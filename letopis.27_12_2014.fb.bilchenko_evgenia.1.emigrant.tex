% vim: keymap=russian-jcukenwin
%%beginhead 
 
%%file 27_12_2014.fb.bilchenko_evgenia.1.emigrant
%%parent 27_12_2014
 
%%url https://www.facebook.com/yevzhik/posts/766356203399492
 
%%author Бильченко, Евгения
%%author_id bilchenko_evgenia
%%author_url 
 
%%tags bilchenko_evgenia,emigracia,poezia
%%title БЖ. Эмигрант
 
%%endhead 
 
\subsection{БЖ. Эмигрант}
\label{sec:27_12_2014.fb.bilchenko_evgenia.1.emigrant}
\Purl{https://www.facebook.com/yevzhik/posts/766356203399492}
\ifcmt
 author_begin
   author_id bilchenko_evgenia
 author_end
\fi

БЖ. Эмигрант.

Отчизна. 
В столице – сплин.
В окопе – погибший взвод...
Ты знаешь, сынок, Берлин
Едва ли тебя спасёт.
Шахтёрские блокпосты.
Задумчивый львовский сад.
Тебе не комфортно, ты,
Мой маленький демократ?
Здесь есть и журналы ню,
И пиво за евро есть.
Не бойся: я не виню
Ни совесть твою, ни честь,
Ни страсть по чужим мехам,
Ни сладкий попсовый вкус:
Я – тоже не без греха,
Я тоже – немного трус.
Но если у Храма – гроб,
И юная попадья
Ведёт беспилотник, чтоб
Твои же спасти края,
И молится на звезду, 
Оплакав в Крыму маяк, −
Ты знаешь… 
Я не пойду
С тобой распивать коньяк. 
27 декабря 2014 г.

\ifcmt
  pic https://scontent-lga3-1.xx.fbcdn.net/v/t1.18169-9/10801944_766356143399498_5578086997859884448_n.jpg?_nc_cat=107&ccb=1-3&_nc_sid=730e14&_nc_ohc=vsYCUd_unhkAX8Nc6Y8&_nc_ht=scontent-lga3-1.xx&oh=367a1c60cdebdc58c83870cc22b58621&oe=60E92AE7

	pic https://scontent-lga3-1.xx.fbcdn.net/v/t31.18172-8/10887677_766356176732828_1380631338212626344_o.jpg?_nc_cat=104&ccb=1-3&_nc_sid=730e14&_nc_ohc=lmiIXKBAvQQAX-lwMtu&_nc_ht=scontent-lga3-1.xx&oh=e0c20ff272db02856c55cc25516e49c7&oe=60E8B6B1
\fi

Марина Марченко
!
 · 6 г.
Наталья Андреева
Супер,Женя!
 · 6 г.
Марк Вайнер
Сделано хорошо: чётко и сильно. Бьёт!
 · 6 г.
Роксана Ахвердян
Женечка! Молодец!
 · 6 г.
Larisa Novitskaya
Сердце защемило...
 · 6 г.
Ядгор Норбутаев
!!
 · 6 г.
Юлия Супруненко
Сильно...
 · 6 г.
Ирина Руднева
Женечка... мы искупим, исправим... верь.... Просто другого ведь не дано
 · 6 г.
Женя Чистый
Привет, тёзка! Вот эта строка: "Ты знаешь, сынок, Берлин Едва ли тебя спасёт" - она сегодня символична и в переносном смысле. В том смысле, что преступна и цинична эксплуатация боевой славы советских дедов, дошедших до Берлина, в сегодняшней антиукраинской пропаганде "русскомирых". Ничто не оправдает братоубийства, свершаемого ура-фанатиками под шаманистское позвякивание медалями усопших героев Отечественной. Попытка реставрации тоталитарного Совка не имеет ничего общего с нашей совместной победой над фашизмом в те годы.

Ирина Руднева
Мне кажется, и Женя об этом.
 · 6 г.
Женя Чистый
Мне кажется, сюжетно - о сегодняшнем Берлине (в смысле "европанампоможет") и обывателях, безответственно и трусливо полагающих, что за них "кто-нибудь" вкупе с берлинами/еэсами разрулят ситуацию, и можно продолжать лениво потягивать коньячок по кабакам.
 · 6 г.
Ирина Руднева
Женя придет, расскажет. Хотя поэты не любят объяснять стихи. Да может и не надо.
 · 6 г.
Сергій Богаченко
Сподобалось,дякую ....такого багато,людей бракуэ свiдомостi,..серце не болить,э i такi,що втомились чути про вiйну,чужу бiль,смерть....але пройде час стануть мудрiше,або перетворяться на привiдiв самотнiх i бездушних....

Ирина Иванова
(y)
 · 6 г.
Женя Чистый
Да она уже рассказала, Ирина. Вплоть до того, что названием самим - речь о моральной эмиграции, о душевном дезертирстве. Впрочем, не стану навязывать своё видение, ибо такой же рядовой Женин почитатель. Каждые ракурс и всколыхнутые произведением в читателе ассоциативные множества имеют право быть)
 · 6 г.
Marina Kiriyak
Женя !!!!!!!!!!!!!!!!!!!!! ❤

Олег Тихонов

\ifcmt
  ig https://scontent-lga3-1.xx.fbcdn.net/v/t1.18169-9/10891697_390648114443857_9136877882961082922_n.jpg?_nc_cat=110&ccb=1-3&_nc_sid=dbeb18&_nc_ohc=pi5mvmJF3sYAX8sV04T&_nc_ht=scontent-lga3-1.xx&oh=e8ec55eff9046d1644548fe6dd5a601c&oe=60E84AED
  width 0.4
\fi


Oksana Romanova
Спасибо.
 · 6 г.
Евгения Бильченко
Жене Чистому: ты молодец, я это под-сознательно, а ты хороший интерпретатор, внимательный. Так ведь оно и есть. В рунете многие обиделись. Решили ,чт оя эмиграциб обижаю. Это не так. Там не в том дело. Я люблю вынужденных политэмигрантов. Бродского обожаю. мажоров не люблю. Я ниже покажу, что я написала. Не люблю оправдываться, но я людей мира люблю ,и я хочу объяснить. Поэты не умеют объяснять, но я учиитель еще, у меня мораль должна быть ,я должна. Сейчас...

Евгения Бильченко
Мой оппонент (хороший человек, но читает меня как-то своеобразно): " понятие эмиграции на сегодняшний день размыто. Те украинцы, которых многие считали украинцами, могут оказаться коренными немцами, которые занимались сельским хозяйством при Екатерине, итальянцами, чьи предки строили Одессу в 19 столетии и пару поколений прожили в Бессарабии, и так далее. Тема стихо снова упирается в потолок диады "космополит- ура-патриот" " Я (в ответ): " Наталия, спасибо за посвящение меня в историко-философский дискурс (ясное дело, вам кажется ,что я его не знаю), но позвольте вам ответить проще. Люблю Простоту. Начнем с примитивных определений. Миротворец - человек, который, рискуя собой и не поддерживая ни своих, ни чужих, будет кормить конфетами и булочками их детей. Кормить под пулями.Для меня это - идеал священника. Так называемый "пацифист" - человек, который залезет под виртуальный диван своей соцсети и оттуда будет вещать о конфетах и булочках для своих и чужих, не накормив ни тех, ни этих. С такими общаюсь некультурно, ибо злая стерва и вообще "невротичка", "шизофреничка"и "злобная девочка, которая делит мир на "черных и белых" ))). Об эмиграции. Когда перед Бродским и Бердяевым СССР поставил невозможные условия для творчества и эти люди (перед которыми я преклоняюсь) служили своей Родине на чужбине (чего стоит один только журнал "Путь"), - они делали то, что обязан делать не только патриот - нормальный, не замусоренный медиа человек. А если сынок богатеньких родителей бездельничает на просторах Европы, за которую тут его братья и сестры проливают кровь (впрочем, не за нее - на фиг она нам сдалась? ))) - за Свободу и Любовь, мы же тут романтики все и идиоты: боремся не за колбасу))), - эта проблема иного плана, которую блестяще описал Милан Кундера. В завершении - личный пример. Дорогой мне друг из Союза Европы, из Бельгии (дорогой, потому что был не буржуа, а католик-марксист, носил дырявый свитер, читал Кундеру, плевал на капиталистическую жизнь и целовал стены славянской Лавры) предложил мне переехать в Дортмунд (богословский факультет Технологического института) и там защитить мою докторскую по левинасизму без проблем и с очень крупным европейским резонансом - бесплатно. Я отказалась. Мою фразу "I love Motherland" (это был 2008 год) - он понял. Поэтому я буду где угодно: на Волыни, в Енакиево, в Киеве, во Львове, в Баре - не деля свою Неньку-Маму на своих и чужих, но ура-патриоты, путинисты в обостренном состоянии "крымнаша головного мозга", как и безродные сущности с масками космополитов, - мне противны. Как противны и трусы в масках миротворцев. Патриотизм - это преданность. Космополитизм - умение набираться мировой культуры у ВСЕХ ДРУЗЕЙ Мира. Русских, индусов, бельгийцев... Миротворчество - это мужество, самопожертвование и сила воли. Эмиграция - это вынужденный шаг, если фашисты в твоей стране угрожают тебе смертью (вспомните суицид Беньямина), а не бросание Родины в беде, если максимум, что тебе угрожает, - это переезд в другой город с менее удобной квартирой. Ничего не мешает юным принцам эмигрировать из Песок в Острог и там работать на благо Родины - здесь еще нет (пока) повального концлагеря, так что этим оправдываться не надо. Я устала от вранья, от лжи, которая прикрывает трусость и малодушие. От того, с каким бесстыдством искажают мои строки и мое имя, переворачивая их на противоположные вещи, - если это не глупость, то либо способ показать себя любимых на моем паблике (который ни на что не претендует, кроме моего желания говорить то, что я думаю СВОБОДНО), либо заведомое коварство). Извините за длинноты. Пусть мой муж не переживает - он абсолютно прав, просто у меня мужества нет сказать этого. Но, с другой стороны, когда, выныривая из кошмара, в котором я живу в последнее время, я вижу подобные Вашим комментарии - это гарант того ,что я хоть каким-то раком народу нужна. Мое личное достижение последнего периода - перестала бояться одиночества. И пожалуйста, поверьте: слава приходит к тому, кто желает не ее, а Пути. Или не приходит. Но Путь продолжается. Солдаты научили. Всем мира, мужества, чести и добра.

Евгения Бильченко
" Мой оппонент: " Евгения, согласна, конечно, в обществе у всех своя роль, своё предназначение. Бездельничать- это вообще последнее дело, я считаю. Хорошо, когда человек сделал много добрых дел, посвятил себя благотворительности, например, занимается вопросами детей из Перу или еще каких-то стран. Но разве так важно, кому служить? Например, если мама итальянка, папа немец, а бабушка украинка, и ребенок везде жил и ходил в разные школы, а родители, например, дипломаты, которые, без сомнения, служат благу Родины, но в разных странах и женаты тоже на выходцах из других стран, а ребенок ходил в английскую школу (программа которой не меняется от переезда из страны в страну) то где Родина человека? Он должен выбрать ее сам. Нельзя быть категоричным, мы живём в многогранном и очень сложном мире. И Левинас, насколько я знаю, не украинец) Мой посыл был в том, что слово "эмигрант" мне резануло глаз, только и всего. Начиная с вопроса о том, что считать эмиграцией, и заканчивая, что в ней были не последние люди)) Поэтому спасибо за пост, в котором выразили отношение к Бродскому! Было очень интересно это отношение. А откуда вообще взялись фигуры мажоров в Ваших комментариях? Проще говоря, для меня эмигрант- это не сразу мажор!)) Странно, откуда такая связка? ". Я: "Перу причем? Для меня роднее люди из Жабьего или Горловки. Жабье - это Верховина, где такие, как я, злобные бандеровцы, формируются ) А Горловка... сами знаете. Перу - это так, отговорка для ничего-не-деланья. Меня на Африку не хватает, звыняйте великодушно. И так не беспокойтесь. Говорю о мировой открытости, отвечают: "А, Родину предает". Скажу о предательстве: "Говорят ,а своих выгораживает". Что бы я ни сказала - ярлыков больше, чем меня, но я зато - одна. И запомните: любое явление имеет две стороны, оттенки нужно ощущать, а не умеете - не мои же проблемы, правда? Гармонии и ума вам. "Твое дело - писать. Ты никому не должна ничего объяснять" (Андрей Балабуха, писатель, Санкт-Петербург, из личной беседы). ".

\emph{Евгения Бильченко}

Одного из двух: либо я круглая дура, либо что-то с рунетом, либо что-то с
планетой. если люди не понимают, чем отличается эмигрант Кундеры ,которым могла
стать я, останься у нас мсье Янык, от мажора, которому некомфортно ,когда в
своих стреляют, то я не знаю ,как дальше в этом мире жить. Я не знаю ,как жить
в мире. где философ Беньямин равен Димебилану. Вот честно.

 · 6 г.
\emph{Яков Осмоловский}

Евгения, Вы всё правильно пишете и чувствуете. Просто не ждите, что Вас поймут
те, с кем не стоит пить коньяк. Наша вечная проблема: у многих декларируемая
шкала ценностей "для других" не совпадает с поведенческой шкалой ценностей "для
себя". И как только абстрактный разговор касается конкретных интересов -
меняются реакции.

\emph{Евгения Бильченко}

Вот))) а мне сейчас коньяк вообще нельзя, потому пью его только с теми у кого
совпадает - на 99, 9. Даже если это оппоненты - уважаю за совпадение. Вот пила
недавно с пацаном, который (пока его дом не обстреляли) искренно считал
ополченцев людьми. И кормил за последние деньги. А я же своих бандеровцев
кормила. Но он реально за свои. И реально кормил. Получил правда потом. Вот с
ним я буду пить коньяк до конца жизни. А с патриотом, который ходит в
сине-желтой куртке, живет за родительские денежки в теплой Европе, никакой
угрозы ни для кого не составляет, но при этом, ища, где глубже, хочет выпить со
мной "за Родину" - basta. "Мне доктор запретил", - толерантно улыбаясь, говорю
я и иду бухать с батальоном ПС и с нашими ребятами из Польши, которые на
семинары приезжают. ))) Я вон с Веронским университетом виртуально
перебухиваюсь уже несколько месяцев... Вот какая недобрая Бильченко.

\emph{Дмитрий Королев}
да
 · Показать перевод · 6 г.
\emph{Елена Суханицкая}
Ненавистная моя Родина,
Нет постыдней твоих ночей!
Как везло тебе на юродивых,
На холопов и палачей.
Как плодила ты верноподданных,
Как усердна была губя –
Тех не купленных и не проданных,
Обречённых любить тебя…
 · 6 г.
\emph{Елена Суханицкая}
Женечка, Вы пишете очень талантливые стихи!

\emph{Евгения Бильченко}

И что мне делать теперь ,если каждый понимает буквально и на свой лад? Когда Чепмен убил Леннона, он сел читать Сэлинджера тут же над трупом великого артиста - "Над пропастью во ржи". Так Сэлинджер виновен? Тогда я виновна - простите. Но я не могу писать не то ,что мне приходит - я не отвечаю за это как субъект. Пишется то ,что пишется. Сознательно я могу выразить менее глубокие смыслы ,чем родит сам Текст. Если это моя вина, я ее понесу - у меня достаточно своих и чужих крестов на лопатках. если это закономерность творчества - как мне свернуть с Пути? если это проблемы самих читателей (их внутренние совести - я буду хирургом. Если они не хотят скальпеля ,а хотят терапии, я, как видно из текста о собаках, могу и зализывать. Только не подлизываю никогда. По Фрейду ничего не происходит просто так. Броский и кундера жестоко описывали СВОИ же муки. Как и Фройд. Как и я, когда я здесь сдуреваю в столице, хотя должна быть на Востоке. Просто сдуреваю. На транках. Вы не поняли этого? Но меня по здоровью и занятиям не берут, и я приму любой упрек от бойцов (которые меня пока что просто зачем-то нянчат, видно, за мой идиотизм) за то ,что я здесь в Киеве сду-ре-ва-ю. Извините, конечно. Но сорвалась. Впервые ,может. Обидно стало.

\emph{Евгения Бильченко}
Боже Упаси! Я поеду в диаспору - это прочту: народ выйдет из зала. Я поеду в батальон - прочту о том ,что гопники тоже люди - народ выскочит из окопа. Я поеду к гопникам - они уже раз десять меня как нацика мочили... Такова жизнь поэта. Пришлось гуглить слово "копирайт" : оказывается. это что-то связанное с авторским правом. Неевропейский я человек ) Кстати ,сегодня Игорь Шуб мне предложил-таки его оформить, чтобы защититься от плагиата, но мне лень. Меня много - сплагиатируют одно - напишу другое. Это как Басе, начали копировать его символы, он перешел к "каруми" - прозрачной повседневности без символов. "Эмигрант" - это мое каруми. А Родины еще больше, чем ничтожной меня, - ее на всех хватит. И она переживет все копирайты, меняясь вместе с ними, благодаря им и вопреки им. если ее будут делить ,как ребенка на суде Соломона, я не дам разрубить - мне легче отдать. Правда ,что будет с матерью потом - в Ветхом Завете не говорится. Я давно не обижаюсь. Я просто обижаю своими текстами. А люди по Фрейду тянутся к тому, что их травмирует. Могу уйти в письмо в стол - письмо от этого не пострадает. Я после периода медитативных практик поняла, что пострадаю меньше, чем мне это представлялось в амбициозной юности. Но у меня солдаты есть. И беженцы. И ребята из Болотной. И те, кто меня любит - непонятно за что, за истерии в текстах, наверно.

\emph{Andrew Sadowsky}
Согласен 🙂 Лучшее против воровства - отдать, дабы не было возможности украсть. Посему , как альтернативу копирайту, предпочитаю Creative Commons.
 · 6 г.
\emph{Евгения Бильченко}
Молодец, мой буддист. Помнишь ,как в притче с дерьмом и яблоками? кому что есть отдать.
 · 6 г.
\emph{Andrew Sadowsky}
Кстати, очень порадовался возобновлению сайта Creative Commons Украина: \url{http://creativecommons.org.ua/}
