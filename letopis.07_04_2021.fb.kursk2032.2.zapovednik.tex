% vim: keymap=russian-jcukenwin
%%beginhead 
 
%%file 07_04_2021.fb.kursk2032.2.zapovednik
%%parent 07_04_2021
 
%%url https://www.facebook.com/groups/kursk2032/permalink/2933651163543279/
 
%%author 
%%author_id 
%%author_url 
 
%%tags 
%%title 
 
%%endhead 

\subsection{Курский заповедник - Стрелецкая степь}
\label{sec:07_04_2021.fb.kursk2032.2.zapovednik}
\Purl{https://www.facebook.com/groups/kursk2032/permalink/2933651163543279/}

\ifcmt
  pic https://scontent-bos3-1.xx.fbcdn.net/v/t1.6435-9/169795294_321755716030776_6503495457819359782_n.jpg?_nc_cat=103&ccb=1-3&_nc_sid=b9115d&_nc_ohc=GUfiQ9a5lz4AX-2eUyc&_nc_ht=scontent-bos3-1.xx&oh=2646b5afa86b9aca87007d8d9384c6f3&oe=609616F5

	pic https://scontent-bos3-1.xx.fbcdn.net/v/t1.6435-9/169814082_321755802697434_532806362818296226_n.jpg?_nc_cat=106&ccb=1-3&_nc_sid=b9115d&_nc_ohc=O3zTNCTcuhQAX-hzOFf&_nc_ht=scontent-bos3-1.xx&oh=5807dbacd9d4d00b47b0ac1e8540126f&oe=6096CD09

	pic https://scontent-bos3-1.xx.fbcdn.net/v/t1.6435-9/170213664_321755876030760_4761611411717086190_n.jpg?_nc_cat=100&ccb=1-3&_nc_sid=b9115d&_nc_ohc=gLo0NULnsMYAX9AM52s&_nc_ht=scontent-bos3-1.xx&oh=b7c9e408dbe9438488a719e80ae0bd2f&oe=6096BA3E

	pic https://scontent-bos3-1.xx.fbcdn.net/v/t1.6435-9/169956506_321756132697401_4441185762734989363_n.jpg?_nc_cat=108&ccb=1-3&_nc_sid=b9115d&_nc_ohc=UCyvQjxpJ70AX-Hmz8-&_nc_ht=scontent-bos3-1.xx&oh=deac0643113f3cec93c85ade7e70b512&oe=6093F196
\fi


Я считаю, что Курский заповедник – Стрелецкая степь необходимо внести в список
Всемирного природного наследия.  

Курский Центрально- Чернозёмный государственный природный биосферный заповедник
имени профессора В.В. Алёхина "Стрелецкая степь" его полное название.

Стрелецкая степь, является разнотравной степью с широколистными злаками,
настоящей лабораторией под открытым небом. На небольшой территории
произрастает 860 выдов различных трав, кустарников и деревьев.  12 видов
занесены в Красную книгу России: 

• волчеягодник боровой,
• проломник Козо-Полянского,  
• ковыль Залесского,
 • венерин башмачок настоящий,
• рябчик кизильник алаунский.
• пион тонколистный (нигде больше в мире не растет)     
• Касатик безлистный;
• Рябчики русский и шахматный;
• Ковыли перистый, опушеннолистный и красивейший.
        С ранней весны до поздней осени в луговой степи можно увидеть смену картин. В середине апреля степь покрывается лиловыми пятнами сон-травы. Следом зацветают горицвет и первоцвет весенние. Весна заканчивается цветением ветреницы лесной и касатика безлистного. В конце мая - начале июня синие соцветия шалфея лугового в серебристом море ковыля перистого. Обильное цветение таволги шестилепестной, нивяника обыкновенного и клевера горного.   Во второй половине июля в побуревшей степи выделяются только синие метёлки живокости клиновидной и тёмно-лиловые свечи чемерицы черной. Еще его называют «Страной живых ископаемых». Здесь можно увидеть такие редкие растения, как пальчатокоренники мясо-красный и кровавый, кувшинка белоснежная, вольфия бескорневая – самое маленькое цветковое растение мира. Из околоводных млекопитающих здесь обитают европейская и американская норки, выдра, выхухоль. В пойме реки располагается одна из самых больших в Курской области колоний серой цапли. Здесь  гнездятся такие виды птиц как кряква, чирок-трескунок, болотный лунь, желтоголовая трясогузка, тростниковая овсянка, усатая синица.
     В Центрально-Черноземном заповеднике известно 1260 видов высших растений, а это около 80 процентов флоры Курской области. Зарегистрировано 140 видов мхов, более 200 видов водорослей, 188 видов лишайников и около 950 видов грибов, два из которых (грифола зонтичная и рогатик пестиковый) занесены в Красную книгу России.
      Богат и разнообразен животный мир заповедника. Здесь обитает 50 видов млекопитающих, обычен кабан, косуля, лось, лисица, барсук. . Только в ЦЧЗ встречаются такой вид как темная мышовка. Отмечено 225 видов птиц. В луговых степях обитает множество куропаток, перепелов, жаворонков, луней. В дубравах заповедника гнездятся: обыкновенный канюк, черный коршун, обыкновенная пустельга, ястреб-тетеревятник и чеглок,  регулярно гнездится редкий вид – орёл-карлик. В заповеднике отмечено 5 видов пресмыкающихся: прыткая и живородящая ящерицы, веретеница, уж обыкновенный, степная гадюка и 10 видов земноводных. В реке Псёл обитает около 30 видов рыб. Из беспозвоночных в заповеднике только насекомых свыше 4000 видов. Многочисленны жуки – 2039 видов, бабочки – 856, двукрылые – 451, перепончатокрылые – 289 и клопы – 190. 19 видов насекомых занесены в Красную книгу России. На участках заповедника обитает более 200 видов пауков.
    Центрально-Черноземный биосферный заповедник – это богатство и гордость не только курян, а и всех россиян!
