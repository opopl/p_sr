% vim: keymap=russian-jcukenwin
%%beginhead 
 
%%file 03_12_2020.news.ru.lenta_ru.2.putin_vaccination_start
%%parent 03_12_2020
 
%%url https://lenta.ru/brief/2020/12/02/go/#numeric_card-2
 
%%author 
%%author_id 
%%author_url 
 
%%tags 
%%title Путин поручил начать массовую вакцинацию от коронавируса
 
%%endhead 
 
\subsection{Путин поручил начать массовую вакцинацию от коронавируса}
\label{sec:03_12_2020.news.ru.lenta_ru.2.putin_vaccination_start}
\Purl{https://lenta.ru/brief/2020/12/02/go}

\ifcmt
pic https://icdn.lenta.ru/images/2020/12/02/17/20201202170600967/brief_412d8cdd29ec64e0e1dc202ea83255a6.jpg
\fi

\begin{center}
\bfseries\em\Large\color{orange}
Когда она начнется и кто первым получит прививки?
\end{center}

\subsubsection{Вакцинация начнется на следующей неделе}

Президент России Владимир Путин поручил на следующей неделе начать в стране
массовую вакцинацию от коронавируса. Об этом сообщает ТАСС в среду, 2 декабря.

По его словам, в ближайшие дни объем произведенной в России вакцины достигнет
двух миллионов доз, что «дает возможность начать если не массовую, то
масштабную вакцинацию».

«Если вы считаете, что мы к такому шагу подошли вплотную, то просил бы вас
организовать работу таким образом, чтобы в конце следующей недели мы уже
приступили к этой масштабной вакцинации», --- сказал президент, обращаясь по
видеосвязи к вице-премьеру Татьяне Голиковой.

Первыми прививку должны получить врачи и учителя.

\subsubsection{В правительстве подтвердили, что для начала вакцинации все готово}

Вице-премьер правительства Голикова отметила, что сейчас идет оценка двух
приоритетных групп населения --- врачей и педагогов.

«Мы оценили наши возможности. Сейчас завершаем подведение итогов, у нас есть
возможности начать масштабную вакцинацию уже в декабре», --- сказала она.

\subsubsection{Ранее сроки вакцинации назвал Минздрав}

О том, что группы риска начнут вакцинировать уже в декабре 2020 года, ранее
сообщал глава Минздрава Михаил Мурашко. Это будут медработники, учителя,
сотрудники служб жизнеобеспечения, пациенты с сахарным диабетом, ожирением,
гипертонией и люди старшего возраста.

Массовая вакцинация, по словам министра, начнется в январе или феврале. Для
этого в настоящее время ведется масштабирование производства препаратов.

\subsubsection{Сейчас в стране зарегистрированы две вакцины}

В России на сегодняшний день зарегистрированы две вакцины против COVID-19.
Первым регистрацию прошел препарат «Спутник V». Препарат был разработан в
Центре им. Гамалеи, его клинические испытания прошли в июне и июле. 14 октября
президент Владимир Путин объявил о регистрации «ЭпиВакКороны» от центра
«Вектор». В стране также разрабатывается третий препарат --- этим занимается
центр имени Чумакова. Планируется, что массовый выпуск вакцины начнется в
феврале 2021 года.

\subsubsection{Число заболевших в России превышает 2,3 миллиона}

Всего с начала пандемии в России официально подтвержден 2 347 401 случай
COVID-19. Это четвертый показатель в мире. Больше заболевших только в США (13,7
миллиона), Индии (9,5 миллиона) и Бразилии (6,3 миллиона).

В России зафиксировано 41 053 летальных исхода, выздоровело 1 830 349 человек.
За последние сутки выявили 25 345 новых случаев заражения COVID-19, больше
всего --- в Москве (5191), Санкт-Петербурге (3684) и Московской области (1148).
