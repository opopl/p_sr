% vim: keymap=russian-jcukenwin
%%beginhead 
 
%%file 17_02_2022.stz.news.ua.fraza.1.kak_zapadnyje_partnery_opuskajut_ukrainu.4.parohody_samolety
%%parent 17_02_2022.stz.news.ua.fraza.1.kak_zapadnyje_partnery_opuskajut_ukrainu
 
%%url 
 
%%author_id 
%%date 
 
%%tags 
%%title 
 
%%endhead 

\subsubsection{И пароходы не плывут, и самолеты не летают}
\label{sec:17_02_2022.stz.news.ua.fraza.1.kak_zapadnyje_partnery_opuskajut_ukrainu.4.parohody_samolety}

О том, что, в связи с околовоенной истерией, воздушное движение над Украиной
оказалось, как минимум, наполовину парализованным, очевидно, слыхали многие.

Если кратко, то это было вызвано тем, что самолеты чаще всего не являются
собственностью авиакомпаний, а находятся в лизинге, и принадлежат различным
банкам, инвестфондам и так далее. Многие из них потребовали вывести воздушные
суда из Украины по причине возросших военных рисков. К вопросу подключились
международные страховые и перестраховочные компании, отказавшиеся страховать
воздушные суда и пассажиров, либо резко взвинтившие расценки на страховку.
Многие воздушные суда оказались прикованными к земле, поскольку, по
существующим правилам, их вообще нельзя поднимать в воздух.

Показательно, что такого не было даже в разгар боевых действий в 2014-2015
годах, когда на Донбассе шла настоящая война с применением авиации и средств
противовоздушной обороны и когда случилась трагедия со сбитым Боингом.

15 февраля, причем после (!) того, как стало известно об отводе российских
войск от наших границ, страховая ассоциация Lloyds Market Association внесла
украинские и российские воды Черного и Азовского моря в зону пиратства,
терроризма и войны. Это очень сильный удар по нашему экспорту и импорту,
идущему через черноморские порты. Это если не паралич перевозок, то
существенное их удорожание из-за роста стоимости страхования. А это — потеря
конкурентоспособности, имиджа и прочее.

Словом, и здесь англо-саксонские «друзья» изрядно «удружили»...

Сколько-нибудь внятных оценок последствий проблем со страховщиками пока найти
не удалось, но думается, что это будет серьезным ударом по экономике Украины,
которая и без того бьется в агонии кризиса и воровства.

\textbf{За всем этим просматривается стремление обесценить Украину, остатки ее
экономики и экономических ресурсов, что даст возможность прибрать ее к рукам за
бесценок.} Причем прибрать именно Западу, поскольку Путину Украина, по большому
счету, совершенно не нужна. Он и с Донбассом-то не знает, что делать.

