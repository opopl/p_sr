% vim: keymap=russian-jcukenwin
%%beginhead 
 
%%file 26_05_2018.stz.news.ua.mrpl_city.1.porty_mariupolja
%%parent 26_05_2018
 
%%url https://mrpl.city/blogs/view/porty-mariupolya
 
%%author_id burov_sergij.mariupol,news.ua.mrpl_city
%%date 
 
%%tags 
%%title Порты Мариуполя
 
%%endhead 
 
\subsection{Порты Мариуполя}
\label{sec:26_05_2018.stz.news.ua.mrpl_city.1.porty_mariupolja}
 
\Purl{https://mrpl.city/blogs/view/porty-mariupolya}
\ifcmt
 author_begin
   author_id burov_sergij.mariupol,news.ua.mrpl_city
 author_end
\fi

Берег Кальмиуса у впадения его в мутноватые воды Меотиды, - так древние греки
называли Азовское море, - мореплаватели посещали еще в глубокой древности. Кто
первый причалил к этому берегу? Историки на этот счет молчат. Но никто не
оспаривает, что приходили сюда \enquote{чайки} запорожцев. С давних давен они
облюбовали естественную гавань в устье реки. Более того, с наступлением холодов
оставались здесь зимовать.

В конце XVIII века Екатерина II Запорожскую Сечь ликвидировала, а на месте
бывшей казацкой крепости – центра Кальмиусской паланки - был заложен город
Павловск, через два года после этого события в него переселились недавние
обитатели Крыма – православные греки. Пристань запорожских казаков пригодилась
и им. Из Крыма в Мариуполь парусники привозили в бочках виноградный сок –
шарап, соль, ткани, другие товары, а обратно – кожи, вяленую рыбу и зерно.
Пристань постепенно обустраивалась новыми причалами, складами, амбарами для
зерна. \textbf{\em Так постепенно образовался первый мариупольский порт.} 

К началу XIX века здесь уже была таможенная застава и портовое правление. В
1840 году была сооружена каменная набережная, с начала 70-х годов регулярно
проводились дноуглубительные работы в устье Кальмиуса. И все же из-за
мелководья далеко не все суда могли подойти непосредственно к причалам порта.
Те из них, что имели осадку более двух метров, бросали якоря на рейде в пяти
верстах от берега. Между берегом и рейдом сновали азовские дубки – мелкосидящие
парусники. Они-то и доставляли на борт судов, стоящих на рейде, грузы. Сначала
это была главным образом пшеница, выращенная в приазовских степях, а позже – с
развитием Донецкого бассейна – и каменный уголь. В середине XIX века в порт
заходило от шестидесяти до ста иностранных судов и до трехсот с лишним судов
каботажных.

\textbf{Читайте также:} 

\href{https://mrpl.city/news/view/otkrytyj-kerchenskij-most-ogranichivaet-prohozhdenie-sudov-v-mariupolskij-port}{%
Открытый Керченский мост ограничивает прохождение судов в Мариупольский порт, Іоанна Вишневська, mrpl.city, 16.05.2018}

В 1882 году к Мариуполю была подведена железная дорога. Поток грузов резко
возрос, порт с рейдовой погрузкой уже не мог обеспечить их переработку.
Тогда-то и возникла идея строительства нового порта. Его хотели разместить у
Белосарайской косы, но потом решили сооружать близ Зинцевой балки в четырех
верстах от конца путевого хозяйства станции Мариуполь. Работы по строительству
нового порта начались в 1886 году, через три года его первая очередь была
торжественно открыта. До наших дней дошла и точная дата этого события, – 29
августа 1889 года, - и название первого судна, ставшего под погрузку донецкого
угля. Это был пароход \enquote{Медведица}.

Еще шли завершающие работы по сооружению порта, а к нему уже подвели
специальную ветку, соединяющую с железнодорожной станцией Мариуполь. Как только
заработал порт, поток традиционных грузов – донецкий уголь и пшеница –
направился к российским и иностранным судам, ожидавшим погрузку у причалов. А в
порту продолжалось углубление акватории, строилась эстакада для выгрузки
чиатурской марганцевой руды, велись работы по устройству внутреннего мола.
Скоро, очень скоро новый порт занял по грузообороту одно из первых мест в
Российской империи, уступив первенство лишь Одессе. Продолжал принимать суда и
порт в устье Кальмиуса. Тот самый, что назван был позже \textbf{гаванью Шмидта}. Здесь
же была пристань для пассажирских судов, которая просуществовала до начала 60-х
годов ХХ века.

Конечно, коренные мариупольцы знают, где находятся и гавань Шмидта,
Мариупольский морской торговый порт. Знают и о \textbf{третьем мариупольском}, причалы
которого разместились вдоль рудного двора доменного цеха комбината \enquote{Азовсталь}.
Этот порт был построен в начал в 30-х годов, его назначение было принимать суда
с керченской железной рудой и керченским же железорудным агломератом. Эту
функцию он выполнял до 90-х годов ХХ века. Позже его использовали для отгрузки
азовстальской прокатной продукции.

Но был еще один порт в Мариуполе, вернее пристань. В память о ней на Правом
берегу есть улица – \textbf{Старая пристань}. Ее причальная стенка находилась в трех
километрах вверх по течению от впадения Кальмиуса в Азовское море. Это было в
те времена, когда река была судоходной. Построена пристань в самом конце XIX
века. К ней пришвартовывались пароходы и баржи с железной рудой керченского
месторождения, предназначенной для доменных печей только что построенного
металлургического завода \enquote{Русский Провиданс}. Теперь это площадка \enquote{Б} концерна
\enquote{Азовмаш}. 

Коль скоро речь зашла о мариупольских портах, следует отметить, что два из них
- \textbf{Мариупольский морской торговый порт и порт комбината \enquote{Азовсталь}} связаны
внутренними водными путями не только со всеми портами Азово-Черноморского
бассейна, но и с портами Белого, Баренцевого и Балтийского морей, а,
следовательно, и с Мировым океаном.
