% vim: keymap=russian-jcukenwin
%%beginhead 
 
%%file 09_12_2020.sites.ru.zen_yandex.yz.rusichi.1.film_beresta_gramota
%%parent 09_12_2020
 
%%url https://zen.yandex.ru/media/politinteres/v-rossii-ekranizirovali-berestianuiu-gramotu-5fca81f3788eda75c7be5d6b
 
%%author Русичи (Яндекс Zen)
%%author_id yz.rusichi
%%author_url 
 
%%tags beresta,film,russia,ancient
%%title В России экранизировали берестяную грамоту
 
%%endhead 
 
\subsection{В России экранизировали берестяную грамоту}
\label{sec:09_12_2020.sites.ru.zen_yandex.yz.rusichi.1.film_beresta_gramota}
\Purl{https://zen.yandex.ru/media/politinteres/v-rossii-ekranizirovali-berestianuiu-gramotu-5fca81f3788eda75c7be5d6b}
\ifcmt
	author_begin
   author_id yz.rusichi
	author_end
\fi

\index[rus]{Русь!История!Фильм, берестяная грамота, 09.12.2020}

\textbf{Можно оценить, что получилось}

\ifcmt
  pic https://avatars.mds.yandex.net/get-zen_doc/1926194/pub_5fca81f3788eda75c7be5d6b_5fcdd51f7e300d7cca38f02f/scale_1200
  caption Кадр из фильма "Я прислал тебе Бересту"
\fi

Вообще, было бы интересно, если бы в поисках свежих сценариев для отечественных
блокбастеров Федор Бондарчук, скажем, обратился бы к той тысяче с лишним
древнерусских берестяных грамот, что найдены за последние семьдесят лет.

Они, конечно, посвящены разным повседневным, бытовым вопросам. Но при должном
желании можно было бы объединить ряд разрозненных грамот, представить себе тех
людей, что их писали, связать их узами общего сюжета...

Но это дело будущего. Пока же за экранизацию столь оригинального письменного
источника взялись энтузиасты из Музея ратной истории Москвы.

\ifcmt
  pic https://avatars.mds.yandex.net/get-zen_doc/1687249/pub_5fca81f3788eda75c7be5d6b_5fcdd7978f8c7853ed6d70c6/scale_1200
  caption Кадр из фильма "Я прислал тебе Бересту"
\fi

\index[rus]{Русь!Берестяные грамоты!Грамота 752, Я посылала к тебе трижды}

Они взяли и сняли короткометражный фильм по одной из самых известных грамот. Мы
про эту грамоту на нашем канале даже писали отдельный пост, если помните.
Археологи обозначают ее скучным номером 752, а в народе она популярна под
названием \enquote{Я посылала к тебе трижды!}.

\enquote{Что за зло ты против меня имеешь, что в эту неделю ты ко мне не приходил? А я
к тебе относилась как к брату! Неужели я тебя задела тем, что посылала? А тебе,
я вижу, не любо. Если бы тебе было любо, то ты бы вырвался из-под глаз и
примчался...} - так писала смелая новгородская девушка своему возлюбленному. Мы
привели только фрагмент, там есть и другие строки.


\ifcmt
  pic https://avatars.mds.yandex.net/get-zen_doc/3373607/pub_5fca81f3788eda75c7be5d6b_5fcdd742788eda75c72777ad/scale_1200
  caption Кадр из фильма "Я прислал тебе Бересту"
\fi


Считается, что это первое в отечественной истории документально зафиксированное
любовное послание. Береста была разорвана, словно в досаде. Но кем? Может, и не
парнем вовсе? Может, попало оно тому, кто подобных свиданий не одобрял? Тут
широкое поле для догадок.

Так вот, реконструкторы из московского музея додумали весь антураж этой истории
и сняли по ней ролик. Он не очень длинный, восемь с половиной минут, так что
можете сразу его и посмотреть:

\video{https://youtu.be/wvE5b4k8tUU}

\begin{itemize}
\item \cusr{Александр Чернышёв}

Бесподобно! История наша настолько глубока, героична и трагична одновременно,
настолько широка география наших народов, культура, языки,
кухня, обряды, верования, традиции, что проф. Соловьев только
на изучение грамот в Императорском архиве 20 лет потратил, аж
ямки в паркетном полу протер! Вон, как \enquote{Холоп} взлетел, даже
эта поделка неслыханный интерес вызвала! и вот, глядя на этот
короткометражный фильм, можно надеяться, что найдется команда,
способная на \enquote{Хочу, могу и сниму!}...
\end{itemize}
