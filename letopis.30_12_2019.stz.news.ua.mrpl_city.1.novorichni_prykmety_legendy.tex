% vim: keymap=russian-jcukenwin
%%beginhead 
 
%%file 30_12_2019.stz.news.ua.mrpl_city.1.novorichni_prykmety_legendy
%%parent 30_12_2019
 
%%url https://mrpl.city/blogs/view/novovorichni-prikmeti-i-legendi
 
%%author_id demidko_olga.mariupol,news.ua.mrpl_city
%%date 
 
%%tags 
%%title Нововорічні прикмети і легенди
 
%%endhead 
 
\subsection{Нововорічні прикмети і легенди}
\label{sec:30_12_2019.stz.news.ua.mrpl_city.1.novorichni_prykmety_legendy}
 
\Purl{https://mrpl.city/blogs/view/novovorichni-prikmeti-i-legendi}
\ifcmt
 author_begin
   author_id demidko_olga.mariupol,news.ua.mrpl_city
 author_end
\fi

Наближення Нового Року – це завжди очікування дива. Сам процес обрання
подарунків для близьких людей чи прикрашання будинку дарує безліч позитивних
емоцій і приємних вражень. А спогади про новорічні свята можуть гріти і
протягом всього життя. Зокрема, мій дідусь часто любив розповідати про свій
найбільш веселий і неординарний Новий рік. Зібралися не тільки всі його
найближчі друзі, але й багато нових людей, які були комусь родичами, а комусь
сусідами. Спочатку було складно розслабитися серед незнайомців, тому дідусь
вирішив трошечки пожартувати. Він з серйозним виглядом сказав, що знає безліч
новорічних прикмет і легенд і готовий поділитися з найближчими людьми. Однак
все, що він розповідав було вигадкою. Хоча друзі звикли сприймати діда
серйозною людиною, тому не тільки вірили, а ще й виконували те, що говорив мій
веселий дідусь. Так за однією з дідових \enquote{прикмет}, щоб здійснилося
найзаповітніше бажання, необхідно бігати навколо святкового столу і дуже
голосно викрикувати свою мрію. Спочатку діда спитали, чи серйозно він, на що
той відповів, що завдяки цьому ритуалу здійснилися чотири його мрії. Тоді всі
вирішили спробувати. Але кімната виявилася малою, стіл великим, тому друзі
винесли його на вулицю. І тут таке почалося... Всі як божевільні побігли. Крик
стояв неймовірний. \enquote{Хочу заміж}, \enquote{Хочу розбагатіти!},
\enquote{Хочу яхту!}, \enquote{Хочу в навколосвітню подорож!}. Майже годину
тривало це дійство. Всім було весело і час пролетів непомітно. Сам дідусь
сміявся до сліз. Але цікаво, що здебільшого всі озвучені бажання здійснилися...

Насправді про Новий Рік можна говорити багато, як і довго до нього готуватися.
Дорогі читачі, у наступному блозі я розповім вам про неймовірну історію
кохання, яка сталася в Маріуполі саме у Новий рік. А зараз пропоную
ознайомитися зі справжніми новорічними прикметами і легендами, які, можливо,
вас здивують чи допоможуть здійснити ваші найбільш заповітні бажання. Так,
дюдину, яка випила за новорічним столом останній келих вина або шампанського з
пляшки, чекає удача. Якщо в новорічну ніч з вами сталося щось незвичайне, то
весь наступний рік чекайте сюрпризів. Якщо за новорічним столом хтось чхне, то
прийдешній рік буде вдалим для всіх присутніх. Хто весело зустріне Новий рік,
той буде веселитися протягом усього року. Якщо 31 грудня або 1 січня до вас в
будинок попроситься собака або кішка, це неспроста. Якщо не можете забрати
тварину – хоча б нагодуйте. Тоді фортуна вам буде посміхатися протягом усього
року.

Добре відомо правило, що як Новий рік зустрінеш, так його і проживеш. Тому на
час потрібно забути про проблеми, помиритися з друзями і домочадцями, якщо
напередодні ви з ними посварилися. Не бійтеся попросити вибачення, це по
достоїнству оцінять всі навколишні. Навчіться радіти святу незалежно від
фінансового становища, успіхів і невдач.

Постарайтеся, щоб в святковий день у вас на столі, в кишені і холодильнику не
було порожньо, тоді і протягом року ваш будинок буде сповнений.

Поділюся з вами ще цікавими легендами, які мені вдалося знайти. Цікаво, що
традиція прикрашати ялинку, згідно з повір'ям, дуже давня і налічує дві тисячі
років. Раніше люди вважали, що всі дерева наділені добрими силами і в них
живуть добрі духи. Люди намагалися задобрити їх і вішали на них подарунки,
частування, фрукти, горіхи. Одна з легенд говорить, що образ Діда Мороза
першими придумали гуни. У гунів був бог Йєрлу, який в перший день року приходив
на землю. Цього дня потрібно було ставити в будинках ялинки, оскільки у гунів
ялина вважалася священним деревом. Виходить, що цій традиції вже 5 тисяч років.

Ще одна легенда свідчить, що різдвяну ялинку придумав засновник протестантизму
Мартін Лютер. Для того, щоб якось зблизити тотемне дерево з християнством, він
попросив паству прикрасити верхівки святкових ялинок Віфлеємської зіркою, а на
гілки помістити свічки. Існує легенда про те, чому ми прикрашаємо ялинки
блискучою срібною мішурою. Колись давним-давно в одному селі жила-була добра
бідна жінка, в якої було багато дітей. Увечері перед Різдвом вона прикрасила
ялинку, але у неї було дуже мало прикрас. Вночі на ялинці побували павуки, і
переповзаючи з гілки на гілку, залишили на собі павутину. В нагороду за доброту
жінки, Христос благословив дерево, і павутина перетворилася на сяюче срібло.

Ось такі існують прикмети і легенди про Новий рік. Сподіваюся, що про деякі з
них ви і самі зможете розповісти за святковим новорічним столом. А наостанок
хочу всіх вас, дорогі читачі, привітати з прийдешнім Новим 2020 роком! Бажаю
вам миру, добра, здоров'я і щастя, адже це найголовніше. А ще казкового настрою
щирих посмішок, вірних друзів навколо. Нехай найзаповітніші мрії збуваються, а
рідні і близькі дарують тільки радість. З новим роком!
