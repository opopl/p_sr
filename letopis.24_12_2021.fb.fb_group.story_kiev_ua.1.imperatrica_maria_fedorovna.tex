% vim: keymap=russian-jcukenwin
%%beginhead 
 
%%file 24_12_2021.fb.fb_group.story_kiev_ua.1.imperatrica_maria_fedorovna
%%parent 24_12_2021
 
%%url https://www.facebook.com/groups/story.kiev.ua/posts/1825480984315361
 
%%author_id fb_group.story_kiev_ua,poljakova_galina.kiev
%%date 
 
%%tags imperatrica.maria_fedorovna.rus,kiev,rusimperia
%%title Императрица Мария Федоровна нежно любила Киев. Как? Почему?
 
%%endhead 
 
\subsection{Императрица Мария Федоровна нежно любила Киев. Как? Почему?}
\label{sec:24_12_2021.fb.fb_group.story_kiev_ua.1.imperatrica_maria_fedorovna}
 
\Purl{https://www.facebook.com/groups/story.kiev.ua/posts/1825480984315361}
\ifcmt
 author_begin
   author_id fb_group.story_kiev_ua,poljakova_galina.kiev
 author_end
\fi

Императрица Мария Федоровна нежно любила Киев. Как? Почему? К ее услугам был
любой уголок любого королевства Европы. Ее, дочь датского короля, тетку
английского короля, вдовую российскую императрицу и мать царствующего монарха с
величайшим почтением бы принялии где душе ее угодно. Но – нет. Когда ей стало
тошно оставаться в Петербурге, он перебралась в Киев. В дивном, зачарованом
Киеве она укрылась от дрязг, распутинщины, злобы и суеты.

\enquote{Мила душка Мінні!}

Принцеса датська створила диво: шлюб Олександра ІІІ з
Марією-Софією-Фредерикою-Дагмар вважається ідеальним. Це було єдине дійсно
щасливе подружжя у розгалуженому клані Романових. 

\ii{24_12_2021.fb.fb_group.story_kiev_ua.1.imperatrica_maria_fedorovna.pic.1}

Дагмар, у православ’ї Марія Федорівна, (1847 р. н.), була маленькою на зріст і
тендітною. Поруч з велетнем-чоловіком вона здавалася зовсім крихіткою. Не бувши
красунею, чарувала якоюсь неповторною жіночістю. Вона малювала морські пейзажі,
грала на фортепіано, надзвичайно граціозно танцювала, та ... Якось невдовзі
після заручин, коли Олександр і Дагмар мирно розмовляли, майбутня цесарівна
раптом рвучко встала, сперлася руками на два крісла і зробила оберт через
голову. Олександр знав, що його наречена щоранку робить гімнастику, обливається
холодною водою, щодня тренується, але такого \enquote{фортелю} аж ніяк не очікував. 

Імператор обожнював свою дружину, яка мала на нього неабиякий вплив. Шведський
історик Стаффан Скотт описує такий випадок. Під час відвідування князівства
Фінляндського високих гостей зустріли національним гімном \enquote{Наша країна} замість
очікуваного \enquote{Боже, царя бережи}. Незадоволений цар не зняв шляпи. Марія
Федорівна смикнула чоловіка за рукав і суворо наказала зняти шляпу. Упертий
самодержець не підкорився. Тоді маленька цариця, просто на балконі, перед
великою юрбою народу, настільки переконливо повідомила чоловікові все, що про
нього думає, що на останньому куплеті Олександр ІІІ все ж зняв шляпу на знак
поваги до фінського гімну.

Двадцять вісім років подружнього життя не затьмарилися жодною сваркою. \enquote{Мила
душка Мінні! Сумно і пусто без тебе, - пише Олександр ІІІ невдовзі після
срібного весілля до дружини, яка провідувала батька в Копенгагені, - Гатчина
цілком змінилася без твоєї присутності, все не те!} З кожним днем \enquote{душка Мінні}
ставала ще більш коханою для Олександра Олександровича. Вона була опорою,
надією, джерелом сил, бо вміла наповнювати оточуючий світ радістю. Життя
спалахувало яскравим світлом в її присутності. З його важкою хворобою вони
боролися удвох. Але не перемогли. У віці 47 років овдовіла.

Початок Першої світової війни застав імператрицю-удову в Лондоні, де вона
гостювала у племінника, короля Англії. Того ж самого дня, 1 серпня, вона
вирушила додому. В Росії на неї чекали. І хоча їй було майже 70, Марія
Федорівна, занурилася у роботу – майже щодня, незважаючи на слабкість та
недуги, вона відвідувала госпіталь. 

Політична криза в державі дедалі глибшала. Переконання, що Росією \enquote{правлять не
ті і не так} поділяло все більше людей. Імператриця-мати говорила з сином і
просила зважити на настрої у суспільстві, але Микола ІІ вже не міг нічого
змінити. Марія Федорівна ненавиділа Распутіна і не знайшла порозуміння з
невісткою, його вірною прихильницею. В салоні Олександри Федорівни
імператрицю-мати  називали \enquote{Гнівною}. Залишатися в столиці більше не хотілося. 

В жовтні 1916 року Марія Федорівна переїхала до Києва і оселилася у
Маріїнському палаці. Вона щодня відвідувала госпіталь, подовгу бесідувала з
пораненими. Часто їздила на прогулянки у відкритому екіпажі і милувалася
улюбленим містом, яке стало для неї затишним притулком і відрадою. Кияни
пізнавали імператрицю-мати на вулицях і тепло вітали. 

Революційний вихор захопив країну. Миколу ІІ з усією сім’єю заарештовано і
відправлено у Сибір. Мало не силою рідня відвезла Марію Федорівну на київський
залізничний вокзал, а звідти – в Крим. Перебування в Криму більше нагадувало
ув’язнення – охорона, приниження, заборона залишати Ай-Тодор. Численні обшуки
дратували стару імператрицю. Особливо її скривдила конфіскація Біблії, тієї
самої, з якою вона приїхала до Росії більше ніж 50 років тому. 

Рокові постріли в Тобольську вже пролунали. З в’язниці Крим перетворився на
пастку для Романових. Король Англії прислав за своєю тіткою і кількох вцілілих
родичів військовий корабель \enquote{Марлборо}. Проте вона відмовилася їхати без своїх
численних друзів. Король дозволив запросити на борт і їх. 11 квітня 1919 року
Марія Федорівна покинула країну, де її двоє синів, п’ятеро онуків і багато
інших близьких родичів були варварські вбиті. До останнього подиху вона не
хотіла вірити, що вони загинули.    

Одного дня вона отримала посилку від датського дипломата. В пакеті була
викуплена в антикварному магазині Біблія, та сама, колись конфіскована в Криму.
З нею в руках вона й померла 1928 року.

\ii{24_12_2021.fb.fb_group.story_kiev_ua.1.imperatrica_maria_fedorovna.cmt}
