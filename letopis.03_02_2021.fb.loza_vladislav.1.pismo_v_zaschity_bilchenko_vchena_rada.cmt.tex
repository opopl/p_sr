% vim: keymap=russian-jcukenwin
%%beginhead 
 
%%file 03_02_2021.fb.loza_vladislav.1.pismo_v_zaschity_bilchenko_vchena_rada.cmt
%%parent 03_02_2021.fb.loza_vladislav.1.pismo_v_zaschity_bilchenko_vchena_rada
 
%%url 
 
%%author 
%%author_id 
%%author_url 
 
%%tags 
%%title 
 
%%endhead 
\subsubsection{Комментарии}
\label{sec:03_02_2021.fb.loza_vladislav.1.pismo_v_zaschity_bilchenko_vchena_rada.cmt}

\begin{itemize}
\iusr{Nataly Pasichnyk}

робіть позов у суд

\iusr{Влад Лоза}

Nataly Pasichnyk це не так просто, оскільки ніхто нікого не звільнив. Рішення
\enquote{призупинити} викладання, скоріш за все, буде юридично бездоганно оформлене. З
чим іти в суд конкретно проти університету (не беручи до уваги погрози, наклепи
та іншу уголовщину, за які установа \enquote{відповідальності не несе})?

\iusr{Nataly Pasichnyk}

Влад Лоза ну призупинити означає відсторонити від викладання на невизначений
термін. зарплату виплачуватимуть? якщо ні, то треба ще й робити лінгвістичну
експертизу і лекцій, і постів.

\iusr{Nataly Pasichnyk}

Влад Лоза іти проти незаконного відсторонення від викладання на підставі
написання приватного допису у фб.

\iusr{Nataly Pasichnyk}

Влад Лоза найважливіше - подавати все сухо і без емоцій, по закону.

\iusr{Влад Лоза}

Nataly Pasichnyk зарплату платять начебто. дякую за думку. передам Вашу пропозицію правозахисникам

\iusr{Nataly Pasichnyk}
Влад Лоза о, так це - канікули)

\iusr{Владислав Сушков}

Не исключено, что нагрузка будет просто сдвинута на более поздний срок.
Преподаватель университета в этом более свободен, нежели школьный учитель.

\iusr{Vladislav Petrenko}

Про малоросійську номенклатуру це певний ідеологічний маркер. Думаю що не варто
використовувати риторику опонентів. А так цілком згоден. Шкода тільки що інші
університети не зробили своїх заяв.

\iusr{Влад Лоза}

Vladislav Petrenko тут все - ідеологічні маркери. + вся медіа-артикуляція
тримається на апропріації риторики опонентів. інакше виходить, що треба діяти
за логікою \enquote{якщо означник вже юзнули, то він зашкварений}, і ми
\enquote{не маємо на нього права.} Так можна зовсім без означників лишитись, з
огляду на те, що боротьба розгортається у шизофренічній площині)

\iusr{Vladislav Petrenko}

Влад Лоза він просто методологічно не вірний. І семантично пов'язаний з малою
Грецією. Думаю що варто використовувати власну риторику. Але це не суть
важливо.

\iusr{Влад Лоза}

Vladislav Petrenko дякую, що просвітили, я ж думав, що Ruthenia Minor придумав
Путін) Ви плутаєте медійне перекидання мемами і етимологічний диспут. В теорії
я ЗА те, щоб повертати правильно артикульований означник \enquote{Мала Русь} у дискурс
про українську ідентичність, бо він апелює до привілейованого права на
цивілізаційну спадщину Русі. Але використовується націоналістами, з одного
боку, як маркер стигматизації, а рос. імперіалістами, з іншого боку - як маркер
імперської провінційної периферії. І я писав про це не раз. Але, боюся, нашим
опонентам байдужі такі філологічні тонкощі, як байдужі їм і реальні проекти
інклюзивної укр. ідентичності. Тож на даному етапі тактично доцільним є
сократичне \enquote{обертання} маркеру опонента проти нього самого - в дусі
звинувачення опонента у невідповідності своєму власному дискурсу, витісненні
того, чим є він сам. Я подивився б, як би Ви повели у цій шизофренії виважений
етимологічний диспут, із методологічно вивіреним використанням "власних"(?)
означників.

\iusr{Vladislav Petrenko}

Влад Лоза 1 я не збирався вас просвіщати. 2 я з вами вже погодився 3 я не думаю
що дискусія можлива. Судячи з коментарями під вашими дописати. 4 не будемо
\enquote{мірятися} ліпше солідаризуватися. 5 з терористами і фанатиками перемовин не
ведуть ви і самі це знаєте. 6 успіху вам у вашій справі.

\iusr{Влад Лоза}
Владислав Петренко дякую

\iusr{Stanislav Tykhonko}

А які сценарії ви запустите, якщо Більченко не реабілітують?

\iusr{Alexey Nedviga}

Бестиарий

\iusr{Евгения Финчук}

Я хоть и с другого факультета, но тоже с вами! На самом деле много тех, кто
против её увольнения только из-за её политической позиции, просто молчат..
боятся высказывать. Мои с моего факультета тоже пытались пристыдить за
поддержку Евгении, хотя явно же понимают что она верные вещи говорит!

\iusr{Сергій Іванович Запоржанин}

взагалі раджу не дочікуватися висновків невідь-кого щодо змісту лекцій, а
замовити таку ж саму експертизу паралельно, хоча би лінгвістичну у київському
лінгвістичном, зібрати висновки експертів (у програм же є рецензенти) про
програми та лекції і все) а те, що депутат від слуги народу у відео одразу
сказав \enquote{що у лекціях щось не те} - ще один доказ, що розбір лекції
медиком-фрілансером замовили, може і слуги.

\begin{itemize}
\iusr{Влад Лоза}

Сергій Іванович Запоржанин дякую

\iusr{Влад Лоза}

Сергій Іванович Запоржанин а про якого депутата і про яке відео йдеться? можете
сюди скинути, якщо під рукою?

\iusr{Сергій Іванович Запоржанин}

Влад Лоза те, що в Жені на стіні, найновіше, з каналу НАШ, послухайте його уважно до кінця)

\iusr{Влад Лоза}

Сергій Іванович Запоржанин на 2.34 вона сама про це згадує, що могло дати слузі
зачіпку. Тобто він все ж не проговорився перший, а зловив на слові. хоча ви
праві, без попереднього наміру і знання контексту таке питання не поставиш, це
секундна омовка

\iusr{Сергій Іванович Запоржанин}

Влад Лоза авторка аналізу в одному з коментарів зізналася, що цей аналіз їй, як
фрілансеру-копірайтеру, замовили. не від щирого серця вона це робила, і ще
зазначала, що її заробіток чесний. мені здається, було би варто зробити аналіз
тієї ж лекції, де розбити в пух і прах її, притягнуті за вуха тези. звісно, що
Женю замовили політики, а не бідні педагоги. Лише в політика є можливість
профінансувати цілу кампанію булінгу, включно із спеціальним відео-випуском із
Стерненком. Гадаю, що спонсор Стерненка і є замовником.

\iusr{Влад Лоза}

Сергій Іванович Запоржанин погоджуюсь, що це доцільно. Всі ваші зауваги я
передав, ідея з експертизою дискутується. Але сама ситуація поки надто
підвішена. Все здійснюватиметься вже при підготовці юридичної позиції для суду.
\enquote{Аналіз} ви маєте на увазі зробити в рамках лінгвістичної/фахової експертизи,
чи постом у фейсбуку? якщо друге, то Більченко вже спростовувала всі ці речі
публічно: там на цих аргументах ніде клейма ставити. Плюс подібні \enquote{розбори}
давно тиражуються, критика фрілансерки вже далеко не єдина, і це буде
примножуватись. Відповідати на кожен такий закид не вийде. Вони зрозуміли, що
по мовному питанню і по закону нічим крити, значить, треба перевести цькування
у вимір профпридатності.

\iusr{Сергій Іванович Запоржанин}

Влад Лоза Звісно, що все має бути офіційним і закріпленим печаткою. І навіть
звернення до преза має бути офіційним, щоб вам дали офіційну відповідь. Він не
зобов'язаний реагувати на кожен відеозапис, але мусить відреагувати на офіційне
звернення. Адресу його офісу можна дізнатися на сайті :)

\iusr{Сергій Іванович Запоржанин}

Влад Лоза всі програми, які зробила професор БІльченко, наскільки я розумію,
відрецензовані докторами наук, треба підняти ці рецензії, це вже полегшить
роботу. Крім того, як виявилося, по цих програмах читають інші викладачі. Хай і
вони напишуть відгуки, що програми - хороші. Вони ж не писатимуть самі проти
себе

\iusr{Сергій Іванович Запоржанин}

Влад Лоза просто з програмами зрікатися колеги вийде важче, бо ж доктори наук
підтвердили і неодноразово, що все чікі-пікі)

\iusr{Сергій Іванович Запоржанин}

а аналізувати можна лише ті лекції, які є в записі, все решта - необгрунтований
донос, тому експертизу треба робити усіх відеоматеріалів. це затратно, але по
факту)

\iusr{Влад Лоза}

Сергій Іванович Запоржанин затратність - один із факторів. Путін нас фінансує
за залишковим принципом) + ще той факт, що кафедри більшості київських
університетів, де таке можна замовити, окуповані нациками. Треба ретельно
підібрати заклад.

\iusr{Сергій Іванович Запоржанин}

Влад Лоза ці рецензії були написані вже давно, їх просто треба підняти,
потворної експертизи програм не потрібно. треба робити аналіз лише відеолекцій.

\iusr{Сергій Іванович Запоржанин}

Влад Лоза щодо мовної екпертизи, так її замовляють при різних кримінальних
справах, думаю, таких спеців є багато.

\end{itemize}

\iusr{Sergey Titsky}

Пане Лоза! Вам не страшно учиться на факультете, ученый совет которого
составляют, по вашим словам, шизофреники, кафкисты-асурдисты и шизоиды
(\enquote{в обстановці шизофренічного залякування і кафкіанського структурного
абсурду}, \enquote{безграмотні та поверхові квазі-методологічні звинувачення
пост-єльцинських шизоїдних бабусь}), гнобящие \enquote{гениального ученого с
мировым именем} и его не менее гениальных неофитов-учеников?...

\begin{itemize}
\iusr{Сергій Іванович Запоржанин}
Sergey Titsky пан Лоза, звісно, занадто емоційний у своїх формулюваннях, але й
опоненти Більченко не мають, що пред'явити, окрім голих емоцій)

\iusr{Влад Лоза}

Сергій Іванович Запоржанин вважаю, що з огляду на почуте і побачене мною, це ще вельми виважена діагностика)

\iusr{Сергій Іванович Запоржанин}

Влад Лоза це все поезія, щоб виграти справу, оперуйте законом і фактами )

\iusr{Влад Лоза}

Сергій Іванович Запоржанин оперуємо. але і без \enquote{поезії} не обійтись, бо значна
частина справи досі - медіа-герилья, ідеологічна артикуляція і ті політичні
заяви, які ми складаємо. Надалі (дуже на це сподіваюсь) все буде переводитись у
юридичну площину. Тут вже робота для юристів, а не студентських активістів. Я
за юридичну частину справи не відповідаю, на це є окремі люди. Пост присвячений
Вченій раді, де нам вдалося зламати монологічне мовлення обвинувачів-ідеологів,
які фактами не оперують, а закон тлумачать вельми своєрідно. "Юридичного" на
цьому етапі було мало - йшлося про те, щоб наша позиція була почута як така.
Надалі - так, вступають факти і закон. Але кому, як не вам, знати, яку роль у
подібних дискусіях виконують полемічні метафори)

\iusr{Sergey Titsky}

Дело не в эмоциях, а соответствии произнесенных слов и их смысла - реально
существующему. Как довольно квалифицированный историк могу утверждать, что
лучшее средство против экзальтированно-эмоциональных и эпатажных
общественно-политических демонстраций есть абсолютное и последовательное их
игнорирование, а также - методический, отчасти рутинный, созидательный трудна
благо народу, который чаще всего, и в лучшем случае, не воздает за этот труд
ничего... Аминь!

\iusr{Сергій Іванович Запоржанин}

Sergey Titsky ну от хай професор в опалі і працює, хіба ж вона поганий науковець?

\iusr{Влад Лоза}

Sergey Titsky от я і намагаюсь, разом з кількома небайдужими, вберегти від
подібних екзальтацій той вимір, де ми могли би спокійно займатись творчою
науковою працею під керівництвом нашого вчителя. Для цього нам доводиться
вдаватись до радикальних полемічних стратегій, бо по-іншому, на жаль, ми би не
були почуті. І заважає нам у переслідуванні нашої шляхетної консервативної мети
(яка полягає лише у тому, щоб самовіддано і рутинно, як Ви зазначили, служити
народу та Істині), з одного боку, цькування радикалів, з іншого - академічний
конформізм таких, як перераховані у пості. Зверніть, увагу, пане Тицький, я не
зазначив прізвищ тих, хто був \enquote{за} - яку ще більшу послугу я можу зробити своїй
alma-mater після такого ганебного вчинку з її боку? А щодо \enquote{страху}, який я маю
відчувати - то, як сказав на засіданні один професор, \enquote{не треба мені
погрожувати}!)

\iusr{Sergey Titsky}

Влад! Ваша группа поддержки так увлекалась борьбой в защиту Евгении, что не
видите, как профессиональные промывальщики мозгов из \enquote{Евразийского
пространства} на вашей борьбе создают в информационном пространстве России и
Европы образ Мученицы за Русское Дело и Русский Мир. Если Вас устраивает роль
преданных сподвижников Русской Жанны д'Арк, угнетаемой и преследуемой
\enquote{американскими наймитами}, то продолжайте в том же духе и услышите в
конце концов обращенную к вам лично фразу \enquote{А почему это преподаватели
вашей кафедры ведут занятия не на русском языке?}, как это было в 1979 году на
нашей кафедре Киевского педагогического института имени А. М. Горького, во время
проверки общественных кафедр института комиссией от министерства высшего и
специального среднего образования СССР.

\iusr{Влад Лоза}

Sergey Titsky тоді давайте розділимо два моменти. Перший момент - це академічна
дискусія, яка може торкатись питань зовнішньополітичної орієнтації,
інформаційної війни, критики ідеологій. І другий момент - це цькування і
відсторонення від викладання на підставі поглядів. 

До Вашого відома, у мене особисто з Більченко немало розбіжностей у вельми
непростих теоретичних тонкощах, які нікого, за винятком трьох чоловік, у цій
країні не цікавлять. І ми дискутуємо про ці речі мовою високого академічного
дискурсу. 

Ви не враховуєте один неочевидний момент: Більченко створила школу, а не секту
неофітів. Об'єднала людей із дуже різними політичними, науковими, естетичними
та ін. уподобаннями на підставі спільності методологій (які ми нерідко
тлумачимо і застосовуємо по-різному) та відданості ідеалам академічної
доброчесності. 

От цю школу - поліфонічний вимір, де ще можна вести диспути і розробляти
проблематику, не будучи звинуваченим у політичній неблагонадійності - ми і
захищаємо. Не чиїсь геополітичні орієнтації, а цей невеликий майданчик, де
ведуться дискусії і пишуться праці українською мовою, і від якого не в останню
чергу залежить політичне та філософське майбутнє України. 

Натомість, опонентам вигідно представити нас як секту політично індоктринованих
інфантилів, які знаходяться під \enquote{чарами} риторики свого вчителя. Мій
вчитель не вимагає від мене, як і від інших, нічого, окрім раціональної та
послідовної логіки аргументації. При цьому вона може мати свої власні політичні
уподобання, які НЕ МОЖУТЬ НІ ЗА ЯКИХ УМОВ ставати підставою для переслідування,
звільнення, призупинення викладання. 

Принаймні, поки у нас діє Конституція, і поки конкретні переконання не
проголошені поза законом (та й навіть в останньому випадку це ще не все,
оскільки закон можна прийняти який завгодно, всупереч легітимності,
справедливості та відповідності принципам права.) 

Своїми тезами Ви фактично виправдовуєте прямі репресії за ідеологічним
критерієм - мовляв, якщо твої погляди хоча б трохи вкладаються у
\enquote{євразійський} мейнстрім, тебе зроблять вигнанцем, і це буде
справедливо. Подумайте над тим, наскільки відверто тоталітарну риторику Ви
відтворюєте, і наскільки це сумісне з Вашими власними ідеалами сумлінної
наукової роботи.

\iusr{Vladislav Petrenko}

Сергій Тицький а чем вас русский мир не устраивает. Чисто с научной точки
зрения. Есть хорошая работа у В. В. Бибихина \enquote{мир}. Русский мир
существует. Так же как украинский, французский, испанский и тд.
Мир...миро....на миру. В общем и целом не нужно повторять глупости из
телевизора. То же самое касается Малороссии...и других вещей которые почему то
подаются в качестве маркера политического исходя из идеологического нарратива.

\iusr{Александр Нечипоренко}

Сергій Тицький пане доцент, а чи не стало б пристойнішим виражати вашу думку
українською, чи це ви так хочете підтримати Більченко і не впасти в грязь. Мов
дивіться Женю, пишу російською, а патріотам цією ж мовою буду її обгаджувати,
що ж, дійсно достойно доцента НПУ

\end{itemize}

\end{itemize}
