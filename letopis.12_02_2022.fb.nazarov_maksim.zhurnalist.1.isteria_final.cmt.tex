% vim: keymap=russian-jcukenwin
%%beginhead 
 
%%file 12_02_2022.fb.nazarov_maksim.zhurnalist.1.isteria_final.cmt
%%parent 12_02_2022.fb.nazarov_maksim.zhurnalist.1.isteria_final
 
%%url 
 
%%author_id 
%%date 
 
%%tags 
%%title 
 
%%endhead 
\zzSecCmt

\begin{itemize} % {
\iusr{Олег Аптекарь}

игра игре рознь, товарищ и брат..)..как я понимаю, актеру предложили сыграть
президента за достойный недостойного гонорар.. но играть пришлось с шулерами
мирового масштаба, а такого актер не ожидал.. в финале пьесы, по всей
вероятности, придется стать жертвенным ягненком.. если ноги не сделает
вовремя..)


\iusr{Polina Zelenyak}

Когда озабоченные дяди и тети из за океана, которые и 2013-2014 гг были у
власти, озвучивают даты вторжения Путина на Украину, хочется задать именно им
вопросы. А что их всевидящая и всезнающая разведка тогда не видела и не знала,
не просчитала этого развития ситуации в Украине? Почему наш большой
\enquote{друг} Байден тогда дал команду Турчинову отдать Крым без единого
выстрела? Почему они тихонько молчали, когда Россия подтягивала войска к
Донбассу? Разве Порошенко, Яценюк, Турчинов и тд не знали, не понимали куда
приведет Украину, выбранный для нее Белым домом путь? Наши заокеанские
\enquote{друзья} и московские \enquote{враги} ведут между собой игру, а Украина
для них - игровое поле, а мы, простой народ Украины, для них лишь мячики и наша
судьба их совсем не волнует.

\begin{itemize} % {
\iusr{Виктор Счастливый}
\textbf{Polina Zelenyak} да вроде бы не Байден, а Абама...??

\iusr{Лидия Воробьева}
\textbf{Polina Zelenyak} 

Отдали Байден и Обама Крым, потому что поняли, что Россия ни за что не отдаст
Севастополь, город русских моряков, под НАТО, и испугались. Как я теперь
понимаю, мы были на грани Третьей Мировой.

\iusr{Олег Ротанов}
\textbf{Polina Zelenyak}, при чем тут знали или не знали? Изначально был такой план: превратить Украину в вечный хаос у российских границ.

\iusr{Олександр Косвінцев}

Помню русский мир в Одесі теж кричав про рюський город і рюських моряків, не
помогло запашок зажарених любителів росії аж до Києва чутно було, приємний
запах))

\end{itemize} % }

\iusr{Дмитрий Матущак}
Вангую... следующая дата нападения 29 февраля!

\begin{itemize} % {
\iusr{Viktor Rubis}
\textbf{Дмитрий Матущак} а победа шароварников и парад на Красной Площади 30-го

\iusr{Вадим Литвин}
\textbf{Дмитрий Матущак} вообще-то Псаки говорила с 29-30 февраля - в ночь

\iusr{Наталия Мельник}
\textbf{Вадим Литвин} Вася плавал, Вася знает. Псаки тоже плавала. В чем плавала, тем и попахивает
\end{itemize} % }

\iusr{Олег Амиров}

Зачем какие то домыслы ?! 14 февраля в Думе поставят на голосование признание
ДНР и ЛНР. Голосование, очевидно, будет успешным. А 15 или 16 все и начнётся.
Какие провокации, если Украина пойдет отвоевывать свои территории, а Россия
предоставит защиту признанным новосозданным государствам - согласно заключённым
накануне правительственным соглашениям !

\end{itemize} % }
