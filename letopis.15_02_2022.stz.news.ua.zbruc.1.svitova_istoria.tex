% vim: keymap=russian-jcukenwin
%%beginhead 
 
%%file 15_02_2022.stz.news.ua.zbruc.1.svitova_istoria
%%parent 15_02_2022
 
%%url https://zbruc.eu/node/110591
 
%%author_id lemko_ilko
%%date 
 
%%tags istoria,napadenie,rossia,ugroza,ukraina
%%title Наввипередки зі світовою історією
 
%%endhead 
 
\subsection{Наввипередки зі світовою історією}
\label{sec:15_02_2022.stz.news.ua.zbruc.1.svitova_istoria}
 
\Purl{https://zbruc.eu/node/110591}
\ifcmt
 author_begin
   author_id lemko_ilko
 author_end
\fi

Одного чудового ранку цієї зими всі ми, громадяни України, прокинулися у
цілковито іншій реальності. Миттєво, дуже несподівано, може якось і непомітно,
утім фатально й невідворотно ми стали частиною вільного демократичного світу.
Сидіння «на грані двох світів», за «Декалогом українського націоналіста»
Степана Ленкавського, закінчилося, бо на грані сидіти дуже й дуже незручно.
Сьогодні ця грань раптово опинилася на наших східних і північних кордонах і
збіглася з кордонами ерефії та її сателіта Білорусі. Ми стали явно вже по цей
бік барикад. Бо наша доля містка між двома світами, наша гібридність, наша
буферність, наше ні те ні се – це вже минуле, яке не виправдало якихось
найменших сподівань на щасливе майбутнє.

\ii{15_02_2022.stz.news.ua.zbruc.1.svitova_istoria.pic.1}

У перші два десятиліття новітньої незалежності нашої держави ми перебували в
омані – нам наче здавалося, що буферність України, тобто оте ні те ні се, саме
і є запорукою хоча б примарного, але примирення між двома ворожими
цивілізаціями, фактично між демократією і автократією. А виявилося, що буфер
для того й служить, аби пом’якшувати можливі удари з обох сторін, приймаючи їх
на собі.

Ми на підсвідомому рівні вірили у геополітичну містичність України, де Євразія
плавно переходить через межу Харківської чи Луганської області і, просуваючись
на захід, остаточно губиться десь на Правобережжі, а Європа так само
насувається із заходу і десь за Збручем непомітно зникає власне там, де у
неділю гарують на городах і весело «відриваються» на дискотеках у Страсний
тиждень.

У нашій уяві ворожі Європа і Євразія, обійнявшись, пристрасно кружляли у танці
Україною, а евентуальна серйозна воєнна загроза, яка могла б потенційно
виникнути, швидко розсмоктувалась десь в українських степах по обидва боки
Дніпра... Та це була лише наша примарна ілюзія, бо рано чи пізно ворожість двох
цивілізацій мала подолати уявне примирення, уявну консолідацію, і відлік нової
конфронтації розпочався у березні 2014-го...  

Зараз ми вже невідворотно тут, у Європі, а наше минуле залишилося там. Гнана
російським багнетом у спину, Україна швидко побігла попереду паровоза світової
історії наввипередки зі світовою історією. Це, безумовно, радісна подія, що ми
стали частиною цивілізованого світу, стали нею з нашою різною, але завжди
недолугою владою, з нашими нерозпочатими і розпочатими, але не завершеними
реформами, з неподоланою корупцією, з необраним керівником САП, з усіма
трухіними, дубінськими, бужанськими, з усією рідною і нерідною мерзотою. Ми
вистраждали стати такими собі неофітами вільного світу, незаперечною його
частиною, не ставши поки що ані членами ЄС, ані членами НАТО. Такий от
подарунок долі.

Бо колективний Захід нарешті усвідомив істину, так стисло й точно висловлену
нашою поетесою Ліною Костенко:

\begin{zznagolos}
\obeycr
І жах, і кров, і смерть, і відчай,
І клекіт хижої орди,
Маленький сірий чоловічок
Накоїв чорної біди.
Це звір огидної породи,
Лох-Несс холодної Неви.
Куди ж ви дивитесь, народи?!
Сьогодні ми, а завтра — ви.	
\restorecr
\end{zznagolos}

І тому нарешті сьогодні вільний світ тою чи іншою мірою, але все ж здебільшого
за нас, ми сьогодні маємо від нього таку політичну, фінансову, військову
підтримку, про яку не могли навіть і мріяти рік чи навіть пів року тому. Також
не можемо не подякувати й тій самій Росії під керівництвом того самого Путіна
за небачену за всю історію України консолідацію українського суспільства – як
Заходу, так і Сходу, за шалену русофобію більшості українців (і за стриманішу
русофобію меншості – стриманіше, але теж подякувати), за патріотизм і героїзм
українського війська, за відданість волонтерів, за усвідомлення тою незначною
частиною наших громадян, які плекали ще якісь наївні ілюзії про миролюбність
віроломного сусіди, того незаперечного факту, що ми вже стовідсотково ніколи не
будемо «братьями».

І ось нам треба бігти чимдуж попереду згаданого паровоза історії, блискавично
використовувати цю вкрай сприятливу нагоду, це відчинене вікно можливостей,
поки в нього не влетіла російська міна чи чергова криза Заходу. Посилення
обороноздатності армії, зміцнення тероборони. І шаленими темпами – реформи,
реформи, реформи! Як сьогодні вкрай актуально звучить істина, вербалізована
багатьма українцями ще на початку цієї війни – страшніше від Путіна може бути
лише наша повальна корупція, бо вона органічно долучає нас до «рускава міра», і
вочевидь Путін, коли писав про «адіннарот», мав на увазі: «ви такі ж
корупціонери, злодії і нездари, як і ми, куди ж вам до Європи, там не крадуть і
працюють, а вам це треба? Тож приєднуйтесь краще до нас». І беззаперечно
головним критерієм нашої остаточної і безповоротної відірваності від жахливого
рускава міра і долучення до омріяної Європи будуть ефективно проведені реформи
і подолання корупції. Без цього нам до Європи зась.

\ii{15_02_2022.stz.news.ua.zbruc.1.svitova_istoria.pic.2}

Сталося так, що на нинішньому цивілізаційному тектонічному розламі наші
національні цінності цілковито збіглися з цінностями загальнолюдськими. На нашу
одвічну боротьбу з Московією за право бути вільними, за нашу мову, культуру, за
наші звичаї наклалася загальноєвропейська система координат. Зараз ми на
передньому краї, на передку, на нулику світової боротьби не на життя, а на
смерть, за демократію проти авторитаризму і тоталітаризму, за прогрес проти
регресу, за людські цінності проти нелюдськості, за життя проти смерті.
Надзвичайно важка й водночас почесна місія. Це наша історична карма. Європа –
або смерть! Відступати нам нікуди, бо за нашими плечима – весь цивілізований
світ, бо справді, якщо впадемо «сьогодні ми, то завтра – вони». 
