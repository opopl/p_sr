% vim: keymap=russian-jcukenwin
%%beginhead 
 
%%file 07_12_2020.fb.nazar_volodymyr.1.z_chogo_pochynalas_moskovia
%%parent 07_12_2020
 
%%url https://www.facebook.com/permalink.php?story_fbid=749633678976377&id=100017892540365
 
%%author Назар, Володимир
%%author_id nazar_volodymyr
%%author_url 
 
%%tags 
%%title З чого починалась Московія? З гнилих і смердючих боліт
 
%%endhead 
 
\subsection{З чого починалась Московія? З гнилих і смердючих боліт}
\label{sec:07_12_2020.fb.nazar_volodymyr.1.z_chogo_pochynalas_moskovia}
\Purl{https://www.facebook.com/permalink.php?story_fbid=749633678976377&id=100017892540365}
\ifcmt
	author_begin
   author_id nazar_volodymyr
	author_end
\fi

\ifcmt
pic https://scontent.fiev6-1.fna.fbcdn.net/v/t1.0-9/130663209_749632858976459_3641937653415061740_o.jpg?_nc_cat=110&ccb=2&_nc_sid=730e14&_nc_ohc=RAesAv3CqxQAX9SsVHf&_nc_ht=scontent.fiev6-1.fna&oh=74d8685a4543ba8a8a13fd7b8f377cab&oe=5FF57B0E
\fi

З чого починалась Московія? З гнилих і смердючих боліт. \enquote{Брехнею є те, що
Москва заснована Юрієм Довгоруким у 1147 р. Це міф, який не має доказового
підтвердження. Москва як поселення була заснована 1272 року. Цього ж року був
проведений третій перепис населення Золотої Орди. При першому переписі
(1237–1238 pp.) і другому (1254–1259 pp.) поселення – Москва не згадується.}
історик Я. Дашкевич.  Довідка: МОКВА́, и, жін., діал.

\begin{itemize}
\item 1. Мокротеча. Надворі була негода: моква, туман, темрява (Степан Чорнобривець, Визволена земля, 1950р.)
\item 2. Низина, залита водою. Ішло вояцтво, братство ратних діл, І сходило із пагорба на діл, Шикуючись у лави бойові На оболоні, на грузькій мокві (Микола Бажан) Прилягти нема де — під ногами моква, болото (Олесь Гончар, Тронка, 1963р.)
\item 3. Мокрота, сирість. Гончар цілий день у мокві робе (Словник Грінченка);  Ото воно гниє і далі і смердить весь час.
\end{itemize}

\begin{itemize}
\item \fbusr{Олександр}
Там, де я народився(Велика Писарівка, Сумської) біля Ворскли є місце, яке завжди називали - Московське. На мої питання була відповідь: за річкою живуть москалі...
Протягом останнього року провів дефрагментацію мовного контенту і побачив таку картину:
Наша мова тривимірна(вимова, письмо, тлумачення) тож підлягає іншому адмініструванню.
Тож Московське є кінцем(мисом) острівця, який утворюється під час весняної повені, початок звуть Тітаренкове.
Тож Москва - це назва пагорбу під кремлем та Красною площею.
Красна бо вранішне сонце світить саме на неї.
Всі ці назви створені українськими колоністами, вірніше - русьетнос!!!
Пізніше руськолонії потрапили під контроль злодіїв який існує посьогодні
\end{itemize}
