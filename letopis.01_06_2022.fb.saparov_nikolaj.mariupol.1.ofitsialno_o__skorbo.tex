%%beginhead 
 
%%file 01_06_2022.fb.saparov_nikolaj.mariupol.1.ofitsialno_o__skorbo
%%parent 01_06_2022
 
%%url https://www.facebook.com/permalink.php?story_fbid=pfbid0j3sAghj68mvrTpBEmDMjLBnkFaGUtjZuTKiwKDhBbfUpovkRgN1j2pNAbmuzpMb3l&id=100015424741499
 
%%author_id saparov_nikolaj.mariupol
%%date 01_06_2022
 
%%tags mariupol
%%title Официально о «Скорботе» - Поделитесь - важно для родственников умерших
 
%%endhead 

\subsection{Официально о \enquote{Скорботе} - Поделитесь - важно для родственников умерших}
\label{sec:01_06_2022.fb.saparov_nikolaj.mariupol.1.ofitsialno_o__skorbo}

\Purl{https://www.facebook.com/permalink.php?story_fbid=pfbid0j3sAghj68mvrTpBEmDMjLBnkFaGUtjZuTKiwKDhBbfUpovkRgN1j2pNAbmuzpMb3l&id=100015424741499}
\ifcmt
 author_begin
   author_id saparov_nikolaj.mariupol
 author_end
\fi

Здравствуйте, друзья. 

Рад всем, кто выжил и в безопасности!

Официально о «Скорботе».             

Поделитесь - важно для родственников умерших.

Как и большинство зданий и помещений в городе, 70\% наших- тоже или разрушены
или сгорели. Мы не работаем с середины марта. После того,как нашему
сотруднику оторвало ноги, мы уже не могли рисковать жизнями живых во имя
мертвых. 

Кто бы сейчас не представлялся «Скорботой»,  это не мы! Мы не работаем!
Совсем! Впервые, за 25 лет...

На момент прекращения работы, в Похоронном доме , на Володарском шоссе
находилось 16 тел и около 20умерших-в цеху ,на улице Торговой,65-а.Понятно, что
из-за постоянных обстрелов, захоронить их было негде, да и некому. Тогда, в тот
момент, это был единственный способ хранить тела. Было холодно и, в то
время, ещё под охраной. Ни кто и предположить не мог, что боевые действия
растянутся на два месяца.

В течении которых не были захоронены эти тела. 

Так подробно рассказываю, потому что люди, которые не были в то время в
Мариуполе, просто не понимают, как это - нельзя взять и отвезти на кладбище
умершего. Не то что нельзя было, а смертельно опасно!!! Поэтому и хоронили, кто
где: от парков до разделительных полос на дорогах.

По моим ,лично не проверенным, но заслуживающим доверия данным, тела из
Похоронного дома захоронили за его забором с левой стороны от  центрального
входа. 8 мужчин и 8 женщин. Тела,которые хранились на улице Торговой,
постигла более ужасная участь. В один из дней туда прилетело. Сгорел
цех, вместе телами и техникой. Состояние тел, можете представить. В конце
апреля, начале мая, туда заселились ДНР. Они, видимо, и вызвали своё МЧС. Тела
были вывезены в неизвестном направлении и захоронены. Судьба их мне пока не
известна.

Некоторые тела хранились в мешках, это те люди, которых собирали с улиц.

Некоторые в гробах, это близкие наших заказчиков. Все были с сопроводительными
документами. Всё сгорело. Жестко.

За захоронение последних, которые в гробах, были уплачены деньги. Мы ни в коем
случае не отказываемся от своего материального долга перед родственниками.
Порядок и способ возврата я объясню лично, каждому в мессенджерах.

Что со «Скорботой»? Она как и большая часть города разрушена...    

Часть сотрудников - в городе, других разбросало, как и всех мариупольцев.

Имущество, техника, - что разграбили, что сгорело, что мы сами пораздавали для захоронений. 

Мы, как и все жители - с разбитой жизнью и потерей любимой работы.

Я лично, и часть сотрудников пытаемся собирать в кучу беженцев и
переселенцев. Наша диаспора мариупольцев (и не только) - беглецов очень
большая. Уверен, что только вместе мы сможем адаптироваться и приспособиться
к новой жизни...

О новом проекте напишу в другой раз.    Знаю точно, что руки опускать
нельзя, нужно держаться вместе и друг другу помогать!!!  Тогда обязательно
выползем из этого пиз.. реза!!!

%\ii{01_06_2022.fb.saparov_nikolaj.mariupol.1.ofitsialno_o__skorbo.cmt}
