% vim: keymap=russian-jcukenwin
%%beginhead 
 
%%file 02_08_2021.fb.bryhar_sergej.1.lvov_jazyk.cmt
%%parent 02_08_2021.fb.bryhar_sergej.1.lvov_jazyk
 
%%url 
 
%%author 
%%author_id 
%%author_url 
 
%%tags 
%%title 
 
%%endhead 
\subsubsection{Коментарі}

\begin{itemize}
%%%fbauth
%%%fbauth_name
\iusr{Українською Будь Ласка}
%%%fbauth_url
%%%fbauth_place
%%%fbauth_id
%%%fbauth_front
%%%fbauth_desc
%%%fbauth_www
%%%fbauth_pic
%%%fbauth_pic portrait
%%%fbauth_pic background
%%%fbauth_pic other
%%%fbauth_tags
%%%fbauth_pubs
%%%endfbauth
 

так, галичани добряче підсмоктують в узкоязиких.. а в києві намагаються
переходити на рашоязик. це особливо огидно. та що казать — в тернополі на
фестивалі днями з успіхом гастролювали москвинські артисти, а у львові москвини
взагалі збирають стадіони з сольними концертами. навіть у києві їм не вдається
взяти такий масштаб (hammali \& navai)


%%%fbauth
%%%fbauth_name
\iusr{Олександр Кондратюк}
%%%fbauth_url
%%%fbauth_place
%%%fbauth_id
%%%fbauth_front
%%%fbauth_desc
%%%fbauth_www
%%%fbauth_pic
%%%fbauth_pic portrait
%%%fbauth_pic background
%%%fbauth_pic other
%%%fbauth_tags
%%%fbauth_pubs
%%%endfbauth
 

Складається враження., що пересічні галичани трохи розслаблені, бо їх
(україномовних) поки більшість, і саме тому вони не відчувають і не розуміють
загрози зросійщення. Ризикують бути непомітно звареними на малому вогні.

\begin{itemize}
%%%fbauth
%%%fbauth_name
\iusr{Українською Будь Ласка}
%%%fbauth_url
%%%fbauth_place
%%%fbauth_id
%%%fbauth_front
%%%fbauth_desc
%%%fbauth_www
%%%fbauth_pic
%%%fbauth_pic portrait
%%%fbauth_pic background
%%%fbauth_pic other
%%%fbauth_tags
%%%fbauth_pubs
%%%endfbauth
 

\textbf{Олександр} юні та молоді галичани всі тотально слухають москвинський музон, а
малим дітям ставлять москвинські мультики (маша і мєдвєдь, фіксікі). тож це не
просто розслабленість. і мова потихеньку обростає русизмами


%%%fbauth
%%%fbauth_name
\iusr{Олександр Кондратюк}
%%%fbauth_url
%%%fbauth_place
%%%fbauth_id
%%%fbauth_front
%%%fbauth_desc
%%%fbauth_www
%%%fbauth_pic
%%%fbauth_pic portrait
%%%fbauth_pic background
%%%fbauth_pic other
%%%fbauth_tags
%%%fbauth_pubs
%%%endfbauth
 
\textbf{Олег Олег} це вже — наслідки розслабленості. Кияни змушені бути
зубатішими задля створення навколо себе україномовного середовища. Або
вибірковішими. Львів'яни мають мовний комфорт за замовчанням, без додаткових
зусиль. Поступове розмиття середовища російськомовним елементом відбувається
непомітно. Встигають звикнути. І за рахунок міграції, і за рахунок зросійщених
мас-медіа. Те, що дісталося без зусиль — менше цінується і захищається, ніж те,
що відвойовано тобою.

%%%fbauth
%%%fbauth_name
\iusr{Serhii Bryhar}
%%%fbauth_url
%%%fbauth_place
%%%fbauth_id
%%%fbauth_front
%%%fbauth_desc
%%%fbauth_www
%%%fbauth_pic
%%%fbauth_pic portrait
%%%fbauth_pic background
%%%fbauth_pic other
%%%fbauth_tags
%%%fbauth_pubs
%%%endfbauth
 

От русизми... Мене дратують, а місцеві кажуть: "не лізь у це, немає сенсу, в
нас кажуть саме так...". Тобто, по ідеї, я мав прийняти "шаріки", "жесть",
"качєльки", "топор", "відділ продАЖ"... Але я з моїм досвідом... Усе постійно
виправляю(. Навіть підсвідомо.


%%%fbauth
%%%fbauth_name
\iusr{Volodymyr Mylenko}
%%%fbauth_url
%%%fbauth_place
%%%fbauth_id
%%%fbauth_front
%%%fbauth_desc
%%%fbauth_www
%%%fbauth_pic
%%%fbauth_pic portrait
%%%fbauth_pic background
%%%fbauth_pic other
%%%fbauth_tags
%%%fbauth_pubs
%%%endfbauth
 
\textbf{Олександр Кондратюк} так і є. не розуміють. коли я розказував про
запорізькі мовні реалії - то як мінімум дуже дивувалися, а то і просто з
недовірою слухали

%%%fbauth
%%%fbauth_name
\iusr{Solomia Shtephan}
%%%fbauth_url
%%%fbauth_place
%%%fbauth_id
%%%fbauth_front
%%%fbauth_desc
%%%fbauth_www
%%%fbauth_pic
%%%fbauth_pic portrait
%%%fbauth_pic background
%%%fbauth_pic other
%%%fbauth_tags
%%%fbauth_pubs
%%%endfbauth
 
\textbf{Олександр Кондратюк} не "за замовчуванням", а "за початкових умов", або "без додаткових вказівок", або "без додаткових умов"....

%%%fbauth
%%%fbauth_name
\iusr{Олександр Кондратюк}
%%%fbauth_url
%%%fbauth_place
%%%fbauth_id
%%%fbauth_front
%%%fbauth_desc
%%%fbauth_www
%%%fbauth_pic
%%%fbauth_pic portrait
%%%fbauth_pic background
%%%fbauth_pic other
%%%fbauth_tags
%%%fbauth_pubs
%%%endfbauth
 
\textbf{Solomia Shtephan} дякую за виправлення, візьму до уваги ці конструкції. Моя українська далеко не перфектна — вчив її самостійно в дорослому вже віці.

%%%fbauth
%%%fbauth_name
\iusr{Українською Будь Ласка}
%%%fbauth_url
%%%fbauth_place
%%%fbauth_id
%%%fbauth_front
%%%fbauth_desc
%%%fbauth_www
%%%fbauth_pic
%%%fbauth_pic portrait
%%%fbauth_pic background
%%%fbauth_pic other
%%%fbauth_tags
%%%fbauth_pubs
%%%endfbauth
 
\textbf{Solomia} якось криво і навантажено звучить/читається. є елегантніші пропозиції?

%%%fbauth
%%%fbauth_name
\iusr{Solomia Shtephan}
%%%fbauth_url
%%%fbauth_place
%%%fbauth_id
%%%fbauth_front
%%%fbauth_desc
%%%fbauth_www
%%%fbauth_pic
%%%fbauth_pic portrait
%%%fbauth_pic background
%%%fbauth_pic other
%%%fbauth_tags
%%%fbauth_pubs
%%%endfbauth
 
\textbf{Олександр Кондратюк} дякую за те, що вивчали та зараз так сприймаєте поради щодо мови. Я теж дякую усім за подібні поради. Зростаймо разом, творімо українське слово...

%%%fbauth
%%%fbauth_name
\iusr{Олександр Кондратюк}
%%%fbauth_url
%%%fbauth_place
%%%fbauth_id
%%%fbauth_front
%%%fbauth_desc
%%%fbauth_www
%%%fbauth_pic
%%%fbauth_pic portrait
%%%fbauth_pic background
%%%fbauth_pic other
%%%fbauth_tags
%%%fbauth_pubs
%%%endfbauth
 
\textbf{Solomia Shtephan} 

але я готовий трохи подискутувати:

варіант "без додаткових вказівок" не передає того, що я хотів сказати. Якщо "за
замовчанням" — русизм, "без додаткових вказівок" — не зовсім адекватний
переклад того русизму.

"Без додаткових умов" — також не зовсім те.

"За початкових умов" — імовірно, коректно передає суть, але звучить трохи
офіційно.

Могло би бути "Без додаткових зусиль", але також є якийсь нюанс, який я
відчуваю, але не можу передати словами.

Коротше, ідеально-еквівалентного перекладу на українську російському
"по-умолчанию" я досі не знайшов для себе.

%%%fbauth
%%%fbauth_name
\iusr{Тетяна Лук'янова}
%%%fbauth_url
%%%fbauth_place
%%%fbauth_id
%%%fbauth_front
%%%fbauth_desc
%%%fbauth_www
%%%fbauth_pic
%%%fbauth_pic portrait
%%%fbauth_pic background
%%%fbauth_pic other
%%%fbauth_tags
%%%fbauth_pubs
%%%endfbauth
 
\textbf{Solomia Shtephan} , теж дякую, бо зловживаю цією фразою.

%%%fbauth
%%%fbauth_name
\iusr{Олександр Кондратюк}
%%%fbauth_url
%%%fbauth_place
%%%fbauth_id
%%%fbauth_front
%%%fbauth_desc
%%%fbauth_www
%%%fbauth_pic
%%%fbauth_pic portrait
%%%fbauth_pic background
%%%fbauth_pic other
%%%fbauth_tags
%%%fbauth_pubs
%%%endfbauth
 
\textbf{Solomia Shtephan} 

Погугливши, знайшов цікавий текст на цю тему. Я також сприймаю цю конструкцію
як переклад англійського комп'ютерного терміну "default; by default".

\url{http://ena.lp.edu.ua/bitstream/ntb/2646/1/18.pdf}

Ще раз дякую, що спонукали до пошуку.

%%%fbauth
%%%fbauth_name
\iusr{Lina Kacmar}
%%%fbauth_url
%%%fbauth_place
%%%fbauth_id
%%%fbauth_front
%%%fbauth_desc
%%%fbauth_www
%%%fbauth_pic
%%%fbauth_pic portrait
%%%fbauth_pic background
%%%fbauth_pic other
%%%fbauth_tags
%%%fbauth_pubs
%%%endfbauth
 
\textbf{Олександр Кондратюк} українською це буде «само-собою», або «по замоченню».

\end{itemize}

%%%fbauth
%%%fbauth_name
\iusr{Олена Калмикова}
%%%fbauth_url
%%%fbauth_place
%%%fbauth_id
%%%fbauth_front
%%%fbauth_desc
%%%fbauth_www
%%%fbauth_pic
%%%fbauth_pic portrait
%%%fbauth_pic background
%%%fbauth_pic other
%%%fbauth_tags
%%%fbauth_pubs
%%%endfbauth
 

Місцеві львів'яни казали мені, що донецькі москвороті ,ЩЕ ДО ВІЙНИ, скуповували
у Львові квартири. То нема чого дивуватися. ЗА совка собі неможливо було таке
уявити

\begin{itemize}
%%%fbauth
%%%fbauth_name
\iusr{Олександр Кондратюк}
%%%fbauth_url
%%%fbauth_place
%%%fbauth_id
%%%fbauth_front
%%%fbauth_desc
%%%fbauth_www
%%%fbauth_pic
%%%fbauth_pic portrait
%%%fbauth_pic background
%%%fbauth_pic other
%%%fbauth_tags
%%%fbauth_pubs
%%%endfbauth
 
\textbf{Олена Калмикова} проблема в тому, шо вони там не асимілюються через українізацію.

%%%fbauth
%%%fbauth_name
\iusr{Дарія Матвіїв-Веклин}
%%%fbauth_url
%%%fbauth_place
%%%fbauth_id
%%%fbauth_front
%%%fbauth_desc
%%%fbauth_www
%%%fbauth_pic
%%%fbauth_pic portrait
%%%fbauth_pic background
%%%fbauth_pic other
%%%fbauth_tags
%%%fbauth_pubs
%%%endfbauth
 
\textbf{Олександр Кондратюк} бо така толерантна до них була завжди політи ка держави.

%%%fbauth
%%%fbauth_name
\iusr{Олена Калмикова}
%%%fbauth_url
%%%fbauth_place
%%%fbauth_id
%%%fbauth_front
%%%fbauth_desc
%%%fbauth_www
%%%fbauth_pic
%%%fbauth_pic portrait
%%%fbauth_pic background
%%%fbauth_pic other
%%%fbauth_tags
%%%fbauth_pubs
%%%endfbauth
 
\textbf{Олександр Кондратюк} Бо треба жорстко вимагати знання мови та історії України, як для іноземців. Як не отримають роботу через незнання мови, то махом вивчать ВСЕ, що треба.

%%%fbauth
%%%fbauth_name
\iusr{Дарія Матвіїв-Веклин}
%%%fbauth_url
%%%fbauth_place
%%%fbauth_id
%%%fbauth_front
%%%fbauth_desc
%%%fbauth_www
%%%fbauth_pic
%%%fbauth_pic portrait
%%%fbauth_pic background
%%%fbauth_pic other
%%%fbauth_tags
%%%fbauth_pubs
%%%endfbauth
 
\textbf{Олена Калмикова} і якщо прийдеться вчити дітей за власний рахунок

%%%fbauth
%%%fbauth_name
\iusr{Олена Калмикова}
%%%fbauth_url
%%%fbauth_place
%%%fbauth_id
%%%fbauth_front
%%%fbauth_desc
%%%fbauth_www
%%%fbauth_pic
%%%fbauth_pic portrait
%%%fbauth_pic background
%%%fbauth_pic other
%%%fbauth_tags
%%%fbauth_pubs
%%%endfbauth
 
\textbf{Дарія Матвіїв-Веклин} ОТО Ж!
\end{itemize}

%%%fbauth
%%%fbauth_name
\iusr{Лиза Ребар}
%%%fbauth_url
%%%fbauth_place
%%%fbauth_id
%%%fbauth_front
%%%fbauth_desc
%%%fbauth_www
%%%fbauth_pic
%%%fbauth_pic portrait
%%%fbauth_pic background
%%%fbauth_pic other
%%%fbauth_tags
%%%fbauth_pubs
%%%endfbauth
 
Другий день в Одесі я балакаю українською, звісно 2 тижні мені насрати, що вони про мене думають

\begin{itemize}
%%%fbauth
%%%fbauth_name
\iusr{Олександр Кондратюк}
%%%fbauth_url
%%%fbauth_place
%%%fbauth_id
%%%fbauth_front
%%%fbauth_desc
%%%fbauth_www
%%%fbauth_pic
%%%fbauth_pic portrait
%%%fbauth_pic background
%%%fbauth_pic other
%%%fbauth_tags
%%%fbauth_pubs
%%%endfbauth
 
\textbf{Ліза Ребар} лише так і треба. Я до війни щороку, від самих 80-хх по
кілька разів на рік бував на Донеччині, в тій її частині, що зараз під
окупацією. Спілкувався українською. Ніколи не мав з цим проблем. Востаннє — на
різдво 2014-го гуляв вулицями вечірнього Донецька, розмовляв українською як із
супутницею, так і з місцевими незнайомими мешканцями — без проблем. Всі нас
розуміли. Вовком не дивилися.

%%%fbauth
%%%fbauth_name
\iusr{Олег Щербан}
%%%fbauth_url
%%%fbauth_place
%%%fbauth_id
%%%fbauth_front
%%%fbauth_desc
%%%fbauth_www
%%%fbauth_pic
%%%fbauth_pic portrait
%%%fbauth_pic background
%%%fbauth_pic other
%%%fbauth_tags
%%%fbauth_pubs
%%%endfbauth
 
\textbf{Ліза Ребар} в обслуговування надається українською? Переходять на вимогу?

%%%fbauth
%%%fbauth_name
\iusr{Лиза Ребар}
%%%fbauth_url
%%%fbauth_place
%%%fbauth_id
%%%fbauth_front
%%%fbauth_desc
%%%fbauth_www
%%%fbauth_pic
%%%fbauth_pic portrait
%%%fbauth_pic background
%%%fbauth_pic other
%%%fbauth_tags
%%%fbauth_pubs
%%%endfbauth
 
\textbf{Олег Щербан} У Сільпо на Палія

%%%fbauth
%%%fbauth_name
\iusr{Олег Щербан}
%%%fbauth_url
%%%fbauth_place
%%%fbauth_id
%%%fbauth_front
%%%fbauth_desc
%%%fbauth_www
%%%fbauth_pic
%%%fbauth_pic portrait
%%%fbauth_pic background
%%%fbauth_pic other
%%%fbauth_tags
%%%fbauth_pubs
%%%endfbauth
 
\textbf{Ліза Ребар} а в інших місцях?

%%%fbauth
%%%fbauth_name
\iusr{Лиза Ребар}
%%%fbauth_url
%%%fbauth_place
%%%fbauth_id
%%%fbauth_front
%%%fbauth_desc
%%%fbauth_www
%%%fbauth_pic
%%%fbauth_pic portrait
%%%fbauth_pic background
%%%fbauth_pic other
%%%fbauth_tags
%%%fbauth_pubs
%%%endfbauth
 
\textbf{Олег Щербан} ні, або язик, або якась мішанина
\end{itemize}

%%%fbauth
%%%fbauth_name
\iusr{Сергей Миколаенко}
%%%fbauth_url
%%%fbauth_place
%%%fbauth_id
%%%fbauth_front
%%%fbauth_desc
%%%fbauth_www
%%%fbauth_pic
%%%fbauth_pic portrait
%%%fbauth_pic background
%%%fbauth_pic other
%%%fbauth_tags
%%%fbauth_pubs
%%%endfbauth
 

Без заходів з боку держави протистояти ізику важко, бо пропаганда працює. А до
державних органів бидло обрало кого? І змінити ситуацію надважке завдання, бо
бидлу питання його ідентичності не може бути цікавим апріорі.

\begin{itemize}
%%%fbauth
%%%fbauth_name
\iusr{Олександр Кондратюк}
%%%fbauth_url
%%%fbauth_place
%%%fbauth_id
%%%fbauth_front
%%%fbauth_desc
%%%fbauth_www
%%%fbauth_pic
%%%fbauth_pic portrait
%%%fbauth_pic background
%%%fbauth_pic other
%%%fbauth_tags
%%%fbauth_pubs
%%%endfbauth
 
\textbf{Serhij Mykolaienko} Держава трохи згаяла час і упустила слушний момент для початку українізації: мовний закон треба було приймати році так в 2015-му, раз вже побоялися зобити це 2014-го. За 4 роки до наступних виборів вже мали б позитивні наслідки від його запровадження.

%%%fbauth
%%%fbauth_name
\iusr{Олег Кобизський}
%%%fbauth_url
%%%fbauth_place
%%%fbauth_id
%%%fbauth_front
%%%fbauth_desc
%%%fbauth_www
%%%fbauth_pic
%%%fbauth_pic portrait
%%%fbauth_pic background
%%%fbauth_pic other
%%%fbauth_tags
%%%fbauth_pubs
%%%endfbauth
 
\textbf{Олександр Кондратюк} Бо сивочолий гадав собі преференції перед виборами таким чином вибити,щоб мати свіжіший ефект від україноцентричної частини суспільства не спрацювало.

%%%fbauth
%%%fbauth_name
\iusr{Олександр Кондратюк}
%%%fbauth_url
%%%fbauth_place
%%%fbauth_id
%%%fbauth_front
%%%fbauth_desc
%%%fbauth_www
%%%fbauth_pic
%%%fbauth_pic portrait
%%%fbauth_pic background
%%%fbauth_pic other
%%%fbauth_tags
%%%fbauth_pubs
%%%endfbauth
 
\textbf{Олег Кобизський} Це було однією з серйозних помилок Порошенка. Але зі зневажливим прізсвиськом "Сивочолий" я не згоден: при всіх своїх помилках Порошенко виявився найефективнішим з усіх українських президентів. Тому не заслуговує на зневагу.

%%%fbauth
%%%fbauth_name
\iusr{Олег Кобизський}
%%%fbauth_url
%%%fbauth_place
%%%fbauth_id
%%%fbauth_front
%%%fbauth_desc
%%%fbauth_www
%%%fbauth_pic
%%%fbauth_pic portrait
%%%fbauth_pic background
%%%fbauth_pic other
%%%fbauth_tags
%%%fbauth_pubs
%%%endfbauth
 
\textbf{Олександр Кондратюк} ті "помилки" чітко прораховані дії які в першу чергу були спрямовані на власну вигоду.

%%%fbauth
%%%fbauth_name
\iusr{Олександр Кондратюк}
%%%fbauth_url
%%%fbauth_place
%%%fbauth_id
%%%fbauth_front
%%%fbauth_desc
%%%fbauth_www
%%%fbauth_pic
%%%fbauth_pic portrait
%%%fbauth_pic background
%%%fbauth_pic other
%%%fbauth_tags
%%%fbauth_pubs
%%%endfbauth
 
\textbf{Олег Кобизський} Не згоден з вашою оцінкою. Вважаю її упередженою.

%%%fbauth
%%%fbauth_name
\iusr{Lina Kacmar}
%%%fbauth_url
%%%fbauth_place
%%%fbauth_id
%%%fbauth_front
%%%fbauth_desc
%%%fbauth_www
%%%fbauth_pic
%%%fbauth_pic portrait
%%%fbauth_pic background
%%%fbauth_pic other
%%%fbauth_tags
%%%fbauth_pubs
%%%endfbauth
 
\textbf{Олександр Кондратюк} 

це в нього не помилки, а свідома українофобська політика. Дивуюся, як
нормальні, розумні люди цього не розуміють? Чи просто не слідкували за його
«руками»? І зверніть увагу на те, який він називає справжній рейтинг
українськості, бо хто-хто, а він точно знає справжній рейтинг по країні між
українцями і руцкамірцамі! І на те, як він сушив собі голову, щоб і вівці були
цілі, і вовки ситі? Подумайте добре над його словами і зрозумієте - це помилки,
чи свідома ціль затвердити в Україні руцкій мір?

\href{https://www.youtube.com/watch?v=awSp0bvTMqw}{%
порошенко защитник русского языка, Елена Лукаш, youtube, 20.07.2019%
}

\end{itemize}

%%%fbauth
%%%fbauth_name
\iusr{Павло Шубарт}
%%%fbauth_url
%%%fbauth_place
%%%fbauth_id
%%%fbauth_front
%%%fbauth_desc
%%%fbauth_www
%%%fbauth_pic
%%%fbauth_pic portrait
%%%fbauth_pic background
%%%fbauth_pic other
%%%fbauth_tags
%%%fbauth_pubs
%%%endfbauth
 

Не ідеалізуйте галичан. Вони варяться від 1939 року з усіма Українцями в одному
болоті. А за понад 80 років і з галичанина толераста легко зробити можна.
Тернопільські діти віком до 10 років сьогодні розмовляють жахливим суржиком,
хоча в школі московської мови не вивчають. Інтернет робить свою справу. І
батьки цьому аж ніяк не перешкоджають. Бо самі батьки, обравши клоуна за
президента, сповідують принцип какойразніци.

\begin{itemize}
%%%fbauth
%%%fbauth_name
\iusr{Serhii Bryhar}
%%%fbauth_url
%%%fbauth_place
%%%fbauth_id
%%%fbauth_front
%%%fbauth_desc
%%%fbauth_www
%%%fbauth_pic
%%%fbauth_pic portrait
%%%fbauth_pic background
%%%fbauth_pic other
%%%fbauth_tags
%%%fbauth_pubs
%%%endfbauth
 

Але то все землі (і Тернопіль - теж), де за клоуна загалом проголосували
найменше виборців. Теж достобіса, але найменше. Це перше. А друге - я достатньо
предметно вивчив середовище, щоб не мати ілюзій. Тобто, думаю, ані недооцінки,
ані переоцінки зараз уже бути не може. 

Просто маю висновок, який мені дуже не
подобається: бути україномовним одеситом складніше, ніж російськомовним
львів'янином.


%%%fbauth
%%%fbauth_name
\iusr{Павло Шубарт}
%%%fbauth_url
%%%fbauth_place
%%%fbauth_id
%%%fbauth_front
%%%fbauth_desc
%%%fbauth_www
%%%fbauth_pic
%%%fbauth_pic portrait
%%%fbauth_pic background
%%%fbauth_pic other
%%%fbauth_tags
%%%fbauth_pubs
%%%endfbauth
 
\textbf{Serhii Bryhar} ваш висновок свідчить про те, що українська мова в
Україні має загрожені позиції і за кілька поколінь може перетворитися на
артефакт.

%%%fbauth
%%%fbauth_name
\iusr{Дарія Матвіїв-Веклин}
%%%fbauth_url
%%%fbauth_place
%%%fbauth_id
%%%fbauth_front
%%%fbauth_desc
%%%fbauth_www
%%%fbauth_pic
%%%fbauth_pic portrait
%%%fbauth_pic background
%%%fbauth_pic other
%%%fbauth_tags
%%%fbauth_pubs
%%%endfbauth
 
\textbf{Павло Шубарт} що Ви сплітаєте? Ще скажіть, що молодь, яка приїжджає зі
сіл на навчання до Тернополя зразу ж починає ізиком розмовляти.

%%%fbauth
%%%fbauth_name
\iusr{Лиза Ребар}
%%%fbauth_url
%%%fbauth_place
%%%fbauth_id
%%%fbauth_front
%%%fbauth_desc
%%%fbauth_www
%%%fbauth_pic
%%%fbauth_pic portrait
%%%fbauth_pic background
%%%fbauth_pic other
%%%fbauth_tags
%%%fbauth_pubs
%%%endfbauth
 
\textbf{Павло Шубарт} усе залежить від батьків, так, моя дочка дивиться Тік Ток ,але я завжди виправляю іншомовні слова

%%%fbauth
%%%fbauth_name
\iusr{Павло Шубарт}
%%%fbauth_url
%%%fbauth_place
%%%fbauth_id
%%%fbauth_front
%%%fbauth_desc
%%%fbauth_www
%%%fbauth_pic
%%%fbauth_pic portrait
%%%fbauth_pic background
%%%fbauth_pic other
%%%fbauth_tags
%%%fbauth_pubs
%%%endfbauth
 
\textbf{Дарія Матвіїв-Веклин} 

я вам скажу інакше. Українська мова в Тернополі тримається завдяки напливу до
міста селян. Але цей ресурс обмежений, а тому московщення Тернополя триватиме.
І перший етап в цьому - жахливий суржик.


%%%fbauth
%%%fbauth_name
\iusr{Лиза Ребар}
%%%fbauth_url
%%%fbauth_place
%%%fbauth_id
%%%fbauth_front
%%%fbauth_desc
%%%fbauth_www
%%%fbauth_pic
%%%fbauth_pic portrait
%%%fbauth_pic background
%%%fbauth_pic other
%%%fbauth_tags
%%%fbauth_pubs
%%%endfbauth
 
\textbf{Павло Шубарт} а ви звідки знаєте?

%%%fbauth
%%%fbauth_name
\iusr{Дарія Матвіїв-Веклин}
%%%fbauth_url
%%%fbauth_place
%%%fbauth_id
%%%fbauth_front
%%%fbauth_desc
%%%fbauth_www
%%%fbauth_pic
%%%fbauth_pic portrait
%%%fbauth_pic background
%%%fbauth_pic other
%%%fbauth_tags
%%%fbauth_pubs
%%%endfbauth
 
\textbf{Serhii Bryhar} 

пане Сергію, рашамовних до Львова наїхало повно ще зі сорокових минулого
століття. Звичайно, ніхто з них і не подумав опанувати українську. Я після
школи, в далекому 1972, влаштувалася в обласну клінічну лікарню, приїхавши туди
із села на Сокальщині. Здивуванню мому не було меж-російськомовними було все
лікарняне начальство з типовими росіянськими прізвищами, медсестри-дружини
військових, етнічні євреї. 

Та серед нас, місцевих, не було жодного москвоязикого! На все життя
запам'яталися "уроки" патріотизму саме від них. У Львові живуть мої діти,
багато знайомих-ніхто з них не омоскалений, всі розмовляють гарною українською
мовою. Без усяких там, до речі, діалектів чи полонізмів.


%%%fbauth
%%%fbauth_name
\iusr{Дарія Матвіїв-Веклин}
%%%fbauth_url
%%%fbauth_place
%%%fbauth_id
%%%fbauth_front
%%%fbauth_desc
%%%fbauth_www
%%%fbauth_pic
%%%fbauth_pic portrait
%%%fbauth_pic background
%%%fbauth_pic other
%%%fbauth_tags
%%%fbauth_pubs
%%%endfbauth
 
\textbf{Павло Шубарт} не вірю!

%%%fbauth
%%%fbauth_name
\iusr{Марта Козак}
%%%fbauth_url
%%%fbauth_place
%%%fbauth_id
%%%fbauth_front
%%%fbauth_desc
%%%fbauth_www
%%%fbauth_pic
%%%fbauth_pic portrait
%%%fbauth_pic background
%%%fbauth_pic other
%%%fbauth_tags
%%%fbauth_pubs
%%%endfbauth
 
\textbf{Павло Шубарт} 

ні УКРАІНСЬКА мова в Тернополі є бо це українське, виключно українське, місто
.Там просто живуть українці .На рахунок селян це не завжди так ,бо як то власне
селяни щоб довести свою"містяність"калічать по російські. Містяни ж стійкі
,свідомі УКРАІНЦІ. І ЦЕ Ж У ЛЬВОВІ., ФРАНКІВСЬКУ І У ВСІЙ ЗАХІДНіЙ УКРАЇНІ .Ми
УКРАІНЦІ і цим все сказано .Можуть бути помилки чи русизм але це перш за все
УКРАЇГСЬКА МОВА.


%%%fbauth
%%%fbauth_name
\iusr{Олександр Кондратюк}
%%%fbauth_url
%%%fbauth_place
%%%fbauth_id
%%%fbauth_front
%%%fbauth_desc
%%%fbauth_www
%%%fbauth_pic
%%%fbauth_pic portrait
%%%fbauth_pic background
%%%fbauth_pic other
%%%fbauth_tags
%%%fbauth_pubs
%%%endfbauth
 
\textbf{Дарія Матвіїв-Веклин} 

Тернопіль лінгвістично — мабуть найбільш український з усіх обласних центрів
України. Більш український, ніж Львів, якщо вірити статистиці. Але ж не можна
сказати, що львів'янам байдуже на мову. Не байдуже. Просто недооцінюють
небезпеку.


%%%fbauth
%%%fbauth_name
\iusr{Дарія Матвіїв-Веклин}
%%%fbauth_url
%%%fbauth_place
%%%fbauth_id
%%%fbauth_front
%%%fbauth_desc
%%%fbauth_www
%%%fbauth_pic
%%%fbauth_pic portrait
%%%fbauth_pic background
%%%fbauth_pic other
%%%fbauth_tags
%%%fbauth_pubs
%%%endfbauth
 
\textbf{Олександр Кондратюк} 

а що можуть зробити львів'яни? Закрити російські школи? Погромити розкладки з
московськими журнальчиками?

%%%fbauth
%%%fbauth_name
\iusr{Олександр Кондратюк}
%%%fbauth_url
%%%fbauth_place
%%%fbauth_id
%%%fbauth_front
%%%fbauth_desc
%%%fbauth_www
%%%fbauth_pic
%%%fbauth_pic portrait
%%%fbauth_pic background
%%%fbauth_pic other
%%%fbauth_tags
%%%fbauth_pubs
%%%endfbauth
 
\textbf{Дарія Матвіїв-Веклин} 

не ставити дітям мультфільми російською мовою, зупиняти тих, хто розпочинає
акції на кшталт "Львов говорит по-русски", є й інші форми мовного спротиву, які
не застосовуються саме через недооцінку небезпеки, я вважаю. Як в балтійських
республіках СРСР часом відповідали російськомовним? — просто відмовлялися їх
розуміти. Бо відчували і усвідомлювали небезпеку, вважали їх окупантами.


%%%fbauth
%%%fbauth_name
\iusr{Дарія Матвіїв-Веклин}
%%%fbauth_url
%%%fbauth_place
%%%fbauth_id
%%%fbauth_front
%%%fbauth_desc
%%%fbauth_www
%%%fbauth_pic
%%%fbauth_pic portrait
%%%fbauth_pic background
%%%fbauth_pic other
%%%fbauth_tags
%%%fbauth_pubs
%%%endfbauth
 
\textbf{Олександр Кондратюк} 

я особисто так роблю багато років. І таких галичан багато, але погоджуюся зі
сказаним Вами,що цього недостатньо.

Та знаю точно, що без належної уваги на державному рівні в питаннях освіти,
патріотичного виховання-нічого не зміниться.

%%%fbauth
%%%fbauth_name
\iusr{Lina Kacmar}
%%%fbauth_url
%%%fbauth_place
%%%fbauth_id
%%%fbauth_front
%%%fbauth_desc
%%%fbauth_www
%%%fbauth_pic
%%%fbauth_pic portrait
%%%fbauth_pic background
%%%fbauth_pic other
%%%fbauth_tags
%%%fbauth_pubs
%%%endfbauth
 
\textbf{Serhii Bryhar} 

повністю з вами погоджуюсь, що українцем в Одесі, і не тільки в Одесі!
незрівнянно тяжче, чим москворотим у Львові чи Тернополі. Бо українці
надзвичайно добренькі. Толерують москалоту мовчки, що мене надзвичайно бісить.
І злить мене навіть не стільки ті упороті москвороті, як мої земляки, які це
дозволяють. Я ніколи не мовчу і свою злість найчастіше висловлюю не стільки
упоротим любітєлям узкава язика, як тим, що дозволяють срати собі на голову.


%%%fbauth
%%%fbauth_name
\iusr{Марта Козак}
%%%fbauth_url
%%%fbauth_place
%%%fbauth_id
%%%fbauth_front
%%%fbauth_desc
%%%fbauth_www
%%%fbauth_pic
%%%fbauth_pic portrait
%%%fbauth_pic background
%%%fbauth_pic other
%%%fbauth_tags
%%%fbauth_pubs
%%%endfbauth
 
\textbf{Дарія Матвіїв-Веклин} 

у Львові нема російскіх шкіл. Але рос мова як ібурян стійка і пхається в життя
дітей через мережі і гаджети .Це боротьба вічна .АЛЕ МИ ПЕРЕМОЖЕМО.

\end{itemize}

%%%fbauth
%%%fbauth_name
\iusr{Volodymyr Genyk}
%%%fbauth_url
%%%fbauth_place
%%%fbauth_id
%%%fbauth_front
%%%fbauth_desc
%%%fbauth_www
%%%fbauth_pic
%%%fbauth_pic portrait
%%%fbauth_pic background
%%%fbauth_pic other
%%%fbauth_tags
%%%fbauth_pubs
%%%endfbauth
 
То не донецькі, чи ще якісь, хто 'псує картину", там і домашніх рагулів вистачає.

%%%fbauth
%%%fbauth_name
\iusr{Roman Romanow}
%%%fbauth_url
%%%fbauth_place
%%%fbauth_id
%%%fbauth_front
%%%fbauth_desc
%%%fbauth_www
%%%fbauth_pic
%%%fbauth_pic portrait
%%%fbauth_pic background
%%%fbauth_pic other
%%%fbauth_tags
%%%fbauth_pubs
%%%endfbauth
 

сергій перестаньте дрочити на львів. львів зміцнів випадково і залишився
укранським теж майже випадково, це при всій ткскть повазі до моїх
предків-западенців. Київ і вся серцевина від хмельницька і до.. далеко - от де
Україна. і - так - вона зараз понівечена і просто тихенько лежить і набираєцця
сил. а не львіви і прочі черновіци))

\begin{itemize}
%%%fbauth
%%%fbauth_name
\iusr{Maryna Yatsunda}
%%%fbauth_url
%%%fbauth_place
%%%fbauth_id
%%%fbauth_front
%%%fbauth_desc
%%%fbauth_www
%%%fbauth_pic
%%%fbauth_pic portrait
%%%fbauth_pic background
%%%fbauth_pic other
%%%fbauth_tags
%%%fbauth_pubs
%%%endfbauth
 
\textbf{Roman Romanow} і зовсім випадково єдиний проголосував за Пороха на виборах. І зовсім випадково масово був на Майдані в Києві. І зовсім випадково чистить пики тим, хто посягає на пам‘ятник Бандері або в центрі волає ка@апські пісні. Не треба на нас дрочити. Ми і самі собі раду дамо.

%%%fbauth
%%%fbauth_name
\iusr{Roman Romanow}
%%%fbauth_url
%%%fbauth_place
%%%fbauth_id
%%%fbauth_front
%%%fbauth_desc
%%%fbauth_www
%%%fbauth_pic
%%%fbauth_pic portrait
%%%fbauth_pic background
%%%fbauth_pic other
%%%fbauth_tags
%%%fbauth_pubs
%%%endfbauth
 
\textbf{Maryna Yatsunda} Ви не розумієте (або не хочете розуміти що в принципі для Вас же гірше) про що пишете. так львів найукраїнськіший. так він у певному розумінні став каталізатором процесів. але ще сто років тому ситуація влигядала зовсім не так. а ці сто років стали великими випадковостями. от і весь х.....

%%%fbauth
%%%fbauth_name
\iusr{Максим Меркулов}
%%%fbauth_url
%%%fbauth_place
%%%fbauth_id
%%%fbauth_front
%%%fbauth_desc
%%%fbauth_www
%%%fbauth_pic
%%%fbauth_pic portrait
%%%fbauth_pic background
%%%fbauth_pic other
%%%fbauth_tags
%%%fbauth_pubs
%%%endfbauth
 
\textbf{Roman Romanow} Чому понівечена? Я свого часу поїздив по Київщині -
всюди розмовляли живою і колоритною українською. Спілкувався з вінничанами -
казали, що у їхньому місті все ж домінує українська. Про Полтавщину узагалі
мовчу, це для мене як друга Галичина. А Львів таки молодець - тримає планку.

%%%fbauth
%%%fbauth_name
\iusr{Roman Romanow}
%%%fbauth_url
%%%fbauth_place
%%%fbauth_id
%%%fbauth_front
%%%fbauth_desc
%%%fbauth_www
%%%fbauth_pic
%%%fbauth_pic portrait
%%%fbauth_pic background
%%%fbauth_pic other
%%%fbauth_tags
%%%fbauth_pubs
%%%endfbauth
 
\textbf{Максим Меркулов} 

бо не може непонівечена країна вибирати на президента красіве зелене ніщо. от,
якщо починати з самого важного.


%%%fbauth
%%%fbauth_name
\iusr{Максим Меркулов}
%%%fbauth_url
%%%fbauth_place
%%%fbauth_id
%%%fbauth_front
%%%fbauth_desc
%%%fbauth_www
%%%fbauth_pic
%%%fbauth_pic portrait
%%%fbauth_pic background
%%%fbauth_pic other
%%%fbauth_tags
%%%fbauth_pubs
%%%endfbauth
 
\textbf{Roman Romanow} Тоді й Тернопіль з Франківськом теж понівечені, виходить

%%%fbauth
%%%fbauth_name
\iusr{Roman Romanow}
%%%fbauth_url
%%%fbauth_place
%%%fbauth_id
%%%fbauth_front
%%%fbauth_desc
%%%fbauth_www
%%%fbauth_pic
%%%fbauth_pic portrait
%%%fbauth_pic background
%%%fbauth_pic other
%%%fbauth_tags
%%%fbauth_pubs
%%%endfbauth
 
\textbf{Максим Меркулов} про них я навіть мову не веду. вони не в тій лізі грають

\end{itemize}

%%%fbauth
%%%fbauth_name
\iusr{Марія Вирасток}
%%%fbauth_url
%%%fbauth_place
%%%fbauth_id
%%%fbauth_front
%%%fbauth_desc
%%%fbauth_www
%%%fbauth_pic
%%%fbauth_pic portrait
%%%fbauth_pic background
%%%fbauth_pic other
%%%fbauth_tags
%%%fbauth_pubs
%%%endfbauth
 
Жаль,що все так відбувається.
А ще більше злість.

%%%fbauth
%%%fbauth_name
\iusr{Лариса Артем'єва}
%%%fbauth_url
%%%fbauth_place
%%%fbauth_id
%%%fbauth_front
%%%fbauth_desc
%%%fbauth_www
%%%fbauth_pic
%%%fbauth_pic portrait
%%%fbauth_pic background
%%%fbauth_pic other
%%%fbauth_tags
%%%fbauth_pubs
%%%endfbauth
 

Також почала помічати в своїй місцевості все більше російськомовних.Можливо
раніше не звертала уваги на це.А ще пише молодь російською,певно щоб
похизуватись ,що знають іноземну.,можливо завдяки різним росмовним
передачам,блогерам і т.д.Тут треба мати імунітет щоб протистояти русифікації.


%%%fbauth
%%%fbauth_name
\iusr{Nataliia Miniailo}
%%%fbauth_url
%%%fbauth_place
%%%fbauth_id
%%%fbauth_front
%%%fbauth_desc
%%%fbauth_www
%%%fbauth_pic
%%%fbauth_pic portrait
%%%fbauth_pic background
%%%fbauth_pic other
%%%fbauth_tags
%%%fbauth_pubs
%%%endfbauth
 
Там справді багато білорусів і вони всі російськомовні(

%%%fbauth
%%%fbauth_name
\iusr{Леся Менкуш-Мальків}
%%%fbauth_url
%%%fbauth_place
%%%fbauth_id
%%%fbauth_front
%%%fbauth_desc
%%%fbauth_www
%%%fbauth_pic
%%%fbauth_pic portrait
%%%fbauth_pic background
%%%fbauth_pic other
%%%fbauth_tags
%%%fbauth_pubs
%%%endfbauth
 

Окупація йде повним ходом і на всіх фронтах. Лише сліпий того не бачить...
Зарадити можуть лише заходи з боку держави. Але де нам взяти на тих посадах
тих, хто б це хотів і зміг зробити?

%%%fbauth
%%%fbauth_name
\iusr{Сергій Лащенко}
%%%fbauth_url
%%%fbauth_place
%%%fbauth_id
%%%fbauth_front
%%%fbauth_desc
%%%fbauth_www
%%%fbauth_pic
%%%fbauth_pic portrait
%%%fbauth_pic background
%%%fbauth_pic other
%%%fbauth_tags
%%%fbauth_pubs
%%%endfbauth
 

На позицію львів'ян, очевидно, впливає те, що вони і так " найкращі"... На кого
рівнятися? В Черкасах і в Полтаві мова ще гірша

\begin{itemize}
%%%fbauth
%%%fbauth_name
\iusr{Павло Шубарт}
%%%fbauth_url
%%%fbauth_place
%%%fbauth_id
%%%fbauth_front
%%%fbauth_desc
%%%fbauth_www
%%%fbauth_pic
%%%fbauth_pic portrait
%%%fbauth_pic background
%%%fbauth_pic other
%%%fbauth_tags
%%%fbauth_pubs
%%%endfbauth
 
\textbf{Сергій Лащенко} хто придумав, що львів'яни "найкращі"? Це совок намагався протиставляти Галичину Україні, але 30 років незалежної усіх зрівняв - і Полтаву, і Львів, і Одесу з Дніпром. Скрізь манкуртів вистачає. Ідеалів і бастіонів вже не існує.

%%%fbauth
%%%fbauth_name
\iusr{Сергій Лащенко}
%%%fbauth_url
%%%fbauth_place
%%%fbauth_id
%%%fbauth_front
%%%fbauth_desc
%%%fbauth_www
%%%fbauth_pic
%%%fbauth_pic portrait
%%%fbauth_pic background
%%%fbauth_pic other
%%%fbauth_tags
%%%fbauth_pubs
%%%endfbauth
 
\textbf{Павло Шубарт} Не "найкращі" - це точно. Але якщо витворити якусь шкалу, то все одно були б у числі перших. З Чернівцями ж не порівняєш! І з Луцьком теж. Там суржик існує як норма. Мій знайомий досі при зустрічі каже "Прівєт!", хоча й працює у сфері культури

%%%fbauth
%%%fbauth_name
\iusr{Павло Шубарт}
%%%fbauth_url
%%%fbauth_place
%%%fbauth_id
%%%fbauth_front
%%%fbauth_desc
%%%fbauth_www
%%%fbauth_pic
%%%fbauth_pic portrait
%%%fbauth_pic background
%%%fbauth_pic other
%%%fbauth_tags
%%%fbauth_pubs
%%%endfbauth
 
\textbf{Сергій Лащенко} мову пересічних львів'ян теж українською не назвеш. Галицький діялект тепер зустрінеш хіба що в словниках. А мова львів'ян - це московсько-український суржик, бо львів'яни полюбляють дивитися москвоязичне телебачення.

%%%fbauth
%%%fbauth_name
\iusr{Наталя Дойкова}
%%%fbauth_url
%%%fbauth_place
%%%fbauth_id
%%%fbauth_front
%%%fbauth_desc
%%%fbauth_www
%%%fbauth_pic
%%%fbauth_pic portrait
%%%fbauth_pic background
%%%fbauth_pic other
%%%fbauth_tags
%%%fbauth_pubs
%%%endfbauth
 
\textbf{Сергій Лащенко} львів'яни просто не відчули ще на собі жаху змосковщення. За час мого проживання у Львові я також була така собі лояльна якарізниця. Але лише в Києві відчула приниження від того, що твою українську тупо не розуміють і не збираються розуміти.

%%%fbauth
%%%fbauth_name
\iusr{Павло Шубарт}
%%%fbauth_url
%%%fbauth_place
%%%fbauth_id
%%%fbauth_front
%%%fbauth_desc
%%%fbauth_www
%%%fbauth_pic
%%%fbauth_pic portrait
%%%fbauth_pic background
%%%fbauth_pic other
%%%fbauth_tags
%%%fbauth_pubs
%%%endfbauth
 
\textbf{Наталя Дойкова} ви ще в Одесі не були...

%%%fbauth
%%%fbauth_name
\iusr{Наталя Дойкова}
%%%fbauth_url
%%%fbauth_place
%%%fbauth_id
%%%fbauth_front
%%%fbauth_desc
%%%fbauth_www
%%%fbauth_pic
%%%fbauth_pic portrait
%%%fbauth_pic background
%%%fbauth_pic other
%%%fbauth_tags
%%%fbauth_pubs
%%%endfbauth
 
\textbf{Павло Шубарт} останній раз була в Одeсі два роки назад. На диво, в
готeлі, в супeрмаркeті, в кафe (тобто, майжe всюди , дe доводилось
комюнікувати), мeні відповідали нашою мовою. А от в Києві в 80-х трамвайний
талончик бeз москвоязу купити було нeможливо. (

%%%fbauth
%%%fbauth_name
\iusr{Максим Меркулов}
%%%fbauth_url
%%%fbauth_place
%%%fbauth_id
%%%fbauth_front
%%%fbauth_desc
%%%fbauth_www
%%%fbauth_pic
%%%fbauth_pic portrait
%%%fbauth_pic background
%%%fbauth_pic other
%%%fbauth_tags
%%%fbauth_pubs
%%%endfbauth
 
\textbf{Сергій Лащенко} Гм... Мені Полтава видалася україномовнішою за Київ. Хоча й там російської не бракує.

%%%fbauth
%%%fbauth_name
\iusr{Максим Меркулов}
%%%fbauth_url
%%%fbauth_place
%%%fbauth_id
%%%fbauth_front
%%%fbauth_desc
%%%fbauth_www
%%%fbauth_pic
%%%fbauth_pic portrait
%%%fbauth_pic background
%%%fbauth_pic other
%%%fbauth_tags
%%%fbauth_pubs
%%%endfbauth
 
\textbf{Наталя Дойкова} Коли саме Ви відчули таке приниження в Києві? Просто
мене, україномовного, усі тут чудово розуміють. Щоправда, я перейшов на
українську в нульових, себто не так уже й давно.

%%%fbauth
%%%fbauth_name
\iusr{Сергій Лащенко}
%%%fbauth_url
%%%fbauth_place
%%%fbauth_id
%%%fbauth_front
%%%fbauth_desc
%%%fbauth_www
%%%fbauth_pic
%%%fbauth_pic portrait
%%%fbauth_pic background
%%%fbauth_pic other
%%%fbauth_tags
%%%fbauth_pubs
%%%endfbauth
 
Максим Меркулов А коли ви були в Полтаві? Який \% розмовляє українською?

\end{itemize}

%%%fbauth
%%%fbauth_name
\iusr{Ольга Семиволос}
%%%fbauth_url
%%%fbauth_place
%%%fbauth_id
%%%fbauth_front
%%%fbauth_desc
%%%fbauth_www
%%%fbauth_pic
%%%fbauth_pic portrait
%%%fbauth_pic background
%%%fbauth_pic other
%%%fbauth_tags
%%%fbauth_pubs
%%%endfbauth
 
Абсолютно згідна

%%%fbauth
%%%fbauth_name
\iusr{Зоряна Мацько}
%%%fbauth_url
%%%fbauth_place
%%%fbauth_id
%%%fbauth_front
%%%fbauth_desc
%%%fbauth_www
%%%fbauth_pic
%%%fbauth_pic portrait
%%%fbauth_pic background
%%%fbauth_pic other
%%%fbauth_tags
%%%fbauth_pubs
%%%endfbauth
 

Думаю, все-таки більше йдеться про переселенців. Вони у Львові та передмісті
часто купують цілі під'їзди у новобудовах. Не асимілюються. Помітила, що зі
збільшенням їхньої кількості у Львові побільшало чоловіків, одягнутих у стилі
"аля я тільки вийшов із в'язниці" і всіляких осіб, які "косять" під "крутих".
Такі зазвичай ходять із величезними динаміками у кепках і слухають російський
чи то шансон, чи то що. Ну і не треба забувати, що нащадки тих осіб, які
приїхали у 40-50-60-х роках із Росії чи зі зрусифікованої Центральної або
Східної України, також досі розмовляють винятково російською. Ці принципово не
переходять на українську, хоч часто її досить добре знають. Деякі мої родичі,
на жаль, також досі послуговуються у побуті тільки російською. Хоч живуть у
Львові вже не один десяток років.

\begin{itemize}
%%%fbauth
%%%fbauth_name
\iusr{Марта Козак}
%%%fbauth_url
%%%fbauth_place
%%%fbauth_id
%%%fbauth_front
%%%fbauth_desc
%%%fbauth_www
%%%fbauth_pic
%%%fbauth_pic portrait
%%%fbauth_pic background
%%%fbauth_pic other
%%%fbauth_tags
%%%fbauth_pubs
%%%endfbauth
 
\textbf{Zoryana Matsko} це не про Львів

%%%fbauth
%%%fbauth_name
\iusr{Зоряна Мацько}
%%%fbauth_url
%%%fbauth_place
%%%fbauth_id
%%%fbauth_front
%%%fbauth_desc
%%%fbauth_www
%%%fbauth_pic
%%%fbauth_pic portrait
%%%fbauth_pic background
%%%fbauth_pic other
%%%fbauth_tags
%%%fbauth_pubs
%%%endfbauth
 
Це про Львів. Це - мої власні щоденні спостереження.

%%%fbauth
%%%fbauth_name
\iusr{Руслан Шеремета}
%%%fbauth_url
%%%fbauth_place
%%%fbauth_id
%%%fbauth_front
%%%fbauth_desc
%%%fbauth_www
%%%fbauth_pic
%%%fbauth_pic portrait
%%%fbauth_pic background
%%%fbauth_pic other
%%%fbauth_tags
%%%fbauth_pubs
%%%endfbauth
 
\textbf{Zoryana Matsko} Так і є, маргінали не знають сходинок вверх.

%%%fbauth
%%%fbauth_name
\iusr{Зоряна Мацько}
%%%fbauth_url
%%%fbauth_place
%%%fbauth_id
%%%fbauth_front
%%%fbauth_desc
%%%fbauth_www
%%%fbauth_pic
%%%fbauth_pic portrait
%%%fbauth_pic background
%%%fbauth_pic other
%%%fbauth_tags
%%%fbauth_pubs
%%%endfbauth
 

Все дуже відносно. Можна послуговуватись українською і бути маргіналом. Тут
радше про вперте небажання вивчати мову або користуватися нею. Для них вона
чужа.


%%%fbauth
%%%fbauth_name
\iusr{Руслан Шеремета}
%%%fbauth_url
%%%fbauth_place
%%%fbauth_id
%%%fbauth_front
%%%fbauth_desc
%%%fbauth_www
%%%fbauth_pic
%%%fbauth_pic portrait
%%%fbauth_pic background
%%%fbauth_pic other
%%%fbauth_tags
%%%fbauth_pubs
%%%endfbauth
 
\textbf{Zoryana Matsko} Таке теж є. Хоча більшість з них із сіл...

%%%fbauth
%%%fbauth_name
\iusr{Зоряна Мацько}
%%%fbauth_url
%%%fbauth_place
%%%fbauth_id
%%%fbauth_front
%%%fbauth_desc
%%%fbauth_www
%%%fbauth_pic
%%%fbauth_pic portrait
%%%fbauth_pic background
%%%fbauth_pic other
%%%fbauth_tags
%%%fbauth_pubs
%%%endfbauth
 
Коли я була на Донеччині, то їхні селяни досить незле говорять українською.

%%%fbauth
%%%fbauth_name
\iusr{Dina Wild}
%%%fbauth_url
%%%fbauth_place
%%%fbauth_id
%%%fbauth_front
%%%fbauth_desc
%%%fbauth_www
%%%fbauth_pic
%%%fbauth_pic portrait
%%%fbauth_pic background
%%%fbauth_pic other
%%%fbauth_tags
%%%fbauth_pubs
%%%endfbauth
 
\textbf{Zoryana Matsko} починається.
\begin{itemize}
  \item - Хто всравсь?
  \item - Невістка!
  \item - А де вона?
  \item - Свиней пасе. Од неї вітер несе.
\end{itemize}

%%%fbauth
%%%fbauth_name
\iusr{Serhii Bryhar}
%%%fbauth_url
%%%fbauth_place
%%%fbauth_id
%%%fbauth_front
%%%fbauth_desc
%%%fbauth_www
%%%fbauth_pic
%%%fbauth_pic portrait
%%%fbauth_pic background
%%%fbauth_pic other
%%%fbauth_tags
%%%fbauth_pubs
%%%endfbauth
 

Я ще помічаю досить значну кількість продавців, які от по телефону (чи з
колегами) приблизно так: "Льоша, альо, Льоша, бля, ти що поц?, ану падайді к
Маше і вазьмі там пакєт, і давай бєгом, бля, сука, нє бєсі мєня...", - кадр
змінюється, з'являється посмішка, і: "доброго дня, пане, вам щось підказати?")


%%%fbauth
%%%fbauth_name
\iusr{Олександр Кондратюк}
%%%fbauth_url
%%%fbauth_place
%%%fbauth_id
%%%fbauth_front
%%%fbauth_desc
%%%fbauth_www
%%%fbauth_pic
%%%fbauth_pic portrait
%%%fbauth_pic background
%%%fbauth_pic other
%%%fbauth_tags
%%%fbauth_pubs
%%%endfbauth
 
\textbf{Serhii Bryhar} і сміх, і гріх... Сценка про продавця і Льошу.

%%%fbauth
%%%fbauth_name
\iusr{Дарія Матвіїв-Веклин}
%%%fbauth_url
%%%fbauth_place
%%%fbauth_id
%%%fbauth_front
%%%fbauth_desc
%%%fbauth_www
%%%fbauth_pic
%%%fbauth_pic portrait
%%%fbauth_pic background
%%%fbauth_pic other
%%%fbauth_tags
%%%fbauth_pubs
%%%endfbauth
 
\textbf{Serhii Bryhar}

тупі створіння не дуже мовою переймаються. Їх вистачає, на жаль, і у Львові,
але це ж наївно думати, що галичани всі грамотні і виховані. Я про
інше: галичан, що змінили рідну мову на московську - одиниці.


%%%fbauth
%%%fbauth_name
\iusr{Максим Меркулов}
%%%fbauth_url
%%%fbauth_place
%%%fbauth_id
%%%fbauth_front
%%%fbauth_desc
%%%fbauth_www
%%%fbauth_pic
%%%fbauth_pic portrait
%%%fbauth_pic background
%%%fbauth_pic other
%%%fbauth_tags
%%%fbauth_pubs
%%%endfbauth
 
\textbf{Zoryana Matsko} Для мене, неукраїнця з російськомовної родини,
українська теж апріорі мала б стати чужою. Але, як бачите, не стала. 😉

\end{itemize}

%%%fbauth
%%%fbauth_name
\iusr{Руслан Шеремета}
%%%fbauth_url
%%%fbauth_place
%%%fbauth_id
%%%fbauth_front
%%%fbauth_desc
%%%fbauth_www
%%%fbauth_pic
%%%fbauth_pic portrait
%%%fbauth_pic background
%%%fbauth_pic other
%%%fbauth_tags
%%%fbauth_pubs
%%%endfbauth
 

Там багато мешкає колишніх військових і переселенців. Сихів один із львівських
спальників. Ну і угорці з молдаванами можуть виключно російською спілкуватися,
і львівські поляки теж.


%%%fbauth
%%%fbauth_name
\iusr{Павло Шубарт}
%%%fbauth_url
%%%fbauth_place
%%%fbauth_id
%%%fbauth_front
%%%fbauth_desc
%%%fbauth_www
%%%fbauth_pic
%%%fbauth_pic portrait
%%%fbauth_pic background
%%%fbauth_pic other
%%%fbauth_tags
%%%fbauth_pubs
%%%endfbauth
 

1939 року у Львові сталася цивілізаційна катастрофа: прийшов руССкій мір. Зі
своїм язиком і поняттями. Тому від 1939 року і дотепер Львів активно
змосковщується і навіть ззовні уподібнюється до здеградованих Одеси, Маріуполя
чи Тамбова - цивілізаційний простір же один. На кого рівнятися Львову на цьому
просторі? В кому сили черпати, як не в самому собі? Але Львів не здатний
опиратися. Тому руССкій мір і його перетравлює.

\begin{itemize}
%%%fbauth
%%%fbauth_name
\iusr{Maryna Yatsunda}
%%%fbauth_url
%%%fbauth_place
%%%fbauth_id
%%%fbauth_front
%%%fbauth_desc
%%%fbauth_www
%%%fbauth_pic
%%%fbauth_pic portrait
%%%fbauth_pic background
%%%fbauth_pic other
%%%fbauth_tags
%%%fbauth_pubs
%%%endfbauth
 
\textbf{Павло Шубарт} про зовні уподібнюється - це ви конкретно перегнули палицю

%%%fbauth
%%%fbauth_name
\iusr{Павло Шубарт}
%%%fbauth_url
%%%fbauth_place
%%%fbauth_id
%%%fbauth_front
%%%fbauth_desc
%%%fbauth_www
%%%fbauth_pic
%%%fbauth_pic portrait
%%%fbauth_pic background
%%%fbauth_pic other
%%%fbauth_tags
%%%fbauth_pubs
%%%endfbauth
 
\textbf{Maryna Yatsunda} ні, не перегнув. Поїдьте до галицьких Тарнова чи Кракова на польському боці і одразу все зрозумієте. Львів просякнутий совком.
\end{itemize}

%%%fbauth
%%%fbauth_name
\iusr{Maryna Yatsunda}
%%%fbauth_url
%%%fbauth_place
%%%fbauth_id
%%%fbauth_front
%%%fbauth_desc
%%%fbauth_www
%%%fbauth_pic
%%%fbauth_pic portrait
%%%fbauth_pic background
%%%fbauth_pic other
%%%fbauth_tags
%%%fbauth_pubs
%%%endfbauth
 

Мова в дописі про те, що бути в Одесі украінцем значно важче, ніж
російськомовним у Львові. Моя думка, украінцем важче бути будь де у нас.

\begin{itemize}
%%%fbauth
%%%fbauth_name
\iusr{Павло Шубарт}
%%%fbauth_url
%%%fbauth_place
%%%fbauth_id
%%%fbauth_front
%%%fbauth_desc
%%%fbauth_www
%%%fbauth_pic
%%%fbauth_pic portrait
%%%fbauth_pic background
%%%fbauth_pic other
%%%fbauth_tags
%%%fbauth_pubs
%%%endfbauth
 
\textbf{Maryna Yatsunda} Українцем найлегше бути за межами України. Там повітря інше і вільно дишеться на повні груди.

%%%fbauth
%%%fbauth_name
\iusr{Мирослава Слюсаренко}
%%%fbauth_url
%%%fbauth_place
%%%fbauth_id
%%%fbauth_front
%%%fbauth_desc
%%%fbauth_www
%%%fbauth_pic
%%%fbauth_pic portrait
%%%fbauth_pic background
%%%fbauth_pic other
%%%fbauth_tags
%%%fbauth_pubs
%%%endfbauth
 
\textbf{Maryna Yatsunda} українцем легко бути в Канаді ((

%%%fbauth
%%%fbauth_name
\iusr{Дарія Матвіїв-Веклин}
%%%fbauth_url
%%%fbauth_place
%%%fbauth_id
%%%fbauth_front
%%%fbauth_desc
%%%fbauth_www
%%%fbauth_pic
%%%fbauth_pic portrait
%%%fbauth_pic background
%%%fbauth_pic other
%%%fbauth_tags
%%%fbauth_pubs
%%%endfbauth
 
\textbf{Павло Шубарт} на жаль так і є. Навіть не знаю, чи змогла би порадити
щиро п. Сергію переїжджати до Львова, бо не впевнена, що він знайде тут те, що
шукає. Це дуже прикро.

%%%fbauth
%%%fbauth_name
\iusr{Максим Меркулов}
%%%fbauth_url
%%%fbauth_place
%%%fbauth_id
%%%fbauth_front
%%%fbauth_desc
%%%fbauth_www
%%%fbauth_pic
%%%fbauth_pic portrait
%%%fbauth_pic background
%%%fbauth_pic other
%%%fbauth_tags
%%%fbauth_pubs
%%%endfbauth
 
\textbf{Maryna Yatsunda} 

Гм... Особисто мені легко бути україномовним навіть у Києві. Хоч я й ні разу не
українець.

\end{itemize}

%%%fbauth
%%%fbauth_name
\iusr{Олег Щербан}
%%%fbauth_url
%%%fbauth_place
%%%fbauth_id
%%%fbauth_front
%%%fbauth_desc
%%%fbauth_www
%%%fbauth_pic
%%%fbauth_pic portrait
%%%fbauth_pic background
%%%fbauth_pic other
%%%fbauth_tags
%%%fbauth_pubs
%%%endfbauth
 

То біженці або люди які переїхали зі сходу. Львів'янам давно промивають мізки
толерантності. Ось наслідки.

%%%fbauth
%%%fbauth_name
\iusr{Олег Щербан}
%%%fbauth_url
%%%fbauth_place
%%%fbauth_id
%%%fbauth_front
%%%fbauth_desc
%%%fbauth_www
%%%fbauth_pic
%%%fbauth_pic portrait
%%%fbauth_pic background
%%%fbauth_pic other
%%%fbauth_tags
%%%fbauth_pubs
%%%endfbauth
 

Мені здається що ми іноді надто завишуємо планку вимог до галичан. Ми теж люди
і різні, але ми зберегли мову і традиції.

\begin{itemize}
%%%fbauth
%%%fbauth_name
\iusr{Павло Шубарт}
%%%fbauth_url
%%%fbauth_place
%%%fbauth_id
%%%fbauth_front
%%%fbauth_desc
%%%fbauth_www
%%%fbauth_pic
%%%fbauth_pic portrait
%%%fbauth_pic background
%%%fbauth_pic other
%%%fbauth_tags
%%%fbauth_pubs
%%%endfbauth
 
\textbf{Олег Щербан} люди звикли до ікон.

%%%fbauth
%%%fbauth_name
\iusr{Лиза Ребар}
%%%fbauth_url
%%%fbauth_place
%%%fbauth_id
%%%fbauth_front
%%%fbauth_desc
%%%fbauth_www
%%%fbauth_pic
%%%fbauth_pic portrait
%%%fbauth_pic background
%%%fbauth_pic other
%%%fbauth_tags
%%%fbauth_pubs
%%%endfbauth
 
\textbf{Олег Щербан} тим більше що ми в окупації були з часів Речі Посполитої

%%%fbauth
%%%fbauth_name
\iusr{Павло Шубарт}
%%%fbauth_url
%%%fbauth_place
%%%fbauth_id
%%%fbauth_front
%%%fbauth_desc
%%%fbauth_www
%%%fbauth_pic
%%%fbauth_pic portrait
%%%fbauth_pic background
%%%fbauth_pic other
%%%fbauth_tags
%%%fbauth_pubs
%%%endfbauth
 
\textbf{Ліза Ребар} то якому з окупантів ви чинили сильніший опір - польському, звільнившись від якого, ви зберегли українську мову, чи московському, мови якого ви не позбулися навіть за 30 років існування начебто власної держави?

%%%fbauth
%%%fbauth_name
\iusr{Лиза Ребар}
%%%fbauth_url
%%%fbauth_place
%%%fbauth_id
%%%fbauth_front
%%%fbauth_desc
%%%fbauth_www
%%%fbauth_pic
%%%fbauth_pic portrait
%%%fbauth_pic background
%%%fbauth_pic other
%%%fbauth_tags
%%%fbauth_pubs
%%%endfbauth
 
\textbf{Павло Шубарт} я за себе відповім: моя українська досконала,хоч я і не філолог

%%%fbauth
%%%fbauth_name
\iusr{Олег Щербан}
%%%fbauth_url
%%%fbauth_place
%%%fbauth_id
%%%fbauth_front
%%%fbauth_desc
%%%fbauth_www
%%%fbauth_pic
%%%fbauth_pic portrait
%%%fbauth_pic background
%%%fbauth_pic other
%%%fbauth_tags
%%%fbauth_pubs
%%%endfbauth
 
\textbf{Павло Шубарт} це не місцеві розмовляють, а ті що приїхали і то в останні роки.

%%%fbauth
%%%fbauth_name
\iusr{Павло Шубарт}
%%%fbauth_url
%%%fbauth_place
%%%fbauth_id
%%%fbauth_front
%%%fbauth_desc
%%%fbauth_www
%%%fbauth_pic
%%%fbauth_pic portrait
%%%fbauth_pic background
%%%fbauth_pic other
%%%fbauth_tags
%%%fbauth_pubs
%%%endfbauth
 
\textbf{Олег Щербан} а чому вони не асимілюються? Тому що галичани їх до цього не схиляють.

%%%fbauth
%%%fbauth_name
\iusr{Олег Щербан}
%%%fbauth_url
%%%fbauth_place
%%%fbauth_id
%%%fbauth_front
%%%fbauth_desc
%%%fbauth_www
%%%fbauth_pic
%%%fbauth_pic portrait
%%%fbauth_pic background
%%%fbauth_pic other
%%%fbauth_tags
%%%fbauth_pubs
%%%endfbauth
 
\textbf{Павло Шубарт} ну так. Раніше такого не було. Проте Садовий під патронатом того ж Фрідмана просувають ідеї толерантності.

%%%fbauth
%%%fbauth_name
\iusr{Олег Щербан}
%%%fbauth_url
%%%fbauth_place
%%%fbauth_id
%%%fbauth_front
%%%fbauth_desc
%%%fbauth_www
%%%fbauth_pic
%%%fbauth_pic portrait
%%%fbauth_pic background
%%%fbauth_pic other
%%%fbauth_tags
%%%fbauth_pubs
%%%endfbauth
 
\textbf{Павло Шубарт} у більшості людей які в радянські часи потрапили на Галичину діти стали українськомовними.

%%%fbauth
%%%fbauth_name
\iusr{Павло Шубарт}
%%%fbauth_url
%%%fbauth_place
%%%fbauth_id
%%%fbauth_front
%%%fbauth_desc
%%%fbauth_www
%%%fbauth_pic
%%%fbauth_pic portrait
%%%fbauth_pic background
%%%fbauth_pic other
%%%fbauth_tags
%%%fbauth_pubs
%%%endfbauth
 
\textbf{Олег Щербан} бо Садовий працює на рашистські спецслужби.

%%%fbauth
%%%fbauth_name
\iusr{Олег Щербан}
%%%fbauth_url
%%%fbauth_place
%%%fbauth_id
%%%fbauth_front
%%%fbauth_desc
%%%fbauth_www
%%%fbauth_pic
%%%fbauth_pic portrait
%%%fbauth_pic background
%%%fbauth_pic other
%%%fbauth_tags
%%%fbauth_pubs
%%%endfbauth
 
\textbf{Павло Шубарт} так і глянь як працює наше телебачення і що вони нав'язує населенню України. Це теж має свій вплив..

%%%fbauth
%%%fbauth_name
\iusr{Лиза Ребар}
%%%fbauth_url
%%%fbauth_place
%%%fbauth_id
%%%fbauth_front
%%%fbauth_desc
%%%fbauth_www
%%%fbauth_pic
%%%fbauth_pic portrait
%%%fbauth_pic background
%%%fbauth_pic other
%%%fbauth_tags
%%%fbauth_pubs
%%%endfbauth
 
\textbf{Олег Щербан} а може забалакаємо на діалекті? Я туво прочитала про галичан- і насеруматері

%%%fbauth
%%%fbauth_name
\iusr{Олег Щербан}
%%%fbauth_url
%%%fbauth_place
%%%fbauth_id
%%%fbauth_front
%%%fbauth_desc
%%%fbauth_www
%%%fbauth_pic
%%%fbauth_pic portrait
%%%fbauth_pic background
%%%fbauth_pic other
%%%fbauth_tags
%%%fbauth_pubs
%%%endfbauth
 
\textbf{Ліза Ребар} то хіба вар'яти про таке пишуть))

%%%fbauth
%%%fbauth_name
\iusr{Дарія Матвіїв-Веклин}
%%%fbauth_url
%%%fbauth_place
%%%fbauth_id
%%%fbauth_front
%%%fbauth_desc
%%%fbauth_www
%%%fbauth_pic
%%%fbauth_pic portrait
%%%fbauth_pic background
%%%fbauth_pic other
%%%fbauth_tags
%%%fbauth_pubs
%%%endfbauth
 
\textbf{Олег Щербан} далеко не у всіх. Інакше звідки б набирали дітей у російські школи?
\end{itemize}

%%%fbauth
%%%fbauth_name
\iusr{Ivan Melobensky}
%%%fbauth_url
%%%fbauth_place
%%%fbauth_id
%%%fbauth_front
%%%fbauth_desc
%%%fbauth_www
%%%fbauth_pic
%%%fbauth_pic portrait
%%%fbauth_pic background
%%%fbauth_pic other
%%%fbauth_tags
%%%fbauth_pubs
%%%endfbauth
 

У мене особисто кілька друзів переїхали з Адєси до Львова і звісно ж не
перейшли на українську 🙁

\begin{itemize}
%%%fbauth
%%%fbauth_name
\iusr{Serhii Bryhar}
%%%fbauth_url
%%%fbauth_place
%%%fbauth_id
%%%fbauth_front
%%%fbauth_desc
%%%fbauth_www
%%%fbauth_pic
%%%fbauth_pic portrait
%%%fbauth_pic background
%%%fbauth_pic other
%%%fbauth_tags
%%%fbauth_pubs
%%%endfbauth
 
Сфера IT?

%%%fbauth
%%%fbauth_name
\iusr{Ivan Melobensky}
%%%fbauth_url
%%%fbauth_place
%%%fbauth_id
%%%fbauth_front
%%%fbauth_desc
%%%fbauth_www
%%%fbauth_pic
%%%fbauth_pic portrait
%%%fbauth_pic background
%%%fbauth_pic other
%%%fbauth_tags
%%%fbauth_pubs
%%%endfbauth
 
\textbf{Serhii Bryhar} так. вони майже НІКОЛИ не асимілюються... а зачєм мнє ета нада?

%%%fbauth
%%%fbauth_name
\iusr{Ivan Melobensky}
%%%fbauth_url
%%%fbauth_place
%%%fbauth_id
%%%fbauth_front
%%%fbauth_desc
%%%fbauth_www
%%%fbauth_pic
%%%fbauth_pic portrait
%%%fbauth_pic background
%%%fbauth_pic other
%%%fbauth_tags
%%%fbauth_pubs
%%%endfbauth
 
вони не розуміють, какаяразніца, не розуміють, що війна йде в тому числі через
слабкі позиції всього українського у нашій країні...

%%%fbauth
%%%fbauth_name
\iusr{Dmytro Dzyuba}
%%%fbauth_url
%%%fbauth_place
%%%fbauth_id
%%%fbauth_front
%%%fbauth_desc
%%%fbauth_www
%%%fbauth_pic
%%%fbauth_pic portrait
%%%fbauth_pic background
%%%fbauth_pic other
%%%fbauth_tags
%%%fbauth_pubs
%%%endfbauth
 
\textbf{Іван Мелобенський}, 

бо "ані ж с адєсси! Адєсса асобєнний ґорад! В Адєссє бєз русскава нікак!"

Власне, така ситуація всюди на змосквинених теренах, і у нас на Запорожжі
також. А після 2019-го взагалі всє рєзко сталі узкоязичнимі, тому, відчуття
наче в окупації...


%%%fbauth
%%%fbauth_name
\iusr{Ivan Melobensky}
%%%fbauth_url
%%%fbauth_place
%%%fbauth_id
%%%fbauth_front
%%%fbauth_desc
%%%fbauth_www
%%%fbauth_pic
%%%fbauth_pic portrait
%%%fbauth_pic background
%%%fbauth_pic other
%%%fbauth_tags
%%%fbauth_pubs
%%%endfbauth
 
\textbf{Serhii Bryhar} 

що цікаво, у програмістів там справді більше перспектив... Але проблема у тому,
що ринок айті орієнтується на світ - на англійську... А інші мови - то вже йде
як какаяразніца

\end{itemize}

%%%fbauth
%%%fbauth_name
\iusr{Михайло Ковальчук}
%%%fbauth_url
%%%fbauth_place
%%%fbauth_id
%%%fbauth_front
%%%fbauth_desc
%%%fbauth_www
%%%fbauth_pic
%%%fbauth_pic portrait
%%%fbauth_pic background
%%%fbauth_pic other
%%%fbauth_tags
%%%fbauth_pubs
%%%endfbauth
 

Пристосуванці на "бал маскараді"!

Ганьба тим, що за лаштунками, ні ілюзії!

Війна! А гібриди🌶️ множаться в Українському середовищі, центрі культури і
духовності

%%%fbauth
%%%fbauth_name
\iusr{Михайло Ковальчук}
%%%fbauth_url
%%%fbauth_place
%%%fbauth_id
%%%fbauth_front
%%%fbauth_desc
%%%fbauth_www
%%%fbauth_pic
%%%fbauth_pic portrait
%%%fbauth_pic background
%%%fbauth_pic other
%%%fbauth_tags
%%%fbauth_pubs
%%%endfbauth
 

\ifcmt
  ig https://scontent-cdg2-1.xx.fbcdn.net/v/t1.6435-9/229471726_508346203795758_8853672202945162511_n.jpg?_nc_cat=100&ccb=1-3&_nc_sid=dbeb18&_nc_ohc=81yc7pW8z-UAX-0lgiV&_nc_ht=scontent-cdg2-1.xx&oh=b97691cc0eb8bc2762e56de21737ef31&oe=613147DE
  width 0.3
\fi

%%%fbauth
%%%fbauth_name
\iusr{Інгвар Кумар}
%%%fbauth_url
%%%fbauth_place
%%%fbauth_id
%%%fbauth_front
%%%fbauth_desc
%%%fbauth_www
%%%fbauth_pic
%%%fbauth_pic portrait
%%%fbauth_pic background
%%%fbauth_pic other
%%%fbauth_tags
%%%fbauth_pubs
%%%endfbauth
 

Так Львів був заселений савєцкімі людьмі у першу окупацію, і, особливо, в
другу. Я дуже часто бував в місті 2005-2007 роках, купа російськомовних. І,
навіть в 2019-му коли ми повертаючись із Прикарпаття провели добу у Львові,
дружина із старшими пішли гуляти, я з наймолодшою чекали їх в парку біля ЛНУ,
зустріли багацько кривощелепних. І вірити в те, що вони із понаїхавших, як і
ми, не буду...  Єдиний із обласних центрів де, майже не чути російську - Луцьк.

\begin{itemize}
%%%fbauth
%%%fbauth_name
\iusr{Ярослав Дацко}
%%%fbauth_url
%%%fbauth_place
%%%fbauth_id
%%%fbauth_front
%%%fbauth_desc
%%%fbauth_www
%%%fbauth_pic
%%%fbauth_pic portrait
%%%fbauth_pic background
%%%fbauth_pic other
%%%fbauth_tags
%%%fbauth_pubs
%%%endfbauth
 
\textbf{Ingvar Kumar} думати що україномовні львів'яни стали масово навертатись в рос.яз. мені здається якось тупо і не логічно. Максимум діти суржик масово вживають через рутуб і тікток. Залишаються понаїхавші, і повірте їх немало

%%%fbauth
%%%fbauth_name
\iusr{Оксана Ковалишин}
%%%fbauth_url
%%%fbauth_place
%%%fbauth_id
%%%fbauth_front
%%%fbauth_desc
%%%fbauth_www
%%%fbauth_pic
%%%fbauth_pic portrait
%%%fbauth_pic background
%%%fbauth_pic other
%%%fbauth_tags
%%%fbauth_pubs
%%%endfbauth
 
\textbf{Ярослав Дацко} І тих, у кого спина зам'яка чималенько. Якщо скажете, що ні, то самі себе обманюєте, перепрошую.

%%%fbauth
%%%fbauth_name
\iusr{Оксана Ковалишин}
%%%fbauth_url
%%%fbauth_place
%%%fbauth_id
%%%fbauth_front
%%%fbauth_desc
%%%fbauth_www
%%%fbauth_pic
%%%fbauth_pic portrait
%%%fbauth_pic background
%%%fbauth_pic other
%%%fbauth_tags
%%%fbauth_pubs
%%%endfbauth
 
\textbf{Ingvar Kumar} Ви правду пишете. Стверджувати, що то тільки понаєхавшиє - себе обманювати. А це, як прийти до лікаря і він, щоб вас не засмучувати замість діагнозу "рак легень" ставить вам ГРЗ.
\end{itemize}

%%%fbauth
%%%fbauth_name
\iusr{Любов Марущак}
%%%fbauth_url
%%%fbauth_place
%%%fbauth_id
%%%fbauth_front
%%%fbauth_desc
%%%fbauth_www
%%%fbauth_pic
%%%fbauth_pic portrait
%%%fbauth_pic background
%%%fbauth_pic other
%%%fbauth_tags
%%%fbauth_pubs
%%%endfbauth
 
У Львові живе моя тітка з сім’єю, вони розказують, що за ці декілька років дуже побільшало російськомовних , дуже

%%%fbauth
%%%fbauth_name
\iusr{Отто Фон Штирлиц}
%%%fbauth_url
%%%fbauth_place
%%%fbauth_id
%%%fbauth_front
%%%fbauth_desc
%%%fbauth_www
%%%fbauth_pic
%%%fbauth_pic portrait
%%%fbauth_pic background
%%%fbauth_pic other
%%%fbauth_tags
%%%fbauth_pubs
%%%endfbauth
 

Підтримую. В Харкові подібна ситуація. Люди не дивуються, коли чують на вулицях
арабську чи вірменську, проте обертаються, коли чують українську (не
суржик!!!). Українська має тримати межі. І жодним чином не здавати позиції у
Львові, Франківську, Луцьку, Рівному, Тернополі, Чернівцях. І має підкорити
столицю! Це засади безпеки держави! І далі рухатися на південь і схід.

\begin{itemize}
%%%fbauth
%%%fbauth_name
\iusr{Максим Меркулов}
%%%fbauth_url
%%%fbauth_place
%%%fbauth_id
%%%fbauth_front
%%%fbauth_desc
%%%fbauth_www
%%%fbauth_pic
%%%fbauth_pic portrait
%%%fbauth_pic background
%%%fbauth_pic other
%%%fbauth_tags
%%%fbauth_pubs
%%%endfbauth
 
\textbf{Otto Von Stierlitz} Про столицю

\href{gazeta.ua/ru/articles/opinions-journal/_novij-kiyiv-bude-ukrayinskij/499688}{%
Новий Київ буде український, Олександр КУРИЛЕНКО; Ігор ЛУБ'ЯНОВ; Дмитро СКАЖЕНИК, gazeta.ua, 29.05.2013%
}

%%%fbauth
%%%fbauth_name
\iusr{Ярослав Білозір}
%%%fbauth_url
%%%fbauth_place
%%%fbauth_id
%%%fbauth_front
%%%fbauth_desc
%%%fbauth_www
%%%fbauth_pic
%%%fbauth_pic portrait
%%%fbauth_pic background
%%%fbauth_pic other
%%%fbauth_tags
%%%fbauth_pubs
%%%endfbauth
 

Ця стаття - вологі мрії її автора. Не бачу жодної причини і обставини, яка би
зробила Київ таким самим україномовним містом як Львів, Тернопіль, Франківськ,
Луцьк.


%%%fbauth
%%%fbauth_name
\iusr{Отто Фон Штирлиц}
%%%fbauth_url
%%%fbauth_place
%%%fbauth_id
%%%fbauth_front
%%%fbauth_desc
%%%fbauth_www
%%%fbauth_pic
%%%fbauth_pic portrait
%%%fbauth_pic background
%%%fbauth_pic other
%%%fbauth_tags
%%%fbauth_pubs
%%%endfbauth
 
\textbf{Ярослав Білозір}, дарма. Дуже повільно, але української на вулицях
столиці таки стає більше. А чи стає більше української у тих містах, що
зазначили ви, тут питання.

\end{itemize}

%%%fbauth
%%%fbauth_name
\iusr{Мирослава Слюсаренко}
%%%fbauth_url
%%%fbauth_place
%%%fbauth_id
%%%fbauth_front
%%%fbauth_desc
%%%fbauth_www
%%%fbauth_pic
%%%fbauth_pic portrait
%%%fbauth_pic background
%%%fbauth_pic other
%%%fbauth_tags
%%%fbauth_pubs
%%%endfbauth
 

так Львів вже декілька років просто вражає кількістю російськомовних. Раніше з
Києва у Львів їздила, щоб насолодитися україномовним середовищем і тепер ось ..

\begin{itemize}
%%%fbauth
%%%fbauth_name
\iusr{Максим Меркулов}
%%%fbauth_url
%%%fbauth_place
%%%fbauth_id
%%%fbauth_front
%%%fbauth_desc
%%%fbauth_www
%%%fbauth_pic
%%%fbauth_pic portrait
%%%fbauth_pic background
%%%fbauth_pic other
%%%fbauth_tags
%%%fbauth_pubs
%%%endfbauth
 
\textbf{Miroslava Slusarenko} Мені, щоб почути українську, не треба виїздити кудись з Києва. Достатньо прогулятися навколишніми дворами. Це Позняки, коли що.
\end{itemize}

%%%fbauth
%%%fbauth_name
\iusr{Олександр Кондратюк}
%%%fbauth_url
%%%fbauth_place
%%%fbauth_id
%%%fbauth_front
%%%fbauth_desc
%%%fbauth_www
%%%fbauth_pic
%%%fbauth_pic portrait
%%%fbauth_pic background
%%%fbauth_pic other
%%%fbauth_tags
%%%fbauth_pubs
%%%endfbauth
 

В певному сенсі Галичині пощастило на пару десятиліть менше перебувати в складі
СРСР, ніж східнішим областям. Та і Австро-Угорщина була суттєво ліберальнішою
імперією, ніж російська.

Сміливо можна стверджувати, що і Україні пощастило з Галичиною, якій вдалося
протриматися українською до самого розпаду совка.

І обидва Майдани, і голосування на виборах 2019-го — яскраві свідчення того, що
патріотизму львів'янам не бракує, і гасло "Мова. Віра. Армія" — не порожній
звук для них. Але за великим і масштабним можна не помітити дрібненької
повзучої зарази зросійщення туристами, приїжджими, переселенцями зі зросійщених
країв. Дуже здивувала (неприємно) акція "Львов говорит по-русски" в 2014-му
році. Назвати це помилкою — надто м'яко виходить.

\begin{itemize}
%%%fbauth
%%%fbauth_name
\iusr{Serhii Bryhar}
%%%fbauth_url
%%%fbauth_place
%%%fbauth_id
%%%fbauth_front
%%%fbauth_desc
%%%fbauth_www
%%%fbauth_pic
%%%fbauth_pic portrait
%%%fbauth_pic background
%%%fbauth_pic other
%%%fbauth_tags
%%%fbauth_pubs
%%%endfbauth
 
З усім згоден. Ситуація саме так і виглядає. Зараза помалу бере нові рубежі..
\end{itemize}

%%%fbauth
%%%fbauth_name
\iusr{Roman Kucharenko}
%%%fbauth_url
%%%fbauth_place
%%%fbauth_id
%%%fbauth_front
%%%fbauth_desc
%%%fbauth_www
%%%fbauth_pic
%%%fbauth_pic portrait
%%%fbauth_pic background
%%%fbauth_pic other
%%%fbauth_tags
%%%fbauth_pubs
%%%endfbauth
 

Сьогодні слухаю радіостанцію "Львівська хвиля": "...Він відправився в Індію
поодинці". Ось так розділився на кілька частин, які мандрують незалежно одна
від одної, і "відправився". Розшифровка: "...Він вирушив до Індії на самоті".


%%%fbauth
%%%fbauth_name
\iusr{Світлана Бардіна}
%%%fbauth_url
%%%fbauth_place
%%%fbauth_id
%%%fbauth_front
%%%fbauth_desc
%%%fbauth_www
%%%fbauth_pic
%%%fbauth_pic portrait
%%%fbauth_pic background
%%%fbauth_pic other
%%%fbauth_tags
%%%fbauth_pubs
%%%endfbauth
 

Я в Одесі 30 років. Останні 8 років "тримаю наш рубіж". Слідкую за Вашими
дописами і вражена Вашою стійкістю.

Підозрювала про Львів щось подібне... Тут, в Одесі, нас меншість. Там, у
Львові, занадто багато безпечності і дурної толерації "неляканих ідіотів"...(((
А коли накриє, як сто років тому....ну, граблі не вчать

%%%fbauth
%%%fbauth_name
\iusr{Сергій Литвиненко}
%%%fbauth_url
%%%fbauth_place
%%%fbauth_id
%%%fbauth_front
%%%fbauth_desc
%%%fbauth_www
%%%fbauth_pic
%%%fbauth_pic portrait
%%%fbauth_pic background
%%%fbauth_pic other
%%%fbauth_tags
%%%fbauth_pubs
%%%endfbauth
 
Однозначно потрібно тримати стрій і гуртуватися разом.

%%%fbauth
%%%fbauth_name
\iusr{Svitlana Chub-Krywuzka}
%%%fbauth_url
%%%fbauth_place
%%%fbauth_id
%%%fbauth_front
%%%fbauth_desc
%%%fbauth_www
%%%fbauth_pic
%%%fbauth_pic portrait
%%%fbauth_pic background
%%%fbauth_pic other
%%%fbauth_tags
%%%fbauth_pubs
%%%endfbauth
 
Переселенці(

\begin{itemize}
%%%fbauth
%%%fbauth_name
\iusr{Ігор Мазепа}
%%%fbauth_url
%%%fbauth_place
%%%fbauth_id
%%%fbauth_front
%%%fbauth_desc
%%%fbauth_www
%%%fbauth_pic
%%%fbauth_pic portrait
%%%fbauth_pic background
%%%fbauth_pic other
%%%fbauth_tags
%%%fbauth_pubs
%%%endfbauth
 
\textbf{Svitlana Chub-Krywuzka} перепрошую, додам: срані
\end{itemize}

%%%fbauth
%%%fbauth_name
\iusr{Руслан Янісевич}
%%%fbauth_url
%%%fbauth_place
%%%fbauth_id
%%%fbauth_front
%%%fbauth_desc
%%%fbauth_www
%%%fbauth_pic
%%%fbauth_pic portrait
%%%fbauth_pic background
%%%fbauth_pic other
%%%fbauth_tags
%%%fbauth_pubs
%%%endfbauth
 
Мені таксист у Львові показував новобудову, де цілими під'їздами викупили житло понаєхавші зі Сходу.

\begin{itemize}
%%%fbauth
%%%fbauth_name
\iusr{Ігор Мазепа}
%%%fbauth_url
%%%fbauth_place
%%%fbauth_id
%%%fbauth_front
%%%fbauth_desc
%%%fbauth_www
%%%fbauth_pic
%%%fbauth_pic portrait
%%%fbauth_pic background
%%%fbauth_pic other
%%%fbauth_tags
%%%fbauth_pubs
%%%endfbauth
 
\textbf{Руслан Янісевич} А смороду на півміста

%%%fbauth
%%%fbauth_name
\iusr{Ярослав Дацко}
%%%fbauth_url
%%%fbauth_place
%%%fbauth_id
%%%fbauth_front
%%%fbauth_desc
%%%fbauth_www
%%%fbauth_pic
%%%fbauth_pic portrait
%%%fbauth_pic background
%%%fbauth_pic other
%%%fbauth_tags
%%%fbauth_pubs
%%%endfbauth
 
\textbf{Руслан Янісевич} район новобудов Пасічний. Чтокають і какають на кожному кроці

%%%fbauth
%%%fbauth_name
\iusr{Оксана Ковалишин}
%%%fbauth_url
%%%fbauth_place
%%%fbauth_id
%%%fbauth_front
%%%fbauth_desc
%%%fbauth_www
%%%fbauth_pic
%%%fbauth_pic portrait
%%%fbauth_pic background
%%%fbauth_pic other
%%%fbauth_tags
%%%fbauth_pubs
%%%endfbauth
 
\textbf{Руслан Янісевич} І так по всіх містах. Наче прорвало відстійники. Лайно розтіклося по всій Україні.

%%%fbauth
%%%fbauth_name
\iusr{Руслан Янісевич}
%%%fbauth_url
%%%fbauth_place
%%%fbauth_id
%%%fbauth_front
%%%fbauth_desc
%%%fbauth_www
%%%fbauth_pic
%%%fbauth_pic portrait
%%%fbauth_pic background
%%%fbauth_pic other
%%%fbauth_tags
%%%fbauth_pubs
%%%endfbauth
 
\textbf{Оксана Ковалишин} ми їх українізуємо. Вірніше, їх нащадків. Інакше гріш нам ціна.
\end{itemize}

%%%fbauth
%%%fbauth_name
\iusr{Svitlana Rudych}
%%%fbauth_url
%%%fbauth_place
%%%fbauth_id
%%%fbauth_front
%%%fbauth_desc
%%%fbauth_www
%%%fbauth_pic
%%%fbauth_pic portrait
%%%fbauth_pic background
%%%fbauth_pic other
%%%fbauth_tags
%%%fbauth_pubs
%%%endfbauth
 

Таке явище є! Дуже точно Ви підмітили!

При чому, нічого не вдієш: такі ми є - м'які, і такі є вони - грубі (прямі) і
агресивні. І так, згодна абсолютно з Вами: стійкість б і наполегливості б нам у
захисті своїх кордонів!

%%%fbauth
%%%fbauth_name
\iusr{Павло Шубарт}
%%%fbauth_url
%%%fbauth_place
%%%fbauth_id
%%%fbauth_front
%%%fbauth_desc
%%%fbauth_www
%%%fbauth_pic
%%%fbauth_pic portrait
%%%fbauth_pic background
%%%fbauth_pic other
%%%fbauth_tags
%%%fbauth_pubs
%%%endfbauth
 

\ifcmt
  ig https://scontent-cdt1-1.xx.fbcdn.net/v/t1.6435-9/230296033_128763422766436_2272557178260454974_n.jpg?_nc_cat=110&ccb=1-3&_nc_sid=dbeb18&_nc_ohc=WSyae-3jwiAAX_nUBMm&_nc_ht=scontent-cdt1-1.xx&oh=f08ebf38690ba95862030cebf90c3d05&oe=61312DEE
  width 0.4
\fi

\begin{itemize}
%%%fbauth
%%%fbauth_name
\iusr{Serhii Bryhar}
%%%fbauth_url
%%%fbauth_place
%%%fbauth_id
%%%fbauth_front
%%%fbauth_desc
%%%fbauth_www
%%%fbauth_pic
%%%fbauth_pic portrait
%%%fbauth_pic background
%%%fbauth_pic other
%%%fbauth_tags
%%%fbauth_pubs
%%%endfbauth
 
Та я намагаюся так і робити).
\end{itemize}

%%%fbauth
%%%fbauth_name
\iusr{Павло Шубарт}
%%%fbauth_url
%%%fbauth_place
%%%fbauth_id
%%%fbauth_front
%%%fbauth_desc
%%%fbauth_www
%%%fbauth_pic
%%%fbauth_pic portrait
%%%fbauth_pic background
%%%fbauth_pic other
%%%fbauth_tags
%%%fbauth_pubs
%%%endfbauth
 

\ifcmt
  ig https://scontent-cdg2-1.xx.fbcdn.net/v/t1.6435-9/230864874_128763449433100_3866462886698248167_n.jpg?_nc_cat=111&ccb=1-3&_nc_sid=dbeb18&_nc_ohc=sq7X2g6e2DwAX_NOf0v&_nc_ht=scontent-cdg2-1.xx&oh=8383c60c235e3d5bf568f8be17c70aca&oe=612E51BA
  width 0.3
\fi

%%%fbauth
%%%fbauth_name
\iusr{Olha Gontar}
%%%fbauth_url
%%%fbauth_place
%%%fbauth_id
%%%fbauth_front
%%%fbauth_desc
%%%fbauth_www
%%%fbauth_pic
%%%fbauth_pic portrait
%%%fbauth_pic background
%%%fbauth_pic other
%%%fbauth_tags
%%%fbauth_pubs
%%%endfbauth
 

У Львів з’їхалося повно рузкоговорящих біженців і не біженців, починають
насаджувати свій язик і свої правила у всьому!

Маємо відстоювати свою рідну мову і ще мають виконуватися державні мовні
закони!!!

\begin{itemize}
%%%fbauth
%%%fbauth_name
\iusr{Ігор Мазепа}
%%%fbauth_url
%%%fbauth_place
%%%fbauth_id
%%%fbauth_front
%%%fbauth_desc
%%%fbauth_www
%%%fbauth_pic
%%%fbauth_pic portrait
%%%fbauth_pic background
%%%fbauth_pic other
%%%fbauth_tags
%%%fbauth_pubs
%%%endfbauth
 
\textbf{Olha Gontar} Слід на кожному кроці давати їм знати, що відносини з ними - важка для нас інклюзія.

%%%fbauth
%%%fbauth_name
\iusr{Натали Эльф}
%%%fbauth_url
%%%fbauth_place
%%%fbauth_id
%%%fbauth_front
%%%fbauth_desc
%%%fbauth_www
%%%fbauth_pic
%%%fbauth_pic portrait
%%%fbauth_pic background
%%%fbauth_pic other
%%%fbauth_tags
%%%fbauth_pubs
%%%endfbauth
 
\textbf{Olha Gontar} , абсолютна маячня. До тисячі зареєстровано. В той час, як в Києві та Дніпрі сотні тисяч.

%%%fbauth
%%%fbauth_name
\iusr{Victor Pavlyshyn}
%%%fbauth_url
%%%fbauth_place
%%%fbauth_id
%%%fbauth_front
%%%fbauth_desc
%%%fbauth_www
%%%fbauth_pic
%%%fbauth_pic portrait
%%%fbauth_pic background
%%%fbauth_pic other
%%%fbauth_tags
%%%fbauth_pubs
%%%endfbauth
 
\textbf{Olha Gontar} 

у Львові все були руськоязичники. Від совка ще. Я знаю це добре. Брати Львів як
перлину українства це смішно. Все що було українське знищувалося. Ті хто
говорить по українськи толерантно чи з інакших мотивацій ліберально дивляться
на руцьогаварящіх а чому? Тому що вони самі виросли при совдепі і їм якось то
всерівно хто як говорить. Шлунок. Маленька кількість людей яка дійсно не при
яких обставинах не хоче згубити свою українську ідентичність і кладе квіти
Шептицькому в соборі св. Юра. Це старі бабці дідусі і клаптик молоді яким
розказали що таке УПА і хто такий Шухевич. Не треба Львів брати як мекку
українського націоналізму. Він давно вже став містом таким як всі міста в
Україні. 

По друге ще при активності Українського супротива в час війни і після в Львів
заселяли руських бо так було. Армянська діаспора у Львові не говорить
українською а чому? Мікоян ще заселив їхніх дідів нквдістів у Львів. Все що там
заселилося після винищення корінного населення там і виросло. Все що говорить
по українськи це заселенці з області. Одним словом переєбано і перетрахано. 

ЗВИЧАЙНО коли приїзжає з Одесси патріотично налаштований турист який бачить
Львів у вишиванках він потрапляє в кумедну ситуацію бо йому манкурт таксист
відповідає на веліком. 

Я у Львові на звертання до мене на російському все відповідаю "Розмовляй
Українською або помри!" тоді трошки на агресивність люди начинають включати в
своїй макітрі міксер свого недоєбаного новорічного олівє яке у них гангреної
стоїть в чашці і кисне.

\end{itemize}

%%%fbauth
%%%fbauth_name
\iusr{Olha Gontar}
%%%fbauth_url
%%%fbauth_place
%%%fbauth_id
%%%fbauth_front
%%%fbauth_desc
%%%fbauth_www
%%%fbauth_pic
%%%fbauth_pic portrait
%%%fbauth_pic background
%%%fbauth_pic other
%%%fbauth_tags
%%%fbauth_pubs
%%%endfbauth
 

Львів‘янам просто потрібно їх не розуміти !!!

Особливо в сфері обслуговування і на роботі !!!

Хай розмовляють язиком між собою !!!

\begin{itemize}
%%%fbauth
%%%fbauth_name
\iusr{Ярослав Білозір}
%%%fbauth_url
%%%fbauth_place
%%%fbauth_id
%%%fbauth_front
%%%fbauth_desc
%%%fbauth_www
%%%fbauth_pic
%%%fbauth_pic portrait
%%%fbauth_pic background
%%%fbauth_pic other
%%%fbauth_tags
%%%fbauth_pubs
%%%endfbauth
 
Львів'нам потрібно їм буків надавати.
\end{itemize}

%%%fbauth
%%%fbauth_name
\iusr{Ihor Dmytryshyn}
%%%fbauth_url
%%%fbauth_place
%%%fbauth_id
%%%fbauth_front
%%%fbauth_desc
%%%fbauth_www
%%%fbauth_pic
%%%fbauth_pic portrait
%%%fbauth_pic background
%%%fbauth_pic other
%%%fbauth_tags
%%%fbauth_pubs
%%%endfbauth
 
Є новобудови, де більшість квартир викуплено біженцями зі сходу.

\begin{itemize}
%%%fbauth
%%%fbauth_name
\iusr{Оксана Ковалишин}
%%%fbauth_url
%%%fbauth_place
%%%fbauth_id
%%%fbauth_front
%%%fbauth_desc
%%%fbauth_www
%%%fbauth_pic
%%%fbauth_pic portrait
%%%fbauth_pic background
%%%fbauth_pic other
%%%fbauth_tags
%%%fbauth_pubs
%%%endfbauth
 
\textbf{Ihor Dmytryshyn} А як вам таке: волонтери добивалися, щоб квартири давали "руцкагаварясчім мальчікам", які лікувалися у Львівському госпіталі. Натомість місцевих скалічених хлопців відправляли до старої матері в село, де ні води ні зручностей?

%%%fbauth
%%%fbauth_name
\iusr{Оксана Ковалишин}
%%%fbauth_url
%%%fbauth_place
%%%fbauth_id
%%%fbauth_front
%%%fbauth_desc
%%%fbauth_www
%%%fbauth_pic
%%%fbauth_pic portrait
%%%fbauth_pic background
%%%fbauth_pic other
%%%fbauth_tags
%%%fbauth_pubs
%%%endfbauth
 
\textbf{Ihor Dmytryshyn} Я б, скоріше, використала лапки для "біженців", бо більшість з них, навряд чи можна так назвати. Якось ходила на курси з дівчиною з нашого міста. Першою до них придула сестра чоловіка і добилася всього, чого їй було треба. Чоловік цієї "біженки" ...тадам-тадам... воює в, так званій, ДНР. Батькам же знайомої, патріотично налаштовані українці (говорять українською - "яка дивина"🙄) ледве вдалося вирватися. Приїхали голі і босі. Кинули, все, що мали.

%%%fbauth
%%%fbauth_name
\iusr{Оксана Ковалишин}
%%%fbauth_url
%%%fbauth_place
%%%fbauth_id
%%%fbauth_front
%%%fbauth_desc
%%%fbauth_www
%%%fbauth_pic
%%%fbauth_pic portrait
%%%fbauth_pic background
%%%fbauth_pic other
%%%fbauth_tags
%%%fbauth_pubs
%%%endfbauth
 
\textbf{Ihor Dmytryshyn} Інший приклад: сусідка питає одну таку, чого та сюди приперлась, якщо вона так "нєнавідіт фашистав бєндєравцев"? А та каже:"у вас тут чіста, красіва, а ПАРЯДАК ми навідьом". Як ви думаєте, вона вулиці підмітати зібралася?

%%%fbauth
%%%fbauth_name
\iusr{Ihor Dmytryshyn}
%%%fbauth_url
%%%fbauth_place
%%%fbauth_id
%%%fbauth_front
%%%fbauth_desc
%%%fbauth_www
%%%fbauth_pic
%%%fbauth_pic portrait
%%%fbauth_pic background
%%%fbauth_pic other
%%%fbauth_tags
%%%fbauth_pubs
%%%endfbauth
 

Згідний, що біженці це умовна назва, не знаю їхнього юридичного статусу, а
використав це як умовне позначення. Знаю турка, що за 10 років вивчив
українську, користується нею, діти його теж розмовляють з ним українською, а
його дружина з українським прізвищем, сім'ї совєцкого військового на українську
так і не перейшла, а розмовляє з чоловіком окупантомовною, хоч живе у Львові
мабуть років з 20-ть.


%%%fbauth
%%%fbauth_name
\iusr{Iryna Terletska}
%%%fbauth_url
%%%fbauth_place
%%%fbauth_id
%%%fbauth_front
%%%fbauth_desc
%%%fbauth_www
%%%fbauth_pic
%%%fbauth_pic portrait
%%%fbauth_pic background
%%%fbauth_pic other
%%%fbauth_tags
%%%fbauth_pubs
%%%endfbauth
 
\textbf{Ihor Dmytryshyn} Біженці це окреме питання. Проблема в тім, що питома частка галичан толерує москвояз.

%%%fbauth
%%%fbauth_name
\iusr{Ihor Dmytryshyn}
%%%fbauth_url
%%%fbauth_place
%%%fbauth_id
%%%fbauth_front
%%%fbauth_desc
%%%fbauth_www
%%%fbauth_pic
%%%fbauth_pic portrait
%%%fbauth_pic background
%%%fbauth_pic other
%%%fbauth_tags
%%%fbauth_pubs
%%%endfbauth
 
\textbf{Iryna Terletska} 

Так, русофільство має давні корені, втім не тільки воно. А яка різниця між
галицьким русофілом, та русофілом з Донецька? Галицький, окрім російської
володіє українською в першу чергу. А поляколюби чи германофіли, котрих значно
менше часто і речення не можуть зв'язати мовою своїх кумирів. То люди різні,
від недалеких до клоунів, що так заробляють.


%%%fbauth
%%%fbauth_name
\iusr{Iryna Terletska}
%%%fbauth_url
%%%fbauth_place
%%%fbauth_id
%%%fbauth_front
%%%fbauth_desc
%%%fbauth_www
%%%fbauth_pic
%%%fbauth_pic portrait
%%%fbauth_pic background
%%%fbauth_pic other
%%%fbauth_tags
%%%fbauth_pubs
%%%endfbauth
 
\textbf{Ihor Dmytryshyn} 

Ну, від Донецького русофіла русофільство очікуване. Його так формували
десятиліттями. Натомість, галицький мав усі можливості бути нормальною людиною.
Тому до нього вищі вимоги.

\end{itemize}

%%%fbauth
%%%fbauth_name
\iusr{Вікторія Валевська}
%%%fbauth_url
%%%fbauth_place
%%%fbauth_id
%%%fbauth_front
%%%fbauth_desc
%%%fbauth_www
%%%fbauth_pic
%%%fbauth_pic portrait
%%%fbauth_pic background
%%%fbauth_pic other
%%%fbauth_tags
%%%fbauth_pubs
%%%endfbauth
 

Як я вас розумію. Їдеш до Львова з такою надією. що навкруги виключно свої, які
розмовляють однією з тобою мовою. І кожного разу боляче як вперше. А в 14-му
році, якраз перед Іловайськом, зі мною взагалі сталася жахлива історія. Мене
мали зустріти у Львові, я мала знаходитися там день, а потім мене б посадили на
потяг до Праги. Людей я цих не знала, бачила вперше. А це якраз після Майдану,
війна, добровольці, піднесення, що ми нація, а Львів колиска цієї нації. Ну
коротше, привозять ці милі люди мене до себе додому, я заходжу в кімнату, а там
стоїть триколор, і всі в квартирі розмовляють російською. Досі як згадаю, так
хочеться помитися.

\begin{itemize}
%%%fbauth
%%%fbauth_name
\iusr{Ігор Мазепа}
%%%fbauth_url
%%%fbauth_place
%%%fbauth_id
%%%fbauth_front
%%%fbauth_desc
%%%fbauth_www
%%%fbauth_pic
%%%fbauth_pic portrait
%%%fbauth_pic background
%%%fbauth_pic other
%%%fbauth_tags
%%%fbauth_pubs
%%%endfbauth
 
\textbf{Вікторія Валевська} таки маємо проблему з тими непрошеними понаєхавшими.

%%%fbauth
%%%fbauth_name
\iusr{Вікторія Валевська}
%%%fbauth_url
%%%fbauth_place
%%%fbauth_id
%%%fbauth_front
%%%fbauth_desc
%%%fbauth_www
%%%fbauth_pic
%%%fbauth_pic portrait
%%%fbauth_pic background
%%%fbauth_pic other
%%%fbauth_tags
%%%fbauth_pubs
%%%endfbauth
 
\textbf{Ігор Мазепа} вони народились у Львові, хоча їх батьки так, понаєхавші. Здається, навіть з московії.

%%%fbauth
%%%fbauth_name
\iusr{Юлія Зак}
%%%fbauth_url
%%%fbauth_place
%%%fbauth_id
%%%fbauth_front
%%%fbauth_desc
%%%fbauth_www
%%%fbauth_pic
%%%fbauth_pic portrait
%%%fbauth_pic background
%%%fbauth_pic other
%%%fbauth_tags
%%%fbauth_pubs
%%%endfbauth
 
\textbf{Ігор Мазепа} у Львові і асвабадітєльскіх внучков вистачає...

%%%fbauth
%%%fbauth_name
\iusr{Оксана Ковалишин}
%%%fbauth_url
%%%fbauth_place
%%%fbauth_id
%%%fbauth_front
%%%fbauth_desc
%%%fbauth_www
%%%fbauth_pic
%%%fbauth_pic portrait
%%%fbauth_pic background
%%%fbauth_pic other
%%%fbauth_tags
%%%fbauth_pubs
%%%endfbauth
 
\textbf{Ігор Мазепа} Але не будьмо ж і такі сліпі, проше пана. Скількі
українців готові тим руцкоговорясчім кланятися! Дуже вже толерантні. Аж гидко!
\end{itemize}

%%%fbauth
%%%fbauth_name
\iusr{Галина Шаповал}
%%%fbauth_url
%%%fbauth_place
%%%fbauth_id
%%%fbauth_front
%%%fbauth_desc
%%%fbauth_www
%%%fbauth_pic
%%%fbauth_pic portrait
%%%fbauth_pic background
%%%fbauth_pic other
%%%fbauth_tags
%%%fbauth_pubs
%%%endfbauth
 

Це Сихів, найбільший спальний район Львова. З початком війни тут з'явились
переселенці з Донбасу. А ще перестарілі нащадки савєтскіх дєдов, які продали
свої хороші квартири в австрійській чи польській забудові центральної частини
міста, купили непорівняно дешевші тут і "на решту" живуть і горя не знають.
Звідси така кількість цього всього вколись україномовному районі...


%%%fbauth
%%%fbauth_name
\iusr{Валентина Журбенко}
%%%fbauth_url
%%%fbauth_place
%%%fbauth_id
%%%fbauth_front
%%%fbauth_desc
%%%fbauth_www
%%%fbauth_pic
%%%fbauth_pic portrait
%%%fbauth_pic background
%%%fbauth_pic other
%%%fbauth_tags
%%%fbauth_pubs
%%%endfbauth
 

Сергію, я думаю, що причина в міграції російськомовного населення з окупованого
Донбасу. Деякі з них толерантні і швидко вчать мову тої держави, в яку
переїхали жити. Але є такі, які вперто не хочуть, та ще й підтасовують під себе
оточуючих. А ми - україномовні - люди інтелігентні, виховані, тому мовчимо і
поступово переходимо на їхню, щоб догодити постраждалим...

\begin{itemize}
%%%fbauth
%%%fbauth_name
\iusr{Оксана Поєдинкова}
%%%fbauth_url
%%%fbauth_place
%%%fbauth_id
%%%fbauth_front
%%%fbauth_desc
%%%fbauth_www
%%%fbauth_pic
%%%fbauth_pic portrait
%%%fbauth_pic background
%%%fbauth_pic other
%%%fbauth_tags
%%%fbauth_pubs
%%%endfbauth
 
\textbf{Валентина Журбенко} Я на російську не перехожу ні де. Ні на роботі, ні в магазинах, ні у спілкуванні з людьми.

%%%fbauth
%%%fbauth_name
\iusr{Iryna Terletska}
%%%fbauth_url
%%%fbauth_place
%%%fbauth_id
%%%fbauth_front
%%%fbauth_desc
%%%fbauth_www
%%%fbauth_pic
%%%fbauth_pic portrait
%%%fbauth_pic background
%%%fbauth_pic other
%%%fbauth_tags
%%%fbauth_pubs
%%%endfbauth
 
\textbf{Валентина Журбенко} На жаль, і серед галичан вистачає таких, що охоче
переходять на російську. Інакше мігранти не почувалися б так комфортно зі своїм
москвоязом.

%%%fbauth
%%%fbauth_name
\iusr{Валентина Журбенко}
%%%fbauth_url
%%%fbauth_place
%%%fbauth_id
%%%fbauth_front
%%%fbauth_desc
%%%fbauth_www
%%%fbauth_pic
%%%fbauth_pic portrait
%%%fbauth_pic background
%%%fbauth_pic other
%%%fbauth_tags
%%%fbauth_pubs
%%%endfbauth
 
Згодна. Ми повинні самі себе поважати, а не співбесідника, який, по суті, нашу мову розуміє, або зобов'язаний розуміти.
\end{itemize}

%%%fbauth
%%%fbauth_name
\iusr{Юлія Зак}
%%%fbauth_url
%%%fbauth_place
%%%fbauth_id
%%%fbauth_front
%%%fbauth_desc
%%%fbauth_www
%%%fbauth_pic
%%%fbauth_pic portrait
%%%fbauth_pic background
%%%fbauth_pic other
%%%fbauth_tags
%%%fbauth_pubs
%%%endfbauth
 
Не лише на Сихові. І вже давно...

%%%fbauth
%%%fbauth_name
\iusr{Tetyana Stepanivna}
%%%fbauth_url
%%%fbauth_place
%%%fbauth_id
%%%fbauth_front
%%%fbauth_desc
%%%fbauth_www
%%%fbauth_pic
%%%fbauth_pic portrait
%%%fbauth_pic background
%%%fbauth_pic other
%%%fbauth_tags
%%%fbauth_pubs
%%%endfbauth
 
Катерина ІІ знищила козаків. Вона знала що робить.

От нам і не вистачає козацького духу, їхніх нащадків, того міцного коріння.

Не можемо й досі відродитися.

%%%fbauth
%%%fbauth_name
\iusr{Natalija Kireeva}
%%%fbauth_url
%%%fbauth_place
%%%fbauth_id
%%%fbauth_front
%%%fbauth_desc
%%%fbauth_www
%%%fbauth_pic
%%%fbauth_pic portrait
%%%fbauth_pic background
%%%fbauth_pic other
%%%fbauth_tags
%%%fbauth_pubs
%%%endfbauth
 
Тримаємо стрій!

\ifcmt
  ig https://scontent-cdg2-1.xx.fbcdn.net/v/t39.30808-6/226994873_974082960092408_971215731814423256_n.jpg?_nc_cat=102&ccb=1-3&_nc_sid=dbeb18&_nc_ohc=mx44540x_LwAX9JDj24&_nc_ht=scontent-cdg2-1.xx&oh=e74fb8dd463c6cab262819b86eaaa722&oe=610F5BE5
  width 0.3
\fi

%%%fbauth
%%%fbauth_name
\iusr{Анжеліка Олексович}
%%%fbauth_url
%%%fbauth_place
%%%fbauth_id
%%%fbauth_front
%%%fbauth_desc
%%%fbauth_www
%%%fbauth_pic
%%%fbauth_pic portrait
%%%fbauth_pic background
%%%fbauth_pic other
%%%fbauth_tags
%%%fbauth_pubs
%%%endfbauth
 

Підтримую! Нещодавно в Форумі молоденька продавщиня так лепетала російською, бо
,бачте, грошовита коієнтка попалась, яка прекрасно розуміла українську.

\begin{itemize}
%%%fbauth
%%%fbauth_name
\iusr{Оксана Ковалишин}
%%%fbauth_url
%%%fbauth_place
%%%fbauth_id
%%%fbauth_front
%%%fbauth_desc
%%%fbauth_www
%%%fbauth_pic
%%%fbauth_pic portrait
%%%fbauth_pic background
%%%fbauth_pic other
%%%fbauth_tags
%%%fbauth_pubs
%%%endfbauth
 
От найбільша проблема. Якби не це, то з "понаєхавшимі" ми б легко справилися!
\end{itemize}

%%%fbauth
%%%fbauth_name
\iusr{Taras Oliynyk}
%%%fbauth_url
%%%fbauth_place
%%%fbauth_id
%%%fbauth_front
%%%fbauth_desc
%%%fbauth_www
%%%fbauth_pic
%%%fbauth_pic portrait
%%%fbauth_pic background
%%%fbauth_pic other
%%%fbauth_tags
%%%fbauth_pubs
%%%endfbauth
 

Нема жодної загрози русифікації Львова. Принаймі зараз, поки існує, хоч і
формальна Українська Держава.

Я пам'ятаю наскільки був русифікований Львів за пізнього совка. Особливо в
центрі. І всіх їх перемолола україномовна більшість. Всі вони і їхні нащадки
змушені на публіці розмовляти українською. При чому ніхто їх не бив і не
особливо гнітив. Просто така атмосфера загальна.

І не треба нам переймати традицій орків. Бо ми просто інші. Тому і методи
асиміляції в нас свої - не переходити на російську. І це є переважаюча тактика
уЛьвові. Звісно є ще досить тих хто прогинається. Але весь час їх відсоток
зменшується.

І ще не таких узкоязичних перемелював. Ну або випльовував.

\begin{itemize}
%%%fbauth
%%%fbauth_name
\iusr{Helena Dubravina}
%%%fbauth_url
%%%fbauth_place
%%%fbauth_id
%%%fbauth_front
%%%fbauth_desc
%%%fbauth_www
%%%fbauth_pic
%%%fbauth_pic portrait
%%%fbauth_pic background
%%%fbauth_pic other
%%%fbauth_tags
%%%fbauth_pubs
%%%endfbauth
 
\textbf{Taras Oliynyk} На жаль, Львів дійсно змінюється не в кращу сторону. А ви ж останній форпост! По українську атмосферу їдуть саме туди!

%%%fbauth
%%%fbauth_name
\iusr{Serhii Bryhar}
%%%fbauth_url
%%%fbauth_place
%%%fbauth_id
%%%fbauth_front
%%%fbauth_desc
%%%fbauth_www
%%%fbauth_pic
%%%fbauth_pic portrait
%%%fbauth_pic background
%%%fbauth_pic other
%%%fbauth_tags
%%%fbauth_pubs
%%%endfbauth
 
\textbf{Taras Oliynyk} Оптимістично звучить. Хочеться вірити, що усе саме так).
Просто я вже багато років їжджу, спостерігаю. І як на мене, кількість "язика"
постійно зростає... Я ж не кажу, що їх треба бити). Мене теж в Одесі ніхто не
б'є))). Середовище мене ламає і намагається підлаштувати під себе. Ніколи не
казав, що мені тут легко і комфортно. Україномовні в цьому місті - бідна і
малочисельна діаспора. А от коли дивишся на "рускоязичними" у Львові, то
складається враження, що вони аж ніяк не бідні, не малочисельні, що в усіх
сферах почуваються цілком собі комфортно.. Оцей контраст мені якось не
подобається.

%%%fbauth
%%%fbauth_name
\iusr{Оксана Ковалишин}
%%%fbauth_url
%%%fbauth_place
%%%fbauth_id
%%%fbauth_front
%%%fbauth_desc
%%%fbauth_www
%%%fbauth_pic
%%%fbauth_pic portrait
%%%fbauth_pic background
%%%fbauth_pic other
%%%fbauth_tags
%%%fbauth_pubs
%%%endfbauth
 
\textbf{Taras Oliynyk} Ваші слова та богові б вуха.

%%%fbauth
%%%fbauth_name
\iusr{Anatolii Gumenyuk}
%%%fbauth_url
%%%fbauth_place
%%%fbauth_id
%%%fbauth_front
%%%fbauth_desc
%%%fbauth_www
%%%fbauth_pic
%%%fbauth_pic portrait
%%%fbauth_pic background
%%%fbauth_pic other
%%%fbauth_tags
%%%fbauth_pubs
%%%endfbauth
 
\textbf{Helena Dubravina} 

Іноді дрбре і зручно сприймати бажане за дійсне, але дійсність не така, як вам
хочеться її бачити. Совкова русифікаторська традиція не оминула Львів. А те, що
тут багато місцевих айтішників означаж, що вони часто вимушені спілкуватися
російською (бо ж інтернет ресурси!) Але головне - відсутність непримиренних
патріотичних переконань у молодого покоління, розуміння необхідності
відстоювання рідної мови і доброзичливих "порад" гостям зі сходу України вчити
українську. А можливо і глузування та висміювання їх небажання або нездатність
її вивчити. Так само, як вони вболівають і глузують з поодиноких україномовних в
Одесі, Миколаєві чи Харкові...


%%%fbauth
%%%fbauth_name
\iusr{Максим Меркулов}
%%%fbauth_url
%%%fbauth_place
%%%fbauth_id
%%%fbauth_front
%%%fbauth_desc
%%%fbauth_www
%%%fbauth_pic
%%%fbauth_pic portrait
%%%fbauth_pic background
%%%fbauth_pic other
%%%fbauth_tags
%%%fbauth_pubs
%%%endfbauth
 
\textbf{Helena Dubravina} 

Не сказав би. Франківськ, Тернопіль і Коломия значно україномовніші. У
Переяславі (Київщина, година їзди від столиці) я чув менше російської, ніж
Львові. А це аж ніяк не Галичина.

\end{itemize}

%%%fbauth
%%%fbauth_name
\iusr{Валерій Лапікура}
%%%fbauth_url
%%%fbauth_place
%%%fbauth_id
%%%fbauth_front
%%%fbauth_desc
%%%fbauth_www
%%%fbauth_pic
%%%fbauth_pic portrait
%%%fbauth_pic background
%%%fbauth_pic other
%%%fbauth_tags
%%%fbauth_pubs
%%%endfbauth
 

Сиділи-балакали про все і ні про що. З розбору фінського серіалу перейшли на
фінську ментальність. І тут вигулькнуло питання: чому свого часу фіни не
з'ясовували, хто пролетар, а хто інтелігент, а взялися всі разом до зброї і
влиндили Червоній армії? І чому ми так не можемо?

Відповідь спливла одразу: у фінів не було понаєхів.

\begin{itemize}
%%%fbauth
%%%fbauth_name
\iusr{Петро Кагала}
%%%fbauth_url
%%%fbauth_place
%%%fbauth_id
%%%fbauth_front
%%%fbauth_desc
%%%fbauth_www
%%%fbauth_pic
%%%fbauth_pic portrait
%%%fbauth_pic background
%%%fbauth_pic other
%%%fbauth_tags
%%%fbauth_pubs
%%%endfbauth
 
І суттєве те, що оці "понаєхи",відразу ж, повели себе-як дома, начисто ігноруючи положення про гостей! Ну, та- "какаяразніца"!..

%%%fbauth
%%%fbauth_name
\iusr{Оксана Ковалишин}
%%%fbauth_url
%%%fbauth_place
%%%fbauth_id
%%%fbauth_front
%%%fbauth_desc
%%%fbauth_www
%%%fbauth_pic
%%%fbauth_pic portrait
%%%fbauth_pic background
%%%fbauth_pic other
%%%fbauth_tags
%%%fbauth_pubs
%%%endfbauth
 
Якби ми не були такі "толерантні", то справилися б і з понаєхами. А то гнучкі вже задуже спини.

%%%fbauth
%%%fbauth_name
\iusr{Оксана Ковалишин}
%%%fbauth_url
%%%fbauth_place
%%%fbauth_id
%%%fbauth_front
%%%fbauth_desc
%%%fbauth_www
%%%fbauth_pic
%%%fbauth_pic portrait
%%%fbauth_pic background
%%%fbauth_pic other
%%%fbauth_tags
%%%fbauth_pubs
%%%endfbauth
 
\textbf{Петро Кагала} Так, а чом би їм це не робити, як дозволяють. Я не кажу
про ті часи, коли треба було голими грудьми проти танків (хоча й тоді такі
траплялися), а що заставляє зараз українців так гнутися? Розумію, що та наволоч
встигла захопити і владу і гроші, але ж честь то у людей мала лишитися?
\end{itemize}

%%%fbauth
%%%fbauth_name
\iusr{Olha Gontar}
%%%fbauth_url
%%%fbauth_place
%%%fbauth_id
%%%fbauth_front
%%%fbauth_desc
%%%fbauth_www
%%%fbauth_pic
%%%fbauth_pic portrait
%%%fbauth_pic background
%%%fbauth_pic other
%%%fbauth_tags
%%%fbauth_pubs
%%%endfbauth
 

Потрібно зрозуміти або українці ( не плутати з населенням) будуть або не будуть !!!

Потрібно мати свою Гідність по всій Україні !!!

\begin{itemize}
%%%fbauth
%%%fbauth_name
\iusr{Оксана Ковалишин}
%%%fbauth_url
%%%fbauth_place
%%%fbauth_id
%%%fbauth_front
%%%fbauth_desc
%%%fbauth_www
%%%fbauth_pic
%%%fbauth_pic portrait
%%%fbauth_pic background
%%%fbauth_pic other
%%%fbauth_tags
%%%fbauth_pubs
%%%endfbauth
 
\textbf{Olha Gontar} Так, якщо не плутати, то гідність є. Але ж скільки таких?
На жаль, в основній масі тієї гідності нема. От часто доводиться чути:"не
називайте їх українцями". А як, перепрошую? Українці - це національність, а не
чеснота.

%%%fbauth
%%%fbauth_name
\iusr{Olha Gontar}
%%%fbauth_url
%%%fbauth_place
%%%fbauth_id
%%%fbauth_front
%%%fbauth_desc
%%%fbauth_www
%%%fbauth_pic
%%%fbauth_pic portrait
%%%fbauth_pic background
%%%fbauth_pic other
%%%fbauth_tags
%%%fbauth_pubs
%%%endfbauth
 
\textbf{Оксана Ковалишин} українці бережуть свою мову і історію , звичаї ,
вчить дітей рідної мови а населення просто проживає на цій території.
Розмовляємо українською мовою усюди по Україні а там може і населення зрозуміє
!
\end{itemize}

%%%fbauth
%%%fbauth_name
\iusr{Іван Ковівчак}
%%%fbauth_url
%%%fbauth_place
%%%fbauth_id
%%%fbauth_front
%%%fbauth_desc
%%%fbauth_www
%%%fbauth_pic
%%%fbauth_pic portrait
%%%fbauth_pic background
%%%fbauth_pic other
%%%fbauth_tags
%%%fbauth_pubs
%%%endfbauth
 

\ifcmt
  ig https://scontent-cdt1-1.xx.fbcdn.net/v/t39.30808-6/229447747_2336656076467968_449062478644885886_n.jpg?_nc_cat=110&ccb=1-3&_nc_sid=dbeb18&_nc_ohc=rO9paq3f2wIAX-bPNtE&_nc_ht=scontent-cdt1-1.xx&oh=0fc8fbd74b4b8ca7bad834b211928a48&oe=610F39B4
  width 0.3
\fi

%%%fbauth
%%%fbauth_name
\iusr{Галя Капиця}
%%%fbauth_url
%%%fbauth_place
%%%fbauth_id
%%%fbauth_front
%%%fbauth_desc
%%%fbauth_www
%%%fbauth_pic
%%%fbauth_pic portrait
%%%fbauth_pic background
%%%fbauth_pic other
%%%fbauth_tags
%%%fbauth_pubs
%%%endfbauth
 

Я прийшла в поліклініку в Сихові !!!!!на прийом до невропатолога. Загалом біля
дверей нас стояло четверо в черзі. Троє з нас розмовляли російською!!! Це жах!!!

\begin{itemize}
%%%fbauth
%%%fbauth_name
\iusr{Serhii Bryhar}
%%%fbauth_url
%%%fbauth_place
%%%fbauth_id
%%%fbauth_front
%%%fbauth_desc
%%%fbauth_www
%%%fbauth_pic
%%%fbauth_pic portrait
%%%fbauth_pic background
%%%fbauth_pic other
%%%fbauth_tags
%%%fbauth_pubs
%%%endfbauth
 
Нові реалії?(.. дехто, правда, стверджує, щось на зразок "так било фсєхда", але, ми ж бачимо, що протягом останніх років зміни таки є..

%%%fbauth
%%%fbauth_name
\iusr{Галя Капиця}
%%%fbauth_url
%%%fbauth_place
%%%fbauth_id
%%%fbauth_front
%%%fbauth_desc
%%%fbauth_www
%%%fbauth_pic
%%%fbauth_pic portrait
%%%fbauth_pic background
%%%fbauth_pic other
%%%fbauth_tags
%%%fbauth_pubs
%%%endfbauth
 
\textbf{Serhii Bryhar}
Дуже багато розвелося москалів у Львові.І це біда!!Воно нагле, воно домінує.

%%%fbauth
%%%fbauth_name
\iusr{Сергій Лейчук}
%%%fbauth_url
%%%fbauth_place
%%%fbauth_id
%%%fbauth_front
%%%fbauth_desc
%%%fbauth_www
%%%fbauth_pic
%%%fbauth_pic portrait
%%%fbauth_pic background
%%%fbauth_pic other
%%%fbauth_tags
%%%fbauth_pubs
%%%endfbauth
 
\textbf{Галя Капиця}
так то воно так, але, мо' й тішитися треба, що в поліклініку?!
в їх ще є шанс.
Може вилікуються...

%%%fbauth
%%%fbauth_name
\iusr{Світлана Літвінова}
%%%fbauth_url
%%%fbauth_place
%%%fbauth_id
%%%fbauth_front
%%%fbauth_desc
%%%fbauth_www
%%%fbauth_pic
%%%fbauth_pic portrait
%%%fbauth_pic background
%%%fbauth_pic other
%%%fbauth_tags
%%%fbauth_pubs
%%%endfbauth
 
\textbf{Галя Капиця} Домінує?
Ви це толеруєте?
Толерантність, мед.термін, назва тяжкої хвороби.

%%%fbauth
%%%fbauth_name
\iusr{Оксана Ковалишин}
%%%fbauth_url
%%%fbauth_place
%%%fbauth_id
%%%fbauth_front
%%%fbauth_desc
%%%fbauth_www
%%%fbauth_pic
%%%fbauth_pic portrait
%%%fbauth_pic background
%%%fbauth_pic other
%%%fbauth_tags
%%%fbauth_pubs
%%%endfbauth
 
\textbf{Галя Капиця} Не давайте!

%%%fbauth
%%%fbauth_name
\iusr{Олена Шержукова}
%%%fbauth_url
%%%fbauth_place
%%%fbauth_id
%%%fbauth_front
%%%fbauth_desc
%%%fbauth_www
%%%fbauth_pic
%%%fbauth_pic portrait
%%%fbauth_pic background
%%%fbauth_pic other
%%%fbauth_tags
%%%fbauth_pubs
%%%endfbauth
 
\textbf{Галя Капиця} ви не вважаєте це наслідками війни на донбасі та руху переселенців країною? бо мені саме так здається, що у нас в Харкові, який прийняв величезну кількість переселенців, ситуація суттєво погіршилась для українців;

\end{itemize}

%%%fbauth
%%%fbauth_name
\iusr{Володимир Костенко}
%%%fbauth_url
%%%fbauth_place
%%%fbauth_id
%%%fbauth_front
%%%fbauth_desc
%%%fbauth_www
%%%fbauth_pic
%%%fbauth_pic portrait
%%%fbauth_pic background
%%%fbauth_pic other
%%%fbauth_tags
%%%fbauth_pubs
%%%endfbauth
 

\ifcmt
  ig https://scontent-cdg2-1.xx.fbcdn.net/v/t39.30808-6/229240803_348761566726132_3231041118358514864_n.jpg?_nc_cat=107&ccb=1-3&_nc_sid=dbeb18&_nc_ohc=0wsij2HvKUkAX-NkNpb&_nc_ht=scontent-cdg2-1.xx&oh=41d78cda96cbe070a34052948694c131&oe=610FA02A
  width 0.4
\fi


%%%fbauth
%%%fbauth_name
\iusr{Iryna Hovorko}
%%%fbauth_url
%%%fbauth_place
%%%fbauth_id
%%%fbauth_front
%%%fbauth_desc
%%%fbauth_www
%%%fbauth_pic
%%%fbauth_pic portrait
%%%fbauth_pic background
%%%fbauth_pic other
%%%fbauth_tags
%%%fbauth_pubs
%%%endfbauth
 

Треба, щоб було як колись. Навіть у потягу до Львова тобі не відповідали, якщо
ти суржиком заговориш. Колись їхали і спитали "у вас є спічки"?, То провідниця
не відреагувала, поки не сказали "сірники". Треба, щоб така реакція була по
всій Україні

\begin{itemize}
%%%fbauth
%%%fbauth_name
\iusr{Serhii Bryhar}
%%%fbauth_url
%%%fbauth_place
%%%fbauth_id
%%%fbauth_front
%%%fbauth_desc
%%%fbauth_www
%%%fbauth_pic
%%%fbauth_pic portrait
%%%fbauth_pic background
%%%fbauth_pic other
%%%fbauth_tags
%%%fbauth_pubs
%%%endfbauth
 

От розповідали ж таки байки. І моя мама колись їздила до Львова зі своїм
нерафвнованим подільським суржиком, і теж стверджувала, що реалували так собі.
Або навіть просили говорити українською! І куди ж то все поділося? Де воно?
Більше немає навіть натяків. Тепер місцеві й самі регулярно "пєрєходять на
русскій".


%%%fbauth
%%%fbauth_name
\iusr{Олеся Чапцева-Білоус}
%%%fbauth_url
%%%fbauth_place
%%%fbauth_id
%%%fbauth_front
%%%fbauth_desc
%%%fbauth_www
%%%fbauth_pic
%%%fbauth_pic portrait
%%%fbauth_pic background
%%%fbauth_pic other
%%%fbauth_tags
%%%fbauth_pubs
%%%endfbauth
 
\textbf{Iryna Hovorko} провідниця просто не розуміла, що таке спічки)

%%%fbauth
%%%fbauth_name
\iusr{Iryna Hovorko}
%%%fbauth_url
%%%fbauth_place
%%%fbauth_id
%%%fbauth_front
%%%fbauth_desc
%%%fbauth_www
%%%fbauth_pic
%%%fbauth_pic portrait
%%%fbauth_pic background
%%%fbauth_pic other
%%%fbauth_tags
%%%fbauth_pubs
%%%endfbauth
 
\textbf{Olesya Chaptseva-Bilous} ні, тоді вони так реагували на рос язик. Якби це було так і зараз, то не було б там напрієхавших
\end{itemize}

%%%fbauth
%%%fbauth_name
\iusr{Євгения Подмаркова}
%%%fbauth_url
%%%fbauth_place
%%%fbauth_id
%%%fbauth_front
%%%fbauth_desc
%%%fbauth_www
%%%fbauth_pic
%%%fbauth_pic portrait
%%%fbauth_pic background
%%%fbauth_pic other
%%%fbauth_tags
%%%fbauth_pubs
%%%endfbauth
 
Абсолютно згодна

%%%fbauth
%%%fbauth_name
\iusr{Микола Паламарчук}
%%%fbauth_url
%%%fbauth_place
%%%fbauth_id
%%%fbauth_front
%%%fbauth_desc
%%%fbauth_www
%%%fbauth_pic
%%%fbauth_pic portrait
%%%fbauth_pic background
%%%fbauth_pic other
%%%fbauth_tags
%%%fbauth_pubs
%%%endfbauth
 
То є так. Не той тепер Львів ,що був у 90-х, і тепер все більше ,,опускається" ,і в культурі теж.

%%%fbauth
%%%fbauth_name
\iusr{Helena Dubravina}
%%%fbauth_url
%%%fbauth_place
%%%fbauth_id
%%%fbauth_front
%%%fbauth_desc
%%%fbauth_www
%%%fbauth_pic
%%%fbauth_pic portrait
%%%fbauth_pic background
%%%fbauth_pic other
%%%fbauth_tags
%%%fbauth_pubs
%%%endfbauth
 

Головная зброя українця в цьому питанні - не переходити на російську з
російськомовним. Принаймі вимушені розуміти, якщо не говорити, словниковий запас
накопичують мимохіть.

\begin{itemize}
%%%fbauth
%%%fbauth_name
\iusr{Serhii Bryhar}
%%%fbauth_url
%%%fbauth_place
%%%fbauth_id
%%%fbauth_front
%%%fbauth_desc
%%%fbauth_www
%%%fbauth_pic
%%%fbauth_pic portrait
%%%fbauth_pic background
%%%fbauth_pic other
%%%fbauth_tags
%%%fbauth_pubs
%%%endfbauth
 

Так, переходити не можна! Але й тут ситуація така собі... Інформаційний фон то
в них не надто й відрізняється від київського, одеського, харківського. Тут усе
більше переконання у тому, що "нема різниці". Рускомірська зараза активно
поширюється..


%%%fbauth
%%%fbauth_name
\iusr{Оксана Ковалишин}
%%%fbauth_url
%%%fbauth_place
%%%fbauth_id
%%%fbauth_front
%%%fbauth_desc
%%%fbauth_www
%%%fbauth_pic
%%%fbauth_pic portrait
%%%fbauth_pic background
%%%fbauth_pic other
%%%fbauth_tags
%%%fbauth_pubs
%%%endfbauth
 
\textbf{Helena Dubravina} А я б сказала, що головна зброя - це почуття власної
гідності. Бо, як його нема, то хто ж буде використовувати вашу?
\end{itemize}

%%%fbauth
%%%fbauth_name
\iusr{Ntina Ntoubrova}
%%%fbauth_url
%%%fbauth_place
%%%fbauth_id
%%%fbauth_front
%%%fbauth_desc
%%%fbauth_www
%%%fbauth_pic
%%%fbauth_pic portrait
%%%fbauth_pic background
%%%fbauth_pic other
%%%fbauth_tags
%%%fbauth_pubs
%%%endfbauth
 

переселенці з Криму та Донбасу у Львів приїхали та заселилися. Бо 

1) Польща близько, якщо війна пошириться за межі Криму та Донбасу - їм буде
куди далі тікати. 

2) кордони на виїзд саме в Польщі дуже добре облаштовані.

Місцеві дурки та дурепи цього досі не прохавали.

%%%fbauth
%%%fbauth_name
\iusr{Wlad Wasyliuk}
%%%fbauth_url
%%%fbauth_place
%%%fbauth_id
%%%fbauth_front
%%%fbauth_desc
%%%fbauth_www
%%%fbauth_pic
%%%fbauth_pic portrait
%%%fbauth_pic background
%%%fbauth_pic other
%%%fbauth_tags
%%%fbauth_pubs
%%%endfbauth
 
Єдіная страна!

%%%fbauth
%%%fbauth_name
\iusr{Maria Nakonecna}
%%%fbauth_url
%%%fbauth_place
%%%fbauth_id
%%%fbauth_front
%%%fbauth_desc
%%%fbauth_www
%%%fbauth_pic
%%%fbauth_pic portrait
%%%fbauth_pic background
%%%fbauth_pic other
%%%fbauth_tags
%%%fbauth_pubs
%%%endfbauth
 

Лізуть до Європи зі своїми лаптями. Тундра не вичухана.

З таким нашим народом нема нічого дивного.

У Прибалтиці такого бути не буде.

\begin{itemize}
%%%fbauth
%%%fbauth_name
\iusr{Ярослав Білозір}
%%%fbauth_url
%%%fbauth_place
%%%fbauth_id
%%%fbauth_front
%%%fbauth_desc
%%%fbauth_www
%%%fbauth_pic
%%%fbauth_pic portrait
%%%fbauth_pic background
%%%fbauth_pic other
%%%fbauth_tags
%%%fbauth_pubs
%%%endfbauth
 
"Прибалтика" - то московське визначення. Просто Балтика, або країни Балтії.
\end{itemize}

%%%fbauth
%%%fbauth_name
\iusr{Оксана Дідик}
%%%fbauth_url
%%%fbauth_place
%%%fbauth_id
%%%fbauth_front
%%%fbauth_desc
%%%fbauth_www
%%%fbauth_pic
%%%fbauth_pic portrait
%%%fbauth_pic background
%%%fbauth_pic other
%%%fbauth_tags
%%%fbauth_pubs
%%%endfbauth
 

В Трускавці в санаторії "КАРПАТИ" на процедурах, відчиняються двері, лікар:"Слєдующій!"

\begin{itemize}
\item Я.
\item "На што жалуєтєсь?"
\item Я: "Чого ви до мене російською?
Я що в Росію приїхала лікуватися?"
\item Лікар: "Бо тут багато російськомовних."
\item Я: "То й що? Хіба донецькі не розуміють української? Чого ви, як слуги росіян? Чого така неповага до своєї мови? Чого, як другосортні?
Я як поїду в Польщу, чи в Чехію, чи в Росію, то ніхто не намагається зі мною
говорити українською. Це мої проблеми їх мова. Тому ми вчимо їх мову!!!
От, як хочуть лікуватися в Україні, хай вчать українську мову. "
\end{itemize}

%%%fbauth
%%%fbauth_name
\iusr{Оксана Ковалишин}
%%%fbauth_url
%%%fbauth_place
%%%fbauth_id
%%%fbauth_front
%%%fbauth_desc
%%%fbauth_www
%%%fbauth_pic
%%%fbauth_pic portrait
%%%fbauth_pic background
%%%fbauth_pic other
%%%fbauth_tags
%%%fbauth_pubs
%%%endfbauth
 

Скільки українців дивляться, слухають, читають, лайкають, поширюють
руцкоязичноє, "поліглоти" довбані! А лиш хто в такому "вишуканому товаристві"
хоча б заїкнеться про мову, накинуться, мов кури на кров. Недавно прочитала у
Бакмена :"Культура - це рівною мірою те, що ми заохочуємо, і те, що ми
ДОЗВОЛЯЄМО"

%%%fbauth
%%%fbauth_name
\iusr{Ліда Завойко}
%%%fbauth_url
%%%fbauth_place
%%%fbauth_id
%%%fbauth_front
%%%fbauth_desc
%%%fbauth_www
%%%fbauth_pic
%%%fbauth_pic portrait
%%%fbauth_pic background
%%%fbauth_pic other
%%%fbauth_tags
%%%fbauth_pubs
%%%endfbauth
 

Це правда, що у Львові стало більше чути російську мову, на жаль. Але найгірше
що привезли її теперішні переселенці. Без сумніву їх жаль, але сюди приїхали
багаті і вони думають що можуть наводити свої порядки ( повискакували цілі
під'їзди та котеджі і користуються всіма пільгами біженців. Але я вірю, що наші
батьки пережили не одну нашесть і ми це переживемо.

\begin{itemize}
%%%fbauth
%%%fbauth_name
\iusr{Юлія Клименко}
%%%fbauth_url
%%%fbauth_place
%%%fbauth_id
%%%fbauth_front
%%%fbauth_desc
%%%fbauth_www
%%%fbauth_pic
%%%fbauth_pic portrait
%%%fbauth_pic background
%%%fbauth_pic other
%%%fbauth_tags
%%%fbauth_pubs
%%%endfbauth
 
\textbf{Ліда Завойко} так скрізь.... і Київ, і Бровари, та й в Одесі
додалося.... тепер майже всі власники нового житла там - то
бєженці.... бідолашні.... Скрізь вони вважають себе хазяєвамі жизні ....а ми
-раби...

%%%fbauth
%%%fbauth_name
\iusr{Оксана Дідик}
%%%fbauth_url
%%%fbauth_place
%%%fbauth_id
%%%fbauth_front
%%%fbauth_desc
%%%fbauth_www
%%%fbauth_pic
%%%fbauth_pic portrait
%%%fbauth_pic background
%%%fbauth_pic other
%%%fbauth_tags
%%%fbauth_pubs
%%%endfbauth
 
\textbf{Юлія Клименко} не завжди.

До нас ставляться рівно так, як ми їм це дозволимо.

Якщо підкреслено відповідатимете українською і через раз, ба більше,
ігноруватимете їх, то так будуть старатися вам догодити, бо я це відчула на
собі.

Гарно обслуговувала українок, рускоговорящім через раз відповідала і то нехотя.
То так дякували і "до побачення, а не дасвіданія..."

Треба всім виглядом демонструвати своє незадоволення. Хай стараються переходити
на МОВУ.


%%%fbauth
%%%fbauth_name
\iusr{Юлія Клименко}
%%%fbauth_url
%%%fbauth_place
%%%fbauth_id
%%%fbauth_front
%%%fbauth_desc
%%%fbauth_www
%%%fbauth_pic
%%%fbauth_pic portrait
%%%fbauth_pic background
%%%fbauth_pic other
%%%fbauth_tags
%%%fbauth_pubs
%%%endfbauth
 
\textbf{Оксана Дідик} справа не тільки в мові, тут питань немає.....

%%%fbauth
%%%fbauth_name
\iusr{Оксана Дідик}
%%%fbauth_url
%%%fbauth_place
%%%fbauth_id
%%%fbauth_front
%%%fbauth_desc
%%%fbauth_www
%%%fbauth_pic
%%%fbauth_pic portrait
%%%fbauth_pic background
%%%fbauth_pic other
%%%fbauth_tags
%%%fbauth_pubs
%%%endfbauth
 
\textbf{Юлія Клименко} треба мати стержень, проявляти характер. Біля мене не буде почувати себе "хазяєвами" в Україні.
Іноді мовчання і є відповідь.
\end{itemize}

%%%fbauth
%%%fbauth_name
\iusr{Tamara Kharkivskykh}
%%%fbauth_url
%%%fbauth_place
%%%fbauth_id
%%%fbauth_front
%%%fbauth_desc
%%%fbauth_www
%%%fbauth_pic
%%%fbauth_pic portrait
%%%fbauth_pic background
%%%fbauth_pic other
%%%fbauth_tags
%%%fbauth_pubs
%%%endfbauth
 

Те саме відчуваю на Дніпропетровщині. Зросійщене містечко з красномовною назвою
- Новомосковськ. Вже сама назва ріже слух. А моя українська - постійний предмет
дискусій - від стриманої терпимості - до відвертої зневаги.

Але я все одно її несу в ці зомбовані маси. Де пояснюю, де пропускаю повз вуха
і увагу агресивні випади, але не переходжу на мову окупанта.

І маленького онука навчаю, бо скрізь, починаючи від садочка і закінчуючи масою
занять, виключно російська мова.

Дуже сумно.

\begin{itemize}
%%%fbauth
%%%fbauth_name
\iusr{Юлія Клименко}
%%%fbauth_url
%%%fbauth_place
%%%fbauth_id
%%%fbauth_front
%%%fbauth_desc
%%%fbauth_www
%%%fbauth_pic
%%%fbauth_pic portrait
%%%fbauth_pic background
%%%fbauth_pic other
%%%fbauth_tags
%%%fbauth_pubs
%%%endfbauth
 
\textbf{Tamara Kharkivskykh} спасибі Вам!!.

%%%fbauth
%%%fbauth_name
\iusr{Оксана Дідик}
%%%fbauth_url
%%%fbauth_place
%%%fbauth_id
%%%fbauth_front
%%%fbauth_desc
%%%fbauth_www
%%%fbauth_pic
%%%fbauth_pic portrait
%%%fbauth_pic background
%%%fbauth_pic other
%%%fbauth_tags
%%%fbauth_pubs
%%%endfbauth
 
\textbf{Tamara Kharkivskykh} тому, що влада така в Україні. Привести до влади патріотів-націоналістів. Міністром культури обрати Ірину Фаріон - защебечуть, як соловейки. І назву міста змінять.

%%%fbauth
%%%fbauth_name
\iusr{Tamara Kharkivskykh}
%%%fbauth_url
%%%fbauth_place
%%%fbauth_id
%%%fbauth_front
%%%fbauth_desc
%%%fbauth_www
%%%fbauth_pic
%%%fbauth_pic portrait
%%%fbauth_pic background
%%%fbauth_pic other
%%%fbauth_tags
%%%fbauth_pubs
%%%endfbauth
 
\textbf{Юлія Клименко} Гуртуймося!)🇺🇦

\end{itemize}

%%%fbauth
%%%fbauth_name
\iusr{Леся Пристанська}
%%%fbauth_url
%%%fbauth_place
%%%fbauth_id
%%%fbauth_front
%%%fbauth_desc
%%%fbauth_www
%%%fbauth_pic
%%%fbauth_pic portrait
%%%fbauth_pic background
%%%fbauth_pic other
%%%fbauth_tags
%%%fbauth_pubs
%%%endfbauth
 

Якось в Болгарії одна " дочь палковніка" сказала до жіночки- працівниці кафе:"
Учітє рускій! Как ви собіраєтєсь работать?" Нащо болгарка відповіла:" Єто ви к
нам прієхалі! Учітє болгарскій!"

\begin{itemize}
%%%fbauth
%%%fbauth_name
\iusr{Оксана Ковалишин}
%%%fbauth_url
%%%fbauth_place
%%%fbauth_id
%%%fbauth_front
%%%fbauth_desc
%%%fbauth_www
%%%fbauth_pic
%%%fbauth_pic portrait
%%%fbauth_pic background
%%%fbauth_pic other
%%%fbauth_tags
%%%fbauth_pubs
%%%endfbauth
 
\textbf{Леся Пристанська} І все ж більшість болгар дуже вже промосковські.

%%%fbauth
%%%fbauth_name
\iusr{Оксана Дідик}
%%%fbauth_url
%%%fbauth_place
%%%fbauth_id
%%%fbauth_front
%%%fbauth_desc
%%%fbauth_www
%%%fbauth_pic
%%%fbauth_pic portrait
%%%fbauth_pic background
%%%fbauth_pic other
%%%fbauth_tags
%%%fbauth_pubs
%%%endfbauth
 
\textbf{Оксана Ковалишин}, бо наслала компартія "струдніков", в свій час, для поліработи. Свою справу зробили. Тепер викорінювати треба.

%%%fbauth
%%%fbauth_name
\iusr{Леся Пристанська}
%%%fbauth_url
%%%fbauth_place
%%%fbauth_id
%%%fbauth_front
%%%fbauth_desc
%%%fbauth_www
%%%fbauth_pic
%%%fbauth_pic portrait
%%%fbauth_pic background
%%%fbauth_pic other
%%%fbauth_tags
%%%fbauth_pubs
%%%endfbauth
 
\textbf{Оксана Ковалишин} ні! Російську знає старше покоління, бо в школі вчили. Молоде покоління спілкується англійською. Принаймі мені такі зустрічались)
\end{itemize}

%%%fbauth
%%%fbauth_name
\iusr{Ірина Гриців}
%%%fbauth_url
%%%fbauth_place
%%%fbauth_id
%%%fbauth_front
%%%fbauth_desc
%%%fbauth_www
%%%fbauth_pic
%%%fbauth_pic portrait
%%%fbauth_pic background
%%%fbauth_pic other
%%%fbauth_tags
%%%fbauth_pubs
%%%endfbauth
 

Як тільки заселялИ Львів я собі подумала, до ці люди не дозволять собі говорити
рос. навчаться української і внас буде багато україномовних, але я глибоко
помилилася Так, дуже ріже слух рос і найбільше на Сихові


%%%fbauth
%%%fbauth_name
\iusr{Люба Безручко}
%%%fbauth_url
%%%fbauth_place
%%%fbauth_id
%%%fbauth_front
%%%fbauth_desc
%%%fbauth_www
%%%fbauth_pic
%%%fbauth_pic portrait
%%%fbauth_pic background
%%%fbauth_pic other
%%%fbauth_tags
%%%fbauth_pubs
%%%endfbauth
 

Згадалося - приїхали ми з сім'єю у Львів десь в 80 роках ,зустрілися з давніми
друзями, біля будинку вони нас попереджають - говоріть Українською, або ж
мовчить, бо

ЗАРАЗ ПОВЕРНУЛИСЯ З ЗАСЛАННЯ БАНДЕРІВЦІ(25років !) то вони на дух не переносять
, хто говорить російською...

Не нариваємося на небезпеку....

А в минулому році я проїздом була у Львові, визвала таксі, то таксист зі мною
разгаварівал, звертався паруські, хоча я демонстративно говорила українською..

ЩО ЗІ ЛЬВОВОМ НЕ ТАК?????

\begin{itemize}
%%%fbauth
%%%fbauth_name
\iusr{Лариса Бондаренко}
%%%fbauth_url
%%%fbauth_place
%%%fbauth_id
%%%fbauth_front
%%%fbauth_desc
%%%fbauth_www
%%%fbauth_pic
%%%fbauth_pic portrait
%%%fbauth_pic background
%%%fbauth_pic other
%%%fbauth_tags
%%%fbauth_pubs
%%%endfbauth
 
\textbf{Люба Безручко} Вати там з Донбасу понабігало забагато. Та ще й грошовитої, яка взагалі нікого крім себе за людей не вважає.

%%%fbauth
%%%fbauth_name
\iusr{Любовь Войтович}
%%%fbauth_url
%%%fbauth_place
%%%fbauth_id
%%%fbauth_front
%%%fbauth_desc
%%%fbauth_www
%%%fbauth_pic
%%%fbauth_pic portrait
%%%fbauth_pic background
%%%fbauth_pic other
%%%fbauth_tags
%%%fbauth_pubs
%%%endfbauth
 
\textbf{Люба Безручко} ДУЖЕ ДИВУЄ РОСІЙСЬКА У ЛЬВОВІ. ДУЖЕ.

%%%fbauth
%%%fbauth_name
\iusr{Любовь Войтович}
%%%fbauth_url
%%%fbauth_place
%%%fbauth_id
%%%fbauth_front
%%%fbauth_desc
%%%fbauth_www
%%%fbauth_pic
%%%fbauth_pic portrait
%%%fbauth_pic background
%%%fbauth_pic other
%%%fbauth_tags
%%%fbauth_pubs
%%%endfbauth
 
\textbf{Лариса Бондаренко} ЧОМУ ДАЛИ ЇМ ТАКУ ВОЛЮ? НЕ РОЗУМІЄТЕ РОСІЙСЬКОЇ-І
ВСЕ. В КИЄВІ ТАКОЖ ПОВНО ЦИХ АФЕРИСТІВ, ВОРЮГ З ДОНБАСУ. ДУЖЕ БУЛО НЕПРИЄМНО
БАЧИТИ В ПАСАЖІ НА ХРЕЩАТИКУ ШАНТРАПУ ЗІ СХОДУ.
\end{itemize}

%%%fbauth
%%%fbauth_name
\iusr{Анатолій Шолудько}
%%%fbauth_url
%%%fbauth_place
%%%fbauth_id
%%%fbauth_front
%%%fbauth_desc
%%%fbauth_www
%%%fbauth_pic
%%%fbauth_pic portrait
%%%fbauth_pic background
%%%fbauth_pic other
%%%fbauth_tags
%%%fbauth_pubs
%%%endfbauth
 

У 82-му із переважно російськомовних Черкас завітав у Львів. Мене попереджали,
що про всяк випадок краще говорити мовою. Але мені тоді кинулось у вічі, що
відповіді частіше були московською...на автоматі... так, як і у Черкасах...


%%%fbauth
%%%fbauth_name
\iusr{Віра Ватуляк}
%%%fbauth_url
%%%fbauth_place
%%%fbauth_id
%%%fbauth_front
%%%fbauth_desc
%%%fbauth_www
%%%fbauth_pic
%%%fbauth_pic portrait
%%%fbauth_pic background
%%%fbauth_pic other
%%%fbauth_tags
%%%fbauth_pubs
%%%endfbauth
 
Сумно і боляче, за Львів і львів'ян

%%%fbauth
%%%fbauth_name
\iusr{Павел Мельник}
%%%fbauth_url
%%%fbauth_place
%%%fbauth_id
%%%fbauth_front
%%%fbauth_desc
%%%fbauth_www
%%%fbauth_pic
%%%fbauth_pic portrait
%%%fbauth_pic background
%%%fbauth_pic other
%%%fbauth_tags
%%%fbauth_pubs
%%%endfbauth
 

Мова для нас це як східні терени нашоі держави Украіна,яку захищають і
оберігають як зіницю ока від чухоньсь коі орди наші хлопці,а наша справа
зберегти нашу ідентичність і мову, бо то є ,святе...

%%%fbauth
%%%fbauth_name
\iusr{Olya Berezhnytska}
%%%fbauth_url
%%%fbauth_place
%%%fbauth_id
%%%fbauth_front
%%%fbauth_desc
%%%fbauth_www
%%%fbauth_pic
%%%fbauth_pic portrait
%%%fbauth_pic background
%%%fbauth_pic other
%%%fbauth_tags
%%%fbauth_pubs
%%%endfbauth
 

Та йой, у Львові вже давно як в Києві. Знайти львів'ян дуже складно. Серед моїх
знайомих/друзів львів'ян є дуже мало. Більшість все ж таки не толерує ізик. І
ще, дома і між собою на вулиці розмовляють рускою, але в компанії, громадських
місцях українською.


%%%fbauth
%%%fbauth_name
\iusr{Андрій Міщенко}
%%%fbauth_url
%%%fbauth_place
%%%fbauth_id
%%%fbauth_front
%%%fbauth_desc
%%%fbauth_www
%%%fbauth_pic
%%%fbauth_pic portrait
%%%fbauth_pic background
%%%fbauth_pic other
%%%fbauth_tags
%%%fbauth_pubs
%%%endfbauth
 
Львів давно вже ж столицею Хрунів

%%%fbauth
%%%fbauth_name
\iusr{Андрій Андрусяк}
%%%fbauth_url
%%%fbauth_place
%%%fbauth_id
%%%fbauth_front
%%%fbauth_desc
%%%fbauth_www
%%%fbauth_pic
%%%fbauth_pic portrait
%%%fbauth_pic background
%%%fbauth_pic other
%%%fbauth_tags
%%%fbauth_pubs
%%%endfbauth
 
Шувар – це шаурма по-модньому?

%%%fbauth
%%%fbauth_name
\iusr{Svitlana Lystopad}
%%%fbauth_url
%%%fbauth_place
%%%fbauth_id
%%%fbauth_front
%%%fbauth_desc
%%%fbauth_www
%%%fbauth_pic
%%%fbauth_pic portrait
%%%fbauth_pic background
%%%fbauth_pic other
%%%fbauth_tags
%%%fbauth_pubs
%%%endfbauth
 

Була у Львові у 2014, 2018, 2021... Дійсно,.. і це помітно: кількість
російськомовних збільшилася суттєво. Жаль... Тут ще й мої зЄмлякі зі Сходу
"попрацювали": у них і групи у ФБ російською називаються - " Северодончане
Львова" (наприклад), і аеробіка та у-шу у Стрийському парку пАрускі.

\begin{itemize}
%%%fbauth
%%%fbauth_name
\iusr{Serhii Bryhar}
%%%fbauth_url
%%%fbauth_place
%%%fbauth_id
%%%fbauth_front
%%%fbauth_desc
%%%fbauth_www
%%%fbauth_pic
%%%fbauth_pic portrait
%%%fbauth_pic background
%%%fbauth_pic other
%%%fbauth_tags
%%%fbauth_pubs
%%%endfbauth
 

От про ушу і аеробіку в Стрийському парку не знав. Це взагалі страшно. Це
публічний простір... Ну не знаю. Як матиму час і натхнення, спробую когось
підняти і щось вирішити). Я ж трохи вмію..

\end{itemize}

%%%fbauth
%%%fbauth_name
\iusr{Volodymyr Tarnavskyi}
%%%fbauth_url
%%%fbauth_place
%%%fbauth_id
%%%fbauth_front
%%%fbauth_desc
%%%fbauth_www
%%%fbauth_pic
%%%fbauth_pic portrait
%%%fbauth_pic background
%%%fbauth_pic other
%%%fbauth_tags
%%%fbauth_pubs
%%%endfbauth
 
Це ніякі не переселенці, а місцеві

%%%fbauth
%%%fbauth_name
\iusr{Валентина Хмельовська}
%%%fbauth_url
%%%fbauth_place
%%%fbauth_id
%%%fbauth_front
%%%fbauth_desc
%%%fbauth_www
%%%fbauth_pic
%%%fbauth_pic portrait
%%%fbauth_pic background
%%%fbauth_pic other
%%%fbauth_tags
%%%fbauth_pubs
%%%endfbauth
 
Підтримую автора. Російська злоякісна пухлина розростається. Отже, потрібне оперативне втручання\Laughey[1.0][white].

%%%fbauth
%%%fbauth_name
\iusr{Andro Gorbacio}
%%%fbauth_url
%%%fbauth_place
%%%fbauth_id
%%%fbauth_front
%%%fbauth_desc
%%%fbauth_www
%%%fbauth_pic
%%%fbauth_pic portrait
%%%fbauth_pic background
%%%fbauth_pic other
%%%fbauth_tags
%%%fbauth_pubs
%%%endfbauth
 
Толеранція у питанні мови призводить до окупації і винищення народу швидше від танків.

\begin{itemize}
%%%fbauth
%%%fbauth_name
\iusr{Volodymyr Tarnavskyi}
%%%fbauth_url
%%%fbauth_place
%%%fbauth_id
%%%fbauth_front
%%%fbauth_desc
%%%fbauth_www
%%%fbauth_pic
%%%fbauth_pic portrait
%%%fbauth_pic background
%%%fbauth_pic other
%%%fbauth_tags
%%%fbauth_pubs
%%%endfbauth
 

Ну ви ж маєте прекрасно розуміти, що проблема не в місцевих російськомовних.
Проблема в самій державі Україна, яка є російськомовною по суті. Коли нам всі -
від Зеленського до Порошенка розповідають, що російська це добре, нею воюють на
сході тощо, то який ви результат хочете? Без України загалом не було б такою
проблеми, ба навіть, толерувати не було б що...

\end{itemize}

%%%fbauth
%%%fbauth_name
\iusr{Максим Меркулов}
%%%fbauth_url
%%%fbauth_place
%%%fbauth_id
%%%fbauth_front
%%%fbauth_desc
%%%fbauth_www
%%%fbauth_pic
%%%fbauth_pic portrait
%%%fbauth_pic background
%%%fbauth_pic other
%%%fbauth_tags
%%%fbauth_pubs
%%%endfbauth
 
У Львові багато російської мови. Зауважив це ще в нульові.

%%%fbauth
%%%fbauth_name
\iusr{Myroslava Vesna}
%%%fbauth_url
%%%fbauth_place
%%%fbauth_id
%%%fbauth_front
%%%fbauth_desc
%%%fbauth_www
%%%fbauth_pic
%%%fbauth_pic portrait
%%%fbauth_pic background
%%%fbauth_pic other
%%%fbauth_tags
%%%fbauth_pubs
%%%endfbauth
 
Їх в Сихівському районі дуже... дуже...

%%%fbauth
%%%fbauth_name
\iusr{Олександр Слюта}
%%%fbauth_url
%%%fbauth_place
%%%fbauth_id
%%%fbauth_front
%%%fbauth_desc
%%%fbauth_www
%%%fbauth_pic
%%%fbauth_pic portrait
%%%fbauth_pic background
%%%fbauth_pic other
%%%fbauth_tags
%%%fbauth_pubs
%%%endfbauth
 

"...українські посадовці цідять через зуби із себе українське слово як матюк –
раз в рік, щоб миттєво перейти на общєпанятний.

Я не можу забути відео, як російськіі окупанти збиткувалися над смертельно
пораненим українським бійцем, який просив водички. «Разгаварівай па
чєлавєчєскі!»- кричали вони і гиготіли.

Чим ви тоді відрізняєтесь від них?"

\begin{itemize}
%%%fbauth
%%%fbauth_name
\iusr{Оксана Ковалишин}
%%%fbauth_url
%%%fbauth_place
%%%fbauth_id
%%%fbauth_front
%%%fbauth_desc
%%%fbauth_www
%%%fbauth_pic
%%%fbauth_pic portrait
%%%fbauth_pic background
%%%fbauth_pic other
%%%fbauth_tags
%%%fbauth_pubs
%%%endfbauth
 
\textbf{Олександр Слюта} Це питання до кожного! Чим ви відрізняєтесь, на чиїй ви стороні, "харошиє люді"?
\end{itemize}

%%%fbauth
%%%fbauth_name
\iusr{Jurko Zełenyj}
%%%fbauth_url
%%%fbauth_place
%%%fbauth_id
%%%fbauth_front
%%%fbauth_desc
%%%fbauth_www
%%%fbauth_pic
%%%fbauth_pic portrait
%%%fbauth_pic background
%%%fbauth_pic other
%%%fbauth_tags
%%%fbauth_pubs
%%%endfbauth
 

ja Vam, pane \textbf{Serhii Bryhar}, sze skiko rokiw tomu nazad kazaw, że bude torba
tomu wśiomu, jikszo Galìczėnu ne vysmyknuty z malorusskogo bolota?

Oś torba i je u wśij svoji «krasi»....

\begin{itemize}
%%%fbauth
%%%fbauth_name
\iusr{Андрій Міщенко}
%%%fbauth_url
%%%fbauth_place
%%%fbauth_id
%%%fbauth_front
%%%fbauth_desc
%%%fbauth_www
%%%fbauth_pic
%%%fbauth_pic portrait
%%%fbauth_pic background
%%%fbauth_pic other
%%%fbauth_tags
%%%fbauth_pubs
%%%endfbauth
 
\textbf{Jurko Zełenyj} теперішні галичани зразу в гівні втопляться, якщо відійдуть від України.

%%%fbauth
%%%fbauth_name
\iusr{Volodymyr Tarnavskyi}
%%%fbauth_url
%%%fbauth_place
%%%fbauth_id
%%%fbauth_front
%%%fbauth_desc
%%%fbauth_www
%%%fbauth_pic
%%%fbauth_pic portrait
%%%fbauth_pic background
%%%fbauth_pic other
%%%fbauth_tags
%%%fbauth_pubs
%%%endfbauth
 
\textbf{Андрій Міщенко}, пане Міщенко, а можна без оцих глибокодумних сентенцій
на тему гівна? В чому, скажіть, не правий Юрко чи інші думаючі люди, які цю
ситуацію прогнозували вже давно? Свобода хоч якось займається прогнозуванням,
аналізом, моделюваннях мовних процесів, зокрема на Галичині? Чи у вас на
мовному напрямку одна пані Фаріон?)))
\end{itemize}

%%%fbauth
%%%fbauth_name
\iusr{Сергій Василюк}
%%%fbauth_url
%%%fbauth_place
%%%fbauth_id
%%%fbauth_front
%%%fbauth_desc
%%%fbauth_www
%%%fbauth_pic
%%%fbauth_pic portrait
%%%fbauth_pic background
%%%fbauth_pic other
%%%fbauth_tags
%%%fbauth_pubs
%%%endfbauth
 
Я ж казав: якщо переїжджати з родиною, то до столиці. Вона українізується)

\begin{itemize}
%%%fbauth
%%%fbauth_name
\iusr{Volodymyr Tarnavskyi}
%%%fbauth_url
%%%fbauth_place
%%%fbauth_id
%%%fbauth_front
%%%fbauth_desc
%%%fbauth_www
%%%fbauth_pic
%%%fbauth_pic portrait
%%%fbauth_pic background
%%%fbauth_pic other
%%%fbauth_tags
%%%fbauth_pubs
%%%endfbauth
 
Ага) аж біжить, бідна, і спотикається)))

%%%fbauth
%%%fbauth_name
\iusr{Dajan Jarosław Monastyrski}
%%%fbauth_url
%%%fbauth_place
%%%fbauth_id
%%%fbauth_front
%%%fbauth_desc
%%%fbauth_www
%%%fbauth_pic
%%%fbauth_pic portrait
%%%fbauth_pic background
%%%fbauth_pic other
%%%fbauth_tags
%%%fbauth_pubs
%%%endfbauth
 
\textbf{Сергій Василюк} столиці урср?

%%%fbauth
%%%fbauth_name
\iusr{Сергій Василюк}
%%%fbauth_url
%%%fbauth_place
%%%fbauth_id
%%%fbauth_front
%%%fbauth_desc
%%%fbauth_www
%%%fbauth_pic
%%%fbauth_pic portrait
%%%fbauth_pic background
%%%fbauth_pic other
%%%fbauth_tags
%%%fbauth_pubs
%%%endfbauth
 
\textbf{Dajan Jarosław Monastyrski} ого, ополячені українці ставлять під сумнів статус Києва? Щось новеньке

%%%fbauth
%%%fbauth_name
\iusr{Dajan Jarosław Monastyrski}
%%%fbauth_url
%%%fbauth_place
%%%fbauth_id
%%%fbauth_front
%%%fbauth_desc
%%%fbauth_www
%%%fbauth_pic
%%%fbauth_pic portrait
%%%fbauth_pic background
%%%fbauth_pic other
%%%fbauth_tags
%%%fbauth_pubs
%%%endfbauth
 
\textbf{Сергій Василюк} ади, малорос назива мене укром, та ше й споляченим) певне, нацик і украінєц народжений в совку

%%%fbauth
%%%fbauth_name
\iusr{Serhii Bryhar}
%%%fbauth_url
%%%fbauth_place
%%%fbauth_id
%%%fbauth_front
%%%fbauth_desc
%%%fbauth_www
%%%fbauth_pic
%%%fbauth_pic portrait
%%%fbauth_pic background
%%%fbauth_pic other
%%%fbauth_tags
%%%fbauth_pubs
%%%endfbauth
 
\textbf{Сергій Василюк} 

Не стави би з тобою тут сперечатися. Думаю, все так і є. Просто є ще кілька
важливих аспектів:

\begin{itemize}
\item 1. Одеса і Львів - "одиниці" майже однакові. В логістичному, та й у суто
відчуттєвому сенсі, мені у Львові комфортно.

\item 2. Львів - це трохи інша ментальність. Близькість до кордону. Більше
дисципліни. Трохи. Так, це та сама Україна, але трохи дисциплінованіша, трохи
кільтурніша. Принаймні, поки що.

\item 3. А оце найголовніше: люблю Київ... Думаю, я би там звик і прижився. Але він
здоровенний... В ньому багато логістичних проблем. І от тут на перший план
виходять ціни. Дозволити собі прийнятну квартиру в зручному районі з гарною
транспортною розв'язкою, я думаю, завдання фактично неможливе. У Львові ціни
теж високі, але порівняти їх з київськими навіть і не вийде. Отже, я навіть не
думав про столицю після предметного аналізу цін. Віддати стільки я наразі не
можу).
\end{itemize}

%%%fbauth
%%%fbauth_name
\iusr{Volodymyr Tarnavskyi}
%%%fbauth_url
%%%fbauth_place
%%%fbauth_id
%%%fbauth_front
%%%fbauth_desc
%%%fbauth_www
%%%fbauth_pic
%%%fbauth_pic portrait
%%%fbauth_pic background
%%%fbauth_pic other
%%%fbauth_tags
%%%fbauth_pubs
%%%endfbauth
 

Я ставлю під сумнів мовний статус Києва. Те, що ви називаєте українізацією, я б
назвав розкаянням наркомана чи алкоголіка - якась мить, сльози жалю та каяття,
а потім знов повернення до тваринного стану. Українізація Києва настільки
квола, повільна та фрагментарна, що годі говорити про якусь системність чи
прогнозованість, а отже дієвість та ефективність. Так, оази в пустелі, де добре
та затишно.


%%%fbauth
%%%fbauth_name
\iusr{Оксана Ковалишин}
%%%fbauth_url
%%%fbauth_place
%%%fbauth_id
%%%fbauth_front
%%%fbauth_desc
%%%fbauth_www
%%%fbauth_pic
%%%fbauth_pic portrait
%%%fbauth_pic background
%%%fbauth_pic other
%%%fbauth_tags
%%%fbauth_pubs
%%%endfbauth
 
\textbf{Сергій Василюк} Хіба що в порівнянні з АдєССай.

\end{itemize}

%%%fbauth
%%%fbauth_name
\iusr{Тарас Гаврилюк}
%%%fbauth_url
%%%fbauth_place
%%%fbauth_id
%%%fbauth_front
%%%fbauth_desc
%%%fbauth_www
%%%fbauth_pic
%%%fbauth_pic portrait
%%%fbauth_pic background
%%%fbauth_pic other
%%%fbauth_tags
%%%fbauth_pubs
%%%endfbauth
 
Це явище, на жаль не тільки у Львові. Загляньте у телефони українців. У багатьох налаштування рос. мовою.

%%%fbauth
%%%fbauth_name
\iusr{Степан Сторонський}
%%%fbauth_url
%%%fbauth_place
%%%fbauth_id
%%%fbauth_front
%%%fbauth_desc
%%%fbauth_www
%%%fbauth_pic
%%%fbauth_pic portrait
%%%fbauth_pic background
%%%fbauth_pic other
%%%fbauth_tags
%%%fbauth_pubs
%%%endfbauth
 
І на Левандівці я їх багато зустрічав. Всі місцеві. 😠

%%%fbauth
%%%fbauth_name
\iusr{Віктор Католик}
%%%fbauth_url
%%%fbauth_place
%%%fbauth_id
%%%fbauth_front
%%%fbauth_desc
%%%fbauth_www
%%%fbauth_pic
%%%fbauth_pic portrait
%%%fbauth_pic background
%%%fbauth_pic other
%%%fbauth_tags
%%%fbauth_pubs
%%%endfbauth
 

\begin{itemize}
\item 1. У Львів заселяли доволі багато ще в радянські часи з метою роботи з
неблагонадійним населенням. Було багато російськомовних шкіл, навіть зараз іще
є.

\item 2. Зараз понаїхало багато, бо Львів пропонує багато можливостей порівняно з
іншими містами. Купують нерухомість, вкорінюються. Про кількість переселенців
свідчать темпи забудови міста.

\item 3. Дуже багато львів'ян виїхали з міста, хто в Київ, хто за кордон. Святе місце
пустим не буває.
\end{itemize}


%%%fbauth
%%%fbauth_name
\iusr{Oleksandr Kirkov}
%%%fbauth_url
%%%fbauth_place
%%%fbauth_id
%%%fbauth_front
%%%fbauth_desc
%%%fbauth_www
%%%fbauth_pic
%%%fbauth_pic portrait
%%%fbauth_pic background
%%%fbauth_pic other
%%%fbauth_tags
%%%fbauth_pubs
%%%endfbauth
 

там де меншість - продовжувати відстоювати мову і добиватися всіми можливими і
обережними шляхами збільшувати українську в просторі. наперекір москворотим
москвоголовим.
\end{itemize}

