% vim: keymap=russian-jcukenwin
%%beginhead 
 
%%file 05_01_2022.fb.kuzmin_renat.1.ukraina_fashizm
%%parent 05_01_2022
 
%%url https://www.facebook.com/RRKuzmin/posts/4653041851438737
 
%%author_id kuzmin_renat
%%date 
 
%%tags fashizm,nacionalizm,ukraina
%%title Украина. 14 признаков фашизма по Лоуренсу Бритту
 
%%endhead 
 
\subsection{Украина. 14 признаков фашизма по Лоуренсу Бритту}
\label{sec:05_01_2022.fb.kuzmin_renat.1.ukraina_fashizm}
 
\Purl{https://www.facebook.com/RRKuzmin/posts/4653041851438737}
\ifcmt
 author_begin
   author_id kuzmin_renat
 author_end
\fi

Украина. 14 признаков фашизма по Лоуренсу Бритту: 

1. Мощный и продолжительный национализм — фашистские режимы постоянно
используют националистические лозунги, девизы, символы, песни и так далее. 

— Тут без комментариев. Все все видят и слышат. А «Слава Україні и Героям
слава»- стали официальными приветствиями в украинской армии. 

\ii{05_01_2022.fb.kuzmin_renat.1.ukraina_fashizm.pic.1}

2. Пренебрежение к общепризнанным правам человека — из страха перед врагом и
под предлогом обеспечения безопасности фашистские власти убеждают, что права
человека могут игнорироваться в определенных случаях ради некой общественной
«необходимости». 

— Применение санкций к неугодным гражданам, лишение их собственности и
блокировка счетов по решению СНБО — только верхняя часть айсберга. 

3. Выявление врага / искупительные жертвы как объединительная основа — народы
при фашистских режимах сплачиваются в патриотичном движении в борьбе против
общей опасности или общего противника. 

— Путин, российская агрессия, пятая колонна Кремля. Бузина, Шеремет,
Калашников. В общем, все как по лекалам. 

4. Преимущественное положение вооруженных сил — даже если в стране есть много
острых внутренних проблем. Пропаганда навязывает привлекательный образ военных
и военной службы.

— В общем, «армія, мова, віра», где на первом месте, конечно, армия. 

5. Сильная дискриминация по признаку половой принадлежности — в фашистских
правительствах доминируют мужчины. 

— «Корабельные сосны» — это как раз оттуда. 

6. Контроль над СМИ — они контролируются непосредственно властью или косвенно
через сочувствующих журналистов либо руководителей СМИ. Распространена цензура.

— Закрытие оппозиционных телеканалов, электронных СМИ, использование
проплаченных блогеров и журналистов для шельмования оппозиции в Украине давно
уже никого не удивляют. 

7. Маниакальное увлечение национальной безопасностью — страх используется как
мотивационный инструмент власти для давления на массы.

— Показательные расправы над инакомыслием на заседаниях СНБО, использование
правоохранительных органов для запугивания оппозиции — в Украине стали
элементами государственной внутренней политики. 

8. Переплетение религии и власти — правительства фашистских стран используют
религию как инструмент управления общественным мнением.

— Та же «армія, мова, віра», только теперь уже «віра» впереди паровоза. 

9. Защита корпораций — промышленная и деловая аристократия в фашистских
государствах часто является единственной силой, ставящей лидеров во власть,
создавая взаимовыгодные деловые отношения с властной элитой.

— Украинские олигархи являются частью государственной системы управления. Плати
и живи — основной принцип выживания украинского бизнеса. 

10. Притеснение профсоюзов и рабочего движения. 

— Кто-то может вспомнить хоть одного украинского профсоюзного лидера? Их всех
давно уже загнали пол лавку. 

11. Презрение к интеллигенции и искусству — фашистские государства поощряют или
терпимо относятся к проявлениям открытой враждебности к высшему образованию и к
учёным. Свобода самовыражения в искусстве подвергается открытым нападкам.

— Массовые увольнения русскоязычных ученых и преподавателей давно стали нормой,
как и запрет русскоязычной научной литературы. А запрет на въезд в Украину
русскоязычным деятелям искусства, запрет книг, фильмов и упоминание фамилий —
вообще за гранью понимания нормального человека. 

12. Навязчивая идея преступления и наказания — при фашистских режимах
правоохранительным органам даются почти неограниченные полномочия. Люди во имя
патриотизма во многих случаях предпочитают не замечать их злоупотреблений, даже
нарушение своих гражданских свобод. Часто создаются репрессивные органы с
неограниченной властью.

— Охота на «государственных изменников и врагов украинского народа» давно уже
никого не удивляет, как и антиконституционное создание новых репрессивных
органов, типа НАБУ, ГБР, БЭБ, НАПК и. т. д.

13. Необузданное кумовство и коррупция — фашистскими режимами почти всегда
управляют кланы приятелей и партнеров, которые назначают друг друга на
правительственные должности и используют власть для защиты членов своего клана
от ответственности. Зачастую правительственные лидеры присваивают или даже
напрямую разворовывают государственные ресурсы и казну.

— Квартал 95 в Украине стал кузницей кадров для высших государственных постов,
а про «велике крадівництво» не говорит только ленивый. 

14. Мошеннические выборы — выборы в фашистских государствах часто превращаются
в фарс. Нередко проводится клеветническая кампания против кандидатов от
оппозиции, законодательство используется для манипулирования численностью
избирателей, границами округов, средствами массовой информации.

— Украинские власти лишили Донбасс права голоса на выборах, изменив границы
избирательных участков, введя военно-гражданские администрации и приняв
соответствующие законы.

Вывод: действующий в Украине режим имеет ВСЕ признаки фашистского. Осталось
толко решить что со всем этим делать...
