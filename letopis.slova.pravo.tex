% vim: keymap=russian-jcukenwin
%%beginhead 
 
%%file slova.pravo
%%parent slova
 
%%url 
 
%%author 
%%author_id 
%%author_url 
 
%%tags 
%%title 
 
%%endhead 
\chapter{Право}

%%%cit
%%%cit_head
%%%cit_pic
%%%cit_text
1529 року з'явився Перший (Старий) Литовський статут, який з'єднав усі норми
руського, литовського, а також почасти польського і німецького \emph{права}. Серед
найважливіших відмінностей від судебників минулого – статут гарантував
політичні та майнові \emph{права} шляхти за типом Великої хартії вольностей (тобто
самі \emph{права} існували з незапам'ятних часів, але періодично їх мали
підтверджувати чергові правителі). 1566 року з'явився Другий статут – повніший
і точніший
%%%cit_comment
%%%cit_title
\citTitle{Українській Конституції – 2300 років}, 
Сергій Громенко, gazeta.ua, 26.06.2021
%%%endcit
