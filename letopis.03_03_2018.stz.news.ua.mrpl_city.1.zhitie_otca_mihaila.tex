% vim: keymap=russian-jcukenwin
%%beginhead 
 
%%file 03_03_2018.stz.news.ua.mrpl_city.1.zhitie_otca_mihaila
%%parent 03_03_2018
 
%%url https://mrpl.city/blogs/view/zhitie-ottsa-mihaila
 
%%author_id burov_sergij.mariupol,news.ua.mrpl_city
%%date 
 
%%tags 
%%title Житие отца Михаила
 
%%endhead 
 
\subsection{Житие отца Михаила}
\label{sec:03_03_2018.stz.news.ua.mrpl_city.1.zhitie_otca_mihaila}
 
\Purl{https://mrpl.city/blogs/view/zhitie-ottsa-mihaila}
\ifcmt
 author_begin
   author_id burov_sergij.mariupol,news.ua.mrpl_city
 author_end
\fi

\ii{03_03_2018.stz.news.ua.mrpl_city.1.zhitie_otca_mihaila.pic.0.arnautov}

В 1995 году программа \enquote{Мариуполь. Былое} на телеканале \enquote{Сигма} только-только
набирала обороты. И с первых же выпусков ее верным другом и помощником стал
Игорь Андреевич Налчаджи – человек, беззаветно влюбленный в Мариуполь, в его
историю, сохранивший в своей памяти множество преданий из жизни прошлых
поколений мариупольцев. Его любовь к своей малой родине была активной: он был
среди тех, кто добивался возвращения исторических имен и городу, и его улицам,
он был среди немногочисленных энтузиастов, которые воссоздали Мариупольское
греческое общество. И нет ничего удивительного в том, что он время от времени
подбрасывал темы для краеведческих передач.

Однажды при встрече он спросил: \enquote{Не хотите познакомиться с Маргаритой
Михайловной Астаховой? Это племянница Виктора Михайловича Арнаутова и внучка
мариупольского священника отца Михаила}. Как можно было отказаться от такого
предложения? И вот 21 октября 1995 года автор этих строк и телеоператор Дмитрий
Федоров во дворе многоэтажного дома на Черемушках. А перед нами ведет неспешный
рассказ Маргарита Михайловна. Через неделю телефильм о трагической судьбе отца
Михаила вышел в эфир. К счастью, и копия телефильма, и полная стенограмма
воспоминаний М. М. Астаховой сохранились до сих пор. Эти материалы и были
использованы при написании того, что предлагается здесь читателю.

*  *  *

Маргарите Михайловне Астаховой в наследство от деда достались план
принадлежавшего ему некогда подворья на Митрополитской улице, квитанции об
уплате налогов за сте­ны дома, сожженного немцами в сорок третьем году, и еще
пачка семейных фотографий разных лет...

Каким был ее дед? Семилетней Рите он запомнился высоким, худощавым и очень
добрым. К дедушке и бабушке ее с сестренкой привозили родители из Бердянска,
где жила в довоенные годы их семья. Еще в памяти остался огромный фруктовый
сад. Впрочем, так ли он был велик? Ребенку всегда все кажется большим.

В рассказах Маргариты Михайловны причудливо смеша­лись детские воспоминания и
то, что довелось уже в более зрелом возрасте услышать от родителей: \enquote{В саду
были масса цветов и небольшой огород. Были и ульи. За всем этим ухаживал
дедушка сам, и бабушка ему помогала. Он приучал нас с сестренкой к труду. Если
шла обрезка деревьев, нам поручали сносить хворост, собирать листья. Мы любили
работать с дедушкой, потому что он знал много историй из жизни растений,
живот­ных, птиц и рассказывал их нам об этом. Вот любовь к труду и к природе
сохранилась у нас на всю жизнь. Я до сих пор люблю цветы, люблю растения. У
меня дома много комнатных растений.

Дедушка вел переписку с Иваном Владимировичем Мичуриным и по его совету сделал
прививку на абрикос. Получился новый сорт, названный им \enquote{Ивановкой}. Может
быть, в честь Мичурина?}.

Кем же был дедушка Маргариты Михайловны? Агрономом, садоводом-мичуринцем?

Не тем и не другим.

Ее дед Михаил Васильевич Арнаутов - православный священник.

Нет, не он выбирал себе профессию. То было решение отца - ялтинского
поселянина, как принято было в те далекие годы говорить. Когда братья
Арнаутовы подросли, им было сказано: \enquote{Ты, Иван, будешь учителем, а ты, Михаил,
- священником}. Сыновья отцовскую волю безропотно исполнили...

Двадцатилетний воспитанник Екатеринославской семинарии Михаил Арнаутов пишет
своему другу: \enquote{Счастлив тот, кто мог понять, что жить и существовать - две
разные вещи}. Для себя Михаил выбрал жизнь. После окончания семинарии он
женится на юной красавице-казачке Аделаиде Кравцовой, вскоре обретает сан
священника и начинает служить в одном из мариупольских храмов. Матушка дарит
ему сыновей: Виктора, Евгения, дочь Лидию, а уж напоследок появляется на свет
Божий Леонид.

\ii{03_03_2018.stz.news.ua.mrpl_city.1.zhitie_otca_mihaila.pic.1}

Он служил в Успенской церкви, что на Марьинске, а когда ее взорвали, перешел в
кладбищенскую \enquote{Всех Святых}. Когда оставался один, подолгу молился перед
домашним киотом, освещенным мерцающим светом лампады; просил Бога сохранить
его семейство от напастей, хвори и невзгод. Он старался выполнять заповеди
Божьи и жить по-христиански. В большом своем многокомнатном доме занимал с
матушкой каморку ближе к черному ходу, а в гостиной, спальнях сыновей
разместились беженцы из Финляндии, вдовая матушка с дочкой, обездоленные
приятельницы матушки. Все эти жильцы жили бесплатно, а порой также и
столовались.

Еще одно воспоминание Маргариты Михайловны: \enquote{В одной из комнат дедушкиного
дома жили три сестры Батиевские. Одна из них - Мария Николаевна - была
совершенно слепая, а дедушка заставлял нас каждое утро водить ее на прогулку и
дарить ей цветы. Он говорил: \enquote{Самые лучшие, самые красивые цветы срезайте для
Марии Николаевны}. Когда мы удивлялись: \enquote{Так она их все равно не увидит}, -
дедушка отвечал: \enquote{Зато она их почувствует, и это принесет ей здоровье}}.

Увы, не принесли цветы здоровья Марии Батиевской, она тихо ушла в мир иной. От
сыновей перестали приходить письма, один за другим были взорваны все
мариупольские храмы. В начале июля тридцать седьмого года скончалась матушка
Аделаида Ивановна. \enquote{Что ж, на то воля Божья}, - смиренно думал
шестидесятисемилетний пастырь. Он начал тихо готовиться к принятию монашеского
чина. Но не знал отец Михаил, что дни его сочтены, не знал, какая мученическая
кончина уготована ему.

19 сентября тридцать седьмого года раздался стук в дверь. Начался обыск,
составили протокол, отца Михаила увели, и он исчез навсегда...

\ii{03_03_2018.stz.news.ua.mrpl_city.1.zhitie_otca_mihaila.pic.2.viktor_arnautov}

Прошло четверть века. В Мариуполе появился старший сын отца Михаила Виктор. Он
приехал в родной город из Сан-Франциско почти через пятьдесят лет отсутствия,
вернулся известным художником, профессором одного из университетов США.
Местные художники окружили его уважением и заботой, городские власти через
короткое время выделили квартиру, обеспечили работой (мозаичные панно на Доме
связи и в зале ожидания Мариупольского аэропорта - результат творчества
Виктора Михайловича). Через какое-то время к нему в гости из Чехословакии
приехал Леонид Михайлович, инженер-строитель - самый младший из детей старого
священника.

Братья после долгой разлуки о чем-то подолгу наедине говорили. О чем? Кто
знает. Впрочем, разве мало было у них тем для разговоров. От своей племянницы
Маргариты - дочери их сестры Лидии - они узнали об аресте и исчезновении
своего отца, но уточнять подробности ни у кого не пытались...

Дети мариупольского священника, в миру известного как Михаил Васильевич
Арнаутов, ушли из жизни, так и не узнав ни времени, ни места гибели своего
отца. Лишь внучке его Маргарите Михайловне Астаховой довелось увидеть и шитое
белыми нитками \enquote{дело контрреволюционера Арнаутова}, и добиться возвращения ему
доб­рого имени, и узнать дату его расстрела. Только неизвестно до сей поры,
где погребены бренные останки невинной жертвы.

С 6 сентября 1872 года по 1 марта 1938 года простиралось житие отца Михаила -
мариупольского священника, любителя-садовода, так и не ставшего монахом.

*  *  *

В 2007 году вышла книга известной правозащитницы и писательницы Галины
Михайловны Захаровой \enquote{Хранить вечно (О жертвах политического террора г.
Мариуполя и Приазовья)}. В ней есть пространный очерк об отце Михаиле
Арнаутове. Приведем лишь одну цитату из этой книги.

\enquote{Уважаемая Маргарита Михайловна! Ваш дедушка Арнаутов Михаил Василье­вич
родился 6 сентября 1872 года в селе Ялта Мариупольского уезда Донецкой
области, работал священником кладбищенской церкви. По национальности - грек,
до ареста проживал по улице К. Либкнехта, дом 62. Арестован был 19 сентября
1937 года Мариупольским городским отделом НКВД по необоснованному обвинению в
том, что он якобы проводил \enquote{контрреволюционную деятельность}. По постановлению
\enquote{тройки} УНКВД Донецкой области 25 октября 1937 года был приговорен к
расстрелу. Приговор приведен в исполнение 1 марта 1938 года. Сведений о месте
захоронения в деле не имеется. И.о. начальника подразделения Г. А. Ткаченко}.
