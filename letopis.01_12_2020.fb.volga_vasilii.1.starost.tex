% vim: keymap=russian-jcukenwin
%%beginhead 
 
%%file 01_12_2020.fb.volga_vasilii.1.starost
%%parent 01_12_2020
 
%%url https://www.facebook.com/Vasiliy.volga/posts/2766717983645616
 
%%author Волга, Василий Александрович
%%author_id volga_vasilii
%%author_url 
 
%%tags 
%%title СТАРОСТЬ
 
%%endhead 
 
\subsection{Старость}
\label{sec:01_12_2020.fb.volga_vasilii.1.starost}
\Purl{https://www.facebook.com/Vasiliy.volga/posts/2766717983645616}
\ifcmt
	author_begin
   author_id volga_vasilii
	author_end
\fi

Помню в детстве я всегда присматривался к рукам старых людей. Кожа на руке
старика другая --- она словно глянцевая и в то же время сухая. Если провести
пальцами по наружной стороне ладони, то такая кожа соберется в гармошку и не
сразу разгладится, если палец убрать. Морщинки на руках старика сухие длинные
тонкие и фаланги пальцев особенно выделены кожаными чехольчиками из морщин. 

И еще пятна. Не родинки, которые с самого детства с нами, а именно новые
желтовато-коричневые пятна, которые появляются только в старости. Вернее,
появляться они раньше - они начинают сигнализировать о том, что старость на
подходе. 

Последнее время я стал рассматривать свои руки. Они гораздо громче, чем мое
внутреннее ощущение меня самого, говорят мне о прожитых мною годах. И я начинаю
к ним привыкать. У меня словно идет с ними своеобразный разговор. Они будто
говорят мне: «Ну что, брат? Ты стареешь». А я улыбаюсь и отвечаю им: «И мне это
нравится. Хоть это и удивительно, хоть всю свою жизнь я был убежден в том, что
я-то никогда не состарюсь, но мне нравится все, что со мной происходит. И
спасибо вам, руки, за то что вы служили мне всю мою жизнь и за то, что вы так
разговариваете со мной своими морщинами и пятнышками. Спасибо Вам, что вы
трудились для меня, что обнимали любимую женщину, что держали моих детей, что
обняли моего отца незадолго до смерти, спасибо вам за это. И вот даже сейчас,
когда вы пишите этот текст, я сморю на вас, как на старых своих добрых друзей,
которые никогда меня не подводили». Руки мне говорят: «Но так будет не всегда.
Как и твои глаза мы рано или поздно начнем отказывать тебе, да ты и уже не
особенно можешь попасть ниткой в иголку».  

\ifcmt
pic https://scontent.fiev6-1.fna.fbcdn.net/v/t1.0-9/129217855_2766718413645573_3608340348991872912_n.jpg?_nc_cat=102&ccb=2&_nc_sid=730e14&_nc_ohc=one2drvqwtYAX_nxmB8&_nc_ht=scontent.fiev6-1.fna&oh=72084fe2c12941cad2b9fac3ed7927b2&oe=5FEA7D7E
fig_env wrapfigure
width 0.7
\fi

Это, конечно, правда. И на баяне я не сыграю так, как мог сделать это тридцать
лет назад, но мои руки сегодня могут передать моим внукам тепла гораздо больше,
чем были способны передавать когда-то моим детям. Наверное это такой закон
старения: чем слабее руки, тем больше они пригодны для прикосновения к детям.

Я помню теплые руки своей бабушки. Бабушки Ани. Теплые, сухие, немного пухлые и
полные любви. 

Берегите своих бабушек, друзья. Да и дедушек берегите. Не оставляйте их в
одиночестве. 

Мира и любви Вашим семьям.

\subsubsection{Комментарии}

\begin{itemize}

\item Viktor Peknno

\ifcmt
pic https://scontent.fiev6-1.fna.fbcdn.net/v/t39.1997-6/p320x320/72232259_2363266403926026_5287256169037955072_n.png?_nc_cat=1&ccb=2&_nc_sid=0572db&_nc_ohc=2dRIDjtCWbQAX9dZk3o&_nc_ht=scontent.fiev6-1.fna&oh=7482ef6b84467afe22a656451577fe62&oe=5FEB5CE9
\fi

\item Viktor Peknno
Красиво написано, и с душой. И Вы себя Василий берегите.

\item Rustam Azizov
Когда меня не станет --- я буду петь голосами
Моих детей и голосами их детей
Нас просто меняют местами, таков закон сансары
Круговорот людей (с) Баста

\item Ирина Гончар
Очень трогательно, Василий Александрович.

\item Лариса Халайджи
Стареть надо красиво!

\item Oleg Vasinsky
Жаль что это изменение промелькнуло в такой быстрой перемотке, оглянуться не успел - уже внуки.

\item Елена Елкина

Романтик, лирик, философ, сподвижник, политик, бизнесмен, работяга, истинно
верующий - всё в одном.. так складывается образ Василия Александровича Волги.
Дьявол прячется в мелочах, как говорится. Всегда прокалываются в
комментах,вроде бы коротких и почти не важных. В словах, когда хотят быстро
сформулировать и донести людям понятное им..Как говорил Николай Григорьевич
Чернышевский : у человека три характера. Первый,который видят люди, второй, что
он сам о себе думает и говорит. И третий-который есть на самом деле. Какой же
Волга на самом деле?)

\item Инна Пасечникова
Чуток не втему...Вот, друг Олег Абдуллин (Oleg Abdullin)
стихотворение написан. Мне очень нравится..

\begin{multicols}{2}
	\obeycr
И когда я стану очень стар,
С милой сидя на лесной опушке,
Я ей вновь, и снова повторю,
Я люблю тебя, моя старушка.

И когда ложится будем спать
Рядышком, и на одной подушке,
Я опять ей буду повторять,
Я люблю тебя, моя старушка,

Время подойдет окончить жизни бег,
На последнем вдохе ей шепну на ушко
Смерть пускай ещё секунду подождёт,
Я люблю тебя моя старушка,

Даже если вдруг на оборот
Ты уйдешь чуть раньше
Все решают с выше,
Я шепну - по прежнему люблю,
Знаю, всё равно меня услышишь... 
	\restorecr
\end{multicols}

\end{itemize}
