% vim: keymap=russian-jcukenwin
%%beginhead 
 
%%file 14_10_2021.fb.bielskaja_elizaveta.1.jazyk_zaschitniki_ukrainy
%%parent 14_10_2021
 
%%url https://www.facebook.com/elizabeth.bielska/posts/4440448822657249
 
%%author_id bielskaja_elizaveta
%%date 
 
%%tags donbass,jazyk,mova,ukraina,ukrainizacia,vojna
%%title ПРО РОСІЙСЬКОМОВНИХ ЗАХИСНИКІВ УКРАЇНИ
 
%%endhead 
 
\subsection{ПРО РОСІЙСЬКОМОВНИХ ЗАХИСНИКІВ УКРАЇНИ}
\label{sec:14_10_2021.fb.bielskaja_elizaveta.1.jazyk_zaschitniki_ukrainy}
 
\Purl{https://www.facebook.com/elizabeth.bielska/posts/4440448822657249}
\ifcmt
 author_begin
   author_id bielskaja_elizaveta
 author_end
\fi

ПРО РОСІЙСЬКОМОВНИХ ЗАХИСНИКІВ УКРАЇНИ

Коли пишу про важливість українізації, а радше — дерусифікації України та
повернення до власної національної мови, часто-густо натрапляю на агресивний
спротив зі словами: «Расскажитє ето рєбятам на фронтє, там всє русскоязичниє,
оні што же, мєньшиє патріоти от етого? Хватіт нас дєліть і прінуждать к етой
вашей мовє!»

\ifcmt
  ig https://scontent-frt3-1.xx.fbcdn.net/v/t1.6435-9/245786553_4440441145991350_2088773672719336652_n.jpg?_nc_cat=104&ccb=1-5&_nc_sid=730e14&_nc_ohc=nO8oLJ2dPqQAX9h3Rk0&_nc_ht=scontent-frt3-1.xx&oh=084acc2e0e07f1f19d66e961e3c1c869&oe=618DBC8C
  @width 0.4
  %@wrap \parpic[r]
  @wrap \InsertBoxR{0}
\fi

Торік їздили з Антіном у гості до наших захисників, що лікуються в
реабілітаційному центрі в Ірпені. Влаштували їм творчий вечір.

Питаю Руслана, який не так давно вийшов із коми:

— Скажіть, а правда, що більшість бійців в АТО говорить російською?

— Хто вам таке сказав? Таких — меншість. У нашому підрозділі, наприклад, не
було жодного російськомовного. Ну так, мова різна, не літературна, але не
російська точно. Її ми використовували тільки з місцевим населенням, на
блок-постах, наприклад, щоб швидше порозумітися. — Його слова підтвердили
побратими.

У спеціальній реабілітаційній тренажерній залі зустріла справжнього красеня
років 30 з Чернігівської області — Олександра. На його голові — грубий шов, а
на нозі — фіксаційна пластина.

— Що з вами трапилося?

— Три роки тому в новорічну ніч на посту в Золотому російський снайпер поцілив
мені в голову (все це він говорить дуже повільно, немовби спотикаючись на
кожному слові).

— А що з ногою?

— Через ту кулю я не відчуваю пів тіла. Не можу читати вголос. І говорити
швидко. Губи не слухаються. Три роки тому я не міг вимовити ні слова. А зараз
уже говорю з вами.

Він іще був готовий спілкуватися, та мої очі почали неконтрольовано
наповнюватися сльозами від розуміння, скільки всього пережив цей мужній молодий
чоловік, який не мав шансів вижити.

Колись російськомовний Олександр після втрати мови та 12 см кістки черепа через
кулю російського військового знову опановує вміння говорити. І природно, що
мова ця уже не може бути російською.

Тож тепер, коли мені розповідатимуть, що всі військові на фронті
російськомовні, відповідатиму, що такими є не наші, а російські військові. Якщо
ж хтось із наших захисників спілкується іншою мовою, він усе одно не має нічого
проти української та хоче, щоб його діти були українськомовними.

А цього року були з іншими бійцями, що проходять реабілітацію на Закарпатті, на
екскурсії в Мукачівському замку. Керівництво постаралося, нам дали найкращого
екскурсовода, який виявився російськомовним (для мене загадка, навіщо взагалі є
опція цієї мови у музеї). З нами був Олександр, який сильно обгорів у бою. Він
був ображений тим, що гід говорить мовою тих, хто хотів спалити його живцем.
Тож ми просто сиділи на лавці й розмовляли. «Я перейшов на українську тільки у
2014-му. І дружина теж. Усі родичі поступово переходять. Мій взвод весь зі
Східної України, і ми разом вчимось».

Тим, хто виправдовує власну російськомовність начебто російською на полі бою,
нагадаю Статтю 13 Закону про ЗСУ: У Збройних Силах України мовою службової
діяльності є державна мова. До речі, про існування цієї статті я дізналася від
керівництва 95-ї окремої десантно-штурмової бригади у Запоріжжі. Вони мені її
процитували, коли я запитала про мову армії. Так і сказали: «Ми служимо Україні
й дотримуємося її законів».

Це був уривок із нашої спільної з \textbf{Антін Мухарський} книжки "Як перейти на
українську", яка побачить світ вже за три тижні. Передзамовити її, щоб отримати
з нашими автографами та браслетом з написом "Шляхетні люди розмовляють
українською", можна на сайті UKRIDEABOOK.

Зі святом! СЛАВА УКРАЇНІ!
