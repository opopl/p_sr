% vim: keymap=russian-jcukenwin
%%beginhead 
 
%%file 16_10_2021.fb.chistopolcev_oleg.1.istoria_park_shevchenko_mova_rebenok.cmt
%%parent 16_10_2021.fb.chistopolcev_oleg.1.istoria_park_shevchenko_mova_rebenok
 
%%url 
 
%%author_id 
%%date 
 
%%tags 
%%title 
 
%%endhead 
\subsubsection{Коментарі}

\begin{itemize} % {
\iusr{Sergiu Devdyk}
Але це батьки так ''ліплять''...

\begin{itemize} % {
\iusr{Олег Чистопольцев}
\textbf{Sergiu Devdyk} так. Але якби окрім батьків навколо був українськомовний простір, то наврядче б він так сказав

\iusr{Sergiu Devdyk}
\textbf{Олег Чистопольцев} , більшості зараз діти черпають з інету , а це чи контролюють батьки. Таке...

\iusr{Андрій Дворніченко}
\textbf{Sergiu Devdyk} 

типовий випадок мій 9річний син. У школі тільки класний керівник розмовляє українською.
Зі мною спілкується українською і раз на тиждень на пласті.
Мультики та фільми дома дивимось дубльовані українською.
Книжки зараз майже усі українською.
Але.
Між собою діти спілкуються російською, більшість слів, бо батьки російськомовні 95 відсотків.
У родині у нас із мамою також російською.
Тренера по айкідо, роботетехніці, плаванню, шахам також російськомовні.
І ютубконтент майже увесь, що малий дивиться інколи московитський  @igg{fbicon.frown} 
По футболу намагається українською розмовляти і репетитор англійської також намагається українською спілкуватись.
Українізація Дніпра тільки на початку  @igg{fbicon.frown} 

\iusr{Sergiu Devdyk}
\textbf{Андрій Дворніченко} 

вчора, малював з хлопцями карту, в мене їх двоє..., карту як з московії іде газ
по нашій території, і що московити зараз роблять. В кінці сказав , шо моск@лі
на скоро нас ввалять. Треба було бачити погляд старшого. З старшим в мене
легше, там фільтрація мосслів. + Розуміння свій , наш, чужий - вже ''привито''.
А от з меншим, там в розмові дуже багато мосслів, і все через ютюб..., оточення
у нас все на наші мові говорить(регіон такий).

Тому я включаю тяжку ''артилерію'' , вбитих мамів і татів московитами на сході,
показую дітей які лишились, але це коли ловлю, що в ютюбі дивиться московський
контент....

А також їм показую і розповідаю про наших Великих українців, про наші Перемоги,
про нашу історію, не ділячи українців по теренах .

Що зрозумів, що наш - чужий діє, і дитина сама оберає нашу, свою сторону....

\end{itemize} % }

\iusr{Ганна Собко}
жахлива історія!
і не соромно цьому манкурту просити "на руССкам"!
От що,, жахливо!
Поки їм не буде стрьомно визнавати, що не знають мови- ми програємо!

\begin{itemize} % {
\iusr{Олег Чистопольцев}
\textbf{Гандзя Собко} питання до дорослих
\end{itemize} % }

% -------------------------------------
\ii{fbauth.kuharchuk_anatolij.ukraina}
% -------------------------------------



На таке зазвичай відповідаю: що не розумію іншомовних , так як в Україні єдина
мова українська

\begin{itemize} % {
\iusr{Ганна Собко}
\textbf{Анатолій Кухарчук} а лічинкам- пофігу(

\iusr{Анатолій Кухарчук}
\textbf{Гандзя Собко} , в танк не пускати @igg{fbicon.wink} 

\iusr{Олег Чистопольцев}
\textbf{Анатолій Кухарчук} тут дитина була. Я й так йому дав поживу для думок коли запитав чи він приїхав з росії.

\iusr{Тушті Лаліта}

Діти не дорослі, їх треба не відштовхувати, а приваблювати. Вони не
відповідальні ще. Це не був їхній свідомий вибір, що дорослі позбавили їх
можливості знати державну мову.

\iusr{Анатолій Кухарчук}
\textbf{Олег Чистопольцев} , він тут в як в Росії
\end{itemize} % }

\iusr{Евгений Пахомов}
Підтримую! А в першу чергу, звісно, не обирати до влади тих хто явно чи тайно сповідує православ'я – російську релігію!

\iusr{Тушті Лаліта}

У мене питання, як таке могло трапитися з учнем української школи. Зазвичай
часто у перший клас приходять стерильні діти, щодо української мови, але за рік
за найгірших умов, вони всі принаймні прекрасно розуміють українську. У якій
мовній ізоляції була ця дитина???

\iusr{Ольга Шамлова}

100 відсотків даю то по прикладу батьків так сказали поумнічать схотіли @igg{fbicon.anger} . Бо
брехня, що не розуміє, може й вчителі не завжди спілкуються українською та всі
підручники і додаткові матеріали українською. Знаю родини що спілкуються
російською, діти прекрасно розуміють українську і можуть переходити в
спілкуванні легко.

\begin{itemize} % {
\iusr{Олег Чистопольцев}
\textbf{Ольга Шамлова} можливо він бере приклад з батьків. Але тут щось пішло не так)

\iusr{Ольга Шамлова}
Прикрий випадок, ще й на свято @igg{fbicon.face.persevering} .
Наступного разу в техніку пускати лише після того, як заспіває гімн чи батько наш Бандера  @igg{fbicon.wink} .

\iusr{Lina Kacmar}
\textbf{Олег Чистопольцев} бере приклад з батьків просити кац‘язик, бо він не розуміє українську мову? Оце подібне на правду.
\end{itemize} % }

\iusr{Инна Богуславская}
Йдіть у владу, бо ніхто нічого із старих можновладців не робитиме

\iusr{Юрій Руф}
Це вже виросли діти тих чиї діти, як запевняли їхні батьки, "будут гаваріть па украінскі")))

\begin{itemize} % {
\iusr{Ганна Собко}
\textbf{Юрій Руф} так!
До того ж, це діти батьків-українофобів, які не схотіли вчити мову, бо ненавидять Україну!
Інакше, дитина ніколи б не сказала дорослому, що не розуміє українську!
А раз хлопець так сказав, отже, він пишається, що не знає української...

\iusr{Юрій Руф}
\textbf{Ганна Собко} десь так. Це фетіш "РЯПУ", який запустили ще колись в контактіку. Воно спрацювало лише в мінус - купа руских по-факту з укрпаспортами, які ненавидять Україну і українське, збільшують свій вплив за рахунок того, що їх не відрізниш від таких же руских по-факту з укрпаспортами, які люблять Україну і прихильні (лояльні) до українського
\end{itemize} % }

\iusr{Олексій Пономаренко}

\obeycr
Зі мною одного разу подібна історія теж трапилася....
Дівчинка(6-8 років) була з батьками й так само зробила мені" таке зауваження".
Я відповів, що не можу бо то моя рідна мова.
Її батьки зніяковіли та швиденько "вбігли")
То ж не будемо звертати на те увагу. Нам треба своє робити.
\restorecr

\begin{itemize} % {
\iusr{Олег Чистопольцев}
\textbf{Олексій Пономаренко} та це ж я не про себе. А про ситуацію. Нездорову. Роботи ще багато.

\iusr{Ганна Собко}
\textbf{Олексій Пономаренко} будемо звертати увагу.
Сміятитися із зайд, які не поважають нашу країну та не спроможні вивчити нашу мову!
Лише так ми повернемо ситуацію на краще.
\end{itemize} % }

\iusr{Іхь Бін Я-я}
Підросте і вилетить за борт українських сенсів разом з батьками.

\iusr{Yevhen Piskun}
Шкода, але більшість батьків навіть мультфільми дітям на Ютюбі, включають російською мовою

\iusr{Артем Сергійович Гунько}

Нажаль у нас в місті, навіть деякі депутати міськради, принципово проводять
публічні дитячі заходи російською! А коли їм робиш зауваження спускають на тебе
армію кишенькових ботів! @igg{fbicon.face.confused} 

\end{itemize} % }
