% vim: keymap=russian-jcukenwin
%%beginhead 
 
%%file slova.lenin
%%parent slova
 
%%url 
 
%%author 
%%author_id 
%%author_url 
 
%%tags 
%%title 
 
%%endhead 
\chapter{Ленин}
\label{sec:slova.lenin}


В Курске продолжаются работы по украшению улиц. В центре города устанавливают
инсталляции «Цветочное дерево». В таких композициях, которых по планам будет 6,
используется петуния ампельная.  Всего на улице \emph{Ленина} и Театральной
площади установят 18 цветочных конструкций. На Театральной площади появится
фигура павлина высотой 5 метров. На углу улиц \emph{Ленина} и Садовой установят
11 цветочных страусов высотой более 2 метров.  Напомним, недавно на Театральной
площади появился \enquote{Город цветов}. Яркие клумбы были установлены рядом с
домиками и фигурками сказочных персонажей.  Известно, что на улице
\emph{Ленина} в этом году высадят почти 36 тысяч цветов различных сортов, а
также более 3 тысяч роз, саженцев барбариса и клена,
\citTitle{В Курске на улице Ленина устанавливают \enquote{цветочные деревья}}, РИА Курск, riakursk.ru, 06.07.2021

%%%cit
%%%cit_pic
%%%cit_text
Для вірних \emph{ленінців} мета виправдовувала будь-які засоби. Будь-які! Коли \emph{Ленін}
одного разу почув про мораль у політиці, він здивовано поворухнув бровами й
зауважив: \enquote{Моралі в політиці не існує, в політиці є лише доцільність}. Цей
принцип був настільки зведений комуністами в абсолют, що про мету вони
часто-густо забували. Залишалась сама доцільність. Хоча справжньою метою тирана
було утримання влади, бо він не лише чудово знав, а й активно практикував
принцип Vae victis – горе переможеним...
%%%cit_comment
%%%cit_title
\citTitle{У політику треба повернути мораль. Інакше ми приречені залишатися сталіністами}, 
Геннадій Друзенко, gazeta.ua, 14.06.2021
%%%endcit

