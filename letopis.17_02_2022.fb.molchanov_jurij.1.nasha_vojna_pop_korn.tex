% vim: keymap=russian-jcukenwin
%%beginhead 
 
%%file 17_02_2022.fb.molchanov_jurij.1.nasha_vojna_pop_korn
%%parent 17_02_2022
 
%%url https://www.facebook.com/george.molchanov.9/posts/4695591780553753
 
%%author_id molchanov_jurij
%%date 
 
%%tags rossia,ugroza,ukraina,vtorzhenie,zapad
%%title Для них наша война, это максимум поп-корн и боевик по телевизору
 
%%endhead 
 
\subsection{Для них наша война, это максимум поп-корн и боевик по телевизору}
\label{sec:17_02_2022.fb.molchanov_jurij.1.nasha_vojna_pop_korn}
 
\Purl{https://www.facebook.com/george.molchanov.9/posts/4695591780553753}
\ifcmt
 author_begin
   author_id molchanov_jurij
 author_end
\fi

Нравятся мне рефлексии на тему, мол, ну, и что теперь скажут мировые и
неполживые столпы журналистских стандартов по поводу 16 февраля.

Да ни хрена они говорить не будут. 

И не собирались. Кому говорить? Аудитории США, из которой лишь треть способна
указать хоть приблизительное место на карте, где находится Украина? Или
европейцам, которые вспоминают об Украине преимущественно в разрезе найма
домработницы или недорогой раб.силы?

Для них наша война, это максимум поп-корн и боевик по телевизору. Один из
виртуальных источников адреналина. Вспомните, много ли вы рефлексировали на
войну в Сирии, особенно до ввода туда контингента РФ? Много вам говорили
названия Хомс, Алеппо или Дейр-эз-Зор? Сильно переживали из-за жертв арабской
весны? Многие из нас с первого раза покажут, где на карте Ливия, Бахрейн, Тунис
или тот же Ирак? Я уже молчу об эмпатии по бомбежкам Югославии или трагедии
Северной Ирландии.

Наша Станица Луганская для них немногим понятнее, чем какой-нибудь ваххабитский
схрон в афганском ущелье.

Западный обыватель, неискушенный жвачкой политических ток-шоу (этот медиаформат
живет и размножается лишь в кризисных условиях) видит это все контурными
картинками больших медийных нарративов – портретами Путина, Байдена и, в лучшем
случае, Си. Что-то там они между собой не поделили. А в остальном у них своих
проблем и хлопот, особенно в постковидные времена, по горло.

Мантры про «весь свiт з нами», тиражируемые крыголамами и грантоедами, для
которых война и противостояние есть хлеб насущный, погрузили нас в чудовищную
топь политического инфантилизма. Мы нащупали зону комфорта, где все проблемы
можно легко заретушировать плакатом, площадной кричалкой или происками пятой
колонны. А еще мы можем топать ножками и решительно требовать помощь от мира.

И когда реальность вдруг неожиданно, как зима каждый год в Киеве, щелкнула по
лбу рядового украинца, мы оцепенели: как так? Как собрали посольские манатки и
во Львов? Как на олигархосамолеты и семьями из страны? А как же борьба за
демократию и европейские ценности? А как же мы?

И эта реальность только начала прорисовываться. Это только первые, робкие
росточки сквозь толстый слой асфальта постреволюционного угара, иллюзий и
многолетнего политического мошенничества.

Впрочем, как бы я не хотел здесь поставить точку фразой welcome to reality,
физиономия очередного оффшорного патриота, изрыгающего в мозг телезрителя
порцию примитивной полит.туфты, продолжит этот спектакль абсурда. Дешевое
представление под вывеской агонии постсоветской системы.
