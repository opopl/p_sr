% vim: keymap=russian-jcukenwin
%%beginhead 
 
%%file 10_12_2021.stz.news.lnr.lug_info.1.nagrazhdenie_konkurs_dostojanie_respubliki
%%parent 10_12_2021
 
%%url https://lug-info.com/news/nagrazhdenie-pobeditelej-konkursa-dostoyanie-respubliki-sostoitsya-13-dekabrya-v-luganske
 
%%author_id 
%%date 
 
%%tags obschestvo,lnr,konkurs,nagrazhdenie,donbass,zhizn,lugansk
%%title Награждение победителей конкурса "Достояние Республики" состоится 13 декабря в Луганске
 
%%endhead 
\subsection{Награждение победителей конкурса \enquote{Достояние Республики} состоится 13 декабря в Луганске}
\label{sec:10_12_2021.stz.news.lnr.lug_info.1.nagrazhdenie_konkurs_dostojanie_respubliki}
\Purl{https://lug-info.com/news/nagrazhdenie-pobeditelej-konkursa-dostoyanie-respubliki-sostoitsya-13-dekabrya-v-luganske}

Награждение победителей официального республиканского молодежного конкурса
\enquote{Достояние Республики} состоится 13 декабря в Луганской государственной
академии культуры и искусств имени Михаила Матусовского. Об этом сообщила
пресс-служба Министерства культуры, спорта и молодежи (МКСМ) ЛНР.

\ii{10_12_2021.stz.news.lnr.lug_info.1.nagrazhdenie_konkurs_dostojanie_respubliki.pic.1}

\enquote{Победители в каждой из 14 номинаций конкурса будут оглашены на торжественной
церемонии награждения. Также будет оглашено имя лауреата главной номинации
конкурса – \enquote{Достояние Луганской Народной Республики}. Мероприятие начнется в
13:00}, - говорится в сообщении.

В ведомстве отметили, что церемония награждения будет проводиться с учетом
требований Чрезвычайной санитарно-противоэпидемической комиссии \enquote{в связи с
угрозой распространения коронавирусной инфекции с использованием средств
индивидуальной защиты}.

Напомним, что более 80 молодых жителей ЛНР стали победителями регионального
этапа конкурса.

Организатором мероприятия выступает МКСМ при поддержке администраций городов и
районов ЛНР, а также органов государственной власти и профсоюзных организаций
Республики. Конкурс направлен на поддержку молодежных инициатив, консолидацию
усилий по реализации государственной молодежной политики в ЛНР, обмен опытом и
инновационными технологиями. За семь лет проведения конкурса в нем приняли
участие более 1 350 молодых людей со всех территорий Республики.

Впервые молодежный конкурс \enquote{Достояние Республики} прошел в ЛНР в 2015 году.
Тогда в конкурсе приняли участие 124 человека. По результатам пяти этапов
состязания жюри из 10 независимых экспертов определило победителей в 24
номинациях, представлявших различные сферы жизни Республики.

В 2020 году в конкурсе приняли участие более 250 активных молодых людей из всех
городов и районов ЛНР.
