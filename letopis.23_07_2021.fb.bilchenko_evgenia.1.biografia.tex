% vim: keymap=russian-jcukenwin
%%beginhead 
 
%%file 23_07_2021.fb.bilchenko_evgenia.1.biografia
%%parent 23_07_2021
 
%%url https://www.facebook.com/yevzhik/posts/4088216334546779
 
%%author Бильченко, Евгения
%%author_id bilchenko_evgenia
%%author_url 
 
%%tags bilchenko_evgenia,biografia,travlja.bilchenko,ukraina
%%title БЖ. Немного моей биографии для вас, друзья
 
%%endhead 
 
\subsection{БЖ. Немного моей биографии для вас, друзья}
\label{sec:23_07_2021.fb.bilchenko_evgenia.1.biografia}
 
\Purl{https://www.facebook.com/yevzhik/posts/4088216334546779}
\ifcmt
 author_begin
   author_id bilchenko_evgenia
 author_end
\fi

БЖ. Немного моей биографии для вас, друзья.

Родня, привет. Я знаю, что вы меня читаете, продираясь сквозь хейтерские
угрозы, оскорбления и пожелания смерти, которые цензура здесь не банит.
Заметьте, как хорошо проявляются правила кибернетической игры в спецоперациях.
Я также знаю, что вы ждёте новостей, которые касаются моего здоровья,
состояния, места пребывания и стадии сопротивления. Сообщаю.

Здоровье. Врачи все более склоняются к хроническому сепсису - постоянному
циркулированию очага инфекции в организме. Сразу скажу, что после смерти
дедушки я растеряла все медицинские связи и сейчас могу консультироваться лишь
у частников. Ещё по контакту мне советуют из России. Впереди у меня -
развернутые анализы крови, я начала новый антибиотик - левомак, - всё по
виртуалке пока решается. Здоровье - хуже. Наверное, меня загнали уже вконец. Я
терплю. Я - не маг, над телом не властна. Принимаю.

Состояние. Да, не буду лгать. Из меня сильно хотели сделать психиатрическую
больную, чтобы дискредитировать диссидента, но не вышло: мне ничего тяжелее
бифрена или антидепрессанта для "нервных дам" не дают, хотя мои нагрузки -
посильнее, чем у "нервных дам". Я помню о судьбе Бродского. Украина мощно
десоветизировалась, но зачем-то защищает методы карательной медицины, для
расправы с неугодными, при этом игнорируя все лучшее от СССР: от промышленности
до социальных льгот. Вот где дурдом зарыт. Это тоталитаризм. Новый. Он так
выглядит.

Место пребывания. Украина. По этому поводу мои преследователи сами признались в
разжигании гражданской войны против собственных соотечественников. Я - не
доброволец русского фронта, не ополченец, не воюю с оружием в руках. Я - такой
же гражданин Украины, как они, десятилетия отдавший наук, литературе,
педагогике. Я выражаю свою личную точку зрения на войну, на Донбасс, на Россию
и на русский язык, отличную от идеологической доминанты. Статья 34 о свободе
слова. Конституция Украины. В условиях войны? Да, именно, потому что у меня уже
много лет из уст слетает один и тот же вопрос: зачем националисты воюют против
нас? Зачем нас лишают средств к проживанию, подвергают травле, угрожают,
мучают, доводят до болезни и иногда просто убивают? Факт? Факт. Это
тоталитаризм. Новый. Он так выглядит.

\ifcmt
  pic https://scontent-lga3-1.xx.fbcdn.net/v/t1.6435-9/222344291_4088216277880118_901206215893885802_n.jpg?_nc_cat=102&ccb=1-3&_nc_sid=8bfeb9&_nc_ohc=6bKeU1ct3isAX88PBqg&_nc_ht=scontent-lga3-1.xx&oh=83e230b1730e34670e148859f1151d1c&oe=6126890B
  width 0.4
\fi

Итак, кто я? Этнически - коренная украинка, только один прадед - русский. Я
этносу в духовном развитии человека не уделяю особого внимания. Это для порядка
справка, дабы известно было, что мои предки никуда не выезжали столетиями из
этой земли. Я - не сторонник этой убогой концепции почвы, это для тех, кому она
важна. Кто я цивилизационно? По ценностям, языку, духовному воспитанию? Ну,
конечно, русская. Я выросла на Пушкине и Шевченко одновременно. Что-то потом
домысливалрсь и переосмысливалось, но факт культуры есть факт. Да, мы не делили
народы. Бытие украинской культуры и бытие русской во мне составляют единое
историческое целое, оба языка всегда были у меня в свободном использовании. Это
неделимо и это безумно обогащает. 

Быть проукраинским и значит быть прорусским. Надеюсь, что "наоборот" сможет
сохранится по отношению к нормальным людям, националистов и фанатиков это не
касается, отравленного ими общественного начала, начала массы, - тоже. А теперь
признайтесь, сколько нас таких? Сотни тысяч? Нет, миллионы. И они всех хотят
сослать в Сибирь, своих же граждан представив как пришлых, случайных,
заимствованных, второсортных? Это тоталитаризм. Новый. Он так выглядит.

Стадия сопротивления. Ну, я не сдамся. О том, что реорганизуют вуз или, по
крайней мере, кафедру, я знала ещё в январе. Мне сказали, что это будет крайняя
мера. Крайняя? Это же великолепно, что может один человек делать со структурой.
Вдумайтесь. Они не смогли доказать моей профнепригодности и аморального
поступка. Хотя как прием дискредитации инакомыслящего могли бы: это было бы
потешно, но даже на это они не пошли. А я ждала. Я готова к любой дискуссии. Я
публично могу защитить все свои научные разработки. А мои политические взгляды
- мое нерушимое право. Конституция запрещает увольнять по политическим
взглядам. Факт нарушения конституции? Факт.

Что я намерена делать? Мы работаем с адвокатом по предлагаемым должностям. Это
нужно, чтобы пройти стадии борьбы с несправедливостью до конца и получить то,
что нам причитается. Поскольку я готова к любым жизненным испытаниям, я не
сильно буду плакать по НПУ, даже, если предлагаемые должности - формальность. Я
ничего не делаю без адвоката. Действительно, мне давно в универе своем тесно.
Весь спектр научного и литературного общения - всегда за его пределами. 

Потеря зарплаты? Да, очень пугает. Я же недаром получала от ректора награды как
преподаватель года. Я умею лишь учить. Но справимся. Что делать? Я - типичный
японец: я никогда не меняла место работы, в моей трудовой книжке за 18 лет (с
22 лет) - ОДНА запись. Так меня дед и бабушка воспитали. Потому, конечно, это
было травмой - потерей жизненного смысла, утратой призвания. Вы много видели
кандидатов наук в 26 и докторов в 32? Здесь, в/на Украине, как угодно. Я - нет.
"Ботан" - мой тип. Даже самой смешно. Многие считали, что я круче, но на самом
деле всю жизнь я лишь училась и учила. Много. 

Но надо привыкнуть, надо выжить. Я же сознательно отказалась извиняться, мне в
ректорате предлагали. Нет. Нет. Нет. Этого не будет. Я не извиняюсь перед
нацизмом. Я диагностирую это именно как неонацизм в современных условиях. Так
выглядит новый тоталитаризм. Акцентирую ваше внимание на неототальности.

Я понимаю, почему я их раздражаю больше тех людей, кто сразу был против и кого
легко можно было удалить на старте. Я не отвечаю идеологией на идеологию. У
меня факты жизни. Против этого тяжело идти. В 2014 меня погнали туда, куда
погнали половину - добрую половину, это надо признать - населения: за все
хорошее против всего плохого. В 2015 я как "сознательный гражданин" пошла
защищать свою Родину от... (все мое бывшее окружение и весь университет этим
занимается до сих пор).

Я не смотрю медиа. Вообще. В последнее время стала, и то, чтобы найти базу для
критики пропаганды в медиа-теории. Все свои выводы я делаю сама. Это убивает
моих оппонентов. Я говорю то же, что русские СМИ, говорят они? Ну, да,
наверное. Я, которая лично прошла период майдана, которая лично наблюдала за
войной и военными и пыталась помочь, - говорю то же самое. Это было ошибкой.

Да, одни граждане нашей многострадальной страны убивают других. Так я сказала
пять лет назад... Но в 2016 году оказалось, что у нас в стране нельзя менять
взгляды. Когда меня на общественном суде отстраняли (мне студенты сделали
запись), мой бывший руководитель, Виктор Бондаренко сказал что-то вроде: "Это
последняя капля. Мы ее и так долго терпели". 

Да, пять лет я "переобувалась" с незначительными потерями: меня убрали из всех
учёных советов, меня преследовали на допросах и постоянно запугивали. Я могу
теперь это сказать. Пять лет я писала посты, почему я ошибалась насчёт нового
курса Украины и пять лет меня подвергали моральному террору за это. Так я жила.
Я очень любила своих детей. Я втайне надеялась, что наука, профессионализм и
исполнительность победят политоту. Нет. Это тоталитаризм. Новый. Он выглядит
так.

Я ощущаю сейчас определенный итог. Да, я на развалинах. Да, я знаю, что многие
не прощают на Донбассе и в России, многие. Я это всё принимаю безропотно. Фу...
Сказала, даже легче стало. Я сделала все, что требовала совесть: поддержала
новую украинскую власть,  не веря ей, но веря народу, была там с этой частью
народа, где он, к несчастью был, как писала Ахматова, увидела, что народ
ошибся, сказала, попросила прощения, искупаю, отгребаю. 

Многие думают, что у меня есть спонсор: нет. Его не было на майдане, его не
было и нет в России. Все мои доходы - ядро - это зарплата, очень тяжёлый
фриланс (я недавно, почти не двигаясь, за две недели сделала перевод романа в
пару-тройку сотен страниц, украинский-русский: владение в совершенстве),
продажа книг, редактура, научные консультации, поэтические мастер-классы и ваша
посильная помощь. Всё. Если бы не читатели, я не знаю, как бы я вытянула.

\ii{23_07_2021.fb.bilchenko_evgenia.1.biografia.cmt}
