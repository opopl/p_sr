% vim: keymap=russian-jcukenwin
%%beginhead 
 
%%file 03_11_2021.fb.voloshin_oleg.opzzh.1.obschestvo_krizis
%%parent 03_11_2021
 
%%url https://www.facebook.com/oleg.voloshin.7165/posts/4837466502953039
 
%%author_id voloshin_oleg.opzzh
%%date 
 
%%tags krizis,obschestvo,politika,strana,ukraina
%%title Состояние общества - кризис
 
%%endhead 
 
\subsection{Состояние общества - кризис}
\label{sec:03_11_2021.fb.voloshin_oleg.opzzh.1.obschestvo_krizis}
 
\Purl{https://www.facebook.com/oleg.voloshin.7165/posts/4837466502953039}
\ifcmt
 author_begin
   author_id voloshin_oleg.opzzh
 author_end
\fi

Настроение в зале Верховной Рады за последний месяц радикально изменилось. Всем
очевидно, что в стране системный кризис и идеальный шторм, перед которым
зелёный режим беспомощен. Более того, жадность, тупость и близорукость команды
Зеленского ежедневно множит их собственные проблемы. В частных беседах слуги
признаются, что в полном шоке от происходящего. Оттопыренные от пачек долларов
в карманах пиджаки ещё заставляют их послушно жать кнопку, но понимание
неизбежности конца «зелёного праздника» приходит с неизбежностью смены осени
холодной зимой. 

Но важна даже не судьба режима Зеленского. С ним все ясно. Тектонические сдвиги
надвигаются в базовых установках общества. Два из трёх опорных столпов
доминировавших после 2014 года сил рухнули. Во-первых, огромному большинству
украинцев очевидно, что постмайданные политики не только не искоренили
коррупцию, но возглавили разграбление общественных богатств в беспрецедентных
масштабах. Во-вторых, сказка о «будущем в Европе» развеялась, как утренняя
дымка. Настроение «нас там никто не ждёт» стало доминирующим даже среди тех,
кто искренне пылал лозунгом «Геть від Москви». В силу объективных причин ни
США, ни, тем более, ЕС на практике не готовы предпринимать никаких весомых
усилий для оживления постмайданных «евроиллюзий». 

Лагерь соросят дискредитирован, демотивирован и расколот. Печальная участь
развалившейся до мышей фракции «Голос» - лишь наиболее очевидное отображение
этого процесса. Прочный контроль Порошенко над своим ядерным электоратом при
его антирейтинге делает невозможным выдвижение сильного лидера, способного
консолидировать прозападно-националистический лагерь. А взявший на вооружение
идеологию предшественника, но демонстрирующий гораздо худшие результаты в
управлении Зеленский ежедневно отталкивает ранее эмоционально ушедшее в
«ура-патриотизм» электоральное болото. Они все еще не любят Россию, но уже
утратили веру в значимость и, главное, перспективность борьбы с ней.  На
следующие выборы они могут вообще не пойти. 

Одновременно нарастает радикализация наиболее активной части этого сегмента
общества. На первый план вновь будут выходить деятели вроде Яроша и Билецкого,
что ещё больше оттолкнёт более умеренных украинцев от «патриотического лагеря».
Таким образом, такие процессы, как эмиграция, апатия и маргинализация будут
определять динамику в этой среде. 

Признаю, анализ сделан широкими мазками. Но он отражает общие тенденции. И если
эскалация конфликта на Донбассе или вооружённый мятеж радикалов в столице не
поставят на грань закрытия украинский проект в целом, то очень серьёзные
перемены неизбежны. Или, если угодно, реванш.

\ii{03_11_2021.fb.voloshin_oleg.opzzh.1.obschestvo_krizis.cmt}
