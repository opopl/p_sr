% vim: keymap=russian-jcukenwin
%%beginhead 
 
%%file slova.mechta
%%parent slova
 
%%url 
 
%%author_id 
%%date 
 
%%tags 
%%title 
 
%%endhead 
\chapter{Мечта}

%%%cit
%%%cit_head
%%%cit_pic
%%%cit_text
В ту пору все младшие отпрыски знатных родов \emph{мечтали} поскорее надеть длинный
белый плащ с черным крестом - традиционное облачение
рыцарей-тамплиеров. Уже от самого слова "тамплиер" веяло духом дальних
странствий и подвигов, оно вызывало в воображении корабли, идущие на
Восток с гордо раздутыми парусами, страны, где вечно сини небеса, коней,
мчащих всадников в атаку через пески пустыни, все сокровища Аравии,
богатый выкуп за пленников, отбитые у врага города, отданные на поток и
разграбление, крепости, к которым от морского берега ведут гигантские
лестницы. Говорили даже, что у тамплиеров есть свои тайные гавани, откуда они
отплывают в неведомые земли... И свершилась \emph{мечта} Жака де Молэ: одетый в
роскошный плащ, складки которого спадали до золотых шпор, он горделиво шагал
по далеким городам... Он достиг высших ступеней иерархии Ордена, таких,
которых никогда и не надеялся достичь, получил все титулы и
наконец по выбору братьев-тамплиеров занял высший пост Великого магистра
Франции и заморских стран, имел под своим началом пятнадцать тысяч рыцарей. И
все кончилось этим склепом, этой грязью, этими лишениями. Редко на чью долю
выпадала такая неслыханная удача и на смену ей приходило столь глубокое
унижение
%%%cit_comment
%%%cit_title
\citTitle{Железный король}, Морис Дрюон
%%%endcit
