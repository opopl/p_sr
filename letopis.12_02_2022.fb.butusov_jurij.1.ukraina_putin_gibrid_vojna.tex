% vim: keymap=russian-jcukenwin
%%beginhead 
 
%%file 12_02_2022.fb.butusov_jurij.1.ukraina_putin_gibrid_vojna
%%parent 12_02_2022
 
%%url https://www.facebook.com/butusov.yuriy/posts/7242133682493532
 
%%author_id butusov_jurij
%%date 
 
%%tags infvojna,napadenie,putin_vladimir,rossia,ugroza,ukraina
%%title Что происходит? НАТО и Украина побеждают Путина в гибридной войне
 
%%endhead 
 
\subsection{Что происходит? НАТО и Украина побеждают Путина в гибридной войне}
\label{sec:12_02_2022.fb.butusov_jurij.1.ukraina_putin_gibrid_vojna}
 
\Purl{https://www.facebook.com/butusov.yuriy/posts/7242133682493532}
\ifcmt
 author_begin
   author_id butusov_jurij
 author_end
\fi

Что происходит? НАТО и Украина побеждают Путина в гибридной войне

1. Начнет ли Россия широкомасштабную войну против Украины 16 февраля? 

Нет. Российские войска вокруг Украины развернуты, но готовности к немедленным
действиям против Украины у врага нет. 

2. Почему Запад ежедневно делает заявления о неизбежной войне и нагнетает
обстановку, значит ли это, что будет война?

Нет. США, Великобритания и их союзники своими заявлениями о неизбежной войне
заставляют все европейские страны НАТО определиться со своей политической
позицией. США своими действиями радикализируют обстановку в Европе, чтобы
европейские страны создали единый фронт поддержки Украины, и таким образом,
разломали сепаратные договоренности с Путиным о невмешательстве. Байден хочет
обеспечить солидарность европейских стран в вопросе Украины. Теперь в гибридную
войну в Украине по-настоящему включается НАТО. 

3. Президент Зеленский ругает Запад за нагнетание обстановки и провоцирование
паники среди инвесторов, выгодна ли нам на самом деле политика США и их
союзников?

Да. На самом деле, впервые за 8 лет нашей войны ведущие страны НАТО заняли
активную наступательную позицию против угроз Путина, впервые нам широким
потоком пошла военная и финансовая помощь. Впервые у границ Украины НАТО
развертывает группировку для вероятного противодействия российскому вторжению в
Украину. Это огромная и бесценная помощь для народа Украины. А президент
Зеленский либо утратил понимание, что происходит, либо пророссийские силы в его
окружении - Ермак и Демченко, манипулируют президентом и пользуются его
невежеством, чтобы не испортить свои контакты с Кремлем. Позиция НАТО
заслуживает только глубокой благодарности, это исторический перелом в
отношениях Украины с Западом.

4. Не подорвет ли нашу экономику такая паника и заявления о войне?

Нет. Нашу экономику и государство подрывает не война с Россией, которая идет 8
лет, и не очередные угрозы Путина, и уж тем более, не помощь НАТО, а
некомпетентность власти и не умение пользоваться возможностями и огромной
международной поддержкой, которую предоставляет этот кризис. 

5. Может ли Россия начать нападение ударами авиации и ракет, а уже затем
бросить в атаку наземные силы?

Да. Но Россия не атаковала с воздуха даже в 2014-м, потому что  это приведет к
большим жертвам, и более жесткой антироссийской реакции в мире. Удары с воздуха
без наземного наступления усилят сопротивление и международную поддержку
Украины. Здесь находятся журналисты всех мировых СМИ, и видео российского
воздушного террора будет главной темой во всем мире. Европарламент принял
резолюцию о полной блокаде российского энергетического импорта в Европу в
случае большой войны. Такой блокады экономика РФ не выдержит.

6. Заинтересована ли Россия в нападении?

Да. Россия присоединила к себе Донбасс и Крым де-факто, и  российской армии
необходимо заставить украинские войска прекратить стрелять и отбросить  от
Донецка и Горловки. Мы создаем угрозу ликвидации \enquote{ДНР} в любой момент, и Россия
не сможет там стабилизировать свою власть. Именно Донбасс является фронтом, где
Россия заинтересована провести наземное наступление. Которое можно политически
назвать действиями «ополченцев», а не регулярной армии.

7. Надо ли гражданам уезжать, эвакуироваться?

Нет. Надо занимать свое место в строю. Не надо всем брать в руки оружие - да
его и нет на складах, каждый адекватный человек необходим на своем месте, в
логистике, обеспечении, в тылу, на передовой. Надо понимать, что делать в
условиях кризиса, войны, это нормально в современном мире. Этот кризис - не
последний.

8. Закончится ли этот кризис этой зимой?

Нет. Эта война на годы, на долгие годы. Украина - слишком богатая и слишком
важная геополитически страна, чтобы нас когда-либо наши соседи оставили в
покое. Пока Путин у власти, и пока Украина остается слабым, плохо
организованным государством, где правители пытаются умиротворять Путина,
\enquote{смотреть в глаза}, \enquote{просто прекратить стрелять}, проваливать спецоперации
собственных спецслужб против России, мы будем оставаться жертвой в глазах
агрессора. 

9. Можем ли мы выиграть в войне с Россией?

Да. Мы уже выигрываем в войне с Россией. Поддержка НАТО резко повышает нашу
обороноспособность и создает многие новые возможности. Путин не смог выполнить
свои цели - вернуть марионеточный режим Януковича, вернуть к власти
пророссийские силы, заставить Украину прекратить сопротивление по модели
Приднестровья. Мы добились беспрецедентных успехов в развитии своей
независимости, и с каждым годом наши силы растут. Путин как и Гитлер и Сталин
не вечен. Когда мы построим государство с компетентными лидерами, которые
уделают внимание прежде всего вопросам безопасности и обороны, Россия с
Путиным или без Путина осознает бесперспективность дальнейшей войны, и тогда
либо они сами вернут Крым и Донбасс, либо мы сделаем это самостоятельно.
