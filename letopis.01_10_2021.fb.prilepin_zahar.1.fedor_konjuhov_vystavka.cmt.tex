% vim: keymap=russian-jcukenwin
%%beginhead 
 
%%file 01_10_2021.fb.prilepin_zahar.1.fedor_konjuhov_vystavka.cmt
%%parent 01_10_2021.fb.prilepin_zahar.1.fedor_konjuhov_vystavka
 
%%url 
 
%%author_id 
%%date 
 
%%tags 
%%title 
 
%%endhead 
\subsubsection{Коментарі}

\begin{itemize} % {
\iusr{Вера Арнгольд}

Люблю отца Фёдора. Был неоднократно у нас в области,на Соль- Илецкие озера
приезжает. Встречалась и с ним, и с матушкой. Говорили ночью под огромными
звёздами о космосе! Он настоящий, искренний, русский патриот, глубокая вера.
Матушка его все за руку держала, чтобы опять не уехал)))

\iusr{Юрий Круглов}

Захар, вы очень аналитично и проникновенно описали .Спасибо

\iusr{Дмитрий Михайлин}

Мне тут какое-то время назад для одного проекта понадобилось понять, кто самые
популярные люди в стране - из содержательных, а не "звезд". И я устроил опрос
тут, в фейсбуке. С приятным удивлением выяснил, что среди моей аудитории первое
место занял именно Конюхов.


\iusr{Оксана Дарчук}

Удивительнейший человек... он ещё и рисует. Чудеса просто.


\iusr{Татьяна Драгныш}
Похоже на Рокуэлла Кента. Неплохо!

\iusr{Вероника Пономарёва-Коржевская}

Это мой герой. Ты абсолютно прав! такие люди должны находится в центре внимания
общества, но к большому сожалению, это не так. Я прочла все его труды, книги,
была у него в мастерской. Счастлива знакомству с таким человеком.

\iusr{Сергей Русаков}

Удивительное сочетание у этого героя нашего времени ( безусловно ) - великое
человеческое мужество и тонкая душевная организация, построенная на любви и
Вере. Может быть это и есть идеал гражданина русского? Федор - воистину человек
из прошлого, как то явившийся в наш сегодняшний мир.

\begin{itemize} % {
\iusr{Milarus Rusova}
\textbf{Сергей Русаков} он из будущего, из вечности, сын человеческий от Отца, житель вселенной
\end{itemize} % }

\iusr{Илья Дубенский}

Великий Человек. Мы ещё пока не в полной мере осознаем КТО ЭТО для России....
Да и для всей нашей планеты... До того как с ним познакомился, я только в
житиях о таких читал

\iusr{Юлиана Бачманова}
Надо сходить посмотреть. Мощный такой, искренний примитивизм. Да, человек чудный

\iusr{Кованова Галина}

Поражает, когда вижу, читаю о нем иногда такие снисходительно- иронические
отзывы. Я искренне восхищаюсь Конюховым, понимаю его, понимаю обьем его
таланта.


\iusr{Lubov Goncharova}
Да, но пол жизни в одиночном плавании! Когда успел детей сделать! Картины рисовать? Жене поклон...

\iusr{Юрий Кононов}
По цветовой гамме на буддийские танки похоже.

\iusr{Дмитрий Ленский}

Красивые работы, и светлые, но я бы на месте Фёдора не смог бы такое о себе
читать... ну вы понимаете, он не кажется себе героем, он просто такой.... опять
же, как мне видится..: слежу за ним, уважаю, люблю. Вот он говорит о
загрязнении океана... а вы? Что вы или мы делаем, чтобы изменить кардинально
подход к экологии? То-то и оно. А он говорит, а мы и не слушаем...

\begin{itemize} % {
\iusr{Илья Дубенский}
\textbf{Дмитрий Ленский} а он и не спорит ни с кем обычно... Только приходит в мастерскую новый человек и... "нате ка, поносите вериги с мощевиком..." и за чаем, даже в разговоре непринуждённом, он никогда без дела не сидит. То лепит что-то, то рисует, то сам этот самый чай наливает... И чувствуешь себя рядом с ним как будто.... Как будто попал к Преподобному Сергию в скит и ты тут свой...)
\end{itemize} % }

\iusr{Milarus Rusova}
Спасибо за пост. Картины восхитительные, написанные в Духе. В них вся
вселенная, "на земле как и на небе"(Библия) , его руки вёл Отец.

\iusr{Irina Vladimirova}

Спасибо, Вам, Захар, за такой интересный и душевный комментарий. Да, Фёдор
Конюхов - уникальный , многоталантливый человек, он мне напоминает русских
первопроходцев, и он такой пример невероятных возможностей и способностей
человека. И , пусть это и пафосно звучит, я бы его назвала Человеком ( вот так
с большой буквы). А признания ему , наверное, достаточно. Он любит природу и не
утратил чистого любопытства к тому, что она нам даёт , и восхищения ее
чудесами.

И мне нравятся его картины , наивные и мудрые, написанные от души и с любовью к
природе и к людям.

Дай Бог ему ещё много сил и здоровья! А семье - терпения.

\iusr{Валентина Бондаренко}

С огромным уважением отношусь давно к Фёдору Конюхову!

\iusr{Irina Hirsh}
А каковы багеты. Отдельное искусство.

\iusr{Димитрий Леонтьев}
Человек существо удивительно сбалансированое
Если на виду выдающиеся достоинства то в нем таятся и чудовищные недостатки!
Че дома то не сидится?

\iusr{Андрей Петров}
Самородок, сын своего Отечества.

\begin{itemize} % {
\iusr{Надежда Колышкина}
\textbf{Андрей Петров} Помимо Отечества есть еще и так называемое "глубинное государство", которому на сынов Отечества наплевать. Уйди его суденышко на дно океана, реакции было бы не больше, чем по поводу летчиков, расстрелянных в небе партнерами.
\end{itemize} % }

\iusr{Евгения Позина}

Абсолютно согласна! Года два назад открыла для себя этого человека( раньше и
слышала, и писала о нем, но не вникала)А потом пришло осознание его силы и
величия ,и гордости, что живёшь в одно время с таким человеком. Специально
поехала посмотреть на его приход в Москве. Это тоже- целый, не похожий ни что
мир. Думаю, что он сам старается держаться в тени и ,наверное, так ему легче жить.

Вообще - глыба! Второго такого ни у кого нет .Слава Богу - у нас...

\iusr{Евгений Тухто}
Давным давно Фёдор Конюхов достоин иметь звание, как ЗМС, так и Герой России.

\iusr{Максим Казуров}
Согласен с оценкой, мне то же видится.

\iusr{Алла Мушкина}
Девочка, дочка?

\iusr{Иван Запольский}
А я вот поклонник чистого цвета, сочетаний чистых красок. Хотя мне нравится и сумрачная "грязная живопись".

\emph{Татьяна Горбунова}

Очень очень уважаю этого человека! Такой цельный, бесстрашный, талантливый и в
письме, и в живописи, фонтанирующий невероятным идеями своих путешествий.
Доказывающий, что человек все может! Спокойный и уверенный. Это очень
вдохновляет. Отрадно, что он есть, такой вот пример человека. Вспоминаю, что
радовалась как ребёнок, когда возвращаясь из Индонезии, в иллюминаторе самолета
увидела воздушный шар! Он как раз тогда начал свой облет Земли на шаре. От души
ему здоровья, долгих лет и в ожидании новых приключений! В ближайшие дни тоже
пойду на выставку)


\iusr{Тиана Строганова}
А мне картины понравились. В русском сказочном духе.

\iusr{Станислав Бартенев}
Сокровища всегда на дне океана, а на поверхности плавает известно что.  @igg{fbicon.smile} 

\iusr{Татьяна Дайбова}
русский гений

\iusr{Татьяна Горбунова}
Как же Захар замечательно его описал! Спасибо)

\iusr{Алексей Соколов}
И все же мне нравится соцреализм

\iusr{Даниил Шагин}
Интересно, даже не знал, что рисует

\iusr{Сергей Суразаков}
Чистота взгляда - это да.

\iusr{Алёна Курбатова}

На недавней выставке работ Федора Конюхова в Екатеринбурге меня впечатлила
картина "Гребец". Сильно, ярко. Так надоели эти депрессивные страдания
современных "талантов". Спасибо путешественнику-художнику!!!!


\iusr{Андрей Скворцов}
Браво!

\iusr{Наташа Капралова}

Его каждое слово, каждую интонацию ловила, когда была несколько лет назад на
встрече с ним в одном вузе. Это потрясающая личность, какой-то невероятный
масштаб. У меня была масса вопросов, да все они крутились в голове, томились и
таяли. Потому, что казались глупыми, ужасно для него глупыми. Однако, ответ на
один из вопросов мне удалось получить. Правда, не я его озвучила...

- Федор Филиппович, вы путешествуете почти всегда один. А взяли бы напарника?
Какого?, - вдруг спросили из зала.

- Кого бы я взял в свою экспедицию? – размышляет Конюхов. –
Человека­-романтика, смелого, любознательного и любопытного. Того, кто по ночам
любит наблюдать Млечный путь, кого вдохновляет небо. Молодые, прессуйте свое
время! Оно так быстро летит! Мне 65 лет, а я хочу еще столько всего сделать!..

...И теперь у меня есть важный автограф в книге "Сила веры". @igg{fbicon.sparkles} 

\iusr{Александр Мячков}
Не надо может в центр ? Он хорош на своём месте. Зачем искушать его.

\iusr{Игорь Караулов}
Живопись кошмарная, конечно. Человек удивительный.

\begin{itemize} % {
\iusr{Vladimir Vishnyakov}
\textbf{Игорь Караулов}, надо вживую глянуть.

\iusr{Игорь Караулов}
\textbf{Vladimir Vishnyakov} ну скажем, он имеет на это право.

\iusr{Vladimir Vishnyakov}
\textbf{Игорь Караулов}, даже здесь есть разные картины. надр посмотреть внимательнее (мне), может на них видно, какой он человек. может, где-то живое прорывается.

\iusr{Тихон Синицын}

Графика у него превосходная. Ну, а в сравнении с тем, что выставляют наши
"актуальные художники" на выставках в Киеве, Питере и Москве - это несомненно
очень круто. Конюхов - мастер путевых зарисовок, картин-путешествий, это всё
более понятно в контексте его личности. И уж точно интересно смотреть, читать и
слушать.

\iusr{Дмитрий Белоногов}
\textbf{Игорь Караулов} вот прямо согласен. А человек-удивительный.

\iusr{Елена Лапшина}
\textbf{Игорь Караулов} Игорь, да это неважно.

\iusr{Тихон Синицын}

Федор Филиппович - гениальный человек. Его очень много и он не умещается просто
в один жанр. Так же как не умещался, например, Макс Волошин, которого до сих
пор ряд искусствоведов не долюбливает. Похожая история с У. Блейком или У.
Моррисом. Они захватывали все возможные територии для творческой энергии. Это
нормально, одно дополняет другое. Открытые цвета Конюхова и обводка контуров
фигур это вполне нормально мне кажется. Образы работают. Видел в разное время
массу композиций.

\end{itemize} % }


\end{itemize} % }
