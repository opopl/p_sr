% vim: keymap=russian-jcukenwin
%%beginhead 
 
%%file 22_06_2020.fb.buzhanskii_max.1.den_pamjati_i_skorbi
%%parent 22_06_2020
 
%%url https://www.facebook.com/permalink.php?story_fbid=1664168443747628&id=100004634650264
 
%%author 
%%author_id buzhanskii_max
%%author_url 
 
%%tags 1941,22_jun,germania,istoria,nacizm,napadenie,pamjat,sssr,ukraina,vojna,vov,vov.22.06.1941
%%title День Памяти и Скорби как раз сегодня
 
%%endhead 
 
\subsubsection{День Памяти и Скорби как раз сегодня}
\label{sec:22_06_2020.fb.buzhanskii_max.1.den_pamjati_i_skorbi}
\Purl{https://www.facebook.com/permalink.php?story_fbid=1664168443747628&id=100004634650264}
\ifcmt
 author_begin
   author_id buzhanskii_max
 author_end
\fi

День Памяти и Скорби как раз сегодня.
Понимаете, в этом то и весь фокус того, что с нами пытались сделать, поменять местами в сознании отношение к 9 мая и 22 июня.
Не получается.
Домой к предкам каждого из вас, в этот день пришла Война.
Не раньше, не позже, именно в этот.
Жизнь целых поколений определилась как до и после.
И тут никакие напоминания не нужны, все прекрасно помнят.
Бабушку свою вспомните, как она про этот день рассказывала, одна и вторая.
--------------------------------------------------
22 июня 1941 года.
До рассвета еще пара часов.
Ночь нежна, зной умер накануне и ещё не родился вновь.
Воздух приятно холодит выбритые до синевы щёки, запах кофе щекочет ноздри.
Тысячи офицеров, одинаково механическим жестом бросают взгляд на часы.
Время.
Никакого возбуждения, даже азарта нет, холодны и спокойны.
Работа.
Бесшумными тенями снуют сзади солдаты, стаскивают с танков маскировочные сетки, вышибают клинья-стопоры из под колес тягачей.
Пора, время.
Из леса выходят бесконечные колонны, угольками тлеют в зубах огоньки сигарет, рукава мундиров закатаны, ничто не должно мешать.
Рокот накрывает неожиданно, будто из ниоткуда возник.
Огромная плотная туша, сотни идущих вместе бомбардировщиков, на секунду зависла над головой, будто замерла, и поползла дальше.
Ещё один короткий взгляд на часы, пора.
Сзади чихнул, заводясь, танковый двигатель, выплюнул в темноту сизое облачко дыма, за ним другой, третий, десятки, сотни.
Взгляд назад, короткий взмах рукой.
Пора!
Бесконечные колонны бесшумно двинулись вперёд, выброшенные щелчком сигареты летят в стороны, умирая в воздухе.
Первые взрывы.
Первый, второй, сплошной ад взрывов, штурмовики с рёвом прошли над головами, засыпая бомбами приграничную полосу.
Один заход, второй, волна за волной, совсем низко, прижавшись к земле, не боясь ничего и никого.
Сзади взревела артиллерия.
Багровые вспышки озаряют ночь, пляшут чудовищные тени, кажется, будто не руки режут колючую проволоку, но оживший кошмар беснуется в ночи.
Никакой больше тишины, рокот тысяч моторов, лязг гусениц, взрывы, хриплое дыхание солдат.
————————————————————
Рокот тысяч моторов, лязг гусениц, взрывы, хриплое дыхание солдат.
Застава в ружьё!!!
Воет сирена, отчаянно пытаясь прорваться сквозь ад бомбардировки!
Надрываются собаки, захлебываются лаем, виснут на привязи, душат себя ошейниками, рвутся навстречу врагу.
Люди мечутся, кто то неслушающимися пальцами пытается застегнуть крючок гимнастерки у самого горла, будто сейчас это самое важное в жизни!
Штурмовики растопырили крылья прямо над головой, поливают огнем.
С того берега бьют пулеметы, бойцы мечутся, путаются среди мертвых и живых.
Кто то прорвался к оружию, лупит в рассвет короткими очередями, по бесконечным, развернувшимся в цепи колоннам.
Что то орет в трубку радист, ничком уткнулся в землю командир, рвутся с привязи овчарки, вцепиться в горло, умереть, но вцепиться.
22 июня 1941 года.
Я не знаю, какой день скорби нужно придумывать.
Это так дико и нелепо, так фальшиво и глупо.
Не было беды страшней, не было скорби чудовищней.
Знаете, всё одновременно просто и сложно.
Нужно знать и чувствовать.
Вот как можно не чувствовать этой вибрации, этой армады бомбардировщиков над головой, этого рёва моторов, этого хрипа овчарок, этого мерного топота сотен тысяч, миллионов сапог?
Как не чувствовать этой дрожи Земли?
Как можно этого не знать?
Как зная и чувствуя можно пороть чушь и бред, позорные и мерзкие?
Побежали???
Это наши то побежали???
Да если бы я слушал эти истории, я бы решил, что немцы победили, что Москва, Ленинград и Сталинград пали, что мы всё проиграли и всё потеряли!
Бежали??
Это кто бежал??
Манштейн, Йодль, Гудериан, Кейтель , криком кричат- они не бегут!!!
Криком кричат, мы с боем берем каждый метр!!
А вы кричите побежали?
Хотите понравится Геббельсу?
Поздно.
Не оценит.
Побежали?
Харьковские заводы вывезли.
Вывезли заводы, вы понимаете, что такое вывезти завод?
Вы понимаете, что такое вывезти завод, когда Рейх навалился всей силой и прёт, прёт безудержной чудовищной машиной??
ХИИТ вывезли.
Не, не просто вывезли.
Он начал работать дальше.
Студенты пошли учиться дальше.
А с заводов пошли танки.
Наши 2 миллиона пленных, это наш заград отряд.
Тот, который встал впереди, и прикрыл собой всех остальных.
Дал время.
Сдались в плен, когда расстреляли все патроны, сожрали всю кору с деревьев и не могли больше стоять на ногах.
Дали время остальным.
Наши пленные, это наша цена, цена за то, чтобы устоять.
Это не рынок, тут не торгуются.
Это подвиг, а не позор, и мне унизительно от того, что приходится объяснять это.
Плохие командиры?
Кто то победил лучше?
Остановились на Волге?
Это для кого басня, а?
А под Москвой остановились, не хотите?
Или не остановились, сдали Столицу, да?
Может быть остановиться на Волге это хуже, чем сбежать за Ла Манш?
Вранье.
Вы им пропитались, вы в нем тонете, убогие рассказчики, не понимая ни смысла, ни сути, и рассчитывая только на одно.
Что вам не возразят.
Возразят, ещё как возразят.
Сталин был чудовищем, но его сын, бросился на колючую проволоку в концлагере, а не встал в очередь за пайком немецкого офицера.
И когда вы рассказываете о первом, но забываете сказать о втором, вы врете.
И это все понимают.
22 июня, день Скорби.
Самой огромной, немыслимой скорби, которую только можно себе представить.
И тут ничего не нужно прививать, внедрять и вводить указами.
Это прекрасно понимает и чувствует каждый человек.
Каждый нормальный человек, естественно.
Наше дело было правым, и мы победили.
Мир смотрел с надеждой, затаив дыхание.
Не подвели.

\ifcmt
  pic https://scontent-lga3-2.xx.fbcdn.net/v/t1.6435-9/104895205_1664168420414297_8089253603208202629_n.jpg?_nc_cat=104&ccb=1-3&_nc_sid=8bfeb9&_nc_ohc=tiTFNgMugmQAX-qIDHZ&tn=ntrKbsW_7ChXu3v-&_nc_ht=scontent-lga3-2.xx&oh=c95c0b15fc0ec96ab0db0b5168016860&oe=60D67DD3
\fi

\begin{itemize}
  \item \url{T.me/MaxBuzhanskiy}
  \item \url{https://www.facebook.com/max.buzhanskiy.1}
\end{itemize}
