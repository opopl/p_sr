% vim: keymap=russian-jcukenwin
%%beginhead 
 
%%file 14_05_2023.stz.news.ua.donbas24.1.u_stolyci_zahysnycja_mrpl_prezent_novi_virshi.txt
%%parent 14_05_2023.stz.news.ua.donbas24.1.u_stolyci_zahysnycja_mrpl_prezent_novi_virshi
 
%%url 
 
%%author_id 
%%date 
 
%%tags 
%%title 
 
%%endhead 

Ольга Демідко (Маріуполь)
14_05_2023.olga_demidko.donbas24.u_stolyci_zahysnycja_mrpl_prezent_novi_virshi
Маріуполь,Україна,Мариуполь,Украина,Mariupol,Валерія Суботіна,Київ,Киев,Kyiv,Kiev,date.14_05_2023

У столиці захисниця Маріуполя презентувала нові вірші (ВІДЕО)

У Києві відбувся творчий вечір, присвячений маріупольчанці Валерії Суботіної

13 травня у Будинку кіно в Києві всі охочі нарешті зустрілися з
маріупольчанкою, поетесою та захисницею Маріуполя Валерією (Навою) Суботіною
(Карпиленко), яка тільки 10 квітня повернулася з полону. Дівчина 10 місяців
перебувала в оточенні ворогів і, оговтавшись, зустрілася зі своїми земляками.
Зустріч організувала мати її загиблого чоловіка Людмила Суботіна. Це був
зворушливий творчий вечір, який зібрав у Червоній залі Будинку кіно друзів,
знайомих, посестер та побратимів Нави.

Читайте також: Історія сталевої Нави — захисниця Маріуполя розповіла про весілля на Азовсталі та російський полон (ВІДЕО)

На заході друзі Валерії і всі, кого вона надихала, виходили на сцену, аби
поділитися життєвими історіями, які їх поєднують з військовослужбовицею, та
почитати її вірші зі збірки «Квіти і зброя». Водночас Валерія презентувала три
власні поезії. Перша була присвячена Маріуполю та маріупольцям, які не могли
говорити, тому що у них не було зв'язку. 

«Росія робила, все, аби велика земля, весь світ не міг почути цих людей. і в
цьому був найбільший цинізм. Вони не просто вбивці, не просто вбивці людей,
вони ще робили все і роблять все, аби світ не знав правду. Аби світ не почув і
не побачив того, що вони роблять. Але світ побачив і почув, в тому числі,
завдяки пресслужбі Азову», — наголосила Валерія.

Другий вірш поетеса присвятила військовим, які наразі перебувають у полоні -
цей вірш вона написала декілька днів назад.

«Ми знали, що нас тут чекають, ми розуміли, що інакше бути не може. Але майже
рік...Майже рік знаходитися серед ворогів, серед людей яких ти ненавидиш і геть
нічого не можеш зробити... Це важко. І тому нам треба робити все, аби поскоріше
їх звідти забрати... Будь ласка, друзі, пам'ятайте про героїв Азовсталі. Будь
ласка, представники ЗМІ, не забувайте про них, продовжуйте про них говорити», —
наголосила Валерія Суботіна.

Читайте також: У Києві відбувся захід на підтримку військовополоненої з
Азовсталі (ВІДЕО)

«Нас усіх об'єднує війна... Серед нас є сім'ї, які дочекалися рідних з полону.
Ось і наша сталева донечка разом із нами, дякувати Богу. Але ніхто із нас не
може спокійно жити, бо дуже багато наших рідних і близьких зараз у полоні. Ми
маємо разом боротися, щоб якомога швидше наші воїни повернулися з полону. І ми
продовжуватимемо боротьбу, бо материнські серця не можуть по-іншому», —
зауважила мати загиблого захисника Маріуполя Людмила Суботіна.

Останній вірш, який представила Валерія, є найбільш важливим для неї, адже він
про коханого чоловіка Андрія, з яким Валерія пробула у шлюбі три дні: 5 травня
2022 року було зареєстровано шлюб, а 7 травня він не повернувся з бойового
завдання. На творчому вечорі жінці подарували картину, присвячену їй та Андрію...

«Не просто пам'ять про нього живе. Він живе. Ті, хто знали мого Андрія знають,
що він завжди робив неймовірні речі для мене. Ця друкарська машинка, яку мені
сьогодні подарували, всі квіти... Я впевнена, що все це робить він вашими руками.
І він зараз поруч. І все те, що зараз відбувається, я впевнена, йому дуже
сильно подобається», — зазначила поетеса.

Читайте також: У Києві презентують проєкт «Портрети Маріуполя» — як потрапити
на виставку

На зустрічі всі присутні могли підійти до Валерії, висловити їй слова подяки за
захист, підтримати дівчину, отримати афтограф та обійняти її. 

Нагадаємо, раніше Донбас24 публікував історії п'яти жінок з Маріуполя, чия
непересічність, самовіддана діяльність і винятковий талант надихають.

Ще більше новин та найактуальніша інформація про Донецьку та Луганську області
в нашому телеграм-каналі Донбас24.

Фото: Наталі Дєдової та з архіву Донбас24
