% vim: keymap=russian-jcukenwin
%%beginhead 
 
%%file 20_04_2021.fb.sokyra_space.1.borsch
%%parent 20_04_2021
 
%%url https://www.facebook.com/sokyra.space/posts/1213007152497895
 
%%author 
%%author_id 
%%author_url 
 
%%tags 
%%title 
 
%%endhead 
\subsection{Бо нині борщ — це теж політика. Борщ наш!}
\label{sec:20_04_2021.fb.sokyra_space.1.borsch}
\Purl{https://www.facebook.com/sokyra.space/posts/1213007152497895}

Спочатку ми поступилися Колобком. Потім без кінця поступалися борщем. А тоді у
нас з кров'ю й сльозами відтяли Крим. Так буває завжди, коли твій сусід Росія.

Спочатку вона поглинає твою культуру, а далі знищує, лишаючи після себе
пустелю.

\ifcmt
  pic https://scontent-dfw5-1.xx.fbcdn.net/v/t1.6435-9/176328408_1213007102497900_4642832177913586624_n.jpg?_nc_cat=1&ccb=1-3&_nc_sid=730e14&_nc_ohc=jfvbp8hqNYEAX-XQa3M&_nc_ht=scontent-dfw5-1.xx&oh=ad2070dd0575bed49ab4ffe0a740b69f&oe=60A6C416
\fi


Сьогодні Google та Ростуризм опублікували топ-10 типу своїх і типу традиційних
страв. І там немає тюрі і щей, зате там БОРЩ. НАШ БОРЩ! Знову вони намагаються
прилаштувати собі те, що їм не належить і ніколи не належало.

Ми надто довго легковажили українською кулінарією, типу це й не культура, так
собі щось там, їдло. Господині варять, по селам щось ліплять ну і що в тому
такого? А все!!! Кулінарія — це величезний і надзвичайно важливий пласт
культури. У ній не лише капусточка з картопелькою — у ній, якраз, відображення
національної ідентичності, ходу історії і справжня культура без ураження рясним
шароваризмом. 

В Києві ще три роки тому ресторанів італійської кухні було набагато більше, ніж
ресторанів, які б нормально і смачно готували вітчизняні страви. Українська
кухня відкидалася, як надто проста, надто селянська. 30 років незалежності, ми
тільки те й робили, що легковажили цим. Борщ був немодним. Борщ був несучасним.
Борщ був неважливим. Але головне, що борщ був. Він в усі часи, за всяких
обставин лишався на наших столах. 

А що робили наші сусіди? Додавали в борщ якогось говна і вперто на світовому
культурному тлі вказували російськість борщу. Варили якусь синю гливку убогість
і звали то «варєнікамі». Так і йде культурна, чи то пак, кулінарна експансія. А
потім, треба порно-кулінарній блогерці доводити, що борщ, то зовсім не Росія. А
потім нам на кожному кроці все ще доводиться пояснювати, що ми — не Росія.

Я аплодую стоячи Клопотенку, який нині робить те, що слід було б зробити ще на
початку 90-х років, після проголошення Незалежності. Він зараз не просто зібрав
величезний стос документів і оббив сотні порогів, щоб добитися врешті подачі
заявки на внесення борщу до списку кулінарної спадщини ЮНЕСКО. Він для світу
розставляє кулінарні, а отже і культурні маркери, які всім на всіх континентах
скажуть — БОРЩ НАШ!!!

Білий, червоний, зелений, жовтий. З яблуками, з вишнями, з грушами, сливами і
навіть полуницями. З буряком і без буряку, з білим буряком. З квасом,
виготовлення якого більше схоже на магічний ритуал. З м'ясом, з рибою або
бобами. Все це — наш український борщ. Борщ, який має всі шанси пояснити світу
нашу самобутність, нашу національну важливість у світовому контексті. 

Не легковажте борщем. Дозвольте йому бути різним. Дозвольте йому бути скрізь.
Дозвольте йому бути навіть з ананасами чи папайєю, якщо це допоможе рознести
всьому світові, що ми народ, ми нація.  Наша культура сягає глибоко у віки. Ми
не Росія і навіть тоді, коли вона з нас випивала кров, ми не стали і не були
нею.

Не кожна країна, не кожен народ має такий культурний козир в кулінарії, не всім
так пощастило у віках, окупації та нищенні культури вберегти справжні перли. То
якого дідька ми маємо дарувати ворогові свій безцінний спадок. Немає ніякого
російського борщу — бо Росія ніколи не вміла навіть нормально зберегти те, що
вкрала.

Тож, їжте борщ. Варіть борщ за своїми смаками і вподобаннями. Нехай він буде
різним. Нехай він буде по-полтавськи густим, нехай буде по-вінницьки солодким,
по-бесарабськи оксамитовим, по-запорізьки гострим, по-львівськи ніжним. Нехай
він просто буде! 

Бо нині борщ — це теж політика.

Борщ наш!

І пам’ятай, кожного разу, коли ти зневажаєш борщ, десь у мисці проступає лик найогиднішого з росіян!

Татуся Бо, спікерка Демократичної Сокири з питань освіти та борщу, а ще
колекціонерка рецептів борщу та вареників (колекція налічує понад сотню
рецептів і постійно поповнюється).
