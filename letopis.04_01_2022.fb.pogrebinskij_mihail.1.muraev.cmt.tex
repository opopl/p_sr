% vim: keymap=russian-jcukenwin
%%beginhead 
 
%%file 04_01_2022.fb.pogrebinskij_mihail.1.muraev.cmt
%%parent 04_01_2022.fb.pogrebinskij_mihail.1.muraev
 
%%url 
 
%%author_id 
%%date 
 
%%tags 
%%title 
 
%%endhead 
\zzSecCmt

\begin{itemize} % {
\iusr{Вячеслав Беленький}

Давайте уж откровенно. Признание Москвой Порошенко в 2014 году было с подачи Медведчука.
Это очевидно.
И является самой главной ошибкой-подставой.
Выставление заведомо непроходного Бойко на президентских - так же. Как минимум - с согласия.
Если и был какой хитрый план, то он с треском провалился.
Или так и Должен был провалиться?
Вопросы, вопросы...
Откол Мураева от ОПЗЖ был предсказуем.
Попытка атаковать его в информ поле не принесёт голосов ОПЗЖ, как выпады Евгения в адрес Шария не несут голосов "НАШим"
Пустой раздрай, отбивающий охоту электората идти на выборы вообще.

\begin{itemize} % {
\iusr{Михаил Погребинский}
\textbf{Вячеслав Беленький} это Вам Медведчук сказал, что с его подачи? Ну Вы, Вячеслав, вроде, разумный человек, но берётесь судить о вещах, в которых ни х-ра не понимаете. Зачем?

\iusr{Вячеслав Беленький}
\textbf{Михаил Погребинский}
Свечку не держал, но много лет наблюдений в основании.
У Москвы не было, да и нет, более весомых знакомых фигур на месте, которые могут выступить консультантом в столь фундаментальном шаге.
А без консультации по местному пасьянсу такие решения не принимается.
24 июня 2014
Официальный сайт Кремля сообщил о посреднической роли Виктора Медведчука в урегулировании конфликта на востоке Украины.
Представитель Украины в Трёхсторонней контактной группе.
Но это уже неважно - прошлое.
Вопрос, что ОПЗЖ собирается делать дальше.
Пока, кроме эпатажа Кивы - все по кустам.

\iusr{Mariya Romanova}

Всё верно – именно опзж псевдо-озициция, слившая реальную оппозицию. Сейчас,
судя по всему, заказали Мураева, потому что он будет выдвигаться на следующих
выборах. Но, не получится

\iusr{Татьяна Жмуркова}
\textbf{Вячеслав Беленький} 

правы 100\%. Чем больше будут \enquote{мочить} друг друга ОПЗЖ и НАШИ, тем выгоднее для
Порошенко, Тимошенко и др. Можно представить их реакцию, если бы выступили
единым фронтом. Сор из избы не выносили бы.

\iusr{Mariya Romanova}
\textbf{Татьяна Жмуркова} опзж были всегда в паре с рошеном

\iusr{Ирина Попова}
После того, как Мураев обнимался с Арестовичем, с ним все стало ясно

\iusr{Андрей Ковалев}
\textbf{Ирина Попова} И что Вам ясно?

\iusr{Андрей Ковалев}
\textbf{Вячеслав Беленький} 

Я все-таки полагаю, что решение о признании нового президента Украины, со
стороны Кремля было частью закулисной сделки с Западом после переворота. В
последствии оформленной в качестве Минских соглашений. Именно их то Порошенко и
подписал. Подписал но не выполнил.


\iusr{Ирина Попова}
\textbf{Андрей Ковалев} что он такой же переобувалка, как все там

\iusr{Нина Евсеева}
\textbf{Ирина Попова} Мураев с Арестовичем знакомы 10 лет, что не мешало Мураеву открыто в эфирах критиковать Арестовича. А вы наверное любите зрелища - мордобои в эфире, ругань и брызганье слюной? )

\iusr{Наталия Карпенко}
\textbf{Вячеслав Беленький} согласна в части не проходного Бойко а также Вилкула который оттяпал 5\% от опзж пусть внутри разберутся

\iusr{Женя Демакова}
\textbf{Нина Евсеева} , как Вы наивны ........ Лет-то Вам сколько ?☹️

\iusr{Alex Verber}

Москва признала Порошенко, п.ч. он обещал закончить войну. Не помню срок, но
очень быстро.

И обманул. Славяне, это очень доверчивая народность. Большой процент
душевности, поэтому очень хочется кому-то верить. Вот, нынче, Мураеву всё еще
верят, не смотря на фу-проколы.

\iusr{Надежда Бугаенко}
\textbf{Вячеслав Беленький} а что ОПЗЖ делала все эти 7 лет? Как оппозиция она никакая. Ноль. Они даже сами себя не могут защитить.

\iusr{Mariya Romanova}
\textbf{Alex Verber} 

А углём, сжиженным газом и дизелем по Прикарпатзападтранс с Порошенко тоже
Мураев))? Или, всё-таки, товарищ Медведчук)? И почему господин Погребинский не
говорит, что он его политтехнолог))?

\iusr{Михаил Бельтюков}
\textbf{Вячеслав Беленький} Какая ошибка?) чем был не выгоден парашка из вне?)
Да и кто у выиграше оказался?)

\iusr{Семен Борисенко}
\textbf{Вячеслав Беленький} 

Вы лишний раз подтверждаете тот факт, что на Украине даже политики с очень.
близкими позициями и взглядами не способны договариваться, сближать позиции,
выступать единым строем. Москва в 1914 г. не была готова к полному разрыву с
Украиной, потому была вынуждена признать Порошенко. И при всей моей симпатии к
Мураеву было совершенно очевидно, что у него тогда не было шансов. Наиболее
сильной оппозиционной партией тогда была ОПЗЖ, а её лидером был Юрий Бойко. И
этим всё объясняется...

\iusr{Раиса Френкель}
\textbf{Михаил Погребинский} Михаил Борисович, Гордон на "Нашем" это плохо, а Гордон в 2018-19 гг.на " 112" это хорошо?

\iusr{Eldar Atakulov}
\textbf{Вячеслав Беленький} Свечку не держал, но осуждаю. )) Аха-ха... Нормальная логика.

\iusr{Владимир Алексеев}
\textbf{Вячеслав Беленький} Думаю, здесь главную роль сыграл Зурабов.

\end{itemize} % }

\iusr{Валерий Дейнека}

Как бы то ни было, Мураев говорит по делу, и у его партии правильная программа.
На парламентских выборах, если пройдут, будут неизбежно блокироваться с ОПЗЖ.
На президенских, если его рейтинг будет ниже кандидата от ОПЗЖ, Мураев должен
сняться в его пользу. И это реально сломает \enquote{хитрый план} ОПЗе. Не снимется,
значит, грош ему цена и голосовать за Мураева бессмысленно.

\begin{itemize} % {
\iusr{Ольга Арканова-Чёрная}
\textbf{Валерий Дейнека} голосовать за этого скользкого мурчащего, вообще бессмысленно изначально, если не играть с ним заодно

\iusr{Михаил Погребинский}
\textbf{Ольга Арканова-Чёрная} Вы правы, увы не все это понимают

\iusr{Tanya Michaylova}
\textbf{Валерий Дейнека}, на зелень тоже уповали. Правильные вещи говорил, да только ничего не сделал.

\iusr{Раиса Френкель}
\textbf{Ольга Арканова-Чёрная} Почему бессмысленно? У него правильные взгляды и на политику и на жизнь! Он любит нашу страну и СССР!

\iusr{Павел Павел}
\textbf{Валерий Дейнека} между тем, что говорят и что будут делать может быть пропасть. Мураев это часть некого проекта.

\iusr{Сергей Костенко}
\textbf{Валерий Дейнека} между говорить и сделать- в основном Марианская впадина ☺ ️  @igg{fbicon.face.grinning.smiling.eyes}  @igg{fbicon.wink} 

\iusr{Роман Котенко}
\textbf{Валерий Дейнека} 

В 2019г. на президентских выборах я и подобные мне благодушные дураки тоже
ждали, что Мураев снимется в пользу Бойко (пусть даже в последний момент). Т.е.
я и подобные мне до последнего надеялись. Более дальновидные, например, тот же
Михаил Погребинский знали цену Мураеву, а главное, тому, кто за ним стоит,
несколько раньше. Но на дворе уже 2022-ой. Как можно продолжать питать иллюзии
относительно Ахметова и его проектов?

\iusr{Ольга Арканова-Чёрная}

Он любит только себя, на Вас, Родину и СССР ему глубоко наплевать, а взгляды
свои он будет преподносить так, как Вам нравится, но с целью заработать очки у
зелёных и наличку у Ахметова. Неужели, Вам мало «правильных» взглядов
сегодняшнего гаранта?

\iusr{Елена Гилберт}
\textbf{Павел Петрашов} 

а как вы можете узнать, что Мураев будет делать, или вы знали программу Зе
команды? Да у Зе команды была программа играть на сцене, делать видосики и
выполнять приказы Коломойши и партнеров. Это было ясно и понятно, мне во всяком
случае, а народ голосовавший за него 73\% почему-то не видел и не понимал, что
у него нет планов, а только одна болтовня, он никто, он клоун, он паяц, ни
одного дня нигде не проработавший, и даже диплом у него фейковый, он его
получил играя в КВН, он не учился даже, но народ повелся на слова этого клоуна.
У Мураева есть образование, есть опыт работы руководителем, руководил районом в
Харьковской области, есть опыт работы в политике. А теперь кто-то может сказать
о другом политике, который может взять на себя ответственность в такое время и
вывести страну из тупика, в который завёл страну Порошенко и Зе? Мураев не
сумасшедший, чтобы взять на себя разваленную страну и продолжить ее
разваливать, то что сделал Зе, за что его народ и возненавидел, и готов порвать
его на мелкие кусочки. Так что мое мнение вот такое, и я за Мураева. И имею
право высказать своё мнение и привести свои аргументы в его поддержку, а то что
он мурчит или кричит, я не хочу обсуждать, я слушаю, что он говорит, его
позицию и у него есть команда, которая его поддерживает. И ещё не корректно
говорить о том, что Мураев забивает мозги журналистам из его команды. Я считаю
у каждого журналиста есть свои мозги, и они ими думают. Вот те кто думает и
доверяет Мураева тот и с ним, в его команде.

\iusr{Инесса Команова}
\textbf{Валерий Дейнека}

мечты-мечты, где ваша сладость..
Мураев-балласт. И уже давно. Говорящая игрушка в руках ЗЕ. Не более.

\iusr{Игорь Гудков}
\textbf{Инесса Команова}

мураев уже был известный политик, когда Зе-клманды и партии \enquote{Слуга народа} еще
не существовали.

\iusr{Павел Павел}
\textbf{Елена Гилберт} 

голосовали не за Зе, а против Порошенко. Знать что Мураев будет делать мне
неизвестно, чтобы он не говорил, но то что он участвует в проекте против ОПЗЖ
оттягиванием голосов это ребенку понятно. Чтобы делать политику надо опираться
на какой нить ресурс: то ли на росс., как ОПЗЖ, то ли на американский как
Порошенко, то ли на олигархов, которые все одно под США, как Зе. На кого
расчитывает Мураев, если что!?

\iusr{Sergey Ozhegov}
\textbf{Валерий Дейнека} 

Мураев трудится на власть. Это очевидно.
У него индульгенция на любую критику Зеленского, слуг, США и проч.
Потому и канал его работает, в отличие от уже 5 закрытых каналов ОПЗЖ.
Он не будет объединяться с оппозицией.
Верить в это могут лишь адепты этой секты.

\iusr{Pankratova Nadiy}

Мураев дальновидный, адекватный политик, но умные у нас сейчас не в моде или их
боятся и поливают грязью.

\iusr{Инесса Команова}
\textbf{Игорь Гудков}

Вы правы. Он политик со стажем. Умный политик, постоянно мечущийся в надежде на
лидерство. Патриотизма недостает.....

\iusr{Алексей Владимирович}
\textbf{Валерий Дейнека} Они никогда не пройдут

\iusr{Олег Протасов}
\textbf{Валерий Дейнека} 

все это многосерийный спектакль для гоев: \enquote{Выборы без выбора!}
Как говорят раввины: \enquote{Какая разница кто победит на выборах?! При власти будут все равно только наши люди!}
И так и есть... Остальное - шелуха...

\iusr{Мила Подлесная}
\textbf{Валерий Дейнека} 

Не снимется. Конфронтация с некоторыми представителями непреодолима, можно было
ранее сделать вывод при расколе \enquote{За жизнь}.

\iusr{Мила Подлесная}
\textbf{Роман Котенко} \enquote{Снимется в пользу Бойко}? Или Левочкина?)

\iusr{Владимир Медвидь}
\textbf{Валерий Дейнека} говорит ! Катастрофические последствия его лицемерия Украина уже видела на президентских и парламентских выборах. Позор.

\iusr{Владимир Медвидь}
\textbf{Роман Котенко} Роман Респект

\end{itemize} % }

\iusr{Нина Ворушило}

Да, если бы Мураев тогда согласился на предложения Медведчука, то Порошенко
проиграл бы ОП, но у Мураева, очевидно, другая задача. Красиво болтать, не
значит не совершать подлости. А он поступил, как предатель, помог выиграть
Порошенко, вот и весь его расклад.

\begin{itemize} % {
\iusr{Нина Евсеева}
\textbf{Нина Ворушило} А какие у Медведчука были предложения? Иди на фик, я тут главный? )) Партию ЗА Жизнь разве не Мураев с Рабиновичем создавали? Не Медведчук ли сделал это раскол по подлому?

\iusr{Александр Конарев}
\textbf{Нина Евсеева} .

Мураев сам говорил, что после вхождения Виктора Медведчука в партию за Жизнь,
ему предлагали быть Премьер-Министром от партии за Жизнь, но Мураев отказался !


\iusr{Ольга Подщанская}
\textbf{Нина Ворушило} . Сколько можно эту дурь писать?

\iusr{Ольга Подщанская}
\textbf{Александр Конарев} , не смешите. Что это пародийная должность? У вас совести и чести вообще нет!

\iusr{Александр Конарев}
\textbf{Ольга Подщанская} .
Приведите пример, что у меня нет чести или совести .
Где и когда и в чём и кого я обманул или подвёл ?
Требую Ваших извинений !
Кстати в моей ленте рекламируются 10-тки видео-роликов Мураева и пока я не принял решение кому доверю свой голос, поскольку в нашем споре должна выкристализоваться истина .

\iusr{Ольга Подщанская}
\textbf{Александр Конарев} . 

У Мураева Медведчук отжал канал, по его наводке Россия внесла Мураева в
санкционный список и арестовала активы семьи, а вы все здесь ноете об
объединении. Погребинский - холуй Медведчука такую грязь и враньё льёт на
Мураева - это точно для объединения?!? Поэтому нет у вас чести, если вы
требуете объединения с бесчестными людьми. Мураев никогда не позволял себе
такого хамства по отношению к ОПЗЖ, всегда защищал закрытые каналы, много раз
говорил о неправомерности ареста Медведчука. А что в ответ? Циничное враньё и
сплошные оскорбления!

\iusr{Александр Конарев}
\textbf{Ольга Подщанская} .

1. Я не писал, что ОПЗЖ и Мураев должны объединиться - так что Вы пишете, мягко
говоря НЕ ПРАВДУ.

2. Кто у кого что отжал - мне это не известно, поскольку я не видел подлинных
документов и какие у кого права на канал они должны были выяснять в суде, тем
более, что была информация, что канал принадлежал другому депутату, а не
Мураеву, а Вам это известно или снова не правда ?

3. Если Мураев внесён в санкционный список России, то где Мураев будет брать
300 млрд.\$ денег на развитие Украины под свою Программу на ближайшие 10 лет,
учитывая что США и Европа за 30, а особенно за последние 7 лет не вложили в
развитие экономики Украины ни одного миллиарда, а в Афганистан США вложили 2
триллиона \$ за 20 лет и все в войну, а не в развитие промышленности
Афганистана.

Так откуда деньги у Мураева под его Программу ? Программа - это Блеф ?

Т.е. опять Ваша не правда.

И Вы мне не дали ответ в чём проявилось отсутствие у меня совести !

Вы пишете, что все и в том числе я ною об объединении Мураева и Медведчука, а я
вообще не писал об их объединении, а писал, что Мураев заявлял, что Медведчук
предлагал ему из партии не уходить, а после победных выборов стать
Премьер-Министром Украины по представлению депутатов от партии \enquote{За ЖИТТЯ}. Т.е.
Ваше утверждение, что я ною об объединении - тоже Ваша ЛОЖЬ ?

Мураев за дружбу и экономические отношения со всеми соседями, в т.ч. с Россией,
а Вы, как сторонница Мураева, за дружбу с Россией ?

И я нигде не утверждал, что я против Мураева, но ряд его негативных поступков и
Глобальных проектов требуют разъяснений с его стороны.

К ОПЗЖ и к ряду их руководителей на местах у меня тоже есть вопросы.

Желаю Вам Здравия !

Жду ответа и Ваших извинений, за то, что Вы обвинили меня без оснований.

Или Вы не обучены этике ?

\end{itemize} % }

\iusr{Наталия Сафонова}

Очень точная характеристика ЭТОГО... явления!! Знает на чем играть и, к
сожалению, \enquote{играет} очень многих!

\iusr{Ferik Mur}

Реальная оппозиция опасна и её уничтожают, Олесь Бузина до последнего дня думал
не посмеют, не решатся. А вот когда сказал публично: \enquote{За шо стояли на Майдане
...} Раздались сразу выстрелы. Это Рефлексия. У Вас сейчас самый востребованный
товар - это оппозиционер который огрызается, говорит умные вещи и его можно
погладить по головке. Он безопасен и берет с руки все что дают с барского
стола. Народу кроме зрелищ надо подарить иллюзию выбора. И они дарят. Самый
главный вопрос Украины почему к сожалению во главе государства всегда
становятся претенденты с гнильцой, не одного приличного безупречного человека
за все эти годы.

\begin{itemize} % {
\iusr{Sergio Moscalenko}
\textbf{Ferik Mur} сегодня на Украине по факту оппозиции нет

\iusr{Наталья Примакова}
\textbf{Ferik Mur} 

потому что, даже только подать документы на выбры, как кандидата, надо
официально заплатить государству около миллиона грн, а если вместе с остальными
расходами, то там десятки миллионов \$. Круг замкнулся(( Кто из порядочных это
потянет? Да никто! Савченко Надя обьясняла когда то, почему не стала
кандидатом, ей просто негде взять столько миллионов! Так что, все кто смог
пройти этот отбор, реальные воры, мошенники или продажные подстилки...

\iusr{Ferik Mur}
\textbf{Наталья Примакова} 

Я думаю если Михаил Погребинский снимет с головы шляпу и как Садко из русской
сказки попросит на благое дело у народа, то по 5 гривен ему эту сумму соберут
за несколько месяцев. Другое дело это не безопасно. Кандидат это прежде всего
Глава ОПГ и должен быть в авторитете. Или как мэр Кличко иметь руки как
кувалды.

\iusr{Наталья Примакова}
\textbf{Ferik Mur} да, именно, вы сами ответили на вопрос, почему же у руля нет честных и порядочных!

\iusr{Александр Тищенко}
\textbf{Ferik Mur}

\enquote{претенденты с гнильцой}

О то ж великая закадка.

Да кто ж ему без \enquote{гнильцы} то там быть позволит.

Да и самое главное в этой связи решается не в у-р-не.

\iusr{Лидия Литвинова}
\textbf{Ferik Mur} Был один.

\iusr{Наталия Горбунова}

А мне кажется -.он троянский конь и этот пост Погребинского замечательного -
только подтверждение. Ой, зря, наверное, вслух написала)

\iusr{Виктор Павлухин}
\textbf{Ferik Mur} 

Увы, Михаил Борисович реалист и он туда так не пойдёт, хотя деньги бы собрали и
быстро и далеко не по 5 грн. Только тут вопрос а нужно ли вновь по такому
порочному кругу двигаться, ибо все наши ОПГ это шентропа шелудивая, которую
взрастила главная ОПГ планеты, которая находится за океаном, а руки у неё
оказались очень длинными. Они уже и НАБУ нам организовали и носками их
дорогущими до 400 грн за пару снабдили и миллиарды на них потратили, а
криминально-воровское сообщество - власть как процветало, так и продолжает
процветать. Государственность убита, негодяи пануют, пора самоликвидироваться,
ибо бочка не бездонна.

\iusr{Алла Пашанова}
\textbf{Наталья Примакова} Савченко в президенты? Хорошее сравнение. Без образования. Да ещё и убийца.

\iusr{Наталья Примакова}
\textbf{Алла Пашанова} 

Наде́жда Ви́кторовна Са́вченко — украинский государственный и политический
деятель. Герой Украины. Депутат Верховной Рады Украины VIII созыва с 27 ноября
2014 по 29 августа 2019. Член постоянной делегации Украины в ПАСЕ.

Звание: капитан ВСУ (в запасе)

Образование: Харьковский национальный университет Воздушных Сил имени Ивана
Кожедуба.

Не пишите гадости на достойнешего гражданина украины!


\iusr{Владимир Алексеев}
\textbf{Ferik Mur} 

Это вполне закономерно. И эту ситуацию еще 90 лет назад гениально предвидел
польский мыслитель, политик и государственный деятель Роман Дмовский - один из
отцов возрожденной Польши. В своей работе 1930 года он разложил по полочкам все
прелести от реализации как тогда говорили, \enquote{украинского проекта}:

"Впрочем, найдутся лица, готовые заняться практической реализацией «украинского
проекта», но здесь начинается трагедия. Нет такой силы, которая предотвратила
бы превращение оторванной от России и ставшей независимым государством Украины
в прибежище аферистов всего мира, которым тесно у себя на родине: капиталистов,
искателей денег..., спекулянтов, интриганов и организаторов разного рода
проституции, включая немцев, французов, бельгийцев, итальянцев, англичан и
американцев. Им поспешили бы на помощь местные и соседи в лице русских,
поляков, армян и, наконец, самых многочисленных тут евреев...

Все эти элементы при участии самых пронырливых из украинцев... образовали бы
элиту страны. Но это была бы весьма специфическая элита, ибо ни одно другое
государство не могло бы похвастаться столь богатой коллекцией международных
каналий..."

"Украина стала бы язвой на теле Европы; люди, мечтающие о культурном, здоровом
и сильном украинском народе..., убедились бы в том, что вместо собственного
государства они получили некое международное предприятие, а вместо здорового
развития – прогрессирующий распад и гниение...».

Роман Дмовский, польский политик и мыслитель. \enquote{Украинский вопрос}.

Dmowski R. Swiat powojenny i Polska. Wydanie drugie. 1930 год.

\iusr{Зинаида Елькова}
А кого же убила Надя, вы в своём уме?!

\end{itemize} % }

\iusr{Евгений Гарец}

Браво, мои мысли вслух и судя по коментам, я, не один, значит будем жить. Всем
удачи.

\iusr{Анна Мисюк}
Я абсолютно согласна с каждым словом!

\iusr{Elen Obelec}
Волк в овечьей шкуре

\iusr{Сергей Соболь}

® К Вам, как к политологу с большим опытом, вопрос:

Неужели Вы реально считаете, что есть хоть малейшие шансы выиграть
президентские выборы у любого представителя от оппозиции?.( Бойко, Медведчук,
Мураев и т д., не важно).

Я конечно же не эксперт в этих играх, но хорошо помню, что даже когда был Крым
и Донбас голоса избирателей примерно составляли 50×50.

А сегодня, когда в добавок уехало из Украины столько народу, какие ещё
результаты могут быть у оппозиции?.

\begin{itemize} % {
\iusr{Надежда Бугаенко}
\textbf{Сергей Соболь}

хочу добавить, что и оппозиция-то вся переругалась между собой. Единства нет. Голоса избирателей дробятся.

\iusr{Сергей Соболь}
® \textbf{Надежда Бугаенко} , это уже не важно(от слова совсем).

Даже если вся так называемая оппозиция возмётся за руки, для решения
украинского вопроса это не поможет.

Точка не возврата пройдена. Любого представителя юго-восточной части Украины
западная не примет. Причём если условный Восток толерантный, то Запад
агрессивный и не примиримый.

Какой выход?.

Лично моё мнение, это только федеративное устройство страны. Причём это тоже не
гарантирует свет в конце тоннеля. Но для меня это реальная НАДЕЖДА при жизни
увидеть хоть какие-то перспективы на развитие и будущее.

В этом варианте есть шансы проявить себя хоть Медведчуку, хоть Мураеву ( в
принципе не важно, там уже кого выберут люди).

Другие варианты, это демагогия, иллюзии прцесса...  @igg{fbicon.thinking.face} 

\iusr{раиса степанова}
\textbf{Сергей Соболь} 

Если Медведчук попридержит своих информационных киллеров типа Погребинского, то
у Мураева шанс появится.

\iusr{Александр Конарев}
\textbf{Сергей Соболь} .

55 на 45, но с Западной в Европу уехало больше, поэтому возможно сейчас 50 на
50, но для этого бедные и слабые люди должны прийти на избирательные участки !

\iusr{Надежда Бугаенко}
\textbf{Сергей Соболь} согласна на 100\%. Ещё Вячеслав Черновол об этом говорил.

\end{itemize} % }

\iusr{Галина Бридковская}

С 2013 года не хожу на выборы. Из политиков не верю никому, чтобы кто не
говорил. Всё решается за лужей.

\iusr{Женя Демакова}

Это УЖАСНО ..........Нами опять манипулируют. Спасибо Вам, Михаил, правда !
Вам - ВЕРЮ. У меня возникали некоторые сомнения и несоответствие речей его и
ведущих. Диссонанс в оценках происходящего. Чем заинтересовал .....Изменением
построения государства на парламентскую 2-х палатную республику..... КУПИЛАСЬ !.
Думаю, не я одна. НЕ сотвори себе кумира........ @igg{fbicon.face.confused} 

\iusr{Евгений Атапин}
Полностью согласен !

\iusr{Кира Берестенко}
Какой-то вечный мальчик - даже видом своим говорит о своей несостоятельности.

\begin{itemize} % {
\iusr{Раиса Френкель}
\textbf{Кира Берестенко} Занимается спортом, не злобствует, не делает гадости - вот и выглядит хорошо! Настоящий Мужчина!

\iusr{Кира Берестенко}
\textbf{Раиса Френкель} инфантильность всегда приветствовалась нашими женщинами  @igg{fbicon.smile} 

\iusr{Нина Евсеева}
\textbf{Кира Берестенко} А не сходить ли вам в гугл и почитать об инфантильных мужчинах. А то уже совсем глупость пишете. И выставляете себя смешной на всеобщее обозрение.

\iusr{Раиса Френкель}
\textbf{Кира Берестенко} Каким-то женщинам возможно нравится инфантильность, но не мне. Люблю настоящих мужественных мужчин, войнов, но обязательно интеллектуалов, умных. Мураев - галантный, культурный!

\iusr{Лидия Литвинова}
\textbf{Раиса Френкель} Ой, не смешите. Мальчик из Змиева.

\iusr{Раиса Френкель}
\textbf{Лидия Литвинова} А что плохого в Змиеве?

\iusr{Кира Берестенко}
\textbf{Лидия Литвинова} е- мамочкам нравится  @igg{fbicon.smile} 

\iusr{Pankratova Nadiy}
\textbf{Кира Берестенко} А вам, важнее вид или умственные способности. если человек молодо выглядит, что. в этом плохого.

\iusr{Кира Берестенко}
\textbf{Pankratova Nadiy} ничего не поняли, но пройти мимо не смогли.  @igg{fbicon.smile} 
\end{itemize} % }

\iusr{Кира Берестенко}
А где сейчас Вилкул? Успешно отгрыз на выборах свои 5\% у Бойко и все - миссия выполнена?

\begin{itemize} % {
\iusr{Luda Milo}
\textbf{Кира Берестенко} 

а я никак не могла понять почему он перед самыми выборами снял свою кандидатуру
в пользу Вилкула, хотя он был гор

\iusr{Luda Milo}
гораздо более популярен, чем Вилкул.

\iusr{Кира Берестенко}
\textbf{Luda Milo} и те, кто тогда топил за Вилкула, сейчас топят за Мураева.

\iusr{Наталия Карпенко}
\textbf{Кира Берестенко} да все они работают на власть чтоб их не трогали

\iusr{Елена Хорошевская}
\textbf{Luda Milo} 

думаю, что Вилкул больше финансов вложил в выборы. Поэтому сложилась связка
Вилку президент, Мураев премьер-министр. Как-то так.

\end{itemize} % }

\iusr{Вадим Романенко}

Михаил Борисович, с большим уважением и вниманием читаю все ваши оценки. В этом
случае - даже 300 Евгениев Мураевых с телеканалами, не отгрызут столько голосов
у опзж, сколько один Кива, находящийся в кадре возле Виктора Медведчука.

\iusr{Вадим Колесниченко}

Вот именно ...(

\iusr{Татьяна Чугаенко}

Да! Для меня это абсолютно очевидно, и уже давно. Странно, что многие до сих
пор не понимают

\iusr{Владимир Очеретный}

Какая разница - Мураев, не Мураев. У Украины в её нынешнем состоянии в принципе
нет хорошей программы - кого ни избери. Хоть Медведчука, хоть ещё кого. Украина
разделена сама в себе на взаимоуничтожающие части. Пока не отделятся друг от
друга, ничего путного не будет.

\end{itemize} % }
