% vim: keymap=russian-jcukenwin
%%beginhead 
 
%%file 08_12_2021.fb.lesev_igor.1.strategicheskaja_pobeda_zelenskii.cmt
%%parent 08_12_2021.fb.lesev_igor.1.strategicheskaja_pobeda_zelenskii
 
%%url 
 
%%author_id 
%%date 
 
%%tags 
%%title 
 
%%endhead 
\subsubsection{Коментарі}

\begin{itemize} % {
\iusr{Максим Гольдарб}

\ifcmt
  ig https://i2.paste.pics/bef50a1dc0e5522a2cd251c2ac02b32e.png
  @width 0.4
\fi

\iusr{Энрике Менендес}

Всё верно с одной поправкой - ЛДНР это супер-бедное продолжение Ростовской
области. Без какой-либо экономической интеграции, кроме трудово-мигрантской (((
Получил паспорт - ищи работу в РФ за з/п меньше чем у россиянина первого сорта
раза в три

\begin{itemize} % {
\iusr{Игорь Лесев}
\textbf{Энрике Менендес} знаю не понаслышке, какая там жопа. Но эта жопа теперь там и очень сильно вряд ли, что когда-то будет здесь. Это их теперь уже проблемы, их жизнь и они сами будут планировать, что и как там обустраивать.

\iusr{Сергий Федорынчык}
\textbf{Энрике Анатольевич Менендес} Тут якісь дивні у Вас нотки - майже русофобські! На просторах Сибіру, Далекого Сходу та Крайньої Півночі трудящих донбасівців чекає багато робочих місць з високими зарплатами. І ніхто не набридатиме українською мовою!

\iusr{Энрике Менендес}
\textbf{Сергий Федорынчык} Я не живу в чёрно-белом мире. Работа ждёт везде. Но это не дома

\iusr{Сергий Федорынчык}
\textbf{Энрике Анатольевич Менендес} Рідний Дім - це справді висока вартість, як і Мир. Але чи не варто певним жителям Донбасу заради цих вартостей визнати, що вони припустились деяких серйозних помилок, зокрема - брали участь у незаконних референдумах?

\iusr{Алексей Иванов}
\textbf{Сергий Федорынчык} да в 1991 был незаконный референдум. И это была ошибка.

\iusr{Алексей Иванов}
\textbf{Сергий Федорынчык} люди бегавшие с флагами , значками и ленточками с символикой ЕС , могут свалить в ЕС.

\iusr{Сергий Федорынчык}
\textbf{Alexey Ivanov} Ви нагадуєте даму, яка вважала, що увесь полк крокує неправильно, крім її сина. Референдум 1991 р був визнаний всіма країнами світу, а референдуми в Донецькій та Луганській областях - ні.

\iusr{Елена Джабраилова}
\textbf{Игорь Лесев} но в этой жопе минпенсия на 1 тыс грн больше с 1 января и жкх в 5-10 раз ниже, чем в европейской Украине....Да и печи доменные начинают задувать)) Так надо разобраться какая жопа больше

\iusr{Алексей Иванов}
\textbf{Энрике Менендес} тут нужно признать один факт. До 2020 года в России
были железно уверенны, что Донбасс нужно будет отдавать, сейчас или потом, но
отдавать. По этому считали, что развивать и как то вкладываться, как в Крым не
нужно. Теперь пришло осознание, что таки ЛДНР это уже де -факто часть России.
По этому года через три посмотрим. И да, расчитывать Украине на ЛДНР не стоит.

\iusr{Юлия Коломойцева}
\textbf{Энрике Менендес} , не беднее Таганрога, как оказалось. Ростовская область тоже разная.
(Деградацию экономики на неподконтрольных Украине территориях я не отрицаю.)

\iusr{Юлия Коломойцева}
\textbf{Сергий Федорынчык} , просто с некоторых пор государство Украина перестало быть родным для этих территорий, поэтому не нужно обвинять жителей в этих проблемах, явно не они их создали на референдумах. Референдум был в мае, а отряды "добровольцев" начали собирать на майдане в марте-апреле.

\iusr{Юлия Коломойцева}
\textbf{Сергий Федорынчык} , "весь світ" всегда признает то, что ему выгодно, и не признает то, что ему невыгодно. Вся легитимизация.

\iusr{Сергий Федорынчык}
\textbf{Alexey Ivanov} 

Україні не потрібні ЛДНР, їй потрібно повернути ОРДЛО. Ваші побажання
прибічникам ЄС в Україні переадресуйте прибічникам Росії на Донбасі та в Криму.
Всі, кому подобається Росія більше ніж Україна, можуть туди їхати зі своїми
валізами, але не з українською територією. Варто згадати приклад судетських
німців. А соціологічні опитування показують, що більшість населення України
підтримує євроінтеграцію України, і 48\% - за вступ до НАТО. Завдяки діям Росії
ці показники лише зростатимуть.

\iusr{Алексей Иванов}
\textbf{Сергий Федорынчык} 

какая такая украинская территория, есть украденная территория
Донецко-Криворожской ре юспублики присоединенная Лениным и Сталиным к Украине.
Ленина и Сталина вы считаете преступниками, сносите им памятники. Так вот
крупнейший им памятник это границы Украины.

\iusr{Сергий Федорынчык}
\textbf{Юлия Коломойцева} 

А чому увесь світ має бути дурним і не розуміти власної вигоди? Україна- не
Росія, і має право на самостійне існування. Попри купу негараздів, з часом
Україна стане країною благополучною, а Донбас буде найбільш квітучою її
частиною. Як у Австралію колись зсилали каторжників з Британії, а нині вона
серед найуспішніших країн.

\end{itemize} % }

\iusr{Василий Январев}

Ну это окно возможностей так себе - Украине выгоднее потерять ОРДЛО, чем
выполнить Минск.

\begin{itemize} % {
\iusr{Алексей Иванов}
\textbf{Василий Январев} пусть теряет.

\iusr{Сергий Федорынчык}
\textbf{Василий Январев} 

Україна не буде втрачати ОРДЛО, бо "Найкращі українські націоналісти - з
Донбасу!" (історик руху опору, радянський політв'язень Василь Овсієнко). В
майбутньому Донбас стане найбільш квітучим регіоном України. Мінські угоди були
підписані представниками України під тиском російської зброї біля Іловайську та
Дебальцева, але написані, начебто на Донбасі - внутрішній конфлікт. Вони
брехливі за формою і за змістом, сама Росія не збирається їх виконувати своїми
маріонетками, а себе ліцемірно проголошувалає, начебто не є стороною конфлікту.
З моменту видачі десятків тисяч російських паспортів "ополченцам" це є прямою
брехнею.

\end{itemize} % }

\end{itemize} % }
