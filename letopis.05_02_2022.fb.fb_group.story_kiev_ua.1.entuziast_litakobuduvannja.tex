% vim: keymap=russian-jcukenwin
%%beginhead 
 
%%file 05_02_2022.fb.fb_group.story_kiev_ua.1.entuziast_litakobuduvannja
%%parent 05_02_2022
 
%%url https://www.facebook.com/groups/story.kiev.ua/posts/1854918558038270
 
%%author_id fb_group.story_kiev_ua,nuribekova_vika
%%date 
 
%%tags 1937,aviacia,aviastrojenie,harkov,kiev,kpi.kiev,repressii,sssr,ukraina
%%title Ентузіаст літакобудування
 
%%endhead 
 
\subsection{Ентузіаст літакобудування}
\label{sec:05_02_2022.fb.fb_group.story_kiev_ua.1.entuziast_litakobuduvannja}
 
\Purl{https://www.facebook.com/groups/story.kiev.ua/posts/1854918558038270}
\ifcmt
 author_begin
   author_id fb_group.story_kiev_ua,nuribekova_vika
 author_end
\fi

Ентузіаст літакобудування.

5 лютого 2022 року виповнюється 135 років із дня народження талановитого
авіаконструктора, першого директора Харківського авіазаводу Костянтина
Олексійовича Калініна. 

У двадцятих - тридцятих роках минулого століття він був одним із
найперспективніших авіаконструкторів не лише СРСР, але й світу. Його літаки,
виконані за аеродинамічною схемою «літаючого крила», лягли в основу надзвукової
авіації майбутнього. Сікорський пропонував Калініну роботу в США, Європа
нагородила калінінський К-4 золотою медаллю. На вершині слави радянська влада
закатувала найуспішнішого авіаконструктора країни...

\ii{05_02_2022.fb.fb_group.story_kiev_ua.1.entuziast_litakobuduvannja.pic.1}

Костянтин Калінін народився 5 лютого 1887 року у Варшаві, в родині польки та
кадрового офіцера царської армії, вихідця з Воронезької губернії. 

В 1909 р. вступив до елітного Одеського піхотного юнкерського училища. 

У 1916 році пройшов Гатчинську авіаційну школу, діставши звання військового
льотчика та чин штабс-капітана. 

\ii{05_02_2022.fb.fb_group.story_kiev_ua.1.entuziast_litakobuduvannja.pic.2}

У 1918 - 1919 рр. служив в авіації Армії Української Держави. Але,
розчарувавшись у діях директорії УНР, перейшов на бік більшовиків. 

На початку 1918 року познайомився з працівниками й випускниками Київського
політехнічного інституту Вікторинином Бобровим і братами Іваном та Андрієм
Касяненками, які відіграли в його подальшому житті вирішальну роль.

\ii{05_02_2022.fb.fb_group.story_kiev_ua.1.entuziast_litakobuduvannja.pic.3}

Закінчити Московський авіаційниц технікум, куди його направили у 1920 році, як
досвідченого льотчика так і не вдалось.  Внаслідок «чистки» слухачів був
відрахований — «як колишній царський офіцер і дворянин».

Тож, у 1923 році Костянтин Калінін повертається до Києва і вступає на четвертий
курс Київського політехнічного інституту.

В той самий час Калінін увійшов до керівництва Авіаційного науково-технічного
товариства (АНТТ) і був призначений начальником виробництва Київського
авіаремонтного заводу «Ремповітря-6». Він запропонував колегам створити власний
вітчизняний пасажирський літак. Концепція та головні схеми цієї машини під
назвою К-1 на той час істотно відрізнялися від традиційних. Як схему літака
було вибрано підкісний аероплан з оригінальним еліптичним крилом, яке давало
низку переваг: найменші втрати на формування вихору, підвищення бокової
стійкості, зниження енергії двигуна, що забезпечувало вищу швидкість і більшу
дальність польоту. Це стало проривом в авіабудуванні. Ефективність такої форми
теоретично обґрунтував відомий німецький гідроаеродинамік Л. Прандтль. Але
пріоритет К. Калініна в розробці конструкції такого крила був підтверджений
патентом 1923 року.

Це був другий випадок в історії авіації - після Ігоря Сікорського, - коли
студент КПІ повністю спроектував літак і впровадив його у серійне виробництво.

Державні випробування нової машини, які пройшли у квітні 1925 року, засвідчили,
що машина задовольняє всім вимогам до пасажирських літаків і придатна для
використання в Цивільному повітряному флоті з повним навантаженням — три
пасажири і льотчик — літак досягав швидкості понад 160 км/год і стелі в три
тисячі метрів.

1 травня 1925 року, під пілотуванням льотчика-випробувача київського
авіаремонтного заводу «Ремповітря-6» Станіслава Косінського цей літак здійснив
свій перший успішний переліт за маршрутом Київ—Харків—Москва.

У цей час починаючим авіаконструктором зацікавилося харківське товариство
«Укрповітрошлях» (УПШ). Калініну запропонували зайняти пост головного
конструктора і начальника виробництва майбутнього авіазаводу. Разом з ним із
Києва переїхали найближчі помічники по створенню літака К-1, які сформували
невеличкий конструкторський колектив. Почалася робота над іншими літаками. 

З 17 вересня 1926 року розпочав свою історію ХАЗ, а його ідейним лідером став
Костянтин Калінін. 

Роботи було чимало, завод упевнено розвивався - уперше в заводській практиці
було вирішено будувати літаки не поштучно, а серійно — по чотири і більше. Це
дозволило заощаджувати кошти і час на підготовку виробництва.

Згодом розробки Калініпа стали найкращими у Европі - отримав головний приз
Берлінської виставки — Золоту медаль. Це зробило Костянтина Калініна одним із
найвідоміших авіаконструкторів не лише в СРСР, але й на Заході. Його запросив
на роботу до своєї американської компанії інший випускник КПІ Ігор Сікорський.
Але Калінін вважав, що він має робити свою справу у себе на Батьківщині.

Звісно літакобудування це не проста річь - бувають і злети, і падіння і навіть
катастрофи...

Тож історія дуже сумна.  Політична хвиля репресій 1937 року накрила оборонну
промисловість: у галузі був підготовлений план заходів щодо «викриття і
попередження шкідництва та шпіонажу». На заводах, у конструкторських бюро і в
науково-дослідних інститутах авіаційної промисловості розпочався справжній
терор. Були заарештовані практично всі керівники ЦАГІ - славнозвісні
конструктори літаків і двигунів. Згодом дійшла черга і до Костянтина Калініна:
22 жовтня 1938 року як «ворог народу» був засуджений «за підрив радянського
літакобудування» і наступного дня розстріляний.

Пам'ятаємо це.
