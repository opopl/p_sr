% vim: keymap=russian-jcukenwin
%%beginhead 
 
%%file 25_04_2022.stz.pc.ua.dou.1.geroizm_pogljad_kapellana.5.istorii_vadyma
%%parent 25_04_2022.stz.pc.ua.dou.1.geroizm_pogljad_kapellana
 
%%url 
 
%%author_id 
%%date 
 
%%tags 
%%title 
 
%%endhead 

\subsubsection{Історії Вадима}

З дозволу Вадима я розкажу декілька історій чудес, які ми бачили в житті його
та нашої команди. Вадим є сміливою людиною. Він один з пасторів нашої церкви. В
нього троє дітей. Він міг би спокійно поїхати разом з сім'єю в Європу.

Але він тут. Боронить Україну. Його команда евакуювала більше ніж 2000 людей з
Бучі, Ірпеня, Ворзеля, Києва, Чернігова та Донбасу. Зараз він на Донбасі. Я
вважаю, що він має отримати зірку Героя України.

\ii{25_04_2022.stz.pc.ua.dou.1.geroizm_pogljad_kapellana.5.istorii_vadyma.pic.1}

На момент написання цієї статті, сьогодні, 22 квітня, Вадим підірвався на міні
біля Лисичанська. Вони вижили, але це було реальне чудо. Може в наступній
статті я це опишу.

Далі цитати Вадима, передаю його історії, як є.

\paragraph{Жінка з Сєвєродонецька}

Вчера одного человека в Северодонецке нужно было забирать с адреса,
женщина-инвалид. Туда отказались ехать военные, там всё простреливается,
сказали, идите сами если хотите. Мы рискнули и слава Богу забрали человека.

Эта женщина всю войну просидела одна, её дом крайний в городе, кроме неё там
никого не осталось, он полностью разбит, она говорит, что лежала и кричала
«помогите», когда слышала что кто-то ходит под окнами. Зимой когда было −12, в
её квартире не было окон, на кровать где она лежала прилетали горячие осколки
от ракет.

\ii{25_04_2022.stz.pc.ua.dou.1.geroizm_pogljad_kapellana.5.istorii_vadyma.pic.2}

Чтобы не сойти с ума, она вела календарь, где записывала как прожила день. Она
слышала крики русских и стрельбу из автоматов. Ну, короче, там полная жесть,
она реально чудом осталась жива!

В дороге она радовалась всему: дождю, людям, воде, птицам.

\paragraph{Історія евакуації дитини поліцейського з Чернігова}

Мы припарковались возле полицейского отделения в Чернигове (На моєму бусі
написано «Капелан», а на інших машинах «Евакуація»)

К нам подошёл полицейский и спросил не могли бы мы эвакуировать его грудного
ребёнка вместе с супругой.

\ifcmt
  ig https://s.dou.ua/storage-files/RZ5FHKT_1.jpeg
	@caption Немовля з Чернігова
  @wrap center
  @width 0.6
\fi

У него реально не было возможности в те дни, а именно в конце марта, просто
вырваться и отвезти свою семью в безопасное место. Этот человек до конца
выполнял свой долг!


