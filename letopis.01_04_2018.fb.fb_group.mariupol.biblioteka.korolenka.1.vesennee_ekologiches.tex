%%beginhead 
 
%%file 01_04_2018.fb.fb_group.mariupol.biblioteka.korolenka.1.vesennee_ekologiches
%%parent 01_04_2018
 
%%url https://www.facebook.com/groups/1476321979131170/posts/1633246170105416
 
%%author_id fb_group.mariupol.biblioteka.korolenka,kibkalo_natalia.mariupol.biblioteka.korolenko
%%date 01_04_2018
 
%%tags ekologia,mariupol,ptica,priroda,vesna
%%title Весеннее экологическое ассорти
 
%%endhead 

\subsection{Весеннее экологическое ассорти}
\label{sec:01_04_2018.fb.fb_group.mariupol.biblioteka.korolenka.1.vesennee_ekologiches}
 
\Purl{https://www.facebook.com/groups/1476321979131170/posts/1633246170105416}
\ifcmt
 author_begin
   author_id fb_group.mariupol.biblioteka.korolenka,kibkalo_natalia.mariupol.biblioteka.korolenko
 author_end
\fi

Весеннее экологическое ассорти

31 марта, накануне Международного  дня птиц, в Центральной городской библиотеке
им. В.Г. Короленко прошло весеннее экологическое ассорти, посвященное пернатым
друзьям. Его анонсированная насыщенная программа привлекла многих мариупольцев.
Библиотекари рассказали об истории «птичьего» праздника: о далеком 1894 годе,
когда обычный учитель из маленького городка в США Чарльз Бабкок инициировал
детский праздник по защите птиц, о подписанной 1 апреля 1906 года
«Международной конвенции по охране птиц». Небольшое видеопутешествие «Пернатые
друзья» и книжная выставка были посвящены интересным фактам из жизни пернатых.

К Международному дню птиц приурочили и праздничное открытие новой экспозиции в
Мобильной галерее  «Мир увлечений». Автор творческих работ, представленных  на
выставке, Татьяна Александровна Чебанова -  большой любитель природы, и в
частности птиц. Татьяну Александровну – увлеченного человека, аматора,
руководителя  клуба-студии декоративно-прикладного искусства «Ярослава» - знают
многие мариупольцы, особенно в творческом пространстве города. Она - давний
друг Центральной библиотеки им. В.Г. Короленко. В 2012 году в рамках Мобильной
галереи библиотекари презентовали ее первую персональную выставку, много раз
она выступала в библиотечных мероприятиях в качестве волонтера. 

Татьяна Александровна рассказала о своем творчестве: создании композиций из
ракушек и уникальных украшений из бисера, о превращенных в чудных зверьков
электрических лампочках. Особое восхищение вызвали представленные на выставке
птицы, созданные Татьяной Александровной из ракушек. Среди птичьего
разнообразия широко известные мариупольцам воробей, аист, ласточка, дятел,
филин, сорока, кукушка, лебеди, и достаточно редко встречаемые сизоворонка,
зяблик, зорянка, горихвостка, свиристель, чемга, серощекая поганка и шурка
золотистая. Все творческие работы Татьяны Чебановой излучают добро, свет,
жизнелюбие.

Всем участникам экологического ассорти очень понравились рисунки учащихся
художественного отделения Школы искусств (зав. отделением Макаренко Л.В.), с
которыми можно было познакомиться на выставке "На крыльях весны". 

Создать хорошее настроение.любителям природы старалась и учащаяся музыкальной
школы № 5 Полина Кириченко. Девочка исполнила веселые песенки «Апрель», «Мистер
жук» и «Красная Шапочка».

Завершающим элементом весеннего экологического ассорти стал мастер-класс по
рукоделию  "Фетровая птичка-невеличка". Птичек делали и взрослые, и дети.
Особенно увлеченно создавали пернатых девчонки-и-мальчишки из ОО «Искренность».
Очень уж они старались сделать и аккуратную строчку, и ниточки подобрать
соответствующие, и глазки-клювики приклеить правильно! И радовалась детвора
фетровым птичкам искренно и трогательно! Материалы для мастер-класса
предоставлены в рамках реализации проекта «MRPLife: знайомимось, гуртуємось,
розвиваємось» при поддержке Международной организации по миграции (МОМ).

Итогом экологического ассорти стало всеобщее решение: подкормить птиц зимой,
сделать и повесить скворечник весной, элементарно не шуметь в местах обитания
птиц – вполне посильно каждому!
