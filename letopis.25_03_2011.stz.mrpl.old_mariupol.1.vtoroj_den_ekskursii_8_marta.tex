% vim: keymap=russian-jcukenwin
%%beginhead 
 
%%file 25_03_2011.stz.mrpl.old_mariupol.1.vtoroj_den_ekskursii_8_marta
%%parent 25_03_2011
 
%%url http://old-mariupol.com.ua/vtoroj-den-ekskursii-15-marta
 
%%author_id mrpl.old_mariupol
%%date 
 
%%tags 
%%title Второй день экскурсии - 8 марта
 
%%endhead 
 
\subsection{Второй день экскурсии - 8 марта}
\label{sec:25_03_2011.stz.mrpl.old_mariupol.1.vtoroj_den_ekskursii_8_marta}
 
\Purl{http://old-mariupol.com.ua/vtoroj-den-ekskursii-15-marta}
\ifcmt
 author_begin
   author_id mrpl.old_mariupol
 author_end
\fi

\ifcmt
  ig https://i2.paste.pics/PKL28.png?trs=1142e84a8812893e619f828af22a1d084584f26ffb97dd2bb11c85495ee994c5
  @wrap center
  @width 0.9
\fi

\begin{quote}
\em\bfseries
Предлагаем вашему вниманию вторую главу книги \enquote{Мариуполь и его окрестности}.

Напоминаем, что при размещении текста на других сайтах, не забывайте о ссылке
на \enquote{Старый Мариуполь}.

Еще раз хочу поблагодарить всех, кто работал и работает над данным проектом.
\end{quote}

\ifcmt
  ig https://web.archive.org/web/20220121044336im_/http://old-mariupol.com.ua/wp-content/uploads/2011/03/%D1%83%D0%BB%D0%B8%D1%86%D0%B0-%D0%9D%D0%B8%D0%BA%D0%BE%D0%BB%D0%B0%D0%B5%D0%B2%D1%81%D0%BA%D0%B0%D1%8F.jpg
  @wrap center
  @width 0.9
\fi

\begin{quote}
\em\bfseries
Все участвовавшие собрались в гимназии в 12 часов дня. Ученики пропели молитву
перед учением. После этого и об. инспектора Вавилов сделал следующее сообщение.	
\end{quote}

\subsubsection{Топография и климат г. Мариуполя}

Мариуполь, уездный город Екатеринославской губернии, находится под 47°5'5"\ 
северной широты и под 7°14'17"\ восточной долготы от Пулковской обсерватории
(колокольня собора во имя св. Харлампия) и расположен на северном берегу
Азовского моря  по правую сторону устья реки Кальмиус. Город тянется с одной
стороны по берегу от устья означенной реки до Кленовой балки на расстоянии с
небольшим 2 верст, с другой стороны по берегу Кальмиуса на расстоянии около
трех верст до поворота р. Кальчика (притока первой). Таким образом границей
города служат с юга – море, с востока – Кальмиус, который впадает в море почти
под прямым углом, с севера – Кальчик. 

Западная граница остается открытой и город в своем росте направляется
преимущественно в эту сторону. За Кленовой балкой по берегу моря расположены
дачи и сады местных жителей вплоть до Зинцевой балки, где находится
Мариупольский порт. Расстояние между устьем Кальмиуса и началом порта не много
более пяти верст. Площадь, занимаемая городом состоит из двух резко
отличающихся между собою частей: низменной, где расположена пригородная
слобода, и возвышенное, где находится самый город.  Вдоль всего морского берега
тянется низменная полоса, состоящая вблизи моря из чистого сыпучего песка.
Около устья Кальмиуса лежит не глубокое, но значительное по своим размерам
озеро Домаха, бывшее в недавнем прошлом одним из рукавов реки. Ширина
низменного берега в различных местах не одинакова. Начиная у озера Домаха с
ширины в 350 саж. Береговая полоса постепенно суживается, у Кленовой балки
достигается только 140 с., далее по направлению к порту суживается еще более и
местами не превышает 25 саж.; у Зинцевой балки низменный берег опять
расширяется до 200 с. и образует площадь, занимаемую различными портовыми
сооружениями. 

По реке Кальмиусу идет также низменная долина довольно значительной ширины, на
границе с морской береговой полосой ширина ее имеет 160 саж., далее к северу на
расстоянии одной версты она расширяется до 285 саж, затем постепенно суживается
и вместе с рекою теряется вне пределов города.  Почва этой полосы у города
состоит из илистого грунта, в котором пробивается много родников и ключей,
отчего та часть представляет сплошное болото. Здесь же находится так называемый
фонтан – весьма обильный ключ, снабжающий город водой.  Вода ключа собирается в
три каменных резервуара, из которых она берется для городских потребностей.
Излишек воды стекает в Кальмиус по особой канаве, для которой в дамбе железной
дороге устроена труба. По долине Кальмиуса с севера идет полотно железной
дороги, которое от озера Домахи поворачивает на запад и по берегу моря доходит
до порта. Железнодорожная станция и вокзал расположены в нескольких десятках
сажень от моря, против самой середины города. Высота над уровнем моря как той,
так и другой низменной полосы на большом расстоянии от моря и реки весьма
незначительна и редко превышает 1 саж. Вследствие этого, прилегающая часть
слободки подвергается иногда наводнению во время разлива весенних вод. За
низменной полосой, как со стороны моря, так и со стороны реки сразу идет
значительный крутой подъем и вводит в нагорную часть города.  Поверхность
возвышенности не представляет собой ровного плоскогорья, а не глубокой
Куконовой балкой, идущей с востока на запад, разделяется на два холма с
значительными уклонами. 

Более ясное представление об устройстве поверхности
высокой части города можно составить при рассмотрении данных измерения
расстояний и высот над уровнем моря. Эти данные взяты с различных точек из
материалов, собранных при составлении плана города, а также нивелировок,
произведенных учениками VII и VIII классов Мариупольской гимназии. Длина части
первого от моря холма, занимаемой городом от востока на запад 830 саж. (от
подъема у собора до начала шоссе к порту), ширина его в восточной части 450
саж., в западной 720 саж. Для определения высот над уровнем моря взяты те
пункты, которые лежат на самой выпуклой части этого холма и расположены по
одной улице (Екатерининской). Самая восточная точка – колокольня собора во имя
св. Харлампия лежит на высоте 10,5 саж., мужская гимназия 15,75 саж., женская
гимназия 24,5 саж. И самая западная точка – начало шоссе в порте – 31,7 саж.,
расстояние меду собором и началом шоссе немного более 700 саж. Так что от
востока к западу поверхность города имеет подъем в 0,03 (приблизительно
равномерный). Из данных нивелирования в другом направлении оказалось, что
поверхность города от морского берега до Екатерининской улицы имеет подъем в
0,025 и приблизительно такой же отклон по направлению к Куконовой балке. За
этой балкой лежит второй холм, длина части его, занятой городскими постройками,
имеет 620 саж., ширина, как в западной, так и в восточной, около 730 саж.
Высота второго холма несколько меньше, чем первого, а также и уклоны его
поверхности менее значительны со стороны первого холма, к Кальчику же он
спускается довольно круто. В западной части оба эти холма сливаются почти в
одну линию и образуют равную поверхность. На первом холме расположена лучшая
часто города: казенные и общественные учреждения, дома более состоятельных
граждан, магазины и лавки; здесь же в юго-западной части, на крутом морском
берегу находится городской общественный сад, одно из лучших мест города, с
прекрасным видом на море. На втором холме живет более бедная часть
народонаселения, здесь же по преимуществу находятся хлебные  амбары, а в
обрыве, выходящем к реке Кальмиус, находятся каменоломни. Высокие обрывы,
которые тянутся вдоль морского берега, изрезаны, как в черте города, так и на
всем протяжении до порта многочисленными оврагами и рытвинами и состоят из
песчаной глины, частью из глины с хрящем; в обнажениях заметны значительные
слои чистого песка, однородного с песком, из которого образовался морской
берег. В некоторых местах обрывы подходят близко к морю и носят следы
разрушения.

Лучшее сооружение в Мариуполе – его порт, лежит приблизительно, в четырех
верстах от города: восточный мол находится немного более 5 верст от устья
Кальмиуса, а западный почти в 6 верстах. Ширина портовой набережной между
молами 455 саж., длина западного мола до 720 саж., восточный мол на расстоянии
230 саж. от набережной поворачивает на запад и постепенно приближается к
западному на 250 саж. В порту двое  ворот, восточные находятся в 175 саж. от
набережной и южные вблизи западного мола: площадь, занимаемая портом, разделена
сваевым волноломом на две части. Река Кальмиус, отделяющая земли города
Мариуполя, а также Екатеринославской губернии от земли Войска донского, перед
своим впадением в Азовское море имеет направление с северо-востока и в 1 ½
верстах тот своего устья принимает направление с севера на юг и впадает в море,
как было выше сказано, под прямым углом. В настоящее время ее воды вливаются в
море двумя рукавами, главный рукав имеет направление к югу и единственный,
который бывает постоянно открыт; второй рукав течет к юго-востоку и большую
часть года бывает закрыт. 

Раньше было указано, что существовал еще третий
рукав, юго-западный, который в настоящее время представляет озеро Домаха.
Кальмиус при своем устье был подробно исследован комиссией по устройству
мариупольского порта на протяжении 6 верст от моря до впадения Кальчика. Этот,
последний, протекал около города в 3 верстах от моря, при своем впадении делает
изгиб и удаляется от города на 6 верст. По данным исследования, Кальмиус в
указанных пределах не имеет почти никакого падения. Горизонт воды в этой части
реки зависит от изменений горизонта моря. Течение в реке находится в
зависимости от направления ветра, так что поплавки, пущенные в реку для
определения скорости течения то поднимались вверх по реке, то опускались вниз;
наибольшие скорости не превосходили ½ ф. Такое состояние устья реки приводит к
заключению, что исключая время половодья, означенная часть  реки имеет свойство
морского залива.

Выше места разделения Кальмиуса на два рукава ширина его
доходит до 110 саж., вверх по реке ширина постепенно уменьшается и у плавучего
моста доходит до 60 саж. Глубина по фарватеру на этом протяжении от 3 до 5
фунтов. За мостом до Кальчика река имеет ширину от 54 саж. до 33 с., а глубину
от 5 до 8 ½ фут. Восточный рукав реки имеет ширину не более  30 саж., а
наибольшую глубину  около 3\textbackslash\ 4 саж. Озеро Домаха, которое образовалось от
заграждения и засорения песком устья западного рукава реки, имеет наибольшую
длину 420 саж. и наибольшую ширину 230 саж., глубину не более 2 ¾ ф.;
железнодорожной дамбой это озеро разделяется на две части и во время сильной
жары летом вода в нем значительно усыхает и озеро обращается в гниющее болото.
Русло реки состоит из илистого и частью песчаного грунта, глубина его свыше 9
саж. В главном рукаве реки устроена на правом берегу ее набережная длиной 260
саж. и она служит в настоящее время гаванью г. Мариуполя. Ширина этого рукава у
набережных – 240 саж. и только в одном месте у самого устья – 25 саж. Глубина
против набережной от 10 до 11 фут. Причем, ширина русла с этой глубиной от 18
до 20, 25 саж. в 1880-1882 гг. была сделана попытка улучшения старого
Мариупольского порта прорытием открытого канала глубиной 10 фут. Через бар р.
Кальмиуса и углублением самого русла реки вдоль набережной. Уже во время
производства работ образовывались в канале заносы и обмеление его, в реке же
значительных заносов не замечалось. Для дальнейшей расчистки заносов и
поддержания канала была назначена землечерпательная машина. По словам лиц,
производивших углубление канала, весенние воды Кальмиуса весьма много
содействовали образованию заносов, которые преимущественно замечались после
сильных восточных и юго-западных бурь.

Эти работы показали возможность
поддержания землечерпанием глубины входа значительно большей против
существовавшей до начала работ, но достигнуть глубины в 10 футов оказалось
почти невозможным и средней глубиной входа следует считать 6 футов. Как
продолжение набережной к северу существует так называемый Ковш – короткий
морской канал, отведенный от реки в северо-западном направлении и отделенный от
нее предохранительной дамбой. Первоначальное назначение его было служить местом
стоянки в зимнее время для землечерпательных машин, работавших при устройстве
большого порта. В настоящее время в нем   зимуют небольшие суда каботажного
плавания. Непосредственно у набережных р. Кальмиус расположены амбары, лесные
склады, торговые конторы и конторы пароходных обществ; эта местность носит
название биржи и во все время навигации полна народом и кипучей деятельностью.
Река Кальмиус, по сведениям, сообщенным местными жителями, обыкновенно
замерзает и вскрывается от льда раньше моря. При  вскрытии непосредственно у
выхода реки в море образуются зажары, вследствие чего является подъем воды,
обыкновенная высота которой от 4 до 5 фут., но бывали годы, когда вода
поднималась до 9 фут. до уровня шоссе. Когда зажара прорвана, вода и лед
устремляются с чрезвычайной скоростью в море, ломают поставленные в реке и не
защищенные каботажные суда и лодки и производят значительные подмывы, которые
углубляют некоторые места у набережной. Продолжительность весеннего поводка
бывает, по заявлению местных жителей, от 3 до 4 дней, причем вода в это время
чрезвычайно мутная и несет большое количество камыша, земли и навоза от
разрушенных в верховьях реки запруд. Камыш и частью навоза впоследствии
выбрасываются на морской берег, собираются жителями слободки и служат топливом.

Пользуясь материалами, собранными комиссией по устройству Мариупольского порта
остается сказать несколько слов о качествах Азовского моря около города
Мариуполя. Это море, образуя Мариупольский рейд, вдается в материк. Если
провести прямую линию от начала Ляпиной косы, находящейся в 7 верстах от города
к востоку, до начала Белосарайской косы, отстоящей на 25 верст к западу, то
наибольшее углубление моря в материк против Кленовой балки будет около 3 ½
верст. Берег моря около города имеет направление от востока к западу, за
Кленовой балкой принимает юго-западной направление и сохраняет его до порта;
очертание берега совпадает с направлением наиболее сильных ветров. Море по
линии от города к порту имеет различную глубину; по обе стороны устья Кальмиуса
существуют две подводных песчаных косы и в том месте море на большом расстоянии
имеет незначительную глубину. Более ясное представление о глубине моря можно
составить при рассмотрении приводимой ниже таблицы линий разных глубин. Эти
линии против устья Кальмиуса значительно удаляются от берега; против Зинцевой
балки линии приближаются к берегу. В прилагаемой таблице показаны расстояния до
линий равных глубин по направлениям нормальным берегу.

\begin{center}
\begin{longtable}{|p{2cm}|p{2.5cm}|p{2.5cm}|p{2cm}|p{2cm}|p{2.5cm}|}
\hline
\textbf{Глубина в футах} &	\textbf{Против Кальмиуса} &	\textbf{Против Домахи} &	\textbf{У Кленовой балки} &	\textbf{У начала Зинцевой балки} &	\textbf{У конца Зинцевой балки} \\
\hline
5  & 450  & 450  & 150 & 90   & 25-50  \\
\hline
10 & 700  & 1000 & 700 & 280  & 70-120 \\
\hline
14 & 1250 & 1200 & —   & 1200 & 680    \\
\hline
18 & 3500 & —    & —   & 3000 & 2800   \\
\hline
20 & 4900 & —    & —   & 4700 & 4200   \\
\hline
\end{longtable}
	
\end{center}

Грунт морского дна около берега, как и сам берег, состоит из чистого песка.
Линия залегания песка не совпадает с какой-либо определенной глубиной и
представляет большие неправильности в своем распространении. Как выше было
замечено, по обе стороны устья Кальмиуса находятся две песчаных косы из них
левая выдвигается на 6350 саж. от берега, правая же называемая Домахинской
россыпью, простирается почти на 1050 саж. от берега; притом в первой песок
доходит до 9 футовой глубины, во второй до 11 и даже до 13 футов. Против
кленовой балки песок доходит до 5 футов глубины на расстоянии около 110 саж. от
берега у Зинцевой балки до 9 фут. в расстоянии 70-100 саж. от берега. Вообще
следует заключить, что далее 13 фут. глубины песок не встречается. За линией
песков морское дно состоит из слабого ила, представляющего весьма хороший грунт
для якорной стоянки.

Сообразно с осадкой судов, посещающих Мариупольский порт, существует два места
якорной стоянки, а именно малый рейд против устья Кальмиуса на 12-футовой
глубине для каботажных судов и большой рейд для судов заграничного плавания,
против Зинцевой балки на 18-20-футовой глубине. Горизонт Азовского моря у г.
Мариуполя подвергается довольно сильным колебаниям. Изменение горизонта
зависит, главным образом, от направления ветров, господствующих в данное время;
причем ветры северо-восточной четверти компаса сгоняют воду и понижают горизонт
моря, ветры же юго-западной четверти нагоняют воду и повышают горизонт.
Возвышение над ординаром бывает обыкновенно в течение месяцев: часть марта,
апрель, май, июнь, июль и часть августа, в остальное время преобладает
понижение, хотя в частных случаях в каждый из этих месяцев случается и обратное
явление. Наибольшее понижение, какое наблюдалось при сильном восточном ветре
доходило до 46 дюймов над ординаром, наибольшее же возвышение наблюдалось до 33
дюймов, так что амплитуда колебания уровня моря достигает до 74 дюймов или 6
футов 2 дюймов.

Преобладающие в известный промежуток времени ветра служат причиной течений,
какие замечаются в море. Из наблюдений над течениями оказалось, что направление
течения зависит от направления ветров; причем, при ветрах северо-восточной
четверти наблюдается в большей части случаев течение от востока к западу, при
ветрах юго-западной четверти обратное течение. Наибольшие скорости течения
наблюдались при направлении O-W до 120 фут. в минуту, при обратном же
направлении до 220 фут. Наибольшие скорости последнего направления замечались в
первый момент после изменения направления течения с одной стороны в
противоположную. Необходимо заметить, что общих строгих выводов из
произведенных наблюдений сделать нельзя, так как течения подвержены сильным
изменениям и неправильностям. Различные ветры, дующие около города Мариуполя,
производят неодинаковое волнение в прилегающей к городу части моря: ветры южные
и юго-восточные производят самое малое волнение, самое сильное волнение
производят юго-западные, а затем  восточные ветры. Высота волны производимой
части достигает до 0.9 саж., что наблюдалось по свае, забитой на 10-футовой
глубине против железнодорожной станции. Высота волны вообще зависит от глубины
моря и расстояния до ближайшего берега со стороны ветра. При волнениях морская
вода становится мутной от поднимаемого со дна ила, содержание мути воды бывает
весьма различно. Если выразить это содержание отношением объемов полусухого ила
и воды, то оно колеблется в пределах от 0.00002 до 0.005.

Замечено также, что во время волнения при глубине дна от 7 до 12 ф. Происходит
движение ила, составляющего морское дно.

Состав воды Азовского моря, по определению технолога Шполянского, занимающегося
соляными прослами в Приазовском крае, следующий:

\bigskip
\begin{tabular}{lll}
NaCl  & Хлористого натрия –      & 0.96583\%  \\
KaCl  & Хлористого калия  —      & 0.01279\%  \\
MgCl  & Хлористого магния –      & 0.08870\%  \\
MgBr  & Бромистого магния –      & 0.00035\%  \\
CaSo4 & Сернокислой извести  —   & 0.02879\%  \\
MgSo4 & Сернокислой магнезии –   & 0.07642\%  \\
CaSo3 & Двууглекислой извести –  & 0.00221\%  \\
MgSo3 & Двууглекислой магнезии – & 0.01286\%  \\
\end{tabular}
\bigskip

и следы органических веществ\par
\bigskip

\textbf{Общая сумма} 1.18795\%

Удельный вес воды при 14º по Реомюру – 1.00097.

Удельный вес воды черного моря от 1.0114 до 1.0209.

Удельный вес воды Средиземного моря – 1.0289.

Сравнение этих цифр указывает на ничтожную соленость воды Азовского моря, у
города же Мариуполя содержание солей еще более уменьшается присутствием пресной
воды, вливаемой рекой Кальмиус, здесь обычное явление – цветение моря во время
жарких летних месяцев.

По приметам местных жителей море замерзает и освобождается от льда позднее, чем
река Кальмиус. Рейд замерзает обыкновенно в декабре, а в марте очищается от
льда; толщина льда в морозные зимы достигает ¾ аршина. Обыкновенно море,
замерзнув, стоит покрытое льдом до весны, река же в течение зимы. Замечено
также, что редко случается, чтобы замерзала средняя часть Азовского моря,
прибрежный же лед не остается на всю зиму неподвижным, при начале воды он
поднимается, а при спаде опускается вертикально вместе с водой, при сильных же
северных ветрах иногда отгоняется от берега, уходит в море и потом
юго-восточным  ветром снова пригоняется к берегу в такое обстоятельство
повторяется иной год по несколько раз в одну зиму.

Подобное же явление бывает весной во время таяния льда: силой ветра лед то
отгоняется от берега, то вновь возвращается и служит причиной понижения
температуры.

В Мариуполе метеорологической станции нет, но с конца 1887 года при портовом
управлении производятся метеорологические наблюдения главным образом над
направлением ветров и колебаниями уровня моря. Попутно с этим делаются
наблюдения и над изменениями температуры три раза в день в 7 часов утра, 1 час
пополудни  и 9 часов вечера. Благодаря любезности заведующего портовыми
работами Л. И. Янышева я имел возможность познакомиться с этими наблюдениями.
Само собой понятно, что наблюдений за четыре года весьма недостаточно для того,
чтобы сделать какие-либо положительные выводы в заключении о климате данной
местности, так как при наблюдениях в более продолжительный промежуток времени
многие данные примут другой вид, неудобство это будет видно при рассматривании
приводимых ниже таблиц. К крайнему сожалению необходимость заставляет
ограничиться тем, что есть и насколько возможно выяснить основные особенности
местного климата. Прежде всего рассмотрим таблицу, выражающую годовой ход
температур в Мариуполе по месяцам и временам года:

% 17
%\begin{tabular}{lllllllllllllllll}
%\small\hline
%январь & 	февраль &	март & апрель &	май &	июнь &	июль &	Август &	сентябрь &	октябрь &	ноябрь &	декабрь &	зима &	весна &	Лето &	осень &	год \\
%\hline
%— 4.8 &	— 2.7 &	4.4 & 10.9 &	15.1 &	18.8 &	21.1 &	18.1 &	11.7 & 	5.7 &	— 0.2 &	— 5.1-7 &	— 4.8 &	10.1 &	19.3 & 	5.7 &	7.6 \\
%\hline
%\end{tabular}

Относительно средней месячной температуры за декабрь приходится сделать
оговорку: — 5.1 град. средняя температура этого месяца за рассматриваемые
четыре года, включая и 1891 год с исключительно  высокой средней температурой
0.8 вместо обычных –7 и –8 градусов, поэтому будет более правильно считать
среднюю температуру за этот месяц –7 градусов. Приведенная таблица показывает,
что самый жаркий месяц – июль, повышение от июня и понижение к августу почти
одинаково, далее понижение к зиме идет быстрее, чем повышение от зимы к лету;
самый холодный из зимних месяцев – декабрь, а самый теплый – февраль, что
ставит Мариуполь по климату наряду с остальным материком. Значительная разница
между средней температурой самого теплого месяца июля 21.1 и самого холодного
декабря –7 градусов, разность равняющаяся 28.1 градусов указывает на климат с
крайностями, резкость его. Рассмотрим далее колебания средней месячной
температуры: первая строка дает самую высокую среднюю температуру каждого
месяца в четырехлетний промежуток, вторая – самую низкую среднюю за этот же
переход. Третья строка дает разность или амплитуду колебания

%Январь 	февраль 	март 	апрель 	Май 	июнь 	июль 	Август 	сентябрь 	октябрь 	ноябрь 	декабрь 	Год
%-3.5 	-1.8 	5.7 	11.6 	16.6 	20.2 	23.5 	19.5 	13.2 	7.6 	1 	0.8 	8.1
%-6.9 	-3.2 	3 	9.8 	13.6 	17.8 	19.4 	17.2 	10.5 	3.4 	-1.7 	-8 	6.8
%3.4 	1.4 	8.7 	1.8 	3 	2.4 	4 	2.3 	2.7 	4.5 	2.7 	8.8 	1.8

%https://web.archive.org/web/20180212123518/http://old-mariupol.com.ua/vtoroj-den-ekskursii-15-marta
