% vim: keymap=russian-jcukenwin
%%beginhead 
 
%%file 20_10_2020.sites.ru.zen_yandex.yz.ja_kladoiskatel.1.poslednjaja_gramota_novgorod
%%parent 20_10_2020
 
%%url https://zen.yandex.ru/media/yakladoiskatel/posledniaia-naidennaia-novgorodskaia-berestianaia-gramota-otkryla-informaciiu-o-kotoroi-ranee-izvestno-ne-bylo-5f84a5ebcae5a83b554a41d9
 
%%author Я Кладоискатель (Яндекс Zen)
%%author_id yz.ja_kladoiskatel
%%author_url 
 
%%tags novgorod,beresta,russia
%%title Последняя найденная Новгородская берестяная грамота открыла информацию, о которой ранее известно не было
 
%%endhead 
 
\subsection{Последняя найденная Новгородская берестяная грамота открыла информацию, о которой ранее известно не было}
\label{sec:20_10_2020.sites.ru.zen_yandex.yz.ja_kladoiskatel.1.poslednjaja_gramota_novgorod}
\Purl{https://zen.yandex.ru/media/yakladoiskatel/posledniaia-naidennaia-novgorodskaia-berestianaia-gramota-otkryla-informaciiu-o-kotoroi-ranee-izvestno-ne-bylo-5f84a5ebcae5a83b554a41d9}
\ifcmt
	author_begin
   author_id yz.ja_kladoiskatel
	author_end
\fi
\verb|20 October 171k full reads 1,5 min|
\index[cities.rus]{Новгород, Великий!Берестяная грамота, последняя найденная, 20.10.2020}
\index[rus]{Русь!История!Берестяные грамоты}

На данный момент в Новгороде найдено более 800 берестяных грамот. Но сегодня
пойдет речь о послании, найденном во время последней археологической
экспедиции. В обнаруженной грамоте исследователи прочитали информацию, о
которой ранее известно не было.

Берестяные грамоты, это новый вид исторических источников, о котором до
середины 20го века ученые даже и не подозревали. Они являются средневековыми
письмами. Это был дешёвый материал и на нем легко можно было нацарапать
послание.

В начале осени 2020 года археологи обнаружили удивительную находку во время
проведения раскопок на Софийской стороне города Великий Новгород. Ею оказалась
берестяная грамота, вернее её фрагмент, который датируется XIV веком.

Изучая новгородский культурный слой, археологи пришли к выводу, что в
средневековье в этой местности часто случались крупные пожары, так как дома, в
большинстве случаев, строили из дерева. В одном слое археологи зафиксировали 3
разрушительных пожара.

\ifcmt
pic https://avatars.mds.yandex.net/get-zen_doc/1852544/pub_5f84a5ebcae5a83b554a41d9_5f84a6088e35355ad19414c8/scale_2400
caption https://www.pomorie.ru/2020/10/01/5f75e7fc92a11b328d183fe2.html
\fi

Перед застройкой жилого дома в современном Новгороде, археологи провели
исследования участка, где ранее находилась центральная часть старого Новгорода.
Именно здесь и был обнаружен остаток берестяной грамоты, у которой, к
сожалению, сохранилась только нижняя часть текста (около 2х строк). Что было
выше остается только догадываться.

\ifcmt
  pic https://avatars.mds.yandex.net/get-zen_doc/1875669/pub_5f84a5ebcae5a83b554a41d9_5f84a6d1a144c35a2714c516/scale_1200
  caption Берестяная грамота, фото: Институт археологии РАН
\fi

В ней упоминается о рати, которая идет через Заволочье или Волок. Автор задает
вопрос: «Чи пойдёт рать за Волок» и просит осведомить его, чего ожидать от
этого военного отряда. Эта находка уникальна тем, что из документов редко
удается получить сведения о ходе каких-либо военных операций. Да и вообще в
источниках очень мало сведений о Заволочье.

Этот фрагмент наводит на мысли о каком-то военном конфликте в те времена.

\ifcmt
  pic https://avatars.mds.yandex.net/get-zen_doc/3985451/pub_5f84a5ebcae5a83b554a41d9_5f84a722cae5a83b554cb428/scale_2400
  caption https://www.pomorie.ru/2020/10/01/5f75e7fc92a11b328d183fe2.html
\fi

Рядом с фрагментом была найдена печать новгородского посадника, который занимал
эту должность в 1354-1380 годах, это и помогло датировать документ. Археологи и
сейчас находят много интересных артефактов XIV-XVI веков, среди которых чаще
всего встречаются нательные кресты и различные украшения. Все эти предметы
говорят о достойном социальном статусе жителей, которые, возможно, были
торговцами или занимали высокие посты.

Археологи зафиксировали уже более 800 берестяных грамот X-XIV вв. Написаны они
представителями всех социальных классов, даже низших сословий, что говорит о
достаточно высоком уровне грамотности.

Из таких, казалось бы, незначительных находок и строится наша история – история
России.

Загляните в наш блог)

\begin{itemize}

\item \cusr{Иван Корн}

Автор несколько поскромничал относительно "достаточно высокого уровня
грамотности". Грамотность была всеобщей. Известны грамоты,
написанные детьми. В раннем Средневековье! В то время, когда в
Европе редкий король умел сносно читать, а писать и того реже.
Более того, филологами установлено, что в текстах грамот (а их
найдено около 1200), нет ни единой грамматической ошибки! Ни
единой! При том, что это не официальные документы, а бытовая
переписка горожан, такое средневековое СМС. Проверьте свои
СМС-ки, много ошибок делаете? Вот то-то. Так что, эти скромные
артефакты - свидетельства невероятно высокого культурного
уровня Древней Руси.

\item \cusr{Отец Этлау}

Иван Корн, да, а потом пришли Романовы и сначала всех сделали безграмотными и
загнали в кабалу, а потом переписали историю под себя. Только
благодаря кровавым совкам и их наследию у нас есть на
сегодняшний момент оставшиеся крохи образования, науки и
медицины... Но ничего, ещё пару реформ и будет как при
Романовых.

\item \cusr{Андрей Виноградов}

Иван Корн,Так уж и всеобщей? Города где найдены берестяные грамоты можно
пересчитать по пальцам. В основном это северо-запад,частично
северо-восток. Основная масса в Новгороде ,потом в городах ,где
новгородцы имели торговые связи Псков16 шт),Старая Русса(51
шт), Смоленск(16) ,Псков (8)Торжок(19) ,Тверь(9) Москва,Рязань
и тд единичные находки.Всё. То есть берестяные грамоты это
чисто новгородский феномен и общерусского значения не имел.Да и
сами новгородцы судя по тем же грамотам русскими себя не особо
считали. В одной грамоте так и написано-"Еду во Владимир в Русь
из Новгорода". или в новгородской летописи-Въ лѣто 6657 [1149].
Иде архепископъ новъгородьскыи Нифонтъ въ Русь, позванъ
Изяславомь и Климомь митрополитомь: ставилъ бо его бяше
Изяслав. И таких примеров можно привести далеко не один.Все они
говорят,что сами новгородцы Новгород Русью не считали и так
почти до 15 века.

\iusr{Павел К.}

Отец Этлау, не надо клеветать на Романовых. Все произошло задолго до них.
Новгородская республика, отличавшаяся высоким уровнем
грамотности, была уничтожена московским князем Иваном III, а
затем повторно вырезана уже царем, Иваном Грозным (новгородский
погром). Эти разрушители русской культуры были не Романовы, а
Рюриковичи.

\iusr{Павел К.}

Андрей, что значит "новгородцы не считали себя Русью"? Да именно они начтоящей
русью и были. Русское государство было создано Рюриком в 862 г.
в городе Ладога (сейчас это Ленинградская обл.), а затем Рюрик
переместился в Новгород. Это и есть исконная Русь. Другие
территории, включая Киев, были завоеваны позже.



\iusr{Андрей Виноградов}

Павел К., То и значит.Я вам привёл два примера,где прямо так и написано-Из Новгорода в Русь. Хотите поищите ещё примеры ,уверен найдёте. Новгородцы считали себя отдельным народом.Мало того при анализе берестяных грамот лингвисты ,в частности Зализняк делают вполне обоснованные выводы ,что к 14-15 векам в Новгороде складывался отдельный язык,родственный русскому,но самостоятельный.И не присоедини Иван 3 Новгород к Москве сейчас бы существовало не три(русский,украинский и белорусский) а четыре восточнославянских языка.Да кстати когда было летописное призвание Рюрика никаких русских вообще не существовало как единого народа а были племенные союзы полян,древлян,радимичей ,и т.д. Варягов призывали четыре племенных союза словене( безусловные славяне),кривичи(до сих пор не определили в какой степени они славяне) чудь и меря(не славяне,финно-угры). И никаких русских.

\iusr{Сола Камели}

Андрей, новгородцы действительно были несколько в стороне от Руси, что
Внутренней, что Внешней, если вспомнить деление её Константином Багрянородным.
Но "во Владимир в Русь" должно подсказать Вам, где была подлинная Росия,
известная и грекам, и арабам.

\iusr{Павел К.}

Андрей, Вы невнимательно читали ПВЛ. Племена призвали не просто варягов, а
варягов-русь. Причем во-первых, ясно указано, что русь - это название народа, а
во вторых, что русь и до этого была там, собирая дань с окрестных племен. То
есть русь сначала изгнали на время, а потом уже призвали обратно. Русские - это
и есть союз пяти племен: руси, словен, кривичей, чуди и веси (иногда мерю
добавляют), названный по главному народу.

Насчет завоевания Новгорода Москвой. Эта политика привела не к объединению Руси
(как утверждают московские историки), а к смутному времени, и полному развалу
русского государства. Русь была объединена уже союзом ополчений - Минина и
Пожарского, Ляпунова, и прочих - рязанских, нижегородских и т.д. После чего и
был избран царь. Надеюсь вы знаете, что Москва возвысилась над русскими
городами не как русский город, а как ордынский - там был центр сбора дани с
русских городов. Поэтому "родственный русскому язык" мог возникнуть только в
Москве, но никак не в исконно русском Новгороде. А все обвинения в адрес
Новгорода - не более чем пропаганда Московского княжества. Представьте себе,
пропаганда в то время уже существовала.

Ну и насчет ученых, которые делают глубокомысленные выводы на основании пары
грамот. Я недавно читал у одного из украинцев, что жители Львова считают
киевлян москалями, а жители более западных мест, считают москалями самих
львовян. Вот примерно такой же уровень оценок по бытовым записям тех времен,
как и по современным постам. Для диссертации и создания научной школы годится,
но действительность не отражают.

\iusr{Андрей Виноградов}

Павел К., Ну и чушь вы нагородили. Во первых русь указывается в списке .И
список этот не содержит ни одного славянского.В основном это германцы. Во
вторых - русь это не племя,а профессия и слово имеет явно финское
происхождение.Все остальные англы, фряги, и т.д. и только варяги-русь. Именно
так финны называли шведов. Роутси-гребцы. Кстати именно так они шведов называют
по сей день роутсилайнен,а нас по сей день вения .В третьих Язык менялся и в
раннем древнерусском наречии не было слова русь вообще.Это слово псалось через
букву "от"(сейчас не используется) и читалось как Руотсь или Роутси.

\iusr{Андрей Виноградов}

Павел К., Я уж не говорю об отрывке с описанием прихода "послов от кагана
русов" к византийцам,где все русские почему-то оказываются германцами
-скандинавами."Мы от рода русского — Карлы, Инегелд, Фарлаф, Веремуд, Рулав,
Гуды, Руалд, Кари, Фрелав, Руар, Актеву, Труан, Лидул, Фост, Стемид— посланные
от Олега, великого князя русского..."

\iusr{Константин}

Андрей, где в словосочетании "Еду во Владимир в Русь из Новгорода" есть хотя бы
намёк на "разные народы"? Это как прочитав "еду в Сибирь, в Новосибирск"
считать, что Сибирь - это народ. Вы и другого бреда понаписали, ещё и слова
воспитанного, неглупого и явно образованного человека (Павел К) чушью
называете. Дождётесь, насуют и опустят вас в этом чятике :). Лучше исчезните,
пока не поздно :)

\iusr{Lora Shurpik}

Андрей, а в голову не приходит , что русы - это и есть союз племен, и язык
русский - это продукт смешения племенных языков.

А уж украинцев не было и тогда, когда Русь уже сложилась. Малая Русь, окраинная
- была.

\iusr{станислав}

Андрей, большинство так называемых летописей были сфабрикованы в основном при
Софии Августе Фредерике Ангальт-Цербстской, хотя повесть временных лет была
состряпана в городе кёнигсберг к приезду петра. все остальное это списки со
списков с дополнениями и исправлениями так что не надо лохматить бабушку.
ученые не в курсе где на самом деле был Великий Новгород, ну так на минуточку.

Павел К., а вы за Софию Августу Фредерику, принцессу Ангальт-Цербстскую не
хотите рассказать, убила мужа захватила престол и самые страшные годы
крепостного права были именно при ней. так что не гони волну они и романовыми
не были никогда, от слова совсем

\iusr{Андрей Виноградов}

Lora Shurpik, А в голову не приходит ,что свои бредовые предположения надо
подтверждать ссылкой на источники? А украинцы тут ни при чём.Как впрочем и
русские с белорусами. Никого из них ещё не было века до14-15.Была древнерусская
народность и то не сразу,а постепенно складывалась из племенных союзов не одно
столетие.

\iusr{Александр}

Иван Корн, за 300 лет при всеобщей грамотности новгородцы должны были написать
НЕСКОЛЬКО МИЛЛИОНОВ берестяных грамот. Найдено 1200. Допускаю, что нашли лишь
один процент, но это 120 тыс. Т.е по 400 грамот каждый год на 40-50 тыс
населения. Да, возможно каждый тридцатый был грамотным или даже каждый десятый
в 15 веке ( наибольший расцвет Новгорода), но не больше.

Для тех времен это прилично. В европейских городах примерно также. Ведь Франсуа
Вийон в 14 веке не только для дворян писал.

\iusr{Олег Ермаков}

Павел К., Да ладно! По какому же недомыслию в Правде Ярославовой в пункте
первом читаем -

1. Оубьеть моужь моужа, то мьстить братоу брата, или сынови отца, любо отцю
сына, или братоучадоу (а), любо сестриноу сынови; аще не боудеть кто мьстя, то
40 гривенъ за голову; аще боудеть роусинъ, любо гридинъ, любо коупчина (б),
любо ябетникъ, любо мечникъ, аще (в) изъгои боудеть, любо словенинъ, то 40
гривенъ положити за нь.

Русин ниразу не словенин выходит! О какЪ!


\iusr{Евгений Гришуков}

Отец Этлау, Это точно! По нем образование это,, аксиомы,, договорных величин..
Что по сути не какого отношения к настоящему образованию не имеет . Но, есть
ещё кое-что, обратная сторона этой медали, которую почему-то считают
образованием, это суждения обыкновенных невежд, кто во что горазд, в меру своей
некомпетентности... Вы правы, образованные бывают совершенно разные, то есть
это проблема в нас, способность отождемтаить, понять, определить факты, а не
проблема окружающего нас Мира! Интереснв е которые комментарии, о Руси, России
, один плюс, в том что задумались о том что собственно из себя представляет
Русь, Россия, Русская Цивилизация!Судя по комментариям мы не знаем в какой
Стране Живем!? Но тем не менее, где то на глубинах сознания осознаем,
сокральную суть России!

\iusr{Владимир Владимиров}

Андрей, Да, да, да, русских не было, но именно праславяне создали тот
праиндоевропейский язык, отца всех индоевропейских языков, что был суть
праславянский, о чем уже начинают говорить даже западные лингвисты, как
например П. Фридрих,утверждающий ,что санскрит родился на территории
праславяне, что подтвердила новейшая наука днк-генеалогия,доказав ,что
арийский,корневой субклад z645 R1a имел местом своего зарождения центральную
часть современной РФ.

Славяне ведут свою историю с корневого примерно субклада z645(z283)

\ifcmt
  pic https://avatars.mds.yandex.net/get-zen_pictures/3403730/1093869923-1603251844160/orig
\fi

\iusr{Владимир Владимиров}

Андрей, Вандалы- славяне,о чем существует множество свидетельств

Вильгельма Рубрука из книги \enquote{Путешествие в восточные страны} (13 век):

Язык русских, поляков, чехов (Воеmorum) и славян один и тот же с языком
вандалов, отряд которых всех вместе был с гуннами

Так, к примеру, Блонд говорит, что вандалы названные так по имени реки Вандал,
впоследствии стали называться славянами. Иоанн Великий Готский пишет, что
вандалы и славяне одной нации и отличаются только по названию. М. Адам во II
кн. «Истории церкви» говорит, что славяне — это те, кто прежде назывались
вандалы.

Его соотечественник, писатель XII в. Гельмольд в полном согласии с ним говорит,
что славян в древности называли вандалами, а в его время - винитами, или
винулами.

Фламандский монах Рубрук писал в 1253 г., что «язык русинов, поляков, богемов
(чехов. - С. Ц.) и славян тот же, что и у вандалов».

Также и уроженец словенской Каринтии Сигизмунд Герберштейн (первая половина XVI
в.) утверждает, что в период своего могущества вандалы «употребляли... русский
язык и имели русские обычаи и религию». Далее он поясняет, что немцы именуют
всех славян «виндами, вюндами и виндитами, производя их имена от одних только
вандалов».

Об этом же пишет, ссылаясь на не дошедшую до нас «Историю вандалов» Альберта
Кранция, хорватский просветитель из Далмации Мавро Орбини (XVII в.): «Вандалы
имели не одно, а несколько различных названий, а именно: вандалы, венеды,
венды, генеты, венеты, виниты, славяне и, наконец, валы». Для подкрепления
своего утверждения о тождестве вандалов и славян он приводит выдержки из
вандало-славянского словаря Карла Вагрийского, свидетельствующие о языковой
близости этих двух народов.

Схожее наблюдение принадлежит географу XVI в. Меркатору, который заметил о
языке населения острова Рюген, что у них в ходу «славянский да виндальский»
языки.

Этно-языковое родство вандалов и славян утверждается также во многих
средневековых русских источниках и славянском фольклоре - в частности, об этом
говорит легенда о старейшине Словене и его сыне Вандале.

\iusr{Геннадий Н. Б.}

Берестяные записки говорят о высоком проценте грамотности русского населения
тех времён и, естественно, о глубоких древних культурных его корнях,
существовавших со времён языческой Руси, которое, пришедшее ему на смену
христианство, уничтожило.

Самый действенный способ уничтожения реализовался с переходом письменности от
глаголицы к кириллице. Именно это привело к потере всей древнейшей
многотысячной истории Руси и утраты поголовной грамотности её населения.
Множество сказаний, мифов и былин Древней Руси христианская кириллица отвергла
и, естественно, нигде не зафиксировала с применением новой письменности,
заменив их древнесемитскими сказками и мифами христиан. Вот таким простым
способом лишили нас великого исторического прошлого!

Сейчас подобное происходит в бывших советских республиках, но уже с заменой
русского алфавита на латинский, ставя крест на всём советском прошлом, потому
что никто из них не будет переводить на латиницу все старые тексты, а только
то, что укладываются в их современную национальную парадигму. Так что можно с
уверенность ожидать - все они скоро перейдут из орбиты влияния России, на
орбиту американцев, да и у наших, которые исподволь поднимают подобный вопрос,
надо думать, тоже такие задумки есть!

\iusr{валентина полякова}

Самым большим и охраняемым секретом России - является её реальная история!

КТО СОЗДАВАЛ НАМ ИСТОРИЮ?

Сейчас мы последовательно перечислим ВСЕХ АКАДЕМИКОВ-ИСТОРИКОВ Российской Академии наук, как иностранцев, так и Отечественных, начиная от её основания в 1724 году вплоть до 1918 года. (справочное издание, книга 1) Мы приводим также год избрания.

1) Коль Петр или Иоганн Петер (Kohl Johann Peter), 1725,

2) Миллер или Мюллер Федор Иванович или Герард Фридрих (Mu»ller Gerard Friedrich), 1725,

3) Байер Готлиб или Теофил Зигфрид (Bayer Gottlieb или Theophil Siegfried), 1725,

4) Фишер Иоганн Эбергард (Fischer Johann Eberhard), 1732,

5) Крамер Адольф Бернгард (Cramer Adolf Bernhard), 1732,

6) Лоттер Иоганн Георг (Lotter Johann Georg), 1733,

7) Леруа Людовик или Пьер-Луи (Le Roy Pierre-Louis), 1735,

8) Мерлинг Георг (Moerling или Mo»rling Georg), 1736,

9) Брем или Брэме Иоганн Фридрих (Brehm или Brehme Johann Friedrich), 1737,

10) Тауберт Иван Иванович или Иоганн Каспар (Taubert Johann Caspar), 1738,

11) Крузиус Христиан Готфрид (Crusius Christian Gottfried), 1740,

12) ЛОМОНОСОВ МИХАИЛ ВАСИЛЬЕВИЧ, 1742,

13) Модерах Карл Фридрих (Moderach Karl Friedrich), 1749,

14) Шлецер Август Людвиг (Schlo»zer Auguste Ludwig), 1762,

15) Стриттер или Штриттер Иван Михайлович или Иоганн Готгильф (Stritter Johann Gotthilf), 1779,

16) Гакман Иоганн Фридрих (Hackmann Johann Friedrich), 1782,

17) Буссе Фомич или Иоганн Генрих (Busse Johann Heinrich), 1795,

18) Вовилье Жан-Франсуа (Vauvilliers Jean-Francois), 1798,

19) Клапрот Генрих Юлий или Юлиус (Klaproth Heinrich Julius), 1804,

20) Герман Карл Федорович или Карл Готлоб Мельхиор или Карл Теодор (Hermann Karl Gottlob Melchior или Karl Theodore), 1805,

21) Круг Филипп Иванович или Иоганн Филипп (Krug Johann Philipp), 1805,

22) Лерберг Август или Аарон Христиан (Lehrberg August Christian), 1807, 23) Келер Егор Егорович или Генрих Карл Эрнст (Ko»ler Heinrich Karl Ernst), 1817,

24) Френ Христиан Данилович или Христиан Мартин (Fra»hn Christian Martin), 1817,

25) ЯРЦОВ ЯНУАРИЙ ОСИПОВИЧ , 1818,

26) Грефе Федор Богданович или Христиан Фридрих (Gra»fe Christian Friedrich), 1820,

27) Шмидт Яков Иванович или Исаак Якоб (Schmidt Isaac Jacob), 1829, 28) Шенгрен Андрей Михайлович или Иоганн Андреас (Sjo»rgen Johann Andreas), 1829,

29) Шармуа Франц Францевич или Франсуа-Бернар (Charmoy Francois-Bernard), 1832,

30) Флейшер Генрих Леберехт (Fleischer Heinrich Lebrecht), 1835,

31) Ленц Роберт Христианович (Lenz Robert Christian), 1835,

И т.д...

\iusr{Тата Мир}

Ндяяя !У некоторых диванность в полной форме !

Кирик Новгородец (нач. XIIв.) в своих "Вопрошаниях" к новгородскому епископу
Нифонту : " Нъсть ли въ томь гръха, аже по грамотамъ ходити ногами аже кто
изрЪзавъ помечеть, а слова будуть знати? "

Тем , кто топит за исключительно новгородский феномен (?) рекомендую
поинтересоваться не только грамотами (малое количество в других др.городах
объясняется ТМН , составом почвы , застройкой местности и т.д. ), но и
средневековыми граффИти на стенах церквей , и таки , не только Вел.Новгорода .

//...Зализняк делают вполне обоснованные выводы ,что к 14-15 векам в Новгороде
складывался отдельный язык,родственный русскому,но самостоятельный...//-
глупость несусветная от А.Виноградова !Зализняк и т.д. пишут не об отдельном
языке , а о новгородском ДИАЛЕКТЕ , причём как раз таки к XIV-XV в.в. различия
в диалекте от "стандартного" древнерусского становятся МЕНЕЕ заметны , чем в
Xl-Xllв.в.

Судя по некоторым , скажем так мягко - сепаратистским глупостям об
исключительности Вел.Новгорода и плохих "москалях" , кто-то начитался
нью-хронолоджи от ВКЛ ? Вопрос риторический !

\iusr{riktiki r}

Уже Великие Шумеры набежали. 100 лет отроду с папой Лениным но память уже
девичья. Ничерта не помнят. Все путают. Доказывать будут что их Куев или Х..в
незакрытый пуп земли и Роcсия не Россия, и Русь ненастоящая, и до Куева, и
письменности не было:)

\iusr{валерий чирков}

riktiki r, верно, но вы пишите о хохлах а не о шумерах. Паны хохлы к шумерам
никакого отношения не имеют, к хазарам да!

\iusr{riktiki r}

валерий, Ну их давно в шутку так зовут. Да и к хазарам навряд ли они относятся.
Ту степь монголо-татары зачистили качественно. Земли заселялись заново из
глубинки России.

«… а всех юродивых и убогих ссылать на Окраину, там им дуракам место»

Иван Грозный

\iusr{Руслан Солнышко}

Здравствуйте. Извините за беспокойство. Хочу поделиться замечательным городом.
Город Чухлома - это небольшой, но очень провинциальный городок Костромской
области. В нашем уютном городке проживают чуть более пяти тысяч человек. Его
называют глубинкой Костромского края. В Чухломе есть красивая Успенская
церковь. Она украшает город и работает практически каждый день, кроме четверга.
Ещё в городе есть восхитительный парк, который находится недалеко от озера. В
этом удивительном парке растут красивые цветы: тюльпаны, нарциссы, красные и
жёлтые розы, за которыми очень хорошо ухаживают садоводы. Ещё есть сказочное
озеро - это замечательно место для купания жителей города и всего Чухломского
района. В Чухлому приезжают люди за десятки и сотни километров, чтобы погулять
по улицам и полюбоваться красивыми домиками. У нас ведь много крутых домов с
изумительными наличниками на окнах. А недавно, лет пять назад, появился большой
народный праздник - Пуговка. Чухломичи, а также все приезжие, желающие
насладиться этим прекрасным праздником, приезжают посмотреть его. Также
работает кинотеатр, где показывают 3D и даже 5D фильмы. Ещё в городе есть
несколько больших магазинов - Пятёрочка, Дикси, 10 Баллов, Магнит и
Перекрёсток. Не так давно этих магазинов не было. Их построили в 21 веке. Кроме
супермаркетов, в городе работают минимаркеты: Фортуна, Лесная Сказка,
СуперЛюкс, Надежда, Эврика, Алекс, Улыбка и другие магазины. Есть также и
просторные аптеки: Красные Цены, Антей, Фарм Лига, Губернская, и куча
салонов-парикмахерских, и очень хороший краеведческий музей, где научные
сотрудники рассказывают историю о нашем городе, русской старине, а также о
знаменитых людях, творческих деятелях и так далее. В городе есть недорогая
сауна - это прекрасное место для оздоровления любого человека. Вот так мы и
живём - очень хорошо и достаточно богато. Мы жители приспособленные, нам всё,
что надо - хватает. Работаем для себя и для других - стараемся! Также в городе
есть Майкова гора - это прекрасный лес. Летом здесь мы собираем землянику,
малину и подосиновики. А зимой катаемся на лыжах и на ватрушках. Приезжайте к
нам пожить, у нас тихо и спокойно, в отличии от Костромы. А возможностей очень
много, как в крупном городе. Также в Чухломе есть три больших гостиницы: две в
центре города, а одна на автозаправочной станции. Все гостиницы работают и
недорогие.

\iusr{Юлия Л.}

Ну, по поводу тотальной грамотности без единой ошибки в комментариях
погорячились.

Я пишу иконы. Часто это списки с очень древних икон. Так вот, авторы их были
грамотными, но ошибки в орфографии встречаются часто, поэтому при изготовлении
списка древней иконы надо проверять надпись на ошибки. И нельзя ее просто
списать.

Свободно читаю на старославянском, но надписи бывают очень древние, старше
нынешней версии церковно-славянского языка.

Иногда надписи приносят филологам, такое там бывает, что без специалиста не
прочитать.

Новгородская школа в силу древности сохранилась не очень хорошо, мало икон 12
века. Есть еще деревенское письмо, которое писали в северных деревнях. Вот там
ошибок бывает много.

В общем, помимо летописей, берестяных грамот есть еще пласт текстов на образах.

И я с ним имела дело довольно долго.

Бывали и очень забавные случаи при прочтении текстов на музейных иконах.
Особенно радует Южно-Русская школа. Они любили писать многофигурные иконы для
храмов, в руки Святых вкладывать назидательные свитки. Так, то свиток с
молитвой не тому Святому, то не тот Праздник. Орфографические ошибки по
сравнению с этими свитками - уже мелочь мелкая.

Так что по работе много имела дела с очень древней орфографией.

Южно-Русская школа писала слабей Новогорода 12 века, слабей Московской школы,
поэтому списков с нее обычно не делают, не шедевры, но изучают.

\iusr{Владимир}

Хорошая статья, только к ней надо писать предисловия, всё-таки не все хорошо
знают историю и историю Новгорода особенно. (Я уж про ЕГЭ не говорю) И в
советской школе это изучение было поверхностным. Да, типа: "попадаются
отдельные свитки, грамотными были 5 человек на весь город, а вот благодаря
достижениям социализма - у нас поголовная грамотность"!!!

Во-первых, начать надо с того, что Новгород и Русь (Московия) - дав абсолютно
разных государства. Я думаю это знают 30\% не более.

Во-вторых, эти разные государства населяют два различных народа!!!

А вот это, УПС!, знают в лучшем случае 1-2\%.

И именно берестяные грамоты дали возможность подтвердить этот факт.

Хотя если подумать - он вполне очевиден.

Русь заселялась с Киева. Введенный термин Киевская Русь в корне неверен,
критики (а особенно нэзалэжники) совершенно справедливо указывают что такого
упоминания нет ни у кого из соседей, и делают неверный вывод: Киевской Руси не
было.

Да, её и не было!!! Была просто Русь (и упоминаний об этом - море) со столицей
в Киеве.

И была залесская Русь, со столицей в Ярославле, Суздале, даже Твери, и в конце
концов в Москве. Так вот пришедший с Киева Князь приводил с собой своих
дворовых людей (слуг), в последствии ставших дворянами.

Новгород заселялся с запада, из Литвы.

Этому есть несколько подтверждений, не вполне очевидных:

Во-первых, политика Новгорода была ориентирована на Запад (сейчас её почему-то
называют антирусской, наверное в угоду современной политике).

Во-вторых, как раз те самые грамоты!!! Причем в конце 20-го века научное мнение
сходилось на том, что грамотными были только верхи (грамот было найдено мало),
да и те были безграмотными (писали с ошибками!!!). И именно в ошибках и речь -
это с точки зрения русского языка они писали с ошибками, а с точки зрения
новгородцев - они то писали грамотно, просто язык был не русский.

И вот два подтверждения:

Прямо эта грамота: что такое "чи"? Нет в русском языке такого слова!!! Значит
безграмотно? Ан, нет!! это просто другой язык. И "чи" есть и в литовском, и в
белорусском, и в польском.

Другой пример, в другой грамоте: "иос". Нет в русском такого слова и тут уже
говорить о безграмотности трудно.

Зато в литовском есть "jos".

И как только кто-то догадался, что новгородский язык не русский, а
западно-славянский, грамоты сразу стали написаны грамотно, просто на другом
языке.

Правда, не соглашусь с автором о "100\% грамотности", и пример с СМС - просто
супер. Грамоты это и есть СМС XV века!!! И ошибок в них ДОЛЖНО столько же

\iusr{Ипполит Воробьянинов}

Владимир, в то время русского языка в нашем понимании ещё не было. Язык в
каждом княжестве отличался от другого. Они, конечно были похожи. Люди друг
друга понимали. Но было очень много диалектов. Это не удивительно. Ибо люди
жили всю жизнь на одном месте. В другие края не ездили, ну, кроме купцов,
которых совсем не много. Ну, ещё совместные военные походы. В них тоже было
задействовано не так много людей. Современный язык для всей страны появился
лишь где-то в 19 веке. Увеличилась мобильность населения. Крепче стала
центральная власть. Так что новгородский язык тогда отличался от тверского или
владимирского не больше, чем те друг от друга.


\iusr{Aborigen СССР}

Ипполит,

Гавариш "русСкого языка в нашем понимании ещё не было"?

А какой тогда был? Можэт ты тагда жыл и на нём гаварил???

"Но было очень много диалектов." Каких? На пример хоть не много, а десяток воспроизведи для общево развития!

Выдумки свои сочиняеш людям на смех!!!

\iusr{nice.ninel}

Владимир, ещё в 60 тых годах в каждой деревне был свой говор. И даже недалёкий
пример, в 90 году из Казахстана приехала в глухую деревню Алтайского края и
встретилась с незнакомыми словами на бытовом уровне, так что не смешите с"
разными" государствами

\iusr{Владимир}

Ипполит, То что вы пишите это настолько очевидно, что как то и обсуждать не
ловко. Но я поясню написанное мною. Русский язык Киева, это один язык. Русский
язык Владимиро-Суздальской Руси, а в описываемый период Московского княжества
это другой язык. Русский язык Великого Новгорода это совсем не Русский язык
Московского княжества. Вот об этом я и пишу.

Можно было бы написать киевский язык, московский язык, новгородский язык, но
тогда это бы было бы совсем не понятно большинству читающих, они также, как вы
считают что это был некий усреднённый Русский язык, который все понимали, такой
своеобразно лингвистический "сферический конь, в вакууме".

Однако исторические описания говорят об обратном. Московские князья для общения
в новгородскими послами приглашали "литовских толмачей", что переводится, как
переводчиков из ВКЛ (Великого княжестве Литовского). Были ли они предками
литовцев или белорусов неизвестно.

И кстати, отправляясь на Переяславскую раду, московские бояре тоже брали с
собой "литовских толмачей". Здесь подозреваю это были уже малороссы или
черкассы.

\iusr{Яков POIZON}

А кто может утверждать, что новгородские грамоты написаны грамотно. Грамматики,
в современном понимании этого слова, тогда не существовало. Писали, как
говорили, не разделяя слов, не ставя ни запятых, ни точек. Современные СМСки
порой ничуть не грамотнее.

Новгородская земля не была Русью. Только в 15 веке известные московские князья
(два Ивана), частью истребив коренные новгородские рода, частью выслав их в
центральные области, заселив Новгород московскими людьми, присоединили Новгород
к Великому княжеству московскому (тогда еще не России).

Так же глупо называть Старую Ладогу столицей Руси. Эта крепость была всего лишь
резиденцией "князя". Это был наемник из варягов, который не имел никаких
наследственных прав в Новгородской земле. Русью (Rootsi) финны называли всех
германоязычных скандинавов. Финны и по сей день называют так шведов. Это слово
означало тогда не национальность, а род занятий - варяг (Varanger).

После смерти Рюрика новгородцы не стали «подписывать контракт» ни с его
малолетним сыном, ни с Олегом, военачальником. Видимо полагали, что Олег
склонен к узурпации власти. И не зря полагали. Олег отправился вниз по Днепру,
беря города. Наконец, жестоким обманом захватил власть в Киеве. Вот там он и
решил обосноваться. Киев и есть «мать городов русских», т.е. первая столица
Руси.

Берестяные грамоты хорошо сохранились во влажной земле новгородчины. Этим и
объясняется большое количество находок.

\iusr{kostik k}

Яков POIZON, грамотно,значит по Русски.Тогдаипрелки писали так,ка слышится -
так и пишется! Измени хоть одну букву и все слово поменяет
смысл.Например,"свабода"- это достижение Сварги небесной,божественной
бесконечности."Свобода"- свод,ограничение . Вот так нам и поменяли на
противоположные смыслы все слова "гусские" лингвисты...
\end{itemize}
