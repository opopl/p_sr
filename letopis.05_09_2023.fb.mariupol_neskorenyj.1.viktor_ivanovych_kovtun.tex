%%beginhead 
 
%%file 05_09_2023.fb.mariupol_neskorenyj.1.viktor_ivanovych_kovtun
%%parent 05_09_2023
 
%%url https://www.facebook.com/100066312837201/posts/pfbid031zL7gMYxBmX3PPjt98KzCMm8cDVC6TRWVujMe4VeSqBya4QwFbE5uRkgRH5HTEYjl
 
%%author_id mariupol_neskorenyj
%%date 05_09_2023
 
%%tags 
%%title Віктор Іванович Ковтун
 
%%endhead 

\subsection{Віктор Іванович Ковтун}
\label{sec:05_09_2023.fb.mariupol_neskorenyj.1.viktor_ivanovych_kovtun}

\Purl{https://www.facebook.com/100066312837201/posts/pfbid031zL7gMYxBmX3PPjt98KzCMm8cDVC6TRWVujMe4VeSqBya4QwFbE5uRkgRH5HTEYjl}
\ifcmt
 author_begin
   author_id mariupol_neskorenyj
 author_end
\fi

✨️Ще один учасник проєкту \enquote{Маріуполь нескорений} -  художник,
науковець, педагог  Віктор Іванович Ковтун.

🖼  Народний художник України, професор кафедри малюнку Харківської державної
академії дизайну і мистецтв  Віктор Ковтун з перших днів війни разом з
рятувальниками перебував  у місцях найбільш гарячих подій свого міста. На своїх
полотнах він відобразив хроніку війни у Харкові, зокрема героїчну роботу
рятувальників. Сюжети брав  з світлин, які сам робив на місті страшних
руйнувань. Митець від лютневого вторгнення написав понад 250 картин. Героям
своїх картин у подарунок  він робить копії, а кошти від проданих картин
передаються  на спецтехніку для ДСНС.👏

✅️На полотнах Віктора Ковтуна зображений поранений, змучений Харків, зруйновані
будівлі, біль втрат, страждання, самотність. Художник розповідає: \enquote{Приліт до
Харківської обладміністрації зачепив частину моєї квартири та майстерню, вибито
шибки, пошкоджено полотна. Ми всі тут переживаємо  це все. Як почесний
громадянин Харкова я нікуди не збирався їхати. Адже ми там, де ми є}. 

🏆  За найкраще висвітлення військової тематики у творах мистецтва  йому
вручили  премію ім. Богдана  Хмельницького. 

📍  Маріупольці пам'ятають. що Віктор Іванович Ковтун був постійним учасником
Міжнародних Меморіалів ім. А. І. Куїнджі у нашому місті, дарував свої картини
художньому музею. Тому і трагедія Маріуполя не могла  не викликати біль і
смуток у серці  митця.😓  

🗓Картини відомого українського художника можна побачити на виставці
\enquote{Маріуполь нескорений} до 21 вересня поточного року у Київській галереї
мистецтв \enquote{Лавра}.

Департамент культурно-громадського розвитку Маріупольської міської ради \par
Diana Tryma \par
Галерея мистецтв \enquote{Лавра} \par
КУ \enquote{Маріупольський краєзнавчий музей} \par
Маріупольська міська рада \par
Костянтин Чернявський \par
Група \enquote{Національна Спілка Художників України} \par
Місто Марії \par
\#Маріупольнескорений \#виставка \#культурнадеокупація \#Маріуполь \#Київ \#художники\par
