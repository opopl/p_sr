% vim: keymap=russian-jcukenwin
%%beginhead 
 
%%file 14_10_2021.fb.miheev_vladislav.1.banalnost_cifrovogo_zla
%%parent 14_10_2021
 
%%url https://www.facebook.com/vladislav.mikheev.5/posts/4538392932894113
 
%%author_id miheev_vladislav
%%date 
 
%%tags chelovek,internet,obschestvo,psihologia,zlo
%%title Банальность цифрового зла
 
%%endhead 
 
\subsection{Банальность цифрового зла}
\label{sec:14_10_2021.fb.miheev_vladislav.1.banalnost_cifrovogo_zla}
 
\Purl{https://www.facebook.com/vladislav.mikheev.5/posts/4538392932894113}
\ifcmt
 author_begin
   author_id miheev_vladislav
 author_end
\fi

Банальность цифрового зла

Устойчивое развитие - замечательная идея и ценность. Такая же как, например,
социальная справедливость и всеобщее равенство. 

Жаль только, что при воплощении в реальную жизнь эти идеи дают на выходе
жутковатые антиутопии  - статистика за прошлый век накоплена.

В природе нет никакого равенства и справедливости.

 Точно так же нигде не встречаются идеальные квадраты или шары - все это плод
 абстрактного мышления человека. 

Точно так же нет в природе и никакого устойчивого развития.

Антиутопические перекосы возникают именно тогда, когда пытаются натянуть
человеческую абстракцию на природную реальность.

На самом деле, развитие - принципиально неустойчивый процесс. 

Устойчивость и развитие - понятия взаимоисключающие.

Бытие - катастрофично. Попытка исключить из него катастрофичность - эта попытка
создать внутри бытия зону свободную  от бытия - небытие или антибытие.

Природные циклы, эволюционные процессы, рождение и смерть - все  это включает
неустойчивость и кризис как необходимый элемент, без которого бытие попросту не
бытийствует.

Напротив, антиутопии стремятся остановить время, объявить "конец истории",
отменить развитие через кризис.  Но эта устойчивость застывшего бытия  - всегда
временная и кажущаяся.

И, вероятно, неким образом тоже вписана в природную цикличность.

Предыдущие антиутопии проповедовали "новые подходы" к деньгам и частной
собственности, делали ставку на научно-технический прогресс. 

Устойчивое развитие предполагает, что искусственный интеллект и технологии
блокчейна сотрут границы между государственным (общественным) и частным. 

 Цифровой и финансовый надзор придет на смену партийной бюрократии и
 сотрудникам спецслужб.

	Даже "новая" идея финансовой системы, которая должна обеспечить устойчивое
	развитие, сильно напоминает  "второй контур" финансовой системы плановой
	советской экономики.

Выходит, что СССР - это не прошлое, а будущее мира? Но уже на новом
технологическом этапе и в рамках глобального проекта, а не 1/6 части суши.

Можно не сомневаться, что это будущее наступит в ближайшие годы. 

Да, несомненно, это может дать толчок грандиозным инфраструктурным проектам и
глобальной модернизации. Но есть и обратная сторона.

Регламентации и контроль -  для рынков и собственности,  для производства и
потребления, для поступков и убеждений, для друзей и врагов. 

Что такое "хорошо" и что такое "плохо" никто больше не вправе решать сам для
себя ни по одному вопросу. Есть общественная польза и она - высшая ценность.
Поэтому в интересах всех вам будет предписано, как правильно  потреблять и
думать, как зарабатывать и тратить.

На высшую бюрократию эти правила, естественно, распространяться не будут.
Принцип статусного потребления, который на самом деле  является одним из
источников неустойчивости, ни одному, даже самому устойчивому и справедливому
обществу из реальности вычеркнуть не удастся.

При этом сомнения в правильности официального курса - тоже будут запрещены,
поскольку нарушают устойчивость.

"Учение Маркса всесильно, потому что оно верно". Вопрос "почему?" граничит с
преступлением против общества, поэтому реальное, внеидеологическое мышление под
запретом.

Некоторым такое будущее может прийтись не по вкусу. Но большинство человечества
примет новые правила. Ведь они - в интересах большинства и более того,  в
интересах планеты и всего живого на ней!

Страдания отдельных индивидов - это статистическая погрешность -  на пути к
правильным Целям "щепки" неминуемо полетят. Помниться, однажды миллионы уже не
вписались в строительство коммунизма в отдельно взятой стране, потом они не
вписались в рынок. Теперь не впишутся ещё и в устойчивое развитие...

Но не думайте, что ваша личная позиция поможет человечеству избежать этой
трансформации. "Капитализм стейкхолдеров"  скорее всего явление закономерное,
так же  как, по Марксу, закономерен переход от классического капитализма к
коммунизму.

Отвертеться от него не получится. Представляется, что глобальная антиутопия -
это закономерный  "частный случай" циклической катастрофичности бытия. Это
всего лишь точка на амплитуде колебаний единой космической вибрации. 

В этом смысле все наши абстракции, концепции устойчивости или неустойчивости
мало что значать. Цифровые технологии и искусственный интеллект ставят перед
собой амбициозную "устойчивую"  задачу - контроля над случайностью и
необходимостью. То есть, над  бытийственными  свойствами той же самой
Вселенной, порождением которой является и сам искусственный интеллект. Минувшие
культурные эпохи усмотрели бы в этом самое оголтелое люциферианство или "новый
дивный мир".

Нам же,  достаточно помнить, что времена не выбирают... Устойчивое развитие -
это реальность антиутопии, в которой нам предстоит жить буквально завтра.

Мир безвольно надевший маски,  безропотно принявший реальность локдаунов и
вакцин,  внимающий Грете Тумберг и апокалиптике климатических прогнозов,
вполне к этому готов. 

Человечество, измученное неопределенностью,  радостно заплатит за спасительную
устойчивость тем, что обитает где-то на вершине пирамиды Маслоу...

Формально, внешне - выбора у нас нет. Но у нас есть внутренний выбор:
нравственного  и интеллектуального соучастия или несоучастия в антиутопии. 

"Банальность зла", описанную Ханной Аренд, которая исследовала  психологию
"маленького человека" внутри большой тоталитарной системы, никто не отменял.
Если на город надвигается чума, опасно и недальновидно встречать ее
апплодисментами, желательно сделать все, чтобы ею не заразиться. Чтоб потом, на
следующем цикле исторического развития,  осуждая цифровой троцкизм, цифровые
воронки  и цифровой ГУЛАГ,  не удивляться: кто же это написал 300 миллионов
цифровых доносов?

\ii{14_10_2021.fb.miheev_vladislav.1.banalnost_cifrovogo_zla.cmt}
