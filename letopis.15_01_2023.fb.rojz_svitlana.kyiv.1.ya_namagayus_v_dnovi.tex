%%beginhead 
 
%%file 15_01_2023.fb.rojz_svitlana.kyiv.1.ya_namagayus_v_dnovi
%%parent 15_01_2023
 
%%url https://www.facebook.com/svetlanaroyz/posts/pfbid02H4KUqv2rrcgCkwBpXGWybj49DZwjwGop73U7LeTTuwkaV8drjyZYn2pR5ivymb9rl
 
%%author_id rojz_svitlana.kyiv
%%date 15_01_2023
 
%%tags 
%%title Я намагаюсь відновити в пам'яті вчорашній день
 
%%endhead 

\subsection{Я намагаюсь відновити в пам'яті вчорашній день}
\label{sec:15_01_2023.fb.rojz_svitlana.kyiv.1.ya_namagayus_v_dnovi}

\Purl{https://www.facebook.com/svetlanaroyz/posts/pfbid02H4KUqv2rrcgCkwBpXGWybj49DZwjwGop73U7LeTTuwkaV8drjyZYn2pR5ivymb9rl}
\ifcmt
 author_begin
   author_id rojz_svitlana.kyiv
 author_end
\fi

Я намагаюсь відновити в пам'яті вчорашній день. Бо сьогодні я усвідомила, що
він в відчуттях зовсім фрагментований. І я й сьогодні в стані ступору й
оніміння. Сьогодні вночі вперше за весь час лежала в ліжку і тремтіла. І
боялась поруч із донькою заснути й не почути тривоги. Думала, було б краще,
якби сон був поверхневим. А сьогодні вранці ледь прокинулась. І, наче,
відходила від наркозу.

Зловила себе на тому, що накриваю на стіл і кладу доньці котлету в чашку
замість тарілки. Засміялась з себе і "прокинулась". 🙂

Що нам зараз важливо тримати в фокусі - часу і можливостей нервової системи на
відреагування, опрацювання і відновлення у нас (у багатьох) вже мало чи немає.
І таке можливо - якщо раніше у нас була на все швидка реакція - дієвістю,
усвідомленістю станів та задач, зараз це може бути вже реакція ступору,
загальмованісті, оніміння. Сонливісті. І відчуття вразливості. (Важливо бути ще
обережнішими із всіма словами та коментарями, рішеннями та реакціями). Це
нормально. 

Але - якщо найближчими днями будуть знов удари (а вони ж можуть бути) нам буде
потрібна наша притомність та реактивність. 

Тому - нам так важливо для самозбереження і безпеки наших дітей - відволікатись
хоч трохи від новин, нашій тілесності потрібно тепло, і тепла їжа, рух. Нам
важливо, якщо є можливість, йти до людей, в будь-який  можливий момент -
відпочивати.

.... Я подумала, що сьогодні мене геть не турбує багатогодинне відключення
світла. Наче це взагалі неважливо. Подивилась в дзеркало - лице застигле, і
волосся, що відросло з минулого походу до перукаря - зовсім сиве. маківка мов
притрушена снігом. І пригадала, як вчора, коли ми їхали додому повз наш ліс - я
попросила зупинитись. Думала, може наодинці, хоч кілька хвилин пореву. Зайшла
трохи - до таблички "обережно, міни". Дивувалась такій змерзлій казковій красі.
Все, що бачилось - велично сріблясте. Але замерзле, нерухоме.

А поверх цього, уявіть, як у кіно - живі, наче весняні переспіви птахів.

...А їхали ми з дитячого дня народження - доньку запросили до подружки. В
торговельному центрі діти при ліхтариках святкували в ігровій. Смаколики не
змогли приготувати, бо в центрі не було світла. А коли пролунала тривога - всіх
попросили звільнити простір. І діти зібрались далі святкувати в безпечному
місці.

А до цього - я сумнівалась, чи їхати взагалі на свято - бо було ж зрозуміло, що
вдень будуть ще обстріли. Але для дитини так важливі будь-які можливості
радості.

А до цього - після вибуху вранці розбудила доньку, з одягом та її іграшками
перевела вниз. Світла не було, батько топив пічечку. А я пішла шукати кішку,
щоб була з нами. Повертаюсь з кішкою на руках і застигаю від свисту... Уявіть,
тривога. Тільки був вибух. І свист зовсім поруч. А це чайник на плитці почав
кипіти. Вже нервово сміялась, коли зрозуміла, що це "хороший звук". Тепер всіх
попереджаю про звук чайника. 

Свист чайника і котлета в чашці - будуть моїми внутрішніми мемами. 🙂

А ще - у мене звучить всередині січневий весняний спів птахів в нашому
засніженому занімілому лісі вчора.  Це надія на життя, що зберігається
всередині, навіть, коли зовні все застигло. Це Життя, яке точно перемагає. 

Піду робити доньці і собі какао - це її улюблений смак і аромат затишку і
безпеки. Какао- терапія 

Обіймаю, Родино ❤️ як вірю в Перемогу
