% vim: keymap=russian-jcukenwin
%%beginhead 
 
%%file 16_12_2017.news.ua.argumentua.1.leontovych_mykola_chekist_smert
%%parent 16_12_2017
 
%%url http://argumentua.com/stati/kompozitora-nikolaya-leontovicha-zastrelil-chekist
 
%%author 
%%author_id 
%%author_url 
 
%%tags leontovych_mykola,death
%%title Композитора Николая Леонтовича застрелил чекист
 
%%endhead 
 
\subsection{Композитора Николая Леонтовича застрелил чекист}
\label{sec:16_12_2017.news.ua.argumentua.1.leontovych_mykola_chekist_smert}

\Purl{http://argumentua.com/stati/kompozitora-nikolaya-leontovicha-zastrelil-chekist}

\ii{16_12_2017.news.ua.argumentua.1.leontovych_mykola_chekist_smert.pic.1}

«Родные сидели в соседней комнате со связанными руками и слышали, как Грищенко
кричал на полусознательного Леонтовича»... Как жил и погиб создатель
легендарного «Щедрика».

«Отец ни на минуту не бывал днем дома. Приходил только поздними вечерами. Мы с
мамой до тех пор не спали, прислушиваясь к знакомым шагам по темной пустой
улице. Только услышав его быстрый шаг, мама заставляла нас ложиться спать. Сама
пыталась не выдать волнения. Отец, увлеченный своими произведениями, не боялся
ночных улиц в те неспокойные времена», — вспоминает о жизни в Киеве весной
1919 года Галина, дочь композитора Николая Леонтовича.

«Леонтович внешне выделялся своим изяществом. Не любил носить служебной формы и
всего казенного, — пишет дирижер Николай Покровский в очерке \enquote{Со страниц
прошлого}. — Во время выступлений всем своим видом уже настраивал и
исполнителей, и зал. Одет в темную визитку, рубашку с твердым стоячим
воротником и хорошим галстуком, он, худощавый и стройный, с большим лбом и
черной бородкой-эспаньолкой и коротко подстриженными усами, словно возвышался
над стихией нежных звуков. Николай Дмитриевич устойчиво боролся с нелегкой
жизнью учителя пения того времени. Его семья еле сводила концы с концами. Но он
редко бывал печален и мрачен, всегда шутил и надеялся на лучшие времена».

\begin{leftbar}
  \begingroup
    \em\Large\bfseries\color{blue}
В тему: \href{http://argumentua.com/stati/rasstrelyannoe-vozrozhdenie-neizvestnaya-istoriya-pisatelei-iz-rassekrechennykh-arkhivov-kgb}{Расстрелянное возрождение Неизвестная история писателей из рассекреченных архивов КГБ}
  \endgroup
\end{leftbar}

Живут на Баггоутовской улице в двухкомнатной квартире. Ее оставила матери
сестра, которая с семьей уехала в деревню. Помещение расположено почти на краю
города. Поэтому часто гуляют заросшими оврагами неподалеку, слушают пение птиц.

«Чтобы не срывать учение, репетиции хора Николай Дмитриевич проводил во
внеурочное время. Занятия начинались в 9 часов. Леонтович приходил в 6-7-й,
когда мы еще спали, — пишет в статье \enquote{Мы все его любили} выпускник Киевской
учительской семинарии Иван Барчук. — Он заходил в спальню, снимал с нас одеяла
и призывал на репетицию. В 8 пили чай. Он с нами тоже садился. Звучала песня.
Не было случаев, чтобы его кто-то не слушал или высказывал недовольство».

Кроме семинарии, преподает хоровое дирижирование в музыкально-драматическом
институте имени Николая Лысенко и в народной консерватории. Как инспектор
музыкального отдела Народного комиссариата просвещения занимается
государственным украинским оркестром и национальной хоровой капеллой. Читает
лекции в школах и гимназиях.

\ii{16_12_2017.news.ua.argumentua.1.leontovych_mykola_chekist_smert.pic.2}

«Лишения истощали нашу семью, — писала в очерке об отце Галина Леонтович. —
Ясно было, что надо возвращаться в Тульчин (городок в Винницкой области. —
Країна). Но все мы чувствовали, как тяжело ему будет порывать с Киевом. Теперь,
когда его талант входил в расцвет, когда он работал с близкими ему музыкантами,
имел с ними общие интересы, чувствовал их поддержку в своей творческой работе,
возвращение в Тульчин было бы драматическим событием в его жизни».

\ifcmt
  ig https://lh3.googleusercontent.com/wa4pMYfyhzz70PE-WoG0r7_rYMhhnsIBl_zRPR9GixhleBrBvBDZAaz6jCq6HzcRQttzhbTxXap7lxJ8o1v0Nf0C57vq2GJtTgSsl3NfgTiccAvKMaOI1Toi_zFaayH9Cmwow2zL
	@caption Николай Леонтович — сидит в третьем ряду четвертый справа — среди участников Первого украинского хора в Киеве, 1919. Фото предоставлено Анатолием Завальнюком
	@width 0.6
	@wrap center
  %@minipage 0.4
  %@wrap \parpic[r]
\fi

Жена Клавдия с дочерьми — 17-летней Галиной и 5-летней Надеждой — покидают Киев
летом 1919-го. Муж договаривается о местах в товарном поезде. Сажает их и идет
на работу.

«Осенью 1919 года Николай Дмитриевич в ветхом летнем пальтишке на плечах и
неуклюжей шапочке. Сильно осунувшийся, простуженный, пешком пришел из Киева в
Тульчин и снова поселился здесь», — пишет его друг Игнат Яструбецкий.

\ifcmt
  ig https://lh3.googleusercontent.com/i0a5SZEBHM64mUNSFAmi3ym_vkVXuLvTl5S4fQCfuV9fsuHwZ-9iZBTATcAFmMWKxdhIzdUYjtJ9SBFDQE1YueLCaoitJYU3ITf6LrdeGFBMNA5jxO42cGOpEpNtGEaijT4UEQwl
	@caption Николай Леонтович с женой Клавдией и дочерью Галиной, 1905. Фото: Винницкий областной краеведческий музей
	@width 0.6
	@wrap center
\fi

Композитор берется за учительство в созданной на базе епархиального училища
трудовой школе. Руководит самодеятельными хорами и выступает с концертами в
воинских частях. Он поддерживает связи с киевскими коллегами и надеется на
возвращение в город. С радостью откликается на приглашение композитора Кирилла
Стеценко приехать в Каменец-Подольский для организации украинских хоров.

«Ночь. Сильный ветер с холодным дождем. Военные события. Город набит беженцами,
которые прибывают и прибывают. Наш дом стоит у широкой дороги в город. Мы не
можем отказать голодным, измученным людям, которые остались без крова. И вот
тогда кто-то сказал, что здесь есть Леонтович, — писала в очерке \enquote{Минуты из
жизни} Ольга Приходько, сестра хорового дирижера Александра Приходько. — Моя
сестра Елена, которая несколько лет работала с ним в девичьей Тульчинский
школе, сразу узнала его, хотя он был таким изможденным, что еле стоял ».

Хозяева приглашают композитора в комнату отдохнуть.

— Я такой грязный, с болотом. Не пойду. Благодарен, что имею крышу над головой
и теплый дом, где не дует злой ветер и не сечет в глаза холодный дождь, —
отказывается.

О появлении композитора слышат другие гости. Все вместе убеждают его зайти,
хотят пообщаться. Он отказывается.

«Нам было интересно с ним поговорить. Но видели, что ему следует немедленно
лечь и отдохнуть, — дальше вспоминает Ольга Приходько. — Я сделала ему постель
на диване в столовой. Мы все разошлись по своим углам. А утром его уже не было.
Когда и куда он ушел, не сказал».

В конце октября 1920 года Леонтовича в Тульчине посещают композитор Кирилл
Стеценко и поэт \textbf{Павло Тычина}. Оба работали в странствующей капелле
Днепровского союза кооперативных обществ. После выступления в Одессе коллектив
по железной дороге прибыл на станцию Вапнярка — теперь Томашпольский район
Винницкой области. Члены капеллы узнали, что в 18 километрах живет автор
«Щедрика» и других произведений из их репертуара. Решили завернуть к нему.

«Николай Дмитриевич прислонился головой к стене, слушает. Первым хлопает в
ладоши», — записал в дневнике о концерте в Тульчине Павло Тычина.

Леонтович показывает и исполняет фрагменты своей оперы «На русалчин Великдень».
Кирилл Стеценко хвалит и советует как можно быстрее ее закончить, чтобы
поставить в Киеве.

Леонтович хочет написать народно-фантастическую оперу в трех действиях. Для
создания либретто обращается к своей ученице Надежде Танашевич, которая живет в
селе Стражгород — теперь Теплицкий район Винницкой области. Девушка сначала
отказывается. Соглашается, когда композитор говорит, что главное — писать
правдиво, и обещает поддержку.

— Она напела мне много песен, — говорит о Надежде бывшему коллеге Акиму Греху.
— Может, полюбила старика-лысого. Пусть любит, будет лучше писать текст. Любовь
— поэзия, это лучшие переживания каждого человека.

Осенью 1920-го семье Леонтовича постоянно не хватает еды и одежды. На
религиозные праздники отправляют дочь Галину погостить к деду в Марковку.
Рассчитывают, что она принесет каких-то продуктов.

— Надо мне, Клава, как-то добраться до отца самому, — говорит Николай Леонтович
жене в начале 1921 года.

— Как же ты будешь добираться по такому холоду?

— Во время ходьбы не замерзну. Не могу же я спокойно смотреть, как дети
недоедают, а купить кусок сала, масла или еще чего денег не хватает. От отца
свинины, муки и каких-нибудь круп можно привезти. И картошку, и дров.
Рассчитывает зайти к Надежде Танашевич, которая живет недалеко от отца. Хочет
обсудить дальнейшую работу над совместным произведением. По дороге заворачивает
к Акиму Греху.

\begin{leftbar}
  \begingroup
    \em\Large\bfseries\color{blue}
В тему: \href{http://argumentua.com/stati/odna-pulya-v-dve-golovy-75-let-nazad-kommunistami-rasstrelyana-elita-ukrainskoi-natsii}{%
Одна пуля в две головы. 75 лет назад коммунистами расстреляна элита украинской нации}
  \endgroup
\end{leftbar}

«Это было после Рождества 1921 года. Я очень удивился, когда увидел во дворе
направляющегося ко мне Леонтовича, — пишет Грех в воспоминаниях. — Он был одет
в старое пальто. На голове — оригинальная шапка, которую сшила его жена из
старого одеяла. На руках рукавицы на один палец — тоже работа жены. А штаны —
серо-черного цвета с большими фиолетовыми пятнами. Он еще нес на палочке
завязанный в большой платок гостинец для меня — калачи».

— Ох, мама! Эта линия жизни совсем кончилась, зачем же ты неправдиво говоришь,
— возмущается Надежда Танашевич, увидев, как ее мать гадает по руке композитору
Николаю Леонтовичу и говорит, что его линия жизни «вроде бы скоро кончается».

— Значит, когда буду от вас идти и из леса выйдет бандит меня убивать, то скажу
ему: «Э, нет! Стой. Матушка Танашевич мне еще нагадывала долго жить», —
рассмеялся гость.

Надежда ведет его в другую комнату и просит больше так не шутить.

— На днях давал концерт в казармах, — говорит Леонтович. — На пианино забыл
свой портфель с документами. Меня сейчас же «пригласили на чашку чая», забрали
документы, проверили. Потом — вернули. Жду результатов, а может, и убьют.

Через четыре дня в него стреляет из обреза охотничьего ружья агент Гайсинского
уездного ЧК («чрезвычайная комиссия». — Країна) Афанасий Грищенко. Произошло
это в родительском доме в селе Марковка. Чекист попросился переночевать. С ним
— ездовой Федор Грабчик, крестьянин из Киблич — теперь Гайсинский район
Винницкой области. Хозяева согласились.

В разговоре выясняется, что гость приехал бороться с «бандитизмом». Хвастается
сетью информаторов по селам, от которых знает обо всем, что здесь происходит. В
полночь мужчины ложатся спать в одной комнате. Во второй — мать композитора
Мария, сестра Виктория и дочь Галина.

Утром, в 7 часов раздается выстрел.

— Папа, папа! Что это? Взрыв? — первым отзывается Николай Дмитриевич.

Отец бросается к сыну. Тот хочет подняться с постели, но не может. На правом
боку — рана. Кровь заливает простыню. Напротив стоит босой, раздетый до белья
чекист. Держит обрез. Достает из него гильзу и закладывает новый патрон.

— Ступай отсюда! — кричит на хозяина и выталкивает за дверь. Приказывает
Грабчаку связать всем руки. Для этого снимает полотенце со стены и кромсает
найденную юбку.

«Родные сидели в соседней комнате со связанными руками и слышали, как Грищенко
кричал на полусознательного Леонтовича. Они не могли ничего сделать. Тем
временем убийца требовал золото и деньги. Забрал столовые ложки, часы и
найденные в кошельке и шкафах деньги. Набросил на себя кожух хозяина, прихватил
сапоги его сына и выбежал из дома», — пишет Яструбецкий.

\begin{leftbar}
  \begingroup
    \em\Large\bfseries\color{blue}
				В тему: \href{http://argumentua.com/stati/chekisty-byli-maroderami-nkvd-v-1937-m-repressiroval-grazhdan-radi-ikh-kvartir-i-sberezhenii}{Чекисты были мародерами: НКВД в 1937-м репрессировал граждан ради их квартир и сбережений}
  \endgroup
\end{leftbar}

На крики отца совпадают люди. Одни бросаются за убийцей, другие — за врачом в
Теплик, третьи — к пострадавшему.

— Воды, — просит Николай Дмитриевич. Когда подают, напиться не может. — Света.
Дайте света. Папа, умираю.

Через несколько минут его сердце перестает биться.

Крестьяне догоняют подводу Грабчака, но Грищенко в ней нет. Его преследует
милиция. Тот отстреливается, ранит милиционера и убегает.

«Убывшему в служебную командировку районному информатору Грищенко полагать в
таковой и исключить с провиантского и чайного довольствия с 12 сего декабря, —
подтверждает его должность найденная в архиве в Виннице справка. — Выданные ему
5000 аванса на секретные расходы выписать в расход по приходно-расходному
журналу».

На похороны из Тульчина приезжают жена Клавдия и дочь Надежда. Метет, поэтому
дорога дается им тяжело. Возле дома Леонтовичей собираются люди. Тело покойного
кладут в гроб из тополя. Хоронят босиком, потому что не во что обуть — сапоги
забрал убийца. 25 января около 17:00 гроб опускают в могилу.

На следующий день из Старжгорода привезли на могилу венок от Надежды
Танашкевич. На нем была надпись: «Вечная память. Спи спокойно».

«Леонтович убит. Траур сухой у меня. Дико, — записал в дневнике 27 января
1921-го поэт Павло Тычина. Позже он еще вспоминает о трагедии: «Выписываю ноты
для хора. „Музыка Леонтовича“. Странно. Вся Украина спела Леонтовича. А что
ему, Николаю Дмитриевичу? Лежит в гробу, никому не нужен. Тихим будешь — тебя
убьют. Сильным станешь — должен убивать. Вот в чем логика жизни».

«Пуля чекистского посланца сознательно целила в сердце признанного носителя
духовности нации — Николая Леонтовича. Это был знак страшной казни, что ждет
каждого самостийника, не угодного системе», — писала в одной из публикаций
искусствовед Валентина Кузык.

160 обработок народных песен создал Николай Леонтович. Самые известные —
«Щедрик», «Дударик», «Пряля», «Козака несуть», «Зашуміла ліщинонька», «Ой, з-за
гори кам'яної». Из них 50 произведений для детского хора издал в учебных
пособиях «Нотная грамота» и «Сольфеджио».

\subsubsection{Работал на Донбассе}

1877, 13 декабря — Николай Леонтович родился в семье священника в селе
Монастырок Брацлавского уезда — теперь Немировский район Винницкой области.

Дед и прадед также были священниками. Отец Дмитрий Феофанович играл на цитре,
балалайке, гитаре, скрипке. Мать Мария Иосифовна — хорошо пела. Брат Александр
и сестра Мария стали профессиональными певцами, Елена училась в классе
фортепиано Киевской консерватории, Виктория играла на нескольких инструментах.

1892 — вступает в Подольскую духовную семинарию в Каменце-Подольском, которую
заканчивает в 1898 году.

1901 — издает первый сборник песен Подолья. Через два года выходит второй — с
посвящением Николаю Лысенко.

Работает учителем двухклассной школы в селе Чукив. Играет на скрипке и флейте.
Организует школьный оркестр. За собственные средства покупает пять скрипок,
виолончель, флейту, корнет и тромбон. При месячной зарплате учителя 27 руб за
каждый инструмент отдает от 25 до 40 руб.

1902 — работает учителем музыки в Тыврове — теперь райцентр Винницкой области.
Здесь знакомится и женится на Клавдии Жовткевич. Она на 2 года старше и
приехала с Волыни. Вдвоем работали в Тыврове, затем — в Виннице. Там в 1903
году рождается дочь Галина.

1904 — уезжает работать на Донбасс преподавателем пения и музыки в
железнодорожной школе. Живет с семьей на станции Гришино (ныне город Покровск
Донецкой области. — Страна) в бараке для железнодорожников.

Во время революции 1905 года организует хор рабочих, который выступает на
митингах. Деятельность Леонтовича привлекает внимание полиции. Он вынужден
вернуться в Тульчин. Преподает музыку и пение в епархиальном училище для
дочерей сельских священников.

С 1909-го — берет уроки композиции, полифонии у профессора теории музыки
Болеслава Яворского, которого периодически посещает в Москве и Киеве.

1916 — вместе с хором Киевского университета выполняет свою обработку
«Щедрика».

1919 — во время захвата Киева деникинцами вынужден бежать в Тульчин. Там
основывает первую в городе музыкальную школу.

1919-1920 — работает над народно-фантастической оперой «На русалчин Великдень»
по одноименной сказке Бориса Гринченко.

1921, 23 января — убит агентом ВЧК (Всероссийская чрезвычайная комиссия по
борьбе с контрреволюцией и саботажем. — Країна) Афанасием Грищенко в селе
Марковка — теперь Теплицкого района Винницкой области. Текст рапорта,
раскрывающий имя убийцы композитора, был обнародован в 1997 году.

—

Вита Довбыш, опубликовано в журналу «Країна»

Перевод: «Аргумент»
