% vim: keymap=russian-jcukenwin
%%beginhead 
 
%%file 17_12_2020.news.ua.day_kiev_ua.jakovenko_igor.1.virusy_rusofobii
%%parent 17_12_2020
 
%%url https://day.kyiv.ua/uk/blog/suspilstvo/virusy-rusofobiyi
 
%%author Яковенко, Ігор
%%author_id jakovenko_igor
%%author_url 
 
%%tags 
%%title Віруси русофобії - З кого починати лікування
 
%%endhead 
 
\subsection{Віруси русофобії - З кого починати лікування}
\label{sec:17_12_2020.news.ua.day_kiev_ua.jakovenko_igor.1.virusy_rusofobii}
\Purl{https://day.kyiv.ua/uk/blog/suspilstvo/virusy-rusofobiyi}
\ifcmt
	author_begin
   author_id jakovenko_igor
	author_end
\fi

Тимчасового повіреного у справах РФ Олександра Комарова 12.12.2020 запросили до
МЗС Казахстану, де заступник міністра закордонних справ Марат Сиздиков вручив
йому ноту і висловив подив у зв’язку з висловлюваннями депутата Держдуми
В’ячеслава Ніконова на «Першому каналі».

Річ у тім, що голова комітету Держдуми з питань освіти і науки В’ячеслав
Ніконов ось уже два роки за сумісництвом веде на «Першому каналі» ток-шоу
«Велика гра», де він разом із «політологом» Дмитром Саймсом управляється в
патріотизмі, який В’ячеслав Олексійович розуміє виключно як приниження інших
країн і народів.

Ось і цього разу депутат В’ячеслав Ніконов повідомив росіянам, а заодно казахам
і всьому світові своє цілком екзотичне бачення історії і географії казахського
народу.

«Казахстану просто не існувало», — з поблажливою посмішкою оголосив пан
Ніконов, — «Північний Казахстан узагалі не був заселений». — «Але казахи
існували!» — вирішив втрутитися ведучий іншої програми «Першого каналу»
Володимир Познер, який опинився в студії. — «Казахи існували», — після
хвилинного роздуму неохоче визнав Ніконов. І тут же додав: «Але вони існували
набагато південніше, і, власне, територія Казахстану — це великий подарунок з
боку Росії!»

\ifcmt
pic https://day.kyiv.ua/sites/default/files/20-12-52/27_bogorad.jpg
cpx Малюнок Віктора Богорада
\fi

Потім Ніконов вирішив узяти більший план: «Взагалі, середньоазіатські, ну, там
усе було складно! Держави там національні — їх не було!» Тут у розмову вступив
колега Ніконова, Олексій Пушков, який очолює аналогічний комітет у Раді
Федерації: «Ось на злобу дня — половина України подарована їй Росією!» — «Ну,
як мінімум!» — радісно погодився Ніконов.

Імовірно, доктор історичних наук В’ячеслав Ніконов не в курсі, що держава під
назвою Казахське ханство існувала на території сучасного Казахстану з 1465
року, а Середню Азію він узагалі даремно згадав, оскільки древні осередки
науки, культури і державності там виникли задовго до того, як на території
Росії з’явилися перші міста. Що стосується національних держав, то д.і.н.
Ніконов, мабуть, не в курсі, що їх система в Європі почала складатися тільки
після Вестфальського миру 1648 року.  Втім, після його знаменитого
висловлювання про «гілку арійського племені», яка «спустилася з Карпатських
гір» і «мирно заселила Велику Російську Рівнину, дійшла до Тихого океану і
заснувала Форт Росс», розмовляти з депутатом Ніконовим на теми історії та
географії було б цікаво здебільшого письменникам-фантастам, які шукають сюжети
для романів у жанрі альтернативної історії...

Напередодні того дня, коли депутат Ніконов позбавив Казахстан історії і
частково географії, його колега, віце-спікер Держдуми Петро Толстой, повністю
знищив Україну, позбавивши її не лише державності, а й мови і взагалі
будь-якого окремого існування. На щастя, тільки вербально.  Це сталося в
програмі Тіни Канделакі «Спеціальний гість» на каналі RTVi.  Канделакі
поставила Толстому запитання, чому на російських ток-шоу весь час обговорюють
США й Україну, хоча в країні є безліч своїх проблем.  Відповідь віце-спікера
Держдуми Петра Толстого заслуговує на те, щоб її вибили золотом на мармурі в
стінах Держдуми, МЗС РФ як надгробний напис на цих могилах російського
парламентаризму, дипломатії і в цілому російської державності.

«Проблеми України ми обговорюємо тому, що Україна — це частина Росії.  Проблеми
України ми обговорюємо тому, що це такі самі російські люди, на яких поставили
і довели до кінця певний експеримент, який не пройшов у нас.  Тому це цікаво,
на мій погляд», — пояснив Толстой.

Після чого думка Толстого перемістилася в область медицини.

«Це грубо кажучи... ось уявіть собі, що у тебе дуже-дуже болить нога, і ти
приходиш до своїх друзів і починаєш говорити, що у тебе болить нога. І в
принципі ніхто тобі не дорікне, що ти така суперегоїстична... ну, не можна про
це постійно говорити. Ти постаралася полікувати... Ми говоримо, але поки не
лікуємо. Сподіваємося, що пройде. Гангрена не проходить. Її треба лікувати.
Тому будемо лікувати».

«Україна — це нога?» — про всяк випадок уточнила Канделакі, явно не досвідчена
в політичній анатомії. — «Так, нога!» — рішуче відрізав Толстой. Віце-спікеру
пощастило, що Канделакі не стала уточнювати, якою ж частиною тіла на цьому
анатомічному атласі є Росія, оскільки відомо, звідки ростуть ноги. Замість
цього ведуча запитала, що робити, якщо українці не вважають себе «ногою»,
наївно вважаючи свій народ «іншою людиною». Толстой тут же продемонстрував
знання літератури, згадав гоголівський «Ніс» і пояснив, що «коли частина тіла
вважає себе самостійною, це погано закінчується.  «У України шансу немає
побудувати свою окрему від Росії самостійну державу», — засудив Толстой...

Депутати Петро Олегович Толстой і В’ячеслав Олексійович Ніконов є живими
втіленнями двох пов’язаних між собою явищ — деградації і русофобії.  Кожен із
них — нащадок дуже різних, але великих історичних постатей. Ні, я не порівнюю
великого письменника зі сталінським наркомом, чий підпис стоїть не тільки під
пактом, який став прологом до Другої світової війни, а й під рекордною
кількістю розстрільних вироків.  Але тим не менше, важко заперечувати, що
Молотов був великою політичною фігурою. На відміну від онука. Не кажучи вже про
нікчемного нащадка Льва Миколайовича, який зганьбив свій рід не лише
українофобією, а й печерним антисемітизмом.

Тут уже йдеться про здрібніння навіть не до мишей, а, швидше, до вірусів.
Причому до вірусів «русофобії».  Оскільки саме такі, як депутат Ніконов і
віце-спікер Толстой, є причинами того, що Росію в більшій частині
цивілізованого світу сприймають як джерело постійної загрози не тільки для
найближчого оточення, а й усього людства. І для того, щоб «русофобія» зникла,
треба вилікувати від таких вірусів російський телевізор, Державну думу, Раду
Федерації, МЗС і всю російську державність.
