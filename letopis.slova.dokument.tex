% vim: keymap=russian-jcukenwin
%%beginhead 
 
%%file slova.dokument
%%parent slova
 
%%url 
 
%%author 
%%author_id 
%%author_url 
 
%%tags 
%%title 
 
%%endhead 
\chapter{Документ}

%%%cit
%%%cit_head
%%%cit_pic
%%%cit_text
\enquote{Переселить западных украинцев}. Заметки к \enquote{Плану Ост}.  Одним
из главных \emph{документов} нацистов по перспективам Украины является так
называемый \enquote{план Ост}, который писался уже после оккупации УССР.  В
цельной форме этот план создать не успели. Однако его усиленно готовили - как
по ведомству Розенберга, так и рейхсфюрера Генриха Гиммлера, возглавлявшего СС.
Упоминания об этом плане сохранились в \emph{нацистских документах} - письмах и
докладных записках, что курсировали между немецкими ведомствами, которые
занимались оккупированными территориями.  Главный из этих \emph{документов} -
\enquote{Замечания и предложения по генеральному плану \enquote{Ост}
рейхсфюрера войск СС }. Этот \emph{документ} был подписан 27 апреля 1942 года
Эрхардом Ветцелем - начальником отдела колонизации 1-го главного политического
управления \enquote{восточного министерства} - то есть ведомства Розенберга
%%%cit_comment
%%%cit_title
  \citTitle{22 июня - 80 лет нападения на СССР. Что немцы готовили для украинцев}, Максим Минин, strana.ua, 22.06.2021
%%%endcit
