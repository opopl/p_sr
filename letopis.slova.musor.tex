% vim: keymap=russian-jcukenwin
%%beginhead 
 
%%file slova.musor
%%parent slova
 
%%url 
 
%%author 
%%author_id 
%%author_url 
 
%%tags 
%%title 
 
%%endhead 
\chapter{Мусор}
\label{sec:slova.musor}

%%%cit
%%%cit_head
%%%cit_pic
%%%cit_text
На тему \emph{київського сміття} я написав багато текстів, провів купу акцій,
командою виходили на прибирання, але ситуація у столиці не змінюється. Чуємо
лише чергові обіцянки Віталія Кличка, що колись у Києві з'явиться
\emph{сміттєпереробний} завод. Поділюся з вами 10 фактами, які треба знати про
ситуацію з \emph{відходами} у столиці та за що міському голові має бути
соромно. 1. Київ генерує один мільйон триста тисяч тонн \emph{відходів} на рік.
Це як 58 стадіонів НСК Олімпійський, повністю завалених \emph{сміттям}
%%%cit_comment
%%%cit_title
\citTitle{Сором Кличка. Київ перетворюють на смітник}, 
Єгор Фірсов, blogs.pravda.com.ua, 23.06.2021
%%%endcit

