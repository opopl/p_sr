% vim: keymap=russian-jcukenwin
%%beginhead 
 
%%file 23_01_2022.fb.fb_group.respublika_lnr.1.sorev_sambo_lugansk
%%parent 23_01_2022
 
%%url https://www.facebook.com/groups/respublikalnr/posts/959738651328593
 
%%author_id fb_group.respublika_lnr,zimina_olesja
%%date 
 
%%tags donbass,lnr,lugansk,sambo,sorevnovanie,sport,zhizn
%%title СПОРТ. Спортсмены из ЛНР и ДНР приняли участие в соревновании по боевому самбо в Луганске
 
%%endhead 
 
\subsection{СПОРТ. Спортсмены из ЛНР и ДНР приняли участие в соревновании по боевому самбо в Луганске}
\label{sec:23_01_2022.fb.fb_group.respublika_lnr.1.sorev_sambo_lugansk}
 
\Purl{https://www.facebook.com/groups/respublikalnr/posts/959738651328593}
\ifcmt
 author_begin
   author_id fb_group.respublika_lnr,zimina_olesja
 author_end
\fi

СПОРТ. Спортсмены из ЛНР и ДНР приняли участие в соревновании по боевому самбо
в Луганске

В Луганске 22 января состоялось соревнование по боевому самбо, организованное
активистами проекта «Дружина» Общественного движения «Мир Луганщине». В
соревнованиях приняли участие спортсмены из ЛНР и ДНР. Об этом сообщили в
пресс-службе движения.

\ii{23_01_2022.fb.fb_group.respublika_lnr.1.sorev_sambo_lugansk.pic.1}

Самбо –  это боевой вид спорта, а также система защиты без оружия, на основе
борьбы дзюдо, джиу-джитсу. Участников турнира поприветствовали координатор
проекта «Дружина» ОД «Мир Луганщине» Михаил Голубович и исполняющий обязанности
руководителя проекта «Народная Дружина» ОД «Донецкая Республика» Владимир
Тараненко. Они пожелали ребятам удачи и хороших результатов.

\ii{23_01_2022.fb.fb_group.respublika_lnr.1.sorev_sambo_lugansk.pic.2}

На турнир приехали юноши и парни из разных городов ЛНР и ДНР. Их разделили на
группы по возрасту и весу. После разминки и инструктажа спортсмены боролись на
ринге по три минуты.

\ii{23_01_2022.fb.fb_group.respublika_lnr.1.sorev_sambo_lugansk.pic.3}

Михаил Голубович рассказал, что планировали проводить соревнования давно
потому, что в Республиках ЛНР и ДНР много спортсменов и любителей боевых видов
спорта.

\ii{23_01_2022.fb.fb_group.respublika_lnr.1.sorev_sambo_lugansk.pic.4}

– Все спортсмены показали высокий уровень подготовки. Ребята специально
приехали из разных городов, чтобы побороться за победу, потренироваться и
обменяться опытом. Мы наблюдали, болели за ребят, поддерживали. Спасибо всем, –
сказал Михаил Голубович.

\ii{23_01_2022.fb.fb_group.respublika_lnr.1.sorev_sambo_lugansk.pic.5}

Победителями турнира стали Никита Шилин, Иван Наумов,  Никита Калюжный, Василий
Раю, Ренат Раю, Сергей Реутский, Илья Гурьянов, Иван Михайлюк и лучшим бойцом
стал Василий Раю.

– Спасибо за участие, сегодня вы показали множество интересных элементов. Я
думаю, что каждый из вас получил опыт и всем всё понравилось. Надеюсь, что в
дальнейшем мы будем вместе развиваться, расти и работать на результат, –
отметил Владимир Тараненко.

Победитель в категории 78 кг Иван Наумов из Донецка рассказал, что занимается
боевыми видами спорта четыре года. Ранее занимался кикбоксингом, боксом,
рукопашным боем.

– Соревнования очень понравились! Буду тренироваться усерднее и стремиться к
победам и высоким результатам, – сказал Иван Наумов.

Никита Шилин из города Молодогвардейск занявший первое место в весе 85-90 кг
рассказал, что занимается борьбой три года в клубе смешанных единоборств SKIF
FIGHTERS.

– Мне понравилась организация турнира. Спарринги были захватывающие, наблюдать
было очень интересно, соперники были сильные,  – отметил Никита Шилин.

Призерам соревнований вручили грамоты, кубки, медали и подарочные наборы от
проекта «Дружина» ОД «Мир Луганщине».
