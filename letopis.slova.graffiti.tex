% vim: keymap=russian-jcukenwin
%%beginhead 
 
%%file slova.graffiti
%%parent slova
 
%%url 
 
%%author 
%%author_id 
%%author_url 
 
%%tags 
%%title 
 
%%endhead 
\chapter{Граффити}
\label{sec:slova.graffiti}

%%%cit
%%%cit_head
%%%cit_pic
\ifcmt
  pic https://img.pravda.com/images/doc/c/3/c3d2a14-001---------.jpg
	caption Напис на стіні Софії: \enquote{Блуд створив у святий день}; Фото: із особистих архівів Вячеслава Корнієнка
	width 0.4
\fi
%%%cit_text
Але цінність \emph{графіті} не тільки в цьому. За літописним текстом людини не
видно.  А от графіті дозволяють побачити за літерами людське.  – Приміром,
переживання людини: \enquote{Купріяне, гарний же ти, скільки мені ще даремно
розпалятися}. Або \enquote{того не пиши, того не проклинай, а Івану погано}.
Читаю, а у мене перед очима старий буркун, який всім незадоволений і навіть
нашкрябати йому не дають, тож він жаліється.  Хтось довіряв стіні свої мрії:
Свята Софія, допоможи мені стати художником.  Або прокльони: Козьма тать – ти
вкрав м'ясо. Бодай тобі ноги сплутало. Амінь!
%%%cit_comment
%%%cit_title
\citTitle{Непозбувний лох, стіна-психолог та вічна молитва. Про що говорить Київ на стінах, парканах і асфальті}, 
Олена Струк, www.pravda.com.ua, 02.07.2021 
%%%endcit

%%%cit
%%%cit_head
%%%cit_pic
\ifcmt
  tab_begin cols=3
     pic https://img.pravda.com/images/doc/5/6/56ee2df-5-graphity.jpg
     pic https://img.pravda.com/images/doc/d/9/d9ba77d-4-oleksiy-kashtani.jpg
		 pic https://img.pravda.com/images/doc/a/c/acaa8f7-6-lidia-grigorivna.jpg
  tab_end
\fi
%%%cit_text
Єгор Лєтов та Лідія Іванівна.  Стрункі ряди хрущовок та сталінок у Жовтих Водах
змістом наповнюють розписані стіни.  Між рядками, якими рясніють розписані
будинки, читається різне: ніжність, есхатологічні мотиви, заклик до опору,
зневага до державної системи та до голови місцевого водоканалу.  "Харитонова –
вор!". Словесний самосуд, влаштований комунальній чиновниці, місцеві коментують
так: "У нас "куб" холодной воды один из самых дорогих в стране – больше 60 грн.
А горячей уже лет десять нет". У \emph{графіті} жовтоводського Бенксі
чуттєвість змішалася зі знанням текстів Єгора Лєтова та ізгоя Костянтина
Ступіна: "Я слышу колокол ада"; "Кайф вперемешку с кровью"; "Убей в себе
государство"; "Бунтуй, угарай"; "Спільний сон інтимніше за секс; "Беспощадные
глубины морщин".  70-річна Лідія Григорівна на ці послання не звертає уваги.
Настав час збирати гілки.  – Метлу з них буду робить і листя замітать, –
озвучує плани жінка, яка звикла наводити лад із часів роботи у міськкомунгоспі.
– О-о-ой! – вигукує вона. – В Жовтих Водах квітники кругом були, цвіло місто! А
зараз вже ЖЕКі сокращають, мусор не прибирають.  Молодь не держиться у нас: то
по Польщах, то по Чехіях, то по Естоніях
%%%cit_comment
%%%cit_title
\citTitle{Уран із присмаком кави. Яким стало режимне місто Жовті води, де видобували сировину для атомних бомб}, 
Євген Руденко; Ельдар Сарахман, www.pravda.com.ua, 09.11.2021 
%%%endcit
