% vim: keymap=russian-jcukenwin
%%beginhead 
 
%%file 11_02_2022.stz.news.ua.hvylya.1.vse_zlo_ot_ressentimenta.2.istoricheskij_koll_pamjati
%%parent 11_02_2022.stz.news.ua.hvylya.1.vse_zlo_ot_ressentimenta
 
%%url 
 
%%author_id 
%%date 
 
%%tags 
%%title 
 
%%endhead 

\subsubsection{Ресентимент исторический, коллективной памяти, примордиальный и зеркальный}

Ресентимент является результатом не только индивидуальных травм, но и
групповых, социальных, исторических, геополитических травм. Носителем
Ресентимента может быть не только индивид, но и группа, страна, государство и
его политика, группа стран и их политика, цивилизация.

Психическое заражение у Шелера — это феномен некритического «следования за
толпой» или менталитет толпы, подобно леммингам, прыгающим со скалы.
Положительные примеры — добродушные толпы в пабе или на спортивных
мероприятиях; отрицательный пример — насильственные беспорядки. В таком виде
Ресентимент становится уже социальной проблемой, потому что начинает
транслироваться на других посредством «психического заражения».

Использование таких методов негативного психического заражения можно
рассматривать как движущую силу таких исторических личностей или движений, как
Нерон (сожжение Рима), Французская революция (Ресентимент в исходной
концепции), Гитлер (геноцид евреев, арийская господствующая раса и проект
«Лебенсборн»), «красные кхмеры» 1975-1979 годов (социальная инженерия
геноцида), Руанда 1994 года (геноцид племен), международный фундаменталистский
исламский терроризм, национализмы Восточной Европы, «Русский мир» в России,
система социального рейтинга в Китае, левая повестка дня (Black Lives Matter,
радикальный феминизм, «культура отмены», «проверь свои привилегии»).

Имеющийся в обществе и соответственно у историков Ресентимент, как правило,
обуславливает ресентиментальные исторические нарративы. Ресентимент также
заражает коллективную память установками на подозрение, зависть, ненависть,
месть, то есть превращает ее в коллективную злопамятность.

В то же время, историки постоянно сталкиваются с трансляцией Ресентимента между
поколениями, народами, империями, и это влияет на их нарративы, несмотря на
правила академической добропорядочности. Рим сам по себе не был
ресентиментальным, но породил Ресентимент варваров. Мы можем подозревать, что
татаро-монголы были ресентиментальны и в своем завоевании транслировали и таким
образом глубоко укореняли Ресентимент в большинство покоренных народов.

Ресентиментальная политика памяти обычно имеет одномерную структуру памяти, где
есть вина обидчика и нет структур памяти вины жертвы. В ресентиментальной
памяти виноват только обидчик, а жертва невинна, как агнец.

Парадоксально и скандально то обстоятельство, что Ресентимент преодолевается не
тогда, когда обиды, унижения, холокосты и геноциды забываются, а когда они
осмысливаются как связка Ресентимента обидчика и Ресентимента жертвы и при этом
рефлексивно меняется содержание памяти: c ресентиментального комплекса жертв на
осознание эпохальных ограничений ресентиментального мышления,
ресентиментального эмоционального интеллекта, ресентиментальной воли и
ресентиментальной веры.

Скандальность иудейского Ресентимента заключается не в отрицании признания
Холокоста, который, безусловно, был и в котором безусловно виноват
национал-социалистический режим Гитлера, а в формулировке его глубинных причин,
которые находят в библейских нарративах.

Ницше и Вебер рассматривали Ресентимент на примере иудаизма и библейской
истории древних иудеев. Это примордиальный ресентимент, который понимается так,
что в начале любой цепочки Ресентимента стоит или иудейский Ресентимент, или
Ресентимент татаро-могольской орды, или еще какие-то Ресентименты из прошлого.
Это плохое мышление не потому, что ненаучное, а потому что упрощает
представление о Ресентименте, снимает ответственность с последующих
ресентименталов и делает Ресентимент неизбежным. Тем не менее, хоть и
примордиальные цепочки Ресентимента концептуально слабы, представление о
зеркальном Ресентименте вполне может быть использовано.

Геополитика Ресентимента очень часто носит зеркальный характер:
национал-социалистический Ресентимент противостоит классовому Ресентименту
(Германия — СССР времен войны), один националистический Ресентимент
противостоит другому (современные Украина и Польша), империалистический
Ресентимент противостоит националистическому ресентименту (современные Россия и
Украина), талассократический имперский Ресентимент противостоит
телурократическому имперскому Ресентименту (отношения Великобритании и России).

Зеркальный Ресентимент суть возникающий компенсаторно как отражение в
индивидуальном или коллективном сознании последствий взаимодействия с чужим
Ресентиментом. То есть чужой Ресентимент во взаимодействии с ним компенсируется
собственным Ресентиментом. Так в коллективной памяти обеих сторон создается
стойкая пара взаимного Ресентимента, существование которого гарантируется
взаимосвязанными структурами коллективной памяти. Зеркало Ресентимента может
быть кривым. Зеркальный Ресентимент — это фундаментальная трагическая
неоригинальность.

История тоже бывает зеркальной. Можно предположить, что русский Ресентимент
происходит от ордынского рабства России и удерживается противостоянием с
западной цивилизацией, порождая при этом украинский зеркальный Ресентимент.

Зеркальный Ресентимент гораздо глубже, чем незеркальный, то есть какой-нибудь
оригинальный Ресентимент. К примеру, Россия уже несколько сотен лет лелеет свой
русскоязычный Ресентимент. В зеркальном отражении также и Украина столько же
лелеет свой украиноязычный Ресентимент. Россия на языковом Ресентименте
построила свой «Русский мир». В то же время Украина на языковом Ресентименте
построила свой национал-патриотический проект.

Украина оказывается вообще в издевательском выборе: убегая от российского
рабства через зеркальный Ресентимент, она отдается на растерзание западному
рабству, программируя новый виток Ресентимента. Если из Украины убрать
зеркальные (прямые и искривленные), в основном ресентиментальные, отражения
России, Польши, Венгрии, Германии, Франции и Запада в целом, что останется?
Может ли Украина «выйти из зеркала»?
