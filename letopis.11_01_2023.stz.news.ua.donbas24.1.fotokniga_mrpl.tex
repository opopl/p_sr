% vim: keymap=russian-jcukenwin
%%beginhead 
 
%%file 11_01_2023.stz.news.ua.donbas24.1.fotokniga_mrpl
%%parent 11_01_2023
 
%%url https://donbas24.news/news/viiskovosluzbovec-zsu-vidav-fotoknigu-prisvyacenu-mariupolyu
 
%%author_id demidko_olga.mariupol,news.ua.donbas24
%%date 
 
%%tags 
%%title Військовослужбовець ЗСУ видав фотокнигу, присвячену Маріуполю
 
%%endhead 
 
\subsection{Військовослужбовець ЗСУ видав фотокнигу, присвячену Маріуполю}
\label{sec:11_01_2023.stz.news.ua.donbas24.1.fotokniga_mrpl}
 
\Purl{https://donbas24.news/news/viiskovosluzbovec-zsu-vidav-fotoknigu-prisvyacenu-mariupolyu}
\ifcmt
 author_begin
   author_id demidko_olga.mariupol,news.ua.donbas24
 author_end
\fi

\ii{11_01_2023.stz.news.ua.donbas24.1.fotokniga_mrpl.pic.front}

\begin{center}
  \em\color{blue}\bfseries\Large
  Воїн ЗСУ Мирослав Кобилянський видав унікальну фотографічну книгу \enquote{Мій Маріуполь та світ довкола}
\end{center}

Наприкінці 2022 — на початку 2023 року побачила світ фотокнига захисника
України, фотографа, мандрівника \emph{\textbf{Мирослава Кобилянського \enquote{Мій Маріуполь та світ
довкола}}}. Після повномасштабного вторгнення рф Маріуполь зазнав нищівного
руйнування, тому світлини, на яких зафіксована природа, архітектура Маріуполя і
навколишня місцевість, сьогодні мають ще більшу цінність та важливість.

\begin{leftbar}
\emph{\enquote{У пам'яті немає терміну придатності... В цій книзі я зібрав власні світлини
Маріуполя, створені в період його розквіту. Фотокнига є своєрідною
візитівкою міста. В ній глядач побачить головну гавань Приазов'я,
привабливу вітрину Донбасу — неймовірний Маріуполь до повномасштабного
вторгнення. Маю надію, що світлини, зібрані в цій книзі, закарбуються у
пам'яті кожного, хто хоча б раз перегорне глянцеві сторінки...}}, —
зазначив автор книги Мирослав Кобилянський.
\end{leftbar}

\ii{11_01_2023.stz.news.ua.donbas24.1.fotokniga_mrpl.pic.1}

\textbf{Читайте також:} \href{https://donbas24.news/news/smert-xodila-za-toboyu-jevgen-maloljetka-pro-zrobleni-kadri-viini-u-mariupoli}{\emph{\enquote{Смерть ходила за нами} — Євген Малолєтка про зроблені кадри війни у Маріуполі}}%
\footnote{\enquote{Смерть ходила за нами} — Євген Малолєтка про зроблені кадри війни у Маріуполі, Яна Іванова, donbas24.news, 27.11.2022, \par%
\url{https://donbas24.news/news/smert-xodila-za-toboyu-jevgen-maloljetka-pro-zrobleni-kadri-viini-u-mariupoli}%
}

Мирослав народився в Коломиї, з 2009 року служить у Збройних силах України. З
2015 до 2022 року він проходив службу в Маріуполі. Світлини фотохудожника з
висоти пташиного польоту не тільки Маріуполя, але й інших міст України та
Європи завжди відрізнялися особливою майстерністю та виразністю. Саме тому,
коли маріупольці дізналися про вихід фотокниги Мирослава, одразу ж намагалися
її якомога швидше придбати.

\begin{leftbar}
\emph{\enquote{Кожна його робота — це освідчення в коханні рідному місту. Я із захопленням,
болем та сумом знайомилася з фотоспогадами автора. Панорамні фото — з
таких ракурсів своє місто я побачила вперше. Місто-фортеця, місто у
моря, місто мрійників, місто якого нема... Але ще буде, і всі ми,
маріупольці, мріємо і знаємо, що Маріуполь це — місто-перемога!  Ця
книга, пам'ять і надія, і вона вже у фонді незламної Донецької обласної
бібліотеки для дітей!}}, — наголосила Директорка Донецької обласної
бібліотеки для дітей \textbf{Юлія Василенко}.
\end{leftbar}

\textbf{Читайте також:} \href{https://donbas24.news/news/lorem-ipsum-is-placeholder-text-commonly-used-in-the-graphic}{\emph{Поранене та ув'язнене місто у роботах маріупольського художника}}%
\footnote{Поранене та ув'язнене місто у роботах маріупольського художника, Ольга Демідко, donbas24.news, 24.10.2022, \par%
\url{https://donbas24.news/news/lorem-ipsum-is-placeholder-text-commonly-used-in-the-graphic}%
}

\ii{11_01_2023.stz.news.ua.donbas24.1.fotokniga_mrpl.pic.2}
\ii{11_01_2023.stz.news.ua.donbas24.1.fotokniga_mrpl.pic.3}

Книга включає п'ять розділів: \emph{\enquote{Приморський район}}, \emph{\enquote{Центральний район}},
\emph{\enquote{Лівобережний район}}, \emph{\enquote{Кальміуський район}} та \emph{\enquote{Світ довкола}}. Кожен з розділів
містить унікальні та короткі авторські описи як зображених локацій чотирьох
адміністративних районів міста, так і території в межах 100 км навколо
Маріуполя. Директор Львівського мистецького видавництва \enquote{Світло й Тінь}, яке і
випустило фотокнигу, \textbf{Володимир Пилип'юк} зауважив, що це один з найособливіших
та захоплюючих проєктів видавництва.

\begin{leftbar}
\emph{\enquote{Без перебільшення можу сказати: за всі 5 років мого керування видавництвом
ця книга — виняткова в культурному, історичному та художньому планах
праця. Її ідея, вміст та виконання, впевнений, неодмінно відгукнеться
багатьом із вас. Сторінки видання формують цілісний фотоспогад про
Маріуполь, яким він був до війни — красивим, омитим солоними азовськими
хвилями, мрійливим, дещо строгим та, на превеликий жаль, тимчасово
ефемерним — а проте таким, яким назавжди залишиться в пам'яті тих, хто
тут проживав і гостював}}, — поділився думками Володими Пилип'юк.
\end{leftbar}

\textbf{Читайте також:} \href{https://donbas24.news/news/slovo-2022-roku-yakii-viraz-stav-golovnim-v-rik-viini}{\emph{Слово 2022 року: який вираз став головним в рік війни}}%
\footnote{Слово 2022 року: який вираз став головним в рік війни, Яна Іванова, donbas24.news, 10.01.2023, \par%
\url{https://donbas24.news/news/slovo-2022-roku-yakii-viraz-stav-golovnim-v-rik-viini}%
}

\ii{11_01_2023.stz.news.ua.donbas24.1.fotokniga_mrpl.pic.4}
\ii{11_01_2023.stz.news.ua.donbas24.1.fotokniga_mrpl.pic.5}

Важливо, що книга є чотиримовною: містить переклади з україн\hyp{}ської на
англійську, німецьку та іспанську. Придбати книгу можна через львівське
видавництво \enquote{Світло і тінь} чи звернувшись особисто до Мирослава.

\begin{leftbar}
\emph{\enquote{Ця книга про тебе й для тебе, місто зі сталі — Маріуполь}}, — підсумував автор
фотографічної книги \enquote{Мій Маріуполь та світ довкола} Мирослав Кобилянський.
\end{leftbar}

\ii{11_01_2023.stz.news.ua.donbas24.1.fotokniga_mrpl.pic.6}

Раніше Донбас24 розповідав, що танцювальний \href{https://donbas24.news/news/tancyuvalnii-kolektiv-z-ukrayini-light-balance-kids-viklikav-furor-na-amerikanskomu-sou-talantiv-video}{\emph{колектив з України Light Balance
Kids}}%
\footnote{Танцювальний колектив з України Light Balance Kids викликав фурор на американському шоу талантів, Наталія Сорокіна, donbas24.news, 09.01.2023, \par\url{https://donbas24.news/news/tancyuvalnii-kolektiv-z-ukrayini-light-balance-kids-viklikav-furor-na-amerikanskomu-sou-talantiv-video}}
викликав фурор на американському шоу талантів.

Ще більше новин та найактуальніша інформація про Донецьку та Луганську області
в нашому телеграм-каналі Донбас24.

ФОТО: з особистого архіву Мирослава Кобилянського

\ii{insert.author.demidko_olga}
%\ii{11_01_2023.stz.news.ua.donbas24.1.fotokniga_mrpl.txt}
