% vim: keymap=russian-jcukenwin
%%beginhead 
 
%%file 22_12_2021.fb.fb_group.story_kiev_ua.1.kiev_bani_rejting.cmt
%%parent 22_12_2021.fb.fb_group.story_kiev_ua.1.kiev_bani_rejting
 
%%url 
 
%%author_id 
%%date 
 
%%tags 
%%title 
 
%%endhead 
\zzSecCmt

\begin{itemize} % {
\iusr{Владимир Бондаренко}
на сковороды напротив церквы Мыколы Доброго. сейчас там БЦ

\iusr{Виктор Задворнов}

С легким паром! Посещал баню на Кременецкой (рядом с военкоматом Святошинского
района), на Брест-Литовском просп. - теперь реконструированное помещение
конторы бывшего \enquote{Брокбизнес банка}. Но легендарной считаю баню в поселке
Коцюбинский. Там отапливали банные \enquote{залы} и грели воду благодаря
функционированию кочегарки. Парадокс: идешь принять душ и попариться, а из
двери котельной выглядывает \enquote{темная} от угольной сажи личность - пьяненький
кочегар. Мыться пожалуйте! Цены в 60-е-70-е были весьма сносными.


\iusr{Sasha Jampolsky}

У меня немного менее исторические заметки про бани. Выросши в общей квартире на
Крещатике, без ванны, то есть ванна была но она не работала наверно с ещё
довоенных времен, ходили мы в Караваевские бани. Раз неделю, я с папой а мама с
бабушкой, ходили иногда в общее , а зачастую в номера. Кажется это стоило тогда
рупь за час. А потом, подросши и распознавши прелести противоположного пола, я
ходил туда с избранницами так как там паспорта у меня не просили потому что
знали с детства и ещё за два рубчика ( один сверху) можно было занять номер на
полные два часа.

А заодно и помыться


\iusr{Елена Мельникова}

В 60-70г. 20 века многие посещали общественные бани. Подол например ходил в баню
по Героев Триполья (ныне Спасская). Общие номера с парилкой, отдельные
номера-душ и ванная. Мылись как правило раз в неделю. Пиво и раков не помню, а
маникюр, парикмахер и замечательная косметичка Галина Григорьевна курившая
только \enquote{Беломор} по фронтовой привычке.

Может кто помнит?

\begin{itemize} % {
\iusr{Vladimir Dotsenko}
\textbf{Елена Мельникова}

С 1954 по 1961 посещали эти бани всей семьей, после в буфете для Отца было
бочковое пиво, а для нас, детишек - сладкий вкуснющая газировка с двойным
сиропом...


\iusr{Ольга Кирьянцева}
\textbf{Елена Мельникова} да, посещали эту баню). Номер с душем и ванной 1 рубль.
\end{itemize} % }

\iusr{Владимир Хворов}

у Нас на Андреевском спуске вода была по среди двора (колонка) Мы всей семьей
по субботам ходили в баню на Малой Житомирской Иногда в номера ст 1 рубль в час
Там были душ и ванная Ванная была просто роскошь Меня тщательно мыли а потом
разрешали плюхаться в ванной Это был кайф


\iusr{Катерина Силина}

Вы забыли о бане на Постышева. Мы всей семьей ходили туда, когда проживали на
Михайловской в коммунальной квартире. Я была ещё маленькая, но помню там был
бассейн в лягушками по углам.

\begin{itemize} % {
\iusr{Олег Чудновский}
\textbf{Катерина Силина} Да это сейчас ул. Малая Житомирская, там о ней есть.

\iusr{Катерина Силина}
\textbf{Олег Чудновский} Спасибо, теперь понятно, что не забыли нашу баню

\iusr{Светлана Николаевна}
\textbf{Олег Чудновский} в детстве жили в коммуналке на Тарасовской и тоже ходили в баню Караваевскую, память осталась на всю жизнь, спасибо напомнили, чудные были времена!!

\iusr{Олег Чудновский}
\textbf{Светлана Николаевна} Приятно, но спасибо надо сказать Юрию Бубнову.

\iusr{Лариса Кушниренко}
\textbf{Катерина Силина} Очень даже помню . Жили на Десятинной и это была наша баня. @igg{fbicon.face.happy.two.hands} 
\end{itemize} % }

\iusr{Петро Гарматюк}
А на вулиці Михайлівській баня була ?
Чи я помиляюся ?

\begin{itemize} % {
\iusr{Irena Visochan}
\textbf{Петро Гарматюк} это на Постышева рядом.

\iusr{Петро Гарматюк}
\textbf{Irena Visochan}
Спасибо.

\iusr{Давид Вайнблат}
\textbf{Петро Гарматюк} Була. Я жив на Прорізаний і туди ходів
\end{itemize} % }

\iusr{Владимир Шевцов}
Печерская Московская угол Суворова. Сейчас там Зоряный

\iusr{Алла Ковтун}
Троицкие до конца 80-х ещё работали.

\iusr{Татьяна Жалнина}
\textbf{Алла Ковтун} точно, я там была ,теперь только воспоминания

\iusr{Микола Ахаладзе}
А ви забули Лазню на проспекті Перемоги (бувший Брест-Литовський проспект).

\iusr{Анатолий Борозенец}

Маю сумнів щодо присутності \enquote{Московських бань} на вул. Ярославів Вал,40.
Будівлю, де нині інститут ім.Карпенка-Карого, звели для економічної школи в
1907-му.

\begin{itemize} % {
\iusr{Елена Гнеденко}
\textbf{Anatoly Borozenets} 

Похоже, что эти бани находились на ул. Бульварно-Кудрявской, приблизительно в
\#6 или8, (точно не помню). Там жили родственники ( и в пятидесятых годах 20
столетия), в чердачных помещениях. Сейчас там административное здание.


\iusr{Анатолий Борозенец}
\textbf{Елена Гнеденко} Ні. \enquote{Московські бані} знаходилися на вул. Ярославській, 17/22.
\end{itemize} % }

\iusr{Pasha Paolo}

Бани Каплера, которые \enquote{Московские} НИКОГДА не располагались на ЯрВалу.
Правильный адрес - Константиновская 22/17. Фактически, это гостиница \enquote{Сион},
сами бани были во дворе и теперь это действительно корпус Театрального
института.

\end{itemize} % }
