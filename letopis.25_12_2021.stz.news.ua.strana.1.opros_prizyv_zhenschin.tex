% vim: keymap=russian-jcukenwin
%%beginhead 
 
%%file 25_12_2021.stz.news.ua.strana.1.opros_prizyv_zhenschin
%%parent 25_12_2021
 
%%url https://strana.news/news/368717-voennyj-uchet-zhenshchin-v-ukraine-opros-antoniny-belohlazovoj.html
 
%%author_id beloglazova_antonina
%%date 
 
%%tags 
%%title "Имущество Зеленского защищать не собираюсь". Что говорят в Киеве о призыве женщин в армию. Опрос "Страны"
 
%%endhead 
\subsection{\enquote{Имущество Зеленского защищать не собираюсь}. Что говорят в Киеве о призыве женщин в армию. Опрос \enquote{Страны}}
\label{sec:25_12_2021.stz.news.ua.strana.1.opros_prizyv_zhenschin}

\Purl{https://strana.news/news/368717-voennyj-uchet-zhenshchin-v-ukraine-opros-antoniny-belohlazovoj.html}
\ifcmt
 author_begin
   author_id beloglazova_antonina
 author_end
\fi

Министерство обороны Украины утвердило приказ, который обязывает многих женщин
возрастом от 18 до 60 лет становиться на воинский учет - в зависимости от их
профессии.

Список этих профессий настолько широк, что под воинский учет подпадают миллионы
женщин. Причем власть имеет право регулярно вызывать их на военные сборы. 

Мы решили поинтересоваться, что об этой инициативе думают киевляне.

Большинство опрошенных категорически против этой инициативы. 

\begin{multicols}{2} % {
\setlength{\parindent}{0pt}

\headTwo{\enquote{Вчера варили борщ, а завтра за автоматы?}}

\enquote{Плохо отношусь к этому. Особенно [если брать] женщин, которые никогда не
служили. В Израиле люди готовятся к этому. То есть мужчин могут через 10 лет
призвать, которые на военной кафедре учились. И они хоть что-то понимают. А
женщин, которых в 55 лет могут призвать, я слабо себе представляю}, - поделился
своим мнением киевлянин. 

\ii{25_12_2021.stz.news.ua.strana.1.opros_prizyv_zhenschin.pic.1}

Женщины тоже от военного призыва не в восторге. \enquote{Вчера варили борщ и ухаживали
за детьми, а завтра автоматы и всё прочее - это, конечно, перебор}, -
соглашается с предыдущим участником опроса киевлянка.

\ii{25_12_2021.stz.news.ua.strana.1.opros_prizyv_zhenschin.pic.2}

\enquote{Если всех призовут в армию, то кто будет здесь? Бабушки-дедушки и дети
маленькие останутся что ли? Ну это ненормально, я считаю. Никогда так не было.
Всегда мужчины воевали. А женщины были в тылу}, - говорит участница опроса. 

Ей сложно ответить на вопрос, с какой целью вообще в Минобороны пошли на то,
чтобы призвать миллионы женщин: \enquote{Не знаю, нехватка мужской силы?}, -
недоумевает она. 

\ii{25_12_2021.stz.news.ua.strana.1.opros_prizyv_zhenschin.pic.3}

\enquote{В Израиле все барышни служат. Почему бы и нет? Только бы обеспечили условия,
чтобы мы понимали, ради чего это всё делается. Сейчас условия не обеспечены. Я,
например, так просто не поеду. Нужны хотя бы бытовые условия, четкое понимание
задач, организация. Ну, это же барышни. Одним приказом там не отделаться}, -
убеждена киевлянка.

\ii{25_12_2021.stz.news.ua.strana.1.opros_prizyv_zhenschin.pic.4}

\enquote{Безобразие! Ну как можно относиться! Это какой-то беспредел! Туда еще и
женщин. Они и так работают, тянут домой последнее, еще их в армию. Я вообще не
знаю, что вам сказать. Я в шоке. Это же не Израиль. Женщин и так беречь надо, а
мы еще их в армию}, - поделился своим возмущением мужчина.

Он уверен, что перенять израильский опыт невозможно из-за различий в
менталитете: \enquote{Что перенимать? Может, надо за своё как-то держаться. Женщин
всегда берегли}.

\headTwo{\enquote{Неудачная калька с Израиля}}

\ii{25_12_2021.stz.news.ua.strana.1.opros_prizyv_zhenschin.pic.5}

\enquote{Неудачная калька с Израиля как по мне. Потому что быстро такие дела не
делаются. Без каких-то предварительных общественных концепций выдвижения,
обговариваний и т.д. Больше повлияли последние события вокруг нашей страны и
вокруг наших соседей. Все эти взаимоотношения старые и недобрые, которые
последние 7 лет. По моему мнению, это так}, - ответил мужчина.

Люди уверены, что опыт Израиля нельзя бездумно переносить на реалии жизни в
Украине. Кадры должны тщательно готовить, смотреть, что в нашей специфике
подходит, а что нет. 

\ii{25_12_2021.stz.news.ua.strana.1.opros_prizyv_zhenschin.pic.6}

\enquote{Мне кажется, что это совершеннейший идиотизм. Это первый вариант. Второй
вариант (но это, наверное, вырежут): так как нами правят люди определенной
национальности, нет ну а чего здесь такого, евреи - те же украинцы, то они
видимо хотят сделать, чтобы у нас было так же, как в Израиле. А в Израиле
женщины военнообязанные. Но Израиль живет постоянно в окружении врагов
постоянно - арабских стран. У них это видимо оправданно. Но у нас враги
появились только недавно и я не думаю, что нас окружают. Я думаю, что скоро это
безобразие закончится}, - поделилась своим мнением женщина. 

Она уверена, что ситуация последних лет создана искусственно.

\ii{25_12_2021.stz.news.ua.strana.1.opros_prizyv_zhenschin.pic.7}

\headTwo{\enquote{Воевать нет смысла}}

Следующая участница опроса очень резко высказалась против этой инициативы.

\enquote{Отлично, прекрасно, могу завтра пойти записаться, но так чтобы сразу с бухлом,
чтобы посидеть отлично... - иронизирует молодая девушка и через минуту говорит
следующее: - Конечно, это @реновая идея! Мы не Израиль, ребята, простите! У нас
уравнять права на работе не могут, у нас дискриминация, но зато в армию все
вместе пойдём. Прекрасно! Уравняли женщин в правах. Просто спасибо, ребята!}, -
эмоционально говорит она. 

При этом девушка говорит, что никак не налажена даже работа с призывом парней.

\enquote{Все сваливают. Подписывают какие-то белые билеты. Да и если бы у нас был смысл
служить. А так, например, у меня парень служил с начала войны, и я очень много
знаю интересных вещей, которые там происходят. И если бы у нас в армии всё было
чисто, красиво и правильно, больше бы людей хотели бы туда идти служить. А так
вопрос не в том, что у нас люди не ответственные, а в том, что они не видят в
этом смысла}.

\ii{25_12_2021.stz.news.ua.strana.1.opros_prizyv_zhenschin.pic.8}

Опрос в целом показывает неготовность женщин к войне: \enquote{Не знаю, что ответить.
Никак [не отношусь к этому] Мне это не нравится. Вот допустим, как женщина до
60 лет может пойти служить, скажите? Это бабушка, тем более сейчас здоровье
женщин не настолько крепкое и в 59 лет служить?}, - интересуется у нас
представительница этой категории.

\ii{25_12_2021.stz.news.ua.strana.1.opros_prizyv_zhenschin.pic.9}

Очередная девушка призналась, что и так служит в армии, но идею отправить на
службу всех женщин до 60 лет назвала глупой. 

\enquote{Это будут девушки и женщины до 60 лет, а так как у нас со здоровьем и
социальными пакетами в стране очень плохо, поэтому я считаю это не очень
вариант}, - ответила она.

\ii{25_12_2021.stz.news.ua.strana.1.opros_prizyv_zhenschin.pic.10}

Часть девушек даже думать не хотят о службе в армии и об учете. 

\enquote{Ужасно! Потому что это дебилизм. Каждый должен заниматься своим делом: женщина
- своим, мужчина - своим}, - ответила нам киевлянка на бегу. 

\ii{25_12_2021.stz.news.ua.strana.1.opros_prizyv_zhenschin.pic.11}

\enquote{Девушка - это мать, это жена, это не воин. Я так думаю. В Израиле девушки
служат. Но нашему менталитету этот опыт не подходит. Я думаю, мы более
домашние, знаете. Я лично по себе сужу. Мне это не надо. И так же моим детям,
моим дочерям так же само}, - говорит женщина.

С ней согласна следующая участница опроса, которая сразу же признается, что
воевать не хочет, а женский призыв должен быть по желанию. \enquote{Кто хочет служить,
защищать, это его призвание, внутренние ощущения, - тот должен идти. А так...
Те девушки, кто хотят, пусть те и идут}, - пожимает плечами она. 

\ii{25_12_2021.stz.news.ua.strana.1.opros_prizyv_zhenschin.pic.12}

\headTwo{\enquote{Президент наш не служил}}

\enquote{Ужасно, потому что это дебилизм}, - так начинали свой ответ на вопрос \enquote{Страны}
многие опрошенные. 

\enquote{Зачем там бухгалтер. Врачи были, все хорошо было. А бухгалтер или менеджер
зачем? Для чего это? - Просто усложнить людям жизнь}, - говорит мужчина. Он
уверен, что это делается только с целью денежного обогащения.

\enquote{Денег больше выдрать [хотят]. Работодатель будет платить штраф и всё. Или
хотят увеличить число войск? Так у нас куча желающих - их не берут. У нас армия
укомплектована. Если бы был недобор страшный. Плюс это вызывать еще на сборы,
обучать чему-то. Это просто выделение бюджетных денег, которые будут списывать
на проведение сборов и еще какую-то ерунду. А толком там ничего проводиться не
будет}, - размышляет он. 

\enquote{Кстати, президент наш не служил, правильно? Отмазывался от армии и сам это же
мутит. Я думаю, что его жена тоже пойдет, военнообязанная получается будет?
Хотел бы я посмотреть, как они там будут ползать где-то по лесам житомирским},
- добавляет с улыбкой киевлянин.

\ii{25_12_2021.stz.news.ua.strana.1.opros_prizyv_zhenschin.pic.13}

\enquote{Как и мужчины, женщины должны проходить определенную подготовку. Это такой
длительный процесс, который занимает не два дня, не неделю, не две недели. На
это должны решаться определенные женщины, не все должны призываться под одну
копирку. Женщины имеют право, но это не должно быть обязательным. И призыв
обязательно должен включать какую-то серьезную подготовку}, - уверена
жительница столицы.

\ii{25_12_2021.stz.news.ua.strana.1.opros_prizyv_zhenschin.pic.14}

Еще одна девушка, которая идет под руку с мужчиной ответила нам, что к
новшеству Минобороны относится отрицательно.

\enquote{Я думаю, что это просто нагнетание обстановки среди людей и населения. Что
будет война и так далее. Что женщины должны воевать. Просто всё в этом роде}, -
ответила девушка. Она призналась, что старается жить сегодняшним днем и не
анализировала, к чему могут привести подобные законодательные решения.

\enquote{Надеюсь, что когда-нибудь к власти придет умный человек, который это все
отменит}, - добавила она.

\ii{25_12_2021.stz.news.ua.strana.1.opros_prizyv_zhenschin.pic.15}

\enquote{У нас что мужчины закончились или что? Не женское это дело}, - звучит еще один
короткий ответ от женщины.

Дальше корреспондент \enquote{Страны} задает вопрос пенсионерки и получает неожиданный
ответ, мол призывать можно тех, кому за 60, чтобы могли есть приготовить и
военным носки связать. 

\enquote{Надо призывать тех, которым уже за 60-70 лет. А другие пусть детей рожают, за
мужиками ухаживают, кормят, вот такое моё видение. Я люблю семью, люблю детей,
всю жизнь мою мама говорила, что радость - это дети. А как это мама пойдет в
окопы. А детки с кем будут? Как вот это сейчас в Польшу едут и какие семьи
неполные и что дети тогда делают? Разве же это по-государственному? - Нет, это
неправильно}.

\ii{25_12_2021.stz.news.ua.strana.1.opros_prizyv_zhenschin.pic.16}

\enquote{Девушки что воевать должны? Это что за новости такие. Кошмар}, - возмущаются
две женщины - противницы призыва.

\headTwo{\enquote{В службе нет ничего плохого}}

Впрочем, есть люди, которые ничего плохого в военной службе не видят и даже
поощряют такое желание своих детей.

\enquote{Моя дочь военнообязанная. Она очень хотела быть военной. И когда в институте
появилась возможность пойти на военную кафедру, два года отработала. И если бы
не трое детей она была бы уже в АТО прямо с первых дней... Потому что патриот,
за государство, за родную Украину}, - сказала нам женщина. 

Она говорит, что армию женщина еще должна заслужить, воевать пойдут не все.
\enquote{Чтобы приняли, чтобы подготовили, это сильные личности. Отказаться от благ
каких-то, комфорта, пойти в окопы, в те же блиндажи, те суровые условия}, -
восхищается она женщинами-военными и добавляет, что 60\% украинок готовы идти
воевать.

\ii{25_12_2021.stz.news.ua.strana.1.opros_prizyv_zhenschin.pic.17}

Нам также встретились мужчины, которые относится к этому положительно. 

\enquote{В Израиле все служат и в этом ничего плохого нет. Если девушки хотят идти [в
армию], они идут. В принципе, я думаю, будет нормально}, - сказал он.

\ii{25_12_2021.stz.news.ua.strana.1.opros_prizyv_zhenschin.pic.18}

Еще один собеседник не против перенять израильский опыт, предполагающий призыв
представительниц женского пола. 

\enquote{В принципе, положительно. Опыт Израиля показывает, что всё нормально. Почему
нет. Наши женщины вообще воинственные. Если воевать - то всем и везде}, -
убежден киевлянин.

\ii{25_12_2021.stz.news.ua.strana.1.opros_prizyv_zhenschin.pic.19}

Еще один аналогичный ответ дает третий мужчина: \enquote{Отлично. Равноправие, -
называет он первую причину с улыбкой на лице. - Это стандарт НАТО}. Он уверен,
что Украине эти новшества подходят.

\enquote{Всегда так было. Во времена казачества. Женщины нам всегда помогали}, -
пояснил он.

\ii{25_12_2021.stz.news.ua.strana.1.opros_prizyv_zhenschin.pic.20}

\enquote{Ну хорошо, возьмём Израиль, там все одинаково служат - и женщины, и мужчины.
Это же не значит, что она будет выполнять какие-то физические нагрузки. Есть
конечно разные женщины, разная физиология. Все же кричали \enquote{Гендерное
равенство!} Вот чем женщина хуже мужчины? Что она не имеет права делать? В
Советском союзе были женщины-трактористы. Паша Ангелина, слышали такую?}, -
отвечает нам еще один опрошенный.

\ii{25_12_2021.stz.news.ua.strana.1.opros_prizyv_zhenschin.pic.21}

\headTwo{Нечем семью кормить, а они хотят женщин - в армию}

Две девушки в один голос поражаются перспективе воевать наравне с мужчинами. 

\enquote{Надеюсь, нас это не коснется. Не хотелось бы. Человек должен сам принять
решение, хочет он служить или нет}, - говорит участница опроса.

Ее подруга тоже согласна, что должно быть по желанию.

\ii{25_12_2021.stz.news.ua.strana.1.opros_prizyv_zhenschin.pic.22}

\enquote{Мне, как матери, жалко женщин. Потому что дети остаются, как она пойдет...
Пусть бы только мужчины. Только жаль у нас рождаемость маленькая, не хватает
мужчин. А женщин я бы, конечно, не хотела отдавать в армию}, - отвечает нам
киевлянка.

\ii{25_12_2021.stz.news.ua.strana.1.opros_prizyv_zhenschin.pic.23}

\enquote{Негативно. Мне уже почти 50 лет, какой из меня воин в данном случае? Если
будет угроза нашему государству, то, конечно, это будет ответственность каждого
человека - идти защищать или нет. Но я считаю, что сейчас идет политика
уничтожения украинцев. Отбирание земель, мы самая бедная в мире страна, поэтому
я отношусь негативно. Точно так же в этот котёл женщин бросают. Мне кажется,
что женщины должны объединяться вокруг другого - развитие нашего государства,
рождение детей, создание и укрепление семей, наша национальная
самоидентичность}, - сказала нам противница женского призыва.

\ii{25_12_2021.stz.news.ua.strana.1.opros_prizyv_zhenschin.pic.24}

Мужчина отвечает, что это всё ерунда абсолютная. Он говорит, что делается это
отсутствия ума у украинских властей. 

Еще один прохожий пожилого возраста поясняет: \enquote{Женщина - это дома хозяйка,
блюститель очага, воспитатель детей, создатель хорошей ауры в семье, любящая
жена и женщина. Ну какой воинский учет? Без женщин мужики должны решать эти
вопросы. А не женщин должны туда привлекать}.

\ii{25_12_2021.stz.news.ua.strana.1.opros_prizyv_zhenschin.pic.25}

Еще одна женщина говорит, что всех женщин нельзя поставить на военный учет. Да
и у женщин под 60 уже здоровье не то, чтобы куда-то ехать. 

\enquote{Это нереально просто. И женщины, которые хотят служить в армии, их не надо
призывать, они сами идут туда служить. Я думаю, что это все отменится, потому
что это нереально. И это, если мягко сказать, бздык президента}, - убеждает нас
пенсионерка.

Она уверена, что делается это для того, чтобы отвлечь от проблем более
глобальных. \enquote{Это сколько военкоматов надо. А сколько денег? У нас денег нет. Не
платят пенсии людям, зарплаты... Вот сегодня шла и мужчина чуть не плачет: три
месяца не выплачивают зарплату, нечем семью кормить, а они хотят женщин в
армию}, - поделилась она.

\ii{25_12_2021.stz.news.ua.strana.1.opros_prizyv_zhenschin.pic.26}

\enquote{Я думаю, что это что-то странное, его не введут. Нас, наверное, от чего-то
отвлекают. Я не думаю, что так это и будет, потому что уже все мои подруги
говорят: \enquote{Вы шо?} Типа это же сумасшествие. Так что я не думаю, что это будет},
- говорит о призывном постановлении Минобороны девушка. 

\ii{25_12_2021.stz.news.ua.strana.1.opros_prizyv_zhenschin.pic.27}

\enquote{В Отечественную войну женщины шли добровольцами на фронт. А кто-то у нас
пойдет добровольцем на фронт из женщин? Что-то я очень сомневаюсь. А если и
пойдут, то какие-то может быть единицы. А массово на фронт, когда бросали своих
детей в Отечественную войну, такого, мне кажется, быть не может. Уже не то
сознание, не тот менталитет. Не то отношение к родине, извините}, - сказала
киевлянка.

\ii{25_12_2021.stz.news.ua.strana.1.opros_prizyv_zhenschin.pic.28}

\enquote{Слышала такую информацию, но подробностей не знаю. В принципе, я наверное
против. Я вообще против войны, поэтому против этого всего. У меня дедушка был
на войне и рассказывал, что это на самом деле очень ужасно}, - говорит
киевлянка.

\ii{25_12_2021.stz.news.ua.strana.1.opros_prizyv_zhenschin.pic.29}

Следующие ответы были аналогичными. Женщины отвечали, что не хотели бы иметь
какое-то отношение к армии.

\enquote{Мужчины защитники! Не считаю нормой, что девушек должны на учет ставить}, -
поделилась своим мнением одна из опрошенных.

\enquote{Просто поставить на учет, не проведя спецподготовку, сенс и коэффициент
полезного действия в этом будет однозначно нулевым. Это будет просто
формальность}, - ответила еще одна респондентка.

\enquote{Мы не подготовлены к этому ни морально, ни физически}, - говорит третья.

Они затруднялись ответить, с какой целью это делается. 

\enquote{Наверное, не от большого ума, чтобы держать всех под контролем}, - раздумывает
девушка. 

\ii{25_12_2021.stz.news.ua.strana.1.opros_prizyv_zhenschin.pic.30}

\enquote{Это неправильно. Женщина должна быть матерью, мужики должны воевать}, -
отвечает нам участник опроса, который от нас узнал про обязательный призыв
женщин, а раньше даже не задумывался об этом. 

\ii{25_12_2021.stz.news.ua.strana.1.opros_prizyv_zhenschin.pic.31}

\enquote{Пусть Верховная Рада идёт. Не сидят и штаны просиживают, а идут нашу страну
защищать}, - говорит 50-летняя женщина. 

\ii{25_12_2021.stz.news.ua.strana.1.opros_prizyv_zhenschin.pic.32}

Молодые парни убеждены, что женщина не должна брать в руки оружие. Они называют
эту идею плохой. 

Также часть опрошенных еще не разобралась в вопросе: \enquote{Пока не поняла, чем
грозит. Видела только про штрафы что-то такое, а как это будет на практике
проходить и что последует за этим дальше, пока непонятно. Глобально, когда
такой сосед со странностями, наверное, есть смысл задуматься о том, чтобы
усилиться, но для этого нужны еще и другие действия. Экономические, финансовые
и так далее. Это не решение вопроса, если девчонки сейчас все запишутся,
сильнее мы от этого не станем и врагу отмашь не дадим. Поэтому нужно усиливать
то, что есть}, - объяснила женщина. 

\ii{25_12_2021.stz.news.ua.strana.1.opros_prizyv_zhenschin.pic.33}

\enquote{Нужно было спросить у самих женщин. Мне сын это сбросил и я, честно говоря,
просто в шоке. Хотя я дочь советского офицера и я не стесняюсь этого. Я была в
окружении советских ребят, которые прошли Афганистан. Поверьте мне, что моё
поколение, хоть у меня и есть друзья-медики, которые военнообязанные, и я
работала в медицине, я понимаю... но нынешнее поколение и сын мой не был еще в
этой армии. Но это очень и очень вызывающе! Вызов обществу, щелчок, удар!} -
рассказывает нам участница опроса.

\ii{25_12_2021.stz.news.ua.strana.1.opros_prizyv_zhenschin.pic.34}

\enquote{Глупость в постановке вопроса и том, как он решается. Пропускная способность
военкомата 30 человек в день. Теперь пересчитываем количество женщин,
количество военкоматов, [учитывая], что в помещении находиться нельзя из-за
ковида, а нужно на улице. Теперь самой женщине нужно вычислить, кто она, имеет
ли она, должна ли она... В общем, это дикость}, - отвечает мужчина. 

На вопрос, с какой целью это делается он ответил так: \enquote{Я считаю, это
диверсанты, которых выбрали 73\% (столько проголосовало за Зеленского во втором
туре). Вот 73\% сами являются этими диверсантами, которые приводят к тому, к
чему идёт страна во время войны}.

\ii{25_12_2021.stz.news.ua.strana.1.opros_prizyv_zhenschin.pic.35}

Еще одна женщина негативно относится к нововведениям, говоря о том, что ранее
похожей практики в нашей стране не было:

\enquote{Ощущение, что всё втихаря принялось, поэтому сейчас это больше вызывает
негатива, чем позитива. Что-то новое всегда вызывает отторжение, потому что
если разобраться, это будет ок. Но пока не очень}.

\ii{25_12_2021.stz.news.ua.strana.1.opros_prizyv_zhenschin.pic.36}

\enquote{Мало информации и немножко пугает. Я вообще боюсь войны. Но так делают во
многих странах, поэтому не знаю, может, это и справедливо}, - пытается
взглянуть на проблему чуть шире киевлянка.

\ii{25_12_2021.stz.news.ua.strana.1.opros_prizyv_zhenschin.pic.37}

Напоследок мы встречаем жителей Израиля, которые приехали в Киев. Они против
призыва женщин. \enquote{Не знаю, у нас это по-другому. Не знаю, как это тут. Но я не
думаю, что женщина может делать то, что мужчина может. И все, кто туда хотят,
им сразу советуют не идти. Я там не была, я не воевала, но это не тот уровень},
- говорит молодая девушка, которая служила, но не воевала.

\ii{25_12_2021.stz.news.ua.strana.1.opros_prizyv_zhenschin.pic.38}

Последняя опрошенная женщина к идее относится однозначно отрицательно.

\enquote{О-три-ца-тель-но. И еще я вам хочу сказать: землю продали, нам отстаивать
нечего. Имущество Порошенко, Зеленского и иже с ними я отстаивать не собираюсь
вообще. Вот им}, - показала она средний палец.

\end{multicols} % }
