% vim: keymap=russian-jcukenwin
%%beginhead 
 
%%file 21_01_2022.stz.news.lnr.lug_info.1.kniga_donbass_nina_ischenko_kulturolog
%%parent 21_01_2022
 
%%url https://lug-info.com/news/luganskij-kul-turolog-predstavila-v-dnr-knigu-o-puti-donbassa-v-civilizacionnoj-vojne
 
%%author_id news.lnr.lug_info
%%date 
 
%%tags dnr,donbass,identichnost,ideologia,kniga,kulturologia,lnr,mentalnost,rossia,ukraina,vojna,zapad
%%title Луганский культуролог представила в ДНР книгу о пути Донбасса в цивилизационной войне
 
%%endhead 
\subsection{Луганский культуролог представила в ДНР книгу о пути Донбасса в цивилизационной войне}
\label{sec:21_01_2022.stz.news.lnr.lug_info.1.kniga_donbass_nina_ischenko_kulturolog}
 
\Purl{https://lug-info.com/news/luganskij-kul-turolog-predstavila-v-dnr-knigu-o-puti-donbassa-v-civilizacionnoj-vojne}
\ifcmt
 author_begin
   author_id news.lnr.lug_info
 author_end
\fi

Культуролог, кандидат философских наук, доцент кафедры философии Луганского
государственного педагогического университета, член Союза писателей ЛНР Нина
Ищенко представила в Донецке свою книгу \enquote{Южный фронтир: Россия – Украина –
Донбасс}.

\ii{21_01_2022.stz.news.lnr.lug_info.1.kniga_donbass_nina_ischenko_kulturolog.pic.1}

Презентация, организованная при поддержке общественных палат Луганской и
Донецкой Народных Республик, состоялась в Донецком агентстве новостей.

\enquote{Луганский ученый, философ и участник крупнейшего, не побоюсь этого слова, в
мире донецкого литературного форума \enquote{Звезды над Донбассом} Нина Сергеевна
Ищенко привезла в Донецк свою научную работу, благодаря которой термин
\enquote{фронтир} получил научное обоснование}, - представил автора труда председатель
Общественной палаты ДНР Александр Кофман, добавив, что Донбасс в очередной раз
таким образом \enquote{показывает единую политику, единые цели и единое культурное
пространство}.

\ii{21_01_2022.stz.news.lnr.lug_info.1.kniga_donbass_nina_ischenko_kulturolog.pic.2}

\enquote{Я хочу рассказать о работе, которая представляет собой философское осмысление
тех культурных процессов, которые идут в последние годы в районе, обозначенном
названием}, - продолжила Ищенко.

Она сообщила, что при написании сосредоточилась на обобщении наработанного за
годы войны материала, особенно опираясь на события последних лет в Донбассе и
на Украине, в стремлении ввести понимание происходящего в научный оборот.

\enquote{Мне интересно это как ученому, и в общем для науки, считаю, это необходимо}, -
подчеркнула автор.

Также Ищенко уточнила, что в книге рассмотрены те культурно-социальные проекты,
с которыми люди так или иначе сталкивались в повседневной жизни.

\enquote{Моя книга – это шанс для людей увидеть \enquote{кто}, \enquote{зачем}
и \enquote{почему}}, - рассказала писательница.

\subsubsection{ВОЙНА ЦИВИЛИЗАЦИЙ}

По мнению культуролога, конфликт в Донбассе носит цивилизационный характер –
это война Русской и Западной цивилизаций, в которой сталкиваются \enquote{разные
антропологические типы, воспитанные каждой из них}. Западная воспитывает
индивидуалиста, либерального человека, который ставит себя и свою личность
превыше всего. Русская – человека, для которого Родина и общность важнее
личного. Русский человек проявляет инициативу ради этих высших целей, а не
собственной выгоды.

\enquote{Человек берет из общества культурные, семейные нормы и даже язык, на котором
он формулирует эти нормы. Если он принципиально отказывается приумножать,
сохранять и беречь все это и ставит себя в позицию универсального потребителя,
то это ведет к разрушению такой культуры, как русская, где общий интерес
традиционно превалирует над личным}, - добавила автор.

Она объяснила, что на Западе есть механизмы, которые позволяют конвертировать
энергию разрушения в устойчивое развитие и экспансию, в то время как в русской
культуре такие механизмы отсутствуют. В то же время Украина стремится изменить
свою цивилизационную идентичность и применить культурные артефакты и нормы,
свойственные для западного общества. Подводя итог, Ищенко отметила, что работа
\enquote{Южный фронтир: Россия – Украина – Донбасс} свидетельствует о развитии
Донбасса, о том, что здесь заняты не только \enquote{хлебом насущным}, но и
осмыслением множества глобальных процессов.

Кроме того, как считает культуролог, труд служит одним из элементов победы в
цивилизационной войне, поскольку в нем вскрываются реальные механизмы
культурного противостояния, их цели и средства.

\subsubsection{ОТВЕТ ЗАПАДУ}

\enquote{Мы приехали в очередной раз подтвердить всем, что ДНР и ЛНР – это
сестры-близнецы, которые идут плечом к плечу, и те сложности, которые возникают
сегодня в процессе становления государственности, преодолеваются, - сказал
председатель Общественной палаты ЛНР Алексей Карякин. - Что касается книги, то
это достойный ответ навязываемой культуре извне}.

Он добавил, что \enquote{все потуги Запада обречены на провал}, жители Донбасса
являются достойными своих предков и сумеют сберечь свою идентичность.

Книга выпущена ограниченным тиражом. Презентация прошла в Луганске чуть более
месяца назад. Ознакомиться с изданием можно в электронном виде, также в
ближайшее время несколько экземпляров будут переданы в Донецкую республиканскую
универсальную научную библиотеку имени Крупской, а затем книга будет размещена
на электронных ресурсах в ДНР.
