% vim: keymap=russian-jcukenwin
%%beginhead 
 
%%file 15_09_2021.fb.zharkih_denis.1.interview_savenkova
%%parent 15_09_2021
 
%%url https://www.facebook.com/permalink.php?story_fbid=3078199575726772&id=100006102787780
 
%%author_id zharkih_denis,savenkova_faina
%%date 
 
%%tags donbass,interview,literatura,savenkova_faina,ukraina,vojna
%%title Год назад я взял интервью по ФБ у детской писательницы и, по совместительству, ребенка Фаины Савенковой
 
%%endhead 
 
\subsection{Год назад я взял интервью по ФБ у детской писательницы и, по совместительству, ребенка Фаины Савенковой}
\label{sec:15_09_2021.fb.zharkih_denis.1.interview_savenkova}
 
\Purl{https://www.facebook.com/permalink.php?story_fbid=3078199575726772&id=100006102787780}
\ifcmt
 author_begin
   author_id zharkih_denis,savenkova_faina
 author_end
\fi

Фаина Савенкова: "Я не знаю, что произошло потом и почему всё так изменилось,
но по какой-то причине Украина не захотела быть европейским государством, где с
уважением относятся к чужим ценностям". 

Год назад я взял интервью по ФБ у детской писательницы и, по совместительству,
ребенка Фаины Савенковой. Ребенок войны, она пишет интереснейшие вещи, которые
очень чисто и откровенно призывают к человечности. Фаина ведет общественную
работу в борьбе за мир на Донбассе. При этом она не стала иконой, плакатом,
который используют политики. Не думаю, что где-то могу разместить это интервью,
так что читайте его тут, а сам я рекомендую его к прочтению. 

\ifcmt
  pic https://scontent-frt3-1.xx.fbcdn.net/v/t1.6435-9/242062093_3078192892394107_6816385452011809422_n.jpg?_nc_cat=108&_nc_rgb565=1&ccb=1-5&_nc_sid=730e14&_nc_ohc=0aXgiiLFC4sAX_D-NPA&_nc_ht=scontent-frt3-1.xx&oh=6fb4cf2b995abb4514b85bfd3c44b86c&oe=616E8403
  @width 0.7
  %@wrap \parpic[r]
\fi

Фаина, прошел год, как мы с тобой говорили. Ты много где побывала, участвовала
во многих творческих проектах. Что изменилось за этот год, чему он тебя научил?

Добрый день, Денис . Рада с Вами снова поговорить. Этот год, в первую очередь,
заставил учиться на собственных ошибках и заниматься творчеством с их учетом. В
основном, это касается свободного времени, которое оказалось не таким уж и
«свободным». Теперь приходится учитывать, что в сутках 24 часа и нужно
выбирать, на что их потратить. Из-за этого приходится отказываться или
откладывать некоторые проекты.

Какие проекты больше всего запомнились, с какими людьми было интересно
работать? Почему? 

Может показаться, что в этом году у меня меньше работ, но это не совсем так.
Просто эти проекты были немного крупнее, чем предыдущие и каждый из них важен
для меня. С моим постоянным соавтором Александром Конторовичем мы написали
рассказ для сборника фантастики «Донбасс живет», роман «Стоящие за твоим
плечом» и сейчас дописываем «На несколько мгновений вперед». А люди… Может
быть, это из-за того, что у меня не такой большой опыт, но мне интересно
работать со всеми. У каждого человека можно научиться чему-то новому.

Назови трех самых интересных людей, с которыми пришлось работать. Какие они,
что тебе открыли?

Как я и говорила, у меня еще не такой большой опыт. Наверное, это мой соавтор –
Александр Конторович, мой учитель – Глеб Бобров и итальянский писатель и
журналист – Маринелла Мандоини.

Ты ребенок войны, тем не нее никогда это не подчеркиваешь не только в
общественной жизни, но и в творчестве. Это скромность или скрытая боль?
Насколько глубоко война в твоей душе?

Я всегда стараюсь избегать напрямую тему войны. Она, конечно, присутствует
иногда в моем творчестве и общественной жизни, но как фон. Мне неинтересны
военные действия, мне интересны люди, их судьбы. Ведь война показывает какой
человек на самом деле. Не думаю, что это скромность. Ну, а насколько глубоко
она в моей душе покажет время. Или расскажут близкие. Самой мне сложно о чем-то
говорить, потому что для меня всё происходящее привычно.

Понимаешь ли ты отчего началась эта война и как ее можно закончить? Твоя
версия.

Война началась, потому что она была нужна. Когда-то же Украина была единым
государством. Я не знаю, что произошло потом и почему всё так изменилось, но по
какой-то причине Украина не захотела быть европейским государством, где с
уважением относятся к чужим ценностям. Донбасс для меня многонациональный
регион, в котором люди с разными убеждениями и вероисповеданием могли долгое
время жить мирно, не навязывая свои ценности другим, но и не позволяя ущемлять
их права. Но потом кто-то решил, что национализм важнее мирной жизни и
процветания и решил навязать его Донбассу. Думаю, никому не понравится, когда
что-то заставляют делать насильно, вот и результат: разношерстный Донбасс
объединился против идеи украинского национализма. А решение… Нужно
договариваться. Разговаривать, а не стрелять. Иначе потом разговаривать будет
уже невозможно.

Ты обращалась к политикам и общественным организациям, как ты считаешь, они
тебя поняли или отписались? Нужны ли им были твои обращения? Был ли от них
толк?

Вы знаете, думаю, толк есть. Меня хотя бы услышали, вспомнили лишний раз, что
Донбасс восьмой год живет под обстрелами. А это большой плюс. С письмом Макрону
вышло смешно: мне ответил начальник его канцелярии. Вроде бы отписка. Мне
кажется, президент Франции его и не видел-то. Но на фоне гробового молчания
центральных изданий Франции, эту тему подняла независимая пресса. Как-то так
получилось, что заметили даже в Италии и Германии. Оказалось, что простых
жителей война на Донбассе волнует куда больше, чем правительства этих стран.
Поэтому мне кажется, что писать и делать публичные обращения обязательно нужно,
ведь это рано или поздно поспособствует прекращению обстрелов.

Ты говорила, что не против как-то приехать в Украину. Желание не прошло? С кем
бы там встретилась, какие бы проекты тебе были бы там интересны?

Не прошло. На Украине много красивых городов и хороших людей. Но пока у власти
такие люди, как сейчас, не уверенна, что смогу это сделать. А так с
удовольствием бы встретилась с Вами, с Любовью Титаренко. Хотела бы сходить на
выставку работ Владислава Ерко. Надеюсь, такая проводилась? Или будет. Насчет
проектов пока не знаю, как-то не задумывалась над этим.

Что главное для тебя в творчестве? О чем ты пишешь, что хочешь сказать людям?
Это удается?

Для меня творчество это возможность создавать то, что хочется. Сейчас объясню.
Многие творческие люди делают то, что хочет их зритель, читатель, издательство
или ориентируясь на модные конкурсные темы. Их интересуют продажи и рейтинги,
поэтому они становятся заложниками мнения других. Творчество превращается в
работу. Я пока свободна от рейтингов и желаний читателей и пишу, как хочется.
Из-за того, что я пишу для детей, немного сложно. Не хочется думать и решать
всё за читателей, поэтому пытаюсь оставить им в своих работах право выбора. А
для этого нужно задуматься и принять решение. Пока вроде бы получается.

Как ты относишься к своей известности? Люди понимают тебя такой какой ты есть,
или видят что-то другое?

Спокойно воспринимаю. Отношение людей зависит от того, как ты себя показываешь.
Я могу написать и пожелать доброго утра многим известным людям, поэтому все
почему-то считают, что я очень общительная и вообще ничего не боюсь. На самом
деле это не так. Не так давно мы с братом ходили в гости к ** и когда вернулись
домой, он сказал маме: «Представляешь, сидим мы в тишине, пьем чай и у меня
сустав хрустнул. Этот хруст потом эхом отражался от стен. Так неудобно было». У
меня вообще хватило сил только на то, чтобы глупо улыбаться. Зато когда пришли
домой, наперебой делились впечатлениями и рот было не закрыть в течение часа.
Это сейчас смешно вспоминать, а тогда было не так уж и весело. Мы с братом
очень домашние и общаться с кем-то вживую сложно. Интернет эту проблему с
легкостью решает. Но создает другую: реальный человек и картинка из соцсетей
получаются разными и не все понимают, почему так происходит, особенно те, кто
намного старше.

Назови три свои самые любимые книги. Чему они тебя научили?

Кир Булычев «Приключения Алисы», «Тайные начала» Пулмана и рассказы Чехова.
Они, пусть и по-разному, научили не сдаваться, если что-то не получается.

Если бы была волшебницей, что бы подарила людям?

Чувство сострадания и сопереживания к ближнему.

Бывают ли у тебя творческие неудачи?

Неудачи бывают у всех. Для меня неудача в творчестве – это когда ты не можешь
выстроить логику сюжета, ведь читатель не дурак, он все заметит. Есть у меня
несколько набросков, которые так и не стали готовыми произведениями именно по
этой причине.

Вот если Бог говорит: «С этого дня ты напишешь только одно произведение, но его
прочитают все в этом мире», то что бы написала, о чем?

Наверное, самый простой ответ: «Возлюби ближнего своего, как самого себя», если
я правильно помню цитату. Без этого люди перестанут быть людьми.

Как ты считаешь – вдохновение идет из боли или нет?

У кого как. У меня лично вдохновение идет от чистого голубого неба, дождика,
паучка на ветке. Или из любви к своим питомцам. Рассказ «Глициния и кошка» о
моей умершей кошке Соне. Может показаться, что он от боли, тем более, что
рассказ действительно выглядит грустным. Но это не так. Мне хотелось оставить в
памяти свою любимицу и запечатлеть ее, как героя этой истории, заодно подняв
тему бездомных животных. Соню ведь мы нашли на улице, замерзшую и голодную, а
потом забрали домой и подарили ей тепло.

Чему в жизни еще хотелось бы научиться?

Хотелось бы научиться рисовать. И выучить японский язык. Но я слишком ленивая,
к сожалению.

\ii{15_09_2021.fb.zharkih_denis.1.interview_savenkova.cmt}
