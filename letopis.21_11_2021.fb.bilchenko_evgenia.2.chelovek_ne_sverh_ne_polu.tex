% vim: keymap=russian-jcukenwin
%%beginhead 
 
%%file 21_11_2021.fb.bilchenko_evgenia.2.chelovek_ne_sverh_ne_polu
%%parent 21_11_2021
 
%%url https://www.facebook.com/yevzhik/posts/4460742243960851
 
%%author_id bilchenko_evgenia
%%date 
 
%%tags bilchenko_evgenia,chelovek,rusdusha,rusmir,russkij_chelovek,sobornist
%%title БЖ. Человек: не сверх- и не полу-
 
%%endhead 
 
\subsection{БЖ. Человек: не сверх- и не полу-}
\label{sec:21_11_2021.fb.bilchenko_evgenia.2.chelovek_ne_sverh_ne_polu}
 
\Purl{https://www.facebook.com/yevzhik/posts/4460742243960851}
\ifcmt
 author_begin
   author_id bilchenko_evgenia
 author_end
\fi

БЖ. Человек: не сверх- и не полу-

Михаил Юрьевич Елизаров. Лермонтовский отрешенный демон и насмешливый Дед
Лимонов в одном лице. Егор Летов, доживи он до симулякров чистого постмодерна и
озлобись до полной утонченности духа. Левый антиглобалист, потому что поет о
Сталине? Правый антиглобалист, потому что поет о Логосе и традициях?
Человек-избыток вне делений на красных и белых, на булкохрустов и пролетариев,
на "белой акации гроздья душистые" и "диктатуру пролетариата".

\ifcmt
  ig https://scontent-frt3-1.xx.fbcdn.net/v/t39.30808-6/258544905_4460742127294196_1025759145768473646_n.jpg?_nc_cat=104&ccb=1-5&_nc_sid=8bfeb9&_nc_ohc=e8G4MFECJqoAX-6eu3r&_nc_ht=scontent-frt3-1.xx&oh=5ac76a740388cfb4eb828666388da5f1&oe=619EAED3
  @width 0.4
  %@wrap \parpic[r]
  @wrap \InsertBoxR{0}
\fi

Говорят, что он - постмодернист. Так говорят консерваторы. Говорят, что он -
консерватор. Так говорят постмодернисты. Человек, который в одном тексте
сочетает аллюзии на Дугина и Жижека, с интертекстами и цитатами из советского
соцреализма, святоотеческой литературы и русской классики, где снобистские
термины номадического Делеза одобряются отборным подзаборным матом. Елизаров -
номад с больными от бездомности корнями, хуторянин, закосивший под хипстера.

Ужас для салонов? "Желчь мужей, услада жен"? Что бы сказал Пушкин? Цветаева?
Эдичка?

Эклектика ли в нем? Нет. Соборность. Соборность, снимающая деление на левых и
правых, красных и белых, марксистов и монархистов. Амплитуда русской души от
храма до кабака, от купола до бани, от крайности до крайности, как в кичевых,
но трушных приемах к/ф "Сибирский цирюльник" Никиты Сергеевича. В нем есть
хтонический темный Логос и Фаворский свет: он стоит на грани миров Самайна и
Иоанна Кронштадского, как и вся Русь, и от этого страшно и свято.

Елизаров похож на ту бабку, что за патриархальную семью. То есть, он правый.
Елизаров похож на ту бабку, что за государственную пенсию. То есть, он левый.
Бабку, если спросить, она - левая или правая, на три буквы пошлет. Елизаров -
тоже. Елизаров - бабка. Огромная снежная баба на русском поле экспериментов,
цельный скифский камень.

То, что он несёт высокий метанарратив русского, славянского, советского мира -
единого в своей духовности без всякой дурацкой его идеализации - очевидно. То,
что он непонятен ура-патриотам, - тоже очевидно. Им же нужны прямые клише и
трюизмы, а не двухходовки и оригинальные метафоры. Если Елизаров выражает
традицию в новой форме, интерпретируя Логос классики через постмодерную иронию,
это не значит, что он выстебывает классику. Имитируя постмодерн, он, как Дэвид
Линч, бичует сам постмодерн

Имитируя гротескный консерватизм, он высмеивает монстров наивного
традиционализма, вынося традицию на неимоверную трансцендентную высоту вне
догматизма и вне либерализма, одинаково ему чуждых и одинаково традицию
убивающих. Я думаю, многие его недолюбливают, потому что он делает русский мир
трендовым, модным, боевитым и конкурентоспособным, а не роется в комоде, чисто
по форме, ритуально. Он несёт живой орден деда так, что от крови сочится его
грудь: он не знает, как это делать ритуально и делает, как в окопе, бах - и
надел. Бах - и пошел петь.

В плане многоходовок - это трикстер. На Руси их звали "скоморохами", но в нем
больше сакрального юродства, чем профанной клоунады. Больше Мамонова, чем
Сукачева. Вон он стебется и матерится, а глаза - уязвимые и печальные. Мальчик
не доиграл в солдатики. "Плюшевый Мишутка шел войною прямо на Берлин..." Смело
крушил все, что попадалось ему под руку: выгоду, конформизм, ложь, двуличие,
толерантность...

Так до нас доживают изменившиеся до Пелевина и Прилепина, до Хаски и Рича -
Маяковский, Летов, Лермонтов и где-то даже Бродский: мешая слова в смачную кашу
сочных корневитых рифм, аллюзий и фонетических игр Мандельштама без ущерба
смыслам, без той сетевой лирики, в которую играет тик-ток-молодежь из мести
ватным своим предкам в стиле "розы/морозы".  Язык его/наш - наш/его пациент:
сам ломаю, сам и чиню, сам себе - деконструкция и экзегеза, ибо уже нет
никакого "Я", а только бытие и дом бытия - хацдеггерраский, мать его, душу нам
измотавший, язык наш.

Не думаю, что он мнит себя сверхчеловеком. Он внутренне считает себя тварью
дрожащей, но, максимум, на что он имеет право: обратиться к Господу на нашем
дурном чумном озорном больном - Одном - языке...

\#рецензииБЖ \#русскимбытьмодно
\ii{21_11_2021.fb.bilchenko_evgenia.2.chelovek_ne_sverh_ne_polu.cmt}
