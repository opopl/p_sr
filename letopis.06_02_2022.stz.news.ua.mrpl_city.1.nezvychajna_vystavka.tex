% vim: keymap=russian-jcukenwin
%%beginhead 
 
%%file 06_02_2022.stz.news.ua.mrpl_city.1.nezvychajna_vystavka
%%parent 06_02_2022
 
%%url https://mrpl.city/blogs/view/nezvichajna-vistavka
 
%%author_id demidko_olga.mariupol,news.ua.mrpl_city
%%date 
 
%%tags 
%%title Незвичайна виставка
 
%%endhead 
 
\subsection{Незвичайна виставка}
\label{sec:06_02_2022.stz.news.ua.mrpl_city.1.nezvychajna_vystavka}
 
\Purl{https://mrpl.city/blogs/view/nezvichajna-vistavka}
\ifcmt
 author_begin
   author_id demidko_olga.mariupol,news.ua.mrpl_city
 author_end
\fi

1 лютого в Музеї історії комбінату імені Ілліча відбулося відкриття  унікальної
виставки \enquote{Від душі і серця – в чарівний світ мистецтва}, присвяченої 125-річчю
комбінату.

Як наголосила директорка музею Ірина Бадасен, 

\begin{quote}
\enquote{виставка цікава тим, що вона
дуже різноманітна за якістю та кількістю робіт. Тут можна побачити роботи
авторів різного віку. Взагалі-то це дійсно від щирого серця і від усієї душі
привітання учасників виставки зі 125-річчям підприємства, на якому вони зараз
працюють, або працювали раніше. Хочеться, щоб люди у своїх роботах показували
власні емоції, передавали те, що вони відчувають, саме тоді їхні витвори
мистецтва виходять живими і більш цікавими}.
\end{quote}

\ii{06_02_2022.stz.news.ua.mrpl_city.1.nezvychajna_vystavka.pic.1}

На виставці представили свої роботи близько 30 художників та майстрів
декоративно-прикладного мистецтва. Авторами робіт є не тільки працівники
комбінату, але й їхні дружини та діти. Тематика виставки необмежена. Тут можна
побачити і вироби з мила, і скульптури з полімерної глини, і картини, вишиті
хрестиком та бісером. А майстриня декоративно-прикладного мистецтва Наталя
Толмачова вирішила представити широкому загалу свої унікальні квіти-світильники
з ізолону. Жінка розповіла, що цей матеріал витримує високу температуру і в
роботі плавиться, тобто Наталя може зробити з нього квітку будь-якої форми.
Спочатку вона працювала з фоаміраном, а потім, коли дізналася, що є ізолон,
перейшла на нього. Жінка дуже захопилася  виготовленням незвичних світильників,
прикрашених квітами. Вони можуть бути різного розміру і кольору. Якщо це
невелика квітка, Тетяна може впоратися протягом одного дня. Майстриня у свої
роботи вкладає всю свою любов та професіоналізм. У неї вже є багато замовників.
Цікаво, що якщо з електрикою жінці спочатку допомагав чоловік, то наразі жінка
свої вироби робить повністю самостійно. Три роки назад вона навчилася
встановлювати проводку для своїх світильників і відтоді нікому не довіряє цю
відповідальну справу. Для неї цей процес нагадує справжнє диво. Важливо, що за
квітами-світильниками з ізолону дуже легко доглядати. Самі квіти можна зняти і
окремо очистити від бруду за допомогою води або іншої рідини, для того, щоб
вони і через декілька років не втрачали своєї вишуканої краси.

\ii{06_02_2022.stz.news.ua.mrpl_city.1.nezvychajna_vystavka.pic.2}

Також на виставці відвідувачі побачать чудові картини з пейзажами  природи та
маріупольською архітектурою, автором яких є талановитий маріуполець Олександр
Туляков. Художник щоденно невтомно працює над відточенням власної майстерності
і незабаром можна  буде побачити персональну виставку чоловіка. Для виставки
чоловік обрав найбільш професійні, на його думку, картини. В них дуже  багато
дрібних деталей, тому працювати над ними було не так легко. Роботи, виконані в
настільки насичених тонах, що спочатку здається, що це не картини, а справжні
фото.

\ii{06_02_2022.stz.news.ua.mrpl_city.1.nezvychajna_vystavka.pic.3}

На виставці вирішила представити свої роботи і  Тетяна Наконечна – одна з
перших в Маріуполі майстринь, яка працює в техніці джутової філіграні. Її
донька, Дорош Наталя, за  допомогою цієї техніки створює картини, а сама Тетяна
працює над більш об'ємними роботами. Може зробити і лампи, і шкатулки, і
серветниці. В Україні ця техніка існує всього 10 років. Джутова філігрань
дозволяє створювати витончені ажурні вироби як для себе, так і  для дому. Це
мистецтво виникло ще в давнину, коли в якості матеріалу використовували
дорогоцінні металеві нитки. Сьогодні їх чудово замінює шпагат джутовий, з яким
працювати набагато простіше. Весь сенс техніки полягає в скручуванні товстих
ниток, які згодом формують ажурний візерунок. За основу взята ювелірна техніка
– скань. Різниця полягає лише в тому, що джут набагато легше скріплювати клеєм,
щоб картинка не розпалася. Багато новачків для створення шедеврів
використовують готові шаблони та трафарети, а от досвідчені майстри самі
вигадують схеми завитків. І той і інший варіант не забороняється. Вироби можуть
фарбуватися та освітлюватись. Тетяна в цій техніці працює вже 2 роки. За цей
час вона вдосконалила свою майстерність. Над великими виробами може працювати
близько місяця. Для жінки створення оригінальних витворів мистецтва стало
справжньою віддушиною та можливістю саморозкриття.

\ii{06_02_2022.stz.news.ua.mrpl_city.1.nezvychajna_vystavka.pic.4}

На закритті виставки відбудеться ярмарок, на якому можна буде придбати будь-яку
роботу, що сподобалася найбільше. Майстри та художники можуть придбати одне у
одного роботи, це ж можуть зробити і відвідувачі виставки. Ціни досить
невисокі, а товари дійсно унікальні. Зокрема, наручний годинник з мікровишивкою
з авторською розробкою схеми коштує всього 300 грн. Водночас автори робіт
залишають свої візитівки і кожен охочий може зробити індивідуальне замовлення з
власними побажаннями. Виставка \enquote{Від душі і серця – в чарівний світ мистецтва}
триватиме до 1 березня. Всім рекомендую відвідати цю незвичайну виставку і на
власні очі подивитися на унікальні витвори мистецтва, які створили талановиті
маріупольці.

\ii{06_02_2022.stz.news.ua.mrpl_city.1.nezvychajna_vystavka.pic.5}
