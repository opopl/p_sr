% vim: keymap=russian-jcukenwin
%%beginhead 
 
%%file 17_06_2021.fb.filatova_elena.1.foto_vov_soldat_dusha_vzgljad.cmt
%%parent 17_06_2021.fb.filatova_elena.1.foto_vov_soldat_dusha_vzgljad
 
%%url 
 
%%author 
%%author_id 
%%author_url 
 
%%tags 
%%title 
 
%%endhead 
\begin{itemize}

\iusr{Antonio Zzr}

Годы холодной, жаркой погоды. Переживания и переживания во время
харизматической схватки меняют не только внутреннее состояние человека.
Давайте благодарим Бога за мир и давайте позаботимся о хороших межличностных
отношениях.

\iusr{Константин Бабенко}

Год за 10

\iusr{Spasenija Sefčik}
Потрясающе ! Ах , война , каждая дорога - Смоленская дорога!

\iusr{Радован В. Маљковић}

То је чудо шта све рат учини од човека.

\iusr{Павел Русский}

Редки были те которые в бое прожили вся 4 года той войны, это не была
обыкновенная война, это была самая страшная война которая в истории
человечества случилась. Этот парень очень много того видел и пережил. Были
редкие те люди которые пережили 4 года на передовой. Историю парня я не знаю,
возможно он ещё был и в плену где его пытали, но я уверен по его взягляду что
он к сожалению, после всего что ему случилось не мог вернуться совсем
нормальной жизны. На этому фото он не постарел 10, а 20 и больше лет. Но и
человек после 20 лет своей нормальной жизни не получит такой дикий взгляд в
глазах, сразу видно что у его были сильные психические травми, но какая это
была война, это, к сожалению, вообще не удивляет. Вечная память герою!

\iusr{Елена Филатова}

Павел Русский Да, он был в концлагере.2 года.Бежать из концлагеря Кобытеву и
его товарищам удалось только в 1943 году. Скрывались по болотам, прятались у
местных крестьян, добрались до своих. Евгений Степанович снова встал в строй. С
боями он прошел через Украину, Молдавию, Польшу, Германию. Войну Е. С. Кобытев
окончил в Дрездене, в звании старшего сержанта. Был награжден орденом \enquote{Красной
Звезды} и медалью \enquote{За Победу над Германией}.

\iusr{Павел Русский}

Елена Филатова Спасибо за ответ, вот это я не знал. Он был известным художником
после войны в СССР, я тоже прочитал статью где его дочь рассказывает про своего
отца и видел его рисунки немецких преступников которые он нарисовал в плене.
Видел и его послевоенные фотографии когда он уже был профессором, можно сказать
слава Богу его настоящее лицо вновь вернулось в порядок, глаза успокоились.
Вечная память герою и спасибо ему за победу!

\iusr{Елена Филатова}

Павел Русский Вернувшись после войны в Красноярск, Евгений Кобытев работал,
преподавал, стал главным художником города. В 1958 году был приглашен в только
что открывшиеся Красноярское художественное училище им. В. И. Сурикова. За 6
лет работы Евгения Степановича педагогическое отделение закончили 60 человек,
20 из них стали членами Союза художников.

\ifcmt
  pic https://scontent-mia3-2.xx.fbcdn.net/v/t1.6435-9/201074648_321554302784719_4733745410961984479_n.jpg?_nc_cat=105&ccb=1-3&_nc_sid=dbeb18&_nc_ohc=cr7Dc32ozdAAX-crUqW&_nc_ht=scontent-mia3-2.xx&oh=9024e783d87bc17eb7e21fc925f5567d&oe=60D015AA
	width 0.5
\fi

Большинство горожан знает Евгения Кобытева как монументалиста, автора большого
мозаичного панно на бывшем кинотеатре \enquote{Родина} (ныне Дом дружбы
народов), материалом для этой мозаики послужила енисейская речная галька.

\iusr{Елена Филатова}

Павел Русский Но главными работами в жизни художника стали произведения,
посвященные \enquote{Хорольской Яме}. Две серии графических работ - «До
последнего дыхания» (реквием по узникам), «Люди, будьте бдительны!» (приговор
палачам) и книга \enquote{Хорольская Яма}. В 1960 году он приезжал в г. Хорол,
где встречался с выжившими бывшими узниками и крестьянами, которые помогали
заключенным выживать. Через четыре года Евгений Степанович вновь вернулся в эти
места и привез сюда выставку своих работ, посвященных \enquote{Хорольской Яме}.
На выставку пришли тысячи людей. В его работах не было вымышленных персонажей,
зрители узнавали и себя и своих мучителей. Для Кобытева это был творческий
отчет перед очевидцами тех страшных событий и выполнение обещания данного
самому себе в лагере смерти – выжить и рассказать в своих работах правду о
великом страдании и великом подвиге народа.

\ifcmt
  pic https://scontent-mia3-1.xx.fbcdn.net/v/t1.6435-9/201162329_321555092784640_8622690036640252216_n.jpg?_nc_cat=100&ccb=1-3&_nc_sid=dbeb18&_nc_ohc=EnzRb0JMH9cAX99AFdU&tn=ntrKbsW_7ChXu3v-&_nc_ht=scontent-mia3-1.xx&oh=1fe70c9c3ce71e6fdc532e5231e72e73&oe=60D0BE73
	width 0.3
\fi

\iusr{Роман Ойра-Ойра}
От большинства, бывшего на фронте с 41го по 45й, остался бугорок с жестяной звездой. Парню повезло.

\iusr{Tatiana Dovnar}

Светлая память и низкий поклон!
\end{itemize}
