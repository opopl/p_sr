% vim: keymap=russian-jcukenwin
%%beginhead 
 
%%file 21_01_2022.fb.fb_group.respublika_lnr.1.ljudi_dela
%%parent 21_01_2022
 
%%url https://www.facebook.com/groups/respublikalnr/posts/958646534771138
 
%%author_id fb_group.respublika_lnr,zimina_olesja
%%date 
 
%%tags dnr,donbass,lnr,vojna
%%title ЛЮДИ ДЕЛА. ПОЕДИНОК ВОЛИ И ДУХОВНОЕ ИСПЫТАНИЕ
 
%%endhead 
 
\subsection{ЛЮДИ ДЕЛА. ПОЕДИНОК ВОЛИ И ДУХОВНОЕ ИСПЫТАНИЕ}
\label{sec:21_01_2022.fb.fb_group.respublika_lnr.1.ljudi_dela}
 
\Purl{https://www.facebook.com/groups/respublikalnr/posts/958646534771138}
\ifcmt
 author_begin
   author_id fb_group.respublika_lnr,zimina_olesja
 author_end
\fi

ГАЗЕТА \enquote{РЕСПУБЛИКА} (№3, 2022г).

ЛЮДИ ДЕЛА. ПОЕДИНОК ВОЛИ И ДУХОВНОЕ ИСПЫТАНИЕ

22 января 2015 года развернулось масштабное наступление объединенных сил
Луганской и Донецкой Народных Республик

В дни Чернухино-Дебальцевской операции, когда небо и земля горели от шквального
огня врага, славные потомки победителей в Великой Отечественной войне яростно
сражались с оккупантом за каждый метр родной земли. Именно благодаря лучшим
сынам Отечества в этом году мы будем отмечать 77-ю годовщину Великой Победы.

\subsubsection{ДЕЛО СОВЕСТИ}

Старший уполномоченный по особо важным делам Управления уголовного розыска МВД
ЛНР Сергей Жуков служит в правоохранительных органах почти 30 лет. Считает
борьбу с преступностью своим призванием. Воспитанный советской школой на
высоких идеалах, после армии твердо решил идти в органы внутренних дел. А
эталоном лучших человеческих качеств для него всегда был отец.

– Он всю жизнь работал. Например, весь водопровод по всей Луганской области он
прокладывал. Для меня это самый идеальный человек, который вдохновлял на
свершения. А государство нас воспитывало на героях Великой Отечественной войны.
Пример этих достойных людей сподвиг нас в 2014 году встать на защиту нашей
земли. 

\ii{21_01_2022.fb.fb_group.respublika_lnr.1.ljudi_dela.pic.1}

Отслужив в органах 25 лет, в 2014 году Сергей Жуков мог спокойно выйти на
заслуженный отдых, но наш собеседник такой вариант даже не рассматривал. 

– Я остался в Луганске и уверен, что сделал правильно. У меня сын спрашивает:
«Папа, почему ты не уехал, ведь многие уехали?» Я говорю ему, что если бы
уехал, то как бы я ему в глаза смотрел. Ведь рано или поздно он обязательно
спросил бы, почему я не остался на Родине. И никто меня не упрекнет в том, что
я струсил и не отстоял свою точку зрения. 

Оставаясь в рядах правоохранителей, он с сослуживцами вступил в ополчение. В
составе 32 человек служили в разведке. Новосветловка, оборона Луганска,
счастьинское направление, аэропорт… Полученные знания в Советской армии и
боевой опыт 2014 года пригодился в Чернухино-Дебальцевской военной кампании в
составе сводного полка МВД ЛНР.

– Зимой 2015 года у нас спросили: кто поедет в Чернухино? С нами были еще
ребята из общественной безопасности во главе с Тутовым Романом Николаевичем и
Череповским Алексеем Викторовичем. Нам поступила команда выдвинуться к
Чернухинской птицефабрике. Мы должны были проводить зачистку территории
поселка. Нас было 52 человека. Мы приехали на окраину Чернухино, только
спрыгнули с брони, начался артобстрел. Позиции заняли кто где успел. У меня был
первый номер на пулемете, со мной еще был сослуживец с позывным «Академик», и
мы заняли позиции в полуразрушенном доме возле креста при въезде. 

Спустя семь лет Сергей Жуков уже с усмешкой вспоминает о своем укрытии, которое
сложно назвать надежным. А тогда подразделение на трое суток попало в
окружение. С бойцами у командования была только связь, а подвоз боеприпасов был
полностью блокирован противником. 

– Мы уже по опыту летних боев знали, что тебя спасает только то, что у тебя с
собой. Поэтому боекомплекта брали столько, сколько могли унести на себе. Это в
большей степени нас всех и спасло. Держали оборону, выполняли задачу «Позиции
не оставлять». Через трое суток к нам смогли прорваться ребята. Позиции мы свои
не оставили.

Сейчас весь накопленный боевой опыт Сергей Анатольевич использует и в жизни, и
в работе. Выезжая на полигоны, он уже наставляет молодых полицейских,
подсказывает, в какой ситуации как лучше действовать, чтобы выполнить
поставленную задачу и сохранить жизнь себе и товарищам.

Сергей Жуков награжден медалями «За оборону Луганска», «За отвагу» II степени,
«За веру, за волю».

\subsubsection{РАЗВЕДКА ЖИЗНЬЮ}

Мальчишеское увлечение оружием и военной историей переросло для старшего
разведчика разведвзвода отдельного танкового батальона старшего сержанта
Народной милиции ЛНР Бориса Дорофеева в профессию. Как и большинство советских
ребят, он самостоятельно изучал виды вооружения, а его кумиром был маршал
Жуков.

– В детстве я не мечтал быть военным, но мысли такие посещали. В школе больше
для себя изучал историю, особенно Второй мировой войны, потому что тогда
появилось вооружение, которое мне очень интересно – это и стрелковое, и
бронетехника, и авиация. Если взять Первую мировую, там оружие очень
слабенькое.

Судьба все-таки связала Бориса Дорофеева с силовыми структурами, где он получил
солидный боевой опыт.

– Я считаю, что военные делятся на три основных категории: солдат, боец и воин.
Я себя причисляю, может это и громко прозвучит, к воинам. 

Во время проведения Чернухино-Дебальцевской операции Борис служил в
спецподразделении СОБР и, как десятки сослуживцев, примкнул к сводному отряду
МВД. Он так же, как и Сергей Жуков, пережил трехсуточное окружение. 

– Когда я поступал на службу в СОБР, командир сразу нам сказал, что возможно
участие в боевых действиях. Все были согласны. Прибыли на крест, начался
обстрел. Как были все в чистеньком, так и попадали тут же в грязь. Бежать особо
некуда, все кругом заминировано. Помню, случай был – одно отделение зачищало
правую сторону дороги, второе – левую. День солнечный был. Открываю дверь в
один дом, вижу силуэт. Я пулемет вскинул, но вовремя рассмотрел. Бог меня
отвел... Женщина гражданская была и паренек с ней. Местные оказались. Не уехали,
потому что боялись оставить дом...

Наш герой утверждает, что служба в спецназе и армии отличается в корне именно
тактическими действиями. Спецназ работает узконаправленно – провели штурм
здания, зачистили. А у военных поле действий гораздо обширнее. Но ежедневные
тренировки с оружием, практические стрельбы, тактические занятия – все это не
просто пригодилось зимой 2015 года, а и сказалось на выносливости. В ходе боев
за освобождение Чернухино Борис Дорофеев получил ранение – вражеский осколок
попал в коленную чашечку. Бойца этот факт не смутил, он самостоятельно извлек
осколок и продолжил воевать.

Борис Дорофеев награжден медалью «За отвагу» II степени

\subsubsection{СОЛДАТЫ В БЕЛЫХ ХАЛАТАХ}

Наравне с военнослужащими, в Чернухино-Дебальцевской кампании участвовали и
медики Луганской Народной Республики. Они спасали, лечили, ставили на ноги
бойцов и мирных жителей. И опыт 2014 года им стал в этом не просто серьезным, а
жизненно важным подспорьем.

– Уже было отработано и оказание первичной медицинской помощи непосредственно в
Народной милиции, – рассказывает главный внештатный хирург Министерства
здравоохранения ЛНР, ректор Луганского государственного медицинского
университета имени Святителя Луки, почетный гражданин Луганска Александр Торба.
– Пациентов уже доставляли обработанными, с остановленным кровотечением. Это
было очень важно для нас. И доставляли не только бойцов, но и мирных жителей,
которые попадали под обстрелы. На местах по пути следования от Чернухино до
Луганска были рассредоточены пункты оказания помощи в крупных медицинских
учреждениях – это районные и городские больницы. А венчала это все Луганская
республиканская клиническая больница. 

Круглосуточно работали и бригады скорой медицинской помощи ближе к полю боя.
Медики уже имели опыт оказания помощи при боевой травме, и, конечно же, это
положительно сказалось на качестве лечения. Уже было меньше осложнений, лечение
протекало легче, снизился процент инвалидности. 

– Конечно, характер повреждений был значительный – это и минно-взрывные травмы,
было очень много и огнестрельных ранений. Из-за позиционных боев было много
повреждений головы, шеи, – вспоминает Александр Владимирович. – Также было
много повреждений от снайперского огня. Они хоть и тоже огнестрельные, но
отличались тем, что наносились точечно. Некоторые такие травмы были не типичны
для характера ведения боя, а наносились с неким форсом: сначала в стопу –
человек падает на колени, затем в бедро, и последний выстрел в голову.

Еще одним характерным моментом для Чернухино-Дебальцевской операции было
большое количество мин-ловушек, на которых при зачистке территорий подрывались
наши бойцы и мирные граждане. 

Подчеркивает Александр Торба и высочайшее значение в работе всей медицинской
службы среднего и младшего персонала, который не покладая рук ухаживал за
ранеными, круглыми сутками дежурил в отделениях.

– Слаженность работы медицинской службы Народной милиции и Министерства
здравоохранения сложилась благодаря санинструкторам, то есть людям, которые
оказывают первую помощь, – поясняет наш собеседник. – И от того, как она
выполнена, зачастую зависит жизнь человека. Средний и младший медицинский
персонал оказывает помощь не только на поле боя, но и в госпиталях, в
клинических учреждениях выхаживает пациентов: выполняют назначения врачей,
участвуют в операциях. Например, операционные медсестрички остаются у
операционного стола с пациентом постоянно, в то время как врачи сменяются:
полосной хирург сделал операцию, затем травматолог, после зашел нейрохирург… А
в условиях боевых действий пациентов подвозят постоянно, а сестричка одна и
одна санитарочка. А тогда был острый дефицит персонала, но люди работали
самоотверженно. Ведь медики начали активно возвращаться как раз после
Чернухино-Дебальцевской операции.

Помимо работы в Луганске, хирурги выезжали оперировать в другие города.
Санавиация не прекращала работать, в частности, оперировались тяжелые пациенты,
которых нельзя было транспортировать в столичные медучреждения. 

Весь колоссальный багаж знаний не пылится на полках в сухих отчетах, а
используется для воспитания следующих поколений медиков.

– Все полученные нами знания и опыт уже преподаются и студентам, и ординаторам,
и врачам на повышении квалификации. Проводятся курсы по боевым повреждениям, а
в последующем это выльется и в научные работы, – утверждает Александр Торба. –
Нас, например, учили на характерах повреждений времен Великой Отечественной
войны и Афганистана. Но с тех времен сильно поменялось вооружение, а
соответственно, и характер повреждений, что повлекло и изменения в оказании
медицинской помощи.

Помимо своих наработок, мы используем в первую очередь опыт российских коллег.
Так как одна из самых лучших доктрина оказания медицинской помощи в войсках
именно в Российской Федерации. Также нам интересен опыт коллег по лагерю СНГ,
интересные работы и китайских военных медиков. 

Лично для Александра Владимировича 2014 год не стал первым боевым опытом.
Впервые он столкнулся с военными травмами в Африке. 

– Но у нас, конечно, характер повреждений гораздо серьезнее. Если первый опыт
был связан преимущественно с огнестрельными ранениями и единичными случаями
минно-взрывных травм, то в 2014 году мы столкнулись с последствиями
массированных артиллерийских обстрелов, и среди пострадавших в 70 процентах
случаев было гражданское население. 

Но все равно проще работать на своей земле, чем на чужбине. Горько и обидно за
то, что государство стало расстреливать своих граждан, но здесь наша Родина, и
наши медики наравне с военными защищают свою землю. 

Александр Торба, кандидат медицинских наук, доцент:

«Мы используем опыт всех стран, потому что в медицине нет политических
разграничений, и помощь мы оказываем как нашим бойцам, так и раненым с
сопредельной стороны. Никогда мы не делали разграничения, хотя у нас работают
хирурги, которые потеряли в войну свои семьи. Мы оказываем помощь людям, хотя
не всегда их так можно назвать. Для остальных разбирательств с ними есть
компетентные в этом вопросе органы. Для нас долг превыше всего. На этом
зиждется и российская медицина, и наша».

Оксана ЧИГРИНА, 

фото Ксении ЩЕРБИНЫ

ГАЗЕТА \enquote{РЕСПУБЛИКА} (№3, 2022г).

\#газета \#республика \#годовщина\_Чернухинско\_Дебальцевкой\_операции
\#воспоминания\_участников\_событий
