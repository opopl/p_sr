% vim: keymap=russian-jcukenwin
%%beginhead 
 
%%file 24_06_2021.fb.buzhanskii_max.1.knizhnii_arsenal
%%parent 24_06_2021
 
%%url https://www.facebook.com/permalink.php?story_fbid=1975723755925427&id=100004634650264
 
%%author 
%%author_id buzhanskii_max
%%author_url 
 
%%tags festival.kiev.knizhnii_arsenal,kiev,kniga,kultura,literatura,mechta,ukraina
%%title Всю жизнь мечтал попасть на Книжный Арсенал
 
%%endhead 
 
\subsection{Всю жизнь мечтал попасть на Книжный Арсенал}
\label{sec:24_06_2021.fb.buzhanskii_max.1.knizhnii_arsenal}
\Purl{https://www.facebook.com/permalink.php?story_fbid=1975723755925427&id=100004634650264}
\ifcmt
 author_begin
   author_id buzhanskii_max
 author_end
\fi

Всю жизнь мечтал попасть на Книжный Арсенал.
Зря.
Сегодня наконец то попал.
Зря.
Мечта должна оставаться мечтой.
Имеете представление, как выглядит блошиный рынок?
Груда абсолютно случайно подобранного барахла, среди которого случайно же попадается что то интересное, вокруг которой бесцельно бродит десяток человек, уныло это разглядывающих?
Вот ровно та же история.
Какие то горы классики, которую, будем откровенны, мало покупают, разве что в рамках школьной программы и которая ну никак не вызывает у посетителей-читателей-покупателей интерес.
Переводы Мартина, которому уже сто лет в обед, и который за новинку никак не выдашь.
Майдан, Сони Кошкиной, которому 6 лет!!!
Вопрос не в Соне, и не в Майдане, просто за 6 лет работы издательства, для него просто не может найтись место на прилавке среди новинок.
А нет новинок.
Книг на русском языке- ну процентов 5 максимум, и это с ооочень большим запасом.
Вопрос, в данном случае, не языковый абсолютно, это вопрос рынка.
Просто представьте себе ресторан, на дверях которого вы вешаете объявление - брюнетов не обслуживаем, и это при том, что у вас по статистике брюнеты чаще обедают.
Как думаете, успешный бизнес выйдет?
Нет?
А теперь представьте, что вы табунами бегаете за государством и требуете содержать ваше издательство( ресторан), потому что сами вы финансово нежизнеспособны.
Должно ли государство помогать?
Конечно, это важно.
Но помогать, а не манную кашу в рот класть.
Мне возразят, что коронавирус здорово ударил по издательствам, и это правда.
Но, тем не менее, ощущение случайности издаваемого, блошиности рынка и по масштабам, и по сути, не покидает.
Жаль, что ушёл без книг, надеюсь другим повезёт намного больше.
(Ну вру, одну таки сфотографировал, заинтересовала, погуглю, и наверно куплю, Клеопатра издательства Фабула, русскоязычного варианта нет, как покупатель я им не нужен, но хрен там, державна так державна, иногда)

\ifcmt
  pic https://scontent-lga3-2.xx.fbcdn.net/v/t1.6435-0/p526x296/207156002_1975723732592096_5277524346427948545_n.jpg?_nc_cat=109&ccb=1-3&_nc_sid=8bfeb9&_nc_ohc=coYE4fe-j_kAX_ktlVh&_nc_oc=AQmDbH9U3NdxgwSp8QTflKPl0m_HUVk8cnqQA4I-qU4DWYK2_oBUAV2ezNPywatCSXE&_nc_ht=scontent-lga3-2.xx&tp=6&oh=9df2d10398b29f71f3b8acaeff5597db&oe=60DA1225
\fi
