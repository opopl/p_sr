% vim: keymap=russian-jcukenwin
%%beginhead 
 
%%file 21_11_2021.fb.bilych_andrij.tjachiv.ukraina.1.den_gidnosti_i_svobody.pic.9.cmt
%%parent 21_11_2021.fb.bilych_andrij.tjachiv.ukraina.1.den_gidnosti_i_svobody
 
%%url 
 
%%author_id 
%%date 
 
%%tags 
%%title 
 
%%endhead 

\iusr{Georg Tenevoy}
Рояль здорової людини

\iusr{Lana Rössner}
Гідна людина грає на роялі руками...

\iusr{Al Jazira}

\ifcmt
  ig https://scontent-frx5-2.xx.fbcdn.net/v/t39.1997-6/s168x128/16781161_1341101952618574_7704631035023065088_n.png?_nc_cat=1&ccb=1-5&_nc_sid=ac3552&_nc_ohc=wCsSskO9MIYAX_IWgnv&_nc_ht=scontent-frx5-2.xx&oh=abc1a27bcd228deacb83ed2573198ab7&oe=61A2442B
  @width 0.1
\fi

\iusr{Al Jazira}

\ifcmt
  ig https://scontent-frx5-2.xx.fbcdn.net/v/t39.1997-6/p480x480/91521538_1030933857302751_5093925307199520768_n.png?_nc_cat=1&ccb=1-5&_nc_sid=0572db&_nc_ohc=MMTXJEU6G7gAX8SkEUO&_nc_ht=scontent-frx5-2.xx&oh=d9496d52ca86285372f799b52f11667e&oe=61A206F7
  @width 0.1
\fi

\iusr{Daria Dosyn}
Білоруський варіант боротьби з диктаторським режимом

\begin{itemize} % {
\iusr{Alona Trohova}
\textbf{Daria Dosyn} при чем Белоруссия. @igg{fbicon.shrug} 

\iusr{Daria Dosyn}
\textbf{Alona Trohova} - я не знаю при чому тут ваша Белоруссия, і де вона находиться ; Я про Білорусь ( Біла Русь, а не Біла Росія ) і про її післявиборний протест 2020 року без» майданутих»з квіточками і роялями на площі.

\iusr{Татьяна Луговская}
\textbf{Daria Dosyn} 

а при чому Білорусь до Української Революції Гідності, до Майдану? Михайло
Жизневський - громадянин Білорусі загинув на Майдані, але в Україні не було
ніякого білоруського варіанту боротьби з диктаторським режимом. У нас свій,
український.

\iusr{Daria Dosyn}
\textbf{Татьяна Луговская} - Ви мабуть забули як починався Майдан, за що стояв
і як називався? Революцією гідності і достоїнства він почав називатись з 1
грудня після побиття студентів. І це зовсім два різні Майдани з вимогами,
учасниками і емоціями. А тепер пригадайте як він все-таки починався наш майдан
( піаніст вам в підказку), його динаміку і чим все закінчилось. І порівняйте
його з Білорусь літа 2020. Хіба з не з такою надією і впевненістю в свою
правоту і справедливість мирними починами хотіли ми змін в державі в листопаді
14 рокі? Здається один народ тільки в різних порах року з нац прапорами,
квітами, танцями, співами і піаністами на площах. До того моменту коли нас і їх
почали бити. Ми вже дозріли, у нас за плечима були два майдани, не рахуючи
різні протести, коли биті небитих возили, у нас терпець урвався, ми оскалилися
і дали по зубам. І стояли до кінця. З цього часу нас і почали називати
майдануті. А Євромайдан різко перейшов в Революцію. Ми різні. Білоруси довго
перебували в летаргічному сні. Лише літом 2020 проснулися, і то не всі. І то їх
розбудила сотня молодих мінчан. Тай по менталітету вони наївні, добрі,
безобідні, як діти. Не має в них ще сили духу і свободи ,як і генів бунтарства.
Ми потомки козаків, гайдамаків, бандерівців. Вони тихі, спокійні, як їхні ліса
і болота і картопляні поля.. І на послідок, щоб ви зрозуміли в чому різниця,
нагадаю пророчі слова С. Бандери: «той хто між хлібом і свободою вибирає хліб, в
кінцевому результаті залишиться і без свободи і без хліба « 

\end{itemize} % }


