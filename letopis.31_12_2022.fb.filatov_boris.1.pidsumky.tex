% vim: keymap=russian-jcukenwin
%%beginhead 
 
%%file 31_12_2022.fb.filatov_boris.1.pidsumky
%%parent 31_12_2022
 
%%url https://www.facebook.com/permalink.php?story_fbid=pfbid02SyJUcFvFvygnVjwLBiWfqiZ3pTvV19EakUV2SiYzg94RwdwdEKFtVAMKdSvDHkm2l&id=100002157183088
 
%%author_id filatov_boris
%%date 
 
%%tags 
%%title Шановні земляки, дорогі друзі! Ось і настав час підбити підсумки 2022-го року
 
%%endhead 
 
\subsection{Шановні земляки, дорогі друзі! Ось і настав час підбити підсумки 2022-го року}
\label{sec:31_12_2022.fb.filatov_boris.1.pidsumky}
 
\Purl{https://www.facebook.com/permalink.php?story_fbid=pfbid02SyJUcFvFvygnVjwLBiWfqiZ3pTvV19EakUV2SiYzg94RwdwdEKFtVAMKdSvDHkm2l&id=100002157183088}
\ifcmt
 author_begin
   author_id filatov_boris
 author_end
\fi

\obeycr
Шановні земляки, дорогі друзі! 
Ось і настав час підбити підсумки 2022-го року. 
Ми прожили разом всі його складнощі. Стояли пліч-о-пліч, працювали, допомагали, підтримували, раділи, переживали, турбувалися одне про одного і про наше велике та славне Місто. 
...
Ми впоралися.
Бо Дніпро — опора українських захисників.
Багато зусиль ми доклали, аби забезпечити наших оборонців необхідним і укріпити об'єкти критичної інфраструктури – насамперед підстанцій ДТЕК і Укренерго. 
Силами міста було вирито близько 100 км протитанкових рвів. 
Також комунальна техніка була задіяна для облаштування кількох рівнів оборони не лише Дніпра, а й всієї області. 
- На потреби військових було вивантажено та розвезено близько 3 тис. кубів лісу.  
- Забезпечено понад 600 вантажівок і пікапів, 7 машин швидкої; 
- Близько 7 тис. бетонних блоків для ЗСУ та ТРО і майже 200 бетонних споруд;
- Близько 10 тис. сталевих протитанкових їжаків;
- 200 тис. літрів пального – це майже 20 бензовозів. 
- Близько 5,5 тис. одиниць екіпірування – від бронежилетів і шоломів, до термобілизни і зимових спальних мішків;
- Майже 350 дронів різних моделей і функцій;
- Понад чверть сотні антидронових рушниць;
- Майже пів тисячі радіостанцій;
- Майже повтора десятка ретрансляторів;
- Понад 3 тис. ноутбуків, планшетів, камер відеоспостереження, портативних навігаторів та різної комп'ютерної техніки;
- Майже 100 тепловізорів;
- Майже 1,3 тис. акумуляторних батарей;
- Близько 700 ІР-телефонів;
- Близько 100 генераторів;
- 7,5 тис пар берців.
А ще – зимовий одяг, предмети побуту, продукти харчування, питна вода і медикаменти. 
...
Я вимагав від фахівців мерії бути максимально корисними також і технологічними рішеннями.
Так було створено вже відомий проєкт захисту неба від ворожих дронів, для якого ми допомогли бійцям модернізувати деяку зброю. 
А ще місто виплачує субвенції військовим частинам ЗСУ. 
У бюджеті на 2023 рік на підтримку оборонців передбачено пів мільярда.
...
Місто, де допоможуть всім, хто цього потребує.
Ми монетизували допомогу нашим землякам, які постраждали від ракетних атак рф. 
Це мільйони гривень на відбудову житла, лікування або ж фінансову підтримку тим, хто втратив близьких. 
У найкоротші терміни місто замінило вікна і покрівлі у понад 310 багатоповерхівках, понівечених ворожими обстрілами.
Дніпро прийняв і роздав тим, хто цього потребував, більше 270 фур різноманітних гуманітарних вантажів. 
Це понад 6 тис тонн продуктів та предметів побуту. 
Місто зберегло соціальні послуги та ініціативи і розробило нові програми. 
Бо у цій війні Дніпро став новим домом для майже 190 тисяч переселенців. 
Це більше ніж у Львові чи інших західних регіонах.
Ми можемо запропонувати переселенцям роботу.
Місто має достатньо вакансій. Будь ласка, за всіма консультаціями звертайтеся в
мерію або використовуйте наш телеграм-канал: \url{http://telegram.me/zanyatost_bot}.
...
Дніпро – місто, де дбають. 
Як і у всі попередні роки, розвиток медицини залишається пріоритетом. 
Якщо в цифрах, то попри війну цьогоріч було відремонтовано відділення чотирьох лікарень. 
Закуплені нові рентгени, ангіограф, ендоскопічне, УЗД та ЛОР обладнання.
Місто відкрило 5 амбулаторій та один з найбільших в Україні хоспісів. 
Аби піклуватися про самотніх і невиліковно хворих. 
Лікарні й медцентри працюють стабільно. Мають необхідне обладнання, запаси води, медикаментів, а також генератори. Наші славетні лікарі працюють 24/7.
...
Місто, що має друзів по всьому світу.
За останні місяці Дніпро уклав угоди про партнерство з містами Осака (Японія) та Кельн (Німеччина). 
Вони допомагають медикаментами, гуманітарними вантажами, автівками та консультаціями.  
Кельн також передав нам велику партію генераторів. 
Зараз ми ведемо низку перемовин про дружбу і партнерство з іншими містами світу. 
Бо це відкриває великі можливості для Дніпра – зараз та вже після Української Перемоги. 
...
Місто, де готові до будь-яких викликів.
Дніпро уникнув тотальної кризи під час блекаутів. 
Ще з лютого ми прораховували багато різних сценаріїв. 
Не лише як велике мільйонне місто, а і як центр військової промисловості та стратегічних підприємств.
Ми продовжували реновацію обладнання:
Модернізували майже 40 котлів;
Поставили пів сотні енергоощадних насосів;
Оновили десятки кілометрів тепломереж.
Місто придбало близько 20 великих та потужних генераторів для найбільших котелень і водоканалу. Щоб підтримувати системи і швидко відновлювати їх після повного знеструмлення.
На правому березі є майже половина зі 100 запланованих ємностей з технічною водою – на 2,2 тис літрів кожна.   
На лівобережжі продовжують облаштовувати 51 свердловину з  колонками для набору. 
І ще раз – ми робимо так, виходячи з особливостей геологічних порід ґрунтів, а не тому, що нам так хочеться. 
Комунальники і міські служби мають чітку вказівку: робити все,  аби в оселях людей було світло та тепло. А депутатам, керівникам департаментів і управлінь, заступникам мера чітко доведено завдання чи не цілодобово залишатися в контакті з усіма містянами, які чекають вирішення проблеми в своїх будинках довше, ніж інші.
Крім цього, ми продовжували і інші міські справи. 
Ремонтували дороги та зливову каналізацію. 
Дерусифіковували назви наших вулиць, прибираючи з них символи тоталітарного спадку. 
У майже 650 багатоповерхівках зробили різні ремонти. 
Зокрема і в будинках ОСББ, для яких місто виконує зобов'язання за програмою підтримки. 
Ми збудували майже 200 укриттів у школах і на вулицях. 
Впоралися після того, як російський «іскандер» знищив понад сотню автобусів і зуміли з нуля налагодити комунальні перевезення.
...
Ми доклали максимум зусиль, аби місто жило нормальним життям – наскільки це можливо. 
І будемо надалі робити для цього все. 
Тож працюємо, віримо в ЗСУ і разом йдемо у 2023-й — рік Перемоги.
Усіх обіймаю!
На зв'язку.
Ваш міський голова. 
Злий, але щирий))
\restorecr
