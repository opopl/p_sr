% vim: keymap=russian-jcukenwin
%%beginhead 
 
%%file 25_04_2022.stz.pc.ua.dou.1.geroizm_pogljad_kapellana.2.raketnyj_udar
%%parent 25_04_2022.stz.pc.ua.dou.1.geroizm_pogljad_kapellana
 
%%url 
 
%%author_id 
%%date 
 
%%tags 
%%title 
 
%%endhead 

\subsubsection{Історія про ракетний удар}

Тиждень тому, 15 квітня окупанти нанесли декілька ракетних ударів по моєму
місту та ще декілька по околицях. Я в той час не міг заснути, чув і бачив майже
всі вибухи. Від четвертого удару мій п'ятиповерховий будинок почав хитатись і я
зрозумів, що, мабуть, пора вже їхати до бомбосховища. Чесно кажучи, я був в
шоці. Але це було десь хвилини три, потім я взяв себе в руки, та почав думати,
кому я зараз ще можу допомогти.

Страх — це нормально. Важливо, що ти будеш з ним робити. Запанікуєш чи візьмеш
себе до рук.

\ii{25_04_2022.stz.pc.ua.dou.1.geroizm_pogljad_kapellana.2.raketnyj_udar.pic.1}

Спускаючись, я постукав до своїх сусідів і запропонував поїхати разом зі мною.
Тому до підвала разом зі мною доїхали ще три людини + кошеня. До речі це було в
перший раз за всю активну фазу війни, коли я поїхав у підвал. Наступного ранку
я мав їхати до Львова на зустріч з сином та дружиною, вони якраз мали
проїжджати кордон з Польщею. Тому, не поспавши і години, о п'ятій ранку я
поїхав по Житомирській трасі на зустріч з сім'єю. Від поворота на Стоянку до
початку Житомирщини немає майже нічого цілого. Дуже багато будинків та інших
споруд було зруйновано.
