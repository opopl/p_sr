% vim: keymap=russian-jcukenwin
%%beginhead 
 
%%file 14_05_2022.fb.fb_group.ua.armia.24_ombr.1mp.1mb.1.medik
%%parent 14_05_2022
 
%%url https://www.facebook.com/groups/495136843994910/posts/2144890915686153
 
%%author_id fb_group.ua.armia.24_ombr.1mp.1mb
%%date 
 
%%tags 
%%title Вивожу поранених, - дуже скромно каже бойовий медик Дмитро
 
%%endhead 
 
\subsection{Вивожу поранених, - дуже скромно каже бойовий медик Дмитро}
\label{sec:14_05_2022.fb.fb_group.ua.armia.24_ombr.1mp.1mb.1.medik}
 
\Purl{https://www.facebook.com/groups/495136843994910/posts/2144890915686153}
\ifcmt
 author_begin
   author_id fb_group.ua.armia.24_ombr.1mp.1mb
 author_end
\fi

- Вивожу поранених, - дуже скромно каже бойовий медик Дмитро. Вже третій місяць
він на східному фронті рятує українських воїнів. 

- Я, коли без роботи, всі радіють. Хлопці розуміють, якщо я нічого не роблю, то
значить все добре.

Та за нинішньої інтенсивності боїв, днів без роботи небагато. Дмитро готовий 24
на 7, будь-якої миті, виїжджати на евакуацію.

- Визначена точка, куди ми під'їжджаємо і забираємо трьохсотого. Хлопці до цієї
точки виносять його на ношах. Це може бути кілометр, а може бути й більше.
Евакуаційна машина не заїжджає прямо на позиці, бо росіяни не мають честі, у
них не існує правил війни. По моїй маркованій хрестами машині стріляли з танку.
Водій мій отримав контузію. Було й таке, що під'їхати не міг і хлопці 3,5
кілометри виносили пораненого посадками. Винесли й попадали. Але я їм дуже
вдячний за це. 

Дмитро тримає руку на своєму рюкзаку. Там все, що перемотує рани і спиняє кров.
Це найнеобхідніше. Для медика, його рюкзак - це його зброя. Бо має єдину бойову
задачу - врятувати життя воїну. 

- Вони дуже чекають і дякують. А мені більше нічого й не треба. Привіз і
розумієш, що передав живого. Вивозили й гусянкою. Погода така була. Але за
людьми треба їхати. Хлопці дуже сподіваються на мене. Вони чекають мене і я
мушу дістатися до них за будь-яких умов.

Раніше Дмитро не служив. Коли почалося повномасштабне вторгнення в Україну, був
за кордоном.

- Два місяці тому я знаходився у славному місті Дубаї з паралімпійською збірною
України. Я медик нашої збірної зі стрільби з луку. З чемпіонату світу я приїхав
за дві доби у військкомат. Мені сказали медики  потрібні і я швидко пішов. Ну,
життя трохи перемінилось. Кульбіт такий нормальний.

Життя Дмитра перевернулося з ніг на голову. Та медик каже, найбільший дар мати
шанс рятувати тих, хто наближає перемогу для України.

\ii{14_05_2022.fb.fb_group.ua.armia.24_ombr.1mp.1mb.1.medik.cmt}
