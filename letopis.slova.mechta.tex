% vim: keymap=russian-jcukenwin
%%beginhead 
 
%%file slova.mechta
%%parent slova
 
%%url 
 
%%author_id 
%%date 
 
%%tags 
%%title 
 
%%endhead 
\chapter{Мечта}

%%%cit
%%%cit_head
%%%cit_pic
%%%cit_text
В ту пору все младшие отпрыски знатных родов \emph{мечтали} поскорее надеть длинный
белый плащ с черным крестом - традиционное облачение
рыцарей-тамплиеров. Уже от самого слова "тамплиер" веяло духом дальних
странствий и подвигов, оно вызывало в воображении корабли, идущие на
Восток с гордо раздутыми парусами, страны, где вечно сини небеса, коней,
мчащих всадников в атаку через пески пустыни, все сокровища Аравии,
богатый выкуп за пленников, отбитые у врага города, отданные на поток и
разграбление, крепости, к которым от морского берега ведут гигантские
лестницы. Говорили даже, что у тамплиеров есть свои тайные гавани, откуда они
отплывают в неведомые земли... И свершилась \emph{мечта} Жака де Молэ: одетый в
роскошный плащ, складки которого спадали до золотых шпор, он горделиво шагал
по далеким городам... Он достиг высших ступеней иерархии Ордена, таких,
которых никогда и не надеялся достичь, получил все титулы и
наконец по выбору братьев-тамплиеров занял высший пост Великого магистра
Франции и заморских стран, имел под своим началом пятнадцать тысяч рыцарей. И
все кончилось этим склепом, этой грязью, этими лишениями. Редко на чью долю
выпадала такая неслыханная удача и на смену ей приходило столь глубокое
унижение
%%%cit_comment
%%%cit_title
\citTitle{Железный король}, Морис Дрюон
%%%endcit

%%%cit
%%%cit_head
%%%cit_pic
%%%cit_text
Звичайно, все це наразі виглядає таким далеким від нагальних проблем пандемії,
вагнергейту, енергетичної кризи, концентрації ворожих військ на українських
кордонах... Позатим той, хто не \emph{мріє} про власний дім, хто у своїх мріях
не малював сотню разів кожну кімнату, кужню, санвузол, коридор, не продумував
електичну проводку, подачу та відведення води, екстер'єр та інтер'єр, ніколи не
житиме у комфортному домі...  У другій розмові з Ruslan Rokhov ми говоримо про
переваги і недоліки прем'єрської республіки. Бо це тільки здається, що прем'єр
приречений бути набагато слабшою постаттю за президента. Нагадаю, що саме
прем'єрами були батько ФРН Аденауер, батько Сінгапуру Лі Куан Ю та батько
Ізраїлю Бен Гуріон.  Прем'єрами були легендарні Вінстон Черчиль, Маргарет
Тетчер, Віллі Брандт, Гельмут Коль та Голда Меїр. Прем'єр-міністрами є Віктор
Орбан та Нарендра Моді, яких ніяк не назвеш слабкими лідерами...  Так, замість
змагатися за місце за кермом заіржавілої української державної колимаги, ми
\emph{мріємо} про майбутню українську теслу. Ми почали проговорювати
конструкцію держави, яку нам ще лише належить побудувати
%%%cit_comment
%%%cit_title
\citTitle{Пересесть со старой ржавой колымаги на новую теслу / Лента соцсетей / Страна}, 
Геннадий Друзенко, strana.news, 26.11.2021
%%%endcit

%%%cit
%%%cit_head
%%%cit_pic
%%%cit_text
Не случайно теперь для большинства рядовых украинцев не осталось ни денег, ни
защиты, ни даже \emph{мечты}. Почему мы дошли до этой «точки»? Да потому, что у
наших топ-чиновников и у представителей крупного капитала мозги должны стать
суверенными, а не их дома на солнечной стороне Швейцарии или Испании.  Пора
перестать говорить, что эти украинцы - «хорошие», а эти – «плохие», что эти -
«патриоты», а эти - нет. У нас по-прежнему идет очень жесткое сталкивание
одного общественного слоя с другим, противопоставление одних регионов другим.
Не поэтому ли возникает некая фобия, и мы упорно ищем в своем прошлом не лучшее
для подражания, а худшее. Но не будет у нас уже никогда иного прошлого.
Друзья, давайте осознаем, что мы сами сделали такими наших доморощенных
политиков и олигархов, мы сами им позволили такими стать. И теперь эти наглые и
малообразованные ребята во власти навязывают нам свои принципы, мораль и
религию, очень быстро превратив нас из космической державы в страну-попрошайку.
А мы что, а мы боимся, боимся ответить и навести порядок. Мы нерешительны,
почему? А вдруг вой поднимется? А вдруг скажут, что это наступление на свободу
слова и государственность, а вдруг грубо ударят по голове.  Ну, и дальше готовы
это терпеть? Ведь эта проблема во сто крат страшнее из-за того, что это уже
устойчивая тенденция для Украины. Но кто-то же должен обратить внимание на эти
миллионные несправедливости в нашей стране
%%%cit_comment
%%%cit_title
\citTitle{Для обычного украинца справедливость важнее, чем закон / Лента соцсетей / Страна}, 
Александр Гончаров, strana.news, 08.12.2021
%%%cit_url
\href{https://strana.news/opinions/366081-dlja-obychnoho-ukraintsa-spravedlivost-vazhnee-chem-zakon.html}{link}
%%%endcit
