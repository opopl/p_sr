%%beginhead 
 
%%file 24_01_2023.fb.marchuk_marina.mariupol.1.v_s_m_rok_v_nazad_ma
%%parent 24_01_2023
 
%%url https://www.facebook.com/permalink.php?story_fbid=pfbid02Yaz2yFmKkYiq22kwJ4hj3eUftNQB3RSntF54AqGscz1G1v6g6toq2oEJ3MpZM4dCl&id=100046294254050
 
%%author_id marchuk_marina.mariupol
%%date 24_01_2023
 
%%tags mariupol
%%title Вісім років назад маріупольці вперше на собі відчули, що таке "руський мир"
 
%%endhead 

\subsection{Вісім років назад маріупольці вперше на собі відчули, що таке \enquote{руський мир}}
\label{sec:24_01_2023.fb.marchuk_marina.mariupol.1.v_s_m_rok_v_nazad_ma}

\Purl{https://www.facebook.com/permalink.php?story_fbid=pfbid02Yaz2yFmKkYiq22kwJ4hj3eUftNQB3RSntF54AqGscz1G1v6g6toq2oEJ3MpZM4dCl&id=100046294254050}
\ifcmt
 author_begin
   author_id marchuk_marina.mariupol
 author_end
\fi

😥Вісім років назад маріупольці вперше на собі відчули, що таке \enquote{руський мир}.
Тоді, 24 січня 2015 року, проросійські терористи обстріляли житлові квартали
мікрорайону \enquote{Східний}

🥺Загинула 31 людина, понад 100 - отримали поранення різного ступеня тяжкості.
Постраждали близько 100 житлових будинків і споруд. 

😔Серед тих, хто постраждав - 6-річна дівчинка Мілана, яка залишилася без
матері. Її мама загинула, прикриваючи її від залишків снаряду. Дівчина втратила
ніжку, пережила кілька операцій і заново вчилася ходити. Ця історія вразила
весь світ.  У 2018 році на 15-поверховому будинку створили відомий мурал
\enquote{Мілана}. Він став символом трагедії та надії на мир й відновлення🕊

Ми сподівалися, що більше подібне не повториться. Але 24 лютого 2022 року
росіяни вирішили знищити мирний Маріуполь. Після окупації стерли й мурал про
Мілану, аби переписати історію міста та свої ганебні злочини. 

Проте ми пам'яємо про кожну трагічну дату в історії Маріуполя! У злочинах рф
проти народа України не має строка давності...
