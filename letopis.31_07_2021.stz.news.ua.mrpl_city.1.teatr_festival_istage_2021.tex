% vim: keymap=russian-jcukenwin
%%beginhead 
 
%%file 31_07_2021.stz.news.ua.mrpl_city.1.teatr_festival_istage_2021
%%parent 31_07_2021
 
%%url https://mrpl.city/blogs/view/teatralnij-festival-istage-2021
 
%%author_id demidko_olga.mariupol,news.ua.mrpl_city
%%date 
 
%%tags 
%%title Театральний фестиваль iStage 2021
 
%%endhead 
 
\subsection{Театральний фестиваль iStage 2021}
\label{sec:31_07_2021.stz.news.ua.mrpl_city.1.teatr_festival_istage_2021}
 
\Purl{https://mrpl.city/blogs/view/teatralnij-festival-istage-2021}
\ifcmt
 author_begin
   author_id demidko_olga.mariupol,news.ua.mrpl_city
 author_end
\fi

\ii{31_07_2021.stz.news.ua.mrpl_city.1.teatr_festival_istage_2021.pic.1}

Вже незабаром у Маріуполі відбудеться ще одна унікальна по\hyp{}дія. З 11 до 15
серпня у Центрі сучасного мистецтва \enquote{Готель Континенталь} (Харлампіївська,17) в
рамках проєкту \enquote{Маріуполь – велика культурна столиця} проходитиме \emph{iStage 2021 –
п'ятиденний фестиваль-лабораторія ідей актуального театру}. Це нова фестивальна
платформа для діалогу мовою театрального мистецтва. У рамках заходу будуть
презентовані театральні проєкти з різних регіонів України, що віддзеркалюють
контекст сьогодення.

\ii{31_07_2021.stz.news.ua.mrpl_city.1.teatr_festival_istage_2021.pic.2}

За словами куратора фестивалю \emph{\textbf{Андрія Палатного}}, програма ІStage складається з
дуже цікавих театрів, які є лідерами в своїх містах. Кожен з них – це унікальна
візитівка міста і всі вони об'єдна\hyp{}ні такими складовими, як \emph{актуальність,
інноваційність, контекст}. Важливо, що за всіма театрами стоїть дуже цікава
ідея. В рамках фестивалю вперше будуть об'єднані і найкращі зразки, які існують
наразі в театральній індустрії країни. Звісно, представлені будуть не всі
театри, але це певна панорама і саме маріупольці зможуть насолодитися та
дізнатися більше про найбільш інноваційні театри України. Після перегляду
вистав будуть створені додаткові дискусії-спілкування з творцями кожного
спектаклю, де можна буде напряму задати питання, які виникнуть під час
перегляду. Окрім цього планується провести перший стрімінговий театральний
фестиваль, тобто всі ці вистави будуть транслюватися на всю країну. Водночас
організатори фестивалю працюють над додатковою програмою – це ексклюзивні
покази вже записаних відео-вистав, які відбувалися під час карантину, що будуть
показані офлайн просто неба у дворі ЦСМ \enquote{Готель Континенталь} і онлайн. Вистави
йтимуть різними мовами: більшість українською, деякі російською, в міжнародних
проектах –  англійською. Всі спектаклі будуть представлені у Центрі сучасного
мистецтва \enquote{Готель Континенталь} – на великій, малій сценах та на подвір'ї.

Програма фестивалю об'єднає 7 українських театральних команд з 5 міст: Києва,
Львова, Івано-Франківська, Харкова та Дніпра:

\begin{itemize} % {
\item \href{https://www.facebook.com/teatrDakh}{Центр сучасного мистецтва \enquote{ДАХ}}\footnote{\url{https://www.facebook.com/teatrDakh}} \enquote{Про що казав Заратустра} [Цифрові театральні уроки];  
\item \emph{Незалежний театральний проєкт} \enquote{Осінь на Плутоні};
\item \href{https://www.facebook.com/teatrVerim}{Театр \enquote{Віримо}}\footnote{\url{https://www.facebook.com/teatrVerim}} \enquote{Однокласники};
\item \href{https://www.facebook.com/wildtheatre.ua}{Дикий театр}\footnote{\url{https://www.facebook.com/wildtheatre.ua}} \enquote{Нові Шрами};
\item \href{https://www.facebook.com/leftbanktheater}{Київський академічний театр драми і комедії на лівому березі Дніпра}%
\footnote{\url{https://www.facebook.com/leftbanktheater}} \enquote{Клас};
\item \href{https://www.facebook.com/teatr.neft}{\enquote{Нефть}}%
\footnote{\url{https://www.facebook.com/teatr.neft}} \enquote{Страждання на Гончарівці};
\item \href{https://www.facebook.com/dramteatr.if}{\emph{Івано-Франківський національний академічний обласний му\hyp{}зично-драматичний театр імені Івана Франка}} та команда \href{https://www.facebook.com/bigprojectw}{\enquote{Project \enquote{W} – Veterans, Volunteers and William}}%
\footnote{\url{https://www.facebook.com/bigprojectw}} \enquote{Дванадцята ніч, або Що захочете}.
\end{itemize} % }

\ii{31_07_2021.stz.news.ua.mrpl_city.1.teatr_festival_istage_2021.pic.3}

Ексклюзивною спец-подією фестивалю iStage 2021 стане перформанс \enquote{R u there? We
r here} (\enquote{Ви там і тут}) – проєкт, що створений у колаборації між Маріуполем та
Лондоном! У рамках програми міжнародних театральних резиденцій Taking the Stage
2.0 режисерка  \emph{\textbf{Josie Dale-Jones}}, спільно з командою Народного Театру
\enquote{Театроманія}, під керівництвом \textbf{Антона Тельбізова}, досліджувала соціальні теми
за допомогою слів, зображень та руху. Проєкт готувався протягом 3 місяців і
дозолив ще більше розкрити потенціал маріупольських акторів. Кожен глядач зможе
побачити себе зі сторони та стати учасником цієї вистави. Антон Тельбізов
сподіається, що подібні фестивалі дозволять подивитися на театр по-новому та
краще зрозуміти його універсальність.

\ii{31_07_2021.stz.news.ua.mrpl_city.1.teatr_festival_istage_2021.pic.4}

Паралельно з фестивалем iStage 2021 триває \emph{\textbf{Марафон міжнародних театральних
резиденцій.}} Це проєкт, що поєднує ідеї європейських режисерів та мистецьких
команд міста у спільних театральних експериментах. Першою із запрошених митців
стала \emph{\textbf{Madeleine Bongard}} –  швейцарська режисерка, акторка та педагогиня. Вона
проведе серію акторських практик, що досліджують усвідомлення присутності.
Результатом стане перформанс \enquote{Enter the Lab}, що увійде в основну програму.

Фестиваль iStage 2021 реалізується в рамках проєкту культурно-мистецьких
ініціатив \emph{\enquote{Діалог мовою мистецтва} за підтримки Українського культурного фонду.}

Потрапити на всі події фестивалю можна буде безкоштовно за попередньої
реєстрації, про яку буде повідомлено на сторінці Великої культурної столиці
України. Маріуполь та івенті фестивалю iStage. Ніяких обмежень щодо кількості
реєстрацій не буде, тобто один глядач зможе зареєструватися на всі події
фестивалю. Головне – встигнути зареєструватися вчасно. Кількість відвідувачів
буде обмежена відповідно до місць в залах приміщення. Куратор фестивалю Андрій
Палатний наголосив, що 
\begin{quote}
\em\enquote{цей захід потрібний, щоб показати, що український театр
за останні 6 років зробив дуже серйозний крок вперед, адже такі фестивалі
допомагають зафіксувати певний етап, на якому знаходиться театральна індустрія
та зрозуміти, яку роль театр відіграє в суспільстві і чи відіграє взагалі.
Головний меседж iStage 2021 (фестивалю-лабораторії ідей актуального театру) –
зрозуміти, в якому напрямі рухається театральна індустрія, які теми глядач хоче
бачити на сцені театрів та яким буде майбутній розвиток театрі України}.
\end{quote}

Про що нині говорять українські театри? Що за ідеї стоять за кожною з вистав?
На ці питання маріупольці зможуть знайти відповіді вже під час театрального
фестивалю iStage 2021.
