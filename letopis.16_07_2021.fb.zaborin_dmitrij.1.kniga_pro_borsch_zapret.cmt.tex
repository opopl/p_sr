% vim: keymap=russian-jcukenwin
%%beginhead 
 
%%file 16_07_2021.fb.zaborin_dmitrij.1.kniga_pro_borsch_zapret.cmt
%%parent 16_07_2021.fb.zaborin_dmitrij.1.kniga_pro_borsch_zapret
 
%%url 
 
%%author_id 
%%date 
 
%%tags 
%%title 
 
%%endhead 
\subsubsection{Коментарі}
\label{sec:16_07_2021.fb.zaborin_dmitrij.1.kniga_pro_borsch_zapret.cmt}

\begin{itemize} % {
\iusr{Александр Якутёнок}
Однозначно работа выполнена, вдувать через суд.

\iusr{Aleksei Dremov}
Борщ наш- это сильно!  @igg{fbicon.face.tears.of.joy} 

\iusr{Валентина Гаташ}
Вы "Фолио" и его директора недооценили )

\iusr{Михаил Намятов}
А какие "щедрые" расценки. 25 грн за тыЩу знаков - заоблачная плата. Очередной пример, как "выгодно" как писательское, так и любое другое ремесло в/на Украине.

\begin{itemize} % {
\iusr{Александр Карпец}
\textbf{Михаил Намятов} Вот и я о том же ниже написал. На нищем сайте и то немного больше платят )
\end{itemize} % }

\iusr{Ольга Леонова}
Це однозначно зрада

\iusr{Иван Бодрийяров}
жесть какая))

\iusr{Vitalii Kulyk}

Печатайся в «Яузе»:) как раз твой формат

\begin{itemize} % {
\iusr{Дмитрий Заборин}
А там наверняка скажут, что в книге явный бандеровский душок и нужно доказать, что борщ - русский.

\iusr{Алексей Стаценко}
\textbf{Дмитрий Заборин} А можно спросить напрямую, если есть желание.

\iusr{Алексей Стаценко}
\textbf{Дмитрий Заборин} В личку стукнись, если тебе оно надо.
\end{itemize} % }

\iusr{Вера Едемская}
А какой общий объем рукописи, если не секрет? в знаках или а.л.

\begin{itemize} % {
\iusr{Дмитрий Заборин}
8 а.л.

\iusr{Вера Едемская}
\textbf{Дмитрий Заборин} фундаментально
\end{itemize} % }

\iusr{Максим Небесный}
Мрак. Ответ от издательства писала явно свидомая кастрюля

\iusr{Igor Igor}
Ой хорош! Ай да макнул мордой в это самое. В борщ!

\iusr{Сергей Киселев}

Письмо издательства прекрасно. Отличная реклама может получиться.

А второй лист письма можно выложить?

\begin{itemize} % {
\iusr{Дмитрий Заборин}
\textbf{Сергей Киселев} тогда надо все четыре уже

\iusr{Сергей Киселев}
\textbf{Дмитрий Заборин} выкладывайте! Насладимся слогом)))

\iusr{Miroslava Berdnik}
\textbf{Сергей Киселев} это какой-то ПЦ! Красовицкий настолько тютюкнулся?!
\end{itemize} % }

\iusr{Артем Сергеевич}

В то время как наш гарант, не щадя живота своего, расповидает про единство
нации, это странное выдавництво пропагандирует нарративы кремля про различия,
федерализацию, сепаратизм и сеет сомнения в легитимности войны на Сходи.

Куда только СБУ смотрит?

\iusr{Сергей Киселев}
Кроме шуток, такая книга могла бы быть очень успешной

\iusr{Денис Рафальский}
После всех лоббистских потуг Красовицкого бегу от любого стенда продукции "Фолио". Пусть печатает на растопку.

\iusr{Евгений Ясенов}
Следовало ожидать

\iusr{Вавилова Елена}

В этой истории удивляет только одно: заказывая книгу, директор не понимал с кем
имеет дело? Или торопился сварганить борщ и мысли не допускал, что кто-то
посмеет посягнуть на святое? Насчёт суда терзают сомнения. Сейчас борщ-наш -
такой же маркер как Крым. Но уверена, что есть издательства, которые возьмутся
напечатать книгу. Могу судить об этом хотя бы по Вашим постам. Их всегда очень
интересно читать - литературный дар присутствует определенно.

\iusr{Константин Бондаренко}

Сохраню для истории)))

\iusr{Алексей Заровский}
Так не нужно было врать про триединство, нужно было написать обжигающую правду про шумеров. Заплатили бы по три тысячи за лист.

\iusr{German Polovnikov}
Какой знатный взрыв 5-й точки у директора  @igg{fbicon.smile} 
Дмитрий - 5 баллов (или 12 по новому)

\iusr{Ольга Семченко}
Это просто прекрасно! Поздравляю  @igg{fbicon.hands.raising} 

\iusr{Ирина Шабанова}
Предварительно можно было заглянуть на страницу фб Красовицкого.... там так всё однозначно руссофобски...

\iusr{Юрий Ткачук}
Света Крюкова, по поводу книги, опять бы пост написала. Может и не один...

\iusr{Владислав Гребенюк}
Рука Кремля,, @igg{fbicon.grin}{repeat=3} 

\iusr{Андрей Блинов}

Снимаешь видео. И его посмотрят куда больше, чем тираж Красовицого в пару тыщ
единиц. Пишешь пост по каждой главе - дискуссия куда содержательнее, чем на
литературном форуме одобрямсов

\begin{itemize} % {
\iusr{Дмитрий Заборин}
\textbf{Андрей Блинов} А кто мне даст тыщу гривен? А? И расписаться на чем? "От автора"

\iusr{Андрей Блинов}
\textbf{Дмитрий Заборин} тебе нужен курс по монетизации в YouTube. Но ты прав: тыщу не дадут, дадут больше, но тебе же нужна именно тысяча от Фолио  @igg{fbicon.wink} 

\iusr{Derlyuk Andrey}
\textbf{Дмитрий Заборин}, поддерживаю Андрея Блинова по части реализации труда в массы и его монетизации, тема хайповая и разлетится как пирожки по просторам ютуба. А по поводу суда - бессмысленная трата времени, средств и нервов. Лучшее наказание для Фолио будет успех твоей работы.
\end{itemize} % }

\iusr{Дмитрий Железняк}
Дмитрий Дмитрий... А вы ж вроде бы Взрослый, Серьезный человек... А пошли посотрудничать с фашистским книгопроизводством... Не стоило оно того изначально, на мой взгляд.

\iusr{Дмитрий-Собор Соколов}
Условия договора выполнены, должны заплатить.
Дальше их право - печатать или сжечь на площади.
\textbf{\#Славанации}!

\iusr{Александр Карпец}
С патриотизмом головного мозга в этом издательстве усьо ясно, хотя когда-то вроде было уважаемой конторой.
Не ясно, шо это за такой "щедрый" гонорар ))

\iusr{Алексей Рубцов}
Дмитрий, спасибо вам, простое спасибо  @igg{fbicon.hands.pray} 

\iusr{Елена Бугло}
 @igg{fbicon.man.facepalming}{repeat=3} 

\iusr{Алексей Тесаловский}
Договор бы почитать.

\iusr{Ivan Vesnin}

Дядя Сэм в своем беспробудном и безоглядном русофобстве поднял на вершины этой
территории самых отъявленных, безнадежных лузеров. Они теперь глаголят с полной
уверенностью "колпака белого господина"... Ну, в Афганистане было то же самое.

\iusr{Vladimir Yegorov}
Альтернативно обделенным нет покоя
"Изменим прошлое к лучшему!"

\iusr{Артем Сергеевич}
Пан Красовицкий в каждом предложении своего труда упоминает о России и русских, тогда как Украину вспоминает значительно меньше, что не совместимо с должностью гендиректора уважаемого издания Фолио!

\iusr{Vladimir Green}
Обязательно жарить! )))

\iusr{Игорь Киржнер}
Саша Красовицкий - редкостный персонаж. Кроме холуйской любви к деньгам - ничего

\iusr{Григорий Шувалов}
Огонь! Особенно про упоминание России большее число раз, чем Польши. Про перевод слова "русский" тоже зачетно.

\iusr{Наталья Дмитриченко}
Вот это ты описал борщец, что тебя посылали на на целых 4 страницы...  @igg{fbicon.wink} , почитать интересно, без редактирования... Чтобы судиться, нужно наверное договор почитать, что там у них об отказах пишут...

\iusr{Roman Jet Li}

Все основания обратиться в суд и взыскать с них полную сумму, обозначенную в
договоре.

Процесс нужно делать публичным, ибо животные иного не разумеют. После первых
заседаний скорее всего иск признают и попросят добровольное возмещение, без
морального вреда.

Нужен профессиональный юрист, который помножит их на 0

\iusr{Александр Тимошенко}
Поздравляю!

\iusr{Сергей Ракитин}
Продать конкурентам? )

\iusr{Margo Weis}
Любой приличный адвокат порвет издательство как Тузик тряпку.
P.S. Борщ не ем. С детства терпеть не могу.

\iusr{Miroslava Berdnik}

А еще как доказательство украинскости борща Красовицкий мог бы вспомнить
амеркианского прокурора Марту Борщ, которая поднялась сильно на деле лазаренко

\begin{itemize} % {
\iusr{Александр Тимошенко}
\textbf{Miroslava Berdnik} и вышла замуж
\end{itemize} % }

\iusr{Наталия Сафонова}

Ой, как интересно!! Истина никого не интересует - главное чтобы быть в струе
госполитики!!)) Ведь всё равно ЭТо безумие когда-то кончится и им будет стыдно
за свои "идеологии украинскости борща"!

\begin{itemize} % {
\iusr{Ольга Войтенко}
\textbf{Наталия Сафонова} я опасаюсь, что это безумие очень нескоро кончится

\iusr{Наталия Сафонова}
\textbf{Ольга Войтенко} лучше бы скоро, но увидим....

\iusr{Ольга Войтенко}
\textbf{Наталия Сафонова} дожить бы до этого дня

\iusr{валентин генчев}
\textbf{Наталия Сафонова} Стыдно не будет. Они будут называть себя тогда "гонимыми" и бегать к дяде Сэму и МИ6 за поддержкой.
\end{itemize} % }

\iusr{Ольга Войтенко}

Какой потрясающе наукообразный бред! И эти люди что-то смеют говорить о
манипуляциях?! \textbf{Дмитрий Заборин} странно, что они вообще обратились к вам, а не
к, допустим, Ницой

\begin{itemize} % {
\iusr{Ирина Ольховка}
Потому что у ницой не набралось бы 40 тыс. слов. У не было бы три: Борщ-наш. ТОЧКА!)
\end{itemize} % }

\iusr{Victor Ostrovsky}
Вам нужно издать книгу о Вашей поисковой деятельности. У вас талант писателя публициста, можно просто ничего не меняя издать все ваши посты и фотографии на эту тему, и это будет бомба.

\iusr{Валентин Жаронкин}
Давай еще напиши, что цветочные мотивы в вышиванках - это с каталогов Брокара 19 века, которые коробейники по селам продавали. Так и без гражданства остаться можно.

\iusr{Tashpoisk Tatyana}
Печатай книгу самиздатом, миллион платформ, раз им не надо.

\iusr{Александр Минский}
Мне кажется, это отличная реклама для этой книги, Дмитрий. Как сейчас говорят "хайп". Но и по суду ещё стрясти с болезных надо )

\begin{itemize} % {
\iusr{Михаил Викторович Никипелов}
\textbf{Alexzender Minskiy}, тьоплого тобі за комір.

\iusr{Александр Минский}
\textbf{Михаил Викторович Никипелов} смотри в штаны теплого не подпусти
\end{itemize} % }

\iusr{Jacques Felix Chouan}

Хм... Поржал. А ,,науковый" специалист С. Сегеди, приведенный в качестве
примера в отмазке, где и когда доказал приоритеты украинцев на слова ,.Русь" и
,.русин"? Мне как венгру это безумно интересно. (Ну, и как выпускнику истфака
Геттингенского универа тоже.) Ссылку на эту таинственную работу можно получить
от директора издательства?


\iusr{Jacques Felix Chouan}

Димка, и вообще забей на идиотов. Если что - я тебе и на венгерский, и на
немецкий перевести могу талмуд о Борще) Читательская аудитория резко
расшириться. )))

\begin{itemize} % {
\iusr{валентин генчев}
\textbf{Jacques Felix} Chouan Я я на български, но за очень отдельную плату))).
\end{itemize} % }

\iusr{Бундурчак Бурунбучак}

1. По пункту 1 непонятно, как составлен договор. Нужен практикующий юрист.

2. Стоимость допечатной подготовки зависит от оформления, цветности печати,
количества полос, иллюстраций, качества бумаги, тиража. Всё очень конкретно для
каждого случая. Каких-то единых формул расчёта нет.

Книга может быть как лысым текстом на газетке, так и шедевром на хорошей
меловке.

\begin{itemize} % {
\iusr{Сергей Дода}
\textbf{Бундурчак Бурунбучак} ага, на 350-ке )
\end{itemize} % }

\iusr{Ярослав Литвиненко}
Надо договор почитать и тогда можно понять можно ли что то сделать. Можете мне скан скинуть почитаю, скажу.

\iusr{Сергей Левитин}
суки

\iusr{Сергей Сабиров}

Внимательно и и интересом прочёл замечания в письме от издательства.
Изобретателей «реальности минувшего» очень уж много в наши дни - и русских, и
украинских и таджикских и тд.)

\iusr{Игорь Пиляев}

А если назвать сей опус "Борщ, як національний орієнтир модернізації від
ракетно-космічної до великої аграрної держави Трипілля". Может, издательство
передумает? Но если серьезно, то договор, в котором не предусмотрена
компенсация автору добросовестных трудозатрат и содержится возможность полного
"кидка" одной из сторон - это не договор. Конечно, суд может присудить
определенные выплаты автору. Но расчет, видимо, на то, что ввиду незначительной
суммы упущенного гонорара пострадавшая сторона сочтет нецелесообразным
обращение в суд, чреватое затратами на адвоката и другими судебными издержками.


\iusr{Адвокат Артём Бугрим}
Заказ есть, произведение - тоже.... Нехорошо, Folio, нехорошо.

\iusr{Сергей Борисыч Иванов}
Красовицкий превратил свое "Фолио" в рассадник нацистских идей. Теперь вот еще одно подтверждение. Убогие хуторяне с комсомольской закалкой!

\iusr{Михаил Погребинский}

Мою книгу "Политический советник, изданную "Фолио" Красовицкий изъял из продажи
в 2014 через пару месяцев после выпуска. А я попал в список МИнкульта
запрещённых авторов только через год.

\ifcmt
  ig https://scontent-lga3-1.xx.fbcdn.net/v/t1.6435-9/217367216_1428293870868316_1487629040534481063_n.jpg?_nc_cat=102&ccb=1-5&_nc_sid=dbeb18&_nc_ohc=NCytuzGmyQEAX9__PUM&_nc_ht=scontent-lga3-1.xx&oh=7b4fdfcfb37f1f110abe6a74fc5cfe4e&oe=6171BE54
  @width 0.3
\fi

\begin{itemize} % {
\iusr{Dmitry Gubin}
А было и такое в том же издательстве

\ifcmt
  ig https://scontent-lga3-1.xx.fbcdn.net/v/t1.6435-9/217664831_4356928577678923_1766466901647717685_n.jpg?_nc_cat=101&ccb=1-5&_nc_sid=dbeb18&_nc_ohc=VClm-jVb6aYAX_QaKhk&_nc_ht=scontent-lga3-1.xx&oh=1a97bef5f9fd77d3b504100c79e9323e&oe=61721022
  @width 0.3
\fi

\iusr{Dmitry Gubin}
И даже такое там же.

\ifcmt
  ig https://scontent-lga3-1.xx.fbcdn.net/v/t1.6435-9/217395490_4356932184345229_2827089101995978653_n.jpg?_nc_cat=102&ccb=1-5&_nc_sid=dbeb18&_nc_ohc=gpIt8n3UXHcAX8fXcqV&_nc_ht=scontent-lga3-1.xx&oh=63e3584ff11d1343617e12a6694b19ae&oe=6173A806
  @width 0.3
\fi

\iusr{Адвокат Артём Бугрим}
\textbf{Михаил Погребинский} думать - это опасно)

\end{itemize} % }

\iusr{Конюшня Старая-Пристань}
Дима, ели статью не пропустили, то может выложи её на странице! Интересно почитать.

\iusr{Станислав Хохель}

Основания подать в суд есть, хотя и не гарантируют выплаты. Однако они
увеличивают вес в возможном полюбовном решении, если таковое не принято до сих
пор.

\iusr{Денис Казанский}

Не всякий графоман, написавший хуйню, является запрещённым писателем.

\begin{itemize} % {
\iusr{Дмитрий Заборин}
\textbf{Денис Казанский} зато лишь истинный долбоёб может объявить книгу хуйней не читая )

\iusr{Slava Em}
\textbf{Дмитрий Заборин} это же известная мразь, чего от него ожидать.

\iusr{Александр Давидович}
\textbf{Slava Em} интересно, чего ж ник не сменил, на хэрсонский

\iusr{Igor Kogan}
\textbf{Денис Казанский} Забавная самокритика)
\end{itemize} % }

\iusr{Андрей Крамаренко}
страна победившего идиотизма. похоже, без принуждения к разуму маразм будет крепчать

\iusr{Алексей Крымский}
Дима, а что ты ожидал? Не этого?

\iusr{Александр Алексеевич}

\obeycr
Пропал калабуховский дом!
А ведь приличное издательство было.
Ну, пусть пан выезжає за рахунок саг про бандеру и од и панегириков про шухевича.
Хотя его доляка в ЭКСМО (Москва) не мешает ему оставаться злобным русофобом.
Бабло не воняє, воно приемно пахне.
\restorecr

\end{itemize} % }
