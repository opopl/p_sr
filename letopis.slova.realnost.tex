% vim: keymap=russian-jcukenwin
%%beginhead 
 
%%file slova.realnost
%%parent slova
 
%%url 
 
%%author 
%%author_id 
%%author_url 
 
%%tags 
%%title 
 
%%endhead 
\chapter{Реальность}

%%%cit
%%%cit_head
%%%cit_pic
%%%cit_text
І з часом починаєш помічати її вплив на те, що відбувається в твоїй
\emph{реальності}.  Не їсти після 18-ої, уступати місце старшим, не матюкатись,
кожного дня робити зарядку, не їсти мучного, не давати хабарів, не перевищувати
швидкість, не обманювати, відкладати 10\% з кожного заробітку, перейти на
українську, ніколи не запізнюватись, не звинувачувати нікого, не відповідати
злом на зло тощо – будь-яке рішення поводити себе певним чином може стати
фактором, що трансформує наше життя.  Сюди відносяться всілякі Кодекси честі
всіх часів і народів, самурайство в будь-якій сфері, релігійне подвижництво,
яке давало можливість будувати та трансформувати своє життя, – як персональне,
так і соціальне
%%%cit_comment
%%%cit_title
\citTitle{Чи можна примусити збірну України розмовляти державною мовою?}, 
Євген Лапін, www.pravda.com.ua, 06.07.2021
%%%endcit


%%%cit
%%%cit_head
%%%cit_pic
\ifcmt
  pic https://gdb.rferl.org/FA660D87-01F2-47B2-8994-4D512F490B87_cx21_cy0_cw78_w650_r1_s.jpg
	caption Любов Цибульська, керівниця Центру стратегічних комунікацій та інформаційної безпеки
  width 0.4
\fi
%%%cit_text
Є в статті і невербальний меседж, говорить Цибульська, це психологічний вплив
«типового аб’юзера», садиста, що отримує задоволення від знущання зі своєї
жертви.  «Це типова психологічна маніпуляція над жертвою, коли аб’юзер каже:
«ти недолуга, в тебе без мене нічого не вийде, ти невдаха». Це сигнал
пригнічення волі. І наших успіхів, які також є, Путін не помічає, він продовжує
жити у своїй уявній \emph{реальності}, і кожного разу нам її нав’язує. І це
продумана тактика, мета якої примусити жертву постійно сумніватися у собі, в
своїх силах», – пояснює Любов Цибульська і нагадує, що психологічні операції
прописані в офіційній воєнній доктрині Росії та вважається одним з трьох
головних компонентів воєнної агресії
%%%cit_comment
%%%cit_title
\citTitle{Спецоперація під назвою «стаття Путіна»}, 
Марія Щур; Сашко Шевченко, www.radiosvoboda.org, 13.07.2021
%%%endcit
