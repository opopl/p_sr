% vim: keymap=russian-jcukenwin
%%beginhead 
 
%%file 20_03_2022.fb.turchak_svitlana.1.mariupol_on_esche_zhyv
%%parent 20_03_2022
 
%%url https://www.facebook.com/permalink.php?story_fbid=pfbid0L3BVarAvo8RBHP5dY5Nbm1jxWqx7fSWSsvaf5KNv8yccyEvqJr1P9iSGxoyT7YTXl&id=100013441451076
 
%%author_id turchak_svitlana
%%date 
 
%%tags 
%%title МАРИУПОЛЬ... ОН ЕЩЁ ЖИВ!
 
%%endhead 
 
\subsection{МАРИУПОЛЬ... ОН ЕЩЁ ЖИВ!}
\label{sec:20_03_2022.fb.turchak_svitlana.1.mariupol_on_esche_zhyv}
 
\Purl{https://www.facebook.com/permalink.php?story_fbid=pfbid0L3BVarAvo8RBHP5dY5Nbm1jxWqx7fSWSsvaf5KNv8yccyEvqJr1P9iSGxoyT7YTXl&id=100013441451076}
\ifcmt
 author_begin
   author_id turchak_svitlana
 author_end
\fi

МАРИУПОЛЬ...ОН ЕЩЁ ЖИВ!

Автор Надежда Сухорукова, г.Мариуполь 

Я выхожу на улицу в перерывах между бомбежками. Мне нужно выгулять собаку. Она
постоянно скулит, дрожит и прячется за мои ноги. Мне все время хочется спать.
Мой двор в окружении многоэтажек тихий и мертвый. Я уже  не боюсь смотреть
вокруг.

Напротив догорает подъезд сто пятого дома. Пламя сожрало пять этажей и медленно
жуёт  шестой. В комнате огонь горит аккуратно, как в камине. Черные обугленные
окна стоят  без стекол. Из них, как языки,  вываливаются обглоданные пламенем
занавески. Я смотрю на это спокойно и обречённо.

Я уверена, что скоро умру. Это вопрос нескольких дней.  В этом  городе все
постоянно ждут  смерти. Мне только хочется, чтобы она была не очень страшной.
Три дня назад к нам приходил друг моего старшего племянника и рассказывал, что
было прямое попадание в пожарную часть. Погибли ребята спасатели. Одной женщине
оторвало руку, ногу и голову. Я мечтаю, чтобы мои части тела остались на месте,
даже после взрыва авиабомбы.  

Не знаю почему, но мне это кажется важным. Хотя, с другой стороны, хоронить во
время боевых действий все равно не будут. Так нам ответили полицейские, когда
мы поймали их на улице и спросили, что делать с мертвой бабушкой нашего
знакомого. Они посоветовали положить ее на балкон. Интересно на скольких
балконах лежат мертвые тела? 

Наш дом на проспекте Мира единственный без прямых попаданий. Его дважды по
касательной задело снарядами, в некоторых квартирах вылетели стекла, но он
почти не пострадал и по сравнению с остальными домами выглядит счастливчиком. 

Весь двор покрыт несколькими слоями пепла, стекла, пластика и металлических
осколков. Я стараюсь не смотреть на железную  дуру,  прилетевшую  на детскую
площадку. Думаю, это ракета, а может  мина.  Мне все равно,  просто неприятно.
В окне третьего этажа вижу чьё -то лицо и меня передёргивает. Оказывается, я
боюсь живых людей. 

Моя собака начинает выть и я понимаю, что сейчас  снова будут  стрелять. Я стою
днём на улице, а вокруг кладбищенская тишина. Нет ни машин, ни голосов, ни
детей, ни бабушек на лавочках. Умер даже ветер. Несколько человек здесь все же
есть. Они лежат сбоку дома и на стоянке, накрытые верхней одеждой. Я не хочу на
них смотреть. Боюсь, что увижу кого-то из знакомых. 

Вся жизнь в моем городе сейчас тлеет в подвалах. Она похожа на свечку в нашем
отсеке. Погасить ее - нечего делать. Любая вибрация или ветерок и наступит
тьма. Я пытаюсь  заплакать, но у меня не получается.  Мне жаль себя, моих
родных, моего мужа, соседей, друзей. Я возвращаюсь в подвал и слушаю там
мерзкий железный скрежет. Прошло две недели, а я уже не верю, что когда-то была
другая жизнь. 

В Мариуполе в подвале продолжают сидеть люди. С каждым днём им все тяжелее
выживать. У них нет воды, еды, света, они даже не могут выйти на улицу из-за
постоянных обстрелов. Мариупольцы должны жить. Помогите им. Расскажите об этом.

Пусть все знают, что мирных людей продолжают  убивать - Надежда Сухорукова,
Мариуполь
