% vim: keymap=russian-jcukenwin
%%beginhead 
 
%%file 21_05_2021.fb.buzhanskii_max.1.volnosti_nacia
%%parent 21_05_2021
 
%%url https://www.facebook.com/permalink.php?story_fbid=1949860681845068&id=100004634650264
 
%%author 
%%author_id buzhanskii_max
%%author_url 
 
%%tags nacia,nacidea,ukraina,vyshyvanka
%%title Код нашей нации, никакая не вышиванка, а "вольности"
 
%%endhead 
 
\subsection{Код нашей нации, никакая не вышиванка, а \enquote{вольности}}
\label{sec:21_05_2021.fb.buzhanskii_max.1.volnosti_nacia}
\Purl{https://www.facebook.com/permalink.php?story_fbid=1949860681845068&id=100004634650264}
\ifcmt
 author_begin
   author_id buzhanskii_max
 author_end
\fi

Теперь, когда все уже красиво сфотографировались и не менее красиво высказались, можно сказать пару слов неприятной правды.

Не нужно, но можно.

Код нашей нации, никакая не вышиванка, а \enquote{вольности}.

Полное и солидарное от Запорожья до Львова презрение к закону и порядку.

А вышиванка, безусловно красивая и исторически достоверная рубаха, призванная это презрение к законности хоть чего-либо как то скрыть.

От нас самих.

Согласитесь, гордиться вышиванкой намного приятней, чем гордиться вечным
хаосом.

Raisa Prisetska

ГОрдитися вишитою сорочкою не варто. Факт. Бо вона є або її немає. Як власне
коріння.Як мамина любов в тій заполочі. Як память поколінь. "роль" вишиванки не
в "сокритіїї" законності. Якщо це незрозуміло, це не проблема історичної
вишитої сорочки. І Вишиванка не відповідає за тих манкуртів, які використовують
її саме з такою метою!

Олександр Щербаков

код нации-это ее историческая миссия.Украинцы еще не осознали,зачем существует
Украина и какова ее миссия.

Виталий Триль

Если законы, написанные и принятые людьми , получившими такое право в
результате предвыборного обмана граждан , не способствуют развитию большинства
а являются инструментом сохранения власти кучки проходимцев то вполне
естественно что такие законы граждане будут пытаться игнорировать . Имея
примером пренебрежение законами со стороны представителей власти . В Украине
люди презирают не закон а представителей закона .

Сергей Удовик

Вольности у нас обусловлены пограничьем, что буквально и обозначает слово
украина, и психология пограничья это бегство от государства. Со свободой это
мало связано, поскольку свобода подразумевает ответственность, и ей надо
учиться

Александр Диденко

Ввшиванка, однозначно, красивая! Мы, украинцы, очень наивные и романтичны! Но,
вчера наблюдал, как сгоняли студентов на марш вышиванок в Харькове, а
отметившись, они тут же ныряли в ближайшую станцию метро. Ничего не поменялось
в методах!

Александр Шибанов

Просторная рубаха с орнаментом- древняя одежда практически ВСЕХ
народов.Уникальности не вижу.Уникальность вижу в
Амосове,Королеве,Бубке,Патоне!А вольности?Ну так мы же везде кричим о козацьком
роде.А это кто- несогласные с властью да беглые каторжники!

Михаил Журавлев

А взяли и запретили русский язык

Тамара Логинова

Когда больше нечем гордиться..

Андрей Васильович

Вышиванку уже вознесли и обесценили!!! Это уже как паспорт без печати,нет
вышиванки и ты уже не украинец!!! А то что это должно идти из души,так то
такэ!!! Началось уже у кого дороже,так как у кого что длинней,,,,, А взять и
похвастаться,что сам,сама вышила,так это тоже такэ!!! Печалька(

Виктор Михайлов

В точку, полностью согласен, но хочу добавить, что постоять друг за дружку в
беде тоже никогда не было желания...

Eugene Demenok

Все буде вишиванка!:)

Eugene Demenok

Если бы кто-нибудь сказал, например, французам, что генетический код их нации -
какая-то вышитая рубашка, интересно было бы посмотреть на реакцию.

Леся Куруч

Евгений Деменок Они бы просто не поняли. Бред необъясним.

Владимир Глядченко

Який ще Марс, нам - упоротим нацистам важніше вишиванка, борщ і щоб кацапам
погано жилося. Це ж національна ідея - краще ми в лайні жити будемо, тільки щоб
сусідам насолити

Наталия Литвиненко

Сегодня это стало фетишем, выставляется напоказ, словно для того, чтобы
клеймить друг друга нам, а это неправильно и губит наши души.  Мы сами, только
сами, своими руками себя уничтожаем, своей ненавистью и злобой, ищем врагов не
там, не уважаем друг друга- в этом и есть главная печаль.  Не замечаем, что
позволяем злу поселиться в наших сердцах и скрыться под масками лицемерия и
фальши.  Света и Добра всем нам.

Ольга Подщанская

Хорошо, если бы это была вышиванка, а не русская косоворотка, в очередной раз
подтвердившая, что Зе - сам неуч и все его окружение такое.

Наталия Литвиненко

Время придёт и для вышиванок...
Сегодня это подиум показухи и лицемерия, неуважения своего Народа.
Когда нечем гордиться в своей стране, нет достижений и мира нет, то верх недопустимости этот подиум показухи и лицемерия, устроенный властью для возвеличивания себя любимых.
Довести страну до ручки и устроить пир во время чумы, печально.

Оставили для украинцев лишь атрибуты государства- флаг, гимн, народные костюмы,
а саму страну и государство украли, теперь ещё и Земли с недрами отобрать
хотят.

Но почему МЫ это допустили?!

Нашего шестого президента сделали своей "шестёркой" все, кому не лень...

И только НАМ лень и безразлично ВСЁ, что происходит в НАШЕЙ стране.

ДА что же НАС разбудит и пробудит в НАС Силу Духа?!

Люди в масках, это не напрасно

Безобразен образ жизни каждого из нас.

\ifcmt
  pic https://scontent-iev1-1.xx.fbcdn.net/v/t1.6435-9/189010773_858105494791949_8675065357192167557_n.jpg?_nc_cat=103&ccb=1-3&_nc_sid=dbeb18&_nc_ohc=t33yyaj98k8AX_4CnUC&_nc_ht=scontent-iev1-1.xx&oh=4db6a226e4036371d5789a33325ed0b0&oe=60CD7B2F
\fi

Alexander Shramko

Со сторны смотрится глуповато, если честно. Если б на уровне полушутливого
флешмоба, это одно. Но когда становится на такой уровень... "день вышиванки",
президент гарцует нарочито в чем-то, хотя и говорят, что это и не
вышиванка...Просто смешно. Если б еще было в стране так хорошо, что просто
больше заняться нечем...

Katya Perova

Для меня, генетический код любой нации это духовные, культурные, учённые лидеры
нации, её лучшие представители. И пока нация способна генерировать таких людей,
она и будет жить. Всё остальное вторично.

Илья Хорев

Исходя из ряда исследований в социопсихологии и социоэкономике, в любой
популяции (подчёркиваю, в ЛЮБОЙ человеческой популяции), люди делятся на три
группы, в отношении соблюдения/нарушения закона, при том всегда в одинаковой (с
изменениями, близкими, или в пределах статистической погрешности) пропорции.

\begin{itemize}
\item - 45\% населения всегда живёт по закону. Соблбдает любые законы, будь они
				хорошо продуманы и рациональны, либо же явно идут вопреки логике и
								здравому смыслу;

\item - 10\% имеют неприодолимую тягу к девеантному маргинальному поведению, и
				всегда будут пытаться обойти любые законы и правила;

\item - и теперь самая интересная группа: 45\% населения колеблются между
				первой и второй группами, в зависимости от ряда объективных, и
								вытекающих из них субъективный факторов. От возможности
								соблюдать законы и правила; от того, видят ли они в этом смысл;
								от того, для всех ли закон един для всех, или присутствуют
								двойные стандарты, и т.д.
\end{itemize}

Теория неправильных людей не работает, хотя явно напрашивается для объяснения ряда проблем

Сергій Тугай

Ну там де в українців вольності - там у декого бунт, причому обов'язково
"безсмислєний і бєзпощадний"!

Marina Goloverda

у нас тут какая то аномальная территория ,когда сами по себе то никакого ладу
нет,недаром Рюриков позвали ,с которых по сути и Русь началась,после монголо
татар ,кого тут только не было ,там где началась восстанавливаться Русь, от
Рюрика Ивана третьего до второй Екатерины сменилось двадцать царей и это
включая смутное время и всех лже Дмитриев, тут же я сбилась со счёта после
стапятидесяти гетьманов,да и не было тут государства ,так гуляй поле .

Владимир Платоненко

А почему это - хаос? Это именно то, что отличает Украину от Московии. И не
случайно Махно появился именно на Украине. В данном случае именно на, а не в.
Ибо не в украинском государстве, а на уераинской земле.  Кстати, в РФ посмотри,
кто идёт против власти, у бол шей части отыщутся корни с Украины. Даже у тех,
кто считает, что Крым - не бутерброд.

Mila Bondar

Почему то сразу при переезде через границу эти "вольности" заканчиваются.
Переходят дорогу только в установленных местах. Там полиция не боится
активистов - Пан, пашпорт. И, штраф. )

Lina Kovtunovic

Национальная одежда есть у ВСЕХ народов. Украинцы носятся со своей вышиванкой ,
как с писаной торбой, будто это что-то особенное. Гордиться ведь нечем, поэтому
придумали гордиться вышиванкой. У вышиванки есть свое место в истории народа:
украинская национальная одежда, ее элемент, и не более!

Татьяна Шульга

Внук спросил у деда:"дед,а куда ты ложишь свою бороду,когда засыпаешь на
одеяло,или под одеяло...???"((борода у деда была знатная)),дед хотел ответить
внуку,задумался и промолчал,а когда лёг спать спросил сам себя,"а куда на самом
деле я ложил бороду???",положил под одеяло,не удобно,положил на одеяло не
удобно и так до утра ложил бороду туда,сюда и с тех пор он больше никогда не
спал спокойно из-за бороды,с которой он прожил много лет не задумываясь,где она
лежит и много лет крепко спал и отлично себя чувствовал,но вопрос внука изменил
его жизнь,дед стал плохо спать,всю ночь ворочался и заболел...Так и наше
общество,жили не тужили,пока перед всеми не встали вопросы,которые раньше даже
не возникали в головах большинства людей,многие казались обыденными мелочами,но
нет...!!!Из-за сплошной патологической и циничной лжи одних,желающих иметь в
этой жизни все,все,все...до неприличия и ничего за это не иметь,другие стали
искать ответы,а как было раньше???Когда ложь кругом,а в Украине за тридцать лет
ложь отточена до блеска и большинство одели на себя кто маски,кто вышиванки,кто
спорткостюмы,кто очень дорогие,фирменные одежды и аксесуары,став играть роли
какие угодно в корыстных для себя целях с далеко идущими планами наживы да
так,что кругом и в общем,все не так как у людей,а в частности,с кем не
поговори,или кого не послушай все Ангелы!!!

Игорь Бабенко

Решил тоже прикупить вышиванку - присмотрел на рынке размер и расцветку. Присмотрелся к этикетке. Сделано в Китае. На этом все желание повышивать пропало.
