% vim: keymap=russian-jcukenwin
%%beginhead 
 
%%file 29_08_2021.fb.mischishin_mihail.1.elita_ukr_gosudarstvo.cmt
%%parent 29_08_2021.fb.mischishin_mihail.1.elita_ukr_gosudarstvo
 
%%url 
 
%%author_id 
%%date 
 
%%tags 
%%title 
 
%%endhead 
\subsubsection{Коментарі}

\begin{itemize}
%%%fbauth
%%%fbauth_name
\iusr{Andrei Raschuk}
%%%fbauth_url
%%%fbauth_place
%%%fbauth_id
%%%fbauth_front
%%%fbauth_desc
%%%fbauth_www
%%%fbauth_pic
%%%fbauth_pic portrait
%%%fbauth_pic background
%%%fbauth_pic other
%%%fbauth_tags
%%%fbauth_pubs
%%%endfbauth
 
Так будет всегда, если строить национальное государство, а не демократическое

%%%fbauth
%%%fbauth_name
\iusr{Андрей Лукашов}
%%%fbauth_url
%%%fbauth_place
%%%fbauth_id
%%%fbauth_front
%%%fbauth_desc
%%%fbauth_www
%%%fbauth_pic
%%%fbauth_pic portrait
%%%fbauth_pic background
%%%fbauth_pic other
%%%fbauth_tags
%%%fbauth_pubs
%%%endfbauth
 

Напоминаю, в декабре 1991 года люди проголосовали за Независимость Украины на
условиях «Декларации о государственном суверенитете Украины», в которой
ОТСУТСТВУЕТ должность президента Украины.

\begin{itemize}
%%%fbauth
%%%fbauth_name
\iusr{Михаил Мищишин}
%%%fbauth_url
%%%fbauth_place
%%%fbauth_id
%%%fbauth_front
%%%fbauth_desc
%%%fbauth_www
%%%fbauth_pic
%%%fbauth_pic portrait
%%%fbauth_pic background
%%%fbauth_pic other
%%%fbauth_tags
%%%fbauth_pubs
%%%endfbauth
 
\textbf{Андрей Лукашов} Да, наверное, Вы правы. Одного премьера нам хватило бы.
Надежды на то, что в Украине появится президент-реформатор, и ему нужна будет
независимость, уже практически нет. Правда, период сравнительного неплохого
правления Кучмы связан именно с сильным конституционно президентом. Дальше его
полномочия начали урезать в пользу Рады и правительства. А, как только Янукович
их восстановил, то его сразу же и не стало на посту президента. После чего
восстановленный статус сильного президента революционеры почему-то тут же и
отменили.
\end{itemize}

%%%fbauth
%%%fbauth_name
\iusr{Feran Tarek}
%%%fbauth_url
%%%fbauth_place
%%%fbauth_id
%%%fbauth_front
%%%fbauth_desc
%%%fbauth_www
%%%fbauth_pic
%%%fbauth_pic portrait
%%%fbauth_pic background
%%%fbauth_pic other
%%%fbauth_tags
%%%fbauth_pubs
%%%endfbauth
 

При Ельцине в 90-ых Запад тоже активно поддерживал Россию, пока Ходорковские,
Гусинские, Березовские и иже с ними лихо присваивали народное добро и
передавали управление активами западным партнерам.

Как только Россия в 2000-ых решила пойти своим путем, а не скакать на задних
лапах перед Западом, мгновенно стала изгоем.

Появился список Магнитского, секторальные санкции, а Крым просто стал очередным
поводом для дальнейшего противостояния.

Видимо Украине нужно определиться, либо становиться вассалом Запада и послушно
поставлять сырье и трудовые ресурсы, либо прыгать под крыло России.
Самостоятельно устоять Украине практически невозможно.

И нужно еще одно понять-перспектива у Украины не как у Польши, а как у
Афганистана.

\begin{itemize}
%%%fbauth
%%%fbauth_name
\iusr{Viktor Yeryomin-Khokhljuk}
%%%fbauth_url
%%%fbauth_place
%%%fbauth_id
%%%fbauth_front
%%%fbauth_desc
%%%fbauth_www
%%%fbauth_pic
%%%fbauth_pic portrait
%%%fbauth_pic background
%%%fbauth_pic other
%%%fbauth_tags
%%%fbauth_pubs
%%%endfbauth
 
\textbf{Feran Tarek}
У них всё раздеребанено, как в Украине. Только "семей" побольше, а коррупция ещё похлеще.

%%%fbauth
%%%fbauth_name
\iusr{Михаил Мищишин}
%%%fbauth_url
%%%fbauth_place
%%%fbauth_id
%%%fbauth_front
%%%fbauth_desc
%%%fbauth_www
%%%fbauth_pic
%%%fbauth_pic portrait
%%%fbauth_pic background
%%%fbauth_pic other
%%%fbauth_tags
%%%fbauth_pubs
%%%endfbauth
 
\textbf{Feran Tarek}

В целом согласен, кроме - прыгать под крыло России.
Теоретически понятно. Практически сейчас это трудно реализуемо. Мне кажется,
для Украины был бы сейчас очень неплох - более спокойный и взвешенный путь
Финляндии, Норвегии, Вьетнама и т.д., не сильно влазящих в разборки России и
Запада, а теперь - и Китая. Да, Россия - рядом. И это тот фактор, который в
Украине нельзя не учитывать. То, что наши революционеры позволили сделать из
себя вторую Грузию и Прибалтику - так мы получим размеры Прибалтики и
грузинскую войну у себя в итоге. Но все почему-то надеются, что будет так, как
было раньше, когда Прибалтика росла на российском транзите и выведении средств
из России. А из Грузии при Саакашвили пытались лепить витрину западных реформ
любыми средствами и, забывая подтирать внутри. Россия серьезным образом
окрепла. А мы верим всем этим байкам западным. Как дети, честное слово.
\end{itemize}

%%%fbauth
%%%fbauth_name
\iusr{Алла Круглова}
%%%fbauth_url
%%%fbauth_place
%%%fbauth_id
%%%fbauth_front
%%%fbauth_desc
%%%fbauth_www
%%%fbauth_pic
%%%fbauth_pic portrait
%%%fbauth_pic background
%%%fbauth_pic other
%%%fbauth_tags
%%%fbauth_pubs
%%%endfbauth
 
Не хотят, не собираются это делать и не будут, к огромному нашему сожалению. 😞

\begin{itemize}
%%%fbauth
%%%fbauth_name
\iusr{Михаил Мищишин}
%%%fbauth_url
%%%fbauth_place
%%%fbauth_id
%%%fbauth_front
%%%fbauth_desc
%%%fbauth_www
%%%fbauth_pic
%%%fbauth_pic portrait
%%%fbauth_pic background
%%%fbauth_pic other
%%%fbauth_tags
%%%fbauth_pubs
%%%endfbauth
 
\textbf{Алла Круглова} Да, Аллочка. Пришел к такому же выводу.
\end{itemize}

% -------------------------------------
\ii{fbauth.grigorjev_ivan.moskva}
% -------------------------------------

Я может обижу чем-то украинцев, но они никогда не строили и не хотели строить
государство. Более 300 лет им обьясняли из Москвы как строить.теперь уже 30 лет
обьясняют из Вашингтона. И никакие партаппаратчик Кравчука никого в космос не
отправляли, они выполняли команды из центра. Украина не будет нормальным
государством, пока не развалится и не начнет сама собираться. Всё что было у
Украины было сделано не ею, да при её участии, но без её воли. Промышленность
уже развалили, медицину тоже, теперь теряются территории. Вот когда останется
что-то маленькое, но то, что будет представлять ценность для своих граждан, и
если граждане смогут этой ценностью увлечь других, тогда и сможет возникнуть
украинское государство.

% -------------------------------------
\ii{fbauth.shabrina_svetlana.kiev.ukraina}
% -------------------------------------
 
Никак не пойму, зачем и почему называют их "элитой"? Что в них элитарного )))

% -------------------------------------
\ii{fbauth.rozhankovskij_vladimir.moskva.rossia.ekspert.finansy}
% -------------------------------------

Думаю, что потому, что политика является производной от экономики, а не наоборот

% -------------------------------------
\ii{fbauth.abroskin_vladimir.rossia}
% -------------------------------------

Все, казалось бы, есть для построения успешного, процветающего государства -
огромная территория, богатая земля, прекрасный климат, трудолюбивый,
талантливый, патриотичный народ, мощное наследство империи и т.д. и т.п. А
государства не получается! В чем же дело? На мой взгляд, одна из существенных
причин - отказ от общей с Россией истории. 

Ведь история - мощнейший неисчерпаемый государствообразующий ресурс, духовный
стержень, на который нанизываются все государственные институты. 

История - питательная среда для формирования элиты, осознающей свою
ответственность за судьбу страны! 

Украина сама порвала в клочья, изрезала на куски все свое славное прошлое. И
первый шаг - отказ от русского имени! Ну нельзя же всерьез относиться к
двуглавым эвфемизмам-уродцам в виде Украины-Руси или Руси-Украины! 

Смешны стенания украинцев о том, что Россия украла их русское имя. С такой
радостью и готовностью от дорогого не отказываются! А ведь можно было строить
независимое государство, не отрекаясь от общей с Россией историей! 

Вам не нравится российская азиатчина? Вы - еуропейцы! Прекрасно, стройте чисто
европейский вариант России, свободный от влияния Кавказа, Средней Азии, Сибири!
Ведь в вашей общей с Россией историей был и такой вектор! Используйте его! Вас
воротит от российской имперскости, от вовлеченности России в дела Сирии,
Венесуэлы, Африки. Да, это крест России, который она обязана нести, если хочет
оставаться Россией. Вам это претит? Пожалуйста! Сбросьте эту ношу, на вас нет
бремени имперских сверхрасходов. Был и такой период в богатейшей русской
истории.

Стройте альтернативную Русь по образцу Швейцарии! Что мешает-то? Да то, что
смыслом вашего государства является АнтиРоссия, а не иная, лучшая на ваш взгляд
Россия!

\index{АнтиРоссия}

\begin{itemize}
%%%fbauth
%%%fbauth_name
\iusr{Михаил Мищишин}
%%%fbauth_url
%%%fbauth_place
%%%fbauth_id
%%%fbauth_front
%%%fbauth_desc
%%%fbauth_www
%%%fbauth_pic
%%%fbauth_pic portrait
%%%fbauth_pic background
%%%fbauth_pic other
%%%fbauth_tags
%%%fbauth_pubs
%%%endfbauth
 
\textbf{Vladimir Abroskin} 

Да. Но надо понимать, что сейчас мы в значительной степени ведомы. Украинцы,
как ребенок в переходном возрасте, сам себе выдает чужие, вздорные и часто
подлые мысли за свои. Но они нам чужие. Со временем большая часть из нас это
поймет. Меньшая и не желающая меняться, предпочитающая жить ненавистью, -
оторвётся от Украины и уйдет в свое самостоятельное плавание.
\end{itemize}

%%%fbauth
%%%fbauth_name
\iusr{Андрей Клочко}
%%%fbauth_url
\urlFriend{https://www.facebook.com/vyborkherson}\par
%%%fbauth_place genichesk,herson_obl,ukraina
%%%fbauth_id
%%%fbauth_front
%%%fbauth_desc
Изучал Экономическая кибернетика в Херсонский национальный технический университет
Живет в г. Геническ
Из г. Геническ
%%%fbauth_www
%%%fbauth_pic
%%%fbauth_pic portrait
\ifcmt
  ig https://scontent-frx5-1.xx.fbcdn.net/v/t1.6435-9/118319663_2813809978849704_8775544108430920947_n.jpg?_nc_cat=100&ccb=1-5&_nc_sid=09cbfe&_nc_ohc=5MuFLztElpMAX_8RZCN&_nc_ht=scontent-frx5-1.xx&oh=8660c572f98079eac5b3e0cdbb42f9c1&oe=615C5D93
  width 0.15
\fi
%%%fbauth_pic background
\ifcmt
  ig https://scontent-frt3-1.xx.fbcdn.net/v/t1.6435-9/61536585_2441622749401764_1972410142381047808_n.jpg?_nc_cat=104&ccb=1-5&_nc_sid=e3f864&_nc_ohc=58J2B6yNf28AX-dMkCA&_nc_ht=scontent-frt3-1.xx&oh=8aa68d7cb6decef9a6907af3a9fdfeac&oe=615C32CD
  width 0.15
\fi
%%%fbauth_pic other
\ifcmt
  ig https://scontent-frx5-1.xx.fbcdn.net/v/t1.6435-9/204457592_3052423894988310_803137943173968556_n.jpg?_nc_cat=105&ccb=1-5&_nc_sid=174925&_nc_ohc=YHREWfEnOdsAX8x7ue-&tn=lCYVFeHcTIAFcAzi&_nc_ht=scontent-frx5-1.xx&oh=969dd02956b6eec33a0407f15f18df14&oe=6159E529
  width 0.15

	ig https://scontent-frt3-1.xx.fbcdn.net/v/t1.6435-9/84745881_2640071776223526_3761587041360412672_n.jpg?_nc_cat=102&ccb=1-5&_nc_sid=174925&_nc_ohc=PprS4sOK5bAAX-c4RGs&_nc_ht=scontent-frt3-1.xx&oh=7d2be7f8962022950996d4260296ad77&oe=61591662
  width 0.15
\fi
%%%fbauth_tags
%%%fbauth_pubs
%%%endfbauth
 

В Украине, тех кто хотел создать достойное государство, почему-то системно
убивают! Вывод. Для решения задачи описанной в публикации, необходим сильный
силовой блок иначе убьют. Поэтому без учёта исторического опыта не реально
прийти к успеху, а историческая ретроспектива всегда вращается вместе с
Россией!
\end{itemize}

