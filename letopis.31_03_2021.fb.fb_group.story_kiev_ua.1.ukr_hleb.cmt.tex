% vim: keymap=russian-jcukenwin
%%beginhead 
 
%%file 31_03_2021.fb.fb_group.story_kiev_ua.1.ukr_hleb.cmt
%%parent 31_03_2021.fb.fb_group.story_kiev_ua.1.ukr_hleb
 
%%url 
 
%%author_id 
%%date 
 
%%tags 
%%title 
 
%%endhead 
\subsubsection{Коментарі}

\begin{itemize} % {
\iusr{Полина Полякова}
Відразу відчула смак \enquote{Українського} хліба. Аж, слина потекла. Обожнюю його.

\iusr{Наталия Давиденко}

Хлебушек вкусный, жаль, что у нас в стране так мало маленьких частных пекарен
@igg{fbicon.exclamation.mark.double}

\iusr{Ігор Крижановський}
ПІДУ... КРУМОЧКУ ВІДЛОМЛЮ..... ДЯКУЮ)

\iusr{Sara Shpilman}
Прямо до слёз...

\ifcmt
  ig https://scontent-frx5-1.xx.fbcdn.net/v/t1.6435-9/167551833_283167820060903_2700705601996121034_n.jpg?_nc_cat=111&ccb=1-5&_nc_sid=dbeb18&_nc_ohc=kw_q6aor2t8AX_x1lMq&_nc_ht=scontent-frx5-1.xx&oh=00_AT9G5uqmQoS5F98vqsIgJpgpR5OmdqfeWY0JKva5Rp-Ivw&oe=61DEC75D
  @width 0.3
\fi

\iusr{Елена Сидоренко}

Умеете Вы написать так, что наяву представляешь себя ребёнком, грызущим корочку
свежего хлеба. @igg{fbicon.face.savoring.food} Завод этот работает до сих пор,
и на Подоле пахнет хлебом. @igg{fbicon.heart.beating} 

\iusr{Олег Самченко}

\ifcmt
  ig https://scontent-frx5-1.xx.fbcdn.net/v/t39.1997-6/p240x240/47294511_540271073156464_2498170897404264448_n.png?_nc_cat=107&ccb=1-5&_nc_sid=0572db&_nc_ohc=iHcKZzNY2mkAX_JWvcZ&_nc_ht=scontent-frx5-1.xx&oh=00_AT90iXMN0ohtEU2KyO7w-G6O5sdG82UqRsszON_CP3mXtw&oe=61BD0699
  @width 0.2
\fi

\iusr{Ольга Яворовская}
Дякую за смачний екскурс!
Смачн!шого не !ла !

\ifcmt
  ig https://scontent-frx5-1.xx.fbcdn.net/v/t39.1997-6/p240x240/47294511_540271073156464_2498170897404264448_n.png?_nc_cat=107&ccb=1-5&_nc_sid=0572db&_nc_ohc=iHcKZzNY2mkAX_JWvcZ&_nc_ht=scontent-frx5-1.xx&oh=00_AT90iXMN0ohtEU2KyO7w-G6O5sdG82UqRsszON_CP3mXtw&oe=61BD0699
  @width 0.2
\fi

\iusr{Lola Shpilskaya}
Я тоже всегда прошу привезти из Киева хлеб

\iusr{Любов Огородня}

Дуже смачно, Дякую Петюнчику! Але, чому на Кожум'яки за хлібчиком?
Гончарна-Воздвиженська. На Комуж'яках був продовольчий магазин @igg{fbicon.wink} . А щодо
хлібзаводу на Подолі, між двома вулицями - Щекавицькій і Турівській. Пам'ятаю,
у 1987 році, ВО \enquote{Київоблгаз}, який розташовувався на Гончарній, 3 (я там
працювала), переїхав на Турівську, якраз навпроти хлібзаводу, поруч з магазином
Галантерея. Аромат навкруги неповторний. Наше керівництво домовилося з
хлібзаводом про харчування працівників Облгазу в їдальні заводу. Ми на обід
ходили на територію заводу. Хліб там був на столах завжди свіжий, гарячий і
безкоштовно.

\iusr{Петр Кузьменко}

Бачиш, Любаня, ти пам'ятаєш ці неповторні миті нашого дитинства краще ніж я.
Старішаю... @igg{fbicon.laugh.rolling.floor} 

\iusr{Semyon Belenkiy}

А хлебозавод, который вы упоминаете, это хлебокомбинат \#2, специализировавшийся
на изготовлении украинского хлеба. Где то до 80-х, по-моему, это был хлебозавод
\#2, а затем произошло объединение с хлебозаводом \#7 ( ул. Кирилловская 65,
бывшая Фрунзе), который производил продукцию из пшеничной муки, бублики,
баранки, кондитерские изделия и в том числе Киевский, фирменную Ромашку и т. д.
и получился хлебокомбинат.

В молодости и сам работал на 7-ом хлебозаводе. Такая история, а за рассказ
большое спасибо!

\iusr{вита шаповал}

Дякую. Дуже смачно написано. Але зараз цей хліб не смачний, вони там хімічать)
Мої сільські родичі і їх сусіди завжди чекали приїзду городських) бо смачного
хліба привезуть. Тітка вийшла заміж за француза, приїхала в 1980 р. до Києва
погостювати. Коли відїждала до Франції взяла декілька буханців українського
хліба та пахучу соняшникову олію

\iusr{Раиса Карчевская}
Да пост прекрасный.

Написано так вкусно, что сразу почувствовала запах и вкус Украинского хлеба. Я
его обожаю и всегда, когда едешь за границу все очень ждут наш прекрасный
вкуснющий хлеб.

\iusr{Константин Панасенко}

Тоді це був дійсно ХЛІБ! А зараз - це порнографія, а не хліб! Допоки теплий, ще
нічого, а тільки став холодним їсти не можливо! Є з чим зрівнять. Хлібзавод
номер 6 випікав в 60х-70х такий хліб, що з магазина до дому доносив тільки
половинку і ту без країв!

\begin{itemize} % {
\iusr{Ірина Оснач}

Колись той хліб був мені смачніший за будь-яке печиво чи торт. Потім його
замінили на \enquote{Дарницький}, то вже був інший смак, на жаль. І паляницю тепер, на
жаль, не печуть, її хрумкий гребінець з"їдався одразу, за що мене постійно
сварили, бо по магазинах з раннього дитинства в сім"ї ходила я.

\iusr{Галина Компанченко}
\textbf{Константин Панасенко} Та яким би він не став- не можна так про хліб..

\iusr{Игорь Бала}
\textbf{Галина Компанченко} не можно выпекать такое \enquote{Г} как сейчас
\end{itemize} % }

\iusr{Ольга Григор'єва}
Сьогодні купила половинку  @igg{fbicon.wink}  Цікаво куштувати і нові смаки, але "Український" купляю завжди.

\iusr{George Vishnevsky}

Нажал, сьогоднішній Український відрізняється від того, що був колись. Немає
тієї хрусткої скоринки, смак схожий , але не той. І головне - інгрідієнти інші,
бо, хліб довго не сохне, а у поліетилені пліснявіє.  Теж саме з Батоном, тільки
той з другого дня починає не сохнути, а сипатися крихтками.  Взагалі, ХЛІБ в
Києві став гірше...

\iusr{Екатерина маковецкая}

самый вкусный всегда был с подольского хлебзавода, бабушка жила рядом на
Щекавицкой, аромат украинского хлеба там был всегда)))) аж слюнки текли!
вкуснее нигде не пекли, т.к. печь украинский сложно, он кислый, плохо подходит
и пропекается, для этого нужны специальные кирпичные тонельные печи, они
сохранились на Подольском хлебзаводе

\iusr{Marina Gromova}

Шматочок свіжого українського, змастити соняшниковою, не рафінованою олією і
посипати сіллю, смакота з дитинства  @igg{fbicon.wink} 

\iusr{Наталья Ершова}
\textbf{Marina Gromova} и сейчас так вкусно!!!

\iusr{Наталия Вигерина}
і паляниць вже не печуть @igg{fbicon.cry} 

\begin{itemize} % {
\iusr{George Vishnevsky}
\textbf{Наталия Вигерина}

\ifcmt
  ig https://scontent-frt3-1.xx.fbcdn.net/v/t1.6435-9/166882013_4075385425834399_7599132278821478542_n.jpg?_nc_cat=102&ccb=1-5&_nc_sid=dbeb18&_nc_ohc=b7cCot6BJB4AX-1lgon&_nc_ht=scontent-frt3-1.xx&oh=00_AT8Xh1ePAEinJJyNoasJuTZLPlkhZpu3BLYn5Xt6ftrKsg&oe=61DE5358
  @width 0.4
\fi

\iusr{Игорь Васильевич Переводов}
\textbf{George Vishnevsky} І де?

\end{itemize} % }

\iusr{Лариса Жабкіна}
Запах хлеба и хлебного магазина на Константиновской я помню до сих пор.

\iusr{Наталья Ершова}
Сейчас хлеб украинский совсем другой, тот украинский делали честно и с душой!!! А сейчас его и покупать нет желания!!!

\iusr{Вера Долженко}
\textbf{Наталья Ершова} Вот именно, я и покупать перестала. А был хлеб вкуснейший.

\iusr{Александр Полищук}

В моїй родині тільки одне слово вимовляється з неправильним наголосом - хліб
укрАїнський, а все інше украЇнське. Чомусь здається, що, якщо ти назвеш хліб
украЇнським, він не буде таким смачним.

\begin{itemize} % {
\iusr{Volodymyr Nekrasov}
\textbf{Александр Полищук} тут нема нічого неправильного. УкрАїнський може бути ще й похідним від «укрАяти», себто відрізати. А ще є містечко УкрАїнка, що на Київщині. Теж всі називають з наголосом на А. Зрештою і назва Україна є похідною від КрАй, крАяти.
\end{itemize} % }

\iusr{Ирина Шевцова}
А ещё арнаутка серая какая была ароматная!!!

\iusr{Olga Zyrina}
Я дуже любила свіжу четвертинку українського з пломбіром після школи (якщо ті
цінні 20 копійок на морозиво були).  @igg{fbicon.smile} 

\iusr{Виктория Довгая}

Дааа, везде, куда бы мы не ездили вне Киева, нас просили привезти буханки
Украинского хлеба. Мама ходила и закупала буханки. А я очень любила бублики с
маком и булочки за 3 копейки. Купишь бублик, съешь - вкуснотища..

\iusr{Маргарита Марченко}
Дуже гарний і неповторний був смак!

\iusr{George Vishnevsky}

\href{https://m.facebook.com/groups/1671741806381479/permalink/2819977861557862}{%
На тлі сучасного розмаїття хлібо-булочних виробів в Києві, я сумую за смаком %
Паляниці ..., George Vishnevsky, facebook, 31.03.2021%
}

\iusr{Людмила Краснюк}
Дійсно, тоді Український хліб був дуже смачний!  @igg{fbicon.thumb.up.yellow} А
скоринка ароматного теплого буханця - смакота!!! @igg{fbicon.heart.red}

\iusr{Irma Krotenko}
Hmmmm smakota❤

\iusr{Ludmila Tkachuk}
Ну это было очень давно, теперь такого хлеба нет.

\iusr{Vitali Andrievski}

Очень хорошо написано. И не забыт наш главный подольский хлебный магазин на
Жданова. Я специально написал старое название главной подольской улицы потому
что оно соответствует тем временам и тому хлебу

\iusr{Iryna Irena}
Мій домашній  @igg{fbicon.smile} 

\ifcmt
  ig https://scontent-frt3-1.xx.fbcdn.net/v/t1.6435-9/166993812_2879936952285440_1935764393342542595_n.jpg?_nc_cat=106&ccb=1-5&_nc_sid=dbeb18&_nc_ohc=omfmfSBFPoEAX8e1_hs&_nc_ht=scontent-frt3-1.xx&oh=00_AT8aTS7Nyczvy5k4MBhnp5u7SH5KchAir7aOQ7iXh_EJ3Q&oe=61DB47F6
  @width 0.3
\fi

\iusr{Yelena Rozengurtel}

Мой прадед Соломон Самсонов (отец доктора Ривы Соломоновны Гутман, автор поста
был знаком с ней) работал с 1930х лет в одном из этих хлебных, и даже заведовал
им потом. Мама говорит он приходил с работы в 12 ночи, ел полный ужин из трёх
блюд, и дожил до 96 лет. Добавлю фото с моим прадедом около нашего дома,
Боричев Ток 22а.  @igg{fbicon.heart.eyes} 

\ifcmt
  ig https://scontent-frt3-1.xx.fbcdn.net/v/t1.6435-9/167043444_10158925805253405_1539791809776297683_n.jpg?_nc_cat=108&ccb=1-5&_nc_sid=dbeb18&_nc_ohc=PA1TUHOItpMAX_NrAng&_nc_ht=scontent-frt3-1.xx&oh=00_AT_DIHW711LrlxdmK0NZmyTPB5K9W_dv8vdQ_M8RVMwHYQ&oe=61DDD012
  @width 0.4
\fi

\begin{itemize} % {
\iusr{Vitali Andrievski}
\textbf{Yelena Rozengurtel} помню Риву Соломоновну. Ее знали все на Подоле

\iusr{Yelena Rozengurtel}

\ifcmt
  ig https://scontent-frt3-2.xx.fbcdn.net/v/t1.6435-9/166805518_10158926038053405_568491081696187048_n.jpg?_nc_cat=103&ccb=1-5&_nc_sid=dbeb18&_nc_ohc=xRhmUhuw_ogAX8N0Iog&_nc_ht=scontent-frt3-2.xx&oh=65629cf3c196d505fc200eae778ecea9&oe=61DF1CFB
  @width 0.25
\fi

\end{itemize} % }

\iusr{Ирина Нечай}
Читаю, а возле меня лежит горбушка ...

\iusr{Hanna Po}
Приємно читається українською мовою.

\iusr{Valentina Syssoena}
\textbf{Hanna Po} и русским не плохо

\iusr{Татьяна Чернышева}

Для меня первое воспоминание о Киеве- это хлебный магазин на Майдане. 1988 год.
Этот неповторимый аромат свежевыпеченного хлеба.... Вкус Украинского, запах
батона и новая любовь- арнаутский! Второе воспоминание- это кулинария с
самоваром, пирожками и бубликами! Я сразу поняла, что уже люблю этот город, он
самый гостеприимный, вкусный, тёплый, уютный и волшебный!


\iusr{Володимир Курило}

Хліб з салом чи ковбасою, помідор та огірок, але останнім часом щось змінилось,
може мука нета, а може люди не ті.

\begin{itemize} % {
\iusr{Лидия Бражник}
\textbf{Володимир Курило} а кусок украинского хлеба, намазанного вареньем с черной смородины, и стакан молока!!! Это шедевр!

\iusr{Татьяна Чернышева}
\textbf{Лидия Бражник} дети думают, что мы извращенцы. Все эти макароны с яйцом и сахаром, томатный сок, батон с повидлом)))

\iusr{Лидия Бражник}
\textbf{Татьяна Чернышева} в 2000 годах, когда зарплата учителя была очень небольшой, ещё и не вовремя платили, мы с дочкой жили вдвоём, так в день зарплаты у нас было лакомство: молочные сосиски, тогда они были ещё вкусные, пара вареных в смятку яиц, свежий батон с маслом, тоже нормального вкуса, и томатный сок!!! Мы это вспоминаем до сих пор и иногда устраиваем себе маленький пир. Вкусно!!!!
\end{itemize} % }

\iusr{Валентин Тим}

\enquote{..Не той тепер Миргород, Хорол-річка -не та..} Не знаю, чи то борошно беруть
інше, чи я вже смак втратив?...

\begin{itemize} % {
\iusr{Вера Олецкая}
\textbf{Валентин Тим} Не тільки ви...

\iusr{вита шаповал}
\textbf{Валентин Тим} борошна там дуже мало. Розпушувачі і всяка всячина, щоб подешевше. Я торт домашній спекти не можу, бо борошно єрунда.

\iusr{Людмила Хомич}
\textbf{Валентин Тим} Поліпшувачі додають!

\iusr{Светлана Сергеева}
\textbf{Валентин Тим} зі смаком у вас усе гаразд. У всіх одночасно смак не втрачається.... хоча... при ковіді....
\end{itemize} % }

\iusr{Irina Popova}
как вкусно вы описали любимый хлебушек и Подол, спасибо Вам!

\iusr{Wiktoria Arto}

Поема про хліб! Дуже вдячна за вибір такої надзвичайної теми, як Хліб, любов
всіх киян - Український хліб. Вже два роки я в Кракові, але такого натурального
смаку, яким є наш чудовий буханець - напевно не знайшла! Дякую за чудову
розповідь! @igg{fbicon.heart.red}


\iusr{Наталья Борчанинова}

А ещё помню в начале 60-х годов ,,жулики,, - ржаные сладкие булочки с изюмом!!!

Название житних булочек,, жулики,, произносила моя бабушка, которая была
коренной киевлянкой 1899 года рождения...

\begin{itemize} % {
\iusr{Valentina Syssoena}
\textbf{Наталья Борчанинова} я обожала свежие жудики

\iusr{Алла Козовец}
\textbf{Наталья Борчанинова} да, 'жулики" это помнят многие киевляне. Правда не помню какие это года - 60-70?..
\end{itemize} % }

\iusr{Yurii Fenskyi}
Не кажить, зараз хлiб вже не такий смачний...

\iusr{Elena Kharchenko}
По-моему, самый вкусный был на по. Калинина(Майдане)

\begin{itemize} % {
\iusr{Лидия Бражник}
\textbf{Elena Kharchenko} 

тогда везде хлеб был очень вкусный! Каждый имел хлеб в том района, где жил, и
корочка украинского хлеба, или козырек паляницы, остались в памяти каждого
ребенка того времени. А ещё - жулик.... Ммммм...

\iusr{Elena Kharchenko}
\textbf{Лидия Бражник} точно!

\iusr{Volodymyr Nekrasov}
\textbf{Elena Kharchenko} Паляниці там справді були найсмачніші
\end{itemize} % }

\iusr{Julia Prima}
Смачний та запашний! Як свіжий особливо!

\iusr{Vicki Seplarsky}
Да, и в Центральном гастрономе «Хліб» на Бессарабке.

\iusr{Елена Бойко}

Так, гарно все описано, все так і було. Але на сьогодні, хліб споганився,
нажаль. Є інші більш пристойні варіанти, але дорожчі


\iusr{Oksana Shulga}

За те 4 минуты, что я шла из гастронома на Ирининской обратно домой на
Калинина, я успевала половину горячего душистого Украинского слопать.

\iusr{Александр Семенище}

Пять лет студентом разгружал по ночам в гастрономе в 80 е по запаху
подъезжающей хлебной машине знал с какого хлебзавода привезли украинкий

Буханка в зависимости от веса стоила 16, 20 или 28 коп.

Вкусный козырек на палянице для всего Киева надрезался вручную на сыром тесте
перед подачей в печь И был вкусный только свежий поэтому магазины заказывали
его мало. При 3 хлебзаводе был цех номер 8 для власть имущих. В булочки по 3
коп. в рецепте было сгущеное молоко

\begin{itemize} % {
\iusr{Наталья Борчанинова}
\textbf{Александр Семенище} В те времена, как более ранние, так чуток поздние, при всех пищевых предприятиях были ,,спецеха,, для власть-имущих... ( я работала в 1976- 89 г. на Киевском (Дарницкий) мясокомбинате технологом...) Знала многое, кое о чем до сих пор молчу...

\iusr{Нина Моцак Грищева}
\textbf{Наталья Борчанинова} а что ж молчите то? Расскажите, это же интересно.

\iusr{Наталья Борчанинова}
\textbf{Нина Моцак Грищева} 

Маленький пример- колбаса ,,Чайная,, 2 сорт (мясо и сало сортами пониже, с
чесноком, цена 1-60 руб/ кг и колбаса ,,Любительская,, - по 2-80 руб/ кг. То
для ,,них,, все было качеством лучше, а ценой дешевле. Рабочие были высокой
квалификации, новичков в бригаду не брали, заходили в цех, закрывались изнутри,
и муха не залетит.

Честно скажу, что за 13 лет я ни разу не была в том цехе, как рядовой технолог,
а туда изредка заходил главный технолог и др. спец-ты высокого ранга...

Экспедиторы были тоже ,,проверенные,, люди. Машина, одна и та же, ездила на
хлебозавод, молокозавод, вин-спирт, кондитерку, мясокомбинат и тд и так возили
в ЦК и др. ,, спецучреждения,,...

\iusr{Нина Моцак Грищева}
\textbf{Наталья Борчанинова} и о чем же тут молчать? Не поверите, это давно совсем не тайна. И никогда ею не было.
\end{itemize} % }

\iusr{Валентин Тим}

Дуже багато залежало від того, які печі на комбінатах. Розповідали, що
найкращий український хліб вдававсь майстрам -пекарям з подільського хлібзаводу
№2. Це там випікавсь справжній хліб на \enquote{черені} (т.з. подовий).


\iusr{Наталья Лапига}

Український хліб мав надзвичайний смак, а який аромат! Він був смачнішим від
тістечка! Неможливо було втриматись, щоб не з'їсти добрий шматок зі скоринкою!
Хіба можна порівняти його з сучасним, дуже низької якості!


\iusr{Александр Семенище}
На память о тех годах осталась вилка которой проверяли свежесть хлеба

\iusr{Наталья Борчанинова}
\textbf{Александр Семенище} О, точно!!! На верёвке привязана к стеллажу...

\iusr{Anastasia Zagorska Michuda}

Ага — мне его всегда не хватает, где бы я не была...

А ещё мне так не хватает булочек по 6 копеек, из которых получались лучшие
сухарики в мире... я вынимала из них колечки подсушенной мякоти, а затем уже
съедала корочку...

\begin{itemize} % {
\iusr{Anastasia Zagorska Michuda}

\ifcmt
  ig https://scontent-frt3-1.xx.fbcdn.net/v/t1.6435-9/167346447_3850728171673272_4198001718764303382_n.jpg?_nc_cat=106&ccb=1-5&_nc_sid=dbeb18&_nc_ohc=l5qBaZc_amcAX83EVt-&_nc_ht=scontent-frt3-1.xx&oh=00_AT_AgKkEGi4ZdCEiqBcLoZMvMarFNRXRcK-SygJT2eqTIQ&oe=61DB9373
  @width 0.4
\fi

\iusr{Marina Pletiukh}
\textbf{Anastasia Zagorska Michuda} рогалики

\end{itemize} % }

\iusr{Виктория Лунная}
Нє такий зараз хлібчик, який ви описуєте. Нажаль.(

\iusr{Ольга Барановская}

Зараз зовсiм iнша рiч. Украiнський став якимсь пласким, гливким, зараз у мене
вiд нього виникае печiя. А ранiше... Нiчого смачнiшого не було i, коли iхав в
iнше мiсто, завжди везли Украiнський, тому що люди просили.

\iusr{Dariia Radchenko}
\textbf{Ольга Барановская} а ще немає такої смачної скоринки, як була раніше...

\iusr{Vasilenko Yulia}

Дякую за таку теплу розповiдь. Мiй тато працював на цьому хлiбзаводi. Кожного
вечора додому приносив смачний, запашний та ще теплий буханець Украiнського
хлiба. Дуже теплi спогади. Ще пам'ятаю в дитинствi коли тато вiдводив до
садочка, який знаходився по вулицi Волошськiй, дуже смачно пахло свiжим хлiбом.
Смак дитинства.  @igg{fbicon.face.smiling.hearts} 


\iusr{Елена Корниенко}
Всегда везла из Киева наш Украинский, куда бы не ехала.

\iusr{Anatolij Palamarchuk}
Полити водичкою і поцукрити

\iusr{Олександр Буйницький}
Самий смачний печуть в Фастові.

\iusr{Оксана Скопец}
16копеек

Я помню, как в детстве, чтобы купить совсем свежий и ещё горячий украинский
хлеб мы ждали машину под магазином \enquote{Хлеб}, которая приезжала к 21:00, а то и
позже... Лето, ещё светло... Но человек 10-15, зная, что вот-вот приедет
машина, пока водитель разгружал, а дежурный грузчик загружал хлеб в магазин, нам
удавалось купить свежий хлеб. Пока стояла машина, нам продавали хлеб(\enquote{без
сдачи}) И все уходили с буханкой горячего украинского или батона, были
счастливы, и та вкусная корочка, которая не доносилась до дома на всю жизнь
останется в памяти... Один аромат и тепло того хлеба останется на
всегда-сытный, ароматный наш украинский хлеб, киевский...

Такого вкусного хлеба, к сожалению сейчас нет...


\iusr{Elena Dubovenko}
Смачно так написано! Дякую. Хочеться українського гарячого хлібця

\iusr{Ксенія Ковальська}

Явно з хлібом щось не те. І це не старече, мовляв, от колись... Може, скоротили
час дозрівання, чи як там воно називається те дічйство, тіста? А раніше, коли я
приїжджала до батьків, вони з порога запитували, чи привезла хліб. І я
привозила. І на хліб кликали сусідів, до ж не самим таку смакоту їсти.


\iusr{Vladimir Umanetc}

Немає, нажаль, тих смачних хрустких скоринок як раніше.. зараз весь хліб
фасують по пакетах, в результаті ми їмо щось незрозуміле... @igg{fbicon.face.weary} 


\iusr{Iryna Saltanova}
А я любила поедать беличанский батон по дороге от киоска домой  @igg{fbicon.wink} 

\iusr{Нелли Двойных}

Це просто ода смачному духняному Українському хлібу, я теж є його
шанувальницею. Мамина сестра Марія завжди, коли приїздила до нас у гості,
обов.язково привозила буханець за нашим проханням. Чомусь зараз він не такий
духняний...


\iusr{Елена Ходорко}
...что может быть вкуснее украинского, сверху маслом, томатной пастой и пластинками чеснока... вкуснота!

\iusr{Светлана Митюненко}
А говорить о патриотизме можно часто и долго А где наша, гордость Украинский хлеб его покупали даже иностранцы а теперь что халтура

\iusr{Polina Feldman}
Такого хлеба как мы ели в Киеве здесь нету а паляныца кирпичик и французские.
Булочки это осталось в памяте  @igg{fbicon.thumb.up.yellow}  @igg{fbicon.moon.cake}{repeat=3} 

\iusr{Сергей Черняк}
Нема його ,тай не буде поїхав за Десну....

\iusr{Александр Петренко}

Це найкращий хліб який я їв коли небудь за все життя. Земляки, може хтось
викладе повний його рецепт. Може є якісь тонкощі. Ніяк не можу відтворити смак
того київського українського хліба. З повагою не киянин з 85-го року.

\begin{itemize} % {
\iusr{Светлана Сергеева}
\textbf{Александр Петренко} Кияни - вони на все життя кияни, де б і скільки вони не були, і де б не жили...

\iusr{Aleks Faershtein}
\textbf{Александр Петренко} 

Вам никто не даст секретный рецепт \enquote{Украинского} хлеба!!! Сегодня такой
рецепт стоит очень больших денег... Такого хлеба как в Украине не было, нет и
не будет уже, увы!!! А \enquote{Паляныця} когда была свежая???!!!!
@igg{fbicon.hands.pray}  @igg{fbicon.100.percent} @igg{fbicon.hand.victory} ☮️ @igg{fbicon.hand.ok}
@igg{fbicon.face.zany} 

\end{itemize} % }

\iusr{Yefim Rozovsky}

Прочитал и слюна идёт. А ведь Я тоже любил этот хлебушек. Посылали меня в
магазин на Глубоческой а жил на Мирной. Пока до дома доходил всю корочку и
съехал. Какая же она .ыла вкусная. И в правду деликатес! Вкуснее этого больше
ничего нет.

А после тренировки на стадионе Динамо, идёшь домой и по дороге зайдёшь в
магазин воды, соки на Крещатике возьмёш воды с сиропом и булочку под названием
Жулик. И больше целый вечер сыт.

\iusr{Yefim Rozovsky}
*.

\iusr{Olena Radzijewska}

Так, правда - найсмачніший хліб - наш київський український. Був таким у моєму
дитинстві у 70-80х роках і залишився на все життя. А шкірочку по дорозі додому
обгризти - це святе.


\iusr{Natalia Ciesielska}
Як не крути, а найсмачніший хліб в Україні. Ще дуже люблю «заварний» або
«Бородінський»

\iusr{Галина Компанченко}
Було колись !! Не той тепер..

\begin{itemize} % {
\iusr{Наталья Борчанинова}
\textbf{Галина Компанченко} продовжую вас далі (як писав поет) ,,...Миргород, Хорол річка не та...,, (І вся їжа не та((()...

\iusr{Галина Компанченко}
\textbf{Наталья Борчанинова} До речі, я колись була на тій річці - залишилася тільки назва. А зараз може, і зовсім її не стало

\iusr{Aleks Faershtein}
\textbf{Наталья Борчанинова} Це написав Павло Тичина!!!
"...літом я робила в полі, а як втілила тіль-тіль, я товаришок питала, поступала у артіль..."
Я боюсь за правильность оригинала, ведь прошло почти 60 лет с того, как я учил это произведение.....
\end{itemize} % }

\iusr{Светлана Лещенко}

Як можна порівняти український хліб той, що був двадцять п'ять років тому, який
можна було намастити маслом і запити кавою якобс, і це було неймовірно смачно,
з тим що продають зараз

\begin{itemize} % {
\iusr{Aleks Faershtein}
\textbf{Светлана Лещенко} В то время был кофе индийский растворимый, а кофе \enquote{Якобс} появилось позже...
Если я не ошибаюсь!

\begin{itemize} % {


\iusr{Светлана Лещенко}
\textbf{Aleks Faershtein} я мала на увазі 90-ті @igg{fbicon.wink} . Якщо відняти 25 років так і буде. Ви тільки що заставили мене перераховувати скільки мені років @igg{fbicon.laugh.rolling.floor} 

\iusr{Aleks Faershtein}
\textbf{Светлана Лещенко} 

Я конечно дурак, но не настолько, девушка!!!!!!!!
@igg{fbicon.hands.pray}  @igg{fbicon.heart.with.ribbon} @igg{fbicon.peace}
@igg{fbicon.hand.ok}  @igg{fbicon.100.percent}  @igg{fbicon.hand.victory}
@igg{fbicon.face.wink.tongue} 

\iusr{Светлана Лещенко}
\textbf{Aleks Faershtein} ну так якобс в 90-ті і був, 93-95 точно, ми ще колотили його з цукром щоб була пінка

\iusr{Aleks Faershtein}
\textbf{Светлана Лещенко} 

Да, сейчас это делает машинка... Жизнь не стоит на месте... Буквально 10 минут
тому получил плохую весть из Киева, сегодня ночью умер мой
новый и ставший мне близким друг Леша... Земля ему пухом, АМЕН
@igg{fbicon.hands.pray}  @igg{fbicon.wilted.flower}
@igg{fbicon.heart.broken}  @igg{fbicon.place.of.worship} @igg{fbicon.flag.ukraina}
@igg{fbicon.flag.izrail} @igg{fbicon.flag.usa} @igg{fbicon.face.anxious.sweat}
@igg{fbicon.synagogue}  @igg{fbicon.church}  @igg{fbicon.kaaba} 

\iusr{Светлана Лещенко}
\textbf{Aleks Faershtein} співчуваю, від поганих новин ніхто не застрахований. Але життя продовжується

\iusr{Aleks Faershtein}
\textbf{Светлана Лещенко} Дякую за співчуття, а вам в Україні скоріше отримати вакцину від будь-якої компанії та вийти з цього небезпечного положення! Я не говорив та не листовав українською мовою! Можливі помилки, на все добре з Ізраїлю!!!
@igg{fbicon.hand.victory}  @igg{fbicon.hands.pray}  @igg{fbicon.100.percent} @igg{fbicon.peace}

\iusr{Светлана Лещенко}
\textbf{Aleks Faershtein} я не збираюся вакцинуватися)))

\iusr{Aleks Faershtein}
Я не розмовляв та не листовав майже 30 років!!!

\iusr{Aleks Faershtein}
\textbf{Светлана Лещенко} Поцікавтесь який стан на цю хвилину в Ізраїлі!!! Ми займаємо перше місце за вакцинуванням в світі!!! Це і є результат вакцинування... @igg{fbicon.hand.ok}  @igg{fbicon.face.zany}  @igg{fbicon.face.wink.tongue}  @igg{fbicon.hands.pray} 

\iusr{Светлана Лещенко}
\textbf{Aleks Faershtein} ну і що, мене це не цікавить. Ні вакцини, ні штучно створені віруси

\iusr{Oksana Shukayeva}
\textbf{Aleks Faershtein} дякуємо:)

\end{itemize} % }

\iusr{Oksana Shukayeva}
\textbf{Светлана Лещенко} зараз кава краща та і укр.хліб, як на мене, теж смачнючий @igg{fbicon.face.savoring.food} 

\iusr{Светлана Лещенко}
\textbf{Oksana Shukayeva} я сама печу

\ifcmt
  ig https://scontent-frx5-1.xx.fbcdn.net/v/t1.6435-9/168524925_3909990375743402_1207870341342479223_n.jpg?_nc_cat=110&ccb=1-5&_nc_sid=dbeb18&_nc_ohc=u-MoGSfU4XYAX9Nroya&_nc_ht=scontent-frx5-1.xx&oh=00_AT8Gk61waashMAbVaDIndXvNPlYorEXwQf08rGGvUFCI2Q&oe=61DEABBC
  @width 0.3
\fi

\end{itemize} % }

\iusr{Ирина Иванченко}
Нет сейчас его. Технологиии, говорят, накладны по теперешним временам...

\iusr{Vladimir M. Pushkarev}

Да, когда-то был хороший хлеб! Помню, уезжая из Киева в походы - всегда брал с
собой пару буханок. И даже вылетая за границу прихватывал с полбуханочки. Тем
более, то что у них продается хлебом вообще нельзя назвать! И хранился хорошо.
А сейчас черт знает что! Нарезать нельзя - вся середина крошится. О вкусе уже
не говорю!


\iusr{Luba Severinovsky}
Паляниця - найкращий хліб мого дитинства

\iusr{Дмитро Піталов}

Український хліб вже давно не такий, як був! Технологію точно змінили! Тепер
він гливкий, крихкий та покривається пліснявою на другий день! Київхліб
спаскудив наш улюблений хліб! Треба боротися за повернення старої технології!
Те що зараз їсти неможливо!


\iusr{Ольга Еремеева}

К сожалению, украинский хлеб сейчас совсем не тот, что был раньше. Пеку дома по
рецепту, но в духовке такой корочки не получается.


\iusr{Vladimir M. Pushkarev}
Говорят хороший украинский можно купить в киоске при хлебозаводе (не помню №) -
рядом с пивзаводом \enquote{Оболонь}. Но туда специально заезжать надо...

\iusr{Ніна Клименко}

В Одесі можна придбати смачний хліб, який не пліснявіє на завтра. В Києві
такого хліба давно немає. Чи мені не траплявся? На Печерську точно нема.

\begin{itemize} % {
\iusr{Леонід Лавренов}
\textbf{Ніна Клименко} На Печерську завжди живуть \enquote{нові кияни}.  @igg{fbicon.smile} 
\end{itemize} % }

\iusr{Татьяна Глущенко}
Пам'ятаю смак скоринки, і на село до бабусі везли завжди укр хліб і плетінку

\iusr{Дмитро Піталов}
Зараз продають якусь бодягу нашпиговану прискорювачами... Нема тієї смачної шкуринки та внутрішня консистенція не та! Бізнесюки спаскудили хліб!

\iusr{Evgeniy Maslov}
Самый лучший Украинский хлеб в Киеве выпекал хоебзавод \#6, что на Левобережной...
Тоже помню эту хрустящую корочку.
Мой покойный дедушка, приходя с базара всегда отрезал себе горбушку украинского, натирал её чесноком, нарезала ломтик купленного сала - лучшая закуска к 100 гр.
 @igg{fbicon.smile} 
К сожалению, сегодняшний Украинский уже не тот...
 @igg{fbicon.frown} 

\iusr{Георгий Шинковский}
Вже років 20, а то і більше не вживаю бо то не український хліб.

\iusr{Василий Барбон}

настоящего украинского давно нет. разница между украинским и дарницким, а также
другими перечисленными наименованиями такая же как между подолом и печерском

\iusr{Вера Карпенко}
Супер!

\iusr{Leonid Chavoulski}
Щиро Вам дякую, за слушну історію про Хліб.

\iusr{Наталия Борщева}

Донести и не обгрызть было невозможно! Недавно как раз вспоминали с мужем, как
получали за обгрызенный \enquote{мышами} хлеб. А запах на всю улицу, когда приезжал
хлебный фургон и распахивал дверцы!

\iusr{Роза Миколаївна Кириченко}

Спасибо за рассказ. Я работала на хлебозаводе на Щекавицкой в 50 годах, учась
на вечернем отделении Киевского пищевого института. После окончания института в
1964 г. снова работала на заводе. Выпекали украинский и арнаутку. Киевляне
помнят этого хлеб. Сейчас вкус украинского другой и черствеет быстро.

\begin{itemize} % {
\iusr{Татьяна Петрачек}
\textbf{Роза Миколаївна Кириченко} арнаутку очень любила, сейчас ее не найдешь. Хлеб стал очень плохой, что и есть не хочется
\end{itemize} % }

\iusr{Ольга Топчий}
Ох, не травите душу воспоминаниями о том, КИЕВСКОМ, хлебе, самым вкусным на свете!

\iusr{Елена Стольникова}
Самый лучший хлеб канул в бездну... Сейчас даже хлебом не пахнет...

\iusr{Ruslanka Kraevskaya}
Сейчас хлеб горечью отдает или у меня вкус поменялся

\iusr{Inna Voronenko}
\enquote{Рецептура не змінилася}, а смак (нажаль) змінився, я вважаю(((

\iusr{Tatiana Goncharova}

Спасибо! Мой хлебный был на улице Константиновской, дом 19. Там всегда был хлеб
свежий, горячий. Иногда заходила в хлебный на Нижнем Валу, больше был похож на
лавку (старенький магазин) и там действительно всегда также выкладывали хлеб на
оконный подоконник (как на фото). И я тоже кусала хрустящую корочку украинского
хлеба, паляницы, бородинского, городской булки, студенческой, батона или
съедала большой бублик с маком, булочку-малютку (помните такую вкусную). И уже
дома (улица Межигорская, дом 13) свежий хлеб, да с холодным молоком из
стеклянной бутылки или вкуснее из треугольного пакета (молочный чуть дальше по
Константиновской, около кинотеатра Жовтень)... ах, как же тогда было вкусно! А
мама любила рогалик с повидлом или сдобную булку с изюмом, кусочек со сливочным
маслом (ещё тем!) и чашечкой настоящего индийского чая, такого больше качества,
вкуса хлеба нет... даже у нас на Подоле. И запах был божественный! Ах и ох...


\iusr{Раиса Игнатенко}

У нас в Донецке также выпекали этот хлеб подписываюсь под каждым вашим словом.
Эмоции те же. Вкус детства.


\iusr{Екатерина Степаненко}

На привеликий жаль, справжнього українського хліба тепер немає. Є аналог, який
на наступний день їсти не хочеться.

\iusr{Лидия Васильевна}

Новус на Княжем затоне- пекут разный хлеб, нравились маленькие булочки,
гречишный хлеб, булочки с начинкой... Все бы ничего, НО частенько хлеб сырой, а
в маленькие булочки стали что то добавлять, что их жевать надо очень долго и ни
чай, ни другая жидкость не помогает разжевать эту жевачку !!! Значит раньше
старались ВЫПЕКАТЬ ХЛЕБ!!! А сейчас \enquote{пекут} деньги.......

\iusr{Magdalena Helga}
А цена......какие 20 копеек? Помимо того, что это .больше не хлеб......
Никогда не могла подумать, что настанет время, когда \enquote{хлеб} станет дорогим и ежедневно дорожающим.
И килограмм, как настоящий Украинский, больше ничто не весит.
Как и никогда больше не будет литра молока, 1000 гр растительного масла.
Стали продавать фасованные какие-то подлые половинки....завелись полбатона...немыслимо..

\iusr{татьяна гордиенко}
Украинский хлеб не вкусный и даже не пахнет хлебом и корочки нет

\iusr{Людмила Зилинская}
Эх, какой был хлеб! Вкуснее ничего нет!

\iusr{Наталья Борчанинова}

В детстве жила с прабабушкой, бабушкой и дедушкой в Дарнице. Посылали купить
свежий хлеб в определенном магазине. Пока пройду через сквер ДК
«Железнодорожников» обгрызу чуток корочки (много нельзя было, а то перебью
аппетит и не буду кушать суп). Однажды бабушка разыграла прабабушку
«Бессовестные продавцы ребёнку продали хлеб, который мыши погрызли...» От
прабабушки было что слышать тем ,,бессовестным,, продавцам... Правда ее
возмущение слышали я и бабушка...


\iusr{Евгения Мамичева Яцкевич}

І знову Ви торкнулися душі... ми мешкали на Набережно-Хрещатицькій, і в будинку
був розташований хлібний магазин. Той аромат пам’ятаю крізь роки! Найсмачніший
український хліб подільського дитинства!

\iusr{Наталья Борчанинова}

Лет 10 тому назад я летала в Израиль и везла с собой 2 буханки «Украинского»
(качеством уже с современным вкусом, но с Украины) и шматок сала... Наблюдала,
как родич делил тот хлеб и сало - другим родичам, сотрудникам, друзьям ... Я
видела у одного из его друзей слёзы, когда он держал в руках такой бутерброд и
наслаждался запахом...


\iusr{Наталья Борчанинова}

А потом настала эпоха «Экономика должна быть экономной!» И началась подмена
ингредиентов, ускорение выпечки и тд...


\iusr{Юлія Шока}

Нажаль зараз український хліб зовсім не такий як раніше. Немає тої скоринки яку
можна було натерти часничком, і смак зовсім інший якийсь кислуватий, на
наступний день він вже зовсім не смачний.

\iusr{Наталли Монашненко}
Так, смак дитинства..
Помню, как в детском саду, нянечка тётя Оля, заносила в группу на большом подносе нарезанный украинский хлеб... я помню этот запах, этот неповторимый аромат хлеба, хотя прошло уже много лет. Боже, как же он пах!
Спасибо за то, что написали о святом, о нашем украинском хлебе.
Благодарность людям, которые заботятся о нас и пекли и дай Бог будут ещё долго печь этот замечательный хлеб.

\iusr{Наталья Борчанинова}
А потом настала эпоха «Экономика должна быть экономной!» И началась подмена ингредиентов, ускорение выпечки и тд...
И все кануло в Лету...

\iusr{Наталия Водяницкая}

Спасибо большое, Петр, за такое тёплое воспоминание и вкус
детства.@igg{fbicon.heart.red} Я жила на ул. Бассейной, 11 (Бессарабка) и
всегда с дворовыми друзьями бегала в хлебный магазин на Крещатике (напротив
\enquote{Бессарабского Крытого рынка}, около \enquote{Овощного} на углу с бул.
Т. Шевченко - тогда \enquote{Ленина}) за буханками \enquote{Паляницы} и
\enquote{Украинского}. И, действительно, домой доносиоа только половины и того,
и другого. Уж очень ароматнымм, аппетитнымм и необыкновенно вкусными были эти
хлебные буханочки. Всегда свежие (чаще тёплые), с хрустящими корочками, как Вы
описываете. Я щедро угощала по дороге своих друзей. Меня не ругали. Просто
иногда приходилось два раза ходить в магазин, если почти ничего домой не
доносила. @igg{fbicon.face.grinning.big.eyes}  И воспоминания эти - ТАКИЕ
приятные, добрые, солнечные... Спасибо, что напомнили. Супер! Вкус
незабываемый! Ням, ням, ням!... @igg{fbicon.thumb.up.yellow}
@igg{fbicon.face.happy.two.hands}  @igg{fbicon.heart.eyes}

\begin{itemize} % {
\iusr{Vladimir Machulenko}
\textbf{Наталия Водяницкая} если подумать, то Ленина - это нынче Б.Хмельницкого.

\begin{itemize} % {
\iusr{Наталия Водяницкая}
\textbf{Vladimir Machulenko} точно. Заработалась... Прошу прощения. Ошиблась. Спасибо. @igg{fbicon.smile} 

\iusr{Vladimir Machulenko}
\textbf{Наталия Водяницкая} этого я Вам не прощу никогда.
Шутка. Удачи.

\iusr{Наталия Водяницкая}
\textbf{Vladimir Machulenko}

\ifcmt
  ig https://scontent-frx5-2.xx.fbcdn.net/v/t1.6435-9/167375937_1170420153414662_6625545675724574577_n.jpg?_nc_cat=109&ccb=1-5&_nc_sid=dbeb18&_nc_ohc=A2RAcxrIZrQAX9ZgB__&_nc_ht=scontent-frx5-2.xx&oh=00_AT9eQVUnQgCQL259QQTwQpsQWDLaHT7ssD0PJmzyM17eMA&oe=61DBC0C0
  @width 0.4
\fi

\iusr{Vladimir Machulenko}
\textbf{Наталия Водяницкая} а у Вас вся спина белая

\iusr{Наталия Водяницкая}
\textbf{Vladimir Machulenko} нет. Под купальником только. @igg{fbicon.face.grinning.sweat}  А в остальном, согласна, - очень на меня похожа!  @igg{fbicon.beaming.face.smiling.eyes} 
\end{itemize} % }

\end{itemize} % }

\iusr{Helen Ferrum}
Саме зараз і зжувала м'якенький окраєць і шматочок сала !!! @igg{fbicon.face.tears.of.joy}{repeat=5} 
@igg{fbicon.index.pointing.up} @igg{fbicon.thumb.up.yellow}{repeat=4} 

\iusr{Елена Свец Баркан}

Моя бабушка работала в хлебном в гастрономе Аполоник на углу Нижнеюрковской и
Фрунзе. Я бывало на каникулах помогала ей. Вот что интересно, раньше у хлеба
была такая треснутая хрустящая корочка сбоку. Я выбирала такую буханку, чтобы
была корочка. А сейчас как-то так пекут, что корочек нет, а жаль((

\iusr{Ірина Сирота}

Сейчас совсем не тот по вкусу украинский, что был в 90х, я думала мне это
кажется, но спросила у супруга и он тоже такого мнения. Нет той корочки и того
специфического вкуса.

\iusr{Людмила Капшученко}

Сейчас Украинский не имеет того вкуса, что раньше. То ли технологию тогдашнюю
утратили, то ли ее просто перестали соблюдать с целью удешевления ингредиентов,
но вкус и качество основательно испортились, к сожалению.

\iusr{Nina Volynska}

Далеко куцому до зайця, це вже не той УКРАЇНСЬКИЙ, що був в дитинстві і
молодості. Тепер він став ніякий

\iusr{Irina Vaiyman}

Ох раздразнили! Живу в стране, где масса вкусностей (Бельгия), а вот хлеба на
мой вкус найти никак не могу! « испортил»меня этот Украинский ☺ ️ , да и Арнаутка
тоже!


\iusr{Ольга Бондар}
Це СМАК нашого ДИТИНСТВА! Запашний, гарячий, хрустка скориночка, важко було цілим донести додому з гастроному.

\iusr{Maryna Chemerys}
З маслом і сіллю!))

\begin{itemize} % {
\iusr{Константин Панасенко}
\textbf{Maryna Chemerys} А ще насіннячком зверху присипать!
\end{itemize} % }

\iusr{Генрих Киевский}
Есть хлебзавод напротив АТБ на подоле

\iusr{Наталья Борчанинова}
\textbf{Генрих Киевский} Завод - то есть, а вот рецептура и технология уже совсем не те...Эх...(((

\iusr{Lola Madino}

Хоч і не люблю цю тональність - але тут маю погодитись з більшістю...

Український став практично неїстівний. Ще хоч якось можна їсти від Кулінічів.
Ані Київхліб ані Цар-хліб практично непридатні...

Страшенно сумую за можливістю купити чвертинку свіжого і зжерти прямо на ходу.

\iusr{Татьяна Вахрушева}

Да, хлеб был вкусный и, во времена моей молодости, можно было купить его прямо
с машины! Как аппетитно он пах! А, если черствел то получались

\iusr{Ирина Мазур}

А помните, хлебный магазинчик, на нынешнем Майдане, чуть левее остановок 16 и 18
троллейбуса? Я, тогда ещё маленькая, любили покупать с моим дедушкой
\enquote{АРНАУТКУ},- огромную душистую буханочку!. Вот где корку можно было ещё
отгрызть! @igg{fbicon.face.grinning.big.eyes} 

\begin{itemize} % {
\iusr{Ирина Кучер}
\textbf{Ирина Мазур} а ещё паляницу , вот где корка была !

\iusr{Татьяна Шульга}
Мы обожали этот магазин. Катались в детстве на 18 троллейбусе до конечной-покупали хлеб и пока доезжали до пл.Шевченка- обгрызено было все по кругу

\iusr{Наташа Котляренко}
И почему-то там хлеб всегда был теплый, с пылу-жару.....

\iusr{Татьяна Шульга}
\textbf{Наташа Котляренко} потому и ездили. В магазинах порой твердый уже, а там горячий даже.
\end{itemize} % }

\iusr{Татьяна Вахрушева}
Невероятно вкусные сухарики с солью и часнычком!

\iusr{Сергій Тім}
Здається опис хліба більш відповідає Житньому домашньому хлібу зараз

\begin{itemize} % {
\iusr{Lola Madino}
\textbf{Сергій Тім} Житомирський фермерський з'явився непоганий. Але небагато де продається...
\end{itemize} % }

\iusr{Наталья Степанова}
После 2000 хлебный магазин ещё существовал, рядом Макдональц.

\iusr{Rudenskiy Aleksandr}

К моему сожалению, пост написан очень плохо. И дело не в содержании, а в том,
что называется \enquote{стиль} и логическое изложение информации. Дорогой автор, так не
излагают свои мысли. Нельзя так писать!. А по поводу хлеба - он был не
украинский хлеб, а назывался хлеь \enquote{Украинский}. По поводу рецерта этого хлеба -
он всегда был на уровне государственной тайны, а дело в том, что хлеб
\enquote{Украинский} очень долго не высыхал, долго оставался мягким и это делало его
стратегическим проуктом для любых экстремаьных ситуаций, и в том числе войны. И
учитывая специфику его производства и то, что он был более дешевый, чем другие
хлебные изделия, делали его самым главным продуктом в рационе многих
организаций, которые должны были кормить большие группы населения. И
действительно, вкус этого хлеба забыть невозможно, как и сравнить его с любым
другим. Это тот случай, когда можно сказать : "Слава хлебу "Украинский"!

\begin{itemize} % {
\iusr{Юрий Махиня}
\textbf{Rudenskiy Aleksandr} 

Ви пишите тільки правильно назву, ну а решта Вашого допису не зовсім відповідає
дійсності, відносно рецептури, яка державна таємниця, рецептура хліба була
відома на всіх хлібозаводах і всім, кому це було цікаво. Ну черствів цей хліб
так само швидко, як і решта хлібної продукції, все залежало від умов
зберігання


\iusr{Юрий Махиня}
\textbf{Rudenskiy Aleksandr} 

ну а відносно дешевої вартості цього хліба, то ціноутворення цього продукту
завжди контролювалася на державному рівні і хлібозаводи реалізовували цей хліб
майже по собівартості, така тоді була установка ЦК КПРС, що хліб всьому
голова, тому вартість його була дуже низькою, основний прибуток хлібозаводи
отримували за рахунок булочно - кондитерських виробів

\iusr{Rudenskiy Aleksandr}

Вы пишете то, что не знаете. Рецептура в данном случае хлеба \enquote{Украинский} была
известна только одному человеку на хлебозаводе ( может быть двум. Была должность
Главный технолог. А у каждого хлеба был не просто свой рецепт, а своя
определенная зпкваска теста. Вот она и была в данном случае никому не известна,
кроме тех, кто за это отвечает. Иначе бы такой хлеь пекли бы где угодно.

\begin{itemize} % {
\iusr{Юрий Махиня}
\textbf{Rudenskiy Aleksandr} 

я працював років з п'ять на хлібокомбінаті 2 на Щекавицькій, 55 і саме цей хліб
там тоді вироблявся і розповісти про технологію виготовлення цього хліба я можу
мабуть досить добре, і ще раз кажу, що те що Ви пишите про якусь супер секретну
технологію,не відповідає дійсності

\iusr{Ирина Микитюк}
\textbf{Rudenskiy} 

Вы слишком много киношек смотрите. Это ж СССР, утвержденные чуть ли не в обкоме
партии, любые рецептуры и процессы. Какие тайны? О чем Вы?:))

\iusr{Раиса Винник}
\textbf{Ирина Микитюк} Руденский Саша в те годы был фирменным киевским экскурсоводом, с отличной эрудицией, интеллектом, опытом, подвижный как ртуть. Поддерживаю его точку зрения.
\end{itemize} % }

\iusr{Muravov Dmitriy}
\textbf{Юрий Махиня}, ви б не могли, будь ласка, дати рецепт хліба?  @igg{fbicon.smile} 

\begin{itemize} % {
\iusr{Юрий Махиня}
\textbf{Muravov Dmitriy} 

все залежить від дуже багатьох складових, а саме якісті борошна, потім якості
питної води, а також якісті дрожжів, крім того температури, при якій визрівало
тісто і саме головне, це конструкція самих печей, температурного режиму в цих
печах, тільки вдале поєднання цих факторів давало якісну продукцію і тому хліб
випичений на різних хлібзаводах завжди відрізнявся своєю якістю, в той час, я
маю на увазі 80 роки, дуже якісний хліб випікався на хлібокомбінаті 1, х/к 2
також випікав дуже якісну продукцію, ну відносно конкретної рецептури то раджу
подивитися по цьому нормативні документи того часу, я працював не технологом, а
начальником відділу збуту, тому якісь моменти може вже й забув, адже пройшло з
того часу багато років

\end{itemize} % }

\iusr{Юрий Махиня}

Крім всіх перерахованих вище складових якості хліба Украінского на той час дуже
впливала вага виробу, сама технологія була розроблена на буханку вагою 1 кг, і
такий хліб з самого початку був дуже смачний, але потім партійні боси і
керівники промисловості з метою покращання економії хліба, дали вказівку
хлібозаводам випікати хліб Український вагою 0,75кг і відповідно якість виробу
знизилася, а потім намагалися в угоду партії на деяких хлібзаводах робити хліб
ще меншим, ну і відповідно смак цього хліба був зовсім інший

\end{itemize} % }

\iusr{Любовь Чернявская}

По роду службы проводила проверку на 2 хлебокомбинате в начале 80-х.
Руководитель рассказала, что их специалисты делают, практически, невозможное.
Пекут вкусный хлеб из канадского кормового зерна, а наше наилучшее идёт на
экспорт. Сейчас хлеб совсем не тот, специалистов нет, либо рецептура иная. Жила
в те годы на ул. Набережно-Луговой, 1, поблизости от хлеб завода. Запомнился
запах свежего хлеба.

\begin{itemize} % {
\iusr{Юрий Махиня}
\textbf{Любовь Чернявская} саме так, в той час не вистачало своєї пшениці і Союзом проводилися закупівлі канадського товарного зерна з низьким рівнем клейковини, саме тому в наших магазинах не було довгих макаронних виробів, а продавалися рожки. Десь в той час, що Ви пишите, я працював начальником відділу збуту х/к 2 на Щекавицькій 55
\end{itemize} % }

\iusr{Мария Константиновская}
Приводжу із Києва украінський хліб, сало. Потім їмо в родині із салом та
малосольним огірком, закусуючи горілочку з морозу... @igg{fbicon.sheaf.of.rice} @igg{fbicon.hand.victory}

\iusr{Maryna Chemerys}

В дитинстві бачила не раз, як який-небудь дядько чи тітка набирали повний мішок
буханок \enquote{Українського} - казали, в село, свиней годувати. Мені це було так
дивно!

\iusr{Анна Самарина}
\textbf{Maryna Chemerys} я теж саме згадала)))

\iusr{Вероника Приходько}
Самый вкусный завод 1

\iusr{Валентина Корниенко}
Тепер не той смак ! Я теж вожу в Ізраіль український хліб... тепер це не той хліб.

\iusr{Iya Yunyk}
Як \enquote{смачно написали}, @igg{fbicon.smile}  @igg{fbicon.thumb.up.yellow} 

\iusr{Леонід Лавренов}

Ви нагадали про звичайну у побуті киян річ - хліб. Український, арнаутка,
паляниця, французька, сайка, \enquote{жулик}, \enquote{здобна} і за кожною назвою - спогади
дитинства.  @igg{fbicon.smile} 

\iusr{Алла Гузий}
\textbf{Леонід Лавренов}  @igg{fbicon.hands.pray}{repeat=2}  @igg{fbicon.thumb.up.yellow}{repeat=2}  @igg{fbicon.grin} 

\iusr{Михайло Надкреничний}
Хліб на фото - це совецько-самостійний, незалежно-більшовицький хліб.
Український хліб - це був Хліб!
Навіть ті, хто в 70-ті випікали домашній хліб в печі, просили привезти з Києва український Хліб.

\begin{itemize} % {
\iusr{Леонід Лавренов}
\textbf{Михайло Надкреничний} Не плутайте праведне з грішним. Якщо випікали хліб власноруч, то з Києва хліб був потрібен для годування худоби. Вартість буханця була 12, 14, 16 копійок за кілограм. Ви за 16 копійок зпечете хліб? А домашній - він завжди кращий.

\iusr{Валентина Корниенко}
\textbf{Леонід Лавренов} годували худобу кирпічиком, а не Українським. Ми хоть самі і пекли, але як була можливість то привозили із столиці круглий український. Ми їли аж давилися. А свій печений нам надоїв.

\iusr{Леонід Лавренов}
\textbf{Валентина Корниенко} Ви, мабуть, мали на увазі, що свій набрид. Так, приїдався і періодично міняли свій вибір. Так і зараз, але тепер доступно всім.

\iusr{Ніна Клименко}
\textbf{Михайло Надкреничний} Кращого подарунка не було. Надзвичайно смачний був хліб.
\end{itemize} % }

\iusr{Muravov Dmitriy}
А рецепта у вас случайно нет?  @igg{fbicon.smile}  А то мои опыты с закваской не очень удачные.
Хлеб получается, но не такой по вкусу и текстуре..

\iusr{Валентина Корниенко}
\textbf{Muravov Dmitriy} дома такий не виходить. Чи мука не та чи іще щось.

\iusr{Алла Тихонова}
Теперь без корочки, запаянный в пластик...  @igg{fbicon.frown} 

\iusr{Виктория Кустова}
Точно - была внизу зажаренная корочка которую мы дети, пока довезем буханку домой, обгрызали.

\iusr{Елена Морозова}
Смачий був

\iusr{Света Света}
Родственники из Италии всегда просят передать хлеб и гречку. При всем изобилии
продуктов, только это может и не хватать для полного тамошнего счастья.

\iusr{Алла Вайнерман}
А мне не хватает Арнаутки. Той из детства.

\iusr{Нина Бегус}
\textbf{Алла Вайнерман} точно!

\iusr{Natalya Olyanishyna}
Я з собою завжди беру в Естонію шматочок., хоча там теж є смачний хліб, але не такий

\iusr{Станіслав Теліцький}

Даруйте! Багато пафосу, але від Українського лишилася одна назва. Смак і
структура відрізняються помітно. Псуватится він почав у 80-і. Тоді якість у
різних виробників могла вже відрізнятися. Нині, якості нема ні в кого.. Звісно,
зараз можна знайти дуже непоганий хліб, але це, точно - не український. Його
нема і, напевно, вже не буде. Нема арнаута, сайки, міської, плетінки... Батон -
жалюгідна подоба старого запашного батона зі скоринкою і смачезним м'якушем.

\begin{itemize} % {
\iusr{Анатолий Дидковский}
\textbf{Станіслав Теліцький} Нема \enquote{Паляниці}, найдорожчої, за 28коп.

\iusr{Станіслав Теліцький}
\textbf{Анатолій Дідковський}, вона, здається, зникла серед перших. Дуже пристойний був виріб.

\iusr{Виталий Кулумбеков}

Не в останню чергу тому все так, що годують людей мукою з фуражного зерна,
зараженого грибком зерна. А кондиційне зерно йде на експорт, збагачуючи
нуворишів та олігархів. Стандарти УРСР знищено(


\iusr{Станіслав Теліцький}
\textbf{Виталий Кулумбеков}, 

я не досліджував це питання детально. Та, припускаю, що проблема не лише в
якості борошна. Ймовірно, відійшли від первинної рецептури. Коли в Бучі закрили
пекарню (а сталося це в 70-і), то хліб почали завозити з Києва (здається - з
Дарниці). Новий хліб вже трохи відрізнявся смаком (був трішечки кислуватий) і
вже не було тієї класичної хрусткої скоринки. При м'якуша були дрібнішими.

\end{itemize} % }

\iusr{Светлана Токарева}
Тяжело уже читать на рiднiй мовi...

\begin{itemize} % {
\iusr{Василий Шалыга}
\textbf{Светлана Токарева} а ви у собі уроки візьміть... він нагадає рідну мову...

\iusr{Василий Шалыга}
А ви у Соловйова уроки візьміть... він нагадає рідну мову...(виправлено)

\iusr{Полина Кравченко}
\textbf{Светлана Токарева} тяжко читати тільки якщо мало читаєте. Просто поставте собі за правило читати виключно українською

\iusr{Василий Шалыга}
\textbf{Полина Кравченко} Світлана у РФ... там нема української, там вона обожнює Соловйова ..

\iusr{Полина Кравченко}
\textbf{Василий Шалыга} було б бажання))
\end{itemize} % }

\iusr{Natalya Olyanishyna}

От, нащо Ви це написали? Я вночі ковтнула слину і потіліпалася на кухню
нарізати українського з салом  @igg{fbicon.smile} 

\iusr{Наталия Жиляева}

\ifcmt
  ig https://i2.paste.pics/90c4d817ebd481ca86580bcd8f36768a.png
  @width 0.2
\fi

\iusr{Михайло Майстренко}

Самий смачний УКРАЇНСЬКИЙ був з 4 хлібозаводу з вул Дегтярівськоі,( нині
покійний), батон нарізний з 10, що на Шулявці, сухарі КИЇВСЬКІ, також з 4.


\iusr{Марина Соснина}
А мне Чумацкий был самым вкусным

\iusr{Олександр Пелешко}
От якійсь він дивний став то кислий то ні @igg{fbicon.thinking.face} 

\iusr{Костя Гнатенко}
На жаль, легендарна скоринка у пакетованому хлібові, зовсім ніяка не
скоринка((( у тієї був особливий хрускіт!

\iusr{Людмила Киенко}
Спасибо за добрые слова про украинский хлеб.

\iusr{Виктор Матюрин}
А батоны были по 12 коп., а Украинский - 16 коп.)))

\begin{itemize} % {
\iusr{Thomas Fairmark Mykola}
\textbf{Виктор Матюрин} 

гигиги! нетра про ціни нагадуати, бо тєлік кошував 600р. (3 зп), машина
5000...10000р. (3...6річна зп)

\iusr{Alexander Krivoshapka}
\textbf{Thomas Fairmark Mykola} Так автомобілів у вільному продажі не було і за такі гроші
\end{itemize} % }

\iusr{Елена Светлая}

Дякую за чудові спогади про одну з візитівок нашого міста! Ми можемо тільки
поспівчувати тим, котрі не мають уявлення про смак Українського хліба,
Київського торта, Котлети по-київськи та багато іншого. А Київські каштани!


\iusr{Полина Кравченко}
Пост, який пахне

\iusr{Alexander Krivoshapka}

пам'ятаю в дитинстві, коли батьки приносили гарячий духм'яний \enquote{Український}, я
відривав від нього підгорілу скоринку і ...А батьки бурчали-чого не відріжеш?!

\iusr{Alexander Krivoshapka}

А ще пригадую розповідь куми, яку доля занесла в кінці 90-х в Самару. І коли вона
приїхала в Київ, то розповідала, як зайшла в магазин, купила буханку
\enquote{Українського} і не втрималась, зразу відкусила кусень, після чого закотила очі і
простогнала від задоволення. Бачить, а продавчиня і покупці дивно на неї
дивляться. Сказала їм: \enquote{Просто я 8 лет не ела нашего хлеба}


\iusr{Viktoriia Kosyk}
у студентські 80-ті обов'язково купляла мамі, коли їхала додому @igg{fbicon.heart.red}

\iusr{Игорь Лысенко}
тоже грыз . пошел наливать

\iusr{Галина Галиченко}
А какой он был вкусный, натёртый чесночком да со свежим борщиком! Вкус детства!

\iusr{Кирило Анатолїйович Козлов}

ранее любил Украинский хлеб. всегда очень вкусный был, да и свежий всегда.
нынче уже не то. не тот Украинский. вкус и качество уже далеко не те, что были,
что помню.

\iusr{Валентина}
Український хліб 10 комбінату їсти не можливо.

\iusr{Ириша Эс}
Все вірно! Моя подруга вже давно диве в Іспанії і я завжди везу їй цей хліб і
сало, більше вона нічого не просить.  @igg{fbicon.grin} 

\iusr{Ольга Dzhun}

В Киеве был самый вкусный хлеб, его вкус и запах забыть невозможно! Жаль, что
теперь такого хлеба нет. Спасибо за воспминания.

\begin{itemize} % {
\iusr{Алёна Бидная}
\textbf{Ольга Dzhun} а ещё качество Хлеба отличалось по районам. Дольше всего Хлеб оставался кусным на Печерске.
\end{itemize} % }

\iusr{Thomas Fairmark Mykola}

Зараз є чисто житній баварський, білоруський, прибалтійський- у нас в сім'ї ці
сорти розходятьса на ура... українский хліб давно зіпсувався, особливо фірми
\enquote{Кулінічі}.

\iusr{Алёна Бидная}

Нет уже нашего хлеба.... ни Украинского, ни Паляныци, ни-че-го... всегда
выбирала с самой большой корочкой.... домой уже приносила половину Хлеба...

\iusr{Олеся Рыбак Дедерлайн}

Шкода, що нема вже того \enquote{українського}, палягиці, арнаутки та сайки
тоненької такої по 6 коп.

\iusr{Людмила Берникова}
Слюнки потекли...

\iusr{Олег Сафронов}

Мне нравились Арнаутка и Паляница. Мы покупали их в булочной на Ленина (теперь
Б. Хмельницкого). Пока я нёс их домой на Леонтовича, четверти арнаутки или
гребешка у паляницы уже не было.

\begin{itemize} % {
\iusr{Анна Сафронова}
\textbf{Олег Сафронов} Вон оно, что оказывается!

\iusr{Олег Сафронов}
\textbf{Анна Сафронова} да, из-за меня тебе доставалост меньше. Но уж больно вкусными были и арнаутка, и паляница.

\iusr{Анна Сафронова}
\textbf{Олег Сафронов} Вот, когда правда всплыла!

\iusr{Анна Сафронова}
\textbf{Олег Сафронов} Зато я помню, как ты мне картошечку жарил, когда я приходила из школы.
\end{itemize} % }

\iusr{Natalie Bugay}

Завжди везли з Києва на гостинець @igg{fbicon.thumb.up.yellow} 

\iusr{Dmytro Poltavcev}
Дуже любив Український хліб, а також сухарі з нього

\iusr{Анна Сафронова}

Какой же вкусный кусочек Арнаутки, посыпанный сверху сахаром. Я помню это
лакомство.

\iusr{Евгения Сергеева}

Только сегодня с мамой вспоминали \enquote{тот} украинский и слюной
обливались.... вот та треснувшая корочка, это что то.... Нет сейчас такого
хлеба....

\iusr{Василь Максимов}

Така смачна і душевна Ода українському Хлібу, що пригадалися
далекі \enquote{київські} роки, коли ми, безтурботні студенти і нарізАли, і наливали, і
випивали, і закушували... Далека молодість і вічна Любов... Це неповторно...

\iusr{Дмитрий Карлов}

Ещё был маленький магазинчик на Крещитике, напротив Бессорабки, возле магазина
Каштан. Там в 80-х продавали ТОЛЬКО бублики, а возле кассы стоял ТЕН с горячим
молоком. ВКУСНОТИЩА ЖЕ!!!

\begin{itemize} % {
\iusr{Нина Омельченко}
\textbf{Дмитрий Карлов} , а назывался этот магазин « Молоко «
\end{itemize} % }

\iusr{Alla Andrushchak}

Його любила я і обожнювала моя мама, нині покійна. Зараз віддаю перевагу
крафтовому гречаному власного виробництва \enquote{Сільпо}.  Ваша розповідь навіює
теплі спогади дитинства та юності. Дякую!

\iusr{Ольга Кореневская}
Нашего УКРАИНСКОГО хлеба НЕТ !!!!

\iusr{Арт Юрковская}

Почему-то он стал невкусный. Никогда не покупаем. Перешли на белорусский, 
бородинский и проч. Но такого хлеба, аналога нет. Его ничем не заменишь. Да,
корочку помню, обрызала, пока до дома доносила. Со сливочным маслом и солью -
очень вкусный был.

\iusr{Виталий Кулумбеков}

В Осетию привозил хлеб Украинский, и там тоже были в восторге от него) Плюс
сало!)

\iusr{Ludmila Vakulenko}
Дякую за чудову розповідь, про нині, таки буденну присутність «Украінського» хліба на нашому столі
Пригадався й смак і дитячі відчуття
Й що характерно, пам‘ятаю, що саме цей хліб можна було істи сам по собі, такі смачнючі нотки в ньому відчувались
А ще трохи додам посмішки,- коти, яких обов‘язково мала за хатніх питомців,- могли, й хотіли істи шматочки «Украінського» хліба навіть після м‘яса
Ніякий інший, а цей завжди
Це мене не аби як й дивувало, й веселило
Та перелік вітамінів, та корисних речовин, додатково смак, -мабудь багато чого пояснює, чому його так любимо
Дякувати пекарям- винахідцям!!!

\iusr{Ольга Симак}

Як Ви затишно пишете. Перечитую всі Ваші пости в цій групі бо сама тут недавно.
Дякую @igg{fbicon.heart.red}

\iusr{Любовь Кузан}

Це правда, наш \enquote{Український} любили і за межами України. В 90-х до нас в гості
приїджали родичі з Ашгабаду, так завжди з собою брали пару буханок хліба і шмат
сала @igg{fbicon.wink}  На жаль зараз \enquote{Український} вже не такий....

\begin{itemize} % {
\iusr{Наталія Савченко}
\textbf{Lybov Kuzan} 

Мій дядя з сім'єю жив в Конотопі. Коли ми до них їхали, просив, щоб 4 буханки
українського привозили, не менше. На зауваження, що він буде дуже черствим
казав: \enquote{То й що?}

\end{itemize} % }

\iusr{Олександр Длігач}
Дуже вдячний за пост!

А магазин на Сагайдачного (тоді - Жданова) був спроєктований мною завдячуючи
особистій ініціативі Салія Івана - доброго господарника тих часів (часів іще
ритмічного походу в нікуди)

\end{itemize} % }
