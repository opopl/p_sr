% vim: keymap=russian-jcukenwin
%%beginhead 
 
%%file 10_03_2021.fb.fb_group.story_kiev_ua.1.bulgakov_smert.cmt
%%parent 10_03_2021.fb.fb_group.story_kiev_ua.1.bulgakov_smert
 
%%url 
 
%%author_id 
%%date 
 
%%tags 
%%title 
 
%%endhead 
\subsubsection{Коментарі}

\begin{itemize} % {
\iusr{Елена Никитчук}
Браво, за жизнь, за любовь браво за поэзию! Мастер БРАВО! ПОМНИМ И ЛЮБИМ!

\ifcmt
  ig https://i2.paste.pics/10fc8a934f8d1e40b5bdd90db0c5401f.png
  @width 0.2
\fi

\iusr{Виктор Михайлюк}
Нет, нельзя его читать, он же на русском языке писал!...

\begin{itemize} % {
\iusr{Володимир Володимир}
\textbf{Виктор Михайлюк} ваш сарказм не доречний і межує з українофобством.,

\begin{itemize} % {
\iusr{Виктор Михайлюк}
\textbf{Володимир Володимир} Добирайте слова, шановний!!! Я зовсiм не шуткую, Я вже пiднiмаю питания про знесення памятника Пушкiну у Киiвi!!!

\iusr{Неля Архипова}
\textbf{Виктор Михайлюк} Но мы- то с вами точно знаем, что шутите, я сама так люблю, особенно в подобных случаях.

\iusr{Alex Aleksandrov}
\textbf{Виктор Михайлюк} ты тяжелее рюмки водки сроду ничего не подымал

\iusr{Alex Aleksandrov}
\textbf{Valentyna Venglovska} ваши пролетарии как власть захватили первым делом памятники и начали сносить а вторым неугодных расстреливать, не понимаю к чему сейчас ваше нытье

\iusr{Ирина Сологуб}
\textbf{Виктор Михайлюк} ну вообще с ума сошел
\end{itemize} % }

\iusr{Елена Семененко}
\textbf{Виктор Михайлюк} а пусть переведут
А Ви знаете что он бил врачом и тогда существовала методика борьби с дифтиритом при помощи трубочек через которие
Витягивали дифтирийние пленки
Бабушка говорила если в селе заболевал ребенок умирали все
До 10 лет
Она била сельской учительницей
Я вам скажу что если би он етого не делал то и памятника и доброй памяти на Украине о нем би не било

\begin{itemize} % {
\iusr{Люба Микицька}
\textbf{Елена Семененко} ОН в КИЕВЕ бьіл венерологом. ЕГо соседи так и запомнили как \enquote{Миша венеролог.}
\end{itemize} % }

\iusr{Володимир Володимир}
\textbf{Виктор Михайлюк} ви, ще про пам'ятник М. Магомаєву питання підніміть...

\iusr{Ольга Гуменюк}
\textbf{Володимир Володимир}
Ну Магомаев же не москвич, он Заслуженный Азербайджанской Республики. Хотя, нашим нацикам, все, кто не Бандера - враг.

\iusr{Олена Мельникова}
Неужели не шутите?

\iusr{Володимир Володимир}
\textbf{Olga Gumeniyk} не несіть дурню!

\iusr{Люба Микицька}
\textbf{Володимир Володимир} чому? АДМИНЬІ одобряют подобньіе вещи.

\iusr{Володимир Володимир}
\textbf{Lyuba Mukutska} нажаль...(((

\iusr{Руслан Трушков}

Господи, сколько же у нас идиотов... одни с памятниками воюют, другим Булгаков
как кость в горле... главное, чтобы своей же блевотой не удавились... или, не
загнили, как те ананасы из рассказа Коровьева...

\end{itemize} % }

\iusr{Маргарита Мышанская}
Думаю, он реально вечен! Любил повторять: \enquote{Главное-не терять
достоинства}! @igg{fbicon.face.smiling.eyes.smiling} 

\begin{itemize} % {
\iusr{Slava Kogan}
\textbf{Маргарита Мышанская} но пьесу в честь Сталина написал \enquote{Батум}

\iusr{Катерина Мазур}
\textbf{Маргарита Мышанская} но при этом, как русский шовенист мог унижать Украину.

\begin{itemize} % {
\iusr{Юрий Блохин}
\textbf{Катерина Мазур} 

Я думаю, что у Вас было бы другое мнение, если бы Вы были коренной киевлянкой,
родившейся в семье интеллигентов .... в конце 19 ст. А делать вывод о том что
было более 100 лет назад с сегодняшнего сумасшедшего времени, по -меньшей мере
неправильно.

\iusr{Неля Архипова}
\textbf{Катерина Мазур} 

шовинист, если позволите. Он лишь родился в Киеве и безмерно и навсегда полюбил
свой Город и, кстати, прославил его на века, чем сделал для Украины больше, чем
все лжепатриоты и националисты вместе взятые. Тем более, что они в основном
только позорили страну своей ненавистью и ограниченностью. А был он, конечно,
русским, как бы это не было вам противно.

\iusr{Катерина Мазур}
\textbf{Неля Архипова} 

он полюбил этот город, как все русские любят, только без украинцев, считая их
недоравитым плебсом. Умными, достойными могут быть только великороссы, ну еще
русскоговорящие украинцы на вторых ролях, которые забыли свои корни. А все
остальные украинцы это рабы недоразвитые.

\iusr{Alex Aleksandrov}
\textbf{Юрий Блохин} 

те \enquote{коренные киевляне} с московитскими корнями имели очень короткую память и
очень быстро забыли как здесь появились. В 18м году Петлюра немного поправил
ситуацию но им на смену быстро пришла новая волна \enquote{коренных киевлян} с красными
тряпками на перевес. Причём начали они свое киевлянство с тредневного обстрела
города и последующей недельной резни имени муравьёва. Сейчас сново выровняли
погнав их вместе с овощем домой в мск

\iusr{Юрий Панчук}
\textbf{Юрий Блохин} Совершенно согласен, человек другой эпохи, выросший в русской семье, глупо его винить. Родиться в такой семье как Леся Украинка скорее исключение. Родись он в другой семье - возможно не стать бы ему Мастером

\iusr{Alexandr Vlasenko}
\textbf{Катерина Мазур} Это Вы серьезно или иронизируете ? )Я не очень-то рефлексирую. Но Киев без украинцев - это как ? Как пасека без пчел ? @igg{fbicon.face.smiling.eyes.smiling} 

\iusr{Ольга Гринь}
\textbf{Катерина Мазур} 

Госпожа, это не Вы-ли потеряли переписку Энгельса с Каутским? Бегите, пока в
печке не сожгли @igg{fbicon.face.savoring.food}  Вы элементарной грамматике не обучены, не стыдитесь писать
такую чушь беспросветную? @igg{fbicon.face.grinning.squinting}  кто такой шовенист, пилять?!?! И это в присутствии
людей с университетским образованием!


\iusr{Alex Aleksandrov}
\textbf{Ольга Гринь} а вы опровергните сказанное) без козыряний не пойми каким образованием и общих оптекаемых фраз))

\iusr{Ольга Гринь}
\textbf{Alex Aleksandrov} а зачем? Мы ужинаем на дне рождения моего обожаемого папочки в компании любимых, мне не до Вас @igg{fbicon.face.blowing.kiss}  Вы и дальше обижайтесь на весь мир, это конструктив @igg{fbicon.face.savoring.food} 
\end{itemize} % }

\iusr{Маргарита Мышанская}

Знаете, что я думаю.... Да, М. Булгаков -русский монархист. В его биографии
написано, что он собирался, пребывая на Кавказе, эмигрировать с белогвардейцами
за границу, но тиф помешал. Я его понимаю.... Он не жил, а мучился при
советах. Скажите, пожалуйста, как ему было продвигать свое творчество? Он великий
писатель, писатель, не изменивший своему духу

\begin{itemize} % {
\iusr{Alex Aleksandrov}
\textbf{Маргарита Мышанская} внимательно перечитайте \enquote{хождение по мукам} потом нам расскажете чему он там не изменил)

\iusr{Igor Popell}
\textbf{Alex Aleksandrov} \enquote{войну и мир} еще можно перечитать внимательно.

\iusr{Alex Aleksandrov}
\textbf{Igor Popell} перечитайте я не препятствую)
\end{itemize} % }

\iusr{Михаил Фролов}
\textbf{Slava Kogan} что значит \enquote{пьесу в честь Сталина} написал? Когда и где состоялась её премьера? Или премьеры вообще не было?
Расскажите историю создания пьесы?

\begin{itemize} % {
\iusr{Slava Kogan}
\textbf{Michael Frolov} зачем расказывать, для этого Google есть. Не поленитесь и найдите инфо сами. Хотя я guy отнюдь не осуждаю. Время такое было.
\end{itemize} % }

\iusr{Маргарита Мышанская}

Если бы да кабы... Он родился так как родился, как каждый из нас. Я русская по
национальности и не могу мыслить как китаец, японец и даже поляк. И напишу о
любом событии так, как мне дано моими, национальными пониманиями.

\begin{itemize} % {
\iusr{Alex Aleksandrov}
\textbf{Маргарита Мышанская} точно, у вас же руских скрепы и лишняя хоромосома, ещё в 14м году выявили и научно обосновали) потому и не можете мыслить как китаец или поляк))
\end{itemize} % }

\iusr{Маргарита Мышанская}
А вы, которые без скрепов и лишней хромосомы, можете мыслить как китаец или поляк? @igg{fbicon.laugh.rolling.floor} 

\iusr{Михаил Фролов}
\textbf{Алекс Александров} а зачем умному человеку опровергать сказанную кем-то глупость?

\iusr{Igor Popell}
то ли про аптеку, то ли про оптику?

\ifcmt
  ig https://scontent-frt3-1.xx.fbcdn.net/v/t1.6435-9/159245382_10159030637246136_355845014517356467_n.jpg?_nc_cat=102&ccb=1-5&_nc_sid=dbeb18&_nc_ohc=_fFHrnJUvwUAX9hqi1W&_nc_ht=scontent-frt3-1.xx&oh=00_AT_VQYrF0EvxxTsfOiOWtUM1wUN_tSAwiaIcnqOsC25iyw&oe=61E10805
  @width 0.4
\fi

\iusr{Михаил Фролов}
\textbf{Slava Kogan} 

я это знаю, когда ещё гугла не было. И какая же это пьеса \enquote{в честь Сталина}? Вы
сначала написали, а на уточняющие вопросы не отвечаете. Булгаков написал пьесу
по своей инициативе? Или инициатором был кто-то другой?

\begin{itemize} % {
\iusr{Slava Kogan}
\textbf{Michael Frolov} это пьеса о батумской о демонстрации 1902 года и начала рев.деятедьности некоего Джугашвили. Булгаков написал её ,когда его пьесы снимали с репертуара и он был в опале.
\end{itemize} % }

\iusr{Людмила Губианури}

Посмотрите, какие интересные материалы мы опубликовали на стр. нашего музея к
этому дню. Bugakov museum in Kyiv


\iusr{Михаил Фролов}
\textbf{Slava Kogan} 

действительно, пьеса об этом и не только об этом. Но написал пьесу Булгаков
совершенно по другому поводу, а не потому что его пьесы снимали с репертуара.
Вы в курсе истории написания пьесы?

\begin{itemize} % {
\iusr{Slava Kogan}
\textbf{Michael Frolov} насколько я знаю из разных источников, он начал писать пьесу в 1929 году, как раз ,когда в прессе началась травля самого Булгакова и его работ. Может в пьесе и есть какие-то ещё сюжетные линии, но в основе молодой Сталин. Кстати пьесу запретили всё равно к постановке
\end{itemize} % }

\iusr{Михаил Фролов}
\textbf{Slava Kogan} 

пьесу он начал писать и написал в 1939 году. Это год 60-летия Сталина. Тогда
многие театры хотели поставить пьесу, приуроченные к этому событию. Руководство
МХАТа тоже решило поставить пьесу. Зная, что в стране нет драматурга лучше, чем
Михаил Афанасьевич, и что он напишет лучшую пьесу, МХАТовцы обратились к нему.
Михаил Афанасьевич понимал, что если пьеса \enquote{наверху} не понравится, это может
иметь для него негативные последствия. И он сказал, что напишет пьесу о молодом
Сталине (уход из Духовной семинарии, организация Батумской стачки, ссылка в
Сибирь и т.д.). Руководство театра согласилось. Была проведена генеральная
репетиция, которая прошла успешно. Главную роль исполнял Николай Хмелев. После
генеральной репетиции вся труппа выехала на поезде в Батуми в место юношеской
деятельности героя пьесы. На станции Серпухов Булгакову в поезде вручили
телеграмму : надобность в поездке отпала.  Булгаков с супругой вернулись в
Москву, впрочем как и вся труппа.

\begin{itemize} % {
\iusr{Slava Kogan}
\textbf{Michael Frolov} ну так и я о том.
\end{itemize} % }

\iusr{Михаил Фролов}
\textbf{Slava Kogan} 

не в 1929 он начал писать пьесу. А в 1939. Не по своей инициативе, а по просьбе
руководства лучшего в СССР Драматического театра.

\begin{itemize} % {
\iusr{Slava Kogan}
\textbf{Michael Frolov} я читал, что задумал он пьесу в 29, писать начал в 36 и работа шла медленно и закончил в 39. Ну а по просьбе или нет ,получилось то что получилось. Не думаю, что он этим сильно гордился. И мы его почитаем и любим не of эту пьесу.
\end{itemize} % }

\iusr{Михаил Фролов}
\textbf{Slava Kogan} 

\enquote{получилось, то что получилось}. И где же в пьесе восхваление Сталина? Вы
пьесу, вообще-то читали? Почему же её запретили? Объяснения, которые дошли до
Булгакова якобы \enquote{с самого верха} лично мне трудно назвать убедительными. У меня
своя версия причины запрета пьесы. И, уверен, эта пьеса должна быть зачтена в
творчество Булгакова со знаком \enquote{+}. Здесь нет никакого восхваления, лозунгов,
клятв в верности. Это о юноше, выбравшем свой путь в жизни.

\end{itemize} % }

\iusr{Ольга Яворовская}

\ifcmt
  ig https://i2.paste.pics/b8221e79398468ee1155cc8720b6b808.png
  @width 0.2
	@name scr.hands.applause
\fi


\iusr{Vitaliy Tretiakov}
Талант

\iusr{Ludmila Debourdeau}
Один из моих любимых писателей @igg{fbicon.heart.red}

\iusr{Ludmila Teslenko-ponomarenko}
Да, рукописи не горят, а самые любимые всегда с нами!

\iusr{Марина Немезида}

\ifcmt
  ig https://i2.paste.pics/d5c1f43db755ce86a999caa90151b2ea.png
  @width 0.2
	@name scr.rose.hearts
\fi

\iusr{Zinaida Mamontova}
Виртуоз слова и мысли на все времена! Гений...

\iusr{Людмила Гаврилюк}
\textbf{Zinaida Mamontova} а хто не розуміє можно тільки співчувати.. ..

\iusr{Ирина Алексеенко}
В гостях у Булгакова

\ifcmt
  ig https://scontent-frt3-1.xx.fbcdn.net/v/t1.6435-9/159763792_2811300479086801_8408082661034046709_n.jpg?_nc_cat=107&ccb=1-5&_nc_sid=dbeb18&_nc_ohc=OSxofmzIVJQAX89nC6k&_nc_ht=scontent-frt3-1.xx&oh=00_AT995T1F5P9v9bnV2us01nyxsxfY8Va5hBioFderryjVqw&oe=61E10DA5
  @width 0.4
\fi

\iusr{Ганна Путова}
\textbf{Ирина Алексеенко} Дякую за це фото. А не панібратськи обіймаючи за плечі, і не на руках вмостившись.

\iusr{Віктор Маслєєв}
Вже дуже не актуальний писака... навіть в Києві... може десь на квжд ще
читають....

\begin{itemize} % {
\iusr{Олена Мельникова}
\textbf{Віктор Маслєєв} А Ви актуальний? Напевно-Швондер...

\begin{itemize} % {
\iusr{Маргарита Мышанская}
\textbf{Олена Мельникова} Шариков и Швондер всегда актуальны! @igg{fbicon.face.tears.of.joy} 

\iusr{Олена Мельникова}
\textbf{Маргарита Мышанская} Да, да. Прочла некоторые комменте и поняла, Шариков и Швондер-бессмертны.

\iusr{Маргарита Мышанская}
Еще и \enquote{навіть в Києві}... \enquote{Как говориться, желаю, чтобы все}... (Собачье сердце)

\iusr{Віктор Маслєєв}
\textbf{Олена Мельникова}
Вже ні ...не актуальний...
А сучасний Швондер це хто....
Не має...
Значить і булгаков зараз до одного місця...
Дівчата... правда... читають до одруження... ну і після 80- ти...
А ви в якій категорії...?

\iusr{Маргарита Мышанская}
\textbf{Віктор Маслєєв} Ну, тогда и Вы до одного места... @igg{fbicon.beaming.face.smiling.eyes} 

\iusr{Маргарита Мышанская}
Так валите побыстрее из \enquote{ватно-совкового зоопарка}. Вы еще тут? @igg{fbicon.laugh.rolling.floor}{repeat=2} 

\iusr{Alex Aleksandrov}
\textbf{Маргарита Мышанская} с какого такого перепугу?)) я буду вас неучей великовозрастных уму разуму учить, свет правды нести в ваши задурманеные саветскай пропагандой головы))

\iusr{Маргарита Мышанская}
Не обольщайтесь! @igg{fbicon.face.grinning.big.eyes}  На мессию Вы не потянете! @igg{fbicon.beaming.face.smiling.eyes} 

\iusr{Alex Aleksandrov}
\textbf{Маргарита Мышанская} ах бросьте сударыня))

\iusr{Маргарита Мышанская}
бросьте Вы

\iusr{Alex Aleksandrov}
\textbf{Маргарита Мышанская} што бросить?)) и главное куда)

\iusr{Ольга Гринь}
\textbf{Alex Aleksandrov} растерял знаки препинания, господин соврамши. Учитель года.

\iusr{Елена Колобанова}
\textbf{Маргарита Мышанская} , я бы сказала, что Шариков и Швондер скорее неистребимы и неизлечимы.
\end{itemize} % }

\iusr{Александр Шафран}
\textbf{Віктор Маслєєв} Это не писака, а писатель мирового уровня.

\begin{itemize} % {
\iusr{Віктор Маслєєв}
\textbf{Александр Шафран}
А я думаю... пісяка...
\end{itemize} % }

\iusr{Ros Iolin}

Грустно не то, что вот такое \enquote{чудо} высказалoсь, а то, что вы, господа,
на полном серьезе на него реагируете


\iusr{Віктор Маслєєв}
\textbf{Ros Iolin}
Господа...
Капець... зрештою як і пани...

\iusr{Олена Мельникова}

Я в категории бальзаковского возраста. А читать Пушкина и Булгакова будут читать
вечно и в 18, и в 40, и в 80.

\begin{itemize} % {
\iusr{Віктор Маслєєв}
\textbf{Олена Мельникова}
Ну ще Булгаков... а що в такого в Пушкіна... хіба те що російською...
Росіяни читають... типу ...а всерівно думка одна... завоювати і вбити... чи не так... отож...

\iusr{Олена Мельникова}
\textbf{Віктор Маслєєв} 

Вы наверно молоды и выросли на ненависти к русскому языку. Чем же провинился
гений-Пушкин? И тот же Толстой, Чехов, Достоевский, Есенин? Тем что творили на
русском языке? Смею напомнить-Тараса Шевченко выкупили творческие люди говорящие
на русском.

\iusr{Віктор Маслєєв}
\textbf{Олена Мельникова}
80 ще не маю ...

\iusr{Валентина Володимирівна}
\textbf{Віктор Маслєєв} . Певно, 79 рочків і 11 місяців @igg{fbicon.laugh.rolling.floor} 

\iusr{Віктор Маслєєв}
\textbf{Валентина Володимирівна}
Ви ванга... ні нострадамус... мої дані не приховані... у ФБ як у більшості...
\end{itemize} % }

\iusr{Виталина Горбаченко}
\textbf{Віктор Маслєєв} какая чушь!))) По себе судите?))))

\begin{itemize} % {
\iusr{Віктор Маслєєв}
\textbf{Виталина Горбаченко}
Не бачу ваших даних у ФБ.... за 80 точно... Булгаков точно для вас...
\end{itemize} % }

\iusr{Alla Zinovych}
\textbf{Віктор Маслєєв} глупости! \enquote{Рукописи не горят}, как и не иссякает интерес к творчеству их авторов...

\iusr{Alex Aleksandrov}

Не группа а какой то ватно-совковый зоопарк бальзаковского возраста) все,
успокойтесь уже, ваше время прошло красные не вернутся белые тоже)

\begin{itemize} % {
\iusr{Александра Кузьменко Хавчин}
\textbf{Alex Aleksandrov} 

так вот глубокоуважаемый Алекс на нас держитесь вы легко и беспринципно мысля
мыслящие и не думающие вообще, что вы из себя представляет? Пустота в голове


\iusr{Віктор Маслєєв}
\textbf{Alex Aleksandrov}
А бальзаківський вік це лише за 35 років... як пише Бальзак...

\iusr{Alex Aleksandrov}
\textbf{Alexandra Kuzmenko Havchin} очень мощная мысль с точки зрения смысловой нагрузки))

\iusr{Alex Aleksandrov}
\textbf{Віктор Маслєєв} 

совки редко читали О Де Бальзака, им нравился Булгаков бежавший из Киева от
большевиков потом верой и правдой им служивший за брошеную кость. Интересно
раскаялся ли он за предательство своих убеждений хотя бы мысленно?)

\iusr{Ольга Гринь}
\textbf{Alex Aleksandrov} 

видно, Вы в костях разбираетесь @igg{fbicon.face.grinning.squinting}  это очень
заметно, поверьте. Ваши попытки провоцировать незнакомых людей более чем
красноречивы. Согласитесь, странно бухтеть на личность, имя которой будет в
веках, а Вас никто никогда не полюбит.

\iusr{Alex Aleksandrov}
\textbf{Ольга Гринь} 

дамочка вы бы поменьше пафоса выкладывали в свои слова глядишь и звучало бы
убедительнее) а так обычный совковый треп, такая же как вы
\enquote{интеллигенция} вечерами о высоком размышляли в квартирах расстреляных
комунистами хозяев

\iusr{Ольга Гринь}
\textbf{Alex Aleksandrov} 

если бы вы хоть одну книгу прочитали, то, может, из Вас что-то и вышло б. А
так, переписка Энгельса с Каутским @igg{fbicon.face.tears.of.joy}  В топку Вас,
осла безграмотного!

\iusr{Alex Aleksandrov}
\textbf{Ольга Гринь} 

так ведь швондеры это те кто вашим поколением правил, это те чьи портреты вы
счастливые носили 1го мая, те о ком вы сочинения в школе писали, Украина вас от
них освободила, причём уже дважды сначала в 91м потом в 2014 и у вас хватает
совести и ума мне что на эту тему говорить?))

\iusr{Ольга Гринь}
\textbf{Alex Aleksandrov} 

Вы бы уже могли запросто на \enquote{ты} ко мне обращаться
@igg{fbicon.face.grinning.squinting}  А то мне Ваше жлобковатое \enquote{вы} с
маленькой буквы как-то претит @igg{fbicon.face.savoring.food}  Да, мы, совки,
такииие, с достоинством!  Книжки читали, родители нас учили доброму-вечному, и
в первую очередь- уважению к Родине, обществу, культуре. И я так воспитываю
своих деток, в любви и уважении. Но Вам этого не понять, Вы провокатор. Это
Ваша работа. Но Вы даже спровоцировать не можете, интеллект по-пьяни сделанный.

\end{itemize} % }

% -------------------------------------
\ii{fbauth.grinj_olga.kiev.ukraina}
% -------------------------------------

Ей-Богу, как странно кичиться бескультурьем и пытаться высказаться против
личностей такого масштаба, как Булгаков! Как говорится, когда меня нет, можете
меня даже немножечко побить @igg{fbicon.laugh.rolling.floor}{repeat=3} 

На нападки по проводу неактуальности Михаила Афанасьевича отвечу его же
крылатой, неописуемо всеобемлющей фразой: \enquote{А вас что, уже выпустили из
сумасшедшего дома?} @igg{fbicon.face.grinning.squinting}  Ибо пъеса
\enquote{Иван Васильевич} принадлежит этому прекрасному и необъятному автору.
Если кто-то скажет, что Иван Васильевич с Шуриком неактуальны, готова на
кровавое побоище @igg{fbicon.face.savoring.food} 

\begin{itemize} % {
\iusr{Alex Aleksandrov}
\textbf{Ольга Гринь} 

прокомментируйте плс \enquote{хождение по мукам} как так вышло что трилогию начинал
писать монархист а закончил коммунист?)

\iusr{Маргарита Мышанская}
\textbf{Alex Aleksandrov} 

Я не понимаю, к чему Вы везде тычете А. Толстого \enquote{Хождение по мукам}? Вы мне
написали \enquote{бросьте, сударыня}. А я Вам ответила \enquote{Бросьте Вы}

\iusr{Ольга Гринь}
\textbf{Alex Aleksandrov} 

я тоже думала об этом. Поймите, творческие люди, равно как и врачи, очень
нуждаются в признании общества. Да и все мы хотим быть признанными, ибо человек
создан быть в социуме! Каждый хочет, чтобы его любили, а не устраивали травлю в
газетах, заставляя изворачиваться и приспосабливаться. Он же не виноват, что не
Ван Гог или Церетелли. То, что он пишет, читать нужно. Где Вы увидели в пъесе
ИВ меняет профессию коммунизм? Это такое произведение, что не передать словами!
Как бережно Гайдай работал с авторским текстом... \enquote{Что, вас уже выпустили из
сумасшедшего дома?}- где Вы тут коммунизм увидели? А вот многоранность...
Ладно. Оно Вам, наверное, не надо.


\iusr{Alex Aleksandrov}
\textbf{Маргарита Мышанская} 

я с вами исключительно на \enquote{вы}, а вот что и куда бросить не понимаю) хождение
по мукам это как пример нравственной стойкости и железного характера великого
пейсателя и непризнаного при жизни гения)


\iusr{Alex Aleksandrov}
\textbf{Ольга Гринь} 

то есть другими словами Булгаков на поверку оказался обычным приспособлен цем,
верно?) но даже прогнувшись под тех кого он так искренне ненавидел и высмеивал
в начале своего творческого пути кончил жизнь в забвении и полуголодной нищите.
Что скажете?)

\iusr{Ольга Гринь}
\textbf{Alex Aleksandrov} 

в юности все набивают ошибки, за которые потом придется расплатится бумерангом
по башке. Я не считаю для себя возможным осуждать кого-бы то ни было. Если Вы
кого-то осуждаете, значит, Вы чего-либо хотите от человека. Я не знаю, чем Вам
насолил МА. Что-то личное? Не представляю себе, как можно так обижаться,
чего-то выторговывать-выуживать. Я на такую шлое.. нь жизнь не трачу. Я всех
люблю. Чего и Вам желаю. В конце концов, Иисус (надеюсь, к нему у Вас нет
претензий? Например, что он смоковницу проклял или в храме разгон делал, мало
ли) четко дал понять нам, неразумным, какой булыган прилетит за осуждение @igg{fbicon.face.squinting.tongue} 

\iusr{Alex Aleksandrov}
\textbf{Ольга Гринь} 

я перечитал его практически полностью вместе с автобиографией, разочаровался в
нем как в человеке. Убеждения предал, украинцев не любил и считал вторым
сортом, шовенизмом насквозь пропитана белая гвардия, которую он так восхвалял и
продал за кусок хлеба. За что мне его любить или уважать?


\iusr{Alex Aleksandrov}
Бесхребетный приспособленец ваш М,А

\iusr{Ольга Гринь}
\textbf{Alex Aleksandrov} но дайте ему самому за это поплатится, или я переписываюсь с Господом Богом? @igg{fbicon.face.grinning.squinting} 

\iusr{Ольга Гринь}
\textbf{Alex Aleksandrov} смею Вас уверить, что Вам не удастся меня спровоцировать. Никак. У меня псизика стабильная, это Вам любой подтвердит. Откуда мне знать, может МА как раз и вытворял все это, ибо предвидел Ваше появление на свет и такие нападки на его память.

\iusr{Alex Aleksandrov}
\textbf{Ольга Гринь} между нашим с ним появлением на свет 100 лет и первым родился не я а он, улавливаете логику или сложно?)

\iusr{Alex Aleksandrov}
\textbf{Ольга Гринь} 

а вот то что вы согласны с тем что он именно \enquote{вытворял} и с тем что он
действительно не любил украинцев это уже какой никакой а прогресс) всего час
убеждений и картонная стена пала в не равном бою))


\iusr{Ольга Гринь}
\textbf{Alex Aleksandrov} 

удивительное многообъясняющее слово \enquote{шовЕнизм} в Вашей экстравагантной
интерпритации упоминается в здешних комментах уже дважды. Это какой-то тупой
десант высадился, что -ли @igg{fbicon.face.grinning.squinting} 

\iusr{Ольга Гринь}
\textbf{Alex Aleksandrov} 

я никуда падать не собираюсь. И ничего не признаю. Борюсь с живыми подонками с
открытым забралом. Чего цеплятся к классикам, имя которых будут помнить
всегда @igg{fbicon.face.grinning.squinting}  как у Гашека: \enquote{Сойка останется сойкой, хоть Ваш редактор обосрись}.
Цитата.


\iusr{Alex Aleksandrov}
\textbf{Ольга Гринь} 

да этот тупой десант в вашем лице высадился на эту бренную землю уже лет 50
как)) сначала красным пол жизни оды пели, переписку энгельса с каутским взахлеб
читали, тряпки красные носили коммунизм строили потом страну свою просрали
теперь колбасу по 2-20 и булгакова вспоминаете))

\iusr{Dan Rozental}
\textbf{Alex Aleksandrov} Это вы что женщине на возраст намекаете? Должен вам заметить, все там будем! А Рабами людей и сегодня можно сделать, это, по историческим меркам, недолгий процесс, и не только в советском Союзе

\iusr{Ольга Гринь}
\textbf{Alex Aleksandrov} да, мы такие! И какой же нам приговор? @igg{fbicon.face.grinning.squinting} 

\iusr{Ольга Гринь}
\textbf{Alex Aleksandrov} да, так и есть, гыыыы. А это как-то преследуется по закону?
Это Вам, дурачек Вы наш золотой. Скрасил мое время.

\ifcmt
  ig https://scontent-frt3-2.xx.fbcdn.net/v/t1.6435-9/159416972_2840150666196582_3557846873159333762_n.jpg?_nc_cat=101&ccb=1-5&_nc_sid=dbeb18&_nc_ohc=IwT1e4C0O3UAX9JZphu&_nc_ht=scontent-frt3-2.xx&oh=00_AT_bRiNOPIbCECzpQOuV5hz_liKHp7LEzAe0GDAChnvv2A&oe=61DE67C5
  @width 0.4
\fi

\end{itemize} % }

\iusr{Виталий Маслов}
чего спорить с ботом, который даже на украинском языке пишет с ошибками непростительными?

\iusr{Элеонора Шпикула}
\textbf{Віктор Маслєєв} мне жаль вас и ваше окружение. Вы не поняли, но и понять уже, наверно, поздно....

\begin{itemize} % {
\iusr{Віктор Маслєєв}
\textbf{Элеонора Шпикула}

Так... не зрозумію... піздно... та й все це для примітивних мізків... початку 20-
но століття... дехто застряг нам навічно... думаю ві поняли о чьом
я... господа... тфу... панове...


\iusr{Виталина Горбаченко}
\textbf{Віктор Маслєєв} оно и \enquote{вiдно о чьом вi})))))
\end{itemize} % }

\iusr{Stela Kushnir}
\textbf{Віктор Маслєєв} что вы несёте?

\iusr{Евгения Бобылева}
\textbf{Віктор Маслєєв} если Вы не понимаете, значит Вы не доросли, а не многие дураки

\iusr{Наталия Удовченко}
\textbf{Віктор Маслєєв} не пишіть дурниць.

\end{itemize} % }

\iusr{Маргарита Мышанская}
Ваше мракобесие даже уже не удивляет...

\iusr{Кирило Анатолїйович Козлов}
гениальный писатель. любимый. неподражаемый.

\begin{itemize} % {
\iusr{Віктор Маслєєв}
\textbf{Кирило Анатолїйович Козлов}
Там не має чому подражать... один нафталін... погодьтеся... ну може що з І Бабелем порівняти...

\begin{itemize} % {
\iusr{Кирило Анатолїйович Козлов}
\textbf{Віктор Маслєєв} кожному своє.

\iusr{виктория швейцер}
\textbf{Віктор Маслєєв} відносно людських рис обидва були дерьмо. А писменники хороші.

\iusr{Ольга Гринь}
\textbf{Віктор Маслєєв} фильм \enquote{ИВ меняет профессию} Вам нравится? Это тоже Михаил Афанасьевич! \enquote{Живьем брать демонов!}(ну, Вы поняли). Я не верю, что такое можно писать, ужас.

\iusr{Кирило Анатолїйович Козлов}
\textbf{Ольга Гринь} это перефраз. который сразу не поймешь правильно.

\iusr{Ольга Гринь}
\textbf{виктория швейцер} ого... а Иисус Господь презрел Марию Магдалину! Не забывайте про камень! Никогда!

\iusr{Віктор Маслєєв}
\textbf{Viktoria Chveitsher}
Он як...

\iusr{Ольга Гринь}
\textbf{Віктор Маслєєв} 

ну не з Вами ж рівняти, погодьтеся. А можна поцікавитися, Швейк Гашека Вам теж
не подобається? Я щиро вважаю, що в книгах Булгакова, Бабеля, Гашека, Гоголя,
Салтикова-Щедріна, Лєскова є відповіді на всі життєві перипетії. Та і Тараса
Григоровича важко назвати життєрадісним веселуном. Івана Франка теж. І Лесю
Українку... А ось Ільф-Петров - це на злобу дня! І Зощенко... та взагалі, де б
я була без цієї чудової безодні @igg{fbicon.grin}  а от Рей Бредбері не зайшов, у юності двічі
розпочинала. Котлован Платонова, ГУЛАГ Солженіцина... перші публікації, як
ковток повітря. Книга йде з людиною по життю рука об руку, бо це - віковічна
мудрість.

\end{itemize} % }

\end{itemize} % }

\iusr{Andrew Rudyakov}
Напевно

\iusr{Ванда Бандура}

Не люблю і не розумію. Декілька разів перечитувала \enquote{Мастер и Маргарита},
дивилася в театрі спектакль по цьому роману але нічого навіть не залишилося в
пам'яті. Якесь відторгнення. Хоча \enquote{Собачье сердце} дійсно варте уваги і
захоплення.

\begin{itemize} % {
\iusr{Thomas Fairmark Mykola}
\textbf{Ванда Бандура} Майстер - це такий собі сюр... Щоб його зрозуміти треба не абияка підготовка. Наприклад перечитати Біблію

\iusr{Валентин Майборода}
\textbf{Thomas Fairmark Mykola} Или принять на грудь пару литров)))

\iusr{Татьяна Данилова}
\textbf{Ванда Бандура} Цей поман не для масового читача, не читайте, не псуйте свій час!

\begin{itemize} % {
\iusr{Віктор Маслєєв}
\textbf{Татьяна Данилова}
Поман точно не цікавий ...вже...
Як балет... чи опера... охають... аха
ють ...а пояснити ні собі... ні внукам не можуть...

\iusr{Ольга Гринь}
\textbf{Татьяна Данилова} смішно, у Булгакова такий творчий наслідок, а всі намагаються ММ зрозуміти. Щось полегше ж можна. Про лікаря, наприклад. Життя без прикрас.
\end{itemize} % }

\iusr{Віктор Маслєєв}
\textbf{Ванда Бандура}
Собачаче серце ще куди не йшло...

\iusr{Светлана Гончаренко}
Ну такое... На вкус и цвет товарищей нет.

\iusr{Віктор Маслєєв}
\textbf{Светлана Гончаренко}
Це точно... їсти краще самому...

\iusr{Ольга Гринь}
\textbf{Ванда Бандура} а Иван Васильевич, который меняет профессию? Вы в себе вообще? Да Вы не читали даже элементраного, куда Вам до ММ.

\begin{itemize} % {
\iusr{Валентин Майборода}
\textbf{Ольга Гринь} 

Вы не правы... Кому-то нравится шансон, кому то классическая музыка... Ну или
попроще - кому-то мандарины, а кому-то нравятся ящики из под мандарин.

\iusr{Ольга Гринь}
\textbf{Валентин Майборода} 

я просто спрашиваю: есть кто-либо, кому Иван Васильевич не нравится?!?!
\enquote{Ей-Богу, я никому не скажу!}(цитата).

\iusr{Ванда Бандура}
\textbf{Ольга Гринь}, ну да куда ж мне! Я не голубых кровей, как вы! Читала и не таких авторов, которых вы и не слышали, не то что читали.

\iusr{Ольга Гринь}
\textbf{Ванда Бандура} действительно, странно, что со мной пытается полемизировать тот, кто не знает, что \enquote{В} при обращении к человеку в единичном количестве следует писать с большой буквы @igg{fbicon.face.grinning.squinting}  @igg{fbicon.face.kissing}{repeat=2}  Или Вы пытаетесь мне этим досадить? Напрасно, уверяю Вас. Чтение не пошло Вам на пользу, больше этим не занимайтесь @igg{fbicon.face.blowing.kiss}{repeat=3} 

\iusr{Ванда Бандура}
\textbf{Ольга Гринь}, вы с большой буквы если с уважением, а так вы пишите бред!

\iusr{Виктория Данькова}
\textbf{Ольга Гринь} хотела бы заметить, что правилами предусматривается обращение на вы с большой буквы только в деловой переписке, а иначе, выходит, что и к близкой подруге в письме нужно обращаться с большой буквы( ну ведь любишь её и уважаешь)

\iusr{Ольга Гринь}
\textbf{Ванда Бандура} поверте, Ваше суждение, милостивый государь, трогает меня именно так же, как прошлогодний снег @igg{fbicon.face.blowing.kiss}{repeat=3}  дурашечка, пишите, все же, \enquote{Вы}! Ненавистно, злобно, но пишите! Не нужно меня вообще уважать. Можете меня даже немножечко побить @igg{fbicon.heart.eyes}{repeat=3} 

\iusr{Ольга Гринь}
\textbf{Виктория Данькова} 

странная подруга, к которой Вы обращаетесь на Вы. Причем близкая. Логика
железная @igg{fbicon.face.grinning.squinting}  и опять же, с чего Вы взяли, что
я тут над Вами прикалываюсь не по делу?!?! Просто все какие-то отмороженные,
обиженные боты с кактусом на странице @igg{fbicon.face.grinning.squinting}
получается, что мы тут с кактусом литературный диспут устраиваем
@igg{fbicon.face.grinning.squinting} ну а я, как всегда занимаюсь самопиаром и
самолюбованием, ибо я как умница, так и красавица, это знают все.

\end{itemize} % }

\iusr{Александра Кузьменко Хавчин}
\textbf{Ванда Бандура} 

уважаемая, у читала в молодости Мастера ничего не поняла, потом читала в
зрелости. Дошло и перечитываю до сих пор, желаю и вам того же. Кстати по
театральной постановку судить о таком произведении бессмысленно, не зная хорошо
произведение.


\iusr{Валентин Майборода}
\textbf{Ольга Гринь} Всякое бывает)) Может и так бывает)))Кому то Моцарт не не нравится)))

\iusr{Ольга Гринь}
\textbf{Валентин Майборода} 

я училась скрипичному мастерству @igg{fbicon.face.grinning.squinting}  Моцарт и \enquote{не нравится} не укладываются в
голове @igg{fbicon.laugh.rolling.floor}  для разнообразия можно, чего уж там @igg{fbicon.face.smiling.sunglasses} 


\iusr{Валентин Майборода}
\textbf{Ольга Гринь} Можно... \enquote{Лесоповал} или типа того))

\iusr{Magdalena Helga}
\textbf{Ванда Бандура} 

Фильм посмотрели, многосерийный, с невероятной музыкой Корнелюка, с красивейшей
и уникальной Анной Ковальчук, великим артистом, которого Россия оказалась
недостойна- Владом Галкиным? Взгляните, там много хороших попаданий. Может, это
переменит Ваше мнение о Романе.

\begin{itemize} % {
\iusr{Ванда Бандура}
\textbf{Magdalena Helga}, спасибо. Один адекватный ответ. Набросились, как шакалы. Каждый мнит себя исключительным знатоком литературы.

\iusr{Magdalena Helga}
\textbf{Ванда Бандура} то, правда, очень красивый фильм. Начните смотреть с нуля, как будто Вы ничего не читали. Там созвездие потрясающих артистов, некоторых уже нет, чего стоит Пилат - Кирилл Лавров! Абдулов хорош, Филипенко чудесен. А музыка такая, что захлебнуться. Вам понравится. @igg{fbicon.heart.red} с Богом! Хочу, чтоб понравился.
\end{itemize} % }

\iusr{Игорь Белинский}
\textbf{Ванда Бандура} якщо декілька разів перечитували, значить сподобалось @igg{fbicon.laugh.rolling.floor} ))
\end{itemize} % }

\iusr{Юлия Чернышева}
Обожаю Булгакова- от Собачьего сердца до Записок на манжетах

\iusr{Татьяна Данилова}

\ifcmt
  ig https://scontent-frx5-2.xx.fbcdn.net/v/t39.1997-6/s168x128/93118771_222645645734606_1705715084438798336_n.png?_nc_cat=1&ccb=1-5&_nc_sid=ac3552&_nc_ohc=lc93vluqWdkAX_GnZio&tn=lCYVFeHcTIAFcAzi&_nc_ht=scontent-frx5-2.xx&oh=00_AT9c-ypoPq91xHH5i95HySbT7i7gzZKtbmk4N226grV0tQ&oe=61BED6FF
  @width 0.1
\fi

\iusr{Alexey Novozhylov}

Нічого не маю проти людей, що захоплюються Булгаковим, але мені видається, що
якщо людина володіє 1-2 основними світовими мовами, то вона врядчи помітить
Булгакова і його творчість, просто часу не вистачить, тобто що називається на
бєзрибьї. Хоча мушу визнати що М і М в перекладі Ларисса Валохонськи та Річард
Півір мені сподобався.  @igg{fbicon.smile} 

\begin{itemize} % {
\iusr{Віктор Маслєєв}
\textbf{Alexey Novozhylov}
Браво... не зрозуміло... але круто ...я в захваті... дві мови звичайно скромненько... китайська думаю одна з них...

\iusr{Alexey Novozhylov}
\textbf{Віктор Маслєєв} браво

\iusr{Ольга Гринь}
\textbf{Alexey Novozhylov} а ИВ меняет профессию? Зачем такие сложности, это тоже Булгаков.

\iusr{Alexey Novozhylov}
\textbf{Ольга Гринь} забєйтє

\iusr{Anhela Gomez}
\textbf{Alexey Novozhylov} Господи не позорьтесь если слабоумием страдаете. Книга переведена более чем на 40 языков и считается величайшим произведением в литературе.

\iusr{Alexey Novozhylov}
\textbf{Anhela Gomez} розмовляти навчись спочатку
\end{itemize} % }

\iusr{Алла Поповская}
Обожаю, Белую гвардию, с героем бегу по улицам Малопровальной, Александровской и книга, и фильм прекрасны!

\iusr{Андрей Надиевец}
Дом на спуске Андреевском, ну где доска, кто в нем жил???

\iusr{Борис Лушельский}

\ifcmt
  ig https://scontent-frx5-2.xx.fbcdn.net/v/t39.1997-6/s168x128/93118771_222645645734606_1705715084438798336_n.png?_nc_cat=1&ccb=1-5&_nc_sid=ac3552&_nc_ohc=lc93vluqWdkAX_GnZio&tn=lCYVFeHcTIAFcAzi&_nc_ht=scontent-frx5-2.xx&oh=00_AT9c-ypoPq91xHH5i95HySbT7i7gzZKtbmk4N226grV0tQ&oe=61BED6FF
  @width 0.1
\fi

\iusr{Татьяна Чута}

\ifcmt
  ig https://scontent-frx5-2.xx.fbcdn.net/v/t39.1997-6/s168x128/16781161_1341101952618574_7704631035023065088_n.png?_nc_cat=1&ccb=1-5&_nc_sid=ac3552&_nc_ohc=iYQJ7jtF9aMAX_4UBUn&_nc_ht=scontent-frx5-2.xx&oh=00_AT8aC1Upd0z-QRsFj3qy5Tv7qV3Fv3YEdQWOVoHfSJDFwg&oe=61BDF3AB
  @width 0.1
\fi

\iusr{Naty Lukina}
Спасибо за пост...  @igg{fbicon.rose} @igg{fbicon.heart.red} @igg{fbicon.hands.pray} 

\iusr{Lara Klimuk}
Произведения Булгакова не могут устареть!!!

\iusr{Lidiya Novynska}

Сами таким скоро станете и будете ностальгировать по.... пейждерам или любимой
ВАМИ книжкой-картиной ...

\iusr{Юрий Панчук}

Наверняка те кто разминает тут свое красноречие не читал ни Мастера ни Белую
Гвардию

\begin{itemize} % {
\iusr{Ольга Гуменюк}
\textbf{Юрий Панчук}
А что? Сплошь Швондеры и Шариковы....
Они не читают. Единственная книга их жизни - Букварь или Читанка. И то, судя по грамотности многих, и это не осилили...

\begin{itemize} % {
\iusr{Дмитрий Лиховид}
\textbf{Olga Gumeniyk} Каждый постсоветский интеллигент мнит себя непременно преображенским, на меньшее он не согласен.

\iusr{Юрий Панчук}
\textbf{Дмитрий Лиховид} Кстати положительность профессора тоже под вопросом, некоторые критики считают, что он возомнил себя Творцом и получил Шарикова

\iusr{Дмитрий Лиховид}
\textbf{Yury Panchuk} Так далеко постсоветские интеллигенты не заходят, для них глубокомысленная фраза
\enquote{что будет с паровым отоплением} или про разруху - это венец всей мировой культуры.

\iusr{Ольга Гуменюк}
Я понимаю, что это сарказм. Но лучше мнить себя потомком Преображенского, чем быть похожим на Швондера или еще хуже, на Шарикова....

\iusr{Дмитрий Лиховид}
\textbf{Olga Gumeniyk} Лучше всего не пытаться сводить реальность к вымышленным образам. Хотя если это форма эскапизма, то я даже понимаю.
\end{itemize} % }

\end{itemize} % }


\iusr{Людмила Синицкая}
Сколько же людей со сдвинутыми мозгами. Вылезли из под плинтусов.

\iusr{Irina Kaminsky}
\textbf{Алла Поповская} и я всегда с Вами

\iusr{Ольга Балаба}

\ifcmt
  ig https://scontent-frx5-2.xx.fbcdn.net/v/t39.1997-6/s168x128/16781161_1341101952618574_7704631035023065088_n.png?_nc_cat=1&ccb=1-5&_nc_sid=ac3552&_nc_ohc=iYQJ7jtF9aMAX_4UBUn&_nc_ht=scontent-frx5-2.xx&oh=00_AT8aC1Upd0z-QRsFj3qy5Tv7qV3Fv3YEdQWOVoHfSJDFwg&oe=61BDF3AB
  @width 0.1
\fi

\iusr{Элеонора Шпикула}
Вечный и Великий Мастер Булгаков. Единственный автор, произведения которого цитируем и перечитываем много лет всей семьёй.

\iusr{Marina Tsvetaeva}
Первая мысль, как хорошо, что до ужасных событий ...

\iusr{Дмитрий Артюк}
На сьогодні самий актуальний письменник.

\iusr{Віктор Маслєєв}
Все ...більше не коментую ... відпочивайте..
Якби не Гайдай... про Іоанна Васільєвича ніхто б і не згадав...

\iusr{Mila Sirikova}
\textbf{Віктор Маслєєв} по себе судите?

\iusr{Галина Горичева}

\ifcmt
  ig https://scontent-frt3-1.xx.fbcdn.net/v/t1.6435-9/159619146_490007895339558_5274284681149382232_n.jpg?_nc_cat=108&ccb=1-5&_nc_sid=dbeb18&_nc_ohc=GCDu0aKcOUMAX9iFlx6&_nc_ht=scontent-frt3-1.xx&oh=00_AT-VSyI6Pv-_c6td8c_20M4K1RxJon3Xm4ICySsDs5YfxA&oe=61E0C35E
  @width 0.4
\fi


\iusr{Руслан Трушков}
Мастер на Века! Ибо, рукописи - не горят!!!)

\iusr{Ирина Фролова}

\ifcmt
  ig https://scontent-frx5-2.xx.fbcdn.net/v/t39.1997-6/s168x128/16781161_1341101952618574_7704631035023065088_n.png?_nc_cat=1&ccb=1-5&_nc_sid=ac3552&_nc_ohc=iYQJ7jtF9aMAX_4UBUn&_nc_ht=scontent-frx5-2.xx&oh=00_AT8aC1Upd0z-QRsFj3qy5Tv7qV3Fv3YEdQWOVoHfSJDFwg&oe=61BDF3AB
  @width 0.1
\fi


\iusr{Світлана Проценко}

Майстер не вмирає, він завжди Є, він і Андріївський Узвіз Завжди Живий

\ifcmt
  ig https://i2.paste.pics/153315ca841cb86710f34e7597d3ca1e.png
  @width 0.2
\fi

\iusr{Anhela Gomez}

По сообщению «Deadline» в США началась работа над экранизацией культового
романа Михаила Булгакова «Мастер и Маргарита». «Булгаков Фонд» владеющий
правами на роман, неоднократно позволял адаптировать произведение для кино,
телевидения и даже для анимации, но большинство из них было сделано для
зарубежных аудиторий. Роман был переведен на более чем 40 языков и считается
одним из величайших произведений мировой литературы. Голливуд собирается снять
полнометражный фильм.

Продюсерами проект выступят российский продюсер Светлана Мигунова-Дали,
работавшая над фильмом «Легенда №17», и бывший президент кинокомпании «New
Crime Productions» и продюсер фильма «Машина времени в джакузи» Грэйс Ло.
Исполнительными продюсерами выступят – создатель музыкального фестиваля Вудсток
Майкл Ланг, а также Невин Шалит и Андрей Фурманчук. Но кто претендует на роли в
фильме, пока не оглашалось.

Михаил Булгаков писал роман с 1928 по 1940 год, но произведение не было
опубликовано вплоть до 1967 года. Впервые роман был экранизирован польским
режиссером Анджеем Вайдой в 1971 году и телекартина получила название «Пилат и
другие – Фильм на Страстную пятницу». В 1972 году французским режиссером и
сценаристом Александром Петровичем по роману был снят одноименный фильм. В
1980-х годах кинокартину попытался снять известный французский режиссер Роман
Полански, но проект закрыли. В 1988 году по роману снял одноименный мини-сериал
польский режиссер и сценарист Мацей Войтышко. В 1994 году режиссер и сценарист
Юрий Кара закончил первую российскую экранизацию, но фильм долго не выходил на
экраны из-за разногласий постановщика и продюсера. В 2005 году свою версию
романа представил Владимир Бортко.

Напомним, что в конце прошлого года британскими студентами были сняты короткие
мультфильмы по мотивам русских классических произведений, и видео по роману
«Мастер и Маргарита» заняло одно из призовых мест.

\begin{itemize} % {
\iusr{Руслан Трушков}
\textbf{Anhela Gomez} и, ещё...лет 10 так назад, забыл какой голливудский режиссёр собрался было снимать киноэкранизацию по этому роману, были выбраны даже актёры на главные роли (Марго - Кетрин Зета-Джонс, и, Воланд - Аль Пачино)...но у Кетрин в то время болел серьезно муж - Дуглас - раком, и она сказала, что пока не вылечу своего мужа - сниматься не буду...так всё и ушло в небытие, к сожалению...но актёры были подобраны просто идеально)
\end{itemize} % }

\iusr{Таміла Трітенко}
любимая книга

\iusr{Лина Агаркова}
Помним, любим, будем читать и перечитывать. А все остальное - пыль

\iusr{Валентина Дорошенко}

\ifcmt
  ig@ name=scr.hands.applause
  @width 0.2
\fi

\iusr{Nina Vakula}

Герой знаменитого романа М. Булгакова \enquote{Мастер} так описывает свою первую
встречу с Маргаритой:

\enquote{Она несла в руках отвратительные, тревожные желтые цветы... Черт их знает, как
их зовут, но они первые почему-то появляются в Москве. И эти цветы очень
отчетливо выделялись на черном ее весеннем пальто.}

(В руках Маргариты были Мимозы.)

\ifcmt
  ig https://scontent-frx5-1.xx.fbcdn.net/v/t1.6435-9/159380005_833466607237696_8011300194395096875_n.jpg?_nc_cat=111&ccb=1-5&_nc_sid=dbeb18&_nc_ohc=Ea72E1Kf7lsAX9lYwy-&_nc_ht=scontent-frx5-1.xx&oh=00_AT-vuAlKxMSAFg_duzZePKmKOn5QeC3Pe-XB1ZlUWBMqRQ&oe=61DE9C4A
  @width 0.4
\fi

\iusr{Elena Sivolapova}

\ifcmt
  ig https://scontent-frx5-1.xx.fbcdn.net/v/t1.6435-9/159085319_270224537812893_1124141479753119560_n.jpg?_nc_cat=100&ccb=1-5&_nc_sid=dbeb18&_nc_ohc=No41hqVhHQcAX_bs57J&_nc_ht=scontent-frx5-1.xx&oh=00_AT84U1cUGJNSEvVYkDwvLmtrRtudVHMsTKFuPfuQm3ty1g&oe=61DE709A
  @width 0.4
\fi

\iusr{Igor Popell}
Как в анекдоте просто!
\enquote{Хождение по мукам} А. Толстой написал, а памятник М. Булгакову. Ничего не понимаю!  @igg{fbicon.laugh.rolling.floor} 

\ifcmt
  ig https://scontent-frx5-1.xx.fbcdn.net/v/t1.6435-9/159363759_10159030343726136_392239406630469691_n.jpg?_nc_cat=111&ccb=1-5&_nc_sid=dbeb18&_nc_ohc=TCI4q3EMnDQAX_EDez-&_nc_ht=scontent-frx5-1.xx&oh=00_AT_Tzu7s-if9dF6-Phdg4cGsfc1ZuD3iJIVi4SWduCHvpA&oe=61E10F5A
  @width 0.4
\fi

\iusr{Валерий Ищенко}
Булгаков таки вечен.

\enquote{ -... так кто ж ты, наконец? - Я - часть той силы, что вечно хочет зла, и вечно совершает благо.}

\iusr{Роман Воловенко}

почитайте о нем воспоминания его ки евской хозяйки квартиры. и как он обгадил в
своем опусе БЕЛАЯ ГВАРДИЯ почтенных киевлян У КОт ОРЫХ ЕГО СЕМЬЯ сн мала кварт
иру. Морфий в руках врача может сделат ь его писателем

\begin{itemize} % {
\iusr{Dan Rozental}
\textbf{Роман Воловенко} 

Ну допустим что сами владельцы подольских доходных домов скорее всего не были
ангелами в перьях. Во вторых бытовуха никогда и никого не украшала. И начиная
от героев до Гениев и заканчивая последним обывателем, все надо заканчивать как
в сказке « поженились и жили долго и счастливо», потому-то это и сказка, а
иначе и Золушка-стерва-жена. Ну и последняя наркоманов и алкоголиков полно
среди нас, даже великий, я слабину даю порой. Но не все становится талантливыми
музыкантами писателями и танцорами. Давайте же порадуемся люди что они есть
среди нас.


\iusr{Ольга Морозова}
\textbf{Роман Воловенко} В данном случае нужно отделять творчество от личной жизни. Ваша жизнь тоже не образцовая. Не все наркоманы становятся писателями.

\begin{itemize} % {
\iusr{Роман Воловенко}
\textbf{Ольга Морозова} и плагаторами и сталинскими подхалимами

\iusr{Ольга Морозова}
\textbf{Роман Воловенко} Докажите в суде \enquote{плагат}.

\iusr{Роман Воловенко}
\textbf{Ольга Морозова} доказать что Пушкин передрал Белоснежку и семь гномов а Волков передрал Волшебника Изумрудного города из Мудреца из страны ОЗ а передрать Буратино из Пиноккио а Крылову дедушке передрать басни Эзопа так что Мишка- морфинист не первый

\iusr{Ольга Морозова}
\textbf{Роман Воловенко} А вы и передрать не способны. Завидуете.

\iusr{Роман Воловенко}
\textbf{Ольга Морозова} страсть люблю передирать все но только для себя
\end{itemize} % }

\iusr{Татьяна Кот}
\textbf{Роман Воловенко} Рома, стареем. Сколько желчи в каждом слове.
\end{itemize} % }


\iusr{Тамара Кинчая}
Одно слово - Мастер!

\iusr{Валерій Макаренко}
Творчість геніїв від Бога, все інше- від людей!...

\iusr{Ирина Галегова}

\ifcmt
  ig https://scontent-frx5-2.xx.fbcdn.net/v/t39.1997-6/s168x128/16781161_1341101952618574_7704631035023065088_n.png?_nc_cat=1&ccb=1-5&_nc_sid=ac3552&_nc_ohc=iYQJ7jtF9aMAX_4UBUn&_nc_ht=scontent-frx5-2.xx&oh=00_AT8aC1Upd0z-QRsFj3qy5Tv7qV3Fv3YEdQWOVoHfSJDFwg&oe=61BDF3AB
  @width 0.1
\fi

\iusr{Светлана Токарева}

Мир делится на 2 части - кому-то нравится, а кто-то считает роман \enquote{Мастер и
Маргарита} бредом умирающего наркомана... Я больше люблю его ранние рассказы.
Там СОВЕРШЕННО другой стиль....

\begin{itemize} % {
\iusr{Михаил Фролов}

Для того чтобы понять о чем идёт речь в \enquote{Мастере и Маргарите}, нужно не
бездумно пробегать глазами текст, а задумываться над каждым словом текста.
Каждое его слово весит тонну.

\iusr{Светлана Токарева}
\textbf{Михаил Фролов} Я редактор... И как ЧИТАТЬ тексты меня учить не надо!

\iusr{Михаил Фролов}

А зачем мне вас учить читать тексты. Для того, чтобы ПОНИМАТЬ написанное, не
обязательно быть редактором. Если бы вы понимали то что читаете, уверен,
никогда бы не написали фразы \enquote{бред умирающего наркомана}. Хотя это только
подтверждает, что вы даже с биографией Михаила Афанасьевича Булгакова не
знакомы.


\iusr{Светлана Токарева}
\textbf{Михаил Фролов} Мир делится на 2 части - кому-то нравится, а кто-то считает роман \enquote{Мастер и Маргарита} бредом умирающего наркомана... Я больше люблю его ранние рассказы. Там СОВЕРШЕННО другой стиль....

\iusr{Светлана Токарева}
\textbf{Михаил Фролов} Если плохо читаете, читайте ещё раз. И вас и меня учить уже поздно. Учите внуков...
\end{itemize} % }

\iusr{Татьяна Максюк}
Была сдесь не раз. Андреевский спуск. любимый подол, здесь все одна мистика.

\iusr{Igor Pachev}

Воланд воздал и прибрал  @igg{fbicon.grin}  @igg{fbicon.face.wink.tongue}  не
надо флудить. Любил Миша только сралина. Кто читал мишину пьеску
\enquote{батум}

\iusr{Igor Pachev}

Зачем нам этот украинофоб, плагиатор, пресмыкавшийся прерд сралиным? Миша
наркоман венеролог - ненавидел Украину, её язык, культуру. Был обычным русским
нацистом.

\begin{itemize} % {
\iusr{Ольга Морозова}
\textbf{Igor Pachev} Опишите свою любовь к Украине и перечислите дела, которые сделают вас знаменитым, как Булгаков.

\iusr{Люся Киевская}
\textbf{Igor Pachev} откуда у вас эта фамильярность ?! И вы не слишком ли много себе позволяете ?! У вас нет никакого морального права вторгаться в личную / интимную жизнь любого человека, тем более известного писателя ! Держите свои умозаключения и инсинуации при себе ! Не стоит вываливать на публику чужое « грязное белье» ., следите лучше за своим ...

\iusr{Irina Somova}
\textbf{Igor Pachev} это вы о себе?

\iusr{Igor Pachev}
Морозова. \enquote{Знаменитость} Миши венеролога построенная на 100\% плагиате.
А тебе плевать, что он ненавидел ураицев, наш язык и культуру

\begin{itemize} % {
\iusr{Ольга Морозова}
\textbf{Igor Pachev} Виктор Некрасов, который любил Украину и сделал для для нее столько, что вам не осознать, любил Булгакова. Что значит ваше ничтожное мнение против его мнения? Да о вас никто не знает и никто не вспомнит.
\end{itemize} % }

\iusr{Igor Pachev}

Люся. Это не белье. Булгаков русский фашист, нацик, украинофоб. Его следа не
должно быть в Украине. Это уже вопрос морали.

\begin{itemize} % {
\iusr{Люся Киевская}
\textbf{Igor Pachev} « вопрос морали « - это то, что вы пишете здесь ! Если писатель не может ответить вам сам, и подать иск в суд - значит можно его « обвалять в грязи « ?! А за украинцев не беспокойтесь - у них более, чем достаточно , и чести , и достоинства !! ...
\end{itemize} % }

\iusr{Igor Pachev}
Сомова это я о вас малороссах. Честь надо иметь с рождения

\iusr{Alexey Novozhylov}
\textbf{Igor Pachev} не позбавляйте людей фєтішу!  @igg{fbicon.face.tears.of.joy} 

\iusr{Владимир Козуб}
\textbf{Igor Pachev} Вам к доктору, батенька. Уверен, что он запретит вам телевизор.

\iusr{Igor Pachev}
Фетиш мазохистов. Им нравится читать, что булгаков считал их второсортными

\iusr{Igor Pachev}
Вова. Ты сначала вернись, а потом разводи тут сепарские щи. Телевидение? Да я его отчасти содержу.

\iusr{Igor Pachev}

Люся - у тебя факел горелой ваты за спиной. Не рвись так. Культурный диспут
предполагает аргументы. У ваших латентных сепаров их нет в пользу булгакоаа.
Все что вы можете - перейти на личности  @igg{fbicon.beaming.face.smiling.eyes}   @igg{fbicon.face.smiling.eyes.smiling}   @igg{fbicon.face.smiling.eyes.smiling} 


\iusr{Igor Pachev}

Морозова. А Гитлер любил Вагнера. И что будем делать с Вагнером. Некрасов
вытащил из забытия Мишу венеролога как Киевляниин Киевлянииа. Сам в Париже
охренел когда понял кого вытащил.

\iusr{Ольга Морозова}
\textbf{Igor Pachev} Игорь , вам не нравится, что Булгаков талантливо описал таких, как вы - шариковы.

\iusr{Igor Pachev}

Шариков был мАцквабадским алкашем климом чугункиным. Меня оскорбляет отношение
Миши венеролога к Украине, украинскому языку, культуре. Судя по тому, что тебя
это восхищает - тебя скоро будут ждать товарные вагоны на лугандон


\iusr{Михайло Наместник}

Пане Ігорю, складається враження, що у вас до Булгакова більше претензій не як
до письменника, а як до венеролога. Це щось особисте?

\end{itemize} % }

\iusr{Igor Pachev}

Почему француз должен восхищаться Булгаковым, если много образов и сюжетных
ходов тот позаимствовал из романа Пьера Мак Орлан \enquote{Ночная Маргарита}, изданного
в Москве в 1927? Главные герои здесь профессор Георг Фауст, продавший душу
дьяволу (таинственном Леону, который, конечно, хромает) и благодаря этому
превратился в молодого человека, и рыжая красавица Маргарита. Придет в голову
французу и роман Александра Дюма \enquote{Жозеф Бальзамо}, - считает писатель.

\enquote{Американец, читая Булгакова, сразу вспомнит \enquote{Таинственного
незнакомца} (1898) Марка Твена, особенно бал и общие философские идеи. Немецкий
читатель заметит множество реминисценций из романа Густава Майринка
\enquote{Ангел Западного окна}, а кто-то еще начитаниший будет просто ошарашен
удивительными совпадениями с \enquote{Приключениям авантюриста Гуго фон
Габенихта} классика венгерской литературы Мора Йокаи (1825-1904). Здесь есть и
теологические дискуссии, похожие на те, что велись на Патриарших, и версия о
том, что Иисус был мистификатором, а настоящее его имя Йошуа Бен Ганоцри, здесь
и бал у Сатаны, и отрезанные ради развлечения головы, и исчезающие деньги, и
женщина на кабане, и полеты ведьм},
-

\begin{itemize} % {
\iusr{Тарас Барабанчик}
\textbf{Igor Pachev} Можно добавить ещё \enquote{Блистающий мир} А. Грина. Случай в цирке очень напоминает лиходеевское варьете)

\iusr{Михаил Фролов}

Вообще-то надо понимать о чем идёт речь в \enquote{Мастере и Маргарите}. Что изображает
Мастер в романе о Пилате? Почему его роман разбит на главы, а не идёт отдельной
новеллой? Если понимания этого нет, то и обсуждать нечего.

\iusr{Люся Киевская}
\textbf{Igor Pachev} 

убеждена, что все перечисленные Вами книги - если и знает кто- то ( кроме Вас),
то это единицы., а Булгакова читали тысячи или десятки тысяч читателей ( на
своём родном языке)., а за прошедшие несколько десятилетий после первых изданий
- уже сотни тысяч, причём в разных странах , людей - знающих русский язык ! ...


\iusr{Irina Somova}
\textbf{Igor Pachev} 

и что? Вот вся музыка состоит из семи нот. Следуя вашей \enquote{логике} (скорее
выпендриванию), все композиторы сплошь плагиаты. Вы \enquote{любительскую} диссертацию
защитить не пробовали? После признания американского президента Украинской
академией наук наследником украинских казаков Байды и династии Рюриковичей
(Теперь Джо Байдан - Байда Рюрикович ) и не такое пройдет.

\begin{itemize} % {
\iusr{Ольга Криворучко}
\textbf{Irina Somova} Ваш аргумент смішний і недоречно заполітизований з переходом на особистості. Це не дискусія, А казна-що. До речі, алфавіт складається з 33 букв, якщо ви про 7 нот згадали
\end{itemize} % }

\iusr{Ольга Криворучко}
\textbf{Igor Pachev} 

Давно читала такі дослідження, що свідчить про велику начитанність Булгакова і
вміння по-своєму викласти чужі твори. Все одно-плагіат


\iusr{Настя Кипиш}
Хохо, а Приключения Никодима старшего? Шо никто не слышал?

\iusr{Igor Pachev}

Люся. То есть украл и молодец? То что ты жила в инкапсулированном мире, где не
было доступа к мировой литературе ты ставишь как достижение Миши венеролога


\iusr{Igor Pachev}

Сомова. Он украл сюжет, героев, даже мелочи. А вам Байден и Украина не дают
покоя, как этому украинофобу булгакову

\begin{itemize} % {
\iusr{Наталия Удовченко}
\textbf{Igor Pachev} 

Ну началост. О Булгакове ведь такой хороший пост. заячем все с ног на ногу
переворачивать, политику приписывать. Как говорится в известной книге \enquote{... и
тут Остапа понесло...}

\end{itemize} % }

\iusr{Igor Pachev}

Начитанность? Да это плагиат сюжетов, героев. Кстати Миша венеролог ненавидел
украинцев, наш язык и культуру. Это очевидно. Нам это божья росса?


\iusr{Наталия Киевская}
\textbf{Igor Pachev} 

Вы представляете! в музыке 7 нот(октавы опустим) и эти \enquote{композиторы} постоянно
друг у друга \enquote{заимствуют} разные их сочетания.. ну прям как в лит-ре. Только
одни известны и почитаемы миром, а другие иногда на слуху.

\iusr{Igor Pachev}

Наташа в музыке плагиатом считается заимствование трех тактов - ноты лишь
материал. Как кирпичики из которых каждый архитектор строит СВОЙ дом. Твой
булгаков крал сюжеты, героев, даже мелочи и имена

\iusr{Геннадий Городецкий}
\textbf{Igor Pachev} где крал, извините?

\iusr{Natalya Dobrovolska}
\textbf{Igor Pachev} 

Дуже цікаво, не знала! Ніколи не подобався твір \enquote{Майстер та Маргарита},
відчувала штучність всього того - і подій, і героїв.


\iusr{Natalya Dobrovolska}
\textbf{Igor Pachev} Цікаво було б почитати всі ці твори, щоб порівняти. А звідки така інформація?

\iusr{Igor Pachev}

Я увидел много ссылок от лица украинских культурологов, которых достал культ
этого нацика. Просто загуглите \enquote{булгаков плагиатор}

\iusr{Люся Киевская}

Только сегодня прослушала т/ п с известным телеведущим РФ - Александром
Плющевым ( он отвечал на вопросы радиослушателей и телезрителей) - на вопрос :-
« Ваша любимая книга ?» - ответ :-« Мастер и Маргарита» ! Это на Ютубе ., не на
« Раша - ТВ» ...


\iusr{Геннадий Городецкий}
\textbf{Igor Pachev} ооо, так вы и до мышей дотрахаетесь...!!!

\iusr{Français Pour Tous}
\textbf{Igor Pachev} 

Вам название произведения Гёте Фауст надеюсь известно? Почитайте в оригинале
рецензии на Ночную Маргариту и станет понятно кто что у кого списал... Скорее
француз у Гете, причём создавая облегчённый вариант древней легенды о Фаусте...

\end{itemize} % }

\iusr{Нина Опольская}

Спасибо за публикацию, Петр, Памяти Мастера, Гения .Помним ..Всегда. .

\iusr{Zhanna Borodina}
\textbf{Нина Опольская} почитайте ті твори що рекомендуем Ігор може попусте вас

\iusr{Вадим Абрамов}

Конеч много можно вылить помоев на Булгакова, особенно тем кто и двух строчек
без ошибок написать не умеют. Вспомнить все его грехи, но это дозволено только
тем, кто смог достигнуть высших высот и возвыситься над Мастером. Жаль до сих пор
таких не нашлось!

\begin{itemize} % {
\iusr{Люся Киевская}
\textbf{Вадим Абрамов} почему же жаль ?!

\iusr{Вадим Абрамов}
\textbf{Люся Киевская} это в переносном значении, а если честно, то таких просто нет!

\iusr{Roman Kovtun}
\textbf{Vadim Abramov} с такой логикой Сталина может критиковать только Гитлер или наоборот

\iusr{Владыслав Волосенко}
\textbf{Vadim Abramov} по моему наоборот)))люди русского мира более ограниченные и не знанием языков и как следствие,не знанием литературы этих культур

\begin{itemize} % {
\iusr{Вадим Абрамов}
\textbf{Владыслав Волосенко} 

люди любых языков бывают ограниченны. Это факт. А не знание других языков и
культур-просто незнание. Но если вы не знаете языка и культуры, к примеру,
Англии, то тогда Шекспир будет непонятен? И ещё пример: если вы не знаете языка то
Кальдерон вами будет непрочиитан?


\iusr{Владыслав Волосенко}
\textbf{Vadim Abramov} конечно же, людям русского мира ближе читать о том, что москвичей испортил квартирный вопрос, а не Шекспира в переводе)))))

\iusr{Вадим Абрамов}
\textbf{Владыслав Волосенко} 

не хочу вас огорчать, но квартирный вопрос испортил не только СССР, но и Великую
Британию, почитайте Конан Дойла-там в записках Шерлока Холмса полно убийств на
почве квартирного вопроса. Далее Рекс Стаун-американец, там тоже квартирный
вопрос стоит во главе угла. Только там этот вопрос в частной плоскости, а у
Булгакова в социальной, или общественной, даже не знаю, как сформулировать. Короче
они а разных плоскостях-но квартирный вопрос не отменяли никто и никогда. И как
не крути его, к сожалению, не отменят никогда.


\iusr{Аліса Забой}
\textbf{Vadim Abramov} Починаючи з печер... На жаль...

\end{itemize} % }

\end{itemize} % }

\iusr{Людмила Каштан}
Он просто Гений. И спорить больше не о чем.

\iusr{Петро Гарматюк}
Несколько раз брался прочесть \enquote{Мастер и Маргарита} и не смог.
Мистика какая-то.
Ну, разве можно в это поверить.
Даже в театре им. И. Франко спектакль не досмотрел. Стало плохо Б. Ступке и спектакль был прерван.

\begin{itemize} % {
\iusr{Михаил Фролов}

А Вы отбросьте все предрассудки и заставьте себя прочитать \enquote{Мастера} до конца.
После этого Вы будете мысленно возвращаться к прочитанному. И сама рука
потянется к книге, чтобы перечитать её ещё раз. И это будет не последний раз.


\iusr{Тина Шевченко}
\textbf{Петро Гарматюк} 

Согласна на все 100\%!!!! Только, увы, это не мистика, а жестокая реальность: дьявол
и бесы существуют, в романе они очень активно фигурируют, поэтому и плохо
становится и актерам, и многим, кто чувствителен, при более глубоком \enquote{контакте}
с этим произведением...

\iusr{Lara Ilich}
\textbf{Петро Гарматюк} 

у дитинстві, підлітковому віці добре сприймається. Багато казкового.Я знаю
хлопчика, який у 9 років цю книжку майже напам'ять знав. Читав безліч разів.


\iusr{Михаил Фролов}
\textbf{Тина Шевченко}, \textbf{Lara Illich} 

проблема в том, что в романе вообще ничего сказочного нет. А пером Мастера
Булгаков, чего вы, к сожалению не понимаете, описывает современную Мастеру
жизнь, а не события 2000-летней давности. Жизнь, современниками которой были и
Булгаков, и герой романа Мастер.

\begin{itemize} % {
\iusr{Lara Ilich}
\textbf{Михаил Фролов} спасибо за разъяснение. А то бы так ничего и не поняла и умерла невеждой! @igg{fbicon.heart.eyes} 
Впрочем, поделюсь \enquote{тайной}: я до сих пор считаю, прожив 70 лет, что жизнь - это сказка. Но, в отличие от вас, вполне понимаю., что у других в жизни ничего сказочного нет и не было.
Не проецируйте своё мировосприятие на других - это скучно и не соответствует реальности.
\end{itemize} % }

\iusr{Михаил Фролов}

@Lara Illich я же ни слова не сказал, кто какую жизнь прожил. Кто-сказочную, а
кто-то совсем иную. Разговор здесь о романе. Если вы прожив 70 лет, считаете,
что Булгаков, который вложил в свой роман всего себя и назвал его \enquote{последним,
закатным} и будучи уже тяжелобольным, постоянно повторял \enquote{чтобы знали, чтобы
знали}, написал сказку, то это ваши личные проблемы.


\iusr{Lara Ilich}
\textbf{Michael Frolov} я очень тронута, что вас волнуют мои личные проблемы. Как говорят: щиро" дякую за увагу!"

\iusr{Людмила Викторова}
\textbf{Петро Гарматюк} Попробуйте начать читать опять! Войдете во вкус и не сможете оторваться ! Посмотрите...

\iusr{Виталий Дубас}

Читал \enquote{Мастера и Маргариту} 4 ре раза... и скаждым взрослением открывал в этой
книге все новое и новое и вечное... Но, не всем это дано....

\end{itemize} % }

\iusr{Alex Dudchak}
Наверное, \enquote{Мастер и Маргарита} должны появиться в нужное время) Мне
повезло, книга попалась в студенческие годы, прочитал за 2 ночи.

\iusr{Maryna Chemerys}
\textbf{Alex Dudchak} В Афінах стоїть пам'ятник Байрону, напис грецькою, а не англійською. І так в усьому світі - прикладів безліч. А Вам державна мова муляє, ніяк не заспокоїтесь.

\iusr{Roman Kovtun}
Так как он не любил Украину ещё поискать

\end{itemize} % }
