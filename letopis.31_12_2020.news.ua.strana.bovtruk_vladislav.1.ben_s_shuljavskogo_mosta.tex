% vim: keymap=russian-jcukenwin
%%beginhead 
 
%%file 31_12_2020.news.ua.strana.bovtruk_vladislav.1.ben_s_shuljavskogo_mosta
%%parent 31_12_2020
 
%%url https://strana.ua/articles/interview/309498-temnokozhij-stroitel-ben-s-shuljavskoho-mosta-tak-i-ne-poluchil-ukrainskij-pasport-ot-zelenskoho.html
 
%%author 
%%author_id bovtruk_vladislav
%%author_url 
 
%%tags 
%%title Бен с Шулявского моста: "Год назад я хотел получить паспорт от Зеленского, а сейчас прошу паспорт у Господа"
 
%%endhead 
 
\subsection{Бен с Шулявского моста: \enquote{Год назад я хотел получить паспорт от Зеленского, а сейчас прошу паспорт у Господа}}
\label{sec:31_12_2020.news.ua.strana.bovtruk_vladislav.1.ben_s_shuljavskogo_mosta}
\Purl{https://strana.ua/articles/interview/309498-temnokozhij-stroitel-ben-s-shuljavskoho-mosta-tak-i-ne-poluchil-ukrainskij-pasport-ot-zelenskoho.html}
\ifcmt
	author_begin
   author_id bovtruk_vladislav
	author_end
\fi

Владислав Бовтрук 13:55, сегодня

\begin{leftbar}
  \begingroup
    \em\Large\bfseries\color{blue}
Парень из Конго, который работает в Киеве, рассказал "Стране", как он уже год
ждет получения украинского гражданства, о Большой стройке и движении Black
Lives Matter
  \endgroup
\end{leftbar}


\ifcmt
  pic https://strana.ua/img/article/3094/temnokozhij-stroitel-ben-98_main.jpeg
	caption Темнокожий строитель Бен с Шулявского моста так и не получил украинский паспорт. Фото: Страна 
  width 0.7
\fi


Ровно год назад мы писали про Бена, темнокожего парня с дредами, который
укладывал асфальт на Шулявском мосту.  

Электрик из Конго стал звездой в один день, когда киевляне
сфотографировали необычного работягу за укладкой асфальта. Сегодня Бэн
трудится на проекте "Большая стройка".

Год назад в нашем интервью Бен обратился к президенту Владимиру
Зеленскому, чтобы тот помог ему с гражданством. "Зеленский говорил, что
готов предоставить гражданство всем, кто хочет работать в Украине. Я очень
хочу здесь работать, я люблю эту страну и не хочу никуда уезжать. Поэтому
если в новом году я получу паспорт Украины - мое самое большое желание
сбудется", - заявил нам Бен.

За год ничего не произошло. Бен продолжал строить дороги без документов и
паспорт так и не получил. Мы встречаемся с Бэном год спустя, на том же
Шулявском мосту, и опять в преддверии праздника.

— Бэн, почему за год вы так и не смогли оформить гражданство?

— Я делаю все, что от меня требуют, но у меня нет паспорта. Из-за этого в
миграционной службе мне говорят, что помочь не могут. Я не понимаю, мне
постоянно без документов ходить или как? Проблема точно не во мне, а в
системе.  

— Были ли у вас проблемы с законом за этот год из-за отсутствия
документов?

— Как-то на мосту Патона из-за ДТП образовалась пробка. Я начал помогать
автомобилям разъехаться, какой-то-то женщине это не понравилось, она
видимо хотела объехать нас, а я ее не пропустил. После чего она обматерила
меня и вызвала полицию. Полицейские попросили меня извиниться. Я не считал
себя виновным, но извинился. Мне показалось, что ее оскорбило, что я
черный и что-то ей запрещаю.  

\ifcmt
  pic https://strana.ua/img/forall/u/0/25/photo_2020-12-30_11.22_.32_.jpeg
  width 0.4
\fi

— К слову, а как вы относитесь к движению Black Lives Matter? 

— Я плохо к ним отношусь. Многие темнокожие очень ленивые, они не хотят
работать, много пьют. Я в Киеве знаю многих ребят, которые живут по семь
человек в одной квартире из-за того, что не хотят работать. А я работаю, у
меня есть еда, жильё, я помогаю своим родным в Конго. Нужно ходить на
работу, а не на митинги. 

— Болели ли вы коронавирусом? 

— Нет. Я работаю на улице, здесь воздух чище, во время работы особо ни с
кем не общаюсь, а когда работы нет, я сижу дома.  

— Боитесь заболеть? 

— Как и все. Не хочу болеть, но надеюсь, что если заболею, все пройдет
гладко. Я не пью, не курю, болею очень редко, у меня хороший имунитет.  

— Болеют ли коронавирусом у вас на родине? 

— Да. Люди очень бояться заболеть, так как система медицины очень плохая.
Ну, или нужны большие деньги для лечения, которых у большинства нет. Если
не ошибаюсь, в Демократической Республике Конго коронавирусом заболело
почти 17 тысяч человек, выздоровело – около 14 тысяч, а умерло примерно
500 человек.  


\ifcmt
  pic https://strana.ua/img/forall/u/0/25/photo_2020-12-30_11.22_.38_.jpeg
	caption Темнокожий строитель из Конго Бен
  width 0.5
\fi


— Слышали ли вы про фонари, которые недавно упали на Шулявском мосту? 

— Да, конечно, слышал. Про них все говорят. Моя бригада занимается
укладкой асфальта, я к фонарям не имею отношения, поэтому не знаю, что там
произошло. Но это стройка, здесь может произоти все, что угодно.  

— Как долго еще будут строить Шулявский мост? 

— Не знаю, это известно только руководству, но думаю, еще не меньше года.
Работы проводят медленно, задействовано небольшое количество людей. А если
выпадет снег, так работы вообще остановят.  

Бен с Шулявского моста ждет украинский паспорт год

— Что вам больше всего запомнилось за этот год? 

— У меня обычная жизнь, я работаю и провожу время дома. Я не люблю ходить
по клубам. Мне запомнилось несколько конфликтов с руководством. Но
это неромантические истории, и я не хочу много про них говорить.  

— Появилась ли у вас девушка? 

— Нет. Я уже говорил, что люблю сидеть дома. А девушки хотят ходить по
клубам, кафе. Мне это не нужно. У меня друзей особо-то и нет. Многие
темнокожие, которые, как и я, без документов, сидят дома, объясняя это
тем, что их могут арестовать, но они просто не хотят работать, я с такими
не хочу общаться.  

— За этот год у вас возникали мысли о том, чтобы уехать из Украины,
допустим, в Париж, куда вы изначально направлялись из Конго? 

— Нет. Мне и здесь хорошо. В Париже также есть безработные и бездомные,
там также есть эмигранты. Какая разница, где работать? Главное, чтобы
работа была. В Украине очень хорошо. Здесь много работы. 

— В прошлом году вместо желания для Деда Мороза вы обратились к президенту
с просьбой о паспорте, но результата не последовало. Что вы хотите
загадать в этом году? 

— В этом году я загадаю такое же желание. В прошлом году я хотел получить
паспорт гражданина Украины от президента Зеленского, а сейчас прошу
паспорт у Господа. Мне больше ничего не остается делать.


