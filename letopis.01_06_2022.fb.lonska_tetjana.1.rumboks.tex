% vim: keymap=russian-jcukenwin
%%beginhead 
 
%%file 01_06_2022.fb.lonska_tetjana.1.rumboks
%%parent 01_06_2022
 
%%url https://www.facebook.com/tatiana.lonskaja/posts/pfbid02bG3Dd7Jkrv29UEE8iQAQVp5qKHXzRreNr4BccxM6CDjkvuoJPWyY96jSBZoyJK7yl
 
%%author_id lonska_tetjana
%%date 
 
%%tags 
%%title Сегодня буду вас радовать, отвлекать от тревожных мыслей и одновременно хвастаться
 
%%endhead 
 
\subsection{Сегодня буду вас радовать, отвлекать от тревожных мыслей и одновременно хвастаться}
\label{sec:01_06_2022.fb.lonska_tetjana.1.rumboks}
 
\Purl{https://www.facebook.com/tatiana.lonskaja/posts/pfbid02bG3Dd7Jkrv29UEE8iQAQVp5qKHXzRreNr4BccxM6CDjkvuoJPWyY96jSBZoyJK7yl}
\ifcmt
 author_begin
   author_id lonska_tetjana
 author_end
\fi

Сегодня буду вас радовать, отвлекать от тревожных мыслей и одновременно
хвастаться.  Потому что хочу рассказать вам о своем новом увлечении, и сразу же
показать результат работы нескольких месяцев. 

Знакомьтесь – это «румбокс», дословно «комната в коробке» - довольно новый
вид рукоделия, который мгновенно приобрел миллионы почитателей во всем
мире. Тематика домиков и их внутренних интерьеров  самая разнообразная:
вилла у моря, домик звездочета, кафе-кондитерская, свадебный салон, студия
звукозаписи, японский ресторан... и еще сотни всевозможных вариаций, какие
только можно придумать.

Наш домик называется «Розовая мечта» и, согласитесь, он таки
соответствует своему названию. В домике два этажа и шесть помещений, в
каждом из которых горит собственный свет в крошечных люстрах и
светильниках. Также в него встроен звуковой механизм, как в музыкальной
шкатулке. Размер всего домика 25х20 см, высота 17 см – вот такой он
миниатюрный. Но можете не сомневаться – в нем есть все необходимое для
жизни – от рояля до лака для ногтей. На краешке умывальника лежит
микроскопическое мыло, а в раковине на кухне есть пробочка, чтобы не
вытекала вода.  Домашние тапочки уютны и пушисты, а серебряные туфельки
по-настоящему гламурны. Кроватка застелена кружевным одеяльцем, подушечки
мягонькие, хрустальные бусинки на занавесках красиво переливаются. Цветут
цветы в горшочках, помпончик на кошачьем домике ждет домашнего любимца, а
диванчик так и манит присесть и отведать лимонный кекс из малюсеньких
ажурных тарелочек с серебристыми вилочками.  Есть книжки и журналы,
корзинка для рукоделия, посудка со сладостями, баночки и пузырьки с
косметикой и духами. Словом – заезжай и живи в свое удовольствие,
исполняй свою розовую мечту.

Скажу сразу – делается это все не так быстро и красиво, как выглядит.
Когда нам подарили этот набор, казалось, что справиться с ним можно за
пару дней. Ага! Не тут-то было. Если не заниматься этим с утра до ночи, то
вам будет, чем заняться вечерами, примерно, с полгода. Потому что все, что
есть в домике, вам предстоит склеить самому. В наборе есть выкройки и
инструкция (правда, на китайском), а еще пакетики с лоскутками,
проволочками, бусинками и фанерками. Все. Дальше - ваше терпение и
многочасовой кропотливый труд.  Учитывая то, что деталей тысячи (только
одна люстра, например, состоит из семидесяти деталей), и большинство из
них имеют размеры в пару миллиметров, делать все нужно очень медленно и
практически все пинцетом. Конечно, не «подковать блоху», но где-то близко
с этим. Дети при этом могут только наблюдать, в крайнем случае – подбирать
детальки для конкретной мебели. Это хобби для взрослого, ребенок, даже
подросток, с таким занятием вряд ли справится. Но радости от процесса это
не убавит, наоборот, сплотит всю семью, потому что и для пап работа тоже
найдется.

А результат вы видите сами. На него хочется смотреть и смотреть, тем
более, что домик запросто может служить ночником. Сверху он накрывается
прозрачным куполом от пыли. Вся мебель у нас была готова еще до войны.
Конечно, уезжая в эвакуацию, мы все бросили. Но, вернувшись, с
удовольствием приступили к сборке самого домика и расстановке предметов.
Поверьте, очень отвлекает от тревожных мыслей во время воздушных тревог.
Каждый, в наше непростое время, находит для себя «таблетку» от депрессии и
страха, подходящую именно для него.

Дорогие друзья, я очень надеюсь, что вы улыбнулись, читая этот пост, и на
пару минут улетели мыслями в приятные эмоции. Сегодня начинается лето –
своеобразный новый рубеж жизни. Я от всей души желаю вам, чтобы это лето
было радостным и спокойным для каждого из вас, и принесло мир и
долгожданную победу в нашу Украину.

\ii{01_06_2022.fb.lonska_tetjana.1.rumboks.orig}
\ii{01_06_2022.fb.lonska_tetjana.1.rumboks.cmtx}
