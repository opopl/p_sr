% vim: keymap=russian-jcukenwin
%%beginhead 
 
%%file slova.jazyk.language
%%parent slova
 
%%url 
 
%%author 
%%author_id 
%%author_url 
 
%%tags 
%%title 
 
%%endhead 
\chapter{Язык (Мова, Language)}
\label{sec:slova.jazyk.language}

%%%cit
%%%cit_head
%%%cit_pic
%%%cit_text
Є бeзлiч фaктopiв, чepeз якi люди гoвopять caмe \emph{piднoю мoвoю}, aлe нaйгoлoвнiший
- цю \emph{мoвy} вoни вивчили вiд бaтькiв. Taк, цe пpиpoдний пpoцec - \emph{мoвa} пepeдaєтьcя
нaм y cпaдoк, зaбeзпeчyючи тяглicть пoкoлiнь i cтвopюючи нaцioнaльнy cпiльнoтy.
Baжливим, xoч i нe зaвжди визнaчaльним, є фyнкцioнaл \emph{мoви} - чи мoжeмo ми
poзвивaтиcя, бyдyвaти кap'єpy, зaдoвiльняти aмбiцiї, cтвopювaти бiзнecи тoщo -
пocлyгoвyючиcь cвoєю мoвoю. B yмoвax кoлoнiзaцiї цe нeмoжливo, aджe \emph{мoвa}
кoлoнiзaтopa ввaжaєтьcя пpecтижнiшoю i вiдчиняє бiльшe двepeй. Цe ми дoбpe
знaємo з icтopiї Укpaїни i з cимвoлiчнoгo пpиклaдy Mини Maзaйлa
%%%cit_comment
%%%cit_title
\citTitle{Справа не в милозвучності}, Андрій Любка, day.kyiv.ua, 25.06.2021
%%%endcit

%%%cit
%%%cit_head
%%%cit_pic
%%%cit_text
Пізніше виявилося, що це фейк, чутка, яку недочули й перебрехали, але вона
встигла поширитися серед найширших суспільних верств. Як стверджують історики,
почалася вона з публікацій в українській газеті «Свобода», яка видавалася в
Америці. В одній із статей була така фраза: «Нашу \emph{руска мова} якъ каже одинъ
ученый прфесоръ Бантишъ-Камҍнскiй есть второю по италіяньской що до звучности,
що до солодкости, що до придатности подъ безсмертни ноти всесвҍтных музикальних
композиторовъ»
%%%cit_comment
%%%cit_title
\citTitle{Справа не в милозвучності}, Андрій Любка, day.kyiv.ua, 25.06.2021
%%%endcit

%%%cit
%%%cit_head
%%%cit_pic
\ifcmt
  pic https://gdb.rferl.org/BE77E58B-9F52-49C8-BD7F-589F144B44E6_w1597_n_r1_st.jpg
	caption Під час однієї з акцій проти проєкту так званого «мовного закону Ківалова-Колесніченка», які передували «Мовному майдану». Київ, 24 травня 2012 року 
	width 0.4
\fi
%%%cit_text
З ухваленням Основного закону закінчився один з етапів на довгому шляху
унормування \emph{мовної} ситуації в Україні, що триває досі. Очевидно, що
Конституція 1996 року стала одним із ключових \emph{мовних актів}, які змінили
і змінюють нашу державу.  Те, що переслідувана Москвою \emph{українська мова} є
надзвичайно важливою для суспільства, засвідчили перші такі акти, проголосовані
ще за існування маріонеткової держави УРСР. Під тиском різних факторів, які
збіглися в часі – вулиці, перебудовчих процесів у Москві, справжніх змагальних
виборів до останнього совєтського парламенту і першої демократичної Верховної
Ради України – офіційний маріонетковий Київ ухвалив зміни до «Конституції
Української РСР»
%%%cit_comment
%%%cit_title
\citTitle{Бастіон державності і мови. Конституція України витримує наскоки руйнівників}, 
Тарас Марусик, www.radiosvoboda.org, 02.07.2021
%%%endcit


