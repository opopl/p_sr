% vim: keymap=russian-jcukenwin
%%beginhead 
 
%%file 29_10_2021.fb.kenigshtein_ilja.1.manaslu.cmt
%%parent 29_10_2021.fb.kenigshtein_ilja.1.manaslu
 
%%url 
 
%%author_id 
%%date 
 
%%tags 
%%title 
 
%%endhead 
\subsubsection{Коментарі}

\begin{itemize} % {
\iusr{Ilia Kenigshtein}
Daniel Hart - A Love Adventure
\url{https://youtu.be/_yl0hx6mUcs}

\iusr{Владимир Смирнов}
круто, нереально  @igg{fbicon.hands.applause.yellow}{repeat=3} 

\iusr{Sana Mamedieva}
Это невероятно красиво ...

\iusr{설악산}

\ifcmt
  ig https://scontent-frt3-1.xx.fbcdn.net/v/t39.1997-6/s480x480/22158190_1959413330993891_4330428087889035264_n.png?_nc_cat=102&ccb=1-5&_nc_sid=0572db&_nc_ohc=A05uToQsLHAAX8Js7uT&_nc_ht=scontent-frt3-1.xx&oh=3bc3918bd43fdee46c24a4aed4257e17&oe=619D3B5A
  @width 0.3
\fi

\iusr{Ілля Бахвалов}
 @igg{fbicon.hands.applause.yellow} 

\iusr{Любовь Рябичева}

Странное чувство феномена от « первого источника»: выставляя фото тогда, была
передача всей гаммы чувств и обратной связи  @igg{fbicon.smile}. Эти, тоже хороши и спасибо !


\iusr{Андрій Шевченко}
ФАНТАСТИКА!!!!!

\iusr{Валентина Чепурина}
Сказать, что это здорово и красиво, значит не сказать ничего! Это можно смотреть и читать с замиранием сердца и дыхания!!!

\iusr{Alexandr Podvalny}
Потрясающе !

\iusr{Ярослава Уторова}

Илья, а Непал сейчас безопасен для туризма? Думаем может туда полететь по
духовным местам, но новости пугают что очень опасно там стало

\begin{itemize} % {
\iusr{Евгений Ихельзон}

\textbf{Utorova Yaroslava} почему опасно?

\iusr{Алексей Денисов}
\textbf{Utorova Yaroslava} , 

да спокойно все в Непале - уже 2 месяца как живу в Катманду. Смотря какие
духовные места хотите посетить - если буддистские монастыри, то во внутрь вряд
ли получится попасть, их временно закрыли для посторонних посетителей из-за
ковид ограничений. А так - в принципе все нормально, все больше и больше
туристов приезжают сюда, не вижу никаких проблем.

\iusr{Ярослава Уторова}
\textbf{Алексей Денисов} благодарю, написала вам в личку  @igg{fbicon.hands.pray} 
\end{itemize} % }

\iusr{Оксана Сорокина}
Кожне фото - шедевр  @igg{fbicon.flame} 

\iusr{Татьяна Сухенко}
Невероятно красиво, а сколько пережито, сколько эмоций, ощущений, за этими фото.

\iusr{Thaneswar Guragai}
Awesome  @igg{fbicon.hands.applause.yellow} 

\iusr{Лилия Кавецкая}
Фантастика.

\iusr{Цейтнот Дедлайн}
В наушниках? А откуда играла музыка и при какой температуре?

\begin{itemize} % {
\iusr{Ilia Kenigshtein}
\textbf{Zeitnot Deadline} часы
\end{itemize} % }

\iusr{Олег Далевін}
Ці світлини щось неймовірне. Сюди і музика PINK FLOYD лягає.

\iusr{Nataly Moritz}
Наглядное и очень реальное доказательство того, что люди, достигшие любых вершин, не взлетели туда, а дошли своими ногами, взобрались шаг за шагом, преодолев немыслимое  @igg{fbicon.glowing.star}  @igg{fbicon.hands.applause.yellow}  @igg{fbicon.dizzy} 

\iusr{Андрій Атаманенко}
До мурах !!!!! @igg{fbicon.biceps.flexed}  @igg{fbicon.flame}{repeat=2} 

\iusr{Levon Papoyan}
Очень красивые и вдохновляющие фотографии.

\iusr{Вера Чванова-Барковская}
Це щось неймовірне, аж не можу собі уявити як люди можуть бути такими сміливими, так ризикувати життям, захоплення!!!!

\iusr{Nataliya Podgornaya}
Неимоверно!!! @igg{fbicon.face.flushed}  @igg{fbicon.hands.applause.yellow}  дух захватывает!

\iusr{Tetyana Nechepa}
Фотки обалдеть! Спасибо что поделились!

\iusr{Andriy Yurchenko}
Как туда попасть ? Откуда старт?

\begin{itemize} % {
\iusr{Ilia Kenigshtein}
\textbf{Andriy Yurchenko} вы имеете ввиду - с чего начать?

\iusr{Andriy Yurchenko}
\textbf{Ilia Kenigshtein} да.

\iusr{Ilia Kenigshtein}
\textbf{Andriy Yurchenko} начните с зимних Карпат. Поднимитесь (обязательно с опытным гидом) на несколько вершин: Поп Иван, Говерлу, Петрос. Так, чтобы почувствовать вообще ваше ли это. Нет прекрасней рассвета в Карпатах, когда солнце начинает всходить, а вы уже на склоне (на фото - склон горы Поп Иван рано утром).

\ifcmt
  ig https://scontent-frt3-1.xx.fbcdn.net/v/t39.30808-6/250974525_4407175176002229_8575835027079725086_n.jpg?_nc_cat=102&ccb=1-5&_nc_sid=dbeb18&_nc_ohc=h7zQCgqRZmMAX9SOK0A&_nc_ht=scontent-frt3-1.xx&oh=29365f34535afde20ead308f5eb4ab52&oe=619CE8EA
  @width 0.4
\fi

\iusr{Andriy Yurchenko}
\textbf{Ilia Kenigshtein} был на Хомяке, понравилось. Люблю хайкинг ...
\end{itemize} % }

\iusr{Ivan Bohomaz}
Космос

\iusr{Max Shakurov}
Оч круто!

\iusr{Anna Vyshniakova}
 @igg{fbicon.hands.applause.yellow}{repeat=4} 

\iusr{Герасимчук Вадим}
действительно, шедевры!

\iusr{Ольга Грибович}
Мені здається, що після такого надзусилля, після такого пройденого етапу життя поділяється на "до" і "післЯ"

\iusr{Юрий Юркив}

Те що ти пройшов знаєш тільки ти, тут важливі деталі... і добре що ти пишеш про
це. Бо люди можуть ніколи не почути про ботінок який підтерав, але ти вперто
йшов, маска яка не удобно сповзла, а поправити її зараз не варіант, бо зверху
каска чи капішон який на такому вірі не знімеш, ну і так далі  @igg{fbicon.face.tears.of.joy}.

Ніхто цього не зрозуміє так, як відчував це саме ти @igg{fbicon.wink} .

Для людей це класні фоточки і гарно написаний текст, вони не передадуть ті
відчуття коли пересохло піднебіння від нестачі води в організмі, або губи наче
наждачка при мінус 20, ти під вершиною, вже немає сил, а попереду ще 6 годин
виснажливого спуску, навколо тільки сніг і каміння, і ти такий спускаєшся з
гори, залізаєш в палатку, в свій теплий спальник і думаєш: було класно))) і
тобі знову хочеться в гори))).

Тут своя романтика, і щоб її зрозуміти, треба її прожити - прожити там в горах  @igg{fbicon.mountain.snow.capped} 

\iusr{Himalayan Traverse}
Namaste

\end{itemize} % }

