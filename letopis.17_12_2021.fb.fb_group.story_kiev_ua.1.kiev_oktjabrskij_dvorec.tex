% vim: keymap=russian-jcukenwin
%%beginhead 
 
%%file 17_12_2021.fb.fb_group.story_kiev_ua.1.kiev_oktjabrskij_dvorec
%%parent 17_12_2021
 
%%url https://www.facebook.com/groups/story.kiev.ua/posts/1820420824821377
 
%%author_id fb_group.story_kiev_ua,fedjko_vladimir.kiev
%%date 
 
%%tags gorod,istoria,kiev
%%title Кіевскій інститут Императора Николая I
 
%%endhead 
 
\subsection{Кіевскій інститут Императора Николая I}
\label{sec:17_12_2021.fb.fb_group.story_kiev_ua.1.kiev_oktjabrskij_dvorec}
 
\Purl{https://www.facebook.com/groups/story.kiev.ua/posts/1820420824821377}
\ifcmt
 author_begin
   author_id fb_group.story_kiev_ua,fedjko_vladimir.kiev
 author_end
\fi

Кіевскій інститут Императора Николая I

***

В радянські часи Жовтневий палац був однією з візитних карток Києва. В шкільні
роки мені довелося декілька разів побувати в ньому на новорічних ялинках. ЦІ
приємні спогади залишилися на все життя.

Але чи завжди ця будівля архітекторів Вікентія і Олександра Беретті виглядала
так, як вона виглядає зараз? 

Ні! Працюючи у фондах Державного архіву Київської області по своїм темам –
військова історія і історія спецслужб, – я знайшов плани будівництва і
реконструкцій будівель інституту за 1855 – 1911 роки! Чим і хочу поділитися з
клубним товариством.

\begin{multicols}{2} % {
\setlength{\parindent}{0pt}

\ii{17_12_2021.fb.fb_group.story_kiev_ua.1.kiev_oktjabrskij_dvorec.pic.1}

\ii{17_12_2021.fb.fb_group.story_kiev_ua.1.kiev_oktjabrskij_dvorec.pic.2}
\ii{17_12_2021.fb.fb_group.story_kiev_ua.1.kiev_oktjabrskij_dvorec.pic.2.cmt}

\ii{17_12_2021.fb.fb_group.story_kiev_ua.1.kiev_oktjabrskij_dvorec.pic.3}
\ii{17_12_2021.fb.fb_group.story_kiev_ua.1.kiev_oktjabrskij_dvorec.pic.3.cmt}

\ii{17_12_2021.fb.fb_group.story_kiev_ua.1.kiev_oktjabrskij_dvorec.pic.3a}

\end{multicols} % }

***

У 1834 році, покидаючи Київ, генерал Дмитро Бегічев подарував свою садибу з
триповерховим будинком «на пользу открытого ныне в Киеве Университета Князя
Владимира или же на другой таковой же общеполезный предмет без всякой за то мне
зарплаты или иного какого-либо вознаграждения». 

За рішенням міської управи ділянку було передано під будівництво Київського
інституту імператора Миколи I (Інституту благородних дівчат). Будівництво
почалося в 1838 році і тривало протягом п'яти років.

\ii{17_12_2021.fb.fb_group.story_kiev_ua.1.kiev_oktjabrskij_dvorec.pic.4}

Відомий київський архітектор, Вікентій Беретті, обрав стиль пізнього
класицизму. Але довести почате до кінця не склалося – він помер раніше і роботу
закінчував його син Олександр. 

\ii{17_12_2021.fb.fb_group.story_kiev_ua.1.kiev_oktjabrskij_dvorec.pic.5}

Пізніше до головного корпусу були прибудовані інші споруди.

***

Після революції 1917 інститут був закритий. Зруйнована під час громадянської
війни будівля була відремонтована та націоналізована. У різні часи у ній
розташовувалися гуртожиток, комунальні служби, інститут шкіряної промисловості. 

У 1934 році, з поверненням Києву статусу республіканської столиці, будівля
стала резиденцією НКВС. У її підвалах було влаштовано камери для «ворогів
народу», а також приміщення для тортур та розстрілів. Тут були закатовані та
вбиті найкращі представники української інтелігенції — письменники, художники,
актори, лікарі, юристи, військові. 

\begin{multicols}{2} % {
\setlength{\parindent}{0pt}

\ii{17_12_2021.fb.fb_group.story_kiev_ua.1.kiev_oktjabrskij_dvorec.pic.7}
\columnbreak
\ii{17_12_2021.fb.fb_group.story_kiev_ua.1.kiev_oktjabrskij_dvorec.pic.7.cmt}

%\ii{17_12_2021.fb.fb_group.story_kiev_ua.1.kiev_oktjabrskij_dvorec.pic.6}
\end{multicols} % }

В 1943 році палац був бомбардуваний радянською авіацію внаслідок чого зазнав
значної руйнації.

Відбудова будівлі (у 1952–1957 роках) здійснювалася під керівництвом команди
архітекторів на чолі з  Олексієм Заваровим. На будівництві працювали мешканці
Києва – робітники, студенти, старшокласники, а також діячі культури та
мистецтва. Крім того, використовувалася праця військовополонених. З цих часів
офіційна назва будівлі стала вимовлятися як Жовтневий палац. Київ відчинив
двері палацу для перших глядачів 24 грудня 1957 року.

\ii{17_12_2021.fb.fb_group.story_kiev_ua.1.kiev_oktjabrskij_dvorec.pic.8}

До середини 1970-х років це місце стало головною сценою країни. Зал для
глядачів приймав до 2200 осіб. Тут йшли найважливіші виступи, концерти, збори
та з'їзди України.

***

\ii{17_12_2021.fb.fb_group.story_kiev_ua.1.kiev_oktjabrskij_dvorec.cmt}
