% vim: keymap=russian-jcukenwin
%%beginhead 
 
%%file 07_12_2021.yz.maj_dnr.1.ukr_propaganda_doneckie
%%parent 07_12_2021
 
%%url https://zen.yandex.ru/media/id/5f8f226b1fe36c1d9e02a36b/pochemu-na-doneckih-ne-deistvuet-ukrainskaia-propaganda-61afa22a08b773024e82437f
 
%%author_id yz.maj_dnr
%%date 
 
%%tags 
%%title Почему на донецких не действует украинская пропаганда?
 
%%endhead 
\subsection{Почему на донецких не действует украинская пропаганда?}
\label{sec:07_12_2021.yz.maj_dnr.1.ukr_propaganda_doneckie}

\Purl{https://zen.yandex.ru/media/id/5f8f226b1fe36c1d9e02a36b/pochemu-na-doneckih-ne-deistvuet-ukrainskaia-propaganda-61afa22a08b773024e82437f}

\ifcmt
 author_begin
   author_id yz.maj_dnr
 author_end
\fi

\ifcmt
  pic https://avatars.mds.yandex.net/get-zen_doc/5365952/pub_61afa22a08b773024e82437f_61afa287aef7ef1bc2d33afa/scale_1200
  @width 0.7
\fi

Недавно один украинский аферист назвал Донбасс «раковой опухолью» на теле
Украины. Не поверите, после нахлынувшего сперва чувства неумения и протеста (ну
да, я до сих пор нее могу привыкнуть к оскорблениям от Украины) я обрадовалась.
Подумалось вдруг – а если это намек? Тогда называйте нас как угодно, только бы
это обозначало, что вы, наконец, отлипните от Донбасса, перестанете утюжить
наши поселки своими бомбами. Но пришлось тут же одернуть саму себя: Ты что,
Майя, - сказала я самой себе, - не знаешь разве, что каждое слово там
выверяется и произносится с определенным расчетом. Так что...

А там и права каждое слово пишется и говорится с определенной целью, в
основном, не с самой доброй, а как раз наоборот. Зачем были нужны истории про
древних укров, которые выкопали Черное море? Про древнюю украинскую нацию и
древний же желто-голубой флаг, который на самом деле подарен Галиции
Францем-Иосифом, когда Галиция была периферией Австро-Венгрии? Зачем была нужна
придуманная история борьбы за независимость и как могли украинцы поверить в то,
что Бандера – герой, а не подлый предатель? Зачем было нужно сносить памятники
воинам-освободителям? Почему на Украине вообще относятся к мемориалам памяти
Великой Отечественной войны с пренебрежением, и заметьте, это началось не
сегодня! Сотни уникальных объектов культуры пришли в негодность за 30 лет
«незалежности», тысячи памятников погибшим героям Великой Отечественной сломаны
осквернены и разбиты, но главное – попрана Память.

Приведу всего один пример:

В селе Воля-Высоцкая Жовковского района Львовской области был замечательный
музей Петра Нестерова. Напомню, первый в мире воздушный таран осуществил
легендарный летчик Петр Николаевич Нестеров 26 августа 1914 года. В 1984 году
на месте гибели Нестерова построили поразительной красоты музей и обелиск с
барельефом летчика и самолетом, выходящим из виража. Посетители, а их было
немало, читали на памятной плите надпись: «На этом месте 26 августа (8 сентября
по старому стилю) 1914 года в воздушном бою, впервые в истории авиации применив
таран, геройски погиб славный сын русского народа, выдающейся военный летчик,
основоположник высшего пилотажа, первый в мире выполнивший «мертвую петлю»,
Нестеров Петр Николаевич. Вечная слава герою!». Но с приходом Украины музей
забросили, теперь там дикие руины и разруха. И посетителей нет. А надпись
вандалы регулярно и настойчиво обливают краской. К чести местных жителей, они
отмывают плиту, но на содержание музея за 30 лет украинская власть не выделила
ни копейки.

\ii{07_12_2021.yz.maj_dnr.1.ukr_propaganda_doneckie.pic.2}

Не так в Донбассе.

Совсем не так. Многие наши Дома культуры, и правда, стоят пустые, заброшенные, многие разрушаются, но за памятниками и мемориалами ухаживают сами жители. Да и местная власть всегда выделяла какую никакую копейку для поддержания их в нормальном состоянии

Прошлым летом я поездила по Республике в поисках безымянных могил солдат,
погибших в боях за Донбасс. Многие нашла и сняла на камеру, во многих местах
расспрашивала местных, кто ухаживает за скромными обелисками, и всегда, всегда
слышала ответ: «Так мы же! А как иначе?». Глава администрации моего родного
Пролетарского района Елена Рязанцева, а потом и Глава Буденовского района
Владимир Гетов рассказали мне, что и при Украине умудрялись выцарапывать из
бюджета средства для реставрации памятников, а в каждом районе есть даже
специальный человек, у которого есть специальный талмуд с полным перечнем всех
мемориалов и братских могил Донбасса. «Это не просто наша память, это наша
честь», - сказал мне тогда Владимир Алексеевич.

В 2014 году у нас тоже объявились вандалы, которым наша память мозолила глаза,
но они как-то быстро пропали. Говорят, одну из таких компашек поймали местные
жители и отметелили как следует, дабы неповадно было. И сейчас даже в самых
отдаленных местах наши места Славы никогда не остаются без живых цветов, хотя
бы даже и полевых.

У меня для вас есть еще одна уникальная история

На самом Юге Республики есть село Саханка, в котором был уникальный монумент на
братской могиле погибших защитников Родины в 1941-1943 годах. Вот он:

В 2014 -2015 году украинские военные прицельно стреляли по памятнику
артиллерией до тех пор, пока он не разрушился полностью. По Саханке вообще ВФУ
шмаляют все 7 лет, но по памятнику колошматили так как будто мать, на руках
которой лежал умирающий солдат, олицетворяла главного врага Украины. Для
жителей Саханки это была настоящая трагедия, и однажды, когда там проводил
съемку последствий обстрелов мой друг, военкор Александр Гриценко с позывным
Индеец, они поделились своей главной бедой. Я не буду долго рассказывать, каких
трудов стоило найти сперва мастеров, которые повторили памятник в точности, а
потом деньги на материалы но, начиная с февраля прошлого года и по самые
осенние холода Саша все свои выходные проводил там, восстанавливая памятник.
Это была адова работа, да еще и под прицелом снайперов – я не преувеличиваю!
Центральная площадь села находится всего в километре от украинских снарядов и,
в силу местного ландшафта, в прямой видимости. Да иногда приходилось прекращать
работу и прятаться, но Индеец на то и индеец, что упорен и настойчив. Иногда
Саша что-то записывал, в итоге получился «Дневник донецкого военкора Александра
Гриценко о расстрелянном и воскресшем памятнике» Я нашла его в интернете:

\href{https://rg.ru/2021/08/11/dnevnik-doneckogo-voenkora-aleksandra-gricenko-o-rasstreliannom-i-voskresshem-pamiatnike.html}{%
Я вернулся, мама!  Дневник донецкого военкора Александра Гриценко о расстрелянном и воскресшем памятнике, %
rg.ru, 11.08.2021%
}

А вот по этой ссылке результаты моей поездки в Саханку – я не могла не увидеть
все собственными глазами!

\href{https://youtu.be/3v7-lnzUc1k}{%
Враги разбили памятник, но память не убили, МАЙ ДНР, youtube, 23.09.2020%
}

\begin{multicols}{2} % {
Есть такие люди. 

Их называют упертыми, сумасшедшими романтиками, от них отмахиваются, но они все
равно добиваются своего, если это ДЛЯ ЛЮДЕЙ

ТАКОЙ АЛЕКСАНДР ГРИЦЕНКО. 

Индеец - это не просто позывной, это Машин образ жизни, его Честь и Совесть.

Памятник в честь павших в Великой Отечественной Войне в селе Саханка украинские
фашисты уничтожили прямым попаданием - специально в него целились. 

Несколько лет Индеец болел идеей восстановить этот памятник, и таки Сделал это!

Нашел изготовителя деталей, довоенные старые фото, нашел все, что нужно. И с
апреля этого года упорно ездит в Саханку и САМ!!! иногда с друзьями, монтирует
точную копию уничтоженного украинскими фашистами. 

Сегодня памятник почти готов...
\end{multicols} % }
\ii{07_12_2021.yz.maj_dnr.1.ukr_propaganda_doneckie.scr.1}

А еще несколько фотографий, на которых разрушенный украинскими вояками памятник
и история его восстановления.


Как хотите, а я лично считаю Индейца, Александра Гриценко Героем, и вех до
единого волонтеров, которые откликнулись на его просьбу о помощи, тоже героями.
И огромное спасибо россиянам, которые помогли Александру, да еще и сделали
новую мраморную плиту с фамилиями захороненных на этом месте героев.

\textSelect{А вот теперь и ответьте сами себе на вопрос: почему на донецких не действует
украинская пропаганда?}
