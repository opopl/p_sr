% vim: keymap=russian-jcukenwin
%%beginhead 
 
%%file 16_05_2020.fb.zharkih_denis.1.ukraina_raspad
%%parent 16_05_2020
 
%%url https://www.facebook.com/permalink.php?story_fbid=2703169923229741&id=100006102787780
 
%%author 
%%author_id 
%%author_url 
 
%%tags 
%%title 
 
%%endhead 

\subsection{Украина - Распад - Мысли}
\url{https://www.facebook.com/permalink.php?story_fbid=2703169923229741&id=100006102787780}

Часто вижу в комментах и в постах, что никакая Украина не нужна, короче, пусть
Ненька распадается, а адекватные украинцы записываются в русские. Так надо! 

Господи, сколько я на своем веку слышал таких рассуждений. Вот пусть Союз
распадается кричали, хорошо это помню. В Союзе всякое было, но чтобы на
Донбассе (Абхазии, Карабахе, Осетии, Чечне) людей годами убивали... Нет,
конечно, кто-то за это получил бочку варенья и корзину печенья, а кого-то даже
взяли в Буржуинство жить, но погибают дети, старики, женщины. А начиналось-то с
неграмотных истеричных жлобов. 

Потом вот демонтаж уже украинской государственности на первом, а особенно на
втором майдане. Скинем Януковича - будем лучше жить. И уже лучше живем без
Крыма и идет бойня на Донбассе. И кричат те же рожи, с тем же выражением. 

Теперь на манеже все те же. Они не понимают, что распад СССР для миллионов
людей стал трагедией, как и майдан, который тысячи людей лишил жизни, а сотни
тысяч крова. Понятно, что малограмотные люди не видят связи силового захвата
власти в Киеве и конфликта на Донбассе. Они разрушили основы государства, вот
окраины и посыпались. 

Теперь о главном. Разрушить государство Украина не так и сложно. Она сейчас
управляется некомпетентной властью и не в интересах граждан, но вопрос в
студию: а разрушение страны дает гарантию, что власть станет компетентной и
будет слушать народ?

Когда только под восторги сограждан убежал Янукович, я сказал, что народ теперь
точно слушать не будут, ведь за нынешнюю власть пролилась кровь, что делает ее
ценной, сакральной, священной. И оказался прав. Попав в другое государство
украинцы станут вторым сортом, как нация не способная построить государство.
Это независимо от этнической принадлежности. Это тут мы еще и русские, поляки,
евреи и болгары. А за границей мы все украинцы. 

Украинская государственность нужна, как система организации жизни. Другой
вопрос, что она должна быть в первую очередь государственность, а украинская
уже потом. А нам начали строить что-то украинское, но, скорее, племя, банду,
хутор, а не государственность. 

То есть, не все украинское государственность, и не все государственность
украинское. И тут не надо выбирать. Нам нужна и Украина, и государство, просто
нельзя мешать этнические элементы в государственное строительство. Вот человек
любит и борщ, и компот, но не мешает же. 

Государство обязательно нужно строить, это наш дом. Культуру нужно развивать,
но она разная, не только этническая. И всегда найдутся дураки,провокаторы,
которые скажут, что раз дом плох, то его надо поджечь. Вот у них-то никакой ни
государственной, ни политической культуры нет. Хотя, возможно, они считают себя
носителями этнической культуры, но это вряд ли, больше прикидываются.
