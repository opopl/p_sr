% vim: keymap=russian-jcukenwin
%%beginhead 
 
%%file 30_12_2020.fb.bilchenko_evgenia.2.gorod_noch_nervy
%%parent 30_12_2020
 
%%url https://www.facebook.com/yevzhik/posts/3517256338309451
 
%%author Бильченко, Евгения
%%author_id bilchenko_evgenia
%%author_url 
 
%%tags bilchenko_evgenia,gorod,kiev,nervy,ukraina
%%title Ночной город. Дождь. Бродим с мужем по пустынным гулким стоянкам, успокаиваем мои нервы
 
%%endhead 
 
\subsection{Ночной город. Дождь. Бродим с мужем по пустынным гулким стоянкам, успокаиваем мои нервы}
\label{sec:30_12_2020.fb.bilchenko_evgenia.2.gorod_noch_nervy}
\Purl{https://www.facebook.com/yevzhik/posts/3517256338309451}
\ifcmt
 author_begin
   author_id bilchenko_evgenia
 author_end
\fi

Ночной город. Дождь. Бродим с мужем по пустынным гулким стоянкам, успокаиваем
мои нервы (я перестала спать вообще, снедают страх смерти и панические атаки, а
днем физическая слабость уже не дает двигаться: от этого я становлюсь жестче и
рискованнее). Картинка Апокалипсиса немного напоминает психоделические кадры
российской современной декадентско-постмодерной драмы о трансгрессии желания
\enquote{Аутло}, мне не близкой. Но где-то внутри огней и паркингов даже здесь таится
тот же традиционный и давно забытый, милостивый Логос. Вчера рыдала всю ночь:
мол, не успела написать в стол мемуары, которые бы всех убили наповал больше,
чем то, что мне шьют сейчас.

\begin{verbatim}
	#психо #Киев #грань #болезнь #труд #травля
\end{verbatim}


\ifcmt
  tab_begin cols=2

     pic https://scontent-lga3-2.xx.fbcdn.net/v/t1.6435-9/134801766_3517255318309553_467279202688458263_n.jpg?_nc_cat=100&ccb=1-3&_nc_sid=8bfeb9&_nc_ohc=yEnCmd8nN0oAX-8iTLi&_nc_ht=scontent-lga3-2.xx&oh=4176f706c835fa4c3139369851007aca&oe=60CC029C

     pic https://scontent-lga3-2.xx.fbcdn.net/v/t1.6435-9/133917288_3517255501642868_3175354409244670267_n.jpg?_nc_cat=104&ccb=1-3&_nc_sid=8bfeb9&_nc_ohc=u4adY4NUprEAX_Vd6MS&_nc_ht=scontent-lga3-2.xx&oh=a2efbca53b5b44f8a595c71bc130fac4&oe=60CC18AF

		 pic https://scontent-lga3-2.xx.fbcdn.net/v/t1.6435-9/134387954_3517255624976189_2920497697781451993_n.jpg?_nc_cat=103&ccb=1-3&_nc_sid=8bfeb9&_nc_ohc=zzyrQYya-8QAX_nZ7Hk&_nc_ht=scontent-lga3-2.xx&oh=7b0a0d6c671218e43c17d86ca5bca210&oe=60CB6908

		 pic https://scontent-lga3-2.xx.fbcdn.net/v/t1.6435-9/134764624_3517255918309493_8089752444962235023_n.jpg?_nc_cat=105&ccb=1-3&_nc_sid=8bfeb9&_nc_ohc=Eh9exEEigp0AX-KlGD4&_nc_ht=scontent-lga3-2.xx&oh=3d4ef44166112b8ed03acef03c2bcf7b&oe=60CB0957

  tab_end
\fi

\emph{Владимир Павлович}

Панические атаки это применяемое воздействие на всех сейчас

\emph{Ольга Сукманова}

Женечка, - нам не хватает детской непосредственности и Веры - в Чудеса...
Верь в волшебство зимней сказки и загадай желание,,, И мы - все тоже сообща
подумаем о тебе - Здоровья желаем !!! 
Быстрого выздоровления ,- Солнце  - оно в тебе!

\emph{Зоя Григорьевна Яременко}

Держитесь, Женя, если не вы, то кто же?

\emph{Людмила Беднарчик}

\textbf{Светлана Миргород}
Женя, твои друзья тебя любят. А что врагов много - гордись ! Значит, все правильно делала в своей жизни !!!

\emph{Светлана Миргород}

Болячки наши пройдут!!! Моего сына и меня хоронили сто раз...

\emph{Svetlana Sheychenko}

Женечка! Извините за мою фамильярность, ведь Вы меня не знаете. Заочно узнала о
Вас с того времени, как Вы были в гостях у Юрия Подоляки и выходили в эфир.
Душа болит за Ваше состояние( физическое и моральное). Мне помогли год назад от
панических атак, страха, повышенной тревожности - один из антидепрессантов
группы СИОЗС - \enquote{эсцитам} или аналог \enquote{гиацинтия} ( назначала
врач-невропатолог от депрессии). Болело тоже все внутри, была бессонница,
плаксивость, не было сил. Я не могу ничего навязывать ни в коем случае, просто
хочу поделиться тем, что мне реально помогло. не могу оставаться равнодушной.
Извините. Сил Вам и терпения.

\emph{Евгения Бильченко}
Спасибо, родня.

\emph{Anna Krivda}
Гулять это прекрасно!

\emph{Valentina Nechay-Velkova}
ждем!

\emph{Tanya Ponomareva}
Я просто помню о тебе. Каждый день.

\emph{Leonty Kostur}
вам в помощь мои слова

\emph{Сергей Возняк}

Жень, просто обними мужа и поспи... Поверь, поможет от всех этих вещей... Ты
умная, вот и все штуки. У меня вот были какие то такие штуки, просто именно так
(не с мужем, а с женой) просто рбнялся и ззаснул... Потом такой, ясный был...

\emph{Yuri Rag}

Апельсин вечером - помогает уснуть

\emph{Любовь Новикова}
Yuri Rag а утром - проснуться с желанием жить и радоваться ))

\emph{Наталья Стригун}
Yuri Rag и проносное утром, и неделька удалась

\emph{Ludmila Martsevich}

Женечка, много людей думают о Вас с любовью и желанием помочь! Надо лечиться,
быть в позитиве, то, что врачи рекомендуют, выполнять обязательно, но и самой
думать, анализировать. И хорошо кушать

\emph{Ludmila Martsevich}

И рядом - друг, достойный друг. Это хорошо, Думайте о хорошем! З прийдешніми
святами!

\emph{Мария Чистопольская}

Дорогая Женечка! Не грустите! Все будет хорошо! С наступающим Новым 2021 Годом!

\emph{Катарина Синчилло}

А на фото всё очень-очень красиво, и ты прекрасна!

\emph{Евгения Бильченко}
\textbf{Катарина Синчилло} Я очень долго приводила себя в порядок.

\emph{Наталья Стригун}
Катарина Синчилло где,где,эта прекрасная женщина?вот эта? Дорогая наденьте очки,и вас нельзя будет различить.

\emph{Наталья Стригун}

Евгения Бильченко и так и не привела.

\emph{Наталья Стригун}
Боже, сколько сепарской блядоты...жах

\emph{Katrin Tregubova}

С наступающим Праздником! Вопреки всему и несмотря ни на что. Сил, сил, сил,
терпения, спокойствия и здоровья!

\emph{Любовь Новикова}

Напиши мемуары

\emph{Евгения Бильченко}

\textbf{Любовь Новикова} Вчера ночью думала, но, боюсь, не успею.

\emph{Alexandr Kuzmenko}

В этом году, был в больнице, могу сказать что отношение к жизни когда болезнь
может реально убить, и когда ничто не угрожает жизни, отличается.

Брился под: Пилот - жить назло.

Осложнений не вылезло, пока не вылезло, пока живу, а остальное...суета, зона
влияния, и зона переживаний сильно отличаются, а у поэтов они ещё и не имеют
чётких границ
