%%beginhead 
 
%%file 03_08_2023.fb.muzej_literatury.ua.kiev.1.poetychna_zbirka_sergija_pantjuka_znykoma_zyma
%%parent 03_08_2023
 
%%url https://www.facebook.com/NMLUmuseumlit/posts/pfbid02LtZtwxjQ3dHMt576C4UDtiuRS5FPHS7rSejXaKNRGSaUNKRsvty54oNNveZJYykml
 
%%author_id muzej_literatury.ua.kiev
%%date 03_08_2023
 
%%tags 
%%title Презентація поетичної збірки Сергія Пантюка "Зникома зима"
 
%%endhead 

\subsection{Презентація поетичної збірки Сергія Пантюка \enquote{Зникома зима}}
\label{sec:03_08_2023.fb.muzej_literatury.ua.kiev.1.poetychna_zbirka_sergija_pantjuka_znykoma_zyma}

\Purl{https://www.facebook.com/NMLUmuseumlit/posts/pfbid02LtZtwxjQ3dHMt576C4UDtiuRS5FPHS7rSejXaKNRGSaUNKRsvty54oNNveZJYykml}
\ifcmt
 author_begin
   author_id muzej_literatury.ua.kiev
 author_end
\fi

2 серпня в Національному музеї літератури України в рамках музейного проєкту
\enquote{ритМИ і риМИ війни} відбулася презентація поетичної збірки Сергія Пантюка
\enquote{Зникома зима}.

Сергій Пантюк – відомий український письменник, автор численних книг для
дорослих і дітей, серед них і 19 поетичних збірок. Крім того, він перекладач,
редактор, журналіст, громадський діяч і видавець, член Спілки письменників
України, її секретар. Нині він головний сержант БПКР 23 окремого батальйону
спеціального призначення Збройних сил України. 

Презентацію своєї збірки автор почав з \enquote{Балади про пораненого Херувима}, яку
автор присвячує своїм загиблим побратимам. 

Збірка \enquote{Зникома зима} – дев'ятнадцята поетична книжка Сергія Пантюка. В ній
присутні і традиційні для поета мотиви, і нові тексти про війну, учасником якої
є сам автор від її початку. Частина віршів збірки написана під Бахмутом, де він
воював разом з побратимами у найгарячіший період осені 2022 – весни 2023 рр.
Окремий розділ книжки складають лінгвоекспериментальні тексти, в яких він з
гумором демонструє багатство й тонкощі рідної мови. За обкладинку взято
фрагмент картини його близького друга – поета, прозаїка, художника та
військового Бориса Гуменюка. Ілюстрації створила юна художниця Стефанія
Андрусяк.

Модератором зустрічі був відомий письменник Іван Андрусяк. Під час презентації,
він зазначив: \enquote{Хотів би, щоб ми пам'ятали: якщо в окопах під Бахмутом пишуться
вірші, то все буде Україна!} А як редактор збірки, визнав, що редагувати було
нічого, підправив кілька ком та тире – і книжка була готова. Адже у Сергія
Пантюка неймовірні поезії! 

Привітати поета прийшли друзі, побратими, колеги, музиканти, журналісти,
громадські діячі. Першим до слова було запрошено професора Юрія Коваліва, який
побачив у збірці справжнього поета, надзвичайно відвертого зі своїм читачем. Ця
книжка \enquote{без обмежень}, містить і дуже цікаві ліричні сюжети, й болісні
переживання війни, без жодної риторики, адже автор все це пережив особисто. 

Також із привітальним словом виступив український громадський і політичний
діяч, голова ОУН Богдан Червак. Він підкреслив, що дуже символічним є те, що
презентація збірки військового Роти УВО відбувається в музейній залі, де триває
виставка художниці Марини Соченко, на якій представлені портрети борців за
українську державність Дмитра Дорошенка, Симона Петлюри, Андрія Мельника,
Олексія Алмазова та Євгена Коновальця. А присутні в залі побратими Сергія
Пантюка – це воїни, які нині продовжують справу славетних попередників. Пан
Богдан також зауважив, що участь наших поетів у війні – це жахлива ціна, яку ми
платимо за нашу перемогу.

Літературна критикиня та журналістка Ніна Головченко заначила, що
патріотично-поетичне літературне слово Сергія Пантюка не розійшлося з його
чином захисту України зі зброєю в руках. Вона підкреслила, що автор стояв біля
витоків української комбатантської літератури і не зрадив собі як митцеві:
попри жахи війни, залишився поетом високого регістру. 

Дружина Сергія Пантюка – поетка, науковиця, музейниця Тетяна Шептицька –
підтвердила, що в збірці автор показав те, чим живе його душа та переймається
серце. Також висловила своє щире захоплення умінням чоловіка писати вірші
будь-якої форми: від сонетів до хоку. Пані Тетяна зізналася, що втішена
спільним творчим життям, відданістю, повсякденними проявами гумору, якими
фонтанує Сергій Пантюк. 

У презентації взяли участь побратими автора. Один із них – відомий бандурист,
народознавець, рок-музикант, композитор Живосил Василь Лютий, який написав
пісні на тексти зі збірки та презентував їх для Сергія та всіх присутніх. На
завершення зі своїми патріотичними піснями виступив бард із Ніжина Олег
Кошовий. 

Під час заходу було оголошено збір коштів для БПКР 23 окремого батальйону
спеціального призначення Збройних сил України, де служить Сергій Пантюк.

Дякуємо автору за його талант, багатогранність творчого вияву і щиру дружбу з
Національним музеєм літератури України, а всім гостям – за теплий та цікавий
вечір.
