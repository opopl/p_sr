% vim: keymap=russian-jcukenwin
%%beginhead 
 
%%file 02_12_2020.news.ua.strana.venk_victoria.1.mova_shtrafy_ukrainizacia.avakov_mozhno
%%parent 02_12_2020.news.ua.strana.venk_victoria.1.mova_shtrafy_ukrainizacia
 
%%url 
 
%%author 
%%author_id 
%%author_url 
 
%%tags 
%%title 
 
%%endhead 

\subsubsection{Авакову можно, другим - нельзя}

В первые же дни своей работы Креминь отметился двойными стандартами: он заявил,
что не против русскоязычных выступлений Арсена Авакова. Поскольку этот министр
- слишком масштабная личность и обременен весьма тяжелыми государственными
заботами. 

Впрочем, разгадка скорее в том, что Креминь - однопартиец главы МВД и еще
раньше пел ему хвалебные оды. 

Что же до остальных русскоязычных украинцев, то здесь позиции \enquote{шпрехенфюрера}
совершенно непримиримые. С введением этой осенью украинизации русских школ
Креминь обещал разобраться с учителями, которые ведут уроки на русском.

В начале ноября он заявил, что получает десятки жалоб на то, что педагоги в
столице переходят с детьми на русский язык. А внешкольное образование - кружки,
секции - \enquote{грешит} этим вообще повсеместно. 

Конечно, такие претензии в XXІ веке звучат максимально дико. Но украинизатора
из \enquote{Народного фронта}, а главное - тех, кто его назначил, это совершенно не
беспокоит. 

Интересно, что Креминь в этом смысле даже не скрывает духовного родства с
Порошенко, который летом собирал акции под Радой против законопроекта
\enquote{слуги народа} Максима Бужанского. Документ дает отсрочку русским школам на
три года по украинизации.

Креминь тогда выступил с экс-президентом на одной сцене и призвал не принимать
законопроект. В итоге \enquote{слуги} даже не вынесли его на рассмотрение. 

А в последнее время \enquote{шпрехенфюрер} просто фонтанирует инициативами.

Так, после встречи с коллегой из Латвии, он загорелся идеей мобильного
приложения, через которое можно будет жаловаться на нарушителей мовного закона.
И выступил за \enquote{цифровой мониторинг} использования государственного языка. А
также, по примеру Латвии, введение института \enquote{старших инспекторов}, которые
будут проверять все и вся. 

У Эстонии, где тоже есть подобная должность, украинский \enquote{шпрехенфюрер}
планирует позаимствовать систему проверок заведений торговли. Эта маленькая
страна делает по три тысячи проверок в год - видимо, в более масштабной Украине
хотят проводить гораздо больше рейдов. 

А буквально пару дней назад Креминь обеспокоился, что преподаватели для
слабослышащих детей все еще используют русский язык жестов. То есть
деятельность чиновника доходит уже до полного абсурда. 

Судя по заявлениям Тараса Креминя, он хочет контролировать буквально всё. Так,
он заявил недавно, что дистанционное обучение во время карантина несет угрозы.
Но не в падении качества образования - этот вопрос, видимо, не так важен
омбудсмену - а в том, что облегчает переход учителей на русский язык. Поскольку
по время онлайн-конференций их никто не контролирует.

Чтобы это пресечь, Креминь предложил фотографировать, снимать на видео и
записывать аудио нарушений.

Правда, здесь возникает сразу два вопроса. Первый - об адекватности самого
омбудсмена. Каким образом фотография может передать, на каком языке говорил
педагог? Второй - о юридической стороне вопроса. Будут ли суды принимать во
внимание любительские видео и тем более аудио? При этом нет сомнений, что таких
доказательств самому омбудсмену вполне хватит.

Также показателен сам факт, что учеников открыто призывают \enquote{стучать} на
учителей. 

\ifcmt
pic https://strana.ua/img/forall/u/0/92/%D0%BA%D1%80%D0%B5%D0%BC%D0%B8%D0%BD%D1%8C(1).png
\fi
