% vim: keymap=russian-jcukenwin
%%beginhead 
 
%%file 28_08_2017.stz.news.ua.mrpl_city.1.mariupol_osobnjak_z_grifonami
%%parent 28_08_2017
 
%%url https://mrpl.city/blogs/view/mariupolskij-osobnyak-z-grifonami
 
%%author_id demidko_olga.mariupol,news.ua.mrpl_city
%%date 
 
%%tags 
%%title Маріупольський особняк з грифонами
 
%%endhead 
 
\subsection{Маріупольський особняк з грифонами}
\label{sec:28_08_2017.stz.news.ua.mrpl_city.1.mariupol_osobnjak_z_grifonami}
 
\Purl{https://mrpl.city/blogs/view/mariupolskij-osobnyak-z-grifonami}
\ifcmt
 author_begin
   author_id demidko_olga.mariupol,news.ua.mrpl_city
 author_end
\fi

Кажуть, у будинку, як і в людини, є душа. Думаю, це цілком можливо, особливо у
тих будинків, які мають довгу історію і зберігають в собі безліч таємниць.
Світу відомі загадкові особняки і замки, чиї легенди бентежать і наводять на
романтичні роздуми. Чи є в Маріуполі будівлі з душею? Чи можуть вони розповісти
власну історію, вже давно втративши свого господаря і перебуваючи у
напівзруйнованому стані?

\ii{28_08_2017.stz.news.ua.mrpl_city.1.mariupol_osobnjak_z_grifonami.pic.1}

Так, можуть! А найбільший маріупольський промовляє до нас і немов благає почути
його біографію і нарешті врятувати. Адже саме він, маючи неповторні
архітектурні риси, несправедливо покинутий власним містом.

Чудовий двоповерховий особняк на підвалах, розташований на вулиці Італійській,
12, побудовано у 1900 р. Фасад будівлі зі здвоєними напівколонами, піднятими на
високий цокольний поверх, і широкі віконні арки справляють враження парадного
залу під відкритим небом. Ошатний ліпний карниз з грифонами опоясує весь
будинок над вікнами другого поверху. Відомо, що у грецькій міфології грифони –
це жахливі птахи з орлиним дзьобом і тілом лева, які стережуть золото,
охороняють будинок свого власника.

\ii{28_08_2017.stz.news.ua.mrpl_city.1.mariupol_osobnjak_z_grifonami.pic.2}

Господар особняк з
грифонами\footnote{По улицам Мариуполя: 360° тур по дому купца Регира (ВИДЕО), Ігор Романов, mrpl.city, 25.08.2017, \url{https://mrpl.city/news/view/po-ulitsam-mariupolya-360-tur-po-domu-kuptsa-regira-video}}
– Петро Петрович Регир – підприємець і судновласник, купець 1-ї гільдії,
народився в 1851 р. в сім'ї німецьких колоністів, був одружений на
маріупольській грекині Єфросинії Прокопівні. Він став власником вантажних
пароплавів \enquote{Прогрес}, \enquote{Великоросія}, \enquote{Бессарабія} та
перевозить щорічно до 100 тисяч тонн донецького вугілля в порти Петербург,
Кронштадт, Ревель, Лібава. Це було свого роду \enquote{Купецьке пароплавство}.

Не може не вразити, що за 23 роки без будь-якої підтримки з боку уряду Петро
Регир зумів створити найбільшу судноплавну компанію в Російській імперії, яка
зайняла п'яте місце по тоннажу серед вітчизняних пароплавств.

У роки революції сім'я Петра Регира переїхала за кордон, куди були перевезені
всі капітали. Але сам Петро Петрович по невідомим причинам вирішив залишитися у
власному особняку. І незважаючи на високий соціальний статус, наприкінці життя
його спіткала дуже складна доля. З будинку його вигнали \enquote{червоні}, після чого
він починає жити у себе на подвір'ї до 1919 р., поки місто не зайняли
денікінці. Помер при загадкових обставинах у тому ж 1919 році. Є декілька
версій: чи від голоду, чи від випадкової кулі...

\ii{28_08_2017.stz.news.ua.mrpl_city.1.mariupol_osobnjak_z_grifonami.pic.3}

Цікаво, що грифони, які повинні оберігати і господаря, і будинок, після
загибелі Петра Петровича, неначе відвернулися від будівлі. Хтозна, може в
будинку до цих пір мешкає привид першого господаря, адже Петро Петрович до
останніх днів не хотів виїжджати з особняку, який неначе його не відпускав...
Можливо, з цим і пов'язана подальша доля будинку, в якому так і не з'явився
постійний власник.

Особняк переходив з рук у руки доволі довгий час. Спочатку його переоблаштували
під житловий квартирний будинок. Наприкінці 60-х років швейна фабрика імені
Дзержинського орендувала будинок під гуртожиток. Трохи пізніше особняк передали
в оренду тресту \enquote{Азовстальконструкція}, після чого він знову змінив власника на
СУ-118 \enquote{Донбасстальконструкція}.

Остаточно руйнуватися і приходити в непридатність він став з продажем його
київській фірмі, яка, в свою чергу, продала його одеському торгівельному
товариству \enquote{Основа} в 2007 році.

Будинок неначе приречений на загибель, вже давно втратив надію на порятунок...
Хоча у тому ж 2007 році Італійська національна асоціація \enquote{Альпіна} і
генеральний директор італійської фірми \enquote{In.com.impex} Бруно Пальмьєрі
запропонували провести реконструкцію особняку і перетворити його на
\enquote{Італійський дім}, проте зв'язатися з одеськими власниками так і не вдалося...

Сьогодні особняк перетворився на руїни. Провалені сходові прольоти, виламані
двері, скло усюди, купи сміття всередині. Знаходитися на території особняку
стало небезпечно. Як так сталося, що потенційно найбільш яскрава архітектурна
пам'ятка міста, що могла б привертати увагу як маріупольців, так і туристів,
стати прикрасою Маріуполя, виявилася у власності Одеси? Той факт, що будинок
одеситам зовсім не потрібен, не викликає заперечень, господарів так і не
вдалося знайти. Проте, байдужість і абсолютна незацікавленість влади протягом
стількох років викликає лише обурення і нерозуміння... Невже Маріуполь не
заслуговує на збереження власної історичної та архітектурної спадщини? Мова йде
про будівлю, яку треба було давно перетворити в музей, яка з легкістю зробила
би наше місто більш привабливим у туристичному плані. Дуже шкода, що в
більшості випадків все залежить від тих, хто незацікавлений і не розуміє всієї
важливості вирішення проблеми.

Кажуть, що людську душу врятувати може тільки любов. Можливо, це парадокс, але
душу найбільшого особняку Маріуполя врятує також любов! Але це любов не однієї
людини, а цілого міста...

Чи зможуть грифони знайти нового господаря і продовжити оберігати вже
реконструйований особняк тепер залежить від наполегливості і рішучих дій
місцевої влади, відділу архітектури і всіх небайдужих мешканців Маріуполя.
