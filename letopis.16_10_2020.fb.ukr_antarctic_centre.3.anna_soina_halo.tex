% vim: keymap=russian-jcukenwin
%%beginhead 
 
%%file 16_10_2020.fb.ukr_antarctic_centre.3.anna_soina_halo
%%parent 16_10_2020
%%url https://www.facebook.com/AntarcticCenter/posts/1497418133801392
%%author anna soina,stancia vernadskii

%%endhead 

\subsubsection{Фото з іншого боку Землі від гідрометеоролога 25-ї УАЕ Анни Соіної}

\url{https://www.facebook.com/AntarcticCenter/posts/1497418133801392}

Атмосферне оптичне явище, яке виглядає як веселка навкруги Сонця має назву
гало. Це не що інше, як дисперсія на кристаликах льоду. В залежності від того,
якої вони форми утворюються різні види гало. Можна спостерігати його навкруги
Місяця, Сонця і навіть вуличних ліхтарів. В нас сьогодні ось таку красу на небі
показували))

\ifcmt
pic https://scontent.fiev21-1.fna.fbcdn.net/v/t1.0-9/121382712_2701470206779072_9190064772673523381_o.jpg?_nc_cat=108&_nc_sid=8bfeb9&_nc_ohc=4GasMUYbIFAAX-PuubJ&_nc_ht=scontent.fiev21-1.fna&oh=a69ed1f97941b2fd3e45db64d41547cc&oe=5FAEA974
\fi
