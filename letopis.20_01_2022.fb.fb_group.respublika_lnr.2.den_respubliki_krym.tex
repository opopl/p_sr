% vim: keymap=russian-jcukenwin
%%beginhead 
 
%%file 20_01_2022.fb.fb_group.respublika_lnr.2.den_respubliki_krym
%%parent 20_01_2022
 
%%url https://www.facebook.com/groups/respublikalnr/posts/957997014836090
 
%%author_id fb_group.respublika_lnr,zimina_olesja
%%date 
 
%%tags istoria,krym
%%title ДЕНЬ В ИСТОРИИ. 20 января - День Республики Крым
 
%%endhead 
 
\subsection{ДЕНЬ В ИСТОРИИ. 20 января - День Республики Крым}
\label{sec:20_01_2022.fb.fb_group.respublika_lnr.2.den_respubliki_krym}
 
\Purl{https://www.facebook.com/groups/respublikalnr/posts/957997014836090}
\ifcmt
 author_begin
   author_id fb_group.respublika_lnr,zimina_olesja
 author_end
\fi

ДЕНЬ В ИСТОРИИ. 20 января - День Республики Крым

20 января отмечается День республики Крым. Эта дата установлена законом РК \enquote{О
праздниках и памятных датах в Республике Крым} и учреждена в честь
восстановления Крымской автономии по результатам референдума 20 январе 1991
года.

\ii{20_01_2022.fb.fb_group.respublika_lnr.2.den_respubliki_krym.pic.1}

Тогда, более четверти века назад, прошел первый всекрымский референдум. На нём
жители высказались за восстановление Крымской автономии.

Референдум о государственном и правовом статусе Крыма состоялся 20 января 1991
года и стал первым в истории Советского Союза. Плебисцит провели на основании
\enquote{Временного положения о референдуме и порядке его проведения на территории
Крымской области} и \enquote{Декларации о государственной и правовом статусе Крыма},
которые Крымский областной Совет народных депутатов принял в ноябре 1990 года.

В частности, в декларации отмечалось: \enquote{Крымский областной Совет народных
депутатов считает указ президиума Верховного Совета СССР от 30 июня 1945 года и
Закон РСФСР от 25 июня 1946 года, упразднившие Крымскую АССР,
неконституционными и заявляет о праве народов Крыма на воссоздание
государственности в форме Крымской Автономной Советской Социалистической
Республики как субъекта Союза ССР и участника Союзного договора}.

На референдум был вынесен вопрос: \enquote{Вы за воссоздание Крымской Автономной
Советской Социалистической Республики как субъекта Союза ССР и участника
Союзного договора?} Положительно на него ответили 1 млн 343 тыс 855 человек
(93,26\%). Всего же в референдуме участвовали 81,3\% крымчан, обладавших правом
голоса.

Итоги такого волеизъявления предусматривали воссоздание на полуострове Крымской
АССР, которая на момент упразднения находилась в составе Российской Советской
Федеративной Социалистической Республики. Однако Верховный Совет Украинской
ССР, в состав которой в момент проведения референдума входил Крым, 12 февраля
1991 года принял закон \enquote{О восстановлении Крымской Автономной Советской
Социалистической Республики}, которым, признав референдум и его результаты,
оставил Крым в составе УССР.

Согласно данному Закону высшим органом государственной власти на территории
Крымской АССР временно (до принятия Конституции Крымской АССР и создания
конституционных органов государственной власти) был признан Крымский областной
Совет народных депутатов.

Традиционно в День Республики всех крымчан с праздником поздравляет руководство
региона, а по всему Крыму проходят различные праздничные мероприятия и акции,
направленные на воспитание у всех жителей Крыма чувств патриотизма, любви и
гордости за свою малую Родину.

С 18 марта 2014 года политический статус Автономной Республики Крым – субъект
Российской Федерации – Республика Крым. Крым наравне с городом Севастополем
вошел в состав России. Соответствующий документ подписали президент России В.В.
Путин, руководители Крыма и мэр Севастополя.

16 марта 2014 года в Крыму был проведен референдум, на который были вынесены
два вопроса: \enquote{Вы за воссоединение Крыма с Россией на правах субъекта Российской
Федерации?} и \enquote{Вы за восстановление действия Конституции Республики Крым 1992
года и за статус Крыма как части Украины?}. Более 96\% избирателей проголосовали
за вхождение в состав РФ. 17 марта Крым был провозглашен суверенным
государством – Республикой Крым – а власти Крыма обратились к РФ с предложением
о принятии ее в состав федерации. Что впоследствии и произошло.

Таким образом два новых субъекта – Республика Крым и город федерального
значения Севастополь – вошли в состав России, на их территории действуют
российские законодательные акты. А сама дата 18 марта является официальным
праздником – Днем воссоединения Крыма с Россией.
