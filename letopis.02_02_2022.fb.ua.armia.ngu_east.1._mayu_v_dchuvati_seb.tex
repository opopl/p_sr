%%beginhead 
 
%%file 02_02_2022.fb.ua.armia.ngu_east.1._mayu_v_dchuvati_seb
%%parent 02_02_2022
 
%%url https://www.facebook.com/ngu.east/posts/pfbid02zUyhyA55Pf9RQtwR8g51vJ85yJSmrjxx9CvdVriyVVVWfNSrpWVh7B8F3MbT8Lfrl
 
%%author_id ua.armia.ngu_east
%%date 02_02_2022
 
%%tags mariupol,teatr,armia,kultura
%%title "Маю відчувати себе здатним захищати Батьківщину": як гвардієць з Маріуполя став актором театру
 
%%endhead 

\subsection{\enquote{Маю відчувати себе здатним захищати Батьківщину}: як гвардієць з Маріуполя став актором театру}
\label{sec:02_02_2022.fb.ua.armia.ngu_east.1._mayu_v_dchuvati_seb}

\Purl{https://www.facebook.com/ngu.east/posts/pfbid02zUyhyA55Pf9RQtwR8g51vJ85yJSmrjxx9CvdVriyVVVWfNSrpWVh7B8F3MbT8Lfrl}
\ifcmt
 author_begin
   author_id ua.armia.ngu_east
 author_end
\fi

«Маю відчувати себе здатним захищати Батьківщину»: як гвардієць з Маріуполя
став актором театру

Сьогоднішня розмова присвячена новому поколінню гвардійців, які руйнують
стереотипи та не бояться самовираження. Спілкуємось з хлопцем, якому вдається
поєднувати службу з грою у театрі.

Страший солдат Віталій Кайсаров проходить службу за контрактом у патрульному
підрозділі військової частини 3057. Майже весь свій вільний від військової
справи час хлопець проводить у Першому недержавному театрі Донеччини «Терра
Інкогніта». 24-річний гвардієць розповів, як йому вдається поєднувати мистецтво
та службу у прифронтовому регіоні.

– Чим займався до служби у Нацгвардії?

– До служби я заробляв на життя тим, що займався озвучуванням відео, рекламних
та промороликів для різноманітних компаній, але крім цього завжди був у захваті
від сцени.

– Звідки в тебе взялася пристрасть до сцени?

– Все почалося ще у шкільні роки, коли мій клас об'єднали з паралельним класом
і я познайомився зі своїм новим однокласником. Ми одразу знайшли спільне
захоплення – це музика. Пізніше це переросло у щось більше ніж просто
захоплення - через деякий час ми створили свою музичну групу. Спочатку це були
дрібні виступи у межах Маріуполя, потім їздили з концертами по Донецькій
області і, навіть, в Києві виступали. Але, напевно, першу популярність нам
принесли виступи у нашій школі. Звідти в мене, мабуть, і з'явилось бажання
вийти на сцену. 

– А як ти потрапив до театру?

– З початком війни музика відійшла у нас з першого плану. Місто було, так би
мовити, шоковане від того, що відбувалося і багато талановитих й перспективних
музикантів виїжджали до інших міст. Наш гурт розпався. Але ми не зневірились і
знайшли себе у кінематографі: мій товариш –  режисером, а я – у якості актора.
Знімали ми короткі ролики, здебільшого це були кумедні випадки з життя у жанрі
фільмів з Леслі Нільсеном – тупо, але смішно. І одного дня, на зйомку до мого
товариша прийшла дівчина, яка, як з'ясувалось пізніше, працювала хореографом у
театрі. Ми трохи поспілкувались, роззнайомились показали свої відео. Пройшов
якийсь час, мені телефонує ця дівчина і каже, що показала ті ролики своєму
режисеру і тому сподобалась моя гра, і чи не хочу я спробувати себе у театрі. Я
погодився. Прийшов туди, зіграв пробну роль і режисеру схвалив. Так я і
потрапив до театру.

– Яку останню роль ти грав?

– Це була роль охоронця у комедію «Дорослі ігри завзятих дівчат» за п'єсою
драматургині Наталі Уварової. Зараз готуюсь до образу Гумберта у спектаклі за
романом Набокова «Лоліта». Досить складна роль, до підготовки якої треба
докласти чимало зусиль, але думаю впораюсь.

– Якісь кумедні чи казусні випадки траплялися на сцені? 

– Ну, я таким багатим досвідом ще не володію, але з кумедного можна виділити
моменти, коли тобі складно утриматись від сміху глядача. Буває під час виступу
глядачі дуже йдуть на контакт і починають сильно сміятись, що я розуміючи саму
комічність ситуації, не можеш стримати свій сміх. 

– Чи хвилюєшся перед виходом на сцену і як з цим борешся?

– На генеральних репетиціях, оцей останній час перед самим виступом, всі
починають лажати. Кожний забуває текст, плутає фрази. І ти розумієш, що через
годину тут вже будуть сидіти глядач і буде дивитись на тебе, а в тебе вже не
буде права на помилку, ось тоді починаєш хвилюватись. Але коли ти виходиш на
сцену і чуєш позитивні емоції від глядача, то хвилювання зникають самі собою.
Це якась магія сцени. 

– Як тобі вдається поєднувати службу та творчість?

– Мабуть, це є водночас і найцікавіша, і найскладніша задача. На репетиціях
треба багато працювати, тому, якщо я маю вільний час, то він розподілений між
дівчиною та театром. Через службу іноді важко спланувати репетиції чи виступи,
але поки що справляюсь. Приємно, що мене підтримують мої рідні, близькі та
друзі. 

– Чи не замислювався ти коли-небудь здобути вищу освіту за цим фахом?

– Так, було така ідея. Я навчався на програміста в одному з  Маріупольських
вишів, але після 2-го року навчання зрозумів, що це не моє. Забрав документи і
хотів їхати поступати до Київського національного університету театру, кіно і
телебачення на майстерність диктора та ведучого програм телебачення, але щось
пішло не так. Сам не знаю що стало поштовхом тому, але вирішив піти служити.
Прийшов до військкомату, так і сказав «Хочу служити у Нацгвардії», –і ось я
досі тут.

– Чи є творчі плани на майбутнє? 

– Далі планую виконувати свій прямий службовий обов'язок, і звичайно ж вносити
краплинку мистецтва та творчості у будні звичайних людей. Живу, намагаючись
нікого не ображати, і себе в образу не давати. У житті надихає та мотивує думка
про те, що для досягнення будь-яких результатів та мрій необхідно просто
працювати над цим. 

Розмову вів Вадим Вигінний

Військова частина 3057 Національної гвардії України

Terra Incognita свій театр для своїх
