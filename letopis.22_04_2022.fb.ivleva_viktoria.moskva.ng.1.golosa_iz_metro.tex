% vim: keymap=russian-jcukenwin
%%beginhead 
 
%%file 22_04_2022.fb.ivleva_viktoria.moskva.ng.1.golosa_iz_metro
%%parent 22_04_2022
 
%%url https://www.facebook.com/victoria.ivlevayorke/posts/10160021815850987
 
%%author_id ivleva_viktoria.moskva.ng
%%date 
 
%%tags 
%%title ГОЛОСА ИЗ МЕТРО
 
%%endhead 
 
\subsection{ГОЛОСА ИЗ МЕТРО}
\label{sec:22_04_2022.fb.ivleva_viktoria.moskva.ng.1.golosa_iz_metro}
 
\Purl{https://www.facebook.com/victoria.ivlevayorke/posts/10160021815850987}
\ifcmt
 author_begin
   author_id ivleva_viktoria.moskva.ng
 author_end
\fi

ГОЛОСА ИЗ МЕТРО

Спустилась на одну из станций харьковского метро и поговорила с жителями,
спасающимися там от российских «освободителей».

- Мы пробыли один самый первый день в бомбоубежище в школе, но там невыносимо
было сидеть, так трусИло всю школу, мы домой прибежали – стекол нет, газа нет..
Мы жили на Северной Салтовке, 5. Потом приехала полиция, а мы как мыши бегали
по подвалам то с одной стороны дома, то с другой, не знали , куда прятаться,
бухало со всех сторон. Страшно было.. снаряды летят. Самолеты. Дома горелые.
Спасибо что здесь нас приютили, а то там  бы сгинули.

- Я мало когда выхожу на улицу отсюда, потому что когда стреляют я боюсь. Я
вообще боюсь выходить, на улице может раз два была, и все. Боюсь. После такого
стресса у меня ноги отказывают, состояние здоровья очень пошатнулось. Плачу
часто. Хотелось бы верить в победу, и я верю в нее, но   столько разрушено,
столько разрушено, иногда не очень получается верить.. 

- А я успокаиваюсь, когда вяжу шарфики. Связала уже четыре соседям.

-А я просто тыкаю в телефон. Сижу то на одном боку, то на другом, и тыкаю.

- Я еще переживаю за вторую дочку, она на поселке Жуковского живет, никуда
ехать не хочет.

- Мы увидели что дома напротив начали складываться  и гореть. Две бомбочки
было, он кружил над головой и выбирал цель.

ГОЛОС ПО РАДИОСВЯЗИ:

Уважаемые эвакуированные! Нужно пять человек волонтеров на выходе в город.

- Когда начиналось все это дело, мы выходили в коридор, чтобы если стекло
разлетится, то в нас не попало. А оно над головой жужжало, очень страшно было.
Потом увидели, как пятнадцатый и шестнадцатый дома буквально сложились, а
вторую бомбу он скинул возле девятиэтажки, она рядом с домом взорвалась, так-то
ничего, но взрывной волной, видимо, газ повредило, и вспыхнуло очень большое
пламя прямо на несколько подъездов. У нас окна на юг, вот мы и смотрели.

-Люди здесь на станции все время приезжают-уезжают. В самый первый день здесь
было до двух тысяч народа, друг на дружке сидели. Сейчас приблизительно
шестьсот. Иногда мы выходим на улицу. Я вот сегодня вышел – и забежал обратно,
бахали так. А с утра было нормально.

- Кормят здесь отлично. Фельдману (владелец частного зоопарка под Харьковом)
привет. В районе четырех часов приедут узбеки, плов привезут с мясом. Они
каждый день приезжают, в первый раз привезли котел, ну а что такое котел на всю
ораву такую? Теперь два привозят. Ходить к ним нужно со своей посудой. Еще суп
привозят в Гриль-бар. Сегодня был гороховый – как полагается, с ребрышками
копчеными, волонтеры привозили, они тоже каждый день обеды привозят сюда. А
вечером вот у нас чайник, если чайку попить. Его нагреть здесь можно. 

- Дочку нашу я прямо выгоняю на улицу, дышать нужно ходить, а ей страшно. В
восемь часов комендантский час, закрывают железом двери, и никто уже не войдет
не выйдет.

- Ночью здесь спать очень холодно, здесь же бетон. Кто повыше, в вестибюле,
устроился - там теплее. Мы могли бы, наверное, принести себе из дома еще
матрасы, но мы домой не ходим, боимся. Мы на Наталье Ужвий живем, это край
города, сто двенадцатый дом. У нас там началась война, как раз по Бобровке
стреляли, мы сразу это увидели. Такой грохот был – в первый же день на северной
стороне стекла повылетали, у кого деревянные окна были, и половина дома, если
не больше, сразу уехала. Мы там ни разу с тех пор не были, с шестого числа
марта как уехали. Нам сказали солдаты, чтобы туда не ходили, там постоянные
автоматные очереди.

- Я хожу на свою квартиру смотреть каждый день, я рядом здесь живу. Она на
первом этаже, я завалила лоджию как могла, но все равно хожу проверяю. Спать
очень холодно и неудобно – смотрите, я на скамье сплю, а раньше вообще на трех
табуретках, матраса нет, сплю в одежде, заворачиваюсь в одеяло сверху еще.

- Помыться здесь можно. Голову вымыть, просто в чайнике нагреть воды. Но и в
кране вода горячая тоже бывает. Я стараюсь за собой следить хоть как-то, брови
вот каждый день рисую.

- Здесь каждое утро обязательно моют пол. И специальные краны из-под пола
вывели и установили, можно всегда попить воды.

- Здесь живет несколько собак – алабай, такса и просто. Ну и кошки. Я видел,
что кошек хозяйки часто тоже держат на поводке, чтобы не потерялись.

- Цветы мне принес сосед наш. Он на участок к себе ходит, там тюльпаны уже
расцвели, вот он их сначала принес. А потом еще и черемуху.

- Я вот сижу здесь, может, до самой зимы придется сидеть, пока их выгонят, а
мне так хочется скорее в деревню, сажать. И чтобы руки все в земле были.
