% vim: keymap=russian-jcukenwin
%%beginhead 
 
%%file 29_03_2022.fb.dykyj_evgen.1.vijna_vijna.cmt
%%parent 29_03_2022.fb.dykyj_evgen.1.vijna_vijna
 
%%url 
 
%%author_id 
%%date 
 
%%tags 
%%title 
 
%%endhead 
\zzSecCmt

\begin{itemize} % {
\iusr{Yevheniia Levcheniuk}

З росією ні про що не можна домовлятись! Ми вже це не одноразово проходили і в
цей історичний час ми не маємо права схибити. Треба душити до кінця!

\begin{itemize} % {
\iusr{Pavlo Polamarchuk}
\textbf{Yevheniia Levcheniuk} а є ресурс? Є ресурс "придушити до кінця" ядерну державу?

\iusr{Evgen Dykyj}
\textbf{Pavlo Polamarchuk} 

от в тому і унікальність ситуації, що наразі є, чи не вперше за історію. Але ж
для початку варто це припустити. А не вважати це неможливим, як до 24.02
вважали непереможною їхню армію.

\iusr{Yevheniia Levcheniuk}
\textbf{Evgen Dykyj} так, цілком згодна і особливо зараз, коли треба готуватись до нової хвилі!

\iusr{Yevheniia Levcheniuk}
\textbf{Pavlo Polamarchuk} на то є все світове співтовариство. Це не тільки наша проблема.

\iusr{Nataliia Iakovenko}
\textbf{Pavlo Polamarchuk} 

а як ця ядерна держава буде стріляти по Україні балістичними ракетами високої
дальності, розрахованими на війну з США?

З Камчатки їх буде запускати, чи що?

\iusr{Oleg Safronkov}
\textbf{Nataliia Iakovenko}, 

от це якраз не є аргументом. Тактичні ядерні заряди - приблизно до сотні
кілотон - мають величезний набір засобів доставки - включно із ракетами для
літаків (причому для запуску як зі стратегічних бомбардувальників, так навіть і
з винищувачів) та артилерійськими снарядами великих калібрів - і можуть бути
застосовані з відстані хоч і в кілька десятків кілометрів до цілі. При цьому,
щоб зрозуміти, що таке діапазон потужностей до 100 кілотон, згадайте про
Хіросіму та Нагасакі, вщент розтрощені приблизно 20-кілотонними ядерними
бомбамию

\iusr{Антарктичний Лікар}
\textbf{Nataliia Iakovenko} 

Використання тактичної ядерної зброї буде супроводжуватися важкими втратами і
деяким радіоактивним забрудненням. Безумовно, це жахливо. Втім, якщо таке
станеться, єдиним, що слід буде робити - міцніше стиснути зуби й продовжувати
боротьбу. Змінити тактику (уникати великих скупчень сил, які можна уразити
тактичним зарядом). Буде дуже важко. Але треба не втратити духу і продовжити
воювати. Інакше наслідки відмови від боротьби будуть страшнішими за ядерну
зброю. Втім, є шанс, що до цього не дійде.

\end{itemize} % }

\iusr{Pavlo Polamarchuk}
\textbf{Evgen Dykyj} 

інформація, що буде знято санкції повністю не відповідає дійсності.

Звідси і помилки в інших розмірковуваннях.

Санкції - червона лінія, санкції мають знищити росію. У грі вдовгу, не в
коротку.

\begin{itemize} % {
\iusr{Evgen Dykyj}
\textbf{Pavlo Polamarchuk} 

тобто Ви допускаєте що можлива угода на умовах, озвучених нашою делегацією у
Стамбулі, але при тому зберігається нинішній обсяг санкцій проти РФ? Цікава
гіпотеза  @igg{fbicon.smile}  Нинішні санкції привязані саме до збройної агресії РФ проти
України, притому саме до останньої фази цієї агресії, з 24.02.2022. До чого
вони будуть привязані в разі укладання угоди, і як довго зберігатимуться? І яка
підстава так вважати?

\iusr{Pavlo Polamarchuk}
\textbf{Evgen Dykyj} 

не я \enquote{допускаю думку}.

Охххх...

Наш (і, сподіваюсь, ваш) головнокомандувач прямо про це каже.

\iusr{Pavlo Polamarchuk}
\textbf{Evgen Dykyj} до факту збройної агресії. Факту.
Факт вже відбувся.
Почитайте про Іранські санкції, про поправку Джексона- Вейніка...
Яку ввели по факту порушення прав людей, зокрема євреїв, на вільну еміграцію з СРСР.
Яке потім припинилося, почалася репатріація до Ізраїлю і інших країн з СРСР, але поправка діяла до 2012 а щодо деяких екс- республік діє і до цього часу

\iusr{Pavlo Polamarchuk}
\textbf{Evgen Dykyj} 

щоби \enquote{відкатати санкції} країна-агресор має поновити статус-кво.

Репараціями, іншими механізмами.

Поки статус-кво на 24.02 чи на 18.03.2014 не відновлено - санкції будуть
працювати і посилюватися.

Життя загиблих дітей, дорослих, розтрощені мирні міста не повернеш назад, то де
логіка, що по факту відведення військ будуть припинені санкції?

Це ж основи міжнародного права.

\iusr{Maksym Kozub}
\textbf{Pavlo Polamarchuk}, технічне: мабуть, мали на увазі поправку Джексона — Веніка.

\iusr{Pavlo Polamarchuk}
\textbf{Maksym Kozub} Т9.
Вона навіть Вейніка в українському правопису)

\iusr{Oleksandr Krysan}
\textbf{Pavlo Polamarchuk}

хоч поправку і відмінили в 2012 але її з 90х регулярно призупиняли і не діяла
вона колиб з 92 чи 93

\iusr{Ігор Харченко}
\textbf{Pavlo Polamarchuk} Міжнародного права по факту більше не існує з 2014 року. Остаточно воно добите в лютому цього року.

\end{itemize} % }

\iusr{Hlib Kozak}

\enquote{для якої нема жодної підстави з точки зору військової ситуації.} -
Євгене, чому ви вважаєте, що нема жодної підстави?

\begin{itemize} % {
\iusr{Oleg Safronkov}
\textbf{Hlib Kozak}, 

може це не зовсім на Ваше питання буде відповідь, але про всяк випадок, щоб
мені не повторюватися, прошу прочитати мій камент (той із двох моїх каментів,
котрий довший) під оцим-о постом іншої людини на цю ж тему:

\url{https://www.facebook.com/talerster/posts/10222110246730949}

\iusr{Hlib Kozak}
складно знайти там ваш пост

\iusr{Oleg Safronkov}
\textbf{Hlib Kozak}, 

на конкретно поставлене Вами питання найближча до відповіді інформація є ось
тут, до речі сформульована військовими фахівцями:

\url{https://www.facebook.com/LevinYigal/posts/1382417955569642}

\url{https://www.facebook.com/LevinYigal/posts/1382538208890950}

, а мій камент - не пост, а камент - я зараз покажу.

\iusr{Oleg Safronkov}
\textbf{Hlib Kozak}, мій камент (скріншот) тут:

\ifcmt
  ig https://scontent-mxp1-1.xx.fbcdn.net/v/t39.30808-6/277555937_4974613109294115_8171024742863764123_n.jpg?_nc_cat=102&ccb=1-5&_nc_sid=dbeb18&_nc_ohc=AsQj0DRd8OMAX8RNHIB&_nc_ht=scontent-mxp1-1.xx&oh=00_AT__oMpzwPAmVl3PfmVzn0zCbTda9qsARpMAvUMPYXTo9w&oe=626C8D30
  @width 0.3
\fi

\iusr{Evgen Dykyj}
\textbf{Hlib Kozak} 

тому що на сьогодні військова ситуація залишається вкрай складною, однак в
цілому перевага на українській стороні, і з кожним днем ця перевага повільно,
але зростає. Докладний опис чому саме так зайняв би дуже багато тексту, але в
принципі це багато де обговорюєтся і можна знайти. Наразі ми маємо унікальний
історичний шанс на повноцінну військову перемогу над РФ, напевно вперше з часів
поразки Мазепи, яка забезпечить і абсолютно іншу позицію на перемовинах, і
запуск потрібних нам політичних процесів у РФ. Зірвати цю перемогу можуть лише
а) застосування РФ ядерної зброї (навіть хімічної не достатньо), або ж б)
передчасні мирні угоди на невигідних нам умовах.

\end{itemize} % }

\iusr{Іван Макар}

Всім, хто збирається обговорювати такі питання, я б радив хоча б переглянути
підручники з історії держави і права, теорії держави і права, державного права
зарубіжних країн!

А то таке виникає враження, як обговорення першокласниками диференціального та
інтегрального числення!

\begin{itemize} % {
\iusr{Stanislav Shumlianskyi}
\textbf{Іван Макар} у них тисячі лайків і сотні репостів, нащо це їм, смішний Ви якийсь

\iusr{Іван Макар}
\textbf{Stanislav Shumlianskyi} 

\obeycr
Хіба не жах: своєї зброї
Не маєш ти в ці скорбні дні...
У тебе так: два-три герої,
А решта — велетні дурні.
\restorecr

\iusr{Evgen Dykyj}
\textbf{Stanislav Shumlianskyi} це Ви наприклад про мене? рілі?

\iusr{Stanislav Shumlianskyi}
\textbf{Evgen Dykyj} конкретно про цей Ваш пост - на жаль так

\iusr{Evgen Dykyj}
\textbf{Stanislav Shumlianskyi} 

тобто Ви всерйоз переконані що мною керує бажання зібрати лайки та репости на
фоні некомпетентності, а не інші фактори - наприклад, 33 роки участі у
українському національно-визвольному русі, досвід кількох воєн та революцій,
певний рівень експертизи та аналітики (наприклад авторство підручнику з
гібридних воєн, за яким навчають офіцерів країн Балтії), непогане володіння
інфою щодо сучасної ситуації (далеко не лише з відкритих джерел), та бажання не
допустити програшу цієї вирішальної для країни війни нашими малокомпетентними
політиками? Ви абсолютно впевнені в тому що правильно оцінили мої мотивації та
рівень? А також свої відповідні досвід та компетенції для такої оцінки?

\iusr{Larysa Chornopyska}
\textbf{Evgen Dykyj} це провокатори оба, які працюють давно на Рашу. Не дивуйтеся

\iusr{Oleg Safronkov}
\textbf{Larysa Chornopyska}, це Іван Макар провокатор і працює на рашу? Ви з його біографією ознайомитися не пробували?

\iusr{Stanislav Shumlianskyi}
\textbf{Evgen Dykyj} 

«Гібридна війна» - то натівська категорія, дуже раціональна, англо-германська
якась. Штати і Нато накладали це на те, що Росія робить, і весь час програвали
їй, не могли «вхопити», бо одне «ядерне», а друге -«хімічне» - ну, я умовно,
звісно. Оце тільки останню віійну - перед 24-м лютого нарешті в Росії виграли.
Може когось з росіян залучили, хтозна. Те, що Росія робить - це пєлєвінщина,
теорія віртуальних реальностей і т.д., навіть слова нелегко підібрати, щоб
«запеленгувати» (і, отже, відбити) - «фейк», «деза», «наратив» - користуємося,
ну але так соб, як воду ситом. Я насправді досвіду того всього не маю, так -
поруч пробігав, давно дуже - тому про переговори, скажімо, лише відблиски якісь
бачу. Що там їхня сторона мутить, що наша - не знаю, бо треба на їхньому місці
бути, всередині процесу, а не зовнішнім спостерігачем. Розумію просто що наші
роблять все правильео. Ви ж, умовно кажучи, вриваєтесь посеред операції і такий
«аааа! Убійци!!! У вас руки в крові!». Ну їм нічо, навіть не помітять. Ні ви ні
я на владу зараз взагалі не впливаємо - воєнний стан - це як авторитаризм
майже, конституційно і хронологічно обмежений). Ми з суспільством працюємо - з
настроями людей - щоб не кіпішували, не тривожились, не множили напругу-тертя,
допоки ті, хто воює, ті, хто воює на переговорах - робили свою справу. І кожен
з нас, хто де - на своїх місцях теж. Щоб ми разом, спільно всією країною
боролись, а не по сторонах підозріло поглядали - хто що не так поруч робить.
Уявіть тепер, що Ваш пост людина в окупованому Херсоні прочитала. Просто її
очима погляньте - який в неї буде настрій, стан. Вам (я знову умовно) життя
багато набоїв видало, може й стінгера якогось - а Ви - по своїх.

\end{itemize} % }

\iusr{Марія Берлінська}

ще невідомі ніякі умови, нічого навіть близько не підписано, в ЗМІ дуже різна
інформація, але пости 'великої зради' вже набувають популярності.

ніякої серйозної аналітики, просто \enquote{зрада}

національний вид спорту.

\begin{itemize} % {
\iusr{Oleg Safronkov}
\textbf{Марія Берлінська}, 

та ні, просто навіть я зі своїм політдосвідом - а в Дикого він іще набагато
більший за мій - в курсі, що таких "якісних" політиків, як у нас, у жодній,
крім не-балтійських пострадянських країн, країні Європи та Північної Америки у
правлячих партіях за останні 100 років не траплялося:

\url{https://www.facebook.com/oleg.safronkov/posts/4715894681832627}

\url{https://www.facebook.com/oleg.safronkov/posts/4507090512713046}

\url{https://www.facebook.com/oleg.safronkov/posts/4377942428961189}

, а відтак навіть найменшій довірі до цих кадрів просто нема де взятися.

\iusr{Evgen Dykyj}
\textbf{Марія Берлінська} 

особисто я користувався виключно пубілчними заявами учасників офіційної
делегаії України панів Арахамії та Чалого та офіцйного речника ОП пана
Подоляка. Мені цього більш ніж вистачає, нема потреби у \enquote{дуже різній}
інформації з медіа.

\iusr{Pavlo Polamarchuk}
\textbf{Мария Берлинская} підтримую.
Є червоні лінії, і немає жодної експертизи, що їх перейдено.

\iusr{Oleg Safronkov}
\textbf{Pavlo Polamarchuk}, чи вам, як помічнику-консультанту у ВР, треба пояснювати, наскільки керованою є державна українська експертиза?

\iusr{Larysa Chornopyska}
\textbf{Марія Берлінська} 

а просто почитати результати переговорів і трошки проаналізувати ви не в стані?
То треба 10 академій закінчувати, зрозуміти такі красномовні речі?....

\iusr{Ірина Оснач}
\textbf{Марія Берлінська} 

Пропозиції переговорної групи від України:

\url{https://www.facebook.com/photo.php?fbid=3026136041033620&set=p.3026136041033620&type=3}

Член делегації України Валерій Чалий пояснив, що означатиме український нейтралітет:

 @igg{fbicon.red.triangle.pointed.down}  Відмова розміщувати іноземні військові бази та військові контингенти.

 @igg{fbicon.red.triangle.pointed.down}  Відмова вступати у військово-політичні спілки.

 @igg{fbicon.red.triangle.pointed.down}  Військові навчання на території України зможуть проводитись лише за згодою
країн-гарантів.

 @igg{fbicon.red.triangle.pointed.down}  Для України відкриті двері до ЄС та країни-гаранти повинні будуть це
підтримувати та зобов'язуються цьому не заважати.

 @igg{fbicon.speech.baloon}  Чалий також заявив, що якщо остаточних домовленостей буде досягнуто на
високому рівні, готуватиметься багатостороння конференція з вищими посадовими
особами країн-гарантів, на якій буде підписано цей договір.

Еміль Дубров

Що саме цікаво, якраз Олександр Чалий, який з 22 березня 1993 р. по 12 травня
1995 р. — займав посаду начальника Договірно-правового управління МЗС України,
герой сьогоднішніх переговорів, свого часу був і головним творцем "ялового"
Будапештського меморандуму!..

\iusr{Oleg Safronkov}
\textbf{Ірина Оснач} , 

Валерій Чалий - дипломат у відставці з 2019 року. Олександр Чалий - діючий
дипломат ще з часів Кучми, нині введений ОП до складу переговорної делегації.
Ви читати не вмієте і не можете прочитати про них хоча б статті у вікі, чи
навіть імена не розрізняєте?

\iusr{Stanislav Shumlianskyi}
\textbf{Мария Берлинская} 

альфред кох влучно про це: «така кількість перемінних у цьому у рівнянні, що на
виході може бути все - від катастрофи до тріумфу»

\end{itemize} % }

\iusr{Ігор Харченко}

Міжнародні гарантії нічого не варті без власної спроможності не тільки \enquote{закрити
небо}, а й завдати агресору таких непоправних втрат, після яких він не зможе
продовжувати наступальні дії.

Єдиними повноцінними гарантами збереження української нації і країни є ЗСУ,
спецслужби та інші воєнізовані формування включно з в обов'язковому порядку
озброєними усіма громадянами України.

\iusr{Pavlo Polamarchuk}

Звідки інформація, що повністю не відповідає дійсності, про зняття санкцій з
країни агресора?

Санкції - червоні лінії.

Як і \enquote{демілітаризація}.

\iusr{Stanislav Shumlianskyi}
У Стамбулі не домовились ні про що

\iusr{Oleg Safronkov}

Значицця, так, панове, хто там один одного та Євгена Дикого обвинувачує у
непрофесіоналізмі, невігластві, не-експертності, ще чомусь: є кар'єрний
дипломат Валерій Олексійович Чалий, з дипломатичним стажем порядка 20 років,
зараз у відставці, в яку пішов з посади посла у США, дипранг Надзвичайний і
Повноважний Посол. Він не став розлого коментувати результат переговорів,
поставив лише десяток формальних питань з позиції кар'єрного дипломата, на тему
\enquote{хто, з якими повноваженнями, з яким юридичним оформленням, та з якими
службовими джерелами інформації, а також за якою процедурою та в якому порядку
ці переговори веде}. Ці формальні питання назвав у зовсім короткому ось цьому
пості:

https://www.facebook.com/permalink.php?story_fbid=4798022416976266&id=100003056659575

. Відповідей він не давав - вони кожному очевидні.

Зі своєї позиції, на базі очевидних відповідей на ці його питання, констатую
наступне: це точно той самий підхід, який я описав ось у цьому пості з зовсім
іншого тоді приводу:

\url{https://www.facebook.com/oleg.safronkov/posts/4377942428961189}

Тобто у черговий раз як на професіоналізм, знання та вміння допущених до
важливої роботи (в даному випадку дипломатичної) осіб, так і на дотримання норм
права (як національного, так і міжнародного) грубо наплювали. Бо порівняно з
вірнопідданістю вождю, ні те ні те нічого не варте - така вже традиція.

\begin{itemize} % {
\iusr{Ірина Оснач}
\textbf{Oleg Safronkov} - Валерия Иваненко

Переговори Росії та України у Стамбулі. Що там робить Олександр Чалий?

Цим питанням задається відомий український політолог Олексій Гарань у своєму
блозі \enquote{НВ погляди}.

На його думку, Олександр Чалий — «знакова» постать української дипломатії —
багаторічний прихильник українського нейтралітету і противник вступу до НАТО.

14 лютого 2020 року він опинився в центрі скандалу на Мюнхенській безпековій
конференції, підписавши разом з російським проурядовим центром 12 кроків для
зміцнення безпеки України, в одному з пунктів якого йшлося про «запуск нового
національного діалогу щодо ідентичності», який мав врахувати погляди, зокрема,
Росії та Угорщини на «теми історії, національної пам’яті, мови, ідентичності і
досвід меншин».

У підсумку, Гарань задається питанням, звідки у нинішньої влади талант витягати
з нафталіну саме такі карикатурні постаті і як таку людину можна допускати до
самого серця секретних переговорів?

- Віктор Шишкін

Саша Чалий як дзеркало української капітуляції.

- Nila Voroshilova

Саша Чалий вже \enquote{гарно} попрацював над Будапештським меморандумом, тепер
працює над Стамбульським.

\iusr{Oleg Safronkov}
\textbf{Ірина Оснач}, 

яке відношення кучмівський придворний дипломат Олександр Чалий, про якого
пишете Ви, має до дипломата у відставці Валерія Чалого, про якого написав я?
Між іншим, мотивом відставки Валерія Чалого була саме незгода з дикими
принципами кадрового підходу до дипломатичної служби після обрання Зеленського.

Якщо ж Ви не переплутали імена цих двох однофамільців-дипломатів, а справді
цікавитеся питанням \enquote{що ці нафталіново-проросійські кадри роблять у
Зеленського}, то на це питання я якраз можу відповісти. У Зеленського значна
частина діючих та екс-придворних кадрів, починаючи хоча б з екс-спікера
Разумкова, класифікувалися сторонніми спостерігачами як \enquote{младорегіонали}. Тобто
зі старим кадровим ПР-складом, таким як Татаров, Ахметов зі Шмигалем, Порошенко
тощо, ці \enquote{младорегіонали} співвідносяться як колишній ВЛКСМ із колишньою КПСС.
Причому і за тими й за тими стоїть один і той самий корпус старих кучмівських
олігархів, тісно пов'язаних своїми бізнесами з російськими олігархами та
організованими злочинними угрупуваннями. Далі все ясно, сподіваюся?

\end{itemize} % }

\iusr{Степан Сивик}

Україною повинні керувати українці з гарячим серцем, холодним розумом,
сталевими яйцями і лютою ненавистю до ворога!

\iusr{Ірина Оснач}
\textbf{Степан Сивик} Чому тоді укранці голосують за ставлеників олігархату?

\iusr{Sunny Smith}
Це тупо злив країни народу і ганьба від всього світу....  @igg{fbicon.man.facepalming} 

\iusr{Олесь Шевченко}
Dyakuyu, pane Evhene!

\iusr{Максим Кречетов}
Брехня і здрадофілія

\iusr{Іван Макар}

Хотів би запитати у автора посту \textbf{Evgen Dykyj}: а чим в анонсованій угоді
передбачено зняття санкцій з РФ?!

\begin{itemize} % {
\iusr{Stanislav Shumlianskyi}
\textbf{Іван Макар} чи є в угоді, якої ще немає - я би додав). Це ж не проект ще навіть - а розмови про проект

\iusr{Іван Макар}
\textbf{Stanislav Shumlianskyi} отож! Мені дивно, що колишній айдарівець Evgen Dykyj активно сприяє ворожій пропаганді!

\iusr{Oleg Safronkov}
\textbf{Іван Макар}, 

Ви людина шанована усіма, хто про Вас знає, і мною в тому числі. Але я все ж не
вважаю, що це є підставою для блазнювання, бо виходить майже так само невдобно,
як на скріншоті:

\ifcmt
  ig https://scontent-mxp1-1.xx.fbcdn.net/v/t39.30808-6/277524055_4979716122117147_5924701021453166793_n.jpg?_nc_cat=106&ccb=1-5&_nc_sid=dbeb18&_nc_ohc=UKaAF8BJruoAX-XzM-d&_nc_ht=scontent-mxp1-1.xx&oh=00_AT-g4bRJDs6XDRkAkgtelKKKprpj99Xg9FcEIzVgBMTaaQ&oe=626C8390
  @width 0.3
\fi

\end{itemize} % }


\end{itemize} % }
