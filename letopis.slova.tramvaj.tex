% vim: keymap=russian-jcukenwin
%%beginhead 
 
%%file slova.tramvaj
%%parent slova
 
%%url 
 
%%author_id 
%%date 
 
%%tags 
%%title 
 
%%endhead 
\chapter{Трамвай}

%%%cit
%%%cit_head
%%%cit_pic
%%%cit_text
Але з часом, коли ви здобудете собі якесь помітне місце в світі, ми, можливо, й
повернемось до сьогоднішньої розмови.  Через кілька хвилин я вже наздоганяв
кембервельський \emph{трамвай} у тумані листопадового вечора, рішуче певний, що не
мине й дня, як я знайду нагоду вчинити якесь геройство, гідне моєї прекрасної
дами. Але хто б в усьому світі міг уявити, яких неймовірних форм набуде це
геройство і яким дивовижним шляхом прийду я до нього!  Цей вступний розділ, як,
може, здається комусь із читачів, не має ніби нічого спільного з подальшою моєю
розповіддю. Але ж без нього не було б і самої розповіді. Бо порвати, як то
зробив я, з усім своїм попереднім життям і поринути в таємниче, загадкове,
невідоме, де на тебе чекають великі пригоди й велика нагорода, здатний тільки
той, хто пройнятий певністю, що героєм можна бути скрізь, тільки той, хто
докладає всіх зусиль до героїчного чину
%%%cit_comment
%%%cit_title
\citTitle{Утрачений світ}, Артур Конан Дойл
%%%endcit
