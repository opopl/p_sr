% vim: keymap=russian-jcukenwin
%%beginhead 
 
%%file 10_08_2023.stz.news.ua.mrpl.0629.1.ljudmyla_zavalej
%%parent 10_08_2023
 
%%url https://www.0629.com.ua/news/3642625/smatocok-mariupola-u-centri-stolici-istoria-ludmili-zavalej-aka-budue-viziu-povernenna-u-misto-foto
 
%%author_id news.ua.mrpl.0629
%%date 
 
%%tags 
%%title Шматочок Маріуполя у центрі столиці. Історія Людмили Завалей, яка будує візію повернення у місто, - ФОТО
 
%%endhead 
 
\subsection{Шматочок Маріуполя у центрі столиці. Історія Людмили Завалей, яка будує візію повернення у місто}
\label{sec:10_08_2023.stz.news.ua.mrpl.0629.1.ljudmyla_zavalej}
 
\Purl{https://www.0629.com.ua/news/3642625/smatocok-mariupola-u-centri-stolici-istoria-ludmili-zavalej-aka-budue-viziu-povernenna-u-misto-foto}
\ifcmt
 author_begin
   author_id news.ua.mrpl.0629
 author_end
\fi

\begin{qqquote}
Людмила Завалєй разом з іншими маріупольськими \enquote{Берегинями} відкрила шматочок
Маріуполя у самому центрі Києва на провулку Музейному 8Б. Коворкінг
\enquote{BEREHYNY space}, який вона з командою започаткувала,  - це результат
її власного ПТСРа, отриманого внаслідок пережитого в Маріуполі, і мрії
повернутись назад, після деокупації. 
\end{qqquote}

\enquote{Мені потрібно було щось робити, бути корисною. Цією справою я лікую себе від
стресу. Робота – це єдине, що допомагає багатьом маріупольцям зараз впоратись
із ситуацією}, - каже Міла.

\ii{10_08_2023.stz.news.ua.mrpl.0629.1.ljudmyla_zavalej.pic.1}

Людмилу Завалей добре знають у волонтерському середовищу, хоча вона сама
зізнається, що дуже довго була \enquote{за чоловіком}. Її чоловік, лікар Андрій
Пазушко, з перших днів російського вторгнення, ще у 2014 році, долучився  до
волонтерського середовища, яке на той час називалось Новий Маріуполь. А Людмила
просто допомагала йому коли це було потрібно. 

\enquote{Прийняти, посортувати,  допомогти розвезти тому, кому це було потрібно та
адресувалось. Наша машина, яку я лагідно називала  \enquote{Сінічка}, де тільки не
побувала за 8 років. Під час таких подорожей я побачила людей у тих селах, які
потребували допомоги, з якими можна було працювати і які серцем були за
Україну}, - пригадує Людмила Завалей.

Вона вже мала великий досвід організаторської роботи, в тому числі і на
державній службі. За фахом Людмила – бухгалтер, фінансист. І до декретної
відпустки багато попрацювала на різних посадах.

"Мені дуже пощастило працювати у таких державних підрозділах, які допомагали
дійсно змінювати країну. Коли в 2014-м розпочались всі ці проросійські танці з
бубнами, у податковій був організований новий підрозділ –
інформаційно-довідковий департамент. Ось там я і працювала. Ми займались
практичними консультаціями підприємців – як їм виживати в тих складних і
абсолютно незрозумілих умовах. Реальна допомога реальним людям.

Офіс у нас був у Кальміуській райадміністрації. І я бачила на власні очі, як
над будівлею по декілька разів на день змінюються прапори – то українські, то
\enquote{денеерівські}. Я тоді телефонувала в керуючий  підрозділ, який в той час
знаходився в Донецьку, попереджала про загрозу захоплення, волала про
необхідність вивозити документацію. А мені відповідали там: \enquote{Ну шо ти
кипішуєш. Все нормально}. 

Звісно, ніхто ніякі документи так з Донецька і не вивіз. Хоча можливість була.
Я звільнилась. Після цього мала дуже цікавий досвід роботи в офісі великих
платників податків. Ми працювали з підприємствами, які потрапили в окупацію,
але сплачували зарплати людям у гривні і платили податки в Україні. Цікава була
робота. Поштового зв'язку з окупованими територіями Донецької області вже не
було, а платники податків були, всі документи передавались потайки кур'єрами. 

Після економічної блокади окупованої частини Донеччини, наша робота втратила
сенс".

Тож Людмила Завалей пристала на пропозицію попрацювати в
«Маріупольтепломережі». Це найбільш складне комунальне підприємство в
Маріуполі, нереформоване, немодернізоване, із великою корупційною складовою.
Вже нова команда у міськраді, яка складалася переважно з менеджерів Метінвесту,
запросила Людмилу Завалей в якості кризового менеджера, разом з новою командою,
на підприємство.

\enquote{Я протягом шести місяців вивчала ситуацію на підприємстві,  розробила план
модернізації \enquote{Маріупольтепломережі}, намагалася впоратися з багатомільйонними
боргами, які залишили попередники в умовах непрацюючого на той час закону про
реструктуризацію боргів, але після вивчення моїх пропозицій мене звільнили.
Пізніше новий директор підприємства Мірошниченко, озвучував тези модернізації з
плану, який розробила наша команда за пів року, вже як власні розробки}. 

Після такої стресової роботи Людмилою було прийнято рішення про організацію
власної справи, вона захотіла віддавати більше часу громадському сектору. Після
декількох років роботи з Дмитром Чичерою, з Мариною Пугачовою, в 2020 році
з’явилась \enquote{Берегиня. Павлопілля}. Це одне із селищ, яке знаходилось майже на
самій лінії зіткнення, яке багато хто називав вимираючим, але там була активна
проукраїнська спільнота., яка хотіла відновити колись багате село. Тому і
заснували там.

\enquote{За роки співпраці з Дмитром Чичерою та його \enquote{Халабудою} була рідна команда,
яка допомагала реалізовувати ініціативи, було розуміння створення
культурного-освітнього простору, була волонтерська \enquote{бульбашка}, і це дуже
допомагає зараз, під час організації роботи \enquote{Берегиня space}. Але  мені дуже не
вистачає зараз поруч Дмитра}, - зізнається Людмила Завалей.

Вона бачилась з ним востаннє 14 березня, зустрічались у \enquote{Халабуді}.

"Ми ще на початку повномасштабного вторгнення поділили місто на зони
відповідальності. Тому що через обстріли важко було пересуватись. Я відповідала
за Приморський район, Черьомушки. Моя дорога в центр міста раз на декілька днів
виглядала так: завершився черговий обстріл, значить, у мене є хвилин 30, щоб
сісти в машину і летіти до гуманітарного штабу. І так само у зворотному
напрямку.

Одного разу я вимушена була шукати тимчасове укриття, тому що за ці 30 хвилин
перерви не встигла дістатись місця. Тож заїхала у санаторій-профілакторій
\enquote{Чайка}. Я була шокована тим, що побачила там. Людей з інвалідністю, старих із
будинку ветеранів ніхто не евакуював, і медичний персонал, медсестрички на
каталках, по одному перевозили самотужки цих людей в укриття. У них там не було
їжі, води. Це був просто жах..."

Відсутність евакуації в перші дні війни – це одне із найболючіших питань, які
ставить Людмила Завалей перед владою.

"Насправді, я знала, що буде війна, і що вона буде не такою, як у 2014-му, теж
здогадувалась. Тому коли все почалося, ми з чоловіком одразу відправили дітей
разом із мамою подалі від Маріуполя. Але для себе було прийнято рішення
залишатись стільки скільки зможемо бути корисними.

Чоловік разом з дружньою організацією Маріупольська асоціація жінок \enquote{Берегиня}
взяв участь у евакуації  жінок, дітей та літніх людей, які були в зоні ризику
через різні підстави. Андрій – лікар, тож він мав супроводжувати евакуаційний
автобус до Польщі, а потім повернутися по наступний евакуаційний рейс. Я була
впевнена, що залишиться хоча б якісь коридор для виїзду із Маріуполя,
сподівалася, що ми, волонтери, зможемо бути корисними для наших захисників, і
для цивільних теж. Але сталося страшне.

Мені пощастило купити в одній із аптек на початку березня медичні аптечки,
досить непогані, в яких було все необхідне для надання екстреної медичної
допомоги. Я забрала всі ці аптечки і розвезла по найбільших цивільних сховищах
в Приморському районі. 

\enquote{Черема} спочатку здавались безпечними, але потім там почалось страшне. Там
взагалі була срака. Приїжджаємо, дивлюсь, а там один з будинків на Латишева,
вже не пам'ятаю номеру, прилетіло всередину. Дивлюсь, стоять пожежні,
дивляться, а воно горить. Ніхто не гасить, ніхто не метушиться. І я така, а що
ви стоїте? Вони кажуть, а води ж немає. І все, і стоять. Я кажу, може там... Ну
ти розумієш. А вони мені,  ні, ні, ні, евакуювати нікого не треба. Всі
евакуювалися, хто міг.

Біля мого будинку в квітні впав снаряд, так бетонну відмостку метрову просто
вирвало зі стіни.

Те, що Лівобережжя пережило у лютому-березні, в порту та на Кіровці розпочалось
у квітні. Розх...ли мікрорайон повністю. Біля мого будинку ще пара будиночків
вціліла, а навколо – пустеля.

Знаєш, я ж 24 лютого бігала по поселку де я мешкала, вмовляла людей
евакуюватися, а вони дивилися на мене як на божевільну. А тепер вже самі були,
як божевільні.

Я бачила стареньку жінку, яка сиділа біля свого зруйнованого вщент дому, і
причитала: \enquote{Чому мене не завалило... Хочу здохнути тут, біля свого будинку...}.
Багато такого я бачила.

Я впевнена, в Маріуполі не менше 100 тисяч загинуло. Я на власні очі бачила
тисячі тіл. Так я ж поїхала у березні, а після ще більше місяця бомбили".

14 березня чоловік Людмили Завалєй зміг прорватися в місто, і сказав що можна
виїжджати, є коридор.  Автобус евакуаційний який їхав для евакуації в місто
заїхати не зміг, по евакуаційному коридору лежали нерозірвані снаряди і міни.
Він із Польщі віз гуманітарку для маріупольців  – її росіяни та денеерівці
забрали. Але головне, що він живий.

Скрізь одне з селищ Людмила з друзями виїхали з міста і дісталися наступного
населеного пункту, там ще чекали людей, кого б могли забрати із собою, і рушили
на Запоріжжя, потім Дніпро і далі – Івано-Франківськ. 

Два місяці, які Людмила Завалей разом з родиною, друзями, собаками, котами,
черепашками провела у друзів в Івано-Франківську, жінка називає реабілітацією.

«Ми просто жили. Намагались не думати ні про що. Не планували свій день. Ми
відходили від всього, що довелося пережити, передивитись у Маріуполі. В певний
момент я зрозуміла, якщо не повернусь до справи, збожеволію. Тому весь цей
простір – це результат мого власного посттравматичного синдрому, точніше,
спроба його позбавитися.

Ми отримали грант від німецької організації «AMICA», яка підтримує жінок, що
потрапили у біду. За ці гроші ми з дівчатами з Маріуполя, часткою команди
Халабуди, яка тут знаходиться – Галею Балабановою, Ірою Кондратенко та іншими,
нашими маріупольськими волонтерами змогли орендувати приміщення та зробити його
«своїм» настільки, наскільки це можливо. Обирали його колегіально. Юля Паєвська
(Тайра), яка є частиною нашої команди та нашого простору, сказала, що під час
Революції Гідності у цьому дворі на вулиці Музейній, 8б,  були сутички з Омоном
і наші відстояли урядові квартали. Тож для мене це місце виявилось символічним
таким.

Ми його перефарбували, підлаштували під свої потреби і 1 червня розпочали
роботу. Тепер тут є коворкінг, невеличка зала для зустрічей та презентацій,
зручна кухня. Є такий простір суто військовий, із зібраними воєнними трофеями,
що залишив ворог вже тут, на Київщині. Зараз у нас тут просто своєрідний штаб
утворився. Тут військові можуть залишитися на ночівлю, якщо їдуть транзитом,
тут збираються волонтери. Тут ми проводимо творчі презентації.

Зараз таких культурних заходів небагато, але з вересня ми плануємо розпочати
цикл важливих дискусій щодо візії повернення у Маріуполь. У нас є думка, що вже
пора. Я цей проєкт називаю \enquote{Візія повернення}. Я ж не міська рада. Я хочу про
це поговорити. Я хочу про це поговорити з нашими", - каже Міла.

Вона впевнена, що дискусію треба проводити у двох напрямках – серед дорослої
частини міста, і серед молоді.

\enquote{Якщо молодь не буде повертатися у Маріуполь, то не буде ніякого реборну,
ніякого відновлення. Буде місто, яке повільно помирає. А коли молодь зможе
почати повертатись? Якщо вона буде залучена до відновлення міста, якщо
побачить, що тут зможе реалізуватись. Тому це дуже важливо – працювати у цьому
напрямку}, - каже Людмила Завалей.

В її найближчих планах – окрім серії дискусій,  - розпочати курси
терапевтичного малювання, терапевтичного письма, провести зустрічі з цікавими
особистостями, експертами, політологами.

\enquote{Наш простір відкритий для всіх. Він буде працювати до тих пір, поки окупований
Маріуполь. А потім ми всі з командою будемо повертатись і разом відбудовувати
наше українське місто}, - запевняє Людмила Завалей.
