% vim: keymap=russian-jcukenwin
%%beginhead 
 
%%file 18_12_2021.fb.mitrjajev_aleksandr.kiev.1.chast_chetyre_kiev_istoria_1968
%%parent 18_12_2021
 
%%url https://www.facebook.com/aleksandr.mitryaev.35/posts/632552294768521
 
%%author_id mitrjajev_aleksandr.kiev
%%date 
 
%%tags 1968,kiev
%%title 4-я заключительная  часть моей Киевской Истории в 1968 году
 
%%endhead 
 
\subsection{4-я заключительная  часть моей Киевской Истории в 1968 году}
\label{sec:18_12_2021.fb.mitrjajev_aleksandr.kiev.1.chast_chetyre_kiev_istoria_1968}
 
\Purl{https://www.facebook.com/aleksandr.mitryaev.35/posts/632552294768521}
\ifcmt
 author_begin
   author_id mitrjajev_aleksandr.kiev
 author_end
\fi

4-я заключительная  часть моей Киевской Истории в 1968 году. Спасибо.

Пролог.

Здесь привожу не те события, которые я переживал в 1968 году. Не те чувства,
которые вели меня по жизненной дорожке, а тот контекст тех лет, месяцев и дней
жизни в СССР. Я жил не на необитаемом острове, а в Советском Союзе за железным
занавесом от всего остального Мира.

\ii{18_12_2021.fb.mitrjajev_aleksandr.kiev.1.chast_chetyre_kiev_istoria_1968.pic.1}

Появилась программа «Время» по телевидению.  Погибли  Гагарин и Серегин на
самолете.  Пражская весна и социализм с человеческим лицом, который ввел в
Прагу танки, расстрелявшие гражданское население,  В США убили М.Л.Кинга и
Р.Кеннеди. 

\ii{18_12_2021.fb.mitrjajev_aleksandr.kiev.1.chast_chetyre_kiev_istoria_1968.pic.2}

Невозможно было купить туфли – 4 часа на задах магазина «Обувь» на Крещатике,
тысячная очередь, давали обувь стран народной демократии, по паре в руки.. Мои
любимые книги – Фихтенгольц и  Кикоин, Джек Лондон и Жюль Верн. Мама у
телевизора сильно болела за «Динамо», киевляне, чемпионы СССР по футболу,
Маслов после Соловьёва.  Игры Динамо с Селтиком, звезда – Бышовец,. 

\ii{18_12_2021.fb.mitrjajev_aleksandr.kiev.1.chast_chetyre_kiev_istoria_1968.pic.3}


В ИПМ был
физрук, он приезжал в Клавдиево и рассказывал, что футболист Хмельницкий, его
друг, зарабатывает 1000 руб. в месяц:  как капитан милиции 250 рублей («Динамо»
- милицейская команда»), остальное - они выезжали в областные города УССР и
играли там за призовые с местными командами. У него 20 000 рублей сбережений,
он получил квартиру и однокомнатную квартиру для родителей.  1005 рублей – это
зарплата нашего Академика в ИПМ. Моя – 78 рублей.                                

\ii{18_12_2021.fb.mitrjajev_aleksandr.kiev.1.chast_chetyre_kiev_istoria_1968.pic.4}

. Никсон приехал в Киев. Я был в старом Ботсаду возле Университета.
Смотрел быстрый его проезд возле Владимирского собора. За деревьями
стояла милиция и военные, близко не подпускали.  По «Голосу Америки» он
сказал, посмотрев  нашу нищету:

- Вы не представляете какой сейчас расцвет Соединенных Штатов
Америки!!

Я наивно верил, что если буду учиться и трудиться, то добьюсь всего.  И день за
днем во мне крепла во мне эта вера. И 31 декабря 1968 года я не дотянул до 12
часов, когда в полночь били куранты в Кремле. Оставив Маму в ее комнате, в
которой стоял телевизор, я пошел в свою большую комнату, оставил зажженной
сосну, украшенную игрушками, гирляндами и пятиконечной звездой, забрался под
одеяло, натянул его на голову и моментально заснул.

\ii{18_12_2021.fb.mitrjajev_aleksandr.kiev.1.chast_chetyre_kiev_istoria_1968.pic.5}

Проснулся в девять утра 1 января 1969 года совершенно другим человеком, полным
надежд на будущее. Я был молод, и не понимал, что есть только два дня в году,
когда нельзя ничего сделать. Один называется вчера, а другой завтра, поэтому
сегодня - это лучший день, чтобы любить, верить, действовать и прежде всего
жить!

\ii{18_12_2021.fb.mitrjajev_aleksandr.kiev.1.chast_chetyre_kiev_istoria_1968.pic.6}

Лежу в кровати, не хочу вылазить из теплой.  Уже совсем светло в комнате. Мама
заботливо укрыла ноги мои полушубком и положила в изголовье на полочку Галину
открытку из Запорожья.             

Вы же помните, что 31 декабря 1968 года и 1 января 1969 года на «галеры» не
надо идти.

\ii{18_12_2021.fb.mitrjajev_aleksandr.kiev.1.chast_chetyre_kiev_istoria_1968.pic.7}
