%%beginhead 
 
%%file 06_10_2020.fb.fb_group.mariupol.nekropol.1.otchet_volonter_rabota_3_4_oktjabrja_2020
%%parent 06_10_2020
 
%%url https://www.facebook.com/groups/278185963354519/posts/380553319784449
 
%%author_id fb_group.mariupol.nekropol,arximisto
%%date 06_10_2020
 
%%tags 
%%title Отчет о волонтерской работе в Некрополе 3-4 октября 2020
 
%%endhead 

\subsection{Отчет о волонтерской работе в Некрополе 3-4 октября 2020}
\label{sec:06_10_2020.fb.fb_group.mariupol.nekropol.1.otchet_volonter_rabota_3_4_oktjabrja_2020}
 
\Purl{https://www.facebook.com/groups/278185963354519/posts/380553319784449}
\ifcmt
 author_begin
   author_id fb_group.mariupol.nekropol,arximisto
 author_end
\fi

\bigskip
\textbf{Отчет о волонтерской работе в Некрополе 3-4 октября 2020}

На выходных волонтерам удалось вернуть из забвения три имени создателей
Мариуполя и приступить к благоустройству древнего участка вокруг склепов Гофов
и Хараджаева.

\textbf{Открытия}

Александр Шпотаковский обнаружил старинную плиту \emph{Андрея Ивановича Хазанджи}
(1886-1889) буквально в шаге от плиты Николая Мелекова (ум. 1874)! Это
подтверждает, что на этом участке делали захоронения в 1870-80-х и есть
вероятность, что Феоктист Хартахай был на нем похоронен.

Впрочем, расчистка соседнего участка возле Ивана Николаевича Мелекова (ум.
1880) никаких результатов не принесла.

Рядом с Хазанджи и Н. Мелековым открыли плиту \emph{Чангли} (предположительная дата
смерти - 1930-е). Вероятно, холм из битого кирпича скрывает новые находки...

Елена Сугак и Александр Шпотаковский в зарослях обнаружили разваленный памятник
\emph{Феофании Алексеевны Антоненко} (урожд. \emph{Франтова}, ум. 1900 г.). И - страшная
находка. Когда очищали памятник от земли, наткнулись на человеческий череп. На
глубине меньше штыка лопаты! Ни костей, ни скелета... Оставили на месте...

\textbf{Благоустройство}

В субботу волонтеры Илья Луковенко, Андрій Никифоренко и Андрей Марусов
принялись за обустройство участка вокруг склепов Гофа и Хараджаева. Он весь
усеян обломками памятников советского времени. Восстановили пять памятников. Но
следует признать, что бОльшую часть восстановить не удастся - непонятно, где
они находились...

Благодаря усилиям Ильи Луковенко склеп Спиридона Гофа постепенно предстает в
своем первозданном виде.

Наконец, вместе с активистами молодежной общественной организации немцев
М.О.Д.Н.О. Марком и Сергей Штамбур стали расчищать завалы зарослей, под
которыми кроется плита немецкого колониста Нарцисса Якобсона (1823-1903)...

Огромное спасибо всем волонтерам!

Друзья, приглашаем вас присоединяться к волонтерской инициативе! Мы очень хотим
до наступления холодов создать в Некрополе хотя бы один благоустроенный
участок! Без вашей помощи мы вряд ли достигнем этой цели...

Как правило, мы работаем на выходных. Точка и время сбора - в 14:00 возле
белого Памятного креста в центре Некрополя. Анонс о мероприятиях 10-11 октября
сделаем чуть позже. Контактный телефон - 096 463 69 88.

До встречи, друзья!

\#mariupol\_necropolis\_report

\#хазанджи

\#антоненко\_франтова

%\ii{06_10_2020.fb.fb_group.mariupol.nekropol.1.otchet_volonter_rabota_3_4_oktjabrja_2020.cmt}
