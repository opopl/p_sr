% vim: keymap=russian-jcukenwin
%%beginhead 
 
%%file 13_12_2020.sites.ru.zen_yandex.yz.vestnik_istorii.1.njanja_pushkina
%%parent 13_12_2020
 
%%url https://zen.yandex.ru/media/vestnikistorii/taina-proishojdeniia-ariny-rodionovny-niani-pushkina-kak-ee-zvali-na-samom-dele-i-kem-ona-prihodilas-poetu-5fd4cd8933ed420c3ff5822b
 
%%author Вестник Истории (Yandex Zеn)
%%author_id yz.vestnik_istorii
%%author_url 
 
%%tags pushkin_aleksandr
%%title Тайна происхождения Арины Родионовны, няни Пушкина: как её звали на самом деле и кем она приходилась поэту?
 
%%endhead 
 
\subsection{Тайна происхождения Арины Родионовны, няни Пушкина: как её звали на самом деле и кем она приходилась поэту?}
\label{sec:13_12_2020.sites.ru.zen_yandex.yz.vestnik_istorii.1.njanja_pushkina}
\Purl{https://zen.yandex.ru/media/vestnikistorii/taina-proishojdeniia-ariny-rodionovny-niani-pushkina-kak-ee-zvali-na-samom-dele-i-kem-ona-prihodilas-poetu-5fd4cd8933ed420c3ff5822b}
\ifcmt
	author_begin
   author_id yz.vestnik_istorii
	author_end
\fi

\index[writers.rus]{Пушкин, Александр Сергеевич!Няня Пушкина, 13.12.2020}

\ifcmt
pic https://avatars.mds.yandex.net/get-zen_doc/1936066/pub_5fd4cd8933ed420c3ff5822b_5fd53d325a2c8e1f2c70e998/scale_2400
caption Арина Родионовна, няня Пушкина
\fi
\begin{leftbar}
	\begingroup
		\em Все мы знаем, что Арина Родионовна являлась няней великого поэта. Она привила
Александру Сергеевичу любовь к русскому языку, разделила с ним все тяжести
ссылки в Михайловском, стала очень близким для него человеком. Что нам известно
об этой женщине? К сожалению, немногое. Неизвестно даже, как она выглядела и
где была похоронена. Но кое-какие факты биографии Арины Родионовны, к счастью,
остались целыми, не уничтоженными временем. Давайте узнаем получше биографию
этой замечательной женщины.
	\endgroup
\end{leftbar}

\subsubsection{Арина Родионовна. Жизнь до службы няней}

\ifcmt
  width 0.5
\fi


Арина Родионовна родилась 10 апреля 1758 года. На самом деле няню Пушкина звали
Ирина. Но все домашние звали девочку Арина. Родители Арины Лукерья Кириллова и
Родион Яковлев (1728 – 1768 годы) были крепостными крестьянами и имели семь
детей. Семья Арины жила в деревне Суйда Копорского уезда Санкт-Петербургской
губернии.

Деревня Суйда принадлежала подпоручику лейб-гвардии Семеновского полка графу
Федору Алексеевичу Апраксину. В 1759 году деревню Суйду и прилегающие к ней
некоторые деревни у Апраксина купил Абрам Петрович Ганнибал — прадед Александра
Сергеевича Пушкина. А в 1781 году Арина Родионовна в возрасте двадцати трех лет
вышла замуж и переехала в Кобрино, к своему мужу, Федору Матвееву, от которого
позже у Арины родилось четверо детей. В Кобрино Арина стала служить у Осипа
Ганнибала, дяди поэта.

\subsubsection{Служба няней}

\ifcmt
  pic https://avatars.mds.yandex.net/get-zen_doc/3690562/pub_5fd4cd8933ed420c3ff5822b_5fd4ce369480ec78dcbf8e55/scale_2400
  caption Портрет Ольги Сергеевны Павлищевой —старшей сестры Александра Сергеевича Пушкина
  width 0.5
\fi

В 1792 году Арина Родионовна служила Марии Алексеевне Ганнибал в качестве няни
для племянника Марии Алексея. А в 1797 году с рождением старшей сестры
Александра Сергеевича Пушкина Арина стала няней и кормилицей для Ольги — внучки
Марии Алексеевны Ганнибал, а затем няней для двух внуков — старшего Александра
(будущего поэта) и младшего Льва. Ранее Арина Родионовна служила нянею для
матери будущего поэта Надежды Осиповны. В 1795 году за безупречную службу Мария
Алексеевна подарила Арине Родионовне отдельную избу в Кобрине. Это был
прекрасный подарок для крепостной крестьянки.

\subsubsection{Интересный факт}

Кроме Арины Родионовны в семье Пушкиных была еще одна няня — Ульяна Яковлевна,
которой было поручено стать няней для маленького Александра. Однако она не
оставила явного следа в творчестве поэта. Как и Никита Козлов – дядька Пушкина,
ставший свидетелем смерти Александра Сергеевича и провожавший гроб с телом
поэта в последний путь на кладбище. Хотя есть версия, что Козлов являлся
прототипом Савельича, верного слуги Петра Гринева из романа «Капитанская
дочка».

\ifcmt
  pic https://avatars.mds.yandex.net/get-zen_doc/3507111/pub_5fd4cd8933ed420c3ff5822b_5fd4ce9664f2df1897261c1f/scale_2400
  caption Юный Лев Пушкин, брат поэта
  width 0.5
\fi

\subsubsection{Вдова}

В 1801 году Арина Родионовна стала вдовой. Муж ее Федор умер от пьянства. Няне
приходилось заботиться и о барских, и о собственных детях, что было совсем
нелегко. Особенно, если учесть, что семья жила бедно.

\subsubsection{Смерть Марии Алексеевны Ганнибал. Возвращение в Санкт-Петербург}

В 1807 году семейство Ганнибалов продало вместе с крестьянами земли в
Петербургской губернии и поселилось в уезде Опочецкий Псковской губернии.
Однако Арине Родионовне продажа не грозила. Она была не прикреплена к земле, а
потому вместе с хозяевами переехала в Псковскую губернию. До 1811 года, года,
когда Александр Сергеевич поступил в Лицей, няня прожила с поэтом под одной
крышей. Юный Александр очень любил няню. В письмах своих он называл ее
«мамушкой» В 1818 году умерла Мария Ганнибал. Арина тогда проживала у Пушкиных
в Петербурге, приезжая вместе с ними на лето в Михайловское.

\subsubsection{Ссылка Александра Сергеевича в Михайловское}

В 1824 – 1826 годах Арина Родионовна разделила с поэтом всю тяжесть ссылки в
селе Михайловском. Она стала ему самым близким человеком и вдохновителем. По
признанию самого поэта Арина была прототипом няни Татьяны Лариной, Дубровского,
мамки Ксении Годуновой, а также нескольких женских образов из романа «Арап
Петра Великого».

Арина Родионовна знала очень много народных сказок. Пушкин записывал их,
тщательно перерабатывал и на основе этих сказок получал сюжеты для своих
гениальных произведений.

В ноябре 1824 года Пушкин написал брату: \emph{«Знаешь ли мои занятия? до обеда пишу
записки, обедаю поздно; после обеда езжу верхом, вечером слушаю сказки - и
вознаграждаю тем недостатки проклятого своего воспитания. Что за прелесть эти
сказки! Каждая есть поэма!»}

\subsubsection{Интересный факт}

Известно, что Пушкин со слов няни записал семь сказок, десять песен и несколько
народных выражений, хотя, конечно, слышал от нее гораздо больше. Няня знала
очень много сказок и умела передавала их как-то по-особенному. Именно от нее
Пушкин впервые услышал про избушку на курьих ножках и сказку о мертвой царевне
и семи богатырях.

\subsubsection{Последняя встреча Александра Сергеевича Пушкина и Арины Родионовны. Смерть няни}

\ifcmt
tab_begin cols=2
	caption Арина Родионовна - Няня Пушкина 

  pic https://avatars.mds.yandex.net/get-zen_doc/1857554/pub_5fd4cd8933ed420c3ff5822b_5fd4cde85a2c8e1f2cdb2144/scale_2400
  caption Портрет Арины Родионовны (Все изображения няни Пушкина были сделаны после ее смерти)

  pic https://avatars.mds.yandex.net/get-zen_doc/3144955/pub_5fd4cd8933ed420c3ff5822b_5fd4cec633ed420c3ff75ce2/scale_2400
  caption Портрет Марии Алексеевны, бабушки Пушкина

  pic https://avatars.mds.yandex.net/get-zen_doc/112297/pub_5fd4cd8933ed420c3ff5822b_5fd4cefa0b82510af53c38da/scale_2400
  caption Главный усадебный дом в Михайловском — действующем музее-заповеднике

  pic https://avatars.mds.yandex.net/get-zen_doc/3431006/pub_5fd4cd8933ed420c3ff5822b_5fd4cf785a2c8e1f2cdd7bd4/scale_2400
  caption Памятная доска, посвященная Арине Родионовне
tab_end
\fi

14 сентября 1827 года в Михайловском состоялась последняя встреча Пушкина и
Арины Родионовне. К моменту встречи няне уже исполнилось 69 лет.

В январе 1828 года старшая сестра Пушкина Ольга против воли родителей вышла
замуж за Николая Павлищева. Молодая семья поселилась в Петербурге. В марте
родители Ольги выделили ей несколько крепостных для ведения хозяйства. Среди
них находилась и её старая няня.

31 июля 1828 года после недолгой болезни семидесятилетняя Арина Родионовна
скончалась в доме Ольги. Арину Родионовну похоронили на Смоленском кладбище в
городе Санкт-Петербург. К сожалению, могила ее была утеряна. Александр
Сергеевич пытался два года спустя найти могилу няни (ни поэт, ни его старшая
сестра по неизвестной причине не были на похоронах), но попытка не увенчалась
успехом. В 1977 году при входе на кладбище была установлена памятная доска.

\textbf{(с) Рыжая Молли}
