% vim: keymap=russian-jcukenwin
%%beginhead 
 
%%file 05_12_2021.fb.pogrebinskij_mihail.1.vranje_blinkena.cmt
%%parent 05_12_2021.fb.pogrebinskij_mihail.1.vranje_blinkena
 
%%url 
 
%%author_id 
%%date 
 
%%tags 
%%title 
 
%%endhead 
\subsubsection{Коментарі}

\begin{itemize} % {
\iusr{Вячеслав Беленький}
Всё он прекрасно знает.
Против лома нет приёма, кроме другого лома.
Лаврову надо было прямо заявить Оладину, что майдан расстреляли Их, снайперы.
И в СМИ.
Тем более, это не далеко от правды по сути.

\begin{itemize} % {
\iusr{Михаил Погребинский}
\textbf{Вячеслав Беленький} я бы именно так и отреагировал бы

\iusr{Геннадий Овчаренко}
\textbf{Михаил Погребинский} поэтому, вы, Михаил, не дипломат...

\iusr{Юрий Зверев}
\textbf{Вячеслав Беленький} Единственная дочь Лаврова постоянно живет и работает в Москве. Да, родилась в Нью-Йорке, но ещё при СССР когда отец работал дипломатом в Постпредстве СССР при ООН.

\iusr{Татьяна Шевченко}
\textbf{Вячеслав Беленький} Такие как Лавров, бывают редко и избирательно...(не в каждой стране), поэтому пусть поработает ещё на Славу своему Отечеству и на радость нам, хоть есть кого послушать и за кого гордится, в отличии от украинских представителей власти...

\iusr{Вячеслав Беленький}
\textbf{Геннадий Овчаренко} Современная дипломатия ничего общего не имеет с книжными эталонами.
В век постправды, опирающаяся на рассудок и здравомыслие аудитории, академическая дипломатия проигрывает с треском.
Во времена Фриков надо играть по законам Фриков.

\iusr{Светлана Захарова}
\textbf{Вячеслав Беленький} я думаю , что им/ блинкенам это постоянно говорят - они не хотят и не будут это слышать , их устраивает « своя правда», это очень видно по создавшейся истерии на сегодняшний момент

\iusr{Геннадий Овчаренко}
\textbf{Вячеслав Беленький} это абсолютно верное, но поверхностное рассуждение.
Мы не имеем и десяти процентов информации о причинах развития именно такого хода событий, и в том числе, - поведения Лаврова.

\iusr{Светлана Захарова}
\textbf{Вячеслав Беленький} 

\href{http://сергей-лавров.рф/doch-lavrova/}{%
Дочь Сергея Лаврова — Екатерина Лаврова: личная жизнь, биография и гражданская позиция, %
сергей-лавров.рф %
}

\begin{multicols}{2}

\headTwo{Дочь Сергея Лаврова — Екатерина Лаврова: личная жизнь, биография и гражданская позиция}

Личная жизнь глав государства, известных политиков и «партийных цезарей» во все
времена была надежно скрыта от посторонних глаз. «Сильные мира сего», по
понятным причинам, никогда не распространялись о своих близких, не давали
никаких объяснений и комментариев. Их дети и внуки всегда находились за ширмой
общественных перипетий.

Екатерина Лаврова, дочь известного российского дипломата, министра иностранных
дел России, пожалуй, является исключением в этой политической плеяде тайн. Она
впервые рассказала о том, какой персоной является ее отец – Лавров Сергей
Викторович.

\headTwo{Автобиографические заметки}

\ifcmt
  ig http://сергей-лавров.рф/wp-content/uploads/2017/10/lavrov-s-dochkoi-e1508698093447-300x294.jpg
  @width 0.4
\fi

Екатерина Лаврова родилась в Соединенных Штатах Америки, детство и юность
провела там же. Это объясняется тем, что ее отец в 1982 году был назначен
полномочным представителем СССР при Организации Объединенных Наций, поэтому он
был вынужден отправиться в Нью-Йорк на постоянное место жительства.

Мать Екатерины, Мария Александровна, филолог по образованию, всю жизнь
старалась привлечь дочери любовь к искусству. Екатерина получила блестящее
воспитание; занималась в хореографической студии, посещала выставки, музеи,
оперу и балет.

\headTwo{Образование}

Екатерина Сергеевна обучалась в школе Манхэттена, после которой поступила в
университет «Колумбия». Сам Сергей Викторович, как отмечает дочь, всегда очень
ответственно относился к образованию дочери.

«Хоть я и была единственным ребенком в семье, всего должна была достигать сама,
чтобы не зависеть от родителей, меня так учили», — рассказывает Екатерина, —
«поэтому усердно училась».

\headTwo{Замужество}

Окончив Колумбийский университет, Екатерина отправляется в Англию, решив стать
магистром. В Лондоне она знакомится с Александром Винокуровым, своим будущим
мужем. Сам Александр из очень влиятельной семьи медиамагнатов в сфере
фармацевтики. Сегодня муж Екатерины Лавровой является совладельцем компании
«Marathon Group».

\ifcmt
  ig http://сергей-лавров.рф/wp-content/uploads/2017/10/doch-lavrova-s-mujem-300x240.jpg
  @width 0.4
\fi

\figCaption{Фамилия Екатерины в супружестве — Винокурова}

Собравшись узаконить свои отношения, Екатерина и Александр отправились в
Россию, отметить свое вступление в брак.

Торжество проходило в Доме приемов при Администрации Президента РФ.

Молодоженов ожидал сюрприз: романтический видеоролик, повествующий об их
знакомстве.

На свадьбе присутствовали такие именитые гости, как: певец Валерий Леонтьев,
искусствовед Мария Байбакова, и другие.

\headTwo{Деятельность Екатерины Сергеевны}

\ifcmt
  ig http://сергей-лавров.рф/wp-content/uploads/2017/10/lavrov-i-doch-200x300.jpg
  @width 0.4
\fi

Сегодня Екатерина – гражданка РФ, постоянно проживает и работает в Москве. Не
смотря на свое привилегированное общественное положение, Екатерина старается им
никогда не пользоваться, предпочитает не прибегать к помощи своего отца или
мужа.

Десять лет Лаврова проработала в крупной мировой компании «Cristies», которая
занимается публичными торгами; в ней она смогла добиться поста директора, но
вскоре покинула эту должность, когда появилась возможность открыть собственное
дело в сфере продвижения и развития искусства. Фирма Екатерины занимается
посреднической деятельностью между художниками и коллекционерами.

\headTwo{Бизнес и семейная жизнь}

Екатерина всегда хотела, чтобы у нее была большая семья. В 2010 году у нее
родился долгожданный первенец – сын Леонид. Сейчас, на радость счастливым
родителям, у пары появилась дочь.

\ifcmt
  ig http://сергей-лавров.рф/wp-content/uploads/2017/10/katya-lavrova-200x300.jpeg
  @width 0.4
\fi

\figCaption{Бизнес Екатерины Лавровой}

Александр и Екатерина – ровесники, поэтому имеют массу общих интересов.
Екатерина говорит, что чувствует себя защищенной и любимой, что супруг во всех
начинаниях поддерживает ее.

Семья Винокуровых ведет здоровый образ жизни: Екатерина с мужем регулярно
посещают бассейн и спортзал, увлекаются триатлоном, любят путешествовать по
горам.

Лаврова призналась, что никогда бы не смогла выйти замуж за иностранца,
поскольку ей не нравится их мировосприятие; поэтому встречу с Александром
Винокуровым она называет – своей судьбой.

В целом, Екатерина Сергеевна благодарна своему отцу за то, что он с ранних лет
смог привить ей уверенность в себе и возможность воспитать силу воли.

\end{multicols}

\iusr{Светлана Захарова}
\textbf{Вячеслав Беленький} 

дочь Лаврова родилась в США, тк Лавров с 1982 г работал при ООН, сейчас она
проживает с семьей в России , нужно перед тем как писать/ утверждать хотя бы
поинтересоваться

\iusr{Светлана Захарова}
\textbf{Вячеслав Беленький} печально уподобляться им

\iusr{Вячеслав Беленький}
\textbf{Светлана Захарова}

В штатах в школах проводят своеобразный конкурс - класс делится на две группы,
дается условие неопределенности. Каждая группа должна доказать противоположное
к другой группе решение. На следующий день они меняются местами и начинают тем
же рвением доказывать прямо противоположное.

Наши восхищаются, мол это гибкость ума развивает, а по факту имеем воспитание
устойчивой субъективности - когда мозги затачиваются не на поиск истины с
учетом всех известных фактов, а на сочинение аргументации под изначально
заданный критерий (выгоду)

А потому удивляемся, как какой-либо западный ЭКС-клерк после отставки "вдруг"
начинает так здраво и красиво рассуждать! "что ж ты такой слепой был, когда при
должности??" А он просто "отрабатывал установку" - косил под дурачка, продвигая
свою и своих работодателей Выгоду.

Так и тут, Блинкин отлично знает правду

Но действует по принципу Геббельса - "Мы добиваемся не правды, а эффекта"

И перебить этот ход, рассчитанный на эмоцию - рассудком невозможно. Только
встречной эмоцией.

\iusr{Светлана Захарова}
\textbf{Вячеслав Беленький} 

по такой схеме у них и 2 партии работают/ руководят, согласна с Вами, детей
натаскивают со школьной скамьи на это, в дальнейшем они « прекрасно» с этим
справляются

\iusr{Александр Блашко}
\textbf{Вячеслав Беленький} Лаврову ещё там жить))))

\iusr{Олексій Романов}
\textbf{Вячеслав Беленький} 100\%

\iusr{Якова Бриз}
\textbf{Вячеслав Беленький} А вместо Лаврова вас?!

\iusr{Валерий Oven}
\textbf{Вячеслав Беленький} 

Это не \enquote{недалеко}, а так и есть. Не случайно дело по этой \enquote{небесной сотне}
закрыли, а вещественные доказательства уничтожили... Кстати, к какому послу эти
\enquote{лидеры майдана} бегали на консультации?

\iusr{Валерий Oven}
\textbf{Вячеслав Беленький} 

Я думаю, что сами эти чревовещатели во главе с нынешним "пожилым человеком"
прекрасно знают, что они лгут. И общественность тоже не очень им верит, так как
у них имеется повод не принимать эту инф. кампанию на веру. Умные люди, как и
без оного, есть везде. Одни работают на лукавого, другие всё понимают. В этой
ситуации нужно себя вести хотя бы как китайцы, которые не ведутся эмоционально
на всю эту постановку. Кстати, от Ким Чен Ына эти ястребы отстали?

\iusr{Вячеслав Беленький}
\textbf{Валерий Oven} Ким проигнорил угрозы и создал ЯО.

Западная модель политики подразумевает реакцию в процессе и не умеет долго
надувать тему, которая \enquote{закрыта} к изменению. Природа капитализма не терпит
неэффективных трат ресурсов.

Пока не было ракет - давили, когда уже есть - остыли в интересе.

Заметьте, по Крыму реальных движений нет, даже санкции бутафорские, потому что
всем очевидно - откат в среднесрочной перспективе невозможен никаким боком,
\enquote{как минимум до развала РФ}.

По Донбассу вопрос не закрыт - потому и в ходу на всю.

Что касается Китайцев, то они прямая противоположность Кремля - никаких
PR-выпадов не спускают, всегда отвечают.

Молча, без лишних слов или тем более позорных \enquote{упрашиваний не нарушать линии} -
просчитывают ситуацию и бьют в слабое звено.

Литва открыла представительство Тайваня - Пекин молча взял и стер резинкой
Литву с мировой карты своей таможни. Циркулярно. Без лишних разговоров.

Потому что это Важно - символ обязательной расплаты и урок другим.
Вильнюс уже в истерике побежал в Брюссель, но вряд ли им там помогут.

\href{https://lenta.ru/news/2021/12/04/litva/}{%
Литва попросила Европу о помощи из-за Китая, lenta.ru, 04.12.2021%
}

И на том же направлении возьмите РФ.... сколько говна прибалты влили на Москву?
несчетно.

А транзит как шел через них российский - так и идет. За десятилетия немного
снизили, Но продолжает идти! Одни пустые угрозы и разговоры о прекращении.
Будет кто уважать РФ?

ответ очевиден.

\iusr{Валерий Oven}
\textbf{Вячеслав Беленький} 

Поддерживаю. Вообще Россия с 1991-го сидит на шпагате. Одна нога ещё опирается
на Советское мировоззрение, другая - на рыночное, буржуазное. От первого
откреститься страшно - народ не поддержит, а второе не будет полноценным при
сохранении первого, которое в принципе выходит из христианства: свобода,
равенство, братство.


\iusr{Пургина Наталья}
\textbf{Светлана Захарова} зачем им говорить, если сценарий они сами и корректировали

\iusr{Валерий Oven}
\textbf{Вячеслав Беленький} 

Так это классика, весь западный истеблишмент заточен на служение мамоне, хотя
многие симпатизируют российским политикам. Тот же Блинкин с умилительным
выражением обнимает Лаврова, и уже через минуту начинает нести ересь, словно
дурачок... А Черчилль? Читая воспоминания этой лисы (\enquote{Вторая мировая война}, в
которую включена его переписка со Сталиным, которого он почти что боготворил),
удивляешься, как можно быть таким двуличным человеком... У нас так не принято.

\iusr{Юрий Ватник}
\textbf{Вячеслав Беленький} точнее, встречной демагогией, ещё более категоричной.
Грубо, но на войне, как на войне.
Кстати, и нам бы здесь не грех почаще пользоваться этим приёмом в спорах с местными оранжево-зелено-коричневыми.
А то мы все про совесть, про права человека, "да, как вам не стыдно"....
Им не стыдно.

\end{itemize} % }

\iusr{Вадим Цымбаленко}

Вы про мораль спрашиваете тех, кто тряс перед мировой общественностью
пробиркой, а затем создал азиатских хаос и уничтожение ... У них есть одна
мораль - доллар!


\iusr{Игорь Тимашков}

Не исключено, что обе стороны сознательно выходят на \enquote{карибский кризис-2}.
Потому что он позволит переустроить этот мир. А переустройство без кризиса -
люди не воспримут ... )

\begin{itemize} % {
\iusr{Любовь Гамова}
\textbf{Игорь Тимашков} 

обе стороны? Это вряд ли. У нас еще помнят, что война дала нам неисчислимые
жертвы и отбросила страну на десятки лет назад. А вот американцы от каждой
войны только выигрывали: обогащались и процветали, выходя из кризисов. И
убеждать их в том, что в этот раз им за океаном отсидеться не удастся,
бесполезно. Потому что это их ракеты стоят в пяти минутах от целей на нашей
территории - причем, уже, фактически, на нашей же территории, и потому что у
них будет первый ход - они этого и не скрывают - такова их военная доктрина. А
в современных условиях, как в шахматах, именно белые фигуры являются решающим
фактором.


\iusr{Игорь Тимашков}
\textbf{Любовь Гамова} 

вы не поняли. ) Первый карибский кризис - сделал мир двухполярным. Союз стал
второй сверхдержавой и распространил свое влияние на половину мира. Это было бы
невозможно сделать без кубинского кризиса. Новый кризис выполнит аналогичную
роль..

\iusr{Любовь Гамова}
\textbf{Игорь Тимашков} 

Путин - не Хрущев, а главное - Байден - не Кеннеди. Байден ( или Столтенберг)
свои ракеты из Польши не отведет, а Путин на Кубе не поставит. Скорее всего, не
сегодня-завтра мир опять станет однополярным, только полюс из Вашингтона
сместиться в Пекин. Нам от этого лучше не будет. Кстати, мир стал двухполярным
в 1945 году.


\iusr{Игорь Тимашков}
\textbf{Любовь Гамова} При выборе - ракеты в Польше, Украине и ядерный удар по Вашингтону - американцы думать не будут. Вашингтон ближе к телу, чем какая-то там Польша .. )

\iusr{Игорь Тимашков}
\textbf{Любовь Гамова} в 45м году у Сталина были танки ИС3, а у Трумена была бомба. Поэтому равенство между СССР и США было только в Европе, но не в мире. )

\iusr{Любовь Гамова}
\textbf{Игорь Тимашков} 

В том-то и дело, что они, мне кажется, убеждены, что, во-первых,Россия не
применит ядерные бомбы - мы ж не Америка, нам людей жалко, во-вторых - не
успеет. Их так долго убеждали, что в 90-е Россию разорвали в клочья и
оправиться она не смогла, что они свято в это верят. Это убеждение очень ясно
видно даже по комментариям наших бывших соотечественников за рубежом: у нас же
ничего нет - ни теплых туалетов, ни науки, ни медицины, ни экономики , ни
закона, ни прав - ничего у нас нет, кроме "диктатора с золотым ершиком".


\iusr{Юрий Швед}
\textbf{Любовь Гамова} В Польше ракет пока ещё нет. Но зато есть в Калининграде.

\iusr{Любовь Гамова}
\textbf{Юрий Швед} наберите в яндексе или в гугле где, в каких странах расположены базы нато. Очень наглядная картинка.

\iusr{Любовь Гамова}
\textbf{Юрий Швед}

\ifcmt
  ig https://scontent-frx5-2.xx.fbcdn.net/v/t39.30808-6/263475647_4384234935039376_3998470536636697505_n.jpg?_nc_cat=109&ccb=1-5&_nc_sid=dbeb18&_nc_ohc=09OYOtxYQpYAX_dj_s6&_nc_ht=scontent-frx5-2.xx&oh=36d34729133b0a1aaac7e9bc5e4d8d3c&oe=61B417D3
  @width 0.4
\fi

\iusr{Юрий Швед}
\textbf{Любовь Гамова} Ракет в Польше пока нет, я не про базы нато.

\iusr{Любовь Гамова}
\textbf{Юрий Швед} будут. И очень скоро. Как вишенка на торте. Но и без этой вишенки отравы столько, что вишенка эта на самом деле ничего не решает, а только демонстрирует изощренный цинизм пекарей.

\iusr{Игорь Тимашков}
\textbf{Любовь Гамова} Не нужно бояться кризиса. Кризис должен закончиться хорошо. )

\end{itemize} % }

\iusr{Девид Грегори}

Приручили народ верить им, что они последняя инстанция и только они доносят
правду и т д. Позор что отдельные особи под эту дудку на цыпочках ходят.


\iusr{Любовь Гамова}

Печально признавать, но внучка неправа: ложь и лицемерие, к сожалению,
работает, да еще как! В современной политике честность, к сожалению,
проигрывает, потому что борется честно, по правилам, а лжец не гнушается
подлостью, подножками и мошенничеством. А я, кстати, допускаю, что Блинкен
ничего не видел про майдан и не знает: а зачем ему это, если есть собственные
интересы, есть пропагандистские клише и есть убежденность в том, что ты (США) -
тот самый Юпитер, для которого все остальные подобны быку, которому положено
быть в стойле и не мычать.

\begin{itemize} % {
\iusr{Сергей Левин}
\textbf{Любовь Гамова} "честность, к сожалению, проигрывает"
Нет. Я считаю, что Россия ведёт себя честно. И на длительной дистанции это лучше чем врать.

\iusr{Любовь Гамова}
\textbf{Сергей Левин} Хорошо бы , если так. Мне нравится ваш оптимизм. Мне его, к сожалению, не хватает.

\iusr{Александр Каспир}
Политика и честность,?.. политика есть сокрытие истинных намерений .
\end{itemize} % }

\iusr{Valeri Paul}
Может быть одной из сторон поступить так, как случилось в конце второй мировой и показательно использовать циркон или кинжал(((

\iusr{Grigory Ioffe}
Блинкен крайне примитивен. Не уверен, что он вообще имеет политический вес.

\begin{itemize} % {
\iusr{Геннадий Панферов}
\textbf{Grigory Ioffe} Не имеет вес, потому что врет? А кто у вас не врет, кто имеет вес? Очевидно же, что имеем дело с системным сознательным враньём, которое проистекает из-за ощущения силы. Дело совсем не в каком-то Блинкене....

\iusr{Grigory Ioffe}
\textbf{Геннадий Панферов} Конечно, все профессиональные политики так или иначе врут. Поэтому политический вес той или иной персоны никак с этим не коррелирует. Или почти никак. Кто имеет вес? Пелоси, Мкконелл, Шумер имеют определенный вес. А в администрации Жаннетт Еллен. Из не действующих политиков, безусловно, Трамп и Обама.
\end{itemize} % }

\iusr{Анатолий Живолуп}
Югославия, Афганистан, Ирак, Ливия и Сирия не дадут соврать, что НАТО оборонительный союз.

 @igg{fbicon.circus.tent}  @igg{fbicon.face.clown} @igg{fbicon.flag.usa}

\begin{itemize} % {
\iusr{Вероника Беллиссима}
\textbf{Анатолий Живолуп} точно!

\iusr{Владимир Дименков}
\textbf{Анатолий Живолуп}, и не только, а Япония, Вьетнам....?
\end{itemize} % }

\iusr{Edvard Crain}

Потому, что тридцать лет ОНИ считали себя абсолютно правыми во всём. Победители
СССР. И привыкли, что их никто не ловит за язык, не надо отвечать за
сказанное. В этом году русские впервые оскалились по-взрослому, но блинкины
этого не понимают.Их надо публично, не отходя от кассы ловить на слове и бить,
может тогда ситуация с циничной брехнёй может изменится... Если не попрут
буром, но последняя остановка это ядерный апокалипсис.

\iusr{Галицкая Любовь}
Надо, чтобы и у них таким же образом действовали, не предусмотренным конституцией методом.)

\iusr{Валерий Джаббаров}
Когда слушаешь этих т н высших чиновников США и ЕС начинаешь понимать, то что
даже ясно дебилам не ясно им !!

\begin{itemize} % {
\iusr{Александр Каспир}
Америка населена не дебилами если четверть мирового по ва и половина изобретений и технологий ...
\end{itemize} % }

\iusr{Геннадий Тарадин}
Ложь у высоких чиновников США - естественная форма освещения всего чего угодно.

\iusr{Ирина Кузнецова}

Все они понимают. Лгут нагло. Так было задумано. Пора объединяться и
становиться в ряды сопротивления информационной войне.

\iusr{Виктория Куколкина}
под кастрюлями работает! да еще и безотказно и многоразово!

\iusr{Наталия Когденко}

Страшно то, что он реально мог и не видеть всех кадров правды. Оно ему не надо,
чтобы легче идти той дорогой, на которую он поставлен.

\iusr{Архимандрит Серафим Вереш}
Их мораль - доллар...

\iusr{Кира Берёзкина}
У них правда в силе - где сила, туда подтягивают и правду.

\begin{itemize} % {
\iusr{Лидия Рошан}
\textbf{Кира Берёзкина} У них нет ни силы, ни правды, ни совести. Даже нет собственного лица. Они представляют собой мировой вирус - лжи, обмана, фальсификации, невежества. Все эти показатели излечимы. Просто их лечением никто не занимался потому, что ранее в Европе была культура, были ценности. Сейчас почти все утрачено.

\iusr{Кира Берёзкина}
\textbf{Лидия Рошан} кем?  @igg{fbicon.smile}  Не смешите.

\iusr{Лидия Рошан}
\textbf{Кира Берёзкина} Нисколько не смешно. Попытайтесь сделать анализ происходящего не из комментариев, а из анализа источников не касающихся СМИ. СМИ - это инструмент подконтрольный владельцу, что хочу, то и напишу. Всего доброго. Думайте, если интересно.
@igg{fbicon.index.pointing.up}	 
@igg{fbicon.beaming.face.smiling.eyes}  @igg{fbicon.hot.beverage} 

\iusr{Кира Берёзкина}
\textbf{Лидия Рошан} Вы бы Пелевина почитали - хотя бы СНАФФ - может, не писали бы так много букв ни о чём.

\iusr{Виталий Руссу}
\textbf{Кира Берёзкина} , у них есть ресурсы... большие надо признать.
И пока , наша правда проигрывает.
Но это пока..

\iusr{Кира Берёзкина}
\textbf{Виталий Руссу} не важен метод (причина) - важен результат.
\end{itemize} % }

\iusr{Елена Верич}
У меня такой впечатление, что в США вообще не понимают, что происходит и
разбираться желания нет.

\iusr{Тамара Богданович}

Да всё они понимают, эти блинкены. Они же это все и планируют. А потом, при
помощи народа, наделённого той бесконечностью, о которой говорил Эйнштейн, всё
осуществляют.


\iusr{Алла Мышко}
Лжет Блинкен, не может же он признать, что этот переырот спланирован ими в
вашингтончком обкоме, ии ручной порошенко их ставленик.

\begin{itemize} % {
\iusr{Николай Гайсенок}
Как сказала Нуланд, они потратили 5 лярдов на Украину, на переворот

\iusr{Алла Мышко}
\textbf{Николай Гайсенок} да так дешево порох сдал страну
\end{itemize} % }

\iusr{Татьяна Тихомирова}

Везде и во всем у них, у всех - отвратительная ложь, подмены понятий и дешевый,
мелкий аферизм!

\begin{itemize} % {
\iusr{Gala Lebedynska}
\textbf{Татьяна Тихомирова} не мелкий и не дешёвый. Эти люди все отлично понимают и знают - зачем . Кто освоил Америку? .Бандиты, криминал, коммерсы

\iusr{Татьяна Тихомирова}
\textbf{Gala Lebedynska} Я хорошо это понимаю. Согласна с вами, но... ПЕРЕД БОГОМ - да шушера они мелкотравчатая, лживая... Сорри!

\iusr{Gala Lebedynska}
\textbf{Татьяна Тихомирова} так и есть

\iusr{Gala Lebedynska}
\textbf{Татьяна Тихомирова}

\end{itemize} % }

\iusr{Оксана Шведова}

Знаете, что страшно - до сих пор существует множество людей, которые верят, что
наглая ложь от блинкенов и ко., это святая правда.


\iusr{Ольга Лосикова}
А что не ложь? - там все построено на лжи. Они ее и не стесняются.

\iusr{Светлана Радионова}
Хвост виляет собакой

\iusr{Константин Кондратьев}

- (да что уж \enquote{силы, лояльные Януковичу} - мне тогда мои майданутые друзья в
лицо кричали без тени сомнения: \enquote{ПУТЕНСКИЕ СНАЙПЕРЫ!..} , мол... - (с того всё
под горку неудержимо из покатилось...((


\iusr{Людмила Хрипкова}

На моей памяти уже ни один десяток лет мы наблюдаем неоспоримую очевидность
того, что по части вранья американские гос. деятели просто монополизировали
\enquote{Оскар} и, уже не заботясь о изобретательности и красоте формулировок, никому
не собираются его уступать.


\iusr{Светлана Захарова}
Они все « вместе взятые» давно переплюнули Гёббельса  @igg{fbicon.man.facepalming} 

\iusr{Митко Хитов}
Пока мы живем в их (американском) мире, хотим этого или нет, мы всегда от их милостью зависим!
Когда начнем жить в своем (евразийском) мире, тогда нам будет все равно что говорят всякие блинкены!
Другой вопрос, что это будет менее мащабная система разделения труда, но...

\iusr{Валентина Врубель}
Я думаю не знает, ему это не интересно и он читает то что ему на бумажке написано

\iusr{Сергей Юзовский}
Брехунами были есть и будут. В этом их сущность, для достижения своих
собственных целей любыми способами и средствами.

\iusr{Валерий Вахнюк}

\enquote{Неужели для того, чтобы эта простая истина дошла до блинкенов, нужно поставить
мир на грань мировой войны?} - К сожалению, это так. Этим господам очень
хочется заглянуть в бездну. А потом бездна посмотрит на них...

\begin{itemize} % {
\iusr{Борис Матвеевич Бурлаченко}
\textbf{Валерий Вахнюк} 

именно так. Вторая мировая вывела сша из глубочайшего кризиса. Пока Европа
захлёбывалась в крови, сша жирели на горе стран. Граждане США не видели взрывов
бомб на своей территории, не испытали горя от гибели детей, стариков, родных и
близких и поэтому война в их понимании нечто их не касающееся, компьютерная
игра, в которой есть несколько жизней. Нет страха и желания подумать, поэтому
так смелым и безрассудны

\end{itemize} % }

\iusr{Георгий Фокин}

да все они (блинкины) понимают, просто они считают, что под свою выдаваемую за
правду ложь, могут творить выгодные для себя дела, грабить другие страны,
осуществлять латентную, а иногда и открытую оккупацию. Им так выгодно, вот и
вся правда


\iusr{Igor Nikolaievitch}
просто, в английском языке нет слова \enquote{совесть}.. че уж тут удивляться

\iusr{Вадим Меркурьев}
Только став эксами, они начинают рассказывать правду. Частями.

\iusr{Suz Weimar}
Не поможет. Наука история доказывает, что просветление наступает только после грандиозного поражения. Но спустя какое-то время все возвращается на круги своя

\iusr{Михаил Бурмистров}
\textbf{Михаил Погребинский} Михаил Борисович , меня терзают смутные сомнения на предмет того, что заявлял Мишико в своё время. Сколько снайперов в Грузии тогда майдайданутые навербовали
... .

\iusr{Константин Колбасев}
Блинкен, Климкин.. Они не в одном роддоме родились случайно? @igg{fbicon.laugh.rolling.floor} 

\iusr{Виктория Глумова}
Сто раз повторенная ложь на сто первый раз станет похожа на правду.

\iusr{Андрей Неверов}

Дымзавеса. Лавров и Блинкин таращат глаза, хмурят брови, извергают страшные
ругательства и ломают стулья. Пипл хавает. Бильды поражают Укрь на карте
указательным перстом. А за кулисами идёт реальная переговорщина. Циничный
холодный торг. Чётко по фон Нейману. Минимакс, максимин. "Падающего толкни"...


\iusr{Николай Гайсенок}
Надо было ему, Блинкину, напомнить, что было в Одессе 2 мая.

\iusr{Николай Александров}

Они не могут признать истину, потому что тогда им придется сказать, вдобавок ко
всему прочему, что именно они-то всё это похабство и возглавляли...

А то мы не помним бесконечное шастанье Нуланд в Киев, ее походы на майдан,
эйфорическое блаженство Маккейна, снимающего на телефончик толпы народа на
улицах вечернего Киева, приземляющиеся самолёты с баблом...

\iusr{German Gorozhanski}

Блинкен не такой идиот, чтоб нести откровенную пургу, да ещё и в такой
вышиватно-хамоватой манере. Важно увидеть цель, которую он преследовал, неся
откровенный, не по статусу госсекретаря, бред. Предположу, что всё это сделано
специально, перед разговором Байдена с Путиным, чтоб показать промайданной
Украине, что "Америка с вами", и своим домашним патриотам и противникам
Байдена, смотрите, мол, как мы строги с русскими и можем чуть-ли не отхлестать
их министра. Всё это делается, имхо, перед тем, как Байден пойдёт на очередные,
видимые и не очень, уступки Путину.

Надо бы последить за их риторикой в СМИ накануне телефонного разговора.

\iusr{борис романенко}
Януковича слили, и устроили террор в стране

\iusr{Алиса Лисина}

Интересно, а не хочет ли БЛИНкен пояснить украинцам действия американских копов
при штурме БД сторонниками Трампа во время выборов в США?

\iusr{Надежда Гурьянова}
Господи что еще можно ожидать от Блинкина. Пора уже привыкнуть врут как дышут.
Но бумеранг никто не отменит.

\iusr{Стефания Губина}
Базы НАТО. \enquote{Оборонные}..... ага...

\ifcmt
  ig https://scontent-lhr8-2.xx.fbcdn.net/v/t39.30808-6/264234615_6534600599946337_3137352436979788266_n.jpg?_nc_cat=105&ccb=1-5&_nc_sid=dbeb18&_nc_ohc=jbb14gnh0UwAX9o5KYN&_nc_ht=scontent-lhr8-2.xx&oh=5adb1eb8d838c801300189b3ec04f5bd&oe=61B4D950
  @width 0.4
\fi

\iusr{Наталья Бальская}
Лжёт, потому что считает всех нас и весь мир идиотами.. «Лжём, а вы жрите»
Господи, пусть у него будет ровно столько здоровья, сколько правды в его словах  @igg{fbicon.hands.pray}{repeat=4} 

\iusr{Alex Yurochkin}

Гибель любой империи сопровождается глобальной ложью как \enquote{средством} дающим
шанс на выживание! Но ложь спасает на время... дальше распад!

\iusr{Константин Лобанов}
В отличие от ... я не считаю что Украина это просто на 100\% миньон штатов.

Конечно есть и такое, но укрополитики при всей своей карикатурности хорошо
усвоили какие речи надо произносить и когда чтобы американская машина
поворачивала куда-то

Напоминает ввод советских войск в Афганистан. Вопреки расхожему мнению СССР
вовсе не хотел вводить войска ибо руководители в общем то понимали, что это
будет. Но Бабрак Кармаль хорошо понимал как работает советская управленческая и
идеологическая машина и, так сказать, владел умением идеологических заклинаний
советского дракона и непрерывно десятки раз совершая поездки в СССР раз сумел
таки, произнося много слов на Политбюро об интернационализме, социализме
империализме и т.д., уговорить ввести войска в Афганистан.

Пятнадцать лет примерно в ту же ловушку попали и американцы во Вьетнаме.

\iusr{Михаил Болотовский}
Такая работа, М.Б.

\iusr{Виолетта Кумылко}
Михаил! Сегодня 7 декабря- поздравляю вас с Днем Рождения! Здоровья вам и всех благ!

\ifcmt
  ig https://scontent-lhr8-1.xx.fbcdn.net/v/t39.30808-6/262443428_4498302530291447_6519720384399102825_n.jpg?_nc_cat=100&ccb=1-5&_nc_sid=dbeb18&_nc_ohc=gdENG-XfL00AX9Nwz71&_nc_ht=scontent-lhr8-1.xx&oh=fe94221f62c34615d0cef72d7e3d8eff&oe=61B48757
  @width 0.4
\fi

\iusr{Борис Михайлов}
"Поставить на грань" - не работает. Работает - поставить на колени!

\iusr{Анастасия Романова}
Все игроки, у каждого есть своя роль, которую они с успехом играют. Не прошло и
8 лет. Чем это закончится? Да все прекрасно знают.

\iusr{Света Выгановская}
А он не признался, кто переворот устроил???

\iusr{Марина Айдова}
А почему не работает? Всё отлично работает.

\iusr{Michael Kot}

"лояльные тогдашнему президенту Виктору Януковичу, обстреляли мирных
демонстрантов в Киеве и убили более 100 человек" - это краткий пересказ
постмайданной мифологии, принятой за официальную версию украинской властью.  @igg{fbicon.smile} 
"Глава Госдепа также назвал НАТО «оборонным альянсом»." - в любой стране
военное ведомство называется Министерством обороны.  @igg{fbicon.smile} 


\iusr{Mikhail Birukov}
Совести у них нет в этом НАТО.

\iusr{Елена Исаева}
Увы, это работает, особенно для тех, кто хочет трактовать как Блинкен

\iusr{Natalia Mieshkova}
Как сказал Ридли Скотт:
-Мы сдохнем с этой политкорректностью...

\iusr{Vadim Slim}
Это им очень выгодно

\iusr{Andrei Studenetskii}

Это высказывания для своей стороны. Они исходят из того, что западной аудитории
слышен только их голос. Соответственно и пропагандируют. Холодная война, она и
есть холодная ВОЙНА.

\iusr{Константин Лобанов}

Блинкен (и его помощники) ничего не читают, а если и читают, то это не служит
им никаким фактом и никакой точкой раздумья!!! Для них существует только свое
мнение и неважны документы.

Как это не покажется странным это так и есть.

\enquote{Все читают и все знают} - да понятие \enquote{знают} носит для этих странных людей
несколько не то, что раньше.... Это Бжезинский знал, это Киссинджер знал но не
эти


\iusr{Алексей Манмарёв}

Михаил Борисович, чему Вы удивляетесь. Прекрасно этот бывший киевлянин всё
знает. Обама же сказал, что они были брокерами майдана, поэтому Блинкин и
нападает, т.к. лучшая защита - это нападение.


\iusr{Vitaliy Lashcenko}
Ага, мырни протэстувальныкы  @igg{fbicon.monkey.see.no.evil} 

\iusr{Вера Буданова}
Всё они знают!

\iusr{Antonij Melnik}
...боюсь спросить: "кровавий воєнний пєрєворот", - то коктейлі молотова?

\iusr{Валентин Леднев}
К сожалению, спесь и чванство, как и наглость, лечатся только хирургически. Или
когда становится очевидной неотвратимость наказания.


\iusr{Алексей Чернега}
Хуцпа - как основа государственной политики.

\iusr{Василий Шишов}
Уже поставили !

\iusr{Mykola Shcherbyna}
\textbf{Михаил Погребинский} Михайло! Що з тобою сталося? В 1990 році ти був інший!

\iusr{Алексей Ч.}

\enquote{Хазария им. троцкого} берет истореский реванш и по-обычаю- проигрывает Руси
православной. \enquote{Бог не выдаст, свинья не съест...}

\iusr{Вероника Беллиссима}
Это не доходит и до 80\% населения Украины.. увы

\begin{itemize} % {
% -------------------------------------
\ii{fbauth.zubko_viktoria.harkov.ukraina}
% -------------------------------------

\textbf{Вероника Беллиссима} доходит, и даже до 80 проц!! Только эти проценты в отчаянии и не знают, делать-то что????

\iusr{Вероника Беллиссима}
\textbf{Виктория Зубко} ничего уже.. поздно.

\iusr{Виктория Зубко}
\textbf{Вероника Беллиссима} 

никогда не поздно. После 41 был 45. Многие зимой 41 думали, что поздно...
Давайте думать о нашей победе!!!!

\iusr{Вероника Беллиссима}
\textbf{Виктория Зубко} да!! @igg{fbicon.hands.shake} 

\iusr{Valentyna Zakharova}
\textbf{Вероника Беллиссима} 

80\% это наверное многовато, но безголовых и непонимающих вообще, что произошло и
тем более что происходит действительно много.

\iusr{Вероника Беллиссима}
\textbf{Valentyna Zakharova} та я просто по своим друзьям сужу. А так то да, думаю меньше.

\iusr{Виктория Зубко}
\textbf{Valentyna Zakharova} в Харькове-10-15 проц, и то, это уже совсем упоротые.

\iusr{Valentyna Zakharova}
\textbf{Виктория Зубко} У меня печальнее....

\iusr{Виктория Зубко}
\textbf{Valentyna Zakharova} да.. у вас там тяжко... Но мы не паримся, уж извините, давно ваших за своих не считаем

\iusr{Вероника Беллиссима}
\textbf{Виктория Зубко} ну люди разные есть и там.. просто их там меньше конечно, гораздо, но есть.
\end{itemize} % }

\iusr{Татьяна Божко}

Ну послушать Тони так и его деда давили русские танки, там просто личная
плебейская русофобия замешаная на пещерной ненависти и тупой некчемной зависти,
к силе духа и величии славян, вот жаба его давит, комплексы, комплексы ....

\iusr{Сергей Ткебучава}

\obeycr
Я задаю этот вопрос с 2014 года.
Внятного ответа пока не услышал.
Если это Янукович санкционировал "обстрел мирных протестующих в Киеве и убийство более сотни людей, вышедших на митинг", с целью подавить протест, почему среди погибших нет ни одного самого захудалого сотника?
Я уже не говорю о вожаках майдана.
Все погибшие или рядовые протестующие, или, вообще случайные прохожие.
А для подавления массовых протестов работать надо по заводилам, вожакам.
Кстати, среди застреленных правоохранителей офицеров много.
\restorecr

\iusr{Юрий Липатов}

Блинкен феерический лжец. Так нагло врать может только чудовище. Ну как
официальное лицо может обвинять в убийствах кого-то без решения суда?


\iusr{Сергей Самфопалов}
Блинкен находился под впечатлением тронной речи подкидыша "к нации..."!)))

\iusr{Виктор Рудов}

А в 2014 году подобные посты Вам, и другим лидерам (за некоторым исключением)
общественного мнения, слабо было написать?? Что непонятно было после 2004 года?
Ваше молчание и трусость, в том числе Ваша, привели к сегодняшней ситуации

\iusr{Виктор Павлухин}

Очень импонирует фраза Михаила Борисовича "оборонительный альянс". Под
впечатлением вчерашней поездки НАШего в Полтаву и посещения там музея авиации
вчера написал пост про этот "оборонительный альянс", далеко не все и всё знают
об этом и фильмом подкрепляется:

\href{https://m.facebook.com/story.php?story_fbid=2955691878076186&id=100009062079713}{%
Советский самолёт ТУ-160, Виктор Павлухин, facebook, 05.12.2021%
}

\begin{multicols}{2}
Так \enquote{наши} \enquote{стратегические партнёры}, построившие в своё время 744 единицы
бомбардировщиков Боинг В-52, 104 единицы ( подобного, но более скромного по
всем параметрам чем ТУ-160) В1 \enquote{Лансер}, уничтожали непревзойденный ни кем по
сей день Советский самолёт ТУ-160 нескольких модификаций, построенный всего в
количестве 35 единиц! У них же была такая \enquote{оборонительная} доктрина, а у нас с
35-ю самолётами \enquote{наступательная}. 

А сколько выбросов углекислого газа в атмосферу планеты они сделали только
от производства этих 744+104 самолётов, и 11 авианосцев набитых
\enquote{оборонительными} истребителями, от строительства 51-й атомной подводной лодки,
против 28 Советских, а потом Российских. Вот их не здоровая на головку Грета
Тунберг с трибуны Давоса и обеспокоена этими выбросами парниковых газов, а они
утверждают, что коровы и мужики в этих выбросах больше вреда приносят (больше
нежданчиков выпускают из ЖКТ), какие они всё таки смешные, если не сказать
больше, \enquote{наши} ( не мои точно) \enquote{стратегические партнёры}!

\ifcmt
  ig https://external-lhr8-1.xx.fbcdn.net/safe_image.php?d=AQHzu3ITmBz_-Hr3&w=500&h=261&url=https%3A%2F%2Fi.ytimg.com%2Fvi%2FJZnqusD4AUo%2Fmaxresdefault.jpg&cfs=1&ext=jpg&_nc_oe=6f2b7&_nc_sid=06c271&ccb=3-5&_nc_hash=AQGzygshiX1P99-b
  @width 0.4
\fi

\href{https://www.youtube.com/watch?v=JZnqusD4AUo}{%
\enquote{Білий лебідь} Ту-160, останній стратегічний бомбардувальник-ракетоносець в Україні, %
Українська правда, youtube, 24.06.2021%
}

\begingroup
\em

Найбільший, найпотужніший і найважчий бойовий літак. Найшвидший бомбардувальник
у світі. На момент розпаду СРСР в Україні було 19 надзвукових Ту-160. Тепер
один екземпляр – музейний. Репортаж УП з Полтави

Текстовий повний репортаж з фото читайте на УП: 

\href{https://www.pravda.com.ua/articles/2021/06/25/7298381/}{%
Таємниці "Білого лебедя". Пілоти бомбардувальної авіації розповідають про ліквідовані в Україні Ту-160, %
pravda.com.ua, 25.06.2021%
}

Наприкінці 1980-х XX століття, коли помітно вщухла Холодна війна між Кремлем і
Білим домом, радянський воєнпром завдав точковий удар по ексзаступнику
директора ЦРУ, міністру оборони США Френку Карлуччі, який прилетів з робочим
візитом до СРСР.

На хвилі перебудови і гласності гостю зі Штатів вирішили показати гордість
радянської дальньої авіації – надзвуковий стратегічний бомбардувальник Ту-160 з
крилом змінної стріловидності.

\ifcmt
  ig https://i2.paste.pics/6abf35b176875acab710ccbf6440452d.png
  @width 0.4
\fi

Екскурсія у кабіну засекреченого ракетоносця ледь не коштувала здоров'я главі
Пентагону. Незважаючи на невеликий зріст, він зачепився за один із перемикачів,
коли сідав у крісло командира. Розбиту до крові брову Френка Карлуччі обробили
і заліпили пластиром.

\ifcmt
  ig https://i2.paste.pics/48da32fab92531ba387c56c6a1a39d35.png
  @width 0.4
\fi

З 1987 року і до моменту розвалу Союзу в небо встигли підняти 21
бомбардувальник, прозваний радянськими пілотами "Білим лебедем", а в країнах
НАТО – Blackjack. 19 з них базувалися в українських Прилуках і всього два – на
території РРФСР.

Мине не так багато часу, і спогади про конфуз із Карлуччі будуть не такими вже
й веселими для військової верхівки і партійної еліти. І Пентагон, і Білий дім,
а заодно і весь колективний Захід у своєму суперництві з СРСР опиняться в ролі
тих, хто сміється добре і останнім.

\ifcmt
  ig https://i2.paste.pics/2a3e0620f0bf5d4e753ecfbdd295bb30.png
  @width 0.4
\fi

Для багатьох пілотів дальньої авіації настане кінець їхньої кар'єри – в
незалежній Україні відмовляться від ядерного озброєння. На металобрухт поріжуть
десять Ту-160, а вісім віддадуть РФ замість боргів за газ.

Ще один, останній в Україні екземпляр, у 2000 році, похитавши на прощання
крилами, своїм ходом прилетить із Прилук до Полтави. Там його перетворять на
єдиний поки в світі музейний експонат Ту-160.

Що стало з командиром ескадрильї, який прокатив коханку на Ту-160? Якими були
польоти на швидкості, що вдвічі перевищує швидкість звуку? Та як у США з’явився
знімок секретного ракетоносця? Репортаж УП з Музею важкої бомбардувальної
авіації в Полтаві.

У репортажі використані 
\href{https://www.youtube.com/c/ВасилийХмелевский/featured}{%
відео Василя Хмельовського %
}

Автори: Ельдар Сарахман, Євген Руденко/Українська правда

\headTwo{Коментарі до відео}

\iusr{Дмитрий Нецвет}

Жуткое зрелище: лётчики, которым обрезали крылья. С какой любовью они
рассказывали о Белом Лебеде...

\begin{itemize} % {
\iusr{владимир васильев}

Тот кто любил эти машины тот летает а это т летун за пайку продался

\iusr{Николай Кузменка}

Это просто пипец, я представляю састояние лётчика, какая у него обидана этих долбодятлов

\iusr{Созвездие Цефей}
Вот..

\iusr{Andrej Fedorenko}

@Николай Кузменка  ему пофиг ... Это самое главное... Он хочет лишь
покрасоваться, какой он был зайчик...

\iusr{Веталь Штобза}
@владимир васильев. Свой словесный понос поясни внятно

\iusr{0niky}

@Веталь Штобза  Посмотри фильм 72 метра, там все понятно показано, только про
флот. Кто то согласился резать свои крылья, но жить в теплой мазанке, кушая
абрикосы. А вот часть людей гнали свои машины в Россию, что бы жить там в
ужасных условиях, месяцами не видя зарплату, но при этом оставляли себе шанс
когда нибудь снова полететь на спасённых машинах... И они полетели. А те кто
остался в теплой Украине, сейчас тебе в видео рассказывают какие классные у
них были самолёты, когда то.

\iusr{Веталь Штобза}

@0niky   каждый делает свой выбор . Кто-то думает о доме благополучии своей
семьи а кто-то ради ради красивых понтов и грозного вида готов жить в параше
хуже бедной Украины, имея при этом четверть Мировых запасов полезных
ископаемых. История поставляет всё на свои места вы  все деньги пропивать на
всякие дубинушки которые годятся только индейцев бомбить и при этом собирать
смс ками деньги на лечение детей и жить без канализации и возмущается что
морковка подорожала

\iusr{Юрий Ким}
@Веталь Штобза  

Не перевирайте.  На тот момент Украина была намного богаче России.  И вам бы я
не советовал обзывать великую страну. Это вас не красит.  Да и время все
расставило на свои места.  Теперь вас называют страной 404, а Россию даже
Байден называет великой державой.  Что у вас в голове?)))

\iusr{0niky}
 @Веталь Штобза  

Ты можешь сколько угодно переворачивать все с ног на голову, это вполне
понятно, когда нет смелости посмотреть правде в глаза. По факту, отдали вы свой
суверенитет за сладкий пряник, а по итогу ни пряник не получили ни суверенитет
не оставили. Про СМСки ты верно сказал, и про благополучие, только перевернул
всё, это на Украине нет бесплатной медицины от слова совсем и это на Украине
зарплаты и пенсии кратно меньше российских, поэтому и расползается народ из
вашей страны, в том числе и в Россию бегут. Не бывает сильной страны без
сильной армии, но до вас это за 30 лет не дошло, а сейчас уже поздно, дурака
учить - только портить.

\iusr{Олег Онищенко}

У меня сосед по гаражу в Васильков-Подгорцы служил, вышел на пенсию, за пять лет
прогорел, весной похоронили, мужик был, золото.

\iusr{Веталь Штобза}

@Юрий Ким   тормоз в Великой стране 20 милионов людей живут без тёплого
туалета . В школах дети на дырки ходят. Ответь мне на один вопрос величие
ваше всём заключается ???????

\iusr{Веталь Штобза}

@Юрий Ким   страну 404  я недавно видел как на Олимпиаде без флага выходила.
Проспали четверть Мировых запасов полезных ископаемых и живёте в параше как
курильщики. Морковка подорожала и пол страни завыло. Великие Бл...ть

\iusr{Юрий Ким}
@Веталь Штобза  

Я тоже видел "великую державу" на Олимпиаде. Видел ее великого "сына", очень
загорелого. Его, правда, потом в Киеве хотели послать в Африку гулять и банан
на дорогу давали) Из достижений что еще?  Тризуб из сала?))))

\iusr{Юрий Ким}

@Веталь Штобза  

Вот мне интересно, где вам рассказывают про 20 миллионов без тёплого туалета?)
Что ни )(0)(0л, то одна и та же методичка)))
Вы реально тупые или прикидываетесь?))))

\iusr{Веталь Штобза}

@Юрий Ким   по кисель тв тебе точно не расскажут. Научись интернетом
пользоватся.

\iusr{Веталь Штобза}
 @Юрий Ким   

представитель великих ответь мне почему вы завыли когда морковка подорожала?
Или ты тоже не слышал об этом?

\iusr{Юрий Ким}
 @Веталь Штобза  

Морковка подорожала?
Как вам трудно то живётся!
День и ночь про цены на моркву и туалеты в стране-агрессоре думаете)))

\iusr{Юрий}

Так хуторянам  такой самолет в Агро - державе  не нужен. Селюки  сами их
распилили на металл за амерские фантики, а остатки за долги России отдали... 

А те  украинские летчики, с любовью о нем вспоминавшие  на выборах за
конфетного короля или игрока  членом на рояле голосовали?  Они имея такую
профессию и штучную квалификацию остались на Украине? Ради чего? Что у них в
голове?

\iusr{Кэш}

@Веталь Штобза  

почему подорожал газ налоги хлеб свет вода продукты ,почему пенсия в
1600гривен и зарплаты у депутатов в 500.000гривен в месяц а у некоторых 1миллл

\iusr{Gerhart}

@Юрий Ким  лапоть стрелки не переводи  @igg{fbicon.face.clown} 

\iusr{Дмитрий Маковий}

@0niky  какая теплая Украина, средние зарплаты и пенсии в РФ выше, да и
самолеты там новые изобретаются и производятся, а ТСН конечно интересный
канал, но советую вам поменьше его смотреть

\end{itemize} % }

\iusr{qpigaro}

С пониманием и сожалением смотрю на советских летчиков. Хорошее душевное видео,
крошечного момента,  жизни - самолета и человека.

\iusr{Небоскрёбы России}

Прекрасный репортаж о лётчиках и легендарном самолёте. Спасибо.

\iusr{Igor Muller}
Сел посмотреть историю самолета, а посмотрел историю человека.

\iusr{VietCong}

Я тоже служил на этом аэродроме в Полтаве 89-90 г.г. в 666 ОБС,  батальон
назывался «десятка» по номеру в/ч 10391.  К самолетам моя служба напрямую не
относилась, служил на ЗАС в узле связи штаба дивизии 13 Гв. ТБАД. 

Тогда на аэродроме в Полтаве стояли только Ту-22 и Ту-16. Про Ту-160 в другом
полку дивизии слышал, но их не видел. Состав батальона был интернациональный,
русские, украинцы, грузины, армяне, азербайджанцы, литовцы, узбеки, башкиры,
мордвины, казахи. Никакой неприязни между русскими и украинцами вовсе не
ощущалось, даже если украинцы были с западной части Украины. 

Обособленно и даже несколько высокомерно себя держали только прибалты, что было
заметно уже даже тогда. 

Общение шло исключительно на русском языке и никого это не напрягало. Конечно
недовольство какое-то и проскальзывало, особенно заметно было как украинцев
обижали насмешки над их ридной мовой, когда те или иные слова казались русским
смешными. Но не было и в помине злобы и ненависти, ставшие сейчас чуть ли не
нормой наших отношений.

Тогда такого что творится сейчас и в страшном сне невозможно было представить
:(

Об Украине и её людях у меня на всю жизнь сложилось самое уважительное,
приятное и положительное впечатление, категорически идущее в разрез с тем как
сейчас наша российская пропаганда пытается показать украинцев какими-то
ненормальными  кастрюлеголовыми придурками. 

То что сейчас мне никак невозможно снова увидеть Киев, Харьков, Полтаву просто
не укладывается в голове. Это на грани какого-то сумасшествия. 

Огромное «спасибо» всем кто тем или иным образом поспособствовал разжиганию
ненависти между нашими братскими народами, очень сильно надеюсь что вы за это
ответите. 

Ну а исторический опыт показавший что все даже смертельно враждующие народы
рано или поздно примиряются, как к примеру СССР с Германией, то надеюсь что и
Россия с Украиной помирятся. Жаль только что не всем нам получится этого
дождаться.

\iusr{Евгений Ковалев}

ту-160 это произведение искусства в авиастроении .

\endgroup

\end{multicols}

\end{itemize} % }
