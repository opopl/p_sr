% vim: keymap=russian-jcukenwin
%%beginhead 
 
%%file 27_08_2021.fb.mogilnickij_maksim.1.albina_sokolova_jazyk_britania
%%parent 27_08_2021
 
%%url https://www.facebook.com/permalink.php?story_fbid=6244494212287391&id=100001806235447
 
%%author Могильницкий, Максим
%%author_id mogilnickij_maksim
%%author_url 
 
%%tags britania,diskriminacia,jazyk,ukraina,ukrainizacia,uvolnenie
%%title Суд признал незаконным увольнение Альбины Соколовой за то, что в рабочее время говорила на русском языке
 
%%endhead 
 
\subsection{Суд признал незаконным увольнение Альбины Соколовой за то, что в рабочее время говорила на русском языке}
\label{sec:27_08_2021.fb.mogilnickij_maksim.1.albina_sokolova_jazyk_britania}
 
\Purl{https://www.facebook.com/permalink.php?story_fbid=6244494212287391&id=100001806235447}
\ifcmt
 author_begin
   author_id mogilnickij_maksim
 author_end
\fi

Суд признал незаконным увольнение Альбины Соколовой за то, что в рабочее время
говорила на русском языке. Кроме того, за оскорбление чувств и косвенную
дискриминацию суд взыскал с работодателя почти одиннадцать тысяч компенсации.

Удивительно, правда? На самом же деле, ничего необычного. Просто дело было не в
Печерском районном суде Киева и даже не в Украине, а компенсацию взыскали не в
гривнах, но в фунтах стерлингов. Да-да, мы говорим о Великобритании.

В Украине же, граждане которой, в большинстве своем, общаются на русском,
увольняют за него сплошь и рядом. И ни один из пострадавших от дискриминации не
помышляет идти со своей бедою в родной украинский суд. Он рад и тому, что
работодатель вышвырнул его на улицу без «волчьего билета».

Такая вот Европа у нас получилась.

