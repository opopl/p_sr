% vim: keymap=russian-jcukenwin
%%beginhead 
 
%%file 29_12_2020.news.ua.strana.venk_victoria.1.sssr_ss_crime
%%parent 29_12_2020
 
%%url https://strana.ua/news/309295-kak-v-ukraine-presledujut-za-simvoly-sssr.html
 
%%author 
%%author_id venk_victoria
%%author_url 
 
%%tags 
%%title Пять лет за шапку-ушанку. Как в Украине сажают за символы СССР и не трогают за SS
 
%%endhead 
 
\subsection{Пять лет за шапку-ушанку. Как в Украине сажают за символы СССР и не трогают за SS}
\label{sec:29_12_2020.news.ua.strana.venk_victoria.1.sssr_ss_crime}
\Purl{https://strana.ua/news/309295-kak-v-ukraine-presledujut-za-simvoly-sssr.html}
\ifcmt
	author_begin
   author_id venk_victoria
	author_end
\fi

\ifcmt
  pic https://strana.ua/img/article/3092/kak-v-ukraine-95_main.jpeg
	caption "Гражданский арест" парня за шапку с кокардой Советской армии. Фото "Дело" 
\fi

Во Львове хотят отправить в тюрьму 19-летнего парня, который гулял по городу в
шапке с красной звездой. Как оказалось, донес на него в полицию местный депутат
от "Голоса", который и произвел "общественное задержание". 

То есть истории с приговорами людям, которые демонстрируют советскую символику,
продолжаются. При этом нет ни одного приговора тем, кто марширует с нацистскими
символами.

"Страна" разбиралась в деталях нового скандала. 

\subsubsection{Как депутат от "Голоса" подвел под тюрьму туриста}

Местная полиция вчера сообщила о "страшном" преступлении: молодой парень из
Киева расхаживал по городу в шапке-ушанке с кокардой советского образца. 

Точнее, не расхаживал, а сидел в одном из кафе на улице Валовой. Оттуда и
поступил сигнал. На место выехала целая оперативно-следственная группа. 

Оказалось, что "преступнику" 19 лет.

У него изъяли орудие преступления - шапку - и завели уголовное дело по статье
436-1. Она обещает до пяти лет лишения свободы и конфискацию имущества за
изготовление, распространение и пропаганду тоталитарных режимов -
коммунистического и нацистского. 

\ifcmt
  pic https://strana.ua/img/forall/u/0/92/133923192_5388396457867382_6371690138075176007_o[1].jpg
\fi

В сообщении полиции указывается именно эта санкция статьи - то есть самая
тяжелая ее вариация. Хотя очевидно, что привлечь парня можно только за
"пропаганду" - изготовлением, продажей или раздачей символики он явно не
занимался. 

Львовские СМИ рассказали подробности этой истории. Оказывается, полицию вызвал
местный студент-журналист Игорь Шолтыс, который участвовал в АТО. Он и провел
"общественное задержание". По его словам, парней из Киева было двое. 

\ifcmt
  pic https://strana.ua/img/forall/u/0/92/132622950_1242967269495604_2617042568871135775_n-768x1024[1].jpg
	caption "Общественный арест" ведет "преступника" в ушанке с кокардой времен СССР. Фото dilo.net.ua 
\fi

"Расспрашивал, где они взяли эту символику. Сказали, что приобрели на
"барахолке", потому что было холодно. Что такая символика запрещена, говорят,
не знали. Мы провели разъяснительные работы. Оказалось, что они русскоязычные
киевляне", – рассказал Шолтис.

По его словам, потом на место вызвали полицию, но у одного из патрульных "были
откровенно не проукраинские взгляды", что чуть было не сорвало всю
"спецоперацию".

"Мы зафиксировали и его данные, проведем разъяснительные работы, потому что
человек совсем не соответствует своей компетенции", - пообещал Шолтыс. 

Интересно, что этот активист на местных выборах прошел от партии "Голос" во
Львовскую объединенную территориальную громаду и стал там самым младшим из
депутатов - ему 24 года. Тем не менее, судя по его Фейсбуку, парень успел
повоевать на Донбассе в батальоне ОУН.  

\ifcmt
  pic https://strana.ua/img/forall/u/0/92/%D0%BE%D1%83%D0%BD.jpg
\fi

Также он принимал участие в "Форуме молодых политиков", одним из организаторов
которого была бывшая С14, сегодня носящая имя "Общество будущего". 

Шолтыс - классический активист националистического толка. Он участвовал в
митингах против министра образования Сергея Шкарлета, называл "манкуртом"
боксера Василия Ломаченко за его высказывания о России и так далее. 

\subsubsection{Как в Украине преследуют за коммунизм}

За время работы закона о запрете символики тоталитарных режимов дела заводились
только на тех, кто был замечен с коммунистическими символами. При этом вторая
часть закона - которая про нацизм - де-факто не выполнялась. Хотя примеров ее
демонстрации была масса. 

Законодательство позволяет привлекать к ответственности даже тех, кто публикует
"неправильные" картинки в соцсетях. Первый приговор в Украине по статье 436-1
УК касался именно пропаганды коммунизма. Причем тоже во Львове.

Приговор был вынесен в апреле 2017 года Галицким судом (это тот же район, где
задержали парня в шапке-ушанке). Студент-третьекурсник Львовского университета
публиковал в соцсетях цитаты Владимира Ленина. Какие - не уточняется, но
известно, что парня вела СБУ, которая провела у него дома обыск и забрала
компьютер. А также книжку "Капитал" Карла Маркса. 

Суд дал ему 2,5 года, но с отсрочкой на год - и если за это время не будет
нарушений, то срок отменяется. 

Это стало итогом сделки подсудимого со следствием - то есть защищать себя
молодой человек не стал и признал вину. Но интересно, что первой жертвой
антикоммунистов стал не возрастной человек, а очень молодой. Как, кстати, и
сейчас - в случае с шапкой-ушанкой. 

Впрочем, при Порошенко это никого не удивляло. Во время его президентства,
кстати, не только коммунистическая символика попала под запрет. В 2017 году
Рада признала нарушением демонстрацию георгиевской ленты,\Furl{https://strana.ua/news/71024-v-rade-progolosovali-za-zapret-ispolzovaniya-georgievskoj-lenty-v-ukraine.html} установив за ее
ношение штрафы и арест в 15 суток. 

Однако приговоры за советскую символику продолжили выноситься и при Зеленском.
Причем в его родном Кривом Роге. В октябре 2019 года местный житель - простой
мойщик окон - получил год условно за футболку с гербом СССР. \Furl{https://strana.ua/articles/analysis/230817-hod-uslovno-za-herb-sssr-sazhajut-li-sejchas-v-ukraine-za-sovetskuju-simvoliku.html}

\ifcmt
  pic https://strana.ua/img/forall/u/0/92/111(62)[1].jpg
\fi

Мужчина также подписал соглашение с прокурором и таким образом избежал тюрьмы.
Интересно, что вынес этот приговор Дзержинский районный суд - по имени
создателя ВЧК Феликса Дзержинского. 

В мае 2020 года на родине Зеленского завели дело на пенсионера, который
расклеивал листовки со словами "Слава Красной армии и русскому солдату". \Furl{https://strana.ua/news/265891-natspolitsija-ukrainy-zaderzhala-muzhchinu-kotoryj-risoval-na-otkrytkakh-simvoliku-sssr.html}


\ifcmt
  pic https://strana.ua/img/forall/u/0/92/91_main[1](1).jpeg
\fi

В том же месяце дело открыли на селянина из Одесской области, который к
Первомаю вывесил над своим домом красный флаг.\Furl{https://strana.ua/news/265084-protiv-vyvesivsheho-na-dome-sovetskij-flah-odessita-zaveli-uholovnoe-delo.html}

\ifcmt
  pic https://strana.ua/img/forall/u/0/92/84_main[1](1).jpeg
\fi

\subsubsection{За СС Галичину - ни одного приговора}

Параллельно националисты, которые выкладывают в соцсетях нацистскую символику
или маршируют с ней по улицам городов, в глазах украинской Фемиды ни в чем не
повинны. 

Показательной стала попытка запретить символику дивизии СС "Галичина",
созданную нацистами на оккупированных территориях из местных
коллаборационистов. 

В мае 2020 года  Окружной админсуд Киева отменил решение Института нацпамяти,
который заявлял, что символика "Галичины" не подпадает под уголовную статью. 

Решение было крайне осторожным, но вызвало резкое неудовольствие ультраправых,
которые привыкли ежегодно маршировать с флагами этого соединения. И после угроз
судьям, апелляция решение суда отменила. 

Детальнее об этом - в материале "Страны" "Лижете ноги оккупантам". Как после
угроз националистов суд отменил запрет на символику дивизии СС "Галичина".

Но не только за символику "Галичины", но и за нацистские символы вроде свастики
или флага Третьего рейха никаких дел в Украине не заводят. Хотя примеров такой
пропаганды масса среди уличных радикалов.

Вот, например, националист и агент СБУ Алексей Цымбалюк, который у себя в
Фейсбуке публикует фото нацистских символов и "зигует". 

\ifcmt
  pic https://strana.ua/img/forall/u/0/92/%D0%BE%D1%80%D0%B5%D0%BB[1](1).png
\fi

Или активист-радикал Виталий Регор. В его соцсетях полно фотографий примерно
такого содержания.

\ifcmt
  pic https://strana.ua/img/forall/u/0/92/76_main[1](1).jpeg
\fi

Больше подобных историй - в материале "Страны" От "Мертвой головы" до "великого
Гитлера". Как украинцы попадали в нацистские скандалы.\Furl{https://strana.ua/news/176575-istorii-ukraintsev-simpatizirujushchikh-hitleru-i-natsistskoj-hermanii-foto-video.html}

