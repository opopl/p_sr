% vim: keymap=russian-jcukenwin
%%beginhead 
 
%%file moje.kremlevskie_narrativy.jazyk.obsuzhdenie
%%parent moje.kremlevskie_narrativy.jazyk
 
%%url 
 
%%author_id 
%%date 
 
%%tags 
%%title 
 
%%endhead 

\paragraph{Обсуждение в обществе взаимоотношения русского и украинского языков}
\label{sec:moje.kremlevskie_narrativy.jazyk.obsuzhdenie}

Можно было б уже сделать целую книжку на десятки тысяч страниц, была бы воля.
Что касается обсуждения в народе этой проблемы, то можно выделить такие тезисы,
как, например:

\begin{itemize} % {
\item русский язык часто называется языком московским, мокшанским, московитским - то есть, считается, по-видимому,
что он ведет свое начало от Москвы, что он также был исскуственно насажден в Украине;
\item также, русский язык называют языком оккупантов;
\item настойчиво проталкивается мысль о том, что во времена СССР была так
называемая \enquote{русификация}, что русскоязычные жители крупных городов,
таких, как Харьков, Одесса, Киев, являются \enquote{зросійщеними};
\item есть тезис о том, что Россия (\enquote{Московия}), в различные
исторические периоды постоянно и последовательно пыталась
\enquote{уничтожить} украинский язык и культуру, приводятся длинные
списки ограничений и запретов, так например Эмский указ или же
Валуевский циркуляр;
\end{itemize} % }

\ifcmt
  tab_begin cols=2,no_fig,center

     pic https://i.ytimg.com/vi/CrXtezJ8gKQ/maxresdefault.jpg
     @caption Екатерина Жарких, киевский журналист и блоггер

     %pic https://pbs.twimg.com/media/EB8RmtpXsAAucfs.jpg
     pic https://pbs.twimg.com/media/Esb3HRgXAAEuv6e.jpg
     @caption Евгения Бильченко, профессор культурологии, университет Драгоманова в Киеве 

  tab_end
\fi

Стоит заметить, что комментирование постов по языковую тематику (например, если
взять наиболее популярные посты Евгении Бильченко или Екатерины Жарких) обычно
сопровождается необычайным накалом эмоций, с откровенными оскорблениями
личности оппонента, что, безусловно, не есть хорошо. В ответ на природные
человеческие аргументы, защищающие русский язык и его носителей, например, что
русский язык в Украине имеет право на существование, что русский язык -
является для его носителей родным языком, на котором мама пела колыбельные, или
же что русский язык нужно сделать вторым государственным, поскольку его 
носители наравне вносят свой вклад в строительство государства, или же то, что
русский язык защищен 10-ой статьей Конституции, и тому подобное, в ответ можно
услышать такие словесные конструкции (некоторые есть буквальные комментарии
пользователей):

\begin{itemize} % {
\item Не подобається мова, геть до Ростова; 
\item Чемодан-вокзал-Россия;
\item Думай хоть на арабском или испанском, а если ты живёшь в Украине, то будь
любезная общайся, когда надо, на українській мові;
\item Тільки бидло не може вивчити мову своєї Країни;
\item Якщо ви живете в Україні, Ви повинні знати українську мову;
\item Той, що не знає української, прстійно мешкаючи в Україні називається розумово неповноцінним, а той, що не бажає ­ звичайним покидьком­шовіністом;
\item Ніколи не забороняли російську, а навпаки ­ більше 100 указів було заборони української мови в Україні і робили це з росії... ніхто не обурюється, як завжди, тільки москалі;
\item Для применения русского языка имеется огромная территория под названием РФ;
\item Ненавидять українську мову тільки малороси;
\item Тільки окупант татаромонгольского віросповідання не приймає національні традиції і всюди насаджує свій ЯЗЫК, який набагато молодший і бідніший за українску мову;
\item брєд говорити російською в Україні, а не українською, як і брєд, що в Україні половину рускоговорящих;
\item Всі руцкацєлюсниє та лаптєногіє, на родіну!!!!
\item марш звідси в ростов. Якшо не влаштовує українська мова вперед за поребрик;
\item Для допиту полонених рузька підійде:))
\item Російська мова як мова корінної титульної нації повинна бути в РФ. А в Україні ­ українська мова ­ мова корінної титульної нації;
\item На нашій планеті та, скоріш за все, і в усій Галактиці прийнято в тій країні де ти проживаєш розмовляти мовою тієї країни, а не своєю рідною;
\item Це взагалі дикунство жити в Україні і виступати проти мови корінного населення;
\end{itemize} % }

\paragraph{Статус украинского языка как стержня украинской государственности}

Как видно по эмоциальной окраске подобных комментариев, вопрос об
сосущенствовании украинского и русского языков является в Украине одним из
наиболее острых в психологическом плане. Украинский язык считается одним из
стержневых факторов Украинской государственности, самой гарантией независимости
Украины как суверенного государства, в то время как русский - наоборот, как
инструмент, подрывающий саму идею Украины, как государства, и ставящий под
сомнение само существование украинской нации и культуры, как совершенно
отдельных от русской нации и культуры. Люди, ратующие за русский язык в
Украине, а именно, за предоставление русскому языку статус государственного, за
то, чтобы русский язык преподавался в школах и университетах, объявляются
агентами Кремля/Путина, на них навешиваются ярлыки пророссийских/прокремлевских
партий/организаций. И с момента объявления независимой Украины, на
государственном уровне шел непрерывный процесс укрепления позиций украинского
языка, то есть, принятие дополнительных законов, таких как закон о тотальной
украинизации, или же введение квот на телевидении и радио. Также в школах,
постепенно, уменьшалось количество часов, выделенных на изучение русского языка
и литературы, русский язык и литература были объявлены как \enquote{иностранный
язык}, и \enquote{зарубежная литература}, соответственно, и есть предположение, что глубинной
целью этого процесса в Украине является вообще полное упразднение изучения
русского языка и культуры, совершенно и всецело, и как следствие, воспитание
поколения, говорящего исключительно по-украински.

\ifcmt
  tab_begin cols=2,no_fig,center
     @caption Голосеевский район, Киев, взято из интернета

     pic https://www.5travel.net/uploads/images/00/00/02/2013/05/16/00060c.jpg
     pic https://kievvlast.com.ua/project/resources/2018/05/LdYAFiyX.jpg
  tab_end
\fi

