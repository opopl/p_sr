% vim: keymap=russian-jcukenwin
%%beginhead 
 
%%file 26_09_2021.fb.tregubov_viktor.1.identichnost_model.cmt
%%parent 26_09_2021.fb.tregubov_viktor.1.identichnost_model
 
%%url 
 
%%author_id 
%%date 
 
%%tags 
%%title 
 
%%endhead 
\subsubsection{Коментарі}

\begin{itemize} % {
\iusr{Дан Бялик}
слушна думка, але більшістю не сприймається, на жаль

\iusr{Maryna Shandropol}
пост про бокс?

\iusr{Paul Potseluiev}
Розумію аргумент і здебільшого згоден
Але, знаючи, наслідком яких дискусій стали ці думки, маю питання
Чи коливається в своїй ідентичності Анатолій Шарій?

\begin{itemize} % {
\iusr{Victor Tregubov}
\textbf{Paul Potseluiev} ні.

\iusr{Viktor Velychko}
\textbf{Paul Potseluiev} шарій - пропагандон кремля, путінизма. Це інше. Він - відкритий ворог

\iusr{Арман Исабеков}
У випадку Шарія взагалі складно казати про ідентичність. У медіашльондр вона навряд чи є.

\iusr{Paul Potseluiev}
\textbf{Victor Tregubov} 

тоді варто мати визначення отого "коливання в ідентичності", межу, за якою
стрілочка ідентичності після коливання перестає рухатись на протилежному від
нас боці.

Я для себе її не визначив досі.


\iusr{Victor Tregubov}
\textbf{Paul Potseluiev} 

так у Шарія і не було коливання. В нього і національної ідентичності немає.
Запитувати, чи він українець, чи росіянин, має рівно стільки ж сенсу, скільки
питати це про неандертальця печери Киїк-Коба чи фікус на підвіконні. За
національною ідентичністю він космополіт.


\iusr{Paul Potseluiev}
\textbf{Арман Исабеков} 

окей, що робить Шарія медіашльондрою, а інших людей, що розповсюджують
абсолютно ті самі тези, що й він — ні? Наявність сплаченого чеку?

Тобто ідейний аналог Шарія у вакуумі міг би бути людиною, що коливається?

P. S. Маю на увазі медійних людей.

\iusr{Арман Исабеков}
\textbf{Paul Potseluiev} 1. Так. 2. Так.

\iusr{Paul Potseluiev}
\textbf{Victor Tregubov} ну це я вже не лише про Шарія, а в цілому про визначення пишу.
З якого моменту коливання зупиняється не там, де нам треба, з якого моменту ймовірний ватник стає ще й ворогом.
Окрім очевидних випадків, коли він бере до рук зброю або загалом починає на ворога працювати.

\iusr{Сергей Анохин}
\textbf{Victor Tregubov} то он кого чоловіки викликають для інтимних послуг - космополітів)))))))))

\iusr{Володимир Борисюк}
\textbf{Paul Potseluiev} Шарій - не коливається. Частина його аудиторії - коливається.

\iusr{Роман Курбатов}
Шарий предпочитает жить в Европе, получая российские деньги за то, что он поливает Украину. Какая тут идентичность?

\iusr{Bohdan Holub}
\textbf{Paul Potseluiev} Ви можете уявити собі Шарія, Медведчука, Шуфрича, які гордо тримають синьо-жовтий прапор і танцюють гопака?Тому не варто змішувати праведне зі святим  @igg{fbicon.smile} 

\iusr{Leonid Tonkovyd}
\textbf{Roman Kurbatov} Русскомирская.
Вполне себе.

\iusr{Роман Курбатов}
\textbf{Bohdan Holub} Шуфрича нет, потому он и враг. А Шария вполне.

\iusr{Роман Курбатов}
\textbf{Leonid Tonkovyd} русскомирцы идейные, Шарий - продажный. Купили бы другие, продался бы другим.

\iusr{Leonid Tonkovyd}
\textbf{Roman Kurbatov} Одно другому не мешает. Я заметил Шария несколько раньше, чем вчера. Он на себя уже давно русское дворянство пытался напяливать, противопоставляясь украинцам, сохраняющим свою идентичность, как чему-то низшему.

\iusr{Роман Курбатов}
\textbf{Leonid Tonkovyd} хоспади, я срался с Шарием еще в мохнатом 2009-ом, на Дурдоме у Шрайка, когда он искал себе площадки для своих журналистских высеров. Как был attention whore, так и остался. Какими бы ни были его взгляды в попытке примазаться к русскому и имперскому, он предаст их за кусок колбасы. Шарий - плохая и низкобюджетная версия Соловьева и Киселева.

\iusr{Leonid Tonkovyd}
\textbf{Roman Kurbatov} И опять одно другому не мешает. Второсортность малороса и тяга в "борули" у него неподдельные. А сбыт налево некоторой части ресурсов русского мира с целью получения шкурной выгоды - это в русском мире явление вполне обычное.

\end{itemize} % }

\iusr{Oksana Kyivska}

то це один не дуже відомий поет, але дуже відомий спортсмен, не будемо називати
імен, за цією теорією «колєблющійся», і дійсно може за український «пряник»
піддатися вихованню? з ночі ледь не плачу від таких романтичних припущень від
багатьох, на перший погляд, просмалених циників)) але най буде. час покаже

\begin{itemize} % {
\iusr{Сергей Гарник}
\textbf{Oksana Kyivska} может. потому что он по сути - ебанькоу и что ему вложить, то он и будет транслировать. правда, я считаю, что кнут тоже не помешает

\iusr{Sergey Koloney}

мені здається тут справа в тому що "цинічний" погляд не залишає ніяких шансів
на перемогу взагалі, а "романтичний" - хоч які, але дає. 

Історія з категорії "ти можеш бути нескінченно правий, але який в тому сенс
коли твої ватніки з коливаючихся перейшли до відкритих сепарів".


\iusr{Oksana Kyivska}
\textbf{Sergey Koloney} 

та звідки оця наївна впевненість, що той поет «колєблється»? гопак і
синьо-жовті тоуселя завжди буди в його арсеналі, після тієї пресухи, де говорив
українською, після бою всі інтерв’ю російською. рукавиці?)) ну це смішно
просто. враховуючи футбу з Алєксандром Нєвскім, канонізованим рпц. чому все за
нього вирішили?))

\iusr{Oksana Kyivska}
і так, відкритий сепар краще, ніж тухла консерва, загорнута у синьо-жовтий фантик

\iusr{Наталія Гурошева}
\textbf{Oksana Kyivska} бо у когось забагато романтизЬму і замало спроможності дивитися правді у вічі

\iusr{Oksana Kyivska}
\textbf{Наталія Гурошева} та у автора посту з цим наче норм було. списую цей дуже розгорнутий емоційний допис на скромноє абаяніє усіка, угу))

\iusr{Наталія Гурошева}
\textbf{Oksana Kyivska} ну оскільки я того "обаянія" бачити не здатна, то і намагатися не буду))

\iusr{Oksana Kyivska}
\textbf{Наталія Гурошева} і я теж)

\end{itemize} % }

\iusr{Victor Molokanov}

Тут має бути коментар:
"Чому недостатньо шевченківською державною??!!!11"

\iusr{Danila Volo}

какая логика, блеать. только жесткий ипринт бытием

вата веает про внешнее управление в телевизоре, на уточняющий вопрос сливного
"а что если мы поставим своих у власти", тут же отвечает "им не дадут,народ тут
же выйдет протестовать". все это укладывается в одном межушном узле как
непротиворечивое, моральное, должное. на вопрос "хотите как в Украине" надо
отвечать "хотим куак на раиссе - чтобы внесудебные казни, гулаг и бульдозеры
пармезан давили"

потому что мораль одна - вероломство и вранье ради росийского царя.

\iusr{Кирилл Данильченко}

Тот факт, что любители РПЦ, божьего Крыма и братушек не едут на курорты ДНР в
Седово и не покупают акции Стирола говорит о том, что с мозгами там все
нормально, в общей массе. С другой стороны, классический ватник (наши миги
сядут в Риге) может спокойно жить себе в Чехии под зонтиком страны НАТО, быть
застрахованным в компании со столетней историей, но голосовать за радикалов,
которые любят Путина и при случае завернут свободы как только заедут в
парламент) Клиника же? Лечится это с трудом и долго. Потому что это вообще
побочный эффект внутренней РФ повестки по удержанию режима. И что чела уебало
розеткой от ТВ с миллиардными бюджетами или духовником РПЦ - ну несчастный
случай.

\begin{itemize} % {
\iusr{Ярослав Ластовецкий}
\textbf{Кирилл Данильченко} 

ну так само уєбало і мудрий нарід, який чомусь хотів дивитись на чєла на
халяву, за 99 грн мегого був готовий задушити. Де ж це блд бачено щоб
промоутери і професійний спорт заробляв?

\iusr{Кирилл Данильченко}
\textbf{Ярослав Ластовецкий} 

Ну РПЦ как организация весьма понятна. Хочет, например, достроить самого
высокого голубя как в Черкассах или как я Житомире недавно видел - требуется
экономка, прачка, повариха, спросить сестру Зачатию условную. Они там за века
надрочились мозги промывать. И неплохо себя чувствуют - садовники у них,
экономки, Усики небось десятину нормальную кашляют. У них система, а у нас
нарид. Который хочет и на хер сесть, и за трансляцию не платить. Опять
возвращаемся к лозунгам времен выборов, короче )

\iusr{Александр Красильников}
\textbf{Кирилл Данильченко} судя по электоральным предпочтениям в той же Германии, немалая часть избирателей, в общем, непротив закрутить себе гайки в угоду стабильности и "чтоб эти богатеи платили больше налогов". Плюс молодежь нехило подтягивается под знамена тех же Зеленых, и не только в Германии, которые тоже по сути радикальные социалисты.

\iusr{Ярослав Ластовецкий}
\textbf{Кирилл Данильченко} то що, доб'єм нарід остаточно і хай купують pay per view по повній вартості. 33 євро було якщо не помиляюся
\end{itemize} % }

\iusr{Oleksii Dubynka}
Що цей право-ліберал собі дозволяє?

\iusr{Oleg Cherevatov}
Сказати трішки легше ніж зробити  @igg{fbicon.face.pensive}  але будемо намагатися  @igg{fbicon.wink}  @igg{fbicon.beaming.face.smiling.eyes} 

\iusr{Елена Рышкова}
на 100\%!

\iusr{Alina Kostiuchenko}
Нє, є ж тільки біле і чорне... у головах деяких людей((

\iusr{Василь Артем'єв}

Боксер ефективно боксує. Добре що на нашому боці. А очікувати розумної
громадянської позиції від нього, це все рівно що чекати чогось розумного від
клоуна

\begin{itemize} % {
\iusr{Alexander Derkach}
та хай боксує. аби не триндів.
\end{itemize} % }

\iusr{Антон Гринь}

Толерантность в борьбе со злом не есть добродетель... - на самом деле пост об
этом а не о "вихованні справжніх патріотів"... Набагато важливішими е питання
активності патріотів у внутрішньому дискурсі а не у роздумах - що там в голові
у вати...

\iusr{Oleksandr Petrov}
Так а какая модель должна быть?

\begin{itemize} % {
\iusr{Андрей Пирогов}
\textbf{Oleksandr Petrov} работать на просвещение и пропаганду среди сомневающихся или проросийских, а не вбивать в головы тем кто и так за Украину что надо быть за Украину
\end{itemize} % }

\iusr{Alex Nikishin}

Люди вибирають одну з сторін. Одні вибрали одну, інші - другу.
Вони і є росіяни. Пів Росії ходить з українськими прізвищами. Але вони росіяни. І тут теж.
Питання не у прізвищах. Колишні українці є набагато більшими ворогами всього українського ніж природні росіяни. Рано чи пізно ця ситуація призведе до відповідних наслідків.

\iusr{Роман Костромицкий}
Якщо тому, хто коливається у визначенні своєї ідентичності не вистачає горя війни, то які ще аргументи можуть бути сильнішими?

\begin{itemize} % {
\iusr{Андрей Пирогов}
\textbf{Роман Костромицкий} каким образом они чувствуют это горе ?

\iusr{Роман Костромицкий}
\textbf{Андрей Пирогов} хіба він не з Криму?
\end{itemize} % }

\iusr{Алексей Панич}

Перший токсичний мотив має глибоке коріння. Схоже було в Біблії: якщо людина
намагається служити Ягве та Ваалу, вона неодмінно завершить тим, що відкине
Ягве заради Ваала. Тому тільки виключна відданість і демонстративно суворе
покарання за відступництво.


\iusr{Андрей Пирогов}

Вполне дачниц тезис что с наридом надо работать и ура патриоты приносят больше
вреда чем пользы, а такая деполяризация и деление населения работают только в
минус. Ума и сердца , всегда

\iusr{Карiна Шиншилова}
Дуже згідна.
І це стосується багатьох сфер.
Зараз актуально стосовно вакцинації, наприклад

\begin{itemize} % {
\iusr{Андрей Пирогов}
\textbf{Карiна Шиншилова} давайте не будем трогать вопрос вакцинации так как там много спорных моментов

\iusr{Alex Buzdugan}
\textbf{Андрей Пирогов} уберите халат с аватарки

\iusr{Андрей Пирогов}
\textbf{Alex Buzdugan} прекратите вести себя как глупый сектант и отрицать очевидные вещи )

\iusr{Александр Артемов}
А ось і яскравий приклад намалювася!
І що з ним робити? От як з таким працювати? Він інших "лікує", начебто...  @igg{fbicon.smile} 

\iusr{Андрей Пирогов}
\textbf{Александр Артемов} а вы кто такой что бы со мной что то делать ?) с чего вы взяли что я дурак, а не вы?)

\iusr{Андрей Пирогов}
\textbf{Александр Артемов} я ещё ничего против вакцин не сказал, а вы начали нести хрень что делает вас не лучше антиваксов)

\iusr{Карiна Шиншилова}
\textbf{Андрей Пирогов} лол, ви сказали не зачіпати тему вакцини, бо там багато спірних моментів, а антивакс хтось інший))))
Візьміть і поясніть, раз мамєнти спорні

\iusr{Андрей Пирогов}
\textbf{Карiна Шиншилова} вакцина не прерывает путь передачи и не защищает от заболевания. То есть маска и соц дистанция вообще нужна после вакцинации и создание зон для вакцинированных бред и не работает. А у меня кто то спросил про моменты? Сразу говном начади бросатья

\iusr{Андрей Пирогов}
\textbf{Карiна Шиншилова} и следующее побочки после вакцин, кто отвечает и заплатит компенсацию ?
\end{itemize} % }

\iusr{Олег Максимович}

А мені для прикладу влом розбиратися в різних сортах г...

Ці українців називають "іудамі" , і одночасно називають "частью рускава народа"
(смішно якщо подумати, "рускій народ" є "народом іуд"? бо як іще ми можемо бути
їх частиною?).

До них були щеодні "Наполеони", гітлерівські німці - ті одну половину Європи
без спросу записали в "арійці" тобто "своїх" - а іншу частину також без спросу
і без жодних раціональних підстав записали в "унтерменші", над якими треба
панувати ("арійцям" звісно) і яких треба винищувати.

Історична практика показує що найкраща дискусія з такими народами є тотальне
бомбардування їхніх міст і сіл, з мільйонами загиблих і скалічених. Бо інакше
ця хвороба у таких народів переходить у хронічну стадію. А це вже не просто
погано а дуже погано.


\iusr{Тарас Пакош}

Ви описуєте крайнощі, у них справді, є свої догми і до інших поглядів вони
глухі, але більшість нормально спілкується і явно, чи таємно вже визначилась у
своєму світогляді.

\iusr{Дори Рыбка}

\obeycr
От дайте я вас обніму за такі слова!
Міняю паспорт, в паспортного дивляться що я народилася в росії, "що, москалька" зверхньо каже мені.
Ні, кажу, я з куркулів, яких вислали колись, поки ваші дупу гріли.
Я не кажу що мене треба поважати за мою криву вимову. Але якщо не знаєш людину, її шлях, як можна засуджувати?
\restorecr

\begin{itemize} % {
\iusr{Евгений Подорван}
\textbf{Дори Рыбка} до чого тут мова?

\iusr{Дори Рыбка}
\textbf{Evgeny Podorvan} до того, що у вас фейсбук російською.

\iusr{Олег Гундяк}
Тицьнув вподобайку, але тут трохи про інше.

\iusr{Евгений Подорван}
\textbf{Дори Рыбка} если Бойко оденет вышиванку что-то изменится?

\iusr{Дори Рыбка}
\textbf{Олег Гундяк} я розумію, що тут про цькування усика. Про цькування тих, хто не так і не тоді вишиванку одягає.
Але ж нам треба за кожного українця (навіть недоукраїнця) дратися в цій інфовійні, а ми цькуємо один одного.
\end{itemize} % }

\iusr{Марина Ященко}
Державна пропаганда та відміна загального виборчого права.

\begin{itemize} % {
\iusr{Alex Buzdugan}
\textbf{Марина Ященко} это Вы сейчас серьёзно?

\iusr{Марина Ященко}
\textbf{Alex Buzdugan} Так, виборче право не має бути загальним.

\iusr{Алексей Воронец}
\textbf{Марина Ященко} монетизация права. Любители гречи ее получат от государства, а не от рвущейся к корыту мрази. Сразу и много.
\end{itemize} % }

\iusr{Oleksandr Chubaruk}
Василю Стусу з Левком Лук'яненком ви б теж порадили збавити обороти, пане Вікторе?  @igg{fbicon.face.nerd} 

\begin{itemize} % {
\iusr{Лариса Маркова}
\textbf{Oleksandr Chubaruk} аякже і комсою в мохєрових бєрєтах нарікли

\iusr{Юрій Човган}
А яке відношення Ви маєте до них?
\end{itemize} % }

\iusr{Aleksandr Vaskiv}

Ну і як тоді боротися з кремлівською пропагандою про "адіннаріт"? Беруть
сумніви щодо того, що маророси та ватніки полюблять країну. Але можливо треба
боротися і за них.

\ifcmt
  ig https://scontent-lga3-1.xx.fbcdn.net/v/t1.6435-9/243116610_2998235280492543_7301718030123572431_n.jpg?_nc_cat=105&_nc_rgb565=1&ccb=1-5&_nc_sid=dbeb18&_nc_ohc=6dGa0XkOFRMAX-H26M8&_nc_ht=scontent-lga3-1.xx&oh=4e9be50eecd82b0d46ec762881b3171e&oe=61750F1A
  @width 0.3
\fi

\iusr{Oleksandr Chubaruk}
А, згадав, ваш пост ідеологічно співпадає з спостереженням Олени Зеленської щодо того, що деякі українці занадто українці  @igg{fbicon.laugh.rolling.floor}{repeat=3} 

\iusr{Оксана Городецкая}

Славяне обьединяйтесь!!-зачем? - как зачем, сук.. и !!!ми близки духовно!!! ви че
внатуре !!!- а ми не бажаємо!!- ви че абарзели??? обьединяйся сказали су.. ки!!!


\iusr{Andriy Afanasenko}
ого накинулись... скоро, Вікторе, Ва білим пальтом наречуть)))

\iusr{Эдуард Орешков}

Мягкая сила. В этом спасенье. Украинство нужно продавливать, но так, чтоб
бывшие оппоненты и сами не заметили как стали уже патриотами. Тонкая игра.

\iusr{Myroslava Balan}

"Згідно цієї моделі, будь-який українець, що коливається у своїй ідентичності,
вже є зрадником та зомбі. І працювати з ним не треба, а треба стріляти в голову
та не давати себе вкусити, бо все одно зрадить."Справа в тому, що вони уже
багато разів так робили, Висока вірогідність, що як би з ними не працювали,
зроблять знову. Тож категорична вимога визначитися зі стороною у воєнний
фактично час та не з'їжджати на "нє всьо так однозначно, там тоже люди" і т. п.
містить , я думаю, частіше всього не якісь там токсичні мотиви, а елементарне
прагнення почуватися у безпеці та не чекати ножа у спину. Але, якщо маєте
вільний час, міцні нерви, а головне, володієте відповідними методиками (бо "та
сторона"обробляє дуже професійно, тож "розкодування"теж дилетанту навряд чи
вдасться) - бажаю успіху.

\iusr{Alexander Derkach}

за скрепанутістю деякі наші "патріоти" дадуть фору навіть ватникам. слейвянскоє
братство, арійці, вайт повар і т. д.

\begin{itemize} % {
\iusr{Vasya Grab}
І маляр з відня.
\end{itemize} % }

\iusr{Наталія Гурошева}

Авторе шановний! Повірте на слово, що то зовсім не комплекс меншовартості
говорить у мені, коли я не вірю у "коливання" і у можливість українізації тих
"потенційних патріотів", яких і війна, і біда розуму не навчила! Я маю єдине
життя, і на ідіотів, що не здатні думати і відрізняти ворога від свого
захисника, його не витрачатиму. Краще виховуватиму дітей - патріотів , краще
фронту допоможу чи ветеранам. А хто має час на йолопів, нехай спробує, не
заперечую.


\iusr{Сергей Лукошин}
Софістика, і дуже хрінова

\iusr{Кирилл Тыцкий}
АААА фсьо ви Арєстовічь!!!!!11111

\begin{itemize} % {
\iusr{Алексей Воронец}
\textbf{Кирило Тицький} кстати, и у Арестовича есть умные мысли)

\iusr{Кирилл Тыцкий}
\textbf{Oleksiy Voronets} в нього є імітація. Дуже вторинна та поверхнева. І раціо там закінчується на другому слові.

\iusr{Олег Гундяк}
\textbf{Олексій Воронець}, є, але ну його нахуй.
\end{itemize} % }

\iusr{Pavel Gol'din}

це Ви про Усика? українець під українським прапором з надписами на рукавицях
"Сімферополь" та "Україна" здобув чемпіонські пояси для України, а
вболівальники не задоволені, бо він їм недостатньо патріот? таке напевно може
бути тільки у нас

\begin{itemize} % {
\iusr{Олег Гундяк}

\ifcmt
  ig https://scontent-lga3-1.xx.fbcdn.net/v/t39.30808-6/242608162_4065139430261190_207769407698034236_n.jpg?_nc_cat=110&ccb=1-5&_nc_sid=dbeb18&_nc_ohc=Hq-MyMEjWVQAX9zNp37&_nc_ht=scontent-lga3-1.xx&oh=07054fedb69a9ba7006e325b058081de&oe=61570CD8
  @width 0.3
\fi

\end{itemize} % }

\iusr{Мирослава Горностаєва}

В "коливання ідентичності" не вірю від слова "зовсім" . Ідентичність або вона
є, або її нема. Її можна накинути силою як за радянських часів-батьки-українці
україномовні, син уже росомовний, бо закінчив вуза, ясна річ російською,
натужно акає та ґекає, але ще співає за чаркою" Маруся раз, два, три" знайшов
собі таку ж Проню і вуаля... Третє покоління уже русскіє, бо разом з мовою
всмоктали і культуру і ідентичність. А "коливання" оці, це коли людина
придивляється, ким їй бути вигідніше. Якби на неї надавили, так, як свого часу
давили на українців, ідентичність стала б на місце, як вивихнутий суглоб. Але ж
це лише їм можна давити.


\iusr{Алексей Воронец}

Сейчас Вам виромовы насуют за "зрадливі думки". И Вы, как ветеран и как
активист здорового человека, для них тоже окажетесь "малоросом". Боже, как же
Сокири не хватает в Раде... Был бы глоток воздуха среди зелени, мертвечуков и
"патриотов".

\begin{itemize} % {
\iusr{Виктор Малиновский}
\textbf{Oleksiy Voronets} вангую, что тот же Виктор вам сейчас насует за "виромовы", как ветеран и как активист здорового человека. Ещё "СУГСа" не хватает для полноты картины. И "армовира". Тоже долбоклюи любят.
\end{itemize} % }

\iusr{Guy Crouchback}
Власне, більшість камєнтів цілком підтверджують думку автора(

\iusr{Игорь Безвозюк}
Хочеш щось змінити, почни з себе - проста істина, але яка банально ігнорується.

\iusr{Федор Меньшаков}

Если бы текст был написан неделю назад можно было бы даже согласиться.

Но если натягивать пост на Усика, то все не так как Виктор написал. Это не
колебания это ролевая модель социально приемливого для россиян украинца. Ещё
один довод в копилку, что есть хорошие украинцы. Вот на Усика посмотрите, на
Шария. А есть Д7 и прочие активисты, которым будет пизда если мы упадём. Без
вариантов, смерть, тюрьма или эмиграция. Без полутонов. Поэтому нахер со своим
Усиком идите.

Я в шоке, что мы это в 2021г обсуждаем.

\begin{itemize} % {
\iusr{Наталія Гурошева}
\textbf{Федор Меньшаков} і на восьмому році війни!
\end{itemize} % }

\iusr{Юрий Возий}

Так то воно так, але ж воно так чому? Бо вчорашні герої дійсно потім одного
вечора опиняються у Москві співаючи караоке з якоюсь скабєєвою.


\iusr{Andrii Molchan}

таке враження, що автор старанно не знає історію українського
народу... толерантність завжди вбивала українців, а його впевненість у здатності
сучасної україноцентричної, територіальної еліти здійснювати активні,
цілеспрямовані, інтелектуальні дії щодо навернення, залучення, втягнення тих,
що коливаються, до українського світу (будь якого), ....є, прямо скажемо,
необгрунтованою. І, крім желанія бить правільним і саврємєнним, за цими словами
нічого немає.


\iusr{Oleg Maryevych}
Прекрасно! Одразу і олімпійці згадуються і Усик і музиканти

\iusr{Володимир Петренко}

)) ... "поплакала і знов ... "

Справа в тому що проти НАС ВСІХ вся сила і міць двох Держав РФії і України - бо
в Україні ... ну самі знаєте хто при владі.

Обнадіює тільки одне - МОВА ВІРА АРМІЯ .

\iusr{Юрий Повх}
Щодо віршика - КГАМ)))

% -------------------------------------
\ii{fbauth.kurbatov_roman.ukraina.web_developer.vakoms}
% -------------------------------------

Самая большая часть ультрапатриотов (если не считать Фарион на зарплате) - это
бывшие русскоязычные, которые только-только переметнулись, многие даже свои
русские ники не сменили. Дебильные неофиты, которых и сюда набежало. Они как
бывшие курильщики или новые веганы - бросили дурную, по их мнению, привычку,
страдают, и хотят, чтобы все остальные страдали вместе с ними. Такие же бегают
у Поцелуева на Альфе и постоянно его упрекают в недостаточном патриотизме,
такие же борются с "богомерзким суржиком" и визжат "чому не українською". Тут
вряд ли что-то можно объяснить, это у людей компенсация так работает.

\begin{itemize} % {
\iusr{Станіслав Андрійович Черкас}
\textbf{Роман Курбатов} спірна й неоднозначна думка, з якою, втім, складно не погодитися. Вона була б іще переконливішою, якби Ви, пане Роман, спробували викласти її, все ж, Мовою. У вас це б добре вийшло. Шануймося...

\ifcmt
  ig https://scontent-lga3-1.xx.fbcdn.net/v/t39.30808-6/242877885_1488318904878885_8041699948529632234_n.jpg?_nc_cat=105&_nc_rgb565=1&ccb=1-5&_nc_sid=dbeb18&_nc_ohc=DEKLJ9xdPCoAX-iFgdu&_nc_ht=scontent-lga3-1.xx&oh=24d25b9d4d77bee543dc25272b584c12&oe=61578598
  @width 0.3
\fi

\iusr{Роман Курбатов}
Какая чудесная иллюстрация. Невероятная просто.

\iusr{Володимир Борисюк}
\textbf{Roman Kurbatov} 

ну, нет. Самая большая часть ультрапатриотов - это украиноязычные из
украиноязычных регионов, которые не знают и не хотят знать, что бывает по
другому. Как та же Фарион, да, если бы она была искренней.


\iusr{Наталія Гурошева}
\textbf{Роман Курбатов} 

яка дурня. Вибачайте, вихопилось... Дебільні неофіти, кажете?.. Може, якоюсь
мірою і це є. Але чим кращі "непохитні мудреці", що не в змозі вивчити історію
країни і народу, серед якого живуть (навіть часто себе до цього народу
відносячи) - і застосувати ці знання до сучасної ситуації? Чиї "принципи"
полягають у намаганні "злетіти над битвою" і звідти щось "вумне" брякнути? В
"принциповому" ігноруванні мови і послуговуванні мовою рос.пропагандонів? Чим
вони, такі, цінніші для Украіни?

\iusr{Наталія Гурошева}
\textbf{Роман Курбатов} 

"Юпітере, ти гніваєшся..." Чому у Вас і подібних до Вас (бо Ви зовсім не
самотні у Ваших поглядах) українські патріоти - завжди "ура-", або "ультра-",
або "псевдо-"? Як Ви впевнено їх класифікуєте! Підкреслюючи - це головне - що
всі вони не є СПРАВЖНІМИ патріотами. А де визначення справжніх?.. І "судді
хто"? І чи не є всі ці епітети проявом Вашої власної сублімаціі? Багато хто з
моїх реальних і ФБ- знайомих і друзів перейшли на украінську (у дописах, у
побуті - хто на що спромігся). Серед них кілька ветеранів війни з рашою. Часом
вони "грішать" суржиком або не зовсім грамотним письмом... Але всі вони дорогі
мені, бо зробили свідомий і для багатьох не дуже легкий крок у бік українства .
Не "ультра", не "псевдо", не "квазі" - а просто українства.

\iusr{Сергей Давыдко}
\textbf{Наталія Гурошева} такими "ярличками" вони намагаються принизити, знецінити. такі як Курбатов - найгірші вороги України.

\iusr{Роман Курбатов}
\textbf{Володимир Борисюк} 

посмотри выше, посмотри ниже. Посмотри по треду. Никто так не топит за
девственность как вчерашняя проститутка. А за патриотизм и мову - как вчерашний
комсомолец и русскомирец. Это как Гончаренко в миниатюре, каждый раз.

\iusr{Роман Курбатов}

Потому что до сих пор болит, потому что нужно доказывать, что ты уже
нормальный, что ты уже патриотичный, а не как раньше. Себе же доказывать, путем
наезда и открещивания.


\iusr{Володимир Борисюк}
\textbf{Roman Kurbatov} 

сорри, но нет, ошибка выборки. Специфика аудитории такая, что может создаться
такое чувство. В группах условной околосвободы будет несколько другая ситуация.

\end{itemize} % }

\iusr{Yar Stseshyk}
А надо сознанье и уд изменить!

\iusr{Станіслав Андрійович Черкас}

Згадався мені у цьому історичному зв'язку кремезний галичанин із твердим
характером @Ярослав Ґалан. Нехай царствує й мене Там дожидається. Нам буде про
що з ним посперечатися...


\iusr{Оксана Євдокимова}
Хто буде Римом, а хто Карфагеном?

\iusr{Gala Ka}

І що, все прощати тим "колеблющимся"? Щоб здавалось, що все, як треба?

Мені здається, треба критикувати і хвалити певні дії, до певної межі не зачіпаючи людину, як таку..

\iusr{Vitaliy Kovalchuk}
Це як рибу літати вчити...

\iusr{Слава Хоббит}
Вipш Бродського?

\iusr{Dmytro Leontiev}
Ну,..

Росія без України, це колонія без метрополії. Це не я сказав
@igg{fbicon.smile}  але це факт. Цей факт дуже образливо вони сприймають.

\iusr{Vasya Grab}

\obeycr
Як добре, що я менше половини українець.
Хоч не стидно, що малорос, які витягли собі на голову гнусаве гiBHo.
Маю знайомих галичан.
Якісь безідейні люде, тільки гроші та маєтки.
Навіть такі сякі комуняки і то мали якусь ідею.
Але так і раніше було - за кавалок ковбаси прокатували на виборах Каменяра.
Огидно на душі.
Хоч би якась вже революція, аби розвіяла це болото Вакул-голємів.
\restorecr

\iusr{Natalya Zayats}

А оце мій улюблений вірш цієї тематики. Стільки болю і образи та ще й від єврея
в еміграції

\href{https://www.culture.ru/poems/30468/na-nezavisimost-ukrainy}{%
На независимость Украины, Иосиф Бродский, www.culture.ru%
}

\iusr{Роман Костенко}

"що он у нього там серед віршів українською мовою є й російськомовний"

Ви, Вікторе, успішно дрейфуєте у бік "білої кістки українства": уже пишете
пости Мовою; і все більше й більше в постах націоналізму (Василь Лісовий
антологію був видав - це найкращий посібник на тему, if any). Ось згадаєте
мене: ще за рік за два і для Вас російська мова із вуст громадянина України
буде ознакою невисокого патріотизму такого україн... громадянина України. Ми як
суспільство або агресивно і безповоротно порвемо із Зайвохромосомією - як
поляки чи прибалти - або знову на якийсь час розчинемося у некультурному,
примітивному, пропитому але дикому і агресивному соціумі Зайвохромосомії - як
нинішні білоруси.

Причому, такі люди, як Ви, будуть у авангарді повного розриву з культурою
"етіхь", із-за парєбріка.


\iusr{Владимир Богданов}

Росіяни за прийменник інклюзивно вдавити готові, навіть ліберали. Українці
готові пробачати будь що, за умови примарної надії на перевиховання вати. А
преревиховання, як показує практика, здебільшого, збійснюється чарівним
перделєм і добрим солом краще, ніж добрим словом.

\iusr{Olga Mare}
Хороший пример... \href{https://www.facebook.com/100000892405451/posts/4618914334814934/}{%
Вчера, на боксерском ринге, для всего мира победила Украина, именно Украина и только для самой Украины это не так,%
Игорь Ларин, facebook, 26.09.2021%
}

\iusr{Рустам Гівазович}
При чому тут усік???

\end{itemize} % }
