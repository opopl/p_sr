% vim: keymap=russian-jcukenwin
%%beginhead 
 
%%file slova.vstrecha
%%parent slova
 
%%url 
 
%%author 
%%author_id 
%%author_url 
 
%%tags 
%%title 
 
%%endhead 
\chapter{Встреча}

%%%cit
%%%cit_head
%%%cit_pic
%%%cit_text
Їхня остання \emph{зустріч} відбулася в Роттердамі 23 травня 1938 року
приблизно о 12.15 в центрі міста. Агент вручив Коновальцеві бомбу, замасковану
під коробку цукерків. Полковника розірвало на шматки (4, с. 248).  1941 року
Другий Великий Збір Організації Українських Націоналістів постановив щороку 23
травня святкувати День Героїв. Такий тезовий виклад життя людини від життя до
смерти у безсмертя
%%%cit_comment
%%%cit_title
\citTitle{Евген Коновалець: від життя до смерти у безсмертя (тези за телепрограмою \enquote{Ген українців})}, 
Ірина Фаріон, blogs.pravda.com.ua, 14.06.2021
%%%endcit

