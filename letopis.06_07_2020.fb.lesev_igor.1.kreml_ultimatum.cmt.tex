% vim: keymap=russian-jcukenwin
%%beginhead 
 
%%file 06_07_2020.fb.lesev_igor.1.kreml_ultimatum.cmt
%%parent 06_07_2020.fb.lesev_igor.1.kreml_ultimatum
 
%%url 
 
%%author_id 
%%date 
 
%%tags 
%%title 
 
%%endhead 
\subsubsection{Коментарі}

\begin{itemize} % {
\iusr{Юрий Лукшиц}
А что будет, если ультиматум не сработает? Аннексия Донбасса?

\begin{itemize} % {
\iusr{Игорь Лесев}
В смысле "аннексия"? Украина и так половину Донбасса не контролирует. Будет то что есть, без возвращения того, что было.

\iusr{Юрий Лукшиц}
\textbf{Игорь Лесев} аннексия подразумевает вхождение в состав РФ.

\iusr{Игорь Лесев}
\textbf{Юрий Лукшиц} есть куча способов владеть без документов. Посмотрите на Западную Сахару - на всех картах мира своим цветом выделена, а по факту - продолжение Марокко. Или более близкие примеры - ЮО и Абхазия. Формально в РФ не входят, хотя ее прямое продолжение на Кавказе.

\iusr{Feran Tarek}
Даже не важно войдет Донбасс в РФ или не войдет и на каких условиях...
Важно, что для Украины это однозначно утраченная территория...

\iusr{Татьяна Сидоренко}
А для ЛДНР важно прекращение обстрелов.На сегодня это самое важное.

\iusr{Сергей Киселев}
Аннексия уже произошла.
Вопрос в другом, какая территория будет у русского Донбасса. С Мариуполем и Сватово или без.
Я склоняюсь что в среднесрочной перспективе линия разграничения не изменится. Хотя...

\iusr{Олег Хавич}
Потому этого и не будет. Киев и Кремль заинтересованы в продолжении стрельбы, каждый по своей причине.

\iusr{Руслан Болдырев}
Опять пришел Хавич и козырями всех побил.
\end{itemize} % }

\iusr{Елена Скидан}

Судя по тому , что Сивохо раздает интервью о том , как он занимается
"возвращением Донбасса" и в Мариуполе, и в Краматорске , и в Славенске, и в
Дружковке, то вторую половину тоже не совсем контролируют ))))

\begin{itemize} % {
\iusr{Tatyana Meylakhs}
Вы имеете ввиду, что не могут управлять тем, что говорит Сивого и что он там раздает интервью самостоятельно?
\end{itemize} % }

\iusr{Матвей Кублицкий}

напомню то, что писал несколько лет назад. Экономика ЛДНР без Украины вырастет
благодаря сохраненным связям с РФ и отсутствием необходимости обслуживать
киевский госдолг. Куда интереснее и непредсказуемее судьба территорий Донбасса,
оставшихся под контролем Киева

\begin{itemize} % {
\iusr{Игорь Лесев}

нихера она особо не вырастет. Главная трагедия Донбасса - разорванный единый
промышленный комплекс. Донбасс без Лисичанска, Краматорска и Мариуполя - это
уже не Донбасс. Другое дело, что ЛДНР таки интегрируется хозяйственно в РФ,
станет таким себе убыточно-депрессивным осколком Ростовской области с
последующей миграцией населения и переориентацией хоздеятельности на что-то
более простое, нежели тяжелая металлургия и машиностроение, но более насущное.
Но вот Украины в этом процессе не видится

\iusr{Матвей Кублицкий}
\textbf{Игорь Лесев} про переориентацию экономики соглашусь. Но в современных условиях это будет лучше для них, чем в составе Украины.

\iusr{Игорь Лесев}
\textbf{Матвей Кублицкий} дело в том, что никто людей на Донбассе не спрашивает - ни Там ни Тут, что для них лучше. Что нарисуют в высоких кабинетах, то и получат.

\iusr{Матвей Кублицкий}
\textbf{Игорь Лесев} везде так. Страной правит элита, а не народ. Проблема в том, что иногда элита не понимает, как это делать правильно
\end{itemize} % }

\iusr{Олег Резник}
Так тому и быть, к этому все шло.
Тем более, совпадает с состоянием Кремля, когда он начал огрызаться серьезнее.
Последнее голосование 1 июля еще больше укрепит их в правильности таких действий.

\iusr{Дмитрий Новаковский}
а Украина собиралась вообще присоединять тот ЛДНР? по-моему, нет

\begin{itemize} % {
\iusr{Тимур Филоненко}
\textbf{Дмитрий Новаковский} на своих условиях
\end{itemize} % }

\iusr{Оксана Высоцкая}

Тут некоторые дружественные политологи предрекают в октябре-ноябре стирание с
лица Земли непокорных республик. Что Вы об этом думаете?

\begin{itemize} % {
\iusr{Юрий Кировский}
\textbf{Оксана Высоцкая} а кто стирать то будет? А может подштаники в очередной раз постирают и на этом узбагоятся?

\iusr{Оксана Высоцкая}
Возможно, но они очень настойчивы) и ладно бы один, а то прям хором , в унисон.

\iusr{Сергей Киселев}
Поскорее бы

\iusr{Oksana Kudrya}
Это сезонное. Обещали уже. Пройдет.
\end{itemize} % }

\iusr{Сергей Киселев}
Хороший анализ

\iusr{Олег Хавич}

Игорь, ну какой ультиматум? Выполните наши условия, не то... мы и дальше с
причмокиванием будем сосать у Запада? Россия - оккупированная территория,
которая претендует на роль доминиона. Но не с её экономикой.

\begin{itemize} % {
\iusr{Евгений Отовчиц}
\textbf{Олег Хавич} с немцами снюхаются и будут доминионировать)

\iusr{Олег Хавич}
РФ - 2\% в мировом ВВП. Только мастурбировать.

\iusr{Евгений Отовчиц}
\textbf{Олег Хавич} технологии в обмен на энергоносители и армию - прекрасная синергия для четвертого рейха.

\iusr{Евгений Отовчиц}
\textbf{Олег Хавич} немцам тоже неуютно между США и Китаем.
\end{itemize} % }

\iusr{Тимур Филоненко}

А почему Украина должна соглашаться на условия России? Вы постоянно пишите о
том, что Киев должен признать для себя невыгодные условия

\begin{itemize} % {
\iusr{Игорь Лесев}

Где же я пишу, что "Украина что-то там должна признать"? Я пишу о том, что
Украина ничего ничего не делает, никаких планов не имеет и ни на что уже не
влияет. А так, наверное, и в России есть люди, которые считают "Аляску своей".

\iusr{Игорь Писаренко}
\textbf{Тимур Филоненко} недоговороспособность — важный признак делинквентного поведения

\iusr{Тимур Филоненко}
\textbf{Игорь Лесев} 

с Аляской все было по-любовно и с договорами. Меня тоже смущает, что нет плана
когда мы продвигаем свои интересы. Напрягает почему нет инициативы по
динансации Минска и на всякий случай выстроения военного плана и доктрины.


\iusr{Тимур Филоненко}
\textbf{Игорь Писаренко} договариваться с оккупантом? Я понимаю Россия для вас ближе, вы этнический русский. Но большинство народа не хочет мира на условиях Кремля.

\iusr{Игорь Писаренко}
\textbf{Тимур Филоненко} как Вы этничность определяете? По Геббельсу или Розенбергу?

\iusr{Тимур Филоненко}
\textbf{Игорь Писаренко} Я вас читаю уже четыре года. Сложилось такое впечатление. Несмотря на кардинальные расхождения в политических взглядах, нравится читать когда вы пишите про медицину. Жалко не так часто(

\iusr{Игорь Лесев}
\textbf{Тимур Филоненко} Минские соглашения вроде бы тоже полюбовно подписаны были. Аж два раза.

\iusr{Тимур Филоненко}
\textbf{Игорь Лесев} так если не работают чего тянуть кота за яйца?

\iusr{Игорь Лесев}
\textbf{Тимур Филоненко} абсолютно с вами согласен

\iusr{Тимур Филоненко}
\textbf{Игорь Лесев} ну хоть в этом. Я просто против лицемерия. Но у нас просто диаметрально противоположные представления о государсвености.

\iusr{Игорь Лесев}
\textbf{Тимур Филоненко} и какие же у вас представления о государственности?

\iusr{Игорь Писаренко}
\textbf{Тимур Филоненко} значит по Розенбергу.

\iusr{Тимур Филоненко}
\textbf{Игорь Лесев} политическая нация с правоцентрским толком и парламентской республикой. Простите, нет должного образования, чтобы правильно выразить свои мысли.

\iusr{Тимур Филоненко}
\textbf{Игорь Писаренко} а кто такой? Простите за безграмотность.

\iusr{Игорь Лесев}
\textbf{Тимур Филоненко} и чем же угрожают политической нации с правоцентристским толком и парламентской республикой Минские соглашения?

\iusr{Тимур Филоненко}
\textbf{Игорь Лесев} фактическая федерация (я только за, если не ебанутый сосед). Плюс мы уже почувствовали что такое автономия в составе государства.

\iusr{Игорь Писаренко}
\textbf{Тимур Филоненко} Рейхсминистр восточных оккупированных территорий

\iusr{Тимур Филоненко}
\textbf{Игорь Писаренко} оооо, не эт. Я не по такому хардкору выступаю.

\iusr{Игорь Лесев}
\textbf{Тимур Филоненко} интересные умозаключения. "Я не против федерализации, но у нас есть ебанутый сосед". И назло соседу не будем возвращать свой же кусок земли.

\iusr{Тимур Филоненко}
\textbf{Игорь Лесев} на его условиях - нет.

\iusr{Игорь Сидоренко}
\textbf{Тимур Филоненко} , сказали "А", говорите "Б". На "его условиях", т.е.,
на условиях подписанного Порошенко Минска- не хотим, на своих- не можем.
Значит, дарим?

\iusr{Тимур Филоненко}
\textbf{Игорь Сидоренко} значит морозим. Но если бы порох не прлебывал последние 5 лет в плане ВПК и армии, то отвоевывать.

\iusr{Игорь Сидоренко}
\textbf{Тимур Филоненко} , морозим, как Приднестровье?

\iusr{Тимур Филоненко}
\textbf{Игорь Сидоренко} можно как и северный Кипр.

\iusr{Игорь Сидоренко}
Тимур, значит, "дарим". По поводу Северного Кипра ведь ни у кого уже нет иллюзий?

\iusr{Тимур Филоненко}
\textbf{Игорь Сидоренко} 

вы хотите сложные вопросы, решать простыми методами. Думайте на несколько шагов
вперёд и моделируйте. Да в краткосрочной перспективе согласие на Минск приведут
к короткой стабилизации региона. Но а дальше что? Будет как Албания или как
Косово? А другие регионы тоже захотят автономии. При таких слабых институциях
это поползет все очень быстро.


\iusr{Игорь Сидоренко}
\textbf{Тимур Филоненко}, 

а Вы хотите сделать вид, что решаете, а сами- оставить всё как есть. На данный
момент вилка разрешения Донбасского кризиса проста: выполняем Минск и получаем
обратно ОРДЛО под украинскую юрисдикцию- да, пусть с особым статусом и
полномочиями или, как Вы выражаетесь "морозим", и тем самым даём отмашку на его
дальнейшую интеграцию в экономико- культурное пространство РФ. Просто сказать
"замри" не получится. Это не игра "море волнуется раз". Если честно, уже сейчас
на первый вариант развития событий очень мало шансов. Пройдёт 2-3-4 года- и их
не останется вообще... @igg{fbicon.face.confused} 

\iusr{Тимур Филоненко}
\textbf{Игорь Сидоренко} я тоже это прекрасно понимаю. Но это реальная палка о двух концах. А вид что что-то делаем не стоит изображать. Нужно честно признаться, что мы со своей стороны морозим.

\iusr{Тимур Филоненко}
\textbf{Serge Vigor} отсыпь.

\iusr{Евгений Отовчиц}

Господа, вы ведёте дискуссии с малограмотным (что признано им самим)
дитём-журналистом Фиаловско-гомосексуального Нового Времени.

О чём вообще можно общаться с человеком, у которого в голове только говно и
пропаганда, а фамилия Розенберга вызывает оторопь? Да, не Маск, не Маск...

Подумаешь, не знает основ ни истории, ни политологии, ни политэкономии - но
зато знает, как надо страну строить. Знает, что большинство украинцев против
России. Новое время ж об этом пишет регулярно. А, и о медицине немного знает -
почерпнул из постов в Фейсбуке.

Господа, я впечатлён. Мне всегда дико жаль потраченного времени на дискуссии с
дискусами. Вы очень великодушны, Игорь Лесев, Игорь Писаренко.


\iusr{Тимур Филоненко}
\textbf{Евгений Отовчиц} о, простите светоч! Как я посмел подумать только высказать свою точку зрения и мнение в кругу людей которых недостойный?

\iusr{Евгений Отовчиц}
\textbf{Тимур Филоненко} "в присутствии двух людей с университетским образованием позволяете себе давать советы космической важности и космической же глупости!.."©

\iusr{Тимур Филоненко}
\textbf{Евгений Отовчиц} "свобода, блядь! Свобода, блядь! Свобода!"©

\iusr{Евгений Отовчиц}
\textbf{Тимур Филоненко} "ты такая же Свобода, как та, шо в Раду не прошла"©

\iusr{Тимур Филоненко}
\textbf{Евгений Отовчиц} короче, Евгений. Вы мне не вправе указывать, что говорить, а что нет.мкак и я вам. Так что давайте им на этом разайдемся?

\iusr{Евгений Отовчиц}
\textbf{Тимур Филоненко} а я к вам даже не обращался, обращу ваше внимание. Вообще с вами не заговаривал, а уж тем более не указывал, чем вам заниматься.
Я восхитился выдержкой и великодушием уважаемых людей.
А вы свободны делать что угодно. Просто у вас профдеформация. Вы можете пиздеть и врать безнаказанно, а в свой адрес находите угрозы даже там, где их не было.

\iusr{Тимур Филоненко}
\textbf{Евгений Отовчиц} я не говорил об угрозах, где вы это прочитали. А уважаемым людям вы говорили в моем присутствии. Я мимо пройти не могу, тем более вы говорите обо мне такие вещи резная меня лично. Я могу предположить, что пиздеть можете и вы и говно в голове именно у вас. Но это догатки. Что знаю точг, так это то что вы невоспитанный человек. Интернет общение тоже общение и тут действуют такие же нормы как и в реальной беседе.

\iusr{Евгений Отовчиц}
\textbf{Тимур Филоненко} 

то есть вы пройти мимо не можете, но настоятельно советуете проходить мимо мне,
попутно пытаясь, будучи абсолютным невежеством, спорить с образованными людьми,
и при этом упрекать меня в невоспитанности?))

Так я вам скажу как пролетарий с Борщаговки - я в рот ебал твои оценки моей
воспитанности, твоего лысого гомосексуального шефа, тебя и твоё мнение.
Политкорректность осталась где-то в центре Детройта.

Образование. Прежде, чем качать права, было бы неплохо получить образование в
тех вопросах, в которых хочется вести дискуссию.

Хотя, безусловно, всё зависит от цели. Если цель - стать настоящим гражданином
и патриотом своей родной страны - необходимо образование и труд. А если
попиздеть и стать вторым "гончаруком" - то и так сойдёт. Главное, побольше
вонять о своих правах, свободе и демократических ценностях.

За сим откланиваюсь. И ещё раз тьфу в ваше задроченное издание, полное
компрадоров и мудаков. Слыхал, у радио Новое Время финансовая беда. Как и у
Лигамедиа. Надеюсь, вы все закроетесь нахер и пойдёте асфальт укладывать на
Большой Стройке Зеленского.

А то Прямой канал всех не потянет, треснет.

\iusr{Наталья Краснова}
\textbf{Евгений Отовчиц} 

А говорят, что женщины склочны:)) Перестаньте, вы ведёте себя как базарные
тётки. Попробуйте, пожалуйста, выражать свое мнение и при этом уважать личность
другого человека, особенно грязно смотрится освещение личной жизни и
подробностей личной жизни другого человека.

Поднятая здесь тема может быть интересной дискуссией с обменом мнений и
предположений, но важно понимать, что это лишь дискуссия, решения ни один из
нас принимать не будет!

\iusr{Евгений Отовчиц}
\textbf{Наталья Краснова} чушь какая.

\iusr{Тимур Филоненко}
\textbf{Евгений Отовчиц} я тебе как пролетарий с Соломы (Чоколовка) скажу: иди на хуй. Увижу, дам по ебалу.

\iusr{Евгений Отовчиц}
\textbf{Тимур Филоненко} договорились)) Обещаю первым не бить.
Но в ответ - пеняй на себя, школота  @igg{fbicon.wink} 

\iusr{Евгений Отовчиц}
\textbf{Тимур Филоненко} маме привет.

\ifcmt
  ig https://scontent-frx5-1.xx.fbcdn.net/v/t1.6435-9/107181416_1176502756034581_2200827287335657228_n.jpg?_nc_cat=100&ccb=1-5&_nc_sid=dbeb18&_nc_ohc=MtywH1dH_WQAX__v-UN&_nc_ht=scontent-frx5-1.xx&oh=e859399cae6320e0d685bac540aaf8f2&oe=61A6148F
  @width 0.4
\fi

\end{itemize} % }

\iusr{Alexei Prolov}

Всегда читаю и наслаждаюсь слогом. Как же вкусно вы пишите о политике.

\iusr{Maksim Kammerer}

к этому все всегда и шло - как, впрочем, и украина искусственная не должна существовать:) а зачем?

\iusr{Анатолій Тимчук}

Будь ласка максимальний рeпост. В мeнe проблeма житлом. Якщо хочeтe
пeрeконатись давайтe созвонимся чи зустрінeмось. Благаю нe пройдіть мимо. Я
знаю шо в кожного у вас є проблeми. Якщо ви хочeтe пeрeконатись я зрадістю можу
доказати що я нe афeрист і дійсно потрeбую допомоги!
@igg{fbicon.face.downcast.sweat}  @igg{fbicon.face.anxious.sweat}
@igg{fbicon.hands.pray} 

\url{https://www.facebook.com/466703640811004/posts/656639245150775/}

\iusr{Александр Мильнер}

Анал итиков по по поводу захвата части территории со предельного государства
,как бродячих собак в этом ,,осколке,,! Причём ,,мнение,, зависит от цены
заказщика...

\end{itemize} % }
