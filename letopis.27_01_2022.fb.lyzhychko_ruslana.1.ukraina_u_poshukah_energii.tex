% vim: keymap=russian-jcukenwin
%%beginhead 
 
%%file 27_01_2022.fb.lyzhychko_ruslana.1.ukraina_u_poshukah_energii
%%parent 27_01_2022
 
%%url https://www.facebook.com/ruslana.lyzhychko.5/posts/4949660788429425
 
%%author_id lyzhychko_ruslana
%%date 
 
%%tags energetika,energia,ukraina
%%title Україна: у пошуках енергії
 
%%endhead 
 
\subsection{Україна: у пошуках енергії}
\label{sec:27_01_2022.fb.lyzhychko_ruslana.1.ukraina_u_poshukah_energii}
 
\Purl{https://www.facebook.com/ruslana.lyzhychko.5/posts/4949660788429425}
\ifcmt
 author_begin
   author_id lyzhychko_ruslana
 author_end
\fi

Надзвичайна атмосфера зарядила мене під час заходу «Україна: у пошуках
енегіїї», який проводив рух Україна 30+ у Києві. Дискусія щодо енергетичної
кризи 2021/2022, виклики, гібридну енергетичну війну, Україну майбутнього,
чисту енергію та пов'язані з цим глобальні можливості. 

\ii{27_01_2022.fb.lyzhychko_ruslana.1.ukraina_u_poshukah_energii.pic.1}

Разом з Viktor  Kurtev (компанія Метрополія) ми продовжуємо професійно,
системно і на постійній основі здійснювати інформаційну (і не тільки) кампанію
щодо чистої енергії.

\ii{27_01_2022.fb.lyzhychko_ruslana.1.ukraina_u_poshukah_energii.pic.2}

♻️ Це практичне лобіювання еко-філософії життя. 

@igg{fbicon.flag.ukraina} Це практичне лобіювання чистої енергії України. 

Я дивлюся сьогодні програми світового рівня, в яких озвучуються наміри
провідних країн реалізовувати амбітні проекти по енергії майбутнього,
створення національних енергетичних проектів з потенціалом, якому немає рівних
у світі, і я задаюсь питанням:  «Невже моя країна, при такій кількості
науковців, не в змозі зробити це ж саме?!...». 

Україна може стати «хабом» чистої енергії.

Україна може виробляти чисту енергію, експортувати її і бути однією з
найбагатших країн світу. У нас є для цього достатній потенціал. Перш за все,  -
людський. 

А ще ми прекрасні тим, що ми самодостатнє суспільство!

З соціальної точку зору ми дуже прогресивні. Ми, як суспільство, «визріли». (І
в дечому «перезріли» наших так званих лідерів, яким вже немає, що нам сказати).

Ми «визріли» і ми не безсилі. Ми навпаки занадто сильні. Нам просто треба
повірити в себе!

Відео, яке ми презентували поки що в закритому форматі (це навіть була не
презентація, а попередній показ ще не завершеного відео), про чисту енергію у
всіх значеннях цього слова. 

Ми зняли його в Бессарабії за підтримки Elementum Energy!

Насправді, кожен куточок України для мене рідний. 

І я відчула, що Бессарабія для мене  теж рідна. Тому кожен куточок України
особливий. @igg{fbicon.heart.red}

І я рада, що я можу показати ці кадри. Вже скоро у широкому доступі.

@igg{fbicon.camera} : Jenya Malchyk
