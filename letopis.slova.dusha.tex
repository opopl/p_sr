% vim: keymap=russian-jcukenwin
%%beginhead 
 
%%file slova.dusha
%%parent slova
 
%%url 
 
%%author 
%%author_id 
%%author_url 
 
%%tags 
%%title 
 
%%endhead 
\chapter{Душа}
\label{sec:slova.dusha}

%%%cit
%%%cit_head
%%%cit_pic
%%%cit_text
Фаина ищет и находит простые слова, образы, сюжеты. У нее, в отличие от
современных потребителей, болит \emph{душа}. Не слышать ее может лишь тот, у кого \emph{души}
нет. Для кого патриотизм это не жертвенность во имя страны, а способ
обогащения, для кого война только прибыльный бизнес, для кого своя рубашка
ближе к телу, а хата с краю.  Послушайте ее обращение в ООН. Никакой
театральности, надрыва, пафоса. Она, скорее, робеет, чем кого-то обвиняет,
требует. Ей как-то неудобно, ведь она понимает, что дети не помнят мира, а
большие дяди в политике не понимают. Точнее, все эти дяди понимают, но просто
дяди негодяи, и заняты другим. А ей, Фаине, за этих дядей неловко и стыдно.
Хотя стреляют не по им, а по ней. Вот это и есть крест настоящего человека,
испытывать стыд за чужую подлость. Маленький человек, но большая \emph{Душа},
%%%cit_comment
%%%cit_title
\textbf{Гибель людей на Донбассе говорит о том, что наше общество больное},
Денис Жарких, strana.ua, 04.06.2021
%%%endcit

%%%cit
%%%cit_head
%%%cit_pic
%%%cit_text
Любовь \emph{душевная} – сложнее. Она может избегать выражений, присущих полу, но не
чуждается телесности. Так ребенок, любя мамочку, обхватывает ее за шею, не
отпускает, хочет вжаться в материнское тело до неразличимости. Но кто из нас
скажет, что любовь ребенка к маме – ненастоящая? Любящий любимого действительно
хочет даже съесть, и поэтому мать кормит дитя собою, равно как и Господь кормит
нас Своим Телом и Кровью. И старики могут любить подлинно и нежно, уже не имея
особых сил для телесных чувственных проявлений.  Грусть сопутствует \emph{душевной}
любви. Грусть со всем синонимическим рядом: с тоской, печалью, меланхолией,
томлением, жаждой неведомого, желанием распахнуть окно и смотреть на звезды.
Юношеское томление ищет выхода, старческое отличается созерцательностью. Но
часто это – лишь балансирование на жердочке. \emph{Душевное} в человеке непостоянно и
нетвердо. \emph{Душевность} либо соскальзывает вниз, в тот самый разгул плоти,
причастившейся злому духу, либо же стремится насытиться вверху, в духе,
объединяющем и тело, и \emph{душу}
%%%cit_comment
%%%cit_title
\citTitle{Возвращение в Рай}, Андрей Ткачев
%%%endcit

%%%cit
%%%cit_head
%%%cit_pic
%%%cit_text
Я поцілував її. Вона заплющила очі, ніби прислухалася до свого відчуття. І не
стало снігу, зірок, проблем часу й простору. Замкнувся світ радості.  Кладу
перо. Не хочу більше писати. Житиму в світі безконечного щастя. Може, це і є
оте народження, яке я пережив недавно вві сні? Мене переслідували потвори, а я
тепер вибухнув коханням і лечу у таємничі сфери Повноти! Якби-то!..  Що зі
мною? Я знову ніби падаю. Біль у серці. Сум у \emph{душі}. Коли це сталося? Де
поділася повнота?  Ми були разом. Найінтимніші доторки, найніжніші обійми, її
слабіючий голос між хвилями нестями. І потім... майже нечутний легіт сорому. Не
збагну...  Невже вся грандіозна фантасмагорія кохання — лише прелюд до зачаття,
яке потрібно природі? А ніжність, а містерія єднання — то лише рефлекси? О
небо!  Хай це буде лише химерою мого втомленого розуму
%%%cit_comment
%%%cit_title
\citTitle{Зоряний Корсар}, Олесь Бердник
%%%endcit

%%%cit
%%%cit_head
%%%cit_pic
%%%cit_text
Ми повинні зрозуміти, що йде не просто гаряча війна, а в першу чергу війна за
серця і \emph{душі} людей. Безумовно, ми хочемо перемоги в цій війні, хочемо врятувати
нашу Вітчизну, наших людей.  З іншого боку – проблема контрагітації. Що ми
можемо, що потрібно і як протидіяти інформаційній агресії. Написані тисячі
статей, сотні монографій, захищені десятки дисертацій з цього питання. Лише
бери і втілюй у життя проєкти, які пропонують науковці.  Але ми знаємо, хто
прийшов до влади. Настільки ці люди освічені. І годі сподіватися на те, що вони
задіють цей матеріал. З іншого боку ми розуміємо чиї інтереси захищає і
обслуговує наша влада. Немає і ніколи не буде України в їхніх серцях.  Першою у
Європі, хто застосував інформаційну і політико-психологічну війну, була
Німеччина під час Першої світової війни. І хоча вона програла, але уроки не
минули даремно. Їх перейняла радянська Росія. Вона втілює їх у життя і
сьогодні. Нас намагаються переконати, що ми провалилися в епоху
радянсько-центризму. А якщо так, то є держава, яка всіх об’єднує. Всі повинні
об’єднуватися навколо Москви. Хто цього не хоче – той ворог. Таким чином
Україну розглядають як ворога, якого треба знищити. Те ж саме пропонують
здійснити стосовного всього іншого світу
%%%cit_comment
%%%cit_title
\citTitle{Як виграти інформаційну війну? – Слово Просвіти}, ,slovoprosvity.org, 01.11.2021
%%%endcit

%%%cit
%%%cit_head
%%%cit_pic
\ifcmt
  tab_begin cols=4
     pic https://strana.news/img/forall/u/11/33/doktorKomarovskij.jpg
     pic https://avatars.mds.yandex.net/i?id=425c54af228755d2ae1c7e0203eb2fef-4589829-images-thumbs&n=13
		 pic https://avatars.mds.yandex.net/i?id=56ddb08d39c12fc46ba5cd01064f2493-5233191-images-thumbs&n=13
		 pic https://avatars.mds.yandex.net/i?id=359797c650c194dec61e05ab0d3fb6db-4429068-images-thumbs&n=13
  tab_end
\fi
%%%cit_text
Самое громкое заявление вчерашнего дня сделал известный доктор Комаровский.
Точнее, это было даже не заявление, а нечто вроде \emph{крика души} во время беседы
Евгения Комаровского с журналистом Дмитрием Гордоном, который вырезали из
записи разговора и превратили в вирусное видео.  Суть сводится к одной фразе:
"Я и 73\% лоханулись, когда выбрали Зеленского". По большому счету, Комаровский
имеет права говорить от имени этих 73\%, поскольку два с половиной года назад он
был, пожалуй, самым ярким представителем зе-лобби (все ждали, что он даже
станет министром здравоохранения), и его разочарование соответствует
настроениям значительной части многомиллионной армии избирателей Владимира
Зеленского.  Сейчас этот \emph{крик души} активно распространяют в соцсетях
порохоботы, пытаясь доказать, что в 2019-м правы были они. Однако мини-ролик с
Комаровским уничтожает их вождя не слабее, чем президента. Доктор просто
признает, что при Порошенко сегодня ситуация была бы не хуже, чем при
Зеленском, однако он говорит о том, что в 2019-м был выбором "между мухомором и
неизвестным" и что он выбрал неизвестное
%%%cit_comment
%%%cit_title
\citTitle{Антирекорд смертности от ковида, 73\% лохов, "гибридная война" Бацьки. Итоги "Страны"}, 
, strana.news, 10.11.2021
%%%endcit
