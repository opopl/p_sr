% vim: keymap=russian-jcukenwin
%%beginhead 
 
%%file slova.diplomatia
%%parent slova
 
%%url 
 
%%author 
%%author_id 
%%author_url 
 
%%tags 
%%title 
 
%%endhead 
\chapter{Дипломатия}

%%%cit
%%%cit_head
%%%cit_pic
\ifcmt

\ifcmt
  tab_begin cols=2
    pic https://avatars.mds.yandex.net/get-zen_doc/5233283/pub_60da2f307403af22afc8bde2_60da2f907403af22afc9a94e/scale_1200

    pic https://avatars.mds.yandex.net/get-zen_doc/1590219/pub_60da2f307403af22afc8bde2_60da331d307ed50db5ce7bfa/scale_1200
  tab_end
\fi

\fi
%%%cit_text
Я долго подбирал эпитеты для сотрудников украинского МИДа и считаю, что лучшим
будет \enquote{бандерложья \emph{дипломатия}}.  Предоставьте себе стаю диких обезьян, которые
решили, что они занимаются \emph{дипломатией}. Их \enquote{работа} будет громкой, кичливой,
истеричной, при этом бандерлоги будут постоянно пытаться нагадить как на
политических оппонентов, так и на союзников, которые рано или поздно устанут от
них. Тогда былые союзники брезгливо отойдут от них подальше, ведь бандерлог
способен лишь клянчить подачки, при этом оскорбляя не только врага, но и
благодетеля...  Рассмотрим нескольких, самых ярких и шумных членов стаи, их
повадки, привычки и манеру поведения
%%%cit_comment
%%%cit_title
\citTitle{\enquote{Гении} украинской дипломатии}, 
Мак Сим, zen.yandex.ru, 29.06.2021
%%%endcit
