%%beginhead 
 
%%file 15_03_2023.fb.kipcharskij_viktor.mariupol.1.r_k_tomu___18__den_
%%parent 15_03_2023
 
%%url https://www.facebook.com/permalink.php?story_fbid=pfbid0eVVdtx2iooPCR7tKDZ9pxtiUFQXYNZ7jYczhntDb5DuVVcBpD2cDYQWYnYAACxgCl&id=100006830107904
 
%%author_id kipcharskij_viktor.mariupol
%%date 15_03_2023
 
%%tags mariupol,mariupol.pre_war
%%title Рік тому: (18+) День 20. 15-03-22. Вівторок
 
%%endhead 

\subsection{Рік тому: (18+) День 20. 15-03-22. Вівторок}
\label{sec:15_03_2023.fb.kipcharskij_viktor.mariupol.1.r_k_tomu___18__den_}

\Purl{https://www.facebook.com/permalink.php?story_fbid=pfbid0eVVdtx2iooPCR7tKDZ9pxtiUFQXYNZ7jYczhntDb5DuVVcBpD2cDYQWYnYAACxgCl&id=100006830107904}
\ifcmt
 author_begin
   author_id kipcharskij_viktor.mariupol
 author_end
\fi

Рік тому:

(18+)

День 20. 15-03-22. Вівторок

Усю ніч рідкі, але дуже гучні \enquote{гупи}. Час від часу кулеметні черги. Таке
враження, що біля банку на Ілліча танк веде бій.

6:10. Снідав вівсянкою. Гупнуло так, що захиталися стіни і підстрибнули кришки
на каструлях. Вікна витримали.

\ii{15_03_2023.fb.kipcharskij_viktor.mariupol.1.r_k_tomu___18__den_.pic.1}

Накип'ятив чайник та бідон. Щось прилетіло на селище нижче Гонди. Пока заносив
чайник, гупнуло дуже голосно: кажуть, що був пуск ракети, яка злетіла трохи
вище 46-го будинку і впала у приватний сектор, але не вибухнула - це бачили
кілька людей біля вогнища. Женя з Толиком пішли в балку за стовбуром, який
знайшли вчора. Довго не поверталися. Гадав, що вони ловлять зв'язок біля
Квартального магазину. Вийшов на перехрестя і побачив їх без стовбура - його
хтось вже забрав. Гілок теж нема - дрова треба десь шукати.

По радіо сказали, що вчора з Маріуполя виїхало 160 машин.

У провулку перед будинком та на Гонди стоять машини - люди выезжают на своих
машинах. Збір на Драмтеатрі о 10:00.

Взяв телефон (я пишу на ньому свій щоденник, тому зазвичай він вдома), пішов до
Квартального: не зловив зв'язок. Повернувся по Нокію - вона зовсім розряджена.
Поставив її заряджатися від акумулятора. Толик запропонував зателефонувати з
його телефону - пішли вдвох. Біля перехрестя  з Блажевича почули дивний звук: з
правого боку від дороги лежить бетонний стовп - на ньому стоїть дівчина, яку
пригорнув до себе чоловік, а вона страшенно виє. Зі стовпа телефон краще
ловить, тож дівчина на нього залізла, аби впіймати зв'язок і дізналася, що
машину її батьків, які поїхали вчора, розстріляли і батьки з тими, хто був у
машині - загинули. Вона безперервно виє. Ані у житті, ані у кіно  такого не чув
і дай Боже, ніколи більше не чути.

Я сказав Толику номери Люби та Миколи  Ми дочекалися, поки звільнилося місце на
парканчику біля будинку на розі, залізли на той парканчик і, задираючи руку з
телефоном угору, почали чекати на з'єднання. Здається першим зв'язок впіймав
телефон Толика: я повідомив Миколу, що ми живі. Потім теж саме сказав Любі.
Вона сказала, що напередодні вони замовили молитву за нас \enquote{правильному
батюшці}. Зателефонував невістці, попросив, аби вона виклала на моїй сторінці у
ФБ про те, що ми живі. (Пізніше Ліля з Київа сказала: \enquote{Я побачила це - там за
п'ять хвилин 150 лайків}).

Коли поверталися, побачили кілька машин з білими стрічками на ручках дверцят -
люди збираються їхати. Пішов до дому, повідомив рідних про зв'язок.

Онучка, вставляючи USB-підсвітку у павербанк зігнула контакти - вони чи то
тріснули, чи то замкнулися: напруга повний нуль.

Прилетіло у приватний будинок за двохповерховим - там полум'я та стовп чорного
диму. Прилетіло у п'яти поверхівку за Куйбишевськими \enquote{сталінками}.

Хлопець,  що винаймав двокімнатну на першому поверсі, поїхав: щойно він виніс
свої речі, в квартиру зайшло з десяток людей з двома немовлятами.

Біля вогнища продавали горілку по 500 грн за пляшку та конь'як по 550-750.

Син з хлопцями пішли по воду: принесли воду та \enquote{цеглини у штанях} - прилетіло у
гаражі дуже близько. 

Усе зважили: ризики виїзду та ризики залишатися і вирішили їхати: почали
збиратися, бо настає казкове життя: чим далі - тим страшніше. Як жити, якщо
вікна полетять або дах пошкодить і почне заливати з верхніх поверхів?

Акумулятор в Епіці повністю розряджений: тестер показує 3 вольта. Мабуть колись
вночі від \enquote{гупів} спрацювала сигналізація, але ніхто не почув.

11:25. Поставили акумулятор заряджатися: він спочатку не хотів, потім почав
приймати заряд. Я виставив 10 ампер замість 6. 

Згадав що треба запитати Лілю, чи зможуть нас прийняти, якщо поїдемо через
Київ. Пішов до Квартального. На дорозі велика калюжа крові, а в провулку нижче
Гонди лежить тіло чоловіка з половиною голови. Моторошне видовище - рожево-сірі
від крові  мізки та білі кістки, розкидані повсюди навколо.

Кров на асфальті - пораненого чоловіка, якого поліцейські повезли до госпіталю
- йому пошматувало спину. Я вирішив не наражатися на черговий приліт і
повернувся у двір.

12:00. Онучка отримала на Нептуні консервовану кукурудзу, горошок, томатну
пасту, п'ять пачок печива та вологі серветки.

Акумулятор зарядився до 12-ти вольт, але то через те, що він нагрівся.

13:30. Голосно гупає. Чорний дим правіше домни 2 - може прилетіло у батарею?
Кулеметні черги, гарматні постріли майже без перерви. Вікна тремтять дуже
сильно.

14:30. Нагрів бідон та чайник. Завели генератор, тож поставив акумулятор
заряджатися. На обід заносив його до Калини.

Віталій переїхав до нас, бо в них вибило вікна. В нього дуже розряджений
акумулятор - ледве завівся і майже нема бензину.

До вогнища прийшов сват. Я сказав йому що завтра о 9:00-й виїжджаємо. Він
сказав, що вони залишаються, але прийдуть до нас попрощатися.

Я пропонував Жені, Толику, ще комусь їхати Дев'яткою - газ в ній є, бензину,
аби заводитися я дам. У тих, хто має права, є машини. Інші прав не мають. Я
сказав, аби шукали водія з розбитою машиною. Хтось сказав, що не хоче, аби його
розстріляли як мародера, що поцупив суду машину. Я ж віддав би права та їхали б
разом. Не допомогло.

15:00 дуже голосно. (Вже по виїзді ми дізналися, що то скинули бомби на школу
20. Було кілька заходів літаків. Кажуть, що там був дуже міцний підвал і кілька
днів з нього лунали голоси людей. Зараз школу зруйнували повністю і залишки
школи разом з залишками людей вивезли на звалище...).

Переливали по машинах бензин, злитий з Дев'ятки. Залишили дав банки: раптом в
одній з машин (у якій першій - не вгадаєш) закінчиться - будемо доливати. Люди
підходили, питали де \enquote{надибав} заправку. Дмитро Ш. побіг у свій гараж зливати
бензин з косарок. Люди просили продати бензин за скажені гроші: а нам на чому
їхати?

17:30. Поставили акумулятор у Епіку і спробували завести. Крутить, але не
завелася. Під'їхав Калиною, крутили від неї - так само. Завтра спробуємо
завести з буксиру. Якщо не заведеться - усі поїдемо Калиною.

Пішли додому. Невістка нервує. Онук отримав завдання написати на аркушах А4
\enquote{ДЕТИ}.

19:00. Гупає, але відносно того, що було вдень - доволі тихо.

Написалося рік тому:

\begingroup
\raggedcolumns
\begin{multicols}{2} % {
\setlength{\parindent}{0pt}
\obeycr\noindent
\em
И вновь, как восемь лет назад,
Листки на стеклах лобовых висят
И вновь они притягивают взгляд:
Четыре буквы прямо в мозг летят:
\smallskip
И гнев, и страх вдруг воскресили буквы эти:
Четыре буковки, два слога, слово "Дети".
Что значит: "Бросили мы дом!
Со всем привычным, нажитым добром!
Спасаем найценнейшее на свете!
Не цельтесь! Ведь в машине этой дети!!!"
\smallskip
Но оркам-нелюдям на это наплевать -
У них приказ: стрелять и убивать!
А на допросах дурака валять:
"Мы на учениях! Мы не хотим стрелять..."
\smallskip
Но души детские на небеса летят
И просят Господа убийц их покарать.
И молят, чтобы не было войны,
Кому, скажите, смерти детские нужны?
\restorecr
\end{multicols} % }

\endgroup

%\ii{15_03_2023.fb.kipcharskij_viktor.mariupol.1.r_k_tomu___18__den_.cmt}
