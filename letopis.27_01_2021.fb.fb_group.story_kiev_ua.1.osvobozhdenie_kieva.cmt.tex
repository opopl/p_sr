% vim: keymap=russian-jcukenwin
%%beginhead 
 
%%file 27_01_2021.fb.fb_group.story_kiev_ua.1.osvobozhdenie_kieva.cmt
%%parent 27_01_2021.fb.fb_group.story_kiev_ua.1.osvobozhdenie_kieva
 
%%url 
 
%%author_id 
%%date 
 
%%tags 
%%title 
 
%%endhead 
\zzSecCmt

\begin{itemize} % {
\iusr{Елена Мирошниченко}
Плачу.

\begin{itemize} % {
\iusr{Мария Каменцова}
\textbf{Елена Мирошниченко} . 

Я сама плакала, когда писала... И мама, когда мне рассказывала. Я постаралась
воспроизвести ее рассказ с фотографической точностью. Все детали...


\iusr{Лариса Заболотная}
\textbf{Елена Мирошниченко} и я...

\end{itemize} % }

\iusr{Александр Лесик}
кошмар. моей бабке повезло она в каком то другом Киеве оккупацию пережила.

\begin{itemize} % {
\iusr{Мария Каменцова}
\textbf{Александр Лесик} . Да жить в оккупацию, мама рассказывала, можно было... Только когда немцев стали выбивать из города, они начали свирепствовать!

\begin{itemize} % {
\iusr{Александр Лесик}
\textbf{Мария Каменцова} 

а моя говорила что тихо ушли. за пару дней до освобождения. просто вдруг много
бродячих собак появилось. (немцы собак беспризорных отстреливали). не стало
немцев. пришли собаки. а потом коммунисты начали Киев обстреливать в котором
кроме жителей не было уже никого (но это понятно они ж предатели не
самоубились). а еще говорила - когда советские Киев сдавали то повзрывали все
эшелоны с зерном на Дарницком вокзале. чтоб не досталось никому. а фашисты когда
бежали-склады с едой (которую не успели вывезти) не взрывали. наверно взрывчатку
пожалели.

\iusr{Мария Каменцова}
\textbf{Александр Лесик} . 

Я так понимаю, что немцы тихо ушли, но до этого пытались уничтожить то и тех,
кого могли. Передаю просто рассказ моей мамы. Она девочкой была, рассказала то,
что сама видела. Может быть, другие видели и знали больше!

\iusr{Александр Лесик}
\textbf{Мария Каменцова} наверно.у мамы медали были ?

\iusr{Мария Каменцова}
\textbf{Александр Лесик} . Ей было 14 лет. Медали у нее были, но не боевые, юбилейные, и довольно много! Они сохранены.

\iusr{Александр Лесик}
\textbf{Мария Каменцова} 

ага у всех так. и дед и бабушка вся грудь в медалях. как начал вникать оказалось
все юбилейные. 20 лет победы 30. и т д. даже орден вов. и тот юбилейный. а за
реальные боевые достижения дед ..благодарность.. от товарища Сталина.. (письмецо
на тетрадном листе написанное) получил. ъ

\iusr{Tatiana Thoene}
\textbf{Александр Лесик} 

моя мама тоже вспоминала, что немцы ушли быстро и вроде «тихо». У них во дворе,
на Коминтерна, была немецкая полевая кухня. Когда наши стали приближаться к
Киеву, собрали манатки и пошли строем к вокзалу. Наверное, не везде
свирепствовали и кому-то повезло.

\end{itemize} % }

\end{itemize} % }

\iusr{Aleksandra Gorovaya}
Огромное спасибо! Слов нет - только слезы.

\begin{itemize} % {
\iusr{Мария Каменцова}
\textbf{Aleksandra Gorovaya} . Я, пока писала (вначале на компьютере), тоже целый день плакала...

\iusr{Мария Каменцова}
\textbf{Aleksandra Gorovaya} . Мама и папа, оба пережившие оккупацию в Киеве, рассказывали много, я здесь еще покажу! Некоторые рассказы прямо необыкновенные! Хотя они почти дети еще были, 14 лет...

\begin{itemize} % {
\iusr{Aleksandra Gorovaya}
\textbf{Мария Каменцова} Мои смогли уехать из Киева, хотя соседки по коммунальной квартиры, уговаривали бабушку остаться. (Как потом выяснилось , дворник выдал немцам всех евреев в доме). Школьная подруга моей матери, которая осталась со своей мамой в оккупированом Киеве, рассказывала, что немцы разрешили учиться в школе только детям, которые занимались балетом.
\end{itemize} % }

\end{itemize} % }

\iusr{Тома Храповицкая}
Спасибо... больше нет слов, Спасибо Вам и Вашей Маме за такие тяжёлые но нужные всем Нам воспоминания!!!!!!!!!

\iusr{Мария Каменцова}
\textbf{Тома Захарова} . Без памяти нет Родины.

\iusr{Ирина Иванова}

Машуня, до слез!!!! И какой же нежной, чистой как первая снежинка на фото твоя
мама. По фото не скажешь, что у этой девочки столько силы..... .Светлая,
добрая память живет во мне о тете Нине

\begin{itemize} % {
\iusr{Мария Каменцова}
\textbf{Ирина Иванова} . Вот она и побыла с нами как будто воочию... Вечная память! Рассказов как ее, так и папы множество. Я тут еще покажу...

\iusr{Оксана Тернавская}
\textbf{Ирина Иванова} мы все помним и её, и всех наших близких с ул.Герцена. Тем более, жизнь так связала наши судьбы.
\end{itemize} % }

\iusr{людмила петрухина}

Моя бабушка тоже пережила этот ужас, пряталась в Голосеевском лесу со своей
подругой, тоже рассказывала об освобождении Киева, о том, что кругом были одни
развалины, сколько же им пришлось пережить.

\iusr{Мария Каменцова}
\textbf{людмила петрухина} . Мы обязаны об этом помнить и рассказывать. Чтобы не повторилось...

\iusr{Giorgi Begeluri}
Kakaia krasivaiajenshchina!

\begin{itemize} % {
\iusr{Мария Каменцова}
\textbf{Giorgi Begeluri} . Потом она немного снималась в кино, ее пригласили... Стала еще красивее!

\iusr{Giorgi Begeluri}
\textbf{Мария Каменцова} ona vasha mama?

\iusr{Мария Каменцова}
\textbf{Giorgi Begeluri} . Да, конечно!  @igg{fbicon.smile} 
\end{itemize} % }

\iusr{Виктория Аксененко}

Мой дедушка был военный и успел вывезти бабушку с моей мамой и тетей. А как
только освободили Киев они вернулись в свою квартиру на ул. Чкалова. Мама тоже
много расказывала уже про освобожденный Киев. Страшное время было..... Мы должны
знать и помнить те года.....


\iusr{Татьяна Чернышева}
Мама такая красивая@igg{fbicon.heart.red}

\iusr{Мария Каменцова}
\textbf{Татьяна Чернышева} . Спасибо! Я еще буду рассказывать, покажу ее фото постарше, когда расцвела – она была прямо кинозвезда!

\iusr{Marina Chernaya}
Как ценны воспоминания очевидцев, когда так нещадно переписывают историю!

\begin{itemize} % {
\iusr{Мария Каменцова}
\textbf{Marina Chernaya} . Критерий всего – ИСТИНА.

\iusr{Marina Chernaya}
\textbf{Мария Каменцова} 

истину знает только Бог, порядочные историки, используя научные методы, вне
всякой политики и пропаганды, пытаются к ней приблизиться. В истории всех
народов и стран были светлые и темные страницы, нельзя этим манипулировать. Как
сказал А. Мень, правда одна, остальное - ложь.

\end{itemize} % }

\iusr{Елена Свец Баркан}

Моя бабушка с двумя детьми 36 и 41 года рождения (мой дядя и мама) была все
время оккупации в Киеве. Не могу даже себе представить, каково ей было

\begin{itemize} % {
\iusr{Александр Лесик}
\textbf{Елена Свец Баркан} может надо было спросить?

\begin{itemize} % {
\iusr{Елена Свец Баркан}
\textbf{Александр Лесик} 

кое что бабушка рассказывала, но она умерла, когда мне было только 16. Что
успела - рассказала. Как ходила по селам под Киевом менять вещи на продукты.
Когда проходила через посты если стояли немцы, то проверяли документы и
пропускали, а если были полицаи, то часто забирали вещи или продукты.

\iusr{Александр Лесик}
\textbf{Елена Свец Баркан} ага. а энкаведисты с 1918 до 39 просто стреляли и даже не закапывали

\iusr{Ольга Жук}
\textbf{Александр Лесик} вас, лично стреляли?
\end{itemize} % }

\iusr{Лариса Заболотная}
\textbf{Елена Свец Баркан} и мои... Бабушка 1905 года, мама 1929...

\iusr{Наталия Погорелова}

Моя бабушка 1908 года с двумя дочками 1928 и 1933 тоже остались в оккупации.
Жили на Сталинке в подвале деревянного домишки на 4 семьи. Мама очень много
рассказывала о том времени...

Перед освобождением Киева немецкий офицер, занимавший в их доме пустую
квартиру, предупредил, что город будет взорван и чтоб уходили поскорее. Мамина
семья с соседями, со своим каким-то скарбом, ушли в сторону Мышеловки. Ночевали
и дневали (в ноябре!) под открытым небом. Но ночами мама с её тетей (1920 г.р.,
8 лет разницы всего, так что были практически подругами) ходили в Киев
покормить лежачую бабушку тетиного жениха (он воевал, а тётя за ней
присматривала)...

Тетю эту, кстати, отправляли в начале оккупации в Германию, но она как-то
умудрилась сбежать - выпрыгнула из вагона и вернулась в Киев.

\iusr{Victor Rozhkov}

Мой отец тоже был в оккупации. Но такого как расстрел мирных не было. Наоборот
немцы уводили людей чтоб они не пострадали. А хуже всех в окупации были полицаи
из наших. Это слова отца. И он ещё жив.

\end{itemize} % }

\iusr{Тетяна Бондаренко}
Читала про саме цей момент в одній книзі про Київ в окупації!

\iusr{איליאנה ז׳וקוב}
 @igg{fbicon.face.downcast.sweat}  @igg{fbicon.flag.izrail}

\iusr{Lola Madino}

Щось я чула, та й документи свідчать - німці ВИСЕЛЯЛИ мирне населення. А не
вбивали... Так, за непідкорення наказу - розстріл. Але можна було скоритись
наказу та піти.

\begin{itemize} % {
\iusr{Тетяна Бондаренко}
\textbf{Lola Madino} Піти куди?
А хто не міг?

\iusr{Наталия Погорелова}
\textbf{Тетяна Бондаренко} хто не міг через хворобу, наприклад - залишався. Мама розповідала мені.

\iusr{Юлия Ильенко}
А про Бабий Яр вы чулы?

\iusr{Lola Madino}
\textbf{Julia Ilienko} а ви? Що саме чули?

\iusr{Оксана Волкова}
\textbf{Lola Madino} в Бабий яр тоже \enquote{добровольно} отправляли @igg{fbicon.cry} 

\begin{itemize} % {
\iusr{Lola Madino}
\textbf{Oksana Volkova} 

Так, сусіди, літня єврейська пара - пішли. \enquote{Добровільно} Хоч знали, що на
смерть. Їх умовляли - \enquote{не може бути, це ж культурна нація, вас вивезуть} - вони
сказали \enquote{ні, ми знаємо, нас уб'ють... Коли повернуться сини (з Червоної армії)
- скажіть, де наші кісточки}

Але сини не повернулись.
\end{itemize} % }

\iusr{Юлия Ильенко}

\ifcmt
  ig https://scontent-frt3-1.xx.fbcdn.net/v/t1.6435-9/142419677_3602589599859654_11061196312617109_n.jpg?_nc_cat=102&ccb=1-5&_nc_sid=dbeb18&_nc_ohc=diCpdINf764AX8MwH_W&_nc_ht=scontent-frt3-1.xx&oh=00_AT_ibODpXN11YVSJO4gLVrJrMC0nQlfw2wQMp150ux9ghw&oe=6211A5CB
  @width 0.2
\fi

\iusr{Юлия Ильенко}
А отсюда мой мирный дед с мирным другом успели сбежать

\ifcmt
  ig https://scontent-frt3-1.xx.fbcdn.net/v/t1.6435-9/142566451_3602590373192910_4530757805676079825_n.jpg?_nc_cat=107&ccb=1-5&_nc_sid=dbeb18&_nc_ohc=CBHwWoi3WcAAX-W7o53&_nc_ht=scontent-frt3-1.xx&oh=00_AT8arrKZezOxD-Zd9xUHdzu8VmMS4Gi27m-A3SQPs-bbTA&oe=620F9842
  @width 0.2
\fi

\begin{itemize} % {
\iusr{Владислав Проноза}
\textbf{Юлия Ильенко} это фото сделано вашим дедушкой? Он бежал и фотографировал?
\end{itemize} % }

\iusr{Lola Madino}
Мої батьки пережили окупацію обоє в Києві.

Коли німці оголосили наказ перед наступом Червоної армії у листопаді 43-го -
\enquote{для уникнення зайвих жертв серед населення звільнити зону військових дій} -
взяли, що змогли та пішли. У Мотовилівку наприклад.

\begin{itemize} % {
\iusr{Мария Каменцова}
\textbf{Lola Madino} . Ангелы кротости. Что-то люди от них прятались по ярам, лесам и канализационным трубам...  @igg{fbicon.wink} 

\iusr{Lola Madino}
\textbf{Maria Kamentsova} 

\enquote{тактику випаленої землі} застосували спочатку совєти, відходячи з
Києва. Наприклад спалили запаси продовольства а не роздали людям, яких залишали
напризволяще.

Німці були ворогами, це без сумніву.

Але \enquote{вбити всіх жителів} - це ви загнули...


\iusr{Оксана Волкова}
\textbf{Lola Madino} в предыдущем комменте было \enquote{щось чула} , теперь уже родители мирно ушли из зоны боевых действий  @igg{fbicon.thinking.face} 

\iusr{Lola Madino}
\textbf{Oksana Volkova} 

Так, вони виконали наказ, за невиконання якого загрожував розстріл. І від них я
це чула. Не \enquote{ВСІМ ЖИТЕЛЯМ}, а тим, хто порушував наказ.

А ще я читала документи. Чого і вам бажаю. А не \enquote{савєцкую історію вєлікай
атєчествєннай}

\iusr{Оксана Волкова}
\textbf{Lola Madino} 

\enquote{советская история Великой отечественной} написана кровью моей семьи, моих
дедов и бабушек, которые воевали с первого до последнего дня войны!

\iusr{Lola Madino}
\textbf{Oksana Volkova} менше патетитки. Більше документів.

\iusr{Оксана Волкова}
\textbf{Lola Madino} начинайте выкладывать пруфы. Только не книжечки, а реальные доки.

\iusr{Ирина Попова}
\textbf{Lola Madino} бот без фото и истории.

\iusr{Игорь П.}
\textbf{Lola Madino} ссылку на документы можно? Только вову вятровича не предлагать.
\end{itemize} % }

\iusr{Мария Каменцова}
\textbf{Lola Madino} . Щось я чула – это не свидетельство очевидца, а сплетни и домыслы.

\begin{itemize} % {
\iusr{Lola Madino}
\textbf{Maria Kamentsova} мені лінь шукати документи для вас, спробуйте самі.
Або почитайте Малакова. Може - допоможе.

\iusr{Tatiana Thoene}
\textbf{Maria Kamentsova} да бросьте, здесь полно коментов - пересказы воспоминаний. Не все претендуют на историческую достоверность.
\end{itemize} % }

\iusr{Юлия Ильенко}
\textbf{Lola Madino}, то, что вашим родственникам повезло выжить, не даёт вам право плевать на тысячи убитых немцами родственников других людей

\iusr{Lola Madino}
\textbf{Julia Ilienko} можете вказати пальцем - де я на когось плюнула?

\iusr{Наталия Погорелова}
\textbf{Lola Madino} 

мама моя теж казала, та й їх сім'ї німецький офіцер, що мешкав у сусідній
квартирі (сусіди встигли виїхати вчасно), порадив швиденько забратися з Києва.
То й зібрали, що могли, й пішли у Мишоловку. І їх сусіди всі з вулиці - теж. Не
розстрілювали німці, а видали наказ звільнити мешканцям місто, бо все буде
знищено.


\iusr{Юлия Ильенко}
\textbf{Lola Madino}, а почему вы упорно избегаете темы Бабьего Яра? И как эта тема сочетается с вашим утверждением, что нiмцi не вбивали?

\iusr{Людмила Новікова}

Так, медсестра літнього віку, що працювала в лікарні номер 1, в якій я лежала,
що навпроти Володимирського собору, (Тамарою її звали), розповідала, що їх німці
виселили з Ірпеня, бо там проводили бойові дії, в Мироцьке

\end{itemize} % }

\iusr{Nata Pylypenko Shmorgun}

Взрослые дети войны.. до слёз и мурашек, это ж какой ужас и страх был и все
вынесли!!! У меня вдруг возникла ассоциация с детками нынешней нашей войны,
дети которые видели и слышали обстрелы, взрывы и многие дети сироты, какой
страх и ужас пережили и они и ведь также будут рассказывать своим детям и
внукам.... война, оккупация всегда страшно, и ценна каждая минута с близкими! И
важно помнить!

\begin{itemize} % {
\iusr{Ольга Авраменко}
\textbf{Nata Pylypenko Shmorgun} Вы правы. Даже в самом страшном сне не могло присниться, что русские будут воевать с украинцами.

\iusr{Sokolova Oksana}
\textbf{Ольга Авраменко} то есть история ни чему не учит?) Столетиями воевали, а тут вдруг не могло приснится.
Не было и вот опять.

\iusr{Оксана Тернавская}
\textbf{Sokolova Oksana} кто с кем воевал?

\iusr{Арт Юрковская}
\textbf{Sokolova Oksana} украинцы с поляками воевали.

\iusr{Sokolova Oksana}
\textbf{Art Yurkovska} воевали. Но не один львовский музей поляки не ограбили.

\iusr{Sokolova Oksana}
Дамы я вас оставлю при вашем мнение, а себя при своем.
Вы можете опять про единую страну, а я порадуюсь что этой страны больше никогда не будет.
И если вам не нравится жить в Украине, то чемодан, вокзал и знакомое направление.
\end{itemize} % }

\iusr{Тетяна Бондаренко}

А ще, в родині мого зятя був такий життєвий момент: чоловік, військовий в
чинах, покинув свою дружину з маленьким сином через те, що вони були в окупації
в Ірпіні. Ось так!

Батьком для хлопця стала зовсім інша людина, яка і дала йому своє прізвище!

\iusr{Надежда Василенко}
\textbf{Тетяна Бондаренко} это был подонок, который просто нашел предлог. Не было бы окупации, он бы просто так ушел. Мало ли таких, которые после войны поменяли жен

\iusr{Лариса Чепурна}
Рассказ похож на какой то бред и кич

\begin{itemize} % {
\iusr{Александр Лесик}
\textbf{Лариса Чепурна} пустая страница с полупустой спорит.полк победосесов проснулся

\iusr{Оксана Тернавская}
\textbf{Александр Лесик} страница закрыта для таких, как вы

\iusr{Александр Лесик}
\textbf{Оксана Тернавская} ваша открыта)) вы искренний совкодрот победобес

\iusr{Оксана Тернавская}
\textbf{Александр Лесик} 

мне на Ваше мнение искренне наплевать. Во времена СССР я искренне его
критиковала и, кстати, не имела с этим никаких проблем. И в то, и в нынешнее
время, я жила хорошо, но всё это благодаря моим родителям, моим способностям,
и, безусловно, обстоятельствам. Никогда не буду охаивать своё прошлое, так как
без него нет будущего. А вот за Победу глотку перегрызу!!! Хорошего Вам дня!

\iusr{איליאנה ז׳וקוב}
\textbf{Лариса Чепурна} не то написала не в тот день .

\iusr{Таня Краснова}
\textbf{Лариса Чепурна} Лариса, это ты ещё про куклу цыганку «мыло» не читала @igg{fbicon.laugh.rolling.floor} 

\iusr{Мария Каменцова}
\textbf{Таня Краснова} . Из таких, как вы, немцы как раз и варили мыло. После удушения в газовых камерах.
\end{itemize} % }

\iusr{Ефросиния Батурина}
Такая Красивая !

\iusr{Мария Каменцова}
\textbf{Ефросиния Батурина} . Спасибо!  @igg{fbicon.smile} 

\iusr{Lubov Vladimirovna}

Моя свекровь Панченко София Филипповна освобождала Киев. Зенитные войска.
Святая была женщина!!! Много рассказывала... страшные были годы...светлая ей
память! На фото справа. Награждена орденом Красной звезды. Орден храним как
реликвию.

\ifcmt
  ig https://scontent-frx5-1.xx.fbcdn.net/v/t1.6435-9/142832175_418898539374085_6058853585748773206_n.jpg?_nc_cat=110&ccb=1-5&_nc_sid=dbeb18&_nc_ohc=Dun_vjzbOiwAX8phGDI&_nc_ht=scontent-frx5-1.xx&oh=00_AT_REmr7nl0idPoNO44AWyv6TNsoXgBvGlFDvWnHtywkDg&oe=620F69E9
  @width 0.2
\fi

\begin{itemize} % {
\iusr{Ирина Владиславовна Остапчук}
\textbf{Lubov Vladimirovna} доземний уклін Вашій свекрусі!

\iusr{Lubov Vladimirovna}
\textbf{Ирина Владиславовна Остапчук} Дуже вдячна Вам за такі слова! Які вони були молоді дівчатка і скільки випало на їхню долю... але це вже інша історія...
\end{itemize} % }

\iusr{Инна Игоревна}
Читаешь, аж мурашки по телу бегут и слезы наворачиваются на глазах.

\iusr{Мария Каменцова}
\textbf{Инна Игоревна} . 

Я, когда вспоминала рассказ мамы и писала его на компьютере, прежде чем тут
разместить, тоже весь день плакала... Сколько им пришлось пережить!!!

\iusr{Ирина Горобец}
Сколько же им пришлось пережить! Благодарю, что делитесь!

\iusr{Лариса Волошина}

Моя мама пережила оккупацию под Киевом. Кошмар... Когда началась война, ей было
5 лет. Отступление наших помнит эпизодически, но помнит. Чем дальше, тем больше
воспоминаний...

Ещё при наступлении немцев их семью пытались расстрелять. Бабушка была
верующая, а дед - ярый атеист. Для немцев - атеист значит коммунист. Кто-то
донёс. Немцы всю семью - бабушка, дед, четверо детей (от 1 года до 7),
поставили к стенке (к печке), собирались расстрелять... Но кто-то увидел,
уговорил подождать старосту, которого нашли и привели... Стояли так, может час,
может два, может больше... Староста помог. В общем отстояли их. ..

Но самое страшное, говорит мама, это - голод. Постоянный, никуда не исчезающий
голод... Во время налетов, это уже 43-й год, отступление немцев, они прятались
в погребе. Но как только самолёты улетали, дети, 6, 7, 8.....лет, сразу же
бежали к воронкам от бомб, надеясь в сумках, карманах убитых найти хоть что-то
съедобное, корочку, крошки... УЖАС...

Жили в непосредственной близи от железной дороги. Пришлось видеть взорванные
партизанами немецкие поезда - нечеловеческие крики, кровь, изувеченные тела,
части тел.... Мама не может простить немцев до сих пор, не может слышать
немецкую речь, но картина человеческого месива, пусть и врагов, потрясла её на
всю жизнь.

Отступление немцев, наступление наших, это же не единовременный момент, помнит
хорошо...

Налеты, бои, бесконечное количество убитых (хотя это правый берег Киева, здесь
наступление с форсированием Днепра все же не сравнить)...

Много чего было, сразу не перессказать...

Мама забыть весь этот ужас не может...

После войны тоже было не легко, ох как нелегко, но с войной, оккупацией не
сравнить. Так говорит мама.

Не хочу провоцировать политические споры, но не сказать не могу, для неё День
победы - это праздник, самый большой праздник.

Не буду продолжать...

\begin{itemize} % {
\iusr{Мария Каменцова}
\textbf{Лариса Волошина} . 

Я не ставила себе задачу – осветить ВСЁ освобождение Киева. У меня – камерный
рассказ только одного отдельного человека, непосредственные впечатления
14-летней девочки. Мне представилось, это ценно... и в самый раз для группы
Киевские истории, это истории О НАС, киевлянах!

\begin{itemize} % {
\iusr{Лариса Волошина}
\textbf{Мария Каменцова} 

Спасибо за Ваш рассказ. Очень проникновенный и настоящий. У меня тоже возникло
желание поделиться воспоминаниями, воспоминаниями, которые за пределы семьи не
выносились. Наверное пришло время, спасибо Вам...

\end{itemize} % }

\iusr{Чорна Галина}
\textbf{Лариса Волошина}
Душевно, эмоционально написали о войне...

\iusr{Лидия Суворовская}
\textbf{Лариса Волошина} Ее можно понять - день Победы стал для нее днем надежды на жизнь без голода и ежеминутного страха смерти

\begin{itemize} % {
\iusr{Алла Тихонова}
\textbf{Лидия Суворовская}, \enquote{для неё} только?! И для меня тоже праздник. Для полицаев и фашистов - нет.
А для Вас?

\iusr{Светлана Ковтун}
\textbf{Алла Тихонова} 

для меня день победы скорее день скорби, чем праздник у меня оба деда погибли
на войне, а бобкука с новорожденной мамой пережили окупацию Киева, война это
ужас, но не человеческое отношение к своим гражданам только в СССР было
,поэтому не вижу я чем особо гордится - сколько мирных жителей, сколько содат мы
потеряли, скольких победителей после войны в лагеря сослали, все не так
однозначно..

\iusr{Rimma Turovskaya}
\textbf{Светлана Ковтун} Что же Вы во всем ищете только плохое?

\iusr{Светлана Ковтун}
\textbf{Rimma Turovskaya} почему плохое?, просто пытаюсь смотреть без розовых очков..
\end{itemize} % }

\iusr{Ирина Попова}
Страннно....вы даже как то извиняетесь за то, что для вашей мамы День Победы -
Праздник?

\end{itemize} % }

\iusr{Зоя Майстренко}

Мои мама и бабушка тоже пережили оккупацию, в их дом попала бомба, они выбежали
с него в последние минуты. Бабушка осталась с двумя детьми, дедушка погиб при
обороне Киева.

\iusr{Ольга Васильевна}

Мою маму освобождали американцы под Кельном в небольшом городке! Когда она
рассказывала о своих чувствах, я рыдала! Попав в то рабство и не имея никакой
информации о ситуации, они просто старались выжить. А с падением Сталинграда
заметили удрученность немцев, у многих появились траурные ленты, после этого
ждали освобождения!

\iusr{Лариса Маковецкая}

Моего отца в 14 лет во время облавы, возле памятника Богдана Хмельницкого,
немцы забрали на работу в Германию, моя бабушка даже не знала где он и что с
ним, но вот каким то чудом она через год получила открытку \enquote{я живой, только
очень хочется жрать} эта открытка была вся в слезах радости и горя.

\iusr{Wiktoria Arto}

Какой необычный рассказ.. очень интересный, но я никогда не слышала, о том,
моменте, когда немцы отходили, что люди прятались.. и вот такое потрясающее
воспоминание, как люди выживали. Это же описала свидетель событий, тогда
девочка, а потом пожилая женщина, ведь помнила эти волнующие моменты всю свою
жизнь! Спасибо большое @igg{fbicon.heart.red}

\begin{itemize} % {
\iusr{Мария Каменцова}
\textbf{Wiktoria Arto} . 

Прятались, потому что немцы расстреливали всех оставшихся. Советской армии не
должно было достаться НИЧЕГО. Выжженная земля и мертвые жители! Получите вашу
победу!!! Но не вышло... Люди выжили и город остался. Мама рассказывала – после
освобождения жители сами, без принуждения и приглашения, каждый день ходили на
Крещатик разбирать завалы, оставшиеся от разрушенных зданий.

\begin{itemize} % {
\iusr{Наталия Погорелова}
\textbf{Мария Каменцова} и моя мама ходила разбирать завалы Крещатика... Тоже говорила, что люди сами шли.

\iusr{Алекс Вельт}
\textbf{Мария Каменцова} 

Вы бредите ?? Киев был объявлен Кампфзоной - моя бабушка помнила это. Зашли
двое солдат - сказали \enquote{5 минут - пошел вокзал- шнель}. Они думали что их будут
убивать и плакали. У бабушки двое детей было и сестра с ними ее. Собрали всех в
колону и привели на Пост-Волынский. Там всех в вагонах вывезли в Ирпень и
Бородянку. Вернулись они через три дня пешком все домой. В городе были уже
советы. Никто никого не стрелял. Только голодные были страшно. Все соседи
вернулись живыми.

\iusr{Алла Тихонова}
\textbf{Алекс Вельт} , может это для полицаев \enquote{шнель-вокзал}?
И потом, почему \enquote{пошел вокзал}? Зачем Бородянка?
И все здоровы, никто не бредит. Если Вы не умеете общаться вежливо, то лучше не вступать в разговор вообще. Ферштейн?

\iusr{Мария Каменцова}
\textbf{Алекс Вельт} . Их просто не успели расстрелять, была неразбериха при оставлении города. Повезло! Я за них рада!
\end{itemize} % }

\end{itemize} % }

\iusr{Ирина Архипович}
До слёз!!.. Мои тоже пережили оккупацию!!!!....

\iusr{Наталия Багинская}

Моя мама родилась в Киеве в 1933 году (голодомор), всю войну с бабушкой в Киеве
находились, прятались по ярам. Много чего рассказывали, голод сильный, на Подоле на
Красной площади сидели семьи умершие от голода. Слышали каждый выстрел в Бабьем
яру, они жили на Куреневке. Мкиа всю жизнь была подвержена этим страхам. Ей было 8
лет когда началась война.

\begin{itemize} % {
\iusr{Мария Каменцова}
\textbf{Наталия Багинская} . Жуть... Вот такого мама мне не рассказывала!

\iusr{Наталия Багинская}
\textbf{Мария Каменцова} Рассказывали, когда начался голод, были такие, которые ели людей, это наши украинцы, немцы сами их расстреляли.

\iusr{Наталия Багинская}

Ходили на обмен по селам. Мама болела тифом, 40 днейбыла без сознания, это когда
наши уже вошли в Киев. Соседи донесли и по ехала санитарная машина из больницы (
из нашей больницы и наши врачи),хотели маму забрать в больницу. Бабушка завалила
маму периной и подушками и сказала, что ее нет. И сама выходила, ходила брала
молоко козье и вливала теплое ей в рот. Потом мама выздоровела. Рассказывпла
хотела зажечь Лампадку, начали гореть волосы. Отгодода мама теряла
сознание, бабушка от голода опухла по пояс, если бы опухоль дошла до сердца ,она
бы умерла. Баьушкп работала в брд нице Павлова, главврач узнал, что она свой обед
не ест, а носит домой ребенку, дал распоряжение выдавать обед и на ребенка. Очень
много рассказывали. Всего и не помню уже.

\iusr{Марья Заславская}
\textbf{Наталия Багинская} хорошо, что рассказывали... мои мне мало рассказывали. хотя пережили всё
\end{itemize} % }

\iusr{Ирина Нищимная}
Сердце разрывается от боли,как же они выжили,, если выжили,,, война самое страшное, что может быть на этой планете,,,

\iusr{Lola Madino}
***Когда немцам стало ясно, что город не удержать, они применили тактику «выжженной земли».*** - правда
***Все, что можно было уничтожить, должно было быть взорвано*** - правда.
***ценности – вывезены*** - правда
***все жители – убиты*** - очень трагично и патетично - НО НЕПРАВДА.
На этом у меня все.

\begin{itemize} % {
\iusr{Валентина Бабченко}
\textbf{Lola Madino} ,а факты?

\begin{itemize} % {
\iusr{Таня Куликова}
\textbf{Валентина Бабченко} 

Киев не был мёртвым накануне освобождения и это известный факт. Многие мои
киевские соседи пережили оккупацию. Моя бабушка Куликова Елизавета
Савельевна, входившая в группу киевских подпольщиков, была расстреляна в Бабьем
Яру, по доносу соседей, за день до освобождения Киева. Все годы оккупации город
жил своей тяжёлой, голодной жизнью

\ifcmt
  ig https://scontent-frt3-2.xx.fbcdn.net/v/t1.6435-9/143741033_3642061582578265_3853686079117016794_n.jpg?_nc_cat=103&ccb=1-5&_nc_sid=dbeb18&_nc_ohc=qHXmPKVh6o4AX_s6HPT&_nc_ht=scontent-frt3-2.xx&oh=00_AT8Vk2dzISA-nER0s_8tql9Q8unmEColAXkG-R_szOGbAA&oe=620F5B70
  @width 0.3
\fi

\end{itemize} % }

\iusr{Люба Заря}
\textbf{Lola Madino} а ви знаєте правду?

\iusr{Ирина Козина}
\textbf{Lola Madino} 

читала во многих мемуарах, также слышала от людей, переживших оккупацию - город
должен был быть полностью освобождён от жителей и сожжён. людей массово
угоняли, вывозили в Германию; сопротивлявшихся - уничтожали. также убивали тех,
кто Германии не был нужен: стариков, детей, больных и инвалидов. запомнилась
сцена (к сожалению, не помню ни автора, ни названия книги), в которой очевидец
и участник события, на тот момент мальчишка 14 лет, описывает, как на его
глазах вытолкали из дома семью соседей: его товарищ такого же возраста
попытался бежать - был застрелен во дворе; мать и старшую девочку увезли, а
старуху бабушку и двух младших детей завели в сарай и расстреляли за
ненадобностью. оставшиеся немногочисленные жители Киева старались прятаться,
где могли; немецкие патрули прочесывали буквально каждый дом, сарай, подвал,
развалины и вообще все; куда не могли или боялись зайти - забрасывали гранатами
или поджигали. спастись удалось очень немногим

\iusr{Ирина Козина}
\textbf{Lola Madino} 

лет 10 назад общалась с очень пожилой женщиной с Татарки, опекуншей местных
собак. как-то случайно зашёл разговор о войне, оккупации. случайно, ибо
изначально говорили о заболевшей дикой собаке, которая пряталась в Татарских
оврагах, и опекунша не могла ее там выловить. плавно разговор перешёл на сами
овраги, и женщина (насколько помню, ее звали Лена) вспомнила, как она ребенком
пряталась там в диких зарослях от немцев, чтобы не угнали в Германию. там же
пересидели они, и многие соседи, и те дни, когда немцы бежали. один из соседей,
полицай, пользуясь знанием местности, привел туда немецкий патруль, и
большинство тех, кто прятался, были переловлены или убиты. спастись удалось
всего нескольким, ей в том числе

\begin{itemize} % {
\iusr{Ольга Гудеева}
\textbf{Ирина Козина}, хуже полицаев злодеев нет!!!(((

\iusr{Алла Тихонова}
\textbf{Ирина Козина} , о полицаях много рассказывала бабушка! Ненавижу этих бандеровцев.

\iusr{Ирина Козина}
\textbf{Алла Тихонова} а нынче они стали нацгероями. от этого так мерзко...
\end{itemize} % }

\iusr{Liudmyla Fedorenko}
Полицаи - бандеровцы? Как мило, однако.

\iusr{Алла Тихонова}
\textbf{Liudmyla Fedorenko} , а Вы не знаете, из каких людей формировались эти отряды полицаев?

\iusr{Liudmyla Fedorenko}
Меня, вероятно, в отличие от Вас с господином медведчуком и ему подобных, там не было.

\iusr{Мария Каменцова}
\textbf{Lola Madino} . Присутствовали? Как моя мама?..  @igg{fbicon.wink} 

\end{itemize} % }

\iusr{Tanja Zvolinskaja}
спасибо

\iusr{Мария Иванова}
Спасибо

\iusr{Ольга Гондза}

Мой дедушка, по маминой линии из Росии освобождал Киев. Когда приезжал в Киев,
мы тогда жили в Святошино на Чернобыльской. Там оказывается проходила линия
обороны Киева. Он был военным водителем. А бабушка по папиной линии. С
детьми-моими двумя тетями и папой. Пережила оккупацию, прятала евреев. От бомбежек
потеряла практически слух и поседела совсем молодой.

\begin{itemize} % {
\iusr{Мария Каменцова}
\textbf{Ольга Гондза} . Ужасы войны наши родители, бабушки и дедушки пережили в одном городе и практически вместе!  @igg{fbicon.heart.sparkling} 

\iusr{Ольга Гондза}
\textbf{Мария Каменцова} к сожалению. Странно, то некоторые люди, до сих пор не знают или не хотят знать все ужасы жизни во время оккупации.
\end{itemize} % }

\iusr{Владислав Проноза}

И сколько тех, которые подошли с кусочком сала и парой сухариков к танку, сколько
этих врагов савецкага народа было убито и сколько было сослано в Сибирь, где
большая часть \enquote{освобождённых} умерла от голода и
холода, замучена, изнасилована, истерзана? Только за то, что они остались на
оккупированной территории. Не читали?

\begin{itemize} % {
\iusr{Александр Лесик}
\textbf{Владислав Проноза} 

а танк на плоту переправили. ибо все мосты были
взорванны. представляю себе этот плот... 26. 6 тонн выдержать.

\begin{itemize} % {
\iusr{Роман Басов}
\textbf{Александр Лесик} что вы хотели сказать, что вам не нравится?

\iusr{Николай Гребенкин}
\textbf{Александр Лесик} А понтонную переправу знаешь? Какое невежество... И туда же, комментируют.

\iusr{Мария Каменцова}
\textbf{Александр Лесик} . Танки спокойно проходили по дну реки, тщательно закупоривая щели. Это было предусмотрено в конструкции Т-34.
\end{itemize} % }

\iusr{Роман Басов}
\textbf{Владислав Проноза} думаю вашим предкам стыдно за вас, какой вы негодяй!

\begin{itemize} % {
\iusr{Владислав Проноза}
\textbf{Роман Басов} 

нет, можешь спать спокойно. Мои не убивали и не ссылали людей умирать в
концлагеря.

Прадедушка был асвабажден из немецкого концлагеря и на три года \enquote{уехал} в
савецкий концлагерь (полицаем не был), второй пулемётчиком прошёл всю войну-то,
что жив остался - чудо, их немцы убивали в первую очередь. Ни-че-го о войне не
рассказывал; по словам дедушки - напивался с друзьями на 9 мая, материл
савецкую власть; все они давно умерли.

Что ещё? Посмотри на победителей и побежденных - на Германию и россию, живущую
в 15 веке - нищета, грязь, войны как в самой россии, так и за её пределами.

\iusr{Ольга Гудеева}
\textbf{Владислав Проноза}, 

ага, \enquote{побежденная Германия} покоя не даёт!!! Нужно было не побеждать? И дать
истребить ваших предков!? Хорошо было бы? Есть одно мааааленькое \enquote{но}: вы бы не
родились, совсем!.......


\iusr{Алла Тихонова}
\textbf{Владислав Проноза} , да, негодяи были всегда.

\iusr{Роман Басов}
\textbf{Владислав Проноза} 

во первых мы с вами не на ты. Во вторых не было у вас никаких дедушек сидевших
за то что в концлагере был. Вы обычный малолетний бот, возомнивший себя
нацистом. Таких как вы много лазит по таким публикациям, стараясь насрать по
больше. Ваши дедушки живут плохо, они не получают пенсии в вашей стране?
Почему?

\iusr{Александр Лесик}
\textbf{Ольга Гудеева} вы с ботами из победобесного полка спорите .зайдите на страницу оппонента .

\iusr{Ольга Смаль}
\textbf{Роман Басов} так вы и есть бот. Что вы тут разводите:).

\iusr{Роман Басов}
\textbf{Ольга Смаль} Оля, вам сути есть что сказать? Нет, не пишите мне глупости.

\iusr{Владислав Проноза}
\textbf{Ольга Гудеева} 

не родились десятки миллионы людей, которых убили камунисты- убили
голодом, холодом, их сжигали, расстреливали, отправляли умирать в лагеря. Это были
миллионы безвинно убитых людей и миллионы не рождённых.


\iusr{Ольга Гудеева}
\textbf{Владислав Проноза}, ну вы то не родились бы по другой причине!;))

\iusr{Роман Басов}
\textbf{Владислав Проноза} 

ещё раз, мы с вами не на ты. Держите себя в руках. В лучших традициях, а нас за
что. На ваши оскорбления отвечать не буду, цель ваша понятна, и дешёвые
манипуляции оставьте при себе.


\iusr{Владислав Проноза}
\textbf{Ольга Гудеева} я родился, как и вы, но савецкая власть убила десятки миллионов людей. И десятки миллионов нерожденных.

\iusr{Ольга Гудеева}
\textbf{Владислав Проноза}, но для вас то фатальной стала бы Германия, не освободи советские люди Киев и Европу!!!

\iusr{Владислав Проноза}
\textbf{Роман Басов} 

это из разряда \enquote{вы скотина?} Написать незнакомому человеку \enquote{малолетний нацист},
бот, после чего писать \enquote{мы с вами не на ты}. У тебя проблемы, не у меня.


\iusr{Роман Басов}
\textbf{Владислав Проноза} 

смешно читать, ваши потуги, пишите ерунду и чушь вам уже здесь на тыкали в
комментах. Не тыкайте и не врите людям. А что вы осуждает нацизм на Украине?


\iusr{Роман Басов}
\textbf{Владислав Проноза} истерикой вашей удовлетворён.

\iusr{Владислав Проноза}
\textbf{Роман Басов} покойся с миром.

\iusr{Мария Каменцова}
\textbf{Роман Басов} . Он просто несчастный человек, который самоутвердается за счет гадостей, которые пишет. Это заболевание.

\iusr{Tatiana Thoene}
\textbf{Роман Басов} он прав. Многих за то, что остались в Киеве обвинили в сотрудничестве с оккупантами и сослали в лагеря.

\iusr{Роман Басов}
\textbf{Tatiana Thoene} кто сотрудничал тот и был сослан, что вас удивляет?

\iusr{Tatiana Thoene}
\textbf{Роман Басов} в том то и дело, что ссылали тех, кто НЕ сотрудничал!

\iusr{Роман Басов}
\textbf{Tatiana Thoene} всех кто был в оккупации садили в лагеря, садили ли в лагеря узников концлагерей?

\iusr{Tatiana Thoene}
\textbf{Роман Басов} садили тех, кто попал в плен на поле боя и оказался в концлагере.

\iusr{Роман Басов}
\textbf{Tatiana Thoene} мой дедушка был в плену, бежал, потом участвовал в формировании Днепра, потом освобождение Чехословакии. Награждён, значится в списках награждённых.

\iusr{Tatiana Thoene}
\textbf{Роман Басов} Спасибо вашему дедушке за победу. Хорошо, что он выжил. Может потому, что война тогда еще не закончилась и он успел заслужить почет на поле боя.

\iusr{Роман Басов}
\textbf{Tatiana Thoene} по второй линии, моя бабушка была в оккупации со всей семьёй в их большом сельском доме стоял штаб немцев, они с матерью жили в хлеву. После освобождения, к ним никто на воронке не приезжал.

\iusr{Tatiana Thoene}
Вашим бабушке и дедушке повезло. И повезло, что им не завидовали и не донесли «стукачи» придуманный донос. Другие были не такие везучие.

\iusr{Владислав Проноза}
\textbf{Роман Басов} 

роман, ты врёшь. Человек, который был в плену, должен был искупить своё
предательство смертью. Как он мог, савецкий челавек, гаварящий языком
родным, попасть в плен? Почему он не погиб, сражаясь за родину, за сталина, а
трусливо сдался в плен врагу? Стыдно это читать.

И второе - твой дедушка \enquote{формировал Днепр} - это пишет человек, родной язык для
которого российский.

Поїдь в Грозний- там в університеті дуже гарна кафедра російської мови, ти
навіть зможеш \enquote{павтаріть} героїчний вчинок дєдушкі, прогулявшись по чудовим
горам цієї прекрасної Республіки. Чечня, звичайно, це росія, не Чехословаччинна, її
звільняти не потрібно.

Почитай, обов’язково, людям Політковську, Литвиненка, і, можливо, ти приїдеш з
орденом. Можливо. Наш юний філолог. Для початку вивчи - \enquote{Чечня-это круто}. На
російськом языке. А далі будеш \enquote{співати} на чеченській мові через дві години.

Спи спокійно, рома.

Грузи бочки апельсинами.


\iusr{Роман Басов}
\textbf{Владислав Проноза} 

сала тухлого объелся? Или самогонка прокисла? Великолепная истерика, браво! А ну, жги исчо!

\iusr{Владислав Проноза}
\textbf{Роман Басов} 

романчик, спи спокойно, \enquote{формируй Днепр}.

ПС: ты в Киеве в районе Ольгино живёшь? Или ты из Бортничей питерских выполз?


\iusr{Роман Басов}
\textbf{Владислав Проноза} другого и не ожидал, таблетки не выпили с утра....? Не пишите мне вашу чушь.

\iusr{Владислав Проноза}
\textbf{Роман Басов} 

хотел бы я послушать твои рассказы на российском языке о твоём же
герое-дедушке, о его ратных подвигах во Второй мировой. Я возьму тетрадь и
ручку, выслушаю твои воспоминания о герое-дедушке, всё запишу. Мне очень
понравилась история о твоём деде. Я живу на Борщаговке, могу быть в центре через
полчаса. Ты как?

\iusr{Роман Басов}
\textbf{Владислав Проноза} 

вот и живите на своей Борщаговке, у меня нет никакого желания с вами
встречатьсяя я не светлоголубой, это вам видимо нравится встречаться с
мужчинами. Вот и встречайтесь, мне ваши психические девиации не к чему. Ещё
раз, таблеточка не забывайте.


\iusr{Владислав Проноза}
\textbf{Роман Басов} 

Откроют границы - лети в Германию, посмотришь на побеждённых.

Или в Чехию - это ведь их твой мифический дед освободил? Тебя \enquote{поблагодарят} за
тридцатилетнюю свободу, которую приволок им твой дед.

Плохо отстрачиваешь на расийском языке - отправят тебя в Сирию, где не знают
российскаго. Приедешь оттуда в 200-м виде.

Пока, Паниковский. Бросай гусей, извини, Фейсбук.

\iusr{Роман Басов}
\textbf{Владислав Проноза} таблетки или лечить электричеством.

\end{itemize} % }

\iusr{Нонна Юшкова}
\textbf{Владислав Проноза} читай дальше, читатель

\iusr{Микола Киевский}
\textbf{Владислав Проноза}:

\enquote{Не читали?} - Читал, и блогерские заметки, и их тупые пересказы. А вот
родители жены, молоденькими пережившие всю оккупацию, этого не помнят. Они
работали на стройках, разборах завалов как и многие их знакомые, долго жили на
Подоле на дебаркадере и не скулили. Может вы это о бедненьких полицаях и прочей
нечисти? Ну так и правильно! Сейчас эти упыри и их последователи, брешут и
отравляют жизнь, забивают дурь в молодые головы.

\iusr{Николай Гребенкин}
\textbf{Владислав Проноза} 

Я родился в Киеве перед освобождением. Отец в оккупации работал в Киеве на ЖД.
После освобождения к моим родителям не было претензий.

\begin{itemize} % {
\iusr{Роман Басов}
\textbf{Николай Гребенкин} не старайтесь, таким вы ничего не докажите их цель розжиг ненависти и не больше.

\iusr{Олег Евко}
\textbf{Николай Гребенкин} 

садили по совокупности: сдался в плен, вышел оттуда, работал при немцах. Ну
никого не интересовало, как выкормить троих детей в оккупации.


\iusr{Tatiana Thoene}
\textbf{Николай Гребенкин} вам повезло. А мой дед был сослан в лагеря, и пробыл там до смерти Сталина и последующей реабилитации.
\end{itemize} % }

\iusr{Татьяна Герасименко}
\textbf{Роман Басов} вот не надо корчить из себя киногероя

\begin{itemize} % {
\iusr{Роман Басов}
\textbf{Татьяна Герасименко} почему нельзя, не завидуйте.

\iusr{Татьяна Герасименко}
\textbf{Роман Басов} 

Знаете, если вьі напялите на себя костюм космонавта, вьі все равно им не
станете. ПЕРЕВИРАЯ цитату героя Л.Бьікова, тоже им не станете... а вот
фашистеньіш ординарньій со всех сторон из вас вьіглядьівает... чему тут
завидовать?!

\iusr{Роман Басов}
\textbf{Татьяна Герасименко} вы меня вашими провокациями не проймёте, мне плевать. Но истерикой вашей удовлетворён.

\iusr{Татьяна Герасименко}
\textbf{Роман Басов} 

Другой бьі спорил - я не буду! Конечно УДОВЛЕТВОРЕН, для тролля - трехрублевки
каждьій коммент єто офигенньій заработок. Для фашистеньішей какой еще заработок
может бьіть кроме єтого?!

\iusr{Роман Басов}
\textbf{Татьяна Герасименко} 

вашими оскарбления и вы лишь показываете свой уровень развития. В принципе не
способны меня чем либо задеть, или обидеть.

\iusr{Татьяна Герасименко}
\textbf{Роман Басов} 

Скорее всего, єто ВЬІ демонстрируете свой уровень недалекого, и почему-то
уверенного в своей безнаказности хама и циника .

\iusr{Роман Басов}
\textbf{Татьяна Герасименко} кто вас оскарбил, и где? К чему вы изображает из себя бота? В стиле сам дурак. Вы к чему такое пишите?

\iusr{Татьяна Герасименко}
\textbf{Роман Басов} 

Вы уже нахамили многим участникам группы. Этого более, чем достаточно. Прежде
чем гадить в группе \enquote{Киевские истории} не лучше ли вам свалить к себе на вашу
историческую родину, и там удобрять словесным навозом свои местные болота.


\iusr{Роман Басов}
\textbf{Татьяна Герасименко} 

кому конкретно, вы как всегда, не о чём кисель размазал. Болота, историческую
родину? Что у вас в голове.... Так и думал, одно и тоже. Одно враньё от вас.
Словесным навозом себе удобряйте ваши болота. Бред какой-то.


\iusr{Татьяна Герасименко}
\textbf{Роман Басов} Тролль-трехрублевка?! Много платят побирушникам-коментушникам?!

\iusr{Роман Басов}
\textbf{Татьяна Герасименко} конкретно опять ничего, вы с пронозой с вашим, прям одно лицо.
\end{itemize} % }

\iusr{Мария Каменцова}
\textbf{Владислав Проноза} . 

Киевляне, по рассказам моей мамы, после победы над фашистами в Киеве и их
изгнания испытывали такой душевный подъем, что в считанные месяцы разобрали все
огромные завалы в центре и начали разметку под новое строительство! Как-то не
согласуется с \enquote{замученными и изнасилованными}...

\begin{itemize} % {
\iusr{Владислав Проноза}
\textbf{Мария Каменцова} 

через десять лет после войны граждане побеждённой Германии, испытывая душевный
подъем, стали ведущей экономикой мира. С душевным подъемом граждане ГДР бежали в
ФРГ, и в итоге Германия стала единой страной.

Эсэсэр сдох, а россия- правоприемница этой тюрьмы,-это страна третьего мира.

Всего доброго, не болейте.

\iusr{Оксана Тернавская}
\textbf{Владислав Проноза} если Россия - страна третьего мира, просто интересно, к какому миру отнести современную Украину?

\iusr{Роман Басов}
\textbf{Владислав Проноза} кто стал ведущей экономикой? Опять несёте ложь несусветную, какой душевный подъём, назовите ВВП ФРГ на 1955г?

\iusr{Tatiana Thoene}
Оксана, причем здесь Украина? Речь ведь не об этом.
\end{itemize} % }


\end{itemize} % }

\iusr{Валентина Бабченко}
Столько лет прошло, а читать об этом невозможно без слёз.

\iusr{Nataly Popovych}
Благодарю за интересный рассказ

\iusr{Марина Бобова}
А сейчас помнят об этом в сми не сообщают

\iusr{Ирина Владиславовна Остапчук}

Яке красиве дівоче обличчя! Без татуажа, накачаних губ, наклеєних повік і
намальованих брів. Природня чудова краса.

\begin{itemize} % {
\iusr{Мария Каменцова}
\textbf{Ирина Владиславовна Остапчук} . Сасибо. Мамочке было бы приятно это услышать.  @igg{fbicon.heart.sparkling} 
\end{itemize} % }

\iusr{Лариса Кушниренко}
А потом всех, кто оставался в городе, считали предателями.

\begin{itemize} % {
\iusr{Татьяна Грицай}
\textbf{Лариса Кушниренко}, 

бред не несите. Мои многие родственники не смогли уехать в эвакуацию и пережили
оккупацию в Киеве. Бабушка, имея своих двоих малолетних детей, умудрилась
спрятать у себя, даже не подругу, а просто знакомую еврейку с ребёнком.

\begin{itemize} % {
\iusr{Лариса Кушниренко}
\textbf{Татьяна Грицай} 

Это, к сожалению, не бред. Бабушка и мама, на время уехали в деревню в
Житомирскую обл. Вернувшись, обнаружили, что их квартира на Б. Шевченко занята.
Остались без жилья. И обзывали их немецким овчарками. Но об этом даже боялись
говорить, тогда. Рассказали уже позднее.


\iusr{Татьяна Грицай}
\textbf{Лариса Кушниренко}, 

все зависит от конкретных людей. Во все времена были и дураки, и негодяи, и
подлецы, но ведь далеко не все. У меня много родственников находились в
киевской оккупации. Конечно, самые яркие воспоминания бабушки. Они жили на
Прорезной. Дворник дома стучал немцам и полицаям. Он знал, что дедушка в
Красной Армии и догадывался, что бабушка кого-то прячет у себя, буквально под
кроватью. Начал шантажировать и приставать к бабуле, так что две девчонки 26 и
27 лет с тремя детьми вынуждены были сбежать с квартиры, а потом даже ушли из
города к каким-то дальним родственникам за Таращей. 150 км по оккупированной
территории лесами и отдаленными сёлами. Ой, такая длинная история, что здесь
все не опишешь. Буквально ещё пару слов: моя бабушка и тетя Ида подружились на
всю жизнь и считали себя сёстрами.


\iusr{Ольга Агафонова}
\textbf{Татьяна Грицай} не бред, аполитика кпсс. Спасали люди.

\iusr{Лариса Кушниренко}
\textbf{Татьяна Грицай} 

Люди разные. Но политика тогда была такая. Относились к людям, пережившим
оккупацию, военнопленным, угнанным в Германию, мягко сказано, с недоверием.


\iusr{Татьяна Грицай}
\textbf{Лариса Кушниренко}, 

вот сами и пишете — «люди разные». И политика не смогла задушить человечность,
духовность и бескорыстность. Впрочем, люди не меняются со времён, когда Каин
убил Авеля и никакой технический прогресс не меняет человеческой сути. И нам,
порядочным и добрым людям, всегда нужно держаться вместе.

\iusr{Tatiana Thoene}
\textbf{Татьяна Грицай} это не бред. Моего деда отправили в лагеря. Моя бабушка растила одна трех дочерей.

\iusr{Ольга Агафонова}
\textbf{Лариса Кушниренко} жестоко
\end{itemize} % }

\iusr{Ірина Оснач}

Я уже рассказывала, как моя мама осталась в оккупации: их бросили на линии
обороны Киева, девчонки-студентки пединститута все лето там копали
противотанковые рвы. Мама и подружка днем прятались, а ночами шли.

Мама и ее семья жили в погребе, а в их доме в Броварах - немецкие офицеры.
Когда немцы отступали, а наши наступали, был страшный обстрел, их дом сгорел.

А потом маму таскали в НКВД на ночные допросы, светили в лицо лампой, угрожали,
требовали признаться в том, чего не было, что якобы сотрудничала с немцами, не
одну ее, всех так, они все (кто был в оккупации, кто был угнан в Германию, как,
например, мой свекор) считались неблагонадежными.

Были женщины, которые с немцами спали. Были полицаи.

Трамвай из Броваров в Киев с началом войны перестал работать. Бабушка
периодически посылала маму с кое-какими продуктами в Киев - к жене своего брата
Жоржа, которую он не успел вывезти с 2-мя малолетними детьми, а сам
эвакуировался спешно. Они страшно голодали. Мама одевалась в тряпье, мазала
лицо сажей, закутывалась в платок, горбилась, как старуха, и шла пешком туда и
назад - на постах иногда проверяли корзину и кое-что забирали. Кое-где на
улицах валялись или висели трупы.

Однажды была облава, ходили по всем квартирам и проверяли документы, хватали
людей, молодежь отправляли в Германию. Тетка сориентировалась быстро: втолкнула
маму в кладовку на кухне, быстро схватила молоток и гвозди, на дверь набила
гвоздями газеты, повесила какие-то черпаки и кастрюльки, придвинула тумбочку.
Хорошо, что патруль был без собаки, маму не нашли.

А потом стал работать оперный - и тетка снова стала петь в хоре, ей давали
паек.

Мама говорила, что было голодно и страшно. Но ночами она ходила в лес искать
партизанов!

Немцы были разные: жестокие - хотели застрелить, а также грустные, один
приходил к ним, показывал фото своих деток, приносил их малышам хлеб и говорил
что война - это плохо. Мама рубила дрова и загнала щепку в руку, рана
нагноилась, рука опухла, температура, некуда деваться - мама пошла в нем.
госпиталь и готовилась умереть, врач обработал рану и дал лекарство, фактически
спас ее. Много чего рассказывала... До конца жизни она говорила: так тяжело
вспоминать и забыть нельзя, вот бы так чик! - и отрезать часть памяти...

А про то, как ее отца схватили в облаве на базаре, держали в подвале на
Институтской, как колонну гнали на Сырец, как мама попрощалась с ним взглядом,
как его кум спасся, а дедушка погиб, я тут писала.

\begin{itemize} % {
\iusr{Оксана Тернавская}
\textbf{Ірина Оснач} 

эта графа была так долго только потому, что её почему-то не отменили. А
тщательно проверяли по ней в послевоенное время только лиц мужского пола,
которые во время войны были призывного возраста, но по каким-то причинам
остались на оккупированной территории.

\end{itemize} % }

\iusr{Надія Чуй}
\textbf{Лариса Кушниренко} 

так, і до 1980року при влаштуванні на роботу заповнювався Анкетний благк, де
було питання: \enquote{пребиваніє на окупированной территории вас і ваших
родственников,}, .. Ми ВСІ пристосовуємось до умов і правил життя у ВСІ часи..

\begin{itemize} % {
\iusr{Галина Тороповская}
\textbf{Надія Чуй}, я у 1984 же ту графу заповнювала. А народилася у 1967.

\iusr{Оксана Тернавская}
\textbf{Галина Тороповская} 

это уже было механически. Мы все заполняли. Мне особисты рассказывали, что
после того, как стали пенсионерами мужчины призывного возраста 41-45г.г. на неё
уже никто не обращал внимание, а отменить не отменили...

\end{itemize} % }

\iusr{Vyacheslav Sliva}

По-поводу графы, про оккупированные территории. Я тоже в свое время заполнял.
Но чему тут удивляться, если до сих пор в независимой стране, уже как 30 лет до
сих пор работают нормативные документы советского союза! Страны которой давно
нет. Это не бред?


\iusr{Мария Каменцова}
\textbf{Лариса Кушниренко} . 

Киевляне, по рассказам моей мамы, после победы над фашистами в Киеве и их
изгнания испытывали такой душевный подъем, что в считанные месяцы разобрали все
огромные завалы в центре и начали разметку под новое строительство! Как-то не
согласуется с \enquote{замученными и изнасилованными}...

\iusr{Лариса Кушниренко}
\textbf{Мария Каменцова} 

Никто, ничего не придумывает, критикует, или прославляет. Просто делимся
тем, что удалось узнать от родителей, которых, к сожалению, уже нет.


\iusr{Арт Юрковская}

Что за бред? Мои пережили оккупацию.

\end{itemize} % }

\iusr{Lidiia Kindzerskaya}
Спасибо. Очень трогательно.

\iusr{Наталья Сенниченко}

Моя мама, 1925 года рождения,была в оккупации в Киеве вместе со своими
родителями. Её брат, работавший на заводе \enquote{Арсенал}, эвакуировался вместе с
заводом, остальных не брали. Много рассказывала о времени, когда Киев был
оставлен советскими войсками. Часто были облавы ( в основном на рынках) и
молодых увозили на работу в Германию, если не было справки, что работаешь в
городе. Было трудно с продуктами, меняли вещи и ценности, у кого они были. Мама
ходила с Подола в Дарницу, там на минных полях можно было найти мёрзлый буряк
или картошку, из которых делали лепёшки или подобие супа. Когда в 1943 году
наши войска подошли к Днепру, немцы стали укреплять береговую оборону и
приказали всем жителям покинуть дома на тех улицах, которые были расположены
вдоль берега. Потом осматривали квартиры, дворы, чердаки, искали, нет ли там
диверсантов. В нашей квартире из ценностей были только книги, большая дедушкина
библиотека. Он написал небольшой плакат на русском и немецком языках
\enquote{Убедительная просьба не трогать книги}. Книги не тронули, взяли только пишущую
машинку с латинским шрифтом. Моя мама и её отец (бабушка погибла в 1942 - , но
это отдельная история), перебрались к родственникам на Лукьяновку, там в
небольшом домике собрались в основном старики и женщины с детьми. Видели, как
уходили немецкие войска. Первыми, кто постучал к ним в дом, были наши
разведчики, они спросили, есть ли в окрестностях немцы. Потом в течении многих
лет графа в анкете о пребывании в оккупации была как \enquote{чёрная метка} при
устройстве на работу или карьерном росте.

\begin{itemize} % {
\iusr{Лариса Кушниренко}
\textbf{Наталья Сенниченко} вот и мне тоже о таком рассказывали родители.

\iusr{Елена Заяц}
\textbf{Наталья Сенниченко} зато сейчас многократная судимость не является препятствием быть депутатом, президентом.

\iusr{Мария Каменцова}
\textbf{Наталья Сенниченко} . Ну не знаю, мои отец и мать пережили в Киеве всю оккупацию, мама окончила после техникум и институт, работала в министерстве легкой промышленности; папа тоже спокойно поступил и закончил институт, работал в Минавтотрансе... Пребывание в оккупации вовсе ничему не помешало!

\begin{itemize} % {
\iusr{Наталья Сенниченко}
\textbf{Мария Каменцова} 

Моя мама тоже закончила институт, правда, не медицинский, как мечтала, там у
неё документы не приняли, а биофак. Диссертацию защищать \enquote{не рекомендовали},
писала их для других, также по службе, будучи не членом партии, не особо
продвигали, хотя работник она была очень грамотный. Возможно, сыграло роль то,
что происхождение её было не из рабочих и крестьян, да и особым честолюбием она
не страдала.

\end{itemize} % }

\iusr{Yelena Barwick}
\textbf{Наталья Сенниченко} да, полностью совпадает с воспоминаниями бабушки и мамы.

\end{itemize} % }

\iusr{Марина Балаценко}
Огромное спасибо за рассказ! До слез!

\iusr{Мария Каменцова}
\textbf{Марина Балаценко} . И вам спасибо за добрые слова и неравнодушие! Я сама плакала, когда писала (вначале на компьютере)...  @igg{fbicon.frown} 

\iusr{Tamara Malezhyk}

Часть людей в силу не зависящих от них обстоятельств оставались на
оккупированных территориях, какие же Они предатели. Сплошные заблуждения,
читайте больше. Предатели — это о другом.

\begin{itemize} % {
\iusr{Арт Юрковская}

А многие, которые ненавидели советскую власть были \enquote{предателями}. Что тут
удивительного? Во время революции не все ее принимали и когда пришли немцы
начали сотрудничать. Причем это уже вторая немецкая оккупация была - первые
\enquote{предатели} гетьман Скоропадский и иже с ним. Могли остаться шпионы. Вот их
выявляли. Нормально. В Париже француженок что с немцами гуляли брили
налысо, раздевали догола и убивали даже.

\end{itemize} % }

\iusr{Леся Фандралюк}
Огромное @igg{fbicon.heart.red} спасибо за память, за историю, пока читала-и
рыдала и улыбалась и нервничала и радовалась

\iusr{Gary Sorokin}
Скажите, кто подорвал много домов на Крещатика и взорвал собор в лавре, ответ
от экскурсовода я не получил

\begin{itemize} % {
\iusr{Dima Matsarsky}
\textbf{Gary Sorokin} 

красная армия, оставляя оборону города на женщин, стариков и детей, перед тем
как предательски покинуть город, мобилизировав почти всё мужское население,
взорвала практически весь исторический центр.

\iusr{Ірина Оснач}

Тут по условиям группы нельзя давать ссылки. А я только что у себя выложила
видео - рассказ историков о том, кто взрывал Крещатик и Киево-Печерскую лавру
и почему. Можете посмотреть у меня на странице. Там говорят, что и часть
Красной армии, и диверсионная группа из Москвы с участием киевлян, и немцы.

\iusr{Мария Каменцова}
\textbf{Gary Sorokin} . 

По законам военного времени все действия, способные привести к победе над
врагом, считаются целесообразными. Так поступали ВСЕ АРМИИ.

\iusr{Tatiana Thoene}
\textbf{Gary Sorokin} красная армия
\end{itemize} % }

\iusr{Yelena Barwick}

Спасибо! Мои с лета 1943-го прятались в склепе на Байковом кладбище. Когда Киев
освободили, оказалось, что их дом был уничтожен бомбой. Вообще слушая рассказы
бабушки, не представляю, как она все пережила.

\begin{itemize} % {
\iusr{Леа Соколова}
\textbf{Yelena Barwick} 

моя родная тетя Вера и ее сын Витя Снигирь(2 годика) жили в склепе на
Байковском кладбище!

Она - жена военного прокурора, пропавшего без вести в первые дни наступления
нацистов в Киев.

Сразу же убежала и спряталась там- в склепе польского сектора!

Я забыла фамилию этого склепа, но зрительно его помню - над ним было 2 ангела!

Выжили!

После вернулась в свой дом на Большой Житомирской, 38- а он занят другими
людьми!

ььОтвоевала себе комнату и так наша квартира стала коммуналкой... Квартира,
купленная прадедом, где родились дед, где родилось 5 его детей.

Маму угнали в Германию в 12 лет. Соседка выдала её за свою умершую дочку. Мама моя
светленькая была, поверили, что украинка.

И, как украинку, увезли на работу в Германию!

Никто не выдал, низкий поклон этим людям и вечная им память @igg{fbicon.face.sleepy} 

А остальные члены моей семьи ушли навсегда в один день.

23 человека.

Бабий Яр. @igg{fbicon.face.crying.loudly} 

Вчера был День Холокоста - вспоминали и рыдали.

\iusr{Tatiana Thoene}
\textbf{Yelena Barwick} моя мама тоже там пряталась с сестрами.
\end{itemize} % }

\iusr{Lana Green}
Спасибо за бесценные воспоминания.

\iusr{Лариса Юровецкая}

А мои родные, моя прабабушка и её сын, он же брат моей бабушки, с женой, жившие
на Пушкинской, дедушка моего мужа, живший нам Подоле, не верили, рассказам про
зверства фашистов. Они, пережившие и воевавшие в Первую мировую, даже не могли
себе это представить! И не хотели эвакуироваться. Остались в Киеве. Все погибли
в Бабьем Яру!

\iusr{Микола Щербань}

Можливо, маму у 90 років трохи пам'ять підводила, що вона замість реальних
подій \enquote{назгадувала} байку в стилі сталінської пропаганди ?! Чи, може, той,
хто записував, з якоюсь метою сам нафантазував ?!

Так чи інакше, а такої дурні писати не треба, бо з брехні і неправди
починається все погане на цьому світі ! Вже достатньо написано про період
окупації і звільнення Києва та усієї країни, навіть тут, у коментарах, люди
пишуть правду. То яка мета такої публікації ??

\begin{itemize} % {
\iusr{Almira Yusupov}
\textbf{Микола Щербань} 

может некоторым и было при немцах хорошо. Мои кто не уехал в земле лежат. Не
надо говорить что человек не помнит только потому что кому то это не хочется
слушать

\begin{itemize} % {
\iusr{Микола Щербань}
\textbf{Almira Yusupov} 

лжи не хочется слушать всем п о р я д о ч н ы м людям. А остальные - ловят
рыбку в мутной воде. Лгать не нужно, даже если родственники погибли - этим
ничего не измените, только помогаете лживой рашистской пропаганде скрывать
преступления против своего народа.

P.S. взглянул на Ваш профиль - стало все ясно. Но если помните преступления
немецких нацистов, то почему не помните сталинских преступлений ?! А 6-7
миллионов выморенных голодом украинских крестьян, всего за 7-8 месяцев, с
осени 1932 до лета 1933, это как ?! А окружение сел воинскими частями , чтобы
никто не выбрался из села и все вымерли, взрослые и дети, - это как называется
?! Для сравнения: во время Холокоста, с 1933 по 1945 г., т.е. за 12 лет, во
всей Европе было уничтожено м е н ь ш е е количество людей ! При этом нужно
учитывать, что уничтожали не только евреев, но и цыган, гомосексуалистов,
психически больных. Так что, уважаемая представительница народа избранного,
думайте, о чем пишете ! Лживая рашистская пропаганда безнаказанно проходить не
будет !

\iusr{Arkady Stolyarov}
\textbf{Микола Щербань} 

а кто судья и кто назначает кто порядочный? Я вижу как вы оскарбляете старого
человека и вы ловите рыбку в мутной воде

\iusr{Almira Yusupov}
\textbf{Микола Щербань} я на ваш профиль не смотрела но с вами теперь и так ВСЕ понятно.

\iusr{Владимир Поспешилов}
\textbf{Микола Щербань} А где захоронены эти млн.крестьян?

\iusr{Анна Кучевская}
\textbf{Микола Щербань} 

выражение \enquote{избранный народ} надо всегда!!! брать в кавычки, потому что
это условно и подразумевается только этим народом.

\iusr{Михайло Наместник}
\textbf{Владимир Поспешилов} да, ты Вова, отпетый нег@дяй

\iusr{Владимир Поспешилов}
\textbf{Михайло Наместник} Типа /сам дурак/ не ответ.

\iusr{Михайло Наместник}
\textbf{Владимир Поспешилов} да, кто ж тебе отвечать собирается? Отпетый нег@дяй - это констатация факта

\iusr{Светлана Токарева}
\textbf{Микола Щербань} 

Про заморенных голодом это вы сильно!!!! А сколько погибли от голода по всей
стране, в Поволжье, например. Опять русские виноваты. Напомню, что страной
правил грузин...

\iusr{Светлана Токарева}
\textbf{Микола Щербань} 

Дурковате хавство це про тебе. Голодом ор был по ВСЕЙ СТРАНЕ!!! РУКОВОДИЛ ВСЕМ
ГРУЗИН. Эта близкая история и переворачивать её не позволю Вам и вам подобным
националистам.

Кто по национальности Каганович, так же знают все. Пора бы поумнеть к старости!

\end{itemize} % }

\iusr{Оксана Тернавская}
\textbf{Микола Щербань} 

у её мамы, которую я знала с детства, до конца жзни была прекрасная память. Тем
более, то же самое я узнала и из уст своей бабушки

\begin{itemize} % {
\iusr{Микола Щербань}
\textbf{Оксана Тернавская} 

уважаемая, главное в данном вопросе - не память, а л о ж ь. Тактику
\enquote{выжженной земли} немцы ни в Киеве и нигде в другом месте не применяли. Это
первое. Второе - несколько лет назад газета, если на ошибаюсь, \enquote{Бульвар
Гордона}, публиковала цикл рассказов \enquote{Жизнь в оккупации} - там правдиво и
довольно интересно описано все, что тогда происходило. Третье. Мои родители:
отец воевал, мать пережила оккупацию; родители жены: отец убежал с поезда,
когда вывозили на работу в Германию; когда пришли наши - воевал на фронте,
инвалид войны; мать - пережила оккупацию. Много родственников тоже жили при
немецкой оккупации. Поэтому я знаю о том времени достаточно и правдивой
информации. И еще раз повторяю: ложь, даже \enquote{неполная}, искажение фактов и
т.п. - не нужны совсем, от этого только вред.

\end{itemize} % }

\iusr{Светлана Токарева}
\textbf{Микола Щербань} Дожил до старости, а ума не нажил. Бывает, сочувствую.

\begin{itemize} % {
\iusr{Микола Щербань}
\textbf{Светлана Токарева} прояви свой ум, хамка.
Или, кроме плевка желчи, ничего больше и не умеешь ?

\iusr{Светлана Токарева}
\textbf{Микола Щербань} Надо меньше пить...

\iusr{Микола Щербань}
\textbf{Светлана Токарева} так и пей же меньше ! Пока лексикон у тебя, как у Эллочки. А то, возможно, будет шире ...

\iusr{Микола Щербань}
\textbf{Светлана Токарева} так и пей же меньше ! Может, и говорить научишься, а то лексикон, как у Эллочки ...
\end{itemize} % }

\iusr{Мария Каменцова}
\textbf{Микола Щербань} . 

А что, правда глаза колет?! Хотелось бы, чтобы было так, как вам хочется?! Так
нет, было так, как было, и вам этого не изменить и не переврать!

\begin{itemize} % {
\iusr{Микола Щербань}
\textbf{Мария Каменцова} 

это ты так примитивно врешь о \enquote{тактике выжженной земли} и еще обвиняешь меня в
каком-то \enquote{переврать} ?! Да, ботоферма Путина в этой группе слабовата ...


\iusr{Оксана Тернавская}
\textbf{Микола Щербань} и здесь без Путина не можете... Больные люди...

\iusr{Микола Щербань}
\textbf{Оксана Тернавская} 

якщо ти пішла \enquote{екмпертом} в групу, то тиражувати і підтримувати сталінську (а
тепер і путінську) брехню типу \enquote{вєроломноє нападєніе фашістской Германіі на
мірно спящій Савєцкій Саюз}, \enquote{нємци прімєнялі тактіку вижженой землі} і т.п.
не треба. Якщо бабці в старості зайнятися нічим, то для початку треба хоч трохи
знати історію - не за сталінськими штампами, а ПРАВДИВУ. Для початку можеш
почитати підбірку \enquote{Жизнь в оккупации}, публікував \enquote{Бульвар Гордона}.

\iusr{Оксана Тернавская}
\textbf{Микола Щербань} 

со слов \enquote{Бульвар Гордона} надо было и начать свой комментарий. Это так, о
брехне. Так как я никогда не опускаюсь до публичных оскорблений и не перехожу
на \enquote{ты} с незнакомыми мне людьми, могу тоько пожелать вам всего хорошего и
больше мне свои комментарии не адресовывать. Жаль только одного - сколько
молодых красивых парней полегло в ту страшную войну, чтобы такие *****, как вы,
жили сейчас...

\iusr{Оксана Тернавская}
\textbf{Микола Щербань} 

и пусть вам будет стыдно! Послушайте замечательного Василия Ланового, Артиста,
Человека, Гражданина, который ушел от нас этой ночью. Светлая память
прекрасному артисту, украинцу по национальности, который тоже пережил оккупацию
у бабушки на Украине!

\ifcmt
  ig https://i2.paste.pics/18684f7b40616c82539c99848c6ca626.png
  @width 0.3
\fi

\iusr{Любов Шкіль}
\textbf{Микола Щербань} И надо же такое сочинить??????

\iusr{Оксана Тернавская}
\textbf{Любов Шкіль} 

ага! Мне вот ЭТО особо понравилось - «сталінську (а тепер і путінську) брехню
типу \enquote{вєроломноє нападєніе фашістской Германіі на мірно спящій Савєцкій Саюз}»!
Это же надо, сколько отпрысков коллаборантов нашу землю топчут...

\iusr{Микола Щербань}
\textbf{Оксана Тернавская} 

1. Я \enquote{Ви} кажу незнайомим - починаючи з підліткового віку, та людям, котрих
поважаю. Непорядних людей, які спекулюють на темі війни, поважати нема за що.

2. За більше ніж 30 років видано масу документальної, правдивої літератури і
про підготовку до війни, і про саму війну, і про післявоєнні події та
наслідки. Тому ще раз кажу - повторювати з впертістю маніяка сталінські брехні
не треба. Тим більше, що вони звучать в унісон з кремлівським \enquote{русскім міром}.

3. Якщо вже по-іншому не можеш чи не хочеш, то не треба істерики, коли хтось
висловлює свою незгоду з брехнею, бо завжди знайдуться люди, для яких брехня -
неприйнятна !

\iusr{Любов Шкіль}
\textbf{Микола Щербань} 

Руссофобия-(фобос) страх,

- это психическое заболевание, вызванное лживой западной пропагандой. Богдан
Ступка. Незараженных этой болезнью, больше людей, вы нас не переубедите, не
старайтесь.

\iusr{Любов Шкіль}
\textbf{Оксана Тернавская} 

Мне тоже больше всего понравилось это выражение. Сразу поняла, с такими мы не
договоримся. Ну их. Надеюсь, скоро они замолчат. Все ещё надеемся.

\iusr{Микола Щербань}
\textbf{Любов Шкіль} 

агресивним представникам расєйського шовінізму давно відповідали нормальні
люди, в т.ч. і росіяни. Від себе лише додам, що брехня про \enquote{вєроломноє
нападєніє...} і про \enquote{распятих мальчіков} - одного поля ягоди. А щодо надій
расєйських шовіністів - то оптимізму тут не передбачається ...

\ifcmt
  ig https://scontent-lhr8-1.xx.fbcdn.net/v/t1.6435-9/144146455_469271380872864_174247586558087870_n.jpg?_nc_cat=103&ccb=1-5&_nc_sid=dbeb18&_nc_ohc=8wWPBkX9E0gAX-V23LV&_nc_ht=scontent-lhr8-1.xx&oh=00_AT_JLcbWBJ0floy034FDy3rueFWAzHTPzAJxAA7eGjw4WA&oe=6214D55C
  @width 0.2
\fi

\iusr{Мария Каменцова}
\textbf{Микола Щербань} . Бульвар есть бульвар, а там недалеко и ПАНЕЛЬ! :)))))))

\iusr{Мария Каменцова}
\textbf{Микола Щербань} . Вам убедительно ответили 1700 лайков от самых разных людей. Вы в удручающем меньшинстве, как ни беснуйтесь... :)))))))

\end{itemize} % }

\iusr{Чумак-Пашкова Наташа}
\textbf{Микола Щербань} она написала правду

\begin{itemize} % {
\iusr{Микола Щербань}
\textbf{Чумак-Пашкова Наташа} 

писати ((перед отступлением )) \enquote{немцы применили тактику выжженной земли - все,
что можно, взорвать - взорвали, все, что можно, вывезти - вывезли, всех,
кого можно, убить - старались убить} - брехня по усих пунктах, давно
розвінчана, викрита, висміяна; давно вже і в періодиці, і у книгах, і на ТБ все
описано і показано. Навіть тут, у коментарах, люди вказують на це і описують,
як воно було насправді. Що Сталіну для стимуляції \enquote{ярости благородной}
потрібна була така брехня, то це ясно. А от навіщо вона ніби нормальним людям
у 2021 р. - зовсім незрозуміло. Тим більше незрозуміла впертість, з якою
групка пенсіонерок відстоює право повторювати брехливу сталінську пропаганду.

\ifcmt
  ig https://scontent-lhr8-1.xx.fbcdn.net/v/t1.6435-9/144035472_469474297519239_8637395518456852092_n.jpg?_nc_cat=106&ccb=1-5&_nc_sid=dbeb18&_nc_ohc=-mTS7iHnCfcAX950TvK&_nc_ht=scontent-lhr8-1.xx&oh=00_AT_HLg1yOiggXUd7c0IkWUK2Wn3x-FwJYq1IlJ6PWSPYUA&oe=6213AED3
  @width 0.3
\fi

\end{itemize} % }

\iusr{Клавдия Плаксий}

Нажаль звинувачення в неправді це є нерозуміння суті викладеного. Тема не про
форму. Тема про те як важко було і окупації вижити. А це правда

\iusr{Микола Щербань}
\textbf{Клавдия Плаксий} 

шановна, розкриваючи тему - як важко було вижити в окупації, - можна обійтися
без шаблонів сталінської брехні; навіть небагато краще вдалося б розкрити тему.

До речі, вже 7-й рік на Донбасі люди теж змушені досить часто ховатися у
погребах, підвалах, виїжджати з районів інтенсивних боїв. А \enquote{тактику
выжженной земли} там не використовує ніхто ...

\end{itemize} % }

\iusr{Людмила Новікова}
Яке миле, янгольське личко на світлинах

\begin{itemize} % {
\iusr{Мария Каменцова}
\textbf{Людмила Новікова}. Спасибо! Маме было бы приятно это услышать!  @igg{fbicon.smile} 

\iusr{Людмила Новікова}
\textbf{Maria Kamentsova} всі ми любимо своїх мам, їхнє покоління прожило каторжне життя, на жаль

\ifcmt
  ig https://scontent-lhr8-1.xx.fbcdn.net/v/t1.6435-9/143823469_977726739710397_8651162317829158164_n.jpg?_nc_cat=110&ccb=1-5&_nc_sid=dbeb18&_nc_ohc=zrt_F0urV7EAX8nR1xY&_nc_ht=scontent-lhr8-1.xx&oh=00_AT_3vEPPUGVXcBXDVPnnZajflAI9wxf_kzTWnY3eaIcaoQ&oe=62136824
  @width 0.2
\fi

\begin{itemize} % {
\iusr{Мария Каменцова}
\textbf{Людмила Новікова} . Ваша мама просто прелестна! Чудесное лицо, смелый взгляд... Такая всего добьется, не смирится и не отступит!  @igg{fbicon.smile} 

\iusr{Людмила Новікова}
\textbf{Maria Kamentsova} дякую, за ваше добре серденько, мама викладала математику
\end{itemize} % }

\end{itemize} % }

\iusr{Валентина Снигур-Шпаковская}
Спасибо,за правду.

\iusr{Надежда Дунаева}
@igg{fbicon.cherry.blossom}  @igg{fbicon.dove} @igg{fbicon.dove}
@igg{fbicon.star}{repeat=3}
@igg{fbicon.cherry.blossom} 

\iusr{Elena Zavadskaya}
Браво!

\iusr{Михайло Наместник}
Типичный образец мифотворчества. Надеюсь, простодушного.

\begin{itemize} % {
\iusr{Елена Филиппова}
\textbf{Михайло Наместник} вы просто мерзавец

\iusr{Мария Каменцова}
\textbf{Михайло Наместник} . 

Ну конечно, немцы пришли на нашу землю, чтобы принести настоящую европейскую
культуру! Двести тысяч замученных и расстрелянных киевлян не в счет, такие
мелочи!

\iusr{Михайло Наместник}
\textbf{Мария Каменцова} не нужно передёргивать и давить на эмоции. Есть исторические факты и есть их художественные интерпретации.
\end{itemize} % }

\iusr{Олег Евко}

Спасибо автору сего \enquote{патриотического} опуса. Путин даст печеньку. А я забанил с
десяток ватников, чтобы не портить свое мнение о киевлянах.

\begin{itemize} % {
\iusr{Мария Каменцова}
\textbf{Олег Евко} . 

Ну конечно, немцы пришли на нашу землю, чтобы принести настоящую европейскую
культуру! Двести тысяч замученных и расстрелянных киевлян не в счет, такие
мелочи!

\iusr{Олег Евко}
\textbf{Maria Kamentsova} 

учите историю, а не методички Кремля. Фашистов я ненавижу не меньше вас, но
дураков и предателей ненавижу еще больше. На войне всегда ужас у мирного
населения. По моему отцу тоже стреляли, когда он пацаном на ходу воровал
продукты с вагона. Так действовали немцы, воров расстреливали на месте без
разбора. Самое интересное, что ваша бабушка ничего не сказала о нехватке
продуктов для киевлян. А это хорошо помнят все.

\iusr{Оксана Тернавская}
\textbf{Олег Евко} 

хоть я с Вами и разных взглядов, но читала Ваши опусы в группе, можно сказать,
с интересом. Но, очевидно, ошиблась. Зачем так себя ограничиваете? Две фразы
«Путин даст печеньку» и «методички Кремля» всё за Вас о Вас говорят((( Обычно,
когда патриЁтам Украины нечего сказать, они всегда вспоминают Путина или
Кремль. И да, пост автора об отдельном эпизоде из жизни её матери, а не об
оккупации Киева вцелом, поэтому Ваши слова «самое интересное, что ваша бабушка
ничего не сказала о нехватке продуктов для киевлян» выглядят глупо и
абсурдно...

\end{itemize} % }

\iusr{Нина Гранковская Коваль}

Мои родные во время войны были в Киева. Рассказывал, что были облавы и всех,
кого поймают, отправляли на запад или работать на хозяина, или в концлагеря.
Родственницу мужа молодую девушку вот так угнали в Италию рабатать на хозяина,
а когда освободили тот городок где она работала, то она в Киев пешком пришла
через пол Европы. Но здесь её не очень радостно встретила власть: сказали, что
ей нельзя жить в Киеве и отправили в Крым. Так она там и прожила всю оставшуюся
жизнь. Мать и сестра в Киеве, а она в Крыму. Те мои родственники, которые при
немцах жили в Киеве очень голодали, все что можно было : постельное белье,
посуду, одежду носили в села менять на продукты. Немцы запрещали это делать, но
они рисковали (могли на месте расстрелять), чтобы не умереть от голода. Дров не
было - жгли мебель. Я родилась в 1946г и помню, что у нас был 1 пододеяльник и
1 простыня и наволочки по одной на каждую подушку. Стирали утром и до конца дня
их нужно было высушить. Очень тяжёлое было время и во время войны, как и
сейчас, люди вели в себя по разному: кто-то помогал, а кто-то немцам выдавал.
Мне мама рассказывала.

\begin{itemize} % {
\iusr{Люда Бабич}
\textbf{Нина Гранковская Коваль} 

Мама рассказывала что днём простынь использовали вместо занавески на окно -
чтобы прохожие не заглядывали в окна. Всю оккупацию она с старшей сестрой и
родителями прожили на Новогоспитальной улице в частном доме, напротив
хлебзавода который уже не существует.

\iusr{Natalia Voytulevich}
\textbf{Нина Гранковская Коваль} 

Моя бабушка, киевлянка, рассказывала, что во время облавы её забрали в чём была
затолкали в вагон и увезли на работу в Германию. И оооченнь сложно потом, после
возвращения на Родину, было вернуться в Киев. Она тяжёлым трудом на стройке
заработала право опять жить в родном городе... Такая у нас история. Очень жаль,
что некоторые злобствуют, вместо того чтобы написать свою...

\end{itemize} % }

\iusr{Nina Volynska}

Пожежу на Хрещатику перед появою німців влаштували московити і перерізали
шланги, які кияни провели з Дніпра, щоб врятувати Хрещатик, АЛЕ.... Успенський
собор, а раніше і Михайлівський знищили московити. Взагалі в Києві було знищено
46 церков

\begin{itemize} % {
\iusr{Sokolova Oksana}
\textbf{Nina Volynska} 

з Успенським не все так просто.

Зараз знайшли посилання, що можливо його німці підірвали.

Це слова робітників Печерського заповіднику.

Але про Хрещатик, Михайлівський та десятки церков то страшна правда, як
знищували наше місто(

\iusr{Оксана Тернавская}
\textbf{Nina Volynska} 

ну да, в Киеве ведь и жили раньше только \enquote{московити}, украинцы только после
1990 года появились. Так по Вашей логике? И в госуправлении СССР никогда не
было украинцев, не так ли? И в компартии они не состояли? И на местах - в
областях и районах тоже одни \enquote{московити} руководили? Не надоело манипулировать?

\begin{itemize} % {
\iusr{Зита Грицюнайте}
\textbf{Оксана Тернавская} 

читаю и диву даюсь, получается все сделали московиты, а где были украинцы
интересно, жалко что мы начинаем историю и ту неправильно трактовать.

\iusr{Екатерина Ковалёва}
\textbf{Зита Грицюнайте} 

там что то с головой...! Вот такие и начинают киевлянам рассказывать где какая
улица тут неправильно называется

\iusr{Светлана Ковтун}
\textbf{Оксана Тернавская} 

ну если Вам так больше нравиться, то Советский режим совершил эти преступления
и в Киеве и во многих других городах. Ну а Москва официально позицирует себя
приемником СССР. @igg{fbicon.face.eyebrow.raised} 

\iusr{Оксана Тернавская}
\textbf{Светлана Ковтун} 

а Украина - преемник УССР (про прИемник насмешили). Если Вам так нравитЬся. А
если не нравитЬся, тогда отдайте \enquote{московитам} всё, что при них здесь создали и
построили. Первый президент Украины Кравчук, идеолог КПУ, тоже московит???

\iusr{Светлана Ковтун}

В этом случае московиты должны раздать все, что путем рабского труда построили
граждане всех национальностей СССР. Да, и про интелектуальную собственность не
забыть, а то у нас все русские открывали, другий наций умных в СССР не было.. Я не
хочу спорить, для Вас СССР это верх блаженства, для меня тяжелая, кровавая эра в
истории моего и других народов СССР.

\iusr{Зита Грицюнайте}
\textbf{Екатерина Ковалёва} 

меня интересует другое, за 30 лет независимость мы даже приличную буду не
построили, всеми благами, а именно, заводами фабриками, большими магазинами ЦУМ, 
универмаг Украина все построена при СССР, а сколько продали, черноморском
порозодство Кравчук продал англичанам и все нам плохо, давайте строить свое,
все что было построено снести и строить свое лучшее, так что слабо, кишка
тонка.


\iusr{Зита Грицюнайте}
Черноморское пороходство

\iusr{Инна Игоревна}
\textbf{Светлана Ковтун} 

так уже все отдали Украине, еще 30 лет назад, только сами же украинцы забрали у
других украинцев. Предъявляейте к ним притензии. 30 лет назад все заводы и тд
были народа Украины, а сейчас олигархов Украины. Требуйте у наших олигархов все
вернуть государству. Все в Ваших руках, кто Вам запрещает

\iusr{Светлана Ковтун}
\textbf{Зита Грицюнайте} 

При бесплатной рабочей силе строить можно сколько хочешь и что хочешь, а вот
если рабочим денежку платить много не настроишся. ну а вместо одного ЦУМа и
Украины в Киеве куча торговых центров, а сколько жилья построено, не спорю что
все в состоянии его преобрести, но покупают и молодые и не очень

\iusr{Светлана Ковтун}
\textbf{Инна Игоревна} 

ну так нас всех граждан СССР сделали, что требовать, все олигархи это бывшие
высокопоставленные деятели СССР или их дети. Все вклады в советский банк куда
делись?? или это только виртуальные деньги были???, это я прo \enquote{прекрасный} СССР
говорю.

\iusr{Зита Грицюнайте}
\textbf{Светлана Ковтун} 

тогда все получали жилье безплатно, и хрущевки и гостинки, знаю много людей и
дети уже выросли, а люди живут на съёмных квартирах, конечно есть плюсы и
минусы, тогда и сейчас.

\iusr{Светлана Ковтун}
\textbf{Зита Грицюнайте} 

если вы считаете, что работать без права увольться, что бы не потерять очередь
на жилье или место в общежити на предприятии как минимум лет10, это бесплатно то
я так не считаю, и кстати, поскольку я родилась в Киеве, и у нас была жилплощадь
мы даже на кооперативную очередь втать не могли, метры не позволяли. Насчет с
ьемных квартир открою вам тайну - на Западе у многих тоже нет собственного
жилья, так удобней, особенно молодым, главное что бы были деньги нс оренду этого
жилья.

\iusr{Зита Грицюнайте}
\textbf{Светлана Ковтун} за то сейчас можете купить все что вы хотите.

\iusr{Светлана Ковтун}
\textbf{Зита Грицюнайте} 

во всяком случае больше чем тогда. За мебелью, бытовой
техникой, одеждой, продуктам, машинами, жильем в очередях ни кто не стоит и не
отмечется. Толку что раньшк был типо стабильный зароботок, что тогда можно было
купить а свободнлй продажк???, ракеты в космос запускали, весь мир пугали, а люди
в той стране не могли приличных трусов купить и носки,, колготки штопали @igg{fbicon.anger} 

\iusr{Зита Грицюнайте}
\textbf{Светлана Ковтун} 

все можно было купить, и на л. украинки был подарочный магазин, и на юности
салон молодожёны х и универмаг, правда ездила на родину Литву мы ездили в Ригу
,вещи покупали, конечно сейчас мне так же лучше нравится, что хочешь то и
покупай, только денег мало, я и говорю везде есть свои плюсы и минусы, лично
однозначно не могу сказать, когда было лучше или хуже.

\iusr{Светлана Ковтун}

Зита мы тоже все покупали, благодаря знакомствам в торговле, мамына подруга была
заведущей отдела в \enquote{Укарине}, у тети мнтго подружек в продтоварных магазинах
работали, свекровь моя в продовольственном магазине работала, но думаю сейчас
лучше, хотя те годы вспоминаю часто ведь это детство и юность. разве для кого то
могут быть плохими детво и юность.???. Грицюнайте

\iusr{Зита Грицюнайте}
\textbf{Светлана Ковтун} это моя девичья фамилия, так нужно, фамилия по мужу другом устройстве.

\iusr{Sokolova Oksana}
Ось одні з перших посилань у школі на цю тему.
В архівах Лаври є списки цінностей вилучених.

\iusr{Инна Игоревна}
\textbf{Светлана Ковтун} олигархи деятели не из СССР, а Украинской ССР. Они же ни куда не уехали, а остались здесь у нас в Украине, бывшей УССР

\iusr{Светлана Ковтун}
\textbf{Зита Грицюнайте}
У меня нет притензий в Вашей фамили  @igg{fbicon.thinking.face}  @igg{fbicon.face.smiling.eyes.smiling}{repeat=2} у меня тоже фамилия мужа☺☺☺

\end{itemize} % }

\iusr{Marina Chernaya}
\textbf{Nina Volynska} 

в 1941 году \enquote{46 церков} в Киеве не подрывали, об Успенском соборе - споры до сих
пор ( читайте все версии, а то я подумаю, что вы историк и исследовали этот
вопрос, так категоричны), центр Киева и мосты подрывали советские диверсионные
группы с целью ликвидации штабов гитлеровцев и не дать им дальше продвигаться
по нашей земле... но речь в посте не о 1941 годе, читайте внимательно! А сколько
культурных ценностей вывезли немецкие оккупанты из Киева - изучите вопрос.

\begin{itemize} % {
\iusr{Sokolova Oksana}
\textbf{Marina Chernaya} 

а сколько из той же Лавры вывезенного в 30-40 годы)

Там немцам достались крупицы.

А вывезенное так и не вернули русские за столько лет

\iusr{Marina Chernaya}
\textbf{Sokolova Oksana} 

какие документы об этом свидетельствуют? В эти годы Лавра работала как музейный
заповедник, я ещё в 80- е годы посещала Лавру как музей.

\iusr{Sokolova Oksana}
\textbf{Marina Chernaya} 

я упомянула Лавру в контексте музея, который и сейчас там находиться.

Есть опись изъятых предметов.

Так же огромная часть Золотой кладовой Эрмитажа и Грановитой палаты это вывезенные из Украины ценности.
Как и мозаики и фрески снятые и вывезенные из Михайловского собора.
Вернули пару штук, все остальное не хотят отдавать

\iusr{Marina Chernaya}
\textbf{Sokolova Oksana} 

что именно из Лавры? А каким боком здесь Михайловский собор, снесённый в 30-е
годы? Фрески, действительно сняли и разместили в других соборах, часть вывезли
в Москву и Ленинград( тогда это была одна страна- СССР), во время войны немцы
украли оставшиеся фрески и вывезли, СССР их вернули.

\iusr{Sokolova Oksana}
\textbf{Marina Chernaya} 

\href{www.radiosvoboda.org/amp/30690956.html}{%
Частину коштовностей, які належать Україні, було вивезено до Росії – видання (огляд преси), %
radiosvoboda.org, 26.06.2020%
}

\url{https://uk.m.wikipedia.org/wiki/Українські_цінності_за_кордоном}

\iusr{Marina Chernaya}
\textbf{Sokolova Oksana} 

Ещё раз повторяю: мы жили в одной стране - СССР, Москва была столицей этой
страны, там, как и во всём мире, находились главные музеи. А в Лувре откуда
экспонаты?

\iusr{Оксана Тернавская}
\textbf{Sokolova Oksana} 

тогда по Вашей логике \enquote{русские} просто обязаны забрать все фрески Владимирского
Собора, Кирилловской церкви, потому что это творения великих русских художников
Васнецова и Врубеля? А Вы не забыли, что до 17-го года прошлого века это тоже
была ОДНА страна?

\iusr{Sokolova Oksana}
\textbf{Оксана Тернавская} это не моя логика. Для снятия фресок они уничтожили древний собор. Варварски.

\iusr{Sokolova Oksana}
\textbf{Marina Chernaya} это не оправдывает изъятия ценностей у других музеев.
А то это поведение напоминает Гитлера и его желание устроить самый великий музей.

\iusr{Marina Chernaya}
\textbf{Sokolova Oksana} 

изъятие - это как? Насильно, а люди сопротивлялись? В
венские музеи, парижские тоже производили изъятие? Это была ОДНА СТРАНА!


\iusr{Marina Chernaya}
\textbf{Sokolova Oksana} Михайловский собор снесли ради строительства официальных учреждений, не ради фресок.

\iusr{Sokolova Oksana}
\textbf{Marina Chernaya} 

нет люди были в восторге от этого)

И если вы даже 5 раз с большой буквы напишете что это была одна страна, то это
не поменяет факта.

Венский музей после принятия закона про реституцию отдавал забранные вещи.

Не скажу что добровольно, но пришлось.

А вот количество ценностей забранных в немецких музеях и спрятанного в фондах
росийских музеев.

Ну не хотят они поступать по закону, и не только в отношении картин.

\iusr{Marina Chernaya}
\textbf{Sokolova Oksana} 

у них были артефакты из ДРУГИХ СТРАН, краденные или вывезенные незаконно... Я
имею в виду собирание со всей страны артефактов для экспозиции в крупнейших
музеях, которые посещают туристы и любители искусства, студенты и специалисты.

\iusr{Sokolova Oksana}
\textbf{Marina Chernaya} 

ну да, коллекцию которую повезли в Москву на временную выставку и там оставили
это конечно же законно. Это ж одна страна.

\iusr{Marina Chernaya}
\textbf{Sokolova Oksana} какую? когда?

\iusr{Оксана Тернавская}
\textbf{Sokolova Oksana} кто ОНИ???

\iusr{Оксана Тернавская}
\textbf{Sokolova Oksana} реституцию со времени Российской Империи предлагаете начать?

\end{itemize} % }

\iusr{Светлана Токарева}
\textbf{Nina Volynska} Содя по посту вы алкоголик? Надо меньше пить. Душ, огурчик, рассольчик...

\iusr{Екатерина Ковалёва}
\textbf{Nina Volynska} чето Вы, явно не об освобождении Киева коментируете.....

\iusr{Мария Каменцова}
\textbf{Nina Volynska} . Страшной длины шланги были у \enquote{киевлян}, четыре километра! :)))

\iusr{Лена Лагута}

Давайте уже тогда поговорим и о вывозе культурных ценностей за последние 30
лет, о разграблении запасников... Ющенко, например...и другими...

\end{itemize} % }

\iusr{Светлана Патюк}

Дякую за таку чудову розповідь і Вам і мамі... бо дійсно, в той час не було
«московитів», «хохлів», «кацапів». Було спільно горе й незалежно від нації на
національності всі боронили спільну країну й робили спільне завдання -
наближення перемоги та звільнення власноі землі від фашизму.


\iusr{Ищук Валентин}

Мій дід, Ксенофонт Тодосійович Журавель, мешканець Татарки, робітник Київського
кабельного заводу після визволення Києва.

\ifcmt
  ig https://scontent-lhr8-2.xx.fbcdn.net/v/t1.6435-9/143633908_1124474351351477_1603724750354427352_n.jpg?_nc_cat=101&ccb=1-5&_nc_sid=dbeb18&_nc_ohc=yW0tDKj4zFQAX_jPdg5&_nc_ht=scontent-lhr8-2.xx&oh=00_AT8jGpzejaw-YyoMKHR3S3XIJAlIAlYPYl5XLz60Erobqg&oe=62135850
  @width 0.2
\fi

\begin{itemize} % {
\iusr{Мария Каменцова}
\textbf{Ищук Валентин} . Наша семья тоже жила совсем близко от Кабельного завода! Получается - соседи!  @igg{fbicon.heart.sparkling} 

\iusr{Ищук Валентин}
\textbf{Maria Kamentsova} Спасибо. Получается, что дед освободил Киев для нас и я очень благодарен ему за это.

\iusr{Ищук Валентин}

В 1982 году дед провел экскурсию для меня, 10-летнего паренька, по местам своей
молодой жизни: Татарка, Подол. Жил он на ул. Татарской, номер дома не помню, с
семьей брата Емельяна и его женой Клеопатрой, ох и имена ж тогда давали. Очень
хорошо я запомнил дом по адресу Боггоутовская, 16 (напротив Фуршета), что-то
там у деда было связано с ним.

\iusr{Ищук Валентин}

Дед рассказывал, что Татарка не сильно была разрушена как центр Киева.
Кабельный завод был эвакуирован на Урал и на его месте было пусто. Фотография
сделана там же, где-то в нынешнем месте Фуршета.

\iusr{Ищук Валентин}

Очень деду понравилось в музее ВОВ, но не монумент, а пушки на спуске - 76 мм
ЗИС-3. Он воевал на такой пушке, просто оторвать нельзя было, все крутил
наводки и подавал сам себе команды.

\end{itemize} % }

\iusr{Ольга Лубягина}
Правда жизни. Читаю со слезами

\iusr{Ирина Писаренко}
Моя мама, тоже пережила похожую историю.
Несколько женщин с детьми прятались на чердаке и когда немцы подошли, бабушка сказала, молитесь
Все, даже совсем маленькие молчали и молились.
И так случилось-в этот дом не зашли

\iusr{Владимир Полищук}

Розповідь, як розповідь. По своєму цікава. Правда і спогади речі різні. Навіть
щоденники Хорошунової, яка пережила окупацію, документи по суті, суб'єктивні.
Хоча вона була вірною комсомолкою-комуністкою, і в щоденниках багато \enquote{наші,
Актябрь, Ленін} і тд... Видали їх після розвалу Союзу. І видали не тому, що
німці там хороші.... А тому що вона чесно жахається якими гнидами були люди.
Рівноцінно і червоні і коричневі. А одинокий танк.... Це гарна, символічна
розповідь. Я такуж подібну читав. Тільки там курінь УПА перебиа енкаведистів і
селяни плачучи стрічали бійців.... Все йде по колу.

\begin{itemize} % {
\iusr{Marina Chernaya}
\textbf{Владимир Полищук} А УПА здесь к чему? Они, что, Киев освобождали?

\iusr{Владимир Полищук}
\textbf{Marina Chernaya} 

це просте порівняння. Дам інше - кияни плакали і раділи, пригощали німецьких
саперів які не дали відступаючій ркака підірвати пів Хрещатику. Факт. Так
краще?

\iusr{Marina Chernaya}
\textbf{Владимир Полищук} 

коллаборанты всегда врага хлебом-солью встречают...это было и в других городах,
только это подавляющее меньшинство, которые обижены были на Советы, даже евреи
Киева радовались, хотели предпринимательством заниматься. А что было потом, вы
знаете. \enquote{Новый порядок} оказался очень страшным и поднялась вся страна...

\iusr{Владимир Полищук}
\textbf{Marina Chernaya} 

да... В 32-33 меншість теж здавала, де більшість хліб ховала.... А у 17-21 теж
стрічала червоних хлібом та сіллю. Все іде по колу. Не будемо сперечатися. Для
мене Безручко наприклад герой, для вас Ватутін. Кожен при своєму.

\end{itemize} % }

\iusr{Евгения Верченко}
И я плАчу...

Мне очень жаль, что в своё время не хватило ума и времени послушать мою бабушку
Марию Георгиевну Брухно - светлая ей память... Они с дедом не любили вспоминать
то страшное время... Радовались всему, что имели после войны... Комнатку в
коммуналке где жили ещё четыре семьи, на Коминтерна, а в конце 69 года их
разселили и дали малюсенькую квартирку на Борщаговке. Но как же они искренне
радовались и отдельной ванне и отдельному туалету и маленькой кухоньке, где
можно было посидеть и попить чай, глядя в окно... Удивительные были люди,
только я этого тогда не знала...

Светлой и доброй им всем памяти...

\iusr{Наталья Иванова}
Спасибо огромное Вам за рассказ!!!

\iusr{Алла Рублюк}
Самое ценное - это воспоминания очевидцев. Спасибо, что поделились. Интересно читать. Прекрасный слог.

\iusr{Мария Каменцова}
\textbf{Алла Рублюк} . 

Спасибо, Алла! Я вообще-то писательница, сюда пригласили, я с удовольствием
делюсь тем, что имею! Драгоценны мнения всех, кто ответил, даже те, с которыми
я не согласна!  @igg{fbicon.smile} 

\iusr{Violeta Tetiana Stetsenko}

Был приказ немцев всем жителям покинуть город, патрули ходили по квартирам и
предупреждали, говорили, что Киев будет вторым Сталинградом. Киевляне уходили
по Житомирскому шоссе, был даже отправлен поезд, который остановили уже где-то
на Пост Волынском. Некоторые люди сразу вернулись, когда немцы отступили, как
моя соседка, а некоторые дошли до Польши. Так что, не сами уходили, а их
выганяли. Уходить не хотели, мамина подруга со своей мамой, пряталась в
Мышеловке. Рассказывала, как они ночью вернулись в свою квартиру за какими то
вещами, как они шли по абсолютно пустому городу на Владимирскую, рядом с
Оперным, и как их шаги раздавались в абсолютной тишине.

\iusr{Mikhailo Sherstiuk}

Когда моя бабушка вернулась в Киев, доблестная красная армия подготовилась
вывозить мебель из квартиры.

\begin{itemize} % {
\iusr{Rimma Turovskaya}
\textbf{Mikhailo Sherstiuk} Что прямо вся Красная армия из вашей квартиры? А кто ж тогда на фронте воевал, чтобы бабушка могла вернуться в Киев?

\iusr{Mikhailo Sherstiuk}
не вся, не волнуйтесь так как фрау в Берлине.
\end{itemize} % }

\iusr{Mikhailo Sherstiuk}

Автор немного врет, немцы как раз пытались спасти жизни местного населения,
естественно своим нацистским бесчеловечным способом, кто не хочет выжить того
расстреляем, но то такэ... - Осенью 1943 года Киев был пустым. По приказу
немецкого коменданта Киева еще 26 сентября киевлянам было приказано покинуть
город, советские войска уже были в Броварах. Всех жителей выселили из той части
столицы, которая входила в боевую зону — три километра от Днепра. Граница этой
зоны проходила по ул. Саксаганского и Дмитриевской. Дома стояли пустыми на
Подоле и Печерске. Эти кварталы патрулировали фашисты и расстреливали на месте
любого, кто окажется в этой зоне.

\end{itemize} % }
