% vim: keymap=russian-jcukenwin
%%beginhead 
 
%%file 12_02_2022.stz.news.ua.strana.1.vojna_nachnetsja_15_fevralja.4.one
%%parent 12_02_2022.stz.news.ua.strana.1.vojna_nachnetsja_15_fevralja
 
%%url 
 
%%author_id 
%%date 
 
%%tags 
%%title 
 
%%endhead 

\subsubsection{1. Подготовка к большому компромиссу}

Выше уже говорилось, что новый виток истерии уже становится прологом для новых
контактов между Вашингтоном и Москвой. Причем на самом высшем уровне.

И, если после подобного накала страстей Байден с Путиным подпишут некий
прорывной документ, никто не станет записывать в пассив Белому дому те уступки
Кремлю, которые могут в нем оказаться. 

Еще от начала истории с \enquote{вторжением} появилась гипотеза, что это -
маскировка переговоров Запада с Россией по неким стратегическим вопросам. Чтобы
по итогу подать финальные договоренности как спасение Европы от войны. И не
дать при этом республиканцам навесить на Байдена ярлык \enquote{агента Кремля}. 

Какие это договоренности - информации немного. Но, судя по предложениям,
которые поступили от США в Москву, речь может идти о гарантиях того, что
Украина не получит стратегических вооружений и туда не будут вводиться боевые
единицы НАТО. Еще одна точка соприкосновения - снижение ядерной угрозы в
Европе.

Впрочем, пока непонятно, как Москва на самом деле к этим предложениям
относится. Внешне риторика Кремля достаточно жесткая: там говорят, что ключевое
требование - не расширение НАТО на восток должно быть выполнено (но от этого
отказывается публично Запад). Но какие варианты прорабатываются в негласном
режиме - это вопрос. По крайней мере такие проработки идут - судя по десанту
западных политиков к Путину и его министрам. 

Только сегодня в Москву прилетал министр обороны Британии Уоллес, а вчера там
побывала глава МИД Лиз Трасс. До этого Кремль посещал президент Франции
Эммануэль Макрон. То есть в такая интенсивная дипломатия вряд ли может
возникнуть на пустом месте, без каких-то реальных проектов, которые лежат на
переговорном столе. 

Но это, скажем так, оптимистический сценарий. Хотя и вполне вероятный - потому
что наиболее простой в исполнении. 

