% vim: keymap=russian-jcukenwin
%%beginhead 
 
%%file 27_02_2017.stz.news.ua.mrpl_city.1.istoria_kolonia_pri_zavode_nikopol
%%parent 27_02_2017
 
%%url https://mrpl.city/blogs/view/koloniya-pri-zavode-nikopol
 
%%author_id burov_sergij.mariupol,news.ua.mrpl_city
%%date 
 
%%tags 
%%title История: колония при заводе "Никополь"
 
%%endhead 
 
\subsection{История: колония при заводе \enquote{Никополь}}
\label{sec:27_02_2017.stz.news.ua.mrpl_city.1.istoria_kolonia_pri_zavode_nikopol}
 
\Purl{https://mrpl.city/blogs/view/koloniya-pri-zavode-nikopol}
\ifcmt
 author_begin
   author_id burov_sergij.mariupol,news.ua.mrpl_city
 author_end
\fi

\ii{27_02_2017.stz.news.ua.mrpl_city.1.istoria_kolonia_pri_zavode_nikopol.pic.1}

Колония при заводе \enquote{Никополь} - так для краткости в прежние времена называли
поселок, построенный Никополь-Мариупольским Горным и Металлургическим Обществом
для служащих и рабочих металлургического завода, который был возведен близ
станции Сартана Екатерининской железной дороги.

Историк предприятия - доцент Ждановского металлургического института (ныне –
Приазовский государственный технический университет. – Авт.) Дмитрий Николаевич
Грушевский в своем труде \enquote{Имени Ильича} цитирует документ, опубликованный в
газете \enquote{Приазовский рабочий} № 123 от 28 мая 1941 г. \enquote{За исключением казарм и
отдельных домиков, вмещающих в себя около 4000 человек, для остальных 3757
рабочих сколочены ша­лаши из тонких досок, без помоста. В этих шалашах люди
живут летом и зимой. Затоки дождевой воды образуют внутри шалашей
разлагающуюся грязь. Нары, где без разделения помещаются мужчины и женщины,
усыпаны насекомыми. Бань на обоих заводах не имеется. Все это низводит жизнь
рабочих к уровню жизни животных}. Так описали представители департамента
полиции, обследовавшие в конце 1899 года жилищные условия рабочих
мариупольских заводов. И более ни слова о колонии. Автора, приведшего
пространную цитату, можно понять, во время написания и выхода в свет книги
(1966 г.) не принято было писать что-нибудь положительное о жизни пролетариев в
царское время. Между прочим, условия жизни строителей предприятий первых
пятилеток в советской стране были нисколько не лучше, чем у  рабочих
мариупольских заводов. Но это так, к слову.

\ii{27_02_2017.stz.news.ua.mrpl_city.1.istoria_kolonia_pri_zavode_nikopol.pic.2}

Что же представлял собой поселок при заводе \enquote{Никополь}? На \enquote{Плане
Никополь-Мариупольского завода}, начерченном не позже 1901 года (в 1901 г. при
заводе освящен храм во имя св. Петра и Павла, на плане  его нет), помимо цехов,
вспомогательных сооружений, сети железнодорожных путей, отражена и жилая зона.
К тому времени она состояла из особняка директора завода, двух коттеджей
помощников директора, четырех  - старших служащих, шести  – начальников служб,
а также более 80 жилых строений для рядовых служащих и рабочих. Они состояли из
одноэтажных домов, состоящих из одной комнаты и кухни, из двух комнат и кухни,
из трех комнат и кухни, четырех комнат и кухни, а также двухэтажных домов по 16
квартир, каждая из которых имела одну комнату и кухню. Имелось также шесть
казарм для холостяков по два отделения на 20 человек в каждом. Помимо этого, на
плане отмечены места для будущего строительства более 90 домов. На плане
приведены условные обозначения домов по количеству комнат, но, к сожалению,
качество имеющейся копии плана не позволяет сделать распределение одноэтажных
домов по количеству комнат. Что касается двухэтажных домов, то их было 10.

\ii{27_02_2017.stz.news.ua.mrpl_city.1.istoria_kolonia_pri_zavode_nikopol.pic.3}

При всех одноэтажных домах были предусмотрены небольшие садики, а для всеобщего
пользования – сквер, который, увы, в наши дни находится в крайне запущенном
состоянии. Кроме того, в колонии были школа, больница на 30 коек, работали там
врач, два фельдшера, медсестра и провизор,  рядом с больницей обозначено место
для постройки отдельного корпуса для  заразных больных. Были также общественная
баня, гостиница, колодец, из которого жители близлежащих домов могли брать
воду, почтовая контора, - мы бы назвали это учреждение отделением связи. Раз
уже зашел разговор о почте, то стоит, наверное, сказать, что адрес поселка был
станция Сартана Екатерининской железной дороги, колония завода \enquote{Никополь}.
Только 1 июля 1927 года поселки городского типа заводов \enquote{А} и \enquote{Б} имени Ильича
вошли в черту города Мариуполя.

На план нанесены водовод и канализация (\enquote{водослив}, как названо в описываемом
документе). Назначение их понятно – обслуживание цехов и служб завода. Забор
воды осуществлялся из реки Кальмиус (там была построена насосная станция –
\enquote{водокачка}), сброс стоков производился ниже по течению от насосной станции.
Были ли в то время обеспечены дома в поселке централизованными водоснабжением и
канализацией, - неизвестно. Правда, их трассы проходят подозрительно близко от
домов, где проживала  \enquote{верхушка} завода. 

Поскольку здесь затронута социальная сфера обслуживания  рабочих завода
\enquote{Никополь}, коснемся такой темы, как их страхование. При заводе действовала
больничная касса, созданная при заводе Никополь-Мариуполь\hyp{}ского Горного и
Металлургического Общества задолго до принятия закона Российской империи от 23
июня 1912 года (\enquote{Об обеспечении рабочих на случай болезни}. – Авт.). В 1897
году была устроена амбулатория, развернутая впоследствии по тем временам во
вполне современную больницу, где оказывали медицинскую помощь рабочим
предприятия и их иждивенцам. С принятием закона, о котором  здесь идет речь,
это лечебное учреждение вошло в сферу деятельности больничной кассы при заводе.
Пока не удалось найти документов, которые бы свидетельствовали об эффективности
работы больничной кассы при заводе \enquote{Никополь}, кроме одного - это удостоверение
больничной кассы с таким текстом: \enquote{Сим удостоверяется, что рабочий
Никополь-Мариупольского завода Литвиненко Петр отправляется для климатического
лечения в село Кременное Купянского уезда Харьковской губернии}. Приходилось
слышать, что в здравнице Кременного поправляли свое здоровье работники завода
имени Ильича даже в послевоенное время. Так ли это?

В завершение нужно отметить, что в предреволюционные годы в колонии завода
\enquote{Никополь} действовала православная церковь во имя св. Петра и Павла, о ней
вспоминалось выше, а также церковь евангелистско-лютеранская, работал частный
кинотеатр, в 1905 году  был открыт Народный дом с библиотекой при заводе
\enquote{Никополь} – первое культурно-просветительское учреждение в нашем городе для
рабочих. Знаменитый мариупольский фотомастер Анисим Михайлович Стояновский
открыл при заводе филиал своего ателье. 
