% vim: keymap=russian-jcukenwin
%%beginhead 
 
%%file 16_04_2021.fb.makarenko_oksana.1.gumilev_nikolaj
%%parent 16_04_2021
 
%%url https://www.facebook.com/permalink.php?story_fbid=2901072030178100&id=100008259933050
 
%%author 
%%author_id 
%%author_url 
 
%%tags 
%%title 
 
%%endhead 
\subsection{Вчора був день народження у Ніколая Гумільова}
\label{sec:16_04_2021.fb.makarenko_oksana.1.gumilev_nikolaj}
\Purl{https://www.facebook.com/permalink.php?story_fbid=2901072030178100&id=100008259933050}

Вчора був день народження у Ніколая Гумільова. Одного з найулюбленіших поетів,
ще з часів юності в театральній школі. Мої ніжні Арлекіни Срібної доби -
Гумільов, Сєвєрянін, Блок... Ніщо не змусить перестати вас любити.

Один з найкращих віршів Гумільова.
*   *   *   *
На полярных морях и на южных,
По изгибам зеленых зыбей,
Меж базальтовых скал и жемчужных
Шелестят паруса кораблей.

Быстрокрылых ведут капитаны,
Открыватели новых земель,
Для кого не страшны ураганы,
Кто изведал мальстремы и мель,

Чья не пылью затерянных хартий, —
Солью моря пропитана грудь,
Кто иглой на разорванной карте
Отмечает свой дерзостный путь

И, взойдя на трепещущий мостик,
Вспоминает покинутый порт,
Отряхая ударами трости
Клочья пены с высоких ботфорт,

Или, бунт на борту обнаружив,
Из-за пояса рвет пистолет,
Так что сыпется золото с кружев,
С розоватых брабантских манжет.

Пусть безумствует море и хлещет,
Гребни волн поднялись в небеса,
Ни один пред грозой не трепещет,
Ни один не свернет паруса.

Разве трусам даны эти руки,
Этот острый, уверенный взгляд
Что умеет на вражьи фелуки
Неожиданно бросить фрегат,

Меткой пулей, острогой железной
Настигать исполинских китов
И приметить в ночи многозвездной
Охранительный свет маяков?

