% vim: keymap=russian-jcukenwin
%%beginhead 
 
%%file 25_01_2022.stz.news.lnr.lug_info.1.den_studenta
%%parent 25_01_2022
 
%%url https://lug-info.com/news/prazdnichnye-meropriyatiya-proshli-v-regionah-respubliki-v-den-studenta
 
%%author_id news.lnr.lug_info
%%date 
 
%%tags donbass,lnr,obrazovanie,prazdnik,prazdnik.tatjanin_den,studenty
%%title Праздничные мероприятия прошли в регионах Республики в День студента
 
%%endhead 
 
\subsection{Праздничные мероприятия прошли в регионах Республики в День студента}
\label{sec:25_01_2022.stz.news.lnr.lug_info.1.den_studenta}
 
\Purl{https://lug-info.com/news/prazdnichnye-meropriyatiya-proshli-v-regionah-respubliki-v-den-studenta}
\ifcmt
 author_begin
   author_id news.lnr.lug_info
 author_end
\fi

Награждения самых активных и талантливых учащихся, праздничные, развлекательные
и профориентационные мероприятия прошли в регионах Республики в День студента.

День студента установлен в Республике указом главы ЛНР и отмечается ежегодно 25
января.

ЛУГАНСК

Церемония награждения победителей конкурса \enquote{Студент года} состоялась в
Луганской государственной академии культуры и искусств имени Михаила
Матусовского. 

\ii{25_01_2022.stz.news.lnr.lug_info.1.den_studenta.pic.1}

Победителем конкурса \enquote{Студент года} стала студентка Луганского государственного
медицинского университета имени Святителя Луки Надежда Мосягина.

АЛЧЕВСК

Конференция, посвященная патриотическому воспитанию и студенческому
самоуправлению, прошла в Донбасском государственном техническом институте
(ДонГТИ).

\ii{25_01_2022.stz.news.lnr.lug_info.1.den_studenta.pic.2}

В Алчевской гимназии имени Петра Липовенко прошел час профориентации \enquote{Я –
будущий студент}. Будущие абитуриенты обсудили условия, необходимые для
правильного выбора профессии, постарались выявить свои склонности к тем или
иным профессиям, познакомились с конкретными профессиями.

АНТРАЦИТ

Студенты Антрацитовского колледжа информационных технологий и экономики
организовали развлекательное мероприятие. Празднование началось с поздравления
учащихся и педагогического коллектива, учащиеся также показали творческие и
юмористические номера о студенческой жизни. Третьекурсники посетили
Свято-Покровский храм, где узнали о житие святой мученицы Татианы.

\ii{25_01_2022.stz.news.lnr.lug_info.1.den_studenta.pic.3}

\enquote{Студенческие годы – замечательная пора в жизни любого человека, когда
закладываются основы будущего, сбываются надежды, впереди еще вся жизнь, полная
замечательных идей и грандиозных свершений. Годы, проведенные в высших и
средних учебных заведениях, – самое счастливое время, когда происходит выбор
собственной жизненной позиции}, - сказал глава администрации Антрацита и
Антрацитовского района Сергей Саенко.

Во Дворце культуры имени Ленина прошла концертно-развлекательная программа с
участием молодежных вокальных и хореографических коллективов, которая
завершилась дискотекой.

БРЯНКА

В Брянковском комплексном центре состоялось мероприятие для студентов
университета III поколения. Работники отделения социально-бытовой адаптации,
активисты первичного отделения \enquote{Терцентр} общественного движения (ОД) \enquote{Мир
Луганщине} и проекта \enquote{Забота о ветеранах} ознакомили присутствующих с историей
возникновения праздника, а также поздравили женщин с именем Татьяна.

\ii{25_01_2022.stz.news.lnr.lug_info.1.den_studenta.pic.4}

Директор комплексного центра, координатор проекта \enquote{Забота о ветеранах} в Брянке
Людмила Герасимова отметила, что студенческие годы – это годы молодости, \enquote{но
многие брянковчане в университете III поколения повторяют свою молодость}. Она
сообщила, что в 2021-2022 году в университете обучаются 67 жителей города.

\ii{25_01_2022.stz.news.lnr.lug_info.1.den_studenta.pic.5}

Завершилось мероприятие традиционным чаепитием.

КИРОВСК

Торжественное мероприятие прошло в Кировском транспортном техникуме. Его
гостями стали первый заместитель главы администрации Кировска Василий Ткачев,
руководитель исполкома Кировского теротделения ОД \enquote{Мир Луганщине} Александр
Коржан, член Общественной палаты ЛНР Нина Пшеничкина, представители молодежных
и ветеранских организаций.

\ii{25_01_2022.stz.news.lnr.lug_info.1.den_studenta.pic.6}

\enquote{День Татьяны не зря совпал с Днем студента. Татьяна – это символ верности и
романтики, а все эти качества присущи молодежи. Студенчество – это самый
радостный, самый прекрасный, самый насыщенный событиями период в вашей жизни,
когда рождаются неожиданные идеи, новые стремления. Это именно тот период,
когда человек выбирает свой жизненный путь... Вам продолжать все наши традиции,
вам строить и развивать будущее нашей молодой Республики!} – обратилась к
учащимся Пшеничкина.

\ii{25_01_2022.stz.news.lnr.lug_info.1.den_studenta.pic.7}

Творческие номера юных артистов Кировского Дворца культуры, веселые сценки,
студенческие приметы, факты о студенческой жизни разнообразили программу
мероприятия. В ходе торжества состоялась церемония посвящения в студенты
первокурсников.

КРАСНОДОН

В краснодонском Дворце культуры имени \enquote{Молодой гвардии} состоялась
торжественная церемония награждения самых активных и талантливых студентов
региона.

\enquote{Именно вам предстоит взять на себя ответственность за благополучие нашего
общества, будущее Луганской Народной Республики. Уверен, что незабываемая
студенческая пора подарит вам массу памятных событий, ярких впечатлений,
интересных открытий, преданных друзей. Поздравляю вас с Днем студента —
праздником молодости, оптимизма, больших ожиданий}, – обратился к молодым людям
глава администрации Краснодона и Краснодонского района Сергей Козенко.

\ii{25_01_2022.stz.news.lnr.lug_info.1.den_studenta.pic.8}

Он вручил грамоты и благодарственные письма горрайонной администрации
победителям республиканских молодежных конкурсов.  

Студентов поздравили творческие коллективы и солисты ДК имени \enquote{Молодой
гвардии}, а также Суходольского ДК имени Олега Кошевого.

КРАСНЫЙ ЛУЧ

Мероприятие, в ходе которого присутствующие узнали о возникновении праздника,
посмотрели выступления городских художественных коллективов и поучаствовали в
конкурсах, прошло в Краснолучском дворце культуры имени Ленина.

ПЕРВОМАЙСК

Сотрудники Дома культуры имени Островского провели для юных первомайцев
развлекательное мероприятие, посвященное Татьяниному дню и Дню студента.

\enquote{Детям было показано познавательное видео о происхождении двойного праздника, и
как они связаны. Узнав происхождение праздника, дети играли в различные
подвижные игры}, - рассказали в городском учреждении культуры.

ПЕРЕВАЛЬСКИЙ РАЙОН

В районной администрации состоялась встреча учащихся Перевальского колледжа и
будущих студентов – учеников 11-го класса Перевальской гимназии № 1. Заведующая
сектором профессионального обучения и профессиональной ориентации
территориального отделения Республиканского центра занятости ЛНР Нина Андреева
рассказала молодым людям о том, как избежать ошибок при выборе профессии и
поделилась основами самопрезентации.   

\ii{25_01_2022.stz.news.lnr.lug_info.1.den_studenta.pic.9}

Затем собравшиеся провели викторину, ответив на вопросы об истории
возникновения праздника, жизни студентов и их традициях. Все участники
мероприятия получили призы от администрации Перевальского района.

\ii{25_01_2022.stz.news.lnr.lug_info.1.den_studenta.pic.10}

\enquote{Данное мероприятие объединило и сплотило нашу молодежь. Все участники получили
полезные знания, опыт и массу незабываемых эмоций. Ребята помогали и
поддерживали друг друга, несмотря на то, что были соперниками. Самое главное,
что все они получили полезные знания, опыт и массу положительных впечатлений},
- сказала заместитель начальника управления - начальник отдела молодежи и
спорта районной администрации Ольга Лапина.

РОВЕНЬКИ

Сектор спорта и молодежи отдела культуры, спорта и молодежи горадминистрации
провел в ровеньковской библиотеке имени Григория Сковороды круглый стол на тему
\enquote{День студента}. В мероприятии приняли участие активисты Молодежного совета при
администрации города и студенты.

\ii{25_01_2022.stz.news.lnr.lug_info.1.den_studenta.pic.11}

\enquote{Целью встречи является духовно-нравственное воспитание молодого поколения. В
ходе встречи участники узнали историю возникновения праздника, поделились
историями студенческой жизни, обсудили приметы и суеверия, распространенные
среди студентов, а также с участниками была проведена викторина}, - рассказали
в секторе спорта и молодежи.

СВЕРДЛОВСК

Молодежный совет при администрации Свердловска и Свердловского района провел
для всех желающих онлайн-викторину \enquote{Ученье – свет!} об истории и традициях
празднования Дня студента.

\ii{25_01_2022.stz.news.lnr.lug_info.1.den_studenta.pic.12}

В городском краеведческом музее прошло мероприятие, посвященное этому
празднику, гостями которого стали свердловчане.  

Сотрудники Свердловской центральной городской библиотеки подготовили
мультимедийную презентацию \enquote{Татьяна: имя и судьба}.

СЛАВЯНОСЕРБСКИЙ РАЙОН

Студенты Славяносербского техникума Луганского государственного аграрного
университета посетили богослужение в Свято-Архидиаконо-Стефановском храме,
приуроченное ко дню памяти святой мученицы Татианы. В ходе богослужения
присутствующие узнали о жизни святой и почему она считается покровительницей
студентов.

\enquote{В этот праздник хочу пожелать всем студентам защиты святой Татианы. Пусть она
дарует всем светлый и чистый разум, чтобы с легкостью преодолевались все
трудности обучения}, - сказал заведующий сектором молодежи и спорта отдела
культуры, молодежи и спорта райадминистрации Артур Кленов. 

СТАХАНОВ

В стахановском Дворце культуры имени Максима Горького состоялась
интеллектуальная молодежная игра \enquote{День Татьяны – для тебя, студент}.
Участниками состязаний стали команды Стахановского инженерно-педагогического
института менеджмента Луганского государственного университета имени Владимира
Даля, стахановского педагогического колледжа Луганского государственного
педагогического университета (ЛГПУ), стахановского отделения медицинского
колледжа Луганского государственного медицинского университета имени Святителя
Луки, промышленно-экономического и машиностроительного техникумов.

Находчивость, знания и эрудицию участников оценивали представительницы
различных профессий, которые носят имя Татьяна.

\enquote{Пусть ваши студенческие годы будут насыщены яркими событиями. Никогда не
отступайте перед трудностями, не останавливайтесь на достигнутом, не бойтесь
ставить перед собой самые смелые задачи}, – обратилась к студентам почетный
гражданин Стаханова Тамара Ковтун.

В ходе игры молодые люди разгадывали ребусы, называли фильмы по представленным
музыкальным композициям, определяли истинность или ложность высказываний,
отвечали на вопросы наставников. Один из раундов был посвящен деятельности
краснодонской подпольной организации \enquote{Молодая гвардия}.

Победу в интеллектуальных состязаниях одержала команда Стахановского
педколледжа ЛГПУ, участники которой получили диплом, кубок, ценный подарок и
сладости.
