% vim: keymap=russian-jcukenwin
%%beginhead 
 
%%file 15_03_2023.fb.krutenko_maryna.mariupol.1.15_03_22_dvadtsatii_.cmt
%%parent 15_03_2023.fb.krutenko_maryna.mariupol.1.15_03_22_dvadtsatii_
 
%%url 
 
%%author_id 
%%date 
 
%%tags 
%%title 
 
%%endhead 

\qqSecCmt

\iusr{Виктория Сорокотяг}

Марина, дякую за... важкі спогади. Кожного ранку чекаю на твої оповідання...

\iusr{Inna Drozdenko}

Я тільки нещодавно почала більш-меньш нормально спати. І то такє. Сни всі про Маріуполь. Тільки що вже звично.

\iusr{Александр Мартемьянов}

Привет, Марина. Продолжай писать, если Бог ложит на сердце.

\begin{itemize} % {
\iusr{Maryna Krutenko}
\textbf{Александр Мартемьянов} уже скоро нечего будет писать)))? Ещё пару дней и все)))
\end{itemize} % }

\iusr{Віктор Куліков}

А мы ещё две недели были в Мариуполе...

\begin{itemize} % {
\iusr{Maryna Krutenko}
\textbf{Віктор Куліков} в комментариях мне писали, что то что мы пережили, это были цветочки.... Ягодки пошли позже.... Так они и есть! Я вообще не понимаю как там хоть кто-то выжил???

\iusr{Віктор Куліков}
\textbf{Maryna Krutenko} да...ягодки были позже....
А мы выбирались пешком под обстрелами.

\iusr{Maryna Krutenko}
\textbf{Віктор Куліков} вы считай вообще под Азовтсалью жили

\iusr{Віктор Куліков}
\textbf{Maryna Krutenko} да...наш посёлок попал сильно...
\end{itemize} % }


\iusr{Валентина Самарина}

Да ягодки были... четко помню как с 17 на 18 марта ночевали просто на подднах в
подъезде на нижних этажах дом содрагался и даже фонарик было страшно включить а
утром увидели коллону \enquote{асвабадителей} мы думали что ад немного стихнет а зря
ужас ждал впереди

\iusr{Макс Лютов}

Как Костя позвонил.... Связи не было вообще 😱😱😱

\begin{itemize} % {
\iusr{Maryna Krutenko}
\textbf{Макс Лютов} мы каждый день ходили на то место, где была связь. Мы каждый день звонили родственникам и говорили, что мы живы.

\iusr{Макс Лютов}
\textbf{Maryna Krutenko} везёт))) мы один раз только за всё время смогли дозвониться....
\end{itemize} % }

\iusr{Ната Чернова}

Мариночка, привет!! Это Юлина мама, я очень благодарна, что Сережа с Женей, меня с
мужем спасли🙏🙏🙏. 15марта 2022г, я никогда не забуду!!! Пусть Господь
оберегает вас!!!!!❤❤❤

\iusr{Дима Секачёв}

Марина пиши, очень конечно страшно представить такое что вы пережили, не могу
даже такое представить или сравнить. Слава Богу что выбрались !!!

\iusr{Lana Khudiakova}

Маринка, читаю и плачу!! Слава Богу вы остались живы!! Мой мозг отказывается
понимать, как те кто ждал ацвабадителей, пережив такое могут петь им серенады
сейчас!!!! Сколько невинных людей просто стерли - жизни, надежду, счастье! Но ты
молодец❤️надо продолжать жить! Люблю тебя, обнимаю💕😘

\begin{itemize} % {
\iusr{Maryna Krutenko}
\textbf{Светлана Худякова} мне рассказывали, что одна женщина скала тело своего мужа... он погиб в квартире и они убежали, она не смогла его похоронить. Так вот, в морге на Левом берегу было 87.000 тел, а в Мангуше 24.000. 100 000 человек взяли и освободили от жизни. В городе жили 470 000, 90 000 сейчас проживают. Вопрос к знатокам, сколько людей освободили от имущества?.....(((((
\end{itemize} % }

\iusr{Виктория Варварига}

Ты описуешь в точности и наш выход из города, только мы выезжали на день позже
16. 03 вечером-в этот день скинули бомбу на драмтеатр. До сих пор перед глазами
тлеющие останки театра. Ты как будто попал в фильм ужасов.... все как ты
описуешь: провода по всей дороге, снаряды в асфальте, осколки, сгоревшие дома,
машины, танки, трупы.... только уже на выезде не было наших солдат, блокпост был
разбит, стояла сгоревшая техника... ужас. Первая ночёвка в Демьяновке. Никогда не
забуду то, как нас встречали \href{https://www.facebook.com/profile.php?id=100005210680976}{Наталия Тимакова} и ее команда, потом
Токмак - совершенно незнакомые люди стали ангелами-спасителями, никогда не
забуду. Тогда я поняла - такой народ не победить!!!!

\begin{itemize} % {
\iusr{Maryna Krutenko}
\textbf{Виктория Варварига} блокпоста не было и когда мы выезжали. Стояли пару солдат на конечной остановке «посёлок Моряков». Блокпост был разбит, стояли пару сгоревших танков....
\end{itemize} % }

