%%beginhead 
 
%%file 10_10_2021.fb.loskutova_natalia.mariupol.1.speaking_club_frants
%%parent 10_10_2021
 
%%url https://www.facebook.com/permalink.php?story_fbid=pfbid0xLfS88mhHzc58Azx3rksx9GsnmC98ocmYsCKcDT4C2tv4yrN1e4YN5JUrtosdBjtl&id=1427894275
 
%%author_id loskutova_natalia.mariupol
%%date 10_10_2021
 
%%tags mariupol,language,ucheba,obuchenie,francia
%%title Speaking Club французькою у невимушеній обстановці
 
%%endhead 

\subsection{Speaking Club французькою у невимушеній обстановці}
\label{sec:10_10_2021.fb.loskutova_natalia.mariupol.1.speaking_club_frants}

\Purl{https://www.facebook.com/permalink.php?story_fbid=pfbid0xLfS88mhHzc58Azx3rksx9GsnmC98ocmYsCKcDT4C2tv4yrN1e4YN5JUrtosdBjtl&id=1427894275}
\ifcmt
 author_begin
   author_id loskutova_natalia.mariupol
 author_end
\fi

Speaking Club французькою у невимушеній обстановці 

08 жовтня 2021 відбулася чергова зустріч з француженкою пані Матильдою Муліс
(Mathilde Moulis), інженеркою французької інжинірингової компанії «Beten
International». Цього разу студенти зустрілися з гостею у кафе з французькою
назвою «Хлеб дю солей» (soleil – у перекладі з французької «сонце»). Як і
завжди захід «Club d'art oratoire» (Клуб ораторського мистецтва) проводився у
формі спікінг-клубу: студенти ставили питання пані Матільді, вона давала
вичерпну відповідь і щось питала у студентів. Темою зустрічі були стереотипи і
студенти мали можливість розкрити для себе Францію, розвіявши багато з них.
Так, не зважаючи на репутацію найромантичнішої країни у Європі, з'ясувалося, що
французькі чоловіки є менш галантними ніж україніці: вони не пропускають жінок
при виході, не подають руки, коли дама виходить з транспортного засобу.
Француженки майже не використовують макіяж на роботі, а нігті фарбують лише з
нагоди урочистих заходів. Натомість, на думку французів, українці майже кожного
дня вживають горілку, а українська мова нічим не відрізняється від російської.
Звичайно, студенти розвіяли ці стереотипи! А ось думку про те, що українські
жінки є найгарнішими, вони спростовувати не стали. Наприкінці жовтня
відбудеться остання зустріч, оскільки термін перебування пані Матильди у
Маріуполі, на жаль, спливає.
