% vim: keymap=russian-jcukenwin
%%beginhead 
 
%%file 13_08_2022.stz.news.ua.donbas24.1.mrpl_spisok_shindlera_istoria_elizaveta_birjukova.txt
%%parent 13_08_2022.stz.news.ua.donbas24.1.mrpl_spisok_shindlera_istoria_elizaveta_birjukova
 
%%url 
 
%%author_id 
%%date 
 
%%tags 
%%title 
 
%%endhead 

«Маріупольський список Шиндлера»: вражаюча історія Єлизавети Бірюкової (ВІДЕО)

Історія Єлизавети Бірюкової під час повномасштабної війни набуває особливої актуальності та важливості

У роки Другої світової війни судетський німецький промисловець Оскар Шиндлер
врятував майже 1200 євреїв під час Голокосту, надавши їм роботу на своїх
заводах в Польщі та Чехії. Єлизавета Бірюкова, як і Оскар Шиндлер, намагалася
рятувати людей в Маріуполі, які не забули про вчинки цієї відважної та сміливої
жінки. Сьогодні з її історією варто ознайомитися кожному українцю, який через
військову агресію росії змушений долати важкі випробувування та життєві
труднощі. Приклад Єлизавети Бірюкової доводить, що завжди, навіть у найбільш
важкі життєві періоди, необхідно залишатися Людиною... 

Передісторія

У 2013 році українська письменниця, історик Олена Стяжкіна знайшла матеріали по
справі Єлизавети Бірюкової, молодої жінки з вищою технічною освітою. Саме
завдяки цій події громадськість вперше заговорила про невідомий подвиг
маріупольчанки. Під час окупації Маріуполя нацистами Єлизавета була змушена
залишитися в місті й працювати на рибоконсервному комбінаті. Але під час роботи
вона починає абсолютно системно рятувати людей. Пізніше вона була засуджена за
зв'язок з шефом комбінату і за те, що нібито на когось доносила. Проте після її
арешту люди, яких вона врятувала, проявивши справжню сміливість, почали писати
листи Вишинському, Берії та Сталіну. Завдяки онучкам Єлизавети Пилипівни,
Оксані та Наталі, мені вдалося дізнатися багато унікальних деталей з біографії
Бірюкової. Водночас онучки поділилися світлинами з сімейного архіву.

Факти з біографії Єлизавети Бірюкової

Народилася Єлизавета 11 жовтня (27 вересня) 1914 року в Маріуполі, проживала на
вулиці Артема, 15 (до окупації росіянами — вулиця Куїнджі) в дружній родині.
Батько — Пилип Семенович — робочий столяр. Мама Єлизавети — Єфросинія
Прокопівна, за розповіддю онучок, мала знатне походження. А братик Костя,
незважаючи на різницю у віці, став найближчим і найкращим другом для Лізи.

Батьки зробили все можливе, щоб Ліза з братом отримали вищу освіту і ні від
кого не залежали. Єлизавета була чесною, справедливою і відповідальною
дівчиною. З юності привчала себе до праці, віддавалася роботі повністю, була
справжнім трудоголіком. У свої 27 років вона вже була дипломованим
інженером-технологом маріупольського рибоконсервного комбінату, працювала
старшим лаборантом.

Героїзм Бірюкової під час окупації

Коли в червні 1941 року німецькі війська напали на Радянський Союз, ніхто в
Маріуполі не очікував, що довгий шлях до них в 1300 кілометрів подолають так
швидко. 1 вересня маріупольці почали зводити захисні споруди, а вже 8 жовтня
місто захопили. Мама Лізи виїхала з Маріуполя, брат вирушив на фронт. А батько
залишився з донькою, яка захворіла на скарлатину, що підтверджує лікарняний
лист, який зберігся в справі Єлизавети Пилипівни. Дівчина, з огляду на
обставини, була змушена продовжувати роботу на рибоконсервному комбінаті. Вона
добре володіла німецькою мовою, тому до неї часто зверталися німці, вважаючи
Лізу добросовісною і старанною робітницею.

Завдяки матеріалам справи, знайденим О. Стяжкіною, вдалося дізнатися, що
Єлизавета Пилипівна намагалася зберегти молодь від поїздки до Німеччини. До
того ж Бірюкова робила все можливе, щоб прийняти на роботу всіх, кому
загрожувала небезпека з боку німецького командування: дружин комуністів, євреїв
(які мають дітей), комуністів і комсомольців. Також вона, ризикуючи власною
безпекою, звільняла військовополонених з табору, пояснюючи це тим, що їм треба
працювати на заводі, адже робітники потрібні завжди. До цих пір достовірно не
відомо, скільки насправді було врятовано маріупольців. Відповідно до матеріалів
справи й того, скільки всього робила Єлизавета Пилипівна, ця цифра може бути
більшою ніж 100 чи навіть 200 чоловік.

Арешт і заслання

Суд присудив Єлизаветі Пилипівні 20 років каторжних робіт і 5 років ураження в
правах. У тих складних обставинах усе-таки Єлизаветі Бірюковій пощастило не
бути розстріляною. Військовий трибунал постановив, що підсудна діяла за
вказівкою шефа рибоконсервного комбінату, і хоча вона була названа зрадницею,
проте отримала кваліфікацію «пособниця», що і стало для неї порятунком. Однак
Єлизавета Пилипівна своєї вини не визнавала.

Завдяки матеріалам справи відомо, що відпустити Єлизавету Пилипівну просили 11
маріупольців, але з огляду на всі тогочасні обставини, ці люди, характеризуючи
Бірюкову як рятівницю, проявили неабияку сміливість.

Засуджена Єлизавета Бірюкова відбувала покарання у Воркуті, там, незважаючи на
важкі каторжні роботи, на всю несправедливість долі, вона продовжувала
зберігати людське обличчя. За розповіддю онучки, Оксани Миколаївни Кушнір, у
колонії бабусі написали портрет. Онучки знали, що бабуся була в колонії, але їм
нічого не було відомо про причини та про героїзм Єлизавети Пилипівни під час
окупації.

І все ж завдяки листам батька Єлизавети, небайдужих маріупольців та її скарзі
головному військовому прокурору Збройних сил СРСР, яку вона написала, коли
отримала дозвіл на листування, справу Є. Бірюкової все ж таки переглянули.
Повторний суд відбувся у лютому 1948 році в м. Сталіно. Єлизавета Пилипівна
все-таки була визнана винною в пособництві, формою якого назвали ефективну
роботу на керівній посаді рибоконсервного комбінату. Проте мірою покарання було
названо лише чотири роки й чотири місяці каторжних робіт, які на момент суду
були відбуті. Втомлена, але незламна жінка не погодилася з вироком і все життя
намагалася отримати реабілітацію.

Життя після колонії 

Після колонії життя сміливої й відважної Єлизавети Пилипівни продовжувало
дивувати. Вона не боялася вступити в листування з архімандритом Вільнюського
монастиря, і це в роки тоталітаризму і проголошеного атеїзму. У 36 років
Єлизавета стала мамою. І це теж драматична та унікальна історія. Брат Костянтин
розповів, що знає нещасну і незаможну дівчину, але у неї на руках маленька
дитина — дівчинка 9 місяців. Костянтин запропонував Єлизаветі поглянути на
дитину. Але від побаченого Єлизавета Пилипівна не прийшла в захват. Дитина
погано одягнена, в ліжечку клопи. Онучка, Оксана Миколаївна, згадує розповідь
бабусі.

«Єлизавета Пилипівна з батьками сиділи вдома, пили чай з печивом. З роботи
прийшов брат Костянтин і підкреслив: „Ви тут бенкетуєте, а дитя помирає“. Ці
слова справили досить сильне враження на Єлизавету і вона зробила все можливе,
щоб удочерити дівчинку, яку назвала Ірою».

Єлизавета Пилипівна мала близькі й теплі стосунки з онуками як своїми, так і
брата.

«Бабуся була дуже доброю, м'якою, поблажливою, все прощала своїм онукам. Крім
цього, з бабусею були довірчі відносини. Вона дуже смачно готувала. Завжди
шанувала традиції, збирала разом всю родину. Прекрасно грала на фортепіано,
дуже любила твір Бетховена „До Елізи“. Все життя працювала. Сьогодні це назвали
б кар'єризмом. Бабуся була дуже прямою, справедливою. У чоловіках найбільше
цінувала інтелект», — розповіла онука Єлизавети Пилипівни Оксана.

Єлизавета Пилипівна ніколи нікому з рідних не говорила про лихоліття, яке їй
довелося пережити. Не розповідала про врятованих нею маріупольців. Вона була
для онуків чудовою бабусею, для робочого колективу — цінним співробітником, але
ніхто не знав її до кінця. Ніхто з маріупольців і уявити не міг, наскільки
сильна, вольова жінка проживала в їхньому місті на вул. Артема, 15, наскільки
довго вона чекала реабілітації. Залишитися вірною самій собі, пересилити біль,
який неможливо забути, і прожити гідне життя після важких випробувань,
несправедливості, жорстокості під силу далеко не кожному. Все ж Єлизавета
Пилипівна була реабілітована в 1992 році, але за фатальним збігом обставин і
незрозумілим сценарієм життя цього ж року вона померла.

Справжній героїзм одразу ніколи не привертає увагу. І тільки згодом люди
оцінюють благородство і мужність душ, які ризикують заради порятунку ближніх...

Нагадаємо, раніше Донбас24 розповідав про злочини росіян проти культурної
спадщини Донбасу.

ФОТО: із сімейного архіву онук Єлизавети Бірюкової.
