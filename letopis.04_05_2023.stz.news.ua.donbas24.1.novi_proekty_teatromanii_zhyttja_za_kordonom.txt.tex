% vim: keymap=russian-jcukenwin
%%beginhead 
 
%%file 04_05_2023.stz.news.ua.donbas24.1.novi_proekty_teatromanii_zhyttja_za_kordonom.txt
%%parent 04_05_2023.stz.news.ua.donbas24.1.novi_proekty_teatromanii_zhyttja_za_kordonom
 
%%url 
 
%%author_id 
%%date 
 
%%tags 
%%title 
 
%%endhead 

Нові проєкти «Театроманії 2:0» та життя колективу за кордоном

Колектив «Театроманії 2:0» готує нові постановки, привячені актуальним і
наболілим проблемам

Попри низку труднощів, пов’язаних із життям за кордоном, колектив Театроманія
2:0 продовжує свою діяльність у Ганновері — місті в Північній Німеччині,
столиці і найбільшому місті Нижньої Саксонії. Художній керівник і режисер
колективу Антон Тельбізов розповів, які вистави були показані у квітні 2023
року та що заплановано представити найближчим часом.

Читайте також: Які театральні проєкти та культурні заходи, присвячені
Маріуполю, реалізуються закордонними митцями

Нагадаємо, театральна студія «Театроманія 2:0» почала працювати після
повномасштабного вторгнення рф в Україну. Це творчий осередок української
молоді від 12 до 20 років, які опинилися в Німеччині під час військових дій в
Україні. Студія працює на базі Української Спілки Нижньої Саксонії у місті
Ганновер. Керівником студії є Тельбізов Антон. Протягом 11 років він є
режисером Народного театру «Театроманія», який було засновано у 2011 році у
Маріуполі. Головний напрям роботи — опрацювання сучасних проблем у театральних
постановках та розробка соціальних проєктів. У репертуарі театру — камерні,
музичні та драматичні вистави. Антон Тельбізов зауважив, що у Німеччині дуже
потужна соціальна політика, але до деяких моментів й досі важко звикнути. 

«За рік життя за кордоном я відчув неймовірну турботу від Німеччини. Приємно
вражає дуже якісна соціальна допомога як елемент системи соціального захисту
населення. Завдяки наданому житлу, медичному страхуванню і загальній соціальній
політиці відчуваю себе захищеним. Але все ж не так легко звикнути до іншої
культури та іншої мови», — зазначив режисер «Театроманії 2:0».

Читайте також: Маріупольський театр «Conception» показав у Тернополі свою
виставу (ФОТО)

Та попри наявні труднощі, колектив «Театроманії 2:0» продовжує працювати на
німецьких сценах. Зокрема, 15 квітня театр успішно представив лялькову казку
«Сусідка». Вистава відбулася українською мовою. Ця казка, що пройшла з повним
аншлагом, була поставлена для діточок з України, які бачили війну, втратили
домівки, але все ще вірять в дива. Заплановано виступити з казкою і в інших
містах Німеччини.

«Те, що наші діти виступають в Німеччині, дуже надихає і їх самих, і їхніх
батьків. Адже коли дитина щаслива, щасливі і батьки», — наголосив Антон
Тельбізов.

Крім того, 27 березня «Театроманія» запустила унікальний проєкт, який колектив
спільно з друзями з Берліну, театром «Ogalala Kreuzberg», почали готувати ще у
2021. Це фільм про те, що таке віра і чому вона має значення, набув особливої
актуальності сьогодні. Окремі частини, присвячені духовному пошуку, свободі,
старінню та хворобі, були зняті у відомих місцях Маріуполя, зокрема у
бібліотеці ім. В. Г. Короленка, Центрі сучасного мистецтва «Готель
«Континеталь» та на морі. Ці відео, преставлені на сторінці театру у Facebook,
нагадують маріупольцям про безтурботні і мирні часи.

Читайте також: Маріупольські актори зіграли благодійну виставу в Києві (ФОТО)

Наприкінці травня відбудеться ще одна прем'єра, яку режисер Антон Тельбізов
готує спільно з драматичним театром Ганновера.

«Я ще не визначився з жанром. Але ця постановка розкаже не тільки про
українців, але й людей з інших країн, які опинилися в Ганновері з різних
причин. У кожного свої проблеми, свої страхи та відчуття. Та всіх нас об'єднує
те, що ми є носіями своєї культури та своєї мови. Дім знаходиться всередині
нас. Це те, що зберігає нас і нашу ідентичність», — поділився режисер театру.

Унікальна інтернаціональна вистава про важливі і актуальні проблеми, яка
об'єднає українських, німецьких та артистів інших національностей на одній
сцені, відбудеться 24 травня у Ганновері. А 13 травня 2023 року о 17:00 на
сцені Hölderlin Eins (Hölderlinstr.1, 30 625 Hannover) відбудеться
документальна вистава «Крила», у якій розповідається про любов кожного актора
«Театроманії» до свого рідного міста.

Раніше Донбас24 розповідав про поетичну імпрезу «Лінії життя», яку підготував
Театр авторської п'єси Conception.

Ще більше новин та найактуальніша інформація про Донецьку та Луганську області
в нашому телеграм-каналі Донбас24.

Фото: з архіву Донбас24
