% vim: keymap=russian-jcukenwin
%%beginhead 
 
%%file 19_05_2021.fb.uljanov_anatolij.1.solidarnost
%%parent 19_05_2021
 
%%url https://www.facebook.com/dadakinder/posts/4365385206813942
 
%%author Ульянов, Анатолий
%%author_id uljanov_anatolij
%%author_url 
 
%%tags narod,obschestvo,shotlandia,solidarnost
%%title Пример солидарности
 
%%endhead 
 
\subsection{Пример солидарности}
\label{sec:19_05_2021.fb.uljanov_anatolij.1.solidarnost}
 
\Purl{https://www.facebook.com/dadakinder/posts/4365385206813942}
\ifcmt
 author_begin
   author_id uljanov_anatolij
 author_end
\fi

Пример солидарности: когда агенты иммиграционных служб приехали выселять, и
задержали двух иммигрантов в Глазго, активисты окружили их машину,
заблокировали проезд, и заставили отпустить задержанных. Подоспевшая подмога из
30 полицейских фургонов захлебнулась в людях, которых становилось всё больше, и
все они действовали сообща. 

\ifcmt
  pic https://scontent-cdt1-1.xx.fbcdn.net/v/t1.6435-9/186459699_4365336023485527_1206441089605645802_n.jpg?_nc_cat=109&ccb=1-5&_nc_sid=730e14&_nc_ohc=kO9qm7CB4uAAX8_gwOV&_nc_ht=scontent-cdt1-1.xx&oh=329b1e73aa01149f0e2dfb2cf1718c90&oe=6159664F
  width 0.4
\fi

Статус-кво не случайно напаривает нам идею атланта, – эгоцентричного
индивидуалиста-одиночки, – потому что знает, что она выгодна тому, кто уже
вырвался вперёд за счёт других, – верхушке, элитам, барину, папику, людям
хозяйским, – а для обычных людей: один в поле не воин. 

Другое дело, когда власть сталкивается с людьми, численно превосходящими её
церберов. Массы состоят из индивидов. В данном случае, индивидов, образующих
активный коллектив. А что такое активный коллектив? Это – организация власти.
Поэтому мы и не должны забывать: сила – в нас, в каждом из нас и нас вместе;
панов всегда меньше.
