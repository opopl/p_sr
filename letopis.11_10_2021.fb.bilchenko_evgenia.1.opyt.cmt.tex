% vim: keymap=russian-jcukenwin
%%beginhead 
 
%%file 11_10_2021.fb.bilchenko_evgenia.1.opyt.cmt
%%parent 11_10_2021.fb.bilchenko_evgenia.1.opyt
 
%%url 
 
%%author_id 
%%date 
 
%%tags 
%%title 
 
%%endhead 
\subsubsection{Коментарі}
\label{sec:11_10_2021.fb.bilchenko_evgenia.1.opyt.cmt}

\begin{itemize} % {
\iusr{Катерина Милютина}
Худи прекрасен! А ломка стереотипов полезна)

\begin{itemize} % {
\iusr{Евгения Бильченко}
\textbf{Катерина Милютина} Ну, чисто психологически, да.
\end{itemize} % }

\iusr{Lara Petrovna}
Уже надоело, одно и то же

\begin{itemize} % {
\iusr{Евгения Бильченко}
\textbf{Lara Petrovna} мне тоже надоело, но в этом магазине нет других товаров. С Донецком вообще айс у журналистов, это просто анекдот: они вас лунатика и видят, народ живёт внутри сказки, Алиса и Зазеркалье
\end{itemize} % }

\iusr{Мишель Поляков}
Так Вас люблю, что если б не был женат....
Шутка глупая, знаю... Не претендую. Но честная!

\iusr{Anna Korf Stepanova}
Любит , это точно
@igg{fbicon.heart.red}⚘и восхищается...
Всё очень талантливо , цельно, чётко и смело.

\iusr{Алексей Бажан}

Неменский еще во времена Ющенко писал, чем вызвал грандиозный скандал в тогда
еще живом ЖЖ: "Реабилитация "пронацистского" прошлого - это общий процесс для
большинства стран Центральной и Восточной Европы, он имеет глубокие основания и
никакие "угрозы пальчиком" из Кремля его не остановят. Для Украины он имеет тем
более актуальный характер - если из здания украинской государственности вынуть
кирпич украинского националистического движения, то все здание рухнет - этот
кирпич лежит в его фундаменте. Реабилитация ОУН-УПА - необходимый процесс при
создании новой единой украинской нации, и это надо понимать. Противостояние ему
на Украине носит не украинский национальный характер, оно идет со стороны
русскоязычных сил и потому, по сути, является антиукраинским. Тем более, его
делает таковым поддержка Москвы, которая при этом до сих пор не пожелала
принять и понять тот простой факт, что идеология украинской государственности
носит откровенно антирусский и антироссийский характер и никогда не будет иной.
При полном отсутствии идеологической наполненности и стратегического характера
политики России в отношении Украины, Кремлю стоит выработать какой-нибудь иной
стиль реагирования на эти процессы, так как просто отменить или заморозить их
все равно уже невозможно".

\begin{itemize} % {
\iusr{Евгения Бильченко}
\textbf{Алексей Бажан} если пальчиком, то нет. Вот это ОПЗЖ. Блин, мне реально они больше в кайф, чем режим, но надо по-другому делать. Всё, причем.

\iusr{Алексей Бажан}
\textbf{Евгения Бильченко} Почему же, как о Иване Елагине или Моршене, ОПЗЖ и прочая оппозиция очень принципиальные "антифашисты".

\iusr{Евгения Бильченко}
\textbf{Алексей Бажан} я их не знаю, многое пропускаю, слишком загрузила в литературе и работе копирайтера. Уже не в себя. Горе такое. Реально горе.

\iusr{Алексей Бажан}
\textbf{Евгения Бильченко} Бродский, широкой души был человек, обоих считал лучшими поэтами второй волны эмиграции, притом что у Моршена темная военная биография и соответственно не очень хорошая репутация. Киевляне.

\iusr{Евгения Бильченко}
\textbf{Алексей Бажан} буду знать

\iusr{Алексей Бажан}
\textbf{Евгения Бильченко} У Моршена отец хороший писатель Николай Нароков, был следователем НКВД в 1937-ом, единственный писавший изнутри. Военные его годы тоже покрыты мраком неизвестности.

\end{itemize} % }

\iusr{Alexander Violin}

Ты умная сестра! Всех видишь насквозь! Сытного вторника! Спаси Бог!

\begin{itemize} % {
\iusr{Евгения Бильченко}
\textbf{Alexander Violin} спаси Бог.
\end{itemize} % }

\end{itemize} % }
