% vim: keymap=russian-jcukenwin
%%beginhead 
 
%%file 23_11_2020.news.ua.1.death_sumy_pcr_test_63_starushka.dve_nedeli
%%parent 23_11_2020.news.ua.1.death_sumy_pcr_test_63_starushka
 
%%url 
%%author 
%%author_id 
%%tags 
%%title 
 
%%endhead 

\subsubsection{Тест сделали только через две недели, когда стало совсем плохо}
\label{sec:23_11_2020.news.ua.1.death_sumy_pcr_test_63_starushka.dve_nedeli}

Что касается якобы потерянного теста, то Чижик уверяет, что это не так. \enquote{Ну это
уже совсем, чтобы тест потерять! Мы в понедельник хотели взять ПЦР-тест. Но его
в Сумы в лабораторию отвозит специальная мобильная бригада. Я им позвонила, они
мне сказали, что поедут в Сумы уже в среду. Какой смысл брать материал и
держать его 48 часов? Поэтому я пригласила пациентку пройти в среду ПЦР-тест. И
он был взят, сейчас материалы находятся в областной лаборатории, результатов
пока нет}, \dshM говорит семейный врач.

\ifcmt
pic https://i.obozrevatel.com/gallery/2020/11/19/alla-chizhik-semejnyij-vrach-o.jpg
caption Семейный врач Алла Чижик. Источник: Facebook Аллы Чижик
\fi

В среду, не найдя место в больнице для Любови Юрченко, семейный доктор все-таки
подготовила все документы для госпитализации и выписала электронное
направление.

\ifcmt
pic https://i.obozrevatel.com/gallery/2020/11/19/gettyimages-1209359788.jpg
\fi

\enquote{Женщина вышла за дверь и куда-то ушла, я ее не выгоняла. Я позвонила ее
родным, сказала, что она в таком состоянии, что я буду искать для нее место в
больнице. Она, оказывается, поехала в Глухов, потом в больницу. И уже оттуда
позвонила мне и попросила еще одно направление на рентген. Не знаю зачем,
недавно мы делали его, и он показал только бронхит. Нужно делать КТ, но раз она
так хочет, я выписала ей электронное направление, СМС должно было ей прийти на
телефон. А уже через час мне позвонила главный врач больницы Путивльской ЦРБ и
сообщила, что пациентка умерла}, \dshM говорит семейный врач.

Эта история показательная для сегодняшней ситуации с лечением ковида.
Медицинская система страны уже захлебнулась. Мест нет, кислорода не хватает,
больных не берут, потому что просто некуда их класть. Сами пациенты борются за
то, чтобы сделать лишний глоток кислорода. А украинцы умирают прямо под стенами
больниц.

В правительстве так и не могут решиться на кардинальный шаг и вновь объявить
локдаун, как это делают в Европе. Принимают лишь половинчатые решения, которые,
кроме раздражения у людей, ничего не решают. Да и расширять больницы и
обустраивать кислородные станции и трассы нужно было еще летом, теперь уже
поздно.
