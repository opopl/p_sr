% vim: keymap=russian-jcukenwin
%%beginhead 
 
%%file 09_08_2021.fb.fb_group.story_kiev_ua.1.shkola_pamjat
%%parent 09_08_2021
 
%%url https://www.facebook.com/groups/story.kiev.ua/posts/1725205847676209/
 
%%author Киевские Истории
%%author_id fb_group.story_kiev_ua
%%author_url 
 
%%tags istoria,kiev,pamjat,shkola,ukraina
%%title Воспоминание о школе
 
%%endhead 
 
\subsection{Воспоминание о школе}
\label{sec:09_08_2021.fb.fb_group.story_kiev_ua.1.shkola_pamjat}
 
\Purl{https://www.facebook.com/groups/story.kiev.ua/posts/1725205847676209/}
\ifcmt
 author_begin
   author_id fb_group.story_kiev_ua
 author_end
\fi

Воспоминание о школе.

Когда 1 сентября 1960 года я пошел  в 1 класс 85 средней школы,  то на  линейке
мне доверили произнести  торжественное обещание. Помню до сих пор: "Мы -
ученики 1-А класса обещаем учиться отлично!" Этот короткий текст я произнес без
заминки и был очень доволен собою.  В ту минуту я был совершенно уверен, что
так оно и будет - я буду радовать своими успехами родителей, учителей, буду
примером для других учеников. Однако, довольно скоро я понял, что грызть гранит
науки - занятие довольно скучное. При входе в школу красовались два кумачовых
транспаранта.  Первый: "Коммунистом стать можно лишь тогда, когда обогатишь
свою память знанием всех тех богатств, которые выработало человечество". И
подпись: В. И. Ленин.

Второй транспарант появился когда я уже был в 3 классе: "Нынешнее поколение
советских людей будет жить при коммунизме!"  Появление этого транспаранта
вызвало у учеников и учителей (если, конечно, не придуривались) полнейший
восторг. Наша учительница, Анастасия Михайловна, даже написала довольно
длинный восторженный текст, который поручила выучить наизусть моему
однокласснику, который славился отличной памятью. Содержание не помню, но одна
фраза осталась в памяти: "Через 20 лет мне будет всего 29 лет!" (Если кто не
знает - Никита Сергеевич Хрущев с трибуны очередного съезда провозгласил, что
через 20 лет будет построена материально-техническая база коммунизма). Что же
касается первого лозунга, то как только я научился читать, мне сразу стало
ясно, что коммунистом мне не быть, так как условие, обозначенное в лозунге,
совершенно не реальное. И, надо сказать, как в воду глядел! Более того, я не
стал не только коммунистом, но даже комсомольцем мне побывать не удалось.

Теперь про учебу. Я перестал выполнять домашние задания, как только научился
более менее быстро писать. На большой перемене мне удавалось "скатать" уроки у
более прилежных учеников. Еще я придумал такое "ноу хау" - писать упражнения по
русскому языку, которые нам задавали в достаточно большом объеме, через одно
предложение, то есть писать только первое, третье, пятое и т. д. предложение.
Как правило, это прокатывало при беглом просмотре тетради учителем. По
математике я мог написать: " После очевидных преобразований получаем", а далее
ответ из последних страниц задачника. Во время контрольных работ я (а я сидел
на первой парте прям перед учительским столом) я нагло доставал учебник и
демонстративно листая его, списывал прямо перед носом учителя. Учитель, который
внимательно следил за тем, чтобы не списывали из-под парты, не мог даже
предположить такой наглости и ничего не замечал. Ну, а про такие общеизвестные
штуки, как два дневника - один для двоек, другой для пятерок или про вырванные
и вставленные из запасного дневника страницы даже писать не интересно. Вот еще
забавный эпизод. Учительница русского языка - Лариса Ивановна запрещала писать
авторучками класса до восьмого. А мне как раз купили роскошную китайскую
авторучку с золотым пером. Стоила она тогда сумасшедшие деньги - 6 рублей (10
процентов минимальной зарплаты). И вот, в конце урока я стал записывать
домашнее задание в дневник авторучкой. Лариса Ивановна потребовала, чтобы я
немедленно отдал ручку ей. Я в ужасе отказался, представляя себе, что меня
ожидает дома ( Как же так? Мы тебе купили такую дорогую ручку! Где она?). - Раз
так, сказала учительница,- Ты теперь для меня пустое место, можешь не приходить
на мои уроки, а если и придешь - я все равно буду считать, что тебя нет! И
велела старосте класса при перекличке отмечать меня отсутствующим. 

Я очень расстроился, но не надолго! Взвесив за и против, я понял все
преимущества такого моего положения. Во-первых, не надо бояться, что тебя
вызовут и ты получишь двойку. 

Во-вторых, можно не готовить домашние задания, не учить стихи. Короче, не
жизнь, а малина! Перед каждым уроком, я напоминал старосте, чтобы он не забыл
сообщить, что я отсутствую, а то не ровен час, Лариса Ивановна забудет о моем
"наказании". Я догадывался, что она уже и не рада, но ходу назад не было. Так
продолжалось довольно долго, до тех пор, пока в школу не нагрянула, какая-то
комиссия и ее представители сидели на уроках. Не желая подставлять Ларису
Ивановну, я сказал старосте, чтобы он не вздумал сообщать о моем отсутствии.

Все закончилось хорошо. Контрольную я написал и сдал. Лариса Ивановна сказала,
что наказание закончено, хоть я и не сделал должных выводов, но чувствовалось,
что она в чем-то благодарна мне... И еще. Так как авторучками писать
запрещалось, я придумал вставлять в обычную ручку сразу четыре пера. И теперь,
при обмакивании ручки, за счет капиллярных явлений, набиралось чернил столько,
что хватало написать целую страницу, не обмакивая ручку. Увидав это шедевр
изобретательства, Лариса Ивановна в "восторге" произнесла: " Печерный! Ты еще
одним пером писать не научился, а уже четыре вставил!"

\ii{09_08_2021.fb.fb_group.story_kiev_ua.1.shkola_pamjat.cmt}
