% vim: keymap=russian-jcukenwin
%%beginhead 
 
%%file 26_01_2022.fb.uljanov_anatolij.1.vojna_mir
%%parent 26_01_2022
 
%%url https://www.facebook.com/dadakinder/posts/5191368614215593
 
%%author_id uljanov_anatolij
%%date 
 
%%tags mir,rossia,ugroza,ukraina,vojna
%%title Так много разговоров о войне, и так мало о мире
 
%%endhead 
 
\subsection{Так много разговоров о войне, и так мало о мире}
\label{sec:26_01_2022.fb.uljanov_anatolij.1.vojna_mir}
 
\Purl{https://www.facebook.com/dadakinder/posts/5191368614215593}
\ifcmt
 author_begin
   author_id uljanov_anatolij
 author_end
\fi

Водопад флагов на авах. В дело пошла идентичность. Чтобы было понятно, кто наш,
кто не наш, и кто кринж. Мысль, что на нас нападают, приятнее, чем мысль, что
нас используют. Она придаёт чувство веса разменной монете. Эта история
становится «про нас», тогда как она – про «нами». Наконец-то можно отключить
критическое мышление, поставить флаг на аву, заявить выгодную пошлость, и
пустить слюну, которая придаёт пускающему блеска. «Символический обмен и
смерть».

\ii{26_01_2022.fb.uljanov_anatolij.1.vojna_mir.pic.1}

Бомбардировка страхом удалась. Я не знаю, следите ли вы за издательской
деятельностью НАТО, но недавно эта организация выпустила исследование
«Когнитивная война». Так вот, ребза, это она – ничего не произошло, а анусы
квакают.

С одной стороны, интересно наблюдать эти технологии в работе. С другой – жалко
людей. В том числе тех, которые лезут в капкан, ведь они не перестают от этого
быть людьми. Бесят не люди, а те, кто это делает с людьми. Потому что делать
такое с людьми может только подлая сука.

Пресс-секретарь Белого Дома Джен Псаки прямым текстом называет страны, куда США
стягивают войска, тем, чем они и являются: «наши страны восточного фланга».
Уверен, среди украинских «патриотов» найдётся множество тех, кто не только не
распознаёт в этом имперский язык и придаточный статус, но был бы рад им
обладать – быть их флангом. А вы хотите?  

Оценивая движение российских войск по российской территории как агрессию, США
(750 военных баз в 80 странах мира) не считают таковыми свои манёвры за
тридевять земель от собственных границ. Это – другое.

Интересно, понимают ли «патриоты», которые поют «All We Need is NLAW», что это
саботирует экономику? Никто не инвестирует в страну, которая кричит «война!».
Разве что оружейное лобби. Уже хотя бы поэтому можно было додуматься брать
нотку пониже в своих вариациях на тему «вставай страна огромная».

Семьи дипломатов эвакуируют, а оружие завозят, хотя должно быть наоборот. Если,
конечно, цель деэскалировать, а не разжечь. Это не агрессивно? Или агрессор
бывает только русским?

«Гражданское общество» повторяет агитки американских медиа. Друзья, эти агитки
не для нас. Они этими агитками пытаются отвлечь своих от внутренних проблем,
продать войну в соответствии с законами телевизионного жанра. Подменяя
предпринимателей, включая Путина, марвеловскими злодеями, вы покидаете
территорию реальности, которая требует от вас рациональных мыслей и поступков,
а не символических жестов в метавселенной.

От кого странам НАТО защищаться? От чувака, который к ним только что трубу
провёл, чтобы делать мани, и ещё три дачи себе выдолбить в скале? В интересах
России держать остатки Украины в амбивалентном напряжении, чтобы НАТО не
приближалось, а брать на содержание руину – зачем? Это дорого. И не выгодно.

Так много разговоров о войне, и так мало о мире, о компромиссах и опциях,
необходимости договорняка. Где антивоенное движение? Где акций протеста против
концентрации войск и стволов со всех сторон? Я понимаю, проще флаг поставить на
аву, перекреститься. Но если цель – мир, а не война – нужно больше говорить о
мире, а не о том, как вот-вот, и всё. Не всё. Ничего ещё не случилось. Просто
однополярный мир, возникший в результате краха СССР, заканчивается, и Америка
не может этого пережить. Это её паника. Которая не должна становится нашей.

Нам нужен МИР. Залогом мира является внеблоковый статус Украины. Ни Москва, ни
Вашингтон. Нейтралитет. Никаких военных баз и чужих армий на нашей земле.
Многовекторное экономическое партнёрство. Демократическая децентрализация с
правом регионов самим решать какой им там бандера и язык ближе. Хватит с нас
однополярного мира титульных наций. Нужно строить современные мультикультурные
общества со множеством образов жизни и центров тяжести.
