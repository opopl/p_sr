% vim: keymap=russian-jcukenwin
%%beginhead 
 
%%file 13_01_2022.fb.fb_group.story_kiev_ua.2.vzlety_i_padenia
%%parent 13_01_2022
 
%%url https://www.facebook.com/groups/story.kiev.ua/posts/1839339742929485
 
%%author_id fb_group.story_kiev_ua,novickij_vladimir
%%date 
 
%%tags kiev,kievljane,pamjat,semja
%%title Были взлёты и падения, но всегда оставалась любовь к любимому Киеву
 
%%endhead 
 
\subsection{Были взлёты и падения, но всегда оставалась любовь к любимому Киеву}
\label{sec:13_01_2022.fb.fb_group.story_kiev_ua.2.vzlety_i_padenia}
 
\Purl{https://www.facebook.com/groups/story.kiev.ua/posts/1839339742929485}
\ifcmt
 author_begin
   author_id fb_group.story_kiev_ua,novickij_vladimir
 author_end
\fi

Между этими первыми двумя фото, где я маленький и  я в подаренной мне внуками
футболке с надписью  Amazin - (Удивительный) дедушка» которые я выставил первыми
— расстояние в долгих 83 года.  Большая часть из которой прошла в нашем родном
городе Киеве, где на улице Дмитриевской, как раз под старый Новый Год, 13
января 1939 года  в № 42 родился я,  и нарекли меня, в честь Великого Киевского
князя — Владимиром. 

\raggedcolumns
\begin{multicols}{3} % {
\setlength{\parindent}{0pt}

\ii{13_01_2022.fb.fb_group.story_kiev_ua.2.vzlety_i_padenia.pic.1}
\ii{13_01_2022.fb.fb_group.story_kiev_ua.2.vzlety_i_padenia.pic.1.cmt}

\ii{13_01_2022.fb.fb_group.story_kiev_ua.2.vzlety_i_padenia.pic.2}
\ii{13_01_2022.fb.fb_group.story_kiev_ua.2.vzlety_i_padenia.pic.2.cmt}

\ii{13_01_2022.fb.fb_group.story_kiev_ua.2.vzlety_i_padenia.pic.3}

\ii{13_01_2022.fb.fb_group.story_kiev_ua.2.vzlety_i_padenia.pic.4}
\ii{13_01_2022.fb.fb_group.story_kiev_ua.2.vzlety_i_padenia.pic.4.cmt}

\ii{13_01_2022.fb.fb_group.story_kiev_ua.2.vzlety_i_padenia.pic.5}
\ii{13_01_2022.fb.fb_group.story_kiev_ua.2.vzlety_i_padenia.pic.5.cmt}

\ii{13_01_2022.fb.fb_group.story_kiev_ua.2.vzlety_i_padenia.pic.6}
\ii{13_01_2022.fb.fb_group.story_kiev_ua.2.vzlety_i_padenia.pic.6.cmt}

\ii{13_01_2022.fb.fb_group.story_kiev_ua.2.vzlety_i_padenia.pic.7}

\ii{13_01_2022.fb.fb_group.story_kiev_ua.2.vzlety_i_padenia.pic.8}
\ii{13_01_2022.fb.fb_group.story_kiev_ua.2.vzlety_i_padenia.pic.8.cmt}

\ii{13_01_2022.fb.fb_group.story_kiev_ua.2.vzlety_i_padenia.pic.9}

% На слонике в Первомайском саду
\ii{13_01_2022.fb.fb_group.story_kiev_ua.2.vzlety_i_padenia.pic.10}
\ii{13_01_2022.fb.fb_group.story_kiev_ua.2.vzlety_i_padenia.pic.10.cmt}

% Фото с бала окончания школы
\ii{13_01_2022.fb.fb_group.story_kiev_ua.2.vzlety_i_padenia.pic.11}

% Неделя после нашей с Лилей свадьбы
\ii{13_01_2022.fb.fb_group.story_kiev_ua.2.vzlety_i_padenia.pic.12}
\ii{13_01_2022.fb.fb_group.story_kiev_ua.2.vzlety_i_padenia.pic.12.cmt}

% Фото с Выпускного альбоиа Киевского Политехнического Института
\ii{13_01_2022.fb.fb_group.story_kiev_ua.2.vzlety_i_padenia.pic.13}

% После окончания работал в Констукторском Бюро завода Большевик
\ii{13_01_2022.fb.fb_group.story_kiev_ua.2.vzlety_i_padenia.pic.14}

% Читаю последние известия
\ii{13_01_2022.fb.fb_group.story_kiev_ua.2.vzlety_i_padenia.pic.15}

% Такими мы приехали в Америку
\ii{13_01_2022.fb.fb_group.story_kiev_ua.2.vzlety_i_padenia.pic.16}
\ii{13_01_2022.fb.fb_group.story_kiev_ua.2.vzlety_i_padenia.pic.16.cmt}

% На круизе по Карибским островам
\ii{13_01_2022.fb.fb_group.story_kiev_ua.2.vzlety_i_padenia.pic.17}
\ii{13_01_2022.fb.fb_group.story_kiev_ua.2.vzlety_i_padenia.pic.17.cmt}

% 60 летие нашей свадьбы
\ii{13_01_2022.fb.fb_group.story_kiev_ua.2.vzlety_i_padenia.pic.18}
\ii{13_01_2022.fb.fb_group.story_kiev_ua.2.vzlety_i_padenia.pic.18.cmt}

\ii{13_01_2022.fb.fb_group.story_kiev_ua.2.vzlety_i_padenia.pic.19}
\ii{13_01_2022.fb.fb_group.story_kiev_ua.2.vzlety_i_padenia.pic.19.cmt}

% До скорой встречи друзья.
\ii{13_01_2022.fb.fb_group.story_kiev_ua.2.vzlety_i_padenia.pic.20}
\ii{13_01_2022.fb.fb_group.story_kiev_ua.2.vzlety_i_padenia.pic.20.cmt}

\end{multicols} % }

Между этими символическими фото  было всё - и трудное детство в окупированном
немцами Киеве, без кормильца отца, которого я впервые увидел уже после войны,
так как его в марте 1939 года, через 2 месяца после моего рождения забрали на
военные сборы, а  1 сентября началась Вторая Мировая война и ему пришлось
поучаствовать  во всех её войнах «Финской», «Польской» и «Отечественной». Эти
войны лишили нас общения - отца и сына, которого так не  хватало нам обеим.

Благодаря моей мамочке — которая счастье вложила своею рукой в мою колыбель при
рождении — мы выжили в те трудные для всех Киевлян времена.  И я был живым
свидетелем того — как она это делала, благодаря своей природной одарённости,
смелости, смекалке и умению  буквально всё делать. А ведь ей было  в ту пору
всего 31 год и на руках у неё остались двое детей и три старика, её родители и
папина мама. Ровно два года назад я впервые прочитал рассказ о тех временах,
который написала «Девочка с Евбаза» Галина Васильевна  Дёмина, которая тоже
кстати родилась 13 января, правда на два года раньше меня. Мне очень
понравились её рассказы  о тех временах, они подтверждали то, что я сам видел
или то, что в конце каждого прожитого дня мне рассказывала мама. и я подумал,
что и я  тоже мог бы  рассказать молодому поколению Киевлян  о том тяжёлом
времени, как очевидец тех событий. Первые рассказы у меня не получались,  так
как я раньше, кроме школьных сочинений ничего не писал. Но меня поддержали
читатели  своими доброжелательными комментариями, а так же руководство нашего
клуба « Киевские Истории», благодаря которым я теперь имею такое количество
прекрасных и искренних  друзей, а также тесную  связь с моим родным Киевом.

Слава Богу, после войны папа остался жив, во многом благодаря тому талисману,
который ему дала его мама, моя бабушка, об этом я  написал в рассказе - «
Талисман».

После войны я учился балету в Киевском Хореографическом Училище, которое в то
время находилось на углу улиц Тургеневской и Воровского. Выступал в Киевском
театре Оперы и Балета. К сожалению закончить его и продолжить заниматься
полюбившимся мне балетом  помешало то, что у меня обнаружили порок сердца (
сказались видимо тяжёлые времена, когда в детстве не хватало не только нужных
витаминов, но и вообще еды) и по рекомендации врачей мама перевела меня в 155
среднюю школу на улице Артёма №5, где так же как в Училище преподавание было на
украинском языке и  как иностранный преподавали — французский.. Увлечение и
любовь к Балету так и остались у меня  на всю мою оставшуюся жизнь. О Балете и
выдающихся танцовщиках  -  Киевлянах Сергее Лыфаре и Вацлаве Нежинском — я
написал несколько рассказов. По окончании школы я поступил в Киевский
Политехнический Институт, на факультет  «Машины и Аппараты Химических
производств»  и  по окончании его  пошёл работать в Конструкторское Бюро завода
« Большевик».  Об этом периоде своей жизни я тоже написал  подробно.  О том как
люди жили в то время, а мы прожили  более четверти века рядом со знаменитым «
Евбазом», что добавляло колорита в  нашу жизнь  и о жизни  после  сносе нашего
дома на улице  Дмитриевской, и  переезде в 1964 году на  красавицу «
Русановку».  Какой  Русановка  была в том далёком 1964  году  ( пять домов и
пески, пески, пески.) я тоже описал в предыдущих рассказах. Написал о  дружбе
между соседями с которыми жили рядышком на Дмитриевской и  потом продолжали
жить в одном подъезде на Русановке.  Уникальной эта дружба было тем, что наши
дедушки и бабушки начали эту соседскую дружбу ещё в 19 веке и продолжаем её
сейчас мы в 21 веке.

О том как мне пришлось руководить несколькими Киевскими  предприятиями я описал
в рассказах « Исповедь» И наконец, как в 1992 году из-за разгула бандитизма в
Киеве и опасности потерять не только  организованный бизнес, но и жизнь — нам
пришлось бежать к сыну в Америку, где и живём сейчас. Как мы уже в Америке
построили магазин  Европейской  пищи, так как у американцев несколько другое
пристрастие к еде, чем у нас европейцев  и назвали его в честь нашего любимого
города  Киева — \enquote{КIEV – FOOD}.                                        

За свою долгую и интересную жизнь было всё. Были взлёты и падения, но всегда
оставалась любовь к любимому Киеву — городу в котором родился и прожил более
полувека.

Промчалась жизнь - как кинолента, как мексиканский сериал.

Не всё себе я доказал, но ценность жизни я познал.

ЛЮБЛЮ, СМЕЮСЬ, СЕРЖУСЬ и ПЛАЧУ

За хвост хочу поймать УДАЧУ.

БОЮСЬ, НАДЕЮСЬ, ВЕРЮ, ЖДУ и радуюсь — ЕЩЁ ЖИВУ

Уходят годы всё быстрей, их не догнать и не вернуть.

Я ощущаю всё острей необратимости сей путь.

Шепчу годам « не торопитесь, ведь мне уже не 25.

Вы отдохните, отдышитесь, а я могу и подождать».

Но годы, будто сговорились подальше от меня сбежать.

Им говорю « Остановитесь, но не хотят они стоять».

Кричу годам своим »не смейте, не убегайте от меня»

Они проносятся как ветер, их только провожаю Я.

Уходят годы, как ночной покой уходит с по заранок,

Как в небо лёгкий дым печной или как кровь из рваной раны.

Уходят, как в песок вода или же как песок меж пальцев.

Как при болезни иногда вдруг из костей уходит кальций.

Уходят годы из под ног, как палуба во время качки.

Как, хлопнув дверью, за порог уходят женщины — гордячки.

Уходят раз и навсегда — как от звезды фотоны света.

Во мраке тая без следа, как догоравшая  ракета.

Порою тягостно, ненастно и горько наше бытиё

Но всё же жизнь друзья прекрасна. Давайте выпьем за неё!!!

Позвольте поздравить всю нашу дружную команду «Киевских Историй» с наступающим
Новым  ( Старым) Годом!!!

Вы — из разных поколений, интересов, увлечений. 

Всех люблю Вас — как родных. Вы верны и бескорыстны.

Терпеливы и чисты.... Мне видней со стороны!!!
