% vim: keymap=russian-jcukenwin
%%beginhead 
 
%%file 22_11_2021.fb.miheev_vladislav.1.zavorozhennost_maidan.cmt
%%parent 22_11_2021.fb.miheev_vladislav.1.zavorozhennost_maidan
 
%%url 
 
%%author_id 
%%date 
 
%%tags 
%%title 
 
%%endhead 
\zzSecCmt

\begin{itemize} % {
\iusr{Влад Войцеховский}

Суть поста: человечишко - знай свое место, маленькое говно! Шустер и Гордон
тебе в помощь...

\begin{itemize} % {
\iusr{Владислав Михеев}
\textbf{Влад Войцеховский} суть поста в буддийской истине: сердце мудрого человека есть деяние без борьбы! )
Хотя маленькое говно, согласен, должно знать свое место. Хотя бы потому, что место повыше занимает, как правило, большое говно)

\iusr{Влад Войцеховский}
\textbf{Vladislav Mikheev} Жизнь не буддийский монастірь. И как ты быстро разделил людей на два сорта говна) Это не по-буддистски)

\iusr{Владислав Михеев}
\textbf{Влад Войцеховский}
Я просто перешёл на твою терминологии) нас Будда учил, что можно разными путями достить просветления)

\iusr{Влад Войцеховский}
\textbf{Vladislav Mikheev} Слава Будде, что я на твою терминологию не переходил. Так и оскотиниться недолго - или по-твоему: достичь просветления)

\iusr{Владислав Михеев}
\textbf{Влад Войцеховский} вот и не надо, должен же кто-то оставаться гностическим светочем среди оскотинившегося мира )

\iusr{Валентина Гаврилова}
\textbf{Влад Войцеховский} 

Слепой не увидит, глухой не услышит... Каждый воспринимает поданную информацию
соответственно своей нозологии, т. е. патологии (глухой/слепой и т. п.). И это
тоже имеет право на существование... Увы...

\iusr{Влад Войцеховский}
\textbf{Валюша Гаврилова} 

Тобто вы оставляете каждому право воспринимать информацию только соответственно
нозологии/патологии? Воспринимать информацию через интеллект, опыт, чувства -
это вы, стало быть, отвергаете?) Так у кого тут нозология/патология?

\iusr{Валентина Гаврилова}
\textbf{Влад Войцеховский} 

Нет. Вы снова всё гипертрофировали, позволив своим эмоциям, вычленить не суть,
а \enquote{найти виноватых} якобы в чем-то Вас обвинить или унизить...Ну Вы же сами
обозначили утвердительно/осудили (что не позволительно, на мой взгляд) цитирую:
\enquote{Суть поста: человечишко - знай свое место, маленькое говно!} Что это, если не
провокация/манипуляция и ...Далее я свое мнение высказала. Имею право на СВОЁ
мнение! Т.к. в посте Владислав Михеев ничего о \enquote{человечешках} и \enquote{говне} я
лично не увидела. А о нозологии/патологии-это профессиональное считывание... Не
эмоционируйте. Вредно.

\iusr{Влад Войцеховский}
\textbf{Валюша Гаврилова} 

Умение закрывать глаза на очевидное и пытаться закрыть их другим - это
неоценимо!) Искренне желаю вам жить среди таких, как вы

\iusr{Валентина Гаврилова}
\textbf{Влад Войцеховский} 

Да это Ваше \enquote{очевидное} !!! Что ж Вы навязываете то, что видите Вы и только
Вы???!!! Ваша правда и зрячесть - это только ваше \enquote{зрение} ЛИЧНОЕ!!! Поэтому
меня и возмутило ваше вколачивание \enquote{человечишки } и \enquote{говно}. Оставьте себе
Ваше личное жизнеощущуние!!! Это и есть \enquote{нозология/патология} на моём
профессиональном языке.

\iusr{Влад Войцеховский}
\textbf{Валюша Гаврилова} 

Не стоит так нервничать) Понятия не имею о вашем \enquote{профессиональном языке}, но
на моем языке профессионального медийщика пост Влада - это сплошь подмена
понятий, игра слов и манипуляции. Как и ваши комменты. Только вы пишете не так
искусно, как Михеев и, в отличие от сабжа, похоже, сами искренне верите в
написанное)

\iusr{Валентина Гаврилова}
Нужно слушать, чтобы понять, а не слушать, чтобы ответить... Вот в чём различие между нами...
\end{itemize} % }

\iusr{Евгений Вагнер}
Ну слава Богу.... Ожил...
А то я думал: ну как же Владислав \enquote{день гидности} пропустит..)))

\begin{itemize} % {
\iusr{Олег Климов}
\textbf{Евгений Вагнер} сам-то не забыл)))

\iusr{Евгений Вагнер}
\textbf{Олег Климов} та не... я такое на всю жизнь запомнил...

\iusr{Олег Климов}
\textbf{Евгений Вагнер} Ты не один)
\end{itemize} % }

\iusr{Станислав Шпиль}

А вот я, если и завороженный, то не майданом - этим аутодафе испанского стыда,
а беспрецендентной ковидной глобализьмой... До сих пор под впечатлением от
масштабов операции... Особенно умиляет, что для наших сервильных хуторянских
экспердов, такой темы и проблемы вообще не существует...

Такие лапусики... Все о своём... О национальном. О зе.. И конечно о ненавистном
пу...

\iusr{Baolisy Yalishanta}
Далеко ли ты от песчинки? Фразы грамотно строить, возможно тоже революция, но не та, Vuladico)

\iusr{Сергей Безгин}
То торнадо, а то убитые люди

\begin{itemize} % {
\iusr{Владислав Михеев}
\textbf{Сергей Безгин} именно поэтому использована эта метафора - ураганы, землетрясения и цунами убивают людей

\iusr{Сергей Безгин}
\textbf{Владислав Михеев} совершенно разные вещи - стихия, в которой гибнут люди, и спланированные насильственные действия с целью погубить

\iusr{Владислав Михеев}
\textbf{Сергей Безгин} 

диалектически это все очень условно. Более того социальные и экономические
явления носят иногда стихийных, а иногда и циклический характер. Участвует в
них и человек своей проектной, плановой деятельностью. О степени это участия
можно споорить, но это именно степень. Классическая диалектика случайности и
необходимости и тд и тп.

\iusr{Сергей Безгин}
\textbf{Владислав Михеев} 

для меня это не тонкая, а толстая и ясная красная линия- планирование нанесения
человеку ран, увечий-это сознательное нарушение не только этических, но и
социальных законов, ведущее к разрушению предыдущей жизни. При этом я не
пацифист, я, скорее, миротворец, и хочу, чтобы и внуки пожили в мире и теплоте


\iusr{Сергей Безгин}
А Майдан и был демонстрацией - мы перешли все границы и забыли все законы. И
понеслось. Не только на Украине понеслось

\iusr{Владислав Михеев}
\textbf{Сергей Безгин} с этим-то я как раз согласен, но могут ли лемминги себя контролировать, когда несутся в пропасть? )

\iusr{Сергей Безгин}
\textbf{Владислав Михеев} их кто-то направил. Найти и - Нюрнберг. С неизбежностью. Иначе нам к миру не вернуться
\end{itemize} % }

\end{itemize} % }
