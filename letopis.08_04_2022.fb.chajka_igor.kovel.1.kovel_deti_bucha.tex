% vim: keymap=russian-jcukenwin
%%beginhead 
 
%%file 08_04_2022.fb.chajka_igor.kovel.1.kovel_deti_bucha
%%parent 08_04_2022
 
%%url https://www.facebook.com/chaykakovel/posts/382689163863019
 
%%author_id chajka_igor.kovel
%%date 
 
%%tags 
%%title У Ковелі діти виклали іграшками слово БУЧА...
 
%%endhead 
 
\subsection{У Ковелі діти виклали іграшками слово БУЧА...}
\label{sec:08_04_2022.fb.chajka_igor.kovel.1.kovel_deti_bucha}
 
\Purl{https://www.facebook.com/chaykakovel/posts/382689163863019}
\ifcmt
 author_begin
   author_id chajka_igor.kovel
 author_end
\fi

У Ковелі діти виклали іграшками слово БУЧА.... 

Сьогодні в нас день жалоби. 

Буча - місто-партнер, місто-побратим Ковеля з 2001 року. 

З перших днів війни воно опинилося буквально у вирві біди разом із Гостомелем,
Ворзелем,  Бородянкою та іншими містами та селами країни. 

Стало українським містом-мучеником. І з 25 березня — \enquote{Містом-героєм}.

Тут велися не просто бойові дії. 

Тут не просто армія воювала з армією. 

\ii{08_04_2022.fb.chajka_igor.kovel.1.kovel_deti_bucha.pic.1}

Тут сталося те страшне, від чого холоне кров: озброєні окупанти, нелюди -
цілеспрямовано, ницо і з особливою жорстокістю знищували мирне,  цивільне,
беззахисне населення. 

Тільки тому, що це - українці.

Чоловіки. 

Жінки. 

Діти. 

Мер Бучі Анатолій Федорук вже заявив, що у практично 90\% загиблих -  кульові
ураження, а не осколкові.

Це - геноцид. 

Знищення українців у центрі Європи в XXI столітті. 

\ii{08_04_2022.fb.chajka_igor.kovel.1.kovel_deti_bucha.pic.2}

За дітей болить невпинно. Цілодобово... 

У Бучі виявлено три місця, де окупанти скидали тіла людей із зав’язаними
руками. 

І одне з них - дитячий табір «Променистий».... 

Сьогоднішній день жалоби за загиблими в Бучі від рук російського окупанта - це
наш спосіб розділити цей біль з усією країною. 

Так, Буча - це наш спільний біль! 

Це велика туга за загиблими дітьми. 

І тому сьогодні на Площі Героїв Майдану, на знак скорботи і пам'яті, ковельські
діти виклали слово \enquote{Буча}... іграшками.

По завершенні громадянської панахиди виступив свідок тих подій, місцевий
священник отець Сергій Мірошниченко. Зізнався, що коли побачив сьогодні слово
\enquote{Буча}, то на хвильку ніби побував вдома. Між тими людьми, яких добре
знав, з якими багато спілкувався. І які, бездиханні, потім лежали на вулицях ще
зовсім недавно прекрасного куротного містечка Буча, про яке всі знали, що воно
поблизу Києва. 

Говорив сьогодні й повторюю знову — ми маємо подвоїти допомогу Бучі. І сьогодні
ми на зв`язку з мерією, маємо список потреб, готуємо до відправки. 

... Україна переможе і загоїть свої рани.

Але Україна ніколи цього не забуде! І не дозволить світові закрити очі на ці звірства. 

Помста і покарання за невинно пролиту кров - неминучі. 

Вічна пам'ять загиблим! 

Слава Україні! 

\#буча \#день\_жалоби \#буча\_ковель \#містогерой
