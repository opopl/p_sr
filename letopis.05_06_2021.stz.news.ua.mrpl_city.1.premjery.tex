% vim: keymap=russian-jcukenwin
%%beginhead 
 
%%file 05_06_2021.stz.news.ua.mrpl_city.1.premjery
%%parent 05_06_2021
 
%%url https://mrpl.city/blogs/view/pro-premeri-na-stsenah-mariupolskih-teatriv
 
%%author_id demidko_olga.mariupol,news.ua.mrpl_city
%%date 
 
%%tags 
%%title Про прем'єри на сценах маріупольських театрів
 
%%endhead 
 
\subsection{Про прем'єри на сценах маріупольських театрів}
\label{sec:05_06_2021.stz.news.ua.mrpl_city.1.premjery}
 
\Purl{https://mrpl.city/blogs/view/pro-premeri-na-stsenah-mariupolskih-teatriv}
\ifcmt
 author_begin
   author_id demidko_olga.mariupol,news.ua.mrpl_city
 author_end
\fi

Початок червня ознаменувався виходом низки унікальних і дійсно цікавих вистав.
Саме про них я і хотіла розповісти в цьому блозі, оскільки наразі дуже уважно
слідкую, за всіма подіями, що відбуваються в театральному житті міста.
Насправді сьогодні Маріуполь перетворився на справжню театральну столицю
Приазов'я і тепер кожен глядач може обрати виставу на свій смак.

\ii{05_06_2021.stz.news.ua.mrpl_city.1.premjery.pic.1}

Якщо ви хочете побачити легку виставу з вишуканими декораціями і відправитися в
минуле та дізнатися, про що мріяли і переживали діти наприкінці XIX ст., вам
варто відвідати виставу \textbf{Театру юного актора \emph{\enquote{Ввечері третього дня}}}. Режисерка
вистави \emph{\textbf{Ірина Анатоліївна Руденко}} довгий час шукала п'єсу, розраховану на
середній вік дітей, перечитала багато матеріалу і вирішила поєднати декілька
творів (розповідей, віршів і монологів) Аркадія Аверченка та Надії Теффі для
створення нового унікального спектаклю. Дія відбувається в одній із дворянських
садиб, адже раніше не було оздоровчих таборів, батьки привозили всіх дітей в
садибу і залишали на гувернантку та нянечку. Ось що відбувається, коли діти
залишаються наодинці з нянею, і розповідається у виставі. Для юних акторів це
незвичний, але дуже цікавий досвід. Завдяки режисерці вони більше дізналися про
минулі часи. Прем'єра вистави \enquote{Ввечері третього дня} відбулася 4 червня. Вдруге
спектакль покажуть 13 червня.

\ii{05_06_2021.stz.news.ua.mrpl_city.1.premjery.pic.2}

Ще одна прем'єра зацікавить всіх, хто полюбляє театр абсурду. Це дипломна
вистава актриси Донецького академічного обласного драматичного театру
(м. Маріполь) \emph{\textbf{Альони Горгоц}} трагікомедія \emph{\textbf{\enquote{У відкритому морі}}} – п'єса польського
драматурга Славомира Мрожека. Сюжет, на перший погляд, простий: троє елегантно
одягнених чоловіків дрейфують на плоту у відкритому морі. Дивом переживши
корабельну аварію, вони залишилися практично без провізії. Коли їстівні запаси
добігають кінця, перед джентльменами постає питання, як вижити. Недовго
думаючи, вони вирішують... А що саме, можна буде дізнатися під час перегляду.
Прем'єра відбулася 21 травня. Другий показ відбудеться 6 червня в ЦСМ \textbf{\enquote{Готель
Континенталь}} (\textbf{ПК \enquote{Молодіжний}}).

\ii{05_06_2021.stz.news.ua.mrpl_city.1.premjery.pic.3}

Що означає \enquote{бути собою?}, як, проживши життя, залишитися вірним своїм
ідеалам та не зрадити власному \enquote{я}?! Відповіді на ці  питання 14 червня
шукатиме на сцені \textbf{Маріупольської камерної філармонії} герой прем'єрної
вистави Пер Гюнт.  Прем'єра вистави відбулася 31 травня. Ідея поставити таку
виставу в Маріуполі належить директорці департаменту культурно-громадського
розвитку  \emph{\textbf{Діані Володимирівні Тримі}}. Режисером стала
\emph{\textbf{Ірина Анатоліївна Руденко}}, одним із завданням якої було
скоротити п'єсу.  Драма \emph{\textbf{\enquote{Пер Гюнт}}} була написана
норвезьким драматургом і поетом Генріком Ібсеном в середині ХІХ сторіччя. У
творі розповідається про драматичні пригоди і життєві  випробування  звичайного
сільського хлопця Пера Гюнта. \emph{\textbf{Михайло Загреба}} виконує головну роль. Він веде
багато діалогів, що ускладнило його завдання. Проте Михайло вважає, що цей
образ йому дуже підходить і радіє що отримав унікальну можливість співпрацювати
з \emph{\textbf{камерним оркестром \enquote{Ренесанс}}}.

Наприкінці XIX століття видатний норвезький композитор Едвард Гріг написав
музику до цієї п'єси, що надало цьому твору ще більшої популярності.  Для
камерного оркестру \enquote{Ренесанс} робота над цією виставою – дуже цікавий досвід.
Спектакль обов'язково залишиться в репертуарі оркестру.

\ii{05_06_2021.stz.news.ua.mrpl_city.1.premjery.pic.4}

Про наймолодших мешканців міста театральні працівники не забули. У \textbf{Донецького
академічному обласному драматичному театрі (м. Маріуполь)} підготували космічну
казку \emph{\textbf{\enquote{Прибульчик}}}. Це друга казка режисери-постановниці \emph{\textbf{Євдокії Тіхонової}} на
маріупольській сцені. Вона вже ставила казковий спектакль \enquote{Їде Святий Миколай}.
Історія про міжгалактичну дружбу сподобалася і акторам, які з інтересом та
ентузіазмом віднеслися до нової вистави. Фантастичні пригоди для дітей на дві
дії створив український режисер, актор і драматург Євген Тищук. Задіяні актриси
дуже активно брали участь у створенні пісень, віршів і танців для вистави.
Відмінність цієї казки від інших полягає не лише в космічному сюжеті, але й в
тому, що її  можна буде побачити не тільки на сцені театру, а й на інших
дитячих майданчиках міста, оскільки вона задумана як виїзна. 

\ii{05_06_2021.stz.news.ua.mrpl_city.1.premjery.pic.5}

Якщо ви любите українську класику, драматичні історії кохання та готові
поринути у світ фантастичних істот, то вам варто побачити драму-феєрію \textbf{\emph{\enquote{Лісова
пісня}}}, режисером якої є \emph{\textbf{Анатолій Миколайович Левченко}}. Ця вистава присвячена
150-річчю від дня народження Лесі Українки і поставлена силами учнів \textbf{Першої
театральної школи-студії}. П'єса \enquote{Лісова пісня} не втрачає своєї актуальності і
знайде відгук в аудиторії, адже вистава у колективу вийшла дуже сучасною. До
того ж вистава представлена як перше фентезі в світі, адже тут є своя
суперчетвірка героїв: е водяник, лісовик, русалки та духи лісу. Прем'єра
вистави відбудеться 5 червня. Наступні покази анонсуватимуться пізніше.

\ii{05_06_2021.stz.news.ua.mrpl_city.1.premjery.pic.6}

Для любителів детективів та непростих сюжетів варто переглянути виставу \textbf{\emph{\enquote{Гра}}}
\textbf{Театральної артілі \enquote{Драмком}}. Прем'єра відбулася 22 та 23 травня. Спектакль
\enquote{Гра} поставлений за мотивами п'єси британського журналіста і автора
детективних розповідей Ентоні Шаффера. Літній аристократ, процвітаючий автор
детективних романів, який звик сприймати життя як гру і граючи принижувати
людей, не очікував зустріти гідного суперника, здатного відповісти тією ж
монетою. І в цій непростій історії, звичайно ж, замішана жінка... У виставі,
що має гостру інтригу і в якій задіяно всього два актори (\emph{\textbf{Євген Сосновський та
Микита Армен}}), також піднімаються актуальні соціально-психологічні теми. Весь
реквізит актори виготовляють самостійно. Загалом ставити спектакль режисерці
\emph{\textbf{Наталі Гончаровій}} було непросто, але всі зусилля були того варті, адже
колективу вдалося створити захоплюючу психологічну історію з елементами
детективу. Побачити цей спектакль можна буде вже восени.

Також в червні на глядачів чекає справжній сюрприз від \textbf{Театру авторської п'єси
\enquote{Conception}}. У прем'єрному спектаклі персонажі говоритимуть трьома мовами.
Нова постановка, написана за мотивами роману Оскара Уайльда \enquote{Доріан Грей},
віднесе глядачів на кілька років в майбутнє. Варто відзначити, що в основі
сюжету вистави лежить боротьба чистоти з аморальністю. На глядачів чекає
неймовірна атмосфера, буря сильних емоцій, нестандартні режисерські рішення та
навіть кілька відвертих сцен. Цей спектакль передбачає і обмеження за віком
(18+).

Таким чином, всі охочі відпочити культурно, мають можливість обрати і улюблений
театр, і виставу, що більше припаде до смаку. Головне, не пропустити прем'єри і
знайти час, щоб їх побачити.
