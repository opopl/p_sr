% vim: keymap=russian-jcukenwin
%%beginhead 
 
%%file 27_10_2021.fb.ukrcentrmos.1.radiodiktant
%%parent 27_10_2021
 
%%url https://www.facebook.com/ukrcentrmos/posts/4515391191886191
 
%%author_id ukrcentrmos
%%date 
 
%%tags diktant,edinstvo,mova,radio,rossia,ukraina,ukrcentr.moskva
%%title Всеукраїнський радіодиктант національної єдності - Москва, Росія
 
%%endhead 
 
\subsection{Всеукраїнський радіодиктант національної єдності - Москва, Росія}
\label{sec:27_10_2021.fb.ukrcentrmos.1.radiodiktant}
 
\Purl{https://www.facebook.com/ukrcentrmos/posts/4515391191886191}
\ifcmt
 author_begin
   author_id ukrcentrmos
 author_end
\fi

Шановні друзі!

Запрошуємо Вас долучитися до участі в написанні ХХІ Радіодиктанту національної єдності.

Місце проведення: 

Національний культурний центр України у м. Москві (вул. Арбат, буд. 9/1).

Дата і час: 9 листопада, 10.00 та 17.30 (коли зручно). Вхід вільний.

Всеукраїнський радіодиктант національної єдності проводиться з 2000 року в День
української писемності та мови.

\ifcmt
  pic https://scontent-lhr8-1.xx.fbcdn.net/v/t39.30808-6/247513913_4515388181886492_8053392682393105521_n.png?_nc_cat=107&ccb=1-5&_nc_sid=730e14&_nc_ohc=_S7geMZieh8AX8XI2To&_nc_ht=scontent-lhr8-1.xx&oh=3e1606e3a3ee770c2d08666738d86dc6&oe=61804C43
  @width 0.8
\fi

Цього року автором тексту і читцем радіодиктанту буде відомий український
письменник Юрій Андрухович.

Роботи будуть прийматися онлайн та офлайн протягом астрономічної доби.
Паперового листа з диктантом потрібно надсилати за адресою: вул. Хрещатик, 26,
м. Київ, 01001. Фото чи скан своєї роботи можна надіслати на електронну
скриньку rd@ukr.radio. Текст диктанту буде оприлюднено на сайті Українського
радіо 11 листопада. Після оприлюднення тексту диктанту електронні листи
прийматися не будуть.

Для зручності організатори підготували електронну форму для надсилання робіт.
За допомогою форми свою роботу можна надіслати з будь-якого гаджета. Електронну
форму буде оприлюднено на сайті Українського радіо 9 листопада.

Усі надіслані роботи перевірить спеціальна фахова комісія. Учасники, які
написали роботи без помилок, отримають подарунки від партнерів проекту.
Результати оголосять упродовж 1-2 місяців.

Трансляція XXI Радіодиктанту національної єдності відбудеться на таких
платформах:

\begin{itemize}
  \item - Українське радіо \url{http://www.nrcu.gov.ua/}
  \item - UA: ПЕРШИЙ \url{https://tv.suspilne.media/}
  \item - ютуб-канал Українського радіо \url{https://www.youtube.com/channel/UCMEMEhlnxpkTphipw-EpFMA}
  \item - фейсбук-сторінка UA: ПЕРШИЙ \url{https://www.facebook.com/suspilne.tb/}
\end{itemize}
