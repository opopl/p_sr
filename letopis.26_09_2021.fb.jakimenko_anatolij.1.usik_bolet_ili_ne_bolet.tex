% vim: keymap=russian-jcukenwin
%%beginhead 
 
%%file 26_09_2021.fb.jakimenko_anatolij.1.usik_bolet_ili_ne_bolet
%%parent 26_09_2021
 
%%url https://www.facebook.com/permalink.php?story_fbid=5106206239406085&id=100000502778735
 
%%author_id jakimenko_anatolij
%%date 
 
%%tags __sep_2021.usik.pobeda.dzhoshua,boks,identichnost',politika,rossia,sport,ukraina,usik_aleksandr
%%title Олександр Усик: вболівати чи не вболівати?
 
%%endhead 
 
\subsection{Олександр Усик: вболівати чи не вболівати?}
\label{sec:26_09_2021.fb.jakimenko_anatolij.1.usik_bolet_ili_ne_bolet}
 
\Purl{https://www.facebook.com/permalink.php?story_fbid=5106206239406085&id=100000502778735}
\ifcmt
 author_begin
   author_id jakimenko_anatolij
 author_end
\fi

Олександр Усик: вболівати чи не вболівати?

Треба розуміти, що люди вболівають за різне. Вболівати за своє і вболівати за
красиве – це не одне і теж. Французи, італійці чи поляки також вболівали за
когось з пари Усик-Джошуа, але зовсім не тому, що когось з цих двох відчували
своїм. Їх вибір виходив з естетики. Комусь подобався бокс українця, а комусь
бокс британця. Тобто естетика боксу проти відчуття спорідненості. 

\ifcmt
  pic https://scontent-yyz1-1.xx.fbcdn.net/v/t1.6435-9/243198584_5106203386073037_5546204574642323537_n.jpg?_nc_cat=104&_nc_rgb565=1&ccb=1-5&_nc_sid=730e14&_nc_ohc=eCSWktmQpTMAX9AHqMj&_nc_ht=scontent-yyz1-1.xx&oh=27bfa1d2620d3de7cc291219278b0b74&oe=61772534
  @width 0.7
\fi

Переконаний, що всі, хто проігнорував  бій Усика з причини його
московсько-праволславної ідентичності спочатку не цікавляться боксом, а потім
уже мають претензії до московської духовності. Неприхильників боксу вболівати
за Усика могло примусити лише відчуття спорідненості. Звинувачувати людину за
відсутність почуттів не можна. Хоча можна, але це безсенсенсово. 

Уваги заслуговують ті українці, які бажали невдачі Усику у бою з Джошуа. Це уже
більш серйозна проблема, яка є загрозою для спільного існування. Для того, щоб
її розв’язати потрібно запропонувати новий концепт спрямований на досягнення
спорідненості. До певної міри концепт біідентичності долає цю проблему. До
певної міри, тому що він передбачає любов до України як обов’язкову, а любов не
лише до України як допустиму. 

На жаль в Україні є багато людей, котрі не люблять і навіть не навидять
українське. І їм концепт біідентичності не підійде. Є люди котрі нічого не
люблять. Вони просто обирають те, що їм приносить матеріальне благополуччя. Є
люди, котрі не хочуть бути біідентичними, тому що вважають це ущербною
роздвоєністю, а у людини закладена цілісність. Перелік можна продовжити.

Усик якраз яскравий приклад біідентичного громадянина. Він любить і Україну і
Росію. Тому бажати йому поразки було невірно. Це якщо прийняти українську
реальність. 

Але і це ще не все. Феномен Усика виходить за границі українського політичного
контексту. Усик як  джміль, який за законами аеродинаміки не  мав би літати.
Просто джміль цього не знає і тому літає. Усик за законами антропології  та
логіки не мав би перемогти Джошуа. Усик був не просто менший і легший, він не
мав навичок швидкого і безпечного зближення, яким володів Майк Тайсон, не
володів технікою ближнього бою і не мав сильного удару, хоча б такого як у
Гвоздика. Але на відміну від джмеля Усик знав про цю неможливість перемогти
Джошуа. І незважаючи на це він це зробив, переробивши себе. Не доростив руки чи
ноги, на що сьогодні спрямована спортивна наука, і на що пішли б китайці, а
спираючись на існуючу фізичну основу створив потрібний функціонал. Це
фантастична здатність людини.

Усик показує всім патріотам і не дуже патріотам, що для того, аби перемогти
Росію зовсім не потрібно досягати її розмірів чи мати стільки ж грошей.
Потрібно бажання, зусилля і розуміння в якому напрямку ці зусилля докласти.
Тому навіть, якби Усик був ватник, його перемога заслуговує на ретельне
осмислення. Я б, звичайно, не вболівав би за ватника, але ігнорувати спортивний
подвиг – це дуже нераціонально і короткозоро.

\ii{26_09_2021.fb.jakimenko_anatolij.1.usik_bolet_ili_ne_bolet.cmt}
