%%beginhead 
 
%%file 03_04_2023.fb.korovanenkova_natalia.mariupol.1.31_03_2022_blok_post
%%parent 03_04_2023
 
%%url https://www.facebook.com/natali.korovanenkova/posts/pfbid0312YgZVRZ8F7d2aNidd6poHjBsqJANYnd7XUVAcckBdPKu85paXEkGPhMMaLsq8AFl
 
%%author_id korovanenkova_natalia.mariupol
%%date 03_04_2023
 
%%tags 31.03.2022,mariupol,mariupol.war,dnevnik
%%title 31.03.2022 Блок пост
 
%%endhead 

\subsection{31.03.2022 Блок пост}
\label{sec:03_04_2023.fb.korovanenkova_natalia.mariupol.1.31_03_2022_blok_post}

\Purl{https://www.facebook.com/natali.korovanenkova/posts/pfbid0312YgZVRZ8F7d2aNidd6poHjBsqJANYnd7XUVAcckBdPKu85paXEkGPhMMaLsq8AFl}
\ifcmt
 author_begin
   author_id korovanenkova_natalia.mariupol
 author_end
\fi

\#Маріуполь\_2022 \#як\_це\_було

31.03.2022

Блок пост.

Прошло больше часа. Мы все еще стоим на выезде из хорода, нет ни одного
автобуса, мы ждем..

Страх и холод, два чувства слились вместе, единственное желание, опуститься на
землю и больше не вставать.

Машины проезжаю мимо, выезжают из

города грязные, разбитые перепуганные.

\ii{03_04_2023.fb.korovanenkova_natalia.mariupol.1.31_03_2022_blok_post.pic.1}

В город вьезжают \enquote{умытые} и холенные, подьезжает джип, выходит
женщина, ухоженная с отличным макияжем, я смотрю на нее, как на что то инородное
..

Ребята  связанные скочем, так и продолжают стоять, сын закуривает нз сигарету,
подносит к каждому  по одной затяжке.

Кто то говорит,: пройдите дальше,  там стоят частники, может договоритесь"

Хватаю собак и мы идем дальше к заправке, там действительно много частников,
увидев нас,  все на перебой: \enquote{вам куда??}

Как в лучшии времена, когда  ты приезжаешь на вокзал,   выходишь на перрон.... и
таксисты предлагают свои услуги.

\enquote{Нам в Бердянск}

Цена вопроса всего 400 уе,  Нам хватит! у меня 500..

Мы едем.

До Бердянска около 100 км, мы ехали около четырех часов, это не долго.

Семь блок постов. на  каждом одна и та же процедура, документы, выходим из
машины, личный досмотр, сына раздевают до нижнего  белья, осматривают его
синяки, у него разбита голова,  синяки на лице, куртка вся в крови, все время
надо было обьяснять, что это последствия  завалов.

Его спасло наверное то, что он ехал со мной, с матерью, 

Еще и собаки отвлекали на себя внимание.

На всех блок постах, кроме одного были чечены.

Пост, где досмотр проводили буряты был самый жуткий, они вели себя как то
отрешонно, безразлично, но от них исходило ощущение жестокости и в этом их
безразличии чувствовалась безисходнось для нас..

Водитель на каждом блок посту давал \enquote{проверяющим}, то сигареты, то кофе, и
\enquote{благодарочку} в ноги.

Последний блок пост, мы вьезжаем в город,  водитель  помогает устроиться в дом
к хозяйке.

Оккупированный Бердянск, живет мирной жизнью, на автовакзале майорять жовто
блакитні прапори.

Транспорт ходит, магазины,  рынок, приват работает.

Сын спрашивает у меня: \enquote{ты слышишь???}

Я начинаю прислушиваться, неужели самолет?? 

Слушай: \enquote{это Город ! Звуки мирного города!}

Мы устроились с комфортом, комната, большая кухня, электропечка, а главное
обогреватель и душ, и все это за 300 гривен в день.

Хозяйка принесла нам горячий суп, пока мы ели она просто стояла и плакала.

Рынок Бердянска шумит, Хлеб! Бердянские кобасы! огромные чебуреки, все
работает! Приват, сервисные центры, где можно зарядить и пополнить телефон.

Город в оккупации, но войны здесь нет.

Идем на рынок, первым делом заходим в сервисный цент зарядить  телефон и
пополнить счет.

Местные говорят, что в городе связь плохая.

Отличная связь, вы не знаете, что такое \enquote{плохая связь}, и дай бог не узнать
никогда.

Я  не была  абонентом Kyivstar, но  только благодаря им можно было хоть  как
то  связаться в самые сложные моменты в Мариуполе, а потом в Бердянске

Щіро дякуємо хлопці!

Рынок шумит.

Меняю все свое оставшееся богатство, 100 уе в обменике,

,мне хватит на три дня прожить, тыс гривень за жилье, а остальное \enquote{прогулять} на
рынке.

Цены на рынке фантастические, бублики сушка 500 гр, пакетик кофе 25, сахар
150, а вот бердянские колбасы цена приятная, как и была.

Но сказали, что уже цены пошли на спад, неделю назад, когда был основной
наплыв из Мариуполя, было гораздо дороже.

Практически на каждом шагу встречаю знакомых с Мариуполя.  

\ii{03_04_2023.fb.korovanenkova_natalia.mariupol.1.31_03_2022_blok_post.pic.2}

Соседи! Живы!

Дима с семьей, как же мы были рады увидеть друг друга. Мы еще грязные не мытые,
голодные, а они уже выглядят \enquote{респектабельно}, они уже дней десять в Бердянске.

В сервисном уже зарядился телефон, пополнили счет, чтобы можно было сделать
звонок. Спрашиваю: \enquote{сколько с меня?}, рябята отказываются брать деньги,
говорят, что вам они нужнее, 

Спасибо)

И вот он, первый звонок. я слышу! голос Маши, говорить не могу, горло
стиснуло. все что помню, что смогла сказать: \enquote{Мы в Бердянске, мы выбрались из
города}.....


Возвращаемся к хозяйке, там остались закрыты собаки.

С хлебом и сосисками! с едой!

ТЕПЕРЬ мы  будем сутки отсыпаться..

Нам нужны силы на дорогу, через пару дней надо выехать с оккупированого
Бердянска на Запорожье.

И это, как оказалось, сделать будет не просто.

%\ii{03_04_2023.fb.korovanenkova_natalia.mariupol.1.31_03_2022_blok_post.eng}
%\ii{03_04_2023.fb.korovanenkova_natalia.mariupol.1.31_03_2022_blok_post.cmt}
