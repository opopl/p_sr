% vim: keymap=russian-jcukenwin
%%beginhead 
 
%%file 12_02_2023.fb.fb_group.mariupol.pre_war.6.nash_lyubimii_gorods.cmt
%%parent 12_02_2023.fb.fb_group.mariupol.pre_war.6.nash_lyubimii_gorods
 
%%url 
 
%%author_id 
%%date 
 
%%tags 
%%title 
 
%%endhead 

\qqSecCmt

\iusr{Maryna Holovnova}

Як гарно! Стільки снігу. Це який рік?

\begin{itemize} % {
\iusr{Люба Копейкина}
\textbf{Maryna Holovnova} январь 2022(
\end{itemize} % }

\iusr{Yuliya Varshavskaya}

Родные места, я жила на ул. Семенишина и ходила в детский сад который раньше
был в этом сказочном доме. Парк такой родной, каждый уголок знаком до боли.
Спасибо огромное за фотографии.

\iusr{Ольга Орел}

Я жила на ул. Куинджи (тогда это была ул. Артема) и ходила в этот же садик.
Интересно - здание уцелело? Как жаль исторический центр....

\iusr{Люба Копейкина}

Снимки сделаны 26 января 2022 года, был мой пост с такими словами-""Снег укрыл
белым покрывалом город и создал дивную красоту! Наслаждайтесь - ведь это все
временно!"

\iusr{Sergey Drovorub}

Чудесное место, почти священное для меня.

\iusr{Катерина Коваль}

А что с домом Гампера? Уцелел?

\begin{itemize} % {
\iusr{Vadim Melnik}
\textbf{Катерина Коваль} относительно. Разбит но восстановить можно.
\end{itemize} % }

\iusr{Наталья Бандурко}

Я жила на улице Семенишина почти рядом с парком, знаком каждый уголок. Парк
уцелел!

\iusr{Грек Толя}

Это наш парк

