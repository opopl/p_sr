% vim: keymap=russian-jcukenwin
%%beginhead 
 
%%file 02_01_2022.stz.news.lnr.lug_info.1.marochko_itogi
%%parent 02_01_2022
 
%%url https://lug-info.com/news/nedel-ya-glazami-eksperta-kontakty-pervoj-stepeni-godnye-zhenshiny-i-zalozhnik-shantazha
 
%%author_id 
%%date 
 
%%tags donbass,dnr,lnr,vojna,ukraina
%%title НЕДЕЛЯ ГЛАЗАМИ ЭКСПЕРТА: Контакты первой степени, годные женщины и заложник шантажа
 
%%endhead 
\subsection{НЕДЕЛЯ ГЛАЗАМИ ЭКСПЕРТА: Контакты первой степени, годные женщины и заложник шантажа}
\label{sec:02_01_2022.stz.news.lnr.lug_info.1.marochko_itogi}

\Purl{https://lug-info.com/news/nedel-ya-glazami-eksperta-kontakty-pervoj-stepeni-godnye-zhenshiny-i-zalozhnik-shantazha}

\begin{zznagolos}
Своим видением событий прошедшей недели, так или иначе связанных с Луганской
Народной Республикой, с ЛИЦ делится военный эксперт, общественный деятель,
подполковник запаса Народной милиции ЛНР Андрей Марочко.
\end{zznagolos}

\ii{02_01_2022.stz.news.lnr.lug_info.1.marochko_itogi.pic.1}

\subsubsection{НА ЛИНИИ СОПРИКОСНОВЕНИЯ}

Минувшая неделя для жителей Луганской Народной Республики прошла относительно
спокойно, несмотря на это обстановка на линии соприкосновения по-прежнему
оставалась стабильно напряженной.

По данным наблюдателей представительства ЛНР в СЦКК, нарушений режима
прекращения огня со стороны вооруженных формирований Украины не зафиксировано.

Украинскими диверсантами похищен военнослужащий Народной милиции ЛНР.

Всего с 00:00 часов 21 июля 2019 года режим всеобъемлющего устойчивого и
бессрочного прекращения огня со стороны вооруженных формирований Украины был
нарушен 1000 раз, из них 666 раз – после вступления в силу допмер; 42 защитника
Республики погибли, 24 получили ранения. Среди мирного населения 5 человек
погибли, 38 получили ранения, повреждено 192 объектов гражданской
инфраструктуры.

\subsubsection{КОНТАКТЫ ПЕРВОЙ СТЕПЕНИ}

В ночь с 30 на 31 декабря состоялся телефонный разговор президента России
Владимира Путина и его американского коллеги старины Джо. 

Несмотря на то, что встреча носила закрытый характер, некоторые подробности
все-таки были обнародованы. 

Пресс-секретарь Белого дома Джен Псаки заявила, что \enquote{президент Байден призвал
Россию к деэскалации напряженности с Украиной}. 

Кремль же сообщил, что Байден четко пообещал, что США не будут размещать на
Украине ударные наступательные вооружения.

Обе стороны выразили надежду на конструктивные переговоры в разных форматах,
которые запланированы на январь: в Женеве между РФ и США 10 января, на
заседании Совета Россия - НАТО 12 января в Брюсселе и 13 января в Вене на
заседании Постоянного совета ОБСЕ.

Делать какие-либо далеко идущие выводы преждевременно, но уже окончательно
стало понятно, что судьбу Украины будут решать без ее участия. 

Информационный шум, исходящий из Киева, по поводу \enquote{постоянных контактов с США}
- не более чем блеф и попытка замазать тот факт, что контакты, может быть, и
есть, но для Вашингтона они в лучшем случае третьестепенные.

Через неделю лидеры России и США будут дистанционно следить и при необходимости
контролировать очень сложный переговорный процесс, руководствуясь интересами
собственных стран, и никто не будет учитывать мнение Украины. 

Начинается выработка правил, по которым в дальнейшем придется жить. Безусловно,
это положительный момент, но не для украинцев. Потеря субъектности привела к
тому, что \enquote{нэньку} либо будут \enquote{женить} без ее согласия, либо вообще выгонят из
дому. 

Второй вариант для нас более желательный, поскольку даст возможность
переформатировать это государство, исходя из исторической справедливости. То,
что когда-то было оторвано и пришито к непонятному территориальному
образованию, должно быть возвращено, и жители Юго-Востока на это очень
надеются.

\subsubsection{ГОДНЫЕ ЖЕНЩИНЫ }

Под конец года украинские власти, видимо, пришли к выводу, что страна
недостаточно эуропеизирована, и решили укрепить равенство полов: 17 декабря
2021 года вступил в силу приказ Министерства обороны нэзалежной, согласно
которому практически все представительницы прекрасной половины украинцев должны
встать на воинский учет.

Теперь в течение 2022 года украинки в возрасте от 18 до 60 лет должны явиться в
военкомат и встать на учет. Те же женщины, которые еще не избавились от
традиционного нежелания идти в армию, должны будут заплатить штраф до 1500
гривен (порядка 4000 рублей). 

Возмущение общественности не заставило себя ждать, а Минобороны выпустило
разъяснение, которое только подлило масло в огонь домашних очагов. В
пресс-службе Главного управления разведки пояснили, что на воинский учет
планируют поставить даже беременных, кормящих и одиноких матерей \enquote{на крайний
случай}. Какое отношение разведчики имеют к мобилизации, разведка не доложила.

Зачем украинскому силовому ведомству женщины-бухгалтеры, музыканты, социальные
работники, сотрудники ресторанов и отелей, а уж, особенно, литературоведы - не
ясно. А как они представляют себе службу находящихся на поздних сроках
беременности женщин, кормящих мамочек, или тех, кто воспитывает малолетних
детей, особенно без мужей?

Абсурдность приказа признали даже в самой нэзалежной, многие склонялись к тому,
что это фейк и работа \enquote{российской пропаганды}, но, зайдя на официальные
ресурсы, с недоумением убеждались в обратном.

В сети появилась масса комментариев и роликов на эту тему, причем их число с
каждым днем увеличивается в геометрической прогрессии, забивая все
информационное пространство. 

Обойти эту тему не смог, конечно же, большой любитель переодеваться в женщин
Карамелька-Арестович. Впрочем, он всем доволен, считая, что женщины давно
добивались равенства с мужчинами и \enquote{добились наконец: одни и те же требования к
мужчинам и женщинам}.

Однако в общественном пространстве всем не до шуток: в комментариях жители
нэзалежной разбивают нововведение в пух и прах в основном на языке
\enquote{агрессора} и в матерных выражениях. 

Такую реакцию населения Украины спрогнозировать было весьма несложно, так для
чего тогда этот приказ № 313? Если использовать логику нынешнего украинского
руководства и преследуемые им цели, то причин на самом деле множество. Прежде
всего – это отвлечь внимание населения Украины от экономической катастрофы.
Теперь на кухне обсуждают не бешеную стоимость продуктов питания, космические
цены ЖКХ, безработицу и прочие негативные моменты правления Зеленского, а как
\enquote{откосить} от армии. 

Демографический кризис, который на Украине искусственно вызван действиями США с
целью сократить население, теперь получит новые перспективы развития и
возможности. Американцам нужна земля, а не аборигены. Те, кто в поисках лучшей
жизни еще не покинул Родину, будут брошены в такие условия, где о продолжении
рода вообще думать будет невозможно.

\subsubsection{ЗАЛОЖНИК ШАНТАЖА}

Пока Киев демонстрирует приверженность западным идеалам, в зоне так называемой
ООС по-прежнему действуют законы джунглей.

Украинские боевики вновь совершили акт агрессии, похитив военнослужащего
Народной милиции ЛНР. На месте захвата обнаружены следы борьбы и крови, что
свидетельствует о применении в отношении военнослужащего физической силы. 

Буквально сразу в пропагандистских украинских СМИ появилась новость о том, что
наш военнослужащий, якобы будучи в сильном наркотическом опьянении,
самостоятельно перешел линию соприкосновения и сдался в плен. Нелепая попытка
прикрыть свое преступление разбивается о массу фактов, на которые многие
обратили внимание. 

Действия украинских диверсантов полностью повторяют два предыдущих похищения:
военнослужащего Народной милиции в ноябре, а также наблюдателя от ЛНР в
Совместном центре контроля и координации Андрея Косяка в октябре. Также были
опубликованы фотографии нашего военнослужащего, где его лицо было закрыто
\enquote{смайлом}, который скрывает побои.

Мотив пленения, на мой взгляд, тоже вполне очевиден: военнослужащего планируют
использовать для дальнейшего шантажа. Украина очень хотела успеть в уходящем
году провести еще одни переговоры в Контактной группе, но наши представители
заняли четкую позицию, обозначив, что возможен только конструктивный диалог, а
болтология и встреча ради встречи никому не нужны. 

Не следует забывать о новогоднем поздравлении Зеленского, в которое нужно было
по максимуму засунуть \enquote{перемог}, которых, по сути, нет. 

Вот и получается, что так называемое \enquote{европейское государство} Украина,
которое именует республики Донбасса террористами, на самом деле пренебрегает
всеми конвенциями и нормами международного гуманитарного права, ведя борьбу
террористическими методами, весьма схожими с методами ИГИЛ.  К сожалению,
мировое сообщество сейчас слепо и глухо, но люди, способные адекватно оценивать
обстановку, прекрасно понимают, что происходит на самом деле. И я просто
уверен: правда все равно восторжествует.
