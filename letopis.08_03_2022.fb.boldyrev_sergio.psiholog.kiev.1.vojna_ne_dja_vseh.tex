% vim: keymap=russian-jcukenwin
%%beginhead 
 
%%file 08_03_2022.fb.boldyrev_sergio.psiholog.kiev.1.vojna_ne_dja_vseh
%%parent 08_03_2022
 
%%url https://www.facebook.com/sergio.boldirev/posts/3156366997968681
 
%%author_id boldyrev_sergio.psiholog.kiev
%%date 
 
%%tags 
%%title Война не для всех, и не стоит смотреть с упреком на мужчин в тылу
 
%%endhead 
 
\subsection{Война не для всех, и не стоит смотреть с упреком на мужчин в тылу}
\label{sec:08_03_2022.fb.boldyrev_sergio.psiholog.kiev.1.vojna_ne_dja_vseh}
 
\Purl{https://www.facebook.com/sergio.boldirev/posts/3156366997968681}
\ifcmt
 author_begin
   author_id boldyrev_sergio.psiholog.kiev
 author_end
\fi

Я по ходу пьесы буду здесь делать записи по возможности, ибо время идёт,
происходят события и мне важно выделять, чтобы не лопнуть.

Начну с того, что война не для всех.

В смысле, солдатом способен быть не каждый.

Воевать это не скакать на лошади с шашкой на голо с криками Ура.

Воевать это значит выживать в условиях не предназначенных для жизни.

Вот, к примеру мои бойцы шестой день к ряду живут в условиях караула. Цикл
жизни таков: 2 часа на посту в открытом поле, 2 часа в резервной группе (полный
экип, броник, оружие, каску можно снять, можно сидеть, и 2 часа сна на полу.

Итого 6 часов в цикле, 4 цикла за сутки. 6 дней без остановки.

Если хотите пойти вслед за русским военным кораблём спросите такого какой
сегодня день, число или прочую ерунду.

Всё смешалось в едином колесе времени без бытия.

Существуют только полярности: день /ночь, поел/голоден, тепло/холодно,
жив/мёртв.

Поэтому реальный солдат на войне похож скорее на бомжа в мирное время.

Грязный, небритый, пахнущий потом, в карманах россыпью патроны и конфеты, кусок
сала замотанный в тряпку, в сумке для сброса магазина кусок батона в целофане.

При этом он выживала. У него в бауле тёплые старые тряпки, каремат, если
повезло спальник. Есть минимум столовых принадлежностей.

Его куда не привези, он обустроиться и начнет жить, а раз пойдёт жизнь, то
пойдёт служба.

Ну, или наоборот. @igg{fbicon.smile} 

Поэтому, когда вы видите обгорелый вражеский танк на фото, знайте что это
просто вишенка на тортик сделанный из пота, недосыпа и мозолей.

И ещё.

Не стоит смотреть с упреком на мужчин в тылу.

Жизнь солдата не для всех. Пускай делают свою работу где лучше у них
получается.

Ваш Сержио.  @igg{fbicon.smile} 
