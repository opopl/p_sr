% vim: keymap=russian-jcukenwin
%%beginhead 
 
%%file 05_07_2019.stz.news.ua.mrpl_city.1.den_voenno_morskogo_flota_v_mariupole
%%parent 05_07_2019
 
%%url https://mrpl.city/blogs/view/den-voenno-morskogo-flota-v-mariupole
 
%%author_id burov_sergij.mariupol,news.ua.mrpl_city
%%date 
 
%%tags 
%%title День Военно-Морского Флота в Мариуполе
 
%%endhead 
 
\subsection{День Военно-Морского Флота в Мариуполе}
\label{sec:05_07_2019.stz.news.ua.mrpl_city.1.den_voenno_morskogo_flota_v_mariupole}
 
\Purl{https://mrpl.city/blogs/view/den-voenno-morskogo-flota-v-mariupole}
\ifcmt
 author_begin
   author_id burov_sergij.mariupol,news.ua.mrpl_city
 author_end
\fi

\ii{05_07_2019.stz.news.ua.mrpl_city.1.den_voenno_morskogo_flota_v_mariupole.pic.1}

После войны, еще при Сталине, было установлено несколько профессиональных
праздников: День радио, День шахтера, День железнодорожника. Праздники,
выпадавшие на летнее время года, в Мариуполе обязательно отмечались в Городском
саду так называемыми народными гуляньями. Большинство праздничных \enquote{гуляний}
мало чем отличались от обычных воскресных. Разве что перед входом в сад ставили
щит с соответствующей надписью да перед заходом солнца на эстраде какой-нибудь
лектор из Общества по распространению научных и политических знаний —
предшественника общества \enquote{Знание} — читал по бумажкам лекцию содержания,
соответствующего текущему празднику.

Другое дело — \textbf{День Военно-Морского Флота}, празднование которого проводили в
последнее воскресенье июля. Еще с утра в Городской сад из морского клуба
ДОСФЛОТ привозили экспонаты выставки. Это были водолазные костюмы, водолазные
помпы и с ними горы шлангов, щиты с образцами морских узлов, шарообразные мины,
многочисленные, с необыкновенной тщательностью изготовленные модели боевых
кораблей, нечто длинное, сигарообразное — то ли торпеду, то ли элемент
заградительного устройства. Обязательно присутствовали также штурвал, небольшой
якорь, несколько бело-красных пробковых спасательных кругов. Здесь же крутились
сбежавшиеся с близлежащих улиц мальчишки. Поглядывая на занятых расстановкой по
своим местам морских диковинок устроителей выставки — морских офицеров в
белоснежных кителях с надраенными до солнечного блеска пуговицами, в фуражках в
белых чехлах с \enquote{крабами}, мальчишки норовили все потрогать руками, а что могло
крутиться, а то и покрутить. \enquote{Морские волки} делали вид, что ничего не
замечают. В это же время через главную площадь в нескольких направлениях
развешивали флотские флаги расцвечивания, прикрепляли к фонарным столбам
портреты прославленных флотоводцев.

\textbf{Читайте также:} 

\href{https://mrpl.city/news/view/kontsert-i-vystavka-mariupol-otprazdnuet-den-voenno-morskih-sil}{%
Концерт и выставка: Мариуполь отпразднует День Военно-Морских Сил, Олена Онєгіна, mrpl.city, 01.07.2019}

На смотровой площадке над обрывом пиротехник и его помощники, предварительно
расставив деревянные рамы, прилаживали к ним длинные картонные трубки, хлипкие
конструкции из тоненьких жердочек, хитроумно соединяли все это хозяйство
шнурами. Ближе к вечеру центральная часть Городского сада постепенно начинала
заполняться празднично одетой публикой. То там, то здесь встречались на головах
мужчин флотские фуражки. Заметим, что не все их обладатели действительно имели
отношение к флоту. В послевоенное время среди молодежи особым шиком считалось
носить невесть откуда добытые форменные морские головные уборы. Кто помоложе,
толпился у выставки, где бывалые моряки с ловкостью фокусников с невероятной
быстротой завязывали и развязывали морские узлы, ученики школы ДОСФЛОТ
передавали сообщения друг другу флажным семафором. Наконец объявлялось
соревнование по перетягиванию каната между командой ребят, одетых в полосатые
флотские фуфайки, и добровольцев из публики. Можете поверить, что волонтеров
всегда было с избытком, и не всегда победа была за \enquote{морячками}. Пока зрители
громкими выкриками и свистом подбадривали соревнующихся, на эстраде
рассаживался духовой оркестр. Его концерт начинался с попурри на темы морских
песен.

Солнце постепенно скрывалось за горизонтом. Морская выставка сворачивалась.
Темнело. На фонарных столбах загорались лампы. Оркестр замолкал. Теперь все
внимание присутствующих было обращено на площадку, где во тьме лишь угадывалось
замысловатое произведение пиротехника. Внезапно электричество выключалось,
раздавался оглушительный хлопок. В небо винтом взлетала, шипя и разбрасывая по
сторонам искры, ракета-шутиха. Сначала одна, затем другая, третья. Каждая
своего цвета: красная, зеленая, лимонно-желтая. Толпа, умолкшая от
неожиданности на секунду, издавала восторженный вздох и разражалась
аплодисментами, с пуском каждой ракеты окрест разносился визг девчонок и
пронзительный посвист их кавалеров. Огненное действо набирало силу. Маленькие
огоньки быстро продвигались по путям, проложенным шнурами, запуская то три
ракеты сразу, то пять. Ракеты со свистом прочерчивали темноту неба, разрывались
в вышине на огромные цветные шары из разбегавшихся от центра цветных стрел,
разбрасывали в вышине мириады сверкающих блесток. У земли же крутились
мельницы, приводимые в движение огненными струями ракет, все больше
раскручиваясь, они образовывали ослепительные разноцветные круги. Всего
несколько минут длился фейерверк, затем запасы пороха иссякали и лишь
догорающие угольки в темноте ночи напоминали о только что завершившемся
зрелище.

Вновь включали электрическое освещение, духовой оркестр играл прощальную
мелодию, люди начинали расходиться, обменивались впечатлениями о пережитом
празднике. На этом День Военно-Морского Флота в Мариуполе завершался...

\textbf{Читайте также:} 

\href{https://mrpl.city/news/view/frantsuzskij-uikend-besplatnuyu-putevku-v-parizh-razygrayut-sredi-mariupoltsev}{%
Французский уикенд: бесплатную путевку в Париж разыграют среди мариупольцев, Анастасія Селітріннікова, mrpl.city, 05.07.2019}
