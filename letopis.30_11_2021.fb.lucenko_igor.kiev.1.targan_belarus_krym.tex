% vim: keymap=russian-jcukenwin
%%beginhead 
 
%%file 30_11_2021.fb.lucenko_igor.kiev.1.targan_belarus_krym
%%parent 30_11_2021
 
%%url https://www.facebook.com/igor.lutsenko/posts/4837067499650534
 
%%author_id lucenko_igor.kiev
%%date 
 
%%tags belarus,krym,lukashenko_aleksandr,rossia,ukraina
%%title Головний білоруський тарган назвав Крим де-юре і де-факто російським
 
%%endhead 
 
\subsection{Головний білоруський тарган назвав Крим де-юре і де-факто російським}
\label{sec:30_11_2021.fb.lucenko_igor.kiev.1.targan_belarus_krym}
 
\Purl{https://www.facebook.com/igor.lutsenko/posts/4837067499650534}
\ifcmt
 author_begin
   author_id lucenko_igor.kiev
 author_end
\fi

Головний білоруський тарган назвав Крим де-юре і де-факто російським. 

Чому тарган це робить зараз, майже на 6 років пізніше України? Ми це зробили ще
у 2015 році, у часи тоді ще не такого сивого гетьмана, коли державна компанія
''Укрінтеренерго'' підписала контракт для поставки електрики в, як було
зазначено у контракті, "Кримський федеральний округ".

В мене є дві версії. 

Перша - це те, що усім стало очевидним, що мінський формат себе віджив, і можна
перестати вважати Мінськ нейтральним майданчиком. Ердоган уже пропонує Анкару
(віртуально чи в реалі, на вибір) як посередника у перемовинах по заручниках. 

Можна Мінську далі не вдавати нейтралітет і злитися з Москвою у цьому питанні,
натомість виторгуавати преференції для себе деінде.

Друга версія - можливо, тому що тарган прекрасно знає, що Україна зацікавлена у
закупках білоруської електроенергії, і сильно кіпішувати не буде. За це слід
подякувати Ахметову, котрий не дотримав графіки накопичення вугілля і чомусь
думає, що ми за це у нього ніколи не конфіскуємо електрогенерацію (жарт).

Чи не жарт?

\ii{30_11_2021.fb.lucenko_igor.kiev.1.targan_belarus_krym.cmt}
