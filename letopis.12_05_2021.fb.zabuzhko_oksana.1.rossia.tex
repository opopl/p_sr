% vim: keymap=russian-jcukenwin
%%beginhead 
 
%%file 12_05_2021.fb.zabuzhko_oksana.1.rossia
%%parent 12_05_2021
 
%%url https://www.facebook.com/oksana.zabuzhko/posts/10159323609423953
 
%%author 
%%author_id 
%%author_url 
 
%%tags 
%%title 
 
%%endhead 

\subsection{Те головне, що треба знати про сучасну Росію}
\label{sec:12_05_2021.fb.zabuzhko_oksana.1.rossia}
\Purl{https://www.facebook.com/oksana.zabuzhko/posts/10159323609423953}

Стисло й вичерпно, на пальцях, те головне, що треба знати про сучасну Росію.
Рекомендується особливо для тих, хто не читав \enquote{І знов я влізаю в танк...} (для
тих, хто читав, нового тут буде менше, але все одно лектура корисна). Єдине, в
чім я не згодна з автором, - це в його тезі, ніби повернення КГБ до влади, вже
безроздільної й тотальної, в Росії завершилося щойно в 1999 р., коли очільником
хунти став Путін: на мою думку, це сталося значно раніше - в ході т.зв.
\enquote{конституційної кризи} 1992-93 рр., а вже після штурму Білого Дому й майже 200
загиблих КГБ (чи як там воно звалось) водив Єльцина на мотузку покірного, мов
той циганський ведмедик, і купу непрямих доказів, що саме з 1994 р. ці
нові-старі господарі Росії заходились \enquote{відновлювати СРСР}, я в своїй книжці,
зрештою, й наводжу... Але, так чи інакше, читати інтерв'ю Фельштинського (і
думати) варт, бо це те знання, яким має володіти кожен:

\url{https://www.facebook.com/felshtinsky/posts/1451801645164341}
