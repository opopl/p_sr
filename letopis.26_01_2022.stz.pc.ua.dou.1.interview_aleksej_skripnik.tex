% vim: keymap=russian-jcukenwin
%%beginhead 
 
%%file 26_01_2022.stz.pc.ua.dou.1.interview_aleksej_skripnik
%%parent 26_01_2022
 
%%url https://dou.ua/lenta/interviews/founding-eleks-skrypnyk
 
%%author_id babak_anatolij
%%date 
 
%%tags interview,programmirovanie,ukraina,ussr
%%title 82-річний Олексій Скрипник щодня програмує на С++ і ходить 6 км. Інтерв’ю зі співзасновником ELEKS
 
%%endhead 
 
\subsection{82-річний Олексій Скрипник щодня програмує на С++ і ходить 6 км. Інтерв’ю зі співзасновником ELEKS}
\label{sec:26_01_2022.stz.pc.ua.dou.1.interview_aleksej_skripnik}
 
\Purl{https://dou.ua/lenta/interviews/founding-eleks-skrypnyk}
\ifcmt
 author_begin
   author_id babak_anatolij
 author_end
\fi

Співзасновник компанії ELEKS Олексій Іванович Скрипник має 40-річний досвід у
програмуванні. Нині він у свої 82 роки щоденно по кілька годин кодить на С++, а
перед тим уранці пішки проходить шість кілометрів. Олексій Іванович розповів
нам про програмування на великих машинах у 1970-х і розробку моделі
електромережі для атомних субмарин.

\ii{26_01_2022.stz.pc.ua.dou.1.interview_aleksej_skripnik.pic.1}

Крім того, найстарший із родини Скрипників поділився спогадами, як засновували
сімейну ІТ-компанію, порівняв ринок технологій в УРСР і сучасний, а також дав
поради новому поколінню програмістів.

\subsubsection{Про кодування на машинах типу ІВМ System/360 і написання першої програми у 1980-х}

— Розкажіть, будь ласка, про свій перший комп’ютер і перший досвід, коли почали
кодити.

Якщо починати просто з програмування, то це 1973 рік. То було програмування на
великих машинах типу IBM System/360. Наш радянський аналог — М-222.

Якщо говорити про персональний комп’ютер, то було так: 1983 року мене
відправили на стажування в Москву. Там теж нормального комп’ютера не було. Але
паралельно я вирушив за програмою обміну досвідом у В’єтнам. Там уже був Apple.
Це перший персональний комп’ютер, який я побачив, програмуючи до того майже 10
років на великих машинах.

У В’єтнамі я написав маленьку програму на BASIC. Що цікаво, то був мій перший
комерційний продаж програми. Коли вже повернувся додому, колеги розрекламували
її. Програмою зацікавилися, знайшовся користувач, який оплатив її. Вона вміла
рахувати струми короткого замикання для електричної мережі. Примітивна
програмка була.

— Розкажіть про свої перші мови програмування, якими ви писали.

Першим був BASIC. Потім FORTRAN, ALGOL і Pascal. Зараз зупинився на С++ та її
варіантах. Ця мова дає змогу досліджувати необхідну на сьогодні частину
моделей.

— Які були перші проєкти?

У 90-ті я робив багато обчислень на великих машинах для України. Зокрема, для
Трипільської ТЕС ми розробляли тренажер. Тобто в оператора енергоблоку був
реальний щит, а на щиті була модель і котла, і турбіни, і електричної частини.
Моя розробка стосувалася саме електричної частини.

Однак цю модель так і не почали використовувати. Її довго реалізовували, а
через це технічна частина імітаторів дуже застаріла.

— Розкажіть про свої перші місця роботи. Де ви працювали до того, як заснували
власну компанію?

У часи Союзу я працював над проєктом моделі атомних підводних човнів. Там
найрізноманітніші програми використовували... Певною мірою це секретна частина.
Однак там усе було досить відокремлено. Тобто ми цікавимося тільки моделлю,
передаємо лише її, а як далі її використовуватимуть — нас уже не стосувалося.
Згодом також працював у Львівській політехніці доцентом.

— Пригадаєте, які зарплати тоді були?

Зарплати там обмежувалися 320 рублями. Мінімальна зарплата інженера становила
110 рублів. Тобто різниця у компенсації початківця і досвідченого програміста
була не дуже великою.

\subsubsection{Про створення ELEKS та свою нинішню роль у компанії}

— Ви, ваша дружина та син 30 років тому заснували компанію ELEKS. Розкажіть, як
це все відбувалося?

Компанія ELEKS була заснована у 1991 році для виконання замовлень великих
енергетичних підприємств. На початку 90-х IT-ринку як такого ще не було. ELEKS
фактично найстаріша IT-компанія України. Інші компанії з’явилися трохи пізніше.

Ідея заснувати власну компанію належала більше дружині. Її трохи гризло те, що
я працюю та отримую мізер. Уже тоді вона мала досвід організації побутових
підприємств — був кооператив «Сонечко». Син теж був молодий, енергійний і мав
досвід роботи в енергетиці. Я бачив, що ми здатні розв’язувати задачі,
пов’язані з енергетикою. Тож вирішили створити підприємство.

У 1996 році син Олексій, можливо, навіть випадково зустрів групу молодих
амбітних інженерів. Вони спочатку навчалися у Львівському математичному ліцеї,
а потім закінчили Львівську політехніку. У них було бажання працювати, тому все
вдало склалося. То було перше солідне розширення.

Майже з самого початку в компанії працювали троє Скрипників (два Олексії та
Євгенія Іванівна), Тарас Гречин, Петро Коновалов, Роман Павлюк, Павло Мосорін,
Юрко Солодкий та Андрій Герасіка. Трохи пізніше, але одночасно, прийшли два
політехніки — Володимир Складанівський і В’ячеслав Вороненко.

Далі почалася плідна робота. Переломним став 2000 рік. Якщо у січні 2000-го
було 13 співробітників, то вже в серпні команда налічувала 34 людини. Цікаво,
що на початку 90-х у нас ще не було градацій на Junior, Middle чи Senior. Вони
з’явилися вже ближче до 2000-х.

\ii{26_01_2022.stz.pc.ua.dou.1.interview_aleksej_skripnik.pic.2}

— Як і де шукали перших замовників?

Тривалий час у нас не було замовлень або вони були мізерними. Спочатку,
наприкінці 1990-х, шукали клієнтів через інтернет. А після перших вдалих
закордонних проєктів стали отримувати нові замовлення. Ми й досі підтримуємо
зв’язки з одним із перших замовників та ведемо його проєкт, який зазнавав
суттєвих змін з роками.

Після 2000 року в компанії вже були різні задачі: фінансові, економічні,
організаційні. Пішли замовлення з Англії, США. На початку 2000-х команда ELEKS
вперше відвідала США, а закордонні замовники приїздили до Львова. У 2001 році
компанія отримала статус Microsoft Certified Partner.

Узагалі, команда збільшувалася поступово, відповідно з’являлася потреба
розширювати офіс. Якщо 1996 року це була маленька кімната на 36 кв. м, то в
кінці 2000-го то вже був цілий поверх, а згодом і ціла будівля.

У 2003 році компанія переїхала у власний 6-поверховий офіс загальною площею
2200 кв. м. Його обладнано найновішою технікою, що дає змогу забезпечити
максимально комфортні умови праці, які відповідають міжнародним стандартам.

І так поступово ми масштабувалися до сучасного рівня. Тоді, як і нині, саме
кількість спеціалістів у компанії не була для нас першочерговою. Ми передусім
відповідаємо за якість, нам цікаві складні технологічні рішення. Отже, важливо,
щоб команда складалася з людей зі схожими цінностями, з відповідальних
професіоналів. Тепер, через 30 років, з нами вже понад 2000 фахівців. Компанія
проходила кілька криз за свою історію становлення, тому колись ми не загадували
так масштабно. Досі активно розвиваємося.

\ii{26_01_2022.stz.pc.ua.dou.1.interview_aleksej_skripnik.pic.3}

— Якою на той момент була ваша роль у компанії?

Приблизно до 2012 року я виконував функцію головного бухгалтера. Спочатку ці
обов’язки взяла на себе дружина, але її взаємодія з податковою закінчилася не
зовсім успішно.

Я знайшов маленьку програму «Фінанси без проблем», яка веде бухгалтерію для
невеликої компанії. Використовуючи ту програму, я успішно контактував з
податковою. Різні варіанти контактів були. Залежно від зміни президента,
змінювалося і ставлення податкової.

У нашій компанії я частково опікувався і питаннями ремонту офісу. Тобто
виконував і технічні, і бухгалтерські функції. Таким чином, я взяв на себе
господарську і фінансову частини керування компанією, встигаючи при цьому
писати нову версію DAKAR під Windows.

— Розкажіть про роль дружини та сина. Олексій вів повністю компанію на початку?
Тобто організаційно, шукав команду, розробників тощо?

Переважно розширення компанії йшло через Олексія і ту амбітну молоду групу
інженерів. Вони шукали й нових замовників. Мій син якийсь час працював у
«Львівенерго», а потім перейшов на постійну роботу в ELEKS. На дружині були
певні організаційні моменти.  

— Яка зараз ваша роль у компанії? Які обов’язки виконуєте?

Раніше ми з дружиною передали дітям свої частки в компанії як співвласників.
Нині я співзасновник і технічний директор компанії. Займаюся підтримкою
програми DAKAR. Це такий собі математичний монстр, який допомагає досліджувати
процеси в енергосистемі, визначати наявність нових елементів. Ми продовжуємо
наповнювати його додатковими модулями. До речі, там є й замовлення від компанії
ДТЕК, яка нині розвиває сонячні й вітрові станції.

Син вважає, що основою успішної компанії є інженерна думка, і з огляду на це
формує команди. Тож у нас є група людей з великим досвідом, і я в тому числі.
Зокрема, входжу до команди з 5–7 осіб, які ту програму підтримують уже 25
років.

У компанії діє програма визнання найкращих технічних експертів, які зробили
вагомий внесок у розвиток компанії, впровадили нові технології та ініціювали
створення нових продуктів.  

— Чи згодні ви цілком з тим курсом, за яким нині
розвивається компанія? Чи ви зробили б щось по-іншому?

Мені важко відповісти. Я зробив би, можливо, щось по-іншому, але який би це
дало результат — складно передбачити. Вважаю, що нинішній успіх компанії —
заслуга не тільки сина, а й того моноліту, який був створений спочатку. Наразі
він і є визначальним, ті перші працівники є основою всього успіху.

Молоді інженери, які приєдналися до компанії ще з самого початку, є частиною
ELEKS і досі. Зокрема, Роман Павлюк, Тарас Гречин, Павло Мосорін, Петро
Коновалов тощо. Ці експерти високого класу і є головним прикладом технічної
експертизи для спеціалістів компанії. Вони продовжують створювати та розвивати
проєкти, як і 30 років тому.

\subsubsection{Про розпорядок дня: щоденне програмування та ранкові пробіжки}

— Відомо, що ви щодня програмуєте по кілька годин. Про це розповідав ваш син в
інтерв’ю DOU. Що це дає? Які мови програмування, технології використовуєте
сьогодні?

Мова програмування — С++. Технології — інженерно-наукове програмування. Нині в
моєму розпорядженні є модель енергосистеми України або й Європи, і я можу
досліджувати процеси, які відбуваються в цих енергосистемах за певних умов. І
цю математично описану модель енергосистеми маю втілити в програму.

Оскільки енергосистеми постійно розвиваються, відбувається перехід на
нетрадиційну енергетику (сонячні, вітрові електростанції), то з’являється
потреба додати їхні моделі. І моє завдання — запрограмувати моделі цих нових
елементів. Тобто ми той комплекс, який існує вже 25 років, підтримуємо, додаємо
нові моделі.

\ii{26_01_2022.stz.pc.ua.dou.1.interview_aleksej_skripnik.pic.4}

— Чи освоюєте нові технології або мови програмування?

З 1973-го до 2000 року я це робив постійно. Багато мов освоїв. Зараз уже
бажання вивчати щось нове втратилося. Та мене С++ абсолютно задовольняє в
роботі. Для моїх цілей цілком достатньо.

— Це ви зараз бекендом переважно займаєтеся? А фронтенд пробували? Вивчати
JavaScript, нові фреймворки?

Ні, для моїх задач ті фреймворки не потрібні. Має бути потреба. От нещодавно
надійшло замовлення з Німеччини від споживачів електроенергії. Вони цікавилися,
звідки надходить енергія в їхні будинки: сонячна, атомна тощо.

Тож я створив таку програму, що дає кожному споживачу змогу визначити, від якої
станції надходить електроенергія. А саме: яка частка від теплової, сонячної,
атомної або іншої станції. Це було абсолютно нове завдання, яке ми виконали
буквально за останній місяць.

— Відомо, що ви щодня вранці проходите по шість-сім кілометрів (зі слів вашого
сина). Звідки берете сили й енергію в такому поважному віці?

Останній рік, а то й більше, я щодня регулярно проходжу по декілька кілометрів.
Приблизно шість кілометрів долаю за годину. Це досить швидко. Я встаю о 5:30, о
6:00 вже виходжу на трасу.

Щоправда, я змінив маршрут, бо траса, якою я ходив влітку, зараз о 6-й ранку не
освітлюється. Доводилося постійно зміщувати час, усе пізніше й пізніше
прокидатися. Мені це не подобалося. Вирішив, що стабільно о 5:30 ходитиму, тож
знайшов трасу, де є освітлення. І тепер уже о 7:10 я повертаюся додому, приймаю
душ, і все — готовий до роботи.

— Це вам додає сил і життєвої енергії?

Так, я добре почуваюся. Маю всі оптимальні показники. Тиск, який у мене був 140
на 80, зараз — 120 на 65. Він нормалізувався після того, як я почав регулярно
ходити. Вага була 90 кг, а зараз — 75 кг. Пульс підтримується під час ходьби в
допустимих межах, за високі показники не виходить. Зокрема, середній пульс 105
ударів на хвилину, максимум 120, коли допустимо за віком 130.

\subsubsection{Про сучасні тенденції на ІТ-ринку та поради новому поколінню програмістів}

— Що, на вашу думку, відрізняє IT-середовище (ринок, спільноту) зараз і 30
років тому?

Головна відмінність полягає в людях. Тоді здебільшого працювали науковці,
аспіранти, які хотіли знайти інструмент для розв’язання своїх дисертаційних
проблем, потім інших. Тобто це було середовище наукових працівників, інженерів,
які цікавилися галуззю. Зрозуміло, що воно було обмежене. При цьому
розв’язували більше технічні задачі, а економічними та фінансовими мало
займалися.

Сьогодні ІТ-компанія вирішує різні задачі, причому як усередині країни, так і
глобально у світі. Нині програмування займає дуже широку нішу побуту. Тому ще
одна важлива відмінність — у розширенні задач. Тепер ми з цифровими продуктами
стикаємося у повсякденному житті. Наприклад, «Дія». Сьогодні я витягую
мобільний телефон, показую, що в мене є COVID-сертифікат, і все.

Тобто ІТ-продуктів нині є набагато більше, ніж тоді. Причому навіть не в
десятки, а подеколи в сотні разів. Так само й завдань значно більше, і
кількість працівників злетіла. Айтішники посіли провідне місце за розміром
оплати праці. Це теж важливий момент.

\ii{26_01_2022.stz.pc.ua.dou.1.interview_aleksej_skripnik.pic.5}

— Ви, мабуть, не очікували, що можуть настільки зрости зарплати з того часу?

Так. Колись я під час підвищення отримував невеличку надбавку. Якщо подвійну
зарплату давали — це вже було дуже багато. А сьогодні... Так само й технології
кардинально змінилися.

— Як ви ставитеся до тих, хто йде в ІТ лише за зарплатою?

Це звичайна проблема, звичне бажання. Насамперед люди йдуть за зарплатою. Але,
можливо, паралельно з цим вони знайдуть своє покликання. Це ж чудово, якщо їхня
діяльність буде приносити, крім фінансового, ще й емоційне задоволення.

Кожен шукає щось краще. Зараз ніби трохи ліпше саме в ІТ-сфері, у розробці
нових технологій. Звісно, не всім вдасться зайти в індустрію, дехто відсіється.
Це еволюційний процес, який мусить бути.

— У вас багато внуків. Всі вони в ІТ пішли або планують?

Ні. У мене з дорослих є два внуки, які фактично пішли в ІТ. Внучка поки що в
університеті, ще не визначилася. І молодші ще в школі. Хтось піде, хтось ні.
Один внук нині розробляє «розумні» чайники, холодильники тощо. Тобто штучний
інтелект вкладають у технічні пристрої. Це теж цікава ІТ-технологія.

— Які поради можете дати сучасним спеціалістам, спираючись на свій досвід в ІТ?

Тут одна порада — працювати, шукати, сподіватися, і досягнете результату.
Прикладом може бути мій старший онук, який після закінчення університету трохи
працював у ELEKS, але хотів бути більш самостійним. Створив маленьку фірму з
пошиття м’яких меблів. Це був бізнес, який зовсім не пов’язаний з
програмуванням. А нині він має потужну ІТ-компанію, яка певним чином навіть
співмірна з ELEKS. Тобто він працював, шукав і знайшов.

Тож основна ідея — знайти те, що цікаво і приносить прибуток. Тому що, крім
цікавості, ще є добробут, сім’я.
