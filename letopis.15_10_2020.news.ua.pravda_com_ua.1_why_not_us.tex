% vim: keymap=russian-jcukenwin
%%beginhead 
 
%%file 15_10_2020.news.ua.pravda_com_ua.1_why_not_us
%%parent 15_10_2020
%%url https://www.pravda.com.ua/columns/2020/10/15/7270005/
 
%%endhead 

\section{"А чому не ми?": якою має бути українська стратегія розвитку? }
\label{sec:15_10_2020.news.ua.pravda_com_ua.1_why_not_us}
\url{https://www.pravda.com.ua/columns/2020/10/15/7270005/}

"А чому не ми?" --- це гасло має стати нашою національною ідеєю. 

Багато хто засуджує українців за бажання постійно порівнювати нашу ситуацію з
іншими, але я бачу у цьому ключ. Ми можемо ставити амбітні цілі і досягати їх
швидше і краще, ніж це роблять інші. А чому ні? 

Кожна держава --- це перш за все ідея, яка лежить у її основі. Яскрава метафора,
яка пояснює мільйонам людей, а для чого ми всі тут, власне, зібралися? Для чого
існує наша окрема незалежна держава? Чому ми готові віддавати за неї свої
життя? 

Сучасна міжнародна політика функціонує насамперед як арена для конкуренції
таких ідей. Хто запропонує кращу метафору, той завжди й перемагає у
довгостроковій перспективі.

Напевно найвідоміша і найкраще артикульована ідея --- "американська мрія".
Америка --- країна можливостей, де кожен своєю працею здатен здобути те, чого
бажає і на що заслуговує. Кожен може стати self-made person, потрібно просто
докласти зусиль. 

Німеччину всі знають як країну, що живе за принципом "Ordnung muss sein". Порядок має бути. Законність має бути дотримано. Правила повинні працювати. Якщо ми про щось домовляємось, це має бути саме так, як ми домовились. Ти можеш розраховувати на те, чого очікуєш. 

Найкращий вислів, на мій погляд, який описав би ідею Росії, сказав імператор
Олександр ІІІ: "У Росії лише два союзники --- армія та флот". Цим принципом можна
пояснити і російську дипломатію, і внутрішню політику. Не довіряти нікому і
нічому, крім мови сили, чекати від інших симетричної відповіді. Одвічний стан
закритої "фортеці в облозі". 

А якою є наша ідея? Якою є місія і ціль України в сучасному світі? На жаль, ми
досі не маємо однозначної відповіді.

Ремонтувати дороги, лікарні, школи, будувати нові об’єкти інфраструктури?
Вступити до НАТО та ЄС? Повернути Крим і Донбас? Цілі чудові і ми обов’язково
їх досягнемо. Але що ми будемо робити далі? Цього очевидно недостатньо в
довгостроковій перспективі.

Багато хто намагається шукати відповідь у нашому минулому, часто надмірно
героїзуючи його. Так, у нас історія, яка заслуговує на повагу. Так, вона
сповнена героїчних сторінок. Ми повинні її знати і ми можемо нею пишатися. 

Але давайте будемо чесними, ми не можемо собі дозволити, як італійці, спиратися
на спадщину Риму чи доби Відродження, не можемо спочивати серед пам’яток
античності і Візантії, як греки, не можемо сперечатися про те, хто більше
заслуговує бути похованим в паризькому Пантеоні, як французи. 

Так, ми не можемо перетворити нашу країну на один великий музей минулої величі.
Але чи не є це нашою перевагою? Наше велике минуле ще попереду --- у майбутньому. 

Українці здатні на неймовірні речі. Погодьтеся, багато проєктів радянських
часів, які зараз вражають світ своїм масштабом, робилися в більшості українцями
– від освоєння цілини до запуску першої людини в космос. Це робили руки і
голови наших батьків. Українці вміють йти і по протоптаній стежці. Вони успішні
в Канаді, в Європі і де завгодно. Чому тоді не можуть досягати таких же
результатів тут, в Україні?

Так, ми пройшли дуже нелегкі роки. В час, коли більшість з нас чекала хоча б
якоїсь зарплати, не говорячи про її розмір, було не до того, щоб думати про
великі ідеї. 

В час, коли на нас напала одна з найсильніших армій світу, нам було не до того,
щоб думати про далеке майбутнє. Було складно, але ми встояли і ми будемо. В
цьому вже не може бути сумнівів. Настав час, коли ми можемо і повинні ставити
собі амбітніші цілі.

Багато хто критикує українців за бажання постійно порівнювати своє становище із сусідами. Чому у нас не так, як у Польщі, Європі, США? Але я не бачу у цьому нічого поганого. Навіть навпаки, на мою думку, в цьому ключ до нашої місії.

Давайте на рівні держави поставимо питання: а чому не ми? Чому не ми запускаємо
людину на Марс? Чому не до нас їдуть навчатися європейські студенти? Чому не ми
приймаємо наступну Олімпіаду? 

Давайте ще більше дивитися, а що хорошого у сусідів і у світі, і запитувати
себе: а чому не у нас? Що заважає нам це зробити першими і зробити краще? 

Якщо будувати дороги, то найкращі. Якщо вступати в союз, то щоб очолити його.
Якщо робити щось, то робити краще за інших. 

Наша перевага в тому, що за нами немає якогось чіткого шлейфа. Світ досі мало
знає про те, що таке Україна та хто такі українці. Це наша стартова позиція. І
цим варто користатися.

Україна --- країна можливостей. Тут роблять неможливе. Так говоритимуть про нас у
світі в найближчому майбутньому. Потрібно лише докласти зусиль і здорової
заздрості. Тим паче і одне, і інше у нас є.

Настав час бути тими, хто нав’язує світу нові правила гри. Ну а чому не ми? 
