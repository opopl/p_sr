% vim: keymap=russian-jcukenwin
%%beginhead 
 
%%file topics.kiev.poezia.pestun_maksim.idet_ohota_na_doma
%%parent topics.kiev.poezia
 
%%url 
 
%%author_id 
%%date 
 
%%tags 
%%title 
 
%%endhead 

\url{https://stihi.ru/2021/09/14/3618}

Идет охота на дома
Максим Пестун
Идет охота на дома, идет охота
На сердцу милые места из детских снов.
На беззащитный особняк у поворота
И каждый метр наших стареньких дворов

Ревут бульдозеры и башенные краны.
Они, как танки на войне, готовы в бой.
Родному городу они наносят раны,
И оставляют лишь руины за собой

Стоит стеной титушек строй у горсовета,
Стоят у каждого объекта за бабло.
Вы чьи ребята? Только нет у них ответа...
Как нету совести, что продана давно.

Сидят застройщики в высоких кабинетах,
И планы новые готовят в тишине.
Что украдут они у нас на тех советах,
Увидим мы по телевизору в огне.

Все на продажу! Без разбора и без счета.
И старый сад, усадьба или луг...
Они вдруг исчезают из учета,
Как на войне погибший старый друг

Пройти боятся центром киевляне,
Где детство их и молодость прошли.
Чтоб сохранить хоть в памяти экране
Места, где были счастливы они.
