% vim: keymap=russian-jcukenwin
%%beginhead 
 
%%file 27_06_2021.fb.fb_group.story_kiev_ua.1.basta_repper_kiev
%%parent 27_06_2021
 
%%url https://www.facebook.com/groups/story.kiev.ua/posts/1693942914135836
 
%%author_id fb_group.story_kiev_ua,majorenko_georgij.kiev
%%date 
 
%%tags basta.rossia.repper,gorod,kiev,kievljane
%%title ВАСЯ, ШУЛЯВКА И БАБИЙ ЯР
 
%%endhead 
 
\subsection{ВАСЯ, ШУЛЯВКА И БАБИЙ ЯР}
\label{sec:27_06_2021.fb.fb_group.story_kiev_ua.1.basta_repper_kiev}
 
\Purl{https://www.facebook.com/groups/story.kiev.ua/posts/1693942914135836}
\ifcmt
 author_begin
   author_id fb_group.story_kiev_ua,majorenko_georgij.kiev
 author_end
\fi

ВАСЯ, ШУЛЯВКА И БАБИЙ ЯР.

Хочу сказать пару слов о человеке, который хотя и не родился в Киеве, но имеет
к нашему городу непосредственное отношение. Это достаточно известный репер Вася
Вакуленко, которого многие знают под псевдонимом Баста. 

\ifcmt
  tab_begin cols=3

     pic https://scontent-frx5-1.xx.fbcdn.net/v/t1.6435-9/207544622_2675123786120179_8056156996412601645_n.jpg?_nc_cat=111&ccb=1-5&_nc_sid=b9115d&_nc_ohc=Zgue-3nO17oAX_-_owY&_nc_ht=scontent-frx5-1.xx&oh=5c243fb02fb0179ec3c834091f186f99&oe=61B2D17A

     pic https://scontent-frt3-1.xx.fbcdn.net/v/t1.6435-9/209053926_2675123812786843_4043145761934177220_n.jpg?_nc_cat=102&ccb=1-5&_nc_sid=b9115d&_nc_ohc=qNn4M_E7KnkAX_dBBm8&tn=lCYVFeHcTIAFcAzi&_nc_ht=scontent-frt3-1.xx&oh=8d114d7777cffafb0f20ff8e7a8e17ac&oe=61B19C80

		 pic https://scontent-frx5-1.xx.fbcdn.net/v/t1.6435-9/206868578_2675123862786838_6998598899041566538_n.jpg?_nc_cat=111&ccb=1-5&_nc_sid=b9115d&_nc_ohc=3iID12h_pBwAX8Qe_F_&_nc_ht=scontent-frx5-1.xx&oh=5bcf7b727594bc39d0a62b4c2e42a251&oe=61B2CB46

  tab_end
\fi

С Киевом его связывает
то, что по линии бабушки он киевлянин не менее чем в трех поколениях. Поколения
эти проживали на Шулявке, на улице Борщаговской и доля им выпала непростая. На
детство его бабушки выпала Великая Отечественная, на детство прабабушки -
революция, гражданская. Прапрадед не дожил до смутных времен, погиб в
железнодорожной аварии в 1910 году. Много горечи познали киевские предки
Василия, но духом не падали, пытаясь найти свое счастье. Некоторое время эта
семья проживала и на Предмостной Слободке. И хотя Слободка до революции была
предместьем, народ побережья жил - не тужил! Кругом фруктовые сады, пляжи,
дачники на лодках, свежее молоко! Парк "Венеция", трактирЪ, театральные
представления. Не зря на месте Слободки потом создали знаменитый Гидропарк!
Место "намоленное".

В 1912 году семья вернулась обратно на Шулявку, которая хоть и считалась
окраиной, но народ там проживал необычайно колоритный! Крестьяне, мещане,
дворяне. Конечно же, студенты, пролетариат с Завода Гретера и Криванека. Я уже
молчу о знаменитой шулявской босоте!

В те старинные времена и в речке Лыбеди рыба водилась, и пацаны плавали по ней
на плотах. Центром культурной жизни Шулявки была Марие-Магдалиновская церковь,
где скрепляли венчанием брачные союзы, крестили детей. На слуху были уважаемые
фамилии жителей Шулявки - Куличенко, Рутенко, Янковские, Павловские, Габеры,
Балтушевичи. Шулявка жила одной большой семьей!

Люди ходил к друг другу в гости, крестили детей, вместе отмечали праздники...
Коснулись далекой родни Василия Вакуленко и события революции, гражданской и
Великой Отечественной. Слом привычного образа жизни, обыски, конфискации,
гибель близких. А самой печальной была доля родной сестры прабабушки Василия -
Анны. Ее за связь с подпольем расстреляли в Бабьем Яру.

Конечно же, Вася посетил это место, трагическое и святое для каждого
киевлянина.

И хотя родился Василий в Ростове на Дону а сейчас проживает в Москве, Киев для
него - родной город и на его вчерашних киевских концертах был аншлаг. Вася
любит Киев, киевляне любят Васю!

\ii{27_06_2021.fb.fb_group.story_kiev_ua.1.basta_repper_kiev.cmt}
