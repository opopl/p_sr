% vim: keymap=russian-jcukenwin
%%beginhead 
 
%%file 21_01_2022.stz.bib.dnr.doneck.nauchnaja_krupskoj.1.kvest_den_studenta
%%parent 21_01_2022
 
%%url http://www.lib-dpr.ru/index.php?link=news&num=7932
 
%%author_id bib.dnr.doneck.nauchnaja_krupskoj
%%date 
 
%%tags biblioteka,dnr,donbass,doneck,kvest,molodezh,studenty
%%title Библиотека – территория молодежи
 
%%endhead 

\subsection{Библиотека – территория молодежи}
\label{sec:21_01_2022.stz.bib.dnr.doneck.nauchnaja_krupskoj.1.kvest_den_studenta}
 
\Purl{http://www.lib-dpr.ru/index.php?link=news&num=7932}
\ifcmt
 author_begin
   author_id bib.dnr.doneck.nauchnaja_krupskoj
 author_end
\fi

\begin{zznagolos}
Квест ко Дню студента в рамках Года молодежи
\end{zznagolos}

21 января в Донецкой республиканской универсальной научной библиотеке имени
Н. К. Крупской состоялся увлекательный квест «Библиотека – территория молодежи»,
приуроченный к российскому Дню студента и в рамках Года молодежи в ДНР. В игре
принимали участие студенты экономического факультета Донецкого национального
университета. Ребята разделились на две команды. Им предстояло пройти шесть
локаций, где их ждали интересные задания, связанные с биографией и творчеством
Фёдора Михайловича Достоевского – русского писателя, мыслителя, философа и
публициста, произведения которого пользуются популярностью среди молодых
интеллектуалов.

\ii{21_01_2022.stz.bib.dnr.doneck.nauchnaja_krupskoj.1.kvest_den_studenta.pic.1}

На первой локации, в «Доме Раскольникова», студенты прошли проверку на знание
экранизаций великого писателя. Далее маршрут привел участников в условный
«Михайловский замок». Здесь ребят ждали вопросы о жизни Фёдора Михайловича. Из
множества представленных фактов биографии писателя ребятам нужно было выбрать
только правдивые.

\ii{21_01_2022.stz.bib.dnr.doneck.nauchnaja_krupskoj.1.kvest_den_studenta.pic.2}

В общей картине Петербурга Достоевского Сенная площадь – концентрированное
выражение всех пороков городской цивилизации, символ предельного унижения
человеческой личности. На этой локации студентам было предложено проверить свои
знания не только о Достоевском, но и о других русских классиках.

\ii{21_01_2022.stz.bib.dnr.doneck.nauchnaja_krupskoj.1.kvest_den_studenta.pic.3}

«Дом старухи-процентщицы» расположился в музее библиотеки, где среди реальных
экспонатов нашли свое место предметы, характеризующие известных персонажей из
романов Достоевского. В «Доме Сони Мармеладовой» наши гости должны были
вспомнить знаменитое Пятикнижие и разобраться в его действующих лицах. Ребята
показали хороший результат, распределив всех героев верно.

«Троице-Измайловский собор» – конечная станция путешествия. Именно здесь в 1867
году писатель обвенчался с Анной Сниткиной, ставшей его преданной помощницей и
впоследствии публикатором писателя. В финале квеста студентов ждали иллюстрации
к некоторым книгам Фёдора Михайловича.

Обе команды полностью справились с поставленными задачами, и все-таки чуть
более внимательной оказалась команда первого курса, которая и стала
победителем.

Квест получился увлекательным и интересным! Ребята смогли с пользой провести
время и отвлечься от студенческих будней. 
