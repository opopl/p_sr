% vim: keymap=russian-jcukenwin
%%beginhead 
 
%%file 18_01_2022.tg.bilchenko_evgenia.1.rodina
%%parent 18_01_2022
 
%%url https://t.me/bilchenkozhenya/5753
 
%%author_id bilchenko_evgenia
%%date 
 
%%tags bilchenko_evgenia,chelovek,psihologia,rodina,rossia,rusmir,zhizn
%%title БЖ. Как обретают Родину: урок выдержки
 
%%endhead 
\subsection{БЖ. Как обретают Родину: урок выдержки}
\label{sec:18_01_2022.tg.bilchenko_evgenia.1.rodina}

\Purl{https://t.me/bilchenkozhenya/5753}
\ifcmt
 author_begin
   author_id bilchenko_evgenia
 author_end
\fi

БЖ. Как обретают Родину: урок выдержки

Надо немного саморефлексии. Я изменилась. Когда меня в очередной раз накрывает
депрессией, я вспоминаю, что меня посвятили в семью. Меня приняли те, о ком я
мечтала, и это всё - свое, сердечное, русское, родное, без всякой задней мысли
говорю, без политики, без идеологии.  Бывают дни, когда я снова начинаю себя
глупо жалеть, за всё вместе: за отсутствие работы, за больное тело, которое
никак не хочет поправляться, за страх будущего. Больше всего меня пугает
лечение и что по болезни я потеряю рабочую активность и не смогу на это лесчние
заработать. Именно вот меня преследует картина смерти от болезни, а не военной.

\ii{18_01_2022.tg.bilchenko_evgenia.1.rodina.pic.1}

Но, в отличие от себя прежней, я уже немного умею с этим бороться. Питер меняет
людей: он делает их суровее, жёстче, севернее, что ли, убирает глупую суету и
заставляет привычно, безропотно сжимать губы в борьбе с бытом. И мне это стало
нравиться. Мужества мне никогда было не занимать, но я преодолевала свой страх
показной истерией и риском, я реально так ощущала, а теперь - все больше
молчанием и терпением. И ещё - я внутренне свободна. Я реально говорю, что
хочу, и, если мне что-то не нравится, тоже говорю. Никто не унижает меня так,
как на прежней Родине. 

Хотя Родина у меня одна - Русский Міръ, это даже на красивом дипломе написано,
и не буду я бояться этих двух слов. Как вам новое старое начертание?
\enquote{Міръ} означает \enquote{лад}, \enquote{собор}, \enquote{гармонию
Вселенной}. Так красноречивее, на мой взгляд, и не вызывает ассоциаций с клише,
которого все боятся. Я, вот, на боюсь искажений слов и не предпочитаю молчать о
последних словах, дабы избавить их от банальности. Я перестала бояться быть
банальной. Могу себе это позволить.

Все равно порою мне очень страшно, даже с этим новым одухотворением. Но мне уже
начало хотеться чего-то: делать какие-то стримы, писать какие-то книги, - после
Киева ведь сразу ничего не хотелось, только - всем \enquote{на ручки}, а сейчас уже
хочется, но сил пока (уже?) нет. 

Вообще, легче стало, когда я решила не гнаться за реализацией. Когда я
поставила на себе крест, я внутренне освободилась. И научилась видеть маленькие
радости жизни вокруг себя. Я принимаю тот факт, что, возможно, никогда больше
не буду преподавать и никогда не стану официально признанным писателем. Зато я
- свободна! Не так, как в политике, а именно всем духом - свободна.

У меня в жизни появилось много такого, о чем на Украине нельзя было и
помыслить: милые гости на праздники, детские ритуалы с бытом, горячие булочки
под самым домом, постоянные прогулки по старинным сказочным местам, встречи с
интересными единомышленниками, счастье любимого человека, частые небольшие, но
яркие для меня, литературные выступления и, вообще, - чувство покоя. И иногда -
счастья. Я реально обрела Родину. Семью. Большую семью своего мира через "і".

У меня остался просто страх, что болезни сломят меня. А так - ничего. И я
стараюсь не думать о том, что будет завтра: просто делаю каждый день какое-то
мелкое нужное дело или молча терплю боль. Конечно, надо снова идти к врачу, но
меня охватывает детская паника. Это нескончаемый поток: недугов, лекарств,
материальных средств и сил. Но я так рада, что я нашла Родину: мои враги пишут,
что я себя уговариваю, но мне на самом деле есть, с чем сравнить. Мне не
страшно больше, я не напряжена так, как раньше. 

Меня песни советской эстрады в кафе, как ребенка радуют. Если бы не тело,
ощущала бы себя на двадцать лет. Вот такие пироги. 

Кому из мигрантов или репатриантов помог этот пост саморефлексии? Недаром
писала? Пишите в личку. Вообще, цивилизационная и духовная Родина - это очень
круто, за это можно все отдать, я знаю. Очень глупые люди думают, что Родину
выбирают за статус. Это для русских - немыслимо. Вообще так настоящий русский
человек (не в смысле крови, в смысле милосердия и доблести) не думает. Точно!
