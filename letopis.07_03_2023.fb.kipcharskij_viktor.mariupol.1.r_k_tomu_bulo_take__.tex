%%beginhead 
 
%%file 07_03_2023.fb.kipcharskij_viktor.mariupol.1.r_k_tomu_bulo_take__
%%parent 07_03_2023
 
%%url https://www.facebook.com/permalink.php?story_fbid=pfbid0G2Wynrim4RyVH8dg735Sz9sV75uRnqiTmamSnW29VR9JePjQgfCVu2Uk12FkDokVl&id=100006830107904
 
%%author_id kipcharskij_viktor.mariupol
%%date 07_03_2023
 
%%tags mariupol,mariupol.war,dnevnik
%%title Рік тому було таке:  День 12 - 07.03.22. Понеділок
 
%%endhead 

\subsection{Рік тому було таке:  День 12 - 07.03.22. Понеділок}
\label{sec:07_03_2023.fb.kipcharskij_viktor.mariupol.1.r_k_tomu_bulo_take__}

\Purl{https://www.facebook.com/permalink.php?story_fbid=pfbid0G2Wynrim4RyVH8dg735Sz9sV75uRnqiTmamSnW29VR9JePjQgfCVu2Uk12FkDokVl&id=100006830107904}
\ifcmt
 author_begin
   author_id kipcharskij_viktor.mariupol
 author_end
\fi

Рік тому було таке: 

День 12 - 07.03.22. Понеділок. 

3:15. Тихо. У щілину між шторами щось світле: підсвічено будівлі госпіталю до
кута меблевого. Не міг зрозуміти що то таке, поки світло не почало пересуватися
- то була машина на Блажевича.

3:45. Поодинокий постріл?

4:02. Постріл та розрив?

%\ii{07_03_2023.fb.kipcharskij_viktor.mariupol.1.r_k_tomu_bulo_take__.pic.1}

4:18. Відкриваю двері гаражу та бачу на підлозі мокрі відбитки ніг та колеса. У
паніці "Викрали Дев'яточку!' ... прокидаюся.

4:20. Біглий вогонь. Це звичайно - це заспокоює (тремтіння вікон - заспокоює?)!

6:45. Як гріти їжу без світла та газу? За допомогою запальнички? Піти в гараж
шукати паяльну лампу, яку вже шукав та не знайшов. А може вона із 2014-го року
у підвалі? А якщо спалити бензин у лампі, то на чому поїдемо, якщо буде
коридор?

Варіант 1 - закріпити банку чи  кастрюлю на стільчику-каркасі з дроту?
Ненадійно. Буде хитатися.

Варіант 2 - гвинтом прікріпити банку до палиці, яку затиснути у лещата?

Варіант 3 - переможець: на низу духовки встановити закріплену у лещатах
запальничку, а над нею - на решітці для тостів - металеву банку з водою чи
каструлю з рідкою їжею. Трохи  нагрів воду у металевій кружці та запарив собі
вівсянку. Поділився ноу-хау з сином - він приготував вівсянку собі і онукам.
Запальничка гріє слабо та економно.

Десь о шостій сусіди почали звідкись носити пиляні дрова, з дитячого садочка
навпроти - скамійки та іншу деревину. Пізніше сусід сказав, що поцупили дрова з
якогось складу, незважаючи на сторожа - йому сказали: "Нам дітей треба
годувати!"

У дворі навпроти парковки почали облаштовувати вогнище.

Я зробив "намаз" - намазав ногу Ліотромбом та замотав обидві ноги еластичними
бинтами, поверх яких натягнув гетри.

З Олексієм та Толиком (ровісники наших дітей - кажуть на мене Дядько чи
Батько!) пішли до АТБ, де "позичили" металеві перекладини від стелажів, аби
зробити з них решітку на вогнище. Все це під постійним "гупанням". Принесли.
Дорогою у Толика через проблеми з хребтом затікали ноги. Він час від часу
опирався ношею на дерево і відпочивав. Під вогнище викопали заглиблення,
виклали стінки бордюрними блоками, на які поклали принесені нами перекладини.
Порубали гілки, розпалили.

Накіп'ятив кілька разів воду у чайнику та трилітровому металевому бідоні ще з
Кандалакші  (мій ровісник!), розлив по термосах, заніс додому. Сусіди варять
картоплю, яйця, смажать котлети - їдять те, що без холодильника може
зіпсуватися. Перший під'їзд влаштував собі окреме вогнище - другий, третій та
четвертий готують на іншому.

11:20. Ми перекипятили бульйон, нагріли суп, зварили бурак. Невістка з онуком
вийшли на вулицю - малу поклали спати. Дем'ян кидав у вогонь гілочки,
фотографував вогнище. Підійшли свати - невістка дуже за них непокоїлася.
Розповіли, що їх знайомого після 17-ї пограбували, загрожуючи пістолетом. На
Покришкіна у будинку,  де колись була молочна кухня, уламками скла посікло
хлопця років 30-ти і він стік кров'ю. Його поховали у дворі і поставили хрест.

Син приніс воду від гаражів. Аби не переплутати набрану з колодязю (для їжи) з
"річковою" (технічною), підписали пляшки...

Новини з приймача: Харкову та Маріуполю присвоїли звання міст-героїв. Зелений
коридор вкотре не погодили. ОБСЄ з Маріуполя втікло - їм тут надто небезпечно.
О 16:00 почнеться новий раунд перемов.

Онука співає джаз (дідусь поганому на навчить!): "Па-па-пі-ду!" та розмальовує.
Онук втикає у смартфон. Я на дивані "рівняю" спину та даю відпочинок ногам.

Втомився. Не записав, що було ввечері...

Фото дня робила Оля.

%\ii{07_03_2023.fb.kipcharskij_viktor.mariupol.1.r_k_tomu_bulo_take__.cmt}
