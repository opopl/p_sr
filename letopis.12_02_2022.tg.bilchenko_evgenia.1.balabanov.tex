% vim: keymap=russian-jcukenwin
%%beginhead 
 
%%file 12_02_2022.tg.bilchenko_evgenia.1.balabanov
%%parent 12_02_2022
 
%%url https://t.me/bilchenkozhenya/6082
 
%%author_id bilchenko_evgenia
%%date 
 
%%tags balabanov_sergej.rezhisser.rossia,bilchenko_evgenia,kiev
%%title БЖ. Живой Балабанов: как я люблю существовать
 
%%endhead 
 
\subsection{БЖ. Живой Балабанов: как я люблю существовать}
\label{sec:12_02_2022.tg.bilchenko_evgenia.1.balabanov}
 
\Purl{https://t.me/bilchenkozhenya/6082}
\ifcmt
 author_begin
   author_id bilchenko_evgenia
 author_end
\fi

БЖ. Живой Балабанов: как я люблю существовать.

Вчера день зашкалил по количеству творческих проектов. Прямо из музея Рерихов,
с чудесного вечера киевского закрытого почти канала, посвященного героям
Великой Отечественной Войны и жизни Олеся Бузины, поехала помогать
девушке-поэту Алисе вести вечер памяти Алексея Балабанова к Олегу Красное
Солнышко Гаркуше. Приехала не злая, как подумала чудесная автор проекта Алиса,
а с плотно сомкнутыми губами, с какими всегда веду политику. Приехала с
киевлянами, которые чтут Бузину. 

\ii{12_02_2022.tg.bilchenko_evgenia.1.balabanov.pic.1}

Какая я ещё должна была приехать? Я ужасно болею корнями по Киеву до оккупации.
И белой завистью завидую тем киевлянам, которые живут на грани жизни и смерти,
в вечной опасности, не от болезни, а как я до выезда. \enquote{Я тоже хочу!}

Олег Алексеевич Гаркуша был, как всегда, светоносный, и мне удалось раскрыть
балабановскую тему через себя, через Андрея Мединского (Севастополь), через
Александра Г. (Киев). Я читала стихи, мои пацаны пели о войне и всякое жесткое
и навылет, в духе \enquote{Брата} и \enquote{Груза 200}.

За вечер спасибо хозяйке - поэту Алисе Mirror Рудо. Спасибо ей, что терпит меня
и даже читала мои стихи про то, как ополченцы и атошники мочат друг друга без
пояснений трепетной клубной публике.  Ещё раз прошу прощения, что опоздала: у
меня одно тело, причем больное сильно, и вчера оно прижималось со всей
эмигрантской дрожью к последним украинцам в своей жизни.

Также вчера, встретившись у Гаркуши с бывшим помощником депутата из Одессы,
уехавшим после 2 мая 2014 года и с целой кучей храбрых киевлян, которые рискуют
всем и сидят со мной, я поняла, что я окончательно живу жизнью героев фильмов
Балабанова и что мне это нравится. И только это делает меня живой. Вообще, у
меня было дежавю: мне показалось, что я в Праге. Та же ночь, те же разговоры на
подтекстах, те же визитки в кармане и то же счастье тайны, как будто это я
рискую и возвращаюсь в Киев.

Обожаю вас, храбрые киевляне! Никаких ваших фамилий и наших фото. Свое
выступление я также выложу в видео. Видео из рериховского музея ниже. Огромное
спасибо моим зрителям, пришедшим на меня и моего Данилу Багрова.

А сейчас падаю: отшастала больная "Минскую инициативу", балабановшину и вечер
киевских соратников за три дня. Сил нет.

Сила в правде (с).
