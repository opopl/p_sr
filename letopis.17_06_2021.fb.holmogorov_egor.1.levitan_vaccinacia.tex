% vim: keymap=russian-jcukenwin
%%beginhead 
 
%%file 17_06_2021.fb.holmogorov_egor.1.levitan_vaccinacia
%%parent 17_06_2021
 
%%url https://www.facebook.com/holmogorov.egor/posts/10226999191071249
 
%%author Холмогоров, Егор
%%author_id holmogorov_egor
%%author_url 
 
%%tags covid,levitan.sssr.diktor,mentalitet,narod,proval,rossia,russkie,vaccinacia
%%title Если положение серьезное - включайте Левитана. Если Левитана нет, то Страна Огромная в очередь за вакциной не встанет
 
%%endhead 
 
\subsection{Если положение серьезное - включайте Левитана. Если Левитана нет, то Страна Огромная в очередь за вакциной не встанет}
\label{sec:17_06_2021.fb.holmogorov_egor.1.levitan_vaccinacia}
\Purl{https://www.facebook.com/holmogorov.egor/posts/10226999191071249}
\ifcmt
 author_begin
   author_id holmogorov_egor
 author_end
\fi

В чем ошибка всей провалившейся кампании по вакцинации? 

Она вся строится на индивидуалистической либеральной идеологии: важен каждый,
драгоценна каждая жизнь, твое решение, я уже вакцинировался (дальше показывают
холеную рожу какого-нибудь артиста-миллионера).

Западный человек понимает все правильно: Генеральная Линия приказала привиться
и у нее не забалуешь. А русский человек по своей природе индивидуалист, а после
эпохи коммунизма стал хитрым, скрытным, злым и упорным индивидуалистом.

Если вы апеллируете к его индивидуалистическим ценностям, то он и реагирует как
индивидуалист: если ему надо в его индивидуальной логике - он прививается, если
не надо - не прививается. И физиономией Машкова вы его не переубедите.

В России работает только неиллюзорный патриотический подъем. То есть когда вы
зовете русских что-то сделать за веру, царя и Отечество, то русские делают.

Если бы было сказано: у нас демографическая катастрофа. Выпилились 700 тысяч за
год, больше 300 тысяч внепланово. Спаси нацию от катастрофы, привейся и
соблюдай правила - это бы сработало. Люди бы стояли в длинных очередях в пункты
вакцинации, для того, чтобы послужить Отечеству (в отличие от либероты я не
вижу в службе ничего плохого, это форма нашей самоорганизации).

Государство у нас понимается не как продавец услуг в сфере порядка для
индивидов, а как защитник холистического целого в рамках которого индивид
только и возможен. И его \enquote{послание} должно быть обращено только к защите этого
целого. А для работы с индивидуальным нужны другие механизмы.

И когда холистический механизм не включается, русский справедливо полагает, что
вопрос рассматривается государством не как действительно жизненно важный. Ну
как если бы во время войны висели плакаты: \enquote{Ищем добровольцев. Я
записался в десант - Даня Моргенрот}. Ну а если власть сама не уверена - нужно
ли это или нет, то с чего бы индивиду быть уверенным. 

Обвинять людей что они не бегут вакцинироваться по первому призыву Машкова -
странно. Они все-таки ждут Левитана.

Если положение серьезное - включайте Левитана. Если Левитана нет, то Страна
Огромная в очередь за вакциной не встанет.

\emph{Андрей Василич}

Интересное мнение. Только на Западе (по крайней мере, в Америке), массово
побежали вакцинироваться, потому что: 1) были испуганы; 2) надоело носить
маску; 3) «посоветовали» работодатели; 4) это бесплатно, в отличие от остальной
медицины. Как-то так.

\emph{Холмогоров Егор}

Андрей Василич
Это и называется Генеральная Линия посоветовала

\emph{Андрей Василич}
Ну, пусть будет так.

\emph{Юрий Гусаков}

Но всё же существует презумпция здравого смысла. Привиться это рационально. Это
защищает от болезни. Выходит, у нас в стране рационально мыслящих людей всего
10\%?

\emph{Сергей Жегло}
Русские не садаются. Поклоняйтесь своему глобалистскому тараканищу, только русскими после этого не называйтесь.

\emph{Svetilkin Lanna}
Браво! ...ЗЫ...как же вы с Красовским по-разному смотрите на русского человека

\emph{Igor Golikov}

У людей страх перед вакциной больше, чем страх перед болезнью.

Ведь многие считают, что это \enquote{обычный грипп} или вообще \enquote{лохотрон}.

Ну, и выстрел Михалкова против вакцины Билла Гейтса оказался настолько сильным, что сдетонировал по всем вакцинам.

Массовое сознание не отличает западные вакцины от российских. Они слышат только одно слово: \enquote{вакцина}.

И весь негатив о западных вакцинах, о той же AstraZeneca, который смакуют наши
СМИ, автоматически переносится на ВСЕ вакцины, в том числе и наши, российские.

Ещё причиной являются диссидентские настроения части медперсонала,
подогреваемые популярными среди медиков интернет-сайтами, группами и
сообществами в соцсетях, непонятно, кем управляемыми.

Нужно навести порядок в информационной сфере.
У людей просто истерическое неприятие вакцинации.
За этот год населению промыли мозги разными способами.
Одним рассказали про жидкий чип и печать антихриста.
Другим - что у них будет импотенция/бесплодие.
Третьих припугнули инфарктом/инсультом/тромбозом вследствие вакцинации.
Четвертым - что вакцину \enquote{придумали жиды} (\enquote{Вакцина Гинцбурга}!!!)

Провинциальных врачей зацепили за их амбиции (\enquote{мало ли что там, в
Москве, академики говорят! у нас в городе что, врачи - дураки???}) И т.д.

\emph{Igor Golikov}

Пример того, как работает массовое сознание. Разместил найденную в ленте инфу,
что американская лаборатория в Ухани принадлежит Pfizer, производителю одной из
вакцин.  Народ увидел только одно слово: \enquote{вакцина}.  И стали писать,
что они не будут вакцинироваться.  То, что российские вакцины к Pfizer и Биллу
Гейтсу отношения не имеют ну ни малейшего, массового обывателя не интересует.
Они читают только \enquote{вакцина}, и всё.  Особенность восприятия толпы.
Толпа не видит нюансов.

\emph{Мария Белоусова}

Я тоже вчера глянула краем на унылое ровно в телике (там ролики с новостями
чередовались, Белы нитки видно) и подумала внезапно, что это была главная
причина. Да, много анти-роликов, всего такого, но средний гражданин не
потребитель конспирологии. Он просто говно не любит. В ролике была истерика
Евгения Миронова (типа лайф, звонок уважаемому худруку) про сограждан, которые
толпами отдыхают. Никогда не слышала ничего более бездарного... Актёр один из
лучших. Гнев директриссы сельской школы.

\emph{Leonid Naumov}

Мне кажется, что главная причина в недоверии. Люди не верят себе, от себя не
ждут ничего хорошего, поэтому и \enquote{другим} (\enquote{государству},
\enquote{науке} и т.д.) не верят. Да, согласен, что сейчас выход в
\enquote{мобилизации}. \enquote{Левитан} не \enquote{звал на фронт
добровольцев}, а говорил с \enquote{мобилизованными}. Совершив поступок человек
начинает приобретать веру в себя, а дальше уже и к другим

\emph{Andrey Lysenko}

Леонид Наумов Себе люди как раз верят. Люди не верят государству РФ и его
пропаганде. И правильно делают.

\emph{Leonid Naumov}
Андрей Лысенко Вам виднее)))

\emph{Bovt Georgy}

Стоять в очереди, чтоб послужить Отечеству? В нынешней России? Смешно, ей-богу.
Только приказ, штраф и дубина. А с причинами провала частично согласен:)

\emph{Виталий Пенской}
Георгий Бовт У Вас есть другая, альтернативная Россия, которой Вы готовы послужить?

\emph{Bovt Georgy}
Виталий Пенской хочется, видимо, подъебнуть, но не знаете как:)

\emph{Karen Tsaturyan}

А есть примеры успешных кампаний последнего времени, выстроенных по
предлагаемой Вами схеме, или это мнение исключительно теоретическое, пусть и
основанное на понимании русского национального характера?

\emph{Elvira Dubois-Ilina}

Провакцинальный догматизм ушел в прошлое. И слава Богу.

\emph{Alexandra Sazykina}

Все верно. Живу во Франции. Тут, конечно, другой менталитет, но вме равно. В
первую волну все упирались: нас нельзя закрывать, дома сажать - мы
свободолюбивые. Потом выступил Макрон и сказал: «у нас война! Чтобы победить,
надо сидеть дома» дословно. И всех закрыл, и все спокойно сидели по домам и
передвигались по улице короткими перебежками в масках и перчатках.

\emph{Anton Lyubavin}

\enquote{...русский человек по своей природе индивидуалист, а после эпохи коммунизма
стал хитрым, скрытным, злым и упорным индивидуалистом} - согласен полностью,
поэтому государство в его настоящем виде у нас понимается исключительно как
продавец услуг в сфере порядка для индивидов, а скорее как некий бездушный и
выхолощенный сервис. И отношение населения к вакцинации - это отношение к
государственной власти в целом.

\emph{Наталия Осипова}

У нас государство внезапно стало индивидуалистом. Добровольная вакцинация,
каждый за себя, делай что хошь - и медицину покоцало. Да и пропаганду покоцало.
Те же антипрививочные паблики - это что такое? Почему расплодились? Кто их
ведет? Про революцию боятся писать, а про вакцинацию - нет. Полный хаос. Видно,
что нет железной руки и единой политики. Да, и плюс оптимистическая ложь -
«победили вирус».

\emph{Андрей Крылов}

Этому государству служить, нынче тоже вряд-ли много желающих найдется, если
завтра война, то хрен кто пойдет, кроме самых примитивных, государство наш
враг, с недавних пор и мы ему противостоим, как можем,пока пассивно.

\emph{Дмитрий Макаров}

Андрей Крылов примитивны ли были Минин и Пожарский? А на гос.службе всегда
много приспособленцев считающих себя не примитивными.

\emph{Дмитрий Макаров}

Точно! Левитана!:)

\emph{Andrey Lysenko}

Нет, никакая \enquote{правильная} пропаганда не помогла бы. За 100 лет государственного
террора и оболванивающей пропаганды русские выработали против госпропаганды
очень сильный иммунитет. В этом отношении у нас сейчас, конечно, большое
преимущество перед западными людьми, которые оказались совершенно беспомощными,
когда их государства начали работать против них.

\emph{Eugene Fedorov}

Да, конечно, побегут колоться неизвестными растворами за веру, царя и
Отечество. Вот только тут такое дело. Колоться бегут в основном те, кто уже
никогда не даст никакого демографического прироста, а молодежь вакцинацию
активно сабботирует. И у молодежи есть для этого все основания: практически
никто среди молодых серьезно не болеет.

\emph{Николай Фигуровский}

Отлично сформулировано! Браво, Егор Станиславович!

\emph{Alexandr M-spb}

Что, если рассмотреть проблему с баранофокусом таким образом: он пришёл (и
видимо изготовлен) от западных «партнеров». Персоналии, пиарящие всю эту
проблематику - не гуманисты, а наоборот - за сокращение населения. Вектор
продвижения противобаранофокусных мероприятий - от глобалистических структур
(ВОЗ) к структурам, видимо подчиненным, у нас(РПН). Отечественные управляющие
персоналии в гуманизме тоже замечены не были, некоторые открыто и положительно
поддерживают идею о сокращении населения. Таким образом вся придумка с
«пандемией» и факцинацией рассматривается нашим народом как некое
либералистическое мероприятие, аналогичное повышению пенсвозраста. Либерализм в
чистом виде, как известно поддерживает около 10\% России, что соответствует
кол-ву уколотых. Как говорят в математике: ч.т.д.(что и требовалось доказать).

\emph{Alexander Zhiglo}

Егор, ну что ж теперь по каждому поводу петь \enquote{Вставай страна огромная?} Эта
мотивация может обесцениться от слишком частого употребления.

\emph{Evgeny Tolstikhin}

Разве западный человек не индивидуалист?

\emph{Andrey Ouzer}

интересно. Вроде более распространен взгляд, что русский человек - врожденный
общинник, а не индивидуалист. Включая советский период.

\emph{Andrey Lysenko}

Андрей Оузер Ну так именно советские товарищи этот \enquote{русский коллективизм} и
пропагандировали старательно. Чтобы загонять русских в разные колхозы и
заставлять бесплатно вкалывать \enquote{в интересах коллектива}. Но как и все
остальное, что говорят советские - это ложь.
