% vim: keymap=russian-jcukenwin
%%beginhead 
 
%%file 11_08_2021.stz.news.ua.mrpl_city.1.viktor_grammatikov
%%parent 11_08_2021
 
%%url https://mrpl.city/blogs/view/viktor-grammatikov-budte-dobrishimi-odin-do-odnogo
 
%%author_id demidko_olga.mariupol,news.ua.mrpl_city
%%date 
 
%%tags 
%%title Віктор Грамматіков: "Будьте добрішими один до одного!"
 
%%endhead 
 
\subsection{Віктор Грамматіков: \enquote{Будьте добрішими один до одного!}}
\label{sec:11_08_2021.stz.news.ua.mrpl_city.1.viktor_grammatikov}
 
\Purl{https://mrpl.city/blogs/view/viktor-grammatikov-budte-dobrishimi-odin-do-odnogo}
\ifcmt
 author_begin
   author_id demidko_olga.mariupol,news.ua.mrpl_city
 author_end
\fi

Цього разу пропоную познайомитися ближче з людиною, чий життєвий і творчий шлях
не може не вразити. Це заслужений журналіст України, член Національної спілки
театральних діячів, редактор радіодайджесту \enquote{Донецьке Приазов'я}, дуже світла і
щира людина \emph{\textbf{Віктор Андрійович Грамматіков}}.

\ii{11_08_2021.stz.news.ua.mrpl_city.1.viktor_grammatikov.pic.1}

Народився наш герой під Саратовом в селі Велика Рельня. Мати – колишня балерина
в саратовському театрі. Грала в масових сценах. З мамою були близькі стосунки.
Вона дуже вплинула на творче становлення юнака. У 1947 році сім'я Віктора
Грамматікова переїхала до Маріуполя. Переїзд був пов'язаний з роботою батька.
Саме у Маріуполі хлопець закінчив школу. Найбільше подобалося вивчати
літературу. Був дуже активним і творчим учнем. Часто займався концертними
ранками. У шкільні роки Віктор працював на радіо \enquote{Шкільна Іскорка}. На великих
святах встановлювали трибуну на проспекті Леніна (сьогодні проспект Миру). Там
була машинка з мікрофонами, звідти і вели репортаж, який полюбляли слухати всі
містяни. З дитинства Віктор колекціонував знаки, які стосувалися радіо чи
телебачення. Також полюбляв збирати книги.

Цікавим і насиченим вийшов театральний шлях у Віктора Андрійовича.  Спочатку
він грав в Народному театрі ПК \enquote{Азовсталь}. Пізніше, у 1958 році,  його
запросили до драматичного театру, де він отримав роль Шурика у виставі \enquote{Два
кольори}. Режисером тоді був \emph{\textbf{Володимир Кріпець}}. Виступали в літньому театрі.
Вітя дуже хотів, щоб батьки його побачили. Приніс квитки, поклав на стіл перед
ними і наголосив, що дуже треба піти на цей спектакль. Жили недалеко від
міського саду, де і розташовувався Літній театр. Оскільки потрібно було ще й до
іспитів готуватися, Вітя зіграв свою сцену і повернувся додому вчити білети.
Прийшовши додому, батьки поділилися, що спектакль сподобався, але один юнак їм
дуже нагадав сина. Після цього наш герой дістав програмку, в якій вони побачили
прізвище: \enquote{Тарганщук-Грамматіков} (тоді ще було подвійне прізвище, пізніше
Віктор Андрійович вирішив залишити лише другу частину). Мати була приємно
здивована, а батько, який не звик ділитися власними думками, того вечора
зауважив, що у сина ця роль вийшла дуже непогано.

З теплом Віктор Андрійович згадує урочисте відкриття маріупольського театру: 

\begin{quote}
\em\enquote{2 листопада 1960 року очікуване відбулося.  Біля входу в театр молоді
актори-учасники допоміжного складу, зустрічали глядачів і прикріплювали їм на
груди жетони. На жетонах були проставлені номери, за якими після вистави
розігрувалися квитки на безкоштовне відвідування протягом місяця п'яти вистав.
Для маріупольців цей день був святковим. Настрій у глядачів був піднятий – вони
входили до \enquote{храму мистецтва}, що пахне ще свіжою фарбою і магією театру. І,
звичайно ж, у залі для глядачів, всі захоплювалися величезною кришталевою
люстрою, яка була виготовлена за спецзамовленням}.
\end{quote}

\ii{11_08_2021.stz.news.ua.mrpl_city.1.viktor_grammatikov.pic.2}

В театр оформили Віктора у допоміжний склад, там же грала і Ольга Бурцева, яка
теж починала в Народному театрі ПК \enquote{Азосталь}. Грамматікова призначили
старостою групи допоміжного складу. На той момент він отримував 40 рублів.
Оскільки в театрі не було помічника режисера, Володимир Кріпець відправив Вітю
на режисерські курси в Москву. Там він закінчив курси режисера театру ляльок.
Коли повернувся, почав грати у виставі \enquote{Четверо у одного даху}. За цю виставу
актор навіть отримав подяку, адже зіграв у  ній 100 разів. Роль була непростою:
він за кулісами озвучував собаку, але настільки правдоподібно, що ніхто з
глядачів не міг повірити, що це гавкала людина. Віктор Андрійович згадує, що в
театрі тоді пахло квашеною капустою. Під театром працював цех з капустою. Це
заважало репетирувати, але рятував дуже дружний колектив. Всі були як одна
велика родина. Завдяки активній гастрольній діяльності театру, наш герой
побачив багато міст. До 1964 року був помічником режисера. У 1964 році став
завідувачем трупи. Проте вже у 1965 році виникла неприємна ситуація з тодішнім
директором театру, яка вплинула на рішення Віті – йти в армію. Так він опинися
у Кіровограді, де була розташована військова частина. Після армії Грамматікова
запросив на кіровоградське телебачення головний режисер телеканалу Михайло
Барський. Він погодися і почав працювати асистентом режисера на телебаченні.
Згодом активний маріуполець створив при телебаченні молодіжний ансамбль. Саме в
Кіровограді Віктор Андрійович одружився.

Про Маріуполь пан Віктор не забував. І, як виявилося, про нього теж не забули.
Запропонував йому повертатися до міста Іван Васильович Коробов, який на той час
працював редактором газети \enquote{Приазовський робочий}. Грамматікова тягнуло до
рідних, тому він погодився. Почав працювати для Донецького радіо, готував
сюжети для обласних та республіканських теленовин. 

У 1982 році він переходить до студії звукозапису Азовського морського
пароплавства, де готує передачі для моряків. Заочно Віктор здобув освіту
журналіста в Ростовському університеті журналістики (1984 рік). Цікаво, що
керівником одного з розділів дипломної роботи Грамматікова був відомий
радянський актор і диктор \emph{\textbf{Юрій Борисович Левітан}}, який нашому герою
запам'ятався як дуже талановита і доброзичлива людина. В альбомі Грамматікова
зберігається і лист від Юрія Борисовича. Смерть Левітана у 1983 році для
Віктора стала справжнім ударом. Завдяки зусиллям Грамматікова і дикторської
групи Всесоюзного радіо новому радянському контейнервозу було присвоєно ім'я
Юрія Левітана.

\ii{11_08_2021.stz.news.ua.mrpl_city.1.viktor_grammatikov.pic.3}

Із вісімдесятого року пан Віктор уже працював у телецентрі з керівництвом у
Донецьку, щоправда безоплатно. З 1988 – у Маріуполі стало виходити у прямий
ефір по проводовій мережі \emph{\textbf{\enquote{Радіо Приазов'я}}}. Тоді його слухали майже всі
містяни. З кожним роком розширювалася тематика передач, збільшувалося число
зацікавлених слухачів. Хвилі \enquote{Радіо Приазов'я} несли від Бердянська до
Таганрога, від Волновахи до Єйська оперативну інформацію про життя і трудові
будні металургів, машинобудівників, хіміків, моряків, вчених, медиків,
працівників культури, спортсменів. Водночас Віктор Андрійович разом з \emph{\textbf{Богданом
Васильовичем Слющинським}} намагався відкрити у Маріуполі театр ляльок. Пізніше
до них приєдналася і \emph{\textbf{Ірина Анатоліївна Руденко}}. Самі створювали ляльок, самі
ставили вистави. Зокрема, й досі йде вистава Віктора Андрійовича \enquote{Дюймовочка}.
А незабаром ми зможемо побачити прем'єру, над якою зараз працює Грамматіков, –
\enquote{Бабусі}. Це вистава про дружбу у віршованій формі, авторка \emph{\textbf{Марія Сладкова}}.

\ii{11_08_2021.stz.news.ua.mrpl_city.1.viktor_grammatikov.pic.4}

Працездатність і ентузіазм відомого маріупольця не можуть не надихати. Він не
тільки актор, журналіст, режисер, редактор, він ще й викладач. Своїми знаннями
Віктор Андрійович залюбки ділився зі студентами Маріупольського держаного
університету. Сьогодні Віктор Андрійович Грамматіков є редактором
радіодайджесту \enquote{Донецьке Приазов'я}, де використовує матеріали різних міських
каналів з посиланням на авторів. Вважає, що наразі найкращим телеканалом
Маріуполя завдяки різноплановим програмам, прямим ефірам та унікальним
матеріалам є \emph{\textbf{Маріупольське телебачення}}. 

\emph{\textbf{Порада маріупольцям:}} \emph{\enquote{Будьте добрішими один до одного і тоді жити стане набагато легше}.}
