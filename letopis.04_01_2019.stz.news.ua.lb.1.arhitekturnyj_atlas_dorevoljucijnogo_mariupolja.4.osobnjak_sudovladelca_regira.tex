% vim: keymap=russian-jcukenwin
%%beginhead 
 
%%file 04_01_2019.stz.news.ua.lb.1.arhitekturnyj_atlas_dorevoljucijnogo_mariupolja.4.osobnjak_sudovladelca_regira
%%parent 04_01_2019.stz.news.ua.lb.1.arhitekturnyj_atlas_dorevoljucijnogo_mariupolja
 
%%url 
 
%%author_id 
%%date 
 
%%tags 
%%title 
 
%%endhead 

\subsubsection{Особняк судовладельца Регира}

В конце ХІХ века мариупольский порт стал мощной базой по перевалке угля и
зерна. Одним из крупнейших частных судовладельцев не только Мариуполя, но и
всей Российской Империи, был Пётр Регир. Остатки дома этого состоятельного
человека можно увидеть сегодня на улице Итальянской. Помпезный двухэтажный дом
начала ХХ века, с входным порталом в левой части фасада, богато украшенный
орнаментами и маскаронами, по праву занимает одно из первых мест среди
архитектурных памяток Мариуполя. В нём проживала семья Регир, а также
размещалось управление пароходством. Сразу после революции Регир-младший
перебрался за границу, а Регир-старший умер в 1919 году в Мариуполе.

\ii{04_01_2019.stz.news.ua.lb.1.arhitekturnyj_atlas_dorevoljucijnogo_mariupolja.4.osobnjak_sudovladelca_regira.pic.1}

При советской власти в здании располагалось общежитие и различные учреждения.
Последний капитальный ремонт проводился в 1968 году. После того, как особняк
покинули сотрудники управления треста \enquote{Донбасстальконструкция}, он остался
необитаем. Пережив все лихолетья двадцатого века, дом Регира разрушился за
короткий период века двадцать первого. В 2008 году краеведы писали, что дом
прекрасно сохранился, а сегодня это величественное строение стоит в руинах. Но
даже в таком плачевном состоянии особняк Регира способен произвести
впечатление.

\ii{04_01_2019.stz.news.ua.lb.1.arhitekturnyj_atlas_dorevoljucijnogo_mariupolja.4.osobnjak_sudovladelca_regira.pic.2}
\ii{04_01_2019.stz.news.ua.lb.1.arhitekturnyj_atlas_dorevoljucijnogo_mariupolja.4.osobnjak_sudovladelca_regira.pic.3}
\ii{04_01_2019.stz.news.ua.lb.1.arhitekturnyj_atlas_dorevoljucijnogo_mariupolja.4.osobnjak_sudovladelca_regira.pic.4}
\ii{04_01_2019.stz.news.ua.lb.1.arhitekturnyj_atlas_dorevoljucijnogo_mariupolja.4.osobnjak_sudovladelca_regira.pic.5}
\ii{04_01_2019.stz.news.ua.lb.1.arhitekturnyj_atlas_dorevoljucijnogo_mariupolja.4.osobnjak_sudovladelca_regira.pic.6}
\ii{04_01_2019.stz.news.ua.lb.1.arhitekturnyj_atlas_dorevoljucijnogo_mariupolja.4.osobnjak_sudovladelca_regira.pic.7}
