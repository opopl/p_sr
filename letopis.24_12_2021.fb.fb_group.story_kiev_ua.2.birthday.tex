% vim: keymap=russian-jcukenwin
%%beginhead 
 
%%file 24_12_2021.fb.fb_group.story_kiev_ua.2.birthday
%%parent 24_12_2021
 
%%url https://www.facebook.com/groups/story.kiev.ua/posts/1825491784314281
 
%%author_id fb_group.story_kiev_ua,gorbov_vadim.kiev
%%date 
 
%%tags birthday,kiev,kievljane
%%title Киевские Истории - День Рождения
 
%%endhead 
 
\subsection{Киевские Истории - День Рождения}
\label{sec:24_12_2021.fb.fb_group.story_kiev_ua.2.birthday}
 
\Purl{https://www.facebook.com/groups/story.kiev.ua/posts/1825491784314281}
\ifcmt
 author_begin
   author_id fb_group.story_kiev_ua,gorbov_vadim.kiev
 author_end
\fi

Я вспоминаю один весенний вечер. В воздухе было немножко сыро, на вокзале -
сотни людей. От чемоданов рябит в глазах. Все взволнованы - все хотят уехать. И
среди этих взволнованных нервно сидел один... Сидел он на своём деревянном
сундуке и думал горькую думу. К нему подошёл некий изящный молодой человек и
спросил: Чего пригорюнился, добрый молодец? (с). 

\ii{24_12_2021.fb.fb_group.story_kiev_ua.2.birthday.pic.1}

Примерно так как описано в киноповести Василия Шукшина «Калина Красная»
четыре года назад состоялось мое первое знакомство с фундатором и
Алминистратором Фейсбук группы «Киевские Истории» Олегом. 

Олег познакомил меня со своим буквально вот только с пылу с жару проектом
киевлян, киевоведов, любящих свой Город и память о нем, и любезно предложил
сотрудничество. Тогда эта группа насчитывала едва ли несколько тысяч
участников. 

-Зачем вам нужен автор -  сатир и сатирик? - удивлённо спросил я тогда
Администратора. Со мной хлопот не оберёшься с цензурой. Жаловаться будут на
меня граждане. Как товарищи, так дамы и господа и пани та панове. - Вы и
ваши тексты - именно то, что нам нужно  в группе в качестве баланса,
кайеннского перца и острой приправы. - Ну что же, Воля ваша, тогда по рукам.
- И ещё с вас колонка «Киевские некрополи». - Уже договорились. И есть у
меня идея как вишенка на тортике - рубрика «Прогулки с кузиной». 

И вот сегодня, в День Рождения «Киевских Историй», когла нам исполнилось
четыре года и сообщество насчитывает уже более 100 000 участников, живущих в
Киеве или разбросанных по всему земному шару, но хранящих частичку Города  в
своих сердцах, 

я поднимаю свой бокал с игристым вином производства бывшего  «Киевского завода
шампанских вин» за всех нас, за отца-основателя и за «Киевские Истории». 

P.S. На фото дружеский коллаж на Администратора и модераторов группы
«Киевские Истории» работы Вова Нестеренко.

\ii{24_12_2021.fb.fb_group.story_kiev_ua.2.birthday.cmt}
