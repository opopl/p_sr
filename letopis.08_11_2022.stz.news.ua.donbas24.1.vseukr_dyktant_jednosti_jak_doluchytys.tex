% vim: keymap=russian-jcukenwin
%%beginhead 
 
%%file 08_11_2022.stz.news.ua.donbas24.1.vseukr_dyktant_jednosti_jak_doluchytys
%%parent 08_11_2022
 
%%url https://donbas24.news/news/vseukrayinskii-diktant-nacionalnoyi-jednosti-yak-dolucitisya
 
%%author_id demidko_olga.mariupol,news.ua.donbas24
%%date 
 
%%tags 
%%title Всеукраїнський диктант національної єдності — як долучитися
 
%%endhead 
 
\subsection{Всеукраїнський диктант національної єдності — як долучитися}
\label{sec:08_11_2022.stz.news.ua.donbas24.1.vseukr_dyktant_jednosti_jak_doluchytys}
 
\Purl{https://donbas24.news/news/vseukrayinskii-diktant-nacionalnoyi-jednosti-yak-dolucitisya}
\ifcmt
 author_begin
   author_id demidko_olga.mariupol,news.ua.donbas24
 author_end
\fi

\begin{center}
  \em\color{blue}\bfseries\Large
9 листопада всі охочі українці зможуть взяти участь у традиційному всеукраїнському Радіодиктанті
\end{center}

Цієї середи, 9 листопада, у День \href{https://archive.org/details/28_07_2022.olga_demidko.donbas24.jak_shvydko_perejty_na_movu_porady}{\emph{української писемності та мови}}, 

в Україні
відбудеться традиційний Радіодиктант національної єдності-2022. Свято розвитку
державної мови було започатковане 9 листопада 1997 року Указом Президента
України Леоніда Кучми. Дата обрана не просто так — 9 листопада українці також
відзначають День вшанування Нестора Літописця — послідовника творців
слов'янської писемності Кирила і Мефодія. Про те, як приєднатися до
всеукраїнського диктанту та перевірити його, розповіло Міністертво культури та
інформаційної політики України.
