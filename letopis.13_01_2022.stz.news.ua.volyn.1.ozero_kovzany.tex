% vim: keymap=russian-jcukenwin
%%beginhead 
 
%%file 13_01_2022.stz.news.ua.volyn.1.ozero_kovzany
%%parent 13_01_2022
 
%%url https://www.volyn.com.ua/news/203133-na-volynskomu-ozeri-zmahalysia-bihuny-na-kovzanakh-foto-video
 
%%author_id news.ua.volyn
%%date 
 
%%tags 
%%title На волинському озері змагалися бігуни на ковзанах (Фото, відео)
 
%%endhead 
\subsection{На волинському озері змагалися бігуни на ковзанах (Фото, відео)}
\label{sec:13_01_2022.stz.news.ua.volyn.1.ozero_kovzany}

\Purl{https://www.volyn.com.ua/news/203133-na-volynskomu-ozeri-zmahalysia-bihuny-na-kovzanakh-foto-video}
\ifcmt
 author_begin
   author_id news.ua.volyn
 author_end
\fi

\ii{13_01_2022.stz.news.ua.volyn.1.ozero_kovzany.pic.1}

\begin{zznagolos}
Перегони відбулися у Шацьку	
\end{zznagolos}

Цьогорічний січень порадував шачан по-справжньому зимовою погодою і морозом,
який скував озера міцним льодом. Цим і скористалися активісти громадської
організації смт Шацьк «Шаковель», створивши на одному з найпопулярніших у
селищі озер – Великому Чорному – чудовий каток, йдеться на
\href{https://shsrada.gov.ua/news/1642064840/}{сайті Шацької ОТГ}.

\ii{13_01_2022.stz.news.ua.volyn.1.ozero_kovzany.pic.2}

12 січня на замерзлій водоймі відбувся цікавий захід, який організували вперше,
–  змаганням з бігу на ковзанах. Ідею реалізували спільно з Шацькою селищною
радою та Молодіжною радою Шацька. І хоч часу в організаторів було обмаль, до
участі в змаганнях зареєструвалися 60 учасників віком від 9 до 20 років. Всім
безплатно видали ковзани та засоби захисту. На початку запросили до слова
рятувальників, які нагадали учасникам правила поведінки на кризі, а впродовж
заходу стежили за тим, щоб катання було безпечним. Крім того, на березі озера
був присутній медик – на випадок травм.

\ii{13_01_2022.stz.news.ua.volyn.1.ozero_kovzany.pic.3}

«Менш ніж за добу наші активісти Володимир Кубай і Сергій Грицюк змогли
організувати захід, запросивши на нього людей і створивши оптимальні умови для
його проведення. Такі змагання у нас відбуваються вперше, але я впевнений, що
не востаннє, і кожні наступні будуть досконаліші і яскравіші. Всім учасникам
зичу гарного настрою і задоволення від катання!» – такими словами привітав
ковзанярів Богдан Тимошук, секретар Шацької селищної ради.

\begin{zznagolos}
До участі в змаганнях зареєструвалися 60 учасників віком від 9 до 20 років.
Всім безплатно видали ковзани та засоби захисту.	
\end{zznagolos}

Змагання відбувалися в таких номінаціях: індивідуальні (хлопці і дівчата віком
до 16 і від 16 років) та командні забіги. Хлопці бігли на ковзанах дистанцію
650 м, дівчатка – наполовину меншу. В результаті грамоти і призи отримали ті,
які в своїй номінації показали найкращу швидкість, а саме: наймолодша учасниця
Оксана Степанюк, володарі перших місць: Світлана Турич, Володимир Семерей,
Богдан Натальчук та команда у складі Івана Кропивника і Володимира Семерея
(хлопцям дістався ще й головний приз – кубок). Нагороди і подарунки вручили
також тим, хто посів друге і третє місця.

І ковзанярів, і вболівальників організатори впродовж двох годин зігрівали
гарячим трав’яним чаєм та пропонували їм солодощі.

Захід подарував позитивні емоції і дітям, і дорослим. А також додав натхнення
організаторам, аби й надалі реалізовували свої креативні ідеї, розвиваючи в
Шацькому краї зимовий туризм.

\ii{13_01_2022.stz.news.ua.volyn.1.ozero_kovzany.pic.4}
\ii{13_01_2022.stz.news.ua.volyn.1.ozero_kovzany.pic.5}
