% vim: keymap=russian-jcukenwin
%%beginhead 
 
%%file 25_08_2021.fb.nikonov_sergej.1.strana_proshloje_nepolnocennost
%%parent 25_08_2021
 
%%url https://www.facebook.com/alexelsevier/posts/1566823740329602
 
%%author Никонов, Сергей
%%author_id nikonov_sergej
%%author_url 
 
%%tags antirossia,istoria,nepolnocennost,nezalezhnist,pamjat,strana,ukraina
%%title СТРАНА, НЕ ЗНАЮЩАЯ ИСТИННОГО ПРОШЛОГО, НЕПОЛНОЦЕННА
 
%%endhead 
 
\subsection{СТРАНА, НЕ ЗНАЮЩАЯ ИСТИННОГО ПРОШЛОГО, НЕПОЛНОЦЕННА}
\label{sec:25_08_2021.fb.nikonov_sergej.1.strana_proshloje_nepolnocennost}
 
\Purl{https://www.facebook.com/alexelsevier/posts/1566823740329602}
\ifcmt
 author_begin
   author_id nikonov_sergej
 author_end
\fi

Зеленский хочет считать точкой отсчета украинской государственности год
основания Киева, – столицы Киевской Руси-Украины, "которую не зря считают
местом, где все начинается", сказал Владимир Зеленский. 

СКАЖУ ТУТ ВАЖНУЮ ВЕЩЬ. СТРАНА, НЕ ЗНАЮЩАЯ ИСТИННОГО ПРОШЛОГО, НЕПОЛНОЦЕННА.
ДУМАЮ, ЧТО ОФИЦИАЛЬНАЯ ИДЕОЛОГИЯ СЕГОДНЯ ТАКОВА, ЧТО В ЭТОМ ПЛАНЕ КУДА ХУЖЕ
СОВЕТСКОЙ В ПЛАНЕ СОЧЕТАНИЯ ПРАВДЫ, ЗАМАЛЧИВАНИЯ, ЛОЖНЫХ ТРАКТОВОК И ЛЖИ. 

Во-первых, слово Украина появилось в летописях 12 века. Или похожее на него.
Так что считайте точкой отсчета например Запорожскую сечь. Нет, с
патриотической точки зрения мне самому хочется чтобы мы были древнее Египта. Но
главное качество жизни и истина.

Во-вторых, это не попытка построить будущее в том числе путем объективного
изучения истории. ЭТО ПОЛИТИКА АНТИРОССИИ. Мол, не было общего славянского
корня трех народов, а был вот один главный  украинский европейский народ, от
которого "ответвления" пошли с какими-то примесями. Или примеси под
Украину-Русь замаскировались.  А раз они - не главные, значит и общего у нас
нет. Это политика Антироссии, а не Украины. 

В-третьих, хорошо, хотим глубоких корней. Так берите из этих корней лучшее и не
трогайте каноническую церковь, идущую от того времени. Берите христианское
мировоззрение, а не "Велике крадивництво" и братоубийственную  войну, как
следствие "феодальной разбленности", то есть господарство олигархов в отдельных
местностях. А пока взяли только закрепощение крестьян и деградацию структуры
экономики в пользу феодальной. Берите структуру ВВП или ВНП для анализа по
видам экономической деятельности. 

В-четвертых, для евроинтеграции мало быть европейцами в каком-то далеком
прошлом. С институциональной точки зрения нужно зрелое государство с успешной
постмодерной экономикой  и всеми конкурентоспособными сферами жизни. А что у
нас? Деградация. Из-за государственной политики. Например, недофинансирование
науки и культуры  и примат псевдопатриотизма. Кроме того, должно быть
европейское мировоззрение. Берем черты, которые считаются лучшими. Где
законопослушание и демократия? А ТАК ЛИ ХОРОШО ТО РЫНОЧНОЕ МИРОВОЗЗРЕНИЕ, ЧТО
ПРЕДЛАГАЮТ НАМ?

В-пятых, увы, утопичная мечта. Предоставьте слово реальным честным
профессионалам, а не Зеленскому, Порошенко, Притуле, разным Вятровичям и
Дрогобычам и т.п.. Но вместо этого гонения на Толочко  и на Евгению Бильченко,
Дудкина. ПО АРГУМЕНТАЦИИ НИЧЕГО СКАЗАТЬ НЕ МОГЛИ. 

Предпочли травить "патриотами".  Очень показателен список запрещенной
российской литературы. Там не только радикальная реально антиукраинская
литература вроде трудов соратников Гиркина-Стрелкова или Жириновского, там все,
что не нравится нашей пропаганде. Да, там есть книга Толочко. 

ЧТО ДЕЛАТЬ? БОРОТЬСЯ ЗА УКРАИНУ С ЧЕСТНОЙ ПОЛИТИКОЙ И ОБЩЕСТВЕННЫМИ НАУКАМИ. НА
ПРАКТИКЕ ЗНАКОМИТЬСЯ С РЕАЛЬНОЙ ИСТОРИЕЙ И ПО ВОЗМОЖНОСТИ ЗНАКОМИТЬ ДЕТЕЙ И
МОЛОДЕЖЬ.
