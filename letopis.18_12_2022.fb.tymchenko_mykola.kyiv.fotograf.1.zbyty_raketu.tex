% vim: keymap=russian-jcukenwin
%%beginhead 
 
%%file 18_12_2022.fb.tymchenko_mykola.kyiv.fotograf.1.zbyty_raketu
%%parent 18_12_2022
 
%%url https://www.facebook.com/nick.tymchenko1/posts/pfbid02H72sSoxJKkpt2z7fNbqj5KWeeunzDnYvxpwZxZ6UsaMYvH1nTZtgVRfzj5tkWV9ql
 
%%author_id tymchenko_mykola.kyiv.fotograf
%%date 
 
%%tags 
%%title Шансів збити російську ракету під Києвом, з трофейного кулемету, один на мільярд
 
%%endhead 
 
\subsection{Шансів збити російську ракету під Києвом, з трофейного кулемету, один на мільярд}
\label{sec:18_12_2022.fb.tymchenko_mykola.kyiv.fotograf.1.zbyty_raketu}
 
\Purl{https://www.facebook.com/nick.tymchenko1/posts/pfbid02H72sSoxJKkpt2z7fNbqj5KWeeunzDnYvxpwZxZ6UsaMYvH1nTZtgVRfzj5tkWV9ql}
\ifcmt
 author_begin
   author_id tymchenko_mykola.kyiv.fotograf
 author_end
\fi

Шансів виграти чемпіонат світу у Мессі або Мбаппе сьогодні були 50 на 50.

Шансів збити російську ракету під Києвом, з трофейного кулемету, один на
мільярд.

Але 16 грудня, під час чергового масованого ракетного обстрілу, Олег Василишин,
боєць 112 Бригада територіальної оборони міста Києва , знаходячись на позиції
на відстані близько кілометра від важливого енергетичного об’єкта у місті
Києві, збив з трофейного кулемета, який його побратими захопили у боях на сході
України, одну з ракет, яка мала вцілити в будівлю.

Те, що так швидко є тепло в наших оселях, маємо дякувати Олегу.

Олег − киянин, військового досвіду не мав, до початку російського наступу
працював пакувальником у \enquote{Книгарні Є}. За дві години після початку
повномасштабного вторгнення вже був в черзі до військкомату.

Олег розповів у коментарі для \enquote{Рубрики}, як влучив в ракету − читайте за лінком
в першому коментарі
