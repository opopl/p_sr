% vim: keymap=russian-jcukenwin
%%beginhead 
 
%%file 08_03_2023.fb.kipcharskij_viktor.mariupol.1.r_k_tomu_bulo_take__.cmt
%%parent 08_03_2023.fb.kipcharskij_viktor.mariupol.1.r_k_tomu_bulo_take__
 
%%url 
 
%%author_id 
%%date 
 
%%tags 
%%title 
 
%%endhead 

\qqSecCmt

\iusr{Марина Солошенко}

Може на щастя для маріупольців, випав сніг, дав хоч трохи води. Вчора, вперше з
начала війни, в Харкові включили вуличне освітлення, ми виходимо з темряви!!!

\iusr{Елена Девина}

А ми в цей день готували їжу на вогнищах біля дому. Раптом під'їхала якась
приватна машина, з неї вийшла незнайома жіночка. Відкрила багажник й почала
виносити з нього дуже красиві рози. Потім підходила до кожної жінки біля вогнища
й дарувала ці квіти. При цьому казала: \enquote{Берегла саме для цього дня. Зі
святом, жіночки!} Ми плакали від несподіваних подарунків! Я поставила ці мабуть
півметрові рози у найкрасивішу вазу! А увечері я давала \enquote{концерт}, граючи на
ф-но(саме для нього не була потрібна електрика). На вулиці мороз, у кімнаті
холодно (бо вікна вже вибиті), навкруги бахкає... А музика лунає й жінки співають
у різних квартирах (бо грала мелодії з кінофільмів та знайомі пісні). Потім грала
вже кожен вечір (до від'їзду)...
