% vim: keymap=russian-jcukenwin
%%beginhead 
 
%%file 18_04_2023.stz.news.ua.donbas24.1.den_pamjatok_istorii_ta_kultury
%%parent 18_04_2023
 
%%url https://donbas24.news/news/den-pamyatok-istoriyi-i-kulturi-tradiciyi-svyata
 
%%author_id demidko_olga.mariupol,news.ua.donbas24
%%date 
 
%%tags 
%%title День пам'яток історії та культури — традиції свята
 
%%endhead 
 
\subsection{День пам'яток історії та культури — традиції свята}
\label{sec:18_04_2023.stz.news.ua.donbas24.1.den_pamjatok_istorii_ta_kultury}
 
\Purl{https://donbas24.news/news/den-pamyatok-istoriyi-i-kulturi-tradiciyi-svyata}
\ifcmt
 author_begin
   author_id demidko_olga.mariupol,news.ua.donbas24
 author_end
\fi

\vspace{1cm}
\ii{18_04_2023.stz.news.ua.donbas24.1.den_pamjatok_istorii_ta_kultury.pic.front}
\vspace{0.5cm}
\begin{center}
  \em\color{blue}\bfseries\Large
Щороку 18 квітня відзначається Міжнародний день пам'яток і визначних місць  
\end{center}

Світова спільнота 18 квітня відзначає День пам'яток та визначних місць. На
підтримку ініціативи вчених, архітекторів, реставраторів, працівників державних
органів охорони \href{https://archive.org/details/11_08_2022.olga_demidko.donbas24.zlochyny_rosian_proty_kulturn_spadchyny_donbasu}{\emph{пам'ятників історії та культури}}%
\footnote{Злочини росіян проти культурної спадщини Донбасу, Ольга Демідко, donbas24.news, 11.08.2022, \par%
\url{https://donbas24.news/news/zlocini-rosiyan-proti-kulturnoyi-spadshhini-donbasu}, \par%
Internet Archive: \url{https://archive.org/details/11_08_2022.olga_demidko.donbas24.zlochyny_rosian_proty_kulturn_spadchyny_donbasu}%
} України було ухвалено встановити в країні відповідне свято, відзначати яке слід також 18 квітня. В
Україні цей день \href{https://mkip.gov.ua/news/7175.html}{отримав} назву — День пам'яток історії та культури.

\textbf{Читайте також:} \href{https://donbas24.news/news/fotografi-pokazali-yak-vizivaje-maize-povnistyu-zruinovana-avdiyivka-foto}{\emph{Фотографи показали, як виживає майже повністю зруйнована Авдіївка}}%
\footnote{Фотографи показали, як виживає майже повністю зруйнована Авдіївка, Тетяна Веремєєва, donbas24.news, 17.04.2023, \par%
\url{https://donbas24.news/news/fotografi-pokazali-yak-vizivaje-maize-povnistyu-zruinovana-avdiyivka-foto}%
}

\subsubsection{Історія дати}

У 1983 році, за пропозицією Асамблеї Міжнародної ради з питань охорони пам'яток
та визначних місць (ICOMOS), створеній при ЮНЕСКО, Центр культурної спадщини
ЮНЕСКО оголосив 18 квітня Міжнародним днем пам'яток та визначних місць
(Inter\hyp{}national Day of Monuments and Sites). Основне завдання цього свята — ще
раз нагадати світовому суспільству про різноманіття культурної спадщини,
створеної зусиллями багатьох народів за всю історію людства, її вразливість та
необхідність збереження для майбутніх поколінь. Девізом Міжнародного дня
пам'яток та історичних місць стали слова: \emph{\enquote{Збережемо нашу історичну
батьківщину}.}

Відповідно до Указу Президента України від 23 серпня 1999 року № 1062/99 в
Україні цей день отримав назву — День пам'яток історії та культури. Український
комітет Міжнародної ради з питань пам'яток і визначних місць (ICOMOS) є
дорадчим органом ЮНЕСКО, та разом із Міністерством культури України та
Національною комісією ЮНЕСКО вирішує, зокрема, які об'єкти номінувати до Списку
Всесвітньої Спадщини.

\textbf{Читайте також:} \href{https://donbas24.news/news/stalevar-veza-ta-more-u-lvovi-prezentuyut-pisanki-z-mariupolskimi-motivami-foto}{\emph{Сталевар, Вежа та море: у Львові презентують писанки з маріупольськими мотивами}}%
\footnote{Сталевар, Вежа та море: у Львові презентують писанки з маріупольськими мотивами, Наталія Сорокіна, donbas24.news, 17.04.2023, \par%
\url{https://donbas24.news/news/stalevar-veza-ta-more-u-lvovi-prezentuyut-pisanki-z-mariupolskimi-motivami-foto}%
}

\ii{18_04_2023.stz.news.ua.donbas24.1.den_pamjatok_istorii_ta_kultury.pic.1}

\subsubsection{Як відзначатиметься в Україні?}

Згідно із Законом об'єкти культурної спадщини, що перебувають на території
України, охороняються державою. З початку повномасштабного вторгнення Росії в
Україну зафіксовано понад 380 випадків пошкоджень та руйнувань російськими
військовими українських пам'яток архітектури, археології, релігії й історії.
Зокрема, понад 100 закладів та пам'яток культури знищені у Луганській області.
На Донеччині зруйновано 89 пам'яток культури та архітектури, з них 53 в
Маріуполі. Архітектурний та культурний фонди Маріуполя майже повністю знищені.
Загалом у місті пошкоджено 90\% будівель. Більша частина пам'яток архітектури не
підлягають реконструкції. Всі воєнні злочини країни-агресорки, спрямовані проти
культурної спадщини України, мають бути зафіксовані і ретельно досліджені.
Залучення міжнародної фінансової підтримки для збереження, оцифрування та
відновлення нашої культурної спадщини через програму Велика Реставрація —
стратегічна задача Міністерства культури та інформаційної політики України.

Цьогоріч 18 квітня в містах України пройдуть акції та виставки на підтримку
пам'яток історії та культури країни. Так, о 14:00 в Києві у приміщенні
Національного музею історії України відбудеться відкриття виставки \emph{\enquote{Вписані в
історію. Зруйновані війною}.}

\ii{18_04_2023.stz.news.ua.donbas24.1.den_pamjatok_istorii_ta_kultury.pic.2}

\begin{leftbar}
\emph{\enquote{Наслідки повномасштабного вторгнення рф призвели до пошкодження та знищення
значної кількості історичних об'єктів, включаючи пам'ятники архітектури та
монументального мистецтва в Центральному районі Маріуполя. Повний масштаб втрат
буде визначений після деокупації міста}}, — наголосили організатори виставки.
\end{leftbar}

\textbf{Читайте також:} \href{https://donbas24.news/news/mariupol-do-ta-pislya-ruinuvan-yak-zaraz-viglyadayut-istoricni-budivli-mista-mariyi-foto}{\emph{Маріуполь до та після руйнувань: як зараз виглядають історичні будівлі міста Марії}}%
\footnote{Маріуполь до та після руйнувань: як зараз виглядають історичні будівлі міста Марії, Алевтина Швецова, donbas24.news, 09.09.2022, \par%
\url{https://donbas24.news/news/mariupol-do-ta-pislya-ruinuvan-yak-zaraz-viglyadayut-istoricni-budivli-mista-mariyi-foto}%
}

На виставці будуть представлені світлини, що демонструють руйнування, взяті з
відкритих джерел та з архівів Маріупольського краєзнавчого музею. Темою
виставки є збереження та відродження історичної спадщини міста.

Раніше Донбас24 писав, що \href{https://donbas24.news/news/fotografi-pokazali-yak-mediki-rizikuyut-zittyam-zaradi-poryatunku-viiskovix-na-peredovii-foto}{\emph{фотографи показали}},%
\footnote{Фотографи показали, як медики ризикують життям заради порятунку військових на передовій, Тетяна Веремєєва, donbas24.news, 17.04.2023, \par\url{https://donbas24.news/news/fotografi-pokazali-yak-mediki-rizikuyut-zittyam-zaradi-poryatunku-viiskovix-na-peredovii-foto}}
як медики ризикують життям заради порятунку військових на передовій.

Найсвіжіші новини та найактуальнішу інформацію про Донецьку й Луганську області
також читайте в нашому телеграм-каналі Донбас24.

ФОТО: з відкритих джерел

\ii{insert.author.demidko_olga}
%\ii{18_04_2023.stz.news.ua.donbas24.1.den_pamjatok_istorii_ta_kultury.txt}
