% vim: keymap=russian-jcukenwin
%%beginhead 
 
%%file 23_06_2021.fb.buzhanskii_max.1.operacia_bagration
%%parent 23_06_2021
 
%%url https://www.facebook.com/permalink.php?story_fbid=1974826492681820&id=100004634650264
 
%%author 
%%author_id buzhanskii_max
%%author_url 
 
%%tags 1944,23_jun,germania,istoria,mirvojna2,operacia.vov.bagration,sssr,vojna,vov
%%title 23 июня 1944 года, началась операция "Багратион"
 
%%endhead 
 
\subsection{23 июня 1944 года, началась операция \enquote{Багратион}}
\label{sec:23_06_2021.fb.buzhanskii_max.1.operacia_bagration}
\Purl{https://www.facebook.com/permalink.php?story_fbid=1974826492681820&id=100004634650264}
\ifcmt
 author_begin
   author_id buzhanskii_max
 author_end
\fi

Я этот текст написал давно, три года назад, погиб вместе с аккаунтом.
Но, не вижу повода не повторить.
23 июня 1944 года, началась операция "Багратион".
41й год, только уже наоборот, совсем наоборот.
Нет, это не вернуло павших, мёртвые не встали из могил, но то, что они хотели сделать, старались изо всех сил, сделали живые.
И да, тупые не бывают храбрыми, никогда не бывают, они просто не в состоянии этого чувства понять.
---------------------------
Операция "Багратион".
Клац, клац...
Стрелка часов щёлкает монотонно и мерно, нагло врываясь в вязкую тишину, повисшую в кабинете.
Вызывающе щелкает, будто хочет показать, что собравшиеся могут быть властны над чем угодно, кроме одного.
Времени.

\ifcmt
  pic https://scontent-lga3-2.xx.fbcdn.net/v/t1.6435-0/p526x296/204222994_1974826462681823_4825385661772135879_n.jpg?_nc_cat=110&ccb=1-3&_nc_sid=8bfeb9&_nc_ohc=ra0dxH-AUMgAX-16VAJ&_nc_ht=scontent-lga3-2.xx&tp=6&oh=ee1abf8c76ce04d0af7e60fb96febedb&oe=60D92814
\fi

Лампы небрежно бросают свет, рассыпают по золоту погон, окутывают им комнату, будто пытаясь загнать обратно крадущиеся отовсюду тени.
Зеленое сукно стола, карта, поверх неё еще одна, брошена поперек, накрыла собой пару забытых карандашей, свисает со стола краем.
Василевский не видит ничего вокруг, никого не видит, тянется куда то циркулем, что то бормочет под нос.
Бронзовым истуканом застыл Жуков, кулак тяжело впечатался в столешницу, мундир трещит на плечах, мощь ощущается, даже когда он неподвижен.
Штеменко украдкой смотрит на остальных, бросает взгляд на карту, ловит движения циркуля Василевского, и снова на лица.
Мягкие шаги за спиной.
Сталин молчит, молчит, шагает вдоль стола за спинами генералов, трубка мерно постукивает о большой палец руки.
Ворс ковра глушит шаги, но все чувствуют их, не оборачиваются, но чувствуют.
Жуков разворачивается, выпрямляется, словно возможно держать спину ещё прямее.
Бесстрастное лицо, подбородок высоко поднят.
Предлагаю наносить удар здесь!!!
Сказал, как впечатал.
Он всегда такой, да и Верховный не любит неопределенности, это знают все.
Василевский оторвался от карты, крутит в руках циркуль, ловит взгляд Сталина, согласно кивает.
Все кивают, синхронно, как строевым шагом на параде, все согласны.
Два удара!
Здесь, и вот здесь!
Голос Рокоссовского звучит сдержанно, почти равнодушно, но будто кто то с грохотом и лязгом уронил гирю на пол.
Жуков разворачивается, трещит мундир, трещит стол, под напором кулака.
Глаза мечут молнии.
Равнодушно смотрит вперед Рокоссовский, затаил дыхание Штеменко.
Сталин останавливается.
Тишина, замерли на миг, будто споткнулись, стрелки часов.
Клац, клац, звучит как то жалобно, робко.
Два удара!- Рокоссовский повторяет ещё раз, смотрит в глаза Сталину.
Тишина.
Минута, другая...
Подумайте еще раз, товарищ Рокоссовский...- голос Верховного так же спокоен и равнодушен.
Тишина.
Минута, другая...
Я настаиваю на своём предложении, товарищ Верховный Главнокомандующий.
Кулак Жукова мелко дрожит на столе, кажется, будто гнев рванет сейчас, разорвет Маршала на части!
Мы предлагаем- мягкие шаги за спиной, Рокоссовский смотрит перед собой, не оборачивается, -Мы предлагаем вам выйти и ещё раз всё обдумать...
Скрип каблуков, Рокоссовский разворачивается и выходит.
Тишина.
Минута, другая, третья...
Бесшумно открывается дверь.
Я настаиваю на своём предложении!
Без вызова, спокойно.
Василевский раздраженно бросает циркуль на стол, звякает ложка в стакане с недопитым чаем.
В кабинете много людей, но Рокоссовский один.
Пустота сомкнулась вокруг него, плещет в глазах пыточным подвалом, сломанными ребрами и пальцами.
Все смотрят на карту, только Голованов смотрит в глаза, с интересом смотрит, возвышаясь над остальными.
Мягкие шаги.
Мы предлагаем вам ещё раз выйти и подумать, товарищ Рокоссовский.
В голосе уже нет троеточия, Сталин сух и сдержан.
Скрип каблуков, разворот через плечо.
Тишина.
Минута, другая, третья...
Дверь открывается, свет льется за спиной Рокоссовского, словно провожая его.
Три шага вперёд, к столу.
Я настаиваю на своём предложении, товарищ Сталин.
Мундир Жукова трещит.
Жуков в ярости.
От Жукова разит кабаном, зверем разит, лицо багровое, капли пота на лбу.
Подался вперед, всей массой, навис над Рокоссовским, будто скала, кулаки сжаты, вены на шее вздулись буграми.
Рокоссовский равнодушно смотрит перед собой, прячет усмешку в уголках рта Голованов.
Клац, клац...
Стрелке страшно.
Времени страшно, оно боится этой жуткой тишины, оно боится этих людей, пусть взорвутся, пусть бросятся друг на друга, что угодно, только не молчат!!!
Мягкие шаги за спиной.
Минута, другая...
Клац, клац, всхлипывает стрелка...
Сталин подходит к столу.
Присутствующие разворачиваются к нему, выпрямляются.
Прячет за спину руку с циркулем Василевский.
Жуков дышит, как загнанная лошадь, бешенство душит его, схватило за горло и сжимает пальцы.
Сталин молчит, поднимает голову, смотрит почему то на Голованова.
-Раз товарищ Рокоссовский так уверен в своей правоте...
Мягкие шаги, трубка стучит по пальцу...
-Раз товарищ Рокоссовский так настаивает,- слово -Так,брошено в лицо присутствующим, как перчатка, дергается, как от удара, Жуков- Мы примем его предложение...
Клац, клац, клац...
Стрелка часов щелкает, обессилевшая, с облегчением.
Тишина лопнула.
———————————————————-
22 июня 1944 года, началась операция Багратион.
Операция, после которой Сталин обращался к Рокоссовскому по имени- отчеству.
Знаете, на Нюрнбергском процессе, умные, холеные немецкие генералы, как зомби твердили.
Это не мы.
Это Гитлер, а мы что, мы солдаты, приказов не обсуждаем:
В своих мемуарах, умные немецкие генералы, в один голос твердили.
Мы умные, мы всё знали, всё понимали,Гитлер дурак, но мы боялись возразить.
Маршал Советского Союза, Маршал Польши Константин Константинович Рокоссовский, был репрессирован перед войной.
Маршал Советского Союза Константин Константинович Рокоссовский, не побоялся возразить Сталину.
Позже, когда Маршал Советского Союза Георгий Константинович Жуков был снят со всех постов и попал в опалу, пустота образовалась вокруг него.
К нему не пришел никто.
Кроме Главного Маршала Авиации Александра Евгеньевича Голованова.
Голованов не любил Жукова.
Жуков не любил Голованова.
Но Голованов не боялся, не побоялся прийти к Жукову.
В этом вся фишка, кто то побоялся, а кто то нет.
Полная внезапность.
Когда говорят, как можно было не заметить, пропустить удар трех миллионов немцев 23 июня 1941 года, как можно было дать себя обмануть, не слушайте.
Уже несколько лет шла война, и немцы дали себя обмануть, не поняли, что полтора миллиона человек, войска нескольких Фронтов, будут наступать.
И этот день настал.
Где то там, в этой огромной массе людей, идущих вперед, командовал танком в отдельном гвардейском тяжелом танковом полку прорыва мой дед.
Где то там, в огромной массе людей, идущих вперед, был второй мой дед, отдельный , Горьковско-Варшавский дивизион бронепоездов.
И ваши были.
Разведчики, саперы, летчики, артиллеристы, связисты.
Все были.
Вы, когда слушаете байки про тупых генералов, вы не забывайте эту историю.
Запомните, тупые, никогда не бывают храбрыми.
Тупые никогда не берут на себя ответственность.
Тупые никогда не стоят на своём тогда, когда над ними нависает туша Жукова и глушатся ковром шаги Сталина за спиной.
22 июня 1944 года, советские войска перешли в наступление, закончившееся освобождением Белоруссии, части Прибалтики и Польши.
Тут дело не в количестве разгромленных дивизий, я знаю, что все вокруг стратеги, кроме меня.
Дело в сути.
Ничто не далось и не дается даром, ничто не дается случайно, ничто не дается тем, кто труслив и глуп.
И самая большая глупость и трусость, назвать глупцами и трусами тех, кто ими никогда не был.
И да, пленных немцев, 57 тысяч, тех, кто мог идти, потом провели по Москве.
А после того, как прошли, помыли улицы.

\url{https://t.me/MaxBuzhanskiy}
