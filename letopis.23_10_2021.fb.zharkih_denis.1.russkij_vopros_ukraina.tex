% vim: keymap=russian-jcukenwin
%%beginhead 
 
%%file 23_10_2021.fb.zharkih_denis.1.russkij_vopros_ukraina
%%parent 23_10_2021
 
%%url https://www.facebook.com/permalink.php?story_fbid=3105601499653246&id=100006102787780
 
%%author_id zharkih_denis
%%date 
 
%%tags obschestvo,rossia,rusmir,strana,ukraina
%%title Русский вопрос в Украине
 
%%endhead 
 
\subsection{Русский вопрос в Украине}
\label{sec:23_10_2021.fb.zharkih_denis.1.russkij_vopros_ukraina}
 
\Purl{https://www.facebook.com/permalink.php?story_fbid=3105601499653246&id=100006102787780}
\ifcmt
 author_begin
   author_id zharkih_denis
 author_end
\fi

Русский вопрос в Украине.

Государству легко завоевать территории, если на них живет население сходное по
языку и обычаям к коренному населению государства. Немцы легко брали Австрию,
Судеты (а заодно и Чехословакию), Данию. Уже с Голландией, Польшей и Францией
пришлось повозиться (ну, хоть постреляли). Французы быстро возвращали Эльзас.
Россия легко брала не только Украину, но даже Прибалтику и Польшу. С Кавказом
возились значительно больше. 

Получается, что русское население для Украины -  удобная среда для экспансии
РФ. Об этом и говорит идеология Русского мира. И тут становится русский вопрос.
Что с ними делать? Депортировать? Украинизировать? Уничтожить?

На самом деле не все так просто. Есть и другие примеры, просто украинские
политики не читают историю, они ее пишут. Например, обожаемые ими США.
Англоязычные колонисты выписали Империи люлей, и у Империи никогда не возникает
желания сунуться туда еще разик. И никто не собирался заниматься изучением
языка индейцев или, например, ирландского. 

Все проще, американцы сделали другую идеологию, другую ментальность, другую
экономику. И эта экономика была лучше, чем у Империи. Собственно, все. Да, и
они с Британией потом дружили, и дружат до сегодняшнего дня. Также поступили
канадцы, австралийцы, ирландцы и новозеландцы.Они дружат с бриттами и не
называют их оккупантами. 

Перенять эту стратегию Украине никто не мешал. Тем более, что, например,
Белоруссия и Казахстан так и поступили. И все у них там относительно хорошо. Но
Украина сразу подняла национальный, а не экономический вопрос, в результате
потеряла экономику, но зато получила национальную проблему. А решать
национальную проблему без экономики невозможно. Тогда Украина становится легкой
добычей не только РФ, но и Польши, Венгрии и Румынии. Сталин легко депортировал
народы, поскольку у него на это были силы. У украинцев нет сил депортировать
русских. И нет ума построить экономику лучше, чем в РФ. И нет дальновидности,
чтобы наладить отношения с сильным соседом. А как ни ума, ни фантазии, ни
порядочности (кто-то считает украинских политиков порядочными?), то Украина
сегодня сырье для других государств. Если у элиты нет воли, нет ума и нет
чести, то страна потихоньку прекращает свое существование. Для спасения
Украины, как проекта, нужно не только элиту поменять, но и правила ее
поведения. Слишком мало времени для этого, и слишком много спесивых идиотов в
этой самой элите.

2016

\ii{23_10_2021.fb.zharkih_denis.1.russkij_vopros_ukraina.cmt}
