% vim: keymap=russian-jcukenwin
%%beginhead 
 
%%file 21_04_2022.fb.solovjov_mikita.harkov.demsokyra.1.hronika
%%parent 21_04_2022
 
%%url https://www.facebook.com/Mikita.Solovyov/posts/7425327227537626
 
%%author_id solovjov_mikita.harkov.demsokyra
%%date 
 
%%tags 
%%title Хроника Харькова, 21-е апреля
 
%%endhead 
 
\subsection{Хроника Харькова, 21-е апреля}
\label{sec:21_04_2022.fb.solovjov_mikita.harkov.demsokyra.1.hronika}
 
\Purl{https://www.facebook.com/Mikita.Solovyov/posts/7425327227537626}
\ifcmt
 author_begin
   author_id solovjov_mikita.harkov.demsokyra
 author_end
\fi

Хроника Харькова, 21-е апреля. 

Обстрелы сегодня были примерно на вчерашнем уровне, то есть заметно меньше, чем
в начале недели. Вчера вечером после обстрела загорелись две высотки на
Салтовке. Насколько я знаю, жертв нет. Единственные мне известные жертвы
сегодня, это трагическая случайность. Чуть ли не единственный упавший на улице
снаряд (или что там было) попал прямо в едущий автомобиль, причем в двигатель.
Самые обстреливаемые районы сегодня это Салтовка, Пятихатки, в меньшей степени
Жуковского. Судя по всему, обстрелы ведутся из районов Циркуны-Тишки и Русской
Лозовой. 

\ii{21_04_2022.fb.solovjov_mikita.harkov.demsokyra.1.hronika.pic.1}

По центральной и северной части области и меня сегодня данных вообще
практически нет. Мне известно только о нескольких эпизодических обстрелах, но
по большей части я просто не имею информации. Основные обстрелы и боевые
действия ведутся на изюмском направлении. Там сейчас ситуация сложная. Основные
действия ведутся к югу от Изюма, но и с других сторон тяжелые бои.
Анализировать результаты боев я не могу. Причем не только за отсутствием
информации, но и по причине собственной полной профнепригодности в этой сфере.
Известно только, что в Харьковской области тяжелые бои, никаких заметных
успехов у русских в наступлении нет, но они ломятся со страшной силой. А, и
известно о нескольких сбитых за день русских самолетах и вертолетах. 

В Харькове обстановка достаточно спокойная. Бои идут не в непосредственной
близости от города, и народ немного переключается. Опять же не берусь оценивать
плохо это или хорошо. Но кажется, подготовка к Пасхе заботит харьковчан намного
больше, чем ход боевых действий на Донбассе. С одной стороны, меня это немного
резануло когда сегодня ходил городом и слушал разговоры. А с другой, если люди
не могут прямо сейчас как-то активно повлиять на ситуацию, то чем поможет их
дополнительная тревожность? 

По моим ощущениям, людей на улицах немного меньше, чем было неделю назад.
Погода повлияла, плотные обстрелы в начале недели или что-то другое, судить не
берусь. Большинство разговоров крутятся вокруг Пасхи и всех связанных с ней
ритуалах. Что радует, большинство уже смирилось с тем, что Всенощные служиться
будут в церквях без прихожан. Что не радует, что большинству не особо интересно
какого патриархата церковь, в которую они привыкли ходить. Но это не
воцерковленные, а просто ходящие 2-3 раза в год в церковь за ритуалом. 

Я опять напоминаю, и буду это делать до конца недели. Пожалуйста, постарайтесь
воздержаться от посещения церквей в эти выходные. Ну или хотя бы не подходите к
ним если уже видите там скопление людей. Очень высока вероятность "праздничных"
обстрелов. Мало того, что это полностью в духе русского мира. Так еще и их чуть
ли не прямое предупреждение, что они готовятся обстреливать церкви на Пасху. Ну
а что УПЦ МП это филиал ФСБ, моим читателям напоминать не нужно, я надеюсь. 

Харьков стоит!

Слава Украине!

Низкий поклон нашим защитникам!

\#ХроникаХарькова

\ii{21_04_2022.fb.solovjov_mikita.harkov.demsokyra.1.hronika.cmt}
