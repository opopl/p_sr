% vim: keymap=russian-jcukenwin
%%beginhead 
 
%%file 03_10_2021.fb.zaharova_maria.mid_rf.1.zelenskii_politika_naglost.cmt.3.alunkachev_tema_ukraina
%%parent 03_10_2021.fb.zaharova_maria.mid_rf.1.zelenskii_politika_naglost.cmt
 
%%url 
 
%%author_id 
%%date 
 
%%tags 
%%title 
 
%%endhead 
\paragraph{Абдурахман Алункачев - Любимая тема Украина ваша}

\begin{itemize} % {
\iusr{Абдурахман Алункачев}
Любимая тема Украина ваша внутри страны нечего хорошего нету независимые Сми не осталось

\begin{itemize} % {
\iusr{Александр Софинский}
\textbf{Абдурахман Алункачев} , а где три телеканала закрыли? Неужто в России? @igg{fbicon.wink}  @igg{fbicon.face.smiling.sunglasses} 
\iusr{Василий Шабунин}
\textbf{Абдурахман Алункачев} Захарова пресс секретарь МИДа, занимается своей работой, причём здесь внутренние проблемы страны?

\iusr{Iakov Le Ffoleg}
\textbf{Абдурахман Алункачев} ЗАТКНИСЬ!

\iusr{Лариса Михайлова}
\textbf{Абдурахман Алункачев} вы по стране часто путешествуете? Не надо таки за всю страну вешать из своего региона)))
\end{itemize} % }

\iusr{Jurij Olechnovic}
Это как с 277 миллиардами запрошеными Зеленским на трансформацию Украины.

- На что будут потрачены эти деньги?

- Мы пока не знаем. Пока мы просто цифру придумали. Специально не круглую,
чтобы сильно в глаза не кидалось то, что она взята просто "с потолка".   @igg{fbicon.face.smiling.eyes.smiling} 

\iusr{Алесь Карніенка}
ваша кремлёвская - сама взвешенность и миролюбие!

\begin{itemize} % {
\iusr{Iakov Le Ffoleg}
\textbf{Алесь Карніенка} ВОТ!

\ifcmt
  ig https://scontent-mia3-2.xx.fbcdn.net/v/t1.6435-9/243952098_255831713052434_7172874655766684472_n.jpg?_nc_cat=110&ccb=1-5&_nc_sid=dbeb18&_nc_ohc=ZEfEGsgE59UAX_CSmZz&_nc_ht=scontent-mia3-2.xx&oh=159c0c24a056ab1c743888714261b247&oe=6186DBBE
  @width 0.4
\fi

\end{itemize} % }

\iusr{Mikael Multipolar}
Не могу дождаться этого дня

\ifcmt
  ig https://scontent-mia3-1.xx.fbcdn.net/v/t1.6435-9/244317036_167324928891795_7948194380767472576_n.jpg?_nc_cat=111&ccb=1-5&_nc_sid=dbeb18&_nc_ohc=DDwrmdHiXfwAX9U3uni&_nc_ht=scontent-mia3-1.xx&oh=4f0fe7b304ff0e6cafef38df32dbf60d&oe=6186B2EF
  @width 0.4
\fi

\begin{itemize} % {
\iusr{Валера Короткий}
\textbf{Mikael Multipolar} Миша не поможет?)))..

\iusr{Николай Буренков}

Путин включился в процесс интеграции России и Белоруссии. Вся эта ду * * цкая
комедия с " объединением " тянется уже пятый год, если не больше, и
замечательно иллюстрирует способности наших чиновников к поганой волоките с
одновременным симилурованием кипучей деятельности. За это надлежит вешать на
вонючих верёвках, как рекомендовал В. И. Ленин. Но не за шею.

\iusr{Кирилл Жирихин}
\textbf{Mikael Multipolar} обнулённого своего раньше в гробу увидишь.

\iusr{Irina Neverec}
\textbf{Кирилл Жирихин} а давайте навального в президенты Украины

\iusr{Adam Shohat}
\textbf{Mikael Multipolar}

\ifcmt
  ig https://scontent-mia3-1.xx.fbcdn.net/v/t39.30808-6/243780901_134217755611250_113132306755928647_n.jpg?_nc_cat=100&ccb=1-5&_nc_sid=dbeb18&_nc_ohc=afCD1FJ9dYIAX_aCRFx&_nc_ht=scontent-mia3-1.xx&oh=cb937e593bd5008d91f761837aff9926&oe=6165C8D9
  @width 0.4
\fi

\iusr{Yu Riy}
\textbf{Mikael Multipolar} кто же его посадит, он же памятник. А вот Владимира Ильича пора подменить в мавзолее на Владимира Владимировича, считаю что он заслужил. И желательно не затягивать с этим делом.

\iusr{Emil V Beck}
Передвижной зоопарк?

\iusr{юрий логвинов}
\textbf{Mikael Multipolar} не дождетесь @igg{fbicon.beaming.face.smiling.eyes} 

\iusr{Алекс Кубичек}
\textbf{Mikael Multipolar} и не дождешься

\iusr{Сергей Богданов}
\textbf{Mikael Multipolar} рядом с ним, надо посадить порошенко и остальную братву, которая нажилась и наживается на Украине, грабя свой народ.

\end{itemize} % }

\iusr{Ilya Schu}
А у России какое развитие? ВВП 0\% роста за 10 лет….

\begin{itemize} % {
\iusr{Игорь Рубцов}
\textbf{Ilya Schu}
А вы умеете считать ввп?

\iusr{Елена Шабанова}
\textbf{Ilya Schu} Сказал человек без лица .Как вы достали работники. Ничего в жизни не делающие. Вот это работа. И блин ,общество такого кормит

\iusr{Башкатов Владимир}
\textbf{Ilya Schu} тролль, как говорила Верка Сердючка, сама придумала, сама поверила

\iusr{Natasha Cannon}
\textbf{Ilya Schu} ну, если учесть, как почти весь капитал ваших исключмтельных хозяев из демократий и их еврподсывал все эти годы гробится на изоляцию нашей "бензоколонки", то даже просто ни шагу вниз для нашего валовой продукта - это уже победа над вами.

\iusr{Ilya Schu}
\textbf{Башкатов Владимир} а с чем вы не согласны? Где рост?

\iusr{Ilya Schu}
\textbf{Игорь Рубцов} а зачем мне его «уметь» считать? Его считает ВБ и ЦБ.

\iusr{Ilya Schu}
\textbf{Елена Шабанова} вы в Киеве не согласны с тем, что ВВП России не растёт 10 лет? А вот фразу «ничего в жизни не делающие» применительно ко мне оставьте себе, так как вы, мягко говоря, ошибаетесь.

\iusr{Елена Шабанова}
\textbf{Ilya Schu} Смотрим на аватарку. И, что же там? @igg{fbicon.face.tears.of.joy}{repeat=3} 

\iusr{Ilya Schu}
\textbf{Елена Шабанова} там всего лишь машина, которая мне понравилась. И вы не ответили на вопрос

\iusr{Игорь Рубцов}
\textbf{Ilya Schu}
ВБ и ЦБ никакого отношения к реальной экономике не имеют.
Если исходить из их методик, то при нынешнем уровне цен на нефть и газ, ВВП РФ за этот год вырос на 30%.

\iusr{Ilya Schu}
\textbf{Игорь Рубцов} но он не вырос. 1.6 трлн много лет подряд.

\iusr{Игорь Рубцов}
\textbf{Ilya Schu}

Меня абсолютно не интересует, какие цифирьки рисуют западные банкиры, к реальной экономике это не имеет никакого отношения.

Ещё по ППС как-то можно сравнивать, да и то ...

\iusr{Ilya Schu}
\textbf{Natasha Cannon} 

у меня к вам два вопроса - почему вы поддерживаете режим и живете в оплоте
демократии штате вашингтон? Вы не пробовали любить бензоколонку, находясь на ее
территории, а не в прекрасном тихом месте на берегу океана ? Вы правда
считаете, что при столь высоких ценах на нефть последнее десятилетие в
отсутствии роста экономики виноват коллективный запад, а не тотальная коррупция
и несменяемость власти? И кстати кто виноват в убыли населения последние годы?
Кто виноват в высоких налогах? Низкой пенсии, плохих дорогах? А кто сократил
количество больниц вдвое за 20 лет? А кто увеличил количество чиновников в три
раза? Запад?


\iusr{Елена Шабанова}
\textbf{Ilya Schu} Таких людей в соц. сетях очень много. Вас ведь видно за версту.))) А вы не знаете почему такие проблемы в стране? Вы посчитали бы ВВП Украины. Или вы начитались и наслушались украинской токсичной пропаганды?

\iusr{Ilya Schu}
\textbf{Игорь Рубцов} 

если вас данный вопрос не интересует, зачем вы лезете в дискуссию? А самое
главное, что вы хотите мне доказать? Сформулируйте мысль? Или ответьте на
простые вопросы- Доходы граждан всей страны растут? Пенсии в реальном выражении
растут? Бензин дешевеет? ЖКХ дешевеет? Пром производство может растёт? Может
цены на продукты падают? Я понимаю, что ввп вас не волнует, но может простые
вещи вас волнуют, которые можно потрогать?


\iusr{Ilya Schu}
\textbf{Елена Шабанова} начнём с первого - меня Украина не интересует. Давайте вы там как-то без нас разбирайтесь. Какое-то неведомое государство аннексировало исконно русские территории и переделывает русских в неведомую нацию. А второе - каких «таких» людей?

\iusr{Игорь Рубцов}
\textbf{Ilya Schu}
Я у вас спросил. Вы считали ВВП или оперируете цифирьками банкиров?
Во сколько они оценивают наличие комплексов Авангард, Посейдон, Кинжал, С-500? Как это соотносится с бюджетом к примеру пентагона у которого и близко ничего подобного нет?
Если вы человек в здравом уме, то никогда не будет все эту ахинею писать и ссылаться на неё.
По всем остальным вопросам, все эти показатели в реальном выражении падают по ВСЕМУ МИРУ С 2008 года.
И связано это с мировой долларовой системой, а не с конкретной страной.
А если брать чистыми, то внешний долг США во сколько раз вырос за 10 лет?
И наоборот сколько РФ накопила резервов за тот же срок?

\iusr{Елена Шабанова}
\textbf{Ilya Schu} 

Где там? Да кто же спорит? Что Украина это на самом деле территория России .Кроме
возможно наших пришибленных нациков гуцулов. Вы же понимаете ,что страну душат
санкциями и внутри ,что творят либералы. А ведь много бед именно из за Украины.


\iusr{Ilya Schu}
\textbf{Игорь Рубцов} 

я оперирую цифрами банкиров, потому что это позволяет в одних цифрах сравнивать
экономики, в неработающих с500, как и арматах, как и пак-5, которые все на
картинках и в мультиках все ещё, сложновато. Так вот в одинаковых цифрах наша
экономика 10 лет 1.6трлн долларов, наши звр как были 500 с небольшим млрд, так
и остались, наш госдолг при этом вырос с 50 до 300 млрд. ВВП США 10 лет назад
был 16трлн, сейчас 21трлн, госдолг изменился точно так же. Так что наш госдолг
вырос в разы, американский на проценты. Средний доход американских семей растёт
все последнее десятилетие и превысил 60 тысяч. так что падает только у нас.

\iusr{Ilya Schu}
\textbf{Елена Шабанова} 

либералы сидят по тюрьмам. Все что могут творят чиновники и казнокрады. Санкции
небольшое влияние оказывают. А вообще Либерализм провозглашает права и свободу
каждого человека высшей ценностью и устанавливает их основой общественного и
экономического порядка. Не пойму, чего плохого вы вкладываете в это слово.


\iusr{Natasha Cannon}
\textbf{Ilya Schu} 

о ты, завистливый, ленивый и бестолковый любитель замочных скважин, подальше
полистай и увидишь, что я много где живу - и вовнаукре, и на Урале, и в Баку, и
в Лондоне, и в Стамбуле, в Дохе даже, в Колорадо, и в Москве, и на Каме. Ты не
пробовал изучать мир в его реальности, а не по указивкам наглецов-"демократов"?

Ты никогда не задумывался - как это к нас не было даже слова коррупция до
самого прихода либеральных ценностей 30 лет назад, а то место откуда оно пришло
- его родина, соответственно, демокоатия с либералом - его родители, это их
дитя и менталитет. Про больницы - размеры больниц считаешь или на 5 коек и на
400 - тебе либеральной пофиг?

Кого ты считаешь чиновником - перечисли чины по порядку.

Несмегяемость власти - в сша, где всего две партии почти 300 лет, а байдены с
пелосями и маккейнами по полвека, сидят в капитолиях и прочих сенатах.

\iusr{Елена Шабанова}
\textbf{Ilya Schu} 

В том то и дело ,что люди не имеющие никакого отношения к либерализму так
очернили само это понятие. Что слышать уже невозможно. Да кого же вы считаете
либералами седящими по тюрьмам? У вас чиновников хорошо трусят. И сажают их. В
России всегда все жалуются. Никто и не говорит, что нет проблем. Я сама русская. И
поверьте мне есть с чем сравнивать. Не так все у вас и плохо.

\iusr{Ilya Schu}
\textbf{Natasha Cannon} 

Начнём с того, что как только собеседник переходит на личности - это означает,
что разумные аргументы кончились. И кстати, без ошибок можете писать, а то не
понятно, что такое вовнаукре? Несменяемость власти это Путин с 1999 года, а
Клинтон, Буш, Обама, Трамп и Байден за тот же период - это не похоже на
несменяемость. Либерализмом в России не пахнет, в РФ попахивает ЦК и застоем.
Что касается отсутствия иностранного слова в лексиконе не отменяло
существование казнокрадства, которое при совке было и обуславливало прекрасную
жизнь близких к «телу» поваров, завхозов, начальников распределителей и тп, я
уж молчу про отдельную жизнь номенклатуры с «березками», ресторанами и прочими
благами.

\iusr{Ilya Schu}
\textbf{Елена Шабанова} если сравнивать с зимбабве, то в России все прекрасно. Если сравнивать в целом с приличными экономиками, то мы много потеряли за последнее десятилетие. И станет хуже. Потому что ничего не происходит кроме воровства и повышения поборов.

\iusr{Natasha Cannon}
\textbf{Ilya Schu} 

первым ты перешёл на мою личность, когда стал шариться по моему аккаунту, это
для меня характерный признак определённых персон из интернета.

Вовнаукра это то несчастье, что нынче выделывается на остатках территории
бывшей усср.

Про что такое несменяемость власти не греми погремушкой, не убедительно.
Перечисленные тобой клинтоны и обамы ничто, так как власть не у них, а у
партии, для которой они служат петрушками, которых никто всерьёз не принимает.
Повдыхать вонючий либерализм России хватило 8-10 лет, до сих пор острая
аллергия на такой. У нас было другое понимание о том, что значит, быть
либеральным, что такое демократия, это не то, что несут ястребы и гиены в
шкурах ослов и слонов.

Если слова в лексиконе не было, то, значит, не было и того, что им описывалось.
А вот отдельные блага - это как раз из разделов западных демократий с их
благами исключительно для элит.


\iusr{Елена Шабанова}
\textbf{Ilya Schu} 

Ну зачем же Зимбабве. Но зря вы так. Много в России делается Да, бедность это бичь
страны. Не забывайте, что КОВИД то никто не отменял. И в странах которыми мы
считаем благополучными ничуть дела обстаят не лучше.

\iusr{Игорь Рубцов}
\textbf{Ilya Schu}

Ещё раз, это не одинаковые цифры, от слова совсем. У нас экономика рублевая.
Если хотите сравнивать, то хотя бы нужно применять индекс бигмака, согласно ему
наша уже повышается в три раза.

Далее, львиная часть ввп сша в услугах, которые отношения к реальной экономике
не имеют. Один и тот же консалтинг может стоить 1000\$, а может в стони раз
больше. Поэтому показатель экономики США завышен, как минимум в 2 раза.

Про мультики Турции, Индии, Китаю расскажите, а то стоят в очередь за покупкой
мультиков.

А уж скромное долг сша увеличился пропорционально экономике повеселило.

Можете впаривать это все школьникам, но они на других ресурсах и ветках сидят,
а здесь лучше не позориться.

\iusr{Emil V Beck}
За последние 10 лет ВВП России вырос почти в два раза. Это легко проверить.

\iusr{Ilya Schu}
\textbf{Игорь Рубцов} 

госдолг был в 2013 17млрд, сейчас 26. Где тут разы, о которых вы говорили?
Львиная доля в сфере услуг? Открыл данные, 29\%. Это львиная? дальше
дискутировать о ввп не вижу смысла. Вы сначала с цифрами в своей голове
разберитесь. Теперь что касается мультиков. Сколько С500 на вооружении стоит?
Ноль? Сколько Армат на вооружении? Я знаю про пару. Сколько истребителей пятого
поколения? Ноль. И очереди за этим не видно. Особенно со стороны китайцев.
Можете впаривать школьникам, а здесь лучше не позориться.

\iusr{Ilya Schu}
\textbf{Natasha Cannon} 

я не помню, чтобы мы с вами переходили на ты... но уточните, признак каких же
персон из интернете является то, что мне стало интересно, откуда пишет
собеседник, если вас так, конечно, можно назвать? Я с вашего позволения, заменю
слово либерализм на «свободное общество» что собственно это слово и обозначает.
Так вот времена настоящего свободного общества пришлись на нефть по 7\$ и на
руины советской экономики, которая никогда не могла конкурировать на мировой
арене, а дальше власть взяли чекисты, нефть стала расти и идеи свободы стали
душиться ради того, чтобы все блага от высоких цен доставались узкому числу
близких людей, которые по факту контролируют большую часть экономики и не дают
ей развиваться, все больше втравливая страну в репрессии ради удержания себя у
этой жирной кормушки. По итогу мы имеем нулевой рост экономики десятилетие,
отсутствие дорог, аэропортов, образования, больниц и единственным экспортным
продуктом страны являются природные ресурсы, нищенские пенсии и средние
зарплаты, при этом имеем платные дороги, шведские налоги в размере почти 50\% на
физических лиц и почти 40\% на юридических, отсутствие гарантии частной
собственности и справедливого суда. Хотя, понимаю, резидентов штата вашингтон
это не сильно волнует.


\iusr{Ilya Schu}
\textbf{Emil V Beck} 

В 2013 ввп был 2.2трлн \$. Сейчас 1.6. В рублях можно что угодно рисовать.
Десять лет назад он был 70трлн рублей, сейчас 100. Завтра рубль будет 200 и ввп
станет 200. Кстати двумя разами не пахнет. Напомню рубль с тех пор упал с 24
рублей до 36 и потом до 75.

\iusr{Andrey Vardugin}

Это какие данные вы открыли? По разным оценкам (включающими или нет те или иные
отрасли в услуги, например, платное образование ), доля услуг колеблется от 65
до 78 \% ВВП. Если у населения нет денег или желания их тратить на услуги, эта
отрасль быстро схлопывается. Можно напечатать и раздать денег населению, как в
пандемию, но инфляция это неминуемая плата.

А за наш ВВП вы не переживайте, растёт, как бы его не считали. Строительный бум
и дефицит автомобилей о чём-то должны говорить.

\iusr{Natasha Cannon}
\textbf{Ilya Schu} 

Есть персоны в интернетах - типа тебя, которые лазят по аккаунтам оппонентов,
вот таким я сразу начинаю тыкать, так как они мне становятся этакими знакомыми
из породы сплетниц.

Словосочетание "свободное общество" так же требует пояснений - от чего оно
свободно, для кого оно свободно, за счёт чего и кого оно свободно, ну и тд.
Советская экономика была мощнейшей в мире в 20м веке, особенно если учесть, чтО
против неё вытворяли "свободные".

Нулевой рост экономики - это точно не про Россию, не болтай своей ерундой
зазря)) Шведские налоги и нравы нас не колеблют, пусть живут как хотят. Про
резидента вашингтона не понятно - это ты про себя или про своего хозяина?

\iusr{Ilya Schu}
\textbf{Andrey Vardugin} например вот 

\url{https://www.statista.com/statistics/248004/percentage-added-to-the-us-gdp-by-industry}

где же 70 процентов?

\iusr{Andrey Vardugin}
Ну так просуммируйте сервисы. 67\%

\iusr{Ilya Schu}
\textbf{Natasha Cannon} 

у вас открыт профиль, пыхтеть за Путина из вашингтона - это клинический случай.
Вернитесь, будем говорить. Если советская экономика была самая мощная в мире,
что же она рухнула при минимальных ценах на нефть? Если она была такая сильная,
что же все бросились покупать Машины в Европу? Что же вы мечтали о джинсах
ливайс? Что же магнитофоны Панасоник рванули доставать? А чего на Жигули
очередь была годы? А чего пустота на прилавках была в 80е? Откуда дефицит? А
чего простую колбасу надо было доставать? Ну если вы отрицаете, что рост ввп
ноль за 10 лет, то сидите в своём вашингтоне и наслаждайтесь. У вас там и
правда рост ввп.

\iusr{Emil V Beck}

Что за глупости? Причем тут доллары? Да, в результате кризиса 15 года, доллар
вырос, и что? Это "падение" - чисто формальное. Мы в Рлссии живем, и наша
денежная единица - рубль. Никакого спада производства или жуткой инфляции не
было. Так что, свои нарисованные цифоы можете свернуть в трубочку и сами
знаете, чтр сделать. ВВП вырос за 10 лет в два раза. Это немного. Но он вырос.

\iusr{Natasha Cannon}
\textbf{Ilya Schu} 

что тебя так на вашингтоне заело-то? почему не на Москве или львове, у меня
оттуда тоже есть фоточки)) Советская экономика рухнула, потому, что тогда
пришло время сбросить с хвоста всяких нацнахлебников, которые до сих пор
поверить не могут, что их оставили без халявы, некоторые реформации в самой РФ
после этого потребовали времени.

кому нафер сдался магнитофон панасоник, у нас дома были комета, горизонт, зил,
жигули, зенит, тд Пустота в 80х была временная, такая сегодня в демостранах
случается регулярно. Колбасы хорошей сегодня фиг купишь в капиталистических
супербазарах, даже хлеба нет хорошего у них.

Подпишись на "сделано у нас" и на "время вперёд", и узнаешь правду, а то даже
не представляю, в какой инфопомойке ты роешься.

\iusr{Башкатов Владимир}
\textbf{Ilya Schu} тролль, обратись к узкопрофильному специалисту срочно

\iusr{Егор Шалыгин}
\textbf{Ilya Schu} Очень шаблонно, вы бот или вообще нейросеть?

\iusr{Ilya Schu}
\textbf{Emil V Beck} 

не было инфляции? Бензин стоил в 2008 году 17 рублей, сегодня 50. Средняя цена
лосося была 200 рублей, сейчас 1000, молоко в те годы стоило около 25 рублей,
сейчас 80 в среднем, машина типа Рено Логан стоил 350-400 тысяч, а сейчас это
почти миллион. Нет инфляции? Она вас не затронула? Единица рубль? Курс вообще
не влияет на вас? Вы эту соловьевскую пропаганду мне-то не заливайте. А вы как
получали 50-150 тысяч так и получаете.

\iusr{Emil V Beck}

Цены всегда и везде растут, во всем мире. И зарплаты растут не так быстро. И
инфляция есть. Я написал, что не было резких скачков инфляции. Если вы бъетесь
в истерике, и вам всюду мерешится соловьев, то вам лучше к врачу.


\end{itemize} % }

\iusr{Марина Синицына}
Как это точно вы сказали! @igg{fbicon.thumb.up.yellow} 

\iusr{Кира Берестенко}
Так никто же не обуздает, а мы, его население, в заложниках у этих "слуг".

\iusr{Роберт Давтян}
"Режим" только у вас в Киеве плохая хорошая, но избранная власть

\iusr{Leo Rossi}
Жизнь по принципу «наглость - второе счастье» примитивна и смешна  @igg{fbicon.laugh.rolling.floor} 

\iusr{Karine Bagdasarova}
Слишком много эмоций в тексте у вас. Слишком по-женски, цитаты из песен в постах политика -это конечно уровень дипломата.

\begin{itemize} % {
\iusr{Надежда Владова}
\textbf{Karine Bagdasarova} Это личная страница, Вы не забыли?)

\iusr{Лана Стар}
\textbf{Karine Bagdasarova} вот Карине бы написала так написала  @igg{fbicon.biceps.flexed}  @igg{fbicon.face.clown} 

\iusr{Гарик Шахназаров}
\textbf{Karine Bagdasarova} расскажите немного о своих успехах в дипломатии

\iusr{Роман Лавров}
\textbf{Karine Bagdasarova} это уровень и более того! И цитаты, а чем проще цитаты, тем выше качество - простота это достоинство сильных.

\iusr{Наталья Герцен}
\textbf{Karine Bagdasarova} милейшая, вас там талибы должны волновать, а не наша внешняя политика и наши дипломаты!!! 
@igg{fbicon.hand.victory}{repeat=3}

\iusr{Лариса Михайлова}
\textbf{Karine Bagdasarova} ну, на своей личной страничке Мария вольна в выражении эмоций. Это вам не официальный сайт МИДа)))

\iusr{Татьяна Хейфец}
Девочку в Лондонской школе видимо научили стилю Хайли Лайкли
\end{itemize} % }

\iusr{Ирина Юрьевна}
Какая прелесть! Действует по принципу " наглость-второе счастье"  @igg{fbicon.laugh.rolling.floor}{repeat=3} 

\begin{itemize} % {
\iusr{Serhii Yushko}
\textbf{Ирина Юрьевна} так это главный машкин принцип, вы только заметили?

\iusr{Igor V Borisenko}
В зеркале снбя давно видел?

\iusr{Serhii Yushko}
\textbf{Igor V Borisenko} на смену заступило? Не отвлекайся

\iusr{Елена Шабанова}
\textbf{Serhii Yushko} Какой ужас @igg{fbicon.face.tears.of.joy}{repeat=3} 

\iusr{Serhii Yushko}
\textbf{Елена Шабанова} ага @igg{fbicon.face.tears.of.joy}{repeat=3} 

\iusr{Андрей Ш.}
\textbf{Ирина Юрьевна} Политика - как и восток - весч тонкая...

\iusr{Дмитрий Космодемьянский}
\textbf{Ирина Юрьевна} обычная хуцпа

\iusr{Igor V Borisenko}
\textbf{Shehil Yushko}, брысь под лавку!

\iusr{Андрей Ш.}
Наглость - не счастье, а - необходимость в современной политической ситуации.
\end{itemize} % }

\iusr{Марина Троицкая}

Публикация лишь" ради красного словца " в дипломатии не органична, ибо
поднимает " бурю в стакане", трёп на завалинке.Из пустого в порожнее - для
чего?

\iusr{Сергей Корычев}
Когда тебе кажется что ты наглосакс, но, оказывается, просто наглый, без сакса.

\iusr{Julia Po}

Помню, смотрела видео, где Путин отвечал на вопросы .. Он сказал, что Украине
предлагали газ по самой выгодной цене, дешевле чем всем в Европе. Вот я живу в
Европе, и не понимаю как можно от такого отказаться  @igg{fbicon.shrug}{repeat=2}  страна, в которой
правят деньги не может от такого отказаться ..

\begin{itemize} % {
\iusr{Лариса Михайлова}
\textbf{Julia Po} что там правит, одному Богу известно))) да и Он, наверное, недоумевает от такой логики: покупать российский же газ, только по реверсу, да ещё и с наценкой. Логика - мы нищие, но гордые))

\iusr{Олег Передерий}
\textbf{Лариса Михайлова} ну не понять вам, ну шо ж тут поделаешь

\iusr{Лариса Михайлова}
\textbf{Олег Передерий} так пояснили бы. Или слабо?)))

\iusr{Олег Передерий}
\textbf{Лариса Михайлова} та конечно слабо. как можно объяснить рабу, шо такое свобода?

\iusr{Лариса Михайлова}
\textbf{Олег Передерий} ага))) особенно, когда путают понятия свободы и вседозволенности.)) Когда без указаний и денежных вливаний от зарубежных кукловодов шагу не ступить - вот это ваша свобода.  @igg{fbicon.face.tears.of.joy} 

\iusr{Олег Передерий}
\textbf{Лариса Михайлова} ну я ж говорю, бессмысленно
\end{itemize} % }

\iusr{Людмила Гладишенко}
Лучше бы танцевала с листиками!

\iusr{Магомед Ториев}

Захватили Крым, часть Донбасса, часть Грузии, часть Молдовы и рассуждаете о
большой политике. Это как начальник Дахау или Освенцима написал бы статью о
гуманизме и милосердии. Стыд конечно не дым, глаза не выест. Можете продолжать
отрицать реальность, заниматься США, ЕС и Украиной больше, чем Россией, но всё
имеет свой конец. Конец СССР уже мы видели, а из его 70 лет у вас уже есть 30.
Но там и конструкция была покрепче, идеология жёстче, а что вы можете
предложить городу и миру? Подождём.

\begin{itemize} % {
\iusr{Андрей Ш.}
\textbf{Магомед Ториев А}, людей спрашивали?

\iusr{Edward Berenstein}
\textbf{Магомед Ториев} Не дождёшься ублюдок

\iusr{Александр Туман}
Лучше не скажешь.

\iusr{Elka Eleni}
\textbf{Магомед Ториев} что курите, или пьёте, или жрете!!?? Чапай в свою страну и пиши о ней, магомед тариев!

\iusr{Serga Rudy}
\textbf{Магомед Ториев} какая каша... @igg{fbicon.beaming.face.smiling.eyes} просто поразительно

\iusr{Serga Rudy}
Маг, Вы не в ту страну уехали... @igg{fbicon.beaming.face.smiling.eyes} 

\iusr{Владимир Липатов}

\href{https://youtu.be/3u4WKbrxFd4?t=28}{%
Александр Блок, "Скифы", youtube, 25.01.2015%
}

\iusr{Магомед Ториев}
\textbf{Андрей Ш.} Вы имеете ввиду. 75 тысяч изгнанных российской армией в 1992 году ингушей? 300 000 изгнанных грузин? Или 250 000 убитых чеченцев или 13 000 убитых украинцев, украинцев изгнанных из Крыма и Донбасса, кого именно спросить то надо. Уточните! Пожалуйста!

\iusr{Наталья Михалёва}
\textbf{Магомед Ториев} историю изучайте лучше,а не глупости пишите.

\iusr{Магомед Ториев}
\textbf{Elka Eleni В} ..... ваше направление.

\iusr{Денис Плебейский}
"Захватили Крым, часть Донбасса, часть Грузии" пока всё идёт неплохо, не находите?

\iusr{Влад Щукин}
\textbf{Магомед Ториев} так ей и положено заниматься США, ЕС И Украиной, Мария же в МИД работает, если вы не в курсе

\iusr{Андрей Ш.}
А, ещё - 28000000 граждан СССР в период 1941-1945.

\iusr{Борис Заблоцкий}
\textbf{Магомед Ториев} ой, смотрите какой смелый, поглядывает он из Праги  @igg{fbicon.face.grinning.squinting}  @igg{fbicon.laugh.rolling.floor}  ... Не устань ждать, и штаны не протри.

\iusr{Магомед Ториев}
\textbf{Наталья Михалёва} Мой народ и я в том числе на своей шкуре изучили вашу кровавую историю.

\iusr{Магомед Ториев}
\textbf{Борис Заблоцкий} А надо рядом с Навальным сидеть и ждать?

\iusr{Наталья Михалёва}
\textbf{Магомед Ториев} Русские потеряли только в Великой отечественной войне 27 миллионов лучших людей.А сирот,вдов,искалеченных не сосчитать.А
малые народы многие получали отсрочку от армии.40\% промышленного потенциала СССР было уничтожено.

\iusr{Борис Заблоцкий}
\textbf{Магомед Ториев} Ну у каждого свои примеры для подражания, если вам рядом с овальным комфорно, то как вам угодно.

\iusr{Игорь Рубцов}
\textbf{Магомед Ториев}
Кто отстроил Крым, Донбасс, Грузию, Молдавию?
А теперь сравним, что отстроили на похожих территориях местные князьки за 30 лет независимости, как развили, сколько населения прибавилось?
Давай расскажи про небывалые успехи триебалтов, как вся Европа стоит в очереди к ним на гражданство, как процветает солнечный Тбилиси по сравнению с тоталитарным совком.
Как счастливые украинские рабочие авиа, авто, вертолёто, кораблестроения приветствуют своего президента.

\iusr{Зинаида Воловидник}
\textbf{Магомед Ториев} 

никто ничего не взял, а взяли свое... Крым был российским всегда, и
Приднестровье также. А еще надо было обходиться нормально с людьми живущими на
этих территориях нормально. А не творить беспредел. Тогда бы не было всяких
взятий.


\iusr{Ruslan Yurin}
\textbf{Магомед Ториев} пражское имя - магомед  @igg{fbicon.smile} 

\iusr{Неля Гринюк}
\textbf{Зинаида Воловидник}. Я так понимаю, что весь мир принадлежит рассеи? А желание этих народов вам не интересно?

\iusr{Магомед Ториев}
\textbf{Igor Rubtsov} 

И в странах Балтии и в Украине нет руин Грозного, фильтрационых лагерей
Чернокозово и Моздок, концлагеря Майрамадаг, их матери не получали гробов из
Чечни, Сирии, Грузии и Украины. А теперь, когда вы захватили части Украины и
Грузии и воюете везде по всему миру, не вам с вашими Тимченко, Кадыровым и
Ротенбергами рассказывать про местных князьков и счастливых рабочих авиапрома.
Космодром Восточный достройте и зарплату не забудь пожалуйста выплатить рабочим


\iusr{Ruslan Yurin}
\textbf{Магомед Ториев} шкурами быть не надо

\iusr{Игорь Рубцов}
\textbf{Магомед Ториев}
Положим чеченская история началась опять же с местных князьков, которых активно поддерживали внешние силы.
Что было в период между 1 и 2 Чеченской напомнить?
Что уже кто-то из бывших республик свой космодром уже достроил, мы уже отстаём?
Прикольно наверно сидя в Праге, переживать за матерей из России, за Молдавию, Грузию, Украину. За это хоть платят или так душа болит от безделья?
Есть ещё возможность попереживать за население Ирака, Ливии, Йемена, Сирии далее по списку.

\iusr{Анастасия Шмелёва}
\textbf{Магомед Ториев} да, аж в 18 веке присоединен был к территории империи. Донбасс тоже входил в состав империи. Все возвращается на законные места.

\iusr{Лариса Михайлова}
\textbf{Магомед Ториев} хорошо вещать из дальнего зарубежья, да? @igg{fbicon.face.tears.of.joy} 

\iusr{Мурат Еникеев}
\textbf{Магомед Ториев} ходьба по граблям, любимое у нас занятие

\iusr{Andrii Kolesnikov}
\textbf{Наталья Михалёва} , ты и трактор))

\iusr{Andrii Kolesnikov}
\textbf{Игорь Рубцов} , Крым отстроила Украина за счёт своего бюджета. С абсолютного нуля.

\iusr{Наталья Михалёва}
\textbf{Andrii Kolesnikov} ума палата.

\iusr{Наталья Михалёва}
\textbf{Магомед Ториев} учите историю. Нет "вечной" дружбы ,пора и честь знать,отдай то ,что тебе не принадлежит.


\ifcmt
  ig https://scontent-frt3-1.xx.fbcdn.net/v/t1.6435-9/243736117_2163712270454357_7638226775968629919_n.jpg?_nc_cat=108&ccb=1-5&_nc_sid=dbeb18&_nc_ohc=H3NeWpR_njIAX8SxHpX&_nc_ht=scontent-frt3-1.xx&oh=a780139ba90277625f8ce00ac3f1e185&oe=6187F30D
  @width 0.4
\fi

\iusr{Наталья Михалёва}
\textbf{Магомед Ториев} 

Грозный русский город. Грозный — окружной город Терской области, на обоих
берегах реки Сунжи, притока Терека, на высоте 419 фт. над уровнем моря. Только
с 1870 г. переименован в окружной город; до того на этом месте находилась
крепость Грозная, построенная в 1818 г. по приказанию Ермолова для борьбы с
чеченцами. Благодаря удачно выбранному месту сооружения крепости в тот же год
была завоевана полоса земли между Тереком и Сунжей. Крепость Г. послужила
началом образования так называемой "Русской дороги", проложенной от одной
крепости к другой через недоступную до тех пор Малую Чечню. Более 6 тыс.
жителей; два раза в году ярмарки.


\iusr{Наталья Михалёва}
\textbf{Andrii Kolesnikov} Крым-Таврида всесоюзная здравница и все ,что там было и есть ,сделано на деньги России.

\iusr{Andrii Kolesnikov}
\textbf{Наталья Михалёва} , на деньги Украины. Учите историю, используйте Гугл, отключите раша-зомбоящик).

\iusr{Наталья Михалёва}
\textbf{Andrii Kolesnikov} 

После завоевания Батыем Русской земли и основания Золотой орды Таврида подпала
под власть татар. Сношения России с Крымом начались еще при Иване III, который в
1474 году заключил с Менгли-Гиреем союз, пообещав ему убежище у себя в случае
нужды. С тех пор сношения поддерживаются постоянно и носят то дружественный, то
воинственный характер. Главным предметом для дружественных сношений служила
торговля, причем крымцы старались получить как можно больше выгод, добиться
беспошлинной торговли. Это им нередко и удавалось ввиду того, что внезапные и
опустошительные набеги крымцев были очень опасны и тяжелы для Москвы, и
последняя старалась всячески предупреждать их. Благодаря этому, например,
крымским торговцам удалось освободиться при Василии III от платежа тамги и т.
п. Другим предметом для сношений с Москвой служили подарки, которых требовал не
только хан, но его двор, бей, чиновники и пр. Москве приходилось платить очень
много, чтобы только обезопасить себя кое-как от нападений. В 1648 г. московский
царь предлагал даже хану запретить чиновникам обращаться за подарками к
московскому царю. Подарки эти обратились в своего рода дань, которая вымогалась
при всяком удобном и неудобном случаях. В 1592 г. татары требовали у русского
царя денег на постройку городов, как обычной дани, а в 1631 г. татарское
посольство укоряло царя, что город Перекоп обветшал, и требовало, чтобы царь
починил его. Малейшее несогласие или недовольство крымцев влекло нападение их
на русские пределы. Нападения эти были так часты, что трудно даже уследить за
ними по источникам. Носили они характер чисто разбойничьих наездов: приходили
татары, обыкновенно, неожиданно, грабили жителей, жгли села и затем уходили,
уводя за собой толпы пленников и пленниц, которые продавались затем в Кафе. В
первое время после основания крымского X. набеги были очень часты: в 1532,
1535, 1536, 1562, 1571, 1591, 1592, 1595 и т. д. В 1571 г. крымский хан доходил
даже до Москвы и сжег ее. В XVII в. набеги повторяются уже реже. Это
объясняется, с одной стороны, теми неурядицами, которые господствовали в Крыму,
и необходимостью для крымских ханов участвовать в многочисленных войнах,
которые вела Турция, с другой — и тем, что Русь значительно укрепила к этому
времени южную границу рядом крепостей и колонизовала ее служилыми элементами.

\iusr{Наталья Михалёва}

С конца XVII столетия, особенно после присоединения Малороссии, Россия
деятельно начинает стремиться на юг в Крым. Первой серьезной, хотя и неудачной
попыткой осуществить это стремление, если не считать малоизвестного похода
русских в 1559 г. — были крымские походы в правление Софьи Алексеевны. Затем
следует поход графа Миниха при Анне Ивановне в 1736 г., когда русские проникли
в Крым и опустошили его. В 1737 и 1738 гг. проникал в Крым граф Ласси. В 1771
г., во время первой турецкой войны, в Крым был послан кн. Василий Mиx.
Долгоруков, который скоро овладел всем полуостровом. С этого времени Россия
считала себя уже обладательницей Крыма, стараясь свергнуть с него опекунство
Турции. Поэтому по Кучук-Кайнарджийскому миру (1774), Крым был объявлен
независимым от Турции, а России была предоставлена свобода мореплавания на
Черном море. Шагин-Гирей начал после этого преобразовывать X. на европейский
образец, но это взволновало духовенство, а через него и народ, который особенно
был недоволен тем обстоятельством, что государственные доходы были отданы на
откуп и ему приходилось тяжело. При таких условиях хану нужно было скрываться в
горах. Здесь, по поручению Потемкина, ухавшего в то время в С.-Петербург,
явился к хану с конвоем генерал Игельстром и стал уговаривать "отдаться
милостям монархини". Хану ничего не оставалось делать, как последовать этому
совету. Он отправился в Керчь, а оттуда был выслан на житье в Воронеж ("Записки
Л. Н. Энгельгардта"). Вскоре, 9 апреля 1783 г, Крым был объявлен присоединенным
к России. Турция признала это совершившимся фактом только по миру в Яссах 25
декабря/6 января 1792 г. В 1861 г. крымский народ исчезает фактически. Он
оставил Крым, примкнул и затерялся в массе мусульманства в Турции.

\iusr{Andrey Vardugin}

Мага,э! Ингуши, нохчи, даги, авары, лаки, все дома живут, род свой ведут, дела
делают. Ты один как пёс из под Праги гавкаешь. Нет у тебя права жаловаться и
требовать. Про фашистов своим друзьям в Белых касках рассказывай. Они тебе
поверят, поддержат тебя, погладят, поцелуют ну ичто вы там дальше делаете.

\iusr{Наталья Михалёва}
\textbf{Andrii Kolesnikov} 

Украины, как государства никогда не было. Украиной называлась часть русских
земель, временно находившихся под управлением Польши, которая сама ибыла сто лет
в составе России. @igg{fbicon.face.grinning.squinting} 

\end{itemize} % }

\iusr{Slava Taymazov}
Опусы Захаровой как эталон языковой безвкусицы. Вы бы ей в хоть копирайтера наняли, чтоб не позорилась.

\iusr{Владимир Дзюра}

Вот!! Наконец-то правду сказал!!! Ведь это же наглость. Газпром заключил
контракт с Венгрией. У Украины истерика. Венгрия @igg{fbicon.flag.vengria} их не спросила, лишила их
транзита. Две суверенные страны забыли спросить у третьей, что им делать!!!
Офигеть наглость. Тоже самое с СЕВЕРНЫМ ПОТОКОМ. Почему Германия вам (Украине)
что то должна???  @igg{fbicon.thinking.face}  Поражает!!! Это даже не наглость, это уже маразм!!!

\begin{itemize} % {
\iusr{Serhii Yushko}
\textbf{Владимир Дзюра} а когда чувак двадцать лет у власти без смены это как называется? Слово маразм тут слишком легкое

\iusr{Татьяна Кравцова}
\textbf{Serhii Yushko} да уж...у вас меняется правитель.поэтому вы и живете все " лучше и лучше" . Ну.по вам заметно очень...

\iusr{Наталья Герцен}
\textbf{Serhii Yushko} а вы до  @igg{fbicon.face.clown}  уже довыбирались! Очень заметен квартал 95 во власти!!!

\iusr{Андрей Воронов}
\textbf{Serhii Yushko} это вы про Меркель (16лет у руля) или Рузвельта, прошедшего четверо выборов?

\iusr{Serhii Yushko}
\textbf{Tatjana Kravtsova} жалко по вас незаметно, потому что нет реального профиля и фоток

\iusr{Olga Kuznetsova}
\textbf{Serhii Yushko} А мы посчитали и решили, что подобный калейдоскоп президентов, как у вас, слишком дорог для нас. Так и страну можно потерять. Это вы живете без оглядки, а нам такие эксперименты не нужны.

\iusr{Serhii Yushko}
\textbf{Наталья Герцен} а почему не Огарева?  @igg{fbicon.laugh.rolling.floor}{repeat=4} 

\iusr{Serhii Yushko}
\textbf{Андрей Воронов} в России реально есть выборы?  @igg{fbicon.laugh.rolling.floor}{repeat=5}  это уже анекдот

\iusr{Serhii Yushko}
\textbf{Olga Kuznetsova} вы? Посчитали? А на выборах разве вас кто-то спрашивает? У нас это всё реальное, пробы и постсовковые ошибки

\iusr{Владимир Дзюра}
\textbf{Serhii Yushko} во первых не чувак! Во вторых сколько сделано за эти 20 лет. Вспомнить где, в какой заднице была российская армия, медицина, дефолт наконец-то!!!! Так что всего за 20 лет этот как вы выразились "чувак" (это для вас!!! Для меня уважаемый человек!) вернул страну в правильное русло!!! А кто был ни кем, так и останется!!! Не надо все беды на руководство спихивать!!! Особенно, лёжа на печи!!!

\iusr{Татьяна Кравцова}
\textbf{Serhii Yushko} ну уж. чья бы... у меня все реально. а ползать всяким по моему профилю нечего... глазки протри или проспись...

\iusr{Olga Kuznetsova}
\textbf{Serhii Yushko} Спрашивали, мы голосовали. В очереди даже стояли.

\iusr{Serhii Yushko}
\textbf{Владимир Дзюра} какие чувства возникают, когда в рф не живешь... как оно в Германии, не дует? @igg{fbicon.face.wink.tongue} 

\iusr{Владимир Дзюра}
\textbf{Serhii Yushko} не дует!!! Вы уже всем надоели, и Германии в том числе!!! Поэтому, даже и не мечтайте о Европе!!! Не будет этого НИКОГДА!!!! А теперь, смейтесь!!! Хоть ухохочись!!!!

\iusr{Serhii Yushko}
\textbf{Владимир Дзюра} да чего смеяться, тут грустно как раз... что ж не возвращаетесь в такой построенный за 20 лет рай?

\iusr{Владимир Дзюра}
\textbf{Serhii Yushko} я с Казахстана. И почему я должен куда-то возвращаться? Это у вас на Украине такой бред. Чуть, что чемодан, вокзал. Мне нравится внешняя политика Путина, а ещё Трампа, а ещё Китая и что? По вашей логике мне туда? Какая то странная формула у вас!!!

\iusr{Serhii Yushko}
\textbf{Владимир Дзюра} ну так в Китае и РФ - недемократические авторитарные режимы, а Германии - демократия... какой разброс, однако... а в Украине у нас демократия, люди сами делают свой выбор, в отличие от РФ и КНР

\iusr{Владимир Дзюра}
\textbf{Serhii Yushko}  @igg{fbicon.face.tears.of.joy}{repeat=3} а я думал день скучно пройдёт!!! Вот теперь действительно смешно!!!!

\iusr{Владимир Дзюра}
\textbf{Serhii Yushko} 

про Америку забыл сказать! Всё не надо больше отвечать. Живите дальше в своих
дебрях. Где вам все должны непонятно за что? Только и можете с протянутой рукой
бродить по миру!!!! Да слезы  @igg{fbicon.face.crying.loudly}  лить!!! Возьмите нас в НАТО, ну возьмите нас в
@igg{fbicon.flag.eu}  ЕС. Мы с Россией воюем за вас!!! А Россия начала только учения вблизи
границ Украины, так тут же ВОЙ на весь мир. Они нападут на нас. Вы же воюете с
Россией уже 7 лет? Сказочники!!! Сами рассказываете сказки, сами в них
верите!!! Всё досвидос!!!

\iusr{Serhii Yushko}
\textbf{Владимир Дзюра} меньше смотри раша тв и всё нормализуется

\iusr{Владимир Дзюра}
\textbf{Serhii Yushko} 

у меня то как раз всё прекрасно, в отличии от вас. А вот вы побольше смотрите
ваше  @igg{fbicon.television}  тв!!! Они ещё много сказок вам расскажут!!! И видимо не без успехов!!!
Но нечего, в 24ом, опять какое нибудь брехло выберете!!! Лучше 20 лет у руля
реально лидер, ну или 16 как в Германии, чем не пойми что каждые пять лет!!!!
На этом всё.

\iusr{Наталья Герцен}
\textbf{Serhii Yushko} ну может ваш Зелебобик и с Огарева? Вам виднее. Вижу, что у вас ну очень «глубокие знания» о жизни в РФ! Повеселил сегодня нас! @igg{fbicon.laugh.rolling.floor}{repeat=3} 

\iusr{Лариса Михайлова}
\textbf{Serhii Yushko} а когда один клоун сменяет другого - это как называется?)) Меркель почти 17 лет у руля стояла, но вы немцев маразматики и не называли)))

\iusr{Лариса Михайлова}
\textbf{Владимир Дзюра} , берегите бисер, не мечите его перед... таким товарищам все равно ничего не докажешь)))

\iusr{Елена Бакунина}
\textbf{Владимир Дзюра} , 

Именно! Даешь Россию без газа! Напомню: 1920-е годы Венгрия не наладила
отношений с СССР. Советский Союз Хорти считал источником «вечной красной
опасности» для всего человечества и выступал против установления с ним любых
отношений. А в ВОВ наладила военное производство для Германии. Догадайтесь,
против кого Венгрия воевала? Подскажу: против СССР

\iusr{Владимир Дзюра}
\textbf{Елена Бакунина} я знаю с кем в коалиции была Венгрия во время ВОВ. Мне не надо играть в угадайку. Вопрос, к чему вы это? Мы в 21 веке. Конечно забывать историю не нужно, но и враждовать не стоит. (И меньше слушайте Темнибока!!) И ещё, по вашему Россия до сих пор не должна иметь нечего общего с Германией из-за нападения Гитлера? К чему был ваш комментарий, ещё раз?  @igg{fbicon.thinking.face} 
\iusr{Serhii Yushko}
\textbf{Лариса Михайлова} Меркель реально выбрали на свободных выборах, а кто выбирал и обнулял вашего пыню?

\end{itemize} % }

\iusr{Михаил Суханкин}
Стоша Говнозад Тихона пальцем.

\end{itemize} % }
