% vim: keymap=russian-jcukenwin
%%beginhead 
 
%%file 09_11_2020.fb.fb_group.ukrainska_mova_dlja_vsih.1.jazyk_moskovity_mova
%%parent 09_11_2020
 
%%url https://www.facebook.com/groups/367613903441695/posts/1692249730978099
 
%%author_id fb_group.ukrainska_mova_dlja_vsih
%%date 
 
%%tags jazyk,moskovia,mova
%%title мова московитів - штучна мова
 
%%endhead 
 
\subsection{Мова московитів - штучна мова}
\label{sec:09_11_2020.fb.fb_group.ukrainska_mova_dlja_vsih.1.jazyk_moskovity_mova}
 
\Purl{https://www.facebook.com/groups/367613903441695/posts/1692249730978099}
\ifcmt
 author_begin
   author_id fb_group.ukrainska_mova_dlja_vsih
 author_end
\fi

Деякі наші вперті друзі почали поширювати вже геть неоковирні речі. Річ в тім,
що це так зване "мовне дерево" - абсолютно хибне, адже мови постають не так, як
тут зображено, а зовсім навпаки.

Наприклад, мова московитів - штучна мова. Вона постала зі староболгарської
книжної та безлічі запозичень, передовсім татарських, потім інших народів, але
іноземних, та тільки не з мов своїх власних автохтонів (з мови автохтонів
лишилася хіба що гідротопонімика та подібні речі).

\ifcmt
  @width 0.9

  pic https://scontent-lga3-2.xx.fbcdn.net/v/t1.6435-9/124704166_3445100322193451_9087129850487678715_n.jpg?_nc_cat=102&ccb=1-5&_nc_sid=825194&_nc_ohc=eGkIpsgFcvAAX-RASHf&_nc_ht=scontent-lga3-2.xx&oh=0911676399db50388a1b55b7b2611bfe&oe=618287A6
	@rotate angle=180,origin=c
  %@wrap \parpic[r]
\fi

Мова ж народів, що лягли в основу етногенезу москвинів - лишилася хіба в
фіно-угорських анклавах, а подекуди взагалі лише в книжках.

Але й староболгарська мова - не аж такий взірець "слов'янскості", адже вона теж
своєрідний слоєний пиріг - це симбіоз мови слов'ян та племені булгар, яке
прийшло на Балкани з берегів Волги (Болги). Власне, це була частина гуннів,
коли Атілла разом з антами пішов на захід, хан Кубрат лишився в нашому
Причорномор'ї, де й заснував Велику Булгарію. Згодом вона розпалася на Булгарію
волзьку та Булгарію дунайську, але це вже інша історія. Волзькі булгари, до
речі, стали джерелом для багатьох народів - волзьких татар, ногайців, чувашів,
тощо. Але на малюнку, так звана російська мова - виростає напряму зі
слов'янської гілки, без жодних посередників чи проміжних етапів, наче як
головний пагін, тоді як решта мов ютяться поблизу, наче бідні родичі.

Далі, до прикладу, англійська. Ця мова формувалася на базі мови кельтів, потім,
з V-го століття мови германських племен англів, саксів та ютів. Злиття мов цих
племен з місцевими кельтами дало так звану староанглійську. Через 5 століть на
цей пиріг прийшли нормани, і французька норманська мова разом з латиною також
суттєво вплинули на староанглійську мову. Цей процес тривав майже до 1500 року,
допоки не постало те, що зараз називається ранньосучасна англійська мова, тобто
те, що сучасний англієць +/- вже може розуміти.

Мова - як страва, з додаванням кожного нового інгредієнта змінюється смак усієї
страви. Ви варите юшку з м'яса - це одне. Додали овочів - вже інше. Заправили
смаженим буряком, покришили капусту - вийде борщ. А в цьому дереві і близько
цього немає. Тут "борщ" неначе розпадається на окремі елементи. Це повна
маячня, тому розглядати його, як якусь аксіому - просто безглуздо.

Я вже мовчу, що там взагалі пройшли повз такий факт, як існування мов койне -
спільномов для спілкування, що були поширенні в Європі в різні періоди.
Кельтська мова - це була койне-мова протягом багатьох століть, а не просто
якийсь карликовий відросток-апендицит, як на малюнку. Готська - майже 300 років
також була койне для понад половини Європи. Скіфська мова, хоча і була рідною
лише для скіфів, - але також була койне від Балкан та аж до Волги. А на малюнку
цих двох мов взагалі немає.

Насправді, це - небезпечна чергова АНТИНАУЧНА ретрансляція цілої низки міфів,
чималою мірою запущених в світ московитами - вони на це наполегливо працювали і
працюють. Тому нехай ніхто не оманюється "закордонними джерелами". Тепер ми в
цих "закордонних джерелах" легко можемо прочитати про скіфське золото, як про
культурне надбання Росії, про московський борщ і варєнікі, гопак та кАзаков -
як про суто російські явища, про три "братніх" народи, що постали з одного
кореня, та про українців, як про різновид росіян і цілу купу инших. Часто-густо
та ж сама англомовна Вікіпедія по всіх питаннях, які на їх думку тематично
пов'язані з москвинами - просто роблять переклад москвомовних статей.

Усі ми пам'ятаємо героїчного українського хлопчика з нашої діаспори у США, ЯКИЙ
ЧЕРЕЗ СУД домігся, аби англомовна друкована Енциклопедія спростувала свою
статтю щодо літака "Мрія", як "російського літака". Так, Енциклопедія визнала
помилку, в майбутніх накладах пообіцяла її виправити, десь там на сайті навіть
написала вибачення, яке навряд чи хтось згадає. Тоді як багатотисячний тираж
тим часом вже давно розпродався - і назад нічого не повернеш. Тож тисячі
любителів знання і далі поколіннями будуть читати про "руссскій самальот Мрія".
Отакі вони, ці "закордонні джерела".

Наша Українська мова зросла на мові автохтонів, а не десь там виповзла поблизу
москвинської, як це накреслив горе-малювака. Вона, як і будь-яка (нештучна)
мова світу проходила різні етапи, і 1000 років тому, звісно ж, звучала дещо
інакше, так само як і та ж староанглійська, наприклад. Але це все одно було
наша мова. 

Та ворог дуже постарався, аби нам не залишилось майже жодних письмових згадок
нашої мови (згадаймо, як у першу світову вони похапцем вивезли увесь архів
Почаївської лаври разом з друкарнею - його так і не знайдено досі, як
знищувались усі книги та рукописи, написані "малоросійськім нарєчієм", як
"випадково" в повоєнні роки вщент вигоріла секція україністики у столичних
бібліотечних архівах - тоді безповоротньо були втрачені тисячі рукописів та
стародруків, згадаймо, що Пересопницьке Євангеліє свого часу уціліло ВИПАДКОВО,
якби не цей випадок ми б взагалі не МАЛИ ЖОДНОГО ПРИМІРНИКА). Все що лишилося -
це чудом вцілілі скупі уривки, якісь окремі фрази, панегірики, примітки на
полях Пересопниці, але й вони дають нам хоча б уявлення про живу розмовну мову
принаймні 500 років тому... Той, хто хоч трохи цікавився лінгвоцидом нашої мови
- знає, наскільки зухвало та знущально звучать закиди на зразок - "а докажитє!
покажитє нам ваши пісьмєнниє істочнікі"!

Але ще більше дратує хор "українських" підспівувачів, які дружно приєднуються
до подібних знущаннь, роблячи вигляд, що їм нічого невідомо про методичне,
планомірне й наполегливе знищення будь яких слідів нашої мовної культури
протягом щонайменше останніх 3-х столітть...

 @igg{fbicon.face.symbols.mouth} 

\ii{09_11_2020.fb.fb_group.ukrainska_mova_dlja_vsih.1.jazyk_moskovity_mova.cmt}
