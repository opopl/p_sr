% vim: keymap=russian-jcukenwin
%%beginhead 
 
%%file 04_12_2021.fb.kutepova_nadezhda.1.kniga_tajny_zakrytyh_gorodov
%%parent 04_12_2021
 
%%url https://www.facebook.com/nadezhda.kutepova/posts/7205312512815954
 
%%author_id kutepova_nadezhda
%%date 
 
%%tags kniga,kultura,literatura,rossia
%%title Про книгу "Тайны закрытых городов". Опус 2
 
%%endhead 
 
\subsection{Про книгу \enquote{Тайны закрытых городов}. Опус 2}
\label{sec:04_12_2021.fb.kutepova_nadezhda.1.kniga_tajny_zakrytyh_gorodov}
 
\Purl{https://www.facebook.com/nadezhda.kutepova/posts/7205312512815954}
\ifcmt
 author_begin
   author_id kutepova_nadezhda
 author_end
\fi

Про книгу "Тайны закрытых городов". Опус 2.

Предтавляете, уже на некоторых сайтах мою книгу не купить, а на Озоне у нее
стоит этикетка "Бестселлер" Огооо!

\ifcmt
  ig https://scontent-frt3-1.xx.fbcdn.net/v/t39.30808-6/262010473_7205309279482944_1489110446234978469_n.jpg?_nc_cat=107&ccb=1-5&_nc_sid=730e14&_nc_ohc=M8JBwB_FyUIAX8J3bwZ&_nc_ht=scontent-frt3-1.xx&oh=40040f64d342b83c2a6e1153a0e3558b&oe=61B2DDA5
  @width 0.4
  %@wrap \parpic[r]
  @wrap \InsertBoxR{0}
\fi

Но сегодня меня осчастливил Максим. Он прислал мне фото с Книжной выставки Нон
фикшн, которая проходит в Москве до 6 декабря. Мой книга - достойный участник..  

Да и сегодня -  открывок о том, почему я написала эту книгу.

Вы удивитесь наколько все просто. Запугали ребёнка.

"Почему я написала эту книгу?

Потому что в детстве (с 1 года и до 16 лет) я очень боялась неправильно
ответить на вопрос солдата, когда проходила проверку личности на КПП. Сделать
ошибку означало подозрение стать шпионом. И тюрьма. Я хочу, чтобы люди не
боялись тайны, которая уже 30 лет не существует.

Когда я была маленькая, я очень хотела знать вот что.

1. Почему наш город, у которого нет собственного имени, охраняют солдаты и
почему проверяют наши документы каждый раз, когда мы приезжаем и уезжаем из
него. Начать заикаться на вопрос о моем имени или имени моих родителей —
это означало попасть под подозрение, что, может быть, я и не та, за кого
себя выдаю. 

2. Почему мы так хорошо живем в нашем безымянном городе, у нас есть все на
полках магазинов — колбаса, конфеты, даже черная икра в буфете местного
Дома культуры, одежда, мебель, немецкие куклы в универмаге, а в других
городах, которые не закрытые, люди живут плохо, ничего нет, а за молоком
очереди с пяти утра. Мне казалось это несправедливым, потому что люди,
которые жили в тех, других городах, выглядели так же, как и мы, обычные
люди... 

Строго сдвинув брови, мама отвечала мне: «Потому что наш город особый. А почему
он особый — об этом мы вопросов не задаем».

Про самые обычные «необычные» вещи говорить было совершенно нельзя. Зато можно
и даже нужно было знать, что наш город интересует шпионов, потому что он
секретный.

А почему секретный? Вокруг лес, озера? «Об этом мы вопросов не задаем!»—
твердила мама. 

Мне казалось странным, что все вокруг знали о существовании нашего города, но
сразу делали большие глаза и многозначительно поджимали губы, как только в
беседе мы вдруг на него натыкались. "

Продолжение следует. В следующий раз о том, кому я ее адресую..
