% vim: keymap=russian-jcukenwin
%%beginhead 
 
%%file 24_08_2022.stz.news.ua.donbas24.1.ukrzaliznycja_potjag_okupovani_mista
%%parent 24_08_2022
 
%%url https://donbas24.news/news/ukrzaliznicya-prisvyatila-potyag-okupovanim-mistam-cim-vin-osoblivii-foto
 
%%author_id news.ua.donbas24,veremeeva_tetjana.zhurnalist.donbas24.sumy
%%date 
 
%%tags 
%%title "Укрзалізниця" присвятила потяг окупованим містам: чим він особливий (ФОТО)
 
%%endhead 
 
\subsection{\enquote{Укрзалізниця} присвятила потяг окупованим містам: чим він особливий (ФОТО)}
\label{sec:24_08_2022.stz.news.ua.donbas24.1.ukrzaliznycja_potjag_okupovani_mista}
 
\Purl{https://donbas24.news/news/ukrzaliznicya-prisvyatila-potyag-okupovanim-mistam-cim-vin-osoblivii-foto}
\ifcmt
 author_begin
   author_id news.ua.donbas24,veremeeva_tetjana.zhurnalist.donbas24.sumy
 author_end
\fi

\ifcmt
  ig https://gcdnb.pbrd.co/images/QbTCIGJSpaWb.png?o=1
  @wrap center
  @width 0.9
\fi

\begin{center}
  \em\color{blue}\bfseries\Large
	Героїзм українців у боротьбі з військами країни-окупанта художники
відобразили у малюнках на вагонах
\end{center}

В Україні почав курсувати потяг, присвячений окупованим територіям та
захисникам нашої країни. Про це \href{https://t.me/UkrzalInfo/3161}{повідомляє АТ \enquote{Укрзалізниця}}.

Українські митці розписали 7 вагонів одного із потягів, який курсує у нашій
країни. Поїзд курсуватиме за маршрутами Київ — Ужгород та Запоріжжя — Львів. У
вівторок, 23 серпня він відправився із залізничного вокзалу у Києві й сьогодні,
у День Незалежності, прибуває до Ужгорода.

Повідомляється, що митці відобразили у своїх ілюстраціях подвиги українського
народу. 

\ii{24_08_2022.stz.news.ua.donbas24.1.ukrzaliznycja_potjag_okupovani_mista.pic.1}
\ii{24_08_2022.stz.news.ua.donbas24.1.ukrzaliznycja_potjag_okupovani_mista.pic.2}

\begin{leftbar}
	\begingroup
		\bfseries
\qbem{У своїх ілюстраціях Андрій Присяжнюк закарбував мужність Богдана Зізи, який
облив синьо-жовтою фарбою міську адміністрацію в Євпаторії; Сергій Туркіневич —
мешканців Енергодара, які беззахисні вийшли проти танків; Дмитро Касянюк —
сталевих оборонців Маріуполя та сміливих залізничників, які евакуювали людей із
Харківщини; Роман Синенко розкрив історію лікарів з Луганщини, які рятували
життя, залишаючись в окопах і підвалах; Аліна Коник — фермерів Миколаївщини,
які збирали врожай на замінованих та палаючих полях, Максим Кільдеров — рух
опору Херсонщини}, — зазначають в \enquote{Укрзалізниці}.
	\endgroup
\end{leftbar}

%3_5
\ii{24_08_2022.stz.news.ua.donbas24.1.ukrzaliznycja_potjag_okupovani_mista.pic.3}

У кожному вагоні є QR-код, відсканувавши який можна перейти на сайт проєкту й
дізнатися докладніше про те, що зображено на вагоні.

\ii{24_08_2022.stz.news.ua.donbas24.1.ukrzaliznycja_potjag_okupovani_mista.pic.6_7}

Раніше Донбас24 писав, що \href{https://donbas24.news/news/z-pokrovska-skasuvali-evakuaciinii-potyag-ci-mozna-viyixati-z-doneccini}{із Покровська скасували евакуаційний потяг}.

Ще більше новин та найактуальніша інформація про Донецьку та Луганську області
в нашому \href{https://t.me/donbas24}{телеграм-каналі Донбас24}.

Фото: \enquote{Укрзалізниця}

\ii{insert.author.veremeeva_tetjana}
