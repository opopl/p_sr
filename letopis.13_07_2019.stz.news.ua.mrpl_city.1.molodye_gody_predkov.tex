% vim: keymap=russian-jcukenwin
%%beginhead 
 
%%file 13_07_2019.stz.news.ua.mrpl_city.1.molodye_gody_predkov
%%parent 13_07_2019
 
%%url https://mrpl.city/blogs/view/molodye-gody-predkov
 
%%author_id burov_sergij.mariupol,news.ua.mrpl_city
%%date 
 
%%tags 
%%title Молодые годы предков
 
%%endhead 
 
\subsection{Молодые годы предков}
\label{sec:13_07_2019.stz.news.ua.mrpl_city.1.molodye_gody_predkov}
 
\Purl{https://mrpl.city/blogs/view/molodye-gody-predkov}
\ifcmt
 author_begin
   author_id burov_sergij.mariupol,news.ua.mrpl_city
 author_end
\fi

\ii{13_07_2019.stz.news.ua.mrpl_city.1.molodye_gody_predkov.pic.1}

Бабушка с дедушкой приехали в Мариуполь на постоянное житье-бытье, если не
изменяет память, в 1906 году. Да, пожалуй, этот год называла бабушка. Добрались
в город у моря  вскоре после венчания в родном селе Варва Прилукского уезда
Полтавской губернии. Застряло в памяти, что бабушка Татьяна Кирилловна бывала в
наших краях еще до замужества. Она приезжала сюда на заработки в Волонтеровку,
тогда еще пригородное село. Там ей пришлось работать в экономии. В юности и
позже казалось, что это собственное имя некоего сельскохозяйственного
предприятия. Сейчас же Википедия дала такой ответ: \enquote{Экономия — до 1917 года
название крупного помещичьего хозяйства на юге России и Украины, в котором
использовались наемный труд и сельскохозяйственные машины}.

\ii{13_07_2019.stz.news.ua.mrpl_city.1.molodye_gody_predkov.pic.2}

Приезжала не одна, а с односельчанками, имевшими опыт работы на Мариупольщине.
Что особенно запомнилось из рассказов бабушки? Сезонные рабочие, а именно
такими были она и ее товарки - выращивали овощи, в том числе и баклажаны,
по-мариупольски – \enquote{синенькие}, \enquote{красненские} - помидоры, болгарский перец
разных сортов. Запомнился рассказ о том, как произошло ее знакомство с
экономией. \emph{\enquote{Дівчата казали, що коли посадять за стіл, хазяїн буде дивиться хто,
як їсть. Хто погано буде їсти, таку він на роботу не візьме. Так я, бувало,
давлюся, але їм}}. И еще один штрих. Бабушку и ее товарок поставили на сбор
баклажан. Ну, как же не попробовать невиданный ранее овощ? Резюме девчат после
\enquote{дегустации} было такое: \emph{\enquote{Хай свині це їдять}}. Странно было слушать этот
рассказ. Ведь в нашем доме и бабушка, и мама, и тетя готовили из баклажан и
икру, и соус с картошкой, и зажаренные полоски  \enquote{синеньких}, обваленных  в
муке. Но \enquote{примой} в этом кулинарном оркестре были квашеные баклажаны,
начиненные морковкой и чесночком.

У дедушки Петра Савельевича, - в украинской интерпретации это звучало, как
Петро Саввич, - опыта жизни в приморском городе было куда больше. Когда ему
стукнуло около десяти лет, может и чуть больше, отец Савва Васильевич,
крестьянин из уже упоминавшегося села Варва, привел к каменщику для обучения
ремеслу. У Петра от рождения была так называемая впалая грудь. И по разумению
его отца, для крестьянского труда он был мало приспособлен. Вот и привез свое
чадо в Мариуполь. Но оказалось, что ученье состояло из грязной домашней работы.
Колка дров, мытье полов, чистка самовара и подсвечников, беготня за водкой,
уборка сарайчика, где откармливался боров и тому подобные \enquote{приемы кладки
кирпича}. За каждую промашку Петя получал подзатыльник или пинок. Когда стало
совсем невмоготу, поплелся он на Итальянскую улицу к полицмейстеру.
Подробности как-то пролетели мимо ушей. Но итог запечатлелся. Полицмейстер
вызвал хозяина, велел надрать уши пострелу, чтобы отучить его бегать в полицию
с доносами. Сразу ли убежал Петро от каменщика или спустя какое-то время? Не у
кого теперь не спросить...

В сохранившейся членской книжке больничной кассы написано, что на иждивении
токаря броневого цеха завода Никополь – Мариупольского Горного и
Металлургического Общества Петра Литвиненко находятся: жена - Татьяна
Кирилловна 32 лет, дети - Александра 6 лет, Лидия 3 лет, Валентина одного
месяца от роду, а также мать - Анастасия 60 лет. Это единственный сохранившийся
документ. Остальное  - память. Из детских впечатлений. У дедушки на лице были
кое-где оспины. Услышанное когда-то от бабушки: \emph{\enquote{Я за рябого пішла заміж тому,
що була перестарком}}. Намного позже, прикинув возраст умершего в младенчестве
первенца бабушки, получалось, что под венец пошла она двадцатилетней.

Первое время молодожены снимали комнатушку где-то на Малофонтанной. Столовались
в \enquote{Обжорке} - харчевне на базаре – главном торжище Мариуполя. Здание, в котором
располагалось это достославное учреждение дореволюционного общепита,
сохранилось до наших дней. И будто бы в нем располагается автошкола. Меню в
\enquote{Обжорке} было питательно, дешево и ... неизменно. За пятачок - \enquote{полумысок}
борща, сваренного на крепком бульоне из говяжьих костей, полфунта жареной
баранины или ливера, бесплатно ржаной хлеб \enquote{досхочу}.

Давно это было, очень давно.

\textbf{Читайте также:} 

\href{https://mrpl.city/news/view/mariupoltsy-tselymi-semyami-aktivno-vklyuchayutsya-v-prazdniki-dvora-foto}{%
Мариупольцы целыми семьями активно включаются в праздники двора, Ігор Романов, mrpl.city, 12.07.2019}

\clearpage
