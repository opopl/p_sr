% vim: keymap=russian-jcukenwin
%%beginhead 
 
%%file 14_02_2022.fb.voloshin_oleg.opzzh.1.vtorzhenie
%%parent 14_02_2022
 
%%url https://www.facebook.com/oleg.voloshin.7165/posts/5211934328839586
 
%%author_id voloshin_oleg.opzzh
%%date 
 
%%tags rossia,ugroza,ukraina,vtorzhenie
%%title Ни один здравомыслящий человек не верит в спонтанное российское вторжение
 
%%endhead 
 
\subsection{Ни один здравомыслящий человек не верит в спонтанное российское вторжение}
\label{sec:14_02_2022.fb.voloshin_oleg.opzzh.1.vtorzhenie}
 
\Purl{https://www.facebook.com/oleg.voloshin.7165/posts/5211934328839586}
\ifcmt
 author_begin
   author_id voloshin_oleg.opzzh
 author_end
\fi

Ни один здравомыслящий человек не верит в спонтанное российское вторжение. У
Москвы мало поводов дальше терпеть проводимый режимом Зеленского курс, но у неё
хватает иных, невоенных инструментов. Между тем Газпром в последнее время
только нарастил транзит через отечественную ГТС.

А вот во что я верю и чему имею подтверждения из самых разных источников, так
это во всплеск агрессии, который на фоне раздуваемой Западом военной истерии
планируют радикальные националистические группы в столице и других городах. Они
уже несколько месяцев готовятся, составляют списки, выделяют транспорт,
определяют адреса для быстрых и ни каким законом не сковываемых действий. Какой
бы риторикой при этом не прикрывались эти действия в их основе только одно -
жажда наживы. Как сказал мне один знакомый, имеющий контакты в этой среде: «Те,
кто не добежал до Межигорья в 2014, хотят взять реванш сейчас». Недавние кадры
из охваченной погромами Алма-Аты должны дать представление, о чем идёт речь. 

Естественно, люди, открыто и последовательно изобличавшие рост неонацистских
группировок, критиковавшие националистический крен в разных сферах политики,
выступавшие за компромисс с Россией - цель номер один. И здесь у них выбор
широкий с представителями правящей команды включительно. А стремление
«экспроприировать экспроприаторов» избавит от излишней разборчивости. 

Концепцию «гуманитарных интервенций» придумали на Западе. Но применить ее может
любая великая держава. И грех тогда будет внешним силам не использовать такой
повод. Потому наряду с готовностью серьезно и динамично двигаться по пути
реализации Минских соглашений для избежания катастрофы для всей нашей страны
действующая власть обязана предотвратить малейшие проявления внутреннего
террора и мародёрства. 

Кстати, во многом из-за этого, а не от страха перед российскими войсками
(трудно себе представить, чтоб в любом случае Москва не гарантировала их
безопасность) западные страны и эвакуируют посольства.
