% vim: keymap=russian-jcukenwin
%%beginhead 
 
%%file 23_05_2018.fb.lesev_igor.1.taras_chernovol_klassik_predatelstva
%%parent 23_05_2018
 
%%url https://www.facebook.com/permalink.php?story_fbid=1927393593958401&id=100000633379839
 
%%author_id lesev_igor
%%date 
 
%%tags chelovek,chernovol_taras.ukr.politik,politika,predatelstvo,ukraina
%%title Тарас Черновол - Классик постоянного предательства
 
%%endhead 
 
\subsection{Тарас Черновол - Классик постоянного предательства}
\label{sec:23_05_2018.fb.lesev_igor.1.taras_chernovol_klassik_predatelstva}
 
\Purl{https://www.facebook.com/permalink.php?story_fbid=1927393593958401&id=100000633379839}
\ifcmt
 author_begin
   author_id lesev_igor
 author_end
\fi

У каждого из нас наверняка есть свой образец абсолютно омерзительного существа.
Выносим за скобки личные обиды, и выносим за те же скобки убийц, садистов и
насильников, перешагнувших рамки законности. И в таком случае непременно
выходим на какую-то публичную особу, в 99\% политика.

\ifcmt
  ig https://scontent-frt3-2.xx.fbcdn.net/v/t1.6435-9/33221580_1927393493958411_7957542765523369984_n.jpg?_nc_cat=103&ccb=1-5&_nc_sid=730e14&_nc_ohc=OKB-cCeX7ngAX-1vgjO&_nc_ht=scontent-frt3-2.xx&oh=cedf5577842e408d9deea1eb3af9073f&oe=61B79F79
  @width 0.4
  %@wrap \parpic[r]
  @wrap \InsertBoxR{0}
\fi

Для меня таким антагонистом является Тарас Черновол. Уникальнейшее по своим
омерзительным высказыванием и поступком существо. Классик постоянного
предательства. Фантастический раздражитель в озвучивании нелепостей и
одновременно моральных низостей. Предал своего покойного отца, а потом и
фракцию «Наша Украина», пойдя на высокооплачиваемую работу к Виктору Януковичу.
Затем предал Януковича, начав кулуарные переговоры с Тимошенко. Сейчас «гасит»
Тимошенко, будучи на содержании у Порошенко. Непременно предаст Порошенко.
Непременно и уже скоро.

Кстати, когда Порошенко будет сидеть в Малаге и раздумывать, ну почему так все
получилось, в череде его воспоминаний должны всплыть нанятые порохоботы. В
рекламе и личном пиаре ничего плохого нет. Но порохоботят ведь не Борис Патон,
не Лина Костенко. Даже Зеленский и Вакарчук могли бы порохоботить, но и те идут
«в президенты». Почему так происходит? В том числе и потому, что Порошенко
подбирает запредельную аморальную падаль, разложившееся мясо, которое далеко не
в каждом морге примут на хранение. А Порошенко принимает. Ну и получает свой
рейтинг.

Но вернемся к Тарасу. На днях наткнулся на ТВ случайно на его физиономию.
Рассуждал о повышении тарифа на газ для населения. Сказал, что это чудесно и
по-европейски. Что бояться нечего, всех спасут субсидии. На вопрос ведущего,
что субсидии урезаются, ответил в стиле – «это ерунда». Вспомнил злочинную
власть Януковича (дважды от партии которого был народным депутатом), и сказал,
что вот тогда людям было по-настоящему тяжело. Тут же захотел поговорить о
Тимошенко, которая живет не по доходам. В общем, типичный набор слов,
формирующих идеально отрицательное отношение к персонажу. Такого можно
использовать разве что для разрушения имиджа оппонентов. Например, если бы это
существо начало рассказывать, что Тимошенко – это не такой плохой вариант в
качестве будущего президента Украины, то Юлии Владимировне в таком случае
следовало бы напрячься. 

Ну а завершить хочу писульку совсем другим. Мне непонятен пиетет на
государственном уровне перед отцом Тараса Вячеславом Черноволом. Последний в
Украине возведен в культ национального героя. Такой себе аналог Николая
Баумана, которому «посчастливилось» быть убитым возле технического
университета, и теперь его именем назван один из лучших вузов России. Черновол
был украинским диссидентом. Имел несколько судимостей. Видел Украину непременно
независимой. Был ярым антисоветчиком. Это все да. Но если бы Черновола вдруг не
было, пошла бы история Украины как-то иначе? Может быть Советский Союз
сохранился? Не было бы ГКЧП и вышедших на улицу москвичей, которые и
предопределили независимость Украины? Наконец, а что такого важного сделал
Черновол в годы независимости? Он занимал какой-то государственный пост? Провел
реформирование хоть чего-то? Создал какую-то институцию, повлиявшую в
последствии на украинское общество? Лично мне он запомнился во главе постоянных
политических срачей в его же партии, которые приводили регулярно к расколам.

Это не попытка умалить заслуги Черновола-старшего, хотя бы потому, что я не
понимаю, что можно умалять. Государственный деятель в моем понимании – это
деятель пришедший к власти и получивший рычаги влияния, определяющие
жизнедеятельность руководимого им сообщества. В особых случаях это может быть и
оппозиционный деятель, но опять же, влияющий на действия властей. На что
когда-то влиял Вячеслав Черновол – мне непонятно. Возможно, из-за
ограниченности знаний. Не претендую на категоричность. Но вот что у Черновола
получилось хуже всего – это воспитать из своего сына достойного человека. Его
Тарас – это эталон проходимца, приспособленца и негодяя.

\ii{23_05_2018.fb.lesev_igor.1.taras_chernovol_klassik_predatelstva.cmt}
