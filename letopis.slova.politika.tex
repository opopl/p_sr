% vim: keymap=russian-jcukenwin
%%beginhead 
 
%%file slova.politika
%%parent slova
 
%%url 
 
%%author 
%%author_id 
%%author_url 
 
%%tags 
%%title 
 
%%endhead 
\chapter{Политика}
\label{sec:slova.politika}

%%%cit
%%%cit_head
%%%cit_pic
%%%cit_text
Но, в системе, где даже Главный закон страны в лучшем случае \enquote{как дышло}, в
худшем - просто слова, нет смысла просто его переписывать - не растут груши на
Марсе. Нужно прежде всего кардинально менять практику его правоприменениния и
наказания чиновников за игры с Конституцией, при этом создавая условия для
будущего реформирования Конституции с цель перехода на чистую модель управления
- парламентскую (предпочтительнее) или президентскую - только в таких условиях
возможно обеспечить хоть какую-то ответственность \emph{политиков}
%%%cit_comment
%%%cit_title
\citTitle{Из 161 статьи Конституции Украины системно нарушаются 125 / Лента соцсетей / Страна}, 
Руслан Бортник, strana.ua, 28.06.2021
%%%endcit

%%%cit
%%%cit_head
%%%cit_pic
\ifcmt
  pic https://avatars.mds.yandex.net/get-zen_doc/5226854/pub_60e1daa72dcce634f5902d51_60e200dd2dcce634f5f8eeda/scale_1200
  caption это юмор из Украины. На русском, кстати, языке!!!
\fi
%%%cit_text
Это даже не глупость, это \emph{ПОЛИТИКА}, \emph{политические игры} и хайп.
Сами же украинские политики и активисты в быту по-русски разговаривают, и еще
не известно на каком языке Ницой в поисковике запросы забивает.
Теперь немного о русско-украинских отношениях через призму ЧЕ по футболу:
%%%cit_comment
%%%cit_title
\citTitle{Мова, политика и сборная Украины по футболу}, Мак Сим,
zen.yandex.ru, 04.07.2021
%%%endcit

