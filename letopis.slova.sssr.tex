% vim: keymap=russian-jcukenwin
%%beginhead 
 
%%file slova.sssr
%%parent slova
 
%%url 
 
%%author 
%%author_id 
%%author_url 
 
%%tags 
%%title 
 
%%endhead 
\chapter{СССР}
\label{sec:slova.sssr}

%%%cit
%%%cit_pic
%%%cit_text
\enquote{Червону руту} играли и пели в концертных залах, ресторанах, на танцплощадках
по всему \emph{СССР} и за его пределами. Белорусский композитор Эдуард Ханок в одном
интервью признался, что под влиянием \enquote{Червоной руты} он написал другую
популярную песню на украинском языке - \enquote{Верба} на стихи Юрия Рыбчинского
%%%cit_comment
%%%cit_title
\citTitle{Какая легенда вдохновила Владимира Ивасюка написать \enquote{Червону руту} 
(украинскому шлягеру - 50)}, 
Татьяна Кроп, zen.yandex.ru, 07.12.2020
%%%endcit

%%%cit
%%%cit_pic
%%%cit_text
Вскоре на телеэкраны вышел первый в \emph{советской Украине} музыкальный фильм
\enquote{Червона рута}, в котором в главных ролях вместе с названными Зинкевичем и
Яремчуком снялась еще одна восходящая звезда - молодая преподавательница
культпросветучилища, очаровательная София Ротару.  На долгие годы украсила ее
репертуар \enquote{Червона рута}, которую София Михайловна любит исполнять в своих
сольных концертах. Во многом благодаря именно этой песне сформировался
сценический образ Софии Ротару - романтической, загадочной, умеющей любить и
влюблять в себя красавицы
%%%cit_comment
%%%cit_title
\citTitle{Какая легенда вдохновила Владимира Ивасюка написать \enquote{Червону руту} 
(украинскому шлягеру - 50)}, 
Татьяна Кроп, zen.yandex.ru, 07.12.2020
%%%endcit

bunyod yunusov
Мой адресс \emph{Советский Союз}, приезжайте в гости братья, сестры, друзья. Я Вас всех ждууу

Кирик Лонгинов
Мне 45 и я горжусь что из \emph{СССР}. Вроде мужик но хочется аж плакать,
старею чтоли, в детство тянет 

Владимир Гусев
Дорогой друг, не стыдись слёз о прекрасном... Ты не одинок в своём душевном
смятении. Мне седьмой десяток и печалюсь также. Эта композиция должна быть
вечной, но к сожалению \enquote{ни что не вечно}. Поверь, что моральные уроды,
воспитанные в \emph{СССР} и, сделавшие многое для утилизации \emph{Страны} и её искусства,
под эту песню тоже пускают сопли.... 

Lord БЭКЕТ'ов
2021 год! Мне 46, и я плачу! Поддержите! Хочу вернуться назад! 05.02.2021г.

Marat Mamadaliev
Мне 43 года и я рад что у меня было \emph{советское дедство}. 

Erbolat Saduov
Пацаны! А мне 22 мая исполнилось 64 года... я тоже рад, что \emph{советский} и тоже... 
реветь охота. 

TheTarantulll
Дедство у вас еще не началось)) 38, тоже из \emph{Советского детства}) 

Alex Porter
\emph{Советское детство}, самое светлое!!!! 

Валерия Ланская \enquote{Прекрасное далёко}
Хор мальчиков Московского хорового училища имени А.В.Свешникова
Музыка: Е. Крылатов, Слова: Ю. Энтин
к\ф \enquote{Гостья из будущего},  режиссер: П. Арсенов
Концерт из цикла \enquote{Живая музыка экрана} - \enquote{Мастера. Поэты Леонид Дербенёв и Юрий
Энтин}.  Концертный зал им. П.И. Чайковского
Январь, 2019 г.

%%%cit
%%%cit_pic
%%%cit_text
Формально названный \enquote{Обращением фюрера к германскому народу} текст в
05.30 по берлинскому времени (в 07.30 по Москве) зачитал третий человек в
Третьем Рейхе Йозеф Геббельс. Заявление \emph{советского} правительства - в
12.15 по московскому времени - второй человек в \emph{СССР} Вячеслав Молотов.
Именно в этих двух речах прозвучала основа версий причин германского нападения
на \emph{Советский Союз}, ставшего началом Великой Отечественной войны.
Нацистская версия состояла в том, что Москва якобы готовилась к наступлению и
сосредоточила на германской границе \enquote{все свои вооруженные силы, готовые
к броску}, в связи с чем Гитлер отдал приказ о наступлении, которое должно
спасти Европу от \enquote{еврейско-большевистских орд}
%%%cit_comment
%%%cit_title
\citTitle{Как начиналась война Германии против СССР 22 июня 1941 года}, 
Дмитрий Коротков, strana.ua, 21.06.2021
%%%endcit

%%%cit
%%%cit_head
%%%cit_pic
%%%cit_text
Рівно 80 років тому, 22 червня 1941 року з повідомлень радіостанцій громадяни
\emph{СРСР} дізналися про напад гітлерівської Німеччини та початок війни, яку пізніше
назвуть \enquote{Великою Вітчизняною}.  Для Сталіна і Червоної армії битва з колишнім
союзником Гітлером стартує провально. Мільйони вбитих і взятих у полон
\emph{радянських} солдатів не обіцяли \emph{Радянському Союзу} нічого доброго. Але перелом
все ж відбувся – Гітлер програв.  Як зазначає публіцист, письменник і історик
Марк Солонін, до перемоги \emph{СРСР} привело декілька основних причин і ще багато
різних за масштабом подій, які визначили перебіг війни
%%%cit_comment
%%%cit_title
\citTitle{Мовчання Сталіна, помилки Гітлера, окупація Києва і провалений бліцкриг – Марк Солонін про війну СРСР і Німеччини}, 
Євген Руденко, www.pravda.com.ua, 22.06.2021
%%%endcit

