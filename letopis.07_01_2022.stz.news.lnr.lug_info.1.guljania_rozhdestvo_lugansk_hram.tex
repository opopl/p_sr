% vim: keymap=russian-jcukenwin
%%beginhead 
 
%%file 07_01_2022.stz.news.lnr.lug_info.1.guljania_rozhdestvo_lugansk_hram
%%parent 07_01_2022
 
%%url https://lug-info.com/news/narodnye-gulania-v-cest-rozdestva-hristova-prosli-na-territorii-luganskogo-hrama
 
%%author_id 
%%date 
 
%%tags rozhdestvo,prazdnik,lugansk,lnr,donbass,cerkov,hram,prazdnovanie
%%title Народные гуляния в честь Рождества Христова прошли на территории луганского храма
 
%%endhead 
\subsection{Народные гуляния в честь Рождества Христова прошли на территории луганского храма}
\label{sec:07_01_2022.stz.news.lnr.lug_info.1.guljania_rozhdestvo_lugansk_hram}

\Purl{https://lug-info.com/news/narodnye-gulania-v-cest-rozdestva-hristova-prosli-na-territorii-luganskogo-hrama}

Духовно-культурное мероприятие, посвященное Рождеству Христову, прошло на
территории Луганского храма в честь иконы Божией Матери \enquote{Умиление}. Об этом с
места события передает корреспондент ЛИЦ.

\ii{07_01_2022.stz.news.lnr.lug_info.1.guljania_rozhdestvo_lugansk_hram.pic.1}

Горожан с праздником поздравили митрополит Луганский и Алчевский Пантелеимон,
глава администрации Луганска Манолис Пилавов, духовенство, творческие
коллективы столицы.

\ii{07_01_2022.stz.news.lnr.lug_info.1.guljania_rozhdestvo_lugansk_hram.pic.2}

\enquote{Дорогие братья и сестры, радость Рождества Христова коснулась наших сердец.
Начиная с полуночи колокола храмов призывают людей на молитву. Сердечно
поздравляю вас, желаю пребывать в мире, здравии, благополучии и вторить добрые
дела во славу Божию, для спасения наших душ – это будет самым лучшим даром
Богомладенцу Христу}, - приветствовал пришедших к храму митрополит.

\ii{07_01_2022.stz.news.lnr.lug_info.1.guljania_rozhdestvo_lugansk_hram.pic.3}

Мэр столицы пожелал \enquote{всем доброго здоровья, мира и добра}.

\enquote{Мы с вами заслужили того, чтобы сегодня Бог Иисус Христос оберегал и защищал
нас}, - подчеркнул Пилавов.

\ii{07_01_2022.stz.news.lnr.lug_info.1.guljania_rozhdestvo_lugansk_hram.pic.4}

На сцене под открытым небом выступили коллективы Луганского дома культуры имени
Ленина, а в нижнем пределе храма – сводный детский хор музыкальных школ города,
который исполнил рождественские песнопения. Детей за их усердие архиерей
угостил новогодними сладкими подарками.

\ii{07_01_2022.stz.news.lnr.lug_info.1.guljania_rozhdestvo_lugansk_hram.pic.5}

\enquote{Желаю, чтобы радость этого праздника была полной, гармоничной, чтобы она
соединялась с добрыми делами, и именно ими мы прославляли родившегося Христа},
- обратился к детям владыка.

\ii{07_01_2022.stz.news.lnr.lug_info.1.guljania_rozhdestvo_lugansk_hram.pic.6}

Традиционную рождественскую выставку своих изделий организовали мастера
декоративно-прикладного искусства народного клуба \enquote{Левша} Луганского центра
народного творчества.

\ii{07_01_2022.stz.news.lnr.lug_info.1.guljania_rozhdestvo_lugansk_hram.pic.7}

\enquote{Для людей православных – это один из главных праздников в нашей культуре. Мы
пришли сегодня с премьерой песни и с радостью и пожеланиями добра, мира,
счастья, благоденствия и всего самого хорошего}, - отметил руководитель
Народного ансамбля казачьей и авторской песни \enquote{Любо} Михаил Ивановский.

\ii{07_01_2022.stz.news.lnr.lug_info.1.guljania_rozhdestvo_lugansk_hram.pic.8}
\ii{07_01_2022.stz.news.lnr.lug_info.1.guljania_rozhdestvo_lugansk_hram.pic.9}
\ii{07_01_2022.stz.news.lnr.lug_info.1.guljania_rozhdestvo_lugansk_hram.pic.10}
\ii{07_01_2022.stz.news.lnr.lug_info.1.guljania_rozhdestvo_lugansk_hram.pic.11}

