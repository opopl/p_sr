% vim: keymap=russian-jcukenwin
%%beginhead 
 
%%file slova.literatura
%%parent slova
 
%%url 
 
%%author 
%%author_id 
%%author_url 
 
%%tags 
%%title 
 
%%endhead 
\chapter{Литература}
\label{sec:slova.literatura}

%%%cit
%%%cit_head
%%%cit_pic
%%%cit_text
Еще раз подчеркнем, феерией называлась не \emph{литературная} основа, а само шоу,
действо. Пьесы, которые ставились таким образом, драматурги не обозначали как
феерию. Потому что играть их можно было по-разному.  Пока не пришел Метерлинк.
В 1905 году бельгийский символист Морис Метерлинк пишет свою самую знаменитую
пьесу. Она называется \enquote{Синяя птица}. И впервые он сразу ставит на обложке ее
жанр - пьеса-феерия.  Так термин перекочевал из театральной среды в
\emph{литературную}. Теперь читатель, даже не видя самой постановки, должен был
представлять то действие, которое разворачивается в пьесе, именно как красочную
и волшебную феерию
%%%cit_comment
%%%cit_title
\citTitle{В двадцатом веке было написано только три феерии. Одна бельгийская, две наши}, 
ЛИТИНТЕРЕС, zen.yandex.ru, 03.06.2021
%%%endcit

%%%cit
%%%cit_head
%%%cit_pic
%%%cit_text
Фраза про популярність в Росії «неймовірної української \emph{літератури}» належить до
порожньої, позбавленої сенсу риторики, бо українськими творами цікавились у
РСФСР хіба що поодинокі українці, що там мешкали.
Варте уваги й означення «неймовірна», вжите щодо \emph{української літератури}.
Очевидно, президент хотів вжити це слово у позитивному значенні «велика; така,
що виходить за межі звичайного». Визначення доволі абсурдне, оскільки у
застосунку до \emph{літератури} воно не вживається. Насправді ж тут можна вбачати
обмовку за Фрейдом: на підсвідомому рівні прикметник «неймовірна» він радше
вжив у першому його значенні – «така, яку важко чи неможливо уявити, якої
насправді не існує; неможлива, нереальна» і саме це значення визначає його
реальне ставлення до \emph{української літератури}
%%%cit_comment
%%%cit_title
\citTitle{Українська мова в культурному просторі держави. Протистояння триває}, 
Лариса Масенко, www.radiosvoboda.org, 11.07.2021
%%%endcit

%%%cit
%%%cit_head
%%%cit_pic
%%%cit_text
Гоголь не прийшов у російську \emph{літературу} бідним родичем. Він приніс у
неї слово свого народу, влив у російську мову свіжий струмінь молодої, розкутої
сили, СЛІДОМ за Пушкіним оновив, збагатив \emph{літературну} мову новими
життєвими зворотами й інтонаціями, надав їй небаченої образності, розширив її
межі. Сила різьблено-точної мовної характеристики гоголівських персонажів була
така вражаюча, що сучасники вигукували: "Автор-стенограф!" Жоден з письменників
не створив такої кількості типів, які ввійшли в побут \emph{літературний},
політичний, щоденний, як імена прозивні, як символи тих чи інших людських
пороків.  Та не слід вважати, ніби \emph{літературний} шлях для Гоголя стелився
широко і вільно, як Невський проспект
%%%cit_comment
%%%cit_title
\citTitle{Три долі. Гоголь, Шевченко, Чехов}, Павло Загребельний
%%%endcit
