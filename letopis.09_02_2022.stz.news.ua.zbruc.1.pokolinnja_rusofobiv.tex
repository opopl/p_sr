% vim: keymap=russian-jcukenwin
%%beginhead 
 
%%file 09_02_2022.stz.news.ua.zbruc.1.pokolinnja_rusofobiv
%%parent 09_02_2022
 
%%url https://zbruc.eu/node/110422
 
%%author_id ljubka_andrii
%%date 
 
%%tags napadenie,rossia,russofobia,ugroza,ukraina
%%title Покоління русофобів
 
%%endhead 
 
\subsection{Покоління русофобів}
\label{sec:09_02_2022.stz.news.ua.zbruc.1.pokolinnja_rusofobiv}
 
\Purl{https://zbruc.eu/node/110422}
\ifcmt
 author_begin
   author_id ljubka_andrii
 author_end
\fi

Я не знаю, чи буде війна. Думаю, сьогодні цього не знає ніхто, навіть її
основний винуватець, Той-що-в-Кремлі-сидить.

Щодня змінюється забагато факторів, які почергово то підтверджують неминучу
ескалацію, то дозволяють зітхнути з полегшенням. Мабуть, саме на такий ефект ця
зловіща тактика й розрахована: від цих емоційних гойдалок ми маємо настільки
втомитися, щоб аж вигоріти й збайдужіти. Тоді й буде ідеальний момент для
повномасштабного вторгнення.

\ii{09_02_2022.stz.news.ua.zbruc.1.pokolinnja_rusofobiv.pic.1}

Не знаю, чим така війна може закінчитися. Можливо, нам вдасться дати загарбнику
в писок і змусити відступити. Можливо, він точить зуби на Харків, Маріуполь чи
коридор до Криму.

Чи можлива повна окупація України, втрата державності? Нині така думка здається
повною нісенітницею, абсолютною фантастикою, але навіть я ще народився в
умовах, коли Україна була повністю окупована росіянами. За останні 300 років ми
рівно вдесятеро більше часу були без держави, ніж із нею, тож суто математично
варіант повної втрати незалежності цілком реальний.

Словом, я багато чого не знаю, ще більше не розумію і щодо багатьох речей
вагаюся, зате одне знаю напевно: з кожним проклятим днем, з кожною страшною
хвилиною цієї вже восьмирічної війни Росія все більше віддаляється від
перемоги. Це не означає, що переможемо ми, що над Кремлем замайорить
український прапор. Але точно означає одне: Путін може нападати чи не нападати,
але війну за Україну він уже програв.

Бо які б карколомні події не трапилися найближчим часом, найголовніші зміни
відбуваються не на контурних картах у військових штабах, а в головах людей. І 8
років війни залишають по собі такий глибокий слід, який навряд чи вдасться
вимазати за наступні 80 років.

Бо вже 8 років ми відбиваємося від російського нападу. Агресії, яку так влучно
й відповідно до кремлівської риторики можна назвати віроломною. За ці роки
народилися діти, які вже встигли вирости й піти в школу, а по телевізору весь
цей час одна тема – війна. Ці діти виростають із уже вбудованим знанням про те,
що Росія – наш ворог. Для них – навічно.

Та й ми, дорослі, уже 8 років поспіль ходимо в школу русофобії. Бо що б там хто
не говорив, але ненавидимо ми не тільки Путіна. За ці роки в нас виробився
рефлекс сахатися всього російського, втікати від нього, мов від прокази. Мова
тут не тільки про некупування російських продуктів, це взагалі найлегше, бо
Росія майже нічого нормального й не виробляє. Мова про російське як таке.

Безумовно, це неправильно, це суперечить засадам гуманізму. Не можна всіх
росіян гребти під одну «грєбьонку», неправильно погано ставитися до російської
мови й культури, це не по-європейськи. Так, неправильно, але нічого не можу з
собою вдіяти. Мене, моє покоління, всіх нас ця війна й особисто Путін загнали в
цей стан. Це вже навіть не про ненависть – більше про огиду.

Кожної блядської миті, коли я не міг заснути, думаючи про те, що буде з моєю
родиною в разі початку повномасштабного наступу, кожної сраної секунди, коли
уявляв, що Росія може бомбардувати наші міста, я складав собі самому присягу –
за будь-якої можливості уникати всього російського, витрачати час і зусилля,
доплачувати за те, щоб моя донька мала якомога менше контакту в житті з
будь-чим російським: бомбами, мовою, культурою, людьми.

Уявіть собі, якщо така зміна трапилася зі мною, внуком учительки російської
мови і літератури, людиною, яка в дитинстві і юності більшість книжок прочитала
російською мовою, то що буде з нашими дітьми, які ростуть зі щоденними новинами
про те, що Росія збирає проти нас війська?
 
