% vim: keymap=russian-jcukenwin
%%beginhead 
 
%%file 19_04_2021.fb.backandalive.1.stanislavenko_mihail_smert_vojna
%%parent 19_04_2021
 
%%url https://www.facebook.com/backandalive/posts/1986434088180685
 
%%author 
%%author_id 
%%author_url 
 
%%tags 
%%title 
 
%%endhead 

\subsection{Вчора, 19 квітня, Михайло Станіславенко міг би відсвяткувати 38 день народження}
\Purl{https://www.facebook.com/backandalive/posts/1986434088180685}

Вчора, 19 квітня, Михайло Станіславенко міг би відсвяткувати 38 день
народження. А сьогодні, 20 квітня, вже 7 роковини з дня його загибелі під час
першого значного бою з проросійськими бойовиками на блокпосту під Билбасівкою в
околицях Слов'янська.

\ifcmt
  pic https://scontent-dfw5-1.xx.fbcdn.net/v/t1.6435-9/176033618_1986425611514866_5228676878749650260_n.jpg?_nc_cat=1&ccb=1-3&_nc_sid=730e14&_nc_ohc=0VMhevSQOg8AX_VnVfU&_nc_ht=scontent-dfw5-1.xx&oh=92234e2257936115bb90d61a440fc0f4&oe=60A3E345
\fi

З моменту його загибелі мама чекала 2 роки, аби поховати Михайла.

З-під Слов'янська бійцям "Правого сектору" довелося відступити: декілька
хлопців були поранені, а половина техніки – розстріляна російськими найманцями.

Михайло залишився лежати на полі бою. Батькам повідомили, що син загинув, мама
купила все для похорону, але Слов'янськ був окупований, а бойовики тіло
чоловіка не передали.

В одному зі списків Служби соціального захисту добровольців ДУК ПС значився
"водій Михайло", який просто “загинув”, але жодної інформації про контакти чи
хто він, не було.

Пошуки інформації були довгими – через 3 дні по тому, як Михайло загинув, морги
Слов'янська перестали працювати й майже вся інформація була знищена. Згодом,
побратимами з "Правого сектора" все ж поїхали на кладовище, де їм показали
могилу. На ній був хрест, табличка з датою смерті та іменем Михайла. Його
поховали тоді, як тіло, яке ніхто не забрав. А в той час його чекали вдома...

16 квітня 2016 року, тіло Михайла перепоховали в мирній Україні, вдома, в Києві.

Пам'ятайте кожного полеглого захисника! Навіть якщо його війна тривала один
бій. Особливо, якщо його війна тривала один бій, а шлях назавжди додому
затягнувся на роки.

Світла пам'ять!  Фонд "Повернись живим"
