% vim: keymap=russian-jcukenwin
%%beginhead 
 
%%file 12_06_2021.fb.bilchenko_evgenia.3.chelovek_lichnaja_zhizn
%%parent 12_06_2021
 
%%url https://www.facebook.com/yevzhik/posts/3972280072807073
 
%%author Бильченко, Евгения
%%author_id bilchenko_evgenia
%%author_url 
 
%%tags chelovek,ukraina,zhizn,zhizn.lichnaja
%%title БЖ. Когда у людей нет личной жизни
 
%%endhead 
 
\subsection{БЖ. Когда у людей нет личной жизни}
\label{sec:12_06_2021.fb.bilchenko_evgenia.3.chelovek_lichnaja_zhizn}
\Purl{https://www.facebook.com/yevzhik/posts/3972280072807073}
\ifcmt
 author_begin
   author_id bilchenko_evgenia
 author_end
\fi

БЖ. Когда у людей нет личной жизни.

Тут прогрессивная часть украинской научной патриотической общественности,
избыток досуга у которой вследствие неспособности оной к написанию
фундаментальных научных трудов, приводит ее к тщательному отслеживанию страницы
БЖ (по моим подсчётам, за это время можно было написать две монографии по 200
страниц). Несмотря на тотальные баны этого милейшего цирка, великие умы нашей
страны находят исполнительных посредников, которые вычитывают, что я ем на
завтрак, как целуюсь с мужем, в какое время дня и ночи поклоняюсь Дугину на
шахидском коврике для йоги и т.д. Сегодня обострение умов дошло до того, что
прогрессивная общественность Украины решила, что на литературном квартирнике
книги о Христе \enquote{Пьета}, я буду обсуждать некие политические письма. Я-то,
дурочка, и Ассанжа обсуждать не собиралась, формат не тот, но спасибо за идею.

Трудно сказать, как эти акробатические трюки камеры Хичкока связаны с поэзией,
но, видимо, работа бывших членов парткома в нынешних трендовых схронах,
осуществляется чисто по инерции.

Всё это квазисоветское райцентровое начало бытия меня уже начало веселить,
особенно, когда мне приходят в личку не письма о Гегеле и Рорти, а копипасты с
украинских СМИ, естественно, самых гуманных СМИ в мире. Вот одно из них, откуда
меня исключили в 2017 г., опубликовало (без ссылки на оригинал, а хотелось бы:
английским я владею) написанное, как по бжезинским нотам, трепетное письмо в ЕС
в поддержку великого диссидента Пратасевича:

\url{https://www.liga.net/ua/politics/opinion/kak-universitet-mojet-udarit-po-diktature-putina-i-lukashenko-otkrytoe-pismo-uchenyh?fbclid=IwAR091069G7tzDXM0ZBDFX12Ws4uYt0E-eowjdZUhSxj342Iz3CDTk3hayV8}

Нет, меня не удивило, что его подписал Жижек, как не удивляют ни массовые
превращения марксистов в new lefts, ни коллаборации Хайдеггера с нацистами.
\enquote{Добро пожаловать в пустыню Реального} и \enquote{Бытие и время} все
равно мною будет читаемы и почитаемы.

Просто, Robert Žák, Oxana Chelysheva, Roman Blaško, can you give me original
text of this letter with authentic signitures of all following scientists? 

Впрочем, реакция левого блока ЕС немного опередила письмо левых (если верить
тексту из Лиги) в виде интервью представительницы эко-блока ЕС Катержины
Конечна журналисту  Анастасии Коваленко (\enquote{Звезда}: в Европе давать интервью
русским медиа не запрещено, а за зиг могут посадить: сама видела):

- Насколько оправданы, на Ваш взгляд, санкции, принятые Евросоюзом против
Беларуси, и почему они были приняты в спешке, еще до начала расследования
инцидента с Ryanair?

Я в данный момент с трудом могу судить, были или являются они оправданными. В
любом случае, инцидент должен быть расследован органом, назначенным для этого,
а это, по моему мнению, не Европарламент. Лишь после независимого расследования
можно перейти к следующим практическим шагам.

- В связи с этим событием можно вспомнить аналогичную ситуацию с самолетом
президента Боливии. Но тогда ЕС не возражал против этого безобразия. Почему
этот случай не вызвал абсолютно никакого осуждения или санкций со стороны
европейского руководства?

Есть еще много примеров из истории. Что делает самолет президента Моралеса
особенным, так это то, что они без колебаний посадили правительственный
самолет. Проблема в том, что некоторые государственные чиновники и
журналистское сообщество уже давно практикуют систему двойных стандартов,
поэтому, как это ни парадоксально, меня это даже не очень удивляет.

- Задержанный в Минске оппозиционер имел отношения к украинскими неонацистами.
Почему Европа предпочитает закрывать на это глаза?

Европа уже давно закрывает глаза на проблемы в Украине. Марши неонацистов,
зверства, которые там совершаются, военизированные батальоны, которые грабят,
закрытие телеканалов, запрет политических партий - все это сегодня Украина, но
в европейских средствах массовой информации вы об этом ничего не прочтёте. Это
делается целенаправленно, и это просто доказывает практику двойных стандартов,
о которых я говорила.

Катерина КОНЕЧНА

Kateřinu Konečnou europoslankyni KSČM

Jak oprávněné jsou podle vašeho názoru sankce přijaté Evropskou unií proti
Bělorusku a proč byly přijaty ve spěchu, ještě před zahájením vyšetřování
incidentu se společností Ryanair?

Já v tuto chvíli mohu jen těžko posoudit, zda byly nebo jsou oprávněné.
Každopádně incident má prošetřit orgán, který je k tomu určený, a to Evropská
rada podle mého soudu není. Po nezávislém prošetření teprve přistupme k dalším
krokům.

V souvislosti s touto událostí si můžeme vzpomenout na podobnou situaci s
letadlem prezidenta Bolívie. Ale pak EU proti tomuto pobouření nenamítala.
Proč ten případ nezpůsobil absolutně žádné odsouzení ani sankce ze strany
evropského vedení?

Těch příkladů z historie je mnohem víc. Letadlo prezidenta Moralese je
výjimečné tím, že takto neváhali uzemnit vládní speciál. Problém je v tom, že
někteří představitelé států a novinářská obec už dlouho praktikuje systém
dvojího metru, takže mě to paradoxně ani příliš nepřekvapuje.

Opozice zadržovaná v Minsku měla mít vztah s ukrajinskými neonacisty. Proč tomu
Evropa raději zavírá oči?

Evropa dlouhodobě zavírá před problémy na Ukrajině oči. Pochody neonacistů,
zvěrstva, která tam páchají, polovojenské prapory, které rabují, zavírání
televizí, zákazy politických stran-to vše je dnešní Ukrajina, ale v médiích se
o tom moc nedočtete. Je to účelové a jen to dokazuje dvojí metr, o kterém jsem
hovořila.

Kateřina KONEČNÁ
