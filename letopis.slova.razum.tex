% vim: keymap=russian-jcukenwin
%%beginhead 
 
%%file slova.razum
%%parent slova
 
%%url 
 
%%author 
%%author_id 
%%author_url 
 
%%tags 
%%title 
 
%%endhead 
\chapter{Разум}
\label{sec:slova.razum}

%%%cit
%%%cit_pic
%%%cit_text
С интересом наблюдаю за славагеройской эпопеей на футболках сборной Украины. В
УЕФА сказали «да», в УЕФА сказали «нет», в России возмущены, власти Украины
восхищены… Не удивлюсь, если «героям слава» нарисуют в виде пиктограммы, а
может даже тайно зашьют на внутреннюю часть труселей. Тут ведь уже движ пошел и
его никакой \emph{разум} не остановит
%%%cit_comment
%%%cit_title
\citTitle{Бандеровцы при власти создали в Украине уютненькую Уганду}, 
Игорь Лесев, strana.ua, 13.06.2021
%%%endcit

%%%cit
%%%cit_head
%%%cit_pic
%%%cit_text
Организовывать фрагменты жизни в упорядоченные формы, противостоять хаосу есть
первооснова любой осмысленной деятельности человека. «Во всяком порядке скрыт
зародыш разрушения. Всякий порядок обречён, но биться за него имеет смысл» (Н.
Уэст).  Государство – это удивительная идея человеческого \emph{разума}. Мы должны
быть признательны всем теоретикам и практикам государственного управления, и
строительства, стараниями которых оно до сих пор существует и развивается,
служа человеку и обществу. Их фантастическому чувству гармонии и соразмерности,
ритмов и темпа, времени и народов
%%%cit_comment
%%%cit_title
\citTitle{Замысел украинского государства: социальность, самодостаточность, независимость. Пятая часть}, 
Акулов-Муратов В. В., analytics.hvylya.net, 18.10.2021
%%%endcit

%%%cit
%%%cit_head
%%%cit_pic
%%%cit_text
Авжеж. Ви ж самі згадали, що мільйони людей сміливо кидалися на штурм загадок
природи й \emph{розуму}, самозречено мучили свою плоть і душу в голоді, холоді... але
не осягали еволюційного результату. Чому? Передусім вони провели в свідомості
роковании кордон своєї гріховності, мізерності, ущербності, слабосилості...
По-друге, вони прохали рішення, рятунку, звільнення, осяяння у певної істоти,
яка перебуває поза ними, у істоти, котра ніколи не проявляла ні своєї любові,
ні, зрештою, ненависті! Принц, володар всесвіту, прохав милостині у привида.
Так ось — самообмеження, відмова від примітивної енергетики та функціонування
— це лише один з важелів, той первісний імпульс, що дає змогу вибратися на
високу скелю для вирішального стрибка...
%%%cit_comment
%%%cit_title
\citTitle{Вогнесміх}, Олесь Бердник
%%%endcit
