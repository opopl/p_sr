% vim: keymap=russian-jcukenwin
%%beginhead 
 
%%file 29_03_2022.fb.jakimenko_anatolij.1.prichina_agressii
%%parent 29_03_2022
 
%%url https://www.facebook.com/permalink.php?story_fbid=5727395563953813&id=100000502778735
 
%%author_id jakimenko_anatolij
%%date 
 
%%tags 
%%title Причина військової агресії росії проти України
 
%%endhead 
 
\subsection{Причина військової агресії росії проти України}
\label{sec:29_03_2022.fb.jakimenko_anatolij.1.prichina_agressii}
 
\Purl{https://www.facebook.com/permalink.php?story_fbid=5727395563953813&id=100000502778735}
\ifcmt
 author_begin
   author_id jakimenko_anatolij
 author_end
\fi

Повертаючись до запитання про причину військової агресії росії проти України.
Дуже часто серед причин війни називається екзестенційна загроза для москви з
боку України. Під загрозую чомусь розуміється побудова демократичної, заможної,
щасливої України. Мовляв, побудова під боком успішної держави з демократичними
інститутуами делегітимізує і десакралізує царство московське. У мене запитання
до «інтелектуалів», які поширюють таку версію – ви, це серйозно? З яких пір в
Україні будується щаслива і заможна країна? Чи насправді хтось переконаний, що
росія бачить у нас справедливість? 

Як це не дивно, але наші оцінки того, що ми будуємо тут в Україні і оцінки
москалів дивним чином збігаються. До 24 лютого ми будували світову клоаку.
Українці відмовлялися жити, що проявлялося у демографічній катастрофі
спричиненою небажанням народжувати, молодь масово залишала країну, гетманцеви і
радуцькі, а до цього кадри порошенко розвалювали та розграбовували країну.
Демократію скабєви представляли як джерело хаосу і бідності. 

Не дивлячись на це росія напала на неуспішну країну у якій живуть нещасливі
люди. До речі, ці факти були одним з чинників на які опирався Кремль
розраховуючи на підтримку. Чому вона це зробила? – ось у чому запитання. Путіну
логічно було б тримати Україну як пугало, наочний приклад результату
демократії, яка призводить у владу клоунів. Тим більше, що саме мир і
стабільність були запорукою економічного занепаду України. 

Україна насправді несе загрозу Росію, але не військову, а екзестенційну. Битва
іде не за ратний спадок Київської Русі і не за міфічний центр християнства з
центром у Києві. Все це повна дурня у ХХІ столітті для світської країни з
вкрапленнями містичного світогляду та культової обрядовості. Росія боїться
прикладу проявлення волі, яка розкриє красу людини. Росія ненавидить красу,
тому прояв волі є кримінальним злочином, тому волю потрібна зламати. Тобто мова
йде про естетичне сприйняття світу. Красивим може бути тільки раб. Путін
дивиться на Кадирова і отримує естетичне задоволення. Собака в образі людини,
інтимні стосунки з ворогом – це і є рускій мір. І якщо народ не хоче мати
вигляд Кадирова, Шойгу чи Наришкіна тощо, якщо не хоче спати з ворогом, то його
треба стерти з землі. Звідси сенс життя москаля і сенс соціального ліфту –
пробитися на верх соціальної піраміди, щоб отримати законне право ламати волю і
перетворювати всіх навколо у прислужливих собак. Когось в агресивних
(кадирівці), когось у покірних (буряти).

Все це важливо розкривати, якщо так хочеться говорити і домовлятися з ворогом,
хоча б для того, аби пояснити Заходу, що означає превентивний хімічний чи
ядерний удар у випадку виникнення екзестенційної загрози для росії. Ця загроза
уже давно осягнута москалями і всі рішення також давно прийняті. Росія
застрягла на точці обґрунтування використання зброї масового враження.
Катастрофою для нас буде не лише фізичний біль, але й трактування Заходом
можливого удару росією по нам психічними розладами Путіна. Існуючі наміри
Путіна – це не психіатрія однієї людини, а свідомий етичний вибір народу, який
претендує на захист вічного і позиціонує себе провідником консервативної ідеї і
моральної культури. У цьому весь жах. На жаль, поки що, ми не справляємося з
поставленою задачею, навіть гірше – виправдовуємося за нав'язаними
звинуваченнями у військовому нападі. 

Після падіння Берліну у 1945 році розпочався Нюрнберзький процес на якому була
засуджена ідеологія нацизму. Україні не вдасться взяти Москву, але в наших
силах провести свій аналог Нюрнбергу, на якому засудити криміналізацію волі,
яка призводить до дегуманізації людини. Криміналізація волі – це коли
проявлення волі трактується як кримінальний злочин.

\ii{29_03_2022.fb.jakimenko_anatolij.1.prichina_agressii.cmt}
