% vim: keymap=russian-jcukenwin
%%beginhead 
 
%%file 21_12_2021.fb.fb_group.story_kiev_ua.1.vasyl_vyshyvanyj.cmt
%%parent 21_12_2021.fb.fb_group.story_kiev_ua.1.vasyl_vyshyvanyj
 
%%url 
 
%%author_id 
%%date 
 
%%tags 
%%title 
 
%%endhead 
\zzSecCmt

\begin{itemize} % {
\iusr{Мария Константиновская}
Спасибо! К стыду своему не знала!

\iusr{Людмила Сидоренко}
Визначна постать на історичній мапі Європи.

\begin{itemize} % {
\iusr{Сергей Удовик}
\textbf{Людмила Сидоренко} Очень мутная личность, читайте Тимоти Снайдера

\iusr{Сергей Удовик}

Австрия выдвинула Васыля Вышиванова, чтобы помешать Скоропадский создавать
Украïнську державу, они сами хотели владеть Украиной. Тогда Украина была
разделена зонами контроля - немцы на севере, в Киеве, а австрийцы на юге, с
центром в Одессе, вот здесь они и продвигали Васыля, но безуспешно из-за его
сомнительных качеств

\end{itemize} % }

\iusr{Елена Полякова}
Выдающийся человек

\iusr{Ольга Васильевна}
Удивительно! Спасибо.

\iusr{Надежда Кулиш}

\ifcmt
  ig https://scontent-frx5-2.xx.fbcdn.net/v/t39.1997-6/s168x128/17527812_1652591011433966_4391041969200037888_n.png?_nc_cat=1&ccb=1-5&_nc_sid=ac3552&_nc_ohc=9CsRD8iKhy0AX-0C9UV&_nc_ht=scontent-frx5-2.xx&oh=00_AT_ZaQiZMSv7L4W0kcQ19IKwhlF1e7kqvJo0QvJaQwOVOg&oe=61D02218
  @width 0.1
\fi

\iusr{Оксана Буржимська}
Хотіла вже написати, що про Ігнатьєва прочитала з задоволенням, а про Вишиваного знають всі. А тут - упс. Не всі

\iusr{Галина Полякова}
\textbf{Оксана Буржимська} Тепер знатимуть більше!

\iusr{Татьяна Аверкиева}
Злая ирония судьбы

\iusr{Ludmiła Slyesaryeva}
Очень печальная судьба. Хорошо хоть памятник уже есть в Киеве этому уникальному человеку...

\begin{itemize} % {
\iusr{Галина Полякова}
\textbf{Ludmiła Slyesaryeva} Да! И это стало для меня неожиданной радостью.

\iusr{Светлана Лебедева}
\textbf{Ludmiła Slyesaryeva} Підкажіть будь ласка, де памятник? Дякую.

\begin{itemize} % {
\iusr{Ludmiła Slyesaryeva}
\textbf{Svetlana Lebedeva} 

поруч із школою № 61, неподалік від метро \enquote{Лук'янівська}, вул. Юрія Іллєнка
(раніше цей її відрізок, здається, називався вул. Герцена і зупинка тролейбуса
тут була)


\iusr{Светлана Лебедева}
\textbf{Ludmiła Slyesaryeva} Дякую. Знайшла.

\ifcmt
  ig https://scontent-frx5-1.xx.fbcdn.net/v/t39.30808-6/269729997_4565324673582394_3866932943537212471_n.jpg?_nc_cat=110&ccb=1-5&_nc_sid=dbeb18&_nc_ohc=ctUebBCehiMAX9Ca0Z5&_nc_ht=scontent-frx5-1.xx&oh=00_AT_JtHThhoMEGksCMHasBPpHsaXS3jtCkoz2Gc6wS2olUw&oe=61D18D9F
  @width 0.2
\fi

\iusr{Ludmiła Slyesaryeva}
\textbf{Svetlana Lebedeva} саме так

\iusr{Ludmiła Slyesaryeva}
А ось мої фото:

\ifcmt
  %tab_begin cols=2,no_fig,center,no_height,width=0.3,resibox=0.5
  tab_begin cols=2,no_fig,center,resizebox=0.7

     pic https://scontent-frt3-1.xx.fbcdn.net/v/t39.30808-6/269731164_2112613578893616_4724614392684891913_n.jpg?_nc_cat=106&ccb=1-5&_nc_sid=dbeb18&_nc_ohc=5CWDGMiCIU4AX-wrJuN&_nc_ht=scontent-frt3-1.xx&oh=00_AT9hVrNs8ewz6sswo7XjSGdKnEX0PYGVuOGtU7mlJz2EyQ&oe=61D14111

		 pic https://scontent-frx5-1.xx.fbcdn.net/v/t39.30808-6/269693705_2112614395560201_6147021150172089501_n.jpg?_nc_cat=111&ccb=1-5&_nc_sid=dbeb18&_nc_ohc=0pfY_53aa7oAX_PGzxY&_nc_ht=scontent-frx5-1.xx&oh=00_AT_xsg-WMzG3RR-WvuNEVRNCDcWdbE7POCtIwNLIGZxUnw&oe=61D0959C

  tab_end
\fi

\iusr{Lena Prusinowska}
\textbf{Ludmiła Slyesaryeva} Адреса школи була вул. Мельникова 39. Я в ній вчилася  @igg{fbicon.smile} 

\end{itemize} % }

\end{itemize} % }

\iusr{Надія Коміссарова}
Радянська влада нищила все, що дихало вільно!
Самі не жили і іншим не давали! До цих пір нічого не змінилось - \enquote{к свободной
жизни их умы непримиримы}...

\begin{itemize} % {
\iusr{Shilov Sergei}
\textbf{Надія Коміссарова} чьи умы? Их нет уже давно....

\iusr{Светлана Моисеенко}
\textbf{Надія Коміссарова} да, \enquote{все нищила}  @igg{fbicon.anger}  та чогось не знищила

\iusr{Ольга Морозова}
\textbf{Надія Коміссарова} Не все радянська влада знищила, до цього часу продовжують нищити.
\end{itemize} % }

\iusr{Таня Коваленко}
Спасибо, очень интересно! Каждый раз ловишь себя на мысли :как много не знаешь!

\iusr{Dmitry Agafonov}

За переказами був вельми порядною людиною, навіть у дрібницях. Виходячи з нашої
сучасної псевдоетики - був би меньш порядним жив би краще

\iusr{Петр Кузьменко}
Достойная судьба благородного аристократа, офицера и патриота Украины! Честь и совесть превыше всего!

\iusr{Вадим Горбов}

Вы забыли упомянуть богемный период жизни Василия в Париже двадцатых годов,
ярко описанных в американском бестселлере «Красный князь». Там совсем другая
мутная история. Оргии, гей-вечеринки, махинации, аферы и тому подобное.

\begin{itemize} % {
\iusr{Галина Полякова}
\textbf{Вадим Горбов} 

Меня меньше всего интересует эта сторона жизни. Я не люблю, когда мне
заглядывают под подол и сама этого не делаю. Вероятно, это мой большой промах,
но и не все бестселлеры я читаю и уважаю. Ценю хорошие дела и не собираю
сплетен.

\begin{itemize} % {
\iusr{Вадим Горбов}
\textbf{Галина Полякова} 

Книга профессора Истории Йельского университета - не сплетня. Вы напрасно его
героизируете, как по мне. И не верьте Википедии откуда у вас целые абзацы в
посте. Там всё врут.))


\iusr{Галина Полякова}
\textbf{Вадим Горбов} 

\enquote{Все врут} - сказал доктор Хаус. Профессора из Йельского университета мі можем
воспринимать как поборника правді?

\iusr{Вадим Горбов}
\textbf{Галина Полякова} Никому нельзя верить, - сказал Мюллер.

\iusr{Елена Корниенко}
\textbf{Вадим Горбов} Даже вам.

\iusr{Вадим Горбов}
\textbf{Елена Корниенко} даже Википедии

\end{itemize} % }

\iusr{Наталья Эгатова}

Спасибо за исторические миниатюры, мне нравится как вы пишите. А насчёт
правдивости, скажем, что в любой правде всегда есть немного лжи!

\begin{itemize} % {
\iusr{Люба Потемкина}
\textbf{Наталья Эгатова} вернее не лжи, а литературного оформления. .Галине Поляковой спасибо.
\end{itemize} % }

\end{itemize} % }

\iusr{Станислав Кшановський}

Булаву и трон ему предлагали исключительно активисты- авантюристы- мечтатели.
И корпуса УСС не существовало. Был Легион/ полк УСС в Австро- Венгрии, бригада
УСС в ЗУНР. А также Корпус Січових Стрільців в УНР. Но Габсбург не командовал
ими.

\begin{itemize} % {
\iusr{Галина Полякова}
\textbf{Станислав Кшановський} 

История УСС сложная и трагичная. Вот цитата из Википедии: \enquote{Поповнені українцями
з російської армії, УСС сформували в Києві у листопаді 1917
Галицько-Буковинський курінь Січових Стрільців (СС), який згодом розгорнувся в
полк СС, а далі в корпус і групу СС, що були однією з найкращих формацій
української армії.} И вот еще одна: \enquote{1 квітня 1918 р. Вільгельм Габсбурґ
перейняв командування УСС біля Херсона}.

\begin{itemize} % {
\iusr{Станислав Кшановський}
\textbf{Галина Полякова} этого никто не отрицает. Корпуса УСС не существовало, он им НЕ командовал.

\iusr{Галина Полякова}
\textbf{Станислав Кшановський} Успокойтесь, не волнуйтесь Вы так. Не было. Ничего не было.

\iusr{Станислав Кшановський}
\textbf{Галина Полякова} мне чего волноваться?)

\iusr{Станислав Кшановський}
от этого корпус УСС не появится и Габсбург им командовать не будет)

\iusr{Галина Полякова}
\textbf{Станислав Кшановський} 

Была полная неразбериха, все менялось, разваливалось, строилось заново и снова
рушилось. Коммуникации не было никакой. Кто-то что-то делал, чем-то и кем-то
командовал. Были коши, полки, курени, сотни, был даже ОСАДНИЙ КОРПУС СІЧОВИХ
СТРІЛЬЦІВ. Это не из ненавистной Вами Википедии, это строка из текста,
опубликованного Институтом Истории Украины НАН. Не Йель, конечно... Ну,
вибачайте.


\iusr{Станислав Кшановський}
\textbf{Галина Полякова} 

конечно был Осадний Корпус СС, потом Корпус СС, потом Группа СС. Я же этого и
не пытаюсь отрицать. Но Корпуса УСС не было. И корпусом СС/УСС Габсбург не
командовал. Всё просто)


\iusr{Станислав Кшановський}
\textbf{Галина Полякова} Ваш текст замечательный, есть всего два уточнения:

1. он никогда командовал ни корпусом УСС, ни Корпусом СС ( т. к. корпуса УСС не
существовало, корпусом СС с момента создания командовал отаман / полковник
Евген Коновалец).

2. Он не мог отказаться от булавы либо трона, т.к. реальной силы, способной
его поставить на трон императора, короля или гетмана не было.

\end{itemize} % }

\iusr{Георгий Горбенко}
\textbf{Станислав Кшановський} Таки да.

\end{itemize} % }

\iusr{Oleg Chorny}
Є цікава книжка про Василя Вишиваного \enquote{Червоний принц}, автор Тімоті Снайдер

\iusr{Татьяна Оржеховская}
За что боролся на то и напоролся

\begin{itemize} % {
\iusr{Елена Безродная}
\textbf{Татьяна Оржеховская} таке как вы и обрекают Украину на бездарных правителей
\end{itemize} % }

\iusr{Юрий Блохин}

Гей и романтик-авантюрист, такими персонажами наполнена история революций и
переворотов, в мирное время им скучно и не уютно  @igg{fbicon.wink} 

\begin{itemize} % {
\iusr{Галина Полякова}
\textbf{Юрий Блохин} 

А в добрые помыслы, в двоих деток верить неохота? Нам бы чего жареного,
пикантного, клубнички бы! Кто с кем спал и, главное, как! Простите, но мне это
не интересно.


\iusr{Юрий Блохин}
\textbf{Галина Полякова} 

Вы знаете, меньше всего меня интересует т.н. клубничка, но одно без другого не
бывает, красивая романтическая история с грустным концом, можно добавить еще
парижскую шерше ля фам с авантюрными наклонностями и предательством, порочные
увлечения, ссору и непонимание со стороны с отца и братьев и получится
неплохая мелодрама .... главное не вникать  @igg{fbicon.wink} 

\end{itemize} % }

\iusr{Ірина Дебкалюк}

Людина воювала, захищаючи молоду Українську республіку, а ви, сидячі на дивані
зараз пишете якусь ахінею у коментарях! Є доступ до листування, спогадів...
Читайте! І взагалі, як легко обливати помиями людей порядних, що вже пішли!
Дякую за допис автору!

\iusr{Александр Венге}
Благодарю Вас за эту удивительную историю. Остаётся недоумевать по поводу того, что не знал ее раньше.

\iusr{Надежда Лабик}
Хочется сказать словами поэта: \enquote{Да, были люди в наше время, не. то, что нынешнее племя....}.

\iusr{Елена Корниенко}

Времена, конечно были насыщеные событиями, и столько судеб
перемололось. Когда-нибудь наши правнуки будут читать биографии теперешних
героев и антигероев. Как Россия воевала с Украиной, а немцы и французы их мирили.

\begin{itemize} % {
\iusr{Александр Дегтярёв}
\textbf{Елена Корниенко} На городі - бузина, а в Києві дядько !! Із-за подобніх комментаторов приходится покидать круппу.

\iusr{Елена Корниенко}
\textbf{Александр Дегтярёв} 

что вас не устраивает. Вы почитайте все комменты. Там всё и герой и предатель и
мученик и гомосексуалист. Не то же самое?


\iusr{Александр Дегтярёв}
\textbf{Елена Корниенко} Мне проще показать, что меня устраивает.
\end{itemize} % }

\iusr{Людмила Скомаровська}
Такие личности, как правило, долго не живут**((*

\begin{itemize} % {
\iusr{Ірина Дебкалюк}
\textbf{Людмила Скомаровська} 

Особенно когда их спустя 27 лет похищает кгб в Европе и везут в Лукяновскую
тюрьму, где человек умирает при невыясненных обстоятельствах...


\iusr{Ірина Дебкалюк}
\textbf{Людмила Скомаровська}

\ifcmt
  ig https://scontent-frx5-1.xx.fbcdn.net/v/t39.30808-6/269744881_1044907039388751_258545079058619754_n.jpg?_nc_cat=111&ccb=1-5&_nc_sid=dbeb18&_nc_ohc=E2hOR1_tt4oAX_ovsTq&_nc_ht=scontent-frx5-1.xx&oh=00_AT85eF8Nt2sFIXZzc01ptS2ZsmuOc4gBnKGlzkFKHBGVLg&oe=61D0BE15
  @width 0.3
\fi

\iusr{Ірина Дебкалюк}
Таким его привезли в Киев и каким он стал в советской тюрьме.

\end{itemize} % }

\iusr{Арт Юрковская}

Я вот не могу понять. По Брестскому мирном договору Украина была оккупирована
австрийскими и немецкими войсками. Так о о какой УНР идет речь? Оккупация она и
есть оккупация. Еще немцы по договору вывозили продукты в гигантских количествах
( а начал изымание продуктов и зерна еще царь, потом Керенский).А ругают за
продразверстку большевиков. Если бы случайно не революция в Германии - Украины
бы не было и фашистов с Гитлером тоже. Была бы Великая Германия или вариант -
Великая Польша (оккупировали Киев и только Красная армия их оттуда выбила).

\begin{itemize} % {
\iusr{Ірина Дебкалюк}
\textbf{Арт Юрковская} 

а вы в курсе, что борьба за государство Украина шла с 1917 по 1921 год? И
немцев призвали, чтобы защитится от большевицкий России? Почитайте, книг море
сейчас об этом периоде! И зарубежных и наших авторов!

Надо знать историю страны, в которой живёшь. Немцы только эпизод в этой
истории!

Українську державу визнано 30 державами, в Києві розташовувалися постійні
представництва 10-ти з них; Україна мала дипломатичні місії в 23 країнах (на
рівні послів Німецька імперія, Османська імперія, Болгарське царство,
Швейцарія, Швеція, Норвегія; дипломатичні представництва Грузинської
Демократичної Республіки, Азербайджанської Демократичної Республіки,
Фінляндії).

\begin{itemize} % {
\iusr{Олег Курилов}
\textbf{Ірина Дебкалюк} 

Немцев никто не призывал, они сами пришли никого не спрашивая! \enquote{Надо знать
историю страны, в которой живёшь} Очень хорошее замечание! Респект! Но к
сожалению, история Украины крайне скучна, как для меня. А вот события
Гражданской войны 1917-1921 гг. очень интересны, по этому я немножко знаю про
этот период. По этому и делаю вам маленькое замечание. Вы немножко не правы.
УНР в 1917-начала 1918гг. было никто и ничто и с ними никто особо не считался.
по этому только Германия и Австро-Венгрия решала. А УНР априори немогли кого то
звать или не звать  @igg{fbicon.wink} . Немцы в 1918 году, это не просто Эпизод в Истории, а и
создатели Украинской Державы!!! ( Или вы думаете без разрешения немцев её кто
то бы признал?) и уже на основании границ УНР - Ленин создал УССР которую
наследовала современная Украина. А вы говорите эпизод! Совсем НЕТ!
П.С. \enquote{Українську державу визнано 30 державами} а Вы можете перечислить эти
страны с сылкой на источник? А то у меня большие сомнения по поводу 30 стран.
Это написано в укрВики. но это источник ещё тот... Они за страны принимают
Крым, Кубань, какую нибудь Донецко-Криворожскую республику. А в реальности я
очень сомневаюсь в таком количестве стран! Потому что страны Антанты и им
симпатизирующие не признавали Украинскую Державу, а их всё же было больше, чем
симпатиков Четверного союза.

\end{itemize} % }

\iusr{Станислав Кшановський}
\textbf{Арт Юрковская} 

По Брестскому миру Германия, Австро-Венгрия, Турция и Болгария признавали УНР. 
Также, как и РСФСР. И германские и австро-венгерские войска наступали как
союзные УНР. Какое отношение это имело к Гитлеру? Какое отношение имели фашисты
к Германии? У Германии не было шансов на победу, как минимум, после
вступления САСШ в войну. И Брестский мир только продлил агонию Центральных
государств.


\iusr{Volodymyr Nekrasov}
\textbf{Art Yurkovska} 

Окупантів не запрошують. А щодо зерна - Німеччина розраховувалась за нього
сукном і паливно мастильними матеріалами. Гігантська кількість це яка? Адже при
цьому на осінь 1918 було зібрано достатньо і фуражного зерна, і такого щоби
пережити зиму 1919.

\begin{itemize} % {
\iusr{Олег Курилов}
\textbf{Volodymyr Nekrasov} 

гигантская, это совсем не мало, точных цифр не помню конечно и лень искать. Но
помниться они были довольно существенными. Хотя, конечно, всё в мире
относительно, и как вы правильно заметили: \enquote{на осінь 1918 було зібрано
достатньо і фуражного зерна, і такого щоби пережити зиму 1919}. То есть, вы
считаете вполне естественным платить дань своему сюзерену -Германии? Ну в общем
то, конечно Вы правы.... \enquote{Німеччина розраховувалась за нього сукном і паливно
мастильними матеріалами.} а можно отсюда по подробнее? воюющей стране -
Германии, эти материалы самой нужны, зачем их отдавать? Или они были в избытке
у Германии и им небыло куда их девать? Просто интересно, зачем Германии это
кому то отдавать? Может продавали? А замечание: \enquote{Окупантів не запрошують.} было
немного не по адресу  @igg{fbicon.wink} 

\iusr{Volodymyr Nekrasov}
\textbf{Олег Курилов} 

Шановний пане Курилов, якщо Вас справді цікавить історія Української Держави
(гетьманат), раджу для початку прочитати \enquote{Воспоминания} самого Гетьмана
Скоропадського, потім \enquote{Історію України 1917-1923рр.} Дмитра Дорошенка (міністр
закордонних справ в уряді Скоропадського). Там в додатках Ви знайдете чимало
документів про розрахунки, кредитування, і навіть сплату мита. Також на сайті
сучасного історика Павла Гай-Нижника зібрано чимало документів, статей зокрема
про фінансову та господарчу діяльність урядів Української Держави та її
союзників. Ще мабуть варто почитати В'ячеслава Липинського, бо саме він став
ідеологом українського консерватизму і одним із засновників Союзу
хліборобів-державників.

\iusr{Олег Курилов}
\textbf{Volodymyr Nekrasov} 

Спасибо! обязательно посмотрю. Дорошенко читал, не помню у него такого. У меня
большие сомнения, что Германия нам что то давала! Хорошо обязательно поищу про
это. А про союзников Украинской державы, это интересно кто? Сюзерены Германия и
Австро-Венгрия? Или государство ВВД Краснова? Сильно громко звучит  @igg{fbicon.smile}  Ещё раз
спасибо за Павла Гай-Нижника. Незнаю такого. Почитаю.

\end{itemize} % }

\end{itemize} % }

\iusr{Volodymyr Nekrasov}

Читав у Снайдера «Червоний Князь», що у австрійців не дуже добре складались
відносини зі Скоропадським та й з самими німцями. Тож вони розкручували проєкт
«Королівство Україна» на чолі з Вільгельмом. Звісно Січовим стрільцям,
громадянам Австро-Угорщини «Вишиваний» був ближчим. А для самих австрійців
Габсбург був більш зрозумілим ніж Скоропадський, чи тим паче Петлюра.

\begin{itemize} % {
\iusr{Арт Юрковская}

Ясно одно - никакой самостийной Украины не планировалось, даже поляками. На
Парижской мирной конференции, несмотря на заявления украинских сил из разных
стран ,Великобритания, Франция и США образовать Украину не разрешили.
Образовали Польшу, Австрию, Венгрию, Турцию и проч. страны. В дурнях остались
украинцы и курды. Почему? Потому что Галицию и Волынь прибрать хотели другие, а
Малороссия была \enquote{слишком русская} и большая, а давать кучке нищих политических
\enquote{украинцев} в управление огромную страну посреди Европы - это было не в
интересах держав-победительниц Первой мировой войны. Тем более народ бредил
помещичьей землей, которую обещали и дали большевики. Тем более украинские
рабочие и крестьяне знать не хотели никаких автрийских Габсбургов. С какой
радости спрашивается???

\begin{itemize} % {
\iusr{Volodymyr Nekrasov}
\textbf{Art Yurkovska} 

Частково Ви маєте рацію. Хоча я би не погодився з твердженням \enquote{слишком русская}
Малоросія. Не більш \enquote{русская} ніж Україна 90-х років ХХ ст. У тому що країни
співдружності не хотіли підтримати створення нової держави Україна, я вбачаю
суто меркантильний інтерес. Ці країни очікували на повернення боргів. Тому і
підтримували представників \enquote{старої} влади Росії. Адже більшовики одразу заявили
що \enquote{всі, кому ми винні - всіх прощаємо}. А щодо бажання робітників і селян... я
би взагалі не наважився щось стверджувати.

\iusr{Арт Юрковская}
\textbf{Volodymyr Nekrasov} 

\enquote{Слишком русская} - это в том смысле, что вообще не украинская. И это не мое
выражение, а аргументы стран-победительниц на Пар. конференции и переговоров
,касающихся послевоенного устройства украинских земель. Никто не думал отдавать
австрийские Галицию и польскую Волынь с Холмщиной в руки УНР, которая совершенно
не могла самостоятельно защищаться. Отдать им эти земли значило вернуть их в
Россию и возродить империю в еще большем размере. Ясно, что никто этого не хотел
делать! А немцам было приказано ждать войск Антанты для оккупации Украины
французами, англичанами и американцами. Немцы же сразу свернулись и ушли, чтобы
назло французам ее заняли русские. Мол, \enquote{не доставайся же ты никому!}. Если не
нам, то и не вам. Украина была вырвана немцами у русских ценой долгой и кровавой
войны, а французы захотели прибрать ее и зажать немцев в голодных тисках
навсегда, так как хлеб в Европу поставляли американцы и Рос. империя. Потому что
немцы сдали Украину большевикам, в благодарность большевики начали поставлять
хлеб Германии и, когда была запрещена немецкая армия - начали с ними военное
сотрудничество. Именно поэтому только бывшую Малороссию смогли занять
большевики (с помощью украинских рабочих и крестьян, которые бредили землей и
поэтому не поддерживали буржуйскую УНР, при которой немцы жестоко давили
крестьянские восстания). А поляки, Пилсудский, так сразу заявили курс на
восстановление Польши в границах 17872г. и заняли Киев. Если бы Красная армия их
не выбила - то это была бы Польша, а вовсе не УНР и Украина. Так что никакой
самостийной и незалежной не планировалось. Современную территорию Украины
слепили Ленин, Сталин и Хрущев, а вовсе не \enquote{наши европейские партнеры}, Петлюра и
Скоропадский.


\iusr{Галина Полякова}
\textbf{Арт Юрковская} С Вашим видением ситуации я не согласна. Но спорить не хочу. Не вижу в єтом смісла.

\iusr{Арт Юрковская}
\textbf{Галина Полякова} 

Это не мое видение, а Парижская мирная конференция, которая решала послевоенное
устройство мира и Европы. На заявления украинских деятелей об образовании, по
примеру Польши, Украины - ответили примерно так: Польша всегда была
антирусской, а Малороссия - как они выразились \enquote{русская}. Поэтому к этой земле
присоединить австрийскую Галицию и польскую Холмщину и Волынь - не
захотели. Отказали. Вместо Украины образовали Польшу, которая сразу и начала
воевать, пытаясь вернуть себе Украину, Белоруссию и Литву. Я все это прочитала в
книге \enquote{Нескорена Волынь}. Очень интересная книга. Там всякие подробности и
акценты, которые замалчивали и советские и несоветские историки. А Габсбурги
вообще войну проиграли и Вышиваный канул вместе с Австро-Венгерской империей в
Лету.

\iusr{Галина Полякова}
\textbf{Арт Юрковская} 

Спасибо за ссылку. Боюсь. что книга одного автора, отражает позицию одного
автора. Акценты и подробности представлены так, как видит именно этот автор.
Для большей объективности желательно вооружиться еще парой-тройкой источников.

\iusr{Арт Юрковская}
\textbf{Галина Полякова} 

Был четко поставлен вопрос на мирной конференции - и были представители
украинских политиков в Париже - они сделали заявление о создании государства
Украина из Галиции, Холмщины, Волыни и УНР. Им отказали. И объяснили
нецелесообразностью по причине абсолютной исторической разнородности земель и
потому что УНР не имела сильной армии. Большевики же имели армию в пять
миллионов, Белая армия имела в лучшие времена 500 тыс. а армия УНР от силы 200
тысяч. Поэтому было решено Галицию отдать Австрии, Холмщину и Волынь Польше -
вновьобразованным странам. А территорию УНР, оккупированную немцами,
оккупировать армией Антанты. И было дано распоряжение ждать прибытия войск, но
Германия, не дожидаясь их, быстро ушла. Это не мнение, а исторические факты. Всем
известные. А затем две страны-изгои - СССР и Германия начали друг другу
помогать. Германия ела исключительно российский(украинский) хлеб всегда. Если бы
Украину оккупировала Франция и начала контролировать зерно - то такой страны, 
как Германия мы бы не знали. А большевики, в благодарность за то, что Германия
ушла, не дождавшись Антанты, и таким образом способствовала Красной армии занять
Украину, начали снабжать ее хлебом. А Австрийская империя Габсбургов -
развалилась и исчезла с карты Европы. У Василя Вышиваного не было никаких шансов
.

\end{itemize} % }

\end{itemize} % }

\iusr{Валентина Маляр}
Я вже читаю

\iusr{Марина Маташ}
Спасибо, за увлекательную, хоть и печальную историю  
@igg{fbicon.thumb.up.yellow}  @igg{fbicon.hands.pray} @igg{fbicon.exclamation.mark.double}

\iusr{Леонид Яннау}
Все, до кого дотягивались лапы НКВД, заканчивали жизнь в их застенках. Даже незатейливый певец Лещенко

\begin{itemize} % {
\iusr{Галина Полякова}
\textbf{Леонид Яннау} Но он, вроде, умер в румынской тюрьме? Хотя, конечно, разница не большая.

\begin{itemize} % {
\iusr{Леонид Яннау}
\textbf{Галина Полякова} я ведь о том, что если их руки дотягивались, то человек пропадал в застенках

\iusr{Галина Полякова}
\textbf{Леонид Яннау} Вы абсолютно правы. Доставали везде. Впрочем, и сейчас не перестали. Примеров хватает. И Березовский, и Литвиненко, и Скрипаль ...

\iusr{Леонид Яннау}
\textbf{Галина Полякова} это тайные операции, это другое.

\iusr{Галина Полякова}
\textbf{Леонид Яннау} И снова я с Вами согласна. Но результат тот же.
\end{itemize} % }

\end{itemize} % }

\iusr{Татьяна Горовенко}

Спасибо за публикацию! Век живи и век учись. Никогда ничего подобного не
слышала. Мои \enquote{западенские} знакомые передают впечатления старых родственников,
что за \enquote{австрияків жилось краще, ніж за Польщі} не говоря уже про \enquote{советов}.
Насколько важна знаковость личности а истории!!! Как нам в Украине не хватало
грамотных, интеллигентных, харизматичных лидеров! Ещё раз спасибо за
информацию.

\begin{itemize} % {
\iusr{Галина Полякова}
\textbf{Татьяна Горовенко} 

Я родилась в Крыму, а выросла в Западной Украине. Действительно, старожилы
считали, что при Австро-Венгрии жилось намного лучше, нежели при поляках и
\enquote{советах}. Факты неоспоримы.

\end{itemize} % }

\iusr{Александр Войтов}
Спасибо, очень интересно.

\iusr{Алла Парадня}

Любая выдающаяся личность в истории со временем мифологизируется и окутывается
легендами, но от этого не теряет своего значения и всё равно заслуживает нашего
внимания и памяти. Что до некоторых уточнений, нелегко было и тогда
разобраться, и сейчас, можно интерпретировать по-разному. Но человек был,
человек творил - помянем его добром.

\iusr{Татьяна Каменева}
Какая потрясающая история жизни!

\iusr{Арт Юрковская}

Ясно одно - никакой самостийной Украины не планировалось, даже
поляками. Основной целью руководства Польши во главе с Юзефом Пилсудским было
восстановление Польши в исторических границах Речи Посполитой 1772 года, с
установлением контроля над Белоруссией, Украиной, Литвой и геополитическим
доминированием в Восточной Европе. На Парижской мирной конференции, несмотря на
заявления украинских сил из разных стран, Великобритания, Франция и США
образовать Украину не разрешили. Образовали Польшу, Австрию, Венгрию, Турцию и
проч. страны. В дурнях остались украинцы и курды. Почему? Потому что Галицию и
Волынь прибрать хотели другие, а Малороссия была \enquote{слишком русская} и большая, а
давать кучке нищих политических \enquote{украинцев} в управление огромную страну
посреди Европы - это было не в интересах держав-победительниц Первой мировой
войны. Тем более народ бредил помещичьей землей, которую обещали и дали
большевики. Вы сами подумайте своей головой! Тем более украинские рабочие и
крестьяне знать не хотели никаких автрийских Габсбургов. С какой радости
спрашивается???

\begin{itemize} % {
\iusr{Ірина Дебкалюк}

Вы прям как путлер все объясняете! Вы его пресс-секретарь? Особенно вражає:
«украинцы» в кавычках! Не любите вы свою Батьківщину!

\iusr{Михайло Наместник}
\textbf{Ірина Дебкалюк} в каждой киевской группе есть свои городские сумасшедшие.
\end{itemize} % }


\end{itemize} % }
