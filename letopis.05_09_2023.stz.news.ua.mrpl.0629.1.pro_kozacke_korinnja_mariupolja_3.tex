% vim: keymap=russian-jcukenwin
%%beginhead 
 
%%file 05_09_2023.stz.news.ua.mrpl.0629.1.pro_kozacke_korinnja_mariupolja_3
%%parent 05_09_2023
 
%%url https://www.0629.com.ua/news/3655464/pro-kozacke-korinna-mariupola-castina-3-akim-bulo-selise-kalmiuska-palanka
 
%%author_id korobka_julia.mariupol,korobka_vadim.mariupol,news.ua.mrpl.0629
%%date 
 
%%tags 
%%title Про козацьке коріння Маріуполя. Частина 3. Яким було селище Кальміуська паланка
 
%%endhead 
 
\subsection{Про козацьке коріння Маріуполя. Частина 3. Яким було селище Кальміуська паланка}
\label{sec:05_09_2023.stz.news.ua.mrpl.0629.1.pro_kozacke_korinnja_mariupolja_3}
 
\Purl{https://www.0629.com.ua/news/3655464/pro-kozacke-korinna-mariupola-castina-3-akim-bulo-selise-kalmiuska-palanka}
\ifcmt
 author_begin
   author_id korobka_julia.mariupol,korobka_vadim.mariupol,news.ua.mrpl.0629
 author_end
\fi

\begin{quote}
\em
Про козацьке коріння Маріуполя. Друга частина статті доцентів кафедри історії
та археології Маріупольського державного університету Вадима Коробки та Юлії
Коробки, які наводять доводи та факти щодо справжньої дати заснування
Маріуполя. 

Перша частина тексту за \href{https://www.0629.com.ua/news/3653256/pro-kozacke-korinna-mariupola-castina-1-davni-sprobi-osagnuti-cinnik-ukrainskogo-kozactva-v-istorii-mariupola}{посиланням}. Друга частина тексту \href{https://www.0629.com.ua/news/3654781/pro-kozacke-korinna-mariupola-castina-2-novi-sprobi-privernuti-uvagu-do-dati-zasnuvanna-mariupola-ta-dii-komandi-bojcenka-na-comu-napramku}{тут}.
\end{quote}

Якщо вести мову про поселення як місце осілого життя людей на теренах, де виник
Маріуполь, то завдяки зараз відомому картографічному джерелу, можна дійти
помилкового висновку, що таке існувало, ймовірно, в XVІІ ст. Про це свідчать
залишки городища, що фіксувались вже як руїни на карті Пітера Бергмана,
шведського картографа на московській службі, яку він уклав 1702 р.  До 1696 р.
терени Надазов'я були під неподільним пануванням Османської імперії та
Кримського ханату, тому неможливо уявити наявність у цей час тут одного з
військово-адміністративних центрів Запорозької Січі. Це дає підстави дійти
висновку, що згадане  городище, найімовірніше, належало кочовикам (нестабільній
спільноті ногайців) і не мало стосунку до козацької паланки та Маріуполя, не
вплинуло на їх виникнення ані цивілізаційно, ані адміністративно, припинило
своє існування задовго до заснування паланки та міста і не мало продовження. 

Історичні джерела підтверджують наявність постійного прагнення в запорозького
козацтва до ведення господарської діяльності на узбережжі Азовського моря та
Кальміуса в ХVII cт. Утім, в цей час азовське узбережжя знаходилось під владою
Османської імперії та її васала Кримського ханату. 

У XVII – ХVІІІ ст., за  відсутності воєн між Османською імперією та Московською
державою (Російською імперією), на узбережжя Азовського моря рушали ватаги
запорожців-рибарів для відповідного промислу. Розпорошені узбережжям
ватажани-промис\hyp{}ловці стабільного поселення не створили. Щонайменше дослідникам
не відомі джерела, які відображають їх існування, зносини промисловців із
Кошем. Водночас, не можна уявити функціонування козацького поселення або
\enquote{сторожового посту} військово-адмініст\hyp{}ративного значення, який би належав
підданим Речі Посполитої чи Московської держави на теренах, що контролювала
Осман\hyp{}ська імперія. З історичних джерел відомі спроби запорожців господарювати
на кальміуських та Азовських берегах, що закінчувались плачевно – потраплянням
у полон. Його наслідком могло стати перетворення бранців на живий товар, продаж
невільників у рабство. Правда, відомі певні форми мирного співжиття християн та
кочовиків мусульман під юрисдикцією Кримського ханства в межах так званої
Ханської України на північних теренах Буго-Дністровського межиріччя у XVIII ст.
Утім, поясненням цьому є пріоритет здобуття політичних та економічних вигід
татарським керівництвом.

\ii{05_09_2023.stz.news.ua.mrpl.0629.1.pro_kozacke_korinnja_mariupolja_3.pic.1}

Спираючись на чималу джерельну базу, зараз можна тезово описати українське
козацьке селище, яке підготувало ґрунт для розвитку Маріуполя в наступні часи.
Населений пункт запорожців – Кальміуська паланка (саме така назва) або Кальміус
– був одним із кількох окружних адміністративно-військових центрів Вольностей
Війська Запорозького низового часів Нової Січі (1734 – 1775 рр.). Тут був
осередок заселення й господарського освоєння Надазов'я (переважно риболовецький
промисел), велась торгівля. Стабільну Кальміуську паланку було засновано 1746
р.  До нашого часу зберігся документ (донесення київського генерал-губернатора
в Правительствуючий Сенат), в якому повідомлялося, що запорозькі козаки
внаслідок каральної операції, організованої старшиною донського козацтва,
опинилися на р. Кальміус і \enquote{ныне стоят поланкою тамо}.

\ii{05_09_2023.stz.news.ua.mrpl.0629.1.pro_kozacke_korinnja_mariupolja_3.pic.2}

Можливість господарського опанування Надазов'я з'явилася завдяки послабленню
конфронтації між імперіями, козацтва з тюркомовними сусідами. Селище
Кальміуська паланка було влаштоване на бар'єрній (нейтральній, буферній,
нічийній) території між Російською та Османською імперіями. Бар'єр простягався
в Надазов'ї південіше прямої лінії, яку можна умовно провести від місця
впадіння ріки Каратиш у Берду. Такий спосіб розмежування було визначено мирними
трактатами між Московською державою (Російською імперією)  1700 – 1742 рр., які
укладались як форма юридичного припинення стану війни між державами, та
договорами, що впорядковували міждержавні кордони.

\ii{05_09_2023.stz.news.ua.mrpl.0629.1.pro_kozacke_korinnja_mariupolja_3.pic.3}

Кальміуська паланка мала символіку, що відобразилася на відбитках відповідних
печаток на документах службового листування. У її центрі схрещення козацької
шаблі зі стрілою. По боках цього зображення у два рядки розташовувалася
абревіатура \enquote{ППКП} (Полкова печатка Кальміуської паланки). 

\ii{05_09_2023.stz.news.ua.mrpl.0629.1.pro_kozacke_korinnja_mariupolja_3.pic.4}

На чолі паланки стояла старшина, яку комплектували під час виборів у
Запорозькій Січі. Вона призначалася Кошем із числа заслужених козаків; до її
складу входили полковник, писар, осавул, підосавул. Полковник, ймовірно, мав
клейнод (символ влади) – пернач. Населення паланки – козаки та наймити (люди,
які виконували роботу за наймом) – були підданими Російської імперії.

У зоні адміністративних обов'язків кальміуського полковника знаходилась велика
територія між річками Вовча і Кальміус та Азовським морем. У розпорядженнях
Коша, адресованих кальміуському полковнику як регіональному керівнику, цей
простір частіше за все іменувався \enquote{ваше відомство} або відомство паланки. В
сучасних історичних творах та науково-популярній літературі паланкою у
більшості випадків називають регіон (округу), на який розповсюджувалась
адміністративна відповідальність паланочного полковника.

Українські козаки почали системне, безперервне господарське освоєння теренів,
де виріс Маріуполь, майже за три десятка років до їх включення до складу
Російської імперії. В паланці та на території, що підлягала її
відповідальності, не існувало кріпосного права, як і на всій території
Вольностей Війська Запорозького Низового, поширеною була індивідуальна праця
або вільний найм.

Вогнищами виробничого життя на території Кальміуської паланочної округи, як в
усьому Запорожжі, були зимівники – відокремлені господарства (за фермерським
типом), що влаштовувались разом із житлом власника та його наймитів. За
відомостями Івана Синяка 1768 р., перед російсько-турецькою війною, на теренах
Кальміуської округи нараховувалось 73 зимівники.

\ii{05_09_2023.stz.news.ua.mrpl.0629.1.pro_kozacke_korinnja_mariupolja_3.pic.5}

За відомостями Дмитра Вортмана, запорозька термінологія відрізняла зимівник від
хутора: перший тип поселення вільно засновували неодружені козаки, члени
січової громади, другий – \enquote{військові піддані} (одружені козаки та посполиті) з
дозволу Коша Запорозької Січі. У джерелах, що мають не-запорозьке походження,
обидва типи поселення можуть називатися хутором.

Важливими напрямами у господарстві зимівників були полювання, скотарство, менше
– землеробство. Найбільшими ж місцями виготовлення товарної продукції були
риболовецькі заводи, розкидані на північно- та південно-східному узбережжі
Азовського моря, наближених до найбільш сприятливих для рибальства угідь.
Вочевидь, деякі рибні заводи були у складі зимівників. Їх власники
спеціалізувались на переробці та реалізації риби. Весь процес ґрунтувався на
приватній власності на засоби виробництва, особистій праці власників та
найманих ними осіб, а також ринковій спрямованості. 

Кальміуський полковник регулював розподіл рибальських угідь. Відомий випадок
втручання в цей процес кошового отамана. Успіхи козацького рибальства в межах
Кальміуської паланочної округи заклали безперервну до нашого часу традицію
освоєння біологічних ресурсів басейну Азовського моря. Товарна спрямованість
рибальський промислу сприяла встановленню господарських зв'язків, інтересу
чумаків та інших торговців до нашої місцевості, включення її в
загальноімперський ринок.

Одним із найбільш обізнаних  у справах Кальміуської паланки та її відомства був
славнозвісний Петро Калнишевський, кошовий отаман (найвищий виборний керівник у
Запорозькій Січі), який обіймав цю посаду більше 10 років. До нашого часу
збереглись документи службового листування кальміуських полковників із
Калнишевським. 

Сучасні українські історики Володимир Полторак та Іван Синяк окремо один від
одного склали списки кальміуської паланочної старшини. В цих письмових
переліках близько 20 полковників, до 10 писарів та підписарів, 5 осавулів та
підосавулів. Найбільш відомі представники кальміуської паланочної старшини –
полковники: Василь Леонтьєв, Андрій Чорний (1746 р.), Андрій Порохня (1754 р.),
Василь Маґро (1756 р.), Андрій Вербицький (1758), Петро Велегура (1772 –1774
рр.); писарі – Данило Малиновський (1754 р.) та Григорій Швидкий (1761 р.).

\ii{05_09_2023.stz.news.ua.mrpl.0629.1.pro_kozacke_korinnja_mariupolja_3.pic.6}

При Кальміусі знаходилась залога (команда) із запорожців. На кальміуську
старшину покладалось завдання організації постійного прикордонного
спостереження за військовими приготуваннями ханату, розшуково-каральних заходів
стосовно розбійництва, підтримання порядку на підвідомчій території, збирання
податків. Паланкова старшина фіксувала усілякі випадки нападів злодіїв, татар
та ногайців, а також донського козацтва на козаків-промисловиків та торговців,
що мали приїхати до Кальміусу чи виїхали з нього. Відомий випадок у Кальміусі
козацького суду за звичаєвим правом. 

У 1768 році на теренах селища Кальміуська паланка одне з картографічних джерел
зафіксувало резиденцію полковника, курені – житла козаків, конюшню, кузню,
базар та навіть школу, де \enquote{...обучаются музыкѣ и пѣть} та деякі інші об'єкти.

З початком російсько-турецької війни 1768 – 1774 рр. команда кальміуського
полковника виступила в похід на з'єднання з Військом Запорозьким Низовим, що
мало входити до складу Другої російської армії і діяти в межах наказів генерала
Петра Рум'янцева. Водночас, за правилами тактики випаленої землі, яка
передбачала знищення всього того, що може слугувати життєзабезпеченню ворожого
війська, на вимогу Коша селище Кальміуську паланку було спалено. Церковні
цінності та цивільне населення підлягали евакуації разом з усією худобою і
майном у більш безпечне місце. Як висловився російський генерал, князь
Олександр Прозоровський, характеризуючи цю акцію, \enquote{...дабы их люди, живущие в
разных местах по их зимовникам не сделались неприятелю жертвою}. Правда, нам
точно не відомо, чи було виконано останній зазначений тут припис кошового
керівництва. Ймовірно зимовчани злегковажили їм.

За відомостями А. Скальковського у ході кампанії 1769 р. татарська орда, яка
здійснила прорив форпостної лінії Другої армії, винищила команду кальміуського
полковника та скоїла жахливі спустошення у відомствах Самарської, Орільської та
Протовчанської паланок. Цей сумний факт певним чином затьмарював у цілому
успішну участь запорожців у війні.  У викладі маріупольських офіційно визнаних
краєзнавців реальна подія, розказана А. Скальковським, набула легендарного
вигляду: начебто татари зруйнували сторожове укріплення Кальміус, а увесь загін
запорозьких козаків на чолі з полковником загинув у нерівній боротьбі з
ворогом. 

1774 р. із завершенням  російсько-турецької війни (1768 – 1774 рр.), за умовами
Кючук-Кайнарджійського мирного договору між імперіями, Росія отримала частину
Чорноморського узбережжя, південну частину міжріччя Дніпра та Південного Бугу,
а Кримський ханат було проголошено вільним від османської залежності, й він
потрапив під протекторат Росії. Водночас, бар'єрні землі в Надазов'ї та на
Північному Кавказі були визнані власністю Росії. Так, Надазов'я, де
розташовувалось селище Кальміуська паланка (Кальміус), а згодом було
задекларовано місто Маріуполь увійшло до складу Російської імперії.

Селище Кальміуська паланка, за свідченням унікального комплексу джерел,
утвореного внаслідок діяльності Коша Нової Запорозької Січі та військової
канцелярії Війська Запорозького Низового, відзначався яскраво вираженими
ознаками складних адміністративних функцій. Виняткове значення поселення
засвідчується низкою виключних фактів, на яких до нашого часу не
зосереджувалась увага дослідників. 

По-перше, в Кальміусі існувала перша в нашому краї православна парафія,
справами якої опікувався, з-поміж інших,  і Митрополит Київський і Галицький.
Центр парафії – дерев'яна церква (оновлена 1754 р.) на честь св. Миколая. 1767
р. церковне начиння було перевезене в с. Кам'янка навпроти Нового Кодака (нині
на тер. м. Дніпро). Новий храм було поставлено, ймовірно, після повернення
козаків з війни 1768 – 1774 рр. Його парафіяни рахували себе правонаступниками
попереднього та власниками вивезеного начиння.

\ii{05_09_2023.stz.news.ua.mrpl.0629.1.pro_kozacke_korinnja_mariupolja_3.pic.7}

По-друге, відомо декілька справ, пов'язаних із Кальміусом, що розглядалися
найвищим державним органом Російської імперії – Правительствуючим Сенатом. 

По-третє, Кальміус неодноразово потрапляв у поле зору Кирила Розумовського,
останнього гетьмана України та київського генерал-губернатора.

\ii{05_09_2023.stz.news.ua.mrpl.0629.1.pro_kozacke_korinnja_mariupolja_3.pic.8}

У зв'язку із ліквідацією імперським урядом Запорозької Січі (1775 р.) нищився й
оригінальний устрій Війська Запорозького Низового, його землі перетворились на
здобич петербурзької влади. Вони увійшли до складу Новоросійської та Азовської
губерній. Водночас, російські урядовці проектували перетворення територіальних
відомств паланок на повіти. У випадку із підвідомчою Кальміуській паланці
територією було запроектовано Кальміуський повіт у складі Азовської губернії.
Імперські адміністратори здійснили переписи населення, загалом, та колишніх
запорожців, зокрема. Вони доводять, що селище Кальміус не припинило свого
існування. Так, за відомістю, підписаною ймовірно 1775 р. генерал-поручиком П.
Текелієм, у Кальміусі на той час мешкали один козацький старшина, 311 – козаків
та 23 – робітника. 

При спробі створення імперської адміністративно-територіальної одиниці –
Кальміуського повіту – його першим начальником було призначено запорозького
старшину Якова (?) Бершадського. Найбільш імовірним місцем новопроголошеного
повітового адміністративного центру був саме Кальміус.

Ще один перепис зафіксував імена та прізвища більшості козаків новостворених
повітів, з-поміж інших, й Кальміуського. Цей облік відображено у \enquote{Списку
Кальмиусского уезда казакам...}, підписаному 13 січня 1776 р. земським
комісаром Петром Горлинським. Перелік складався з 247 осіб. У ньому були
зареєстровані місця баталій, де брали участь всі зазначені запорожці, у війні
1768 – 1774 рр. (здебільшого під Очаковом), а також перебування в полоні та
поранення. Водночас цей перелік містив висновок про гідність колишніх
запорожців отримувати платню та провіант за \enquote{службы свои в минувшую войну}.

Таким було селище Кальміуська паланка (Кальміус) та підвідомча його старшині
округа від виникнення до ліквідації Запорозької Січі та деякий час після її
знищення.

ЧИТАЙТЕ ПОЧАТОК дослідження:

\href{https://www.0629.com.ua/news/3653256/pro-kozacke-korinna-mariupola-castina-1-davni-sprobi-osagnuti-cinnik-ukrainskogo-kozactva-v-istorii-mariupola}{Про козацьке коріння Маріуполя. Частина 1. Давні спроби осягнути чинник українського козацтва в історії Маріуполя}

\href{https://www.0629.com.ua/news/3654781/pro-kozacke-korinna-mariupola-castina-2-novi-sprobi-privernuti-uvagu-do-dati-zasnuvanna-mariupola-ta-dii-komandi-bojcenka-na-comu-napramku}{Про козацьке коріння Маріуполя. Частина 2. Нові спроби привернути увагу до дати заснування Маріуполя та дії команди Бойченка на цьому напрямку}
