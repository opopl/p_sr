% vim: keymap=russian-jcukenwin
%%beginhead 
 
%%file 07_09_2020.sites.ru.zen_yandex.yz.11_eku.1.god_1220_800_let_nazad
%%parent 07_09_2020
 
%%url https://zen.yandex.ru/media/11ecu/eto-byl-1220i-ili-800-let-tomu-nazad-5f4b516711a7d1138b0d7c80
 
%%author 
%%author_id yz.11_eku
%%author_url 
 
%%tags god_1220
%%title Это был 1220-й. Или 800 лет тому назад.
 
%%endhead 
 
\subsection{Это был 1220-й. Или 800 лет тому назад.}
\label{sec:07_09_2020.sites.ru.zen_yandex.yz.11_eku.1.god_1220_800_let_nazad}
\Purl{https://zen.yandex.ru/media/11ecu/eto-byl-1220i-ili-800-let-tomu-nazad-5f4b516711a7d1138b0d7c80}
\ifcmt
	author_begin
   author_id yz.11_eku
	author_end
\fi

Ну что, уважаемые господа/товарищи. Вот мы и в году 1220-м, а там ух как
несладко. Даже наш 2020-й кажется мирным тихим временем.

Время Чингисхана..И от этих слов задрожала Азия. Европа пока еще не дрожала,
подобно тем людям, кто считает, что "не было никаких монголов, а потом немцы
всё придумали, лишь бы опорочить нашу славную многотысячелетнуюю историю".

\ifcmt
pic https://avatars.mds.yandex.net/get-zen_doc/3518430/pub_5f4b516711a7d1138b0d7c80_5f4b6930157070066dc342c8/scale_1200
\fi

Время Чингисхана. Монголы успешно продвигаются на Запад. В 1220-м году взят
Самарканд. Вообще города Средней Азии падают перед завоевателем один за другим.
Отрар, Бухара, Ходжент, Мерв, Бенакент и многие другие. Всё это достижения
монголов, которых не было, только в этом году. Чингисхан основывает свою
столицу, город Каракорум. Это недалеко от современного монгольского города
Хархорин, в центральной части страны.

Впереди Западный Поход. Битва при Калке произойдет уже всего через 3 года.

У Киевского князя тогдашнего Мстислава Романовича годы жизни как раз
оканчиваются 1223-м годом, он сдастся в плен монголам после той самой битвы и
будет убит.

Мстислав Черниговский в 1220-м году успешно отражает набеги литовцев. Он тоже
будет убит в Битве при Калке.

Многим другим князьям также оставалось всего 3 года до ратной смерти:
Туровскому, Несвижскому, Шумскому, Путивльскому....многим.

\ifcmt
pic https://avatars.mds.yandex.net/get-zen_doc/3443049/pub_5f4b516711a7d1138b0d7c80_5f4b69b524f6974cac97e0e0/scale_1200
\fi

Великий князь Владимирский Юрий Всеволодович, в тот год воюет еще с волжскими
булгарами. Этот князь будет убит монголами уже 1238-м в Ситской битве.

Не особо дули в ус и западнее. Будущий известнейший венегрский король Бела IV,
тогда еще 14-летний юноша получает в управление несколько регионов от отца.
Монголы придут, когда Бела будет уже королем. Он ничего не сможет
противопоставить этой силе, едва спасет свою жизнь бегством.

В Священной Римской империи в 1220-м становится император Фридрих II, которому
и представится увидеть вторжение монголов. Этот человек, зная об угрозе востока
пытался использовать ее в политических центрах, набирая очков перед Папой
Римским Гонорием III, который в свою очередь поначалу очень возлагал надежды
относительно нового Крестового похода на нового императора.

\ifcmt
pic https://avatars.mds.yandex.net/get-zen_doc/3947075/pub_5f4b516711a7d1138b0d7c80_5f4b69e0f464170bc2ce3315/scale_1200
caption Это вот он. Знаток птиц. При чем тут птицы? Читайте дальше
\fi

Этому Фридриху II писал Батый, требуя подчинения, тот же отвечал, что неплохо
разбирается в соколиной охоте и мог заведовать у монголов птицами, коль так уж
дело встало. Видимо, настолько не воспринимал всерьез угрозу. И ему повезло,
монголы нещадно потрепали венгров, провели несколько вылазок по направлению к
Вене, но потом всё же решили развернуться. Уж слишком далеко они зашли, и там
где хорошо укрепленные каменные замки, где и степей то нет. Сложно воевать
широкой коннице.

Ну а пока всё тот же 1220-й. Поднадоели монголы-интервенты? Давайте о другом.

В то время существует Латинская империя с центром в Константинополе, она
образовалась после того как во время IV Крестового похода бравые освободители
Гроба Господня решили вместо Святой Земли завоевать христианское государство,
Византийскую империю. Ещё в 1219 году императором стал там Роберт де Куртене,
но он был во Франции на тот момент, и в 1220-м еще не добрался до места
правления, еще ждала страна монарха. Он объявится для исполнения своих
обязанностей лишь в 1221 году. Ненадолго. Через 6 лет он убежит от своих
подданных после того как они из мести убъют его тещу, а жену изрядно изуродуют.
Ну как бы, намекали они ему так, что политически неверно он женился.

\ifcmt
pic https://avatars.mds.yandex.net/get-zen_doc/1936915/pub_5f4b516711a7d1138b0d7c80_5f4b6a44d3217e243e65633d/scale_1200
caption Взятие крестоносцами Константинополя
\fi

Может лучше к монголам привычным вернемся? Ладно ...ладно.. продолжим
безмонгольский мир смотреть.

В Японии эпоха Камакура, там исповедуют дзэн, но страну сотрясает междоусобица
и внутренние конфликты. Ну а с кем еще на острове конфликтовать, как не самим с
собой. Особенно хорошо чувствуют себя тогда самураи. Они самые уважаемые и
великие люди в те годы.

В Южной Америке живет и благополучно царствует самый первый Инка, его звали
Манко Капак. Его родителями были боги Солнца и Луны, что очень логично, ведь от
этого человека пошла целая династия правителей империи Инков. 

\ifcmt
pic https://avatars.mds.yandex.net/get-zen_doc/3965742/pub_5f4b516711a7d1138b0d7c80_5f4b6a7b25c0b248d302d0af/scale_1200
caption Это вот Манко Капак.
\fi

Такое вот сказочное время. Такой вот 1220-й. Но сложно говорить об этом периоде
без оглядки на тучи с востока наступающие. Но монгольского нашествия, как
известно (знатокам настоящей правды) никакого не было. Поэтому мы и
останавливаем свое повествование о том периоде.

Но предлагаем Вам наши рассказы про другие 20-е года.

\begin{itemize}
  \item 20 год
  \item 1000 год
  \item 1120 год
  \item 1420 год
  \item 1520 год
  \item 1620 год
  \item 1720 год
  \item 1820 год
  \item 1920 год
\end{itemize}
