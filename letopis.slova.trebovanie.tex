% vim: keymap=russian-jcukenwin
%%beginhead 
 
%%file slova.trebovanie
%%parent slova
 
%%url 
 
%%author 
%%author_id 
%%author_url 
 
%%tags 
%%title 
 
%%endhead 
\chapter{Требование}

%%%cit
%%%cit_head
%%%cit_pic
%%%cit_text
Нет, неправда, мы \emph{требовали} «ничего без нас о нас». Помните же эту
фразу? Это справедливости ради. Я должна сказать. Это несерьезно просто. Ну
что значит \emph{требовали}? Мы \emph{требовали}, как кто? Ну кто обращает
внимание на \emph{требование} какого-то раба? Никто не обращает.
\emph{Требовать} – это когда ваше слово, \emph{требование} имеет вес. А когда
вы никто и звать вас никак, и вы на мусорнике истории и не можете с собственным
ресентиментом разобраться и собственной коррупцией, стяжательством и грабежом,
кто нас будет обращать внимание? Нам же прямым текстом сказали: «Ваша элита
вороватая, коррупционная, стяжательская не имеет права присутствовать за столом
переговоров господ». Нам прямым текстом об этом сказали. Чего еще нужно? Мало
ли что мы \emph{потребовали}
%%%cit_comment
%%%cit_title
\citTitle{Сергей Дацюк: Украина сегодня - не просто попрошайка, она на мусорнике истории}, 
Сергей Дацюк; Людмила Немыря, hvylya.net, 28.06.2021
%%%endcit

