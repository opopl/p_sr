% vim: keymap=russian-jcukenwin
%%beginhead 
 
%%file 31_01_2022.fb.baumejster_andrej.kiev.filosof.1.progress.cmt
%%parent 31_01_2022.fb.baumejster_andrej.kiev.filosof.1.progress
 
%%url 
 
%%author_id 
%%date 
 
%%tags 
%%title 
 
%%endhead 
\zzSecCmt

\begin{itemize} % {
\iusr{Maria Lypych}
Щирі вітання з Днем народження, дорогий Андрій Баумейстер!  @igg{fbicon.party.popper}   @igg{fbicon.sparkles} 

\ifcmt
  ig https://scontent-frt3-1.xx.fbcdn.net/v/t39.30808-6/273169653_4811276175588492_8471609935175694598_n.jpg?_nc_cat=107&ccb=1-5&_nc_sid=dbeb18&_nc_ohc=G3-idN8pmwsAX8HDjoG&_nc_ht=scontent-frt3-1.xx&oh=00_AT-OT5mih6soGoTOEC5s0cMXcCQry0YXIw8t4VOVRIAr2w&oe=61FDCFD7
  @width 0.3
\fi


\iusr{Андрей Баумейстер}
\textbf{Maria Lypych} Марiя, дуже вдячний за вiтання!!

\iusr{Андрей Мельник}

Сейчас закончил читать Ивана Франка \enquote{Що таке поступ?}. Мне кажется, он
поднимает очень похожий вопрос и дает ответ, что прогресс - это скачкообразное
явление, которое совершенно не линейно. В чем-то прогресс, а в чем-то регресс.

Его ответ в моей интерпретации, что важно реализовывать развитие через любовь в
совместной деятельности - созиданию и учитывать уникальности каждого в этом
движении. Развитие как ээзадача каждого.

И И. Франко возвращая нас к Гете, показывает, что материальный и духовный голод
являются источниками нашего развития.

Очень вовремя Ваша запись!

\begin{itemize} % {
\iusr{Andriy Kryvtsun}
\textbf{Andrii Melnyk} развитие - это разве не прогресс?


\iusr{Андрей Баумейстер}
\textbf{Andriy Kryvtsun} 

мировые религии обходились без идеи прогресса. Христианство, ислам, буддизм.
Древний Египет никуда не прогрессировал. Развитие и прогресс не тождественные
понятия. В следующей беседе на эту тему постараюсь данный тезис обосновать

\iusr{Andriy Kryvtsun}
\textbf{Andrii Baumeister} 

вот интересно как раз про разницу между прогрессом и развитием. Интуитивно
кажется, что это одно и то же. Т.е. мы говорим прогресс и имеем ввиду движение
вперед, т.е. развитие.


\iusr{Андрей Мельник}
\textbf{Andriy Kryvtsun}

РОЗВІЙ (розвиток)

1. Розмотування навитого на що-небудь матеріалу.

2. Що-небудь розмотане з чогось.

Мои вопросы на которые я отвечаю часто в разрезе бизнеса:

Как отличить:

\begin{itemize} \item 1. Изменение; \item 2. Рост; \item 3. Развитие; \item 4.
Трансформацию; \item 5. Преобразование.  \end{itemize}

Что работает с внешней формой, а что с сутью.

На мой взгляд развитие идет от цело с сутью и не меняет сути (сущности), а
развивает отдельные частности целого.

Преобразование - это изменение сути. То есть основного смысла целого и вокруг
целого выращивание уже новой внешней формы.

Где тут встроить прогресс я пока не плеимаю. Как я понимаю, прогресс - это
всеобщее развитие всех существующих идей (сущностей, объектов, систем и т.п.)
одновременно. И мы видим, что в чем-то развитие, в чем-то деградация. Мало в
чем преобразование.

\iusr{Сергей Черненко}

Очень странно отрицать прогресс как идею, как таковой. Прогресс - это идея
пошагового движения от маленького к большому. Сам по себе прогресс - это
математическое понятие (пошаговое движение вперёд к повышению).

Регресс - пошаговое движение к понижению.

Вы же не отрицаете идею сложения и вычитания. Или идею производной или
интеграл.

Возможно под словом прогресс вы понимаете какой-то особый подвид прогресса,
например социальный прогресс, или социокультурный прогресс или общественный
прогресс, или потребительский прогресс.

Прогресс может быть в чтении сложного текста. Разве это плохая идея?

Прогресс может быть у раковой опухоли. Вряд ли это хорошо для
человека-носителя, но с помощью этой идеи можно выполнять математические
расчеты.

То, что прогресс применяют к счастью или к благосостоянию наций - это не значит
что математика плохая и нужно от нее отказаться.

Скорее это значит, что человечество пока не придумало ничего получше.

\iusr{Андрей Баумейстер}
\textbf{Сергей Черненко} 

Сергей, прежде чем употреблять слово \enquote{странно}, посмотрите словари. Это
во-первых. А во-вторых, слово и идея часто не одно и то же. И мы пользуемся
словами, которые изначально имели другие значения. Например, пафос, апатия,
логос, экономика и тому подобное.

\iusr{Андрей Баумейстер}
\textbf{Сергей Черненко} 

я отрицаю не латинское слово с узким значением, а концепцию, авторы которой
присвоили это слово и создали, заметьте, только в 18 веке, определённую
доктрину. А слово, разумеется, существовало за много столетий до этой доктрины.

\iusr{Vyacheslav Artiukh}
\textbf{Andrii Melnyk} у зв'язку з гете франко трохи помилився. там повинен бути шиллер
\end{itemize} % }

\iusr{Евгений Деменок}
Прекрасное видео! С днём рождения!!!

\iusr{Людмила Куклева}
Цивилизация непроизрастает из варварства, Врварство произрастает из цивилизации

\iusr{Александр Воструев}

Все хотят лучше, прогрессировать, но, очень мало кто понимает, что главное,
чтобы хуже не становилось...(((

\iusr{Александр Ладик}

Прогрес, Поступ - це крокування вперед і вгору, це Розвиток, справжній, що не
викликає сумнівів. Я за такий справжній Прогрес (іншого прогресу для мене на
існує!), бо задача людей — за Законами Божої Воля розвиватися в Поступі, в
якому дух спрямовує розум і розширює межі пізнання і стверджує світову гармонію
разом із суспільним благом.

\iusr{Амина Кхелуфи}
\textbf{Андрій Олегович}, вітаю з Днем народження! Відкриттів, наснаги та нових вражень!

\iusr{Alexander Levine}

На мой взгляд, аргументация обоснования заявленного тезиса как 'отсутствие идеи
прогресса' при всём моём огромнейшим уважении к автору не совсем получила своё
логическое обоснование, которое в итоге выразилось в виде обилия большого
количества интересной и важной информации. Возможно было бы нужным сделать
акцент на специфике регресса современной цивилизации, нежели делая акцент на
навязанный 'мнимый прогресс'.

Было не совсем понятный динамика развития аргументации, потому как уже в
исходном материале беседы находятся совершенно очевидные для всех понятия о
том, что например в культуре прогресса быть не может, а в авиастроении вполне и
даже очень.

\iusr{Андрій Шиманович}

Прийміть щирі вітання із Днем народження і величезну подяку за всю ту
неймовірно важливу роботу, яку Ви здійснюєте.

\iusr{Иван Порало}

Несмотря на то, что прогрессом называли технологический прорыв, новое
достижение в материалопроизводстве, сегодня это дефиниция восходящее развитие,
как удобная формулировка.

На мой взгляд осмысление следует начинать от обратного: для чего нужен
прогресс? Имея ответ, можно анализировать качественные стороны восходящего
развития

\iusr{Александр Анохин}
Благодарю.

\iusr{Энна Кот}
С Днём рождения! Ура! Наши взгляды совпадают!

\iusr{Ilona Volkova}
Сердечно вітаю Вас з Днем народження! Дякую Вам за бесіди, від них на серці стає тепліше та навкруги світліше 
@igg{fbicon.heart.red}

\iusr{Виталий Сидоркин}
Уже не хватает продолжения

\end{itemize} % }

\subsubsection{Комментарии к youtube-видео}
\label{sec:31_01_2022.fb.baumejster_andrej.kiev.filosof.1.progress.cmt}

\begin{itemize} % {
\iusr{Tician Delvig}

Благодарю вас Андрей Олегович! Жаль, что люди до сих пор апеллируют к тому, что
\enquote{Мы же перестали есть друг друга, произошел же значит какой-то прогресс}. Лично
мне представляется, что прогресс это не когда люди перестают есть друг друга,
прогресс это когда люди даже в мыслях не допускают подобного и испытывают
естественную и здоровую брезгливость от такой дикости. Когда они осознают ужас
и варварство этого действия и скорее умрут чем пойдут на такое. Достоевский
писал, что хочет не такого общества, где бы он не мог сделать зла, а такого,
чтоб он мог делать всякое зло, но не хотел бы его делать сам... И ведь правда,
прогресс в том, что даже имея возможность совершить зло, человек его не
совершает. Как прекрасно определил свободу Руссо в своих прогулках одинокого
человека: Я никогда не думал, будто свобода человека состоит в том, чтобы
делать, что хочешь: она в том, чтобы никогда не делать того, чего не хочешь...
А Современный человек при всем уважении ко всем, при необходимости пойдет на
поступки на много страшнее каннибализма, ибо в нем нету духовного оплота, нету
благородства и принципов, нету характера и конечно нету никакого понимания
ценности жизни которое удержало бы его от этого. Осматриваясь вокруг, я вижу
как люди в наше время, отстаивают ценность человеческой жизни с оружием в
руках. Да, человек перестал приносить человеческие жертвы, но так это не
потому, что он стал более гуманным и осознанным, потому, что если ему завтра
скажут, что его страна должна вступить в войну с другой страной, ибо та
представляет угрозу, люди возненавидят представителей той страны и будут
вопить, что их надо уничтожить. Я сейчас даже не о том, что основанием
ненависти людей чаще всего служат новости из телевизора, и сомнительные факты,
а о том, что хоть человек и перестал приносить жертвы в виде человеческих
жизней, он не против этого, и до сих пор, с огромным удовольствием приносит
жертвы в виде добродетелей своему многоликому Богу невежества. 

\end{itemize} % }
