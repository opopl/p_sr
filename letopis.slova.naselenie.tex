% vim: keymap=russian-jcukenwin
%%beginhead 
 
%%file slova.naselenie
%%parent slova
 
%%url 
 
%%author 
%%author_id 
%%author_url 
 
%%tags 
%%title 
 
%%endhead 
\chapter{Население}

%%%cit
%%%cit_head
%%%cit_pic
%%%cit_text
Зацитируем этот небольшой, но очень показательный документ из восьми пунктов.
Украинцы в нем определяются эпитетом \enquote{так называемые}. 
"Многоуважаемый партейгеноссе Розенберг!
По поручению фюрера я довожу до Вашего сведения его пожелание, чтобы Вы
соблюдали и проводили в жизнь в политике на оккупированных восточных
территориях следующие принципы.
\begin{itemize}
\item 1. Мы можем быть только заинтересованы в том, чтобы сокращать прирост \emph{населения}
оккупированных восточных областей путем абортов. Немецкие юристы ни в коем
случае не должны препятствовать этому. По мнению фюрера, следует разрешить на
оккупированных восточных территориях широкую торговлю предохранительными
средствами. Ибо мы нисколько не заинтересованы в том, чтобы \emph{ненемецкое
население} размножалось.
\item 2. Опасность, что население оккупированных восточных областей будет
размножаться сильнее, чем раньше, очень велика, ибо само собою понятно, его
благоустроенность пока намного лучше. Именно поэтому мы должны принять
необходимые меры против размножения ненемецкого \emph{населения}.
\item 3. Поэтому ни в коем случае не следует вводить немецкое обслуживание для
\emph{местного населения} оккупированных восточных областей. Например, ни при каких
условиях не должны производиться прививки и другие оздоровительные мероприятия
для \emph{ненемецкого населения}
\end{itemize}
%%%cit_comment
%%%cit_title
  \citTitle{22 июня - 80 лет нападения на СССР. Что немцы готовили для украинцев}, Максим Минин, strana.ua, 22.06.2021
%%%endcit
