%%beginhead 
 
%%file 20_09_2021.fb.teatralna_brama.2.zavershennja_festivalu
%%parent 20_09_2021
 
%%url https://www.facebook.com/TeatralnaBrama/posts/pfbid0gQFKRYJH5aZ58mpZLymsJigKp16DMybTPaPLX2h6JZr7pxwn88w9foWwG8Xdc1B3l
 
%%author_id teatralna_brama
%%date 20_09_2021
 
%%tags 
%%title Завершення фестивалю
 
%%endhead 

\subsection{Завершення фестивалю}
\label{sec:20_09_2021.fb.teatralna_brama.2.zavershennja_festivalu}

\Purl{https://www.facebook.com/TeatralnaBrama/posts/pfbid0gQFKRYJH5aZ58mpZLymsJigKp16DMybTPaPLX2h6JZr7pxwn88w9foWwG8Xdc1B3l}
\ifcmt
 author_begin
   author_id teatralna_brama
 author_end
\fi

🎊 Другий обласний відкритий фестиваль театрального мистец\hyp{}тва ТЕАТРАЛЬНА
БРАМА-2021 завершено!🎊

Останній фестивальний вечір розпочали з приємних урочистостей: відзначили
Подяками міські колективи і заклади, що допомогли нам створити яскраву святкову
атмосферу відкриття – це Маріупольський фаховий коледж культури і мистецтв,
Народний театр \enquote{ТЕАТРОМАНІЯ}, Театральна артіль \enquote{ДРАМКОМ}, Оркестр та
співробітники Головного управління Національної поліції в Донецькій області,
Народний естрадно-цирковий колектив \enquote{ВЕСЕЛАЯ АРЕНА} і Суспільство реставрації
середньовічних боїв!

Потужним фінальним акордом фестивалю стала поетична драма \enquote{МАРУСЯ}. По
закінченню вистави виконавців вітали тривалими оплесками, після чого відбулась
урочиста церемонія закриття. Щирі слова вдячності за професійну роботу в журі
та маленькі пам'ятні подарунки отримали від генерального директора театру,
голови оргкомітету фестивалю ВОЛОДИМИРА КОЖЕВНІКОВА наші поважні експерти:
театрознавиця, народна артистка України, лауреатка премії імені Марії
Заньковецької СВІТЛАНА ОТЧЕНАШЕНКО; магістр мистецтвознавства, театральний
критик, історик театру, стипендіат Програми Уряду Польщі імені Костянтина
Калиновського ДМИТРО ЄРМОЛОВИЧ-ДАЩИНСЬКИЙ; театрознавиця, кураторка міжнародних
і всеукраїнських театрально-мистець\hyp{}ких проєктів, старша викладачка кафедри
театрознавства ХНУМ імені І. П. Котляревського ОЛЕНА ЛИБО та художній керівник
акторсько-режисерського курсу \enquote{Майстерня 55} ХНУМ імені І.П. Котляревського
ВАЛЕРІЙ ГУЙВІН.

💜 Ось на цій теплій ноті наш фестиваль офіційно завершився, але – не
відключайтесь, незабаром оголосимо ПІДСУМКИ!!🤩

фото Лев Сандалов Фотограф%
\footnote{\url{https://www.facebook.com/sandalov.lev}}
