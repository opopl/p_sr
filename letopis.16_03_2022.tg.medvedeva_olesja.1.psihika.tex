% vim: keymap=russian-jcukenwin
%%beginhead 
 
%%file 16_03_2022.tg.medvedeva_olesja.1.psihika
%%parent 16_03_2022
 
%%url https://t.me/Medvedeva_Olesya/7819
 
%%author_id medvedeva_olesja
%%date 
 
%%tags 
%%title Очень важно сейчас сохранить свою психику
 
%%endhead 
 
\subsection{Очень важно сейчас сохранить свою психику}
\label{sec:16_03_2022.tg.medvedeva_olesja.1.psihika}
 
\Purl{https://t.me/Medvedeva_Olesya/7819}
\ifcmt
 author_begin
   author_id medvedeva_olesja
 author_end
\fi

Очень важно сейчас сохранить свою психику. Война рано или поздно закончится, а
если кукуха поедет, то вернуть ее на место крайне сложно, а в некоторых случаях
почти невозможно.

Сейчас уже десятки тысяч публикаций на тему, что вам думать, что вам делать,
что чувствовать и что не чувствовать.

Наша психика и тело очень тесно связаны. Ненависть, злость, ярость провоцирует
сердечно-сосудистые заболевания, нарушает работу головного мозга и все дальше
по списку, вытекающее. Все дело в напряжении, тревожности, кортизоле и прочих
реакциях в организме на наши эмоции.

К примеру, меня периодически встряхивают панические атаки, сопровождающиеся
тошнотой, дикой головной болью, тремором, отдышкой, головокружением. Все
потому, что я переживаю войну второй раз. Второй раз я вынуждена собирать себя
по кускам и начинать сначала.. жизнь в одном рюкзаке.

Нужен позитив. Любой. Зелёная трава и выбивающие из-под земли подснежники,
голубое ясное небо, родные рядом, любимая кошка мурлычет, любимая Музыка в
наушниках, если есть возможность, то прогулка по любимым местам.

Никто не знает как надо, каждый пытается навязать своё «как правильно». А
единого для всех варианта нет. Это война, никто нас к этому не готовил, сейчас
нужно опираться только наши личные ощущения и делать выбор, принимать решение.

Все будет хорошо. Война закончится и мы будем дома. Будем строить мосты и
возрождать нашу любимую Украину.
