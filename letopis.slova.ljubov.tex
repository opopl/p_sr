% vim: keymap=russian-jcukenwin
%%beginhead 
 
%%file slova.ljubov
%%parent slova
 
%%url 
 
%%author 
%%author_id 
%%author_url 
 
%%tags 
%%title 
 
%%endhead 
\chapter{Любовь}
\label{sec:slova.ljubov}

У, здавалося б, такій абсолютно політичній категорії як незалежність є
абсолютно чуттєва сторона – це \emph{любов}. Доброзичливість та \emph{любов}
приходить в життя звільнених від приниження та гноблення людей. Принаймні так
мало б бути.  Проте значна частина власників українських паспортів відчувають
взаємну \emph{нелюбов} та непотрібність у власній країні. Якщо бути об’єктивним
і більш точним: не люблять нас, але \emph{не любимо} і ми.  У незалежності
також є естетична сторона – це краса. Ми збудували некрасиву країну. І ми
розуміємо, що некрасиве виходить з \emph{нелюбові}. В українській мові є два
слова, котрі відображають естетичне: гарний і красивий. Перше більше
характеризує створене природою, а інше – створене людиною. Наприклад, жінка
може бути гарною, але не красивою. Країна у нас гарна, але не красива,
\citTitle{Нелюбов та некраса української незалежності}, Анатолій Якименко, analytics.hvylya.net, 10.06.2021

Про яку \emph{любов} ми говоримо. Не можна сказати, що тема \emph{любові} є
панівною в суспільному дискурсі, але заклики до \emph{любові} в Україні можна
почути часто. Вони лунають звідусіль: з церков, семінарів, тренінгів,
соціальних та політичних бордів й так далі. Бувають періоди, коли здається, що
у нас працює \emph{«любовна»} армія, яка складається з священиків різних
конфесій, коучів, менторів, фасилітаторів, лідерів громадської думки і,
звичайно, політиків. Серед останніх беззаперечним лідером тут є Юлія
Володимирівна. Згадаймо хоча б її промову на молитовному сніданку в США у
лютому 2018 року.  Отже, про що говорять всі ці люди? Безперечно, вони говорять
про \emph{безумовну любов}. Принаймні намагаються говорити саме про неї. І з
цього місця уже починаються проблеми, 
\citTitle{Нелюбов та некраса української незалежності}, Анатолій Якименко, analytics.hvylya.net, 10.06.2021

%%%cit
%%%cit_pic
%%%cit_text
Очень искренно и нежно. И мне кажется, что это одна из лучших песен о маме и
\emph{материнской любви}. И нельзя пройти мимо семейного дуэта мамы и дочки - Нины и
Тони Матвиенко, которые по своему не спели, а рассказали эту песню. Очень
нежно, глубоко и сердечно:
%%%cit_title
\citTitle{5 задушевных украинских песен, которые пели наши родители, а теперь поем мы}, 
Кино Вояж И Не Только, zen.yandex.ru, 07.06.2021
%%%endcit

%%%cit
%%%cit_head
%%%cit_pic
%%%cit_text
Помню ли я, простой горловчанин, 27 июля 2014 года? А помните ли вы, как
впервые поцеловались? Так вот, свой \enquote{поцелуй войны} я вряд ли когда-то забуду.
Кровавые коридоры городской хирургии, горящее админздание крупнейшей
энергетической компании Донбасса, несущие на руках окровавленных детей люди в
камуфляже, истошные крики женщин над телами погибших и ушедший за полчаса в
подвалы город. Достойное свидание со Смертью? Была ли ненависть? Была. Это со
временем она растворилась в анестезии рассуждений, осознании войны и
привыканиии к \enquote{труду} артиллеристов. Город, рожденный жить (как и тысячи других
мирных городов на планете), впервые ощутил то, как иногда государство может
страстно \enquote{\emph{любить}}
%%%cit_comment
%%%cit_title
\citTitle{Gorlovka.ua: Город, рожденный жить - Блоги}, 
Егор Воронов, gorlovka.ua, 29.07.2016
%%%endcit

%%%cit
%%%cit_head
%%%cit_pic
%%%cit_text
Гоблин со стропил следил за прячущимся монахом, который шпионил за ученым.
Гоблин ненавидел монаха и имел для этого все основания. Монах никого не
ненавидел и не \emph{любил}: он был фанатичен и честолюбив. Ученый тайком
спрятал рукопись, найденную им в переплете книги.  Был поздний час, в
библиотеке затихало. Где-то украдкой скреблась мышь. Свеча, стоявшая на столе,
над которым согнулся ученый, почти догорела
%%%cit_comment
%%%cit_title
\citTitle{Зачарованное паломничество}, Клиффорд Саймак
%%%endcit

%%%cit
%%%cit_head
%%%cit_pic
%%%cit_text
А тепер опинився на воді ще більшій. Сам би й не попав туди, але ж був у мене
старший брат Несміян. Брат починається з віків і правіків. З братом завжди
краще. Він пішов до князя і сказав: "От коли б мій Сміянко поплив з Добринею".
— "То хай попливе", — згодився Володимир і не від надмірної \emph{любові} до
мого брата Несміяна, а лиш тому, що брат належав до старшої дружини князевої,
яка живе з князем на єдинім хлібі.  Дурне то було плавання. Десь на передніх
ладьях сам уй князів Добриня з послами до імператорів царгородських, а ми
позаду, княжі отроки і конюхи з кіньми, яких не знати й навіщо везли до
Царгороду — чи то для потреби, чи просто для пихи. Нас було троє: я, Сміянко,
та ще Корчак і Мовчак, старійшина над нами Порій, а веслярі мовби й безіменні,
бо коли чоловік не має мови, то не має він імені також, веслярі ж робили своє
діло мовчки, тільки й показували нам та коням то затяті обличчя, то понурі
спини, і все те плавання стояло мені перед очима, ніби суцільне гойдання
понурих спин
%%%cit_comment
%%%cit_title
\citTitle{Тисячолітній Миколай}, Павло Загребельний 
%%%endcit

%%%cit
%%%cit_head
%%%cit_pic
%%%cit_text
О \emph{любви} за всю историю человечества написано и сказано столько, что, кажется,
нового не добавишь. Даже тем, кто не отличается особым усердием к чтению и
размышлениям, – и тем все вроде бы ясно с \emph{любовью}. Но стоит лишь попытаться
дать себе отчет в том, что же именно «ясно», как почва под ногами становится
шаткой.  Эти беглые строки – еще одна попытка сказать несколько осмысленных
слов о \emph{любви человеческой}.  Скульптор отсекает от глыбы все лишнее, освобождая
заключенную внутри статую.  Так красиво может сказать мастер о своем искусстве
или ценители – о мастере, хотя за изяществом фразы стоят годы трудов и неудач,
пот и бессонные ночи. Тем не менее формула верна, и верна не только для
скульптуры, но и для других видов творчества. И мыслит человек так же – отсекая
лишнее.  Насколько важен предмет размышления, настолько важно умение
определить, чем не является этот предмет. Путем постепенного отсечения того,
чем он не является, мы приближаемся к определению его сути.  Этот принцип важен
в правильном разговоре о Боге. Размышляя о Боге, мы окружаем Его частицами
«не», оставляя невысказанным то, что прячется в смысловой сердцевине. Бог
невидим, неизречен, неизобразим, непостижим – и так далее. И чем дальше вглубь,
тем тише слова, тем значимей молчание.  Мыслить о Боге – значит отрешаться от
мыслей о мире, обнажать ум от всяких образов. Такое богословие именуется
апофатическим, и, быть может, кто-то из читателей окунется вскоре в умный мир
средневековых мистиков и глубоких мыслителей о Существе Высочайшем. Это будет
мир, где отброшены ветхие одежды, мир приближения к реальности, с трудом
вмещающейся в слова.  Ну а нам предстоит разговор более легкий, хотя не менее
важный – разговор о \emph{любви}
%%%cit_comment
%%%cit_title
\citTitle{Возвращение в Рай}, Андрей Ткачев
%%%endcit

%%%cit
%%%cit_head
%%%cit_pic
%%%cit_text
Ничего, топаем потихоньку. Поскольку регулярно поротая задница — прибавляет
трезвости чиновничьему уму. Украине до этого... как до Китая пёхом, уж простите.
Только с треском выгоняют совсем клинических пациентов из Правительства. Чтобы
не очень остроумные промахи комика хоть как-то прикрыть.  Ничего не изменилось
в Незалежной. Как не возьмёшь списки «королевской рати» президента Зеленского,
тут же смех разбирает. А ну-ка, добры молодцы, рассчитайсь! Соросюк — коломоец,
соросюк — коломоец, ахметовец — порошенковец, гейпарадовец –бандеровец! Расчёт
окончен! Не так? А глядя на мятущегося Гаранта, на ум просится сцена признания
Хлестакова в \emph{любви} сразу двум дамам. Такая милая «разновекторность», столь
привычная со времён Кучмы
%%%cit_comment
%%%cit_title
\citTitle{Такие смешные: украинская политика... через призму творчества Гоголя}, 
Исторические напёрстки, zen.yandex.ru, 28.10.2021
%%%endcit

%%%cit
%%%cit_head
%%%cit_pic
%%%cit_text
От цю Україну перекінчиків, яничарів чужого пана і володаря, ненавидів Шевченко
за те, що «чужим богам пожерли жертви, омерзились»; що своїм існуванням
поганили його прекрасну, вольнолюбиву країну. \emph{Любити} цю "рідну" погань
він не міг. Певно, він прагнув всім своїм, сповненим \emph{любові}, великим
серцем "\emph{любить} людей", та коли стрічав "недолюдків", благає Бога дати
йому силу "проклинать і світ запалити". Щоби в тім очищуючім вогні на попіл
згоріло все плюгаве, все плебейське, продажне й гниле. Знав бо, що «застарілі
недуги лікуються героїчними засобами». Де не поможе лік, поможе залізо; не
поможе залізо, поможе вогонь. Хотів, щоб з вогненної купелі встала його вічно
юна, сильна духом і горда Україна; та, яку бачив в столітніх очах діда-козака,
що "як зорі сіяли"; Україна, яка з степових могил вставала перед ним привидами
велетнів; яка «туго начиняла землю своїм і ворожим трупом, своєї свободи на
поталу не давала, ворога деспота під ноги топтала, і свобідна й нерозтлінна
вмирала»
%%%cit_comment
%%%cit_title
\citTitle{Заповіт Шевченка}, , pravyysektor.info, 10.03.2018
%%%endcit
