% vim: keymap=russian-jcukenwin
%%beginhead 
 
%%file 15_09_2021.fb.nikonov_sergej.2.recenzia_film_strasti_po_zoje
%%parent 15_09_2021
 
%%url https://www.facebook.com/alexelsevier/posts/1581786628833313
 
%%author_id nikonov_sergej,bilchenko_evgenia
%%date 
 
%%tags bilchenko_evgenia,film,film.strasti_po_zoje,kino,kultura,recenzia
%%title БЖ. Очень необычная рецензия на фильм "Страсти по Зое"
 
%%endhead 
 
\subsection{БЖ. Очень необычная рецензия на фильм "Страсти по Зое"}
\label{sec:15_09_2021.fb.nikonov_sergej.2.recenzia_film_strasti_po_zoje}
 
\Purl{https://www.facebook.com/alexelsevier/posts/1581786628833313}
\ifcmt
 author_begin
   author_id nikonov_sergej,bilchenko_evgenia
 author_end
\fi

Реца от Евгении Витальевны на фильм Страсти по Зое. А я копирую  ее без
страсти, лайков, дизлайков, комментов...

Евгения Бильченко

БЖ. Очень необычная рецензия на фильм "Страсти по Зое"

И не смогли меня подвигнуть украинские нацисты к суициду. И на колени не
поставили. И извиниться за Россию не заставили. Потому что никогда русского
человека не сломит враг, не возьмёт его "на слабо", и не права Марина Ивановна
была, что в петлю полезла, не надо было, милая вы моя.

И целую травлю на меня СБУ запустила, используя ботоферму свою, и полчища сил
по информационно-психологическим операциям ВСУ накинулись на меня, на стихи
мои, на тело мое, на хвори мои, на семью мою, на внешность мою, на нос мой и на
улыбку мою, на зубы мои и на волосы мои, на одежды мои и на семью. 

\ii{15_09_2021.fb.nikonov_sergej.2.recenzia_film_strasti_po_zoje.pic}

И на улице улюлюкают мне вслед. И сотни наивных присоединились к фюрерам 21
века, искренно присоединились от души. И замолчали дверные звонки мои, и
телефоны оглохли, и все друзья меня оставили, и имя мое поминать страшатся, и
выговаривать его запретили по университетам, а на улице, коли видятся,
единичные уцелевшие всё "хвост" ищут... Боятся. И лишили меня репутации, и
работы лишили, и скоморохом к позорному столбу на площади прибили, и заработков
лишили, и трудиться на панщине заставили, и имя честное отобрали, и сцену
закрыли, и от науки отлучили, и со студентами разлучили...

Потому что нет сейчас Германии 1933 года, и печей нет, и виселиц, хотя Одесса
есть и Донбасс есть, а есть еще вот это вот всё - новое, либеральное, изощренно
жестокое, - когда человека мучают морально и доводят до смерти медленно. Это
тоже - босиком по снегу. Это тоже - в рубище в клетке. Это тоже - в колодках и
в плевках.

Но не смогли они меня сломать.

И тогда лишь возникла у меня мысль: "А хорошо было бы красиво в "Англетере"
выпить пачку транквилизаторов под бутылку", - когда к украинским нацистам
присоединились русские граждане. Никак они мне прошлое не прощают - штрафбат
мой, обитель моя, сколько я уже сделала ради русского мира больше их, но их
фора - всегда супротив моей, просто по рождению в нужном окружении.

Не будем обо мне. Мне не за себя больно. Я, если честно, и так так тяжело живу,
что поставила себе штрафбат до первой крови.

Больно мне за Россию. Больно мне, что они - заодно с моими украинскими
гонителями. Ситуативно так вышло, - а вдруг не ситуативно? А вдруг - это один и
тот же коллективный Запад, который рвет русскую землю с двух сторон? ему какая
разница, как братский народ рвать? Мальчиками с Азова или диванными патриотами
из Москвы?

И еще больно мне, что создают мои русские гонители имидж России как не
прощающего чудовища, бросающего своих, чем весьма радуют украинцев.

И вот, когда я прочла эти пожелания смерти, захотелось мне умереть: ибо что от
врагов - адреналин, от своих - нож в спину.

И подумала я, и решила: ЭТО - НЕ РОССИЯ, ЭТО - НЕ РУССКИЙ МИР, Это - такие же
выруси, как их либералы и наши националисты. Потому что велика душа у русского
мира. Это провокаторы и глупцы, дай Бог, чтобы глупцов было больше, чем
провокаторов. Не значит то, что я вины своей не искупаю.

А значит это, что я люблю тебя, Россия.

Вот об этом для меня - фильм про Зою. Очень рекомендую для психологического
укрепления. Вынести сцены можно. Кто-то скажет: "Пропаганда". Ну, и пусть. Я в
Дэвиде Линче и элитарщине тоже круто разбираюсь. Но она, Зоя-то, сказала: "Я
всё выдержу".

Я ВСЁ ВЫДЕРЖУ.

\ii{15_09_2021.fb.nikonov_sergej.2.recenzia_film_strasti_po_zoje.cmt}
