% vim: keymap=russian-jcukenwin
%%beginhead 
 
%%file 21_12_2021.fb.druzenko_gennadiy.1.vojna_kak_shans
%%parent 21_12_2021
 
%%url https://www.facebook.com/gennadiy.druzenko/posts/10158674327093412
 
%%author_id druzenko_gennadiy
%%date 
 
%%tags gosudarstvo,identichnost',nacia,rossia,ukraina,ukraincy
%%title ВІЙНА ЯК ШАНС
 
%%endhead 
 
\subsection{ВІЙНА ЯК ШАНС}
\label{sec:21_12_2021.fb.druzenko_gennadiy.1.vojna_kak_shans}
 
\Purl{https://www.facebook.com/gennadiy.druzenko/posts/10158674327093412}
\ifcmt
 author_begin
   author_id druzenko_gennadiy
 author_end
\fi

ВІЙНА ЯК ШАНС

Найкраще поточну ситуацію в Україні описує євангельське повчання, яке Лука
вкладає в уста Христа: «Тоді люди їли, пили, одружувалися, аж до того дня, доки
Ной не ввійшов в свій ковчег. Тоді настав потоп і їх усіх знищив. Це буде так
само, як було в часи Лота. Тоді люди їли, пили, купували, продавали, саджали
дерева і зводили будинки. Але в день, коли Лот пішов з Содома, з небес пролився
дощ із вогню та сірки і всіх їх знищив».

\ii{21_12_2021.fb.druzenko_gennadiy.1.vojna_kak_shans.pic.1}

Російське федеральне агентство з технічного регулювання та метеорології вперше
в історії сучасної Росії затверджує національний стандарт РФ «Захоронение
срочное трупов в военное и мирное время» (ГОСТ Р 42.7.01—2021), який набере
чинності 1 лютого 2022 року. Американці вперше серйозно розглядають питання, як
надати майбутній українській партизанці ефективності моджахедів і перетворити
її на Талібан для росіян. А український президент в цей самий час продовжує
закликати НАТО (і провокувати Росію) надати чітку перспективу членства України
в альянсі вже наступного року. Його потенційний конкурент на наступних виборах
і київський градоначальник перевірятиме наступного тижня електричні сирени, але
замість навчання столичних мешканців, що робити, коли вони заревуть
по-справжньому, нас просять зберігати спокій і не реагувати на незвичний звук…

Ні, я не алярміст і не намагаюсь передати куті меду. Я не вважаю, що ми
приречені на широкомасштабний конфлікт з Росією. Але я бачу, що з кожним днем
такий конфлікт стає все реальнішим. Перемогти НАТО в Україні (яке за Україну не
воюватиме) – ідеальний сценарій для Путіна, аби спробувати переформатувати світ
за парадигмою «концерту великих держав», яка так подобається не тільки
підстаркуватому хазяїну Кремля, а й старому генію американської дипломатії
Генрі Кіссинджеру (див. його «Світовий порядок. Роздуми про характер націй в
історичному контексті»).

Я думаю, що виток інформації про те, що у Вашингтоні стали серйозно
обмірковувати можливу підтримку української партизанки – найкраща новина за
останній рік. Це значить, що американці готові поборотись за Україну, а не
тільки висловлювати «глибоке занепокоєння». І в нас є шанс вижити і
переродитися у майбутній катастрофі.

Дозвольте ще раз вдатися зацитувати Святе Письмо: «якщо пшеничне зерно, упавши
на землю, не помре, то залишиться одне, а якщо помре, то принесе багато
плодів». Іноді нації, як і окремій людині, слід пережити екзистенційну кризу –
справжній катарсис – аби отримати шанс на новий початок. 

Так з попелу Османської імперії народилась сучасна Туреччина, з виступу де
Голля на BBC 18 червня 1940 року – надія на відродження незалежної Франції, з
руїн Третього Райху – сучасна Німеччина, з хаосу громадянської війни постав
Радянський Союз… Насправді таких прикладів небагато. Значна частина
екзистенційних криз завершувались для держав руїною і занепадом. Ніхто не знає
наперед, чи переживе пацієнт клінічну смерть. Тому кликати велику війну точно
не варто. Але якщо вона вже трапиться, потрібно зробити все, аби винести уроки
з нашого недоладного минулого і перетворити її на новий шанс і новий початок
для України.

Що для цього потрібно? 

Насамперед когорта людей, здатних очолити спротив і стати батьками-засновниками
нової України. Джордж Вашингтон ніколи б не виграв війну за незалежність, якщо
б навколо нього не згуртувались ті, кого американський історик Джозеф Елліс
назвав «братами-засновниками» США. Насправді їх було небагато – з ключових
фігур це Бенджамін Франклін, Джон Адамс, Томас Джефферсон, Джон Джей, Джеймс
Медісон та Александер Гамільтон. Четверо з них були першими президентами США,
Джон Джей – першим очільником Верховного суду, а Гамільтон – батьком фінансової
системи новоствореної федерації. Бенджамін Франклін, найстарший з усіх, не
займав у США офіційних посад, але був єдиним з батьків-засновників, хто скріпив
своїм підписом всі три найважливіші історичні документи, що лежать в основі
утворення Сполучених Штатів Америки як незалежної держави: Декларацію
незалежності США, Конституцію США і Версальський мирний договір 1783 року.

Те саме ми можемо сказати про Ізраїль. Якщо б поруч з Бен Ґуріоном не було
Ісаака Бен-Цві, Моше Шарета, Леві Ешколя, Ґолди Меїр, Елізера Каплана, Дов
Їосефа, навряд чи єврейський народ отримав би свою державу і зберіг її у
здавалося б безнадійній боротьбі за існування Ізраїлю. Те саме можна сказати
про батьків-засновників сучасної Індії чи постапартеїдної Південної Африки.
Втім обмежусь лише повторенням загальновідомої істини: один в полі не воїн.
Звичайно, має бути справжній харизматичний лідер, але також когорта тих, кого
нащадки назвуть «братами-засновниками» нової України.

Друге, потрібна візія майбутньої України. Здається, всі, хто має голову думати,
переконались, що «Україна – не Росія» – то погана національна ідея. Ви не
можете збудувати комфортний та надійний дім, якщо в основі вашого бачення цього
дому лежить просте заперечення: аби не як у сусіди. І американські колоністи, і
сіоністи, і борці проти апартеїду чітко знали не тільки чого вони НЕ хочуть, а
і власне яким вони прагнуть бачити власний дім. Не в подробицях і не в деталях
– у будівництві держав так не буває, але у загальних обрисах і принципах. 

Ми за 30 років так і не почали серйозну розмову про те, що означає бути
українцем, які правила гри ми ладні залишити на розсуд професійних політиків, а
які залишити у власних руках, аби завжди тримати владу на налигачу, як і навіщо
нам жити разом – і це Ахіллова п’ята нашої державності. Іншими словами, аби
виграти нашу війну, слід мріяти не тільки про надійний паркан навколо
українського обійстя, а й про архітектуру та функціонал майбутнього
українського дому. Воювати ЗА завжди перспективніше, аніж тільки ПРОТИ.

Третє, нам потрібно буде переконати світ, що формальна легітимність важить
менше за наше фундаментальне право самим визначати свою долю. Коли де Голль
звернувся до французів із закликом чинити опір нацистам, він не мав жодного
демократичного мандату: ніхто його не обирав лідером нації і не уповноважував
говорити від імені французів. Але саме тому де Голль – найвеличніша постать у
французькій історії минулого століття. 

До речі, перші вибори в Ізраїлі відбулись лише в січні 1949 року – Бен Ґуріон
проголосив створення держави Ізраїль і сформував тимчасовий уряд за вісім
місяців до того – 14 травня 1948 року. Просто тому, що вірив, – він мав так
чинити. Ми маємо затямити, що в час смертельної загрози для держави лідерів
народ не обирає – він їх впізнає. Врешті-решт легітимним очільником Франції
часів нацистської окупації був маршал Петен, а генерал де Голль був
самозванцем...

Четверте. Нам слід виучити дуже простий урок, який сформулювала киянка Ґолда
Меїр: «Доля маленьких країн завжди залежить від наддержав, а ті завжди
захищають передовсім власні інтереси». Ми маємо перерости свій інфантилізм,
який або очікує, що американські політики мають прокидатися і лягати спати з
думкою про Україну, або впадає в розпач і сприймає як зраду цілком прагматичне
відстоювання Вашингтоном власних інтересів насамперед.

Як колись сказала Меркель Ющенку після Бухарестського саміту: Україна лише тоді
матиме шанс приєднатись до НАТО, коли її вступ до альянсу збільшить, а не
зменшить безпеку усіх його членів. Зараз для тверезих голів у європейських
столицях очевидно: навіть надання Україні ПДЧ – зменшить європейську безпеку, а
не збільшить. Тому нам потрібно навчитись тверезо бачити, де інтереси
українського бика співпадають з інтересами американського Юпітера і користатися
з цього, не претендуючи аби Юпітер думав про бика більше, аніж про своє місце
на геополітичному Олімпі.

І останнє. Нам потрібно усвідомити, що наш головний ворог хоче нас не знищити,
а асимілювати. Росії не потрібна українська державність, але потрібні українці.
Велика спокуса полягає в тому, що українці від часу створення Російської
імперії робили і роблять карколомні кар’єри в сусідній державі. Якщо б Корольов
народився в незалежній Україні, він ніколи б не відправляв в космос ракети.
Хоча можливо і відправляв би, але після еміграції до США, Китаю чи Росії. 

Якщо почитати історію всіх цих розумовських, кочубеїв, прокоповичів,
безбородьків, а пізніше – підгорних, малиновських, семичасних, кириченків, то
важко не погодитись, що українці – співтворці російської імперії, а не її
жертви. Скільки б ми не волали «геть від Москви!», Росія для значної частини
українців залишається якщо не ментально ближчою, то зрозумілішою за Європу.
Якщо вірно служиш імперії, ніщо не заважає піднятися до самих її верхів. І саме
тому аби виграти нашу війну нам недостатньо воювати ПРОТИ Росії. Ми маємо
відокремиться від неї не так мовою, як принципово іншою світоглядною системою
координат. Як колись американці від британців. Імперія завжди (чи принаймні ще
довго) краще годуватиме своїх підданих. Отже нам потрібно запропонувати аби в
українському домі було щось, важливіше за російський холодильник. Свобода?
Повага до людської гідності? Справедливість? Рівні правила гри для всіх?

Насамкінець. Найменше я хочу, аби мої прогнози виявились пророцтвами. Війна –
це завжди погано. Це обірвані життя, понівечені тіла, покалічені долі та
зруйновані домівки. Тому, як казала Ґолда Меїр, війні потрібно спробувати
запобігти – майже за будь-яку ціну. Але якщо доведеться воювати, то ми
воюватимемо й переможемо, втім спершу треба випробувати всі інші варіанти
виходу з кризи.

Втім, якщо нам все ж судилося заплатити велику ціну за збереження власної
державності, потрібно аби заплативши її ми отримали шанс на новий початок.
Шанс, який ми не матимемо жодного права змарнувати. І щоб не сплюндрувати його,
готуватись потрібно вже зараз...

\ii{21_12_2021.fb.druzenko_gennadiy.1.vojna_kak_shans.cmt}
