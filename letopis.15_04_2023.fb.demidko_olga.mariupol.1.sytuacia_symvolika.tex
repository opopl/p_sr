%%beginhead 
 
%%file 15_04_2023.fb.demidko_olga.mariupol.1.sytuacia_symvolika
%%parent 15_04_2023
 
%%url https://www.facebook.com/100009080371413/posts/pfbid05aLdhiYt2pozdjAEhVUqvY4fYiCwM6WZZwudbQfqg4wjVAq1qGcMFXmyLDEkJ8xql
 
%%author_id demidko_olga.mariupol
%%date 15_04_2023
 
%%tags 
%%title Ситуація, мабуть, буденна. Але з голови тепер не виходить...
 
%%endhead 

\subsection{Ситуація, мабуть, буденна. Але з голови тепер не виходить...}
\label{sec:15_04_2023.fb.demidko_olga.mariupol.1.sytuacia_symvolika}

\Purl{https://www.facebook.com/100009080371413/posts/pfbid05aLdhiYt2pozdjAEhVUqvY4fYiCwM6WZZwudbQfqg4wjVAq1qGcMFXmyLDEkJ8xql}
\ifcmt
 author_begin
   author_id demidko_olga.mariupol
 author_end
\fi

Вже більше року мій  верхній одяг прикрашають значки з українською символікою.
З'явились вони в мене ще під час мого тримісячного перебування у Празі. Відтоді
я їх не знімаю,  можу тільки додати нові. Не зустрічала ніколи негативної
реакції на них. До сьогодні... Сталося це для мене настільки неочікувано,  що я
була до цього зовсім не готова... Жінка у маршрутці, яка їхала в Ірпінь, (про
що теж повідомила всім присутнім) дуже голосно говорила зі своєю знайомою. Її
розмова з самого початку обурила всіх присутніх, бо йшлося про її щасливе
перебування в Німеччині і негативне ставлення до України. Вона розповідала про
допомогу у розмірі 500 євро, які щомісяця отримує там, про те, що в Україну
приїхала тільки  на свята, але насправді  більше її тут нічого не тримає, бо ця
держава \enquote{їй нічого не дала!}. На цих словах я не витримала, і подивилась на
неї, чим одразу ж привернула її увагу. Вона прямо зловила мій погляд. 395
маршрутка не дуже велика, тому її розмова була чутна всім, навіть водію. І от
вона, дивлячись на мене, питає, нащо я причепила ці значки, я що депутат?!...
Моєї відповіді вона не чекала. Продовжувала далі. Явно зранку щось не те було з
настроєм. Спитала, що мені дала Україна, чому я молода не виїжджаю, коли є
стільки можливостей. В маршрутці почався галас. Один чоловік сказав, що самі
вирішемо, де нам краще. Інша жінка, не витримавши, наголосила, що   не всі
українці вважають, що треба шукати кращої долі закордоном, а коли почалась
повномасштабна війна, багато хто використав цю ситуацію для себе, - маючи ціле
житло, відправляється за такими грошовими допомогами. Але жінка, яка почала всю
цю перепалку, зауважила, що її будинок в Ірпені, на щастя, цілий, але вона жити
під повітряними тривогами ніколи б не стала. Я все це слухала і не могла
зрозуміти власні почуття. Питання \enquote{а що мені дала Україна?} я чула неодноразово
навіть від моїх однолітків ще до 24 лютого 2022 року. Я відчувала одразу і
якесь обурення, і незрозумілий мені спокій. Розуміла, що жінку, яка так щиро
вважає, що всі їй винні, все одно не переконаю. Але все ж вирішила сказати, що
держава ніколи нікому нічого не винна! В Німеччині ця жіночка не почує \enquote{Що мені
дала Німеччина?}. Але я теж задала їй питання \enquote{А що вона особисто зробила щоб
закінчилась ця страшна війна в Україні?}. Адже вона отримує допомогу, перш за
все, як українка. І без громадянства України не мала б 500 євро в Німеччині. В
умовах воєнного стану дуже важливо підтримати економіку власної країни, тому
особисто я повернулась 11 місяців назад в Україну, живу в Києві,  працюю та
плачу податки саме тут, оскільки розумію, що це вкрай важливо! Кожен зробив
свій вибір. Я нікого не засуджую. І бути закордоном після пережитого для декого
дійсно правильний вибір, адже це і питання безпеки. Але все ж я оптимістка і
сподіваюсь, що є багато  українців, які обов'язково повернуться в свою країну.
Вони візьмуть з іноземних країн найкраще і привезуть додому, щоб розвивати
Батьківщину. Бо Україна нікому нічого не винна! Україна - це і є ми, кожен з
нас! А та жінка подивилась на мене навіть якось по-доброму, сказала, що я ще
дуже наївна, але життя мене навчить. Мені треба було вже виходити, тому
наостанок сказала їй: \enquote{Все буде Україна!}. І вийшла. Ситуація, мабуть, буденна.
Але з голови тепер не виходить...
