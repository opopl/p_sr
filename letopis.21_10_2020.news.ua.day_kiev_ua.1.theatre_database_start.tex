% vim: keymap=russian-jcukenwin
%%beginhead 
 
%%file 21_10_2020.news.ua.day_kiev_ua.1.theatre_database_start
%%parent 21_10_2020
%%url https://day.kyiv.ua/uk/news/211020-v-ukrayini-zapracyuvala-teatralna-baza-danyh-digital-theatre-archiv
 
%%endhead 

\subsection{В Україні запрацювала театральна база даних}
\url{https://day.kyiv.ua/uk/news/211020-v-ukrayini-zapracyuvala-teatralna-baza-danyh-digital-theatre-archiv}

Днями Національна спілка театральних діячів України презентувала Цифровий
театральний архів. Зручний та доступний веб-ресурс з повноцінною національною
театральною базою даних створено для театральної спільноти, науковців та
глядацької аудиторії. На сайті вже 113 театрів, 333 вистав та 1563 персоналій.

Проєкт реалізується НСТДУ за підтримки Українського культурного фонду. Куратор
проєкту – український театрознавець, доктор мистецтвознавства, професор Ганна
Веселовська.

У багатьох країнах процес діджіталізованого архівування та фіксації
театрального процесу триває вже давно. Існує чимало електронних театральних
архівів у Великобританії, Німеччині, США, Франції. А державні інституції, що
опікуються театрами в центральноєвропейських країнах (Інститут театру імені
Збігнева Рашевського в Польщі, Угорський театральний музей та інститут,
Театральні інститути у Словаччині, Чехії, Словенії та інші) забезпечують
збереження інформації на спеціальних сайтах.

Вивчивши досвід колег з європейських інституцій, НСТДУ й реалізувала запуск
повноцінної вітчизняної театральної бази даних. На сайті вже 113 театрів, 333
вистави та 1563 персоналій за період 2015-2019 рр. Організатори зазначають, що
попереду ще багато роботи, доповнення та розширення інформації. Також
закликають всіх бажаючих активно користуватися архівом та допомагати у
вдосконаленні ресурсу.

Прес-служба НСТДУ
