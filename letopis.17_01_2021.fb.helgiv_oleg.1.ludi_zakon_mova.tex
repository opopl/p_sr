% vim: keymap=russian-jcukenwin
%%beginhead 
 
%%file 17_01_2021.fb.helgiv_oleg.1.ludi_zakon_mova
%%parent 17_01_2021
 
%%url https://www.facebook.com/oleg.helgiv/posts/410083400262812
 
%%author Хелгив, Олег
%%author_id helgiv_oleg
%%author_url 
 
%%tags mova,ukraina,ukrainizacia,zakon
%%title ці люди борються не за право говорити на російській мові, а за право не вчити мову країни де вони живуть
 
%%endhead 
 
\subsection{Ці люди борються не за право говорити на російській мові, а за право не вчити мову країни де вони живуть}
\label{sec:17_01_2021.fb.helgiv_oleg.1.ludi_zakon_mova}
 
\Purl{https://www.facebook.com/oleg.helgiv/posts/410083400262812}
\ifcmt
 author_begin
   author_id helgiv_oleg
 author_end
\fi

По завиванням як пересічних вузькоговорящих на одном языке, так і їх духовних
вождів тіпа Бойко, Рабіновича та інших. В черговий раз пересвідчуємося - ці
люди борються не за право говорити на російській мові, а за право не вчити мову
країни де вони живуть. 

Причому це не український феномен. Таких неандертальців від філології можна
зустріти в близькому та далекому зарубіжжі. Латвії, Литві, Естонії, Грузії чи
США та Ізраїлі. Тільки якщо тут вони активно то там пасивно саботують мову
країни проживання. Пам'ятаєте як емігрант учасник штурму Капітолію в
Вашингтоні, в поліції вимагав перекладача на російську бо не розумів
англійської, живучи в США.

Знаково, що такий феномен зустрічається тільки серед русскоговорящих громадян
різних країн. Цікаво - це якийсь розумовий феномен та нездатність освоїти іншу
мову, чи природна лінь і пофігізм.

\ii{17_01_2021.fb.helgiv_oleg.1.ludi_zakon_mova.cmt}
