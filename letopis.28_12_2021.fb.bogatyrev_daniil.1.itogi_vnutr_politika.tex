% vim: keymap=russian-jcukenwin
%%beginhead 
 
%%file 28_12_2021.fb.bogatyrev_daniil.1.itogi_vnutr_politika
%%parent 28_12_2021
 
%%url https://www.facebook.com/bogatyriov.daniil/posts/1101413960672950
 
%%author_id bogatyrev_daniil
%%date 
 
%%tags itogi,politika,strana,ukraina
%%title Политические итоги года. Внутренняя политика
 
%%endhead 
 
\subsection{Политические итоги года. Внутренняя политика}
\label{sec:28_12_2021.fb.bogatyrev_daniil.1.itogi_vnutr_politika}
 
\Purl{https://www.facebook.com/bogatyriov.daniil/posts/1101413960672950}
\ifcmt
 author_begin
   author_id bogatyrev_daniil
 author_end
\fi

Политические итоги года. Внутренняя политика.

Ну, а теперь о главных политических тенденциях по итогам года в нашем родном
уютном уголке Третьего мира.

Тенденции:

1. Закручивание гаек. На волне спада первоначальной эйфории и постепенного
разочарования электората во Владимире Зеленском и \enquote{Слуге народа}, власть,
вполне прогнозируемо, ещё в феврале начала жёстко подавлять политических
конкурентов, под шумок дерибаня их экономические активы. Первым под раздачу
попал Виктор Медведчук и его медиа-активы. Затем, при помощи тех же санкций
СНБО, власть попыталась ограничить медийное влияние \enquote{Страны.уа}, ресурсов
Анатолия Шария, и т. д. Под конец года даже выписали подозрение, чтобы
припугнуть П. Порошенко и стали угрожать Р. Ахметову отжимом ТЭС через введение
на них \enquote{временных госадминистраций}. Как и в период позднего Порошенко, чем
ниже рейтинг власти, тем туже она затягивает гайки. Уже и свой аналог \enquote{плана
\enquote{Шатун}} появился.

2. Попытки саботажа указаний Запада ради сохранения за \enquote{своими} людьми хлебных
мест и власти. В этом году, Зеленский \enquote{отбился от рук} вашингтонского обкома и
тихо саботировал его требования по реформе корпоративного управления и судебной
реформе, назначая своих людей \enquote{на кормление} в госкомпании и затягивая
переназначение судей иностранцами. Но не стоит обманываться. Делается это не в
интересах государства Украина, а в интересах властных группировок, приближённых
к Офису президента.

3. Тотальная фискализация. Попросту - повышение старых налогов и введение
новых. В основе этой тенденции лежат попытки власти хотя бы отчасти
компенсировать бюджетный дефицит, сдирая средства на это с нищего и бесправного
населения. Уже в следующем году ВСЕХ граждан Украины заставят декларировать
доходы и введут проверки соответствия декларации при всех значимых покупках
(квартиры, машины, земля, и т. п.). Со всего, что не вписывается в декларации
(например, с валютных накоплений, на которые вы решите купить машину) придётся
уплатить 18\% налог. Кроме того, 5\% налогом на пользование своей ЧАСТНОЙ
СОБСТВЕННОСТЬЮ будут облагаться все собственники с/х земли от 0,5 гектара.
Делается это для того, чтобы вынудить их продать свои наделы крупным дельцам.
Также здесь можно вспомнить о введении кассовых аппаратов и драконовских
штрафов для ФЛП.

Украинское государство давно перестало быть социальным. Но такого налогового
ада, при полном отсутствии соц. гарантий, многие ещё 8 лет назад даже
представить себе не могли. Эффект от него будет строго противоположным
ожидаемому властями - уход в тень и в нал ещё большего процента экономики.

Больше читайте на моём канале по ссылке и подписывайтесь: \url{t.me/polit_bogatyr}

\ii{28_12_2021.fb.bogatyrev_daniil.1.itogi_vnutr_politika.cmt}
