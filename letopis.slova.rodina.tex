% vim: keymap=russian-jcukenwin
%%beginhead 
 
%%file slova.rodina
%%parent slova
 
%%url 
 
%%author 
%%author_id 
%%author_url 
 
%%tags 
%%title 
 
%%endhead 
\chapter{Родина}
\label{sec:slova.rodina}

КАК не нужно торговать \emph{Родиной}...  Урок Армении. Нас это тоже
касается... Спросите сколько ПОЛЯКИ сдоили с ЕС за аналогичные вещи ))) 10
июня, на заседании правительства Армении было одобрено соглашение с Европейским
Союзом о финансовой поддержке республики в размере трех миллионов евро за отказ
Армении от использования названия \enquote{коньяк} при маркировке алкогольной
продукции.  Требование об отказе от названия \enquote{коньяк} предусмотрено Соглашением
о всеобъемлющем и расширенном партнерстве, которое было подписано еще в ноябре
2017 года. В силу соглашение вступило с 1 марта 2021 года.  Согласно этому
документу Армения должна отказаться и от называния \enquote{шампанское}.  Сообщаю... По
минимуму такое решение стоит 3 МИЛЛИАРДА... но никак не миллиона ))),

%%%cit
%%%cit_pic
\ifcmt
	pic https://avatars.mds.yandex.net/get-zen_doc/5098316/pub_60c3819603088a3a1a36d8ea_60c38e2be72f9332f7e8d619/scale_1200
  caption К середине XVII века почти весь Киев лежал в заброшенных руинах. Рисунок Абрагама ван Вестерфельда, посетившего Киев в 1651 году
\fi
%%%cit_text
А южные русские за несколько столетий не приобрели совсем ничего. Их угнетали
монголы, потом поляки с литовцами, то есть совершенно чуждые народы, которые
кроме поборов ничего больше с захваченных ими земель получать не хотели. Потому
и развалилось сельское хозяйство на Украине, и не возникло никаких промышленных
центров. Малороссы могли только воевать под знаменами кого угодно, и то только
большими массами, без проявления какой-либо индивидуальности. И когда их братья
великороссы принялись за восстановление своей былой \emph{родины}, они столкнулись с
искренним непониманием своих бывших братьев
%%%cit_title
\citTitle{Почему великороссы стали такими сильными, а малороссы остались такими слабыми?}, 
Исторический Понедельник, zen.yandex.ru, 11.06.2021
%%%endcit

%%%cit
%%%cit_pic
%%%cit_text
Эти песни мы поем с сёстрой и вспоминаем свою \emph{малую родину} которую у нас отобрали. Спасибо за эти песни
%%%cit_comment
Валя Юдина
%%%cit_title
\citTitle{5 задушевных украинских песен, которые пели наши родители, а теперь поем мы}, 
Кино Вояж И Не Только, zen.yandex.ru, 07.06.2021
%%%endcit


%%%cit
%%%cit_head
%%%cit_pic
%%%cit_text
На сегодня наша \emph{Родина} - самая северная и самая холодная страна в мире. И это
факт. Но при этом самая богатая и живучая. В том же мире. И это тоже не
оспаривается. И неспроста она овеяна легендами, загадками, домыслами и
помыслами... такой удивительной страны нигде в мире больше нет!  Друзья,
сегодня, в единственный летом большой и значимый праздник - День России -
давайте говорить о России. Нашей огромной, любимой, иногда проблемной, но всё
равно очень и очень привлекательной стране
%%%cit_comment
%%%cit_title
  
%%%endcit

%%%cit
%%%cit_head
%%%cit_pic
%%%cit_text
Олимпийский чемпион Барселоны - 1992, шестикратный чемпион мира, введенный
Международной федерацией фехтования (FIE) во Всемирный зал славы. Все это – об
украинском спортсмене Георгии Погосове.  Его опыт и спортивные навыки оценили
даже в США. Там он прославлял отечественную школу фехтования. В 2020-м вернулся
на \emph{Родину}. Тут и выяснилось, что указом президента его лишили
государственной стипендии.  Подобные стипендии, чтоб вы понимали, назначаются в
знак признания выдающихся достижений спортсмена, повлиявших на формирование
позитивного имиджа Украины.  Закон, по которому человека, положившего здоровье
и юность ради спортивных достижений во славу родного государства, лишали
копеечной стипендии, считаю позорным и несправедливым. Покидая Украину, он
заслуги перед Украиной в чемодан не упаковывал и за границу не вывозил. Все
оставил на \emph{Родине}
%%%cit_comment
%%%cit_title
\citTitle{Государству плевать на соотечественников, прославляющих Украину / Лента соцсетей / Страна}, 
Максим Могильницкий, strana.ua, 05.07.2021
%%%endcit


%%%cit
%%%cit_head
%%%cit_pic
\ifcmt
  tab_begin cols=2
		 caption Во Львове окончательно снесли Монументы славы, сняв фигуры Родины и солдата. Фото: Facebook 
     pic https://img.strana.ua/img/article/3453/vo-lvove-okonchatelno-37_main.jpeg

     pic https://strana.ua/img/forall/u/10/85/218735943_179665077547738_8205208462023198692_n.jpg
  tab_end
\fi
%%%cit_text
Во Львове окончательно уничтожили простоявший полвека Мемориал в память о
победителях над нацизмом в Великой Отечественной войне на улице Стрыйской,
снеся основную часть композиции - фигуры \emph{Родины-матери} и советского
солдата.  Хотя до последнего часть львовян не верила, что фигуры демонтируют.
Была версия, что их все же оставят как память и часть истории Львова.  Местные
власти объявили памятник аварийным и решили снести еще три года назад.  Хотя
многие расценили "аварийность" лишь как прикрытие для "декоммунизации" в угоду
радикально настроенной части общества.  30-метровую стелу Монумента славы, как
называют памятник горожане, повалили в марте 2019 года (причем только с третьей
попытки), весной этого года снесли барельефы с изображениями советских солдат,
а теперь - и фигуры \emph{Родины} и солдата.  Их погрузили на грузовики и
увезли в "музей" - так называемую "Территорию террора", где хранятся советские
символы
%%%cit_comment
%%%cit_title
\citTitle{Без Родины. Как и зачем во Львове снесли Монумент советским воинам}, 
Александра Харченко, strana.ua, 23.07.2021
%%%endcit

%%%cit
%%%cit_head
%%%cit_pic
%%%cit_text
Ты любишь свою \emph{родину}. Любить \emph{родину} на той ступени, которой ты достигнул, не
грех, а долг. В мире еще мало тех, которые переросли эту \emph{любовь}, которые имеют
право сказать: «Мое отечество — Истина». Все остальные, которые отрекаются от
этой любви, еще не доросли до нее. Они еще полузвери... Ибо и зверь не знает
\emph{Родины}. Люби \emph{Родину}.  Но ты дорос до ступени, чтобы знать правду о своей земле.
Йоги решили, что ты можешь видеть. Ибо другие имеют глаза и слепы, имеют уши —
и глухи. Ты можешь видеть и слышать. И потому — иди...  — Я буду с тобой, —
сказал йог. —   Не бойся. В крайности, умрет твое тело. Твое «я» умереть не
может. Ты бессмертен, как и я, как и все, как и всякая жизнь. Но и тело твое не
умрет на этот раз. Но будь справедлив. Не дай красной ненависти ослепить
зоркость зеленых глаз. Пусть вокруг тебя идет изумрудное сияние терпимости. И
смерть отступит перед тобой. Иди.  Он подтолкнул меня, и я очутился близко от
границ России
%%%cit_comment
%%%cit_title
\citTitle{Три Столицы}, В. В. Шульгин
%%%endcit
