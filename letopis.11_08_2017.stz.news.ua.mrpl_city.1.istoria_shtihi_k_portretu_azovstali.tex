% vim: keymap=russian-jcukenwin
%%beginhead 
 
%%file 11_08_2017.stz.news.ua.mrpl_city.1.istoria_shtihi_k_portretu_azovstali
%%parent 11_08_2017
 
%%url https://mrpl.city/blogs/view/istoriya-shtrihi-k-portretu-azovstali
 
%%author_id burov_sergij.mariupol,news.ua.mrpl_city
%%date 
 
%%tags 
%%title История: Штрихи к портрету "Азовстали"
 
%%endhead 
 
\subsection{История: Штрихи к портрету \enquote{Азовстали}}
\label{sec:11_08_2017.stz.news.ua.mrpl_city.1.istoria_shtihi_k_portretu_azovstali}
 
\Purl{https://mrpl.city/blogs/view/istoriya-shtrihi-k-portretu-azovstali}
\ifcmt
 author_begin
   author_id burov_sergij.mariupol,news.ua.mrpl_city
 author_end
\fi

1929 год. Центральный Комитет предложил срочно разработать план использования
керченских руд и в связи с этим строительство приазовских заводов, учитывая
особую важность широкого исполь­зования фосфористых руд как для металлургии,
так и для сельского хозяйства.

Идея использования керченских железных руд, как и ее воплощение в виде заводов,
была не нова.  В 1898 году близ станции Сартана началось строительство
металлургического завода акционерного общества  \enquote{Русский Провиданс}. В апреле
следующего 1899 года предприятие выдало первую продукцию. Специалисты знают,
что введенный в эксплуатацию 12 августа 1933 года завод \enquote{Азовсталь}  по
сырьевой базе, по способу доставки сырья, по технологической схеме, по
использованию металлургических шлаков в сельском хозяйстве, по номенклатуре
выпускаемой  продукции  являлся  довольно продолжительное время увеличенной
копией завода \enquote{Русский Провиданс}.

Действительно, сырьевая база: фосфористая железная руда с Керченского
полуострова, известняки из карьеров у станции Еленовка, донецкие коксующиеся
угли, марганцевая руда из Никопольского месторождения. Доставка железной руды
водным путем. На \enquote{Русском Провидансе} в 1898 году работало две доменные печи,
две мартеновские печи, три томасовских конвертера, 126 коксовых батарей,
крупносортный и рельсобалочный цеха. Железную руду из Камыш-Буруна  на
пароходах доставляли на пристань, расположенную в полутора верстах от устья
Кальмиуса вверх по его течению. От пристани на склады завода руда
переправлялась по канатной дороге. Известняк, коксующиеся угли, марганцевую
руду и огнеупорные материалы и изделия привозили по железной дороге. Продукция
–  балки, швеллеры, двутавры, уголки, а также шахтные рельсы и крепь. Кроме
того, фосфористые удобрения из шлаков томасовских конвертеров.

Завод \enquote{Азовсталь} в середине 50-х годов прошлого века. Доставка железорудного
агломерата лихтерами из  Камыш-Буруна  по морю в собственный порт завода. Кокс
с соседнего коксохимического завода. Выплавка чугуна в доменных печах,
производство стали в уникальных качающихся мартеновских печах. Обжимной стан,
где слитки превращались в заготовки для рельсобалочного и крупносортного цехов.
Продукция – товарный чугун, рельсы, балки, двутавры, уголки и другой сортовой
металл, а также фосфористые удобрения из шлаков мартеновских печей. 

Между прочим, прорабатывался вариант, который предусматривал реконструкцию и
расширение бывшего \enquote{Русского Провиданса} (к тому времени он назывался площадкой
\enquote{Б} завода имени Ильича), а не строить новый металлургический завод в
Мариуполе. Однако от этой идеи отказались из-за стесненности площадки. Так что
проектирование Ленгипромез  вел  с учетом размещения цехов и служб будущего
завода \enquote{Азовсталь} на левом берегу Кальмиуса.  Недостатком этого решения
является неблагоприятная роза ветров для исторической части города. Жители
центра города, мягко говоря, неважно себя чувствуют, когда дуют восточные
ветры.

Довелось услышать то ли байку, то ли легенду, что еще до строительства завода
\enquote{Русский Провиданс} кто-то из крупных российских предпринимателей надумал
построить металлургический завод на том же самом месте, где сейчас
располагается \enquote{Азовсталь}. Но номер, как говорится, не прошел. Земли за
Кальмиусом  принадлежали  Войску Донскому, и разрешение на строительство должен
быть давать Наказной атаман, ставка которого находилась в г. Новочеркасске.
Когда городские власти Мариуполя узнали о будущем соседстве с металлургическим
заводом за Кальмиусом, от которого никакой прибыли не будет, а только одна гарь
и копоть, то собрали денежки и отправили с ними гонцов в Новочеркасск. Атаман
отказал деловым людям строить завод на территории Войска Донского...

В апреле 1930 года на площадке будущего завода приступили к планировочным
работам, к сооружению складов, к монтажу линий электропередач, к проклад­ке
железнодорожных  путей. К июню 1941 года на заводе \enquote{Азовсталь} действовало
четыре доменные, шесть мартеновских печей и некоторые вспомогательные цеха. 8
октября 1941 года Мариуполь захватили гитлеровцы. Вскоре на въезде на завод
появилась вывеска на русском и немецком языках: \enquote{Акц. О-во Фридрих Крупп.
Азовские заводы в Мариуполе}. Оставляя Мариуполь в сентябре 1943 года под
ударами войск Южного фронта Красной Армии, оккупанты взорвали доменные и
мартеновские печи, паровоздуходувную станцию, вывели из строя энергетическое
хозяйство, разрушили железнодорожные пути.

10 сентября 1943 года Мариуполь был освобожден от немецко-фашист\hyp{}ских
захватчиков. В этот же день в Москве Нарком черной металлургии  СССР Тевосян
подписал приказ под названием \enquote{О возобновлении деятельности металлургического
завода \enquote{Азовсталь} и организации восстановительных работ}. Этим приказом
предписывалось командировать на завод группу специалистов  для оценки вреда,
нанесенного предприятию врагом. На следующий день эта группа прилетела в
Мариуполь на военно-транспортном самолете. С этого момента началось
восстановление \enquote{Южной Магнитки}. К концу 1948 года \enquote{Азовсталь} по выпуску
продукции достиг довоенного уровня.

За всю историю завода директорами, генеральными директорами были двенадцать
человек. Но дольше всех на этом посту был Владимир Владимирович Лепорский. Его
директорский стаж – 25 лет.  Среди азовстальцев было немало ярких личностей.
Инженер Василий Воропаев проводил экскурсию школьников  в одном из цехов
завода. Неожиданно на рольганге начали резать раскаленный рельс, сноп искр
полетел в сторону детей. Воропаев своим телом закрыл детей, но сам пострадал.
Искры лишили его зрения. Но он продолжал напряженно  работать. Подготовил и
защитил кандидатскую диссертацию. Правда, ему самоотверженно помогала жена.
Николай Берилов, прошедший с боями тысячи километров военных дорог, окончил
металлургический институт, работал инженером-исследователем, а в свободное от
работы время писал стихи, издал несколько поэтических сборников.
