% vim: keymap=russian-jcukenwin
%%beginhead 
 
%%file slova.gorod
%%parent slova
 
%%url 
 
%%author 
%%author_id 
%%author_url 
 
%%tags 
%%title 
 
%%endhead 
\chapter{Город}
\label{sec:slova.gorod}

%%%cit
%%%cit_pic
%%%cit_text
\emph{Город} стоит на болотах. Он постоянно подтоплялся, и раз за разом настилалась
новая бревенчатая мостовая. До 28 слоев потом археологи насчитали! Нижние
подмокшие слои, лишенные, как и во всяком болоте, доступа кислорода,
превращались в идеальные капсулы, надежно сохраняющие органику. А береста, как
вы понимаете, это и есть органика.  Вспомните, сколько историй вам в школе
рассказывали о доисторических животных, попавших в трясину. Именно они лучше
всего сохранились. Для палеонтолога болота - это настоящее благословение.  Так
что в Новгороде найдено больше грамот не потому, что это был единственный оазис
просвещения на Руси, а потому что там просто-напросто были наиболее подходящие
условия для их сохранности. Московская береста, а также береста других русских
\emph{Городов} по большей части просто истлела
%%%cit_comment
%%%cit_title
\citTitle{Почему в Москве нашли всего четыре берестяные грамоты, а в Новгороде - свыше тысячи?}, 
Русичи, zen.yandex.ru, 18.05.2021
%%%endcit

%%%cit
%%%cit_head
%%%cit_pic
%%%cit_text
Тут можно долго беседовать со своим бессознательным, выясняя причины подобного
сюрреализма. Но даже не будучи фугасом, превратившимся в траву, за эти два года
я тоже пришел к простому выводу: Я - Человек, рожденный жить. И \emph{Город}, в
котором я живу, не создан для войны. Не создан подсчитывать \enquote{полученные
повреждения в ходе артобстрела}. Не создан считать \enquote{200}, \enquote{300} и \enquote{без вести
пропавших}. Не создан быть полигоном для утилизации боеприпасов и
тестостероновых \enquote{меряний} мужчинами, кто из них лучше умеет убивать. Этот \emph{Город}
рожден жить, любить, вдохновлять, производить и желать доброй ночи
%%%cit_comment
%%%cit_title
\citTitle{Gorlovka.ua: Город, рожденный жить - Блоги}, 
Егор Воронов, gorlovka.ua, 29.07.2016
%%%endcit

%%%cit
%%%cit_head
%%%cit_pic
%%%cit_text
Как часто вы разговариваете с \emph{Городом}, в котором живете? Звоните ли ему
время от времени по стационарному телефону? Справляетесь ли о его самочувствии?
Вижу, что вы смеетесь над моими вопросами.  Чудненько. Вы, наверняка, уверены,
что \emph{Город} - безмолвен и не умеет говорить. На самом же деле он никогда
не молчит.  И слова его - это поступки и чувства живущих в нем людей.
\emph{Город} декламирует сонеты и штампует некрологи, чеканит заметки и
сплетает романы, упражняется в рассказах и плюется афоризмами. Детективы,
любовная \enquote{бульварщина}, мистический реализм, сатира, эпосы, военная
лирика...  Все это он проговаривает своими \enquote{-ами} (французы бы
сказали \enquote{mon ami}). Вот так и Горловка. Хочешь услышать ее - наблюдай
за горловчанами, каждый день составляя тезаурус печалей и радостей этого
\emph{Города Вечной Осени}
%%%cit_comment
%%%cit_title
\citTitle{Gorlovka.ua: Говорить с Городом - Блоги}, 
Егор Воронов, gorlovka.ua, 06.11.2016
%%%endcit

%%%cit
%%%cit_head
%%%cit_pic
%%%cit_text
Впрочем, люди помогают расчищать не только свои дворы, но и общие улицы. В Ялте
жители организовываются через соцсети, приезжают волонтеры из Симферополя и
других мест. Также в \emph{город} прислали военных. Он практически закрыт на
въезд и поэтому на трассе, которая ведет в Ялту огромные пробки.  В южной
столице Крыма проблемы не только с мусором и затоплениями. Потоки местных рек,
потерявших берега, сносят мосты и заливают \emph{город}. Военные отправляют на
место их истока в горах беспилотники, чтобы выяснить перспективы бедствия, а
спасатели тем временем возводят временные дамбы
%%%cit_comment
%%%cit_title
\citTitle{Ялта - наводнение в Крыму. Что происходит 21 июня и как ликвидируют катастрофу}, 
Максим Минин, strana.ua, 21.06.2021
%%%endcit

%%%cit
%%%cit_head
%%%cit_pic
\ifcmt
  pic https://strana.ua/img/forall/u/0/36/2021-07-05_15h50_26.png
	width 0.4
\fi
%%%cit_text
\enquote{А кто это решил вообще?}.  Инфраструктура донецкого Нью-Йорка не сравнится с
его американским тезкой - повсеместная разруха, свалки, нуждающиеся в
капремонте дороги, советские трехэтажки, жилые дома из шлакоблока и с мозаикой
1917 года на стенах, школы, которые отапливаются дровами и углем, поросший
сорняком местный парк.  \emph{Город}, мягко говоря, не соответствует названию
Нью-Йорк, считают жители. Он затухает, наблюдается замирание жизни, а не ее
развитие.  \enquote{Раньше здесь было больше различных предприятий. В настоящий
момент осталось одно предприятие - вот так \enquote{развивается} поселок. Чтобы
соответствовать тому названию, которое сейчас есть, он должен быть лучше, чем в
то время, когда был Новгородским. Нужно много вкладывать средств и труда и
всего остального}, - отвечает нам мужчина на вопрос о переименовании родного
поселка
%%%cit_comment
%%%cit_title
\citTitle{Появление Нью-Йорка на Донбассе - как к этому относятся местные жители. Опрос Страны}, 
Антонина Белоглазова, strana.ua, 06.07.2021
%%%endcit

%%%cit
%%%cit_head
%%%cit_pic
%%%cit_text
Всюду было множество мух. Их выгоняли из комнат полотенцами, закрывали
предварительно ставни. Не удивительно — по дворам стояли мусорные ящики,
содержимое которых вывозилось по мере их переполнения.  По \emph{городу} ходила своего
рода знаменитость — София Александровна. Это была психически больная женщина,
бездомная, грязная, со старой горжеткой на шее, в шляпе с перьями и большими
полями. Одежда ее — грязные лохмотья. Ее всегда сопровождала такая же грязная
собака», — вспоминала одна из старейших жительниц города, врач Мария Плахотина.
И только сатирик Дон-Аминадо (местный уроженец Аминад Пейсахович Шполянский)
язвил: «Держался \emph{город} на трех китах. Вокзал. Тюрьма. Женская гимназия»
%%%cit_comment
%%%cit_title
\citTitle{Игорь Тамм. Нобелевский лауреат из Елизаветграда}, Дмитрий Губин, ukraina.ru, 08.07.2021
%%%endcit

%%%cit
%%%cit_head
%%%cit_pic
%%%cit_text
Ювелирная мастерская талантов.
История же, однако, показала, что не женская гимназия прославила \emph{город}, а
мужская, Александровская. И в ней не учились ни Гриша Радомысльский (он же тов.
Зиновьев, уровень знаний и происхождение мешали поступить), ни Серёжа Костриков
(тов. Киров, он же мальчик из Уржума, жил совсем в других местах), ни Марко
Кропивницкий (родился и рос до её открытия, а учился в соседнем Бобринце), чьи
имена носил этот \emph{город} в разные годы. Кстати, ныне это учебное заведение носит
имя Тараса Шевченко, который тоже не был гимназистом, как и все три «тотема»
этого \emph{города}
%%%cit_comment
%%%cit_title
\citTitle{Игорь Тамм. Нобелевский лауреат из Елизаветграда}, Дмитрий Губин, ukraina.ru, 08.07.2021
%%%endcit

%%%cit
%%%cit_head
%%%cit_pic
\ifcmt
  pic https://strana.news/img/forall/u/0/0/Kremin.jpg
  @width 0.4
\fi
%%%cit_text
Еще одной темой вчерашнего дня были переименования. Старт ей дал мовный
омбудсмен Тарас Креминь, который обратился к Верховной Раде с предложением
переименовать целый ряд \emph{городов}, которые, по его мнению, имеют неправильные
названия.  Причем речь уже идет не о декоммунизации – все, что подпадало под
статьи этого закона, давно переименовано (кроме Днепропетровской и
Кировоградской областей).  Теперь неправильными оказались названия \emph{городов}, в
которых есть русскоязычная составляющая. К примеру, Северодонецк, Первомайск
или Южноукраинск.  Почему Креминя бесят эти названия, понять нетрудно: наличие
Северодонецка или Южноукраинска означает, что далеко не вся Украина является
этнически украинской, и как бы напоминает, что в регионах с подобными
названиями русский язык имеет право на официальный статус.  Однако мовный
омбудсмен тут явно недоработал, поскольку есть еще целый ряд населенных
пунктов, в названиях которых содержится русский корень – к примеру, Вышгород,
Красноград.  Наконец, если пользоваться аргументами о приоритете
государственного языка, то нужно стереть с карты страны и такие греческие
названия как Мариуполь и Севастополь
%%%cit_comment
%%%cit_title
\citTitle{СБУ на службе у "активиста", Рада "разгромила" ЗСТ с ЕС, Киеврада уничтожила Молодогвардейцев. Итоги "Страны"}, 
, strana.news, 05.11.2021
%%%endcit

%%%cit
%%%cit_head
%%%cit_pic
\ifcmt
  tab_begin cols=3
     pic https://avatars.mds.yandex.net/get-zen_doc/5313760/pub_61928f6703e600440635f1a8_61928f74ebd3945421a935bc/scale_1200
     pic https://avatars.mds.yandex.net/get-zen_doc/5234097/pub_61928f6703e600440635f1a8_61928f82bf6d9e6207019e16/scale_1200
		 pic https://avatars.mds.yandex.net/get-zen_doc/96506/pub_61928f6703e600440635f1a8_61928f90c44b240f24d91e7e/scale_1200
  tab_end
\fi
%%%cit_text
У каждого взрослого человека должно быть хобби. Во всяком случае так всегда
казалось мне. Кому-то нравится ходить на рыбалку, кто-то собирает монеты. Я
собираю города России. Если точнее областные центры. Посетить все \emph{города} нашей
огромной страны физически невозможно в течении жизни. Пока что я не объехал
даже половины наших областных и краевых центров, но явно видел больше чем
многие мои знакомые. И естественно в поездках я пользуюсь простым общественным
транспортом. В какой-то момент — это стало даже частью обязательной программы.
То есть если в \emph{городе} имеется метро, то я обязательно туда спускаюсь, но таких
\emph{городов} у нас немного. Если есть трамвай, то я стараюсь на нём попасть из
какой-то одной точки в другую. Ну и так далее. Естественно больше всего я люблю
трамваи, но и троллейбусы тоже дело неплохое. В некоторых наших \emph{городах}
трамвайные маршруты радуют, в каких-то не очень. Иногда выявляются вообще
ужасные вещи вроде уничтожения такого роскошного транспорта, иногда можно
увидеть и новый подвижной состав. Но в целом от всех наших \emph{городов} у меня
осталось лишь одно общее впечатление. Наш общественный транспорт находится в
упадке. И ярче всего это видно по наличию маршруток. Увы, \emph{городов} где нет
маршруток я с ходу назвать не смогу
%%%cit_comment
%%%cit_title
\citTitle{Упадок общественного транспорта России}, 
soullaway soullaway, zen.yandex.ru, 16.11.2021
%%%endcit

%%%cit
%%%cit_head
%%%cit_pic
%%%cit_text
Степан тримав свою руку на теплих Надійчиних пальцях і замислено дивився на
річку, піщані круті береги й самотні дерева на них. Раптом Надійка випросталась
і, махнувши рукою, промовила: — А вже Київ близько.  Київ! Це те велике \emph{місто},
куди він їде учитись і жити. Це те нове, що він мусить у нього ввійти, щоб
осягнути свою здавна викохувану мрію. Невже Київ справді близько? Він
збентежився і спитав: — А де ж Левко?  Вони оглянулись і побачили на кормі гурт
селян, що розташувались там із обідом.  На розгорнутій свитці перед ними лежав
хліб, цибуля і сало. Левко, студент-сільськогосподарник з їхнього ж села, теж
сидів коло них і живився. Він був лагідний і грубший, ніж дозволяв його зріст,
отже, з нього був би колись ідеальний панотець, а тепер — зразковий агроном.
Сам з діда-прадіда селюк, він чудово вмів би допомогти селянинові чи то
казанню, чи науковими порадами.  Учився він дуже акуратно, ходив завсігди в
чумарці й над усе любив полювання.  За два роки голодного перебивання в місті
цілком виробив і оформив основний закон людського існування. З поширеного за
революції гасла: «хто не робить, той не їсть» він вивів собі категоричну тезу:
«хто не їсть, той не робить» і прикладав її до всякого випадку й нагоди. Селяни
тут, на пароплаві, охоче почастували його своїми немудрими харчами, а він зате
розповів їм цікаві речі про планету Марс, про сільське господарство в Америці
та про радіо. Вони дивувались і обережно, трошки насмішкувато, потай віри ие
ймучи, розпитували його про ці дива і про Бога
%%%cit_comment
%%%cit_title
\citTitle{Місто}, Валер'ян Підмогильний
%%%endcit

%%%cit
%%%cit_head
%%%cit_pic
%%%cit_text
Комитет Рады вчера, 2 декабря, одобрил переименование одного из древнейших
\emph{городов} Украины - Владимира-Волынского. Теперь это должен утвердить парламент.
У \emph{города} давняя история, и до 18 столетия он назывался просто Владимиром. Но
расцвет и промышленное становление города совпало с тем временем, когда он
вошел в состав Российской империи. И в 1795 году в названии \emph{города} появился
топоним Волынский - чтобы разграничить два похожих по названию \emph{города} -
Владимира-на-Клязьме и Владимира на берегу реки Луги Волынской губернии. 
А заодно подчеркнуть их равный губернский статус в рамках империи. 
После распада Союза Владимир-Волынский долгое время украинскую власть не
интересовал. Пока не началась волна переименований.
\enquote{Страна} разобралась, зачем это снова затеяли
%%%cit_comment
%%%cit_title
\citTitle{Можно просто Владимир. Зачем в Украине хотят переименовать один из древнейших городов}, 
Анна Копытько, strana.news, 03.12.2021
%%%cit_url
\href{https://strana.news/news/355751-vladimir-volynskij-pereimenujut-kohda-i-kak.html}{link}
%%%endcit
