% vim: keymap=russian-jcukenwin
%%beginhead 
 
%%file slova.nauka
%%parent slova
 
%%url 
 
%%author 
%%author_id 
%%author_url 
 
%%tags 
%%title 
 
%%endhead 
\chapter{Наука}
\label{sec:slova.nauka}

%%%cit
%%%cit_pic
\ifcmt
  pic https://gdb.rferl.org/9685EC99-92CA-46AD-B6CF-C5DCA0F34508_w1597_n_r0_st.jpg
	caption Аркуш 271 Остромирового Євангелія, однієї з найвидатніших пам'яток письменства періоду України-Русі. Переписане у 1056–1057 роках
\fi
%%%cit_text
Робота з адаптації старослов'янського тексту Святого Письма відбувалася не лише
у Києві. «Від початку кожен зі слов'янських народів однакові літери
старослов'янської мови вимовляв трохи по-своєму. Образно кажучи, у мертвий
трафарет вкладав свою мовну культуру. Саме це і створювало неповторне обличчя
старослов'янської мови, яке відрізнялося від усіх інших. Ці різновиди
\emph{науковці} назвали редакціями. Редакція сербська, македонська, українська,
білоруська...» – пише Ганна Куземська, автор книги «Якою мовою молилася давня
Україна»
%%%cit_comment
%%%cit_title
\citTitle{Чому написане у Києві Остромирове Євангеліє – «достояние России»?},
Ірина Костенко; Ірина Халупа, radiosvoboda.org, 13.06.2021
%%%endcit

%%%cit
%%%cit_head
%%%cit_pic
%%%cit_text
Сюрпризы начались сразу же: выяснилось, что на станции не открылась крышка
отсека \emph{научной аппаратуры} и отказали шесть из восьми бортовых вентиляторов
станции. Это ставило под угрозу большинство \emph{научных экспериментов} и могло
вызвать серьезные осложнения в работе системы жизнеобеспечения космонавтов.  В
первопроходцы выбрали самый опытный экипаж. За плечами командира Владимира
Шаталова и бортинженера Алексея Елисеева было уже по два полета. Исключение
составлял лишь инженер-исследователь Николай Рукавишников, который еще не бывал
в космосе. Среди основных задач миссии значились стыковка и доставка первых
космонавтов
%%%cit_comment
%%%cit_title
\citTitle{«Они были обречены» 50 лет назад погиб экипаж «Союза-11». Кто
виноват в главной трагедии советской космонавтики?: Космос: Наука и техника:
Lenta.ru}, Сергей Варшавчик, lenta.ru, 30.06.2021
%%%endcit


%%%cit
%%%cit_head
%%%cit_pic
%%%cit_text
Тепер ситуація змінилася. Боголо зажив світової слави, його ідеї змусили
задуматися \emph{наукову} еліту. Папуаського вченого гаряче підтримав Борис Гук, учень
Глушкова (до речі, він мій давній знайомий— ще з університетських років). Я
його одразу впізнав за столом президії. Потім увагу присутніх прикувала жвава
динамічна постать Боголо, його іскристі очі, що дружелюбно поблискували з-під
кучми волосся, подібного до гігантської чорної кульбаби.  Стисла, дивна,
парадоксальна доповідь папуаса була вислухана в цілковитій тиші, але потім зал
вибухнув ураганом овацій. Боголо промовляв просто, і хоч спеціальних термінів я
не втямив, проте збагнув основну суть його «математики Повноти» (або, як він
часто наголошував, видозмінюючи назву, — «математики Цілості, Одності, Любові,
Реальності»).  Які пристрасті клекотіли в конференц-залі, які вулканічні
виверження скепсису та іронії! А разом з тим — громовиці захоплення й
схвалення.  Вирішено затвердити творчу групу «Резозансу Еволюції» («Резев») для
сформулювання основних принципів альтернативної ЕОМ (так знаної «живої ЕОМ»),
Доручено редколегії часопису «Вирій» об’єднати на своїх сторінках ентузіастів
проекту «Резев», публікуючи необхідні матеріали та результати експериментальних
досліджень
%%%cit_comment
%%%cit_title
\citTitle{Вогнесміх}, Олесь Бердник
%%%endcit
