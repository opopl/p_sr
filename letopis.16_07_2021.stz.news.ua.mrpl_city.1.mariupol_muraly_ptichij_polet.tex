% vim: keymap=russian-jcukenwin
%%beginhead 
 
%%file 16_07_2021.stz.news.ua.mrpl_city.1.mariupol_muraly_ptichij_polet
%%parent 16_07_2021
 
%%url https://mrpl.city/news/view/mariupolskie-muraly-pokazali-s-vysoty-ptichego-poleta
 
%%author_id kovalenko_bogdan.mariupol,news.ua.mrpl_city
%%date 
 
%%tags mariupol,mariupol.pre_war,mural,mural.mariupol.milana
%%title Мариупольские муралы показали с высоты птичьего полета
 
%%endhead 
 
\subsection{Мариупольские муралы показали с высоты птичьего полета}
\label{sec:16_07_2021.stz.news.ua.mrpl_city.1.mariupol_muraly_ptichij_polet}
 
\Purl{https://mrpl.city/news/view/mariupolskie-muraly-pokazali-s-vysoty-ptichego-poleta}
\ifcmt
 author_begin
   author_id kovalenko_bogdan.mariupol,news.ua.mrpl_city
 author_end
\fi

Муралы Мариуполя запечатлели с квадрокоптера. Видео опубликовано на
YouTube-канале \enquote{Маріуполь – туристичне місто}.

Мурал \enquote{Оптимист}, который находится по проспекту Мира, 60, согласно замыслу
художника Александра Писарчука, должен вдохновить на достижение новых целей.

\ii{16_07_2021.stz.news.ua.mrpl_city.1.mariupol_muraly_ptichij_polet.pic.1.optimist}

Еще один мурал этого же автора – \enquote{Динозаврик Несси} по проспекту Мира, 63
изображает сказочного персонажа, который перевозит на себе двух детей. 

\ii{16_07_2021.stz.news.ua.mrpl_city.1.mariupol_muraly_ptichij_polet.pic.2.dinozavr_nessi}

Художник Валерий Лагутенко на стене дома по проспекту Мира, 13 изобразил
велосипедиста, уезжающего от дома с грозовым небом. 

\ii{16_07_2021.stz.news.ua.mrpl_city.1.mariupol_muraly_ptichij_polet.pic.3.velosipedist_groza}

Один из самых известных муралов в Мариуполе – \enquote{Милана}, посвященный Милане
Абдурашитовой, которая осталась сиротой в результате обстрела микрорайона
\enquote{Восточный}.  Рисунок выполнил на стене дома по проспекту Мира, 101 Александр
Корбан. 

% aleksandr korban milana
\ii{16_07_2021.stz.news.ua.mrpl_city.1.mariupol_muraly_ptichij_polet.pic.4.milana}

Неподалеку он же нарисовал мурал \enquote{Навстречу мечте}. На нем изображена женщина с
клеткой в руке, из которой вылетела птица. Мурал затрагивает тему домашнего
насилия.

% aleksandr korban navstrechu mechte
\ii{16_07_2021.stz.news.ua.mrpl_city.1.mariupol_muraly_ptichij_polet.pic.5.navstrechu_mechte}

Еще один мурал Александра Корбана – \enquote{Береги воду – береги жизнь} - можно
увидеть на бульваре Хмельницкого, 22. В его основе – рисунок на тот момент
ученицы 11 класса Алины Николюк.

Художники Сергей и Виталий Грех выполнили мурал \enquote{Держи баланс}. По словам
авторов, таким образом они призывают людей находить баланс между работой и
отдыхом.

\ii{16_07_2021.stz.news.ua.mrpl_city.1.mariupol_muraly_ptichij_polet.pic.6}

Мурал \enquote{АСРЗ}, созданный арт-студией \enquote{Микс} под руководством Максима Светлова,
посвящен мариупольским морякам и символизирует любовь, верность и ожидание. 

\ii{16_07_2021.stz.news.ua.mrpl_city.1.mariupol_muraly_ptichij_polet.pic.7.asrz}

На бульваре Шевченко, 64 художники Phonz и Mouz изобразили острова, которые летают над лесом.

\ii{16_07_2021.stz.news.ua.mrpl_city.1.mariupol_muraly_ptichij_polet.pic.8.ostrovy}

Японский художник Миязаки Кенсуке нарисовал огромную \enquote{Рукавичку} на стене ОШ
№68, которая показывает толерантное украинское общество. 

\ii{16_07_2021.stz.news.ua.mrpl_city.1.mariupol_muraly_ptichij_polet.pic.9.rukavychka}

На ОШ №17 художники Александр Писарчук и Дмитрий Вовк изобразили мурал \enquote{Украина}.

\ii{16_07_2021.stz.news.ua.mrpl_city.1.mariupol_muraly_ptichij_polet.pic.10.ukraina}

Напомним, городской голова Вадим Бойченко \href{https://mrpl.city/news/view/v-razvitii-turizma-mariupol-budet-ravnyatsya-na-gruziyu-i-frantsiyu}{рассказал о дальнейших перспективах туристического развития Мариуполя.}

\href{https://www.youtube.com/watch?v=TOPSGors-ew}{%
Мурали міста Марії. Маріуполь, Місто Марії, youtube, 16.07.2021%
}
