% vim: keymap=russian-jcukenwin
%%beginhead 
 
%%file slova.zapad
%%parent slova
 
%%url 
 
%%author 
%%author_id 
%%author_url 
 
%%tags 
%%title 
 
%%endhead 
\chapter{Запад}
\label{sec:slova.zapad}

%%%cit
%%%cit_pic
%%%cit_text
А вот \emph{западные} русские, которые остались под властью Польши, долго искали свое
уникальное самоназвание, но так и не нашли. За неимением лучшего они приняли
польское определение этой части русской нации, так как, по сути, русскими уже
не являлись – ассимиляция русского населения в польской среде происходила
достаточно интенсивно. И в то время как русский язык в землях, принадлежавших
Польше, никак не развивался с древних времен и очень сильно пополнился
полонизмами и германизмами («папир», «файно», «накшталт», «кошт», «страва» и
прочее), менялось и самосознание \emph{западных русских}. Эти русские, с одной
стороны, уже забыли, что они русские, а с другой – все же не хотели становиться
поляками до конца
%%%cit_title
\citTitle{Почему современную Украину назвали «окраиной», а не более престижно - Киевской Русью?}, 
Исторический Понедельник, zen.yandex.ru, 22.02.2021 
%%%endcit


%%%cit
%%%cit_head
%%%cit_pic
%%%cit_text
И теперь задача \emph{Запада} - обезвредить Россию в ее новом качестве.  Московский
центр Карнеги вдруг выкатил достаточную годную статью по встрече Путина и
Байдена. Приведу отрывки. Кому интересно - вот ссылка на полную версию.
\url{https://carnegie.ru/commentary/84786} За последний год демонстративно
завершен условный горбачевско-ельцинский этап развития России. Разрыв с этим
этапом оформлен обновлением конституции. Многочисленные поправки знаменуют не
столько начало этапа, сколько переводят количество накопившихся изменений в
качество, институциализируют путинскую Россию. Путин поехал в Женеву говорить
от имени этой новой России, которая больше не развивается через строительство
или даже имитацию западных институтов. Задача снята с повестки. Россия больше
не идет суверенным путем к общей цели, теперь суверенна и сама цель
%%%cit_comment
%%%cit_title
\citTitle{Превратить Россию в часть западного мира не удалось}, 
Максим Войтенко, strana.ua, 19.06.2021
%%%endcit

