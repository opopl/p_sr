%%beginhead 
 
%%file 18_01_2023.fb.semenihina_ljudmyla.poltava.1.lyub__mo__druz____ka
%%parent 18_01_2023
 
%%url https://www.facebook.com/sigida.ljuda/posts/pfbid02ZuXrV6KXRe4JVBgjpLEyeK6XPA9of8nHfvrx5Qz4LADJGMt9XeUxKQJe6V1h4JV8l
 
%%author_id semenihina_ljudmyla.poltava
%%date 18_01_2023
 
%%tags mova,literatura,poltava,skovoroda_grigorii,gogol_nikolai,kotljarevskij_ivan,kultura
%%title Любі мої друзі ! Кажуть, Полтавщина - колиска мови
 
%%endhead 

\subsection{Любі мої друзі ! Кажуть, Полтавщина - колиска мови}
\label{sec:18_01_2023.fb.semenihina_ljudmyla.poltava.1.lyub__mo__druz____ka}

\Purl{https://www.facebook.com/sigida.ljuda/posts/pfbid02ZuXrV6KXRe4JVBgjpLEyeK6XPA9of8nHfvrx5Qz4LADJGMt9XeUxKQJe6V1h4JV8l}
\ifcmt
 author_begin
   author_id semenihina_ljudmyla.poltava
 author_end
\fi

Любі мої друзі ! 💚 Кажуть, Полтавщина - колиска мови. У нас на Полтавщині
народилося 16 видатних письменників України ! Бо немає мови без письменників її
світочів, творців, носіїв, хранителів. 

1. 📗Маруся Чурай. Легендарна піснярка вже народних пісень «Ой, не ходи, Грицю,
та й на вечорниці», "Розпрягайте, хлопці, коней" та інших, жила у Полтаві за
часів Хмельницького.

2. 📙Григорій Сковорода. «Всякому місту – звичай і права, всяка тримає свій ум
голова» – ці слова належать мандрівному поету та філософу, що вирушив у свій
путь з полтавської землі, а саме – з села Чорнухи. 

3.📗 Іван Котляревський. Батько "Енеїди" народився в Полтаві. 

4. 📘Євген Гребінка. «З панами добре жить, водиться з ними хай тобі господь
поможе, із ними можна їсти й пить, а цілувать їх — крий нас боже!». Видатний
український байкар та поет, чиї вірші стали популярними піснями, народився на
хуторі Убіжище біля Пирятина. 

5.📕 Микола Гоголь. Російський письменник, що зробив український фольклор
літературним трендом 19 сторіччя, народився у Великих Сорочинцях, провів
дитинство та юність у Василівці (Гоголеве), навчався у Полтаві. 

6. 📒Леонід Глібов. Поет та байкар, на чиїх творах зросло не одне покоління
українців, народився в селі Веселий Поділ, що зараз входить до Семенівського
району Полтавської області. Вірші почав писати під час навчання у Полтавській
гімназії. 

7. 📗Панас Мирний. Автор геніальних романів у стилі реалізму «Хіба ревуть воли,
як ясла повні» та «Повія» народився в Миргороді, навчався в Гадяцькому
повітовому училищі та половину життя прожив у Полтаві. 

8. 📙Михайло Драгоманов. Український публіцист, критик, історик, філософ,
фольклорист та літературознавець народився в Гадячі та навчався у Полтавській
гімназії. 

9.📘 Олена Пчілка. Українська письменниця, перекладачка та етнограф, чиї дитячі
проза та поезії цікаві й для сучасного читача, народилася в Гадячі.  

10. 📙Архип Тесленко. Український письменник, над чиїм «Школярем» плакав не
один сучасний школяр, народився в селі Харківці Лохвицького району. 

11.📗 Олесь Гончар. «Собори душ своїх бережіть!..» Автор знакового роману
«Собор» народився на Катеринославщині, але у віці трьох років його забрали до
себе на виховання дідусь та бабуся, що жили в селі Сухе Кобеляцького району.
Відтоді все життя Олесь Гончар вказував у автобіографіях та анкетах своє місце
народження саме село Сухе (колись слобода Суха). 

12-13. 📘📒Григорій та Григір Тютюнники. Брати-письменники, майстри прози.
Григір - започаткував новелістику 60-х, неодноразово бував та писав у Лубнах.
Обидва народилися в селі Шилівка Зіньківського району. 

14. 📕Василь Симоненко. Видатний поет-шестидесятник, автор легендарної поезії
«Лебеді материнства» народився в селі Біївці Лубенського району. 

15. 📙Павло Загребельний. Автор легендарних історичних романів «Диво»,
«Євпраксія», «Роксолана» народився в селі Солошине Кобеляцького району. 

16.📘 Борис Олійник. Український поет, автор зворушливої поезії «Пісня про
матір» народився в селі Зачепилівка Новосанжарського району.

Від себе додам, хоча Володимир Малик родом із Київщини, але саме в Лубнах були
написані його знамениті історичні романи.

Отож колиска наша неймовірна.

 🌺🌺🌺🌺🌺🌺🌺

💚🐿
