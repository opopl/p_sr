% vim: keymap=russian-jcukenwin
%%beginhead 
 
%%file 11_12_2020.news.ru.vesti.semenova_anna.1.sputnik_v_83_regiona_russia
%%parent 11_12_2020
 
%%url https://www.vesti.ru/article/2497705
 
%%author Семенова, Анна
%%author_id semenova_anna
%%author_url 
 
%%tags sputnik_v,russia,covid,vaccine
%%title "Спутник V" уже в 83 регионах России, "ЭпиВакКорона" пока в пяти
 
%%endhead 
 
\subsection{\enquote{Спутник V} уже в 83 регионах России, \enquote{ЭпиВакКорона} пока в пяти}
\label{sec:11_12_2020.news.ru.vesti.semenova_anna.1.sputnik_v_83_regiona_russia}
\Purl{https://www.vesti.ru/article/2497705}
\ifcmt
	author_begin
   author_id semenova_anna
	author_end
\fi

\index[rus]{Коронавирус!Вакцина!Спутник V, в 83 регионах России, 11.12.2020}

\video{https://player.vgtrk.com/iframe/video/id/2249429/start_zoom/true/showZoomBtn/false/sid/vesti/isPlay/true/mute/true/tid/2366/?acc_video_id=2359833}

\ifcmt
pic https://cdn-st1.rtr-vesti.ru/vh/pictures/xw/307/768/5.jpg
\fi

Несмотря на достаточно тяжелую эпидемическую обстановку в стране, вакцинация
поможет держать ситуацию под контролем, заявили в Роспотребнадзоре. А
премьер-министр Михаил Мишустин отметил, что Россия занимает лидирующие позиции
в мире по организации вакцинации.

Вторая российская вакцина "ЭпиВакКорона" выходит на новый этап испытаний. Об
этом на совещании с производителями российских вакцин сообщил Михаил Мишустин.

"Важная новость сегодня, начались пост клинические испытания новой
соответственно вакцины производства компании "Вектор" Роспотребнадзора. Их
результат будут подведены в январе. И тогда же начнется еще более широкое
применение этой вакцины. Ну, уже сегодня можно сказать, что Россия занимает
лидирующие позиции в мире по организации вакцинации людей", – сказал глава
кабмина.

50 тысяч доз уже произведено силами самого "Вектора". "В дальнейшем каждую
неделю до Нового года мы рассчитываем вводить в гражданский оборот по 10 тысяч
доз вакцины еженедельно. Для масштабирования нашего производства мы нашли
индустриального партнера – компанию "Герофарм", с помощью которой мы будем
производить один из ключевых компонентов нашей вакцины – это белок-носитель", –
отметил генеральный директор ГНЦ вирусологии и биотехнологии "Вектор" Ринат
Максютов.

"ЭпиВакКорона" уже есть в 5 регионах, кроме столицы и Петербурга – это
Ростовская, Тульская и Новосибирская области. Именно там вакцина нужна в первую
очередь

"Вакцина может быть доставлена в любой уголок Российской Федерации. даже в
самое отдаленное село. Температура хранения у нее от +2 до +8 градусов. И это
позволяет нам использовать существующую ходовую цепь для доставки вакцины и
обеспечит все логистические процессы", – заявила руководитель Роспотребнадзора,
главный государственный санитарный врач РФ Анна Попова.

Перевозка первой российской вакцины "Спутник V" требует особых условий. В
Адыгею в термоконтейнерах сегодня доставили первую партию – 200 доз. В
Хабаровск привезли 400 доз, как и везде, в первую очередь вакцина адресована
врачам. Еще тысячу доставили в столицу Урала, где число зараженных неуклонно
растет.

Всего же в декабре регионы получат 480 тысяч доз. Уже сформирован график
распределения препарата по стране. Даже в новогодние праздники прививочные
пункты должны работать, подчеркивают в Белом доме.

"Особенно, с 1 по 10 января, чтобы все желающие, кто захотят это сделать, могли
сделать прививку. Как вы помните, по поручению главы государства у нас в первую
очередь такая возможность должна быть обеспечена для учителей, работников
медицинский и социальных учреждений. Тех, кто сегодня на передовой", – сказал
глава правительства.

"Спутник V" сейчас есть в восьмидесяти трех регионах из восьмидесяти пяти. В
Магадан препарат привезут в воскресенье, на Чукотку в понедельник.

"Президентом РФ была поставлена задача до конца этой недели обеспечить доставку
во все регионы вакцины "Спутник V". Задача эта выполнена", – сказал министр
промышленности и торговли РФ Денис Мантуров.

Одна из крупнейших производственных площадок, где будут производить вакцину
"Спутник V" – в Москве. Строительство должно завершиться до Нового года.
Полторы тысячи человек работают круглосуточно.

"Проложено порядка 70 километров различных коммуникаций, подведены газ, вода,
электроэнергия, построена парогазовая котельная, ну, и смонтированы сами
корпуса. Сейчас идет полным ходом монтаж оборудования", – отметил мэр Москвы
Сергей Собянин.

Технологический процесс – начинается с маленькой колбочки, а завершается
литрами вакцины. 156 волновых биоконтейнеров – в них выращивают основу
препарата – клеточные культуры.

Перед тем как запустить эту технику ее проверяют по множеству параметров. По
нормативу каждый такой биоконтейнер должен качаться со скоростью от двух до 30
раз в минуту, чтобы клетки перемешивались в нужном темпе.

Выйдя на полную мощность, производители обещают выпускать в месяц по 10
миллионов доз "Спутника V".
