% vim: keymap=russian-jcukenwin
%%beginhead 
 
%%file 22_10_2020.news.ua.obozreval.1.ivan_bunin_150
%%parent 22_10_2020
%%url https://www.obozrevatel.com/russia/segodnya-ivanu-buninu-150-on-nauchil-menya-russkomu-yazyiku.htm
%%tags ivan bunin,poezia,russia
 
%%endhead 

\subsection{Сегодня Ивану Бунину --- 150: он научил меня русскому языку}
\label{sec:22_10_2020.news.ua.obozreval.1.ivan_bunin_150}

\url{https://www.obozrevatel.com/russia/segodnya-ivanu-buninu-150-on-nauchil-menya-russkomu-yazyiku.htm}

\ifcmt
img_begin 
	tags ivan bunin,russia,poezia
	url https://i.obozrevatel.com/news/2020/10/22/5.jpg?size=972x462
	caption Иван Бунин
	width 0.7
img_end
\fi

Сегодня, 10 октября по старому стилю. Сегодня 150 лет назад в мебелированных
комнатах комнатах в Воронеже родился Иван Бунин. 150 лет назад - целая
вечность. А, между тем, мне посчастливилось быть его современником. Целый год,
даже чуть больше мы жили вместе на этой земле. Я - младенец, он - старик.

Наверное мне родители читали какие-то его стихи, которые обычно читают детям -
Листопад, "и шмели, и колосья"... Не помню. Но в отроческом возрасте Бунин
обрушился на меня как ураган, смел меня и сделал иным. В эти годы выходило его
"красное" девятитомное собрание сочинений. Низкий поклон Твардовскому,
"пробившему" его в ЦК и Главлите. Первые тома прошли мимо меня. Но вот в "Доме
Книге" на Новом Арбате я подошел к прилавку подписной литературы (на первом
этаже по центру) и попросил седьмой том - "Темные аллеи". Я открыл наугад и не
помню сейчас точно, какой мне выпал рассказ, кажется - "Холодная осень".
Оторваться уже не мог. Купил все, имевшиеся в наличии тома, потом правдами и
неправдами собрал остальные.

И я, и мой старший брат Сергей ушли в Бунина, жили им, перечитывали по много
раз, так что некоторые рассказы я могу и сейчас воспроизводить страницами на
память, например, любимое, про Марка Аврелия, "Возвращаясь в Рим".

Бунин научил меня русскому языку, его совершенной красоте, простоте и
выразительности, влюбил меня в мой родной язык на всю жизнь. Он научил меня
ценить импрессию более сюжета, описание чувств - более катарсиса. Впрочем, у
Бунина само описание и есть катарсис.

Бунин открыл мне Россию. Её убогость и её безмерную красоту, её бескрайность,
бездонность и ее всецелое явление в бесчисленных мелочах жизни, природы,
архитектуры, говора. Он подвел меня к Богу через свою именно простую и верную
веру, которую не удалось разрушить врагу никакими соблазнами и падениями.

Только никогда не оставлявший Бога человек мог уйти из жизни со словами:

Ледяная ночь, мистраль
(Он еще не стих).
Вижу в окна блеск и даль
Гор, холмов нагих.
Золотой недвижный свет
До постели лег.
Никого в подлунной нет,
Только я да Бог.
Знает только Он мою
Мертвую печаль,
Ту, что я от всех таю...
Холод, блеск, мистраль

Его приятия и отталкивания, его отношение к захлестнувшему Россию большевизму и
к воинам Белого Дела, боровшимся против разливавшейся по стране стихии зла,
стало и моим виденьем трагического прошлого моего отечества и моих отцов. А сам
Иван Алексеевич - образом непокорившегося злу русского человека, образом,
ставшим примером для меня.

В дописанном в Одессе под большевиками стихотворении, которое начинал он еще в
иной, настоящей России, Бунин сказал, вспоминая евангельский образ мудрых
невест Христовых : "немного нас, елей хранивших, для тьмы, обещанной Тобой,
немного верных, не забывших, что встанет день над этой тьмой". И я старался
собрать этот елей памяти и любви, умножить и сохранить его, и я верил Бунину,
что "встанет день над этой тьмой", верил тогда, когда студентом читал том
"Литературного Наследства" с этим стихотворением, верю и сейчас, на склоне лет.
Как верил и он.- "Ей, гряди!"

Бунин открыл мне тайну и чудо любви к женщине, научил видеть в подруге
"удивительное, никогда не познаваемое и неисчерпаемое". Как и в смерти, в любви
всегда тайна, бездна, далеко выходящая за область плотской чувственности, но в
ней раскрывающаяся с обыденной, и потому головокружительной простотой.

Его проникновение в иные культуры - буддизм, ислам, индуизм - в иные формы
божественного столь же глубоки и верны для меня, как и проникновения в человека
- от орловского мужика-пьяницы, жаждущего красоты и смысла жизни, до мудрейшего
императора Рима, познавшего эти красоту и смысл в свои последние, предсмертные
часы на Антибской вилле "Очаг".

И так совпало, поверьте, совершенно случайно, что именно в день рождения
Бунина, 38 лет назад мы с женой венчались в одном из московских храмов.
Венчались тайно, при закрытых дверях, без гостей, без хора. Это были времена
суровых гонений на Церковь. Но Бунин научил меня, что возлюбленная - это
"прекраснейшая солнца" и что бездонность человеческой, супружеской любви
сознается и раскрывается только в беспредельности Божественной красоты и добра.

И тот холодный, осенний московский день через все годы помнится мне кипением
золотых блесток в воске горящих венчальных свечей, в простых вопросах о
верности дорогого нашего батюшки, отца Георгия, и в той единственной,
прекрасной, стоявшей рядом со мной и пившей со мной одну жизнь из одной чаши
тогда впервые, а ныне уже 38 лет.

Связывающий судьбы и даты, Правящий миром с любовью и нежностью, Он соединил
нашу жизнь с душой Ивана Бунина и в тот день и навеки.

"Скажи поклоны князю и княгине, целую руку детскую твою о той любви, которую
отныне ни от кого я не таю"...

Редакция сайта не несет ответственности за содержание блогов. Мнение редакции
может отличаться от авторского.
