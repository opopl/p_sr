% vim: keymap=russian-jcukenwin
%%beginhead 
 
%%file 24_11_2021.fb.dzhangirov_dmitrij.kiev.1.maidan_evrosud
%%parent 24_11_2021
 
%%url https://www.facebook.com/permalink.php?story_fbid=1315792698866137&id=100013062576677
 
%%author_id dzhangirov_dmitrij.kiev
%%date 
 
%%tags janukovich_viktor,kiev,maidan2,revgidnosti,sud,sud.evropejskij.prava_cheloveka,ukraina
%%title МАЙДАН ДОШЕЛ ДО ЕВРОПЕЙСКОГО СУДА
 
%%endhead 
 
\subsection{МАЙДАН ДОШЕЛ ДО ЕВРОПЕЙСКОГО СУДА}
\label{sec:24_11_2021.fb.dzhangirov_dmitrij.kiev.1.maidan_evrosud}
 
\Purl{https://www.facebook.com/permalink.php?story_fbid=1315792698866137&id=100013062576677}
\ifcmt
 author_begin
   author_id dzhangirov_dmitrij.kiev
 author_end
\fi

МАЙДАН ДОШЕЛ ДО ЕВРОПЕЙСКОГО СУДА

Официальный Киев обязали ответить на иск ЯНУКОВИЧА 

Очередная годовщина Майдана – это всегда гремучая смесь скорби, выяснения
отношений с властью в привычном дискурсе «зрада / перемога», поиск виновных во
всех бедах, постигших Украину после 2013 года и, в обязательном порядке –
обновление кейсов по обвинению экс-президента Виктора ЯНУКОВИЧА сотоварищи в
расстреле Майдана и сдаче Крыма. Вот и сейчас офис генпрокурора сообщил, что
завершено досудебное расследование в отношении экс-президента Виктора Януковича
и девяти высокопоставленных должностных лиц в деле о расстрелах на Майдане, а
лично генпрокурор Ирина ВЕНЕДИКТОВА заявила на брифинге 17 ноября, что всем
подозреваемым грозит пожизненное заключение. 

Однако, в традиционном ноябрьском прокурорском обострении в этом году имеется и
новый момент. Действия генпрокуратуры в данном случае вполне укладываются в
принцип «лучшая защита – это нападение».

От кого же защищается наша правоохранительная система? От Европейского Суда по
правам человека – ЕСПЧ.   

ЕСПЧ никогда ни придавал искам ЯНУКОВИЧА и его соратников приоритетного
статуса, а потому их рассмотрение шло медленно и печально. Медленно – для
истцов, печально – для украинской судебной системы. 

И вот «дело ЯНУКОВИЧА» в ЕСПЧ прошло все предварительные этапы, и суд – в
рамках рутинной процедуры – потребовал от украинской власти разъяснений по
основным пунктам иска. А это уже очень печальная новость для отечественных
правоохранителей, ибо, с одной стороны, до этой стадии доходит только один иск
из десяти, а с другой, абсолютно «топорная» работа на всех уровнях –
следователей, прокуроров и судей в этом деле общеизвестна. Уровень
доказательств прокурорских эпохи ЛУЦЕНКО – «ниже плинтуса».  

Собственно говоря, главные претензии состоят в нарушении ст. 6 Европейской
конвенции о правах человека – «Право на справедливое судебное разбирательство».
С процедурной точки зрения речь идет о недопуске ЯНУКОВИЧА к судебным процессам
по видеосвязи плюс множество других нарушений на всех стадиях процесса, включая
нарушение презумпции невиновности. 

Кроме того, адвокаты ЯНУКОВИЧА заявляют, что приговор Оболонского суда был
написан в Администрации президента ПОРОШЕНКО; во всяком случае, как мы помним,
сайт «Страна» опубликовал текст приговора еще до его оглашения. 

Но, пожалуй, что еще важнее – украинские суды отказались рассматривать другие
версии бойни на Майдане. 

Имеются публикации и серьезные свидетельства того, что обстрел начала группа во
главе с Зиновием ПАРАСЮКОМ – отцом будущего депутата Владимира ПАРАСЮКА. И если
семья ПАРАСЮКОВ хотя бы отрицает эти обвинения, то львовянин Иван БУБЕНЧИК еще
в 2016 году в интервью изданию Bird In Flight с гордостью заявил, что это
именно он начал убивать сотрудников «Беркута». Такая наглость буквально
вынудила следователей открыть дело, но Генеральный прокурор Юрий ЛУЦЕНКО лично
не позволил вручить подозрение и закрыл дело БУБУНЧИКА.

Трое граждан Грузии сделали признание, что это они вели огонь и по
протестующим, и по «Беркуту». Так, бывший военнослужащий грузинской армии Коба
НЕРГАДЗЕ рассказал, что снайперы в Киев приехали благодаря усилиям
экс-советника бывшего президента Грузии Михаила СААКАШВИЛИ – Мамука
МАМУЛАШВИЛИ, а непосредственно в расстреле Майдана участвовали депутат Сергей
ПАШИНСКИЙ, отец и сын ПАРАСЮКИ, а инструктором выступал бывший американский
военнослужащий Кристофер БРАЙАН. Однако, эти заявления даже формально не
расследовались.

Понятно, что фильмы с версиями, альтернативными официальной –производства
Израиля, Италии и США – просто игнорировались. 

Не хочется забегать вперед, но если ЕСПЧ примет аргументы адвокатов ЯНУКОВИЧА,
то это будет означать не только моральную делигитимацию ТУРЧИНОВА, ПАРУБИЯ,
ЯЦЕНЮКА и прочих, но и лавину исков от семей погибших на Майдане «беркутовцев»
и солдат внутренних войск, а также различных чиновников, проходивших по «делу
Майдана». 

Ну, а пересмотр «дела ЯНУКОВИЧА» поднимет такие пласты неудобной информации,
что «пленки МЕЛЬНИЧЕНКО» или «Вагнергейт» для лидеров Майдана и постмайданной
Украины покажутся доброй сказкой дедушки ПАНАСА.  

Для понимания слабости юридических позиций украинских правоохранителей, хочется
сослаться на «свежайшую» пресс-конференцию адвоката Виктора ЯНУКОВИЧА – Виталия
СЕРДЮКА от 23 ноября.    

Господин СЕРДЮК заявил: «Интерпол еще в 2017 году высказал свое мнение по
данным фактам, признав политическим преследованием ЯНУКОВИЧА. Европейский суд
каждый год – в связи с фактами отсутствия справедливого суда и доступа к
правосудию, отменяет санкции вводимые ЕС, называя их незаконными. Мы далеки от
справедливости в данных процессах и явная политическая заангажированность
является тем якорем, который тянет на дно правду в делах Майдана и в делах о
потере Крыма».

Между прочим, подробности о возбуждении дела за сдачу Крыма, о которых напомнил
адвокат СЕРДЮК можно было бы назвать вопиющими, если бы они не были бы столь
типичны для периода президентства Петра ПОРОШЕНКО и прокурорства его кума –
Юрия ЛУЦЕНКО. Кстати, не все еще забыли, что под назначение ЛУЦЕНКО Генеральным
прокурором был изменен закон, а, точнее, отменена норма, носящая характер
аксиомы: главным обвинителем страны должен быть человек с профильным
образованием и стажем прокурорской работы. Вместо этого был назначен человек с
техническим образованием и стажем комсомольской работы. 

Виталий СЕРДЮК заявил, что «дело о сдачи Крыма возникло между допросами
Януковича в 2016 года по делу Беркута. Следователи генпрокурора ЛУЦЕНКО за два
выходных дня подготовили заведомо ложный, заведомо неправдивый текст сообщения
о подозрении. (…) Данное дело было начато путем зачитывания выключенному
телевизору в помещении Святошинского районного суда текста подозрения. Причем
все понимали, что ЯНУКОВИЧ отсутствует, но жажда пиара, но жажда перебить
информационную картинку у ЛУЦЕНКО преобладала».

От себя отмечу, что по делу о «сдаче Крыма» до сих пор не получили ответ
следующие вопросы:    

i. почему и.о. президента Украины Александр ТУРЧИНОВ назначил 1 марта 2014 года
командующим Военно-морским флотом Украины именно Дениса БЕРЕЗОВСКОГО, который
сразу же издал приказ не оказывать сопротивления и сдавать оружие, а уже 2-го
марта стал командующим военно-морскими силами Автономной республики Крым –
вплоть до 18 марта, дня референдума. После чего стал заместителем командующего
Черноморского флота России. Согласитесь, весьма знаковые переходы турчиновского
назначенца.

ii. Почему командующий внутренними войсками Украины Степан Полторак приказал
выводить воинские части из Крыма «с гордо поднятыми знаменами»? Боюсь даже
уточнить, как сочетается сдача территории с «гордо поднятыми знаменами»?

iii. Почему СНБО под руководством мысленно беспощадного к врагам Андрея ПАРУБИЯ
отказалось рассматривать предложение и.о. министра обороны – адмирала Игоря
ТЕНЮХА «идти на прорыв»;

iv. Почему и.о. президента ТУРЧИНОВ поддержал приказ о выводе войск из Крыма
без единого выстрела?

Я сейчас не даю оценку всем этим решениям, возможно, они были мудры и
человеколюбивы. Я даже готов поверить, что именно в этот момент турчиновы и
парубии израсходовали весь свой запас мудрости и человеколюбия на много лет
вперед.

Но, согласитесь, даже простая постановка этих вопросов заставляет нас совсем
другими глазами смотреть на проблему «так кто же сдал Крым?»   

Кстати, напоследок хотел бы уточнить и для себя: правду ли говорит тогдашний
литовский посол в Киеве Пятрас ВАЙТЕКУНАС, что те три недели (перед
референдумом) которые он по поручению послов ЕС провел в Крыму, он выполнял
задачу ЦИТАТА «контролировать недопущение пролития крови и эскалации
конфликта»? Господин ВАЙТЕКУНАС считает, что справился с этой задачей, но мне
сложно представить, что он каким-то образом мог влиять на российскую сторону.
Значит, методом исключения, он от имени ЕС курировал сдачу Крыма Украиной? 

Интересно бы приобщить и его показания к этому многообещающему расследованию.

\ii{24_11_2021.fb.dzhangirov_dmitrij.kiev.1.maidan_evrosud.cmt}
