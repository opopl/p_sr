% vim: keymap=russian-jcukenwin
%%beginhead 
 
%%file 19_02_2022.fb.baumejster_andrej.kiev.filosof.1.disciplina_emocij
%%parent 19_02_2022
 
%%url https://www.facebook.com/andriibaumeister/posts/4778530298935158
 
%%author_id baumejster_andrej.kiev.filosof
%%date 
 
%%tags emocii,myshlenie,napadenie,rossia,ugroza,ukraina
%%title Как раз время учиться проявлять дисциплину эмоций
 
%%endhead 
 
\subsection{Как раз время учиться проявлять дисциплину эмоций}
\label{sec:19_02_2022.fb.baumejster_andrej.kiev.filosof.1.disciplina_emocij}
 
\Purl{https://www.facebook.com/andriibaumeister/posts/4778530298935158}
\ifcmt
 author_begin
   author_id baumejster_andrej.kiev.filosof
 author_end
\fi

Как раз время учиться проявлять дисциплину эмоций, дисциплину воли и дисциплину
мышления.

- Как можно меньше хаоса в эмоциях. Эмоциональный хаос и панические настроения
только мешают принимать правильные решения и делать нужные шаги. В таких
ситуациях это не всегда просто. Но это необходимо. 

- Воля к максимально концентрированному решению конкретных проблем, которые
появляются в каждый конкретный момент. Никаких метаний, гаданий, истерических
обвинений друг друга. Это только блокирует волю. Сейчас нужна воля к
солидарности и к общему взаимодействию граждан единой страны. 

- Дисциплина ума, без ложных умствований \enquote{мы же знали}, \enquote{мы так и думали} и
прочей интеллектуальной шелухи. Это патологии объяснений.

Чтобы ни случилось, толпы умников тут же устраивают \enquote{мыжезнамсы}.

Если конфликт остановится на крайней черте, нам будут говорить, \enquote{мы так и
думали, \enquote{они} давно обо всем договорились за нашей спиной}.

Если конфликт разгорится, нам будут говорить, \enquote{мы так и знали, что враг
нападёт} и \enquote{его ничто не остановит}. 

Если будут заключены компромиссы, нам будут говорит, что это результат
неимоверных усилий наших западных партнёров.

Все эти умствования задним числом только усиливают туман и хаос.

Важно концентрироваться на том, что реально произошло в каждый конкретный
промежуток времени. Что именно сказано в данный момент теми, кто принимает
решения. Видеть в цепочках утверждений и конкретных шагов логическую
последовательность для того, чтобы строить адекватные прогнозы. И только на
этом основании делать рациональные заключения и принимать взвешенные решения.

Всё время не забывая практиковать дисциплину эмоций, дисциплину воли и
дисциплину мышления.
