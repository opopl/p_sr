% vim: keymap=russian-jcukenwin
%%beginhead 
 
%%file 07_01_2022.stz.news.ru.ria_kursk.1.novyj_god
%%parent 07_01_2022
 
%%url https://riakursk.ru/multfilmy-yelki-i-risovanie-do-utra-kak-znamenitye-kuryane-otmechayut-novyy-god
 
%%author_id 
%%date 
 
%%tags kursk,novyj_god,obschestvo,prazdnik,rossia,zhizn
%%title Мультфильмы, ёлки и рисование до утра: как знаменитые куряне отмечают Новый год
 
%%endhead 
 
\subsection{Мультфильмы, ёлки и рисование до утра: как знаменитые куряне отмечают Новый год}
\label{sec:07_01_2022.stz.news.ru.ria_kursk.1.novyj_god}
 
\Purl{https://riakursk.ru/multfilmy-yelki-i-risovanie-do-utra-kak-znamenitye-kuryane-otmechayut-novyy-god}

\begin{zznagolos}
Курские политики и творческие люди — о новогодних и рождественских традициях
\end{zznagolos}

\ii{07_01_2022.stz.news.ru.ria_kursk.1.novyj_god.pic.1}

Новый 2022 год вступил в свои права, и праздничные дни в разгаре. В это время
важно как можно больше времени уделять своим близким.  У многих курян
существуют семейные и личные традиции, как встречать Новый год и Рождество. И
знаменитости — не исключение. Мы попросили курских политиков и музыкантов
рассказать, как принято праздновать в их семьях.

\subsubsection{Мандарины и ёлка}

Виктор Карамышев, заместитель губернатора Курской области:

\ii{07_01_2022.stz.news.ru.ria_kursk.1.novyj_god.pic.2}


\begin{zzquote}
— Моё детство пришлось на 70-е годы. Родители много работали, кроме того, нужно
было следить за хозяйством, я рос в сельской местности. Воспитывали и опекали
меня старшие брат и сестра. Очень яркое воспоминание из детства: когда я шёл на
новогодний утренник в первом классе, костюм мне мастерили всей семьёй.

Такую дружную обстановку, заботу друг о друге мы стараемся сохранить и сейчас.
Уже выросли мои дети, у старшего сына своя семья, однако в Новый год мы
обязательно собираемся все вместе, празднуем его дома.

А ещё с самого детства Новый год, как и у многих, наверное, ассоциируется с
запахом мандаринов и хвои. Мы и сейчас обязательно всей семьёй наряжаем
новогоднюю ёлку.
\end{zzquote}

\subsubsection{Мультфильмы из детства}

Сергей Проскурин, руководитель и дирижёр Губернаторского камерного оркестра:

\begin{zzquote}
— В Новый год я смотрю детские мультфильмы. Вот сколько себя помню, столько
и следую этой традиции. В детстве мы могли смотреть мультики только по
телевизору, но это всегда для нас было настоящим праздником.

Сейчас наши дети — и мы вместе с ними — можем выбирать какую угодно технику
и любые на вкус мультфильмы. Но я, конечно, больше всего люблю мультфильмы
своего детства. Чего стоит, например, новогодний выпуск «Ну, погоди!» или
хоккейная дёжкинская эпопея «Шайбу, шайбу!» и «Матч-реванш»...

Интересно, что это никогда не надоедает. Наоборот, если в детстве,
например, в таком роскошном мультфильме как «Маугли» мы воспринимали только
саму историю, то с возрастом там можно наслаждаться ещё и прекрасной
музыкой Софии Губайдулиной. Но самое важное состоит в том, чтобы всегда
оставаться немного детьми — и Новый год нам весьма в этом помогает.
\end{zzquote}
