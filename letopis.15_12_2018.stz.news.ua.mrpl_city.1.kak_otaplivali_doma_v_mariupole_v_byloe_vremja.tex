% vim: keymap=russian-jcukenwin
%%beginhead 
 
%%file 15_12_2018.stz.news.ua.mrpl_city.1.kak_otaplivali_doma_v_mariupole_v_byloe_vremja
%%parent 15_12_2018
 
%%url https://mrpl.city/blogs/view/kak-otaplivali-doma-v-mariupole-v-byloe-vremya
 
%%author_id burov_sergij.mariupol,news.ua.mrpl_city
%%date 
 
%%tags 
%%title Как отапливали дома в Мариуполе в былое время
 
%%endhead 
 
\subsection{Как отапливали дома в Мариуполе в былое время}
\label{sec:15_12_2018.stz.news.ua.mrpl_city.1.kak_otaplivali_doma_v_mariupole_v_byloe_vremja}
 
\Purl{https://mrpl.city/blogs/view/kak-otaplivali-doma-v-mariupole-v-byloe-vremya}
\ifcmt
 author_begin
   author_id burov_sergij.mariupol,news.ua.mrpl_city
 author_end
\fi

\ii{15_12_2018.stz.news.ua.mrpl_city.1.kak_otaplivali_doma_v_mariupole_v_byloe_vremja.pic.1.kuhnja_s_plitoj}

Бывает же так, что в голову приходит шальная мысль, совершенно не относящаяся к
сегодняшнему дню. Например, чем отапливали жилища в прошлом обитатели наших
мест? Что сжигали в своих очагах, чтобы согреться казаки Кальмиусской паланки?
Каким топливом располагали греки, переселенные из Крыма? Как решал топливную
проблему прочий пришлый люд – отставные солдаты и матросы, поселившиеся в 40-х
годах XΙX века у берега моря, образовав Слободку? А украинцы, русские, евреи,
поляки и представители иных народов, которым было дозволено после 30 марта 1859
года постоянно проживать в самом Мариуполе?

\ii{15_12_2018.stz.news.ua.mrpl_city.1.kak_otaplivali_doma_v_mariupole_v_byloe_vremja.pic.2.pechka}

Ответ на этот вопрос не удалось найти в довольно обширной мариупольской
краеведческой литературе. Что ж, придется домысливать. Итак, наш город был
основан в безлесной зоне. До ближайших лесов сотни верст. Велико-Анадольское
лесничество и лесной массив близ Володарского – не в счет. Это искусственные
насаждения, первое заложено в 1834 – 1845-м, а второе – в 1876 г. Кстати, к
этому приложили силу немцы-колонисты. Следовательно, на дрова рассчитывать не
приходилось. Каменный уголь? Залежи угля в нынешнем Донбассе были найдены в
первом двадцатилетии XVΙΙΙ века. Добыча его поначалу была крайне мала и не
имела товарного значения, добывать донецкий уголь в промышленных масштабах
стали только конце XIX века.

\ii{15_12_2018.stz.news.ua.mrpl_city.1.kak_otaplivali_doma_v_mariupole_v_byloe_vremja.pic.3.russkaja_pech}

\textbf{Читайте также:} \href{https://archive.org/details/08_12_2018.sergij_burov.mrpl_city.svetilniki}{Светильники: когда в Мариуполе не было электроэнергии}

Но люди в Приазовье, тем не менее, как-то отапливали зимой свои жилища, на
чем-то готовили пищу. И тут вспомнилось детство, военное время, проведенное в
селе близ Мариуполя у маминой тетушки Дарьи Силовны. Хатка под соломенной
крышей, пристройка к ней – коровник. В нем корова Марта, кормилица бабушки
Даши. Для Марты устраивалось ложе - подстилка из соломы. Время от времени
хозяйка Марты выгребала вилами подстилку, превратившуюся в навоз, во двор. В
этот \enquote{продукт жизнедеятельности коровы} добавлялась мелко насеченная солома,
все перемешивалось, полученная масса набивалась в деревянную рамку – форму.
Форма снималась, ее снова набивали массой, состав которой приведен выше. Так
форма за формой и почти все пространство двора покрывалось рядами квадратных
пластин \enquote{кизяка}. Именно так называется как исходный, так и конечный продукты
описанного здесь материала. Когда пластины кизяка подсыхали, набрав достаточную
прочность, из них составлялись клетки, наподобие тех, что складывают из
игральных карт. Пластины находились под воздействием ветра и летнего палящего
солнца до наступления дождей, а потом прятались в клуню, по определению Гугла –
в хозяйственную постройку для молотьбы и хранения хлеба. Кизяк был топливом
русской печки, которая обогревала хату в холодное время года и одновременно
служила для приготовления пищи. Кроме кизяка в качестве топлива использовались
стебли и выбитые шляпки подсолнухов, стебли кукурузы, \enquote{одежка} ее початков и
кочерыжек, картофельная ботва, скошенный бурьян, подобранные на дороге высохшие
коровьи \enquote{лепешки} и \enquote{яблоки} лошадей. Конечно, \enquote{решение топливной проблемы} не
бабушкой Дашей было придумано. Видно, пришло оно от первопоселенцев Приазовской
степи, в том числе и населявших наш город.

Можно предположить, что в середине XΙX века могли доставлять каменный уголь в
Мариуполь волами на возах. Вероятно, целиком обитатели нашего славного города
перешли на отопление углем с момента, когда Екатерининская железная дорога была
продлена от станции Еленовка до нашего города. А произошло это знаменательное
событие в 1882 году. Для любознательных читателей приведем отрывок из очерка
Д. А. Хараджаева, помещенного в сборник \enquote{Мариуполь и его окрестности}: \enquote{\em До
проведения железной дороги в Мариуполь хлеб и все товары привозились и
вывозились морем от пристани и сухим путем – гужом волами. В одном из прошений,
поданных на Высочайшее имя, говорится, что воловьих подвод в год проходило
через Мариуполь до 50 тысяч}.

\textbf{Читайте также:} 

\href{https://mrpl.city/blogs/view/de-znajti-naturalni-ta-ekologichni-produkti-yak-v-seli}{%
Де знайти натуральні та екологічні продукти \enquote{як в селі}, Маріанна Бойко, mrpl.city, 14.12.2018}

Пожалуй, с конца XΙX века в нашем городе появились печи-голландки. Справочники
дают следующее определение этим тепловым агрегатам: \enquote{Голландская печь —
изначально облицованная кафелем или изразцами прямоугольная печь для обогрева
комнат, имеющая вертикальные дымообороты, и за счет этого высокую отдачу
тепловой энергии}. Печи эти были в домах многих зажиточных мариупольцев. Но
более распространенными были печи с грубкой - сочетание очага, накрытого
чугунной плитой, с вертикальными дымоходами, спрятанными в стену. На очаге, или
плитке, как называют его у нас в Мариуполе, готовили пищу и грели воду.
Впрочем, и сейчас в кварталах с одноэтажной застройкой можно найти такие
плитки.

\textbf{Собственные воспоминания послевоенных лет.} Топили печку углем, для
растопки применяли древесную щепу. Топливо получал папа на заводе, где он
работал. Уголь был разным. Лучший – антрацит, его сразу можно было узнать по
блестящим сколам.  Полведра антрацита обеспечивало квартиру теплом сутки, даже
при морозе. Вторым по привлекательности был кокс-\enquote{орешек}. Но
\enquote{орешек} не так часто попадался.  Выдавали кокс в виде кусков с кулак и
больше. Дробить эти куски было мучением.  Но норма топлива, положенная
работникам завода, конечно, состояла не только из антрацита и кокса, добрую
половину ее \enquote{отоваривали} так называемым курным углем. В нем было много
пустой породы, соответственно, и шлака больше, жужалки – по-мариупольски. Золу,
просыпавшуюся сквозь колосники в поддувало, периодически выбирали совком, пока
топилась печка. Утром, перед тем, как растопить печку, выбирался из топки шлак.
Спекшиеся крупные куски выбрасывали сразу, а мелочь просеивали с помощью
решета. Тем, что проваливалось сквозь решето, посыпали тротуар при гололеде. Из
того, что оставалось на решете, выбирали несгоревшие кусочки угля, чтобы снова
засыпать их в печку.

\textbf{Читайте также:} \href{https://mrpl.city/news/view/na-donetchine-lesorubu-grozit-do-5-let-lisheniya-svobody}{%
На Донетчине лесорубу грозит до 5 лет лишения свободы, Олена Онєгіна, mrpl.city, 14.12.2018}

Рассказывая о печах, грешно не упомянуть их создателей-печников, людей весьма
уважаемых. Автору этих строк известны, к сожалению, только трое из них.
Григорий Рябов и Голощуков, имя его выветрилось из головы, а Антона Зандера
довелось видеть неоднократно. И когда он клал печь в родительском доме,
пострадавшем при бомбежке, и восстановленном только что, и когда приходил
трусить сажу из печи. Антон был педантичен и аккуратен. Впрочем, тут нет ничего
удивительного – он был уроженцем одной из немецких колоний, которых в Приазовье
было много. В сентябре 1941 года он разделил участь своих соплеменников, и был
выселен с семьей за Урал. Зандер вернулся в Мариуполь только к середине 50-х
годов. На улицах, прилегающих к старому базару, он был нарасхват.

К середине XX века центр Мариуполя пришел с пятью трехэтажными домами. Это
Дворец труда (бывшая гостиница \enquote{Континенталь}), школа №3 (бывшее Реальное
училище Гиацинтова), музей краеведения (бывший учебно-ремесленный дом-приют
инвалидов войны), гостиница \enquote{Спартак} и управление 238-го стрелкового полка
(бывшее Епархиальное училище, а теперь первый корпус Приазовского
государственного технического университета). Зданий с бóльшим количеством
этажей не было. Как отапливались четыре первых названных здания, неизвестно. А
вот о том, как обогревалось Епархиальное училище, довелось узнать от кандидата
технических наук, доцента ПГТУ Сергея Сергеевича Данилова, большого знатока
истории нашего города. Оказывается, Епархиальное училище обогревалось
централизованно. Во дворе было специальное помещение, где топили большую печь
(топку), а нагретый воздух (дымовые газы) проходили по дымоходам под полами.

\textbf{Читайте также:} \href{https://mrpl.city/news/view/vizitnuyu-kartochku-mariupolya-otrestavriruyut-v-2020-godu-foto}{%
Визитную карточку Мариуполя отреставрируют в 2020 году, Роман Катріч, mrpl.city, 07.12.2018}

В средине тридцатых годов прошлого столетия в Мариуполе появились первые
пятиэтажки. Вот их современные адреса: \emph{улица Архитектора Нильсена, №1и №39,
проспект Мира, 43 и 45}. Каким образом обогревались квартиры в этих домах?
Старожил нашего города Георгий Сергеевич Котельников, который с довоенных лет
жил в доме №1 по улице Энгельса вспоминал: \enquote{\em У нас в квартире на кухне была
печь. Я сам приносил уголь из подвала, сам выносил золу. Было центральное
отопление, котельная находилась в подвале}.

Валентина Александровна Кочеткова рассказала: \enquote{\em В доме №45 по проспекту Ленина
до сих пор остались дымоходы печей для отопления углем. Во дворе были сараи для
угля, теперь на их месте гаражи}.

Интересно, что наличие печей для приготовления пищи в многоэтажных домах,
которые отапливались углем или дровами, не было характерно только для
Мариуполя. Вот цитата из Интернета: \enquote{\em В Ленинграде жилые дома (в том числе новые
образцово-показательные районы массовой застройки) вплоть до середины 30-х
годов строили с печным отоплением, по той причине, что для централизованного
парового отопления не было ни соответствующей инфраструктуры, ни материалов и
специалистов для ее создания. В каждом доме были либо дровяные сараи во
внутренних дворах, либо специальные поквартирные отсеки в подвалах для хранения
запаса дров. Причем надо понимать, что дрова использовались не только для
отопления зимой, но и постоянно для готовки пищи, поскольку массовая
газификация города началась уже после Великой Отечественной войны}. 

Газификация городских объектов в Мариуполе началась в 1939 году, когда
предприятием \enquote{Донюжгаз} было проложено 6 километров газовых сетей от
коксохимического завода - и в городскую сеть начал подаваться коксовый газ.
Когда Ставропольское газовое месторождение было введено в эксплуатацию, а это
произошло в 1956 году, постепенно начали переводить тепловые агрегаты заводов и
жилье на природный газ.

\textbf{Читайте также:} 

\href{https://mrpl.city/news/view/novogodnij-podarok-otoplenie-dlya-mariupoltsev-podorozhaet-s-1-yanvarya}{%
Новогодний подарок. Отопление для мариупольцев подорожает с 1 января, Денис Росін, mrpl.city, 11.12.2018}

Вот и все, что удалось домыслить, вспомнить или узнать об отоплении домов в нашем городе.
