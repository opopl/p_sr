% vim: keymap=russian-jcukenwin
%%beginhead 
 
%%file 28_07_2020.fb.lesev_igor.1.sudebnaja_reforma_planeta_obezjan
%%parent 28_07_2020
 
%%url https://www.facebook.com/permalink.php?story_fbid=3403647442999668&id=100000633379839
 
%%author_id lesev_igor
%%date 
 
%%tags reformi,strana,sudebnaja_sistema,ukraina
%%title Судебная реформа на планете обезьян
 
%%endhead 
 
\subsection{Судебная реформа на планете обезьян}
\label{sec:28_07_2020.fb.lesev_igor.1.sudebnaja_reforma_planeta_obezjan}
 
\Purl{https://www.facebook.com/permalink.php?story_fbid=3403647442999668&id=100000633379839}
\ifcmt
 author_begin
   author_id lesev_igor
 author_end
\fi

Судебная реформа на планете обезьян

Смотрел на днях британский фильм по одноименному роману Ивлина Во «Пригоршня
праха». Чуть тяжеловесно, но вот история весьма занятная. Аристократичная
дамочка из начала 30-х вроде бы все имеет, что только велено иметь от красивой
жизни – загородный особняк, респектабельный муж, ребенок. Да, потом случился
адюльтер, и вроде бы тоже не критично. Ну кто из нас святой? Но затем у этой
дамочки, равно как и у ее семьи, все стремительно пошло вразнос. И вот она уже
покупает на последние пенсы пирожки, а муж застрял с концами в плену у старого
фрика в дебрях Амазонии. А сынок в могиле.

\ifcmt
  ig https://scontent-frt3-1.xx.fbcdn.net/v/t1.6435-9/116340349_3403647226333023_6962992606151472995_n.jpg?_nc_cat=102&ccb=1-5&_nc_sid=730e14&_nc_ohc=CdmawqWpQcsAX_A2J8C&_nc_ht=scontent-frt3-1.xx&oh=01adf3f17e6db0bff71cef74a1b267ea&oe=61A3D855
  @width 0.4
  %@wrap \parpic[r]
  @wrap \InsertBoxR{0}
\fi

Вообще, типичная для Во история с падением из князей в грязи. А писал ведь
почти о нашей стране. Мы как-то незаметно запустили в нашу постельку заезжих
интуристов и вот они уже не только потрахивают членов нашей семьи. Они пилят
наше кровное и искренне возмущаются, когда мы им на это указываем. А ведь и
указывать уже не успеваешь. Собаки крутят хоровод, раздалбывая страну с диким
ускорением.

Только под прикрытием карантина забрали у нас землю и тут же не успели
опомниться – всунули нам легализацию игорки. Только-только переваривали слова
Третьяковой о «некачественных детях» и тут же изделие доктора Франкенштейна нам
впаривает идею и легализации проституции. Только всматриваемся в цифры, как
залетная ширинка из Литвы с погонялом Абромавичус за год развалила
«Укроборонпром», как другая ширинка из Грузии нам уже предлагает за раз
крякнуть всю судебную систему Украины.

Я вот нисколько не юрист, но от 4-минутного ролика Саакашвили почувствовал себя
просравшимся правоведом. Интурист на голубом глазу предлагает на раз херануть в
нашей стране корявую, неповоротливую, несомненно, коррумпированную, но в целом,
работающую судебную систему.

Просто напомню. Саакашвили – это та самая истеричная Бе-бе-бе, которая
скрывается от судебных органов Грузии. Той самой Грузии, от которой Саакашвили
по итогу своего яркого президентства отминусовал Абхазию и Южную Осетию. А на
оставшейся территории так чудесно отреформировал тамошние суды, что до сих пор
гастролирует в наших краях.

Итак, что этот многоскиталец предлагает сделать у нас? Убрать все существующие
системы контроля за судебной властью и сколотить вместо них один контрольный
орган. Он один будет проводить аттестацию судейского корпуса, мониторить их
деятельность и материально подкармливать. Проще говоря, Саакашвили предлагает
создать одну институцию, которая будет раком ставить всех судей страны, а те от
нее полностью зависеть, а значит и выносить решения с оглядкой на эту контору.
По сути, это означает, кратное упрощение контроля за судьями со стороны власти.

Дальше. Саакашвили предлагает убрать все «лишние» суды – административные,
хозяйственные – и залепить на их месте «единые окружные». Ну и сократить их
количество с существующих 764 до «не более 200». Еще раз. Грузин нам предлагает
200 судов на всю страну. И тут же добавляет, что работать они станут быстрее. И
ведь он нас держит даже не за идиотов, а за какие-то организмы попроще.

Еще одно «ноу-хау» - суд в смартфоне. Это, сука, реальность, говорит нереальный
Саакашвили. Нам уже построили страну в смартфоне, провели диджитализацию и все
это в турборежиме. Теперь еще предлагают судить по телефону.

Продолжаем идти в ад. Саак предлагает создать «маленький суперсуд из самых
честных в стране людей», который пипец как может решать всё-всё в нашей стране.
Это уже прямая калька на Верховный Суд Штатов. По сути, многоподданный нам
пытается втюхнуть англо-саксонскую модель правосудия, которая вроде бы не
совсем о нас.

Ну и закавказская вишенка на торт – Конституционный Суд нам, оказывается, не
нужен. Типа, атавизм. И, действительно, зачем в стране конституционный суд, где
постоянно сужаются конституционные права людей? Все эти пошлые права на труд,
на бесплатную медицину, на доступное образование…

Смотрите. Личность Саакашвили любому, кто не балуется наркотой, глубоко
противна. Мартышка, меняющая родину чаще, чем его соплеменники по вольеру
бананы, вызывает только брезгливость. Но почему мы с вами допускаем, что эти
заезжие зверушки торгуют нашей родинкой, мне вот совершенно непонятно. Ну
ладно, адюльтер случается по согласию сторон, а тем, кому изменяют, видимо
чем-то непривлекательны. Но ведь это уже не измена. Наш дом разносят бешеные
животные, которые демонтируют базовые основы государственности. Нам строят
зоопарк. И вот уже наши бабы на последние покупают пирожки, а мужики катаются с
протянутой гастарбайтерской рукой по всему миру. Потому что в Украине – царство
заезжих обезьян.
