% vim: keymap=russian-jcukenwin
%%beginhead 
 
%%file 23_11_2020.news.ua.strana.1.covid_suicides_ukr
%%parent 23_11_2020
 
%%url https://strana.ua/news/302049-samoubijstvo-iz-za-koronavirusa-kak-lechitsja-ot-depressii.html
 
%%author 
%%author_id korzun_julia,romanova_maria
%%author_url 
 
%%tags 
%%title "Найти точку опоры". Почему корона-суициды накрыли Украину и как не впасть в ковид-депрессию
 
%%endhead 

\subsection{\enquote{Найти точку опоры}. Почему корона-суициды накрыли Украину и как не впасть в ковид-депрессию}
\Purl{https://strana.ua/news/302049-samoubijstvo-iz-za-koronavirusa-kak-lechitsja-ot-depressii.html}
\Pauthor{Корзун, Юлия}
\Pauthor{Романова, Мария}

\ifcmt
pic https://strana.ua/img/article/3020/49_main.jpeg
caption Человек в больнице. Фото: Pixabay 
\fi

\index[rus]{Коронавирус!Самоубийства, Украина}
 
В Украине было совершено очередное самоубийство больного коронавирусом. В
минувшую пятницу, 20 ноября, с собой покончила пожилая пациентка - женщина
повесилась прямо в больнице.\Furl{https://kiev.strana.ua/302006-v-bolnitse-kieva-povesilas-patsientka-kotoraja-bolela-sovid-19.html}

Этот случай - далеко не первый в нашей стране. Пациенты решают уйти из жизни
по-разному. Наиболее распространенный способом - выбрасываются из окон
инфекционной больнице. Но известны и другие прецеденты.

\enquote{Страна} рассказывает о волне самоубийств ковид-больных.

Также мы попытались разобраться в причинах трагедий с точки зрения
психологии. Иностранные ученые уже ввести в оборот новый термин -
ковид-депрессия, которая вызвана самой болезнью или самоизоляцией.
Публикуем советы специалистов, как от нее уберечься. 

\ii{23_11_2020.news.ua.strana.1.covid_suicides_ukr.suicides}
\ii{23_11_2020.news.ua.strana.1.covid_suicides_ukr.obstanovka_v_bolnicah}

\subsubsection{Что рекомендуют психологи}

\enquote{Страна} спросила психологов, с какими проблемами к ним обращаются люди во
время пандемии коронавируса, и как сохранить ментальное здоровье в таких
условиях.

\index[names.rus]{Карась, Анжела!психолог!коронавирус}

\textbf{Анжела Карась, психолог:}

\enquote{Посттравматический синдром работает так: сначала шок, отрицание, злость и
агрессия, а потом депрессия. Мы сейчас в стадии подхода к депрессии. У
многих людей наблюдается такая подавленность. 

Мы сейчас живем по сути в мирное время, но все равно есть какой-то
невидимый враг в виде эпидемии. Каждый день мы видим ужасающие цифры
заболевших и этот враг все ближе и ближе подступает. Наши знакомые,
знакомые знакомых болеют. Моя не то, что рекомендация, но нужно найти в
себе точку опоры, чтобы это выдержать.

Когда у человеку угрожает опасность, есть две стратегии выживания:
замереть или бежать. Бежать некуда, остается замереть. Но это не значит
умереть, это значит найти в себе точку опоры, чтобы попробовать это
пережить и выдержать.

Поэтому только внутренняя выдержка, психологическая гигиена, здоровье и
вера в то, что это пройдет}.

\index[names.rus]{Драпей, Анжелика!психолог!коронавирус}

\textbf{Анжелика Драпей, психолог:}

\enquote{Я работаю консультантом на телефоне доверия. К нам довольно часто
обращаются люди. У кого-то активизировались панические атаки, не просто
тревожность, а страх заболеть, умереть. По поводу эпидемии коронавируса,
сейчас чаще обращаются люди. В большинстве случаев - это панические атаки
и нападения депрессии. Вначале пандемия сильно влияла, а сейчас, поскольку
психике свойственно защищаться, в обществе встречается обесценивание этой
опасности, довольно игривое отношение к проблеме. Это признаки того, что
люди защищаются от психологической травмы, которая возникла в результате
этих событий. 

Женщины чаще звонят, но и мужчин тоже много. Нужно меньше читать, смотреть
новости, а людям, которые склонны к паническим атакам и тревожности,
вообще не читать и смотреть их. Для них это психическая смерть. Это никак
их не обезопасит, для их психики - это нетерпимые вещи. Чем больше человек
наполняет себя этой информацией, тем больше повышается уровень
тревожности. До коронавируса намного меньше было звонков. С паническими
атаками звонили люди, которые болеют хроническими заболеваниями. Такого
всплеска вообще никогда не было.

Нам звонят молодые люди по 17-20 лет, они говорят: \enquote{Я боюсь выйти из дома,
мне страшно, как мне общаться с людьми, я не знаю, что мне делать}. Кто-то
склонен к навязчивым состояниям — это люди, которые постоянно моют руки,
дезинфицирует или стирают одежду. Пандемия активизировала невротические
вещи, которые глубоко сидели}.

\textbf{Спартак Суббота, психотерапевт, советник главы Минздрава}:

\enquote{Изменения настроений у людей, конечно, есть. Оно изменилось в худшую
сторону. Причина: может, нарушиться качество работы тех или иных структур,
в том числе, средний и малый бизнес могут пострадать. У людей нет четкого
понимания, куда все движется и какие будут последствия.

Однако за последние несколько месяцев, настроение у людей улучшилось. Если
вначале пандемии оно просело значительно, то сейчас чаще обращаются с
тревожными настроениями, меньше с депрессивными, настроение немного стало
лучше, все нормализуется}.

\subsubsection{Covid-расстройства и как их победить}

\textbf{Коронавирус поражает мозг}. Британские ученые выяснили, что коронавирус
может атаковать мозг человека и вызывать в поведении инфицированного
Covid-19 психоневрологические симптомы. 

В числе перечисленных симптомов у зараженного человека может наблюдаться
депрессия, забывчивость, тревога, спутанное сознание и бессонница. В
тяжелых случаях могут быть психозы, делирии (резкое изменение образа
мыслей и действий человека), слабоумие и другие расстройства.

Российский врач-невропатолог и психотерапевт Наталья Терещенко поясняет
механизм, при котором коронавирус поражает мозг человека.

По ее словам, необъяснимую тревогу, возбуждение, невозможность спать и
прочие вегетативные симптомы при ковиде происходят по той причине,
что \enquote{тромбики поражают сосудики мозга, находящиеся в зоне вегетативной
регуляции}. Некоторые пациенты описывают яркие кровавые сны или даже
галлюцинации. У многих слабость остается на много недель после ковида.

Именно поэтому терапевты иногда советуют пить тем, кто болен
коронавирусом, успокоительные параллельно с антикоагулянтами (лекарствами,
которые снижают активность свертывающей системы крови и препятствуют
чрезмерному образованию тромбов).

\textbf{Проблемы с психикой на самоизоляции}. Немецкие ученые уверяют, что
депрессию может вызвать банальное ограничение общественной жизни или
долгое нахождение в самоизоляции. Специалисты считают, что последствия для
психики затронут даже здоровых людей. 

Экономические заботы и мысли о небезопасном будущем могут легко разрушить
привычный уклад многих семей. При этом психологических последствий
особенно высокий для больных, которые попадают в клиники.

\enquote{... они попадают в потенциально опасную для жизни ситуацию, в отделение
интенсивной терапии, где их необходимо лечить инвазивно, проводить
вентиляцию - это влияет на психику}, - уверен директор Центрального
института психического здоровья в Мангейме Андреас Майер-Линденберг.

Данные о влиянии вируса на психическое состояние людей подтверждают ученые
Китая, Норвегии и Канады. Особенно сильно пострадала возрастная группа от
18 до 25 лет, где выявили случаи тяжелой депрессии. Такие данные приводит
совместный опрос частной высшей школой Геттингена, университета
Регенсбурга, норвежского университета прикладных наук и Карлтонского
университета в Оттаве.

По словам профессора клинической психологии Юсеф Шиба, ожидалось, что доля
случаев тяжелой депрессии среди опрошенных не превышает 1\%. Но этот
показатель достиг 5\%. Более того, есть признаки того, что психические
последствия пандемии будут ощущаться длительное время.

Согласно недавно опубликованной работе Оксфордского университета, большое
количество пациентов с Covid 19, выписанных из клиник, все еще имели такие
симптомы, как тревога и депрессия, через 3 месяца после заражения вирусом.
В целом почти 20\% людей получили психиатрический диагноз в течение 90 дней
после заражения коронавирусом.

Результаты анализа также показали, что люди с ранее имевшимся диагнозом
психического расстройства имели на 65\% больше шансов заболеть Сovid-19,
чем те, кто не имел проблем с психическим состоянием.

\enquote{Это открытие стало неожиданностью и требует исследования. Вместе с тем
наличие психического расстройства следует добавить в перечень факторов
риска для Covid-19}, - рассказал один из исследователей доктор Макс Таке.

Несколько исследований, проведенных в Китае, показали, что практически у
всех таких пациентов были симптомы стрессового расстройства.

На Западе уже появился термин covid-depression - ковид-депрессия -
состояние, вызванное ситуацией с коронавирусом в целом, самоизоляцией и
ограничительными мерами. О том, как впала в депрессию на самоизоляции,
рассказывала известная американская певица и актриса Дженифер Лопес.  

\enquote{Вы знаете, это было довольно трудно. Во время пандемии, я думаю, у всех
был такой момент, когда они осознали, что находятся в депрессии и серьезно
переживают за свое будущее}, - рассказала 51-летняя звезда в интервью на
радиостанции El Zol 107.9.

\textbf{Профилактика ковид-депрессии}. По данным Всемирной организации
здравоохранения, в мире депрессией страдает 300 миллионов человек.

По данным Минздрава Украины, в нашей стране депрессия занимает первое
место среди всех психических нарушений. При этом из года в год показатели
заболеваемости растут. Частично это объясняется низкой психологической
осведомленностью населения. А частично - стрессогенной ситуацией.

Вот градация симптомов депрессии, которая приводится на сайте МОЗ:

\begin{itemize}
  \item * угнетенность,
  \item * потеря интереса,
  \item * невозможность испытывать радость,
  \item * уменьшение жизненной энергии,
  \item * пассивность,
  \item * тревога,
  \item * сонливость или бессонница,
  \item * низкая концентрация,
  \item * потеря аппетита,
  \item * низкая самооценка,
  \item * чувство вины или отчаяния,
  \item * мысли о самоубийстве.
\end{itemize}

Депрессию в Украине констатируют, но не лечат системно. В стране до сих
пор нет государственной программы поддержки психического здоровья людей,
психологического сопровождения военных, жертв войны, ветеранов и раненых.

При этом толчком к депрессии становятся негативные события: потеря работы,
смерть кого-то из родных, психологическая травма, тяжелое стрессовое
событие. Сердечно-сосудистые болезни тоже могут приводить к депрессивным
состояниям.

Здесь важно не держать в себе эмоции: делитесь мыслями и страхами с
друзьями и близкими, не пытайтесь всем угодить, ставьте достижимые цели и
задачи, придерживайтесь режима работы и отдыха.

Занимайтесь спортом и активным образом жизни. Откажитесь от алкоголя,
которые усиливают депрессию. Если чувствуете усталость, а тревога не
покидает больше двух недель, обратитесь к врачу: к семейному врачу, а при
необходимости к узкому специалисту.

Легкую форму депрессии можно вылечить без медикаментов, с помощью определенных
психотерапевтических методов. При средней и тяжелой форме может потребоваться
медикаментозное лечение, поэтому любые антидепрессанты нужно принимать
исключительно по назначению врача. 

