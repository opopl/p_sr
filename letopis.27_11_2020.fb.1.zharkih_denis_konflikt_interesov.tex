% vim: keymap=russian-jcukenwin
%%beginhead 
 
%%file 27_11_2020.fb.1.zharkih_denis_konflikt_interesov
%%parent 27_11_2020
 
%%url https://www.facebook.com/permalink.php?story_fbid=2865344507012281&id=100006102787780
 
%%author 
%%author_id zharkih_denis
%%author_url 
 
%%tags 
%%title Нет конфликта национальностей, есть конфликт интересов.
 
%%endhead 
 
\subsection{Нет конфликта национальностей, есть конфликт интересов}
\label{sec:27_11_2020.fb.1.zharkih_denis_konflikt_interesov}
\Purl{https://www.facebook.com/permalink.php?story_fbid=2865344507012281&id=100006102787780}
\ifcmt
	author_begin
	author_id zharkih_denis
	author_end
\fi

Сегодня спорил с Сергеем Веселовским. Вот хочу и на ФБ написать, что не считаю
конфликт в/на Украине, и российско-украинский конфликт культурным,
цивилизационным и этническим. Да там просто все, как угол дома: есть интересы
Штатов, как мирового лидера, запереть русского медведя в берлоге, чтобы носа не
высунул. А то всю малину поест. Тут нет ненависти лично к русским, а есть
понимание, что земной шарик маленький и на всех может не хватить. 

Достаточно долго США это удавалось, они развалили СССР, уничтожили Второй мир
(большую часть поставили себе под контроль), начали активный развал России.
Этот сценарий хорошо виден каждому, но давайте посмотрим против какого сценария
действовали США.  Допустим, перестройка в СССР удалась, теперь это не казарма с
ядерной дубинкой, а мировая мастерская, способная дешево и качественно
производить товары, нужные в мире. К началу Перестройки СССР был ближе к этой
роли, чем современный Китай, который это роль получил. В этом случае СССР не
только не сдает своих союзников по СЭВ и Варшавскому договору, но начинает
экономическую экспансию на Запад. Нет, танки на Лиссабон не идут, а идет масса
недорогих и качественных товаров, которые европейцам дешевле покупать, чем свои
собственные. Европа оказывается в должниках у СССР, и активно экономически с
ним кооперируется, поскольку от нее уже не требуют строить коммунизм, ходить
строем и размещать советские военные базы. Но не все так просто - от Европы
100\% потребовали бы послать подальше американские военные базы, подвинуть
американские банки, фонды и биржи, уменьшить импорт американских товаров,
поскольку есть отличные советские. 

Ну и как такой сценарий США? А они его хорошо просчитали. В этом сценарии
распад СССР не предусмотрен, а значит принцип советского интернационализма
остается, просто меняется цель развития и способ управления. Цель развития
сменить было необходимо, во времена Холодной войны СССР и США пытались навязать
всему миру свой способ жизни. У США это удавалось в Первом мире, у СССР - во
Втором.  Проблема была в том, что в чем собственно советский способ жизни,
советская элита с каждым годом понимала все менее. Китай пошел по другому пути,
он не навязывает свой способ жизни, он навязывает покупку своих товаров, что
мог вполне сделать СССР. Но перейти к новым целям можно было только сменив
методы управления. Это и попытались сделать в Перестройку. 

Советская система была основана на безоговорочном подчинении, лояльности
начальнику. Это не самый плохой принцип, он действует, например, в армии, в
спорте и даже в балете. Но идеальный рынок контролируется перетоком финансов, а
не милиционером или партийным собранием. Теперь нужны были не те, кто подчинит,
а те, кто удовлетворит больше потребностей. Именно к этому и шла советская
управленческая мысль. До экономической супердержавы был практически один шаг.
Но тут-то и вмешались США, подсадив советское пространство (в том числе и
русских) на национализм. 

И приехали. Вопрос об эффективности, справедливом распределении, правильном
карьерном росте, распределении ответственности исчез, унося с собой перспективы
сверхдержавы. Теперь каждая нация начинала плакать о том, что ее обижали и надо
вернуть свое. И тут не важно, что это за нация, обижали ли ее вообще, и что она
считает своим. Главное возмущение, ненависть к соседу, желание получить \zqq{свое}
на халяву. 

Вот это желание пожить на халяву за счет реальных или мнимых обидчиков
разрушает стратегию экономической экспансии. Если раньше за продвижение империи
нужно было платить жизнями воинов, то сегодня захватываются не территории, а
рынки, их нужно покупать, это стоит денег, а не жизни людей. А вот деньги мы
получаем из перераспределения доходов наших граждан. Так делают китайцы и имеют
успех. И пока мы возимся в том, кто русский, кто украинец, где у нас
цивилизационный разлом, у нас скупают самое важное для нас. А потом повесят нас
на веревке, которую мы сами и купим.
