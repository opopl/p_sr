%%beginhead 
 
%%file 07_05_2022.fb.bilous_olena.mariupol.1.pochemu_fraza_glavnoe_sami_zhivy_ne_rabotaet
%%parent 07_05_2022
 
%%url https://www.facebook.com/permalink.php?story_fbid=pfbid034UdvhRKoKZTLeTYnLLk3Gc3Rh58mTDKVn6v5cTwoCH9BGJh45fZS9BhkG5BnY8Dyl&id=100004129691965
 
%%author_id bilous_olena.mariupol
%%date 07_05_2022
 
%%tags mariupol,mariupol.war,psihika,psihologia,dusha,chelovek
%%title Почему фраза "главное сами живы" не работает??????
 
%%endhead 

\subsection{Почему фраза \enquote{главное сами живы} не работает??????}
\label{sec:07_05_2022.fb.bilous_olena.mariupol.1.pochemu_fraza_glavnoe_sami_zhivy_ne_rabotaet}

\Purl{https://www.facebook.com/permalink.php?story_fbid=pfbid034UdvhRKoKZTLeTYnLLk3Gc3Rh58mTDKVn6v5cTwoCH9BGJh45fZS9BhkG5BnY8Dyl&id=100004129691965}
\ifcmt
 author_begin
   author_id bilous_olena.mariupol
 author_end
\fi

Почему фраза «главное сами живы» не работает??????

Самое бесполезное, что вы можете сказать человеку, который потерял всё - это:
\enquote{главное, вы сами живы}. 

Да, живы, это самая большая награда - жизнь. 

Но! Жизнь - это подарок, который мы не просили, не зарабатывали и не мы к этому
прикладывали усилия, как минимум, рожали нас мамы. Да, война показала нам, ещё
раз, как важно ценить каждый прожитый день, как важно ценить всё, что ты имеешь
и людей, которые рядом с тобой. Но! Когда человек потерял всё, что он
зарабатывал и нарабатывал много-много лет: дом, работу, имущество, привычный
круг друзей. Когда человек потерял улицы, по которым он гулял, дерево, под
которым впервые поцеловался или шум моря, который слышал его самый звонкий смех
и самые горькие слёзы. Когда человек потерял свой диван, который так долго
выбирал и только поставил в квартиру, не успев его даже собрать, когда он
потерял свою любимую чалку, телевизор, на который он копил годами... когда он
боится открыть чат со списками погибших, потому что там, в любом случае есть
его знакомые...

Фраза: главное ты жив, звучит как насмешка. Это прямое обесценивание всего, что
человек делал до войны. А ещё, это запрет на чувства: смотри, Вот у него брат
погиб, а ты тут за своей сковородкой плачешь. Любой запрет на проживание чувств
- это болезни в дальнейшем. Любое обесценивание беды или сравнивание твоей беды
с бедой других людей - это как сравнить твою жизнь до войны с нулём!

Почему люди так говорят? Ответ прост. Никто не знает, что нужно говорить в
таких случаях. Никакие слова не помогут этому горю. Его надо пережить, прожить
все стадии, вплоть до принятия. И не ждите, что это будет завтра или через
месяц. Нееееттт.  Через 2-3 месяца у вас только начнётся тоска. Тоска по дому,
по кухне, спальне, по быту. У детей начнётся настоящая тоска по их любимым
игрушкам. И будет ещё хуже. Первое время, мы на адреналине. Организм выделяет
кортизол, работает защитная функция психики. Многие моменты мы блокируем. Плюс,
привычка с детства: летние каникулы 3 месяца. А вот потом, идёт этап осознания,
что на самом деле произошло. Будут особые вспышки. На праздники, значимые дни
или когда прийдется покупать новые ложки (нож/постельное белье/кухонные
мелочи/гаечные ключи и т. д.). 

Если вы не знаете, что ответить человеку, который всё потерял  - просто
обнимите его или просто МОЛЧА ВЫСЛУШАЙТЕ без оценки и советов. Мы имеем право
оплакивать свои квартиры, работу, свои любимые кафе так, как ОПЛАКИВАЕМ БЛИЗКИХ
людей. Почему? Потому, что это и есть: составляющая вот этого самого «главное -
ты жив».

На каждое горе, нужна одна весна, одно лето, одна осень и одна зима. Раньше
легче не станет.  Моментами, перепадами - да. Но не стабильно. Не постоянно. Не
навсегда.
