% vim: keymap=russian-jcukenwin
%%beginhead 
 
%%file 26_02_2023.tg.suhorukova_nadia.mariupol.1.malina_i_detstvo
%%parent 26_02_2023
 
%%url https://t.me/maripol_hope/1327
 
%%author_id suhorukova_nadia.mariupol
%%date 26_02_2023
 
%%tags babushka,detstvo,malina
%%title Малина и детство
 
%%endhead 
 
\subsection{Малина и детство}
\label{sec:26_02_2023.tg.suhorukova_nadia.mariupol.1.malina_i_detstvo}
 
\Purl{https://t.me/maripol_hope/1327}
\ifcmt
 author_begin
   author_id suhorukova_nadia.mariupol
 author_end
\fi

А до прихода русни у меня не было грустных историй. 

Истории пахли малиной и детством. 

Но все они  погибли вместе с аккаунтом. Одну случайно обнаружила вчера. Она
была на одном из сайтов. Пусть будет здесь. 

***

\enquote{Пусть ваша девочка первая поест. Мы же ребенку будем давать. Должны
быть уверены, что все в порядке}.

Мы с бабушкой продаем, только что собранную, малину.

Женщина в очках и соломенной  шляпе с подозрением смотрит на ягоды. 

\enquote{А что так дёшево? Она брызганная?}

Бабушка уверяет:  ни в коем случае. Дама не верит.

Это  удивительно. 

Бабушка никогда не обманывает. 

И мне всегда рассказывает, что вруны, как  предатели, а ябед никто не любит.

\enquote{Доносчику - первый кнут}, -  это тоже ее выражение.   

Чтобы доказать бабушкину правоту, беру горсть ягод и запихиваю в рот. 

Малина на рынке мне кажется  вкуснее, чем дома. 

И я, глядя в глаза зануде - покупательнице, наклоняюсь над банкой с горкой
малины, вытягиваю губы трубочкой  и захватываю ягоды без рук с помощью  воздуха
и языка. 

Я горжусь этим трюком,  получается изящно,   как у  пылесоса. 

Вместо восхищённых глаз -  удивление и испуг. 

Но малину,  дама - неверующая, покупает.  

Когда нет покупателей я  сижу на деревянном ящике и учу стихи.

Из-под  прилавка иногда  мелькает  мой, обгоревший на солнце, нос и
поцарапанные руки.  

Стихотворения мне читает бабушка. 

Я должна запомнить и повторить. 

И желательно с теми же интонациями, что у нее. 

Она знает, как правильно.

Бабушка в прошлом актриса.

Ее зовут Вера и взрослые говорят, что она красивая. 

Мне смешно: \enquote{Как бабушка может быть красивой? Она же  бабушка!}

Годом раньше мы   отдыхаем  в пансионате.

Бабушка подрабатывает  кастеляншей, а я  при ней -  Санчо Панса. 

В ее распоряжении комната с  сугробами  скрипучих   простыней и мягких
полотенец. 

Внутри постоянно  пахнет арбузом.  

Слово \enquote{кастелянша} мне категорически не нравится. 

Оно напоминает  Бабу Ягу - костяную ногу.  

Поэтому я называю  бабушку \enquote{помощница  директора по постели}.  

Так и говорю взрослым, когда задают вопросы о нашей  жизни в пансионате. 

Они  от этой трактовки в восторге.

Повторяют вслед за мной. 

Мне приятно. Чувствую себя пупом земли. 

В пансионате произошли два важных события: меня впервые кусает оса и бабушка
снова выходит на сцену. 

Она читает что-то для отдыхающих в летнем театре. 

Я сначала боюсь, что ее не будут слушать. 

Но после первых же слов люди замирают, как статуи. 

А я  перестаю узнавать женщину на сцене.

Это  не моя бабушка. 

Я срочно хочу вернуть свою: добрую, домашнюю и любимую.

Статная незнакомка с красивым голосом и сильными движениями - чужая. 

Со мной истерика.

Я рыдаю за спинами зрителей и повторяю шепотом, как заклинание: 

\enquote{Хочу к бабушке Вере}. 

Летом  царапины покрывают мою кожу  от лба до пяток. 

Есть даже под глазами и за ухом. 

Это последствия сражений с кустами малины за урожай.

Самые неприятные ранки бабушка мажет йодом и  они  загораются огнем. 

Я не сдерживаюсь и ору. 

Мне всего шесть и терпение не главное мое достоинство.

Дедушка называет это \enquote{кошачьим концертом}, и уверяет, что это \enquote{напрасная
мазня},  на мне все и так  заживёт, как на собаке.

Бабушке не нравится, что человеческого  ребенка сравнивают с животными и она
наказывает дедушку молчанием. 

Дедушка встретил бабушку в Ростове  и сразу влюбился. 

У ее родителей он снимал комнату с питанием. 

Такой  себе аll inclusive после Второй мировой.

Семья бабушки   по тем временам считалась  зажиточной. 

Бабушкин папа работал зубным техником, а  мама занималась дочками, шила наряды
у хорошей портнихи и ходила на рынок через день. 

Дедушка тогда учился в мореходке. Худой, весёлый и свободный, как ветер в
парусах. 

В съемной комнате он прожил неделю, а потом с гастролей вернулась бабушка. 

И дедушка -  \enquote{пропал}.

Он вздыхал, днями лежал на кушетке и печально смотрел в потолок. Бабушка его
не замечала.
