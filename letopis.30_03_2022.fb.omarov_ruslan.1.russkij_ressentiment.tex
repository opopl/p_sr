% vim: keymap=russian-jcukenwin
%%beginhead 
 
%%file 30_03_2022.fb.omarov_ruslan.1.russkij_ressentiment
%%parent 30_03_2022
 
%%url https://www.facebook.com/ruslan.omarov.5836711/posts/692075122237394
 
%%author_id omarov_ruslan
%%date 
 
%%tags 
%%title Русский ресентимент
 
%%endhead 
 
\subsection{Русский ресентимент}
\label{sec:30_03_2022.fb.omarov_ruslan.1.russkij_ressentiment}
 
\Purl{https://www.facebook.com/ruslan.omarov.5836711/posts/692075122237394}
\ifcmt
 author_begin
   author_id omarov_ruslan
 author_end
\fi

Путинское поколение, в той массе, в которой оно переживет эту новую
реинкарнацию фашизма и его поход против человечества, будет, конечно, искать
убежище в извращённой ностальгии. В этом я не сомневаюсь. Это можно было
наблюдать в восьмидесятых применительно к сталинской эпохе, и это же повторится
вновь. 

Те, кому сейчас 25-30 лет (и интеллектуальный класс, и люмпены) в свои 50
станут живописать  путинской ампир как энергичный рай, разрушенный алчным
Западом (или, может, Китаем, в зависимости от течения событий, которые сейчас
трудно предсказать). Вывернутая наизнанку гибель богов, античная трагедия,
неронова драма. Сам Путин и вся остальная банда тонкошеих вождей будут выведены
в качестве культурных героев, богатырей и асуров, застигнутых предательской
стихией у кормила государственной машины ровно в тот миг, когда эта машина
должна была взломать небеса. И так далее. Все будет облагорожено, вычищено,
воспето и украшено венками. Это неизбежно, потому что такова вообще природа
ресентимента. 

Но есть одна особенность, которая присуща только русскому ресентименту, и это
рефрен: 'Зато нас боялась вся Европа! (весь мир, etc.)' 

Нас боялись! Сама идея, что предметом ностальгии может быть внушаемый тобой
ужас (не расово-цивилизационное превосходство, как это было у немцев, не
афинская культурная колыбель и прочие мифологемы, а вот - голый страх) - хорошо
показана, например, у Блока в его \enquote{Скифах}. Чистая энергия бессмысленного
разрушения, возведенный в абсолют хаос войны, нашествие орд, ядерный удар,
гумилевская \enquote{длинная воля}. В этом будут искать (и находить!) величественность,
забывая что \enquote{Нас боялись!} это лишь отражение центральной эмоции, веками
свойственной человеку внутри русского государства - личного страха. Обыватель
боится начальника, начальник - своего начальника, и все вместе - государства,
которое и само сконструировано из страха. Этот страх настолько органично
вплетен в русскую жизнь, что почти уже не воспринимается как отдельное чувство.
И, перевернувшись, как в призме, транслируется в коллективное \enquote{мы}. \enquote{Нас}
боялся весь мир потому что \enquote{мы} и есть квинтэссенция страха, каждый по
отдельности. \enquote{Мы} - здесь просто искусственно созданный мультипликатор
человека, который смотрится в зеркало.

Путинский летчик бомбит Мариуполь не из личной ненависти к украинскому ребенку,
который прячется в подвале разрушенного дома. Ненависть это привилегия вождей.
Не из чувства расового превосходства, не из инстинкта завоевания - это
галлюцинации идеологов. И даже не из животной жажды крови. Банальность Zла в
том, что он делает это именно из страха, который подобно ядовитому плющу
опутывает всю эту сатанинскую Vертикаль. От рядового до генерала. Храбрость
заключается в том, чтобы НЕ нажать на кнопку и не бубнить \enquote{Есть!}, сидя в
пятнадцати метрах от свихнувшегося диктатора - это к вопросу о том, сработает
ли вся человеческая цепь принятия решений в случае атомного сценария?
Сработает, не надо питать иллюзий. Сработает именно потому, что страх давно
парализовал волю. Выжить и сохранить жизнь на Земле - это волевое, осознанное
решение. А обречь мир и себя самого на гибель в страхе перед гипнотизирующим
крысиным оскалом в бункере - панический инстинкт. Строго говоря, весь
пресловутый проект \enquote{русского мира} это ни что иное как способ  бесконечно
умножать этот личный страх на завоеванную человеческую массу - в инфантильной
попытке убежать от него самому.

\ii{30_03_2022.fb.omarov_ruslan.1.russkij_ressentiment.cmt}
\ii{30_03_2022.fb.omarov_ruslan.1.russkij_ressentiment.cmtx}
