%%beginhead 
 
%%file 02_03_2023.fb.kipcharskij_viktor.mariupol.1.den_7___2_03_22__ser
%%parent 02_03_2023
 
%%url https://www.facebook.com/permalink.php?story_fbid=pfbid0EqVmrmjCLD1Z224qXkboL1fhJU1hmYVnPSJB9tQ2jhxNym2uiA8u1B83sAYftiCnl&id=100006830107904
 
%%author_id kipcharskij_viktor.mariupol
%%date 02_03_2023
 
%%tags mariupol,mariupol.war,dnevnik,02.03.2022
%%title День 7 - 2.03.22. Середа. Зникли електрика і, відповідно, інтернет
 
%%endhead 

\subsection{День 7 - 2.03.22. Середа. Зникли електрика і, відповідно, інтернет}
\label{sec:02_03_2023.fb.kipcharskij_viktor.mariupol.1.den_7___2_03_22__ser}

\Purl{https://www.facebook.com/permalink.php?story_fbid=pfbid0EqVmrmjCLD1Z224qXkboL1fhJU1hmYVnPSJB9tQ2jhxNym2uiA8u1B83sAYftiCnl&id=100006830107904}
\ifcmt
 author_begin
   author_id kipcharskij_viktor.mariupol
 author_end
\fi

День 7 - 2.03.22. Середа. 

Зникли електрика і, відповідно, інтернет.

Пішли до магазину Гастроном  N1: я в прибудові купив дві малі сітки картоплі,
моркву, яблука, пару лимонів. Неймовірно, але майже не було скандалів "вас тут
не стояло", чи "чого ти колупаєшся". Оля у магазині купила чупа-чупси онукам,
кетчупи, згущене молоко та якесь печиво за ціною золота, та ще щось. Довго
стояла у черзі на касу. Розраховувалися грошима - щодо карток я помилився. З
магазину вийшов охоронець - колишній наш студент Влад А., дав номер свого
телефону. Номер так в залишився у той куртці.

Зайняло це кілька годин. Чомусь не знайшли до аптеки.

Заїло замок у підвал. Їого збили кувалдою. У підвалі ми залишали стіл, стільці
ти шезлонг, тож його треба закривати, бо без електрики не працює замок на
дверях у під'їзд. Ми поїхали у гараж, де був запасний навісний замок: довго
шукали - не знайшли, тож зняли з воріт. Також взяли воду (колись набирав, аби
мити руки) та акумулятор з Дев'ятки. Навіть не згадали про якусь їжу, що у
2014-му занесли до підвалу разом з одягом - раптом ховатися доведеться.

Підійшов здоровань з підбитим оком, сказав, що він з товаришем добираються
додому, що Лівий "накривають" вогнем аж до Пашковського, попросився, щоб їх
підвезли. Було трохи страшно, але... Приятель чекав біля виїзних воріт. Довезли
майже до лікарні. Чому їм так треба було порожньою вулицею проїхати оті метрів
800?

М'ясо зварили.

Темніє, світла нема. До акумулятора під'єднав гніздо припалювача, до нього -
розгалуджувач, перехідник 12/5 вольт, подовжувач, який по килиму підняли під
стелю і ЛЕД-підсввтку. Більш-менш світло. Можна заряджати телефони(навіщо?),
планшет (читати), приймач (слухати канал Рада з приємними новинами. Приймач
постійно намагається "піти" з хвилі). Дуже цікаво слухати телевізійні новини: "
Ви бачите за моєю спиною...". Судячи з новин, ми вже майже перемогли... Через
слабенький перетворювач 12/220 навіть можна заряджати Нокію. 

Бойченко запевняє, що місто не здадуть: "Я розмовляв з Азовцями і вони мене
запевнили: Вадим Сергійович, місто не здамо:...

Сьогодні ми знаємо, що на той час місто вже кілька  днів було у повному
оточенні. Що багато депутатів виїхали самі і вивезли свої родини...

Що чиясь "світла голова" додумалася збирати в купу їжу, де її знищили -
повторення Бадаєвських складів у Ленінграді. Так само пізніше зібрали в купу та
втратили комунальний транспорт. Але рік тому ми цього не знали...

%\ii{02_03_2023.fb.kipcharskij_viktor.mariupol.1.den_7___2_03_22__ser.cmt}
