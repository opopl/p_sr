% vim: keymap=russian-jcukenwin
%%beginhead 
 
%%file slova.literatura
%%parent slova
 
%%url 
 
%%author 
%%author_id 
%%author_url 
 
%%tags 
%%title 
 
%%endhead 
\chapter{Литература}
\label{sec:slova.literatura}

%%%cit
%%%cit_head
%%%cit_pic
%%%cit_text
Еще раз подчеркнем, феерией называлась не \emph{литературная} основа, а само шоу,
действо. Пьесы, которые ставились таким образом, драматурги не обозначали как
феерию. Потому что играть их можно было по-разному.  Пока не пришел Метерлинк.
В 1905 году бельгийский символист Морис Метерлинк пишет свою самую знаменитую
пьесу. Она называется \enquote{Синяя птица}. И впервые он сразу ставит на обложке ее
жанр - пьеса-феерия.  Так термин перекочевал из театральной среды в
\emph{литературную}. Теперь читатель, даже не видя самой постановки, должен был
представлять то действие, которое разворачивается в пьесе, именно как красочную
и волшебную феерию
%%%cit_comment
%%%cit_title
\citTitle{В двадцатом веке было написано только три феерии. Одна бельгийская, две наши}, 
ЛИТИНТЕРЕС, zen.yandex.ru, 03.06.2021
%%%endcit

%%%cit
%%%cit_head
%%%cit_pic
%%%cit_text
Фраза про популярність в Росії «неймовірної української \emph{літератури}» належить до
порожньої, позбавленої сенсу риторики, бо українськими творами цікавились у
РСФСР хіба що поодинокі українці, що там мешкали.
Варте уваги й означення «неймовірна», вжите щодо \emph{української літератури}.
Очевидно, президент хотів вжити це слово у позитивному значенні «велика; така,
що виходить за межі звичайного». Визначення доволі абсурдне, оскільки у
застосунку до \emph{літератури} воно не вживається. Насправді ж тут можна вбачати
обмовку за Фрейдом: на підсвідомому рівні прикметник «неймовірна» він радше
вжив у першому його значенні – «така, яку важко чи неможливо уявити, якої
насправді не існує; неможлива, нереальна» і саме це значення визначає його
реальне ставлення до \emph{української літератури}
%%%cit_comment
%%%cit_title
\citTitle{Українська мова в культурному просторі держави. Протистояння триває}, 
Лариса Масенко, www.radiosvoboda.org, 11.07.2021
%%%endcit

%%%cit
%%%cit_head
%%%cit_pic
%%%cit_text
Гоголь не прийшов у російську \emph{літературу} бідним родичем. Він приніс у
неї слово свого народу, влив у російську мову свіжий струмінь молодої, розкутої
сили, СЛІДОМ за Пушкіним оновив, збагатив \emph{літературну} мову новими
життєвими зворотами й інтонаціями, надав їй небаченої образності, розширив її
межі. Сила різьблено-точної мовної характеристики гоголівських персонажів була
така вражаюча, що сучасники вигукували: "Автор-стенограф!" Жоден з письменників
не створив такої кількості типів, які ввійшли в побут \emph{літературний},
політичний, щоденний, як імена прозивні, як символи тих чи інших людських
пороків.  Та не слід вважати, ніби \emph{літературний} шлях для Гоголя стелився
широко і вільно, як Невський проспект
%%%cit_comment
%%%cit_title
\citTitle{Три долі. Гоголь, Шевченко, Чехов}, Павло Загребельний
%%%endcit

%%%cit
%%%cit_head
%%%cit_pic
\ifcmt
  pic https://avatars.mds.yandex.net/get-zen_doc/4350071/pub_618312bce8eeee6665ae858c_618312f737011142658acd1c/scale_1200
  @width 0.4
\fi
%%%cit_text
В немецкой \emph{литературе} - братья Гримм, в русской литературе – Стругацкие,
Вайнеры. В украинской \emph{литературе} тоже есть такой братский творческий
тандем – братья Капрановы, Виталий и Дмитрий. Известные книгоиздатели и
писатели.  Казалось бы, явление в \emph{литературе} не уникальное. Ну, братья и
братья, пишут романы, повести, рассказы и прочую прозу, да и ладно. Их читают,
они успешны. Всё хорошо, всё замечательно.  Чем они так уж интересны и почему
вдруг я решила о них рассказать? Нет-нет, с ними лично я не знакома, хотя
приезжают они в Николаев довольно часто, выступают с презентациями своих
произведений в библиотеках города. И вообще, почти земляки. Учились братья
десять лет в Очакове Николаевской области. В школе с русским языком обучения.
Заметьте, от изучения украинского языка они были освобождены. По заявлению
родителей. Такое тогда, увы, случалось часто
%%%cit_comment
%%%cit_title
\citTitle{Украинские писатели-братья}, Украинский русский, zen.yandex.ru, 04.11.2021
%%%endcit

%%%cit
%%%cit_head
%%%cit_pic

\ifcmt
  tab_begin cols=3
     pic https://gdb.rferl.org/72661C11-9FFF-4979-B31C-84B5CADE5B34_w650_r0_s.jpg
		 pic https://gdb.rferl.org/914F243D-2CFE-44E6-B66D-B70E6B6F4F1E_w650_r0_s.jpg
     pic https://gdb.rferl.org/37A2DFE8-CA92-46AB-86CC-401B15A8331B_w650_r0_s.jpg
  tab_end
\fi
%%%cit_text
Михайло Драгоманов у доповіді на \emph{літературному} конгресі в Парижі 1878 року
зазначив: «Що стосується Літопису Нестора, Слова про похід Ігоря і т.д., то ці
твори були безпосередніми плодами місцевого життя стародавньої України. Ці
твори започатковують українську \emph{літературу}».
Прикметно, що в Росії «Повість» Нестора Літописця називають «великим наследием
России» і чіпляють до «истоков русской \emph{литературы}». Докладно про природу таких
імперських міфологем ми писали в публікації «Чому написане у Києві Остромирове
Євангеліє – «достояние России»?»
%%%cit_comment
%%%cit_title
\citTitle{Повість Нестора Літописця – перша писемна історія України}, 
, www.radiosvoboda.org, 07.11.2021
%%%endcit

%%%cit
%%%cit_head
%%%cit_pic
\ifcmt
  pic https://media.1istochnik.ru/attachments/istochnik/publications/9/98537/large_1617185620-60f9f2e47f.jpeg
  @width 0.4
\fi
%%%cit_text
Код, подчеркнула Е. В. Митягина, это всегда соотношение постоянства и
изменчивости. Константы известны нам со времен «Повести временных лет» - они
проходят через все реперные точки \emph{русской литературы} от Пушкина до Блока, от
Блока до Алексея Иванова. На них зиждется идея русского мира и русского
зарубежья.  В какой степени сегодня отражены ценности тысячелетней истории? В
поисках ответа на этот вопрос исследователи обратились к большим данным,
способным помочь пониманию психологии и поведенческих реакций русского
человека; идей, вызывающих наибольший отклик в душах россиян. 
В ходе пилотажного этапа исследования был проведен анализ отзывов на
художественные произведения, получившие признание. Было отобрано более 150
произведений, ставших в 2000-2020 гг. лауреатами и победителями таких премий,
как «Большая книга», «Писатель года», «Русский Букер», «Нацбест», «Большая
\emph{литературная} премия», «Ясная Поляна», «Государственная премия», «Патриаршия
премия». Команда проекта проанализировала сотни тысяч отзывов – в результате
был сделан вывод о том, какие ценности на уровне ментальности находят в душе
россиянина наибольший отклик
%%%cit_comment
%%%cit_title
\citTitle{«Наша сила – в нашей идентичности, самобытности, уникальности и неповторимости»}, 
, syktyvkar.1istochnik.ru, 31.03.2021
%%%endcit
