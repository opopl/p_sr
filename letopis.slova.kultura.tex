% vim: keymap=russian-jcukenwin
%%beginhead 
 
%%file slova.kultura
%%parent slova
 
%%url 
 
%%author 
%%author_id 
%%author_url 
 
%%tags 
%%title 
 
%%endhead 
\chapter{Культура}
\label{sec:slova.kultura}

%%%cit
%%%cit_pic
%%%cit_text
Оскільки було силове впровадження росмовної \emph{субкультуру}, має бути силове витіснення її з України
%%%cit_comment
Ярослав Трінчук
%%%cit_title
\citTitle{...ми ще на стадії формування українського музичного ринку - відвойовуємо його у Росії}, 
Про Мову, facebook, 18.06.2021
%%%endcit

%%%cit
%%%cit_head
%%%cit_pic
\ifcmt
  pic https://img.vz.ru/upimg/m11/m1104023.jpg
\fi
%%%cit_text
Когда Путин в послании Федеральному собранию сказал фразу о Киплинге, которая
сразу стала мемом, он, с одной стороны, отсылал, понятно, к нравам «Книги
джунглей», прозрачно намекая на ситуацию в современной международной политике.
Но, с другой стороны (и этот более глубокий месседж также, думаю, был считан
всеми, кому он предназначался), реагировал на ту беспрецедентную атаку на
традиционную христианскую европейскую \emph{культуру}, которую мы наблюдали весь 2020
год, и которая сегодня продолжает лавинообразно нарастать
%%%cit_comment
%%%cit_title
\citTitle{Россия должна жить по заветам Киплинга}, 
Владимир Можегов, vz.ru, 18.06.2021
%%%endcit

%%%cit
%%%cit_head
%%%cit_pic
%%%cit_text
Не менее сложной и весьма затратной задачей является суверенизация
информационного пространства и установление государственного контроля над ним.
Феномен \emph{культурного влияния} на общество через социальные сети, видеоплатформы,
которые ранее заявлялись «свободными», а оказались четко регулируемыми и жестко
модерируемыми инструментами информационного и \emph{культурного воздействия}, доказал,
что власти Китая были полностью правы, когда распространения такого влияния на
своей территории не допустили
%%%cit_comment
%%%cit_title
\citTitle{Революция Духа – единственный путь спасения России}, 
Юрий Барбашов, voskhodinfo.su, 30.06.2021
%%%endcit

%%%cit
%%%cit_head
%%%cit_pic
%%%cit_text
Результаты незаконченной \emph{культурной вестернизации} России в 19-начале 20 века
были полностью уничтожены во времена сталинской деспотии. Начиная с хрущевских
реформ, страна опять стала открываться миру, \emph{культурная} вестернизация
постепенно возобновилась.  Новой попыткой движения в Европу стала Перестройка.
\enquote{Жить как на Западе} - это в то время была идея-фикс советских людей
%%%cit_comment
%%%cit_title
\citTitle{Как разведенный диктатор-отравитель собирается защищать \enquote{традиционные ценности}}, 
Игорь Эйдман, opinions.glavred.info, 06.07.2021
%%%endcit

%%%cit
%%%cit_head
%%%cit_pic
%%%cit_text
Ключовим тут є слова «\emph{одна культура}». У свідомості президента українська
культура за імперської доби не існувала окремо від російської, \emph{культура} була
спільною для обох народів. Фраза про популярність в Росії «неймовірної
української літератури» належить до порожньої, позбавленої сенсу риторики, бо
українськими творами цікавились у РСФСР хіба що поодинокі українці, що там
мешкали
%%%cit_comment
%%%cit_title
\citTitle{Українська мова в культурному просторі держави. Протистояння триває}, 
Лариса Масенко, www.radiosvoboda.org, 11.07.2021
%%%endcit

%%%cit
%%%cit_head
%%%cit_pic
%%%cit_text
Строить правовую, а не моноэтническую страну.  Современная политическая
ситуация в Украине ставит перед миллионами русскоязычного населения непростую
дилемму. Либо сохранить свой язык и \emph{культуру}, но предать украинское
государство, стать гражданами другой страны, в данном случае России, тем более
что из Украины их выталкивают, а в Россию приглашают. Либо, если они не хотят
предавать Украину, то должны забыть свой язык, свою историю, свою \emph{культуру} и
быть до конца покорными украинской власти, а что самое страшное – быть вместе с
радикалами и националистами, с их красно-черными знаменами, атаками на русский
язык, нетерпимостью к «неправильному православию», оправданием нацизма и
презрением ко всему советскому без разбора, будь то признание подвига
советского народа в победе в Великой Отечественной войне или понимание, что
весь нынешний промышленный потенциал Украины был построен советскими людьми в
советское время по приказу советской власти. Радикалы и власть имущие провели
четкую границу: все хорошее – например, успехи «Динамо» (Киев), полет в космос
Поповича, творчество композитора Ивасюка, самолет «Мрия» – украинское, а все
плохое – Чернобыльская трагедия или Голодомор – наследие советской эпохи. Дошло
до того, что украинские власти заявили, что Освенцим освободили войска Первого
Украинского фронта, не упомянув, что это был один из фронтов Красной армии
%%%cit_comment
%%%cit_title
\citTitle{О будущем украинского и русского народов}, 
Виктор Медведчук, strana.ua, 15.07.2021
%%%endcit

%%%cit
%%%cit_head
%%%cit_pic
%%%cit_text
Чому протягом 2017–2020 рр. Держкомтелерадіо дозволив ввезти в Україну 31 тис.
найменувань книг з Росії? Нібито більшість із дозволених книг – це твори
світової класики, дитяча та науково-популярна література. І все ж, подібна
\enquote{нейтральна} експансія нічого доброго не принесе в Україну, оскільки навіть
такий книжковий продукт РФ, що нібито не є загрозою національній безпеці, є
сутнісним складником політики \enquote{руського міра}. Адже \enquote{закохані} в російську
мову, літературу, \enquote{велікую \emph{культуру}}, \enquote{історіческоє велічіє} українські
громадяни не є носіями українськості, а радше їх світогляд формується в ніші
панросійськості. Саме цей людський капітал (матеріал), створений поза
українською мовою та \emph{культурою}, кремлівці розглядають як плацдарм для своїх
майбутніх агресій і проголошення сурогатних лялькових держав типу
\enquote{придністров’я, абхазії, північної осетії, лнр чи днр}. Гуманітарна зачистка на
території України через мову, освіту, інформацію, книговидавництво, концертну
діяльність, кіно, телеканали, радіо-FM, компрадорську інтелігенцію та
чужомовних владоможців, що і донині продовжують упевнено \enquote{правити} слухняним і
мовчазним народом, і є тією найбільш небезпечною і безупинною війною проти
України
%%%cit_comment
%%%cit_title
\citTitle{Правда в Рідному Слові}, Георгій Філіпчук, slovoprosvity.org, 12.07.2021
%%%endcit

%%%cit
%%%cit_head
%%%cit_pic
%%%cit_text
Активная \emph{культурная} жизнь – один из показателей жизнеспособности
государства. С этим в Донецкой народной республике все в порядке, здесь
развитию \emph{культуры} и искусства уделяют пристальное внимание.  В честь Дня
города, Дня шахтера и своего 10-летнего юбилея художественный музей
«Арт-Донбасс» проведет самую масштабную выставку из фондовой коллекции музея
«Квинтэссенция АРТа».  На выставке будет представлено более 250 работ донецких
художников, за практически семидесятилетний период, с начала 50-х годов XX
века, до наших дней, многие из них будут выставлены на суд зрителей впервые. По
результатам выставки будет издан каталог, в который войдут изображения всех
работ, представленных на вернисаже
%%%cit_comment
%%%cit_title
\citTitle{Донецк отметит День города культурными событиями}, , voskhodinfo.su, 26.08.2021
%%%endcit

%%%cit
%%%cit_head
%%%cit_pic
%%%cit_text
Ви, певно, знаєте — хоч за нашої \emph{напівкультурної} доби ви, мабуть, цього не
знаєте, — що деякі місцевості в басейні Амазонки досліджено лише частково і що
ця річка має численні притоки, багато з яких ще не нанесено навіть на карти. Я
хотів одвідати ці малознані краї та вивчити їхню фауну, щоб здобути матеріал
для кількох розділів великої монументальної праці в галузі зоології, яка,
сподіваюся, виправдає моє земне існування. Закінчивши цю свою роботу й
повертаючись-уже назад, я випадково спинився на ніч у маленькому індіанському
селищі коло самого гирла однієї з приток Амазонки; де саме ця притока і яка її
назва — цього я не скажу. Тамтешні тубільці належать до племені кукама. То дуже
приязний, але напівзвироднілий народ, розумові здібності якого навряд чи
перевищують здібності пересічного лондонця
%%%cit_comment
%%%cit_title
\citTitle{Утрачений світ}, Артур Конан Дойл
%%%endcit

%%%cit
%%%cit_head
%%%cit_pic
%%%cit_text
— Он будет в порядке, — сказал Джоунз. — Свои вопросы он еще сможет задать
позже. Он уже понял, что здесь нечто большее.  — Хорошо, — согласился Сторож. —
Мы могли бы создать величественную \emph{культуру}, единственную в Галактике, а может
и во всей Вселенной. Потому что у нас у первых появился разум и мы начали
намного раньше остальных. Мы могли бы создать образ жизни, превосходящий всякое
воображение, но среди нас в древние времена нашлись мудрецы, которые поняли,
что если мы пойдем таким путем, то окажемся в одиночестве, в изоляции от
остальных разумных существ. Мы должны были принять решение, и оно было принято.
Мы решили жить не для себя, а для других разумов, которые могли возникнуть в
Галактике
%%%cit_comment
%%%cit_title
\citTitle{Зачарованное паломничество}, Клиффорд Саймак
%%%endcit

%%%cit
%%%cit_head
%%%cit_pic
%%%cit_text
Сам Шевченко никогда не разграничивал малороссов и русских. Нигде и никогда не
назвал себя «украинцем». Поэт прекрасно знал русский язык, литературу и
\emph{культуру} в целом, которые были полноправными преемницами древнерусского языка и
\emph{культуры}. Большая часть прозы Шевченко, а также часть стихов написаны на
русском языке. Южнорусский поэт был «продуктом» именно русской культуры.
Представители \emph{русской культуры} (Жуковский, Брюллов, Григорович) и другие
помогли ему освободиться от крепостной неволи, стали учителями, помогли стать
на ноги. Сам Шевченко был частью столичной интеллигенции. В итоге поэт никогда
не разделял «Милую Украйну» и Россию. Даже в своём дневнике свою родину он
только пару раз называет Украиной, а остальных случаях Малороссией
%%%cit_comment
%%%cit_title
\citTitle{Шевченко без украинизма}, , topwar.ru, 09.03.2019
%%%endcit

%%%cit
%%%cit_head
%%%cit_pic
%%%cit_text
Окремі поетичні збірки, про кожну з яких думаєш, ніби така остання.  Окремі
театральні вистави, які от-от закриють.  Окремі художні виставки, після яких
обов'язково полетять голови.  Окремі кінофільми, про які відомо, що їх жахливо
порізали, однак і в такому, порізаному вигляді їх от-от заборонять.  Постійне
відчуття шпарини, крізь яку пролізуть далеко не всі.  Постійне відчуття того,
що навіть і цю вузесеньку шпарину от-от замурують.  Постійне відчуття, що
звідси треба їхати (тоді ще не казали валити) – наприклад, у Таллінн, Пітер або
Москву. Туди, де вільніше. Де немає цього особливого нагляду.  Український
\emph{культурний} стиль являв собою максимум обережності і, в кращому випадку,
завуальованої лояльності. В гіршому ж – лояльності відвертої, себто вже й не
лояльності, а самого що не є вірнопідданства, з обов'язковим ідеологічним
набором, який засигналізує всякого штибу кураторам твою світоглядну
правильність
%%%cit_comment
%%%cit_title
\citTitle{Українську культуру довелося витягати з дна і перестворювати з уламків}, 
Юрій Андрухович, gazeta.ua, 29.10.2021
%%%endcit

%%%cit
%%%cit_head
%%%cit_pic
\ifcmt
  pic https://zz.te.ua/wp-content/uploads/2011/01/bandera.jpeg
  @width 0.4
\fi
%%%cit_text
За традицією в Тернополі 1 січня відбувається урочиста академія з нагоди дня
народження провідника Організації Українських Націоналістів  Степана Бандери.
Цьогорічне вшанування Бандери пройде в особливій атмосфері. Тернопіль – єдине в
Україні місто обласного підпорядкування, де праворадикальна ВО \enquote{Свобода}
володіє всією повнотою влади. На місцевих виборах головою міста обрано
представника \enquote{Свободи} Сергія Надала, а в міській раді партія володіє
однопартійною більшістю.  Сьогодні о 15 годині урочистості стартують в палаці
\emph{культури} \enquote{Березіль}.  Також з ініціативи міського голови Сергія
Надала у рамках заходу пройде виставка документів та листівок визвольних
змагань періоду дії УПА в Україні, трансляція документальних фільмів часів
ОУН-УПА та урочиста академія
%%%cit_comment
%%%cit_title
\citTitle{В перший день Нового року Тернопіль святкує день народження Бандери}, 
, zz.te.ua, 01.01.2011
%%%endcit

%%%cit
%%%cit_head
%%%cit_pic
\ifcmt
  pic https://avatars.mds.yandex.net/get-zen_doc/1592767/pub_6180d6d880f2655b868c60cf_6180e405fd136c619517a4b3/scale_1200
  @width 0.4
\fi
%%%cit_text
После «самостийности» \emph{культура} на Украине стала бурно развиваться? А до
этого был колониальный застой и ущемление всех выступающих причиндалов
национального самосознания? Возьмём главный продукт этой сферы ХХ века. Кино. В
УССР существовали самые продвинутые в техническом плане и мощнейшие киностудии
— Киевская и Одесская. Выдали на экраны более 1 500 фильмов.  Многие из них —
шедевры мирового киноискусства. Современным националистам ничего не скажут
имена величайших художников: Довженко, Параджанов, Лесь Курбас, Марлен Хуциев,
Юнгвальд-Хилькевич, Кира Муратова, Пётр Тодоровский, Говорухин... Да что там
советские классики! Вы уверены, что кино родилось в Париже, его придумали
братья Люмьеры?  Три раза «ха». Значит, вы никогда не были в Одессе. Посетите
Второе христианское кладбище, найдите могилу с памятником из чёрного лабрадора.
Там высечено: «под ним лежит изобретатель первого в мире киноаппарата Иосиф
Андреевич Тимченко»
%%%cit_comment
%%%cit_title
\citTitle{Как Украина была колонией СССР... Не наоборот ли?}, 
Исторические напёрстки, zen.yandex.ru, 02.11.2021
%%%endcit

%%%cit
%%%cit_head
%%%cit_pic
%%%cit_text
Тому питання фізичного виживання може бути в конфлікті зі збереженням своєї
\emph{культурної} ідентичності. Не можна тупо протиставляти одне другому, а треба
вміти узгоджувати, розводити на різні поверхи важливості. Стратегічний рівень –
це найвищий рівень; тактичний рівень – другий, а ресурси і засоби – це третій
рівень. Ця система працює коли одночасно є і стратегічна мета, і тактична
майстерність, і володіння ресурсами та засобами. Саме тоді українська Україна
може мати майбутнє.  Інші варіанти – неукраїнська сильна Україна перестає бути
собою. А українська Україна, яка не здатна себе захистити, якщо вона буде
навіть дуже українською – просто програє у протистоянні з сильним ворогом.  Ми,
як мисляча частина нації, як частина духовної еліти, безумовно повинні
роз’яснювати важливість кожного з цих елементів, тому що нація – це система, а
не якісь фрагменти чогось.  Тактика завжди має бути наступальною. Почитайте
праці будь-якого теоретика інформаційних війн, де сказано, що сторона, яка
захищається, завжди програє. І наша стратегічна помилка дотримуватися мужицької
психології. Є така російська приказка: "Пока гром не грянет, мужик не
перекрестится". А ми не мужики, ми еліта. Ми нащадки Ноя. А Ной готувався до
того, щоб рятувати все живе від потопу. Він мислив стратегічно. А якби думав
ситуативно, то був би таким самим розгубленим, як і всі тварі, яких він посадив
на човен
%%%cit_comment
%%%cit_title
\citTitle{Як виграти інформаційну війну? – Слово Просвіти}, ,slovoprosvity.org, 01.11.2021
%%%endcit


