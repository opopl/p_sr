% vim: keymap=russian-jcukenwin
%%beginhead 
 
%%file 12_02_2022.fb.chernev_egor.1.inf_operacia_zapad_rossia
%%parent 12_02_2022
 
%%url https://www.facebook.com/cherneve/posts/5015568358481906
 
%%author_id chernev_egor
%%date 
 
%%tags infvojna,napadenie,nato,rossia,ugroza,ukraina,vtorzhenie,zapad
%%title Інформаційна спецоперація Заходу проти Росії
 
%%endhead 
 
\subsection{Інформаційна спецоперація Заходу проти Росії}
\label{sec:12_02_2022.fb.chernev_egor.1.inf_operacia_zapad_rossia}
 
\Purl{https://www.facebook.com/cherneve/posts/5015568358481906}
\ifcmt
 author_begin
   author_id chernev_egor
 author_end
\fi

США проводять проти Росії одну з наймасштабніших інформаційних спецоперацій в
історії. І це ми повинні чітко розуміти, коли чуємо про неминуче вторгнення. 

Завдання цієї спецоперації лежать на поверхні: 

1. Мобілізація країн НАТО і відновлення єдності Альянсу та Заходу в цілому

2. Демонізація РФ у світі та створення їй стійкого токсичного образу

3. Нанесення без війни якомога більших втрат для економіки РФ

4. Розпалювання антивоєнних настроїв у самій Росії

5. Деморалізація російської військової еліти через публічне викриття їхніх
секретних матеріалів 

Мета цієї спецоперації також є зрозумілою - остаточно витіснити Путіна з Європи
та унеможливити створення єдиного євразійського простору \enquote{від Владивостоку до
Лісабону} та проекту \enquote{Один пояс - один шлях}, де Росія відігравала би ключову
роль, а Україна була би її васалом. 

Для цього США хочуть змусити Путіна перестати гратися в гібридні війни та
зробити остаточний вибір: або наважитися на широкомасштабне вторгнення, або
остаточно відступити і замкнутися всередині власної країни. 

Паралельно з цим Вашингтон та Лондон вживають усіх заходів, щоб можливе
вторгнення стало економічною та військовою катастрофою для Росії, і щоб вибір
для Путіна став очевидним. 

Звичайно, як і кожна спецоперація, ця також несе у собі ризики. Для України,
насамперед, це означає суттєві економічні втрати через відтік інвесторів та
панічні настрої. Але це війна, і вона не буває без втрат. 

В той же час ці втрати є набагато меншими, ніж ті, яких ми могли би зазнати у
випадку широкомасштабного війського вторгнення. І ці втрати ми можемо зменшити
самі, припинивши паніку. 

Не треба панікувати навіть якщо Путін спробує вислизнути з американської пастки
і все ж віддасть наказ про вторгнення. Наша армія готова влаштувати росіянам
пекельний прийом, а санкційна відповідь Заходу швидко поховає економіку РФ. І
тоді на Росію може чекати розпад, а ми на довгі десятиліття отримаємо спокій
для мирного будівництва та розвитку. 

Keep calm
