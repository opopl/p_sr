% vim: keymap=russian-jcukenwin
%%beginhead 
 
%%file 03_01_2023.fb.ellie_greenflame.1.foto
%%parent 03_01_2023
 
%%url https://www.facebook.com/permalink.php?story_fbid=pfbid02dPtZFTmhE33843KCaNiagSHWfKft2RCRFR2xrkpjYZezPugF6pnSDf1t82yCfpDVl&id=100008651767938
 
%%author_id ellie_greenflame
%%date 
 
%%tags 
%%title Це найстрашніші фотки цього століття
 
%%endhead 
 
\subsection{Це найстрашніші фотки цього століття}
\label{sec:03_01_2023.fb.ellie_greenflame.1.foto}
 
\Purl{https://www.facebook.com/permalink.php?story_fbid=pfbid02dPtZFTmhE33843KCaNiagSHWfKft2RCRFR2xrkpjYZezPugF6pnSDf1t82yCfpDVl&id=100008651767938}
\ifcmt
 author_begin
   author_id ellie_greenflame
 author_end
\fi

Спочатку я не зрозуміла, що це. А потім зрозуміла. Це найстрашніші фотки цього
століття. Може, хтось скаже, що це фотошоп. Бо, знаючи, що там відбувається, я
не уявляю, що у когось був час "перекладувать їх туда-сюда". Найстрашніше у цих
фотках те, що ми посміхаємось. І я. Так їм і треба. Та жарти зі смертю не
смішні. там, на тих полях і наші теж. І не всіх знайдуть, і не всіх забируть. 

"4 дні нас не забирали з позиції. Ми дуже хотіли їсти і пити", а потім у
мертвого вояки з ДНР в рюкзаку знайшли пиріжки. І з'їли. Ще теплі. Мені
принесли хлопці, бо я був самий голодний. І я думав, хтось же йому їх готував.
Я не хотів їх забирати. Але дуже хотів їсти... Схід... я не знаю, про що це.

Найкраще почуття гумору у цивільних в чатіках, яких війна торкнулась
опосередковано. У рідних тих, хто на війні, почуття гумору відмерло, зламалося. 

...про смерть - не смішно. Страшно. Дико. Тому що реально. Хіба що вони мають
боятися навіть дивитися у наш бік. Так у давні віки натикали голови ворогів на
палі. Але це не смішно. Ні.
