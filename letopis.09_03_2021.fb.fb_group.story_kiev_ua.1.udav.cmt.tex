% vim: keymap=russian-jcukenwin
%%beginhead 
 
%%file 09_03_2021.fb.fb_group.story_kiev_ua.1.udav.cmt
%%parent 09_03_2021.fb.fb_group.story_kiev_ua.1.udav
 
%%url 
 
%%author_id 
%%date 
 
%%tags 
%%title 
 
%%endhead 
\zzSecCmt

\begin{itemize} % {
\iusr{Павел Пауль}
А с фамилией такой и ранг мог быть повыше. Недооценили?

\begin{itemize} % {
\iusr{Михайло Наместник}
\textbf{Павел Пауль} не, провалил тест на гибкость хребта

\iusr{Павел Пауль}
\textbf{Михайло Наместник} знакомо.
\end{itemize} % }


\iusr{Vadim Basovskiy}

\ifcmt
  ig https://i2.paste.pics/d4863c2ee2e4fa1e17855d948926fcc5.png
  @width 0.4
\fi

\begin{itemize} % {
\iusr{Ирина Иванченко}
\textbf{Vadim Basovskiy} , \enquote{грецкого полку} прибыло...

\iusr{Vadim Basovskiy}

Что же, всё-таки, Ирина, Вас так мучает?

\iusr{Ирина Иванченко}
\textbf{Vadim Basovskiy} ещё один \enquote{психиатр}?

\iusr{Vadim Basovskiy}
\textbf{Ирина Иванченко}
А что, уже кто-то пытался лечить?

\iusr{Vadim Basovskiy}
\textbf{Ирина Иванченко}
Вообще-то, я ничего не советовал, это Вы сами про психиатра вспомнили.

\iusr{Ирина Иванченко}
\textbf{Vadim Basovskiy} ,тогда прошу прощения
\end{itemize} % }

\iusr{Vladimir M. Pushkarev}

ВОПРОСЫ КРОВИ

«Человек будущего – это тот, у которого самая длинная память» (Ф. Ницше).

За годы советской власти нас как-то отучили интересоваться своим
происхождением. «Совку» родословная была ни к чему. Не положено. Вся
человеческая история была поделена на 2 периода – темный – до 17-го года и
светлый – после. В темном периоде тоже были отдельные светлые пятна, но они
касались лишь борьбы пролетариата за освобождение от цепей и будущий
коммунистический рай. Поэтому, если ты из голодранцев – ну, можно еще было
погордиться и даже воспользоваться этим в качестве козыря для карьеры (сын
трудового народа) или для пролетарского снисхождения при посадке. А если, не
дай бог, другого происхождения – лучше и не вспоминать – сильно отягчающее
обстоятельство. Так, что деда еще кое-как помним, а дальше – туман и полная
неизвестность. Я в этом плане – не исключение.

Родился я в Черновцах – уже поэтому ничего нельзя гарантировать. Такого жуткого
смешения наций и народов, наверное, мало где можно встретить. Часть моей крови
– украинская. Не исключено, что были в роду и румыны – бабушкина фамилия Ракоча
– звучит как-то подозрительно. К тому же у меня наблюдается нездоровая тяга к
румынской музыке, особенно в обработке Владимира Космы. Без всякого сомнения –
есть русские (отец – «оккупант»). Дед и прадед из Смоленской губернии. Говорят,
что одна из прабабок была полячкой (губерния до декабря 1793 г. была под Речью
Посполитой). Я так думаю, что все дерьмо в моем характере именно от той
прабабки. Евреев, скорее всего у меня в роду не было, хотя многие мои знакомые
в этом сомневаются, главным образом из-за моих «деловых» качеств. О турках,
татарах и прочих монголах ничего не могу сказать – возможно все, но
документальных данных у меня нет. Правда, согласно проведенному недавно в
Японии генетическому анализу - я не какой-нибудь там узкоглазый азиат, а, в
отличие от основной массы населения, настоящий, подлинный европеец, то есть
человек Белой расы! Так, что, по всей видимости, татаро-монголы до моих предков
не добежали.

В происхождении фамилии возможны два варианта – военный и производственный. И
если папа был пушкарь, то дети – Пушкаревы. По слухам, какой-то родственник
участвовал в Бородинском сражении. Ну, чего не знаю – того не знаю. Среди
смоленских Пушкаревых были мелкопоместные дворяне и какие-то советники, но не
выше надворных. Не могу сказать, чтобы я этим сильно гордился. Не знаю, связано
ли это с «высоким» происхождением, но дед, не занимаясь активной антисоветской
деятельностью, в 37-м присел. Есть еще остров, названный в мою честь (Пушкарев)
где-то в районе Чукотки. Значит кто-то из предков или однофамильцев добежал аж
туда!

Со стороны матери – почти никаких данных. Вроде бы дед заявился в Черновцы из
Бессарабии («сам я извиняюсь с Кишинева»).

В общем – генеалогическое «древо» – жидковатое. Будем надеяться, что потомки
будут относиться к нему более бережно.

Мне кажется, что людей можно разделить по типу рождения на 2 группы. Рождение
особей из первой, к которой относится, например, моя жена, является практически
предопределенным, независимым от геополитической ситуации. Ее мама с папой жили
в одном селе, где и встретились друг с дружкой с высокой степенью вероятности.
Я отношусь к другой группе людей, рождение которых является крайне
маловероятным, очень случайным событием в результате перемешивания наций,
народов и всяческих политических катаклизмов. Условия моего появления на свет
начали формироваться где-то в районе средневековья. Ведь как-то надо было моей
бабке-полячке и ее предкам появиться в Смоленской губернии. Если же
рассматривать более близкие к нам события, то необходим был распад
Австро-Венгрии, захват Буковины и Бессарабии румынами в 1918, захват этих же
территорий Советами в 1940 г., вторая мировая война, после которой мой предок
из-под Смоленска оказался на Буковине. В результате я получил дикое смешение
российской, польской, румынской и украинской кровей. Это не считая еще
возможных примесей.

В общем, таких как я надо заносить, как минимум, в книгу Гиннеса, а лучше в
Красную книгу человечества.

\begin{itemize} % {
\iusr{Елена Абдаллах Матющенко}
\textbf{Vladimir M. Pushkarev} 

Обязательно, конечно, надо упомянуть годы Советской власти, только не уточнить,
какого периода и разлива были эти годы. Не помню, чтобы меня отучали
интересоваться своими предками. Да и Вас, видимо, не отучили.

\iusr{Lesik Machynsky}
А что мне думать и предполагать???

\ifcmt
  ig https://scontent-frt3-2.xx.fbcdn.net/v/t1.6435-9/159105898_5179260218812932_6656690874772184184_n.jpg?_nc_cat=103&ccb=1-5&_nc_sid=dbeb18&_nc_ohc=UrRDBVQ4G7wAX98fPri&_nc_ht=scontent-frt3-2.xx&oh=00_AT8jGqJKFRI2Oj--38N5a8EmXfGlX2gZYtLxvkv4M8guaQ&oe=6202A4AC
  @width 0.3
\fi

\iusr{Ирина Иванченко}
\textbf{Lesik Machynsky}, Ваше високоблагородіє, чего изволітє... Мачин Таврічєскій...

\iusr{Vladimir M. Pushkarev}
\textbf{Lesik Machynsky} Ну, с такой родословной надо в турецкие президенты баллотироваться!

\iusr{Lesik Machynsky}
\textbf{Vladimir M. Pushkarev} Мачинские македонцы.
Кто-то принял Ислам и сделал карьеру в Порте.

\iusr{Lesik Machynsky}
\textbf{Ирина Иванченко} моє коріння з Бесарабії, біженці від турків. Православіє зберегли.

\iusr{Vladimir M. Pushkarev}
\textbf{Lesik Machynsky} Мой дед тоже из Бессарабии. Только подробностей я не знаю.

\iusr{Александр Муратов}

В. М. Пушкарев. Вы на само деле считаете, что по крои можно определить
национальность. Я не слыхал, чтобы где-то делали анализ крови человека не его
принадлежность к какой-либо нации?  @igg{fbicon.wink} 

\end{itemize} % }

\iusr{Валерий Рогожников}

У меня есть собака Лизка. У неё 50\% хаски, 25\% немецкая овчарка и 25\%
бельгийская овчарка. Мама у неё была, похоже, блядь. Но мы нашу Лизку-падлу
любим, а она нас. И это главное.

\begin{itemize} % {
\iusr{Юрий Загорий}
\textbf{Валерий Рогожников}, 

касательно вашей собаки: \enquote{тут без водолаза не обошлось... Потаскуха была моя
бабушка, царство ей небесное, старушке...} (с) \enquote{Собачье сердце}

\end{itemize} % }

\iusr{Шира Жемчужная}
Такое...

\iusr{Александр Квитницкий}
\textbf{Михайло Наместник} радует, что не \enquote{Ваше преосвященство}))

\iusr{Oksana Shukayeva}

Странненький пост... типа, если в роду были люди выдающиеся, то не надо
гордиться... сиди тихо и не высовывайся, что ты киевлянка в 5м поколении
@igg{fbicon.thinking.face}  @igg{fbicon.face.monocle} 

\begin{itemize} % {
\iusr{Михайло Наместник}
\textbf{Oksana Shukayeva} вовсе нет. Если у Вас есть что высунуть - высуньте.

\iusr{Галина Бойко}
\textbf{Oksana Shukayeva} а на аватаре написано у Вас - думай...

\begin{itemize} % {
\iusr{Oksana Shukayeva}
\textbf{Галина Бойко} и вам советую @igg{fbicon.thinking.face} 

\iusr{Галина Бойко}
\textbf{Oksana Shukayeva} 

ой, Вы ещё и из Страны Советов, вот откуда манеры залезть в чужие аккаунты и
смотреть, кто где учился. Примите к сведению, внучка почётного гражданина, в
приличном обществе за это бьют.

\iusr{Oksana Shukayeva}
\textbf{Галина Бойко} 

Не тебе мне писать... разбираюсь получше твоего, \enquote{хлопского} ума... у меня в
родне дворяне и дед мой был \enquote{Почетный гражданин Киева}... не каждому дано... и
есть чем гордится! А ты, лимита Казанская, брысь в бан @igg{fbicon.face.money.mouth} 

\end{itemize} % }

\iusr{Sergiy Pohvata}
\textbf{Oksana Shukayeva} да хоть в десятом) что выпячивать то?)

\iusr{Oksana Shukayeva}
\textbf{Sergiy Pohvata} давит-таки, да @igg{fbicon.frog} 

\iusr{Sergiy Pohvata}
\textbf{Oksana Shukayeva} по поводу?))

\iusr{Pavlo Bondarenko}
\textbf{Oksana Shukayeva} первое поколение вме равно были, \enquote{понаехами} @igg{fbicon.beaming.face.smiling.eyes} 

\iusr{Маргарита Мышанская}

Важно ваши наработки, личные, а не в каком поколении Вы киевлянка. Я видела
обезумевших, неприятных киевлянок в разном количестве поколений. А алкоголичек
тоже.

\begin{itemize} % {
\iusr{Oksana Shukayeva}
\textbf{Маргарита Мышанская} 

а еще в КПИ учились... про мои наработки не вам судить, а они есть, а вот предки
мои очень много сделали для Киева... первый трамвай был запущен в тч моим
дедом, \enquote{почетным гражданином г. Киева}! И да, я этим очень горжусь... но вам не
понять, как и многим, что \enquote{понаехали}...


\iusr{Маргарита Мышанская}
\textbf{Oksana Shukayeva} 

Оксана, моя бабушка родилась в Киеве в 1913 году на Соломенке. В 37-ом, там же,
на Соломенке родилась моя мама. Мой дед, штурман летной бригады тяжелой
авиации, погиб в июле 1941 года под Киевом. Так что Вы поняли... Бабушка
работала экономистом в Летном училище, мама-успешный врач. Я много лет работала
в недвижимости, я видела \enquote{аппартаменты} коренных киевлян. Иногда в их
квартиры страшно было зайти. Горы мусора, хлама, грязи, провода висели прямо
над головой. К сожалению, пример аккуратности, красивого общения, благородства
они не демонстрировали


\iusr{Маргарита Мышанская}
Это Ваш дед был \enquote{почетным гражданином г.Киева.} А в Вас я вижу пренебрежение к людям и высокомерие.

\iusr{Oksana Shukayeva}
\textbf{Маргарита Мышанская} 

по всему видать, вы не из \enquote{благородных кровей}, раз так их ненавидите... плебс,
он всегда был такой, даже в Nном поколении... и да - мне не интересно в этой
рабочей группе якобы \enquote{коренных киевлян} - в бан

\end{itemize} % }

\iusr{Маргарита Мышанская}
Сгинь, нечистый дух! @igg{fbicon.laugh.rolling.floor} 

\end{itemize} % }

\iusr{Світлана Проценко}
Браво, Киянине

\iusr{Інна Кривенко}
Гарно написали, підкажіть де такий тест зробити ?

\begin{itemize} % {
\iusr{Наталя Кудря}
\textbf{Інна Кривенко} гарне питання!!! особливо в нашій країні- тут немає такого обладнання як в американців- трішки слини й миттєво - ну принаймні в серіалах.... @igg{fbicon.wink} 

\iusr{Інна Кривенко}
\textbf{Наталя Кудря} жаль , кума в Канаді дітям теж зробила, думаю може в Києві є щось таке

\iusr{Наталя Кудря}
\textbf{Інна Кривенко} 

можливо й є - питання скільки коштує - бо в ті часи коли ми його робили це було
25 кубиків крові й 4 дні роботи лабораторії - а це не дешево зараз ... @igg{fbicon.wink} а якщо
врахувати \enquote{точність} хоч би сучасних тестів на вірус - то чи варто воно
того @igg{fbicon.wink} 


\iusr{Інна Кривенко}
\textbf{Михайло Наместник} Дякую !

\iusr{Ирина Сербина}
\textbf{Інна Кривенко} погуглите сайт MyHeritage
Я делала у них. Набор для забора биоматериала присылают по почте. И вы по почте отправляете. Делают в Чикаго! Ответ приходит на электронную почту. Мне результат пришел менее, чем за месяц.
По скидке (это у них часто,) - 60 долларов

\iusr{Інна Кривенко}
\textbf{Ирина Сербина} благодарю !
\end{itemize} % }

\iusr{Галина Бойко}
Спасибо Вам за этот пост. Полностью согласна.

\iusr{Ирина Иванченко}

Кошевых атаманов в роду не искали,вашбродие?С таким
процентным \enquote{украинством}, точно \enquote{гетьманский нащадок}... И хфамилия очень в
тему... \enquote{В чём сила, брат?}

\begin{itemize} % {
\iusr{Helga Etcetera}
\textbf{Ирина Иванченко} 

Причем тут украинство? Вам совсем о другом писали. \enquote{Фима, читайте буквы,
сойдете за грамотного, устроитесь завхозом!}

\iusr{Елена Абдаллах Матющенко}
\textbf{Helga Etcetera} 

Как раз об украинстве и писали, даже проценты соответствия привели. Другое
дело, что полная хня этот отчёт об истоках, купленный на день рождения. И ещё
писали, что даже такой высокоблагородный находит в себе достаточно
благородства, чтобы упомянуть об этом, когда никто не спрашивает о голубизне
его крови, и указать всякому сброду, что он-де не кичится столетиями,
проведенными в благословенном Киеве, а мог бы. О чем вообще этот пост? Я нигде
ни разу не встретила здесь человека, который бы пытался мерить себя и
окружающих географической точкой рождения. С кем надо общаться, чтобы возникла
такая тема, требующая обсуждения? Хороши же благородия

\iusr{Helga Etcetera}
\textbf{Елена Абдаллах Матющенко} 

Генеалогию обсуждать не буду, не моя квалификация, а вот комментарии \enquote{понаехали
тут}, к сожалению, встречаются в группе. Я не говорю, что массово. Но имеет
место быть. Помню очень трогательный пост некой киевлянки о покинутых сельских
хатах. Милый такой, лиричный. Господи Боже! Я просто поразилась, сколько народу
ей пришло объяснять, как им надоели приезжие, что сельские все жлобы и быдло,
портят Киев. Ради справедливости должна добавить, что этот пост, кажется,
администраторы удалили потом. По крайней мере, я не нашла. А искала. С целью
извиниться. От себя за всех киевлян. Потому что мне было очень стыдно.

\iusr{Павел Масалов}
\textbf{Елена Абдаллах Матющенко} \enquote{благородия} закончились в 1917-м  @igg{fbicon.grin} 

\iusr{Маргарита Мышанская}

Реально стыдно. Если бы в селах были хорошие условия для
жизни, инфраструктура, работа, то люди массово не выезжали бы в крупные города в
поисках работы. То, что мы родились в Киеве и нам повезло больше, чем им - не наша
заслуга личная. Поэтому я воздерживаюсь унижать людей только за то, что место их
рождения - не Киев...

\end{itemize} % }

\iusr{Александр Асатуров}
про що цэ?

\iusr{Петр Кузьменко}

+2

До чого ця стаття у нашій групі? Доречніша вон була би у \enquote{Клубі корінних киян}.
Наскільки я розумію ця група для всіх закоханих у наше Місто.

\begin{itemize} % {
\iusr{Оксана Дубинина}
\textbf{Михайло Наместник} 

Вам, Ваше високоблагородіє, попередження з занесенням в особову справу! Якщо
бажаєте ділити території, то Вам не до нас...

\begin{itemize} % {
\iusr{Михайло Наместник}
\textbf{Оксана Дубинина} слушаю и повинуюсь.

\iusr{Ирина Иванченко}
\textbf{Оксана Дубинина}, и где эти \enquote{особисті особові} хранятся, в каком \enquote{отделе кадров}?

\iusr{Оксана Дубинина}
\textbf{Ирина Иванченко}
Приходится запоминать @igg{fbicon.face.smiling.sunglasses}  @igg{fbicon.face.grinning.smiling.eyes}{repeat=2} 
Мы ж обязаны блюсти покой наших уважаемых КИ-ишников!)))
\end{itemize} % }

\iusr{Петр Кузьменко}
\textbf{Михайло} навіщо провокувати мене? В мене достатньо особистої інформації на
сторінці, на відміну від Вас (зауважте я пишу \enquote{Вас} з великої літери, знов на
відміну). Якщо Ви прочитаєте хоч деякі з моїх постів у нашій групі, зрозумієте
відповідь на Ваше питання.

\iusr{Александр Асатуров}
чем-то пахнет......точно не благородием

\iusr{Діана Шевченко}
\textbf{Петр Кузьменко}  

@igg{fbicon.laugh.rolling.floor}{repeat=3} розумієте залежно від числа колін
різна інтенсивність любові @igg{fbicon.laugh.rolling.floor}{repeat=3} 

\begin{itemize} % {
\iusr{Петр Кузьменко}
\textbf{Діана Шевченко} 

не згоден. В мене тих київських колін до біса. Маю багато друзів - киян у
другому, та навіть першому коліні, які шалено закохані у Місто. Не меньше, ніж
Ваш покірний слуга.


\iusr{Діана Шевченко}
\textbf{Петр Кузьменко} 

шановний, ну на жаль я стикалась з людьми, які вважають, що число колін на щось
впливають...от буквально як 3-4 дні, як пані у 4 коліні родповфдала мегі які
скатертини та постіль були в її мами, дуже накрахмалені... але коли я їй такі
запропонувала, вона сказала, що я дурна??? @igg{fbicon.thinking.face}  я вже мовчу що в моєї бабусі,
мами та тіток були такі самі, хоч і не у Києві....тому начзати мені на ті
коліна

\iusr{Александр Асатуров}
\textbf{Діана Шевченко} а в мэнэ тильки два колина... ливэ и правэ

\iusr{Петр Кузьменко}
\textbf{Діана Шевченко} 

ну, зовсім начхати, вважаю, не вірно. Людина повинна знати своє коріння та
родовід, якщо це можливо. Можна і потрібно пишатись пращурами, якщо вони були
достойними людьми. Доречі, моя мама теж завжди крохмалила постільну білизну, в
нас на Андріївському узвозі.

\iusr{Діана Шевченко}
\textbf{Петр Кузьменко} 

а хто каже що я не пишаюмь, я пишаюсь кожною краплиною крові в своєму
тілі.... бо кожну дали мені ті хто був раніше... і я вдячна.... і обожнюю Київ,
але число колін не впливає на інтенствність моєї любові... як і те що любили мої
пращури.... це мій досвід тут і зараз.... як і кожного хто живий.... тому одні
залишаються, інші їдуть, хтось любить, хтось ненавидить... хтось все життя живе
на одній вулиці, а хтось все житття переїзжає....

\iusr{Діана Шевченко}
\textbf{Александр Асатуров} ну і в мене також @igg{fbicon.laugh.rolling.floor}{repeat=4} 

\iusr{Александр Асатуров}
\textbf{Діана Шевченко} та не можэ такого буты!!!))))))

\iusr{Ирина Иванченко}
\textbf{Александр Асатуров}, \enquote{полный комплект!}

\iusr{Александр Асатуров}
\textbf{Ирина Иванченко} правда без запаски
\end{itemize} % }

\end{itemize} % }

\iusr{Оксана Омельяненко}
Моя бабуня - Явдокія Остапівна теж так казала: не хвались, а Богу молись

\iusr{Любовь Чернявская}

Прочитала с большим интересом. Как Ваше повествование воспримут современные
\enquote{патриоты} Украины, которые на каждом шагу кричат о чистоте своей крови? Да, мы
не интересовались родословной, а теперь и не особо восстановишь. Примерно в
91-92 году на похоронах двоюродной сестры отца узнала от старшей сестры
покойной, что их и отца прапрадед был выслан из Варшавы на Херсонщину за
участие в революции 1848г, отсюда фамилия. А я всегда считала себя чистокровной
украинкой. (Но встречала однофамильцев поляков и евреев). У сына-фамилия
русская, хотя у мужа отец был белорус, а мать-украинка. Невольно вспомнишь: \enquote{да,
скифы мы...}

\end{itemize} % }
