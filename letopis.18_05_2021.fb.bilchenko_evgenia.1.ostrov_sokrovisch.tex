% vim: keymap=russian-jcukenwin
%%beginhead 
 
%%file 18_05_2021.fb.bilchenko_evgenia.1.ostrov_sokrovisch
%%parent 18_05_2021
 
%%url https://www.facebook.com/yevzhik/posts/3903889906312757
 
%%author 
%%author_id 
%%author_url 
 
%%tags 
%%title 
 
%%endhead 
\subsection{БЖ. Остров сокровищ}
\label{sec:18_05_2021.fb.bilchenko_evgenia.1.ostrov_sokrovisch}
\Purl{https://www.facebook.com/yevzhik/posts/3903889906312757}

Купила я себе детский раскрашенный пиджачок,
Как радужный паучок, как кистью по шёлку - чёрк.
Как сверчок зелёный у Буратино, как красный лук,
Которым закусывают всухую серую грусть разлук.

\ifcmt
  pic https://scontent-iad3-1.xx.fbcdn.net/v/t1.6435-0/p180x540/186564302_3903889819646099_7957236329556320840_n.jpg?_nc_cat=110&ccb=1-3&_nc_sid=8bfeb9&_nc_ohc=r7XbB1A9scsAX87AVke&_nc_ht=scontent-iad3-1.xx&tp=6&oh=066bf5566772dbeedbd7024b4d4e4218&oe=60CBDFD2
\fi

Пиджачок похожий на "разукрашку", коль нелитературно.
На раскраску жакетов у Евтушенко, на пряник цветной из Тулы,
А если сказать культурно, если не придираться,
То на всё киношное и всамделишное пиратство,
Где музыка Быстрякова и Стивенсона бравада:
В таком пиджачке докторам наук явно ходить не надо.

Но Паоло Пазолини сказал, что глядят поэты,
Как покойник в движущемся гробу, снизу и вверх, - на это.
Так начинается чудо - литературы, камеры и кино.
Так возвращаются краски лета в содом земной.
Так из горящего города цвета утекают, тая,
На гуашной спине кита до стены Китая.

Я верю в искренность фильмографии, в смелость слога,
В бескорыстность живописи Ван-Гога, в импрессионизм от Бога.
Я чувствую, более ждать нельзя: надо спешить прощаться
И возвращаться - в сопли, в книги, в песни и танцы, в счастье.

Родные мои, пожалуйста, не задерживайте меня!
За коня не полцарства - царство жалую за коня.
Купила себе пиджачок цветной, чтоб думать о том не сметь,
Что, если вы меня не вернёте обратно в Эдем, - мне смерть.
17 мая 2021 г.

Илл. "Детство" из ресурса "Избранное", автор не известен.
