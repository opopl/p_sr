% vim: keymap=russian-jcukenwin
%%beginhead 
 
%%file cmds.resizebox
%%parent body
 
%%endhead 
\section{\textbackslash resizebox}
\url{https://latexref.xyz/_005cresizebox.html}
  
\vspace{0.5cm}
 {\ifDEBUG\small\LaTeX~section: \verb|cmds.resizebox| project: \verb|latex2e| rootid: \verb|p_saintrussia| \fi}
\vspace{0.5cm}

\subsubsection{Synopsis:}

\begin{verbatim}
	\resizebox{horizontal length}{vertical length}{material}
	\resizebox*{horizontal length}{vertical length}{material}
\end{verbatim}

Given a size, such as \verb|3cm|, transform \emph{material} to make it that size. If either
horizontal length or vertical length is an exclamation point \verb|!| then the other
argument is used to determine a scale factor for both directions.

This example makes the graphic be a half inch wide and scales it vertically by
the same factor to keep it from being distorted.

\begin{verbatim}
	\resizebox{0.5in}{!}{\includegraphics{lion}}
\end{verbatim}

The unstarred form \verb|\resizebox| takes \emph{vertical length} to be the box’s height
while the starred form \verb|\resizebox*| takes it to be height+depth. For instance,
make the text have a height+depth of a quarter inch with

\begin{verbatim}
	\resizebox*{!}{0.25in}{\parbox{1in}{This box has both height and depth.}}.
\end{verbatim}

You can use \verb|\depth|, \verb|\height|, \verb|\totalheight|, and
\verb|\width| to refer to the original size of the box. Thus, make the text two
inches wide but keep the original height with
\verb|\resizebox{2in}{\height}{Two inches}|. 
  
