% vim: keymap=russian-jcukenwin
%%beginhead 
 
%%file 30_09_2021.stz.news.ua.misto_marii.1.filarmonia_nova_lokacia_shlub_za_dobu
%%parent 30_09_2021
 
%%url https://mistomariupol.com.ua/uk/filarmoniya-nova-nezvychajna-lokacziya-proyektu-shlyub-za-dobu
 
%%author_id news.ua.misto_marii
%%date 
 
%%tags 
%%title Філармонія - нова локація проєкту "Шлюб за добу"
 
%%endhead 
 
\subsection{Філармонія - нова локація проєкту \enquote{Шлюб за добу}}
\label{sec:30_09_2021.stz.news.ua.misto_marii.1.filarmonia_nova_lokacia_shlub_za_dobu}
 
\Purl{https://mistomariupol.com.ua/uk/filarmoniya-nova-nezvychajna-lokacziya-proyektu-shlyub-za-dobu}
\ifcmt
 author_begin
   author_id news.ua.misto_marii
 author_end
\fi

\ifcmt
  ig https://i2.paste.pics/PJUHC.png?trs=1142e84a8812893e619f828af22a1d084584f26ffb97dd2bb11c85495ee994c5
  @wrap center
  @width 0.9
\fi

\begin{quote}
Сьогодні у Маріуполі відкрилася нова локація проєкту \enquote{Шлюб за добу}. Тепер
створити родину можна прямо в стінах Маріупольської камерної філармонії. Нову
локацію презентували заступнику міського голови Маріуполя Михайлу Івченко,
представникам Східного міжрегіонального управління Міністерства юстиції (м.
Харків) та керівникам органів державної реєстрації актів цивільного стану
Донецької області.	
\end{quote}

\ii{30_09_2021.stz.news.ua.misto_marii.1.filarmonia_nova_lokacia_shlub_za_dobu.pic.1}
\ii{30_09_2021.stz.news.ua.misto_marii.1.filarmonia_nova_lokacia_shlub_za_dobu.pic.2}

\emph{\enquote{Ми співпрацюємо з Лівобережним РАЦС вже більше року, і за цей рік було
створено більше 700 нових родин. Це були абсолютно різновікові пари, що
особливо приємно. Дивлячись на цей досвід та попит на послугу, ми вирішили
створити нову локацію в Центральному районі міста, аби це було зручно
маріупольцям та трішки знизило навантаження на Лівобережний РАЦС}}, – розповіла
директорка ККП \enquote{m.EHUB} Марія Сльота.

\ii{30_09_2021.stz.news.ua.misto_marii.1.filarmonia_nova_lokacia_shlub_za_dobu.pic.3}
\ii{30_09_2021.stz.news.ua.misto_marii.1.filarmonia_nova_lokacia_shlub_za_dobu.pic.4}

\enquote{Шлюб за добу} – спільний проєкт ККП \enquote{m.EHUB} та Міністерства юстиції України
при співпраці з РАЦС. За офіційними даними серед молодят не тільки маріупольці,
але й мешканці інших міст та країн. Закоханих приваблювали чарівні локації й
можливість створити родину швидко, без жодних бюрократичних аспектів.

\ii{30_09_2021.stz.news.ua.misto_marii.1.filarmonia_nova_lokacia_shlub_za_dobu.pic.5}

\emph{\enquote{Беззаперечно скажу, що з моменту запуску пілотного проєкту \enquote{Шлюб за добу},
локації були різними. І ці місця дозволяють кожній людині зробити той вибір,
який вона собі намріяла. ми одружуємо в філармоніях, музеях, вежах. По всій
Україні це абсолютно різні локації. Люди обирають щось цікаве. У вашій
філармонії можна обіграти нічні церемонії, наприклад, із свічками. І в цю
філармонію точно увійде любов}}, – розповіла Ірина Дубиківська, начальник
Управління державної реєстрації актів цивільного стану Міністерства юстиції
України.

\ii{30_09_2021.stz.news.ua.misto_marii.1.filarmonia_nova_lokacia_shlub_za_dobu.pic.6}
\ii{30_09_2021.stz.news.ua.misto_marii.1.filarmonia_nova_lokacia_shlub_za_dobu.pic.7}


Офіційні цифри \enquote{Шлюб за добу}: 

\begin{itemize}
  \item усього 716 молодят створили родину завдяки цьому проєкту у Лівобережному Мультицентрі
  \item 90 учасників пар завітали до нас з інших країн
  \item перше золоте весілля на даху урочисто відсвяткували цього вересня
  \item найстарші молодята одружилися у віці 90 років
\end{itemize}

\ii{30_09_2021.stz.news.ua.misto_marii.1.filarmonia_nova_lokacia_shlub_za_dobu.pic.8}

Так проєкт маріупольського \enquote{Шлюб за добу} вплинув на міжнародний
весільний туризм нашого міста. Адже послуги обрали для себе мешканці Німеччини,
США, Молдови, Польщі, Латвії, Грузії, Республіки Білорусь, Нідерландів,
Вірменії, Літви, Казахстану, Узбекістану та Франції, які завітали до міста біля
моря.

\ii{30_09_2021.stz.news.ua.misto_marii.1.filarmonia_nova_lokacia_shlub_za_dobu.pic.9}

У планах рухатися тільки вперед, створювати нові локації або трансформувати
відомі місця, а також вдосконалювати послуги.

\ii{30_09_2021.stz.news.ua.misto_marii.1.filarmonia_nova_lokacia_shlub_za_dobu.pic.10}
\ii{30_09_2021.stz.news.ua.misto_marii.1.filarmonia_nova_lokacia_shlub_za_dobu.pic.11}
\ii{30_09_2021.stz.news.ua.misto_marii.1.filarmonia_nova_lokacia_shlub_za_dobu.pic.12}
\ii{30_09_2021.stz.news.ua.misto_marii.1.filarmonia_nova_lokacia_shlub_za_dobu.pic.13}
