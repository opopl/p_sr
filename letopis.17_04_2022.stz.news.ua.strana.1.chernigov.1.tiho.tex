% vim: keymap=russian-jcukenwin
%%beginhead 
 
%%file 17_04_2022.stz.news.ua.strana.1.chernigov.1.tiho
%%parent 17_04_2022.stz.news.ua.strana.1.chernigov
 
%%url 
 
%%author_id 
%%date 
 
%%tags 
%%title 
 
%%endhead 

Тихо. Такая устрашающая тишина бывает только на войне.

Угрюмое безмолвие нарушает ветер, холодный и пронизывающий. Слышно, как железо
бьется о железо — это ветер играет остатками дырявого металлического забора,
перекошенного и изрешеченного пулями. Сквозняк гуляет по обвалившимся стенам
домов и воет, как брошенная собака.

\ii{17_04_2022.stz.news.ua.strana.1.chernigov.pic.1}

Всюду разбросаны обломки кирпичей, досок и порванные детские игрушки. Окна
домов, в которых еще месяц назад горел свет, стали чёрными бесформенными
пятнами. Оконные рамы обуглились, стекла выбиты, крыши обвалились. Дома
выглядят так, будто их прожевали и выплюнули, а затем еще и растоптали. Там,
где раньше стояли чьи-то жилища, сейчас — груда камней и пепел. 

Сплошная череда обломков и развалин — вот то, что осталось от села Новоселовка
под Черниговом. Ему пришлось принять на себя сильнейший удар противника.
Новоселовке не повезло встать на пути российских войск, которые пытались через
нее прорваться в Чернигов. А когда не смогли, практически стерли село с лица
земли. 

Еще несколько недель назад здесь не стихали канонады: постоянные выстрелы,
взрывы, обстрелы. Но враг отбит — российские войска ушли в начале апреля.
Сейчас в Новоселовке тихо. Даже слишком. 

Местные признаются, что жить под бомбежками было страшно. Но привыкли. А
сейчас, когда стало тихо — еще страшнее. \enquote{Все время думаешь: а вдруг опять?..}

Снова слышен звук металла. Но другой. Ритмичный скрип... Оборачиваюсь и не верю
глазам: это мальчик катается на качели. 

Удивительно, как в этом полуразрушенном селе под бомбежками уцелела детская
площадка. Совсем новая, цветная, желто-красная — сюрреалистичное яркое пятно на
фоне черно-серого послевоенного пейзажа. Как будто эту площадку вырезали в
фотошопе из снимка довоенного времени и искусственно вставили сюда, словно все
это — картина неизвестного постмодерниста. Но реальность превосходит любой
художественный вымысел. О том, что здесь кровавой поступью прошлась война,
красноречиво напоминает воронка от снаряда прямо возле детской горки. 

\ii{17_04_2022.stz.news.ua.strana.1.chernigov.pic.2}

В нескольких метрах от этой воронки на качели катается мальчик по имени Степан.
Он живет в этом селе, и в свои 11 лет уже знает, что такое война. Он видел ее
своими глазами. 

— Что сказать... Война, разрушения... Я волновался, мне было страшно. Особенно
когда шли обстрелы. Но когда стреляли, я много спал. Может, такая защитная
реакция. Звуки выстрелов не мешали спать, я к ним привык. Стреляли с той горы
по нам. Мы все время прятались в погребе, — вспоминает мальчик. 

К счастью, его семье удалось выжить под постоянными обстрелами. 

— Мой дом пострадал, но не сильно. Так, полкрышы обвалилось просто, — Степан
говорит об этом так спокойно, будто речь идет о сломанном конструкторе. 

Поразительно, как дети умеют вопреки всему сохранять оптимизм. Единственное,
что сейчас печалит Степана — то, что разъехались все его друзья, и что пока
нельзя свободно гулять по селу. Много неразорвавшихся снарядов, мин, растяжек. 

— Но когда все закончится, мы тут погуляем, — не сомневается мальчик.

\enquote{Страна} побывала на Черниговщине и своими глазами увидела, как местные жители
больше месяца выживали в блокаде и под постоянными обстрелами российских войск. 

