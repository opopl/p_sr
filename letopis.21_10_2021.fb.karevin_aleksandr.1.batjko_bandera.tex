% vim: keymap=russian-jcukenwin
%%beginhead 
 
%%file 21_10_2021.fb.karevin_aleksandr.1.batjko_bandera
%%parent 21_10_2021
 
%%url https://www.facebook.com/permalink.php?story_fbid=3032845550288078&id=100006879888184
 
%%author_id karevin_aleksandr
%%date 
 
%%tags bandera_stepan,nacionalizm,obschestvo,patriotizm,ukraina
%%title БАТЬКО ИХ – БАНДЕРА
 
%%endhead 
 
\subsection{БАТЬКО ИХ – БАНДЕРА}
\label{sec:21_10_2021.fb.karevin_aleksandr.1.batjko_bandera}
 
\Purl{https://www.facebook.com/permalink.php?story_fbid=3032845550288078&id=100006879888184}
\ifcmt
 author_begin
   author_id karevin_aleksandr
 author_end
\fi

БАТЬКО ИХ – БАНДЕРА

Наверное, все интересующиеся уже посмотрели хотя бы один (вообще их много)
гуляющий в Интернете ролик, где альтернативно одарённая украинская молодежь
поёт (иногда при этом подтанцовывая) о том, что Бандера – это их батько и что
они будут воевать за Украину.

Ну а как же воевал за Украину сам Бандера?

В скольких боях он принимал участие? А ни в одном!

Сколькими военными операциями руководил? А нисколькими!

Сколько сражений выиграл? А нисколько!

Убийства из-за угла и прочая уголовщина в его биографии были. Участия в боевых
действиях – нет.

Правда, воевать можно не только оружием, но и словом. Тем самым словом,
которым, как сказал поэт, «можно полки за собой повести».

Однако и тут не складывается: не был Бандера ни пламенным оратором, увлекающим
за собой полки, ни способным публицистом, чьи статьи зажигали стремлением к
борьбе множество сердец.

И автором книг, изменивших мировоззрение значительного количества людей, он
тоже не был.

Тогда, может быть, он был успешным предпринимателем, зарабатывавшим деньги на
национально-освободительную борьбу?

И снова – нет. Денег Бандера зарабатывать не умел. Умел только тратить.

Получается, Степан Бандера был всего лишь лихим разбойничьим атаманом? 

И это не совсем так. То есть – бандитом он, безусловно, являлся. Лихости в нём
никакой не было.

Лихим разбойничьим атаманом был, например, батько Махно. Тот гулял по степям,
разделяя вместе со своими хлопцами все риски и тяготы разбойничьей жизни. 

Бандера же отсиживался на Западе, в комфортных условиях и далеко от театра
военных действий. Он отправлял на смерть других, получая за это вознаграждение
от иностранных разведок (сначала германской, потом английской и американской).
А сам вовсе не рвался на Украину, предпочитая пребывать в безопасности. В
относительной, конечно, безопасности, ибо возмездие его в конце концов
настигло. Но, вновь-таки, не в бою, ибо, повторюсь, он не воевал.

Так кому же после этого захочется под руководство такого «батьки»? Да только
малолетним недоумкам. У недоумков ведь в головах мозга нет. 

Откуда у них мозг? Батько их – Бандера.

\ii{21_10_2021.fb.karevin_aleksandr.1.batjko_bandera.cmt}
