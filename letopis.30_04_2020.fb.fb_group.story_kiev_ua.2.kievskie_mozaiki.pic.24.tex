% vim: keymap=russian-jcukenwin
%%beginhead 
 
%%file 30_04_2020.fb.fb_group.story_kiev_ua.2.kievskie_mozaiki.pic.24
%%parent 30_04_2020.fb.fb_group.story_kiev_ua.2.kievskie_mozaiki
 
%%url 
 
%%author_id 
%%date 
 
%%tags 
%%title 
 
%%endhead 

\ifcmt
  ig https://scontent-frx5-1.xx.fbcdn.net/v/t1.6435-9/95098170_3174547865912132_97897504565100544_n.jpg?_nc_cat=110&ccb=1-5&_nc_sid=b9115d&_nc_ohc=G8zbf9PzW5cAX_sYLzc&_nc_ht=scontent-frx5-1.xx&oh=2f4bbafe82f44b1aaa136b18917b8c24&oe=61B5B2FC
  @width 0.4
\fi

\iusr{Alexandr Torgrim}
А це не центральний Палац дітей та юнацтва??

\iusr{Ирина Петрова}
Так, на площі Слави. В роки створення мозаік це був Центральний дворець піонерів. Ця композиція називається "Дружба народів"

\iusr{Alexandr Torgrim}
\textbf{Ирина Петрова} одразу згадав дитинство, коли на гуртки бігав туди...

\iusr{Nadiya M Shana}
Композиция так себе... @igg{fbicon.face.hand.over.mouth} 

\iusr{Ирина Петрова}
Яка дружба - така й композиція @igg{fbicon.face.zany} 

\iusr{Nadiya M Shana}
\textbf{Ирина Петрова} Оце так!  @igg{fbicon.laugh.rolling.floor} 

\iusr{Ирина Петрова}

Чотири зрозуміло - індіанський хлопчик, білява європейська дівчинка, азійська
жовтошкіра дівчинка та темношкірий хлопчик! А ось посередині - не можу
втокмачити - ніби й темні шкіра, чи засмага у Артеці, голівонька білява - то
точно піонер з Артеку після зміни!))) І чогось він рук не подає друзям???

\iusr{Nadiya M Shana}
\textbf{Ирина Петрова} Там питання до осього, руки, ноги, голови... Щось їх збентежило зверху...

\iusr{Ирина Петрова}
\textbf{Nadiya M Shana} та вони на сонечко!!! дивляться! @igg{fbicon.laugh.rolling.floor} 

\iusr{Ірина Маслова}
Мабуть то афроамериканець @igg{fbicon.wink} 

\iusr{Ирина Петрова}
Хлопчик з Африки трохи зблід шкірою, живучи в Америці?))) Може, може))))

\iusr{Ірина Маслова}
\textbf{Ирина Петрова} та ні народився там, тому і рук не подає, він наче "не наш" @igg{fbicon.face.grinning.squinting}{repeat=2} 

\iusr{Ирина Петрова}
\textbf{Ірина Маслова} оце вже сюжет для оповідання на тему життя меньшинств в капіталістичному краї  @igg{fbicon.laugh.rolling.floor} 

\iusr{Ірина Маслова}
\textbf{Ирина Петрова} та які вони меншини, що ви, вони зараз дуже поважні і мають дуже суттєві привілеї, мабуть автори щось знали наперед @igg{fbicon.face.tears.of.joy}{repeat=2} 

\iusr{Nina NinaNina}

Мозаїчне панно у Палаці дітей та юнацтва на вул. І. Мазепи, 13(автори А. Рибачук та В. Мельниченко; 1960-т
