% vim: keymap=russian-jcukenwin
%%beginhead 
 
%%file writers.stevenson_louis_robert
%%parent writers
 
%%url https://24smi.org/celebrity/13083-robert-liuis-stivenson.html
%%author 
%%tags 
%%title 
 
%%endhead 

\section{Роберт Льюис Стивенсон}
\Purl{https://24smi.org/celebrity/13083-robert-liuis-stivenson.html}

\subsection{Биография}

Имя Роберта Льюиса Стивенсона с детства знакомо всем, кто не представляет
жизни без книги. Невероятные и захватывающие приключения, которые
поджидают героев его произведений на каждом шагу, не раз заставляли
читателей часами просиживать за страницами «Острова сокровищ» и «Черной
стрелы». И хотя именно эти произведения считаются самыми известными в
библиографии писателя, список книг Стивенсона не ограничивается ими.

\subsection{Детство и юность}

Будущий писатель родился в Эдинбурге 13 ноября 1850 года. Отец мальчика
владел необычной профессией — был инженером, проектировавшим маяки. С
раннего детства мальчик много времени лежал в постели — серьезные диагнозы
заставляли родителей беречь сына.

\ifcmt
pic https://24smi.org/public/media/resize/800x-/2018/5/22/101_6FEwLb2.jpg
caption Портрет Роберта Льюиса Стивенсона
\fi

Стивенсону диагностировали круп, а позднее и чахотку (туберкулез легких),
которые в те времена часто становились смертельными. Поэтому маленький Роберт
много времени проводил в «одеяльной стране» - так писатель позднее напишет о
детстве.

Возможно, постоянные ограничения и постельный режим и помогли воображению
Роберта Льюиса Стивенсона развиться настолько, что тот начал придумывать
воображаемые приключения и путешествия, которые не мог совершить в жизни.
Кроме того, няня мальчика воспитывала в нем литературный вкус и чувство слова,
читая стихотворения Роберта Бернса и рассказывая сказки перед сном.

\ifcmt
pic https://24smi.org/public/media/resize/800x-/2018/5/22/102_6wnVADp.jpg
caption Роберт Льюис Стивенсон в детстве
\fi


Уже в 15 лет Роберт Льюис Стивенсон закончил первое серьезное
произведение, получившее название «Пентландское восстание». Отец Роберта
поддержал сына и издал эту книгу в 100 экземплярах за собственные деньги в
1866 году.

Примерно в то же время Стивенсон, несмотря на состояние здоровья, начал
путешествовать по родной Шотландии и Европе и записывать впечатления и
случаи из поездок. Позднее эти очерки вышли под обложкой книг «Дороги» и
«Путешествие внутрь страны».

\ifcmt
pic https://24smi.org/public/media/resize/800x-/2018/5/22/103_5ILMiRs.jpg
caption Роберт Льюис Стивенсон в молодости
\fi

Став постарше, Роберт Льюис Стивенсон поступил в эдинбургскую академию, а
затем и в Эдинбургский университет. Поначалу молодой человек пошел по
стопам отца и начал изучать инженерное дело. Однако позднее перешел на
факультет юриспруденции и в 1875 году стал дипломированным юристом.

\subsection{Литература}

Первым серьезным произведением Стивенсона, принесшим писателю известность,
стал рассказ под названием «Ночлег Франсуа Вийона». А уже в 1878-м
прозаик, находясь в очередной поездке во Франции, заканчивает цикл
рассказов, вышедших как единое целое.

\ifcmt
pic https://24smi.org/public/media/resize/800x-/2018/5/22/104_Z6BFcq8.jpg
caption Писатель Роберт Льюис Стивенсон
\fi


Этот сборник получил название «Клуб самоубийц» и позднее стал одним из
самых известных произведений Стивенсона. «Клуб самоубийц», а также цикл
рассказов «Алмаз раджи», были напечатаны во многих литературных журналах
Европы. Постепенно имя Стивенсона становилось узнаваемым.

Однако серьезную славу писатель узнал в 1883 году, когда был напечатан,
пожалуй, лучший роман Стивенсона - «Остров сокровищ». Как и многие
гениальные произведения, эта книга началась с шутливых рассказов, которыми
Стивенсон развлекал своего маленького пасынка. Роберт Льюис даже нарисовал
для мальчика карту придуманного острова, которая почти без изменений была
напечатана в предисловии к изданию.

\ifcmt
pic https://24smi.org/public/media/resize/800x-/2018/5/22/105_Yd53iOh.jpg
caption Иллюстрация к книге Роберта Льюиса Стивенсона «Остров сокровищ»
\fi

Постепенно разрозненные эпизоды начали складываться в полноценный роман, и
Стивенсон сел за бумагу. Изначально писатель дал книге название
«Корабельный повар», но позднее изменил его на «Остров сокровищ». В этом
произведении, как признавался Стивенсон, отразились его впечатления от
книг других авторов — Даниэля Дефо и Эдгара По. Первыми читателями
готового романа стали пасынок писателя и отец, но вскоре о книге
заговорили и другие любители приключенческой литературы.

Следующей из под пера писателя выходит «Черная стрела», в 1885 году
появляется «Принц Отто» и культовая повесть «Странная история доктора
Джекила и мистера Хайда». Годом позднее Роберт Льюис Стивенсон закончил
работу над очередным сборником рассказов, получившим название «И еще новые
тысяча и одна ночь» (или «Динамитчик»).

\ifcmt
pic https://24smi.org/public/media/resize/800x-/2018/5/22/106_9CmaRTi.jpg
caption Иллюстрация к книге Роберта Льюиса «Странная история доктора Джекила и мистера Хайда»
\fi


Примечательно, что Стивенсон писал и стихи, однако относился к стихотворным
экспериментам как к дилетантству и даже не пытался публиковать их. Но часть
стихотворений писатель все же собрал под одной обложкой и решился издать. Так
появился сборник поэзии Стивенсона, навеянный воспоминаниями о детских годах.
На русском языке стихотворения вышли в 1920 году и получили переводное название
«Детский цветник стихов».  Позднее сборник несколько раз переиздавался и
изменял первоначальное название.

К тому времени семья Стивенсона, благодаря «Острову сокровищ», жила безбедно.
Но, к сожалению, здоровье автора больше и больше давало о себе знать. Медики
посоветовали писателю сменить климат, и Роберт Льюис Стивенсон переехал из
родной страны на острова Самоа. Местные жители, поначалу настороженно
воспринявшие незнакомцев, вскоре стали постоянными гостями в гостеприимном доме
этого добродушного человека.

\ifcmt
pic https://24smi.org/public/media/resize/800x-/2018/5/22/107_xwvTyMw.jpg
caption Роберт Льюис Стивенсон на острове Самоа с гостями
\fi

За Стивенсоном даже закрепилась кличка «вождь-сказитель» - так называли
писателя аборигены, которым тот помогал советами. Зато белые колонизаторы
недолюбливали Роберта Льюиса Стивенсона за те настроения вольнодумия, что
писатель сеял в умах местных жителей.

Ну и конечно, экзотическая атмосфера острова не могла не отразиться в
произведениях сказителя: на Самоа написаны романы и рассказы «Вечерние
беседы на острове», «Катриона» (ставшая продолжением «Похищенного» -
романа, вышедшего раньше), «Сент-Ив». Некоторые произведения писатель
сочинял в соавторстве с пасынком - «Несусветный багаж», «Потерпевшие
кораблекрушение», «Отлив».

\subsection{Личная жизнь}

Первой влюбленностью писателя стала дама по имени Кэт Драммонд, работавшая
певицей в ночной таверне. Пылкий Стивенсон, будучи неопытным юношей,
настолько увлекся этой женщиной, что собрался жениться. Однако отец
писателя не позволил сыну взять в жены Кэт, которая, по мнению
Стивенсона-старшего, не подходила для этой роли.

\ifcmt
pic https://24smi.org/public/media/resize/800x-/2018/5/22/108_F1uxkxI.jpg
caption Роберт Льюис Стивенсон и его жена Фанни
\fi

Позднее, во время путешествий по Франции, Роберт Льюис Стивенсон
повстречал Фрэнсис Матильду Осборн. Фанни — так Стивенсон ласково называл
возлюбленную — была замужем. Кроме того, у женщины было двое детей и она
была старше Стивенсона на 10 лет. Казалось, это способно помешать
влюбленным быть вместе.

Поначалу так и произошло — Стивенсон уехал из Франции один, без
возлюбленной, оплакивая неудавшуюся личную жизнь. Но в 1880 году Фанни,
наконец, удалось развестись с супругом и обвенчаться с писателем, ставшим
в одночасье счастливым мужем и отцом. Общих детей у пары не было.

\subsection{Смерть}

Остров Самоа стал не только любимым местом писателя, но и последним
пристанищем. 3 декабря 1894 года Роберта Льюиса Стивенсона не стало.
Вечером мужчина по обыкновению спустился к ужину, но внезапно схватился за
голову, сраженный ударом. Через несколько часов писателя уже не было в
живых. Причиной смерти гения стал инсульт.

\ifcmt
pic https://24smi.org/public/media/resize/800x-/2018/5/22/109.jpg
caption Могила Роберта Льюиса Стивенсона на горе Веа
\fi

Там же, на острове, до сих пор сохранилась и могила писателя. Аборигены,
по-настоящему опечаленные смертью своего героя и «вождя-сказителя»,
похоронили Роберта Льюиса Стивенсона на вершине горы под названием Веа,
водрузив на могилу надгробие из бетона.

В 1957 году советский писатель Леонид Борисов написал биографию Роберта
Льюиса Стивенсона, получившую название «Под флагом Катрионы».

\subsection{Библиография}

\begin{itemize}
	\item * 1883 - "Остров сокровищ"
	\item * 1885 - "Принц Отто"
	\item * 1886 - "Странная история доктора Джекила и мистера Хайда"
	\item * 1886 - "Похищенный"
	\item * 1888 - "Черная стрела"
	\item * 1889 - "Владелец Баллантрэ"
	\item * 1889 - "Несусветный багаж"
	\item * 1893 - "Потерпевший кораблекрушение"
	\item * 1893 - "Катриона"
	\item * 1897 - "Сент-Ив"
\end{itemize}

