% vim: keymap=russian-jcukenwin
%%beginhead 
 
%%file 15_10_2020.fb.reklama.illienko_andrii
%%parent 15_10_2020
%%tags vo svoboda,andrii illienko,reklama
 
%%endhead 

\subsection{Андрій Іллієнко - Передвиборча декларація}
\label{sec:15_10_2020.fb.reklama.illienko_andrii}

Конференція Київської «Свободи» висунула мене на посаду Київського міського
голови та першим номером списку до міської ради.

Дякую друзям за довіру і хочу сказати кілька слів про свою мотивацію:

\begin{itemize}
	
\item 1. Київ --- моє рідне місто. Тут народився, виріс, живу і планую жити далі. Мені
боляче, коли Київ перетворюється на малопридатну для життя територію. Мені
боляче, що моїм рідним містом керує якась мутна шобла рішал без жодного
розуміння, що таке Київ і яка його місія. Яка йде навіть за межі України. 

\item 2. Я живу звичайним життям, ходжу по вулицям, їжджу на метро. В мене немає
маєтку за високим парканом і броньованого джипа. Ходжу по роздовбаному
асфальту, миюсь холодною водою влітку, бачу засмічені двори. Мені не треба
питати у соціологів, які у киян побутові проблеми --- бо це і мої проблеми також. 

\item 3. Я політолог за освітою і політик за покликанням, але дві каденції народного
депутата-мажоритарника від київського округу навчили розбиратися у міському
господарстві, хоч це і не були мої прямі повноваження. Знаю що таке бюджет,
податки, тарифи, міжквартальні проїзди, ліфти та дитячі майданчики. Якби по цих
питаннях усі кандидати проходили щось типу ЗНО --- точно обійшов би там багатьох
типу «господарників», у тому числі чинного мера Кличка. Впевнений у цьому)

\item 4. У мене немає жодних бізнес-інтересів. За мною не стоять мафіозні структури.
У мене жодних зобов’язань, окрім публічних зобовязань перед виборцями. Мій
інтерес --- успіх Києва. 

\item 5. Я за українську ідентичність в усіх проявах. Я за українську мову,
українську культуру і проти московської п’ятої колони. Київ заслуговує на
патріотичну владу. Дуже хочеться показати на цих виборах, що Київ не
зеленуватий, а жовто-блакитний.

\item 6. Я не сам. Моя команда --- це команда Київської «Свободи», з якою я пройшов
купу випробувань і в якій впевнений. Це не збіговисько випадкових людей, а
потужна досвідчена команда, за якою --- сотні успішних історій захисту інтересів
киян. Ці хлопці та дівчата мають бути у Київраді --- і я їм допоможу, чим зможу. 

\item 7. Я не йду заради спортивного інтересу. Буду боротися за перемогу.
Кожного дня проводжу зустрічі з людьми, ганяю по всьому місту і оберти
буду лише збільшувати.  Кияни, сподіваюсь на вашу підтримку)

\end{itemize}
