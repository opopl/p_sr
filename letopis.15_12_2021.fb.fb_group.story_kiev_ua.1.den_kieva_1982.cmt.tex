% vim: keymap=russian-jcukenwin
%%beginhead 
 
%%file 15_12_2021.fb.fb_group.story_kiev_ua.1.den_kieva_1982.cmt
%%parent 15_12_2021.fb.fb_group.story_kiev_ua.1.den_kieva_1982
 
%%url 
 
%%author_id 
%%date 
 
%%tags 
%%title 
 
%%endhead 
\zzSecCmt

\begin{itemize} % {
\iusr{Светлана Манилова}
\textbf{Петр}, напомни, пожалуйста. На следующий день, 31 мая было сочинение- первый экзамен? @igg{fbicon.smile} 

\begin{itemize} % {
\iusr{Петр Кузьменко}
\textbf{Светлана}, очень возможно. Мы с Нелли не помним. День Киева был ярким. Потом, точно, первым сочинение... Числа экзамена не помню.

\iusr{Любов Огородня}
\textbf{Светлана Манилова} так, щось писали @igg{fbicon.beaming.face.smiling.eyes} . Я це дуже гарно пам‘ятаю, тому що після феєрверків і салюту, у мене голосу не було і я рада була, що письмовий екзамен.  @igg{fbicon.beaming.face.smiling.eyes} 

\iusr{Лариса Беленчук}
\textbf{Светлана Манилова} Ах! Как же Вы память разворушили! Вспомнила даже тему сочинения, на которую писала-))))....

\begin{itemize} % {
\iusr{Светлана Манилова}
\textbf{Лариса}, и я об этом думаю. Что-то по Маяковскому, по-моему.

\iusr{Лариса Беленчук}
\textbf{Светлана Манилова} я писала на свободную. А тема: «И будет священной каждая пядь земли, политая кровью беззаветных героев». Хотела что-то прокомментировать ещё, но наверное не сейчас...

\iusr{Светлана Манилова}
\textbf{Лариса}, на свободную тему я писала при поступлении в университет.
\end{itemize} % }

\end{itemize} % }

\iusr{Vitali Andrievski}
Спасибо Петр. Точно частичка вашего настроения и того времени уже передалась мне и думаю многим подолянам.

\iusr{Елена Мельникова}
Не помню как было, но было...

\ifcmt
  ig https://i2.paste.pics/1c439a7e11bbe8809a406bcb8ab8b2d8.png
  @width 0.2
\fi

\iusr{Александра Пихур}
Прошло почти 40 лет. Где-то среди людей и я, фотографии передают праздничное настроение

\iusr{Тамара Мельничук}

це було свято, на підприємствах давали медалі до дня 1500 річчя Києву, яку
отримала і я. Вирували гуляння, ярмарки, морозиво! Було спокійно
,святково, урочисто.

\begin{itemize} % {
\iusr{Regina Kostetska}
\textbf{Tamara Melnichuk} Именно так и было;; Медаль, Удостоверение, подписанное самим В. Згурским! И премия в кассе бухгалтерии!

\iusr{Regina Kostetska}

\ifcmt
  ig https://scontent-frx5-2.xx.fbcdn.net/v/t39.1997-6/p480x480/105941685_953860581742966_1572841152382279834_n.png?_nc_cat=1&ccb=1-5&_nc_sid=0572db&_nc_ohc=Kq8EgqfccboAX8GywK8&_nc_ht=scontent-frx5-2.xx&oh=00_AT8Yg5oaecqkbnAWymVzdFe4OjX4-Gve_6nXaUl68BqHaA&oe=61C1E08B
  @width 0.2
\fi

\end{itemize} % }

\iusr{Ereena Shvarts}

И я где-то, наверняка, в этой толпе довольных и счастливых людей топаю. Жданова
6 мой второй дом родной! Бабушка пожила всю свою жизнь в этом доме и все
демонстрации, празднества, парады, все проходило мимо ее двора... приятные
воспоминания... спасибо  @igg{fbicon.hands.pray}  за пост и фотографии


\iusr{Elena Tsirulnik}
И мы где то там с мужем и маленькой доцей!

\iusr{Алла Тракслер}
Добре пам'ятаю цей день, це свято.

\iusr{Ірина Терещенко}

\ifcmt
  ig https://i2.paste.pics/77c8834dfd6d2c44821f76818f058714.png
  @width 0.2
\fi

\iusr{Татьяна Сирота}

А я в этот день спускалась по Андреевскому, где нескончаемым потоком в двух
направлениях (вверх и вниз) шли радостные и весёлые жители и гости нашего
Города.

Я в модельных туфельках на тоненьком каблучке, нарядная, молодая и красивая)))
по булыжной мостовой стойко прошла этот путь! Но... мои туфельки не выдержали
этого \enquote{похода}. Ушли из моей жизни навсегда, раньше времени)))

\begin{itemize} % {
\iusr{Ольга Патлашенко}
\textbf{Татьяна Сирота} Про туфельки, к сожалению, ничего сказать не могу...
А вот все остальное: что была молодой, нарядной, красивой, могу подтвердить.
И еще от себя добавить: помню Танюшу в те годы очень жизнерадостной, обаятельной, общительной и всегда веселой

\iusr{Татьяна Сирота}
\textbf{Ольга Патлашенко} Спасибо, Оленька!
Раньше мы были молодые и красивые. А теперь... только красивые)))

\iusr{Лидия Перкина}
\textbf{Татьяна Сирота} Аналогичная ситуация была со мной, спускалиь с детьми вниз,возвращаться нехотелось и так доковыляла до Красной площади,в дальнейшем всегда учитывала этот момент.

\iusr{Світлана Богданова}
\textbf{Татьяна Сирота} Да пам*ятаю таке, @igg{fbicon.face.smiling.eyes.smiling} 
\end{itemize} % }

\iusr{Pavel Prischepa}
Спасибо.

\iusr{Светлана Гаврилко}
И я помню этот день!! красиво, красочно!!!

\iusr{Maryna Bilous}
Да, настроение тогда было отличное!
Есть, что вспомнить  @igg{fbicon.face.smiling.eyes.smiling} 

\iusr{Roman Sytnyk}

А почалося все з приїзду Брежнєва на відкриття монументу \enquote{Родіни-матєрі} у
травні 1981-го. Кажуть, що Лєоніду Іллічу сподобався новостворений парк і він
Щербицькому пообіцяв кошти із всесоюзного бюджету для Києва, мовляв, \enquote{ви там
прідумайтє под что видєліть дєньгі}. І кияни придумали, що у наступному році
Києву - 1500 років.  @igg{fbicon.smile} 

\begin{itemize} % {
\iusr{Natalie Michael}
\textbf{Roman Sytnyk} Гроші отримали, вклали в місто, зробили парк на пейзажній алеї замість сараїв, відремонтували будинки на Володимирській. А не вкрали. Ото злодії!

\begin{itemize} % {
\iusr{Roman Sytnyk}
\textbf{Natalie Michael} Тоді ще толком і не вміли красти... Хіба, що хтось поцупив пару мішків цементу на дачу...

\iusr{Tatyana Mazerati}
\textbf{Roman Sytnyk} ETO fURZEVA,ZA CHTO ZAPLATILA DOROGO-SVOEY JIZNYU

\iusr{Roman Sytnyk}
\textbf{Tatyana Mazerati} Фурцева?

\iusr{Tatyana Mazerati}
\textbf{Roman Sytnyk} da, ee ubrali iz Politbyuro za stroitelstvo dachi, yakobi
na dengi gosudarstva. Esli vas interesuet ee sudba posmotrite prekrasniy film
Furzeva

\iusr{Roman Sytnyk}
\textbf{Tatyana Mazerati} Фурцевої не стало у 1974-му році, а святкування 1500-річчя Києва відбулося у 1982-му. Вони ніяк не пов'язані.
\end{itemize} % }

\iusr{Анатолий Золотушкин}
А давность 40 летняя)

\end{itemize} % }

\iusr{Dmitrii Pisanenko}
Я даже вижу свой дом (черные балконы)!

\iusr{Наталия Жминко Сычевска}

Здорово !

\iusr{Tatiana Goncharova}

Спасибо большое! О нашем Подоле и всегда так тепло, с настроением и любовью!
@igg{fbicon.heart.red} @igg{fbicon.clinking.glasses} 

\iusr{Margarita Kaminsky}

Спасибо за навеянные воспоминания.

\iusr{Victoria Novikov}

Мне было 17. Прекрасно помню этот день! Мы дошли по Красноармейской до
Крещатика, а оттуда до Андреевского. @igg{fbicon.heart.red}

\iusr{Alex Geyfman}
Я тоже в том году был десятиклассником - но этот день уже не помню...

\begin{itemize} % {
\iusr{Виктор Ниренберг}
\textbf{Alex Geyfman} так мы по одним улицам ходили! СШ 164. Чоколовка форевер!)) Я жил на искровской 1/7

\iusr{Alex Geyfman}
\textbf{Виктор Ниренберг} а я на Космонавтов 18. СШ №64.

\iusr{Виктор Ниренберг}
\textbf{Alex Geyfman} немного поправлю- площадь космонавтов была рядом, а школа находилась на ул. Донецкой

\iusr{Alex Geyfman}
\textbf{Виктор Ниренберг} ой, ошибся. Улица Культуры. Потом переименовали в Мартиросяна.

\iusr{Alex Geyfman}
\textbf{Виктор Ниренберг} на Донецкой разве не СШ 69 была?
\end{itemize} % }

\iusr{Светлана Манилова}
\textbf{Леонід Чекайда}, в связи с чем Ваше \enquote{возмутительно}?

\begin{itemize} % {
\iusr{Леонід Чекайда}
\textbf{Светлана Манилова} шо

\iusr{Светлана Манилова}
\textbf{Леонід}, была недовольная оценка этой публикации. Уже Вы ее убрали.

\iusr{Леонід Чекайда}
\textbf{Светлана Манилова} то случайно
\end{itemize} % }

\iusr{Виктор Ниренберг}

Здорово! Благодарю за память! А я в этот день праздновал свои 18 лет на нашей
даче в Осокорках на 48 садовой... Жарили шашлык, пили пиво... Фото не осталось,
увы..

\iusr{Валентина Валентина}

\ifcmt
  ig https://scontent-frx5-2.xx.fbcdn.net/v/t39.1997-6/s168x128/47270791_937342239796388_4222599360510164992_n.png?_nc_cat=1&ccb=1-5&_nc_sid=ac3552&_nc_ohc=hSdFnkRMAt8AX-KHr1I&_nc_ht=scontent-frx5-2.xx&oh=00_AT9Hi990n2fJpOowUyslNnf9EHL_TPwuQMYi7MTmRBAKxQ&oe=61C105F6
  @width 0.1
\fi

\iusr{Alex Jäger}

В тот день младшая школа маршировала в пилотках и погонах, жара миллион людей и
парад детей юных военных - мне не зашло. Но день этот я помню, Федорова уж
постаралась

\begin{itemize} % {
\iusr{Елена Ивановна}
\textbf{Alex Jäger} Сколько было радости, хлопот в подготовке, счастья у детей маршировать на парадах! А вы все таки стараетесь плевок свой оставить! Порадуйте сегодня чем то детей, кроме майданов и ходов...

\iusr{Alex Jäger}
\textbf{Елена Ивановна} радости? На солнцепеке три часа торчали, я потому этот день и запомнил.
\end{itemize} % }

\iusr{Виктория Крижановская}
Добре памятаю цей день. Дякую за публікацію.

\iusr{Анна Сидоренко}
Гулянье, это хорошо, сейчас там тоже много людей, когда праздник, а ещё мне нравится как задорно играют уличные музыканты.

\iusr{Таня Гур}
Спасибо и я там где то гуляла, мы так ждали этот день!@igg{fbicon.heart.red}{repeat=3}

\iusr{Аркадий Лопухов}

Праздновал , мне было 35 лет .

\iusr{Marta Smekhova}
20-ти летней давности?
хотя, и мне кажется, что 90-е десять лет назад были... @igg{fbicon.face.smiling.eyes.smiling} 

\iusr{Lina Gorodetsky}

Вау, как я помню этот день, и праздник в городе. А за улицу Жданова отдельное
спасибо, Петр! Кажется, что вижу за ближайшим поворотом налево свой дом...

\iusr{Ludmila Bureva}

А я была тогда студенткой 1-го курса театрального института и мы участвовали
всем институтом в грандиозном праздновании на республиканском стадионе... Какое
счастливое было время! Такая щемящая ностальгия!

\begin{itemize} % {
\iusr{Ludmila Bureva}

И Подол - это мой родной и любимый район! Мы там жили, там же и поженились и
там же родился первый сын! Столько прекрасных воспоминаний, связанных с
Подолом! @igg{fbicon.heart.red}{repeat=3}

\end{itemize} % }

\iusr{Георгий Майоренко}
Золотые времена!

\iusr{Елена Чучулашвили}

Я тоже где-то в этой толпе! и тоже-10 класс, впереди-экзамены и вся жизнь,
миллион интересных поездок, встреч, событий, которые, в итоге, и привели меня в
другую страну, в которой живу уже много лет.


\iusr{Елена Полишко}
Помню, мне тогда было 30лет... Хорошие были времена!

\iusr{Люда Невзгляд}

Фото радостное Приятно смотреть на людей. До взрыва в Чернобыле в Киеве все было
прекрасно, всегда, а в праздники особенно. Сппсибо за фото.


\iusr{Любовь Чернявская}
Да, Подол тогда преобразился, расцвел. Приятные воспоминания...

\iusr{Vicki Seplarsky}
Да, все ещё живы... и до Чернобыля осталось 4 года...

\iusr{Helen Ferrum}
40 лет прошло .
Не вериться  @igg{fbicon.face.sad.but.relieved} 

\iusr{Michael Fishman}

Когда готовились к 1500-летию и красили Подол, я вышел из арки двора на
Жданова, где находилась наша ЭВМ и пошел к Почтовой. Передо мной шла пара,
мужик повернулся к жене и сказал: \enquote{Когда они вернутся, они не узнают свой
Подол. Он будет красивее, чем их Нью-Йорк}

\iusr{Валентина Габдрахимова}
\textbf{Michael Fishman} это было ну очень интересно, !!!!

\iusr{Татьяна Приймак}
Как мы ждали такие дни. Жданова, трамвайчик ходил магазинов сколько было. Здорово было!

\iusr{Наталя Задорожна}

Сколько людей и, практически, больше ничего! Никаких аттракционов, еды,
товаров! Сейчас пытаюсь вспомнить, а чем занимались люди на таких праздниках?

\begin{itemize} % {
\iusr{Roman Sytnyk}
\textbf{Наталя Задорожна} Картини і різні мистецькі вироби купували на Андріївському узвозі.

\iusr{Тамара Ар}
\textbf{Наталя Задорожна} Зато, сейчас, аттракционы, кавы, еда уличная, а спокойствия даже у денежных тузов нет!

\iusr{Тамара Ар}
\textbf{Наталя Задорожна} Театры, кинотеатры, концерты, музеи, походы, поездки в Таллин, Питер, Вильнюс, Москву, Ригу, Тбилиси, Ереван, Ашхабад, Северный Кавказ и так далее

\begin{itemize} % {
\iusr{Наталя Задорожна}
\textbf{Toma Ar} секунду, я только о данном празднике спрашивала, а не о всей жизни тогда

\iusr{Тамара Ар}
\textbf{Наталя Задорожна} да ладно вам!

\iusr{Наталя Задорожна}
\textbf{Toma Ar} не поняла?) Если Вы о том, как и куда мы тогда ездили, то я помню, что по родной стране) я пытаюсь вспомнить, что делали на Днях города. Тогда не было так много ресторанов и кафе с летними площадками, уличной еды ( перепичка и пирожки не в счет) не было. Кофе в чашках. Ну, вот смотрю на людей на фото - ничего нет на площади. Только много людей.

\iusr{Тамара Ар}
\textbf{Наталя Задорожна} здоровее были! Сладкое и много кофе вредно для здоровья!,,,,,, Извините, продолжать дисскутировать не делаю, все сказано выше,

\iusr{Тамара Ар}
\textbf{Наталя Задорожна} не желаю
\end{itemize} % }

\end{itemize} % }

\iusr{София Бойко}

\ifcmt
  ig https://i2.paste.pics/54cfa443777593f61b5c98fc280764a9.png
  @width 0.2
\fi

\iusr{Елена Фросталь}

Я с подругой и ее дочкой - подростком была на Андреевском в этот день, народу
было по обе стороны немеряно, картины, музыка , все было интересно. Потом
ходили каждый год, но этот первый не забылся


\iusr{Светлана Кравченко}
Я добре пам'ятаю, менi було тодi 20 рокiв.

\iusr{Olena Rybalchenko}

Город счастья. Мы гуляем и радуемся своей беззаботной жизни. Выпуск со школы только через год, в 1983.

\iusr{Павел Пауль}
Где-то там и мы гуляем. Все мы там виделись.

\iusr{Нина Харченко}
Где-то там и мы))) 10 класс 214 школы

\iusr{Тамара Ар}

Тогда, было классное настроение! И не только по молодости! Мы все чувствовали
стабильность в наших семьях, в стране!

\iusr{Galya Lebedyeva}

Очень хорошо помню тот день! Многие были в вышиванках, очень многие! На фото
этого не видно! Я была в компании своих сокурсников студентов КПИ! У нас была
гитара, мы пели песни, гуляли на Майдане! Но запомнилось, что возле нас был
милиционер, который слушал, что мы поем и в 11 вечера сказал нам, что последняя
песня и расходитесь!!!! Сейчас такое трудно представить!!! Это страшная
действительность тех времён!!!

\iusr{Сергей Мироненко}
\textbf{Galya Lebedyeva} Нет,времён нынешних !

\iusr{Тома Мангушева}

Прослухала зустріч Гордона і Богдана. Не знаю чи ми всміхнемось хоть ще раз!!!
Страшно за країну! А там була стабільність, повага до людини- щасття

\begin{itemize} % {
\iusr{Веле Штылвелд}
\textbf{Тома Мангушева} щастя...

\iusr{Тома Мангушева}
\textbf{Віктор Шкідченко} помилково, але це не так важливо
\end{itemize} % }

\iusr{Сергей Мироненко}
Я не Подолянин, но я с Вами!

\iusr{Arutyun Avetisyan}
Хорошо помню этот праздничный день!

\iusr{Валентина Аулова}
Спасибо за фото! Один из прекрасных дней моей молодости на любимом Подоле!

\iusr{Dima Kievsky}

Как писал Городницкий: \enquote{... обратно в реку прошлого войти нельзя... да это и не
надо...} Но как хочется вернуться туда, где был счастлив! Юность? Да. Вся жизнь
впереди? Да. Все ещё живы? Да. И наверное что то ещё, что не измеряется
количеством денег или сортов колбасы.

\iusr{Анна Сафронова}
\textbf{Dima Kievsky} Абсолютно верно!!

\iusr{Татьяна Зубко Маркина}

\ifcmt
  ig https://i2.paste.pics/d168dcc4a46407f9bcc9bb1653e17ef8.png
  @width 0.2
\fi

\end{itemize} % }
