% vim: keymap=russian-jcukenwin
%%beginhead 
 
%%file 30_09_2021.fb.jermolenko_vladimir.1.raznoobrazie_imperia
%%parent 30_09_2021
 
%%url https://www.facebook.com/volodymyr.yermolenko/posts/10158430813103358
 
%%author_id jermolenko_vladimir
%%date 
 
%%tags identichnost',imperia,nacia,nacidea,obschestvo,raznoobrazie,ukraina,vopros
%%title Як зберігати розмаїття, і при цьому не бути імперією?
 
%%endhead 
 
\subsection{Як зберігати розмаїття, і при цьому не бути імперією?}
\label{sec:30_09_2021.fb.jermolenko_vladimir.1.raznoobrazie_imperia}
 
\Purl{https://www.facebook.com/volodymyr.yermolenko/posts/10158430813103358}
\ifcmt
 author_begin
   author_id jermolenko_vladimir
 author_end
\fi

особливість, і навіть унікальність українського політичного і культурного
досвіду порівняно з іншими країнами Європи для мене втілюється в питанні: як
зберігати розмаїття, і при цьому не бути імперією? і інакше: як бути nation
state і при цьому зберігати множинність?

ми забуваємо часом, наскільки дискурс розмаїття та багатокультурності був
породженням імперського спадку. Як він утвердився у ХХ столітті передусім в
імперіях, що спробували розкаятися в своїй імперськості (з різним успіхом), - і
передусім в грандіозній і напруженій зустрічі між колишніми колонізаторами і
колишніми колонізованими. Більшість сьогоднішніх проблем західної Європи, від
міграції до кризи мультикультуралізму, - звідти. 

Центральна та Східна Європа були у ХІХ-ХХ століттях грандіозним опонентом
імперській ідеї. Ми теж часом забуваємо, наскільки антиімперський принцип
"самовизначення націй" завдячує собі національним процесам у Центральній та
Східній Європі ХІХ-ХХ століть, і наскільки сучасна об'єднана Європа націй
завдячує своїм існуванням цьому принципові.

але повстання національного проти імперського теж не є завжди аж таким
позитивним - у радикальних проявах воно може вести до ксенофобії і ненависті до
іншого. І цікаво, що головний для пізнього ХХ ст. міф Центральної Європи, який
створив Мілан Кундера - "максимум розмаїття на мінімумі простору" - є радше
імперським Австро-Угорським міфом, ніж міфом національним. Саме тому він
сьогодні в Центральній Європі терпить поразку - і саме тому тут можливі Орбан і
Качинський, які на самій тільки ксенофобській риториці можуть роками утримувати
владу

український досвід тут гіперцікавий - і, можливо, унікальний, але ще не
осмислений: бо це досвід природного розмаїття в неімперському суспільстві. У
нас справді є шанс реалізувати "максимум розмаїття на мінімумі простору". Поява
в останні роки вагомого кримськотатарського мусульманського голосу і досвіду,
все цікавіший діалог між українцями та євреями, розмова між різними
християнськими конфесіями, між релігійними і світськими дискурсами - тут дуже й
дуже важлива. Особливо на тлі нового імперіалізму з півночі, який уже ніякого
розмаїття не несе.  

тому ось вони, важливі українські екзистенційні питання: як зберігати
розмаїття, і при цьому не бути імперією? і інакше: як бути національною
державою і при цьому зберігати множинність?

\ii{30_09_2021.fb.jermolenko_vladimir.1.raznoobrazie_imperia.cmt}
