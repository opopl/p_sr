% vim: keymap=russian-jcukenwin
%%beginhead 
 
%%file 17_12_2022.stz.news.ua.donbas24.1.kyiv_zaxid_pidtrym_polonena_azovstal.txt
%%parent 17_12_2022.stz.news.ua.donbas24.1.kyiv_zaxid_pidtrym_polonena_azovstal
 
%%url 
 
%%author_id 
%%date 
 
%%tags 
%%title 
 
%%endhead 

У Києві відбувся захід на підтримку військовополоненої з Азовсталі (ВІДЕО)

17 грудня у Києві провели літературні читання віршів маріупольської поетеси Валерії Суботіної
(Карпиленко), яка наразі перебуває у полоні

17 грудня у Будинку кіно в Києві відбулися літературні читання віршів
маріупольської поетеси та письменниці, доброволиці полку «Азов», української
військовополоненої. Літературні читання «Ковток повітря для книги Валерії
Суботіної (Карпиленко) "Квіти і зброя": читаємо з тобою» організували рідні і
друзі Валерії. Захід присвятили перевиданню її поетичної збірки, яку
письменниця у 2021 році презентувала в Маріуполі та Києві та підтримці мужньої
маріупольчанки.

Читайте також: У Києві з'явилася вулиця Героїв полку «Азов»

Історія Валерії

Валерія Суботіна (Карпиленко) на позивний Nava — маріупольська поетеса,
кандидатка наук із соціальних комунікацій, у минулому випускниця й викладачка
Маріупольського державного університету, яка разом з «Азовом» захищала рідний
Маріуполь. Історію кохання Валерії Карпиленко та прикордонника Андрія Суботіна
(Бороди) записав побратим, азовець Дмитро Козацький (Орест). Пара
познайомилась, коли Валерія працювала в пресслужбі Донецького прикордонного
загону. В облозі на «Азовсталі» вирішили одружитись. Датою обрали 5 травня, бо
це день народження полку «Азов». Проте 7 травня Андрій загинув. Валерія вже 7
місяців перебуває в полоні.

«Ти три дні був моїм законним чоловіком. І цілу вічність ти — моє кохання. Мій
рідний, мій турботливий, мій мужній... Ти був і є найкращий. Мені лишилося твоє
прізвище, твоя любляча родина і спогади про щасливий час разом», — написала
Валерія у травні. 

Читайте також: Із російськогo полону повернулися 64 українця та один громадянин США: список звільнених (ФОТО)

Захід, присвячений Валерії

Напередодні Дня Святого Миколая, як уявний подарунок і сердечна підтримка для
Валерії, в Будинку кіно відбулися літературні читання «Ковток повітря для книги
Валерії Суботіної (Карпиленко) "Квіти і зброя": читаємо з тобою». Мати
загиблого чоловіка Валерії, Людмила Суботіна, та її друзі організували
невеликий, камерний, але дуже душевний творчий захід, присвячений перевиданню
її поетичної збірки, яку письменниця в 2021 році презентувала в Маріуполі й
Києві. На жаль, у багатьох маріупольців-переселенців книжки першого видання
збірки з підписами авторки були втрачені в зруйнованих будинках. Ведучим заходу
був актор та режисер Станіслав Сукненко. 

Серед запрошених гостей заходу кобзар, лірник, бандурист, народний артист
України, лауреат Національного конкурсу кобзарського мистецтва ім. Г. Китастого
Тарас Компаніченко, який прочитав вірші Валерії та заспівав душевні пісні на
підтримку маріупольчанки та її побратимів.

«У цьому горнилі зараз формується справжня еліта. Ці люди стали в оборону
національної ідентичності. Кожен з них це потужний творець. Всі вони
реалізовані люди. В обороні стоїть справжня національна еліта, яка ніколи не
буде мімікрувати, зраджувати та кудись перебігати. Бо вони інакшими бути не
можуть. Саме такій еліті володарювати на цій землі», — підкреслив Тарас
Компаніченко.

Читайте також: Росіяни планують судилище над захисниками Азовсталі

Вірші Валерії читали громадські активісти Маріуполя, представники міської
влади, журналісти, колеги і друзі Валерії.

«Я знаю Леру близько 20 років. Спочатку, як студентку, потім, як колегу, адже
ми разом викладали у Маріупольськомуу державному уніврситеті. Ми завжди
підтримували стосунки, тому що ми дуже споріднені. Вона говорила, що я для неї
взірець, але насправді це вона для мене справжній взірець для наслідування», —
наголосила Вікторія Рогожина.

«Знаючи Леру, можу точно сказати, що вона навіть в найгіршій ситуації може
зберігати оптимізм та віру. Головне, щоб вона повернулася з полону не тільки
живою, а щоб вона залишилася собою. Змогла зберегти свою здатність — бачити
навколо добро, яка б біда не відбувалася», — зауважив заступник міського голови
Денис Кочубей.

В окремому відео можна було почути вірш Валерії у виконанні відомого
українського актора Дмитра Лінартовича.

Читайте також: «Ми розуміли, що шляху назад немає» — прикордонник про бої за
Маріуполь, поранення та полон (ВІДЕО)

Також представили відеозвернення Сергія Жадана, який теж зачитав один з віршів
маріупольчанки. З відеозверненням виступив і Дмитро Козацький (Орест). Чоловік
поділився спогадами про службу з Валерією, наголосив на її неповторності та
виключному таланті і висловив сподівання, що вже незабаром Nava буде вільною.
Ведучий Станіслав Сукненко прочитав зворушливого листа доктора політичних наук,
професора, радника ректора Маріупольського державного університету Костянтина
Балабанова до Валерії Карпиленко. 

Про нову збірку віршів

Зі слів самої Валерії, збірка «Квіти і зброя» написана нею під час служби в
добробаті, в полку «Азов» (2015−2016 рр.), і пізніше, з 2017 по 2020 р., в
Донецькому прикордонному загоні. Весь цей час Валерія записувала історії бійців
і писала вірші про них. Кожний вірш — це спогад про хлопців і дівчат, які були
там, на Сході, всі ці 8 років російсько-української війни, і для яких є
настільки дорогою і небайдужою Україна, що вони готові віддати за неї
найдорожче — свої життя.

Читайте також: «На їжу давали 30 секунд» — захисник Маріуполя Михайло Діанов розповів про знущання в полоні

Збірка складається з чотирьох розділів: «Пріоритети», «Вона», «Замовляння» та
«Душа». Кожний із них має свою атмосферу і свою історію написання. «Пріоритети»
— це про власний вибір захищати країну на фронті та спогади про побратимів, в
основі кожного вірша реальні історії бійців. Розділ «Вона» — душа ліричної
героїні, її бунтівна натура, сумніви й переконання, розчарування і любов.
«Замовляння» не тільки за назвою й тематикою, а й за художнім ритмом та образам
— абсолютно фольклорно-відьомське поетичне «зілля». Останній розділ поетичної
збірки прочиняє для читачів віконце у світ душі ліричної героїні, відкриває її
пантеїстичний світогляд.

Раніше Донбас24 розповідав, що Гурт Tvorchi присвятив пісню захисникам
«Азовсталі».

Ще більше новин та найактуальніша інформація про Донецьку та Луганську області
в нашому телеграм-каналі Донбас24.

ФОТО: з відкритих джерел та з особистого архіву Ольги Демідко
