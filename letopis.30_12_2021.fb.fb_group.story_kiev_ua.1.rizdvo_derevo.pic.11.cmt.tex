% vim: keymap=russian-jcukenwin
%%beginhead 
 
%%file 30_12_2021.fb.fb_group.story_kiev_ua.1.rizdvo_derevo.pic.11.cmt
%%parent 30_12_2021.fb.fb_group.story_kiev_ua.1.rizdvo_derevo
 
%%url 
 
%%author_id 
%%date 
 
%%tags 
%%title 
 
%%endhead 

\iusr{Maksym Oleynikov}

В лютому 1938р. Постишев, переведений на той час з України на пост 1-го
секретаря Куйбишевського обкому і міськкому партії, був арештований, а ще через
рік – розстріляний, як «японський шпигун» і «правий троцькіст».

Моя мама згадувала, як в школі вони співали пісню на слова М.Рильського:

«...Чи бачите ви посмішки і сяєво очей?
Павло Петрович Постишев сидить серед дітей».

Та потім їх змусили вирвати з підручників сторінку з портретом Постишева, а
якось на перерві приїхала машина і співробітники НКВС забрали вчителя, який цю
пісню розучував з учнями...

\iusr{Antoniy Malynovskiy}
\textbf{Maksym Oleynikov} працювали товариші віддано та чесно з усіх боків....
