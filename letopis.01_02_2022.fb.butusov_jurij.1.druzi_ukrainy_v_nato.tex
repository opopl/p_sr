% vim: keymap=russian-jcukenwin
%%beginhead 
 
%%file 01_02_2022.fb.butusov_jurij.1.druzi_ukrainy_v_nato
%%parent 01_02_2022
 
%%url https://www.facebook.com/butusov.yuriy/posts/7178651765508391
 
%%author_id butusov_jurij
%%date 
 
%%tags napadenie,nato,rossia,ugroza,ukraina
%%title До друзів України в НАТО
 
%%endhead 
 
\subsection{До друзів України в НАТО}
\label{sec:01_02_2022.fb.butusov_jurij.1.druzi_ukrainy_v_nato}
 
\Purl{https://www.facebook.com/butusov.yuriy/posts/7178651765508391}
\ifcmt
 author_begin
   author_id butusov_jurij
 author_end
\fi

До друзів України в НАТО

Українське суспільство наразі розколото ставленням до політичної та військової
кампанії країн НАТО проти російської загрози. 

Керівництво України переконує, що заяви про загрозу російського вторгнення з
країн Заходу ніби шкодять нашій економіці та створюють паніку у суспільстві,
інвестори через це тікають. Наша влада робить заяви, ніби-то США, Велика
Британія та інші - провокатори, і розгортання російських військ не становить
загрози на яку варто активно реагувати. 

Але це не є думкою більшості українського суспільства.

І багатьом громадянам України боляче  бачити закиди у провокаціях та
"внутрішніх політичних підставах" на адресу НАТО, під час цієї безпрецедентної
дипломатичної битви з російською загрозою.

Ми розуміємо, що  політична кампанія у країнах НАТО - це не змагання
екстрасенсів за передбачення точної дати наступу. Це намагання відхилити
дамоклів меч, яким Путін шантажує Україні 8 років.

Ми розуміємо, що кожного дня російські війська на українській землі намагаються
вбити громадян України. Саме тому Росія - єдиний в цій війні провокатор,
головний ризик для інвесторів,  основне джерело паніки для суспільства.  

Ми розуміємо, що  наші друзі  у НАТО намагаються створити единий політичний
антипутінський фронт, солідарну позицію. Наші союзники в НАТО підштовхують
європейців визначитись, надати матеріальну підтримку Україні, і саме для цього
потрібно говорити про загрозу з РФ максимально голосно. 

Ми розуміємо, що США, Велика Британія, Литва, Латвія, Естонія, Польща, Чехія
дарують нам зброю для того, щоб підсилити нашу безпеку, і змусити Росію
замислитись, що постачання новітньої зброї від НАТО буде ще більшим під час
війни, і українська армія завдасть важких втрат росіянам. Зброя  - це найкраща
та найпотужніша підтримка України, і це безпрецедентні масштаби допомоги. 

Ми зробимо з часом Україну потужною державою, яка так само без вагань прийде на
допомогу своїм друзям. Можливо, далеко не всі чують у світі наш голос, але
навіть якщо це побачить хтось один, нам важливо сказати:  Ми вдячні. Ми
розуміємо.
