% vim: keymap=russian-jcukenwin
%%beginhead 
 
%%file 18_10_2021.yz.nealternativnaja_istoria.1.kiev_mti_gradom_ruskim.cmt
%%parent 18_10_2021.yz.nealternativnaja_istoria.1.kiev_mti_gradom_ruskim
 
%%url 
 
%%author_id 
%%date 
 
%%tags 
%%title 
 
%%endhead 
\subsubsection{Коментарі}
\label{sec:18_10_2021.yz.nealternativnaja_istoria.1.kiev_mti_gradom_ruskim.cmt}

\begin{itemize} % {
\iusr{Elena Kornilova}

вообще-то эта \enquote{мать городов} - калька с греческого \enquote{метрополия}, т.е.
\enquote{материнский, главный, город}. Так что род тут вообще без разницы.

\begin{itemize} % {
\iusr{VLADIMIR B}
\textbf{Elena}, во всех славянских языках "мати" (позже "матица") означало "основа", "главный".

\iusr{сергей костин}
\textbf{Elena}, Да убили правителей стали править Мать совершенно не причем Греческого не знали в то время

\iusr{Elena Kornilova}
\textbf{сергей костин}, монахи не знали? да Вы что! Вот новость....

\iusr{paul bratanoff}
\textbf{сергей костин}, сомнительно, что Нестор не знал греческого, а слова эти с его лёгкой руки в летописи появились.

\iusr{сергей костин}
\textbf{Elena}, Столица спо гречески метрополис Главный город Мать приплели позднее

\iusr{сергей костин}
\textbf{paul}, Знали метрополис это главный город

\iusr{сергей костин}
\textbf{paul}, Знал и не допустил бы что бы метрополис превратился в мать город

\iusr{сергей костин}
\textbf{paul}, Вполне возможно переврал Нестор слова Олега

\iusr{Bладимир Ыванов}
\textbf{VLADIMIR B}, про материнскую плату вспомните во славянских языцех, она же - материнка...

\iusr{Bладимир Ыванов}
\textbf{сергей костин}, а по какому поводу позднее приплели мать? Интересно же!

\iusr{altaikukui}
\textbf{сергей костин}, можеть ктонибудь киев почему не переводиться на русский язык также москва . россия называние получил 1721 году.мир всем.
\end{itemize} % }

\iusr{сергей костин}

Город как женское начало не упоминается нигде и никогда Отцом отчим домом но не
мать Автор уперт выдумывает на ходу Почему Москву никогда не упоминают матерью
государства русского А вот иметь Имать город русские будут а не хазары это
точно

\begin{itemize} % {
\iusr{Rod Jarrow}
\textbf{сергей костин}, 

В словах Олега речь о материнской опеке... К слову: Город Он, Крепость
Она... Дело в том что летописный город это буквально современное понятие
Крепости, укрепленное сооружение с дружиной внутри. Летописные дела Владимира
(Крестителя) по созданию линии крепостей по рекам на границе Руси со степью X
века:

...и реч̑ Володимеръ . се не добро єсть мало городовъ ѡколо Кыєва . и нача
ставити городы по Деснѣ . и по Оустрьи . по Трубешеви и по Сулѣ . и по Стугнѣ .
и нача нарубати мужи лутши ѿ Словенъ . и ѿ Кривичъ А . и ѿ Чюдии . и ѿ Вѧтичь .
и ѿ сихъ насели и грады бѣ бо рать ѿ Печенѣгъ и бѣ воюӕсѧ с ними и ѡдолѧӕ имъ.

Это важный исторический документ о том, кого первым Владимир начал селить на
территории современной восточной Украины: Словенъ (Новгородцы). и ѿ Кривичъ
(Полочане) . и ѿ Чюдии (угро-финские народы). и ѿ Вѧтичь (корневое славянское
население современной Москвы и московской области)

\iusr{Сергей}
\textbf{сергей костин},

\enquote{Город как женское начало не упоминается нигде и никогда} - едешь в Прагу. Там
на старом гербе города читаешь \enquote{Praha matka mest}. Потом бьёшся головой об
стену повторяя \enquote{какой же я дурак}

\iusr{Фёдор Крюкин}
\textbf{Сергей},  @igg{fbicon.thumb.up.yellow}  @igg{fbicon.face.grinning.squinting}  @igg{fbicon.face.tears.of.joy}{repeat=3} 

\iusr{сергей костин}
\textbf{Сергей}, Прага это по русски порог. Порог родного дома По чешски женского рода наверное

\iusr{Сергей}
\textbf{сергей костин}, порог по-чешски prah. А Прага - Praha
А ваши личные фантазии рассказывайте самому себе на кухне

\iusr{сергей костин}
\textbf{Сергей}, Название деревни закрепилось В чешском БРНО Братислава

\iusr{Сергей}
\textbf{сергей костин}, когда прийдёте в себя посмотрите на карту. Брно и Братислава РАЗНЫЕ города

\iusr{сергей костин}
\textbf{Сергей}, Поэтому я написал БРНО среднего рода город типа село Братислава женского рода как Прага

\iusr{Кот Сибирский}
\textbf{сергей костин}, это тупо дословный перевод греческого (византийского) слова метрополис.

\iusr{сергей костин}
\textbf{Кот Сибирский}, Город метрополис Не мать

\iusr{сергей костин}
\textbf{Rod Jarrow}, Матерински поимел Киев Олег

\iusr{Кот Сибирский}
\textbf{сергей костин}, метрополис знаете, как переводится дословно?

\iusr{Rod Jarrow}
\textbf{Alexandr}, Опека, Забота понятия женского рода - об этом речь Олега.

\iusr{Rod Jarrow}
\textbf{сергей костин}, 

Да так поимел, что со всей округи дань собирал кроме полян... Увы, кроме дани с
полян Хазарам в летописи нет ни слова о дани с полян Олегу, в отличии от всей
округи! Более того с Византии Олег собирает дань на Киев и на другие русские
города...

\iusr{Светлана Инусова}
\textbf{сергей костин}, 

В поздних летописях: Москва есть мать городовъ русьскыхъ.

\end{itemize} % }

\iusr{Павел Рубец}

Нужно учитывать контекст, и в зависимости от него и переводить. \enquote{Мти} может быть
как \enquote{мать} и может быть как \enquote{иметь}. Город с мужским названием называть матерью
выглядит как-то натянуто. Ростов мама, Одесса папа. Так что-ли?

А ещё нужно знать использовался ли такой оборот речи, как \enquote{мать городов} в те
времена.

\begin{itemize} % {
\iusr{Rod Jarrow}
\textbf{Павел Рубец}, 

Увы, как не крути, не может быть, что иное...! Автор русским языком по тексту
летописи примерами пояснил, что во всех случаях, где в летописи появлялось
слово "мти", это была только мать и ни что иное... 2) Автор статьи также
пояснил что это не Мать городОВ, а Мать городАМ Русским, речь о материнской
ОПЕКЕ, а не детоРОЖДЕНИЮ...! Разница понятна?

\iusr{Павел Рубец}
\textbf{Rod Jarrow}, [речь о материнской ОПЕКЕ,]Города,у которого мужское название?

\iusr{Rod Jarrow}
\textbf{Павел Рубец}, 

Тоже самое мужское название ОТЕЦ в женском роде ОТЧИЗНА - вас не смущает? Тоже
самое мужское РОД и производное от этого мужского РОДа - женское РОДИНА! - Как
же так, мужской РОД переходит в женский РОД ? Что бы все это понимать, надо
просто быть русским человеком, для котрого издревле Материнская ОПЕКА над
детьми, по-умолчанию женского рода, а не мужского. (При том, что существует и
отеческая забота, но речь тут о более высшей форме заботы, речь про ОПЕКУ,
которая только женского рода. И, к слову: Одесса тоже город, и, тем не менее
\enquote{мама} у местных! И, еще к слову: Княжий стол - мужского рода, а вот уже любая
Столица - женского рода.)

\iusr{Павел Рубец}
\textbf{Rod Jarrow}, 

думаю уместным было бы (после убийства Аскольда и завладение городом Олегом)
Сказать \enquote{это будем иметь градом русским.} То есть он просто объявил
Киев градом русским.

И причем здесь вообще опека? Судя по описанию Киев был небольшой городок возле
\enquote{перевоза}. То есть переправы через Днепр. По сути жители Киева
оказывали услугу по переправе с одного берега на другой. О какой опеке речь?

\iusr{Rod Jarrow}
\textbf{Павел Рубец}, 

Ну и какая может быть цель в выдумывании того, чего нет в летописи? Выдумывая
некую \enquote{уместность} удобную современной евро-политике по распиливанию Руси на
Русь и не Русь? Понятно о чем речь в современных попытках переписать летопись?
Увы, Анкара тоже небольшой городок, зачуханная деревня в сравнении с
мегаполисом Стамбул, который по населению больше чем Москва и тем не менее
столица турции Анкара! Тоже самое США в сравнении Вашингтона и Нью-Йорка!
Столица не обязательно должна быть космических размеров, тому пример Ватикан -
совсем малюсенький и тем не менее столица всего католического мира! Тоже самое
и Киев для Олега (как Ватикан для католического мира). Я напомню вам летопись,
что титул князя Русского Олег получил именно в Киеве, а не в Новгороде! За
титулом Олег шел в Киев! Иначе ни как НЕ получить титул владыки народа и земли,
кроме как в столице народа и земли

\iusr{Павел Рубец}
\textbf{Rod Jarrow}, ну вас и понесло

\iusr{Павел Рубец}
\textbf{Rod Jarrow}, 

открываем церковнославянский словарь. Находим, помимо матери ниже мати-маю-имати. Имати нужно переводить?
Так вот повторюсь. Нужно переводить, в зависимости от контекста.

\ifcmt
  ig https://avatars.mds.yandex.net/get-zen_pictures/4012878/315195717-1634669090063/orig
  @width 0.4
\fi

\iusr{Rod Jarrow}
\textbf{Павел Рубец}, 

Алло? Где на вашей страничке "Мти"? Или, вы не поняли о чем речь? Повторяю: в
летописи, везде, где встречалось Мти - значение было только Мать иного значения
у летописного слова "Мти" не было нигде в летописи. Увы, тут, как не крути, но
летописное Мти" может быть только "матерью" и НЕ может быть как "иметь"! Тут
как не крути, не получится переписать летопись под новый (хорошо забытый старый
евро-план) по распилу Руси на Русь и НЕ Русь..! Вы, слишком сложно пытаетесь
заморочить голову и себе и людям! Все гораздо проще, увы, в словах Олега речь
буквально о Столице, которую издревле в традициях многих народов сравнивались с
матерью..

\iusr{Павел Рубец}
\textbf{Rod Jarrow}, 

вы наверно не в теме, что летопи писались с сокращением гласных? Но мы же оба
знаем, что Мти это МАТИ. Или вы будете это отрицать?. Если бы вы читали
внимательно, там есть и понятие МАТИ градовом. Так вот это означает мать
городам. Но в летописи написано МАТИ градом, что переводится как иметь городом.

\iusr{Rod Jarrow}
\textbf{Павел Рубец}, 

Вы опять лукавите! Потому что, если вы пытаетесь апеллировать к украинскому
языку (научно- "западно-русский письменный", который коммунисты переименовали в
украинский), так и тут, вы, не в теме, также, как и в языке летописи! Потому
что согласно украинской морфологии так просто невозможно сложить фразу: "Отже
буде мати (иметь) городом Руським" - Это чушь, белиберда, набор абстракций с
гугл Переводчика, а не украинский язык! В украинском так просто невозможно
сложить фразу! При том факте, что большинство украинцев не украино-язычны, тем
не менее (Sorry за тавтологию)"Украино-язычный украинец" скажет только так: "Це
мiсто МАЕ бути Руським" - Увидели это слово "МАЕ"..? Увы, в вашем варианте (с
намеком на украинский) может быть только это слово "МАЕ" если следовать
украинской морфологии. Аналогичный пример украинской морфологии: "Я МАЮ вам
щось сказаты" ДОСЛОВНЫЙ перевод: Я, ИМЕЮ вам, что-то сказать. А, теперь
смысловой перевод согласно Морфологии русского языка Я ХОЧУ вам, что-то
сказать... Поняли разницу, при том факте что слово ХОЧУ тоже есть в словаре
украинского языка но Морфология, увы иная! - Так, к чему последний пример? А, к
тому: ОПАСАЙТЕСЬ гугл Переводчика МАТИ значит ИМЕТЬ! Увы, НЕ ЗНАЧИТ! Потому что
перевод должен быть смысловым, а не дословным! Тем более что в летописи иное
слов "Мти", которое вы искусственно подгоняете под другое слово, да еще и потом
пропускаете его (другое слово) через кривой Гугл-Переводчик, так, что на выходе
получается у вас полная белиберда!

\iusr{Павел Рубец}
\textbf{Rod Jarrow}, 

что вы плетёте? Причем здесь Гугл переводчик? Есть церковнославянский
словарь. Скрин я сделал сам. Не доверяете, проверьте.

\iusr{Rod Jarrow}
\textbf{Павел Рубец}, 

Ё-маё! У вас, что амнезия? Повторяю еще раз свой вопрос из моего сообщения на
ваш скрин: "Где вашем скрине "Мти"?...Где? Если не понятно, расширим вопрос: С
какого фонаря вы сняли утверждение, что Киев (по умолчанию) не был Русским в
период истории ДО Олега? При том факте, что в летописи черным по белому,
русским языком написано, то титул князя Русского Олег получил лишь по факту
своего вокняжения в Киев..., увы НЕ до того! НЕ в период новгородского княжения
Олега! (Летопись: "...пѣрваго лѣт̑ Ѡлгова . Рускаго кн҃зѧ . пѣрваго лѣт̑ Ѡлгова .
понелѣже (потому что) сѣде в Києвѣ...)

Тоже самое, вот о чем эти две фразы из исторических Хроник арабской и
западно-европейской:

а) Ибн ХордадБех, "Книга Путей и Стран" 840 год: "Что могу сказать о купцах Руси, так эти из славян";

б) "Баварский географ" (список народов восточной Европы, относительно границы
государства Франков, время до 830 года). По списку "Баварского географа"
соседом к имени народа Ruzzi стоит имя Хазары, которые никогда в истории не
были соседом Словенам Ильменским (Новогородцам). Смотри таблицу составленную
по списку "Баварского географа" имя Ruzzi под номером 41 (хазары под номером
40):

\ifcmt
  ig https://avatars.mds.yandex.net/get-zen_pictures/4475186/857853061-1634675266139/orig
  @width 0.4
\fi

\iusr{Павел Рубец}
\textbf{Rod Jarrow}, вы походу не русский. Бред несёте. Мти это сокращённое слово МАТИ. Где оно в скрине? Оно там есть я вас уверяю.

\iusr{Rod Jarrow}
\textbf{Павел Рубец}, 

Летописное "Мти" - это мать, как не крути! Об этом речь. И, в скрине есть мать,
как не крути! (Я, вам приводил пример формы написания украинского "иметь" в том
предложении) В отличии от вашего "намека", что русский это обязательно должен
быть невежественный человек, который по умолчанию должен не владеть иными
языками кроме русского...(!), и, которому поэтому можно вскую лапшу на уши
вешать, в том числе и про летопись Русскую. А, когда не получается лапшу вешать
упражнениями с подменой понятий, сразу же переходим к обвинениям в Нерусскости!
Ну, а, как же! Русские ведь исключительно лаптем щи хлебают! Не так ли?...
Отсюда опрос: А, вы, не пытались предположить, что летописное Мти (мать) в
словах Олега по отношению к Киеву означает не ДЕТО-рождение, а буквально
Столица Руси? Или по-умолчанию такого быть не может у вас, потому что быть не
может никогда (в духе нашего времени)? Вот интересно, если бы Новгород ушел под
Литву в 15 веке, и в итоге по-сей день оказался вне последующей Русской
истории, что бы сегодня придумали наши "писатели-фантасты" на тему Русской
истории в отношении Рюрика, поселившегося в свое время в Ладоге и основавшего
Новгород?

(О том, что за период СССР было стерто из памяти: Оказывается украинцы и есть
русские, с самого начала истории Руси и до времени СССР, в котром придумали
"украинскую нацию". Однако, вот что мы находим в тексте официального
королевского поздравления запорожцам по случаю их многочисленных побед над
Османами в 1575 году - В тексте написано официальное обращение к национальному
имени чести и благородства (шляхетства) казачьего, запорожского, и к
национальному титулу всех казачьих запорожских гетманов (в период до 1648 года
/ что, в период политики украинизации в СССР было стёрто из всех учебников по
истории России) язык оригинала 1576 г. обращение к запорожцам:

«шляхетство Русское в чынах, урядах и реестровом козацстви знайдуючоесь,
едность и равенство имуть с шляхетством Польским и Лытевским, якоже зьедноченье
Руси с Польшою и Лытвою улажено есть и утверждено, и мы то подтверждаем и
заховуем Трыбуналу (закону) Русскому. Отправовать дила свои по прылычности в
новосозданном гроди нашом Батурыни, а як потреба укажет, то и в Черкасах; теж и
Гетьманови Русскому резыдовать в тым городи, ....; а бунчук мы жалуем Гетману,
на знак звытяжства его з войском своим над народом Азиатычным, от кого и
клейнод сей добут працею Гетманского и кровию козацкою.» -

К слову: Про авторов украинизации мозгов, смотрим дату:

\ifcmt
  ig https://avatars.mds.yandex.net/get-zen_pictures/4475186/857853061-1634709361648/orig
  @width 0.4
\fi

\iusr{алексей смирнов}
\textbf{Rod Jarrow}, 

\enquote{Баварский географ} по видимому говорил о своих соседях - Королевстве русов,
расположенном на Дунае, на месте нынешней Австрии. Соседом с востока им будут
Венгры, тогда конфедерация из восьми племен, как то - Мадьяры, Угры (в
славянских источниках Обры). Называли по разному, по известным
племенам. Восьмое, присоединившееся позже всех - Хазары. Учитывая отсутствие
следов Хазар на Волге это звучит логичней. Что касается \enquote{Повести}, то она
скомплектована не ранее шестнадцатого века. \enquote{Сын наследует отцу а в случае его
малолетства правит регент} - идея утверждается именно тогда, до этого правил
старший в роду. Сама \enquote{повесть} явно политический заказ Киева, тогда культурной
столицы днепровских казаков начавших борьбу за самостоятельность. Надо
отслеживать логику поступков политика а не его слова переданные кем-то. \enquote{Из
этого Киева мы будем иметь их всех} (в смысле города). Так смысла больше, да и
с дальнейшей реальностью сходится.

\iusr{Павел Рубец}
\textbf{Rod Jarrow}, 

мать городам пишется только так \enquote{МАТИ градовом}. Именно так, а не иначе это
выражение встречается и в других источниках. В \enquote{повести} написано \enquote{МАТИ
градом}. Разницу видите? Или Нестор допустил грамматическую ошибку, или написал
правильно, но в таком случае не \enquote{мать городам}, а \enquote{иметь город}.

И даже если быть точным есть разница в словах имать и иметь. \enquote{Имать} означает
брать, взять. А \enquote{иметь}, собственно иметь. И выходит Олег сказал это возьмём
городом русским.

Ну вы и фантаст. У вас Рюрик основал Новгород. Понятно

\iusr{Rod Jarrow}
\textbf{алексей}, 

Посмотрите на карту, номер 41, там Днепр, а не река Дунай - по которому Хазары
никогда не были соседями Руси. 2) Касаемо летописи: (которую я цитировал) вы,
очевидно, путаете с Пересопницким Евангелие, которое действительно датировано
16 веком! Потому что: а) Цитируемая мною летопись написана не в 16 веке, а в
11-12, когда ни каких казаков еще не было, поэтому летопись и заканчивается
временем Ярослава (Мудрого) и Ярополка... а это вообще-то 11 в. б) летопись
называется Ипатьевская на имя Ипатьевского монастыря где эта летопись и была
найдена, а это Рязань, где никогда ни каких запорожских казаков не было, что бы
ПЕРЕПИСЫВАТЬ киевский оригинал 12 века. 3) Где вы увидели не соответствие
текста летописи в том что Олег княжил при юном сыне Игоря? Я напомню, что
согласно тексту летописи Олег РОДИЧ Рюриков, возможно дядя Игорю, поэтому и
имел право на Регенство до совершеннолетия прямого наследника.

Могу лишь отчасти согласиться с вашим толкованием слов Олега: "Из этого Киева
мы будем иметь их всех" исключительно как забавная шутка!

(К слову: Книга Путей и Стран 840 года арабского географа Ибн ХордадБех у вас
тоже политический заказ Киева? Интересно, сколько Киев заплатил этому арабу,
что бы он в своем 840 году назвал Русь славянами?)

\iusr{Rod Jarrow}
\textbf{Павел Рубец}, 

Хватить уже чушь эту повторять про "иметь город"! В летописи "Мти", а это Мать,
как не крути, при этом во всех случаях, где встречается Мти в Лаврентьевской
летописи и, в Ипатьевской в том числе - в которой ("неграмотный") Нестор не
упоминается вообще ! Лучше подумайте о другом, почему согласно тексту летописи
титул князя Русского Олег получает не в Новгороде, а только лишь по факту в
вокняжения в Киев в 882 ? При этом сам Рюрик вообще нигде в летописи не
упоминается с этим титулом "Князя Русского" в отличии от Олега....

\iusr{Павел Рубец}
\textbf{Rod Jarrow}, 

[Лучше подумайте о другом, почему согласно тексту летописи титул князя Русского
Олег получает не в Новгороде, а только лишь по факту в вокняжения в Киев в 882
? При этом сам Рюрик вообще нигде в летописи не упоминается с этим титулом
"Князя Русского" в отличии от Олега...]

Как Олег мог называться князем до смерти Рюрика? У Рюрика был маленький Игорь
наследник. Которого Олег показал киевлянам, говоря вот настоящий князь, сын
Рюрика. Олег шурин Рюрика.

[В летописи Мти мать как ни крути.] Это вам так кажется.

\iusr{Rod Jarrow}
\textbf{Павел Рубец}, 

...Упс! Вот это да! Оказывается у вас уже и Рюрик еще жив при вокняжении Олега
в Киев! Надо же! Я напомню, что Олег не сразу в Киев рванул из Новгорода, как
только умер Рюрик! Цитирую энциклопедический ПОМЕСТНЫЙ титул Олега: князь
Новгородский, князь Киевский...(Сам Рюрик в энциклопедии упоминается
только-лишь с одним титулом князя Новгородского). Так я повторю вопрос, почему
титул "Князя Русского" Олег получает не в Новгородский период княжения (сразу
после смерти Рюрика), а позже, только лишь по факту в вокняжения в Киев в 882 ?
(При этом сам Рюрик вообще нигде в летописи не упоминается с этим титулом
"Князя Русского" в отличии от Олега ! ) К слову, пояснение о том, чем ПОМЕСТНЫЙ
княжеский титул отличается от Велико-княжеского титула - титула Главы / Владыки
всей Державы (всего народа и всей земли в Державе), в котром упоминается не
город княжения (как в ПОМЕСТНОМ титуле, город ПО МЕСТУ княжения), а
исключительно название всего народа и всей земли (по аналогии с титулами,
короля ПОЛЬСКОГО, короля Шведского, Французского, Английского ..и тд., точно
также и у нас на Руси был титул Владыки всего народа и всей земли, это титул:
Великий князь Русский - в киевский период / Царь Всея Руси - в московский
период) Летописный пример упоминания этого титула у Олега (в отличии от
Рюрика), начиная с времени Олегова вокняжения в Киев:

"...иже по̑слани Ѡлга великаго кнѧзѧ Рускаго и всѣх̑ иже суть под̑ рукою єго .
свѣтълыхъ боӕръ . к вамъ Львови и Александру . и Костѧнтину . великьмъ.

Тот же самый титул упомянут и у Игоря после смерти Олега (титул, котрого нигде
нет у Рюрика по-тексту летописи) :

"...послании Игорѧ великого кнѧзѧ Рускаго и всеӕ кнѧжьӕ и всѣх людии Рус̑коє
земли и тѣхъ заповѣдано ѡбъновити ветхыи миръ...

\iusr{Павел Рубец}
\textbf{Rod Jarrow}, где я писал, что Олег вокняжился до смерти Рюрика? Зачем вы так много пишете не по теме?

\iusr{Rod Jarrow}
\textbf{Павел Рубец}, 

Вы не ответили на мой вопрос... (к слову Олег княжеского рода и до и после
смерти Рюрика! Это факт летописи: "азъ єсмь роду кнѧ̑жа" ) Речь вообще-то о
другом, о том в какое время и какой княжеский титул был у Олега и почему? И,
почему этого титула не было у его предшественника князя Рюрика? При том
летописном факте, что Рюрик был официально призван на РУССКОЕ княжение..., а не
на АБСТРАКТНОЕ княжение в КОСМОС, туда не знаю куда, в чистое поле на пустое
место... (в предыдущем сообщении я вам пояснил чем отличается ПОМЕСТНЫЙ
княжеский титул от титула Велико-княжеского - титула Главы Державы, титула
Владыки всего народа и всей земли державы Русской.

\iusr{Павел Рубец}
\textbf{Rod Jarrow}, 

какое отношение все это имеет к теме этой статьи?. Или вам просто хочется
подискутировать на вам интересной теме?

\iusr{Rod Jarrow}
\textbf{Павел Рубец}, 

Какое отношение...? Увы, самое прямое! Потому что: Если, не Новгородское, а
именно Киевское княжение давало право на титул Владыки всей Руси (титул
Великого князя Русского) от Переяславля до Ростова, тогда все вопросы этой
истории (вообще все на 100\%) как пазлы складываются в единую картину!
Становится понятно абсолютно всё:

а) Понятно почему Олег ушел в Киев, покинув Новгород - тот самый Новгород,
основанный не кем-нибудь, а самим Рюриком! Новгород - реально географически
ЛУЧШЕЕ место для столицы, которое контролировало сразу аж два главных торговых
пути 9 века Волжский и Днепровский, в отличии от Киева реально ХУДШЕГО места,
который, хоть и стоял на Днепровском пути, тем не менее был вообще в стороне о
Волжского торгового пути, да еще и постоянно подвергался набегам НЕдобрых
соседей с востока! Что сразу же и случилось по вокняжению Олега в Киев, когда
тут же, следом пришли Черные Угры (покушать) как на ПРОХОДНОМ ДВОРЕ в Европу...
Ибо иной логики для Олега уходить из Лучшего места в Худшее, априори не
существует, если Олег в здравом уме!

б) Если не Новгород, а именно Киев столица (времени Рюрика/Олега), становиться
понятно почему Олег ЕДИНСТВЕННОГО наследника (юного, сына Рюрика княжича Игоря)
буквально младенцем понес в этот рискованный военный поход изгонять из Киева
самозванцев Аскольда и Дира, которые после смерти Рюрика не являясь князьями по
роду, сели княжить в Киев...! Что фактически стало смертельной угрозой юному
наследнику, у котрого единственного было ПРАВО на Киевское княжения! А, у
Аскольда и Дирда такого права не было! Цитирую летописные слова приговора
Аскольду и Дирду, устами Олега:

"...вы неста кнѧзѧ ни роду кнѧ̑жѧ . но азъ єсмь роду кнѧ̑жа . и вынесоша Игорѧ .
сь сн҃ъ Рюриковъ . и оубиша Асколода . и Дирд̑а."

Ибо иной логики для Олега априори не существует, если Олег в здравом уме!

в) Если не Новгород, а именно Киев столица (времени Рюрика/Олега) становится
понятно почему у Рюрика нет титула Князя русского, при том факте, что Рюрик был
официально призван на Русское княжение, но, увы, в Киев так и не сел! В отличии
от Олега и своего сына Игоря, у которых этот титул "Великий князь русский"
упоминается по несколько раз! А, с именем Рюрика ни одного упоминания этого
титула в летописи.

Ибо иной логики для Олега априори не существует, если Олег в здравом уме!

\iusr{алексей смирнов}
\textbf{Rod Jarrow}, 

несколько общих и поверхностных фраз, большей частью не имеющих отношение к
поставленным вопросам. Очень жаль что не услышал конструктивной критики.

\iusr{Rod Jarrow}
\textbf{алексей}, 

Королевство Русов на Дунае никогда не было! Это современная фантастическая
чушь, притянутая за уши.. ! Почитайте лучше "Баварского географа" внимательно,
там по списку соседями имени "Ruzzi" названы не только Хазары но и Древляне
(Форсдерен) увы а это уже совершенно определенная географическая
локализация.... которую ни каким боком не перетянуть на Дунай, тем боле на
Балтику к Ругам, ибо Русь и Руги это точно такие же "синонимы" (в кавычках) как
Молдавия и Мордовия! Совершенно различные корни в том числе и по смыслу...Ваш
путь поиска подобия в исторических названиях - заблуждение! Потому что по
вашему пути прямой выход в государство Урарту где правителями были Руса I, и
Руса II.

Увы, дело вот в чем: Русь неразрывно связана с торговым путем из варяг в греки,
буквально как ИМЯ этого Торгового пути - Русь - поэтому априори не может
возникнуть где-либо в ином месте! Вспомните Балтийские города торгового союза
Ганза? Так вот Русь это по сути города такого же самого Торгового союза, но на
другом торговом пути, на пути из варяг в греки. К слову вам пища к размышлению,
почему сразу аж на трех славянских языках имя Руси имеет прямое отношение к
понятию ДВИЖЕНИЯ (движения торгового пути ) Вспомните русский корень слова
РУСло? Что это? Это буквально направление ДВИЖЕНИЯ реки... Так вот выражение:
"Сидеть НА ДВИЖЕНИИ торгового пути из варяг в греки" - в переводе на
украинский. вы получите буквально: "Сидеть НА РУСИ торгового пути из варяг в
греки! (НА РУСИ это буквально НА ДВИЖЕНИИ в переводе с украинского на
современный русский (при том, что украинский это тоже русский язык, но чуть
древнее) К слову это понятие движения в имени Руси точно в том же значении
сохранилось и в современном Польском, в котором имя Русь говорится и пишется,
только так: "РуЩ" (РуШь) буквально польским корнем понтий ДВИЖЕНИЯ - РуШ (РУШай
на польском - это иди, двигайся, поезжай...)

\iusr{Павел Рубец}
\textbf{Rod Jarrow}, 

1. Рюрик не основывал Новгород. Он стал княжить в Ладоге, при этом Новгород уже
существовал. Известно что правил в нем Гостомысл. У которого погибли все
сыновья.

2 Олег получил власть от Рюрика, став его преемником.

3. Среди тех, кто призывал Рюрика нет полян, древлян. Ни Киев, ни Искоростень
не входил в число тех, кто делигировал Рюрику власть.

4. Олег подчинил Киев перенес столицу по экономическим причинам. Расположение
на Днепре. Трафик товаров, умеренный климат.

5. Киев не был в подчинении Рюрика. Поэтому Рюрик не княжил в нем. потому не
сел в нем.

\iusr{Rod Jarrow}
\textbf{Павел Рубец}, В обратном порядке:

5) Киев был оккупирован Хазарами - этот факт есть в летописи (также как в своё
время Москва поляками) очевидно, поэтому официально призванный на Русское
княжение Рюрик не сел в Киев, в отличии от Олега, который сел в Киев лишь через
20 лет после призвания Рюрика. Я напомню слова Олега после вокняжения в Киев
всем славянским племенам торгового союза Русь Слова Олега из Киева: "Платите
дань мне, а не хазарам, потому что я враг им!" (Очевидно, потому что не враг
славянам Торгового союза Русь)

4) Олег не подчинял Киев, (точно также, как Мини и Пожарский не подчиняли
Москву) Точно также Олег ОСВОБОЖДАЛ столицу, как представитель НАСЛЕДНИКА,
ЗАКОННО ИЗБРАННОГО Князя На Русское княжение - чей титул "Великого князя
Русского"- можно было получить только в Киеве во время Рюрика/Олега. Эту
разницу нельзя путать и переиначивать на мифический "перенос столиц" котрого
никогда не было! Увы, переносить столицу может только глава Державы, а Олег с
юным княжичем Игорем - получат титул главы державы Русской, лишь по факту
занятия Киева! Увы, ни как не ранее того! - Это летописный факт! Поэтому у
Рюрика не было этого титула, потому что Рюрик не сел в Киев а Новгород не
столица Руси... Вам эта чушь понятна о переносе столиц?

3) Что бы понимать кто в списке призвания Рюрика под единым именем Русь ( ркоша
. Русь . Чюдь . Словенѣ . Кривичи . и всѧ землѧ наша велика . и ѡбилна . а
нарѧда въ неи нѣтъ . да поидете кнѧжит̑ и володѣть нами ) - следует летопись
читать СНАЧАЛА! Понятно? СНАЧАЛА летопись нужно читать, а не с 30 стр призвания
Рюрика.... Цитирую летопись история Руси ДО призвания Рюрика о племенах,
которые в тексте призвания выступили под единым именем Русь цитирую предысторию
призвании Рюрика о том кто это под именем Русь в призвании:

"се бо токмо Словѣнескъ ӕзыкъ в Русі . Полѧне . Деревлѧне. Новъгородьци .
Полочане . Дьрьговичи І . Сѣверо . Бужа̑не . зан̑ сѣдѧть по Бугу . послѣже
Волынѧне . -

Увы это список племен Руси стоит в летописи во времени княжения династии Кия
т.е. во времени до Рюрика и до оккупации Киева хазарами.

2) Олег получил ПРАВО на власть от Рюрика, став преемником ОФИЦИАЛЬНО
ПРИЗВАННОГО на Русское княжение! Вот на основании этого ПРАВА как Регент
наследника Рюрика, Олег и получил титул Князя русского увы в Киеве, а не в
Новгороде.

1) Согласно летопись Новгород основал Рюрик, согласно археологии Новгород
основали лишь через столетие после Рюрика.

\iusr{Павел Рубец}
\textbf{Rod Jarrow}, 

вам что выговориться захотелось? Я спорил по поводу мати градовом. И остаюсь при
своем мнении. Зачем вы излагаете все остальное? Это всё зачем пишете?

И как Рюрик мог основать Новгород, если согласно летописи в нем правил
Гостомысл? Где жил Гостомысл, если Новгород основал Рюрик?

\iusr{Rod Jarrow}
\textbf{Павел Рубец}, 

1) Тут никто не претендует на ваше право жить на Луне в своем собственном
воображении...! Увы! Вы можете писать что угодно, однако, в археологии
новгородской земли нет ничего глубже второй половины X века...! Понятно о чем
речь? 

2) Увы, в летописи именно Рюрик основал Новгород цитирую: "....по дъвою же лѣту
. оумре Синеоусъ . и братъ єго Труворъ . и приӕ Рюрикъ власть всю ѡдинъ . и
пришед̑ къ Ильмєрю . и сруби город̑ надъ Волховом̑ . и прозваша и Новъгород̑ .  и
сѣде ту кнѧжа̑ ." - Увы, при том факте что Археология ставит под сомнение дату
летописного оснвования Новгорода Рюриком так что про Гостомысл тут я вообще
молчу ибо он согласно легенде был до Рюрика

\iusr{Павел Рубец}
\textbf{Rod Jarrow}, 

И въ то время въ Новеграде некый бе старейшина именемъ Гостомыслъ, скончаваеть
житіе, и созва владалца сущая съ нимъ Новаграда, и рече: «советъ даю вамъ, да
послете въ Прускую землю мудрыя мужи и призовете князя отъ тамо сущих родовъ.

\iusr{алексей смирнов}
\textbf{Rod Jarrow}, 

интересные рассуждения о происхождении названия русь, спасибо. История
дунайских русов (или ругов, их так же как и киевских называли и так и так)
разработана достаточно подробно, зачем же такие громкие заявления.
Вышеупомянутый вами арабский географ писал о трех группах русов, это необходимо
учитывать. Славянские племена часто называли по именам рек в месте их
проживания. Древляне от реки Драва (правый приток Дуная, протекает в том числе
и по Австрии) а не от деревьев. Как раз соседи и будут.

\iusr{Rod Jarrow}
\textbf{алексей}, 

РуШ (РуЩь - Движа; Буквально "Торговый Движ") Это было последней каплей когда
все сомнения отпали, когда я услышал это собственно польское название Руси
(РуШ), при том факте, что за прошедшую тысячу лет никто не мешал полякам тупо,
бездумно повторить с окончанием С (РуСь) в конце слова, но это оказалось в
принципе невозможно! Потому что РуШ - это собственно польский корень понятия
движения, в отличии от восточных соседей поляков у который польский корень РуШ
переходит в РуС, которые и сидели на Днепре - сидели НА РуСи торгового пути из
варяг в Греки,...Именно, поэтому тот самый, изначальный смысл имени Руси,
замороженный в польском языке, как в тысячелетнем Леднике сохранил СМЫСЛ корня
имени Руси, который также не случайно оказался в словаре русского языка еще и в
слове "РУС(ло) направление ДВИЖЕНИЯ реки...

- Понимаете почему любая "разработка" поиска Руси где-либо вне поднепровья, это
изначально поиск "черной кошки в темной комнате, где её нет, никогда не было и
быть не могло..." Увы, поиск корней Руси по разработанному "НАРОДНОМУ" методу
поиска звукового ПОДОБИЯ (без смысла) рано или поздно обязательно выведет на
прямую дорожку к Шумерам в государство в Урарту где была целая династия
правителей на имя Руса I и Руса II...По аналогии "происхождение" неМцев от
народа неНцев..." Увы, "разработанный "Народный метод подобия" приведет именно
сюда! В эту чушь..! Куда еще ближе НеМцы и НеНцы! При том факте, что в
морфологии языков буквы М и Н взаимозаменяемые..

(К слову, оказывается не только поляки но и НЕ славяне греки в своем далёком
времени ПО СМЫСЛУ перевели имя Руси на греческий - Дромиты (греческое слово
связанное с понятием ДВИЖЕНИЯ...

\iusr{алексей смирнов}
\textbf{Rod Jarrow}, 

интересное суждение. Но может быть сходство самоназвания русов с польским
термином чисто случайное, так-же как и с урартскими именами? Западные русы
точно так-же пользовались дунайским торговым путем а он был помощней
днепровского, так что здесь вы погорячились. По поводу торгового пути"из варяг
в греки", скорей всего это фейк и вот почему. Нева ближе к своему истоку была
сплошь в порогах и перекатах (расчищена в семидесятых прошлого века). Ниже по
карте Ловать в верхнем своем течении с середины лета становится несудоходной.
От Ловати до Днепра волок 150 км, при этом на половине пути он пересекает
Зап.Двину. Двина судоходна большую часть года по всему своему течению, впадает
в Рижский залив, расположенный гораздо западней и освобождается ото льда
намного раньше финского. Затрат меньше.

\iusr{Rod Jarrow}
\textbf{алексей}, 

Польский славянский корень движения - РУШ мог быть случайным совпадением при
одном условии, если бы рядом с поляками не существовал соседский славянский
термин РУС с точно таким же, как у поляков, значением движения, да еще и при
том факте что именно эти восточные славянские соседи поляков вот с этим корнем
РУС сидели НА РУСИ - буквальный перевод на современный украинский из
современного русского - сидели НА ДВИЖЕНИИ торгового пути из варяг в греки. При
том факте, что и у этих восточных славян, соседей поляков существует их
северо-восточные соседи славяне, у которых в словаре (современного русского
языка) одновременно сохранились оба этих корня с тем же самым значением
движения: РУС и РУШ, в слове РУСло (направление движения реки) и в словах
"раз(РУШ)ение" и "об(РУШ)ение" - увы, тоже самое понятие движения хотя и не
"созидательное" в отличии от торгового движа! Как не крути, но сопадений
слишком много, и все географически буквально приклеены к движению торгового
пути из варяг в греки - у котрого, точно должно было быть единое имя
(РуСь/РуЩ), аналогично пути, объединявшему торговлю между городами от русского
Новгорода до немецкого Любека также с единым именем "Ганза"...которое априори
не подразумевает существование некоего народа-племени с именем Ганза, якобы
давшего название этому торговому союзу... Раньше я тоже интересовался поиском
некоего сказочного народа-племени от имени которого, якобы и произошла Русь, но
факты указывают на изначальное появление ТОПОНИМА Русь собственно славянского
топонима, от которого и прозвались Русью все славянские племена, которые
буквально сидели НА РУСИ (по направлению движения этого торгового пути, сидели
буквально по РУСлу торгового пути из варяг в греки 9 века. Именно поэтому
историческая Русь и торговый путь из варяг в греки в нашей истории,
географически Не существуют отдельно друг от друга. (К слову о Западной Двине:
Что бы из Балтики по Двине идти с товаром в Византию, увы, кроме, как из Двины
(на перекладных) в Днепр иначе не попасть в Черное море)

\iusr{Евгений Сафронов}
\textbf{Rod Jarrow}, 

"Автор русским языком по тексту летописи примерами пояснил, что во всех
случаях, где в летописи появлялось слово "мти", это была только мать и ни что
иное..."

Вы передергиваете. Автор приводил примеры не все. Он брал только выборочно, и
только те, где шли описания родственных (а не имущественных) связей. Не зачёт.

Между тем есть вполне себе родственные слова имущественного (а не родственного
значения) к примеру:

ЕМАТИ, ЯТИ, ИМАТИ - брать. Мать кстати может быть не только МАТИ (в
повседневном употреблении), но и МАТЕРЬ (пафосно). Матерь Божия, а не мати
Божия. Так что не всё так ясно и определённо, как однобоко утверждает автор.

\iusr{алексей смирнов}
\textbf{Rod Jarrow}, 

вы упорно все привязываете к Днепру, упуская из вида что кроме днепровских в
десятом веке

существовали еще три группы русов- таманские, дунайские и балтийские. А первые
сведенья о росах относятся к четвертому веку, их еще называли роксоланами. Рохс
по алански светлый или белый. В России было черное сословие (те кто платил
налоги) и белое (те кому платили, т.е. армия). Росы или роксоланы- армия алан,
позднее термин превратился в имя этноса. Термин даже стал модным. В 13 веке в
Нижегородской летописи есть сообщение как карательный отряд вторгся на
территорию мордвы в ответ на их набеги. В перечне мордовских племен была некая
"русь мордовская", видимо отряд профессиональных войнов.

\iusr{Rod Jarrow}
\textbf{алексей}, 

Да, я упорно привязываю происхождение славянской Руси к торговому пути из варяг
греки (по Днепру)! Буквально топонимом - Русь / Руш - названием для торгового
пути из Варяг в Греки - Русь, потому что само это слово собственно славянского
корня происхождения, от котрого и родилось общее название Русины (упомянутое
еще в Договоре Олега с Византией) для всех племен, которые сидели по РУСИ по
РУСлу этого Торгового пути! И, ни как иначе! Согласно тексту летописи Русской.
И, что ни так? Не взирая на историю державы Урарту - где была аж целая Династия
верховных правителей Руса I, Руса II Руса III Руса IV... Что однако с нашей
Русью аж на тысячелетие не пересекается во времени! Поэтому узнайте, что это
такое "Омонимы в морфологии языков мира это слова, которые звучат и даже иногда
еще и пишутся - одинаково! Но которые, увы, имеют не только различное значение,
но и даже происхождение! Тому пример наше простонародное сокращение названия
автоматов Калашникова - Калаши! Так и в горах памира издревле живет народ (при
этом еще и голубоглазый) который издревле так и называется - Калаши! Ну и кто,
от кого произошел по-вашему мнению...? Какие Калаши от каких Калашей произошли?
Увы, ни кто ни от кого не происходил! Это просто Омонимы, хотя и совпадение
шестизначное, как в лото миллион!

В договоре Олега с Византией это термин Русин - касается всех поголовно на 100\%
от верхов до низов, любого роду-племени, кто из Руси пребывает в Византию,
цитата из Договора:

"аще кто оубиєть Хрс̑тьӕна Русинъ или Хрис̑тьӕнъ Русина . да оумреть ."

\iusr{алексей смирнов}
\textbf{Rod Jarrow}, 

жаль что дискуссия у нас не получается. Все мои возражения вы пропускаете мимо
никак не пытаясь опровергнуть а упорно твердите одно и то-же. Могу лишь
добавить, "славяне" термин западно-славянский, в средние века к восточным он не
применялся. Если средневековый источник упомянул славян и русов вместе, то это
скорей всего центральная Европа. Все указывает на дунайскую группу русов, по
соседству с ними словены и видимо сами они ими были, а так же присутствуют
древляне и хазары. Днепровская русь лишь одна из групп и возникла она позже
остальных, когда этот термин уже существовал.

\iusr{Rod Jarrow}
\textbf{алексей}, 

О дискуссии: Вот, как бы вы отнеслись к предложению поискать ваших родителей
среди других родителей, исходя из идеи (идеи снятой с потолка), что ваши
родители (с которыми прожили всю жизнь с самого детства и уже даже похоронили)
это, якобы, были не ваши родители? Увы, это оскорбительная ерунда! Вот также и
я отношусь к подобным идеям поставить летопись Русскую под сомнение! Самому
себе придумать, что летопись русская это якобы выдумка! Так вот, увы, у вас
перепутана хронология потому что русы (русины) появились на Дунае уже после
образования Руси на её материнской территории торгового пути из варяг в греки!
Они (русины) и сегодня там проживают на территории Сербии! Но русины пришли на
Дунай уже позже! Что, не имет ни какого отношения к римским этРУСкам и тому
подобной чуши поиска подобия, как неМцы и неНцы в названиях фактически
идентичны, при том факте, что в морфологии языков букв М и Н взаимозаменяемые!
При этом, друг к другу эти народы не имеют ни какого отношения! Точно также,
как современные Турки, и совсем иные Турки - из византийской хроники около X
века (когда еще ни какие тЮрки с востока не вторгались на территорию Византии)
Вот в этом историческом периоде (предшествующем вторжению тЮрок) в византийской
хронике "турками" назывались современные венгры ! (в то время угро-финны) Так и
сегодня на территории современной Финляндии вы найдете старинный финский
городок "Турку" - который увы не имеет ни какого отношения к современной
Турции! Вам понятна принципиальна ошибка поиска корней истории народов по
тропинке подобия в названиях..., которая, рано или поздно, вас приведет к
правителям государство Урарту от Русы I до Русы IV наал 1 тысячелетия аж До
Нашей Эры!

О соседстве Руси со Словенами: Летописные словены, назвавшиеся своим именем
(Славяне) это три группы, это: Адриатические Словенцы, Карпатские Словаки и уже
исчезнувшие Словене Ильменские) Так вот Ильменские и Адриатические - это
пришлые из центра зарождения европейских Славян, где и сидят современные
Словаки, при этом сидят прямыми соседями Руси откуда и вытекает река РУСава,
притоком в чешскую Мораву, сегодня источник реки РУСава начинается на
территории Словакии, и тем не менее все эти названия более поздние, возникшие
уже от названия державы, которая зародилась на торговом пути из варяг в греки,
чья географическая локализация известна всему миру. Что не нужно ставить под
сомнение, потому что это поиск новых родителей вместо собственных, по дорожке
подобия в названиях

\iusr{алексей смирнов}
\textbf{Rod Jarrow}, 

где это я утверждал что летопись выдумка? Не стоит приписывать мне свои
фантазии. Далее средневековые источники, как европейские так и азиатские. 6
век: ас-Саалиби (росы союзники хазар). Захир-ад-дин-Мараши (русы на
Сев.Кавказе). 626г: Константин Манасси (русы в числе осаждавших
Константинополь). 643г: ат-Табари (русы враги арабов). 773г: русы противники
франков. 8 век: таманские русы вторгаются в Крым. Географ Баварский (русы
соседи хазар, упоминается река Эльба). 839г: Вертинские аналы (послы народа рос
к Людовику Благочестивому) 842г: житие Георгия Амастридского (нападение росов
на Малую Азию) 844г: ал-Якуби (нападение русов на Севилью) 860г: нападение
росов на Константинополь. 861: Кирилл Философ (русы в Крыму) 863: подтверждено
право владения(марка) русов на территории Австрии. 894г: хроника Пулкавы (Русия
в составе Моравской державы) Это лишь некоторые.

\iusr{Rod Jarrow}
\textbf{алексей}, 

Давайте ка прекращать уже перевирание терминов и дат! В 626 осаждали
Константинополь Не Русы! Это современная лапша на уши! В Византийской хронике
упоминаются дословно Славяне чей флот с моря неудачно атаковал Константинополь
где среди погибших византийцы обнаружили воинов женщин...Таже Ни каких
таманских русов в 8 в. не было! И, балтийские Руги это не Русы, как не крути!
(Также как Мордовия это не Молдавия) Аль Якуби умер в 907 Нападение на Севилью
Ар маджусов Ар Русов в его арабской хронике - это вообще ни о чем! Вполне могли
быть Новгородцы, что не противоречит летописи! Потому что в 844 Аль Якуби был
мягко говоря "сильно молод" что бы хронику писать...; Кирилл Философ жил до 869
- так, почему это при его жизни Русы по вашему не могли в Крым попасть из
Днепра, при том факте, что еще до смерти Кирилла летопись упоминает поход Руси
на Византию (и Константинополь) при Аскольде и Дире, а Крымские черноморске
города был частью Византии! ... Так, давайте ка уточним всё, что вы
перепутали...:

Великая Моравия держава образовалась с 830 г, так и сегодня на территории Чехии
течет река Русава притоком главной чешской реки Моравы, река Русава, которая
начало свое берет на территории современной Словакии, где также проживают
современные карпатские русины на своей земле Древней Руси... Так, современная
Словакия и была частью Великой Моравии в 9 в.. И, в летописях Русы и Словены
соседи! И, что не так? К слову есть две реки Русавы, очевидно очерчивающие
границы древней Руси, западную Русаву на территории Словакии и Чехии (бывшей
Великой Моравии я уже упомянул, где, к слову были князья Моравские с именем
Святополк (Упс!) - точно таким же именем, как и у Рюриковичей) , а вторая
Русава река протекает притоком реки Днестр в Винницкой области (современная
западная Украина). О том что карпатские Русины переселялись и на Дунай в
современную Сербию я уже пояснял вам, что всё это не 7,8 и даже не 9 век, всё
это случится позже на несколько столетий! (Хроника Пулкавы это не 894 это 15
век! Увы!:

В Бертинских Аналах упоминаются разоблаченные шведские разведчики, которые под
прикрытием имени славянской Руси пытались попасть на территорию франков;

Баварский географ это до 830 г Потому что в списке "Баварского географа" Моравы
упомянуты отдельным племенем, отдельно от иных племен, которые с 830 станут
частью "Великой Моравии"... Ну и где вы узрели Эльбу по списку Баварского
географа, имя "Ruzzi" под номером 41 на карте, где здесь Эльба?:

\ifcmt
  ig https://avatars.mds.yandex.net/get-zen_pictures/4012878/857853061-1635403616404/orig
  @width 0.4
\fi

\iusr{Василий}
\textbf{Rod Jarrow}, 

не нужно искать сраную кошку в туалете, приплетать украинский язык (которого не
было до советской власти) к обсуждаемой теме, а так же казаков, поляков и
арабов. Столица Древней Руси переносилась из города в город неоднократно и Киев
не был первым и последним. Так называемые титулы здесь вообще не при чем,
правители как хотели, так себя и величали (князь, царь, генсек, президент).
Фраза Олега о Киеве, как о матери городов, говорит только о том, что от ныне
столицей будет Киев и не о чем здесь спорить. Побыл столицей и перестал.

\iusr{Rod Jarrow}
\textbf{Василий}, 

Киев был первым, но не был последним... Читайте летописи, а не современную
фантастику (про летописи) в которой главари как хотят, так себя и называют
исключительно в бандах! В отличии от государства, которое называлось Русь, даже
если это кому-то это очень не нравится. Увы историю не переписать как не крути.

\iusr{Василий}
\textbf{Rod Jarrow}, 

историю переписывают только кастрюлеголовые хисторики, а по факту первой
столицей была Старая Ладога, затем Великий Новгород. Олег захватил и перенёс
столицу в Киев. Об этом упоминается и в Ипатьевской летописи и в ПВЛ.
Доказывать что-то тупоголовым щеневмерликам - себя не уважать.

\iusr{Rod Jarrow}
\textbf{Василий}, 

Резиденция того или иного князя и столица державы - это не синонимы! Это путать
невежественно! Ладога также как и Новгород никогда не были столицей Руси но при
этом были княжьей резиденцией Рюрика... так у него и до призвания на Русь были
княжьи резиденции и что дальше...? Он же не с Луны упал! Что бы понять эту
разницу последуйте совету автора статьи, попробуйте прочитать хотя бы одну
летопись, (в оригинале на церковно-славянском), а не вырванные цитаты из
летописей на страницах желтой прессы... Начиная с того факта, что априори
НЕВОЗМОЖНО призвать кого-либо на княжение, туда, где нет института (традиции)
княжения соответственно уже существующей Велико-княжьей столицы... Так вот если
Рюрика действительно призывали, это по умолчанию означает, что держава УЖЕ
БЫЛА! А, значит БЫЛА И СТОЛИЦА! (Михаила Федоровича Романова тоже не в пустоту
призывали на царствие! Хотя и занятую шведами и поляками) Иначе просто
невозможно призывать кого-либо на княжение в пустоту космоса! Туда,не знаю
куда! В чистое поле на пустое место! СТРОИТЬ Русь местным аборигенам, которым
она и нахрен не нужна, при этом строить руками тех, кто сам понятия не имеет о
том что это такое держава с институтом ЗАКОНА и ПРАВА! - Неужели не понятно,
что это чушь собачья! Которую, начиная еще с немца Шлёцара до сих пор у нас
вешают лапшой на уши лохам...

\iusr{Кот Сибирский}
\textbf{Павел Рубец}, да всё проще. Мать городов это дословный перевод греческого (византийского) метрополис.

\iusr{Владимир Марченко}
\textbf{Павел Рубец}, 

а у вас есть конкретный пример, где \enquote{мти} однозначно употреблено как \enquote{имати}?
Вообще существительное от глагола в контекстах легко отличить

\iusr{Павел Рубец}
\textbf{Владимир}, 

мать городов пишется только так \enquote{мати градовом}. Есть несколько мест где
пишется только так, а не иначе, если речь ведётся о столице. У Нестора написано
мати градом.

Выше я привел картинку из церковнославянского словаря. Там есть мати-маю-имати.

\iusr{Владимир Марченко}
\textbf{Павел Рубец}, 

так я ж не об этом спрашиваю. Меня интересует предлагаемое вами слово \enquote{имати}.
Где пример написания \enquote{мти}? Неужели нет?

\iusr{Павел Рубец}
\textbf{Владимир}, 

Не знаю где ещё найти текст написанный на македонском наречии болгарского
языка. В сокращённой форме с титлами. Он может и есть но уже без сокращений. И в
нем написано уже не мти, а мати.

\iusr{Василий}
\textbf{Rod Jarrow}, 

сам то понял, что написал? Действительно Рюрика призвали (если такое было) в
державу со столицей в Старой Ладоге, затем он перенёс её в Великий Новгороде. В
то время и в старорусском языке не было понятия и слова столица, был стол и
стольный град, где великий князь проживает там и стольный град. Не нужно
пытаться поучать людей, когда сам не компетентен в данном вопросе.

\end{itemize} % }

\iusr{Владимир Дубровский}

Насколько я знаю, старославянские тексты писались без огласовок (без гласных),
поэтому МТ могло означать и МаТь и иМеТь. При более позднем переписывании смысл
фразы мог быть неверно истолкован. Вот и появились МТи вместо иМаТи.

\iusr{Фёдор Крюкин}

Мти - столица?

\begin{itemize} % {
\iusr{Rod Jarrow}
\textbf{Фёдор Крюкин}, 

ОПЕКА материнская в первую очередь! Столица вытекает из понятия Материнской
ОПЕКИ... Если почитаете летопись на церковно-славянском то обнаружите что титул
Князя Русского Олег получает не в Новгороде, а по факту вокняжения в Киев, увы,
титул Владыки земли и народа во всех странах мира во все времена истории
получить можно было только в столице этой земли и этого народа. Летопись:

"пѣрваго лѣт̑ Ѡлгова . Рускаго кн҃зѧ . пѣрваго лѣт̑ Ѡлгова . понелѣже (потому что)
сѣде в Києвѣ

\end{itemize} % }

\iusr{Alexandr Sviridov}

И теперь эта фраза даёт щирым утверждать что де "Киев - это столица, а ваш
Новгород появился позже и построили его украинцы". Вы, автор, не знаю "из какой
тарелки едите", но такой формой перевода поднимаете огромный слой ненависти
между народами. Вот как по мне, так перевод "станет городом русским" куда как
более нейтрален и не даёт повода к противостоянию. Был хазарским - стал
русским. Всё ровно и хорошо.

\iusr{Николай Костенко}

Господа и с т о р и к и - вы хорошо знаете - все ПОВЕСТИ - списки (копии!) в
кубе. Достоверных источников нет. Выборку делаете по принципу нравится или нет.

\iusr{сергей костин}
\textbf{Ольга} имела сына Где тут была мать

\iusr{сергей костин}

Придумано это позднее Часто неизвестно кто был матерью Анны королевы франции
или других людей Не было такого понятия мать всего Отец другое дело Назвал бы
Киев будет отцом городов русских и по смыслу и по правилам понятно А так город
он стал вдруг матерью Не было тогда голубых нравов хотя смысл понятен захватили
убили поимели

\iusr{сергей костин}

Мать нигде никогда не упоминается в ранний трудах Отчизна отчий край Родина
женского рода поэтому разъяснение сразу родина мать Женщина существо ниже
мужского тогда считалось назвать кого либо женщиной просто оскорбить

\begin{itemize} % {
\iusr{Rod Jarrow}
\textbf{сергей костин}, Вы не русский человек если Мать-Земля. Родина-Мать для вас оскорбление.

\iusr{сергей костин}
\textbf{Rod Jarrow}, Да быстро как кошки Захватили поимели и сразу мать Ну или мачеха
\end{itemize} % }

\iusr{Serge Mark}

Давно известная калька с греческого,поэтому и \enquote{мать} Вопрос давно исчерпан и
только псевдопатриоты хотят вычеркнуть Киев из русской истории.

\iusr{Александр М}

Автор, а вам не приходило в голову, что при таком способе письма, когда
пропускаются гласные буквы, некоторые слова имеют одинаковое написание?

Для полноты доказательства нужно найти слово \enquote{имати, иметь}, как оно написано.
А так похоже на древний еврейский. До сих пор спорят, что написано.

\iusr{сергей ганьжин}

\enquote{Нашёл оригинал летописи}, ты серёзно ? Я знаю, что это список или копия с
копии. И возраст её не старше 18 го века.

\iusr{Марина Байкова}

Да, да..
\iusr{Goch An}

Они так говорили общаясь меж собою КРЩНЬИ? А МТИ не склонялось получается Как
они себе языки не поломали?

\iusr{vasilijbyckov Василий Быков}

Есть и другая версия названия для городов "руских ( смола древесная)" , так
например, в древних летописях употреблялось слово "топо ( от По вкл. Альпы (
Алии ( кии-санскр.) , скифия - " Алипины сар ница")" и калькой будет являться
слово - мять - т.е. "топо" ( топать, топтать).

\iusr{Нурулла Мухамедзянов}

Давно уже известно. На сегодняшний день нет ни одного оригинального документа
отображающего историю средневековой Руси. Все на что ссылаются историографы,
включая различные летописи, являются лишь копиями сделанными неизвестно от
чего. Оригиналы сгорели вместе с библиотекой Ивана Грозного. Остатки документов
по средневековью, загадочным образом, были сожжены недавно в московском
книгощранилище. Так что вся история средневековой Руси является одним большим
глубокоангажированным вымыслом.

\iusr{Vladimir}

какое отношение имеет современная русская речь к древнерусским изречениям....
даже на современном украинском (русинском) языке слово мать - звучит в
определенных падежах как "маты" и слово иметь также "маты", но чтобы
получилось иметь в современном русском варианте опять же вернемся к украинскому
языку, который несколько ближе к древнерусскому, чтобы получилось в данной
фразе "иметь" или "мать" все будет зависеть от применяемого глагола ... если
глагол буди (будет- по -русски) - то мать, а если бы применялся вариант "будем"
- то "иметь" - соответственно говорится именно о матери .... изучайте
украинский язык, господа, который ближе к первоистоку ....

\begin{itemize} % {
\iusr{Aleksadr Tokhtash}
\textbf{Vladimir}, хохлодупый язык (мойва) тухлая смесь.
\end{itemize} % }

\iusr{Rod Jarrow}
Почему мать к названию Киев? А, еще и потому что слово \enquote{Столица} - женского рода.

\iusr{Rod Jarrow}

Совершенно верно: \enquote{мать городам русским}. - Мать не как родительница, а в качестве материнской опеки...

\iusr{Александр Латышев}

Здравствуйте. Спасибо. В условиях жёсткого противостояния современных РФ и
Украины, когда историю взяли в союзники обе заинтересованные стороны с массой
блогеров и их читатель по обе стороны кордона, вопрос приобрёл острое,
политическое значение. А зря! Пример. Раннее Средневековье, Франция- Австразия
и Нейстрия, Германия- Саксония и Франкония, Италия вообще до середины 19 века
расколота , продолжать.... Наличие относительно управляемых и политически
независимых княжеств в составе одного большого княжества это норма. А именно
Киевская Русь новодельный артефакт середины 19 века. Такого государства не
было, Великое Киевское княжество было, Суздальское было, Полоцкое было и так
далее, а все вместе просто - Русь! Как то так. Что тут копья ломать? Братьев с
\enquote{незалежной} понять можно, у них там новодел и срочно требуется биография с
\enquote{историей}, а нам то это зачем? Всем доброго здоровья и благополучия, удачи в
делах!

\begin{itemize} % {
\iusr{Rod Jarrow}
\textbf{Александр}, 

Киевскую Русь, также, как и Новгородскую очевидно для того и выдумали, что бы
создавать противостояние в наше время НОВОЙ феодальной раздробленности...

\iusr{krymvino krymvino}
\textbf{Александр}, 

Читайте летописи, изучайте историю по трудам ученых и Вы узнаете, что Русь
была. Мало того, была не только Русь, но так же была и Руськая земля.

Выкопировка из текста статьи российского историка первой половины XIX века
А.Федотова \enquote{О значении слова Русь в наших летописях} (Русский
исторический сборник. — М., 1837. — т. I, кн. 2., с. 104—121):

\ifcmt
  ig https://avatars.mds.yandex.net/get-zen_pictures/4475186/12174163-1634577538096/orig
  @width 0.4
\fi

\iusr{krymvino krymvino}
\textbf{Rod Jarrow}, 

Лаврентьевская летопись (считающаяся памятником владимиро-суздальского
происхождения) в 1175 г. (въ лѣто 6683) пишет о том, как собравшиеся после
смерти Андрея Боголюбского \enquote{лучшие люди} Суздальского княжества обсуждали
вопрос о наследнике княжеского стола - \enquote{сынокъ его въ Новѣгородѣ, а братья его
въ Руси}. И после обсуждения приняли решение: отправить послов \enquote{по князи наши
въ Русь}.

И таких примеров из летописей можно привести очень много

\ifcmt
  ig https://avatars.mds.yandex.net/get-zen_pictures/4104077/12174163-1634578641513/orig
  @width 0.4
\fi

\iusr{Александр Латышев}
\textbf{krymvino}, 

Здравствуйте. А Вы простите, что читали? Явно не мой комментарий. Там чётко
обозначено- Русь была, а Киевской Руси летописец не знает, новодел из середины
19 века. Что тут непонятного?

\iusr{krymvino krymvino}
\textbf{Александр}, очень даже внимательно читал.

Вот что Вы там пишите: \enquote{Великое Киевское княжество было, Суздальское было,
Полоцкое было и так далее, а все вместе просто - Русь! Как то так.} Но это не
так, поэтому я зделал выкопировки из текстов. Для Вас.

\iusr{Александр Латышев}
\textbf{krymvino}, 

А смысл? С Вами согласны, в чем проблема? Я не специалист по раннефеодальным
гособьединениям на Русской равнине. Но в 1984-85 проводил все сезоны на
раскопах Великого Новгорода, еще раньше Керчь-Херсонес и в 1986-87
Переславль-Залесский, в общем представление имею, общался с умными людьми -
академиком Яниным и профессором Зализняком и, главное с Саниным Геннадием
Александровичем, мой любимый препод по древнерусскому. Вроде, как мысли свои
могу изложить, а может старею - 60 лет мальчику! Доброго здоровья и
благополучия, удачи в делах!

\iusr{Rod Jarrow}
\textbf{krymvino}, Киевская (Ипатьевская) летопись дословно, время Олега:

"и заповѣда Ѡлегъ дати воємъ на кораблии по двѣнатьчать гривнѣ на ключь и по
том̑ даӕти оуглады (дань) на Рускіє городы . пѣрвоє на Кієвъ . таже и на
Черниговъ . и на Переӕславъ . и на Полътескъ (Полоцк). и на Ростовъ . и на
Любечь . и на прочаӕ город...

Увы, никак не получается что Русь это исключительно южные Киев, Чернигов и
Переяславль.... Потому что в словах Олега упомянут Русские города современные
Олегу: Полоцк (Современная Беларусь) и Ростов (Современная Ярославская область)
А, Ростов это точно Север, как не крути! Поэтому в Лаврентьевской летописи дано
указание на всю Русь, (Тот в Новгороде а Братья по всей Руси) а не
исключительно в Киеве, Чернигове, или Переславле... Понятно в чем спрятано
лукавство заявления что Русь это якобы исключительно Киев Чернигов Переяслав? -
Что, увы, сразу же разбивается о слова Олега Вещего, упомянувшего не только
белорусский Полоцк, но и Ростов, расположенный практический на одной Северной
широте с Новгородом.

Увы! И, Великое Киевское княжество, и Владимиро-Суздальское, и Полоцкое, и
Новгород все вместе просто - Русь! Это летописный факт! Факт Киевской
Летописи...

\iusr{нелепый абсурд}
\textbf{krymvino}, 

по поводу Руси и Новгорода, на мой взгляд никакого противоречия тут нет. Ведь
Лаврентьевская летопись описывает время, когда Новгород стал вольной
республикой.

Только после присоединения Новгородской республики в 1478 г. к Московскому
княжеству и окончательной ликвидации монгольского ига в 1480 г. можно говорить
о Руси, как едином государстве.

\end{itemize} % }

\iusr{Тульский мастер Лёвша}

\enquote{Се буди мти градомъ руским}, и что бы означала эта фраза? Почему Олег
провозглашает Киев матерью городам русским? Выдернув из общего понятия эту
фразу, ею манипулируют как хотят. А ларчик просто открывается; Дело в том, что
Рюриковичи, у словен ладожских и кривичей, чуди и муромы, прочая, прочая прочая.
союза племён, у того же Нестора, переведя на нынешний понятийный лад, являются
нанятыми(посадскими) не собственниками тем городам и весям. А тут не родовитые
бояре Аскольд и Дир, оттяпали у хазар мытное местечко, не доложив своему
сюзерену об этом, правят в городке. Это же вопиющее деяние, по тем временам!
Наказав сих бояр, Олег по праву собственника объявляет Киев, своей вотчиной
(собственностью), ибо город взят мечом, а не по договору найма.

\iusr{Voland}

говорил или нет, но он его сделал таковым! Всем альтернативщикам, которые за
Вышгород толкуют, смеюсь в лицо - ха-ха-ха!!!

\begin{itemize} % {
\iusr{сергей костин}
\textbf{Voland}, Человек ищет где лучше Прогнали хазар Святослав тот собирался на Дунае построить столицу для себя
\iusr{Voland}
\textbf{сергей костин}, совершенно верно!
\iusr{Dmitri}
\textbf{сергей костин}, в Переяславце.
\end{itemize} % }

\iusr{Алексей Котельников}

\enquote{Повесть временных лет} ни в коем случае не исторический документ. Это больше
литературное произведение. Кем был Нестор? Монахом Киевском лавры и писал он
то, что было нужно тогдашнему его начальству. А тогда шло противостояние между
Киевским и Московским центрами православия. И вообще, в ПВЛ столько нестыковок
по датам и событиям, что принимать сей опус за исторический документ просто
смешно.

\begin{itemize} % {
\iusr{krymvino krymvino}
\textbf{Алексей}, тогда не было московского центра православия. От слова "вообще".

\iusr{Алексей Котельников}
\textbf{krymvino krymvino}, так ведь ПВЛ была написана много позже описываемых в ней событий. Ты бы хоть немного изучил тему, прежде чем писать.

\iusr{Rod Jarrow}
\textbf{krymvino}, 

Когда все польские короли короновались и сидели в Кракове, увы, на месте
современной Польской столицы Варшавы ничего не было от слова "вообще...! Только
одни жабы квакали даже не польской "мовой ...и, что этот факт, теперь должен
означать лично для вас...? И в Швеции Стокгольм не первая столица Швеции также
как и Варшава в Польше - современная польская столица которая после трех
разделов Польши окажется по другую сторону Границы Между Российской Империей и
Австрийской на территории которой окажется Краков... Ну точно как у нас с
Киевом и Москвой, после оккупации Киева Ливтой! Также и в Японии Токио не
первая столица Японии, раньше столицей был Киото! В Китае за всю истории от
слов ВООБЩЕ аж четыре столицы поменялись... и что теперь вы будете делать от
того факта что все митрополиты Киевские еще в 1255 покинули Киев и ушли во
Владимир затем в 1325 уже из Владимира сели в Москву вот с этим митрополичьим
титулом ВСЕЯ РУСИ сели к Рюриковичам в Москву, которых в Киеве уже давно никого
не было! Киев завоевал Литовский оккупант Гедимин в 1324 году...превратив всю
завоеванную территорию в Великое княжество Литовское...увы, и его литовская
столица тоже была не в Киеве а в Вильно.

\iusr{Кот Сибирский}
\textbf{Алексей}, на месте Москвы во времена Нестора ещё болото с лягушками было.

\end{itemize} % }

\begin{itemize} % {


\iusr{krymvino krymvino}
\textbf{Алексей}, тогда не было московского центра православия. От слова "вообще".

\iusr{Алексей Котельников}
\textbf{krymvino krymvino}, так ведь ПВЛ была написана много позже описываемых в ней событий. Ты бы хоть немного изучил тему, прежде чем писать.

\iusr{Rod Jarrow}
\textbf{krymvino}, 

Когда все польские короли короновались и сидели в Кракове, увы, на месте
современной Польской столицы Варшавы ничего не было от слова "вообще...! Только
одни жабы квакали даже не польской "мовой ...и, что этот факт, теперь должен
означать лично для вас...? И в Швеции Стокгольм не первая столица Швеции также
как и Варшава в Польше - современная польская столица которая после трех
разделов Польши окажется по другую сторону Границы Между Российской Империей и
Австрийской на территории которой окажется Краков... Ну точно как у нас с
Киевом и Москвой, после оккупации Киева Ливтой! Также и в Японии Токио не
первая столица Японии, раньше столицей был Киото! В Китае за всю истории от
слов ВООБЩЕ аж четыре столицы поменялись... и что теперь вы будете делать от
того факта что все митрополиты Киевские еще в 1255 покинули Киев и ушли во
Владимир затем в 1325 уже из Владимира сели в Москву вот с этим митрополичьим
титулом ВСЕЯ РУСИ сели к Рюриковичам в Москву, которых в Киеве уже давно никого
не было! Киев завоевал Литовский оккупант Гедимин в 1324 году... превратив всю
завоеванную территорию в Великое княжество Литовское... увы, и его литовская
столица тоже была не в Киеве а в Вильно.

\iusr{Кот Сибирский}
\textbf{Алексей}, на месте Москвы во времена Нестора ещё болото с лягушками было.

\end{itemize} % }

\iusr{Никифорчин В.}

Ну сказал кто-то там что-то тысячу лет назад. Или не сказал наоборот. Что это
меняет. Факт в том, что Киев никакой матерью не стал. Зато не раз был обматерен
и разгромлен кем только не попадя. Позора насобирал полное лукошко. Как и
сейчас.

\iusr{Николай Сергеев}

Спасибо автору, однако сама фраза Олега означает, что Киев не единственный
город на Руси. И во-вторых она означает, что "матерью" другим городам Олег Киев
хочет "назначить". Т.е. Олег решил сделать Киев главным городом подвластных ему
земель. Т.е. это его личное желание, а не исторически сложившееся положение.
Это его волюнтаризм, если хотите. Т.е. эта фраза означает, что на главенство
или историческое право главенства над большой Русью имели другие города, но
Олег пожелал это положение изменить, навязать каким-то другим городам свою
волю. Из этой же фразы следует, что Киев таким не был. Он не был главным
городом по значению - что совершенно естественно, т.к. это был махонький
городок на фронтире. И своим возвышением он обязан Новгородским Рюриковичам,
которым вероятно просто понравился климат на этих территориях. А так же
ослабление политического соперника - Каганата, не допускавшего до поры в
междуречье Дона и Днепра кочевников. И Олег решил, что так будет теперь всегда.
Но Киев как был, так и остался городком фронтира (до разделов РП во второй
половине 18 века). И, спустя некоторое время уже обжитый Киев Рюриковичи
оставили - стрёмно и опасно жить на фронтире без торговой подпитки. Серьёзной
торговли по Днепру не было никогда. Клады с дихремами по берегам Днепра - не
следствие интенсивной торговли, а свидетельство опасности фронтира. Да и сами
дихремы поступали в тот Киев не с купцами, а посольствами или опосредованно,
через самих русов, т.е. получались где-то в других местах, на более оживлённых
торжищах. Ну, а Киев это типичный княжий городок. Князь избрал другое место и
городок захирел - за 100 лет до Батыя...

\begin{itemize} % {
\iusr{Денис Абасов}
\textbf{Николай}, 

Совершенно верно. А с приходом монгол Киев вообще выбросили как бесполезную и
ненужную больше вещь, это еще одно доказательство правоты ваших слов. И
Чернигов кстати тоже выбросили.

\iusr{Николай Сергеев}
\textbf{Денис}, 

да и вообще, то, что Рюриковичи переместились в Киев о чём говроит? Это говорит
о том, что со смертью Рюрика (или кого там не известно) "Рюриковичи" лишились
всего и были вынуждены искать себе пристанища. Почему-то это естественное
объяснение появления в Киеве Аскольда и Дира, а потом и Рюриковичей, историки
не видят в упор. Рюриковичей просто выгнали из Новгорода. Владимир "Святой"
потом, с помощью конкурента Новгорода Ростова просто завоевал Новгород -
"крещение"....

\iusr{Денис Абасов}
\textbf{Николай}, 

Кто это интересно лишил Рюриковичей всего? Кто их выгнал из Новгорода? Что за
глупости вы пишете. Даже странно что это пишет один и тот же человек по
сравнению с вашим первым постом. Такое резкое перевоплощение из разумности в
глупость.

\iusr{Николай Сергеев}
\textbf{Денис}, 

Видите ли, Владимир "Святой" покорил Новгород, с помощью ростовцев, которых
привёл Путята. Понимаете :о) ПОКОРИЛ СИЛОЙ. Потом это было названо "крещением".
Вообще сегодня люди не понимают, что такое была княжеская власть в раннем
русском средневековье. Сложные взаимоотношения между княжеским сословием и
волостями, т.е. вечевой традиционной властью. Князь должен был понравиться,
чем-то потрафить населению иначе карьера могла и не заладиться. Абсолютная
власть это потом, это Романовы, собственно это Пётр. А в те ранние времена
князья были заняты разборками между собой и сбором податей - на организацию
защиты подведомственных волостей. Что такое были те ранние "князья", видно из
факта их перескакиваний из городка в городок, ловко названное "лествичным"
наследованием. Но это не власть над территориями, это их внутрисемейные
разборки. Волость или городок могли князя, по каким-то причинам, не принять.
Своенравным был городок Киев. Это вовсе не столичное поведение. Таким же
своенравным был и Новгород. Да городки возили подать в Киев, но это так всюду
было устроено, это вообще было единственной формой подтверждения причастности
какой-либо территории к русской общности. Других просто не было. Если дают
деньги, значит свои. Перестали давать - отложились. Это не про княжескую
власть, это о понимании коллективной безопасности. И это совсем не так, как в
Европе устанавливались вассальные зависимости. На Руси не было вассалов и
сюзеренов. Кормления были, но появились гораздо позже киевского периода. Князь
исполнял определённые нужные волости функции администрирования и защиты.
Волость могла изгнать или призвать, и т.д. Собственно отложиться от Москвы
Новгород пытался даже в 17 веке - ждали шведов...

\iusr{Денис Абасов}
\textbf{Николай}, //Потом это было названо "крещением".//

Что значит было названо? Это и было насильственное крещение!

Владимира в Новгород посадил его отец Святослав, и обратите внимание что для
этого ему не пришлось покорять новгородцев. А уже когда сам Владимир княжил в
Киеве он предал языческую веру предков и принял христианство, в то время как
вся остальная Русь принимать христианство не собиралась, вот поэтому ему и
пришлось как вы говорите "покорять".

Владимир был по матери рабичич и помогал ему такой же рабичич Добрыня, брат его
матери. Рабичичи насильно навязали Руси чужую веру - вот что это было. Если бы
он не навязывал чужую веру, то новгородцы бы его приняли как законного князя, а
так им пришлось сопротивляться, с одной стороны они видели в нем законного
князя, а с другой предателя.

Шведы, то есть варяги свеи были бизкородственными варягам русь, как бы сейчас
сказали тогда в 11 веке это был один народ. И жаль что шведы тогда не
вмешались, они обязаны были объявить Владимира предателем и тогда бы история
Руси пошла бы совсем по другому сценарию. Весь этот раздрай на Руси начался
именно с появлением христианства.

\end{itemize} % }

\iusr{Денис Абасов}

Ну нельзя же вот так тупо не учитывать контекста фразы. Мать - это женщина
родившая ребенка. Кого родил Киев? Ну это же несуразица какая-то.

У одного и того же слова может быть несколько разных значений в зависимости от
контекста. Например у слова функция 5 разных значений в словаре, это могут быть
ваши служебные функции на работе, а может быть переменная величина в
математике, все зависит от того в каком контексте вы применяете это слово.
Точно также у слова \enquote{мти} вполне могло быть несколько разных значений. Понятно
же что значение \enquote{мать} применительно к городу не подходит, значит здесь
какое-то другое значение.

Я думаю во фразе Олега слово мти означает - крепость. Киев был тогда на самой
южной границе Руси. Южнее там уже дикое поле начиналось с кочевыми народами. Ну
какой дурак будет делать столицу государства в таком опасном месте на самой
границе? А вот крепость сделать на границе вполне разумно.

Фразу Олега следуте читать так:

\enquote{Это будет русским городом крепостью.}

К тому же сами подумайте, если бы Киев был столицей Руси или матерью, неужели
бы русские так легко отдали его монголам? Ведь за Киев вообще никто даже не
сражался! Так же как и за Чернигов кстати. Значит эти города не были чем-то
ценным для русских.

\begin{itemize} % {
\iusr{галина карандашова}
\textbf{Денис}, киев - владимир- куйва
\end{itemize} % }

\iusr{ПНД}
Киев сделали \enquote{матерью городов русских} в 19 веке.

\begin{itemize} % {
\iusr{Voland}
\textbf{ПНД}, есть доказательства?

\iusr{Станис Лав}
\textbf{Voland}, красавчик!

\iusr{ПНД}
\textbf{Voland}, история написана в 18-19 веках. А Киев взят за основу из-за христианства, так как именно оттуда оно пошло по Руси. А во всякие Вышгорода я тоже не верю.

\iusr{Игорь}
\textbf{ПНД}, Киев захотел прирасти к России, как сейчас к Америке.
\end{itemize} % }

\iusr{Олег Иванович}

Это ещё раз подтверждает то, что эти летописи писали немцы, когда выдумывали
нам историю. Это в немецком языке город женского рода - die Stadt и для них это
звучит вполне естественно Киев мать городов русских, ни одному русскому никогда
не придёт в голову сказать город мать или Киев мать (Ростов мать, Владимир
мать, Новгород мать...), но историки ведутся на этот вымысел.

\iusr{pol 24960}

==="сѣде Олегъ княжа въ Києвѣ и рек Олегъ се буди мти градомъ рускими"==

Если учесть то, что Олег, завоевав территории и стал княжить в Киеве, тогда он
сказал, что этот город будет не "матерью городов русских", а стольным городом,
столицей Руси, столицей государства, котрое он создал.

И не надо про МАТЬ-мужчину Киев, которая выродила города русские.

\iusr{Иван Расстригин}

Как говорится, лучше горькая правда, чем сладкая ложь.

Также, как и ПВЛ, Лаврентьевская летопись была сфальсифицирована писарями под
страхом смерти.

Слава и Светлая Память этим писарям, которые смогли придумать, как оповестить
грядущие поколения о лжи. Скорее всего, на полях писали какой-то жидкостью,
которая со временем стала походить на водяные знаки. Это могло быть простое
молоко.

Читаем: "ПОД СТРАХОМ СМЕРТИ ЗАСТАВИЛ ПИСАТЬ УРЯДНИК МИТЬКА. Б...... Ь." ЛОЖЬ,
ЛОЖЬ, ЛОЖЬ...

Вся придуманная на западе история Руси трещит по швам. И скоро мы узнаем
правду. Все артефакты подписаны и все годы по славянскому календарю указаны.

Например, по надписям на саркофаге Ярослава Мудрого мы знаем, что он умер на
210 лет раньше, чем общепринятая сегодня версия. Это ломает всю генеалогию
Руси, начиная с Рюрика...

Попытка омолодить Русь и сделать из нее самую молодую цивилизацию, построенную
на "более ранних" западных ценностях - провалилась...

\url{https://vk.com/geoglyph?w=wall44671505_414%2Fall}

\ifcmt
  ig https://avatars.mds.yandex.net/get-zen_pictures/4104077/37287217-1635666879756/desktop-2x
  @width 0.4
\fi

\iusr{игорь буев}

Говорил ли Олег эту фразу одному богу известно, ведь летопись писалась через 300
лет после описываемых событий. Как говорит Данилевский Никон скорее всего взял
эту фразу откуда нибудь. Так же нет данных, в Византийских документах, о крещении
Ольги в Византии.

\iusr{Vladik Nelidov}

Как мог князь Олег говорить такое, если незадолго до него впервые увидевшие
Киев русы во главе с Аскольдом и Диром задали вопрос:" А что это за город?" Сам
этот вопрос древних русов, уже предполагает, что русы к нему точно никакого
отношения не имели.

\iusr{Daniel Socolov}

Смысл изменчив не только от слов, но и их интерпретаций. А метафоры вроде \enquote{се
буди мти градомъ рускими} дают максимальный простор для фантазий. Возможных
интерпретаций тысячи. Например, с чего решили, что мать всем градам русским, а
не градам русским одного приграничного региона?! С чего решили, что \enquote{мать} это
самый верхний уровень иерархии, а не подчиненный, например, \enquote{отцу}?! С чего
роль \enquote{матери городов} это центр управления, а не, например, кормовая база для
других центров?! С чего \enquote{се буди} это вообще про место, а не акт его захвата
или вовсе про собственное правление?! Гипотеза на гипотезе.

Читать исторические документы надо осторожно, не примешивая собственных
фантазий к исходному тексту. И если смысл какой-то фразы интерпретируется
неоднозначно, она попросту не ясна. И закладывать такие фразочки фундаментом
каких-либо выводов нельзя. Хотя и можно сопоставлять с другими сомнительными
свидетельствами для формирования общего контекста.

\iusr{Артем Антонов}

\enquote{... смысл его фразы \enquote{Се буди мти градомъ руским} заключается именно, что будет
Киев не просто городом русским, но матерью городов русских.}

А о каких, собственно говоря, \enquote{городах русских} идёт речь?

Новгород и в 13 веке Русью не считался. Смоленск? Очень сильно сомневаюсь.
Любеч? Через сто с лишним лет после Олега сначала Равноапостольный с Ярополком,
а затем Мудрый с Окаянным именно под Любечем бились. Получается, что именно
здесь, скорее всего, и проходила северная граница \enquote{русской земли}.

Так где же они, эти самые \enquote{города русские}, о которых Олег толковал в 882 году?

\begin{itemize} % {
\iusr{Сторонний наблюдатель}
\textbf{Артем}, не ищите тайный смысл, кроме написанного. Что написано, то написано. Остальное ваши вымыслы.

\iusr{Артем Антонов}
\textbf{Сторонний наблюдатель}, так о каких городах говорил Вещий? Уточните.
\end{itemize} % }

\iusr{Кот Сибирский}

Что летописи, что Библия, что саги, вообще любые старинные тексты часто
используют один и тот же прием - результат со всеми подробностями ставят в
намерение. Не пишут - он подбежал, проткнул его мечом в печень, потом отрубил
голову. Вместо этого напишут - он сказал: "Я подбегу, проткну ему мечом печень,
потом отрублю голову" - и так и сделал." Т.е. летописные слова Олега не
означают, что сам Олег что-то такое говорил, он может просто матерно ругался на
скандинавском. Но результат его действий был таков - Киев стал метрополией
(дословно - мать городам) русских городов. А значит летописец легко записал
этот результат в слова Олега, согласно \enquote{нормам} тогдашней литературы.

Это же касается и призыва Рюрика: \enquote{Приди и владей...}

\iusr{евгений иванов}

Следует учитывать, что ПВЛ не просто летопись, бесстрастная объективная хроника
событий. Это апология нового русского государства, у которой три автора (Никон,
Нестор, Сильвестр) и три заказчика - киевские князья. Отсюда крайняя
субъективность документа и наличие его тексте различных легенд и басен. Одна из
таких басен - наличие вещего Олега (то ли князь, то ли родственник князей, то
ли их боярин). Само имя (хельг значило святой) не могло принадлежать данному
персонажу, т. к. он не был крещеным. Такое имя давалось у варягов (скандинавов)
только князьям, принявшим христианство. А вот княгиня Ольга была крещеной
болгарской княжной и возможно была перекрещена (греческим обрядом) самим
Константином Багрянородным. Учитывая произвольность хронологической сетки ПВЛ,
проблемы с датами рождения и возрастами первых русских князей, следует
признать, что Киев был занят не безвестным Олегом, а княгиней Ольгой, вдовой
князя Игоря Старого по прозвищу Рюрик, бежавшей с малолетним Игорем, позднее
ставшим известным как Святослав, из Новгорода после бунта древлян (лесных
людей, обитавших по реке Мста). Потому и стала Олька, севшая в Киеве, мать
городам русским.

\iusr{Арчибальд Иванов}

\enquote{будет Киев не просто городом русским, но матерью городов русских}

матерями являют кошки и самки других животных. А если переложить миф про Рим,
то получается Киев - это типа волчица городов русских. Мать - это и член
домохозяйства, а если муж подкаблучник, то глава домохозяйства. То есть у
метафоры \enquote{матерь городов русских} десятки значений.

\iusr{ган}

Надо разобраться с нарицательностью жительниц данного места, а точнее с
подолянкой соответствующее наличию Подола Материнска Плата донского база
засечной чертой левобережья Днепра.

Всё остальное к западу увы! заподлянка!

\iusr{Владимир Сарбаев}

Древний Киев или Куяб - действительно "Мать городов Русских". Только речь в
летописи идёт о совершенно другом Киеве. Это Иерусалим, Троя, Иерихон,
Царьград, Константинополь... Не случайно Фёдор Тютчев в стихотворении "Русская
география" назвал его наряду с Москвой и Петербургом заветной столицей Руси. Но
у Руси есть ещё и Батя. Это древний Ростов-Папа. Он и дешифруется с
Древнерусского как РОС+ СТ(еСТь)+ Бат(читается в зеркальном прочтении с учётом
переходящих согласных). Кстати, именно в древнем Ростове Великом родилась
Богородица.

О другой не западной версии истории на сайте slovobog.su

\iusr{Гена}

Киев это Хазарский город и переводится, как город зятя. Хазары активно
переселяли Славян, которые были в рабстве в Европейских государствах, на свои
земли на Дон, Днепр и Волгу. Скорее всего и князья Киевские были Хазарами,
которые со временем потеряли свой родной язык и перешли на славянский. Раз Киев
назвали матерью городов Русских, значит был и отец, который стоял над ним.
Скорее всего это столица Хазарии Семендер.

\begin{itemize} % {
\iusr{Rod Jarrow}
\textbf{Гена}, Славу Дурдому! Дурдому Слава!

\iusr{Гена}
\textbf{Rod Jarrow}, Твоему дурдому слава это точно. Ты с какого?

\iusr{Rod Jarrow}
\textbf{Гена}, 

Точно не с вашего персонального... Напоминаю столица хазарского каганата Саркел
- в переводе с хазарского буквально "Белый Дом" - название ничего не
напоминает? При том факте, что власть Хазарская это власть иудейская.

\iusr{Гена}
\textbf{Rod Jarrow}, 

У Хазарии было несколько столиц, первая Семендер(ныне Тарки). Хазария, как
империя была многоконфессиональным государством, где иудеи были религиозным
меньшинством. САРКЕЛЬ кто то, кто не знает языка может перевести, как хочет,
а переводится он правильно МЕСТО, ГДЕ ЖИВЕТ ЦАРЬ ВМЕСТЕ СО СВОЕЙ РОДНЕЙ(РОДОМ)

\iusr{Дмитрий}
\textbf{Гена}, 

\enquote{Скорее всего и князья Киевские были Хазарами, которые со временем потеряли
свой родной язык и перешли на славянский.} - ни один киевский князь в Киеве не
родился. Киев как переходящий приз ходил по рукам.

\end{itemize} % }

\iusr{Николай Попов}

Если исходить из контекста летописи, то получится, что Киев был захвачен
новгородской дружиной и, по логике событий, права старшего города должны были
перейти к Новгороду. Однако этого не произошло. Олег подтвердил (именно так!)
столичный статус Киева, которым этот город уже обладал при Аскольде и Дире.

Как-то совсем упускается из виду, что Олег захватывает не только Киев, но и
возглавляемое им до 882 года государство Русь. "И рече Олегъ: "Се буди мати
градомъ русьскимъ". И беша у него варязи и словени и прочи, прозвашася русью".
То есть разноплеменная дружина Олега принимает имя руси, оседая в Киеве. Не
захватчики передают своё имя государству (общего для "варягов, словен и прочих"
имени ещё просто не существует), а прославленное в 860 году (по летописи - в
866) государство даёт имя захватчикам - они теперь тоже русь!

Вместе с этим именем завоеватели принимают на себя и статус аристократии, что
показывает размещённое под 907 годом предание: "И рече Олегъ: "Исшейте парусы
паволочиты руси (киевлянам - Н.П.), а словеномъ (новгородцам - Н.П.)
кропиньныя", и бысть тако...И воспяша русь парусы паволочиты, а словени
кропинны, и раздра а ветръ. И реша словени: "Имемся своим толстинам, не даны
суть словеном пре паволочиты". В несколько завуалированной форме летописец
сообщает нам - сам Бог так заповедал, что сколько раз Новгород ни будет
завоёвывать Киев, но главным городом в государстве ему не быть и всегда
придётся довольствоваться "своими толстинами".

\iusr{Igor Demidov}

тяжело, дружок сову на глобус натягивать, коли не ведаешь русского языка... А
любой русский тебе скажет, что "мти" = "имати", что означает иметь, владеть...

\iusr{Дмитрий Григорьев}

Думаю вы правильно всё разжевали, но всё же кое что упустили.

Именно из оборота слов, возможно что мать это столица.

А не рожаница.

Раз сел он княжить в Киеве то Киев стал столицей - матерью городам русским а не
основателем их.

\iusr{Valetin Rudenko}

Вот почитал этого засекреченного автора и сразу можно сказать что он пытается
выдать себя за всезнайку и при том любителя приписывать себе и стране как
националисту того что не было и быть не могло. Приписывание себе необоснованные
веще и необоснованность эрудиции по-видимому у него является генетической
наследственность дремучих племен средне-русской равнины, Поэтому патология
находится на таком уровне что ее излечить нельзя а только поставить для
изучения в лабораторных условиях медицинских ВУЗОВ,

\iusr{Aleks Prots}

В оригинальной фразе Олега русского человека: "Се буди пи-пи-пи мти пи-пи-пи
градомъ руским" монахи немного подправили на более приличное без пи-пи.

\iusr{Андрей Волков}

Автор как только объяснишь нам, почему КИЕВ -ОНА. Так и поверим.

Ты лингвист? Филолог? Можешь утверждать, что "имать" и "мать" не могли быть
написаны одинаково, но в нести в контексте разный смысл?

Утверждение "быть Киеву городом русским", полностью ложится в канву
исторических событий.

если ты не в курсе, то есть мнение, которого придерживаются в том числе и
американские и израильские ученые, что Киев-это хазарская крепость Куяба.

И вот тогда фраза Олега действительно будет иметь смысл. Он захватывает
хазарскую крепость и делает ее русским городом.

И именно захват Киева (Куябы) по всей видимости и явилмся причиной начала войны
между русами и хазарами.

Так что не надо тут безальтернативных утверждений.

То, что начиная с 19 века правители великороссов заигрывали с рагулями, (а уж
во времена СССР, это вообще расцвело махровым цветом) и писали им Великую
историю, утаивая и умалчивая очевидные вещи от общественности, не делает эту
историю правдивой..

Слишком много хвалили хохлов.. нахвалили это отребье до того, что они сами в
это поверили. И теперь свысока смотрят на русских дурачков., считая
второсортным быдлом.

Может хватит уже????

\begin{itemize} % {
\iusr{Неальтернативная история}

Андрей, вы рассматриваете древнерусскую летопись с точки зрения современные
грамматики. Вы сами понимаете несостоятельность такого анализа?

Или вы считаете, что 800 лет назад люди говорили по тем же правилам что и
сейчас?

Прекрасно. Обьясните "у грѣховънаго бо корени золъ плодъ бываетъ понеже бѣ была
мти его черницею а второе Володимеръ залежею не по браку прелюбодѣи" как
Владимир имел в прелюбодеянии черницу, которая имела место быть иметь его
черницей, но вдруг родила сына, которому она просто имелась.

Объясните!

Ну не может одно и то же слово использоваться в разных смыслах.

\iusr{Rod Jarrow}
\textbf{Андрей}, 

А вас не смущает ОТЧИЗНА - ОНА произведенная от мужского ОТЕЦ - ОН? Может уже
хватит Туфту нести про летописи, даже одного грамма не понимая
церковно-славянский...

\end{itemize} % }

\iusr{Александр Позин}

Притянул за уши. И город и Киев - мужского рода. Горд не может быть матерью, а
отчиной и дединой может.

\iusr{Valentin}

Автор это ты высасываещь удобный тебе смысл! Потому что не знаешь древнюю речь.
А все очень просто.. И нет никаких других ошибок в построении предложения!! Учи
матчасть."Се буди мти градомъ руским" или - "будем маты градом русским". БУДЕМ
ИМЕТЬ ГОРОД РУССКИМ!! ( "мти" - МАТИ - совр. = Мать / Иметь).

\iusr{Александр Дюрчек}

Как Рось Корсунь Правил.

(Цари Рек.)

.....

Как Велик да Белый Царь

Рось Свет Днепрович

Да взошёл на края Брега Каменна

Да как чуял кору то Великой земли

Да засунул кору себе в руки то правые

Так Корсунские земли осваивал

Как кору, что от Дуба никому не отсечь

Рось сам Царь да копил здесь велики полки

Те полки да освоены Ро́дами

Да те Ро́ды водили во велики то тьмы

Да б Славяне впредь правили всеми народами

Да Дединец на Рось то реке был на правом брегу

На неправом всё было за правыми

Подымал Рось то Царь, коле, праву руку

Знала Рось, что зовёт браной Славою

Подымал Рось то Царь да неправу руку

Все, кто Росы, знали, – неправы мы

Поправляли себя да во Праву Руку

Чтобы снова мы были все Правильны

Права длань как то Права сПравление

От Дидов до Детей утешение.

То есть Дар, да само то Рось Царя

Самого Рось Велико Днепровича.

(Кладь Голяков Всенародного чина Славян)

....

Вот как было. Подымал Праву руку – «Всё. Добро.» Подымал Боярам своим неПраву
(левую) руку – «Не к Добру». И помощники, для того готовые, без устали держали
его подняту неправу руку, пока неправильные события не поправлялись. Вот как
правили. КорСунь. «Сунь» образовалось от слова «тело», потому «КорСунь»
означает «Кора Земли». И в сей былине сказано, что «ежели середина дуба Киев,
то Кора – Корсунь. Во всяком случае понятно, что именно поколение росьЦаря
именно продвинуло поколение Днепра Доновича. Это не Кий, и не его брат Хорив, с
Лыбедью – сестрой Кия. Нет. Дединец и град существовал под названием Днипр. До
Кия только через поколение Ядрёного Корня – Царя Славена, его трёх Детей –
Червня, Велико и Бела, то есть ЧервоноРоса, ВеликоРоса и БелоРоса народились
уже цари поместных званий, каким, например и есть Киев. По древнему – Молот.
Хотя я, как носитель Славянского Предания не исключаю, что то просто звание
Днепра Доновича. Ведь у Днепра Доновича очень похожее прозвище – КуВалда. Что
означает – «КуйВолдай, – Куй Волка и дай нам его» - Молот Ярости. А Киян – это
так же – Молот. Как знать, может в поколениях это так и было всё сведено. Вот
же как.

(Народный Государь жрец Владимир Богомил второй Голяк)

Памятник Росичу на острове что во реке Рось.

\ifcmt
  ig https://avatars.mds.yandex.net/get-zen_pictures/4104077/1482660524-1636300747906/orig
  @width 0.4
\fi

\iusr{Rod Jarrow}

\textbf{Неизвестный Доброжелатель}, Серьезно вы в каком писании нашли слово конунг....?
При этом понятия не имя, что варяг и скандинав это не синонимы в летописи!
Летопись "Афетово колено" (Список народов Европы), увы имя варяги стоит
отдельно от всех скандинавов вместе взятых, в том числе и отдельно от Руси:

Афетово же колѣно и то Варѧзи . Свеи (шведы). Оурманє (норвежцы). Готѣ(остров
Готланд на Балтике) . Русь . Аглѧнѣ(датчане) . Галичанѣ . Волохове . Римлѧнѣ .
Нѣмци . Корлѧзи . Венедици . Фрѧговѣ . и прочии...

\iusr{Неизвестный Доброжелатель}
\textbf{Rod Jarrow},

\enquote{Каган} будет более в духе времени. А вот Вы, батенька, используете
понятие \enquote{скандинав}. Поинтересуйтесь, когда оно появилось.

\iusr{Rod Jarrow}
\textbf{Неизвестный Доброжелатель}, 

А, вы, батенька демагог! Уклоняетесь от главного, что в летописном перечислении
народов Европы по списку "Афетово колѣно" все ваши: Свеи(шведы),
Оурманє(норвежцы), Готѣ(Готландцы) и Аглѧнѣ(датчане) 100\% поголовно
(СКАНДИНАВЫ), все вместе нервно курят в сторонке (ОТДЕЛЬНО !), от имени Варяги!
Увы! А, это означает, что для летописца это точно не синонимы, в отличии от
всех наших современных писателей-фантастов, вешателей лапши на уши лохам про
Сурового Скандинава Викинга Олега Конунга!

\iusr{Rod Jarrow}
\textbf{Неизвестный Доброжелатель}, 

Через 200 лет, какая может быть целесообразность в том что бы христианину
пропагандировать иноземную оккупацию, да еще и языческую..? Увы, ерунду вы
пищите о целесообразности

\end{itemize} % }
