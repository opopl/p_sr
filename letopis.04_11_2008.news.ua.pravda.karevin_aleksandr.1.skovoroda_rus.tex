% vim: keymap=russian-jcukenwin
%%beginhead 
 
%%file 04_11_2008.news.ua.pravda.karevin_aleksandr.1.skovoroda_rus
%%parent 04_11_2008
 
%%url https://www.pravda.com.ua/rus/articles/2008/11/4/4454233/
 
%%author Каревин, Александр
%%author_id karevin_aleksandr
%%author_url 
 
%%tags skovoroda_grigorii,russia,ukraine
%%title О Григории Сковороде, языковой проблеме и недобросовестной полемике
 
%%endhead 
 
\subsection{О Григории Сковороде, языковой проблеме и недобросовестной полемике}
\label{sec:04_11_2008.news.ua.pravda.karevin_aleksandr.1.skovoroda_rus}
\Purl{https://www.pravda.com.ua/rus/articles/2008/11/4/4454233/}
\ifcmt
	author_begin
   author_id karevin_aleksandr
	author_end
\fi

Общеизвестно, что истина рождается в споре. Хорошо, когда полемика ведется
честно, каждая из сторон прямо излагает свою позицию, приводит аргументы и в их
противоборстве выясняется, кто же прав.

Гораздо хуже, если участники спора не заинтересованы в установлении истины.
Такие спорщики обычно используют недобросовестные методы: подменяют факты
выдумкой, пускают в ход сфальсифицированные цитаты и данные, оперируют
фальшивыми доводами.

К сожалению, случаи недобросовестной полемики сегодня не редкость. Одним из них
является, на мой взгляд, статья Владимира Гонского \enquote{Про традиціоналізм,
прагматизм і національну ідею}.

Касаясь путей решения языкового вопроса в Украине, Гонский заявляет: \enquote{Суть
розв’язання мовної проблеми підказав ще Сковорода: \enquote{Кожен повинен пізнати свій
народ і в народі себе. Якщо ти українець, будь ним... Ти француз? Будь
французом. Татарин? Татарствуй! Все добре на своєму місці і своєю мірою. І все
прекрасно, що чисте, природне, тобто непідробне}}.

Далее, в развитие темы следуют собственные рассуждения автора о том, что
украинцы должны не просто \enquote{числиться} таковыми, но и быть в действительности
(перейти на украинский язык), а \enquote{русифікація українців} – это \enquote{неприродний,
шкідливий процес}.

Я не разделяю негативного отношения к русскоязычности своих соотечественников.
Доминирует же в Австрии немецкий язык, а на Кипре – греческий, но развитию этих
государств данное обстоятельство нисколько не мешает.

Официальное закрепление в различных регионах Швейцарии первенствующего
положения французского, немецкого или итальянского языков не привело к распаду
страны.

Несмотря на общий язык, отдельно существуют Сербия, Хорватия и Черногория. Чем
же плоха русскоязычность значительной части населения Украины?

Но, в конце концов, Гонский имеет право на собственную точку зрения. А вот на
что он, по моему мнению, права (морального) не имеет, так это на подтасовку, с
помощью которой пытается подкрепить свое мнение авторитетом выдающегося
философа.

Дело в том, что приведенная выше цитата из Сковороды не верна. На самом деле
знаменитый мыслитель писал: \enquote{Всякий должен узнать свой народ и в народе себя.
Русь ли ты? Будь ею: верь православно, служи царице право, люби братию
\enquote{нравно}. Лях ли ты? Лях будь. Немец ли ты? Немечествуй. Француз ли? Французуй.
Татарин ли? Татарствуй. Все хорошо на своем месте…}

И далее: \enquote{Русь не русская видится мне диковинкою, как если бы родился человек с
рыбьим хвостом или с собачьей головою}.

Понятно, что среди читателей УП с произведениями Сковороды хорошо знакомы
немногие. На что, вероятно, автор и рассчитывал. Он подменил национальное имя,
а заодно сделал \enquote{купюру}, убрав из цитируемой фразы призыв к верной службе
Екатерине II (в перечень признаков \enquote{справжнього українця} такой призыв явно не
вписывался).

Между тем, сам Сковорода Украиной называл территорию Слобожанщины, отличая ее
от Малороссии - Полтавщины и Черниговщины с Киевом.

Под Русью же он подразумевал эти земли вместе с Великороссией. Соответственно,
\enquote{Русью будь!} в устах Сковороды означало единение украинцев (малорусов) с
великорусами, а не отмежевание от них.

Тут, действительно, содержится подсказка к правильному решению языкового
вопроса. Ибо русский литературный язык изначально развивался, как язык
общерусский, общий для всех частей исторической Руси. Для Украины в том числе.

Вклад украинцев в развитие этого языка огромен. Для людей образованных он был
родным. Необразованные же люди на Руси разговаривали на своих местных
просторечиях, которые, впрочем, наряду с литературной речью, считались
разновидностями одного русского языка.

К примеру, во время первой переписи населения Киева и его пригородов в 1874
году 38\% горожан назвали родным русский литературный язык, 30,2\% – малорусское
наречие, 7,6\% – великорусское наречие, 1,1\% – белорусское.

Одним словом, русский язык в Украине был своим. И остается таковым до сих пор,
хотя со второй половины ХІХ века группа политиков из украинофильского движения
принялась вырабатывать другой литературный язык, названный потом украинским.

Русскоязычность в Украине – вовсе не плод русификации. Это результат
нормального, естественного развития. Если же русский язык мешает торжеству
того, что именуется \enquote{украинской национальной идеей}, значит с самой этой
\enquote{идеей} что-то не так.

Ведь, кажется, все согласны с тем, что сущностью национальной идеи любой страны
является стремление человека и нации быть лучшими. Стала ли лучше Украина
оттого, что миллионы ее русскоязычных жителей оказались в положении граждан
второго сорта?

Стали ли лучше украинские школы, в которых уже не изучают русский язык? Стала
ли богаче украинская культура, отказавшись от Гоголя, Короленко, Ахматовой
(Горенко), других писателей, объявленных ныне иностранными потому, что творили
на русском языке?

Все это вопросы риторические. Желать вытеснения из Украины русского языка
означает стремиться к тому, чтобы эта часть Руси перестала быть Русью. Станет
ли она от этого лучше? Григорий Сковорода дал на это исчерпывающий ответ.

Александр Каревин, для УП
\ifcmt
pic https://img.pravda.com/images/doc/r/q/rq_Picture_file_path_10035.jpg
width 0.1
fig_env wrapfigure
\fi
