% vim: keymap=russian-jcukenwin
%%beginhead 
 
%%file 04_09_2019.fb.vishtaljuk_jaroslav.1.ilovajsk_muzhenko_interview
%%parent 04_09_2019
 
%%url https://www.facebook.com/VJaroslaw/posts/2977728178964325
 
%%author_id vishtaljuk_jaroslav
%%date 
 
%%tags donbass,ilovajsk,interview,ukraina,vojna
%%title Журналісти "Донбас. Реалії" - интервью - Муженко - Иловайск
 
%%endhead 
 
\subsection{Журналісти \enquote{Донбас. Реалії} - интервью - Муженко - Иловайск}
\label{sec:04_09_2019.fb.vishtaljuk_jaroslav.1.ilovajsk_muzhenko_interview}
 
\Purl{https://www.facebook.com/VJaroslaw/posts/2977728178964325}
\ifcmt
 author_begin
   author_id vishtaljuk_jaroslav
 author_end
\fi

\href{https://www.radiosvoboda.org/a/30142355.html}{%
Муженко про помилки Іловайської операції, рішення Хомчака і обіцянки Путіна, radiosvoboda.org, 03.09.2019%
}

Журналісти "Донбас. Реалії" взяли напрочуд "зубасте" інтерв'ю по Іловайській трагедії у скотини Муженка. Попри можливість врятувати свого багатолітнього "патрона", зіпхнувши всю провину на АТЦ СБУ і Грицака, який юридично відповідає за дії під час АТО, колишній риг професійно здав Порошенка з усіма тельбухами.
 =============================
– В одному зі своїх інтерв’ю ви сказали, що вторгнення на територію України регулярних російських частин для вас було несподіванкою. Можете пояснити – чому? Адже було очевидно, що Росія всіляко підтримує незаконні збройні формування.
– Є ж відповідні правила і закони, по яких ведеться війна. Міжнародне гуманітарне право чітко визначає ці процедури. Повинно було бути якесь оголошення, якась нота, попередження і таке інше. А тут введені війська з території суміжної держави без будь-яких попереджень, без будь-яких знаків розпізнавання. Тобто вони не позиціонували себе, як регулярні Збройні сили Російської Федерації.
– Тобто ви чекали від Росії виконання міжнародного права?
– Взагалі вона ж повинна, якщо провідне місце займає…
– А хіба попередні конфлікти за участі Росії не показали, що вона не дотримується міжнародного законодавства?
– Показали, до речі, ми передбачали такий варіант. І розглядали можливі варіанти, але ми не думали, що це виконається в такий спосіб. Тому що проводилася низка заходів, в тому числі і в тилу, в питаннях створення військових частин, нових організмів, формування і таке інше. Ми знаємо, в якому стані перебувала українська армія на початку 2014 року. Армія як держаний інститут існувала, але питання її боєздатності і спроможностей було під великим знаком запитання.
– Ви зараз не вважаєте, що було недооцінення такого розвитку подій саме з вашого боку?
– Можливо, так. Ми десь передбачали, тому що одна зі спроб (вторгнення Росії – ред.) була ще в ніч із 12 на 13 липня. Мені доповів один із командирів, який знаходився на одній із висот в районі Ізвариного, що на територію України заходить велика колона – близько кількох сотень машин. Ми нанесли туди вогневе ураження: артилерією, ракетами і таке інше. Три дні відповідна площа – поля, які були засіяні – горіла. Ми побачили 15 залишків-каркасів відповідних автомобілів. І в мене до сьогоднішнього дня – запитання: чи це була колона, чи це було неправильне сприйняття (ситуації – ред.) тими офіцерами і військовослужбовцями, які доповідали. Тобто такий факт мав місце. Ми вважали, якщо це дійсно колона російських регулярних Збройних сил, то це вже означало початок війни. Але нічого не підтвердилося, і ми вважали, що Росія на такий крок не піде. Так, поставки боєприпасів, так, поставки зброї, так, якийсь спецназ, відпускники і таке інше. Але щоб завести регулярну армію, як штатним підрозділом…
– Зараз дуже багато критики в бік і Генштабу, і військово-політичного командування країни лунає у зв’язку з проведенням у той час військового параду в Києві. Як ви зараз ставитесь до цієї події? Чи не вважаєте ви, що це було не зовсім доречно?
– Я був, в принципі, проти проведення цього параду. Але почалася підготовка і вже за кілька днів до визначеної дати на мене вийшов президент Порошенко і запитав: чи є необхідність проведення параду? Я сказав, що раз ми вже пройшли підготовку, і озвучили про те, що такий парад має бути, то відмовитись від нього за кілька днів до його проведення – це покаже наш страх перед росіянами і таке інше. А ми повинні були продемонструвати волю до опору.
– Якби у вас була можливість це рішення ухвалювати, ви б наполягали на параді?
– Якби зараз? Розуміючи оцінки того параду, які прозвучали в період з 2014 по 2019 роки, я думаю, що, мабуть, це було помилкою. Можливо, це якраз і було однією з причин, чому росіяни все-таки вирішили ввести війська саме 24 серпня на територію України.
– У 2016 році ви говорили про те, що пересвідчились остаточно в тому, що росіяни перетнули український кордон 26 серпня. Хоча ми тепер знаємо, що це відбулось із 23 на 24 серпня.
– Я отримав першу інформацію про те, що це, можливо, російські регулярні війська 24-го числа вже по обіді з доповіді полковника Ремигайла. Він сам особисто бачив. Він казав – так, за всіма ознаками це колона регулярних збройних сил, виходячи з кількості техніки, стану цієї техніки і таке інше. Але каже: вона без знаків розпізнавання під прапорами «ДНР». Тобто у нього були сумніви, чи це дійсно колона (російської армії – ред.), чи якесь нове формування, яке було створене за допомогою Росії і з числа цих бандформувань. І вже 26 серпня, коли ми взяли в полон 10 десантників, було чітко зрозуміло і підтверджено, що це росіяни і дійсно регулярні війська.
– Хомчак раніше в інтерв’ю Радіо Свобода сказав, що наказу про вихід із Іловайська чекав не один день. Не дочекався його 29-го числа і сам ухвалив рішення виходити. Це правда?
– Ситуація яка була: 27 серпня було Путін заявив про те, що вони гарантують нам «зелений коридор» для виходу наших підрозділів з Іловайська. 27–28 серпня цілий день йшли ці перемови. Я особисто спілкувався і з Герасимовим (начальник Генштабу ЗС Російської Федерації – ред.), потім його не було, бо він виїхав у Китай. 28 серпня я спілкувався безпосередньо з першим заступником начальника Генштабу Російської Федерації генералом Богдановським. І ми всі ці питання обговорювали. До пізнього вечора узгоджували питання, в тому числі покроково маршрути виходу наших підрозділів. У нас же були ще полонені росіяни. І ми визначали пункти передачі цих полонених при виході наших колон. Але десь біля 23-ї години він вийшов на мене і сказав, що умови змінюються: ми можемо гарантувати вам вихід без техніки і без озброєння. Я йому сказав, що ми на це погодитися не можемо і залишаємо за собою право діяти так, як ми вважаємо за потрібне.
– Хотів би уточнити: ми знаємо про те, що в ніч з 23-го на 24-е заходить російська армія. Резервів для того, щоб підсилити угруповання в Іловайську – бракує. Росіяни заходять з доволі значною кількістю техніки і особового складу – їхні сили переважають. Чому вихід відбувся 29 серпня, а не 24, 25, 26, 27 чи 28-го?
– Тому що йшли бої. Якщо на кожну загрозу реагувати відходом, то невідомо, яку територію ми могли б контролювати і де ми були б.
– Якщо говорити про час. Добре, 24 серпня ця інформація тільки надходила, 26-го – ви вже точно знали, бо взяли в полон російських військових. 26, 27, 28 серпня – чи вважаєте ці дні втраченим часом, який міг би зберегти багато життів?
– Можливо з точки зору сьогоднішнього дня можна було по-іншому вчиняти, але я кажу, що 27-го була зроблена заява, що буде гарантовано цей «зелений коридор» і необхідність прориву туди відпадала. Але вони вчинили потім, як вчинили. Спочатку була умова про те, що буде вихід з технікою, зі зброєю, без будь-яких обмежень. А потім в перебігу вже самих перемовин… Я ще запитав Богдановського (заступник начальника Генштабу Росії – ред.): це кінцеве рішення? А він каже: на жаль, те, що є. Якщо, каже, буде якась додаткова інформація, я на вас вийду. Він більше на мене не виходив аж до самого моменту прориву і розстрілу наших колон. Коли я почав з ним спілкуватися і сказав: що ви робите, де гарантії, одні порушення!? І він мені одну фразу відповів: повірте, рішення ухвалюю не я.
– Чи не вважаєте ви, що перемовини з росіянами і тим паче довіра до них були не зовсім доречними.
– Ми зараз розуміємо, що та фраза, яку в свій час сказав Отто Бісмарк про те, що будь-які домовленості з Росією не коштують паперу, на яких вони підписані. Вона дійсно має місце, підтверджена цими діями і нашою п’ятирічною боротьбою з Росією.
– Чому тоді було ухвалене рішення – давайте, узгодимо з ними вихід, розкажемо про наші плани і домовимось?
– Ви повинні розуміти, що військові ці питання не вирішують.
– Тобто, це було рішення політичне, а не військове?
– Політичне. Було поставлене мені завдання провести відповідні переговори. Уже узгодити саму процедуру і погодити відповідні маршрути по виходу.
– Я правильно розумію, що рішення вести переговори з росіянами і виходити по узгоджених маршрутах приймалося президентом?
– Мені задачу поставив президент провести переговори з росіянами щодо самого механізму виходу цих колон.
– Підсумовуючи тему переговорів і тих днів, які були на них витрачені.
– Реально перемовини проводилися один день – 28 серпня. Якраз узгоджувалися всі ці технічні питання щодо виходу і вже 29-го…
– Але якби 27-го не було заяви про переговори, то…
– Не було б заяви про «зелений коридор», була б зовсім інша ситуація і були б зовсім інші рішення.
– Як ви вважаєте зараз, чи не було це свідомим затягуванням для того, щоб підготувати позиції, які згодом будуть вигідні для обстрілу колон. Тобто чи готові були, в принципі, з самого початку цих переговорів росіяни випустити українських військових? Чи не було це пасткою з першої хвилини цих перемовин?
– Введення в оману, так? Реально можливо і так.
– Тоді цього розуміння не було, як ви вважаєте, зокрема, в політичного керівництва країни?
– Нашої країни? Коли заявляє президент Російської Федерації про те, що ми готові забезпечити «зелений коридор», даємо гарантії цього виходу, то надії були, що це дійсно будуть гарантії і що це дійсно буде вихід. А коли вони в процесі почали міняти відповідні умови, тоді вже було зрозуміло. Крайня заява про те, що вони не дають ніяких гарантій або дають гарантії у тому випадку, якщо ми залишимо зброю і військову техніку – на що ми не могли погодитися. Тоді стало зрозуміло, що дійсно, ситуація, скажімо так, зовсім по-іншому повертається і необхідно ухвалювати рішення про вихід з боєм або про залишення в Іловайську і організації відповідної оборони.
– Хіба Володимир Путін є людиною, якій варто довіряти? Його словам гучним і красивим про «зелений коридор»?
– Виходячи з ситуації сьогоднішнього дня, з нашого п’ятирічного досвіду, він не є тією людиною, якій можна довіряти.
– В 2014 році це не було очевидно?
– Мабуть, не для всіх це було очевидно.
– У глядачів і читачів може скластися таке враження, що наше військово-політичне командування легковажне або наївне. Ви не ловили себе на такій думці?
– Не було у нас там наївності, тому і були віддані відповідні розпорядження щодо виходу і таке інше. Але була надія на те, що, дійсно, ніхто на такий злочин не піде. Це ж порушення всіх міжнародних правил і таке інше. 
– З якими думками ви підходите до цієї дати – п’ять років після Іловайська?
– Мені шкода, що так трапилось, однозначно.
Я розумію, вже з оцінки сьогоднішнього дня, ми б діяли, мабуть, не так, як це було в 2014 році.
