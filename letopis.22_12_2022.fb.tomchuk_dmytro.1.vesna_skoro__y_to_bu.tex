%%beginhead 
 
%%file 22_12_2022.fb.tomchuk_dmytro.1.vesna_skoro__y_to_bu
%%parent 22_12_2022
 
%%url https://www.facebook.com/dmitriy.tomchuk/posts/pfbid02TBsHA6JrHfV7wi1P8Na1yU1y8w4xcE5YqnwdC5G8rVMGj5vWimTgcRhVGyMDPgiql
 
%%author_id tomchuk_dmytro
%%date 22_12_2022
 
%%tags 
%%title Весна скоро. Й то буде вже зовсім інша історія та геть інша війна
 
%%endhead 

\subsection{Весна скоро. Й то буде вже зовсім інша історія та геть інша війна}
\label{sec:22_12_2022.fb.tomchuk_dmytro.1.vesna_skoro__y_to_bu}

\Purl{https://www.facebook.com/dmitriy.tomchuk/posts/pfbid02TBsHA6JrHfV7wi1P8Na1yU1y8w4xcE5YqnwdC5G8rVMGj5vWimTgcRhVGyMDPgiql}
\ifcmt
 author_begin
   author_id tomchuk_dmytro
 author_end
\fi

Це сталося 21 грудня вночі. О 23-47. До наступного дня не вистачило 13 хвилин.
До світанку - шести з половиною годин.  До 2023, у якому ми переможемо - десяти
днів. Але це сталося у свій час, визначений самим Богом у мить створення
Всесвіту, та нема про що дискутувати.

Я пам'ятаю ще, як виглядала вночі зимова Україна з ілюмінатора боїнга з висоти
10 км. То була суцільна біла рівнина, на якій усюди від небокраю до небокраю
під літаком світилися купки жовтих вогників, то були села; ланцюжки, лінії,
рядки - то були дороги та шосе. Іноді пропливали внизу великі та навіть
величезні плями різнокольорового світла - то були міста. Світло було усюди,
неможливо було знайти зверху більш-менш великий клаптик території, де б щось не
світилося б. Підліт до Києва завжди був зрозумілий через появу величезного
сяйва у небі, його неможна було сплутати ні з чим. 

Я уявляю, як вигляда сьогодні Україна з ілюмінатору боїнгу. Точніше - як мала б
виглядати, якщо б над нею літали боїнги. Це зараз та ж сама рівнина, та той
самий сніг на ній там, де він лишився. Але вона лежить внизу темно-блакитна,
освітлена лише луною. Там, де були купи світла, сьогодні поодинокі вогники.
Там, де були ланцюжки, лінії, рядки - іноді повзе маленьке поодиноке світло, це
пересуваються автошляхами військові. Україна нині з літака схожа на море, з
поодинокими вогнями випадкових суден.

Українські міста та села, занурені у темряву та холод, живуть тепер з
обличчями, знову підведеними до неба, як жили покоління наших предків на цій
землі. Раптом стало так, що іноді під час відключення зірки на небі стають
найяскравішим джерелом світла, відразу після світильників, зроблених з
картоплі, олії та бинта. Та люди почали помічати це небесне світло й почали,
виходячи на вулицю, у темряву, холод та первісний жах, дивитися у небо - а що
то воно там у небі світиться? Та заспокоюватися, побачивши що то не двигуни
крилатих ракет а зірки. Так українці 21-го сторіччя відкрили для себе зоряне
небо, та нарешті у цьому знанні встали поряд із усіма, хто населяв цю землю
раніше - скіфами, сарматами, половцями,  трипільцями.

Ми знову почали звертати увагу на сонце. Іншого сенсу набуло тепер, яка
сьогодні погода - чи хмарно, чи ясно та сонячно. Й тепер, як у минулі сторіччя,
стало важливим, о котрій сходить та заходить сонце.

Грудень - найтемніший місяць, коли темрява триває найдовше за добу.  Це дуже
добре знають у країні, де міста побудовані на гнилих та гиблих отруйних
північних болотах, так далеко на півночі, що у грудні  денне світло з'являється
там на кілька годин. Вони на собі знають, як це діє на мозок, нерви, та,
особливо, ДНК,  бо живуть так сотні років, й, виходячи з цього свого буття та
цієї хворої реальності, вони планують стратегію ударів по нас сьогодні, саме у
цей час.

Але вони не встигли скористатися цим часом, а всі удари, що були, виявилися
марними. Тепер час втрачено: 21 грудня вночі, о 23-47, Земля пройшла точку
зимового сонцестояння, й сьогодні це надзвичайно важлива подія для тих, хто
залишається в містах та селах України. Бо це перше зимове сонцестояння за цю
війну. 

Ми пережили найдовшу у році ніч. 

Кожного дня тепер світати буде хоч на трошки раніше, а темніти хоч на трошки
пізніше. Темрява меншатиме, аж доки не перетвориться на декілька майже світлих
годин у липні.

День перемагатиме ніч, як і поклав сам Бог. Світло перемагатиме тьму. Тепло
перемагатиме холод. Надія перемагатиме відчай. Люди перемагатимуть нелюдів.
Зимове сонцестояння - це про те, що попереду в нас гарні часи, про те, що ми
перемогли хоча б у чомусь, переживши найтривалішу ніч, та попереду в нас -
сонячне світло, якого більшатиме.

Так само ми переживемо й зимовий холод, бо незворотньо рухається планета по
орбіті до весни та тепла. Невдовзі темрява та холод залишаться в нас за спиною.
У ворога, одночасно з останніми ракетами, зникне ще одна можливість тиску на
кожного з нас, де б ми не були. 

Буде весна, буле світло, буде тепло. Земля та дороги просохнуть та стануть
твердими замість рідких. Й тоді увесь світ почує, що має сказати ворогові ЗСУ.

Весна скоро. На будь-якому агропідприємстві, де з середини грудня починали
готуватися до весняних робіт, вам скажуть, що часу до весни обмаль, часу маже
нема, не встиг повернутися - вже весна.

Весна скоро. Й то буде вже зовсім інша історія та геть інша війна.
