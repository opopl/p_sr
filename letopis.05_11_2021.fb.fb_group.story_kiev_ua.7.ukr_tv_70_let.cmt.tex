% vim: keymap=russian-jcukenwin
%%beginhead 
 
%%file 05_11_2021.fb.fb_group.story_kiev_ua.7.ukr_tv_70_let.cmt
%%parent 05_11_2021.fb.fb_group.story_kiev_ua.7.ukr_tv_70_let
 
%%url 
 
%%author_id 
%%date 
 
%%tags 
%%title 
 
%%endhead 
\subsubsection{Коментарі}

\begin{itemize} % {
\iusr{Nina Lazarenko}
У нас був такий телевізор який показував одну програму із лінзою яка збільшувала єкран

\iusr{Ольга Кожедуб}
Благодарю, Максим за информацию! Красиво и душевно написано! СПАСИБО!

\iusr{Георгий Горбенко}

У мас дома тоже такой был, только линза квадратная. Позже в сарае был - там у
меня ,,штаб,, был - две программы ловил - без ничего. Хорошоя штука - примерно
до конца 1979 года - работал, потом мы переехали.

\begin{itemize} % {
\iusr{Maksim Pestun}
такой?

\ifcmt
  ig https://scontent-frx5-1.xx.fbcdn.net/v/t39.30808-6/253719338_3049324725337688_8164047446433517493_n.jpg?_nc_cat=100&ccb=1-5&_nc_sid=dbeb18&_nc_ohc=xmcVcj5O4DwAX9R9vtx&_nc_ht=scontent-frx5-1.xx&oh=b1f4249d39d2d1a7ec2f8c3db3310993&oe=618D7AF9
  @width 0.4
\fi


\iusr{Георгий Горбенко}
Точно такая. Однажды мы надолго оставили отодвинутую линзу летом, так у нас стул начал тлеть. Хорошо что пожара не было.

\iusr{Мария Дубровская}
\textbf{Георгий Горбенко} У моей тети тоже с квадратной линзой был
\end{itemize} % }

\iusr{Нина Логвина}
А мы из такой линзы сделали аквариум.

\iusr{Vitali Andrievski}
У нас был один из первых на Подоле ткой телевизор назывался КВН

\begin{itemize} % {
\iusr{Maksim Pestun}
КВН-49

\iusr{Vitali Andrievski}
Спасибо что напомнили. Я помню что были какие то цифры, но не помню какие. 49 это наверное год начала производства этого чуда тогдашней техники
\end{itemize} % }

\iusr{Нина Логвина}

Жили на Горького, в коммунальной квартире (5 семей). Телевизор КВН был только у
нас. Каждый вечер соседи приходили к нам "на телевизор", иногда со своими
стульями, было весело.

\iusr{Вера Гержод}

Был такой же и в нашей семье. Вот только линза была овальная. Долго нам служил,
а потом появился телевизор "Березка". Для нас это был уже "космос", т.к. экран
побольше и линза уже не нужна.

\iusr{Надежда Залескова}
А у нас Знамя , он сразу был с « большим» экраном))))

\iusr{Maksim Pestun}

а у нас на Ленина был такой Ленинград

\ifcmt
  ig https://scontent-mxp1-1.xx.fbcdn.net/v/t39.30808-6/254063010_3049331378670356_4363182306049872749_n.jpg?_nc_cat=105&ccb=1-5&_nc_sid=dbeb18&_nc_ohc=gnQzRvCAsQMAX-B-eEu&_nc_ht=scontent-mxp1-1.xx&oh=d3a4a732a7ba6e667883b76585f1549e&oe=618C273D
  @width 0.4
\fi

\begin{itemize} % {
\iusr{Вадим Кречмар}
\textbf{Maksim Pestun} с приемником и проигрывателей ?

\iusr{Вадим Кречмар}
Мой отец настаивал вышку и оборудование на Крещатике.

\iusr{Maksim Pestun}
вроде бы нет... не помню

\iusr{Людмила Каштан}
\textbf{Maksim Pestun} Да, были приёмник и проигрыватель. У нас потом долго на даче пластинки лежали

\iusr{Анна Солдатова}
\textbf{Maksim Pestun} Там шторка правая двигается, можно закрыть, спрятать экран телевизора
Сверху проигрыватель...
Радио тоже, кажется, есть

\iusr{Людмила Мозговая}
\textbf{Maksim Pestun} такой я помню только у друзей моих родителей...

\iusr{Наталия Педос}
\textbf{Maksim Pestun} Крутой!
Спасибо Вам за интересный рассказ, вернувший в далекие-далекие и безвозвратно ушедшие времена детства!

\iusr{Виктория Штамковская}
\textbf{Maksim Pestun} И у нас первый такой был , левая сторона радиоприёмник.
\end{itemize} % }

\iusr{Павел Бурунов}
Из линз аквариумы делали.

\iusr{Вадим Чемерис}
Тільки лінза заливалась не водою, а гліцеріном

\begin{itemize} % {
\iusr{Maksim Pestun}
это по идее! @igg{fbicon.smile} 

\iusr{Ирина Казачкова}
\textbf{Вадим Чемерис} в аптеке покупали дисцилированную воду и заливали

\iusr{Maksim Pestun}
а если не было "гербовой", то лили простую...
\end{itemize} % }

\iusr{Ірина Оснач}
А у нас линза была квадратная. А сам телевизор у меня на даче, жаль выбрасывать.

\begin{itemize} % {
\iusr{Елена Гугунава}
\textbf{Ірина Оснач} совсем скоро станет антиквариатом! Храните)
\end{itemize} % }

% -------------------------------------
\ii{fbauth.shurhoveckaja_alla.kiev.ukraina}
% -------------------------------------

У нас тоже, один из первых в 2 х эт. Доме на ул. Артема. 12 м. Комната. И каждый
со своим стульчиком нсмотрел эту диковинку по тем временам. Это был 1952-53 г.

\begin{itemize} % {
\iusr{Алла Шурховецкая}

\ifcmt
  ig https://scontent-frx5-2.xx.fbcdn.net/v/t39.1997-6/p480x480/105941685_953860581742966_1572841152382279834_n.png?_nc_cat=1&ccb=1-5&_nc_sid=0572db&_nc_ohc=Hf5CVU3vpT8AX_wQkdB&_nc_ht=scontent-frx5-2.xx&oh=daab3ee9bb0ca206f94e69c2686c79f5&oe=618C7BCB
  @width 0.2
\fi

% -------------------------------------
\ii{fbauth.shvecova_natalia.kiev.ukraina}
% -------------------------------------

\textbf{Алла Шурховецкая} 

Аллочка! У нас та же история - соседи приходили вечером, усаживались, правда
стулья были наши. У соседей был мальчик - на пару лет старше нас с сестрой. А
тогда были фильмы "детям до 14 нельзя". Он оставался смотреть, а нас мама
выпроваживала. Обидно до слёз. А теперь все это вспоминается с таким теплом.

\begin{itemize} % {
\iusr{Алла Шурховецкая}
\textbf{Наталья Швецова} 

Да. Приятные воспоминания.  Наш путь был сложным и одновременно счастливым. У
меня лично нет претензий к той системе и времени.ж

\iusr{Алла Шурховецкая}

\ifcmt
  ig https://scontent-frx5-2.xx.fbcdn.net/v/t39.1997-6/s168x128/47270791_937342239796388_4222599360510164992_n.png?_nc_cat=1&ccb=1-5&_nc_sid=ac3552&_nc_ohc=UxuRvQAuwp8AX8iWqBe&_nc_ht=scontent-frx5-2.xx&oh=f52e49e8b3394e7acfe9778e04e2efbb&oe=618BA136
  @width 0.1
\fi

\end{itemize} % }

\end{itemize} % }

\iusr{Melnicuk Vladimir}
Отето ГАДЖЕТ!!!

\iusr{Виктория Самарина}
Очень интересно!

\iusr{Vadym Shvydkiy}

Да... Помню... И про "пульты" помню и про ДЗЗ на Первомайского и про огромные
ПТСы (пел в Большом Детском Хоре Гостелерадио УССР у Татьяны Николаевны
Копыловой)... Было время...

\iusr{Анна Сидоренко}

Очень хорошо всё помню, первый телевизор не все могли купить, у моей подружки
по школе отец был машинистом поезда, там зарплата была хорошая и они первыми
купили, а мы, дети, ходили к ним на интересные передачи, они сделали даже
маленькие стульчики, а кому не хватало, то просто на полу располагались.

\iusr{Лидия Перкина}

У нас был такой телевизор с линзой, помню бабушка очень любила смотреть
фигурное катание, брат пробовал окрашивать жидкость или наклеивал пленку с
разными цветами, чтобы создать эффект цветного изображения

\begin{itemize} % {
\iusr{Regina Kostetska}
\textbf{Лидия Перкина} Особенно фигурное катание с Белоусовым и Протопоповой!

\iusr{Лидия Перкина}
\textbf{Regina Kostetska} Совершенно верно

\iusr{Лариса Олейникова}
\textbf{Regina Kostetska} Белоусова и Протопопов.
\end{itemize} % }

\iusr{Елена Ходорко}

...у нас такой был, КВН, с линзой, мамочка в ней периодически меняла воду....

\iusr{Kravets Irina}
Поздравляю, Максим!
И спасибо за интересные воспоминания! @igg{fbicon.thumb.up.yellow} 

\iusr{Валентина Белозуб}

Я тоже закончила КПИ электроакустический факультет! Только мы учились на
"собачке" как отдельное государство! Отличное было время! А затем 40 лет было
отдано служению Украинскому телевидению на Крещатике 26, а затем на
Мельникова! Поздравляю, коллеги с праздником Телевидения!

\iusr{Светлана Онойко}

И у нас был такой в моем детстве.

\iusr{Светлана Аникина}

Я маленькая совсем была и пыталась в этой линзе увидеть рыбок - думала, что это
такой аквариум

\iusr{Лариса Давиденко}

Да уж, на подъезд был один такой телевизор.
Каждый со своим стулом приходил.
Если не сделал уроки, плач в кулочок!!

\iusr{зоя Барабанщикова}

Эх, вспомнить бы первого диктора нашего телевидения Ольгу Даниленко! Мы были
влюблены в неё. Может, у кого-то есть фото?

\iusr{Gennady Henry Sergienko}

Спасибо большое за воспоминания!

Помню, в середине 70-х, после того как появилась третья программа, по квартирам
ходили народные умельцы, которые за небольшую мзду настраивали переключатель
телевизора так, чтобы все программы переключались подряд.

\end{itemize} % }
