% vim: keymap=russian-jcukenwin
%%beginhead 
 
%%file 04_08_2018.fb.lesev_igor.1.obezvredit_julju
%%parent 04_08_2018
 
%%url https://www.facebook.com/permalink.php?story_fbid=2032020123495747&id=100000633379839
 
%%author_id lesev_igor
%%date 
 
%%tags politika,timoshenko_julia,ukraina
%%title Обезвредить Юлю
 
%%endhead 
 
\subsection{Обезвредить Юлю}
\label{sec:04_08_2018.fb.lesev_igor.1.obezvredit_julju}
 
\Purl{https://www.facebook.com/permalink.php?story_fbid=2032020123495747&id=100000633379839}
\ifcmt
 author_begin
   author_id lesev_igor
 author_end
\fi

Обезвредить Юлю

Тимошенко, если ориентироваться на социологию, прет танком. Да, разрывы между
оппонентами не столь велики. Но у нас ведь и реалии другие. Постмайданная
Украина – это чуть-чуть правовая страна, с чуть-чуть полуправдой во всем,
включая и желание пипла искренне отвечать социологам. Мы живем в стране
шизофреников, истеричек и стукачей. Это не значит, что их не было до 2014-го.
Но и пьяные матросики до 1917-го тоже встречались. Просто всегда есть дурное
время собирающее дурных людей. Потому и такая социология. Первое место –
Тимошенко. Второе – Гриценко. И только на третьем скромно ковыляет условный
представитель от юго-востока.

\ifcmt
  ig https://scontent-frt3-1.xx.fbcdn.net/v/t1.6435-9/38492644_2032019920162434_8993680166571999232_n.jpg?_nc_cat=106&ccb=1-5&_nc_sid=730e14&_nc_ohc=ms-tEdG519MAX_3Ye5q&_nc_ht=scontent-frt3-1.xx&oh=d133d5d675376fe4623759e50ff080ef&oe=61B94F8C
  @width 0.4
  %@wrap \parpic[r]
  @wrap \InsertBoxR{0}
\fi

Что интересно, Тимошенко особо никакой избирательной кампании не проводит.
Ребята в штабе просто тупо изучили настроения людей, а именно, что их больше
всего волнует – мир, тарифы, социалка, безработица, безопасность. Все. Эти темы
и лузают на бордах. При этом делают все грамотно. Впаривают максимально
абстракции с американизмами из школьного курса а-ля «Новый курс». Еще чуть-чуть
хрени из общего курса философии об общественном договоре (Гоббс, Руссо,
Монтескье). И какая-то срань по экосистеме. Все. Пипл схавает.

Главное не отвечать на вопросы, на которые нельзя ответить. Позиция Юли по
церковному расколу? Что делать с Минском-2? Как решать вопрос с Крымом и
вообще, будут ли прямые переговоры с Москвой без объятий Вашингтона? Будет ли
пересмотрена кабальная ассоциация Украины с ЕС? Если да – то какие конкретно
пункты? Обслуживание внешнего долга после 2018-го? Институциональные реформы –
будут, нужны, и если нужны то что на что меняется?

Счастье Тимошенко в том, что она избирается в президенты в самой нечитаемой
стране Европы. Это я так политкорректно обошел эпитет «тупой». Потому что
втянуть Тимошенко в конкретику – это значит начать ее обличать. Просто
попросить ее назвать свое видение будущего премьера, прокурора, главу НБУ,
министра экономики и пр. – это показать насколько Юлия Владимировна голая на
кадры.

Но, в любом случае, бить Тимошенко рацио – это не про нашу страну. Если пипл
сожрал в 2014-м «жити по-новому», то почему бы этим же самым людям, которые
голосовали за Порошенко, теперь не продать ТО ЖЕ САМОЕ – «новый курс»? Если на
ферме жрут похлебку стоимостью 20 гривен, зачем ее делать как-то
привлекательнее, но по цене уже в 30 гривен? Избиратель и Порошенко, и
Тимошенко эмоционален. Он верует. Поэтому объяснять, что Тимошенко – это и есть
Порошенко минус сам Порошенко – абсолютно бессмысленно.

Для нейтрализации Тимошенко нужна механика. А она начинается с единоначалия,
замыкающегося на Банковой. Порошенко точно не выиграет выборы, если оставит во
главе МВД Авакова. Но давайте прямо, какие шансы, что Порошенко способен снять
Авакова? У нас только на текущей неделе раздолбали прокуратуру в Херсонской
области, а в Киеве разнесли информагентство. Неделей ранее разнесли холл НАБУ.
Вот представим, что при Януковиче такое происходит, а менты никак не
среагировали. Согласитесь, такое даже нелепо представлять. В стране полностью
нарушена силовая вертикаль. Рядовой мент не может просто скрутить гопника,
потому что у нас в стране уже нет гопников, а есть патриоты и активисты.

А ведь в команде Порошенко думают так. Мы тут чуть подрисуем, подхимичем,
комиссии наши, короче, не мытьем так катаньем и хрустящими купюрами сделаем
результат. «Врать не буду, в первом туре мы не победим», - рассуждает Сережа
Березенко. А защищать даже нахимиченные результаты вы как будете? Вас же будут
укатывать прямо на избирательных участках.

Другая проблема – это вопрос согласования Тимошенко. Пока что по всем
параметрам согласовали именно Гриценко. В Вашингтоне, естественно. Но Гриценко
настолько скучный, что он проиграет, даже если за него будут вместе агитировать
Трамп и баба Хиля. А на Банковой все, до чего додумываются, так это называть
Тимошенко «кремлевской консервой». И тем самым делают самую крутую услугу
Тимошенко по продвижению на юго-восточное поле. Тимошенко в принципе абсолютно
грамотно избегает все острые темы, волнующие гигантский юго-восточный электорат
– язык, война на Донбассе, церковь, идентификационные маяки. Более того, если
Тимошенко в лоб спросить, она всегда ответит – я за украинский язык, за ЕС, за
НАТО… Но никто так не продвигает именно Тимошенко на юго-восточное
электоральное поле, как тупые спичрайтеры Порошенко. В сегодняшних условиях
среднестатистическому обывателю Харькова или Кривого Рога достаточно просто за
день не услышать «Слава Украине», «томос» или «Бандера», чтобы о политике
составить плюс/минус благоприятное представление. Тимошенко ежедневно не
славоукраинствует в эфире в отличие от Порошенко.

Наконец, если Порошенко реально собирается как-то бороться за неопределившихся
избирателей майданного типа (другие, в отличие от Тимошенко, ему уже не
светят), ему что-то нужно делать со своей речью и своими месседжами. То что
украинские президенты живут в своих построенных сказочных мирах – это мы знаем.
Но даже в кино, когда артист покупает в магазине батон, он делает это
естественно. Например, не цитирует Шекспира. Не рассказывает продавщице о
парниковом эффекте. Не начинает танцевать, если это только не мюзикл. Порошенко
же даже на открытии детского садика говорит о НАТО, европерспективах,
нанотехнологиях и остаточном прощавай. Каждый раз одна и та же унылая хрень. У
человека нет чувства меры. Начиная от того, что какого хера президент вообще
должен открывать детский сад, и до того, что он городит.

Тимошенко в этом плане с одной стороны легче. Она может позволить себе говорить
о том, о чем не может говорить Порошенко. О коррупции, ворах, схемах, бедности
и тотальной бесперспективности. Тимошенко даже не надо сгущать краски, все на
поверхности. Но у Тимошенко есть другое преимущество. Она говорит человеческим
языком. Да, не без МХАТовских пауз. Чуть тихо. Никогда не перекрикивая. И тут
же уходя от словесного боестолкновения с подсадными персонажами. Юля – это ведь
тоже пафос и наигранные мизансцены. Но все же Тимошенко – это народная актриса.
В том плане, что она понятна улице. Ее можно раскусить, изобличить, можно с ней
не соглашаться, но она остается всегда собеседником. А Порошенко – это такой
себе Нерон, которому подхалимы, вроде Медведева, говорят о чудесности его речи,
а ради «великой театральной постановки» эта пацанва еще и сожжет к ипеням всю
страну.

В общем, у Порошенко есть три базовые проблемы, не решая которых он не решит
вопрос с Юлей:

- единоначалие;

- согласование;

- коммуникация.

А учитывая тот чудесный экспертный состав в окружении Порошенко, Юлии
Владимировне за способность президента переломить ситуацию волноваться не
стоит. Но зато у нее другая проблема – преодолеть свой гигантский антирейтинг в
возможном финальном махаче с Гриценко. Особенно в условиях, когда проигравшему
Порошенко под гарантии из Вашингтона будет дана команда «способствовать» победе
полковника.

\ii{04_08_2018.fb.lesev_igor.1.obezvredit_julju.cmt}
