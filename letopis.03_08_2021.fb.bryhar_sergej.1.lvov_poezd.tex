% vim: keymap=russian-jcukenwin
%%beginhead 
 
%%file 03_08_2021.fb.bryhar_sergej.1.lvov_poezd
%%parent 03_08_2021
 
%%url https://www.facebook.com/serhiibryhar/posts/1755827677950510
 
%%author Бригар, Сергей
%%author_id bryhar_sergej
%%author_url 
 
%%tags jazyk,lvov,mova,obschenie,odin_narod,poezd,ukraina
%%title Напередодні Івана Купала я мав термінові справи у Львові
 
%%endhead 
 
\subsection{Напередодні Івана Купала я мав термінові справи у Львові}
\label{sec:03_08_2021.fb.bryhar_sergej.1.lvov_poezd}
 
\Purl{https://www.facebook.com/serhiibryhar/posts/1755827677950510}
\ifcmt
 author_begin
   author_id bryhar_sergej
 author_end
\fi

Напередодні Івана Купала я мав термінові справи у Львові. Їхав у плацкартіному
вагоні поїзда "Одеса - Ужгород". Біля мене (в моєму і кількох сусідніх
плацкартах) сиділи хлопці й дівчата, що прямували на "Шипіт".

Спочатку я нікого не чіпав - просто слухав їхні розмови. А там було багато
цікавого. Передати весь колорит неможливо, тому я спробую просто трохи
процитувати: "Я много путєшествовал - Камчатка, Чукотка, Сахалін, корочє, там,
гдє живут наши, русскіє...", "о, кстаті, в Сібірі как Одєссу знают - братан,
рєспєкт, уважуха, наш чувак, Одєсса - наши люді, і вот ето всьо", "у мєня
сєструха живьот в Пітєрє, я вот собіраюсь туда щяс, там говорят, одєссіти в
почьтє, потому что нє прогібаются, стоят за свою культуру", "у тєх шипотян, ну
людей, коториє там живут, говор такой інтєрєсний, нє всьо даже понятно, но
потіхоньку, за нєдєлю с одєссітамі оні так потіхоньку начінают нормально
говоріть по-русскі", "с намі в позапрошлом году тусовалісь нєсколько москвічєй,
ми іх сразу прінялі - отлічниє рєбята", "одєссіти на Шипотє самиє центровиє, с
намі затусіть - ето за чєсть"...

Послухав я оце все, та й кажу: "хлопці, а ви що збираєтеся там лише одеською тусовкою?"

\begin{itemize}
  \item - Нє, ну почєму, бивают с Москви, Мінська, Харькова рєбята. Карочє, разниє, но в основном, всьо-такі, наши, русскіє.
  \item - То які ж вони русскіє, з Мінська і Харкова? Це білоруси та українці.
  \item - Ой, ну ето уже всьо такоє накручєноє, ето політика. Для нас всє люді одінакови. Нєт разніци украінєц ти, русскій, бєлорус. Ето одінаково.
  \item - Однакові, але називаєте росіянами. Цікаво.
  \item - Ну давайте назвьом іх украінцамя, раз вам так лєгчє.
  \item - Тобти, національна ідентичність для вас не має жодного значення?
  \item - Нєт. Главноє - люді. Вот ви говорітє по-украінскі, і ми нє протів, і ми всьо понімаєм. Всєм воля і свобода.
  \item - А те, що в Україні триває війна з Росією вас не тривожить, коли ви тусуєтеся з москвичами чи пітерцями?
  \item - Нєт. Ми ету войну нє начіналі. Нам нєчєго дєліть...
\end{itemize}

Наймолодшому з компанії - 23, найстаршому - 35. Мені - 33. Це люди з мого
рідного міста, в якому я провів усе життя. З багатьма із них ми навіть мешкаємо
відносно близько. По ідеї, вони могли би бути моїми друзями, товаришами... Але
між нами - прірва...

\ii{03_08_2021.fb.bryhar_sergej.1.lvov_poezd.cmt}
