% vim: keymap=russian-jcukenwin
%%beginhead 
 
%%file 29_01_2022.fb.buzhanskii_max.1.minsk_soglashenia
%%parent 29_01_2022
 
%%url https://www.facebook.com/permalink.php?story_fbid=2141048969392904&id=100004634650264
 
%%author_id buzhanskii_max
%%date 
 
%%tags donbass,edinstvo,minsk_dogovor,ukraina,vojna
%%title Скажу все таки пару слов о Минских Соглашениях
 
%%endhead 
 
\subsection{Скажу все таки пару слов о Минских Соглашениях}
\label{sec:29_01_2022.fb.buzhanskii_max.1.minsk_soglashenia}
 
\Purl{https://www.facebook.com/permalink.php?story_fbid=2141048969392904&id=100004634650264}
\ifcmt
 author_begin
   author_id buzhanskii_max
 author_end
\fi


Скажу все таки пару слов о Минских Соглашениях.

Нет, не о сути, не о пунктах, не о формуле Штанмайера, которую туда Порошенко
вписал, и не об изменениях в Конституцию, им же согласованных, которые
невозможны в принципе.

Я о другом.

Представьте себе, что отрезан палец.

И стоит задача его пришить.

Я не хирург, но понимаю, что главное, не пристегать нитями к кисти, нет.

Главное, чтобы прижилось, главное, чтобы срослось.

Чтобы не было воспаления, чтобы не было заражения, чтобы не началась гангрена,
которая убьёт организм.

Нужно обрабатывать место операции, промывать, прокалывать антибиотики, и
быстрее, как можно быстрее, пока ткани ещё живы.

Вернёмся к стране, от которой отрезали Донбасс.

Отрезали, отрезался, неважно вообще.

Что нужно было сделать?

Обработать страну, как рану,  открытую рану, которой она является все эти
восемь лет после Майдана.

Как мы живём?

Как мы хотим жить?

Как мы можем жить?

Как мы будем жить?

Какая гуманитарная повестка- общая мечта нас всех( нет такой, и быть не может).

А какая- тот самый неприятнейший компромисс, на который мы все готовы пойти
ради единства нашей страны и мира в ней?

А кто посчитал все или не все, и сколько тех и других?

Беда не в том, что за эти восемь лет ничего не было сделано для реинтеграции,
нет.

Беда в том, что ничего не было сделано для того, чтобы эта реинтеграция не
взорвала страну сразу.

Условия, в которых она должна состояться, вступи она в силу завтра, это машина
времени.

Когда страну, и Донбасс в том числе, бросает назад, в конец февраля 2014 года.

В тот момент, когда все рухнуло.

И вдруг предлагают считать его тем моментом, с которого все само собой вдруг
станет хорошо.

А так не бывает.

Невозможно взаимное приживление чего то, где с одной стороны есть условные
языковые инспекторы, а с другой на них несогласны категорически.

И безусловные языковые инспекторы тут тоже могут быть упомянуты уместно.

Резюмирую.

С нынешней гуманитарной повесткой это невозможно сделать.

Физически невозможно, законодательно невозможно, реально невозможно.

Страна не ушла с Майдана, она живёт на нем, умирает на нем, и вовсе не потому,
что это ей нравится.

Ей просто не предложили ничего другого.

Если не предложить, ничего не получится.

Это, кстати, и обо всей стране в целом, не только о реинтеграции Донбасса.

И да, чуть не забыл.

Все патриоты, которые вот прям сейчас хотят порвать флаг у себя на аватарке в
порыве чего то там, могут напрямую выяснить отношения с нашими американскими
партнёрами, заставлять выполнять Минск будут не из Воронежа, нет, оттуда нас
заставить невозможно.

Max Buzhanskiy
