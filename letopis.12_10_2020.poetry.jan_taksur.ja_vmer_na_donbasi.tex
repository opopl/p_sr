% vim: keymap=russian-jcukenwin
%%beginhead 
 
%%file 12_10_2020.poetry.jan_taksur.ja_vmer_na_donbasi
%%parent 12_10_2020
 
%%endhead 

\subsection{Я вмер на Донбасі. Напис на солдатській могилі}
\label{sec:12_10_2020.poetry.jan_taksur.ja_vmer_na_donbasi}

\obeycr
Замечательное стихотворение Яна Таксюра
"Я вмер на Донбасі. Напис на солдатській могилі"

Тут лежу у землі, мені двадцять було,
Я --- сержант, з Ірпеня, звали Вася.
Ми стріляли крізь дим, все ревло і гуло,
Ну, а потім я вмер. На Донбасі.

Пам’ятаю рибалку, в саду солов’ї
І як мама будила уранці.
Тільки маму пригадую, очі її,
І ще хлопця одного в Слов’янську.

Був той хлопець високим, білявим, худим.
І навіщо він вибіг з підвалу?
Тільки влучив снаряд, був старим його дім,
І ні дому, ні хлопця не стало.

Я впізнав той будинок без даху й стіни,
Як у місто зайшла наша рота.
І хлопчина той мертвий лежав на спині,
Трошки крові текло в нього з рота.

Черевики, як в мене, один впав з ноги.
На цю стрижку і в нас була мода.
І хоч нам говорили, що скрізь вороги?
Того хлопця було мені шкода.

Потім вибух, і ще. Рідна ж мати моя!
Затремтіла невидима сила.
І той дім, що як привид без даху стояв,
Раптом впав. І мене задавило.

Так лежали ми вдвох, розділивши біду,
На землі, що звалась Україна.
Було важко мені, тільки…бачу в саду
Божу Матір і Божого Сина.

Матір Божа молилась до Сина Свого,
Землю нашу позбавити муки.
Вбитий хлопець був поруч, і крові його
Мені впало три краплі на руки.

І Господь подивився з любов’ю Отця,
Не суддею суворим, не катом.
І розвіявсь туман. Я збагнув до кінця:
Значить, брата я вбив, значить, брата…

Я лежу під землею і бачу свій край –
Річка, поле, просте і кохане.
Люба мамо, простіть, загубив я свій рай,
Мов отрути хтось дав чи дурману.

Мамо, рідна, благаю, ідіть та знайдіть,
Де тепер мого брата могила,
І моліть кожен день, кожен час, кожну мить
Його матір, щоб мене простила.

Бо коли запече кров із братових ран,
То не бачу я Божого Сина.
І моліться, щоб зник той пекельний туман,
Що на землю упав України.
\restorecr
