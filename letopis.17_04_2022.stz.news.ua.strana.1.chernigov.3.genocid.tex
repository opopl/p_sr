% vim: keymap=russian-jcukenwin
%%beginhead 
 
%%file 17_04_2022.stz.news.ua.strana.1.chernigov.3.genocid
%%parent 17_04_2022.stz.news.ua.strana.1.chernigov
 
%%url 
 
%%author_id 
%%date 
 
%%tags 
%%title 
 
%%endhead 

\subsubsection{\enquote{Это геноцид какой-то}}

Чернигов с первого дня войны находился под постоянными обстрелами со стороны
России. При этом все объекты инфраструктуры, которые пострадали от бомбежек —
гражданские. Черниговская ТЭЦ, школы, детские садики, отели, больницы, жилые
дома — вот далеко не полный перечень разрушений в городе. По оценке мэра,
уничтожено до 70\% застройки Чернигова. На полное восстановление инфраструктуры
уйдет минимум четыре года. 

Когда 23 марта российская армия разрушила мост через реку Десна, Чернигов
оказался окружен вражескими войсками. 

— По сути, мы стали островом, — описывает тот период войны Александр Ломако,
секретарь Черниговского горсовета.

После этого враг прицельно бил по складам с продуктами и заправкам с топливом,
чтобы в город не могли доставить помощь. Российские войска искусственно
доводили Чернигов до гуманитарной катастрофы. Три недели в Чернигове не было
света и воды, практически не было связи. Отопление не работало, а в это время
как раз ударили морозы. 

За стойкое сопротивление президент Украины Владимир Зеленский присвоил
Чернигову звание города-героя. Несмотря на гуманитарную катастрофу и постоянные
бомбардировки, в Чернигове все это время оставалось около 100 тысяч человек.
Сотрудники коммунальных служб под обстрелами чинили поврежденные электросети,
волонтеры развозили людям воду и еду. 

Многие из них получили ранения, а некоторые — погибли, пытаясь помочь другим.
За время бомбардировок умерло около 700 мирных жителей. Это только те жертвы,
которые уже удалось установить. Сколько точно людей погибло в Чернигове,
возможно, не удастся выяснить никогда. Многие числятся пропавшими без вести —
вероятно, эти люди заживо сгорели в местах применения тяжелых авиабомб.

— Чернигов — город-герой. Но в первую очередь, Чернигов — это город героев, —
говорит Александр Ломако. — Наша базовая больница была обстреляна, там не было
ни одного целого окна, не было света и воды. Но наши врачи под обстрелами
продолжали лечить людей. Бывали дни, когда привозили по 100 раненых. Волонтеры
продолжали печь хлеб, развозили его людям. Пожарные, полиция, водители, местная
власть, которая осталась на своем рабочем месте — все делали большое дело,
чтобы город выжил в эти 38 тяжелых дней окружения. 

\ii{17_04_2022.stz.news.ua.strana.1.chernigov.pic.5}

Черниговский футбольный стадион имени Юрия Гагарина российские войска
разбомбили ночью 11 марта. На этом стадионе базировался футбольный клуб
\enquote{Десна}. Враг с самолета сбросил сюда минимум 5 авиабомб. 

Одна из бомб оставила после себя огромную десятиметровую воронку посреди
футбольного поля. В сетке на воротах застряли куски бетона. Трибуны, на которых
еще недавно собирались болельщики, превратились в сплошное месиво из досок и
огрызков пластиковых сидений. 

Провалившиеся трибуны с жалостью рассматривает мужчина в ярко-оранжевом жилете
с эмблемой \enquote{Десна}. Он проработал на этом стадионе 21 год. 

— Я тут и за охранника был, и за сварщика, и за дворника. До последнего
оставался на дежурном посту. Но когда начали бомбить Чернигов, сотрудников
отсюда сняли, потому что опасно было. Когда обстреляли стадион, меня здесь не
было. А как русские ушли, мне позвонили и попросили охранять стадион от
мародеров. Так что я вернулся, — говорит Михаил.

Глядя на то, что осталось от стадиона, который видел столько спортивных
достижений, столько выкриков футбольных фанатов, столько труда, слез и пота
тренирующихся здесь спортсменов, мужчине трудно подобрать слова. 

— Я даже не знаю, как их назвать. Нелюди! Вот подходящее слово. Только нелюди
такое могут сделать.. Это не военный объект, не стратегический. Тут дети
занимались спортом, радовались жизни. Футбол, бокс, тяжелая атлетика… Брали с
шести лет, выращивали футболистов. Детская спортивная школа тут была на высоте.
У нас базировалась Сборная Украины по тяжелой атлетике. Я со всеми дружил, меня
все уважали. Я на работу ходил, как на праздник. А это все… Очень тяжело.
Тяжело понять, зачем это было сделано. Слов нет, — говорит Михаил, потирая
глаза в попытке сдержать слезы. 

\ii{17_04_2022.stz.news.ua.strana.1.chernigov.pic.6}
