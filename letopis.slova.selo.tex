% vim: keymap=russian-jcukenwin
%%beginhead 
 
%%file slova.selo
%%parent slova
 
%%url 
 
%%author 
%%author_id 
%%author_url 
 
%%tags 
%%title 
 
%%endhead 
\chapter{Село}

%%%cit
%%%cit_head
%%%cit_pic
%%%cit_text
─ Ми зі Світланою народилися в місті ─ в Сєвєродонецьку, але частина нашої
родини з Волині, ми їздили туди щоліта. Це було \emph{маленьке село} з однією вулицею.
Згадую, як я лажу по горищу в купі старих зошитів в пилюці, дивлюся якісь
фотографії. Це в мене викликає відчуття того старого дому, якого вже немає, ─
розповідає Анна.  ─ \emph{Село} досі існує, хоч зникає та сильно трансформується. Ми
хочемо зафіксувати момент його теперішнього стану, — додає Світлана.  \emph{Сільське}
населення та кількість \emph{сіл} поступово зменшуються. Наприклад, на Сумщині за
останні 20 років зникло 45 \emph{сіл}. Чимало людей надає перевагу міським
апартаментам
%%%cit_comment
%%%cit_title
\citTitle{Шукайте хату. Як авторки Old khata projekt подорожують Україною і фіксують час}, 
Влад Яценко, life.pravda.com.ua, 17.06.2021
%%%endcit
