% vim: keymap=russian-jcukenwin
%%beginhead 
 
%%file topics.vojna.my.9.gruppa.6.medved
%%parent topics.vojna.my.9.gruppa
 
%%url 
 
%%author_id 
%%date 
 
%%tags 
%%title 
 
%%endhead 

%https://enot2290.livejournal.com/201787.html?ysclid=l7geqtwm3h930058014

Буквально на днях ученые сделали грандиозное открытие, оказывается такие
огромные и наводящие ужас на человека животные медведи боятся обычного пения
(!). Но почему-то в оное верится с трудом… Наверно когда в лесу наткнешься на
медведя, пение – это последнее, что сможет прийти на ум человеку после
пережитого страха…

На самом деле, при нахождении в лесу, где может появиться медведь (медведи у
нас водятся не во всех лесах) нужно громко разговаривать, шуметь, петь песни –
одним словом делать всё, чтобы медведь заметил Вас раньше чем Вы его и
постарался унести подальше ноги.

Доказано, что если медведь замечает человека заблаговременно/ издали, он в 95%
случае пытается скрыться из вида. Так же ни разу не зарегистрировано случаев
чтобы медведь нападал на группу человек.

Чтобы напугать медведя, нужно держаться толпой, громко разговаривать, кричать и
петь.

Медведи вообще-то боятся человека и избегают встречи с нашим братом. 
