% vim: keymap=russian-jcukenwin
%%beginhead 
 
%%file 17_12_2021.fb.tovt_anastasia.1.kniga
%%parent 17_12_2021
 
%%url https://www.facebook.com/anastasija.tovt/posts/4666506206767746
 
%%author_id tovt_anastasia
%%date 
 
%%tags chtenie,kniga,literatura
%%title Если вы еще не читали книгу Мэттью Макконахи, то многое упустили
 
%%endhead 
 
\subsection{Если вы еще не читали книгу Мэттью Макконахи, то многое упустили}
\label{sec:17_12_2021.fb.tovt_anastasia.1.kniga}
 
\Purl{https://www.facebook.com/anastasija.tovt/posts/4666506206767746}
\ifcmt
 author_begin
   author_id tovt_anastasia
 author_end
\fi

Если вы еще не читали книгу Мэттью Макконахи, то многое упустили. 

Я всегда знала, что Макконахи - офигенный актер. Но не подозревала, что он
такой глубокий человек. 

«Зеленый свет» - автобиография, основанная на его дневниках. Вперемешку с
кадрами из личного архива Макконахи и набросками будущей книги, написанными его
рукой. 

\ii{17_12_2021.fb.tovt_anastasia.1.kniga.pic.1}

С точки зрения эстетики, книга - просто визуальный оргазм. Это вам не мемуары
Юлии Мендель. 

А с точки зрения содержания и смысла - как хорошее вино, которое не
проглатывают залпом, а медленно смакуют. 

Честный рассказ о неизбежности и выборе, сомнениях, поиске, пути, случайностях
и предопределенности, силе намерения и цели, о провалах и успехах. О детстве и
взрослении. О жестких методах воспитания. К примеру, однажды Макконахи выпороли
ремнем за то, что в детском саду он отозвался на имя Мэтт. Мол, мы не просто
так дали тебе это имя, уважай его. 

Его родители - вообще отдельная тема. Они дважды разводились и трижды женились.
Друг на друге. 

Отец четыре раза ломал маме средний палец - «чтобы она не тыкала ему в лицо». 

При этом отец делал искусственное дыхание попугаю, который случайно утонул в
унитазе. И что вы думаете? Попугай прожил еще восемь лет. 

Макконахи - мастер рассказывать истории. Не только на экране, но и на бумаге.

Не припомню, чтобы кто-то еще так писал про испытание славой. Она пришла к
нему, актеру-самоучке, неожиданно после выхода фильма «Время убивать». Чтобы
справиться с популярностью, он уезжал в монастырь. Жил отшельником и три часа,
рыдая, исповедовался монаху в своих грехах, делился сомнениями, а монах молча
слушал. А потом сказал всего два слова: «Я тоже». И Макконахи отпустило. 

Затем он уехал в Амазонку, жил дикарем. Знакомился с собой заново. 

Потом, в период очередного кризиса, в Африку. Где стал свидетелем спора двух
аборигенов о проходящей мимо проститутке. Мол, правильно ли женщине продавать
себя? Макконахи поддержал спорщика, который высказался против. А этот абориген
вдруг наехал на него:

«Мы не о том, кто прав а кто не прав. Мы про «понимаешь или нет»!»

Как это необычно и мудро. Всем стоит взять на заметку такой подход к дискуссии.
Я не пытаюсь тебя переубедить. Я пытаюсь понять, как ты мыслишь. 

Вот несколько цитат из «Зеленого света». Чтобы было понятно, знакомство с каким
человеком ждёт вас на этих страницах. 

Я делал записи не для того, чтобы запомнить события, а наоборот, чтобы их
забыть. 

Когда я в первый раз пригрозил, что уйду из дома, родители собрали мне рюкзак. 

Существует мировой заговор сделать меня счастливым. 

Мне часто попадались чужие стихи, автором которых был я, сам того не зная. 

В нашей семье придерживались строгости и требовали соблюдать правила, до тех
пор, пока у тебя не хватит смелости их нарушить. 

Ставь рамки, чтобы получить свободу. 

Мы не рождаемся в равном положении или с равными способностями, но у нас должны
быть равные возможности.

Если тебе не везет, хотя ты считаешь, что должно, не оставляй им выбора в
принятии решения. Играй как здорово, чтобы это было невозможно отрицать. 

Я готов был превратить «я стал бы, если бы мог» в «я могу и стану».

Желай ее, не нуждайся в ней, и может быть, она покорится. 

Обычно пропажа не украдена, а просто лежит там, где ты ее оставил. 

Когда ты можешь что-то сделать, спроси себя, хочешь ли ты этого, прежде чем это
делать. 

Я по-прежнему звонил маме каждое воскресенье. Только теперь я звонил не маме.
Разговаривала со мной не мама. А женщина, которая любила мою славу больше, чем
я. 

Иногда список приглашенных должен состоять из одного человека. Тебя. 

Дело не в победе или поражении. Дело в том что ты принял вызов. Если ты принял
вызов, значит победил.

Лидеры не всегда впереди. Они знают, за кем идти следом.

Стрела не ищет цель. Цель притягивает стрелу.

Великий человек не все для каждого. Он каждый для всех. 

Гений может делать все, но делает одно за другим, по очереди.

Мне не надо ничего, кроме того, что ты готов мне дать.

Я хотел того, в чем нуждался, и нуждался в том, чего хотел.

Для меня компетентное «удовлетворительно» дороже невежественного «отлично».

Оттого, что стараешься не умереть, дольше не проживёшь. А вот когда занят
жизнью, то живешь дольше.

Есть разница между творчеством и самовыражением. Всякое творчество есть
самовыражение. Но не всякое самовыражение - творчество.
