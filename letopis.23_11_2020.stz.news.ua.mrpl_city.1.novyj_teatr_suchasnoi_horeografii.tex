% vim: keymap=russian-jcukenwin
%%beginhead 
 
%%file 23_11_2020.stz.news.ua.mrpl_city.1.novyj_teatr_suchasnoi_horeografii
%%parent 23_11_2020
 
%%url https://mrpl.city/blogs/view/novyj-teatr-suchasnoi-horeografii
 
%%author_id demidko_olga.mariupol,news.ua.mrpl_city
%%date 
 
%%tags 
%%title Новий театр сучасної хореографії
 
%%endhead 
 
\subsection{Новий театр сучасної хореографії}
\label{sec:23_11_2020.stz.news.ua.mrpl_city.1.novyj_teatr_suchasnoi_horeografii}
 
\Purl{https://mrpl.city/blogs/view/novyj-teatr-suchasnoi-horeografii}
\ifcmt
 author_begin
   author_id demidko_olga.mariupol,news.ua.mrpl_city
 author_end
\fi

Не всім маріупольцям відомо, що в нашому місті є танцюваль\hyp{}но-театральний
колектив, який запропонував новий жанр театрального мистецтва. Тепер містяни
матимуть можливість ходити на вистави театру сучасної хореографії.

Цей колектив ще не має власної назви, але вже готовий вражати глядачів сучасним
високим театрально-танцювальним мистецтвом.

\ii{23_11_2020.stz.news.ua.mrpl_city.1.novyj_teatr_suchasnoi_horeografii.pic.1}

Три талановиті хореографи і керівниці танцювальних колективів \emph{\textbf{Олена Беркова}},
\emph{\textbf{Анна Паніотова}} та \emph{\textbf{Анастасія Петрова}} об'єдналися, щоб створити новий театр
сучасної хореографії. У кожної жінки своя доля і свій творчий шлях, але всі
вони єдині в любові до високого і сучасного танцювально-театрального мистецтва.
Пропоную ближче познайомитися з кожною з них. \emph{\textbf{Олена Беркова}} – випускниця
Московської балетної академії і мати трьох чарівних діточок. У 18 років, ставши
випускницею, Олена мала за плечима солідний досвід роботи на сцені і безліч
гастрольних поїздок – вона виступала в країнах Європи і Азії, Південної та
Північної Америки, в Африці і Японії. За розподілом дівчина потрапила до
московського Державного академічного театру класичного балету. Як солістка
танцювала па-де-де, па-де-труа і сольні партії практично у всіх балетах
класичного репертуару: \enquote{Лускунчик}, \enquote{Лебедине озеро}, \enquote{Жизель}, \enquote{Дон Кіхот},
\enquote{Спартак}. Театр давав вистави як в Росії, так і за кордоном. Повернувшись до
Маріуполя, саме тут вона створила велику сім'ю і почала передавати свої вміння
та знання юним дівчаткам, які мають хист до балету. Так, у 2001 році Олена
Володимирівна стала викладачкою хореографії. З 2010 року очолює громадську
організацію \emph{\enquote{Маріупольська школа класичної хореографії}}. Основним завданням
дитячої громадської організації було об'єднання дітей і молоді для
популяризації прекрасного мистецтва балету. Саме Беркова помітила талант у
\emph{\textbf{Анастасії Петрової}}, батькам якої сказала, що дівчинці потрібно обов'язково
продовжувати навчання. У 2007 Настя  вступила до київського Державного
хореографічного училища, яке закінчила у 2015 році (провчившись, як і Олена, 8
довгих років) та здобула кваліфікацію \enquote{Артист балету, викладач початкових
спеціалізованих мистецьких навчальних закладів}. З 2015 року Анастасія
працювала з балетною групою \enquote{Санкт-Петербург Фестиваль Балет}, гастролюючи
містами Європи. Після отриманої травми дівчину запросила до себе в колектив
Олена Беркова, адже їй не вистачало помічниці. З 2018 року дівчина працює в ПК
\enquote{Молодіжний} \emph{керівницею танцювального колективу \enquote{Балет – центр}}. За цей час
молода Анастасія Петрова створила низку концертних номерів, в яких
розкривається її талант як балетмейстера-постановника і як педагога. У грудні
цього року планує закінчити Харківський національний педагогічний університет
імені Г. С. Сковороди та  здобути ступінь магістра хореографії.

\ii{23_11_2020.stz.news.ua.mrpl_city.1.novyj_teatr_suchasnoi_horeografii.pic.2}

Ще одна учасниця колективу \emph{\textbf{Анна Паніотова}} закінчила Харківську державну
академію культури. Жінка танцювала в різних харківських шоу балетах. Пройшла
курси в Києві з Contemporary. Була учасницею багатьох майстер-класів відомих
вітчизняних та зарубіжних хореографів. Сьогодні Анна Паніотова є керівницею
колективу \emph{сучасного танцю \enquote{Імпульс}} при ПК \enquote{Молодіжний}, який був створений
жінкою у 2014 році. Жінка працює з дітьми від 4 до 15 років, діти вивчають
різні напрями сучасної хореографії. Колективи жінок є лауреатами і призерами
фестивалів та конкурсів різних рівнів.

\ii{23_11_2020.stz.news.ua.mrpl_city.1.novyj_teatr_suchasnoi_horeografii.pic.3}

Декілька місяців назад ці три мисткині створили новий колектив і у своїй
програмі \enquote{OMUT} представили два одноактних модерн-балети. Драматургія вистав
театру передається через хореографічну мову, мову модерн-балету і сучасну
танцювальну пластику. Перша постановка \emph{\enquote{Забуті мемуари}}, створена і перероблена
на основі хореографії \emph{Жана Філіпа Дюрі} (музика Генрі Перселла). У виставі бере
участь і драматична актриса \emph{Єлизавета Чагір}. Ще один модерн-балет колективу
\emph{\enquote{Бережися тихої води}} створений на основі постановки метра сучасної
хореографії, славетного іспанського танцівника і хореографа – \emph{Начо Дуато}. Ця
постановка вже більше 20 років ставиться в найкращих театрах світу. Музика –
\emph{\enquote{Шість романтичних вальсів}}, написаних іспанським композитором \emph{Енріке
Гранадосом}  в пам'ять про загиблу дружину. Хоча в оригіналі хореографія була
створена для чоловіків, три постановниці переробили її під жіноче виконання. У
своїх модерн-балетах актриси дуже тонко і талановито представили долю трьох
жінок і їхні пошуки щастя та зобразили різне розуміння людьми приближення
смерті.

\ii{23_11_2020.stz.news.ua.mrpl_city.1.novyj_teatr_suchasnoi_horeografii.pic.4}

Колектив тільки-но починає свій творчий шлях і наразі лише завдяки невгасаючому
ентузіазму та бажанню завоювати прихильність маріупольців створює нові
унікальні вистави. Найближчим часом тріо хоче поставити виставу \emph{\enquote{Форми тиші і
порожнечі}} на музику Й. Баха. На думку Олени Беркової, доведеться розширити
склад, оскільки в цьому балеті є масові сцени.

Вражає, що цей музично-театральний колектив не боїться експериментувати з
фарбами, звуками, світлом, режисурою та можливостями людського тіла. Водночас в
Маріуполі це єдиний колектив, де можна побачити балетні вистави, виконані на
високому професійному рівні.

