% vim: keymap=russian-jcukenwin
%%beginhead 
 
%%file 08_10_2021.fb.magomedov_musa.avdeevka.1.vojna
%%parent 08_10_2021
 
%%url https://www.facebook.com/musa.magomedov.311/posts/4716841391659571
 
%%author_id magomedov_musa.avdeevka
%%date 
 
%%tags 2014,avdeevka.donbass,donbass,mariupol,ukraina,vojna
%%title Для меня война началась в ночь с 21 на 22 июля в Мариуполе
 
%%endhead 
 
\subsection{Для меня война началась в ночь с 21 на 22 июля в Мариуполе}
\label{sec:08_10_2021.fb.magomedov_musa.avdeevka.1.vojna}
 
\Purl{https://www.facebook.com/musa.magomedov.311/posts/4716841391659571}
\ifcmt
 author_begin
   author_id magomedov_musa.avdeevka
 author_end
\fi

Наталья Емченко запустила флешмоб \#ОдинДень2014. Думаю, у каждого жителя
Донбасса есть свой день, который разделил жизнь на «до» и «после».

Для меня война началась в ночь с 21 на 22 июля в Мариуполе, я после совещания
поехал на дачу к своим друзьям на ночёвку. Они уехали туда с детьми из Донецка
ещё в мае, на пару недель..

\ii{08_10_2021.fb.magomedov_musa.avdeevka.1.vojna.pic.1}

Смотрели на море, пили вино, говорили о том что будем делать, когда «это всё»
закончится. Мы войной «это» ещё не называли, были локальные боевые действия то
там, то тут и мы больше об этом слышали, чем видели, это всегда было где-то
далеко или не очень, и казалось что вот-вот всё закончится. А оно ещё только
начиналось..

Звонок главного инженера Пастернака и его слова: «По нам стреляют, попали в
смолоперегонный цех, мы горим, горим!!!» казались чем-то нереальным,
невозможным, из какого-то сна. 

\ii{08_10_2021.fb.magomedov_musa.avdeevka.1.vojna.pic.2}

Мы жили, как оказалось - призрачными, надеждами, что по заводам и жилым
кварталам стрелять никто не будет, это же невозможно, там же живут и работают
люди, там же опасное производство.. опасной стала вся жизнь, независимо от
того, дома ты или на работе. Сон перестал быть основной потребностью и зависеть
от твоих биоритмов, оказалось что можно неделями спать по два-три часа в те
редкие часы, когда не стреляют. Оказалось что большинство того, что имело
ценность в обычной жизни, в условиях войны не представляет собой ничего. 

Еда. Вода. Связь. Тепло. Свет. Тишина. Всё это стало роскошью, на бесконечные
тёмные времена 14-го, 15-го и первой половины 17-го года.. 

И кратчайшими линиями между двумя точками стали не прямая, а две кривые,
главное- чтобы хоть одна привела в нужную точку и вовремя. 

И часы, когда нет связи с ребятами, которые ушли на ноль восстанавливать с
энергетиками оборванные линии электропередач, а там начался замес, а это ты их
туда отправил, потому что не отправить не мог- это вопрос жизни и смерти для
завода и города, который превращается в вопрос жизни и смерти для мужиков и для
тебя- потому что ты не живёшь все эти часы, пока они не вернутся.. 

«Брат» Алексей Дегтярёв, мой друг, которому два года днём и ночью в любое время
суток мог позвонить и сказать: высылай ребят, мы опять без вводов, у нас есть
сутки... Вживую встретились в 16-м с ощущением, что всю жизнь дружим, и ненадолго
разъезжались..

\begin{multicols}{2}
\ii{08_10_2021.fb.magomedov_musa.avdeevka.1.vojna.pic.3}
\ii{08_10_2021.fb.magomedov_musa.avdeevka.1.vojna.pic.3.cmt}

\ii{08_10_2021.fb.magomedov_musa.avdeevka.1.vojna.pic.4}
\ii{08_10_2021.fb.magomedov_musa.avdeevka.1.vojna.pic.4.cmt}

\ii{08_10_2021.fb.magomedov_musa.avdeevka.1.vojna.pic.5}
\ii{08_10_2021.fb.magomedov_musa.avdeevka.1.vojna.pic.5.cmt}
\end{multicols}

Не люблю я вспоминать про войну и не вспоминал бы, если бы она не продолжалась. 

И я не хочу никому пожелать увидеть её воочию, слышать вой мин, которые
прилетают и ложатся всё ближе и ближе, и осколки мерзко свистят над тобой и ты
думаешь: «только бы не убило, только бы не сюда, не в меня, сейчас закончится,
встану, сяду в машину, уеду и никогда сюда не вернусь..» и.. И не уезжаешь,
потому что это твоя жизнь, потому что люди в тебя верят и думают, что ты знаешь
что-то чего не знают они, и раз ты здесь, значит всё будет норм.

А вот тем, кто считает, что это не война, а «войнушка»- я бы хотел пожелать на
пару минут оказаться на передовой, чтобы понять каково это и научиться ценить
мир и тишину.. И тех кто отдаёт и сейчас свои жизни за нас с вами.

Уверен, Вячеслав Аброськин, Vitalii Barabash, Сергей Магера, есть что вспомнить
про 2014...
