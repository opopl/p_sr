% vim: keymap=russian-jcukenwin
%%beginhead 
 
%%file 22_09_2021.fb.fb_group.story_kiev_ua.1.evgenij_tarle_istorik_sssr
%%parent 22_09_2021
 
%%url https://www.facebook.com/groups/story.kiev.ua/posts/1758374117692715/
 
%%author_id fb_group.story_kiev_ua,olejnikov_maksim
%%date 
 
%%tags istoria,kiev,napoleon,rossia,rusimperia,sssr,stalin_iosif,tarle_evgenij.sssr.istorik,vojna,vojna.1812.rossia.napoleon
%%title Цей уродженець Києва ще за життя став найбільш відомим істориком тодішнього СРСР
 
%%endhead 
 
\subsection{Цей уродженець Києва ще за життя став найбільш відомим істориком тодішнього СРСР}
\label{sec:22_09_2021.fb.fb_group.story_kiev_ua.1.evgenij_tarle_istorik_sssr}
 
\Purl{https://www.facebook.com/groups/story.kiev.ua/posts/1758374117692715/}
\ifcmt
 author_begin
   author_id fb_group.story_kiev_ua,olejnikov_maksim
 author_end
\fi

Цей уродженець Києва ще за життя став найбільш відомим істориком тодішнього
СРСР, був тричі (!) нагороджений Сталінською премією, обраний дійсним членом
Академії наук СРСР, почесним доктором кількох університетів. 

Дякуючи професору Андрію Калістратовичу Буцику прізвище «Тарле» ледве не на все
життя запам’ятали кілька поколінь навіть найбільш ледачих студентів історичного
факультету КДУ…

Євген Вікторович Тарле (при народженні – Герц (Григорій) Вігдорович Тарле)
народився 8 листопада 1874 року у Києві в заможній єврейській родині. Батьком
майбутнього історика був розпорядник магазину готового одягу на Подолі купець
2-ї (пізніше вже 1-ї) гільдії Вігдор Григорович Тарле (наголос на першому
складі), мати – Розалія Арнольдівна Тарле, яка походила з сім’ї, в роду якої
було багато цадиків – знавців і тлумачів Талмуду. 


\ifcmt
  tab_begin cols=3

     pic https://scontent-frt3-1.xx.fbcdn.net/v/t1.6435-9/242618755_920277685228601_932454393492343639_n.jpg?_nc_cat=107&ccb=1-5&_nc_sid=b9115d&_nc_ohc=MYszJmQJ-4oAX-4qmDe&_nc_ht=scontent-frt3-1.xx&oh=31525d8de138f5a746f442da66b04954&oe=6171AA65

     pic https://scontent-frt3-1.xx.fbcdn.net/v/t1.6435-9/242512801_920277741895262_5653683773260732876_n.jpg?_nc_cat=104&_nc_rgb565=1&ccb=1-5&_nc_sid=b9115d&_nc_ohc=EwsKpR6hrdIAX_vEunc&_nc_ht=scontent-frt3-1.xx&oh=f31cc488301a6a65414a15be7252e63e&oe=61716103

		 pic https://scontent-frx5-1.xx.fbcdn.net/v/t1.6435-9/242514196_920277811895255_5648160325262456506_n.jpg?_nc_cat=111&_nc_rgb565=1&ccb=1-5&_nc_sid=b9115d&_nc_ohc=eDNqBj6idiwAX-DbNnf&tn=lCYVFeHcTIAFcAzi&_nc_ht=scontent-frx5-1.xx&oh=b7138a061c44241f5802145ba7671431&oe=6171E581

  tab_end
\fi

Зазнавши комерційної невдачі у Києві, родина повернулась на батьківщину глави
сім’ї до Херсону, де батько влаштувався комерційним агентом у страхове
товариство»Волга», а син закінчив Херсонську чоловічу гімназію № 1 і у
1892р.вступив до Новоросійського університету в Одесі на історико-філологічний
факультет. Через рік Тарле-молодший перевівся на аналогічний факультет до
Київського університету Св. Володимира. Будинок, де він тоді мешкав і звідки
ходив на лекції, досі зберігся – це № 10 на вулиці Паньківській. 

У Києві Тарле одружився на своїй ровесниці, православній дворянці Ользі
Григорівні Михайловій. Задля одруження він у серпні 1893р. хрестився у
Софійському соборі, перейшовши з юдаїзму у православ’я і у зв’язку з цим
обрядом прийняв ім’я «Євгеній». Разом подружжя прожило майже 60 років. 

Своє етнічне походження Тарле ніколи не приховував. Відома його фраза «Я не
француз, а еврей, и моя фамилия произносится Та́рле», сказана ним на лекції
восени 1951р. – в розпалі антисемітської кампанії в СРСР проти «безродних
космополітів»…

Закінчивши із золотою медаллю Київський університет у 1896р., Тарле залишився
на кафедрі загальної історії для підготовки до отримання професорського звання,
викладав у київських гімназіях, друкувався в журнальній періодиці, готував
статті до відомого енциклопедичного словника Брокгауза і Ефрона.

Як і багато студентів Київського університету того часу, він залучився до
марксистських гуртків соціал-демократів. Був арештований і висланий під нагляд
поліції за місцем проживання батьків до Херсона. Як «політично
неблагонадійному» йому заборонялось викладати у імператорських університетах і
казенних гімназіях. Згодом добився дозволу на погодинне викладання у
Петербурзькому університеті як приват-доцент, де він став одним з найбільш
популярних лекторів. Революційні події 1904-1905рр. захопили Тарле, він брав
участь у багатьох виступах опозиційної інтелігенції. Під час одного з таких
виступів він був поранений кінними жандармами. Цей випадок отримав широкий
розголос, про нього писали газети, вивішувалися медичні бюлетені, була випущена
спеціальна листівка з фотографією Тарле на лікарняному ліжку із забинтованою
головою. 

\ifcmt
  tab_begin cols=3

     pic https://scontent-frx5-1.xx.fbcdn.net/v/t1.6435-9/242502166_920277871895249_188314849144704812_n.jpg?_nc_cat=110&_nc_rgb565=1&ccb=1-5&_nc_sid=b9115d&_nc_ohc=lIJjTqWACTsAX-o0lGK&_nc_ht=scontent-frx5-1.xx&oh=568071e1f709ed774f981ac5845d49ec&oe=61702FFB

     pic https://scontent-frt3-1.xx.fbcdn.net/v/t1.6435-9/242599744_920277895228580_7809524520871877634_n.jpg?_nc_cat=102&_nc_rgb565=1&ccb=1-5&_nc_sid=b9115d&_nc_ohc=Rh0SpDy-U3AAX9XNdte&tn=lCYVFeHcTIAFcAzi&_nc_ht=scontent-frt3-1.xx&oh=bd6e251285cc55eaa4f24cfe25cd52ef&oe=6171D7F4

		 pic https://scontent-frt3-1.xx.fbcdn.net/v/t1.6435-9/242548823_920277945228575_5604946633949665096_n.jpg?_nc_cat=104&_nc_rgb565=1&ccb=1-5&_nc_sid=b9115d&_nc_ohc=p7tkPuzSLesAX8owwMW&_nc_ht=scontent-frt3-1.xx&oh=1894f74fcfc04fff9ec64cb3221708d0&oe=6170DE21

  tab_end
\fi

Є.В.Тарле вітав Лютневу революцію 1917р. і насторожено сприйняв революцію
Жовтневу. Однак відмовився емігрувати, хоча йому пропонували місце професора в
Сорбонні. Тим не менш, за радянської влади наукова і викладацька робота Тарле
складалась доволі успішно, він регулярно виїздив до Франції для роботи в
архівах, сприяв налагодженню розірваних більшовиками наукових контактів з
Європою. У 1921р. він - член-кореспондент Російської академії наук, у 1927р. -
дійсний член АН СРСР.

Але з осені 1929 по зиму 1931 років ОДПУ провело арешти 115 відомих
вчених-істориків «старої школи», зокрема і Є.В.Тарле, по т.зв. «Академічній
справі». За вигаданим звинуваченням вчені начебто планували змову з метою
повалення радянської влади, а Тарле у новому уряді відводився пост міністра
закордонних справ. Вчений провів півтора роки в ув’язненні, пройшовши погрози і
принизливі допити, був виключений з Академії наук і відправлений у заслання в
Казахстан.

Міжнародний резонанс і втручання у долю Тарле політичних і наукових діячів
Франції полегшили долю засланця. Йому дозволили викладати історію в
університеті Алма-Ати, згодом відновили на посаді професора Ленінградського
університету, а у 1938р. – і у званні академіка (за особистим  розпорядженням
Сталіна). Проте повністю Є.В.Тарле був реабілітований лише посмертно у 1967
році…

У 1936р. вийшла сама відома і популярна його монографія «Наполеон», що мала
величезний успіх у широкої читацької публіки. А Сталін оцінив її так: «Это,
конечно, немарксистская книга, но лучшая книга по истории, какую я когда-либо
читал».

Саме на прикладі головної наукової праці Тарле – «Наполеон», яка вважається
однією з кращих біографій Бонапарта у світовій історичній науці, можна
побачити, як радянська влада примушувала вчених переписувати історію. 

У виданні 1936р. Є.В.Тарле ставив під сумніви масштаби народної війни 1812р.,
вказуючи, що у французьких мемуарах відсутні згадки про партизанів у перші три
місяці війни в Росії. Натомість справжніми російськими партизанами були загони
під командуванням офіцерів регулярної армії (завжди дивувало, чому офіцер Денис
Давидов був «партизаном», замість служби у одній з армійських частин…). Також
Тарле вважав справедливим застосування терміну «народна війна» лише до війни
проти Бонапарта в Іспанії, а не в Росії. 

У 1937р. у «Правді» і «Известиях» одночасно вийшли дві статті з жорсткою
критикою «Наполеона», який назвали «образцом вражеской вылазки», а його автора
– «троцкистом-двурушником». Жах перед загрозою арешту позбавив Тарле сну. Він
прийняв снодійне, коли ж нарешті заснув, його розбудив телефон. Телефонував сам
Сталін, котрий запевнив, що критика в газетах помилкова і її завтра ж
приберуть. Таким чином «вождь» дав зрозуміти академіку, що йому треба уважніше
(тобто слухняніше) ставитись до побажань партії. І вже у перевиданні 1941р.
Тарле писав, що народна війна була також і у Росії, але в інших формах, ніж в
Іспанії… В іншій книзі, «Нашестя Наполеона на Росію 1812р.» Тарле наголошував
на провідній ролі у перемозі над Наполеоном саме народного спротиву, а не
морозу і безмежних просторів Росії. З книги Тарле також прибрав місця, в яких
йшлося про те, як російські війська грабували власні села через нестачу
провіанту. У такому ж патріотичному дусі Тарле написав книгу «Кримська війна»,
в якій вже посилався на Маркса і Енгельса, хоча насправді «батьки комунізму»
вкрай негативно оцінювали роль Росії у Кримській війні. Так один з кращих
радянських істориків взявся до переписування історії на догоду партії. І партія
цінувала його фальсифікацію історії війни 1812 року: три ордени Леніна, два
ордени Трудового Червоного Прапора, три Сталінскі премії…

Протягом останніх років свого життя Є.В. Тарле викладав, писав нові книги,
підручники історії, укладав збірники документів. Він був обраний почесним
доктором університетів Брно, Праги, Осло, Сорбонни, членом-кореспондентом
Британської академії, членом Норвезької Академії наук, Філадельфійської
академії політичних та соціальних наук у США. Слід визнати, що наукові твори
Тарле вирізняються блискучим літературним стилем, глибоким і колоритним описом
історичних постатей, завдяки чому здобули величезну популярність не тільки у
професійному середовищі, а й у широких колах читачів. Історико-біографічні
праці Тарле, присвячені Наполеону і Талейрану, досі вважаються класичними і
входили до неодмінного читацького «репертуару» радянської інтелігенції
1950-х-1970-х років.  

Помер Є.В.Тарле у 1955 році в Москві, похований на Новодєвічому цвинтарі.
