% vim: keymap=russian-jcukenwin
%%beginhead 
 
%%file slova.informacia
%%parent slova
 
%%url 
 
%%author 
%%author_id 
%%author_url 
 
%%tags 
%%title 
 
%%endhead 
\chapter{Информация}
\label{sec:slova.informacia}

%%%cit
%%%cit_head
%%%cit_pic
%%%cit_text
А  шефиня Центра \emph{информационной} безопасности (ЦИБ) Людмила Цибульская
сообщила, что будет заниматься тремя направлениями. А именно: стратегическими
коммуникациями, противодействием \emph{дезинформации} путем постоянного
\emph{информирования} и адвокацией (чем-чем? – Г.Б.) усилий государства и
общественных организаций на международном уровне. И поведала, что орган успел
запустить несколько «важных проектов».  Интересно следующее. «ЦПД может
получать в установленном порядке от органов исполнительной власти,
правоохранительных и разведывательных органов, прокуратуры, органов местного
самоуправления, предприятий и организаций, независимо от форм собственности и
подчинения, статистические данные, \emph{информацию}, справочные материалы и
т.п.» То бишь любая \emph{информация}, представляющая интерес для СНБО и его
\emph{«дезинформационного»} отростка, будет использоваться, даже несмотря на
то, что персональные данные граждан Украины вроде бы защищены законодательством
%%%cit_comment
%%%cit_title
\citTitle{Как борьба с «дезой» на Украине множит сущности}, Георгий Бородин, odnarodyna.org, 07.01.2021
%%%endcit


%%%cit
%%%cit_head
%%%cit_pic
%%%cit_text
Прежде, чем мы попадём в эпоху \emph{Информационную}, которая вот-вот наступит,
или уже наступила… нам придется пожить в нынешней эпохе,
\emph{Дезинформационной}.  Этот этап неприятный, но вынужденный, – как
прелюдия, увертюра, очищение пространства и разгон (от слова разгоняться).
Зато потом!!!  Представляю, как классно будет жить в \emph{Информационной
эпохе}, которая станет гармоничной, умной и излучающей озарения!  Там все
учатся наукам.., все друг с другом обмениваются йотобайтами.., атмосфера и
стратосфера превращены в ноосферу... Красота!
%%%cit_comment
%%%cit_title
\citTitle{Нынешняя Дезинформационная эпоха когда-то превратится в Информационную}, 
Владимир Спиваковский, strana.ua, 06.08.2021
%%%endcit

%%%cit
%%%cit_head
%%%cit_pic
%%%cit_text
США проиграли войну во Вьетнаме.  Только они проиграли не потому, что были
разгромлены. Не потому, что были ослаблены и не могли дальше продолжать.  Они
проиграли по другой причине. В \emph{информационной} войне.  Знаете как это произошло?
В 1968 коммунистические лидеры Северного Вьетнама были на грани поражения. Они
понимали, что против мощи США им не выстоять и если текущая тенденция
сохранится, то вскоре войну они проиграют. Их просто уничтожат постоянными
бомбардировками. В лагере даже звучали призывы о переговорах. И ситуацию нужно
было срочно менять.  Как?
%%%cit_comment
%%%cit_title
\citTitle{В современной войне можно выиграть благодаря правильной картинке}, 
Павел Вернивский, strana.ua, 09.08.2021
%%%endcit

%%%cit
%%%cit_head
%%%cit_pic
\ifcmt
  pic https://strana.ua/img/forall/u/0/36/QIP_Shot_-_Screen_365.png
  width 0.4
\fi
%%%cit_text
Но подключилась более тяжелая артиллерия, чем блогеры и экс-нардепы.
Минобороны Украины обещало провести беседу с Магучих. Об этом в воскресенье, 8
августа, сообщила замминистра главы ведомства Анна Маляр.  Замминистра
напомнила, что Магучих является младшим лейтенантом вооруженных сил Украины.
Она призвала украинцев прекратить травлю спортсменки, дождавшись ее официальных
объяснений.  При этом Маляр утверждала, что неосторожное поведение спортсменов
может становиться объектом "\emph{информационных спецопераций врага}".
"Спортсмены, представляющие Украину на международных соревнованиях, должны
понимать, что в Украине продолжается российско-украинская война и это
накладывает определенные ограничения и ответственность.  Верное следование
спортивным традициям, к примеру – фото объятий со спортсменами-гражданами РФ,
когда они становятся призерами, сразу использует враг как
\emph{информационно-психологическую пушку}, которая демонстрирует миру, что "их
тут нет", - написала Маляр на своей странице в соцсети
%%%cit_comment
%%%cit_title
\citTitle{\enquote{Поддерживают дискурс вражды}. Как власти кошмарят олимпийскую призерку Магучих за фото с россиянкой}, 
Анна Копытько, strana.ua, 09.08.2021
%%%endcit

%%%cit
%%%cit_head
%%%cit_pic
%%%cit_text
Представителей Донбасса, по мнению, сочинявших эту ересь неадекватов, в
принципе быть не может, а потому их следует называть «приглашенными к работе в
ТКГ лицами в составе российской делегации».  Вместо «Белоруссия» — «Беларусь».
Вместо «Приднестровская Молдавская Республика» — «Приднестровский регион
Молдовы».  Заметьте, сочинителей это \emph{информационно} «пурги» граждане
содержат за счет своих налогов!
%%%cit_comment
%%%cit_title
\citTitle{Краткий словарь грантоедов под редакцией СНБО / Лента соцсетей / Страна}, 
Александр Карпец, strana.news, 26.10.2021
%%%endcit
