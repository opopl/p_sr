% vim: keymap=russian-jcukenwin
%%beginhead 
 
%%file 15_05_2021.fb.bilchenko_evgenia.2.golubi_mira
%%parent 15_05_2021
 
%%url https://www.facebook.com/yevzhik/posts/3892774510757630
 
%%author 
%%author_id 
%%author_url 
 
%%tags 
%%title 
 
%%endhead 

\subsection{БЖ. Голуби мира}
\label{sec:15_05_2021.fb.bilchenko_evgenia.2.golubi_mira}
\Purl{https://www.facebook.com/yevzhik/posts/3892774510757630}

\ifcmt
  pic https://scontent-frt3-1.xx.fbcdn.net/v/t1.6435-9/186902130_3892786844089730_9136928111105458614_n.jpg?_nc_cat=104&ccb=1-3&_nc_sid=8bfeb9&_nc_ohc=uhbfOsciLkQAX-0H3ex&_nc_ht=scontent-frt3-1.xx&oh=ec368373010c61f5d93a0b71b68cf894&oe=60C45B2D
\fi

Они говорят: \enquote{Прекращай борьбу, прекращай борьбу.
Всё равно ведь не изменить судьбу.
Борьба - это гордыня, и всякий солдат не брит}.
В южном городе у меня брат до сих пор горит.

Но они говорят: \enquote{Прекращай борьбу, прекращай борьбу.
Нам Бог возвестил, что фашизм без нас кончится как-нибудь.
В непротивлении Льва Толстого - смысл завершить игру}.
Ребенок радистки Кэт простужается на ветру.

Но они все - такие умные и праведные, - вот, все.
Одна я заблудилась в овсе, на фронте, в слове, на полосе.
\enquote{Война - лекарствочко от морщин}? - Режется о стекло
Внутренний Цой. \enquote{Прекращай борьбу,} -  ответственностью за зло

Поделись. \enquote{Прекращай}. Твоя отработка кончена. Стоп, штрафбат.
Сделай, как мы, когда мы ушли: никто же не виноват.
Брестская улица оборвалась вдоль по пути в Рейхстаг.
\enquote{Прекращай борьбу, ибо есть кому за тебя водрузить твой стяг}.

А дальше - всегда про Христа, всегда. Обязательно - про Христа.
Как будто каждый из них там спал, от хлебов устав,
В Гефсиманском саду Дахау, вкусив от щедрот земли,
За которую не боролись, за которую не смогли

Отдать даже жизнь, то есть, - малость самую, тени тень.
\enquote{Прекращай борьбу}. Пятилетний шкет - фотография на кресте.
\enquote{Кто-то другой выручать придет, а совесть у нас - чиста,
Войной не замарана...}

Сыне Божий, сойди ко мне в бой с Креста!

15 мая 2021 г.

Сергей Никонов

И так, и так думают многие. Просто обычно без ярких образов. А правильный путь
- путь борьбы. Только форма боя бывает разная.  · Reply · 3h

Людмила Некрасова

Убей в себе дракона

Евгения Бильченко

Людмила Долгалева Чтобы там родились тысячи новых по Шварцу? Или чтобы стать Ланселотом?

Алексей Бажан

Евгения Бильченко Пьеса Шварца куда хитрее и изощреннее фильма Марка Захарова,
к тому же весьма амбивалентна. \enquote{Как уже было отмечено, \enquote{дракон} у Шварца - это
солярный символ. Таким образом, победа над Драконом - это победа над Солнцем.
Интересно отметить, что литературное произведение с таким названием
существовало и было хорошо известно Шварцу. Интересно также отметить, что это
была пьеса.  Речь идёт о футуристической \enquote{опере}, поставленной футуристами в
1913 году третьего и пятого декабря, в петербургском Луна-Парке. В подготовке
оперы принимала участие элита русского футуризма: автором сценария-либретто был
Крученых, пролог написал Хлебников, музыку создал Матюшин, оформил постановку
Малевич.} \url{http://haritonov.kulichki.net/stories/schw.html}

Tanya Ponomareva

У каждого - своя война. И прекратить борьбу нельзя. У каждого - свой способ
борьбы. И научить ему нельзя. Мы все - солдаты на передовой. И у каждого - свое
место в строю. И нельзя сказать - займи вон то, потому что я здесь стою. Я -
где надо стою.
