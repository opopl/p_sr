% vim: keymap=russian-jcukenwin
%%beginhead 
 
%%file 26_07_2021.stz.news.ua.mrpl_city.1.studia_tanciv_flamenko
%%parent 26_07_2021
 
%%url https://mrpl.city/blogs/view/studiya-tantsiv-flamenko
 
%%author_id demidko_olga.mariupol,news.ua.mrpl_city
%%date 
 
%%tags 
%%title Студія танців фламенко
 
%%endhead 
 
\subsection{Студія танців фламенко}
\label{sec:26_07_2021.stz.news.ua.mrpl_city.1.studia_tanciv_flamenko}
 
\Purl{https://mrpl.city/blogs/view/studiya-tantsiv-flamenko}
\ifcmt
 author_begin
   author_id demidko_olga.mariupol,news.ua.mrpl_city
 author_end
\fi

\ii{26_07_2021.stz.news.ua.mrpl_city.1.studia_tanciv_flamenko.pic.1}

У Маріуполі працює ще один унікальний і цікавий колектив. Це \emph{\textbf{Студія танців
фламенко}}, яка поєднала пристрасть, вогонь, акторську майстерність та неабиякий
драматизм. Нещодавно я особисто познайомилася з діяльністю цього самобутнього
колективу.

\ii{26_07_2021.stz.news.ua.mrpl_city.1.studia_tanciv_flamenko.pic.2_3}

Студія виникла у 2017 році. Починалося все буквально з 2–3 учасниць. І вже
цього року на сцену виходить близько 10 осіб. Керівниця студії \emph{\textbf{Наталя Гончарова}}
має режисерську освіту, танцями вона почала займатися ще у 2008 році. Танець
фламенко вперше побачила по телебаченню і одразу ж закохалася в цей
танцювальний напрям та зрозуміла, що це точно її стиль. Навчатися танцювати
фламенко Наталя почала в Мінську, де жила до переїзду в Маріуполь. В мінську
школу фламенко часто приїздили носії культури цього танцю – іспанці, унікальні
майстер-класи яких допомагали відточувати майстерність. За словами керівниці
студії, фламенко – це танець душі, який підходить для будь-якого віку. Він
народжується від пройденого досвіду, від того, що накопичилося і, що хочеться
висловити завдяки окремим рухам, які потрібно пережити та відчути, тому для
учасниць важлива і акторська майстерність... 

\ii{26_07_2021.stz.news.ua.mrpl_city.1.studia_tanciv_flamenko.pic.4}

Цікаво, що один і той же танець
можна станцювати зовсім по-різному: весело чи сумно. Водночас танець фламенко –
це індивідуальний танець, оскільки в ньому людина може розкрити себе. Наталя
Гончарова наголошує, що всі учасниці дуже працьовиті, сміливі та терплячі, адже
все виходило не одразу. Зі свого боку учасниці Студії у захваті від цього
пристрасного танцю і вважають, що він допомагає їм продовжити молодість,
подолати страх публічних виступів та стати ще більш жіночними. Зокрема, \emph{\textbf{Олена
Попова}} підкреслила, що вона в полоні фламенко і завдяки цьому танцю вчиться
управляти своїм тілом. Подобається учасницям і те, що вони швидко змінюють свої
ролі. Так танцівниці бувають і гордими іспанками, і веселими кокетками.
Сьогодні у колективу вже є свій невеликий, але дуже насичений репертуар.

\ii{26_07_2021.stz.news.ua.mrpl_city.1.studia_tanciv_flamenko.pic.5}

Якщо заглибитися в історію, то ми дізнаємося, що фламенко – південноіспанський
музичний, пісенний і танцювальний стиль, який склався в Андалусії наприкінці
XVIII-го сторіччя  (перша згадка в літературі – 1774 рік). Секрети техніки
цього танцю іспанці передавали з покоління в покоління. Істотно вплинула на цей
стиль і циганська музика – багато хто вважає основними, істинними носіями стилю
саме іспанських циган. Три століття відособленості і поневірянь зробили
фламенко ще більш унікальним танцем. Окремі рухи розповідають і про гіркоту
втрати батьківщини, і про небезпеку та передчуття нових доріг, радість здобуття
нової країни, знайомство з новим світом Іспанії. Cольні спів і танець,
супроводжуються грою на гітарі, кастаньєтах. По всьому світу шкіл фламенко не
так багато. В останні роки цей стиль отримав велику популярність у всьому
світі, особливо в Японії, де наразі існує більше навчальних закладів, ніж в
Іспанії. Важливою подією для цього танцю став коментар, наданий ЮНЕСКО в 2010
році: \emph{\enquote{Фламенко є одним із шедеврів усної та нематеріальної спадщини людства}}.
В Україні працює декілька танцювальних студій фламенко в Києві, Харкові,
Краматорську та Одесі. У Маріуполі колектив Студії танців фламенко –
першопрохідці. І керівниця, і учасниця сподіваються, що цей напрям буде
розвиватися в Маріуполі.

\ii{26_07_2021.stz.news.ua.mrpl_city.1.studia_tanciv_flamenko.pic.6}

Щорічно до карантинів у липні проводився фестиваль у Львові, де збирали всі
колективи, що танцюють фламенко, які повинні були представити свою творчість та
майстерність. Наталя Гончарова дуже хоче у найближчому майбутньому представити
на цьому фестивалі репертуар і маріупольської Студії.

Наразі танцівниці виступають на різних майданчиках міста. Це можуть бути і
флешмоби, і майстер-класи, де кожен охочий зможе відчути себе іспанцем, вивчити
основні рухи фламенко і навіть станцювати разом із колективом. Студію танців
фламенко можна побачити і в парку \enquote{Веселка}, і на Театральній площі.
