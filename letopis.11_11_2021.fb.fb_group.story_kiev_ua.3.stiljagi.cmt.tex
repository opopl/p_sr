% vim: keymap=russian-jcukenwin
%%beginhead 
 
%%file 11_11_2021.fb.fb_group.story_kiev_ua.3.stiljagi.cmt
%%parent 11_11_2021.fb.fb_group.story_kiev_ua.3.stiljagi
 
%%url 
 
%%author_id 
%%date 
 
%%tags 
%%title 
 
%%endhead 
\zzSecCmt

\begin{itemize} % {
\iusr{Тома Храповицкая}

\ifcmt
  ig https://scontent-frt3-1.xx.fbcdn.net/v/t39.30808-6/255490103_1780228092187002_5959204929586824861_n.jpg?_nc_cat=104&ccb=1-5&_nc_sid=dbeb18&_nc_ohc=PlcRkPgXgykAX_4bkAP&_nc_ht=scontent-frt3-1.xx&oh=00_AT-h3ykbRDKrEyqt4yJ34VyEsETQh8AsZ3VetBkiVEkqZw&oe=61CB8D83
  @width 0.3
\fi

\begin{itemize} % {
\iusr{Тома Храповицкая}
Папа, 1939 года рождения.

\iusr{Людмила Гаценко}
\textbf{Тома Храповицкая}
Гарний, елегантний.

\iusr{Тома Храповицкая}
\textbf{Людмила Гаценко} Дякую Людмила!!!

\iusr{Анна Шустерман}
\textbf{Тома Храповицкая} , стиляга. @igg{fbicon.flame} 

\iusr{Тома Храповицкая}
\textbf{Анна Шустерман} Ага! Было такое, и брюки-дудочки, туфли на \enquote{манной каше}...

\iusr{Анна Шустерман}
\textbf{Тома Храповицкая} , асоциальный элемент. @igg{fbicon.wink} 

\iusr{Тома Храповицкая}
\textbf{Анна Шустерман} Было... песочили...

\iusr{Людмила Гаценко}
\textbf{Анна Шустерман}
У кращому розумінні цього слова.
Денді.

\iusr{Людмила Гаценко}
\textbf{Анна Шустерман}
І який же гарний і приємний, з розумом у погляді, елемент.

\iusr{Людмила Гаценко}
\textbf{Тома Храповицкая}
І шийна хусточка замість галстука. Як гарно.

\iusr{Тома Храповицкая}
\textbf{Людмила Гаценко} Дякую Вам Людмила, шийну хусточку татові подарувала полячка на фестивалі молоді і студентів у 1957 році.

\iusr{Людмила Гаценко}
\textbf{Тома Храповицкая}
А ще носили шовкові хусточки. Мій друг носив такі.

\end{itemize} % }

\iusr{Николай Бурчин}
40-45 см брюки это от чего до чего замер?  @igg{fbicon.smile} 

\begin{itemize} % {
\iusr{Yuriy Bubnov}
\textbf{Николай Бурчин} От сих до сих. @igg{fbicon.face.wink.tongue} 

\iusr{Оксана Гриневич}
\textbf{Николай Бурчин} нижняя ширина брючины

\iusr{Николай Бурчин}
\textbf{Оксана Гриневич} дружинники ходили в клешах полуметровых?  @igg{fbicon.face.nerd} 

\iusr{Оксана Гриневич}
\textbf{Николай Бурчин} да. Сначала самые продвинутые модники)))

\iusr{Людмила Гаценко}
\textbf{Mykola Burchin}

У 90-і були такі брюки модні, а у жінок широкі (довгі й короткі) називалися
брюки- спідниця. Зараз мода повернулася, і тепер жіночі кльошні брюки (і знову
ж таки довгі й короткі) називаються шоти.

\end{itemize} % }

\iusr{Тамара Ар}
Надо же, как всех сейчас интересует прошлое время!

\begin{itemize} % {
\iusr{Наталя Кудря}
\textbf{Toma Ar} 

думаю про цей час і згадати нічого буде- крім масок і сертифікатів!!
@igg{fbicon.wink}  тому минуле й цікаве- просто жили!!

\iusr{Тамара Ар}
\textbf{Наталя Кудря} Это прекрасно, на самом деле! Но, мое мнение, о прошлом или хорошо, или ничего.

\iusr{Наталя Кудря}
\textbf{Toma Ar} минуле- це те єдине що не можуть забрати в людини!!! ну крім Айцгеймера- але тоді вже байдуже...
\end{itemize} % }

\iusr{Игорь Печерный}
Вот вспомнил пару строк из песенки тех лет:

\obeycr
Мы идем по Уругваю
Ночь, хоть выколи глаза
Слышны крики попугаев
И мартышек голоса
Я иду с своей красоткой
Брюки дудочки на мне
Обезьянья походка
Папироса на губе
\restorecr

\iusr{Oksana Garnets}

Ещё туфли на «манной каше» - толстой белой подошве из каучука и очень пестрый
галстук, если рубаха и пиджак. У меня старший двоюродный брат был стилягой. Как
мне нравилось!

\iusr{Всеволод Цымбал}

Папа моего друга дядя Виталик в молодости на танцах в клубе в Дарнице сплясал
рок-н-ролл, за что был задержан дружинниками и записан в стиляги

\iusr{Ludmila Debourdeau}

Расскажу вам одну историю, которая приключилась с моими родителями в 1957 году.
Маме только исполнилось 17 лет, папе было 18. Они были настоящими стилягами.
Мама на свой день рождения попросила брюки. И пошли мама с папой на Крещатик,
мама очень довольная в новых брюках, а папа в клетчатом пиджаке, брюки дудочки
и кок на голове. На встречу им комсомольский патруль. Маме порезали вдоль её
подарок, а папе отрезали кок. Мама была безутешна. А папа потом ходил в
институт с перебинтованной головой пока волосы не отросли. Так что всё так и
было. сейчас они об этом вспоминают с юмором.

\end{itemize} % }
