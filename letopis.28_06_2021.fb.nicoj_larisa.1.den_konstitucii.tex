% vim: keymap=russian-jcukenwin
%%beginhead 
 
%%file 28_06_2021.fb.nicoj_larisa.1.den_konstitucii
%%parent 28_06_2021
 
%%url https://www.facebook.com/nitsoi.larysa/posts/941553576411015
 
%%author Ницой, Лариса
%%author_id nicoj_larisa
%%author_url 
 
%%tags den_konstitucii.ukr,konstitucia,konstitucia.ukr,nicoj_larisa,ukraina
%%title День Конституції
 
%%endhead 
 
\subsection{День Конституції}
\label{sec:28_06_2021.fb.nicoj_larisa.1.den_konstitucii}
\Purl{https://www.facebook.com/nitsoi.larysa/posts/941553576411015}
\ifcmt
 author_begin
   author_id nicoj_larisa
 author_end
\fi

День Конституції. 

Так уже склалося, що я пишу казки про наболілі українцям теми, і навіть про
Конституцію в мене розповідається в \enquote{Казці про славного Орленка} - книжці  для
старшокласників та їхніх батьків. 

Так от, найбільша проблема нашої Конституції - це не Конституція, а ми,
громадяни України, які не дотримуємося  Конституції. У нас кожен другий -
конституційний професіонал. Усі ми великі знавці, що з нашою Конституцією не
так, і як її треба поправити, забуваючи, що можна внести купу правок, але, якщо
продовжувати їх не дотримуватися, то ... нічого не зміниться, будемо знову
розповідати, що винна Конституція, а не ми.

\ifcmt
  pic https://scontent-lga3-1.xx.fbcdn.net/v/t1.6435-9/207836631_941553136411059_734845232739800968_n.jpg?_nc_cat=105&ccb=1-3&_nc_sid=730e14&_nc_ohc=VbYSxEihu1IAX_vKtmY&_nc_ht=scontent-lga3-1.xx&oh=22f407ef399b7ef287c6096a71a5448b&oe=60E2087D
  width 0.4
\fi

От, наприклад, є в Конституції стаття 10 про державну мову, російську мову і
мови меншин. 25 років ми сперечаємося навколо цієї статті. 

Але КРАПКУ в цій суперечці давно, ще 22 роки тому, поставив Конституційний Суд
(КС). 

Ця уставнова (КС) існує саме для того, щоб давати роз'ясннення, кому що не
зрозуміло в Конституції. Що цікаво, якщо є  роз'яснення КС по якійсь статті, то
хто б ти не був, навіть якщо ти "господь Бог", то трактувати цю статтю
Конституції треба так, як сказав КС. А інше трактування вважається
неконституційним і карається штрафами і навіть в Кримінальному кодексі
передбачені штрафи або 3-5 років тюрми.   

Так от, усім українцям Конституційний Суд ще в 1999 році сказав: \enquote{Досить
сперечатися наколо мови і Конституції, статті 10. Українська мова є
обов'язковою, а інші мови національних меншин,  у тому числі і російська мова -
є НЕОБОВ'ЯЗКОВИМИ. Усі мови МОЖУТЬ вільно розвиватися (\enquote{можуть} означає: хочуть
- розвиваються, не хочуть - не розвиваються, ніхто їх не переслідуватиме), але
ОБОВ'ЯЗКОВА лише одна - УКРАЇНСЬКА мова. І держава повинна забезпечувати (тобто
фінансувати) лише державну мову}. Отак сказав Конституційний Суд. 

Чи припинилися після такого Рішення КС суперечки в нашому суспільстві навколо статті 10? Та нічого подібного. Не припинилися.

Маючи таке рішення, таке пояснення Конституційного суду, ми й далі розповідаємо
один одному на телебаченні, на нарадах, у міністерствах, урядам і президентам,
як треба розуміти цю статтю - і це розуміння, всупереч Рішенню КС, є різним. 

Ми всі є порушниками закону.  Зверніть увагу, порушниками є не лише 5 колона,
яка ініціює такі антизаконні дискусії, множачачи пояснення КС на нуль, а
порушниками є всі ми, які беремо 22 роки участь у цих дискусіях, замість дати
рішучу відсіч. Ми винні, що товчемо воду в ступі  - і спонукаємо своєю
нерішучістю далі розводити незаконні розмови порушниками. 

То хто нам винен? Конституція? Що ми своїми дискусіями (замість жорстких дій)
лише заохочуємо порушників порушувати Конституцію і самі стаємо співчучасниками
їхніх порушень.

Висновок. Ми повинні рішучо утвердити в Україні 10 статтю Конституції  -
українську мову. Це повинно стати завдянням кожного українця, бо від цього
залежить не лише національна безпека, а й економічне процвітання України.
Спільна мова формує спільний світогляд. Спільний світогляд допомагає сформувати
спільне бачення нашого майбутнього. А спільнобачення формує народ як спільну
команду та спільні шляхи побудови майбутнього до економічного процвітання.
Мовний роздрай створює роздрай в країні і приводить до неуспіху. 

\begin{verbatim}
  #КонституцІя_Ніцой
  #День_Конституції_Ніцой
\end{verbatim}
