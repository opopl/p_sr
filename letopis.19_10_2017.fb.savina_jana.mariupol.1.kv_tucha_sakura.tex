%%beginhead 
 
%%file 19_10_2017.fb.savina_jana.mariupol.1.kv_tucha_sakura
%%parent 19_10_2017
 
%%url https://www.facebook.com/permalink.php?story_fbid=pfbid02s521sFgNQCWfVsURqmMp1aagXnATg8AcyHASEhjD7nztQxTsTx6mzEe1c2Q5gmhvl&id=100011360442974
 
%%author_id savina_jana.mariupol
%%date 19_10_2017
 
%%tags mariupol,mariupol.pre_war,vystavka,isskustvo,japonia,kultura
%%title Квітуча сакура
 
%%endhead 

\subsection{Квітуча сакура}
\label{sec:19_10_2017.fb.savina_jana.mariupol.1.kv_tucha_sakura}

\Purl{https://www.facebook.com/permalink.php?story_fbid=pfbid02s521sFgNQCWfVsURqmMp1aagXnATg8AcyHASEhjD7nztQxTsTx6mzEe1c2Q5gmhvl&id=100011360442974}
\ifcmt
 author_begin
   author_id savina_jana.mariupol
 author_end
\fi

\enquote{Квітуча сакура}

У червні 2017 року стартувала Міська художня акція декоративно-прикладного та
образотворчого мистецтва «Квітуча Сакура». Вона розроблена відповідно до Указу
Президента України від 11.01.2017 року № 1 «Про оголошення 2017 роком Японії в
Україні». До участі в Акції були запрошені вихованці гуртків, студій, творчих
об'єднань декоративно-прикладного та образотворчого мистецтва і учні початкових
спеціалізованих мистецьких навчальних закладів Маріуполя у віці від 10 до 17
років.

Творчі роботи учасників акції представлені для перегляду в Центральній
бібліотеці ім. В.Г. Короленка. Виставка вийшла дуже яскравою і цікавою. В ній
налічується понад сімдесят робіт. Найактивніші учасники акції – учні
Маріупольської художньої школи ім. А.І. Куїнджі.

Японія – дивовижна країна, яка ретельно зберігає і шанує свої звичаї і
традиції. Надзвичайно різноманітне японське рукоділля: амігурумі, Канзас,
темарі, кусудама, осіе, кінусайга, терімен, фурошики, куміхімо, Сашико та ін.
Багато з них дуже популярні й в Україні, тому деякі роботи дітей виконані в
японських техніках прикладного мистецтва. Також на виставці представлені роботи
по темі «Японія» - це графіка, живопис, аплікації, флористика, різноманітні
композиції.

Відвідати «японську» виставку можна до кінця жовтня. Вхід вільний.
