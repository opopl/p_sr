% vim: keymap=russian-jcukenwin
%%beginhead 
 
%%file slova.feeria
%%parent slova
 
%%url 
 
%%author 
%%author_id 
%%author_url 
 
%%tags 
%%title 
 
%%endhead 
\chapter{Феерия}

%%%cit
%%%cit_pic
\ifcmt
  pic https://avatars.mds.yandex.net/get-zen_pictures/2032820/589758338-1623855207922/orig
	caption Александр Грин, Алые Паруса, обложка
\fi
%%%cit_text
Это не жанр книги (произведения) - \emph{феерия}. \emph{Феерия} - это
представление для Ассоль и жителей городка Каперны, которое разыгрывает автор в
книге, при помощи главного героя капитана Грэя, Грей всё \enquote{волшебство}
создаёт своими руками - разыгрывает феерию... Собственно, по этому и нет других
книг этого жанра, из-за неверного толкования смысла, написанного автором в
заглавии слова \emph{феерия}...  Здесь, \emph{феерия} - не жанр литературного
произведения...  В первом издании 1923 года издательства Л. Д. Френкеля
написано, на обложке: А. С. ГРИН АЛЫЕ ПАРУСА, на первой странице (титульный
лист) - А. С. ГРИН АЛЫЕ ПАРУСА/ПОВЕСТЬ.  Артур Грэй: «Я понял одну нехитрую
истину — чудеса надо делать своими руками. Если душа человека жаждет чуда —
сделай ему это чудо. Новая душа будет у него и новая у тебя»
%%%cit_comment
Литошко Олег, \url{https://zen.yandex.ru/user/df54khxv51y078405mjtfxa288}
%%%cit_title
\citTitle{В двадцатом веке было написано только три феерии. Одна бельгийская, две наши}, 
ЛИТИНТЕРЕС, zen.yandex.ru, 03.06.2021
%%%endcit

%%%cit
%%%cit_head
%%%cit_pic
%%%cit_text
Ну что ж, первая литературная \emph{феерия} двадцатого века у нас есть, она
бельгийская. Осталось еще две.  Перепрыгиваем через несколько лет и попадаем в
1911 год. На этот раз мы отправляемся на территорию Российской империи.
Поэтесса Леся Украинка, которая последние годы жизни проводит в Грузии, пишет
пьесу \enquote{Лесная песня} и дает ей подзаголовок \emph{драма-феерия}. Эту пьесу позже
несколько раз экранизировали в Советском Союзе.  Ну а третью \emph{феерию} вы хорошо
знаете. Она появилась уже в 1923 году и знаменовала собой новую эволюцию этого
термина. Впервые к \emph{жанру феерии} был отнесен целый роман. Автор его - Александр
Грин, а называется книга \enquote{Алые паруса}
%%%cit_comment
%%%cit_title
\citTitle{В двадцатом веке было написано только три феерии. Одна бельгийская, две наши}, 
ЛИТИНТЕРЕС, zen.yandex.ru, 03.06.2021
%%%endcit

