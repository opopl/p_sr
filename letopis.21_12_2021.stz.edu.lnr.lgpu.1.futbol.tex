% vim: keymap=russian-jcukenwin
%%beginhead 
 
%%file 21_12_2021.stz.edu.lnr.lgpu.1.futbol
%%parent 21_12_2021
 
%%url http://lgpu.org/news/8773-tovarischeskaya-vstrecha-po-voleybolu-proshla-v-lgpu.html
 
%%author_id 
%%date 
 
%%tags 
%%title Товарищеская встреча по волейболу прошла в ЛГПУ
 
%%endhead 
\subsection{Товарищеская встреча по волейболу прошла в ЛГПУ}
\label{sec:21_12_2021.stz.edu.lnr.lgpu.1.futbol}

\Purl{http://lgpu.org/news/8773-tovarischeskaya-vstrecha-po-voleybolu-proshla-v-lgpu.html}

В Луганском государственном педагогическом университете 21 декабря состоялась
игра по волейболу между представителями профессорско-преподавательского состава
и студентами вуза.

\ii{21_12_2021.stz.edu.lnr.lgpu.1.futbol.pic.1}

Мероприятие прошло в соответствии с Требованиями ЧСПК по организации работы
образовательных организаций среднего профессионального и высшего образования
ЛНР на 2021/2022 учебный год в условиях сохранения рисков распространения
COVID-19 в спортивном зале первого корпуса вуза.

\ii{21_12_2021.stz.edu.lnr.lgpu.1.futbol.pic.2}

Организатором спортивного мероприятия выступил спортивный клуб «Буревестник»
совместно с кафедрой физического воспитания Института физического воспитания и
спорта (ИФВС) при поддержке Первичной профсоюзной организации преподавателей и
сотрудников университета.

\ii{21_12_2021.stz.edu.lnr.lgpu.1.futbol.pic.3}

В состав сборной команды преподавателей ЛГПУ вошли: председатель
Попечительского совета ЛГПУ, депутат Народного Совета ЛНР Павел Пилавов;
педагоги кафедры физического воспитания ИФВС – заведующий Валерий Слепцов,
доцент Андрей Лимонченко; старшие преподаватели Елена Асташова, Анна Олефир,
Владлен Кострыкин, преподаватели Елена Дюбина и Даниил Колпаков; заведующий
геологическим музеем, ассистент кафедры географии факультета естественных наук
Виталий Рыбалченко; заведующий столовой университета Станислав Деркульский.

\ii{21_12_2021.stz.edu.lnr.lgpu.1.futbol.pic.4}

Их соперниками стала команда студентов. Обучающиеся Института
физико-математического образования, информационных и обслуживающих технологий
Станислав Алексеев, Сергей Перебейнос, Илья Андрющенко; студенты Института
истории, международных отношений и социально-политических наук Игорь Дьячек и
Николай Хахонин; представители филологического факультета Кирилл Лабезник и
Валерия Зоричева, студентка Института педагогики и психологии Валерия Щеткина,
обучающаяся кафедры адаптивной физической культуры и физической реабилитации
Полина Богачева, студент факультета музыкально-художественного образования
имени Джульетты Якубович Михаил Бабенко объединились в сборную под руководством
доцента кафедры физического воспитания ИФВС Надежды Максимовой и старшего
преподавателя кафедры физического воспитания ИФВС Наталии Васецкой.

Активную поддержку участникам товарищеской встречи оказывали болельщики –
представители администрации вуза, педагоги, сотрудники вуза. Среди гостей матча
были ректор ЛГПУ, депутат Народного Совета ЛНР Жанна Марфина, первый проректор
вуза Юрий Филиппов, директор ИФВС Наталья Павлова. Спортсменов воодушевляла и
группа поддержки филологического факультета «FFamily».

Команды сыграли три партии, продолжавшиеся до 25 очков. Абсолютным победителем
стала сборная команда преподавателей и сотрудников ЛГПУ, одержавшая три победы
со счетом: 25:15, 25:23, 25:20.

Участники игры были награждены грамотами. Сборная команда преподавателей и
сотрудников коллегиально приняла решение передать призы от Первичной
профсоюзной организации сотрудников и преподавателей ЛГПУ
студентам-волейболистам.

\begin{zznagolos}
\normalsize
– Это была напряженная и интересная игра. Здоровый образ жизни и занятия
спортом – неотъемлемая часть образа современного человека и замечательно, что
молодежь ЛГПУ придерживается подобных постулатов, а педагоги вуза поддерживают
их и мотивируют собственным примером, - прокомментировал спортивное событие
участник игры Павел Пилавов. - Спасибо организаторам за возможность с пользой
провести время  и померяться силами со студентам. Благодарю участников матча за
желание развиваться физически и болельщиков – за любовь к спорту!
\end{zznagolos}

Поздравляем победителей и участников игры!

Пресс-центр университета,

фото Алексея Волобуева и Дарины Малыхиной
