% vim: keymap=russian-jcukenwin
%%beginhead 
 
%%file 10_04_2023.stz.news.ua.donbas24.2.velykyj_obmin_polonenymy_80_geroiv_20_geroinj
%%parent 10_04_2023
 
%%url https://donbas24.news/news/velikii-obmin-polonenimi-v-ukrayinu-povernulisya-shhe-80-geroyiv-ta-20-geroyin-foto
 
%%author_id news.ua.donbas24,sorokina_natalia.lugansk.zhurnalist.donbas24
%%date 
 
%%tags 
%%title Великий обмін полоненими: в Україну повернулися ще 80 героїв та 20 героїнь
 
%%endhead 
 
\subsection{Великий обмін полоненими: в Україну повернулися ще 80 героїв та 20 героїнь}
\label{sec:10_04_2023.stz.news.ua.donbas24.2.velykyj_obmin_polonenymy_80_geroiv_20_geroinj}
 
\Purl{https://donbas24.news/news/velikii-obmin-polonenimi-v-ukrayinu-povernulisya-shhe-80-geroyiv-ta-20-geroyin-foto}
\ifcmt
 author_begin
   author_id news.ua.donbas24,sorokina_natalia.lugansk.zhurnalist.donbas24
 author_end
\fi

\ii{10_04_2023.stz.news.ua.donbas24.2.velykyj_obmin_polonenymy_80_geroiv_20_geroinj.pic.front}
\begin{center}
  \em\color{blue}\bfseries\Large
  Серед звільнених з російського полону — оборонці Маріуполя, Бахмута, ЧАЕС, ЗАЕС та з інших напрямків
\end{center}

Україні сьогодні, 10 квітня, вдалося провести черговий великий \href{https://donbas24.news/news/viplati-rodinam-viiskovopolonenix-yak-otrimati}{\emph{обмін полоненими.}}%
\footnote{Виплати родинам військовополонених — як отримати, Тетяна Веремєєва, donbas24.news, 08.04.2023, \par\url{https://donbas24.news/news/viplati-rodinam-viiskovopolonenix-yak-otrimati}}

На Батьківщину повернулися 80 наших захисників та 20 захисниць. Серед них,
зокрема, оборонці Маріуполя, Бахмута, Запорізької АЕС, Київщини, острова
Зміїний.

\textbf{Читайте також:} \href{https://donbas24.news/news/zaboroniv-vtracati-sebe-voyin-zsu-rodom-z-donecka-pro-te-yak-vdalosya-pereziti-321-den-polonu}{\emph{\enquote{Заборонив втрачати себе}: воїн ЗСУ родом з Донецька про те, як вдалося пережити 321 день полону}}%
\footnote{\enquote{Заборонив втрачати себе}: воїн ЗСУ родом з Донецька про те, як вдалося пережити 321 день полону, Наталія Сорокіна, donbas24.news, 29.03.2023, \par%
\url{https://donbas24.news/news/zaboroniv-vtracati-sebe-voyin-zsu-rodom-z-donecka-pro-te-yak-vdalosya-pereziti-321-den-polonu}%
}

Про це передає Донбас24 з \href{https://t.me/Koord_shtab/811}{посиланням}%
\footnote{\url{https://t.me/Koord_shtab/811}}
на Координаційний штаб з питань поводження з військовополоненими.

\ifcmt
  tab_begin cols=2,no_fig,center,separate,no_numbering
     pic https://i2.paste.pics/PS131.png?trs=1142e84a8812893e619f828af22a1d084584f26ffb97dd2bb11c85495ee994c5
     pic https://i2.paste.pics/PS13A.png?trs=1142e84a8812893e619f828af22a1d084584f26ffb97dd2bb11c85495ee994c5
  tab_end
\fi

Як повідомляється, визволити вдалося 24 нацгвардійців, 22 прикордонника, 22
представника Військово-морських сил, 21 військовослужбовця Збройних сил України
та 11 тероборонців. Вони брали участь у боях на Херсонському, Запорізькому,
Донецькому, Харківському, Сумському, Київському напрямках.

Майже половина з колишніх бранців мають важкі поранення, хвороби або зазнали
катувань. Наймолодшому зі звільнених героїв — всього 19 років.

\ii{10_04_2023.stz.news.ua.donbas24.2.velykyj_obmin_polonenymy_80_geroiv_20_geroinj.pic.1_3}

\ifcmt
  tab_begin cols=2,no_fig,center,separate,no_numbering
     pic https://i2.paste.pics/PS16S.png?trs=1142e84a8812893e619f828af22a1d084584f26ffb97dd2bb11c85495ee994c5
     pic https://i2.paste.pics/PS1CB.png?trs=1142e84a8812893e619f828af22a1d084584f26ffb97dd2bb11c85495ee994c5
  tab_end
\fi

\begin{leftbar}
\enquote{Це був непростий обмін, і я вдячний всій команді, Координаційному штабу з
питань поводження із військовополоненими за те, що кожен робить те, що часто
багатьом може здаватися неможливим. Робить свою роботу, дуже важливу
сьогодні, — \href{https://t.me/ermaka2022/2414}{написав} %
%\footnote{\url{https://t.me/ermaka2022/2414}}
у своєму телеграм-каналі керівник Офісу Президента
України \textbf{Андрій Єрмак}. — Особливо радий повідомити чудову новину про обмін
родичам звільнених з полону людей. Вони довго чекали своїх чоловіків, жінок,
батьків вдома. Очікування — це завжди дуже складний та нервовий процес}.
\end{leftbar}

\ii{10_04_2023.stz.news.ua.donbas24.2.velykyj_obmin_polonenymy_80_geroiv_20_geroinj.pic.4_6}

Нагадаємо, що Україна докладає максимум зусиль, щоб якнайшвидше повернути з
російського полону усіх своїх громадян. Обмін полоненими з країною-агресоркою
відбувається на постійні основі: минулий \href{https://donbas24.news/news/znovu-vdoma-z-polonu-povernulisya-desyat-zaxisnikiv-donbasu-ta-dvoje-civilnix-video}{\emph{було проведено 3 квітня}}.%
\footnote{Знову вдома: з полону повернулися десять захисників Донбасу та двоє цивільних, Наталія Сорокіна, donbas24.news, 03.04.2023, \par\url{https://donbas24.news/news/znovu-vdoma-z-polonu-povernulisya-desyat-zaxisnikiv-donbasu-ta-dvoje-civilnix-video}}

Ще більше новин та найактуальніша інформація про Донецьку та Луганську області
в нашому телеграм-каналі Донбас24.

Фото: Координаційний штаб з питань поводження з військовополоненими

%\ii{10_04_2023.stz.news.ua.donbas24.2.velykyj_obmin_polonenymy_80_geroiv_20_geroinj.txt}
