% vim: keymap=russian-jcukenwin
%%beginhead 
 
%%file 11_11_2021.fb.fb_group.story_kiev_ua.1.pamjat_slavnoj_kievljanki.cmt
%%parent 11_11_2021.fb.fb_group.story_kiev_ua.1.pamjat_slavnoj_kievljanki
 
%%url 
 
%%author_id 
%%date 
 
%%tags 
%%title 
 
%%endhead 
\subsubsection{Коментарі}

\begin{itemize} % {
\iusr{Natasha Levitskaya}

Спасибо, Георгий, за этот пост памяти о благородных и душевно-богатых
Киевлянах!

К сожалению, таких людей уже нет, их уже не производят...

Чем больше читаешь о Леоноре Натановне, чем больше узнаешь от людей, которым
повезло встречаться и общаться с нею, тем больше восхищаешься этой невероятной
женщиной, которая так любила свой Город, знала его и дарила свои знания людям.

Очень понравился один комментарий, написанный 6 лет назад о ней: "Лена
настолько приподнимала нас над обыденностью, что порой мы чувствовали себя как
бы персонажами Шагаловских картин!"

В свете недавних и горячих споров в нашей группе о переименовании улиц, о
героях, о новоявленных друзьях из далёких стран, чьими именам называют нынче
улицы Киева, хочется надеяться, что всё- таки там "наверху" найдутся разумные
люди и в Киеве, не на окраинах среди новоделов, появятся улицы знаменитых
киевлян, и поколение тик токов и Инстаграм будет знать хот бы так о великих
людях города.

Светлая и добрая память Вам, Элеонора Натановна!

\iusr{Maria Rasin}

Потрясена! Натан Рахлин - фамилия с которой я прожила всю жизнь в Киеве и
такие унижения, чтобы повесить эту несчастную доску таким именам? Стыд и позор
! Люблю свой город, но после некоторых историй настоящее отравление .....

\iusr{Валентина Татаржинская}
Дуже цікаве оповідання. Я згадала це прізвище деригента. А його історії раніше не знала. Дякую

\iusr{Алик Старшой}
Спасибо за такой пост....

\iusr{Наталия Руденко}
Несправедливо, когда величие зависит от бездарности...

\end{itemize} % }
