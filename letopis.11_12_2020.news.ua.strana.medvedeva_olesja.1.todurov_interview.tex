% vim: keymap=russian-jcukenwin
%%beginhead 
 
%%file 11_12_2020.news.ua.strana.medvedeva_olesja.1.todurov_interview
%%parent 11_12_2020
 
%%url https://strana.ua/articles/interview/305908-boris-todurov-intervju-olese-medvedevoj-video-i-tekst.html
 
%%author Медведева, Олеся
%%author_id medvedeva_olesja
%%author_url 
 
%%tags todurov_boris
%%title Борис Тодуров: "Супрун к нам прислали как агента международных корпораций"
 
%%endhead 
 
\subsection{Борис Тодуров: \enquote{Супрун к нам прислали как агента международных корпораций}}
\label{sec:11_12_2020.news.ua.strana.medvedeva_olesja.1.todurov_interview}
\Purl{https://strana.ua/articles/interview/305908-boris-todurov-intervju-olese-medvedevoj-video-i-tekst.html}
\ifcmt
	author_begin
   author_id medvedeva_olesja
	author_end
\fi

\index[names.rus]{Тодуров, Борис!Директор Института Сердца, Киев}

\ifcmt
  pic https://strana.ua/img/article/3059/boris-todurov-intervju-8_main.jpeg
  caption Борис Тодуров и Олеся Медведева. Кадр из фильма-интервью 
\fi

Знаменитый украинский кардиохирург Борис Тодуров дал интервью спецкору "Страны"
и автору блога "Ясно.Понятно" Олесе Медведевой. 

Олеся провела день в Институте сердца, которым руководит Борис Михайлович.

Беседой дело не ограничилось – Тодуров показал журналистке "Страны", как он
делает свои операции: практически нон-стоп, перемещаясь из одной операционной в
другую.

Часть Института сердца сегодня переведена под ковидных больных. В перерывах
между операциями мы поговорили с Борисом Тодуровым о том, когда закончится
пандемия в Украине и как дошло до такой ситуации с коронавирусом, которую мы
имеем сейчас.  

Врач рассказал "Стране", почему уникальный институт с первых же дней работы в
Минздраве невзлюбила Ульяна Супрун. И какой вклад ее реформ в ту медицинскую
катастрофу, которая наблюдается сегодня в ходе пандемии. 

Также Тодуров объяснил, что Украине нужно делать с российской вакциной от
коронавируса. Детальнее – в фильме Олеси Медведевой.

\video{https://youtu.be/M2z7UjtjZqw}

Также ниже приводится текстовая расшифровка интервью. 

\subsubsection{\enquote{Мы взяли в пример худшую в Европе медицину}}

– Когда мы входили в пандемию, наша система здравоохранения оказалась абсолютно
к этому не готова. Несмотря на то, что вот уже несколько лет подряд вроде как
всем рассказывали о том, что была очень эффективная и продуктивная медицинская
реформа. Почему так вышло? Почему мы оказались совершенно не готовыми к тому,
что мы имеем сейчас?

– Несколько лет назад началась так называемая реформа, на самом деле большая
авантюра, которая привела к полной потере управленческой вертикали. В медицине,
в отличие от административных вещей, нельзя было делать децентрализацию, потому
что Украина – большая многомиллионная страна. Мы имеем опыт Чернобыля, мы имеем
опыт техногенных катастроф, и мы должны понимать, что что-то аналогичное может
случиться в любой момент. У нас четыре атомных станции. У нас огромное
количество заводов. В том числе тех, которые работают с химией, с какими-то
препаратами и так далее. Мы уже пережили несколько эпидемий, и мы должны быть
готовыми к тому, чтобы эта управленческая вертикаль работала безукоризненно.

У нас была до 2017 года служба санэпидемстанции, которая контролировала
санитарно-эпидемиологическую обстановку, которая могла реагировать адекватно на
изменение этой обстановки, которая контролировала основные инфекции, которые
появлялись в Украине. Это и дифтерия, и корь, и грипп, и многое другое.

Эта служба была заточена с советских времен на борьбу с этими эпидемиями. И
нужно отдать ей должное: та советская система, которую сейчас все так хают,
именно она победила эти основные эпидемии в то время в Советском Союзе в
послевоенные и довоенные годы. И туберкулез победила, и корь, и дифтерию, и
сибирскую язву, и брюшной тиф, и много-много того, что уносило тысячи жизней. И
это была абсолютно четко построенная система, рассчитанная на профилактику
заболевания и на предотвращение заболевания.

Когда я был школьником, были диспансерные осмотры. Ты рождался, на следующий
день после выписки приходила патронажная сестра из детской поликлиники, она
контролировала все прививки, была карточка в детской поликлинике на ребенка до
16 лет, делались все прививки, никто это не обсуждал, ни у кого это не
спрашивали, и была ситуация, которая до сих пор нас защищает.  

Те прививки АКДС (обеспечивают ребенку иммунитет от коклюша, дифтерии и столбняка. – Ред.) до сих пор защищают бывших жителей Советского Союза даже от сегодняшнего ковида. Вы знаете, что меньше болеют. 

– Так это доказанный факт? Просто очень много противоречивых слухов по этому поводу. 

– Да, статистика по Восточной и Западной Германии говорит о том, что в
Восточной Германии, где живет большинство людей, привитых АКДС, болеют на 30\%
меньше. 

Эта система каким-то образом существовала до последних лет, и вдруг кто-то
решил, что не нужно вкладывать в эту систему деньги. Она, собственно,
существовала на очень экономные деньги – порядка 2,8-3\% от бюджета. И эта
система была более-менее рабочей. Кто-то сказал, что "плохая система",
советская, типа система Семашко не годится никуда, давайте будем делать
реформу.

– А как вы думаете, кто сказал, что нам это не подходит?

– Это те, кто нам последние годы говорят после революции, куда нам надо
двигаться, какую демократию строить, на каком языке разговаривать, какому Богу
молиться, в какую церковь ходить. Кто-то решил из этих людей, что нужно и здесь
немножко подкорректировать, забыв при этом профинансировать эту реформу. 

Такая реформа требовала как минимум 7\% в бюджет. Так, как это делается в
западных странах. Минимум 7\%, а в некоторых странах и 12\% из бюджета страны
тратится на медицину. Вот это могла бы быть реформа, когда под нее бы подложили
какую-то экономическую базу, какой-то экономический расчет. Но, как правило, у
нас этим сильно не озадачиваются. Надо сломать. Знаете, вот последние годы у
нас не строят, а ломают. Вот это было плохо, это был Совок, его надо поломать.

Но мало у кого есть представление, что же построить вместо этого. И начали
строить некую английскую якобы систему, которая в Англии из английского бюджета
потребляет 12\% или 13\%. В десятку бюджетов входит мировых и считается в
Европе наихудшей медициной.

И вот мы решили вдруг построить именно английский вариант с 3\% вместо 13\%,
разрушив при этом полностью и убрав, поломав всю санитарно-эпидемиологическую
службу, а это лаборатории, это кадры, это санстанции в каждом районе, области,
городе. И огромное количество людей, которые контролировали санитарное
состояние.

\ifcmt
pic https://strana.ua/img/forall/u/0/92/%D0%A2%D0%BE%D0%B4%D1%83%D1%80%D0%BE%D0%B2_%D0%BA%D1%80%D1%83%D0%BF%D0%BD%D1%8B%D0%B9_%D0%BF%D0%BB%D0%B0%D0%BD.png
caption Борис Тодуров. Кадр из фильма Олеси Медведевой 
fig_env wrapfigure
width 0.4
\fi

– А как вы считаете, мы бы лучше справились в нашей стране, если бы в свое
время не ликвидировали эту службу?

– А чего тут думать? Когда в Одессе в 70-е годы возникла холера, именно та
советская санстанция сработала. И это не понеслось по всему Советскому Союзу.
Был создан комитет, который включал в себя и военных, и внутренние войска.
Одесса была окружена внутренними войсками. Был карантин жесткий. Никто не
заезжал, не выезжал, пока не ликвидировали очаг холеры. 

– И не было социальных сетей, никто не жаловался и не кричал по этому поводу.

– И мало кто знал в Советском Союзе тогда, что в Одессе такое случилось. А как
боролись с чумой в Советском Союзе? Когда в Москву вдруг привезли из Индии
чуму. И в течение недели или двух полностью ликвидировали все очаги. Это
работала служба. Она включала в себя все инстанции, начиная от КГБ тогдашнего,
внутренних войск, медиков, эпидемиологической службы и так далее. Это все
объединялось в одну какую-то структуру, и ликвидировали очаги эпидемии. То же,
что было в Китае. Они очень быстро справились, благодаря той службе, которая у
них есть, государственная. 

– Мы вроде как начинали перенимать их опыт, почему все прекратилось? Почему
людям до сих пор не хватает кислорода? Почему за семь месяцев эпидемии мы не
справляемся, ведь есть и коронавирусный фонд?

– Потому что нет ответственности за это. Нет человека, который бы отвечал за
это сегодня. Министра лишили всех возможностей и рычагов управления. Сегодня
министр не имеет возможности влиять на районные, городские, областные больницы.
Они сегодня подчинены полностью децентрализации, они подчинены только местным
властям. Санитарной службы как таковой нет. Нет структуры, нет кадрового
потенциала и нет лабораторий, которые бы взяли на себя какую-то работу по
лабораторному выявлению быстрому. Не произошло создание какой-то комиссии, она
должна была быть, на мой взгляд. Как бы я делал с самого начала: есть у нас
СНБО, структура, которой подчиняются в той или иной степени все остальные, да?
И нужно было создать какой-то комитет спасения, как хотите его назовите, куда
бы входили и МВД, и военные, и медики, и районные власти, и все остальные.

Какой алгоритм действий должен быть?

Если есть эпидемия, то какой кадровый потенциал, и его надо усилить; и если его
не хватает, значит нужно тут же начать подготовку врачей другого профиля.
Начитывать им лекции, создать протокол лечения. Тут же локализовать очаги
эпидемиологические. Туда включается и полиция, служба безопасности, какие-то
структуры внутренние, которые могут отслеживать людей и по телефонам и так
далее. Пограничная служба должна быть туда включена и работать абсолютно четко,
чтобы пресекать поступление.  

А дальше – материально-техническая база. Самое сложное. Вы говорите "кислород",
да у нас ПЦРов не купили. У нас анализаторы ПЦР во многих областях выдают
результат через пять дней. До сих пор, спустя восемь месяцев... Кислород,
койки, мониторы, дыхательные аппараты, газовые анализаторы, пульсоксиметры,
маски, шапочки, костюмы (я уже не говорю про капельницы, трубочки
интубационные, катетеры для удаления слизи) – это все требует не только
покупки, но еще и утилизации. Если человек лежит заинтубированный и дышит на
вентиляцию в течение двух недель, то кто-то должен его обслуживать. Кто-то
должен выносить это судно, кто-то должен сливать мочу из катетера, кто-то
должен его кормить через зонд. Это зондовое питание. Это все не просто. Должно
быть специальное питание: внутривенное или зондовое, но кто-то это должен
делать.

Вот эта вся инфраструктура, заточенная на ковид, работу в инфекционных
условиях, инфекционные больницы, боксовые палаты, подготовленный персонал,
который не будет заражаться. 

Всего этого не было сделано. Нам объявили: у нас есть достаточное количество
коек. Чем обслуживать эти койки? Какими людьми? Какими сестрами, какими
санитарками? Как этих людей обслуживать на больничных койках, ведь они
инфекционные больные. Они требуют совсем другого подхода. И отнеслись к этому
очень легкомысленно. И не нашлось, к сожалению, того человека, который бы на
тот момент сказал: "Ребят, здесь надо немного посерьезнее. И давайте какой-то
резервный фонд. Мы закупимся, потому что будет вторая волна". Не нашлось
профессионалов в том руководстве, которое принимает решения о финансировании,
об изменении структуры и т. д.

С 1 апреля Верховная Рада приняла решение о введении второго этапа реформ. Это
на фоне эпидемии ковида. Разве можно было это делать? Они полностью перевели
все клиники в казенные предприятия. То есть сделали предприятия хозрасчетными.
Служба НСЗУ непонятно вообще, на каком основании создана, для чего... Сегодня
она занимается финансированием медицинских услуг по всем регионам. Она
заключала контракты на определенные пакеты медицинских услуг с этими всеми
больницами.

Так вот у нас во время эпидемии количество обычных пациентов, которые приходят
по этим пакетам, которые приходят в терапевтическое или хирургические
отделения, уменьшилось в несколько раз. Некоторые клиники полностью
перепрофилировали в ковидные. И, соответственно, исчез тот поток денег, который
должен был идти за этими пациентами. И многие клиники оказались на грани
банкротства. 

И те деньги, которые дают на ковид, абсолютно их не хватает для того, чтобы
обслуживать этих больных, потому что максимум у них хватает только на какие-то
примитивные антибиотики, купить кислород и так далее.

За восемь месяцев во многих областях до сих пор возят кислород баллонами, что
уже настолько неэффективно, потому что уже давно на Западе все клиники оснащены
либо собственными генераторами кислорода (делают его из воздуха), либо
загружают в жидком виде, и это концентрация в десятки раз больше, поэтому
гораздо выгоднее привезти бочку в 10 тонн жидкого кислорода, которого хватает
на две недели, чем каждый день привозить баллоны, перекручивать эти трубочки,
что взрывоопасно...

– Почему у нас не так? Нет денег или нет людей, в этом заинтересованных?

– Нет структуры, которая бы этим занималась. 

– Теоретически есть Министерство здравоохранения, которое может все закупить.

– Министерство здравоохранения не может. Сегодня забрали все закупки из
Министерства здравоохранения. Есть закупочное агентство, которое занимается
закупками. И, насколько мне известно, министр никак на это не влияет.

Второй волны можно было, честно говоря, избежать, если бы приняли настоящие
противоэпидемиологические мероприятия, как это сделали в Китае. 

– А вы считаете, что у нас было две волны?

– Да. Мы видим это по поступающим больным. Летом было немножко проще. Сейчас мы
заполнены. 3/4 наших пациентов – это ковидные больные. Даже у нас в институте
непрофильном. А есть целые клиники, которые полностью перепрофилировали в
ковидные. Они принимают сотни больных с ковидом сегодня. 

\subsubsection{\enquote{Часть легких просто становится рубцом}}

– Правильно ли я понимаю, что перепрофилировали отделения в Институте сердца, и
часть медперсонала занимается теперь и ковидными пациентами?

– Большая часть медперсонала нашего института занимается сегодня ковидными
больными. Где-то 2/3 клиники сегодня полностью отгорожены территориально. Даже
вентиляция разделена. И мы выделили эти койки, понимая, что есть категория
крайне тяжелых больных, с которыми другие клиники не справляются. А у нас
все-таки очень мощная реанимация с очень серьезными специалистами. Мы
обслуживаем пациентов с сердечной патологией, и у нас больше возможностей
спасать таких пациентов. 

Мы последние годы пережили несколько волн эпидемиологических гриппозных. Был
свиной грипп и так далее. И мы тогда тоже участвовали в спасении людей с
тяжелыми пневмониями. Могу сказать, что, по сравнению с тем гриппом, даже самым
тяжелым (который был в 2009 году. – Ред.), коронавирус гораздо более опасен. 

Коронавирус вызывает такие воспалительные реакции в организме и такой
аутоиммунный ответ на свое внедрение, что даже мы со всеми возможными
препаратами, гормонами и так далее иногда не справляемся с пневмониями, и люди
умирают от дыхательной недостаточности.

\ifcmt
pic https://strana.ua/img/forall/u/0/92/%D0%BE%D0%BB%D0%B5%D1%81%D1%8F(3).png
\fi

В том числе умирают и молодые, не только пожилые люди. То, что сопутствующие
заболевания являются очень серьезным фактором риска, это да. Но даже у нас уже
умерли несколько человек, которые не имели никаких сопутствующих заболеваний.
Это 30-летний молодой человек, занимающийся профессионально спортом.
Аутоиммунная реакция такая, что даже высокие дозы гормонов не подавляют ее.
Происходит фиброз легких.

У нас сейчас есть три человека, которые пережили коронавирус, но стали на
очередь на пересадку легких в результате. Потому что легочная структура очень
тонкая. Альвеолы – микроскопические пузырьки воздуха, окруженные клетками с
тонкой мембраной, через которую происходит газообмен. Поступает кислород –
отдается углекислый газ. Эта мембрана очень тонкая. Всего в несколько клеток:
капилляр, мембрана и воздух, которым мы дышим. И при таком воспалении, особенно
при аутоиммунном воспалении, вся эта воздушная и нежная структура превращается
по плотности в печень.

Нарушается обмен газом, и даже после выздоровления, когда вирус уходит, после
этого воспаления структура не восстанавливается. И люди остаются с тяжелой
дыхательной недостаточностью. Потому что часть легких просто становится рубцом.
Вот что страшно в этом вирусе. 

И я думаю, что еще одна волна, после того как мы уже, даст бог, при помощи
вакцин победим этот коронавирус, у нас появится на листе ожидания десятки и
сотни людей, которые будут нуждаться в пересадке легких. Мы сейчас к этому
готовимся.

Надо очень серьезно к этому отнестись. Мне кажется, уже практически все поняли,
что вот так легко к этому относиться не стоит. Воронки уже все ближе, снаряды
падают все ближе. Практически у каждого в семье кто-то тяжело переболел, а
сегодня уже переболело, наверное, больше 15\% населения. А многие уже и
потеряли близких людей, потому что количество умерших приближается к 20
тысячам.

– Несмотря на все это, 40\% населения категорически против любой вакцинации.  

– Все эпидемии, которые были в Советском Союзе и в Украине за эти годы, были
побеждены исключительно вакцинацией. Корь была побеждена вакцинацией, дифтерия
была побеждена вакцинацией. Оспа в свое время, полиомиелит, от которого люди
становились инвалидами и умирали, это все победили исключительно тотальной
вакцинацией. 

И если бы этой тотальной вакцинации не было, сегодня наша нация была бы гораздо
менее многочисленной и менее здоровой, поэтому, если брать статистику, из
тысячи вакцинированных может быть один человек будет иметь некие осложнения,
нелетальные. Просто температура или тяжело это переживет или еще что-то. Но
если класть на весы риски от вакцинации и риски не быть вакцинированным и
умереть в результате какой-то болезни и стать инвалидом в результате какой-то
болезни, то абсолютно несоизмеримые веса. В сотни, тысячи раз вакцинация
превосходит те риски, которые можно получить без нее. 

Так что я – за довольно жесткую вакцинацию. Есть люди, которых по медицинским
показаниям мы не вакцинируем. Например, родился ребенок с врожденным пороком, и
до того как мы его прооперируем и сделаем его здоровым, мы его не вакцинируем,
чтобы не вызвать бакэндокардит или какие-то осложнения со стороны сердца.

Да, есть противопоказания, но это 2\% населения имеют их, не более. Всех
остальных, если вы хотите ходить в школу, а ваш ребенок может стать источником
дифтерии, пожалуйста, наймите себе частного учителя, если вы такие умные, – и
занимайтесь дома, и не водите своего ребенка ни в какие кружки. И даже во двор
не выводите. 

Если мы живем в социуме и в густонаселенном городе, а не в селе, где сложно
потом локализовать подобные эпидемии, политика должна быть гораздо более
жесткой. 

– Возможно, это происходит так, потому что у людей просто нет доверия к власти?
Они боятся, что им завезут какую-то непроверенную вакцину, что на них будут
тестировать непроверенные лекарства. Не хотят быть подопытными в этом плане.

– Знаете, это же целая политика власти. Мне сегодня написали, когда я сделал
один пост с надписью "СССР", что "так вы из СССР". Я отвечаю, что большую часть
жизни, и могу сказать, что лучшую часть жизни, прожил в СССР. Около 30 лет. У
меня есть, с чем сравнивать. В те годы мнение врача было практически
непререкаемым. Мнение специалистов было непререкаемым. Если профессор что-то
говорил, все молчали, слушали и выполняли. Сегодня любой блогер вообще без
образования может назвать профессора дураком, сказать "да ты просто идиот",
"какие прививки вообще", "какой коронавирус", "что ты тут придумываешь". И
сегодня политика государственная не направлена на то, чтобы поддержать имидж
врача. И не направлена на то, чтобы защитить его, чтобы его мнение было
последней инстанцией, мнением специалиста.

Сегодня мнение десятка идиотов, извините, перевешивает мнение профессора. И кто
громче кричит в "Инстаграме" или "Фейсбуке", того и слушают.

Стало модным критиковать. Стало модным иметь свое мнение. Не важно,
профессиональное оно или нет. Оно просто должно отличаться от общего, и это
стало модным, да? Ты чем-то выделился, особенно если ты критикуешь авторитета,
значит ты что-то знаешь, понимаете? И вот люди сегодня... Какое-то отупление
происходит тотальное, как на мой взгляд. Люди, к сожалению, сегодня
прислушиваются чаще к тем, кто громче кричит, чем к тем, кто 30 лет работал в
этой проблеме и что-то в ней понимает.

Особенно если сегодня такая политика, когда можно приписать человеку "а, ну,
конечно, вы совковый, вы же еще с того времени, что вас слушать, о что вы нам
можете сказать". Люди, которые там не жили, люди, которые не знают, что это
такое, сегодня – самые большие критики того времени и той медицины и той
системы.

Посмотрите, кто сегодня руководит медицинскими какими-то большими программами. 

– Представители разных грантовых структур?

– Есть ли там специалисты? Вот, грантовых структур. И зарплаты у них в десятки
и сотни раз больше. Вот это поменялось очень сильно. И, к сожалению, это
разрушает и образование медицинское, и разрушает сегодня всю структуру
медицинскую, организационную, я имею в виду. И с этим ничего невозможно
сделать, потому что государство не настроено поддерживать специалистов. Не
настроено. Государство сегодня занимается саморазрушением. 

\subsubsection{\enquote{Российские вирусологи победили много эпидемий}}

– Как вы относитесь к российской вакцине?

– Я считаю, что нужно прекратить называть вакцины российскими или французскими,
или немецкими. Есть в медицине такое понятие, как слепое исследование, когда с
флакончиков какого-то препарата снимают этикетки производителя, присваивают
какой-то код и делают клинические испытания, при которых ни пациент, ни врач,
который назначает препарат, не знают происхождения этого препарата. Среди них
могут быть плацебо в том числе. То есть пустышки. И смотрят на результат.
Проводят исследование на сотнях добровольцев и получают статистически
достоверные результаты.

Как правило, это несколько сотен или даже несколько тысяч случаев. И после
этого говорят: эта вакцина хорошая, она эффективная, она дает стойкий иммунитет
на 6, 10 месяцев, на год, а эта вакцина – плохая.

Этого никто не сделал на сегодня. У нас вообще нет лабораторий, которые бы
определили качество этой вакцины. Их уничтожили за эти последние годы. Как для
какой-то африканской республики, мол, вы будете получать то, что мы вам скажем,
и колоть в свой организм то, что мы вам скажем. 

– Получается, что если нам какую-то вакцину поставят, у нас даже не будет
лабораторий и технической возможности, чтобы ее апробировать? 

– На сегодня – нет. Поэтому я бы не хотел сегодня хвалить или хаять московскую
вакцину, российскую вакцину. 

Российская школа вирусологов победила очень много эпидемий. Не только в своей
России родной, но и в Африке. И в других странах. Вирус Эбола победили именно
российские эпидемиологи. И эта школа развивалась много десятилетий. И мы еще по
Союзу знаем, что эта школа обеспечивала нам абсолютно все вакцины, которые мы
тогда не покупали за рубежом. И эти вакцины были очень эффективными.

Эта школа сохранилась. У них сохранилось биологическое оружие. Россия – одна из
немногих, кто работает с вирусами оспы до сих пор и с особо опасными
инфекциями. И у них эти лаборатории очень сильны. У них есть военные
медики-вирусологи, которые этим занимаются. У нас разрушено полностью все. У
нас был институт генетики, который сегодня не работает так, как он работал в те
годы, к сожалению. 

\ifcmt
  pic https://strana.ua/img/forall/u/0/92/%D0%B2_%D0%BF%D0%B0%D0%BB%D0%B0%D1%82%D0%B5_%D1%81_%D1%80%D0%B5%D0%B1%D0%B5%D0%BD%D0%BA%D0%BE%D0%BC.png
  caption Отделение детской реанимации в Институте сердца 
\fi

У нас есть Институт вирусологии во Львове, у нас были
санитарно-эпидемиологические лаборатории в каждой области. Этого всего нет. Это
все разбазарено, разрушено, ликвидировано. Поэтому говорить о том, что
эффективная вакцина российская или нет, можно тогда, когда мы ее исследовали и
сказали: да, есть результат. 

Но сегодня этого никто не делает. Тема заполитизирована. Объявляют нам высокие
чиновники, что мы принципиально не будем покупать российскую вакцину. А я читаю
сегодня в интернете: Трамп хочет принять закон, что, пока не будут
вакцинированы все американцы, американские вакцины не выедут за границу. А это
несколько месяцев. 

– Ряд народных депутатов Украины подавали даже заявления в международные
структуры о том, что на территории нашей страны расположены иностранные базы по
изучению вирусов. Публиковали даже карту размещения этих лабораторий. Как вы к
этому относитесь? 

– Юридический нонсенс. Такого не должно быть в независимой стране. Лаборатории,
которые никому не подчиняются и никем не контролируются, – это бомба
замедленного действия. В любой момент эта лаборатория может стать утечкой
какой-то особо опасной инфекции, которая убьет всю страну. И будет ли это
сделано преднамеренно или случайно, мы не знаем. Никто же не контролирует. Как
это возможно? Как это допустимо? Давайте еще какие-то ядерные лаборатории
разместим у себя. Пусть там экспериментируют. Не жалко же нас, украинцев.
Поэтому нонсенс. Такого не может быть в цивилизованной стране, самостоятельной. 

– Небольшой блиц от подписчиков. Маски на самом деле помогают не заразиться?

– Да. Это единственное, что сегодня должно неукоснительно соблюдаться. Масочный
режим. И дистанция между людьми, которые разговаривают между собой. 

– Делать КТ – да или нет и насколько адекватно она показывает картину. И
насколько это опасно?

– Здесь не получится ответить коротко, потому что в тех странах, где развита
система УЗИ-диагностики легких (например, в Германии), КТ почти не делают. Там
делают УЗИ-диагностику. К сожалению, в Украине очень мало специалистов, которые
делают УЗИ легких. И знают эту картину, и могут это адекватно оценить и
сказать, что здесь пневмония, здесь – не пневмония. И в какой степени
пневмония. Мы только начали эту программу у себя – подготовки узистов для
исследования легких. Поэтому у нас делают КТ. И никуда от этого не денешься.
Это, конечно, вредная штука. В десятки раз больше получаем ретген-излучение,
чем обычная флюорография. Но картинка, конечно, лучше.

– Коронавирус вызывает очень много стрессовых ситуаций. В том числе у пожилых
людей. Можете дать совет, как не нервничать и предостеречь себя?

– На самом деле очень сложно, долго пребывая в тревожном состоянии, сохранять
душевный покой. Можно неделю, две, три пережить какие-то стрессы, а когда это
длится восемь месяцев и накладывается экономическая составляющая (люди работу
теряют), болезни родственников, друзей, потеря близких (а это сейчас сплошь и
рядом).

Конечно, стресс усугубляет любые сердечные заболевания, в том числе ишемическую
болезнь, потому что хронический стресс, спазм сосудов, повышенное давление,
соответственно, и инсультов больше и инфарктов больше. Плюс сам по себе
коронавирус вызывает тромбоз мелких сосудов и риск тромбоэмболии легочной
артерии, риск инфаркта миокарда увеличивается в несколько раз у больного.
Потому мы всем пациентам, как только мы определили у них коронавирус, даем
антикоагулянт (лекарство против тромбов. – Ред.).

– Как вы считаете, когда закончится эпидемия в нашей стране? 

– Будет во многом зависеть от того, как быстро купят вакцину. Потому что за
восемь месяцев у нас переболело 15–20\% от силы нашей популяции украинской, а
этого недостаточно для того, чтобы пандемия остановилась сама по себе. К тому
же мы видим, что через шесть месяцев иммунитет у многих теряется настолько, что
они заражаются повторно. И второй раз болеют ничуть не меньше.

– То есть антитела не спасают на 100\%?

– Не спасают, поэтому просто переболеть не получится. Потому любая эффективная
вакцина, которая появится где-либо в мире, и мы будем готовы ее купить, – это
будет единственное спасение от этой пандемии. Иначе у нас просто будет
экономическая разруха.  

– Постоянно будут новые какие-то волны нас преследовать?

– Да.

– Сколько это может быть – год, два, сколько угодно?

– Говорят, что Англия уже начала вакцинировать своих врачей, полицейских и т.
д. В Америке вроде уже начали вакцинировать. В Китае и т. д. Поэтому сейчас
зависит только от чиновников, насколько быстро они смогут организовать покупку
вакцины, ее доставку и массовую вакцинацию групп риска.

– Как пандемия изменила украинскую медицину? В лучшую сторону или показала
только проблемы?

– Высветила проблемы. И мы очень надеялись, что после этого проявления проблем
к нам начнут по-другому относиться.

– Не заметили этого?

– Нет.

– А мир изменила пандемия, как вы считаете?

– Очень сильно изменила. Мы поняли, насколько хрупкий весь этот мир, насколько
хрупкая наша жизнь. То есть мы, врачи, и так это знали, но вот так для обычных
людей, которые каждый день не сталкиваются с этим, это было показателем. И я
думаю, что многие люди пересмотрели вообще жизненные ценности, связанные с
существованием.

– Однозначно. Прогулка где-то – это уже целое мероприятие.

– И то, что нет возможности уехать куда-то, что нет возможности поехать сейчас
в Германию или в Америку, где-то оперироваться, и то, что рядом близкие тяжело
болеют и ты бессилен что-либо сделать. Понимаете, есть целый ряд людей, которые
привыкли быть хозяевами жизни, они руководят большим бизнесом, они руководят
страной, целыми направлениями, какими-то, отраслями, и вдруг какой-то вирус
радикально меняет их жизнь.

\subsubsection{\enquote{Супрун – исчадие ада}}

– Мы с вами в операционной находимся. Я знаю, что Ульяна Супрун вас настолько
не любила, что в фильме, который она продюсировала, даже был антигерой с вашей
фамилией. А потом так случилось, что этот актер оказался у вас на операционном
столе. Какие ощущения были?

– Он пришел ко мне в кабинет прямо в день премьеры этого фильма. У меня был
шок. Я его узнал, а он меня нет. Потому что с момента съемок прошло полгода, и
он вообще не знал о том, что существует такой Тодуров. Он жил себе в Запорожье,
нормальный человек, актер своего театра. Он о Тодурове даже не слышал. Я ему
ничего не сказал. Я боялся, что это какая-то провокация, честно говоря. Думал,
что он член команды.

– У нас всякое можно ждать.

– Я очень деликатно с ним поговорил, сказал, что вам нужна операция.
Пожалуйста, если вы хотите, приходите, я вас прооперирую. Он сказал, что пришел
лично ко мне, чтобы я его оперировал. Через неделю его прооперировали, все
прошло гладко, ему ничего не говорили. Потом лечащий врач ему рассказал. Он был
в шоке.

Он сейчас приезжает, приходит ко мне в кабинет, мы общаемся. Мы очень хорошо
сейчас ладим. Я думаю, что мы сделали все правильно. Я мог отказаться от
операции, на самом деле. Но это было бы не по-врачебному, не по-человечески. Я
свою работу сделал, как положено, невзирая на какие-то эмоции. Он, в свою
очередь, сумел оценить, что мы остались нормальными людьми.

– Со статистикой заболеваемости коронавирусом происходят странности. То она
рекордно высокая – выше 16 тысяч в сутки, то рекордно низкая – менее 9 тысяч.
Насколько я понимаю, это связано с количеством тестирований. Статистикой кто-то
манипулирует? 

– Я не думаю. То, что эта статистика не соответствует истинному положению дел,
это правда. Потому что очень многие люди не тестируются из материальных
соображений. Например, кто-то в селе переболел тихонечко и даже не едет сдавать
анализы на ковид. Для кого-то это дорого. Кто-то понимает, что если результат
будет через пять дней, то он уже не будет иметь смысла. А считают только
тестированных методом ПЦР и на антитела. Поэтому в среднем эта цифра держится –
15-13 тысяч, она будет колебаться. Но я не думаю, что кто-то этим манипулирует
– просто мы не учитываем всех заболевших. 

– Министр здравоохранения или главный санитарный врач постоянно опираются на
эти данные, политики обращаются к ним, когда хотят ввести карантинные меры.
Поэтому людей волнует вопрос: почему сегодня статистика высокая, а завтра –
низкая. Она меняется буквально за один день, поэтому люди ищут подвох. 

– Более объективными будут показатели разницы между заболевшими и
выздоровевшими. Это определяет нагрузку на койки. Если заболевших намного
больше и это резкий скачок, то наши медицинские учреждения не справляются. Они
становятся переполненными, не хватает медикаментов, кислорода, персонала.

\ifcmt
pic https://strana.ua/img/forall/u/0/92/%D0%BE%D0%BF%D0%B5%D1%80%D0%B0%D1%86%D0%B8%D1%8F.png
caption Борис Тодуров на операции. Фото "Страны" 
\fi

Абсолютные числа заболевших показательны, мы можем видеть тенденцию. Но на
самом деле введение карантина должно определяться другой цифрой, а именно –
заполненность ковидных коек. Если цифра приближается к 80-90 процентам, это уже
критично, тогда нужно предпринимать серьезные меры. 

– Касательно введения тотального локдауна называли разные даты – 7 декабря, 25
декабря, теперь – после Нового года. Если ситуация патовая, почему меры не
вводить сейчас? 

– Принятие решения о локдауне в нашей стране диктуется не только медицинскими
показателями, но и политической и экономической целесообразностью. Есть три
фактора. Основное, конечно, – заполненность коек и приближение к грани
возможности медицинского обеспечения пациента.

Второй фактор, не менее важный, – все понимают, что полный карантин вызовет
экономическое падение: закроется весь мелкий бизнес, а тогда налоги валятся
вниз, бюджет никакой, люди теряют работу, доходы. Это вызовет рост
недовольства, преступности, депрессивных состояний, самоубийств, криминала и
всего, что за этим следует, – голодного бунта. 

Ну и чисто политическая целесообразность – все время они зондируют, как люди
принимают то или иное решение эмоционально, что идет в позитив, в позитив к
избирательной кампании, а что идет в негатив и снижает рейтинг власти. 

На сегодня экономические и политические факторы перевесили медицинские
показатели. Поэтому, думаю, и не введен полный локдаун.

– Вы работали при разных министрах здравоохранения.

– При всех.

– Как вы считаете, кто был самым эффективным?

– Могу сказать, что последние семь лет мы видим снижение уровня руководящего
состава – и не только министерского, а всего, что ниже. Вот Степанов после
Богатыревой (Раиса Богатырева возглавляла МОЗ в правительстве Николая Азарова в
2012-2014 годах. – Ред.) для нас, для медиков, наиболее приемлемый министр. Он
начал нас слышать, слушать. 

– То есть Степанов лучше Супрун? 

– Ну, то вообще исчадие ада. Даже не знаю, с чем сравнить. Демонически
патологическая женщина. Когда я смотрю на нее... Я объективно пытаюсь оправдать
человека, поставить себя на его место, понять, чем продиктовано поведение. Но в
случае с Супрун это сложно. 

– Вы общались с ней лично?

– Конечно. Впечатления крайне негативные, крайне неприятная в общении женщина.
Через неделю после ее прихода на пост министра я пригласил ее сюда, мы
пригласили пациентов после пересадки сердца. Были три пациента с механическими
сердцами. Показали, что мы единственная клиника, которая владеет такими
технологиями. Хотел ей показать, что в Украине такое существует. Она же
приехала из США и вообще ничего здесь не знала, документы не могла прочитать
медицинские.

Я хотел рассказать целую стратегию развития, сделал большую презентацию на 15
минут. Она еле досидела, была крайне недовольна нашими успехами. Сделала губами
куриную жопу (Тодуров показывает, как Супрун скривила губы. – Ред.), вышла,
поймала такси. Я бежал за ней. Говорю: "Что-то не понравилось, почему вы ушли с
презентации?" Она: "Мені потрібно їхати". Прямо из актового зала вызвала такси,
и они уехали с подружкой. 

– После этого вы общались? 

– После этого у нас не получился контакт уже никакой. Она была так раздражена
увиденным здесь. И когда к ней пришли народные депутаты с программой по
трансплантации и механическим сердцами, дали ей депутатский запрос – мы просили
всего лишь 15 миллионов, не так уж много для бюджета МОЗ, – она швырнула эти
бумаги и сказала: "У нього і так всього достатньо". Видимо, она не для того
сюда была прислана, чтобы что-то развивать или радоваться нашим успехам.

– А для чего, как вы думаете?

– Как представитель международных корпораций, которые поняли, что в
40-миллионной стране можно зарабатывать. Там, где можно влиять на принятие
решений, законов, а наши чиновники и депутаты, к сожалению, все это принимают
без оглядки по приказу сверху. Это делается в ущерб нашей медицине, нашим
пациентам, нашим врачам. Это все очень деструктивно. А рынок же очень большой:
препараты, медицинские услуги, трансплантология – это большие деньги. 
