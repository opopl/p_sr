% vim: keymap=russian-jcukenwin
%%beginhead 
 
%%file slova.sad
%%parent slova
 
%%url 
 
%%author_id 
%%date 
 
%%tags 
%%title 
 
%%endhead 
\chapter{Сад}

%%%cit
%%%cit_head
%%%cit_pic
\ifcmt
  tab_begin cols=3
     pic https://regnum.ru/uploads/pictures/news/2021/10/27/regnum_picture_16353551825022945_normal.jpg
     pic https://regnum.ru/uploads/pictures/news/2021/10/27/regnum_picture_16353552514365664_normal.jpg
		 pic https://regnum.ru/uploads/pictures/news/2021/10/27/regnum_picture_16353552035110708_normal.jpg
  tab_end
\fi
%%%cit_text
Бал хризантем «Осенняя рапсодия» — такое название получил бал в том году. На
выставке представлено около сорока пяти тысяч хризантем из 500 сортов и видов.
Впервые выставка хризантем прошла в \emph{Никитском ботаническом саду} в 1953 году, в
этом году выставка проходит в 68-й раз
%%%cit_comment
%%%cit_title
\citTitle{«Осенняя рапсодия»: в Крыму открылся бал хризантем — фоторепортаж}, 
Александр Полегенько, regnum.ru, 27.10.2021
%%%endcit


%%%cit
%%%cit_head
%%%cit_pic
%%%cit_text
За два тижні я самостійно ходив попід стінами, тренуючи м’язи, біцепси. Пив
лише воду і чай, заварений лікувальними травами. Щоденно очищався і приймав
гарячі ванни з настоєм полину, евкаліпту, звіробою. Гімнастика й обтирання
водою теж входили в комплекс лікування.  На четвертому тижні я вже гуляв у
\emph{Ботанічному саду}, обнімав стовбури дерев, плакав від щастя, що можу
знову дивитися у ясне видноколо, дихати запахом квітів, гладити шорстку кору
дуба, милуватися неповторними переливами барв живого світу.  Минуло вісім
тижнів. Я повністю одужав. Розумієте — повністю! Навіть з’явилося хвилююче
відчуття радості буття, чого не було ніколи раніше. Почав входити у нормальний
режим харчування, знову поправився, відновив вагу. Потім пішов до лікарів. Вони
були вражені, здивовані. Але, ясна річ, все збагнули, коли я їм розповів про
голодотерапію. Цей метод тепер широко практикується, ви, безумовно, повинні
знати...
%%%cit_comment
%%%cit_title
\citTitle{Вогнесміх}, Олесь Бердник
%%%endcit
