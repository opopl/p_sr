% vim: keymap=russian-jcukenwin
%%beginhead 
 
%%file 15_01_2022.fb.ovcharenko_vitalij.1.japonia_interview_ukr_rossia.cmt
%%parent 15_01_2022.fb.ovcharenko_vitalij.1.japonia_interview_ukr_rossia
 
%%url 
 
%%author_id 
%%date 
 
%%tags 
%%title 
 
%%endhead 
\zzSecCmt

\begin{itemize} % {
\iusr{Іван Синєпалов}
Ну ти ж сказав, що ми готові за свою честь, не вмирати, а вбивати?)

\iusr{Vitalii Ovcharenko}
\textbf{Іван Синєпалов} я сказав що Росія і головне самі росіяни дорого заплатять за свою чергову атаку.

\iusr{Roman Eridan}
То ти вже медійна особа, слухай )

\iusr{Vitalii Ovcharenko}
\textbf{Roman Eridan} та куди там.

\iusr{Mykola Mykolaiovych Dolynskyi}
We are building the perfect weapon to deliver Emperor's will

\ifcmt
  ig https://i2.paste.pics/6dc7a26764a59be87145bbdd20c9c4b9.png
  @width 0.2
\fi

\begin{itemize} % {
\iusr{Vitalii Ovcharenko}
\textbf{Mykola Mykolaiovych Dolynskyi} найкраща зброя це самурайський дух!

\iusr{Mykola Mykolaiovych Dolynskyi}
\textbf{Vitalii Ovcharenko} це мав бути мій другий коментар, але підступний Овчаренко-сан зіграв на випередження

\iusr{Vitalii Ovcharenko}
\textbf{Mykola Mykolaiovych Dolynskyi} це мудрість столітньої сакури...

\iusr{Mykola Mykolaiovych Dolynskyi}
\textbf{Vitalii Ovcharenko} може ти вже бачив, але принаймні лайка твого там немає
\url{https://www.facebook.com/2170688466513405/posts/3044942879087955/}
\end{itemize} % }

\iusr{Євгенія Чуприна}
Ми воюємо саме за те, щоби нас не вбивали

\iusr{Yasa Sirko}

Норм світосприйняття. В мене таке саме. Усе життя його реалізовую ))) Чесно
мені погуй на євротолерантні цінності з їхньою моделлю демократії, гендером -
фемінізмом та лгбт )))) Я за Українську Україну та наші традиції та цінності.

\begin{itemize} % {
\iusr{Дмитрий Карней}
\textbf{Yasa Sirko} зазвичай толерантність европейців найбільш хвилює всяких латентних представників меньшин. Може пан щось в собі приховує?  @igg{fbicon.wink}  Може до псіхоаналітика?

\iusr{Vitalii Ovcharenko}
\textbf{Yasa Sirko} дивно всі ж інші за гендер воюють!

\iusr{Yasa Sirko}
\textbf{Дмитрий Карней} написано що хвилює ? Написано - погуй . Зовсім протилежне
чи не так ))) Так що не збуджуйся а краще вивчай методичку )))

\iusr{Дмитрий Карней}
\textbf{Yasa Sirko} якщо людині щось "погуй" то вона про те не згадує і не пише. Але вас вочевидь то бентежить і у вас то болить. Тому не ігноруйте мою пораду.

\iusr{Yasa Sirko}
\textbf{Дмитрий Карней} 

якщо мені \enquote{погуй} то я й пишу \enquote{погуй} якщо мене щось \enquote{бентежить} чи
\enquote{болить} я так й пишу \enquote{бентежить} чи \enquote{болить} ))) Якщо у пана його \enquote{Гугл}
підсвідомості до вирваних з контексту слів видає таку \enquote{гуйню} то пану дійсно
потрібно до спеціаліста. Хоча можливо для пана це є норма )))

\iusr{Дмитрий Карней}
\textbf{Yasa Sirko} ага. Хто всрався? Невістка... (с)
\end{itemize} % }

\iusr{Віктор Дихановський}

Японці не стали воювати за свою честь і гідність у 1945 році. І в 1993 році не
забрали свої північні території, коли в Московії був безлад.

\begin{itemize} % {
\iusr{Vitalii Ovcharenko}
\textbf{Віктор Дихановський} у 1993 році уе було б розв’язування війни. А у 45 ще одна ядерна бомба.

\iusr{Родомир Степной}
\textbf{Віктор Дихановський} Ага) у самогубці стояли черги добровольців, але японці не хотіли воювати)))
\end{itemize} % }

\iusr{Valentyn Krasnoperov}
Треба було гандама попросити аби воно парочку прислали нам на фронт

\begin{itemize} % {
\iusr{Vitalii Ovcharenko}
\textbf{Valentyn Krasnoperov} кого?)

\iusr{Valentyn Krasnoperov}
\textbf{Vitalii Ovcharenko} то японський робот в якого залізаєш і мочиш ворогів. З аніме

\iusr{Vitalii Ovcharenko}
\textbf{Valentyn Krasnoperov} ми його на російські позиції і камікадзе!
\end{itemize} % }

\iusr{Родомир Степной}

Коли вже були скинуті ядерні бомби та імператор записав звернення до народу,
знайшлися заколотники, які увірвалися до його палацу. Вони хотіли зірвати
заплановане звернення і продовжувати воювати до кінця!

\begin{itemize} % {
\iusr{Vitalii Ovcharenko}
\textbf{Родомир Степной} справжній японський дух!

\iusr{Родомир Степной}
\textbf{Vitalii Ovcharenko} небезпечні люди)
\end{itemize} % }

\iusr{Vitaliy Kolomiets}
Писав якось про честь дещо в юридичному аспекті. Може тобі пригодиться

\href{https://lb.ua/blog/vitalii_kolomiets/441901_pro_chest_storozhu_simvoliv.html}{%
Про честь та сторожу символів, Віталій Коломієць, lb.ua, 11.11.2019%
}

\raggedcolumns
\begin{multicols}{2} % {
\setlength{\parindent}{0pt}

\ifcmt
  ig https://i.lb.ua/114/45/5db6c8149c9f3.jpeg
\fi

Останні кілька місяців були повідомлення про зняття українських прапорів у
Станиці Луганській, інших містах та навіть з військової техніки.

Якби старомодно це не виглядало, але війна йде саме за символи. Символи - це
знаки певних цінностей. Цінності для багатьох людей складають основу їх
ідентичності. Власне, віра у різні цінності робить біологічно ідентичних гомо
сапієнсів - різними.

Один знак може нести величезний пласт інформації, коли ми бачимо знак Mercedes,
Toyota чи Tesla ми розуміємо про що це. Це \enquote{щось} є результатом життя і
праці тисяч людей.

Людина, яка приймає певні цінності, які визначатимуть її цілі і вчинки, також
приймає символи цих цінностей. Хрест для християн це ж символ, який концентрує
тисячі і тисячі інструкцій життя.

Людина, представляючи себе, не каже, що вона 80-ти кілограмова водно-білкова
емульсія з когнітивними здібностями. Взагалі, відповідь на питання \enquote{хто
ти} є відповіддю \enquote{про цінності}: мама, лікар, правоохоронець,
священник, інженер, військовий, журналіст, українець.

Носій певних цінностей може їх проявляти через інші символи, наприклад, одяг:
військовий, суддя, лікар, монах.

Здатність бути вірним своїм цінностям - є честю. Коли людина чесна в
сповідуванні своїх цінностей це проявляється у всіх її вчинках, тоді така
людина має честь. Честь офіцера, лікаря, священника.

Але коли декларуються одні цінності, а сповідуються інші - це відсутність честі
(сором, ганьба).

Той, хто має честь, з повагою носить символи тих цінностей, які сповідує.
Шеврон з українським прапором - це символ такої цінності як \enquote{свобода}. Власне,
наш прапор є символом двох важливих цінностей: \enquote{хліб} - результат власної праці
та \enquote{небо} - символ свободи. Їх розміщення - ще один символ того, що \enquote{Свобода
вище Хліба}. Це незрозумілі цінності для росіян. Їм чужа \enquote{воля} робити свою
справу в \enquote{щирій праці}, вони ніколи не \enquote{положать тіло за свободу}.

Тому ми чужі, тому й воюємо кожен з вірою в свої цінності.

Наше \enquote{слава} дуже близьке до поняття \enquote{честь}.

Ми дуже мало приділяємо уваги такій моральній категорії як \enquote{честь}, хоча
юридично вона не менш значуща як життя і здоров'я.

Багато, хто згадує про життя і здоров'я людини як найвищу соціальну цінність.
Хоча в ст. 3 Конституції на рівні з іншим згадана Честь: \enquote{людина, її життя і
здоров'я, честь і гідність, недоторканність і безпека визнаються в Україні
найвищою соціальною цінністю}.

Інша стаття, яка зобов'язує дотримуватися законів і забороняє посягати на права
людини також забороняє посягати на честь: \enquote{кожен зобов'язаний неухильно
додержуватися Конституції України та законів України, не посягати на права і
свободи, честь і гідність інших людей} (ст.68).

Нація - концентрує свої цінності в своїх символах. Нерозуміння свого прапору,
гімну, чи гербу є нерозумінням цінностей, нерозуміння цінностей - є
дезорієнтацію, а дезорієнтований - це 60-80-100 кілограмова водно-білкова
емульсія.

Шанування символів є обов'язком кожного громадянина України (ст. 65
Конституції). Шанувати символи - означає також сповідувати цінності, які
закладені в ці символи.

Посягання на символи - це посягання на честь тих, хто сповідує цінності,
зашифровані в ці символи.

Честь військового - це також бути добрим сторожем наших спільних символів. Не
давайте знімати наші символи з нашої землі.

\end{multicols} % }

\iusr{Светлана Пучкова}

Очень правильные вопросы для наций с наличием чувства собственного достоинства.
Я давно слежу за японцами и корейцами. Они очень круты. Они себя точно
сохранят.

\begin{itemize} % {
\iusr{Дмитрий Карней}
\textbf{Светлана Пучкова} 

при этом все их имперские потуги (в отличие от китайцев кстати), всегда
заначивались смачной пизд\&линой и позором. А корейцы так и вообще половину
страны просрали. И сейчас там полоумный жиробас вырастил новый вид человека
\enquote{овощ разумный}. Очень круты, спору нет...  @igg{fbicon.laugh.rolling.floor} 

\iusr{Іван Синєпалов}
Нації з відчуттям власної гідності не розмовляють мовою народа-окупанта.
\end{itemize} % }

\iusr{Pavlo Shatohin}

Я нещодавно відповідав одному \enquote{какая разніца}, що якщо він планує тікати, але
все ще не визначився, за яку країну він таки готовий битися, то тікати він буде
все своє життя і ніде його не зрозуміють, не приймуть і зневажатимуть, як би
він не виправдовувався.

\iusr{Ivan Lozowy}
Надіюсь, згадали про добробатів, кіборґів.

\begin{itemize} % {
\iusr{Vitalii Ovcharenko}
\textbf{Ivan Lozowy} а токар про 156й, 191й 4й та 100500 батальйон. Вони ж дуже розбираються у наших тонкощах.

\iusr{Ivan Lozowy}
\textbf{Vitalii Ovcharenko} Не знаю ці номера, але про рух добробатів знають багато хто, це ключова відповідь на їхні запитання: 8 років тому тисячі людей повстали на захист держави, аж до смерті, як добровольці.
\end{itemize} % }


\iusr{Олег Медведчук}
Японці - потужна нація, тому і вірні акценти вони ставлять в питаннях.

Є чому повчитись з погляду на позицію мешканців селища на Тернопільщині, де
проти розгортання підрозділу тероборони


\iusr{Medvedko Andrii Oleksandrovych}
Бач, демократія твоя нікому не всралась.

\begin{itemize} % {
\iusr{Vitalii Ovcharenko}
Ті хто воліють за тоталітаризм завжди думать що вони уж точно не будуть у розстрільних списках)))

\iusr{Родомир Степной}
\textbf{Vitalii Ovcharenko} кожен хоче Сталіна не для себе, а для сусіда (с)

\iusr{Medvedko Andrii Oleksandrovych}
\textbf{Vitalii Ovcharenko} тоталітаризм твій теж всім пох)))

\iusr{Vitalii Ovcharenko}
\textbf{Medvedko Andrii Oleksandrovych} тоді що не пох?))

\iusr{Medvedko Andrii Oleksandrovych}
\textbf{Vitalii Ovcharenko} Вільна Україна!

\iusr{Vitalii Ovcharenko}
\textbf{Medvedko Andrii Oleksandrovych} Занадто абстрактно))

\iusr{Medvedko Andrii Oleksandrovych}
\textbf{Vitalii Ovcharenko} за то демократія і тоталітаризм - дуже конкретно!))

\iusr{Vitalii Ovcharenko}
\textbf{Medvedko Andrii Oleksandrovych} Та там інтервʼю було для японців, я там намагався загальними фразами говорити.

\iusr{Medvedko Andrii Oleksandrovych}
\textbf{Vitalii Ovcharenko} ну за Імператора не пхнув?

\iusr{Vitalii Ovcharenko}
\textbf{Medvedko Andrii Oleksandrovych} Імператор то святе. Про нього або гарно або ніяк!

\iusr{Михайло Свистович}
\textbf{Medvedko Andrii Oleksandrovych} японцям всралась, в них демократія, і вона їм подобається
\end{itemize} % }

\iusr{Сергей Косяк}

Честь и достоинства изнасилованы в масштабе нации не Путиным, а местными
божками, которые поднялись на волне Майдана, поправ его принципы. Честь, права
и свободы заменили на «патриотические» лозунги о которых можно вечно говорить,
спорить, получая свои бонусы на волне популярности. О достоинствах можно
говорить когда в суды, будут честно судить, полиция - защищать, политики -
служить, и т.д., а так...

\begin{itemize} % {
\iusr{Павло Марчук}
\textbf{Sergey Kosyak} 

За \enquote{честь і гідність} вас хочеться забанити тут і зараз.

Гідні українці цитати із московської пропаганди не використовують.


\iusr{Сергей Косяк}
\textbf{Павло Марчук} 

да бань себе на здоровье. Можешь апельсин назвать яблоком и патриотично за это
умереть, от этого оно яблоком не станет. 17 человек с нашей команды и я в том
числе прошли через плен, 4 через расстрел, 2 убили. За то что бы на выхлопе
получить «віру» и прочую хрень? Или вы сами забыли, с чего начался Майдана?

\iusr{Oleksandr Savchenko}
\textbf{Сергей Косяк} 

Сергію, я тобі дивуюсь! Невже ти не бачиш зв‘язок між червоною окупацією
України і сьогоднішніми бідами- злиднями? Честь і достоїнство робляться не
державою, а нами.

\iusr{Сергей Косяк}
\textbf{Oleksandr Savchenko} 

Когда рейдеры в связки с судьей отбирают территорию церкви во Львове под
строительства отеля-борделя, вот никак не вижу связь с Путиным и оккупацией
Донбасса. Нам постоянно, что-то жить мешает - то Путин, то предощущая влада, то
настоящая, то ещё Америка, то Европа, а по сути, Януковичи, Кивы, україновомни
українски судді та інши корупціонери і злодіі выходцы из нас.

\iusr{Oleksandr Savchenko}
\textbf{Сергей Косяк} 

виходців з нас? Сергію, кого маєш на увазі? Це наслідки російсько- комуняцкої
диктатури, начальники і вертухаї у концтаборі теж виходці з поневолених?

\iusr{Сергей Косяк}
\textbf{Oleksandr Savchenko} 

имеется ввиду из обычных людей. Существующие коррумпированные системы формируют
личности которые в эти системы попадают. Я откровенно говоря не знаю, что
должно произойти, что бы Украина стала другая.

\iusr{Vitalii Ovcharenko}
\textbf{Sergey Kosyak} иногда я удивляюсь что вы не пошли в ополчение. Риторика очень схожа.

\iusr{Сергей Косяк}
\textbf{Vitalii Ovcharenko} когда вы поймёте библейское понимание единства, тогда поймёте и другое.

\iusr{Максим Потапчук}
\textbf{Sergey Kosyak} 

біблійне розуміння єдності це коли ви зараз усіх наших українців, які працюють
над розвитком України з часів Майдану, порівнюєте з бандюками? Тобто і я
бандюк?? Я був на Майдані і цим пишаюсь, а Ви намагаєтесь зараз нагидити на
голову моїх живих та загиблих побратимів. Можливо якось разом попрацюємо над
біблійним єдинством, бо я або не можу наздогнати Вас або мої знання про єдність
вже перевершили Ваші. Не скигліть, дорогий Друже та працюймо далі.
Тримайтеся@igg{fbicon.heart.red} @igg{fbicon.flag.ukraina}

\iusr{Vitalii Ovcharenko}
\textbf{Sergey Kosyak} Проще всего прикрывать преступления и сравнивать палача и жертву - библией.

\iusr{Oleksandr Savchenko}
\textbf{Сергей Косяк} 

Сергію, біблійни істини, які об‘єднують, вони ж і роз'єднують. Українці мають
християнські цінності навіть у своєї ментальності, тому ми збираємося на
Майданах, тому ми обираємо президентів які оголошують ті принципи, і тому ми
зкидаємо їх, коли вони зраджують тим принципам. А роз‘єднують ті біблійні
істини нас з тими, для кого вигода більше за ті істини. Розумієш? Нам по
цимбалах, на якій мові говорять і декларують зрадники і вовки в українських
сорочках, але в нас є свої національні, історичні істини, які формувалися на
тих біблійних цінностях і от за них, ми й готові на все.

\iusr{Сергей Косяк}
\textbf{Vitalii Ovcharenko} 

а где я что сравнивал? Или ты думаешь я оправдываю агрессию Путина, тем, что не
оправдываю то что, принципы Майдана попираются под лжепатриотические лозунги?
Что Майдана стоял за ценности прав, свобод и против коррупции, а потом
изначальный вектор заменили на віру, безвіз и прочі друге степені речі?

\iusr{Михайло Свистович}
\textbf{Sergey Kosyak} 

Майдан стояв і за віру й безвіз. Не треба всіх судити по собі й оточенню. Різні
люди за різне там стояли.

\iusr{Павло Марчук}
\textbf{Sergey Kosyak} ось із цим твердженням навіть я погоджуюсь

\iusr{Сергей Косяк}
\textbf{Михайло Свистович} 

за «віру» я точно не стоял. Максом бы и стоял, то была бы в списке она где то в
хвостике. Так как живем мы в межконфессиональной стране и никакая конфессия, не
должна быть инструментом политических манипуляций. Если о вере, говорят
политики, то это уже не религия, а политика.

\iusr{Михайло Свистович}
\textbf{Sergey Kosyak} 

Ви не стояли, хтось стояв, але Ви чомусь пишете у дусі \enquote{Майдан - це я}, хоча на
Майдані були дуже різні люди. Політики завжди говорили про віру, всю історію
людства. То що, ніякої релігії ніде не було після виникнення держав?

\iusr{Piter Haruna}
\textbf{Sergey Kosyak} Україна має еволюціонувати, а для цього українці повинні вчитися селекціонувати владу.

\end{itemize} % }

\iusr{Павло Марчук}

українською \enquote{достоинство} перекладається як \enquote{гідгінсть}



\end{itemize} % }
