% vim: keymap=russian-jcukenwin
%%beginhead 
 
%%file 16_01_2022.fb.fb_group.story_kiev_ua.1.ljubite_li_vy_teatr.cmt
%%parent 16_01_2022.fb.fb_group.story_kiev_ua.1.ljubite_li_vy_teatr
 
%%url 
 
%%author_id 
%%date 
 
%%tags 
%%title 
 
%%endhead 
\zzSecCmt

\begin{itemize} % {
\iusr{Елена Соколова}
Осень люблю театр, это не обьясняется. Просто люблю...

\begin{itemize} % {
\iusr{Ирина Петрова}
\textbf{Елена Соколова} 

Театр можно обсуждать, осуждать, любить, какой-то спектакль может вызвать
чувство катарсиса, какой-то - омерзения. Уж сколько великих говаривало, что
театр - наша жизнь. С ними не поспоришь)

\iusr{Елена Соколова}
\textbf{Ирина Петрова} полностью согласна с вами
\end{itemize} % }

\iusr{Александр Асатуров}

не пора ли нам замахнуться на Вильяма нашего, понимаете, Шекспира? Ведь
насколько Ермолова могла бы играть лучше вечером, если бы днем она, понимаете
ли, работала у шлифовального станка!

\begin{itemize} % {
\iusr{Светлана Манилова}
\textbf{Александр}, \enquote{Берегись автомобиля}. Как хорошо ты вспомнил. @igg{fbicon.smile} 

\iusr{Александр Асатуров}
\textbf{Светлана Манилова} Евстигнеев - бог!

\iusr{Ирина Петрова}
\textbf{Александр Асатуров} и не поспоришь ведь @igg{fbicon.wink} 
\end{itemize} % }

\iusr{Ирина Иванченко}

Спасибо, отличные тетральные истории, приятно было прочесть про дорогих вашему
сердцу людей, про народный театр, про Пистоньку, и фотографии прикольные.

\iusr{Ирина Петрова}
\textbf{Ирина Иванченко} 

спасибо, Ирина! Такие теплые слова очень нужны автору @igg{fbicon.wink}  театральные байки -
необъятный мир, и, даже на уровне небольшого самодеятельного театрика их можно
рассказывать бесконечно.

\iusr{Виктор Задворнов}

Прекрасный рассказ. Думаю, если бы существовал до сих журнал \enquote{Самодеятельный
театр}, Ваши очерки украсили бы не один номер. Огромное человеческое спасибо!
Роли Предаевич я обожал смотреть в Русской драме. Оказывается, она еще и
режиссировала. Представляю, какой насыщенной была Ваша творческая жизнь. Желаю
всех благ!

\begin{itemize} % {
\iusr{Ирина Петрова}
\textbf{Виктор Задворнов} 

прежде всего - огромная благодарность и поклон за добрую память о нашей любимой
Верочке Леонидовне! Это был такой подарок судьбы - знакомство с Верой
Леонидовной, Александром Герасимовичем ( помню, как он приходил на репетиции, с
тростью, всегда с сигаретой, замечания говорил негромко Вере Леонидовне,
актеров всегда хвалил), с Сашей Онуровым мы начинали репетиции Ионеско \enquote{Король
умирает}, помню свою первую выходную реплику \enquote{здесь мусор и пыль...}

Не сложилось. Саши нет уже шесть лет ...

Спасибо за отклик, это греет.

Вера Леонидовна в конце репетиций всегда говорила нам: \enquote{До свидания!
Кланяйтесь непременно маме ( или супругу, или супруге)}. Это навсегда вошло и в
мой обычай, теперь, правда, вызывающий недоумение...

Светлая им память, Актёрам!
\end{itemize} % }

\iusr{Alik Perlov}

С утра прочёл)))) Спасибо тебе за рассказ @igg{fbicon.bouquet}{repeat=3}
@igg{fbicon.hands.applause.yellow}{repeat=10} Браво Бис.  Народная вы Наша
@igg{fbicon.face.blowing.kiss}{repeat=3} 


\iusr{Ирина Петрова}
\textbf{Alik Perlov} 

Алик, поклон с авансцены! Мерси, мон анж! @igg{fbicon.heart.eyes}
@igg{fbicon.face.blowing.kiss} @igg{fbicon.heart.red} ️@igg{fbicon.face.smiling.eyes.smiling}
@igg{fbicon.heart.with.ribbon} 

\iusr{Тамара Градобоева}

\ifcmt
  ig https://scontent-frt3-1.xx.fbcdn.net/v/t39.1997-6/s168x128/16781161_1341101952618574_7704631035023065088_n.png?_nc_cat=1&ccb=1-5&_nc_sid=ac3552&_nc_ohc=hM1uiDFLgRoAX_BmOxP&_nc_ht=scontent-frt3-1.xx&oh=00_AT82pUAwD2i2YMFZih_O8Tj_FhSyFI6vMKWz5V8oPXHLKg&oe=620524AB
  @width 0.1
\fi

\iusr{Елена Сидоренко}

Браво, Ирочка! Огромное спасибо за рассказ. @igg{fbicon.heart.sparkling} 

\ifcmt
  ig https://scontent-frt3-1.xx.fbcdn.net/v/t39.1997-6/s168x128/17639224_1652591041433963_4031073656246370304_n.png?_nc_cat=1&ccb=1-5&_nc_sid=ac3552&_nc_ohc=GKJX96ftDGkAX-8nBAY&_nc_ht=scontent-frt3-1.xx&oh=00_AT_4CzpxvEY92DYVcpshdr3kcH2lvHYRqs59bCrsAbDiQw&oe=62057ADF
  @width 0.1
\fi

\iusr{Ирина Петрова}
\textbf{Елена Сидоренко} спасибо за теплый отклик!

\iusr{Tetiana Petrovska}
Спасибо. Замечательние воспоминания. Люблю театральное закулисье  @igg{fbicon.flame} 

\iusr{Ирина Петрова}
\textbf{Tetiana Petrovska} 

Спасибо за отклик! И Вы совершенно правы - мир, который зритель видит из зала,
и мир, который есть щакулисами - это, как говорят в Одессе - две большие, я бы
сказала, две огромнейшие разницы. Волею судеб, мне сейчас повезло участвовать
в подготовке и проведении театрального фестиваля. Проходит фестиваль в стенах
любимого театра. Запахи, звуки, встречи - это просто невероятные ощущения.

\iusr{Елена Муравьева}

самодеятельность, как мафия, неискоренима. я \enquote{нечаянно} стала режиссером
самодеятельного театра и вот уже восемь лет мы ставим спектакли. возраст
артисток (в труппе только дамы) за шестьдесят

\iusr{Елена Муравьева}

самодеятельность, как мафия, неискоренима. я \enquote{нечаянно} стала режиссером
самодеятельного театра и вот уже восемь лет мы ставим спектакли. возраст
артисток (в труппе только дамы) за шестьдесят

\begin{itemize} % {
\iusr{Ирина Петрова}
\textbf{Елена Муравьева} 

а вот теперь поподробнее!))) Где, когда, почему нет рекламы? Любопытно ж!!! Где
выступаете? Сейчас есть спектакли? Очень хотелось бы посмотреть!

Ой, увидела на Вашей страничке! Почти понятно) а будут ещё спектакли?

\iusr{Елена Муравьева}
\textbf{Ирина Петрова} 

каждые полгода делаем новую постановку. выступаем перед новым годом и в мае.
один-два спектакля и все. к сожалению, удержать репертуар не получается. пока
мои артистки учат новые роли старые забываются. но процесс творчества увлекает
больше выступлений

\iusr{Ирина Петрова}
\textbf{Елена Муравьева} 

все равно, это очень здорово! Вы - крутые молодцы!!! Если что - я в первых
зрителях, ок? @igg{fbicon.wink}  @igg{fbicon.heart.eyes}
@igg{fbicon.hands.applause.yellow} 

\iusr{Елена Муравьева}
\textbf{Ирина Петрова} конечно
\end{itemize} % }

\iusr{Татьяна Петрюк}

Восторг ! Вы не только талантливая актриса, но и блистательная рассказчица !!!
Браво !

\begin{itemize} % {
\iusr{Ирина Петрова}
\textbf{Татьяна Петрюк} 

Танюша, спасибо, дорогая! Насчёт талантливой актрисы - это преувеличение, а
рассказать могу, что да, то да @igg{fbicon.wink}  Спасибо
@igg{fbicon.heart.eyes} @igg{fbicon.heart.red}

\begin{itemize} % {
\iusr{Татьяна Петрюк}
\textbf{Ирина Петрова}, 

чтобы рассудить нас: нужно попасть на спектакль с Вашим участием, и,
конечно, новых увлекательных историй ! Ваши Рассказы реально вносят в нашу
жизнь лучи добра тепла и света !!!

\iusr{Ирина Петрова}
\textbf{Татьяна Петрюк} 

сонечко, наверное, сцена уже только в воспоминаниях. Я ужасно рада, что именно
в группе нашла девочку из нашего театра. Это уже не первый раз, когда группа
дарит мне людей, с которыми жизнь развела. Честно - иногда во сне выхожу из
кулис, и...не помню слова, мало того - не могу понять, что играем! От всхлипа и
ужаса просыпаюсь. Почти 30 лет уже прошло, а есть ещё в уголочке мозга вот это
- выйти из кулис. Спасибо за теплые слова, они поддерживают в наше непростое
время!

\iusr{Татьяна Петрюк}
\textbf{Ирина Петрова} 

не устаю повторять - группа позитива. Уверенна, что после 30-ти лет творческого
актёрского «простоя» Вам удастся подарить зрителям приятные минуты настоящего
творчества. Сравнивать нельзя, но я когда-то занималась фигурным катанием,
через лет эдак почти 50 повезла внучку кататься на роликах ... чтобы не терять
время зря в ожидании - взяла на прокат ролики и поехала  @igg{fbicon.face.tongue} ... я испытала
невероятный душевный подъем и гордость, словами сложно передать ...

Поэтому уверенна - собирайте своих девочек и на репетиции. Зрители уже готовы и
ждут @igg{fbicon.heart.red}.
\end{itemize} % }

\end{itemize} % }

\iusr{Дмитрий Бартюк}

Есть мнение, что народные театры вскоре вытеснят, наконец, театры
профессиональные! И это правильно! Актер, не получающий зарплаты, будет играть
с большим вдохновением. Ведь кроме того, актер должен где-то работать.
Неправильно, если он целый день, понимаете, болтается в театре!

\begin{itemize} % {
\iusr{Ирина Петрова}
\textbf{Дмитрий Бартюк} 

увы, это мнение, как и множество множеств \enquote{мнений} было неверно в корне. Больше
того, вспоминая \enquote{МСНВ}, даже телевидение не вытеснило театры) \enquote{подвинул}
немного интернет, это да ...

\begin{itemize} % {
\iusr{Дмитрий Бартюк}
\textbf{Ирина Петрова} это просто цитата). Из Евстигнеева.

\iusr{Светлана Манилова}
\textbf{Дмитрий}, 

а мне другая цитата вспомнилась: \enquote{Ничего не будет. Ни кино, ни театра, ни книг,
ни газет – одно сплошное телевидение...} @igg{fbicon.smile} 


\iusr{Дмитрий Бартюк}
\textbf{Светлана Манилова} так это не у нас. Это в Москве)

\iusr{Ирина Петрова}
\textbf{Дмитрий Бартюк} 

Дим,  @igg{fbicon.heart.eyes}. Есть театральная байка, которая заканчивается
фразой : \enquote{Ребята, с вами всегда узнаёшь что-то новенькое!}.
@igg{fbicon.laugh.rolling.floor}{repeat=3} 

\iusr{Ирина Петрова}
\textbf{Светлана Манилова} 

дда, я ж и имела в виду под МСНВ умозаключение Рудольфа (Родиона)
@igg{fbicon.face.smiling.eyes.smiling} 

\iusr{Светлана Манилова}
\textbf{Ирина}, не заметила... @igg{fbicon.smile} 

\iusr{Светлана Манилова}
\textbf{Ирина}, 

зато я заметила в ветке выше \enquote{рассказать могу, что да, то да
@igg{fbicon.wink} }. Так что ждем! @igg{fbicon.smile} 

\iusr{Дмитрий Бартюк}
\textbf{Ирина Петрова} ))))))
\end{itemize} % }

\end{itemize} % }

\iusr{Раиса Карчевская}

Прекрасный рассказ. Спасибо большое

\iusr{Ирина Петрова}
\textbf{Раиса Карчевская} 

спасибо, что понравился рассказ об очень дорогом для меня, кусочке жизни
@igg{fbicon.heart.eyes} 

\iusr{Розалия Гольдецки}

Ирочка, дорогая!! Спасибо огромное за чудные воспоминания, эти счастливые дни
никогда не сотрутся из памяти... я все увидела... как наяву.. ведь мы
действительно были, как одна семья.. я не помню каких-то склок, интриг... все с
уважением относились друг к другу, радовались с блеском сыгранных ролей своих
партнеров..., переживали неудачи и всякие независимые от нас какие-то
проблемы... Я очень тебе благодарна за то, что ты всех помнишь... А еще я играла
в спектаклях \enquote{Любовь Яровая}, не помню названия - про профессора Полежаева,
\enquote{Васса Железнова}, поставленные моим учителем - А. Б. Гранатовым, и еще им
оцененные на дарственной фотографии... он верил, что у меня успешное будущее... И
в дальнейшем, я участвовала в других спектаклях - например играла главную роль -
председателя колхоза, с халой на голове, с орденом на пиджаке, но в другом
коллективе, и с 1987 года участвовала в еврейском театре ШОЛЭМ, пела и сольные
песни и в ансамбле... у нас был большой коллектив... вокальный ансамбль - девушки
, оркестр, мужской вокал, танцевальная молодежная группа, рассказчики,
конферанс . Детский танцевальный коллектив, был и свой композитор, хореограф, и
руководитель ансамбля Давид Мильштейн... очень нас хорошо принимали, мы ездили
по всей Украине с гастролями, т. к. уже было разрешено говорить и петь на
идыш... но настали другие времена. большая. часть ансамбля переехала в Израиль,
Америку, Австралию... И уже здесь, в Израиле я тоже играла в нескольких русских
театрах - \enquote{Ассорти на десерт}, \enquote{Про Фетэдота-стрельца, удалого молодца}, очень
много было постановок на русском, по мотивам рассказов Шолом-Алейхема с моим
партнером Гришей Певзнером из Ленинграда.. потом 15 лет я играла в театре на
иврите.. есть, конечно, постеры, афиши, спектакли \enquote{Это мы}, \enquote{Мосты}, \enquote{Вещи,
которые мы видим оттуда}, , я ушла из театра в 2013 году... я попала в аварию.. и
потом случился микроинсульт.. я не могла высказаться, не помнила текст... очень
жаль... Я восстановилась, но в театр уже хожу, как зритель... .Я желаю тебе
счастья и здоровья, и пусть все, что с нами было, навсегда останется в
счастливых воспоминаниях, целую, люблю...

\begin{itemize} % {
\iusr{Ирина Петрова}

Розочка, милая моя! Мы ж с тобой помним всё-всё. Я писала воспоминания и
плакала. О ком ни вспомню - Царство Небесное...

Я вот и призабыла, что ты начинала ещё у Гранатова! А вдруг сохранились фото? У
меня немного осталось ...основные были у Гавриловых. Какой же ты молодец -
ничего себе, сколько играла! Я уже с начала 90-х, когда начались смутные
времена, на сцену не выходила. Я так рада, что нашла тебя. Ты - единственная, с
кем из наших восстановилась связь. Обнимаю, красотка моя, так вот стоит перед
глазами \enquote{Гуска} и твоё \enquote{бах, у серце передчуття}
@igg{fbicon.laugh.rolling.floor}  @igg{fbicon.heart.eyes}{repeat=3}
@igg{fbicon.heart.red} @igg{fbicon.face.blowing.kiss}{repeat=3} 

Обнимаю, мы вместе! Береги себя!

\end{itemize} % }

\iusr{Розалия Гольдецки}

\ifcmt
  ig https://i2.paste.pics/575a6eff857dca2a35594afcbf0e7e7b.png
  @width 0.2
\fi

\iusr{Розалия Гольдецки}

\ifcmt
  ig https://i2.paste.pics/66cb0f960e54bef19591532b7042dcda.png
  @width 0.2
\fi

\iusr{Миленаа Милена}

А я смотрела почти все спектакли Вашего театра. И даже \enquote{Мышеловку} С
Базилевичем. Он играл Паравечини. И про Буслая.. и даже Притворщики.. И
естественно Гуску. Чесно сказать, такого професионализма от Народного театра, я
не ожидала. Мне очень нравился Ваш театр.

\begin{itemize} % {
\iusr{Ирина Петрова}
\textbf{Миленаа Милена} 

Миленочка, я прекрасно тебя помню ( на \enquote{ты} я по привычке, я помню славную
девочку). Спасибо, дорогая, так приятно, что есть ещё люди, которые были
очевидцами!

\iusr{Миленаа Милена}
\textbf{Ирина Петрова} ) И я Вас хорошо помню. И Сашу и родилей его!
\end{itemize} % }

\iusr{Nadiya M Shana}

Не помню какой спектакль, но по-моему ты за сценой производила звуки открытого
газового баллона. Я реально стала задыхаться! Талант, йо!
@igg{fbicon.thumb.up.yellow}  @igg{fbicon.face.blowing.kiss} 

\iusr{Розалия Гольдецки}

А еще при Гранатове были два одноактовых спектакля по Чехову - \enquote{Предложение} и
\enquote{Юбилей}... В Предложении - играл Валера Делиско и, по\hyp моему, Фаина, которая
потом стала его женой.. и я помню тот день, когда ..они поженились.. это была
сенсация.. она была старше его на 10 лет, и у нее был 18-летний сын.. а Валере
было 25.. а ...нет, я напутала... я играла в Юбилее по очереди с Фаиной, жену
директора банка, которого играл Базилевич... по ходу по роли я врываюсь в его
кабинет, рассказываю всякую всячину, щебечу, но при этом должна сесть к нему на
колени и целоваться ну как бы в губы, на самом деле щекой прикоснуться, боже
как я смущалась, я не могла допустить... как это чужой человек.. взрослый
дядечка.. а я сижу на коленях и целуюсь... в тот год как раз пришла к нам в театр
Вера Предаевич и поставила этих два дипломных спектакля... Ирочка! Я поищу
старые фото и пришлю тебе...

\iusr{Ирина Петрова}
\textbf{Розалия Гольдецки} 

да, я помню Делисок), я пришла, они ещё немного были и ушли. И помню родителей
Андрюши Чернышова). Андрюша теперь звезда!

\iusr{Margarita Kaminsky}
Браво!

\iusr{Natasha Levitskaya}

Спасибо, Ирина! Прекрасные театральные воспоминания! Такая яркая творческая
молодость! @igg{fbicon.hands.applause.yellow} 

\iusr{Ирина Петрова}
\textbf{Natasha Levitskaya} 

Наташа, спасибо! С театром в сердце по жизни - в зале, на сцене, за кулисами - только б рядом)

\iusr{Lola Shpilskaya}

Читала со слезами. Помню почти все спектакли, но особенно \enquote{Мышеловка} и
\enquote{Призраки}. Это был настоящий профессиональный театр. Сашу невозможно было не
любить, море обаяния... Он работал с моими родителями и бывал у нас дома.
Спасибо за чудесный рассказ!

\begin{itemize} % {
\iusr{Ирина Петрова}
\textbf{Lola Shpilskaya} 

спасибо Вам огромное! Я помню фамилию Шпильский, помню много сотрудников Саши.
Саша ушёл 28 октября 2019...и я очень любила Сашу... Спасибо, что помните,
спасибо, что увидели. Группа Киевские истории уже не раз, и не два приносит мне
встречи с людьми, с которыми в обычной жизни встретиться было бы нереально.
Очень тронута Вашими словами, Вашими воспоминаниями. Будьте здоровы!

\iusr{Lola Shpilskaya}
\textbf{Ирина Петрова} 

мне искренне жаль, что Саши нет... возможно это мистика, но я буквально на
прошлой неделе его вспоминала и рассказывала о Вашем театре своему знакомому,
который тоже играет... Желаю Вам здоровья и всего самого хорошего!

\iusr{Ирина Петрова}
\textbf{Lola Shpilskaya} спасибо! Будем жить и помнить...
\end{itemize} % }

\iusr{Liudmila Mironets}

Яке ж змістовне, щасливе життя, дуже радію за всіх, нажаль у часі і просторі
розминулась з вами, дуже би хотіла бути серед таких талановитих людей, ваші
спогади зігрівають душу

\iusr{Ирина Петрова}
\textbf{Liudmila Mironets} 

щиро дякую за теплий, сердечний відгук! Так, це був чудовий час, і спогади
назавжди в серці. Життя триває, та й в цьому часі є багато гарного, красивого,
цікавого та душевного. Я впевнена, що багато залежить від нас самих, від нашого
самосприйняття та сприйняття оточуючого життя! Гарного та доброго Вам!

\end{itemize} % }
