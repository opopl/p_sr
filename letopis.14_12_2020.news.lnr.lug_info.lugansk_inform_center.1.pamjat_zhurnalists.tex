% vim: keymap=russian-jcukenwin
%%beginhead 
 
%%file 14_12_2020.news.lnr.lug_info.lugansk_inform_center.1.pamjat_zhurnalists
%%parent 14_12_2020
 
%%url http://lug-info.com/news/one/predstaviteli-lnr-i-dnr-proveli-videomost-v-pamyat-o-pogibshikh-v-donbasse-zhurnalistakh-foto-62834
 
%%author ЛуганскИнформЦентр
%%author_id lugansk_inform_center
%%author_url 
 
%%tags 
%%title Представители ЛНР и ДНР провели видеомост в память о погибших в Донбассе журналистах
 
%%endhead 
 
\subsection{Представители ЛНР и ДНР провели видеомост в память о погибших в Донбассе журналистах}
\label{sec:14_12_2020.news.lnr.lug_info.lugansk_inform_center.1.pamjat_zhurnalists}
\Purl{http://lug-info.com/news/one/predstaviteli-lnr-i-dnr-proveli-videomost-v-pamyat-o-pogibshikh-v-donbasse-zhurnalistakh-foto-62834}
\ifcmt
	author_begin
   author_id lugansk_inform_center
	author_end
\fi

\ifcmt
tab_begin cols=3
	pic http://img.lug-info.com/cache/b/8/IMG_2672_novyj_razmer.JPG/1000wm.jpg
	pic http://img.lug-info.com/cache/e/a/(1)_IMG_2698_novyj_razmer.JPG/1000wm.jpg
	pic http://img.lug-info.com/cache/e/7/IMG_2695_novyj_razmer.JPG/1000wm.jpg
tab_end
\fi

\index[rus]{День памяти журналистов!Луганская народная республика, 14.12.2020}

Представители ЛНР и ДНР провели видеомост, посвященный Дню памяти журналистов,
погибших при исполнении профессиональных обязанностей. Об этом из Металлистской
средней школы имени Корнелюка и Волошина Славяносербского района, где проходило
мероприятие, передает корреспондент ЛИЦ.

Участие в мероприятии со стороны ЛНР приняли представители руководства
Республики, Общественной палаты и Молодежного парламента, администрации
Славяносербского района, а также работники СМИ, представители общественных
организаций, студенты и школьники.

Перед стартом видемоста представители ЛНР возложили цветы к памятному кресту,
установленному у поселка Металлист на месте гибели журналистов Всероссийской
государственной телерадиокомпании Игоря Корнелюка и Антона Волошина. 

Заместитель руководителя администрации главы ЛНР Вячеслав Матвеев подчеркнул,
что военная журналистика помогает сделать необратимым наказание преступника. Он
отметил, что в современном мире информация стала инструментом войны, зачастую
более действенным, чем классическое боевое оружие.

"Сегодня внимание к информации и к людям, которые эту информацию доносят, порой
ценой собственного здоровья и жизни, это достойно всяческого уважения", -
сказал он.

\ifcmt
tab_begin cols=3

	pic http://img.lug-info.com/cache/b/b/IMG_2689_novyj_razmer.JPG/1000wm.jpg
	pic http://img.lug-info.com/cache/a/3/IMG_2685_novyj_razmer.JPG/1000wm.jpg
	pic http://img.lug-info.com/cache/0/6/IMG_2683_novyj_razmer.JPG/1000wm.jpg

	pic http://img.lug-info.com/cache/0/a/(1)_IMG_2678_novyj_razmer.JPG/1000wm.jpg
	pic http://img.lug-info.com/cache/4/3/IMG_2677_novyj_razmer.JPG/1000wm.jpg
	pic http://img.lug-info.com/cache/5/4/(1)_IMG_2670_novyj_razmer.JPG/1000wm.jpg
tab_end
\fi

Глава администрации Славяносербского района, председатель территориального
отделения общественного движения "Мир Луганщине" Сергей Крамаренко отметил, что
профессия военного журналиста была и остается одной из наиболее опасных в мире.

"Сегодня мы вспоминаем об этих отважных людях: талантливых репортерах,
фотографах, телеоператорах, которые по всему миру, невзирая ни на что,
выполняли свой профессиональный долг. Они навсегда останутся в нашей памяти", -
отметил он.

Председатель Общественной палаты ЛНР Алексей Карякин добавил, что сегодня
жители ЛНР и ДНР чтят память тех людей, которые отдали свою жизнь за правду.

"Военный журналист – это спецназ, но без оружия, но их чаще берут в прицел и их
как раз пытаются сегодняшние враги, которые убивают мирное население, убить,
чтобы они замолчали, чтобы они не несли правду в мир", - сказал общественник.

Он поблагодарил журналистов со всего мира, которые не побоялись приехать в
Донбасс и рассказать правду о преступлениях киевских силовиков.

Помощник министра внутренних дел ЛНР Виталий Киселев рассказал, что встречался
с Корнелюком и Волошиным незадолго до их гибели.

Он отметил, что российские репортеры не побоялись, несмотря на уговоры, поехать
в район Металлиста, где на тот момент разворачивался бой с украинским
батальоном националистов "Айдар".

Киселев отметил, что российские журналисты носили опознавательные знаки прессы,
а также находились среди мирных жителей, покидающих поселок Металлист, что,
впрочем, не остановило киевских силовиков, начавших прицельный минометный
обстрел.

Военный эксперт Андрей Марочко напомнил, что только в 2014 году на Украине было
зафиксировано более 300 случаев избиения журналистов, девять журналистов
погибли.

\ifcmt
tab_begin cols=3
	pic http://img.lug-info.com/cache/a/b/IMG_2666_novyj_razmer.JPG/1000wm.jpg
	pic http://img.lug-info.com/cache/5/3/IMG_2655_novyj_razmer.JPG/1000wm.jpg
	pic http://img.lug-info.com/cache/2/3/IMG_2634_novyj_razmer_exposure.JPG/1000wm.jpg

	pic http://img.lug-info.com/cache/d/f/IMG_2628_novyj_razmer_exposure.JPG/1000wm.jpg
	pic http://img.lug-info.com/cache/7/c/IMG_2610_novyj_razmer_exposure.JPG/1000wm.jpg
	pic http://img.lug-info.com/cache/2/6/IMG_2608_novyj_razmer_exposure.JPG/1000wm.jpg

	pic http://img.lug-info.com/cache/d/5/IMG_2592_novyj_razmer_exposure.JPG/1000wm.jpg
	pic http://img.lug-info.com/cache/7/5/IMG_2589_novyj_razmer_exposure.JPG/1000wm.jpg
tab_end
\fi

"Это говорит о многом. Эти люди (погибшие и пострадавшие из-за агрессии ВСУ
журналисты) доносили правду всему миру. На тот момент никто из мирового
сообщества, кроме России, не поднимал тему конфликта на юго-востоке Украины.
Лишь благодаря этим людям, выполнявшим свой долг, весь мир узнал правду", -
сказал он.

День памяти журналистов, погибших при исполнении профессиональных обязанностей,
учрежден в 1991 году по решению Союза журналистов России и отмечается 15
декабря.

24 мая 2014 года под Славянском погибли итальянский фотокорреспондент Андреа
Роккелли и его переводчик Андрей Миронов.

В ночь на 30 июня 2014 года смертельное ранение получил оператор "Первого
канала" Анатолий Клян, работавший под Донецком в рамках своей 375-й
командировки. До этого он освещал боевые действия в Югославии, Чечне, Ираке,
Афганистане, Сирии.

Съемочная группа ВГТРК, готовившая репортаж об эвакуации жителей расположенного
под Луганском поселка Металлист, 17 июня 2014 года попала под минометный
обстрел. Звукорежиссер Антон Волошин погиб на месте, а корреспондент Игорь
Корнелюк получил тяжелое ранение и скончался в больнице. Памятный знак
журналистам, погибшим в Донбассе, был открыт в Луганске 17 июня 2019 года
главой ЛНР Леонидом Пасечником – в пятую годовщину гибели Корнелюка и Волошина
у поселка Металлист.

6 августа того же года погибли фотокорреспондент МИА "Россия сегодня" Андрей
Стенин и находившиеся в его машине сотрудники информационного подразделения ДНР
"IКОРПУС" Сергей Коренченков и Андрей Вячало. Автомобиль журналиста попал под
обстрел около Дмитровки в Донецкой области, в машине были обнаружены останки
всех троих. Стенин работал в Сирии, секторе Газа, Египте, Ливии, Турции и
других регионах. 

