% vim: keymap=russian-jcukenwin
%%beginhead 
 
%%file 24_12_2016.stz.news.ua.mrpl_city.1.kak_ulicy_nazyvali
%%parent 24_12_2016
 
%%url https://mrpl.city/blogs/view/kak-ulitsy-nazyvali
 
%%author_id burov_sergij.mariupol,news.ua.mrpl_city
%%date 
 
%%tags istoria,mariupol,mariupol.istoria
%%title Как улицы называли
 
%%endhead 
 
\subsection{Как улицы называли}
\label{sec:24_12_2016.stz.news.ua.mrpl_city.1.kak_ulicy_nazyvali}
 
\Purl{https://mrpl.city/blogs/view/kak-ulitsy-nazyvali}
\ifcmt
 author_begin
   author_id burov_sergij.mariupol,news.ua.mrpl_city
 author_end
\fi

% 1-2
\ii{24_12_2016.stz.news.ua.mrpl_city.1.kak_ulicy_nazyvali.pic.1}

У мариупольских властей почти сто лет не доходили руки, чтобы дать названия
городским улицам. Да и кому эти названия были нужны? Ведь к тому времени в
Мариуполе проживало не более 14 тысяч человек обоего пола. Все и так знали, кто
где живет и где что находится. Но вот исторический факт. Улицам Мариуполя 28
сентября 1876 года, а бывшим пригородным селам 15 сентября 1877 года были
присвоены собственные имена. До наших дней не дошли сведения, проводились ли по
этому поводу общественные слушания, учреждались ли комиссии по наименованиям,
спрашивали ли что-нибудь у жителей каждой улицы. Нужно думать, решение по этому
поводу было принято  городской думой по представлению городской управы. Вот и
все.

% 3 Базар
\ii{24_12_2016.stz.news.ua.mrpl_city.1.kak_ulicy_nazyvali.pic.2}
% 4 Магазин братьев Адабашевых
\ii{24_12_2016.stz.news.ua.mrpl_city.1.kak_ulicy_nazyvali.pic.3}
% 5 Собор Святого Харлампия
\ii{24_12_2016.stz.news.ua.mrpl_city.1.kak_ulicy_nazyvali.pic.4}
% 6 Торговая улица
\ii{24_12_2016.stz.news.ua.mrpl_city.1.kak_ulicy_nazyvali.pic.5}
% 7 Магазин и склад братьев Адабашевых
\ii{24_12_2016.stz.news.ua.mrpl_city.1.kak_ulicy_nazyvali.pic.6}
% 8 Екатерининская улица
\ii{24_12_2016.stz.news.ua.mrpl_city.1.kak_ulicy_nazyvali.pic.7}

Итак, названия и принципы наименования площадей и улиц Мариуполя, утвержденные
28 сентября 1876 года.  

\subsubsection{Площади}

\textbf{Соборная}, поскольку  на ней возвышался главный православный  храм – собор св.
Харлампия. Теперь на его месте стоит здание, в народе известное как ДОСААФ.

\textbf{Александровская} площадь названа в честь императора Александра II, по повелению
которого был проведен ряд реформ, в том числе и отмена  крепостного права.  В
наши  дни  этот топонимический объект называется Театральной площадью.

\textbf{Косьмо-Дамьяновская} площадь. На ней стояла церковь Рождества Богородицы, в
которой находилась  особо почитаемая икона Косьмы и Дамьяна. Мариупольцы
называли церковь Карасевской. Ее прихожанами были жители сельца Карасу-Базар.

\textbf{Марьинская} площадь, на которой была построена церковь Успения Божьей Матери.
Здесь молились аборигены предместья Марьинска. Сейчас это район школы №36 и
автовокзала.

\subsubsection{Поперечные улицы}

\textbf{Торговая} улица, на ней было сосредоточено множество торговых заведений.

Улица \textbf{Харлампиевская}, название понятно из приведенных строк о соборе. 

Городская магистраль, которая была разделена на две части Марии-Магдалинин\hyp{}ской
церковью. Она была заложена Азовским губернатором В. А. Чертковым, ее стены
частично были возведены еще до прихода греков. Часть улицы, направленная в
южном направлении, – \textbf{Магдалининская}, другая, направленная  на север, –
\textbf{Греческая}.

Следующая улица была  разъединена Александровской площадью на два отрезка. Один
- сориентированный в сторону моря – улица \textbf{Больничная}, потому что на ней
находилась земская больница, а второй отрезок - в сторону противоположную -
назван  \textbf{Таганрогской} улицей. Это было начало пути в Таганрог. Понятно, что речь
идет о современной улице Куинджи, в недавнем прошлом - ул. Артема.

За ней следовала улица \textbf{Константиновская}. Она была наименована в честь брата
царя Александра ΙΙ,  великого князя Константина Николаевича. Он дважды посещал
наш город: восемнадцатилетним юношей в 1845 г.  и будучи уже  зрелым человеком
- морским министром Российской империи - в 1872 г. 

\subsubsection{Продольные улицы}

\textbf{Малая и Большая Садовые} (современные ул. Семенишина и ул. Пушкина) как память о
том, что ранее эта часть города была занята фруктовыми садами.

Улица \textbf{Итальянская}. На пересечении улиц Торговой и Итальянской был построен  и
18 октября 1860 года освящен католический храм. Поскольку деньги на его
сооружение пожертвовали жившие в то время в Мариуполе итальянцы и итальянский
король Виктор-Эммануил, то в народе этот храм стали называть итальянским.
Отсюда и название улицы.

Улица \textbf{Георгиевская} – в честь святого Георгия – ему был посвящен правый престол
Харлампиевского собора.

\textbf{Екатерининская} улица, самая широкая по тем временам, с первых проектов
застройки Мариуполя предусматривалась как главная. Существуют две версии
наименования: одна - в честь императрицы Екатерины II и вторая – в честь
Екатерининской церкви, стоявшей рядом с Харлампиевским собором. Ни одна ни
другая версии не подтверждены документально.

Улица \textbf{Николаевская} – в честь святого Николая – ему был посвящен левый престол Харлампиевского собора.

Улица \textbf{Митрополитская}, на ней находилось подворье митрополита Игнатия.

\textbf{Фонтанная} улица. Она вела к фонтану, главному источнику водоснабжения Мариуполя.

Улица \textbf{Евпаторийская}, на которой располагалось село, основанное  греками-пере\hyp{}селенцами из Гезлева (Евпатории).

Улица \textbf{Кафайская}, на которой были построено село, основанное  греками-пере\hyp{}селенцами из Кафы (Феодосии).

\textbf{Готская} улица названа в честь древней  православной Готской епархии, находившейся  в Крыму.

Улица \textbf{Карасевская} – это село, основанное жителями Карасу-Базара.
Кстати, русифицированное название населенного пункта к карасям не имеет
никакого отношения.

\textbf{Бахчисарайская} улица. Само ее название говорит, из какого крымского города прибыли его первопоселенцы.

В завершение нужно сказать, что приведенный выше текст написан с использованием
материалов бесценного труда преподавателей Мариупольской Александровской
гимназии \enquote{Мариуполь и его окрестности}, изданного на средства мецената
Д. А. Хараджаева в 1891 году.
