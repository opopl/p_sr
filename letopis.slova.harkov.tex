% vim: keymap=russian-jcukenwin
%%beginhead 
 
%%file slova.harkov
%%parent slova
 
%%url 
 
%%author 
%%author_id 
%%author_url 
 
%%tags 
%%title 
 
%%endhead 
\chapter{Харьков}
\label{sec:slova.harkov}

В \emph{Харькове} скончался ветеран Великой Отечественной войны Михаил Карпеев. В
ноябре ему должно было исполниться 99 лет. Об этом 7 июня написал в Фейсбуке
директор областного Департамента соцзащиты населения Юрий Шпарага.  \enquote{Умер
Михаил Поликарпович Карпеев - известный \emph{харьковчанин}, Герой Советского Союза,
генерал-майор авиации, ему было 99 лет (по данным Википедии, ветеран родился 21
ноября 1922 года, то есть ему было 98 - Ред.)}, - сказано в посте чиновника.  В
комментариях Шпарага добавил, что Михаил Поликарпович был последний живой Герой
Советского Союза в \emph{Харькове},
\citTitle{В Харькове умер Михаил Карпеев - последний в городе Герой Советского Союза}, Наталья Полулях, kharkov.strana.ua, 07.06.2021

%%%cit
%%%cit_head
%%%cit_pic

\ifcmt
  tab_begin cols=4
		width 0.23

     pic https://cdn1.img.ukraina.ru/images/103170/36/1031703638.jpg

     pic https://cdn1.img.ukraina.ru/images/103170/42/1031704215.jpg

		 pic https://cdn1.img.ukraina.ru/images/103170/43/1031704337.jpg

		 pic https://cdn1.img.ukraina.ru/images/103170/44/1031704457.jpg

  tab_end
\fi

%%%cit_text
История \emph{Харькова} богата известными во всем мире деятелями искусства, особенно
художниками. Картины Ильи Репина и Генриха Семирадского украшают крупные
галереи и частные коллекции любителей искусства. Издание Украина.ру собрало
пять самых популярных работ \emph{харьковских} творцов «Запорожцы пишут письмо
турецкому султану» Это одна из самых известных картин Ильи Репина.  Художник
родился в Чугуеве \emph{Харьковской} губернии в 1844 году. Рисовать он начал рано, а в
16 лет покинул родные края ради работы в кочевой иконописной артели, а затем
отправился в Санкт-Петербург
%%%cit_comment
%%%cit_title
\citTitle{От «Запорожцев...» до «Грешницы»: пять всемирно известных картин харьковских художников}, 
Кристина Жалова, ukraina.ru, 23.06.2021
%%%endcit

