% vim: keymap=russian-jcukenwin
%%beginhead 
 
%%file 13_04_2021.fb.1.ukraina_rusmir
%%parent 13_04_2021
 
%%url https://www.facebook.com/Michaell5522/posts/4545463905469722
 
%%author 
%%author_id 
%%author_url 
 
%%tags 
%%title 
 
%%endhead 

\subsection{Насильно мил не будешь... И 'Русский мир' на Украине не построишь}
\label{sec:13_04_2021.fb.1.ukraina_rusmir}
\Purl{https://www.facebook.com/Michaell5522/posts/4545463905469722}

\ifcmt
  pic https://scontent-ber1-1.xx.fbcdn.net/v/t1.6435-9/172542213_4545430638806382_4294921517591310808_n.jpg?_nc_cat=110&ccb=1-3&_nc_sid=8bfeb9&_nc_ohc=LMqD77nKapcAX9k4Qg8&_nc_ht=scontent-ber1-1.xx&oh=23ea4b588a580c5c8925e56d0bee5b30&oe=609E038F
\fi

Вчера в эфире передачи "Место встречи", обсуждая ситуацию на Украине, опять прозвучали одни из самых любимых вопросов моих оппонентов, которые они часто адресуют мне:
"Чё США лезут на Украину? Чё Америка там делает? Чё она забыла там? Где Америка и где Украина?"
С таким же самым успехом  можно задать им встречный вопрос: 
"Что Россия делает в Сирии или Венесуэле? Где Россия и где Венесуэла? Что Россия забыла в Венесуэле?"
"Но Россия действует в Сирии и Венесуэле по приглашению там законно избранных властей!"
А США тоже действуют на Украине по приглашению законно избранных властей в Киеве (если игнорировать, на секунду, что власти в Венесуэле и Сирии в корне нелегитимные, и нельзя поставить их фальшивые выборы на одну доску с украинскими, которые даже Кремль признал легитимными). 
"Но Россия помогает Сирии сохранить территориальную целостность!" – восклицают  мои оппоненты.
А США также помогают украинским властям укрепить свою оборону и защищать Украину от дальнейших нарушений её территориальной целостности. 
Обвинение США в том, что они "вмешиваются" в дела Украины и "лезут" в российское собственное ближнее зарубежье -- это примитивный и нелепый перевод стрелок. 
Все прекрасно понимают, что украинское обращение к США за помощью в области безопасности и американское оказание в какой-то степени этой помощи являются прямым ПОСЛЕДСТВИЕМ российской агрессии 2014 года. 
Если бы не было аннексии Крыма и военной интервенции в Донбассе, Украина, безусловно, сохранила бы свой прежний нейтральный, внеблоковый статус... Россия и Украина сохранили бы свой прежний двусторонний крупный объем торговли....и обе страны сохранили бы свои прежние нормальные и, в целом, добрососедские отношения.
Россия может только себя обвинять в том, что:
-- Украина отвернулась от России и повернулась к Западу;
-- Украина обратилась к США за помощью;
-- США заменили Россию в качестве главного стратегического партнёра Украины;
-- Россия потеряла Украину на многие, многие десятилетия. 
Переведя эту геополитическую ситуацию на простую бытовую, представьте себе ситуацию, при которой муж каждый второй день бил свою жену и отобрал ("аннексировал") принадлежащую ей по закону шикарную летнюю дачу с безумно красивым видом на Чёрное море. 
В результате этого насилия и потери собственности, жена развелась с мужем, выдворив его из дома на законной основе и наняла для её дома частного охранника с оружием 24\7, чтобы муж больше не приходил к ней домой, больше не бил ее, и чтобы он не захватил еще и ее квартиру, на которую она получила права собственности в 1991 г.
И представьте, что после того, как бывшая жена предприняла эти базовые оборонительные меры безопасности, муж, впадая в истерику, начал обвинять именно ОХРАННИКА (!!) в том, что тот "вмешивается" в его частные, семейные дела, что он "лезет туда, куда не надо", и что его бывшая жена "продажная", потому что она "осмелилась" от него уйти (!).
Муж возмущённо кричит охраннику: "Что Вы там делаете в МОЕМ доме? Это же МОЯ жена" -- хотя это уже давно не его дом и не его жена! 
Ведь тот муж считает, что его жена не имеет права уйти от него, поскольку она, мол, относится к его исключительной "сфере интересов" (точно также как Россия считает, что Украина -- по определению -- относится к исключительно российской сфере интересов и должна подчиняться Москве.)
Такие аргументы этого мужа совершенно нелепы --- столь же нелепы, как аргументы моих оппонентов, обвиняющих США в том, что они "лезут" и "вмешиваются" в дела Украины, и вопрошают на голубом глазу: "Какое Америке дело до Украины? Что США там забыли?"
