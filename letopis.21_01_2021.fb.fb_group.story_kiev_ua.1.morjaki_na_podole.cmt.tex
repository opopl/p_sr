% vim: keymap=russian-jcukenwin
%%beginhead 
 
%%file 21_01_2021.fb.fb_group.story_kiev_ua.1.morjaki_na_podole.cmt
%%parent 21_01_2021.fb.fb_group.story_kiev_ua.1.morjaki_na_podole
 
%%url 
 
%%author_id 
%%date 
 
%%tags 
%%title 
 
%%endhead 
\zzSecCmt

\begin{itemize} % {
\iusr{Vitali Andrievski}

Молодец Петр. Я подолянин, но к Морполиту никакого отношения не имел. Зато
много лет работал на Днепре на круизных теплоходах, в основном на ВУЧЕТИЧЕ. Да
тогда Днепр был очень оживленный. Начальником Главречфлота был Н. А. Славов,
благодаря которому и Днепр и многое на Подоле оживилось. Жаль, что после него
все это пришло в запустение, пассажирского флота практически не существует

\begin{itemize} % {
\iusr{Віта Дімінська}
\textbf{Vitali Andrievski} Мій батько працював в відділі кадрів при А. Н. Славові. Ще був теплохід 25 з'їзд

\iusr{Vitali Andrievski}
\textbf{Віта Дімінська} 

був не тiльки 25 зiзд, на якому до речi, менi теж довелося працювати. Загалом
було 12 круiзних теплоходiв. На жаль зараз немае жодного з них


\iusr{Віта Дімінська}

Як казав мій батько, було три флагмани Главрічфлота :25, ВУЧЕТИЧ... третього не
пам'ятаю! На вихідних ходили на Канів, там і зустрічались.


\iusr{Vitali Andrievski}
\textbf{Віта Дімінська} 

було загалом 12 теплоходiв 301 i 302 проектiв. Першим був у 1976 роцi \enquote{Евгений
Вучетич} а потiм \enquote{25 сьезд}, \enquote{Ленин}, \enquote{советская Россия}, \enquote{добролюбов}. Это
были 301 проекта а потом пошли 302 проекта \enquote{ Глушков}, \enquote{Ватутин}, \enquote{Рыбалко},
\enquote{Ватченко}, \enquote{кошевой}, \enquote{лавриненков} и последний был \enquote{Т. Шевченко}. Загалом 12
теплоходiв. Жодного на жаль зараз на Днiпрi не залишилось

\end{itemize} % }

\iusr{Евгений Мартынович}
Спасибо за содержательный рассказ!

\iusr{Елена Свец Баркан}

Я в свое время работала в училище на Андреевской и была в совете молодых
педагогов Подольского района, отвечала за культмассовую работу. Организовывала
вечера отдыха, как они тогда назывались. Проходили они чаще всего в кафе
\enquote{Ярославна} на Константиновской. Молодые педагоги обычно были женского пола и,
для того, чтобы вечера проходили весело и приятно, приходилось прибегать к
помощи комсоргов КВВМПУ. Ходила к ним на проходную, договаривалась о количестве
ребят. Не знаю, может кто-то таким образом свою судьбу устроил?)

\begin{itemize} % {
\iusr{Петр Кузьменко}
\textbf{Елена Свец Баркан} не сомневаюсь! @igg{fbicon.wink} 

\iusr{Анатолий Кашпур}
Так вы, Елена, свахой работали?  @igg{fbicon.beaming.face.smiling.eyes} 

\iusr{Елена Свец Баркан}
\textbf{Анатолий Кашпур} скорее организатором мероприятий (массовиком-затейником) @igg{fbicon.grin} 

\iusr{Нелли Кузьменко}
\textbf{Елена Свец Баркан} Не поверишь, я на танцы ходила только тогда, когда Петюня на вахта стоял дежурный по клубу.

\iusr{Нелли Кузьменко}
\textbf{Елена Свец Баркан} И когда наши ребята, с Петькиного класса, на дискотеки играли, группа \enquote{Наследники}

\iusr{Роман Воловенко}
\textbf{Елена Свец Баркан} 

тогда модно было проводить случки между муж и жен уч заведениями зарождались
новые семьи нас авиакурсантов возили на случку в пердильное училище в Кр Рогу.

\end{itemize} % }

\iusr{Ольга Беляева}

А я с подружками на дискотеку в КВВМПу ходила... Потом выпускной, ребята в
форме с аксельбантами и кортиком-красавцы!!! А еще розы шикарные у главного
входа!!!! Как вчера...


\iusr{Марина Бортницкая}

Спасибо, чудесно написали повествование, Я жила во 2 номере Андреевского
спуска, мои окна выходили на Военно-морское училище, часто смотрела, с братом на
подготовки к парадам, жаль, всё ушло, и якоря убрали, только остались пару кустов
роз,,,

\begin{itemize} % {
\iusr{Петр Кузьменко}
\textbf{Марина Бортницкая} это мой родной дом. О нём есть мой отдельный пост в группе.

\iusr{Марина Бортницкая}
\textbf{Петр Кузьменко} 

спасибо поищу, с удовольствием прочитаю, на Андреевском спуске 2 жили с 1974 года
до отселения 1990 года, училась в 20 школе, мой брат в 100 школе, каждый раз, когда
бываю на Андреевском находит грусть, и в моем доме сейчас типа модное заведение
реберня Пьяная вишня,

\iusr{Петр Кузьменко}
\textbf{Марина Бортницкая} 

просто нажмите на мою аватарку в группе. Увидите все мои публикации.
Практически в каждом моём посте есть упоминание о моём родном доме.

\iusr{Марина Бортницкая}
\textbf{Петр Кузьменко} спасибо, мне интересно

\iusr{Марина Бортницкая}
\textbf{Петр Кузьменко} Спасибо большое, нашла, очень интересно
\end{itemize} % }

\iusr{Оксана Тернавская}
Спасибо большое Вам за рассказ от семьи военного моряка)))

\iusr{Олександр Громов}
Я со второго выпуска, 1972 год. Когда мы поступили было только два здания.

\iusr{Олег Гринюк}

Я жил в этом доме Волошская - Григория Сковороды, ранее
Набережно-Никольская 12/4, т. е. до 1977 года, когда наш дом и был передан КВВМПУ, и
поступал в 1975 экзамены сдавали в Лютеже.

\begin{itemize} % {
\iusr{Павел Пауль}
\textbf{Олег Гринюк} дом известный. Жили там замечательные люди.

\iusr{Олег Гринюк}
\textbf{Павел Пауль} 

дом Водников, капитаны Теплоходов, мой отец закончил в 1955 КИевский речной
техникум, потом ЛИВТ БЫЛ Начальником ОТК КССРЗ им. Сталина, потом просто
судостроительный завод., а в доме действительно жили Прекрасные люди, научные
работники, профессора, механики СТешек-барж, их капитаны, двор был очень дружный.


\iusr{Павел Пауль}
\textbf{Олег Гринюк} 

брат моего деда на Ленкузне мастером котельного цеха был. Марчук Евгений
Григорьевич. Содеди его Русаковы. Начальником Гимс был Виктор. Часто бывал там
в детстве.


\iusr{Олег Гринюк}
\textbf{Павел Пауль} 

Русаковы жили в нашем подъезде, Вова был гребцом бойдарочником чемпионом мы
дружили, потом они жили там где Ген. Прокуратура на Подоле, Раиса Соя гребчиха то
же чемпионка и ее семья мои соседи по коммуналке, да были люди в наше время....


\iusr{Павел Пауль}
\textbf{Олег Гринюк} я старше, а то могли бы и встречаться. Батя мой одно время тоже там жил. Он механиком на теплоходах работал.

\iusr{Олег Гринюк}
\textbf{Павел Пауль} значит моего отца точно знал.

\iusr{Павел Пауль}
\textbf{Олег Гринюк} однозначно. Директором завода помню Матяш был, деканом института Платонов А. М.

\iusr{Олег Гринюк}
\textbf{Павел Пауль} это друзья моего отца, я их хорошо помню, пацаном был,

\iusr{Павел Пауль}
\textbf{Олег Гринюк} на охоте пацаном с ними рос. Начальником пароходства Бобровникова ещё был. Марчук Вильям Павлович-отец мой.

\iusr{Олег Гринюк}
\textbf{Павел Пауль} Бобровникова если не ошибаюсь Анатолием звали.

\iusr{Павел Пауль}
\textbf{Олег Гринюк} Бобровникова Николай Андреевич.
\end{itemize} % }

\iusr{Ольга Кирьянцева}

Спасибо большое Вам за рассказ. Так замечательно написано! С огромной любовью и
гордостью за свое училище. Я жила некоторое время у родственников на Подоле,
все пути мои проходили через Красную площадь. И мне всегда хотелось хоть одним
глазком подглядеть, где и как учатся эти молодые, красивые, подтянутые
ребята-курсанты). Это ж столько девчачьих сердечек замирало и таяло при виде
тех красавцев!)) Знаю по себе)). А если удавалось попасть на их построение на
площади, то это был настоящий празник для души)). Спасибо! Вы были (и
остаётесь, я уверена,) - классными!

И с праздником Вас и всех Ваших друзей - выпускников училища.

\iusr{Елена Лясковская}

Да, попасть к вам на дискотеку было нелегко ! Но мне с подружкой однажды
повезло в далёком 1985 году @igg{fbicon.thumb.up.yellow}  @igg{fbicon.woman.dancing} 

\iusr{Ольга Писанко}
А мы были пару раз на дискотеке с подругой в году 79!

\iusr{Недоля Алексей}

А у нас со второго курса 12 роты, 1 группы Рижского Мореходного Училища в
Киевский Морполит перевёлся в 1987 году Литовченко Нормунд. Удивительная
история: я коренной Киевлянин уехал в Ригу, он, Латыш в Киев.

Хотелось бы узнать его судьбу. Может кто учился с ним вместе?


\iusr{Юлия Лепашева}

А у нас в доме батюшка живет, Сейчас батюшка, а в конце 80 ых бегал в форме
Морполита.:):):) Наверное полит. работники и духовники миссию выполняют
одинаковую.:):):)

\begin{itemize} % {
\iusr{Петр Кузьменко}
\textbf{Юлия Лепашева} об этом я писал в комментариях к посту о библиях в нашей группе. Еще в 1980-м олимпийским году я собирался поступать в Духовную семинарию.  @igg{fbicon.hands.shake} 
\end{itemize} % }

\iusr{Ілона Луценко}

Мои дедушка с бабушкой жили на Спасской улице и всегда гуляли в парке моряков.
Дедушка очень хотел что я его Внучка вышла замуж за военного моряка. Его мечта
сбылась через много лет.

\iusr{Alik Gurevich}

Пётр! Можно поправить несколько неточностей в Вашем замечательном
повествовании... С 1-го июня 1967 года началось формирование училища,
торжественное открытие состоялось 1-го октября того же года! Знаю точно, так
как в январе 1967 года был направлен в школу ОСНАЗ ВМФ и до середины июня
проходил обучение и службу там, в здании на Красной площади. А после, нас
перевели оканчивать обучение на Рыбальский .... Так как начался ремонт зданий и
образование училища.

\begin{itemize} % {
\iusr{Петр Кузьменко}
\textbf{Alik Gurevich} 

спасибо за уточнение, которое уверен с удивлением воспримут наши выпускники.

\iusr{Владимир Картавенко}
\textbf{Петр Кузьменко} 

вначале был командующий Днепровской военной флотилией матрос Полупанов, потом
младший морполит и две школы на Рыбальском. Решением Щербицкого В.В. подписаны
постановления о создании КВВМПУ и 316 УО ОСНАЗ в/ч 20884 ЧФ.

\end{itemize} % }

\iusr{Владимир Решетило}
Спасибо за воспоминания. Немного грустно, что так быстро всё проходит. 1974-1978.

\iusr{Valentina Belkova}
Эти исторические факты нужно передавать из поколения в поколение. Спасибо вам!

\iusr{Alla Zgurzhnitsky}
Я помню, моя одноклассница просила меня идти с ней мимо училища. Ей нравились курсанты.

\iusr{Valentina Belkova}

Мой муж рассказывал нашему сыну и про Реч. вокзал, Ленинскую кузню и про
Могилянку, но к сожалению ему было не интересно. Это только наше поколение
сейчас вспоминает и интересуется.

\iusr{татьяна неверовская}
\textbf{Valentina Belkova} но все равно рассказывать надо, постепенно всему свое время..... !

\iusr{Ирина Шушпанова}

А я выпускница Киевского судостроительного техникума в 1984 г получила в
награду мужа - замполита (выпуск КВВМПу 1985 г)


\iusr{Петр Кузьменко}
\textbf{Ирина Шушпанова} моя жена выпускница КССТ 1985 года. Мы знаем выпускников КВВМПУ 1985 года. С некоторыми давно и плотно дружим.

\iusr{Оксана Гриценко}
Учились мои родственники - Строй Николай !!! где-то в 70-71 годах!

\iusr{Татьяна Смирнова}

У КВВМПУ викладав мій батько, був начальником лабораторії мій брат. А зараз я
викладаю у НаУКМА. Так життя цілої родини виявилось пов'язаним з цими
історичними будівлями.


\iusr{Ольга Ненашева}
Мой папа окончил это училище в 1956 году, а мама университет в этом же году и поехали в Севастополь

\iusr{Анна Мерикова}

В этом училище работала в медсанчасти - стоматолог Людмила Серафимовна - врач от
Бога и с золотыми ручками. Она была хорошей знакомой моей бабушки.


\iusr{Татьяна Лещинер}

Я гостила у бабушки в Киеве, на Подоле, на Спасской, 16. В 1968 году. Так
курсанты ходили мимо дома в баню. Каждую неделю. Один раз пели \enquote{Легендарный
Севастополь}, на сл. неделе \enquote{Ах какая Черноморочка, как такую не любить}. Мы
смотрели в окно.

\iusr{Евгений Бабенко}
Помню ППР ( парт. полит. работу ) - посидели, попи₴дели, разошлись...
Столько людей дурью маялись. А если бы во благо?

\iusr{Андрей Пожарский}
Отлично!

\iusr{Наталья Сенниченко}

Курсанты Высшего Военно - Морского политического училища каждый вечер ходили
строем и с песнями мимо окон дома, где я жила. Я тогда была школьница, но до
сих пор помню слова этих песен ( \enquote{Черноморочка}, \enquote{Легендарный Севастополь} и
т.д.). Потом, когда я заканчивала школу, наш дом ( улица Волошская 10) забрали
для нужд этого училища, а жильцов расселили в разные районы города. Нам
досталась квартира на Левом берегу в панельной девятиэтажке и мой папа почти
двухметрового роста в.первое время постоянно бился головой о форточку с
непривычки ( на Подоле потолки были более 4-х метровой высоты и почти метровой
ширины подоконники). Радовало одно - близость Днепра, мы привыкли жить \enquote{у
воды}. В доме на Волошской жили мои дедушка и бабушка, родилась моя мама, там
же они жили во время оккупации и там же во дворе погибла в 1942-м году моя
бабушка. Сейчас в нём располагаются несколько факультетов Киево - Могилянской
Академии.

\iusr{John Smilerainman}

Одно из сильных впечатлений детства это приход двух настоящих моряков к нам в
детский садик. Форма, какие-то рассказы, а в конце подарили по золотому якорю.

Уверен, что это были курсанты Морполита  @igg{fbicon.smile} 

\iusr{Alexandr KatzNacht}
Это называлось вечерняя прогулка - обязательный ритуал в вооруженных силах.

\iusr{Сергей Удовик}
Я в КВВМПУ переподготовку проходил

\iusr{Valentina Kharenko}

Моя мама Анна Владимировна Харенко много лет преподавала немецкий язык в
училище. К курсантам относилась по-матерински. Заведовала кафедрой иностранных
языков Нинель Михайловна Сазонова.

\begin{itemize} % {
\iusr{Sergey A. Mozgovoy}
\textbf{Valentina Kharenko} у Вас замечательная мама! Анна Владимировна - прекрасный человек и педагог!  @igg{fbicon.hearts.revolving}  @igg{fbicon.rose} 

\iusr{Valentina Kharenko}
\textbf{Sergey A. Mozgovoy} 

Спасибо. Мне кажется, я Вас видела в училище. А мамочка, к сожалению, ушла от
нас в 2016 году. До последних дней декламировала Гете, Гейне на немецком языке,
прекрасно общалась на английском языке, сохранила чистое сознание и тягу ко
всему новому. А в те годы она жила своей работой, уходила с прекрасным
настроением на работу, рассказывала о своих курсантах, очень их любила, писала
методички, бежала на консультации. Я по-прежнему думаю, чтобы сказала мама по
этому поводу, как бы она порадовалась за нас.


\iusr{Мария Константиновская}
\textbf{Valentina Kharenko} а химию преподавала моя подруга Балаба Тамара Александровна...
\end{itemize} % }

\iusr{Татьяна Шиверская}
Боже мой, так все помнить! как я хотела в юности выйти замуж за моряка. Но увы!
Преклоняюсь перед автором

\iusr{Игорь Луговский}
Девчонки все крутились там у центрального входа, ждали дискотеку,!  А, мы, трижды
играли с морполитом в футбол и два раза выигровали Ещё там препадовали боевое
самбо, вход где якорь. со стороны кр. площади И конечно знал троих ребят с района
что служили в оркестре Морпалита Юра мореман стал мужем моей одноклассницы
Марины. Морпрлит это часть истории и украшение района

\iusr{Анна Мерикова}
А еще, когда я была студенткой педучилища мы бегали на дискотеку в морполит....

\iusr{Виктория Салтанова}

Петр, спасибо Вам за рассказы. Наш учитель математики Семен Исаакович, до школы
преподавал в морполите. Мечтала выйти замуж за моряка... Вышла за летчика)))

\begin{itemize} % {
\iusr{Петр Кузьменко}
\textbf{Виктория Масич} 

Семён Исакович сам заканчивал это же училище, когда оно ещё было средним
политическим. Об этом училище я упоминал в моём посте. Когда мы учились,
выпускники того училища, среди которых были и Семён Исакович, наш школьный
преподаватель математики, и тогдашние наши преподаватели КВВМПУ, собирались на
30 - летие своего выпуска. Нам они казались седыми ветеранами. А теперь, 40 -
летие нашего выпуска не за горами. В 2026...

\end{itemize} % }

\iusr{Sergey A. Mozgovoy}
Два вопроса: 1. Где сейчас Знамя училища? 2. Кто и куда дел музей училища?

\begin{itemize} % {
\iusr{Петр Кузьменко}
\textbf{Sergey A. Mozgovoy} 

знамя училища так же как и экспонаты музея хранятся в Украинском Музее Великой
отечественной войны в Киеве. Сейчас он называется музеем второй мировой войны.
По запросу комитета ветеранов КВВМПУ знамя выдают на встречи выпускников.
Правда в сопровождении сотрудника музея.


\iusr{Sergey A. Mozgovoy}
\textbf{Петр Кузьменко} Спасибо!
\end{itemize} % }

\iusr{Наталия Педос}
Очень интересное повествование. Спасибо!

\iusr{Petr Ivanov}

мой дядя... уже давно ушедший ..был в 70годах начальником политотдела училища
..кап 1..... и после школы у меня был только один путь ..стать курсантом.... в
1973г. после окончания школы.. меня повезли в Лютеж.. сдавать экзамены.... и тут
пошло.. что то не так ..и я слава Богу завалил математику... и был послан домой
..на Волошскую... я абсолютно не расстроился... Но родители наехали на дядю
Васю... который был в отпуске и не проконтролировал племяша.... был серьезный
разбор ..и меня тут же по согласованию с начальником училища тут же взяли на
=работу=...слесарем в автопарк училища.... и еще такой же =работник= стал
маляром.... мы приняли присягу и стали заниматься с преподавателями в порядке
так сказать исключения..... Внимательно вникнув в то что меня ждет в дальнейшем
я быстренько убежал совсем в другие совсем не худшие войска... и не пожалел об
этом решении ни разу..... единственное что осталось у меня в военном
билете.... профессия... аквалангист... т. к я закончил курсы в училище.... А товарищ
мой =маляр= поступил в 1974г.. и благополучно оттрубил в Североморске на
под. лодке... ушел на гражданку капитаном первого ранга.... с кучей
проф. болячек... и с вопросом... как жить дальше.... Короче... других уж нет.. а те
далече..... Иногда встречались ..пили по 100гр.... и я ни чуть не жалею... что так
сложилось..... насколько я знаю.. больше гражданских
=специалистов.. слесарей.. маляров=..в КВВМПУ не брали!!!

\iusr{ZabaVa Da}

Петр, спасибо, приятные воспоминания из детства тоже с морским оттенком-в детсад
приходили красивенные морячки, как раз напротив огромные ворота со
звездой.. напротив Фроловского @igg{fbicon.face.rolling.eyes} 

\begin{itemize} % {
\iusr{Петр Кузьменко}
\textbf{Рита Хомик} 

это мой 98 садик. Я писал о нём в своих постах в группе. Кстати монастырь
Флоровский, а улица неподалёку Фроловская.

\begin{itemize} % {
\iusr{Валерий Панчук}
\textbf{Петр Кузьменко} на руси флоров не было поэтому привычней фрол поэтому фроловская да и мы всегда говорили фроловский монастирь

\iusr{Петр Кузьменко}
\textbf{Валерий Панчук} и мы говорили. Но правильно Флоровский.
\end{itemize} % }

\iusr{Валерий Панчук}
Конечно в честь греческого святого флора

\iusr{Валерий Панчук}
\textbf{Рита Хомик} там вч вв сейчас нацгвардии

\begin{itemize} % {
\iusr{Петр Кузьменко}
\textbf{Валерий Панчук} ВВ там было всегда. Новороссийский полк ВВ.

\iusr{Валерий Панчук}
\textbf{Петр Кузьменко} а почему новороссийский проясните

\iusr{Петр Кузьменко}
\textbf{Валерий Панчук} 

мы школьниками были там на экскурсии. Название получено потому, что
подразделение НКВД, на основе которого полк был сформирован проявило героизм в
Великой отечественной войне при боях за Новороссийск.

\end{itemize} % }

\end{itemize} % }

\iusr{Наталия Беляева}

Пойти на танцы в \enquote{морполит}? Замечательная идея! Надо отметить, что играл там
вокально-инструментальный ансамбль, состоящий из курсантов этого же училища, и
играли ребята хорошо!  @igg{fbicon.hand.waving} 

\begin{itemize} % {
\iusr{Петр Кузьменко}
\textbf{Наталия Беляева} 

это были мои однокурсники и друзья. Ансамбль назывался \enquote{Наследники}.
Благодарю, что помните!@igg{fbicon.heart.red}

\iusr{Ирина Нищимная}

Я тоже бегала с подружкой на дискотеки в морполит, живо все так написали,
спасибо, помню как мы ходили смотреть выпускной в Морполите, как жаль, что
утрачено такое заведение,

\end{itemize} % }

\iusr{Андрей Могильный}

Небольшая ремарка: Киевской академии водного транспорта тоже уже нету) она
вошла в дуит (Государственный университет инфраструктуры и технологий )
спасибо. Интересный рассказ!


\iusr{Ирина Архипович}
Спасибо за интереснейший экскурс!!!  @igg{fbicon.hands.applause.yellow}  @igg{fbicon.face.happy.two.hands} 

\iusr{Ольга Рожненко}
Петр, спасибо за рассказ, пишите ще!

\iusr{Сергей Сергеев}

Как не помнить морполит - прм и прд караула на гарнизонной гауптвахте, участие
в соревнованиях КВО, встреча с патрулем от морполита на конечной 62 автобуса,
которые всегда заканчивались кроссом (был такой фетиш у патрулей - поймать не
менее 5 курсантов с нарушением формы одежды) по моему родному Подолу и т.д...
))))


\iusr{Арман Кеосаян}

Помню, когда был выпуск, курсанты срывали и бросали на Ильинской золотистые
якоря, а дети бегали и подбирали. Особенно вечерние песни запомнились -
\enquote{Маруся, кап, кап, кап...} Лет через десять встрелись выпускники Морполита и
оказалось, что это была песня как раз их роты. Встретились как родные: подоляне
и бывшие курсанты. Хотя уже все было в прошлом: и Морполит, и старый Подол


\iusr{Павел Пауль}
Флаг на трубе Тэц 1 - это тоже курсанты...

\iusr{Анна Василенко}

Спасибо за рассказ. Я студентка педучилища бегали с подружкой на дискотеки
Морполита. Было очень круто))) весело. И ребята с которыми мы встречались
бегали к нам в своих комбезах.)))))

\begin{itemize} % {
\iusr{Naty Lukina}
\textbf{Анна Василенко} 

мы там тоже пропадали, компания была внушительная, 3 года дружили до их
выпуска, кто женился таки и со всеми поддерживаем до сих пор отношения, хоть
многие поразъехались...

\iusr{Анна Василенко}
\textbf{Naty Lukina}, да знакомо))) я тоже чуть не вышла)))
\end{itemize} % }

\iusr{Yurii Kadochnikov}

Ще була така класна рота забезпечення учбового процесу. Ото хлопцям повезло
служити моряками та ще й в Києві ))

\begin{itemize} % {
\iusr{Юрий Стебельский}

А чего вдруг не на своём родном русском языке, а на мове, дорогой русский друг
Юрий Кадочников (\textbf{Yurii Kadochnikov})? Угнетают, сломили или добровольный выбор?

\begin{itemize} % {
\iusr{Yurii Kadochnikov}
\textbf{Yuriy Stebelskiy} у мене рідна українська, я навіть школу українську закінчував))))

\iusr{Юрий Стебельский}
\textbf{Yurii Kadochnikov} я тоже украинскую школу заканчивал. Но мой родной язык русский и русского я в себе убивать не собираюсь.
А у папы Кадочникова тоже родным языком был украинский?

\iusr{Yurii Kadochnikov}
\textbf{Yuriy Stebelskiy} так, от дід був російськомовним, до 1948 р.

\iusr{Юрий Стебельский}
\textbf{Yurii Kadochnikov} а из какого папа села? Где на Украине есть сёла с укромовными Кадочниковыми?
А зачем русский дед после 1948 года перестал быть русскоговорящим?

\iusr{Yurii Kadochnikov}
\textbf{Yuriy Stebelskiy} Дід, Олександр Іванович Кадочніков, в одному таборі під Олою з бандерівціями по 58ій чалився. Там і вивчив))))
\iusr{Юрий Стебельский}
\textbf{Yurii Kadochnikov} дед, значит, убил в себе русского а его потомки по сути - дети и внуки янычара. Теперь понятно.
А дед чтил бандеровцев и Бандеру?

\iusr{Yurii Kadochnikov}
\textbf{Yuriy Stebelskiy} там цікава та довга історія, якось при зустрічі розповім

\iusr{Юрий Стебельский}
\textbf{Yurii Kadochnikov} 

особенно интересно, читали ли бандеровцы \enquote{Майн кампф}? Сам-то Бандера наверняка
читал. В Европе в те годы это был хит, бестселлер. И Гитлер писал в \enquote{Майн
кампф}, что Украину надо захватить, заселить её германцами а с коренным
населением поступить так, как с индейцами США.

Как думаешь, Бандера знал о планах Гитлера? Хотя бы уж в 1944-му году знал?

В том году, задолго до конца войны фашисты выпустили Бандеру на свободу и потом
он жил и работал в Германии.
\end{itemize} % }

\iusr{Сергей Сергеев}
\textbf{Yurii Kadochnikov} В Лютеже дислоцировался батальон (БОУП), а не рота.

\iusr{Yurii Kadochnikov}
\textbf{Сергей Сергеев} можливо, то ж давно було

\end{itemize} % }

\iusr{Юлия Кравченко}

А те самые, начищеные до блеска, якоря, что стояли у центрального входа, теперь
стоят на ул.Волошской у входа в "Спортлайф".

\iusr{Alexander Bunin}

\enquote{Ах какая Черноморочка! Как такую не любить!} - громко и вногу пели курсанты
проходя мимо наших окон на Сковороды по дороге в баню на Спасской (Героев
Триполья плохо приживалось). Мылись, похоже, раз в неделю, как все, но ходили и
пели по 3 дня подряд каждую неделю. И так всё время из года в год.

\iusr{Владимир Картавенко}

Подвиги экипажа монитор \enquote{Железняков} в войне 1941-45 подтвержденные пленумом
Верховного Суда 21.09.1966, стали основанием для воссоздания 21.01.1967 КВВМПУ
и создании 316 УО ОСНАЗ 20884 ЧФ. При другом решении Суда, Морполит и
Разведшколы могли быть в иных городах союза.

\iusr{Евгений Дробков}
Шикарная память! Спасибо!

\iusr{Виктор Клименко}

Со временем, конечно, приходят воспоминания о юности. Но вот я никак не пойму,
как из холла центрального входа училища можно было увидеть дома на Андреевском
спуске. Там и Гостиный двор, и сквер заслоняли эту перспективу. Да и второй
номер по Андреевскому с Контрактовой площади не виден. Разве что, зимой и
деревья в сквере спилить, да и то, только боковая стена. Сознайтесь, это чисто
эмоциональная выдумка автора.

Стройные ряды курсантов, иногда марширующие перед вечерней поверкой по моей
Борисоглебской улице, хорошо помню. Видимо, по каким-то причинам отцы-командиры
откланялись от традиционного маршрута по улице Ильинской. И, по какой-то
причине, курсантов в баню водили тоже по городской улице, хотя, видимо, можно
было и через внутренний двор.

Помню, когда жилые дома начала ХХ века на ул. Волошской отдали училищу и они
превратились в безликие одинаковые строения, выкрашенные в один цвет. А сквер
напротив Ильинской церкви перед бытовыми корпусами училища образовался после
снесения гвоздильного завода, мимо которого ходил Павел Кольцов в исполнении
Юрия Соломина в фильме \enquote{Адъютант его превосходительства}. И сами съёмки этого
фильма помню...

\begin{itemize} % {
\iusr{Петр Кузьменко}
\textbf{Виктор Клименко} 

это совсем не выдумка. Несколько дней назад, встречаясь по случаю 35-летия
выпуска, я показывал однокурсникам, как хорошо видны мои окна из окна
центрального входа, с бывшего поста номер один у боевого знамени училища.
Замечу, что во времена КВВМПУ, главный вход с якорями по бокам, был в корпусе
номер один а не в циркульном здании как сейчас.

\begin{itemize} % {
\iusr{Виктор Клименко}
\textbf{Петр Кузьменко}

Ну, возможно, если главный вход считался тот, что в доме на углу ул. Ильинской.
Действительно, якоря были там. Раньше палисадник возле этого дома был окружён
забором, а со стороны Ильинской на стене здания, за забором, были закреплены
антены-четвертушки.

\iusr{Петр Кузьменко}
\textbf{Виктор Клименко} 

не только четвертушки. И другие. С той стороны, которую Вы упоминаете, был
кабинет кафедры технических средств кораблевождения. Эти антенны помогали нам
изучать радионавигацию.

\iusr{Виктор Клименко}
\textbf{Петр Кузьменко}

Если не ошибаюсь, во двор дома №2 по Андреевскому спуску нужно было входить с
площади. Там располагался борцовский клуб, который посещал, когда учился в
старших классах 20-й школы. Но могу и ошибиться, давно это было, в 60-х. А в
доме 2-В жили мои одноклассники. Там был узкий двор-колодец. Да, память...

\iusr{Петр Кузьменко}
\textbf{Виктор Клименко} 

наш двор был проходным. Можно было зайти и с площади, через подворотню между
ателье проката и райисполкомом. Затем налево в парадное по первому этажу в наш
двор. Основной вход был через подворотню на Андреевском спуске. Был ещё третий
вход, с Боричевого тока, через парадное одного из домов. Сейчас в нём
посольство.

\iusr{Виктор Клименко}
\textbf{Петр Кузьменко}

Последний раз был на Подоле с дочкой лет пятнадцать назад. Там так сильно всё
изменилось, что и не хочется этого видеть. А в памяти остался тот старый Подол
и годы мечтаний и дерзаний, неудач и побед, проведённые там.

\iusr{German Shefer}
\textbf{Viktor Klymenko}

Борцовский клуб Спартак был, я в нём занимался 2 года пока не сломал ключицу.
Меня туда отвёл отец, к своему же бывшему тренеру.
\end{itemize} % }

\end{itemize} % }

\iusr{ZabaVa Da}
В детский сад, что напротив, морячки приходили на утренники к детям. Спасибо за
теплые воспоминания

\iusr{Svyatoslav Sovyuk}

Ещё один военно-морской истеблишмент в нашем районе, мало кто знает о оном.

В начале восьмидесятых по инициативе Детской Комнаты Милиции Подольского района
(и конечно же не безучастие Веры Ивановны)

и с согласия командования КВВМПУ, на базе подготовительного центра,
дислоцированного на западном побережье Киевского Водохранилища, для
трудновоспитуемых, не управляемых, не контролируемых и.т.п. был основан
военно-спортивный лагерь «Альбатрос».

Целью мероприятия было, сконцентрировать самую талантливую часть молодёжи,
нашего района, на протяжении летнего периода в одном месте.

Тем самым обеспечить безопасность отпускного сезона на местах.

Лагерь функционировал несколько сезонов.

Мне посчастливилось, по принудительной путёвке, провести лето в этом лагере.
Таким образом это делает меня и автора оригинальной статьи (выпускника КВВМПУ)
в какой-то степени однополчанами.

Без официального статуса о званиях я Юнга ВМФ СССР

\begin{itemize} % {
\iusr{Yurii Kadochnikov}
\textbf{Slava Sovyuk} 

у 1988-му я у складі добровольчої бригади 2-й поверх дачі адмірала добудовував
та поруч баньку зібрав. Віктор Іванович, начальник ВСБ, потім на водолазних
спусках дозволяв мені ввечері там попаритись.


\iusr{Ольга Беляева}
\textbf{Slava Sovyuk}, вы знали Дяченко Веру Ивановну?

\iusr{Svyatoslav Sovyuk}
\textbf{Ольга Беляева}
Да, но только на профессиональном уровне.
Наши организации имели определённые точки соприкосновения. ; )

\iusr{Ольга Беляева}
\textbf{Slava Sovyuk}, моя мама с ней работала, а я ее боялась.. Последнее время она работала в детдоме \enquote{Малятко}

\iusr{Svyatoslav Sovyuk}
\textbf{Ольга Беляева}

Детская Комната Милиции Подольского Района, с её неустанными бдителями
правопорядка и формирования нравственности у подрастающего поколения вверенного
района, Веры Ивановны и Елены Юрьевны, находилась напротив кинотеатра Октябрь,
вход со стороны детской площадки, позволяя онным просматривать оперативную зону
и создавая определённую сложность, для тех кого знали в лицо, прогуливать школу
и посещать кинозаведение.

В процессе общения и обсуждения волнующих моментов и точек соприкосновения
наших институтов, я пришел к выводу что Вера Ивановна, невзирая на
обстоятельства, такие как идеология настоящего государственного органа со всеми
механизмами давления и подавления самоэволюциионного процесса, является
специалистом своего дела, что вообщем позволило мне, не оказаться на учёте.

\end{itemize} % }

\iusr{Lola Madino}
Аж в очах зарябіло від червоного та серпів-молотів з марсовими зірками...

\iusr{Анна Сидоренко}
Спасибо.

\iusr{Раиса Карчевская}
Большое спасибо за очень познавательный пост. Прочла на одном дыхании

\iusr{Владимир Бахтурин}

А я служил в оркестре в/ч 3217, на противоположной стороне площади))) и мы
часто ходили в оркестр Морполитучилища обмениваться творческим опытом, а
офицеры и курсанты у нас часто проводили политзанятия, причём очень
профессионально на разные интересные темы (не ради \enquote{галочки})))...

\iusr{Yurii Kadochnikov}
Ще раз прочитав. Юність згадав, однокашників. Дякую, Петро! Гарно викладаєш, ше
пиши! @igg{fbicon.thumb.up.yellow} 

\iusr{Таня Сидорова}
Дуже змістовна та цікава розповідь, дякую

\iusr{Любовь Ямковая}
Благодарю за рассказ очень интересно было узнать историю Подола!!!!

\iusr{Natalya Tarasenko}
Удивительное повествование! Спасибо огромное с наидобрейшими пожеланиями.
@igg{fbicon.heart.red}

\iusr{Олег Гринюк}

Еще немного истории о старом \enquote{СПАССКОМ ПРИЧАЛЕ} еще не было гранитной
набережной, возле него росли АКАЦИИ с иголками до 30 см. такие акации были в
Бот. саду Фомина.

\iusr{Людмила Порунова}
Вітаю! Бажаю успіхів!

\iusr{Игорь Карпачов}

У меня отец был начальником кафедры физ воспитания, Карпачев В. И и брат его
окончил. Мне как ребенку было как в сказке в летнем лагере КВМПУ в Лютеже на
киевском море, купание, рыбалка, яхты, моторки и свобода. Для меня довольно редкое
светлое пятно социализма, но может быть и детский спорт.


\iusr{Сергей Сергеев}
ЧИПОК - Чрезвычайная Индивидуальная Помощь Оголодавшему Курсанту

\iusr{Владимир Картавенко}

О Василии Шукшине это красивая легенда. В 1940 Днепровская флотилия
переименована в Пинскую, резервный штаб которой был на Рыбальском. Шукшин
учился в Пинске и проездом в Севастополь был в Киеве у друзей. 316 уоосназ
20884 и КВВМПУ созданы в один день 21.01.1967. Поздравляю !

\iusr{Tatiana Sitnik}
чим сьогодні живе училище?

\iusr{Сергей Павловский}

С замиранием сердца вчитыввлся в каждую строку в ожидании хоть слова про ВПО
\enquote{Океан} при КВВМПУ и дождался! Теплые воспоминания об нашем объедении
Океан, об полковнике Тенен, о наших занятиях в классах на Ярославской 40 и
аудитории морполита, о наших парадах по Крещатику и на открытии памятника
Родина-Мать, о лагере Альбатрос и походах на ялах, о курсантов командира
взводов и конечно же форма! От которой все девчонки были без ума! И беска с
ленточкой с одной стороны ВПО \enquote{Океан}, а с другой обязательно Киевское высш
воен-морское полит училище, из-за перевёрнутый ленточки часто приходилось
делать ноги от патрулей по любимым дворам Подола.

Спасибо дорогой!

А в 84 поступил в ДВВПУ...

\iusr{Юрий Панчук}

Спасибо, интересный рассказ! Я как то в конце 90-х пересекся по работе с
выпускником училища. Действительно, чувствовался в нем определенный офицерский
вышкол, набор правильных понятий и даже манеры. Причем выпускался он уже в
постсоветское время. Училище было расформировано кажется в 1995 или 96м годах.
Корпуса Морполита заняла Могилянка - диаметрально противоположное по идеологии
заведение). Очень интересно, чему же учили политработников после распада СССР и
отмены всех коммунистических догм и доктрин?! И как интересно поменялась
учебная программа тех курсантов, которые поступили еще при СССР, а выпускались
уже в независимой Украине?! Все эти вопросы я не успел ему задать, пути
разошлись. Может Вы расскажете в двух словах?

\begin{itemize} % {
\iusr{Петр Кузьменко}
\textbf{Юрий Панчук}, 

к сожалению, я не могу вразумительно ответить на этот вопрос. Я закончил КВВМПУ
в 1986 году. В интересующее Вас время я служил на Тихоокеанском флоте и Киев
видел лишь в отпусках. Может одногруппники - квумпари, закончившие нашу
\enquote{систему} уже после распада СССР, восполнят этот пробел.

\end{itemize} % }

\iusr{Oleksiy Rozhkov}
Реально, написано замполітом, приз від главпура автору

\iusr{Владимир Картавенко}

В 1958 училище младших политработнтиков было расформировано.

21.09.1966 пленум Верховного Суда Союза установил обстоятельства начала ВОВ.

Так монитор Железняков стал памятником Республиканского значения и на базе
20884 созданной 16.091941 созданы КВВМПУ и 316 уо 20884, первым командиром
которой стал Офицер полтучилища Василий Васильевич Пискунов, участник Парада
Победы швырнувший к ногам победителей штандарт разведки и контрразведки АБВЕРа.
Документы рассекречены и есть в свободном Интернет.

\iusr{Willy Poindexter}
Послушница - это не инокиня 

\begin{itemize} % {
\iusr{Петр Кузьменко}
\textbf{Willy Poindexter} не стоит так придираться к мелочам. @igg{fbicon.face.smiling.eyes.smiling}  (Имеется ввиду, что это мелочь, для данного повествования)
\end{itemize} % }

\iusr{Роман Халин}
До сих пор считается, что лучших замполитов для флота, готовили только в Киеве

\iusr{Регина Кучеренко}
Атмосферные воспоминания! Спасибо.

\iusr{Natasha Kotlyarenko}

Были у меня там друзья. Со мной учились девочки из Севастополя и жили в
общежитии, а их одноклассники в училище. Гражданку держали у меня, киевлянки. В
субботу приезжали, переодевались и мчались к подругам. А вечером к 22.00
приезжали на такси, переодевались и возвращались в казарму. А поскольку в
квартире какое-то время находились их ботинки, начищенные до блеска, то после
выходных несколько дней стоял суровый запах ваксы или гутталина. Моя мама, дочь
военного, заходя домой всегда говорила : В детство вернулась..... так пахли
сапоги ее папы.... После окончания училища ребята разьехались, мы потерялись .
Знаю что один служил на крейсере Москва. Этот крейсер пережил тяжелый пожар и,
рассказаывали, что наш знакомый обгорел и тяжело болел. Не знаю о его судьбе.
Жизнь всех нас разбросала... Тогда были танцы, в будние дни договаривались и
приносили ребятам покушать - они забирались на забор и мы передавали кулечки.
Были молодые, веселые.... Все ребята были прекрасные друзья, были
мужчинами.....

\iusr{Оксана Гриневич}
«...Секретка там, как злато чахнет,
Там МорПолит,
Там морем пахнет»

\iusr{Yurii Kadochnikov}
Таки 55 років, є що згадати  @igg{fbicon.face.smiling.sunglasses} 

\ifcmt
  ig https://scontent-lhr8-1.xx.fbcdn.net/v/t39.30808-6/272246178_1157098814694601_7812923707035540028_n.jpg?_nc_cat=111&ccb=1-5&_nc_sid=dbeb18&_nc_ohc=HlTW-w6xfLIAX99hKcc&_nc_ht=scontent-lhr8-1.xx&oh=00_AT-VgIyZOhEfm5iBd0JOX83En1qXlVUV81KvkScboP5ufw&oe=61F1EB75
  @width 0.3
\fi

\iusr{Михайлина Голуб}

Мой любимый коллега был выпускником морполита. К сожалению, после училища,
служить ему довелось недолго. Он заболел и был комиссован. И куда податься
молодому человеку с двумя малыми детьми и женой? Так получилось, что он решил
попробовать учительствовать, и ему очень понравилось. Мы с моей коллегой, как
опытные учителя истории, всячески поддерживали Сашу. И он стал замечательным
учителем. Он не преподавал не только историю, но и был военруком. Но самое
главное: он воспитывал из наших мальчишек-разгильдяев настоящих Мужчин. Именно
так, с большой буквы. Школа, увы, давно превратилась в абсолютно женское
заведение. Дома у многих мальчишек только мама и бабушка, а в школе - только
учительницы. И тут Александр Николаевич. Блистательная выправка, остроумие и
потрясающие рассказы о море и моряках. И настоящее, не поддельное благородство
и уважение к женщине, так присущее морским офицерам. Неудевительно, что ребята
смотрели ему в рот и пытались подражать, а девчонки откровенно влюблялись и
мечтали встретить в жизни такого. И безусловно основу этого флотского
благородства заложила его альма-матер Морполит.

\iusr{Vladislav Priemsky}

Отличный рассказ! Позволю себе дополнить. Продукцию для ВМФ выпускали еще и
завод им. Петровского, и НИИ \enquote{Квант} (если не ошибаюсь, \enquote{24-й ящик} в свое
время). А для НИИ гидроприборов, входящего в НПО \enquote{Славутич}, который занимался
созданием гидроакустического оборудования, кадры соответствующего профиля
готовили в Киевском судостроительном техникуме и в КПИ.

\begin{itemize} % {
\iusr{Петр Кузьменко}
\textbf{Vladislav Priemsky}, благодарю! Наш Город был неразрывно связан с Флотом!

\iusr{Vladislav Priemsky}
\textbf{Петр Кузьменко} Несомненно! Многие предприятия работали на Флот, а учебные заведения готовили для него специалистов!
\end{itemize} % }

\iusr{Наташа Грудцына}

Имела удовольствие некоторое время в начале 90-х годов работать в одном отделе
НИИ \enquote{Квант} с капитаном !-го ранга, бывшим преподавателем этого училища Ткачом
(к сожалению, имя-отчество забыла). Этот моряк запомнился как образец
настоящего морского офицера - умён, галантен, подтянут. Родом из Сибири (Чита
или Иркутск). Может, кто-то вспомнит его? Отзовитесь! Автору спасибо за
интересный рассказ! Хочу только дополнить, что НПО \enquote{Квант} также работал на ВМФ
(радиоэлектронное оборудование), на его территории находился филиал Киевского
судостроительного техникума, где готовили специалистов по специальности
\enquote{Радиоаппаратостроение} и по моей - \enquote{Радиолокационные устройства}). А вот о
Киевском военно-морском техникуме связи я ничего не слышала.

\begin{itemize} % {
\iusr{Vladislav Priemsky}
\textbf{Natalia Grudcyna} 

С 1978-го по 1982-й учился в КССТ по специальности \enquote{Радиоаппаратостроение}.
Основных площадей техникума на Старовокзальной не хватало, где только нам не
приходилось учиться в арендованных помещениях!

\end{itemize} % }

\iusr{Nataliia Shevchenko}

Спасибо за прекрасный, как всегда, рассказ и фотографии. Это не просто личные
воспоминания, это частичка нашей Истории... Тепло и радостно, что это было,
грустно и тяжело, что потеряли... и не только училища, заводы т тд... .

\iusr{татьяна неверовская}
Потрясающе!!!  @igg{fbicon.hands.applause.yellow}  Много связано! Спасибо!!!

\iusr{Людмила Викторовна}
Спасибо, прочитала с интересом!

\iusr{Rita Ritulya}

Речное училище хорошо помню. На Подоле. Практически рядом с моей замечательной
17-й школой. И мы курсантов называли \enquote{колюжники}... Как это давно было..

\begin{itemize} % {
\iusr{Ольга Беляева}
\textbf{Rita Ritulya}, рассказ не о речниках, а о политологов из КВВМПУ, теперь это Киево-Могилянская академия.

\iusr{Елена Михайлова}
\textbf{Ольга Беляева} где-то прочла, что это единственное в Союзе высшее мор-полит.

\iusr{Ольга Беляева}
\textbf{Елена Михайлова}, было единственным...

\iusr{Rita Ritulya}

Да... Я знаю. Меня просветили, так сказать.... Когда моя дочь отслужила армию, мы с
мужем решили все же показать откуда её корни. Прилетели в Киев... Для меня то
Киев в памяти остался ещё когда был СССР... Ну спустились по Жданова. Универмага
нет... Потом показываю МОРСКОЕ УЧИЛИЩЕ... Ну объясняю доче что это.. И вдруг
милый человек мне говорит... Ты шо дура? Какое училище? Это Киево-Могилянская
Академия... Ну ладно... дура, значит дура... Так, что я точно знаю что было и, что
есть в этом здании. А рядом был Комитет комсомола... районный... Все было...

\end{itemize} % }

\iusr{Роман Воловенко}

помню как пригласила меня профорг завода на митинг посвященный 50-летию начала
ВОВ 22 июня 1991 года Но я отказался, так как летал офицером на МИ-8 армейской
авиации 17-й воздушной армии. Но профорг сказала что после митинга в парке им.
Примакова будет накрыт стол с выпивкой и объедками с ресторана по случаю
вчерашнего 40-летия 2-го Киевского гормолзавода. Я спрашиваю : - Почему меня не
пригласили на юбилей в ресторан? - Так тебя же забрали в армиюю

\iusr{Роман Воловенко}

Пришлось пойти Это был выходной день И что меня поразило: На митинге стояли
курсанты Морполита с бело-голубыми андреевскими флагами с привязанными
траурными лентами!!! и это за 2 месяца до развала Союза!!! Вот когда зарождался
рюский миръ!!! И он выстрелил под этими андреевскими флагами по нашим кораблям
через четверть века.

\begin{itemize} % {
\iusr{Сергей Солдатенко}
\textbf{Роман Воловенко} 

а как же иначе? Политруков учили в полной идеологической привязанности к
Центру. Никаких \enquote{мистечковых, националистических} идей  @igg{fbicon.face.tears.of.joy} 

\iusr{Роман Воловенко}
Смотрели в корень
\end{itemize} % }

\iusr{Владимир Гонтар}
А морская практика каждый год была или только после 3-го курса?

\begin{itemize} % {
\iusr{Петр Кузьменко}
\textbf{Владимир Гонтар} ежегодно. На разных флотах. После третьего курса был дальний штурманский поход.

\iusr{Владимир Гонтар}
\textbf{Петр Кузьменко} 

класс! Молодцы, я оканчивал КВОКДКУ. Когда служил в США, то у моего начальника
- зам. Хозяина, капитан 1 ранга Ильяшенко, сын был курсантом КВМПУ.

\end{itemize} % }

\iusr{Наталия Бурковская}

Спасибо за красивую историю! Я жила на ул Сковороды до 1978 и вспоминаю
вечерние строевые с песней по улице где-то в 21, после чего я ложилась
спать...

\iusr{Ванька Стахов}

Одне питання, запорєбріковий вираз: \enquote{Дави их бл..ть} це стилістика старої
школи і разлєтєвшихся по всєму міру чи все-таки невігластво нових і
несознатєльних?

\begin{itemize} % {
\iusr{Петр Кузьменко}
\textbf{Ванька Стахов} нас такому точно не вчили.

\iusr{Владимир Гонтар}
\textbf{Ванька Стахов} 

це сучасна російська пропаганда. Якщо у людини є голова на плечах, якщо людина
старанно вчилася навіть в радянських ВУЗах і здатна була аналізувати ситуацію,
порівнювати реалії й те, про що розповідали на кафедрі марксизму - ленінізму,
то така людина мала стати справжнім громадянином своєї країни. Гадаю, що
офіцери РА і ВМФ поділилися чи не навпіл - одні залишилися з мантрою в голові,
що СРСР сім’я братніх народів, а інші зрозуміли, що це страшна й підступна
система. Але для цього треба мати голову на плечах.

\begin{itemize} % {
\iusr{Ванька Стахов}
\textbf{Владимир Гонтар} 

Щодо розділення згоден. Віктор Суворов як приклад. Хотілося би ще розуміти
співвідношення розуміючих і \enquote{справжніх синов отчізни}.


\iusr{Владимир Гонтар}
\textbf{Ванька Стахов} 

тільки конкретні вчинки людей покажуть, хто є хто. Україна не вперше переживає
такий момент. Вибір перед офіцерством постав і в 1918 році. І багато українців
не хотіли йти до білих - їх благородій, а левова частина не могла сприйняти
фарисейство червоних, свою Державу та армію не змогли тоді утримати...

\end{itemize} % }

\end{itemize} % }

\iusr{Наталья Эгатова}
Очень интересно было прочитать ваш рассказ. Я всю жизнь прожила в Киеве, но не
знала ничего об этом училище.


\iusr{Petr Ivanov}

Я принял присягу.. о неразглашении.. и с мая 1973г.. начал работать автослесарем в
Кввмпу.. родители устроили меня туда.. что бы я ознакомился с военной службой .и
преподавателями... годик я поработал... получил специальность аквалангиста... и
решил не поступать.. а такой же парень как и я.. числившийся
маляром... поступил.... я ушел в ПВ... и не жалею.... Потом.. много лет спустя я
пересекался с ним... он прослужил на подводной лодке.. заработал кучу
болячек.. рано вышел на пенсию... и работал в райисполкоме.. в квартирном
отделе.....

\begin{itemize} % {
\iusr{Yurii Kadochnikov}
\textbf{Petr Ivanov} а я з МЧПВ прийшов в КВВМПУ і не шкодую))

\iusr{Petr Ivanov}
\textbf{Yurii Kadochnikov} кожному свое...я теж не шкодую....
\end{itemize} % }

\iusr{Алёна Бидная}
какие были бальные танцы и дискотеки!!!

\iusr{Юрий Нифатов}

Спасибо, Петр! Все очень мне близко. Позволю себе добавить несколько
фотографий. Это командиры ОСНАЗ 316. До 1967 находился в здании \enquote{Могилянки}.
Фотографии колоризованы.

\ifcmt
  ig https://scontent-lhr8-1.xx.fbcdn.net/v/t39.30808-6/272449025_10219941316948673_6681241652137248532_n.jpg?_nc_cat=110&ccb=1-5&_nc_sid=dbeb18&_nc_ohc=RgJ0P5TPfNUAX9lxU8O&_nc_ht=scontent-lhr8-1.xx&oh=00_AT9qy9mJqJVjVOgBeZNZt-ypRZM4K8OOnIbZL0rku_RR_A&oe=61F15B99
  @width 0.3
\fi

\iusr{Юрий Нифатов}
1966, вероятно.

\ifcmt
  ig https://scontent-lhr8-2.xx.fbcdn.net/v/t39.30808-6/272644802_10219941322428810_4953368931164542976_n.jpg?_nc_cat=102&ccb=1-5&_nc_sid=dbeb18&_nc_ohc=r5mfbjM-0WQAX93thRi&_nc_ht=scontent-lhr8-2.xx&oh=00_AT8k3lTpptEjM-FWTe8NTh4L79ILAvPu1F7u_k8PpwlVhw&oe=61F2843F
  @width 0.3
\fi

\iusr{Юрий Нифатов}
ОСНАЗ 316

\ifcmt
  ig https://scontent-lhr8-1.xx.fbcdn.net/v/t39.30808-6/272629297_10219941324468861_5261730643564363614_n.jpg?_nc_cat=107&ccb=1-5&_nc_sid=dbeb18&_nc_ohc=ZG_GL1KFC40AX-tbczc&_nc_ht=scontent-lhr8-1.xx&oh=00_AT9TBVbr0iECgd3dzkHG5XSqsoaCl2gnPHb3edi4YX7jOQ&oe=61F270F7
  @width 0.3
\fi

\iusr{Юрий Нифатов}
ПКЗ-100 \enquote{Иматра} в гавани на Рыбальском.

\ifcmt
  ig https://scontent-lhr8-2.xx.fbcdn.net/v/t39.30808-6/272390201_10219941326548913_4390199096675102430_n.jpg?_nc_cat=102&ccb=1-5&_nc_sid=dbeb18&_nc_ohc=flB7QBT1NV8AX_gFOaD&_nc_ht=scontent-lhr8-2.xx&oh=00_AT-zDrI61--IsDXKRjTGnv7hb3gkhajfoIKU3aqR3DrSSg&oe=61F26CEB
  @width 0.3
\fi

\iusr{Юрий Нифатов}
День Присяги. ОСНАЗ 316

\ifcmt
  ig https://scontent-lhr8-2.xx.fbcdn.net/v/t39.30808-6/272497871_10219941329068976_9142819563391933723_n.jpg?_nc_cat=104&ccb=1-5&_nc_sid=dbeb18&_nc_ohc=G5Ut0oIwgt8AX9Vh6dy&_nc_ht=scontent-lhr8-2.xx&oh=00_AT_MHhyQE6yRudCnun4aLJzQnk8Rjm7Nh8PD074O7bCkXA&oe=61F2BBD4
  @width 0.3
\fi

\iusr{Юрий Нифатов}
Катера ОСНАЗ 316 в гавани на Рыбальском. 1970гг. Один из них до сих пор там стоит!

\ifcmt
  ig https://scontent-lhr8-1.xx.fbcdn.net/v/t39.30808-6/272397203_10219941329828995_6078107989130835420_n.jpg?_nc_cat=103&ccb=1-5&_nc_sid=dbeb18&_nc_ohc=wJxSWLd05M8AX9WFmbT&_nc_ht=scontent-lhr8-1.xx&oh=00_AT_9Oz_ecVh8-bUaKsz3sdwFGC4cM37zZwGbX6CcqnbPfw&oe=61F24741
  @width 0.2
\fi

\begin{itemize} % {
\iusr{Yurii Kadochnikov}
\textbf{Юрий Нифатов} мабуть на них ми виходили в Київське море

\iusr{Юрий Нифатов}
\textbf{Yurii Kadochnikov} Можливо. Хоча три з цих катерів на початку 1970 списали. Наскільки знаю, ходили на нових - Ярославцях.
\end{itemize} % }


\end{itemize} % }
