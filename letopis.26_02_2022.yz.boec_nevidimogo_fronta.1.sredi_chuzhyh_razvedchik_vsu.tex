% vim: keymap=russian-jcukenwin
%%beginhead 
 
%%file 26_02_2022.yz.boec_nevidimogo_fronta.1.sredi_chuzhyh_razvedchik_vsu
%%parent 26_02_2022
 
%%url https://zen.yandex.ru/media/rodopi84/sredi-chujih-byl-kak-svoi-rasskaz-razvedchika-lejavshego-v-okopah-vsu-6219f95070df981201a67183
 
%%author_id yz.boec_nevidimogo_fronta
%%date 
 
%%tags donbass,razvedchik,razvedka,ukraina,vojna,vsu
%%title "Среди чужих, был как свой" - рассказ разведчика лежавшего в окопах ВСУ
 
%%endhead 
 
\subsection{\enquote{Среди чужих, был как свой} - рассказ разведчика лежавшего в окопах ВСУ}
\label{sec:26_02_2022.yz.boec_nevidimogo_fronta.1.sredi_chuzhyh_razvedchik_vsu}
 
\Purl{https://zen.yandex.ru/media/rodopi84/sredi-chujih-byl-kak-svoi-rasskaz-razvedchika-lejavshego-v-okopah-vsu-6219f95070df981201a67183}
\ifcmt
 author_begin
   author_id yz.boec_nevidimogo_fronta
 author_end
\fi

\enquote{Ещё в апреле месяце прошлого 2021-го года, на Донбассе побывал первое лицо
Украины Вл. зеленский. Он успешно продефилировал по окопам передовой, осмотрел
все достопримечательности на линии соприкосновения с ополченцами Донбасса и
после устроенной им самоличной проверки, уверенно заявил, что ему нравится,что
бойцы ВСУ готовы уже к тому, чтобы успешно штурмовать и Луганск и Донецк.}

\begin{zznagolos}
Кто - нибудь об этом помнит сегодня??? Сегодня - полдень 26 февраля 2022 -го
года. А мы вот решили поворошить недалёкое историческое прошлое и напомнить.	
\end{zznagolos}

То, что сейчас происходит на Украине, оказывается происходит не просто так.
Происходит вопреки воле тех бойцов ВСУ, которые уже тщательным образом
готовились идти в Донецк, воевать с мирным населением и ополченцами, а теперь
вдруг вынуждены включить \enquote{заднюю скорость} и подумать о других
материальных и моральных вещах...

\ifcmt
  tab_begin cols=2,no_fig,center

     pic https://avatars.mds.yandex.net/get-zen_doc/175411/pub_6219f95070df981201a67183_6219f98e16b4077ea9aa10bf/scale_1200
		 @caption В окопах ВСУ летом прошлого года... фото: картинки яндекса.

		 pic https://avatars.mds.yandex.net/get-zen_doc/1907561/pub_6219f95070df981201a67183_621a19aaf195ce4d5676d6e0/scale_1200
		 @caption Весной 2021-го года на Донбассе. фото:картинки яндекса.

  tab_end
\fi

Попалось тут по воле случая одно занятное интервью с профессиональным военным,
который в силу своих возможностей побывал в шкуре разведчика и был в центре
подготовки ВСУ примеряя на себя форму военно-обязанного войск Украины.

Занятные вещи порассказывал человек. Кому интересно, послушайте. (интервью
датировано поздней осенью 2021-го года) Информация немного напоминает
фантастический боевик.

\begin{zznagolos}
"Как опытный боец, служивший по контракту в различных горячих точках получил
предложение однажды поучаствовать в деле и поехать на Украину. С предложением
согласился и поехал. 

Получил необходимые документы и легализовался на территории Украины, вместе
с группой товарищей. Затем, в составе этой группы, все поступили как
наемники в вооруженные силы Украины.	
\end{zznagolos}

Нас привезли в один городок на юго-востоке Украины и поселили в
специализированном тренировочном лагере Вооруженных сил республики, в палатках.

Прапорщик по вещевому довольствию спросил: \enquote{Надолго ли прибыли?}
Ответили: \enquote{Как скажут, столько и будем заниматься.} Он ещё спросил:
\enquote{Вам выдавать всё?} И мы ответили просто: \enquote{Да, всё, конечно!}

На воинское довольствие поставили, и выдали, как офицерам пистолеты. Поначалу
чувствовали себя неуютно в непривычной обстановке. Потом притёрлись и поняли
,что всем \enquote{всё по фиг}.

Любопытных особо - отшивали быстро. Притащили с учебных стрельбищ немного
гранат и показывали очень любопытным. Любопытство сразу проходило. 

\begin{zznagolos}
Там все свои и никаких подозрений. Научились \enquote{гэкать} на местный манер, а когда
донимали вопросами некоторые \enquote{челы}, напускали тумана и говорили, что с
востока.
\end{zznagolos}

Когда назойливые вопросы надоели уже, то сказали, что мы тут по спецзаданию и
тогда разные расспросы прекратились вообще. А в лагере было тысячи таких же
мужиков и всем было не до нас. Потом часто возили нас группами по области,
местность обозревали, для ориентиров, как бы.

Общее впечатление от этой \enquote{подготовки} ( а я повидал многие лагеря), что это
толком не армия. Если кто-то говорит, что в нашей армии бардак, то на Украине
вообще слово \enquote{армия}, к таким войскам - не применимо.

\begin{zznagolos}
Пьянки постоянные, курят всякую дрянь, и когда такой \enquote{обшмаляный} с
автоматом заходит в твою палатку и спрашивает: \enquote{Выпить есть?} - Порой
не знаешь, что ему ответить, ведь и не понятна его реакция будет, когда он
стоит затвором щелкает, вернувшись с поста.
\end{zznagolos}

\ifcmt
  ig https://avatars.mds.yandex.net/get-zen_doc/5231858/pub_6219f95070df981201a67183_621a1ccbf9083a5c68b59dd4/scale_1200
	@caption Солдат на дороге. фото: картинки яндекса.
  @wrap center
  @width 0.8
\fi

Та сторона юга где Херсон, Мариуполь, Николаев ,Одесса, почитай всё российское,
и идея Малороссии де-факто реализована в памяти людской. Но десятилетие
пропаганды украинской сказывается. У многих просто деревенских украинцев
подавшихся по безработице в армию, интеллект отсутствует из-за недостатка
мировоззрения и простых тупых привычек - поесть, выпить, поспать. Такими людьми
проще управлять. На то пропаганда и рассчитана. Им порой долго разъяснять надо и
логическую цепочку вещей пересказывать не раз, чтобы им стало немного понятно.
Они и говорят-то простыми штампами, что увидели из телевизора. 

\begin{zznagolos}
...Через пару месяцев вернулись оттуда, как из пионерского лагеря. Посмотрели,
узнали, что к чему. Настроение среди офицеров ВСУ тоже шаткое. Примерно одна
треть способна при первой возможности перейти на другую сторону. Верной присяге
считают себя примерно пятая часть командования. Остальным - всё по фиг. Лишь бы
денег платили вовремя.
\end{zznagolos}

Половина офицерского состава, что пришлось видеть, "ни рыба-ни мясо", при
реальной угрозе будут просто сдаваться, не желая воевать. Такое пришлось
повидать в окопах ВСУ."

* * *

Вот что рассказал разведчик, который \enquote{среди чужих был как свой}. И это было
задолго до начавшихся событий на Украине, в летнюю пору.  
