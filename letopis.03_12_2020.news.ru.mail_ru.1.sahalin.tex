% vim: keymap=russian-jcukenwin
%%beginhead 
 
%%file 03_12_2020.news.ru.mail_ru.1.sahalin
%%parent 03_12_2020
 
%%url https://hi-tech.mail.ru/news/sahalin_photo/
 
%%author 
%%author_id 
%%author_url 
 
%%tags 
%%title Утраченное наследие. Как выглядит «заброшенный Сахалин»
 
%%endhead 
 
\subsection{Утраченное наследие. Как выглядит «заброшенный Сахалин»}
\label{sec:03_12_2020.news.ru.mail_ru.1.sahalin}
\Purl{https://hi-tech.mail.ru/news/sahalin_photo/}

\index[cities.rus]{Смирных!поселок, Россия, Сахалин}

Сахалинская область — единственный островной регион России. Место, где очень
много нефти, рыбы и красивой природы. И да, тут тоже есть «заброшки».

\subsubsection{Больше не полетим. Легендарный военный аэродром в пос. Смирных}

Этот объект имеет богатую и интересную историю, поэтому заслуживают отдельного
внимания. 

\textbf{Смирных} — военный аэродром на окраине одноименного поселка. Точная
дата его основания неизвестна. Построен он был во времена японского господства
на острове для Императорской армии и назывался — Кетон. С апреля 1946 года на
аэродроме базировались советские войска. Смотрите фото ниже. Таким был этот
объект при разных хозяевах и так он выглядит теперь. Утраченное наследие. 

\ifcmt
tab_begin cols=2
  caption Больше не полетим. Легендарный военный аэродром в пос. Смирных

  pic https://htstatic.imgsmail.ru/pic_original/53d3ff2cac6c2d9f8a443c80aada7a94/1919542/
  pic https://htstatic.imgsmail.ru/pic_original/b5e3bbf9ae837013e32a7bd8565dccaf/1919544/
  pic https://htstatic.imgsmail.ru/pic_original/a1a1b82f005037f6806dff79b295ed7d/1919529/
  pic https://htstatic.imgsmail.ru/pic_original/034b303f891617238b3e0825410fa186/1919537/
tab_end
\fi

В 1994 году авиационный полк, базирующийся на аэродроме был расформирован. МиГи
перегонялись на аэродром Хурба в Комсомольске-на-Амуре, но несколько машин так
и остались в Смирных. Что происходит на аэродроме сегодня:

\ifcmt
tab_begin cols=2
  caption Печальное зрелище современности

  pic https://htstatic.imgsmail.ru/pic_original/a74eb02763597abdc4cbf9ae79122318/1919534/
  pic https://htstatic.imgsmail.ru/pic_original/fd4d8299493e3a9133de6f642cd00eb8/1919530/
  pic https://htstatic.imgsmail.ru/pic_original/0d408f12c2c577590af36aa208257b4b/1919555/
  pic https://htstatic.imgsmail.ru/pic_original/dc32fd37b34c13e355b2b5970b0eb62e/1919531/
tab_end
\fi

\subsubsection{Культовый маяк и промышленное запустение }

\ifcmt
pic https://htstatic.imgsmail.ru/pic_original/53dc988d90bccb54329e4a0026834f51/1919480/
cpx Самый известный объект региона — маяк Анива на скале Сивучья, построенный 1939 году. «Сердце» маяка, маячок, двигался за счет часового механизма (на одном из концов механизма была подвешена гиря весом в 270 кг), который заводили каждые три часа. Устройство освещения располагалось в чаше с 300 килограммами ртути, которая была использована в качестве подшипника. Дальность света маяка была почти 30 км. В 1990-х годах маяк, который работал на дизельном топливе, решено было сделать автономным и переоборудовать в атомный. Но в 2006 году изотопные установки были сняты. С тех пор маяк заброшен. Фото: Wikimedia / Yaroslav Shuraev / CC BY-SA 4.0
\fi

Мы собрали самые разные промышленные и технологические объекты, которые
находятся на территории этого региона. Их объединяет одно — они больше не
функционируют и уже никогда не будут.

\ifcmt
tab_begin cols=2
pic https://htstatic.imgsmail.ru/pic_original/afbea7f98523a1c90e103fc662551d41/1919471/
pic https://htstatic.imgsmail.ru/pic_original/afbea7f98523a1c90e103fc662551d41/1919471/
tab_end
\fi

\subsubsection{Больше «заброшенной России»}

Александр Сухарев почти 13 лет путешествует по заброшкам и фотографирует их,
делая иногда мрачные, иногда пугающие, а иногда просто грустные фотографии. Мы
собрали часть работ фотографа, связанную с крупными предприятиями, заброшенными
стоянками техники и другим технопрошлым нашей страны.
