% vim: keymap=russian-jcukenwin
%%beginhead 
 
%%file 30_03_2022.stz.news.ua.pravda.2.ohtyrka.1.intro
%%parent 30_03_2022.stz.news.ua.pravda.2.ohtyrka
 
%%url 
 
%%author_id 
%%date 
 
%%tags 
%%title 
 
%%endhead 

\ifcmt
  ig https://i2.paste.pics/28750a9e512120c8337b65be3ea749af.png
  @wrap center
  @width 0.8
\fi

До російського вторгнення 48-тисячна Охтирка була типовим зеленим містом на
Сумщині. Усе, що можна було про неї почути за межами області – що там в
Охтирському районі видобувають нафту. 

Втім, з 24-го лютого назва міста не сходить із заголовків великих ЗМІ – і, на
жаль, ці заголовки є дуже болючими. 

Обстріл військової частини, де загинули щонайменше
\href{https://www.pravda.com.ua/news/2022/03/1/7327015/}{70 бійців},
\href{https://www.pravda.com.ua/news/2022/03/1/7326990/}{три вакуумні бомби},
\href{https://www.pravda.com.ua/news/2022/03/4/7328015/}{знищення ТЕЦ}, обстріл
\href{https://www.pravda.com.ua/news/2022/03/8/7329481/}{ракетами та авіацією},
\href{https://www.pravda.com.ua/news/2022/03/26/7334618/}{бомбардування} та
страшенні руйнування.

Стерті з лиця землі вулиці та спалені заживо люди.

Охтирка прийняла на себе величезний удар російської армії та дивом, точніше
титанічними зусиллями українських захисників, вистояла. Окупанти не взяли місто
й не пройшли крізь нього далі. Але бомбардування \href{https://www.pravda.com.ua/news/2022/03/26/7334618/}{тривають}.

\enquote{33-ій день я не знаю відповіді на запитання своєї дитини: \enquote{Па,
а за що вони нас убивають?}}, – каже в одному з останніх
\href{https://www.facebook.com/permalink.php?story_fbid=1133330780777119&id=100023006243706}{відеозвернень}
до містян голова Охтирки Павло Кузьменко.

Відколи почалася війна, він не лише мер міста за 30 кілометрів від Росії, а й
лікар з 20-річним стажем. На другий день наступу Кузьменко пішов оперувати
перших поранених \enquote{Градами}, двох з яких він знав особисто.

Як і ледь не кожну людину, яка приходить у місті на похорони. А вони
відбуваються щоденно. 

Можливо, саме трагічність ситуації дозволяє Кузьменку в розмові з
\enquote{Українською правдою} бути прямим і навіть критичним. 

Він розповідає про проблеми з розподілом гуманітарної допомоги, хаос, який
виникає через надвисоку активність волонтерів, і претензії від місцевого
бізнесу, який вже чекає на компенсацію збитків.

Мер дуже вірить у перемогу і чекає на можливість відбудувати своє місто.
\href{https://www.pravda.com.ua/news/2022/03/25/7334331/index.amp}{Місто-герой Охтирку}.

Нижче – пряма мова Кузьменка.
