% vim: keymap=russian-jcukenwin
%%beginhead 
 
%%file 14_02_2022.fb.fb_group.story_kiev_ua.2.kievskij_hlebushek.cmt
%%parent 14_02_2022.fb.fb_group.story_kiev_ua.2.kievskij_hlebushek
 
%%url 
 
%%author_id 
%%date 
 
%%tags 
%%title 
 
%%endhead 
\zzSecCmt

\begin{itemize} % {
\iusr{Владимир Новицкий}

Супер фото, все мы так делали и правда не возможно было удержаться от этого
запаха вкуснейшего хлеба!!

\iusr{Оксана Дубинина}
\textbf{Владимир Новицкий}  

@igg{fbicon.face.happy.two.hands}  @igg{fbicon.face.smiling.hearts}  А
мальчишки еще могли и к хлебзаводу за горяченьким хлебом
сбегать) @igg{fbicon.wink}
@igg{fbicon.face.smiling.sunglasses} 

\iusr{Владимир Новицкий}
\textbf{Оксана Дубинина} да и я там был!!

\iusr{Светлана Манилова}
Оксаночка, спасибо за такие вкусные воспоминания! Руки так и потянулись за
кусочком хлеба...@igg{fbicon.heart.red}

\begin{itemize} % {
\iusr{Оксана Дубинина}
\textbf{Светлана Манилова} 

@igg{fbicon.heart.red} @igg{fbicon.face.savoring.food}
@igg{fbicon.face.grinning.smiling.eyes}  приятного аппетита!

\iusr{Євгеній Гончарук}
\textbf{Светлана Манилова} Так сегодняшний так не хрустит.

\iusr{Віктор Зайчук}
\textbf{Євгеній Гончарук} Раніше і дерева були високими, а ми - молодими...

\iusr{Олександра Мазаєва}
У нас, в сім'ї \enquote{офіційно} дозволялось обгризаний по колу Український

\iusr{Tamara Bialik}
А забули арнаутку!

\end{itemize} % }

\iusr{Ирина Иванченко}

Это преступление против человечества... человечности... сорри, человеколюбия -
аппетитный пост с такой вкуснятиной и нА ночь, Оксана !!! Придётся ставить
чайник... А ведь так спокойно все шло ко сну...

\begin{itemize} % {
\iusr{Светлана Манилова}
Ирина, я не решилась так высказаться, но поддерживаю Вас! @igg{fbicon.face.tears.of.joy} 

\iusr{Раиса Карчевская}
\textbf{Светлана Манилова} Я тоже

\iusr{Кузя Музя}
Это точно! Уже, \enquote{не до сна} !
\end{itemize} % }

\iusr{Ирина Хмара}

Корочку))) Мы, пока не открыли на Половецкой хлебный, ходили с подружками на
Нагорную. Пока дойдем каждая съест больше половины. Украинский гарячий такой
вкуснючий был. Вначале хрустящие края шли, а потом и мякушка хорошо шла. А
иногда за разговорами и случайно почти весь сьедался. Приходилось идтить за
новым хлебом.

\begin{itemize} % {
\iusr{Оксана Дубинина}
\textbf{Ирина Хмара}
Да, Ирочка, мне тоже особо «украинский» черный нравился.
На Нагорной классный был магазин и еще на Мурашко в «хлебный» горячий хлеб привозили, а еще на Лукьяновке))
Еще вспоминается магазин «Паляниця», где сейчас Макдональдс расположился, там всегда свеженький хлеб был

\iusr{Ирина Хмара}
\textbf{Оксана Дубинина} на Майдане как-то даже вкуснее вроде был. Там и выбор был больше. Паляница с корочкой тоже вкусная была

\iusr{Marina Shepeleva}
\textbf{Оксана Дубинина} а булочка по 6 и по 3 копейки - это было просто наслаждение. Сейчас вообще такого хлеба нет - жаль

\iusr{Анна Анна}
\textbf{Оксана Дубинина} , мы тоже на Мурашко ходили. С Лермонтовской

\iusr{Sasha Ty}
\textbf{Ирина Хмара}
Які ходячі на хлібах, а?  @igg{fbicon.smile} 
І паляниця в них пішла, і жопки..... 

Довелось спостерігати два хлібзаводи з середини: на Шулявці і Дарниці.  Дівчата
там хоть куди були, тому і хліб смачний але б мене придавили б як комара якщо
щось.

\iusr{Оксана Дубинина}
\textbf{Sasha Ty} 

так на таких молодичок любо-дорого дивитися)))
@igg{fbicon.face.grinning.smiling.eyes}  @igg{fbicon.wink} 

\iusr{Eva Evidze}
\textbf{Ирина Хмара} 

А ми жили на Нагорній! В будинку графа Львова в одній квартирі з його внучкой.
Хліб продавали в Гастрономі на площі - три кроки від нашого дому. Але навіть
така коротка відстань дозволяла з'їсти майже весь хліб по дорозі додому. І
внучка графа їла хліб зовсім по-пролетарські

\iusr{Оксана Дубинина}
\textbf{Eva Evidze} навіть королівські особи не втрималися б)) @igg{fbicon.wink}  @igg{fbicon.face.grinning.smiling.eyes} 
\end{itemize} % }

\iusr{Катя Бекренева}
Самая вкусная горбушка была у \enquote{Украинского} за 20 копеек), а ежели ещё и со сливочным маслом и солью...

\begin{itemize} % {
\iusr{Ирина Хмара}
\textbf{Катя Бекренева} по 20 копеек не помню украинского. По 14 и 16коп., а батон 22 коп.

\iusr{Катя Бекренева}
\textbf{Ирина Хмара} 

по 14 он был какой-то серый, по 16 самый распространённый, а по 20 он был чуть
больше, и тесто немного отличалось. У нас в Святошино его можно было купить
только в одном магазине \enquote{У бабы Раи} на Кременецкой (название народное) ))

Видно разные хлебзаводы поставляли.

\iusr{Ирина Хмара}
\textbf{Катя Бекренева} я думала, что цена от веса зависила.

\iusr{Алёна Выгорницкая}
\textbf{Катя Бекренева} 

а в Боярке был обалденный кирпичик по 20 коп, пока домой доходили, был уже без
корки хрустящей

На всю жизнь в память врезалось, и всегда тёплый, душистый, свежий,
хрустящий... эх, словами не передать... сникерс детства

\iusr{Moti Manzon}
\textbf{Катя Бекренева} 

Самая самая вкусная горбушка была у \enquote{Украинского} за 28 копеек на Крещатике,
возле остановки 16 и 18 тролейбусов  @igg{fbicon.face.smiling.hearts} ... а где сейчас выпекают в Киеве самый
вкусный хлеб ? Кто подскажет

\iusr{Лариса Борсук}
\textbf{Moti Manzon} на оболоне остался, но выкупили частники уже не то современные технологии....

\iusr{Natalie Di Giovanni}

Я только по 16 помню. Мы на левом берегу недалеко от хлебзавода жили.
Просыпались под этот запах сумасшедшей с утра... ммм


\iusr{Людмила Мозговая}
\textbf{Катя Бекренева} да помидорку с огорода.

\end{itemize} % }

\iusr{Ludmila Kilimnik}
Справді смачно було!! Чудові спогади...
@igg{fbicon.face.savoring.food}{repeat=3}  Дякую!!!

\iusr{Петр Кузьменко}

Хорошо ещё, если в хлебный магазин, а не в ларёк свежайшего хлеба при
подольском хлебзаводе. Тогда домой приносилась, в лучшем случае, обзрызанная,
но всё равно ароматная, середина хлеба  @igg{fbicon.wink}  или обожаемой мной
лично арнаутки...@igg{fbicon.heart.red}

\begin{itemize} % {
\iusr{Светлана Манилова}
Петр, и ты туда же...Люди, пощадите! @igg{fbicon.face.tears.of.joy} 

\iusr{Оксана Дубинина}
\textbf{Петр Кузьменко} дааа, мальчишки часто к хлебзаводам бегали.)) @igg{fbicon.heart.red}

\iusr{Анжела Жуковская}
\textbf{Петр Кузьменко} самый вкусный украинский пекли там, особенные печи там были - старые
\end{itemize} % }

\iusr{Елена Турчина}
Да нет, лучше уж корочку чесночком натереть @igg{fbicon.sun.with.face}{repeat=3} 

\begin{itemize} % {
\iusr{Оксана Дубинина}
\textbf{Елена Турчина} ага)) и чуть сольки посыпать))

\iusr{Ирина Хмара}
\textbf{Оксана Дубинина} тогда ещё и в ароматное масло подсолнечное макать))

\iusr{Оксана Дубинина}
\textbf{Ирина Хмара} аж слюнки текут)) @igg{fbicon.face.savoring.food}  @igg{fbicon.face.grinning.smiling.eyes}{repeat=2} 

\iusr{Ирина Хмара}
\textbf{Оксана Дубинина} ты первая всех раздразнила на ночь глядя  @igg{fbicon.cat.face} 
\end{itemize} % }

\iusr{Микола Веселий}

И белый и черный хлебушек любил я надгрызть, чтоб корочку. И булочки по 3 коп
уважал, хоть они были и без корочки)

\iusr{Алёна Выгорницкая}

Мы наш хлебушек раньше все хранили в деревянных хлебницах

А теперь в пакетах... оттого он и плесневет.

Вот работала в хлебном, и у нас был возврат... естественно оставляли в
пакете (возврат раз в неделю) так я из-за плесени не могла опознать... бородинский
или житний. Потом думаю проэксперементирую и сниму пакет, так вот он просто
засох.

Делайте выводы, пора опять прежние хлебницы, либо тканевые торбочки, но не
полиэтилен

\begin{itemize} % {
\iusr{Sasha Ty}
\textbf{Алёна Выгорницкая}
Справа в компонентах а не кулечках, хоча сирість сприяє.
В мене \enquote{хліб} по півроку чекає в пакеті, страшно відкрити, плесені ніякої.
Плесінь це показник кволоті

\iusr{Алёна Выгорницкая}
\textbf{Sasha Ty} я вам розповіла, як у мене на роботі було.
В пакеті цвілий, без пакету ні

\iusr{Klara Mezhebovsky}
\textbf{Sasha Ty} Плесень говорит о том, что в этом хлебе нет консервантов

\iusr{Татьяна Годынская}
\textbf{Алёна Выгорницкая} учтите ещё и \enquote{новые технологии}...
\end{itemize} % }

\iusr{Петр Кузьменко}

Оксана! Как удивительно Ваш прекрасный пост перекликается с моей давней
публикацией в нашей группе
\url{https://www.facebook.com/groups/story.kiev.ua/permalink/1630285413834920/}

\begin{itemize} % {
\iusr{Виталий Пидченко}
\textbf{Петр Кузьменко}
В смысле \enquote{перекликается}...

\iusr{Оксана Дубинина}
\textbf{Петр Кузьменко} потому что все мы родом из детства!))@igg{fbicon.heart.red} и так много общих киевских воспоминаний))
\end{itemize} % }

\iusr{Людмила Иванова}
Домой доносили обгрызаный со всех сторон батон!

\iusr{Євгеній Гончарук}
Меня просила мама грызть только в одном месте.

\begin{itemize} % {
\iusr{Оксана Дубинина}
\textbf{Євгеній Гончарук} а слушались и грызли только в одном месте?))

\iusr{Євгеній Гончарук}
\textbf{Оксана Дубинина} Да. У мамы был дар спокойно и аргументировано объяснять, почему стоит так делать или не стоит.
\end{itemize} % }

\iusr{Tekhkhi Shekk}
Я так кушала в детстве. И на фото вообще очень похоже на меня ))))

\iusr{Валентина Козачук}

Коли купляла паляницю в дитинстві, то її шляпка згризалася миттєво.. А хліб до
нас в магазин чомусь завжди привозили в обідню перерву. Всі школярі вже були
дома і під магазином шикувалася черга з дітей і пенсіонерів. А в руки давали
тільки одну буханку, а рогалики з повидлом привозили тільки один лоток.. І яке
це було щастя, коли тобі діставався цей жаданий рогалик, зверху припорошений
цукровою пудрою

\end{itemize} % }
