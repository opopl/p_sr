% vim: keymap=russian-jcukenwin
%%beginhead 
 
%%file 27_12_2020.fb.bilchenko_evgenia.6.kaming_aut
%%parent 27_12_2020
 
%%url https://www.facebook.com/yevzhik/posts/3506798886021863
 
%%author Бильченко, Евгения
%%author_id bilchenko_evgenia
%%author_url 
 
%%tags bilchenko_evgenia,poezia
%%title БЖ. Каминг аут
 
%%endhead 
 
\subsection{БЖ. Каминг аут}
\label{sec:27_12_2020.fb.bilchenko_evgenia.6.kaming_aut}
\Purl{https://www.facebook.com/yevzhik/posts/3506798886021863}
\ifcmt
 author_begin
   author_id bilchenko_evgenia
 author_end
\fi

БЖ. Каминг аут.

Я не слушала Далиду, не кушала \enquote{Bubble Gum}, не смотрела \enquote{Звёздные войны}.
А также я не смотрела \enquote{Матрицу} и \enquote{Терминатора}.
И \enquote{Гарри Поттера} я не читала (звучит почти непристойно,
Но это - чистая правда, я - действительно дурноватая).
Потом я всё это просмотрела - бегло, по диагонали, -
И обрадовалась тому, что времени зря не тратила.
Увлеченья моих родителей совсем меня не догнали,
Хотя в искусстве погони они - замечательные старатели.
Ужас, да: в подобном признаться, как сор из души, разбросанный
По душам чужим, с хамоватой страстью бросить навстречу миру?
Но я читала \enquote{Дубровского}, я читала \enquote{Дубровского},
Я запускала Машу, как Бога, к себе в квартиру.
И ещё - давайте уж до конца - я не слушала западную рок-музыку
И, тем более, то, что стало лайт- её продолжением.
Я стою перед вами, как на суде, и окурками мечу в мусорку,
А Дубровский (он оказался жив) мне шепчет: \enquote{Спокойно, Женя}.
26 декабря 2020 г.
Илл.: Александр Базарин.

\ifcmt
  pic https://scontent-lga3-2.xx.fbcdn.net/v/t1.6435-9/132938679_3506798856021866_3643908626609171452_n.jpg?_nc_cat=105&ccb=1-3&_nc_sid=8bfeb9&_nc_ohc=L7LWl1y2QSMAX8ERujh&_nc_ht=scontent-lga3-2.xx&oh=897f7453a41b561d67527b2da70996a9&oe=60CB65DA
\fi

\emph{Dmitry Vologodsky}
\enquote{... вермут - не кровь, моё next-поколение
Голгофа - не Эверест...}
© Борис Усов/Соломенные Еноты

\emph{Ігор Мітров}
як для фахового культуролога – так собі зізнання

\emph{Евгения Бильченко}
Ігор Мітров \enquote{Ваше осуждение меня только придает мне силы} (М. Врубель).

\emph{Anna Denisova}

А вот я, в свою очередь, никогда не ходила по дискотекам. И вполне смирилась с
тем, что пришла из другого века...🙂

\emph{Евгения Бильченко}

Anna Denisova Ничего: вон соросята выше тебя научат \enquote{Родину любить}. В виде колонии.

\emph{Leonty Kostur}

я тоже этого не читал все некогда

\emph{Евгения Бильченко}

Ребята, если кто-то реально подумал, что я что-то не посмотрела и не прочитала,
он ошибается. Я знаю этот империалистический мир по первоисточникам лучше его
сторонников. Я не просто знаю его в лицо. Я его сделала объектом своей
лаборатории. Глупо было бы противостоять чему-либо, не изучив оружие противника
и не овладев им лучше противника. Именно поэтому они меня так ненавидят: я -
сильнее, чем их стереотип консерvatора. Как там у генерала Аркиллы из Алабамы
или у Ричарда Шафрански, ученика Тоффлера: \enquote{Чтобы бороться с сетью,
надо стать сетью}, Да мы кем угодно станем. В лаптях против водородной бомбы не
пойдем. И, да, нас лучше бояться, чем смеяться над нами. Логос не уничтожим в
условиях смерти тела. Не я - так следующие. Мамона падет. И падет она, потому
что у воинов моих будут лазерные мечи, а не топоры. Поэзия поэзией, метафора
метафорой, война войной, любовь любовью. That is all, колониальные.

\emph{Oleh Shynkarenko}

Дубровский с лазерными мечами двинется на бой с Микки Маусами и Елвисом Пресли.

\emph{Оксана Ляшенко}
я ещё и Капитанскую дочку читала)))

\emph{Евгения Бильченко}
Оксана Ляшенко Это вообще святое. Я даже играла ее в детстве: переодевалась и
изображала)

\emph{Боляк Владимир}

Конечно, все это мы видели, слышали и осмысливали, а потому, не отравившись,
остались живы. Но почему же этот процесс случился? Просто, Кто-то раньше нас
пришёл и сказал: \enquote{Ша, Маша (Женя, Вова etc. ), я - Дубровский!}

\emph{Ирина Любимцева}

Крутое стихотворение. Особенно конец понравился.

\emph{Людмила Соколова}

А Алданова вы читали? «Самоубийство», например? Или Аверченко, Тэффи? Или хомо
советикус не переваривают ничего, кроме «партийной линии»?
