%%beginhead 
 
%%file 19_04_2022.fb.nebesna_anna.mariupol.1.poraneni_dushi
%%parent 19_04_2022
 
%%url https://www.facebook.com/anna.nebesna/posts/pfbid02HhBe5Ai4yiuaQitAHVr5ieVof8VhHNDph2iZNBi1jw9svBJmDVZwEXbaZ6acqwf8l
 
%%author_id nebesna_anna.mariupol
%%date 19_04_2022
 
%%tags mariupol,mariupol.war,bezhenec
%%title "Поранені душі"
 
%%endhead 

\subsection{\enquote{Поранені душі}}
\label{sec:19_04_2022.fb.nebesna_anna.mariupol.1.poraneni_dushi}

\Purl{https://www.facebook.com/anna.nebesna/posts/pfbid02HhBe5Ai4yiuaQitAHVr5ieVof8VhHNDph2iZNBi1jw9svBJmDVZwEXbaZ6acqwf8l}
\ifcmt
 author_begin
   author_id nebesna_anna.mariupol
 author_end
\fi

"Поранені душі"

🇺🇦Україна🇺🇦- це не тільки певні географічні кордони, Це ми- люди. Ще до
початку війни кожен з нас мав певні проблеми з якими вже давно треба звертатися
до психолога. А зараз курс уваги значно змістився і у переважної більшості він
один- ВИЖИТИ. Бажання жити- це нормальне прагнення живої людини. Але емоції,які
людина переживає у період загрози своєму життю перевертають всю систему
координат її поведінки, виховання, звичок. Ми інакше поводимось, ми інакше
розмовляємо, навіть думаємо не так, як завжди. У зв'язку з вимушеними
переховуванням від обстрілів та переселенням може скластися таке враження, що
ми проживаємо якесь чуже життя. Це все- стрес і наша не зовсім адекватна
реакція на нього. Протест психіки на насильне втягнення у "свідки" тотального
знищення всього, що для нас дорогоцінно перевертає душу знову і знову. Це рана,
яка не даватиме спокою нікому. У комплекті з попередніми проблемами які ми так
і не встигли вирішити, насилля, загибель близької людини, втрата майна, житла,
сенсу життя -цей стрес-коктейль може призвести до крайніх форм негативних
реакцій. Я не хочу перераховувати яких, аби нікого ні до чого не "надихнути".
Кожен з нас сам знає на межі чого він зараз балансує. Моє прохання до
найближчого оточення: 

❤️якщо у вас є друзі які "вижили" в певних подіях і ви на зв'язку, просто іноді
цікавтеся їх справами- це їм дозволить відчути себе живими;

❤️якщо ви поруч, просто обійміть, але будьте готові до можливих сліз чи істерики
з нашої сторони, не зупиняйте цей "прорив", людина мусить виплакатись, так вона
очищає свою свідомість від важких емоцій, просто в цей момент будьте поруч. Це
дорогоцінно;

❤️якщо можете залучити до якогось справжнього робочого процесу, зробіть це, ми
дуже раді нарешті вибратися з полону горя і стати корисними для соціуму, це
лікує;

❤️не спішіть давати наполегливі рекомендації як людині ліпше приходити в себе. У
кожного це свій шлях і свої методи. Просто бути поруч як другу- цього,
впринципі, достатньо.

Я думаю, що зараз можу сказати за багатьох, хто вимушено переселений: кожна
людина, яка трапилась на нашому шляху, допомогла чи просто була поруч, ви- наші
янголи. Ми тримаємось тільки тому, що ви нас підтримали. Ви в наших молитвах, в
наших думках, серцях. Ми вас бачимо поруч з такими поняттями як "надія" і
"майбутнє". Нехай в житті кожного з нас зійде сонце надії🇺🇦.

На фото: 

Моя картина "Світанок" 2019 р. 

Вірогідно, назавжди втрачена.

%\ii{19_04_2022.fb.nebesna_anna.mariupol.1.poraneni_dushi.cmt}
