% vim: keymap=russian-jcukenwin
%%beginhead 
 
%%file 30_12_2013.fb.arestovich_alexei.1.sede_vacante.cmt
%%parent 30_12_2013.fb.arestovich_alexei.1.sede_vacante
 
%%url 
 
%%author_id 
%%date 
 
%%tags 
%%title 
 
%%endhead 
\subsubsection{Коментарі}
\label{sec:30_12_2013.fb.arestovich_alexei.1.sede_vacante.cmt}

\begin{itemize} % {
\iusr{Анатолий Неёлов}
Алексей, ты тупо прав... однако...

\iusr{Алексей Арестович}

Однако, общество еще не попало в серьезную беду. Так - насморк.
Когда попадет - тогда все изменится.

\iusr{Andriy Horbal}
Не завалити найбільшого мамонта, а змінити клімат.

\iusr{Roman Bas'kov}
и тут Остапа понесло...)))

\iusr{Stas Boiko}

Эта задача стоит на несколько уровней выше, чем сейчас способно решать наше
общество исходя, как минимум, из текущего умения сотрудничать и доверять,
которое только начинает зарождаться. А в долгосрочной перспективе да, надо
стремиться к такой модели.

\iusr{Jaroslava Igorivna}
Кого виштовхати? Добровільно!

\iusr{Наталья Ната}
Мне кажется , что настолько народ устал смотреть на цирк власти, что уже готов на все. Раньше ляжем раньше Встанем!!!

\iusr{Сусанна Тритяченко}

Власть- это отложенное, но предъявленное насилие, остальное притчи.

\iusr{Алексей Арестович}
Насилие бывает очень разных видов. @igg{fbicon.smile} 

\iusr{Сусанна Тритяченко}
Совершенно верно, моральное, физическое и на благо тоже. заставлять умываться, это тоже насилие)) даже уговорами))

Dmitri Gurewitsch
Я надеюсь все догадались, что народ наконец то должен себя вытолкнуть во власть! Время представителей прошло.

\iusr{Наталья Ната}
Да... :((( :))))) но хочеться думать о позитиве!

\iusr{Лариса Васильченко}

Ви мені неприємний, оце Ваше дозоване інформування, це так противно. Ви наче
зверху і все знаєте. І Вам насрати на цих людішок, а якщо дехто з них оцінить
Вашу мудрість - їх щастя

\iusr{Алексей Арестович}
Природа насилия тоже изменилась. @igg{fbicon.smile} 

\iusr{Лариса Васильченко}
медитація триває

\iusr{Алексей Арестович}
\textbf{Лариса Васильченко}, если бы мне было насрать, зачем бы я писал?:)

\iusr{Юрий Юхман}
вырвать больной зуб - это насилие?

\iusr{Сусанна Тритяченко}
Да, только инстинкт власти пока еще нет, вот и несрастец.))

\iusr{Julia Polibina}

Из опыта: если не появляется лидер, то на сцену выдвигается анти-лидер...
Психологи говорят, что это - закономерность... Может, следует принять меры -
чтобы анти-лидер не появился хотя бы? (я к тому, что пост Президента можно было
бы временно упразднить - это бы сняло накал страстей немного)

\iusr{Dmitri Gurewitsch}
@Лариса: чтоб воспринимать информацию надо до нее дорасти

\iusr{Алексей Арестович}
Нам спешить некуда. Вокруг - Вечность. @igg{fbicon.smile} 

\iusr{Гелена Мурлян}
Нанять немецких администраторов и хозяйственников и все! Пусть научат!

\iusr{Лариса Васильченко}
Дмітрію Гуревичу - якщо пишете російською, то хоч її вивчіть - правильно "дорАсти"

\iusr{Наталья Ната}
Можно вопрос: а как вам львовяне? После семинара? в общем?

\iusr{Татьяна Архипова}

Алексей, а "серьезная беда" -это как? Когда уже умер или парализованный
инвалид? И в том качестве как есть сейчас народ сможет ли этот народ увидеть
того, кто поможет ему? Или, как всегда, нужные дяди таки красиво "подскажут"
кого "нужно". А потом будут убеждать, что все по решению народа... Пока
качество человека не измениться ничего не изменится у нас. Даже лидера, что
сможет помочь, увидеть и вытолкнуть к власти не смогут...

\iusr{Юрий Юхман}
вечность сама по себе, а мы сами по себе.

\iusr{Алексей Арестович}

Так секрет нового лидера для народа, не в лидере, а в народе.
Серьезная беда - это тяжелое положение, коснувшееся лично всех при условии, что никто не знает, что делать.

\iusr{Лариса Васильченко}
Так як Вам, Алєксєй, спєшшить нєкуда і купа у Вас співбесідників, ВЕСЕЛИХ ВАМ СВЯТ!

\iusr{Dmitri Gurewitsch}
@Лариса: пардон... исправлю

\iusr{Андрей Любалин}
Amor wincit omnia

\iusr{Алексей Арестович}
\textbf{Лариса Васильченко}, и Вам!:)

\iusr{Алексей Арестович}
Кто будет субъектом люстрации?

\iusr{Molekula Valentyna}
Зрозуміла кого треба штовхати у владу - одного мого голосу замало. Допоможіть організувати процес!

\iusr{Наталья Ната}
Тот кто будет выдвигаться на пост
\iusr{Olesia Grabovska}

Очень глубокий пост. Единственное, что меня остановило - "лидер, которого
назначили против его воли". Это сразу как-то нивелирует для меня предыдущие
слова "...власть новых времен - это власть, которую не надо навязывать вопреки
сопротивлению тех, над кем она осуществляется". Я думаю, что принцип не
навязывания власти (как и всего другого) должен действовать в обе стороны - и
обществу, и лидеру. И кроме того, очень сомнительно, что можно грамотно
распорядиться навязанной властью... Может, я чего-то недопоняла в выводах?


\iusr{Роман Цапик}

А не готов народ к новым лидерам, не готов... Потому как что б быть новыми
лидерами надо говорить правду, мало того надо делать, очень много делать, а еще
больше надо делать народу, а еще больше надо думать, учиться, проявлять
инициативу.... А народ не готов, делать, не готов думать, не готов учиться он
даже правду слушать не готов... Его этому не учили. Да и лидеры тоже не готовы
(я не о горыныче), у лидеров пока есть что терять, и пока есть что терять
систему не завалить... Система любого лидера съест и не подавиться. Печалька...
Но ничего не поделать, пока.

\iusr{Татьяна Архипова}

ребенок- семья- человек- народ-лидер. Меняться нужно нам. И революция здесь
никак не поможет, только эволюция человека как вида. Вопрос в том, хватит ли
времени что бы пройти череду изменений и стать действительно человеками? Или
"серьезная беда" придет раньше?:)

\iusr{Наталья Ната}
Я уверена , что есть такие люди! Но они наверное скромные вот и надо их выдвигать! Может и насильно...

\iusr{Andrew Sivach}
Достойных людей нужно уговаривать править.

\iusr{Andrew Sivach}
Это факт

\iusr{Александр Грабин}

Да и сам по себе лидер без команды либо впадет в дисперсию, если совестливый,
либо плюнет на все и примет правила игры окружения... Выталкивать во власть
надо целую толпу единомышленников!


\iusr{Julia Polibina}

Есть лидеры, которые не воспримутся сейчас... Люди не готовы их принять. Один
пример того что говорит В. Горбулин в статье "ЗН"... Я читала комменты - и вижу
что здравые мысли воспринимаются в штыки именно из-за того что "харизма лидера"
для текущего момента не воспринимается

\iusr{Andrew Sivach}
а можно ссылочку на статью?

\iusr{Julia Polibina}

\href{http://gazeta.zn.ua/internal/kamo-gryadeshi-_.html}{%
Камо грядеши?, Владимир Горбулин, zn.ua, 27.12.2013%
}

\iusr{Ігор Скальський}

\href{http://www.snob.ru/profile/5232/blog/56510}{%
Борис Акунин: Теория поплавка, snob.ru, 13.01.2013%
}

це хвороба, хворіє зранений
організм, з кожним просуванням Ординської хвороби, в 13ст, в 17ст, 18ст, 20ст,
рецидивами, хвилями, все далі на Захід, ми відбиваємось, а воно знову накочує .
Пора йти назад 

\url{http://www.youtube.com/watch?v=zdrcgLqdbnE}

\iusr{Наталья Ната}

А Соскин !? смотрела на 112 кан сегодня. Тоже, так не скрывая правду говорил.
Мне нравятся его статьи. Умный человек


\iusr{Ирина Стежка}

Есть же такое понятие БРЕМЯ власти, тягар по-нашему)) Когда была жива лидер
сахаджа-йоги, то вибрационно выбирался самый центральный человек (гармоничный), в
подробности сейчас не буду вдаваться, и он был лидером - страны или города, но
лидерство ограничивалось координаторством, а если начинал давить, то у него сразу
конечно падал вибрационный фон и его снимали)) Думаю, что по-большему счету в
будушем лидеры будут такими координаторами, которые будут заинтересованны в
том, чтобы все ладилось, а значит было в руках умных и знающих людей, профи и
конечно порядочных, ведь без этого ничего не скоординируешь. При такой ситуации
эго не задействовано - лидер как бы и не делает ничего, есть команда, где каждый
в ответе за свой участок. Это высшая форма лидерства, когда люди думают, что они
сами все делают и над ними нет никого.

\iusr{Julia Polibina}

Да, Наталья! Я смотрела его - и не раз! Очень разумный подход... НО! Он сам
говорит: его НЕ ЗОВУТ, не просят поделиться мнением и рассказать... в нем не
заинтересованы. Именно об этом я и говорю. Не в лидерах вовсе дело. "Тройняшек"
устраивает ситуация - они вовсе ничего менять не хотят. И не собираются! Именно
поэтому есть опасность возникновения "анти-персонажа", который может негативно
повлиять на ситуацию...

\iusr{Sergey Marchuk}

Для того, чтобы эта абсолютно правильная идея овладела массами и в конце концов
реализовалась, - одного поста в одном фейсбуке недостаточно. Даже с учетом
перепостов. @igg{fbicon.smile} 

\iusr{Роман Цапик}

Достойный, лидер, править, эффективно не бывает.... Ирина Стежка опередила
задача не править а координировать и модерировать работу.

\iusr{Петро Обухов}
habemus papam

\iusr{Віталій Коваляк}

Поки що люди хочуть, щоб прийшов хтось і вирішив всі їхні проблеми. Але без
їхньої участі. Ми завжди знайдемо тисячу відмазок, лиш би не змінюватися самим.

\iusr{Valerii Borys}

Гадаю, духовні лідери є (Філарет, Гузар, Ліна Костенко, ініціатива 1 грудня
тощо) і було (Шевченко, Франко, Бандера) вже їх достатньо, і зробили свою
справу: повстали люди, стоїть Майдан і активісти піднімаються скрізь по країні

\iusr{Максим Банасевич}

Я сегодня много от кого слышу что "есть лидеры"... ребята - это не лидеры. Это
люди которые хотели бы ими быть... Реальный лидер - очень быстро соберет вокруг
себя инициативную группу. Или придет и подчинит уже существующую (я не про ту
которая сегодня сидит в доме профсоюзов). Не знаю как вы - пока что я таких
лидеров не видел. То что люди поддаются риторике оппозиции - тут все понятно.
За не имением других - слушают тех кто есть. Но - предчувствую что это скоро
переменится.

\iusr{Наталья Ната}
Юля! Согласна с вами.

\iusr{Максим Банасевич}
По поводу "уговаривать" - я бы не согласился. Хотя и такой вариант может быть. Всяко история знала.

\iusr{Максим Банасевич}

я бы не согласился что это тот случай =))) в смысле
@igg{fbicon.face.smiling.eyes.smiling} 

% -------------------------------------
\ii{fbauth.polibina_julia.kiev.ukraina.alpha_people.direktor}
% -------------------------------------

Лидер - это человек, который не только согласен "нести лавровый венок", но и
готов во имя ОБЩЕГО ДЕЛА всего себя, полностью и безраздельно ему посвятить. И
быть готовым к тюрьме, к пыткам и к смерти... И что?... Кто-то пойдет на это
если его "заставить"? Тут нужна не только воля и вера, не слепое подчинение
"толпе" - нужно осознанное ПОНИМАНИЕ ЦЕЛИ! Нет лидеров... нужно ситуацию
выравнивать... Алексей написал - стволов на руках много (и это, увы, правда)...

\iusr{Алексей Арестович}
Спасибо, Антон. @igg{fbicon.smile} 

\iusr{Игорь Зюбин}
\textbf{Алексей Арестович}, я выталкиваю Вас во власть! Кто за? Прошу голосовать!

\iusr{Алексей Нужный}

Когда человек сомневается, он колеблется. В буквальном смысле. Делит 100\% своей
воли ровно пополам на два противоположных вектора. Заражает этим других, ищет
поддержки и опоры извне, тонет, вместо того, чтобы вспомнить, что он прекрасно
умеет плавать с рождения, что волна не против, а за, что перед рассветом ночь
темнее, что быть честным и искренним гораздо легче и приятней... и просто
позволить случиться тому, что и так давно случилось... как сказал один шутник -
нам уже давно все можно, просто мы еще не хотим)

% -------------------------------------
\ii{fbauth.medvedjko_andrej.kiev.ukraina}
% -------------------------------------

\obeycr
Восстала нетеpпенья дpожь,
И pвутся удила.
Нам ясен путь,
Нам нужен вождь,
Котоpый - как стpела.
Нам нужен вождь, чей взгляд, как нож,
Вонзиться б в сеpдце мог.
Нам для Джихада нужен вождь,
Котоpый ликом - Бог.
Нам нужен вождь, как чистый дождь,
Скопилась меpзость в нас.
Нам для Джихада нужен вождь,
Котоpый духом - князь.
Тот стpах не в силах пpевозмочь,
В ком голос пpедков смолк.
Нам для Джихада нужен вождь,
Котоpый сеpдцем - волк.
\restorecr

\iusr{Lucy Smith}

Лідером має бути жінка-лікар.

\iusr{Орияна Беленькая}

Турпоход на межигорье, ясно показал, что хоть Гитлера в лидеры поставь, уже не
заметят, мозги отключены, зачем думать, лидеры думают и ведут, вопрос куда?

\iusr{Алексей Нужный}
Красиво, наверняка искренне.. но имхо - вождь - поводырь для слепых.. а время прозревать

\iusr{Алексей Нужный}
всем

\iusr{Орияна Беленькая}
Да, желательно психиатр!

\iusr{Valerii Borys}

Цікаво, що мешканці Південної та Східної України таким лідером вважають овоча /
див. Денис Казанский :: Проти фюрера не бунтують. (c) 

\href{http://tyzhden.ua/Society/97204}{%
Чому не страйкує Донбас?, ДЕНИС КАЗАНСЬКИЙ, tyzhden.ua, 27.12.2013%
}

% -------------------------------------
\ii{fbauth.pozumentschikov_viktor.dnepropetrovsk.ukraina}
% -------------------------------------

Вне зависимости от того, верим мы или не верим, знаем мы или не знаем – Бог
существует (особенно в религиях))). Вне зависимости от того, верим мы или не
верим, знаем мы или не знаем, существует Истина и Путь, который мы еще не
прошли… Новые Системы - в начале существуют в отдельных головах, как
"Утопии"... 

Как известно, затем Утопия становится реальностью (в истории - одной из трех
составляющих марксизма-ленинизма). Иногда Истинное кажется абсурдом для
перевернутого сознания, особенно, если сознание откалибровано, как фанатичное и
зашоренное. Для массы людей необходимы чертежи-схемы, и пошаговое исполнение -
творчески мыслящих (с большой битностью) не так уж много, а вернее - единицы.
Алексей, я в тех же раздумьях... Сегодняшние лидеры - тайные) Их удел мыслить,
желательно вслух)) А пазлы сойдутся потом.. на общей картине... - культуры,
духовности, хлеба насущного.... 

В эпоху Водолея, на мой взгляд, ломая матрицы и мещанские системки, лидеры
отличаются от болтунов-политиков тем, что их мысли, как людей сделавших шаг в
сторону, экстравагантны и отличаются от телевизионного стереотипа, т. е.
утопические)))), но Системные - это факт...  Всему свое время... В начале было
Слово - тривиальнейший факт)))). В воздухе витает экзотерический запах НОВОЙ
СИСТЕМЫ))))!!!!!!

\iusr{Сергей Данилов}

Красиво и содержательно. Браво! Я б Мирослава Поповича выталкивал бы. Давайте
подумаем, как создать этот выталкивающий инструмент? И главное как держать на
плаву этот трон, а может это лифт?

\iusr{Дмитро Купач}
Не з усім погоджуся, але цікаво та корисно

\iusr{Svitlana Gunning}

Что случилось со СТО? Куда она делась? ))) и кого выталкиваем? Противоречие
логическое намечается: предлагается во власть выталкивать, но тут же оговорочка
(правильная кстати), что власть не по плечу случайным людям. Жду продолжения )

\iusr{Ігор Шаповалов}

Ось і з"явились перші "найбільші" краплі в океані - лідери!  @igg{fbicon.frown} 

\iusr{Тарас Куць}

А це не СТО.)))) Це так, "поржать". Вірніше черговий "вброс" інформації до
роздумів. З обов'язковим спостереженням куди ці роздуми переважну більшість
приведуть. Підозрюю, спроба виявити людей з чітко вираженою власною позицією і
що це за позиція. Себто який відсоток читаючих йде за потоком.


\iusr{Ігор Скальський}

2000 років як в Греції вигнали тирана і спокійно обходились без нього, а ви
знов на ті самі граблі - лідера нам. ми зграя, нам потрібний вожак, у нас це
в генах закладено ...як же ш без нього ? Так демократія потребує громадян -
особистостей і ми в стані ними стати.

\iusr{Тарас Куць}
Чи може то Рівненський семінар на Олексія так вплинув)))...

\iusr{Ігор Скальський}
модерує рух думки  @igg{fbicon.smile} 

\iusr{Ольга Рудь}

Что-то я не поняла: первый посыл - никто не хочет в лидеры, второй - совестные
не хотят (тобишь, они, бессовестные, хотят). На фига тогда было говорить, что
никто не хочет? Противоречие какое-то О\_о. 2) Так кто же тот единственный
римский правитель?? 3) для меня здесь сплошь ложные посылы, как то: "совесть",
"надежды и мечты (народа)", "духовный и религиозный", а потом еще и "общество
не готово". ...Не понимаю, чем может помочь информация этого поста? Это ради
"поболтать"? Или собрать очередную порцию лести от массы инертных недумающих
людей, слепо верящих в идеи/правоту/святость или что там еще Алексея? Или,
может, выявить этих инертных?  @igg{fbicon.wink}  ... Аbsurdus.

\iusr{Тарас Куць}

На 2-ге питання можу відповісти: звичайно ж Олексій, він так і написав: "Секрет
избрания власти новых времен в том, что в нее должны выталкивать. Тех людей,
которые от нее бегают, не хотят ею заниматься."  @igg{fbicon.smile}  Мені здається досить таки
прозоро)))

\iusr{Yevgen Yeroshenko}

"Чуда, чуда давай! Требуем чуда!" (с)

\href{https://www.youtube.com/watch?v=s2hmqvh9TjE}{%
Чудо, youtube, 12.03.2012%
}

\iusr{Виктор Позументщиков}

Революция происходит в головах, сердцах, душе. Ломая стереотип мышления,
возвышает Дух. Эволюция меняет формацию, культуру, экономику. Ломая Систему,
возвышает Свободу и Счастье.

\iusr{Светлана Кондратьева}

А власть и не требуется навязывать, ни тем ни другим; власть обеспечивает
действенность законов, и четкую сменяемость людей, приходящих во власть, В
такой стране людям нет дела до политики, она не отвлекает их от дел на благо
страны, ибо каждый занимается полезным и нужным, в силу избранной профессии по
своим способностям. Здесь, где закон защищает интеллектуальную и вещевую
собственность, запрещает взятки, ибо это наказуемо судебн. органами страны,
этому способствует гласность СМИ и общественное мнение. Разве это не выполнимо?
Примеры есть. Политика - дело политиков. Особый респект автору, которого читать
и интересно, и полезно, спасибо.

\iusr{Andre Gluhariv}

Нельзя отказываться от уже изобретенных элементов схемы велосипеда  @igg{fbicon.smile}  Например,
ну нельзя президенту назначать губернаторов или меров, верховные судьи должны
преимущественно быть назначены предыдущими правительствами, по схеме которая
автоматически ставит "смотрителем" человека из оппозиции, число бытовых
вопросов и сфер, в которых центральные законодатели могут принимать законы
должно быть жестко ограничено и т.д. Есть целый ряд простых "механических"
решений, которые ограничивают возможность произвола для любой власти.
Проанализируйте несколько европейских или американских дел против
коррупционеров, например возьмите случай предпоследнего мера Чикаго - очень
хороший способ увидеть где расставлены системные крючки и заборы, делающие
проблему "плохой власти" безличной, решаемой на уровне структуры.

\iusr{Павел Шерман}

na samom dele tyakoj lider v ukraine uge est ...eto ruslana legichko..... no k
sogaleniyu ona poka ne znaet chto delat I ne hochet bit liderom + sama malo
chto smoget.... nugna komnda naprimer + arestovich, lide r 95 kvartala, sobole
v, naem I td ne bolee 7-10 chelovek

\iusr{Владимир Липяцький}

После прочтения, почему то мысль пошла гулять, что Власть - это половой акт с
народом. Одной нравится, другой мало, третьей много. Только не смотря на
некоторую несовместимость, все они должны хранить верность насильнику. И
вспомнился анекдот про Вовочку....

\iusr{Трактин Руслан}
Всегда есть лидер!

\iusr{Vitalii Dem'ianiuk}
Глибоко

\iusr{Валерий Тудвасев}

Вчера послушал Вас на 1+1 про трехстороннюю комиссию для расследования
резонансных преступлений. В этой связи вспомнил о таком механизме как
парламентские следственные комиссии которые легитимно могут вести расследования
и очень эффективно используются в других странах.

\iusr{Алексей Арестович}
У нас они тоже действуют, но их явно недостаточно. В данной ситуации.

\iusr{Алексей Арестович}
\textbf{Владимир Липяцький}. Речь и идет об изменении природы власти.

\iusr{Володимир Ярошенко}
Кажете виталківатель ))). Так люди ж можуть повірити в цю ідею і будуть кричати: "Арестовича на ешафот, тобто в президенти"

\iusr{Lily Rudnytska}

Простите, но Иисус Христос уже изменил природу этой власти, но ей не очень
пользуются. Лука 22:25-26 Он же сказал им: цари господствуют над народами, и
владеющие ими благодетелями называются,

26 а вы не так: но кто из вас больше, будь как меньший, и начальствующий - как
служащий.

\iusr{Сергей Данилов}

\obeycr
прийняття європейського закону про ЛЮСТРАЦІЮ!
Люстрація – головна вимога МАЙДАНУ!
Інакше нас будуть регулярно зраджувати, зливати і т.д.
Я цього не хочу!
Я хочу щоб нами керували кришталево чесні люди, а в цій системі вони працювати не зможуть.
Люстрація розчистить дорогу для патріотів і одним махом змете всіх кровосісів! В тому числі багато теперішніх очільників!!! Тому цю тему або замовчують , або заговорюють!
Буду вдячний за перепости, викладення в блогах і будь-яке інше поширення ідеї ЛЮСТРАЦІЇ!
Це повинно звучати у кожному виступі на НАШОМУ МАЙДАНІ!
Слава Україні!
\restorecr

% -------------------------------------
\ii{fbauth.kovalenko_ljudmila.kiev.ukraina.turizm.kneu}
% -------------------------------------

А кто мне может объяснить? Я о Майдане. А почему вместо нереальных лозунгов –
типа «геть», Майдану не сосредоточиться на созидательной и объединяющей всю
нашу страну цели Возрождения Украины и на всех отсюда вытекающих ближайших и
дальнейших задачах? Я думаю, если бы за стол переговоров от Майдана сели
профессионалы экономисты, дипломаты от гражданского актива и представители
действующего правительства, то можно было бы найти компромисс: волки живут
старыми жировыми запасами, но больше не воруют и т.д. – овцам(сори, из
поговорки слов не выбросишь) не мешают возрождать страну (прогрессивные Законы
общества и контроль за их выполнением). Возможно это тоже тупик, но попробовать
же можно? Такой шанс. Для этого нужен лидер или сами люди должны отстоять СВОИ
лозунги? Неужели у нас нет моральных профессионалов, которым совесть не дает
спать по ночам спокойно, из-за того, что они, могут, что то сделать, но ничего
не делают... Люди на Майдане с чистым сердцем – они распознают своего лидера,
только сколько им ждать?

\iusr{Алексей Арестович}
Друзья, я учел Ваши замечания и отредактировал текст.

\iusr{Natalya Belyukina}
Ну Слава Богу, а то я начала уже бояться за тебя, Алексей. Начать отбрасывать лукавство- это здорово...:))))))))

\iusr{Алексей Арестович}
По мере дозревания ситуации. Вся правда за один раз не проходит. @igg{fbicon.smile} 

\iusr{Алексей Арестович}
\textbf{Игорь Зюбин}.  @igg{fbicon.smile}  спасибо, конечно. Соберете еще миллиона два-три?:)

\iusr{Dmytro Symovonyk}
Олексій, це Ви передбачали теракти в Росії?

\iusr{Molekula Valentyna}
Алексей, спасибо, что это диалог)

\iusr{Алексей Арестович}
Игрок, которому досталась роль Бога, сказал, что он скоро так озадачит Россию, что ей станет не до Украины.

\iusr{Dmytro Symovonyk}
Дякую. На зустрічі у Львові Ви говорили також про московський "Майдан". Це щось інше, наскільки я розумію.

\iusr{Алексей Арестович}
Почему? Это возможное развитие. Если ситуация станет неуправляемой, народ пойдет на улицы. В Волгограде уже паника.

\iusr{Dmytro Symovonyk}
Дякую. Зустріч була також дуже цікавою, можливо не все зрозумів, але треба продовжити аналізувати.

\iusr{Алексей Арестович}
Спасибо Вам за гостеприимство. Надеюсь, еще увидимся.  @igg{fbicon.smile} 

\iusr{Сергей Данилов}
: )

\iusr{Oleg Duchiminskij}

Зібрались мої друзі - десяток свідомих, самодостатніх, патріотично настроєних.
По приколу запропонували: "давай ми тебе висунемо в мери". Я впирався, але
погодився, але сказав що буде важко і буду строгий. Усі сказали що нічого, ми
підтримаємо вибори виграєм, а потім............. ти нам "порішаєш" земельні
ділянки і інші питання... все... приїхали... Усі зрозуміли що щось не так.
Зміювати треба систему.

\iusr{Алексей Арестович}
\textbf{Oleg Duchiminskij}, так отож. @igg{fbicon.smile}  у головах.

\iusr{Алексей Арестович}
\textbf{Irina Muskina}, он сказал, что у нас как раз, все будет хорошо. @igg{fbicon.smile} 

\iusr{Nataliya Bezborodova}

Уже не раз читала тут о параллелях с Сечью, и вот: «Двое из них схватили его
под руки, и как он не упирался ногами, но был наконец притащен на площадь,
сопровождаемый бранью, подталкиваньем сзади кулаками, пинками и увещаньями. –
Не пяться же, чертов сын! Принимай же честь, собака, когда тебе дают ее!

Таким образом введен был Кирдяга в козачий круг.

- Что, панове? – провозгласили во весь народ приведшие его. – Согласны ли вы,
чтобы сей козак был у нас кошевым?

- Все согласны! – закричала толпа, и от крику долго гремело все поле.

Один из старшин взял палицу и поднес ее новоизбранному кошевому. Кирдяга, по
обычаю, тотчас же отказался. Старшина поднес в другой раз. Кирдяга отказался и
в другой раз и потом уже, за третьим разом, взял палицу. Одобрительный крик
раздался по всей толпе, и вновь далеко загудело от козацкого крика все поле»

(Н.Гоголь)

\iusr{Александр Сичкарук}
Общество ищет лидера. Лидер не появляется. Это точно.

\iusr{Andrii Kolpakov}
как военные могут что-то предлагать? разве в Украине есть боеспособные воинские части? кроме МВД тут никого из силовиков не осталось.

\iusr{Иван Бурьян}
Давно так думаю....., и уверен это время прийдёт)

\iusr{Illia Bielov}
стране нужен Бэтмен. \url{http://vodichka.com.ua/library/101/}

\iusr{Oleksandr Krasnoperov}
Ой, как в точку! "Власть - раскаленная сковородка. Уже больше месяца она валяется на улице и никто не хочет ее подобрать". (с)

\iusr{Molekula Valentyna}
\url{http://www.youtube.com/watch?v=3kQJGRLdI6I}

\iusr{Яків Міхеєв}

Когда человеку дана такая власть, как в притче, ему хочется управлять людьми а
не послушными биороботами, думающими, уважающими себя и других Людьми, с
большой буквы

\iusr{Марк Гордиенко}

Браво.

\iusr{Марк Гордиенко}
И еще притча, в тему. Человек пришел к Будде и сказал:

- Я очень богат, у меня нет детей, моя жена умерла. Я хотел бы сделать
какую-нибудь работу для заслуги. Что я могу сделать для бедных и униженных?

Только скажите мне, что я должен делать?

Услышав это, Будда стал очень печален и слеза скатилась по его щеке.

Человек был озадачен такой реакцией. Он спросил:

- В твоих глазах слезы? Ты вдруг опечалился, почему?

Будда ответил:

- К сожалению, ты не сможешь никому помочь до тех пор, пока не поможешь себе.
Твой основной "металл" еще не стал золотом, ты не можешь сделать ничего
сострадательного, ибо твои энергии в самом низу. Ты хочешь помочь людям, но в
тебе еще недостаточно осознанности. Ты не имеешь подлинного центра, откуда
может струиться сострадание.

\iusr{Владимир Липяцький}

Сомневаюсь, что я ошибаюсь. Почитал комментарии и понял никому не нравится как
его имеет власть. Те кому понравилось здесь не ходят. Одни неудовлетворенные. А
хотелось сказать, я ошибаюсь...

\iusr{Мила Коган}
Классно все по полочкам

\iusr{Igor Golfiño Sivak}

\obeycr
"КНИГА О ЖЕРТВЕ
Какою силою утвердитесь?
Как достигнете исполнения Нашего дела?
Властью, Нами данною.
Мне ли говорить о власти?
Когда все глупое и когда все тщеславное к власти устремляется.
Но Я говорю и утверждаю.
Но Наша власть иная -
Наша Власть - Жертва!
Поясню немногими словами.
Когда Курновуу Правитель
Созидал Золотые Врата,
Он стремился во храм.
Но все же донес свою жертву.
Когда Соломон искал власть красоты,
Когда символом Суламифи был показан
Символ нечеловеческой правды,
Он все же остался царем
И донес свою жертву.
Когда духовный учитель Тибета Аллал-Минг
Стремился в горы, где впервые предстал ему Бог.
Он все-таки остался в долине
И принял чашу.
Когда Шейх Россул-Ибн-Рагим
Стремился передать власть сыну,
Он все же услышал Голос
И отдал все, чтобы дойти.
Когда учитель Ориген отдавал
И телесное и духовное, лишь бы
Научить их последнему преданию Христа,
Он все же сохранил тяготу Учительства.
Когда Сергий из Радонеги уклонился
От престола Митрополита,
Когда Он стремился говорить со зверями,
Он все же остался строить дома Общежитий
И Он сохранил около себя учеников.
Когда Акбар, названный Великим,
Слагал камни единения церкви,
Душа Его стремилась под дерево мудрости,
Где сходило Ему просветление,
Но Он все же остался на ступенях трона.
Зная, что есть подвиг,
Зная, что есть власть-жертва,
Если, утверждая завоевание, произнесете:
- Господи, да минет меня чаша сия!
Значит, вы уже имеете право творить,
И дух ваш уже несокрушим.
Запомните эту книгу о жертве,
Ибо она дает вам врата к завершению и готовности.
А будучи готовыми, вы знаете все,
Ибо вам все будет открыто, и принесено, и рассказано.
Но только откройте уши и запомните.
А главное, читайте и повторяйте,
Ибо часто оболочка ваша затемняет знание духа.
И, глядя на сужденный пожар,
Вы скажете: вот почему вчера я вынес вещи мои.
И, глядя на молнию, вы преклоните голову, почитая Веление Бога.
Я сказал и заповедал - храните".
Зов, Сентябрь 23, 1922 г.
\restorecr

\iusr{Oleksandr Krasnoperov}

Простите меня, гада  @igg{fbicon.smile}  Сейчас, кроме Анатолия Гриценко, я не вижу другого
человека, которого можно попросить попытаться нами руководить  @igg{fbicon.smile} 

\iusr{Oleksandr Krasnoperov}

Простите меня, гада  @igg{fbicon.smile}  Сейчас, кроме А. Гриценко, я не вижу
никого, кого можно попросить нами, непутевыми, управлять  @igg{fbicon.smile} 

\iusr{Igor Golfiño Sivak}

Ещё есть Алексей Арестович  @igg{fbicon.smile} 

\iusr{Iryna Iab}

попала на глаза статейка \url{http://www.litsa.com.ua/show/a/826}

"Казак — это не гопак, а (согласно научному определению) лично свободный и
вооруженный мужчина. Подчиняться случайным людям было не в его правилах.
Поэтому в основе школы запорожского лидерства лежала прямая угроза. Прежде чем
выпячивать грудь и раздувать щеки, запорожец задавался вопросом: “А не тонка ли
у меня кишка?” Исходя из многочисленных примеров, он знал, что завоевать здесь
доверие общества и не оправдать его — означает верную гибель.

Каждый казак, решившийся выделяться своими достоинствами из общей массы,
рисковал быть избранным на должность. При этом его согласия никто не спрашивал.
Отказаться от предложенной чести можно было только символически, не более двух
раз. Если кандидат начинал упираться по-настоящему, то, в лучшем случае, его
могли сделать инвалидом, чтобы впредь не выдавал себя за крутого."

\iusr{Сергей Лукьянец}

\url{https://www.facebook.com/groups/223130817802885/permalink/327088357407130/}

\iusr{Konstanty Jakuszewski Cat Donskoj}

\href{https://wol.jw.org/ru/wol/b/r2/lp-u/bi12/66/16#study=discover}{%
Откровение - ОНЛАЙН-БИБЛИОТЕКА Сторожевой башни%
}

\iusr{Konstanty Jakuszewski Cat Donskoj}

\url{https://www.jw.org/ru/библейские-учения/вопросы/что-такое-царство-бога}

\iusr{Inna Shynkarenko}
сказка - ложь, да в ней намёк...  @igg{fbicon.wink} 

\url{http://www.sheremet.org/content/fizik-yadernik-oleksandr-harchenko-stverdzhuie-shcho-yomu-vdalosya-pobuvati-v-maybutnomu}

\iusr{Олександр Бурейко}

Майже 10 років працюю в органах влади, приходив думаючи, що тут сидять старі
пердуни, які досі залишились нам у спадок від КПРС. Хотів показати як треба
працювати.

Те що побачив шоком не назву (бо прийшов з армії), але складність ситуації
коротко не пояснити (відсутність ресурсів, лещата між вищим керівництвом та
населенням, нерозуміння деяких рухів, відсутність методичності і організації
тощо).

Спеціалістів до органів влади, як виконавчих, так і місцевого самоврядування,
потрібно готувати прискіпливо, з психологічним та фаховим відбором. Потрібно
налагоджувати системи звітності та координації між гілками влади. Більшість
осіб, які прийшли до органів влади з надією розбагатіти, принаймні за межами
Київської області та АРК тільки отримали проблеми. В органах влади страшний
дефіцит кадрів, зарплати мізерні - від мінімалки у спеціаліста до близько 4
тис. грн. у голови РДА. Люди тікають, кращі давно пішли, на роботу доводиться
брати вчорашніх студентів, вакансії до третини від особового складу.

\iusr{Олександр Дудка-Булига}

Не нужно смешивать всё в одну кучу. В древности Держава фактически состояла из
четырёх автономных Держав. Каждая каста имела свои внутридержавные правила.
Называлось всё это Теократией из которой потом вышел Республиканизм как одна из
составных частей Теократии. Фактически сейчас стоит вопрос о системе мирного
сосуществования двух Украин - Украины козаков и Украины гречкосиив и
свинопасов. У каждой из них будет своя Держава со своими правилами. Но они
как-то должны научиться уживаться между собой на одной территории. Одни будут
добровольно и сознательно платить налоги на совместные проекты. Со вторых будут
всегда налоги собирать насильно. И это только две касты мы сейчас берём а ведь
их четыре.

\iusr{Ольга Рудь}

Вот обсуждают вполне конкретные и действенные варианты, как разбудить сознание,
как превозмочь индифферентность, пассивность, что может на это повлиять:

\href{http://www.youtube.com/watch?v=MfZI-Pd7Cg8}{%
Пьотр Павленський на Hromadske.TV, hromadske, youtube, 30.12.2013%
}

Про трансформацию объекта в субъект управления, про "готовность", про
действенность Майдана, ожидания и разочарования. +Живое общение всегда заряжает
энергией.

\iusr{Irina Borisova}
SEDE VACANTE = свободный престол.

\iusr{Едуард Моисеев}

Алексей, согласен с Вами. Только престол должен стать рабочим местом! Необходима
новая система, проекты и программы новой системы должны иметь индекс социальной
и экологической капитализации. На самом деле необходимость в кредитах -
иллюзия, навязанная США и Европой. Это они в наших ресурсах нуждаются, а не Мы
в их деньгах. Вся их сила в умении убедить в том что нам без их денег ну никак.
И они пока с этим успешно справляются. Это исправимо, а деньги мы и сами
печатать умеем. Причем делать это будем тоже по совершенно новой системе.


\iusr{Andrey Khavryuchenko}
Олексій,
> Часть из них, конечно, на руках у представителей власти, возможно, большая часть, но все равно речь идет о десятках тысяч стволов в распоряжении граждан, которые, по их словам, власть ненавидят. И что?..
А що Ви зробили зі своїм стволом?
Я зі своїм - знаю що. І ті люди, з якими я домовлявся - теж знають
А Ви?

\iusr{Ігар Леднік}
\textbf{Алексей Арестович} Насилие бывает очень разных видов (c). Проблема перераспределения "потенциала" насилия (по Дугласу Норту) в сфере моих профессиональных интересов. Приходится применять различные вариации насилия по отношению к лидерам и на лидерство претендующим.

\iusr{Ігар Леднік}
\textbf{Антон Волчков}, переход от харизматичного типа легитимизации власти к европейскому, легально-рациональному - самая что ни на есть реальная РЕАЛЬНОСТЬ

\iusr{Nick Feldman}

Природа реальной естественной и спокойной власти народа заключена и в том, что
вот я вижу сейчас пост \textbf{Алексей Арестович} и после прочтения этого поста мне
очень хочется "почему-то" сделать всё для того, чтобы он стал во главе страны.
И если таких, как я - миллион, например, по всей стране наберется, остальные 30
миллионов не смогут с этим не считаться. И если все эти тысячи людей дальше в
спокойном режиме смогут реализовать свой стратегический план - рано или поздно
власть изменится полностью. Поэтому начинать необходимо. Вообще отчего-то
радостно читать такие тексты. После них появляется уверенность в завтрашнем дне
и будущем страны. Это ведь хорошо, когда такие самобытные лидеры рождаются вот
так на глазах. Пусть не все это сразу поймут, пусть тролли занимаются своим
любимым делом, но имеющий глаза - точно увидит  @igg{fbicon.wink} 

\iusr{Сергей Данилов}

Может проблема лидерства в современных условиях связана с тем, что во власти
теперь нечего ловить, кроме не здорового роботняка, разрушенной экономики,
доведенного до инвалидности зомби населения. Личное что можно поиметь давно
разграбили и зарегистрировали за кордрном. Подписали международные договора о
не допустимости изъятий "к старому возврата больше нет". Олигархи упакованы и
сильны - своего не отдадут и все последние ресурсы свободного народа
старательно подмел и подсобрал последний, заключительный президент страны.
Теперь родился последний олегарх и администротивно ресурс больше не
распределяется- его нет. Выжимания на гос закупках доведены до обсурда, а в
остальных секторах надо долго работать. Продать даже себе, больше нечего. Ну
кто теперь возьмется за такую страну, погрязшую в украденных олегархами,
долгах. Это должен быть очень мужественный и отважный человек с непомерной
поддержкой мощнейших элит. Наша политика таких убивала еще в пеленках, их таких
на сегодня нет в нашей природе. Поэтому надо либо с природой что то делать ,
либо с племенным материалом из за рубежа что то решать. Ситуация с местными
лидерами - на сегодняшний день по-моему - очевидна. Узкокорпоративные интересы,
уже не в силах обеспечить достойный уровень даже собственным сотрудников - это
менее 6\% населения. : )

\iusr{Алексей Арестович}
\textbf{Nick Feldman}, чтобы я взялся порулить, меня нужно ну очень просить. @igg{fbicon.smile} 
Не потому, что я типа гордый.
Просто состояние сознания, необходимое для того, чтобы рулить такими жесткими процессами, это не то, что меня особенно сильно интересует.
Но, если попросят, тогда, конечно, поработаю.

\iusr{Nick Feldman}
Да, я согласен. Попросить должен где-то миллион. Но он начинается всегда с одного человека  @igg{fbicon.wink} 

\iusr{Алексей Арестович}
Уже три. @igg{fbicon.smile} 

\iusr{Yuri German}
Арестовича в президенты! А всю СТО - в печку)
\iusr{Сергей Данилов}

Варяги, я замечу это всегда хорошо организованные, вооруженные и развитые в
путешествиях советники. Избавиться от них почти не реально - это колониальный
способ развития территорий.


\iusr{Сергей Данилов}

Замечательные слова! Вещи надо называть только своими именами - специально
существующими для этого словами! Когда все начнут говорить именно так как оно
должно быть, трансформация будет потрясающая!


\iusr{Olexandr Korichev}
+1!...))

\iusr{Михаил Симагин}

Молодец Аристарх! С Новым годом всех! Думаю он будет лучше чем предыдущий, он
не будет легче, но будет лучше. Лёд тронулся и возврата уже нет. Недожав
11числа майдан власть промахнулась тотально.


\iusr{Алексей Цистан}
100\% !

\iusr{Лили Марлен}

хорошая притча ....она говорит о том что НИЧЕГО ИЗМЕНИТЬ НЕЛЬЗЯ  @igg{fbicon.smile}  - в смысле
БЫСТРО И СРАЗУ ...но вот есть такой Панарин А.С, "Искушение глобализмом" - тот
писал как это делали большевики, а потом спецслужбы ...

\ifcmt
  ig https://scontent-frt3-1.xx.fbcdn.net/v/t1.18169-9/1237535_684948188205092_1646142467_n.jpg?_nc_cat=108&ccb=1-5&_nc_sid=dbeb18&_nc_ohc=zqISPztKb1kAX8wKCsH&_nc_ht=scontent-frt3-1.xx&oh=c9140af67bbb72f336d0159e15f1febe&oe=61BA82DA
  @width 0.4
\fi


\end{itemize} % }
