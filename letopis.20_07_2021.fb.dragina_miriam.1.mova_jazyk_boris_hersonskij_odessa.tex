% vim: keymap=russian-jcukenwin
%%beginhead 
 
%%file 20_07_2021.fb.dragina_miriam.1.mova_jazyk_boris_hersonskij_odessa
%%parent 20_07_2021
 
%%url https://www.facebook.com/miriam.dragina/posts/10158876452278411
 
%%author Dragina, Miriam
%%author_id dragina_miriam
%%author_url 
 
%%tags herson,jazyk,mova,obrazovanie,obschestvo,odessa,prepodavatel,prinuzhdenie,ukraina,ukrainizacia
%%title как больно читать посты бориса херсонского. он отказался от преподавания в одессе
 
%%endhead 
 
\subsection{как больно читать посты бориса херсонского. он отказался от преподавания в одессе}
\label{sec:20_07_2021.fb.dragina_miriam.1.mova_jazyk_boris_hersonskij_odessa}
 
\Purl{https://www.facebook.com/miriam.dragina/posts/10158876452278411}
\ifcmt
 author_begin
   author_id dragina_miriam
 author_end
\fi

как больно читать посты бориса херсонского. он отказался от преподавания в
одессе, так как посчитал, что его украинский не достаточно хорош, сколько ни
совершенствует. а закон есть закон. 

и все равно вокруг травля с этими ‘професор з російськомовною щелепою‘,
‚україна для українців‘ и проч.  мало он сделал для этой страны? неочевидна его
позиция, его действия? 

upd. для всех, кто спешит обижаться, не вчитываясь: \verb|___|тут я отбила блоки
текста, как бы обращаясь ко следующему смыслу. второй абзац начинается со слов
"и все равно", которые также означают фразу "несмотря на это", то есть,
несмотря на отказ преподавать по собственному самокритичному решению человека
продолжают травить за его недостаточно хороший украинский в быту\verb|___|

что хотят построить здесь, какой мир, из кого, сидящие на диванах злобные
упыри, которым никак не понять, что полстраны отличается от них языком, как
могли бы отличаться цветом кожи. да, пол-украины говорит на русском, одесса и
киев — русскоязычные города. такой факт. история. и это не заставляет отрицать
украинский как государственный, не делает нас врагами. насилие с травлей это не
изменит. люди будут замолкать, уходить, замыкаться, терять мотивацию. и знаю,
что даже на этот текст возможна волна хейта, вроде: а ты, так учи, враги, война
и прочее. 

upd 2. для всех, кто спешит обижаться, не вчитываясь: \verb|___|в этом месте,
где мы вроде как выяснили, что речь уже не идет о государственном языке, я
перехожу к осуждению травли в быту. за слово упыри прошу прощения, это грубо. 

также поясню: киев я русским городом не называю, говорю "русскоязычный город",
там еще стоит слово "история", так как имеется в виду довольно продолжительный
исторический период. там еще есть фраза "и это не заставляет отрицать
украинский как государственный, не делает нас врагами", которую все
предпочитают не замечать и сразу переходят к верещанию, что мне нужно валить с
россию. о цвете кожи - не самое удачное сравнение, согласна\verb|___|

и, чтобы два раза не вставать: фейсбук — личное дело, не медиа, не университет,
не министерство. и ни тут, ни в остальном быту, никто не имеет права лезть с
советами, какой язык использовать. на войне одинаково гибнут за страну носители
украинского, суржика и русского. а если позволять лезть в личное, то дальше
будет переключение на религию и половые связи, чтобы все стали одинаковыми,
правильными, понятными, сидели на диване и кивали.

прогрессивный мир давно понял: только в разности возможно развитие. но,
несмотря на умниц в науке, культуре и бизнесе, больше прочих слышны вот эти,
которым нужно залезать в личную жизнь других, так как маленькие умы сковывает
ужас, когда другие позволяют себе жить иначе, быть собой и не соответствовать
их представлениям.

upd 3. для всех, кто спешит обижаться, не вчитываясь: \verb|___|удаляйте меня
смело, мне не стыдно за свою позицию, я за разность и диалог, и против травли.
я за то, чтобы говорить в быту (в б-ы-т-у - это так написано, в персональной,
значит, жизни, быт, то есть, личная жизнь, дом, семья, друзья, родильное
отделение, не на работе значит, понимаете?) на своем языке, молиться своему
богу. и не позволю никому решать такие вопросы вместо меня, и буду отстаивать
это право для других тоже, пока еще не все уехали прочь подальше от травли\verb|___|

‘стало неуютно, значит, мы на пороге новых открытий’, — такая вспомнилась
цитата

\ii{20_07_2021.fb.dragina_miriam.1.mova_jazyk_boris_hersonskij_odessa.cmt}
