% vim: keymap=russian-jcukenwin
%%beginhead 
 
%%file 28_02_2023.fb.fb_group.mariupol.pre_war.1.bulvar_bogdana_khmel.cmt
%%parent 28_02_2023.fb.fb_group.mariupol.pre_war.1.bulvar_bogdana_khmel
 
%%url 
 
%%author_id 
%%date 
 
%%tags 
%%title 
 
%%endhead 

\qqSecCmt

\iusr{Natty Burt}

Никто не знает - дом 31 по бульвару уцелел или снесли?

\begin{itemize} % {
\iusr{Ira Iryna}
\textbf{Наталья Буртовая} ремонт.

\iusr{Алексей Ковалёв}
\textbf{Наталья Буртовая} он был на против воинки? была информация, что снесли😭

\iusr{Shvedkina Oksana}
\textbf{Natty Burt} снесли

\iusr{Светлана Кокташ}
\textbf{Natty Burt} \url{https://t.me/c/1461691528/21880}

\iusr{Светлана Кокташ}
\textbf{Natty Burt}, ці фото від 25 лютого, будинок стоїть, будуть ремонтувати
\end{itemize} % }

\iusr{Valeriya Conacher}

Такие родные улицы, сколько километров пройдено по ним...

\iusr{Виктория Ветрова}

Щиро дякую за фото. Годували голубів з онуком, на цьому бульварі він (2014 рік
народження) зробив свої перші кроки. А зараз ми всі зникаємо потроху...

\iusr{Irina Greinert}

Такой уютный бульвар был! Я по нему каждый день в 90-е годы с собакой гуляла. Спасибо за фото.

\iusr{Наталья Цыбалова}

Чудові та душевні фото

\iusr{Светлана Кокташ}

Мій рідний бульвар!!!😥💔

\iusr{Luda Kucher}

Мой любимый бульвар, сколько там было роскошных дубов, нигде столько не было,
интересно живы ли они?

\iusr{Таня Минжуренкова}

Любимый бульвар. Я там работала.

\iusr{Ольга Шевченко}

Вся жизнь здесь прошла. Столько фото детских на нем было снято! Ничего не осталось. Спасибо, Олег, за свидание со счастливой жизнью

\iusr{Моника Белуччи}

Здесь выросли мои дети, внук. теперь пустырь, боль не утихает..

\iusr{Екатерина Силаева}

Жили себе люди, ростили своих детей и внуков, а главное никого не просили их освобождать.

\iusr{Leonid Ehdelshteyn}

У 2013 році \enquote{за часів януковича} розпивали з моїм колишнім колегою по роботі
Юрою, на бульварі, на лавці, Амаретто, яке було привезено мною з Німеччини. За
нами стежила машина ПМГ. Ми ховалися вiд неї в кущах на бульварі.

Мінливий як капрiз, струмує денний бриз,

У Амаретто смак терпкий і запах літа...

Юри вже немає, я є.

Бульвара теж немає? Чи таки вiн є?

\begin{itemize} % {
\iusr{Olga Chentsova}
\textbf{Леонид Эдельштейн} можно сказать, что его нет. Практически все дома от пл. Комсомола и до Апатова снесли😡😭 и от бывшего памятника Апатову единичные дома остались
\end{itemize} % }

\iusr{Анна Котула}

19 февраля 2022 года. Последняя прогулка по бульвару. Погода замечательная,
дети ели булки и кормили хлебом голубей. Мы ходили проведать папочку в этот
день.

\ifcmt
  igc https://scontent-frt3-2.xx.fbcdn.net/v/t39.30808-6/333536737_750603263041324_8590688801107655599_n.jpg?_nc_cat=108&ccb=1-7&_nc_sid=dbeb18&_nc_ohc=CrvZWjWGW8IAX8jG-G-&_nc_ht=scontent-frt3-2.xx&oh=00_AfDq6Ne6JOhxKHB2aiL9WYnkMRUSWntbh3WSy8FPGAVJrw&oe=64296842
	@width 0.4
\fi

\begin{itemize} % {
\iusr{Татьяна Рахмаил}
\textbf{Анна Котула} улица моего детства так называется ❤️
\end{itemize} % }

\iusr{Светлана Хотеенко}

Спасибо большое за фото!

\iusr{Lena Ferrera}

Родной бульвар💔💔💔 по нему ходили каждое утро в 3ю школу в 70х, а мой сын в
9ую. Помню каждую скамейку, каждое дерево на нашем бульваре...

Как учились с подругой кататься на велосипедах за 3м домом- там, где потом
упала бомба, как пахли акации за девятиэтажкой, которой нет теперь...

\iusr{Светлана Шутова}

Всі ще живі!
