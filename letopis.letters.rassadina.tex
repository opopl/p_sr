% vim: keymap=russian-jcukenwin
%%beginhead 
 
%%file letters.rassadina
%%parent letters
 
%%url 
 
%%author_id 
%%date 
 
%%tags 
%%title 
 
%%endhead 

%11:14:32 31-03-23
Ви просто неймовірна Людина!! Дуже дякую за пост!! До речі, я
бачив там зверху дискусію щодо запису та публікації. Це дійсно необхідно
робити, і я цим займаюсь зі свого боку. Я програміст. живу в Києві. Займаюсь
систематизацією, записом та архівацією різноманітних публікацій в фб та взагалі
в інеті. У мене є проект Літопису Війни. Щодо збереження та публікацій - у мене
є власний метод (заснований на системі підготовки публікацій LaTeX - це те, що
використовують науковці як де факто стандарт публікацій в світі фізики та
математики та інших наук) та власне програмне забезпечення, яке я використовую.
Як все це виглядає, можете подивитись у мене на сторінці, я якраз виставив
архівовану копію цього поста (там в біо фб-сторінки також є посилання на мій
телеграм канал та акаунт в Інтернет-Архіві із збереженими постами). Щодо
книжки. Із технічної сторони, я вже майже рік тому зробив книжку на 200
сторінок про Харків під час війни (посилання є в телеграм каналі, в самому
початку, як прогорнете назад). Якщо що, не проблема зробити книжку з усіх ваших
постів + оформити згідно побажанням. Я цим всім (літопис, запис постів і т.д.)
займаюсь вже роки два, і вже добрячи набив на цьому руку. На запис Вашого поста
у мене вчора вийшло дві години (разом із усім, повний запис всього)...
Сподіваюсь, Вам це буде цікаво, і я зможу Вам чимось в цій архіважливій справі
допомогти!

%11:28:17 31-03-23
добре ) так, пишіть! це дуже важливо, щоби світ дізнався правду!

%16:13:36 10-04-23
дуже круто, що будете купляти!! я теж дуже люблю велосипед)) щодо
характеристик. Це залежить від того, як ви його збираєтесь використовувати, ну
тобто, характеристики, залежно від того, чи ви збираєтесь ганяти з гірки? чи ви
просто хочете їздити по місту, чи по лісу? чи збираєтесь щось на ньому возити?
в який сезон? якщо взимку - то дійсно потрібні покришки з шипами, особливо в
Швеції з довгими зимами напевне (я на таких катаюсь взимку в Києві). Потім, є
два види передачі - (1) гірска - на фото (2) планетарна. 

%21:04:09 10-04-23
чудово!! я теж дуже люблю кататись ))) хоча останнім часом закинув... щодо
порад... міг би багато написати. Але! Є ще кращий варіант, як мені здається.
Нещодавно я познайомився тут на фб з неймовірною Maryna Bludsha яка до речі
свого часу в 2020 році зробила просто офігенний сайт про Маріуполь, наскільки
можна судити з того, що збереглось. На жаль, він зараз недоступний, але можливо
свого часу оживе, бо ж всі матеріали зберіглись, наскільки я зрозумів. Але тут
я не про це. Марина, наскільки я бачу, пересувається по Києву усюди на
велосипеді, активно займається пропагуванням велосипедів в Києві (яке на жаль
поки що доволі мало ще пристосоване для велосипедистів, знаю по власному
досвіду). Так от, спитайте в неї, можливо Марина вам все розскаже і
порекомендує в тисячу раз краще за мене. Крім того, вона веде щоденник у себе
на сторінці, дуже гарно і цікаво пише про усілякі різні речі!

10:44:30 23-04-23
https://www.facebook.com/ira.rasadina/posts/pfbid0c5ivBQqRks1tPKzkzzcqV63BLiJoycfnGsiWFojaSXpbvgomodmCddNq5ap3tPQal
оооооооо, шкода, що у вас там в Швеції зараз так, вже ж весна золота! У Києві
вчора був взагалі неймовірний день!
