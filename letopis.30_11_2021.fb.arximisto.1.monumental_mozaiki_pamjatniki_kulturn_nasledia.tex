%%beginhead 
 
%%file 30_11_2021.fb.arximisto.1.monumental_mozaiki_pamjatniki_kulturn_nasledia
%%parent 30_11_2021
 
%%url https://www.facebook.com/arximisto/posts/pfbid02M5js4qBpkMud9Ry2HtXVSm5PKACmbw1wNs84ADN4in5zv9KTnUPpcbwmS8CUZjpSl
 
%%author_id arximisto
%%date 30_11_2021
 
%%tags 
%%title Впервые монументальные мозаики Мариуполя будут признаны памятниками культурного наследия
 
%%endhead 

\subsection{Впервые монументальные мозаики Мариуполя будут признаны памятниками культурного наследия}
\label{sec:30_11_2021.fb.arximisto.1.monumental_mozaiki_pamjatniki_kulturn_nasledia}

\Purl{https://www.facebook.com/arximisto/posts/pfbid02M5js4qBpkMud9Ry2HtXVSm5PKACmbw1wNs84ADN4in5zv9KTnUPpcbwmS8CUZjpSl}
\ifcmt
 author_begin
   author_id arximisto
 author_end
\fi

Впервые монументальные мозаики Мариуполя будут признаны памятниками культурного наследия

\#новости\_архи\_города

Департамент культурно-общественного развития Мариупольской мэрии инициировал
предоставление статуса памяток культуры для двух советских монументальных
мозаик - \enquote{Юность} и \enquote{Покорители космоса}. К концу декабря для них будут
разработаны учетная документация и проекты зон охраны, по данным системы
госзакупок ProZorro. После чего будет подана заявка в управление культуры
Донецкой областной госадминистрации.

Мозаика \enquote{Покорители космоса} является первой работой в Мариуполе Виктора
Арнаутова (1896-1979), всемирно известного ху\hyp{}дожника-монументалиста. Ученик
знаменитого Диего Риверы, в 1930-х он создал в Сан Франциско и Калифорнии более
десятка фресок и муралов. В 1963 он вернулся из США в родной город (тогдашний
Жданов).

Мозаику на Доме связи он создал вместе с художником Григорием Пришедько в 1964
году, по данным исследователей мариупольских мозаик Александра Чернова и
Станислава Иванова.

Мозаичное панно \enquote{Юность} (1971 г.) - результат работы мариупольцев Леля
Кузьминкова (1925-2012) и Валентина Константинова (1923-2012), первопроходцев
монументальной живописи в Мариуполе. Они создали более десятка мозаик в городе
и его окрестностях. Будучи приазовскими греками, они вдохновлялись традициями
византийской живописи.

Всего в Мариуполе насчитывается более двадцати советских монументальных мозаик.
Среди них – мозаики Аллы Горской, выдающейся художницы-диссидентки 1960-х и
продолжательницы идей \enquote{бойчукистов}, киевлян Валерия Ламаха, Эрнста Коткова,
Ивана Литовченко (см. описание и местонахождение мозаик на карте \enquote{Цікавий
Маріуполь} \url{https://cutt.ly/AT5gXwr}). 
%\footnote{\url{https://www.google.com/maps/d/u/0/viewer?mid=116g22D5LAoGthaVxUXaUU8U_47i3KPyC&ll=47.16456997672142%2C37.52782215849429&z=12}}

\ifcmt
  %ig https://i2.paste.pics/PKCT0.png?trs=1142e84a8812893e619f828af22a1d084584f26ffb97dd2bb11c85495ee994c5
	ig https://i2.paste.pics/PKCUA.png?trs=1142e84a8812893e619f828af22a1d084584f26ffb97dd2bb11c85495ee994c5
	@caption Карта Цікавий Маріуполь
  @wrap center
  @width 0.9
\fi

Проекты учетной документации и зон охраны будут разработаны краматорским ООО
\enquote{Историко-архитектурное наследие}, победителем тендера Департамента. Общая
стоимость его услуг – 34 381,87 грн. Осенью 2020 года эта организация
разработала подобные документы для синагоги, дома В. Нильсена и казенного
винного склада рубежа XIX-XX вв. (см. \url{https://cutt.ly/GT5lNeI}
\footnote{\url{https://www.facebook.com/arximisto/posts/699511190747010}}).

Инициатива Департамента – это долгожданный прорыв в сфере охраны культурного
наследия Мариуполя, по мнению Андрея Марусова, директора ГО \enquote{Архи-Город}. В
нашем городе сохранилась одна из самых больших коллекций монументальной
живописи XX в. в Украине. На сегодня ни одна мариупольская мозаика не имеет
статуса памятки, что не позволяет взять их под защиту государства и планировать
реставрацию. 

Это также прецедент для всей страны, поскольку советское монументальное
искусство до сих пор не получило должного признания.

=================

Источники информации 

Звіт про результати проведення процедури закупівлі послуг з розробки облікової
документації та проектів зон охорони на об'єкти культурної спадщини м.
Маріуполя від 6 жовтня 2021 р. \url{https://cutt.ly/mt72o7j}

Договір № 07-21 / 89 від 05.10.21 про надання послуг між Департаментом
культурно-громадського розвитку Маріупольської міської ради та ТОВ
Науково-дослідний проектний центр \enquote{Історико-архітектурна спадщина}
\url{https://cutt.ly/yt79kua}
\footnote{Internet Archive: \url{https://archive.org/details/dogovir.05_10_2021.mariupol.no.07_21_89.mozaiki.budynok_z_levamy}}
