% vim: keymap=russian-jcukenwin
%%beginhead 
 
%%file 31_07_2019.stz.news.ua.mrpl_city.1.andrij_gorbulja_robyty_te_scho_ljubysh
%%parent 31_07_2019
 
%%url https://mrpl.city/blogs/view/andrij-gorbulya-robiti-te-shho-lyubish-i-lyubiti-te-shho-robish
 
%%author_id demidko_olga.mariupol,news.ua.mrpl_city
%%date 
 
%%tags 
%%title Андрій Горбуля: "Робити те, що любиш і любити те, що робиш"
 
%%endhead 
 
\subsection{Андрій Горбуля: \enquote{Робити те, що любиш і любити те, що робиш}}
\label{sec:31_07_2019.stz.news.ua.mrpl_city.1.andrij_gorbulja_robyty_te_scho_ljubysh}
 
\Purl{https://mrpl.city/blogs/view/andrij-gorbulya-robiti-te-shho-lyubish-i-lyubiti-te-shho-robish}
\ifcmt
 author_begin
   author_id demidko_olga.mariupol,news.ua.mrpl_city
 author_end
\fi

\ii{31_07_2019.stz.news.ua.mrpl_city.1.andrij_gorbulja_robyty_te_scho_ljubysh.pic.1}

Завдяки його тренуванням багато маріупольців змогли повірити в себе, зміцнити
власне здоров'я та розкрити глибинний потенціал організму. Незважаючи на багато
труднощів він продовжує надихати тисячі, при цьому дуже рідко користується
інтернетом і не женеться за популярністю. Вчинки \textbf{Андрія Дмитровича Горбулі} –
засновника і керівника маріупольського народного театру військових мистецтв
\textbf{\enquote{Білий лотос}} та міської школи \textbf{\enquote{Хонг-За В'єтнам}} – говорять за нього. У
вихованців нашого героя вже багато перемог, оскільки заняття східними
єдиноборствами дають практикуючим не тільки фізичну, а й духовну силу. Андрій
Горбуля, який більше 20 років займається різними бойовими практиками, зміг
об'єднати воєдино бойові мистецтва ніндзя, українських козаків, ушу, карате та
рукопашного бою і вибрав з них найкраще. Сьогодні театр військових мистецтв
\enquote{Білий лотос} показує унікальні шоу-програми, що не мають аналогів в Україні.

Андрій Дмитрович народився в Маріуполі, який любить змалечку. Батьки працювали
на заводі ім. Ілліча. Зі школи захоплюється єдиноборствами. Чотири роки займався
гімнастикою, 9 років – боксом у тренера – \textbf{Дворцова Володимира Леонідовича},
серед вихованців якого – призери та переможці змагань різного рівня, зокрема
олімпійські чемпіони та призери. Андрій завжди вважав Володимира Леонідовича
тренером від Бога. Загалом заняття боксом з досвідченим тренером зробили Андрія
кандидатом у майстри спорту.

\textbf{Читайте також:} \emph{Спортивный дайджест июля: мариупольские спортсмены блестяще выступили на мировых аренах}%
\footnote{Спортивный дайджест июля: мариупольские спортсмены блестяще выступили на мировых аренах, %
Георгий Федоренко, mrpl.city, 31.07.2019, \par%
\url{https://mrpl.city/blogs/view/sportivnyj-dajdzhest-iyulya-mariupolskie-sportsmeny-blestyashhe-vystupili-na-mirovyh-arenah-1}
}

Юнак отримав середню технічну освіту, закінчивши індустрі\hyp{}альний технікум. В
армії був інструктором з рукопашного бою. У 1989 році, коли східні єдиноборства
перестали бути забороненими, Андрій став інструктором кунг-фу при ОСОУ
(ДОСААФ). Того ж року виїхав до Анапи, де заснував театр військових мистецтв
\textbf{\enquote{Білий лотос}}. В тиждень тренував 600 учнів. Через 4 роки повернувся до
Маріуполя, де \enquote{Білий лотос} продовжив свою діяльність. Дуже допомагав тренер
Андрія Володимир Леонідович Дворцов.

\ii{31_07_2019.stz.news.ua.mrpl_city.1.andrij_gorbulja_robyty_te_scho_ljubysh.pic.2}

Сьогодні \enquote{Білий лотос} успішно гастролює по СНД, став відомий широкій публіці
після шокуючих виступів у 2009 році в талант-шоу \textbf{\enquote{Україна має талант}} (СТБ), у
2010 році – російському аналогу програми \textbf{\enquote{Хвилини слави}} і проекті \textbf{\enquote{Диво-люди}}
(ICTV). Також брав участь в телепрограмах на ТРК Україна.

\ii{31_07_2019.stz.news.ua.mrpl_city.1.andrij_gorbulja_robyty_te_scho_ljubysh.pic.3}

Під час програми \enquote{Хвилини слави}, яка проходила в Москві, глядачі і журі
побачили не тільки ушу, а й елементи жорсткого цигуну – стародавнього
китайського мистецтва виконувати небезпечні трюки без шкоди для себе.
Наприклад, учасники не отримували опіків від вогненних факелів, лежали всією
вагою на вістрях тризуба, розбивали дерев'яну палицю головою. \enquote{Білий лотос}
увійшов в 16 кращих колективів Європи. Вихованці маріупольської школи \enquote{Хонг-За
В'єтнам} стали чемпіонами в Шостому фестивалі бойових мистецтв \textbf{\enquote{Воїн світла}},
який пройшов в Києві (2015 рік). І це при тому, що фестиваль проводився в
масштабах всієї країни, а участь в ньому брали представники 50-ти шкіл бойових
мистецтв. Там учні Андрія Дмитровича пробували багато нових елементів, причому
самі дивувалися, що виходять такі серйозні речі. Зокрема, лягали на плиту голою
потилицею, на лоб також клали плиту, зверху билося все кувалдою, після чого
відразу розбивалися дві плити, – такий серйозний трюк. У вихованців
маріупольської школи \enquote{Хонг-За В'єтнам} голова завжди виявляється міцніша, ніж
плита. І секрет такого успіху простий. Керівник школи наголошує, що \emph{треба
практикувати, займатися і багато речей буде виходити, незалежно від віку}.

\ii{31_07_2019.stz.news.ua.mrpl_city.1.andrij_gorbulja_robyty_te_scho_ljubysh.pic.4}

У 2017 році спортсмени їздили до Франції на Фестиваль бойових мистецтв світу.
Завдяки їхній участі Україна вперше була запрошена на цей фестиваль. Але на
поїздку зібрали фінансування самостійно. Загалом підтримки іноді не вистачає,
але незважаючи на всі труднощі і вчитель, і учні не зупиняються. Тільки завдяки
команді, яку вдалося зібрати Андрію Горбулі, можливі ось такі перемоги. Багато
його вихованців сьогодні самі стали інструкторами. Часом учні досягають
результатів, до яких Андрій Дмитрович йшов все своє життя, за три-чотири роки
занять. Роблять трюки, які в світі роблять одиниці. Це, наприклад, сон на
піках, вис на тризубі. Це складні вправи з ризиком для життя. Все залежить від
наполегливості, від бажання людини, і набагато менше – від здібностей.
Засновник \enquote{Білого лотоса} не ставить собі за мету спеціально набрати здібних
учнів, для нього головне, щоб \emph{були учні віддані своїй справі, відповідальні та
дисципліновані - тоді буде результат}. Маріуполець наголошує, що абсолютно
будь-яка людина може досягти такого рівня, потрібно тільки прагнення і
позитивний настрій.

\ii{31_07_2019.stz.news.ua.mrpl_city.1.andrij_gorbulja_robyty_te_scho_ljubysh.pic.5}

Водночас це не просто тренування, а ціла філософія і особливий спосіб життя.
При цьому він зазначає, що \emph{все життя потрібно бути учнем, а не вчителем,
тому намагається щодня продовжувати навчатися}. На думку нашого героя, люди
повинні об'єдну\hyp{}ватися. Учні \enquote{Білого лотоса} вже давно стали однією
сім'єю. Вони проходять систематичні тренування 2-3 рази на тиждень. Однак в
день може бути 4 тренування. До речі, спортивну залу Андрій Горбуля побудував
сам. Вихованці мають вегетаріанське харчування. Оскільки з 21:00 години вечора
до 1:00 відбувається відновлення гормональної системи, лягати спати
рекомендується о 21:00. Підйом о 4:30 - 5:00 годині ранку. Необхідно завжди
підтримувати здоровий спосіб життя і мати чисті думки: не заздрити, не бажати
зла, позбавлятися від гордині. Воду пити тільки джерельну. Вік учнів
необмежений.  Наймолодші – 3 роки, найстарші – 80. Хоча приходила й бабуся 93
років, яка займалася пів року. З дітьми найголовніше, на думку тренера -
розвинути увагу і дисципліну.

\ii{31_07_2019.stz.news.ua.mrpl_city.1.andrij_gorbulja_robyty_te_scho_ljubysh.pic.6}

Школа \enquote{Хонг-За В'єтнам} спрямована на зміцнення і відновлення здоров'я,
дисциплінування тіла і духу. Займаючись по 8 років, багатьом вдалося відновити
своє здоров'я. Андрій Дмитрович наголошує, що ходьба є основною профілактикою
здоров'я. В день треба проходити близько 7 км – це 1,5 години безперервного
руху. Загалом він лікує людей за допомогою екзотичної гімнастики. Це новий
напрям в Україні, який будується на оздоровленні людей. За допомогою вправ
лікуються остеохондрози, артрити, багато внутрішніх хвороб. Хоча, звичайно,
краще її використовувати як профілактику. Школа благодійна, хто як може, так і
платить, не може – не платить взагалі. Діяльність Андрія дуже підтримує сім'я.
До речі, 2 його онуки теж стали вихованцями \enquote{Білого лотоса}. Загалом родина
творча, адже дружина Андрія Дмитровича – артистка цирку, у неї свій колектив.

\ii{31_07_2019.stz.news.ua.mrpl_city.1.andrij_gorbulja_robyty_te_scho_ljubysh.pic.7}

Наразі всі сили режисера народного театру \enquote{Білий лотос} та його учнів
спрямовані на будівництво храму військових мистецтв, який розташований на 23
мікрорайоні. Всі небайдужі і активні маріупольці можуть долучитися до цієї
благородної справи.

Андрій Дмитрович говорить, що його \emph{покликання – допомагати людям, надихати їх
власним прикладом на відновлення здоров'я і розкриття внутрішнього потенціалу.} 

Надихає любов до людей і учнів.

\ii{insert.read_also.demidko.shevchenko_chas_dlja_sebe}

\textbf{Улюблена книга:} \enquote{Галерні раби} Пульвера Юрія.

\textbf{Улюблений фільм:} \enquote{Пірати XX століття} (1979 р.)

\textbf{Порада маріупольцям:} \enquote{Вести здоровий спосіб життя і розвиватися духовно, адже сьогодні духовність, на жаль, втрачена}.

\ii{31_07_2019.stz.news.ua.mrpl_city.1.andrij_gorbulja_robyty_te_scho_ljubysh.pic.8_9}

\emph{Фото з архіву Андрія Горбулі, 2 останні світлини Євгена Сосновського}
