% vim: keymap=russian-jcukenwin
%%beginhead 
 
%%file 11_01_2022.fb.fb_group.story_kiev_ua.2.stancia_metro_universitet.cmt
%%parent 11_01_2022.fb.fb_group.story_kiev_ua.2.stancia_metro_universitet
 
%%url 
 
%%author_id 
%%date 
 
%%tags 
%%title 
 
%%endhead 
\zzSecCmt

\begin{itemize} % {
\iusr{Татьяна Вакуленко}
Квест! @igg{fbicon.hands.applause.yellow}{repeat=3} 

\iusr{Александр Мурашов}
Максим Горький ещё уцелел. Круто! Это покруче палеолитических раскопок на месте бывшего моря

\begin{itemize} % {
\iusr{Ольга Кирьянцева}
\textbf{Александр Мурашов} тоже об этом подумала)

\iusr{Татьяна Гордиенко}
\textbf{Александр Мурашов} , палеонтологических. Палеолит -это к археологам

\iusr{Александр Мурашов}
\textbf{Татьяна Гордиенко} зато красиво, поэтично и короче писать @igg{fbicon.heart.eyes} 

\iusr{Татьяна Гордиенко}
\textbf{Александр Мурашов} , это разные отрасли знаний - геология и археология.
Писать подобное - проявоять невежество. Впрочем, может вам это нравится

\iusr{Александр Мурашов}
\textbf{Татьяна Гордиенко} я специально так написал, так прикольнее) Вы же отреагировали=значит уже хорошо

\iusr{Татьяна Гордиенко}
\textbf{Александр Мурашов} , глупо

\iusr{Александр Мурашов}
\textbf{Татьяна Гордиенко} спсибо. ваше мнение важно и будет учтено
\end{itemize} % }

\iusr{Ольга Кирьянцева}

Интересно как. А я ещё в детском саду в свежем привезенном морском песке нашла
нечто, - круглое, сантиметров 7-8 в диаметре, состоящее из двух пластин,
соединенных перемычками и на верхней пластине как будто выдавленное изображение
наподобие цветка с пятью лепестками... До сих пор нигде не встречала ничего,
напоминающего ту мою находку. Очень интересно, что бы это могло быть...

\begin{itemize} % {
\iusr{Владимир Аксенов}
\textbf{Olga Kiryantseva} Это могло бы быть, что если бы эту находку сохранили, то сегодня были бы миллионершой и не в чем бы себе не отказывали.

\iusr{Ольга Кирьянцева}
\textbf{Владимир Аксенов} сама так думаю  @igg{fbicon.face.grinning.big.eyes} 
\end{itemize} % }

\iusr{Елена Сидоренко}
Интересно,спасибо. @igg{fbicon.heart.beating} 

\iusr{Тамара Игнатова}

\ifcmt
  ig https://scontent-frx5-2.xx.fbcdn.net/v/t39.1997-6/s168x128/93027172_222645632401274_7176243611145601024_n.png?_nc_cat=1&ccb=1-5&_nc_sid=ac3552&_nc_ohc=ReP062CEyJEAX-liui4&_nc_ht=scontent-frx5-2.xx&oh=00_AT-IOIlVoNgzXkah58GTmjkGWrzuCvab6koxemhq-HwHYA&oe=61E1D2F0
  @width 0.1
\fi

\iusr{Ирина Иванченко}
До Пешкова не добрались пешком \enquote{ревизионисты} ...Наверное, им всё - таки непросто достать его \enquote{На дне}...

\iusr{Vadim Basovskiy}

Чуть-чуть подправлю автора. Не мог он опустить пятачок на следующий день после
открытия метрополитена. Метро было открыто для регулярной работы 12-го ноября
1960-го года. То есть оплата в 5 копеек началась с денежной реформы 1961-го
года. До этого проезд стоил 50 копеек. Пассажир покупал в кассе бумажный
билетик и предъявлял его контролёру. Автоматы, принимавшие пятачки. появились
почти через год после открытия метро.

\begin{itemize} % {
\iusr{Александр Венге}
\textbf{Vadim Basovskiy} благодарю. Думаю, что Вы правы. Мал был и что-то путается. Будем считать историю про пятачок метафорой ))).

\iusr{Владимир Новицкий}
\textbf{Александр Венге} 

Александр, видимо у Вас, как и у меня ассоциация с орпусканием пятачка связана
с тем, что зайдя на станцию, как правильно заметил Вадим по бумажному билетику
- некоторые первые пасажиры на радостях попав впервые на экскалатор бросали
монетки, которые летели вниз и ударялись в стекляную будку за которой стояла
дежурная экскалатора. Таких развлекающихся быстро угомонила милиция. Я был
свидетелем этого первого спуска. Спасибо за интересный рассказ.


\iusr{Александр Венге}
\textbf{Vladimir Novitsky} у меня что-то с чем-то перемешалось в памяти. Мать все время рассказывала, как мы ходили на следующий день после открытия. Я только помню отчётливо праздничное настроение.

\iusr{Nina Steseva}
\textbf{Vadim Basovskiy} вы абсолютно правы, продавались талоны по 50 копееек 1947 года, а после реформы 1961 года - пятачок!
\end{itemize} % }

\iusr{Igor Plaschkin}
А откуда в Киев везли мрамор?

\begin{itemize} % {
\iusr{Александр Венге}
\textbf{Igor Plaschkin} нечаянно отправил в печать черновик. Из Закарпатья.

\iusr{Igor Plaschkin}
\textbf{Александр Венге} А где там мрамор добывают?

\iusr{Александр Венге}
\textbf{Igor Plaschkin} Вчера прочел, что доставляли с Новоселицкого карьера.

\iusr{Igor Plaschkin}
\textbf{Александр Венге} спасибо!!!
\end{itemize} % }

\iusr{Александр Николаевич}

Однажды появилась Украина и Закарпатье, потом Киев... Ты серьёзно? Ближайший
мрамор на территории Украины, в Крыму. Бред!

\begin{itemize} % {
\iusr{Татьяна Гордиенко}
\textbf{Александр Николаевич} , мрамора в Деловом ещё добывают. Автор видимо великий поклонник популярных малограмотных статеек и Википедии

\iusr{Yuri Ginsburg}
\textbf{Александр Николаевич} В Крыму не мрамор, а мраморовидные известняки. И от Киева до Рахова напару сот километров ближе чем до Гурзуфа.
\end{itemize} % }

\iusr{Татьяна Гордиенко}

По поводу метаморфизма несколько нафантазировано. Мрамор это
перекристаллизованный известняк, процесс перекристаллизации происходит под
воздействием температуры и давления, так что \enquote{движения земной коры} тут не
виноваты

\begin{itemize} % {
\iusr{Александр Венге}
\textbf{Татьяна Гордиенко} насколько я понял из прочитанного, изменение температуры и давления как раз и происходит от движения земной коры. Что касается метаморфизма, то на него ссылка во всех статьях, которые попались мне на глаза.

\iusr{Татьяна Гордиенко}
\textbf{Александр Венге} , рекомендую почитать что-то кроме Википедии. Например, вузовский учебник по петрографии

\iusr{Ольга Кирьянцева}
\textbf{Татьяна Гордиенко} а для чего нам, собственно, вузовские знания по какой-то петрографии? Мы же не на лекции научной. Александр очень доступно и интересно поведал нам о природных явлениях-преобразованиях и этого, мне кажется, вполне достаточно для такой аудитории, как КИ. ☺ @igg{fbicon.wink} 

\iusr{Игорь Сирадчук}
\textbf{Ольга Кирьянцева} А куди ж подіти перфекціонізм? Все має бути викладено рівненько до буковки і до камінчика @igg{fbicon.laugh.rolling.floor} .
\end{itemize} % }

\iusr{Максим Анатолійович Чекановський}
Вы молодец.

\iusr{Максим Анатолійович Чекановський}
Настоящий Киевлянин.

\iusr{Владимир Картавенко}

Викентий Хвойка решил полностью не открывать Кирилловскую стоянку первобытного
человека на территории Киева. Это место напротив Фармака где была таможня
ГОРЫНЫЧА. Фото гавани Почайна 1925-30 годов. Скоро и здесь пройдет метро.

\ifcmt
  ig https://scontent-frx5-2.xx.fbcdn.net/v/t39.30808-6/271719397_6807438412660731_1715083780740625732_n.jpg?_nc_cat=109&ccb=1-5&_nc_sid=dbeb18&_nc_ohc=QkpfYS36C0gAX9WbOIh&_nc_ht=scontent-frx5-2.xx&oh=00_AT9VNiSCMIsLCASetpuLS4OFHxvtKkUziilxbpClkvHeyA&oe=61E2E9C2
  @width 0.3
\fi

\iusr{Iryna Petrotchenko}
\textbf{Владимир Картавенко} чому він вирішив не відкривать?

\iusr{Вячеслав Мусиенко}

Меня также мама повела на станцию Университет сразу после открытия и проехали
одну остановку. Очень красиво и ярко. Впереди было светлое будуще...


\iusr{Тамара Ар}

Самая любимая станция с детства, но, даже не знала столько
интересного,,,,,,,,,и что удивительно, бюст уцелел,,,,,,,, наверное, не все
знают,,,,,,,,читали,,,,


\iusr{Тамара Ар}
Спасибо, мне все понравилось, потому что интересно написано

\iusr{Тетяна Слись}
В студенческие времена однокурсник показал мне этот срез. С тех пор, бывая на
Университете, здороваясь со знакомым аммонитом.

\begin{itemize} % {
\iusr{Александр Венге}
\textbf{Тетяна Слись} их там много

\iusr{Тетяна Слись}
Согласитесь, большой один.
\end{itemize} % }

\iusr{Elena Kravchenko}
С детства знала где она и именно там становилась в ожидании поезда ( вправо пройти и смотреть внимательно вниз)

\iusr{Татьяна Третьяченко}

«Університет» - моя улюблена і найближча від дому станція метро! Амоніт під
бюстом Пєшкова - не єдиний. Ми колись з дітьми обстежували і роздивлялися
мармур, але я вже не пам‘ятаю кількість цілих розтинів.


\iusr{Старр Диггер}
Так а снимок где??

\iusr{Александр Венге}
\textbf{Edward Popovitch} извините, Эдвард. Издержки неопытности. Вернул.

\iusr{Микола Гончаренко}

Хлопці ви помиляєтеся. То не мрамор а шліфований граніт. Мрамор (мармур) білого
кольору. Буває ще рожевий

\begin{itemize} % {
\iusr{Александр Венге}
\textbf{Mykola Goncharenko} именно мрамор. Вы правы, что исходный цвет белый. Но потом нередко добавляются природные красители.

\iusr{Игорь Панченко}
\textbf{Микола Гончаренко} Гранит формируется из расплавленных пород (кстати, столешницы кухонь из гранита радиоактивно "фонят"), соответственно температура его была очень высока. Не думаю, что при такой температуре в породе сохранились бы ископаемые остатки, которые можно наблюдать на ст. м. Университет
\end{itemize} % }

\iusr{Оксана Зубец}
Я тоже фотографий не вижу

\begin{itemize} % {
\iusr{Марія Макарова}
\textbf{Оксана Зубець} і я

\iusr{Александр Венге}
\textbf{Оксана Зубець} извините за издержки, Оксана. Фотографии вернул.

\iusr{Оксана Зубец}
\textbf{Александр Венге} ух ты! Никогда даже не догадывалась! Сегодня специально выйду поищу! Спасибо за такую интересную информацию!

\iusr{Александр Венге}
\textbf{Оксана Зубець} очень рад, что Вам понравилось.
\end{itemize} % }

\iusr{Marina Kolomaiko}
А фото ???

\begin{itemize} % {
\iusr{Александр Венге}
\textbf{Marina Kolomaiko} прошу извинить, Марина. Вернул.

\iusr{Marina Kolomaiko}
\textbf{Александр Венге} благодарю за фото и статью!!!!
\end{itemize} % }

\iusr{Светлана Александренко}
Грандиозно! Моя любимая станция метро.

\iusr{Наталья Лаврухина}
Спасибо, интересно, и еще @igg{fbicon.face.smiling.eyes.smiling} интересней уточнения, возражения...... комменты...

\iusr{Світлана Наговіцина}

Оздоблення з каменю, скільки існує людство. Коли ще не було бетону, цегли..
Пещери слугували їхніми природними домівками. Але з часом, вони вже майстрували
фортеці, піраміди, форти,палаци. І звичайно - оздоблення в метро


\iusr{Iryna Naidonova}
Мне этот аммонит показал будущий муж!

\iusr{Вадим Сандино}
Уверен, что фантазия у ЦК КПУ дальше Горького На дне и отложений морских не пошла!

\iusr{Elena Marijchuk}
Там их много, но ОБ ЭТОМ УЖЕ ПИСАЛИ.

\iusr{Дмитрий Мурин}
след аммонита - это спирально закрученный узор?

\begin{itemize} % {
\iusr{Александр Венге}
\textbf{Дмитро Мурін} да

\iusr{Viktor Kozhevnikov}
Єто не след, єто сам аммонит в разрезе
\end{itemize} % }

\iusr{Katrusya Golovash}
Любимая станция!

\iusr{Анна Сидоренко}
Спасибо, интересная история, а пятачки или талончики то уже мелочи...

\iusr{Sergyi Grabar}
Дякую, пане Олександре! Дуже цікаво і гарно написано.

\iusr{Александр Венге}
\textbf{Sergyi Grabar} Дякую.

\iusr{Evgeny Novoselov}
Good observation

\iusr{Александр Венге}
\textbf{Evgeny Novoselov} дякую

\iusr{Инна Валентиновна}

Спасибо, кратко и трогательно! Моя любимая станция. Сколько с ней связано,
сколько свиданий прошло на ней. Да, прошло....


\iusr{Татьяна Щербинина}
Станция метро \enquote{Университет} не самая пустынная. Она весьма активна и многолюдна. А вот \enquote{Кловская} да, пустынна.

\iusr{Nataliia Baranova}
Не менее интересный аммонит можно было разглядеть на полу в Больнице спортивной медициь на Павловской. К счастью давно там не была

\iusr{Тарас Єрмашов}
Чітко видимих амонітів там три (і ще з десяток відбитків інших організмів).

\iusr{Boris Shulman}
Самый большой аммонит - в Мюнстере.

\iusr{Ковальская Татьяна}
Самая чудесная станция метро Золотые ворота, как для меня, а вообще наше метро лучшее в мире, чистое, красивое, удобное!

\begin{itemize} % {
\iusr{Al Pol}
\textbf{Ковальская Татьяна} 

Да уж ... Если киевское меро сравнить с метро в Париже или Милане, то это
примерно как дворец Украина и колхозный клуб на хуторе Козюльки.

Чистота, красота и порядок в киевском метро - это то немногое, чего не успели
уничтожить \enquote{реформаторы}.

Памятники повалили, улицы переименовали, промышленность угробили, медицину и
образование \enquote{опустили}, уровень жизни ниже, чем в Нигерии.

Пора метро \enquote{реформировать}.

\iusr{Iryna Petrotchenko}
\textbf{Al Pol} тільки зіпсували, частково, М рекламою.

\iusr{Ковальская Татьяна}
\textbf{Al Pol} Наша тема розмови, яка станція Метро Києва вам подобаєтся, але не те що ви до цього додали! 28 років замовчуванням, тепер прозріли!
\end{itemize} % }

\iusr{Мария Бойчук}
дякую!

\iusr{Иван Абрамов}
Окаменелость Горького впечатляет

\iusr{Igor Yasen}
Да, помню, \enquote{эту ракушку}. В студенческие годы каждый рабочий день двигался на метро.

\iusr{Neonila Tkachenko}

Любимое занятие в детстве - в ожидании поезда искать и находить \enquote{ракушки} на
станции метро \enquote{Университет} )) Эта на фото - самая большая и самая легко
находимая)


\iusr{Наталя Кудря}
і найкрасивіша станція...

\iusr{Мария Константиновская}
А где фотографии??? @igg{fbicon.shrug} 

\begin{itemize} % {
\iusr{Александр Венге}
\textbf{Мария Константиновская} Вы не видите фотографий?

\iusr{Анатолій Базір}
\textbf{Александр Венге} их нет

\iusr{Александр Венге}
\textbf{Анатолій Базір} спасибо. Мистика. У меня есть. Сейчас буду разбираться.

\iusr{Мария Константиновская}
\textbf{Александр Венге} не вижу!!!

\iusr{Александр Венге}
Анатолий, модераторы видят. 2 шт. Вы по-прежнему нет?

\iusr{Светлана Манилова}
Мария. фото в публикации есть.

\iusr{Мария Константиновская}
\textbf{Светлана Манилова} у меня нет!

\iusr{Светлана Манилова}
Мария, попробую переслать их Вам в личку.

\iusr{Светлана Манилова}
Мария, зайдите, пожалуйста, в мессенджер.
\end{itemize} % }

\iusr{Мария Константиновская}
Мои любимые: Университет и Золотые ворота!!

\iusr{Наталия Калинина}
Из моей коллекции)))

\ifcmt
  ig https://scontent-frx5-1.xx.fbcdn.net/v/t39.30808-6/271734376_4830197247023858_386940596697285139_n.jpg?_nc_cat=100&ccb=1-5&_nc_sid=dbeb18&_nc_ohc=GTEmVNvWN48AX8-xj-V&_nc_ht=scontent-frx5-1.xx&oh=00_AT_37RG-bl3iz8uPXTmuJa9uydYFx_q6YfIk6v7WQewUmA&oe=61E2C930
  @width 0.3
\fi

\iusr{Татьяна Иванова}
Wow! Amazing!

\iusr{Маринэ Рехвиашвили}

Интересно откроют ли станцию на Львовской браме и какой она будет? Говорили,
что там под землёй нашли въезд в город, времен Киевской Руси и поэтому станцию
не открывают, чтобы не рушить ворота.

\iusr{Владимир Картавенко}
\textbf{Marine Rekhviashvili} Подземный въезд был с Юрковицы

\iusr{Мария Константиновская}
Спасибо! Фото появились!!!

\iusr{Наталья Соколовская}

Дякую за дуже цікаву розповідь авторові і дописувачам !!! Я давно побачила ці
амоніти на \enquote{Університеті}. Їх дуже-дуже багато там. Є великі, менші... Коли буваю
там - завжди милуюся. Шукаю очима нові, коли чекаю потяга. Потяг приїхав, повіз,
і знову забулася ця краса до наступної зустрічі з нею.

Чомусь фото не робила. Зроблю.

Ще раз дякую авторові за чудовий екскурс.

\iusr{Александр Венге}
\textbf{Наталья Соколовская} Дякую також

\iusr{Лариса Шелест}
Как интересно! Спасибо!

\iusr{Mariya Demidik}
Супер @igg{fbicon.heart.exclamation} Спасибо

\iusr{Ольга Балаба}
Спасибо!

\iusr{Татьяна Сирота}

 @igg{fbicon.thumb.up.yellow}{repeat=3} 

\end{itemize} % }
