% vim: keymap=russian-jcukenwin
%%beginhead 
 
%%file 18_06_2020.fb.zharkih_ekaterina.1.mova_jazyk
%%parent 18_06_2020
 
%%url https://www.facebook.com/kate.zharkih.5/posts/4531897370169665
 
%%author 
%%author_id 
%%author_url 
 
%%tags 
%%title 
 
%%endhead 

\subsection{МОВА И ЯЗЫК}
\label{sec:18_06_2020.fb.zharkih_ekaterina.1.mova_jazyk}
\url{https://www.facebook.com/kate.zharkih.5/posts/4531897370169665}

Я коренная киевлянка и моя семья растила меня на русском языке. Пели
колыбельные, воспитывали, говорили добрые слова именно на нем. Думаю я на
русском. И мне не стыдно!

Вопрос языка никогда не стоял с адекватными людьми, один говорил на одном,
другой на другом. Кто хотел, переходил на язык собеседника тем самым проявляв
уважение. Заметьте, его не заставляли, он сам это выбирал.

Исторически так сложилось, что почти все киевляне русскоязычные. Как и половина
нашей Родины. Сегодня нас хотят сделать «нацменьшенством», что просто
неприемлемо!

Дискриминация по языковому принципу я вижу каждый день в магазинах, кафе, кино.
Квоты на тв, запрет на книги из России(мне интересно, как книга про кишечник
угрожает целостности страны?), плюс на русском языке книги найти сложно.

Почему нет выбора?

А иногда таблички в заведениях прям оскорбляют меня, русскоязычную украинку.
(Фото ниже, ресторан VANO IVANO на Богдана Хмельницкого)

Нам постоянно хотят навязать какой-то комплекс недогражданина, недоукраинца.
«Мы же живем в Украине, значит должны говорить на украинском». Ну бред же!

Зачем строить из себя того, кем не являешься? Зачем кривляться? Оставайтесь
собой, будьте верными себе! Не комплексуйте и не принижайте себя. Вы —
украинцы, работаете на свою страну, платите налоги, а за ваши деньги вас
унижают и запугивают. Хотят чтобы вы уехали в другую страну или вообще умерли
от кулаков якобы «патриотов», которые между собой тоже на русском говорят. Все
помнят, как «отец нации» в прокуратуре начал матюкаться на русском, где их
патриотизм?

Это клоунада направлена, чтобы у вас не было прав, чтобы вами было легко
манипулировать. Не дайте себя выгнать из своего дома!

На этой почве ссорят людей между собой, превознося одних и унижая других. Это
ненависть дошла до того, что мать-одиночка не хочет отправлять своего ребёнка в
русскоязычный класс, потому что переживет, что его побьют по дороге домой
другие школьники с параллельного класса. Они какое-то время жили в Израиле и не
было возможности изучать украинский. Ребёнок ничего в школе не понимает, за то,
жив-здоров. Понимаю эту бедную женщину.

Я не против украинского языка, отнюдь. Язык красивый и эмоциональный. Но его
нужно развивать не путём запретов на все остальное – а через \verb|#любовь|. Писать
сказки, стихи, сочинять песни – повышать культуру. А власть уже который год
только запрещает и рушит. Не защищает огромную часть страны. А \verb|#Зеленский| шёл
на пост, как русскоязычный президент.

Значит выгодно продолжать стравливать население. Значит все таки потеряем
Донбасс. Потому что не хотим слышать. Мне очень понравился пример Кузьмина на
эту тему, что это как в отношениях мужа и жены — если они перестанут говорить
между собой, обсуждать проблемы, высказывать моменты которые нравится/не
нравится, то в конечном итоге они разведутся. Разве кто-то из нас хочет развал
нашей страны?

Языковую политику нужно менять и начать с того, чтобы слышать и уважать.

\ifcmt
  pic https://scontent-amt2-1.xx.fbcdn.net/v/t1.6435-9/104489582_4531883503504385_9161784228983488042_n.jpg?_nc_cat=111&ccb=1-3&_nc_sid=8bfeb9&_nc_ohc=pziqzLEh6yoAX-pA0Td&_nc_ht=scontent-amt2-1.xx&oh=6e8af41a453106bd6d4e5a5be39f8cab&oe=608EF964
  width 0.3
  fig_env wrapfigure
\fi

\ii{18_06_2020.fb.zharkih_ekaterina.1.mova_jazyk.cmt}

