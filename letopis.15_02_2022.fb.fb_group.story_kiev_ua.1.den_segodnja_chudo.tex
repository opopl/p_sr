% vim: keymap=russian-jcukenwin
%%beginhead 
 
%%file 15_02_2022.fb.fb_group.story_kiev_ua.1.den_segodnja_chudo
%%parent 15_02_2022
 
%%url https://www.facebook.com/groups/story.kiev.ua/posts/1862010323995760
 
%%author_id fb_group.story_kiev_ua,sirota_tatjana.kiev
%%date 
 
%%tags kiev
%%title День сегодня чудо, как хорош!
 
%%endhead 
 
\subsection{День сегодня чудо, как хорош!}
\label{sec:15_02_2022.fb.fb_group.story_kiev_ua.1.den_segodnja_chudo}
 
\Purl{https://www.facebook.com/groups/story.kiev.ua/posts/1862010323995760}
\ifcmt
 author_begin
   author_id fb_group.story_kiev_ua,sirota_tatjana.kiev
 author_end
\fi

День сегодня чудо, как хорош!

Яркое солнце, которое уже согревает, синее небо, на котором нет ни единого
облачка!

И мы с младшим внуком отправляемся на Владимирскую горку, где бываем очень часто
и любим здесь гулять в любое время года.

Маршрут сюда у нас неизменный: автобус 115, что идёт с Березняков через мост
Патона и по набережной на Подол.

\raggedcolumns
\begin{multicols}{4} % {
\setlength{\parindent}{0pt}

\ii{15_02_2022.fb.fb_group.story_kiev_ua.1.den_segodnja_chudo.pic.1}

\ii{15_02_2022.fb.fb_group.story_kiev_ua.1.den_segodnja_chudo.pic.2}
\ii{15_02_2022.fb.fb_group.story_kiev_ua.1.den_segodnja_chudo.pic.2.cmt}

\ii{15_02_2022.fb.fb_group.story_kiev_ua.1.den_segodnja_chudo.pic.3}
\ii{15_02_2022.fb.fb_group.story_kiev_ua.1.den_segodnja_chudo.pic.4}

\ii{15_02_2022.fb.fb_group.story_kiev_ua.1.den_segodnja_chudo.pic.5}
\ii{15_02_2022.fb.fb_group.story_kiev_ua.1.den_segodnja_chudo.pic.5.cmt}

\end{multicols} % }

А мы выходим сразу за Пешеходным мостом и через Почтовую площадь направляемся к
фуникулёру, который поднимает нас на Владимирскую горку.

На улицах города снег уже растаял, и даже повысыхали лужи. А здесь ещё лежат
снежные сугробы, и земля покрыта тонким снежным покрывалом. И, хотя на календаре
только середина февраля, радостно щебечут птицы, и воздух наполнен  запахом
ВЕСНЫ.

\ii{15_02_2022.fb.fb_group.story_kiev_ua.1.den_segodnja_chudo.pic.6_8}

Нагулявшись, решаем идти через Стеклянный мост в парк Хрещатый и посетить Музей
Воды. Но... в Музее оказался выходной день. И мы отправились к Театру
кукол, здание которого больше напоминает замок, нежели театр, смотрится просто
сказочно и вызывает восторг не только у детей, но и у взрослых. Но вот... уже
пора возвращаться домой, готовить уроки к завтрашнему дню.

\ii{15_02_2022.fb.fb_group.story_kiev_ua.1.den_segodnja_chudo.pic.9_11}

И мы идём к станции метро \enquote{Крещатик}.

14 февраля 2022 года.
