% vim: keymap=russian-jcukenwin
%%beginhead 
 
%%file books.istoria_rusiv.abstract
%%parent books.istoria_rusiv
 
%%url 
 
%%author_id 
%%date 
 
%%tags 
%%title 
 
%%endhead 

\begin{zznagolos}
\raggedcolumns
\begin{multicols}{2} % {
\setlength{\parindent}{0pt}

It has been a long time since the Russians wanted to have \enquote{History of
Russians, or the History of Inner Russia}, a work by Georgii Koniskii.

Давно вже малоросіяни бажали бачити друкованою «Історію Русів, або Малої Росії,
твір Преосвященного ГЕОРГІЯ КОНИСЬКОГО».

Багато слів було час від часу, що той чи другий збираються видати її, навіть
друкують; але до сих пір Ті як нема, так нема! Маючи кілька списків цієї
історії, я вибрав ліпший з-поміж них, підвів до нього з інших різночитання, а
потім запропонував Імператорському Товариству Історії і Старожитностей
Російських видати його в світ, що і здійснюється нині. Час від часу я маю намір
те саме зробити і з іншими писемними джерелами Малої Росії, літописами,
записками, описами і т. ін., напр., з Шафонським, Симановським та інш.  А тому
я просив би всіх, хто тільки має і бажає бачити за короткий час ці та подібні
до них пам'ятники надрукованими, надсилати їх мені як секретареві Товариства
для зняття з них списків і негайного поміщення в «ЧТЕНІЯХЬ». Щира вдячність і
признання сучасників, які займаються історією, а також і самого потомства, буде
найкращою нагородою тим, хто відгукнеться на цей заклик. Пора вже, давно пора
не затаювати подібного скарбу і тим, оскільки можливо, полегшити вивчення і
пізнання історії Південних Русів для всіх і кожного, особливо ж самих Руських.
Добре, що є певна і скора нагода виконати цей священний обов'язок справжнього
сина свого народу і батьківщини.

О. Бодянський\footnote{%
Йо́сип (О́сип) Макси́мович Бодя́нський (псевд. О. Бода-Варвинець, Ісько
Материнка, І. Мастак та ін.) (31 жовтня (12 листопада) 1808, за іншими
даними — 3(15) листопада 1808, Варва, Лохвицький повіт, Полтавська
губернія, Російська імперія (нині Чернігівська область, Україна) — 6
(18) вересня 1877, Москва, Російська імперія) — український
філолог-славіст, історик, фольклорист, перекладач, видавець, письменник,
член-кореспондент Петербурзької Академії наук (1854) [Википедия].
}

Лютого 9, 1846, Москва

\end{multicols} % }
\end{zznagolos}

\ii{books.istoria_rusiv.abstract.pic.1}
