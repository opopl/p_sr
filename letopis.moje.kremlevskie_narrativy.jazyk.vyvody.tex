% vim: keymap=russian-jcukenwin
%%beginhead 
 
%%file moje.kremlevskie_narrativy.jazyk.vyvody
%%parent moje.kremlevskie_narrativy.jazyk
 
%%url 
 
%%author_id 
%%date 
 
%%tags 
%%title 
 
%%endhead 

\paragraph{Завершающие выводы}
\label{sec:moje.kremlevskie_narrativy.jazyk.vyvody}

А что насчет языка? Надо бы закончить как-нибудь, и перейти к другим темам. Как
говорил Гоголь, перед вами громада - русский язык! Не трогала б ты, Украина,
русский язык и русскую культуру, - не было бы войны. Не оскорбляла бы россиян,
Россию, русскую культуру тысячами комментов, и ведрами помоев, не было бы
вторжения. Не орали бы военные на параде в Киеве песенку про Путина, - Путин
был бы добрее к нам, он же ж тоже человек, и украинский язык и культуру, и
Украину, он, мы уверены, в душе любит... Еще тут можно добавить...  Не орали бы
протестующие толпами на Майдане \enquote{хто не скаче - той москаль}, - не
утопили бы мирные революционеры в Киеве дом Профсоюзов в огне, не выламывали бы
из улицы Институтской булыжники, которые эти люди не клали, - зачем было
портить то, что ты не строил? - не было бы ни отторжения Крыма, ни восстания на
Донбассе. Потому что посеешь ветер - пожнешь бурю, слово не воробей - вылетит,
не поймаешь. Семь раз отмерь - один раз отрежь. И лучше синица в руках, чем
журавль в небе. А что касается Галичины, которая хотела натянуть свой глобус на
всю Украину, -  не стоит лезть со своим уставом в чужой монастырь, если тебя об
этом не просили. Лучше вежливо постучать сначала.

И вместо того, чтобы плакать за павшими защитниками Отчизны, и за сгоревший
самый большой самолет Мрия в Гостомеле, ты сейчас бы уже строила наверное уже
десятую по счету Мрию, и все было бы хорошо и расчудесно. Нам в Киеве на
четвертый день войны, сидя за ужином и чашкой гарячего какао, это так кажется.
Никуда выходить нельзя, так чем же еще заниматься, как не размышлениями и
разговорами о том, о сем?  Может быть, завтра будет еще все намного хуже, хотя
вроде назначили переговоры в Припяти, и неизвестно, сколько нам еще отпущено
жить, но там, где мы живем, сегодня было тихо. И война-войной, а ужин по
расписанию.

А Вам как кажется?

\ifcmt
  tab_begin cols=2,no_fig,center
     pic https://avatars.mds.yandex.net/i?id=2a0000017a066a147e86318381462bc5d895-4936013-images-thumbs&n=13
     pic https://i.pinimg.com/originals/00/ce/23/00ce23a5a6393b760d9a0e7371d340bf.jpg
  tab_end
\fi
