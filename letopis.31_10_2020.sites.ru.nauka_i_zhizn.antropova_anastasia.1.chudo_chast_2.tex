% vim: keymap=russian-jcukenwin
%%beginhead 
 
%%file 31_10_2020.sites.ru.nauka_i_zhizn.antropova_anastasia.1.chudo_chast_2
%%parent 31_10_2020
 
%%url https://nkj.ru/open/39690/
 
%%author Антропова, Анастасия
%%author_id antropova_anastasia
%%author_url 
 
%%tags 
%%title Чудо, которое произошло. Часть вторая
 
%%endhead 
 
\subsection{Чудо, которое произошло. Часть вторая}
\label{sec:31_10_2020.sites.ru.nauka_i_zhizn.antropova_anastasia.1.chudo_chast_2}
\Purl{https://nkj.ru/open/39690/}
\ifcmt
	author_begin
   author_id antropova_anastasia
	author_end
\fi

\begin{leftbar}
	\bfseries Как снимать насекомых под пристальным взглядом милиционера, почему
				на Ямайке самые красивые пляжи, как размножаются слизни, и как плетут
				паутину пауки? А главное, можно ли стать счастливее, взяв в руки
				камеру? Об этом в заключительной части интервью с Ильёй Долговым,
				видеооператором живой природы, выпускником биофака НГПУ и автором блога
				\verb|@bbc_na_kolenke| в Инстаграм.
\end{leftbar}

\ifcmt
pic https://nkj.ru/upload/iblock/6d6/6d6cfe6bdb12fa0b381c25e8a5fa9a87.jpg
\fi

\textbf{Илья, в первой части интервью вы рассказывали о том, как пришли к работе
видеооператором. Если же говорить о самих съёмках, то как вы снимаете животных
и растения? Какие есть особенности в такой съёмке?}

Есть некоторые сложности во взаимодействии с «актёрами». С людьми ты можешь
договориться, а с животными и растениями всё совсем иначе. Они всё время
норовят уйти из кадра или вырасти куда-нибудь не туда. Из-за этого очень много
съёмочного материала идёт в помойку. Например, недавно под Звенигородом мы
пытались снять, как ночью паук плетёт свою паутину.

В первую ночь наш паук паутину плёл, но был такой густой туман, что линзы на
аппаратуре постоянно запотевали, и ничего толком снять не получилось. Во вторую
ночь погода внезапно изменилась, и пауки напрочь отказались плести сети и даже
начали их разбирать. Когда пауки чувствуют изменением погоды, например,пойдёт
либо дождь, либо обильная роса, они могут собрать круговые паутинки, чтобы под
их напором не обрушались опорные нити, которые существенно сложнее установить
пауку. И только в третью ночь у нас получилась съёмка, хотя и тут всё пошло не
по плану. Когда я впервые снимал паука, это заняло не более часа. В этот раз на
улице было +10, и паук плёл паутину часа три, а закончил уже на рассвете.
Казалось бы, я уже проводил точно такую же съёмку, уже представляю, как снимать
этот сюжет, но часто опыт множится на ноль, потому что в игру включаются
дополнительные факторы.

Часто съёмки длятся очень долго. Один раз я неделю жил в студии, чтобы снять,
как из коконов выходят бабочки. Это происходит довольно быстро и сложно понять,
когда начнётся «вылупление», поэтому я просто сидел возле них и ждал (только
выходил в магазин за дошираком и пивом). В итоге дождался. Сначала куколки
висят вниз головой, потом они начинают изменять цвет, становиться прозрачными.
А потом оболочка мгновенно трескается, лопается, оттуда выходит бабочка со
смятыми крыльями. Бабочка переворачивается вниз крыльями, чтобы под действием
силы тяжести они расправились. И тогда она уже превращается в полноценное
летающее насекомое. 

\ifcmt
pic https://nkj.ru/upload/medialibrary/d15/d157e45f50a8ebd28c72572131953ff8.jpg
\fi

Ещё сложнее снимать самый быстро растущий в мире гриб под названием весёлка.
Этот гриб входит в книгу рекордов Гиннесса, как самый быстрорастущий организм:
он растёт по полсантиметра в минуту, фактически на глазах. Но вся сложность в
том, что сначала гриб развивается в виде шарика, и внутри формируется плодовое
тело. Потом оболочка быстро трескается и оттуда выползает гриб и быстро
вырастает. Но чтобы снять это, нужно знать, когда он появится, а он может под
этой «оболочкой» находиться очень долго. Что бы я ни делал, как бы ни пытался
поторопить гриб «выйти наружу», ничего не помогало. Сколько бы я их не снимал,
наблюдения не помогали выявить какую-то закономерность. Даже учёные не могли
сказать, что влияет на этот процесс.

\textbf{Какие качества необходимы, чтобы стать хорошим видеооператором природы?}

Прежде всего, нужна усидчивость и терпение. Хорошо бы иметь какие-то знания о
том, что ты снимаешь - мне, например, в этом очень помогает моё биологическое
образование. И интерес, способность удивляться. Конечно, немного ребяческие
качества, но, мне кажется, они важны. 

\textbf{Случались ли у вас курьёзные случаи во время съёмок?}

Я жил какое-то время на улице Нагорная. Там есть речка Коршуниха, где я начинал
свои первые операторские потуги. И кроме грибочков я снял там прекрасных
слизняков. В одном би-би-сишном фильме показали, как они феерично размножаются.
Происходит это следующим образом. Два слизня заползают на ветку, сплетаются и
повисают в экстазе на длинной «сопле» из слизи. Я посмотрел эту программу и
понял, что точно такие же слизняки живут у меня за домом. Нашёл их и начал
снимать. И действительно – всё получилось! Но пока я снимал, мной
интересовались патрули. 

\ifcmt
pic https://nkj.ru/upload/medialibrary/230/2300473c3a4f8418afe2192c71c922ae.jpg
\fi

Район не самый благополучный, милиция стабильно каталась по улице, а тут я –
непонятный человек с фонариками. Выглядело очень загадочно. Милиционеры вышли,
представились и посветили фонариками в мою сторону. Я им очень строго сказал,
что нужно выключить свет и только после этого можно ко мне обратиться, иначе
они всё мне испортят. На удивление они послушались, спросили, что я снимаю.
«Половой акт слизней», – ответил я. И они поехали дальше.

На той же улице я снимал для детской программы маленькие грибочки. Сижу на
газоне с камерой и… чувствую чьё-то присутствие. Поворачиваю голову, а надо
мной склонился комичный, почти мультяшный, милиционер, напоминающий Дядю Стёпу.
Он даже на меня не смотрел, смотрел на дисплей камеры, где было видно, как по
грибочку ползёт букашка. Смотрел, смотрел, потом сказал: «Какая красота!», –
почесал затылок, его фуражка съехала на лоб, и он ушёл.

В любом случае, если я снимал в городе, мной так или иначе интересовалась
милиция. Даже ещё до съёмок, у меня были забавные стычки. Когда я работал в
океанариуме, у нас умер мечехвост (это древнее членистоногое, очень похожее на
ксеноморфа из фильма «Чужой»). Какое-то время он просто валялся на полке в
океанариуме и сох. У меня был приятель, который коллекционировал ракообразных –
наш мечехвост был бы отличным подарком для него. Я принёс мечехвоста к себе
домой и вечером после работы в тапочках и с целлофановым пакетом пошёл вручать
подарок (жили мы практически по соседству). Тут же останавливается патрульная
машина, выскакивают двое милиционеров с автоматами, идут ко мне и интересуются
наличием документов. 

\obeycr
–Дома они у меня, – отвечаю я им.
– А куда вы идете?
– В магазин.
– А там нет магазина, – отвечает мне милиционер.
– Там АШАН.
– А в пакете что?
– Мечехвост, – они склонились над пакетом. Один из них не растерялся и спросил:
– А зачем вам в АШАНе мечехвост???
Второй добавил
– Странный вы какой-то...
Я, недолго думая, ответил:
– Первый день что ли работаете, не насмотрелись ещё?
Они поняли, что никакой угрозы во мне нет, да и мечехвост, в общем-то, для общества безвреден, и тут же меня оставили в покое. 
\restorecr

{\bfseries 
Часть вашей работы и жизни – это путешествия. Получается ли отдохнуть во время поездок или всё время вы посвящаете работе?
}

Все мои выезды так или иначе всегда связаны с работой. Я беру с собой камеру, и
отпуск всё равно превращается в работу, на что мои друзья раньше обижались. Им
хочется тусить, а я сижу с камерой и жду, когда из пещеры начнут вылетать
летучие мыши. Но потом они поняли, что наблюдать за моей работой занимательно и
смирились с этим. 

\ifcmt
pic https://nkj.ru/upload/medialibrary/dd1/dd1e49aa1efc4143395c557568c36f92.jpg
\fi

Я много путешествовал по Азии, по её необычным местам. Даже на «попсовом»
курорте пытаюсь найти что-то интересное. Из экзотических мест я был в Китае,
где даже успел и пожить, Тайланде, Вьетнаме, Шри-Ланке, Монголии. И очень
люблю Ямайку. Это очень аутентичная страна, и в силу особенности местного
населения там прекрасная уцелевшая природа. По соседству аграрные Куба и
Доминикана, их ландшафт уже сильно изменён, а на Ямайке люди кайфуют, слушают
Боба Марли, курят и никуда не спешат. Поэтому там нетронутые прекрасные пляжи.
Больше всего мне запомнилась светящаяся лагуна. Я уже и прежде видел подобные
явления: во Вьетнаме Японское море светится по осени. Но на Ямайке особые
условия, и бухта светится круглогодично, ярко и интенсивно. Биолюминесценцию я
до сих пор воспринимаю как маленькое чудо.

\textbf{Вы провели детство и юность в Новосибирске, работаете в Китае, сейчас живёте в
Москве. В чём для вас состоят ключевые отличия жизни в этих местах?}

В регионах всё намного медленнее. Новосибирск по количеству населения примерно
в десять раз меньше Москвы, город более тесный, поэтому люди скованнее, более
зажаты: всё у всех на виду. В головах звучит фраза: «А что подумают
окружающие?» Здесь, в Москве, существенно проще чувствовать себя расслабленным
и делать то, что хочется. Чувствуешь себя абсолютно свободно и спокойно. Кругом
куча людей, которые всё время куда-то спешат. Никому нет до тебя абсолютно
никакого дела. 

\ifcmt
pic https://nkj.ru/upload/medialibrary/95f/95f565ed0f8ba897a0fa12428e38f292.jpg
\fi

Максимально комфортно я себя чувствовал в большом городе в Китае, где я
работал барменом полгода. Много людей, никого вообще не понимаешь, для тебя
люди – это просто декорации. За полгода жизни в Китае я стал чувствовать себя
абсолютно расслабленно в людных местах. А как только вернулся обратно, я
слышал и понимал каждое слово, которое произносили около меня, я не знал, как
от этого абстрагироваться. Едешь ли в метро, идёшь ли по улице, слышишь
каждого человека и зачем-то думаешь о том, что он произносит. В Китае же было
парадоксальное чувство, среди многотысячной толпы мне казалось, будто бы я
один на всей планете.

\textbf{Как изменилась ваша жизнь, после того, как вы взяли в руки камеру?}

Я стал счастливее. У меня появилось больше денег, чего уж там таить. Я стал
намного расслабленнее, потому что больше не нужно было ходить на работу с 10 до
6. Жизнь стала интереснее!

\textbf{Чтобы вы хотели сказать напоследок тем, кто дошёл до конца этого интервью?}

Людям стоит чаще отвлекаться от бытовых забот, дел и чаще смотреть себе под
ноги. Вокруг происходит действительно много интересного, что по-настоящему
заслуживает внимания.

Наш мир и его ценности немного искусственные, и, когда это полностью занимает
твою голову, от этого устаешь: экономика, политика – всё это выдуманные
истории. А самые простые вещи: паук плетёт свою паутину, растёт гриб – могут
намного сильнее увлечь и разум, и воображение. Городским людям стоит чаще
замедляться и смотреть по сторонам. Можно увидеть много стоящего.

\ifcmt
pic https://nkj.ru/upload/medialibrary/73f/73f479ef03242dea7e64c103f5492df8.jpg
\fi

\begin{itemize}
  \item Фото и видео: блог Ильи Долгова \verb|@bbc_na_kolenke| в Инстаграм.
	\item \href{https://www.nkj.ru/open/39683/}{Первая часть интервью.}
	\item Автор: \textbf{Анастасия Антропова}
  \item Источник: «Наука и жизнь» (nkj.ru)
  \item \verb|#интервью #природа #фотографии|
  \item 31 октября 2020
\end{itemize}
