% vim: keymap=russian-jcukenwin
%%beginhead 
 
%%file 10_12_2021.fb.zhuravko_aleksej.1.potomki_pobeditelej
%%parent 10_12_2021
 
%%url https://www.facebook.com/permalink.php?story_fbid=408132394377053&id=100055408874270
 
%%author_id zhuravko_aleksej
%%date 
 
%%tags mirvojna2,pamjat,pobeda,polsha,rkka,sssr,ukraina,vov
%%title Как потомки победителей смогут забыть свою генетическую память
 
%%endhead 
 
\subsection{Как потомки победителей смогут забыть свою генетическую память}
\label{sec:10_12_2021.fb.zhuravko_aleksej.1.potomki_pobeditelej}
 
\Purl{https://www.facebook.com/permalink.php?story_fbid=408132394377053&id=100055408874270}
\ifcmt
 author_begin
   author_id zhuravko_aleksej
 author_end
\fi

Дорогие друзья! 

Еще 8 лет назад мне один поляк говорил: «Вот увидишь – мы ещё доживём, когда в
Европе станут говорить – на самом деле это Советский Союз напал на Германию. А
немцы вполне культурные люди, с ними не случалось ничего плохого: при оккупации
строили школы, больницы, раздавали детям шоколад… не сравнить с ужасными
большевиками».

Я тогда посмеялся – ну, думаю, шикарное у человека чувство юмора и как люди
могут забыть истинную историю? 

\ii{10_12_2021.fb.zhuravko_aleksej.1.potomki_pobeditelej.pic.1}

Однако, не прошло и нескольких лет, как это пророчество начало сбываться. За
последние годы в Польше демонтировали множество памятников солдатам РККА.
Польские улицы, названные в честь погибших поляков-подпольщиков,
переименовываются – просто потому, что эти герои были коммунистами. В
Прибалтике ежегодно устраиваются марши одряхлевших участников местных легионов
СС. В Румынии суд Бухареста признал союзника Гитлера, диктатора Иона Антонеску
невиновным (!) в нападении на СССР, поскольку это было «юридически оправдано».
Правда, позднее суд своё же решение отменил. Но, тут уж, как говорится в
анекдоте: ложечки нашлись, а осадок остался. 

Украина в этом вопросе не отстает. Нынешние власти переименовали все улицы,
запретили ПОБЕДУ и любое упоминает Великой отечественной войны. Как можно было
допустить уничтожение собственной истории и кричать с пеной у рта на почве
русофобии. 

Как потомки победителей смогут забыть свою генетическую память. Это был геноцид
одновременно всех наций Советского Союза. Я всегда объясняю – 9 мая мы отмечаем
наш день независимости, поскольку в той войне был поставлен на карту вопрос
выживания подавляющего большинства народов СССР. 

И это нужно всегда помнить.

С уважением, 

Алексей Журавко
