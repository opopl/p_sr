% vim: keymap=russian-jcukenwin
%%beginhead 
 
%%file 08_03_2018.stz.news.ua.mrpl_city.1.tenditni_neskoreni_sylni
%%parent 08_03_2018
 
%%url https://mrpl.city/blogs/view/tenditni-neskoreni-silni-1
 
%%author_id demidko_olga.mariupol,news.ua.mrpl_city
%%date 
 
%%tags 
%%title Тендітні, нескорені, сильні...
 
%%endhead 
 
\subsection{Тендітні, нескорені, сильні...}
\label{sec:08_03_2018.stz.news.ua.mrpl_city.1.tenditni_neskoreni_sylni}
 
\Purl{https://mrpl.city/blogs/view/tenditni-neskoreni-silni-1}
\ifcmt
 author_begin
   author_id demidko_olga.mariupol,news.ua.mrpl_city
 author_end
\fi

\ii{08_03_2018.stz.news.ua.mrpl_city.1.tenditni_neskoreni_sylni.pic.1}

Що зберігають аннали історії Маріуполя про жінок, чия самовіддана діяльність,
талант і неповторність надихають? Вони такі тендітні та сміливі водночас,
вродливі, ніжні, проте нескорені та незламні духом. Кажуть, що сила жінки в її
слабкості, проте життя доводить, що сила жінки – у силі. Пропоную познайомитися
з ними ближче і запам'ятати їхні імена... Так, Маріуполь – унікальне місто, в
історію якого увійшли сильні жіночі особистості.

Поширеним є твердження, що за \emph{\enquote{кожним великим чоловіком стоїть велика жінка}}.
Можливо, так і є. Наше місто не стало виключенням. Є великий чоловік Архип
Іванович Куїнджі, якого надихнула на неймовірну роботу над собою, на боротьбу
за неї дочка багатого купця, талановита піаністка \textbf{Віра Шаповалова}. Вона своєю
самовідданістю і неповторною жіночністю повністю підкорила ще невідомого
світові художника. Батько Віри, незадоволений тим, що дочка проводить багато
часу зі злиденним художником, запитав її, не зібралася вона за нього заміж? На
що дівчина одразу відповіла: \enquote{Якщо не за Архипа, то тільки в монастир}.
Шаповалов поставив Куїнджі умову: принесеш сто рублів золотом – Віра твоя. Для
Архипа це була непідйомна сума. Він вирішив їхати в Петербург на заробітки. Та
завдяки закоханості у Віру Архип Куїнджі підкорив Санкт-Петербург і,
повернувшись до неї, отримав дозвіл від її батька на одруження. Однак чекати
закоханим довелося довго – цілих 12 років. І все ж таки сильна Віра дочекалась
свого єдиного і неповторного Архипа Івановича. Після вінчання Куїнджі написав
портрет дружини. Її обличчя на картині світилося любов'ю. А щасливий Архип
Іванович з того часу творив один шедевр за іншим.

%\ii{08_03_2018.stz.news.ua.mrpl_city.1.tenditni_neskoreni_sylni.pic.2}
%\ii{08_03_2018.stz.news.ua.mrpl_city.1.tenditni_neskoreni_sylni.pic.3}

Одна з найбільш яскравих актрис Маріупольського театру \textbf{Людмила Радіонова} стала
безпосередньою учасницею воєнних дій. ЇЇ бойовий шлях відзначений орденом
Леніна, двома орденами Червоної Зірки, сімома бойовими медалями, медаллю
Міжнародного Червоного Хреста. Працювала вона в Маріупольському театрі з 22
років. У статистах ходити їй не довелось, у Маріупольському театрі почала вона
блискуче – відразу з головних ролей. Лариса в \enquote{Безприданниці} О. Островського
відразу ж зробила їй ім'я. Незабаром жінка-комісар в \enquote{Оптимістичній трагедії}
В. Вишневського, ролі в \enquote{Любові Яровій}, \enquote{Марійці}. Всього за один рік актриса
встигла зіграти стільки яскравих ролей, скільки іншим, дай Бог, за все життя
зіграти. М. Соколов, народний артист Української РСР, згадуючи про Людмилу
Радіонову, підкреслив: \enquote{Так, талантом її природа обдарувала щедро. Але,
мабуть, всього важливіше її талант людини – мужньої, красивої, яскравої; її
талант жити без оглядки, не шкодуючи себе}.

\ifcmt
  tab_begin cols=2,no_fig,center,separate

  pic https://mrpl.city/uploads/posts/redactor/4xd6lw3rjiff3dot.jpg
  pic https://mrpl.city/uploads/posts/redactor/4j4naclgucmcilum.jpg

  tab_end
\fi

З початком війни Л. Радіонова стала санінструктором 75-го стрі\hyp{}лецького полку. У
с. Миколаївці під Таганрогом вона винесла з вогню і врятувала від неминучої
загибелі сорок п'ять поранених бійців і командирів. У бої під Таганрогом Л.
Радіонова розстріляла з револьвера весь екіпаж нацистського танку, за що була
нагороджена орденом Леніна. З 4 по 6 березня 1944 р. надала допомогу 213 бійцям
і командирам, пораненим в бою. Була нагороджена вищою відзнакою Міжнародного
комітету Червоного Хреста – медаллю \enquote{Флоренс Найтінгейл}. Ця нагорода носить
ім'я однієї з засновниць Міжнародного Червоного Хреста і вручається медсестрам
і санітаркам, які \enquote{відзначилися виключною самовідданістю при догляді за
пораненими та хворими під час війни або громадського лиха}. У 27 років вона
стала інвалідом. Театр довелося залишити...

\ii{08_03_2018.stz.news.ua.mrpl_city.1.tenditni_neskoreni_sylni.pic.5}

Під час Другої світової війни маріупольський інженер рибної промисловості
\textbf{Єлизавета Бірюкова} врятувала не менше 150 чоловік. Вона не встигла
евакуюватися, хворіла на скарлатину. Молода жінка з вищою технічною освітою під
час війни залишилася в місті й наважилася піти працювати на німців, хоч і
прекрасно усвідомлювала, чим це може для неї закінчитися. Працювала на
рибоконсервному комбінаті сумлінно і налагодила настільки потужне виробництво
консервів, що німці не могли натішитися її роботою. Та, судячи з матеріалів
справи, далі вона абсолютно системно починає рятувати людей, допомагає
військовополоненим і містянам. Наприклад, Єлизавета Бірюкова наказувала
залишати на відрізаних головах риб м'ясо, яке потім варила і відносила в табір
військовополонених. Після звільнення Маріуполя жінку звинуватили в
колабораціонізмі й доносах. Їй присудили 15 років позбавлення волі.

Маріупольці, яких врятувала Єлизавета Бірюкова, писали листи Берії, Сталіну,
головам НКВС, благаючи про помилування. Люди заявляли: \enquote{Єлизавета Бірюкова
перетворила рибний комбінат на табір спасіння, забирала чоловіків з таборів для
військовополонених, запевняючи офіцерів вермахту, що це найкращі спеціалісти,
без яких комбінату не обійтись. Годувала у столовій членів сімей комуністів та
євреїв. Немічним та хворим возила харчі додому. Загалом від долі остарбайтерів
врятувала 150 людей}. За 4,5 роки Бірюкову випустили, але все життя жінка
намагалася зняти судимість. Реабілітували її лише на початку 90-х. Немає жодної
світлини сміливої жінки, залишився лише короткий запис у Книзі пам'яті
Маріуполя і листи тих маріупольців, хто хотів їй допомогти.

Афганська війна тривала довгих 10 років, з 1979 по 1989 рік. За цей час службу
в Афганістані пройшли 1,5 млн чоловік. 15 тисяч радянських солдатів і офіцерів
загинули. Серед них понад 3 тисячі – українці, 26 осіб – з Маріуполя: 25
хлопців і одна дівчина, \textbf{Людмила Мошенська}. Вона працювала медичною сестрою
дитячого відділення міської лікарні № 4 в Жданові. У добровільному порядку
через Орджонікідзевський районний військовий комісаріат міста Жданова 7 травня
1983 р. у 27 років була направлена для роботи за наймом в радянські війська, що
знаходилися в Республіці Афганістан. Працювала медичною сестрою інфекційного
відділення 650-го військового госпіталю 40-ої загальновійськової армії
(військова частина польова пошта 94777; місто Кабул. Відважна маріупольчанка,
будучи медсестрою, півтора року працювала в інфекційному госпіталі, рятувала
радянських солдат від всіляких інфекцій, поширених на Сході, проти яких у них
не було імунітету. Однак сама була інфікована і померла в Афганістані 12
вересня 1983 р. від важкої форми черевного тифу.

\ii{08_03_2018.stz.news.ua.mrpl_city.1.tenditni_neskoreni_sylni.pic.6}

Маріуполь пов'язав долі Віри Куїнджі, Людмили Радіонової, Єлизавети Бірюкової,
Людмила Мошенської, які, безумовно, заслуговують особливої уваги та пам'яті,
адже, на жаль, для багатьох маріупольців дотепер вони залишаються маловідомими...
Вони – \emph{тендітні, нескорені, сильні} – сьогодні є важливою часткою історії
Маріуполя.
