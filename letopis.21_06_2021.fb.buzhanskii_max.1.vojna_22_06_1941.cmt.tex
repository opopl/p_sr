% vim: keymap=russian-jcukenwin
%%beginhead 
 
%%file 21_06_2021.fb.buzhanskii_max.1.vojna_22_06_1941.cmt
%%parent 21_06_2021.fb.buzhanskii_max.1.vojna_22_06_1941
 
%%url 
 
%%author 
%%author_id 
%%author_url 
 
%%tags 
%%title 
 
%%endhead 
\subsubsection{Комментарии}
\label{sec:21_06_2021.fb.buzhanskii_max.1.vojna_22_06_1941.cmt}

\begin{itemize}

\iusr{Inessa Mazur}
Да отлично написано, и глубоко... Если представить, просто представить, на сколько вторглось горе в человеческую жизнь

\iusr{Dmitriy Levchenko}
Инесса Мазур умеет написать Макс) это правда)

\iusr{Александр Шибанов}
Буднично и страшно!Всё рядом.
Спасибо Макс за хороший текст🤝

\iusr{Lyudmila Lukyanchuk}
Спасибо, Макс.До слёз...

\iusr{Svetlana Kvasha}
До слез....

\iusr{Julia Sergeeva}
До глубины души

\iusr{Наталія Коваль}
Дуже точно. Бездоганно по стилю. Точні емоції. Рука майстра? Так. Але велика душа справжньої Людини! Дякую. Це зачіпає душу...

\iusr{Станислав Лахтионов}
Просто нет слов...Спасибо Вам👏👏👏

\iusr{Людмила Гурова}
Спасибо,что не даёте забывать

\iusr{Сергей Олексеенко}

По-моему это было с субботы на воскресенье.

\iusr{Людмила Шпакова}
👏✌️👍Макс!💯

\iusr{Nadegda Anfalova}

Эта дата навсегда впечатана в наши души.

\iusr{Олег Сюмко}

А вечером Жуков, Кузнецов и другие штабисты будут спорить о приведении войск в
боевую готовность. По воспоминаниям Кузнецова день был жарким

\iusr{Зинаида Кобец}

\ifcmt
  pic https://scontent-mxp1-2.xx.fbcdn.net/v/t1.6435-9/131726667_612365936822505_6574888189599693540_n.jpg?_nc_cat=101&ccb=1-3&_nc_sid=dbeb18&_nc_ohc=qPMhiX3KKmgAX_yyMHM&_nc_ht=scontent-mxp1-2.xx&oh=fc218e3b5a45211f44fb72d9020658e5&oe=60D71F46
  width 0.2
\fi

\iusr{Андрей Данко}

Сильно, Макс... до слез 😕

Татьяна Жилинская
До слез...

\iusr{Konstantin Siminenko}

Провожу аналогию с сегодняшними днями:

Пошёл ли бы я молча, стиснув зубы, защищать родину?

От кого? Для кого? Ради чего?

Совершенно чужое и безразличное ко мне государство, смотрящее на меня глазами
даже не мачехи, скорее незнакомки, для которой я всего лишь прохожий.

Защищать ради того, чтоб даже в случае «победы» быть забытым, никому не нужным
психическим инвалидом без почестей?

Защищать ради того, чтоб мои сверстники в это время резвились по клубам и
пляжам моей столицы, а иностранцы получали неслыханные и невиданные для меня
деньги в наблюдательных советах предприятий, которые я «защищаю»?

Те люди шли защищать ту страну, потому что она была ИХ.

Сегодня, для десятков, сотен тысяч молодых людей, эта страна, увы, чужая.

Прости, Украина, не мы тебя просрали, не нам тебя и защищать.

\begin{itemize}
\iusr{Валерий Чернолуцкий}
Konstantin Siminenko А если вас призовут, куда вы денетесь?
\iusr{Konstantin Siminenko}
Валерий Чернолуцкий найду куда.
\iusr{Валерий Чернолуцкий}
Konstantin Siminenko Тогда вы нарушите закон и вас можно будет привлекать к уголовной ответственности. Всего лишь. Поймите правильно, я не то чтобы с вами соглашаюсь, но по крайней мере понимаю. Однако понимаете ли вы, что громкие слова это одно, а реальные поступки - другое?
\iusr{Konstantin Siminenko}
Валерий Чернолуцкий я уже не в призывном возрасте, поэтому даже не переживаю. Прописан на неподконтрольной территории, из-за чего меня здесь физически не существует. Могу перечислять дальше, но и этих двух пунктов с головой.
\iusr{Konstantin Isikov}
Konstantin Siminenko Отлично сказано!
\iusr{Наталья Шарипова}
Konstantin Siminenko 👏👏👏
\iusr{Валерий Чернолуцкий}
Konstantin Siminenko Я тоже непризывной, поэтому мы с вами можем довольно спокойно и даже отвлечённо, как бы со стороны, обсуждать этот вопрос. А ведь для кого-то он актуален и приходится делать вполне конкретный выбор.
\iusr{Konstantin Siminenko}
Валерий по-моему выбор для многих очевиден: лучше собирать клубнику в Польше, чем тебя будут собирать по частям на поле брани
\iusr{Валерий Чернолуцкий}
Konstantin Siminenko Но кто-то же всё же сидит в окопах под Донецком и Луганском. Иначе у нас была бы совсем другая военно-политическая ситуация
\iusr{Konstantin Siminenko}
Валерий да, сидят. И с той и с той стороны. Герои для одних, враги для других. И наоборот.
\iusr{Валерий Чернолуцкий}
Konstantin Siminenko Ну, это да. Как говорится, наши - это всегда доблестные разведчики, ихние - презренные шпионы.
\iusr{Alexander Abend}
Валерий Чернолуцкий \enquote{куда вы денетесь?} Возможно туда, куда делись несколько миллионов украинцев за последние несколько лет? В Польшу, например. Да и у северного соседа на стройках часто слышна украинская речь
\end{itemize}

\iusr{Анна Смирнова}
Как обыденно и страшно, после того как вспомнишь последний мирный день. Спасибо, Максим!

\iusr{Vladimir Sterling}
Отличная отповедь тем, кто забыл, что была такая война, Великая Отечественная!

\begin{itemize}
\iusr{Валерий Чернолуцкий}
Vladimir Sterling Кое-для кого сейчас это была всего лишь радянсько-німецька війна.
\iusr{Konstantin Isikov}
Валерий Чернолуцкий таких очень мало
Валерий Чернолуцкий
Konstantin Isikov Но почему-то именно они отвечают у нас за гуманитарную политику.
\iusr{Konstantin Isikov}
Валерий Чернолуцкий К сожалению да.
Таковы директивы из США
\iusr{Людмила Брашевич}
Konstantin Isikov Своим умом никак... но эти люди о ценностях нам глаголят((
\end{itemize}

\iusr{Anatoly Ploshkin}
Очень точные образы. Респект автору!

\iusr{Роман Шеховцов}

Кто были фашисты ТОГДА это понятно. Макс, скажите, а как назвать тех, кто
запрещает СЕЙЧАС даже использовать термин ВОВ? Как назвать тех, кто в раздел
школьных учебников \enquote{Герои Украины} вносит эсесовцев? Как назвать
этих.....людей?!

\iusr{Андрей Буров}
Хороший вопрос! Но без ответ

\iusr{Людмила Шпакова}

Макс, всегда с удовольствием слушаю Вас!!! На НАШем!

\iusr{Макс Бужанский}
Людмила Шпакова, спасибо, Людмила.

\iusr{Инга Тесс}

Макс Бужанский вы как всегда все по полочкам лаконично все в тему доходчиво
спасибо. Без вас программа Народ против была бы неинтересной всегда смотрю с
удовольствием так держать

\iusr{Элина Гаврилюк-Конорезова}
Макс Бужанский
И я вас очень люблю и уважаю, мировоззрение совпадает!!!

\iusr{Elena Ohinskaya Filippova}
До слёз... Спасибо

\iusr{Alena Hoshabaeva}
Спасибо. До слёз...

\iusr{Светлана Семенец}
Спасибо! Это так важно, помнить и не изменять ни себе, ни ушедшим....спасибо.

\iusr{Лариса Лупашко}
Спасибо, Максим. Пронзительно. До мурашек. До слёз.. Спасибо за память.

\iusr{Профессор Фэйт}
Максим , спасибо ...

\iusr{Zhanna Evsa}
Спасибо и поклон, Максим!

\iusr{Макс Бужанский}
Zhanna Evsa, достаточно просто спасибо)

\iusr{Zhanna Evsa}
ОК!

\iusr{Елена Несветайло}

Человечище! Так чувствовать может только человек с огромной душой!

\iusr{Ольга Заг}

Да.. Все были счастливы и не подозревали об этом..  Спасибо, что напоминаете,
как надо ценить мир.

\iusr{Валентин Станев}

Спасибо Макс. Это страшно представить даже. Как выпускники отгуляли последний
звонок, и строили планы на жизнь, а завтра мечты многих оборваться ...

\begin{itemize}
\iusr{Валентина Вильчинская}
Валентин Станев і не тільки мрії,а й життя

\iusr{Валентин Станев}
Валентина Вильчинская именно так
\end{itemize}

\iusr{Наталья Рогова}
Очень сильно написано. Спасибо

\iusr{Светлана Бондаренко}

Максим,как вы думаете, женщинам при Порошенко Рева с Третьяковой придумали 30
лет стажа ,это правильное решение или женщин привратили в раб силу,как вы
считаете это правильно или надо вернуть женщинам стаж 25 лет??? Женщин после 50
лет не берут на работу где им брать 30 лет,предлагают только мужскую тяжёлую
работу.

\iusr{Валентина Вильчинская}
Ви один з небагатьох слуг народу,якого можна слухати з задоволенням.Дякую

\iusr{Макс Бужанский}
Валентина Вильчинская, дякую, пані Валентина

\iusr{Олег Резник}

«...22 июня состоится годовщина нападения нацистской Германии на СССР. После того
как ... была развязана Вторая Мировая война, ... Германия открыла следующую
ужасающую главу. Для нас, немцев, этот день – повод для стыда», – заявила
канцлер.  «ФРГ признает свою ответственность за преступления нацистского
режима. К этому относится и постоянная память о том, что произошло. Это не
должно быть забыто. Мы должны ради миллионов жертв и их потомков.

\begin{itemize}
\iusr{Роман Шеховцов}

Олег Резник Для немцев это повод для стыда, а для украинских коллаборционистов
повод для гордости. Эсесовца в Киеве хоронили с президентским почетным
караулом.

\iusr{Олег Резник}

Роман Шеховцов, а в Харькове недавно хоронили последнего Героя Советского
Союза, летчика- штурмовика, который совершил 300 боевых вылетов, уничтожившего
массу боевой техники врага - без Президентского полка и вообще без малейшего
внимания со стороны Офиса(((

\iusr{Роман Шеховцов}

Олег Резник Да. И за это власть получит свою оценку на выборах. И они это
знают, поэтому будут снова пытаться нас обмануть. Может часть партии во главе с
Максом отколется под Восток, может какую ту новую управляемую партию
создадут...

\iusr{Олег Резник}

Роман Шеховцов, не думаю. У Макса своя ниша.  По хорошему- вообще не ходить на
выборы, пока не возникнет нормальная политсила.

\iusr{Роман Шеховцов}

Олег Резник Я думаю, что будет новое шоу. Увидим, что придумают. Но на выборы
идти надо, хотя бы, чтобы не дать власти украсть твой голос

\iusr{Ирина Грунт}

Роман Шеховцов Посудите сами, 30 років незалежності. 6 президентів, постійні
вибори до верховної ради і органів місцевого самоврядування. Ми що, всі ці роки
не тих вибираємо? Кого б ми не вибрали, нічого не зміниться. Повинно бути
покарання і приводитися в дію.

\iusr{Олег Резник}

Роман Шеховцов, или отдавать голос за настоящую левую силу, пусть даже явно не проходимую.

\iusr{Олег Резник}

Ирина Грунт, а ВР - это вообще какой то паразитный орган.Демагоги, неучи, популисты, болтуны, взяточники.

\end{itemize}

\iusr{Анна Кулинич}
Дякую за важливу роботу Максиме, яку Ви робите.
Сильно і до сліз ...!!!

\iusr{Vladimir Dovzhik}

Как же, Вы, Макс, точно описываете ощущения людей в тот день, хотя и не были
свидетелем тех событий! Я, ребёнок войны, брат и сын фронтовиков, утверждаю,
что всё было именно так. Спасибо, Макс, за память,за талант!

\iusr{Константин Зарукин}
просто сделаю перепост. Вам Максим спасибо!

\iusr{Яна Игнатова}
Максим, это пронзает. И сердце и мозг. Пережила сейчас этот день с вами. Спасибо.

\iusr{Svetlana Scherbakova}
Как всегда - пронзительно! А завтра была война!

\iusr{Валентина Калашнікова}

Ох, и умеете ВЫ, Максим, воспоминанием встревожить душу. Те, у кого погибли или
воевали в то время, будем с уважением помнить их подвиг!

\iusr{Елена Ставицкая}
Какое счастье жить сейчас.

\iusr{Eleonora Vaisfeld}
Спасибо, Максим! Спасибо за память и правду!

\iusr{Наталия Гудзь}

Да, горько сжимается сердце. Все это было до войны. А завтрашний день все
изменит во всех наших судьбах. Бабушка с дедушкой говорили. Я была очень
маленькой. "Я видел его (кого то из знакомых) ещё до войны. И по радио передача
была. Родственники ищут родных, пропавших в годы войны. И мои, серьезно глядя
перед собой, внимательно вслушивались в имена, произносимые диктором. Вот такое
было у меня ощущение войны.

\iusr{Dmitry Balzer-Bondarenko}

Спасибо.  Мой дед принял первый бой в 04:30 22.06.41, на границе у Равы
Русской, в звании лейтенанта, командира танковой роты, будучи раненым в
рукопашном бою, оставил Львов 28.06.41..

\iusr{Марина Горемыкина}

Максим, большое спасибо вам за все, что вы пишите!

\iusr{vlad Gerasimenko}
Тронуло . Спасибо за пост.🤝

\iusr{Людмила Антонишина}

Макс! Всегда стараюсь не пропустить всё,что вы пишите.Часто со слезами,как
сейчас. Мой отец-фронтовик встретил войну на территории теперешнего польского
Пшемышля. Рассказывал о войне мало,говорил,что то,что я помню,слишком страшно и
детям лучше не знать.Но все знали,что войны с немцами не избежать,так он
говорил.

\iusr{Alexandr Aratsky}

До мурашек..

\iusr{Олег Резник}

\ifcmt
  pic https://scontent-mxp1-2.xx.fbcdn.net/v/t1.6435-9/203253057_962772467844095_8803526613896209343_n.jpg?_nc_cat=102&ccb=1-3&_nc_sid=dbeb18&_nc_ohc=MPYo3P1BvTkAX_BGLhR&_nc_ht=scontent-mxp1-2.xx&oh=d96924a03a8352476a0653d8583740c3&oe=60D6C52D
  width 0.2
\fi

\iusr{Maria Yavorskaya}
Макс, всегда очень рада, слышать и видеть, Вас, в эфире, читать Ваши посты. С уважением!

\iusr{Роман Романовский}
Спасибо за правду. Сейчас это редкость. 👍👍👍👍👍👍👍

\iusr{Viacheslav Makarov}
Все рушится в один миг.. И в этом одна из ужасных граней любой войны...

\iusr{Олег Резник}

Рассвет 22 июня 1941 г. Передовой немецкий разведывательный дозор переходит
границу СССР.  Фотограф ещё не знает, что он в одном кадре отразил и начало, и
исход войны.

\ifcmt
  pic https://scontent-mxp1-2.xx.fbcdn.net/v/t1.6435-9/203180008_962802367841105_4145185017743612198_n.jpg?_nc_cat=104&ccb=1-3&_nc_sid=dbeb18&_nc_ohc=Bjoi4zTvZ_IAX8tol-3&_nc_oc=AQnLdOXt1JURP7X3fS3oTmLr1MUIDCFV6mQGzHN-22zkCuQLdk0P4yXPFKl1ps_u5zM&_nc_ht=scontent-mxp1-2.xx&oh=2b5992517a1a9a48e599b44dec94eebb&oe=60D72ECA
  width 0.2
\fi

\iusr{Vitalik Derevyanko}

Страшно все было...тогда, даже тяжело представить слова \enquote{Война}с
репродуктора.. и намного страшнее реальность.. наша, мирная, благодаря им... Маршал
Победы....

\ifcmt
  pic https://scontent-mxp1-2.xx.fbcdn.net/v/t1.6435-9/205531238_593451808286456_8641581706060938971_n.jpg?_nc_cat=102&ccb=1-3&_nc_sid=dbeb18&_nc_ohc=yXnRTGtKSgMAX_weNSI&_nc_ht=scontent-mxp1-2.xx&oh=f26d50b10187ad1f5a61c48c1edf3ed5&oe=60D65CE0
  width 0.2
\fi

\iusr{Надежда Токар}
Страшно то, что сейчас стараются забыть подвиг народа, оболгать.

\iusr{Виталий Пискун}

Максим, знали, что будет война. Руководство страны абсолютно точно знало, что и
как. Политик РФ Никонов (внук Молотова) говорил, что он сто раз расспрашивал
деда о тех, предвоенных событиях. Гитлер принял решение о нападении за неделю,
то есть 17 июня и 18 июня приказ ушёл в войска. Однако, даже 18 июня Молотов
звонил в ставку фюрера с предложением обсудить ситуацию... безрезультатно. И
наступило 22 июня.

\iusr{Mila Bondar}
Как много еще людей, готовых воевать.

\iusr{Евгения Краско}
Мой дед так ушел и не вернулся. Хотя мог не уходить, сам ушел. Красивый. Умный. Образованный... И никогда меня не увидел.

\iusr{Lyudmila Bogdanova}

Студент Киевского театрального Вадим Михайлович Богданов в первые дни войны
добровольцем ушёл на фронт. Был дважды ранен, вернулся в строй. Погиб под
Курском в 1943 г.

\ifcmt
  pic https://scontent-mxp1-2.xx.fbcdn.net/v/t1.6435-9/205039691_3970260953093549_1841391674371904058_n.jpg?_nc_cat=100&ccb=1-3&_nc_sid=dbeb18&_nc_ohc=-igNVxrEu-sAX-AkAOj&_nc_oc=AQl4ZZ-wUzm2-6APtGz4Vt5S0k7LuLBHveznLV4xsj4fOBPxeOjNR7VmvqksZu5tmfM&_nc_ht=scontent-mxp1-2.xx&oh=700cedd1121204aba64b4d58f420e019&oe=60D5E78A
  width 0.2
\fi

\iusr{Михаил Варшавский}
Макс ты может единственный из зелени в адеквате , помнишь историю и не перекручиваеш её в угоду всякому сброду

\iusr{Elena Lebedeva-Giouzeli}
Именно так ......

\iusr{Alexander Abend}
Кстати, в прошлый раз за Севастополь бились 8 месяцев, прежде чем туда ступила нога оккупанта...

\iusr{Egor Trofimoff}
Спасибо

\iusr{Александра Домкович Марченкова}

Мои дедушка и бабушка служили во Владимире-Волынском. Бабушка всю жизнь мечтала
научиться кататься на велосипеде. Они даже купили его. Дедушка обещал, что
завтра будет воскресенье, он возьмет выходной и за один день ее научит. Но
вечером у моей мамы поднялась температура, а наутро было 22 июня. Не дожидаясь
приказа из Москвы, командир части отдал распоряжение на эвакуацию семей
комсостава. Бабушка с дедушкой под бомбами бежали через весь город к вокзалу,
передавая из рук в руки мою завернутую в одеяло пятилетнюю маму. Дедушка
буквально коленом запихнул их в последний вагон. До 1942 года вообще не знали,
кто остался живой. К сожалению, дедушка Шахвердиев Гайбет Шахвердинович
скончался в госпитале от полученных ранений 9 февраля 1945 года. Помним. Чтим

\iusr{Людмила Кривощокова}
Макс Бужанский ,про войну или о войне понятно,а что по существу?!

\iusr{Бадри Шалвович}
Макс ты молодец !!! Уважаю !!

\iusr{Инна Смехова}
И именно завтра - день памяти и скорби 🙏🕊️

\iusr{Наталя Фенога}
До слез .Спасибо .

\iusr{Юлия Печерская}

У меня в роду воевали все, в очень большой семье, Вернулись единицы. Никакого
дня применения ни с кем. Сломаны жизни, но все деды,кто выжил вернулись и по
керпичу поднимали родную землю. Я горжусь каждым своим дедом и прадедом, ни
одного подлеца и предателя, в послевоенные годы, у меня даже бабушки были
бойцами \enquote{фронта труда}. У каждой медаль \enquote{Ветеран труда}. Я
горжусь поколением конца 19 века и начала 20-го.

\iusr{Василий Жиляев}

Класс! Браво!

\iusr{Денис Ковалев}

Че надо сделать простому человеку чтоб не отсасывать не у америки ни у россии а быть нормальной страной защищающей свои интересы?

\iusr{Valery Yurievich}

Все помнят слова Путина, что Россия победила бы в войне и без Украины /которая
потеряла во время войны около 19 процентов своего населения !/? И этот плевок в
прошлое был сделан еще в мирном 2010м году и до сих пор он не покаялся за эти
безумные слова!

\begin{itemize}
\iusr{Misha Mokhovikov}
Valery Yurievich это вы в украинских СМИ прочитали?

\iusr{Valery Yurievich}

Жаль что вы не в курсе, но понятно, что это не вспоминают на российских СМИ ресурсах 
\url{https://www.youtube.com/watch?v=K0FJge8nDxM}

\iusr{Владимир Игнатьев}

\enquote{Россия победила бы в войне и без Украины} и \enquote{Если бы Россия и
Украина были разделены, Россия все равно бы победила} несколько
отличаются по смыслу. Но любителям передёргивать и расчёсывать
выдуманные обидки что либо объяснять бессмысленно.

\iusr{Oleg Raznochenko}

Valery Yurievich на дурацкий вопрос дурацкий ответ. Табаки продолжают тявкать. Дотявкаетесь.

\iusr{Valery Yurievich}

ты меня пугаешь мразь одесская? выйди и скажи это на улице? спрятался под лавку
и шакалиные зубы скалишь?
\end{itemize}

\iusr{Александр Егошин}
Спасибо, Максим!

\iusr{Диана Володина}

Спасибо Вам за Вашу деятельность и низкий поклон. Прошу, не разочаруйте, как
Портнов и Лукаш

\begin{itemize}
\iusr{Макс Бужанский}
Диана Володина , я глубоко уважаю своих друзей Андрея и Елену, никогда в жизни не имел повода разочароваться в них хоть на секунду.

\iusr{Аня Новикова}
Диана Володина чем разочаровали ??!!

\iusr{Диана Володина}
Макс Бужанский И ты , Брут...

\iusr{Макс Бужанский}
Диана Володина , и я, Гай...)

\iusr{Net Irina}
Макс Бужанский а Шарий врёт все про Портнова и квартиры мамы и дочки Лукаш?.
Это ж просто посмотреть декларации и реестры. Просто узнать что Портнов делал в
России в 14м году. Посмотрите и расскажите, мы вам поверим. Скажи мне кто твой
друг.....

\iusr{Диана Володина}
Аня Новикова посмотрите Шария.

\iusr{Макс Бужанский}
Net Irina , я не стану с вами спорить) верьте во что считаете нужным)

\iusr{Ксения Саукова}
Диана Володина ...Шарий просто стал базарной бабой..

\iusr{Александр Трофимов}
Аня Новикова брехнёй своей...создают видимость деятельности. Портнов только и может что в пустую бла-бла-бла

\iusr{Макс Бужанский}
Александр Трофимов , кто то может больше, чем он? Скажите, кто? Или может он не сделал чего то из того, что был должен сделать?

\iusr{Александр Трофимов}
Макс Бужанский я был на него долго подписан...устал слышать пустые обещания закрыть Порошенко...а выходит что это был спектакль.

\iusr{Макс Бужанский}
Александр Трофимов , вам не приходило в голову, что закрытие Порошенко не состоялось вопреки тому, что он для этого делал? Как частное, подчеркну, лицо.
\end{itemize}

\iusr{Tatyana Inozemtseva}

Макс, вы мастер , который позволяет читающему ваши строки, ощутить себя лицом
реально действующих событий!!! Спасибо Вам, что не оставляете эти события в
истории нашей страны! Очень живы в памяти рассказы очевидцев( дедушек,
бабушек), воевавших и в оккупации!! Спасибо за правду, за память,что не даёте
забыть!!!

\begin{itemize}
\iusr{Макс Бужанский}
Tatyana Inozemtseva, спасибо, Татьяна!

\iusr{Tatyana Inozemtseva}

А если провести аналогию... а завтра был 2014 год...  Семнадцать лет работы в
школе, русской,в Днепре , на Парусе! Потом летний лагерь \enquote{голубое
пламя}. Море, солнце, достойная зарплата!! А завтра был Майдан! Я
русскоговорящая! Да и преподавать историю по нынешним учебникам совесть не
позволяет!!! Пенсия ...2200! Говорят, что архив сгорел!!! Как после этого сеять
разумное, доброе, вечное!!! Будущее казалось надежным, спокойным, сытым! А
завтра бал 2014 год!!! Об этом кто-то напишет?! Без претензий лично к вам!

\iusr{Макс Бужанский}
Tatyana Inozemtseva , какой архив у них сгорел, повлияв на вашу пенсию?

\iusr{Tatyana Inozemtseva}
Но слезы не лью!!! Надеюсь, что ваших единомышленников и моих большенство!!! История многому учит, но требует времени !!!

\iusr{Tatyana Inozemtseva}

Макс Бужанский в роно так сказали! Посчитали по минимуму! Не помог звонок на горячую линию меру!

\iusr{Макс Бужанский}

Tatyana Inozemtseva, если нужна какая-то помощь запросом, пишите в личку!

\iusr{Tatyana Inozemtseva}
Макс Бужанский спасибо!!

\end{itemize}

\iusr{Andrey Zakharov}

Макс, Вам бы лекции читать по всему Миру! В том числе и в России. Только понять
не могу, какой идиот поставил Вас в санкционный список, невъездных в Россию?
Или тупо всю партию внесли

\iusr{Людмила Резнікова}

Ви ніби були учасником тих подій, так слізно і пафосно, описали... і ніби не
помітили тих подій, які відбулися на ваших очах, при вас. Майдан, домайданні
підножки рф у вигляді перепон для експорту нашого. Крим, підлість, така ж війна
без прикрас зовсім. Я бережу медалі дідові і бабусин атестат 41 року. Але не
розумію, чому вчепилися за минуле і не хочете бачити таких самих людей в
сьогоденні. Таких самих, які стануть чи ні бабусями і дідусями когось.

\begin{itemize}
\iusr{Александр Шибанов}
Людмила Резнікова Женщина, сколько яда и ненависти надо иметь,чтобы после такой
прозы Макса написать пасквиль с претензией.

\iusr{Людмила Резнікова}
Александр Шибанов розуміння, спроба діалогу. Михальчишина згадайте)

\iusr{Макс Бужанский}

Людмила Резнікова, ещё как заметил, как же не заметить Майдан, убивший нашу
страну. И достаточно подробно описал свое к нему отношение не один и не два
раза.

Но это моя страница, и я пишу на ней о том, что интересно мне, а не вам)

\iusr{Александр Шибанов}

Людмила Резнікова Мне не надо его вспоминать,у меня его портрет всегда с
собой))Он застрял в периоде национально-освободительного движения прошлого
века.

\iusr{Людмила Резнікова}

Макс Бужанский окей. Але є привід подумати, погодьтеся. Гарного вечора.

\iusr{Макс Бужанский}
Людмила Резнікова , гарного вечора!)

\iusr{Аня Новикова}
Людмила Резнікова майдан разрушил нашу страну и нашу жизнь.

\iusr{Александр Шибанов}
Людмила Резнікова Повод подумать есть всегда))

\iusr{Sergej Onopko}
Людмила Резнікова \enquote{домайданні підножки рф}???
Это какие?

\end{itemize}

\iusr{Аня Новикова}

Спасибо Вам! Потеряли в этой войне бабушку и двух дедушек...

\iusr{Дмитро Кучерук}

Довольный обладатель :)
\ifcmt
  pic https://scontent-mxp1-2.xx.fbcdn.net/v/t1.6435-9/203830123_336514434541102_3177267216246709454_n.jpg?_nc_cat=103&ccb=1-3&_nc_sid=dbeb18&_nc_ohc=ckSXbR4MTOAAX8HMBKU&_nc_ht=scontent-mxp1-2.xx&oh=0d7f19bd4ee30fb3b84114e52e1b9f1b&oe=60D6AF16
  width 0.2
\fi

\iusr{Валентина Криворучко}

Господин Макс! Верхушка айсберга точно знала, знали работники НКВД (может не
все). Мой родственник, которого я никогда не видела, знал. Расспросить более
подробно его у моей мамы не получилось (его расстреляли немцы).

\iusr{Сергей Салюк}

Интересно этот вопрос осветил Судоплатов

\iusr{Наталия Андреева}

А меня в преддверии завтрашнего дня радует только один факт: что никто из моих
старших близких родственников НЕ дожил до этого подлого нынешнего времени — до
того, когда так кощунственно начали искажать и перекручивать всё то, что
связано с этой страшной войной. Сейчас даже у меня сердце сжимается и болит от
обиды, от этой подлой лжи, которую нам пытаются навязать, начиная с 2014 года:
а уж каково было бы это слышать тем, кто сам пережил все ужасы Великой
Отечественной (а не Второй Мировой)...

\iusr{Maya Petrujina}

Наталия Андреева и у меня эти же чувства. Страшно сказать, но хорошо, что мама,
папа, дедушка не дожили до этих манкуртовских времен

\iusr{Anton Kropachev}

А у моей бабушки был выпускной

\iusr{Larissa Minkova}
Кстати, семечки: в оккупированной Одессе были запрещены.

\iusr{Надежда Малюкина}
Исторические знания энциклопедический. Восхищаюсь, Макс Бужанский.

\iusr{Iryna Latysheva}

Макс Бужанский, спасибо от дочери/внучки/племянницы фронтовиков из
Днепропетровска и Курска. Родители поженились на фронте. Воевали 10 человек. 2
погибли.4 раненых воевали до Победы.  Светлая БЛАГОДАРНАЯ Память ВСЕМ
воинам-освободителям

\iusr{Наталия Хоружевская}

А сейчас уже можно было две войны завершить...Действительно подлое время.

\iusr{Натали Натали}
\ifcmt
  pic https://scontent-mxp1-2.xx.fbcdn.net/v/t1.6435-9/202944335_501389387865487_6309161904354760445_n.jpg?_nc_cat=106&ccb=1-3&_nc_sid=dbeb18&_nc_ohc=vch8fZ_-61sAX-FFgOB&_nc_ht=scontent-mxp1-2.xx&oh=036695484e198de06ef3cac2c22a43ba&oe=60D77A0C
  width 0.5
\fi



\end{itemize}
