% vim: keymap=russian-jcukenwin
%%beginhead 
 
%%file slova.kosmonavt
%%parent slova
 
%%url 
 
%%author 
%%author_id 
%%author_url 
 
%%tags 
%%title 
 
%%endhead 
\chapter{Космонавт}
\label{sec:slova.kosmonavt}

%%%cit
%%%cit_head
%%%cit_pic
%%%cit_text
В 00 часов 16 минут 30 июня состоялся еще один сеанс связи с кораблем.
Организатор и руководитель подготовки \emph{космонавтов} генерал Николай Каманин
напомнил о необходимости при спуске вести постоянный репортаж на коротких и
ультракоротких волнах. А после посадки действовать строго по инструкции: резко
не двигаться, люк не открывать и ждать группу поиска.  «Желаю мягкой посадки.
До скорой встречи на Земле!» — подытожил Николай Петрович. В ответ командир
экипажа Добровольский сказал: «Вас понял, условия посадки отличные. На борту
все в порядке, самочувствие экипажа отличное. Благодарим за заботу и добрые
пожелания»
%%%cit_comment
%%%cit_title
\citTitle{«Они были обречены» 50 лет назад погиб экипаж «Союза-11». Кто
виноват в главной трагедии советской космонавтики?: Космос: Наука и техника:
Lenta.ru}, Сергей Варшавчик, lenta.ru, 30.06.2021
%%%endcit

%%%cit
%%%cit_head
%%%cit_pic
%%%cit_text
На космодромі Корольов заміняв юним \emph{космонавтам} учителя та батька (як колись у
Харківському колегіумі Григорій Сковорода дбав про своїх вихованців). «Королев
был для нас Богом», — казав космонавт Леонов і, здається, мав абсолютну рацію.
Бо як інакше пояснити той факт, що саме українська пісня першою пролунала в
далекому космосі? Перший \emph{космонавт}-українець Павло Попович під час польоту в
серпні 1962 року на прохання Корольова заспівав «Дивлюсь я на небо, та й думку
гадаю...», а Сергій Павлович у цей час підспівував йому з Землі. Подібну
зухвалість у радянські часи міг собі дозволити лише авторитет, центр, навколо
якого оберталось його оточення, як навколо сонця
%%%cit_comment
%%%cit_title
\citTitle{Про двох хлопців з Житомира: Короленка і Корольова}, 
Олександра Кльосова, day.kyiv.ua, 05.08.2021
%%%endcit
