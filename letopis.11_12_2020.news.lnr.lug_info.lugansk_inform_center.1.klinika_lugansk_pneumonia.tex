% vim: keymap=russian-jcukenwin
%%beginhead 
 
%%file 11_12_2020.news.lnr.lug_info.lugansk_inform_center.1.klinika_lugansk_pneumonia
%%parent 11_12_2020
 
%%url http://lug-info.com/news/one/klinika-meditsiny-katastrof-nachala-osuschestvlyat-reabilitatsiyu-perebolevshikh-pnevmoniei-foto-62697
 
%%author ЛуганскИнформЦентр
%%author_id lugansk_inform_center
%%author_url 
 
%%tags 
%%title Клиника медицины катастроф начала осуществлять реабилитацию переболевших пневмонией
 
%%endhead 
 
\subsection{Клиника медицины катастроф начала осуществлять реабилитацию переболевших пневмонией}
\label{sec:11_12_2020.news.lnr.lug_info.lugansk_inform_center.1.klinika_lugansk_pneumonia}
\Purl{http://lug-info.com/news/one/klinika-meditsiny-katastrof-nachala-osuschestvlyat-reabilitatsiyu-perebolevshikh-pnevmoniei-foto-62697}
\ifcmt
	author_begin
   author_id lugansk_inform_center
	author_end
\fi

\index[rus]{ЛНР!Клиника медицины катастроф, 11.12.2020}

\ifcmt
tab_begin cols=3
	caption Клиника медицины катастроф начала осуществлять реабилитацию переболевших пневмонией, Фото: Марина Сулименко / ЛИЦ

	pic http://img.lug-info.com/cache/a/c/photo_2020-12-11_13-41-14.jpg/1000wm.jpg
	pic http://img.lug-info.com/cache/f/e/photo_2020-12-11_13-41-23.jpg/1000wm.jpg
	pic http://img.lug-info.com/cache/2/7/photo_2020-12-11_13-41-18.jpg/1000wm.jpg

	pic http://img.lug-info.com/cache/7/c/photo_2020-12-11_13-41-18_(2).jpg/1000wm.jpg
	pic http://img.lug-info.com/cache/b/0/photo_2020-12-11_13-41-17.jpg/1000wm.jpg
	pic http://img.lug-info.com/cache/0/6/photo_2020-12-11_13-41-16.jpg/1000wm.jpg
tab_end
\fi



Филиал "Клиника медицины катастроф" государственного учреждения "Луганский
республиканский центр экстренной медицинской помощи и медицины катастроф" начал
осуществлять реабилитацию пациентов, получивших осложнения после коронавирусной
пневмонии. Об этом сообщила заведующая Центром кризисных состояний и
психосоматических расстройств клиники Светлана Шабанина.

"На базе нашего Центра проходит реабилитация и восстановление больных, которые
переболели тяжелой вирусной пневмонией - это, в основном, выраженный
астено-невротический синдром с тяжелыми последствиями, - сказала она. - Клиника
медицины катастроф является сейчас основным и единственным звеном в Республике,
которое занимается реабилитацией больных с постковидным синдромом".

Шабанина сообщила, что одной из основных задач Центра является купирование
осложнений, восстановление трудоспособности граждан, переболевших пневмонией.

Заведующая пояснила, что психосоматические расстройства и другие осложнения
могут возникнуть у пациентов, переболевших пневмоний, вне зависимости от их
возраста и пола. Тяжесть протекания болезни также не всегда является основным
фактором, вызывающим осложнения.

Она отметила, что для каждого пациента подбирается индивидуальный курс лечения
в зависимости от степени и формы осложнений.

"Пациент поступает на дневной стационар, там уже идет дифференцировка, и
пациент может либо продолжить лечение на дневном стационаре, либо поступает ко
мне в отделение (Центра на круглосуточное пребывание), - рассказала Шабанина. -
Обязательно (пациент) проходит курс физиотерапевтических процедур, моей задачей
является психодиагностика и медикаментозная диагностика больных".

Заведующая сообщила, что в среднем курс реабилитации длится от нескольких дней
до трех недель. Она отметила, что в Центре также работает штатный психолог,
который помогает пациентам справиться с психическими расстройствами, вызванными
постковидным синдромом.

Психолог центра Екатерина Рыжевол рассказала, что многие люди, переболевшие
коронавирусом, в особенности его тяжелой формой, получают ряд осложнений, в том
числе психологических.

"Моя стадия оказания помощи является уже закрепляющей, до этого они (пациенты)
работают с врачами (физиотерапевтами). Я работаю с пациентами с тревожными
состояниями: мы снимаем у них и мышечные зажимы, и эмоциональные, и тревогу,
страх", - добавила она.

Психолог сообщила, что первые пациенты, прошедшие реабилитацию в Центре, уже
благополучно вернулись к полноценной жизни. 

\ifcmt
tab_begin cols=3

	pic http://img.lug-info.com/cache/5/e/photo_2020-12-11_13-41-15.jpg/1000wm.jpg
	pic http://img.lug-info.com/cache/c/7/photo_2020-12-11_13-41-15_(2).jpg/1000wm.jpg
	pic http://img.lug-info.com/cache/8/3/photo_2020-12-11_13-41-13.jpg/1000wm.jpg

	pic http://img.lug-info.com/cache/b/4/photo_2020-12-11_13-41-12.jpg/1000wm.jpg
	pic http://img.lug-info.com/cache/3/5/photo_2020-12-11_13-41-12_(3).jpg/1000wm.jpg
	pic http://img.lug-info.com/cache/7/d/photo_2020-12-11_13-41-12_(2).jpg/1000wm.jpg
tab_end
\fi

Ранее Министерство здравоохранения ЛНР
приобрело\Furl{http://lug-info.com/news/one/minzdrav-lnr-priobrel-kompyuternyi-tomograf-dlya-kliniki-meditsiny-katastrof-foto-62507}
для клиники медицины катастроф компьютерный томограф.

Всемирная организация здравоохранения (ВОЗ) объявила о пандемии болезни
COVID-19, вызываемой коронавирусом нового типа, первые случаи заражения которым
были выявлены в конце 2019 года в Китае. К настоящему времени, по данным ВОЗ,
количество заразившихся новым коронавирусом превысило 68,8 млн, из них более
1,5 млн скончались. В ЛНР подтверждены 1865 случаев заражения коронавирусом,
155 заразившихся скончались.

С целью недопущения распространения коронавирусной инфекции власти Республики
ввели режим повышенной готовности и приняли ряд мер, в том числе по изменению
режима работы госорганов, предприятий, учреждений, транспорта.

ЛуганскИнформЦентр — 11 декабря — Луганск

