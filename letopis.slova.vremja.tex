% vim: keymap=russian-jcukenwin
%%beginhead 
 
%%file slova.vremja
%%parent slova
 
%%url 
 
%%author 
%%author_id 
%%author_url 
 
%%tags 
%%title 
 
%%endhead 
\chapter{Время}
\label{sec:slova.vremja}

%%%cit
%%%cit_pic
%%%cit_text
Так и вымрут. Слава!  Вывод. Думаю будет логичным и стройным. На заре
\emph{Времён}, псеродактили отложили потомство в яйцах, тем самым зародили
новую нацию. Нация тысячи лет принимала вертикальное положение и так утомилась
этим действом, что решила не заморачиваться (или это природа, глядя на мучения
и потуги) в развитии дальше и остановилась на стадии когда хватало для жизни
донашивать за кем-то ношеные шкуры (секонд хенд рулит), греться халявным теплом
и доедать за кем-то недоеденного мастодонта. Как-то умудрились попасть в щель
пространственно-временного континуума и вывалится всем нам на головы огромной
геморройной проблемой. Требую узаконить эту версию как единственно верную и
придать ей статус догмы. Сла... ой, АМИНЬ! ЦИНИК
%%%cit_comment
%%%cit_title
\citTitle{От \enquote{псеродактиля} в \enquote{андертальца} и далее в венец человечества: рассуждения никудышного антрополога}, 
Дмитрий Жук, zen.yandex.ru, 13.06.2021 
%%%endcit

%%%cit
%%%cit_pic
%%%cit_text
Нам нужна иная управленческая элита. Прагматичная. Рациональная. Компетентная и
ответственная. Не похожая на тех послушных болванчиков, которые сегодня
обслуживают стратегические интересы внешних игроков. Патриотов Украины,
способных на своем внутреннем калькуляторе просчитывать наши потенциальные
риски и потенциальные выгоды. Профессионалов, для которых понятия «общее благо»
и «интересы Украины» не будут пустым звуком. Справятся ли с этой задачей
украинские университеты? Или надежды оправдает Президентский университет?
Покажет \emph{Время}...
%%%cit_comment
%%%cit_title
\citTitle{Зеркальный потолок Украины}, Андрей Китаев, 
analytics.hvylya.net, 14.06.2021
%%%endcit

