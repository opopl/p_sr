%%beginhead 
 
%%file 20_04_2022.fb.golovnjova_maryna.mariupol.1.maibutn__ukra_ni___v
%%parent 20_04_2022
 
%%url https://www.facebook.com/m.holovnova/posts/pfbid0339QDpVdJGcj89B7DuikqWPSjLJzRjdjPkHCYi6ZVp2usrknJMT3PD37Ken7njUfol
 
%%author_id golovnjova_maryna.mariupol
%%date 20_04_2022
 
%%tags mariupol,mariupol.war,dnevnik
%%title Майбутнє України – в руках молодих
 
%%endhead 

\subsection{Майбутнє України – в руках молодих}
\label{sec:20_04_2022.fb.golovnjova_maryna.mariupol.1.maibutn__ukra_ni___v}

\Purl{https://www.facebook.com/m.holovnova/posts/pfbid0339QDpVdJGcj89B7DuikqWPSjLJzRjdjPkHCYi6ZVp2usrknJMT3PD37Ken7njUfol}
\ifcmt
 author_begin
   author_id golovnjova_maryna.mariupol
 author_end
\fi

Майбутнє України – в руках молодих

Блокадні ночі ми проводили в квартирі п'ятиповерхового будинку. Підвалу в ньому
не було. Після 8 березня центр Маріуполя почали обстрілювати практично без
зупину, снаряди влучали у сусідні будинки, від уламків гинули люди, горіли
квартири і машини на подвір'ях. Пекельне смертельне коло звужувалося.
Залишатися на четвертому поверсі було божевіллям. 

15 березня ми взяли свої рюкзаки, залишки продуктів, води і пішли шукати новий
прихисток. Обійшли навколишні будинки з підвалами. З одного вийшов чоловік і
сказав нам, що місця в них уже немає, і якщо ми намагатимемось спуститися в
їхній підвал під час бомбардувань, то він особисто стрілятиме нам по ногах, як
в мародерів. Рекомендував йти в Драматичний театр. Але там уже був зайнятий
кожен метр на всіх трьох поверхах і в підвалі. Волонтери перестали пускати
новоприбульців.

Я згадала про масивний індустріальний коледж неподалік. Це двоповерхова споруда
кінця ХІХ ст., з арочним входом і товстезними стінами з темної цегли. Перша
чоловіча гімназія, відкриття якої змінило хід історії міста. Я любила
розказувати про неї на своїх екскурсіях і завжди хотіла потрапити всередину,
побачити вітражі та барельєфи. 

Ми пішли перевірити, чи ховалися там люди. У холі були величезні вікна з
дерев'яними рамами, деякі уже без скла. Переміщатись від одного кінця коридору
до іншого треба було бігом, аби у разі близького прильоту не потрапити під
убивчі уламки скла. 

На другому поверсі висів стенд із написом «Майбутнє України – в руках молодих».
Ми обійшли все. Стало очевидно, що від початку війни у коледж взагалі ніхто не
заходив. Для нас він став останнім прихистком у Маріуполі. 

Ми залишился там увісьмох з нашими друзями, зайняли маленьку кімнатку без вікон
під сходами. Вона належала прибиральниці: на стіні висіло кілька іменних грамот
за її взірцеву роботу. На столику лежав акуратно заповнений журнал з останнім
записом за 23.02.2022. Там же знайшлася напівпорожня банка розчинної кави.
Дурниця, але це тоді було за таке щастя! Ми зробили вогнище у внутрішньому
дворику, скип'ятили воду і дозволили собі кожен по чашечці.

Швидко стемніло, і ми зловили себе на думці, що над Маріуполем ніколи не бувало
настільки чистого зоряного неба. Наші металургійні гіганти зупинились майже три
тижні тому і більше не запрацюють.

Вночі було нестерпно холодно, і гриміло, як ніколи. Ніхто з нас не спав. Десь о
другій ночі чергова бомба впала зовсім поряд, за 200 метрів. Це росіяни влучили
в Центральний універмаг на проспекті Миру. Масивна будівля коледжу стояла
міцно, як фортеця. Уперше за багато ночей стіни навколо нас не тряслися, і це
створювало хоча б ілюзію захищеності. Втім, вікна в холі не витримали, і з
останніх вцілілих на підлогу дзвінко посипалося скло. На ранок усе було вкрите
гострими уламками.

Наступного дня росіяни скинули 500-кілограмову бомбу на театр. Коло замкнулося.
Безпечних місць більше не залишилось.

%\ii{20_04_2022.fb.golovnjova_maryna.mariupol.1.maibutn__ukra_ni___v.cmt}
