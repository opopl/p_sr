% vim: keymap=russian-jcukenwin
%%beginhead 
 
%%file 16_12_2021.yz.maj_dnr.1.svjaz_telefon_kod_7
%%parent 16_12_2021
 
%%url https://zen.yandex.ru/media/id/5f8f226b1fe36c1d9e02a36b/pozvoni-mne-pozvoni-donbass-skoro-pereidet-na-7-61bb70798f920a4fc4c92c7a
 
%%author_id 
%%date 
 
%%tags 
%%title Позвони мне, позвони: Донбасс скоро перейдет на +7!
 
%%endhead 
\subsection{Позвони мне, позвони: Донбасс скоро перейдет на +7!}
\label{sec:16_12_2021.yz.maj_dnr.1.svjaz_telefon_kod_7}

\Purl{https://zen.yandex.ru/media/id/5f8f226b1fe36c1d9e02a36b/pozvoni-mne-pozvoni-donbass-skoro-pereidet-na-7-61bb70798f920a4fc4c92c7a}

Для тех, кто еще думает, что в Донбассе осталось что-то украинское, этот текст.

В 2014 году мой телефон (ЮМС) глобально прослушивался, и это мешало. Не в
смысле, что я боялась о чем-то говорить, но телефон все время тормозил, там
что-то шуршало, иногда разговор прерывался, а лежащая на столе в спокойном
состоянии трубка регулярно попискивала и потренькивала. Реально понимаю, что
это было, скорее, не прослушивание на предмет выяснить что-нибудь важно, а
просто давление на нервы. Мол, СБУ не спит и все такое. Остальное было не так
важно – мы уже летом приучились ничего серьезного по телефону не говорить. Но
это я. А были люди, которым подобна прослушка была в тягость. Мало того,
некоторые мобильные операторы вообще отключили свои коммутаторы и перекрыли
связь – например, Киевстар и Лайф. Однако, я сейчас не об этом.

\ifcmt
  ig https://avatars.mds.yandex.net/get-zen_doc/3323992/pub_61bb70798f920a4fc4c92c7a_61bb71990693f87b5428c65b/scale_1200
  @width 0.4
  %@wrap \parpic[r]
  @wrap \InsertBoxR{0}
\fi

К осени 2014 года стало понятно, что Республике будет нужна собственная связь,
свободная от опасности быть внезапно оборванной, от осознания тотального
контроля СБУ, да и вообще, любое самодостаточное государство начинает свое
строительство с надежной связи.

Я не знаю, кто именно придумал название нашему мобильному оператору – Феникс –
но уже в мае 2015-го мы точно знали, «Фениксу» быть и быть гораздо скорее, чем
мы даже могли мечтать.

Свою заветную сим-карту мобильного оператора «Феникс» я получила аккурат к
Новому 2016 году и очень быстро перевела все свои разговоры на нашу,
республиканскую связь. ЮМС-шный номер оставался в моем телефоне до тех пор,
пока все мои друзья и знакомые не обзавелись Фениксами, а это было очень
быстро. Потому и украинская карточка скоро оказалась в помойке.

Кстати, на многих наших бывших номерах потом появились мошенники. Например,
украинского номера моей подруги у ее контактов просили занять денег, мол,
переведи на карточку, а я тебе к вечеру подвезу наличку. Вы же понимаете,
номера мы не теряли, телефоны тем более, а потому писать от имени Лены Тане:
«Танюш....» - т.е. именно так, как Лена ее называла при разговоре по телефону....
Словом, понятно, что это СБУ вот таким способом отрабатывало свой плесневелый
хлеб, обдирая доверчивых «Тань» из Донецка. Но и это скоро прекратилось. Феникс
взял уверенный верх над украинскими номерами, да и то сказать, после нескольких
публикаций об этом способе мошенничества украинская спецура угомонилась.
Началась друга – с воровством аккаунтов, привязанных к ЮМС, но об этих
«приключениях» я расскажу как-нибудь в другой раз.

\headTwo{Километровые очереди}

Вы не представляете себе, какой поначалу ажиотаж был с сим-картами Феникса! Его
захотели все, и ринулись на почты и в главный офис оператора. Гребли, пачками,
стояли в очередях, ругались и снова стояли в очередях. Фениксу даже пришлось
установить лимит – не более 5 симок на один паспорт. Да, да, с самого начала
номера присваивались персонифицировано, по паспорту. И никак иначе. Слава Богу,
это правило сохраняется до сих пор, хотя ограничение уже снято – бери сколько
хочешь. А я хорошо помню, что говорила всем и каждому: «У нас, как в России,
хочешь номер телефона – предъяви паспорт». Кстати, всегда считала российский
опыт единственно правильным, а украинскую анонимность никогда не понимала. Так
вот, к концу 2016 года Феникс был у всех жителей Донбасса. Или почти у всех.
Связь была дешевая: стартовый пакет – 120 рублей, месячная абонплата - 50
рублей, внутри сети звонки бесплатны, да еще на месяц 300 бесплатных смс.
Звонки на украинские номера и в Россию были платные, но тоже стоили копейки,
например, на любой российский номер 5 рублей минута. Пополняли счет
скретч-картами, в любых отделениях «Почты Донбасса» и в офисе Феникса.

\headTwo{Говори сколько хочешь или «Связь для Победы»?}

Не могу сказать, что поначалу все было просто – к своему открытию Феникс
покрывал только 75\% территории ДНР, а может и того меньше. И интернет поначалу
был слабый – статью напишешь, но видео не передашь. Но это был наш,
государственный оператор связи, и даже первый тариф так и назвали – «Народный».
И еще он рос, как на дрожжах. И покрытие, и интернет. К 2019 году в Республике
работало уже 80 базовых станций 4G-LTE, на январь 2020 года количество базовых
станций составило уже 190 штук. 2020-й и 21 тоже не отстали. Так что очень
скоро в Республике не будет закоулка, где мой телефон бы не чихнул, принимая
смс или извещая о новом посте в Телеграм (он у меня чихаетJ). О преодолении
препятствий рассказывать долго. Иногда случаются подвисания иногда тяжело
дозвониться, из-за чего добрые дончане, в шутку, называют Феникс «Связь для
Победы» или ставят приставку «до», но скептикам и ругателям я бы напомнила,
сколько лет прошло от первого мобильного телефона – у моего мужа была такая
чудо-труба – до современной связи в других странах. 5 лет для построения
полноценной мобильной сети – это очень короткий срок. Невероятно короткий срок,
давайте будем честными. И откровенно говоря, у меня сейчас крайне редко бывают
ситуации, когда я ворчу на связь или на интернет.

\headTwo{Что сейчас и что будет дальше}

В ноябре этого года Феникс начал реконструкцию - заменил первичное ядро сетевой
связи, что увеличило ее мощность. Кстати, именно во время этих работ посыпались
жалобы на качество связи. Но слава Богу, проблемы были недолгими, зато
улучшение качества связи я заметила – на днях была в отдаленном районе
Макеевки, где интернет раньше не тянул даже текст, а теперь спокойно залила
видео на свой Ютуб. Феникс обещает, что 4G очень скоро будет везде.

\headTwo{От рублевой зоны к «Направлению на +7»}

Еще летом Феникс объявил, что в Донбассе пройдет тестирование связи с
российским кодом +7. Но до этой недели об этой теме ничего не было слышно. И
вот, на днях директор Феникса объявил, что работы в направлении +7 начаты.

Я прекрасно понимаю, что для полноценной работы российского телефона нужно
время. Но вряд ли бы о этом кто-то говорил, если бы на территории ДНР
российская связь была бы в принципе невозможна.

А это значит, набираемся терпения. В 2014 году мы не очень-то представляли
себе, что в 2015 будем расплачиваться в магазинах рублями, а в 2017-м вообще
забудем, как выглядят украинские гривны. Полагаю, совсем скоро, услышав
название «Киевстар» или «Водафон» мы спросим: А что это такое?

У нас уже давно все российское, а главное – русская душа. Дело за малым – за
«+7».....
