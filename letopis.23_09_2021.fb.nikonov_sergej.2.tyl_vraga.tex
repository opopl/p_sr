% vim: keymap=russian-jcukenwin
%%beginhead 
 
%%file 23_09_2021.fb.nikonov_sergej.2.tyl_vraga
%%parent 23_09_2021
 
%%url https://www.facebook.com/alexelsevier/posts/1587741244904518
 
%%author_id nikonov_sergej,bilchenko_evgenia
%%date 
 
%%tags bilchenko_evgenia,poezia
%%title БЖ. Заброшенная в тыл врага
 
%%endhead 
 
\subsection{БЖ. Заброшенная в тыл врага}
\label{sec:23_09_2021.fb.nikonov_sergej.2.tyl_vraga}
 
\Purl{https://www.facebook.com/alexelsevier/posts/1587741244904518}
\ifcmt
 author_begin
   author_id nikonov_sergej,bilchenko_evgenia
 author_end
\fi


\ifcmt
  ig https://scontent-lga3-1.xx.fbcdn.net/v/t1.6435-9/242488393_1587741084904534_9141635541257036932_n.jpg?_nc_cat=101&ccb=1-5&_nc_sid=730e14&_nc_ohc=1CV53h1SpsgAX9miMJE&_nc_ht=scontent-lga3-1.xx&oh=e061e516b462c484e29235911937ad2f&oe=61732F8D
  @width 0.4
  %@wrap \parpic[r]
  @wrap \InsertBoxR{0}
\fi

Снова стихотворение Евгении Бильченко. Поэт творит, сочиняет  стихи... , а я их репощу. 
БЖ. Заброшенная в тыл врага
У меня день рождения через две недели.
Если я не умру, надо, чтобы чёрное на меня надели.
Муж запрещает мне уходить в монахини под Донецком.
Я мечтаю уметь стрелять и веду себя так по-детски,
Что надо мной насмехается вся восхитительная толпа
Красных и белых, левых и правых, от беса и от попа.
Я хочу научиться нормально так попадать в мишень,
А не сидеть в Берлине и в Праге в кондитерской "Ля Мишель".
Я хочу на свою днюху рабочие боты и камуфло,
Я поняла, что зло побеждает только бабло,
А если его не хватает, сойдёт и маузер,
Бабушка, ты меня научила целиться в прыщик мазями.
Неправильный выбор. Неверный шаг. Какой из меня Штирлиц,
Если рожа - моя шторм, как им покажешь штиль лица?
Как уходить, не хлопнув дверьми, с Вавилона транснационалов?
Победить словами - их ад нельзя, можно купить их налом
И через счёт банковский, но лучше ракетой - сразу.
Товарищ священник, отче, исповедуй во мне заразу.
У меня день рождения, я не хочу жить с любимым, жить с с нелюбимым:
Они же умрут рядом со мной, цивилы и пилигримы.
Когда утром я просыпаюсь, на часах моих - час седьмой.
Я работаю на господина, гонорар - господин мой.
Я забыла свое имя ученого и писателя, и отныне
Мною владеет адская смесь смирения и гордыни.
Но я куплю себе берцы - пацанские сапоги.
Я встану с третьей, вчера оторванной на растяжке, чужой ноги,
Я облачусь в ризы ночные, такой у меня праздник.
Нафиг ступай, пренебречь, вальсируем, конченый мир-лабазник.
Не могу больше ждать Питера. Я умерла в блокаду.
Мама Таня дала мне хлеб на окраине Ленинграда.
Кот умер от голода: знала бы, накормила.
В кафе на Невском, где жрут эклеры, уже ничего не мило.
А здесь деревья стоят, как встарь, - ласковые, пушистые:
Каштаны растут в Украйне пышно при всех фашистах.
Я виновата, да, виновата, - навылет, навзрыд, навыкат.
Я убью себя здесь, ибо войска нет. И я не боюсь.
А вы-то?
Как вы там без меня? Надоело, спасать этот край холеры?
Для веры в Победу без оснований давно не хватает веры?
Конечно, я понимаю. Я всё теперь понимаю.
Я так люблю вас, что задушу, - недаром не обнимаю.
21 сентября 2021 г.
