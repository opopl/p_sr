% vim: keymap=russian-jcukenwin
%%beginhead 
 
%%file moje.kremlevskie_narrativy.jazyk
%%parent moje.kremlevskie_narrativy
 
%%url 
 
%%author_id 
%%date 
 
%%tags 
%%title 
 
%%endhead 

\subsubsection{Положение русского языка в Киеве, в частности, и вообще в Украине}

Тут, к сожалению, ситуация совершенно удручающая с точки зрения поддержки
государством, на содержание которого мы исправно платим налоги. Язык, на
котором писал свои сочинения Григорий Саввич Сковорода, наш великий
философ-просветитель, который в свое время учился в Киевской Академии, и чей
памятник стоит на Подоле, как раз возле Андреевского Спуска, - по факту
систематически изгоняется из сферы образования, науки, культуры. Ну, точнее его
пытаются изгнать, унизить на государственном уровне, хотя априори очевидно, что
это абсолютно бессмысленная и глупая затея, поскольку Украина, - родина Николая
Гоголя и Тараса Шевченка, - как говорила на двух родственных языках, - на
русском и на украинском, так и будет говорить. Так было вчера, так есть сейчас,
и так будет всегда. Но... есть хорошая пословица - научи дурака молиться, и он
лоб расшибет об стену, и у нас действительно есть много дураков, которые
несмотря ни на что, продолжают заниматься этой глупой затеей, вместо того чтобы
тратить свои духовные силы на что-то более творческое и созидательное.

Так вот, знаете, сейчас как раз уже 300 лет нашему
великому философу, и это поразительно, просто поразительно, что язык, на
котором он писал о самых сокровенных вещах - Человеке, Боге, Душе -
систематически оплевывается и унижается. Некоторые недалекие люди говорят, а
иные даже исступлено кричат, испуская потоки желчи по бескрайнему интернету,
что это язык московских оккупантов, говорят, что это московский (москвинский,
мокшанский) язык. Чушь собачья! Русский язык, раз уж вспомнили о Сковороде, это
- прежде всего язык Сковороды, поскольку на 90 процентов или более язык его
произведений совпадает со современным русским языком. Не верите? Ну так
почитайте сами, откройте в оригинале Сад Божественных Песен, глаза ж вам даны
не только для того, чтобы смотреть передачи ТСН, не так ли? Конечно, в его
произведениях есть примеси украинских (малороссийских) слов, и грамматические
конструкции немного старомодны, все-таки писал он более двух столетий назад; но
по факту, язык его произведений - с нашей (Киевской) точки зрения, это самый настоящий русский язык, это уж
будьте уверены. И значит, если унижают русский язык, то по факту унижают язык
Григория Саввича, а значит, это плевок в самого Сковороду.  Задумайтесь об
этом, господа патриоты. Вот, например, песнь 18-ая, красиво, правда?

\raggedcolumns
\begin{multicols}{2} % {
\setlength{\parindent}{0pt}
\obeycr
Ой ты, птичко жолтобоко,
Не клади гнезда высоко!
Клади на зеленой травке,
На молоденькой муравке.
\smallskip
От ястреб над головою
Висит, хочет ухватить,
Вашею живет он кровью,
От, от! кохти он острит!
\smallskip
Стоит явор над горою,
Все кивает головою.
Буйны ветры повевают,
Руки явору ломают.
\smallskip
А вербочки шумят низко,
Волокут мене до сна.
Тут течет поточок близко;
Видно воду аж до дна.
\smallskip
На что ж мне замышляти,
Что в селе родила мати?
Нехай у тех мозок рвется,
Кто высоко в гору дмется,
\smallskip
А я буду себе тихо
Коротати милый век.
\smallskip
Так минет мене все лихо,
Щастлив буду человек.
\restorecr
\end{multicols} % }

\ifcmt
tab_begin cols=2
  @caption Памятник Григорию Сковороде в городе Киеве

  ig https://avatars.mds.yandex.net/get-altay/2776464/2a00000170f4a99f4b8134bb851eb03ec81d/XXL
	ig https://ic.pics.livejournal.com/s_a_plotnikov/85688553/71932/71932_original.jpg

tab_end
\fi

Далее, что касается русского языка, является ли он своим или чужим, мы скажем,
что русский, - это один из двух основных Киевских языков, да, языков Города
Киева, наравне с украинским, на котором, кстати, написан гимн Киева, Як тебе не
любити, Києве Мій. Ничуть не желая унизить или умалить украинский язык (мы про
него напишем в другом месте), мы тем не менее скажем, что русский - это Язык,
великий язык на котором говорят миллионы Киевлян каждый день. В разговорах в
семье, на улицах; в школах и на свиданиях; во время телефонных разговоров и
также во время разговоров по скайпу, во время переписки по вайберу или по
телеграму. На русском языке в Киеве говорят уборщицы и депутаты Верховной Рады;
говорят сотрудники СБУ и врачи скорой; говорят везде, и тут, и там. В
маршрутках говорят и в метро тоже; говорят в институтах Академии Наук, и в
Милицейской Академии тоже; Говорят зимой, и летом; осенью, и весной. На русском
языке матерится Петр Порошенко, когда ему вручают повестку, и на русском же
языке Антон Геращенко призывает россиян остановить Путина. Да, представте себе,
дамы Фарион и Ницой, в Киеве говорят на прекрасном, удивительном русском языке,
и воистину поразительно, что нам нужно вообще об этом писать и напоминать,
поскольку обилие и разнообразие разговорного и письменного русского языка в
Киеве - это очевидный факт. Киев - это возможно вообще самый богатый русским
языком Город в мире, и неудивительно, - поскольку в Киеве была же первая библиотека, первые школы
на Руси, и первый свод законов, - Русская Правда... И именно Киевляне - а
Киевлянин это не только тот человек, кто родился здесь, - а в более широком
смысле - это вообще любой человек, кто душевно связан с Киевом, кто любит Киев,
кому Киев дорог, - Булгаков, родившийся здесь, Гоголь, мечтавший занять кафедру
при Киевском Университете, Пушкин, бывавший здесь в юности, Высоцкий, гостивший
в детстве у своей киевской бабушки, - и многие другие - преображали русскую
культуру и язык, каждый по-своему выводя ее на новый уровень.  На русском языке
поют Киевляне Потап и Настя, на русском языке долгие годы выступает Студия
Квартал 95; на русском языке был снят замечательный сериал Слуга Народа - к
сожалению, он был использован Зеленским в итоге не в лучшую сторону... Молодой
талантливый русскоязычный актер, пришедший на волне народных чаяний, так изящно
и смешно пародировавший украинских политиков, и предавший, - увы, -
впоследствии все идеалы, на которых он пришел к власти...    

Русский язык абсолютно Киеву органичен и Киеву совершенно свой. На русском
говорят и пишут Антон Геращенко и Арсен Аваков, Петр Толочко и Олеся Медведева,
Алексей Арестович и Борислав Береза, Олег Волошин и даже Юрий Бутусов. На
русском языке был написан призыв Мустафы Найема, с которого начался Майдан-2, а
значит, русский язык, - это язык Майдана.  На русском языке говорили воины
Красной Армии, которые ценой огромных жертв освободили Киев, и на русском языке
писал свои книги Михаил Булгаков. Это наш - Киевский язык, корни которого идут
прямиком в Киевскую Русь, во времена, когда на стенах Софиевского Собора
простые Киевляне оставляли свои надписи. Вы хотите сказать, что это все
староукраинский язык, а русский язык - это миф путинской пропаганды, что это
все навязанное, не свое, родное. Что была именно Украина-Русь, и именно так, а
не иначе.  Ну а мы тогда ответим, что Украина-Русь - это какой-то непонятный
конструкт, какой-то вообще выдуманный мираж. Был Киев, Стольный Град Киев,
который столько раз грабили и уничтожали, а он все возрождался, Киев,
объективный, реальный, вот он - здесь - прямо под ногами! Град Киев, куда язык
человеческий вел, стоило только спросить...  а как нам добраться до Киева, не
подскажете? И тут же путнику охотно указывали, - сначала пройди вон тот лес, -
ты ж не балбес? - потом поверни направо, потом - налево, потом будет речка с
русалками, - потом будет еще один лес, - только гляди, такой тебе совет, - не
попади ты волкам на обед, - ну аж потом ты наконец-то увидишь Киев...  И он
здесь стоит уже тысячу лет, наш Киев, в котором были киевляне и князья, бояре и
смерды, тиуны и дружинники. И все это было связано со Псковом и Новгородом,
Суздалем и Ярославлем, и другими городами Северо-Восточной Руси.  И все это
вместе в массовом сознании было Русь. Просто Русь. А потом, - постепенно Русь
разделилась, раздробилась, раскололась на части, и каждая часть пошла своей
дорогой, - одна путем вольницы и освободительных войн, а другая - путем
жесткой, централизованной государственности и имперскости, - потом соединившись
вместе, а потом снова в наши времена разойдясь... А пресловутая Украина-Русь?
Вот именно так, Украина-Русь, или же Русь-Украина, через дефис, и никак иначе,
а то Грушевский на пару с Арестовичем обидятся?  Да чепуха это все. Был Киев -
а потом все остальное. И что сейчас важно, по-настоящему важно, это не то,
насколько именно сегодняшний Киевский русский язык подобен языку графити Софии
Киевской, а то, что Киев сейчас является невероятно замусоренным, загрязненным,
разбитым городом, чье нынешнее запущенное состояние (можно еще вспомнить про
Чернобыль рядом) абсолютно не соответствует его истинной значимости, являясь
наглядным примером порочности и примитивности нынешней украинской
государственной идеологии, а также культурного обнищания и духовной деградации
украинского народа в целом.

\ifcmt
  tab_begin cols=3,no_fig,center

     pic https://gordonua.com/img/article/15213/29_big.jpg?v1601729183
		 pic https://avatars.mds.yandex.net/i?id=515efdebec88624056a69b00ecc6e3b0-5886792-images-thumbs&n=13
		 pic https://avatars.mds.yandex.net/i?id=a951e988dcb2b7168b7301d363197d0a-5747194-images-thumbs&n=13

  tab_end
\fi

Далее, касательно притеснений русского языка и в Киеве, и по Украине, это
действительно правда, и Кремль здесь особо ни причем, поскольку же именно ты,
государство Украина, управляющее Городом Киевом последние 30 лет, хорошо
постаралась в этом, закрывая русские школы, вводя глупые украинизаторские
законы, и вообще систематически унижая русский язык - кровный язык Киева! - как
второстепенный, иностранный язык.  Знаешь, Украина, есть такая пословица:
насильно мил не будешь. Это касательно развития украинского языка. К сожалению,
тоталитарная репрессивная политика в отношении статуса русского языка и русской
культуры не привела к настоящему, духовному подъему украинской культуры и
языка, как ты хотела, не привела. Совсем наоборот.

\ifcmt
  tab_begin cols=3,no_fig,center
     pic https://i2.paste.pics/c4004e2585224bc31f681c37e8d849bc.png
		 pic https://i2.paste.pics/5de7ab7812afc20b9788023f41992509.png
		 pic https://i2.paste.pics/b7b038ded8df6885ce858fc8fc4054af.png
  tab_end
\fi

Alas! Украинский язык - певучий, удивительный язык, - проникающий в самую глубь
сердца, озаряющий самые потаенные уголки души, - стоит только открыть томик
Леси Украинки или же Олександра Олеся, или послушать какую-нибудь песню
Русланы, - ну например, старую добрую песню Свитанок, или же, Знаю Я -
превратился в государственный фетиш, в мертвый символ, в канцелярский язык, в
язык, на котором заставляют говорить насильно, в язык, который многие люди
вообще абсолютно искренне, от души ненавидят, поскольку он ассоциируется с
ненавистью, грабежами, убийствами и насилием, творимым незаконными военными
действиями ВСУ, а также с людоедскими заявлениями Фарион, Дроздова, Ницой, и их
многочисленных единомышленников, которые готовы резать москалей и малороссов,
которые повсюду видят предателей, живя во мраке мысленных чудовищ, порожденных
ими самими... Знаешь, Украина, апостол Павл сказал когда-то, что, \enquote{если
буду говорить красиво, но не буду иметь любви, то буду подобен меди звенящей},
- то есть, даже если ты говоришь самым изысканным украинским языком, но при
этом ты зол и глуп, и душа твоя пуста и не имеет сострадания и сочувствия, а
только бесконечное - росия-ворог, путин..., москалей-на-ножи, то какой толк
тогда в твоем украинском? Как ты думаешь, Украина, если бы например Леся Украинка перенеслась
машиной времени в наше время, она бы тоже орала кричалки, которые орут фанаты
на матче Украина-Россия?

\ifcmt
  tab_begin cols=3,no_fig,center

     pic https://avatars.mds.yandex.net/i?id=2a00000179f5e312985d9263bd58639efd1b-4613271-images-thumbs&n=13
		 pic https://ye.ua/images/news/poster_photo_title-1496732268.jpeg
		 pic https://www.megakniga.com.ua/uploads/cache/Products/Product_images_474227/0c4864_w600.jpg

  tab_end
\fi

Украинский язык в итоге стал скучным пропагандистским орудием... Он потерял
живительную силу сердца и души, потерял то, благодаря чему он жил и развивался
столетиями, несмотря на все преграды, превратившись в итоге в средство унижения
и ненависти. Парадоксально, но факт. Лучшие произведения на украинском языке
были созданы в те времена, когда Украина не имела своей государственности,
будучи разделенной между империями. Именно в те времена жили и творили Леся
Украинка, Иван Франко, Тарас Шевченко и многие другие. Получив же
независимость, Украина дала государственный статус украинскому языку, пнув при
этом ногой русский. И что в итоге получилось, вы и так знаете. И украинские
интеллектуалы, да и просто обычный народ, особенно его \enquote{патриотическая}
часть в массе своей загнали себя в тупик самого слова \enquote{Украина},
разделив свою реальность на Украину и Россию, проукраинские и пророссийские
нарративы, украинцев, малороссов, какаяразница, украина-понад-усе и тому
подобное... загнав поэтому и сам украинский язык в ловушку, в своего рода
информационное гетто, лишив его живительной силы сердца и искренности, разума и
мудрости. А ведь украинский язык же не появился просто так! Он же ж именно
такой и есть певучий и красивый, поскольку в давние времена люди хотели вольно
жить на своей земле, хотели любить и жить в гармонии с природой, - но им
приходилось все время воевать и сражаться за право быть на своей земле, - и
отсюда красота языка, \enquote{смачность} языка,  и также божественная красота
украинских песен, - как отображение внутренних волевых, чистых порывов души и
сердца. Вот, например, такая есть песня с комментариями под видео... 

\href{https://www.youtube.com/watch?v=--iWRp8Jr6k}{%
Українські пісні онлайн. Чом ти не прийшов, youtube}

\raggedcolumns
\begin{multicols}{4} % {
\setlength{\parindent}{0pt}
\obeycr
Чом ти не прийшов,
Як місяць зійшов?
Я тебе чекала.
Чи коня не мав,
Чи стежки не знав,
Мати не пускала?
\smallskip
І коня я мав,
І стежку я знав,
І мати пускала.
Найменша сестра,
Бодай не зросла,
Сідельце сховала.
\smallskip
А старша сестра
Сідельце знайшла,
Коня осідлала:
\enquote{Поїдь, братику,
До дівчиноньки,
Що тебе чекала}.
\smallskip
Тече річенька
Невеличенька,
Схочу - перескочу.
Віддайте мене,
Моя матінко,
За кого я схочу.
\restorecr
\end{multicols} % }

\ifcmt
  tab_begin cols=2,no_fig,center
     pic https://i2.paste.pics/39b5912a65683fffc41863c68bfc5f26.png
		 pic https://i2.paste.pics/dd441b74a982c23a31f5cd313bc89e3d.png
  tab_end
\fi

\begin{itemize} % {
\item Боже, как я люблю украинские песни. Я сразу возвращаюсь в детство, когда все
еще были живы-здоровы, а когда за столом собирались родственники..., украинские
песни лились рекой одна за другой.... как же было здорово!!!

\item Процветания тебе, Украина  @igg{fbicon.heart.sparkling} ... с Любовью и
уважением из России...

\item Люблю Украину и украинские песни. Они самые лучшие, душу выворачивают. Слова и
музыка завораживает. Класс!!!!!!!!!

\item Песня звучит, как звучат струны души у прекрасных и красивый людей. Хочется
слушая, подпевать, хотя  не  всё слова понимаю.  @igg{fbicon.hands.pray} 

\item Я сама русская. Но очень люблю украинские песни.

\item молодцы только украинцы так поют сказочно

\item Слушаю, подпеваю с удовольствием! У меня русско-украинские корни. Люблю Россию,
люблю Украину!!! Господь все управит, все у наших народов будет хорошо!!!
\end{itemize} % }
