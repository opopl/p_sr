% vim: keymap=russian-jcukenwin
%%beginhead 
 
%%file 20_07_2020.fb.lnr.14
%%parent 20_07_2020
 
%%endhead 
  
  
\subsection{Переход углепрома ЛНР под госуправление ломает преступные схемы --- Пасечник}
\url{https://www.facebook.com/groups/LNRGUMO/permalink/2861116533999902/}

\vspace{0.5cm}
{\small\LaTeX~section: \verb|20_07_2020.fb.lnr.14| project: \verb|letopis| rootid: \verb|p_saintrussia|}
\vspace{0.5cm}

Переход угольной промышленности под государственное управление ломает
преступные схемы, созданные в этой отрасли. Об этом заявил глава ЛНР Леонид
Пасечник.

"В ходе реализации вопросов, связанных с внутренней кооперацией, то есть по
закупкам тех расходных материалов, которые используются на предприятиях
угольной промышленности, мы ломаем преступные схемы, когда эти сделки проходили
по завышенным ценам, когда в них имели место "откаты" и так далее", - сказал
он.

Пасечник отметил, что до реорганизации угольной отрасли основную прибыль от
продажи угля получали посредники.

"Самое важное достижение (реорганизации), на мой взгляд, состоит в том, что мы
ломаем схему реализации углей посредниками, которые в ходе реализации получали
основные выгоды и прибыли --- те, которые должны были получать государство и
шахтеры", - пояснил глава ЛНР, отметив, что шахтеры выполняют основную работу,
"рискуют собственным здоровьем, собственной жизнью, добывая уголь и работая на
интересы Республики".

"Но реализатор этой продукции получает основные прибыли и выгоды. Я думаю, что
это неправильно. Это был один из краеугольных камней, анализируя который мы
принимали решение о проведении реорганизации угольной отрасли в целом", -
сказал Пасечник.

Он подчеркнул, что на сегодняшний день предприятие "Восток-уголь" занимается и
добычей, и реализацией угля.

"Мы видим, почем мы добыли уголь, мы понимаем его себестоимость, мы понимаем,
за сколько мы продали этот уголь", - пояснил глава Республики.

ЛуганскИнформЦентр
