% vim: keymap=russian-jcukenwin
%%beginhead 
 
%%file 30_12_2021.fb.fb_group.story_kiev_ua.4.novyj_god_semja_1950.cmt
%%parent 30_12_2021.fb.fb_group.story_kiev_ua.4.novyj_god_semja_1950
 
%%url 
 
%%author_id 
%%date 
 
%%tags 
%%title 
 
%%endhead 
\zzSecCmt

\begin{itemize} % {
\iusr{Ольга Кирьянцева}
Спасибо! И Вас поздравляю с Новым годом! Будьте здоровы и счастливы.  @igg{fbicon.heart.sparkling} 

\iusr{Раиса Канаева}

\ifcmt
  ig https://scontent-mxp1-1.xx.fbcdn.net/v/t39.1997-6/s168x128/34849193_231467377438008_5173102921927098368_n.png?_nc_cat=1&ccb=1-5&_nc_sid=ac3552&_nc_ohc=nxbTzTFT0fkAX98CYk1&tn=lCYVFeHcTIAFcAzi&_nc_ht=scontent-mxp1-1.xx&oh=00_AT8UrLEvmbxtstSbjROgq13UFdJilXeHgHeMAOT8AUwyDQ&oe=61D83454
  @width 0.1
\fi


\iusr{Вікторія Святненко}

Ах, как замечательно Вы все описали... С наступающим Новым Годом Вас! И пусть
всегда этот праздник будет связан с детством и прекрасными воспоминаниями @igg{fbicon.heart.red}

\iusr{Aleksandr Mitryaev}

\ifcmt
  ig https://scontent-mxp1-1.xx.fbcdn.net/v/t39.30808-6/270380712_640063474017403_5108039823257123614_n.jpg?_nc_cat=107&ccb=1-5&_nc_sid=dbeb18&_nc_ohc=k2h5OsT0du0AX-LR0N4&_nc_ht=scontent-mxp1-1.xx&oh=00_AT_i_7zaBdmQse9J9dgehQaOf9ldPVagqHV9AWX-g06PsA&oe=61D8C548
  @width 0.3
\fi

\begin{itemize} % {
\iusr{Анатолий Крупин}

Необходимо напомнить, что у автора воспоминания о 50-х, а не о 90-х годах.
Приведенный на снимке универсам искусственно действительно был пуст, чтобы
усилить ненависть и озлобленность граждан к своей стране. Странно все это
конечно, и это при наличии Госплана. Хотя все шло по задуманному сценарию
развала. Нужно было все же на снимке показать обычный магазин 50-х и что на
полках его было. Да и в те годы Новогодний стол никогда не был пуст, есть еще
свидетели тому, что было. Многим конечно хотелось после лаптей, нищеты и полной
безграмотности все и сразу и сейчас, особенно тем, кто попал из грязи прямо в
князи... Но сам праздник Нового Года тогда был светлым и радостным, а не так
как сейчас с безрадостным ожиданием неизвестности. Время в пути и все меняется,
но во все времена были любители подкинуть в бочку с медом ложку дегтя и это им
доставляло удовольствие. Поэтому, как всегда, как во все времена, как и в те
далекие 50-е, хотелось бы пожелать: С Наступающим!

\begin{itemize} % {
\iusr{Vladimir Sterling}
\textbf{Анатолий Крупин} вот как сегодня выглядят супермаркеты в Европе, - тоже искусственно созданный дефицит для переформатирования общества:

\ifcmt
  ig https://scontent-frt3-1.xx.fbcdn.net/v/t39.30808-6/270371615_1095631101251521_4660248314217479297_n.jpg?_nc_cat=104&ccb=1-5&_nc_sid=dbeb18&_nc_ohc=JZoRLfP1R5kAX-uioyJ&_nc_ht=scontent-frt3-1.xx&oh=00_AT_lQOuYoSs7mwPYQg3V8G632gfPm08-Uq9b2lWmL3G-sQ&oe=61D7AD6B
  @width 0.3
\fi

\end{itemize} % }

\iusr{Лида Еремейкина}
\textbf{Aleksandr Mitryaev} где это такие корзинки и тележки?

\iusr{Ольга Почивалова}
\textbf{Aleksandr Mitryaev} это в 50-е уже были корзины на колесах!? Ну даем! 

%@igg{fbicon.face.grinning.squinting} 

\iusr{Vladimir Sterling}
\textbf{Aleksandr Mitryaev} в совке было, как сейчас в США и Англии)))

\ifcmt
  ig https://scontent-frt3-1.xx.fbcdn.net/v/t39.30808-6/270342160_1095631227918175_545032637678070100_n.jpg?_nc_cat=108&ccb=1-5&_nc_sid=dbeb18&_nc_ohc=IRqdNI73fRkAX-m7QYH&_nc_ht=scontent-frt3-1.xx&oh=00_AT9URcMtJqC8zsZ_rPQqG1tQNv5cc-4Xm5zzM2cr7JoTAA&oe=61D8FCC7
  @width 0.3
\fi

\begin{itemize} % {
\iusr{Ольга Почивалова}
\textbf{Vladimir Sterling} 

блин, это правда, звоню подруге пару дней назад (Калифорния), спрашиваю \enquote{куда
пойдете на каникулах, по магазинам?} Она отвечает, что уже все скупили,
магазины уже пустые. Изначально владельцы закупают одинаковое кол-во товара,
чтобы все распродать


\iusr{Vladimir Sterling}
\textbf{Ольга Почивалова} это фото за месяца два до Рождества)))
\end{itemize} % }

\iusr{Vladimir Sterling}
\textbf{Aleksandr Mitryaev} привет, английская Perestroyka!))

\ifcmt
  ig https://scontent-frx5-1.xx.fbcdn.net/v/t39.30808-6/270362561_1095632017918096_7143845680504838283_n.jpg?_nc_cat=110&ccb=1-5&_nc_sid=dbeb18&_nc_ohc=8vtFGNDfR4cAX_UzBdC&_nc_ht=scontent-frx5-1.xx&oh=00_AT-GieuidvwB92rymRuHAX27J7cK7uSeAHmg7QovsWCU2A&oe=61D8FCA5
  @width 0.3
\fi

\iusr{Vladimir Sterling}
\textbf{Aleksandr Mitryaev} похлеще совка, - там хоть консервы всегда были и хлеб!

\ifcmt
  ig https://scontent-frx5-1.xx.fbcdn.net/v/t39.30808-6/270236110_1095632417918056_4655677952074130609_n.jpg?_nc_cat=100&ccb=1-5&_nc_sid=dbeb18&_nc_ohc=6WuInlXP0G8AX-sGu2X&_nc_ht=scontent-frx5-1.xx&oh=00_AT-KH1785gg84Yq3qOPFUtZC9DE2UaaKWuDt9mikcc81gg&oe=61D7AE83
  @width 0.3
\fi

\iusr{Aleksandr Mitryaev}
1942 год США.

\ifcmt
  ig https://scontent-frt3-1.xx.fbcdn.net/v/t39.30808-6/270933645_640234510666966_8941117714692839519_n.jpg?_nc_cat=107&ccb=1-5&_nc_sid=dbeb18&_nc_ohc=CeMO_Tl4uv4AX_3cNJF&_nc_ht=scontent-frt3-1.xx&oh=00_AT-Xug8sHGM8ODpw66BlDR_iSke1H6nwKrjNosPHlxDQ3Q&oe=61D9097C
  @width 0.3
\fi

\iusr{Aleksandr Mitryaev}
1988 СССР

\ifcmt
  ig https://scontent-frt3-1.xx.fbcdn.net/v/t39.30808-6/270516819_640241780666239_3242392046929514195_n.jpg?_nc_cat=104&ccb=1-5&_nc_sid=dbeb18&_nc_ohc=2kcn11PQXJEAX9SNtds&_nc_ht=scontent-frt3-1.xx&oh=00_AT9u0U9WSpbiBzDc5sSMUCVOdV2b3tD5KqIhW5Sx0COAkA&oe=61D7AA43
  @width 0.3
\fi

\iusr{Aleksandr Mitryaev}
1989 СССР


\ifcmt
  tab_begin cols=2,no_fig,resizebox=0.5

     pic https://scontent-frt3-2.xx.fbcdn.net/v/t39.30808-6/271094343_640242663999484_9050254606997084188_n.jpg?_nc_cat=103&ccb=1-5&_nc_sid=dbeb18&_nc_ohc=cInkWIFfQUoAX99L0T2&_nc_ht=scontent-frt3-2.xx&oh=00_AT9NCC50hJU-Q0RnI0tp-AUFX2aLWRJOMS_RhgHePa4mXA&oe=61D77A77

		 pic https://scontent-frt3-1.xx.fbcdn.net/v/t39.30808-6/270409660_640243393999411_4731831403646923765_n.jpg?_nc_cat=108&ccb=1-5&_nc_sid=dbeb18&_nc_ohc=VgDOzd_7tooAX9OhWTA&_nc_ht=scontent-frt3-1.xx&oh=00_AT9cZcGTdvHxqTLAqVblTf9qAYpqvdxMYSvnmUx9q4hO2A&oe=61D7CDBD

  tab_end
\fi

\iusr{Aleksandr Mitryaev}

Сейчас в США и Великобритании

\ifcmt
  ig https://scontent-frx5-1.xx.fbcdn.net/v/t39.30808-6/270652047_640252183998532_7138551180335905228_n.jpg?_nc_cat=100&ccb=1-5&_nc_sid=dbeb18&_nc_ohc=7-afv-OegL0AX-oLJ8K&_nc_ht=scontent-frx5-1.xx&oh=00_AT_d4o4hQKLmyKrz_5rOKF16P7tx-GK4mIQ5mQ7E_usdGw&oe=61D86756
  @width 0.3
\fi

\iusr{Aleksandr Mitryaev}
Вот вам и ха-ха ...

\end{itemize} % }

\iusr{Людмила Полищук}

@igg{fbicon.snowflake} Спасибо! Именно так всё и было! С НАСТУПАЮЩИМ !!!
@igg{fbicon.christmas.tree}  @igg{fbicon.heart.sparkling}  Всего наилучшего!

\iusr{Ирина Иванченко}
Как замечательно, Анатолий! С наступающим Новым годом , здоровья, счастья и добра!

\iusr{Alex Gluzman}
И тебя с наступающим, страшно подумать, 2022-м! Держись и пиши! Много буков! Не ракет, конечно!

\iusr{Петр Кузьменко}

Тёплые и душевные воспоминания! С наступающим Новым Годом Вас! Здоровья и
счастья! Ждём новых интересных публикаций! @igg{fbicon.snowflake} ️ ☃️

\iusr{Галина Гурьева}
Спасибо! Стало очень тепло от воспоминаний. С наступающим!!!

\iusr{Tamara Malezhyk}
Спасибо . С Наступающим Новым годом !!!

\iusr{Dubizhansky Ludmila}

Как говорила незабвенная гениальная Майя Каганская, праздник ощущается своим,
только когда у него есть запах и вкус детства. Рублевый прозрачный мешочек с
залежавшимися конфетами и маленькой мандаринкой в нашем сознании навсегда
остался главным атрибутом любимого новогоднего праздника с нарядной елкой,
ватными Дедом Морозом и снегурочкой, подарками, елкой в Октябрьском дворце...


\iusr{Светлана Ветрова}
Душевна росповідь з дитинства усіх з прийдешними святами


\iusr{Таня Гур}
 @igg{fbicon.hands.applause.yellow} 

\iusr{Елена Тарасова}

У нас тоже хранились игрушки в посылочном ящике. Такая деталь ... и нахлынули
воспоминания. Спасибо! Сил и здоровья! С Новым годом!!! @igg{fbicon.evergreen.tree}{repeat=4} 

\iusr{Людмила Костина}
Спасибо. Так тепло и хорошо от Ваших слов. С Новым Годом! Счастья Вам, здоровья и радости.

\iusr{Nina Tkachenko}

Вітання з прийдешнім Новим роком 2022 з Києва, на Володимирській вулиці, біля
Софії зараз сяє штучна ялинка і ковзанка і глінтвейн, і музика, і карусель як у
Парижі! Але радості якось не відчувається, такі складні часи настали.... Війна,
Пандемія, Ковід. Пишіть оповідки, бережіть себе!

\begin{itemize} % {
\iusr{Ольга Ершова}
\textbf{Nina Tkachenko} ми стали старші, життя - складніше. Обов'язки сприймаються важче й ми - реалісти. Тези про дива у новорічні дні та особливі прикмети нас дратують. Всі сподівання - на себе та на щасливий випадок та обставини.
Але - з Новим Роком.

\begin{itemize} % {
\iusr{Nina Tkachenko}
\textbf{Ольга Ершова} 

Після Януковича \enquote{йолка} для українців стала абсолютно знаковим \enquote{політичним}
деревом.

З'ясувалось, що у французів є вислів \enquote{ça sent le sapin} - \enquote{пахне ялинкою}. Цей
афоризм означає - \enquote{скоро вже кінець}. Така асоціація у французів склалася
декілька століть тому, бо труну виготовляли з хвойних порід дерев.

В Україні всі знаки, які подає нам \enquote{йолка} вказують на близьку \enquote{політичну
смерть}.

Шо там на Банковій? Чути, як \enquote{йолкою} запахло?

\iusr{Natali Kazakova}
\textbf{Nina Tkachenko} скоріше б вже запахло ялинкою зеленій владі, бо вже далі нема куди

\iusr{Людмила Адаменко}
\textbf{Ольга Ершова} при мародьорі було краще?
\end{itemize} % }

\end{itemize} % }

\iusr{Людмила Викторовна}
Спасибо! С Новым годом!  @igg{fbicon.evergreen.tree} 

\iusr{Мария Константиновская}
Спасибо! Столько воспоминаний! И Вас с Новым годом! @igg{fbicon.christmas.tree}  @igg{fbicon.bottle.popping.cork}  @igg{fbicon.clinking.glasses} 

\iusr{Люсянка Балашова}
Спасибо... с Наступающим Новым Годом !!!

\iusr{Наталья Дрягун}
Спасибо! Прочитала с удовольствием  @igg{fbicon.smile} 
С Новым годом  @igg{fbicon.christmas.tree} ! Здоровья и всех благ!

\iusr{Надежда Грабовская}
С Наступающим Новым годом!

\iusr{Світлана Дровозюк}

Про покупку елки, \enquote{посьілочньіе} ящики, Деда Мороза и игрушки как про мое
детство, только наши игрушки все побились когда коротнула гирлянда. Слава Богу,
все обошлось, но от елки остались только скелетньіе ветки а от игрушек осколки,
перегорели нитки на которьіх они висели. Дед Мороз тоже сгорел. Прошло больше 50
лет а я помню ето как сегодня

\begin{itemize} % {
\iusr{Ольга Ершова}
\textbf{Світлана Дровозюк} а на меня когда-то упала елка в полной комплектации. Кот спал в кресле. Почему она упала - непонятно. Но ничего с нее не упало, все осталось цело

\iusr{Світлана Дровозюк}
\textbf{Ольга Ершова} ветки самортизировали
\end{itemize} % }

\iusr{Елена Палий}
Спасибо!!!! С Новым годом!!!!

\iusr{Алла Пачевский}

@igg{fbicon.snowman.without.snow} @igg{fbicon.snowflake}
@igg{fbicon.snowman.without.snow} Спасибо огромно, вроде в Киеве побывала.. С
наступающим новым годом!

\iusr{Арт Юрковская}

Представьте каково мне было жить возле двух игрушечных магазинов на
Красноармейской! И вчера достали коробку - и один шар разбит. Как это
получается? Уложила старательно. И в этом году радость - стоит реставрированный
Дед Мороз. Скоро будет и Снегурочка!


\iusr{Наталия Любинецкая}
Коснулась детства, потеплела душенька. Благодарю, Анатолий. С наступающим !

\iusr{Сергей Мачеристов}

В конце каждого года, как раз перед новогодними праздниками в муз.школе всегда
были экзамены. Всегда сдавал на \enquote{5}, за что от бабушки получал вознаграждение
-10 рублей. Получив денюжку, всегда бежал в Детский мир и покупал новые игрушки
на ёлку. А потом вечером всегда украшали ёлку, но сначала новыми игрушками, а
потом \enquote{старыми}.


\iusr{Ольга Ершова}

А я, будучи ребенком, ждала таинства украшения елки. И все казалось прекрасным и сказочным.

Будучи подростком мне все предновогоднее казалась убогой безвкусицей.

Будучи мамой маленького мальчика мне хотелось, чтобы все предновогоднее нрааилось и ему и мне.

Став взрослым, мой ребенок заявляет - я сам все вам привезу, такое какое сочту нужным.

Но дед Мороз из ваты у меня есть  @igg{fbicon.snowman.without.snow} @igg{fbicon.snowflake} @igg{fbicon.snowman.without.snow} 

\iusr{Татьяна Шаповалова}
Как хорошо написано, а главное-узнаваемо. Предвкушение Нового года \enquote{по киевски}. Всех с наступающим, здоровья и мира!

\iusr{Ирина Панченко}
 @igg{fbicon.hands.applause.yellow} 

\iusr{Владимир Коваленко}
Душевно! @igg{fbicon.hands.applause.yellow} 

\iusr{Лена Олейник}
Спасибо!
С наступающим Новым годом!

\iusr{Наталья Соколовская}
Огромное спасибо за такие душевные воспоминания. С Новым годом вас, счастья, здоровья!

\iusr{Михайлина Голуб}

Уважаемый Анатолий. Прочитала и поразилась. Словно о моем детстве написано. Все
точно. Даже бабушкин гусь. Только жили мы на Лукьяновке и елочный базар был на
Артема, рядом с кинотеатром \enquote{Коммунар}, а второй - возле клуба трамвайщиков.

\iusr{Ira Yakunenko}
Дякую за теплі спогади. Моя бабуся з дідом жили на Володимирській, 42. І

\iusr{Майя Зигель}
Спасибо за поздравления и добрые воспоминания ! С Новым годом !!!

\iusr{Iryna Domashova}
Спасибо за рассказ и поздравление. Всех с Наступающим! Конечно здоровья и мира!

\iusr{Анна Сидоренко}
СПАСИБО!

\iusr{Августина Тытянчук}

У меня также воспоминания, только район другой.... возле Дорогожич... также
деревянные ящики... Дед мороз и снегурочка!!!, куча игрушек и дождика....
обязательно много ваты, типа снежки... серпантин и много хлопушек с конфети...
замечательное время... ты маленький, папа, мама, сестра все друзья... класс, спасибо
большое за пост и мне вспомнилось... Всех с праздниками!!! Всех благ!!!

\iusr{Валентина Ненько}
Спасибо за воспоминания

\ifcmt
  ig https://scontent-frt3-1.xx.fbcdn.net/v/t39.30808-6/270671633_462686721921134_5983339389366942054_n.jpg?_nc_cat=102&ccb=1-5&_nc_sid=dbeb18&_nc_ohc=RFw0qs1JROIAX93CjnY&_nc_ht=scontent-frt3-1.xx&oh=00_AT8XLkt5wNQ5-0VDj-AjafiusVfH1jHhEm-NPh2IlAMMdA&oe=61D8CCF4
  @width 0.2
\fi

\iusr{Валентина Ненько}
Спасибо за тёплые слова  @igg{fbicon.hands.pray} @igg{fbicon.heart.red}
@igg{fbicon.evergreen.tree}  @igg{fbicon.christmas.tree} 

\iusr{Мила Мила}

У меня и сейчас ещё есть много старых игрушек. Им по 50 - 30 лет. А хранили
их в большом картонном ящике, но каждая игрушечка была завернула в салфетку,
а потом ещё и ватой перекладывали, поэтому все были целыми. Ну и конечно
большой Дед мороз из ваты. Поздравляю всех с наступающим новым годом. Здоровья
и благополучия в новом году

\iusr{Lyudmila Zhuchkova}

Благодарю! Интересные воспоминания! С наступающим Новым годом! @igg{fbicon.snowflake} @igg{fbicon.evergreen.tree} 
@igg{fbicon.snowman.without.snow} Здоровья,
счастья, добра, мира и благополучия!

\ifcmt
  ig https://i2.paste.pics/31573b43d82ef42c460e9da005f847b9.png
  @width 0.2
\fi

\iusr{Нинель Кузницкая}
Анатолий, спасибо большое за замечательный рассказ! С Новым годом Вас и Ваших
близких! @igg{fbicon.snowman.without.snow}  @igg{fbicon.party.popper} 

\iusr{Надія Чуй}
Навзаєм!!!!!

\ifcmt
  ig https://scontent-mxp1-1.xx.fbcdn.net/v/t39.1997-6/s168x128/10734344_610917765700644_325078055_n.png?_nc_cat=1&ccb=1-5&_nc_sid=ac3552&_nc_ohc=oQ8oPR7QdT4AX_9-0w4&_nc_ht=scontent-mxp1-1.xx&oh=00_AT-rDK7d5Tvl_a08Z9uTA5fMxz25RMtbwauVDZYYqc6tOg&oe=61D90045
  @width 0.1
\fi

\iusr{Олена Медведева- Прицкер}

Какая замечательная открытка из детства !С Новым годом уважаемый Анатолий и
Все-Все ! Крепкого здоровья и успехов  @igg{fbicon.christmas.tree} 

\iusr{Ковальская Татьяна}

Спасибо! Старики смотрят в прошлое и я тоже вспоминаю прошлые Новые года! Почему
моему мужу всегда удавалось купить две корявый, почти без веток ёлки и потом он
их связывал с нашей помощью и получалось что-то похожее на ёлку, но может в
этом и была прелесть, помним, это и было счастье!

\iusr{Dina Bitel}

Очень хорошо помню предновогоднюю суету где-то в начале 60тых, поиски ёлки
среди палок. Так хотелось, чтобы папа купил, наконец, хотя бы одну из них, ту,
на которой оставалось ещё какое-то количество иголок. Моё полное разочарование
и даже отчаяние от перспективы остаться без ёлки... Но вот вдруг ОНА найдена,
красавица, пышная, с большими игольчатыми ветками! Какое счастье!!!

С новым годом!

\iusr{Olga Ladanovskaya}
Навзаєм!!!!

\iusr{Ирина Дубчак}
спасибо за воспоминания мира и добра и конечно здоровья

\iusr{Таня Таня}
С наступающим!!!! Здоровья и удачи!!!

\iusr{Алла Беркова}

Был еще базар на площади Калинина и возле памятника Хмельницкому я помню. Всех
с Новым Годом главное будьте здоровы  @igg{fbicon.santa.claus}
@igg{fbicon.christmas.tree} 🤶

\iusr{Natasha Levitskaya}

Спасибо за теплые воспоминания и с наступающим Новым годом вас!
@igg{fbicon.party.popper}  @igg{fbicon.christmas.tree}
@igg{fbicon.bottle.popping.cork}  @igg{fbicon.clinking.glasses} 

\iusr{Dina Kozlova}
С НАСТУПАЮЩИМ НОВЫМ ГОДОМ !!!

\iusr{Константин Ткачев}
Спасибо, взаимно. А кто такoя Ханука?

\iusr{Анатолий Золотушкин}
\textbf{Константин Ткачев} это еврейский праздник. Посмотрите в Гугле

\iusr{Елена Сидоренко}

С новым годом, новым счастьем, новым здоровьем! Вы - один из моих любимых
авторов, жду в новом году новых киевских историй! 
@igg{fbicon.evergreen.tree}  @igg{fbicon.party.popper}  @igg{fbicon.wine.glass}
@igg{fbicon.snowflake} @igg{fbicon.heart.sparkling} 

\iusr{Анатолий Золотушкин}
\textbf{Елена Сидоренко} спасибо

\iusr{Тамара Кипнис-Иванисова-}

Прекрасные воспоминания. Поделюсь своими. Приходил папа с елкой, живой. От пола
до потолка. Обалденно пахла хвоей. Топором делал крест, который прибивал к
деревянному полу. Всей семьей садились за круглый стол и готовили украшения.
Какие? Орехи красили бронзовой краской. Конфеты (золотой ключик и шоколадные:
(Ромашка, Буревестник), привязывали на нитку. Из яичной скорлупы делали
клоунов, клеили на них колпачки из цветной бумаги. Были и стеклянные игрушки.
кое-что сохранилось. Ощущение праздника было. Дед Мороз из ваты под елкой,
картонный домик с подсветкой из лампочки. вата на ветках. Мы с сестрой конфеты
съедали очень быстро. Остались щемящие душу воспоминания об ожиданиях,
подготовке, о большой дружной семье. С наступающим, дорогие друзья!


\iusr{Ксения Приймак}
Мира и тепла!

\iusr{Алёна Шаповалова}
С наступающим Новым Годом!!!
 @igg{fbicon.christmas.tree} @igg{fbicon.snowman.without.snow} @igg{fbicon.christmas.tree} @igg{fbicon.snowman.without.snow} @igg{fbicon.christmas.tree} @igg{fbicon.snowman.without.snow}
 @igg{fbicon.christmas.tree} @igg{fbicon.snowman.without.snow} @igg{fbicon.christmas.tree} @igg{fbicon.snowman.without.snow} @igg{fbicon.christmas.tree} @igg{fbicon.snowman.without.snow} @igg{fbicon.christmas.tree} 

\iusr{Мария Тимошенко}
В парке Шевченко я с дедушкой впервые увидела елку и запомнилась на всегда где то 50 годв

\iusr{Arcadia Olimi}
@igg{fbicon.snowflake} Спасибо  @igg{fbicon.face.happy.two.hands} 
Взаимно @igg{fbicon.snowman.without.snow}

\iusr{Ніна Нагорська}

\ifcmt
  ig https://scontent-frx5-1.xx.fbcdn.net/v/t39.30808-6/271118636_354462186516419_1206369388061869058_n.jpg?_nc_cat=111&ccb=1-5&_nc_sid=dbeb18&_nc_ohc=XxNsLU2SwQkAX-b5CT1&_nc_ht=scontent-frx5-1.xx&oh=00_AT8DNT6lrG6TevcZAKIfk2fTf04PmfW4HG_keCBp7J8--w&oe=61D75F4A
  @width 0.2
\fi

\iusr{Таня Сидорова}

А мій батько ще знизу спілював знизу гілочки та додавал там, де було не дуже пишно.

\ifcmt
  ig https://scontent-frt3-1.xx.fbcdn.net/v/t39.30808-6/270034982_2067876266713282_1130323881669191513_n.jpg?_nc_cat=102&ccb=1-5&_nc_sid=dbeb18&_nc_ohc=svJm4rMQEOcAX8vNEkI&_nc_ht=scontent-frt3-1.xx&oh=00_AT9K0dfB8zPM8zw336fHPRUF7_cNDT2KOTBGI0Kt4F7N1A&oe=61D7796C
  @width 0.2
\fi

\iusr{Маргарита Воронова}

урааа

\iusr{Tatiana Thoene}
Спасибо за тёплый рассказ. С Новым годом!

\iusr{Honcharuk Tatiana}
 @igg{fbicon.confetti.ball}  @igg{fbicon.shooting.star} 

\iusr{Нина Ключникова}
Просто в детстве побывала!

\iusr{Наташа Драпушко}

Слезы навернулись на глаза. Вспомнила, как папа подгонял ствол большой ёлки под
крест. А мы с братом ждали и вдыхали ёлочный запах и готовили игрушки.

\iusr{Віта Ілейко}
з прийдешнім новим роком вас!!

\iusr{Людмила Гнынюк}
С наступающим Новым годом!

\ifcmt
  ig https://scontent-frx5-1.xx.fbcdn.net/v/t39.30808-6/270187667_1515866355456612_566483100610998316_n.jpg?_nc_cat=100&ccb=1-5&_nc_sid=dbeb18&_nc_ohc=jszvtXANv9gAX9ZpoHY&_nc_ht=scontent-frx5-1.xx&oh=00_AT8jKlnEQC1NWQEKbaGnQCrCXxcHDF9d8Z_zJHystfjMtg&oe=61D862FF
  @width 0.3
\fi

\iusr{Анна Сафронова}
Теплый, замечательный рассказ. Так повеяло родным и близким, знакомым с детства! Спасибо! С Новым Годом Вас!

\iusr{Анна Сафронова}
@igg{fbicon.snowman.without.snow} @igg{fbicon.snowflake} @igg{fbicon.heart.sparkling}  @igg{fbicon.party.popper} 

\iusr{Анна Шустерман}
С наступающим, Анатолий! Будьте здоровы пожалуйста и в новом году радуйте нас снова своими рассказами!  @igg{fbicon.hearts.two} @igg{fbicon.snowman.without.snow} @igg{fbicon.snowflake}

\iusr{Nataliya White}
С наступающим Новым годом!@igg{fbicon.snowman.without.snow} @igg{fbicon.party.popper}  @igg{fbicon.face.smiling.eyes.smiling} 

\iusr{Татьяна Петрюк}
Ваши воспоминания уникальны по деталям: светлые и согревают душу: всплывают из
памяти счастливые моменты детства. Спасибо !

\iusr{Valentina Shevchenko}
Дякую за вспомин.
Нібито настальгія, але це теж історія.
Вітаю Вас з наступаючим Новим Роком!
Добра і миру всім нам!
Дякую що пишите для нас.

\iusr{Vicki Seplarsky}
Самый любимый праздник, один без политики.

\iusr{Vadim Shapiro}

Спасибо за рассказ, Анатолий! С Новым Годом Вас! Мне удалось дотащить свои
любимые игрушки до Америки. Теперь ими украшают елку мои внуки - 'Deda's toys
from Kiev!'


\iusr{Людмила Філевська}

Все точнісінько так: і коробка з-під посилки з іграшками, і обовязково декілька
іграшок розбиті, і ялинка в хрестовині, і пух для подушок, і віничок з крила,
і..., і...... . І розуміння, що моє життя - це вже історія)))

\iusr{Svetlana Naulko}
Восхитительно!!!! С благодарностью , Анатолий !!!

\iusr{Светлана Здор}
Спасибо. И Вас с наступающим Новым годом!

\iusr{Виктор Задворнов}

Да, огромное спасибище за воспоминания. Нет рядом уже отца, мамы, старшего
брата. Смотрю на место у окна \enquote{большой} комнаты, где в эти дни стояла елка.
Где-то отлеживаются елочные игрушки... НО! Новый год к нам мчится. Скоро все
случится. Всех благ желаю всем!

\iusr{Natalii Yakovenko}
З наступающим! Всем мире, здровья и добра! @igg{fbicon.heart.beating}  @igg{fbicon.evergreen.tree}  @igg{fbicon.bottle.popping.cork}  @igg{fbicon.clinking.glasses} 

\ifcmt
  ig https://scontent-frt3-1.xx.fbcdn.net/v/t39.30808-6/270741754_641540927043462_5761304841395216083_n.jpg?_nc_cat=104&ccb=1-5&_nc_sid=dbeb18&_nc_ohc=ldRuakW2L1kAX-61z66&_nc_ht=scontent-frt3-1.xx&oh=00_AT9Gnqy2dDP3PwXkCGm0t3CjS2rulNvyznszGr7F2Ld44Q&oe=61D76A65
  @width 0.2
\fi

\iusr{Ольга Нижник}
Огромное спасибо за воспоминание детства! Я, думаю, многим есть, что вспомнить!
С наступающим Новым годом!


\iusr{Наташа Драпушко}

У меня тоже есть новогодняя история. Мне шесть лет, лежу в инфекционной
больнице. Тихо плачу под одеялом, никого ко мне не пускают. И вдруг, открывается
дверь и появляется маленькая украшенная ёлочка. Папа прошел и принес ёлочку.. На
ёлке были конфеты, которые мы всей палатой дружно съели, хотя врачи не
разрешали. Папа был военным, ёлка настоящая, срубленная в лесу. Это было за
полярным кругом. Папа ушел в лучший мир в возрасте Ста лет. Скучаю.

\iusr{Віта Ілейко}

\ifcmt
  ig https://scontent-frt3-2.xx.fbcdn.net/v/t39.30808-6/270588417_4635165149883003_6862513990576899491_n.jpg?_nc_cat=103&ccb=1-5&_nc_sid=dbeb18&_nc_ohc=2JzG7szZ8t8AX_Gp3Fz&_nc_ht=scontent-frt3-2.xx&oh=00_AT_jA-y46qE5Y2ifBovkYcn6uAq16uQpm7DvvaB3cJGHTQ&oe=61D816B2
  @width 0.3
\fi

\iusr{Natalia Tatarynova}

\ifcmt
  ig https://scontent-frx5-1.xx.fbcdn.net/v/t39.30808-6/270713631_1354079478375607_2004785737363082779_n.jpg?_nc_cat=105&ccb=1-5&_nc_sid=dbeb18&_nc_ohc=UreXP4o0P0UAX8wr1FE&_nc_ht=scontent-frx5-1.xx&oh=00_AT9E5NxONtceoXi3PmRtb2IMeosISMZrqbRi_VxCqmej3A&oe=61D868E8
  @width 0.2
\fi

\iusr{Наталья Торопкова}
Просто чудо новогоднее

\iusr{Людмила Белоусова}

\ifcmt
  ig https://scontent-frx5-1.xx.fbcdn.net/v/t39.30808-6/271185044_324570433005535_8688282016544807711_n.jpg?_nc_cat=110&ccb=1-5&_nc_sid=dbeb18&_nc_ohc=1p8PVVKlqGUAX_vm8Ex&_nc_ht=scontent-frx5-1.xx&oh=00_AT_UWBz1atqaZ74MayCF4WJ6MP5nwMPmLZo0i3mRe8uDbQ&oe=61D85574
  @width 0.3
\fi

\iusr{Светлана Александренко}
Замечательный рассказ! Спасибо за настроение, за память!

\iusr{Лариса Безрукова}

\ifcmt
  ig https://scontent-mxp1-1.xx.fbcdn.net/v/t39.1997-6/s168x128/10935975_332668776936759_549883954_n.png?_nc_cat=1&ccb=1-5&_nc_sid=ac3552&_nc_ohc=XhFWHDcjIxEAX_sa1Gz&_nc_ht=scontent-mxp1-1.xx&oh=00_AT_taL3MChREXpGD3tDWzOYpftJWVydydvUnag6Qf0Mcgg&oe=61D84687
  @width 0.1
\fi

\iusr{Валентина Бабченко}

\ifcmt
  ig https://scontent-frx5-1.xx.fbcdn.net/v/t39.30808-6/270935832_3250238555297306_8398323603890439315_n.jpg?_nc_cat=111&ccb=1-5&_nc_sid=dbeb18&_nc_ohc=yWbE9V1LYhMAX-geae9&_nc_ht=scontent-frx5-1.xx&oh=00_AT8T_DgULxc4-qVvFy2M8lVVOKIqUoyIsMevNUnfdcXRrQ&oe=61D8D9C5
  @width 0.2
\fi

\iusr{Alla Kenya}

\ifcmt
  ig https://scontent-frx5-1.xx.fbcdn.net/v/t39.30808-6/270114499_1603924116628601_6308167956108298793_n.jpg?_nc_cat=105&ccb=1-5&_nc_sid=dbeb18&_nc_ohc=NJM90NK5fcEAX8QR7IK&_nc_ht=scontent-frx5-1.xx&oh=00_AT_1LyOmukEKV_AtSRrYRL-wI0IF7I1XetWUn0QJ1n5Svw&oe=61D7C30F
  @width 0.3
\fi

\iusr{Julia Panchul}
я видела это прошлое

\iusr{Люда Невзгляд}
Спасибо за чудные воспоминания.

\iusr{Klara Mezhebovsky}
Я помню эту новогоднюю открытку!

\iusr{Bella Braiman}

Да мы все помним это нельзя забыть, а ведь прошло 60 лет
@igg{fbicon.face.happy.two.hands}  @igg{fbicon.snowman.without.snow} 

\iusr{Klaudia Markh}
Спасибо за интересную историю. Сразу становится как-то уютно  @igg{fbicon.heart.eyes} 

\iusr{Natalie Di Giovanni}
С новым годом @igg{fbicon.snowman.without.snow}{repeat=3}

\iusr{Abraham Chavez}
Happy New Years

\iusr{Оля Оля}
\enquote{И так будет всегда}! к, сожалению, так не будет никогда. Спасибо за
рассказ! С Новым годом! @igg{fbicon.christmas.tree} 

\iusr{Алла Квасницкая}
Спасибо за такой привет из детства! Счастливого Нового Года!

\iusr{Алёна Новиковп}

Наша память хранит много приятных моментов из детства (у меня досих пор игрушки
лежат в деревянном ящике из под сливочного масла (не многие помнят), и пока не
пришел старческий маразм, мы каждый новый год будем тепло вспоминать моменты
нашего детства.... @igg{fbicon.snowman.without.snow}{repeat=3}  @igg{fbicon.wink}{repeat=3} 

\iusr{Наталия Ковалева}
Всех с Новым годом!!!!

\iusr{Наталия Ковалева}

\ifcmt
  ig https://scontent-mxp1-1.xx.fbcdn.net/v/t39.1997-6/s168x128/70086330_2383481515080879_1239518598028853248_n.png?_nc_cat=1&ccb=1-5&_nc_sid=ac3552&_nc_ohc=CrYBJypCYKQAX_J1OZZ&_nc_ht=scontent-mxp1-1.xx&oh=00_AT8wc8wnbFqV_qQ5U2mC0ZKoTv_Bz2qTeq_vafwCWo5OEA&oe=61D756B8
  @width 0.1
\fi

\iusr{Галина Местечкина}
Повністю поділяю ці спогади і вдячна за них!


\end{itemize} % }
