% vim: keymap=russian-jcukenwin
%%beginhead 
 
%%file 22_02_2022.fb.bereza_borislav.1.putin_ukraina
%%parent 22_02_2022
 
%%url https://www.facebook.com/borislav.bereza/posts/7802213716471333
 
%%author_id bereza_borislav
%%date 
 
%%tags __feb_2022.putin.priznanie,dnr,donbass,lnr,putin_vladimir,ukraina
%%title Путин отгрызает кусок за куском части Украины
 
%%endhead 
 
\subsection{Путин отгрызает кусок за куском части Украины}
\label{sec:22_02_2022.fb.bereza_borislav.1.putin_ukraina}
 
\Purl{https://www.facebook.com/borislav.bereza/posts/7802213716471333}
\ifcmt
 author_begin
   author_id bereza_borislav
 author_end
\fi

8 лет назад Путин воспользовался моментом и захватил Крым. Сегодня он
воспользовался Зеленским и захватил ОРДЛО. Путин отгрызает кусок за куском
части Украины. И он не остановится. Ошибаются те, кто думают, что угроза
вторжения исчезла. Нет, наоборот. Аппетит, ведь приходит во время еды. А Путин
пожирает Украину. И этот процесс ему нравится. Политика Зеленского, которая
напоминает тезис Троцкого \enquote{Ни мира, ни войны}, окончательно
провалилась. Сейчас многое зависит от того, какие санкции коллективный Запад
наложит на всех, кто принял участие в этой операции поглощения ОРДЛО и какие
именно санкции будут наложены на РФ. Надежд и иллюзий в отношении нашей власти
у меня нет.

При этом надо обратить внимание на два момента. Фактически Путин своим решением
уничтожил Минские соглашения и освободил Украину от них, но не снял санкции с
РФ. И это для нас освобождает площадку для маневра, а так же открывает окно
возможностей. Кроме того, заявление о том, что Россия будет принимать меры для
поддержания финансовой и банковской систем «ДНР» и «ЛНР», исходя из того, что
платежное средство там - российский рубль, снимает все финансовые обязательства
Украины перед ОРДЛО. И это пора озвучить. А у ОРДЛО теперь будущее Абхазии -
депрессивной территории без будущего. 

При этом надо отметить, что позиция Зеленского крайне слабая и унылая. Как и
его бессодержательное выступление. Он просто тянет время и не принимает никаких
решений. Нет ни мобилизации, ни разрыва дипломатических отношений с РФ, ни
введения виз для граждан РФ, ни даже адресных санкций против всех, кто своим
участием в Совбезе РФ санкционировал признание ОРДЛО. Нет и введения военного
положения на территории Донецкой и Луганской областей. Нет ничего. Вместо
державнической позиции - лишь пук в лужу. 

В такой ситуации именно парламент должен стать инициатором всех действий
направленных на защиту интересов и территориальной целостности Украины. Но
учитывая, что в ВР сегодня сидит импотентное большинство подконтрольное ОП и
перепуганному Зеленскому, то можно не ждать от них каких-то действий. К
сожалению, в период потрясений мы имеем ту власть, которая способна лишь на
дискредитационные или коррупционные скандалы, а не на антикризисные решения. И
как же вновь не вспомнить слова Михаила Грушевского, сказанные им более 100 лет
назад, \enquote{Біда України в тому, що нею керують ті, кому вона не потрібна}.
К сожалению, эти слова не утратили своей актуальности и сегодня.
