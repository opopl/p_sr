% vim: keymap=russian-jcukenwin
%%beginhead 
 
%%file 18_06_2021.fb.fb_group.pro_movu.1.muzyka_rynok_rossia
%%parent 18_06_2021
 
%%url https://www.facebook.com/groups/promovugroup/permalink/991830644724130/
 
%%author 
%%author_id fb_group.pro_movu
%%author_url 
 
%%tags ekonomika,muzyka,rossia,rynok,showbiz,ukraina
%%title ...ми ще на стадії формування українського музичного ринку - відвойовуємо його у Росії
 
%%endhead 
 
\subsection{...ми ще на стадії формування українського музичного ринку - відвойовуємо його у Росії}
\label{sec:18_06_2021.fb.fb_group.pro_movu.1.muzyka_rynok_rossia}
\Purl{https://www.facebook.com/groups/promovugroup/permalink/991830644724130/}
\ifcmt
 author_begin
   author_id fb_group.pro_movu
 author_end
\fi

Роман Муха.

«...ми ще на стадії формування українського музичного ринку - відвойовуємо його
у Росії. Бо до подій на Майдані, до російсько-української війни вони почували
себе у нас привільно. І те, що наші музиканти хаотично пробивалися на
радіостанції та збирали зали, не означало, що ми мали музичний ринок». 

«Музичні радіостанції переважно належать людям, які не особливо підтримують
українську музику й культуру... Для мене українська – це україномовна. Існує
шаблон, мовляв, музика «поза межами», інтернаціональна. Музика - так. Але в
пісні завжди є мова. Якщо артист виробляє в Україні російськомовну пісню, ця
пісня належить російському культурному центру. Насправді якби не квоти, ми б
вже втратили культурне поле».

«У  музичних редакторів чи продюсерів є набір шаблонних відповідей – «не
формат», «музична рада не затвердила», «фокус-група не прийняла». Це
маніпуляції та відмовки... На телеканалі «М2», де я був продюсером, ми давали
виконавцю шанс попасти в ефір, якщо трек якісно записаний. Як наслідок, з 2016
року до 2020 року «М2» вдвічі виріс у рейтингу. Тобто, аудиторія хоче слухати
українську музику».

«...є проблеми зі стрімінгами (четвірка основних - Apple Music, Spotify,
YouTube, Deezer). Як думаєте, де знаходяться регіональні офіси стрімінгових
платформ?

В Росії.

Так, у Москві. Це приблизно те саме, якби музичні редактори, які визначають
чарти для Ізраїлю, сиділи в штаб-квартирі ХАМАС... Ми говоримо про нав'язування
споживацьких смаків країною-агресором. Не може Москва нав'язувати музичні чарти
Києву... Програма максимум – щоб відкрили офіс у Києві. Програма мінімум –
перенесли регіональні офіси з Москви у неворожу для нас країну. У Варшаву чи
Вільнюс. Це питання треба розхитувати. Нехай суспільство підключається,
громадські організації».

«Мовна полігамність – це ще одна проблема в споживчих смаках. Оскільки у нас
величезна частина населення російськомовна, то по звучанню, по вихованих з
дитинства музичних гармоніях російська музика їм ментально ближче. Коли
виходить російський хіт, російськомовна людина швидше на нього відгукнеться...
Маємо пропонувати альтернативу, так само як повинні відвоювати простір у
російського культурного простору».

«...треба розуміти особливості явища підліткової субкультури. Це культура
зграї, тому прийнята авторитетами цих спільнот культура поширюється на всю
спільноту. Підлітки перебувають у віці психологічного бунту проти батьків,
фальшу школи, шукають щось таке, що б «чіпляло» в плані бунту. Російський реп
тонко «зчитує» ці настрої та пропонує їм власну ілюзію бунту.

У цьому плані Росія наслідує Радянський Союз... при комсомольських організаціях
були створені рок-гурти за підтримки держави... У Радянському Союзі була
потужна школа спецслужб, ці специ лишилися в Росії. І те саме зроблено в Росії
з репом, бо реп – це сьогоднішній рок. Проблема в тому, що наші підлітки
виховуються на цьому, слухають цих артистів, готові платити гроші на концертах.
Гроші знову поїдуть у Москву. Цей процес хтось повинен зупинити. Для цього має
бути комплексне рішення, підтримка держави».

\ifcmt
  pic https://external-mia3-2.xx.fbcdn.net/safe_image.php?d=AQELnOO0oEsbOxu8&w=500&h=261&url=https%3A%2F%2Fimages.unian.net%2Fphotos%2F2021_06%2Fthumb_files%2F1000_545_1623242273-3664.jpg&cfs=1&ext=jpg&tp=1&ccb=3-5&_nc_hash=AQHbuzUyPHmmGwuF
\fi

«Громадськість, держава, бізнесмени мають усвідомити, що вони або захищають
свій культурний простір, або позбавляються його. Без цього усвідомлення та
підтримки ми приречені втратити свій культурний простір... Це питання
національної безпеки, охорони свого культурного та економічного простору».

\emph{Юлія Клименко}

Нав"язування--це істина,бо молодь вся висить на російськомовній музиці....те,що
слухають вони--це жах....з цим треба щось робити,бо неминуча
деградація....Матюки в піснях,якісь дикі думки,бридота одним словом....

\emph{Лариса Гошкодеря}

Юлія Клименко 100\%! І без втручання держави, прийняття відповідних законів не
буде діла! Треба припинити цей \enquote{бєзпрідєл}!

\emph{Ольга Новосьолова}

Юлія Клименко Тупі тексти, гундосі голоси, російська мова. І все це слухають
мої внуки. Де українські виконавці? Таке відчуття, що українською мовою нічого
путнього не можна написати. Я маю на увазі музику для дітей, підлітків. Діти
ростуть, а я все сподіваюсь...

\emph{Ярослав Трінчук}

Оскільки було силове впровадження росмовної субкультуру, має бути силове витіснення її з України.

\emph{Лариса Гошкодеря}

Прочитала - аж душа просвітліла, які розумні думки, яка мудра грамотна людина!
З такими завжди є шанс на перемогу! Впевнена, що ми таки зробимо Україну
українською!

\emph{Irena Osiunina}

А міністр культури- какаяразніца, або - пожаліємо расєйськаговорящіх і спотворимо закон про мову і освіту! Тільки громдські організаціі зможуть організувати і надавити на цю зеплісняву.

\emph{Олег Дулембов}

Дякувати Богові, що з'явилися люди які розуміють важливість пісні в боротьбі з агресором

\emph{Jaroslava Sokol}

Як хочеться слухати пісні українською мовою, дивитися кіно також,

\emph{Briston Knight}

На тридцятому році незалежности і 7-му році війни це звучить дуже сумно... Коли
маєш такого агресивного сусіда, до мови і роботи усіх без виключення медійних
ресурсів треба було ставитись більш серйозно з перших років незалежности. Більш
серйозно, ніж навіть до армії... Тоді не було б ні того, що сталось з Кримом,
ні з Донбасом.

Відносно регіональних офісів стримінгів... Чому речі, які мала б робити
держава, СБУ перекладаються на громадськість, ентузіастів? А Рада Національної
Безпеки і Оборони чим займається? Тоді я не дивуюсь тому, що у Раді слуги
агресора роблять з мовними питаннями...

\emph{Larisa Oliinyk}

Згодна!В Україні має має бути своя українська культура, український музичний простір, україномовні пісні та співаки.

\emph{Наталья Покотило}

... Про наші битви на папері голо,
Лише в піснях отой вогонь пашить...
(Л.Костенко)

\emph{Олеся Федорова}

Один в полі, не воїн. Це повинно бути на державному рівні. Якщо я намагаюсь
слухати і дивитись українське, то в мене, невеликий вибір. А більшості
\enquote{какаяразніца}.

\emph{Люба Бандрівська}

Що так звана наша незалежна держава робила ці 30 років? Виявляється вона була
незалежна хіба на папері, а реально ми хіба зараз боремося за її незалежність
,як на воєнному фронті України,так і на культурному фронті.

\emph{Василь Шопук}

Не знаю хто очолює радіостанцію \enquote{шансон}, але їх вже давно потрібно було б
закрити. Це тотальне попирання конїтенту. На протязі години, я підраховував,
прозвучало 10 російських і лиш 2 українських пісні і це на Рівненщині.

\emph{Олександра Решотко}

Так! В Україні повинна всюди панувати українська мова!

\emph{Ірина Копцюх-Богданевич}

Повністю погоджуюсь. Як не парадоксально, українці у своїй країні, мають відвойовувати, свій, культурний простір.

\emph{Алла Гладковська}
Ваші слова, як бальзам на душу. Всією душею підтримую українську пісню.

\emph{Olya Gavrilyuk}

Робіть щось з цим російськомовним блатним контентом!

\emph{Сергій Мінтянік}

Всіх, хто розмовляє мовою окупанта, мовою агресора, всіх сепаратюг до Московії і хай там їх хуйло захищає.

\emph{Сергій Рябухін}

Тільки так, і ні інакше

\emph{Bogdana Brukh}
Дуже правильна і актуальна дукамка.
