%%beginhead 
 
%%file 27_09_2023.fb.mariupol.turystychne_misto.1.vsesvitnii_den_turyzmu_povernennja
%%parent 27_09_2023
 
%%url https://www.facebook.com/mistoMarii/posts/pfbid02ZV6oWWftAXVK33PhzhhvsYa8axYKKZqPgNEJwjKUZJM2KBZwrh8McdCL2GEsVDS1l
 
%%author_id mariupol.turystychne_misto
%%date 27_09_2023
 
%%tags 
%%title Всесвітній день туризму: згадуємо та віримо у повернення до рідного міста
 
%%endhead 

\subsection{Всесвітній день туризму: згадуємо та віримо у повернення до рідного міста}
\label{sec:27_09_2023.fb.mariupol.turystychne_misto.1.vsesvitnii_den_turyzmu_povernennja}

\Purl{https://www.facebook.com/mistoMarii/posts/pfbid02ZV6oWWftAXVK33PhzhhvsYa8axYKKZqPgNEJwjKUZJM2KBZwrh8McdCL2GEsVDS1l}
\ifcmt
 author_begin
   author_id mariupol.turystychne_misto
 author_end
\fi

Всесвітній день туризму: згадуємо та віримо у повернення до рідного міста. 🗺️

☝️До повномасштабного вторгнення рф у Маріуполі, попри близькість до лінії
фронту, активно розвивався туристичний напрям. 

У нашому місті з 2018 року працював в історичній будівлі культурно-турис\hyp{}тичний
центр \enquote{Вежа}: звідти стартували різноманітні екскурсії Маріуполем і Приазов'ям,
у центрі проводилися культурні події. 

• У 2021 році \enquote{Вежа} прийняла 18,8 тис. відвідувачів, з яких більше третини
становили іноземці та жителі інших міст України 

Місто Марії стало справжнім культурним і туристичним центром: проводилася
велика кількість фестивалів – MRPL CITY FEST, який 2021 року відвідали 45 тисяч
людей, GogolFest, Mariupol Cla\hyp{}ssic, Open Book, Goby Fest тощо. Культурні події
та туристичні місця приваблювали безліч гостей з різних міст і країн. 2021 року
Маріуполь здобув звання \enquote{Великої культурної столиці} - тоді в місті пройшла
відома Ніч Музеїв, відбулися мультидисциплінарні театральні проєкти за участю
закордонних митців, резиденції PostMost тощо.

• Протягом літа 2021 року маріупольські пляжі щоденно відвідували до 20 тис.
відпочивальників. За сезон пляжі прийняли більше 1 млн відвідувачів, серед яких
20\% - туристи з інших міст України та закордону.  

Серед напрацювань туристичного напряму Маріуполя до повномасштабного вторгнення
– стратегія розвитку туризму, створення туристичного трикутника та
діджитал-продуктів, розробка туристичної навігації та співпраця з амбасадорами
галузі. 

Сьогодні про трагедію Маріуполя говорять в усьому світі – багато митців і
ідейників розповідають про постраждале місто в своїх творах і проєктах. Зі
свого боку команда m.EHUB акцентує увагу на культурній спадщині тимчасово
окупованого міста Марії через тематичні виставки про знищену архітектуру,
навчання ентузіастів у Школі гідів, співпрацю з Вікіпедією. В Україні та Європі
проходять екскурсії-паралелі з нагадуванням про Маріуполь, працює сувенірна
онлайн-крамничка. 

Заходи у нашому місті завжди були змістовними, креативними та яскравими. 

Пропонуємо вам згадати мальовничі локації міста Марії, його вулички та визначні
пам'ятки, святкові події, з якими пов'язані незабутні емоції. Ми віримо в
Перемогу та повернення до рідного міста, яке ще обов'язково розкриє свій
туристичний потенціал. 

Все буде Україна!🇺🇦

