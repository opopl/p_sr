% vim: keymap=russian-jcukenwin
%%beginhead 
 
%%file 15_04_2019.fb.zharkih_denis.1.o_radosti
%%parent 15_04_2019
 
%%url https://www.facebook.com/permalink.php?story_fbid=2383774228502647&id=100006102787780
 
%%author 
%%author_id 
%%author_url 
 
%%tags 
%%title 
 
%%endhead 
\subsection{О радости}
\Purl{https://www.facebook.com/permalink.php?story_fbid=2383774228502647&id=100006102787780}

За время второго Майдана не потерял ни одного друга. Моя позиция и до Майдана
была понятна, так что ко мне особо не цеплялись. Сразу занял позицию против
убийств и столкновений, как только Майдан организовался. Все знакомые по разные
стороны баррикад отнеслись к этому с уважением. 

А вот сейчас теряю друзей. Не то, чтобы очень много, но все- таки... Позиция
моя простая: Ни Порошенко, ни Зеленский не заслуживают поддержки. И тут
высказал крамольную мысль, что Зеленский может быть худшим президентом, чем
Порошенко. И поехало. Оказалось, что у некоторых друзей просто сорвало крышу.
Как так? Мы, можно сказать, ночи не спим, ждем, когда над Порошенко разорвется
возмездие, а он тут, типа, умничает, сомнение вызывает. Да чего сомневаться?
Все же понятно!

Ну, во-первых, ничего над Порошенко не разорвется, и он войдет, скорее всего, в
ВР с хорошим результатом уже осенью.  Рейтинг у него есть, ресурсы ого-го,
короче, никто креста на нем не поставит. Это не я хочу, это реальность такая,
социальная механика. 

О чем мечтают мои уже недрузья? Наказать Порошенко. Да провал на выборах это,
скорее, его спасение, а не наказание. Сдав дела, он уже не отвечает за страну.
А потом активно будет критиковать Зеленского, а уже есть за что. Так что зря не
спят и надеются на чудо. 

Что хочу я? Я хочу, чтобы силы, которые близки мне политически, набирали
рейтинг. Так они будут его набирать и при Порошенко, и при Зеленском. Тут
главное не очароваться Зеленским, сразу за жабры его хватать. Но мои друзья
хотят порвать три баяна, если Порошенко не выиграет. Только он не заплачет.
Заплачут те, кто очаруется Зеленским. Он им, собственно, ничего не должен. Сами
прыгали, радовались.  

Какие-то маленькие радости у моих теперешних недругов.
Маленькие и гаденькие. "Ширше" смотрите, ребята, значительно ширше!

