% vim: keymap=russian-jcukenwin
%%beginhead 
 
%%file 28_09_2021.fb.fb_group.story_kiev_ua.1.gogol_i_kiev
%%parent 28_09_2021
 
%%url https://www.facebook.com/groups/story.kiev.ua/posts/1762652217264905
 
%%author_id fb_group.story_kiev_ua,zolotushkin_anatolij.hajfa
%%date 
 
%%tags gogol_nikolai,kiev,pushkin_aleksandr
%%title Гоголь и Киев
 
%%endhead 
 
\subsection{Гоголь и Киев}
\label{sec:28_09_2021.fb.fb_group.story_kiev_ua.1.gogol_i_kiev}
 
\Purl{https://www.facebook.com/groups/story.kiev.ua/posts/1762652217264905/}
\ifcmt
 author_begin
   author_id fb_group.story_kiev_ua,zolotushkin_anatolij.hajfa
 author_end
\fi

Как нас учили в советской школе Николая Васильевича Гоголя в детстве звали
просто Колей.

А о чем может мечтать маленький Коля, родившийся под Полтавой?

Конечно о Киеве.

Но с первого раза он туда не попал, проскочил мимо, поскольку поехал учиться в
Нежин. А там и сейчас ничего хорошего нет, кроме соленых огурчиков, а тогда тем
более. Гоголь опять захотел в Киев, но получил письмо от Пушкина с приглашением
в Петербург. Ведь, как мы знаем, Пушкин очень любил детей (в хорошем смысле
этого слова), а Коля Гоголь был тогда ещё маленький. 

Поехал Гоголь на родину Путина, но там ему не понравилось - холодно, мокро, по
ночам мосты разводят и вообще наводнения. Затосковал юноша  и написал \enquote{Вечера
на хуторе близ Диканьки} о родной Украине. 

Пушкин прочитал, прослезился, обнял Николая (в хорошем смысле) и решил в гроб
сходя благословить на манер Державина, но передумал и послал Гоголя в   Киев.

Обрадовался Гоголь, ведь он давно о Киеве мечтал.

А в Киеве в ту пору как раз красный Университет имени святого Владимира
открывали и услышав, что сам Гоголь к ним едет, предложили ему профессорскую
должность, генеральский чин и квартиру на Крещатике, угол Прорезной, где после
Паниковский поселился.

Приехал Гоголь, поглядели киевляне на его нос и передумали.

 - У нас тут своих с носами некуда девать, импортных нам не надобно.

Обиделся Гоголь и пошел на Владимирскую горку гулять. А там красота такая, что
вмиг забыл он обиды свои, сел на камешек, вынул телефон и пост в Фейсбуке
затеял:

\begin{zznagolos}
\enquote{Чуден Днепр при тихой погоде, 
Когда вольно и плавно мчит
Сквозь леса и горы полные воды свои}.
\end{zznagolos}

Пару фоток сделал и добавил:

\begin{zznagolos}
\enquote{Редкая птица долетит до середины Днепра.
Пышный, нет ему равной реки в мире!}
\end{zznagolos}

Что тут началось! 

Первым написал комментарий профессор орнитологии и доказал, что практически все
птицы не только долетят до середины Днепра, но и улетят в Африку.

\begin{multicols}{2}
\setlength{\parindent}{0pt}

\ii{28_09_2021.fb.fb_group.story_kiev_ua.1.gogol_i_kiev.pic.1}
\ii{28_09_2021.fb.fb_group.story_kiev_ua.1.gogol_i_kiev.pic.1.cmt}

\ii{28_09_2021.fb.fb_group.story_kiev_ua.1.gogol_i_kiev.pic.2}
\ii{28_09_2021.fb.fb_group.story_kiev_ua.1.gogol_i_kiev.pic.2.cmt}
\end{multicols}

Потом учитель географии вставил, что при всем уважении к Днепру, Дунай больше,
не говоря уже о Ниле и Амазонке.

Потом и простой народ подключился. Многие сами видали птиц, перелетавших через
Днепр туда и обратно, другие читали об этом в литературе. Некоторые намекали на
слабое зрение Николая, особо злостные писали, что Гоголь ничего не видит из-за
своего длинного носа.

Плюнул тогда украинский классик русской литературы и уехал в Рим, поселился
возле фонтана Треви и начал \enquote{Мертвые души} писать. 

Потом, правда, одумались киевляне, памятник поставили. Но было уже поздно,
Гоголь в Киев не вернулся.

О чем это я? 

О некоторых своих комментаторах.

На фото:

1. Днепр при тихой погоде (картина Куинджи)

2. Памятник той самой птице посреди Днепра

\ii{28_09_2021.fb.fb_group.story_kiev_ua.1.gogol_i_kiev.cmt}
