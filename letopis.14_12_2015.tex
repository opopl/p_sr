% vim: keymap=russian-jcukenwin
%%beginhead 
 
%%file 14_12_2015
%%parent dec_2015
 
%%url https://www.facebook.com/dima.grapchev/posts/1229257050418346
 
%%author 
%%author_id 
%%author_url 
 
%%tags 
%%title 
 
%%endhead 
\section{14-12-2015}
\Purl{https://www.facebook.com/dima.grapchev/posts/1229257050418346}

\ifcmt
  pic https://scontent-ber1-1.xx.fbcdn.net/v/t1.18169-9/12366317_1229257040418347_5080245820890842056_n.jpg?_nc_cat=108&ccb=1-3&_nc_sid=8bfeb9&_nc_ohc=J33toUbyjVEAX_cPEBU&_nc_ht=scontent-ber1-1.xx&oh=3cb2bb82517b0d7432b7381f8e6b4ecd&oe=60C24983
\fi

Этот старик умер в доме престарелых. Все считали, что он ушел из жизни, не
оставив в ней никакого ценного следа. Позже, когда медсестры разбирали его
скудные пожитки, они обнаружили это стихотворение.

Входя будить меня с утра, кого ты видишь, медсестра? Старик капризный, по
привычке, ещё «живущий» кое-как. Полуслепой, полудурак. «Живущий» впору взять в
кавычки. Не слышит – надрываться надо. Изводит попусту харчи. Бубнит всё время
– нет с ним сладу. - Ну, сколько можно, замолчи! Тарелку на пол опрокинул. Где
туфли? Где носок второй? Слезай с кровати! Чтоб ты сгинул…

Сестра! Взгляни в мои глаза! Сумей увидеть то, что за...За этой немощью и
болью, За жизнью прожитой, большой. За пиджаком, «побитым» молью, За кожей
дряблой, «за душой». За гранью нынешнего дня, попробуй разглядеть МЕНЯ…… Я
мальчик! Непоседа, милый. Весёлый, озорной слегка. Мне страшно. Мне лет пять от
силы. А карусель, так высока! Но вон отец и мама рядом. Я в них впиваюсь цепким
взглядом. И хоть мой страх неистребим, Я точно знаю, что ЛЮБИМ…

…вот мне шестнадцать, Я горю! Душою в облаках парю! Мечтаю, радуюсь, грущу. Я
молод, Я ЛЮБОВЬ ищу…… и вот он, мой счастливый миг! Мне двадцать восемь. Я -
жених! Иду с ЛЮБОВЬЮ к алтарю, И вновь горю, горю, горю…… мне тридцать пять,
растёт семья. У нас уже есть сыновья. Свой дом, хозяйство. И жена. Мне дочь
вот-вот родить должна…

…а жизнь летит, летит вперёд! Мне сорок пять – «круговорот»! И дети «не по
дням» растут. Игрушки, школа, институт… Всё! Упорхнули из гнезда! И разлетелись
кто куда. Замедлен бег небесных тел. Наш дом уютный опустел…но мы с ЛЮБИМОЮ
вдвоём! Ложимся вместе и встаём. Она грустить мне не даёт. И жизнь опять летит
вперёд…… теперь уже мне шестьдесят. Вновь дети в доме голосят! Внучат весёлый
хоровод. О, как мы СЧАСТЛИВЫ!

Но вот…… померк внезапно солнца свет. Моей любимой больше нет! У счастья тоже
есть предел… Я за неделю поседел. Осунулся, душой поник. И ощутил, что Я
старик…… теперь живу Я «без затей». Живу для внуков и детей. Мой мир со мной,
но с каждым днём Всё меньше, меньше света в нём. Крест старости взвалив на
плечи, Бреду устало в никуда. Покрылось сердце коркой льда. И время боль мою не
лечит. О, Господи, как жизнь длинна, Когда не радует она…

… но с этим следует смириться. Ничто не вечно под луной. А ты, склонившись надо
мной, Открой глаза свои, сестрица. Я не старик капризный, нет! Любимый МУЖ,
ОТЕЦ и ДЕД … … и мальчик маленький, доселе В сиянье солнечного дня Летящий
вдаль на карусели… Попробуй разглядеть МЕНЯ!

Кирилл Винда
