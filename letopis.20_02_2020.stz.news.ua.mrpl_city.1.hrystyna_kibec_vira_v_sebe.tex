% vim: keymap=russian-jcukenwin
%%beginhead 
 
%%file 20_02_2020.stz.news.ua.mrpl_city.1.hrystyna_kibec_vira_v_sebe
%%parent 20_02_2020
 
%%url https://mrpl.city/blogs/view/hristina-kibets-ne-vtrachajte-viru-v-sebe
 
%%author_id demidko_olga.mariupol,news.ua.mrpl_city
%%date 
 
%%tags 
%%title Христина Кібець: "Не втрачайте віру в себе!"
 
%%endhead 
 
\subsection{Христина Кібець: \enquote{Не втрачайте віру в себе!}}
\label{sec:20_02_2020.stz.news.ua.mrpl_city.1.hrystyna_kibec_vira_v_sebe}
 
\Purl{https://mrpl.city/blogs/view/hristina-kibets-ne-vtrachajte-viru-v-sebe}
\ifcmt
 author_begin
   author_id demidko_olga.mariupol,news.ua.mrpl_city
 author_end
\fi

Наступна героїня моєї серії, присвяченої талановитим та непересічним
маріупольцям, є унікальною і талановитою художницею, роботи якої відрізняються
особливим авторським баченням. Кожне з полотен несе свою інформацію, свій
посил, свою відповідь. Художниця \textbf{Христина Кібець} у своїх роботах закликає
замислитися над багатьма проблемами сучасності, а вишуканий колорит, складна
палітра відтінків підсилюють враження і привертають увагу багатьох
маріупольців...

Народилася Христина в Маріуполі. Батьки розлучилися, коли їй був рік.
Виховувала дівчинку бабуся, проте Христина зберегла спілкування з обома
батьками, яких дуже любить. Загалом сім'я у дівчини дуже велика. Найближчі – це
бабуся, мама, вітчим, який в всьому підтримує, сестрички та брат. У художниці
вже є і власна сім'я – коханий чоловік з улюбленою донечкою та чудові батьки
чоловіка, з якими вони живуть під одним дахом.

\ii{20_02_2020.stz.news.ua.mrpl_city.1.hrystyna_kibec_vira_v_sebe.pic.1}

Любов до Маріуполя почалася зі вступу до університету. Тоді вона вже
зустрічалися з майбутнім чоловіком і часто прогулювалися вулицями міста,
парками, спускалися до моря... Вона вважає, наше місто далеке від ідеалу. І в той
же час Маріуполь неможливо не любити: за його унікальну багату історію,
неповторні пам'ятки, за людей, яким не байдужа доля свого міста і які роблять
все можливе, щоб Маріуполь був сучасним та європейським. Навіть присвятила
місту одну зі своїх картин – \textbf{\enquote{Дим сигарет з формальдегідом}}. Показала не
найкращу сторону, але лише для того, щоб звернути увагу людей на проблеми
Маріуполя, на які не можна закривати очі. І в той же час в образі дівчини я
хотіла передати, що наше місто чарівне.

\ii{20_02_2020.stz.news.ua.mrpl_city.1.hrystyna_kibec_vira_v_sebe.pic.2}

За освітою наша героїня історик. Закінчила історичний факультет МДУ в 2016
році, отримавши диплом спеціаліста.

Після закінчення університету пройшла курси живопису і скетчінга в художній
студії Free Art Studio на віддаленій основі. Паралельно працювала менеджером у
\enquote{Work in Poland}. Недовго пропрацювавши, пішла в декрет і ось зараз
вона мама чудової дівчинки. Поки ніде не працює, здебільшого займається
фрілансом.

Христина бере активну участь у художніх конкурсах зі шкільних років. Останній
був у 2016 році – \textbf{\enquote{Малюємо заради миру}} до Дня Захисту дітей,
де робота нашої героїні була відзначена як одна з найкращих.

\ii{20_02_2020.stz.news.ua.mrpl_city.1.hrystyna_kibec_vira_v_sebe.pic.3}

Один з найцікавіших проєктів для Христини подарувала співпраця з компанією
KADZAMA, яка пропонує обладнання для виробництва шоколаду ручної роботи. Вона
малювала ескізи форм для шоколаду і самого обладнання. Було дуже цікаво. А
потім, менеджер цієї компанії спільно з професійним шоколат'є \textbf{Регіною Паєвською}
вирішили створити свій проєкт \textbf{\enquote{She chocolate}}, і залучили її в якості дизайнера
обкладинки шоколадної обгортки. Їхні шоколадки побували в різних зарубіжних
країнах, дівчата всіляко піарили проєкт. Однак проіснував він недовго, були на
те причини. Втім ця робота принесла Христині безцінний досвід та нових
клієнтів, для яких вона почала розробляти дизайн етикеток, візиток і логотипів.

Найбільше надихає художницю природа та людська діяльність як творча, так і
руйнівна. Христина з легкістю може підготувати яскравий портрет для людини,
якій симпатизує, чи роботу, присвячену болючій темі, яка змусить задуматися,
хоч на мить про те, що відбувається навколо нас. Спектр порушених художницею
тем дуже великий, тому її твори користуються популярністю.

На одну картину йде від одного до 4-х вечорів, тобто, від 5 до 20 годин.
Художниця щодня докладає чимало зусиль, щоб відшліфовувати власну техніку. У
майбутньому вона хоче, щоб картини виходили більш об'ємними, щоб виходило більш
чітко передавати світло і тінь, пропорції. Завдяки цілеспрямованості і
старанності вона помічає, що поступово, маленькими кроками йде до своєї мети.

З напрямом, в якому працює маріупольчанка, складно визначитися. На думку
Христини, її роботи символічні, практично в кожній закладена ідея, але з
часткою сюрреалізму. Між цими двома напрямами проходить дуже тонка грань.

Сподівається найближчим часом підготувати ще декілька робіт на полотні та
зайнятися дизайном поліграфії. У дівчини велика група підтримки – як сім'я, так
і друзі. В цьому плані їй дійсно пощастило, адже для кожної творчої людини дуже
потрібна і важлива підримка близьких. 

\ii{20_02_2020.stz.news.ua.mrpl_city.1.hrystyna_kibec_vira_v_sebe.pic.4}

Дуже цікаво було б побачити на одній з сірих стін Маріуполя картину, що змушує
замислитися, пробуджує людяність та нагадує про важливі проблеми
повсякденності. Думаю, з робіт Христини найбільше б вразили багатьох містян –
\textbf{\enquote{Домашнє насильство} з двох серій,} \textbf{\enquote{Час}} \textbf{або \enquote{Дозволь мені залишитися в твоєму
серці}.}

\textbf{Улюблені місця в Маріуполі:} Драмтеатр, Міський парк.

\textbf{Улюблені книги:} \enquote{Кульбабове вино} Рея Бредбері, наповнена теплом, сонцем і
пахне літом. \enquote{Ті, що співають у терні} Коллін Мак-Каллоу – книга, яка вразила
до глибини душі.

\textbf{Улюблені фільми:} \enquote{Красуня} (1990), \enquote{Дон Жуан де Марко} (1995), \enquote{Богемська
рапсодія} (2018).

\textbf{Хобі:} 

\begin{quote}
\em\enquote{Шалено люблю збирати пазли, це моя пристрасть. Раніше колекціонувала
комах. І якщо врахувати, що я за духом пацифіст, то в моїй колекції кожна
комаха померла природною смертю. Іншими словами – я їх просто знаходила і
підбирала}.
\end{quote}

\textbf{Порада маріупольцям:} 

\begin{quote}
\em\enquote{Віра в себе - ось моя порада маріупольцям, які бажають
привнести щось нове в розвиток міста. Не бійтеся нових починань, не бійтеся
помилок або засуджень. Це нормально. Дійте, не стійте на місці, розвивайтеся! І
головне, щоб це приносило задоволення в першу чергу вам. Кожен з нас здатний
зробити наше місто кращим!}.
\end{quote}
