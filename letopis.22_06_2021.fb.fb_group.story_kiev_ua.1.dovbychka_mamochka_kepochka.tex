% vim: keymap=russian-jcukenwin
%%beginhead 
 
%%file 22_06_2021.fb.fb_group.story_kiev_ua.1.dovbychka_mamochka_kepochka
%%parent 22_06_2021
 
%%url https://www.facebook.com/groups/story.kiev.ua/posts/1690449877818473/
 
%%author_id fb_group.story_kiev_ua,petrova_irina.kiev
%%date 
 
%%tags gorod,kiev,rasskaz,zhizn
%%title Довбычка, мамочка и кепочка
 
%%endhead 
 
\subsection{Довбычка, мамочка и кепочка}
\label{sec:22_06_2021.fb.fb_group.story_kiev_ua.1.dovbychka_mamochka_kepochka}
 
\Purl{https://www.facebook.com/groups/story.kiev.ua/posts/1690449877818473/}
\ifcmt
 author_begin
   author_id fb_group.story_kiev_ua,petrova_irina.kiev
 author_end
\fi

Довбычка, мамочка и кепочка

Моё внимание привлекло слово «Довбычка» и странное утверждение, что «купаться
там мы не сможем».

После прекрасно проведенного дня у подружки Ируси на даче, я ехала в маршрутке.
Желтый бусик лениво тащился вдоль жутких монстров – жилых домов Позняков.
Позади меня, судя по голосам, сидела пара людей, на их разговор я не обращала
внимания до тех пор, пока не прозвучало это дикое, на мой взгляд, утверждение.
Как это – нельзя купаться на Довбычке? А где тогда вообще можно?! Желтый
шелковый песок, небыстрое течение Венецианской затоки, вид на правый берег, от
которого на ум приходит: «Господи, за что мне такая благодать?» И – нельзя
купаться...

\ifcmt
  ig https://scontent-frx5-1.xx.fbcdn.net/v/t1.6435-9/206410636_4419996534700586_5777515327084307690_n.jpg?_nc_cat=110&ccb=1-5&_nc_sid=b9115d&_nc_ohc=9ZHMxKEKx80AX-5Huw7&_nc_ht=scontent-frx5-1.xx&oh=2a79bf91f54c31df63073778d3b5f650&oe=61B2805D
  @width 0.3
  %@wrap \parpic[r]
  @wrap \InsertBoxR{0}
\fi

Невольно я стала прислушиваться к разговору, надеясь услышать объяснение.
Мужской голос сообщал своей собеседнице, что-де сегодня они с мамочкой пошли на
базар, но мама смогла совершить лишь половину запланированных покупок, ибо
из-за сильного ветра у неё голова раскалывалась на куски, а кепочку она забыла
дома.

При этом голос мужчины был монотонен и бесцветен, он немного заикался,
протягивал начало каждого слова, чуть пришепетывал, нечетко произносил букву
«р», а вот букву «ч» выговаривал очень твердо, так её произносят белорусы или
поляки.

Живописание страданий мамоЧЪки без кепоЧЪки заняло немало минут. При этом сын
этой мамоЧЪки покаялся, что был несдержан и вспылил, недовольный маминой
забывчивостью. «Как, как можно было забыть кепоЧЪку?» - вопрошал мой невидимый
попутчик. Но, потом все завершилось неплохо для мамоЧЪки. О чем было понятно из
разговора по мобильному. После тщательного расспроса о самочувствии, пожеланий
здоровья, строгого предупреждения не выходить на ветреный балкон, настоятельных
рекомендаций поесть еще пару раз, и искреннего раскаяния о своей несдержанности
и вспыльчивости на базаре, разговор с мамой был исчерпан. В конце оного было
доложено, что «они гуляют с Ларисой». Теперь я знала имя его спутницы.

Наконец вступила в беседу и Лариса. Робким голосом она спросила, что же должна
откушать мамочка? Заботливый сын подробно описал приготовленную им еду –
молодую картошечку дольками, котлетки из телятинки и протертую с сахарком
клубничку. Всё меню было просюсючено уменьшительно-ласкательными формами, но
толк в поварском ремесле он знал.

И если бы моя подружка не накормила меня, как гуся перед Рождеством, то можно
было бы захлебнуться слюной от описываемого кулинарного мастерства.

Постепенно тема мамочки сошла на нет. В моем воображении уже были готовы
портреты мамоЧЪки, её сыночка и немногословной Ларисы. Сын представился мне
сорокалетним, или около этого, немного помятым мужичонкой, лысоватым, с
«заёмом» из редкой жирноватой пряди волос, в босоножках времен тотального
дефицита на босу ногу, несвежих унылых брючатах и в помятой футболке с
олимпийским мишкой на груди. А Лариса – тоже уже несколько подтоптанная
одинокой жизнью девица бальзаковского возраста, в крепдешиновом платьишке,
шлепанцах, непременных для таких дам «следах», и с «шестимесячной» на
пергидрольной головке. Серая и скучная, как утро понедельника.

\ii{22_06_2021.fb.fb_group.story_kiev_ua.1.dovbychka_mamochka_kepochka.pic.tramvaj}

Она постаралась переключить разговор с мамоЧЪки на предполагаемую поездку
куда-нибудь за город. Ибо, по уже сложившейся традиции, вся наша богатая страна
опять отмечала очередной праздник, и выходные были «длинными».

Он предложил альтернативный Довбычке вариант – озера Пущи-Водицы. Я, мысленно
уже подключившись к сопереживаниям о мамочке, вкусным описаниям ужина с
котлетками, одобрила и этот проект. И в самом деле, волшебные озера берендеевых
лесов Пущи, чистые и теплые, поездка туда на трамвае через лес – тоже неплохой
уик-энд. Вот только выезжать надо бы пораньше, резонно заметила Лариса. Да-да,
согласился ОН (его имя пока оставалась для меня загадкой), но непременно с утра
надо приготовить маме завтрак и проследить за приемом лекарств. Тут услышанное
заинтересовало меня не на шутку – отчего человек, свободно передвигающийся по
базару, пусть даже без кепоЧЪки, не в состоянии сделать себе легкий,
соответствующий возрасту, завтрак и проглотить горсть таблеток? Из-за такой
тотальной беспомощности мамочки, выехать в Пущу раньше одиннадцати не
представлялось возможным. Лариса обреченно согласилась и опять замолкла. А
голос ЕГО всё звучал, ныл и тянулся. Оказалось, что это не первая их совместная
поездка в Пущу, а вот в прошлый раз Лариса позволила себе такое высказывание:
«Вот и дождь собрался, даже погода против нас». Тут последовал тщательный
разбор и допрос с пристрастием, ЧТО имела в виду Лариса, бросая такую фразу.
Почему она считает, что против них может что-то возникнуть у природы, ведь он
настолько хорошо к ней относится, считает её своей спутницей, она не должна ни
в чем сомневаться и тому подобные хрестоматийные утверждения. «Ну, Егор, -
робко возразила Лариса (вот оно, ИМЯ, сестра, имя!) - я же вовсе не это имела в
виду, я просто хотела сказать, что мы так редко куда-нибудь выбираемся, и это
нечастое событие может быть испорчено (и еще несколько оправданий в таком же
духе)». Егор, а у меня он сразу трансформировался в Гошу, возразил
безаппеляционно: «В следующий раз, думай, что говоришь!» - был вынесен строгий
вердикт. Лариса согласилась с таким вариантом. Но безжалостный правдолюбец Гоша
не успокаивался. «И вот еще один момент, - продолжал гундосить мужчинка, - вот
ты как-то нехорошо (!!! прим. автора) спросила, мол, мы уже никуда и поехать не
сможем? Что ты хотела этим сказать?» Лариса тихо объяснила, что она имела в
виду перспективу совместной поездки куда-нибудь за пределы Киева, ведь лето,
отпуск и т.п. Гнев Гоши был не наигранным: «А о мамоЧЪке ты подумала? Как я её
оставлю???»

Лариса пыталась объяснить, что, мол, у мамы в наличие руки, ноги, голова и
прочие части туловища, она самостоятельно передвигается, адекватно реагирует на
происходящее во внешнем мире (что лично у меня вызвало некоторые сомнения), и
некоторое непродолжительное время может побыть без тотальной опеки.

Но, как же категорически не согласен был Гоша. Лариса попыталась выдвинуть
новый аргумент. А вот, если бы (ох, это сослагательное наклонение, как оно
подчас все портит), они были вместе, и у них бы родился ребенок, и если бы этот
выросший ребенок был на каникулах, они что, и тогда не смогли бы никуда с ним
выехать? При этом, очевидно, мамоЧЪке отмеривалось два века априори. Нет!!! -
Гоша отсек эти поползновения на корню, пусть едет в пионЭрлагерь. «Но, что и с
родителями никуда, никуда нельзя съездить? Вообще никуда?» – тихим воплем
застрадала Лариса. Точен и холоден был ответ Гоши – «Никуда! Возможности нет,
значит никуда!». А дальше последовало трудноперевариваемое логикой
умозаключение Егорки – «Я же тебя люблю, я тебе не изменяю (при этих словах мне
захотелось увидеть еще одну, кроме безответной Ларисы, охотницу на такой
непотреб), я хочу на тебе жениться (наконец-то!, вот только как же мамоЧЪка?),
и ты должна понимать всю сложность моего положения». Воистину, куда уж сложнее!
Я почувствовала на своей шее легкое дуновение от вздоха Ларисы. Любопытно,
подумала я, а был ли Гоша сам в пионерлагере? Отпускала ли его туда чадолюбивая
мамочка? И если да, то сколько же раз такому противному мальчишке устраивали
«темную» и мазали зубной пастой?

Никак, ну никак не представлялось возможности обернуться и посмотреть на эту
дивную парочку. Мне было невыносимо страшно, что они, встав, пойдут на выход к
задней двери, и, максимум, что я могу увидеть – их спины. Но судьба преподнесла
мне сюрприз. Мы уже подъезжали к метро, вокруг тянулась промзона – гаражи,
железная дорога, какие-то склады. И тут Егорка запросил остановку. Они встали,
мимо меня направились к передней двери. И тут… Впереди шел атлет-красавец,
роста под метр девяносто, загорелый, с бицепсами, трицепсами, и всякими прочими
группами мышц, количество «кубиков» на плоском животе красавца я сосчитать не
успела, аккуратную маленькую попку обтягивали недешевые белоснежные джинсы,
светлые туфли мягкой кожи сами кричали о своей нескромной цене, торс и плечи
можно было использовать в рекламе фитнесс-клуба. Да какой там Джонни Депп !!!
Под стать ему была и девушка. Роскошная грива пепельных волос, точеная фигурка,
смуглые ножки в супермодных босоножках. У меня отняло дар всего – речи,
соображения, адекватного восприятия действительности. Я лихорадочно стала
оглядываться – может, это другие люди идут к двери, может ЭТИ еще сидят? Но
нет, места позади меня были пусты, а в глубине автобуса виднелась парочка
мужичков и молодая мама с ребёнком.

В следующее мгновение мне захотелось повиснуть на руке этой Ларисы с воплем:
«Остановись, пошел он к такой-то ... мамоЧЪке! Ты – красавица, характер –
бархат!», но, они уже пружинисто выпрыгнули из бусика, и пошли прочь.

Так и остались два вопроса без ответов. Куда он повел красавицу Ларису гулять
по промзоне на закате летнего дня? И второй – почему нельзя купаться на
Довбычке?
