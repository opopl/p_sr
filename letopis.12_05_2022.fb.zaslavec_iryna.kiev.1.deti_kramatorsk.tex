% vim: keymap=russian-jcukenwin
%%beginhead 
 
%%file 12_05_2022.fb.zaslavec_iryna.kiev.1.deti_kramatorsk
%%parent 12_05_2022
 
%%url https://www.facebook.com/irina.zaslavets/posts/4822471917857680
 
%%author_id zaslavec_iryna.kiev
%%date 
 
%%tags 
%%title Я хочу, щоб історії цих дітей побачив світ!
 
%%endhead 
 
\subsection{Я хочу, щоб історії цих дітей побачив світ!}
\label{sec:12_05_2022.fb.zaslavec_iryna.kiev.1.deti_kramatorsk}
 
\Purl{https://www.facebook.com/irina.zaslavets/posts/4822471917857680}
\ifcmt
 author_begin
   author_id zaslavec_iryna.kiev
 author_end
\fi

%\ii{12_05_2022.fb.zaslavec_iryna.kiev.1.deti_kramatorsk.eng}

Поки він пильнував речі в приміщенні вокзалу, на перон, де були мама і рідна
сестричка, прилетіла ракета. 

\ii{12_05_2022.fb.zaslavec_iryna.kiev.1.deti_kramatorsk.pic.1}

Є вислів \enquote{серце розривається}, але він неспівмірний із тим, що відчуваєш, коли
заходиш до них в палату. Три ліжка. На одному - 11-річна дівчинка Яна з
ампутованими ногами. Навпроти - мама Наталя. Вона теж не ходить: ліва нога
ампутована, права - поранена. На третьому ліжку - 11-річний Ярослав. Вони з
сестричкою близнята. 

\ii{12_05_2022.fb.zaslavec_iryna.kiev.1.deti_kramatorsk.pic.2}

Ярославчик не за віком дорослий. Він єдиний вцілів в тій трагедії і взяв на
себе відповідальність по догляду за мамою і сестрою. Принести води, допомогти
одягнутися, збігати в магазин... Дитя сумлінно усе виконує. Він тепер - опора. 

\ii{12_05_2022.fb.zaslavec_iryna.kiev.1.deti_kramatorsk.pic.3}

Наталя час від часу плаче. Не за себе, за дітей. За каліцтва фізичні і душевні.
Каже, що ніколи себе не пробаче, що вийшла з донькою на перон по чай, який
роздавали волонтери. За те, що не лишилася в приміщенні, не вберегла... 

30 поранений в Краматорську на вокзалі діток за останній місяць привезли лише
до нашої дитячої Лікарні Св. Миколая
\href{https://www.facebook.com/1tmolviv/}{Перше медичне об'єднання міста
Львова}.  Історія кожного - суцільний біль. Палата за палатою - це каліцтва,
загибель рідних, страх... Є дівчинка, яка втратила там маму. Вона ховається під
ковдру щоразу, коли до неї заходять...

Я хочу, щоб історії цих дітей побачив світ! Щоб ні в кого більше не лишалося
сумнівів, що так суцільне зло. Щоб ніхто більше не осмілився назвати конфліктом
те, що вбиває і калічить дітей!!!

\ii{12_05_2022.fb.zaslavec_iryna.kiev.1.deti_kramatorsk.cmt}
\ii{12_05_2022.fb.zaslavec_iryna.kiev.1.deti_kramatorsk.cmtx}
