% vim: keymap=russian-jcukenwin
%%beginhead 
 
%%file 21_05_2021.fb.nicoj_larisa.2.kiev_magazin_welfare_mova
%%parent 21_05_2021
 
%%url https://www.facebook.com/nitsoi.larysa/posts/919581438608229
 
%%author Ницой, Лариса
%%author_id nicoj_larisa
%%author_url 
 
%%tags mova,jazyk,kiev,ukraina,magazin,ukrainizacia
%%title Київ. Сепарський магазин "Welfаre"
 
%%endhead 
 
\subsection{Київ. Сепарський магазин \enquote{Welfаre}}
\label{sec:21_05_2021.fb.nicoj_larisa.2.kiev_magazin_welfare_mova}
\Purl{https://www.facebook.com/nitsoi.larysa/posts/919581438608229}
\ifcmt
 author_begin
   author_id nicoj_larisa
 author_end
\fi

\obeycr
Київ. Сепарський магазин \enquote{Welfаre}.
Не встигла відчинити двері, продавчиня до мене:
- Здравствуйтє.
Щоб одразу розставити всі крапки над \enquote{і}, кажу голосно і з притиском:
- Доброго дня!
Продавчиня продовжує гнути своє чужою мені мовою. Ну, думаю, всяке буває. Не
всі університети закінчували. Не всі зразу здогадуються, що мова першого
звернення в магазинах - українська. Хоча, скільки ж уже можна бути такими
вперто \enquote{не здогадливими}, коли давно треба не здогадуватися, а ЗНАТИ.
- Вам памочь с вибарам? - продовжує ігнорувати мої права і Закон. - У нас новая каллєкція! 
- А ще в нас Україна! - дивлюся їй у вічі. - І українська мова.
- Ну канєш-ш-ш-ш-шна! - фиркає продавчиня, розвертається і йде до іншого покупця, - Ну как вам, падходіт?...
- Я не зрозуміла, - кажу їй, - що це за іронія? Що значить ваше \enquote{канєшна}? 
Відповіді не отримала. Може власники магазину мені, та й усім українцям заодно,
пояснять, що мала на увазі їхня продавчиня 21-го травня о 10.40 в їхньому
магазині на вулиці Гончара. Це пряме недопрацювання керівництва з персоналом.
Можете мені подзвонити і вибачитися. Телефон є у вашій базі. 
\restorecr

\ifcmt
  pic https://scontent-lga3-2.xx.fbcdn.net/v/t1.6435-0/p180x540/188965506_919581418608231_5749260326584150764_n.jpg?_nc_cat=110&ccb=1-3&_nc_sid=8bfeb9&_nc_ohc=orJ2m4vi1iMAX9eWMVW&_nc_ht=scontent-lga3-2.xx&tp=6&oh=55a5720b3096d6d8c2747cf6c3fc0666&oe=60CCF7D3
\fi

А поки сюди ні ногою й інших українців закликаю.

Так, товариство, сьогодні через кілька годин після моєї публікації, мені
телефонували з офісу і вибачалися. Молодці, спрацювали оперативно, телефон
знайшли, розмовляли чемно. Окрім того, продавчиня написала в приват листа,
пообіцяла надалі українську мову не ігнорувати. Я вибачення прийняла. Тому,
заклик не ходити до цього магазину - скасовую. Навпаки, можете піти,
насолодитися їхньою українською, а як щось сподобається, то й купити ))


Стефанія Кучерко

Не можу не похвалитися, пані Ларисо, на вашій малій Батьківщині, ще
Кіровоградщині, як не прикро, в Знам"янці, нарешті чую українську в сфері
обслуговування, паростки української. Слава Богу дочекалася, а живу тут майже 40
років. Сьогодні була в Приватбанку, на касі молода дівчина у вишиванці,
звертається українською. Подякувала, сказала що гарна вишиванка, як і
вона. Обличчя її закрите маскою, очі посміхнулися, а на мої навернулися
сльози. Може не всі зрозуміють мої відчуття.

Лариса Ніцой

Стефанія Кучерко Розумію. Так само часто млію ))

Руслан Максимов

Продавчиня заслужила на те, щоб також допомогти їй зі звільненням з роботи.

Andrii Vitrenko

Добре було б скерувати відповідну скаргу на адресу Уповноваженого із захисту
державної мови. Відреагуємо.

Лариса Ніцой

Andrii Vitrenko Я не зняла на відео доказів. Пішла, вже потім подумала. Без
відео приймете?

Andrii Vitrenko

Лариса Ніцой Так, приймемо до розгляду, проаналізуємо. Можливо, були інші
скарги на суб'єкта також.

Юлия Слепенко

Andrii Vitrenko доброго дня. Підкажіть будьласка, чи розповсюджується
повноваження органу з захисту мови на інтернет ресурси?

Andrii Vitrenko

Юлия Слепенко Доброго дня! Так, мовний закон регламентує і це питання. Прошу
скористатись нашим сайтом: www.mova-ombudsman.gov.ua

Andrii Vitrenko

Юлия Слепенко Секретаріат Уповноваженого розглядає скарги щодо порушень і в питанні інтернет ресурсів також.

Юлия Слепенко

Andrii Vitrenko дякую за відповідь

Наталія Лозинська

Пані Ларисо! Щиро ділюсь із вами своїм досвідом! ))) Я в таких випадках починаю
набирати поліцію! Повірте! Діє! Зразу згадують, що треба українською! )))
Манкурти, направду, боягузи!

Stas Chornyy

Дивлюсь шоу політиків і всі суки патріоти і продажні на мові какой разніци
розмовляють....
