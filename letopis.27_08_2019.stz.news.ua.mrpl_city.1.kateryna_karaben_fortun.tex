% vim: keymap=russian-jcukenwin
%%beginhead 
 
%%file 27_08_2019.stz.news.ua.mrpl_city.1.kateryna_karaben_fortun
%%parent 27_08_2019
 
%%url https://mrpl.city/blogs/view/katerina-karaben-fortun-mij-vlasnij-everestshhodenne-samovdoskonalennya
 
%%author_id demidko_olga.mariupol,news.ua.mrpl_city
%%date 
 
%%tags 
%%title Катерина Карабєн-Фортун: "Мій власний Еверест – щоденне самовдосконалення"
 
%%endhead 
 
\subsection{Катерина Карабєн-Фортун: \enquote{Мій власний Еверест – щоденне самовдосконалення}}
\label{sec:27_08_2019.stz.news.ua.mrpl_city.1.kateryna_karaben_fortun}
 
\Purl{https://mrpl.city/blogs/view/katerina-karaben-fortun-mij-vlasnij-everestshhodenne-samovdoskonalennya}
\ifcmt
 author_begin
   author_id demidko_olga.mariupol,news.ua.mrpl_city
 author_end
\fi

\ii{27_08_2019.stz.news.ua.mrpl_city.1.kateryna_karaben_fortun.pic.1}

Таланти та захоплення нашої наступної героїні не можуть не вражати. Вона –
поетеса та письменниця, художниця і викладачка – продовжує невпинно творчо
зростати і займатися щоденним самовдосконаленням. Великий досвід її роботи та
кількість нагород свідчать про високу працездатність та вміння легко і швидко
всьому навчатися. Пропоную вам познайомитися ближче з самобутньою, щирою,
відкритою і унікальною маріупольчанкою \textbf{Катериною Геннадіївною Карабєн-Фортун}.

Народилась Катерина наприкінці листопада у студентській сім'ї в Донецьку. Тоді
у далекому 1986-му сніг був заввишки у півметра. Трохи пожила з батьками в
Красному Лучі Луганської області, але, починаючи зі свідомого дитинства, вона
живе у Маріуполі і вважає його своїм рідним містом.

У 2003 році закінчила з відзнакою фізико-математичний факультет у
Маріупольському міському ліцеї. Того ж року вступила на філологічний факультет
Маріупольського державного університету. Навчалась на денній формі на бюджеті,
закінчила його також з відзнакою як спеціаліст англійської мови та спеціаліст
україн\hyp{}ської філології. Саме тоді вивчила італійську мову. Також Катерина вільно
володіє іспанською, яку пізніше вивчила самотужки та склала іспит у Харкові у
2016 році.

\ii{insert.read_also.burov.ivan_sidorchuk}
\ii{27_08_2019.stz.news.ua.mrpl_city.1.kateryna_karaben_fortun.pic.2}

Загалом жінка має безліч сертифікатів та дипломів різних спеціальностей, бо
вважає що людині XXI століття потрібно весь час розвиватись та
вдосконалюватись. Досвід роботи Катерини також має, адже диплом замість людини
не буде писати статтю або рятувати спину хворому. Тож вона має досвід і
журналіста, і фотокореспондента, і копірайтера / рерайтера, вчителя танців,
масажиста з традиційних технік та майстра креольського масажу бамбуковими
паличками, дизайнерки одягу та приміщень, майстрині народних ремесел,
викладачки трайбл-денсу, художника-декоратора та, звичайно, викладача іноземних
мов.

Після народження двох синочків багато цих навичок використовує вдома, адже, як
наголошує наша героїня, \emph{\enquote{мати – це сотні професій, які об'єдналися завдяки
єдиному і найголовнішому ресурсу – любові}}.

Вчителем англійської мови працювала і в 45-й, і в 24-й школах, зараз вже третій
рік працює в гімназії № 2 (викладачем англійської, іспанської та італійської
мов). Загалом педстаж Катерини становить 12 років. Їй подобається працювати з
підростаючим поколінням, ділитися знаннями та досвідом. 

\ii{27_08_2019.stz.news.ua.mrpl_city.1.kateryna_karaben_fortun.pic.3}

З 12-ти років Катрін почала писати вірші. Після перемоги у конкурсі \enquote{5 очима
молоді} у 2003 році її запросили до літературного гуртка \enquote{Азов'є}, де творчими
вчителями були \textbf{Павло Бессоннов} та \textbf{Наталя Харакоз}. В університеті з дівчиною
працював \textbf{Богдан Васильович Слющинський}. Він дуже багато працював над створенням
літгуртку \enquote{Перлини слова}, був справжнім Гуру поетичного слова для Катерини.
Авторка видала дві збірки поезій українською мовою (\textbf{\enquote{Барви мого серця} – 2012},
\textbf{\enquote{Промінь слова} – 2013}) і одну збірку прози українською мовою (\textbf{\enquote{Пил, попіл та
склянка фіалка} – 2016}).

З віршами нашої героїні можна ознайомитися на порталі \enquote{Проба пера}, а також на
її сторінках у Facebook та Instagram. Вона є лауреатом конкурсів та премій
\textbf{\enquote{Творча юнь Донбасу} (2003)}, \textbf{\enquote{Паперовий літак} (2016)}, лауреат \textbf{Міжнародної
слов'янської премії 2019 р.} (третє місце), лауреат \textbf{Маріупольської книжкової
толоки 2018 р}., \textbf{\enquote{Книги і преси} (Маріуполь 2014, 2018)}. Поетеса та письменниця
з 2016 року є членом \textbf{Національної Спілки письменників України}.

Водночас Катерина Геннадіївна – член \textbf{Ліги вільних художників ARS ALTERA (з 2017
року)}. Художниця працює в дуже різних жанрах: акрил, олія, пастель, туш,
сумі-є. Малювати може на сирій штукатурці, дереві та тілі – брала участь в
\textbf{боді-арт шоу ATLANTIS} в Маріуполі і Бердянську. Також малює мехенді хною, а
починалося все з розпису апельсинових скоринок. Катерина провела дві
персональні виставки. Також її три роботи взяли участь у загальних
маріупольських виставках та дві картини – у всеукраїнських виставках. Водночас
Її роботи представлені у приватних колекціях Маріуполя, Львова, Одеси, Києва,
Лос-Анжелеса (США). Шість робіт художниці взяли участь у цьогорічному
міжнародному проекті \textbf{\enquote{Пост Мост}}. З 2017 р. Катерина – член клубу любителів
японської культури, вже другий рік бере участь у фестивалі \textbf{Східfest} зі своїми
картинами в східному стилі.

\textbf{Читайте також:} \emph{Балерина Лилия: Как девочка стала танцующим лебедем}%
\footnote{Балерина Лилия: Как девочка стала танцующим лебедем, Анатолий Ломакин, mrpl.city, 13.08.2019 \par%
\url{https://mrpl.city/blogs/view/balerina-liliya-kak-devochka-stala-tantsuyushhim-lebedem}}

\ii{27_08_2019.stz.news.ua.mrpl_city.1.kateryna_karaben_fortun.pic.4}

На думку маріупольчанки, 

\begin{quote}
\em\enquote{творчість – це осяяння, тому планів на майбутнє
немає. Приходить осяяння – Всесвіт дає можливість реалізації – і є результат. А
запланувати – створити – це більш ремісничий або робочий підхід}.
\end{quote}

Сім'я та друзі підтримують творчу і енергійну Катерину. Надихає Всесвіт:
природа, сни, люди та події. У Маріуполі найбільше любить гуляти у Приморському
парку та Міському саду.

Для жінки дуже важливий сам процес зростання над собою вчорашньою. Вона
намагається писати сьогодні краще, ніж учора і малювати більш цікаво та
професійно, ніж у минулому. Її власний Еверест – це власне самовдосконалення,
тому це природно, що він зростає разом з нею кожного дня.

\ii{27_08_2019.stz.news.ua.mrpl_city.1.kateryna_karaben_fortun.pic.5}

Все життя Катерини пронизане любов'ю до міфологій народів світу. Починалось з
любові до казок, а далі перейшло у сильне захоплення історичними фактами,
легендами та подіями, що відбувались на нашій планеті протягом її існування. Це
відображається і в її літературних творах, і в картинах. Вона просто живе
ними... Також вона захоплюється кармічною психологією, цікавиться філософськими
течіями різних релігій світу. Її дуже цікавлять культури стародавніх кельтів,
слов'ян, народів Індії та Тибету, Японії та Китаю, корінних народів Африки та
Америки, також Скандинавії, Ісландії, народів Півночі... У кожному з них вона
знаходить те, що імпонує їй найбільше, те, від чого серце розкриває крила і
народжує неповторні твори.

\textbf{Улюблена книга:} 

\begin{quote}
\em\enquote{Дуже багато улюблених письменників, тим паче книг. Так би
мовити топ-5 улюблених письменників та їх творів: Габріель Гарсія Маркес \enquote{Очі
блакитного собаки} та \enquote{Сто років самотності}, Рей Бредбері \enquote{Мар\hyp{}сіанські хроніки}
та \enquote{Вино з кульбабок}, Айзек Азімов цикл романів \enquote{Заснування} та оповідання з
циклу \enquote{Я, Робот}, твори Юрія Андруховича, особливо ті, що були ним використані
у виставі \enquote{Альберт, або Найвища форма страти} та середньовічний роман \enquote{Пісня
про Нібелунгів: життєпис Рогового Зиґфріда}}.
\end{quote}

\ii{27_08_2019.stz.news.ua.mrpl_city.1.kateryna_karaben_fortun.pic.6}

\textbf{Улюблений фільм:} \enquote{Хмарний Атлас}, \enquote{Той, що біжить по лезу}, \enquote{Тринадцятий
поверх} та \enquote{Будинок літаючих кинджалів}.

\textbf{Хобі:} Колекціонує статуетки міфологічних істот, богів, різні цікаві камені та
мінерали. А ще – приємні враження.

\textbf{Курйозний випадок з життя:} 

\begin{quote}
\em\enquote{Усе життя моє насичене цікавими подіями, тому
пригадати щось конкретне важко. Наприклад, ось з недавнього... Сиджу я у парку
та граю на дримбі (такий музичний інструмент на кшталт варгану). Мої діти
звичні до того, тож граються, незважаючи на нього. А інші діти підійшли,
питають: \enquote{тьотю, то ви зливу викликаєте?} Найкумедніше, що того вечора дощ
таки пішов, бо ж і за прогнозом його обіцяли. Тож і таке буває...}.
\end{quote}

\ii{27_08_2019.stz.news.ua.mrpl_city.1.kateryna_karaben_fortun.pic.7}

\textbf{Порада маріупольцям:} 

\begin{quote}
\em\enquote{Знаєте, мені здається, що якщо всі ми і кожен з нас буде
робити свою справу старанно та на 100 відсотків, то і рівень життя покращиться.
Головне – не чекати, не слідкувати, де ж і хто почне старатися, а робити це
самому... Щоб змінити світ навколо на краще, треба спочатку змінитися зсередини
та привнести Світло до свого серця, життя, навколишнього середовища. І чим
більше буде таких людей, тим чистіше буде навколо, тим краще ми будемо жити}.
\end{quote}

\textbf{Читайте також:} \emph{\enquote{Цветы завянут, а ребенок будет жить}: мариупольцам предлагают отказаться от цветов на 1 сентября}%
\footnote{\enquote{Цветы завянут, а ребенок будет жить}: мариупольцам предлагают отказаться от цветов на 1 сентября, Богдан Коваленко, mrpl.city, 24.08.2019, \par\url{https://mrpl.city/news/view/tsvety-zavyanut-a-rebenok-budet-zhit-mariupoltsam-predlagayut-otkazatsya-ot-tsvetov-na-1-sentyabrya}}
