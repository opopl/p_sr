% vim: keymap=russian-jcukenwin
%%beginhead 
 
%%file 11_08_2021.fb.vasiljev_vasilij.dnepropetrovsk.1.zhittja_v_sssr
%%parent 11_08_2021
 
%%url https://www.facebook.com/vasiliy.vasilyev.96/posts/235204311940317
 
%%author_id vasiljev_vasilij.dnepropetrovsk
%%date 
 
%%tags babushka,semja,sssr,zhizn
%%title ТРОХИ ПРО ЖИТТЯ ПРИ СРСР
 
%%endhead 
 
\subsection{ТРОХИ ПРО ЖИТТЯ ПРИ СРСР}
\label{sec:11_08_2021.fb.vasiljev_vasilij.dnepropetrovsk.1.zhittja_v_sssr}
 
\Purl{https://www.facebook.com/vasiliy.vasilyev.96/posts/235204311940317}
\ifcmt
 author_begin
   author_id vasiljev_vasilij.dnepropetrovsk
 author_end
\fi

ТРОХИ ПРО ЖИТТЯ ПРИ СРСР

Ніхто чомусь не розповідав про державну позику в післявоєнні роки аж до
\enquote{відлиги} Микити Сергійовича Хрущова.  Розповім ЩО це таке.  Звичайно ж, зі
слів моєї бабушкі. 

Жілі ми тоді на нинішній Черкащині.  В колгоспі ім.  Чкалова
гроші не платили, а восени на трудодні давали кілька пудів жита, пшениці,
гречки, проса, цукрових і кормових буряків, а також півсотні кавунів, кілька
кошиків "заводський" (сорт такий був, він втрачений) картоплі, моркви, цибулі.
Натуроплату отримувала на свої важко і наша корова Галка.Кстаті, трудоднів у
корови було більше, ніж у бабусі Ганни.  Не знаю чому.  Мабуть коровам писали
більше трудоднів.  І це справедливо.  Корова має кінську силу, яку
використовували на всіх роботах, які сьогодні виконує техніка.  

Наша годувальниця працювала в парі з такою ж сильною коровою тітки Ярини.  Я
питав, а коли ж корова паслася?  Виявляється вночі.  Всю ніч корови гризли
соковиту траву, потім пару годин сну і знову на роботу, звозити снопи на полях,
перевозити молотарки і великий двигун з поля на поле.  Двигуна тягнули більше 5
пар корів.  Без корів не можна було скласти солому в стіг, перевезти зерно з
току в комори. Вот така була наша Галка. Она була в партизанах, а при
форсуванні притоки Дніпра вона почала тонути разом з возом, але зуміла
вирватися з ярма і, задерши хвіст, побігла  прямо на німецький дзот.  Німці
явно пошкодували сміливе тварина і вже після війни бабуся знайшла корову в
іншому селі.  Господарі виявилися добрими людьми і повернули корову бабусі за
два пуди ячменю.  

Але я відволікся на корову не випадково.  Вона брала участь у важливому
державному деле- возила ісолнітеля райради, який агітував працюють в полі вдів
купувати облігації державної позики.  Кожна вдова повинна була купити від 400
до 800 рублів облігацій.  Жінки плакали.  Звідки у них можуть бути гроші.  Але
виконавець стоячи на возі голосно розповідав як необхідна для країни
індустріалізація і як потім буде добре жити: не буде грошей взагалі, а люди
зможуть йти в будь-який магазин і брати все що їм потрібно.  Жінки готові були
повірити і, напевно, вірили, але не знали, де брати гроші на облігації.  Тих,
хто відмовлялися, садили на віз і возили по полях.  

Виконавець зупиняв віз на черговому поле, збирав людей і говорив: подивіться
ось на цих жінок, вони проти індустріалізації СРСР, вони не хочуть, щоб у нас
були безкоштовні магазини і наша країна разгороміла буржуїв у всьому міре.
Женщіни плакали від сорому, але грошей адже  НЕ намалюєш.  Везіть цей
несвідомий елемент в сільраду наказував виконавець і наші корівки повезли
заарештованих жінок в сельсовет. 

Там їх закрили в спеціальний кабінет на всю ніч до ранку.  А вдома ж діти, які
не годовані корови, свині, кури, старенькі родітелі. Закрилі бабусину невістку
Марфу, сусідку Дуньку і ще багато жінок і декількох мужиків, які повернулися з
війни калекамі. Утром випустили.  Потім таке покарання застосовували до інших
ворогам індустріалізації СРСР.  

Бабуся розповідає, що вона викуповувала по 400 рублів пустушок.  Гроші вона
заробляла на самогонку.  Вночі, щоб не бачили сусіди, вона гнала сивуху і
продавала по рублю за пляшку.  Влітку торгувала яблуками, абрикосами, вішнямі.
Могла продати пару курок в Жаботине або Медведівці.  Теж не ближній свет. А по
шляху на роботу вона несла молоко здавати за безкоштовно.  Тисячу літрів треба
було відірвати від сім'ї і віддати государству. Яіц теж треба було здати 500
штук.  Таким чином вистачало на пустушки облігації, щоб не сидіти ночами в
сільраді в кутугузке з гратами на вікнах.  

Вечорами вдови збиралися у баби Вусті і говорили за жізнь. Все
сходилися на думці, що при німцях було легше.  Німці платили гроші на трудодні
і давали більше зерна і овочів на сім'ю.  

Правда при відступі фашисти спалили село до тла, вчителів і комсомольців
загнали в церьковь Казанської Божої Матері і підпалили.  Люди згоріли.  

Однак селяни звинувачували в цьому партизан, які дражнили німців своїми
невдалими наскоками з Холодноярського леса. Но все-таки партизани потім
поселили в глибокому лісі бездомних селян.  Дідуся Данилу німці повісили на
вишні у дворі, а бабуся з дітьми встигла сховатися в лісі.  

З палаючого сараю вдалося вивести корову.  Так вийшло, що вона стала героїнею
моєї розповіді.  Бабуся розлучилася з нею аж за Хрущова.  Микита Сергійович
дозволив здавати молоко державі вершковим маслом.  Не пам'ятаю скільки кг треба
було здати за планом на один двір.  А облігації держава так і не погасило.  І
ніхто не міг згадати куди поділися ці пожовклі папірці.  Їх набралося повну
кошёлку.  Зате індустріалізація удалась. І служить до сих пір.  Чи не відомо
тільки кому.

Створено на підставі спогадів бабусі нашого друга. Іван Іван.

\ii{11_08_2021.fb.vasiljev_vasilij.dnepropetrovsk.1.zhittja_v_sssr.cmt}

