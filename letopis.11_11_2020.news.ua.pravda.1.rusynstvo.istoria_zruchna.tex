% vim: keymap=russian-jcukenwin
%%beginhead 
 
%%file 11_11_2020.news.ua.pravda.1.rusynstvo.istoria_zruchna
%%parent 11_11_2020.news.ua.pravda.1.rusynstvo
 
%%url 
%%author 
%%tags 
%%title 
 
%%endhead 

\subsubsection{Зручної історії не буває}
\label{sec:11_11_2020.news.ua.pravda.1.rusynstvo.istoria_zruchna}

Є на Закарпатті дотепний русинський жарт "Людина людині --- вовк, а руснак
руснаку --- українець".

Я згадую про цей жарт, коли бачу, як поліціянти, що поміж собою спілкуються
"закарпатською", під час сварок з місцевими власниками кав’ярень та ресторанів,
що відкриті попри заборону --- "червона зона", як-ніяк --- одразу переходять на
державну мову.

Шукати заклад, який працює підпільно, ані у мене, ані у Володимира Фенича немає
бажання, тому наша друга зустріч відбувається у затишному сквері. Я розповідаю
йому про нашу розмову з Владиславом. Фенич не погоджується з висновками
Товтина.  

– Так званий "іншій козир" Росії дають не русини, а українські
"декомунізатори", --- каже пан Володимир. --- Саме вони здатні з часом
"декомунізувати" договір по Криму 1954 року або піддати сумніву
радянсько-словацький договір 1945 року про входження Закарпаття до складу СРСР.

Адже декомунізація --- це не просто повалити пам’ятники Леніну і змінити назви
вулиць. Якщо до влади в Україні прийдуть люди, яким декомунізація таких
договорів буде на руку, то вони будуть мати всі підстави для цього.

Щодо русинського питання Фенич взагалі не бачить офіційної позиції України. У
сусідніх країнах це питання було вирішено у демократичний спосіб, а Україна
лише декларує наміри бути європейською країною. Тим паче, що "Русь" та
"руськість" --- це набагато ширше явище, ніж винятково етнографічний сегмент.

– Так, русини не вважають свою рідну мову діалектом української, --- каже Фенич.
– Але зараз ми з вами спілкуємося державною мовою, чи не так? І державна мова у
нас всіх одна --- і в українців, і у підкарпатських русинів. І територіальні
кордони нашої держави є непорушними і суверенними для всіх.

Але чому в мене, громадянина України, у якого мати угорка, а тато русин,
ставлення до своєї держави цілком патріотичне, а ось ставлення держави до мене
– шовіністично-дискримінаційне?

Це імперський підхід. Коли Путін каже, що росіяни та українці --- єдиний народ, а
українську націю та мову придумав австрійський генштаб, мене це обурює і як
громадянина України і як професійного історика. Але чому Україна так само
ставиться до русинів и русинської мови?

\ifcmt
img_begin 
        url https://img.pravda.com/images/doc/f/c/fcc6797-dscf1192.jpg
        caption Володимир Фенич: "Чому русини з ностальгією згадують часи, коли Закарпаття увійшло до складу Чехословаччини? Тому що Чехословаччина --- це демократичне дитя Версаля. Її демократія дозволяла спокійно співіснувати і москвофілам, і українофілам, і угрофілам. Аби тільки не порушували законів".
        width 0.7
img_end
\fi

Точку зору Володимира Фенича я розумію. Але не розумію, чому на мітингах
закарпатські русини й досі використовують триколор, що дуже нагадує російський
прапор.

Коли йдеться про автономію, русини люблять згадувати, що в них вона вже колись
була --- Підкарпатська Русь. Але прапором Підкарпатської Русі було жовто-блакитне
полотнище.

Чому коли йдеться про автономію, Підкарпатська Русь --- це зразок, а коли йдеться
про прапор, то Підкарпатська Русь вже ніякий не зразок? Адже саме через той
триколор постійно виникають непорозуміння.  

Коли в Україні вже шість років йде війна з Росією на Донбасі, коли Крим був
анексований саме під триколором, хіба доцільна впертість щодо традицій, які
були започатковані при Чехословаччині москвофільськими товариствами Фенцика?

– Чесно кажучи, мені це самому незрозуміло і неприйнятно, --- розводить руками
Фенич. --- Так, на прапорі Підкарпатської Русі були жовтий і блакитний кольори,
тому що це кольори династії Габсбургів. Саме тому русинськими природними
кольорами є ті ж самі кольори, які присутні на державному прапорі України.

Але… Ви розумієте, коли Фенцик у 1923 році почав використовувати триколор, він
мав на увазі не більшовицьку Росію, а ту Росію, що боролася з більшовиками,
тобто, була жертвою. Тоді на той триколор взагалі ніхто не зважав, тому що
партія Фенцика була лише однією з 30 політичних партій, які брали участь у
виборах до чехословацького парламенту. Але зараз… так, я згоден з вами --- це
неправильно.

\ifcmt
img_begin 
        url https://img.pravda.com/images/doc/3/1/3109269-dscf1097.jpg
        caption Герб Підкарпатської Русі на стіні Будинку національних спільнот в Ужгороді. Закарпаття отримало власний герб після включення його до складу Чехословаччини в 1919 році. Саме цей герб закарпатські (або підкарпатські) русини позиціонують як символ всіх русинів у світі
        width 0.7
img_end
\fi

Ми з Володимиром Феничем заходимо до внутрішнього двору Ужгородського
художнього музею, що розташований у приміщеннях колишнього будинку колишнього
Ужанського комітату --- адміністративно-територіальної одиниці в Угорському
королівстві.

– Розумієте, --- каже Фенич, --- тутешнє москвофільство --- невід’ємна історія
закарпатського краю. Ми не повинні переписувати історію --- ми повинні її
переосмислювати. Нам треба вчитись жити з незручною історією, бо "зручної"
історії немає у жодної нації.

От скажіть мені, хто є більшим соборником? Августин Волошин, який у березні
1939 року проголосив незалежність Карпатської України, тобто реалізував
сепаратистський проект як громадянин Чехословаччини? Чи комуніст Іван Туряниця
(теж чехословацький громадянин), який у листопаді 1944 проголосив про
возз’єднання Закарпаття з Україною?

Для мене як історика неприпустимо "канонізувати" Волошина та Карпатську Україну
і забути назавжди про Туряницю і про Закарпатську Україну. Тому що усе це
історія. А історія завжди має в собі велику кількість незручностей для
ідеології.

Коли у 2002 році Августина Волошина посмертно нагородили званням Героя України,
на Закарпатті серед русинів виникла напруга, тому що тут ще пам’ятають і про
концтабір Думен поблизу Рахова, і про встановлення у краї авторитарного
волошинського режиму.

Але що можна зробити, щоб ідеологічні концепти не йшли у розріз з історичними
фактами?

– "Pacto del olvido --- договір про забуття. А також заборона державним
чиновникам всіх рівнів коментувати історичні події. Тому що стаючи на один бік
історичних подій чи діячів, вони від імені держави скривджують пам'ять іншої
частини суспільства, предки яких не знали, що їхні нащадки колись житимуть в
Україні, --- вважає Володимир Фенич.

А ще він впевнений, що Україні навіть вигідно з ідеологічної точки зору визнати
русинів за окремий етнос, тому що русинська мова --- це коріння, з якого потім
проросли українська, російська та білоруська мови. Тобто, русинська мова як
релікт збереглася крізь час, попри усі монархічні, імперські та тоталітарні
режими. Українцям лише треба переступити через власний етноегоїзм, породжений
вигаданою національною титульністю, винятковістю та обраністю.

Можливо, саме на цьому грунті й треба шукати компроміс?

Держава може сказати русинам --- так, хлопці, ми не визнаємо вас окремим етносом.
Але первородство у царині української мови та деяких інших слов’янських мов
належить саме вам. Ви --- релікт нашої давньої руськості, наш зв'язок з предками.
Закарпаття може стати колискою слов’янських мов. Хіба це не вигідно Україні з
точки зору ідеології?

Але відповідь на це питання можна отримати лише у Києві.

