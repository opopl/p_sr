% vim: keymap=russian-jcukenwin
%%beginhead 
 
%%file 29_07_2021.fb.glazunov_dmitrij.1.znakomitsja_blizhe
%%parent 29_07_2021
 
%%url https://www.facebook.com/HlaDim/posts/4263363560397961
 
%%author_id glazunov_dmitrij
%%date 
 
%%tags chelovek,puteshestvie,ukraina
%%title Давайте знайомитись ближче?
 
%%endhead 
 
\subsection{Давайте знайомитись ближче?}
\label{sec:29_07_2021.fb.glazunov_dmitrij.1.znakomitsja_blizhe}
 
\Purl{https://www.facebook.com/HlaDim/posts/4263363560397961}
\ifcmt
 author_begin
   author_id glazunov_dmitrij
 author_end
\fi

Давайте знайомитись ближче?

З усіма цим пандеміями, карантинами, й обмеженою кількістю соціальних зв’язків
я зрозумів що наскільки б я не був стовідсотковим інтровертом, а мені не
вистачає комунікації, особливого живого спілкування.

А ще я зрозумів, що я дуже мало знаю про людей, які є в мене друзях та
підписниках.

Якось не дуже прийнято у нашому суспільстві розповідати про себе та свої
досягнення.

Давайте це виправимо, я почну!

Спробую розказати про себе мовою фактів, щоб не бути занадто пафосним. Якщо вам
щось цікаво, то питайте й діліться своєю історією.

Мені 26 і я маю багато веселих історій з мандрів у понад 45 країнах та досвід
життя у форматі 200 євро на півтора року та подорожування при цьому.

\ifcmt
  tab_begin cols=4

     pic https://scontent-frx5-1.xx.fbcdn.net/v/t1.6435-9/225962317_4263312233736427_3651510903084298599_n.jpg?_nc_cat=111&ccb=1-5&_nc_sid=730e14&_nc_ohc=BJVpgtqOF1EAX8_Gxw0&_nc_ht=scontent-frx5-1.xx&oh=aaa0e57f5f550d52a7208a81477c9ed9&oe=617F1F9C

     pic https://scontent-frt3-1.xx.fbcdn.net/v/t1.6435-9/225759946_4263312550403062_7562686925926959513_n.jpg?_nc_cat=108&ccb=1-5&_nc_sid=730e14&_nc_ohc=bpmMaRqwQwgAX_1WO2C&_nc_ht=scontent-frt3-1.xx&oh=b0f338207f031a59c4ec31a801d9f16f&oe=617EC609

		 pic https://scontent-frx5-1.xx.fbcdn.net/v/t1.6435-9/226665010_4263313663736284_1955052854411518456_n.jpg?_nc_cat=111&ccb=1-5&_nc_sid=730e14&_nc_ohc=zZxKItCGoOIAX9Y0DxO&_nc_ht=scontent-frx5-1.xx&oh=f7db8c45b720791618e455343dd52249&oe=61826D03

		 pic https://scontent-frx5-1.xx.fbcdn.net/v/t1.6435-9/225654951_4263360130398304_8703268014733590325_n.jpg?_nc_cat=100&ccb=1-5&_nc_sid=730e14&_nc_ohc=2G2ZFwF-2SMAX99NjQs&_nc_ht=scontent-frx5-1.xx&oh=561df9892e4ca13738939516e8f56eaa&oe=6181B727

  tab_end
\fi

Я залежний від мандрів і подорожую як старі мандрівники — вивчаю світ, ходжу по
15-35 км в день, катаюсь автостопом, спілкуюсь з місцевими, але й не відмовляю
собі в інших форматах.

Накрутив 200 тисяч км автостопом у 2017 й перестав рахувати кілометри.

Через свій образ життя в подорожах я страшенний мінімаліст і всі необхідні речі
для життя разом зі спорядженням поміщаються в мій 34 літровий рюкзак.

Мене можна закинути в будь-яку точку на планеті без грошей й телефону і я
виберусь звідти з крутою історією та ніштяками.

Коли я вперше поїхав за кордон, то мав з собою в кишені 30 гривень, а
повернувся зі 100 баксами та повним рюкзаком вина та сувенірів з Грузії та
Вірменії.

До війни жив в Єнакієво. 

Катав як ультрас донецького Шахтаря. У 17 років вдарився в націоналізм.
Перейшов на українську. А у 2013 очолив новостворений мною єнакієвський
осередок «Свободи». Був помічником народного депутата України 7-скликання,
можливо наймолодшим в історії України. Одного разу російські прикордонники й
фсб-шники цілу добу думали, що я український шпигун. За це тепер маю вибитий
хребець і милу серцю багаторічну заборону на в’їзд у РФ.

У 2015 подався на біженця на Кіпрі. Приїхав туди зі 120 євро в кишені та одним
рюкзаком. Згодом Перетягнув туди батьків і сестру. Через два роки в мене була
мережа хостелів на Кіпрі, тому можу розказати як побудувати і як керувати
мережею хостелів. Зараз, на щастя, цим більше не займаюсь.

Маю 4 річний досвід торгівлі цінними паперами й можу пояснити базиси
довгострокового інвестування в акції, індексні фонди. Організував вихід
українського примірника бізнес книги slicingpie
https://mandruvaka.gumroad.com/l/AeLsM

Поширюю ідею динамічного поділу капіталу в бізнесі.

Топлю за легалізацію усього!

 Бо тільки легалізовані речі в правому полі можна по справжньому контролювати
та регулювати в залежності від потреби суспільства.

Не їм м’яса більше ніж 8 років.

Люблю досліджувати та експериментувати з організмом та свідомістю.

Передивився понад 4 тисячі фільмів й завдяки кіно почав формувати свій
адекватний світогляд, тому зараз вчусь на кіно та телепродюсера в
Карпенка-Карого, а не гнию в окопах ДНР.

Маючи можливості звалити в будь-яку точку світу, щоб почати нове життя, навіть
після відвідин майже 50 країн все одно повернувся в Україну. Важко пояснити
чому. Тут класно. Не зважаючи на весь політичний хаос та наше суспільство, яке
схоже на 3 річного малюка, який бігає набиває гулі об стіл й винить в тому
стіл.

Мені тут комфортно, я хочу тут жити й мені не байдуже на розвиток людства та
України, тому часто бурчу що щось не так і маю свій альтернативний погляд на
багато проблем.

Я повернувся в Київ ще перед ковідом й всі плани, як й в багатьох інших пішли
собаці під хвіст. Хочеться нових планів й рухатись далі.

Маю мрію створити десь в радіусі 35 км від Києва центр сучасної української
культури у форматі природного заповідника, майстерень для творчої молоді та
винахідників, щоб там можна було приймати гостів з усього світу, знайомити їх з
Україною, проводити творчі заходи та акумулювати й знайомити талановитих людей
в цьому просторі.

Мені хочеться більш персоналізованого спілкування та цікаво дізнатись
детальніше про вас. Хто ви? Розкажіть про себе. Який мали досвід.Що вас
надихає?

Та розпитуйте мене, якщо вас щось зацікавило.

А якщо хтось хоче зустрітись, то ми з моєю собакою Кускус будем радо бачити вас
на прогулянці в мар’їнському парку або можете кликати на свої пропозиції.
