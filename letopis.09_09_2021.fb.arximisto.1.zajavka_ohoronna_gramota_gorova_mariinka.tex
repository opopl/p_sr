%%beginhead 
 
%%file 09_09_2021.fb.arximisto.1.zajavka_ohoronna_gramota_gorova_mariinka
%%parent 09_09_2021
 
%%url https://www.facebook.com/arximisto/posts/pfbid02fcUcgXB2g4Lhpt44P5P2NWAig7HKL6s7crXWHLCPQz75XWQCWXDMNuCGo9Yhf9S7l
 
%%author_id arximisto
%%date 09_09_2021
 
%%tags 
%%title Заявка на "охоронну грамоту" для Першої гімназії Маріуполя – готова для подання!
 
%%endhead 

\subsection{Заявка на \enquote{охоронну грамоту} для Першої гімназії Маріуполя – готова для подання!}
\label{sec:09_09_2021.fb.arximisto.1.zajavka_ohoronna_gramota_gorova_mariinka}

\Purl{https://www.facebook.com/arximisto/posts/pfbid02fcUcgXB2g4Lhpt44P5P2NWAig7HKL6s7crXWHLCPQz75XWQCWXDMNuCGo9Yhf9S7l}
\ifcmt
 author_begin
   author_id arximisto
 author_end
\fi

РУС \textbackslash~ УКР Заявка на \enquote{охранную грамоту} для Первой гимназии Мариуполя – готова к подаче!

\#новости\_архи\_города

Одесские архитекторы и историки предоставили Первой гимназии и общественным
активистам полный комплект учетной документации, которая необходима для
присвоения зданию бывшей Мариинской женской гимназии статуса памятника
архитектуры и истории местного значения. 

Здание гимназии имеет \enquote{значительную} историческую и \enquote{редкую}
архитектурную ценность, как считают авторы документации Анатолий Изотов,
кандидат архитектуры, и Андрей Красножон, доктор исторических наук.

Историческая ценность заключается в том, что Мариинская женская гимназия тесно
связана с жизнью и деятельностью Феоктиста Хартахая, выдающегося мариупольца,
историка, этнографа, основателя первых средних учебных заведений в городе,
включая Мариинскую гимназию.

Здание можно отнести к стилистике неоклассики, по мнению авторов документации.
Оно является одним из архитектурных акцентов и составной частью традиционной
исторической среды центра города. В здании в значительной степени сохранилась
старинная система воздушного отопления, включая помещение и печные каналы
подогрева.

Несмотря на исторические перипетии XX века здание отвечает критерию
аутентичности, поскольку в нем сохранилось архитектурно-художественное решение
фасадов и интерьеров.

Авторы документации предлагают сделать здание одновременно памятником истории и
архитектуры местного значения (подробную характеристику см. в учетной
документации по ссылке внизу).

25 ноября этого года наша гимназия будет отмечать свое 145-летие! Благодаря
выполненной работе мы впервые получили независимую профессиональную оценку
исторической и архитектурной значимости ее здания, а также его технического
состояния, как подчеркнула Ирина Олейниченко, директор Первой гимназии. Мы
благодарим инициаторов и благотворителей проекта и с нетерпением ждем
завершения процедуры предоставления статуса памятника архитектуры и истории.

Разработка документации стала настоящим исследованием. Оно впервые выявило
сложную строительную историю Мариинской гимназии с 1894 по 1951 год, как
считает Анна Дежец, директор музея Первой гимназии. Мы признательны Наталье
Капустниковой, директору Мариупольского краеведческого музея, и Виктории
Рязановой, руководителю архивного отдела Мариупольского горсовета за помощь,
оказанную разработчикам. 

Следующий шаг – это подача учетной документации в управление культуры и туризма
ДонОГА от имени общественной организации \enquote{Архи-Город}, как сообщил Андрей
Марусов, директор организации. При положительном рассмотрении управление
предоставит зданию статус \enquote{объекта культурного наследия}. Затем весь пакет
документов отправится в министерство культуры и информационной политики Украины
– для получения статуса памятника архитектуры и истории.

Присвоение зданию такого статуса по инициативе граждан станет беспрецедентным
событием не только для Мариуполя, но и для всей Восточной Украины, как
подчеркнул А. Марусов. Ибо, как правило, инициаторами выступают органы власти.

Необходимые 16 780 грн. были пожертвованы двумя общественными организациями и
одиннадцатью мариупольцами. На сегодня на оплату услуг разработчика ООО \enquote{НИПЦ
архитектурного и градостроительного наследия} было потрачено 16 166 грн.,
банковское обслуживание обошлось в 198 грн. Остаток в 416 грн. сохраняется на
счетах \enquote{Архи-Города} для возможных расходов по дальнейшему сопровождению
процедуры (поездка на заседание ОГА в Краматорск, внесение изменений и т.п.). 

Мы чрезвычайно признательны всем авторам подарка Первой гимназии к ее юбилею: 

общественной организации \enquote{Національна вимога} (директор - Алина Юнге, Adelheid
Unge-Zheludkova) и Мариупольскому обществу греков (глава общества – Надежда
Чапни, Nadia Chapni), инициатору проекта Ярославу Федоровскому, Славе
Ярославскому (Slava Yaroslavskiy), Данцкеру Борису Давидовичу (выпускной 10А
класс 1971 года), Яблуновской Анжеле Леонидовне, Данилову Сергею Ивановичу,
Лиманскому Сергею Александровичу, Рыжих Елене Николаевне (Олена Рижих),
Побегайло Елене Влаленовне, Кучеренко Анне Андреевне, Фатеевой Ирине
Владимировне (Ирина Фатеева) и Удовенко Юлии Руслановне. 

Приглашаем всех желающих ознакомиться с учетной документацией по следующим
ссылкам: \url{https://cutt.ly/7WGT1BF} (pdf) или \url{https://cutt.ly/CWGTHUT} (Word,
Winrar).

Напомним, что общественная кампания по сбору средств для разработки учетной
документации началась в апреле и успешно завершилась в июне 2021 г. В рамках
контракта между ГО \enquote{Архи-Город} и ООО \enquote{НИПЦ архитектурного и градостроительного
наследия}, Анатолий Изотов, архитектор и директор ООО, 23 июня провел
обследование здания Первой гимназии. Одновременно историки и краеведы занялись
сбором и анализом материалов в архивах и музеях Мариуполя... 

Мы будем рады предоставить любую дополнительную информацию: 096 463 69 88,
arximisto@gmail.com (А.Марусов), 098 8229455 (Анна Дежец).

\#перша145річниця

\#мариинке\_охранная\_грамота

=======================================

Заявка на \enquote{охоронну грамоту} для Першої гімназії Маріуполя – готова для подання! 

Одеські архітектори та історики надали Першій гімназії та громадським
активістам повний комплект облікової документації, яка є необхідною для
отримання колишньою Маріїнською жіночою гімназією статусу пам'ятки архітектури
та історії місцевого значення. 

Будівля гімназії має \enquote{значну} історичну та \enquote{рідкісну} архітектурну цінність, як
переконані розробники документації Анатолій Ізотов, кандидат архітектури, та
Андрій Красножон, доктор історичних наук. 

Історична цінність полягає в тому, що Маріїнська жіноча гімназія тісно
пов'язана з життям та діяльністю Феоктиста Хартахая, видатного маріупольця,
історика, етнографа, засовника перших середніх учбових закладів в місті,
включаючи Маріїнську гімназію.

Будівлю можна віднести до стилістики неокласики, на думку розробників
документації. Вона є одним із архітектурних акцентів та складовою частиною
традиційного історичного середовища центру міста. Це рідкісний зразок
архітектури, будівельно сформований протягом тривалого часу не за єдиним
задумом. Причому, в будівлі значною мірою зберіглася система повітряного
опалення, в т.ч. приміщення та канали печі підігріву.

Попри драматичні перипетії XX ст. будівля відповідає критерію автентичності,
оскільки в ньому зберіглося архітектурно-художнє вирішення фасадів та
інтерєрів.

Автори документації пропонують зробити будівлю одночасно пам'яткою історії та
архітектури місцевого значення (детальну характеристику див. за посиланнями
нижче).

25 листопада цього року наша гімназія буде святкувати свою першу 145-ту
річницю! Завдяки виконаній роботі ми вперше отримали незалежну професійну
оцінку історичної та архітектурної значущості її будівлі, а також стану її
збереження, як підкреслила Ірина Олейниченко, директорка Першої гімназії. Ми
дякуємо ініціаторам та благодійникам проекту та з нетерпінням чекаємо на
завершення процедури надання статусу памятки архітектури та історії.

Розробка документації стала справжнім дослідженням. Воно вперше виявило складну
будівельну історію Маріїнської гімназії з 1894 до 1951 року, як вважає Ганна
Дежец, директорка музею Першої гімназії. Ми вдячні Наталії Капустніковій,
директорці Маріупольського краєзнавчого музею, та Вікторії Рязановій, керівнику
архівного відділу Маріупольської міської ради, за допомогу, надану розробникам.

Наступний крок – це подання облікової документації в управління культури та
туризму ДонОДА від імені громадської організації \enquote{Архі-Місто}, як повідомив
Андрій Марусов, директор організації. У разі схвалення управління прийме
рішення про надання будівлі статусу об'єкту культурної спадщини. Після цього
весь пакет документів буде відправлений у міністерство культури та
інформаційної політики України – для отримання статусу пам'ятки архітектури та
історії. 

Надання будівлі такого статусу за ініціативою громадян буде безпрецедентною
подією не лише для Маріуполя, а для всієї Східної України, як підкреслив
А. Марусов. Бо зазвичай ініціаторами є органи влади...

Дві громадські організації та одинадцять маріупольців загалом пожертвували 16
780 грн. На сьогодні на оплату послуг ТОВ \enquote{НДПЦ архітектурного та
містобудівної спадщини} було витрачено 16 166 грн., банківське обслуговування –
198 грн.  Решта у 416 грн. зберігається на рахунку \enquote{Архі-Міста} для
можливих витрат на подальше супроводження процедури (поїздка на засідання
управління ОДА в Краматорськ, внесення змін тощо). 

Ми надзвичайно вдячні всім авторам подарунку Першій гімназії на її ювілей: 

ГО \enquote{Національна вимога} (директорка – Аліна Юнге) та Маріупольському товариству
греків (голова товариства – Надія Чапні), ініціатору проекту Ярославу
Федоровському, Славі Ярославському, Данцкеру Борису Давидовичу (випускний 10А
клас 1971 р.), Яблуновській Анжелі Леонідівні, Данилову Сергію Івановичу,
Ліманському Сергію Олександровичу, Рижих Олені Миколаївні, Побегайло Олені
Влаленовні, Кучеренко Ганні Андріївні, Фатеєвій Ірині Володимирівні та Удовенко
Юлії Русланівні.

Запрошуємо усіх бажаючих ознайомитися з обліковою документацією за наступними
посиланнями: \url{https://cutt.ly/7WGT1BF} (pdf) або \url{https://cutt.ly/cwgthut} (Word,
Winrar).

Надаємо, що громадська кампанія зі збору коштів для розробки облікової
документації розпочалася у квітні та успішно завершилася у червні 2021 р. В
рамках контракту між ГО \enquote{Архі-Місто} та ТОВ \enquote{НДПЦ архітектурної та
містобудівної спадщини}, директор товариства та архітектор Анатолій Ізотов
здійснив обстеження будівлі гімназії наприкінці червня. Одночасно історики та
краєзнавці провели збір та аналіз матеріалів в архівах та музеях Маріуполя...

Ми будемо раді надати будь-яку додаткову інформацію: 096 463 69 88,
arximisto@gmail.com (А.Марусов), 098 8229455 (Ганна Дежец).
