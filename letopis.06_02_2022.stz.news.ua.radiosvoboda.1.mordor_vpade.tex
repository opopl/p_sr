% vim: keymap=russian-jcukenwin
%%beginhead 
 
%%file 06_02_2022.stz.news.ua.radiosvoboda.1.mordor_vpade
%%parent 06_02_2022
 
%%url https://www.radiosvoboda.org/a/nikita-titov-oboronni-symovly-mist/31683129.html
 
%%author_id shtogrіn_іrina
%%date 
 
%%tags hudozhnik,isskustvo,kartina,mordor,nenavist,rossia,ukraina
%%title Художник Нікіта Тітов: «Мордор впаде і Україна задихає вільно»
 
%%endhead 
 
\subsection{Художник Нікіта Тітов: «Мордор впаде і Україна задихає вільно»}
\label{sec:06_02_2022.stz.news.ua.radiosvoboda.1.mordor_vpade}
 
\Purl{https://www.radiosvoboda.org/a/nikita-titov-oboronni-symovly-mist/31683129.html}
\ifcmt
 author_begin
   author_id shtogrіn_іrina
 author_end
\fi

\begin{zznagolos}
Користувачі українського Facebook добре знають роботи художника Нікіти Тітова.
Створені ним образи є візуалізацією тонких матерій, великих трагедій, складних
процесів і водночас простоти та радості буття. Те, про що пишуть романи і
знімають фільми, Тітов уміє вмістити в одному зображенні. Зараз художник
творить «оборонні символи міст» на знак того, що «Україна готова дати відсіч
Росії».

Нікіта Тітов народився в Естонії, закінчив художньо-графічний факультет
Харківського педагогічного університету і довгий час жив у Харкові. Займається
живописом, графікою, книжковою ілюстрацією та дизайном.
\end{zznagolos}

\ii{06_02_2022.stz.news.ua.radiosvoboda.1.mordor_vpade.pic.1}

– Ви художник-філософ, бо Ваші картини-малюнки-плакати це завжди переказана
історія, погляд у саму сутність явищ і одночасно запитання, на яке людина сама
має знайти відповідь. Як Вам це вдається?

– Я не відчуваю себе якимось особливим, філософом чи мислителем, у житті я дуже
проста людина, яку турбують одні й ті самі думки, які приходять багатьом. Я не
вмію і не люблю описувати свої думки текстом, мені набагато простіше передати
це за допомогою образу, а в цьому я маю досить великий досвід.

Багато років тому я працював у рекламі, арт-директором, був відповідальним саме
за художню складову, саме там я навчився просто і образно доносити свої думки.
У своїх роботах я хочу бути зрозумілим, це саме те, що мені найбільш цікаво.

– Як до Вас приходять ці образи? Як відбувається процес творення зображення?

– Кожна чи практично кожна робота, це мій відгук на пережитий день, на ті
думки, які виникли. Це може бути реакція на події країни, а може бути
продовженням спогадів або візуалізація мрії. Буває, що відгук на думки
відбувається миттєво, я просто бачу як її відобразити і сам процес від думки до
її реалізації займає не більше п'ятнадцяти хвилин, але буває, що я ношу думку
кілька днів, щоб знайти їй відповідний образ.

\ii{06_02_2022.stz.news.ua.radiosvoboda.1.mordor_vpade.pic.2}

– Вам вдалося так зобразити Голодомор, що весь жах штучно створеного голоду
«зчитується» навіть тими, хто нічого про це не знали. Що Вас підштовхнуло до
такого образу?

– Декілька років поспіль я малюю плакати на цю болючу тему, все почалося з
того, що в Харкові, на День пам'яті жертв Голодомору, побачив як мало людей
прийшло вшанувати пам'ять загиблих, як мало у вікнах свічок пам'яті і це в
місті навколо якого відбувалися всі ці страшні події. Це було боляче і сумно і
саме це наштовхнуло створення першого плакату, захотілося просто нагадати про
ці страшні події.

\ii{06_02_2022.stz.news.ua.radiosvoboda.1.mordor_vpade.pic.3}

– Розкажіть, будь ласка, про ідею створення «оборонних гербів» для українських
міст. Кому вона належить? Чому меч? Які символізм у цих зображеннях?

– Мій друг, поки що тільки на фейсбуці, Євген Чепелянський, ветеран та людина з
дуже активною громадянською позицією попросив мене поширити інформацію про
створення резерву. Я відреагував на це малюнком із символічним зображенням
Києва, як з'ясувалося не зовсім правильним, бо набір добровольців проводиться
по всій країні, але цей малюнок виявився у нагоді для інших цілей і взагалі
з'явився дуже вчасно.

Мене почали просити про схожі рішення для інших міст, так почала з'являтися
серія.

Головним образом для цього я вибрав меч, захотілося показати, що українські
міста готові дати відсіч Росії. Саме тому, почав шукати символи міст, які можна
органічно пов'язати з мечем.

Треба сказати, що створити такі зображення для всіх міст України завдання дуже
складне, треба знати та відчувати особливості міста, його унікальний код та
символи, а це робота дуже велика і потребує повного занурення та великої
кількості часу. Я поки що не можу точно сказати, що подолаю це завдання,
поживемо-побачимо.

– У Вас багато зображень на тему кохання? Це з Вашого внутрішнього бачення цієї
рушійної, а часом і руйнівної сили, чи ваші «моделі» ходять вулицями, сидять на
лавочках, п’ють каву в кафе?

– Я люблю любити! Закоханість надає життю фарб і піднімає його над буденністю.

\ii{06_02_2022.stz.news.ua.radiosvoboda.1.mordor_vpade.pic.4}

– Що для Вас означає реакція людей на Ваші картини, на Ваші зображення? Які
реакції Вам запали в душу?

– Я отримую дуже багато любові та підтримки, дуже часто це тримає та дає сили.
Мені дуже важливо, що мої думки близькі багатьом людям, це напевно
найголовніше.
