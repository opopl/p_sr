% vim: keymap=russian-jcukenwin
%%beginhead 
 
%%file 26_09_2021.fb.jakimenko_anatolij.1.usik_bolet_ili_ne_bolet.cmt
%%parent 26_09_2021.fb.jakimenko_anatolij.1.usik_bolet_ili_ne_bolet
 
%%url 
 
%%author_id 
%%date 
 
%%tags 
%%title 
 
%%endhead 
\subsubsection{Коментарі}

\begin{itemize} % {
\iusr{Роман Панчошный}

Ну..таке. Це профбокс. В реванші для усатого все може закінчитися тим про що
говорили стосовно першого файту... І буде третій... Бо це гроші Хірна,
стадіону, транслятора і спонсорів.

Особисто мені ця людина огидна - і я знаю цю людину краще ніж абсолютна
більшість дописувачів у ко комментах. Але, це особисте...

Це не сшиває... Хоча, ідея 30+ мені до вподоби

\begin{itemize} % {
\iusr{Анатолій Якименко}
дякую, цікаво

\iusr{Роман Панчошный}
\textbf{Анатолій Якименко} зшити може щось морально на те здатне. Я не хочу і не зможу сприйняти таке "зшиття"

\iusr{Ирина Зайцева}
\textbf{Анатолій Якименко} тут про злиття
\url{https://m.facebook.com/story.php?story_fbid=2420861371378777&id=100003648120104}
\end{itemize} % }

\iusr{Алена Романовская}

Він і є ватник, який ні разу і ніколи не міг відповісти на питання - чий Крим?
Анатолію, Ваш пост це подвійні стандарти...яке Якби був? Відчуття що Вас я к р
а з на цей бій спустили звідкілясь, бо Ви чомусь ігноруєтє ватність Усика, і
всю історію його поведінки

\begin{itemize} % {
\iusr{Анатолій Якименко}
та ні, просто я даю йому шанс виправитися  @igg{fbicon.smile} 

\iusr{Алена Романовская}
\textbf{Анатолій Якименко} а пишуть сусіди з Ворзелю, де нібито живе мама
УСИКА, він тепер зі своїми московськими попами щ е більше знахабніє, а це дуже погано

\iusr{Надія Петрівна}
\textbf{Алена Романовская} чий Крим? І чому за 8 років Крим не повернули в Україну?

\iusr{Надія Петрівна}
Весь фб забитий еспертами по Криму і по Усику

\iusr{Руслан Добровольський}
\textbf{Альона Романовська}
На превеликий жаль, зараз формують ватність і до заходу, який має щодо України свої вузькомеркантильні інтереси...

\iusr{Надія Петрівна}
\textbf{Руслан Добровольський} та це епідемія
І це все йде з москви

\iusr{Алена Романовская}
\textbf{Надія Петрівна} до чого Ви мені то пишете. Судячи з Вашого профіля, христос воскрсна пані, ми з Вами по різні боки, і немає чого мені закидати власні комплекси. Я атеїстка

\iusr{Надія Петрівна}
\textbf{Алена Романовская} я користуюсь правом свободи слова) а ви будьте ким хочете, лише не ганьбіть інших
Тим більше релігія вам далека

\iusr{Надія Петрівна}
\textbf{Алена Романовская} а відповідь на запитання можна почути ?

\iusr{Руслан Добровольський}
\textbf{Надія Петрівна}
На моє бачення, нас тягнуть з вогню, та в полум'я...
Бажання поживитись - з усіх сторін...
А недолуге і непотичне керівництво нашої країни з готовністю розпродує Україну наліво і направо...
Та ще й на південь...

\iusr{Надія Петрівна}
\textbf{Руслан Добровольський} так, погоджуюсь - і так всі 30 років

\iusr{Ирина Зайцева}
\url{https://m.facebook.com/story.php?story_fbid=2420861371378777&id=100003648120104}

\end{itemize} % }

\iusr{Юрий Романенко}

Браво, Толя

\begin{itemize} % {
\iusr{Роман Панчошный}
\textbf{Yuriy Romanenko} С удовольствие посмотрю интервью с человеком, который ковид рекомендовал лечить причащением.
При случае, конечно...

\iusr{Met Albucerce}
\textbf{Юрий Романенко} Юра,а Вы с Анатолием про ковид интервью не проведёте?

\iusr{Юрий Романенко}

\textbf{Met Albucerce} мы их уже сделали кучу

\iusr{Met Albucerce}
\textbf{Юрий Романенко} Про частную медицину и наблюдательные советы - да, а про
ковид ( вакцинация, иммунитет, маски - вот это все ) разве было?
\end{itemize} % }

\iusr{Владимир Маглена}
Дури у него конечно хватает. А в остальном как-то всё не очень. Не верю, одним словом.

\iusr{Alla Vetrovcova}

ЖКХ в Києві - ось вам банальний концепт націїї без пролонгації. Або заробітчани
в Чехії. ті паряться. аж бігом за усиком та рпц. досвідос. бо мені як їхньому
психотерапевту=координатору-перекладчеві завтра на зміну. і єдине, що їх зараз
хвилює, це чи автобус який їх на склади повезе, поїде з тої зупинки, де вони
чекатимуть. Вболівайте, чи вчасно на склади добіжать. Крісмас наближається,
і-шопи рулять, мантра збоку - всім додому. УсиꥥҐҐҐҐ? на Марсі. Жизнь на
складах.

\begin{itemize} % {
\iusr{Роман Панчошный}
Ви що, не хочете зшити РПЦ з КПу ?

\iusr{Роман Панчошный}
\textbf{Alla Vetrovcova} тут є цілий блогєр про те що "народєц нє плохой"

\iusr{Alla Vetrovcova}
та. в мене інша проблема на тому разі. я поширити не можу. всьо широкО і екран пока що не дозволя))

\iusr{Роман Панчошный}
\textbf{Alla Vetrovcova} бачите, як на тому клятому Захаході !!!

\iusr{Alla Vetrovcova}
Я бачу. А вони - нє. вони сподіваються. Подвійні стандарти))

\iusr{Роман Панчошный}
\textbf{Alla Vetrovcova} курва !

\iusr{Alla Vetrovcova}
та вже скоро дОйде. вони тута до листопаду. я їм мляво сповіщаю, але вони тіпо
сподіваються. Правда, деякі з ностальгією згадують, дивлячись на овочі в
супермаркеті Теско, що в них вдома нормальні помідори гниють на городі...
\end{itemize} % }

\iusr{Владимир Маглена}

Да и вера в Бога у него своеобразная. Не похоже это на Православие))) Я понимаю
что он теперь бог и священная корова для многих, но для меня он точно не кумир.
Хотя когда-то нравился, пока о вере не заговорил.

\begin{itemize} % {
\iusr{Екатерина Золотарёва}

\textbf{Владимир Маглена} он как раз умничка) И не сейчас, а всегда  @igg{fbicon.smile} 

\iusr{Владимир Маглена}
\textbf{Екатерина Золотарёва} пофиг

\iusr{Екатерина Золотарёва}

\textbf{Владимир Маглена} а мне нет @igg{fbicon.shrug} 

\iusr{Владимир Маглена}
\textbf{Екатерина Золотарёва} я просто знаю на что способны православные моджахеды. Я их боюсь. Забыли как зимой воевали со шляпой на ёлке?))) Он один из таких.

\iusr{Екатерина Золотарёва}

\textbf{Владимир Маглена} мне его религия совсем не интересна, к счастью) Он боксёр.

\iusr{Владимир Маглена}
\textbf{Екатерина Золотарёва} религия всегда побеждает. Победит и его. Там уже по глазам видно что он побежденный православным Талибаном.

\iusr{Владимир Маглена}
\textbf{Екатерина Золотарёва} аминь. Замолкаю из уважения к вам.)

\iusr{Екатерина Золотарёва}

\textbf{Владимир Маглена} об этом я говорить не могу, я не в теме совсем. Но что он делает как боксёр мне очень нравится) Бокс такого уровня я люблю, это искусство.

\iusr{Владимир Маглена}
\textbf{Екатерина Золотарёва} нехай))) Обнимаю  @igg{fbicon.face.happy.two.hands} 
\end{itemize} % }

\iusr{Володимир Барцьось}
\url{https://www.facebook.com/100001658436692/posts/4428616067203631/?d=n}

\iusr{Сергей Шанюк}
Це коменти зі світської держави. Це якесь середньовічча. Переслідувати за віру зараз це мовітон. Дикунство.

\begin{itemize} % {
\iusr{Oleg Lvivskij}
А ви тільки тепер побачили, що африканське Сомалі і Центральноєвропейське - дві одинакові країни?
Ну вітаю з прозрінням)

\iusr{Сергей Шанюк}
\textbf{Oleg Lvivskij} спершу мав сподівання... від Сомалі це очікувано, але географічно, ми також на перетині торгівельних шляхів. Пірати не тільки на морі
\end{itemize} % }

\iusr{Dmytro V. Ilchuk}
Дуже потрібні слова, потрібні громадянам України

\iusr{Святослав Плачинда}

Любов до Росії в сучасних реаліях України - то є прояв любові до ворога. Той з
українців хто втратив будь що йому дороге від дій російців - той розуміє
безморальність ситуації яку створила влада та її прихильники з 2014 року понині
з одного боку балакаючи казки про "війну та агресора" при цьому досі маючи
Росію самим важливим торговельним партнером зі збереженням дипломатичних
відносин з Кремлем. Такі "громадяни" як цей усік - це продукт поразкового шляху
яким українців ведуть такі дорогі серцю кожного громадянина українські
"патріоти" та політики. І на ці граблі українці як нація наступає всю свою
історію... І винуваті в цьому тільки ми. Результат таких наших безчесних дій не
забариться...

\begin{itemize} % {
\iusr{Виктория Шаповалова}
\textbf{Svyatoslav Plachinda} что попало с чем попало...
\end{itemize} % }

\iusr{Паша Добрый}
Вибачте, але джміль літає за законами фізики. І ніяк інакше.

\iusr{Samij Dobrij}
Проблема в тому що дехто в дуже доброму спортсмені хоче побачити кумира у всьому , так не буває

\iusr{Володимир Воловодюк}

Троянський кінь Усик, є дуже зручним персонажем в умовах інформаційної війни.
Відповідно буде використаний по повній програмі, можливо сам того не розуміючи.

\begin{itemize} % {
\iusr{Poshuk UA}
\textbf{Володимир Воловодюк} 

в першу чергу - він боксер, в другу - українець, який піднімає прапор нашої
країни навіть на росії, співає гімн, має татуювання України і українського
герба ... а для тих кому ще досі не зрозуміло чий Крим, в нього в бою з Джошуа
були рукавиці з написом Сімферополь, шорти і боксерки з прапором України ... І
він переїхав жити в Київ, хоча навіть в Симфірополі ходив по місту в костюмі
олімпійської збірної ... а на рахунок церкви, то серед вірян, перехід до ншої
конфесії означає відступництво ... враховуючи той факт, що Олександр, після
перемоги, будучи на "вершині світу", сказав що цією перемогою він хотів
прославити Іісуса Христа, означає те, що людина глибоко віруюча ... а навіть у
мусульман, є повага до віри інших людей ... щодо відносин з росіянами, то маючи
хороших друзів серед них, ніхто би від них не відрікся, війна це не про кожного
мешканця країни агресора, а виключно про тих, хто ненавидить українців і
Україну ... і в рашці є нормальні люди, а якщо хтось вважає що це не так - то
це нацизм ...


\iusr{Володимир Воловодюк}
\textbf{Poshuk UA} все так, він такий яким є, і має право у вільній країні бути таким. Честь і слава йому, як чемпіону !
Тільки мова була зовсім про інше.

\iusr{Володимир Воловодюк}
\textbf{Poshuk UA} до речі, ось влучно сказано:
\url{https://m.facebook.com/story.php?story_fbid=4530075123725962&id=100001705562678}

\end{itemize} % }

\iusr{Larisa Badun}

Не сприймаю Усика взагалі, навіть без його ватних закидонів ( вискочка), АЛЕ
Ваша підводочка ( відчувається моцна радянська освіта) до "для того, аби
перемогти Росію зовсім не потрібно досягати її розмірів чи мати стільки ж
грошей" - зараховується

\begin{itemize} % {
\iusr{Женя Комаров}
\textbf{Larisa Badun} вискочка? Ви собі уявляєте якого рівня підготовка була до цього бою?

\iusr{Alla Hulovata}
\textbf{Larisa Badun} А ви взагалі боксом цікавитеся? Вболіваєте?

\iusr{Виктория Шаповалова}
\textbf{Larisa Badun} думаю, Усик не огорчается из-за вашего несприйняття- собаки лают, а караван идет!
\end{itemize} % }

\iusr{Ігор Найда}

У значній частині згоден з автором. Вважаю, що Усик, хоч і продовжує перебувати
в парарадигмі «православного кримчанина», розвивається в сторону українства. Не
всім дано відразу народитися націонал-патріотами, особливо коли його батьки
взагалі були далекі до українства, а в оточенні часто переважали московські
шовіністи. На моє глиблке переконання, таким як Усик потрібно допомагати в їх
осимсленні себе свідомим Українцем. А за те, що він представляв Україну (навіть
спортивна форма про це свідчила, не кажучи вже про прапор і гопак), - навіть
подякувати. Дайте таким людям шанс змінюватися на краще (в нашому розумінні
українства). Відчуття нашої прихильності тільки допоможе.

\begin{itemize} % {
\iusr{Александр Беспрозванный}
\textbf{Ігор Найда} а кто ты такой, чтобы ему помогать?

\iusr{Виктория Шаповалова}
\textbf{Ігор Найда} слава Богу, что не все рождаются национал-патриотами

\iusr{Ігор Найда}
\textbf{Александр Беспрозванный} з тикальщиків справді путнього мало.

\iusr{Александр Беспрозванный}
\textbf{Ігор Найда} ты возмести человеку деньги за дом в Крыму. А он кстати живёт на украинской крымской земле. Или ты считаешь, что Крым не наш.

\iusr{Иван Краско}
Это вам нужно дать шанс сходить к психиатру и полечиться от национал- патриотизма.
\end{itemize} % }

\iusr{Максим Циганок}

Забагато уваги до того, що Усик каже. Можно дивитись його бокс і не треба
слухати те що він каже. І питань крім як про бокс йому не слід ставити. Все
просто.  @igg{fbicon.smile} 

\href{https://zn.ua/tech/obratnaja-storona-tsifrovoho-obshchestva-vtoroe-vosstanie-mass-.html}{%
Обратная сторона цифрового общества — второе «восстание масс», Виль Бакиров, zn.ua, 17.09.2021%
}

\iusr{Павел Себастьянович}

Доктор Якименко! Отличный текст!

\iusr{Юрий Троян}
А Вы кто? чтобы вас выслушать можно было...

\iusr{Олег Яковенко}
З власного досвіду.

Пам’ятаю, я як керівник в 2014 році приймав на роботу своїх колег з донецької
та луганської областей, які опинились в жахливій ситуації перебуваючи в статусі
біженців (де хто з них на той момент розмовляв лише російською), і приїхали до
нас на захід бо хотіли жити та працювати в Україні! І дуже добре пам’ятаю
окремих місцевих колег-патріотів, які з осудом та недовірою дивились на них і
казали,- «чого вони сюди просунулись»! Проте час все розставив на свої місця, і
ті, що вимушено приїхали зі сходу нашої країни виявились кращими патріотами
нашоі держави, ніж ті, які їх засуджували! Тому ніколи не судіть та не робіть
поспішних, радикальних висновків!


\iusr{Svitanna Ovchar}
Привичка -віри (іншого не знає) так поводитьця як чуєтьця, стає лекше на душі.

\iusr{Сергей Соломахин}

\ifcmt
  ig https://scontent-yyz1-1.xx.fbcdn.net/v/t1.6435-9/243097697_3048663105377504_3083439939752490457_n.jpg?_nc_cat=103&_nc_rgb565=1&ccb=1-5&_nc_sid=dbeb18&_nc_ohc=Htgcg1qZnFUAX-vuqrD&_nc_ht=scontent-yyz1-1.xx&oh=2a9f06991286c848ddc39e7960b91797&oe=6176C1AE
  @width 0.3
\fi

\iusr{Сергей Соломахин}

\href{https://censor.net/ru/photo_news/93228/poroshenko_stal_diakonom_moskovskogo_patriarhata_i_prinyal_uchastie_v_krestnom_hode_fotoreportaj}{%
Порошенко стал диаконом Московского Патриархата и принял участие в крестном ходе. ФОТОрепортаж, %
censor.net, 15.06.2009%
}

\iusr{Елена Пылыпчак}
Естетика боксу? Хм.

\iusr{Алексей Сериков}
Блин, школьным сочинением, а может и вступительным в вуз застойных годов повеяло.

\iusr{Yuri Petrenko}
Прекрасно!

\iusr{Nataly Bezmen}
Очень красивый был бой. И абсолютно не важно что он думает или говорит. Украинец? Украинец. Победил, и победил красиво? Да. Что еще нужно?

\begin{itemize} % {
\iusr{Dmytro Vernyhor}
\textbf{Nataly Bezmen} как это, что ещё нужно?! Где вопросы журналистов к боксёру про его отношение к Супрун?! А к нашей системе здравоохранения, в целом, а к системе Семашко - особенно)))

% -------------------------------------
\ii{fbauth.golubovskaja_olga.kiev.ukraina.vrach.professor}
% -------------------------------------

\textbf{Nataly Bezmen} иногда мне кажется, что мы живем в дурдоме, где кто-то все время хочет от кого-то идентифицикации

\iusr{Nataly Bezmen}
\textbf{Ольга Голубовская} пора после "мое тело - мое дело" вводить такое же точно отношение к идентификации. Сугубо личное дело ведь, как и вероисповедание.

\iusr{Ольга Голубовская}

\textbf{Nataly Bezmen} я вот тоже не могу понять фразу «треба правильно себе ідентифікувати» ... от слова совершенно ...)
\end{itemize} % }

\iusr{Алексей Гуреев}

Если бы Усика не было - Кремлю стоило бы его создать. Крайне удобный для
путлера персонаж - и Крым у него божий, и живется там украинцам хорошо и
успешно, и МП - святая организация, а не филиал ФСБ...((


\iusr{Александр Крежанивский}
Человек выступает под украинским флагом, какого патриотизма вам тут не хватает?

\iusr{Viktor Plakhuta}

Зверніть увагу на форму представлення Усика від ring announcer Michael Buffer -
він сказав "представляє народ всієї України", а не якесь місто, як це кажуть
зазвичай. Людина з Криму представляє Україну, яку вона по своєму любить.

\begin{itemize} % {
\iusr{Павел Себастьянович}

А что значат слова "по своєму"?

\iusr{Viktor Plakhuta}
\textbf{Pavel Sebastianovich} Да это значит то же самое, как и я ее люблю, и как Вы - по-своему. Все люди разные, и при этом считают эту страну своей Родиной, и видят ее собственными глазами, а не чужими.

\iusr{Павел Себастьянович}

По-моему - одинаково.

\iusr{Екатерина Золотарёва}

\textbf{Павел Себастьянович} Пашунь, всё-таки по-своему каждый. Об этом и пост. Он так. Какой-то Стерненко по-другому.
\end{itemize} % }

\iusr{Ліна Франчук}
А що би дала перемога над Росією? Поміркуйте, шановний авторе,виходячи з незчисленних нинішніх перемог і реалій...

\iusr{Микола Унгурян}

все так, єдине що варто уточнити що це перемога не лише Усика, а швидше
перемога Ломаченка-старшого, який зумів знайти потрібну тактику для Усика

\iusr{Павел Себастьянович}

Одна фраза в посте, с которой я не могу согласиться - "Усик якраз яскравий
приклад біідентичного громадянина. Він любить і Україну і Росію." Откуда мысль,
что он любит Россию? И ладно бы один Усик. Ведь биидентичность любого украинца
предполагает любовь к России, но это совершенно не так. Идентичность русских и
русскоговорящих в Украине совершенно не предполагает любовь к России. Зачем нам
любить Россию, если у нас есть своя страна?

Концепцию "биидентичности" считаю недоработанной  @igg{fbicon.smile} 

Толик, а у нас не может быть одной идентичности? Идентичности принадлежности
вот к такому государству, как оно есть, вот к такой громаде, как она есть.

\iusr{Володимир Брунько}
..то еще Гетьманцев не поднял тему налогов с его, Усика, гонорара! @igg{fbicon.face.grinning.squinting} 
Сразу будет понятно кто какой патриот @igg{fbicon.face.grinning.squinting} 

\iusr{Сергій Король}

Ви вже все майже просрали, прозахідні українці, а ще повчально аналізуєте
Усика. Ви вже і своє домінуюче положення майже просрали, і так мало бути


\iusr{Алла Алексенцева}
ВЕРИМ !!!

\iusr{Иван Райли}

\ifcmt
  ig https://scontent-yyz1-1.xx.fbcdn.net/v/t39.30808-6/242989069_1119811861883521_3682355887747776399_n.jpg?_nc_cat=100&_nc_rgb565=1&ccb=1-5&_nc_sid=dbeb18&_nc_ohc=FSmGjtMoQZEAX8Y3TNn&_nc_ht=scontent-yyz1-1.xx&oh=6fcbe8c06eec0310a8cda06cdfc60221&oe=6157277B
  @width 0.3
\fi

\iusr{Ivan Box}

.після 14-го важко пояснити/виправдати/зрозуміти… позицію «двосторонньої
любові». це вже якийсь джміль шредінгера виходить.

цікаво як, один з найбільш технічних бійців крузер/геві-вейту, свої попередні
бої вигравав без «…навичок швидкого і безпечного зближення…» та техніки
«…ближнього бою…»? жодного нового функціоналу він «не відростив», робив, що й
зазвичай: не перегорів до бою, бився з «холодною головою», підняв/тримав
інтенсивність на позамежному (на тлі джошуа) рівні, але до 8 раунду міг як
виграти так і програти, але виявився краще функціонально готовим за опонента і
в чемпіонських раундах показав майстер клас.

якщо «…Усик показує всім патріотам і не дуже патріотам, що для того, аби
перемогти Росію зовсім не потрібно досягати її розмірів чи мати стільки ж
грошей. Потрібно бажання, зусилля і розуміння в якому напрямку ці зусилля
докласти…», що тоді, починаючи з 14-го, робили волонтери/добробати?!

\iusr{Анатолій Матузко}

Нахрена эта лекция кто есть кто!!  А ну вспомните, кто в какую церковь ходил из
политиков!?  Все и в российскую и украинскою! Повторяю Все!! Даже преданные
патриоты нынешние!  А сейчас в другого бога и в другой церкви молятся! Позор!


\end{itemize} % }
