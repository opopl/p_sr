% vim: keymap=russian-jcukenwin
%%beginhead 
 
%%file 30_03_2022.fb.omarov_ruslan.1.russkij_ressentiment.cmt
%%parent 30_03_2022.fb.omarov_ruslan.1.russkij_ressentiment
 
%%url 
 
%%author_id 
%%date 
 
%%tags 
%%title 
 
%%endhead 
\zzSecCmt

\begin{itemize} % {

\iusr{Андрей Маргулев Иностранный Агент}

\href{https://activatica.org/content/d5f30299-9434-404b-b405-8d8dcfaae368/istoriya-psevdogosudarstva-rossijskogo}{%
История псевдогосударства российского, Андрей Маргулев, activatica.org, 30.03.2022%
}

\raggedcolumns
\begin{multicols}{2} % {
\setlength{\parindent}{0pt}

Я начну с актуального. Альфой и омегой российской государственности являлось и
является принципиальное непризнание любых ошибок на пути ее формообразующей
«исторической» или «божественной» миссии: торжества над другими народами.
Именно по этой причине пока существует режим, выбравший своей целью уничтожение
украинской государственности, любые «мирные процессы» с его участием будут
фикциями. А украинской государственности в российской традиционной парадигме
быть не может и не должно.

Летом 2014 года я написал работу «Рождение нации»
(\url{http://margulev.narod.ru/nationborn.html}), где в качестве исходного
пункта было взято консенсуальное признание большинством россиян некоего
«пункта»:

\ifcmt
  ig https://storage.googleapis.com/activatica/uploads/c6aface6-9623-4471-8578-d656eaffdded
\fi

«Пункт этот не нов, а имеет более чем давнюю историю, так что Путину не
пришлось в этой части ничего придумывать. Звучит он, примерно, так: Украина –
это исторически некая \enquote{окраинная} область Великой России, которая
достаточно искусственным путем получила статус государства. Поскольку это
\enquote{окраина} – русский язык там оказался \enquote{испорчен}, а понятие
\enquote{украинская нация} – националистический миф, выдуманный уязвленным
сознанием части \enquote{окраинной} элиты. И, раз нет \enquote{нации} –
государство является искусственным институтом, который можно безболезненно
\enquote{упразднить}».

Заканчивалась эта работа так:

«То, что происходило в ноябре 2013 – феврале 2014 на Майдане и вообще в Киеве,
внешне напоминало все, что угодно – массовые беспорядки, восстание, революцию –
но только не то, чем являлось в своем глубинном значении: рождением Нации.

В эти считанные месяцы через граждан Украины, непосредственно вовлеченных в
противостояние, проявились все те вызревшие в ее исторической судьбе качества,
которые не имеют ничего общего с качествами \enquote{старшего брата}, прошедшего
многовековую отрицательную византийско-татарскую селекцию. /.../

Самоценность свободы выбора, сохраненная в языковой культуре, истории, и,
главное – в генетической памяти украинцев со времен столь трагического и столь
героического Войска Запорожского, – вот что преобразило Украину в эти месяцы,
вот чего не могли учесть кремлевские политтехнологи.

Ну, а все, что произошло и еще произойдет далее – это просто закрепление \enquote{точки
невозврата} пробудившейся украинской нации к тому летаргическому состоянию, в
который она пребывала, ожидая своего исторического часа...»

Ныне Украина продемонстрировала свою полную состоятельность всему миру. Крах
теории «псевдогосударственности» Украины становится очевиден даже прозревающему
российскому населению. А вот с государственностью российской дело как раз
обстоит совсем иначе.

Собственно, ей я уже посвятил две работы: «Прощание с Химерой» (2020) и
«История с географией» (2021), – целями которых было выявление химерической
природы этой государственности. Химера – существо противоестественное и потому
обреченное на мучительное существование больного организма с неизбежной его
гибелью по катастрофическому сценарию. Противоестественность российской
государственности состоит, как обосновывалось в этих работах, в ее культурной
двухуровневости, когда традиционная «избранническая» культура базового уровня
постоянно конфликтует с модерновой, заимствованной у Запада культурой верхнего
уровня. И вот только что мы стали свидетелями некоего явления, неизбежного
ввиду неизбывности такого конфликта.

Дело в том, что одно из проявлений химеризма в области государственного
управления России состоит в том, что истинными целями этого управления являются
не декларируемые государственные интересы, а интересы самих управителей. В
«нормальных» государствах, формировавших модерновый слой российской культуры,
это явление называется «коррупцией» и подавляется. В России же это явление
становится камуфлируемым, но в реальности системным принципом управления. Путин
– это воплощение химерической российской государственности, стал вершиной всей
этой системы, которую вполне можно назвать «государственно-частным
партнерством» в той полностью извращенной в России форме, когда прибыль идет
частным лицам, а убытки – государству.

Тот грандиозный военный провал, который явила Россия в своей агрессии против
Украины – это «момент истины», который счастливый строитель этой системы как-то
не учел. Дело в том, что вся эта система управления была полностью заточена под
обеспечение финансовых потоков «наверх» – туда, где узкий круг лиц формировал
тайные фонды для тайных экспансий, а попутно, обеспечивал роскошным
существованием причастных к этому формированию лиц. Экспансия с помощью тайных
финансовых операций – подкупа элит и найма агентов – должна была по замыслу
творцов этой системы обеспечивать не только их благоденствие, но и
экономическую, в том числе и военную состоятельность государства.

Что же получилось в результате? Действительной целью любого, в том числе и
оборонного заказа – той целью, которая только и оценивалась вертикалью
заказчиков – стало ответвление «наверх» денежных потоков с каждого уровня.
Заказы перешли в разряд полувиртуальных сущностей, достаточных в форме тех или
иных моделей и «опытных образцов» и безудержной лживой пропаганды. И в тот
момент, когда химерическое псевдогосударство, опьяненное наркотиком собственной
пропаганды до степени потери связи с реальностью, решило обрушиться всей
накопленной, казалось бы, мощью на своего заведомо более слабого соседа, вся
эта полувиртуальная «мощь» рассыпалась в прах – ибо была им изначально.

\end{multicols} % }

\iusr{Максим Миранский}
Абсолютно!

\begin{itemize} % {
\iusr{Ruslan Omarov}
\textbf{Максим Миранский} 

ну я не уверен. Размыто немного. Но суть вот мне кажется в этом, в страхе.
Страх тоже может цементировать общество.

\iusr{Максим Миранский}

А я, наоборот, вижу и чую правоту Вашу... - хоть и больно признавать! Страх
ненадёжен онтологически, да, но построить и строить на нём империю можно - не
вопрос - увы - и мы тому солидное подтверждение! А Страх ответственности - это
инфантилизм высшей пробы!

\iusr{Ruslan Omarov}
\textbf{Максим Миранский} спасибо @igg{fbicon.heart.red}
\end{itemize} % }

\iusr{An Cher}
Идеальный текст. Ухвачена и блестяще сформулирована сама суть.

\begin{itemize} % {
\iusr{Nail Sabiroff}
\textbf{Андрон Вишня} 

нечто похожее говорил Мюллер Штирлицу. ошибся только в одном: имел в виду новое
поколение немцев, а случилось всё в россии.

\iusr{Андрей Чертков}
\textbf{Наиль Сабиров} - 

никто ни в чем не ошибся, ибо сформулировал все эти якобы \enquote{немецкие} мысли
русский писатель Юлиан Семёнов, а в реальности их просто не было, как не было и
никакого Штирлица.

\iusr{Ruslan Omarov}
\textbf{Андрон Вишня} не идеальный. Просто поток мыслей )

\iusr{An Cher}
\textbf{Ruslan Omarov} Ну, ладно. Просто очень правильно сформированный и направленный именно туда, куда надо. Так сойдёт? )

\iusr{Ruslan Omarov}
\textbf{Андрон Вишня} ладно ) мерси  @igg{fbicon.hearts.two} 

\iusr{Nail Sabiroff}
\textbf{Андрей Чертков} спасибо что просветили)
\end{itemize} % }

\iusr{Татьяна Васильева-Сильченко}

Не будет этого уродливого образования на карте мира в сегодняшних границах.
Этой шовинистической ямы, этой черной дыры с ее ресентиментом не должно быть.
Для безопасности человечества. Carthago delenda est

\begin{itemize} % {
\iusr{Ruslan Omarov}
\textbf{Татьяна Васильева-Сильченко} может и не будет. Даже скорее всего. Но земля и люди останутся, люди будут рефлексировать.

\iusr{Татьяна Васильева-Сильченко}
\textbf{Ruslan Omarov} 

По-другому будут рефлексировать, если лишить их возможности даже намека на
имперскую спесь. Они в страхе будут жаться в свой «петушиный» угол. Пусть и в
бессильной злобе

\iusr{Guy Frankovich}
\textbf{Ruslan Omarov} 

рефексируют люди в разных тихих мирных странах по чудовищам своего прошлого...
уверен, что и в Германии есть немало рефлексирующих по рейху, только это
подавлено законами и стандартами тамошнего общества

\iusr{Ruslan Omarov}
\textbf{Guy Frankovich} 

для этого необходимо два условия: разрушение государства и внешнее управление
проектом, который возник на его месте. Как в Германии и Японии после WW2. Будет
ли так с рф мы наверняка не знаем.

\iusr{Guy Frankovich}
\textbf{Ruslan Omarov} хотелось бы

\iusr{Ruslan Omarov}

То есть одного условия недостаточно. Если не станет государства (например,
Россия распадается на некие субъекты) но не будет внешнего управления -
государство быстро восстановится по старым лекалам: централизация, аппарат
насилия, антидемократические механизмы, вождизм и тп. Если наоборот будет
внешнее управление, но государство не будет разрушено, то повторятся 90е годы -
вначале демократическим путем изберут одного популиста, потом он выдвинет
преемника и здравствуй путинщина 2.0

\end{itemize} % }

\iusr{Сергей Бондарев}

Боялись, боялись, ррррраз и перестали боятся. А все почему? Потому, что
рядышком есть точно такой же народ который, так сказать, пораньше встал и
отряхнулся. Он же точно знает, что никакие мы не великие.))

\iusr{Витя Бревис}
Здорово! Если ты это сам придумал/догадался, то ваще круть и завидно по хорошему

\begin{itemize} % {
\iusr{Ruslan Omarov}
\textbf{Витя Бревис} хз, по моему это очевидно. Драйвером русской государственности всегда был страх.

\iusr{Сергей Бондарев}
\textbf{Ruslan Omarov} 

но все же страх не первопричина. Это тоже следствие чего-то. А вот чего, тут
уже стоит порассуждать. Я склоняюсь к отрицательной селекции. Страна, между
Азией и Европой, большая территория которой это климатические неудобья с зоной
рискованного земледелия, сотни лет охваченная выживанием, войнами,
экспансионистской политикой как следствие этого противостояния, просто не может
не растерять всех, выражаясь просто, умных и смелых, когда элитами становятся
худшие. Вот результат отрицательной селекции.

Социолог Питирим Сорокин впервые применил этот термин. «В периоды острых
социальных катаклизмов самыми приспособленными оказываются не лучшие, а
средние, способные слиться с массой в ее инстинктивных мотивах и не
дистиллированных разумом побуждениях». Через это явление проходили все страны и
тем более все империи. Просто российская империя возникла позже остальных -
помните 3-й Рим - поэтому к ней это и пришло позже остальных. Я много поездил
по Европе и мог сравнить уровень развития культуры одного и того же временного
периода у разных народов. Так вот, более развитых всегда побеждали менее
развитые потому, что так сказать спокойствие, это тоже развитие культуры.
Арабскую культуру уничтожили крестовые походы. В Испании культурных мавров
вытеснила Реконкиста и сама стала культурной. Рим развалили варвары и сами
стали Римом и т.д. и т.п. В то время когда в Европе в 12 веке возводили
великолепные готические соборы в России строили избы. То, что на Руси начали
строить в 12 веке в Европе строили ещё в 10-м веке. Ничего нового. Все уже
было.

\iusr{Ruslan Omarov}
\textbf{Сергей Бондарев} 

согласен. Что-то должно лежать в основе. Какая-то внешне бравурная идеология,
но содержащая имплицитную модель иерархии, основанную на страхе и унижении.
Государство как фетиш, например, бюрократия как патрон патронус,
полурелигиозное отношение к государству - возможно, это византийское, как
например считает Игорь Приходько. Очень похоже.

\iusr{Сергей Бондарев}
\textbf{Ruslan Omarov} 

я считаю, что в нынешние времена любая государственная модель направлена против
конкретного индивида. Будь то это так называемые демократические страны или
такая модель, как в РФ. Государство сейчас это целая пирамида идеологий во всех
сферах, будь то партии, религии, карьера, борьба за экологию и т.д. и т.п.
нанизаная на один штатив. Как детская деревянная игрушка пирамидка. Как ни
крути, а мы потихоньку продвигаемся к социализму. Вопрос лишь в том,
естественная ли это стадия развития человечества. Думаю, что в современных
реалиях вполне естественная. Надо ли делать что-то, чтобы не допустить такого
\enquote{развития}? Конечно надо. Будет ли это полезным? Нет, не будет. Потому, что
большинство живущих на земле людей, это толпа, и слепить из нее можно все, что
угодно. Те 5 -15\% людей у которых есть критическое мышление могут делать все,
что угодно и в итоге не сделать ни-че-го просто потому, что их никто не будет
слушать. Могут ли элиты управлять большинством? Да могут. Будут? Нет не будут.
Просто потому, что абсолютно очевидна деградация элит по Питириму Сорокину.
Современные мировые (так называемые) лидеры лучшее тому подтверждение.))

\ifcmt
  ig https://scontent-mxp1-1.xx.fbcdn.net/v/t39.30808-6/277488738_4944850725592419_4937344130691855529_n.jpg?_nc_cat=101&ccb=1-5&_nc_sid=dbeb18&_nc_ohc=EhD9MlVipLIAX8MojNS&_nc_ht=scontent-mxp1-1.xx&oh=00_AT99qkDvFZkB5Pu-r3AebtcmKnuz_Webrbu4PEhG5VGD7w&oe=626E8CA0
  @width 0.3
\fi

\end{itemize} % }

\iusr{Vadim Kos}
Главные скрепы путинизма - хамство, наглость, ложь.

\begin{itemize} % {
\iusr{Ruslan Omarov}
\textbf{Vadim Kos} нет, это как раз вторично. В основе именно страх

\iusr{Ruslan Omarov}

Страх порождает все остальное - агрессию, ложь вот это все. Это орда людей,
охваченных страхом друг перед другом в первую очередь, и перед собственной
массой, перед тем что они называют государством и тп

\iusr{Vadim Kos}
\textbf{Ruslan Omarov} согласен.
\end{itemize} % }

\iusr{Theo Exit'Exists}
Божечки, как же Вы прекрасно пишете!!! И очень глубоко! И очень в сердце...

\begin{itemize} % {
\iusr{Ruslan Omarov}
\textbf{Тео Просто} толку то  @igg{fbicon.frown} 

\iusr{Theo Exit'Exists}
\textbf{Ruslan Omarov} очень понимаю ((
Ну, немного обратной связи и виртуальных обнимашек. Такой толк. Не помогает, да?

\iusr{Theo Exit'Exists}
\textbf{Ruslan Omarov} 

а ещё, например, я всех боюсь, но мне всегда была невыносима мысль, что кто-то
будет бояться меня. Я даже защищал в школе тех, кого травли.

Но ещё меня занимала мысль о том, как, например, люди вступают в банды - там
ведь ещё страшнее просто постоянно. Ну, и как бы вот это Ваше рассуждение о
том, что страх это валюта, которую сначала вымогают у тебя (фигура речи),
которой ты платишь за условную, даже кажущуюся, личную безопасность здесь и
сейчас, ты чувствуешь, что твой личный доход в этой валюте как бы
увеличивается, если кто-то от твоего имени вымогает страх у других, - оно, это
рассуждение, очень крутое!

\iusr{Ruslan Omarov}
\textbf{Тео Просто} 

это очень остроумная метафора с валютой, спасибо!  @igg{fbicon.thumb.up.yellow}
Я в такой плоскости не рассматривал

\iusr{Theo Exit'Exists}
\textbf{Ruslan Omarov} это шульмановская метафора!
\end{itemize} % }

\iusr{Igor Prykhodko}

Поделился. И должен заметить, что даже страх-то не аутентичный, а
заимствованный. Завезён на территории из Византии вместе с двухглавым орлом и
шапкой Мономаха. Хотя выясняется, что про шапку Мономаха они в очередной раз
соврамши...

В остальном, пожалуй, соглашусь. За исключением соображений о красной кнопке и
инстинкте самосохранения, который в некоторых ситуациях всё-таки оказывается
сильнее стремления повиноваться. Вот бросают же они в полях Украины свои
суперсовременные танки и прочие железяки. Отказываются же некоторые садиться в
относительно безопасные истребители какого-то поколения. Отказываются даже
воевать с нами ехать  @igg{fbicon.wink} 

В общем, как говорят на моей исторической родине, это ужас, но не
ужас-ужас-ужас  @igg{fbicon.smile} 

\begin{itemize} % {
\iusr{Ruslan Omarov}
\textbf{Игорь Приходько} 

спасибо! @igg{fbicon.thumb.up.yellow}  От вас похвала всегда ценна  @igg{fbicon.smile}  что касается кнопки - боюсь, тут
немного разные условия. Солдат под огнем да, бросает танк. Тот же солдат или
генерал в теплом бункере с кнопкой - я бы не был так уверен  @igg{fbicon.frown} 

\iusr{Igor Prykhodko}
\textbf{Ruslan Omarov}, 

что-то мне шепчет, что в тёплом бункере куда страшнее, чем в неуютном танке. И
чем теплее бункер (длиннее стол, толще бронежилет) -- тем страшнее обитателю.

\iusr{Ruslan Omarov}
\textbf{Игорь Приходько} 

страх в бункере конкретизирован и персонифицирован. Это фсошник со шприцем, вот
он. А ядерная зима - нечто абстрактное.
\end{itemize} % }

\iusr{Igor Kaytov}
Умно.

\iusr{Alexander Kovrov}

Ты знаешь, 25-30 - поколение, которое ностальгировать не будет. Вот 40-60
будет. 30-40 промежуточно.

\begin{itemize} % {
\iusr{Ruslan Omarov}
\textbf{Александр Ковров} 

будет. Все создают мифы после 40-45. Тут как совпадает - если конец империи
придется на кризис среднего возраста, то это одно. Если ляжет раньше или позже.
Ну в целом это да - не угадаешь конечно.

\iusr{Alexander Kovrov}

Вообще уже сейчас у молодых вижу ностальгию по 2008-2012 - Медведев такой
молодой и юный октябрь впереди. Типа была какая-то развилка, которую мы
проскочили не туда.

\iusr{Ruslan Omarov}
\textbf{Александр Ковров} 

что тоже миф ) не туда проскочили. Реальность это другое - это \enquote{не там
соскочили} ) вот кому повезло соскочить в сытые медведевские годы, грамотно
наскирдовать баблос и уйти в третье гражданство - те реалисты )

\end{itemize} % }

\iusr{Tatarnikov Oleg}

Извини, Руслан, но всё это полная херня...

Имперские народы рефлексируют совсем по-другому.

Читайте Киплинга про то \enquote{как} или поговорите с испанцами...  @igg{fbicon.smile} 

Уж \enquote{испанский-то мир} не в пример больше \enquote{русского} и кого там
больше \enquote{боялись} ещё вопрос.

Ностальгия по \enquote{нас боялись} больше характерна как раз для
\enquote{малых народов}, неудачно притулившихся в какое-то время под чьё-то
крыло и потерявших самоидентификацию...  @igg{fbicon.smile} 

\begin{itemize} % {
\iusr{Ruslan Omarov}
\textbf{Tatarnikov Oleg} 

Олег, я б сильно удивился если бы ты согласился с моими тезисами
@igg{fbicon.wink}  но вообще имперские народы рефлексируют по разному, это
правда. Однако конкретно \enquote{нас все боялись/боятся}, притом стоящее во
главе угла - это вот чисто русское. Ну я хз, может ещё у гуннов было нечто
подобное, но летописей не осталось.

\iusr{Tatarnikov Oleg}
\textbf{Ruslan Omarov} 

Да нет же! Спроси у меня — я ж вятич на 146\%, \enquote{древний великоросс} по
дореволюционному определению и персонаж мемов, когда надо подчеркнуть
\enquote{русскость}, как рязанец.  @igg{fbicon.smile} 

И нас-то реально боялись и киевские князья, и московские, и даже татарские  @igg{fbicon.smile} 

Но рефлексия у нас абсолютно \enquote{испанская}. Мне даже сейчас испанский
крестьянин ближе по менталитету, чем советский колхознек.

Вот у мордовской вохры да, есть такое про "нас боялись" в отношении
государства. Да и то не массово  @igg{fbicon.smile} 

А у нас до сих пор на гербе наш князь Олег, который с Литвой и Мамаем воевал
против московитов.

Сложная эта нация — \enquote{русские}.

«И гордый внук славян, и финн, и ныне дикой Тунгуз, и друг степей калмык»

\iusr{Tatarnikov Oleg}
\textbf{Ruslan Omarov} 

Я, кстати, по этому поводу очень не люблю перевод Гамлета Пастернака (не помню,
говорил я тебе или нет).

Он совершенно не понял \enquote{имперской рефлексии} Гамлета в трагедии, а там
это ВЕДУЩАЯ тема:

The time is out of joint: O cursed spite,

That ever I was born to set it right!

\iusr{Ruslan Omarov}
\textbf{Tatarnikov Oleg} 

о это я помню :)) ты меня ещё в прошлый раз пытался убедить, что Гамлет
рефлексирует по великодержавности патриотизму и вот этому всему. Я хз, я вот
как то не заметил там такого особенного. А или ты у него там гражданственность
искал?  @igg{fbicon.thinking.face}  Ну короче, Гамлет - феодальный принц,
рефлексия там чисто феодальная - трон это трон, семейная собственность,
узурпатор - покусился в первую очередь на мое право по крови, а уже вторично -
на Данию. Ну такой истмат

\iusr{Tatarnikov Oleg}
\textbf{Ruslan Omarov} 

Ну вот, видишь, недоступна тебе рефлексия настоящего имперца!  @igg{fbicon.smile} 

Да, это всё есть, канешна. И право крови, и потреотизЬм, и Дания...

Но основной прикол там совсем не в этом.  @igg{fbicon.smile} 

А в \enquote{бремени белого человека} и личной ответственности за ВСЁ мироздание.

Да, это \enquote{великодержавный шовинизЬм}, который, кстати, не имеет ничего общего ни
с национализЬмом, ни тем более с нацизмом.

Но именно эта \enquote{ответственность} воспитывается у имперской элиты с детства, а не
\enquote{право собственности}.

\iusr{Дмитрий Овсянников}
\textbf{Tatarnikov Oleg} 

извините что влезаю, но прям прогрессорством Стругацких повеяло: все эти
комконы-руматы... тоже ведь по сути реплика имперской рефлексии, нет?

\iusr{Tatarnikov Oleg}
\textbf{Дмитрий Овсянников} 

По сути да, только там оно больше \enquote{рационалистическое}, а у имперцев скорее
\enquote{мифологическое}  @igg{fbicon.smile} 

\end{itemize} % }

\iusr{Victor Buravlev}
Согласен.

\iusr{Tania Chimpoesh}

Интересно. Обобщать всегда опасно (это тоже обобщение))). Но механизм проекции
описан точно))))

\iusr{Ruslan Omarov}
\textbf{Таня Чимпоеш} 

обобщения - кривая дорожка, согласен ) я в общем то не собирался писать
политическое эссе, просто собрал в пучок некоторые мысли

\iusr{Damir Musin}
Шикарно написано!

\iusr{Ruslan Omarov}
\textbf{Дамир Мусин} спасибо )

\iusr{Viktor Bairak}

Вот и оформились в букву наши с тобой одесские беседы. Значит не напрасно было
проведено время. Обнимаю.

\iusr{Vadim Kalinin}

Очень хорошо сказано. Действительно эмоциональное содержание русской культуры
это тем или иным способом упакованный страх. Обертка бывает достаточно яркой,
но страх в глубине все равно прощупывается. Скажем под тактильной феерией
Бунина скрывается страх перед \enquote{глубинным народом}. Достоеский - ужас перед
\enquote{нутром человеческим}. Толстой боится вымышленных субъектов им же и
придуманных. Булгаков боится трусости. И так далее.

\begin{itemize} % {
\iusr{Tania Chimpoesh}
\textbf{Vadim Kalinin} страх общечеловеческое переживание, вообще то)))))

\iusr{Vadim Kalinin}
\textbf{Таня Чимпоеш} 

Я говорю: \enquote{Вася сидит на героине}. Ты отвечаешь: \enquote{Героин действует на всех людей
вообще-то!}. Я знаю, что он действует на всех. Я говорю о другом. О том, что
Вася торчит и скоро сдохнет.

\iusr{Vadim Kalinin}
\textbf{Таня Чимпоеш} 

А так, вообще культуры, где страх выступает в качестве базовой стяжки
социального конструкта на планете чень не много. Это русская, китайская,
северо-корейская, арабская и часть африканских культур. Все остальные люди
боятся только когда на самом деле опасно, а не всю жизнь.

\iusr{Tania Chimpoesh}
\textbf{Vadim Kalinin} очень хочется верить, что так и есть.

\iusr{Ruslan Omarov}
\textbf{Vadim Kalinin} спасибо, Вадим @igg{fbicon.heart.red}

\iusr{Vadim Kalinin}
\textbf{Ruslan Omarov} Не за что! Это тебе спасибо, за прекрасное высказывание.

\iusr{Vadim Kalinin}
\textbf{Таня Чимпоеш} Зачем верить? Всё же очевидно. Вы были за рубежом?

\iusr{Tania Chimpoesh}
\textbf{Vadim Kalinin} я сейчас с германии

\iusr{Tania Chimpoesh}

Европейцы боятся ядерной катастрофы. Просто вам она сейчас гораздо меньше
угрожает. Многие русские кто побогаче срочно вылетели в америку.

\iusr{Vadim Kalinin}
\textbf{Tania Chimpoesh} 

Ну тогда вы должны понимать разницу в укладе жизни. На западе страх не является
социальным клеем. Там все построено на выгоде. У нас на востоке социальные
связи важнее выгоды. В этом есть свои плюсы и свои серьезные минусы.

Здесь тоже все перепуганы ядерной угрозой. Ядерная война может уничтожить всю
цивилизацию. Выживет разве что Океания...

\iusr{Tania Chimpoesh}
\textbf{Vadim Kalinin} ну это упрощение)))

\iusr{Vadim Kalinin}
\textbf{Tania Chimpoesh} 

Что упрощение? Про то что азиаты более социальны? Это не упрощение. Это чистая
правда. У Азиатов другой набор допаминовых рецепторов. Они получают большее
удовольствие от одобрения сообществом и меньшее от путешествий. Здесь
повсеместно клановая структура. И я постоянно вижу, на мой взгляд, ужасные
вещи. Например, когда одного из детей специально воспитывают для ухода за
стариками в семье. Такой человек не получает образования но и на работу не
ходит. Его кормит клан.

Здесь вообще куча людей, которые не парятся по заработку. В клане может
работать процентов 15 людей. Остальные обслуживают работающих. Это как пчелиный
рой. Помягче конечно, но идея та же.

Я очень люблю Азию. Даже больше, чем Европу. Но клановость эта меня пугает.

\end{itemize} % }

\iusr{Tatiana Bonch}

Только они это называют \enquote{нас уважают}

И это не (с)только право подворотни, но и Раскольникова. Такого персонажа,
правда, у Достоевского не было, кто бы утер Родиону Романовичу пот со лба и
избавил от комплексов/поднял с колен

\iusr{Андрей Костриков}

У старорежимного Достоевского в романе была всё же сугубо положительная Соня и
следователь Порфирий Иваныч. Те, кто уломали его покаяться. А тут иное.

Представьте Родиона Романовича, которому

а) сообщили о скорой смерти от рака

б) Соня Мармеладова постоянно говорит о том, что не убивший процентщицу в рай
не вхож

в) следователь вместо изъятого вещдока (топора) протягивает под столом атомную
бомбу со словами \enquote{Вижу я, затейник вы, Родион Романович. Но массовик ли?}

г) ФСБатюшка зачитывает вместо Евангелия Дугина

\iusr{владислав тушнин}

Мне 72... кроме отвращения последние 22 года не вызывают ... и я не один: ...
такой умный

\iusr{Валерий Рысин}

И всё же, как только повернётся излучатель, картина сменится очень быстро


\end{itemize} % }
