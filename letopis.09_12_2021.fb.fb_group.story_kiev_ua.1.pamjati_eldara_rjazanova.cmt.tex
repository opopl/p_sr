% vim: keymap=russian-jcukenwin
%%beginhead 
 
%%file 09_12_2021.fb.fb_group.story_kiev_ua.1.pamjati_eldara_rjazanova.cmt
%%parent 09_12_2021.fb.fb_group.story_kiev_ua.1.pamjati_eldara_rjazanova
 
%%url 
 
%%author_id 
%%date 
 
%%tags 
%%title 
 
%%endhead 
\subsubsection{Коментарі}

\begin{itemize} % {
\iusr{Елена Мельникова}
Как поэтично отозвались о режиссере!

\iusr{Анатолий Золотушкин}
\textbf{Елена Мельникова} Спасибо

\iusr{Chyzhyshyn Oksana}
А де саме був цей кінотеатр Дніпро ?
Саме там, де зараз Ляльковий театр?

\begin{itemize} % {
\iusr{Анатолий Золотушкин}
\textbf{Chyzhyshyn Oksana} да

\iusr{Irena Visochan}
\textbf{Chyzhyshyn Oksana} Гуляешь по Крещатику, идёшь к стадиону,, Динамо,,(тогда) и тут слева, вверху на ,,Дніпрових кручах,, возвышается кинотеатр ,,Днепр,, позже он стал к-том ,,Стерео,,

\begin{itemize} % {
\iusr{Chyzhyshyn Oksana}
\textbf{Irena Visochan}
От дякую, а то поспорили десь в коментарях, чи був Дніпро стерео.
Я його запам, ятала якраз по стереофільмах, бо в Києві просто кінотеатрів вдосталь.
А мені доказували, що тільки Орбіта біля Бесарабки.

\iusr{Alexander Kleimenov}
\textbf{Chyzhyshyn Oksana} «Орбіта» теж була стерео до того, як стерео зробили «Дніпро»

\iusr{Chyzhyshyn Oksana}
\textbf{Alexander Kleimenov}
Знаю, була )
Мені доказували, що тільки Орбіта була стерео

\iusr{Irena Visochan}
\textbf{Chyzhyshyn Oksana} 

На Красноармейской был изначально к-т ,,Кинопанорама,, тоже с какими -то
спецэффектами, не помню точно, исправьте кто знает. Вот ,когда уже, его после
ремонта назвали Орбита, по-моему, у меня не получилось там бывать, мимо проходила
каждый день. Может, Орбита тала стерео, а,, ,,Зоря,, Дружба,, стали кинотеатрами
повторного фильма, возможно, и Днепр как-то перепрофелировали. Не уверена.


\iusr{Лариса Павловская}
\textbf{Irena Visochan} 

Кинопанорама была на Шота руставели возле кукольного (ныне синагоги Бродского).
Я там тоже в специальных очках какой-то фильм смотрела. Виноградник запомнился
с крупными гроздями винограда

\iusr{Irena Visochan}
\textbf{Лариса Павловская} 

А позже Кинопанорама стала Орбита, я не ошибаюсь? В Кинопанораму мы ходили всем
классом (культпоход) выдавали очки, зрелище было запоминающееся на всю
жизнь. Причем, там подолгу шёл один фильм, всегда была очередь (много гостей
киевских), а потом на следующий новый фильм-очередной культпоход. Класс!

\end{itemize} % }

\iusr{Лариса Павловская}

Нет, \enquote{Орбита} была в другом месте, недалеко от Бессарабки, перед поворотом
налево, вверх, к Круглоуниверситетской.

\end{itemize} % }

\iusr{Валентина Бодина}
Зачем Вы так грустно?... Прямо плачется...

\begin{itemize} % {
\iusr{Ольга Морозова}
\textbf{Валентина Бодина} Нельзя же постоянно быть ,,на приколе". Грусть - прекрасное чувство.

\iusr{Валентина Бодина}
\textbf{Ольга Морозова} Да, светлая грусть - это хорошо...

\iusr{Анатолий Золотушкин}
\textbf{Валентина Бодина} Пост памяти

\iusr{Татьяна Ильченко}
\textbf{Анатолий Золотушкин} спасибо за пост. Написано очень душевно и с любовью! Таким теплом все отозвалось в душе и в воспоминаниях!

\iusr{Анатолий Золотушкин}
\textbf{Татьяна Ильченко} спасибо
\end{itemize} % }

\iusr{Мария Константиновская}
Фото - супер! Спасибо!

\iusr{Надежда Лабик}

Я тоже ходила в этот кинотеатр. И маршрут у меня был такой же. А фильмы
Рязанова-это наша тогдашняя действительность или наше представление о счастье.
Сейчас сериалы, похожие друг на друга... Рязанов не повторим.

\iusr{Ян Медников}

Я не только все его фильмы смотрел много раз, но и покупал книги. В магазине
"Мистецтво" на Крещатике".

\iusr{Ольга Морозова}
Спасибо за память.

\iusr{Tetyana Drobot}
Какие сладкие грустные воспоминания @igg{fbicon.hands.pray} .
А мы ходили с мамой на утренние сеансы смотреть фильмы в «Лейпциг». Билет 10 коп.
После сеанса шли домой через парк и напевали «Пять минут, пять минут..».
Окунулась в детство, спасибо @igg{fbicon.face.smiling.hearts} 

\iusr{Анатолий Золотушкин}
\textbf{Tetyana Drobot} спасибо

\iusr{Natasha Levitskaya}

Спасибо, теплые воспоминания! Любимые фильмы любимого режиссера! Может быть и
снимает ТАМ... Ведь там уже целая эпоха, лучшие...

\iusr{Елена Кордан}
Самые лучшие и незабываемые моменты жизни -это детство с мамой.

\iusr{Лариса Тарасюк}

...геніям властиво передбачувати майбутнє, таким був режисер. На жаль, бачим і
\enquote{...небеса}(.

\iusr{Людмила Аскерова}

\ifcmt
  ig https://i2.paste.pics/cefd6828e86cca683480428a65e0510a.png
  @width 0.4
\fi

\iusr{Светлана Манилова}

Перефразируя известную цитату из \enquote{Иронии судьбы}, могу сказать: \enquote{У меня такое
ощущение, что за эту ночь (публикацию) мы прожили целую жизнь...}. Спасибо,
Анатолий! @igg{fbicon.heart.red}

\begin{itemize} % {
\iusr{Анатолий Золотушкин}
\textbf{Светлана Манилова} спасибо, Света
\end{itemize} % }

\iusr{Вероника Васючкова}
Мне 41 год.
Но я с детства обожаю фильмы Эльдара Рязанова. Это бесценный опыт:доверять людям, учиться любить и совершать поступки.
Это вкус моды и стиля, это всегда праздничное настроение.
Я когда-нибудь покажу один из фильмов своему будущему ребенку, уверена что в сердце у него откликнуться те же эмоции ,что и у меня.
Хоть и современные, продвинутые @igg{fbicon.wink} 
А ко ли так, будем устраивать совместные киносеансы.

\iusr{Sasha Jampolsky}

Не для мене, тільки коли втікнувши з уроків та побачивши \enquote{В джазе только
девушки} Some like it hot на англійській мові я полюбив мистецтво кіно.

А всі ці карнавали та новогодні ночи так і залишились для мене м'яким совком..

Повинно я західник..

\begin{itemize} % {
\iusr{Natasha Levitskaya}
\textbf{Sasha Jampolsky}

Вы знаете, господин захiдник, иногда лучше \enquote{пройти мимо}... без \enquote{цi карнавали и
совка}! Или это \enquote{моё мнение имею право} без этого никак???

\begin{itemize} % {
\iusr{Sasha Jampolsky}
\textbf{Natasha Levitskaya} це ваша воля панна модератор

\iusr{Natasha Levitskaya}
\textbf{Sasha Jampolsky}

Я это говорю не с позиции модератор, а с позиции человеческой. Спасибо за
понимание, если поняли, о чем я.

\iusr{Vika Sevostinova}
\textbf{Natasha Levitskaya} то пройдіть мимо. Чи тут існує цензура?

\iusr{Natasha Levitskaya}
\textbf{Vika Sevostinova}
Я написала \enquote{цензура}? Я о \enquote{внутренней} цензуре человека...
Если вам это понятно.

\iusr{Vika Sevostinova}
\textbf{Natasha Levitskaya} І так, я в цьому місті народилась і все життя прожила) на відміну від вас))
\iusr{Inna Leonid}
\textbf{Vika Sevostinova}
Те що ви народилися в Києві, тобто сам факт народження в Києві, не робить вас кращою людиною і більшою патріоткою Києва.

\iusr{Vika Sevostinova}
\textbf{Inna Leonid} тут мова не про це @igg{fbicon.man.facepalming} 

\iusr{Natasha Levitskaya}
\textbf{Vika Sevostinova}
Вы знаете, хамство- это момент воспитания и не имеет значения, где вы родились. Это о вас!
Ваш комментарий в мой адрес удалила не я, админ!

\iusr{Светлана Манилова}

\textbf{Vika}, прочитайте, пожалуйста, приветственный пост нашего администратора, в
котором написано (цитирую): \enquote{Для меня нет принципиальной разницы в
происхождении, на мой взгляд нас всех объединяет любовь к этому прекрасному
городу.} Ваш тон общения неуместен в группе. И один из Ваших комментариев был
удален, как нарушающий правила группы, с которыми Вы тоже можете ознакомиться.

\iusr{Vika Sevostinova}
\textbf{Natasha Levitskaya} прочитайте ще раз свій попередній коментар. Це і є хамство. Повторюю. Тут кожен має право написати що йому подобається а що ні. Ви написали, щоб хтось пройшов мимо, то й скористайтесь своєю порадою.

\iusr{Vika Sevostinova}
\textbf{Светлана Манилова} не погоджуюсь з тим, що я щось порушила.

\iusr{Natasha Levitskaya}
\textbf{Vika Sevostinova}

\enquote{Пройти мимо} взято в кавычки, это не хамство, а пройти мимо публикации и
читать то, что интересно вам. А вы из этой фразы перешли на личности. По
видимому на правах \enquote{коренной киевлянки}. На этом нашу коммуникацию закончим. Не
утруждайте себя комментариями в мой адрес.

\end{itemize} % }

\iusr{Михайлина Голуб}
\textbf{Sasha Jampolsky} 

Пробачте, пане, але ви не західник, а вибачте за слово - сноб та невіглас. Я не
більш за інших полюбляю совок, але яке відношення має справжнє мистецтво до
совка? Так, Голівуд створив деякі чудові фільми, але більшість його продукції,
вибачте знов за слова, відверто огидні, розраховані на американського жлоба,
який ані на йоту не кращий за нашого, \enquote{рідного}. І читати ваші одкровення
сумно. Для мене снобізм - одна з наймерзенніших якостей.

\begin{itemize} % {
\iusr{Sasha Jampolsky}
\textbf{Михайлина Голуб} так,я був 14річним снобом і по-перше побачив на екрані вільних людей у яких небуло страху перед ідеологічними керівниками
Про все інше згідний

\iusr{Vika Sevostinova}
\textbf{Михайлина Голуб}  @igg{fbicon.man.facepalming} 
\end{itemize} % }

\iusr{Алла Вайнерман}
\textbf{Sasha Jampolsky}. Культура не определяется понятием западник или
восточник. Она либо есть, либо нету.

\begin{itemize} % {
\iusr{Sasha Jampolsky}
\textbf{Алла Вайнерман} 

шановна вы не знаєте історію культури, так на приклад в Імперії під кінець 19
віку були славянофіли ( східники) та західники які почитали західні
цівілізаційні ідеї.

\iusr{Алла Вайнерман}
\textbf{Sasha Jampolsky}. У человека должна быть внутренняя культура.

\iusr{Sasha Jampolsky}
\textbf{Алла Вайнерман} що це \enquote{внутренняя культура}? В кожному поколінні, в кожних цивілізаціях були свої культурні цінності

\iusr{Алла Вайнерман}
\textbf{Sasha Jampolsky} Жаль, что вы не понимаете, что такое внутренняя культура. В первую очередь, это воспитание и чувство такта.
\end{itemize} % }

\iusr{Oksana Shevchuk}
\textbf{Sasha Jampolsky}

Мой любимый! До сих пор! Для меня это было \enquote{открытием Америки}. В 90х пыталась
взять этот фильм на прокат * уже в Америке) а меня никто не понимал... я
пыталась переводить на английский русское название...


\iusr{Sasha Jampolsky}
\textbf{Oksana Shevchuk} you are welcome

\iusr{Елена Гонж}
\textbf{Sasha Jampolsky}

Да, нет. Вы просто человек другого времени. Это нормально. Каждому- своё. И у
каждого \enquote{своё} время. Сужу по своим детям.

\end{itemize} % }

\iusr{Светлана Недайбида-Бучко}
Спасибо за ожившие воспоминания

\iusr{Татьяна Петрюк}

Для меня Рязанов - великий режиссёр! Ах, какая трогательная его поэзия!

\ifcmt
  ig https://scontent-mxp1-1.xx.fbcdn.net/v/t39.30808-6/265324491_2540515996080346_2436994615096041298_n.jpg?_nc_cat=109&ccb=1-5&_nc_sid=dbeb18&_nc_ohc=LDLiko4dHU8AX9KojbJ&_nc_ht=scontent-mxp1-1.xx&oh=de3ef7b1131f773ae8e27db0ceac5c22&oe=61B82A67
  @width 0.4
\fi

\begin{itemize} % {
\iusr{Chyzhyshyn Oksana}
\textbf{Tatyana Petryuk}
Ах, які романси на його вірші в фільмі Жорстокий романс

\iusr{Татьяна Петрюк}
\textbf{Chyzhyshyn Oksana} позвольте поинтересоваться чем вызван гневный смайлик ?!

\iusr{Chyzhyshyn Oksana}
\textbf{Tatyana Petryuk}
Випадковість, при друкуванні коментаря )

\iusr{Юрий Симоненко}
\textbf{Chyzhyshyn Oksana} В фильме один романс на стихи Рязанова -
\enquote{Любовь — волшебная страна}.

\iusr{Chyzhyshyn Oksana}
\textbf{Юрий Симоненко}
Я вас тут вже зачекалась  @igg{fbicon.smile} 
Але який розкішний романс!

\iusr{Юрий Симоненко}
\textbf{Chyzhyshyn Oksana} Шикарный...
\end{itemize} % }

\iusr{Татьяна Петрюк}
«Все исчезает - люди и эпоха .искусство сохраняет только тени.
Мы отражаем ощупью, по крохам
Бегущей жизни рванные мгновенья.» - Рязанов Э.

\iusr{Сергей Хромешкин}

\ifcmt
  ig https://scontent-mxp1-1.xx.fbcdn.net/v/t39.1997-6/s168x128/93027172_222645632401274_7176243611145601024_n.png?_nc_cat=1&ccb=1-5&_nc_sid=ac3552&_nc_ohc=ExyAb5Ejq_IAX_hVGIO&_nc_ht=scontent-mxp1-1.xx&oh=160841c29cc72614f2adaa14ae8c35bd&oe=61B84BB0
  @width 0.1
\fi

\iusr{Татьяна Карпенко}
А мы с мамой в Днипро ходили на стерео кино, это было так необычно смотреть кино в очках!)

\iusr{Татьяна Иванова}
Спасибо!

\iusr{Oleg Ivanenko}
А ещё Эльдар - один из немногих россиян, осудивших агрессию против Украины в 2014.
Вечная память

\begin{itemize} % {
% -------------------------------------
\ii{fbauth.bojko_galina}
% -------------------------------------

\textbf{Oleg Ivanenko} не обольщайтесь, не осудил.

\iusr{Oleg Ivanenko}
\textbf{Галина Бойко} Сам слышал интервью. У вас такого не показывают

\iusr{Галина Бойко}
\textbf{Oleg Ivanenko} Вы про торренты слышали? А про VPN? У меня телевизора нет 10 лет. А вот Вам в лирический и прекрасный пост о хорошем режиссёре надо было обязательно вставить 5 копеек информационной войны. Ссылку на интервью дайте. Он тогда уже был тяжело болен и никаких интервью не давал. Его не стало 30.11.2015.

\iusr{Oleg Ivanenko}
\textbf{Галина Бойко} Нам болит агрессия России против нас. А что кому болит, тот про то и говорит (рус.посл.)

\iusr{Галина Бойко}
\textbf{Oleg Ivanenko} это Вам в посте памяти Эльдара Рязанова болит? Вам стоит выйти из группы - здесь очень многие не разделяют Вашу точку зрения. И в правилах группы заявлено - никакой политики. Я Вас не баню и не жалуюсь на коммент админу исключительно по д... Ещё

\iusr{Oleg Ivanenko}
\textbf{Галина Бойко} Посмотрите на число лайков и дизллайков на мой коммент и делайте вывод

\iusr{Галина Бойко}
\textbf{Oleg Ivanenko} посмотрите на количество участников в группе

\iusr{Галина Бойко}
\textbf{Oleg Ivanenko} или хотя бы на количество отметивших сам пост.

\iusr{Галина Бойко}
26 человек, которые верят в агрессию. Впечатляет.

\iusr{Mikhail Yevtushenko}
\textbf{Галина Бойко} Галина, чтобы понять, какую точку зрения не разделяют, на Ваш взгляд, «очень многие»?

\iusr{Галина Бойко}
\textbf{Mikhail Yevtushenko} выхожу из бесмысленного пикирования. Надоело. Хороший был пост. Светлые воспоминания. Хочу остаться с этим, а не с жертвами информационной войны телевизоров.

\iusr{Игорь Бойко}
\textbf{Oleg Ivanenko} только вот Украина не осудила. Не было суда.
\end{itemize} % }

\iusr{Юрий Симоненко}

Любимейший режиссёр. Давно собираю его книги, а теперь и о нём. Светлая память...

Памяти Эльдара Рязанова.

Сергей Алешин 2.

\begin{multicols}{2}
\obeycr
О, что за режиссер явил нам чудо;
Веселый новогодний карнавал!
Как будто он пришел из ниоткуда,
Улыбкой и иронией спасал.
\smallskip
Отсрочек нет. Ушел от нас навеки,
Наверно, воздалось нам за грехи,
Оставил он не денежные чеки,
А фильмы, песни, музыку, стихи.
\smallskip
Себя, как в зеркале в них узнавали,
И далеко еще была гроза,
От глупостей и скуки изнывали,
Но Деточкина помнили глаза.
\smallskip
Там двое как могли любовь спасали,
А на часах двенадцать без пяти.
Огромная страна, как на вокзале,
Жила, ждала на запасном пути.
\smallskip
Небес обетованных не достигли,
Еще не нарастили сильных крыл.
Сердца смягчал. Надолго ли, на миг ли?
Всех дуралеев искренне любил.
\smallskip
Стояли в очереди Нади, Жени,
Приехал попрощаться Ипполит,
России не найти тебе замены,
Исчерпан на таких людей кредит.
\smallskip
Но будет длиться жизнь на белом свете,
Она сумеет злобу превозмочь,
Мохнатый шмель присядет на соцветье,
С любимым вновь уйдет цыганка в ночь.
\smallskip
Оставив миру каплю милосердья,
Уже оттуда их благослови,
Ты тем обрел величье и бессмертье,
Что звал в волшебную страну любви.
\restorecr
\end{multicols}

\ifcmt
  ig https://scontent-mxp1-1.xx.fbcdn.net/v/t39.30808-6/265244779_990731911481692_3411957019648074728_n.jpg?_nc_cat=104&ccb=1-5&_nc_sid=dbeb18&_nc_ohc=BJCI5Q4E5xwAX_yqY1S&_nc_ht=scontent-mxp1-1.xx&oh=6c15ca578c4a2cdd544dbb5d97c42def&oe=61B85FC9
  @width 0.4
\fi

\begin{itemize} % {
\iusr{Лідія Вознюк}
\textbf{Юрий Симоненко} замечательный комментарий, сколько чувств и настроения в стихах о великом Рязанове. Спасибо.

\iusr{Юрий Симоненко}
\textbf{Лідія Вознюк} Ну тут в первую очередь спасибо автору стихов. Да, гениальный режиссёр. Подобных фильмов мы уде не увидим. Но с будем с удовольствием пересматривать его фильмы...
\end{itemize} % }

\iusr{Irena Visochan}
\textbf{Анатолий Золотушкин} Спасибо, пролетела наша молодость вместе с любимыми фильмами Э Рязанова Спасибо тебе огромное, порадовал. @igg{fbicon.hands.applause.yellow} 

\iusr{Анатолий Золотушкин}
\textbf{Irena Visochan} спасибо тебе за отзыв

\iusr{Татьяна Петрюк}
Фильм «гараж « могу смотреть беспрерывно... актуален во все времена ...

\iusr{Микола Веселий}

Да, поход в кино тогда был всегда праздник ! А кинотеатр \enquote{Дніпро} лично пойду
защищать, если на его месте опять ТРЦ надумают строить. Бывает, советские
постройки не очень жалко и можно снести ради чего-то лучшего. Например,
кинотеатр \enquote{Экран} можно обсуждать. Но \enquote{Дніпро} для меня без вариантов

\begin{itemize} % {
\iusr{Chyzhyshyn Oksana}
\textbf{Микола Веселий}
А що, його приміщення ще існує ?

\begin{itemize} % {
\iusr{Irena Visochan}
\textbf{Chyzhyshyn Oksana} Там же кукольный театр?Очень красивое место,

\iusr{Микола Веселий}
\textbf{Chyzhyshyn Oksana} Поки існує. Але Деміївський ринок також існував до цієї зими. Все не було мені часу навіть спитати в продавців, що його знесуть. Закупився і побіг - таке життя, в шаленому темпі. А так би захищав до останнього

\iusr{Микола Веселий}
\textbf{Irena Visochan} Очень красивое

\iusr{Chyzhyshyn Oksana}
\textbf{Микола Веселий}
А де він находиться зараз цей кінотеатр Дніпро ?
Є єдина там стара будівля наверху там, за Ляльковим театром зліва.
Сказали що там колись був Палац піонерів, а потім взимку танці.
Саме перед ним стоїть ота скульптура велика Жаба.

\iusr{Микола Веселий}
\textbf{Chyzhyshyn Oksana} Там він один. Спитаєте літніх людей в парку де бувший стереокінотеатр - всі скажуть

\iusr{Chyzhyshyn Oksana}
\textbf{Микола Веселий}
Кінотеатр Дніпро давно знесли і на його місці Ляльковий театр

\iusr{Микола Веселий}
\textbf{Chyzhyshyn Oksana} Не знесли, а зберегли, нехай і лише незначну його частину. Все ж краще ніж черговий ТРЦ
\end{itemize} % }

\iusr{Marina Lukyanov}
\textbf{Микола Веселий} , я запуталась  @igg{fbicon.face.screaming.in.fear} 
А как же назывался летний кинотеатр на Владимирской горке, кот срезали и музей Ленина построили, а потом он стал Украинским домом ?
Где же этот кинотеатр Днiпро ?

\begin{itemize} % {
\iusr{Irena Visochan}
\textbf{Marina Lukyanov} Стоишь лицом к стадиону Динамо-слева на склоне возвышается к-т Днепр,

\iusr{Микола Веселий}
\textbf{Marina Lukyanov} Спросите бывший стереокинотеатр \enquote{Днепр} у бабушек и дедушек, что прогуливаются в парке у Моста любви. Все покажут и Мост и кинотеатр. Библиотеку КПСС могут и не показать - они туда могли не ходить)

\iusr{Marina Lukyanov}
\textbf{Микола Веселий} ,  @igg{fbicon.face.tears.of.joy}{repeat=3} 
А это что ? Я помню стерео кинотеатр...

\ifcmt
  ig https://scontent-mxp1-1.xx.fbcdn.net/v/t39.30808-6/264439639_1593901374296976_5332618253155403827_n.jpg?_nc_cat=103&ccb=1-5&_nc_sid=dbeb18&_nc_ohc=HXe0LM2uPQIAX8_bCxI&_nc_ht=scontent-mxp1-1.xx&oh=bcede65c338ec67e00bda0229db1c969&oe=61B81335
  @width 0.4
\fi

\iusr{Микола Веселий}
\textbf{Marina Lukyanov} Ну вот уже влезли комерсы в нашу память))). Ладно, как-нибудь за ручку Вас провожу. Бабушки с дедушками тоже могут быть понаехавшими и не знать)))

\iusr{Irena Visochan}
\textbf{Marina Lukyanov} Не помню и я .Недавно был чем-то пост по поводу ресторана ,,Ривьера,,,где упоминали и о Летнем кинотеатре на Владимирской горке.

\iusr{Marina Lukyanov}
\textbf{Микола Веселий} , я сама уже бабушка, в феврале 32 годика, как я покинула Киев. Вот и не помню )

\iusr{Микола Веселий}
\textbf{Marina Lukyanov} 

Тогда ясно. Ну, приедете, жив буду - покажу. А так - к бабушкам местным, они
знают скорее всего, там и гуляют у кинотеатра))). Дедушек так мало, что они
роли не играют. Редко доживают у нас мужчины до дедушек. Леня-космос даже к ним
не обращался. Только к бабушкам. Только не стали бабушки путешествовать по миру
после его правления, как он сам. Путешествуют по базару, где купить дешевле
помидор продавленный. А где-то их и этой радости лишили, например в
Голосеевском районе снесли Демеевский рынок (


\iusr{Irena Visochan}
\textbf{Marina Lukyanov} Да, Марина, это и есть Стереокино, а позже Днепр.

\iusr{Marina Lukyanov}
\textbf{Микола Веселий} , печально ( ...
Живите, пожалуйста, до 120-ти .
Встретимся  @igg{fbicon.thumb.up.yellow} 
Я буду в розовых очках @igg{fbicon.face.smiling.sunglasses} , а не в тех, что выдавали в стерео-кино...

\iusr{Микола Веселий}
\textbf{Marina Lukyanov} Ну теперь придется жить. Есть стимул)

\iusr{Татьяна Кирилюк-Омельчук}
\textbf{Marina Lukyanov} Это напротив

\end{itemize} % }

\iusr{Клим Форманчук}
\textbf{Микола Веселий} , 

никуда по \enquote{Днепру} выходить не надо. На том месте сейчас прекрасное здание
Кукольного театра. Не возбуждайтесь так, а то \enquote{отбитый} вспомнит и очередную
высотку там замутит.

\begin{itemize} % {
\iusr{Микола Веселий}
\textbf{Klim Formanchyk} В районе Амурской площади тоже неплохо отреставрировали старый советский кинотеатр частично в другом стиле. Могут ведь, когда захотят. А все, что хотят снести - решать не подковерно, а на обсуждение, и прежде всего соседей, которые живут рядом ! Промзон еще хватает для ТРЦ и бизнес-центров, туда им и дорога

\iusr{Клим Форманчук}
\textbf{Микола Веселий} , а кто нас спрашивал когда из прекрасного, доступного к-ра \enquote{Нивки} сделали дорогой ТРЦ, с билетами на импортную муть по 180 гр. ?! А надо был лишь косметический ремонт.

\iusr{Микола Веселий}
\textbf{Klim Formanchyk} 

Да, все втихаря. Говорят на Демеевском висели приклееные продавцами объявления
по рынку. Что они приглашают на собрание по поводу сноса. Ну, а тут бежишь, ни
на что не смотришь. Я бы сходил, есть юристы. Были объекты, что и отстояли. А
на то собрание, как мне говорили никто даже из депутатов не приехал, тем более
Виталик. Нивки, да тоже очень жаль. Особый стиль, это не похожий на обычную
коробку к-р \enquote{Экран}


\iusr{Клим Форманчук}
\textbf{Микола Веселий} , а Виталику всё по@уй где бабками не пахнет.

\ifcmt
  ig https://scontent-mxp1-1.xx.fbcdn.net/v/t39.1997-6/s480x480/104990052_953857721743252_4487169675628282415_n.png?_nc_cat=1&ccb=1-5&_nc_sid=0572db&_nc_ohc=hzkoUH7H4ZoAX8tFVYf&_nc_ht=scontent-mxp1-1.xx&oh=00_AT_rzrVAnvNgep64K42V-yf1FJ26clTTxu9LjrxjC9Eubw&oe=61B8EC28
  @width 0.2
\fi

\iusr{Микола Веселий}
\textbf{Klim Formanchyk} 

Ну, в Голосеевском районе уже определились за кого голосовать. Это не снос
старинного здания, людям важнее сейчас на еде экономить. Да и из других районов
туда ездили. У нас в стране рейтинг имеет значение для избрания. Жадность
фраера погубит

\end{itemize} % }


\end{itemize} % }

\iusr{Надія Коміссарова}

Це була мозаїка неймовірної краси, де вона тепер, казкова жар -птиця! На цьому
фото її Не дуже видно, на жаль...

\iusr{Клим Форманчук}

Спасибо за память ! Совсем вылетело с головы, что там, где сейчас прекрасный
Кукольный с красивым парком был когда-то к-р \enquote{Дніпро}. А ведь и я когда-то в
детстве, в конце 50-х там был с мамой и папой ! А потом в парке на склонах
Днепра рассекал на педальной машинке. А подружка на коне педальном ! Помню там
сцену \enquote{ракушку} со скамейками для зрителей. Помню играл оркестр. Я много чего
помню ... Территория вокруг Кукольного не единственное ли место в Киеве куда не
добралась строительная мафия. Пусть и будет так !


\iusr{Anneta Kobets}

Аж слезы наворачиваются только от одних воспоминаний. Какая быстротечная
жизнь!!! Только недавно бегала по послевоенному родному Киеву, а уже закат
опустился и, к сожалению, больше не пройдусь по любимому, родному,
незабываемому городу...

\ifcmt
  ig https://i2.paste.pics/6538f55c07e4c5ddf14c222acf0bdc49.png
  @width 0.2
\fi

\begin{itemize} % {
\iusr{Клим Форманчук}
\textbf{Anneta Kobets} , а чего так печально ! Ничего нет такого, чего человек не может попробовать изменить. Надо верить и действовать ! Уверен, всё у Вас получится.

\begin{itemize} % {
\iusr{Anneta Kobets}
\textbf{Клим Форманчук} От всего сердца благодарю за Вашу уверенность, но просто уже некогда действовать:мало, что я предала 500летие корнями а Киеве и живу (доживаю) в Минске, куда вышла замуж дочь, а меня принесли к ней обстоятельства и собыытия

\iusr{Клим Форманчук}
\textbf{Anneta Kobets} , не отчаивайтесь,- Минск не Аргентина и даже не Израиль. Сели вечером в автобус, и утром - оп-ля \enquote{Золотые ворота}!
\end{itemize} % }

\end{itemize} % }

\iusr{Люба Потемкина}

А я в первые дни войны , без разрешения старших, пошла в Днепре. ЧТО смотрела
не помню, но , Павлином , всегда любовалась. И нынешняя постройка прекрасна, но
стену с Павлином, нужно было сохранить. Жаль сейчас мне уже не подняться на ,
эту вершину, ,СПАСИБО ЗА подробное воспоминание. И вообще то Киев прекрасен.

\begin{itemize} % {
\iusr{Oleg Ivanenko}
\textbf{Люба Потемкина} Павлина сохранили, он в фойе кукольного театра!

\iusr{Люба Потемкина}
\textbf{Oleg Ivanenko} если дождусь лета и отмены всех запретов ,из за Короновируса, обязательно, с помощью своих внуков , там побываю
\end{itemize} % }

\iusr{Leonid Dukhovny}

Очень памятный с детства кинотеатр. Особенно, когда он был \enquote{Кинопанораммой}. И
вокруг него таинственные заросли, Деревья, кусты, стойкий запах цветов - в
общем, всё, что входит в понятие Киев!

\begin{itemize} % {
\iusr{Анатолий Золотушкин}
\textbf{Leonid Dukhovny} он не был Кинопанорамой. Она была возле синагоги Бродского. Тогда кукольный театр

\iusr{Клим Форманчук}
\textbf{Анатолий Золотушкин} , он видимо имел в виду \enquote{Стерео-кино}. Ну типа панорама !

\iusr{Михайло Наместник}
\textbf{Leonid Dukhovny} память подводит, Леонид - до 60-х годов это был кинотеатр повторного фильма, а потом стереокино «Дніпро» При входе выдавались картонные очки для объемного видения.

\begin{itemize} % {
\iusr{Leonid Dukhovny}
\textbf{Михайло Наместник} Верно, Миша. Просто я последовательность не соблюдал. И верно, что стереокино! Sorry!

\iusr{Михайло Наместник}
\textbf{Leonid Dukhovny} без проблем - у самого иногда запаает, как клавиши в моем ноутбуке А стереокино потом переместилось в "Орбиту".
\end{itemize} % }

\iusr{Eugen V Pryhod'Co}
\textbf{Leonid Dukhovny} он не бьіл "Кінопанорама"
Кинопанорама бьіл на Шота Руставели.
ДНІПРО - СТЕРЕОСКОПИЧЕСКИЙ!!!!
И ДАВАЛИ СПЕЦИАЛЬНО-ОЧКИ

\begin{itemize} % {
\iusr{Нина Брусник}
\textbf{Eugen V Pryhod'Co} Це точно

\iusr{Alla Zgurzhnitsky}
\textbf{Eugen V Pryhod'Co} точно, был стерео. Смотрела в детстве цирк Дурова. Было такое стерео кино
\end{itemize} % }

\iusr{Лариса Чубарова}
Он был не кинопанорама, а стериокино.

\iusr{Лариса Юровецкая}
\textbf{Leonid Dukhovny} 

\enquote{Днепр} 

никогда не был кинопанорамой. Это был стереокинотеатр. Для просмотра выдавали
специальные очки, еще теплые после сан. обработки. .

\end{itemize} % }

\iusr{Михайло Наместник}

А на нижней алее собирались филателисты. Там я выменял 5 венгерских треугольных
марок с бабочками на слона республики Того  @igg{fbicon.smile}  А потом от парочки пацанов
получил по ушам и лишился и бабочек, и слона  @igg{fbicon.smile} 


\iusr{Анна Сидоренко}
Спасибо, как вы хорошо написали об Э. Рязанове, что не фильм то шедевр был...

\iusr{Ольга Грудзинская}
Как всегда, чудесно написано.

\begin{itemize} % {
\iusr{Анатолий Золотушкин}
\textbf{Ольга Грудзинская} как всегда, спасибо большое
\end{itemize} % }

\iusr{Алла Вайнерман}
Светлая память великому мастеру.

\iusr{Катерина Кравець}
Не спішіть туди !
Переглядайте фільми тут

\iusr{Лариса Кучерова}

ну вот, опять, ворвались современные апологеты в чудесный пост об ушедшей
эпохи и её кино. На этом месте чудесный кукольный театр. И фильмы Эльдара
Рязанова киевляне любят...


\iusr{Квітницька Ганна}

Я знала этот кинотеатр - как стерео...
Там выдавали очки и крутили обьемные фильмы)))

\begin{itemize} % {
\iusr{Михайло Наместник}
\textbf{Квітницька Ганна} это потому, шо молодая ты ишшо, Аня

\iusr{Квітницька Ганна}
\textbf{Михайло Наместник} ну да...

\iusr{Elena Zhuravel}
Да.. очки выдавали и пирожные там вкусные были... в буфете @igg{fbicon.pancakes} ) И походы туда были тоже с мамой)))
\end{itemize} % }

\iusr{Надія Чуй}

Заходьте на \enquote{Наші котики}, \enquote{Черкаси}, \enquote{Кіборги}, \enquote{Погані дороги},
серіал \enquote{Гвардія}, \enquote{Спіймати Кайдаша}, \enquote{століття Якова} - з чудовими саунд-треками
Христини Соловій


\iusr{Лариса Артемчук}
Павлин был из мозаики

\begin{itemize} % {
\iusr{Ольга Писанко}
\textbf{Лариса Артемчук} а он и есть! И его брат- близнец на занавесе театра! @igg{fbicon.smile} 

\iusr{Леонид Бойко}
\textbf{Лариса Артемчук} он жив

\iusr{Вова Журавский}
Потрясающей красоты павлин из мозаики!
\end{itemize} % }

\iusr{валентина серцова}
моя мама там работала киномехаником и мы смотрели фильмы бесплатно

\begin{itemize} % {
\iusr{Natalia Pigulevska}
\textbf{валентина серцова} Работа мечты по-моему, мы проходили практику в кинотеатре Киев - тоже смотрели фильмы бесплатно. Прекрасное было время  @igg{fbicon.face.relieved} 
\end{itemize} % }

\iusr{Лариса Кушниренко}
А мы там уроки прогуливали.

\iusr{Natalia Pigulevska}
А мы любили ходить в Комсомолец Украины - тоже было целое событие...

\begin{itemize} % {
\iusr{Нина Брусник}
\textbf{Natalia Pigulevska} Там мультики шли!! Особенно Ну, заяц, погоди! @igg{fbicon.face.tears.of.joy}{repeat=3} 
\end{itemize} % }

\iusr{Афанасенко Тетяна}
Мне 35, и для меня настоящий НГ - это стеклянные игрушки и карнавальная ночь.

\iusr{Валерий Стрельцов}
Жалко если просто разбили Павлина, пусть укралиб на дачу. Красота она Красота.

\begin{itemize} % {
\iusr{Анатолий Золотушкин}
\textbf{Валерий Стрельцов} он цел, говорят. Внутри
\end{itemize} % }

\iusr{Виктор Беляев}
Он был первым СТЕРЕОКИНОТЕАТРОМ в КИЕВЕ.
Даже раньше, чем \enquote{ОРБИТА} !!!!

\iusr{Татьяна Зубко Маркина}
Спасибо. Совсем забыла этот кинотеатр. Воспоминания и рассказ прекрасны

\iusr{Анатолий Золотушкин}
\textbf{Татьяна Зубко Маркина} спасибо

\iusr{Екатерина Ганжа}
Где был это кинотеатр? Или ещё есть? Не соображу никак

\begin{itemize} % {
\iusr{Анатолий Золотушкин}
\textbf{Екатерина Ганжа} сейчас там Кукольный театр

\iusr{Екатерина Ганжа}
\textbf{Анатолий Золотушкин} аааа, ясно-понятно) спасибо) я просто \enquote{понаехали}, поэтому не застала там кинотеатра)

\iusr{Tamara Bialik}
А одно время был кинотеатр повторного фильма.
\end{itemize} % }

\iusr{Аркадий Шухман}

\enquote{Если бы не статистика, мы бы не знали, как хорошо мы работаем.} И ещё десятки
актуальных цитат из его фильмов.

\begin{itemize} % {
\iusr{Татьяна Петрюк}
\textbf{Аркадий Шухман} сценарии Рязанов писал совместно с Э. Брагинским ... сложно сказать кому принадлежат крылатые фразы ....
\end{itemize} % }

\iusr{Людмила Моргуновская}
Хорошие воспоминания!

\iusr{Сергей Шевченко}
Уютный был кинотеатр... Павлин, кстати, был не нарисован. Это была мозаика... Сейчас на этом месте кукольный театр, что тоже здорово! @igg{fbicon.smile} 

\iusr{Лина Бек}
Ах  @igg{fbicon.face.wink.tongue} какие были времена и фильмы @igg{fbicon.heart.suit}

\iusr{Eduardo April}
А еще рядом с этим кинотеатром, была вилеодискотека.
Зимняя жаба!

\iusr{Vika Sevostinova}
\textbf{Eduardo April} не пам’ятаю зимову а Жаба точно була))
\end{itemize} % }
