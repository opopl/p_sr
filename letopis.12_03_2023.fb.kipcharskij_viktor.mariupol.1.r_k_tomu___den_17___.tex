%%beginhead 
 
%%file 12_03_2023.fb.kipcharskij_viktor.mariupol.1.r_k_tomu___den_17___
%%parent 12_03_2023
 
%%url https://www.facebook.com/permalink.php?story_fbid=pfbid02Y2NAdW3q4H1Y8iPQ2n3bZCmJxDKbJ3fomkGvynTim4oDKQ6H4s145cpDaxmxvqMwl&id=100006830107904
 
%%author_id kipcharskij_viktor.mariupol
%%date 12_03_2023
 
%%tags mariupol,mariupol.war,dnevnik
%%title Рік тому:  День 17 - 12.03.33. Субота
 
%%endhead 

\subsection{Рік тому:  День 17 - 12.03.33. Субота}
\label{sec:12_03_2023.fb.kipcharskij_viktor.mariupol.1.r_k_tomu___den_17___}

\Purl{https://www.facebook.com/permalink.php?story_fbid=pfbid02Y2NAdW3q4H1Y8iPQ2n3bZCmJxDKbJ3fomkGvynTim4oDKQ6H4s145cpDaxmxvqMwl&id=100006830107904}
\ifcmt
 author_begin
   author_id kipcharskij_viktor.mariupol
 author_end
\fi

Рік тому: 

День 17 - 12.03.33. Субота

Близько 4:00 щось ревло - наче продувають доменну піч (заводи стоять вже зо два
тижні). У вікні щось засвітилося - яскравий спалах, але звуку не почув. Над БК
Іскра світився невеликий, але яскравий вогник - схожий на діодний ліхтарик. Не
рухався. Лівіше ще один. Що це було?

У кімнаті 13,3 градуси.

Субота. - нормальні люди відпочивають. Ненормальні - так само, тобто добрих
новин (коридор, режим тиші і т.д.) чекати не варто .

4:35. Пішов у туалет, за звичкою "увімкнув" світло: два тижня світла нема, а
звичка тримається. Чув якесь постійне гудіння. Віддушина?

4:32. Почало гуркотіти. Поки це писав - застигли руки.

6:30. На вулиці -10.

Грів чайник, бідон, соус.

Прильоти на Шлачку чи Річне селище: якісь дебіл біля вогнища радіє: "Скоріше б
та батарея замовкла: спати заважає".

Не заважає, а захищає!

\ii{12_03_2023.fb.kipcharskij_viktor.mariupol.1.r_k_tomu___den_17___.pic.1}

Сергій і Стас взяли у Олі з другого поверху відро з мотузкою вирушили по воду
до колодязя біля магазину дорогою до гаражів. Сходили без пригод.

Зарядили домашній акумулятор. У Калині 12.8 вольта. Сашко та Люда з другого
поверху спилили тополі: вони вже пустили сік і горять дуже погано - з торців у
вогонь витікає волога. З усіх боків торохтять бензопили - пиляють дерева по
Гонди.

Бомжики (чоловік та жінка), порпаються у купі сміття навколо смітника, щось
пхають у пакети. На кравчучці - купа алюмінієвих кутків. Чоловік з-під сміття
витягнув старий ноутбук з трьохдюймовим дисководом.

Повз нас йшли люди - несли фрукти й овочі, які набрали на Оптовому, на який
напередодні скинули бомби і розбили машини: щось згоріло, щось померзло, щось
залило соляркою. Ходили пішки. Кажуть, що проспект Металургів біля мосту ' у
кількох місцях пересипано купами землі.

Невістка вивела на двір онуків: сонечко світить...

Над нами гудять літаки, повітря тремтить від вибухів, а у пісочниці грається
дитина!


Несподівано знайшов у кишені на дверцятах Калини резервний тютюн з трубкою та
запальничкою! "Ищите и обрящете!"

Біля Нептуна давали печиво та зефір: натовп, штовханина, постріли у повітря.

16:40. Нагрів макарони, соус, накип'ятив води у чайнику та бідоні. Заніс Едіку.
У нього після Людиного борщу бадьорий настрій: "Завтра піду до друга, в нього
хлібопічка, він пече хліб". На чому пече? Електрики ж нема.

Не вечеряю. Аби не так хотілося їсти - пішов до вогнища допалити люльку.

Хтось сказав, що на парковці за адміністрацією на дітей дають печиво. Взяв
свідоцтва про народження онуків і пішов. На зустріч йдуть люди, вони сказали,
що по документах не дають. Пвшов додому, зустрів Сергія, сказав, щоб він пішов
за онуком. Їм дали три пачки печива. Біля вогнища знову дивилися як безкарно
літає та бомбить літак, як стрибає відбиття у склі балкону. На Важмаш та Ілліча
прилетіло чотири ракети.

Вдома онучка пригостила печивом.

Радіоновини: чергова спроба гумконвою дістатися Маріуполя не вдалася. Завтра
попереду поїдуть попи. (Навряд чи щось дасть).

Написалося отаке:

Спочатку ми ховались у підвали.

Це на початку. Потім - перестали.

Ось фото: понад нами щось літає,

Гримить, димить, і навіть десь палає.

І серед гуркоту і вибухів навколо

Дитина у пісочку щось майструє.

Вона зросте, закінчить школу

І згодом наше місто відбудує.

А поки та дитина мирно грає,

І спокоєм дорослих наділяє.

Живи, дитино...

Та даруй надії...

%\ii{12_03_2023.fb.kipcharskij_viktor.mariupol.1.r_k_tomu___den_17___.cmt}
