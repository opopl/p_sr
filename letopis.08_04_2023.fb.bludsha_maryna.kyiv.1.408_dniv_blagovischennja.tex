%%beginhead 
 
%%file 08_04_2023.fb.bludsha_maryna.kyiv.1.408_dniv_blagovischennja
%%parent 08_04_2023
 
%%url https://www.facebook.com/maryna.bludsha/posts/pfbid0q1E6H2dHQFntZ5gMxqQHS7fcdJfwem7SkDWkheDinquk7XvXGZMMqppohM17R7mcl
 
%%author_id bludsha_maryna.kyiv
%%date 08_04_2023
 
%%tags 
%%title 408 днів - Благовіщення. Сьогодні були справді благі вісті і благенькі, від яких ковбасило весь вечір
 
%%endhead 

\subsection{408 днів - Благовіщення. Сьогодні були справді благі вісті і благенькі, від яких ковбасило весь вечір}
\label{sec:08_04_2023.fb.bludsha_maryna.kyiv.1.408_dniv_blagovischennja}

\Purl{https://www.facebook.com/maryna.bludsha/posts/pfbid0q1E6H2dHQFntZ5gMxqQHS7fcdJfwem7SkDWkheDinquk7XvXGZMMqppohM17R7mcl}
\ifcmt
 author_begin
   author_id bludsha_maryna.kyiv
 author_end
\fi

408 днів

Благовіщення. Сьогодні були справді благі вісті і благенькі, від яких ковбасило
весь вечір. Почну з благеньких.

Нова Пошта перетворила мій вечір п'ятниці на трилер з елементами детективу. Три
з половиною години в мене пішло на те, щоб розплутати цей фейспалм, який
попсував мені нерви, настрій і заблокував дві банківські картки. Якщо у вас
недавно було щось подібне, дайте мені знати - може, подамо колективний позов до
суду, щоб компенсувати моральну шкоду. Усе почалося з того, що на телефон
раптом прийшло сповіщення від Приватбанку, який сповістив, що з моєї картки
було списано нуль гривень, і прислав код авторизації картки на сайті
LOOKUPCODE. Чи треба говорити, що ані на цьому, ані на іншому сайті я не
логінилася зі своєю карткою і взагалі в той момент ішла по вулиці, милуючись
фіолетовим небом на горизонті? Я людина проста, тож коли бачу будь-які
непонятні транзакції по своїй карті, яких не робила, то відразу дзвоню в службу
підтримки банку. Співробітниця банку мене підтримала у моїх страхах і наказала
негайно блокувати картку, на якій окрім особистих коштів є ще великий кредитний
ліміт.

- Дайте хоч гроші переказати на іншу картку, - благала я, затиснувши трубку.

- Давайте прямо зараз переказуйте і відразу блокуємо, бо якщо шахраї дізнались
дані вашої картки, то спробують вивести гроші, - жінка з кол-центру була
невблаганна.

- Все, я переказала, що далі?

- Тепер треба заблокувати й другу картку про всяк випадок.

Я випадаю в осад і прошу ще трохи часу, щоб зробити переказ на свою третю
картку з іншого банку. 

- Це тільки під вашу фінансову відповідальність. Ви можете самі заблокувати
другу картку, щойно зробите переказ. 

Я дякую і прощаюсь, намагаюсь знову зайти в додаток для онлайн-банкінгу, але
він уже теж заблокований. На телефон приходить смс: "Ваші картки заблоковані.
Спроба списання коштів третіми особами. Перевипустіть картку в будь-якому
відділенні Приватбанку". На годиннику майже восьма вечора, я стою у відділенні
Нової пошти, куди непомітно для самої себе повернулася з вулиці під час
бурхливих телефонних діалогів. Ще пару хвилин розмови з працівниками Нової
пошти, у яких я намагаюсь дізнатися, кому пішов мій переказ за отриману посилку
- продавцю товара чи НП. Мені показують у чеку, що Novapay, тобто Новій пошті,
яка сама розраховується з продавцями і не передає даних картки третім особам, у
чому мене запевнила навіть співробітниця банку по телефону. Не буду вас далі
втомлювати детальним описом моїх подальших потуг переказати гроші з
заблокованої картки і підтвердженням своєї особи, скажу тільки, що скоріш за
все мої спроби зробити селфі з паспортом у руках під ліхтарем на зарядній
станції для електрокарів надовго запам'ятаються не лише мені й системі
верифікації Приватбанку.

Уже вдома я змогла заспокоїтись і знайти, де були сховані кінці цієї
шахрайської історії, що виявилася суцільним фейком. Зайшовши в додаток Нової
пошти перевірити, чи не прив'язала я бува випадково до нього свою банківську
картку, натиснула на розділ "Банківські картки" у профілі. Відразу після цього
прийшло сповіщення від банку про нульову транзакцію і код авторизації на вже
знайомому мені сайті. І так повторювалося щоразу! Але ж я не заходила в цей
розділ, коли йшла по вулиці з посилкою, просто прибирала сповіщення від додатку
НП, які посипались після отримання. І тут я додумалася зайти в ці сповіщення.
Попередження від банку приходить мені щоразу, коли я натискаю на останнє
повідомлення від Нової пошти. "Сплачуй за відправлення швидше з автосписанням.
Прив'яжи картку у додатку, обери основну та плати у відділенні чи в кур'єра без
зайвих дій". Справді без зайвих дій - Нова пошта все зробить за вас, навіть
підтверджувати нічого не треба й обирати. А заодно автоматично з'являється
культурна програма на цілий вечір.

Блага вість - у Київ знову прийшла весна. У Карпатах лежить сніг, а в столиці
після рекордної за 90 років кількості опадів знову розпогодилося. Можна пити
каву на вулиці, а не ховатися від дощу під капюшон і найближчий дах. Трава
пнеться, а маленькі звірятка роблять київські торти у вітрині посередині
магазину Рошен на кондитерській фабриці. А це означає, що принаймні в мене на
районі дилери Ромашки і гарного настрою продовжують працювати, бо на Бессарабці
Рошен чомусь закрився. Весна, а це значить, що сезон салатів з редиски, зеленої
цибулі та кропу відкритий! А ще ж є молода капуста, петрушка і огірочки. Але з
цього вже завтра салат зроблю. 

На другому фото гніздо в туї з саду моїх друзів на Черкащині. За кадром нарциси
і фіолетові квіти, схожі на півники. Уже можуть квітнути півники?

День закінчився справді благою вістю. Знайшла у вхідних повідомленнях лист від
київського програміста, який веде Літопис Війни й архівує свідчення про неї,
тут частина з них: \url{https://archive.org/details/@kyiv_chronicler} Останнім часом
він зайнявся Маріуполем, бо переживає, що про нього стали забувати. Цей лист
настільки мене вразив і зворушив, що дозволю собі зацитувати його автора, якому
невимовно вдячна за ініціативу і пропозицію реанімувати мій Путівник по
Маріуполю: "І це особливо важливо, оскільки Маріуполь, на мій погляд, потребує
не тільки чисто військового звільнення, а також культурно-духовного
відродження. Так, фізично він був знищений, але в духовному просторі він
залишився у вигляді спогадів, фото, книжок, постів на фб. Але... якщо не
займатись систематично збереженням спадщини Маріуполя, то існує, на мій погляд,
велика загроза в тому, що Маріуполь буде забутий в культурно-духовній площині
також, і це буде означати фактично духовну смерть Маріуполя. І зокрема, як одна
ознак цього процесу... я майже не бачу книжок про Маріуполь в Києві. Оббігав
купу книгарень - книжок про Маріуполь або немає зовсім, або одна-дві назви
(невеличкі книжки Сухорукової та Шишацького - про трагічні події), а про
красивий, гарний, мирний Маріуполь - зовсім немає. Тому я і займаюсь зараз
Маріуполем, записую, систематизую. Маріуполь - неймовірно гарне і красиве
місто, справжня Перлина біля Моря! - яке неодмінно знову воскресне і засяє
новими барвами! Але щоб цього досягти, треба ще дуже багато роботи
попереду...".

Бережіть себе та Україну.

Доброї нам усім ночі.

%\ii{08_04_2023.fb.bludsha_maryna.kyiv.1.408_dniv_blagovischennja.cmt}
