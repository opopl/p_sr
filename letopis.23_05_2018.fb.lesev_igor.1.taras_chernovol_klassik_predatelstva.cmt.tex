% vim: keymap=russian-jcukenwin
%%beginhead 
 
%%file 23_05_2018.fb.lesev_igor.1.taras_chernovol_klassik_predatelstva.cmt
%%parent 23_05_2018.fb.lesev_igor.1.taras_chernovol_klassik_predatelstva
 
%%url 
 
%%author_id 
%%date 
 
%%tags 
%%title 
 
%%endhead 
\subsubsection{Коментарі}

\begin{itemize} % {
\iusr{Светлана Соколова}
как бы там папу не героизировали, а яблоня от яблони...

\begin{itemize} % {
\iusr{Игорь Лесев}

ну его папа у меня не вызывает никаких эмоций, знаковый политик из 90-х,
который погиб в ДТП, будучи уже полностью политическим банкротом... а вот
Тарас... ну я выше о нем написал

\iusr{Светлана Соколова}
У меня вызывает. С его движением Рух в Днепропетровске многие головой двинулись в 90-х

\iusr{Игорь Лесев}
ну, значит определенное влияние чел таки оказывал на умы))

\iusr{Светлана Соколова}
в то время мы на них смотрели, как на клоунов. Но то было начало украинизации

\iusr{Giulia Solnishkovna}
\textbf{Светлана Соколова} абсолютная правда.

\iusr{Владимир Алексеев}
\textbf{Игорь Лесев} Приходилось иметь дело с папой. Должен сказать, что Тарасик ему и в подметки не годится.
\end{itemize} % }

\iusr{Василий Стоякин}

Да ну. То ли дело - еврооптимисты! Я где-то писал уже. что это явление
настолоько мерзкое само по себе. что даже Попрошенке начинаешь сочувствовать

\begin{itemize} % {
\iusr{Светлана Соколова}
ну, соглашусь с Игорем, Тарасик по растяжкам в позицию лидирует. Первый забортовой

\iusr{Игорь Лесев}
Василий, мы как на приеме у психоаналитиков - надо сделать признания в своих фобиях... так значит ты не веришь в еврооптимизм?)

\iusr{Василий Стоякин}
\textbf{Игорь Лесев} 

Конечно не верю. Я его регулярно наблюдаю в виде лещенко, наема и т.п. На их
фоне Тарасик выглядит даже где-то человекообразным... Носом шмыгает, опять же.


\iusr{Игорь Лесев}

Лещенко и Наем (это же разные люди?) - персонажи космополитического характера.
"Цветы жизни". Но они даже срут эстетично. Я бы даже сказал, какают. Не
розочками, но все же. А эстетский негодяй - это уже и не совсем негодяй.

\iusr{Дмитрий Коломийченко}
\textbf{Игорь Лесев} 

"Электрики" тоже были эстетами в своём деле, опять же форма от Хьюго Босс и всё
такое. Но именно их отметили юридически после войны, хотя подвиги совершали
все.

\iusr{Игорь Лесев}
\textbf{Дмитрий Коломийченко} что за "электрики"? подозреваю, не те самые ребята, которые лампочки в подъезде меняют

\iusr{Дмитрий Коломийченко}
\textbf{Игорь Лесев} 

те без кавычек лампочки меняют и форма у них из ИТК, а не от Хьюго Босс. Это
была ремарка о вреде эстетства. Очень хорошо это в своё время показал Пазолини
в одном известном фильме, который немногие смогут досмотреть до конца, слишком
много там эстетики.

\iusr{Игорь Лесев}
у Пазолини я смотрел только Евангелие от Матфея и сдается мне, его работы тяжело досматривать не только по причине жестких тем

\iusr{Дмитрий Коломийченко}
Известен он несколько другим фильмом. Теперь фон Триер эту делянку обрабатывает, ибо поколение смартфона не умеет в Историю.
\end{itemize} % }

\iusr{Оксана Приходько}
Ну не известно кем бы стал папа, дожив до наши дней. Ну а Тарасик мерзкий, омерзительный слизняк

\begin{itemize} % {
\iusr{Игорь Лесев}

Папа погиб уже не молодым, не имея ни президентского, ни партийного рейтинга.
Думаю, его ожидала бы судьба Александра Мороза... хотя как по мне, у Мороза
заслуг перед Украиной на порядок больше, он все-таки неоднократно был спикером
парламента

\iusr{Марина Прохорова}
Папа ещё и федерализацию проповедовал. Так что его могла ожидать судьба не Мороза, а Савченко ))

\iusr{Оксана Приходько}
Не, ну Мороз порядочный. А папа все таки руховец был. Так что сейчас был бы на волне

\iusr{Игорь Лесев}

я пытаюсь от идеологических маркеров и предпочтений отходить, а судить именно
по степени влиятельности на что-либо... Черновол все же влиял на умы, но он не
капитализировал это влияние в какие-то рычаги или институции. По сути, он был
веянием своей эпохи, как и группа "Руки вверх", которую тоже все знали, но
которая вряд ли что-то существенно изменила в поп-музыке

\iusr{Оксана Приходько}
Не успел просто

\iusr{Игорь Лесев}
чел на седьмом десятке лет погиб... когда он должен был "успеть"? В 80 лет?

\iusr{Оксана Приходько}
Вот поэтому я и пишу, что не успел. Был бы моложе и жив, то вот что мне кажется, что прекрасно вписался бы в местную мерзоту

\iusr{Владимир Алексеев}
\textbf{Марина Прохорова} 

Он проповедовал федерализацию, когда во главе Львовской облрады сидел. И хотел
от диктата Центра избавиться. А когда в Киев перебрался и инструменты диктата у
него в руках оказались, то идею федерализации он тут же растоптал.

\end{itemize} % }

\iusr{Марина Прохорова}

Первая пятёрка партии регионов. Рядом с Богословской. Каждый раз, видя Тарасика
Ч., вспоминаю Инну Б. и так и не могу решить, кто же из них омерзительнее.

\begin{itemize} % {
\iusr{Игорь Лесев}

Инна недавно автобиографическую книжку выпустила. Я на студии два раза ее
почитывал. Там самовлюбленная жЭсть)) Чел себя считает великой. Где они всего
этого набираются, будучи в плане полезности никем, понять не могу)

\end{itemize} % }

\iusr{Сергей Бухтияров}
Истинный галычанин!

\begin{itemize} % {
\iusr{Игорь Лесев}
глуповатое клише

\iusr{Сергей Бухтияров}
А вы сами посудите! 99\% мерзавцев - выходцы с западных регионов.

\iusr{Игорь Лесев}

во-первых, 99\% мерзавцев не считают, что они мерзавцы... эт вообще
исключительно оценочное мнение, ваше, например. А во-вторых, Вячеслав Черновол
родился в Киевской области

\iusr{Сергей Бухтияров}
Согласен! Вон у амеляна все дети в штатах рождаются. Какие они галычане?
\end{itemize} % }

\iusr{Виктор Данченко}
а мне понравилось, хорошо написано и четкие характеристики

\iusr{Галина Череповская}
Согласна с Ваме на все 1000\%

\iusr{Елена Круть}

Мне, как львовянке, Черновол старший запомнился тем, что став
мэром, сразу "прославился" в городе прихватизацией особняка. У нас здесь много об
этом писали, все были в шоке, все ожидали СЛУЖЕНИЯ, а не толкотни у
корыта... Поэтому у Тараса был неплохой пример.

\begin{itemize} % {
\iusr{Игорь Лесев}
а разве он когда-то был мэром Львова? насколько я знаю, он главой облсовета области был

\iusr{Елена Круть}
\textbf{Игорь Лесев} Вы правы, ф ошибалась... но не насчёт особняка.

\iusr{Gregory Koretskyj}

Олено, про що ти верзеш? В. Чорновіл ніколи не був Мером м. Львова!

В. Чорновіл не ставав нікуди! Його було обрано Головою Львівської Обласної
Ради, але областю все одно керував Голова Обласної Адміністрації Степан
Давимука.

Будинок, який він відремонтував для себе, був розвалюхою і мав бути знесений.
Люди давали свої особисті гроші в самі працювали на реставрації цього будинку,
так само як заплатили за його викуп. Вже відреставрований будинок був
подарований В. Чорноволу громадою Львівської області. Я був депутатом обласної
Ради тоді і все пам'ятаю. Прошу не придумувати і не писати нісенітниць про
В'ячеслава Чорновола, який загинув за Україну!

\iusr{Gregory Koretskyj}
\textbf{Elena Krut} На рахунок особняка також шановна!

\iusr{Игорь Лесев}
\textbf{Gregory Koretskyj} за детали с домом спасибо, было интересно. Но непонятна концовка вашего поста - почему человек, погибший в дорожно-транспортном происшествии, "погиб за Украину"?

\end{itemize} % }

\iusr{Ирина Кузнецова}
Для меня такой омерзительныц тип - Игорь мосейчук.

\begin{itemize} % {
\iusr{Nataliya Zlochevska}
Во всей красе и симпатии.

\ifcmt
  ig https://scontent-frx5-1.xx.fbcdn.net/v/t1.6435-9/33530513_1857815480905802_8064524177459118080_n.jpg?_nc_cat=105&ccb=1-5&_nc_sid=dbeb18&_nc_ohc=mdxGBCrDL8EAX8h8FOW&_nc_ht=scontent-frx5-1.xx&oh=e6b7bd3aea8bfc0bd947d5f71e1b0a11&oe=61B7DA3D
  @width 0.4
\fi

\end{itemize} % }

\iusr{Дмитрий Коломийченко}

В лице Тараса Ч вы описали деградацию всей нашей политической элиты. То что они
за копейку малую готовы продать себя и не раз и не два унизительно для общества
и страны.

\begin{itemize} % {
\iusr{Игорь Лесев}

ну, правды ради, все мы себя так или иначе продаем... даже поход на нелюбимую
работу - это форма продажи. Но когда после смены работы, каждый раз предыдущего
работодателя называешь говном, вот это уже вызывает известную гамму эмоций

\iusr{Дмитрий Коломийченко}
\textbf{Игорь Лесев} 

Странно, никогда не называл своих боссов говном. Хотя у них были свои
недостатки. А кто их лишен? Эту этическую проблему по-моему еще Пелевин
разбирал в Ganeration-П в диалоге про женщин с пониженной социальной
ответственностью и пиарщиками. Поэтому я с вами не соглашусь. Каждый продаёт
себя в меру своей этичности. Некоторым хорошо быть и клоуном у пи...сов.

\end{itemize} % }

\iusr{Таня Кушнир}

Выбирать самого омерзительного из общей кучи дерьма вообще не хочется.
Ковыряться ещё, калибровать? Ой смотрите, вон та какашка закрутилась в такую
забавную спиральку? Фи... Пофиг, кто гадостнее, вообще пофиг. К Тарасу не
эстетические и нравственные критерии надо применять, а психопатолога. Он болен,
это просто ходячая иллюстрация учебника по шизофрении.  @igg{fbicon.frown}  возмущаться и
негодовать если и стоит - то не по поводу его поступков или слов, а тем, что
откровенно больным людям дают слово на эфирах, приглашают комментировать и
т.д.:((( шизофренический бред добивает размягченные и так мозги масс:((

\begin{itemize} % {
\iusr{Игорь Лесев}

Вот тут не соглашусь. Русофобия - это не болезнь. Клептократия - тоже.
Приспособленчество медицина тоже не лечит. Но если все сложить вместе, выходит
Тарас Черновол, который уже через год будет советником, возможно, у Тимошенко)

\iusr{Таня Кушнир}

Ну а в чем противоречие? Шизофркники любят такие идеи. Даже поклоняются им.
Пусть в разные моменты жизни их набор будет иным - русофобия дополнится
гомоцидными настроенями или чем иным. . Это как раз и показатель. И фанатизм в
следовании таким идеям. Я вообще не заморачивалась по поводу Тарасика, пока
однажды не наткнулась на него в телеке, переключая каналы без звука. Я
наблюдала его меньше минуты даже не зная о чем он говорит. Просто как говорит,
кривит рот, фокусирует взгляд, начинает говорить очень быстро и вдруг зависает,
морща лоб, подхихикивает, почесывается. Потом включила звук и все
подтвердилось. Усугубилось так сказать. Это классическая шизофрения. Вообще
неважно о чем он говорит - русофобия там или порохофилия или ещё что. Сегодня
это - завтра будет другое. Он чистый шиз:((

\iusr{Nataliya Zlochevska}

Зачем вы Тимошенко ставите на один моральный уровень с Тарасиком?

\end{itemize} % }

\iusr{Никита Холодов}
Тарас -шлюха. Чего уж там стыдиться..

\begin{itemize} % {
\iusr{Мирослава Александровна Бердник}

Тарас пододнокм был и в юности - еще до развала Союза
\end{itemize} % }

\iusr{Олег Резник}
Отлично!

\iusr{Мирослава Александровна Бердник}


Со всем согласна, за исключением того, что Вячеслав Максимович не имел рычагов
влияния. Закулисные имел - имел хорошие ЛИЧНЫЕ отношения с Кучмой и приходил
просить за нефтегазовый бизнес Зварича, Костенко и Ко. ВОт когда отказался -
тогда и начались срачи и раскол Руха с изгнанием самого Вячеслава Максимовича и
выбрасываением его треников и тапочек из помещения Руха


\iusr{Irina Sharova}
Глазки сучьи

\iusr{Александр Алиев}
Человек - говно

\iusr{Иван Маслыган}
Сука из параши, этот черноротый.

\iusr{Евгений Верный}
Папа был не лучше, изнасиловал дочь высокопоставленного лица, и обявил что по политическим мотивам, разьве не подонок

\iusr{Лидия Гусак}
Они там сейчас в тренде, подпевалы

\iusr{Татьяна Таран}
100\%

\iusr{Ирина Тарасюк}
А Леша Гончаренко-тот еще рвотный порошек... Никогда не забуду эту гниду, в Доме Профс-ов в Одессе...

\iusr{Станислав Бочкур}
А мне нравился Черновол-старший. Потому что остальные на его фоне ну совсем не алё.

\iusr{Ирина Шиповская}

100\% правы и по поводу младшего - пренепреятнейший тип. Он-то и карьеру свою
политическую просрал, потому что иуда. И по поводу старшего правы, та же
фигня...


\end{itemize} % }
