% vim: keymap=russian-jcukenwin
%%beginhead 
 
%%file 05_09_2023.stz.news.ua.mrpl.0629.1.pro_kozacke_korinnja_mariupolja_3.pic.3
%%parent 05_09_2023.stz.news.ua.mrpl.0629.1.pro_kozacke_korinnja_mariupolja_3
 
%%url 
 
%%author_id 
%%date 
 
%%tags 
%%title 
 
%%endhead 

\ifcmt
  ig https://i2.paste.pics/4353e8b3fed757a79129345c07459df3.png
	@caption Бар'єрні (нейтральні) землі в Надазов'ї  між Російською та Османською імперіями на фрагменті карти О. Рігельмана (1768 р.) позначені зелено-жовтою лінією. Бар'єр простягався південніше зазначеної лінії, яку можна умовно провести від місця  впадіння р. Каратиш у р. Берду до гирла р. Міус. Селище Кальміуська паланка знаходилось на бар'єрній території. Карту редаговано авторами.
  @wrap center
  @width 0.9
\fi
