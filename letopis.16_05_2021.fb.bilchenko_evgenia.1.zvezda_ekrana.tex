% vim: keymap=russian-jcukenwin
%%beginhead 
 
%%file 16_05_2021.fb.bilchenko_evgenia.1.zvezda_ekrana
%%parent 16_05_2021
 
%%url https://www.facebook.com/yevzhik/posts/3896910557010692
 
%%author 
%%author_id 
%%author_url 
 
%%tags 
%%title 
 
%%endhead 
\subsection{БЖ. Звезда экрана}
\label{sec:16_05_2021.fb.bilchenko_evgenia.1.zvezda_ekrana}
\Purl{https://www.facebook.com/yevzhik/posts/3896910557010692}

Поколению наших отцов посвящается
Я редактирую прозу чела, который мечтал стать Боддхисаттвой.
Но спасти этот мир как-то не проканало.

\ifcmt
  pic https://scontent-frx5-1.xx.fbcdn.net/v/t1.6435-9/186982528_3896910523677362_8600515406355246505_n.jpg?_nc_cat=100&ccb=1-3&_nc_sid=8bfeb9&_nc_ohc=HYjAGEUSsDwAX90emRd&_nc_ht=scontent-frx5-1.xx&oh=738b77fe2ea4957f757e707da14cf2a9&oe=60C9C31A
\fi

Теперь из него выползает отнюдь не плохой писатель
Во внутренней эмиграции, как из барахла пенала
В советской школе вдруг вылезал забытый якобы дома циркуль.
Чел этот был когда-то звездой нашего телевидения,
Но, когда пространство предстало адом, а время - цирком,
Больше его не видели на этом постпразднике Veni, vidi...
Пока я была студенткой и ходила пешком под стол
Коллекционных серий русистики в библиотеке Вернадского,
Чел сей казался мне неимоверной звездой, а его "престол":

"Команда" и всё такое - почти апостолами двенадцатью.
Я называла его на "Вы", за глаза, а моя бабулечка
Смотрела все эти телешоу, с либероносным сбродом.

А потом на лужах моей Отчизны вскипели бульбашки,
И даже дебилы поняли, что кончился дождь свободы.
Теперь я говорю ему: "Ты", он же мне: "Вы", - естественно,
Не потому, что в восторге диком он от гражданской доблести
"Жени Бильченко", а потому, что женщина - это тесто,
Которое быстро портится, мужчина же тесто доброе есть.

Я постарела, а он - всё тот же мальчик восьмидесятых.
И теперь он может писать свободно про советской эпохи нежность:
Он больше не числится на ТВ, не звезда и не Боддхисаттва.
Когда кончаются берега, вступает в игру безбрежность.
Я редактирую прозу чела - неправильного, невечного,
Обокраденного ничем и всем (потому и вор никак
Не найдется). Я редактирую ушедшее человечество...
И мне хочется в этот миг не редактором быть, а дворником.
8 мая 2021 г.
