% vim: keymap=russian-jcukenwin
%%beginhead 
 
%%file 28_02_2023.stz.news.ua.donbas24.1.zhyttja_pereselenske_premjera
%%parent 28_02_2023
 
%%url https://donbas24.news/news/zittya-pereselenske-mariupolskii-teatr-pidgotuvav-novu-premjeru
 
%%author_id demidko_olga.mariupol,news.ua.donbas24
%%date 
 
%%tags 
%%title «Життя переселенське» — маріупольський театр підготував нову прем'єру
 
%%endhead 
 
\subsection{\enquote{Життя переселенське} — маріупольський театр підготував нову прем'єру}
\label{sec:28_02_2023.stz.news.ua.donbas24.1.zhyttja_pereselenske_premjera}
 
\Purl{https://donbas24.news/news/zittya-pereselenske-mariupolskii-teatr-pidgotuvav-novu-premjeru}
\ifcmt
 author_begin
   author_id demidko_olga.mariupol,news.ua.donbas24
 author_end
\fi

\begin{center}
  \em\color{blue}\bfseries\Large
  3 та 9 березня Театр авторської п'єси \enquote{Conception} покаже нову прем'єру 
\end{center}

3 та 9 березня маріупольський
\href{https://archive.org/search?query=subject\%3A"Театр+Conception+Маріуполь"}{Театр авторської
п'єси \enquote{Conception}} на сцені
Національного академічного драматичного театру ім. Лесі Українки покаже
прем'єру вистави \enquote{Життя переселенське}. Це буде ще один спектакль, автором та
режисером якого є керівник незалежного маріупольського театру авторської п'єси
Conception Олексій Гнатюк, що буде присвячений життю маріупольців після
повномасштабного вторгнення рф в Україну. 

\textbf{Читайте також:} 

\href{https://donbas24.news/news/u-kijevi-vidnovili-mariupolsku-vistavu}{%
У Києві відновили маріупольську виставу (ФОТО), Ольга Демідко, donbas24.news, 02.12.2022}

\begin{leftbar}
	\begingroup
		\bfseries
\qbem{Нерозуміння... Страх... Заціпеніння... Біль... Ненависть... Розгубленість...
Спустошення... Що відчуває людина, все минуле життя якої тепер може
вмістити в собі невелика дорожня валіза? Як можна забрати з собою все
життя?! В які валізи можна помістити пам'ять, емоції, спогади? Як
забрати з собою краєвид із вікна та улюблене місто? Як, втративши все,
не втратити себе... Життя переселенське... Яке воно?}, - йдеться в анонсі
до вистави. 
	\endgroup
\end{leftbar}

За допомогою художніх образів актори маріупольського незалежного театру
авторської п'єси Conception, які перебували в окупованому місті до квітня
місяця, розкажуть глядачеві про людей, яким вдалося виїхати із захоплених
російськими військами територій на територію, підконтрольну Україні. Вистава
маріупольського театру присвячена життєвим та й досі дуже актуальним історіям.
Зокрема, буде висвітлено декілька розповідей, які об'єднує важкий шлях з
Бердянська до евакуаційного автобусу. Це історії про труднощі, небезпеку та
непередбачені ситуації по дорозі до вільної від агресорів території, як
наприклад, досить жорстокі сцени на блокпостах. Спектакль поставлений за п'єсою
режисера Олексія Гнатюка, який розповідає про власний життєвий досвід.

\textbf{Читайте також:} 

\href{https://donbas24.news/news/ci-lyubiv-marko-kropivnickii-priazovya-do-dnya-stvorennya-ukrayinskogo-realisticnogo-teatru}{%
Чи любив Марко Кропивницький Приазов'я: до дня створення українського реалістичного театру, Ольга Демідко, donbas24.news, 27.10.2022}

\ii{28_02_2023.stz.news.ua.donbas24.1.zhyttja_pereselenske_premjera.pic.1}

\begin{leftbar}
	\begingroup
		\bfseries
\qbem{Це психодрама, дії якої розгортаються, коли ми всі виїжджали, коли
покидали Маріуполь. Це вистава не про адаптацію, а про те, як люди
виїжджали, втративши все, але при цьому зуміли не втратити себе}, —
зазначив Олексій Гнатюк. 
	\endgroup
\end{leftbar}

Спектакль є продовженням історії, яку глядачі могли побачити в іншій виставі
Театру авторської п'єси Conception \enquote{Обличчя кольору війна}. Проте, за словами
режисера, вона є складнішою та більш насиченою. У виставі задіяно 10 акторів.
Перший показ відбудеться 3 березня о 18.30 на Новій сцені Національного
академічного драматичного театру ім. Лесі Українки. Тривалість вистави — 60
хвилин.

Раніше \href{https://donbas24.news}{Донбас24} розповідав про
\href{https://donbas24.news/news/lyudi-yaki-zdivuvali-za-rik-viini-svoyim-dobrim-sercem-foto}{%
людей, які здивували за рік війни своїм добрим серцем, Яна Іванова, %
donbas24.news, 21.02.2023}.

Ще більше новин та найактуальніша інформація про Донецьку та Луганську області
в нашому \href{https://t.me/donbas24}{телеграм-каналі Донбас24}.

ФОТО: Євгена Сосновського

\ii{insert.author.demidko_olga}
