% vim: keymap=russian-jcukenwin
%%beginhead 
 
%%file 24_08_2021.fb.djubanov_oleksandr.1.nezalezhnist
%%parent 24_08_2021
 
%%url https://www.facebook.com/oleksandr.diubanov/posts/3030472077282222
 
%%author Дюбанов, Александр
%%author_id djubanov_oleksandr
%%author_url 
 
%%tags nezalezhnist,ukraina
%%title НЕЗАЛЕЖНІСТЬ
 
%%endhead 
 
\subsection{НЕЗАЛЕЖНІСТЬ}
\label{sec:24_08_2021.fb.djubanov_oleksandr.1.nezalezhnist}
 
\Purl{https://www.facebook.com/oleksandr.diubanov/posts/3030472077282222}
\ifcmt
 author_begin
   author_id djubanov_oleksandr
 author_end
\fi

НЕЗАЛЕЖНІСТЬ.

В наш час яскравих фарб, смачний страв, красивих жінок, корисних гаджетів,
вражаючих спецефектів, великих спільнот і великих грошей, тектоничних процесів,
глобалізації і всепронизуючого, неспинного потоку інформації дуже важко
залишатися незалежним.

\ifcmt
  pic https://scontent-cdt1-1.xx.fbcdn.net/v/t1.6435-9/240641329_3030472053948891_1372496752004855717_n.jpg?_nc_cat=103&ccb=1-5&_nc_sid=8bfeb9&_nc_ohc=wpOhgeFpcD8AX8KrLby&_nc_ht=scontent-cdt1-1.xx&oh=886cdb47765aee4f1a4f9754154f12d0&oe=6155CC21
  width 0.5
	fig_env wrapfigure
\fi

Так привабливо і так просто приєднатися до чогось, розчинитись. Стати одним з
чисельних винтиків вже існуючої системи. Отримувати імпульс і передавати його
далі. Відчувати стабільність і впевненість у майбутньому, сопричасність до
чогось великого. Відчути себе членом чисельної спільноти, відчувати плечі тих
хто стоїть поруч, розчинити тягар персональної відповідальності у коллективній
безвідповідальності. Позбавитися страху самотності.

Яка ціна? Підкоритись. Увійти в підпорядкування іншим. Відмовитися, або надійно
сховати певну частини того твого, що йде у протиріччя з суспільною доктриною
системи. Працювати і відпочивати в межах прийнятих норм, реагувати в межах
прийнятого світогляду. Знати своє місце і роль у системі. Знати про що говорити
а про що мовчати. Знати хто враг за будь яких обставин, а хто друг за будь-яких
обставин, кому все, а кому - закон, або нічого.

Це вибір більшості. Більшість не мислить себе у відриві від середи свого
перебування. Більшість прагне ставати часткою все більших систем, хай і ціною
узагальнення своєї індивідуальності.

Але є люди які цінять іскру унікальності. Зберегти і посилити її, розпалити з
неї огонь який зможе гріти і підтримувати тебе і оточуючих - ось життєвий
пріорітет таких людей. Хай світло і тепло від цього джерела маленьке, хай для
його розвитку і підтримання треба йти на жертви значно більші ніж на співучасть
біля тих великих вогнищ, що вже розпалені. Зато воно унікальне, зато воно своє
і не впідпорядковано нічий іншій волі, воно палає і дає тепло саме для нас,
воно палає нами і палає настільки наскільки ми і тільки ми в нього вклали і
вкладаємо. 

З днем незалежності.

