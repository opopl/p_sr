% vim: keymap=russian-jcukenwin
%%beginhead 
 
%%file 09_11_2022.stz.news.ua.donbas24.2.den_ukr_pysemnosti_movy_fakty_istoria
%%parent 09_11_2022
 
%%url https://donbas24.news/news/den-ukrayinskoyi-pisemnosti-ta-movi-cikavi-fakti-ta-istoriya-svyata
 
%%author_id demidko_olga.mariupol,news.ua.donbas24
%%date 
 
%%tags 
%%title День української писемності та мови — цікаві факти та історія свята
 
%%endhead 
 
\subsection{День української писемності та мови — цікаві факти та історія свята}
\label{sec:09_11_2022.stz.news.ua.donbas24.2.den_ukr_pysemnosti_movy_fakty_istoria}
 
\Purl{https://donbas24.news/news/den-ukrayinskoyi-pisemnosti-ta-movi-cikavi-fakti-ta-istoriya-svyata}
\ifcmt
 author_begin
   author_id demidko_olga.mariupol,news.ua.donbas24
 author_end
\fi

\ii{09_11_2022.stz.news.ua.donbas24.2.den_ukr_pysemnosti_movy_fakty_istoria.pic.front}
\begin{center}
  \em\color{blue}\bfseries\Large
9 листопада в Україні традиційно відзначають День української мови — однієї з головних цінностей нашої країни
\end{center}

Українська мова має багату історію, адже вона є однією з найдавніших у світі.
Її неодноразово забороняли і намагалися викорінити, а над її носіями вчиняли
геноцид. \href{https://donbas24.news/news/yak-svidko-pereiti-na-ukrayinsku-movu-korisni-poradi}{\emph{Наша мова}}%
\footnote{Як швидко перейти на українську мову: корисні поради, Ольга Демідко, donbas24.news, 28.07.2022, \par\url{https://donbas24.news/news/yak-svidko-pereiti-na-ukrayinsku-movu-korisni-poradi}} вистояла і наразі визнана однією з найкрасивіших мов у
світі. А в умовах повномасштабної війни День української писемності та мови має
особливе значення, адже Росія не лише вбиває наших людей, знищує міста, але й
зазіхає на головні державні цінності та символи України, які ідентифікують нас
як українців. Саме українська мова є ключовим символом, головним генератором і
генетичним кодом нації.

\textbf{Читайте також:} \href{https://donbas24.news/news/u-diyi-zyavlyatsya-diplomi-ta-atestati-koli-mozna-skoristatisya}{\emph{У Дії з'являться дипломи та атестати: коли можна скористатися}}%
\footnote{У Дії з'являться дипломи та атестати: коли можна скористатися, Яна Іванова, donbas24.news, 06.11.2022, \par%
\url{https://donbas24.news/news/u-diyi-zyavlyatsya-diplomi-ta-atestati-koli-mozna-skoristatisya}%
}

\subsubsection{Історія свята}

День української писемності та мови було встановлено указом Президента України
Леоніда Кучми від 9 листопада 1997 року \enquote{Про День української писемності та
мови} на підтримку \enquote{ініціативи громадських організацій та з урахуванням
важливої ролі української мови в консолідації українського суспільства}. Дата
була вибрана зовсім не випадково, адже 9 листопада вважається Днем вшанування
пам'яті Преподобного Нестора-літописця. За деякими дослідженнями, саме з його
праць починається українська писемність. Нестор-Літописець був ченцем
Києво-Печерського монастиря, відомим київським літописцем та письменником,
послідовником Кирила та Мефодія, які створили слов'янську писемність. Він є
автором \enquote{Повісті минулих літ}.

\ii{09_11_2022.stz.news.ua.donbas24.2.den_ukr_pysemnosti_movy_fakty_istoria.pic.1}

Носіями української мови є 40−45 мільйонів людей, більшість з яких мешкають на
території України. Поширена наша мова також у Молдові, Польщі, Румунії,
Словаччині, Казахстані, Аргентині, Бразилії, Великій Британії, Канаді, США та
інших країнах, де проживають українці. Українська мова є другою або третьою
слов'янською мовою за кількістю мовців і входить до 30 найпоширеніших мов
світу. Після початку повномасштабного вторгнення Росії в Україну кількість
українців, які почали говорити вдома українською, значно виросла. Водночас
жителі нашої країни суттєво знизили споживання російськомовного контенту,
більше 40\% опитаних взагалі перестали дивитися російські серіали та слухати
російську музику.

\textbf{Читайте також:} \href{https://donbas24.news/news/movni-normi-vstupayut-v-silu-yakix-pravil-treba-dotrimuvatisya}{\emph{Мовні норми вступають в силу — яких правил треба дотримуватися}}%
\footnote{Мовні норми вступають в силу — яких правил треба дотримуватися, Еліна Прокопчук, donbas24.news, 15.07.2022, \par%
\url{https://donbas24.news/news/movni-normi-vstupayut-v-silu-yakix-pravil-treba-dotrimuvatisya}%
}

\subsubsection{Головні традиції святкування}

Цього дня в Україні:

\begin{itemize} % {
\item пишуть Радіодиктант національної єдності, який започаткований у 2000 році;
\item покладають квіти до пам'ятника Нестору-літописцю;
\item відзначають найкращих популяризаторів українського слова;
\item заохочують видавництва, які випускають літературу українською мовою;
\item стартує Міжнародний конкурс знавців української мови імені Петра Яцика — конкурс проводиться за підтримки Міністерства освіти та науки України та Ліги українських меценатів. Щорічна кількість учасників — понад 5 млн із 20 країн світу.
\end{itemize} % }

\ii{09_11_2022.stz.news.ua.donbas24.2.den_ukr_pysemnosti_movy_fakty_istoria.pic.2}

\textbf{Читайте також:} \href{https://donbas24.news/news/skoli-visi-ta-sadocki-ukrayini-perexodyat-na-distanciiku-detali}{\emph{Школи, виші та садочки України переходять на дистанційку: деталі}}%
\footnote{Школи, виші та садочки України переходять на дистанційку: деталі, Яна Іванова, donbas24.news, 10.10.2022, \par%
\url{Школи, виші та садочки України переходять на дистанційку: деталі}%
}

\subsubsection{Цікаві факти про українську мову}

1. Українська дуже тісно пов'язана зі старослов'янською — спільною мовою
предків всіх сучасних слов'ян, так само як санскрит, є мовою, найближчою до
спільної мови перших індоєвропейців (арійців). Цікаво, що в українській мові є
багато слів, які майже ідентичні словам у санскриті: \enquote{повітря}, \enquote{кохати},
\enquote{кінь}, \enquote{дерево}, \enquote{вогонь}.

2. Санскритолог, дослідник, філолог-мовознавець, поет В. Кобилюх довів, що
українська мова сформувалася в Х-IV тисячоліттях до нашої ери. Саме тому
походження найважливіших українських слів слід шукати саме в санскриті, а не в
російській, німецькій, турецькій, грецькій та інших мовах, які виникли значно
пізніше.

3. Наприкінці XVI — на початку XIX століть в Україні використовувалася особлива
система письма (\enquote{козацький скоропис}), де написання деяких букв відрізнялися
від прийнятих у кирилиці.

4. Згідно зі словником Національної Академії Наук України сучасна українська
мова налічує близько 256 тисяч слів і включена до списку мов, які успішно
розвиваються.

5. Відповідно до видання \enquote{Короткий словник синонімів української мови}, в якому
розроблено 4279 синонімічних рядів, на сьогодні найбільшу кількість синонімів
має слово \enquote{бити} — 45 синонімів.

\ii{09_11_2022.stz.news.ua.donbas24.2.den_ukr_pysemnosti_movy_fakty_istoria.pic.3}

\textbf{Читайте також:} \href{https://donbas24.news/news/v-ukrayini-zapuskayut-google-znannya-comu-mozna-navcitisya-na-osvitnyomu-xabi}{\emph{В Україні запускають Google Знання: чому можна навчитися на освітньому хабі}}%
\footnote{В Україні запускають Google Знання: чому можна навчитися на освітньому хабі, Наталія Сорокіна, donbas24.news, 02.11.2022, \par%
\url{https://donbas24.news/news/v-ukrayini-zapuskayut-google-znannya-comu-mozna-navcitisya-na-osvitnyomu-xabi}%
}

6. Українську мову протягом різних історичних періодів називали по-різному:
про́ста, руська, русинська, козацька тощо. Історично найуживанішою назвою
української мови до середини XIX ст. була назва \enquote{руська мова}.

7. Офіційно вважається, що після видання \enquote{Енеїди} Івана Котляревського,
українська мова була прирівняна до літературної мови. Українського поета і
драматурга Івана Котляревського по праву вважають основоположником нової
української мови.

8. В українській мові й донині збереглися назви місяців з давньослов'янського
календаря — сiчень (час вирубки лісу), лютий (люті морози), березень (тут існує
кілька тлумачень: починає цвісти береза; брали березовий сік; палили березу на
вугілля), квітень (початок цвітіння берези), травень (зеленіє трава), червень
(червоніють вишні), липень (початок цвітіння липи), серпень (від слова \enquote{серп},
що вказує на час жнив), вересень (цвітіння вересу), жовтень (жовтіє листя),
листопад (опадає листя з дерев), грудень (від слова \enquote{груда} — мерзла колія на
дорозі).

9. В українській мові найбільша кількість слів починається на літеру \enquote{П}, а
найменш уживаною літерою українського алфавіту є літера \enquote{Ф}.

10. Головною особливістю української мови є те, що вона багата на зменшувальні
форми. Зменшувально-пестливу форму має, як не дивно, навіть слово \enquote{вороги} —
\enquote{вороженьки}.

Текст підготовлено за наступними джерелами: zno.ua, zn.ua, vsviti.com.ua,
glavcom.ua.

Раніше Донбас24 розповідав, про особливості \href{https://donbas24.news/news/vseukrayinskii-diktant-nacionalnoyi-jednosti-yak-dolucitisya}{\emph{всеукраїнського Радіодиктанту національної єдності}}%
\footnote{Всеукраїнський диктант національної єдності — як долучитися, Ольга Демідко, donbas24.news, 08.11.2022, \par\url{https://donbas24.news/news/vseukrayinskii-diktant-nacionalnoyi-jednosti-yak-dolucitisya}} у 2022 році.

Ще більше новин та найактуальніша інформація про Донецьку та Луганську області
в нашому телеграм-каналі Донбас24.

ФОТО: з відкритих джерел.

\ii{insert.author.demidko_olga}
%\ii{09_11_2022.stz.news.ua.donbas24.2.den_ukr_pysemnosti_movy_fakty_istoria.txt}
