% vim: keymap=russian-jcukenwin
%%beginhead 
 
%%file 07_07_2022.stz.news.ua.donbas24.1.otchenashenko_77_rokiv
%%parent 07_07_2022
 
%%url https://donbas24.news/news/7-lipnya-jedinii-narodnii-artistci-ukrayini-v-mariupoli-svitlani-otcenasenko-vipovnilosya-b-77-rokiv-foto
 
%%author_id demidko_olga.mariupol,news.ua.donbas24
%%date 
 
%%tags 
%%title 7 липня єдиній народній артистці України в Маріуполі Світлані Отченашенко виповнилося б 77 років... (ФОТО)
 
%%endhead 
 
\subsection{7 липня єдиній народній артистці України в Маріуполі Світлані Отченашенко виповнилося б 77 років... (ФОТО)}
\label{sec:07_07_2022.stz.news.ua.donbas24.1.otchenashenko_77_rokiv}
 
\Purl{https://donbas24.news/news/7-lipnya-jedinii-narodnii-artistci-ukrayini-v-mariupoli-svitlani-otcenasenko-vipovnilosya-b-77-rokiv-foto}
\ifcmt
 author_begin
   author_id demidko_olga.mariupol,news.ua.donbas24
 author_end
\fi

\ifcmt
  ig https://i2.paste.pics/748ebfd5e9da9b5088b925bc7215e1dd.png
  @wrap center
  @width 0.9
\fi

\begin{center}
  \em\color{blue}\bfseries\Large
3 місяці назад пішла з життя почесна громадянка Маріуполя, народна артистка
України, лауреат премії ім. Заньковецької Світлана Іванівна
Отченашенко, якій 7 липня виповнилося б 77 років
\end{center}

7 квітня внаслідок набряку легенів у 2 міській лікарні Маріуполя померла
народна артистка України Світлана Отченашенко, яка вирішила залишатися у
рідному місті, попри воєнні дії і небезпеку для життя. Видатна маріупольчанка
пройшла довгий творчий шлях, який викликає щирий подив і захоплення кожного,
хто знайомий з її біографією. 7 липня Світлана Іванівна б могла відсвяткувати
77 років, але ця жорстока і цинічна війна забирає найкращих... 

\subsubsection{Сторінки з біографії}

Народилася актриса 7 липня 1945 року в селі Ісківці Полтавської області. Батько
мав три вищі освіти, закінчив Вчительський інститут, Вищу партійну школу та
Сільськогосподарський інститут. Займав посаду голови колгоспу. Батьки
познайомились у військовому шпиталі, куди потрапив батько після поранення, і де
працювала медсестрою мати. Медичною сестрою мама актриси пройшла всю війну.
Винесла більше 80 поранених з поля бою. І до старості працювала операційною
сестрою. На малій Батьківщині Світлана Іванівна завжди говорила українською,
яку вважала рідною мовою.

\ii{07_07_2022.stz.news.ua.donbas24.1.otchenashenko_77_rokiv.pic.1.batjky_svitlany_otchenashenko}

Юна Світлана брала участь в шкільній самодіяльності. З 7 років почала приміряти
на себе образ актриси. Раз на два роки мати вивозила доньку із села до
Ленінграда, де водила в театр. Враження від театру й вистав з його комплексним
мистецьким впливом на свідомість стануть найсильнішими спогадами юності. Бачила
на сцені живих \emph{Ольхіну} та \emph{Смоктуновського} у виставі \enquote{Ідіот} у Великому
драматичному театрі, що було для дівчини дарунком долі.

Освіта майбутньої актриси розтяглась на декілька років і пройшла у різних
містах. Вступила до Всеросійського державного інституту кінематографії у
Москві, але не витримавши напруги мегаполісом, перебралась до Києва, де
закінчила студію імені Франка. Серед викладачів Світлани Отченашенко — народний
артист УРСР \emph{Покотило Михайло Федорович} та народний артист СРСР \emph{Степанков
Костянтин Петрович}. Після закінчення студії працювала в Полтавському театрі,
але як початківець грала дрібні ролі.

\ii{07_07_2022.stz.news.ua.donbas24.1.otchenashenko_77_rokiv.pic.2_3}

Коли Світлана приїхала до Маріуполя, вирішивши спробувати свої сили у місцевому
театрі, з нею трапився курйоз. Оскільки вона не знала, де знаходиться театр, їй
довелося їхати на таксі. Однак, незважаючи на невелику відстань від вокзалу до
театру, таксист вирішив показати місто вродливій дівчині і катав її 40 хвилин.
На відміну від Полтави, у Маріуполі молоду Світлану Отченашенко відразу ж ввели
в поточний репертуар театру.

\subsubsection{Творчий шлях}

Вистава за п'єсою Ю. Едліса \enquote{Крапля в морі} (1966 р.) стала першою роботою
актриси в Маріуполі. Олівцем від руки в програмку вписано: \enquote{Віра —
Отченашенко}. Після ролей Валі Анощенко (\enquote{Російські люди} К. Симонова), Ніси
(\enquote{Дурочка} Лопе де Вега) і Тані Свєтлової (\enquote{Глєб Космачов} М. Шатрова)
закріпився статус, що Світлана Отченашенко — одна з провідних актрис театру. І
як підтвердження тому — \emph{\textbf{звання заслуженої артистки України, отримане в 1972
році в 27 років}}. Звання отримала в один день зі своїм Майстром, Учителем,
головним режисером театру Олександром Кадировичем Утегановим. Вона завжди була
\enquote{утегановською} актрисою. Світлана Іванівна зажди була людиною вдячною, а ця
якість досить рідкісна в театрі. На все життя вона зберегла подяку Олександру
Кадировичу за те, що він навчив її думати на сцені, професійно підкував, сприяв
формуванню індивідуальності. Саме про нього писала вона дипломну роботу на
театрознавчому факультеті Ленінградського державного інституту театру, музики і
кінематографії (закінчила у 1984 році).

\ii{07_07_2022.stz.news.ua.donbas24.1.otchenashenko_77_rokiv.pic.4}

З кожною новою роллю зростала майстерність актриси. Особливо запам'яталася
Світлана Отченашенко в ролі Віктоші з \enquote{Казок старого Арбату} О. Арбузова — юна
дівчина, що змогла перевернути розмірене життя двох літніх чоловіків, ролі яких
виконували \emph{Борис Сабуров} і \emph{Микола Земцов}. На сцені панували такі чисті
стосунки, випромінювалося таке тепло, що ніхто з глядачів не зміг залишитися
байдужим. Світлану Отченашенко в ролі Віктоші вітав автор п'єси \emph{Олексій
Арбузов}, її органічністю захоплювався \emph{Зіновій Гердт}.

\ii{07_07_2022.stz.news.ua.donbas24.1.otchenashenko_77_rokiv.pic.5_6}

Одна з ключових ролей для Світлани Іванівни — це роль Марії Каллас
(\enquote{Майстер-клас} \enquote{Страсті по Марії}). Спочатку актриса не знала, як підійти до
цієї ролі, бо дуже хвилювалася. Разом з режисером Костянтином Володимировичем
Добруновим вирішили не \enquote{ліпити} зовнішньої схожості, а передати характер. І це
було правильним рішенням. Ця роль допомогла Світлані Іванівні висловити власні
переживання, власну біль. Вона спробувала не зіграти Марію Каллас, зірку, що
завдяки драматизму життя і долі стала легендою, а зіграти колегу, Актрису і
Жінку, яку дуже добре розуміє, оскільки й сама — актриса й жінка. 

\ii{07_07_2022.stz.news.ua.donbas24.1.otchenashenko_77_rokiv.pic.7_8}

Загалом, про зіграних персонажів народної артистки України Світлани Отченашенко
можна сказати, що всі вони \emph{сильні, неоднозначні та незабутні}. Попри те, що вона
була актрисою далекого провінційного театру, їй вдалося стати відомою в
столичних професійних колах. Світлані Іванівні неодноразово присвячував статті
відомий журнал \enquote{Театр}. Актриса, в чиєму репертуарі \emph{Аркадіна, Раневська, Аббі
Патнем, Марія Каллас} і театральний критик — в одній особі. Статті С.
Отченашенко в \enquote{Літературці}, \enquote{Радянській культурі} піднімали аж ніяк не
специфічно периферійні, а загальноцехові професійні проблеми. На рахунку С.
Отченашенко ціла низка моноспектаклів (вечори поезії програми віршів Анни
Ахматової, Марини Цвєтаєвої, літературно-музичний спектакль \enquote{Заметіль} за О.
Пушкіним і сценічна новела з вистави \enquote{Ілюзіон} за Крецом, також роль матері з
драми \enquote{Божевільні}, \enquote{Цинкові хлопчики} — про хлопців, які пішли і не
повернулися з Афганістану, \enquote{Мати} за однойменним твором Олександра Довженка та
ін.). Вечори поезії та моновистави допомогли актрисі вистояти, коли не було
ролей. Правда, вистави — це одне, наголошувала актриса, а читання віршів —
інше. 

\begin{leftbar}
	\begingroup
		\bfseries
\qbem{У виставі ти пов'язаний з партнерами, режисером, художником... Коли ж
читаєш вірші, тобі ніхто не допоможе, але ніхто і не завадить. Ось
глядач, а далі — Всесвіт, Бог, Совість. Яке щастя відчувати глядача,
бачити, як від твоїх слів розкриваються обличчя. Відчуваєш таку
свободу! Ці програми допомагають мені вдосконалюватися, знаходити себе
через спілкування з високою поезією. А ще вірші допомогли мені знайти
свого глядача}, — розповідала Світлана Іванівна.
	\endgroup
\end{leftbar}

\ii{07_07_2022.stz.news.ua.donbas24.1.otchenashenko_77_rokiv.pic.9}

Світлана Іванівна Отченашенко невтомно підкорювала маріупольців своєю
талановитою акторською грою, адже завжди їй вдавалося проживати життя своїх
героїнь.

\begin{leftbar}
	\begingroup
		\bfseries
\qbem{Суть професії в театрі — тобто професійна функція акторів — це
допомагати людям у формуванні душі. Адже, здебільшого, людська душа
сьогодні виявляється сиротою. Тілом ми займаємося: фізично розвиваємо
його часто, а ось душа... А душа у нас забута, закинута. Є душі
нерозбуджені... Акторам слід допомогати душам залишатися живими, рости,
формуватися, знаходити себе}, — говорила Світлана Іванівна.
	\endgroup
\end{leftbar}

У актриси був досвід і в кіно. Вона зіграла бабу Зою у фільмі \enquote{Зачароване
кохання} (2008 рік). 

\begin{leftbar}
	\begingroup
		\bfseries
\qbem{Кіно вимагає загибелі всерйоз... Все повинно йти зсередини. Не потрібно
стільки міміки і жестів, як в театрі, потрібно стежити за своєю грою
уважніше} — наголошувала Світлана Отченашенко. 
	\endgroup
\end{leftbar}

\ii{07_07_2022.stz.news.ua.donbas24.1.otchenashenko_77_rokiv.pic.10_11}

\subsubsection{Любов до Маріуполя}

Актриса дуже любила Маріуполь і вважала його найріднішим містом. Саме тут
відбувалися найважливіші події в її житті. Тут вона познайомилася і зі своїм
чоловіком \emph{Харабетом Юхимом Вікторовичем}, скульптором і медальєром, заслуженим
діячем мистецтв України, який підтримував у всьому кохану жінку. Тут народила і
виховувала сина. 

\ii{07_07_2022.stz.news.ua.donbas24.1.otchenashenko_77_rokiv.pic.12}
\ii{07_07_2022.stz.news.ua.donbas24.1.otchenashenko_77_rokiv.pic.13}

Завдяки енергійній діяльності актриса змогла зберегти історичну театральну
спадщину Маріуполя, надихнувши телеканал \enquote{Сигма} на створення унікального
циклу, присвяченого історії та людям Маріупольського театру. Мало кому відомо,
що актриса брала активну участь в створенні бібліотеки в маріупольському театрі
разом із завідувачкою трупи \emph{Іскаковою Неллі Миколаївною}. Світлана Іванівна дуже
любила маріупольців, підкреслювала, що вони мають виняткову театральну
культуру, дуже вдячні глядачі. Вона щиро захоплювалася менталітетом містян,
їхнім бажанням розвивати власне місто. Актриса завжди щиро переживала за своє
місто і дуже боляче переживала за те, що місто почали руйнувати... Великим ударом
для неї було знищення військами рф будівлі Донецького академічного обласного
драматичного театру (м.Маріуполь), в якому вона пропрацювала понад 50 років. 

\begin{leftbar}
	\begingroup
		\bfseries
\qbem{Дорогі мої маріупольці, всі українці, зберіжіть живою вашу душу. Не
пускайте туди ненависть і злість, адже зруйнована душа це набагато
гірше зруйнованого тіла!}, — наголошувала Світлана Отченашенко ще до
повномасштабного вторгнення росії в Україну. 
	\endgroup
\end{leftbar}

\subsubsection{Вірші Світлани Отченашенко}

\ii{07_07_2022.stz.news.ua.donbas24.1.otchenashenko_77_rokiv.pic.14}

З дитинства Отченашенко дуже любить читати. Вона неодноразово відзначала своє
особливе ставлення до поезії. Втім Світлана Іванівна і сама писала вірші, хоча
й не довгий період, оскільки вирішила, що вона не має великого таланту... Однак,
читаючи вірші почесної громадянки Маріуполя, розумієш, що талановита людина,
дійсно, талановита в усьому! Поезія актриси дозволяє зрозуміти наскільки
багатогранний внутрішній світ Отченашенко. Лише один вірш Світлани Іванівни був
опублікований в газеті \enquote{Приазовський робочий}, яким актриса залюбки поділилася
зі мною ще у 2020 році:

\begin{center}
  \em\color{blue}\bfseries

Мне это важно! Поверьте, послушайте

Вдруг на асфальте мокром,

На танцплощадке пустынной, далекой

Листья-веснушки кружатся, кружатся в вальсе осеннем,

Утром и вечером четкие такты

Очень беспомощно, очень доверчиво жмутся к асфальту.

Вы это видели? Вы это помните?

Кто на бульварах рассыпал золото,

Развешал на ветках капли громадные...

Вдруг сорвется и прямо на голову

Солнце на небе давно не видели,

Надо грустить, да кто это выдумал?

Ничто не уходит, ничто не кончается

Листьями снизу земля освещается.

Нет, не сидится в уютной комнате,

Мне хорошо, улыбаюсь прохожим.

Вы ее помните, вы ее помните?

Осень свою на весну похожую... 
  
\end{center}

\ii{07_07_2022.stz.news.ua.donbas24.1.otchenashenko_77_rokiv.pic.15}

В першу розлуку з чоловіком, коли Отченашенко поїхала на гастролі, а мобільних
телефонів тоді не було, вона написала своєму чоловікові цей вірш: 

\begin{center}
  \em\color{blue}\bfseries
Оденусь на улицу, выскочу и побегу к телефону

Краткое, очень привычное в трубку скажу: \enquote{привет}!

Что же тут необычного, но в этом мире сонном,

Нет твоего телефона, и дома нашего нет.

Мне бы привыкнуть к этому, все далеко-далече:

И счастье, и расставанье, и боль тупая в груди.

Встряхнуть головой незаметно, бодро расправить плечи

И дожидаться встречи, которая впереди...

Но не могу привыкнуть, но не хочу поверить

И если в трубке молчание, тебя во всем обвиню.

Сейчас дождевик накину, сейчас я открою двери,

Сейчас на улицу выскочу и все-таки позвоню... 
\end{center}

Всього три вірші прочитала Світлана Іванівна під час останнього інтерв'ю зі
мною. І все ж доторкнутися до яскравого світу художнього слова видатної
маріупольчанки, на мою думку, — унікальна можливість. Цим віршем, як
підкреслила актриса, закінчилася її \enquote{поетична кар'єра}:

\begin{center}
  \em\color{blue}\bfseries
Я не люблю читать свои стихи,

Строка, как призрак, память, ускользая,

Разбудит мертвый уголок души

И требует: пиши, пиши, пиши!

И начинается игра немая

С собой на смерть жестокая игра

И капля крови на конце пера,

И сердце, задыхаясь, умирая,

А тело разом оторвется вдруг и взмоет

В высоту полетом птичьим

Со мной сидящий рядом чуткий друг зевнет и скажет:

\enquote{Ничего, прилично!}

Я не люблю читать свои стихи...
\end{center}

І незважаючи на те, що Світлана Іванівна \enquote{не любит читать свои стихи},
впевнена, що вірші справжньої легенди Маріуполя обов'язково знайдуть відлуння у
серцях читачів. Важливо зазначити, що навіть в останні хвилини життя, за
словами очевидців, Отченашенко читала свої улюблені вірші...

\ii{07_07_2022.stz.news.ua.donbas24.1.otchenashenko_77_rokiv.pic.16}

Нагадаємо раніше, Донбас24 розповідав,
\href{https://donbas24.news/news/na-donbasi-znishhili-superzbroyu-rf}{як на
Донбасі знищили суперзброю рф}.

ФОТО: архів Донбас24, Лева Сандалова та особистого архіва Демідко Ольги.

\clearpage
\ii{insert.author.demidko_olga}

