% vim: keymap=russian-jcukenwin
%%beginhead 
 
%%file 27_10_2021.fb.fb_group.story_kiev_ua.1.ikony_vizantia.pic.2.cmt
%%parent 27_10_2021.fb.fb_group.story_kiev_ua.1.ikony_vizantia.pic.2
 
%%url 
 
%%author_id 
%%date 
 
%%tags 
%%title 
 
%%endhead 
\subsubsection{Коментарі}

\begin{itemize} % {
\iusr{Vladimir Aniskov}

\ifcmt
  ig https://scontent-frx5-2.xx.fbcdn.net/v/t39.1997-6/s168x128/93118771_222645645734606_1705715084438798336_n.png?_nc_cat=1&ccb=1-5&_nc_sid=ac3552&_nc_ohc=5_oh4QOUKYQAX_jXMXP&tn=lCYVFeHcTIAFcAzi&_nc_ht=scontent-frx5-2.xx&oh=a4bf9d1b917f742c38e7c35921e70992&oe=61BAE27F
  @width 0.1
\fi

\iusr{Кретов Андрей}
Спасибо, интересно!

\iusr{Oksana Parfus}

Спасибо за информацию. Обязательно ещё раз схожу и доторкнусь до Святынь.
Как-то была на Синае в монастыре Святой Екатерины. И приобрела точный список с одной из икон.

\ifcmt
  ig https://scontent-frx5-1.xx.fbcdn.net/v/t39.30808-6/247166099_1310940752679006_1828382856745992728_n.jpg?_nc_cat=111&ccb=1-5&_nc_sid=dbeb18&_nc_ohc=Owxosf8s89YAX8mdcS3&_nc_ht=scontent-frx5-1.xx&oh=404f8b135d2688232faf0b9deabe0c8c&oe=61BB2C80
  @width 0.4
\fi

\begin{itemize} % {
\iusr{Оксана Денисова}
\textbf{Oksana Parfus} Ой, я Вам по- хорошему завидую, что Вам посчастливилось побывать в этом монастыре!

\iusr{Сергей Оборин}
\textbf{Oksana Parfus} Можете объяснить, что такое \enquote{точный список}, возможно, копия? (Не разбираюсь в этом лексиконе).
\end{itemize} % }

\iusr{Людмила Старовойтенко}
Спасибо за рассказ и публикацию икон

\iusr{Оксана Денисова}
\textbf{Людмила Старовойтенко} Спасибо Вам!

\iusr{Yaroslav Shkaraputa}
таки вражають, раджу подивитись

\iusr{Svitlana Plieshch}
Огромное спасибо за рассказ. Обязательно пойду в музей. Фантастика, такие иконы рядом, а я не знала

\begin{itemize} % {
\iusr{Оксана Денисова}
\textbf{Svitlana Plieshch} Уверена, что Вам понравятся!

\iusr{Svitlana Plieshch}
Знаете, это даже не про понравится. Не верится, что такие уникальные произведения есть у нас и их можно увидеть
\end{itemize} % }

\iusr{Svitlana Poleva}
Спасибо, Дорогая Оксана! Обожаю «читать» Вас !!!

\iusr{Оксана Денисова}
\textbf{Svitlana Poleva} Спасибо Вам большое!!!

\iusr{Ольга Молодецкая}
Потрясающе, в музее множество раз была, иконы видела, но о такой уникальности( что их всего 12 в мире ) не знала...
Спасибо @igg{fbicon.face.happy.two.hands} 

\iusr{Оксана Денисова}
\textbf{Ольга Молодецкая} Спасибо Вам!

\iusr{Евгения Верченко}
Дякую дуже!

\iusr{Вержбицкая Вероника}

\ifcmt
  ig https://scontent-frt3-2.xx.fbcdn.net/v/t39.30808-6/250087046_1451967351869962_8099712689120702873_n.jpg?_nc_cat=101&ccb=1-5&_nc_sid=dbeb18&_nc_ohc=aII1liddE18AX8tm7tm&_nc_ht=scontent-frt3-2.xx&oh=00_AT_v5Dc476GfZeKL2JB-0py8NG25nxUrk3ZYszhlZyPK1A&oe=61BAEC35
  @width 0.4
\fi

\iusr{Раиса Карчевская}
Оксана!
Большое спасибо за информацию.
Надо будет сходить в музей Ханенко

\iusr{Оксана Денисова}
\textbf{Раиса Карчевская} Обязательно надо! Спасибо Вам!

\iusr{Анна Шпак}
Спасибо большое, Оксана!

\iusr{Оксана Денисова}
\textbf{Анна Шпак} Спасибо Вам!

\iusr{Олена Шелест}
Благодарю за такую ценную информацию.

\iusr{Оксана Денисова}
\textbf{Олена Шелест} Спасибо !

\iusr{Ирина Гузий}
Спасибо!

\iusr{Светлана Александренко}
Какая публикация! Спасибо, Оксана, за интересную информацию!

\iusr{Оксана Денисова}
\textbf{Светлана Александренко} Спасибо Вам!

\iusr{Ольга Кожедуб}
Благодарю, Оксаночка!

\iusr{Оксана Денисова}
\textbf{Ольга Кожедуб} Спасибо Вам!

\iusr{Андрей Чекховской}
Иконам место в храме.

\iusr{Андрей Чекховской}
Он ведь передал в дар Братскому монастырю. Как они оказались в музее ?

\begin{itemize} % {
\iusr{Оксана Денисова}
\textbf{Андрей Чекховской} Братский монастырь закрыли, иконы передали в Музей.

\iusr{Оксана Вишневська}
\textbf{Andrey Tchekhowckiy} а где гарантия, что из церкви эти реликвии не будут украдены московскими попами или подменены? Нет уж, им место в музее.
\end{itemize} % }

\iusr{Андрей Чекховской}
Это государственный музей и иконы были просто украденны при большевиках. Им место в храме.

\iusr{Оксана Денисова}
\textbf{Андрей Чекховской} Но пост же не об этом, это уже другая история, главное, что они сохранились и их можно увидеть!

\iusr{Сергей Оборин}

Благодарю за такую историю. Как же они сохранились во время войны? Ведь столько
ценностей было уничтожено, или вывезено. Вот бы, кто-нибудь рассказал, из
очевидцев событий, кто, и как сохранил эти ценнейшие иконы.

\begin{itemize} % {
\iusr{Оксана Денисова}
\textbf{Сергей Оборин} Может, у нас в группе есть люди, связанные с Музеем Ханенко, и они знают и нам расскажут.

\iusr{Сергей Оборин}

Было бы интересно послушать их. В наше время всплывает столько информации,
недоступной ранее, что диву даешься, - жизнь прожил в неведении, а только
сейчас узнаешь.

\iusr{Нина Светличная}
\textbf{Сергей Оборин} Ці ікони завжди були доступні. Тому завжди кажу - треба з музеєм знайомитись з екскурсоводом)

\iusr{Сергей Оборин}
Важливо не те, що вони доступні в наш час, а те, що їх зберегли під час революцій та переворотів, і т.ін.

\iusr{Barbara Novokhatska}
Просто фантастика

\iusr{Barbara Novokhatska}
Такие раритеты должны
претендовать на создание отдельного музея для них с мировой известностью. Но это уже совсем другая тема...

\iusr{Сергей Удовик}
\textbf{Сергей Оборин} Во время гражданской войны они сохранились чудом, там целая история, а в 1941 их кажется эвакуировали
\end{itemize} % }

\iusr{Наталья Ваховская}
Спасибо

\iusr{Ирина Кучер}
С Вашего позволения поделюсь этой информацией. Как хорошо что у нас и целы невредимы

\iusr{Оксана Денисова}
\textbf{Ирина Кучер} Спасибо!

\iusr{Петр Кузьменко}

Благодарю! Интересно и очень полезно знать это. При ближайшей возможности
обязательно увидим это чудо. @igg{fbicon.hands.shake}{repeat=3} 

\iusr{Оксана Денисова}
\textbf{Петр Кузьменко} Спасибо, рада, что понравилось!

\iusr{Таня Сидорова}

Спасибо Оксана. Получается эти иконы там все время были, а я столько раз могла их
видеть и не знать об их уникальности... Спасибо.

\begin{itemize} % {
\iusr{Оксана Денисова}
\textbf{Таня Сидорова} Тем интереснее будет их увидеть ещё раз!

\iusr{Таня Сидорова}
\textbf{Оксана Денисова} спасибо большое
\end{itemize} % }

\iusr{Света Буренок}

\ifcmt
  ig https://i2.paste.pics/511fa70d643e5cdcb780a6d062aed403.png
  @width 0.2
\fi

\iusr{Ирина Иванченко}
Как всегда у вас припасена какая- нибудь киевская ,,изюминка", очень интересно.
Спасибо вам, Оксана !


\iusr{Оксана Денисова}
\textbf{Ирина Иванченко} Спасибо Вам большое!

\iusr{Lessya Kotovskaya}
Оксана, спасибо.
Нужно ещё раз побывать в музее.
Кажется, я обратила внимание на эти иконы, но не знала об их уникальности.

\iusr{Оксана Денисова}
\textbf{Lessya Kotovskaya} Тем интереснее будет их ещё раз посмотреть !

\iusr{Ирина Маркевич}

Спасибо, как раз собираемся с ребёнком посетить музей на каникулах, будет что
ему рассказать

\iusr{Оксана Денисова}
\textbf{Ирина Маркевич} Как хорошо, что вовремя написала @igg{fbicon.grin} 

\iusr{Александр Топал}

напрасно икон сохранилось множество с 11 века а вот по поводу их истории то
Лука написал 5 икон и 3 дошли до нашего времени это факт

\iusr{Татьяна Вакуленко}
СПАСИБО

\begin{itemize} % {
\iusr{Татьяна Судиловска}
\textbf{Татьяна Вакуленко} давай сходим в музей и посмотрим, когда вернешься.

\iusr{Татьяна Вакуленко}
\textbf{Татьяна Судиловска} давай, я в Киеве
\end{itemize} % }

\iusr{Андрей Зелёный}

\ifcmt
  ig https://scontent-frx5-2.xx.fbcdn.net/v/t39.1997-6/p480x480/105941685_953860581742966_1572841152382279834_n.png?_nc_cat=1&ccb=1-5&_nc_sid=0572db&_nc_ohc=_w9z4sOXh30AX-b0lFW&_nc_ht=scontent-frx5-2.xx&oh=16a04e7db8b371352051bdb4afb8edc2&oe=61BBF1CB
  @width 0.2
\fi

\iusr{Валико Египетский}
Почуваюся ніяково, коли чую, що ікона зберігається в музеї. Святині місце в храмі.
Дякую, Оксана!

\iusr{Junghänel Tetyana}
Просто сенсация!! Спасибо..
Но действительно, иконам место в храме. Единственное, что думаю, в храме они не пережили бы революции и войны..

\iusr{Татьяна Иванова}
Вот это да! Чудо! Спасибо!

\iusr{Оксана Денисова}
\textbf{Татьяна Иванова} Спасибо!

\iusr{Sudorgina Lusia}
Здорово!

\ifcmt
  ig https://i2.paste.pics/ef6233960b3bbb9587e1eb3e3aaf2d2f.png
  @width 0.2
\fi

\iusr{Наталия Волжева}
Дякую!

\iusr{Лариса Карпачева}
Благодарю. Интересная информация

\iusr{Ирина Камнева}
Спасибо! Не знала их историю.

\iusr{Liudmila Baliasna}
Благодарю.., интересно очень!!

\iusr{Inna Shylan}

Оксана, огромное спасибо за такую ценную информацию. Я была на Синае, в
монастыре Святой Екатерины, это словами не передать, как там хорошо. Но про
такие уникальные иконы не знала. И не знала, что 4 из них в Киеве. Еще раз
Благодарю Вас за информацию.

\begin{itemize} % {
\iusr{Оксана Денисова}
\textbf{Inna Shylan} Спасибо Вам! И как здорово, что Вы были на Синае в этом монастыре, по- хорошему завидую Вам!
\end{itemize} % }

\iusr{Oleksandr Barchuk}
Какие прекрасные лица !

\iusr{Dmitriy Grunergarten}

Очень интересно в плане познания. Но что сразу бросается в глаза- жуткий
примитивизм, антропоморфность. И это после античных произведений
искусства. Регресс сумасшедший! А до кватроченто ещё, как до Луны.


\iusr{Anna Nazarko}
Здорово...

\iusr{Лайт Татит}
Очень трудно поверить, что в наше время имеем оригинал, когда целые церкви раздариваются ради политических забаганок.

\iusr{Люала Пилипейко}
Чудо, что сохранились, у дивительно.

\iusr{Владимир Лысенко}
Как произведение искусства - супер!!! Но не как предмет поклонения.

\iusr{Софья Чиркова}
Благодарю Вас! Видела эти иконы, но не подозревала об их судьбе!

\iusr{Вита Вовченко}

Поразительно то, что они до сих пор в Киеве, а не в Эрмитаже, например. Даже в
голове не укладывается. Возможно, это наши обереги?

\begin{itemize} % {
\iusr{Оксана Денисова}
\textbf{Вита Вовченко} Это прекрасные иконы и здорово: что они у нас, пусть нас охраняют!

\iusr{Olena Ivanenko}
\textbf{Вита Вовченко} Справді: хоч ці скарби у нас НЕ вкрали!

\iusr{Iryna Kuzyk}
\textbf{Olena Ivanenko} 

Так це велике щастя, що ці ікони збереглися в Україні! А не сталося так як
сталося з багатьма нашими церквами ,монастирями, старовинними рукописами і
навіть УКРАЇНСЬКОЮ мовою , якою ,, насєлєніє" сьогодні не володіє!

\iusr{Андрей Лебединец}
\textbf{Olena Ivanenko} бо це не золото скифів @igg{fbicon.face.grinning.squinting} 

\end{itemize} % }

\iusr{Наталія Гриб}
СПАСИБО!!! Очень интересно.

\iusr{Natasha Levitskaya}

Была не один раз в Музее Ханенко и видела иконы! Теперь буду знать их
уникальность... Спасибо!


\iusr{Оксана Денисова}
\textbf{Natasha Levitskaya} Спасибо Вам!

\iusr{Oiga Krotova}
Спасибо, невероятно интересно!

\iusr{Оксана Денисова}
\textbf{Oiga Krotova} Спасибо!

\iusr{Irena Visochan}
Я когда-то в юности прочитала книгу Солоухина,, Черные доски,, - очень впечатляюще, по сей день.

\iusr{Elena Kobinskaya}
Благодарю! Пойду в музей обязательно

\iusr{Елена Громозова}

Огромное спасибо за такую ценную информацию! Мы тоже были в монастыре Св.
Екатерины и много раз в нашем музее, но ничего не знали об этих сокровищах!!!

\iusr{Оксана Денисова}
\textbf{Елена Громозова} Спасибо Вам, рада, что интересно!

\iusr{Tatyana Boyko}

Благодарю от всей души Вас, Оксана за уникальные знания, которые нам даёте!
@igg{fbicon.hands.pray}  @igg{fbicon.heart.suit}

\iusr{Оксана Денисова}
\textbf{Tatyana Boyko} Спасибо Вам большое @igg{fbicon.heart.red}

\iusr{Надія Глоба}
Дякую за історію.
Знаю чим зайняти свій день  @igg{fbicon.face.smiling.eyes.smiling} 

\iusr{Наталья Толстова}

Це правда. Я коли їх побачила, не повірила, що то можливо. Тільки розташовані
вони ,на мій погляд, дуже невдало, мало хто їх там знаходить.


\iusr{Oksana Yankovich}
Спасибо!

\iusr{Inna Kontra}
Spasibo, ocheny interesno!

\iusr{Александра Осадчая}
В екскурсии по музею об этом ни слово. А жаль

\iusr{Оксана Денисова}
\textbf{Александра Осадчая} Серьезно? Я там с экскурсией не была, думала, они об этом рассказывают. Очень жаль!

\iusr{Iryna Kuzyk}
Дякуємо Оксана!

\ifcmt
  ig https://i2.paste.pics/63d09abfd044a6561dfaad58ec9c22e0.png
  @width 0.2
\fi

\iusr{Татьяна Третьяченко}
Дякую за допис!

\iusr{Алла Тихонова}
Иконы для молитвы, если для \enquote{посмотреть} - то уже не иконы.

\iusr{Ирина Иванова}
була- бачила - неймовірно!!!

\begin{itemize} % {
\iusr{Ирина Кучер}
\textbf{Ирина Иванова} заздрю по білому. Скільки була не звертала уваги. Тепер
піду тільки побачити і постояти біля них. Така енергетика !?!?
\end{itemize} % }

\iusr{Валентина Петрова}
Спасибо

\iusr{Валентина Комлева}
Дякую

\iusr{Світлана Шалуєва}
Была в этом месяце в музее, иконы видела. Спасибо за историю. @igg{fbicon.face.smiling.hearts} 

\iusr{Ludmila Kilimnik}
Дякую! Дуже цікаво!!!

\iusr{Оксана Денисова}
\textbf{Ludmila Kilimnik} Дякую!

\iusr{Олена Житар}
Спасибо Вам за историю!

\iusr{Оксана Денисова}
\textbf{Олена Житар} Спасибо Вам!

\iusr{Марина Тузова}
Спасибо, не знала об этом - очень интересно!

\iusr{Оксана Денисова}
\textbf{Марина Тузова} Спасибо Вам!

\iusr{Тома Храповицкая}

Сила и Мощь икон за столетия только приумножилась! Создавались Мастерами с
Любовью и Верой! А это очень важно при написании святых ликов! Идёт обмен,
сильный, мощный и оберегающий!!! Спасибо Вам Оксана, Вы как всегда на
вершине!!!

\end{itemize} % }
