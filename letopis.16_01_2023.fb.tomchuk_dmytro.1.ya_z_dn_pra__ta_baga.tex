%%beginhead 
 
%%file 16_01_2023.fb.tomchuk_dmytro.1.ya_z_dn_pra__ta_baga
%%parent 16_01_2023
 
%%url https://www.facebook.com/dmitriy.tomchuk/posts/pfbid033jmQrFn5yiHR6LT75t1W3AWakg5nBiVXYj7M78aSoLYaDaiPm7D2FrugrNC8wNUjl
 
%%author_id tomchuk_dmytro
%%date 16_01_2023
 
%%tags 
%%title Я з Дніпра. Та багато хто в нас з Дніпра. Але в нас немає ненависті
 
%%endhead 

\subsection{Я з Дніпра. Та багато хто в нас з Дніпра. Але в нас немає ненависті}
\label{sec:16_01_2023.fb.tomchuk_dmytro.1.ya_z_dn_pra__ta_baga}

\Purl{https://www.facebook.com/dmitriy.tomchuk/posts/pfbid033jmQrFn5yiHR6LT75t1W3AWakg5nBiVXYj7M78aSoLYaDaiPm7D2FrugrNC8wNUjl}
\ifcmt
 author_begin
   author_id tomchuk_dmytro
 author_end
\fi

Я з Дніпра. Та багато хто в нас з Дніпра. Але в нас немає ненависті.

Бо якщо б ми ненавиділи ворога, ми ненавиділи б так сильно, що в  нас тремтіли
б руки. А якщо так, жодна бомба не влучила б у свою ціль, та наша ненависть
таким чином допомагала б не нашим, а ворогові. 

Ось чому ми шукаємо балансу та рівноваги замість ненависті чи страшних
трагедій, які здатні викликати гострі відчуття. Та знаходимо. Де знаходимо? В
собі. Більш нема де. Ми - наодинці  з собою та Богом, коли на місії, та нема
більш нікого. Нам не можуть допомогти ані волонтери, ані кризові
психотерапевти, ані капелани, бо людина на самоті, коли поряд смерть. Тому
знаходимо в  собі, бо ззовні нема, я бачу це по тих, хто шукав допомогу десь у
зовнішньому, не знайшов та загинув.

Усе наше відношення до ворога давно виставлено на потрібні значення таким
чином, щоб не заважало воювати. Бо воюємо ми з любов'ю. Щоб воювати, потрібна
любов до своїх замість ненависті до ворога. Тоді все виходить. Це ізраїльський
рецепт, його колись розповів мені навіть не полковник, який пройшов на меркаві
усі ізраїльські війни,  а звичайний школяр з Єрусалима, бо цьому там навчають у
школі. У школі - яки дивні слова з минулого. Знищіти ворога можна тільки у
врівноваженому стані. Й для цього потрібна любов до тих, кого захищаєш. 

Цього не знають екзальтовані тьоті у фейсбуці та розумні мальчики на
диванчиках, з тих, що досі бігають від мобілізації. Ті захлинаються від
ненависті, бо вона не заважає їх головній діяльності - здіймати пустий гвалт ні
про що. Тому там можна ненавидити без шкоди для справи. Бо справи ніякої нема.
А в нас так не вийде. Бо нам робити треба.

В нас немає ресурсу рефлексувати. Бо якщо ми реагуватимемо, ми не зможемо
робити своє.  Мене вразив хтось, хто вже через дві години після удару по
будинку в Дніпрі на колінці наклепав душешчипательний вірш про загиблих та
вивісив його у фейсбуці. 

Знаєте, чому немає глибокомисленних творів-роздумів про трагедію від тих, хто
дійсно був там на місці? Бо вони розбирали завали, у крижаній пітьмі, а коли
вже не могли, не йшли у фб писати віршики, а  падали спати десь у теплі, щоб
прокинутися та йти розбирати завали. Бо там ще могли залишатися живі. Доречі -
це також неможливо робити, якщо рефлексувати та гостро відчувати. Неможливо,
рефлексуючи, не тільки врятувати з-під залізобетону когось живого, а навіть
витягнути з уламків понівечене дитяче тільце, ще тепле. Неможливо.

Але, маю зізнатися. Трагедія в Дніпрі мало що додала до наших поточних
моральних налаштувань.

Ми не стали ще більше ненавидіти, чому, я щойно розповів.

Ми не дали обітницю знищити ще більше ворога, бо дуже давно прийняли для себе
кількість вбитих нами: рівно стільки, до скількох ми зможемо дотягнутися, доки
будемо живі. Менше - не рентабельно. Дами у білих пальтах все питають в мене:
"Ви стільки вбили, як ви будете жити після цього?" Але по-справжньому я не
знаю, як житиму, якщо залишу в живих когось, кого міг вбити. Не через те, що я
такий людожер. А через те, що не вбитий мною сьогодні завтра буде у ВАС на
подвір'ї, ґвалтуватиме пляшкою ВАШУ дружину, виколюватиме штикножем очі та
вирізатиме свастики на спині ВАШОЇ дитини.

Ми не присяглися знайти та покарати тих, хто запустив на будинок в Дніпрі
протикорабельну Х-22 з бойовою частиною вагою в тонну та швидкістю 4 000
км/год. Тому що кожен має займатися своєю справою та бути на своєму місці - так
само, як це робили ті, хто розбирав  дві доби завали в  Дніпрі. Тому що ми
абсолютно віримо у тих, хто знайде та покарає усіх причетних, просто виконуючи
службові обов'язки. Доречі, для цього потрібно багато віри. Набагато більше,ніж
щоб увійти у Царство Небесне - там потрібна кількість з гірчичне зерня. А щоб
повірити у служби, які здійснять все, що треба у свій час, потрібно набагато
більше. Але ми віримо.

Ми не переконалися тепер вже достеменно, що на тому боці - нелюди. Бо ми бачили
це вже у перші дні війни, у максимальній HD-якості. Тому ми навіть не прагнемо,
щоб душі вбитих нами йшли прямо до пекла. Бо в них, для початку, немає ніяких
душ. Це наче стародавня історія про деструкцію голема, виліпленого з глини та
приведеного до життя злим чаклуванням. Чвак - й глина перестає жити та
повертається до матері-глини. Прах до праху, попіл до попелу, глина до глини.
Так само можна було б воювати з велетенськими тарганами чи павуками.

Але я можу розповісти, що в  нас є натомість усього цього.

Зібраність. Концентрація. Професіоналізм. Чіткість дій. Спокій всередині, майже
дзен. Розуміння, що і як робити. Тверезий розум. Руки, які НЕ тремтять. Швидка
реакція. Здатність приймати вірне рішення. Воля. Ретельність та педантизм у
виконанні роботи, та контролі виконаного. Розуміння, що окрім нас, це скінчити
буде нема кому. Зовсім трошки фаталізму, тобто розуміння руки Бога над нами.

Й жодних істерик з жодного приводу. Вибачайте, якщо шо.
