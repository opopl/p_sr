% vim: keymap=russian-jcukenwin
%%beginhead 
 
%%file 03_09_2021.fb.dikij_evgenij.dnepropetrovsk.1.arestovich_rusj_ukraina
%%parent 03_09_2021
 
%%url https://www.facebook.com/YevhenDykyi/posts/4069905943135656
 
%%author_id dikij_evgenij.dnepropetrovsk
%%date 
 
%%tags arestovich_aleksei,istoria,pereimenovanija,rusj,rusj.ukraina,ukraina
%%title Олексій Арестович має мрію. Перейменувати Україну на Русь-Україну - щось на кшталт Трансільванії-Румунії
 
%%endhead 
 
\subsection{Олексій Арестович має мрію. Перейменувати Україну на Русь-Україну - щось на кшталт Трансільванії-Румунії}
\label{sec:03_09_2021.fb.dikij_evgenij.dnepropetrovsk.1.arestovich_rusj_ukraina}
 
\Purl{https://www.facebook.com/YevhenDykyi/posts/4069905943135656}
\ifcmt
 author_begin
   author_id dikij_evgenij.dnepropetrovsk
 author_end
\fi

\textbf{Олексій Арестович} має мрію. Перейменувати Україну на РусьУкраїну - щось на
кшталт ТрансільваніїРумунії. Його мрія йде далі, дискутували раніше, але не
скопіював аби процитувати. Там зокрема йшлося про присвоєння російської мови,
відбирання в української мови її історичної назви "руська" та перейменування
російської мови на руську. З усім цим категорично не(!) згоден. 

\ifcmt
  ig https://scontent-yyz1-1.xx.fbcdn.net/v/t39.30808-6/241237385_4069905893135661_3023532154827078375_n.jpg?_nc_cat=104&_nc_rgb565=1&ccb=1-5&_nc_sid=8bfeb9&_nc_ohc=E9gSxQaIRUMAX9oVUiz&_nc_ht=scontent-yyz1-1.xx&oh=2e0f00183aae3ff2cf0d6bcfe98b1e4d&oe=614E0CBA
  @width 0.4
  %@wrap \parpic[r]
  @wrap \InsertBoxR{0}
\fi

У мене теж є мрія. Що українці повернуть собі застарілу назву "русини" та
воз'єднаються з прихильниками руху русинства. Звісно ж не без інституційного
втручання в літературну норму мови. Від повернення українській застарілої назви
"руська" до впровадження переваги для тих діалектизмів, що з 4 діалектів для
котрих характерно вживання застарілих слів і зокрема самоназв. Ті 4 діалекти є
основою для руху русинства, але більшість носіїв тих діалектів, перш за все
яким пощастило залишитися в Україні на своїх бАтьківщинах, вважають себе
українцями і відкидають рух русинства.

Проте перейменування України в цій мрії немає. Воно зайве та деструктивне.
Поняття Україна вже включає поняття Русі, але не навпаки. Олексій вочевидь знає
історію принаймні гірше за мене, якщо заперечує деструктивність варіянту
перейменування в "РусьУкраїну". Що ж до воз'єднання з рухом русинства він
погожується, тому дозволю собі спробувати знайти компроміс між нашими ідеями.

Об'єднання з двома південними племенами, що зрештою породило рух русинства,
врятувало Русь після втрати державності при створенні Речі Посполитої. Далі два
напрямки відновлення державності, монархічний(Князі Острозькі) та
республіканський(створена державою південних руських племен Військом
Запорізьким для Русі під власним протекторатом держава Гетьманщина),
підтримували одне одного, але вижив лише останній напрям. Завдяки зародженню
нації, і то спільної для русин і південних руських племен. Вже при відродженні
нацією республіки в часи УНР усі залишки монархізму працювали на республіку.
Немає Русі без південних племен, їх спадку, а з ними - це вже Україна. Я можу
підтримати в рамках компромісу розширену назву, як от "РУСЬКА Республіка
Україна" за аналогією з "РУСЬКИМ Королівством" чи "ФРАНЦУЗЬКОЮ Республікою" -
де уточнення про королівство та республіку відображає період і форму
державності.
