% vim: keymap=russian-jcukenwin
%%beginhead 
 
%%file 23_01_2022.stz.news.lnr.lug_info.2.javlenie_petra
%%parent 23_01_2022
 
%%url https://lug-info.com/news/nedel-ya-glazami-eksperta-gadyashaya-anglichanka-ushedshij-demoralizator-i-yavlenie-petra
 
%%author_id news.lnr.lug_info
%%date 
 
%%tags arestovich_aleksei,donbass,lnr,marochko_andrej.nm_lnr,nm_lnr,poroshenko_petr,ukraina,vojna
%%title НЕДЕЛЯ ГЛАЗАМИ ЭКСПЕРТА: Гадящая англичанка, ушедший деморализатор и явление Петра
 
%%endhead 
 
\subsection{НЕДЕЛЯ ГЛАЗАМИ ЭКСПЕРТА: Гадящая англичанка, ушедший деморализатор и явление Петра}
\label{sec:23_01_2022.stz.news.lnr.lug_info.2.javlenie_petra}
 
\Purl{https://lug-info.com/news/nedel-ya-glazami-eksperta-gadyashaya-anglichanka-ushedshij-demoralizator-i-yavlenie-petra}
\ifcmt
 author_begin
   author_id news.lnr.lug_info
 author_end
\fi

\begin{zznagolos}
Своим видением событий прошедшей недели, так или иначе связанных с Луганской
Народной Республикой, с ЛИЦ делится военный эксперт, общественный деятель,
подполковник запаса Народной милиции ЛНР Андрей Марочко.	
\end{zznagolos}

\ii{23_01_2022.stz.news.lnr.lug_info.2.javlenie_petra.pic.1}

\subsubsection{НА ЛИНИИ СОПРИКОСНОВЕНИЯ}

Минувшая неделя, несмотря на договоренности о дополнительных мерах контроля
действующего режима прекращения огня, для жителей Луганской Народной Республики
прошла неспокойно, обстановка на линии соприкосновения по-прежнему оставалась
стабильно напряженной.

По данным наблюдателей представительства ЛНР в СЦКК, боевики вооруженных
формирований Украины осуществили один обстрел нашей территории, применив
станковый противотанковый гранатомет и крупнокалиберный пулемет

Всего с 00:00 часов 21 июля 2019 года режим всеобъемлющего устойчивого и
бессрочного прекращения огня со стороны вооруженных формирований Украины был
нарушен 1020 раз, из них 686 раз – после вступления в силу допмер; 43 защитника
Республики погибли, 24 получили ранения. Среди мирного населения 5 человек
погибли, 38 получили ранения, повреждены 194 объекта гражданской
инфраструктуры.

\subsubsection{ГАДЯЩАЯ АНГЛИЧАНКА}

На фоне нагнетания западными СМИ обстановки вокруг мифического предстоящего
\enquote{вторжения России} на Украину Запад активизировал поставки вооружений Киеву. К
нашпиговывающему Украину вооружениями Вашингтону присоединился Лондон, который
отправил в Киев несколько самолетов с оружием. По данными СМИ, в течение недели
на Украину были доставлены около двух тысяч противотанковых комплексов NLAW
(Next Generation Light Anti-Tank Weapon). Данные переносные ПТРК ряд экспертов
уже окрестили \enquote{Javelin на минималках}, что вполне соответствует
действительности. Несмотря на меньшие, чем у американского \enquote{старшего брата}
дальность стрельбы и мощность боевой части, NLAW тем не менее может оказаться
эффективным средством уничтожения бронетехники.

Помимо поставок вооружений Великобритания также направляет на Украину
дополнительную группы военных инструкторов для обучения военнослужащих ВСУ.

Для нас, по большому счету, действия англичан были вполне предсказуемы. В
Донбассе уже давно привыкли к тому, что в нэзалежную практически на регулярной
основе приезжают в качестве инструкторов представители стран НАТО, да и военные
грузы стали обыденностью. А если учитывать ту истерию, которая нагнетается
вокруг \enquote{российского вторжения}, то было бы даже странно если бы Запад повел
себя иначе. Но есть, на мой взгляд, моменты, на которые нужно обратить
внимание.

Прежде всего, это резкая активизация Лондона, который начал продвигать
вашингтонские нарративы о том, что российские войска на территории России вдруг
стали представлять угрозу для Украины, а поставки украинским боевикам
вооружений и направление к ним инструкторов не нужно воспринимать как эскалацию
и угрозу. Возможно, России они и не угрожают, а вот для жителей Донбасса угроза
вполне реальна, ведь именно туда направится поступившее с Запада оружие. А
размещение вдоль линии соприкосновения тысяч ПТРК очевидно не будет
способствовать мирному урегулированию конфликта, так как получение все большего
количества вооружений явно будет толкать киевских боевиков к эскалации. И в
Лондоне это должны прекрасно понимать. Значит, цель британских действий -
перевести конфликт в горячую фазу. США на фоне переговоров с Россией по
гарантиям безопасности сейчас это делать напрямую невыгодно, поэтому они и
попросили британских союзников немного поработать за себя.

\subsubsection{КАРАМЕЛЬКА НА \enquote{МИРОТВОРЦЕ}}

Тяжелая утрата постигла украинскую делегацию в Контактной группе. Из рядов
волынщиков и любителей тянуть резину в долгий ящик выпало яркое и местами даже
самобытное существо - Алексей Арестович, не менее известный как Карамелька.

Видимо, трансгендер в очередной раз сменил(а) свою политическую ориентацию, в
связи с чем на одном из украинских телеканалов языком агрессора рассказал
интимные подробности про всеукраинского клоуна и по совместительству президента
Владимира Зеленского, который нанес \enquote{удар себе по яйцам}, вместо того, чтобы
открутить их олигархам. Откровения спикера украинской делегации и советника
главы Офиса президента Украины его начальство явно не оценило и рекомендовали
выйти вон идти бродить по свету с карамелькой за щекою. Более того, вскоре
Карамельку внесли в базу данных \enquote{Миротворца}, где объявили \enquote{профессиональным
провокатором}, участвовавшим в \enquote{актах гуманитарной агрессии против Украины} и
подрывавшим \enquote{обороноспособность Украины путем деморализации вооруженных сил
Украины}. До кучи обвинили в \enquote{дискредитация государственных органов власти и
управления}, хотя дискредитировать эти украинские органы, на мой взгляд, уже
давно невозможно по определению.

Заявлениям Арестовича о том, что он ушел со своих постов по \enquote{личным причинам},
никто особо не поверил. Что же или кто послужил причиной, по которой \enquote{Люся} не
дожил(а) до запланированной в конце месяца первой в этом году встречи
Контактной группы? Предположений на самом деле хоть отбавляй, и все они в той
или иной степени имеют право на существование. Возможно, он попросту всех
достал своим скудоумием. Не исключено, что мы наблюдаем бегство очередной крысы
с тонущего чумного корабля. Также одной из причин "переобувания" Арестовича
могут быть предстоящие выборы в Раду, в которых он вполне сможет поучаствовать
по квоте для радужных либералов и соросовских грантоедов.

Самым печальным для Карамельки может оказаться расклад, при котором он
действительно ушел по собственному желанию. Посудите сами. После того, что он
наговорил, будучи еще \enquote{Зеленым}, порохоботы его вряд ли примут, да и предал он
вначале Порошенко, переметнувшись к Зеленскому. От \enquote{слуг народа} тоже
отказался, ушел, громко хлопнув дверью и оскорбив всю команду. Есть, конечно,
шанс \enquote{замутить} с каким-нибудь Гордоном, но вряд ли две гниды уживутся вместе.
Актером он вряд ли станет: в украинском шоу-бизнесе еле концы с концами сводят
успешные артисты, поэтому сыгравшему пару эпизодических ролей там ничего не
светит. О продолжении военной карьеры, если она у него, конечно, была, в чем
есть сомнения, тоже придется забыть: \enquote{побратимы} давно точат ножи за его косяки
и оправдание потерь. Остается только самая древняя профессия. А что? Опыт
переодеваний в женщину есть, принципов нет, все, как говорится, сходится. В
общем, жизнь покажет, как судьба распорядится с данной персоной, но однозначно
нам будет немного не хватать его фееричных заявлений.

\subsubsection{ВОЗВРАЩЕНИЕ ПЕТРА}

Начало недели на Украине ознаменовалась \enquote{пришествием Петра}: 17 января
пятый президент нэзалежной Петр Порошенко, обвиняемый в госизмене, вернулся на
родину из \enquote{дипломатического турне}, в которое отбывал, чтобы не
оказаться за решеткой.  То, как было обставлено и освещалось в СМИ возвращение
\enquote{шоколадного короля}, и дает основание говорить о начале активной фазы
президентской предвыборной гонки. В лучших традициях \enquote{совершенно
спонтанно} были организованы \enquote{стихийные} митинги с идеальной
организацией, наглядной агитацией, сценой, микрофонами и громкоговорителями,
массовкой и прочими атрибутами, которые неоднократно успешно использовались в
таких случаях на территории нэзалежной.   

Организованное командой Порошенко представление началось накануне в Варшаве,
где экс-президент провел \enquote{экстренный брифинг}. Примечательно, что свое
заявление Порошенко делал на английском языке и лишь отвечая на вопросы
украинских СМИ перешел на мову. Видимо, таким образом он хотел охватить
западную аудиторию, выставляя себя жертвой политической расправы. Риторика в
ответах и в заявлении также наталкивала на эту мысль. Не забыл Порошенко и
\enquote{подстелить соломки} на случай негативного развития событий, выставив все так,
что при любых действиях власти он окажется на коне. Вообще нужно отдать должное
порошенковским спичрайтерам, политтехнологам, адвокатам и прочим помощникам,
ведь они смогли из уже начавшего гнить политического трупа сделать
великомученика, который всеми силами пытается спасти украинцев от \enquote{зеленой
чумы}, за что и подвергается преследованию и гонениям. Дело о госизмене против
Порошенко приобрело яркий политический окрас и начало работать в его пользу,
гипотетические негативные последствия уголовного преследования полностью
перекрываются резким ростом рейтинга Петра Алексеевича, быстрее которого лишь
падает популярность Владимира Александровича. Лозунг сторонников Порошенко
\enquote{Украине нужен порох, а не ваЗЕлин!} становится в нэзалежной все более
популярным.

Полный провал тактики команды Зеленского, которая не то что не смогла отправить
Порошенко за решетку, а лишь способствовала росту популярности главного
конкурента своего шефа, четко охарактеризовал украинский блогер Анатолий Шарий,
по словам которого \enquote{Владимиру Александровичу прилюдно и официально помочились
на лицо}. Пожалуй, соглашусь с мнением \enquote{агента ФСБ}, а именно так теперь
принято называть на Украине Шария, и добавлю, что со стороны устроенный балаган
выглядел очень качественно. Власть в течение всего дня демонстрировала свою
несостоятельность, проявляя осторожность и даже трусость, а экс-президент с его
сторонниками вели себя уверенно и агрессивно.

В общем, я представлением остался доволен, готов и дальше наблюдать как паучки
в треснувшей банке нэзалежной грызут друг друга.
