% vim: keymap=russian-jcukenwin
%%beginhead 
 
%%file slova.polkovnik
%%parent slova
 
%%url 
 
%%author 
%%author_id 
%%author_url 
 
%%tags 
%%title 
 
%%endhead 
\chapter{Полковник}

%%%cit
%%%cit_head
%%%cit_pic
%%%cit_text
\emph{Полковник} вільно пересуватися світом міг лише завдяки литовському громадянству
як Йозеф Новак. Від 1930 р. за основний осідок обрав Швайцарію, позаяк у Женеві
розташувалася Ліга Націй як центр міжнародної політики. Там він зосередився на
ідеологічній та інформаційній боротьбі з большевизмом і встановленні зв'язків з
Україною. Уряди Польщі і СРСР полювали на Е. Коновальця. Про це він висловився
в приватному діялозі, висаджуючи в товариша дерево: \enquote{Бодай яму викопав минулого
літа. Може, це для себе. Бо гонять мене вороги так, що не знаю де й жити.
Дослужився...} (Св., с. 35).
Операцію з убивства Коновальця розпочали в серпні 1933 року. Чекіст П.
Судоплатов проник у середовище націоналістів через колишнього старшину СС,
галичанина Василя Хом'яка. Останній представив його як розчарованого в
комунізмі та готового долучитися до націоналістів. Контактував з дуже багатьма
націоналістами. Тричі зустрічався з Коновальцем і спілкувався про розвиток
націоналістичного руху в УРСР
%%%cit_comment
%%%cit_title
\citTitle{Евген Коновалець: від життя до смерти у безсмертя (тези за телепрограмою "Ген українців")}, 
Ірина Фаріон, blogs.pravda.com.ua, 14.06.2021
%%%endcit
