% vim: keymap=russian-jcukenwin
%%beginhead 
 
%%file 21_11_2021.fb.myhelson_oleksandr.1.gidnist_svoboda.cmt
%%parent 21_11_2021.fb.myhelson_oleksandr.1.gidnist_svoboda
 
%%url 
 
%%author_id 
%%date 
 
%%tags 
%%title 
 
%%endhead 
\subsubsection{Коментарі}

\begin{itemize} % {
\iusr{Алексей Панич}

Ну, відкриті ліваки насправді на Майдані не прижилися. А ті, хто прижився,
здебільшого не світилися своїми лівацькими гаслами, відклавши їх "до кращих
часів".

\iusr{Валерій Ковельський}

Оскільки це не майдан коло церкви (на всяк випадок додам) де революція іде то дозволю собі зауваження.
Перший Майдан був справжньою подією громадянської свідомості українців.
Другий Майдан то вже спроба виправити свою дурість. Маючи перед очима приклади т. Путіна, лукашенка, туркменбаші, асада обирати януковича було вкрай нерозумно.
Те, що багато українців проявило геройство не відміняє того факту, що спочатку був дурний вибір.

\begin{itemize} % {
\iusr{Олександр Михельсон}
\textbf{Валерій Ковельський} ну якщо шукати початок, то треба з революції на граніті йти. або з визвольних змагань сто років тому. або з хмельниччини.

\iusr{Валерій Ковельський}
\textbf{Olexander Mykhelson} нє, можна але для чого?
Українці у ХХІ столітті не можуть обрати нормальну владу то про що мова за минулі століття?
Кравчук, Кучма 1, Ющенко - закономірності. Кучма 2 - результат успішної спецоперації.
І тільки янукович - точка повороту на однозначно хибний шлях.

\iusr{Олександр Михельсон}
\textbf{Валерій Ковельський} гм. ну, в певному сенсі да, можна дивитись і так.
\end{itemize} % }

\iusr{Solomia Shtephan}

У житті завжди є привід для поділу: ставлення до тієї чи іншої події, ситуації,
до стану справ... Був спротив. Майдан. Наслідок - аж надто далекий від бажаного
з огляду на розвиток держави


\iusr{Max Mykhaylenko}

Чому, можна і обійнятись - кажуть до нас іде два вуглевози з Колумбії, уявіть
скільки там тон? Тож навіщо платити більше?
\url{https://www.youtube.com/watch?v=ft3Sylw3ssA}

\iusr{Алексей Ковалев}

Завжди пам`ятаємо Майдан.

Ганебніша сторінка в історії України - це правління Януковича. Навіть у
Колумбії не допустили того аби Пабло Ескобар став держпосадовцем. А в Україні
населення само обрало кримінальника.

Зек, бидло з промзони, що на тюрмі був "козлом", баландьором, співпрацював із
адміністрацією, став президентом країни. Як таке можливо? Написав дві книжки,
прочитав одну, так про нього казали.

Його просували вгору вертухаї. Його голова адміністрації Льовочкін був сином
головного вертухая країни. Совпадєніє? Навряд чи. Він був "смотрящім" за
Януковичем. А Янукович був "свинею", на зековськом жаргоні. Його заздалегідь
готували на заклання. Його просували саме за тим, аби проти нього українці
підняли повстання. Це був заздалегідь продуманий план верхівки Партії Регіонів
разом з ФСБ. Аби вивезти потім свого "козла" до Росії й надати Путіну привід
для інтервенції.

Хіба на Майдан вийшли тисячі людей тому, що їм не було більше чого робити й
вони тільки й хотіли помайданити?

Ні. Їх фактично примусили вийти, постійним й систематичним приниженням.
Протягом 10 років в них не сходила з язика тема української історії, вони
порочили усіх українських історичних постатей та взагалі все українське. Це все
накопичувалося. А коли ті покидьки ще й прийшли до влади...

Хто пам`ятає ті часи, в того навряд чи є сумніви, що всі "медійні регіонали",
ті хто ходили по ефірах, свідомо, на протязі 10 років працювали на розкол
країни та створення конфлікту між сходом та заходом, між Донбасом та Галичиною.
Вони умисно провокували конфлікт. Це робилося так відверто, що немає тут з чим
сперечатися. Вони виконали замовлення Путіна на відмінно.

У Кремлі все прорахували заздалегідь. План захоплення Криму в них був щонайменш
років за 10 до 2014го. Вони чекали тільки влучного моменту й приводу для
інтервенції. Й не просто чекали у моря погоди, а створювали собі ті умови.

Й звіряче побиття студентів на Майдані було невипадковим. Тому що вони 10 років
чекали цього моменту й в них вже все було "на мазі". Вже була готова нова
"хата" для Зека у Ростові. План захоплення Криму вже був введений у дію. Якби
Майдан раптом розійшовся, усе б потрібно було відміняти, все планувати наново,
знову чекати іншого моменту.

Тому розплітати цей злочинний клубок потрібно з того, хто віддавав наказ на
побиття студентів й хто надав при цьому усний наказ, зробити це якомога
жорстокіше. Це була свідома провокація. Як, взагалі-то, й вся діяльність Партії
Регіонів.

Партія Регіонів була Партією Влади. Громадяни країни надали їм владу над собою,
над всією державою. Це є найвища довіра суспільства. А вони її використали
проти держави, проти суспільства, проти громадян, на користь ворожої держави.

За найвищу довіру - найвищий попит.

Й порівняння з комсомольською/піонерською організацією тут недоречно. Це
реваншистський вброс, мета якого реабілітувати зрадників.

Те, що Росія становить загрозу було зрозуміло ще після 080808, після нападу
Росії на Грузію.

Вірили вони, що Росія нападе, чи не вірили, нікого не повинно їб@ти. Вони були
зобов`язані підготувати належним чином армію до спротиву можливої російської
агресії. Тому що тисячі людей попереджали про такий можливий сценарій розвитку
подій. Вони саботували свій обов`язок перед Державою, перед усіма нами.

Партія Регіонів - злочинна організація та має понести колективну
відповідальність за умисний саботаж обороноздатності держави, який вона
здійснювала з 2010 по 2014 рік. А також за підбурювання мешканців Донбасу до
сепаратизму.

Якщо чекати поки це зроблять по закону, ми ніколи цього не дочекаємося. Тому що
ці зрадники вже пролізли всюди як щури. Під час війни зрадників в тилу потрібно
позбавлятися в будь-який спосіб. А таких "патріотів", хто їм співчуває, каже,
що тільки по закону треба, всі мають засуджувати за малодушність.

Людина, що вступала до лав цих українофобів має вважатися такою, що втратила
честь і гідність та не може займати державні посади. Зрадники мають
відповідати.

Якщо влада саботує свої функції, покаранням зрадників мають зайнятися патріоти.

Жоден регіонал не повинен вільно ходити по українській землі та вільно дихати
українським повітрям. Ми маємо зробити їх життя нестерпним в Україні.

Зустрів регіонала - плюнь йому в єб@ло.  

\iusr{Max Mykhaylenko}
\textbf{Алексей Ковалев}

\url{https://uk.wikipedia.org/wiki/Флуд}

\end{itemize} % }
