% vim: keymap=russian-jcukenwin
%%beginhead 
 
%%file 28_11_2021.fb.fb_group.stari_fotografii.1.sadok_vyshnevyj
%%parent 28_11_2021
 
%%url https://www.facebook.com/groups/923672854777788/posts/1325479554597114
 
%%author_id fb_group.stari_fotografii,tverdohlebova_galina.kiev
%%date 
 
%%tags foto,gorod,kiev,selo,ukraina
%%title "Садок вишневый коло хати..."
 
%%endhead 
 
\subsection{\enquote{Садок вишневый коло хати...}}
\label{sec:28_11_2021.fb.fb_group.stari_fotografii.1.sadok_vyshnevyj}
 
\Purl{https://www.facebook.com/groups/923672854777788/posts/1325479554597114}
\ifcmt
 author_begin
   author_id fb_group.stari_fotografii,tverdohlebova_galina.kiev
 author_end
\fi

\enquote{Садок вишневый коло хати...}

До войны мои бабушка и дедушка, папины родители, жили в самом центре Киева. Оба
много работали. Их дети, мой папа и его сестра, маленькими часто болели.
Поэтому сидели дома с няней. Няня, сельская женщина, лечила детвору больше
народными средствами и всё жаловалась, что виноват во всём плохой городской
воздух. Однажды она таки уговорила бабушку и деда отпустить детей к ней в село
на всё лето. 

\ii{28_11_2021.fb.fb_group.stari_fotografii.1.sadok_vyshnevyj.pic.1}

\enquote{Дети асфальта} вернулись осенью окрепшие с розовыми щёчками. С тех пор каждое
лето папа с сестрой и няней уезжали в село.

...А потом началась война...

Прошли годы, папа женился, родилась я. И как начала болеть и болеть. \enquote{Нам надо
дочку на свежий воздух вывозить},- сказал папа и уехал в то село, где провёл
своё детство. Нашёл своих деревенских друзей, мальчишек, которые давно выросли
и тоже  обзавелись семьями. Папу приняли с распростёртыми объятьями и, конечно,
пригласили в гости всю нашу семью. 

\ii{28_11_2021.fb.fb_group.stari_fotografii.1.sadok_vyshnevyj.pic.2}

Впервые я увидела деревню в 7 лет и сразу оценила преимущества вольной
деревенской жизни: можно целый день гулять на улице, ходить, куда хочешь,
играть, где хочешь. 

\ii{28_11_2021.fb.fb_group.stari_fotografii.1.sadok_vyshnevyj.pic.3}

А черешни, вишни, яблоки, малина, смородина...- ешь сколько
хочешь! А козье молоко полюбила с первого глотка!  Опять же таки собаки, кошки,
куры, утки, поросята! Столько радостных эмоций для ребёнка!  Только корову я
боялась, уж очень большая! Жили мы в семье папиного довоенного друга. Мама
помогала его жене по хозяйству и даже ходила с ней на ферму. А я играла с
хозяйскими детьми. Училась ходить босиком, лазить на деревья и спать на
сеновале.

\ii{28_11_2021.fb.fb_group.stari_fotografii.1.sadok_vyshnevyj.pic.4}

Каждое лето потом, аж до окончания школы, я хоть на месяц, но в это село
ездила. За селом начинался большой лес. Меня научили разбираться в грибах,
защищаться от гадюк, ориентироваться на местности.

А папа много фотографировал. Сейчас это бесценная память. Такого уже нигде не
увидишь: хатки, крытые соломой, ветряк ( мельница), деревянные церквушки.
Красивейшее старинное село на берегу Днепра. Кийлов называется. Теперь оно
совсем другое.

Фото 1-7 период 1061-62 гг. Потом фото 1966 и последнее 1968.

\ii{28_11_2021.fb.fb_group.stari_fotografii.1.sadok_vyshnevyj.cmt}
