% vim: keymap=russian-jcukenwin
%%beginhead 
 
%%file 27_10_2021.fb.fb_group.story_kiev_ua.1.ikony_vizantia.pic.2
%%parent 27_10_2021.fb.fb_group.story_kiev_ua.1.ikony_vizantia
 
%%url 
 
%%author_id 
%%date 
 
%%tags 
%%title 
 
%%endhead 

\ifcmt
  ig https://scontent-frx5-1.xx.fbcdn.net/v/t39.30808-6/248349850_10158758781467198_1830161848712382083_n.jpg?_nc_cat=100&ccb=1-5&_nc_sid=b9115d&_nc_ohc=6I2s_2aYX0AAX_RdO99&_nc_ht=scontent-frx5-1.xx&oh=00_AT9GH54__kzlJ-DzAkIMoZQ5f3sJinRSRDyyH7ZzAzDgAA&oe=61BBCAD1
  @width 0.4
\fi

\iusr{Оксана Денисова}
\figCapA{Мученик и мученица VII век}

\iusr{Sergej Mirnyj}
Там говорили о сверх искустве, о... А тут какой то детский рисунок @igg{fbicon.man.facepalming} 

%\iusr{Людмила Митина}

%\ifcmt
  %ig https://i2.paste.pics/ddea64722f9a343404e37aa4bb07d5b8.png
  %@width 0.2
%\fi

\iusr{Анна Владимировна Лесюк}
По моему у них все хорошо..

\iusr{Татьяна Оксаненко}

Может быть, стиль написания икон, стал более утонченным, но, здесь, привлекает
старина написания икон и удивительный способ изобретения, так называемых
красок. Удивительно, но человеческие таланты—неисчерпаемы! Спасибо Вам,
Оксаночка! Вы, как всегда, преподнесли, все изысканно...

\ii{27_10_2021.fb.fb_group.story_kiev_ua.1.ikony_vizantia.pic.2.cmt}
