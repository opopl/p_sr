% vim: keymap=russian-jcukenwin
%%beginhead 
 
%%file 15_09_2021.fb.jakovenko_vjacheslav.1.donbass_vojna_razmyshlenia
%%parent 15_09_2021
 
%%url https://www.facebook.com/kerchv/posts/114521157627531
 
%%author_id jakovenko_vjacheslav
%%date 
 
%%tags 2014,donbass,mysli,ukraina,vojna
%%title Что произошло тогда в Донбассе – это уникальное для современного исторического момента явление
 
%%endhead 
 
\subsection{Что произошло тогда в Донбассе – это уникальное для современного исторического момента явление}
\label{sec:15_09_2021.fb.jakovenko_vjacheslav.1.donbass_vojna_razmyshlenia}
 
\Purl{https://www.facebook.com/kerchv/posts/114521157627531}
\ifcmt
 author_begin
   author_id jakovenko_vjacheslav
 author_end
\fi

Погружаясь в размышления о периоде, который потряс Донбасс и, в целом, Украину
в 2014 году, снова и снова анализируя происходившие события, были сделаны
некоторые заметки на полях. Эти заметки, возможно, обрывочны и не тянут на
окончательный вывод, но дают достаточные основания для дальнейшего развития
мысли.

\ifcmt
  pic https://scontent-frx5-1.xx.fbcdn.net/v/t1.6435-9/241912787_114521064294207_8170789508291082615_n.jpg?_nc_cat=100&ccb=1-5&_nc_sid=730e14&_nc_ohc=L-313CijnjAAX8MUhOI&_nc_ht=scontent-frx5-1.xx&oh=37a221a43ddc844b512f9b8e5f0592ea&oe=619B2426
  @width 0.7
\fi

То, что произошло тогда в Донбассе – это уникальное для современного
исторического момента явление. Вы зададитесь вопросом, почему оно уникальное?
На фоне давления культа потребления и психологии индивидуализма трудно было
себе представить, что жители Востока [тогда еще] Украины, живущие, как все
остальные, семьей, работой, накоплением, могут самоорганизоваться, объединиться
и дать отпор той силе, которая кровавой поступью двинулась с запада. Все эти
люди с разными интересами, достатком, уровнем образования, разной
национальности оказались объединенными культурным ядром, которое со времен
Киевской Руси держало вместе русских. Именно его, это самое культурное ядро,
чужеродное для нацистов, и вознамерились разрушить агрессоры. Только
разрушением можно было оторвать Донбасс от русского мира.

Но нужно сразу учесть тот факт, что двигателем восстания в Донбассе стали
несколько факторов, без каждого из которых бы, невозможна была бы протестная
активность. Эти «компоненты» в своей совокупности породили «гремучую смесь»,
которая сдетонировала при добавлении «инициирующего вещества». «Инициирующим
веществом» оказалось непостижимое для человека, изуверское насилие со стороны
киевского режима.

Но, к сожалению, эта «гремучая смесь» сама по себе не смогла долго «гореть», и
потом, после 2014 года появились высказывания по типу «мы не за это сражались».
Для того, чтобы «пламя» переросло в «пожар» (в хорошем смысле этого слова),
должен был быть механизм, поддерживающий и направляющий это «пламя» в то русло,
в котором полезность от «горения» была бы максимальной.

Давайте по порядку и без метафор. Украина образца 2014 года – это застой в
тупике. Украинская элита, находившаяся в тот период времени у власти, мягко
говоря, дистанцировалась от народа, закон им стал не писан, и на все стенания
простых людей ей было глубоко безразлично. Все решалось через связи, коррупцию
и «цену вопроса».

Если бы «майдан» прошёл без лозунгов об евроинтеграции, и на сцену не вылезли
украинские нацисты, смена власти произошла бы без шума и пыли. Опыт подобного
Майдана у Украины был в 2004 году. Тогда проигравшая власть Януковича сделала
попытки реванша, но они остались популистскими. «Партия регионов», скорее
всего, в 2014 году не получила бы поддержки со стороны Донбасса, как это было в
2004.

Во время событий января - первой половины февраля 2014 года никакого активного
и протестного к майдану движения в Донбассе не было. Донбасс работал и за всем
происходящим наблюдал с экранов телевизора. А значит, народ был готов принять
майдан. Но нацистские группировки стали генерировать совсем иную, чёрную
энергию, которая стремилась разрушить самое главное, на чем держится народ. И
определенная, по-настоящему патриотически настроенная часть населения стала
готовой подержать Януковича, дабы не допустить украинских нацистов к власти.

Когда «жгут» людей при исполнении своих служебных обязанностей по защите
законности – это одно дело; но когда насильство захватывает простых граждан –
это уже совсем иное. «Инициирующим» фактором стало насилие со стороны
«майдановцев» в отношении людей из «антимайдана». Эти события прекрасно
отражены в фильме «Крым», в сцене с сожженными автобусами.

20 февраля 2014 года украинские нацисты в 135 километрах от Киева начали
сжигать и убивать своих граждан из-за того, что у них иная точка зрения на
происходящее. Это кровавое событие в новой истории Украины отмечено, как
«Корсуновский погром».

«Инициирующее» вещество было добавлено, произошла детонация, и к концу февраля
Донбасс начал самоорганизовываться. Появились первые списки самообороны
Донбасса.

продолжение следует...

\url{https://www.youtube.com/watch?v=cFRrd8ApaKU}

