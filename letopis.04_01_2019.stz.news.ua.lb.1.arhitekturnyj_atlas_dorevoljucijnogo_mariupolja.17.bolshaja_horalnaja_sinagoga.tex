% vim: keymap=russian-jcukenwin
%%beginhead 
 
%%file 04_01_2019.stz.news.ua.lb.1.arhitekturnyj_atlas_dorevoljucijnogo_mariupolja.17.bolshaja_horalnaja_sinagoga
%%parent 04_01_2019.stz.news.ua.lb.1.arhitekturnyj_atlas_dorevoljucijnogo_mariupolja
 
%%url 
 
%%author_id 
%%date 
 
%%tags 
%%title 
 
%%endhead 

\subsubsection{Большая хоральная синагога}

Также в отдельном ряду стоит мариупольская хоральная синагога. Её остатки -
единственный образец культовой архитектуры дореволюционного Мариуполя. В 30-х
годах ХХ века в разгар антирелигиозной кампании большевики уничтожили все
православные храмы и католический костёл, а синагогу по неизвестным причинам
пощадили. Во время немецкой оккупации в 1941-1943 г.г. здесь размещался
госпиталь и пункт сбора рабочей силы для отправки в Германию. После
освобождения города помещения синагоги использовались как гимнастический зал,
контора Гипромеза, больница, медучилище и морская школа. И только к середине
восьмидесятых годов для роскошного здания бывшей синагоги нашли более
возвышенную роль, под стать ей самой. В ней должны были разместить экспонаты
художественной коллекции краеведческого музея, но не успели. Зимой обветшавшее
за много лет здание не выдержало обильного снегопада – рухнувшая крыша
похоронила все планы. Это событие поставило точку в истории сооружения. Сегодня
апокалиптический скелет синагоги ценится как оригинальный антураж для
фотосессий.

\ii{04_01_2019.stz.news.ua.lb.1.arhitekturnyj_atlas_dorevoljucijnogo_mariupolja.17.bolshaja_horalnaja_sinagoga.pic.1}
\ii{04_01_2019.stz.news.ua.lb.1.arhitekturnyj_atlas_dorevoljucijnogo_mariupolja.17.bolshaja_horalnaja_sinagoga.pic.2}
\ii{04_01_2019.stz.news.ua.lb.1.arhitekturnyj_atlas_dorevoljucijnogo_mariupolja.17.bolshaja_horalnaja_sinagoga.pic.3}
\ii{04_01_2019.stz.news.ua.lb.1.arhitekturnyj_atlas_dorevoljucijnogo_mariupolja.17.bolshaja_horalnaja_sinagoga.pic.4}
