%%beginhead 
 
%%file 17_03_2023.fb.kipcharskij_viktor.mariupol.1.r_k_tomu__den_22__17
%%parent 17_03_2023
 
%%url https://www.facebook.com/permalink.php?story_fbid=pfbid0tW8KM11iodKZ5FvS36PsqyyhZv4GBQpEPZHCjVZVo8fdBLmS6WHVjL3kQCW8ubm4l&id=100006830107904
 
%%author_id kipcharskij_viktor.mariupol
%%date 17_03_2023
 
%%tags mariupol,mariupol.war,dnevnik,17.03.2022
%%title Рік тому: День 22, 17-03-23. Четвер - Ночували у Юр'ївці
 
%%endhead 

\subsection{Рік тому: День 22, 17-03-23. Четвер - Ночували у Юр'ївці}
\label{sec:17_03_2023.fb.kipcharskij_viktor.mariupol.1.r_k_tomu__den_22__17}

\Purl{https://www.facebook.com/permalink.php?story_fbid=pfbid0tW8KM11iodKZ5FvS36PsqyyhZv4GBQpEPZHCjVZVo8fdBLmS6WHVjL3kQCW8ubm4l&id=100006830107904}
\ifcmt
 author_begin
   author_id kipcharskij_viktor.mariupol
 author_end
\fi

Рік тому:

День 22, 17-03-23. Четвер

Ночували у Юр'ївці. Я майже не спав. Чому? Лякає тиша? Намагався вмовити себе,
спати, пояснити собі, що на нас чекає довга дорога... 

Коли ми вперше їхали до Львова, доїхали до Запоріжжя десь за три години. А
зараз ми у дорозі вже майже добу...

Їхати, чи залишитися у безпечному місці? Лякали новини про розстріл
попередньою колони, про можливість ночівлі у полі на морозі, про відсутність
палива. Що визволителі забирали в людей автівки, але переважно нові та
особливо дизельні.

Четвертий тиждень \enquote{гарячої} війни почався з підготовки до від'їзду. Хазяйка,
яка спала на гумовому матраці навпроти вхідних дверей, бо усі спальні зайняли
незвані гості (ми спали на розсувному диван,  можна сказати, на кухні),
приготувала сніданок. Нагодували онуків, поїли, позбирали і винесли речі
(спальники, зарядки, тощо). Подякували хазяйку. Намагалися заплатити, чим
образили Надію.

Вночі був сильний мороз. Вікна у машинах геть позамерзали - поливали їх
\enquote{розморожувачем}.

Сьогодні Епіка завелася без проблем, а от Калина вередувала: кілька разів
крутив стартером, поки завелася. Потім не вмикалася задня передача і довелося
виштовхувати її з двору. Таке враження, що хтось чи щось намагався втримати нас
у Юр'ївці.

Напередодні Дмитро казав, що на блок-постах питають сигарети, тож чекали на
відкриття магазину, аби купити на "хабар". Надія сказала, що вже кілька днів в
магазині сигарет нема, але може вранці привезуть. Звідки? Може зі складів у
Бердянську...

Ще раз попрощалися, подякували і поїхали.

В магазині сигарет не було. 

Перед виїздом, порадившись з Надією та Дмитром, прийняли рішення їхати старою
(точніше - колишньою) дорогою, бо чим гірша дорога, тим менше черги на
блок-постах.

Від дороги залишилася хіба що назва: геть уся у ямах. Добре хоч те, що калюжі
позамерзали і багнюки нема. На початку дорогу мало не через дві-три сотні
метрів лежать відірвані пластикові бампери. Син їде дуже обережно, аби не
відірвати бампери та глушники, бо машини важко навантажені - труть по дорозі
бризковиками.  На під'їзді наздогнали колону з п'яти машин, які ледве повзли.
Оскільки невідомо, чи вистачить нам бензину у баках, я вирішив обігнати їх
розкатаним узбіччям - мороз скріпив багнюку. 

Попереду на підйомі, на крутому схилі,  блок-пост (Оля рахувала - всього їх
було 21). Стали. Добре хоч те, що сонечко світить і, головне - гріє.

Незважаючи на попереджувальні написи на узбіччі \enquote{Мини}, якісь тітки побігли \enquote{у
кущики}. Ми сказали, щоб вони бігали туді, звідки уламки не долетять до наших
автівок. Калина почала знову вередувати - не крутив стартер. Син зачепив мене
буксиром і нам дуже крутому підйомі підтягнув до блок-посту. (Машина виявилася
розумнішою за мене: вона вередувала мало не на кожному блок-пості і,
безперечно, це нам дуже допомогло: машина з малими онукам тягне машину з
пенсіонером-інвалідом. В усякому випадку, нам не довелося вивантажувати речі із
машини і жінки навіть не виходили з машин. Я навіть взяв з неї приклад: коли
просили відкрити багажник, я важко кульгав і взагалі демонстрував неміч: важко
вилазив, важко сідав - відволікав увагу на себе. Калина поганому не навчить! Як
Оля стримувала себе від сміху на тому моєму спектаклі - не знаю, я сам
намагався не сміятися). 

Поки син чіпляв буксир, нас об'їхали "наші добрі люди), які не мали часу чекати
(чи совісті?)

До нашою машини широко посміхаючись підійшов бурят чи якут. Я простягнув у
вікно паспорти . Він їх узяв і запитав: \enquote{Отец, оружие есть?}. Я відповів: \enquote{У
машині попереду мої онуки - став би я наражати їх, везучи зброю?}. \enquote{Може
мисливська?}. Я мало не бовкнув, що я не мисливець, а рибалка, але вчасно
прикусив собі язика, аби не наражатися на інші запитання. Ми ще поговорили про
проблеми з машиною, потім він повернув документи, так і на розкривши їх і нас
пропустили. 

Трохи відїхали і Калина завелася.

У Бердянську скрізь майорять наші прапори.

Люди протягують в сторону машин пакети з якоюсь їжею. Я, грішним ділом подумав,
що звиклі заробляти на відпочивальниках бердянці налагодили новий бізнес, аж
поки не причитав на картоні, який притулили до скамійки з пакетам їжі: \enquote{Еда и
ночлег безплатно}! І досі соромно, що я запідозрив бердянців у користі. 

Через десяток блок-постів доїхали до Токмаку. Практично на кожному просили
сигарети. Я на випередження почав питати, де є бензин: тобто - не я їм
відмовляв, а вони мені. Про сигарети я не брехав: в мене їх дійсно не було -
вони лежали у Олі в сумці, в мене була люлька! Тож я казав щиру правду і
реакції у мене були відповідні - я ж не брехав!

На черговому блок-пості, коли доглядали багажник, вояка побачив перчатки. Вони
були геть попечені на вогнищі, через те, що доводилося знімати каструлю, яка
раптово закіпила і загрожувала залити вогнище (та й вода була дефіцитом, аби
так її витрачати), а часу шукати.\enquote{прихватки} не було. Тож, хапаючи каструлі за
розпечені вогнем ручки, я спалив кілька пар перчаток і ці також добряче
попропалював на пальцях. Вояка: "Можна я візьму перчатки?),  показав червоні
поморожені руки і не очікуючи відповіді, взяв їх. Перчатки прикривали руків'я
палиці (зазвичай я возив її у кабіні, аби спиратися коли вставав, але зараз там
не було вільного місця. Побачивши палицю (а може побачивши опіки на
перчатках?), він ніби смикнувся покласти їх назад, але передумав - не поклав і
дозволив закрити багажник.

На зустріч нам проїхали кілька колон БТР-ів з машинами забезпечення:
бензовози, вантажівки з кухнями на причепі. Усі з літерами \enquote{V}.

Час від часу на дорозі стояли спалені танки,  БМП, або БТР з баштами,
повернутими нам назустріч... Біля деяких були сліди волочіння - їх відтягували
з дороги...

На черговому блок-пості молодий вояка запитав, чи маємо з собою ножі, бо
попереду чекає серйозний обшук (він сказав \enquote{шмон}). Я подумав, що він хоче
\enquote{наварити} \enquote{блатний} ножик, а Оля сприйняла це серйозно і почала наполягати,
аби ми повикидали усі ножі. Тож мені довелося порпатися в сумках і викидати
майже усі ножі, бо в Олі був такий вигляд, ніби ми веземо щонайменше атомну
бомбу. Судячи з того, як багато людей почало бігати "під деревця" обабіч
дороги, таке саме було мало не в кожній машині. Мені довелося позбутися мого
улюбленого \enquote{Гербера} і рибальського, а Оля викинули маленький з кухонного
набору, яким зазвичай  чистила картоплю. 

А от сокиру, руків'я якої зробив мій тато ще у Кандалакші  півстоліття тому, я
не викинув. Тай й щоб до неї дістатися, треба було б розвантажити багажник.
Сокирою я у дворі рубав гілки, а зберігав її у багажнику.

На в'їзді до Токмаку і, особливо,  на виїзді ми стояли без руху майже три
години. Весь цей час повз нас туди-сюди їздив Сітроен-дизель (я би сказав -
дуже смердючий дизель) хетчбек без глушника, чомусь розмальований з балончиків
у сине-білий \enquote{камуфляж} типу міліцейського. У відкритому багажнику Ситроену
сидів ну дуже критий вояка к броніку з кулеметом ПК у руках, коробка якого
заважала йому вилазити з багажника: він, вибирав у черзі когось з нетерплячих,
проводив поза чергою до блок-посту і повертався по чергового \enquote{клієнта}. І так
весь час.

Син підійшов до вояки, який наставив на нього зброю і сказав, що у нього у
машині мерзнуть діти. Той дав відмашку проїжджати. Епіка поїхала і потягнула
Калину. Вояка почав махати і кричати, та потім роздивився, що Калина на
буксирі. Коли вояка наставив на сина зброю, я так злякався, що не розблокував
руля і бусирувальна линва розірвала Калині, яка поїхала вбік, передній бампер.

Міст у Василівці підірвали і тому ми поїхали в об'їзд селом через старий місток
і наздогнали дві вантажівки, у кузовах яких були морські контейнери. Вантажівки
довго не могли проїхати під газовою трубою. Нарешті ми під'їхали майже до траси
і стали. Стояли години три, бо поліцейські  повели колону на Запоріжжя і ми у
цій \enquote{сірій зоні} чекали, поки вони повернуться по нас. За весь час ми не бачили
у селі жодної людини, жодного вогника у вікнах чи диму з труби. Не чули навіть
собачого гавкання. Навіть коти не бігали! Село наче мертве, хоч будинки
доглянуті, не занедбані.

Як так вийшло що за три години \enquote{визволителі} зайшли у Херсон, а у Запоріжжя не
зайшли за три тижні?

Між нами і трасою було з десяток машин, в тому числі й ті контейнеровози, через
яких ми не встигли до попередньої колони. Просто в очі світило багряно-червоне
сонце, що сідало. Колір натякав, що вночі буде сильний мороз. В машині стало
дуже холодно, вікна почали пітніти від пари з рота. Я, згадавши розповідь
Едіка, трохи відкрив вікно. Почав молити Небесних Захисників, щоб вивели
онуків, аби вони не ночували на морозі. (Молити Захисників я почав ще у
Маріуполі, коли над нами кружляли ворожі літаки). З машини попереду час від
часу вискакував чоловік ножним насосом качав заднє праве колесо. Потім вийшли
жінки і почали розвантажувати багажник - напевно, шукали домкрат та балонник.
Знайшли. Але відкрутити гайки, аби поставити запаску, не змогли. 

Водій вантажівки попереду заліз у кузов і відкрив контейнер. З контейнеру
вилізли дівчати і пішли \enquote{у кущі}. З кабіни вантажівки вилізли ще троє дівчат,
які полізли у контейнер. Напевно, вони по черзі грілися у кабіні.

Нарешті машини попереду почали рухатися. Епіка витягла Калину на трасу і ми
поїхали. З часом Калина завелася, бо їхати на причепі з \enquote{дубовими}, гальмами
було досить важко. Та ще й за швидкістю колони, об'їжджаючи перепони... Нарешті
Калина завелася. Поки син відчепляа  її, нас об'їхали декілька машин. Настала
повна темрява. Проявилося неповторне видовище: перед нами по пагорбах повзе
нескінченна червона змія (габаритні ліхтарі машини попереду), звиваючись
вгору-вниз та збоку в бік, а позаду - така сама, тільки біла (фари машин, що
їдуть за нами). Напевно, нашу колону приєднали до тих, що збирали а інших
місцях, бо деякий час попереду блимали червоно-сині вогні поліцейської машини,
а потім їх не стало. Але ми не зупинялися - напевно нас \enquote{стикували} на ходу.

Об'їжджати у темряві протитанкові \enquote{їжаки} - теж те ще задоволення. До речі, на
виїзді з Василівки нам перебіг справжній заяць.

По сьомій нас привели на парковку до великого магазину (зараз вже можна
сказати, що то був Епіцентр). Просто на парковці ми показали документи, нас
переписали і порадили зайти у магазин, де втомлені волонтери нам дали суп,
хліб, кашу з м'ясом - все це було складено у пакетах. (Хтось це цілий день
готував, розливав, пакував в привозив: ті люди, що нас зустрвчали , показували
куди пройди, де туалети, роздавали їжу і таке інше, - то лише верхівка
айсбергу. Пізніше писали, що 26-го з Маріуполя виїхали понад одинадцять тисяч
людей на 2500-х автівках! І це лише з Маріуполя).  Окремо можна було взяти чай,
або каву. Столів не було: я тримав перед онуком тарілку з супом, бо в одній
руці в нього була ложка, а в іншій - хліб. Він не їв - він метав як за себе:
майже дванадцять годин у машині тільки щось перекусували. Невістка з сином
годували онуку - та слухняно розкривала рота, коли до нього наближалася ложка.
Оля підносила суп, кашу, хліб.

Онук з'їв не усю кашу - наївся. Оля запитала: \enquote{Чай будеш?}. Він \enquote{М'яско було
ріденьке}, тобто,  з підливою. Малеча \enquote{доганялася} печивом, почали їсти
дорослі. Я з'їв своє і кашу після онука. Голосом вовка з мультика \enquote{Жив-був пес}
сказав: \enquote{Ну всё - щас сплю}. Онука тоненько: \enquote{ У-у-у...}. Відтанула трохи - бо
перед вечерею  була така переполохана, сер'йозна, доросла... Це у чотири
роки... (Онуку на той час було під дев'ять). Сергій зателефонував друзям: у
одного з них у Запоріжжі живе теща і він запросив нас ночувати в неї до них
шістьох +,без тещі, ще й нас шістьох!), але вже настала комендантська година,
тож нас муніципальним автобусом повезли у дитячий школу-садок з промовистою
назвою \enquote{Надія}, де на сьогодні закінчилася подорож від Надії до \enquote{Надії}. Жінок
і дітей розмістили у спальнях, а чоловіків відправили до спортивної зали на
другому поверсі, де на підлозі було розкладено спортивні мати з дитячими
матрацами та подушки з наволочками. Ми зняли куртки, якими накрилися коли
лягли.

Пройшла жінка, записала прізвища і кому куди їхати, аби знати скільки автобусів
треба, аби повезти людей до машин біля Епіцентру, до залізниці та автовокзалу.
Якийсь хлопець запитав в неї, чи усі у Запоріжжі такі добрі, а вона відповіла,
що Запоріжжя не виняток - такі люди на усій Україні  Хтось за телефонами шукав
квартиру у Запоріжжі: за \enquote{однушку} від десяти тисяч. 

Я зателефонував Любі, Миколі, Едіку і заснув. Серед ночі хтось привів двох
собак. Вони почали гавкати, хтось сварився - я чув це скрізь сон але не
прокидався. Прокинувся, як звичайно, десь о першій. Накрив сина вільним
матрацом. Він сказав, що я так хропів, що він розірвав серветку і запхав собі у
вуха. 

Записав щоденник.

Настав ранок 23-го дня.

Вранці нам дали сніданок (дівчата з дітсадка прийшли о третій години ночі, аби
начистити картоплю нам на сніданок!). Ми сиділи за дитячими столами і на
дитячих стільцях; їжу нам приносили. Жіночка пригостила онука цукеркою, яку
дістала з кишені білого халату. 

Оля розказала що звечора вони з жінками сиділи і розмовляли, аж поки до групи
не зайшла жіночка і не запитала: "А чого ви світло не вмикаєте?". А ми  вже
звикли без світла". Разом з Олею були жінки, які були у підвалі Терраспорту,
які розповідали, як серед ночі прийшли їх звідти \enquote{визволяти}. Але це вже не мої
спогади.  

Після сніданку нас автобусом довезли до Епіцентру - до автівок. Ми взяли на
дорогу печиво, чай (діти - каву). У магазинах стояли коробки з новим одягом,
який можна було брати безкоштовно.  

Розпитали таксистів де є заправки, заправилися бензином і газом (розрахувалися
картками!!!) і поїхали до наших старших дітей і онуків... 

Калина вередувала, тож дорогою ми заїхали до моєї двоюрідної сестри Люби до
батькового села, а потім до Тетяни - так ч наново знайомився з родичами...

Але це вже \enquote{мирні спогади}...

Я знайшов адресу електронної пошти того садка (насправді, це ДНЗ Надія -
школа-садок). Сьогодні напишу їм листа, а якщо вдасться - зателефоную та
подякую за те, що надали нам прихисток їжу рік тому..

%\ii{17_03_2023.fb.kipcharskij_viktor.mariupol.1.r_k_tomu__den_22__17.cmt}
