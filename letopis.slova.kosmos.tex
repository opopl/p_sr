% vim: keymap=russian-jcukenwin
%%beginhead 
 
%%file slova.kosmos
%%parent slova
 
%%url 
 
%%author 
%%author_id 
%%author_url 
 
%%tags 
%%title 
 
%%endhead 
\chapter{Космос}

%%%cit
%%%cit_head
%%%cit_pic
%%%cit_text
А у нас раздают \emph{космические} советы \emph{космической} глупости.  Дополню Александр
Рябоконь немного цифрой, чтобы окончательно стало понятно, что философия
господина Макогона мягко говоря - ни о чем.  Как в известном анекдоте: о доме
надо думать.  Как сохранить свою конкурентноспособность и обеспечить
потребителей внутри страны, а не решать вопросы \emph{космического масштаба}, раздавая
советы такого же уровня глупости
%%%cit_comment
%%%cit_title
\citTitle{В первую очередь надо думать о том, как обеспечить население внутри страны / Лента соцсетей / Страна}, 
Валентин Землянский, strana.ua, 03.07.2021
%%%endcit

%%%cit
%%%cit_head
%%%cit_pic
\ifcmt
  pic https://day.kyiv.ua/sites/default/files/21-07-52/21_korolovserhii.jpg
  width 0.4
\fi
%%%cit_text
Звертаючись до біографії Сергія Корольова, неможливо оминути його \emph{космічні}
звитяги, обов’язковою умовою котрих завжди є слово «перший»: перший штучний
супутник Землі, запущений на орбіту, перший знімок зворотної сторони Місяця
(преса всього світу називала його «Знімком століття»), перший \emph{космічний політ}
Білки і Стрілки, перший політ Юрія Гагаріна у \emph{космічний простір}, перший в
історії \emph{космонавтики} вихід у відкритий \emph{космос}, здійснений Олексієм Леоновим.
Відомо, що, окрім Гагаріна, претендентів на першу подорож людини в \emph{космос} було
кілька, серед них — українець Павло Попович і чуваш Андріян Ніколаєв. Однак
всі, включно з самими \emph{космонавтами}, розуміли неможливість їхньої участі в
Першому польоті, адже Першими завжди мають бути росіяни. «Ви, хлопці,
уособлюватимете дружбу народів — українець і чуваш», — пояснював Корольов,
якого вже встигли занести до списку російських учених. Бо Корольов — Головний
конструктор, він Перший, він достойний бути росіянином.
Але помандруймо до витоків. Де ж починався автор перших \emph{космічних} здобутків
«братства народів» та найбільш засекречений радянський учений?
%%%cit_comment
%%%cit_title
\citTitle{Про двох хлопців з Житомира: Короленка і Корольова}, Олександра Кльосова, day.kyiv.ua, 30.07.2021
%%%endcit

