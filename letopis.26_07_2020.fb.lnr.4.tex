% vim: keymap=russian-jcukenwin
%%beginhead 
 
%%file 26_07_2020.fb.lnr.4
%%parent 26_07_2020
 
%%endhead 
  
\clearpage
\subsection{ВЫ УЖЕ ПРОСЧИТАНЫ: КАК РАБОТАЮТ СИСТЕМЫ ИСКУССТВЕННОГО ИНТЕЛЛЕКТА}
\url{https://www.facebook.com/analitikym/posts/628034724476250}

\index{FB!GROUPS!LNRGUMO}
\index{IT!AI}

Обстоятельная лекция исследователя Игоря Шнуренко о развитии и последствиях внедрения искусственного интеллекта в человеческое общество и его повседневный обиход. Текущие вопросы, затронутые в лекции:

\begin{itemize}
\item 1. Как большие данные увязываются с необъявленными войнами за ресурсы планеты.
\item 2. В чём состоят отличия современная война за "новую нефть" (человеческие ресурсы), а точнее человеческое поведение.
\item 3. Как связано управление человеческим поведением с накоплением данных (человеческая big data).
\item 4. Главные бенефициары войны за человеческие данные – пул корпораций FAGMA (Facebook, Amazon, Google, Microsoft, Apple).
\item 5. Феномен устойчивости корпораций FAGMA в кризисные периоды экономики.
\item 6. Перспективы выживания малого и среднего бизнеса в зависимости от FAGMA.
\item 7. FAGMA строит новый мир основываясь на цифре в фарватере своего государства, управляя человеческой популяцией.
\item 8. Два пути эволюции: государство становится цифровой корпорацией, корпорация становится государством.
\item 9. На чем основана бизнес-модель корпораций FAGMA.
\item 10. Что такое поведенческий продукт.
\item 11. Почему человеческая big data лежит в основе цифровой экономики.
\item 12. Почему системы искусственного интеллекта иногда называют поведенческими фабриками.
\item 13. Как работает поисковая система Google на самом деле в плане сбора данных.
\item 14. Почему самым ценным для цифровых платформ о пользователях – «поведенческая добавка».
\item 15. Как создается "поведенческая добавка" о пользователях на цифровых платформах.
\item 16. Как зарабатывают цифровые компании на поведенческих цифровых фьючерсах.
\item 17. Как будут получать информацию о "поведенческой добавке" цифровые компании.
\item 18. Зачем продвигают технологию беспроводной связи 5G.
\item 19. Как измениться точность предсказаний поведения человека с внедрением технологии быстрой связи \item 5G.
\item 20. Как цифровые платформы реализуют на практике формирование роевого поведения людей на базе технологий социальной инженерии.
\item 21. Как Российская Федерация пытается включиться в гонку за Big Data.
\item 22. Существует ли опасность для граждан Российской Федерации при запуске цифрового проекта в Москве.
\item 23. Что является самым страшным для выстаиваемой системы искусственного интеллекта для тотального цифрового контроля общества и каждого человека.
\item 24. Каковы шансы человека доказать свою правоту в случае борьбы с системами искусственного интеллекта.
\item 25. Почему Интернет вещей резко расширит объем собираемых данных, которые необходимы для управления поведением людей.
\item 26. Понятие свойств «компьютерной поверхности» во внедряемых предметах в бытовую среду человека.
\item 27. Почему в выстраиваемом обществе будущего необходимо высокая гарантия надежности средств передачи информации.
\item 28. Почему новая организация общества будет возможна только на "роевых" началах.
\item 29. Какое место внедряемые технологии займут в обществе и какие риски это представляет.
\item 30. В чем заключается неустойчивость выстраиваемой системы тотального контроля.
\item 31. В чем заключается тайна инструментальной экономики, которая выстраивается на наших глазах.
\item 32. Почему была спровоцирована волна эпидемии короновируса с точки зрения ускорения внедрения систем исскуственного интеллекта.
\item 33. Почему борьба на глобальной и внутренней арене ведется за контролем над социальной средой человечества.
\end{itemize}

Подготовлено по материалам телеканала День: \url{https://www.youtube.com/watch?v=kQ5VDkGLz5c}

\href{https://l.facebook.com/l.php?u=https%3A%2F%2Fgeopolitikym.site%2Fot-bilderberga-k-gulagbergu-kak-globalnaya-elita-stroit-elektronnyj-konclager%2F%3Ffbclid%3DIwAR2STwnETlzpyNSDpL7P24IUX3S9WvWJB3Ab1Nc0EsgWvOaiteQc_z4j17c&h=AT1wbjkDVgFxalonLp4tcLk7UcepD4MGhTKyYPTvOLOHSSxiwOokqhs5BrD4Ej6pVt2BnHWiM3IqQA22Bk3bwqIrzI6AwQAWqvau4pGchOJOw6ZvAhA9uLCBIwR61e830dNk3u4f0eC4H1DireSmfYmKig5BljlSpF9wkJgEMhSilbUeVsMKi9WZOOY_KP2KCR9O4X0OLwwnDTcSZXvaq0V-fM3d7k4w4oWEt_MyMW7-IDbmYgecAuf6AEwoImloPpur2S6Kt0Lw_aP9u684QDEK14JAffEYHjL4hl5Th9n_QT-Hx7cVZ6Psm7jTV0q_UI9zuPfmjAg8Z-HFM3nW83cj7A1wyBs5sdDQti1XKVmgM-lvKUx38sj0FCMEFZhJsOEFvoGWmxUYERpTRG_Eb1ykfsTo4aQ-ToXtYXRJYJOth8h_K3f5balQdFRXyTQWP7gqDr1FbJ9w8PxDRgcvjWZn_aV8n6sR05RordJB4GjgJ_mDl6nzpbP8OjCLkHzFBJVnqPaCIswsvR7_DxYwU61bjGU0eGr8hw9H_ogPD_ztb4I_TGGfo77G_OGHku3TUAu2CTkxw-4IbN00iv9sM6jRE_SuBWTviUIfDDtKCyNX1qJF792st5kk5YLmdCmbgBzT0dgi0i3p5L0}{Link} 

\#системыисскуственногоинтеллекта, \#поведенческаядобавка,
\#компьютернаяповерхность, \#тотальныйцифровойконтроль,
\#контрольнадсоциальнойсредойчеловечества,
\#точностьпредсказанийповедениячеловека, \#поведенческиецифровыефьючерсы,
\#поведенческийпродукт, \#поведенческаядобавка, \#FAGMA, \#большиеданные,
\#человеческиеданные, \#цифровыекорпорации,
\#бенефициарывойнызачеловеческиеданные, \#человеческаяbigdata, \#новаянефть,
\#цифровое золото, \#ИгорьШнуренко
  
