% vim: keymap=russian-jcukenwin
%%beginhead 
 
%%file 03_10_2020.stz.news.ua.mrpl_city.1.andrij_lucenko_krascha_versia_sebe
%%parent 03_10_2020
 
%%url https://mrpl.city/blogs/view/andrij-lutsenko-budte-krashhoyu-versieyu-sebe
 
%%author_id demidko_olga.mariupol,news.ua.mrpl_city
%%date 
 
%%tags 
%%title Андрій Луценко: "Будьте кращою версією себе!"
 
%%endhead 
 
\subsection{Андрій Луценко: \enquote{Будьте кращою версією себе!}}
\label{sec:03_10_2020.stz.news.ua.mrpl_city.1.andrij_lucenko_krascha_versia_sebe}
 
\Purl{https://mrpl.city/blogs/view/andrij-lutsenko-budte-krashhoyu-versieyu-sebe}
\ifcmt
 author_begin
   author_id demidko_olga.mariupol,news.ua.mrpl_city
 author_end
\fi

\ii{03_10_2020.stz.news.ua.mrpl_city.1.andrij_lucenko_krascha_versia_sebe.pic.1}

В моєму блозі окреме і дуже важливе місце посідають статті, присвячені
театральному життю Маріуполя. Сподіваюся, що вони знаходять свого читача і
роблять театр ближчим до кожного з нас. Цього разу хочу познайомити
маріупольців з талановитим чоловіком, про життя якого теж з легкістю можна
поставити виставу. Це \textbf{Андрій Анатолійович Луценко} – актор театру і кіно,
режисер-постановник, викладач Маріупольського коледжу мистецтв, учасник АТО.
Понад 70 ролей Андрієм зіграно в \emph{Івано-Франківському академічному національному
драматичному театрі}, але воєнні події привели талановитого актора на схід
України, до Маріуполя. Сьогодні в Маріупольському драматичному театрі актора з
довірою вводять у вистави та радіють, що Андрій став частиною театральної
сім'ї.

Чоловік родом із міста Бориспіль – одного з прадавніх міст Київщини. Актором
хотів стати з дитинства. Хоча хлопця цікавили й інші професії. Зокрема, він
мріяв стати слідчим, льотчиком чи військовим. У старших класах Андрій
більш-менш визначився і зрозумів, що бачить себе саме актором. Він брав участь
у шкільній самодіяльності. А пізніше грав у \emph{народному аматорському театрі
\enquote{Березіль}} в місті Бориспіль. 

\ii{03_10_2020.stz.news.ua.mrpl_city.1.andrij_lucenko_krascha_versia_sebe.pic.2}

Після армії вступив до Київського державного
інституту культури на факультет \emph{\enquote{театральна режисура}}, де отримав професію і
режисера, і актора. Але в житті більше полюбляє грати, ніж ставити вистави.
Немало ролей Андрієм було зіграно в Криворізькому, Білоцерківському театрах, та
найкраще розкрився талант актора в Івано-Франківському національному
драматичному театрі, в якому він почав працювати в 90-х роках і де успішно
відслужив 21 сезон. Саме там сформувався як актор і режисер-постановник.

До Майдану чоловік жив театральною сценою, він і подумати не міг, що колись
доведеться боротися за територіальну цілісність суверенної України. Мужній і
сміливий українець не міг стояти осторонь від подій, що сколихнули всю країну.
Під час Майдану Андрій знав, що має бути саме там. Дуже важко переживав смерть
своїх молодих однодумців. А з початком війни, недовго вагаючись, вирішив йти
служити. Ще юнаком він охороняв ділянку державного рубежу СРСР на стику з
Румунією й Угорщиною, тому у 2014 році попросився відправити його до рідних
прикордонників.

\ii{03_10_2020.stz.news.ua.mrpl_city.1.andrij_lucenko_krascha_versia_sebe.pic.3}

1 березня 2015 року підрозділ нашого героя заїхав до населеного пункту під
Маріуполем - \enquote{Зоря}, там розташовувалася дислокація його підрозділу. Кожен на
бойовому посту робив усе для того, аби вберегти не тільки власне життя, а й
своєчасно захистити свого бойового товариша. Після закінчення мобілізації
Андрій, на щастя, повернуся живим і здоровим до Івано-Франківська. Втім,
незабаром знову добровольцем пішов служити в шосту хвилю мобілізації, по
закінченні якої залишився на два шестимісячні контракти. Після цього сержант
Луценко повернувся до Івано-Франківська. Тоді чоловіку хотілося продовжувати
службу до перемоги. Але, побачивши, що цій війні немає кінця, він втомився
морально і вирішив повертатися додому. Звісно, в житті Андрія війна лишила свій
помітний слід і незабутній відбиток. Це не могло не вплинути на його творче
життя. Відбулася переоцінка цінностей, змінилося ставлення до людей. Актор з
вдячністю згадує допомогу всіх волонтерів.

До Маріуполя Андрія Анатолійовича привела доля. Коли Андрій йшов на війну, він
був холостий. Саме в Маріуполі він зустрів свою кохану жінку і майбутню дружину
– \emph{Наталю}. Вона була волонтеркою і робила маскувальні сітки, які називали
\enquote{кікіморами}. Саме тому волонтерок, які займалися їхнім створенням, називали
\emph{\enquote{Група маріупольських кікімор}}. Андрій та Наталя вирішили, що вони потрібні
одне одному, і одружилися. В Івано-Франківську подружжя прожило 1 рік, після
чого вони вирішили переїхати до Маріуполя, де залишилися рідні дружини. Актору
було складно покидати театр, в якому він служив довгий час. Але життя диктувало
свої правила.

В Маріуполі Андрій успішно пройшов прослуховування і конкурс, після чого він
був зарахований до трупи. Так, з кінця 2018 року Андрій Луценко став артистом
Донецького академічного обласного драматичного театру (м. Маріуполь). Нещодавно
Андрій Луценко, як режисер, \href{https://mrpl.city/news/view/ko-dnyu-politsii-v-mariupole-sostoyalas-premera-spektaklya-osnovannogo-na-realnyh-sobytiyah}{поставив виставу \enquote{Шлях додому}}.%
\footnote{Ко Дню полиции: в Мариуполе состоялась премьера спектакля, основанного на реальных событиях, Олена Онєгіна, mrpl.city, 04.07.2020, \par%
\url{https://mrpl.city/news/view/ko-dnyu-politsii-v-mariupole-sostoyalas-premera-spektaklya-osnovannogo-na-realnyh-sobytiyah}
}
Автор п'єси  – \emph{Ігор Тур}. Це унікальний спільний проєкт Національної поліції Донеччини та Донецького
академічного обласного драматичного театру (м. Маріуполь). Вистава присвячена
українцям, які з різних обставин залишились жити в окупованому Донецьку.
Спектакль про те, як зробити свій вибір і як прийти до свого дому. Режисер
наголошує, що \emph{\enquote{наш дім – це Україна, в нашому домі зараз йде війна. Є люди, які
виявилися зрадниками, але є й ті, хто чекає на звільнення...}}. Саме про цих
людей з діаметрально різним світоглядом і про їхній вибір – ця вистава.

\ii{03_10_2020.stz.news.ua.mrpl_city.1.andrij_lucenko_krascha_versia_sebe.pic.4}

Незабаром відбудеться ще одна \href{https://mrpl.city/news/view/premernye-spektakli-v-dramteatre-mariupolya-uvidyat-li-ih-zriteli}{%
прем'єра Луценка – комедія \emph{\enquote{Пограбування
опівночі}}}.%
\footnote{Премьерные спектакли в драмтеатре Мариуполя: увидят ли их зрители?, Ганна Хіжнікова, mrpl.city, 17.09.2020, \par%
\url{https://mrpl.city/news/view/premernye-spektakli-v-dramteatre-mariupolya-uvidyat-li-ih-zriteli}
} Це невелика вистава, дуже легка і позитивна, яка підійде для
відпочинку з сім'єю. У виставі задіяні такі яскраві актори як Ігор Кітриш,
Олена Біла, Ярослав Прикоп, Юлія Манько, Ігор Курашко (1 склад) та Дмитро
Нестеренко, Ігор Сігнаєвський, Ганна Німайєр, Олексій Панасюк, Надія Лавриненко
(2 склад).

Для Андрія немає улюблених ролей, але є ролі, які запам'ятали\hyp{}ся найбільше.
Однією з таких є роль Герострата у виставі \emph{\enquote{Забути Герострата}}. Більше
подобається грати негативних персонажів. З 2017 року Андрій почав зніматися в
кіно. Цікаво, що партнером по сцені у першому фільмі, де довелося грати нашому
герою (фільм \enquote{Інший}), був один з найулюбленіших акторів Андрія – народний
артист України \emph{Олексій Горбунов}. Луценка можна побачити в серіалі \emph{\enquote{Спіймати
Кайдаша}}, де він грає роль Миколи Довбуша, батька Мотрі. У жовтні на ТРК
Україна вийде серіал \emph{\enquote{Сага}}, в якому Андрій виконує роль реальної людини,
Євгена Сосновського – оперного співака, батька головної героїні. У серіалі
\emph{\enquote{Загадка для Анни}} актор зіграв голову правління банку, чоловіка головної
підозрюваної у вбивстві (9, 10 серії). Влітку Андрій знявся у чотирьохсерійному
серіалі \emph{\enquote{Втікачі}}, де актор грає суддю-корупціонера. А у фільмі \emph{\enquote{Моя жінка, мій
чоловік}} він зіграв голову будівельної компанії. Пропозиції зніматися і зараз
є, але Андрій віддає перевагу театру. Андрій Анатолійович наголошує, що в
Маріупольському театрі колосальна акторська школа, і всі його колеги -
професіонали з великої літери. Також Андрій викладає у Маріупольському коледжі
мистецтв і отримує велике задоволення від спілкування з майбутніми акторами.

Найбільше надихають яскравого актора любов до роботи, до драматургії. Андрій
Анатолійович дуже хотів би, щоб у Донецького академічного обласного
драматичного театру (м. Маріуполь) була розширена гастрольна діяльність, яка є
необхідною складовою для творчого життя будь-якого театру. Мріє, щоб наш театр
отримав своє ім'я. Наш герой наголошує що Маріупольський театр повинен носити
ім'я \emph{Василя Стуса} – одного із найактивніших представників українського
культурного руху шістдесятників, людини, життя якої міцно пов'язане з
Донеччиною, митця, що прагнув змін на краще та завжди мав незламну віру в
торжество правди і справедливості...

\begingroup
\em
\textbf{Улюблена книга:} твори австрійського письменника Стефана Цвейга.

\textbf{Улюблений фільм:} \enquote{Люди добрі} (2009 рік).

\textbf{Побажання маріупольцям:} \enquote{Бути добрішими, бути кращою версією себе, вірити в Бога....}.
\endgroup
