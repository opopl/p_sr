% vim: keymap=russian-jcukenwin
%%beginhead 
 
%%file 14_09_2019.stz.news.ua.mrpl_city.1.basketbolistka_natalia_klimova
%%parent 14_09_2019
 
%%url https://mrpl.city/blogs/view/ko-dnyu-fizkultury-i-sporta-basketbolistka-iz-mariupolya-nataliya-klimova
 
%%author_id burov_sergij.mariupol,news.ua.mrpl_city
%%date 
 
%%tags 
%%title Ко Дню физкультуры и спорта: баскетболистка из Мариуполя Наталия Климова
 
%%endhead 
 
\subsection{Ко Дню физкультуры и спорта: баскетболистка из Мариуполя Наталия Климова}
\label{sec:14_09_2019.stz.news.ua.mrpl_city.1.basketbolistka_natalia_klimova}
 
\Purl{https://mrpl.city/blogs/view/ko-dnyu-fizkultury-i-sporta-basketbolistka-iz-mariupolya-nataliya-klimova}
\ifcmt
 author_begin
   author_id burov_sergij.mariupol,news.ua.mrpl_city
 author_end
\fi

\ii{14_09_2019.stz.news.ua.mrpl_city.1.basketbolistka_natalia_klimova.pic.1}

Мариупольцы гордятся выдающимися спортсменами, жизнь которых связана с нашим
городом. Это \textbf{Александр Семенович Мишаков} - гимнаст и тренер, заслуженный мастер
спорта СССР, заслуженный тренер СССР; \textbf{Виктор Иванович Чукарин} - заслуженный
мастер спорта СССР, заслуженный тренер УССР, абсолютный чемпион Олимпийских
игр, мира, СССР; \textbf{Александр Александрович Сидоренко} - пловец, чемпион летних
Олимпийских игр 1980 года; \textbf{Вячеслав Николаевич Олейник} - украинский борец
греко-римского стиля, заслуженный мастер спорта Украины, чемпион летних
Олимпийских игр 1996 года и другие.

Но среди спортсменов, уроженцев Мариуполя, первое место по спортивным титулам,
безусловно, занимает \textbf{Наталия Генриховна Климова (Назембло).} Она - заслуженный
мастер спорта СССР, чемпионка Олимпийских игр 1976 года, чемпионка мира 1971
года, 1975 года, чемпионка Европы 1972 года, 1974 года, 1976 года, бронзовый
призёр V Спартакиады народов СССР. Ее стихия – баскетбол. В книге З. Хромаева и
А. Волошина \enquote{Баскетбол незалежної України} представлен рейтинг лучших
украинских баскетболистов среди женщин и мужчин. Список возглавили мастера –
легендарные баскетболисты Александр Белостенный и Наталия Климова...

\textbf{Читайте также:} 

\href{https://mrpl.city/blogs/view/mariupol-masshtabno-otprazdnuet-den-fizicheskoj-kultury-i-sporta}{%
Мариуполь масштабно отпразднует День физической культуры и спорта, Георгий Федоренко, mrpl.city, 09.09.2019}

Наташа Назембло родилась 31 мая 1951 года. Ее детство и юность прошли на
Карасевской улице Мариуполя в доме № 23. Путь в большой баскетбол у нее начался
на спортплощадке ее родной 11-й школы. Одна из учениц этой же школы, правда,
более позднего выпуска, говорила, что баскетбол у нее и одноклассников был на
первом месте. У Наталии Назембло было все, чтобы эта игра стала для нее
занятием на всю жизнь: рост 185 сантиметров и отличная прыгучесть. И главное -
для нее баскетбол был любимым занятием. Ее заметил уроженец нашего города
\textbf{\emph{Владимир Мартынов}}, игрок киевского \enquote{Будівельника}. Он увидел стройную девушку
на спортплощадке, отрабатывающую прыжки в высоту. Рассказал о ней тренеру
баскетболистов \emph{\textbf{Владимиру Степановичу Заморскому}}. Тот немедленно отправился в
Жданов (так тогда назывался наш город) и увез в Киев подававшую большие надежды
девушку, которой недавно исполнилось шестнадцать лет. Владимир Заморский
вспоминал: 

\begin{quote}
\em\enquote{Скажи ей выполнять упражнение, она его не бросит, пока не достигнет
ощутимого результата. Более послушного и упорного в обучении ребенка у меня не
было. Как-то показал девушку тренеру киевских динамовок Александру Леонову. Тот
мгновенно нашел для нее общежитие, взял на питание, повез на сборы. Наталья
делала бешеные шаги вперед. Играла настолько мощно, что ее сразу же включили в
юниорскую сборную СССР для участия в чемпионате Европы-1969 в Германии. Игры
проходили в Кёльне, Лёнене, Эссене, Хохенлимбурге и Хагене. Из Германии две
киевлянки - Любовь Остапец и Наталья Назембло привезли золотые медали}.
\end{quote}

Наталия продолжила свое образование в школе-интернате спортивного профиля.
Окончив его, она поступила на исторический факультет Киевского университета. И,
несмотря на то, что ее успехи в учебе были очевидны, она после первого курса
оставила университет. Киевская баскетбольная команда \enquote{Динамо} в начале 70-х
оказалась в сложном положении. Ряд игроков достигли возраста, когда усталость и
болезни давали о себе знать. Пришлось им уйти из команды. Как раз в это время
Наташа стала \enquote{динамовкой}. Команда эта никогда не занимала призовых мест. Тем
не менее, неизменно входила в десятку сильнейших клубов СССР. Наталию
приглашали в более успешные клубы, но она всегда оставалась верна \enquote{Динамо}.

Читайте также: 

\href{https://archive.org/details/21_07_2018.sergij_burov.mrpl_city.viktor_kalognomos}{%
Виктор Калогномос, mrpl.city, 21.07.2018}

К 1971 году Наталия Назембло была включена в сборную команду Украины, которая
заняла третье место в V Спартакиаде народов СССР. Финалы проходили в Москве с
16 по 30 июля 1971 года. Впервые на международной баскетбольной арене вместе со
\enquote{взрослой} сборной СССР прошел для Натальи Назембло Чемпионат мира 1971 года в
Бразилии, проходивший с 15 по 29 мая в пяти городах: Бразилиа, Сан-Паулу,
Ресифи, Нитерой и Араракуара. До того, как женский баскетбол в 1976 году был
включен в программу Олимпийских игр, участие, а уж тем более победа в мировом
чемпионате мира была мечтой для продвинутой баскетболистки. Итогом мирового
чемпионата мира в Бразилии для Наталии Назембло и ее подруг по команде стала
грандиозная победа и почетное звание чемпионок мира.

1972 год. Чемпионат Европы в Болгарии, соревнования проходили в Варне и
Бургасе. Сборная страны выиграла восемь игр из восьми. Первое место. Наталья
впервые становится чемпионкой Европы.

1974 год. Чемпионат Европы, снова восемь побед из восьми, теперь уже в Италии.

1976 год. Чемпионат мира в Колумбии. И вновь - восемь выигранных игр из восьми
у советской сборной. В том же 1976 году на Олимпийских играх в Монреале
баскетболистки стали первыми в истории спорта чемпионками. В 1975 году Наталья
Климова получила звание \enquote{Заслуженный мастер спорта СССР}.

1977 год ознаменовался для Наталии Генриховны окончанием Киевского
государственного института физической культуры. В 1978 году она перестала
участвовать в соревнованиях, а стала преподавателем в том же вузе, в котором
училась сама. За спортивные достижения она была награждена медалью \enquote{За
трудовую доблесть}.

\textbf{Читайте также:} 

\href{https://mrpl.city/news/view/po-stopam-shkolnikov-v-mariupole-proveli-sorevnovaniya-dlya-pervokursnikov-video}{%
По стопам школьников: в Мариуполе провели соревнования для первокурсников, Олена Онєгіна, mrpl.city, 13.09.2019}
