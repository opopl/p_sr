% vim: keymap=russian-jcukenwin
%%beginhead 
 
%%file 11_04_2021.fb.respublikalnr.2.rabota_trud_gvardia_soldaty
%%parent 11_04_2021
 
%%url https://www.facebook.com/groups/respublikalnr/permalink/794751221160671/
 
%%author 
%%author_id 
%%author_url 
 
%%tags 
%%title 
 
%%endhead 

\subsection{Газета "Республика" (№14, 2021г).  Дело мастера. Рабочей гвардии солдаты}
\Purl{https://www.facebook.com/groups/respublikalnr/permalink/794751221160671/}

В современном мире человек без навыков обработки материалов не преуспел бы ни в
одной сфере деятельности. Все механизмы и машины собраны из отдельных
металлических деталей, и, как ни крути, без станочников современная цивилизация
просто немыслима.

\ifcmt
  pic https://scontent-frt3-1.xx.fbcdn.net/v/t1.6435-9/173111410_121067170073391_3773157974321761956_n.jpg?_nc_cat=106&ccb=1-3&_nc_sid=b9115d&_nc_ohc=cd4T_k5uxQgAX8tgHiX&_nc_ht=scontent-frt3-1.xx&oh=3ba1eab623dffc0547da2108d19a6b54&oe=609AF907

	pic https://scontent-frt3-1.xx.fbcdn.net/v/t1.6435-9/172695050_121067163406725_8733534135067456652_n.jpg?_nc_cat=106&ccb=1-3&_nc_sid=b9115d&_nc_ohc=yaYnMDX6tHMAX9QbuNI&_nc_ht=scontent-frt3-1.xx&oh=2c276ef12b995e0af8059122857e9e3b&oe=609E8F08

	pic https://scontent-frx5-1.xx.fbcdn.net/v/t1.6435-9/171814798_121067150073393_8553299065490566513_n.jpg?_nc_cat=100&ccb=1-3&_nc_sid=b9115d&_nc_ohc=ZBEroFjROdkAX8ZTKAw&_nc_ht=scontent-frx5-1.xx&oh=d555b487f9eed864d9af8525b8aeb62e&oe=609BB44E

	pic https://scontent-frx5-1.xx.fbcdn.net/v/t1.6435-9/171852953_121067226740052_1790246377614184799_n.jpg?_nc_cat=110&ccb=1-3&_nc_sid=b9115d&_nc_ohc=36BcoPZEr2UAX_IVknk&_nc_ht=scontent-frx5-1.xx&oh=fad8bb81f9b9c1f029ebf929a4807c2d&oe=609B7AE6
\fi

Методов обработки металлов множество – это и сверление, и строгание, и
шлифование. Но сегодня мы расскажем о фрезерной и токарной металлообработке.
Различие между ними в том, что на фрезерном станке вращается режущий элемент, а
деталь неподвижно закреплена, в токарном станке наоборот – вращается
обрабатываемая деталь. И самое главное: что работа каждого станка не
представляется без искусного мастерства человека, в чьих умелых руках кусок
металла из заготовки превращается в новую сверкающую деталь.

\subsubsection{Токарные изящества}

Все чаще и чаще мы слышим о востребованности той или иной рабочей профессии.
Причины тому всем известны. Спустя годы после уничтожения мощнейшей
производственной базы Советского Союза, отечественная промышленность постепенно
возрождается из руин и ей уже недостаточно мастеров, которые много лет назад
пришли юношами на заводы. Требуются молодые силы, которым предстоит нести
вперед знамя производственных побед.

Чтобы стать мастером в любой профессии, необходимо учиться на протяжении всей
жизни. Современные технологии развиваются семимильными шагами и требуют того же
от специалистов на производствах. Но каждый профессионал начинал с азов,
полученных у стареньких, но надежных станков в училищах, техникумах и
колледжах. 

В учебной мастерской Луганского технологического колледжа шумно, как в
настоящем заводском цехе: будущие токари-универсалы оттачивают на практике
полученные теоретические знания, вытачивая на болванках изгибы деталей. За
процессом внимательно следит старший мастер Петр Ковальчук, который почти 50
лет связан с токарными станками.

– Я о профессии могу часами говорить, – началось наше знакомство с Петром Терентьевичем. – Это же самое настоящее искусство! 

Наш собеседник начал свой путь токарем в Киевском профессионально-техническом
училище. Производственная практика на авиаремонтном заводе и опытные наставники
привили ему, прежде всего, любовь к профессии. 

– Что и нашим ребятам стараемся передавать, – говорит старший мастер,
подчеркивая и важность условий, в которых обучаются будущие мастера. –
Мастерская у нас в хорошем состоянии, и это несмотря на то, что за последние 30
лет ни одной единицы нового оборудования не было. Но у нас спецы толковые, все
станки надежные, поддерживаются в отличном состоянии. 

Да, оборудование несовременное, но чтобы стать токарем, в первую очередь
изучается работа универсального станка. В колледже имеется шесть разновидностей
токарных станков, но принцип работы у всех один. Есть и импортные, но наиболее
распространены станки завода «Красный пролетарий». Кроме станков, учащиеся
полностью обеспечены инструментом. В свое время руководство колледжа проявило
дальновидность, и поэтому инструментальная кладовая забита всей необходимой
оснасткой. 

В первую очередь технологический колледж ориентируется на подготовку
специалистов для машиностроительной отрасли. Потребность в молодых кадрах в
Луганской Народной Республике есть, при том немалая. И учиться на токарей
приходят не только выпускники школ. Один из них Николай Лареоненко, который с
детских лет проявил любовь к металлу, наблюдая за работой отца механизатора.
Для молодого мужчины профессия токаря уже не первая. В Счастьинском
профессиональном лицее он учился на слесаря, но скудная материально-техническая
база не удовлетворила его. Поэтому и было принято решение поступить в Луганский
технологический колледж на токаря. В образовательном учреждении к Николаю
внимательно присмотрелись и, увидев в нем педагогические задатки, воспитывают
будущего мастера производственного обучения. 

Но главные задачи, которые ставят перед собой педагоги, – это найти подход к каждому молодому человеку, пришедшему на обучение. 

– Безнадежных детей для нас нет! – в один голос безапелляционно утверждают и
старший мастер Петр Ковальчук, и директор колледжа Андрей Димитриев.

– К нам приходят ребята с разными характерами, различной успеваемостью, –
продолжил руководитель образовательного учреждения. – Ведь чтобы овладеть
токарным мастерством, необходимо уметь и чертежи читать, и знать
материаловедение, математические науки… И каждый наш учащийся успешно осваивает
все дисциплины.

С нескрываемой радостью в Луганском технологическом колледже говорят о переломе
в отношении рабочих профессий в нашей Республике. Постепенно отходит в прошлое
негативная тенденция низкой оплаты тяжелого труда, а на передний план выходит
справедливый уровень заработной платы и поощрений. 

– Для нас самое главное, чтобы все наши выпускники получили достойный шанс на
будущее, – подчеркивает Андрей Димитриев. – И все наши педагоги, а это
высококвалифицированные специалисты – например, Петр Терентьевич отличник
образования Украины, у меня почетное звание заслуженного работника образования
Луганской Народной Республики, все имеют грамоты и благодарности. Но самая
большая награда для всех нас – это когда о наших обучающихся приходят только
положительные отзывы.

\subsubsection{Фрезерные тонкости}

С филигранной точностью уже 39 лет вытачивает углубления, отверстия и фигурные
профили на металлических заготовках будущих деталей Любовь Шнайдер. Фрезеровщик
5-го разряда, бригадир – она больше похожа на Золушку в исполнении Янины Жеймо
– такая же миниатюрная и белокурая. А будучи 14-летней девчонкой, приехавшей в
Луганск поступать в училище, выбор будущей профессии сделала исключительно из
эстетических соображений – понравилось само слово «фрезеровщик».

Конечно, нам привычнее видеть представительниц прекрасной половины человечества
у станка хореографического, чем фрезерного. И все же есть в тяжелом
машиностроении своя, особенная музыка.

Все годы своей трудовой деятельности Любовь Александровна отдала
машиностроительной отрасли, проработав 22 года на заводе имени Пархоменко, а 17
лет назад вслед за мужем перешла работать на Луганский
электромашиностроительный завод. 

– Работа совсем не монотонная, а больше творческая, – отмечает Любовь Шнайдер.
– Конструкторы что-то придумают, мы испытываем новое изделие, если видим, что
что-то не так, вносим свои замечания, и при доработке модели их обязательно
учитывают.

К слову, вопреки расхожему мнению, подкрепленному низкопробной украинской
пропагандой, машиностроительная отрасль в Республике не просто работает, но и
успешно развивается. Доказательством тому как раз и служит Луганский
электромашиностроительный завод, который производит оборудование, узлы и детали
для подвижного состава Российских железных дорог. Предприятие нуждается в
квалифицированных молодых кадрах, в связи с чем активно взаимодействует с
образовательными учреждениями Республики. В этом году обучающиеся колледжа и
студенты университета имени Владимира Даля прошли производственную практику, и
после защиты дипломов будут приняты на стажировку, после которой лучшие будут
приняты на работу.

– Главное в нашей работе – это качество. От точности нашей работы зависят жизни
людей, ведь детали устанавливаются в электровозы и тепловозы, – подчеркивает
Любовь Шнайдер. – Конечно же, как и в любом деле есть свои хитрости и тонкости.
Все это приходит с опытом – присматриваешься, как лучше, где быстрее.

Качество работы на ЛЭМЗ достигается несколькими путями, но залог успеха – это
оптимизм, которым заряжает всех сотрудников директор предприятия Евгений
Плотников. От чего работа на предприятии не просто идет, а стремительно летит,
словно стружка из-под фрезы станка.

– Сложности в нашей работе, конечно, есть, но все зависит от опыта наших
сотрудников, – отмечает начальник цеха ЛЭМЗ Александр Артамонов. – Коллектив
работает как единый механизм. По-другому у нас нельзя – только работа в
команде.

Есть на Луганском электромашиностроительном заводе и фрезерные станки с
числовым программным управлением, но никогда они не вытеснят с производства
ручной человеческий труд, который вдыхает жизнь в бездушный холодный металл.

Любовь ШНАЙДЕР, фрезеровщик 5-го разряда:

– Я ни дня не пожалела о своем выборе. У меня был прекрасный наставник, который
многому меня научил. И за годы работы у меня было много учеников, которым я
передала то, чему учили меня. Считаю, что фрезеровщик – прекрасная профессия.

Оксана ЧИГРИНА, 
фото Ксении ЩЕРБИНЫ 
и Надежды ПЕРЕСВЕТ
ГАЗЕТА "РЕСПУБЛИКА" (№14, 2021г).
%#газета #республика #Луганский_технологический_колледж_ЛНР #рабочие_специальности
