% vim: keymap=russian-jcukenwin
%%beginhead 
 
%%file 01_04_2021.fb.volga_vasilij.1.gogol
%%parent 01_04_2021
 
%%url https://www.facebook.com/Vasiliy.volga/posts/2862192660764814
 
%%author 
%%author_id 
%%author_url 
 
%%tags 
%%title 
 
%%endhead 

\subsection{ГОГОЛЬ - Малороссийский русский гений, который стал такой неразрешимой проблемой современной националистической Украины}
\url{https://www.facebook.com/Vasiliy.volga/posts/2862192660764814}

ГОГОЛЬ - Малороссийский русский гений, который стал такой неразрешимой проблемой современной националистической Украины.

Сегодня его День Рождения. И что с этим делать? Весь мир отмечает, а нам как
быть? Не уж-то мы признаем «Тройку птицу тройку», которая несется, как Русь, и
другие народы и государства «косясь, постораниваются и дают ей дорогу»?

Нет же. Не признаем. Мы откажемся уже в тридцатый раз от своей памяти, от своей
культуры, от своего языка, от своей истории… от самих себя откажемся, лишь бы
не воздать славы гению Николая Васильевича, самородка и литературного гиганта,
рожденного в Полтавской губернии Российской Империи и воспевшего красоту
Малороссии и верность свободного запорожского козака Вере Православной и Земле
Русской, от которой он не отступал, в отличие от нас современных, даже перед
лицом жестокой смерти.

Гоголь – это моя любовь и моя надежда. Он бесконечно сильнее новопоявившихся
злобных карликов, уцепившихся сегодня за власть на Киевских холмах, и он
победит их рано или поздно, даже если мы, уже совсем не такие, как его Тарас
Бульба, не сможем их победить. Мы все больше становимся андреями, предавшими
своего отца, Родину, товарищество и веру нашу ради призрачной польской услады.

Как же все таки точно нас Гоголь описал, как верно он нас чувствовал.

Поздравляю, друзья! 

Сегодня День Рождения великого нашего земляка – Николая Васильевича Гоголя!


\ifcmt
  pic https://scontent-ams4-1.xx.fbcdn.net/v/t1.6435-9/167579830_2862192637431483_2850865663371179429_n.jpg?_nc_cat=103&ccb=1-3&_nc_sid=730e14&_nc_ohc=Gmokioq0QdgAX9AEqCy&_nc_ht=scontent-ams4-1.xx&oh=71d747f6acf0085c192b56d454a282ed&oe=608F53D0
  width 0.4
\fi

