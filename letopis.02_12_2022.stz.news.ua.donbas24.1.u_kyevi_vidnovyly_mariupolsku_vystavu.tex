% vim: keymap=russian-jcukenwin
%%beginhead 
 
%%file 02_12_2022.stz.news.ua.donbas24.1.u_kyevi_vidnovyly_mariupolsku_vystavu
%%parent 02_12_2022
 
%%url https://donbas24.news/news/u-kijevi-vidnovili-mariupolsku-vistavu
 
%%author_id demidko_olga.mariupol,news.ua.donbas24
%%date 
 
%%tags 
%%title У Києві відновили маріупольську виставу
 
%%endhead 
 
\subsection{У Києві відновили маріупольську виставу}
\label{sec:02_12_2022.stz.news.ua.donbas24.1.u_kyevi_vidnovyly_mariupolsku_vystavu}
 
\Purl{https://donbas24.news/news/u-kijevi-vidnovili-mariupolsku-vistavu}
\ifcmt
 author_begin
   author_id demidko_olga.mariupol,news.ua.donbas24
 author_end
\fi

\ii{02_12_2022.stz.news.ua.donbas24.1.u_kyevi_vidnovyly_mariupolsku_vystavu.pic.front}

\begin{center}
  \em\color{blue}\bfseries\Large
Маріупольський театр авторської п'єси Conception представив киянам свою знакову
виставу \enquote{Рентген усміхнених сердець} 
\end{center}

Наразі маріупольський \href{https://donbas24.news/news/u-kijevi-pokazut-vistavu-pro-mariupol}{%
\emph{Театр авторської п'єси Conception}}%
\footnote{У Києві покажуть виставу про Маріуполь, Ольга Демідко, donbas24.news, 20.07.2022, \par%
\url{https://donbas24.news/news/u-kijevi-pokazut-vistavu-pro-mariupol}%
}
відродив свою діяльність у столиці і вже встиг своїми виставами приємно вразити
вибагливого київського глядача. Зокрема, постановкою театру \enquote{Обличчя
кольору війни}, заснованою на реальних по\hyp{}діях, колектив зміг голосно заявити
про себе. У ній учасники розповідали власні історії та ділилися спогадами і
емоціями, пережитими в окупованому Маріуполі. Театр представив свою
документальну виставу і на гастролях у Дніпрі, Вінниці, Черкасах. Згодом, за
досить короткий час, колектив підготував другу виставу — музично-поетичний
спектакль — \enquote{Тримаємось разом}, де актори продовжили розповідати пережите через
вірші талановитих сучасників. А 30 листопада маріупольський театр показав
спектакль, з якого розпочинав у 2019 році свою роботу у Маріуполі.

\textbf{Читайте також:} \emph{Маріупольці вразили своєю прем'єрою киян}%
\footnote{Маріупольці вразили своєю прем'єрою киян, Ольга Демідко, donbas24.news, 29.07.2022, \par%
\url{https://donbas24.news/news/mariupolci-vrazili-svojeyu-premjeroyu-kiyan}%
}

\subsubsection{Історія вистави \enquote{Рентген усміхнених сердець}}

З трагікомедії \enquote{Рентген усміхнених сердець} театр авторської п'єси
\enquote{Conception} розпочав свою роботу у Маріуполі. Це була аншлагова
вистава, яка мала неабиякий успіх та підкорила багатьох маріупольців. П'єсу для
постановки написав художній керівник театру та режисер \textbf{\emph{Олексій Гнатюк}}. При
цьому назва спектаклю йому наснилася.

\begin{leftbar}
  \begingroup
\emph{\enquote{Назва цієї вистави мені наснилася 9 лютого 2019 року. Я прокинувся і
одразу ж її записав. Потім з'явилася і сама п'єса. Саме завдяки цій
виставі я з акторами вирішив створити театр авторської п'єси}}, —
поділився спогадами режисер театру Олексій Гнатюк. 
  \endgroup
\end{leftbar}

\ii{02_12_2022.stz.news.ua.donbas24.1.u_kyevi_vidnovyly_mariupolsku_vystavu.pic.1}

Олексій Гнатюк не тільки зумів відродити виставу в Києві, але й додав декілька
важливих деталей для київського глядача. Всі події спектаклю розгортаються у
столиці. Актори театру чистою українською мовою розповідають історію молодого
перспективного письменника Максима Алчевського, який переживає творчу кризу.
Давній друг і агент чоловіка приводить до нього талановиту дівчину Ангеліну, що
приїхала з Фастова поступати на філологічний факультет. Дівчині нема де жити, і
агент просить Максима її прихистити. До того ж, Ангеліна — фанатка творчості
письменника, перечитала всі його романи і, надихнувшись творчістю письменника,
написала свій. Прочитавши її твір, Максим одразу ж розуміє, що це дуже
оригінальний стиль і свіжий погляд. Письменник переписує роман Ангеліни слово
слово, видавши за свій.

\begin{leftbar}
  \begingroup
\emph{\enquote{Сюжет цієї вистави для мене дуже зрозумілий, адже я багато разів у своєму
житті стикався з ситуаціями, коли люди видавали мої твори за свої}}, —
наголосив режисер спектаклю Олексій Гнатюк. 
  \endgroup
\end{leftbar}

\ii{02_12_2022.stz.news.ua.donbas24.1.u_kyevi_vidnovyly_mariupolsku_vystavu.pic.2}

Втім дівчина, у якої головний герой вкрав роман, виявилася настільки щирою і
неповторною, що чоловік не зміг не закохатися в неї. Але Ангеліна, яка теж має
почуття до Алчевського, не може зрозуміти, як він міг піти на такий низький і
безсоромний вчинок. Насправді, за змістом вистава зовсім не змінилася.
Щоправда, долучилося декілька нових акторів. Також роль агента, яку в Маріуполі
виконувала жінка, тепер став грати чоловік — талановитий і самобутній
маріупольський актор \emph{\textbf{Євген Сосновський}}.

\begin{leftbar}
  \begingroup
{\em\enquote{Неодноразово дивився цю виставу у Маріуполі. Був на її прем'єрі в 2019
році. \enquote{Рентген усміхнених сердец} став дебютною виставою театру і був
дуже популярним серед маріупольських глядачів. У Маріуполі я грав у
театрі \enquote{ДрамКом} і ніколи не думав, що через деякий час вийду у цій
виставі на київську сцену у складі театру \enquote{Conception}}}, — зазначив
актор Євген Сосновський.
  \endgroup
\end{leftbar}

\textbf{Читайте також:} \emph{Чи любив Марко Кропивницький Приазов'я: до дня створення українського реалістичного театру}%
\footnote{Чи любив Марко Кропивницький Приазов'я: до дня створення українського реалістичного театру, Ольга Демідко, %
donbas24.news, 27.10.2022, \par%
\url{https://donbas24.news/news/ci-lyubiv-marko-kropivnickii-priazovya-do-dnya-stvorennya-ukrayinskogo-realisticnogo-teatru}%
}

\subsubsection{Як пройшла прем'єра?}

Прем'єра вистави \enquote{Рентген усміхнених сердець} відбулася 30 листопада на
камерній сцені Київського академічного театру ляльок, що дуже символічно для
театру авторської п'єси \enquote{Conception}, адже вперше її показали у 2019 році теж
на сцені Театру ляльок, але в Маріуполі.

\begin{leftbar}
  \begingroup
\emph{\enquote{Це дуже цікава локація. Приємно співпрацювати зі справжніми професіоналами
своєї справи, які працюють в цьому театрі. Ми плануємо й інші свої
прем'єри показувати на цій сцені}}, — зауважив режисер театру.
  \endgroup
\end{leftbar}

Глядачі спостерігали за всім, що відбувається на сцені, затамувавши подих. Ця
вистава нікого не залишила байдужим, адже сама історія є досить повчальною та
проникливою.

\begin{leftbar}
  \begingroup
\emph{\enquote{Я встигла за час перегляду і поплакати, і щиро посміятися. Цей спектакль можна
дивитися і дітям, і дорослим. При цьому кожен винесе для себе щось
своє. Моя сім'я залишилася у захваті від цієї прем'єри. І ми
обов'язково прийдемо і на інші вистави цього талановитого колективу}}, —
зазначила глядачка Катерина.
  \endgroup
\end{leftbar}

\textbf{Читайте також:} \emph{\enquote{Тримаємось разом} — маріупольський театр \enquote{Conception} представить нову виставу}%
\footnote{\enquote{Тримаємось разом} — маріупольський театр \enquote{Conception} представить нову виставу, %
Ольга Демідко, donbas24.news, 13.10.2022, \par%
\url{https://donbas24.news/news/trimajemos-razom-mariupolskii-teatr-conception-predstavit-novu-vistavu-foto}%
}

\ii{02_12_2022.stz.news.ua.donbas24.1.u_kyevi_vidnovyly_mariupolsku_vystavu.pic.3}

Актори відчували, що глядач проживає разом з ними усе, що відбувалося на сцені,
адже цей взаємозв'язок дуже важливий і потрібний.

\begin{leftbar}
  \begingroup
\emph{\enquote{Дуже вдячні глядачам, що вони разом із героями вистави переживали їх долю
протягом усього перегляду, це відчувалось та наповнювало нас позитивною
енергією під час гри}}, — наголосила виконавиця головної ролі \emph{\textbf{Марина
Говорущенко}}. 
  \endgroup
\end{leftbar}

Водночас для акторів прем'єра вистави дуже була важливою подією, яка викликала
досить суперечливі почуття.

\begin{leftbar}
  \begingroup
\emph{\enquote{З одного боку я дуже радію, що ми відродили цю виставу, а з іншого — дуже
сумно, тому що розумієш, що свого минулого життя вже не повернути. Вперше, коли
побачила сценарій українською, було дуже незвично, але мені здається, що
українською наша вистава заграла новими кольорами. Працювати з Євгеном
Сосновським — це неймовірне задоволення, дуже талановитий актор і чудова людина
з великим сердцем. Але мені дуже не вистачало Ольги Сівак (вона грала Еллочку),
я все шукала поглядом її за кулісами, а коли розуміла, що сьогодні вона не з
нами, ставало дуже сумно}}, — розповіла актриса \emph{\textbf{Катерина Колмикова}}.
  \endgroup
\end{leftbar}

\textbf{Читайте також:} \emph{Які театральні проєкти та культурні заходи, присвячені Маріуполю, реалізуються закордонними митцями}%
\footnote{Які театральні проєкти та культурні заходи, присвячені Маріуполю, реалізуються закордонними митцями, %
Ольга Демідко, donbas24.news, 31.07.2022, \par\url{https://donbas24.news/news/yaki-teatralni-projekti-ta-kulturni-zaxodi-prisvyaceni-mariupolyu-realizuyutsya-zakordonnimi-mitcyami}%
}

\ii{02_12_2022.stz.news.ua.donbas24.1.u_kyevi_vidnovyly_mariupolsku_vystavu.pic.4}
\ii{02_12_2022.stz.news.ua.donbas24.1.u_kyevi_vidnovyly_mariupolsku_vystavu.pic.5}

Багато проблем виникло з костюмами, реквізитом і декораціями. Адже всі речі
залишилися у Маріуполі. Найбільш зворушливим моментом у виставі став відеоряд,
знятий ще у мирному та щасливому Маріуполі. Серед глядачів були і маріупольці,
які не могли в ці хвилини стримати сліз. Загалом режисер театру Олексій Гнатюк
залишився дуже задоволений роботою акторів і прем'єрним показом. 

\begin{leftbar}
  \begingroup
\emph{\enquote{Кожен з акторів після пережитих подій в Маріуполі дуже змінився. Всі стали
настільки наповненими, що ця вистава заграла новими, більш насиченими
фарбами. Також приємно вразив своїм перевтіленням і наш новий актор
Євген Сосновський. До речі, саме він переклав текст вистави українською
мовою}}, — поділився режисер театру Олексій Гнатюк. 
   \endgroup
\end{leftbar}

Самі ж актори також наголосили на тому, що вони тепер інші, тому і вистава
зазвучала по-новому. Зокрема, актор, який виконує головну роль, \textbf{\emph{Дмитро Гриценко}}
зауважив, що після усього пережитого, після того, як вони були змушені покинути
рідний дім, грати у виставі, яка асоціюється з іншим, мирним і спокійним
життям, емоційно дуже складно.

\textbf{Читайте також:} \emph{Драматичний театр Маріуполя виступив у Польщі — актори взяли участь у міжнародному фестивалі}%
\footnote{Драматичний театр Маріуполя виступив у Польщі — актори взяли участь у міжнародному фестивалі, %
Еліна Прокопчук, donbas24.news, 12.09.2022, \par%
\url{https://donbas24.news/news/dramaticnii-teatr-mariupolya-vistupiv-u-polshhi-aktori-vzyali-ucast-u-miznarodnomu-festivali}%
}

\ii{02_12_2022.stz.news.ua.donbas24.1.u_kyevi_vidnovyly_mariupolsku_vystavu.pic.6}

\begin{leftbar}
  \begingroup
\emph{\enquote{Виносить буквально з перших секунд вистави, як тільки чуєш вступну
музику, а коли заграла фінальна пісня —я взагалі мовчу, що відбувалося
усередині. Чув вже від декількох глядачів, які бачили цю виставу
раніше, в Маріуполі, що тепер актори грають її інакше. І справа не в
перекладі й нових персонажах, справа у тому, що в усіх нас змінилось
наше внутрішнє \enquote{я} після усього, що довелось пережити. І звісно ж, це
накладається і на те, що ми робимо. Не знаю, на краще це чи ні, але
просто сподіваюся, що нам вдалося достукатися до глядача і зробити так,
щоб на дві години він забув про все, що його турбувало і занурився в
нашу історію}}, — зазначив Дмитро Гриценко.
   \endgroup
\end{leftbar}

\subsubsection{Подальші плани театру}

Вже 10 грудня театр авторської п'єси \enquote{Conception} представить дитячу
виставу про святого Миколая. Заплановано благодійний показ для
дітей-переселенців та, ймовірно, ще один показ для всіх зацікавлених. Це буде
спектакль, який зможе відволікти маленьких переселенців та занурити їх у
справжню казку.

Водночас колектив театру розпочав співпрацю з українською письменницею \emph{\textbf{Наталією
Климась}}, яка пише п'єсу для наступної вистави.

\begin{leftbar}
  \begingroup
\emph{\enquote{У наших виставах ми показуємо дуже багато трагічних подій. Тому я
вирішив, що час трохи й посміятися. Наталія Климась, яка пише комічні тексти на
злободенні теми, запропонувала написати для нашого театру п'єсу. У новій
виставі ми з гумором покажемо ті проблеми, які постали перед переселенцями
сьогодні}}, — додав Олексій Гнатюк.
   \endgroup
\end{leftbar}

\ii{02_12_2022.stz.news.ua.donbas24.1.u_kyevi_vidnovyly_mariupolsku_vystavu.pic.7}

Раніше Донбас24 розповідав про новий \href{https://donbas24.news/news/vid-papirusu-do-gadzetu-novii-projekt-mariyi-ta-oleksandra-sladkovix}{\emph{театральний проєкт Марії та Олександра
Сладкових.}}%
\footnote{\enquote{Від папірусу до гаджету} — новий проєкт Марії та Олександра Сладкових, Ольга Демідко, donbas24.news, %
25.10.2022, \par%
\url{https://donbas24.news/news/vid-papirusu-do-gadzetu-novii-projekt-mariyi-ta-oleksandra-sladkovix}%
}

Ще більше новин та найактуальніша інформація про Донецьку та Луганську області
в нашому телеграм-каналі Донбас24.

ФОТО: з архівів Євгена Сосновського та Ольги Демідко
%\ii{02_12_2022.stz.news.ua.donbas24.1.u_kyevi_vidnovyly_mariupolsku_vystavu.txt}
