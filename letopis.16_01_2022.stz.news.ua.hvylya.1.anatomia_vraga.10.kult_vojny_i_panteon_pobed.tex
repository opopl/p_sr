% vim: keymap=russian-jcukenwin
%%beginhead 
 
%%file 16_01_2022.stz.news.ua.hvylya.1.anatomia_vraga.10.kult_vojny_i_panteon_pobed
%%parent 16_01_2022.stz.news.ua.hvylya.1.anatomia_vraga
 
%%url 
 
%%author_id 
%%date 
 
%%tags 
%%title 
 
%%endhead 

\subsubsection{Культ войны и пантеон побед}
\label{sec:16_01_2022.stz.news.ua.hvylya.1.anatomia_vraga.10.kult_vojny_i_panteon_pobed}

\enquote{Россия никогда ни на кого не нападала. Россия всегда защищалась}.
Одним из важных факторов в московской стратагеме выживания является культ
войны, внедряющийся в народе, как осознанная необходимость для выживания всей
государственной модели в целом. Русские, в отличие от нашего антисистемного
мировоззрения \enquote{тихого садочка}, где индивидуальные \enquote{хрущі над
вишнями гудуть}, являются коллективными экспансионистами, где прямая агрессия
называется \enquote{вынужденной защитой} от \enquote{территории хаоса}.

\enquote{Я хату покинул, пошел воевать, чтоб землю в Гренаде крестьянам
отдать}.  Своеобразное словесное камуфлирование агрессии необходимо Москве для
оправдания \enquote{исключительности} национальной формы нападения, как
\enquote{единственного возможного} выхода из конфликтной ситуации, которую в
Кремле определяют, как угрозу основным пунктам стратагемы выживания. Во всех
случаях государственной агрессии клерикальный фактор московской стратагемы
выживания \enquote{сакрализирует} нападение, выводя его на уровень
\enquote{освобождения}, \enquote{спасения}, освящая \enquote{жертвенность} во
имя победы и действуя в большинстве случаев синхронно со всеми государственными
институциями. Война является одним из главных смыслов в мировоззрении русского
человека, так как является эффективным механизмом осуществления одного из
основных пунктов государственной стратегемы выживания - экспансии.

Чтобы культ войны был эффективен, в нем не должно быть поражений. Это у нас
принято скорбеть по Крутам и Голодомору, рефлексировать о поражении Мазепы под
Полтавой, плакать о жертвах Батурина, бередить национальные раны Илловайском и
обороной Донецкого Аэропорта. Москва идет по принципиально иному пути. Культ
войны должен быть подкреплен пантеоном побед и героев, которые стали синонимами
этих побед. Только тогда он будет эффективен. Вся официальная историография
России — это череда практически непрерывных войн с громкими именами настоящих и
не очень героев - начиная от Александра Невского, Дмитрия Донского и Ивана
Сусанина заканчивая Денисом Давыдовым, Минихом, Суворовым, Александром
Матросовым, Зоей Космодемьянской, Рокоссовским и Жуковым.

Официальная власть в России, причем, во все исторические периоды не жалела и не
жалеет огромных средств не только на монументализацию своих побед в виде
названий городов и возведении памятников, но и на создание книжной,
театральной, кинематографической и игровой продукции, создающих
гиперболизированный ореол героики \enquote{войны по-русски}, будь то войны Смуты XVII
века, завоеваний Петра I, русско-турецких войн XVIII века или \enquote{не по плану}
начавшейся войны с Гитлером, заботливо Сталиным названной Отечественной - с
очень удобной аллюзией на Отечественную Войну XIX века, когда император
Александр I изгнал \enquote{хаос} Наполеона из России.

Каждая война Москвы для россиянина - это закономерный и исторический результат
\enquote{защиты}, которая увеличивает \enquote{национальную} территорию влияния, отодвигая от
столицы \enquote{фронтир хаоса}. Все поражения \enquote{эпизодичны} и являются лишь временной
передышкой для последующего реванша.

\enquote{Меня зарыли в шар земной..} (с) В России никто не ставит \enquote{свечек} о погибших
на войне, мертвые становятся \enquote{сакральной} и в то же время \enquote{формальной} жертвой
во имя высшей государственной и коллективной задачи - экспансии мировоззрения
\enquote{стабильного единоначалия}. Смерть на войне в России \enquote{обыденна} и российское
общество, в коллективном бессознательном, согласно терпеть жертвенность ради
государственных инициатив, в отличии от украинского социума, где
постколониальная антисистемность и индивидуализм, на глубинном уровне,
отвергает сам факт возможной насильственной физической смерти за продолжение
существования общей государственной модели, как чуждой украинцу конструкции.

Этой особенности россиян у нас и на западе очень многие не придают особого
значения, рассчитывая количество \enquote{русских гробов} в возможной масштабной
эскалации с Москвой. Болевой \enquote{порог} трупов у российского общества несоизмеримо
выше в силу объективных государственных и описанных нами исторических традиций.
Всем рекомендую внимательно почитать хронологию Зимней войны Кремля с финнами,
ознакомиться с потерями Советской армии во время Второй Мировой войны, о
потерях РФ в первую Чеченскую войну и осознать очевидную для историков формулу
- за свою государственную стратагему \enquote{борьбы с хаосом} Москва, как
государственная модель, осознанно готова платить физическими жизнями, как
чужих, так и своих граждан.



