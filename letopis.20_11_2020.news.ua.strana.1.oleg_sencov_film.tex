% vim: keymap=russian-jcukenwin
%%beginhead 
 
%%file 20_11_2020.news.ua.strana.1.oleg_sencov_film
%%parent 20_11_2020
 
%%url https://strana.ua/reviews/301950-film-oleha-sentsova-nomera-retsenzija-o-chem-film.html
%%author 
%%author_id 
%%tags 
%%title 
 
%%endhead 

\subsection{\enquote{Номер} ноль. О чем новый фильм Сенцова, который журналист \enquote{Страны} смотрела одна в зале в день премьеры}
\Purl{https://strana.ua/reviews/301950-film-oleha-sentsova-nomera-retsenzija-o-chem-film.html}

Анастасия Товт 09:16, сегодня

\ifcmt
pic https://strana.ua/img/article/3019/50_main-v1605821704.jpeg
caption О чем новый фильм Олега Сенцова \enquote{Номера}. Фото: Страна
\fi

В украинский прокат вышел фильм \enquote{Номера}. Он привлек к себе внимание тем,
что снят по одноименной пьесе Олега Сенцова, которую он написал задолго до
того, как оказался в российской тюрьме – еще в 2011 году.

Съемки начались в октябре 2018 года. Минкульт выделил на производство
фильма 10 миллионов гривен.

В трейлере к фильму сообщается: directed from Gulag - \enquote{срежиссировано из
Гулага} (Сенцов вышел на волю и вернулся в Украину лишь в сентябре 2019
года).

Срежиссировать фильм Олегу Сенцову помог крымскотатарский актер и
режиссер Ахтем Сеитаблаев. Рабочий процесс они обсуждали по переписке,
которую Сенцов вел из тюрьмы. Олег Сенцов утверждал актеров, а Сеитаблаев
воплощал замысел пьесы и работал с труппой на съемочной площадке. 

Фильм - копродукция Украины (435 FILMS) и Польши (Apple Film Production)
при поддержке Министерства культуры Украины и Польского института кино.

Премьера фильма состоялась в феврале этого года на Берлинском
международном фестивале (Берлинале). Со вчерашнего дня он идет в прокате в
украинских кинотеатрах.

Отметим, что сейчас снимается еще один фильм Сенцова - \enquote{Носорог}, на
который из бюджета Украины Госкино выделило 25 миллионов гривен.

Тем интереснее посмотреть на первый фильм бывшего заключенного.

Сразу скажем, зрительский интерес к картине пока небольшой. Мягко говоря.

В день премьеры в одном из кинотеатров Киева журналист \enquote{Страны} оказалась
единственным зрителем на сеансе.

\ifcmt
pic https://strana.ua/img/forall/u/0/25/WhatsApp_Image_2020-11-19_at_23.31_.19_(1)_.jpeg
caption На премьерный показ фильма Сенцова "Номера" в Киеве пришел один человек. Это была журналистка "Страны"
\fi

Интересно, что пьеса изначально была написана на русском, но сценарий был
переведен на украинский язык, чтобы получить финансирование Госкино
Украины.

При этом надписи \enquote{старт} и \enquote{финиш} на условном стадионе, где главные герои
участвуют в некоем чемпионате, – на русском. А в одежде и декорациях явно
прослеживается советская стилистика – намек на жизнь в тоталитарном
обществе?

Действие фильма фактически происходит на театральной сцене. Вероятно,
авторов идеи вдохновляли параллели с \enquote{Догвиллем} Ларса фон Триера (но
\enquote{Номерам} до \enquote{Догвилля}, конечно, в плане реализации и актерской игры
очень далеко).

По сути, "Номера" – это пьеса, снятая на камеру и переместившаяся в мир
кино. Герои и разговаривают по-театральному, иногда – даже чересчур. Из-за
перебора с театральным пафосом, особенно в диалогах, в происходящее на
экране мало верится.

У персонажей фильма нет имен – только порядковые номера от 1 до 10. Четные
номера – женщины, нечетные – мужчины. У каждого есть своя пара. У
номеров-героев нет выбора, как жить и даже кого любить. Каждый обязан
строить отношения со своей «парной» – четный номер-женщина с нечетным
номером-мужчиной.

При этом \enquote{встречаться} наедине они могут только в особый праздник –
\enquote{день передачи эстафетной палочки}. Мужчины и женщины ночуют отдельно –
на ночь два стража (их называют судьями) ставят перегородку посреди стадиона.

Все одеты в черные спортивные костюмы с белыми лампасами – они участвуют в
некоем чемпионате. Судьи вооружены винтовками и за неподчинение Правилам
угрожают дисквалификацией – смертью.

А Правила здесь, мягко говоря, странные. Герои обязаны есть и пить по
часам, а во время приема пищи – бегать. Потом каждый день бежать из одного
конца стадиона в другой, но не на скорость – кто первый, – а в
соответствии с порядковыми номерами.

Главные герои-номера живут по книге Правил, с которой вечно носится Номер
1, и поклоняются верховному правителю – Нолю, которого никто не видел.
Периодически герои спорят, существует вообще Ноль или нет – такая себе
аллегория с Богом. Сомнения в действиях Ноля герои называют
\enquote{нолехульством}.

Ноль-Бог – безмолвный наблюдатель, который ездит на подвесной конструкции
над стадионом, где разворачиваются события. Там у него своя мини-комната –
с диваном, шашлыком и пивом (в то время когда герои-номера обедают
кукурузными палочками).

Ноль влияет на то, что происходит внизу, на спортивном стадионе. Стучит в
барабан – гремит гром. Поливает цветы – идет дождь. Прикрывает сценический
софит – наступает вечер, открывает – наступает день. Связь с двумя судьями
он держит через телефон – набирает номер и отдает приказы.

\ifcmt
pic https://strana.ua/img/forall/u/0/25/WhatsApp_Image_2020-11-19_at_23.30_.40_.jpeg
\fi

Но все меняется, когда Номер 7 вступает в отношения не со своей \enquote{парной}
Восьмой – а с Четвертой. От этой связи внезапно в корзине посреди стадиона
появляется ребенок – вернее, кукла в виде ребенка. Его называют
Одиннадцатым.

На следующий день кукла превращается в мальчика, еще через день – в юношу.
Окончательно перестаешь понимать, что происходит на экране, когда этот
юноша вдруг провозглашает себя не Одиннадцатым, а «Первым из первых» –
сыном Ноля. Он стоит в позе креста и предлагает принести себя в жертву,
как Иисус.

\enquote{После обеда буду приносить себя в жертву. Говорят, Нулевых сыновей
периодически нужно казнить}, - говорит он.

Откуда-то на стадионе появляется гильотина, но в ключевой момент она не
срабатывает – Иисус остается в живых.

В итоге Седьмой, настоящий отец Одиннадцатого, устраивает "революцию".
Бросает чем-то в сторону судей, они отворачиваются, за это время он
подбегает к ним, вместе с Девятым номером отбирает у них винтовки и
провозглашает новый порядок. А-ля "жити по-новому". Седьмой объявляет
конец старого мира. Правда, как жить в новом мире, он пока и сам не знает.

В итоге Ноль свержен, и все начинают поклоняться Седьмому. Все становится
еще хуже, чем раньше – общественный порядок скатывается к языческой
диктатуре. Седьмой \enquote{убил Дракона}, и стал Драконом сам.

\ifcmt
pic https://strana.ua/img/forall/u/0/25/WhatsApp_Image_2020-11-19_at_23.30_.31_.jpeg
\fi

В целом, задумка фильма не нова (свергнувшие тиранию сами становятся
тиранами). Да и реализация подкачала.

Очень рафинированные образы: борец, стукач, безответно влюбленная,
любовница и т.д. Моментами фильм выглядит, как школьная постановка
Оруэловской притчи.

Аллегории тоже просты и считываются на первых минутах фильма. Правила –
символ Заповедей. Ноль – Бог. Чемпионат, в котором участвуют герои-номера
– сама жизнь. Точнее - жизнь в тоталитарном обществе по правилам, чей
смысл до конца никому не понятен. За любые попытки что-то изменить и
поддать сомнению – дисквалификация. То есть смерть. Расстрел.

Все попытки индивидуума мыслить нестандартного пресекаются на корню. Будь
как все, не выделывайся. Причем эта мысль тоже подается прямым текстом, \enquote{в
лоб}.

\enquote{Все ничего не делают, и ты повторяй за ними. Свистнули – беги, ещё раз
свистнули – поешь, а потом поспи. А в перерыве отдыхай. Разве это так
сложно?}, - говорит одна из героинь своему 
\enquote{парному}.

В фильме много странных и непродуманных моментов. Почему герои едят только
раз в день и только кукурузные палочки? Почему грозные судьи так легко
отдают оружие бунтарям, которых держали в страхе весь фильм, и вдруг
начинают подчиняться приказам Седьмого?

Общее впечатления - фильм слишком артхаусный для массового зрителя. И
слишком банальный для зрителя артхаусного. Можно предположить, что его
ждет трудная прокатная судьба.
