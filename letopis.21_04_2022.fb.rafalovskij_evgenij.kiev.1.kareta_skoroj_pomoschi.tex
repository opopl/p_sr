% vim: keymap=russian-jcukenwin
%%beginhead 
 
%%file 21_04_2022.fb.rafalovskij_evgenij.kiev.1.kareta_skoroj_pomoschi
%%parent 21_04_2022
 
%%url https://www.facebook.com/evgen.rafalovsky/posts/10221113731458287
 
%%author_id rafalovskij_evgenij.kiev
%%date 
 
%%tags 
%%title Друзья, благодаря Вам, помимо двух джипов мы приобрели карету скорой помощи
 
%%endhead 
 
\subsection{Друзья, благодаря Вам, помимо двух джипов мы приобрели карету скорой помощи}
\label{sec:21_04_2022.fb.rafalovskij_evgenij.kiev.1.kareta_skoroj_pomoschi}
 
\Purl{https://www.facebook.com/evgen.rafalovsky/posts/10221113731458287}
\ifcmt
 author_begin
   author_id rafalovskij_evgenij.kiev
 author_end
\fi

Друзья, благодаря Вам, помимо двух джипов мы приобрели карету скорой помощи!
Которая летит в Украину в Николаевскую область. Я уверен, она спасёт сотни
жизней!

Спасибо Всем и Каждому! Когда я вижу друзей, знакомых и даже совершенно
незнакомых людей которые желают помочь, я понимаю что  у добра и зла нет
национальности! Но как не прискорбно, в канун праздника Пасхи в глазах наших
убийц слишком много религии. Наша, святость она в наших сердцах и делах, мы
сами в ответе за свою Душу.

Вот наша история Друзья! 

Каждая великая мечта начинается с мечтателей. Однажды вечером,
\href{https://www.facebook.com/kostyantyn.storozhenko}{Kostyantyn Storozhenko}
предложил: - Может попробуем взять скорую помощь? И я подумал, а почему бы и
нет. Так начался наш путь к мечте.

\ii{21_04_2022.fb.rafalovskij_evgenij.kiev.1.kareta_skoroj_pomoschi.pic.1}

Промониторив авторынок, мы вышли на британскую машину и рискнули отправится  в
путь. Правда уже в Германии мы осознали, что наших финансов недостаточно для
безвизовой поездки Кости и покупки в Лондоне. Тогда мы принялись искать  машины
скорой помощи по доступной нам цене в Германии и Франции. И наш поиски
увенчались успехом: мы нашли скорые в пределах тех же 6-7 тыс. евро. 

\ii{21_04_2022.fb.rafalovskij_evgenij.kiev.1.kareta_skoroj_pomoschi.pic.2}

Но, оказалось, не все так просто как бы нам того хотелось. Изначально мы
сторговались с одним французом из-под Парижа.  Француз подтвердил сделку,
Костя купил билеты на поезд. Но, оказалось что наш продавец, еще тот жучара
навозный:  вечером отписался, что селяви, но машину отдаёт своему другу.  В
итоге мы потратили бесценное время и потеряли 140 евро, на билетах. А для нас
сейчас  каждая минута и копейка дорога. Но что поделаешь, жуки, увы, водятся
везде. От них никто не застрахован. 

\ii{21_04_2022.fb.rafalovskij_evgenij.kiev.1.kareta_skoroj_pomoschi.pic.3}

Следующая попытка, увы, тоже не сложилась. Предварительно нам выслали фото
машины, нас все устроило,  купили билеты, на автобус – нате-здрасте! По факту,
оказалось, что всю начинку со скорой выгребли: машина была без носилок, сирен и
мигалок. Опять облом. Хоть в этот раз билет был возвратный. 

Значит это не наша машина, говорю я Костику. И в следующее мгновения, открывая
новое объявление, мы не верим своим глазам, читаем : продается скорая в городе
Анси.  А в Анси живёт Костя! Ну вот, если судьба преподносит тебе лимоны —
сделай из них лимонад.

\ii{21_04_2022.fb.rafalovskij_evgenij.kiev.1.kareta_skoroj_pomoschi.pic.4}

Костик едет домой в Анси. Покупает, оформляет на меня скорую. Нужно прзнать это
моя первая покупка машины! Дальше через день подхватывает меня в Аугсбурге, и
мы на всех парах гоним машинку в Украину.

Я благодарен, что жизнь, Паганели и путешествия меня свели с таким ЧЕЛОВЕЧИЩЕМ
как Костик! Он очень крутой парень! Добрый, отзывчивый, знает много иностранных
языков. С ним можно и в огонь, и в воду. Проверено - ходили! Мы с ним вместе
\enquote{Корону в Перу прошли} и еще много чего было!

— Костя, давай так: ты рулишь, а я если что толкаю.

— Почему Жека, мы всегда всё должны делать только по-твоему?

— Ладно, пусть будет по-твоему: я толкаю, а ты рулишь.

— Вот это другое дело, поехали!

А дальше пронеслись три дня: \enquote{Чип и Дейл спешат на помощь}.  

- Костик, а давай хоть на весь лес врубим сирену и мигалки?

— Ты ненормальный?! 

Но сирену мы все-таки врубили, но чуть попозже. Ведь зачем нужны кнопки если на
них нельзя нажимать?

Звонит мне в дороге
\href{https://www.facebook.com/profile.php?id=100004658208358}{Mikołaj
Storożuk}, он когда-то со мной ходил на байдарке на черепашек. Но уже лет 6
живёт в Польше.

- Слушай у меня здесь есть пайки, продукты и 16 польских броников купили. Заберёшь на границу, сможешь передать нашим? 

- Да, конечно, без проблем! 

Николай говорит - Я гуманитарку тебе с Джоном привезу, он из Африки, с Замбии
вроде. Но я его не сильно понимаю, он по английский чешет, а я с английским не
очень дружу. Джон у нас сейчас волонтером работает. 

Я думал Джон черненький, а оказалось, что он беленький, как из фильма\enquote{
Миссионер с пулеметом}. Да, та самая знаменитая легкая родезийская пехота!
Сейчас он уже служит церкви и Богу. Время искупить грехи, поэтому он здесь!
Бог привёл его сюда, а Джон позвал меня к себе на Замбези, у него там свой
лодж. Поэтому, надеюсь, что в скором будущем, после Победы, увидимся на Замбези
или в Кейптауне.

По пути мы встречали только хороших людей. Во Вроцлаве, мы остановились у Саши.
Когда-то мы с ним прошли вместе Перу и Патагонию. Он нас приютил, накормил,
напоил и дал еще позитивчика в дорогу.  А Макс не взял ни копейки и поменял в
скорой винт в креплении к амортизатору, чтобы не стучало колесо. Оля заказала
нам новые наклейки на скорую. Еще один Максим в Перемышле загрузил нас едой.
Так вот, с миру по нитке и шубка готова!

Три дня пути и вот мы на границе Польши и Украины. Я верю, когда-то этих границ
не будет. И нам - людям свободного мира, можно будет также открыто
путешествовать по всему Миру.  

Машинку мы передали нашему Паганелю в Николаев,
\href{https://www.facebook.com/maxim.beznosenko}{Максим Безносенко}. Сейчас там
горячо. Он ее отправит в нужное место, медикам. Будем ждать фоток с фронта!
Пусть летает и спасает жизни людей.

Помните, Друзья, что в Вас есть сила, терпение и страсть, чтобы дотянуться до
своей мечты!  Душа — наш парус. Ветер — жизнь. Пусть ведёт нас через океан
войны к берегам Мира! 

Я понимаю,  что сейчас тяжелые времена для всех, но, если у Вас осталась
возможность финансово помочь, пожалуйста, пишите нам в личку. Буду очень
благодарен любой сумме. Всех обнял, работаем дальше!

\ii{21_04_2022.fb.rafalovskij_evgenij.kiev.1.kareta_skoroj_pomoschi.cmt}
\ii{21_04_2022.fb.rafalovskij_evgenij.kiev.1.kareta_skoroj_pomoschi.orig}
\ii{21_04_2022.fb.rafalovskij_evgenij.kiev.1.kareta_skoroj_pomoschi.cmtx}
