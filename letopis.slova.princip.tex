% vim: keymap=russian-jcukenwin
%%beginhead 
 
%%file slova.princip
%%parent slova
 
%%url 
 
%%author 
%%author_id 
%%author_url 
 
%%tags 
%%title 
 
%%endhead 
\chapter{Принцип}
\label{sec:slova.princip}

%%%cit
%%%cit_pic
%%%cit_text
Ни один \emph{принцип} в истории человечества не унес столько миллионов
человеческих жизней, как \enquote{\emph{принцип} священной территориальной
целостности}. Когда какая-либо часть страны желает отделиться или получить
автономию в силу тех или иных причин, то на ее пути встает это \emph{жуткий
принцип}.  В угоду этому \emph{принципу} все государства принудительно толкали
и продолжают толкать в топку войны массы своих затюканных и малообеспеченных
граждан, которые не смогли откупиться от военных чиновников, потому что
сознательно, будучи в здравом уме, отдавать свою жизнь за этот \emph{принцип}
просто невозможно.  Ведь этот \emph{принцип} чисто государственнический, то
есть бюрократический. А разве может здравомыслящий человек добровольно отдавать
свое здоровье и жизнь за бюрократию и за ее интересы?
%%%cit_comment
%%%cit_title
\citTitle{Мы живем в мире, когда размер перестал иметь значение}, 
Андрей Головачев, strana.ua, 13.06.2021
%%%endcit

%%%cit
%%%cit_pic
%%%cit_text
\emph{Принцип} территориальной целостности или по другому \enquote{принцип никого никуда не
пущать} равносилен принципу запрета на разводы супругов. Представьте что в
стране действует запрет на развод. Глупо, да? А ведь когда-то такие
антигуманные \emph{принципы} действовали. И принуждали супругов мучаться друг с другом
до конца дней своих. Врагу не пожелаешь! Сейчас такой запрет воспринимается как
анахронизм.  Когда то (надеюсь!) таким же антигуманным анахронизмом будет
признан и \emph{принцип} священной территориальной целостности. Скоро логически
обосновать его смысл будет уже невозможно. И чем раньше это произойдет, тем
больше жизней на планете будет сохранено
%%%cit_comment
%%%cit_title
\citTitle{Мы живем в мире, когда размер перестал иметь значение}, 
Андрей Головачев, strana.ua, 13.06.2021
%%%endcit

