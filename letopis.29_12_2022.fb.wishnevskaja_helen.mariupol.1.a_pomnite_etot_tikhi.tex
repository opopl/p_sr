%%beginhead 
 
%%file 29_12_2022.fb.wishnevskaja_helen.mariupol.1.a_pomnite_etot_tikhi
%%parent 29_12_2022
 
%%url https://www.facebook.com/helen.wishnevski/posts/pfbid02d5mpksg8Jee7J4coN5LVCu8s21FWFFA4DGW5B6GPzqokPoGpowWR6RWMq19cUgJEl
 
%%author_id wishnevskaja_helen.mariupol
%%date 29_12_2022
 
%%tags mariupol,more.azov,more,foto,voda,pamjat
%%title А помните этот тихий шёпот прибоя?
 
%%endhead 

\subsection{А помните этот тихий шёпот прибоя?}
\label{sec:29_12_2022.fb.wishnevskaja_helen.mariupol.1.a_pomnite_etot_tikhi}

\Purl{https://www.facebook.com/helen.wishnevski/posts/pfbid02d5mpksg8Jee7J4coN5LVCu8s21FWFFA4DGW5B6GPzqokPoGpowWR6RWMq19cUgJEl}
\ifcmt
 author_begin
   author_id wishnevskaja_helen.mariupol
 author_end
\fi

А помните этот тихий шёпот прибоя? А свежие брызги, которыми оно осыпало, когда
волновалось? А как оно приветливо обнимало волнами и начинало заигрывать, когда
в знойный день ты с разбегу  рассекаешь воду и с наслаждением плюхаешься в нее?
И кажется,  что вода просто закипает, соприкаясь с твоим раскаленным  солнцем
телом.  А когда на Песчанке ныряешь на глубине,  а выныриваешь на мели,
счесывая себе живот и коленки?) Конечно же, вы помните. Да, оно не самое
большое,  не самое глубокое,  но для нас оно самое ласковое, любимое и родное.
Наше, Азовское ..
