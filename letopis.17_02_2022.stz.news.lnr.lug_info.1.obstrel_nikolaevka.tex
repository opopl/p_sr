% vim: keymap=russian-jcukenwin
%%beginhead 
 
%%file 17_02_2022.stz.news.lnr.lug_info.1.obstrel_nikolaevka
%%parent 17_02_2022
 
%%url https://lug-info.com/news/artobstrel-so-storony-vsu-stal-polnoj-neozhidannost-yu-dlya-zhitelej-nikolaevki
 
%%author_id news.lnr.lug_info
%%date 
 
%%tags donbass,ukraina,vojna,vsu
%%title Артобстрел со стороны ВСУ стал полной неожиданностью для жителей Николаевки
 
%%endhead 
 
\subsection{Артобстрел со стороны ВСУ стал полной неожиданностью для жителей Николаевки}
\label{sec:17_02_2022.stz.news.lnr.lug_info.1.obstrel_nikolaevka}
 
\Purl{https://lug-info.com/news/artobstrel-so-storony-vsu-stal-polnoj-neozhidannost-yu-dlya-zhitelej-nikolaevki}
\ifcmt
 author_begin
   author_id news.lnr.lug_info
 author_end
\fi

Артиллерийский обстрел со стороны ВСУ стал полной неожиданностью для жителей
села Николаевка, относящегося к Луганску. Об этом с места события передает
корреспондент ЛИЦ.

\ii{17_02_2022.stz.news.lnr.lug_info.1.obstrel_nikolaevka.pic.1}

ВСУ обстреляли Николаевку из артиллерийских орудий калибра 122 мм. В
направлении населенного пункта было совершено 15 выстрелов. В результате
обстрела были повреждены несколько многоквартирных жилых домов, газопровод,
производственное помещение маслоцеха, около 500 абонентов остались без
электроснабжения.

\ii{17_02_2022.stz.news.lnr.lug_info.1.obstrel_nikolaevka.pic.2}

\enquote{Я была в зале в момент обстрела. Снаряд упал возле дома, но трясло так, что аж
стены трещали. Осколками мне разбило балкон, окна, двери, мебель}, - рассказала
жительница села Анна Михайлова.

\ii{17_02_2022.stz.news.lnr.lug_info.1.obstrel_nikolaevka.pic.3}

Она отметила, что обстрел начался внезапно: хотя \enquote{Николаевка и находится на
линии соприкосновения}, но этот район считается \enquote{относительно тихим}.

\ii{17_02_2022.stz.news.lnr.lug_info.1.obstrel_nikolaevka.pic.4}

\enquote{Нас в конце прошлой осени (в 2021 году) обстреляли, но мы надеялись, что это
единичный случай и больше такое не повторится, но сейчас уже ни на что не
надеемся, похоже взялись (ВСУ) за наше село основательно}, - констатировала
женщина.

\ii{17_02_2022.stz.news.lnr.lug_info.1.obstrel_nikolaevka.pic.5}

Жительница села Анна Лебедева рассказала, что жертв во многом удалось избежать,
потому что взрослые в момент обстрела были на работе, а дети - в школе и
детсаду.

\ii{17_02_2022.stz.news.lnr.lug_info.1.obstrel_nikolaevka.pic.6}

\enquote{Внезапно все (началось). Я вещи вешала, только постирала и зашла в хату (с
балкона в квартиру), секунда дела - и выбило все: и балкон, и в кухню залетело.
Слава богу, дети в садике и школе были, а так бы даже боюсь представить, что
могло быть}, - поделилась подробностями женщина.

\ii{17_02_2022.stz.news.lnr.lug_info.1.obstrel_nikolaevka.pic.7}

Напомним, в ноябре минувшего года семь жилых домов в селе Николаевка получили
повреждения в результате обстрела со стороны ВСУ.

\ii{17_02_2022.stz.news.lnr.lug_info.1.obstrel_nikolaevka.pic.8}

Ранее представительство ЛНР в Совместном центре по контролю и координации
режима прекращения огня сообщило, что ВСУ с начала суток 10 раз нарушили \enquote{режим
тишины}. Обстрелам подверглись районы Золотого-5, сел Сокольники, Веселенькое,
Нижнее Лозовое, Лозовое, Раевка и поселок городского типа Донецкий. Село
Николаевка ВСУ обстреляли из артиллерийских орудий калибра 122 мм, в результате
чего был поврежден газопровод, около 500 абонентов остались без
электроснабжения.

Позднее в СЦКК проинформировали, что киевские силовики сегодня днем еще 10 раз
нарушили \enquote{режим тишины} в зоне ответственности ЛНР. 

Власти Украины начали силовую операцию против Донбасса в апреле 2014 года.
Урегулирование конфликта базируется на Комплексе мер по выполнению Минских
соглашений, подписанном 12 февраля 2015 года в белорусской столице участниками
Контактной группы и согласованном с главами стран - участниц \enquote{нормандской
четверки} (Россия, Германия, Франция и Украина). Документ, в частности,
предусматривает прекращение огня и отвод тяжелых вооружений от линии
соприкосновения. 
