% vim: keymap=russian-jcukenwin
%%beginhead 
 
%%file topics.vojna.msg.ukr
%%parent topics.vojna.msg
 
%%url 
 
%%author_id 
%%date 
 
%%tags 
%%title 
 
%%endhead 

https://www.facebook.com/groups/story.kiev.ua/posts/1914763742053751/

Kiev Fortress Валентина Фищук Відкрийте будь-ласка, твори Григорія Саввича
Сковороди - хоча він й є наш великий український філософ, але писав свої твори
про Бога та Людину російською та латиною. Його теж треба викорінити? І щодо
латини та мови. Специфіка будь-якої мови полягає в тому, що її неможливо просто
так "викорінити", як би те комусь не хотілось би. Будь-яка мова, і російська в
тому числі, є живим організмом, який живе за власними законами, і який
неможливо просто так загнати в якісь межі законами або ж циркулярами. Ось та же
латинська мова. Римська імперія у свій час жорстоко насаджувала свій лад у
поневолених країнах, не менш жорстокіше ніж росімперія в Україні або Польщі або
британська імперія в індії. Римська імперія давно вже зникла, а латина живе.
Хоча й мертва мова в тому, що нею ніхто не розмовляє в звичайному житті, але
вона живе в словах, живе в творах, живе як мова богослужіння. Латина - офіційна
мова Ватикану. Також, "специфіка" , календар, інформація, реформація - всі ці
слова є спадком жорстокої, аморальної римської імперії, в якій Нерон вбив свою
матір та спалив Рим, а Калігула вїжджав в сенат на коні. Тим не менш,
незважаючи на все це, латинська мова пережила 15 сторіч, і вже ніколи не
зникне, хоча й існує лише як писемна мова. Тим більше російська мова, як мова
живого спілкування, як мова, що розповсюджена далеко за межами московії,
переживе і криваву зловісну брехливу московію, переживе і минулу свою імперську
провину перед Україною, і буде жити далі, і в Україні зокрема. А якщо вам не
подобається росмова під час війни - адресуйте, будь-ласка, зокрема свої
занепокоєння до конкретних українців, що розмовляють росмовою і широко відомі в

Відкрийте будь-ласка повість Захар Беркут Івана Франка... і перші рядки це
будут рядки з Пушкіна (цитата з поеми Руслан і Людмила). як це не дивно. Пушкін
не обмежується лише Полтавою або ж Клеветникам России, і зокрема наш Каменяр,
Іван Франко досить високо цінував творчість Пушкіна. Але звісно, гнів та
ненависть з боку українців до сучасної московії є абсолютно законна річ,
злочини сучасної московії просто перетнули всі можливі людські межі, але все ж
таки тут важливо підходити до таких речей, як знесення памятників та
перейменування вулиць, з більшим розумом, ніж це зараз відбувається. На наш
погляд, це є явне розпорошення розумових та емоційних сил, які можна було би
більш продуктивно витратити на допомогу нашим хлопцям, зокрема Маріуполю.

https://www.facebook.com/kyjeslav.borshchahivets/posts/5195164247219634

У вас тута трохи якась каша. То парк чудовий, то він слугував русифікації. Ну -
слугував, ну і шо. росімперія давно вже вмерла - а чудовий парк залишився. От
ви коли ідете в Київскому метро, яке відкрили в 1960 році, ви ж не
задумувались, що його будувала компартія, яка винна в багатьох злочинах. срср
вмер - а чудове гарне метро залишилось, хоч би якою тоталітарною державою совок
не був. Або ж гімн Києва, Як тебе не любити, Києве Мій. Теж був створений в
совку в 1962, і взагалі його першим виконавцем був росіянин, Юрій Гуляєв. І
тому подібне. А щодо Пушкіна, відкрийте Захар Беркут Івана Франка, і там ви
побачите цитату Пушкіна. Іван Франко взагалі досить високо оцінював творчість
Пушкіна. Так само в ширшому значенні і творчість поетів та письменників. Можете
знести хоч всі памятники, всі до одного, але творчість поетів та письменників,
зокрема Пушкіна, буде продовжувати жити в серцях багатьох людей, буде
продовжувати жити також на сторінках книг (через цитати, згадки і тому
подібне), і зокрема також творів українських письменників, таких як Іван
Франко, бо ж Пушкін не обмежується лише Полтавою або ж Клеветникам России, і
звісно справжня цінність Пушкіна не вимірюється кількістю памятників. Щодо
порівняння московії і російської імперії, знаєте, сучасна московія, тобто
путінська московія, навіть гірша і за росімперію і за срср, бо московія, бридке
хиже утворення на тілі Землі, зараз нищить те, що будували таким великим коштом
і такими великими зусиллями і росімперія, і срср, як-от найбільший у світі
літак Мрія АН-225. московія не має ніякого відчуття цінності і власне Пушкіна,
бо вона бомбить Місто, Вічне Місто Київ, де той самий памятник і знаходиться
(або ж Одесу, де навіть є музей Пушкіна).

розумієте, тут ви теж впадаєто в крайнощі, тому що не ви єдиний, хто любить
Україну. І кожен любить Україну трохи по-своєму. Жодного відношення не мають,
чи мають... Ви що у нас геній на кшталт Франка, що є такі самовпевнені і у
всьому на світі розбираєтесь? Все у світі так чи інакше має відношення до
України, тому что Україна є частиною цього великого світу. Крім того, у вас
немає ніякої особистої монополії вирішувати, що має відношення до України, а що
ні. Це справи громади або віча, де збираються люди, що живуть безпосередньо у
Києві, наприклад. Ось громада збирається і вирішує, бути тим чи іншим
памятниками чи ні в цьому районі чи ні. І судячи з досить жвавої дискусії,
якраз досить велика частина української громади вважає, що або це питання не на
часі, або ж памятник треба залишити, так, як він і був. Суть України полягає ж
не тільки у Пушкіні, чи Шевченкові, а у тому, чи здатні ми сукупно та мирно
вирішувати ти чі інші спірні питання. В цьому і є суть демократії. Ми самі
вважаємо, що це питання абсолютно не на часі, бо це тільки розпорошує наші
сили, зараз війна зі страшним аморальним ворогом, ви хіба забули. А після війни
- проголосуємо, обговоримо - якщо що, знесемо, нема питань.

визволяти. справа не зовсім у цьому. це лише один із словесних приводів до
вторгнення, і не самий найголовніший. Основною тезою роспропаганди, на нашу
думку, є міф про те, що в Україні "неонацисти", які служать Заходу, який хоче
знищити расєю-матушку. І вбивця - якщо він хоче вбити - він привід знайде.
Стоїть Пушкін, є ті, хто розмовляє російською - йдемо визволяти молодших
братів. Пушкіна звідусіль знесли, всі перейшли на українську - йдемо мститися
за Пушкіна, Україна вже повністю бандерівська стала, так чому б не знищити її
взагалі одним ядерним ударом. Розумієте. Вони там просто всі як скажені звірі
стали. От ви самі можете зайти на однокласники, і перевірити. Як барани без
мізків, що їм вкладуть, у те вони й вірять. І привід до насильства завжди
можна знайти. І тут справа не в тому, що ми персонально Пушкіна любимо чи ні.
Справа в принципі демократії та громадянського суспільства. Україна зможе
протистояти московії тоді, коли вона стане по справжньому згуртованою, а одним
із чинників цієї згуртованості є в тому, що треба навчитися спірні питання
вирішувати по-дружньому, злагоджено. Іще головне питання щодо московії є в
тому, як зламати, знищити саму ідеологію московії. Але це складне питання, яке
потребує напружених інтелектуальних зусиль.

ви не туди дивитесь. Розумієте, можна спалити книжки (що є поганий прецендент
сам по собі, і тут справа не в змісті цих книжок, а в самому сутті цього явища
спалювання книжок,  що ставить інформаційно для світу нас як варварів ),
викинути памятники пушкіну, але саму мову - в сенсі інструменту людського
спілкування  по собі неможливо просто так знищити. Сама історія заборон
української мови тут є найкращим свідченням. Її забороняли, принижували - а
українська мова все це пережила, і нарешті розквітає по справжньому в
незалежній Україні. І тут справа не в російській мові як такій, а в самому
понятті мови як перш за все інструменту спілкування. Так, можна казати, що
росмова це мова путіна, мова оккупанта, давайте все заборонимо, все спалимо, і
тому подібне, але від того нікому краще не буде. Такі штампи є просто
неправильним осмисленням самого факту росмови в Україні, і що з нею далі
робити. І тут також проблема в тому, що люди не усвідомлюють суть самого явища
мови. Будь-яка мова, українська, російська, англійська - це живий організм сам
по собі, який живе за своїми законами. Так що ви можете бути персонально проти
росмови в Україні, але факт є факт. росмова в Україні є і буде на час нашого
життя, і напевне буде ще надовго, а може і назавжди. Вам напевне це не
подобається, і ви цього не хочете, але це факт. Ось наприклад, ви ж знаєте, що
існує латина - мова Римської імперії. Хіба Рим був краще за московію. Може й
гірше у свої часи, аморальна, жорстока імперія... Але латина надовго пережила
Рим, і продовжує жити в медицині та запозиченнях, і навіть є офіційною мовою
Ватикану. Сковорода наприклад писав латиною (крім росмови), як і багато інших
філософів і вчених. А росмова - це жива мова, і нею говорять не тільки хуйло з
скабеєвою, а наприклад скажімо також той герой родом з Харкова або Одеси, що
вчора підбив черговий ворожий танк десь на Донбасі. Так що питання не в
існуванні росмови як такої, а питання в тому, які меседжі, які смисли генерують
ті люди в Україні, що розмовляють росмовою. Якщо досить велика частина
російськомовних людей в Україні є патріотами України, а таких дуже-дуже багато,
дуже гарних, розумних, хоробрих людей, щирих серцем і душею, якщо вони
підтримують та захищають Україну, так в чому питання? І більш того. Якщо ми
хочемо знищити московію, ми в тому числі повинні знищити її ідеологічно, тобто
перепрошити мізки зомбованим баранам запоребріком, щоби вони більш не
загрожували Україні. А цю контр-пропаганду неможливо згенерувати інакше як
росмовою, бо москалі в більшості своїй просто ніяких інших мов просто не
знають.

% Охтирка оперативна
https://www.facebook.com/groups/498652738278706/?multi_permalinks=552444902899489&comment_id=553493676127945&notif_id=1652342891110460&notif_t=feedback_reaction_generic&ref=notif

Kiev Fortress Julie Tolmacheva так, звичайно. Але ви неуважно читали, що ми
хотіли донести. Так, це мова ворога, звісно. Але це також і мова друзів. До
біса українців, справжніх патріотів України, і не тільки українців, а й друзів
України по всьому світу, розмовляють росмовою, збирають гроші, виходять на
демонстрації на підтримку України, знімають кліпи, пишуть в телеграмах,
фейсбуках, при цьому абсолютно будучи патріотами України. Наприклад, ви
телеграм канал Казанського не читаєте, напевно. Або хроніки Харкова Соловйова.
А чому. Нащо оце зациклення на московії. Що, тільки московія розмовляє
росмовою, чи що. Що, хуйло персонально винайшов росмову, чи написав хоч один
словник росмови, що кажуть, що це мова путіна. Он наш Григорій Саввич Сковорода
набагато гарніше писав на росмові, ніж уся потворна московія разом. Або ми їх,
або вони нас. Абсолютно вірно. Ми маємо справу з жахливим, брехливим,
аморальним ворогом. Але перемогу України над московією, якщо брати ідеологічний
вимір, неможливо досягти простим тотальним відторгненням усього російського.
По-перше, це неможливо в принципі, як би вам не хотілося цього досягти. А
по-друге. А що ж потрібно, власне кажучи. Тут потрібен розум. Крім емоцій і
законної люті, потрібно дуже багато розуму. Дуже-дуже багато розуму, науки.
Вивчення філософії, психології, математики, фізики, суспільних наук. Ми повинні
стати настільки розумними, наскільки можемо. Бо якщо ми не станемо розумними,
ми не зробимо нову якісну зброю, яка буде захищати нас від ворога. Якщо ми не
станемо розумними, ми не побудуємо кращий суспільний лад в Україні. Якщо ми не
станемо розумними, ми не створимо нові технології, та нові робочі місця в
Україні. Якщо ми не станемо по справжньому розумними як суспільство в цілому,
ми так і залишимося найбіднішою наскрізь корумпованою країною Європи, поруч із
злим жахливим сусідом, із постійною небезпекою нових нападів та нових ракетних
ударів. Так що тут напевне варто нагадати. Учітеся, читайте, і чужому
навчайтесь, і свого не цурайтесь.

Kiev Fortress Катя Сосна як ми вибрали не Кравчука, а Чорновола. Якби то, якби
те. Якби ми б всі розмовляли б українською, то путін би не прийшов визволяти.
Якби влада була б українською, то все було б добре. Знаєте, питання не у тому,
а у чому. Приказка є така. Якби бабуся була б дідусем, то у неї була б борода.
Розумієте. Оці осі нескінченні якби б то, якби б це - це ментальна пастка.
Нескінченна демагогія, яка ні до чого не веде. Кравчук вже помер, залиште його
у спокої. І крім того, у свій час він власне і став одним із засновників
незалежної України, підписавши угоди в Біловежській пущі. За це йому честь і
хвала. Також за те, що отримання незалежності пройшло безкровно. У свій час він
був на своєму місці, так, комуняка звичайно, але також і хитрий лис. В Україні,
якщо вам нагадати, не було кровопролиття, незалежність була отримана фактично
безкровно.

щодо спалювання або заборони російських книжок. Тут ми в цьому коментарі не
агітуємо особливо за чи проти самої роскультури або мови в Україні, але
просто... смішно читати про це у цифрову епоху, коли зміст всіх цих книжок
легко доступний в інтернеті. Кому треба - зайде в інет і прочитає. І чим більше
буде таких заборон, тим більше буде охочих скоштувати забороненого. Заборонене
завжди має присмак... солодкого, чи як сказати. Ну, забороните, ну, спалите, ну
і що. Крім того, заборони, викидання на макулатуру чи просте спалювання не
можуть знищити ні росмову, ні рослітературу, ні зокрема зміст цих книжок, не
можуть дати нам перемоги над московією, і не повернуть нам життя наших воїнів
та невинно загиблих. Але сам факт такого створює дуже негарний прецендент,
інформаційний прецендент, оскільки позицінує українців для навколишнього світу
як націю, що не поважає книгу як таку.  Так, звісно, розпіареність роскультури
не відповідає її контексту в Україні, особливо з урахуванням жахливих злочинів,
скоєних російською армією, і ненависть та лють до загарбників і окупантів є
абсолютна законна і зрозуміла річ, але нищити книжки по бібліотеках, особливо
під час війни, коли десь там гинуть наші воїни, і їм потрібна вся наша
допомога, і всі наші зусилля повинні бути спрямованими на перемогу, ну... це
якось  надто тупо. Просто тупо. А українці повинні бути, точніше повинні стати
розумною нацією, найрозумнішою нацією на світі, нацією не тільки хоробрих
воїнів або волонтерів, а по-справжньому нацією вчених, інженерів, філософів,
науковців та письменників. Інакше Україна просто не витримає конкуренції в
навколишньому світі поміж інших держав.  

Це звичайно дуже круто, що ви таки знайшли україномовного тренера для своєї
дитини. Це офігенно як круто. І обовязково треба було пнути словесно
російськомовних, це теж круто. Дитинка ваша буде щаслива і вдоволена, що вона
не буде чути жахливої страшної росмови, а як же інакше. Але у нас зараз війна,
якщо ви забули. Війна зі страшним, жахливим, аморальним ворогом. Дякуйте Богові
що взагалі живі, і що вашу дитинку десь не вбили або не покалічили. Дякуйте
Богові, що Київ взагалі стоїть, що місто відносно ціле, і що орки відійшли.
Видно, що Київ ожив, і деякі особливо розумні вирішили, що можна зайнятись
словесною маячнею і пустослов'ям. Краще б написали що коїться у Херсоні, або ж
у Маріуполі, або ж як живе Харків, і чим можна допомогти тим або іншим. А оця
маячня що ви написали - абсолютно не на часі, це є просто словесний бур'ян у
голові. Так викидуйте якнайшвидше його з голови, і почніть спрямовувати свої
інтеллектуальні та душевні сили на те, що країні по справжньому потрібно для
перемоги над московією. А те, що ви знайшли україномовного тренера для своєї
дитинки - вибачте, але це перемогу над московією аж ніяк не пришвидшує. Якась
херня просто. Ну, знайшла, ну і шо. Купіть собі конфєтку, ви ж такий молодець.

ниче. смеетесь россиянцы. рано смеяться начали скоро перестанете смеяться.
плакать будете. горькими слезами плакать будете. не далек тот час когда рассея
развалится ко всем чертям. вот будет потеха-то.

вы наверное плохо изучали историю. иначе бы знали, что гетьман Сагайдачный в
свое время дошел до Москвы. И украинцы всю свою историю тоже воевали, так что
Украина тоже не в первой воюет, и к тому же на своей родной земле против
наглых, лживых, омерзительных оккупантов. Крейсер Москва хорошо горит, да,
россиянцы. Люлей еще мало получили, видно не нахватались пока еще смертей. Пока
еще в патриотическом угаре поколение Зет. Плавает в говне, а думает, что то
божья благодать. Ниче, время поплакать, постоять на коленях и посыпать голову
пеплом после поражения рассеи у вас еще найдется. Прошлый неуемный рассейский
патриотический угар в 1914 году кончился распадом империи, и в этот раз будет
то же самое, распад рассеи. только второго ссср уже не будет, будет просто
полная жопа. Вот и все.

икак никак. да все так, таки будет. а почему - долго объяснять. сами увидите
скоро. вкратце, нападение на Украину - это просто мега-глупость тысячелетия,
это выстрел себе в голову, это рассея просто убила себя саму идеологически,
ментально, культурно. и наверное, вы при падении рассеии просто умрете, ну то
такое. А то что вы этого не видите, и начнете выть, когда будет уже поздно. Ну
так извините, ваши проблемы. Если вы свою голову употребляете только на то,
чтобы потреблять соловьева и пускать слюнки на путена и великаю рассею и жрать
котлеты за обедом, ну так кто ж вам доктор. Видно, смертей еще не нахватались.
Еще в угаре находитесь. Ну что ж, угар пройдет. А что придет взамен на рассею.
А просто жопа. Полная и всеобъемлющая жопа. Возможно, через пару лет, или
позже, но это случится обязательно. Потому что иначе никак. Потому что таковы
неумолимые законы истории, по которым был Римская империи - а потом империи не
стало. есть москва - а потом и москвы не станет. Все ж так просто. И потому что
за преступлением обязательно будет наказание. Про это ж еще федормихалыч (ТМ)
писал.

да месідж взагалі не про це. ну, пішла - відвела до тренера, ну і добре. Але
нащо з цього розводити цілу демагогію, оце незрозуміло. В контексті війни нас
от більш хвилює питання наприклад, коли звільнять Херсон або чи можливо
врятувати захисників Маріуполя, а не те, що хтось там бідкається, що не може
знайти тренера, або щоби дитинку відгородити від росмови. На жаль, проблема є в
тому, що у багатьох українців відсутня якість виставляти правильні пріорітети,
оце є біда. А в результаті маємо, те, що маємо.

зїздіть тоді краще в Маріуполь або в Херсон, якщо нема чим зайнятись, окрім
відповідати в такому дусі. От тоді і побачите, де у кого які проблеми, якщо
взагалі залишитесь живі.

дякую, що намагаєтесь розібратись в темі. Буду розказувати на прикладі міст, в
яких жила. Донецька обл. історично Україна і історично україномовна, як і решта
України. Конкретно про Донецьку область мені не потрібні супер історичні
розвідки, щоб це знати. Просто приїжджаю в Лиман або Слов‘янськ, або Курахове і
слухаю, як говорять люди. В самому ж Донецьку українська мова була викорінена і
переселенням туди росіян, наративами пропоганди в стилі «українська не
важлива», і «так поймуть», «дєдьі воєвалі», «бандеровци шота там, не ясно шо і
де, але погані». Ці наративи створювали ситуацію, коли за українську на Донбасі
мало хто боровся, бо складена суспільна думка була такою, що тільки скажеш
щось, то викличеш бурю емоцій у людей, що мислять як ви зараз. І деякі з них
матимуть навіть гарні наміри, а проте будуть обслуговувати чужі цілі. І от ми
зараз з вами переписуємось під час війни, яка почалась тому, що росіяни думали,
що можуть швидко перемогти нас. Ми переписуємось про те, чи мають право діти в
Києві ходити на україномовні гуртки. І вас не цікавить, що цих гуртків нема, і
права українських дітей, травмованих війною дітей, порушуються, бо нема цих
гуртків. Ніяких. Ви пишете щось в дусі рос пропаганди про не на часі і решта
все. Чому? Чому вам не здається, що українська дитина може ходити на гурток
українською в Києві. Не колись потім, а зараз? І як ці гуртки повинні
з‘явитись, якщо таким як ви завжди вони не доречні і не на часі?

Olga Maksymchuk Kiev Fortress тільки не розумію чому ви згадали Херсон і
Маріупіль. Це прекрасні міста, які я сподіваюсь, скоро будуть звільнені. Але до
чого вони в розмові про те, що в Києві неможливо знайти український гурток? Це
називається маніпуляція. Вона теж з шаблонів рос методичок. І ви можете цього
не усвідомлювати, проте наслідуєте патерн поведінки.

чудик. здесь никто не истерит. мы просто говорим как оно есть. В отличие от
многих других, мы читаем книги и умеем анализировать что было, что есть, и что
будет потом. Так вот. Жалко вас в чем-то, олухов рассейских. Вроде люди - а по
факту - выродились в диких злобных безмозглых животных. Скоро у вас будет
полная жопа. Знаете такое слово - жопа... замечательное кстати слово - жопа.
Емкое и краткое. Так вот, вас ожидает жопа, настоящая и всеобъемлющая. А вы все
раздуваете лоснящиеся и заплывшие от безумной гордыни щечки. То возьмем! Это
возьмем! Бойтесь нас, мы сильные и страшные! Херсон взяли, Мариуполь взяли, аа!
Буквой Зю все обрисовали, аа! Посмотрите же на нас, какие мы сильные и злобные
обезъяны! Да никакие не сильные и страшные, а просто напросто омерзительные и
глупые до безобразия... Многие уже гниют в украинской земле... а многих еще
ожидает точно такая же участь. Да. Как это ни удивительно, а гордая
тысячелетняя рассея скоро рассохнется и рухнет с громким писком. Вот и все.
Кстати. А че это вы тута во вражеской сети вообще забыли-то. Вам же вроде
роскомнадзор приказал не лазить по фейсбукам, запретили же вам слушать
вражеские голоса, так чо лазите-то, россиянцы-московитики.

ларисочка, нехрен шастать по фейсбукам. вам же вроде роскомнадзор запретил,
да. У вас же там... ээ... росграм... налей сто грам есть, правда. Ну так и
чеши туда, в свой родной росграм-гиппопотам, нефиг засорять фейсбук рассейским
пометом. Фейсбук - это сеть для приличных людей. Московитам сюда вход
воспрещен... а вы все лезете и лезете, никак не уйметесь. Ну а когда интернет
у вас упадет, вообще полная потеха начнется! То, что рутюб упал на 9 мая, или
около того, это такое. Только начало. А цветочки еще впереди. Еще очеень много
всего интересного будет.

потому как если первые лица московии безбожно врали и продолжают врать, то чего
уж ожидать от простых рядовых московитов, истово облизывающих буквочки Зю на
парадах. Ничего кроме лжи, в общем-то. Уж извините, если огорчили.

git restore --source=HEAD :/

Ганна Терпиловська

Kiev Fortress Ні, не так. Ніяк думки про Маріуполь чи Херсон не мають заважати продовжувати виховувати своїх дітей. Особливо людям, які знаходяться в місцях, де є для того умови. В контексті війни, як Ви пишете, нас хвилюють всі питання. Нікому думки про Маріуполь не заважає в Києві снідати і вечеряти, читати новини і жити, як це вдається. Незрозуміло, чому Ви вважаєте, що треба плюнути на виховання дитини, тому що от зараз війна.

А те, що людина поділилася радістю від того, що знайшла україномовного тренера,
розуміють всі, хто таких викладачів шукав. Я живу в Києві, і десь приблизно з
15-16 року постійно була в пошуці гуртків і секцій, де навчання проводиться
українською, і це було дуже важко. Не дарма авторка написала, що це не бойовий
гопак. Тому що українською можна було знайти малювання, фольклорний спів,
фольклорний театр і бойовий гопак. Я шукала звичайний театр дуже довго, і мені
всюди відповідали, що Вишотакехочете, ні, ні, і не буде українською, хочете
українською, йдіть вчіть колядки в етнотеатр, а класика в нас буде класична
російська. Знайшла там, де пообіцяли 50 на 50, насправді було 25 на 75, але, як
виявилося, багато людей теж захотіли українською і поступово стало таки 50 на
50, а потім за пару років почали відкривати групи українською. В Києві була
група пошуку гуртків українською мовою, і, повірте, це була велика проблема,
навіть в гуманітарній сфері, а в спорті ніхто навіть не розглядав можливість
займатися з дітьми українською. Можливо, тим, в кого діти розмовляють
здебільшого російською - все одно. А тим, в кого діти виховані вдома
українською, навіть якщо відкинути ідейну складову, важко займатися російською.
І неможливість реалізувати бажання дитини навчатися рідною мовою в своїй країні
- це дико. Тому радість автора я дуже добре розумію, як мама, яка вишукувала
дев'ять років для своєї дитини україномовну середу існування.

%https://www.facebook.com/groups/story.kiev.ua/posts/1923987197798072

так, звісно ідеологічний вимір важливий, і кількість назв, повязаних с
роскультурою напевне є надмірною. Але ви помиляєтесь у тому, що ракети падають
тому, що 30 років було не на часі.  Ракети падають тому, що в кремлі сидить
божевільний. Ракети падають тому що народ російський перетворився на стадо
диких зомбованих баранів. Ракети падають також тому, що Україна не вкладалась
достатньо в розробку нових озброєнь. Ракети падають, тому що... Взагалі, оце на
часі, не на часі. Не на часі, тому що одразу розгорається лютий срач, одразу
починаються суперечки. Піднімається потужна ментальна хвиля. Люди засмічують
свої мізки цією балаканиною, і просто тупо витрачають свої ментальні сили в
нікуди. А в цей час хтось сидить в окопі, поки ми тута розводимо балаканину і
йому абсолютно по барабану, чи є там памятник чи нема. Знаєте, це як в
анекдоті. Це до того, що є багато речей в Україні, які самі по собі на хід
війни не впливають, але якщо їх підняти зараз, то йде потужне відволікання
уваги від по справжньому важливих речей - як от. Як звільнити Херсон. Як
звільнити Маріуполь. Як визволити бійців Азову. Як добути більше зброї,
броників. Так от, чи є памятник пушкіну, чи його немає прямо зараз - для ходу
війни абсолютно по барабану. Хід війни не визначається тим, чи дивиться пушкін
на корпуса політеху чи вже ні. Поки це питання не підняли, ніхто не пушкіна
особливо не думав. Всі ці суперечки аж ніяк не можуть пришвидшити перемогу
України, але навпаки, можуть сприяти в кінці болісній поразці, бо мізки були
направлені не та те, що потрібно. Так от, анекдот. Звісно, аналогія дуже-дуже
крива, але все ж таки. Йде професор з величезною бородою, назустріч йому
студент. Студент питає - професоре. А коли ви спите, ви кладете бороду на
ковдру, чи під ковдрою. Професор - я ніколи не задумувався. Пройшов день -
студент йде і професор такий злий-злий - каже йому - ах ти ж паскуда! всю ніч
не спав!! то на ковдру кладу, то під ковдрою, все свербить, ніяк не міг
заснути! От. є також інша приказка. Не діліть шкуру невбитого медвідя. От
переможемо - перейменуємо все що завгодно, знесемо одні памятники, поставимо
інші. Народ збереться, обговорить і винесе якесь прийнятне рішення. А зараз це
все є якийсь просто лютий треш, просто голова болить вже від цього. Кажете -
привід для нової війни, якщо не знесемо. Це абсурд. Привід для вбивці завжди
найдеться, було б бажання напасти. Тим більше що привід можна буде ще легше
знайти. А як запобігти новій війні. Бути згуртованими як один, сильними,
розумними. Бути найрозумнішою, найзгуртованішою нацією в світі.

шановний чудік під імям Сашко Папуга. Цей акаунт вже пару місяців як ми
створили, і ми є абсолютно живі люди. Хтось веде акаунт роками і має тисячі
підписників, хтось створює його вчора і починає писати, маючи нуль друзів
спочатку. Треба ж з чогось починати. І в чому взагалі питання, не зрозуміло.
Адекватна людина, може і так, ну і що. А що, адекватні люди у нас вже прям
стали святі, що їх не можна критикувати. Окей. Далі. Ну настрочив ти гнівний
коментар ну і що. По суті того що ми висловили нічого не сказав, по змісту
нічого розумного не сказав (як наприклад шановна пані в попередньому коментарі
вище), просто інформаційного шуму додав. Так що піди купи собі конфєтку теж,
маладєц. Щодо буряну в своїй свідомості, бурян вже давно прополений. Так
барвисті квіточки ростуть і гарні розлогі дерева. Там дзюрчить холодна водичка,
і гріє тепле сонечко, у нашій свідомості. Співають пташки соловїним співом. І
там все у нас зі свідомістю добре, не хвилюйся.

але якщо вам цікаво, чим ми займаємось, окрім цих коментарів. Ми ведемо
літописи війни по різним тематикам, і насправді це досить напружена робота
кожного дня. Якщо вам цікаво, ми можемо кинути вам посилання в приваті. У нас є
збірка і про Харків, а ви як видно з Харкова. У вас до речі видно по
прискіпливості аналітичний склад розуму, не дивно як для людини з науковою
освітою.

Щодо пари місяців спілкуйтеся із самим Фейсбуком, адже це саме він характеризує
так, як показано на картинці.  "Адекватна людина, може і так, ну і що". -
Заздріть мовчки!  "А що, адекватні люди у нас вже прям стали святі, що їх не
можна критикувати". - Критикувати можна. Писати нісенітниці, називаючи це
критикою, теж можна - і виглядати при цьому дурнем ТЕЖ можна.  "Ну настрочив ти
гнівний коментар ну і що". - Гнівний? А, може, більш глузливий, ніж гнівний?
"По суті того що ми висловили нічого не сказав, по змісту нічого розумного не
сказав". - Ви були значно неуважні. 😄 А якщо читати уважно, то можна побачити
те розумне, що було сказано досить чітко: хтось вивалив свою недолугу маячню у
коментарі і зовсім не потурбувався про те, чи буде воно стосуватися теми
розмови. Те інше, що можна було ще сказати по суті (невеличкій суті, але вже
якій є), вже сказано іншими. Я хіба що міг перефразувати.  "Так що піди купи
собі конфєтку теж, маладєц". - Мрійте і далі собі про конфєтки. Якщо згодом
почнете мріяти про цукерки, то це буде суттєвим кроком уперед.  "Щодо буряну в
своїй свідомості, бурян вже давно прополений". - А сувора практика, тобто
наявні коментарі показують зовсім інше.

здравствуйте. Привет из Киева. Если бы как бы, то во рту росли б грибы.
Конечно, воюет именно с народом Украины. И народ Украины дает рассее люлей. А
рассея скоро просто развалится от вони и гнилья, старая мерзкая старушонка
рассея... 
