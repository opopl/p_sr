%%beginhead 
 
%%file 04_03_2023.fb.fb_group.mariupol.pre_war.1.ad_sushchestvuet__i_
%%parent 04_03_2023
 
%%url https://www.facebook.com/groups/1233789547361300/posts/1421103958629857
 
%%author_id fb_group.mariupol.pre_war
%%date 04_03_2023
 
%%tags mariupol,poezia
%%title Ад существует, и теперь я знаю: ужас его в том, что он сделан их обрывков рая или воспоминаний о рае
 
%%endhead 

\subsection{Ад существует, и теперь я знаю: ужас его в том, что он сделан их обрывков рая или воспоминаний о рае}
\label{sec:04_03_2023.fb.fb_group.mariupol.pre_war.1.ad_sushchestvuet__i_}

\Purl{https://www.facebook.com/groups/1233789547361300/posts/1421103958629857}
\ifcmt
 author_begin
   author_id fb_group.mariupol.pre_war
 author_end
\fi

Ад существует, и теперь я знаю: ужас его в том, что он сделан их обрывков рая
или воспоминаний о рае. Нужно иметь внутри себя хаос, чтобы  создать в группе
такие условия для авторов материалов, в группе состоящей на 95\% из беженцев от
войны из оккупированного врагом  Мариуполя, такие условия, когда модерами
группы замалчиваются или удаляются все антивоенные или патриотические
материалы. Или, что ещё хуже, сначала печатают, а потом, на следующий день
такие материалы,  удаляются модерами группы  безвозвратно.

Вот, например, моё последнее стихотворение, сначала напечатанное,
просуществовавшее опубликованным в группе один день, горячо одобренное группой,
а потом безвозвратно удалённое, вместе со всеми другими антивоенными
стихотворениями и материаламиЮ ранее напечатанными и опубликованными мною в
группе.

Любіть Маріуполь більше, ніж раніше...

Облога Марiуполя.
Зі стін вологість тхне іржавая,
Навколо темрява та ніч,
Ми з відчайдушною відвагою,
Ждемо щось поруч, пліч-о-пліч.
Підвальним сутінком охоплені,
Так несподівано близькі,
Раніше час бездумно тратили,
На дурощі та дріб`язкі.
Тепер сторінку перегорнуто?
Цей запит мабуть не простий,
Коли життя все перевернуте,
І бомбовозiв гул частий.
Прийдешнє свiтло щось  стискається,
На темних стінах слід блідий,
Там сумно хороводом мається,
Кружляє дим пороховий.
Твоїх очей  манливих вирази,
I трепет полохливих вій,
Хоча сиджу з тобою поблизу,
Не бачу в темряві густий.
Ти побажай і все загоїться,
Як тільки ранок ніч перерве,
Війна раптово заспокоїться,
І сон страшний від нас піде.
Без горя щастя не трапляється,
Долоні мало для хлопка,
Хвилини в крапельки збігаються,
І час стікає до струмка.
На краще пошепки загадую,
Я повернуся, ти  повір,
Чекай, і зручною нагодою,
Я за водою йду надвір.
Краплинами години тягнуться,
Вже з неба сонечко світить,
Незгасною вогнем лампадою,
Над містом, що давно горить.
Автор: Леонiд Едельштейн.

Я хочу предупредить модеров, мои антивоенныестихи или статьи - это как
приведения, и им случается выходить на сцену без приглашения. Я обращаюсь
также, к друзьям, обретённым мною в группе за поддержкой, потому что друзья -
это ангелы, которые поднимают нас, когда наши крылья больше не могут вспомнить
как летать.

Существует целый ад в двух словах: замалчивание реальности, потому что такое
замалчивание бесполезно, это как мыльный пузырь, который меняет цвет, словно
радужка и лопается, когда до него дотрагиваются.

Требую вынести на дискуссию вопрос: Нужно ли печатать в группе антивоенные и
патриотические материалы, или нет.

Эти фотографии довоенного города, красивы и достижимы, как жена другого, только
в мечтах. Мы все потеряемся в нашем эмигрантском одиночестве, из-за жестокой
войны, если не будем поддерживать или лишимся веры в победу.
