% vim: keymap=russian-jcukenwin
%%beginhead 
 
%%file 01_02_2022.fb.ohmatdet.bolnica.1.128_let
%%parent 01_02_2022
 
%%url https://www.facebook.com/ndslohmatdyt/posts/2994079187569465
 
%%author_id ohmatdet.bolnica
%%date 
 
%%tags birthday,bolnica,deti,kiev,medicina,ohmatdet.bolnica
%%title Охматдиту 128 років
 
%%endhead 
 
\subsection{Охматдиту 128 років}
\label{sec:01_02_2022.fb.ohmatdet.bolnica.1.128_let}
 
\Purl{https://www.facebook.com/ndslohmatdyt/posts/2994079187569465}
\ifcmt
 author_begin
   author_id ohmatdet.bolnica
 author_end
\fi

Охматдиту 128 років @igg{fbicon.cake.birthday} 

Сьогодні найбільша дитяча лікарня України святкує свій день народження  @igg{fbicon.party.popper} 

\ii{01_02_2022.fb.ohmatdet.bolnica.1.128_let.pic.1}

У різні роки й століття, в часи війни та різноманітних криз ця лікарня
трималась, розвивалась і жила завдяки лиш одному — людям. Відданим,
професійним, сміливим та чесним. Саме вони день за днем створювали Охматдит
таким, яким він є сьогодні. На честь свята пропонуємо згадати хороші моменти з
історії лікарні.

@igg{fbicon.heart.exclamation} 1 лютого 1894 відкрила свої двері Київська
безоплатна Цесаревича Миколая лікарня для чорноробів. Збудував її відомий
український підприємець та меценат Нікола Терещенко. Сьогодні на території
Охматдиту стоїть пам’ятник своєму засновнику.

@igg{fbicon.heart.exclamation} Від 1927 року лікарня була клінічною базою для
Інституту охорони материнства та дитинства. А у 1957 їй надали статус
спеціалізованої дитячої лікарні.  Керівництво новоствореного закладу взяла на
себе Тетяна Петрівна Новікова. Вона зробила усе, щоб створити спеціалізовану
лікарню для дітей. Вже за рік після нового статусу відкрито відділення
виходжування недоношених дітей.

@igg{fbicon.heart.exclamation} У 1947 році до закладу приєднали поліклініку №11. Нині
консультативно-діагностична поліклініка — це наш передовий підрозділ: вони
консультують пацієнтів, скеровують їх на лікування у стаціонарі та роками
ведуть пацієнтів до здорового життя. Щороку у поліклініці фіксують понад 300
000 відвідувань!

\ii{01_02_2022.fb.ohmatdet.bolnica.1.128_let.pic.2}

@igg{fbicon.heart.exclamation} Кафедра дитячої хірургії Національного медичного
університету почала співпрацю з Охматдитом у 1957. Професор кафедри Олег
Володимирович Дольницький організував перше в Україні відділення дитячої
мікрохірургії. За розробку та впровадження мікрохірургічних методів лікування
пошкоджень периферичних нервів професору Дольницькому О. В. присуджено Державну
Премію України в галузі науки та техніки. 

\ii{01_02_2022.fb.ohmatdet.bolnica.1.128_let.pic.3}

@igg{fbicon.heart.exclamation} У 1975 році відкрито хірургічний та
неонатологічний корпуси. Фінансували їх будівництво за рішенням міської влади з
коштів киян, зароблених на громадських суботниках.

\ii{01_02_2022.fb.ohmatdet.bolnica.1.128_let.pic.4}

@igg{fbicon.heart.exclamation} Наша кафедра неонатології почала роботу в 1977
році. Вона була першою в Україні та другою в Радянському Союзі.

\ii{01_02_2022.fb.ohmatdet.bolnica.1.128_let.pic.5}

@igg{fbicon.heart.exclamation} З 1989 року вперше в Києві в дитячій практиці
роблять ультразвукову діагностику мозку у новонароджених, а також УЗД легенів,
плевральної порожнини, шлунку, вилочкової залози, м‘яких тканин та
кістково-суглобної системи.

\ii{01_02_2022.fb.ohmatdet.bolnica.1.128_let.pic.6}

@igg{fbicon.heart.exclamation} Професор Бебешко Володимир Григорович з Центру
дитячої онкогематології та трансплантації кісткового мозку НДСЛ «Охматдит» був
першим Головним дитячим гематологом МОЗ України та засновником служби дитячої
гематології. 

\ii{01_02_2022.fb.ohmatdet.bolnica.1.128_let.pic.7}

@igg{fbicon.heart.exclamation} Потенціал нашої команди та відданість покликанню
допомогла отримати державний статус вже в новій незалежній Україні. 3 липня
1996 року Указом Президента України створено українську дитячу спеціалізовану
лікарню «Охматдит» — тепер вже національного значення.

\ii{01_02_2022.fb.ohmatdet.bolnica.1.128_let.pic.8}

Щоб розповісти про всі наші досягнення, довелося б писати книгу в кількох
томах. Та у нашої команди нема на це часу. Бо кожну мить ми віддаємо, щоб
рятувати життя та здоров’я маленьких українців @igg{fbicon.biceps.flexed} 

\ii{01_02_2022.fb.ohmatdet.bolnica.1.128_let.pic.9}

Сьогодні час подякувати кожному, хто працював та працює на благо лікарні. Разом
з вами та завдяки вам ми робимо справжні дива, які неможливо переказати у
кількох реченнях  @igg{fbicon.dizzy} 

\ii{01_02_2022.fb.ohmatdet.bolnica.1.128_let.pic.10}

З Днем народження, Охматдит. У нас попереду великі справи  @igg{fbicon.hands.raising} 

Фільм про історію Охматдиту за посиланням: \url{https://youtu.be/Xvdq9d7ZAuQ}

\#Охматдит \#найкращадитячалікарня

\ii{01_02_2022.fb.ohmatdet.bolnica.1.128_let.pic.11}
\ii{01_02_2022.fb.ohmatdet.bolnica.1.128_let.pic.12}
\ii{01_02_2022.fb.ohmatdet.bolnica.1.128_let.pic.13}
\ii{01_02_2022.fb.ohmatdet.bolnica.1.128_let.pic.14}
\ii{01_02_2022.fb.ohmatdet.bolnica.1.128_let.pic.15}
\ii{01_02_2022.fb.ohmatdet.bolnica.1.128_let.pic.16}
\ii{01_02_2022.fb.ohmatdet.bolnica.1.128_let.pic.17}
\ii{01_02_2022.fb.ohmatdet.bolnica.1.128_let.pic.18}
\ii{01_02_2022.fb.ohmatdet.bolnica.1.128_let.pic.19}
\ii{01_02_2022.fb.ohmatdet.bolnica.1.128_let.pic.20}
\ii{01_02_2022.fb.ohmatdet.bolnica.1.128_let.pic.21}
\ii{01_02_2022.fb.ohmatdet.bolnica.1.128_let.pic.22}
