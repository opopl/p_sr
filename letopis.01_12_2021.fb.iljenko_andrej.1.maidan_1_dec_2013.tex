% vim: keymap=russian-jcukenwin
%%beginhead 
 
%%file 01_12_2021.fb.iljenko_andrej.1.maidan_1_dec_2013
%%parent 01_12_2021
 
%%url https://www.facebook.com/andriy.illenko/posts/7357440007630059
 
%%author_id iljenko_andrej
%%date 
 
%%tags maidan2,ukraina
%%title 1 грудня 2013 року
 
%%endhead 
 
\subsection{1 грудня 2013 року}
\label{sec:01_12_2021.fb.iljenko_andrej.1.maidan_1_dec_2013}
 
\Purl{https://www.facebook.com/andriy.illenko/posts/7357440007630059}
\ifcmt
 author_begin
   author_id iljenko_andrej
 author_end
\fi

1 грудня 2013 року

День був дуже довгий, тому конспективно.

Ще на підходах до парку Шевченка (де було місце збору) стало зрозуміло - що це
буде щось неймовірне. 

Людське море починалося ще від Володимирського собору. Такої кількості людей я
не бачив до того ніколи. І вразила не лише кількість - люди були переповнені
справедливим гнівом і бажанням дій.

\ii{01_12_2021.fb.iljenko_andrej.1.maidan_1_dec_2013.pic.1}

Гігантський марш рушив на Майдан. Сцени не було, підігнали бусік з озвучкою,
який губився в дуже щільному натовпі. Він - на фото.

Стало ясно - почалось. З невеликою групою народних депутатів і активістів
рушаємо до Будинку профспілок. Заходимо через головний вхід. Там сидить дідок,
який в стилі радянського вахтера намагається нас "закликати до порядку".
Просимо його виглянути у вікно - після цього всі претензії у вахтера
відпадають. 

Облаштовуємо у профспілках штаб. Тут мені надходить звістка - Київрада вже теж
наша. Ідемо туди. В колонній залі вже кипить робота з облаштування повстанської
бази. 

Ідемо на Банкову, де почалося протистояння з внутрішніми військами і
"Беркутом". Заходимо на задимлену вулицю і одразу нариваємося на оскаженілий
беркут, який біжить прямо на нас. Витримуємо навалу і без критичних втрат
відходимо. Серйозно постраждали Олексій Кайда та Андрій Міщенко, яким розбили
голови. Момент атаки видно на цьому скріншоті з відео - ми згруповано стоїмо по
центру. 

Вертаємось на Майдан, віддихались від сльозогінного газу. Розуміємо - назад
дороги вже нема. Або ми, або нас.

Таким я запам'ятав той день - день початку Революції Гідності.
