% vim: keymap=russian-jcukenwin
%%beginhead 
 
%%file slova.blog
%%parent slova
 
%%url 
 
%%author 
%%author_id 
%%author_url 
 
%%tags 
%%title 
 
%%endhead 
\chapter{Блог}
\label{sec:slova.blog}

%%%cit
%%%cit_head
%%%cit_pic
%%%cit_text
Итак, со старта начинает \emph{Блогер}, «европейский проект не будет завершенным и
полноценным без Украины». Мне вот даже интересно, а будет завершен «европейский
проект» без России, хотя бы на том основании, что Мордор является самым большим
государством Европы? Или то другое? Но если Россия «то – другое», то почему
Украина не может быть таким же «другим»? Откуда вообще столько спеси у
голодранцев, которые ходят по миру с протянутой рукой, но при этом с
шизофреническим пафосом рассказывают немцам, что такое «завершенный европейский
проект»? У тебя Посольство расписывает, какие вакцины нельзя поставлять, что
запрещено покупать и продавать китайцам и как именно по пунктам должна пройти
судебная реформа. А ты надуваешь щечки и рассказываешь будущему канцлеру,
почему ЕС будет незавершенным проектом без Шепетовки и Горишних Плавней
%%%cit_comment
%%%cit_title
\citTitle{Пацан с района задал четкие вопросы будущему канцлеру / Лента соцсетей / Страна}, 
Игорь Лесев, strana.ua, 29.06.2021
%%%endcit

%%%cit
%%%cit_head
%%%cit_pic
%%%cit_text
Дальше \emph{Блогер} упомянул «революцию достоинства». Все ведь канцлеры знают, что
ради нее люди положили на Майдане свои жизни. Главное не поднимать вопрос, а
кто же их там положил. И \emph{Блогер} по заветам Петра так и не поднимает. Давайте не
копошить раны. Будем просто почитать и тыкать всякую зажравшуюся немчуру нашими
героями.  А затем \emph{Блогер} берет снайперскую винтовку и бьет будущего верховного
немца свинцовым вопросом прямо в висок – «а не утомилась ли Европа так долго
уклоняться от вопроса, когда же Украина станет членом ЕС»? Быдыщ! 
%%%cit_comment
%%%cit_title
\citTitle{Пацан с района задал четкие вопросы будущему канцлеру / Лента соцсетей / Страна}, 
Игорь Лесев, strana.ua, 29.06.2021
%%%endcit
