% vim: keymap=russian-jcukenwin
%%beginhead 
 
%%file 04_12_2020.news.ru.lenta_ru.1.koronavirus_moskva_vaccine
%%parent 04_12_2020
 
%%url https://lenta.ru/news/2020/12/04/priv/
 
%%author 
%%author_id 
%%author_url 
 
%%tags 
%%title В Москве открылась запись на вакцинацию от коронавируса
 
%%endhead 
 
\subsection{В Москве открылась запись на вакцинацию от коронавируса}
\label{sec:04_12_2020.news.ru.lenta_ru.1.koronavirus_moskva_vaccine}
\Purl{https://lenta.ru/news/2020/12/04/priv/}

\index[rus]{Коронавирус!Россия!Запись на вакцинацию в Москве}

\ifcmt
pic https://icdn.lenta.ru/images/2020/12/04/09/20201204090514768/pic_6bcdb072af521b900fe181a583a38504.jpg
caption В Москве открылась запись на вакцинацию от коронавируса, Фото: Георгий Зимарев / РИА Новости
\fi

В Москве на портале mos.ru открылась электронная запись на прививку от
коронавируса. В настоящий момент получить вакцину «Спутник V» могут граждане,
которые контактируют с большим количеством людей: учителя, сотрудники городских
служб, работники медицинских учреждений.

Заместитель мэра Москвы Анастасия Ракова пояснила, что в первую очередь
вакцинируются основные группы риска. «Обращаю внимание, что получить прививку
смогут не только врачи и педагоги. Санитар в больнице, сотрудники школы,
социальные работники трудятся в коллективах, где много контактов с жителями. И
риск заразиться у них не меньше», --- уточнила она.

Отмечается, что москвичи могут записаться в один из 70 пунктов на бесплатную
прививку. После того как учреждения получат новые партии вакцин, список
категорий расширят.

\begin{leftbar}
	\bfseries
{\centering\Huge\color{orange}	70}\par
поликлиник Москвы\par
начнут проводить вакцинацию от коронавируса
\end{leftbar}

\subsubsection{Ранее стал известен порядок вакцинации}

Прививку могут получить только те, кто прикреплен к столичным поликлиникам.
Пункты на их базе будут работать с 8 утра до 8 вечера. На прививку необходимо
прийти с паспортом, полисом ОМС и справкой с работы с указанием должности.

Весь процесс займет около часа. Перед прививкой пациенту проведут 10-минутный
осмотр, после чего для него будет разморожена вакцина, что займет около 15
минут. После прививки медики еще 30 минут будут наблюдать за самочувствием
человека.

\subsubsection{Путин объявил масштабную вакцинацию «Спутником V» со следующей недели}

2 декабря президент России заявил, что в ближайшее время объем произведенной в
стране вакцины достигнет двух миллионов доз, что позволяет начать «если не
массовую, то масштабную вакцинацию» препаратом «Спутник V».

На следующий день мэр Москвы Сергей Собянин объявил о старте массовой
вакцинации от COVID-19. Отмечалось, что прививочные пункты заработают уже 5
декабря.

Препарат «Спутник V» первым из трех российских вакцин прошел регистрацию. Он
разработан в Центре им. Гамалеи, его клинические испытания прошли в июне и
июле.
