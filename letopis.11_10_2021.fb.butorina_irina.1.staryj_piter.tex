% vim: keymap=russian-jcukenwin
%%beginhead 
 
%%file 11_10_2021.fb.butorina_irina.1.staryj_piter
%%parent 11_10_2021
 
%%url https://www.facebook.com/irina.butorina.773/posts/10217585804151644
 
%%author_id butorina_irina
%%date 
 
%%tags arhitektura,gorod,rossia,st_peterburg
%%title ЗА КРАСИВЫМИ ФАСАДАМИ СТАРОГО ПЕТЕРБУРГА
 
%%endhead 
 
\subsection{ЗА КРАСИВЫМИ ФАСАДАМИ СТАРОГО ПЕТЕРБУРГА}
\label{sec:11_10_2021.fb.butorina_irina.1.staryj_piter}
 
\Purl{https://www.facebook.com/irina.butorina.773/posts/10217585804151644}
\ifcmt
 author_begin
   author_id butorina_irina
 author_end
\fi

ЗА КРАСИВЫМИ ФАСАДАМИ СТАРОГО ПЕТЕРБУРГА

Те, кто хоть однажды побывал в Санкт-Петербурге, не устают восхищаться красотой
северной столицы России. В последние годы, когда были отреставрированы дома
старой части города, он еще больше похорошел. Однако гостям города невдомек,
что кроется за красивыми фасадами домов. Вряд ли кто-то задает себе вопрос: так
же они хороши внутри, как и снаружи? Хотя, в принципе, людям постарше несложно
догадаться, что основная часть домов исторической части города представляет
собой, выражаясь языком Ильфа и Петрова, Вороньи слободки, т.е. коммунальные
квартиры. Это полностью заслуга Советской власти, которая, решая жилищный
вопрос населения Петрограда, переселила в роскошные апартаменты буржуев
питерскую бедноту. Прошедшие с той поры сто лет на пользу этим квартирам не
пошли.  

\ifcmt
  tab_begin cols=3

     pic https://scontent-lhr8-1.xx.fbcdn.net/v/t39.30808-6/245186685_10217585800351549_7744990936686423953_n.jpg?_nc_cat=106&ccb=1-5&_nc_sid=730e14&_nc_ohc=cWfVyQiorFMAX8e_4Tw&_nc_ht=scontent-lhr8-1.xx&oh=069bcb4072b9d482c1be1ead69891d1f&oe=617F04F4

     pic https://scontent-lhr8-1.xx.fbcdn.net/v/t39.30808-6/245200540_10217585800951564_7283503144991069112_n.jpg?_nc_cat=111&ccb=1-5&_nc_sid=730e14&_nc_ohc=K-0L8KRWOGgAX8x-05s&_nc_ht=scontent-lhr8-1.xx&oh=02db1dc7ff8c889a5799a51ce564b0aa&oe=617D936E

		 pic https://scontent-lhr8-1.xx.fbcdn.net/v/t39.30808-6/245318513_10217585801911588_6460915281294617178_n.jpg?_nc_cat=109&ccb=1-5&_nc_sid=730e14&_nc_ohc=xD-y0UmSFhMAX8c4QYy&tn=lCYVFeHcTIAFcAzi&_nc_ht=scontent-lhr8-1.xx&oh=024e40d723abaed25eaad85e16cbfd61&oe=617EDB31

  tab_end
\fi

В число таких домов предсказуемо попал и знаменитый «Пряничный дом» - творение
известного русского архитектора Николая Никонова, выполнившего и фасад, и
внутреннее убранство дома в традиционно русском стиле с мозаикой, кокошниками
и прочими приметами русской культуры, сделавшими здание таким же красивым, как
расписной пряник. За более чем сотню лет красота «дома-пряника» поблекла, а
случившийся в 2009 году пожар усугубил его состояние. Только после этого
городские власти решили заодно с восстановлением двенадцати пострадавших от
пожара квартир отреставрировать фасад дома и придать ему статус объекта
культурного наследия регионального значения, что и было сделано в 2011 году.
Однако расселения коммунальных квартир выполнено не было, да и мысль о том,
что «Пряник» можно было бы восстановить и изнутри, превратив дом в воистину
культурное наследие города, как образчик русской дизайнерской культуры, никому
в голову не пришла.

\ifcmt
  tab_begin cols=3

     pic https://scontent-lhr8-1.xx.fbcdn.net/v/t39.30808-6/245046507_10217585802671607_428277051448864165_n.jpg?_nc_cat=110&ccb=1-5&_nc_sid=730e14&_nc_ohc=GaYFdDFZalwAX_TauR8&_nc_ht=scontent-lhr8-1.xx&oh=fd4047ac2aa9b17f0adfb8c043812685&oe=617E8B25

     pic https://scontent-lhr8-2.xx.fbcdn.net/v/t39.30808-6/244876884_10217585897953989_5347201366018610518_n.jpg?_nc_cat=105&ccb=1-5&_nc_sid=730e14&_nc_ohc=q57PFBQvBjYAX-TajDj&tn=lCYVFeHcTIAFcAzi&_nc_ht=scontent-lhr8-2.xx&oh=4195738e9ab8e4b0b5d864225d1538c3&oe=617E4711

		 pic https://scontent-lhr8-1.xx.fbcdn.net/v/t39.30808-6/244749256_10217585898594005_2436269477389649708_n.jpg?_nc_cat=106&ccb=1-5&_nc_sid=730e14&_nc_ohc=7y05jhslQBQAX-THm7c&_nc_ht=scontent-lhr8-1.xx&oh=3d56322018ae37dfa1db7b1eee08afae&oe=617E4168

  tab_end
\fi

Однако, мир не без инициативных людей, и когда в Пряничный дом попала моя
племянница Ольга Полищук – интерьерный дизайнер, она, пробившись по коридору,
заваленному накопившимся за столетие хламом, в комнату, которую намеревалась
реконструировать, ахнула – так великолепна была лепнина на потолке. В этот
момент она поняла, что находится в уникальном помещении, которое необходимо
приобрести  и восстановить в первозданном виде.

\ifcmt
  tab_begin cols=3

     pic https://scontent-lhr8-1.xx.fbcdn.net/v/t39.30808-6/244566634_10217585899114018_8826799410474165452_n.jpg?_nc_cat=111&ccb=1-5&_nc_sid=730e14&_nc_ohc=chm_1r2gLXgAX8euYm9&_nc_ht=scontent-lhr8-1.xx&oh=e8cfeffd0fd4c2d965559a662fbd53c3&oe=617E4629

     pic https://scontent-lhr8-2.xx.fbcdn.net/v/t39.30808-6/244645851_10217585899594030_150285351819322282_n.jpg?_nc_cat=104&ccb=1-5&_nc_sid=730e14&_nc_ohc=PR5BO-dUAr4AX8PB8cm&_nc_ht=scontent-lhr8-2.xx&oh=dd24e871ab74f78e80493d05677a8271&oe=617EB6A3

		 pic https://scontent-lhr8-2.xx.fbcdn.net/v/t39.30808-6/244977401_10217585900114043_6275453561829602126_n.jpg?_nc_cat=101&ccb=1-5&_nc_sid=730e14&_nc_ohc=vS8UGD9ysnEAX_77vaI&tn=lCYVFeHcTIAFcAzi&_nc_ht=scontent-lhr8-2.xx&oh=5bdc249f2e7d91fdcdc6df43f5b7ee3a&oe=617EF00D

  tab_end
\fi

В результате был нанят профессиональный реставратор, который скальпелем и
зубной щеткой расковыривал и расчищал многолетние наслоения краски и штукатурки
на лепнине потолка, а феном выдувал образовавшийся мусор, освобождая
алебастровый узор от вековых наслоений. Когда узоры потолка были освобождены и
покрашены, мастер взялся за стены, которые являли собой папье-маше из десятков
слоев обоев. Там между слоями бумаги нашлась картина Мурильо «Непорочное
зачатие», выполненная печатью на шелке, которая тоже требовала реставрации.
Однако самая замечательная находка ждала хозяйку за стенкой купленной квартиры.
Практически за фанерной перегородкой была обнаружена небольшая комната –
продолжение первой с такой же лепниной потолка. Ее тоже удалось купить, и после
демонтажа стенки стало ясно, что эти два помещения были когда-то единым целым и
представляли собой гостиную или музыкальный салон. 

\ifcmt
  tab_begin cols=4

     pic https://scontent-lhr8-2.xx.fbcdn.net/v/t39.30808-6/244851276_10217585900594055_5111896741194346686_n.jpg?_nc_cat=101&ccb=1-5&_nc_sid=730e14&_nc_ohc=f_HIQsze7y4AX9Asq86&_nc_ht=scontent-lhr8-2.xx&oh=a24ee5d900893745635ce9ae750cb53d&oe=617EC0E8

     pic https://scontent-lhr8-1.xx.fbcdn.net/v/t39.30808-6/245094111_10217585901914088_3306354166390399947_n.jpg?_nc_cat=108&ccb=1-5&_nc_sid=730e14&_nc_ohc=b3Iduysq52sAX97-RLF&_nc_ht=scontent-lhr8-1.xx&oh=98d5b8a751c9c45862749fbf41621487&oe=617E7D8E

		 pic https://scontent-lhr8-1.xx.fbcdn.net/v/t39.30808-6/244593138_10217585955635431_5173638729552715731_n.jpg?_nc_cat=110&ccb=1-5&_nc_sid=730e14&_nc_ohc=Vywtr8nCUhYAX9-8QWx&_nc_ht=scontent-lhr8-1.xx&oh=026a34a665f163f3490d670437393f09&oe=617DAF5D

		 pic https://scontent-lhr8-1.xx.fbcdn.net/v/t39.30808-6/245036420_10217585956035441_5253999797332020372_n.jpg?_nc_cat=110&ccb=1-5&_nc_sid=730e14&_nc_ohc=4sNVQf3jyw8AX8fxogx&_nc_ht=scontent-lhr8-1.xx&oh=17ce6d42d16b0982601abba80e20ee6e&oe=617DAA0D

  tab_end
\fi

Вот эта догадка и подвигла хозяйку создать в этом помещении Арт-пространство.
Вскоре были куплены, отреставрированы и обставлены мебелью начала прошлого века
еще две комнаты коммуналки. Питерские художники предоставили для экспозиции
несколько картин, в результате «АРТ-пространство. Квартира №6» состоялось! Не
беда, что для этого пришлось отказаться от дорогостоящего ремонта собственной
квартиры и летнего отдыха для всей семьи на море.

Я попала в созданное Ольгой АРТ-пространство в начале октября, зная историю его
создания из нескольких телепередач на Санкт-Петербургских каналах. Очарованная
красотой фасада дома-пряника, впечатлившись просторным, но грязноватым
коридором подъезда с витражами окон и великолепно сохранившимися с момента
постройки дубовыми дверями, открыв их, я погрузилась в темноту мрачного,
ободранного и заваленного изломанной мебелью коридора. Мне, никогда не жившей и
не бывавшей в коммуналках, не приходило в голову, что за красивыми фасадами
старого Петербурга скрывается такой хаос, нищета и депрессия. Только
депрессивным состоянием жителей можно объяснить тот факт, что они не хотят
вылезти из своих темных комнатушек и переселиться в нормальные квартиры, даже
когда им предлагают это сделать на взаимовыгодных условиях, и всеми силами
мешают тем, кто хочет и может сделать их жизнь достойной. К счастью, за дверью
из мрачного коридора открывается совершенно иная реальность – великолепные
покои начала 20 века, возрожденные из забытья трудом и настойчивостью Ольги.

Сегодня в «АРТ-пространстве. Квартира №6» замечательная встреча с исполнителем
романсов Сергеем Рогожиным. Он поет под гитару без микрофона, но акустика
музыкального салона, спроектированного талантливым архитектором, великолепна, и
голос певца звучит прекрасно. Слова исполняемого им романса «Королева,
чародейка, госпожа…», наверняка посвящены хозяйке салона, сумевшей вернуть
городу часть его культуры – роскошные внутренние интерьеры «Пряничного дома»,
потратившей на это много денег и душевных сил, которые очень нужны в условиях
Вороньей слободки. Впереди у нее жестокие баталии с соседями, которых, мне
кажется, больше всего бесят даже не редкие концерты и гости, а вылупившаяся
из-под слоя грязи и нищеты прежняя роскошь и красота «дома-пряника», которую
они старались не замечать. 

Остается только надеяться на стойкость Ольги и государство, которое должно
защищать не только фасады культурного наследия города, но и их внутреннее
убранство, если желает сохранить за Петербургом статус города-музея.
