% vim: keymap=russian-jcukenwin
%%beginhead 
 
%%file 27_11_2020.news.ua.strana.2.covid_16000_cases
%%parent 27_11_2020
 
%%url https://strana.ua/news/303301-skolko-v-ukraine-bolnykh-covid-19-dannye-na-27-nojabrja.html
 
%%author 
%%author_id 
%%author_url 
 
%%tags covid,ukraine
%%title Второй день антирекорд. Число новых зараженных коронавирусом в Украине перевалило за 16 000 человек
 
%%endhead 
 
\subsection{Второй день антирекорд. Число новых зараженных коронавирусом в Украине перевалило за 16 000 человек}
\label{sec:27_11_2020.news.ua.strana.2.covid_16000_cases}
\Purl{https://strana.ua/news/303301-skolko-v-ukraine-bolnykh-covid-19-dannye-na-27-nojabrja.html}

\ifcmt
	 pic https://strana.ua/img/article/3033/1_main-v1606457266.jpeg
	 caption В Украине растет число зараженных с каждым днем, фото: pixabay.com
\fi

По состоянию на утро пятницы, 27 ноября, в Украине подтвердилось 16 218 новых
случаев заражения коронавирусом. И это очередное печальное достижение.

Напомним, что последний антирекорд был отмечен только вчера, 26 ноября, когда
за сутки в стране был зафиксирован 15 331 новый зараженный.\Furl{https://strana.ua/news/303063-skolko-v-ukraine-bolnykh-covid-19-dannye-na-26-nojabrja.html}

Об очередных жертвах Covid-19 сообщает глава Минздрава Максим Степанов в
соцсети.

Общее число заболевших на сегодняшний день таким образом достигло отметки в 693
407 человек. Выздоровели за последние 24 часа 8 843 пациента. Умерло за сутки
192 человека (с начала эпидемии число летальных случаев составило уже 11 909).
Госпитализировано 2 069 человек. Среди новых зараженных 666 детей и 655
медработников.

Самое большое количество случаев заражения за минувшие сутки обнаружено в Киеве
(1 520 человек), Запорожской области (1 003), Киевской области (994),
Днепропетровской области (1 773) и Одесской области (966).


\ifcmt
	 pic https://strana.ua/img/forall/u/0/29/27_%D0%BD%D0%BE%D1%8F%D0%B1%D1%80%D1%8F.PNG
\fi

Напомним, что в Словакии не увидели эффекта от поголовного тестирования
населения на коронавирус.\Furl{https://strana.ua/news/303294-v-slovakii-ne-uvideli-effekta-ot-poholovnoho-testirovanija-naselenija-na-koronavirus.html}

Также мы рассказывали, что Швейцария не станет закрывать горнолыжные курорты во
время пандемии коронавируса.
\Furl{https://strana.ua/news/303262-shvejtsarija-otkroet-hornolyzhnye-kurorty-vo-vremja-pandemii-koronavirusa.html}
