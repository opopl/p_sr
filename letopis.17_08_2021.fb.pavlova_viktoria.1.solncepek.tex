% vim: keymap=russian-jcukenwin
%%beginhead 
 
%%file 17_08_2021.fb.pavlova_viktoria.1.solncepek
%%parent 17_08_2021
 
%%url https://www.facebook.com/pavlova1975/posts/815030202549693
 
%%author_id pavlova_viktoria
%%date 
 
%%tags donbass,film,film.solncepek.donbass,kino,vojna,vpechatlenie
%%title Три дня подряд я вижу в ленте "Солнцепек", "Солнцепек", "Солнцепек"
 
%%endhead 
 
\subsection{Три дня подряд я вижу в ленте \enquote{Солнцепек}, \enquote{Солнцепек}, \enquote{Солнцепек}}
\label{sec:17_08_2021.fb.pavlova_viktoria.1.solncepek}
 
\Purl{https://www.facebook.com/pavlova1975/posts/815030202549693}
\ifcmt
 author_begin
   author_id pavlova_viktoria
 author_end
\fi

Три дня подряд я вижу в ленте "Солнцепек", "Солнцепек", "Солнцепек". Честно,
не люблю современное российское кино о войне - уж слишком оно пропагандистским
и карикатурным получается. Я уходила посреди сеанса со "Спасти Ленинград", не
зашел мне и "Т-34", а другие новинки и смотреть не стала.

Но "Солнцепек" стал мейнстримом в социальных сетях. И я, начитавшись отзывов о
том, как его смотрели в три захода, 

плакали, глотали успокоительное, не могли прийти в себя, решилась. Вдруг
действительно стоящее кино получилось, а я его не увижу?

Увы. Первые кадры разочаровали. Мародеры, насилие - зачем и к чему, я так и не
поняла. Хватило на двадцать минут - захотелось выключить. Слишком уж
прямолинейно режиссер показывает изнанку - концентрированно, щедро поливая
кровью и выстрелами в упор каждый кадр. За изобилием жестоких сцен теряется
драматизм ситуации.

Я заставила себя смотреть дальше. Досмотрела. Слез нет, из эмоций - злость и
разочарование. Сюжет предвзятый, изобилует штампами и клише. Все намешано в
кучу: добробаты, американские инструкторы, мародеры, врачи скорой, ополченцы,
мирные жители. 

Не понятно, о чем снимали и зачем это снимали? Очередная спекуляция на нашей
войне? Хотели показать весь ужас происходившего? Напомнить? Так оно и не
забывалось. А ужас передать не получилось. 

Помните, как водитель скорой сказал? Страшно, когда на твоих глазах убивают
других, а ты не можешь ничем помочь.  Это правда, но я актеру не поверила.
Наверное, это сложно понять и еще сложнее показать, передать. Актеры на экране
были какими-то отстраненными, безучастными.

Трагизм ситуации - он ведь не в реках крови. Трагизм ситуации еще и в
непонимании происходящего, растерянности, в человеческом горе - массовом. В том
горе, которое в каждом из нас. В том горе, когда воздуха не хватало, когда в
стену зубами впиться хотелось, когда вереницы машин уезжали, когда колонны
мирных расстреливали, когда собаки выли по ночам, когда в магазинах не было
еды, а в кранах - воды. 

Трагизм ситуации - это очереди за гуманитарной помощью и бесплатным хлебом во
вчера еще благополучном городе. Очереди, в которых люди  терпеливо стояли и
беззвучно погибали под обстрелами.  Трагизм - в церковных службах под свистом
снарядов, в табличках "Мины", в патрулях и блокпостах, в детях, играющих в
футбол на изрешеченном осколками школьном поле. Трагизм - он в деталях.  

Если режиссеру удается передать весь ужас происходившего без крови, а иногда и
без военных сцен, то это высший пилотаж военного кино. Это глубина, нерв,
анализ.

Увы, в "Солнцепеке" я увидела только хаос. Такое впечатление, что режиссер
нагонял ужас на зрителей. Ужас - может, и сильная эмоция, но далеко не
единственная в нашей войне.

Не проняло. Слишком уж однобоко снято. Я всю войну в Донецке. В прифронтовом
районе. Многое видела, многое слышала, многое еще свежо в памяти. Имею право
так думать и так писать.

Скажу честно, я не хочу, чтобы мои дети, внуки и правнуки составляли себе
представление о нашей войне по подобным фильмам. Если бы в жизни все было так
однозначно, как показано в фильме. Если бы...
