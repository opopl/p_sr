% vim: keymap=russian-jcukenwin
%%beginhead 
 
%%file 19_08_2021.fb.bilchenko_evgenia.1.svoboda_lichnosty_hristos.cmt
%%parent 19_08_2021.fb.bilchenko_evgenia.1.svoboda_lichnosty_hristos
 
%%url 
 
%%author 
%%author_id 
%%author_url 
 
%%tags 
%%title 
 
%%endhead 
\subsubsection{Коментарі}

\begin{itemize}

% -------------------------------------
\ii{fbauth.valikova_olga.moskva.rossia.alma_ata.rudn.researcher}
% -------------------------------------
 
Развернуто и - ясно)

\begin{itemize}
%%%fbauth
%%%fbauth_name
\iusr{Евгения Бильченко}
%%%fbauth_url
%%%fbauth_place
%%%fbauth_id
%%%fbauth_front
%%%fbauth_desc
%%%fbauth_www
%%%fbauth_pic
%%%fbauth_pic portrait
%%%fbauth_pic background
%%%fbauth_pic other
%%%fbauth_tags
%%%fbauth_pubs
%%%endfbauth
 
\textbf{Ольга Валикова} Я говорила после тебя ещё с одним человеком.
\end{itemize}

%%%fbauth
%%%fbauth_name
\iusr{Настя Бузиашвили}
%%%fbauth_url
%%%fbauth_place
%%%fbauth_id
%%%fbauth_front
%%%fbauth_desc
%%%fbauth_www
%%%fbauth_pic
%%%fbauth_pic portrait
%%%fbauth_pic background
%%%fbauth_pic other
%%%fbauth_tags
%%%fbauth_pubs
%%%endfbauth
 
Потому что Бог есть Любовь... И она во всем - в действиях, в поступках, в
мыслях, в жизни. Любовь побеждает смерть и страх

\begin{itemize}
%%%fbauth
%%%fbauth_name
\iusr{Евгения Бильченко}
%%%fbauth_url
%%%fbauth_place
%%%fbauth_id
%%%fbauth_front
%%%fbauth_desc
%%%fbauth_www
%%%fbauth_pic
%%%fbauth_pic portrait
%%%fbauth_pic background
%%%fbauth_pic other
%%%fbauth_tags
%%%fbauth_pubs
%%%endfbauth
 
\textbf{Настя Бузиашвили} да, моя нежная.
\end{itemize}

%%%fbauth
%%%fbauth_name
\iusr{Alex Katsai}
%%%fbauth_url
%%%fbauth_place
%%%fbauth_id
%%%fbauth_front
%%%fbauth_desc
%%%fbauth_www
%%%fbauth_pic
%%%fbauth_pic portrait
%%%fbauth_pic background
%%%fbauth_pic other
%%%fbauth_tags
%%%fbauth_pubs
%%%endfbauth
 

Вообще-то понятие Троицы из язычества и в Библии не упоминается как таковое. То
есть Бог Сын не равен Богу Отцу. Сам Христос говорит, что Отец большего него.
Поэтому не православие.

\begin{itemize}
%%%fbauth
%%%fbauth_name
\iusr{Евгения Бильченко}
%%%fbauth_url
%%%fbauth_place
%%%fbauth_id
%%%fbauth_front
%%%fbauth_desc
%%%fbauth_www
%%%fbauth_pic
%%%fbauth_pic portrait
%%%fbauth_pic background
%%%fbauth_pic other
%%%fbauth_tags
%%%fbauth_pubs
%%%endfbauth
 
\textbf{Alex Katsai} Всё в Отце, верно. Дух Святой исходит от Отца и посылается
Сыном, у нас нет filioque. Понятие же триады присутствует в неоплатонизме
Плотина и Прокла как "Единое - Разум - Душа" (также в арабской фалсафа, у
перипатетиков). В эпоху формирования Церкви в период апологетики и патристики
античная триада неоплатонизма была использована для догмата, утвержденного
Никео-Константинопольским символом веры.

%%%fbauth
%%%fbauth_name
\iusr{Евгения Бильченко}
%%%fbauth_url
%%%fbauth_place
%%%fbauth_id
%%%fbauth_front
%%%fbauth_desc
%%%fbauth_www
%%%fbauth_pic
%%%fbauth_pic portrait
%%%fbauth_pic background
%%%fbauth_pic other
%%%fbauth_tags
%%%fbauth_pubs
%%%endfbauth
 
\textbf{Alex Katsai} Архетип три, да, мы можем обнаружить в нумерологии И Цзин,
в японской сакральной символике синто, трёх гунах материи школы санкхья,
бесконечном числе легенд о братьях-основателях в сказках и т.д. Короче говоря,
в любых прообразах диалектики, включая теологическую, и в пространственных
структурах мифа. И?

%%%fbauth
%%%fbauth_name
\iusr{Alex Katsai}
%%%fbauth_url
%%%fbauth_place
%%%fbauth_id
%%%fbauth_front
%%%fbauth_desc
%%%fbauth_www
%%%fbauth_pic
%%%fbauth_pic portrait
%%%fbauth_pic background
%%%fbauth_pic other
%%%fbauth_tags
%%%fbauth_pubs
%%%endfbauth
 
\textbf{Евгения Бильченко} Что «и?»? Вас ничего не смущает? Библия не учит
Троице. Поэтому догмат о ней, которого придерживается православие, не является
подлинно христианским.

%%%fbauth
%%%fbauth_name
\iusr{Alex Katsai}
%%%fbauth_url
%%%fbauth_place
%%%fbauth_id
%%%fbauth_front
%%%fbauth_desc
%%%fbauth_www
%%%fbauth_pic
%%%fbauth_pic portrait
%%%fbauth_pic background
%%%fbauth_pic other
%%%fbauth_tags
%%%fbauth_pubs
%%%endfbauth
 
Из чего следует, что православие имеет в себе язычество

%%%fbauth
%%%fbauth_name
\iusr{Гамельнский Крысолов}
%%%fbauth_url
%%%fbauth_place
%%%fbauth_id
%%%fbauth_front
%%%fbauth_desc
%%%fbauth_www
%%%fbauth_pic
%%%fbauth_pic portrait
%%%fbauth_pic background
%%%fbauth_pic other
%%%fbauth_tags
%%%fbauth_pubs
%%%endfbauth
 
\textbf{Евгения Бильченко} 

Приведенные примеры чужды русской, да и в целом, нордической рассе. Гораздо
ближе Северная традиция, Славянское язычество, которое, собственно и есть
Православие, название, присвоенное церковью греческого толка Orthodoxy, уверен,
целенаправленно, с целью введения в заблуждение и подмены Православия
истинного.

Здесь, можно найти, возможно, исток понятия Троицы - Явь, Навь, Правь,
триединство видимого и невидимых миров. Православие, собственно, есть
прославление той части, где обитают Высшие Боги.

В Северной Традиции тройственность мира представлена в виде древа Игдрасиль,
которое делится на корни, ствол и ветви, где Яви соответствует ствол.


%%%fbauth
%%%fbauth_name
\iusr{Евгения Бильченко}
%%%fbauth_url
%%%fbauth_place
%%%fbauth_id
%%%fbauth_front
%%%fbauth_desc
%%%fbauth_www
%%%fbauth_pic
%%%fbauth_pic portrait
%%%fbauth_pic background
%%%fbauth_pic other
%%%fbauth_tags
%%%fbauth_pubs
%%%endfbauth
 
\textbf{Alex Katsai} 

Абсолютно ничего не смущает, причем, ни с одной точки зрения. Приведем
навскидку шесть онтологических программ, которых мы в той или иной мере
придерживаемся. В каждую из них заложена культурная наследственность. Первая:
онтология присутствия. Наиболее яркий представитель на Западе - Хайдеггер. Всё
есть всё. На Востоке этому соответствует монистическое единство мира. Исходя из
этого, дохристианская (внешняя) философия содержит в себе те же архетипы (Карл
Юнг), что и христианство. Вторая программа - негативная. Все есть Ничто, Сартр
и Лакан. Ничто есть Пустота/Реальное, то есть бессознательное, не выразимое в
языке по полноте своей, как у позднего Витгенштейна. Следовательно, мистическая
триада есть довербальный элемент. Нечто непознаваемое. Третья программа:
циничная, или негативно-диалектическая. Каждый текст есть списком с
предыдущего, смерть автора. В святоотеческой традиции - это анонимность. Тем
более, культурная наследственность предусмотрена как идеалистической (Гегель),
так и материалистической (Маркс) парадигмами. И, наконец, рассмотрим это с
точки зрения программы избытка. Все наследуется каузально, пока не происходит
разрыв - революционное событие прерывания предыдущего ряда ради основания
следующего (в экзистенциализме - абсурд, в христианстве - "Верую ибо абсурдно"
Квинта). Суть догмата Троицы, его древние прототипы и его связь с учением
Церкви отражают единство бытия. А наличие в христианстве культурных
заимствований никак не умаляет чуда веры как события разрыва. Попытки
романтиков Тюбингенской школы и Католического института в Париже (например,
Альфреда Луази) прервать наследственность и отсечь текст Библии от апологетики,
патристики и схоластики на том основании, что в канонических и неканонических
Евангелиях нет догматов, выработанных семью вселенскими соборами, не увенчались
успехом. Можно, конечно, если вы протестант, исходить из sola scripture, я
уважаю эту позицию, но мне ближе принцип взаимосвязи времён, и наличие
культурных привнесений в сакральных текстах для меня не умаляет их содержания,
ибо формально можно использовать разные логики, суть же - вечна, и открывается
человечеству в разных формах. Мне духовно близко прочтение Святой Троицы.

%%%fbauth
%%%fbauth_name
\iusr{Евгения Бильченко}
%%%fbauth_url
%%%fbauth_place
%%%fbauth_id
%%%fbauth_front
%%%fbauth_desc
%%%fbauth_www
%%%fbauth_pic
%%%fbauth_pic portrait
%%%fbauth_pic background
%%%fbauth_pic other
%%%fbauth_tags
%%%fbauth_pubs
%%%endfbauth
 
\textbf{Гамельнский Крысолов} 

Я вполне уважаю эти сходства, потому и толкую о диалоге и солидарности в
русской культуре. Но, когда кто-либо заявляет что-то вроде "нашей расе
чуждо...", я очень напрягаюсь по двум причинам. Первая: немотивированное
выдавание своей позиции за позицию народа (миллионы православных русских в это
время напрягаются). Называется такая театрализация "синдром общих своих",
считается признаком ур-фашистского мышления, ибо в фашизме был свой пласт
эзотерики. Второе. Очень смущает расовая аргументация: я весьма довольна, что в
РФ подобные вещи не поощряются. Мне хватило здесь подобных вещей. Если же брать
культурную историю, без экзальтации, да, это, несомненно, всем известные
религиозные и мифологические нарративы. В таком случае я временно перейду на
европейскую позицию и напомню вам, как минимум, о толерантности и ненарушении
границ. Я не давала оценки язычеству. Я просто описывала свой выбор. У вас он
может быть иным, я уважаю ваш выбор. Спасибо.


%%%fbauth
%%%fbauth_name
\iusr{Alex Katsai}
%%%fbauth_url
%%%fbauth_place
%%%fbauth_id
%%%fbauth_front
%%%fbauth_desc
%%%fbauth_www
%%%fbauth_pic
%%%fbauth_pic portrait
%%%fbauth_pic background
%%%fbauth_pic other
%%%fbauth_tags
%%%fbauth_pubs
%%%endfbauth
 
\textbf{Евгения Бильченко} Для меня лично именно Библия является истиной. Все остальные теории, умозаключения, вселенские соборы и тд - не важны.

%%%fbauth
%%%fbauth_name
\iusr{Евгения Бильченко}
%%%fbauth_url
%%%fbauth_place
%%%fbauth_id
%%%fbauth_front
%%%fbauth_desc
%%%fbauth_www
%%%fbauth_pic
%%%fbauth_pic portrait
%%%fbauth_pic background
%%%fbauth_pic other
%%%fbauth_tags
%%%fbauth_pubs
%%%endfbauth
 
\textbf{Alex Katsai} Я так и поняла.

\end{itemize}

\end{itemize}

