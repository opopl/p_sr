% vim: keymap=russian-jcukenwin
%%beginhead 
 
%%file 21_01_2022.fb.akimov_boris.perejaslavl_zalesskij.rossia.1.inostrancy
%%parent 21_01_2022
 
%%url https://www.facebook.com/boris.akimov/posts/10161422641663572
 
%%author_id akimov_boris.perejaslavl_zalesskij.rossia
%%date 
 
%%tags emigracia,inostrancy,perejaslavl_zalesskij.rossia,rossia,zhizn
%%title Иностранцы в Переславле
 
%%endhead 
 
\subsection{Иностранцы в Переславле}
\label{sec:21_01_2022.fb.akimov_boris.perejaslavl_zalesskij.rossia.1.inostrancy}
 
\Purl{https://www.facebook.com/boris.akimov/posts/10161422641663572}
\ifcmt
 author_begin
   author_id akimov_boris.perejaslavl_zalesskij.rossia
 author_end
\fi

Иностранцы в Переславле. 

Я много раз писал о том, как европейцы и американцы переезжают в Переславль.
Насовсем. С каждым месяцем в наших местах все больше новых жителей, переехавших
на ПМЖ к нам издалека. Вот тут собрали 15 самых интересных соседей Счастливые.
Ярославская область. 

Мы с Большая Земля продолжаем привлекать переселенцев в Россию. Теперь не
только в Переславль и окрестности, но и в Доброград во Владимирской области. 

Иконописец и столяр Джейсон Силуан Кристофер Кэмбэлл с семьей переехал в
Переславль из США совсем недавно. А ещё он многодетный отец.

Анья Пабст, живёт в России с 2004 года, медиатор и  бизнес-тренер, к нам в
Переславский район переехала жить в 2017 году. Anja Pabst

Гудрун Пфлугхаупт переехала из города Росток в Германии в деревню Красногор и
открыла тут небольшой эко-кемпинг «Бабушка Холле». А всё потому, что бабушка в
детстве рассказывала ей истории о далекой России. @babushka\_holle

Ханнес переехал в Переславль из Баварии и с женой Алиной они основали гостевой
дом Styles lodge sky and sand

Фредерик Андриё приехал с юга Франции и открыл тут производство и кафе.
@frederika86

Катя Эчеди приехала в Россию из Венгрии 27 лет назад. Сначала у них была
маленькая картинная галерея, а потом они открыли мини-отель
@pereslavl\_art\_hotel

Лоранс Гийон переехала юга Франции, где была учителем. А тут, в Переславле,
написала книгу о приключениях Ивана Грозного. А ещё она много рисует и изучает
эпоху средневековья. 

Жиль Вольтер тоже переехал из Франции в Переславль и открыл тут чудесное кафе с
традиционными французскими рецептами @laforetcafe

Реставратор старинных книг Ян ден Бизен переехал из Бельгии. Вот уже более 40
лет он восстанавливает средневековые редчайшие книги. 

Дизайнер Исаак Грегсон с семьей переехали из Вашингтона. Присматриваются к
возможности покупки земли под небольшую ферму под Переславлем.

Из Англии к нам приехал Стивен Уильям и с женой Натальей они в деревне Рыково
занимаются семейной цветочной фермой! 

Бенуа и Елена Ларди. Они познакомились в Париже, Бенуа влюбился сначала в
Елену, а потом в Россию. Теперь они вместе живут тут и устраивают увлекательные
путешествия для всех желающих @tsarvoyages\_russie

Эрик и Ольга Рафальске. Переехали сюда из Джорджии. Эрик увлекается
антиквариатом и они с Ольгой всегда рады принять интересных людей у себя в
гостях. @erafalske

Джозеф Глисон или Отец Иосиф переехал с семьей в небольшую деревню Ивашево. До
этого он жил в Техасе и Иллинойсе, 20 лет проработал в сфере IT с большими
компаниями. Теперь отец Иосиф служит в храме и преподаёт английский язык.

Беньямин Форстер, он же Вениамин, прибыл к нам из Швейцарии, занимается тут
пасекой и является казаком и старообрядцем.

В ближайшее время к нам переезжает ещё несколько иностранных семей. Если и вы
интересуетесь переездом из Европы, США  и других стран в наши места - пишите
нам в Директ \#movetorussia  \#большаяземля \#счастливые

В мобильном приложении #Счастливых вы найдете сотни других деятельных людей,
которые здесь и сейчас строят новую жизнь – свою и  жизнь своих территорий.
Качайте приложение Счастливые в App Store и Google Play
