% vim: keymap=russian-jcukenwin
%%beginhead 
 
%%file 29_06_2015.stz.mrpl.old_mariupol.1.teleprogramme_mariupol_byloe_bolee_20_let
%%parent 29_06_2015
 
%%url http://old-mariupol.com/teleprogramme-mariupol-byloe-bolee-20-let
 
%%author_id mrpl.old_mariupol
%%date 
 
%%tags 
%%title Телепрограмме "Мариуполь. Былое" более 20 лет
 
%%endhead 
 
\subsection{Телепрограмме \enquote{Мариуполь. Былое} более 20 лет}
\label{sec:29_06_2015.stz.mrpl.old_mariupol.1.teleprogramme_mariupol_byloe_bolee_20_let}
 
\Purl{http://old-mariupol.com/teleprogramme-mariupol-byloe-bolee-20-let}
\ifcmt
 author_begin
   author_id mrpl.old_mariupol
 author_end
\fi

\begin{quote}
\em\bfseries
Программа \enquote{Мариуполь. Былое} на ТРК \enquote{Сигма} создавалась не как
развлекательное ток-шоу, рассчитанное на погоню за рейтингом. Эта программа
носит образовательный характер, и направлена на популяризацию истории и
культуры.

Ее первый выпуск вышел в эфир 24 июля 1994 года, а регулярная еженедельная
трансляция началась с 7 сентября того же года. 20 июня 2015 года телезрителям
был показан 983-й выпуск.
\end{quote}

\enquote{Мариуполь. Былое} — это истории нашего города и Приазовья, рассказы о
старинных зданиях, улицах, площадях. Программа включает в себя несколько
циклов: \enquote{Жизнь замечательных мариупольцев}, \enquote{Мариуполь в Великой Отечественной
войне}, \enquote{Кальмиусская паланка запорожских казаков}, \enquote{Театр в Мариуполе},
\enquote{Мариуполь литературный}, \enquote{Архип Куинджи — наш земляк}, \enquote{Мариупольские
художники}, \enquote{Мариуполь глазами художников} и многие другие.

\ii{29_06_2015.stz.mrpl.old_mariupol.1.teleprogramme_mariupol_byloe_bolee_20_let.pic.1}

Автор программы — Сергей Буров — рассказал о том, как все начиналось, и о
работе над созданием выпусков.

\enquote{В 1994 году в одном из выпусков информационной программы \enquote{Зеркало} ведущий
говорил, что у многих городов Украины есть свое дерево-символ. К примеру, у
Киева – каштаны, а у Мариуполя такого нет. В это время у нас на улицах зацвели
акации, которых в нашем городе очень много. Я позвонил на телеканал и
рассказал, что есть в Мариуполе деревья, символизирующие наш город, и предложил
рассказать об этом. Так вышел в эфир сюжет о цветущих акациях на улицах,
сопровождаемый стихами мариупольских поэтов, которые озвучивала актриса нашего
театра Светлана Доронченко}.

Сергей Давидович Буров в 1960 году окончил Мариупольский металлургический
институт, долгое время работал по специальности. В студенческие годы и позднее
принимал участие в любительской киностудии, которая одно время снимала
ежеквартальный киножурнал \enquote{Приазовский экран}. Этот киножурнал
показывали перед фильмами в кинотеатрах города.

А тесное сотрудничество с \enquote{Сигмой} у него началось, когда он работал на
\enquote{Азовмаше}. Над программой \enquote{Мариуполь. Былое} работал по вечерам, после
рабочего дня на производстве. \enquote{Помогала жена, — рассказал Сергей Буров. –
Печатала сценарии, искала в библиотеках необходимые материалы, занималась
корректурой}.

Над выпусками работал и работает коллектив \enquote{Сигмы}. Это режиссеры:

Сергей Салма, Алексей Серков, Валерий Миндов, Александр Шлик, Виктор Борец,
Артем Дробот, в настоящее время — Светлана Добрынина. Операторы: Сергей Салма,
Вадим Свиридонов, Дмитрий Федоров, Сергей Коноплев, Дмитрий Кузминский, Сергей
Семенченко, Евгений Воробьев, Виктор Дедов и другие. Тексты читали: Андрей
Василенко, Алексей Серков, Виктор Борец, Роман Толстых.

Естественно, что работа над программой по истории Мариуполя требует
консультаций специалистов, воспоминаний очевидцев событий.

Сергей Буров по этому поводу говорит: \enquote{Консультанты, добровольные помощники
программы и наиболее часто участвующие в ней: ветеран 221-й стрелковой дивизии
Игорь Семенович Смык, ветеран труда Сергей Григорьевич Дьяченко, ветераны флота
Владимир Андреевич Павлий и Виталий Антонович Торорощенко, научные сотрудники
краеведческого музея Рена Ильинична Саенко, Раиса Петровна Божко, Лидия
Ивановна Кучугура, Александра Забело, старший методист библиотеки им. Короленко
Ольга Николаевна Подтынная, краеведы и писательницы Галина Михайловна Захарова
и Валентина Михайловна Зиновьева, коллекционер и искусствовед Александр
Михайлович Чернов и многие другие}.

Сергей Буров – член Национального союза краеведов Украины. Это авторитетная
профессиональная организация, в нашем городе в ней состоят только 13
мариупольцев.

Но вместе с тем в силу своей природной скромности Сергей Давыдович говорит:
\enquote{Я не краевед, я краелюб}.

\enquote{Когда программа только начала выходить, мы над ней работали по ночам,
— вспоминает главный редактор \enquote{Сигмы} Андрей Василенко, который
озвучивал \enquote{Мариуполь. Былое} до 1998 года. — У нас было всего два
монтажных комплекса, и днем они были заняты другими группами. Порой работа
затягивалась до полуночи, и домой приходилось возвращаться пешком.

Для меня сотрудничество с Сергеем Давыдовичем было серьезной школой работы с
короткометражными фильмами, которыми, по сути, и являются выпуски этой
программы. Он при их подготовке всегда опирается на исторические факты,
источники, воспоминания людей старшего возраста, сотрудничает с музеями города.
Очень тщательно и скрупулезно подходит к подбору материалов.

Можно сказать, что благодаря этой программе я для себя открыл неизвестные
страницы истории нашего города. Особенно из всех программ мне понравился цикл о
мариупольских улицах и зданиях – возникало даже такое ощущение, что смотришь на
наш современный город сквозь столетия.

Мы сохранили в архиве, переведя их в электронный цифровой формат, все выпуски
\enquote{Мариуполя. Былое}. Ведь это – телевизионная история нашего города, его
культуры, рассказы об известных и рядовых мариупольцах}.

\textbf{По материалам \enquote{Приазовского рабочего}}
