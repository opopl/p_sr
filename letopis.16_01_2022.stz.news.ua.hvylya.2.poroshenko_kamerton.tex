% vim: keymap=russian-jcukenwin
%%beginhead 
 
%%file 16_01_2022.stz.news.ua.hvylya.2.poroshenko_kamerton
%%parent 16_01_2022
 
%%url https://hvylya.net/analytics/245480-chogo-ya-ne-mozhu-podaruvati-poroshenku
 
%%author_id druzenko_gennadij,news.ua.hvylya
%%date 
 
%%tags poroshenko_petr,ukraina,politika,obschestvo,strana
%%title Чого я не можу подарувати Порошенку?
 
%%endhead 
\subsection{Чого я не можу подарувати Порошенку?}
\label{sec:16_01_2022.stz.news.ua.hvylya.2.poroshenko_kamerton}

\Purl{https://hvylya.net/analytics/245480-chogo-ya-ne-mozhu-podaruvati-poroshenku}
\ifcmt
 author_begin
   author_id druzenko_gennadij,news.ua.hvylya
 author_end
\fi

\begin{zznagolos}
Я не можу подарувати Петру Олексійовичу тільки одного – він зіпсував
український моральний камертон. Змусив його фальшивити.
\end{zznagolos}

Колись радянський дисидент Андрей Синявский, написав знамениту фразу: \enquote{В мене з
радянською владою суто естетичні розбіжності}. В мене з Петро Порошенко та його
прибічниками розходження насамперед етичні.

Як дорослий хлопчик я розумію, що Петро Олексійович – класична політична
тварина (ζῷον πoλιτικόν). І в цьому він – кров від крові та плоть від плоті
Дональда Трампа, Віктора Орбана чи Боріса Джонсона, так само не обтяжених
моральними імперативами.

\ii{16_01_2022.stz.news.ua.hvylya.2.poroshenko_kamerton.pic.1}

Ба більше, останнє ітревʼю Порошенка Politico наче писали спічрайтери Трампа,
розвиваючи улюблену тезу 45 Президента США \enquote{I alone can fix it}. В цьому
інтервʼю пʼятий Президент України стверджує: \enquote{Я – лідер опозиції. Я – лідер
громадської підтримки. Я – пʼятий президент. Я – особа, яка воювала з Путіним,
та разом з моєю командою врятували Україну в найбільш важкі часи її історії. Я
– людина, яка створила армію. І я саме та особа, яка зробила Україну набагато
ближчою до Євросоюзу. Я – той, хто закарбував в Конституції України європйську
та євроатлантичну інтеграцію як напрям нашої зовнішньої політики.}

Так, ви не переплутали, це не цитата з Людовика XIV, якого називали
\enquote{король-сонце} і який вважав, що «L’état c’est moi», тобто \enquote{держава – це я}. Це
свіжа цитата з інтервʼю пʼятого Президента України. Аби не залишалось сумнівів,
наводжу мовою публікації: \enquote{I’m the leader of the opposition,} he said. \enquote{I’m the
leader of public support. I’m the fifth president. I am the person who,
fighting Putin, and with my team saved Ukraine in the most difficult years of
our history. I’m the person who created the army. And I am the person who
[brought] Ukraine much closer to the European Union. I am the person who put in
the Ukrainian constitution, European and Euro-Atlantic integration as the
direction of our foreign policy.}

Нехай ці твердження залишаються на совісті самого \enquote{сивочолого гетьмана}
та його адептів. Як у самовидця та безпосереднього учасника тих важких подій, в
мене сформувалась принципово відмінна версія того, як і завдяки кому вистояла
Україна. І не тільки в мене. Але наразі не про це.

Для мене українська справа має перспективи на перемогу тоді і тільки тоді, коли
національна ідея поєднується з моральною перевагою, з етичною вищістю над
ворогом. Тому моїм дороговказом завжди залишатиметься не Декалог українського
націоналіста, а моральний імператив Бориса Антоненка-Давидовича:

\begin{zzquote}
\enquote{...знаю лише, що жорстокість і непримиренність були, є і будуть до кінця мого
життя чужі мені і далекі. Я не толстовець, що не противиться злу, але я певен,
що не доведе до добра наслідування найгірших рис своїх супротивників. За що ж
тоді боротись і важити своїм життям? Адже зло навіть під українським
національним прапором, виголошене українською мовою, все ж лишається злом...}
\end{zzquote}

Я не готовий ставати на бік морального зла \enquote{під українським національним
прапором, виголошеного українською мовою}, а прихильники Петра Олексійовича
готові. В моїй ієрархії цінностей етика завжди вища за (національну) айдентику.
І я чудово знаю: якщо ставити айдетнику понад етику, власна держава може стати
найгіршим концтабором для власних громадян. Доведено трагічним ХХ століттям.

Втім я доволі спокійно ставлюсь до нагнітання пристрастей навколо
\enquote{повернення Петра}. Бо і на Майдані і на Сході я і мої друзі боролися
не за конкретного політичного лідера, а за свободу і право кожного українця
вірити в те (і в того), у що (і кого) він чи вона вважає правильним. Зокрема і
в деміурга Петра Олексійовича, який \enquote{alone fix it all}. Чим ми гірші за
американців, які – попри все – пристрасно підтримують Трампа?

Така врешті-решт ціна демократії. Бо вона ніколи не гарантувала ухвалення
правильних рішень. В основі демократії лежить право на помилку, за умови якщо
виборець готовий нести відповідальність за наслідки свого вибору. Під цим кутом
зору, і Петро Порошенко, і Володимир Зеленський, і навіть Віктор Янукович –
добра школа демократії. Яка вчить: обирати можна кого завгодно. Але
кінець-кінцем за свій вибір прийдеться заплатити...

Я не можу подарувати Петру Олексійовичу тільки одного – він зіпсував
український моральний камертон. Змусив його фальшивити. Бо до \enquote{доби Петра} я
знав, що такі люди, як безмежно шановані мною Мирослав Маринович чи вже, на
жаль, покійний Євген Сверстюк завжди називатимуть біле білим, а чорне чорним. І
жодна \enquote{національні інтереси} не змусять їх називати зло добром. На жаль, на
превеликий жаль, Петру Олексійовичу вдалось спокусити частину тих, кого не зміг
зламати весь репресивний апарат СРСР. Своє, принаймні для деяких з них, стало
важливішим за моральне.

І знищення цього морального камертона – це єдине, чого я ніколи не подарую
Петру. Надто я люблю цих людей. І надто боляче мені бачити, як вони етично
фальшивлять заради політичної тварини у вишиванці...

Політика ніколи не буває абсолютно моральною. І не має бути. Савонароли у владі
– страшне лихо. Але коли в політиці не залишається навіть дещиці етики, вона
стає дорогою до пекла. Коли мета (якою б великою вона не була) виправдовує
будь-які засоби її досягнення – це і є \enquote{back to USSR}, від якого ми так
відчайдушно намагаємося втекти. Навряд чи варто було так відчайдушно валити
памʼятники Леніну, аби одразу водрузити на його місце \enquote{свого сучого сина}.

Я завжди залишусь на позиції Антоненка-Давидовича і Василя Стуса: \enquote{Зло навіть
під українським національним прапором, виголошене українською мовою, все ж
лишається злом...}
