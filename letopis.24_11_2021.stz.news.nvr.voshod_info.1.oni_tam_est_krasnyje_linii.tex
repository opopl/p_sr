% vim: keymap=russian-jcukenwin
%%beginhead 
 
%%file 24_11_2021.stz.news.nvr.voshod_info.1.oni_tam_est_krasnyje_linii
%%parent 24_11_2021
 
%%url https://voskhodinfo.su/okazat-pomosch-lyudyam/pomosch-semyam/73136-oni-tam-est-krasnye-linii.html
 
%%author_id lebedev_sergej_lohmatyj
%%date 
 
%%tags 
%%title Они ТАМ есть! Красные линии
 
%%endhead 
\subsection{Они ТАМ есть! Красные линии}
\label{sec:24_11_2021.stz.news.nvr.voshod_info.1.oni_tam_est_krasnyje_linii}
\Purl{https://voskhodinfo.su/okazat-pomosch-lyudyam/pomosch-semyam/73136-oni-tam-est-krasnye-linii.html}

\ifcmt
 author_begin
   author_id lebedev_sergej_lohmatyj
 author_end
\fi

Все чаще говорится о «красных линиях» и о том, что мир на пороге войны.
Глобальную войну не хочет никто, кроме самоубийц. В ней не будет победителя,
лишь выжившие. Однако столкновение систем за право существования, вероятнее
всего, будет. И конечно, смысл в такой войне – уничтожение конкурента, который
хочет владеть или владеет определенными ресурсами. А кто у нас на планете
обладает самыми большими ресурсами?

Западу нужно втянуть Россию в войну. И Украина в этой ситуации даже не поле
боя, а трава на поле боя! Украинские власти готовы переходить «красные линии»,
о которых говорил президент Путин. Но готовы ли перейти их владельцы
украинского правительства? Владимир Путин также говорил о том, что удары будут
наноситься и по центрам принятия решений, а это совсем другой поворот.

\ii{24_11_2021.stz.news.nvr.voshod_info.1.oni_tam_est_krasnyje_linii.pic.1}

Но украинская власть переходит уже не только «красные линии», которые
обозначены Владимиром Путиным, но и те, которые существуют на самой Украине у
народа. Для украинской власти удержаться – вопрос выживания. А удержаться можно
только, вводя тотальный диктат. На Украине и так свобод практически нет. И
совершенно оправданно Владимир Путин высказался об отношении к украинским
соцопросам, заявив, что люди на Украине запуганы, и честно ответить на вопросы
рискнет далеко не каждый.

Для усиления диктатуры нужен хотя бы номинальный повод. Таким поводом вполне
могут стать бунты населения, которое останется без отопления и электричества в
городах. Сказки о том, что у Украины есть резервы и все отлично, могут
действовать только на сознание совсем недалекого человека. С радостью встречали
пароход, на котором Ахметов привез 60,5 тыс. тонн угля. Вроде бы огромная
цифра? На самом деле это примерно 900 вагонов с углем, т.е. 17-18 составов. Все
дело в том, как показать цифру, если в килограммах, то это будет вообще
гигантская цифра. А Украине для прохождения зимы нужно примерно 3,5 миллионов
тонн угля. Разницу понимаете?

Человек, который боится, что у него отнимут что-то нужное, может покориться. А
чем запугать человека, у которого уже н чего отнять? Чем можно угрожать
человеку, у которого от голода загибаются дети, а от холода простыли и умерли
родители? Чем можно угрожать тому, кто не может оплатить электричество и
отопление? А если может, но просто нет ни электричества, ни отопления? Именно к
такому рубежу стремится Украина. Веерные отключения уже не за горами. И если в
1990-е инфраструктура была старенькая, но еще живая, то сейчас она уже мертвая,
ей просто припарки ставят и палкой шевелят.

У Зеленского есть два варианта: перенести все проблемы на Юго-Восток, вытягивая
инфраструктуру хотя бы на Западе и в Центре. Таким образом можно нарваться на
бунты, но Юго-Восток – это ментально не Украина. Это НОВОРОССИЯ! Так можно
смело начать еще одно АТО. Причем не возле границ РФ, где может подуть
«северный ветер», а там, где живут отъявленные «сепаратисты». Любому Гитлеру
нужен враг, из-за которого все проблемы. И если у многих стран враг реален, то
на Украине единственный и основной враг – сама украинская власть. Украина
готовится отдать Юго-Восток, но … лишь то, что от него останется.

И если у Зеленского получится, то города Новороссии ждет то, что я видел в
Луганске 2014 года, и что можно сейчас видеть на линии соприкосновения в ДНР и
ЛНР.

Недавно сидели с земляком и его товарищем, вспоминали начало. Товарищ его родом
с Дебальцево, но воевал в ЛНР. Как оказалось, мы пересекались несколько раз.
Один из тех пересечений мы вспомнили, обсуждая фильм «Солнцепек». Фильм
тяжелый, но не показывает и 10\% того, что пришлось пережить местным жителям. В
фильме был показан момент, когда обстреляли детский интернат. Мы оба
присутствовали при похожем моменте. Я как-то рассказывал, как мы откапывали
детей, как я через руины разговаривал с мальчишкой, а он передавал мои слова
другим детям, которые были дальше. Он не выжил тот пацан, я буквально минут на
пять опоздал… на нем лежал кусок стены, он просто задохнулся.

А товарищ вспомнил то, что я всеми силами пытался забыть: камни многие были
скользкими от крови и на некоторых были прикипевшие или прилипшие от взрыва
части тел людей, детей в том числе. Вспомнили, как во время Дебальцевской
операции договорились с Украиной о частичном перемирии, чтобы жители Чернухино
смогли эвакуироваться. Люди собрались на площади, подъехали автобусы с
украинской стороны и со стороны ЛНР. В украинские автобусы село человек 12-15,
все толпились около луганских автобусов, стремясь уехать подальше от Украины.

Увидав такой поворот дел, украинские «воины света» забыли о перемирии и накрыли
площадь артиллерией, снайперским огнем, пулеметными очередями. Там был
показательный случай: наши парни вставали ПЕРЕД автобусами, отстреливаясь и
принимая пули и осколки на себя, спасая мирных жителей. А бронежилет тебя не от
всего защитит, погибло много настоящих русских по духу людей. А второй случай я
видел некоторое время спустя. Украинские военные ИЗ-ПОД автобуса вели обстрел
наших позиций. Они прикрывались людьми в автобусе, понимая, что мы не станем
отвечать.

Мы бы очень не хотели, чтобы такое пришлось пережить Харькову, Одессе,
Днепропетровску… Самое опасное – в этих городах стоят необстрелянные части. А
необстрелянный солдат не понимает, что он творит, стреляя. Только когда ты
видишь дело рук своих или рук своего врага, ты понимаешь, что такое война на
самом деле. Но ведь тогда будет поздно и прольется кровь тысяч обычных мирных
людей. И нам становится от этого страшно.

Да, мы боимся. Мы знаем, что это, мы знаем, как это будет и мы боимся. И лжец
тот, кто был в бою и скажет, что не боится. И боишься обычно не за себя, за
того, кто рядом, за тех обычных граждан, которые вынужденно доверили тебе свои
жизни и жизни своих близких.

И вот уже перед самым боем, у тебя просто мандраж. Не от страха, от ожидания.
Поднимается давление и в ушах так шумит, что ты боишься не услышать
долгожданного приказа. Страшно ли под обстрелом? Конечно! Ну как может быть
иначе, если ты не самоубийца? А потом вспоминаешь страх в глазах детей… Дети не
должны бояться! Никакие дети! Ни русские, ни украинские, ни французские, ни
чернокожие, ни азиаты. Дети – это просто дети. И за страх в их глазах у тебя
поднимается не злость, ни гнев, а ЯРОСТЬ. Она затопит тебя, и ты, внешне
совершенно спокойный, просто механически устраняешь причину страха.

Потом, когда запах пороха, кажется, навсегда въелся, когда в ушах еще звенит от
грохота взрывов, ты вдыхаешь полной грудью и понимаешь, что каким-то чудом во
всем этом ты остался живой. Смотришь на дело рук своих и начинаешь
по-настоящему ненавидеть тех, кто сделал выбор за тебя, приведя войну в твой
дом. Эта ненависть всегда с тобой, даже когда целуешь любимую женщину.

Вот об этом мы говорили до самого утра. Выпито было немало, но мы были
трезвыми. И страшно нам и за тех, кто здесь, и за тех, кто ТАМ, но ведь и
выбора другого нет. И именно нам придется освобождать наши города, неся свою
ненависть и ярость дальше — но лишь в отношении тех, кто всё начал. Россия не
оставит нас без помощи – в этом даже не сомневаюсь. Но сделать должны именно
мы, это наше личное.

Подует ветер с Донбасса. Он будет яростный и бить наотмашь. Думаю, что
Зеленский и многие чиновники укроются в Британии. На оставшийся кусок Украины
приведут президента Авакова, Разумкова, да хоть Медведчука! Новороссия будет
непризнанной республикой – и это очень хорошо. А потом, не так уж скоро, но
обязательно станет Южно-русским федеральным округом Большой России.

Тот товарищ, с которым разговаривали и вспоминали прошлое, начал войну у
Стрелкова в Семеновке. Там попал под фосфорные снаряды. В России его выходили,
дальше он воевал в ЛНР. После освобождения Дебальцево, приехал домой,
посмотрел… и уже в Донецке опять ушел служить. У него молодая жена и маленькая
дочь. А еще после фосфора у него онкология… Вот так и живем и все равно планы
на будущее строим.

Мы не вернемся никогда с войны. Все эти Зеленские, Яроши, Тягнибоки и рангом
пониже военные преступники, к моему сожалению не самоповесятся. Хотя
вероятность сесть в машину с испорченным кондиционером у них очень велика. А
значит нам придется их всех найти и закончить начатое. И если не ради
собственной ненависти, то для того, чтобы дети не боялись.

А вы помогаете с другой стороны нам всем. Вы помогаете выжить тем, кто
пострадал от украинских нацистов, помогаете семьям погибших, поддерживаете в
них и во многих адекватных людях Окраинской территории надежду.

До сих пор некоторые «профессиональные русские» стараются унизить выживших на
ТОЙ стороне, рассказывая, что русские не придут, что у русских свои проблемы и
больше никто им не нужен. И в какой-то степени могу согласиться с пропагандой –
русские не придут! Русские просто никуда не ушли – это нужно понять тем, кто на
ТОЙ стороне живет. Русские ТАМ есть! И они, осознав себя, сопротивлялись
насколько умели. Мы проиграли бой, но не войну. И если иметь ввиду народ,
живущий в РФ, то нам оказана огромная помощь, начиная от добровольцев, которые
пришли сражаться в трудные для нас месяцы и годы, и заканчивая простой
моральной и материальной поддержкой русских людей.

В самом начале хочу рассказать об одной семье, которой вы ранее помогали. Если
помните, была семья, где мать после нескольких лет в СБУ потеряла здоровье, а
дети находились на попечении бабушки-пенсионерки. Бабушка — учитель русского
языка и литературы. На современной Украине по идее не самая популярная
профессия. Но как оказалось, после запретов на преподавание на русском языке,
много родителей, сидя с детьми на удаленке, поняли, что их чада по части
грамотности абсолютно бестолковые. Ситуацию надо исправлять. Кто-то своими
силами, кто-то нанимает репетиторов, причем тоже по удаленке. Так профессия
учителя русского языка и литературы стала очень популярна на украинской
территории.

Когда я связался с их семьей, чтобы перевести очередную помощь, оказалось, что
они не только справляются сами, а даже попросили связать с кем-то на Украине,
кому нужно помочь и сами оказали помощь! Я искренне рад за них!

\textSelect{Один читатель с CONT.WS оказывает помощь одесситке Марине}. Марина пострадала 2
мая, чудом оставшись живой. Множество переломов, повреждение позвоночника. С
Мариной мы на связи, она благодарна всем вам, особенно «товарищу Б». На Украине
давно декоммунизировали такое понятие, как бесплатная медицина. Впрочем, как и
бесплатное образование, бесплатные обеды в детсадах и школах и многое другое.
Сейчас докоммунизируют инфраструктуру, упадут последние мосты и наступит
Европа… правда, раннего Средневековья, но с чего-то надо начинать.

\textSelect{Вы помогли семье из Нежина (город в Черниговской области)}. В семье осталась
восемнадцатилетняя дочь погибшего на майдане милиционера и ее трое младших
братьев и сестер. Мама в начале октября умерла от коронавируса. Старшая дочь
оформила опекунство детей на себя, в противном случае им грозил детский дом,
что само по себе не хорошо, а на Украине это вообще жесткое место.

\textSelect{Вы помогли семье из Николаева}, в которой тоже старшая дочь взяла опекунство над
братом. Их мать убили у них на глазах, пытаясь «разговорить» отца, который
оказывал действительно серьезное сопротивление. Во многом благодаря ему,
Николаев считается очень негостеприимным городом для украинствующих
криминальных нацистов.

\textSelect{Одна девушка из США помогла семье с Донбасса}. Семья живет по ТУ сторону
ленточки. В семье было трое сыновей, у старшего есть дети. Старший сын погиб в
самом начале войны, младший осенью 2016, средний до сих пор сражается против
украинствующих нацистов. Дети старшего живут с бабушкой и дедушкой, именно им
русская американка и вы оказали помощь.

\textSelect{Вы помогли семье из Луцка (Западная Украина), отец и сын погибли}. Сын погиб,
сопротивляясь майдану, отец погиб на Донбассе, во время боев за Дебальцево. У
сына остался ребенок, живет с бабушкой-пенсионеркой. Мать решила не связывать
судьбу с «сепаратистами» и уехала на ПМЖ в Италию. О сыне она все же вспоминает
по праздникам, а вы помогаете ему регулярно.

\textSelect{И конечно огромная благодарность вам от Якута}, который заполняет сайт
\href{https://voskhodinfo.su}{«ВосходИнфо»}. Кто не в курсе – напомню. Этот мужик встал на защиту Луганска.
Вовремя деблокады города он и его товарищи, которые были с ним в УАЗике,
подорвались на мине. Выжил только Владимир (Якут). Потом был ростовский
госпиталь, где его собрали, затем госпиталь в Питере. Хочу обратить внимание,
что там никто с него ни копейки не взял.

Мужик остался инвалидом. Ног своих он не чувствует, хотя ходит. На работу с
такой проблемой здесь не устроишься. Есть пенсия по инвалидности, но есть и
семья, и дети. Практически все, что вы видите на «ВосходИнфо», опубликовал
Якут. Мне особенно нравятся его ежедневные сборники «Новороссия сегодня» и
«Сводки»

Кто может оказать помощь Якуту: Карта Сбербанка РФ 4276 5209 7902 5125 Владимир
Алексеевич С.

Кто хочет связаться с Владимиром напрямую, сделать это можно через
«Одноклассники» \url{https://ok.ru/profile/589768523292}

Вы помогли семье из Ужгорода. Обычный фермер не принял государственный
переворот и не смог жить, не сопротивляясь украинствующим нацистам. Бывший
офицер, прошедший Афганистан, присоединился к ополчению Донбасса. Один из
лучших снайперов, погиб при обстреле во время освобождения Углегорска. Его жена
была арестована, и несколько лет провела в СИЗО, выпустили не так давно, с
больным сердцем и полуослепшую от побоев. У них две дочери, старшая скоро
закончит школу, младшая еще учится. Живут с сестрой погибшего героя. Детей
нормально приняли и ни в чем старались не отказывать. В сентябре сестра умерла
от коронавирусной пневмонии, семье понадобилась помощь, и вы ее оказали.
Огромная вам благодарность от них. Дальше о семье позаботится дядечка-венгр, о
котором я пару раз упоминал.

Вы помогли семье из Херсонской области. В семье мать и два ребенка, их отец
погиб, останавливая «автобусы дружбы», которые весной 2014 пытались проехать в
Крым. До войны у отца было небольшое прибыльное дело, после его гибели, жена
была арестована, дети остались на попечении стариков. Супругу выпустили в конце
2018 года, конечно больную. Старики выходили и ее, но украинской пенсии не
хватает не то что на лекарства, на нормальную оплату жилья. Семье переехала в
область, живут в деревне. С вашей помощью они смогли купить семь кубов дров на
зиму. Ведь зима будет холодная, а украинская власть плевать на граждан хотела.
Хорошо, что у НАШИХ людей есть ВЫ.

Вы помогли семье парня из Донецка. Молодой человек, оставшись без родителей,
погибших при обстреле в 2016 году, стал Мужчиной в семье в 14 лет. Сейчас от
него зависит бабушка и младший брат. Парень работает как вол и старается
успевать учиться на заочном отделении в местном вузе. Я не знаю, смог бы я в
его возрасте стать опорой для семьи? Если бы вы видели во что он одет… Вы
оказали ему помощь небольшую, но очень нужную. И он и его младший брат теперь
нормально одеты к зиме, а у бабушки есть нужные лекарства.

Вы помогли начать собирать помощь, которую отвезу Батюшке Андрею в ЛНР.

Теперь самое впечатляющее чтиво для тех, кто ТАМ есть. Конечно читают они текст
целиком, но эта часть, как пишет большинство людей с ТОЙ стороны – самая
светлая и дающая надежду часть. Ваши комментарии, пришедшие вместе с помощью:

\obeycr
«Из Томска. Наши везде» - спасибо вам большое! Наши везде, мы – цивилизация!
 «Русским! С верой в справедливость. НН.» - спасибо! Справедливость – это и есть русские, какой бы нации или расы они ни были.
 «Спасибо вам за вашу работу. Снежинск» - спасибо, что всегда помните о нас!
 «Нашим» - благодарю вас!
 «Детям и старикам Донбасса» - сделано! Спасибо большое!
 «помощь старикам» - спасибо вам
 «Тверь. Спаси Бог» - Благодарю тверчане!
 «Читаем вас. Ждем домой. СПБ. Нам не все равно!» - Спасибо вам за постоянную поддержку питерцы!
 «Спасибо, что вы ТАМ есть! Мы с вами!» - Благодарю, без вас нас бы наверное уже и не было.
 «Трем сыновьям» - спасибо от младшего, пока все ему досталось
 «С наступающим ДР» - спасибо огромное!
 «Для русских на Донбассе» - благодарим вас
 «Держитесь, родные. Пенсионерка Ирина. Москва» - спасибо большое Столице и вам лично!
 «Для Донбасса» - спасибо вам
 «С праздником 7.11. Нашим ТАМ. СПБ с вами!» - спасибо питерцы, ваш город пережил куда более страшные годы во время прошлой войны, вы знаете как это. А 07.11 для меня – свержение либаралов, которые совершив госпереворот в феврале, начали разбазаривать и уничтожать нашу Родину. Не все гладко было и с большевиками, но победили державники. Сейчас опять либералы и троцкисты хотят взять реванш. ВВП к счастью суровее Николая, но аврора должна быть готова сделать залп.
 «НАШИМ. Вместе победим. Сил ЛДНР. СПБ с вами» - Спасибо огромное. Да, победим, и конечно будем вместе. Пройдет много или мало времени, но будет именно так.
 «Из Нижнего Новгорода» - спасибо Нижний. Много настоящих товарищей у меня с ваших мест. Командир казачьего полка Ратибор, Андрей Тихомиров ... многие другие настоящие русские люди.
 «Детям и старикам Донбасса» - спасибо вам большое!
 «Помощь нуждающимся. Оренбург с вами» - без вас не получилась бы, это вам спасибо, что помогаете.
\restorecr
 

Огромная благодарность всем, кто смог помочь, несмотря на что, эпидемия и ее
последствия внесли свои коррективы. Уровень жизни не поднялся отнюдь, но вы
помогаете чем можете. Не меньшая благодарность тем, кто помогает распространять
стати этого направления. Люди должны знать, что Они ТАМ есть! Особенно люди,
живущие на украинской территории, которые находятся в информационной изоляции
под постоянным прессингом украинских СМИ, лжи и пропаганды. Людям важно
понимать, что они не одни, что их никто не бросил и что мы вернемся за своим.

Мой старший товарищ, который участвовал во многих конфликтах, говорит, что это
самая подлая война, которую он видел. Людей не просто убивают, их так морально
уничтожают, что многие еще внешне живые внутри уже давно мертвые, и просто
существуют без надежды. Вы разрушаете это стену лжи, вы даете надежду. Это
самая важная ваша помощь.

Власти Украины готовы перейти последние «красные линии», а значит нам всем
нужно быть готовыми возвращать свое. Но не менее важно сделать так, чтобы потом
не было войны, в которой будут погибать наши внуки и правнуки. Как мне писали и
не раз: надо все по суду, по закону, соблюдая права человека… Я не согласен!
Надо максимально эффективно, чтобы не было повторения, чтобы слова «бандера»
воспринималось с той же брезгливостью и омерзением, что и слово «педофил».

Никто из тех, кто виновен в смерти детей, в гибели целых семей, не достоин
снисхождения. И не важно, лично он произвел выстрел или поддерживал
преступления сидя в Раде, посылая нацистскую мерзость истязать людей. Чтобы
использовать «права человека», нужно быть Человеком. Человек может оступиться и
раскаяться после. Но не считаю человеком того, кто ОЗОЗНАННО убивал или потакал
убийствам мирных, ни в чем не повинных граждан. Я не считаю, что их жизнь
дороже не то что жизни ребенка, не дороже его слезинки, и уж явно дешевле того
страха в детских глазах, который он принес на эту землю. Я хочу видеть похожий
страх в его глазах и лишь тогда я буду уверен, что мы вернули свое, а «красные
линии» стоят нерушимыми.

Кто сможет оказать помощь НАШИМ людям: 

Карточка «Сбербанк России»: 4276 5219 7344 7208 

Яндекс-деньги: 410013477797584 

Вебмани: 

Доллар: Z933002199485 

Евро: E454061128185


Мы благодарны за любую оказанную помощь, словом, делом, распространением
статей в поддержку НАШИХ людей на ТОЙ стороне. Мы знаем: даже люди при немалых
деньгах читают эти статьи и оказывают помощь НАШИМ людям. Мы знаем: от
Владивостока и до Мюнхена есть НАШИ люди, которые не бросили своих в беде. Мы
убедились, что те, кому помогли, сами не забывают помогать другим.

А киевским переворотчикам не стоит надеяться на демократичный суд российских
граждан, им придется иметь дело с Донбассом. А Донбасс порожняк не гонит!

Любая ваша помощь найдет свою благодарность. Кто-то опять удивится, что «аж в
самой России» знают и помнят, кто-то прочтет эту статью, которую вы помогли
распространить.

Если ты сопротивлялся украинствующим нацистам, и семья твоя из-за этого
пострадала, свяжись с нами через форму обратной связи на сайте ВосходИнфо, чем
сможем – поможем. Можно связаться со мной в любом аккаунте социальных сетей.
Если знаешь тех, кто из-за майдана потерял все, кроме мужества, кто выживает с
трудом, свяжи нас с этим человеком. Но есть одно «НО» - это должен быть вопрос
выживания. Мы, к сожалению, многим помочь не сможем. Потому помогаем тем, у
кого крайняя нужда.
