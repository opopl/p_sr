% vim: keymap=russian-jcukenwin
%%beginhead 
 
%%file 17_03_2021.fb.ozhegov_sergej_775.1.cennosti_semja
%%parent 17_03_2021
 
%%url https://www.facebook.com/permalink.php?story_fbid=108758981292334&id=100064746976226
 
%%author 
%%author_id ozhegov_sergej_775
%%author_url 
 
%%tags cennosti,pamjatnik,rossia,semja,usa
%%title Нам выбирать, что нам ближе
 
%%endhead 
 
\subsection{Нам выбирать, что нам ближе}
\label{sec:17_03_2021.fb.ozhegov_sergej_775.1.cennosti_semja}
 
\Purl{https://www.facebook.com/permalink.php?story_fbid=108758981292334&id=100064746976226}
\ifcmt
 author_begin
   author_id ozhegov_sergej_775
 author_end
\fi

Нам выбирать, что нам ближе.

Слева памятник традиционной семье - символ незыблемых ценностей, которые
являются и будут являться фундаментом общества. Двое взрослых - мама и папа,
рядом дети - старшая дочь, сынишка и младшенькая, плюс ещё малыш в животе у
беременной мамы. 


\ifcmt
  tab_begin cols=2

     pic https://scontent-cdg2-1.xx.fbcdn.net/v/t1.6435-9/161486795_108758911292341_2008575581695154959_n.jpg?_nc_cat=104&ccb=1-5&_nc_sid=8bfeb9&_nc_ohc=-d3y5e6IJwsAX_34jzf&_nc_ht=scontent-cdg2-1.xx&oh=fa2abe31d481ad8fffe1ec01e0e873c1&oe=61428C2C

     pic https://scontent-cdt1-1.xx.fbcdn.net/v/t1.6435-9/161490755_108758941292338_7437577343149924206_n.jpg?_nc_cat=103&ccb=1-5&_nc_sid=8bfeb9&_nc_ohc=L5bNAbEvQnIAX9pfIY7&_nc_ht=scontent-cdt1-1.xx&oh=3eb8980c92b1bdecdd8e3b7ac214d768&oe=61425D4D

  tab_end
\fi

Минимум четверо детей в семье - это уверенное продолжение человеческого рода.

Памятник расположен на площади в г.Саранск. Россия. 2008 год.

Справа памятник геям и лесбиянкам. США г.Нью-Йорк. 1992 год.

Глядя на хронологию,  можно понять, что памятник традиционной семье - это ответ
на навязываемую нам с некоторых пор нетрадиционную культуру Запада, которая
начала становится таковой сразу после распада СССР, когда соревноваться за
уровень жизни Западу стало уже не с кем.

На смену американской мечте о домике с лужайкой и бассейном для каждой семьи с
несколькими детьми начало приходить то, где детям места нет. 

А ведь дети - это наше будущее. 

И каким оно будет зависит именно от нас.
