% vim: keymap=russian-jcukenwin
%%beginhead 
 
%%file 21_04_2022.fb.solovjov_mikita.harkov.demsokyra.2.evakuacia
%%parent 21_04_2022
 
%%url https://www.facebook.com/Mikita.Solovyov/posts/7424524290951253
 
%%author_id solovjov_mikita.harkov.demsokyra
%%date 
 
%%tags 
%%title Об эвакуации
 
%%endhead 
 
\subsection{Об эвакуации}
\label{sec:21_04_2022.fb.solovjov_mikita.harkov.demsokyra.2.evakuacia}
 
\Purl{https://www.facebook.com/Mikita.Solovyov/posts/7424524290951253}
\ifcmt
 author_begin
   author_id solovjov_mikita.harkov.demsokyra
 author_end
\fi

Об эвакуации.

Я не понимаю, как стыкуются между собой две мантры, постоянно звучащие от
городских и областных властей в публичном пространстве. 

«Необходимости в эвакуации Харькова нет» и «Старайтесь не покидать убежища без
крайней необходимости» в разных вариантах, в том числе практически после
каждого отбоя тревоги. 

Ведь всем же уже понятно, что ситуация достаточно стационарна. Что угроза
обстрелов и бомбежек в Харькове будет сохраняться до конца войны просто в силу
географии. То есть это совсем не «день простоять и ночь продержаться». Нужно
готовиться к тому, что это на месяцы. 

И дальше мы возвращаемся очередной раз к моему постоянному вопросу «Зачем?».
Если угроза настолько высока, что есть необходимость постоянно находиться в
убежищах, то какой смысл тогда вообще сидеть в городе? Ведь сидя безвылазно в
убежище никаких особых шансов делать что-то полезное нет. А то немногое, что
вполне можно делать и в убежище, с тем же успехом можно делать и сидя в более
спокойном регионе. И если мы считаем, что пребывание в городе само по себе
несет такой высокий уровень опасности, тогда может быть разумнее на уровне
городских и областных властей активно рекомендовать эвакуироваться? А если
человек каким-то образом задействован в помощи обороне, в обеспечении
жизнедеятельности города, в помощи нуждающимся в ней и т.д., то какой смысл его
призывать к тому, что он физически не сможет сделать, не прекратив свое
занятие? 

При этом массовая централизованная эвакуация даже пяти наиболее пострадавших
микрорайонов не проводится. Хотя в них уж точно пребывание связано с высоким
риском. Там уже большинство оставшихся и так сидят по убежищам постоянно, а
активно перемещаются только волонтеры, обеспечивающие всем необходимым. Причем
волонтеры при этом постоянно подвергаются высокой опасности. Причем как минимум
часть этих районов подвергаются регулярно не только обстрелам, туда еще и время
от времени пытаются влезть российские военные. И в такие моменты мирное
население еще и заметно мешает нашим военным. Казалось бы, уж эти микрорайоны
точно нужно и можно эвакуировать. Но нет, эвакуацией занимаются в
индивидуальном порядке волонтеры, а от города никаких системных шагов не видно.
«Необходимости в эвакуации из Харькова нет!»

Эти два призыва еще как-то сочетались без особых противоречий в первые
неделю-две. Когда многие действительно верили, что все закончится за считанные
если не дни, то недели. Но теперь же вроде бы уже ни у кого из принимающих
решения нет подобных иллюзий? Так какой смысл в том, чтобы фактически
подталкивать людей сидеть месяцами в убежищах? Мне кажется, что самым
правильным действием было бы следующее:

1) Централизованно эвакуировать полностью пять наиболее пострадавших и
продолжающих постоянно получать урон микрорайонов. 

2) Уведомить население и постоянной угрозе и о том, что скорее всего она
продлится еще долго. 

3) Предложить харьковчанам, не задействованным в полезной деятельности в связи
с этими рисками эвакуироваться. Централизованно эвакуировать в другие регионы
все убежища, включая метро. Оказать помощь в эвакуации всем желающим. 

Давайте хотя бы во время войны относиться к харьковчанам не как в малым
неразумным детям, а как к взрослым, людям, могущим брать на себя
ответственность за свои действия, свой выбор и их последствия! Честное слово,
за последние два месяца харьковчане заслужили такое отношение.

\ii{21_04_2022.fb.solovjov_mikita.harkov.demsokyra.2.evakuacia.cmt}
