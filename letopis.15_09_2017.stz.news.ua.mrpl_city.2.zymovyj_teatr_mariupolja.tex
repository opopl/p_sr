% vim: keymap=russian-jcukenwin
%%beginhead 
 
%%file 15_09_2017.stz.news.ua.mrpl_city.2.zymovyj_teatr_mariupolja
%%parent 15_09_2017
 
%%url https://mrpl.city/blogs/view/zimovij-teatr-mariupolya-istoriya-odniei-vtrachenoi-budivli
 
%%author_id demidko_olga.mariupol,news.ua.mrpl_city
%%date 
 
%%tags 
%%title Зимовий театр Маріуполя: історія однієї втраченої будівлі
 
%%endhead 
 
\subsection{Зимовий театр Маріуполя: історія однієї втраченої будівлі}
\label{sec:15_09_2017.stz.news.ua.mrpl_city.2.zymovyj_teatr_mariupolja}
 
\Purl{https://mrpl.city/blogs/view/zimovij-teatr-mariupolya-istoriya-odniei-vtrachenoi-budivli}
\ifcmt
 author_begin
   author_id demidko_olga.mariupol,news.ua.mrpl_city
 author_end
\fi

\ii{15_09_2017.stz.news.ua.mrpl_city.2.zymovyj_teatr_mariupolja.pic.1}

Центральна вулиця Маріуполя – проспект Миру – береже в собі безліч таємниць,
легенд, історій і знакових подій. Це та магістраль з минулого в сучасне, яка
дозволяє нам краще зрозуміти власне місто, подивитися йому в очі і зазирнути в
найпотаємніші куточки його давнього минулого. Однак чи потрібно знати, що
раніше знаходилося на тому чи іншому місці головної вулиці Маріуполя? Думаю, це
дійсно дуже важливо \emph{для розуміння, хто ми є, хто наші пращури, що вони нам
заповіли...}

\ii{15_09_2017.stz.news.ua.mrpl_city.2.zymovyj_teatr_mariupolja.pic.2}

Не всім маріупольцям відомо, що на місці будинку № 24 по пр. Миру наприкінці
XIX ст. збиралися шалені черги за квитками на найбільш яскраві вистави. Тут
можна було побачити чоловіків у гарних костюмах, дам у шикарних вечірніх
туалетах, які поспішали насолодитися грою як місцевої трупи, так і видатних
гастролерів. Адже саме на цьому місці була побудована будівля Зимового театру
чи Концертної зали, яка стала центром театрального і культурного життя не
тільки Маріуполя, але й всього Приазов'я.

\ii{15_09_2017.stz.news.ua.mrpl_city.2.zymovyj_teatr_mariupolja.pic.3}

Появі першого стаціонарного театрального приміщення в місті посприяла активна
діяльність та великі зусилля талановитого артиста, режисера, засновника першої
професійної маріупольської трупи Василя Леонтійовича Шаповалова. Вражаючим є
той факт, що він, грек за походженням, закохався в українку, що стало причиною
відмови його забезпеченого батька допомагати фінансово, адже в ті часи в
Маріуполі вважалося неприйнятним одружуватися знатним грекам на жінках інших
національностей. І все ж таки попри матеріальні труднощі та завдяки
наполегливій праці В. Шаповалов втілив у життя свою мрію. 

\ii{15_09_2017.stz.news.ua.mrpl_city.2.zymovyj_teatr_mariupolja.pic.4}

У 1887 р. Василь Леонтійович закінчив будівництво театру на 800 місць з
буфетом, фойє, касами та назвав його \enquote{Концертною залою}. Зимовий театр
урочисто відкрили 8 листопада 1887 р. прем'єрою  \enquote{Ревізора} Миколи
Гоголя.  При переповненій залі Василь Шаповалов з великим успіхом зіграв роль
Городничого.  Відкриття нового театрального приміщення стало великою подією в
культурному житті міста, про нього писали в столичних і провінційних газетах
Катеринослава, Ростова, Харкова, Новочеркаська, Таганрога. Після побудови
нового театру в Приазов'я були залучені кращі театральні сили України та Росії.

На сцені Зимового театру Маріуполя були показані кращі сучасні вистави, не
могли не вразити й знамениті гастролери, такі як заслужена артистка
Імператорських театрів Гликерія Федотова, видатний російський актор Павло
Орленєв, яскраве подружжя оперних співаків Давид Южин та Наталя
Єрмоленко-Южина; виступала і трупа Всеволода Мейерхольда. Зачаровували своєю
грою зарубіжні актори: алжирська актриса Галіма, японська актриса пані Ганако,
італійська трупа Гонсалеца та багато інших.

На жаль, проблеми зі здоров'єм і фінансові труднощі змусили Василя Шаповалова
продати будівлю І. І. Уварову, господареві чавуноливарного заводу. Останній
віддав театр в експлуатацію приїжджим антрепренерам.

\ii{15_09_2017.stz.news.ua.mrpl_city.2.zymovyj_teatr_mariupolja.pic.5}

Проте нова сторінка в історії будівлі почалася в 1932 р., коли в Зимовому
театрі почав діяти Маріупольський грецький національний театр. Відомо, що він
був першим в СРСР і налічував всього лише 24 актора. З них тільки Данило
Теленчі, Георгій Дегларі та Юрій Дранга були досвідченими професіоналами,
решта – аматори. Тому доводилося починати все з нуля в буквальному сенсі: у
важких житлових умовах, із заборгованістю по зарплаті і нешвидким визнанням
грецького глядача. Та завдяки гарному репертуару і талановитій грі трупі
все-таки вдалося стати на ноги і затвердити своє становище не тільки в місті,
а й у селах. Грецький театр став стрімко розвиватися. У 1934 р. Маріупольська
міська рада дала добро на реконструкцію театрального будинку. З початком
ремонтних робіт розширилася гастрольна діяльність театру, всюди виступи
Грецького театру мали успіх.

\ii{15_09_2017.stz.news.ua.mrpl_city.2.zymovyj_teatr_mariupolja.pic.6}

14 березня 1936 р. відбулося урочисте відкриття міського театру. Великий успіх
серед глядачів мали актори С. Янгічер, М. Чіча, які дивували виразним читанням
та грецькими народними танцями, Ф. Кашкер вражав румунським народним танцем, а
Л. Карнаухова – сольним співом. Багато акторів блискуче володіли музичними
інструментами. Особливий успіх мала Людмила Карнаухова – вона була єдиною
українкою в грецькій трупі. Але завдяки високим професійним якостям та
енергійності вона стала примою державного грецького театру, улюбленицею
публіки. Проте у середині 30-х років настав час партійних чисток, репресій. 28
січня 1938 року творчий колектив грецького театру несподівано для маріупольців
було оголошено \enquote{посереднім театром, збереження якого не є необхідністю}, і
закрито, найбільш активні діячі театру - режисери та актори – репресовані.

\ii{15_09_2017.stz.news.ua.mrpl_city.2.zymovyj_teatr_mariupolja.pic.7}

Завдяки ініціативі та наполегливості громадського діяча і режисера народного
театру ПК заводу \enquote{Азовсталь} І. Налчаджи на будинку № 24 по проспекту Миру
встановлено меморіальну дошку в пам'ять про знищений грецький театр. Однак на
табличці відсутні відомості, що в цьому ж приміщенні знаходився відомий далеко
за межами Маріуполя Зимовий театр...

Кажуть, час – це той матеріал, з якого створене життя. Він біжить невблаганно,
нікого не шкодуючи на своєму шляху. Не пошкодував він і перше театральне
приміщення, про існування якого знають і пам'ятають далеко не всі маріупольці.
Проте ніколи не пізно відновити зв'язок з минулим і почати цікавитися історією
власного міста, своїм корінням, що, безумовно, розширить кругозір і змінить
ставлення до рідного краю. 
