% vim: keymap=russian-jcukenwin
%%beginhead 
 
%%file 08_06_2023.stz.news.ua.donbas24.1.mariupol_aktrisa_virshi_nimecchyna_vystupy_olga_samojlova.txt
%%parent 08_06_2023.stz.news.ua.donbas24.1.mariupol_aktrisa_virshi_nimecchyna_vystupy_olga_samojlova
 
%%url 
 
%%author_id 
%%date 
 
%%tags 
%%title 
 
%%endhead 

Ольга Демідко (Маріуполь)
Маріуполь,Україна,Мариуполь,Украина,Mariupol,Ukraine,date.08_06_2023
08_06_2023.olga_demidko.donbas24.mariupol_aktrisa_virshi_nimecchyna_vystupy_olga_samojlova

Маріупольська актриса з українськими віршами виступає в німецьких містах

Маріупольські митці продовжують популяризувати українську культуру за кордоном

6 червня в місті Білефельд відбувся поетично-музичний український вечір за
участю колишньої актриси Донецького академічного обласного драматичного театру
(м. Маріуполь) та режисерки і актриси Народного театру «Театроманія» Ольги
Самойлової. Актриса прочитала вірші відомих майстрів українського слова Тараса
Шевченка, Лесі Українки, сучасних українських авторів Сергія Жадана та Катерини
Калитко і маріупольської поетеси Оксани Стоміної.

Читайте також: У Києві відбулося прощання з героїчним маріупольцем (ВІДЕО)

Організатором українського вечора стала Німецько-українська організація в місті
Білефельд (Північний Рейн Вестфалія). Ця програма проводилась для підтримки
дітей, що постраждали внаслідок російської агресії. Зібрані кошти будуть
передані в Дніпро. 

«Я дуже багато разів вдома перед дзеркалом репетирувала з цими аркушами.
Доходжу до якогось важкого місця і не можу стримати сліз», — розповіла Ольга
про підготовку до заходу.

Разом з Ольгою Самойловою виступили скрипалька з України Наіра Арзуманян,
вірменка, яка з 1995 року жила у Дніпрі. Також виступила німецька піаністка
Клаудія Кьоль та професійний актор Білефельдського театру Томас Вульф. Ольга
Самойлова читала вірші українською мовою, а Томас читав переклади німецькою
мовою.

Читайте також: «Зустрінемося на драмі»: маріупольцям призначили побачення в
українському звільненому місті (ВІДЕО)

«Все пройшло на такому нерві. Для мене це було досить важко, тому що цього дня
я дізналася про страшні події, які відбулися в Каховці. І тому мені було дуже
важко читати ці вірші після всього, що я почула та побачила. Після всіх цих
подій і „Заповіт“ Тараса Шевченка, і „Contra spem spero“ Лесі України набули
нового сенсу. Вони сьогодні дуже актуальні. Я декілька разів зупинялася, щоб
стримувати свої емоції. Мені здавалося, що я себе віддаю всім по шматочку.
Емоційно було дуже важко», — поділилася актриса.

Читайте також: Маріупольський музей ім. А.І. Куїнджі відтепер можна відвідати,
не виходячи з дому (ФОТО)

Вечір викликав справжній фурор. До актриси підходили глядачі і дякували за
такий потужний та емоційний виступ. Після виступу Томас Вульф запропонував
Ользі створити спільний проєкт. Крім того, Наіра Арзуманян та Клаудія Кьоль
домовилися з актрисою повторити цю програму, але вже для учнів
німецько-української школи.

«У програму для учнів ми вирішили додати ще нові вірші, більш легкі та
зрозумілі для молоді. Але всі вірші, які були прочитані 6 червня, залишаться,
тому що вони дуже виразно передають те, що сьогодні відчувають всі українці», —
наголосила Ольга Самойлова.

Раніше Донбас24 розповідав, що театр авторської п'єси Conception представив
поетичну імпрезу «Лінії життя».

Ще більше новин та найактуальніша інформація про Донецьку та Луганську області
в нашому телеграм-каналі Донбас24.

Фото: з архіву Ольги Самойлової
