% vim: keymap=russian-jcukenwin
%%beginhead 
 
%%file 22_12_2020.news.ua.strana.tovt_anastasia.1.suprun_kniga_mochi_mantu
%%parent 22_12_2020
 
%%url https://strana.ua/reviews/308158-zavualirovannaja-propahanda-pod-prikrytiem-poleznykh-sovetov-o-chem-novaja-kniha-uljany-suprun.html
 
%%author 
%%author_id tovt_anastasia
%%author_url 
 
%%tags suprun_uljana,kniga
%%title Полезные советы и вредные "совєты". О чем книга Ульяны Супрун, где она сравнила себя с инопланетянкой
 
%%endhead 
 
\subsection{Полезные советы и вредные \enquote{совєты}. О чем книга Ульяны Супрун, где она сравнила себя с инопланетянкой}
\label{sec:22_12_2020.news.ua.strana.tovt_anastasia.1.suprun_kniga_mochi_mantu}
\Purl{https://strana.ua/reviews/308158-zavualirovannaja-propahanda-pod-prikrytiem-poleznykh-sovetov-o-chem-novaja-kniha-uljany-suprun.html}
\ifcmt
	author_begin
   author_id tovt_anastasia
	author_end
\fi

\ifcmt
pic https://strana.ua/img/article/3081/zavualirovannaja-propahanda-pod-58_main.jpeg
caption "Страна" прочитала новую книгу Ульяны Супрун "Мочи Манту". Большая ее часть - это не советы о здоровье, а пропаганда против России и СССР 
\fi

С 2016 по 2019 годы Ульяна Супрун исполняла обязанности министра
здравоохранения Украины. Теперь же она стала еще и писателем.
 
Супрун издала книгу с провокационным названием "Мочи манту" и не менее
провокационной обложкой. На ней изображен человеческий мозг с руками и ногами,
разрывающий оковы.
 
Это по совместительству еще и главный герой и символ книги – Здравый Смысл. 

\ifcmt
pic https://strana.ua/img/forall/u/0/25/IMG_20201221_132208(1).jpg
caption Книга Ульяны Супрун "Мочи Манту" поступила в продажу. Фото: Страна
\fi

Книгу можно купить только на специальном сайте с одноименным названием "Мочи
Манту". Цена - 295 грн. А доставка через почтовую службу корреспонденту
"Страны" заняла целых 5 дней. 
 
Отметим, что обложка книги выполнена в довольно "ядерной" цветовой гамме:
сочетание ярко-малинового, розового, желтого, черного, красного… А внутри
размещено несколько иллюстраций в виде комиксов с главным героем – мозгом. И
подарок для читателей – набор наклеек. 

\ifcmt
pic https://strana.ua/img/forall/u/0/25/IMG_20201221_132147.jpg
caption Книга Ульяны Супрун "Мочи Манту" продается по цене 295 грн на единственном сайте. В качестве бонуса - набор наклеек с изображением мозга. Фото: Страна
\fi

На оборотной стороне обложки изображен портрет самой Ульяны Супрун в виде
комикса, а под ней – человекоподобное розовое создание в виде мозга с очками и
бородой. Причем судя по бороде, этот мозг срисован с супруга Ульяны Супрун,
Марко. 

\ifcmt
pic https://strana.ua/img/forall/u/0/25/IMG_20201221_132223(1).jpg
caption Мозг в книге Ульяны Супрун срисован с ее мужа, Марко. Фото: Страна
\fi

О Марко Супруне "Страна" уже неоднократно писала.\Furl{https://strana.ua/articles/rassledovania/258493-facebook-nanjala-dlja-okhoty-za-fejkami-v-ukraine-orhanizatsiju-stopfake.html} Он неоднократно был замечен
на мероприятиях с ультраправыми, участвует в работе организации StopFake,
которая занимается цензурой в Фейсбуке. 
 
Но вернемся к содержанию книги экс-главы Минздрава.
 
Основной книги стали посты Ульяны Супрун из Фейсбука. А посвящается она  "всем
детям, которые мочили Манту и сидели на холодном". В целом же задача Супрун –
развенчать самые популярные мифы о здоровье.  
 
Книга издана на украинском языке. Во вступительном слове подробно описывается,
как в 1974 году 11-летняя Ульяна впервые приехала в Украину, которая тогда была
частью СССР. Причем Супрун описывает советский период не иначе, чем с
прилагательным вместо литературного прилагательного "радянський" почему-то
использует издевательскую кальку с русского "совєцький": "Совєцька
соціалістична республіка", "совєцький Львів" и т.д.

\ifcmt
pic https://strana.ua/img/forall/u/0/25/IMG_20201221_134405.jpg
caption Книга Ульяны Супрун посвящена всем детям, которые мочили Манту и сидели на холодном. Фото: Страна
\fi

В целом ключевой посыл вступления к книге – как было хорошо жить за границей и
как плохо в "совєцькой" Украине. "Совєцький Союз, – пишет Ульяна Супрун, – это
большой концентрационный лагерь для людей и правды".
 
Хотя непонятно, какое отношение все это имеет к мифам про здоровье, которое
экс-глава Минздрава собралась развенчивать.
 
Но спустя несколько абзацев после подробного описания, как плохо было жить в
СССР, Супрун ловко подводит к своему основному тезису.
 
"Знали ли совєцкие медики, что Манту можно мочить? Конечно, знали. Почему же
они запрещали это детям? Потому что вроде как это помогало заставить их не
тереть место инъекции. Так детей приучали молча исполнять приказы. Не задавать
лишних вопросов. И ни в коем случае не сомневаться в целесообразности
распоряжений "сверху". Подобные запреты и правила существовали в каждой сфере
сововєцкой жизни и сопровождали людей на протяжении всей жизни. Правда
считалась преступлением", – говорит Ульяна Супрун.

\ifcmt
pic https://strana.ua/img/forall/u/0/25/IMG_20201222_155907.jpg
caption Значительная часть книги Супрун посвящена тому, как плохо было жить в СССР. Фото: Страна
\fi

Дальше она проводит параллели с Чернобыльской катастрофой – мол, советские
инженеры и КГБ знали об угрозе аварии, но не предотвратили ее, потому что "эту
Манту тоже нельзя было мочить".
 
И та же "совєцкая" медицинская система досталась в наследство Супрун, когда она
возглавила украинский Минздрав в 2016 году. Ульяна называет это "консервацией
совка в медицине".
 
"И она сохранила самое страшное – глубокое убеждение медиков в том, что
пациенту необязательно знать правду или детали о своем состоянии… Совєцкая
медицина не была создана для людей. Ее клиентом был тоталитарный режим,
которому необходимо было имитировать заботу о людях", – делает вывод Супрун.
 
В целом же, с первых строк книги возникает ощущение, что читаешь не
просветительскую книгу о здоровье экс-главы Минздрава, а мемуары главы
Министерства пропаганды.

\ifcmt
pic https://strana.ua/img/forall/u/0/25/IMG_20201222_153142.jpg
caption Ульяна Супрун издала свои посты из Фейсбука в виде книги. Фото: Страна
\fi

Как раз ради пропаганды "здравого смысла", по выражению самой Супрун, она и
начала вести страницу в Фейсбуке. Она стала "платформой для предотвращения
тупости". А книга Супрун, по ее словам, издана для "всех, кто привык
сопротивляться бессмыслице вокруг".
 
"Мочить Манту – больше не про бунт, это о зрелом критическом мышлении и свободе
действовать креативно. Так что мочите Манту, чтобы мы все были здоровы!" -
призывает Ульяна Супрун.
 
Книга неслучайно называется "Мочи Манту" – именно этой теме был посвящен первый
пост Ульяны Супрун на Фейсбуке, с которого и началась блогерская деятельность
главы МОЗ. И в первой же главе книги цитируется этот пост.
 
Дальше – не менее щепетильная тема: можно ли сидеть на холодном. Тоже пост из
Фейсбука Ульяны Супрун. Можно ли есть мороженое, если болит горло; можно ли
заходить в больницу без бахил, нужно ли прятаться от сквозняков, обрабатывать
ли раны перекисью водорода, работает ли оксолиновая мазь в носу – опять же все
то, о чем Ульяна писала у себя на Фейсбуке. Только с картинками. И на бумаге.

\ifcmt
pic https://strana.ua/img/forall/u/0/25/IMG_20201222_153216.jpg
cpx В книге Супрун ее посты из Фейсбука дополнены комиксами с изображениями мозга - как символа критического мышления. Фото: Страна
\fi

На полях страниц размещены QR-коды – перейдя по ним, можно найти в интернете
ссылки на научные источники, которыми пользуется Супрун. 

В некоторых разделах помимо фейсбучной основы материала есть дополнения,
которые, видимо, в свое время не прошли цензуру на Фейсбук-аккаунте Супрун.
 
Например, в главе про гомеопатию Супрун называет препараты с недоказанной
эффективностью "секси".

\ifcmt
pic https://strana.ua/img/forall/u/0/25/IMG_20201222_154839.jpg
cpx Гомеопатические препараты Супрун называет "секси". Фото: Страна
\fi

Середина книги целиком посвящена развенчиванию мифов о здоровье. А вот под
конец, в третьем разделе "Здравый смысл", снова превращается в пропаганду.
 
На этот раз – не против "совка", а против России.
 
Так, в главе под названием "Мочить Манту можно, а друг друга – нет", Супрун
вдруг начинает рассказывать об информационной войне России с Украиной. Опять
же, какая связь со сквозняками, антибиотиками и вакцинами – непонятно.

\ifcmt
pic https://strana.ua/img/forall/u/0/25/IMG_20201222_165940.jpg
cpx Третий раздел книги Супрун полностью посвящен идеологической борьбе против российского шовинизма. Хотя непонятно, как это связано с мифами о здоровье. Фото: Страна
\fi

Под конец в книге также собраны речи Ульяны Супрун на официальных мероприятиях.
 
Также отдельная глава посвящена украинскому языку, который "долгое время делали
второсортным". Там Супрун открыто поддерживает нашумевший закон "Об обеспечении
украинского языка как государственного", на который вскоре должна перейти вся
сфера услуг в Украине.
 
Отдельно Супрун описывает положительный эффект от декоммунизации. "Нет никаких
объективных причин гордиться тоталитарным прошлым и сохранять к нему уважение",
- утверждает Супрун.
 
И даже объясняет, почему употребляет слово "совєцький" вместо "радянський".
 
Она говорит, что в ее семье власть СССР называли именно так – "совєцька". И
Ульяна использует "власну назву окупаційного режиму совєтів. Без українізованої
версії. Без легітимізації режиму окупантів".
 
Тут же экс-глава Минздрава рассуждает о том, что быть украинцем с СССР
приравнивалось к тому, чтобы быть инопланетянином. "Для многих людей в Украине
я – инопланетянин", – утверждает Супрун.
 
"Есть и другие "пришельцы", рожденные в Украине. Они знают, что страна, которую
мы пытаемся отстроить после долгой совєцкой оккупации, действительно будет
независимой", – пишет Ульяна Супрун.

\ifcmt
pic https://strana.ua/img/forall/u/0/25/IMG_20201222_155642(1).jpg
cpx Ульяна Супрун повторяет в книге свои посты из Фейсбука. Фото: Страна
\fi

Она подробно описывает, как вернулась в Украину в 2013 году в той вышиванке,
которую вышили в Украине ее бабушка еще в 1938 году до того, как эмигрировала в
США. Бабушке Ульяны Супрун сейчас 99 лет.
 
Ульяна Супрун вообще много рассказывает о своей семье и о том, как они сбегали
от власти "совєтов", как ее мама Зеновия Юркив в США возглавляла движения за
женские права. И как важна для Ульяны Супрун украинская идентичность.
 
"Политики, которые хотят, чтобы я сложила руки и перестала работать, стремятся
вернуть старую коррумпированную систему медицинской помощи. Это те, кто хочет и
дальше видеть Украину совєцкой. Те, кто стремятся поделить власть между собой,
чтобы удовлетворить собственные меркантильные интересы. Потому что именно так
поступает совок – распространяет российский шовинизм, превращая свободных людей
в рабов", - пишет Супрун.
 
И после этих строк кажется, что все предыдущие медицинские советы в этой книге
– не более чем ширма для распространения идеологии Супрун и ее видения развития
Украины. Причем темы, на которые она рассуждает, выходят далеко за пределы
полномочий бывшего и.о. главы Минздрава. 
 
