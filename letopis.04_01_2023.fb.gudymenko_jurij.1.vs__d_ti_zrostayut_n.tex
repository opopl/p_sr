%%beginhead 
 
%%file 04_01_2023.fb.gudymenko_jurij.1.vs__d_ti_zrostayut_n
%%parent 04_01_2023
 
%%url https://www.facebook.com/hudymenko/posts/pfbid027qjWqgQZHsRieC7oj1t31wTBzN4gTXQS1SWxDygx1JQDmr9oVxTCJvNWHsQyNVR8l
 
%%author_id gudymenko_jurij
%%date 04_01_2023
 
%%tags 
%%title Всі діти зростають на казках. Всі діти, всіх народів без виключення
 
%%endhead 

\subsection{Всі діти зростають на казках. Всі діти, всіх народів без виключення}
\label{sec:04_01_2023.fb.gudymenko_jurij.1.vs__d_ti_zrostayut_n}

\Purl{https://www.facebook.com/hudymenko/posts/pfbid027qjWqgQZHsRieC7oj1t31wTBzN4gTXQS1SWxDygx1JQDmr9oVxTCJvNWHsQyNVR8l}
\ifcmt
 author_begin
   author_id gudymenko_jurij
 author_end
\fi

Всі діти зростають на казках. Всі діти, всіх народів без виключення.

Деякі казки — чиста вигадка. Більшість мають історичне підґрунтя, як от
легендарний Кирило Кожум’яка, який драконом (окей, окей, змієм) прорив цілком
реальні Змієві вали. 

Але казки є в усіх народів.

І вони закладають моральні норми. Перші, нечіткі, але орієнтири. Захищати
близьких – правильно. Зло – це забирати чуже і приносити біль. І зло завжди
програє.

Але перемога не здобувається легко.

У жодній казці народ жевунів не обслуговував злітні смуги з жовтої цегли для
бомбардувальників злої Гінгеми. У жодного народу немає казок про героїв, які
цуплять унітази з хат. 

Але це є. Ми це бачимо. Огидні колишні діти прийшли на нашу землю, аби вбивати,
красти, приносити біль. А інші, з країни жевунів, мовчки ходять на свою роботу:
обслуговують вбивць в магазинах, розчищають злітні смуги для літаків, приносять
їм каву і бажають вдалого вильоту, коли ті готуються скидати бомби на мою
землю.

Казки вчили їх іншому. Але не навчили.

За півтисячі років будуть нові казки. Про окопи під Бахмутом, які, мабуть,
прорив драконом (окей, окей, змієм) Кирило Кожум'яка. Про день, коли Київ був у
осаді. Про богатирів у камуфляжі, які розірвали між деревами двоголового орла. 

За півтисячі років будуть нові казки. 

У мого народу. 

У жевунів та мертвих не буває казок.
