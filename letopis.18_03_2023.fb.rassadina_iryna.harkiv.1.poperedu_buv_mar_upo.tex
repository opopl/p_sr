%%beginhead 
 
%%file 18_03_2023.fb.rassadina_iryna.harkiv.1.poperedu_buv_mar_upo
%%parent 18_03_2023
 
%%url https://www.facebook.com/ira.rasadina/posts/pfbid02XZpiFuFb4vkHQ2jrdTUhvga4tmPGmJvSSWsiDUg3ke3PBkrwJsd5BKncAdiDXJT5l
 
%%author_id rassadina_iryna.harkiv
%%date 18_03_2023
 
%%tags mariupol,mariupol.war
%%title Попереду був Маріуполь
 
%%endhead 

\subsection{Попереду був Маріуполь}
\label{sec:18_03_2023.fb.rassadina_iryna.harkiv.1.poperedu_buv_mar_upo}

\Purl{https://www.facebook.com/ira.rasadina/posts/pfbid02XZpiFuFb4vkHQ2jrdTUhvga4tmPGmJvSSWsiDUg3ke3PBkrwJsd5BKncAdiDXJT5l}
\ifcmt
 author_begin
   author_id rassadina_iryna.harkiv
 author_end
\fi

Прокинувшись сьогодні, але рік тому, о 4, я дивилась у стелю ще хвилин 40,
прислухаючись. 

Абсолютна тиша ночі у окупованому Бердянську після місяця неперестанних ракет у
Харкові жахала мене чи не більше. Здавалося я розучилася назавжди спати, жити у
тиші, без сирени і рокоту прильотів...

Тіло мобілізувалося. Я відмовилася від кави і сніданку, бо не знала, коли я
зможу сходити у туалет. 

Попереду був Маріуполь. 

У той час наші захисники ще контролювали велику частину міста. 

У Центрі безперервно велись запеклі вуличні бої. 

Юлю шукати я їхала саме в Центрі.

Заповнивши бак до горловини, я трішки розслабилася,  бо зрозуміла, що більше
забрати в мене нічого з машини, все паливо всередині,  а стрілку на приборці
вони не дивляться. Й якщо все піде гаразд, мені хватить перетнути лінію фронту
назад, в Україну.

Попрощавшись з волонтерами, я поїхала.


Їхала максимально швидко, дороги були абсолютно пусті, адреналін лився з вух.

Інколи на узбіччях стояли розпатрані пусті цивільні автівки, з відкритими
дверми, багажниками...

Дорога була сильно побита гусеницями.

На зустріч йшли два хлопці, тягнучи валізку на колесах.  

Через кілометр ще один хлопець. Він йшов без нічого, засунувши руки у кармани,
якось безтурботно. Він дуже нагадав мені мого сина...

Мангуш.

Попереду я побачила великий пост ДПІ у вигляді НЛО тарілки й натовп болотної
форми. 

У скронях зашуміло.

Мене зупинили і затребували документи. 

- Ирина Павловна,  ставим автомобиль вон туда на штрафстоянку и вы свободны

- Тобто?

Він подивитися на мене, як на тупу. 

- Я забираю вашу машину, вы свободны

- Але чому?

- Потому, что ваша машина оформлена не на вас, это машина фирмы, вы частное
лицо и у вас нет доверенности управлять ей 

І тут я зробила помилку.

- Але в Україні не потрібно мати довіреність для управління авто.

Його обличчя перекосилось. 

- Я сказал на стоянку!

Я від'їхала на обочину,  вийшла з автівки

- Я хочу побалакати з вашим начальником.

Він хмикнув і мовчки пішов у будівлю-тарілку. Я роздивлялася людей навколо.

Побачила жінку з трьома дітьми

Дівчинка-підліток років 13, хлопчик рочків 10 і маленька дівчинка у
яскраво-рожевому, їй на вид було роки 4. 

Вони стояли попід забором, у жінки було хмуре обличчя з величезними колами під
очами. Діти мовчали, маленька дівчинка стояла і дивилась в нікуди, її обличчя
було нерухоме, як воскова маска й абсолютно без емоцій. 

Я подзвонила Сергію: "Вони забирають авто." 

Відповідь: "Залишай! Щось придумаємо!"

Всмислі залишай? А як я виїду сьогодні? Мене дитина чекає, друзі, мій Стетік і
Вікінг взагалі-то!

З НЛО вийшли двоє. Той, що мене зупинив, вказав на мене пальцем і пішов далі на
смугу, зупиняти автівки. Другий підійшов до мене. Він був спокійний і
ввічливий.

- Мы конфискуем все машины, которые не принадлежат частным лицам,  ваша машина
на фирму.

І тут я почала плакати. З відтягом так, рясно. В захлин!

Я підвела його до машини, відкрила багажник і двері і почала доставити речі,
показуючи чим наповнена.

- В мене.... там.... сестра з дітьми.. це її автівка... вони були під
Драмтеатром...

ці суки скинули на них бомбу.... все знищено... там стільки дітей прямо зараз
вмирають від холоду.... це одежа для них... будь ласка... мені треба туди...
дивіться, це на діток немовлят... а це на жінок... я бачу, що ви порядна
людина, що ви мене розумієте.... будь ласка, дайте мені поговорити з вашим
начальником 

Він почав все те запихувати назад у автівку. 

- Хорошо, идем, вон он как раз собирается уезжать, быстрее.

У машині сидів боров. 

Щоки цього борова звисали на комірець, а пузо впиралося у кермо. Він перебирав
якісь папери і жував гумку.

- Что она хочет?

- Чтобы мы ее отпустили. 

- Всмысле? Она свободна. У нас указания только  на автомобиль.

І тут я знову почала плакати.

- Я їду не звідти,  а туди. Я везу дітям і жінкам одяг. Тим, що залишились живі
під Драмтеатром після вчорашньої бомби 

Будь ласка, пропустіть мене...

- Туда? Хм... Ну ладно, но знай,  что машину на выезде оставишь тут. Все равно
это один путь оттуда, так что имей ввиду.

- Дякую, дякую!

Ззаду підійшов до нас той перший. 

\ii{18_03_2023.fb.rassadina_iryna.harkiv.1.poperedu_buv_mar_upo.pic.1}

Взяв у руки мої документи,  вдивився у моє ім'я і приблизившись впритул
викрикнув мені у обличчя, наголошуючи на кожному складі

- Слышишь, Павловна! Украины больше не существует! Украина умерла в 2014 году!
По-ня-ла, Пав-лов-на?! ЗА ПОМ НИ! УКРАИНА УМЕРЛА!

Я стояла і мовчки дивилась у його риб'ячі очі, спокійно, беземоційно. 

В голові бриніло і була тільки одна думка "мені ще людей вивозити".

Він якось швидко стух і мовчки віддав мені документи. 

Я бігом пішла до авто, сіла і поїхала далі.

До Маріуполя залишалось кілометрів 25.

Більше блокпостів не було.

Я вирішила їхати не так, як веде головна дорога, а вздовж моря. Тому що на
дорозі почали з'являтися автівки і тому, що попереду я чула гуркіт арти.

Мені здавалося правильним заїхати у місто зі сторони порту. Я підозрювала, що
порт росіяни не руйнують, бо їм ще мародерити і вивозити Азовсталь і інші
підприємства. 

Порт ім потрібен неушкоджений. 

І це спрацювало.

Вздовж моря дійсно було тихо і ціло.

По місту я їхала 30 км на годину, бо прочитала в якомусь пабліку, що якщо
швидко рухатись - це гарантія нарватись на обстріл автівки.

Я заблукала, руки тряслись так, що я випустила телефон, роздивляючись офлайн
карту.

З-за рогу вилетіла стара японська автівка з двома молодими хлопцями.

Вони зупинились поряд мене

-Як звідси виїхати? Вивезіть нас, ми вам заплатимо!

- Я не знаю, хлопці. У вас є автівка, то ви можете взяти ще лю...

- Смотри, вон машина какая-то поехала, нам туда!

Вони газонули, доганяючи ще одну автівку, яка хутко завернула за інший куток,
на крутий підйом.

Я нарешті знайшла на мапі де я, порахувала кількість поворотів наліво-направо і
поїхала. 

Навколо було тихо. 

Коли я піднялась у місто (порт в Маріуполі ніби знизу), я побачила...

Дерева, посічені і ніби зрізані чимось великим валялися на узбіччі, тролейбусні
кабелі великими клубками або здоровими петлями звисали і стелились по шляху,
ризикуючи намотатися мені у колеса.

Відкрила вікна, бо хотіла Чути.

Дуже смерділо.

Інколи смерділо так, що виїдало очі.

Я побачила згорівший тролейбус, а біля нього машину пожежної допомоги. Теж
згорівшу. 

За кермом пожежки сиділа людина, вірніше те що від неї залишилось.

В тролейбусі теж було щось, але я не могла дивитись, мене нудило.

Завернувши за ріг, проїхала кілька кварталів і побачила кількох похилих жінок,
вони стояли на кутку будинку.

Я запитала де тут поворот на вулицю (назвала).

Але вони всі не відреагували, ніби я привид. Вони дивились на мене, але не
бачили мене, вони дивились КРІЗЬ. 

Вони ніби застрягли в текстурах. 

Як та маленька чотирирічна дівчинка у Мангуші. 

Я завернула, як мені здалося, у потрібний поворот,  а потім у ще один і почала
дивитися навколо.

Ще увечері я роздивилася на Гуглмепс фото цієї вулиці. Я бачила кольори
будинків і впізнавала їх з тих фото у спаплюжених фасадах з вибитими
вікнами-очами. 

Ось, здається саме тут! 

Я запаркувалася серед битого  скла.

Так, десь тут мають бути ковані двері у двір з ручкою, як у купе поїзда.

Бляяяя! Ось вона, ручка! 

Я тут!

Я забігла у двір п'ятиповерхівки і знову загубилася. 

Де шукати Юлю?!

І я почала кричати.

Я викрикувала її ім'я знову і знову, в мене знову почало бриніти у вухах і
плигали зайчики у очах.

- Юля! Юля! Юля!

Вона швидко вибігла з-за куща, за іншим заборчиком, який я чомусь не помітила
спочатку, мабуть через нерви.

- Я Юля! А ви хто?

- Я Іра, ти мене не пам'ятаєш? Я приїхала за тобою! 

- Ти.... ти своя?

- Так, я своя! Подивись, в мене для тебе є ось дещо! 

Я трясучимися руками відкрила переписку і включила голос її доньки, яка благала
маму послухати мене і робити те, що скажу. А потім голос чоловіка, який казав
те ж саме... 

І тут десь зовсім поряд прилетів Град.

Я добре знаю цей й@баний звук, я з Харкова. 

Юля викрикнула, щоб я бігла за нею.

Я послизнулась, впала навколішки і поповзла, рятуючись від гуркоту навколо.

В голові було одне "машині пизд€ць машині пизд€ць машині пизд€ць"

Я піднялась, забігла у якісь двері і спустилася вниз, а потім вверх. 

Юля звала свого батька.

Вона застрягла у Маріуполі на початку повномасштабної війни, тому що приїхала з
Харкова піклуватися за захворівшим батьком. Це був відомий інженер-конструктор,
який колись винайшов і спроектував  унікальну бурову машину для добичі,
здається металевої руди. Ці машини й досі працюють у рудниках.

Він спустився з другого поверху до нас і сказав, що нікуди не поїде..

Що він залишається. 

Я включила аудіо й для нього. Бо там і для нього були слова. 

Він прослухав, але відмовився. 

Юля його обійняла, дістала з-під стола свій рюкзак і сказала,  що вона готова.

Зовні затихло.

Я вибігла на вулицю, готова побачити знищене авто.

Але, блд, автівка була ціла!!

Яке щастя!!!

Я підбігла, почала викидувати на землю одяг з переднього сидіння, щоб Юля могла
сісти.

Я відчувала жорсткий сором і докори сумління, що цей одяг не віддала комусь, а
просто викинула на дорогу...

Ми сіли, я завела мотор.

На зустріч, по середині дороги йшли кілька молодих людей, маргінальної
зовнішності.

Один з них розкинув руки в сторони і посміхаючись щось кричав нам.

Я розвернулася так, як вмію ще з автоспорту (дякую, сенсей Пітон за вміння!) й
полишивши позаду цих чуваків, чкурнула назад по вулиці.

Мені потрібно було зупинитись і видихнути адреналін.

Тому я знову направилася в тихий порт, я вже запам'ятала шлях.

Юля заспокоювала мене

- Видихни,  спокійно вдихни, знову видихни...

Запаркувавшись, я дістала з карману всі папірці з адресами людей і відкрила
мапу.

Коли я ще вперше їхала у центр, я побачила, що дорога, що вела у сторону, де
жив молодий лікар, його мама і лежача бабуся, була знищена здоровенною
воронкою. Будинки зруйновані ніби катком,  груди обломків.  Обхідний шлях лежав
би через Драмтеатр. Саме там, де тільки що гримів Град.

Картаючи себе, я відклала цю адресу.

БЛ@ДЬ!!! Де адреса мами з трьома дітками?! Де цей й@баний папірчик?!!!

Я вискочила з автівки і почала у лихоманці ритися під сидінням і по карманам. Я
витрушувала все знову і знову, знову і знову... 

Загубила! От пи\$да, я загубила адресу!!!

Розумієте... 

Усвідомлення того, що я тоді, своїми руками, можливо вбила цих всіх людей, мене
жере вже рік!

У мене залишився останній листочок з адресою сім'ї з 4х чоловік і два місця у
автівці. 

Юля сказала, що знає де це. І це недалеко.

Ми поїхали.

Дорога була засипана мусором, відколотими кусками цегли, відірваними кабелями. 

Були згорівші автівки...

Коли ми заїхали у двір звичайної п'ятиповерхівки, я побачила людей, що готують
їжу на вогнищі біля будівель.

Я почала казати їм, що сьогодні у місто виїхало багато волонтерів, що вони
прямують по оцій і оцій вулиці, збираючи людей на евакуацію (я знала план тих
волонтерів, з ким виїхали з Запоріжжя)

Але люди не реагували.

Юля побігла, здається, у другий з правої сторони під'їзд, бо мала у руці
папірець з іменами і точним описом.

Я почала діставати одяг з автівки і складати його на лавку поряд. 

Почали стікатися люди, які рилися і щось вибирали.

І тут я знайшла другий пакет з їжею,  який в мене не віджали росіяни вчора! 

І пакет з водою, ще зі Швеції (я у останню мить закинула цей здоровенний пакет
у авто, коли їхала сюди). Одна жінка з божевільними очима почала хватати ту
воду і консерви та запихувати собі за пазуху, наповнюючи кофту...

До мене підійшла ще одна жінка у ковдрі. Вона питала, чи може поїхати з нами.
Вона плакала і збивчо пояснювала, що приїхала провідати маму, а в їх дім попала
ракета, і вона вибралась, і зараз у підвалі з чужими абсолютно людьми, нікого
не знає тут взагалі...

Але в мене усього два місця!

Вона продовжувала стояти поряд  і плакати.

З'явилася Юля. 

І уся та сім'я, я запамятала тільки ім'я жінки - Альона.

Я подивилась на всіх них і зрозуміла, що якщо дуууууже потіснитися, взяти
підлітків на коліна, влізуть усі п'ятеро!

\ii{18_03_2023.fb.rassadina_iryna.harkiv.1.poperedu_buv_mar_upo.pic.2}

Навіть та жінка у ковдрі, здається її звали Люда.

І я сказала

- Їдуть всі.

Ми швидко закидали рюкзаки у пустий багажник і поїхали. Всі ці люди, за ким я
приїхала, мали зібрані рюкзаки,  миттєво готові, хоча вони не знали, що я
їду... Вони весь цей час жили надією.

Я вирішила виїжджати вздовж моря знову.

І спробувати їхати весь час вздовж моря, скільки зможу, щоб оминути той
блокпост і не втратити автівку.

Біля невеличкого блокпоста на виїзді з міста, в мене знову почали трястися руки
і колінки, я розуміла,  що ми можемо залишились прямо тут без коліс.

Вздовж дороги йшли люди, що пішки покидали місто, багато людей, нескінченна
верениця людей...

Десь через годину у заторі,  ми під'їхали на блокпост.

\ii{18_03_2023.fb.rassadina_iryna.harkiv.1.poperedu_buv_mar_upo.pic.3}

І о чудо, ці болотяні чоловіки не роздягали наших хлопців і не придралися до
документів! 

Яка вдача!!! 

Нас випустили!!!

Всіх, навіть чоловіків!!!

Ми відділилися від маси, поїхавши по узбережжю, а всі люди попрямували на
Мангуш, до НЛО

Появився зв'язок. 

Люда плакала і безперестанку комусь дзвонила, не замовкаючи. І тут в мене
задзвонив телефон. 

Сергій, чоловік Юлі. Я дала їй зняти слухавку. Здається він там загубив дар
мови...

І тут навіть та гімняна дорога, що в мене була - скінчилася.

Всьо, немає.

Тільки у поле. 

Ми повернули у поля, на свіже рілля і всходи пшениці, здається.

Позаду точилися якісь розмови, діти щебетали. Я знайшла кілька пачечок горішків
і яблука. Люда сказала, що не їла два дні.

Позаду мене привязалася якась автівка, потім вона зупинилась, ярісно мигаючи
мені світлом. І почала здавати назад.

І тут я побачила міни.

В грудях захололо, руки спітніли і вчепилися у кермо. 

Здається ніхто з машини не помітив.

Я гримнула, щоб сиділи тихіше, бо гомін мене звинчував, а я й так була на
кінчику адреналінової голки.

Десь кілометрів через 7-10 я побачила польове перехрестя по якому доволі жваво
їхали якісь автівки і я зрозуміла, що ми вибралися.

Мені прижало  в туалет.

Ми зупинилися на більш-менш втоптаному узбіччі і я строго наказала не відходити
більш, ніж на метр від автівки,  бо хз які ще сюрпризи могли б  бути у кущах.

З'явився інтернет і я дивлячись на мапу зрозуміла, що ми оминули Мангуш і
можемо спокійно повернутися на велику дорогу.

У якомусь маленькому селі працював магазинчик, де Люда купила собі плавлений
сирок і ще щось.

Я подзвонила волонтерам, чию сім'ю вивезла й сказала, де і коли ми можемо
перетнутися. Й взяла обіцянку, що вони також відвезуть Люду у дім відпочинку у
Бердянську,  де її чекала подруга.

Це фото, де усі, кого я вивезла з Маріку...

Я мала швидко їхати, бо хотіла як найскоріше перетнути лінію фронту, боялася,
що форточка схлопнеться.

У одному селі ми зупинилися поряд магазину, мені потрібно було дещо найти на
мапі. Саме в цей час в магазин привезли хліб. Знаєте, такі дерев'яні палєти з
викладеними рівненько свіженькими буханочками. У Юлі розширилися очі.

\ii{18_03_2023.fb.rassadina_iryna.harkiv.1.poperedu_buv_mar_upo.pic.4}

- Я маю взяти цей хліб!

Хлібини, цікавої трикутної форми, звичайного сірого хлібу.

 Смакота!

Коли я чекала Юлю з магазину, я побачила знайомця. Вздовж дороги чимчикував той
самий безпечний хлопчина з Маріку, що нагадав мені сина вранці. Як він добрався
сюди, невже пішки?

Ми тронулися і я зупинилася біля нього, бо бачила, що він голосував іншим
автівкам.

Я вирішила вивезти ще і його з окупації.

Цей хлопчик дзвонив мамі, вони були з Краматорську. Він був студентом у
Маріуполі.

Його мама плакала у трубку і просила дати мою картку, щоб заплатити.

- Та що ти плачеш, мам, та все ок, мене в Запоріжжі чекають друзі, ща виїду і
всьо пучком!

Вертатись назад я вирішила не тою дорогою, що заїжджала, бо прочитала у
пабліках, що там під Горіховим бої саме зараз. 

У Василівці був розбомблений здоровенний міст через урвище. Й ми звернули
направо. ХЗ чому туди, а не вліво, мені здалося, що там накатаніше дорога
візуально. 

І я знову не помилилася.

Сільська доріжка вела петляючи вздовж урвища, але я не бачила куди звернути,
щоб проїхати по дну на ту сторону.

За мною якось вистроїлася колона машин так 30, що як мурашки повзли, наповнені
людьми, дітьми, що лишали свої життя тут і зараз.

Їб@ть, чуваки, чого прилипли, камон! Я сама не знаю куди я їду!

І тут я побачила хлопчика років 10 на велосипеді,  місцевого. І запитала чи знає він як дістатися на ту сторону

- Звичайно!

Тільки там міни, я вам покажу як проїхати, я сто раз там вже їздив!

І вся наша колона поповзла за маленьким провідником.

Коли той хлопчина повертався назад, з кожної автівки йому за пазуху сипалися
яблука, цукерки, печиво.

Він виглядав дуже задоволеним і я, здається,  зрозуміла чому саме він сто разів
вже там їздив і що це не перебільшення

На дні урвища я побачила маленькі прапорці, які позначали міни

Українські прапорці, вштрикнуті у м'яку, жирну землю...

І я почала плакати.

І, здається,  Юля також...

Наші воїни посміхалися нам і мені здавалося, що вони найпрекрасніші чоловіки,
кого я бачила у своєму житті!

Передавши нашого хлопчину волонтерам у Запоріжжі,  щоб його доставили по місцю,
ми рванули на Дніпро.

Де мене чекали син,  мій дорогоцінний доберман, друзі і Вікінг...

Попереду лежав ще дуже довгий і важкий шлях.

%\ii{18_03_2023.fb.rassadina_iryna.harkiv.1.poperedu_buv_mar_upo.cmt}
