% vim: keymap=russian-jcukenwin
%%beginhead 
 
%%file 11_11_2020.yz.pozin_aleksandr.1.peremoga_ne_pobeda.cmt
%%parent 11_11_2020.yz.pozin_aleksandr.1.peremoga_ne_pobeda
 
%%url 
 
%%author_id 
%%date 
 
%%tags 
%%title 
 
%%endhead 
\subsubsection{Коментарі}

\begin{itemize} % {
\iusr{Ольга С.}

Мне понравилась статья, особенно семантическое сравнение слов Победа и
Перемога. Слово \enquote{Победа} действительно имеет сакральный смысл, исторически мы в
этом убеждались и не раз. Я не большой специалист в лингвистике, мне было
просто очень интересно это прочитать, кое-над чем задуматься. Спасибо.

\iusr{Ярослав Позин}

Интересные мысли, но очень субъективный взгляд, я считаю. Любительская
лингвистика очень много куда завести может. И получается что-то типа упражнений
по языкознанию Носовского-Фоменко или Задорновские передачи

\iusr{Михаил}

А еще в украинском языке есть слово покращення. Покращення - это маленькие
перемоги, но бояться их нужно не меньше.

\iusr{zvet neba}

Хорошо разобрали. Но обычный человек так глубоко не копает. Просто первый раз
услышанное слово вызывает какие-то смутные ощущения, которые потом или
пропадают, или усиливаются.

Слово ПЕРЕМОГА на русский слух интуитивно звучит вообще не о победе и даже не о
борьбе, а о тяжести преодоления болезни, тяжёлого состояния, вызванного
усталостью или депрессией:

- ну как ты, выздоравливаешь?

- перемогаюсь потихоньку, вроде не хуже...

А мягкое украинское Г, почти Х - ещё больше усиливают это ощущение. Типа того,
что человек уже хрипит, задыхаясь.

\iusr{Леонид Бахарев}

Я патриот России, но данный пост форменное славоблудие.
\iusr{Роман Мурашко}

Если верно утверждение, что в языке каждого народа есть слова, которые
характеризуют мироощущение и характер нации, то \enquote{ЗРАДА} безусловно абсолютно
точно характеризует украинствующих идиотов!

\iusr{Валерий Борисов}

А усмихнэться явно не улыбнётся. Вот с ехидной усмешкой на хохоликов всё и
смотрят.

\end{itemize} % }
