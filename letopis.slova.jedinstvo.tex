% vim: keymap=russian-jcukenwin
%%beginhead 
 
%%file slova.jedinstvo
%%parent slova
 
%%url 
 
%%author 
%%author_id 
%%author_url 
 
%%tags 
%%title 
 
%%endhead 
\chapter{Единство}
\label{sec:slova.jedinstvo}


%%%cit
%%%cit_head
%%%cit_pic
%%%cit_text
К Крыму у коллективного Запада есть какой-то интерес. К Донбассу - никакого.
На прошлой неделе в прессе опять появились сообщения-повторения о заявлениях
про \enquote{стену с Донбассом}. Во-первых, это интересно тем что опять никакой
\enquote{\emph{единой страны}} не обнаружено.  Во-вторых, вспомним что Сивохо по
результатам опроса на Донбассе не стал пи-пи-пи про \enquote{\emph{единую страну}}, и
как раз за это его выпинали -- но \enquote{\emph{единой страны}} всё равно не
обнаружен
%%%cit_comment
%%%cit_title
\citTitle{Никакой единой страны в Украине не обнаружено / Лента соцсетей / Страна}, 
Роман Подолян, strana.ua, 03.07.2021
%%%endcit

%%%cit
%%%cit_head
%%%cit_pic
\ifcmt
  tab_begin cols=3
     pic https://press-center.news/wp-content/uploads/2018/11/foto11.jpg
     pic https://nikopol.informator.ua/wp-content/uploads/2019/11/diktant-2.jpg
		 pic https://ministryofcounterculture.org/media/images/39dab76abdde11e5b30d9c8e99086854-01.jpeg
  tab_end
\fi
%%%cit_text
Отакі вимоги, і це все – про Радіодиктант (саме так, з великої Р) національної
єдності. 9 листопада, себто найближчого вівторка, о десятій ранку, з Києва.
Цього року удвадцятьперше. (Ні, ви тільки погляньте на це дивовижне слово –
удвадцятьперше!) Тобто це вже національна традиція: щороку 9 листопада українці
пишуть диктант, який у прямому етері надає для них їхнє Українське радіо. Коли
2000 року вони робили це не удвадцять-, а просто вперше, то слова "флешмоб"
вони ще переважно не знали. Тим не менше, радіодиктант став одним із
наймасовіших і не одним із, а таки точно найдовговічнішим українським
флешмобом. Протягом швидкоплинних 20-30 хвилин \emph{національна єдність}
перестає бути порожнечею, ілюзією, фантомом – і збувається. Причому в
найпростіший та найприродніший спосіб – як \emph{єдність} у мові.  Радіодиктант
– це перевірка на мислення, пам'ять, уяву, інтелект, начитаність.  Якщо ви
хочете їх розвивати й демонструвати, то непогано для початку зрозуміти, на
якому вони у вас рівні. І граматична адекватність – один із чітких цього рівня
показників.  Але є й міркування трохи вищого ґатунку. Йдеться-бо про захід,
який називається не просто радіодиктантом, а Радіодиктантом національної
\emph{єдності}. Яка – і з цим хто б не погодився – є незаперечною цінністю. Її
нам, щиро кажучи, дуже бракує.  А національна мова для її досягнення є чимось
фундаментальним. Бо вона й передумова, і запорука, і все-все-все, тобто основа
основ. Не просто собі державна, а – підвищуймо статус! – національна, тобто
щось таке, що існує вже на рівні і мертвих, і живих, і ненарожденних. Немає
\emph{національної єдності} без національної мови. Радіодиктант об'єднує нас,
таких різних, навколо мови як цінності, й бодай на пів години робить уважними,
вдумливими, солідарними.  Тож долучайтеся до нації
%%%cit_comment
%%%cit_title
\citTitle{Радіодиктант об'єднує нас навколо мови як цінності. Долучайтеся до нації}, 
Юрій Андрухович, gazeta.ua, 05.11.2021
%%%endcit

%%%cit
%%%cit_head
%%%cit_pic
\ifcmt
  tab_begin cols=3

     pic https://avatars.mds.yandex.net/get-zen_doc/1360848/pub_5bdc178e40910900aab9c609_5bdc5029b327ef00a937361d/scale_1200
		 @caption Эрнест Лисснер. «Изгнание польских интервентов из Московского Кремля в 1612 году»

     pic https://avatars.mds.yandex.net/get-zen_doc/245342/pub_5bdc178e40910900aab9c609_5bdc5188228e4f00aaa623e5/scale_1200
		 @caption Народное ополчение 1611 года

		 pic https://pbs.twimg.com/media/Dq_j3CRWsAALa9O.jpg:large
		 @caption В День народного единства в Оренбуржье пройдет фестиваль национальных культур, zen.yandex.ru, 2019

  tab_end
\fi
%%%cit_text
В первых числах ноября 1612 года Китай-город был взят приступом. У осажденных
был катастрофический голод, поляки стали выгонять из Кремля боярские семьи и
обслугу. На два предложения о добровольной сдаче и сохранении жизней поляки
ответили высокомерным отказом. Но позднее командование гарнизона подписало
капитуляцию и выпустило всех бояр. 7 ноября остатки гарнизона в Кремле сдались
ополчению Минина и Пожарского. Москва снова стала русской. Князь Пожарский
вступил в освобожденную Москву с Казанской иконой Божией матери. А в феврале
1613 года на Земском Соборе был выбран первый русский царь из династии
Романовых - Михаил Фёдорович. Позже, в 1649 году, царь Алексей Михайлович
распорядился отмечать день Казанской иконы Божией матери не только летом, но и
22 октября по старому стилю, в память от избавлении Москвы от польских
интервентов. Тогда же у него родился первый сын, Дмитрий Алексеевич. С 1918
года, когда Россия (РСФСР) перешла на григорианский календарь, этой дате стало
соответствовать 4 ноября. С 2005 года 4 ноября стал российским государственным
праздником, Днём \emph{народного единства}.  Что же касается Лжедмитриев,
спасшихся царевичей, то в 1611-12 гг. их было еще два (третий и четвертый), но
особых успехов в попытках занять русский престол они не достигли
%%%cit_comment
%%%cit_title
\citTitle{День народного единства. Что вообще мы празднуем 4 ноября? Немного ликбеза}, 
ПРОСВЕТИТЕЛЬ, zen.yandex.ru, 04.11.2021
%%%endcit
