% vim: keymap=russian-jcukenwin
%%beginhead 
 
%%file 15_11_2021.fb.holmogorov_egor.1.pushkin_i_dostojevskij
%%parent 15_11_2021
 
%%url https://www.facebook.com/holmogorov.egor/posts/10227944951074658
 
%%author_id holmogorov_egor
%%date 
 
%%tags dostojevskii_fedor,kultura,literatura,pushkin_aleksandr,rossia,rusmir
%%title Пушкин и Достоевский
 
%%endhead 
 
\subsection{Пушкин и Достоевский}
\label{sec:15_11_2021.fb.holmogorov_egor.1.pushkin_i_dostojevskij}
 
\Purl{https://www.facebook.com/holmogorov.egor/posts/10227944951074658}
\ifcmt
 author_begin
   author_id holmogorov_egor
 author_end
\fi

Под самый конец жизни Федор Михайлович узнал, что такое настоящая грандиозная
слава и всеобщее признание. К этой славе привел его с детства любимый Пушкин.

В июне 1880 года в Москве был открыт сооруженный на деньги общественности
памятник Пушкину. Он стоял тогда на другой  стороне площади, напротив
возвышавшегося красавца  Страстного монастыря, разрушенного в 1937 году, после
чего в 1950-м Пушкина перевезли на место образовавшейся духовной пустоты.

Пушкинский праздник грозил стать местом столкновения идеологических партий.
Либералы собирались в своих выступлениях восславить поэта как певца свободы,
вольнодумства и космополитизма. Патриотическое крыло организаторов празднества
решило противопоставить им Достоевского.

Федор Михайлович отнесся к задаче со всей серьезностью – противопоставить
либерально-общечеловеческой карикатуре на Пушкина – настоящего русского поэта,
способного к всечеловеческой мысли и всемирной отзывчивости.

\ifcmt
  tab_begin cols=2

     pic https://scontent-frt3-2.xx.fbcdn.net/v/t39.30808-6/255891273_10227944947394566_7758965874266556030_n.jpg?_nc_cat=101&ccb=1-5&_nc_sid=730e14&_nc_ohc=-Py6hkJDKlkAX_Xpxsd&_nc_ht=scontent-frt3-2.xx&oh=d65390e610614a72e26af0f186c4304f&oe=61995E0A

     pic https://scontent-frx5-2.xx.fbcdn.net/v/t39.30808-6/255625685_10227944946354540_5841766174092836867_n.jpg?_nc_cat=109&ccb=1-5&_nc_sid=730e14&_nc_ohc=18lpffqGjEgAX_VkU08&tn=lCYVFeHcTIAFcAzi&_nc_ht=scontent-frx5-2.xx&oh=ae9309794665329fc6c92dc581b46dbf&oe=619A06BD

  tab_end
\fi

Пушкин, по мнению Достоевского, указал на главную проблему русского
образованного общества, приобретенную в послепетровскую эпоху,  – оторванность
от народа, оторванность от корней, бессильную гордость. Этой болезнью больны и
Алеко в «Цыганах» и Евгений Онегин. Сам Достоевский продолжил галерею этих
одержимых гордыней лишних людей – Ставрогин, Дмитрий и Иван Карамазовы. 

«Почему наш европейский либерал так часто враг народа русского? Почему в Европе
называющие себя демократами всегда стоят за народ, а наш демократ… служит в
руку всему тому, что подавляет народную силу?» - язвительно замечал
Достоевский, полемизируя в «Дневнике писателя» с критиками своей речи.

«Смирись гордый человек и прежде всего смири свою гордость. Смирись праздный
человек и, прежде всего, потрудись на родной ниве» - предлагает Достоевский в
своей Пушкинской речи  единственно возможное решение вопроса – обращение из
западнической гордыни к русской народности.

Пример такого обращения дал сам Пушкин, нарисовавший идеальные русские типы из
которых особое внимание он уделяет  Татьяне Лариной. Татьяна русская душею
постигает пародийность и пустоту Онегина и осуществляющяя в себе нечто высшее,
чем романтическая похоть – нравственный закон. Именно в этом – сознании
нравственного закона - причина её отказа быть с Онегиным.

«Повсюду у Пушкина слышится вера в русский характер, вера в его духовную мощь,
а коль вера, стало быть и надежда, великая надежда на русского человека».

Пушкин для Достоевского – идеальный представитель русского народного духа.
Включая и ту её черту, которую писатель называет «всемирной отзывчивостью»,
способностью понимать и постигать другие народы и их идеи, улавливать
английский, французский, испанский. Однако было бы ошибкой представлять
всемирную отзывчивость по Пушкину и Достоевскому как огромный умственный бак, в
который сливается культурное варево, а то и помои со всего мира.

\ifcmt
  tab_begin cols=2

     pic https://scontent-frt3-1.xx.fbcdn.net/v/t39.30808-6/255701262_10227944946674548_8393052971133924617_n.jpg?_nc_cat=106&ccb=1-5&_nc_sid=730e14&_nc_ohc=riW1pgEFCRQAX9kkFsJ&_nc_ht=scontent-frt3-1.xx&oh=dd3eff8e5972ce99b608872694f0da40&oe=619A3F41

     pic https://scontent-frx5-1.xx.fbcdn.net/v/t39.30808-6/255689055_10227944946554545_7852864383346925066_n.jpg?_nc_cat=110&ccb=1-5&_nc_sid=730e14&_nc_ohc=0g1ka5EhdawAX-L4HJK&_nc_ht=scontent-frx5-1.xx&oh=a9542829cbc14fcfea599bcc24d4bb09&oe=619AA0F6

  tab_end
\fi

Напротив, у Достоевского идея всемирной отзывчивости носит активный, порой
почти завоевательный характер. Французский дипломат Мельхиор де Вогюэ с
раздражением описывал в дневнике свой спор с Достоевским и чеканную формулу
русского писателя и мыслителя: «Мы обладаем гением всех народов, и сверх того
русским гением. Вот почему мы в состоянии понять вас, а вы нас — нет». 

Достоевский пытается примирить русских западников и славянофилов в следующей
идее. Усвоение западной культуры стало для русских частью осуществления
собственной национальной идеи, поскольку в основе этой идеи устремленность «ко
всеобщему общечеловеческому воссоединению со всеми племенами великого арийского
рода… Для настоящего русского Европа и удел всего арийского племени так же
дороги, как и сама Россия, как и удел своей родной земли, потому что наш удел и
есть всемирность».

Россия призвана решить противоречия Европы, внести примирение в Европу – такова
мечта Достоевского, которую «либеральная полиция», - как выразился Федор
Михайлович, немедленно бросилась разоблачать. Но это было после, одумавшись, а
само произнесение речи закончилось всеобщим триумфом.

«Когда я закончил – я не скажу тебе про рев, про вопли восторга: люди
незнакомые между публикой плакали, рыдали, обнимали друг друга и клялись друг
другу быть лучшими, не ненавидеть впредь друг друга, а любить» - писал
Достоевский жене.

«Вы гений, вы больше чем гений», «Вы наш святой, вы наш пророк» - кричали
Достоевскому. Рукоплескания продолжались полчаса. Даже западник Тургенев, и тот
обнимает своего старого врага. 

Не понимаемый, бедствующий и мучимый почти всю свою жизнь русский пророк уходил
к Богу в лучах славы.

Больше здесь: \url{https://www.youtube.com/watch?v=xlnrJU09Jso}
