% vim: keymap=russian-jcukenwin
%%beginhead 
 
%%file 26_12_2021.fb.fb_group.story_kiev_ua.3.orkestr_benni_gudman_grigorij_chapkis_1962.cmt
%%parent 26_12_2021.fb.fb_group.story_kiev_ua.3.orkestr_benni_gudman_grigorij_chapkis_1962
 
%%url 
 
%%author_id 
%%date 
 
%%tags 
%%title 
 
%%endhead 
\zzSecCmt

\begin{itemize} % {
\iusr{Igor Popell}

А еще Гудман посетил студию звукозаписи и принял участие в репетици Киевского
симфонического оркестра УР и ТВ. Он очень впечатлился мастерством музыкантов. В
частности, флейтиста Дмитрия Федькина. Гудман подарил ему флейту Selmer. И
после этого для Федькина начался настоящий концерт. Его затаскали на допросы и
комиссии... Флейта, кстати, испарилась. Видимо, как важное вещественное
доказательство. Знаю об этом от своего отца, который долго работал в
Госоркестре и дружил с Федькиным. Многие старые музыканты знали эту историю. На
фото Д. Федькин за пару лет до свалившегося на него подарка Гудмана )))

Кстати, есть неплохой док. фильм о гастролях Бенни Гудмана в СССР. Ссылки здесь
запрещены, но можно найти на Youtube \enquote{Benny Goodman in USSR}.

\ifcmt
  ig https://scontent-frx5-1.xx.fbcdn.net/v/t39.30808-6/269794346_10159674199401136_1432669746507291945_n.jpg?_nc_cat=110&ccb=1-5&_nc_sid=dbeb18&_nc_ohc=yRdunhAiaqAAX9Erfcp&_nc_ht=scontent-frx5-1.xx&oh=00_AT_lhCqx-gQJsv1l1UOztFWghsv8fQ00ymTC1zB2uaAUoA&oe=61CDBEBD
  @width 0.4
\fi

\iusr{Irena Visochan}

А я могу подтвердить все о чем написал Игорь. Наши папочки служили в Симфонияе
ком оркестре Украинского Радио и телевидения. Они же участвовали в записи этого
знаменитого концерта. Дирижер оркестра был Константин Симеонов.(светлая память)О
том, что Б. Гудман сделал такой царский подарок Д. Федькину знали все
музыканты, @igg{fbicon.musical.score} Но КГБ не дремало...


\iusr{Захар Давыдов}

Я был на этом концерте. Ажиотаж был такой, что снесли одни из дверей Дворца и
многие попали в него без билетов тоже. Дружинники отнимали портативные
магнитофоны \enquote{Весна} у предприимчивых меломанов. Успех был потрясающий.

\begin{itemize} % {
\iusr{Михаил Рабинович}
\textbf{Zahar Davydov} я жил рядом с Дворцом на концерт не попал а разбитые оскоки стекляных дверей видел. Их , от большого ума, не открыли и держали запертыми. Толпа напирала и 5мм каленное стекло рассыпалось на крошки. Я пытался проскочить в другом месте и эти крошки увидел когда всё было кончено.
\end{itemize} % }

\iusr{Maria Rasin}
Я была на этом концерте во дворце спорта и помню этот ажиотаж и значки помню !!!

\iusr{Gary Sorokin}
Сегодня ты играешь джаз, а завтра родину продашь

\iusr{Vladimir Tkachev}
Я был на этом концерте!))) Киев был городом джаз фанов. С артистического входа выхода Дворца Спорта музыкантов и Бенни Гудмана встречала интеллигентная, просвещенная толпа киевлян фанатов любителей музыки джаза)))

\iusr{Ольга Боровских}
Вот это история! Спасибо, что поделились!

\iusr{Владимир Бахтурин}
Спасибо огромное за интересную историю!)))

\ifcmt
  ig https://scontent-frt3-1.xx.fbcdn.net/v/t39.1997-6/s180x540/248288601_4356211641163137_1585777796013288602_n.png?_nc_cat=108&ccb=1-5&_nc_sid=ac3552&_nc_ohc=N2sLblkAOQ4AX8SVO6F&_nc_ht=scontent-frt3-1.xx&oh=00_AT9CU0WaH-ikEcaONe_X877sqseN6HTf9crwx6F7gnEIPw&oe=61CECA66
  @width 0.1
\fi

\iusr{Ирина Иванченко}
Отличный пост, спасибо.

\iusr{Leonid Dukhovny}

Увы, мне тогда не удалось попасть во Дворец Спорта. Я надеялся на случай, как
это было с концертом Дюка Элингтона ! Но, увы, пролетел, о чём по сей день
очень сожалею !

\begin{itemize} % {
\iusr{Irena Visochan}
\textbf{Leonid Dukhovny} Я с папой была на концерте Дюка Элингтона. @igg{fbicon.musical.notes} 

\emph{Leonid Dukhovny}
\textbf{Irena Visochan} 

Вот это здорово! Жаль, что не встретил вас там.

Вот более подробные воспоминания о том концерте.

Я хорошо помню концерт Дюка Элинктона в старом Дворце Спорта, вмещающем 12000
зрителей. Как водилось тогда билеты на этот концерт не продавались, а
распределялись в строгой отчётности по заводам, фабрикам, крупным
госучереждениям...Естественно, билета у меня не было, но за час до начала я
стоял перед Дворцом с протянутой рукой и тривиальным вопросом на губах \enquote{Нет ли
у вас лишнего билетика?!} Бесполезно, может у кого-то и был, но отчётность! Я
отчаялся - оставалось 5 минут до начала, а моя замёрзшая рука так и оставалась
пустой... \enquote{Лёня, что ты тут делаешь?} - раздался за спиной голос. Я обернулся.
Это был знакомый волтарнист из нашей Держоперы! И далее фантастическое: \enquote{Идём,
проведу}!!!!... Он по какому-то пропуску завёл в здание и провёл в последний
ряд на самую верхотуру...В зале свет был уже потушен. Сцена тоже едва
освещалась... И в этой полутьме стали выходить на сцену, по-одному, музыканты,
садились и зажигали свечу, на своём пульте... За роялем никого не было... Тихо и
как-то протяжно запели саксофоны, заработала \enquote{секунда}, в степенном ритме
заиграла остальная медь...Что-то уж бесконечно знакомое...

Господи, это же знаменитый \enquote{Караван}! И когда вступил рояль, дали полный свет -
сам Маэстро играл своё произведение!... В темноте я не видел своего соседа
слева. Силуэт, говорил об очень высоком росте его обладателя. Всё хорошо,
но \enquote{силуэт} самозабвенно отбивал ритм своей правой ногой, при том не замечая,
что \enquote{отбивание} происходит на моей, на левой ступне... Пустяки, главное, вот он
легендарный Дюк... Закончилось первое отделение. Дали свет. И вдруг мой
верзила-сосед произносит: \enquote{Леня, это ты?} Господи, это ж Юра Майборода,
племянник знаменитых композиторов, мы с ним играли в диксиленде, в клубе РАБИС.
Между прочим, классный джазмен, пианист!... Мы вышли в фоей и в толпе были видны
такие же, как у нас с Юрой сомбулические лица .... И вот, что примечательно.
Когда мы вернулись на нашу верхотуру и посмотрели в партер - он был на три
четверти пуст !!! Что вы хотите - билеты-то распределялись -социализм!...

Запомнился финал концерта... Постепенно каждый музыкант тушил свечку на своём
пюпитре и уходил. \enquote{Секунда} тоже потушила свечи и ушла... Свеча горела только
на рояле Маэстро! Он, как-бы, давал сольник своих мелодичных пьес... Неожиданно
он поднялся, и, затушив свечу ушёл... Несмотря на 20-минутную овацию оставшейся
четверти зала, он не вернулся.... Так получилось, что это был его прощальный
концерт.... Через несколько месяцев великого Дюка Элингтона не стало...


\iusr{Irena Visochan}
\textbf{Leonid Dukhovny}  

@igg{fbicon.musical.notes} @igg{fbicon.heart.suit} Спасибо, тронула, мурашки по
коже!

\iusr{Захар Давыдов}
\textbf{Leonid Dukhovny} 

На самом деле было три концерта подряд. Я был на всех трех. На первом Дворец
был заполнен примерно не две трети. На втором уже был полный Дворец - но билеты
можно было свободно купить. И только на третьем было то о чем Вы пишите.

\end{itemize} % }


\iusr{Svetlana Vishniova}
А у меня есть значок и пластинка с автографом Бенни Гудмана.

\iusr{Oleg Chorny}

Є короткометражний документальний фільм про гастролі оркестру Гудмена. Знімали
документалісти з Ленінград, але там чимало кадрів з Києва. А ще контрабасист
оркестру потім написав спогади про ці гастролі


\iusr{Сергей Хромешкин}

\ifcmt
  ig https://scontent-frx5-2.xx.fbcdn.net/v/t39.1997-6/s480x480/47472862_1846683478763869_5271603212267290624_n.png?_nc_cat=1&ccb=1-5&_nc_sid=0572db&_nc_ohc=Wa5Mf4xvcXQAX9K28FS&_nc_ht=scontent-frx5-2.xx&oh=00_AT_KXAez5TtoFtKRuIZNKaJCtJgUdPWmblHGqdgK3aVWdg&oe=61CE4FE7
  @width 0.2
\fi

\iusr{Георгий Майоренко}
Супер!!! Мои родители были на этом концерте. Отец был большим любителем джаза.

\iusr{Alexander Bronstein}

Был я на выступлении оркестра Бенни Гудмана в Киеве.
Воспоминание на всю жизнь. СОБЫТИЕ.

\begin{itemize} % {
\iusr{Marina Astistov}
\textbf{Alexander Bronstein} Я до сих пор помню рассказ папы и мамы о концерте, а особенно о том, как они попали на этот концерт. Билетов не было. Их друг работал инженером во Двоце спорта. Он втащил их через окно мужского туалета. Родители были ещё "подъёмные" Им было лет по 30.

\iusr{Alexander Bronstein}
\textbf{Marina Astistov} Мать моего школьного друга Феликса хотела выяснить, не родственники ли она с Бенни, так как её девичья фамилия была как раз Гудман, и кто-то из родни уехал в своё время в Америку. Не выяснили.  @igg{fbicon.frown} 
\end{itemize} % }

\end{itemize} % }
