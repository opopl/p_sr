% vim: keymap=russian-jcukenwin
%%beginhead 
 
%%file 12_10_2022.fb.butusov_jurij.1.shtanko
%%parent 12_10_2022
 
%%url https://www.facebook.com/butusov.yuriy/posts/pfbid02Vrbt4LTftGVekNQbUjpJWdmQuWrXssnP6XE1nFQ4oNBhySyJdZ2z2dg9sQqfzSArl
 
%%author_id butusov_jurij
%%date 
 
%%tags 
%%title 5 жовтня в бою під Бахмутом загинув піхотинець 93-ї бригади Дмитро Штанько
 
%%endhead 
 
\subsection{5 жовтня в бою під Бахмутом загинув піхотинець 93-ї бригади Дмитро Штанько}
\label{sec:12_10_2022.fb.butusov_jurij.1.shtanko}
 
\Purl{https://www.facebook.com/butusov.yuriy/posts/pfbid02Vrbt4LTftGVekNQbUjpJWdmQuWrXssnP6XE1nFQ4oNBhySyJdZ2z2dg9sQqfzSArl}
\ifcmt
 author_begin
   author_id butusov_jurij
 author_end
\fi

\ii{12_10_2022.fb.butusov_jurij.1.shtanko.pic.1}

5 жовтня в бою під Бахмутом загинув піхотинець 93-ї бригади Дмитро Штанько,
відомий тим, що записав розмови з братом голови Офісу президента Андрія Єрмака,
публікація яких спричинила скандал.

По Dmytro Shtanko була відкрита кримінальна справа, по ньому активно працювали
правоохоронці, він став особистим ворогом Єрмака. Справу по розмовам його
брата, звичайно, зам'яли.

У Штанько було троє дітей, але старший син вже став повнолітнім, він не
ухилявся від мобілізації і не шукав теплого місця у тилу. Він розповідав, що
співробітники військової контррозвідки СБУ постійно контролювали його
перебування в армії, нагляд за ним і його контактами в учбовому центрі та під
час переїзду на фронт був постійним, співробітник СБУ, з його слів,
супроводжував його одного.

Штанько потрапив у вересні в одну з наших кращих частин - 93-тю механізовану
бригаду в Бахмут. Він отримав найнебезпечнішу роботу на війні - став
піхотинцем. Там на передовій ніякого нагляду не було, там складна обстановка і
кожного дня йдуть запеклі бої.

Ось як написав Штанько 24 вересня на своїй сторінці:

"Кожного дня в кожного з нас є можливість померти безліч раз і масою варіантів.
Але до цього звикаєш. Захищати Батьківщину стає роботою.  Це карусель подій.
Людських життів. Емоцій. У багатьох стресу. Емоційного напруження. Для багатьох
рутина. Кожного дня по 10 історій, які стали одним днем життя від того, як
потрапив сюди і до того, як все скінчиться.

Бридко дивитися на майстрів піару, які вішають собі чужі нагороди, не забуваючи
при цьому будувати політику і примножувати статки за рахунок життів і здоров'я
реальних українських воїнів.

Пишаюся тим, що захищаю Батьківщину зараз поруч з моїми побратимами у складі 93
ОМБ "Холодний яр". Ці воїни - люди з великої букви.

Ми обов'язково переможемо".

Треба віддати належне - Штанько, попри відсутність бойового досвіду, проявив
себе як вмотивований та мужній солдат. В перший же день його направили на
спостережний пост, оскільки ситуація напружена, і на фронті потрібні
вмотивовані бійці. Він потрапив під обстріл, поруч з ним загинув наш боєць, але
він виконував усі поставлені завдання. Штанько воював дуже недовго, всього
місяць, але кожен день на тій найгарячіший ділянці фронту - це як маленьке
прожите життя. За місяць отримав дві контузії, але не став "відмовником", не
залишив позиції, а тримався разом з товаришами у строю.

5 жовтня російські війська після кількаденних обстрілів захопили одну з наших
позицій, яка була дуже сильно розбита. Ворог переважав абсолютно у боєприпасах
та артилерії, а у нас не було снарядів та можливості подавити вогонь російських
батарей. Командир роти зробив все що міг - він зібрав групу в 20 бійців та
попрохав відновити становище. В контратаку пішли тільки вмотивовані, тільки
добровольці. Серед перших пішов Штанько.

Ворог відкрив щильний мінометний та артилерийський обстріл, групу накрили.
Майже усіх поранило, кілька бійців загинуло, в тому числі Дмитро Штанько.

Дмитро Штанько був завзятим опонентом діючої влади, вважав Андрія Єрмака
корупціонером та російським агентом, який свідомо провалює підготовку України
до війни з РФ. Але якими б не були його політичні погляди, захист своєї країни
став для нього найважливішим у житті.

Штанько став на захист нашої України, бо свобода - це в тому числі свобода
думок та свобода довести свої погляди. Він взяв на себе відповідальність і
здобув справжній авторитет серед своїх бойових товаришів.

Дмитро Штанько бився за наше майбутнє та загинув у бою гідно та чесно, як
український герой.

Ми не забудемо його вчинок. Нічого не забудемо.
