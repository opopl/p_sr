% vim: keymap=russian-jcukenwin
%%beginhead 
 
%%file 29_05_2021.fb.mova_na_chasi.1.mova
%%parent 29_05_2021
 
%%url https://www.facebook.com/infomova/posts/3725156180929795
 
%%author Мова на Часі
%%author_id mova_na_chasi
%%author_url 
 
%%tags chelovek,jazyk,mova,ukraina
%%title Українською я почала говорити після 20
 
%%endhead 
 
\subsection{Українською я почала говорити після 20}
\label{sec:29_05_2021.fb.mova_na_chasi.1.mova}
\Purl{https://www.facebook.com/infomova/posts/3725156180929795}
\ifcmt
 author_begin
   author_id mova_na_chasi
 author_end
\fi

©️ Наталка Сонечко.

Українською я почала говорити після 20. Свідомо. І весь час мені було нелегко.
Спочатку мене не розуміли знайомі, навіть друзі. Всі думали, що я схиблена. У
той же час я зустріла прото-тітушку: вперше ( і сподіваюся, востаннє) в житті
чоловік на мене підняв руку. Просто на вулиці, серед білого дня п'яне, вибачте,
чмо сказилося не тільки від того, що я відмовилася з ним прогулятися, як те, що
я це зробила українською... Це було в Нікополі, наприкінці 90-х... 

Мені довелося боротися зі своїми комплексами, зі своїм незнанням рідної мови, з
чудернацькою вимовою. В голові весь час йшов синхронний переклад з російської
на українську і навпаки при спілкуванні з моїми російськомовними друзями... І
весь час ця фраза, яку я просто органічно ненавиджу - ґлавноє, чтоби чєловєк
бил хароший! Та не головне! 

\ifcmt
  pic https://scontent-lga3-2.xx.fbcdn.net/v/t1.6435-0/p526x296/192641963_3725156104263136_6682808270109248714_n.jpg?_nc_cat=111&ccb=1-3&_nc_sid=8bfeb9&_nc_ohc=BOPasSoCutAAX8SCTFW&_nc_ht=scontent-lga3-2.xx&tp=6&oh=3a6b432a2307ed91c1bb8a8ebb594756&oe=60DABAAD
\fi

Тобто, звичайно, людина повинна бути хорошою. Але людина - це не просто так
собі особина, що тиняється планетою. Людина - це дитина якихось батьків, якоїсь
країни... Вона - носій певного  генетичного коду, культури, мови. 

Дорогі мої, рідні мої російськомовні друзі! Ви не говорите українською, бо вам
\enquote{нравится этот красивый мелодичный язык} і ви \enquote{не хотите его коверкать}. Але в
той же час ви записуєтеся на англомовні курси. Є такий відомий вислів — скільки
мов ти знаєш, стільки разів ти й людина... а якщо ти не знаєш рідної мови — ти
хто? А якщо ти не маєш поваги до мови автохтонів землі, благами якої
користуєшся — хто ти? 

Мова не може розвиватися тільки в податкових деклараціях
і законах, що приймає ВР. Мова виживе, якщо вона буде скрізь — в науці,
техніці, вишах і дитячих садочках, на семінарах і на побаченнях. Мова житиме,
якщо буде літературною, сленговою, діалектною. Якщо вона буде емоційною,
потрібною, бажаною. 

Люди вмирали за наше право бути вільними. Люди вмирали за наше право мати своє
СЛОВО! Не бійтеся української мови, ви ж не боїтеся бути патріотами, українцями
(ким би ви не були за національністю). Полюбіть її — унікальну, неповторну,
співучу, душевну, багату, влучну, змістовну. Я змогла)...

❤️ Підтримай нашу сторінку:

\begin{itemize}
  \item 5168 7554 3571 4254 приват
  \item 4441 1144 4983 9131 моно
\end{itemize}
