%%beginhead 
 
%%file 28_12_2021.fb.mariupol.turystychne_misto.1.sobor_arhistratiga_mihaila_livoberezzhja
%%parent 28_12_2021
 
%%url https://www.facebook.com/mistoMarii/posts/pfbid02AxWT2PBe7dANX928aKXBDdyepHdvySTj73uoVRdTEuxXheDndgWhKZRmifWq4AfAl
 
%%author_id mariupol.turystychne_misto
%%date 28_12_2021
 
%%tags 
%%title Величний собор Архистратига Божого Михаїла на Лівобережжі
 
%%endhead 

\subsection{Величний собор Архистратига Божого Михаїла на Лівобережжі}
\label{sec:28_12_2021.fb.mariupol.turystychne_misto.1.sobor_arhistratiga_mihaila_livoberezzhja}

\Purl{https://www.facebook.com/mistoMarii/posts/pfbid02AxWT2PBe7dANX928aKXBDdyepHdvySTj73uoVRdTEuxXheDndgWhKZRmifWq4AfAl}
\ifcmt
 author_begin
   author_id mariupol.turystychne_misto
 author_end
\fi

Величний собор Архистратига Божого Михаїла на Лівобережжі.

А ви бували на оглядовому майданчику біля цього храму взимку?

🔹зведено собор в 1997 році в найкрасивішому місці Лівобережного району Маріуполя 

🔹Церква стала візитною карткою Маріуполя та одним з найвідоміших храмів міста.

🔹в народі церква називається \enquote{Поживанівською}, початок будівництва церкви
почався в 1995 р. за каденції мера - Михайла Поживанова, який був ініціатором
та меценатом проєкту.

🔹собор розташований на великому пагорбі у доглянутому сквері й має в своєму
розпорядженні неймовірний оглядовий майданчик з мальовничим краєвидом на
Азовське море та узбережжя.

🔹 Поруч споруджено пам'ятник святителю Ігнатію Маріупольському, небесному покровителю міста. 

🔹 Головною святинею собору є список ікони Божої Матері \enquote{Живоносне джерело},
написаний на початку XIX століття. Під час Другої світової війни ікона
зберігалася у прихожан. Після важкого періоду гонінь і смут образ Божої Матері
відреставрували і знову повернули церкві.

\#Маріуполь \#ТутВарто гуляти містом

Дякуємо за фото та інформацію Miroslav Yuriyoviĉ
