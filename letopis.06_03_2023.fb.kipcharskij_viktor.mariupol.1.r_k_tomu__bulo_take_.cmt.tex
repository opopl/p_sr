% vim: keymap=russian-jcukenwin
%%beginhead 
 
%%file 06_03_2023.fb.kipcharskij_viktor.mariupol.1.r_k_tomu__bulo_take_.cmt
%%parent 06_03_2023.fb.kipcharskij_viktor.mariupol.1.r_k_tomu__bulo_take_
 
%%url 
 
%%author_id 
%%date 
 
%%tags 
%%title 
 
%%endhead 

\qqSecCmt

\iusr{Елена Девина}

Я також вела щоденник війни. Навіть з 22 лютого. Не хотіла зараз до нього
звертатися-боляче. Але... Порівнюю Ваші й наші спогади... В нас \enquote{картинки} були
страшніші, бо у вікнах був \enquote{телевізор}.

\begin{itemize} % {
\iusr{Віктор Кіпчарський}
\textbf{Елена Девина} 

Те, що я викладаю - не сценарій фільму жахів. Це - щоденник подій річної давнини.

Я вже писав, що наш двір був чи не найспокійнішим місцем у Маріуполі.

Чому?

Я не маю на це відповіді.

\iusr{Елена Девина}
\textbf{Віктор Кіпчарський} Я так його (Ваш щоденник) і сприймаю. Мені теж цікаво де і як було

\iusr{Лилия Савицкая}
\textbf{Віктор Кіпчарський} а я вважаю, до 20 березня, наш був найспокійніший

\iusr{Віктор Кіпчарський}
\textbf{Лилия Савицкая} 

Не сперечаюсь. Хлопець з нашого будинку переїхав десь в районі "Комсомольця"
час від часу приїжджав до мами в наш двір - то він казав про найспокійніші
райони: його і наш. До речі, десь 14-го березня він переїхав до нас...

\iusr{Лилия Савицкая}
\textbf{Віктор Кіпчарський} 

в нашему дворі були люди з Восточного, Черёмушек, він невеличкий та
непрохідний, Але 16 березня після прямого попадання, більшість поїхали,
дякувати Господу, ніхто не загинув, ні ті, хто поїхав, ні, ті, хто залишився

\end{itemize} % }


\iusr{Светлана Водзянская-Живогляд}

Прокинулась такой рано, намітила тіста на хліб, звірила запаси, муки ще 2 кіло,
вистачить. Вчьора хлопці сказали що іде підмога! Вийшла за двір черга за водою
у колодязь, люди сваряться поміж собою, довелося заспокоювати бо шукали винного
хто втопив знову відро, довелося виносити вже 3 відро і сваритися аби поміняли
на нову мотузку (мою), люди накручені не хочуть чекати зайві 10 хвилин навіть.
Церква поряд не дає води людям вже (до того давали 5 л у одні руки колодязьної
води, свічка сама манюня по 50грн ), просто зачинили ворота і кажуть начебто
води у колодязі нема!! У нашому вода прибуває більше та чистіша.

\iusr{Светлана Водзянская-Живогляд}

Через 3 години принесла 4 відра води, бо живу я тут чи ні, а в черзі стояла як
усі, та й черга то джерело новин з усього міста. Випікла хліб, понесла половину
кумам. Ти останній раз коли пахло хлібом у моїй хаті. В теплій хаті у тазику
помила дітей, пішла знову по воду, а там прийшли люди з главпоштампу,
розказують що ринок бомбили і тепер звідти тягнуть цигани все що можна. Ми
відвели дітей кумам і бігом за хрещеним (якщо будуть питання він має там точки,
та й їжі в них не багато). Ми в трьох біжимо в сторону ринку тільки но
перебігли трамвайні колії по нам відкрили автоматний вогонь, я повзла під
гнилим дерев'яним парканом і молилася щоб не вбили, доповзли до кута, наче
перестали стріляти бо між нами і стріляючим тепер будинок. Встали і побігли
один за одним тримаючи 2 метри поміж нами щоб якщо вб'ють то не всіх зразу.

\iusr{Светлана Водзянская-Живогляд}

Біжимо, а поруч горять будинки, вирви від снарядів, по ходу оглядаємо всі
канави куди можна впасти якщо «накриє», кілька мертвих людей трохи присипаних
снігом лежать під будинками. Чоловік відвертає мою увагу від них бо знає як я
боюсь мертвих. Пробігаємо до ринку там дотягують те що лише можна- молотки,
труби інструменти, шпалери (вже мене це не лякає, всередині пустота), доходимо
до колій і тут чоловік розвертає мене до себе і каже \enquote{тільки не кричи! Я тебе
прошу не кричи} через секунду я розумію чому...

\iusr{Светлана Водзянская-Живогляд}

Посеред дороги лежить той самий сторож якому я обіцяла наливку. Йому хтось
натягнув шапку на очі та присипало снігом, з дороги не прибрати- прохід
вузький, довелося переступать. Це мій нічний кошмар найближчих місяців… ми
побігли в ринок, ятки всі згоріли, якби не вивезли вчьора товар то він би
згорів. Заходимо в ринок, все виломане, щось взривною хвилею, щось людьми,
більше людьми. Пішли на продукти, все розвороване, нічого крім картону і битого
скла, я у відчаї піднімаю все що потоптане ногами бо мені потрібні харчі дітям,
якщо не знайду то невідомо наскільки нам залишків вистачить. У контейнері
знайшла брудні приправи, згадала як бабуся розказувала що варили суп з трави,
взяла їх бо суп з трави з приправами мені здався смачніше. Схопила 10 пачок
дріжджів Львівських трохи придавлених( нашо якщо вдома 2 кіло борошна всього?).
Контейнери виявилися повністю пустими, їжі нема 😞 почули літак сердце стукає у
вухах, біжу під дах де продають овочі ( дивне бажання щоб тебе не бачили, якщо
не бачать не скинуть бомбу). Поки літак полетів побачила на вітрині пошарпану
банку з нутом, зраділа як дитина, чоловік та хрещений закинули мене у контейнер
бо все у склі, а я одна маленька. Зверху нічого, полізла під купи скла, чоловік
кричить вилазити бо знову літак, а в мене \enquote{чуйка} що тут є їжа!! ЗНАЙШЛА!!!! Ще
2 л банка нуту в перемішав з сочевицею!!! А ще нижче горіхи аж 4 пачки!!! І
пакет з дорогими цукерками курага та чорнослив у шоколаді. Більше нічого нема
😞 в сусідній ятці знайшли розчину каву в пакеті. Хапаємо і біжимо назад. Бігти
важко, але розуміння що тепер ще на кілька тижнів є їда хоч і така незвична
гріє душу!! Ту каву ми потім обміняли біля колодязю на шматок сиру, а частину
горіхів віддали батальйону брата після того як їх базу розбили і хлопці
лишилися голодні

\begin{itemize} % {
\iusr{Віктор Кіпчарський}
\textbf{Светлана Водзянская-Живогляд} Дякую Бога, що в нас такого жаху не було. Поруч було, а наш двір, наш будинок наче й тост прикрив від страхіття.

\iusr{Віктор Кіпчарський}
\textbf{Светлана Водзянская-Живогляд} Запішить свої спогади, хоч як це важко пригадувати та переживати, хоч і у уяві, ще раз.
\end{itemize} % }

\iusr{Светлана Водзянская-Живогляд}

Надалі сніданок моїх дітей був - цукерки та чай, на перекус горіхи, вечорі нормальне гаряче годування.

\begin{itemize} % {
\iusr{Віктор Кіпчарський}
\textbf{Светлана Водзянская-Живогляд} 

Як це не божевільно виглядає, але завдяки моїй болячці, ми заздалегідь робили
запаси на січень-березень. І того року також. Крім того, в перші дні ми
купували їжу, бо нас було двоє, а стало шестеро.

Ну і я приглядався до голубів і чекав, що ось-ось піде бичок.

\iusr{Віктор Кіпчарський}
\textbf{Светлана Водзянская-Живогляд} 

Це було саме в той день рік тому?

Я збираю хронологію.

Звичайно, Ваші спогади без Вашого дозволу я не використаю.

\iusr{Светлана Водзянская-Живогляд}
\textbf{Віктор Кіпчарський} так, тільки може +/- 1 день. Бо дні зливалися вже в 1 великий кошмар

\iusr{Віктор Кіпчарський}
\textbf{Светлана Водзянская-Живогляд} Так, тому я й почав щовечора записувати події за день.

\iusr{Светлана Водзянская-Живогляд}
\textbf{Віктор Кіпчарський} в мене не було сил записувати 😞 хотілося забути як страшний сон, та й світло витрачати лише на себе здавалося марнотратством
\end{itemize} % }

\iusr{Leonid Ehdelshteyn}

Встать, после таких ударов судьбы, мобилизовать внутренние резервы, не
надломиться, сопротивляться несчастьям, - всё это, куда легче написать, чем
применить на практике. И потом, нет работы, нет крыши над головой, нет ничего,
ощущение, что в жизни, уже никогда не будет энергии начать всё сначала, вокруг
полно людей предпочитающих сидеть и жалеть только себя, есть война - какие силы
нужно иметь, чтобы отогнать мысль, что слишком поздно что-то менять, пытаться
вернуть свою жизнь и ещё положительно влиять на изменение, возвращение к жизни
окружающих!

и выстраивать её, жизнь, на новом месте, возрождать в себе желание думать о
будущем!

Сильно, земляки! Вы не убегаете из жизни, вы строите или будете выстраивать
новые планы жизни, новые увлечённости, вы продолжаете жить!

Я хотел бы для себя одного, чтобы жизнь всегда приводила ко мне, таких сильных
людей, как Вы!
