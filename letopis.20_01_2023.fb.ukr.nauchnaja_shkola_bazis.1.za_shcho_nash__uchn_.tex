%%beginhead 
 
%%file 20_01_2023.fb.ukr.nauchnaja_shkola_bazis.1.za_shcho_nash__uchn_
%%parent 20_01_2023
 
%%url https://www.facebook.com/sciencebasis/posts/pfbid02RZWjtsLNtqFMiKRdm53o4RuTQFz92qpTF49GHb76XfNBpkFRQUpnsVY8aiQ1bDYQl
 
%%author_id ukr.nauchnaja_shkola_bazis
%%date 20_01_2023
 
%%tags shkola,deti,nauka,obrazovanie
%%title За що наші учні так люблять онлайн-уроки у "Вечірньому Базисі"? 
 
%%endhead 

\subsection{За що наші учні так люблять онлайн-уроки у \enquote{Вечірньому Базисі}?}
\label{sec:20_01_2023.fb.ukr.nauchnaja_shkola_bazis.1.za_shcho_nash__uchn_}

\Purl{https://www.facebook.com/sciencebasis/posts/pfbid02RZWjtsLNtqFMiKRdm53o4RuTQFz92qpTF49GHb76XfNBpkFRQUpnsVY8aiQ1bDYQl}
\ifcmt
 author_begin
   author_id ukr.nauchnaja_shkola_bazis
 author_end
\fi

За що наші учні так люблять онлайн-уроки у \enquote{Вечірньому Базисі}? 

Ми думаємо, тому що це:

- живі заняття та спілкування з улюбленими вчителями;

- якісні знання, які ми даємо дітям;

- щоразу цікаві завдання та задачі, які наші вчителі спеціально придумують та
відшукують для кожного уроку;

- зручний час проведення уроків (17:40-20:40);

- можливість навчатися перебуваючи будь-де. 

Учні мають свою думку з цього приводу. І ось, що вони кажуть про навчання у
\enquote{Вечірньому Базисі}.

Марія Коваленко — 9-В клас: 

\enquote{Мені дуже подобаються уроки у \enquote{Базисі}. Я завжди в захваті, як вчителі
організовують уроки та відносини з учнями. Я зрозуміла для себе, що чогось не
знати, не розуміти та робити помилки — це нормально. Вчителі хочуть тебе
підтримати й будуть пояснювати хоч 100 разів, поки ти не зрозумієш. Твою думку
завжди  вислухають і з повагою до неї віднесуться. Для мене \enquote{Базис} — це не про
\enquote{бо треба}, а про \enquote{я хочу}}.

Якщо ви бажаєте також навчатися у \enquote{Вечірньому Базисі} — реєструйтеся:
\url{https://www.basis.kiev.ua/basis_online} 

\#Basis \#basisschool \#вечірній\_basis
