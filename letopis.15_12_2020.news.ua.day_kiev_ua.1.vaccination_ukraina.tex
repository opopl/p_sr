% vim: keymap=russian-jcukenwin
%%beginhead 
 
%%file 15_12_2020.news.ua.day_kiev_ua.1.vaccination_ukraina
%%parent 15_12_2020
 
%%url https://day.kyiv.ua/uk/news/151220-pershi-dozy-covid-vakcyny-moz-rozrahovuye-otrymaty-v-lyutomu
 
%%author 
%%author_id 
%%author_url 
 
%%tags 
%%title Перші дози COVID-вакцини МОЗ розраховує отримати в лютому
 
%%endhead 
 
\subsection{Перші дози COVID-вакцини МОЗ розраховує отримати в лютому}
\label{sec:15_12_2020.news.ua.day_kiev_ua.1.vaccination_ukraina}
\Purl{https://day.kyiv.ua/uk/news/151220-pershi-dozy-covid-vakcyny-moz-rozrahovuye-otrymaty-v-lyutomu}

\ifcmt
  pic https://day.kyiv.ua/sites/default/files/styles/460-news/public/news/15122020/2020-09-04t110441z_448976463_rc2nri9bmgka_rtrmadp_3_health-coronavirus-russia-vaccine.jpg?itok=M2KNvMmC
  width 0.5
  fig_env wrapfigure
\fi


Міністерство охорони здоров’я розраховує отримати перші дози вакцини від
COVID-19 у першому кварталі 2021 року.

Як передає Укрінформ, про це міністр охорони здоров'я Максим Степанов сказав в
інтерв'ю Радіо НВ.

«Ми дуже розраховуємо на те, що перші вакцини ми отримаємо вже в I кварталі
наступного року. Зараз намагаємося робити все, щоби перші дози були в Україні в
лютому», - повідомив Степанов.

Так, за його словами, МОЗ вже провело переговори щодо стратегії зберігання
вакцин, деякі з яких, наприклад розробки Pfizer, повинні зберігатися за
ультранизьких температур у -80°C, з промисловими підприємствами країни, що
мають холодильні потужності.

«Близько 1 млн доз ми будемо в змозі зберігати», - сказав Степанов.

За словами міністра, вакциною від Pfizer в першу чергу будуть вакциновані ті
категорії населення, які МОЗ може «дуже чітко проконтролювати»: медпрацівники,
військовослужбовці, співробітники сфери освіти.

Як відомо, Міністерство охорони здоров'я планує вакцинувати за кошти державного
бюджету близько 21 мільйона громадян. 
