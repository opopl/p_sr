% vim: keymap=russian-jcukenwin
%%beginhead 
 
%%file 10_05_2021.fb.respublikalnr.1.lnr_pesnja_fleshmob
%%parent 10_05_2021
 
%%url https://www.facebook.com/groups/respublikalnr/permalink/810106509625142/
 
%%author 
%%author_id 
%%author_url 
 
%%tags 
%%title 
 
%%endhead 
\subsection{ФЛЕШМОБ. Более 600 жителей ЛНР приняли участие в песенном флешмобе ко Дню Республики}
\label{sec:10_05_2021.fb.respublikalnr.1.lnr_pesnja_fleshmob}
\Purl{https://www.facebook.com/groups/respublikalnr/permalink/810106509625142/}

\ifcmt
  pic https://external-frt3-2.xx.fbcdn.net/safe_image.php?d=AQGwnVS1COwAxjbM&w=500&h=261&url=https%3A%2F%2Fi.ytimg.com%2Fvi%2FtyqnPa9KjeM%2Fmaxresdefault.jpg&cfs=1&ext=jpg&ccb=3-5&_nc_hash=AQFxrKG5D7SHBH0a
\fi

ФЛЕШМОБ. Более 600 жителей ЛНР приняли участие в песенном флешмобе ко Дню
Республики Более 600 жителей ЛНР приняли участие в песенном флешмобе ко Дню
Республики, организованном общественным движением (ОД) "Мир Луганщине". Об этом
сообщила пресс-служба движения.  

"Ко Дню Луганской Народной Республики в общественном движении "Мир Луганщине"
провели флешмоб \enquote{Республика поет}, в котором приняли участие более 600 жителей
ЛНР. Каждый желающий поучаствовать в песенном флешмобе к седьмой годовщине
Республики мог записать видео своего исполнения песни \enquote{За тебя, Родина-мать}
группы "Любэ" и выложить его в социальные сети с хештегом #люблюЛНР", –
говорится в сообщении.

В ОД уточнили, что жители Республики загрузили в социальные сети более 80 видеозаписей с хештегом #люблюЛНР.

\enquote{Мы рады, что жители ЛНР так активно принимали участие во флешмобе 
\enquote{Республика поет}. Песню 
\enquote{За тебя, Родина-мать} исполнили разные люди, каждое видео с
хештегом #люблюЛНР – особенное. Пели профессионалы и любители, взрослые и дети,
творческие коллективы и сольные исполнители. Но всех их объединяет душевность
исполнения, неподдельная любовь к Родине и трепетное отношение к ней. 12 мая мы
празднуем День Республики, этот праздник для каждого из нас имеет огромное
значение, ведь семь лет назад мы сделали свой выбор и ни на секунду не
усомнились в нем. Мы не поддались киевскому режиму, не отказались от своей
истории, своих идеалов и ценностей, мы отстояли свою независимость, свое право
говорить на русском языке}, – 
отметили в \enquote{Мире Луганщине}.

Там добавили, что "участники флешмоба "Республика поет" в очередной раз показали единство народа ЛНР".

\enquote{Мы любим свою Республику, хотим жить, работать и учиться, воспитывать детей здесь}, – подчеркнули в общественном движении.

Итоговое видео флешмоба будет транслироваться в День ЛНР, 12 мая, на телеканале
\enquote{Луганск 24}. Также видео можно посмотреть на канале \enquote{Мира Луганщине} в YouTube
или в группе движения в социальной сети \enquote{ВКонтакте}.

