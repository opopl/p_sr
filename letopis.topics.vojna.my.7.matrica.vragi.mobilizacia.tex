% vim: keymap=russian-jcukenwin
%%beginhead 
 
%%file topics.vojna.my.7.matrica.vragi.mobilizacia
%%parent topics.vojna.my.7.matrica.vragi
 
%%url 
 
%%author_id 
%%date 
 
%%tags 
%%title 
 
%%endhead 

\paragraph{12:55:52 21-09-22 Сергей МамонтOFFreplied to Владимир}

Что за радость? К армии РФ +300 тысяч человек и это только первая неполная
мобилизация!!! У нас в Украине заканчивается 4 волна мобилизации тероборонны,
мужики вступали в тероборонну чтобы не попасть на фронт, а теперь их посылают
умерать именно туда!!! Следующая волна мобилизации будут женщины и дети
Zeле-югент 14-18 лет, ещё осенью прошлого года министерство обороны Украины
подписали приказ 313 о постановке на воинский учёт женщин 18-60 лет, а теперь и
они должны умереть за бункерного карлика наркомана Zеленского??? Zеленский 4
раза уклонялся от призыва в армию Украины, а теперь посылает на убой женщин и
детей!!!

\paragraph{23:48:01 21-09-22 💕✧༺Елена ✧Шамрай༻✧💕}
23:44
Александра Себянина
сегодня в 13:03
МОБИЛИЗАЦИЯ
Ну что ж, мы долго ждали и наконец дождались. Вполне ожидаемо в сети начал подниматься вой тех, кому с 24 февраля плели чушь про спец.операцию, успокаивая, что это там где-то, ща кааааак ударим по центрам принятия решений после очередного выхода за красные линии. Мы не сделали ровным счётом НИЧЕГО, чтобы подготовить наше общество к мобилизации и пресловутому "Родина-мать зовёт". И я понимаю, что сейчас практически каждая мать и жена будет цепляться за малейшую призрачную возможность, чтобы не пустить, уберечь, спрятать. Потому что нет понимания, что это война, не локальная, как мне частенько пишут, что тут как в Чечне. Нет, ребята, это не Чечня, это полноценная мировая война, нравится вам это или нет. Да, пока она локализована на сравнительно небольшом пятачке. Но когда тебя кроют французскими, американскими и прочими снарядами, ощущаешь в полной мере, как весь мир ощетинился против тебя.
И пока московские хипстеры и прочие стонут про кровавый режим и принудительную мобилизацию, Донбасс насупился и сжал кулаки. За каждое слово нашего Президента, за каждое кардинальное решение, Донбасс платит кровью. Вот уже 8 лет. Мало кто ждал выступления Верховного так, как ждали здесь. Вслушиваясь в каждое слово. И хоть сейчас все замерли в ожидании, появилась робкая надежда, что всё-таки нас не бросили, что вот-вот, ещё чуть-чуть потерпеть и всё наладится.
Больше всего я хочу, чтобы завтра эти люди проснулись в одной большой и сильной стране - России. Они слишком долго ждут. Все эти годы я испытывала стыд. Мне было стыдно за свою страну, за то, что мы дали надежду и потом просто наблюдали, как Донбасс умывается кровью. Да, помогали, но не спасли. Сейчас я в какой-то мере искупаю свою личную вину, принимая участь этого региона вместе с ним и с теми кого люблю.
Не тратьте время на истерики и панику, подготовьтесь по максимуму сами и подготовьте своих близких, потратьте время на основы оказания первой медицинской помощи и обращения с оружием. Война давно зашла в ваш двор, вы просто старались отгородиться от неё, задёрнув шторы.
#ZOV #Россия #своихнебросаем

