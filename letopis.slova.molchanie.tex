% vim: keymap=russian-jcukenwin
%%beginhead 
 
%%file slova.molchanie
%%parent slova
 
%%url 
 
%%author 
%%author_id 
%%author_url 
 
%%tags 
%%title 
 
%%endhead 
\chapter{Молчание}

%%%cit
%%%cit_head
%%%cit_pic
%%%cit_text
Вы понимаете, что происходит? И правды нет ни в той, ни в другой стороне.
Поэтому собственно я в последнее время принял решение \emph{молчать}. Я не публикую
блогов, я прекратил проект «Субботний кофе».  Прекратили проект «Субботний
кофе»?  Да, прекратили проект «Субботний кофе». Вместе с коллегами закончили
циклы циклов. Вы поймите, что в ситуации ресентимента мы пытались достучаться
до мышления. Но невозможно. И как бы, я считаю, что есть время говорить, есть
время \emph{молчать}. Вполне возможно, что в ситуации ресентимента... из нее не удастся,
это черная дыра, она засасывает всех, поэтому единственное, что я вижу – это
просто \emph{молчание}. Потому что есть вещи, из которых выхода просто нет. Люди,
которые попали в черную дыру ресентимента, не могут выбраться ни
самостоятельно, ни при помощи людей, мыслителей, психологов
%%%cit_comment
%%%cit_title
\citTitle{Сергей Дацюк: Украина сегодня - не просто попрошайка, она на мусорнике истории}, 
Сергей Дацюк; Людмила Немыря, hvylya.net, 28.06.2021
%%%endcit

