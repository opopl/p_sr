% vim: keymap=russian-jcukenwin
%%beginhead 
 
%%file 13_12_2021.fb.fb_group.story_kiev_ua.2.raf_marshrutka
%%parent 13_12_2021
 
%%url https://www.facebook.com/groups/story.kiev.ua/posts/1818013848395408
 
%%author_id fb_group.story_kiev_ua,kuzmenko_petr
%%date 
 
%%tags gorod,kiev,marshrutka,sssr,transport
%%title РАФ - Наши любимые маршрутки детства и юности!
 
%%endhead 
 
\subsection{РАФ - Наши любимые маршрутки детства и юности!}
\label{sec:13_12_2021.fb.fb_group.story_kiev_ua.2.raf_marshrutka}
 
\Purl{https://www.facebook.com/groups/story.kiev.ua/posts/1818013848395408}
\ifcmt
 author_begin
   author_id fb_group.story_kiev_ua,kuzmenko_petr
 author_end
\fi

В эти декабрьские дни вдруг подумалось: \enquote{Старею. Свою коллекцию авто моделек я
стал собирать 50 лет назад, получив в подарок на 6 лет первую. Ещё один
юбилей.} И взгляд, сам по себе, остановился на двух моделях советских
микроавтобусов РАФ. Наши любимые маршрутки детства и юности! 

\begin{multicols}{2} % {
\setlength{\parindent}{0pt}

\ii{13_12_2021.fb.fb_group.story_kiev_ua.2.raf_marshrutka.pic.1}
\ii{13_12_2021.fb.fb_group.story_kiev_ua.2.raf_marshrutka.pic.1.cmt}

\ii{13_12_2021.fb.fb_group.story_kiev_ua.2.raf_marshrutka.pic.2}
\ii{13_12_2021.fb.fb_group.story_kiev_ua.2.raf_marshrutka.pic.2.cmt}

\ii{13_12_2021.fb.fb_group.story_kiev_ua.2.raf_marshrutka.pic.3}
\ii{13_12_2021.fb.fb_group.story_kiev_ua.2.raf_marshrutka.pic.3.cmt}
\end{multicols} % }

Знаменитый и любимый многими киевлянами маршрут с Красной площади до
Бесарабского рынка.  Стоимость проезда, помню, была 15 копеек. Сначала
пассажиров возили 977 РАФ \enquote{Latvija}, с закруглённой \enquote{мордашкой} и очень
просторные. Маленьким я ездил с мамой на Бесарабку за чем-нибудь особенным,
чего было не купить на нашем Подоле. Потом мы гуляли в сторону Подола пешком по
Крещатику. Заходили в вареничную перекусить, в магазины, в Пассаж. 

Если у меня хватало сил, шли неспешно по Владимирскому спуску и Жданова домой,
на Андреевский спуск. А если я уставал в Верхнем городе, мы ехали на 16 трамвае
с площади Ленинского Комсомола до Красной. 


\begin{multicols}{2} % {
\setlength{\parindent}{0pt}
\ii{13_12_2021.fb.fb_group.story_kiev_ua.2.raf_marshrutka.pic.4}
\ii{13_12_2021.fb.fb_group.story_kiev_ua.2.raf_marshrutka.pic.4.cmt}

\ii{13_12_2021.fb.fb_group.story_kiev_ua.2.raf_marshrutka.pic.5}
\ii{13_12_2021.fb.fb_group.story_kiev_ua.2.raf_marshrutka.pic.6}
\ii{13_12_2021.fb.fb_group.story_kiev_ua.2.raf_marshrutka.pic.7}

\end{multicols} % }

Потом наши поездки на этих маршрутках стали менее радужными и яркими. Мы ездили
к папе в больницу рядом с Бесарабским рынком или в госпиталь, поднимаясь по
бульвару Леси Украинки пешком от Бесарабки. Всё чаще, с возрастом, он
становился пациентом лечебных учреждений. Беспокоили старые раны и возрастные
хвори. Их поколение было более изношено жизнью, если можно так выразиться. 

В то время маршрутное такси здорово облегчало нам дорогу и жизнь, ибо мы с
мамочкой вспоминали наши более ранние поездки и прогулки, в которых иногда нас
сопровождал и папа. 

Где-то в этот период устаревшие РАФ-977 были заменены на РАФ-2203, более
современные и имевшие красивый стремительный профиль. Уже на таких машинах мы
совершали начальную часть наших романтических прогулок с Нелли в девятом -
десятом классах, неизменно доезжая от уже появившегося памятника философу
Григорию Сковороде до Бесарабки. 

Потом мы
гуляли в разных местах Города, постепенно приближаясь к родному Подолу. И,
проводив любимую домой на Почайнинскую, уже в районе полуночи или даже позже я
добирался к маме на Автозаводскую. Она всегда ждала меня и волновалась, но
никогда не отчитывала за ночные прогулки. Мамочка очень любила меня и всё
понимала. 

С маршрутным такси тесно связана и дата появления на свет нашей
дочери. В начале июля 1985-го у нас заканчивалась штурманская практика и поход
вокруг Европы. Катюша по плану должна была родиться 1-2 июля. И я, как будущий
папаша, должен был дежурить под окнами роддома. 29 июня день рождения нашей
подруги и Нелли, находясь \enquote{на сносях}, со всеми мерами предосторожности поехала
за подарком на Крещатик на метро. Выйдя из магазина она, бережно придерживая
животик, собиралась последовать домой тем же путём. Однако встретила другую
свою подольскую подругу детства, тоже вышедшую замуж за \enquote{квумпаря}, окончившего
училище на год раньше. 

Татьяне, в первом лейтенантском отпуске, не терпелось
поделиться впечатлениями военно-морской офицерской жены с годичным стажем. И
она увлекла Несю на нашу привычную маршрутку. Киевляне знают дорожное покрытие
Владимирского спуска. Мостовая в 1985-м была ещё послевоенной. РАФ
добросовестно встряхнул мою дражайшую супругу по пути на Подол. 

И, едва придя домой, любимая быстренько уехала в роддом. 28 июня, проходя
турецкий Босфор и усиленно фотографируя Стамбул с обеих сторон - Европы и Азии,
мы услышали по громкой трансляции учебного корабля \enquote{Гангут}
поздравление меня с рождением дочери. 

Хорошо, хоть за пару дней до выписки моих девочек я появился под окном роддома,
чёрный от средиземноморского загара. Позже жена поведала мне удивление своих
соседок по палате. Сначала, вместо мужа, её посетили подруги, с угощениями с
именинного стола, затем мама (свекровью она никогда не была для Нелли) и
наконец темнокожий муж. Одна из соседок высказала удивление: \enquote{Надо же,
родила белобрысую светлую девочку, а муж негр}. 

Так что легендарные киевские маршрутки между Красной и Бесарабской площадями
сыграли весьма значимую роль в судьбе нашей семьи и вряд ли сотрутся из памяти
когда-либо.

\ii{13_12_2021.fb.fb_group.story_kiev_ua.2.raf_marshrutka.cmt}
