% vim: keymap=russian-jcukenwin
%%beginhead 
 
%%file 16_07_2020.fb.zharkih_ekaterina.1.bednost_ukraina_mif
%%parent 16_07_2020
 
%%url https://www.facebook.com/kate.zharkih.5/posts/4659230267436374
 
%%author 
%%author_id 
%%author_url 
 
%%tags 
%%title 
 
%%endhead 


\subsection{Украинская бедность — миф}
\label{sec:16_07_2020.fb.zharkih_ekaterina.1.bednost_ukraina_mif}
\Purl{https://www.facebook.com/kate.zharkih.5/posts/4659230267436374}

Наша страна, вопреки пугалкам политиков (особенно тех, которые обычно при
власти) имеет огромный потенциал и могла бы сама себя прокормить. 


\ifcmt
  pic https://scontent-frt3-2.xx.fbcdn.net/v/t1.6435-9/109563623_4658958440796890_8073717367493923251_n.jpg?_nc_cat=103&ccb=1-3&_nc_sid=8bfeb9&_nc_ohc=UxhlkOP1fmwAX-WAckn&_nc_ht=scontent-frt3-2.xx&oh=f72a1c8579185d7cb944f8fb28fd2074&oe=6098510D
\fi

Мы могли бы не ходить вечно с протянутой рукой, в которую загоняют иглы
кредитов МВФ, как якобы «единственное спасение экономики». 

Украина находится в выгодном географическом положении, граничит с 7 странами,
имеет выход к морю. Имеются огромные территории с черноземом, полезные
ископаемые, леса, вдоволь пресной воды. Торгуй - не хочу, как минимум 7 стран
можно снабжать самой разной продукцией и ещё больше отправлять по морю. 

Только нужно вкладываться в предприятия, а все хотят набить карманы и поскорее
свалить в Англию или США, вариант развития Родины никто не рассматривает.

Нас отвлекают мовами, пятыми колонами, рассказывают о «нужных/ненужных»
гражданах, продолжают войну, чтобы уничтожить лишних людей. 

Зачем? 

Потому что голодными людьми легче управлять и манипулировать. Давайте хлеба и
зрелищ, и все будут довольны.

Поэтому и уничтожается образование, медицина, переписывается история.
Пенсионеры тоже идут на «утилизацию», а то вдруг скажут, что в страшном Союзе,
которым пугают, было не так уж плохо. Вдруг люди задумаются о социальной
справедливости и потребуют каких-то гарантий. Бесплатной медицины, жилья,
например. 

И можно было бы сказать, что просто плохие люди у власти, но они же управляются
извне. Посмотрите только на посла ЕС, который обратился к руководству нашей
страны с угрозами и требованием отменить последние действия Верховной Рады,
направленные на стимулирование работы украинского машиностроения. И потом
кто-то будет рассказывать, что это все «теории заговора, которые ничем не
обоснованы». 

Иногда мне кажется, что людям проще закрыть глаза и делать вид, что ничего не
происходит. Так проще. И чувство, что тебя обманывают неприятное, с этим же
что-то делать придётся. Проще плыть по течению, да? 

А из Украины уже много лет хотят сделать сырьевой придаток и использовать как
кнопку давления на Россию. И пока что внешним силам это удаётся. 

Вопрос для нас в том, что нам самим выгодно? Когда пресекают развитие нашей
промышленности, по внешней указке просто запрещают развиваться, садят на вечные
кредиты, которые будут выплачивать наши праправнуки из своего кармана — это ни
разу не патриотично. На каком бы языке не говорил и в какой бы красивой
вышиванке не одет тот, кто это продвигает. 

Что можем сделать мы как граждане?

Я думаю, во-первых, дать адекватную оценку происходящему. Для этого к делу
нужно подходить с холодным умом, отбросив эмоции и штампы, которыми забивают
нам голову. Потом находить единомышленников, дискутировать и искать пути
решений. 

После – поддерживать лидеров, которые говорят и делают эти самые адекватные
вещи. Да, не все белые и пушистые, но нам нужно смотреть в наше общее будущее и
создавать его вместе. 

А не быть беспомощными свидетелями и заложниками навязанной нам «бедности».
