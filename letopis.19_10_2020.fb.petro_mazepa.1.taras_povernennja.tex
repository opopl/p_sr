% vim: keymap=russian-jcukenwin
%%beginhead 
 
%%file 19_10_2020.fb.petro_mazepa.1.taras_povernennja
%%parent 19_10_2020
%%url https://www.facebook.com/groups/promovugroup/permalink/837843976789465/
%%tags taras shevchenko,cinema
 
%%endhead 

\subsubsection{Він бачив Пекло і Неволю. Тарас. Повернення.}
\label{sec:19_10_2020.fb.petro_mazepa.1.taras_povernennja}

Петро Мазепа, 19 октября в 20:42, Киев.
\index[authors.rus]{Мазепа, Петро}

\url{https://www.facebook.com/groups/promovugroup/permalink/837843976789465/}

Багато українців, як правило, не повністю мають уявлення, що таке феномен
Тараса Шевченка. А це справді є феномен. Олександр Денисенко взявся за
відтворення дуже складного періоду життя Шевченка --- це його солдатчина,
заслання.

Люди ніби й знають, що Шевченко був у засланні 10 років, але нам мало відомо,
що йому там довелося пережити.

Тезисно про фільм:

\begin{itemize}
\item - розповідає про останні три місяці заслання у Казахстані (на той час окупованому московитами) та про подальше звільнення, яке не пройшло гладко та мало не закінчилося печально;

\item - чи не перша спроба в історії українського кіно розкрити постать Кобзаря не з точки зору вихолощеного академізму, а як цікаву та неординарну людину;

\item - забудьте про образ "діда в рушниках". Адже у цій стрічці, як і у житті, Шевченко --- жартівливий, ексцентричний та романтичний авантюрист;

\item - тут є чудові казахські краєвиди, музика Мирослава Скорика та голос Каті Chilly;
\item - розкрито образ Тараса як близьку серцю кожного українця, рідну людину, що в
страшних умовах заслання в пустелі, у "в'язниці без кордонів" виборює власну
свободу (бо саме із свободи особистості починається свобода всього народу);

\item - це не істерн і не екшн (трейлер оманливий), а все ж таки --- більше драма;
\item - цей фільм відмовлялися фінансувати у часи Януковича через критику російського шовінізму;
\item - на жаль, маємо з чим порівняти. Тараса з московської неволі визволяли
				тоді.  Тепер визволяємо багатьох військовополонених та політв'язнів від
								того ж таки "доброго сусіда".
\end{itemize}

Думки щодо фільму розділилися. Половині глядачів сподобалося, а половина
жорстко критикує. Та все ж я рекомендую цю стрічку до перегляду. Хоча б тому,
що це перший фільм про Шевченка для великих екранів за останні 56 років.

P.S. Ходіть, малята, у кіно. Бо дядькові Петрові скоро, мабуть, набридне
відчувати себе наче vip у залах кінотеатрів. --- здесь: Кінотеатр Флоренція КП
``Київкінофільм''
