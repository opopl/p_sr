% vim: keymap=russian-jcukenwin
%%beginhead 
 
%%file 04_10_2021.fb.gorovyj_ruslan.1.jazyk_posty_facebook.cmt
%%parent 04_10_2021.fb.gorovyj_ruslan.1.jazyk_posty_facebook
 
%%url 
 
%%author_id 
%%date 
 
%%tags 
%%title 
 
%%endhead 
\subsubsection{Коментарі}

\begin{itemize} % {
\iusr{Ярослав Матюшин}

Аби писати українською, мені треба замислюватись та зосереджуватися. Але ж
іноді пишеш з бодуна, і тоді лінькі вмикати башню.

\begin{itemize} % {
\iusr{Вікторія Валевська}
Ярославе, не придурюйтесь. З вашими то здібностями до мов...
\end{itemize} % }

\iusr{Соня Поліщук}
Інколи фб нагадує пости тих врємьон,соромно бляха і якісь вони бездушні як і той язик....

\iusr{Інна Долгушина}
Коли знаходжусь в дуже емоційному стані. Наприклад, обурена або розлючена, можу написати російською. Мова агресора, фіглі.

\iusr{Natalya Lyutikova}
Теж не розумію цього.

\iusr{Mariya Syemashkina}

Бо думаю все одно російською. Хоча нею вже майже не читаю книжок. Але є одне
але) пости з відгуками на книжки пишу все ж виключно українською, так що це
питання часу, щоб і все інше писати українською)

\begin{itemize} % {
\iusr{Евгений Кузин}
\textbf{Mariya Syemashkina} 

Це питання внутрішнього переналаштування. Коли я почав курс Upper Intermediate
англійської, почав перемикати звукову доріжку в кіно на англійську, читати
англомовні джерела новин, книги. Перші півроку було важко капець, потім почав
ловити себе на тому, що свої внутрішні діалоги часто веду англійською, не
помічаючи цього, дуже прикольне відчуття. Звісно, англійська в мене -
необхідність, мої робочи дні проходять виключно англійською. Тому думаю, що це
питання умов, якщо середовище сприяє, мова розвинеться до вільного використання
і на рівні підсвідомості.


\iusr{Mariya Syemashkina}
\textbf{Eugene}, 

моя робоча мова теж англійська) і німецькою я думаю іноді, проте мій внутрішній
голос думає переважно російською і, так, живу в російськомовному середовищі

\end{itemize} % }

\iusr{Vitally Sladko}

Все просто. Разговорный язык повседневного общения - русский. Посты в фб - это
как обычное общение, потому пишут на том языке, на котором разговариваю.

В Украине живу, вижу ее исключительно как независимое государство, а РФ как
врага, который пытается в течение веков уничтожить Украину.

Рус язык, как и англ язык на мой взгляд не является маркером принадлежности к
ценностям современной РФ.


\iusr{Ruslan Shloma}

Хто знa. Я от дивлюся по своїм друзям. Всі спілкуються між собою українською. Я
навіть і не згадаю, чи чув від них російську мову. Але у фейсбуці, чи в
інстаграмах всяких пишуть виключно російською. Навіть між собою у вайбері. Щоб
було по-городському. Бо і ми, Химко, люди. Я із-за цього вийшов з групи нашої
компанії.

\begin{itemize} % {
\iusr{Юрій Горішний}
\textbf{Ruslan Shloma}, прикро таке спостерігати... Дійсно, люди, яким проблемно вимовити російською дві фрази, пишуть все російською з жахливими помилками...

\iusr{Halyna Rudyka}
о, і я таке бачила!
\end{itemize} % }

\iusr{Евгений Кузин}

Mother tongue, переважно. Особисто для себе будь-яку мову вважаю перш за все
інструментом обміну інформацією, тому загальний принцип маю - відповідати, за
можливості, мовою того, хто звертається. Пишу першим російською просто тому, що
з усіх мов максимально вільно нею володію. Але, мабуть, не зовсім відповідаю
вимогам, бо в частині "хочу щоб дитина залишалась жити тут" - абсолютно ні, і
хочу щоб дитина мала достатньо знань і вмінь, щоби існувати поза державними
кордонами і обирати місце комфортного існування незалежно від того як
намальована політична карта світу. В контексті мови тому більше переживаю за
знання англійської.

\begin{itemize} % {
\iusr{Наталія Гурошева}
\textbf{Евгений Кузин} Ви вжили слово, яке означає язик як орган , а "мова" англійською - language. Вибачте)

\iusr{Евгений Кузин}
Це усталений англійський вираз, що означає "рідна мова"

\url{https://dictionary.cambridge.org/ru/словарь/английский/mother-tongue}

\iusr{Елена Данкина}
\textbf{Eugene Kuzin} От насправді - вік живи, вік учись) Я теж сіпнулася, а тепер завдяки вам знаю щось нове для себе!

\iusr{Евгений Кузин}

Оооо, англійські ідіоми - це просто прекрасно, і дуже весело ))) Рекомендую
сайт urbandictionary.com, та канал "Fucking English" в Телеграмі - жива мова,
особливо в порівнянні з прямим перекладом слів виразу - дуже весело щось таке
нове дізнаватись.

\iusr{Наталія Гурошева}
\textbf{Евгений Кузин} дякую за науку! Була така підозра...)) Ви іі підтвердили, спасибі, знатиму!

\end{itemize} % }

\iusr{Станислав Кушнаров}
Пишу української. За винятком жартів, коли гра слів виглядає такою саме російською.

\iusr{Лана Рисенко}

Дехто ніяковіє писати українською через те, що у френдах є реальні знайомі, які
знають вас, як російськомовного.

Наприклад, в кулінарній групі до мого френда приколупався його реальний
знайомий типу "а с какіх ето пор, Саньок, ти так на мовє пісать начял? Я етава
нє панімаю, твоя мать к тєбє как абращалась?" Ну і в такому стилі. Я як читала,
то аж кипіла. Але френд дуже гарно і спокійно все пояснив. Не всі витримують
оцих підйопок.

\iusr{Лана Рисенко}

\ifcmt
  ig https://scontent-frx5-1.xx.fbcdn.net/v/t1.6435-9/244448007_1725656014297590_8293952530889128628_n.jpg?_nc_cat=105&_nc_rgb565=1&ccb=1-5&_nc_sid=dbeb18&_nc_ohc=JrJwlfhUWFgAX-QxHW0&_nc_ht=scontent-frx5-1.xx&oh=59b51863e4547e219bb49c8b875dac7d&oe=617FDA6E
  @width 0.4
\fi

\iusr{Ruslan Gorovyi}
\textbf{Лана Рисенко} достойно

\iusr{Valeria Petrova}

Намагаюся писати українською, але певна кількість дописів таки рос. У таких випадках:

1) коли дууууже поспішаю

2) коли щось мене сильно вразило, і є бажання якомога точніше транслювати свої
відчуття, та і взагалі детально розповісти, а українською я буду писати три
години зі словником, і не впевнена, що напишу саме те, що відчуваю

\iusr{Yulia Klochko}

Ще є такий феномен: людина зі мною спілкується українською, а повідомлення пише
виключно російською. Правда, дехто після запитання "чому так?" переходять на
рідну в листуванні.

\iusr{Александр Вектор}

Все розумію, все можу. В розмові деколи незручно, бо голова думає російською і
доводиться підбирати слова. Якщо коротко відповідати, то це звичка. Якщо більш
розлого, то це становиться куди легше, коли навколишнє середовище україномовне.
Якщо ні (а це й досі доволі значний відсоток, менше ніж було, але все одно є),
то відбуваються рецидиви. Мозок сам по собі становиться на старі зручні рельси
і так далі.

Щодо написання російською і чому... Ну поперше таж сама звичка. Іноді не
встигаєш замислитися. Є думка і ти її ізлагаєш. Крім того, деякі речі, як на
мене, російською мені більш реальні (якщо можна так сказати). Але все одно.
Основна проблема, це звичка)


\iusr{Natalia Gnydiuk}

Я можу зрозуміти, чому пишуть пости російською, бо не хочуть писати з
помилками, але отчого взагалі не можу зрозуміти, чому багато хто, хто пише
пости українською з дітьми розмовляє російською.

\begin{itemize} % {
\iusr{Мария Скакунова}
\textbf{Natalia Gnydiuk} бо коли наші батьки з нами розмовляли російською, то і до своїх дітей так. Якщо родина в побуті російською.. Це автопілот. Почати українською хоча б писати, це вже маленький крок до змін  @igg{fbicon.smile} 
\end{itemize} % }

\iusr{Вікторія Валевська}
От зараз читала Бутусова і думала про це. Хоча загалом мені подобається, що він пише.

\iusr{Андрей Свистула}
Буття визначає свідомість.

Нажаль, у повсякденному бутті в російськомовному середовищі і з купою побутових
проблем, проблем зі здоров'ям , роботою і таке інше, іноді, в ситуаціях де те
має принципових ознак, вже не вистачає сил і натхнення перебувати у стані
війни.

Прошу не перекручувати і не додавати моєму коментарю зміст, котрий я в нього не
вкладав

\begin{itemize} % {
\iusr{ірина болгарова}
\textbf{Андрій Свистула} 

навіщо бути у стані війни - просто бути у українському стані.. можна просто у
громадських місцях ,частіше , у магазинах та у фб розмовляти державною..


\iusr{Ruslan Gorovyi}

не зрозумів останній посил... якщо ви переживаєте, що хтось щось не зрозуміє і
перекрутить, сформулюйте так, щоб все було однозначно і зрозуміло...


\iusr{Андрей Свистула}
\textbf{Ruslan Gorovyi}

Гарний приклад наш діалог. Я, людина з російськомовного регіону і майже на 100\%
російськомовного оточення шукаю для себе нішу, де я можу відчути себе українцем
на своїй землі. Для цього я приєднуюся до україномовних спільнот і підписуюся
на україномовних лідерів думок. Але що я маю на виході коли вступаю в діалог чи
ділюся своїми сумнівами? Майже зі 100\% верогідностю я чую у відповідь, що я
недостатньо принциповий, недостатньо наполегливий, що я якийсь неповноцінний
недоукраінець. І все це у середовищі, котрому я не ворог і в якому шукаю сенси
для себе. А що коїться на львівських майданчиках? Там, якщо ти не 770 років
родом зі Львова і в тебе прізвище не Фаріон або Дроздові, то ти маєш тільки
слухати не відкриваючи рота...

А тепер уявіть собі ситуацію, коли ти знаходишся у недружньому середовищі, у
нашому випадку за мовними ознаками. Негативом тебе спалять за лічені дні, може
тижні, може місяці.

Про те я і написав, що в ситуаціях, які не носять принциповий характер,
наприклад у спілкуванні з російськомовними товаришами, використовую російську
не замислюючись про те

\iusr{Ruslan Gorovyi}
\textbf{Андрій Свистула} 

тепер все чітко і зрозуміло. Дякую вам за відповідь. Оскільки я сам був з
такого регіону і середовища, то дуже добре вас розумію.

\end{itemize} % }

\iusr{Віктор Ярмолюк}

У мене більше питань до земляків з Волині, що після переїзду до Києва стали не
тільки розмовляти російською, а й писати свої пости. Не розумію, чому? Це
якийсь прояв совкового комплексу неповноцінності? Поясніть, не розумію.

\begin{itemize} % {
\iusr{Ruslan Gorovyi}

тут комплекс всього,.. на моїй пам'яті дівчина з прекрасною українською в Києві
за 5 років зросійщилася повністю. це було ще до війни.... як зараз я не знаю.

\iusr{Олена Бабенко}
\textbf{Віктор Ярмолюк} 

Моя племінниця навіть не виїжджала нікуди з Волині, блін. 17 років дитині.
Російської навіть у школі не вчила. І при цьому вся її інста в узкоязичних
цитатах штибу ОК/ВК. Звідки це гівно в її голові - не уявляю.  @igg{fbicon.face.woozy} 

\iusr{Віктор Ярмолюк}
\textbf{Олена Бабенко} , 

якщо подивитись тренди Ютубу, то одразу все стає зрозумілим. У мене донька 8р.
недавно питала, чи знаю я про Моргенштерна, бо її однокласники слухають, а там
багато слів "які напевне матюки".. А усі блогери з великою багатомільйонною
аудиторією - російськомовні, тож діти просто тягнуться до того, що їм здається
ознакою популярності та успішності.

\iusr{Катерина Ващук}
\textbf{Віктор Ярмолюк} оо та сама фігня...) мене аж підкинуло як побачила пости однієї землячки

\iusr{Ольга Щербак}
\textbf{Віктор Ярмолюк} 

Щодо Ютубу, то там український контент — це невтомна праця з очищення від
російського. Є такий фокус: здавалося б очистила пропозиції рашаконтенту, але
не дай Боже, навіть українське щось російською подивишся — кришка, лізтиме в
усі щілини. Що цікаво, то ні з англійським, ні з польським, ні з німецьким чи
французьким такого не траплялося ні разу (уточню, французькою не володію,
просто музику якусь слухала). Тому для подивитися щось із підозрілого контенту
у мене на комп'ютері стоїть браузер, де на тому ж Ютубі ніхто не залогований,
щоб не загажувати профіль.  @igg{fbicon.face.rolling.eyes} 


\iusr{Тетяна Бурлак}
\textbf{Олена Бабенко} від блохерів, яких вони дивляться.

\end{itemize} % }

\iusr{Ирина Мастакова}

Записую думку в тому вигляді, як вона виринула у свідомості. Тобто лише
занотовую свій внутрішній голос. Як воно в голові прозвучало - так і буде,
переклад не роблю. Іноді воно українською, а іноді російською. Буває і
мішанина, суржик.

\begin{itemize} % {
\iusr{Виктория Кононова}
\textbf{Ирина Мастакова} 

у мене теж саме так. Іноді у випадках "російськомовної" думки вдається
переключити її на українську, іноді - ніяк слова не підбираються.

Ще помітила, що важко говорити й писати українською, коли дуже втомлена або зпросоння.

\end{itemize} % }

\iusr{Вячеслав Раєвський}

Повністю підтримую, за час війни не скажу що сто відсотків але переважна
більшість російськомовних друзів стали не тільки писатии але і спілкуватись
українською

\iusr{Людмила Храпачевская}

Чому, коли я пишу українською мовою до людини, яка спілкується рузькою, ( до
колеги, до подруги) , я відчуваю провину, що цим її ображаю.... Що робити з
цим?

\begin{itemize} % {
\iusr{Ruslan Gorovyi}
це все говорить про вас, а не про них... вони ж вам не казали, що ображаються?

\iusr{Лана Рисенко}
\textbf{Lyudmila Hrapachevskaya} бо ви відчуваєте себе кращою, ніж вони.

\iusr{Людмила Храпачевская}
\textbf{Ruslan Gorovyi} Я їх не звинувачую. Але ,що зі мною! Чому?

\iusr{Ольга Щербак}
\textbf{Lyudmila Hrapachevskaya} Нас привчали до того, що до тих, хто говорить російською обов'язково переходити на російську, а не роблячи цього Ви наче порушуєте правила. З тої ж причини (точніше її віддзеркалення) деякі російськомовні так негативно ставляться до українізації публічного простору і волають про насилля над ними, хоча ж жоден закон не втручається в особисту сферу — бояться втратити ексклюзивне становище. Просто українцям досі незвично мати право бути українцем в Україні.  @igg{fbicon.eyes} 
\end{itemize} % }

\iusr{Andrii Kryvolap}
Десь з 2016-го став постити солов'їною. Хоча в побуті ще відсотків на 70 спілкуюсь свнсбчою(

\iusr{Natali Blashchyshena}
Мене теж завжди це цікавило, послухаю.
Тим більше тут не катять відмови "боюсь сказать нє правільно" тому що акценту тут не видно, а в гуглі завжди можна глянути потрібне слово якщо вже дуууже хочеться "правильно"

\iusr{Антон Селин}

Язык это личное, его использование не требует пояснений. Исключения прописаны в
языковом законе - это публичные выступления, госслужба и сфера обслуживания,
если я не ошибаюсь.

\begin{itemize} % {
\iusr{Ruslan Gorovyi}
то навіщо ви написали комент? тобто якісь пояснення такі треба?

\iusr{Антон Селин}
\textbf{Ruslan Gorovyi} вы спросили, я ответил. Кажется, в посте было приглашение к дискуссии

\iusr{Ruslan Gorovyi}
\textbf{Anton Selin} я говорив про власний досвід, а ви мені вирішили розповісти про закон.

\iusr{Антон Селин}
\textbf{Ruslan Gorovyi} я ж не про закон писал, а про то, что язык общения это личное, не требующее пояснений.
Проте, коли не очікують пояснень, то зазвичай я переходжу на мову співрозмовника, тобто моя російська це не принцип, це просто моя рідна мова.

\iusr{Ruslan Gorovyi}
\textbf{Anton Selin} дякую. зрозуміло
\end{itemize} % }

\iusr{Оксана Лоза}

Пишу виключно українською , стала помічати , що мені складно іноді знайти
російський відповідник тому чи іншому слову .

\begin{itemize} % {
\iusr{Елена Данкина}
\textbf{Оксана Лоза} Те саме. Але найдивніше коли моя донька, що живе за кордоном і її мови - англійська та іврит, так само в розмові чомусь вставляє українські слова, дуже природньо і невимушено)
\end{itemize} % }

\iusr{Яків Бриль}

Коли ще задовго до війни почав переводити діловодство на українську, дуже
помічним був MS Word з увімкненим жорстким контролем правопису. А от зараз вже
мені значно важче писати російською, аніж українською. Англійською ще важче,
але то таке

\iusr{Натали Натадзе}

У мене багато родичів в Європі і в Грузії, які не розуміють українську

\begin{itemize} % {
\iusr{Яків Бриль}
\textbf{Natali Natadze} ეს ახლა პრობლემაა?
\end{itemize} % }

\iusr{Tatyana Misiyuk}

Звичка перш за все. Бо з російськомовного Запоріжжя (хоча українська була у
бабусі в Умані, тож і з 4 років українською вже читала). А потім освіта
роійською на англійською, робота та спілкування також постійно англійською
була, до Майдану україномовного середовища практично не було. Коли почало
з'являтись, мене за акцент за діаспору вважали). Зараз спілкування письмового
набагато більше, а в житті - ні. Тому звичка залишається, хоча все частіше
стала українською писати, інколи навіть з россійськомвними в месенджері
перехожу. Треба час. Не хочу це робити в вигляді принципу " а от від сьогодні
ніякої російської". Має бути натуральним

\begin{itemize} % {
\iusr{Ruslan Gorovyi}
все має бути так, як вам хочеться. це головне

\iusr{Tatyana Misiyuk}
\textbf{Ruslan Gorovyi} 

ну просто дууууже рідко хтось власне може заявити мені на моїй де сторінці,
мовляв, чого не державною. От тут я на рога встаю. Бо інакше нормально
спілкуюсь українською. Але аби мені не тикали, як мені у себе писати. Реально
дух протиріччя вмикається. А ще у одного друга, у якого глибинне розуміння теми
націі, цікава теорія на тему, що насправді не всім і треба розмовляти
українською, бо це також свого роду маркер. От я, наприклад, точно не можу
сказати, що вишиванка для мене - це мій генетичний код. Бо більше польське
коріння мало значення, українського набагато менше. Тому воно і вкладається в
певну схему


\iusr{Tatyana Misiyuk}
Треба ж не забувати, що дійсно не для всіх українська - рідна мова

\iusr{Ольга Щербак}
\textbf{Tatyana Misiyuk} Ой, леле! Мене уродженку Полтавщини, яка прожила тут усеньке життя регулярно називають западенкою, бо, бачте, не послуговуюся словами: поняв, обідився тощо. Я про що? Хай собі вважають.  @igg{fbicon.wink}  @igg{fbicon.face.smiling.halo} 

\iusr{Tatyana Misiyuk}
\textbf{Ольга Щербак} я і западенців знаю з такими словами  @igg{fbicon.smile}  та чого тут мірятись взагалі?
\end{itemize} % }

\iusr{Oksana Kotich}

як показує практика, українську чудово розуміють навіть самі російськомовні, що
письмову, що розмовну. Більш того, за півроку роботи у переважно
російськомовному колективі мої колеги навчились відповідати мені українською.
Без примусу чи якогось там ультиматуму з мого боку. Просто завжди спілкуюсь
українською.

\begin{itemize} % {
\iusr{Tatyana Misiyuk}
\textbf{Oksana Kotich} ну так мова ж не про діалог, я також українською відповідаю завжди, але особисті дописи - то інше

\iusr{Oksana Kotich}
\textbf{Tatyana Misiyuk} може я просто хотіла похвастатись?

\iusr{Tatyana Misiyuk}
\textbf{Oksana Kotich} ну таке) коли нема примусу і тикання носом. люди поступово звикають використовувати украінську. Бо більшості дійсно не вистачає практики. Кіно. ТБ, Інтернет - це пасивне використання. Тому і те, що в магазинах українською обслуговують - це також на користь і іншим - звичка відповідати українською поступово у багатьох з;являється
\end{itemize} % }

\iusr{Anna Radysh}
Іноді - розраховуючи на френдів з інших країн, що російською володіють, а українською - ні.

\begin{itemize} % {
\iusr{Олександр Рощин}
\textbf{Anna Radysh} для цього є перекладач. Мене читають багато людей з інших країн що української не розуміють

\iusr{Лариса Прядкіна}
\textbf{Олександр Рощин}
Перекладач настільки спотворює текст і зміст, що мої рідні досить часто мене не розуміють зовсім.
Так перекладають, що іноді лишається тільки покрутити пальцем біля скроні.
Хороших автоперекладачив мало, у мене є один адекватний, але в ньому нема української.
Виникає друге питання: чому в українському сегменті ФБ досі нема суто українських адміністраторів та модераторів?
Від кого це залежить?
При вирішенні цього питання, певно, і перекладачи покращалися б.
\end{itemize} % }

\iusr{Андрей Свистула}

Руслане, кожного літа їдемо з сім'єю в Тернопільську область у відпустку (теща
переїхала туди в 14му з Донецька). Намагаюсь якнайбільше спілкуватися з
місцевими, придивляюся і прислухаюся. Іноді волосся на голові шевелиться:

1. Стільки шансону, як там, в Дніпрі в транспорті не почуєш.

2. Ми ходимо до церкви МП тому, що туди завжди ходили наші батьки.

3. Отам, між Кременцем і Збаражем проходить кордон, котрий розділяє галичан і
волинян. Якщо щось і буде, то ми повернемося під Польщу і будемо жити як і
раніше до 1939го року.

4. Ми, тут у Тернопіллі недолюблюемо Львів'ян, тому що вони були під австріяками...

От таке я чую в землях, котрі мають буди колискою всього українського.

І я уявляю відношення тих людей до мене, українця зі сходу. Відкрито, багато
хто з заходу України готові відмовитися від своїх східних співвітчизників разом
з тими ватанами.

Мої спостереження дають підстави мені вважати, що в нас досі не сформульована
національна ідея, котра дає поняття, що нас об'єдную і поєднує.

Але все і не так погано, як могло бути

\begin{itemize} % {
\iusr{Ruslan Gorovyi}

ми всі вражені імперією. хтось більше, хтось менше. хтось чинить опір, хтось
ні... хтось мав змогу говорити українською з дитинства, а хтось повертається
свідомо... такі часи.


\iusr{Оксана Атаманчук}
\textbf{Андрій Свистула} 

є різні люди з різними думками)і все,... можливо ви думаєте що ці землі МАЮТЬ
БУТИ колискою.. та я думаю що ці землі і Є колискою, і не тільки ці землі ,вся
наша земелька є колискою)))бурянчику просто багацько, тому ,полоти-сапати))))і
все буде добре),ну я так думаю))

\end{itemize} % }

\iusr{Валерий Шамрай}
Вчора видалив трьох ,,друзів",яких знаю особисто, за листівки російською мовою, надоїли.
А можливо вже нерви.

\begin{itemize} % {
\iusr{Ruslan Gorovyi}
якщо ви людей не знаєте, то нема чого й жаліти. якщо ж знаєте - тут ви самі вирішуйте. це ваша зона комфорту

\iusr{Валерий Шамрай}
\textbf{Ruslan Gorovyi} Просто я їм натякав, що таких картинок в інтернеті на українській мові повно. Мабуть какая разніца.
\end{itemize} % }

\iusr{Olga Malyshko Demydova}

якщо не заради срача, а заради дослідження, то - я не знаю. інколи слова просто
так складаються у речення - російською. інколи німецькою. інколи англійською.

\iusr{Євгенія Гуржій}
І мені цікаво!
Поділюсь)

% -------------------------------------
\ii{fbauth.vorobchenko_svjatoslav.kiev.ukraina}
% -------------------------------------

1. Привычка. Всю жизнь жил в русскоязычной семье. Русский знаю лучше
украинского.\par Переучиваться-доучиваться лень. Хотя иногда бывает, что пишу на
украинском. Но пока что это исключение из правил, и больше, как уважение к
собеседнику, или если из пени слов не выкинешь, или если есть настроение. Почти
всегда проверяю свои тексты на отсутствие ошибок (не хочу, чтобы проскакивало
что-то типа "останівка"). Уже если писать, то писать правильно. На это нужно
время и желание.

2. Не дорос ментально.


\iusr{Oksana Ruban}
Патамушо какаяразніца.

\iusr{Петр Москалец}

Часто роблю репост інформації мовою оригіналу(московитською), бо вважаю більш
актуальним зміст, а не форму. Багато шанованих авторів пишуть московитською, і
перекладати не бачу сенсу. По великому рахунку, мова, це засіб спілкування, і
від людини залежить як цей засіб використовувати. Яби теж хотів, щоб в Україні
була українська мова, але в найближчому майбутньому так не буде. Може через
покоління.

\iusr{Наталия Чемеркина}

Вік, коли я запоєм читала книжки, прийшовся на тоді, коли українською ще не
говорила. І, ясно що, мовні обороти, синоніми слів, що дозволяють потім якомога
чіткіше донести свої думки - все це було завантажено з російських книжок. Тому
і пишеться російською легше та яскравіше. Українською складніше писати, але
заставляю себе не відступати

\iusr{Ирина Добрянская}
Пишу українською , але можу відповісти росіянською- якщо , наприклад, такою мовою поздоровили.

\iusr{Nina Nizka}

Я російськомовна і пишу російською, багато друзів і знайомих (в тому числі,
близьких) з країн колишнього СРСР, які не знають української. Так склалося, що
українську я в школі не вивчала (навіть англійську почала вивчати раніше за
українську), але дуже стараюся і намагаюся її вивчити якнайкраще.

\iusr{Юля Музика}

теж неодноразово задавалась цим питанням... особливо бісить коли по життю
україномовна особистість, а пости і коменти викладає російською @igg{fbicon.face.eyebrow.raised} 

\iusr{Таня Пилипець}
Досліджуй, бро

\iusr{Таня Пилипець}
То надважливо

\iusr{Оксана Підопригора}
В інстаграм так роблять для збільшення аудиторії. Тобто питання мови побоку, головне, щоб лайків побільше було

\iusr{Svetlana Pavlichenko}

Саме не розмовляти українською то напевно є звичка. Бо в російськомовному
Дніпрі так було завжди. А пости пишу виключно українською, бо прийняла для себе
таке рішення ще у 2014-му. В громадських місцях спілкуюся українською та й в
містах до яких подорожую та які україномовні. А в родині, на роботі та зі
знайомими звичайно важно перейти на українську.

\iusr{Юлия Коляда}

Пишу вже давно українською і страшенно це подобається. Подобається, що під
моїми дописами в коментарях ті, хто пишуть російською, переходять на
українську.


\iusr{Ксеня Мальована}
Я якраз спочатку пости почала писати українською. Потім вже перейшла повністю.

\iusr{Максим Мошковский}

Десь рік тому врешті перевів свій паблік з мемами на українську і вбив купу
зайців)По-перше відвалилася купа заліщан і ватоти і якість підписників
виросла)По-друге і кількість підписників виросла, хоча я думав, що буде значна
просадка і велика кількість скарг в інсті від ватоти)

\iusr{Oksana Velit}

Пишу на русском, потому что мне так удобнее. И потому что мне претит идея
подстраиваться под тех, кто пару-тройку лет назад сам освоил украинский и на
этом фоне приболел синдромом вахтера.

На комментарии, написанные на украинском, отвечаю тоже не украинском. В
реальной жизни тоже перехожу на украинский с украиноговорящими людьми (если они
мне не неприятны).

\iusr{Наталія Гурошева}

Пишу московською лише відповіді тим людям, яких щиро поважаю і які при цьому
погано володіють (чи зовсім не володіють) украінською

\begin{itemize} % {
\iusr{Igor Tsys}
А я пишу вражинською тільки у вражинських пабліках, щоб якомога більше ворогів зрозуміли, що я їм не брат.
P.S. І то , пишу через і, бо їхньої 'и'нема в телефоні принципово.
\end{itemize} % }

\iusr{Оксана Масалітіна}

Гаразд, я спробую пояснити особисто своє.

Це якось пов'язане для мене із моєю власною ідентичністю.

Я не говорила українською до 2014 року. Взагалі. Не вчила ніде окрім короткого
базового курсу в університеті дуже давно.

За ці останні роки я пишу у фб майже завжди коментарі українською, наче навіть
не дуже багато помилок. Можу годинами спілкуватися у приватних розмовах
письмово та не дуже кульгаво у офлайн, живих, розмовах. Більше того, у мене
навіть є деякий комплекс меншовартості і я стидаюся говорити російською з
україномовними співрозмовниками. Також я завжди спілкуюся українською у
магазинах, кафе, та т.ін.

Мене нудить від к.ц..кого кино, музики, акценту та будь-якого іншого чогось
характерного для рашки. Відвернуло від усього, що колись подобалося, що я
слухала, дивилася та читала. І нудить від цього більше ніж когось іншого, бо я
сама звідти, я народилася там та прожила до 18 років, їздила туди у гості, там
досі мої батьки та інші родичі, з якими, окрім батьків, я вже давно зовсім не
підтримую зв'язок.

Але.

Але свої пости у фейсбуці я пишу зазвичай російською. І не перехожу на
українську у побуті. Я не блогір та не пишу чогось цікавого на загал. Мої пости
особисті про особисте. І якщо я пишу їх українською - то це наче не я, а якась
інша людина написала. Або я граю роль когось іншого.

Перепрошую що так багато слів, зачепило.

\begin{itemize} % {
\iusr{Анатолий Удовиченко}
\textbf{Оксана Масалітіна} 

От що цікаво, коли я читаю твої дописи українською, то читаю ту саму Оксану,
яку знав і з якою спілкувався раніше тільки рузькою... Тобто, ніякої "іншої
людини" немає...

\iusr{Ruslan Gorovyi}
дякую

\iusr{Оксана Масалітіна}
\textbf{Анатолій Удовиченко} 

Толю, розумієш, це ж ти про сприйняття мене іншими людьми, а я про власні
відчуття.

І ще тут є така, може не дуже доросла, образа - чому от я маю свою мову
викинути на смітник та соромитися її відтепер назавжди, коли я і так вже майже
усе, пов'язане з тією країною викинула та зпалила. А наприклад моя колега на
роботі, зі Львова родом - у цей же час розказує мені українською мовою
(безграмотно, доречі), як "У расіі взагалі-то прикольно жити, ось один знайомий
поїхав та нічо так, каже"


\iusr{Анатолий Удовиченко}

Ну, я так думаю, тут більше питання не до тих, хто не перейшов з рідної
російської на українську, а до тих, хто ту російську використовує чи не вперше
в житті і саме в пабліку інтернету. Та ще й з такими помилками, що очі
сльозяться.  @igg{fbicon.smile} 

\iusr{Оксана Масалітіна}
\textbf{Анатолій Удовиченко} так, я це вже потім, по коментарях, зрозуміла

\end{itemize} % }

\iusr{Вадим Зябліцев}
Теж завжди цікаво. Навіть в житі повністю україномовні, пишуть москальською..

\iusr{Анна Лисовская}
В мене буває так, що іде саме українська, а буває, що російська. Поки що більш російська...

\begin{itemize} % {
\iusr{Igor Tsys}
Воно видно по написанню хвамілії, що саме іде....

\iusr{Анна Лисовская}
\textbf{Igor Tsys} а що саме вас турбує в написанні мого прізвеща?)))

\iusr{Ruslan Gorovyi}
\textbf{Igor Tsys} я вас дуже прошу. перечитайте пост і моє прохання. наразі ви єдиний які пішли в комети з образами... мета не виправдовує засоби. Тут немає ворогів.
\end{itemize} % }

\iusr{Olena Bondarenko}

Для себе вирішила просто не лайкати пости, написані російською, про що вони не
були б. Маю російськомовних друзів. І часто рука зависає між підтримати пост
лайком і мовною позицією. Складно.

\begin{itemize} % {
\iusr{Ruslan Gorovyi}
\textbf{Olena Bondarenko} розумію

\iusr{Євген Косяненко}
\textbf{Olena Bondarenko} таке саме

\iusr{Olena Bondarenko}
\textbf{Євген Косяненко} розумію
\end{itemize} % }

\iusr{Олена Мазій}

Все\_життя\_російськомовна, у 2014 перейшла на українську не тільки у написанні,
а й у розмові, виступах на аудиторію.

Було нелегко, але я зробила відразу те, що могла зробити саме я, щоб показати,
що Україна є.

І я не переходжу на російську ні з ким, ні з родичами з Білорусі, ні з росії, і
вони мене чудово розуміють.

Пане Руслане, ви поставили дуже правильне питання, мене теж це цікавить.

\begin{itemize} % {
\iusr{Oleksandr Masliukivskyi}
\textbf{Олена Мазій} Родичі в рососії мали б бути залишені ще в тому ж 2014му. Нема у вас там родичів. То вороги.

\iusr{Анна Лисовская}
\textbf{Oleksandr Masliukivskyi}  @igg{fbicon.man.facepalming} 

\iusr{Олена Мазій}
\textbf{Oleksandr Masliukivskyi}
Пане Олександре, я не просила поради, я з своїми родичами сама розберуся.
\end{itemize} % }

\iusr{Анатолий Удовиченко}

Теж таких багато знаю. В житті АБСОЛЮТНО україномовні, вилазять в інтернет - і
прориває. Мабуть, комплекс меншовартості. Стараюсь з такими менше контактувати.
У місцевій групі не раз розгоралися "срачі" з цього приводу, але віз і нині
там. Як пісалі так і пішут...


\iusr{Ігор Чичкань}

Ок, відповім (хоча зазвичай уникаю теми використання мов).

Пишу своєю рідною російською, бо володію нею значно майстерніше, ніж вивченою
українською, бо вважаю, що можу нею гратися словами та сенсами більш цікаво,
ніж робив би те після перекладу в голові. Бо мій рос.вокабулярій значно ширше
ніж укр.

От десь тому.  @igg{fbicon.face.smiling.sunglasses} 

\begin{itemize} % {
\iusr{Ruslan Gorovyi}
зрозуміло

\iusr{Ігор Чичкань}
\textbf{Ruslan} звісно, що у спвлкуванні з україномовною людиною використовую українську зазвичай. Бо це вважаю правильним.

\iusr{Tatyana Misiyuk}
\textbf{Igor Chychkan} от 100\% так само. Хоча впевнена, це таки питання постійної практики.

\iusr{Ігор Чичкань}
\textbf{Tatyana} можливо. Втім це вже хай буде справою більш молодих людей, мені вважаю дещо запізно...  @igg{fbicon.smile} 
\end{itemize} % }

\iusr{Илья Соколовский}
\textbf{Igor Chychkan} 

так все ж можна вивчати і тренувати. Звісно, легко не буде, треба витратити час і зусилля, для опанування нових виразів і лексем.
Теж до певного часу був такої точки зору, бо виріс в російськомовному середовищі. Але відповіді на питання хто я і самоідентифікація розставила все на свої місця!
Немає нічого неможливого було би бажання!

\iusr{Igor Tsys}
Вокабулярій, кажеш, широкий...

\iusr{Ігор Чичкань}
\textbf{Илья} а ще я лінивий))))

\iusr{Илья Соколовский}
\textbf{Igor Chychkan} власне про це іде мова!

А ще ж треба всім знайомим пояснити, чому ти раптом вирішив перейти на
українську. А раптом хтось буде задавати питання, або взагалі відреагує і
перестане з тобою спілкуватись.

Тут тільки варто усвідомити, що розмовляючи московською ти підтримуєш
(опосередковано) їх мову, їх культуру, їх світ, а вони потім приходять в цей
світ ногами своїх солдатів, захищати тебе! навіть якщо ти про це їх і не
просив.

\iusr{Анна Лисовская}
\textbf{Ігор Чичкань} вот в мене також) але не змогла навіть так сформувати @igg{fbicon.monkey.see.no.evil} 

\iusr{Ігор Чичкань}
\textbf{Илья} власне, кремлівським зольдатам вистачить наказа аби прийти будь-куди (крім рос.мови), але я ніколи не погоджусь з твердженням, що копірайт на русскій язик раптом опинився у рашистів. Бо це не так.

\iusr{Вуйко Місь}

Знецінення себе ,як представника українського народу , відсутність
елементарного розуміння , що послуговуватися чужою мовою і відкидати зневажливо
свою, властиво дурням ,або свідомим зрадникам.

\iusr{Sviatoslav Pashchuk}
Бо дибіли.

\begin{itemize} % {
\iusr{Ruslan Gorovyi}
\textbf{Sviatoslav Pashchuk} )) поговорили

\iusr{Sviatoslav Pashchuk}
\textbf{Ruslan Gorovyi} вибачайте, куме, накипіло)

\iusr{Святослав Воробченко}
\textbf{Ruslan Gorovyi} 

кстати, вот еще одна из причин, почему я говорю и пишу по-русски. Когда меня
называют дебилом или предателем только потому что я использую в повседневной
жизни русский язык, меня точно не побуждает переходить на украинский.

Хочется сказать таким патриотам, "идите на ...".


\iusr{Olexander Kononov}
\textbf{Святослав Воробченко} цікава позіція, дуже схожа з "назло всім відморожу вуха"! Вибачайте але саме так сприйняв коментар

\iusr{Святослав Воробченко}
\textbf{Olexander Kononov} 

воспринимайте, как хотите.

Просто попытайтесь представить, что Вас кто-то называет дебилом или предателем,
или еще как-нибудь только потому что Вы делаете что-то не так, как считает
нужным кто-то.

\iusr{Ruslan Gorovyi}
\textbf{Святослав Воробченко} 

що кажуть інші це їхня справа. Мені, наприклад, закидають «та чьо ти
вийобуєшся, гаварі как раньше»... на мене це не впливає, бо я розумію що і як зі
мною відбувається і чому


\iusr{Ruslan Gorovyi}
Чуваки, припиніть ото. Бо з вами хрєн шо з’ясуєш для науки

\iusr{Святослав Воробченко}
\textbf{Ruslan Gorovyi} 

просто с такими "друзьями" и враги не нужны.

Друг друга готовы убить за "инакомыслие". Делать выводы о человеке только на
основе того, на каком языке он разговаривает и пишет, как по мне, за пределами
добра и зла.

А для науки я написал, что это еще один пункт - "баба яга против".

Хотя, наверное, это больше касается не своих постов, а комментариев. Там
основной срач. В моей ленте такого нет, потому что она на 90\% бытовая.

\iusr{Olexander Kononov}
\textbf{Святослав Воробченко} 

Мене дібілом і скатіною називають саме російськомовні совки за те що я пишу
українською мовою. І ще ніхто з свідомої україномовної спільноти не назвав мене
зрадником хоча я на дописи українською мовою перейшов відносно недавно,
спочатку частково а роки чотири вже повністю. І це при тому що я є
російськомовний росіянин з Донбасу і всі це чудово знають і розуміють! Я навіть
в російській історичній групі принципово пишу українською мовою. Хоча вже
навіть не принципово бо маю тільки українську та англійську розкладку
клавіатури! Знаєте, це дуже зручно, відсіках зайве!


\iusr{Ruslan Gorovyi}
\textbf{Olexander Kononov} ти просто танк, Сань... а в світі є й інші види зброї

\iusr{Olexander Kononov}
\textbf{Ruslan Gorovyi} сєпари Термінатором прозвали! Мо'' не помилились?

\iusr{Святослав Воробченко}
\textbf{Olexander Kononov} 

я понимаю. Но в данной ветке я отписался именно потому что пару человек
написали про дебилов и предателей.

Это проблемы тех, кто так пишет.

Я же написал в первую очередь Руслану, как топикстартеру, задавшему вопрос, что
одна из причин - нежелание подстраиваться под таких "патриотов", делящих людей
по языковому признаку. Да, похоже на "назло бабушке отморожу уши". Просто я
считаю, что каждый волен говорить на том языке, на котором хочет, и никто не
вправе судить людях по языку общения. А настоящий патриотизм проявляется в
поступках.

\ifcmt
  ig https://scontent-mia3-1.xx.fbcdn.net/v/t1.6435-9/244390449_4717793438287501_7245770746245738478_n.jpg?_nc_cat=111&_nc_rgb565=1&ccb=1-5&_nc_sid=dbeb18&_nc_ohc=oYjLkwV0IzUAX-F0cqt&_nc_ht=scontent-mia3-1.xx&oh=a8611eea9c79c7727c0d119e40f30752&oe=61805920
  @width 0.4
\fi

\iusr{Olexander Kononov}
\textbf{Святослав Воробченко} 

я не ділю людей по мовній ознаці і не висуваю претензій за російську мову бо то
є моя рідна родинна мова. Але в спілкуванні в фейсбуку дописи російською
сприймаю як ворожи. Але тут ще впливає специфіка мого регіону - в місцевих
луганських групах українською пишуть виключно свідомі патриотичні Громадяни, а
російською широкий спектр від какаяразніци до сєпарів і відвертих ворогів та
зрадників!

Ну й спілкуючись російською ти несеш наратив руssкого міра і розповсюджуєш саме
його, а спілкуючись українською ти стверджуєш Україну! І це так само вагомі
вчинки як і захист Ненькі з калашматом в руках! Так що щиро раджу переходити на
українській бік!


\iusr{Святослав Воробченко}
\textbf{Olexander Kononov} Вы говорите, что не делите людей по языковому признаку и тут же предлагаете мне переходить на украинскую сторону?

\iusr{Святослав Воробченко}
\textbf{Olexander Kononov} 

я выше уже написал, что я родился и вырос в русскоязычной семье. Папа украинец
(Донецкая обл.), мама русская (Смоленская обл.)

Вырос в Украине. Закончил военное училище в Киеве. Потом служил в Монголии,
Бурятии, Москве.

В 1995 году вернулся в Киев и с тех пор живу здесь.

В повседневном общении пользуюсь русским языком. У меня были попытки перейти на
украинский, но не хватило силы воли. Очень быстро возвращался к русскому, как к
более привычному.

Патриотом себя не считаю.

Для меня основное в жизни - семья.

Безусловно, война на востоке Украины - это война с Россией. Крым безусловно -
Украина.

Об участии моей семьи и меня лично в этой войне предпочитаю не
распространяться.

\end{itemize} % }

% -------------------------------------
\ii{fbauth.sugonjako_alina.kiev.ukraina.zhurnalist.suspilne_movlennja.stopfake}
% -------------------------------------

Почала переходити на пости українською поступово. Спочатку їх було 20
відсотків, потім 50, зараз, мабуть, 80-90 відсотків. Так як я від початку
російськомовна, то під час сильної емоції скоріше писала пост російською. Коли
от зараз нахлинуло, пре, емоційно пишу. Бувало, що є час перечитати,
перекласти, але відчуваю, що від того або змінюється емоційне забарвлення,
влучність. Тобто перекладати треба довше. Зараз такого вже менше. У мене
виробилася звичка, беру смартфон, заходжу в фб і думки линуть українською. Ну і
плюс я все ж таки не вважаю, що у нас всі біди через російську мову. Ставлюсь
до неї добре.

\begin{itemize} % {
\iusr{Tatyana Misiyuk}
\textbf{Аліна Сугоняко} у мене навпаки зараз, коли емоція, то короткий допис краще українською, бо в певних сенсах українська виразніша. А от щось розлоге, роздуми, або просто "а поговорити"- російською
\end{itemize} % }

\iusr{Тетяна Шидловська}

З 2014 року встановила скрізь український інтерфейс і в публічній площині стала
писати українською власні дописи і коментарі в соцмережах. В приватному
спілкуванні з друзями і знайомими ще багато російської, бо нею володію краще та
й звичніше.

\iusr{Лариса Криштапович}

Полжизни прожила в Латвии. Учила латышский, на украинском (с диким акцентом)
читала только Кобзаря, когда в школе по литературе проходили. На украинский
пытаюсь перейти с 14, благодаря Руслану и его книгам. На улице стараюсь в
общении тоже перейти на украинский. Но словарного запаса и знания орфографии не
хватает. Посты получаются сухими и малоинтересными. По работе пишу посты
по-украински, но сначала по-русски, потом перевожу. Не получается быть "внутри
языка". Наверное, уже возраст.

\iusr{Любов Ломакіна}

Бо "какаяразніца". Бо мої односельці, в житті суржик, в інтернеті виключно
російська - залишилась совкова меншовартість.

Немаю російської навіть в телефоні, не встановлювала.

Навіть в росгрупах дописую українською, нехай тлумачать, вони ж не переймаються
чи я розумію російську, чи ні. Мене турбує, чому реальні патріоти, які своє
життя вкладають в незалежність і ідентичність України говорять і дописуюють
російською, на них рівняються інші, прості.

Мала таке питання до одного відомого волонтера-політика, якому допомагала по
можливості; два слова "Чому російською?", бо добре володіє українською - була
послана далєко і надолго і заблокована.

\iusr{Тома Коваль}

школа російською, інститут російською, дуже багато прочитаних книжок
російською, більший словниковий запас, думаю російською і легше добирати слова,
щоб передати емоції

\iusr{Ольга Выросткова}

Не знаю  @igg{fbicon.eyes}  для мене це відкрите питання вже багато років.

Я з тих самих, навернутих. Мову не знала, не вчила і навіть не чула, поки не
переїхала до України "взамуж"  @igg{fbicon.face.happy.two.hands} 

В березні 2014-го почала несміло, але безповоротно, хоч і з помилками, писати
українською. І потроху розмовляти (думаю "на мові" давно, як не дивно)))

Вдячна всім, хто терпив/терпить мої помилки. І окрема подяка тим, хто перейшов
до українського сегменту, розуміючи, що різниця є! Я вчуся у вас, наслідую,
підглядаю, беру на озброєння якісь прислів'я, звороти, говірки та балачки.

\iusr{Светлана Филипченко}

Никак не поменяю телефон, на моем нет украинской раскладки, но уже приняла для
себя решение перейти сразу, как только куплю новый.

Скачать и установить сейчас не получается, телефон еле дышит уже)))

\begin{itemize} % {
\iusr{Ludmila Hryniv}
\textbf{Svetlana Filipchenko}
Рос .шрифт дозволяє писати українською. Замість І та Ї ставите 1 . Замість Є Э
\end{itemize} % }

\iusr{Полина Кривцова}

По-перше - звичка. Повністю російськомовне оточення. По-друге - чомусь незручно
перед тими, хто давно знає мене російськомовною. Може дурна причина, але це як
вийти з зони комфорту, взагалі цим психологи займаються і не всім легко
дається. Поступово все буде.

\iusr{Іван Федоришин}
Бо багато філологічних задротів, які виправляють кожне слово.

\begin{itemize} % {
\iusr{Helgis Gerus}
\textbf{Іван Федоришин}
задротів
*швидко тікає  @igg{fbicon.smile} 

\ifcmt
  ig https://scontent-mia3-2.xx.fbcdn.net/v/t1.6435-9/244462470_829528867728117_1348105801017162957_n.jpg?_nc_cat=110&_nc_rgb565=1&ccb=1-5&_nc_sid=dbeb18&_nc_ohc=0YL_k-pSgnYAX9u7ZAQ&_nc_ht=scontent-mia3-2.xx&oh=f709f0ebcce41eb94cf2338ad1f33b58&oe=6181E3DB
  @width 0.4
\fi

\iusr{Іван Федоришин}
\textbf{Helgis Gerus} ну тут просто опечатка, а не діалектизм)) Дякую

\iusr{Helgis Gerus}
\textbf{Іван Федоришин}
Та я жартую, я зрозумів  @igg{fbicon.smile} 

\iusr{Александр Котенко}
\textbf{Іван Федоришин} а ви такі налякані та сором'язливі?  @igg{fbicon.face.tears.of.joy} 

\iusr{Іван Федоришин}
\textbf{Olexandr Kotenko} ну особисто я знаю, що відповісти їм, тай завжди українською спілкуюсь) Але навіть якийсь футболіст недавно писав, що уникав української щоб не засміяли))

\end{itemize} % }

% -------------------------------------
\ii{fbauth.belikova_alla.golaja_pristanj.herson_obl.ukraina}
% -------------------------------------

Я з півдня і живу в селі, маю багато односельчан в друзях на фб. В житті всі
говорять звичним суржиком. А ось паздравляют в фб один одного виключно
російською. Це при тому, що російської мови в нашій сільській школі немає вже
25 років (це багато говорить про граматику тих паздравлєній). Люди не читають
нічого, окрім мемів, не дивляться нічого, окрім 1+1 і їхніх серіалів, просто не
знають жодної мови: ані української, ані російської і користуються словниковим
запасом із пісень Потапа

\begin{itemize} % {
\iusr{Марина Ніколаєнко}
\textbf{Алла Бєлікова} на мене дивляться в селі,як на диво,коли кажу- кава,або цукор,або борошно. І теж таке ,суржик,а у інтернеті майже всі російською спілкуються.
\end{itemize} % }

\iusr{Олена Семибратова}

Давайте розмовляти і писати українською! Бо хто, як не ми!?? Ми ж не хочемо,
щоб мова зникла. Ми ж українці, якщо ми не розмовлятимемо, то мови не буде.

\iusr{Тапа Тапа}

Одного дня я вирішила що писатиму тільки українською. Взагалі. Всім і всюди у
всіх месенджерах, соціальних мережах, письмово. Вже й не пам'ятаю скільки це
2-4 роки... розмовляю українською не так як пишу - костурбато, як можу, але ж
розмовляю і намагаюся втягнути в це якомога більше людей.

\iusr{Yurco AciMov}

Ще варто сказати, що багато українців почали уникати, зменшувати, банити, абощо
дописи кцпською, бо гігієна. Якщо принципових чистюль стане більше, ледачі теж
почнуть писати українською.

\iusr{Vlada Yaroslavska}

Я володію обома мовами, а пишу або коментую тією, якою думаю в даний конкретний
момент. Це не питання якогось "усвідомленого рішення", це виходить автоматично.
І: так, моя перша мова російська, українську знаю тільки зі шкільного віку.

\iusr{Hanna Sokolovska}

Є люди які ніби розуміють що паління, чи алкоголь, шкідлива їжа, токсичні люди,
токсична мова - їм шкодять.

І що «расійська» шкодить не лише їх особистому здоров’ю і психіці а й усій
Українській нації.

Кажуть що варто було б покинути.

Але продовжують і нічого не змінюють, (або типу для відмазки десь скажуть щось
українською, для замилювання очей своїй же совісті і знову скочуються до
лекшого, звичного, пригрітого, хоч і токсичного руського)

А на словах ніби й хочуть все це змінити.

Це схоже із симптомами алкоголізму, не помічаєте?

Тобто це «мовний алкоголізм» виходить?

Перш ніж полетять помідори скажу, що я теж розмовляла російською, і теж була
таким собі мовним залежним.

Здавалось що так лекше.

Загаваріть було легше ніж промовити.

Але насправді це тупо виправдовування перед собою і іншими.

Бо коли я усвідомила, що це не правильно то перестала.

В один момент. Бо спротивила це в собі. Зрозуміла що ця зараза руська в мені, і
я ще й перенощиця її іншим здоровим Українцям. Я набралась сміливості
противитись цьому злу в собі

І поборола дуууже швидесенько.

І дуже вдячна людям котрі мені допомогли це зрозуміти, а не чекали ще хуліон
років доки до мене це може колись само дійде. Не толерували мені. І правильно,
я їм дуже вдячна.

Бо ця звичка сама собою не проходить, треба вчитися і працювати над собою,
будувати як своє тіло, свій мозок, інтелект, свою людяність, так і свою мовну
культуру.

Ти або будуєш це все або течеш по течії як те що не тоне.

Хто лінується сам для себе той психологічно слабка людина, вона не може себе
контролювати і робити те що хоче насправді а робить те що виходить. Вона слабка
як і її гідність.

І навіть якщо для когось це звучить грубо то вибачте, але правда є правда.

Та це ж можна змінити.

Просто встань, зміни себе і більше не виправдовуйся ні перед собою ні перед
іншими чому у тебе не виходить.

Це не так важко як здається!

\begin{itemize} % {
\iusr{Ruslan Gorovyi}
дуже цікаво. дякую

\iusr{Людмила Марченко}
\textbf{Hanna Sokolovska} Дякую Вам! @igg{fbicon.face.happy.two.hands}  @igg{fbicon.heart.sparkling} 

\iusr{Hanna Sokolovska}
\textbf{Людмила Марченко}  @igg{fbicon.heart.orange} 
\end{itemize} % }

\iusr{Tetyana Rozumenko}
підпишусь на коменти. теж цікаво. я не розумію і не читаю москвомовних текстів.

\iusr{Nikolai Karetzky}

Вітаю, Ruslan. Три роки живемо в Україні, якась "не така щелепа" вживати мову
не заважає (до цього жодного слова українською не знав). Вивчаю мову, дивлячись
ТБ, читаючи книжки та новини, спілкуючись з людьми на всіляких заходах, з
друзями тощо.

Листуватися та спілкуватися українською в Україні/з українцями вважаю
правильним ставленням до громадян и країни, демонстрацією позиції відносно
росії-ворога, а також повагою, по-перше, до загиблих українців.

* Зараз писав без ґуґлоперекладача, тому роби висновки. Загалом, тут важливіше
бажання. Саме бажання, це як кинути палити, наприклад.

\iusr{Helgis Gerus}
Цікаве дослідження Русік тут у коментарях організував  @igg{fbicon.smile} 

Дуже показове.

Підпишуся на коментарі.

Щодо себе, я вже колись писав - я зросійщений трішки менше ніж повністю.

Я ДУМАЮ запорєбріковою.

Вільно розмовляти не можу - хоча і тренуюся.

АЛЕ!

Коли пишеш, є час подумати, підібрати слова, покопирсатися у словниках, тобто
фактор часу не тисне  @igg{fbicon.smile} 

Якимось "переломним моментом" було те, що ми створили досить популярну свого
часу групу "Підлога країни", де розбирали фейли путінутих хрюкогавців, які
вважали, що досить увімкнути Гуголь-транслейт, і ми "адіннарот". Тому і
з'являлися перли, на кшталт "Донбас хоче світу"

Зрештою виявилося, що усі модератори і адмін - російськомовні.

Тому поставили питання - треба хоча б у групі переходити на українську, бо
запорєбрікомовна українська лінгвістична група - то якось "па-дебільнаму"  @igg{fbicon.smile} 

Ну, ось так і став писати, а побіжно назбирав архів "Підлоги країни"  @igg{fbicon.smile} , як на
скріні - де Гуглоль перекладач фейлиться повністю.

\ifcmt
  ig https://scontent-mia3-2.xx.fbcdn.net/v/t1.6435-9/244534821_829538867727117_4319243177294374943_n.jpg?_nc_cat=109&_nc_rgb565=1&ccb=1-5&_nc_sid=dbeb18&_nc_ohc=fk3a1p3ZK0kAX_GR8kJ&_nc_ht=scontent-mia3-2.xx&oh=b110948e4edd133155fd7a45ae1657a3&oe=6180DA9C
  @width 0.7
\fi

\begin{itemize} % {
\iusr{Valeriia Voronetska}
\textbf{Helgis Gerus} це не перекладач, а шкідник якийсь і зрадник!

\iusr{Helgis Gerus}
\textbf{Valeriia Voronetska}
Та отож.
Треба здобувати вище утворення, щоб зрештою нам усміхнулась частка  @igg{fbicon.smile} 

\iusr{Helgis Gerus}
\textbf{Valeriia Voronetska} , а у вищому утворенні повинен бути окремий роздягнув для "Підлоги Країни", бо одразу без чогось алкогольного гарної фортеці не розберешся  @igg{fbicon.smile} 
\end{itemize} % }

\iusr{Valeriia Voronetska}
\textbf{Helgis Gerus} це не перекладач, а шкідник якийсь і зрадник!

\iusr{Helgis Gerus}
\textbf{Valeriia Voronetska}
Та отож.
Треба здобувати вище утворення, щоб зрештою нам усміхнулась частка  @igg{fbicon.smile} 

\iusr{Helgis Gerus}
\textbf{Valeriia Voronetska} , а у вищому утворенні повинен бути окремий роздягнув для "Підлоги Країни", бо одразу без чогось алкогольного гарної фортеці не розберешся  @igg{fbicon.smile} 

\begin{itemize} % {
\iusr{Ruslan Gorovyi}
мова не про комети ж, а про власні пости

\iusr{Gennady Pimakhov}
\textbf{Ruslan Gorovyi} , тільки якщо то відверто московитомовний ресурс... але я там й не буваю, бо не бачу сенсу... скоріш піду на англомовні ресурси, з відповідними мовними засобами...
\end{itemize} % }

\iusr{Fill Feaouill}

тому що якби не автокоректор, то я був би еталоном безграмотності (

\begin{itemize} % {
\iusr{Ruslan Gorovyi}
\textbf{Fill Feaouill} не лізь поперед мене)))

\iusr{Зоя Леухина}
\textbf{Fill Feaouill} Не міряйтесь! Є більш безграмотні!
\end{itemize} % }

\iusr{Андрій Прудченко}
Вже більше року, пишу виключно українською! Розмовляю все частіше теж українською. Десь на цей час відсотків на 70.
Намагаюсь якомога швидше досягти 100!)

\iusr{Елена Макаренко}

Для мене писати українською набагато складніше ніж говорити, саме тому мої
пости бувають російською. А ще тому, що мене читають не тільки українці, а і
росіяни, білоруси, та інши люди з снг, що живуть в Польщі


\iusr{Лілія Сергієнко}

Якось задала це запитання на сторінці сільської спільноти. Це там, де всі все і
про всіх знають. Село україномовне, біля столиці. Але дописи роблять
московською, не всі, але часто. І більшість відповідей була "яка різниця".
Обґрунтованих відповідей не отримала.

\ifcmt
  ig https://scontent-mia3-1.xx.fbcdn.net/v/t1.6435-9/244422903_2920338161550501_4046716613348700915_n.jpg?_nc_cat=106&_nc_rgb565=1&ccb=1-5&_nc_sid=dbeb18&_nc_ohc=BW7pWdhjXZcAX9L96sR&_nc_ht=scontent-mia3-1.xx&oh=a5725a750addafca470958e5187da32a&oe=6182C8A7
  @width 0.4
\fi

\iusr{Ольга Колоянова}

Привіт. Я з тих російськомовних, хто після подій 14-го року почав поступово
переходити на українську. Південь України, оточення майже повністю
російськомовне. Було б інше - й перехід відбувся б швидше. Своїм досягненням
вже вважаю, що в сім'ї говоримо 50/50 російською та українською. 5-річна дочка
володіє українською дуже органічно, постійно чуємо запитання "А ви что, в
садікє ізучаєтє укрАінскій?". Дуже дивуються, коли вона відповідає, що говорить
українською з мамою і татом.

У фб теж пишу 50/50. Залежить від того, якою мовою я думаю на момент написання
посту. Інша справа, що я взагалі пишу рідко.


\iusr{Andriy Shalyov}
Написав до пп

\iusr{Елена Пылыпчак}

В російській багато звичних штампів, які зразу приходять в голову. Українською,
щоб сказати, треба спочатку подумати)

\iusr{Дима Лукьяненко}

Ніколи не замислювався, якою мовою пишу, писав у фб на автоматі, - якою мовою
думка прийшла, тією і записував. Виходило десь 90 на 10 на користь російської.
Останнім часом перейшов в постах майже виключно на українську. Причому, це не
було якесь продумане та аргументоване рішення. Це сталося автоматично. От
тільки зараз, після вашого запитання, я замислився над цим. Відповідь виявилась
цікавою і несподіваною для мене самого.)) Просто з‘явилась дуже важлива для
мене людина, яка теж читає мої пости. Вона абсолютно україномовна, російська їй
некомфортно і важко заходить. То ж я просто хочу, щоби їй було комфортно . І
цей переход відбувся автоматично))) Але взагалі вважаю, що при цьому , якщо ти
когось цитуєш, то краще це робити тією мовою, якою автор написав оригінал.
Звичайно, тупо було б тут цитувати Румі арабською, або Гіту санскритом, але
російська, білоруська, польська, англійська, гадаю, більшості моїх френдів
достатньо зрозумілі. Просто, вважаю цитування мовою оригіналу, краще передає
закладені автором меседж і емоцію.

\begin{itemize} % {
\iusr{Ruslan Gorovyi}
\textbf{Dima Lukyanenko} дякую. Дуже цікаво
\end{itemize} % }

\end{itemize} % }
