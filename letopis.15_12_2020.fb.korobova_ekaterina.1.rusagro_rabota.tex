% vim: keymap=russian-jcukenwin
%%beginhead 
 
%%file 15_12_2020.fb.korobova_ekaterina.1.rusagro_rabota
%%parent 15_12_2020
 
%%url https://www.facebook.com/ekaterina.korobova.3/posts/3295833310518301
 
%%author_id korobova_ekaterina
%%date 
 
%%tags kompania.rusagro,rabota,rossia,seljskoje_hozjajstvo,vpechatlenie,zhizn
%%title Все так просят, рассказать как я в Русагро пришла
 
%%endhead 
 
\subsection{Все так просят, рассказать как я в Русагро пришла}
\label{sec:15_12_2020.fb.korobova_ekaterina.1.rusagro_rabota}
 
\Purl{https://www.facebook.com/ekaterina.korobova.3/posts/3295833310518301}
\ifcmt
 author_begin
   author_id korobova_ekaterina
 author_end
\fi

Все так просят, рассказать как я в Русагро пришла @igg{fbicon.shrug} Ладно
@igg{fbicon.face.smiling.hearts} 

Как я писала ранее я переходила в Русагро из западной компании. Как всегда,
сначала агентство, а потом встреча в компании.

На тот момент мне уже сделали 2 оффера от других компаний, но внутри что-то
«екнуло», я не знаю, может, интуиция, судьба, предчувствие, и решила пройти
встречи в Русагро.

\ifcmt
  ig https://external-lga3-1.xx.fbcdn.net/safe_image.php?d=AQFe-u-U2oFouz-A&w=476&h=476&url=https%3A%2F%2Fscontent-lga3-1.cdninstagram.com%2Fv%2Ft51.29350-15%2F130764796_401393324477540_2478514714069758052_n.jpg%3F_nc_cat%3D107%26ccb%3D1-5%26_nc_sid%3D8ae9d6%26_nc_ohc%3D5DOY70tyZVgAX8JsEVq%26_nc_ht%3Dscontent-lga3-1.cdninstagram.com%26oh%3Deee66318eb9bdd65c8bf4d8929f2244a%26oe%3D61536DCC&cfs=1&_nc_oe=6ebdd&_nc_sid=62b3ef&ccb=3-5&_nc_hash=AQFmWBYbhTVAvFLy
  @width 0.4
  %@wrap \parpic[r]
  @wrap \InsertBoxR{0}
\fi

И знаете, я тоже ехала  на собеседование и думала: «Катя, что ты делаешь?  Это
же сельское хозяйство, это крестьяне с вилами и пашут на лошадях. Как ты туда
будешь искать людей?  Отправлять офферы   с голубями? )))))))))))))))))))))»

Сейчас так смешно, а тогда было так.  Нет, я не боялась производства, я тогда
уже его отлично  знала, и людьми пробовала уже управлять, но я боялась новой
отрасли, которая представлялась не очень прогрессивной. 

Собеседование прошло увлекательно, Ольга Федорова потрясающе рассказала о
компании, меня собеседовала она. С тех пор - one love @igg{fbicon.heart.red}.

И вот о чем я хочу сказать. Может это прозвучит не профессионально, но зато
по-человечески. Я после этой встречи внутри на 100\% знала, что если мне эта
компания сделает предложение, то я буду работать в ней. Отмету все предложения
и буду работать тут. И я точно знала, что будет
предложение.@igg{fbicon.heart.red}

Я не анализировала, не копалась в отзывах, не собирала рекомендации, я просто
знала – мое. И я точно знала, что я перехожу к вдохновленному человеку, в этой
компании.

Адаптации в компании у меня практически не было после западной.  @igg{fbicon.dizzy}  Процессы,
системы, подходы, требования уже были как в лучших западных.  А процессы, о
которых я расскажу позже, порой превосходили западные.  Меня поразило
стремление компании меняться, добиваться лучшего результата, проявлять
инициативу.  @igg{fbicon.collision} 

Скоро будет 10 лет как я работаю в АГРО- и я горжусь работой в этой отрасли. И
 @igg{fbicon.100.percent} каждый день из этих 10 лет   мне давали возможность проявить себя,
развиваться, и я становилась лучше.

Сейчас, когда ко мне приходят на собеседования люди, и я не «продаю» кандидату
компанию,  я искренне рассказываю о компании, потому что я так думаю. @igg{fbicon.heart.eyes} 

\begin{verbatim}
	#тамбовскаяобласть 
	#белгород 
	#орёл 
	#курск 
	#москва 
	#АмбассадорРусагро 
	@rusagrosahar
\end{verbatim}
