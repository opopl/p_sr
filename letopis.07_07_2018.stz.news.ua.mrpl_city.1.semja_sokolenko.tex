% vim: keymap=russian-jcukenwin
%%beginhead 
 
%%file 07_07_2018.stz.news.ua.mrpl_city.1.semja_sokolenko
%%parent 07_07_2018
 
%%url https://mrpl.city/blogs/view/semya-sokolenko
 
%%author_id burov_sergij.mariupol,news.ua.mrpl_city
%%date 
 
%%tags 
%%title Семья Соколенко
 
%%endhead 
 
\subsection{Семья Соколенко}
\label{sec:07_07_2018.stz.news.ua.mrpl_city.1.semja_sokolenko}
 
\Purl{https://mrpl.city/blogs/view/semya-sokolenko}
\ifcmt
 author_begin
   author_id burov_sergij.mariupol,news.ua.mrpl_city
 author_end
\fi

\ii{07_07_2018.stz.news.ua.mrpl_city.1.semja_sokolenko.pic.1.zhan_hpi}

На Новоселовке, в доме, что стоит на углу улиц Бахмутской и Тимирязева, жила
большая дружная семья Соколенко. У \textbf{Ивана Ивановича} и \textbf{Марфы Александровны} было
четыре сына: \textbf{Жан, Виктор, Василий, Александр}, один краше другого, и всеобщая
любимица – дочка \textbf{Светлана}. Иван Иванович, Марфа Александровна и их старший сын
Жан покинули нашу бренную землю, а сыновей и дочь разметала судьба по другим
городам. Сейчас в Мариуполе живет только Василий, преподаватель по классу баяна
и аккордеона 5-й музыкальной школы.

\ii{07_07_2018.stz.news.ua.mrpl_city.1.semja_sokolenko.pic.2.semja_sokolenko_s_nevestkoj}

В каникулярное время, когда в школе не было учеников и царила тишина, и
произошла наша встреча с Василием Ивановичем и его бывшим соседом, \textbf{Николаем
Леонтьевичем Гахом}, инженером. Он был известен и на \enquote{Азовмаше}, и на
\enquote{Азовстали}, и на стройке металлургического завода в далекой Индии как
большой специалист в своем деле и умелый руководитель, пребывающий сейчас на
заслуженном отдыхе. Собственно, от них и довелось узнать о семье Соколенко, на
первый взгляд обычной, а на самом деле во многом особенной.

Вот что рассказал Николай Леонтьевич: 
\begin{quote}
\em\enquote{В 1945 году моим родителям, как
многодетной семье, было предоставлено жилье по улице Тимирязева, в доме № 36.
Так мы стали соседями двух братьев Соколенко – Василия Ивановича и Ивана
Ивановича. Дети быстро нашли общий язык, часто бывали в домах друг у друга,
играли вместе, делились между собой скудными лакомствами голодного
послевоенного времени, а то и просто куском хлеба. Так что жизнь наших семей
была как на ладони.

И Иван Иванович, и Василий Иванович работали на заводе \enquote{Сельмаш}, а Марфа
Александровна вела домашнее хозяйство. Василий Иванович – инвалид Отечественной
войны, осенью сорок третьего года в жестоких боях под Мелитополем потерял ногу
выше колена. Может, по этой причине он так и не обзавелся семьей. Для того
чтобы как-то разделить или развеять скуку, Василий Иванович начал заниматься
музыкой. Купил кустарного производства баян и самоучкой, на слух начал
подбирать сначала фронтовые песни, а потом уже пошли и другие произведения,
когда больше освоился с клавиатурой}.
\end{quote}

Пример Василия Ивановича оказался заразительным. Старший сын семьи Жан стал
пробовать играть на баяне, дядя делился своим небольшим опытом, и вскоре ученик
достиг уровня своего наставника. И когда все поняли, что мальчик музыкально
одарен, был приглашен учитель музыки. В семейных преданиях сохранилось его имя
и отчество – \textbf{Борис Викентьевич}, так как в те времена в Мариуполе еще не было
музыкальной школы с обучением игры на баяне. Преподавание Бориса Викентьевича
было столь успешным, а усердие и способности Жана были столь велики, что когда
он после окончания 10 класса сдал вступительные экзамены в Харьковское
музыкальное училище, то его безоговорочно зачислили в это учебное заведение. В
этот же год он подал документы в Харьковский политехнический институт, благо,
что сроки приемных экзаменов в училище и вуз были разные, успешно преодолел
вступительные испытания и немалый конкурс и стал студентом. Жан сделал выбор в
пользу политехнического института, но с музыкой никогда не расставался. Играл в
студенческом оркестре баянистов, брал с собой инструмент, когда ребят из ХПИ
отправили на целину, а когда уже работал на заводе, на праздничных
демонстрациях под баян Жана Ивановича сотрудники пели песни.

\ii{07_07_2018.stz.news.ua.mrpl_city.1.semja_sokolenko.pic.3.semja_sokolenko}

Николай Леонтьевич вспоминал: \emph{\enquote{Сначала Жан дуэтом играл с дядей Васей, потом,
когда баян освоил Виктор, играли втроем, через время к ним присоединился
Василий. С той поры можно было слышать на Бахмутской улице квартет баянистов.
Люди, проходя мимо, останавливались и заслушивались музыкой, которая звучала из
двора Соколенко}}. Василий Иванович добавил к этому: \emph{\enquote{Все, буквально все играли
на этом инструменте, в том числе и сестричка Светлана, и Саша, младший мой
братик. Играли бы и вшестером, да баянов не хватало}}.

Все дети Ивана Ивановича и Марфы Александровны получили высшее образование.
Виктор Иванович, как и Жан, окончил тот же политехнический институт, был
направлен в Киев, там и остался, Светлана Ивановна получила диплом о высшем
образовании на химическом факультете Черновицкого государственного
университета, стала преподавателем химии в общеобразовательной школе, самый
младший из братьев Александр – выпускник Киевского торгово-экономического
института. Но профессиональным музыкантом стал только Василий Иванович. 

Вот что он поведал о своем пути в творческую и педагогическую деятельность: 

\begin{quote}
\em\enquote{Я, как и все мои старшие братья, потянулся к музыке. И в этом большая заслуга,
конечно, была нашего, можно сказать, второго отца – дяди Васи. Я не учился в
музыкальной школе, тогда уже была такая школа, но мы от нее далеко жили. А
стать музыкантом очень хотелось. Поэтому я всегда прислушивался к исполнению
старшего брата Жана, когда к нему приходил учитель на урок. Постепенно освоил
баян, так сказать, самоучкой. Кстати, когда я после окончания средней школы
поступал в Черновицкое музыкальное училище, сольфеджио не сдавал. Этот предмет
я не знал, не проходил. Но когда я сыграл \enquote{Саратовские переборы},
преподаватель, который присутствовал на экзамене, сказал: \enquote{Этого парня надо
брать}. И меня приняли. Было это в 1961 году.

На следующий год был призван на службу в Советскую армию. После демобилизации в
1965 году вернулся в то же Черновицкое музыкальное училище, где и продолжил
свою учебу. Между прочим, в тот период на нашем курсе училась знаменитая теперь
народная артистка София Ротару. С 1972 года я продолжил образование в
Каменец-Подольском педагогическом институте на музыкальном факультете. Через
пять лет я его окончил и в том же городе, где учился, работал директором
музыкальной школы. А в 1978 году переехал к своим родителям в Мариуполь.
Работал преподавателем во 2-й музыкальной школе, а затем в школе № 5. Уже более
тридцати лет играем в ансамбле со своим прекрасным другом, композитором,
музыкантом Владимиром Александровичем Митиным. И, нужно сказать, я получаю от
этого большое творческое удовлетворение}.
\end{quote}

Большим уважением в семье пользовался старший брат - Жан Иванович. Именно он
первым проторил, так сказать, путь к высшему образованию своим братьям и
сестре. Наверное, благодаря не только дяде Васе, но и ему приобщились к музыке
его младшие братья и сестра. Он мог бы вполне стать концертирующим музыкантом
или композитором, но выбрал, как мы знаем, инженерную стезю. Завершив обучение
в Харьковском политехническом институте, некоторое время работал на
предприятии, куда попал по направлению. Но что-то там его не устраивало, и Жан
вернулся в родной город. Незадолго до этого образовался Ждановский завод
тяжелого машиностроения. Его приняли в конструкторский отдел
подъемно-транспортного оборудования. В этом подразделении Жан Соколенко прошел
путь от рядового инженера-конструктора, начальника бюро до заместителя главного
конструктора отдела. Он был командирован в Румынию - в порт Констанца, где был
консультантом по наладке припортового крана, изготовленного тяжмашевцами. Его
уважали коллеги, ценило руководство. Помнится, какими теплыми словами отзывался
о нем его непосредственный начальник, главный конструктор отдела \textbf{Владимир
Авдеевич Михеев}.

Настало время рассказать о внуках Ивана Ивановича и Марфы Александровны. Все
они, как и их дети, стали специалистами с высшим образованием. Сын Жана
Ивановича окончил Мариупольский металлургический институт, его дочь Елена –
филологический факультет Донецкого государственного университета, сейчас
преподает английский язык в Киевском университете им. М. П. Драгоманова. Дочь
Виктора Ивановича Светлана – выпускница Киевского института народного
хозяйства, в настоящее время – финансовый директор страховой компании.
Людмила, дочь Светланы Ивановны, получила специальность историка в Киевском
государственном университете, доцент, кандидат исторических наук, сын Юрий тоже
историк, но окончил он Тернопольский педагогический институт, как и сестра –
кандидат наук. Оксана, дочь Василия Ивановича, получила высшее образование
также в Тернопольском педагогическом институте, вторая дочь Ирина окончила
Приазовский государственный технический университет, экономист. Дочь Александра
Ивановича Марина - выпускница Киевской академии водного транспорта.

Наша встреча закончилась признанием Василия Ивановича: {\em\bfseries\enquote{Я всегда вспоминаю
своих родителей. Их нет уже в живых. Вспоминаю всегда с добрым сердцем, с
любовью к ним. Мои родители были не очень грамотными, но были мудрыми. Они
были, во-первых, тружениками, и нас, своих детей, приучали к этому. Поэтому
навсегда в памяти моей родители остались как добрые, честные, хорошие мои папа
и мама}}.

Вот такая история семьи Соколенко, которая жила в скромном доме на улице
Бахмутской.
