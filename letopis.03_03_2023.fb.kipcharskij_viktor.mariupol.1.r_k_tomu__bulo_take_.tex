%%beginhead 
 
%%file 03_03_2023.fb.kipcharskij_viktor.mariupol.1.r_k_tomu__bulo_take_
%%parent 03_03_2023
 
%%url https://www.facebook.com/permalink.php?story_fbid=pfbid0WinzdiK3ty7v66YG3gNxrTejXJSKc1RDFggogCw1oN3tLgud7cfzJnb9gPxXc9qFl&id=100006830107904
 
%%author_id kipcharskij_viktor.mariupol
%%date 03_03_2023
 
%%tags 03.03.2022,dnevnik,mariupol,mariupol.war
%%title Рік тому  було таке:  День 8 - 3.03.22. Четвер
 
%%endhead 

\subsection{Рік тому  було таке:  День 8 - 3.03.22. Четвер}
\label{sec:03_03_2023.fb.kipcharskij_viktor.mariupol.1.r_k_tomu__bulo_take_}

\Purl{https://www.facebook.com/permalink.php?story_fbid=pfbid0WinzdiK3ty7v66YG3gNxrTejXJSKc1RDFggogCw1oN3tLgud7cfzJnb9gPxXc9qFl&id=100006830107904}
\ifcmt
 author_begin
   author_id kipcharskij_viktor.mariupol
 author_end
\fi

Рік тому  було таке: 

День 8 - 3.03.22. Четвер.

Світла і зв'язку нема, відповідно, нема і новин.

З ранку приходили свати, принесли якусь їжу: їм вдалося купити хліб чи батон. Здається, це був останній свіжий хліб, яким ми ласували в Маріуполі...

Весь час гупає, але до цього вже звикли і звертаємо увагу лише коли на якійсь час стає тихо...

На приймачі слухаємо канал Рада, бо більше нічого піймати не можемо. Бойченко: "Пошкоджено 15 енерговводів, основний та допоміжні водогони, але бійці Азову сказали мені  не хвилюйтеся, запевняємо Вас..."

Діти нервують, хвилюються за онуків, не розуміють, чому не поїхали з Маріуполя 24-го... Кажуть що св'язок від МТС та Водофону ще працюють. Завтра спробуємо купити її карточки в АТБ.

"Новеньких" стає більше: додаються люди з Західного, де дуже гучно...

Тиск води зменьшався десь з 9-ї ранку. Заповнили усі ємності. Близько 15-ї  води не стало. Від усіх благ цивілізації поки що є газ. Чи надовго? У квартирі ще тримається тепло: +21. Зовні увечері - близько нуля.

Близько 23:00 червона пляма, що мерехтить в напрямку Кам'янська або Шлакопереробки.

Навздогін: рік тому, 28.02.22 я виклав  відео з маріупольського телеканалу про те, як лікарі намагаються врятувати життя пораненій у Сартані (передмістя Маріуполя) дитині. Як дорослі люди, професія яких полягає в тому, щоб повертати людей до життя, плачуть, бо на цей раз їх вміння і навички виявилися недостатніми... Вони плачуть від безсилля...

Фейсбук заблокував цей спогад. 

Про таке не можна згадувати...

Пережити можна, а от згадувати - надто жорстоко...

%\ii{03_03_2023.fb.kipcharskij_viktor.mariupol.1.r_k_tomu__bulo_take_.cmt}
