% vim: keymap=russian-jcukenwin
%%beginhead 
 
%%file 20_04_2022.fb.fb_group.story_kiev_ua.1.rock_n_roll
%%parent 20_04_2022
 
%%url https://www.facebook.com/groups/story.kiev.ua/posts/1907000756163383
 
%%author_id fb_group.story_kiev_ua,stepanov_farid
%%date 
 
%%tags 
%%title ХРОНИКИ НЕОБЪЯВЛЕННОЙ ВОЙНЫ. День пятьдесят шестой. Long live rock-n-roll!
 
%%endhead 
 
\subsection{ХРОНИКИ НЕОБЪЯВЛЕННОЙ ВОЙНЫ. День пятьдесят шестой. Long live rock-n-roll!}
\label{sec:20_04_2022.fb.fb_group.story_kiev_ua.1.rock_n_roll}
 
\Purl{https://www.facebook.com/groups/story.kiev.ua/posts/1907000756163383}
\ifcmt
 author_begin
   author_id fb_group.story_kiev_ua,stepanov_farid
 author_end
\fi

ХРОНИКИ НЕОБЪЯВЛЕННОЙ ВОЙНЫ

День пятьдесят шестой. Long live rock-n-roll!

Троюродный брат жены Алексей, проживающий в рф - приятный пример того, как в
голове человека сохраняются мыслительные процессы, если не смотреть росTV, да
и телевизор в принципе. Можете сколько угодно кидать в меня камни гнева, но я
искреннее восхищён его смелостью. Смелостью, с которой он пытается
разговаривать со своими соотечественниками. О этой неравной борьбе напишу
отдельно. Сегодня о rock-n-roll.

\ii{20_04_2022.fb.fb_group.story_kiev_ua.1.rock_n_roll.pic.1}

На днях Алексей переслал нам целый трактат от одной своей знакомой с
пространными рассуждениями о мире и братской любви. Этот трактат она просила
отправить нам. Жена уже, было, собралась отвечать, но я её остановил:

- А ты заметила два ключевых вопроса в тексте?

- Какие?

- \enquote{Что нас не устраивало в ссср?} и \enquote{Почему мы захотели выйти из ссср от
русских?}.

- Как историк, скажи: что ей ответить?

- Ты же - филолог и отлично понимаешь, что слова имеют значение. Во-первых,
\enquote{что нас не устраивало в ссср?} не имеет никакого отношения к тому, что её
страна ведёт войну в Украине. ссср уже нет тридцать лет. Понимаю, к чему она
ведёт. Во-вторых, обрати внимание на формулировку \enquote{... от русских}. Это что,
как неразумный младший сын совершил ошибку и пошёл во все тяжкие? А мудрый
старший брат возвращает его в лоно семьи? Знаешь, как называется принуждение к
любви без согласия одной из сторон? Изнасилование. Мне такая братская любовь не
нужна! 

- Так что ей ответить?

- Пусть напишет нам пять примеров, на основании которых она делает вывод, что в
Украине нацизм. Не аргументов \enquote{это итак понятно}. А откроет книгу или Интернет,
найдёт определение \enquote{нацизм} и его признаки. Её же в Google не забанили? И
каждый из этих признаков нацизма подтвердит примером из современной Украины. А
там уже решим: есть смысл с ней что-то обсуждать или... 

- Она над нами издевается? - прошёл день и Алексей отправил ответ от своей
знакомой.

- И что же она написала?

- Ничего. Она прислала видео. Как она ходила в Екатеринбургскую филармонию и
слушала выступление народного хора Свердловской области, который пел украинские
песни. А зрители аплодировали.

- В XXI веке область до сих пор носит имя организатора Красного террора - Якова
Свердлова? Организатора убийства Николая II, которого их церковь
канонизировала. Канонизировала того, кого ещё при жизни народ называл Николаем
кровавым за 9 января. Что у них в головах!

- Ты лучше текст песни послушай. Там поётся о младшей сестрёнке украиночке.
Младшей! 

- Это то, о чём я тебе говорил - комплекс старшего брата. Который сам себе
присвоил право быть главой семьи. Она направление русского корабля знает?

Мне в жизни очень везёт: я был на концертах величайших музыкальных рок групп
и исполнителей современности. Я стоят от них на расстоянии вытянутой руки,
брал у них автографы. Воспитанный на классической музыке, я очень люблю рок.
Когда сегодня вижу поддержку Украины рок-сообществом, горжусь, что я - его
часть. 

Pink Floyd - моя любимая группа, впервые за двадцать восемь лет записывает
новую песню - украинскую песню \enquote{Ой у лузі червона калина} с Андреем
Хлывнюком. И весь мир поет её. Всегда нарочито аполитичные Deep Purple
раскрашивают свой логотип на официальном сайте в жовто-блакитний и отменяют
все концерты в рф. Возвращают автограф медведеву. Julian Lennon, сын самого
John Lennon, ради Украины нарушает обет не исполнять песню своего отца
"Imagine". Scorpions меняют слова одной из своих культовых песен "Wind of
Change", заменив москва на сердце с Украиной. Sting и Jon Bon Jovi записывают
песни в поддержку моей страны. Metallica и Queen собирают средства для помощи
пострадавшим. Rammstein записывает новую песню в поддержку Украины. Даже сам
великий и ужасный Ozzy Osborne с Украиной! 

Билли Айлиш, Мадонна, Селин Дион, Кэти Перри, Элтон Джон, Майли Сайрус, The
Weeknd, U2, Red Hot Chili Peppers... присоединились к глобальному
онлайн-событию в поддержку Украины: StandUpForUkraine — сбору средств на
помощь украинским беженцам.

А что они мне могут предложить из-за своего железного занавеса? Народный хор
Свердловской области? Да не хочу я слушать народный хор Свердловской области!
Не хочу и не буду! 

Long live rock-n-roll!

\ii{20_04_2022.fb.fb_group.story_kiev_ua.1.rock_n_roll.cmt}
