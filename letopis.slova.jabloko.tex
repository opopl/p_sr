% vim: keymap=russian-jcukenwin
%%beginhead 
 
%%file slova.jabloko
%%parent slova
 
%%url 
 
%%author 
%%author_id 
%%author_url 
 
%%tags 
%%title 
 
%%endhead 
\chapter{Яблоко}

%%%cit
%%%cit_head
%%%cit_pic
%%%cit_text
Подгнившие \emph{яблоки}.  Историю рассказал мой приятель.  «Все детство я прожил в
частном доме.  У нас был большой сад. И погреб.  В нем мы хранили \emph{яблоки}. Их
было очень много. Но некоторые \emph{яблоки} чуть подгнивали.  Мама просила папу
всегда есть их первыми.  Ведь если отрезать половинку, а иногда четвертинку, то
\emph{яблоко} почти хорошее. А выбросить жалко.  Пока мы сьедали одни подгнившие
\emph{яблоки}, успевали подгнить другие. И мы ели уже их.  В итоге, что я помню про
детство?  Все детство я жрал гнилые \emph{яблоки}.  А ведь можно было просто один раз
выбросить гнилые и есть спелые и вкусные»
%%%cit_comment
%%%cit_title
\citTitle{В один момент нужно решиться - и выбросить гнилые яблоки из своей жизни}, 
Евгений Черняк, strana.ua, 13.07.2021
%%%endcit
