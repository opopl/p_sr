% vim: keymap=russian-jcukenwin
%%beginhead 
 
%%file 10_08_2019.stz.news.ua.mrpl_city.1.ko_dnu_stroitelja_aleksandr_potlov
%%parent 10_08_2019
 
%%url https://mrpl.city/blogs/view/ko-dnyu-stroitelya-aleksandr-potlov
 
%%author_id burov_sergij.mariupol,news.ua.mrpl_city
%%date 
 
%%tags 
%%title Ко Дню строителя: Александр Потлов
 
%%endhead 
 
\subsection{Ко Дню строителя: Александр Потлов}
\label{sec:10_08_2019.stz.news.ua.mrpl_city.1.ko_dnu_stroitelja_aleksandr_potlov}
 
\Purl{https://mrpl.city/blogs/view/ko-dnyu-stroitelya-aleksandr-potlov}
\ifcmt
 author_begin
   author_id burov_sergij.mariupol,news.ua.mrpl_city
 author_end
\fi

\ii{10_08_2019.stz.news.ua.mrpl_city.1.ko_dnu_stroitelja_aleksandr_potlov.pic.1}

Тихая мариупольская улица им. Архитектора Нильсена, начало ее у главного входа
в Городской сад. На ней под номером 1 значится дом в стиле советского
конструктивизма – это одно из первых пятиэтажных строений Мариуполя. Говорили,
что дом этот построен по проекту архитектора Петра Теслера в середине 30-х
годов XX столетия. В дореволюционные годы среди горожан улица была известна,
как Константиновская. И названа была так в честь брата царя Александра ΙΙ,
великого князя Константина Николаевича. Он дважды посещал наш город,
восемнадцатилетним юношей в 1845 году и будучи уже зрелым человеком – морским
министром Российской империи - в 1872 году. Видимо, последнее посещение было
предопределено его должностью: Мариуполь уже в ту пору был заметным портовым
городом Российской империи. Но это – к слову.

Подойдем к дому старинной постройки под номером 16. Кто был его архитектором,
кто – его хозяином, кто обитал в нем до войны – неведомо. Существует лишь
легенда, что квартировал тут или был его хозяином некий священник, который во
время гражданской войны покинул нашу страну. А вот тех, кто жил в нем с семьями
в послевоенное время, назвать можно. Прежде всего, личность легендарная, но
теперь, к сожалению, забытая - \textbf{Александр Павлович Поборчий}, управляющий трестом
\enquote{Азовстальстрой}, он-то и восстановил разрушенное войной здание. Затем въехал
сюда с семьей главный инженер этого же треста \textbf{Михаил Петрович Демаков}.
Очередными обитателями дома № 16 стали \textbf{Алексей Иванович Дыкин}, возглавлявший
\enquote{Азовстальстрой}, и его семья. Алексей Иванович был направлен на работу в Киев,
а его должность и квартиру занял \textbf{Александр Федорович Потлов}. Вот этому-то
неординарному человеку и посвящены эти строки.

Так уж получилось, что далеко не все сведения о жизни Александра Федоровича
дошли до нашего времени. Но многое все же удалось узнать от его дочери – \textbf{Ольги
Александровны}. От нее-то стало известно, что родился он 5 мая 1919 года в селе
Караваино Инжавенского района Тамбовской губернии. В 1937 году окончил среднюю
школу в Томске. Почему в Томске? Как попал уроженец Тамбовщины в этот
западносибирский город - неясно. В следующем году он уже студент первого курса
Новосибирского инженерно-строительного института. Сохранилась фотография
Александра Потлова студенческой поры. Открытое лицо, сосредоточенный взгляд
чуть печальных глаз. Он перешел на третий курс института, а тут началась война.
Студенты рвались на фронт, чтобы вместе со своими сверстниками сражаться с
фашистскими захватчиками. Но правительство думало по-иному: скоро, очень скоро
понадобится много инженеров-строителей, поэтому оставило их в тылу...

\textbf{Читайте также:} 

\href{https://mrpl.city/news/view/priemnuyu-me-ra-mariupolya-sproektiruyut-studenty-na-konkurse}{%
Приемную мэра Мариуполя спроектируют студенты на конкурсе, Богдан Коваленко, mrpl.city, 09.08.2019}

Второго февраля 1943 года завершилась великой победой Красной Армии
Сталинградская битва. После бомбардировок, артобстрелов, жесточайших уличных
боев Сталинград представлял собой нагромождение битого кирпича, скопищем
остатков прокопченных стен, испещренных следами пуль и осколков, в которых
трудно узнавались дома, пространством, где невозможно было определить ни четкие
границы кварталов, ни улиц. На площадках заводов вместо цехов – груды
искореженных причудливым образом металлоконструкций, покрытых ржавчиной. Все
это предстояло вернуть к жизни. В этом же, 1943 году, Александр Потлов
оканчивает институт и его с молодой женой Ириной, тоже инженером-строителем,
только что получившей диплом инженера в Куйбышевском инженерно-строительном
институте, направляют на восстановление твердыни на Волге. Там им довелось
работать в чрезвычайно тяжелых условиях четыре года. За активное участие в
восстановлении города-героя Александр Федорович был награжден медалью
\enquote{За доблестный труд в Великой Отечественной войне}.

В 1947 году Потловы уехали по новому назначению в Мариуполь. На новом месте
Александр Федорович и его жена приступили к своим обязанностям в строительных
управлениях треста \enquote{Азовстальстрой}. Семье, в которой уже была
трехлетняя дочка – Ниночка, была выделена комната в коммунальной квартире 45-го
дома на проспекте Республики, так в ту пору назывался проспект Мира. Когда у
Потловых родилась вторая дочь - Ольга, они поселились в более просторной
квартире в доме номер 8 на улице Энгельса. В то время все силы и средства
треста \enquote{Азовстальстрой} были сосредоточены на восстановлении взорванных
гитлеровцами цехов и сооружении новых объектов завода \enquote{Азовсталь}.
Заглянув в летопись важнейших событий этого предприятия, можно перечислить
здесь то, что было сделано строителями, монтажниками, работниками завода: 1947
год – восстановлены 3, 4 и 5 мартеновские печи; 1948 год – введен в
эксплуатацию цех \enquote{Блюминг}, восстановлена мартеновская печь № 6, дал
первый прокат для народного хозяйства рельсобалочный цех. И это все
осуществлено только за два года в израненной стране людьми, которые испытывали
нужду в еде, одежде, более или менее нормальном жилье, при недостатке
строительной техники и материалов. Но, кроме того, велось еще и гражданское
строительство: восстанавливались школы, клубы, кинотеатры, жилье. Темпы
последующих лет были не менее напряженными и еще более результативными.

Александр Федорович Потлов принимал непосредственное участие в ряде этих работ,
занимая посты начальника производственно-технического отдела стройуправления,
затем главного инженера, начальника управления. Работал он самоотверженно, от
зари до зари, прилагая все силы и знания, чтобы строительные объекты сдавались
в эксплуатацию в установленные сроки. Об этом говорят его награды: медаль \enquote{За
восстановление предприятий черной металлургии Юга}, орден \enquote{Знак почета} и
высшая награда советской страны - орден Ленина. В 1951 году Александр Федорович
был направлен в Москву на учебу на высшие курсы Министерства строительства
СССР. По окончании их, в 1952 году, он возвращается домой и приступает к
обязанностям главного инженера строительного управления № 9 - уже ставшим для
него родным, - треста. Но в этой должности Потлов задержался недолго...

\textbf{Читайте также:} 

\href{https://mrpl.city/news/view/rekonstruktsiya-ploshhadi-svobody-kogda-v-mariupole-otkroyut-fontan-za-24-mln-grn-foto}{%
Реконструкция площади Свободы: когда в Мариуполе откроют фонтан за 24 млн грн, Анастасія Селітріннікова, mrpl.city, 09.08.2019}

Следующим этапом его карьеры было назначение в 1956 году на пост управляющего
треста \enquote{Азовстальстрой}. За период его руководства на заводе \enquote{Азовсталь} были
построены двенадцатая мартеновская печь, возведены пятая и шестая доменные
печи, сооружено здание драматического театра. Это, так сказать, самые крупные
объекты. Но помимо этого выполнялись подрядные работы на заводе имени Ильича,
других предприятиях нашего города и области, строились и многоэтажные жилые
дома.

В январе 1961 года Александра Федоровича Потлова направили в Донецк заведовать
отделом строительства обкома компартии Украины. Это сулило большие перспективы
в его дальнейшем продвижении по службе. Тогда принято было \enquote{обкатывать} хорошо
проявивших себя на производстве, относительно молодых инженеров-руководителей
на партийной работе, чтобы готовить из них крупных государственных управленцев:
начальников главков, заместителей министров или даже министров. Но
воспользоваться этой возможностью ему было не суждено. В должности завотделом
он работал менее года. 5 декабря 1961 года Александр Федорович Потлов
скоропостижно скончался. Ему было всего 42 года. 

На руках у его вдовы, Ирины Александровны, остались три дочери:
семнадцатилетняя Нина, двенадцатилетняя Ольга и малышка Татьяна, которой было
всего пять лет. Кто знает, как бы сложилась судьба семьи, если бы не помощь
\textbf{Ивана Петровича Голубева, Бориса Васильевича Николаева} и других близких друзей.
С 1962 года Ирина Александровна работала в тресте \enquote{Ждановжилстрой} заместителем
управляющего по экономическим вопросам, начальником сметно-договорного отдела.
В 1976 году она ушла на заслуженный отдых. Эта мужественная женщина преодолела
все трудности, вырастила трех дочерей, помогла всем им получить высшее
образование. Память о своих предках сейчас хранят не только дочери, но и шесть
внучек и внук \textbf{Александр}, названный так в честь деда.

\textbf{Читайте также:}

%\href{https://mrpl.city/blogs/view/ko-dnyu-metallurga-i-gornyaka-vasilij-voropaev}{%
\href{https://archive.org/details/20_07_2019.stz.news.ua.mrpl_city.1.ko_dnu_metallurga_i_gornjaka_vasilij_voropaev}{%
Ко Дню металлурга и горняка: Василий Воропаев, mrpl.city, 20.07.2019}
