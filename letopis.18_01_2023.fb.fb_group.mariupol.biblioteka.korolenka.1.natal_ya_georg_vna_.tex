%%beginhead 
 
%%file 18_01_2023.fb.fb_group.mariupol.biblioteka.korolenka.1.natal_ya_georg_vna_
%%parent 18_01_2023
 
%%url https://www.facebook.com/groups/1476321979131170/posts/5748929958536996
 
%%author_id fb_group.mariupol.biblioteka.korolenka,lisogor_viktoria.mariupol
%%date 18_01_2023
 
%%tags poezia,literatura,mariupol,kultura,more,more.azov,mariupol.pre_war
%%title Наталія Георгіївна Харакоз...
 
%%endhead 

\subsection{Наталія Георгіївна Харакоз...}
\label{sec:18_01_2023.fb.fb_group.mariupol.biblioteka.korolenka.1.natal_ya_georg_vna_}
 
\Purl{https://www.facebook.com/groups/1476321979131170/posts/5748929958536996}
\ifcmt
 author_begin
   author_id fb_group.mariupol.biblioteka.korolenka,lisogor_viktoria.mariupol
 author_end
\fi

Наталія Георгіївна Харакоз...

Письменниця, журналістка, громадська діячка грецького походження. Член
Національної спілки журналістів України (1971). Перша в історії Маріуполя
жінка, яка стала членом Національної спілки письменників України (1998).

Добрий друг нашої бібліотеки та почесна читачка. Була однією з ініціаторів
створення літературного музею в бібліотеці ім. В. Г. Короленка. За її участю в
стінах бібліотек міста пройшло безліч літературних вечорів та презентацій
книжок авторів Приазов'я.

Була. Бо загинула в окупованому Маріуполі навесні 2022 року.

Але, пам'ять про неї буде жити вічно. У наших спогадах. Та завдяки чудовій
онуці Анна Котыхова, яка зібрала усі сили та видала книжку «Новеллы Азовского
побережья», де надруковано ще не відомі читачу новели відомої маріупольчанки.
Ну і звичайно, дякую за цей подарунок для маріупольської бібліотеки. Завдяки
таким небайдужим людям, ми потроху оновимо краєзнавчий фонд, бо, що сталося з
книжками маріупольських авторів в окупованому місті, наразі невідомо.

Далі буде...

Ну а компанія \#новапошта як завжди на висоті. Дякуємо, за безкоштовну доставку
книжок для нашої книгозбірні.

%\ii{18_01_2023.fb.fb_group.mariupol.biblioteka.korolenka.1.natal_ya_georg_vna_.cmt}
