% vim: keymap=russian-jcukenwin
%%beginhead 
 
%%file sochinenie
%%parent songs
 
%%url 
 
%%author 
%%author_id 
%%author_url 
 
%%tags 
%%title 
 
%%endhead 
\url{http://lib4school.ru/printout.php-id=1&bookid=181-9.htm}

Давно кем-то было сказано: «Если хочешь понять поэта, иди в его страну». Если
хочешь понять душу народа, ее сокровенные глубины, неповторимую
индивидуальность - изучай народные песни. Песня, это замечательное сочетание
слова и музыки, является духовным лицом нации, ее своеобразной визитной
карточкой.



Известно, что украинских песен записано более 200 тысяч и этим далеко не
исчерпывается песенное богатство нашего народа. Специалисты утверждают, что ни
одна другая нация в мире не может похвастаться таким количеством песен. Высокую
оценку украинской народной песни в разные времена давали знаменитые зарубежные
и отечественные деятели культуры:



«Ни в одной другой земле дерево народной поэзии не дало таких величественных плодов, нигде дух народа не оказался в песнях так живо и правдиво, как в украинцев... Действительно, народ, который мог петь такие песни и любоваться ими, не мог стоять на низком уровне образования».



(Ф. Боденштедт, немецкий поэт)



«Украинская народная поэзия самая богатая и самая красивая в Европе. Она обладает большими художественными достоинствами и поэтическим вдохновением, меткостью высказывания и имеет в себе что-то трогательное, величавое, что-то чувствительно-чувственное, унылое и живописный».



(А. Люкшич, югославский ученый)



«Украинские просторы - столица лирической поэзии. Отсюда песни неизвестных авторов часто распространялись по всей Слов'янщині».



(А. Мицкевич, польский поэт)



«Украинские песни будут положены в основу литературы будущего».



(Ю. Словацкий, польский поэт)



«Кажется, каждая ветка дерева на Украине имеет своего поэта и каждый стебелек травы на этих безграничных цветущих равнинах відлунюється песней».



(Тальви, американская писательница)



«Украинцы очень богатые, может, самые богатые между всеми славянами, разноцветными народными песнями.



Они не думают, не потеют, сочиняя свои песни; у них песни вырастают сами, как цветы на полях зеленых, и их такое количество, которой не может похвастаться ни один народ в мире».



(П.-И. Шафарик, чешский ученый, известный славист)



«Украинская песня имеет так много художественных ценностей, что их невозможно перечислить.



Это искусство глубоко народное том, что с него непосредственно обращается к нам своеобразная чистая душа украинского народа. Хорошая эта душа. Такое именно и ее искусство».



(3. Неєдли, чешский ученый и общественный деятель)



«Моя радость, жизнь моя! Песни! Как я вас люблю! Чего стоят все холодные летописи, в которых я теперь копаюсь, перед этими звонкими, живыми летописями!



Песни для Малороссии - все: и поэзия, и история, и отцовская могила.



Покажите мне народ, у которого было бы больше песен. Наша Украина звенит песнями».



(Г. Гоголь)



«Что за мелодичность и красота, не говоря уже о классическую простоту и непосредственность вдохновения. Се - источник, из которого на здоровье долго еще будут пить потомки.



Меткость высказывания чрезвычайная, а краткость подчас просто непередатна!».



(ТЕ. Грабовский)



«Украинская песня - это бездонная душа украинского народа, это его слава».



(О. Довженко)



«Украинская песня всегда встречала не только удивление, восхищение, но и буквально какое-то религиозное преклонение со стороны самых требовательных и строгих критиков всего мира».



(О. Кошиц)



В середине прошлого века известный русский писатель, переводчик украинских песен и произведений Тараса Шевченко на русском языке Николай Берг заключил антологию песен многих народов мира. На первом месте он поставил лирические украинские песни. Вот как он писал о них в предисловии: «Очаровательная, грациозная и нежная украинская песня. Скажу больше: в ней есть что-то, что хватает за сердце, что-то глубоко задушевное. Как справедливо здесь пригодились эти многочисленные ласкательные слова, которые невозможно перевести ни на какой другой язык... Но не на одних только ласковых словах создается обаяние украинской песни. Эта обаяние разлита во всем, в словах и в сравнениях, а иногда и просто не знаешь, в чем: мыло и все!».



В разговоре со студентами-украинцами гениальный Лев Толстой сказал:



«Счастливы вы, что родились среди народа с такой богатой душой, народа, что умеет так чувствовать свои радости и прекрасно изливать свои мысли, свои мри, свои заветные чувства. Кто имеет такую песню, поэтому нечего бояться за свое будущее. Его время не за горами. Верите или нет, что ни одного народа простых песен я не люблю так, как вашего. Под их музыку я душой спочиваю. Столько в них красоты и грации, столько сильного, молодого чувство силы».



в 1911 г. в Париже, в день пятидесятилетия смерти Шевченко, А. В. Луначарский произнес перед эмиграцией доклад, в котором сказал вдохновенные слова о украинскую песню:



«Украинская музыка и поэзия есть самая роскошная, найзапашніша из всех ветвей мировой народного творчества. Минорная по содержанию, грустная даже в своем веселом порыве, украинская песня относится всеми знатоками на первое место в музыке всех народов. Украинские думы, что через века передавались Гомерами Украины - кобзарями, светят своими красками, почуваннями, рыцарством в любви и вражды, размахом казацкой смелости и философской вдумчивостью».



Композитор П. И. Чайковский отмечал:



«Бывают счастливо одаренные натуры и бывают так же счастливо одаренные народы. Я видел такой народ, народ-музыкант, - это украинцы».



Много украинских песен известно далеко за пределами Украины. Одна из таких песен - «ехал казак за Дунай». Об этой песне и ее авторе рассказывает Константин Родик:



«Когда имя Семена Климовского сейчас ничем не отзывается в нашей памяти, то один из его произведений известный во всем мире Это песня «ехал казак за Дунай». Написана в херсонских степях в самом конце XVIII века, она уже в самом начале века XIX появляется в немецком переводе. Во время войны с Наполеоном немецкие солдаты разносят ее по всей Европе. А 1815 года в Филадельфии выходят в свет ноты и английский перевод. Популярность песни приобретает такой размах, что ее в каждом крае начинают воспринимать свою, народную. Людвиг ван Бетховен пишет на эту «народную» мелодию свою знаменитую вариацию.»



Эту историю исследовал и опубликовал в своей книге «Слово и песня» литературовед Григорий Скука.



О влиянии, которое оказывает на слушателя украинская песня, написал замечательное стихотворение киевский поэт Л. Киселев, что рано ушел из жизни:



Я позабуду все обиды,



И вдруг напомнят песню мне.



На мылом и полузабытом.



На украинском языке.



И в комнате где, как батоны,



Чужие лица без конца,



Взорвутся черные бутоны -



Окаменевшие сердца.



Я постою у края бездны



И вдруг пойму, сломясь в тоске,



Что все на свете - только песня



На украинском языке.



Украинский литературовед Федор Погребенник много лет исследует историю создания и распространения украинских патриотических песен:



«Патриотические песни, песни-гимны, что их столько десятилетий запрещено петь, все же не умирали: жили глубоко в душе верных сынов и дочерей народа, согревали веру в лучшие времена, помогали переносить политические преследования, вселяли надежду, что «встанет мать-Украина, счастливая и свободная».



Когда треснула лед имперского абсолютизма, «развалилась руина», патриотические песни и песни-гимны воскресли, словно Феникс из пепла, окрылили всех, кто вышел на площади и улицы, чтобы заманифестировать свою волю жить в независимой, суверенной Украине. Над городами и селами родной земли зазвенели старые казацкие песни, гимны «Ще не вмерла Украина», «Не пора, не пора, не пора», песни, возникшие в начале 1900-х годов и связаны с развитием патриотического спортивно-противопожарного мочевого движения, песни украинских сечевых стрельцов - военных отрядов в составе австро-венгерской армии, которые были предтечей более поздних национальных сил Украины. В конце концов, начали набирать прав гражданства и песни Украинской повстанческой армии.



Никто уже не сможет заставить их замолчать! Это доля нелегкой истории, в которой сквозь призму народной души отразилось величие и духовная красота народа, его свободолюбивые стремления, его боль и трагедия.



Приходят новые времена, когда утверждается идея суверенности нашего народа, за которую столько веков шла священная борьба. По-новому звучат сегодня эти столько лет не певческие и преследуемые песни, но и сейчас, как никогда, согревают наши сердца святой любовью к Украине».



Среди самых известных патриотических песен - «Ще не вмерла Украина» П. Чубинского, «За Украину!» М. Вороного, «Ой у лузи красная калина» С. Чарнецкого, «Молитва за Украину» А. Конисского, «Не пора, не пора, не пора» И. Франко, «Эй, там на горе Янв идет», «Там на горе, на Макушке», «Подул ветер степной» и др.



Сл. О. КОНИССКОГО Муз. М. ЛЫСЕНКО



Молитва за Украину



Боже Великий, Единый,



Нам Украину храни,



Свободы и света лучами



Ты ее осени.



Светом науки и знания



Нас, детей, просвещения,



В чистой любви к краю



Ты нас, Боже, вырасти.



Молимся, Бог Единый,



Нам Украину храни.



Все твои ласки, щедроты



Ты на народ наш обрати,



Дай ему волю,



Дай ему судьбу,



Дай доброго света



. Счастья дай. Боже, народу



И многая, многая лета!



О эти прекрасные песни можно сказать словами из стихотворения Г. Рыльского:



Благословенна ты в веках.



Как солнце наше благовестное.



Как вещий белокрылый птица.



Печаль и радость наша, постное,



Что мужество будиш в сердцах,



Когда над краем туча виснет.



В песнях, рожденных новой эпохой, - вера в Украину, в духовную силу народа, призыв приобретать волю без крови и насилия.



ДМИТРИЙ ПАВЛЫЧКО



Песня



Вставай, Украина, вставай.



Єднай Черное море и Карпаты,



И свой переболений край



Не дай врагам разломать.



Вставай и здіймай знамя



Всечеловеческой согласия и любви,



Чтобы воле святой вино



Употребить без мести и крови!
Давно кем-то было сказано: «Если хочешь понять поэта, иди в его страну». Если хочешь понять душу народа, ее сокровенные глубины, неповторимую индивидуальность - изучай народные песни. Песня, это замечательное сочетание слова и музыки, является духовным лицом нации, ее своеобразной визитной карточкой.



Известно, что украинских песен записано более 200 тысяч и этим далеко не исчерпывается песенное богатство нашего народа. Специалисты утверждают, что ни одна другая нация в мире не может похвастаться таким количеством песен. Высокую оценку украинской народной песни в разные времена давали знаменитые зарубежные и отечественные деятели культуры:



«Ни в одной другой земле дерево народной поэзии не дало таких величественных плодов, нигде дух народа не оказался в песнях так живо и правдиво, как в украинцев... Действительно, народ, который мог петь такие песни и любоваться ими, не мог стоять на низком уровне образования».



(Ф. Боденштедт, немецкий поэт)



«Украинская народная поэзия самая богатая и самая красивая в Европе. Она обладает большими художественными достоинствами и поэтическим вдохновением, меткостью высказывания и имеет в себе что-то трогательное, величавое, что-то чувствительно-чувственное, унылое и живописный».



(А. Люкшич, югославский ученый)



«Украинские просторы - столица лирической поэзии. Отсюда песни неизвестных авторов часто распространялись по всей Слов'янщині».



(А. Мицкевич, польский поэт)



«Украинские песни будут положены в основу литературы будущего».



(Ю. Словацкий, польский поэт)



«Кажется, каждая ветка дерева на Украине имеет своего поэта и каждый стебелек травы на этих безграничных цветущих равнинах відлунюється песней».



(Тальви, американская писательница)



«Украинцы очень богатые, может, самые богатые между всеми славянами, разноцветными народными песнями.



Они не думают, не потеют, сочиняя свои песни; у них песни вырастают сами, как цветы на полях зеленых, и их такое количество, которой не может похвастаться ни один народ в мире».



(П.-И. Шафарик, чешский ученый, известный славист)



«Украинская песня имеет так много художественных ценностей, что их невозможно перечислить.



Это искусство глубоко народное том, что с него непосредственно обращается к нам своеобразная чистая душа украинского народа. Хорошая эта душа. Такое именно и ее искусство».



(3. Неєдли, чешский ученый и общественный деятель)



«Моя радость, жизнь моя! Песни! Как я вас люблю! Чего стоят все холодные летописи, в которых я теперь копаюсь, перед этими звонкими, живыми летописями!



Песни для Малороссии - все: и поэзия, и история, и отцовская могила.



Покажите мне народ, у которого было бы больше песен. Наша Украина звенит песнями».



(Г. Гоголь)



«Что за мелодичность и красота, не говоря уже о классическую простоту и непосредственность вдохновения. Се - источник, из которого на здоровье долго еще будут пить потомки.



Меткость высказывания чрезвычайная, а краткость подчас просто непередатна!».



(ТЕ. Грабовский)



«Украинская песня - это бездонная душа украинского народа, это его слава».



(О. Довженко)



«Украинская песня всегда встречала не только удивление, восхищение, но и буквально какое-то религиозное преклонение со стороны самых требовательных и строгих критиков всего мира».



(О. Кошиц)



В середине прошлого века известный русский писатель, переводчик украинских песен и произведений Тараса Шевченко на русском языке Николай Берг заключил антологию песен многих народов мира. На первом месте он поставил лирические украинские песни. Вот как он писал о них в предисловии: «Очаровательная, грациозная и нежная украинская песня. Скажу больше: в ней есть что-то, что хватает за сердце, что-то глубоко задушевное. Как справедливо здесь пригодились эти многочисленные ласкательные слова, которые невозможно перевести ни на какой другой язык... Но не на одних только ласковых словах создается обаяние украинской песни. Эта обаяние разлита во всем, в словах и в сравнениях, а иногда и просто не знаешь, в чем: мыло и все!».



В разговоре со студентами-украинцами гениальный Лев Толстой сказал:



«Счастливы вы, что родились среди народа с такой богатой душой, народа, что умеет так чувствовать свои радости и прекрасно изливать свои мысли, свои мри, свои заветные чувства. Кто имеет такую песню, поэтому нечего бояться за свое будущее. Его время не за горами. Верите или нет, что ни одного народа простых песен я не люблю так, как вашего. Под их музыку я душой спочиваю. Столько в них красоты и грации, столько сильного, молодого чувство силы».



в 1911 г. в Париже, в день пятидесятилетия смерти Шевченко, А. В. Луначарский произнес перед эмиграцией доклад, в котором сказал вдохновенные слова о украинскую песню:



«Украинская музыка и поэзия есть самая роскошная, найзапашніша из всех ветвей мировой народного творчества. Минорная по содержанию, грустная даже в своем веселом порыве, украинская песня относится всеми знатоками на первое место в музыке всех народов. Украинские думы, что через века передавались Гомерами Украины - кобзарями, светят своими красками, почуваннями, рыцарством в любви и вражды, размахом казацкой смелости и философской вдумчивостью».



Композитор П. И. Чайковский отмечал:



«Бывают счастливо одаренные натуры и бывают так же счастливо одаренные народы. Я видел такой народ, народ-музыкант, - это украинцы».



Много украинских песен известно далеко за пределами Украины. Одна из таких песен - «ехал казак за Дунай». Об этой песне и ее авторе рассказывает Константин Родик:



«Когда имя Семена Климовского сейчас ничем не отзывается в нашей памяти, то один из его произведений известный во всем мире Это песня «ехал казак за Дунай». Написана в херсонских степях в самом конце XVIII века, она уже в самом начале века XIX появляется в немецком переводе. Во время войны с Наполеоном немецкие солдаты разносят ее по всей Европе. А 1815 года в Филадельфии выходят в свет ноты и английский перевод. Популярность песни приобретает такой размах, что ее в каждом крае начинают воспринимать свою, народную. Людвиг ван Бетховен пишет на эту «народную» мелодию свою знаменитую вариацию.»



Эту историю исследовал и опубликовал в своей книге «Слово и песня» литературовед Григорий Скука.



О влиянии, которое оказывает на слушателя украинская песня, написал замечательное стихотворение киевский поэт Л. Киселев, что рано ушел из жизни:



Я позабуду все обиды,



И вдруг напомнят песню мне.



На мылом и полузабытом.



На украинском языке.



И в комнате где, как батоны,



Чужие лица без конца,



Взорвутся черные бутоны -



Окаменевшие сердца.



Я постою у края бездны



И вдруг пойму, сломясь в тоске,



Что все на свете - только песня



На украинском языке.



Украинский литературовед Федор Погребенник много лет исследует историю создания и распространения украинских патриотических песен:



«Патриотические песни, песни-гимны, что их столько десятилетий запрещено петь, все же не умирали: жили глубоко в душе верных сынов и дочерей народа, согревали веру в лучшие времена, помогали переносить политические преследования, вселяли надежду, что «встанет мать-Украина, счастливая и свободная».



Когда треснула лед имперского абсолютизма, «развалилась руина», патриотические песни и песни-гимны воскресли, словно Феникс из пепла, окрылили всех, кто вышел на площади и улицы, чтобы заманифестировать свою волю жить в независимой, суверенной Украине. Над городами и селами родной земли зазвенели старые казацкие песни, гимны «Ще не вмерла Украина», «Не пора, не пора, не пора», песни, возникшие в начале 1900-х годов и связаны с развитием патриотического спортивно-противопожарного мочевого движения, песни украинских сечевых стрельцов - военных отрядов в составе австро-венгерской армии, которые были предтечей более поздних национальных сил Украины. В конце концов, начали набирать прав гражданства и песни Украинской повстанческой армии.



Никто уже не сможет заставить их замолчать! Это доля нелегкой истории, в которой сквозь призму народной души отразилось величие и духовная красота народа, его свободолюбивые стремления, его боль и трагедия.



Приходят новые времена, когда утверждается идея суверенности нашего народа, за которую столько веков шла священная борьба. По-новому звучат сегодня эти столько лет не певческие и преследуемые песни, но и сейчас, как никогда, согревают наши сердца святой любовью к Украине».



Среди самых известных патриотических песен - «Ще не вмерла Украина» П. Чубинского, «За Украину!» М. Вороного, «Ой у лузи красная калина» С. Чарнецкого, «Молитва за Украину» А. Конисского, «Не пора, не пора, не пора» И. Франко, «Эй, там на горе Янв идет», «Там на горе, на Макушке», «Подул ветер степной» и др.



Сл. О. КОНИССКОГО Муз. М. ЛЫСЕНКО



Молитва за Украину



Боже Великий, Единый,



Нам Украину храни,



Свободы и света лучами



Ты ее осени.



Светом науки и знания



Нас, детей, просвещения,



В чистой любви к краю



Ты нас, Боже, вырасти.



Молимся, Бог Единый,



Нам Украину храни.



Все твои ласки, щедроты



Ты на народ наш обрати,



Дай ему волю,



Дай ему судьбу,



Дай доброго света



. Счастья дай. Боже, народу



И многая, многая лета!



О эти прекрасные песни можно сказать словами из стихотворения Г. Рыльского:



Благословенна ты в веках.



Как солнце наше благовестное.



Как вещий белокрылый птица.



Печаль и радость наша, постное,



Что мужество будиш в сердцах,



Когда над краем туча виснет.



В песнях, рожденных новой эпохой, - вера в Украину, в духовную силу народа, призыв приобретать волю без крови и насилия.



ДМИТРИЙ ПАВЛЫЧКО



Песня



Вставай, Украина, вставай.



Єднай Черное море и Карпаты,



И свой переболений край



Не дай врагам разломать.



Вставай и здіймай знамя



Всечеловеческой согласия и любви,



Чтобы воле святой вино



Употребить без мести и крови!

