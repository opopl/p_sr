% vim: keymap=russian-jcukenwin
%%beginhead 
 
%%file slova.kartina
%%parent slova
 
%%url 
 
%%author_id 
%%date 
 
%%tags 
%%title 
 
%%endhead 
\chapter{Картина}
\label{sec:slova.kartina}

%%%cit
%%%cit_head
%%%cit_pic
%%%cit_text
...Малыш лежал на полу в своей комнате и читал книгу, когда снова услышал за
окном какое-то жужжание, и, словно гигантский шмель, в комнату влетел Карлсон.
Он сделал несколько кругов под потолком, напевая вполголоса какую-то весёлую
песенку. Пролетая мимо висящих на стенах \emph{картин}, он всякий раз сбавлял
скорость, чтобы лучше их рассмотреть. При этом он склонял набок голову и
прищуривал глазки.  — Красивые \emph{картины}, — сказал он наконец. — Необычайно
красивые \emph{картины}! Хотя, конечно, не такие красивые, как мои.  Малыш вскочил на
ноги и стоял, не помня себя от восторга: так он был рад, что Карлсон вернулся.
— А у тебя там на крыше много \emph{картин}? — спросил он.  — Несколько тысяч. Ведь я
сам рисую в свободное время. Я рисую маленьких петухов и птиц и другие красивые
вещи. Я лучший в мире рисовальщик петухов, — сказал Карлсон и, сделав изящный
разворот, приземлился на пол рядом с Малышом
%%%cit_comment
%%%cit_title
\citTitle{Карлсон, который живёт на крыше}, Астрид Линдгрен
%%%endcit

%%%cit
%%%cit_head
%%%cit_pic
%%%cit_text
Может это не так и плохо. Многие уверены, что такое постоянное видеонаблюдение
— один из важнейших факторов, обеспечивающих порядок и безопасность.  Но вот
беда, контроль стал распространяться не только на сферу физического.
Социальные сети блокируют контент, который, по их мнению, может вас расстроить.
Более того, делают они это не без нашей помощи. Прежде, чем что-то
опубликовать, нужно сто раз пропустить свои мысли и слова сквозь сито. Вы
можете быть сколь угодно деликатны и порядочны, но среди читателей обязательно
найдется тот, кто посчитает, что вы задели его чувства. А пожаловаться и
запретить сегодня гораздо проще, чем доказать, что в вашем сообщении не было
никакого злого умысла.  Да, все мы люди, можем ошибаться и ненароком кого-то
обидеть, но как быть с классической живописью, скульптурой, литературой? Когда
посты и даже аккаунты музеев и выставок блокируются из-за того, что в них якобы
можно увидеть изображения порнографического толка? И почему \emph{картины}
Модильяни наносят психологическую травму больше, чем отфотошопленное фото
очередной инстадивы в микроскопическом купальнике? Дело дошло до того, что
Туристический Совет Вены зарегистрировал аккаунт в OnlyFans — ресурсе, более
известном сегодня своим порнографическим контентом, хотя изначально он
создавался для публикации и поддержки самого разнообразного творчества, но без
цензуры
%%%cit_comment
%%%cit_title
\citTitle{Мир розовых пони / Статьи}, 
Варвара Фалеева, fraza.com, 03.11.2021
%%%endcit

