% vim: keymap=russian-jcukenwin
%%beginhead 
 
%%file 02_08_2021.fb.bilchenko_evgenia.5.kontr_kultura
%%parent 02_08_2021
 
%%url https://www.facebook.com/yevzhik/posts/4116909265010819
 
%%author Бильченко, Евгения
%%author_id bilchenko_evgenia
%%author_url 
 
%%tags bilchenko_evgenia,kontr_kultura,kultura
%%title БЖ. Контркультура - плохое слово?
 
%%endhead 
 
\subsection{БЖ. Контркультура - плохое слово?}
\label{sec:02_08_2021.fb.bilchenko_evgenia.5.kontr_kultura}
 
\Purl{https://www.facebook.com/yevzhik/posts/4116909265010819}
\ifcmt
 author_begin
   author_id bilchenko_evgenia
 author_end
\fi

БЖ. Контркультура - плохое слово?

Многие мои российские знакомые говорят (и в этом есть доля истины, но только
доля), что мне будет одинаково плохо и в Украине, и в России, потому что я -
"контркультура". Я, конечно, - давно уже не тот анархист, которым была в
юности, оно и понятно: если человек в двадцать - не революционер, у него нет
сердца, если человек в сорок - революционер, у него нет ума. Но, как любому
русскому поэту, определенная доля контркультурности мне естественно присуща, и
это побудило меня поделиться с читателями некоторыми своими размышлениями о
контркультуре.

Есть три простых правила, которые отличают настоящий протест от бунта на
продажу. Первое правило - протест должен быть бескорыстным. Второе правило -
протест должен быть направлен на вышестоящих, а не нижестоящих. Третье правило
- за протест расплачиваются, от него должны быть потери, а не приобретения. В
этом плане я искала контркультуру там, где её давно уже нет. А, может, никогда
и не было. 

\ifcmt
  pic https://scontent-cdg2-1.xx.fbcdn.net/v/t1.6435-9/230285136_4116909218344157_1883004818700029920_n.jpg?_nc_cat=104&ccb=1-3&_nc_sid=8bfeb9&_nc_ohc=SVgaFxWdEDcAX9BADZW&_nc_ht=scontent-cdg2-1.xx&oh=ebbcb5c2433ab384a1989f0606b81f14&oe=61303977
  width 0.4
\fi

Киевские хиппи, как любые хиппи, сопровождавшие меня в первый период
взросления, контркультуру уничтожили сами. Они сейчас благополучно являются
частью системы и подчинены либерально- националистическому тренду нашей власти.
На этом фоне стилизация бунтарства в виде мата и плевков на забор выглядит, как
дешёвый спектакль. Ребята давно в матрице и при этом играют в протест против
русских. За это им позволяют жить и периодически гладят по головке, как гладят
всю нашу творческую богему. Киевская площадь, ворвавшаяся в мою жизнь в 2014,
оказалась таким же спектаклем, только на порядок выше. Никакой контркультуры
нет и не может быть, если колония протестует не против метрополии, а ей в
угоду. Контркультурщикам там делать нечего. Не бывает Башлачева за печенье.
Пришлось уйти, трудно, долго и тяжело, пока вся степень псевдопротеста не вышла
наружу.

"Где же тогда контркультура?" - спросите вы. Да, Россия это. Россия. Если вы -
закоренелый контркультурщик и ищете систему покруче, скажите ваше "Нет!"
глобальному миру. Россия это сделала на уровне политики. Попробуйте вы на
уровне привата. Я попробовала. Что со мной сделали? Результат налицо. Да, это
не газовая камера в Треблинке и не гетто в Лодзи: это газовая камера моральной
травли и гетто полной социальной обструкции. Это - до оснований разрушенное
тело в сравнительно молодом возрасте.

Чем отличается моя жизнь от жизни жителей Лодзи? Пока - питанием. И всё.
Состояние здоровья, режим выхода на улицу в сопровождении, закрытость дома,
перспектива безработицы и постоянный страх, что на улице тебя будут
третировать, ничем от жёлтой звёзды на спине не отличаются. Только по форме, но
и мы в двадцать первом веке, а не в двадцатом. 

Я и представить себе не могла, что я так буду жить: без сцены, без кафедры, без
друзей, без врачей, без поездок, без прогулок. Просто - дома. Это, по сути,
тюрьма. Я знаю, что я попала в число тех, кого расстреливают показательно:
например, каждого десятого. Я просто им оказалась, этим десятым. Но и так
привыкаешь жить. Не думаю, что нынешние хиппи вообще знают, что такое
контркультура: пацифизм нынче в цветном тренде. Разве что ребята реально
полагают, что рэпер Моргенштерн - это "контркультура". Тогда мне сказать
нечего.  Тогда остаётся только тоскующей от "несвободы" лощеной молодежи Питера
и Москвы пожелать прожить один мой киевский день. 

Россия, несмотря на её объективную включенность в глобализм (что есть, то
есть), таки сказала это "Нет". И это есть контркультура в ее мировом масштабе.
Если украинцы думают, что они контркультурны к России, то Россия -
контркультурна к глобальному миру, инструментом которого являются сами
украинцы. Беда многих россиян, что они не понимают и не ценят своей
контркультурности, всё ещё ощущая себя властью, пытаясь говорить пафосно и с
апломбом. Они теряют тот шарм, который и делает контркультуру обаятельным
пристанищем для молодежи. Не хотят обновляться, чиновнически тормозят, понимаю.
Но. Как хочется им сказать: "Не будьте патриархами, будьте, как апостолы,
несите Егора Летова Всея Руси - это и есть живое!"

Повторюсь: во дворе - двадцать первый век. Не двадцатый. Контркультура больше
не ходит в клещах с "Голосом Америки" в наушниках. Мама, папа, пока. Либерализм
давно стал властью, системой, матрицей угнетения. Он заговаривается и впадает в
деменцию, повторяя одно и то же. Он жесток: настолько, насколько жесток палач
во время медленной пытки. Попробуйте на себе: утром вы проснетесь растерзанным
медиа, больным, без работы, со всеми девайсами на проверке спецслужб.
Контркультура теперь ходит в рясе, в толстовке, с православным рэпом в ушах,
если хотите, она ходит с крестом. Традиция теперь - контркультурна. Мамонов это
четко показал. 

Мальчики и девочки, бегущие вслед за Навальным, - это запоздалая часть системы
власти. Огромной, всесильной, беспощадной. Самое время людям традиции сорвать
маску протеста с этой матрицы угнетения. Самое время людям традиции выдернуть
из-под нее табуретку нонконформизма и того лучшего, что ему присуще: азарт,
задор, дерзость, свобода, вольнодумство, обаяние, модность, в конце концов,
жёсткость и крутизна. Потому что всё свежее, модное, крутое, яростное влечёт.
Молодое христианство влечёт. Рэп Саграды влечет. 

Впрочем, по большому счету, контркультура - так себе слово, потому что любовь -
выше контркультуры. И соборность выше. Просто возьмите на заметку. Точнее - на
вооружение.

\ii{02_08_2021.fb.bilchenko_evgenia.5.kontr_kultura.cmt}
