% vim: keymap=russian-jcukenwin
%%beginhead 
 
%%file 15_10_2020.news.ua.from_ua.1_ips_wagner_gate
%%parent 15_10_2020
%%url https://from-ua.com/obzor-pressi/570300-pochemu-u-telegrama-vagner-geit-luganskie-ip.html
 
%%endhead 

\subsection{Почему у Телеграма "Вагнер-гейт" луганские IP?}
\label{sec:15_10_2020.news.ua.from_ua.1_ips_wagner_gate}

\url{https://from-ua.com/obzor-pressi/570300-pochemu-u-telegrama-vagner-geit-luganskie-ip.html}

Некоторые следы ведут на оккупированные Россией украинские территории

На фоне резонанса темы вагнеровцев в Украине в сети появляются телеграм-каналы,
которые, по их словам, публикуют эксклюзивные документы и достоверные данные об
этой «как-бы операции» украинских спецслужб.

Таким каналом, например, является появившийся 11 октября "Вагнер-гейт", который
позиционирует себя как "воскрешение" ранее просуществовавшего всего один день
канала "Vagner". Буквально за несколько дней "Вагнер-гейт" смог набрать
довольно большую аудиторию как для украинских телеграм-каналов --- больше 11
тысяч подписчиков.

Как принято на подобных псевдоинформационных ресурсах, никаких данных про
создателей канала, каналов связи с ними или фактов, подтверждающих
достоверность размещаемой информации, нет. При этом, первичную рекламу и
раскрутку телеграм-каналу "Вагнер-гейт" дал телеграм-канал "Очки Баканова",
который в тот же в день перепостил с "Вагнер-гейта" материал со схемой рассадки
вагнеровцев в самолёте. Сами „Очки Баканова” интернет-сообщество связывает с
близким окружением министра МВД Украины.

Киберсообщество "За Независимую Украину" решило рассмотреть новый вагнер-канал
поближе. И вот что киберактивистам удалось узнать:

Регистрация самого канала и посты были проведены с разных IP-адресов,
предположительно через VPN.

Один из входов в канал и публикация осуществлена с очень интересного адреса IP
178.158.0.21, который ведет прямиком в оккупированный Луганск.

Данный диапазон адресов принадлежит интернет-провайдеру "Магеал", одному из
крупнейших поставщиков компьютерной техники и услуг интернета в т.зв. "ЛНР".

Если посмотреть детали данного IP адреса, можно узнать точное местоположение
компании, ответственного администратора сети и контактные данные для связи.

Теплицкий Евгений, 1988 года рождения, проживает в городе Луганск, Луганская
область. Работает системным администратором в ЧП "Магеал".

В 2019 году искал работу системного администратора в крупных городах Украины,
прежде всего, Киеве, Одессе, Днепре, Харькове. Важно отметить, что искал работу
он удаленно, без переезда в другой город. В "Магеал" работает с марта 2014.
Начало карьеры в компании приходится на время, когда спецслужбы РФ начали
операцию по захвату юго-востока Украины.

По большому счету, Евгений просто представитель обслуживающего персонала и
ответственный за сети "Магеал". Но вот сама компания очень интересная.

Во-первых, она одна из тройки самых больших провайдеров фейковой республики, а,
во-вторых, является поставщиком интернет-соединения в "государственные
структуры террористов ЛНР".

Информацию о «Магеале» можно получить также на «правительственных» сайтах
недореспублики. Это говорит о том, что компания приближена к фюрерам "ЛНР" и
находится под контролем местного МГБ.

Данный провайдер является фактически монополистом в предоставлении услуг
интернета, мобильной связи и IP телефонии на оккупированных территориях
Луганской области. Практически все линии связи там ведут на центральный канал
"Магеала", дальше канал идет в Ростовскую область и после вливается в общую
сеть РФ.


В фокусе

    Лечение становится для украинцев всё менее доступным
    Почему у Телеграма "Вагнер-гейт" луганские IP?	
    Хто з чим іде на вибори до Борщагівської ОТГ: Електоральний аналіз	
    Нелегальное оружие в Украине: эхо решений 2014 года	
    Бездействие власти, или как Турчинов Крым сдавал	

Предложения партнеров
Гіроскопічний еспандер Power Ball
Кистьовий гіроскопічний еспандер. Займася в будь яких умовах!
Настільна лупа з підсвіткою
Корисна для людей зі слабким зором, рукодільницям, ювелірам та іншим
Окуляри з регулюванням дальності
Окуляри з регулюванням діоптрій від -6 до +3
Опрос
Представителей какой политической силы вы поддержите на местных выборах?

    Слуга народа
    ОПЗЖ
    Батькивщина
    Европейская солидарность
    Голос
    За майбутне
    Партия Шария
    Радикальная партия Ляшко
    Сила и честь
    другая политсила
    буду поддерживать не политсилы, а конкретных кандидатов
    А когда у нас выборы?

Все опросы
Спецпроекты

    Украина в 2020 году: самые интересные прогнозы
    Украинцы 1991-2017

Система Orphus
Политика
Почему у Телеграма "Вагнер-гейт" луганские IP?
15.10.2020 18:59
Некоторые следы ведут на оккупированные Россией украинские территории

На фоне резонанса темы вагнеровцев в Украине в сети появляются телеграм-каналы, которые, по их словам, публикуют эксклюзивные документы и достоверные данные об этой «как-бы операции» украинских спецслужб.

Таким каналом, например, является появившийся 11 октября "Вагнер-гейт", который позиционирует себя как "воскрешение" ранее просуществовавшего всего один день канала "Vagner". Буквально за несколько дней "Вагнер-гейт" смог набрать довольно большую аудиторию как для украинских телеграм-каналов --- больше 11 тысяч подписчиков.

Как принято на подобных псевдоинформационных ресурсах, никаких данных про создателей канала, каналов связи с ними или фактов, подтверждающих достоверность размещаемой информации, нет. При этом, первичную рекламу и раскрутку телеграм-каналу "Вагнер-гейт" дал телеграм-канал "Очки Баканова", который в тот же в день перепостил с "Вагнер-гейта" материал со схемой рассадки вагнеровцев в самолёте. Сами „Очки Баканова” интернет-сообщество связывает с близким окружением министра МВД Украины.

Киберсообщество "За Независимую Украину" решило рассмотреть новый вагнер-канал поближе. И вот что киберактивистам удалось узнать:

Регистрация самого канала и посты были проведены с разных IP-адресов, предположительно через VPN.

Один из входов в канал и публикация осуществлена с очень интересного адреса IP 178.158.0.21, который ведет прямиком в оккупированный Луганск.

Данный диапазон адресов принадлежит интернет-провайдеру "Магеал", одному из крупнейших поставщиков компьютерной техники и услуг интернета в т.зв. "ЛНР".

Если посмотреть детали данного IP адреса, можно узнать точное местоположение компании, ответственного администратора сети и контактные данные для связи.

Теплицкий Евгений, 1988 года рождения, проживает в городе Луганск, Луганская область. Работает системным администратором в ЧП "Магеал".

В 2019 году искал работу системного администратора в крупных городах Украины, прежде всего, Киеве, Одессе, Днепре, Харькове. Важно отметить, что искал работу он удаленно, без переезда в другой город. В "Магеал" работает с марта 2014. Начало карьеры в компании приходится на время, когда спецслужбы РФ начали операцию по захвату юго-востока Украины.

По большому счету, Евгений просто представитель обслуживающего персонала и
ответственный за сети "Магеал". Но вот сама компания очень интересная.

Во-первых, она одна из тройки самых больших провайдеров фейковой республики, а,
во-вторых, является поставщиком интернет-соединения в "государственные
структуры террористов ЛНР".

Информацию о «Магеале» можно получить также на «правительственных» сайтах
недореспублики. Это говорит о том, что компания приближена к фюрерам "ЛНР" и
находится под контролем местного МГБ.

Данный провайдер является фактически монополистом в предоставлении услуг
интернета, мобильной связи и IP телефонии на оккупированных территориях
Луганской области. Практически все линии связи там ведут на центральный канал
"Магеала", дальше канал идет в Ростовскую область и после вливается в общую
сеть РФ.

Владелец компании --- Грибов Александр Георгиевич 1950 года рождения, инженер,
электромеханик, преподаватель физики в одном из техникумов Луганщины и успешный
бизнесмен, что очень сложно в реалиях недореспублик без крыши местного МГБ или
приезжих сотрудников ФСБ. Тем более с территории, которая контролируется
террористами как "ЛНР", так и "ДНР", спецслужбы РФ проводят свои мероприятия в
интернет-пространстве Украины.

Если посмотреть на биографию Грибова, то он владеет сетью успешных предприятий,
которые прямо и косвенно завязаны на руководящие органы террористической
организации "ЛНР". Ему принадлежат ЧП "Магеал", ООО "Кронос 1", ЧП «Юф альянс
нерухомість-Луганськ» и ООО «Луганськ-світар».

Причем каждая из компаний имеет внушительный стартовый капитал. Откуда такие
деньги у простого инженера пока остается загадкой.

Немаловажный факт: Александр Грибов баллотировался в депутаты Верховного Совета
Украины в далеком 1998 году от Партии Регионов. Данная информация до сих пор
публично размещена на ресурсах ЦИК.

В итоге получается, что некоторые следы телеграм-каналов, которые откровенно
дискредитируют правительство и спецслужбы Украины, ведут на оккупированные
Россией украинские территории. Ведь лучший VPN это тот, который ведет в
какую-то непризнанную республику, так как, они могут никому не подчиняться и
делать все, что в их интересах, как в просторах интернета, так и в жизни.

Но важно также понимать, что подобные вбросы попадают в украинское пространство
благодаря отечественным журналистам с колоссальной аудиторией. Получается, что
источник информации, которой доверяют топовые представители украинских
масс-медиа и которой они призывают доверять десятки тысяч своих читателей,
имеет гибридное происхождение?

Поэтому внимательно смотрим, что нам принесёт реанимированный телеграм-канал
"Вагнер-гейт" или во что нас опять будут пытаться уверовать!
