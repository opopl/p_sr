% vim: keymap=russian-jcukenwin
%%beginhead 
 
%%file 13_10_2021.fb.stomina_oksana.mariupol.1.muzei_golovnogo_upra.cmt
%%parent 13_10_2021.fb.stomina_oksana.mariupol.1.muzei_golovnogo_upra
 
%%url 
 
%%author_id 
%%date 
 
%%tags 
%%title 
 
%%endhead 

\qqSecCmt

\iusr{Natalia Kovtun}

\ifcmt
  igc https://i2.paste.pics/fd327d099d7161d5977b262812388f13.png
	@width 0.4
\fi

\iusr{Ольга Ковалева}

Были! Интересно!

\iusr{Александр Бондаренко}

Ничего не понимаю. Директором музея был Николай Николаевич, а гл. смотрителем -
Сергей Викторович, который и проводил экскурсии. Его на снимках нет, а директор
музея какая-то дама... 

\begin{itemize} % {
\iusr{Александр Бондаренко}

Директор он или она?

\iusr{Оксана Стомина}
\textbf{Александр Бондаренко} он. Во всяком случае так мне сказали

\iusr{Александр Бондаренко}

Тогда почему \enquote{экскурсию провела директор музея}? Или у полиции, по
советской традиции, все \enquote{шито-крыто}?

\iusr{Оксана Стомина}
\textbf{Александр Бондаренко} провів

\iusr{Оксана Стомина}

Аааа, у тебя, наверное, перевод на русский язык стоит?!

\iusr{Oleg Ukraintsev}
\textbf{Александр Бондаренко} Николай Николаевич не був директором музея)))!
\end{itemize} % }

\iusr{Александр Бондаренко}

Оксана, есть еще более богатый музей, в конце пр. Мира после ул. Зелинского -
музей Вооруженных сил, он еще богаче этого, я там был и писал материал в ПР. 

\begin{itemize} % {
\iusr{Оксана Стомина}
\textbf{Александр Бондаренко} я знаю, Саша. Но он всегда закрыт. Как туда можно попасть?

\iusr{Alex Rud}
\textbf{Оксана Стомина} Да он почти всегда закрыт, но созвонившись с хозяином попасть можно.

\iusr{Oleg Zavgorodny}
\textbf{Александр Бондаренко} Да, там интересно. Но это не сколько музей вооруженных сил, там шире тематика. И знамена, и форма, и разные артефакты.

\iusr{Александр Бондаренко}
\textbf{Оксана Стомина} У меня была их визитка, но где она сейчас не знаю

\iusr{Оксана Стомина}
\textbf{Александр Бондаренко} если соберётесь, возьмите и меня с собой! )

\iusr{Виталий Шол}

И меня))

\iusr{Александр Бондаренко}

Я туда не собираюсь, ищите подходы сами, а посмотреть можно много интересного.
Есть даже немецкие документы периода оккупации Мариуполя. Особо меня удивил
плакат тех времен, к чему-то призывавший горожан, на котором сзади стоял штамп
\enquote{Секретно} 

\end{itemize} % }

\iusr{Олена Сугак}

И я хочу! Это где?

\begin{itemize} % {
\iusr{Оксана Стомина}
\textbf{Елена Сугак} на Георгиевской. Там, где сгоревшее УВД

\iusr{Олена Сугак}
\textbf{Оксана Стомина} спасибо. Вычухаюсь, посещу.

\iusr{Оксана Стомина}
\textbf{Елена Сугак} скорейшего выздоровления!

\iusr{Олена Сугак}
\textbf{Оксана Стомина} спасибо. Лет 10 не болела...

\iusr{Оксана Стомина}
\textbf{Елена Сугак} а сейчас что случилось?
\end{itemize} % }

\iusr{Sergey Drovorub}

То там є плащ Жеглова?

\href{https://novosti.dn.ua/ru/zhuzhalka/252-v-maryupole-nashly-plashh-kapytana-zheglova-ego-otpravyat-v-muzey-polycyy}{%
В Мариуполе нашли плащ капитана Жеглова. Его отправят в музей полиции, novosti.dn.ua, 15.01.2019%
}

\ifcmt
  tab_begin cols=4,no_fig,center,separate

     pic https://i2.paste.pics/37b188daadc97ab776183a742ee0e188.png
		 pic https://i2.paste.pics/885f85c2a097ebc98ad69ca2b890c4ec.png
		 pic https://i2.paste.pics/a151c11dbdbe968421355bdba28114cf.png
		 pic https://i2.paste.pics/81223ed40528685ad6b02bbade1b9340.png

  tab_end
\fi

%https://www.facebook.com/permalink.php?story_fbid=1995520454075542&id=100008528200769

\begin{itemize} % {
\iusr{Оксана Стомина}
\textbf{Sergey Drovorub} так )
\end{itemize} % }

\iusr{Sergey Drovorub}

А! Побачив.

\begin{itemize} % {
\iusr{Оксана Стомина}
\textbf{Sergey Drovorub} 🙂
\end{itemize} % }
