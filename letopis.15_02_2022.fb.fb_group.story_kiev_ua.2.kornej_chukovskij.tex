% vim: keymap=russian-jcukenwin
%%beginhead 
 
%%file 15_02_2022.fb.fb_group.story_kiev_ua.2.kornej_chukovskij
%%parent 15_02_2022
 
%%url https://www.facebook.com/groups/story.kiev.ua/posts/1861898700673589
 
%%author_id fb_group.story_kiev_ua,majorenko_georgij.kiev
%%date 
 
%%tags kiev,kornej_chukovskij
%%title КОРНЕЙ ЧУКОВСКИЙ И "ОТЦЫ" КИЕВА
 
%%endhead 
 
\subsection{КОРНЕЙ ЧУКОВСКИЙ И \enquote{ОТЦЫ} КИЕВА}
\label{sec:15_02_2022.fb.fb_group.story_kiev_ua.2.kornej_chukovskij}
 
\Purl{https://www.facebook.com/groups/story.kiev.ua/posts/1861898700673589}
\ifcmt
 author_begin
   author_id fb_group.story_kiev_ua,majorenko_georgij.kiev
 author_end
\fi

КОРНЕЙ ЧУКОВСКИЙ И \enquote{ОТЦЫ} КИЕВА

Разбирая документы своих далеких предков, обнаружил личное дело Семена
Анреевича Неводовского, который после окончания юрфака киевского Университета
Святого Владимира в 1860 году служил в Канцелярии киевского
генерал-губернатора. А, как известно, эту должность тогда занимал Илларион
Илларионович Васильчиков. И вдруг я вспомнил строки Корнея Ивановича Чуковского
из стихотворения \enquote{Крокодил}:

\obeycr
Лишь один
Гражданин
Не визжал,
Не дрожал —
Это доблестный Ваня Васильчиков.
Он боец,
Молодец,
Он герой
Удалой...
\restorecr

Надо же, Чуковский в \enquote{Крокодиле} увековечил фамилию киевского генерал -
губернатора! Ведь, как известно, в стихах Корней Иванович часто использовал
фамилии конкретных людей.

\ii{15_02_2022.fb.fb_group.story_kiev_ua.2.kornej_chukovskij.pic.1}

Допустим, в сказке про Бибигона Чуковский упоминает о морском великане Курынде.
А в автобиографической повести \enquote{Серебряный герб} повествует о своей любви к
девочке по фамилии Курындина.

\ii{15_02_2022.fb.fb_group.story_kiev_ua.2.kornej_chukovskij.pic.2}

А как с Бармалеем получилось?

Вспоминает писатель Лев Успенский:

"Много лет назад Корней Иванович гулял по Петроградской стороне нашего города
(это такой район его) с известным художником Мстиславом Добужинским. Они вышли
на Бармалееву улицу.

— Кто был этот Бармалей, в честь которого целую улицу назвали? — удивился Добужинский.

— Я, — говорит Корней Иванович, — стал соображать. У какой-нибудь из императриц
XVIII века мог быть лекарь или парфюмер, англичанин либо шотландец. Он мог
носить фамилию Бромлей: там Бромлеи — не редкость. На этой маленькой улице у
него мог стоять дом. Улицу могли назвать Бромлеевой, а потом, когда фамилия
забылась, переделать и в Бармалееву: так лучше по-русски звучит…

Но художник не согласился с такой догадкой. Она показалась ему скучной.

— Неправда! — сказал он. — Я знаю, кто был Бармалей. Он был страшный разбойник.
Вот как он выглядел…

И на листке своего этюдника М. Добужинский набросал свирепого злодея,
бородатого и усатого…

Так на Бармалеевой улице родился на свет злобный Бармалей."

А я вдруг вспомнил песенку зверей из сказки Чуковского про доктора Айболита:

\obeycr
Шивандары, шивандары,
Фундуклей и дундуклей!
Хорошо, что нет Варвары!
Без Варвары веселей! 
\restorecr

Батюшки-светы! Знакомая фамилия! Иван Иванович Фундуклей был киевским
градоначальником и его Чуковский увековечил в своем произведении!

Наверняка эта идея пришла Корнею Ивановичу в голову, когда он прогуливался по
улице Фундуклеевской.

А при каком Генерал - губернаторе служил Фундуклей? Правильно, при Бибикове! 

Дмитрий Гавриилович Бибиков был героем Русско-турецкой и Отечественной войны
1812 года. А у Чуковского есть персонаж - отважный Бибигон. Тут видно
невооруженным глазом - Корней Иванович изменил три буквы и вместо Бибикова
появился Бибигон! Видать, мысль эта пришла к Корнею Ивановичу на Бибиковском
бульваре!

Спасибо Корнею Ивановичу Чуковскому - он сберег в произведениях память об
"отцах" Киева прошлых лет и этим завещал нам помнить историю родного города.
