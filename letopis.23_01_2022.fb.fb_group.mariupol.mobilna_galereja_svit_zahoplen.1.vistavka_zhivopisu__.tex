%%beginhead 
 
%%file 23_01_2022.fb.fb_group.mariupol.mobilna_galereja_svit_zahoplen.1.vistavka_zhivopisu__
%%parent 23_01_2022
 
%%url https://www.facebook.com/groups/148929512488552/posts/919305235450972
 
%%author_id fb_group.mariupol.mobilna_galereja_svit_zahoplen,kibkalo_natalia.mariupol.biblioteka.korolenko
%%date 23_01_2022
 
%%tags mariupol,mariupol.pre_war,isskustvo,vystavka,kartina,hudozhnik
%%title Виставка живопису - Ліза Павлова
 
%%endhead 

\subsection{Виставка живопису - Ліза Павлова}
\label{sec:23_01_2022.fb.fb_group.mariupol.mobilna_galereja_svit_zahoplen.1.vistavka_zhivopisu__}
 
\Purl{https://www.facebook.com/groups/148929512488552/posts/919305235450972}
\ifcmt
 author_begin
   author_id fb_group.mariupol.mobilna_galereja_svit_zahoplen,kibkalo_natalia.mariupol.biblioteka.korolenko
 author_end
\fi

23 січня в Мобільній галереї «Світ захоплень» \href{https://www.facebook.com/groups/1476321979131170}{Центральна міська публічна
бібліотека ім. В.Г. Короленка м. Маріуполь} відкрилась нова виставка живопису.
Творчі роботи представила юна художниця Ліза Павлова. В даний час дівчина
навчається в 7-му класі ЗОШ № 47, а з 2017 року освоює образотворче мистецтво в
Маріупольській школі мистецтв (викладач Олена Сидорова).  Олена Анатоліївна
вважає Лізу дуже перспективною ученицею. У свій вільний час дівчина полюбляє
читати книжки, їй подобаються твори класиків світової літератури. «Коли я
читаю, мені легко вигадувати сюжети майбутніх картин, а також представляти
нестандартні образи,» - розповідає юна художниця. Хоч зараз Лізу приваблюють
реалістичні пейзажи всесвітньо відомих художників, вона мріє про «свій»,
особливий стиль у малюванні та ілюстрування дитячих книжок.

Для своєї першої персональної виставки дівчина підготувала два десятки
яскравих, емоційних, дуже позитивних картин: серія жіночих портретів, милі
мультяшні персонажі, малюнки за темою знаків зодіаку та ін. На відкриття
виставки до Мобільної галереї прийшло чимало любителів образотворчого
мистецтва, батьки і друзі Лізи, а також однокласники разом із класним
керівником. Така велика група підтримки дуже важлива для починаючого художника!
Свято живопису прикрасив своїми гарними музичними імпровізаціями
піаніст-композитор Андрій Гончаренко. Для користувачів бібліотеки з
функціональними порушеннями зору велося тифлокоментування.

Виставка мальовничих робіт Лізи Павлової відкрита до 13 лютого, вхід вільний.
