% vim: keymap=russian-jcukenwin
%%beginhead 
 
%%file 22_11_2021.fb.uljanov_anatolij.1.maidan.4
%%parent 22_11_2021.fb.uljanov_anatolij.1.maidan
 
%%url 
 
%%author_id 
%%date 
 
%%tags 
%%title 
 
%%endhead 
\subsubsection{4}
\label{sec:22_11_2021.fb.uljanov_anatolij.1.maidan.4}

У любого исторического события есть последствия для каждой отдельно взятой
человеческой личности. Те, кто «здобулы» власть и статус по итогам майданов,
чувствуют себя сегодня хорошо, и совершенно искренне празднуют годовщину своей
победы. Вокруг них и для них открываются какие-то заведения, имеется культурная
программа, и классовая среда, позволяющая жить комфортно.

Единственное, что мешает этому комфорту – реальность, в которой сытый класс
окружен всем этим \sem{быдлом, ватой, совкодрочерами} и прочим большинством, без
которого невозможна уютная сытость воцарившейся кучки. Но и это более-менее
поправимо. В отличие от сытых и подкормленных горнов нашего «гражданского
общества», у большинства украинцев нет горна, размер которого позволил бы им
звучать столь же широко и на той же громкости – рассказать о реальности своей
жизни в сотворённой нашими выдающимися «управленцами» экономике.

Большинство молчит. Говорят за него. Отсюда – искажённая иллюзиями «говорящего
класса» картина реальности, где всё «не идеально, но не так уж плохо». У нас. В
ресторане. В кафе. Галерее. На пати. А тот алкаш? Так он же алкаш. Сам виноват.
Не виноваты только мы. У нас война. На нас напали. И кричим мы об этом из
кафетериев громче, чем украинский солдат из окопа.

Солдат не кричит. Потому, что в окопе, и занят. А занят и в окопе он потому,
что не может откосить. Во-первых, у него нет богатого папы, который может
занести в военкомат. Во-вторых, у него есть «чувство патриотического долга»,
которому не учат в арт-скулах. А в сельской школе учат. Вот наш селюк и в
окопе. А мы на капуче, и говорим ему спасибо из фейсбука. Как грохнут, назовём
героем.  

По той же причине народ не кричит – занят выживанием в мире, чьи ценности
воспевают эти достойные, лучшие, более качественные люди, чьё экономически
нежизнеспособное царство мы ещё будем отпевать.
