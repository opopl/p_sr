% vim: keymap=russian-jcukenwin
%%beginhead 
 
%%file 21_01_2020.fb.fb_group.story_kiev_ua.1.elegii_prodolzhenie.cmt
%%parent 21_01_2020.fb.fb_group.story_kiev_ua.1.elegii_prodolzhenie
 
%%url 
 
%%author_id 
%%date 
 
%%tags 
%%title 
 
%%endhead 
\subsubsection{Коментарі}

\begin{itemize} % {
\iusr{Марина Соболевская}
Слушайте, ну просто зачитаешься! И даже вещественные доказательства! Верю! Спасибо!)

\begin{itemize} % {
\iusr{Ирина Петрова}

Мариночка, Вы прям в теме театральной, как незабвенный КС :"Верю!" ))
Требования КИ - все подкреплять фотодокументами выполняю неукоснительн) @igg{fbicon.face.grinning.big.eyes} 

\end{itemize} % }

\iusr{Галина Гурьева Николенко}
Прелестно!

\iusr{Ирина Петрова}
Спасибо) @igg{fbicon.face.kissing.closed.eyes} 

\iusr{Tatiana Loganovska}
Очаровательно @igg{fbicon.heart.eyes}  и все "прообразы" виртуального общения существовали уже давным-давно - "нет ничего нового под луной" @igg{fbicon.smile} 

\begin{itemize} % {
\iusr{Ирина Петрова}
Благодарю! Дааа, Вы правы абсолютно! Даже древнегреческие драматурги не были новаторами в сюжетных линиях @igg{fbicon.face.monocle} 

\iusr{Елена Сидоренко}
Классика, спасибо за интересный рассказ! @igg{fbicon.heart.red}
\end{itemize} % }

\iusr{Ирина Архипович}
Грандиозная история!!! Какие интриги, сколько вещей задействовано, захватывающий сюжет !  @igg{fbicon.face.smiling.eyes.smiling} Зачиталась!!! Спасибо, браво!!!  @igg{fbicon.thumb.up.yellow}  @igg{fbicon.hands.applause.yellow}  @igg{fbicon.heart.eyes} 

\iusr{Леся Фандралюк}
Какой накал страстей!!!)) Кстати, Ваши папа и, особенно, мама-прекрасны!)

\begin{itemize} % {
\iusr{Ирина Петрова}
\textbf{Леся Фандралюк} спасибо большое! Их уже нет с нами тут, но, они живы в наших воспоминаниях)
\end{itemize} % }

\iusr{Алексей Плешко}
Браво!

\begin{itemize} % {
\iusr{Ирина Петрова}
\textbf{Алексей Плешко} благодарю! Приятно!)

\iusr{Алексей Плешко}
\textbf{Ирина Петрова} На бис!!

\iusr{Ирина Петрова}
\textbf{Алексей Плешко} будет, @igg{fbicon.thumb.up.yellow}  @igg{fbicon.face.grinning.smiling.eyes} 
\end{itemize} % }

\iusr{Петрова Ирина}

Фото Ваших родителей потрясающее. Такие люди могли быть только умными, добрыми,
понимающими и просто замечательными.

\begin{itemize} % {
\iusr{Ирина Петрова}
\textbf{Петрова Ирина} так чудесно видеть Иру Петрову )))) а Вы не Владимировна, случайно?)))

\iusr{Петрова Ирина}
Нет, я не Владимировна
\end{itemize} % }

\iusr{Светлана Манилова}

Ирочка, в свете вчерашних обсуждений попрошу обратить Ваше внимание, что в
группе есть не только Ирина Петрова, а и Петрова Ирина... Это Вам не Вадим-
Дмитрий. Тут полное совпадение по имени и фамилии.  @igg{fbicon.wink} А кроме шуток, спасибо за
элегию детства. @igg{fbicon.thumb.up.yellow}  

\begin{itemize} % {
\iusr{Ирина Петрова}
\textbf{Светлана Манилова} Светочка, ведь Киев - город магический, тут всё не просто так, всё с тайным смыслом! Ир Петровых много не бывает!))) За похвалу спасибо! Греет душу!)
\end{itemize} % }

\iusr{Ирина Мазур}

Очень интересно читать..." требую продолжения праздника!" @igg{fbicon.heart.eyes} 

\begin{itemize} % {
\iusr{Ирина Петрова}
\textbf{Ирина Мазур} да, Ирочка, будет, конечно) спасибо за одобрение!

\iusr{Ирина Мазур}
\textbf{Ирина Петрова} Спасибо! @igg{fbicon.heart.sparkling} 
\end{itemize} % }

\iusr{Светлана Хоменко}
Браво! Шедеврально! @igg{fbicon.hands.applause.yellow}  @igg{fbicon.thumb.up.yellow} 

\iusr{Ирина Петрова}
\textbf{Світлана Хоменко} спасибо! Такие похвалы очень приятны!)

\iusr{Ольга Зубенко}
Ох ці жіночки!! @igg{fbicon.hands.applause.yellow}{repeat=3} 

\iusr{Ирина Петрова}
\textbf{Olga Zubenko} це кияночки, а вони такі ще штучки)

\iusr{Светлана Елсукова}
Очень хороши и папочка и мамочка Валюша...

\iusr{Ирина Петрова}
\textbf{Светлана Елсукова} спасибо, мне очень приятны добрые слова о любимых людях  @igg{fbicon.face.smiling.eyes.smiling} 

\iusr{Светлана Иванова}
Как же хорошо написано...!

\iusr{Ирина Петрова}
\textbf{Гончаренко Светлана} я очень рада, что скромный труд приносит минуты радости людям)

\iusr{Natasha Levitskaya}

Браво, Ирина! Хорошо, с настроением написано и с хорошим чувством юмора,
присущим вам! Вчера убедилась "по полной"!!!  @igg{fbicon.laugh.rolling.floor}  @igg{fbicon.grin}  Мама и папа - красивые люди!

\begin{itemize} % {
\iusr{Ирина Петрова}
\textbf{Natasha Levitskaya} даааа уж, вчера мы повеселились от души! @igg{fbicon.laugh.rolling.floor}  И вот опять же - ничего не придумано, режиссером была сама жизнь, а мы просто успевали за ней конспектировать! Спасибо, Наташа, за соучастие!
\end{itemize} % }

\iusr{Наталия Платонова}
Человечно, тепло, чудесно...

\begin{itemize} % {
\iusr{Ирина Петрова}
\textbf{Наталия Платонова} тепло идёт от теплых воспоминаний . Близкие люди, родные, знакомые, хорошие соседи живут в уголочках души и сердца) Спасибо Вам за прочтение! @igg{fbicon.face.smiling.eyes.smiling} 
\end{itemize} % }

\iusr{Tetiana Petrovska}

Да уж! @igg{fbicon.hands.applause.yellow} 

\iusr{Ирина Петрова}
\textbf{Tetiana Petrovska} этот дом был необычайным) как, впрочем, и сотни киевских домов @igg{fbicon.face.smiling.eyes.smiling} 

\iusr{Ирена Шведова}
Спасибо огромное! Написано с юмором и теплом. Браво! Прочитала с интересом и улыбкой. СПАСИБО!!!!

\begin{itemize} % {
\iusr{Ирина Петрова}
\textbf{Ирена Шведова} мне очень приятно, что многим мои воспоминания приносят улыбку) ведь это продлевает жизнь!) Вам спасибо за соучастие и теплые слова!)

\iusr{Ирена Шведова}
\textbf{Ирина Петрова}

\ifcmt
  ig https://scontent-frt3-1.xx.fbcdn.net/v/t1.6435-9/83359390_192292182086112_9073041850712981504_n.jpg?_nc_cat=108&ccb=1-5&_nc_sid=dbeb18&_nc_ohc=FYKzt6vJq3IAX_8feRe&_nc_ht=scontent-frt3-1.xx&oh=e272a9d07e8d84c94a9d8ec62a6d69e0&oe=61B527D8
  @width 0.25
\fi

\end{itemize} % }

\iusr{Wiktoria Arto}

\ifcmt
  ig https://scontent-frx5-2.xx.fbcdn.net/v/t39.1997-6/p240x240/851584_396469193830520_436712995_n.png?_nc_cat=1&ccb=1-5&_nc_sid=0572db&_nc_ohc=rQ6jgBuFJ6IAX8MQTV0&tn=lCYVFeHcTIAFcAzi&_nc_ht=scontent-frx5-2.xx&oh=a3751f56b9ac8b079e2b55c8b499cef0&oe=61931A84
  @width 0.2
\fi

\iusr{Ирина Петрова}
\textbf{Wiktoria Arto} благодарю! @igg{fbicon.face.happy.two.hands}  @igg{fbicon.bouquet} 

\iusr{Wiktoria Arto}
Супер маленький рассказ!

\iusr{Wiktoria Arto}

Автор с юмором, легко, непринужденно описала судьбу женщины. Не перестаёшь
удивляться. Спасибо за тепло, добро, любовь которые пронизывают рассказ.

\begin{itemize} % {
\iusr{Ирина Петрова}
\textbf{Wiktoria Arto} Жизнь щедрой рукой раздает сюжеты, только успевай записывать! @igg{fbicon.face.smiling.eyes.smiling} 

\iusr{Wiktoria Arto}
А тут и «рояль в кустах» - талант рассказчика. Точнее писательский талант. Повествование достойно Зинаиды Гиппиус или Тэффи.

\iusr{Ирина Петрова}
Смущению моему границ нет)))) благодарю, тщеславие довольно)
\end{itemize} % }

\iusr{Виктор Крицкий}
Это было на самом деле?

\begin{itemize} % {

\iusr{Ирина Петрова}
\textbf{Виктор Крицкий} А как бы хотелось Читателю? @igg{fbicon.face.grinning.smiling.eyes} 
\end{itemize} % }


\iusr{Nadiya M Shana}
Папа, как герой из фильма, мама - модница и красавица.  @igg{fbicon.heart.eyes} 

\begin{itemize} % {
\iusr{Ирина Петрова}
\textbf{Nadiya M Shana} спасибо, дорогая, да, теперь вот уже старше тех пап и мам на снимках, смотришь и понимаешь, какими они были)
\end{itemize} % }

\iusr{Надежда Маслак}
У каждого свои таланты

\iusr{Наталия Платонова}
Удивительно папа Володя похож на моего папу! Видимо, такой Киевский типаж...

\begin{itemize} % {
\iusr{Ирина Петрова}
\textbf{Наталия Платонова} 

наверное, Вы правы) я часто хожу на Печерск, в парки, "по местам детства".
Иногда вижу старичка, удивительно похожего на моего папу в 90-х... так сердце
дрогнет...

\iusr{Наталия Платонова}
\textbf{Ирина Петрова} 

А я помню в этих парках Киевских Дам( бабушка жила на Печерске), в чесучевых
костюмах, шелковых блузках и обязательно в шляпках. Когда много лет спустя я
увидела там же этих же дам, была ошеломлена. Потом поняла: это дочери тех Дам!.
Их всё меньше, и уже нет чесучи, но попадаются похожие, сохранившие тот дух...

\iusr{Ирина Петрова}
\textbf{Наталия Платонова} верно! Есть ещё Дамы в Парках! Летними вечерами в Ракушке на симфонических концертах видела))))
\end{itemize} % }

\iusr{Ирина Шмидт}
Интересно

\iusr{Вера Милашич}
Спасибо, почитала с удовольствием! @igg{fbicon.bouquet} 

\begin{itemize} % {
\iusr{Ирина Петрова}
\textbf{Вера Милашич} благодарю, рада, что ещё одна минутка удовольствия появилась в Вашем дне)
\end{itemize} % }

\iusr{Andrei Dvoynos}
Спасибо, у Вас талант!

\begin{itemize} % {
\iusr{Ирина Петрова}
\textbf{Andrei Dvoynos} Похвала автору - бальзам на душу! А Вам спасибо за фотографию обложки профиля. Мой любимый, второй после Долбычки, родной и любимой, пляж!
\end{itemize} % }

\iusr{Людмила Хохол}

Остроумно. Вот как в прошлом, могли развлекаться наши родители, дедушки и
бабушки. А главное был результат.

\begin{itemize} % {
\iusr{Ирина Петрова}
\textbf{Ludmila Chochol} прадедушки, дедушки, прабабушки, бабушки, родители, мы, наши дети, наши внуки - мы все те же люди, с теми же эмоциями, недостатками и достоинствами))) ничто не ново)
\end{itemize} % }

\iusr{Алексей Плешко}

\ifcmt
  ig https://scontent-frx5-2.xx.fbcdn.net/v/t39.1997-6/s168x128/64555015_1231656060339077_207158783925813248_n.png?_nc_cat=1&ccb=1-5&_nc_sid=ac3552&_nc_ohc=Gi-b27LfSc8AX9t2fXV&_nc_ht=scontent-frx5-2.xx&oh=1b70e567a7450492846251ca3f2820ad&oe=619336CE
  @width 0.15
\fi

\begin{itemize} % {
\iusr{Ирина Петрова}
\textbf{Алексей Плешко} спасибо, по современной шкале школьных оценок это 5 с минусом нашего детства, но, все же 5!!!)
\end{itemize} % }

\iusr{Виктор Крицкий}
Она розрушала чужую семью, не хорошо!

\begin{itemize} % {
\iusr{Ирина Петрова}
\textbf{Виктор Крицкий} да, увы, ах, как свидетель событий в оправдание соседки скажу, что семья осталась монолитной) ох, уж эти мужчины(((( он, конечно, абсолютно невиновен!

\iusr{Виктор Крицкий}
\textbf{Ирина Петрова} инициатива была с ее стороны!

\iusr{Ирина Петрова}
\textbf{Виктор Крицкий} понимаете, я была лет 10-ти, подробности, понятно, мне не ведомы. Уже по опыту взрослой жизни, могу утверждать, что в таких ситуациях односторонней инициативы не бывает)

\iusr{Виктор Крицкий}
\textbf{Ирина Петрова} согласен

\iusr{Ирина Петрова}
\textbf{Виктор Крицкий} у Вас не характер, а бархат @igg{fbicon.thumb.up.yellow}  @igg{fbicon.face.tears.of.joy} 

\iusr{Виктор Крицкий}
\textbf{Ирина Петрова} согласен
\end{itemize} % }

\iusr{Светлана Солдатенко}

\ifcmt
  ig https://scontent-frx5-2.xx.fbcdn.net/v/t39.1997-6/s168x128/47192743_1074619842699448_6852476323399467008_n.png?_nc_cat=1&ccb=1-5&_nc_sid=ac3552&_nc_ohc=-5JlsKfFEYcAX9_WDJj&_nc_ht=scontent-frx5-2.xx&oh=9a3e22645f279750e9921322e303603e&oe=6192D78E
  @width 0.15
\fi

\begin{itemize} % {
\iusr{Ирина Петрова}
\textbf{Svetlana Soldatenko}  @igg{fbicon.face.happy.two.hands}  @igg{fbicon.wink} 
\end{itemize} % }

\iusr{Irina Popova}
Душа оттаивает от таких милых, жизненных зарисовок, спасибо

\begin{itemize} % {
\iusr{Ирина Петрова}
\textbf{Irina Popova} очень точное и правильное слово ",зарисовок", именно они, не рассказы, не повести, даже не фельетоны) Спасибо!
\end{itemize} % }

\iusr{Elvis Vladimir}
Прям Осенний марафон

\begin{itemize} % {
\iusr{Ирина Петрова}
\textbf{Elvis Vladimir} а ведь верно! Именно, осенний, на весну героя уже не хватило))))

\iusr{Elvis Vladimir}
\textbf{Ирина Петрова} двойное совпадение ?)

\iusr{Ирина Петрова}
\textbf{Elvis Vladimir} )))) так!
\end{itemize} % }

\iusr{Igor Kurinnoy}

\ifcmt
  ig https://scontent-frt3-2.xx.fbcdn.net/v/t39.1997-6/s168x128/64539451_2575251802485583_6812212130280898560_n.png?_nc_cat=101&ccb=1-5&_nc_sid=ac3552&_nc_ohc=A0arstOoOGIAX8Xt2rq&_nc_ht=scontent-frt3-2.xx&oh=d597a100d618b09f868a5ab7c0d85f6e&oe=6193A57F
  @width 0.15
\fi

\end{itemize} % }
