% vim: keymap=russian-jcukenwin
%%beginhead 
 
%%file 22_02_2022.fb.menendes_enrike.1.sobratsja_s_mysljami
%%parent 22_02_2022
 
%%url https://www.facebook.com/e.menendes/posts/7068469489862083
 
%%author_id menendes_enrike
%%date 
 
%%tags dnr,donbass,lnr,minsk_dogovor,rossia,ukraina
%%title Нужно собраться с мыслями по поводу вчерашних событий
 
%%endhead 
 
\subsection{Нужно собраться с мыслями по поводу вчерашних событий}
\label{sec:22_02_2022.fb.menendes_enrike.1.sobratsja_s_mysljami}
 
\Purl{https://www.facebook.com/e.menendes/posts/7068469489862083}
\ifcmt
 author_begin
   author_id menendes_enrike
 author_end
\fi

Нужно собраться с мыслями по поводу вчерашних событий. Попробую разложить всё
по полочкам. Напоминаю, что это исключительно мои личные переживания и моё
личное мнение. А я человек Донбассу не чужой, стоит меня послушать.

Итак, то, что вчера произошло это водораздел. Это нельзя игнорировать. С одной
стороны, я, конечно, рад за жителей ОРДЛО. Они натерпелись за 8 лет и достойны
того, чтобы испытывать позитивные эмоции. Люди хотят мирной жизни, я их
прекрасно понимаю и мои симпатии всегда будут на их стороне. Донецк и Луганск –
моё сердце всегда с вами. Если ваша жизнь станет проще и безопаснее – я буду
рад. С другой стороны, я не уверен, что признание со стороны России
автоматически гарантирует высокий уровень жизни. Это не лучший сценарий для
Донбасса и любой адекватный человек это понимает.

Второй момент. Ответственность за полный провал мирных переговоров по Донбассу
лежит на политических элитах Украины. Без сомнения. Довыделывались бляди. Нужно
было следовать Минским соглашениями и ничего этого бы не случилось. Лично я
остаюсь сторонником этого пути в урегулировании конфликта как единственно
правильного.

Но ситуация сложилась именно таким образом и все размышления это уже в пользу
бедных. Нужно думать, что дальше.

Первое. Возможность большой войны не уменьшилась, а увеличилась. Вчера все
слушали Путина. Его речь это не про Донбасс, это про всю Украину, а значит
признание - это только первый шаг, что-то должно быть следом. Для России
признание Л/ДНР это стратегическое поражение, потому что глобальная битва за
Украину становится сложнее. Если быть более конкретным, то признание Л/ДНР это
разгромное поражение России на украинском направлении. Теперь даже лояльные к
России украинцы лишены выбора. А Россия может добиться успеха только силой, со
всеми соответствующими издержками.

Второе. У России сузилось пространство вариантов для развития ситуации. У неё
теперь есть конкретные обязательства перед Л/ДНР. Если обстрелы продолжатся,
Россия должна реагировать. А если нам прилетает оттуда и нам приходится
отвечать, как отреагируют теперь в РФ? Войной с Украиной? Стрелять в ответ по
украинскому Донбассу, где раньше поддерживали Россию? А что будет теперь, когда
там начнут гибнуть мирные от российских снарядов?

Третий пункт. Реакция Запада. Тут всё понятно, вчера российский рынок акций
рухнул на 12\% впервые с 2008 года. Приготовитесь - это только начало. Очень
важна будет реакция Китая, ждём новостей сегодня. Надеюсь, Европа соберёт свои
яички в кулак и скажет что-то дельное.

Теперь перейдём к реакции Украины. Все мои подписчики знают мою позицию. Я
всегда выступал за урегулирование конфликта на основе минских соглашений, за
дружбу с Россией и нейтральную Украину. Вчерашний день изменил мою позицию.
Нет, я не стал против России и не стал противником Минска. Но теперь у нас нет
выбора. А мы – это Украина.

Я искренне считаю, что украинские власти должны сделать следующие шаги:

1. Немедленный разрыв дипломатических отношений с РФ. Высылка всех российских
дипломатов.

2. Остановка транзита российского газа через украинскую ГТС.

3. Обращение к ЕC с просьбой заморозить работу Северного потока.

4. Фиксация границы с Л/ДНР.

5. Полная заморозка всех социальных выплат и пенсий на неподконтрольных
территориях Украины до выяснения обстоятельств.

6. Введение режима ЧС на территории Донецкой и Луганской областей.

7. Все, кто хочет уехать из Л/ДНР в Украину должны иметь зелёный коридор. Нужны
государственные программы поддержки таких людей.

8. Никаких военных действий. Украина не стреляет! Мы только за мирное решение
конфликта!

Россия так хочет, чтобы Украина любила её. И я любил Россию до вчерашнего дня.
Верил, что они хотят мира и добра Донбассу.

Они обманули меня. Теперь я хочу справедливости.

\ii{22_02_2022.fb.menendes_enrike.1.sobratsja_s_mysljami.cmt}
