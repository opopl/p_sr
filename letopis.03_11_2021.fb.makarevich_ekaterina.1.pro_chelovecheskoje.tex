% vim: keymap=russian-jcukenwin
%%beginhead 
 
%%file 03_11_2021.fb.makarevich_ekaterina.1.pro_chelovecheskoje
%%parent 03_11_2021
 
%%url https://www.facebook.com/ekmakarevich/posts/4761513833913048
 
%%author_id makarevich_ekaterina
%%date 
 
%%tags chelovek,dusha,language,psihologia
%%title Про человеческое
 
%%endhead 
 
\subsection{Про человеческое}
\label{sec:03_11_2021.fb.makarevich_ekaterina.1.pro_chelovecheskoje}
 
\Purl{https://www.facebook.com/ekmakarevich/posts/4761513833913048}
\ifcmt
 author_begin
   author_id makarevich_ekaterina
 author_end
\fi

Чуть-чуть и я говорю там про человеческое (после середины). 

Ребята вырезали одну историю из моих слов, поэтому здесь ее расскажу. 

Несколько лет назад, когда я ездила в Страсбург, еще когда была журналистом,
там собрали журналистов из разных стран - Армении, Азербайджана, Грузии, а
также из Украины и России. 

\href{https://www.youtube.com/watch?v=x9h_wUSdHkk}{%
Укpaинa: xитpocть Кучмы, кoppупция Пopoшeнкo, peфopмы Зeлeнcкoгo | Кpым, Дoнбacc, Мaйдaн, %
varlamov, youtube, 02.11.2021%
}

\ifcmt
  pic https://external-mxp1-1.xx.fbcdn.net/safe_image.php?d=AQH2m3rHS6VU7wUO&w=500&h=261&url=https%3A%2F%2Fi.ytimg.com%2Fvi%2Fx9h_wUSdHkk%2Fmaxresdefault.jpg&cfs=1&ext=jpg&_nc_oe=6f000&_nc_sid=06c271&ccb=3-5&_nc_hash=AQFXNjvlFpO8C2sa
  @width 0.7
\fi

С лекцией приехала выступать  спикер из России, из журналистского сообщества, с
темой "Язык вражды и язык мира". И вот она говорит, я ее слушаю и понимаю, что
ее слова в сообществе журналистов вызывают провокационные действия. То есть
слова она говорит обычные, но вот результат этих слов отнюдь не к миру
приводит, а наоборот к противоположному. Там все чуть не перессорились из-за
одной злободневной темы, не помню какой, но кстати, само слово - "на злобу дня"
говорит само за себя. Но тогда значит, что есть добродневные темы?) 

После лекции, я помню, подошла к украинским ребятам, спросила их, почувствовали
ли они специфику энергии от слов. Они подтвердили. То есть мы - и спикер, и
ребята, и я, - говорили на русском, но говорили по-разному. На разных языках.
Была разная семантическая окрашенность у слов, была разная эмоциональная
палитра. Значит, дело не в языке, а в том, ЧТО и КАК говорится на этом языке. 

Тогда у меня даже возникла идея провести лингвистическое исследование о
семантике языка мира. Потому что тогда, только переехав в Украину, и решив
постепенно отходить от политики, я серьезно задумалась о том, что наши слова
больше национального языка. Есть гораздо более глубокий язык. Его можно назвать
человеческим, языком мирности или любви. Но он отличается. У него другая
степень твердости. Он более легкий, невесомый, не бьет наотмашь, а словно
обнимает, трогает струны души и потому что-то преображает. И этот язык
отличается от политического, бизнесового, философского, он просто другой. 

Тому исследованию не суждено было появиться, хотя писала что-то типа заявки на
грант. А было бы здорово, потому что это предопределило бы многие сегодняшние
проблемы в мире, и уверена, сделало бы огромные открытия в сфере продуктивного
общения между людьми и между странами. Но видимо, так должно было случиться,
поскольку сейчас именно через Логос, когда снова внутри меня возникла
потребность в этом различении и возвращении на уровне слов к языку мирности,
этим и занимаюсь. Правда, пока в одиночку. 

Но та история впервые меня побудила задуматься о том, что в языке важны как и
произносимые слова, так и то внутреннее состояние, при котором эти слова
говорятся. Можно говорить на разных языках национальных, но понимать друг друга
и тем самым создавать в атмосфере мирность. Вот этот неуловимый язык
человечности так хочется вернуть в атмосферу мира...кто бы знал. Но пока что
имеем. 
