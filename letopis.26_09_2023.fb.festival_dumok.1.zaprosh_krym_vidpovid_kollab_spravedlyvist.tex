%%beginhead 
 
%%file 26_09_2023.fb.festival_dumok.1.zaprosh_krym_vidpovid_kollab_spravedlyvist
%%parent 26_09_2023
 
%%url https://www.facebook.com/opinionfestival/posts/pfbid02MwyDdXmeCvwQ7Licg9QtA5NqsDPRPzRYrkSpgXRMP6kJ6WuhNCRyzMFG5j6cU6o3l
 
%%author_id festival_dumok
%%date 26_09_2023
 
%%tags 
%%title Запрошуємо на обговорення "Чи може кримінальна відповідальність за колабораціонізм відновити справедливість?"
 
%%endhead 

\subsection{Запрошуємо на обговорення "Чи може кримінальна відповідальність за колабораціонізм відновити справедливість?"}
\label{sec:26_09_2023.fb.festival_dumok.1.zaprosh_krym_vidpovid_kollab_spravedlyvist}

\Purl{https://www.facebook.com/opinionfestival/posts/pfbid02MwyDdXmeCvwQ7Licg9QtA5NqsDPRPzRYrkSpgXRMP6kJ6WuhNCRyzMFG5j6cU6o3l}
\ifcmt
 author_begin
   author_id festival_dumok
 author_end
\fi

📍Запрошуємо на обговорення \enquote{Чи може кримінальна відповідальність за колабораціонізм відновити справедливість?}

📍Коли? 30 вересня, 14:45-16:15. 

📍Де? Київ, Національний музей історії України у другій світовій війні, монумент Батьківщина-Мати. 

🚨Після початку повномасштабної збройної агресії рф проти України у
Кримінальному Кодексі з'явилася стаття 111-1, що визначила відповідальність за
колабораційну діяльність.

🚨За даними Офісу Генерального Прокурора України, станом на 19 вересня 2023
року зареєстровано 6 346 таких злочинів і винесено близько 700 вироків,
більшість – обвинувальні. Пропонуємо поговорити про те, чи відповідає практика
притягнення до відповідальності суспільним очікуванням і чи є вона
справедливою, а також який вплив має переслідування за колабораціонізм на
процеси відновлення звільнених територій?

📢Дискусія підготовлена в партнерстві з Центром прав людини ZMINA. Центр прав
людини та Українська Гельсінська спілка з прав людини - УГСПЛ.

🎤 Спікери та спікерки:

Онисія Синюк – правова аналітикиня Центру прав людини ZMINA;\par
Олександра Козорог – керівниця освітнього департаменту УГСПЛ;\par
Максим Єлігулашвілі – експерт ІМіП, секретаріат Коаліції \enquote{Україна. П'ята ранку};\par
Олександр Клюжев – експерт IFES;\par
Ноель Калгун – заступниця голови Моніторингової місії ООН із прав людини. \par
📢 Модераторка – Альона Горова, членкиня правління ГО \enquote{Інститут миру і порозуміння} (ІМіП).\par
