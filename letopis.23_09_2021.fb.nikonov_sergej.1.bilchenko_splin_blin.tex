% vim: keymap=russian-jcukenwin
%%beginhead 
 
%%file 23_09_2021.fb.nikonov_sergej.1.bilchenko_splin_blin
%%parent 23_09_2021
 
%%url https://www.facebook.com/alexelsevier/posts/1587790531566256
 
%%author_id nikonov_sergej,bilchenko_evgenia
%%date 
 
%%tags bilchenko_evgenia,donbass,literatura,poet,poezia,rossia,rusmir,travlja,ukraina,vojna
%%title БЖ. Сплин, блин!
 
%%endhead 
 
\subsection{БЖ. Сплин, блин!}
\label{sec:23_09_2021.fb.nikonov_sergej.1.bilchenko_splin_blin}
 
\Purl{https://www.facebook.com/alexelsevier/posts/1587790531566256}
\ifcmt
 author_begin
   author_id nikonov_sergej,bilchenko_evgenia
 author_end
\fi

Российская поэтесса Стефания Данилова поддерживает Евгению Бильченко и других
поэтов. 

БЖ. Сплин, блин!

И высказала эта девочка очень умную мысль, потому что от сопливой печали ничего
не видевших в жизни поэтов, на лице у меня появляется добрая ироничная улыбка и
мне хочется сказать им: "Дети, не гневите Бога".

Стефания Данилова сказала:

"Читаю стихи Ваши, даже навскидку, из интернета. Много трагики, много боли,
извините. Как возникает у российских поэтов вот эта «печаль из ниоткуда»?
Прекрасная творческая жизнь, масса выступлений, масса поклонников и обратной
связи, когда, кажется, ни на секунду не побудешь один/одна… «до самого дна»,
как у Янки, как у других поцелованных сумерками поэтесс…


\ifcmt
  tab_begin cols=2

     pic https://scontent-lga3-1.xx.fbcdn.net/v/t1.6435-9/242493535_1587789658233010_3832149768268842317_n.jpg?_nc_cat=101&ccb=1-5&_nc_sid=730e14&_nc_ohc=lvcVor8bUcwAX_xb9jA&_nc_oc=AQlAQNyhyA4O4ZFKe6pQGDYzVDapAs2P_KeTpx6he2v_Truq5cYMsx77ZQHah4HmVAs&_nc_ht=scontent-lga3-1.xx&oh=70920ad1c522917835216a56aec20348&oe=61711E77

     pic https://scontent-lga3-1.xx.fbcdn.net/v/t1.6435-9/242621343_1587789641566345_1039814554441658919_n.jpg?_nc_cat=110&ccb=1-5&_nc_sid=730e14&_nc_ohc=u1A5XxEO5wQAX_m8d2j&_nc_ht=scontent-lga3-1.xx&oh=82d3c83a3c78a73e1195b451d5e32824&oe=6173036E

  tab_end
\fi

Это Русь-матушка так калечит своих дочерей, или это русские вчерашние дети
несчастливы на Родине? Огромным пластом больного вдохновения для меня послужили
события моего наркоманского двора на окраине Ленинского проспекта. В частности,
местный наркоман Витька ударил мою бабушку лицом об ступени и отобрал пенсию,
вследствие чего она осталась инвалидом на всю жизнь. Мой «Больничный блюз»,
«Боль» и немало других стихотворений берут начало

именно в этой луже крови в подъезде, которую я не могу забыть, хотя моей
бабушки уже давно нет в живых. Я не хочу искать виноватых. Все равно мне никто
бабушку не вернет".

От себя добавлю.

Поэт Лена Касьян - в тяжелейших муках умирала от рака и за несколько часов до
смерти уже под морфием написала читателям "Все ещё с вами".

Поэт Анна Ревякина - не уехала из Донецка, не раз попадала под бомбежки, чуть
не умерла от короны, родила второго ребенка во время войны. Нянчит дитя под
пулями. Пишет о любви.

Поэт Анна Долгарева - военкор, волонтёр, пережила ад на Донбассе, потерю Родины
Украины, вдова двух ребят, пишет о Боге и русском космосе. Потеряла на войне
близких.

Поэт БЖ - лишена работы, профессуры, имени, сцены, выступлений (всех),
репутации, Родины, друзей, книгоиздания, денег. Все дни проводит дома. На улице
подвергается нападениям. Работает литнегром. Круглосуточно. Иногда на улице. На
выживание. Больна хронической физической болезнью. Приговорена Украиной к
травле и принудительному лечению в психлечебнице за любовь к России. Пишет о
Христе преимущественно, иногда срывается на маузеры.

Молодые поэты с хорошими лицами, вам капучино не хватает или панакоты?
