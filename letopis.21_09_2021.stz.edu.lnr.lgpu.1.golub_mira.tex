% vim: keymap=russian-jcukenwin
%%beginhead 
 
%%file 21_09_2021.stz.edu.lnr.lgpu.1.golub_mira
%%parent 21_09_2021
 
%%url http://lgpu.org/news/8408-v-lgpu-proshel-edinyy-chas-duhovnosti-golub-mira.html
 
%%author_id 
%%date 
 
%%tags 
%%title В ЛГПУ прошел Единый час духовности «Голубь Мира»
 
%%endhead 
\subsection{В ЛГПУ прошел Единый час духовности «Голубь Мира»}
\label{sec:21_09_2021.stz.edu.lnr.lgpu.1.golub_mira}

\Purl{http://lgpu.org/news/8408-v-lgpu-proshel-edinyy-chas-duhovnosti-golub-mira.html}

В Луганском государственном педагогическом университете 21 сентября прошел
Единый час духовности «Голубь мира», призванный напомнить об укреплении идеалов
мира среди всех стран и народов. Участниками акции стали студенты,
преподаватели и сотрудники вуза. 

\ii{21_09_2021.stz.edu.lnr.lgpu.1.golub_mira.pic.1}

«Единый час духовности» проходит с целью создания единого духовного
пространства и сохранения памяти о Великой Победе. Он проходит не только в ЛНР,
но и в более чем 50 государствах-участниках Международного союза «Наследники
Победы»: России, Армении, Республике Беларусь, Болгарии, Греции, Грузии,
Израиле, Казахстане, на Кипре, в Киргизии, Латвии, Литве, Сербии, Республике
Абхазия, ДНР и других странах.

\ii{21_09_2021.stz.edu.lnr.lgpu.1.golub_mira.pic.2}

Ежегодно мероприятия этой бессрочной акции призваны пробуждать людей к поиску
мира, выхода из сложившейся ситуации, а также осмыслению каждым человеком
своего вклада в дело по защите мира на планете, в своей стране, в своем доме.

Участие в акции для многих из присутствующих стало уже ежегодной традицией.

\href{https://www.youtube.com/watch?v=hrBD5VRV5AY}{%
Единый час духовности «Голубь Мира», youtube, 21.09.2021%
}

\ii{21_09_2021.stz.edu.lnr.lgpu.1.golub_mira.pic.3}

Для Луганской Народной Республики, выступающей против фашистской, антигуманной,
милитаристской идеологии, этот праздник имеет колоссальное значение. Студентам
ЛГПУ идеалы мира и гуманности прививают с самого начала обучения, ведь уже
вступая в ряды студентов нашего вуза, первокурсники клянутся «укреплять
интернациональную дружбу молодежи, бороться с любыми проявлениями национальной,
расовой и религиозной нетерпимости».

Для справки: Международный день мира был провозглашен Генеральной Ассамблеей
ООН в 1981 году. Каждый год 21 сентября ООН призывает все страны мира сложить
оружие и прекратить войну.
