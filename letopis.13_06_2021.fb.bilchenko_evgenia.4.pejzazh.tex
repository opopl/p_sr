% vim: keymap=russian-jcukenwin
%%beginhead 
 
%%file 13_06_2021.fb.bilchenko_evgenia.4.pejzazh
%%parent 13_06_2021
 
%%url https://www.facebook.com/yevzhik/posts/3978189612216119
 
%%author Бильченко, Евгения
%%author_id bilchenko_evgenia
%%author_url 
 
%%tags bilchenko_evgenia,foto,kiev,krasota,park,pejzazh,priroda,ukraina
%%title БЖ. Пейзаж без пейзанства
 
%%endhead 
 
\subsection{БЖ. Пейзаж без пейзанства}
\label{sec:13_06_2021.fb.bilchenko_evgenia.4.pejzazh}
\Purl{https://www.facebook.com/yevzhik/posts/3978189612216119}
\ifcmt
 author_begin
   author_id bilchenko_evgenia
 author_end
\fi

БЖ. Пейзаж без пейзанства.

Что-то с берёзами есть у меня - генетическое, но не грёзовое,
А грозовое да боевое, упорствующе живое.
Я не верю в Есенина сантименты, я верю в ту скрупулёзность,
С которою он считал генокоды внутри своего конвоя.
Что-то с Тютчевым у меня - генетическое, но не мая
В начале гроза, а Silentium, не по школьному по учебнику
Прочитанный, а по закладкам бабушки, где она пишет, что понимает,
Что настала её терминалка рака, а Тютчев спасает чем-то
Невыразимым... Есть у меня - гены общие даже с Фетом:
Там, где он опускает во всех предложениях подлежащие.
"Прозвучало над ясной рекою..." Глаголы Бродского - не конфеты
На столе у барина-булкохруста, а русское дао, лежащее
В гробу из цинка, в люльке из дерева, в лукошечке из берёсты.
Вот это - они, берёзы, а не то, что о них слащавится.
Любовь моя, Родина белый мишка, свобода моя, - всё просто:
Чем больше во мне есть берёз без грёз, тем всамделишней мое счастье.
13 июня 2021 г.

\ifcmt
  pic https://scontent-lga3-2.xx.fbcdn.net/v/t1.6435-0/p526x296/200169427_3978189555549458_906940083123586761_n.jpg?_nc_cat=101&ccb=1-3&_nc_sid=8bfeb9&_nc_ohc=8VEmzrGXeNUAX-qUPSq&_nc_ht=scontent-lga3-2.xx&tp=6&oh=d7d3c46040a4c3ae537c4e5d1a3aa5a3&oe=60CB339C
\fi
