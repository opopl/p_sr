% vim: keymap=russian-jcukenwin
%%beginhead 
 
%%file 14_07_2020.fb.fb_group.story_kiev_ua.1.bajka_bessarabki.cmt
%%parent 14_07_2020.fb.fb_group.story_kiev_ua.1.bajka_bessarabki
 
%%url 
 
%%author_id 
%%date 
 
%%tags 
%%title 
 
%%endhead 
\subsubsection{Коментарі}

\begin{itemize} % {
\iusr{Ника Орлова}
Чудесная история!
Приятный слог

\begin{itemize} % {
\iusr{Ирина Петрова}
Спасибо) Мой покойный свёкр был очень интересным рассказчиком, я так любила его слушать)
\end{itemize} % }

\iusr{Тома Храповицкая}

Спасибо Ирина!!! Так хорошо, тепло и с небольшим \enquote{сарказмом} описали отрывок
из жизни Ваших предков! Спасибо, хорошо читается!!!

\begin{itemize} % {
\iusr{Ирина Петрова}

Спасибо, это так важно - знать не только хронологию своего рода, но, и вот
такие жизненные махонькие историйки) мой сын - киевлянин в седьмом поколении),
Гнат Григорович тоже родился в Киеве)

\iusr{Тома Храповицкая}
\textbf{Ирина Петрова} согласна с Вами, хорошо описали, жду ещё воспоминаний, не давлю, по Вашему настроению!!!!!

\iusr{Тома Храповицкая}
\textbf{Ирина Петрова} ждём, я думаю в группе многим понравилось!
\end{itemize} % }

\iusr{Татьяна Гурьева}
Все сплошь таланты, киевские таланты. Ими можно гордиться, и их историями тоже.

\begin{itemize} % {
\iusr{Ирина Петрова}

Да, Танечка, не оскудеет земля киевская самыми разными тьалантами. Это у нас
что-то в воздухе, наверное)))

\iusr{Татьяна Гурьева}
\textbf{Ирина Петрова} такая удивительная земля наша)
\end{itemize} % }

\iusr{Галина Никитенко}

\ifcmt
  ig https://scontent-frx5-2.xx.fbcdn.net/v/t39.1997-6/s168x128/93118771_222645645734606_1705715084438798336_n.png?_nc_cat=1&ccb=1-5&_nc_sid=ac3552&_nc_ohc=w050kTGE1aYAX-qDVNF&tn=lCYVFeHcTIAFcAzi&_nc_ht=scontent-frx5-2.xx&oh=049b5a6ed228224c27b12780aee4d330&oe=61954FBF
  @width 0.1
\fi

\iusr{Ирина Петрова}
благодарю)

\iusr{Кургаева Наталья}
Очень интересно. Благодарю Вам!!!

\ifcmt
  ig https://scontent-frx5-2.xx.fbcdn.net/v/t39.1997-6/s168x128/93025159_222645582401279_8207007965157261312_n.png?_nc_cat=1&ccb=1-5&_nc_sid=ac3552&_nc_ohc=cx_4kk5_uMYAX9UuFn2&_nc_ht=scontent-frx5-2.xx&oh=8ee9579d045329a7efe36dbc26880996&oe=6195CC82
  @width 0.1
\fi

\iusr{Ирина Архипович}
Прочитала на одном дыхании!! Спасибо!!  @igg{fbicon.hands.applause.yellow}  @igg{fbicon.face.happy.two.hands}  @igg{fbicon.heart.eyes} 
\iusr{Alla Kenya}

\ifcmt
  ig https://scontent-frx5-2.xx.fbcdn.net/v/t39.1997-6/s168x128/93025159_222645582401279_8207007965157261312_n.png?_nc_cat=1&ccb=1-5&_nc_sid=ac3552&_nc_ohc=cx_4kk5_uMYAX9UuFn2&_nc_ht=scontent-frx5-2.xx&oh=8ee9579d045329a7efe36dbc26880996&oe=6195CC82
  @width 0.1
\fi

\iusr{Александр Ходор}

И мои, почти все на Байковом.
Земля им пухом и Царствие небесное!!!!

\iusr{Татьяна Лукьяница}

\ifcmt
  ig https://scontent-frx5-2.xx.fbcdn.net/v/t39.1997-6/s168x128/93118771_222645645734606_1705715084438798336_n.png?_nc_cat=1&ccb=1-5&_nc_sid=ac3552&_nc_ohc=w050kTGE1aYAX-qDVNF&tn=lCYVFeHcTIAFcAzi&_nc_ht=scontent-frx5-2.xx&oh=049b5a6ed228224c27b12780aee4d330&oe=61954FBF
  @width 0.1
\fi

\iusr{Татьяна Оксаненко}

Когда о людях, живших в Киеве, о событиях тех лет, ловлю себя на мысли, что все
Киевляне, как родственники, все трогательно в этих воспоминаниях. Спасибо
Ирина, Вас всегда приятно читать!

\begin{itemize} % {
\iusr{Ирина Петрова}
\textbf{Татьяна Оксаненко} благодарю! Есть общность по крови, есть общность по интересам, есть общность по профессиям...разные есть общности. Интересы, профессии меняются, а вот малая родина остаётся навсегда) это и сближает
\end{itemize} % }

\iusr{Неля Архипова}
Замечательные жители нашего великого и любимого Города.

\iusr{Надія Коміссарова}
Дуже яскрава історія, як українці ставали русскімі.

\begin{itemize} % {
\iusr{Ирина Петрова}
\textbf{Надія Коміссарова} 

національность можна визначити тільки за прізвищем? Моя думка - за знанням та
прийняттям історії свого народу, за болем його стражданнь, за радістю його
перемогам, за досконалим володінням мовою нації, за вірність своєму народові.
До речі, мій свекор та моя свекруха , яку я називала мама Галя, були
українськими філологами, дідусь був членом Спілки журналістів України. В домі
завжди звучала прекрасна наша мова, дідусь писав гарні вірші, а прізвище його
мало наприкінці -ов. Прізвище ще мало про що говорить. Головне - душа.

\begin{itemize} % {
\iusr{Надія Коміссарова}
\textbf{Ирина Петрова} звичайно, я з вами згодна.
Але, що сосується трансформації українських прізвищ в Російській імперії, саме так і відбувалось.
На перший погляд, начебто не головне, а наслідок - ми одінарод, укрАінцев нет, Украіну прідумалі в австрійском генштабе.
Я ні в якому разі не ставлю під сумнів патріотичність вашої родини!

\iusr{Lidia Bualo}
\textbf{Надія Коміссарова} ви все влучно сказали, "расставили по полочкам", дякую!

\iusr{Ірина Маслова}
\textbf{Надія Коміссарова} 

трансформували і при совєтах, прізвище мого батька було змінене паспортисткою, а
за незгоду була погроза взагалі відібрати паспорт, батькові вже 92 роки, от
так...

\end{itemize} % }

\iusr{Татьяна Петрюк}
\textbf{Надія Коміссарова} в бочку мёда - ложка дёгтя ?! Зачем ?...?!

\iusr{David Norton}
\textbf{Татьяна Петрюк} Я бы сказал "дерьма на вентилятор"

\iusr{David Norton}

Десь приблизно так і робила колись козацька старшИна, отримавши від руських
царів права нарівні з руським дворянством. А до того - ще й отримала власних
кріпаків з тих же українських селян-грічкосіїв. Зате винна в кріпацтві на
землях теперішньої України Росія...

\begin{itemize} % {
\iusr{Lola Madino}
\textbf{David Norton} отримавши? Від царів? Боже, скільки є охочих разглагольствовать про те, про що зеленої уяви не мають...
До вашого відома - українська шляхта довгими й складними шляхами добивалась визнання свого дворянства.
Впевнена - про своїх предків ви цього ніколи не чули.

\iusr{Валерий Абромович}
\textbf{David Norton} нехера не понял, но зато жутко (интересно!)
\end{itemize} % }

\iusr{Alex Vines}

Господа та добродii. Почему каждый раз, когда речь идет о Городе, о его
знаменитых и простых жителях, обязательно возникает дисскусия о языке, о
хорошей и плохой Советской власти, и т.д. Историю не переделаеш, любая власть
ее пыталась и пытается подогнать под себя. Тут же мы не в политике разбираемся,
НАСЛАЖДАЕМСЯ Великим Городом, его историей, его прошлым, настоящим , его
людьми. Ребята давайте жить дружно, а кто кому друг или враг, нехай разбираются
на других страницах. 

\begin{itemize} % {
\iusr{Олена Білостоцька}
\textbf{Alex Vines} в правы, но у нас война, вот такая история до нее и довела. У людей болит, т. к. люди гибнут каждый день.. Именно по этому не только наслаждаются.
\end{itemize} % }

\iusr{Lola Madino}
Валерий Абромович а оцей опус якою мовою був?  @igg{fbicon.wink} 

\end{itemize} % }

\iusr{Larysa Holovinova}
Да, смесь кровей укрепляет род и рождает таланты

\iusr{Ирина Петрова}

Мені завжди цікаво, дуже цікаво, чому люди знаходять якісь підтексти,
політичні, религійні тощо там, де, здавалось, знайти їх важкувато. Навіть, у
найнайнайнейтральніших оповіданнях про самі найнайнайзабавні та дрібні кумедні
випадки знайдуть щось таке , що одразу викликає водоспад праведного гніву,
збурення всіх "радєтєлєй" правди!. Це треба теж мати талант, здібності, зір
соколиний... "я б так не змогла", як казала теща мого свекра, незабутня наша
баба Тата, упокой , Господи, душеньку її


\iusr{Олена Константинова}
Так душевно написали!

\begin{itemize} % {
\iusr{Ирина Петрова}
благодарю)

\iusr{Олена Константинова}
\textbf{Ирина Петрова} пишите ещё, у Вас хорошо получается!

\iusr{Ирина Петрова}
\textbf{Олена Константинова} "батя, я стараюсь"(с)  @igg{fbicon.face.grinning.big.eyes} 

\iusr{Олена Константинова}
\textbf{Ирина Петрова} только не батя, а маманька
\end{itemize} % }

\iusr{Светлана Недайбида-Бучко}
Спасибо!!! История просто супер!!! Очень интересно!!! @igg{fbicon.thumb.up.yellow}{repeat=3} 

\iusr{Ирина Петрова}
благодарю! От гида похвалу заслужить - очень мило) Спасибо и Вам за экскурсии)

\iusr{Олена Амелина}
Спасибо! интересно!

\iusr{Ирина Петрова}
\textbf{Олена Амелина} спасибо, рада одобрению читателей)))

\iusr{Liudmyla Galaktion}
Хороший, добрый рассказ, приправленный типично киевской легкой иронией. СПАСИБО

\iusr{Ирина Петрова}
\textbf{Liudmyla Galaktion} благодарю! Очень важно знать, что есть отклик в душах людей)

\iusr{Татьяна Скиба}
Очень интересно спасибо

\iusr{Полина Любченко}
Давайте, всех достал национальный вопрос !

\iusr{Ирина Петрова}
\textbf{Полина Любченко} что дать? Мне тоже тут поиски оного непонятны....

\iusr{Юта Кухарева}
Спасибо. классная история

\iusr{Валерий Абромович}
Класные кусачки! аж самому захотелось... шучу я не вор и воров не навижу.

\iusr{Ирина Серова}
Вы просто молодец!!!

\iusr{Ирина Петрова}
\textbf{Ирина Серова} дякую! Сердце автора тане від схвальних відгуків)

\iusr{Valya Jrchenko}

Быль иль небыль, но написано мастерски, блеск. Все герои повествования ожили
перед глазами, слухом воспринимаются интонации голосов. История чудесная,
жизненная и потому правдивая.

\iusr{Олейникова Ольга}

\ifcmt
  ig https://scontent-frx5-2.xx.fbcdn.net/v/t39.1997-6/p480x480/17633008_2041012282792912_5347591075043213312_n.png?_nc_cat=1&ccb=1-5&_nc_sid=0572db&_nc_ohc=0cEh5s0idgAAX8_rNXb&_nc_ht=scontent-frx5-2.xx&oh=de8588fe94a8f49ad5dbf7b303aa5c3c&oe=61955BAA
  @width 0.2
\fi

\iusr{Галина Демченко}
Мiй дiд був ЛугИна, а батько вже ЛугIн.


\end{itemize} % }
