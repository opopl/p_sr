% vim: keymap=russian-jcukenwin
%%beginhead 
 
%%file 12_02_2022.tg.rossia_v_globalnoj_politike.1.ekzaltacia
%%parent 12_02_2022
 
%%url https://t.me/ru_global/18080
 
%%author_id tg.rossia_v_globalnoj_politike
%%date 
 
%%tags napadenie,rossia,ugroza,ukraina,vtorzhenie
%%title Очень много экзальтации вокруг
 
%%endhead 
 
\subsection{Очень много экзальтации вокруг}
\label{sec:12_02_2022.tg.rossia_v_globalnoj_politike.1.ekzaltacia}
 
\Purl{https://t.me/ru_global/18080}
\ifcmt
 author_begin
   author_id tg.rossia_v_globalnoj_politike
 author_end
\fi

Очень много экзальтации вокруг. Попробуем шершавым языком логики, которую
следует применять при анализе подобных ситуаций. 

В конце прошлого года Россия выступила с жёсткими требованиями, сформулировав
их в ультимативной форме. Речь об изменении системы безопасности в Европе, как
она сложилась после холодной войны. То есть вопрос не ситуативный,
принципиальный. Но он принципиален и для Запада, ведь система, существовавшая
до сих пор, была Западу крайне выгодна и комфортна. И Запад не видит, почему он
должен её менять ради того, чтобы она больше устраивала Россию. Так что позиции
обеих сторон весьма тверды. И просто полюбовной договорённостью такие вопросы,
увы, не решаются. В классические времена решались войной и установлением нового
баланса сил, но сейчас война между претендентами, которые обладают ядерным
оружием, практически невозможна по причине крайне высоких рисков. 

Соответственно в ход идут конкурентные преимущества обеих сторон. 

Российское преимущество –  тот факт, что её военные возможности в регионе
конфликта несопоставимы с возможностями США и НАТО, а также Россия при крайнем
развитии событий способна их применить. В отличие от западных стран, которые
категорически подчёркивают, что участвовать в конфликте напрямую не будут. 

Американское преимущество - доминирование в мировом информационном
пространстве, способность создать выгодный для себя и крайне неблагоприятный
для России глобальный нарратив: Москва – хищный и безжалостный агрессор,
противостоять которому – всеобщий долг, а вообще – спасайтесь, кто может, пока
не поздно. И этот свой инструмент США задействуют сейчас по полной программе,
не обращая внимания даже на собственно объект патроната (Украина), который
пытается слабо возражать против такого нагнетания. 

В реальности никто не хочет военного решения, оно станет огромным риском для
всех. Есть основания провести некоторую параллель с Карибским кризисом 1962
года, конечно, в сильно деформированном виде. Без прямой угрозы ядерных
сверхдержав друг другу, с асимметричными ресурсами сторон, со стиранием грани
между реальными и виртуальными опасностями, и с тем, что ставка всё-таки
регионального, хотя и очень существенного, масштаба. Но по степени нагнетаемого
накала ситуация может достигнуть того уровня из-за этих самых информационных
инструментов. 

Благоприятный выход – как и тогда: в какой-то момент признание большой
опасности дальнейшей эскалации и начало прямого разговора по существу о
принципах взаимных гарантий. В 1962 году это сработало и положило начало
выработке системы отношений, которая сделала вторую половину холодной войны
более безопасной с точки зрения фронтальной конфронтации СССР и США. Сейчас это
тоже стало бы оптимальным сценарием.
