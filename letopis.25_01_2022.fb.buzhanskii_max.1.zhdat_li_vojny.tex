% vim: keymap=russian-jcukenwin
%%beginhead 
 
%%file 25_01_2022.fb.buzhanskii_max.1.zhdat_li_vojny
%%parent 25_01_2022
 
%%url https://www.facebook.com/permalink.php?story_fbid=2137798026384665&id=100004634650264
 
%%author_id buzhanskii_max
%%date 
 
%%tags analiz,napadenie,rossia,ugroza,ukraina,vtorzhenie
%%title Ждать ли войны
 
%%endhead 
 
\subsection{Ждать ли войны}
\label{sec:25_01_2022.fb.buzhanskii_max.1.zhdat_li_vojny}
 
\Purl{https://www.facebook.com/permalink.php?story_fbid=2137798026384665&id=100004634650264}
\ifcmt
 author_begin
   author_id buzhanskii_max
 author_end
\fi

Сотни людей каждый день звонят, пишут, переживая по поводу того, ждать ли
войны.

Как любой человек, более того, не обладающий какой то специальной информацией,
могу только предполагать, так вот, предполагаю, категорически НЕТ, не будет.

1.  Причина.

Нет ни одной причины, по которой РФ не просто было бы выгодно, вообще имело бы
смысл её начинать.

Я понимаю, что за годы пропаганды \enquote{они хотят присоединить Украину},
\enquote{поработить}, \enquote{восстановить империю}, \enquote{владеть Коломыей}, уже очень сложно
рассуждать исходя из логики, но смысла начинать войну, нет.

В 21 веке не нужны территории, нужны внешние рынки.

И трудовые ресурсы.

Ресурсы у РФ есть, свои, плюс вся Средняя Азия, которая едет туда работать, а
рынки они себе уже тоже нашли.

В Европе, в той Европе, у которой есть деньги, а это не Польша, и не
Прибалтика, это Германия.

Если вы продаёте бесконечный ресурс с сумасшедшей выгодой, я о газе, и у вас
сложилась конъюнктура рынка- продавать его в 15 раз дороже обычного, вы точно
выберете это моментом для начала войны?

Не думаю.

А задачи, связанные с территориями, они уже решили путем аннексии Крыма и
мятежа на Донбассе.

2.  Время.

Время играет на них, а не на нас.

Можно сколько угодно рассказывать о рассыпающейся бензоколонке, но они
становятся богаче и сильнее, а мы слабее и беднее, это факт.

И речь не об армии, хотя тут тоже можно найти бесконечное количество поводов
для неприятных выводов, а о стране.

Если вы сидите на берегу реки и видите, как ваш противник тонет, точно станете
бросаться с обрыва, чтобы утопить его поскорее?

Нет, посидите, покурите, выпьете кофе, куда ж и зачем спешить?

3.  Цели.

А они их и не скрывают.

Прямо и открыто орут о них благим матом.

Цели очень предметны, раз и навсегда сделать невозможным расширение НАТО на
пространство бывшего СССР, и заставить нас выполнить Минские Соглашения, причём
ровно в том виде, в котором они заключены, без изменений.

Это программа максимум (в отношении нас, уверен, там ещё уйма вопросов), но и
программа минимум несильно отличается, зазор там символический, разве что
пощадить чьё то самолюбие (не наше).

4.  Запад.

Запад продаёт от дохлого осла уши.

Ряд реально имеющих место конфликтов, от Украины до Белоруссии раздуваются до
масштаба вероятности немедленной глобальной ядерной войны, ленты новостей
звенят пулеметными лентами и лязгают гусеницами танков, грохочут сапогами
десантников.

Какие-то части куда то дислоцируются, размещаются истребители, по 3-4 штуки,
что вообще ни о чем, но выглядит красиво, сверкнули крыльями в лучах заката и
покатили по полосе.

Весь этот накал потом вдруг сдуется воздушным шариком, и на столе окажется
договор, который все торжественно подпишут и пойдут рассказывать своим
аудиториям, как превзошли решение Карибского Кризиса.

Россияне, если их до этого не разорвёт на части от переполняющего
самодовольства, когда каждый зам сельского головы с утра даёт НАТО 10-15 минут
на капитуляцию- своим, американцы- своим, слегка унылым, после бегства из
Афганистана.

Маленькая победоносная виртуальная война.

Заканчиваю.

Мы покупаем у них уголь, гсм, уйму всего.

Для того, чтобы нас удушить, не нужно выбрасывать Псковскую дивизию в
Тернополе, достаточно просто прекратить торговлю.

Закрыть границу для наших граждан, работающих там.

Всё это, за месяц создаст проблемы большие, чем все их танковые дивизии вместе
взятые.

Значит ли это, что не будет никаких обострений, провокаций, конфликтов?

Вполне возможно будут.

Но не вторжение ради... а вот никто, заметьте, даже не попытался
сформулировать, ради чего.

Потому что нет ничего такого, ради чего б.

Совсем последнее.

Почему эвакуируют персоналы посольств.

Да потому что наши Западные друзья, преследуя свои цели закошмарили свои
информпространства так, что уже просто не могут не включить соответствующий
протокол.

Иначе будут выглядеть уж совсем странно у себя дома.

А то, как всё это воспринимается у нас дома, как нервничают люди, боятся за
свои семьи, бегут снимать деньги из банков, так им на это наплевать.

Разве ж вы ждали чего то другого, а?

Max Buzhanskiy
