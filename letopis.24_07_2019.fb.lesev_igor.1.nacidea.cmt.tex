% vim: keymap=russian-jcukenwin
%%beginhead 
 
%%file 24_07_2019.fb.lesev_igor.1.nacidea.cmt
%%parent 24_07_2019.fb.lesev_igor.1.nacidea
 
%%url 
 
%%author_id 
%%date 
 
%%tags 
%%title 
 
%%endhead 
\subsubsection{Коментарі}

\begin{itemize} % {
\iusr{Андрей Смитюх}
Это бельгийцы в 1830 на картине?

\iusr{Игорь Лесев}
да

\iusr{Сергей Киселев}
Хорошо написали. Но утопично

\iusr{Игорь Потысьев}
Браво, были отличные тексты, но этот лучший

\iusr{Игорь Лесев}
не, ну если вы это отметили))

\iusr{Halyna Shpak-Oborevych}
Ваши слова - Богу в ухо.

\iusr{Ludmila Nikolaeva}

Сейчас у этих "случайных пацанов" вся полнота власти. Эх, если бы только там
здравомыслящие! Все можно сделать! И идиотские законы отменить, и здравые
принять, и вообще все можно осуществить! Хорошее. Для людей. Для будущего
страны. Только, сдается мне, что напрасны мои мечты...

\begin{itemize} % {
\iusr{Арина Родионова}
\textbf{Ludmila Nikolaeva} , очень напрасны... мечтать полезно и приятно... Но здесь точно зеро ...)

\iusr{Михаил Якобсон}
Есть там и здравомыслящие.
Не стоит так пессимистично.

\iusr{Ludmila Nikolaeva}
\textbf{Михаил Якобсон} Дай то Бог. Мы только "за". За последние годы весь оптимизм иссяк.
\end{itemize} % }

\iusr{Андрей Сергеевич}

Нет у нас никакого шанса построить полноценную страну! И чтобы осознать эту
нехитрую истину, достаточно попытаться ответить самому себе на вопрос: А кто
будет строить?

\iusr{Михаил Подоляк}

Отлично)... Но - опять же - наши новые чемпионы тоже могут не уметь... читать)))

\begin{itemize} % {
\iusr{Игорь Лесев}
да и мы в основном любим писать, так что есть некий паритет))
\end{itemize} % }

\iusr{Олег Хавич}
"Бандера - герой, это нормально и классно".
В.А.Зеленский, 18.04.2019
Всё.

\begin{itemize} % {
\iusr{Игорь Лесев}
Олег, ты когда из польской эмиграции в наше общежитие? Или уже все, ошариился?)

\iusr{Олег Хавич}
\textbf{Игорь Лесев}, как только закроют дело по ч.2 ст.110 УКУ, до десяти лет.

\iusr{Maksim Kammerer}

все отлично - и все в корне не верно. давайте построим - а что? украина - именно
Украина в таких пределах не была и не будет - вывод - Украина должна сузится до
естественных пределов и превратится в этноУкраину - все остальное - это
Российское государство - желательно Республика ,но - вероятнее - всего
-Империя.... но ничего этого не произойдет - в ближайшие лет 50 современная
Украина это недоГосударство, а там как бог даст, как то так:)

\end{itemize} % }

\iusr{Ольга Ремигайло}
Назовём Укросия и суржик - гос.язык ))

\iusr{Fox Renard}

Ряд моментов.

1. Куда девать олигархов, которые хотят сами управлять/получать прибыль от
территории, а остальные укрограждане для них тупо конкуренты? Кстати, местные
олигархи являются продолжением различных глобальных проектных группировок,
поэтому вопрос № 1 частично пересекается с вопросом № 3.

2. Что делать с реально неподъёмными долгами? И здесь же подвопрос: что делать
с диктатом МВФ?

3. Что делать с реально имеющим место внешним управлением, сиречь, практически
полным отсутствием суверенитета?

4. Резюмирующий вопрос: где тупо взять ресурсы (финансовые, человеческие,
административные, организационные и т. д.) на то, чтобы что-то построить, а
потом с этой прибыли делить?

Эти вопросы - ключевые.

\begin{itemize} % {
\iusr{Дмитрий Коломийченко}

\begin{itemize}
  \item 1. Теоретически большое низовое объединение по типу Минин-Пожарский обнулит олигархию.
  \item 2. Отказ. Через переоснование государства.
  \item 3. Большое низовое политическое объединение решает этот вопрос автоматически (Испанские Нидерланды).
  \item 4. Базовым ресурсом любой страны являются люди. Характерный пример - Япония.
\end{itemize}

\iusr{Fox Renard}
\textbf{Дмитрий Коломийченко} 

1. Минин-Пожарский? На Украине? Это шутка?

2. Дефолт? В эпоху глобализации? Когда все мировые легитимные институции
подчинены тем, кто контролирует МВФ м фонды-кредиторы? Ну-ну, успехов.

3. Большое низовое политическое объединение? На Украине? В стране, где людей
проблема в ОСМД объединить? Мне страшно представить, какой должен возникнуть
сверхкризис, чтобы у нас возникли устойчивые горизонтальные связи и все не
передрались со всеми.

4. Разбегается наш базовый ресурс в разные стороны. Деиндустриализация,
деинфраструктуризация и, как следствие, депопуляция. Причём, при полном
содействии всему этому тех, кто у нас именуется элитой.

Может я и ошибаюсь, но я пока не вижу ни единого просвета. Ни единого. Нету
лидера, который может предложить только "тяжёлый труд, кровь, пот и слёзы". Ну
или "впахивание на галерах" (тоже пойдёт).

\iusr{Дмитрий Коломийченко}
\textbf{Fox Renard} 

Я вам чисто теоретически набросал, как это может быть. Теоретичность моего
комментария очевидна, хотя бы исходя из приведенных исторических примеров.

Проблема Украины в том, что кризис носит крайне затяжной характер. И
справедливо описанная вами атомизация общества тоже из области влияния кризиса.
Многолетняя социальная депрессия, миграция и деградация инфраструктуры актом
протестного голосования не преодолеваются. Тем более видео роликами. Это не
акты исторической воли. Игорь предполагает, что мыши станут ежиками. Имеем все
основания сомневаться в реальности этой метаморфозы.

\iusr{Fox Renard}
\textbf{Дмитрий Коломийченко} Блин, отож.  @igg{fbicon.face.pensive} 

\iusr{Энлиль Маратович}
Самый простой ответ на все 4 вопроса: бегом к "маме", под крылышко к Путину. Только, боюсь, уже поздно...

\iusr{Fox Renard}
\textbf{Энлиль Маратович} А нахрена мы Путину? Проще пинка нам дать под зад за всё, что натворили. Да и зрада это.

\iusr{Igor Tenetko}
Пункт 1, бодро и весело!))

\ifcmt
  ig https://scontent-frx5-1.xx.fbcdn.net/v/t1.6435-9/67334663_2363245913995354_5910987593394159616_n.jpg?_nc_cat=111&ccb=1-5&_nc_sid=dbeb18&_nc_ohc=qfsPXfwnDTgAX_U_1nS&_nc_ht=scontent-frx5-1.xx&oh=c10b15e5f64dbae823796e2e2d5fcfc4&oe=61B7815A
  @width 0.4
\fi

\end{itemize} % }

\iusr{Андрей Бородин}
А пока символ всего вот этого - Лёша Гончаренко. Печально.

\iusr{Игорь Димитриев}
Не понимаю, а как вовлечение большинства граждан принесет в страну деньги?

\begin{itemize} % {
\iusr{Fox Renard}
\textbf{Игорь Лесев} 

имеет ввиду, что привлечение граждан уберёт внутреннее противостояние и даст
государству столь необходимое единство граждан.

Он прав во всём. Кроме одного: тупо нет ресурсов. Вот просто нету и всё. И
взять внутри неоткуда. Земля? Возможно! Но это вызовет гигантский социальный
коллапс, сравнимый по своим последствиям с огораживаниями в средневековой
Англии. Остаётся внешний источник. Какой?

- Вот вопрос! Вы ж наповал меня бьёте этим вопросом! (ц).

\iusr{Игорь Димитриев}
\textbf{Сергей Гор} я живу в иллюзиях))))) после вашего коммента у меня вопросов больше нет)
\end{itemize} % }

\iusr{Дмитрий Коломийченко}

Бунт людей которых связывает эмоция. Эмоция же их и развяжет. Им нужна своя
элита, которая будет отрицанием нынешней, а стало быть контрэлитой. Контрэлита
через отрицание формирует общую идею, а та овладевает массами. Пока такой
контрэлиты не видно.

\iusr{Юра Чех}
побачим, хоча "терзают смутньіе сомнения"....

\iusr{Владимир Михайловский}

У автора прекрасные тексты, этот не исключение. Увлекательно
написано, захватывающе, я бы даже сказал. Чтение завораживает. Так малыш внимает
сказке на ночь, а отрок постарше без устали глотает романы, не в силах
оторваться. Сколько утопий на заданную тему, изложенных с разной степенью
талантливости, я прочел за свою жизнь, начиная еще с советских времен! И всякий
раз, ознакомившись с очередным рецептом, я задавал себе два вопроса. Если рецепт
построения государства "не хуже, чем у людей" так прост, то почему он до сих пор
не реализован. И если сделать "не хуже, чем у людей" не получается на
протяжении последней тысячи лет, то почему должно получиться завтра, через
год, десять ? Ну, это вопросы касательно еще той, большой страны. Что же до
Украины, то уместно дополнить еще одним - если последнюю тысячу лет здесь не
было государства вообще, то с чего бы ему появиться сейчас, да еще в более, менее
совершенном виде?

\begin{itemize} % {
\iusr{Halyna Shpak-Oborevych}
\textbf{Владимир Михайловский} , Израиль? Не было государства, и появилось.

\iusr{Владимир Михайловский}
\textbf{Halyna Shpak} 

Ошибаетесь, было, достаточно прочесть библию. Ну, и потом, евреи - это чего-то
особенного ...У них за спиной тысячелетия, а история украинства насчитывает
чуток больше ста лет.

\iusr{Halyna Shpak-Oborevych}
\textbf{Владимир Михайловский} , это почему? Можна же вообще по-другому посмотреть на вопрос: русский - боковая ветвь украинского, изменённая влиянием племен, населявших то, что стало Россией :))))

\iusr{Владимир Михайловский}

"Тому, что стало Россией" порядка 600 лет, то, что стало Украиной вообще еще "не
стало". Так что, кто чья ветвь лучше пропустить. Как по мне, так обоя
рябое. Впрочем, смотрите как хотите, это не приблизит к ответу на мои вопросы.


\iusr{Halyna Shpak-Oborevych}
\textbf{Владимир Михайловский} , 

я к тому, что история - что дышло, куда повернул, то и вышло. Если смотреть на
Украину как наследницу Киевской Руси, то Россия отдыхает :))))


\iusr{Владимир Михайловский}
Если вас интересует, кто, где и как отдыхает, то это не ко мне. Боритесь со своими комплексами и маниями сами.
\end{itemize} % }

\iusr{Андрей Ермаков}

Всё это красивые и заумные слова и тексты... А реально нет ответа. Реальность
такова, что нужен китель, сапоги и трубка, это вся "нерухомiсть"... и будет вам
и электрификация, индустрилизация и всякая там кибернизация... Всё просто...
Менталитет не перекуёшь...


\iusr{Дарина Зарицкая}
Красиво написано, но не реально. Извините)

\iusr{Арина Родионова}

Впервые мне хочется не согласиться с автором) Кто строить-то будет? Тот, кто
"победил" директора "Мотор-Сич"? Или "свадебный фотограф"?) Глядя на эту новую
"раду" считаю что проиграли мы все...

Но текст хороший, просто отличный.

\begin{itemize} % {
\iusr{Виктор Данченко}

ну от того директора в раде, тоже пользы для нас абсолютно никакой. Держать
людей на своем предприятии за рабов, наверное неправильно. Делится же не хочет.


\iusr{Арина Родионова}
\textbf{Виктор Данченко} , охотно верю что пользы никакой, там ни от кого пользы не было. Но рокировка просто потрясающа) но пользы тоже не ждите)

\iusr{Виктор Данченко}
\textbf{Арина Родионова} я и не жду, но и не жалею
\end{itemize} % }

\iusr{Олег Резник}
Отлично! И по сути и по стилю. Люблю читать вообще и если тексты красивые литературно, сразу ощущается.

\iusr{Василий Стоякин}

Мнэээ... Как сказал один умный человек, народу в центральной Украине живется
очень скучно. Они эту скуку терпят-терпят. а потом у них срывает крышу. И
начинается атаманщина. Заканчивается "максимальное вовлечение большинства в
управление" всегда одинаково - гуманитарной оккупацией.

\begin{itemize} % {
\iusr{Игорь Лесев}
не мешай мне думкой богатиты, и вообще, какой тебе еще оккупации не хватает?(с)
\end{itemize} % }

\iusr{Виктор Данченко}

не будут сограждане рвать, за такое государство. В начале идеи и лозунги о
патриотизме работают, но чем дольше происходят события, тем сложнее им на этом
держаться. Все правильно, экономика всегда побеждает гуманитариев, они тоже
хотят благ.


\iusr{Dima Smith}
текст хороший.
только фишка в том, что "нац.идею" нельзя не только сконструировать, но даже и сформулировать, пока её нет.
она вызревает сама.
как и почему это происходит, какие механизмы это обеспечивают - тайна, загадка.
а текст, повторюсь, хороший.

\iusr{Матвей Кублицкий}
не получится. поезд уже ушел...

\iusr{Анатолий Цвиркун}
хотелось бы верить...

\iusr{Марина Прохорова}

Немного странно читать об украинской национальной идее на русском. Потому что
идея эта проста, как вышивка крестом и давно сформулирована: "Украина не
Россия". И ничем не отличается от, например, узбекской национальной идеи, а
именно: русских выгнать или максимально унизить, все командные места отдать
своим (из своего племени, клана, кишлака), править должны местные беи, а
местные декхане должны гнуть на них спину и не выделываться. Или марш за бугор
теньге зарабатывать для любимой родины с тысячелетней историей.

Именно поэтому закон о мове никогда не будет существенно изменён. Только
недобросовестная конкуренция позволит "в своём краю пануваты", а то, что
инженеры, учёные, литераторы и пр. при таком режиме не выживают - никого не
волнует. Зато своя Галя пристроена. И Мыкола из Жашкива может не напрягать свои
мозги - никакой еврейский/русский/венгерский мальчик не составит ему
конкуренции, потому что со школы принуждён будет учиться на неродном языке. И
какая-нибудь Кокотюха соберёт все премии. И т.д. И т.п.

Превосходная национальная идея, по поводу которой давно имеется полный
консенсус именно в центре. И накакая другая им не нужна ((

\begin{itemize} % {
\iusr{Энлиль Маратович}
 @igg{fbicon.cry} 
\end{itemize} % }

\iusr{Ludmila Nikolaeva}

Отличная статья! Впрочем как и все, что пишет Игорь Лесев. Но я вот о чем. Один
раз переметнуться как Леша Гончаренко?! Один раз?! А вот посмОтрите, этот
опарыш зайдет в Раду и переметнется к "Слуге"!!! Ему не впервой. Он будет это
делать всегда.

\iusr{Serge Kirik}
Доктор, я жить буду? - А смысл?

\iusr{Юрий Рубаха}

По-моему, тут делается ошибочная ставка на мифическую "третью силу", о которой
в той или иной степени говорят уже последние лет 15.

На самом деле сил всего две:

1) "Сделать по-нормальному" 2) "Сделать назло русским"

Других сил в стране просто нет, причем функционируют они именно в том виде, в
котором описаны.

При этом вторая сила потратила немало сил на убеждение самих себя в том, что
первая является "пророссийской" или "малоросской". Это понятно, т.к. им нужно
как-то обосновывать свою патологическую русофобию. Но нормальным людям не
следует разделять эти заблуждения.

Конечно, если проникнуться этой идеей, то может показаться, что между
"пророссийской" и "антироссийской" идеологией существует что-то еще, причем
"что-то хорошее". Но на самом деле этим "чем-то хорошим" и является та
идеология, что не является антироссийской. Попытка ее срезать, обвинив в
пророссийскости, приводит к полному мировоззренческому и институциональному
вакууму.

\begin{itemize} % {
\iusr{Serge Kirik}
похоже, что так и будет продолжаться по кругу: Незалежность - Руина - Переяславская Рада - Незалежность - зрада - перемога(
\end{itemize} % }

\iusr{Sergey Light}

Вспомним, сколько праведных и неправедных обид было у россиян на Украину, когда
она отделилась. Один Бродский чего стоит! И тут же и сколько надежд на нее было
- Украина будет развиваться, подтянет Россию, придаст ей импульса... Почти все
тогдашние политики-аналитики-критики начала 90-х на этой теме потоптались. А
вот когда же россияне с этими упованиями окончательно завязали, в 95, 98, 2001,
2004...? Когда поняли, что Украина - не жилец? Кто помнит?

\end{itemize} % }
