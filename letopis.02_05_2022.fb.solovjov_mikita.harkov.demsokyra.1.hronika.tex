% vim: keymap=russian-jcukenwin
%%beginhead 
 
%%file 02_05_2022.fb.solovjov_mikita.harkov.demsokyra.1.hronika
%%parent 02_05_2022
 
%%url https://www.facebook.com/Mikita.Solovyov/posts/7474741482596200
 
%%author_id solovjov_mikita.harkov.demsokyra
%%date 
 
%%tags 
%%title Хроника Харькова, 2-е мая
 
%%endhead 

% topics.vojna.harkov.posts
% topics.vojna.day.68
 
\subsection{Хроника Харькова, 2-е мая}
\label{sec:02_05_2022.fb.solovjov_mikita.harkov.demsokyra.1.hronika}
 
\Purl{https://www.facebook.com/Mikita.Solovyov/posts/7474741482596200}
\ifcmt
 author_begin
   author_id solovjov_mikita.harkov.demsokyra
 author_end
\fi

Хроника Харькова, 2-е мая. 

Сегодня обстрелов немного больше чем в предыдущие два дня. Кроме традиционных
Салтовки и Пятихаток, были прилеты по ХТЗ. То есть все равно заметно меньше чем
неделю назад, но больше чем предыдущие пару дней. Честно говоря, не могу понять
откуда обстреливали ХТЗ.

По области обстрелов много. Из известного мне Малая Даниловка, Золочев,
Богодухов, Чугуев, очень сильный обстрел Слатино. Есть жертвы. Естественно,
много разрушений.

В городе и области, где это возможно, продолжаются активные работы по
восстановлению инфраструктуры. Причем кроме точечного латания дыр чтобы хоть
как-то обеспечить светом и прочим, местами проводятся серьезные работы,
например перепрокладка труб среднего диаметра. (На всякий случай. \enquote{Средними}
называются все кроме магистральных. Фактически это самый большой диаметр и
давление, который вообще можно встретить в городе.) Активно продолжаются
дорожные работы. Не берусь судить об области. Но в городе без света и газа
остаются только те микрорайоны, в которых восстановление сетей невозможно и
приходится полностью тянуть их заново. Как минимум, на многих участках.

Который раз удивляюсь ситуации с эвакуацией. Теперь уже не только в самых
пострадавших микрорайонах Харькова, но и других населенных пунктах области. Вот
на примере Слатино. Регулярные обстрелы. Из коммунальных сетей не работает
вообще ни хрена. Света, воды, газа нет, мобильной связи почти нет. Естественно,
никаких работающих магазинов. Возможностей выехать море. Только волонтеры
постоянно предлагают варианты. Но примерно 200 человек продолжают там упорно
сидеть. Это достаточно небольшой процент, порядка 3-5 от довоенного уровня. Но
люди упираются и отказываются уезжать. И это не исключение, это стандартная
ситуация. В менее пострадавших населенных пунктах еще больше процент
оставшихся. Даже при ежедневных или регулярных обстрелах.

Причем понятны две вещи. Что при настолько массированных разрушениях быстро
восстановить инфраструктуру не получится. Тем более, при регулярно
повторяющихся обстрелах. Вести какую-то хозяйственную деятельность теоретически
можно. (Не удивляйтесь сильно, но многие огороду уже посажены прямо под
обстрелами.) Но это все многократно повышает риски. Мне кажется критически
важным разобраться почему люди отказываются уезжать и попытаться устранить эти
причины отказов. Уговаривать. В критических случаях, как в том же Слатино или
Р. Лозовой, возможно объявлять обязательную эвакуацию. Благо право такое в
временной военной администрации есть.

Возможно, через пару дней смогу подробнее об этих отказниках рассказать.
Собираюсь в ближайшие дни пообщаться с ними, для начала в черте города.

А в центре города уже все больше налаживается какой-то быт и какая-то жизнь.
Это похоже на то, как наступает весна или светлеет утром. Каждый день вроде
никаких заметных изменений со вчера нет. Но проходит пара недель, и все вокруг
уже изменилось. Все больше работает сервисов. Все больше людей на улицах просто
гуляет. Все больше (хотя еще и очень мало, конечно) людей нашли себе какое-то
занятие и способ заработка. Все больше машин. Город выглядит уже совсем не так,
как две-три недели назад. Но чувствуется, что для дальнейшего восстановления
городу нужен какой-то толчок. И я не могу придумать ничего более важного, чем
запуск общественного транспорта.

Харьков стоит!

Слава Украине!

Низкий поклон нашим защитникам!

\#ХроникаХарькова

%\ii{02_05_2022.fb.solovjov_mikita.harkov.demsokyra.1.hronika.eng}

\ii{02_05_2022.fb.solovjov_mikita.harkov.demsokyra.1.hronika.cmt}
