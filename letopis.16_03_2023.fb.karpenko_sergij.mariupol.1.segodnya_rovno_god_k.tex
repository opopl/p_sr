%%beginhead 
 
%%file 16_03_2023.fb.karpenko_sergij.mariupol.1.segodnya_rovno_god_k
%%parent 16_03_2023
 
%%url https://www.facebook.com/sergiykarpenko/posts/pfbid02XkdwmNot1ngwkQw74gzJEHdqc5vgrkbckPHg2ADd6J4fNi2SDhgS4zxLukzJg1Eol
 
%%author_id karpenko_sergij.mariupol
%%date 16_03_2023
 
%%tags mariupol,mariupol.war,dnevnik
%%title Сегодня ровно год как моя семья вышла из окруженного Мариуполя
 
%%endhead 

\subsection{Сегодня ровно год как моя семья вышла из окруженного Мариуполя}
\label{sec:16_03_2023.fb.karpenko_sergij.mariupol.1.segodnya_rovno_god_k}

\Purl{https://www.facebook.com/sergiykarpenko/posts/pfbid02XkdwmNot1ngwkQw74gzJEHdqc5vgrkbckPHg2ADd6J4fNi2SDhgS4zxLukzJg1Eol}
\ifcmt
 author_begin
   author_id karpenko_sergij.mariupol
 author_end
\fi

Сегодня ровно год как моя семья вышла из окруженного Мариуполя. Простите за сумбурность

Хроника

День 0

обычный день. Вчера выступил плешивый с признанием рыцпублик дамбасса... ну
выступил и выступил, он эту балалайку восемь лет чешет... сил для вторжения и
захвата у него явно недостаточно, чего лезть то?!?

К вечеру написали источники в местной полиции, что завтра вторжение. Решил, что
утром виднее будет

День 1

В 4:00 начались разрывы ракет по периметру города. Интернет сказал, что
плешивый выступает. Включил трансляцию: специальная ваенная аперацыя...
блаблабла... выключил... думал ехать на работу, жена отговорила. Пошел на разведку:
очереди к банкоматам, за водой, сигаретами... большие очереди... к вечеру появилось
стойкое ощущение, что надо валить с левого берега, грады шли пакетами, не
прекращаясь, по пригородам, а по городу точечно пытались работать ствольной
артой с востока и северовостока. Если бы я опирался тогда только на опыт восьми
лет, я бы остался дома, но интуиция продолжала клевать в темечко \enquote{надо валить!}

Решили к родителям ехать, это самый запад города, всяко меньше вариантов что
прилетит на голову... забрали кошек и тещу... вещей на пару дней взяли

%1-2
%\ii{16_03_2023.fb.karpenko_sergij.mariupol.1.segodnya_rovno_god_k.pic.1}
\ii{16_03_2023.fb.karpenko_sergij.mariupol.1.segodnya_rovno_god_k.pic.1.large}

День 2

⁃ Наберите воды
⁃ Так есть же в кране
⁃ Наберите...
⁃ Куда?
⁃ В емкость для капельного полива 2 тонны лишними не будут

Поехали затаримся... заправки все уже закрыты, даже захоти я сейчас уехать из
города, я стану в лучшем случае в 300км, а их еще проехать надо, а на холостом
ходу оно тоже жрет...

Ладно в магазин: картошки мешок, масла 5л, мука, вода... о тут еще можно картой
оплатить надо же... [и чего я сигарет не затарил тогда баран?!?] народ нервный
смари, если бы не военные вокруг, уже бы замес был...

%3-5
%\ii{16_03_2023.fb.karpenko_sergij.mariupol.1.segodnya_rovno_god_k.pic.2}
\ii{16_03_2023.fb.karpenko_sergij.mariupol.1.segodnya_rovno_god_k.pic.2.large}

День 5

Столько всего дома забыли... ну да, на пару дней же ехали... говорят город уже в
окружении... ложимся, вечером пропал свет. Вместе с ним и связь, любая. 

Милая, хочешь фокус покажу?

Беру картошку, бинт, блюдце, подсолнечное масло. Вуаля каганец готов) на ближайшее время это наш источник света вечером...

День 6 в городе во всю орудуют мародеры, спрячь машину. У соседки гараж пустой... спрятал

День 7

Выключили воду. А я говориил!

День 8

Вода в баках замерзла. -12 ночью это тебе не это... вечером вышел на крыльцо,
лицом на юг. Что-то не так... зарево, должно быть справа на западе, там, где село
солнце, а оно слева, там, где город. Город горит, это бытовой газ полыхает по
всему городу

% 6-8
%\ii{16_03_2023.fb.karpenko_sergij.mariupol.1.segodnya_rovno_god_k.pic.3}
\ii{16_03_2023.fb.karpenko_sergij.mariupol.1.segodnya_rovno_god_k.pic.3.large}

День 9

выключили газ...

День 10

Надо забрать бабушку из центра города, там уже накрывает. Забрали. Бедняжка не
понимает, где она и зачем, отвезите меня домой без конца...

Нашли керосинку, новейшая совдеповская керосинка... у соседа на огороде зачем-то
провалялась полторашка керосина несколько лет... залили... горит зашибись...

Готовить на костре это прикол... ровно один день, пока не дойдет что дрова все
сырые и это процесс на весь день... блины вместо хлеба, горячее питье вместо
всего... залейте в термосы кипяток!.. остались угли? Зашибись, ставь казан с
технической водой, он хоть чуть нагреет комнату, в которой мы спим, летнюю
комнату без утепления с вечным конденсатом на стенах...

День 13 

сегодня восьмое марта, есть самогон, бутылка шампанского, колонка и музыка в
телефоне. Будем праздновать, петь и танцевать...

День 14

Орки разбомбили роддом трети... к нам тоже прилетел снаряд. В пяти метрах от
родительского дома. Второй этаж прошило навылет. Там ночевали дети, к счастью,
в этот момент их там не было. Стекла вынесло, теперь все ночуют по летним
хозпостройкам... также разбило одну из наших машин

% 9-11
%\ii{16_03_2023.fb.karpenko_sergij.mariupol.1.segodnya_rovno_god_k.pic.4}
\ii{16_03_2023.fb.karpenko_sergij.mariupol.1.segodnya_rovno_god_k.pic.4.large}

День 15 

У соседа есть пленка, затянули окна в доме

День 16 

сливаем бензин черпаком из разбитой машины, теперь у нас какой никакой запас есть...

День 17

Надо сходить за водой... четверо \enquote{молодых} по две баклажки в руки и
пошли на родник... как же хочется курить...

Дошли туда, набрали воды, встретили мужиков, покурили)

Узнали новости: в городе работает один магазин, туда многокилометровые очереди,
бои в северозападной части города, заходят сепароорки на танках их палят, они
валят артой по застройке, мародеры торгуют потихоньку или меняют: блок
сигарет-литр бенза, сто баксов-20000грн, пачка сигарет - 500грн... итп

По дороге куча разбитых домов и машин от того обстрела...

Короче вернулись домой, взяли машину и набрали во все емкости что было, разом
300л. Это немного на толпу в 10 человек

День 19

Говорят, наверху возле школы ловит сеть... пошел... долго лазил по крышам,
черемушки в дыму, надо же... поймал сеть, с десятого раза поговорил с родней,
хватило на то, что мы живы объяснить...

День 20

А пошли еще сеть половим... ой, что это? Наверху куча машин в сторону Мелекино с
белыми повязками и надписями \enquote{дети}, бесконечная вереница до горизонта. Что
это? Можно выехать? Коридор? Запорожье? Погнали советоваться с семьей... я до
того каждый день проверял кран на бочке с водой... всегда ни капли, а в этот день
она размерзлась, чем не знак типа оставайся, еда еще есть, вода тоже... нет, надо
валить... может завтра? Нет сегодня... за полтора часа собрались, на часах около 2х
часов дня, с Богом... 10 людей на две машины и двое кошек на одну
переноску... бензин слитый пригодился

К комендантскому часу успели до Бердянска доехать, по дороге куча блокпостов с
разными неукраинскими лицами в бомжатской зеленой форме с белыми повязками.
«Асвабадители» матьих... везде спасало что в машинах есть дети и старики, в тот
день такая куча машин ехала, что им даже поверхностно было некогда
досматривать.

Доехали почти до Токмака, село на букву Х, там завернули нас асвабадители, мол
возле посадки заночуйте в машинах, не шарьтесь... из села приехали волонтеры,
давайте к нам, детей разместим. Мы перепуганные, куда мы детей отдадим?!? Тогда
в школу давайте, там горячая еда.

Поели, нам даже какие-то маты дали на пол постелить, положили женщин детей,
сами не спали толком, холодно пипец...

День 21

7:00 Едем... Токмак, Васильевка... везде блокпосты тянется и тянется... к часу дня
подъехали к Запорожью. Красавец мост на въезде с юга взорван... надо в объезд.
Там запускают машин по 50 опасаются обстрела... въехали кое как около 16:00... и
тут начала пробиваться сеть... сотни сообщений от друзей с поддержкой и
вопросами, за весь март... даже сейчас слезы на глазах... приехали, кое как
поселились, помылись, прочитали новости за день: разбомбленный драмтеатр с
людьми, корабли начали работать по побережью, там где мы еще вчера жили... вот
так в последний момент выскочили...

Уже потом начали приходить сообщения о зверствах на Киевщине, о масштабах
геноцида в Мариуполе и стало понятно то, что было неочевидно в день 0: сил то
на захват территории у них нет. Если захват — это цель. А вот на геноцид...

%\ii{16_03_2023.fb.karpenko_sergij.mariupol.1.segodnya_rovno_god_k.eng}
%\ii{16_03_2023.fb.karpenko_sergij.mariupol.1.segodnya_rovno_god_k.cmt}
