% vim: keymap=russian-jcukenwin
%%beginhead 
 
%%file 17_04_2022.stz.news.ua.pravda.1.rekviem_za_moskvoju
%%parent 17_04_2022
 
%%url https://www.pravda.com.ua/articles/2022/04/17/7340231
 
%%author_id kazarin_pavel,news.ua.pravda
%%date 
 
%%tags 
%%title Реквієм за Москвою
 
%%endhead 
 
\subsection{Реквієм за Москвою}
\label{sec:17_04_2022.stz.news.ua.pravda.1.rekviem_za_moskvoju}
 
\Purl{https://www.pravda.com.ua/articles/2022/04/17/7340231}
\ifcmt
 author_begin
   author_id kazarin_pavel,news.ua.pravda
 author_end
\fi

В Севастополе для пророссийских людей крейсер \enquote{Москва} был всегда немного
больше, чем просто корабль. Статус \enquote{убийцы авианосцев} был пропитан имперским
масштабом и щекотал эго. Этот корабль строили для Третьей мировой и его боевое
дежурство символически поддерживало уровень притязаний. Мол, раз \enquote{Москва} несет
службу, то от своих амбиций Россия не отказывается.

Теперь \enquote{Москва} лежит на дне, а российские социальные сети проходят пять стадий
принятия горя. Сперва отрицали, затем злились, теперь торгуются. Пытаются
убедить себя, что ракетный крейсер затонул из-за американских козней. Они
исступленно жонглируют версиями, только ради того, чтобы не отдавать украинской
армии победу над собственным символом. Но впереди их ждет лишь отчаяние и
принятие.

Впрочем, у гибели \enquote{Москвы} вообще нет хороших версий. Попытка списать ее на
\enquote{ошибки экипажа} плохо вяжется с имиджем \enquote{лучшей армии континента}. Хотя, этот
статус отныне тоже довольно условен. На фоне происходящего все компьютерные
игры, построенные на идее \enquote{русского вторжения в Европу}, смотрятся максимально
нелепо.

Российская армия грозилась дойти до Ла-Манша, но не смогла добраться даже до
Фастова. То, что должно было сломить Европу – не смогло сломить даже Киевскую
область. Должно быть Владимиру Путину сейчас очень стыдно. Он потратил
миллиарды, чтобы его боялись, но в итоге над ним смеются.

А ведь для российского президента сейчас наступил момент истины. Он шел к нему
начиная с двухтысячного года. Эдакая развилка Достоевского, на которой принято
определяться с тем, тварь ли ты или право имеешь. Путин похож на игрока,
который поставил на зеро абсолютно все – и теперь твердо намерен не проиграть.
Для нас это означает лишь то, что он не прекратит хоронить своих солдат в
украинских полях.

До девятого мая остается три недели. Принято считать, что к этой дате
российская армия попробует добиться каких-то успехов. Москва соорудила из
Второй мировой войны гражданскую религию, назвала ее \enquote{Великой Отечественной} и
десятилетиями черпала легитимность из этого источника. Победа над нацизмом была
объявлена главным \enquote{зато}, призванным оправдывать все, что было до войны, и все,
что было после нее. Российская власть раз за разом объявляла себя наследницей
победителей – и на этом основании требовала от остального мира благодарных
уступок.

24 февраля эта лошадь умерла и скакать на ней больше не выйдет. Отныне Москва
наследует не у победителей, а у проигравших. На российский парад приедут только
те, кого не смущают параллели со странами Оси. Россия до последнего будет
отрицать свой новый статус, но вряд ли ее мнение станет кого-то интересовать.  

Забавно. Россия десятилетиями воспринималась как средоточие бытового
благополучия. В нее ехали за карьерой, деньгами и признанием. Трудовая миграция
служила \enquote{оправданием} для российского высокомерия и заносчивости. Профицит
российского бюджета выполнял роль \enquote{мягкой силы} - выгоды дружбы с Москвой можно
было посчитать при помощи калькулятора. Отныне все это в прошлом. Владимир
Путин превратил Россию в некрасивую подружку, у которой вдобавок тикают часики.

Часто пытаюсь понять, что будут писать об этой войне белорусские учебники. С
украинскими все понятно. С европейскими – тоже. А как будут лет через десять
объяснять себе все эти события беларусы? Как они станут описывать статус своей
страны? Кого из своих соотечественников сочтут героями? Чьи фамилии будут
гравировать на постаментах?

Это не праздный вопрос. Последние пятьдесят дней мы спорим о том, как
разграничить коллективную вину и коллективную ответственность. Пытаемся понять,
где заканчиваются граждане и начинается режим. Сердимся на тех, кто как папа
Франциск – пытается осуждать войну, а не агрессора. И если Путин воспринимается
нами как продолжение российского обывателя, то споры о том, можно ли ставить
знак равенства между Лукашенко и беларусами – все еще идут.

Кстати. На наших глазах происходит еще одно важное изменение. География театра
военных действий вплетает украинский восток и юг в национальный героический
концепт. В дополнение к украинскому западу, который обзавелся этим концептом
еще семьдесят лет назад. Россия своими собственными руками сшивает Украину на
символическом уровне – отправляя в небытие привычные идеологемы про \enquote{лево-} и
\enquote{правобережье}.

Война в нашей стране перестала быть делом одних лишь военных. Она стала
национальной и отечественной. Фронт держат волонтеры и предприниматели,
айтишники и пенсионеры, те, кто платят налоги и те, кто разбирают завалы. В
своей попытке доказать невсамделишность Украины Россия в очередной раз добилась
обратного.

Отрицание. Гнев. Торг. Отчаяние. Принятие. Это ведь не только про реакцию
Москвы на катастрофу \enquote{Москвы}. Это ведь еще и про реакцию России на катастрофу
России. Пятьдесят дней назад эта страна встала на эскалатор, который везет ее
вниз. И никакие слова не изменят траекторию этого маршрута.

Наше поколение напишет отличные мемуары.
