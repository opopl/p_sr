% vim: keymap=russian-jcukenwin
%%beginhead 
 
%%file 01_01_2022.fb.buzhanskii_max.1.stepan_razvalil_vse
%%parent 01_01_2022
 
%%url https://www.facebook.com/permalink.php?story_fbid=2120852964745838&id=100004634650264
 
%%author_id buzhanskii_max
%%date 
 
%%tags bandera_stepan,istoria,ukraina
%%title Степан развалил всё
 
%%endhead 
 
\subsection{Степан развалил всё}
\label{sec:01_01_2022.fb.buzhanskii_max.1.stepan_razvalil_vse}
 
\Purl{https://www.facebook.com/permalink.php?story_fbid=2120852964745838&id=100004634650264}
\ifcmt
 author_begin
   author_id buzhanskii_max
 author_end
\fi

Степан развалил всё.

Нет, он вовсе не боролся ни с Третьим Рейхом, ни даже с Советским Союзом, нет,
с ними боролись другие.

Степан боролся за власть.

И развалил всё, всё, к чему прикоснулся, всё, что было у националистов.

Развалил ОУН до войны, восстав против Мельника, за что самой же ОУН и был
приговорён к расстрелу в 1940м году.

Не берусь судить, кто там из них двоих и тех, кто шёл за ними, был больше или
меньше прав, просто констатирую факт.

Развалил свою же собственную ОУН(б), уже после войны поссорившись с Ребетом.

Великой Отечественной Войны, которую он провел под арестом в бараке у забора
Заксенхаузена, а вовсе не \enquote{в концлагере}, как об этом пишут люди, которые либо
лгут, либо не в курсе.

В бараке, в котором его навещала жена, родившая ему в этот период сына, в
бараке, из которого немцы выпустили его в сентябре 44 го года, бороться с
Советским Союзом.

Как раз после освобождения Украины от нацистов.

Боролся ли покойный Степан с Советским Союзом?

Нет.

В первый и последний раз, после освобождения в сентябре 39го немцами из
брестской тюрьмы, где его держали поляки, на территории Советского Союза, во
Львове, Степан был в 40м году.

И, скорее всего, в конце 39го.

И никогда больше там не был, ни советской власти, ни Красной Армии, ни граждан
советских в глаза не видел, и не особо рвался повидать, будем откровенны.

С Советским Союзом, если уж на то пошло, боролся Шухевич, пришедший сюда с
немцами в составе своего \enquote{Нахтигаля} в 41 году, служивший карателем белорусских
крестьян в 201 батальоне шуцманшафта в 42м году, бывший здесь, оставшийся
здесь, никуда не побежав, и убитый здесь, на территории нашей страны.

Но не Степан.

А что Степан?

А Степан боролся за власть.

Будет ли справедливо сказать, что его интересовала только власть, и ничего
больше?

Я думаю, однозначно нет.

Просто власть была необходимым инструментом, а жизнь сложилась таким образом,
что ни дотянуться до него толком, ни воспользоваться им хоть для чего то, никак
не удавалось, и так и не удалось.

Что, в общем то, очень неплохо, потому что ничего хорошего использование
Степаном этого инструмента украинцам не сулило, даже на фоне всего, через что
они прошли в первой половине 20 века.

Степан развалил всё.

Поэтому, убийство Степана Бандеры 15 октября 1959 года, было большой ошибкой
Советского Союза.

Которому он был очень полезен именно в этом, разваливать всё.

Большая страна и ошибки большие были, но нет, больше, чем большой ошибкой,
большой глупостью.

В этот же, кстати, день, но в 1945 году был расстрелян Пьер Лаваль.

Роковое какое то число для коллаборантов, 15.10.

Это так, короткая памятка для тех, кто прочтет мелькнувшую где то в новостях
информацию о том, как две тысячи человек на сорокамиллионную страну пойдут
отмечать день рождения покойного.

Будет много факелов, но ведь жечь- не строить, как можно понять из биографии
того же Степана.

Max Buzhanskiy
