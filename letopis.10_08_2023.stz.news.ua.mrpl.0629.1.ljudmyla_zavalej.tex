% vim: keymap=russian-jcukenwin
%%beginhead 
 
%%file 10_08_2023.stz.news.ua.mrpl.0629.1.ljudmyla_zavalej
%%parent 10_08_2023
 
%%url https://www.0629.com.ua/news/3642625/smatocok-mariupola-u-centri-stolici-istoria-ludmili-zavalej-aka-budue-viziu-povernenna-u-misto-foto
 
%%author_id news.ua.mrpl.0629
%%date 
 
%%tags 
%%title Шматочок Маріуполя у центрі столиці. Історія Людмили Завалей, яка будує візію повернення у місто, - ФОТО
 
%%endhead 
 
\subsection{Шматочок Маріуполя у центрі столиці.\titlebreak Історія Людмили Завалей, яка будує візію повернення у місто}
\label{sec:10_08_2023.stz.news.ua.mrpl.0629.1.ljudmyla_zavalej}
 
\Purl{https://www.0629.com.ua/news/3642625/smatocok-mariupola-u-centri-stolici-istoria-ludmili-zavalej-aka-budue-viziu-povernenna-u-misto-foto}
\ifcmt
 author_begin
   author_id news.ua.mrpl.0629
 author_end
\fi

\ii{10_08_2023.stz.news.ua.mrpl.0629.1.ljudmyla_zavalej.pic.1}

\begin{qqquote}
Людмила Завалєй разом з іншими маріупольськими \enquote{Берегинями} відкрила шматочок
Маріуполя у самому центрі Києва на провулку Музейному 8Б. Коворкінг
\enquote{BERE\hyp{}HYNY space}, який вона з командою започаткувала,  - це результат
її власного ПТСРа, отриманого внаслідок пережитого в Маріуполі, і мрії
повернутись назад, після деокупації. 
\end{qqquote}

\begin{quote}
\em\enquote{Мені потрібно було щось робити, бути корисною. Цією справою я лікую себе від
стресу. Робота – це єдине, що допомагає багатьом маріупольцям зараз впоратись
із ситуацією}, 
\end{quote}
- каже Міла.

Людмилу Завалей добре знають у волонтерському середовищу, хоча вона сама
зізнається, що дуже довго була \enquote{за чоловіком}. Її чоловік, лікар Андрій
Пазушко, з перших днів російського вторгнення, ще у 2014 році, долучився  до
волонтерського середовища, яке на той час називалось Новий Маріуполь. А Людмила
просто допомагала йому коли це було потрібно. 

% 1 - Прийняти, посортувати ...
\ii{10_08_2023.stz.news.ua.mrpl.0629.1.ljudmyla_zavalej.qt.1}

\ii{10_08_2023.stz.news.ua.mrpl.0629.1.ljudmyla_zavalej.pic.2}

Вона вже мала великий досвід організаторської роботи, в тому числі і на
державній службі. За фахом Людмила – бухгалтер, фінансист. І до декретної
відпустки багато попрацювала на різних посадах.

% 2 - Мені дуже пощастило працювати ...
\ii{10_08_2023.stz.news.ua.mrpl.0629.1.ljudmyla_zavalej.qt.2}

\ii{10_08_2023.stz.news.ua.mrpl.0629.1.ljudmyla_zavalej.pic.3}

Тож Людмила Завалей пристала на пропозицію попрацювати в
\enquote{Маріупольтепломережі}. Це найбільш складне комунальне підприємство в
Маріуполі, нереформоване, немодернізоване, із великою корупційною складовою.
Вже нова команда у міськраді, яка складалася переважно з менеджерів Метінвесту,
запросила Людмилу Завалей в якості кризового менеджера, разом з новою командою,
на підприємство.

% 3 - Я протягом шести місяців вивчала ...
\ii{10_08_2023.stz.news.ua.mrpl.0629.1.ljudmyla_zavalej.qt.3}

\ii{10_08_2023.stz.news.ua.mrpl.0629.1.ljudmyla_zavalej.pic.4}

Після такої стресової роботи Людмилою було прийнято рішення про організацію
власної справи, вона захотіла віддавати більше часу громадському сектору. Після
декількох років роботи з Дмитром Чичерою, з Мариною Пугачовою, в 2020 році
з'явилась \enquote{Берегиня. Павлопілля}. Це одне із селищ, яке знаходилось майже на
самій лінії зіткнення, яке багато хто називав вимираючим, але там була активна
проукраїнська спільнота., яка хотіла відновити колись багате село. Тому і
заснували там.

\ii{10_08_2023.stz.news.ua.mrpl.0629.1.ljudmyla_zavalej.pic.5}

% 4 - За роки співпраці з Дмитром Чичерою ... 
\ii{10_08_2023.stz.news.ua.mrpl.0629.1.ljudmyla_zavalej.qt.4}

Вона бачилась з ним востаннє 14 березня, зустрічались у \enquote{Халабуді}.

% 5 - Ми ще на початку повномасштабного вторгнення ...
\ii{10_08_2023.stz.news.ua.mrpl.0629.1.ljudmyla_zavalej.qt.5}

Відсутність евакуації в перші дні війни – це одне із\par\noindent найболючіших питань, які
ставить Людмила Завалей перед владою.
 
% 6 - Насправді, я знала, що буде війна, і що вона буде не такою ...
\ii{10_08_2023.stz.news.ua.mrpl.0629.1.ljudmyla_zavalej.qt.6}

14 березня чоловік Людмили Завалєй зміг прорватися в місто, і сказав що можна
виїжджати, є коридор.  Автобус евакуаційний який їхав для евакуації в місто
заїхати не зміг, по евакуаційному коридору лежали нерозірвані снаряди і міни.
Він із Польщі віз гуманітарку для маріупольців  – її росіяни та денеерівці
забрали. Але головне, що він живий.

\ii{10_08_2023.stz.news.ua.mrpl.0629.1.ljudmyla_zavalej.pic.6}

Скрізь одне з селищ Людмила з друзями виїхали з міста і дісталися наступного
населеного пункту, там ще чекали людей, кого б могли забрати із собою, і рушили
на Запоріжжя, потім Дніпро і далі – Івано-Франківськ. 

Два місяці, які Людмила Завалей разом з родиною, друзями, собаками, котами,
черепашками провела у друзів в Івано-Франківську, жінка називає реабілітацією.

% 7 - Ми просто жили. Намагались не думати ні про що ...
\ii{10_08_2023.stz.news.ua.mrpl.0629.1.ljudmyla_zavalej.qt.7}

Вона впевнена, що дискусію треба проводити у двох напрямках – серед дорослої
частини міста, і серед молоді.

\begin{quote}
\em\enquote{Якщо молодь не буде повертатися у Маріуполь, то не буде ніякого реборну,
ніякого відновлення. Буде місто, яке повільно помирає. А коли молодь зможе
почати повертатись? Якщо вона буде залучена до відновлення міста, якщо
побачить, що тут зможе реалізуватись. Тому це дуже важливо – працювати у цьому
напрямку}, 
\end{quote}
- каже Людмила Завалей.

\ii{10_08_2023.stz.news.ua.mrpl.0629.1.ljudmyla_zavalej.pic.7}

В її найближчих планах – окрім серії дискусій,  - розпочати курси
терапевтичного малювання, терапевтичного письма, провести зустрічі з цікавими
особистостями, експертами, політологами.

\begin{quote}
\em\enquote{Наш простір відкритий для всіх. Він буде працювати до тих пір, поки окупований
Маріуполь. А потім ми всі з командою будемо повертатись і разом відбудовувати
наше українське місто}, 
\end{quote}
- запевняє Людмила Завалей.

%\ii{10_08_2023.stz.news.ua.mrpl.0629.1.ljudmyla_zavalej.txt}
