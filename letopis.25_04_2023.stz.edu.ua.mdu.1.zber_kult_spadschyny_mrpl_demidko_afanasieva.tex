% vim: keymap=russian-jcukenwin
%%beginhead 
 
%%file 25_04_2023.stz.edu.ua.mdu.1.zber_kult_spadschyny_mrpl_demidko_afanasieva
%%parent 25_04_2023
 
%%url https://mdu.in.ua/news/zberezhennja_kulturnoji_spadshhini_mariupolja_kulturologini_mdu_predstavili_vlasni_proekti_na_vseukrajinskomu_konkursi/2023-04-25-4266
 
%%author_id mdu
%%date 
 
%%tags 
%%title Збереження культурної спадщини Маріуполя: культурологині МДУ представили власні проєкти на всеукраїнському конкурсі
 
%%endhead 
 
\subsection{Збереження культурної спадщини Маріуполя: культурологині МДУ представили власні проєкти на всеукраїнському конкурсі}
\label{sec:25_04_2023.stz.edu.ua.mdu.1.zber_kult_spadschyny_mrpl_demidko_afanasieva}
 
\Purl{https://mdu.in.ua/news/zberezhennja_kulturnoji_spadshhini_mariupolja_kulturologini_mdu_predstavili_vlasni_proekti_na_vseukrajinskomu_konkursi/2023-04-25-4266}
\ifcmt
 author_begin
   author_id mdu
 author_end
\fi

%ia 25_04_2023.edu.ua.mdu.zber_kult_spadschyny_mrpl_demidko_afanasieva
%Маріуполь,Україна,Мариуполь,Украина,Mariupol,Ukraine,Mariupol.MSU,Маріуполь.МДУ,Мариуполь.МГУ,Ольга Демідко,Наука,date.25_04_2023

Представниці Маріупольського державного університету доцентка кафедри
культурології \href{\urlDemidkoIA}{Ольга Демідко} та магістрантка спеціальності \enquote{Культурологія}
Наталія Афанасьєва долучилися до конкурсу наукових проєктів у межах
Всеукраїнської науково-практич\hyp{}ної конференції \enquote{Дослідження молодих вчених: від
ідеї до реалізації}. Організатором події виступив Київський університет імені
Бориса Грінченка.

\ii{25_04_2023.stz.edu.ua.mdu.1.zber_kult_spadschyny_mrpl_demidko_afanasieva.pic.1}

Конкурс об'єднав студентів, аспірантів і викладачів закладів вищої освіти
Києва, Харкова, Ніжина, Луцька, Одеси, Херсона, Маріуполя. Власне бачення
культурологині МДУ представили за напрямом \enquote{Наука та культура: збереження і
популяризація історико-культурної спадщини}, успішно пройшовши відбірковий етап
конкурсу.

На конференції \href{\urlDemidkoIA}{Ольга Демідко} презентувала проєкт пересувної фотовиставки
\enquote{Літопис театрального життя Маріуполя}. Він дозволить проаналізувати розвиток
театральної культури міста-героя протягом середини XIX – кінця XX століття,
висвітлити діяльність професійних та самодіяльних театральних колективів
приазовського краю, а також приїжджих труп.

\begin{quote}
\emph{Сьогодні для багатьох українців залишається невідомим, що найстаріший театр на
Лівобережній Україні був заснований саме у Маріуполі. Проєкт спрямований
змінити ставлення широкої громади до історії міст Східної України та наблизити
до театрального мистецтва героїчного міста. Крім того, він спрямований
зруйнувати радянську \enquote{парадигму закритості} музеїв та архівів і захистити
найбільш вразливу культурну спадщину – паперові об'єкти, історичні фотографії
та документи,}
\end{quote}
– доцентка \href{\urlDemidkoIA}{Ольга Демідко}.

Унікальність та інноваційність проєкту визначає особистий\par\noindent оцифрований архів
дослідниці МДУ, який нараховує понад 1000 світлин. Всі вони були зібрані з 2013
року під час написання дисертації, присвяченій театральному життю Північного
Приазов'я. Колекція містить програмки, афіші та фотодокументи, що зберігалися в
маріупольських театрах, музеях і приватних архівах.

За результатами конференції \href{\urlDemidkoIA}{Ольга Демідко} стала переможницею за обраним
напрямом конкурсу. Це дозволило дослідниці отримати фінансову підтримку на
реалізацію власного проєкту. Викладачка сподівається, що відкриття виставки
відбудеться саме у Маріупольському університеті, після чого буде представлена в
інших освітніх закладах, музеях і театрах Києва. 

У свою чергу, магістрантка Наталія Афанасьєва представила історико-культурний
проєкт віртуального музею пам'яті \enquote{Архітектурна спадщина Маріуполя: проблеми
окупації та випробування часом}. Його основна мета полягає у створенні
веб-сайту, який міститиме інтерактивну мапу значущих архітектурних споруд
героїчного міста із супровідними матеріалами про них: дореволюційними
листівками, світлинами різних періодів, аудіо- та відео-матеріалами, спогадами.

\begin{quote}
\emph{Основна ідея полягає в тому, щоб дослідити архітектурну спадщину Маріуполя,
зокрема у реаліях сьогодення, зберегти та актуалізувати пам'ять про місто, про
його історико-культурні особливості. У межах проєкту буде створена база
важливих архітектурних споруд і проведена фотофіксація їхнього поточного стану,
збір текстових і візуальних матеріалів, що транслюють значення кожного
архітектурного об'єкту для маріупольців, міста, регіону,}
\end{quote}
– магістрантка Наталія Афанасьєва.

Студентка Маріупольського університету не отримала призового фонду на
реалізацію авторського проєкту, тому додатково шукатиме джерела фінансування.
Не дивлячись на це дівчина вже втілює у життя власну ідею і планує довести
справу до кінця навіть без стороннього бюджетування.

\textbf{Читайте також:} \href{https://mdu.in.ua/news/pragnennja_stvorjuvati_novi_mozhlivosti_v_mdu_zavershivsja_konkurs_studentskikh_iniciativ/2023-04-12-4261}{\emph{Студенти МДУ отримали фінансування на реалізацію власних проєктів}}%
\footnote{Студенти МДУ отримали фінансування на реалізацію власних проєктів, МДУ, 12.04.2023, \par%
\url{https://mdu.in.ua/news/pragnennja_stvorjuvati_novi_mozhlivosti_v_mdu_zavershivsja_konkurs_studentskikh_iniciativ/2023-04-12-4261}%
}

%\ii{25_04_2023.stz.edu.ua.mdu.1.zber_kult_spadschyny_mrpl_demidko_afanasieva.txt}
