% vim: keymap=russian-jcukenwin
%%beginhead 
 
%%file 07_11_2019.fb.sotnik_katerina.1.bulgakov_parad_kiev.cmt
%%parent 07_11_2019.fb.sotnik_katerina.1.bulgakov_parad_kiev
 
%%url 
 
%%author_id 
%%date 
 
%%tags 
%%title 
 
%%endhead 
\zzSecCmt

\begin{itemize} % {
\iusr{Катерина Сотник}
На першому фото – О. Козир-Зірка, він же – булгаківський \enquote{полковник Козырь-Лешко}.

\iusr{Глеб Бараев}

\url{https://uk.wikipedia.org/wiki/Олесь_Козир-Зірка}

\begin{itemize} % {
\iusr{Катерина Сотник}
\textbf{Глеб Бараев}
Благодарю Вас.
Однако уровень \enquote{Википедии} мною давно преодолён.
И в сфере непосредственной теоретико-прикладной деятельности, где наличествует и PhD, и Dr.habilitatus (см. инфо в профиле).
В исторических материалах, – являющихся моим эпизодическим хобби: также стараюсь пользоваться оригинальными, а не научно-популярными и герменевтико-мультиинтерпретативными источниками.
Однако – спасибо за внимание и заботу  @igg{fbicon.wink} .

\iusr{Глеб Бараев}
\textbf{Катерина Сотник} Дело не в уровне, а в указании на дату рождения. Вы ведь писали, что неизвестен его возраст. Интересно также указание на место рождения, оно совпадает с местом рождения другого атамана - Андрея Рыбалко-Зирка. Уж не родственники ли?

\iusr{Катерина Сотник}
\textbf{Глеб Бараев}
Спасибо.
Вам – просто следовало уточнить, к чему относится Ваша ссылка. Но, тем не менее, в украинских архивах Мюнхенского украинского университета, Институте украинских исследований Гарвардского университета, украинских спец.-иссл.центрах Канады, имеющих бОльший нежели в Украине объём материалов, даты рождения и смерти данного исторического субъекта – до сих пор спорны.

\iusr{Катерина Сотник}
\textbf{Глеб Бараев}
Спасибо за интересную информацию об А.Рыбалко-Зирка.
Попробую сделать запрос в бумажные архивы.
А пока что – не ли у Вас какой-нибудь эл.ссылки о нём ?

\iusr{Глеб Бараев}
\textbf{Катерина Сотник} О Рыбалко-Зирка специальных работ нет. Он упоминается лишь в связи с предательством 1921 года, но это Вы наверняка знаете.

\iusr{Катерина Сотник}
\textbf{Глеб Бараев}
Нет, вроде бы не знаю об этом событии – я ведь не историк.

\iusr{Глеб Бараев}
\textbf{Катерина Сотник} Тогда имеет смысл начать с этой статьи 

\href{https://zn.ua/SOCIETY/opozdali.html}{%
ОПОЗДАЛИ, Как и почему сорвалась попытка поднять в Украине всенародное восстание против большевиков, %
zn.ua, 14.03.1997%
}

\iusr{Глеб Бараев}

И дополнительно - что остается за пределами исторических публикаций: кто
возглалял противника - Екатеринославскую ГубЧК. Это был Трепалов, вошедший в
историю как \enquote{первый начальник московского уголовного розыска}, о нем в
последние годы даже сериалы снимают, но только о деятельности в МУРе.

\end{itemize} % }

\iusr{Катерина Сотник}

\begin{multicols}{3} % {
\obeycr
« – За що ж боролись ми з ляхами?
– За що ж ми різались з ордами?
– За що скородили списами
московські ребра !??
\smallskip
Засівали,
І рудою поливали...
І шаблями скородили.
Що ж на ниві уродилось??!
\smallskip
Уродила рута... рута...
Волі нашої отрута.
А я, юродивий, на твоїх руїнах
Марно сльози трачу; заснула Вкраїна,
\smallskip
Бур’яном укрилась, цвіллю зацвіла,
В калюжі, в болоті серце прогноїла
І в дупло холодне гадюк напустила,
А дітям надію в степу оддала.
\smallskip
А надію...
Вітер по полю розвіяв,
Хвиля морем рознесла.
Нехай же вітер все розносить
На неокраєнім крилі,
Нехай же серце плаче, просить
Святої правди на землі.
\smallskip
Спи ж, повитий жидовою,
Поки сонце встане,
Поки тії недолітки
Підростуть, гетьмани.
Помолившись, і я б заснув...
\smallskip
Так думи прокляті
Рвуться душу запалити,
Серце розірвати.
Не рвіть, думи, не паліте,
Може, верну знову
Мою правду безталанну,
Моє тихе слово.
\smallskip
Може, викую я з його
До старого плуга
Новий леміш і чересло.
І в тяжкі упруги...
Може, зорю́ переліг той,
А на перелозі...
\smallskip
Я посію мої сльози,
Мої щирі сльози.
Може, зійдуть, і виростуть
Ножі обоюдні,
Розпанахають погане,
Гниле серце, трудне...
\smallskip
І вицідять сукровату,
І наллють живої
Козацької тії крові,
Чистої, святої!!!
\smallskip
Може... може... а меж тими
Меж ножами рута
І барвінок розів’ється,
І слово забуте,
Моє слово тихо-сумне,
\smallskip
Богобоязливе,
Згадається — і дівоче
Серце боязливе
Стрепенеться, як рибонька,
І мене згадає...
Слово моє, сльози мої,
\smallskip
Раю ти мій, раю!
Нехай гинуть
У ворога діти,
Спи, гетьмане, поки встане
Правда на сім світі».
\restorecr
\end{multicols} % }

P.S. Вагомий мемуарист Т. Шевченка,

поляк Афанасьєв-Чужбинський відзначає повсякденну дружбу Шевченка з євреями,
котрих він ніколи не ототожнював з ідеологічними «жидами», з-поміж багатьох
випадків, описує один з епізодів, коли горіли бідні євреї, а всі мешканці
стояли збоку і дивилися:

– «Шевченко у розпачі кинувся їх рятувати й витягати з вогню, зі злобством й
розпачем та плачучи кричав цим обивателям: \enquote{Чого ж ви стоїте? Це ж люди !
Рятуйте їх}.

Т.Шевченко «Полякам»:

\obeycr
– «Отак-то, ляше, друже, брате!
Неситії ксьондзи, магнати
Нас порізнили, розвели,
А ми б і досі так жили.
Подай же руку козакові
І серце чистеє подай!»...
\restorecr

Цей вірш Т.Шевченко присвятив й вперше усно читав своїм найближчим друзям –
З.Сераковському (провідному герою Польського повстання 1863-67 рр.),
А.Коасовському, ксьондзу М.Зельонці (котрий зі сміхом сприймав покаянні
екскурси \enquote{елекцій} Тараса проти ксьондзів, котрий, кожного разу перепрошував
свого друга...).

P.P.S. Найкращим художньо-документальнтм біографістом Т. Шевченка вже багато
років вважається лише поляк Єжі Єнджієвіч, автор книги «Українські ночі, або
Родовід генія».

Нет описания фото.

\end{itemize} % }
