% vim: keymap=russian-jcukenwin
%%beginhead 
 
%%file 04_11_2021.fb.belenjkij_vjacheslav.1.patriotizm_vojna_donbass.cmt
%%parent 04_11_2021.fb.belenjkij_vjacheslav.1.patriotizm_vojna_donbass
 
%%url 
 
%%author_id 
%%date 
 
%%tags 
%%title 
 
%%endhead 
\subsubsection{Коментарі}

\begin{itemize} % {
\iusr{Валентина Николаева}

Всё верно. И ещё они почему-то уверовали в свою исключительность. У них
почему-то есть право не признавать власть и валить её майданом, а у других нет
права не признавать их власть.

\iusr{Валентина Николаева}
Дрюченко и Кулаков, как всегда ничго не поняли, но это нормально для питриотов.

\iusr{Олександр Михельсон}

И пам'ятники развалят, и детей сожрут  @igg{fbicon.smile}  встречный вопрос: а у России какой
мотив? Если что - такой же, что и у Украины. Отстаивать то, что считаешь своим.
Ну просто же все - нет, автору вот надо изгаляться, плести всякую чушь...

\begin{itemize} % {
\iusr{Дмитрий Духонченко}
Хто не скаче, той "плетет всякую чушь ".
Как свежо и интересно!

\iusr{Вячеслав Беленький}
\textbf{Олександр Михельсон}

На каком основании Вы, ничего в Донбассе не имеющие, считаете его своей собственностью?
(А Крым, получается, признаете "не своим")

\iusr{Дмитрий Воронин}
Да, где битвы за Крым???

\iusr{Олександр Михельсон}

М-да. Вера в адекватность оппонентов снова меня подвела  @igg{fbicon.frown}
Что ж. Не впервой... Оставляю вас в компании адептов хазарского заговора.
\end{itemize} % }

\iusr{Энрике Менендес}

Вопрос не ко мне, но я возьму на себя смелость ответить цитатой из книги,
которую сейчас читаю - думаю там содержится и ответ.

"... человеческая природа, как отмечает Фукидид, мотивируется страхом (phobos),
собственной выгодой, "барышом" (kerdos) и славой, мнением других (doxa). "Чтоб
улучшить мир, - пишет Моргентау, - человек должен работать сообразно этим
силам, а не противодействовать им". Таким образом, реализм принимает
человеческий материал таким, каков он есть, каким бы далёким от идеала он ни
был."

Цитата дана по книге Роберта Каплана "Месть географии"

\iusr{Юрий Онищенко}

А был какой-то референдум, из результатов которого понятно, что больше 50\%
жителей Донбасса хотят создать свое независимое государство? Или как
происходило волеизъявление стремления к отделению?

\begin{itemize} % {
\iusr{Денис Зимин}
\textbf{Юрий Онищенко} 11 мая 2014

\iusr{Юрий Онищенко}
\textbf{Денис Зимин} 

\href{https://www.bbc.com/ukrainian/ukraine_in_russian/2014/05/140511_ru_s_donbass_referendum}{%
В Донецке обнародованы результаты "референдума", bbc.com, 11.05.2014%
}

ну такое себе.

\iusr{Денис Зимин}
\textbf{Юрий Онищенко} 

ну конечно, они не идут ни в какое сравнение с теми референдума, которые
проводила Украина, а их как я помню, за 30 лет было аж 2. Первый в 2000,
результаты которого так и не были реализованы, и второй, в октябре 2020, просто
эталон псевдореферендумов, это когда мальчики и девочки у входа у избирательных
участков проводили так называемый опрос. Конечно были и региональные, начиная
от крымского в начале 90-х, Донецкого и луганского 1994, харьковского 2002, но
их результаты естественно были центральной властью проигнорированы. Кстати,
никаких инициатив о проведении референдума, как одной из форм демократии, в
западных или центральных регионах страны я совершенно не припомню. Хотя конечно
это может быть связано, с тем, что там больше демократии предпочитают
охлократию.


\iusr{Юрий Онищенко}
\textbf{Денис Зимин} 

а до второй половины 19 века в России было крепостное право. Значит ли это, что
я могу оправдывать несвободу жителей деревни в СССР?

\end{itemize} % }

\iusr{Валентина Терешкова}
Согласна с автором на все 100.

\iusr{Oksana Kudrya}
Именно шо.

\iusr{Дмитрий Воронин}

Очень интересный пост  @igg{fbicon.thumb.up.yellow}{repeat=3} 

Я много обывателей с Украины слушал и видел,и сложилось у меня такое ощущение-

Хотят что были как все, не высовывались, не выделялись, чтоб не как эти на
северо востоке, чтоб всё чинно и как надо, особость своя, пусть и архаичная.

Это от комплексов к соседу, фобии живущие и конформизм-быть как все и флер типа
украинства-вышиванки, цвета, язык, хотя вон выше патриоты и на родном, но за ,,
патриотизм" и украинизацию


\iusr{Василий Лесорубов}
Всё верно

\iusr{Gavrilova Irina}
О, Михельсона уже выпустили из вытрезвителя, где он ловил собак, тараканов и шмыгающих мышей?

\iusr{Андрей Ювченко}
Какая милая манипуляция  @igg{fbicon.face.tears.of.joy} 

\iusr{Ігор Кулаков}

Дело в том, что здесь дофига и больше дончан и луганчан, которым, гхм, не очень
нравится, что у них забрали Родину. Парни из Хмельницкого им немного-немного
помогут. Потому что мы одна страна, брат за брата. Русские не сдаются.

\begin{itemize} % {
\iusr{Дмитрий Духонченко}
Вот до чего доводит отсутствие информационной гигиены...

\iusr{Ігор Кулаков}
\textbf{Дмитрий Духонченко} Такова жизнь, Дмитрий Духонченко, такова ее правда.

\iusr{Дмитрий Духонченко}
Жизнь это не картинка в телевизоре и даже не мемасик в смартфоне.
Жизнь посложнее и поинтереснее СМИ и ЛОМов была, есть и будет.

\iusr{Ігор Кулаков}
\textbf{Дмитрий Духонченко} Именно. Мне бы очень не хотелось снова пробираться через блокпосты в Лутугино, чтобы хоронить друга. Которого какие-то *** застрелили лишь за то, что он не притормозил на БП. Это жизнь.

\iusr{Дмитрий Духонченко}
Вот неправда, вам хочется.
Для подкрепления вашей картины мира.
Вы же не хотите вернуться в 2013, когда никаких блокпостов и войск в Донбассе ещё не было?
Когда пересiчнi соросята ещё не призывали стрелять, а лишь писали статьи на тему "Хватит кормить Донбасс"

\iusr{Ігор Кулаков}
\textbf{Дмитрий Духонченко} 

конечно, украинцу по происхождению (но не по духу) из СПБ виднее, что хочется
русскому по происхождению, но украинцу по духу. А вернуться я бы хотел,
пожалуй, в 2003-2004, когда в том числе в СПБ стал разрабатываться проект
"Крым". Или действительно в 2013, когда мои харьковско-питерские знакомые
разворачивали в Крыму сетку.

\iusr{Дмитрий Духонченко}

А вы дух Бандеры с духом Украины не путайте.
Хуторянство это не единственный вариант для Украины.
К тому же не забывайте, что революции любят пожирать своих детей, особенно тех, которые только по духу.

\iusr{Ігор Кулаков}
\textbf{Дмитрий Духонченко} 

Гхм. Причем здесь дух Бандеры. Причем здесь хуторянство. Причем здесь
революции. Последняя значимая революция была в 1917 году на вашей новой родине.
Вы высказываете сами себе какие-то тезисы, сами на них отвечаете и с ними
спорите.


\iusr{Дмитрий Духонченко}

Iгор, неужели вы тоже считаете последний майдан переворотов, а не революцией гидности?
Я бы тоже хотел, чтобы и хуторянство, и бандеровщина были не при делах.
Но именно они оккупировали Киев со всеми вытекающими последствиями и при полной поддержке олигархата.
И именно они сейчас считают себя Украиной.
Мне такой дух не нравится, ибо затхлый и давно протухший.
Уверен, что такое не понравилось бы и жителю СПб Тарасу Григорьевичу, украинцу по происхождению

\iusr{Ігор Кулаков}
\textbf{Дмитрий Духонченко} 

Дмитрий, а что такое майдан вообще? Это даже не переворот (хотя вам из каждого
утюга говорят, что да). Это когда группа олигархов (среди кторых один дончанин)
красиво удалила одного донецкого увальня, который немного обнаглел со своей
"семьей". остальные-то остались. Но РФ выгодно считать это переворотом, да.
Иначе как оправдать дальшейшее вмешательство во внутренние дела соседнего
государства, разворачивание проекта "Стоп Нато" (2003 год, кстати) и т.п.

\end{itemize} % }

\iusr{Larissa Conti}

Хазарская кабала проглотила Украину, а Донбасс костью в горле стал. Чтоб они не
делали, кость в горло не полезет. Вы что ещё не поняли что такое Донбасс?


\iusr{Сергей Перевозчиков}

Вячеслав проспал в летаргическом сне 8 лет, и с удивлением вопрошает ко всем,
почему идёт война. Это как летом 1943 года на курской дуге у бойцов ркка
спрашивать, что они тут делают и чего добиваются. Как по мне дурацкий вопрос.

\begin{itemize} % {
\iusr{Вячеслав Беленький}
\textbf{Сергей Перевозчиков}
Эта сова на глобус не натягивается, хотя часто пытаются.
На Курской - либо сопротивляешся, либо ты "абажур" по плану "Ост".
Если же в культурном срезе, то ещё хуже - обратная картина по факту.
Вторжение на Донбасс чужой идеологии.

\iusr{Сергей Перевозчиков}
\textbf{Вячеслав Беленький} у них был план Ост, а у путина план Русский мир.

\iusr{Вячеслав Беленький}
\textbf{Сергей Перевозчиков}
Что такое "русский мир", которым вы постоянно пугаете Украину?

\iusr{Сергей Перевозчиков}
\textbf{Вячеслав Беленький} 

это когда человек безправное существо, придаток государства, как гимн, герб или
флаг. Где существует пиетет перед властью, где сама мысль о вольнодумстве,
своих правах абсурдна сама по себе. Где на протяжении сотен лет эта система
вбивала каторгой, гулагом, ФСБ людям в голову правила, которым они сейчас
безвольно подчиняются. Это ассимиляция местного населения, невозможность
учиться на родном языке, знать родной язык и им пользоваться. Вот что такое
русский мир, если коротко.

\iusr{Вячеслав Беленький}
\textbf{Сергей Перевозчиков}

Вы описали абсолютно всё, что воплощает Украина на Донбассе, и не только там.
Пункты показательного унижения и насильственной ассимиляции бесправного
населения приведены в "украинском мире"

Не находите?

\iusr{Сергей Перевозчиков}
\textbf{Вячеслав Беленький} нет не нахожу. Ассимилировать украинцев украинским языком невозможно.

\iusr{Вячеслав Беленький}
\textbf{Сергей Перевозчиков}

классика манипуляции терминологией, для чего у нас и смешали понятия гражданства с национальностью в один термин.
Половина граждан Украины - русскоязычные, а каждый пятый ещё и прямо русским себя назвал на переписи.

\iusr{Сергей Перевозчиков}
\textbf{Вячеслав Беленький} 

17\% и это было ровно 20 лет назад. Вот я и говорил выше об этом выше. Забыли
украинцы свой язык и стали русскоязычные. В России это сплош и рядом. Одни
кавказцы держатся, и то только потому что стали россиянами на 200 лет позже.

\iusr{Вячеслав Беленький}
\textbf{Сергей Перевозчиков}
Хоть 5\%. Кто вам дал право их насильственно ассимилировать?

\iusr{Игорь Писаренко}
\textbf{Сергей Перевозчиков} 

Гуго – по количеству запросов, и Приватбанк — по количеству обращений говорят о
поиблизительно 80\% русскоязычных. А можно просто на улицу выйти

\end{itemize} % }

\end{itemize} % }
