% vim: keymap=russian-jcukenwin
%%beginhead 
 
%%file 01_01_2022.fb.fb_group.story_kiev_ua.1.izdateli_bratja_korchak_novickie
%%parent 01_01_2022
 
%%url https://www.facebook.com/groups/story.kiev.ua/posts/1831258323737627
 
%%author_id fb_group.story_kiev_ua,jeremenko_evgenia
%%date 
 
%%tags kiev,kniga,tipografia
%%title КИЕВСКИЕ ИЗДАТЕЛИ БРАТЬЯ КОРЧАК-НОВИЦКИЕ
 
%%endhead 
 
\subsection{КИЕВСКИЕ ИЗДАТЕЛИ БРАТЬЯ КОРЧАК-НОВИЦКИЕ}
\label{sec:01_01_2022.fb.fb_group.story_kiev_ua.1.izdateli_bratja_korchak_novickie}
 
\Purl{https://www.facebook.com/groups/story.kiev.ua/posts/1831258323737627}
\ifcmt
 author_begin
   author_id fb_group.story_kiev_ua,jeremenko_evgenia
 author_end
\fi

С новым годом, дорогие близкие по духу киевляне! От всей души благодарю, что вы
появились в моей жизни, за ваш оптимизм, увлечённость, свет и доброту. 

В наступившем году желаю нам всем быть вместе, дарить любовь и радость нашему
прекрасному миру. 

БЫТЬ ДОБРУ!, как замечательно призывает наш Олег. 

Вчера вдогонку уже разместила свой прошлогодний \enquote{должок}, но случайно почему-то
удалила. Повторяю.

КИЕВСКИЕ ИЗДАТЕЛИ БРАТЬЯ КОРЧАК-НОВИЦКИЕ (2)

Продолжая рассказ об известных книгоиздателях дореволюционного Киева
КОРЧАК-НОВИЦКИХ, перелистаем некоторые страницы жизни старшего брата, Георгия
Трофимовича (1859? ‒ после 1915). О младшем, Николае Трофимовиче (1862-1937),
возглавлявшем типографию при Императорском университете св. Владимира,
говорилось в 1-й части.

\ii{01_01_2022.fb.fb_group.story_kiev_ua.1.izdateli_bratja_korchak_novickie.pic.1}

Прежде всего, о годах жизни. Возле дат вынужденно стоят вопросительные знаки.
Во многих источниках годом рождения указан 1859, но, пересмотрев перечень
изданий, вышедших в издательстве Г.Т.Корчак-Новицкого, находим 1878 год
(«Пырятынська ластивка. Кобзарь Ев. Пав. Гребинкы» ‒ название оригинала).
Действительно ли 19-летний юноша был хозяином солидной типографии? 

О годе упокоения в интернет-ресурсах тоже нет достоверных данных. В большинстве
источников стоит прочерк, в одном из них встретилась дата 1911. В
адресно-справочной книге «Весь Киев» фамилия Г. Т. Корчак-Новицкого фигурирует и
в 1912, и в 1915 с указанием места его проживания на Большом (Бибиковском)
бульваре 28 и 50. На 1916 год и далее сведения об адресах киевлян отсутствуют.
Нужно сказать, что это не совсем показательно. Например, Вера Николаевна
Гудшон, упокоившаяся в 1911 году, о чём в газете «Киевлянин» было помещено
несколько объявлений, а год спустя, в 1912, годовое поминовение, упоминается в
книге «Весь Киев» за 1913 (!) год по адресу ул. Резницкая 9а. Но, как бы там ни
было, год кончины Георгия Трофимовича Корчак-Новицкого с достоверностью указать
нельзя.

\ii{01_01_2022.fb.fb_group.story_kiev_ua.1.izdateli_bratja_korchak_novickie.pic.2}

Впрочем, нам важнее знать, каков его вклад в жизнь Киева. 

О том, как обстояло дело в государстве и, в частности, в нашем городе во 2-й
половине XIX века с книгоизданием, уже кратко упоминалось в 1-й части. Хочется
лишь напомнить, что среди многих серьёзных издателей имя Г. Т. Корчак-Новицкого
вовсе не затерялось. 

Начав с литографической мастерской, он удачно повёл дело, специализируясь
сначала на духовной литературе, а со временем расширил круг своих интересов и
включил в него издания светские, словари, справочники, календари. Вскоре он
стал владельцем паровой хромолитографии. Его имя стоит первым в ряду известных
киевских издателей: «В Киеве... на 1 января 1887 г. было 17 типографий  и 2
скоропечатных заведения. Из них в четырёх ‒ в типографиях Г. Корчак-Новицкого,
С. Кульженко, И. Кушнерева и в Лаврской ‒ в производстве применялись паровые
двигатели», ‒ пишет Виталий Ковалинский.   

\ii{01_01_2022.fb.fb_group.story_kiev_ua.1.izdateli_bratja_korchak_novickie.pic.3}

В 1901 году типография по форме собственности была преобразована в «Акционерное
общество печатной и издательской деятельности Г. Т. Корчак-Новицкого». 

\ii{01_01_2022.fb.fb_group.story_kiev_ua.1.izdateli_bratja_korchak_novickie.pic.4}

Увидели свет тысячи экземпляров книжек, среди которых и первые издания
современных ему украинских авторов. 

Опять-таки, назовём наугад некоторые издания типографии Г. Т. Корчак-Новицкого.

\enquote{Исповедь} Блаженного Августина Епископа Иппонийского,1880; Милорадович Г.А.
\enquote{Сказания о роде дворян и графов Милорадовичей}, 1874; Келлог, С. Г. \enquote{Буддизм и
христианство. Сравнение легендарной истории и учения Будды с евангельской
историей и учением нашего господа И. Христа}, 1894 г.; А.А.Андриевский.
\enquote{Русские конфиденты в Турции и Крыму в 1765-1768}, 1894; Сильвестр
(Маливанский) (еп.). \enquote{Опыт Православного Догматического богословия}, 1892; П.И.
Житецкий "Очерк литературной истории малорусского наречия в XVII веке", 1889;
Антонович В.Б. Сборник летописей, относящихся к истории Южной и Западной Руси,
1888; Орест Левицкий \enquote{Внутреннее состояние Западно-Русской Церкви в
польско-литовском государстве}, 1884; Голубев С. \enquote{Киевский митрополит Петр
Могила и его сподвижники}, 1883; Сорокин Г. И. \enquote{Свадьбы и свадебные песни у
малороссов и великороссов}, 1891; Никитин А. Ф. \enquote{Синагоги иудейские как места
общественного богослужения}, 1891; Словарь книжной малорусской речи,1888 и др.  

\ii{01_01_2022.fb.fb_group.story_kiev_ua.1.izdateli_bratja_korchak_novickie.pic.5}

В 1880-90-е годы издавалась серия под названием «Практическая школа», автором
которой была Мариуца-Гринева М. Н. Вот названия нескольких выпусков:
\enquote{Приготовление эдамского или красного голландского сыра}, \enquote{Откорм свиней},
\enquote{Маслоделие}.

\ii{01_01_2022.fb.fb_group.story_kiev_ua.1.izdateli_bratja_korchak_novickie.pic.6}

Совершенно невозможно не упомянуть и о журнале \enquote{Кіевская Старина}, который
весьма длительное время (1882–1900) печатался в типографии Георгия Трофимовича.
Об этом  журнале, который с 1900 по 1906 выходил в университетской типографии
младшего брата, Николая Трофимовича, более подробно говорилось в первой части. 

\ii{01_01_2022.fb.fb_group.story_kiev_ua.1.izdateli_bratja_korchak_novickie.pic.7}

Как многие издательства, типография Г. Т. Корчак-Новицкого принимала заказы на
печатание литографий с видами Киева.

А вот ещё одна страничка деятельности Г. Т. Корчак-Новицкого. Связана она с
сотрудничеством с А.А. Русовым — учёным, экономистом, крупнейшим земским
статистиком, общественным деятелем, а также фольклористом, собирателем
древностей и книгоиздателем, членом Юго-Западного отделения Русского
географического общества. И, к слову сказать, супругом  С.Ф. Русовой ‒
педагога, прозаика, литературоведа и общественного деятеля, одной из пионерок
украинского женского движения.

\ii{01_01_2022.fb.fb_group.story_kiev_ua.1.izdateli_bratja_korchak_novickie.pic.8}

В работе черниговского автора В. Науменко об этом говорится так. 

А. А. Русов, "живя в Киеве в начале 80-х годов, занят был печатанием своего
труда по описанию Нежинского уезда, а также другой временной работой. Этой
другой работой было неофициальное редакторство его в газете «Труд» (официальным
редактором считался издатель Г.Т. Корчак-Новицкий), выходившей в Киеве, если не
ошибаемся, с 1880 по 1882 год. Историю возникновения этой газеты передавали
тогда, может быть, анекдотически, так: состоятельный человек Г.Т.
Корчак-Новицкий купил в Киеве доходный дом, а так как в доме помещалась
типография, то он решил купить и типографию; при типографии печатался какой-то
газетный листок, – буду и я издавать газету, решил г. Корчак-Новицкий, но
только пожелал переделать её в нечто более приближённое к настоящей газете.

\ii{01_01_2022.fb.fb_group.story_kiev_ua.1.izdateli_bratja_korchak_novickie.pic.9}

Было известно также, что вскоре с Г. Т. Корчаком-Новицким вступили в переговоры
некоторые из тогдашних черниговских видных земцев, желая газету «Труд» сделать
не только идейной газетой, но и отвечающей политическим настроениям их группы,
причём в программу их входило способствовать развитию в обществе украинской
идеи. Каковы были финансовые соглашения между Корчаком-Новицким и земцами, я не
знаю, но было поставлено условие, что во главе газеты станет А. А. Русов и те
сотрудники, которые будут работать вместе с ним, а за Корчаком-Новицким
остаётся только официальное представительство". 

\ii{01_01_2022.fb.fb_group.story_kiev_ua.1.izdateli_bratja_korchak_novickie.pic.10}

В 1881 году к этой газете вышло обширное приложение: \enquote{Дело о совершенном 1-го
марта 1881 года злодеянии, жертвою коего пал в Бозе почивший государь
император Александр Николаевич}.

\ii{01_01_2022.fb.fb_group.story_kiev_ua.1.izdateli_bratja_korchak_novickie.pic.11}

Далее в статье В.Науменко сообщается: "Несмотря на... разношёрстный состав, наши
редакционные обсуждения носили очень мирный, задушевный характер; беда только в
том, что среди нас не было ни одного газетного борзописца, благодаря чему мы в
«Труде» во всём запаздывали, а если принять во внимание, что газета-то выходила
три раза в неделю, то совсем станет понятным, почему нам нельзя было угнаться
за «Зарёй». Не без трениев было и в цензурно-административном отношении, так
что не раз тягали раба Божия Георгия (Корчак-Новицкого) к объяснениям с
властями, включительно до генерал-губернатора Дрентельна. Протянув кое-как ещё
1882 год, г. Корчак-Новицкий (как говорили, по требованию соучастников своих,
земцев), заявил, что он прекращает издание. Как раз в это время и А.А. Русов
получил приглашение ехать в Херсонщину для статистических работ».

Редакция располагалась по улице Михайловской, 4. Этот дом и купил
Корчак-Новицкий. 

Ещё один его адрес ‒ Большой (Бибиковский) бульвар, 28 (сейчас на этом месте
отель \enquote{Hilton}). До 1889 г. нумерация бульвара была несколько иной, дом
значился под №26, и проживал в нём другой известный издатель (и несколькими
годами ранее тоже имевший дачу в Святошино) Пётр Александрович Барский, правнук
знаменитого киевского зодчего Ивана Григорьевича Григоровича-Барского.

Этот особняк указывается и как доходный. В разделе \enquote{Лица и учреждения,
выбравшие из киевской гор. Управы промысловые свидетельства на торговые
предприятия 1 и 2 разр. и на промышл. Предпр. 1, 2, 3 и 4 разрядов} (\enquote{Весь
Киев} за 1913 год) значится Корчак-Новицкий Г.Т. (меблированные комнаты)
Бибиковский бул. 28, а среди адресов жителей Киева он фигурирует как
проживающий по «Б.-Бульв. 50» (тот же Бибиковский). 

Неизменным остаётся его святошинский адрес: ул. Северная  43, по сквозной
нумера-ции участок 110. Сейчас это ул. Феодоры Пушиной, сквер на пересечении с
пр. Палладина со стороны ул. Чернобыльской, рядом с трамвайным кольцом,
существовавшим с начала ХІХ и до конца ХХ ст.

В 1910 году Георгий Трофимович купил сразу три участка: № 107, 108 (с одной
стороны ул. Трамвайной, нынче пр. Палладина, примерно №16, где ТЦ “Varus”,
тогда эта земля числилась казённой) и № 110 ‒ по другую. Но уже с 1911 за ним
известен только 110-й участок.

Улица Северная, как следует из её названия, проходила по северной части дачного
посёлка и была главной её артерией. Она начиналась от 2-й просеки (сейчас ул.
Краснова) и заканчивалась, где и сейчас: у поймы речки Борщаговки (иногда её
ошибочно называют Нивкой), позже запруженной. Рядом с Северной улицей некоторое
время находилось Святошинское Общество Красного Креста. На перекрёстке Северной
и Торговой улиц (современных Ф.Пушиной и Семашко) был базар, просуществовавший
до конца ХХ века. 

Напротив него, уч. 83, позже 84, жила семья известных изготовителей музыкальных
инструментов Кернтопф. «Депо роялей, пианино и фисгармоний И.Кернтопф и Сын.
Киев, Крещатик 33. Телефон 808. Собственная фабрика в Варшаве существует с 1840
года», сообщало об их фирме объявление в газетах того времени. 

Чуть дальше за базаром на противоположной стороне ул. Северной, уч. 105, жил
первый настоятель Святошинской церкви свт. Николая прот. Константин
Сементовский с семьёй.

Совсем рядом с дачей  Г. Т. Корчак-Новицкого (участок 112) с 1912 года находилось
городское училище ‒ одно из нескольких учебных заведений дачного посёлка, в
которых учились не только дети постоянно проживавших в Святошино, но и
приезжавших из города. Эта дачная местность привлекала прекрасной природой ‒
соснами и дубами с целебными свойствами, уникальным воздухом и, конечно,
удобным сообщением с Киевом: трамвай ходил от Триумфальной арки (где сейчас
Воздухофлотский мост), поезд от вокзала до станции Святошин (здание
сохранилось).

А ещё через 2 участка от городского училища проживал статский советник
Отто-Адольф Мартинович Редлих с супругой Марией Ивановной. Их надгробья на
Лукьяновской кладбище занесены в реестр объектов, находящихся под охраной
государства. 

В конце Северной улицы (угол с Чернобыльской, тогда Беличанской, где дом с
магазином «Сельпо») была дача госпожи Татаровской, фотография красивого
нетипового особняка и его хозяев сохранилась. 

Оба брата Корчак-Новицкие жили в Святошино, но по разные стороны от Шоссе (пр.
Победы): один на юге, другой на севере посёлка. Наверняка бывали друг у друга в
гостях, обсуждали издательские планы, ведь основные направления их деятельности
постоянно пересекались. 

В Святошино было принято спокойное нешумное общение, часто были
импровизированные концерты. Праздники нередко отмечали сообща, совершали походы
в лес, отправлялись купаться на пруды, посещали парк с его увеселениями. 

А ещё здесь активно входили в жизнь новшества в виде велосипедных гонок и ‒
реже ‒ поездки на первых в городе автомобилях. Это не удивительно, ведь заядлые
автомобилисты к. ХІХ ‒ нач. ХХ с удовольствием проводили на своих Святошинских
дачах немало времени. Неподалёку от Корчак-Новицких располагалась дача
владельца одного из первых авто в Киеве Александра Николовича Терещенко, в
Святошино жили и Каульфусы ‒ наследники и последователи одного из основателей
киевского клуба автомобилистов (а заодно и пианиста, и держателя склада
велосипедов) Бернгарда Каульфуса, и другие.

Но Георгий Трофимович отдавал предпочтение лошадям. 

История рысистых бегов в Киеве официально начинается с 1867, тогда же и
деятель-ность \enquote{Киевского общества испытания лошадей}. Ипподром того времени
находился на Лукьяновке (территория танкового училища). Но это общество
просуществовало недолго, и бега вынуждены были проводить зимой по льду, летом
на заброшенном ипподроме. 

В 1885 году возникает новое общество – Киевское общество охотников конского
бега, президентом его стал князь Николай Васильевич Репнин (-Волконский) –
киевский губернский предводитель дворянства. По его инициативе был создан устав
общества, избрано руководство, в которое вошли известнейшие меценаты и
конезаводчики того времени: Л.И.Бродский, А.Н.Терещенко, В.И.Толли,
Г.Т.Корчак-Новицкий, А.И.Паншин. Именно при их правлении были проведены первые
в истории зимние и весенние бега, а также основано первое в России рысистое
дерби стоимостью десять тысяч рублей. С 1889 года на Киевском ипподроме начал
функционировать тотализатор.

Строится новый ипподром на Печерске, почти рядом с райадминистрацией, на ул.
Суворова. Изначально деревянный (беседка, как тогда назывались трибуны), а с
1915-16 ‒ красивое каменное сооружение (архитектор Валериан Рыков) с крытыми
трибунами, рестораном, залами, одно из излюбленных мест отдыха и возможной
прибыли дореволюционных киевлян.

Ещё одним начинанием, но косвенно связанным с издательской деятельностью
Г.Т.Корчак-Новицкого, было создание при его типографии хора наборщиков. И
произошло это благодаря выдающемуся музыканту Якову Степановичу Калишевскому. 

В 1873 году, 17-летним юношей, обладающим незаурядными регентскими
способно-стями и прекрасным голосом, он организовывает хор наборщиков при
типографии мэтра издательского дела С. Кульженко и через год на Благовещение
даёт первый в Киеве концерт духовной музыки «в зале Дворянского собрания на
Крещатике. Исполняли произведения Д. Бортнянского, А. Веделя, прот. Петра
Турчанинова. Концерт имел огромный успех и был повторен в Вербное воскресенье
того же года в Киеве и 1875 в Харькове».

Владимира Беляков  в своём прекрасном «Эскизе к монографии о Якове Калишевском»
пишет: "В то время в Западной Европе было принято содержать при промышленных
предприятиях любительские ансамбли (оркестры, хоры, театры и т.п.), что нашло
одобрение и подражание у киевских пролетариев. Так, в 1881 г. при Киевских
Главных железнодорожных мастерских открылась музыкальная школа, чтобы
подготовить музыкантов для духового оркестра...

В 1882 г. Я. Калишевский создаёт хор при типографии Корчак-Новицкого и
руководит им два года; в 1883 г. ‒ при школах Сулимовских благотворительных
заведений. Этот хор во время царского посещения 1885 г. пел перед Их
Величеством.

Созданные Я. Калишевским «типографские» хоры скоро обратили на себя внимание
качеством звучания, классическим репертуаром и, главное, интересными образцами
интерпретаций». 

По благословению киевского митрополита Платона (Городецкого) Яков Калишевский
был приглашён возглавить хор при Кафедральном Софиевском соборе. И уже через
два года этот хор стал лучшим в Киеве. Почти 40 лет (1882-1919) трудился Яков
Степанович регентом в Софийском соборе, а с 1919 до своей кончины в 1923 служил
руководителем хоровой капеллы им. Н. Лысенко.

Его мастерство высоко ценил П. Чайковский, Н. Лысенко, художники М. Нестеров и
В. Васнецов. 

Профессионал такого высокого уровня занимался с рабочими типографии
Г. Т. Корчак-Новицкого.

Вот, пожалуй, и всё, что на сегодня можно вкратце рассказать о киевских
издателях братьях Корчак-Новицких.

Первыми фамилиями для наших предков служили прозвища, даваемые по особенностям
внешнего вида либо по роду занятий. Позже иногда присоединялось и название
местности.

«Корчак» во многих славянских языках означало «пень, корень дерева». Так могли
назвать человека кряжистого, крепкого, сильного, выдержанного. А в некоторых
диалектах словом «корчак» называли глиняный кувшин и, порой, его изготовителя ‒
гончара, мастера почётной и незаменимой профессии. Человека, пользовавшегося
уважением.

Издатели Корчак-Новицкие вполне оправдали свою фамилию ‒ прозвище, некогда
полученное их предками. Несколько десятилетий они держали высокую планку в
своём деле, проявляя недюжинное трудолюбие, выдержку и целеустремлённость,
соединённые с мастерством. И были одними из лучших среди тех, кто развивал
книгоиздательское дело в нашем городе.
