% vim: keymap=russian-jcukenwin
%%beginhead 
 
%%file 19_05_2021.fb.taksjur_jan.1.duma_ukr_politolog
%%parent 19_05_2021
 
%%url https://www.facebook.com/taksyur/posts/4203576796355111
 
%%author Таксюр, Ян
%%author_id taksjur_jan
%%author_url 
 
%%tags obschestvo,poezia,politika,politologia,taksjur_jan.pisatel,ukraina
%%title ДУМА ПРО УКРАЇНСЬКОГО ПОЛІТОЛОГА
 
%%endhead 
 
\subsection{Дума про Українського Політолога}
\label{sec:19_05_2021.fb.taksjur_jan.1.duma_ukr_politolog}
\Purl{https://www.facebook.com/taksyur/posts/4203576796355111}
\ifcmt
 author_begin
   author_id taksjur_jan
 author_end
\fi

Политологи у нас становятся популярнее поп-звёзд. Их слушают и читают с
упоением. Миллионы сограждан в их речах напряжённо пытаются различить очертания
своего будущего. И политологи стараются. Умно, блестяще, остроумно они
разъясняют согражданам, чем одна компания мерзавцев отличается от другой. Кто
из них больше украл у этих самых сограждан. И у кого какой рейтинг. 

Признаюсь, я сочувствую нашим политологам, глядя на тот «материал», за жизнью
которого они вынуждены следить. Что касается моих соотечественников, то они,
послушав прогнозы и приняв сердечные капли, нервной рукой пишут в комментах:
«Что же нам делать чтобы не пришёл полный... коллапс?».

Политологи мудро и устало улыбаются, а потом, как доктор из американского
фильма, беспощадно предрекают, глядя на карту Украины: «Боюсь, мы её теряем».
Некоторые говорят: «Держитесь». Кто-то предлагает «держать кулаки». А я вот
думаю, если бы миллион слушающих, вместе с говорящими аналитиками, попросили
Бога, наверняка, что-то изменилось бы. Но мне говорят: «Опять вы со своим
Богом! Мы тут серьёзными делами заняты – обсуждаем, кого из воров избрать в
парламент и от кого из них будет меньше вреда». Хорошо, делаю я последнюю
попытку. Оставайтесь при своём. 

Хотите верить в нуль, в эволюцию, в своих предков-приматов – верьте. Но
попробуйте хоть однажды, в виде эксперимента, сказать: «Господи, если Ты
всё-таки есть, и я ошибаюсь (ну, могу же я иногда ошибаться) прости нас и
помоги нам. И, быть может, Бог, Который «везде сый и вся исполняяй», то есть,
пребывает всюду и всё Собой наполняет – услышит плач неразумных Своих созданий.

Пока писал, вспомнил одно своё стихотворение «в тему».   

ДУМА ПРО УКРАЇНСЬКОГО ПОЛІТОЛОГА

Жив на світі політолог,
Політолог молодий.
В нього був приємний голос,
Гарно вдягнутий завжди.
Де у нас які загрози,
Де підйом, а де застой,
Він робів щодня прогнози.
Аналитик був ще той.
І тому тримавсь зухвало – 
Горда впевнена хода.
Та прогнози не збувались.
Отака була біда.
І журився, хоч таємно,
Молодий отой діяч:
– Як же все це неприємно!
Не збувається, хоч плач!
І від горя бідолага
Якось спав цілу добу,
І прийшов до нього Янгол.
Той, що змалку в нього був.
– Чом, – заплакав політолог, – 
Прогнозую, а дарма?
Он, сміються вже навколо,
Мовляв, успіхів нема!
Янгол каже: – Досить суму.
Не дося̀гнеш ти мети,
Бо Господь інакше дума.
Дума Він не так, як ти.
І тому не вийде путнє,
Не побачиш перемог.
Ти не можеш знать майбутнє,
Бо ти хлопчик, а не Бог.
– Що ж робить? – кричить сердешний, – 
Вже дійшов я до межі!
– Помолись, щоб знать прийдешнє.
Кажи: «Боже, поможи!»
А як будуть десь  дебати,
(Ох, дивися, не нашкодь)
Як почнеш прогнозувати,
Додавай: «Як дасть Господь»
Політолог зблід відверто:
– То вже точно засміють!
Бо для нас, крутих експертів,
Бог чи віра – то все муть.
Ну, дозвольте, я вгадаю
Хоч мізерне: «Де-не-де
Дощ рясний для урожаю
Зранку у четвер піде»
– Прогнозуй… Ну, наче діти, – 
Янгол тихо промовля, – 
Все одно збиравсь полити
Наш Господь Свої поля.
І назавтра «Будуть зливи!» – 
В інтернет пішла стаття.
Дощ піщов, і був щасливий
Аналітик, як дитя. 
Політолог той – це кум мій.
Тож повірте ви мені:
Зовні всі вони розумні,
А всередині дурні.
