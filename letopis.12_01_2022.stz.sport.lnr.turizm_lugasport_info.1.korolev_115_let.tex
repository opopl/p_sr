% vim: keymap=russian-jcukenwin
%%beginhead 
 
%%file 12_01_2022.stz.sport.lnr.turizm_lugasport_info.1.korolev_115_let
%%parent 12_01_2022
 
%%url https://turizm.lugasport.info/12-yanvarya-115-let-so-dnya-rozhdeniya-s-p-koroleva.html
 
%%author_id 
%%date 
 
%%tags 
%%title 12 января – 115 лет со дня рождения С.П. Королева
 
%%endhead 
\subsection{12 января – 115 лет со дня рождения С. П. Королева}
\label{sec:12_01_2022.stz.sport.lnr.turizm_lugasport_info.1.korolev_115_let}

\Purl{https://turizm.lugasport.info/12-yanvarya-115-let-so-dnya-rozhdeniya-s-p-koroleva.html}
\ifcmt
 author_begin
   author_id  
 author_end
\fi

Сергей Павлович Королев – советский ученый и конструктор в области ракетного
строения и космонавтики,  космических кораблей, основоположник практической
космонавтики.

Сергей Павлович родился 12 января 1907 года в Житомире. В 1924 году окончил
строительную профессиональную школу города Одессы. Спроектировал свой первый
планер. В этом же году поступил в Киевский политехнический институт по профилю
авиационной техники.

\ii{12_01_2022.stz.sport.lnr.turizm_lugasport_info.1.korolev_115_let.pic.1}

В 1926 году Королев перевелся в Московское высшее техническое училище имени
Баумана, где проявил свои способности опытного конструктора и планериста.

В 1929 году Сергей Павлович защитил дипломный проект легкомоторного
двухместного самолета СК-4, предназначенный для достижения рекордной дальности
полета. После защиты получил профессию инженера-аэромеханика. Параллельно С.П.
Королев конструировал еще один рекордный аппарат-планер СК-3 «Красная звезда»,
на котором в октябре 1930 года впервые в мире была выполнена петля Нестерова в
свободном полете.

С.П. Королев вместе с советским ученым и изобретателем Ф.А. Цандером в 1931
году организовал одну из первых ракетных организаций – группу изучения
реактивного движения (ГИРД), руководил строительством и полетными испытаниями
ракет.

В апреле 1932 года ГИРД стала государственной научно-конструкторской
лабораторией по разработке летательных аппаратов, в которой создавали и
запускали первые жидкостные отечественные баллистические ракеты ГИРД-09 и
ГИРД-10.

\ii{12_01_2022.stz.sport.lnr.turizm_lugasport_info.1.korolev_115_let.pic.2}

В 1938 году Сергей Павлович был репрессирован и осужден на длительный срок
заключения. Начало срока отбыл на Колыме, а в годы Великой Отечественной войны
участвовал в создании самолета Ту-2 в московском Конструкторском бюро при
Центральном Конструкторском бюро НКВД (Народный комиссариат внутренних дел) в
группе с авиаконструктором А.Н. Туполевым.

В 1944 году освобожден согласно Указу Президиума Верховного Совета СССР и в
1945 году командирован в Германию, где в составе Технической комиссии
знакомился с немецкой трофейной ракетной техникой. За успешное выполнение работ
в годы войны по созданию ракетных ускорителей для самолетов награжден первым
орденом «Знак Почета».

В 1946 году был создан единый научно-исследовательский институт «Нордхаузен» в
Германии, директором которого был назначен генерал-майор Л.М. Гайдуков, а
главным инженером – С.П. Королев. В 1948 году С.П. Королев начал
летно-конструкторские испытания баллистической ракеты Р-1 и в 1950 году успешно
сдал ее на вооружение.

В 1955 году С.П. Королев, М.В. Келдыш, М.К. Тихонравов вышли в правительство с
предложением о выведении в космос при помощи ракеты Р-7 искусственного спутника
Земли. И уже 4 октября 1957 года С.П. Королев запустил на околоземную орбиту
первый в истории человечества искусственный спутник Земли. Работы над
спутниками велись параллельно с подготовкой полета в космос человека.

Создав первый пилотируемый космический корабль «Восток», С.П. Королев 12 апреля
1961 года реализовал первый в мире полет человека Юрия Алексеевича Гагарина по
околоземной орбите.

В 1965 году руководил полетом корабля «Восток-2», когда космонавт Алексей
Леонов впервые в истории вышел в открытый космос. Участвовал в запусках
автоматических межпланетных станций «Луна-5», «Луна-6»,  «Луна-7», «Луна-8»,
«Венера-2», «Венера-3», летательного аппарата  «Зонд-3», спутника связи типа
«Молния-1».

С.П. Королев скончался 14 января 1966 года. Урна с его прахом установлена в
Кремлевской стене на Красной площади.

Память о С.П. Королеве увековечена в многочисленных памятниках:

\begin{itemize}
  \item в городе Королев (Московская область) на проспекте Королева;
  \item возле учебно-лабораторного корпуса МГТУ им. Н.Э. Баумана (Москва);
  \item памятник-бюст на проспекте Королева в Санкт-Петербурге на территории спортивной школы;
  \item в Омске к 300-летию города перед ДК Химик на проспекте Королева;
  \item памятник-бюст в Самаре около национального исследовательского университета имени Королева;
  \item бюст во дворе школы им. С.П. Королева в г. Тольятти;
  \item барельеф-портрет С.П. Королева в Парке покорителей космоса им. Юрия Гагарина, открытом 9 апреля 2021 года в Саратовской области.
\end{itemize}

Также памятники конструктору в области ракетного строения и космонавтики стоят
в Чебоксарах, Железногорске, Байконуре, Киеве, Житомире и Мадриде.

Памятник Юрию Гагарину и Сергею Королеву «Перед полетом» находится на
набережной Волги в городе Энгельсе Саратовской области. Эти деятели увековечены
также в Таганроге и в подмосковном Королеве.
