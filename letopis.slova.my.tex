% vim: keymap=russian-jcukenwin
%%beginhead 
 
%%file slova.my
%%parent slova
 
%%url 
 
%%author 
%%author_id 
%%author_url 
 
%%tags 
%%title 
 
%%endhead 
\chapter{Мы}
\label{sec:slova.my}

%%%cit
%%%cit_pic
%%%cit_text
\emph{Мы} ещё умеем дружить и не прощаем предательства. \emph{Мы} ещё помним,
что любить человека – отнюдь не синоним слова «спать» с ним. \emph{Мы}
пережили развал страны, 90-е, парочку мировых кризисов и несколько воин
(особенно внутренних).  До \emph{нас}, наконец, дошло, что наши родители делали
для \emph{нас} всё, что могли. Даже если не делали ничего.  \emph{Мы} теперь
уже точно осознали, что «круто» – это не клубы каждые выходные, не градусы в
стакане, а здоровый цвет лица, крепкий сон и дорогие люди рядом.  \emph{Мы}
вручную рисовали «поля» в тетрадях, дрались исключительно «до первой крови» и
уже многих похоронили. \emph{Мы} – самое закаленное и, не в обиду другим, самое
прогрессивное поколение. \emph{Нам} повезло родиться ещё до того, как у детей
отобрали свободу и право выбора. \emph{Мы} привыкли никого не слушать, ни на
кого не надеяться и ни от кого ничего не ждать.  \emph{Мы}, дети 60-80-х, всё
привыкли делать сами – потому что лучше \emph{нас} никто не сделает...» Автор -
Галина Ляшевская
%%%cit_comment
%%%cit_title
\citTitle{Советскому поколению повезло родиться до того, как у детей отобрали право выбора}, 
Елена Лукаш, strana.ua, 13.06.2021
%%%endcit

%%%cit
%%%cit_pic
%%%cit_text
Фактически \emph{мы} живем в двух непересекающихся сообществах. Есть правящее. У него
«героям слава», проспекты Бандеры и Шухевича, годовщины Майдана и постоянная
гордость за какую-то иррациональную хрень, вроде украинского борща. И есть
второе сообщество – наблюдателей и терпящих. Оно тоже неоднородное, пестрое и
со своими тараканьими представлениями об этом мире
%%%cit_comment
%%%cit_title
\citTitle{Бандеровцы при власти создали в Украине уютненькую Уганду}, 
Игорь Лесев, strana.ua, 13.06.2021
%%%endcit

%%%cit
%%%cit_head
%%%cit_pic
%%%cit_text
Более того, перед \emph{нами}, по сути, ясно и четко сформулированная идеология – не
только для внешней, но и для внутренней политики. Если \emph{мы} действительно готовы
строить новую Россию, то это может быть только по-настоящему культуроцентричная
страна, духовной осью которой станут князь Владимир, Андрей Рублев, Пушкин и
великая русская литература 19-го века. Литература, которую молодой Томас Манн
назвал «святой», «достойной преклонения» и единственной, в сравнении с другими
великими литературами, христианской литературой Нового времени, несущей
«целительное, освящающее воздействие».  Великая русская литература – главный
мировой бренд России. Он – возможно, самое большое наше национальное достояние.
Толстой, Достоевский, Пушкин, Чехов – это именно то, что создает образ России
во всем мире. И, конечно, этот величайший «русский бренд» \emph{мы} должны взять под
защиту, прежде всего у себя дома. А значит – снова должны стать
культуроцентричной державой. Снова начать воспитывать не обывателей, не
потребителей, а людей прежде всего культуроцентричных. В свое время это не
побоялся сделать Сталин, потеснив Маркса с Лениным с постаментов и поставив в
центр новой строящейся империи Пушкина (в свое время мы об этом писали)
%%%cit_comment
%%%cit_title
\citTitle{Россия должна жить по заветам Киплинга}, 
Владимир Можегов, vz.ru, 18.06.2021
%%%endcit

%%%cit
%%%cit_head
%%%cit_pic
%%%cit_text
Но не \emph{мы}, потому что \emph{мы} уцепились в ресентимент. \emph{Мы} уцепились за вот эти
свойность, группы интересов по языку и по этнической культуре и дальше \emph{мы} не
двигаемся, мышление остановилось в этих мыслительных патернах, в этих
когнитивных представлениях, в этих ментальных моделях. Украину \emph{нам} не
сохранить. Не сохранить. Ее не будет. Как именно она прекратит существование не
очень уже и важно. Но корень здесь, он в этом
%%%cit_comment
%%%cit_title
\citTitle{Сергей Дацюк: Украина сегодня - не просто попрошайка, она на мусорнике истории}, 
Сергей Дацюк; Людмила Немыря, hvylya.net, 28.06.2021
%%%endcit

%%%cit
%%%cit_head
%%%cit_pic
%%%cit_text
Адже потрібно захищати не російськомовних і не україномовних. Варто захищати
\emph{нас} від самих себе, від тієї ненависті, яку \emph{ми} можемо носити в
собі і виявляти до інших. Українці не розуміють одне одного не через мову, а
через небажання слухати, чути і сприймати.  Мова не має нічого спільного з
різними проявами нелюдяності й шовінізму
%%%cit_comment
%%%cit_title
\citTitle{Українці не розуміють одне одного не через мову, а через небажання слухати, чути і сприймати}, 
Юлія Мендель, www.pravda.com.ua, 07.07.2021
%%%endcit

%%%cit
%%%cit_head
%%%cit_pic
%%%cit_text
Українська культура в конкуренції з російською на власній території.  Треба
сказати, що на час здобуття незалежності \emph{ми} мали два потужні ресурси для
того, щоб позбутися спільного з Росією маскульту і сформувати
інформаційно-культурний простір на своїй, питомій основі.  Перше джерело – це
багата українська культура, яка вийшла зі стану «салону у в’язниці», якщо
скористатися метафорою польського філософа Лєшека Колаковського, й поповнилась
масивом заборонених за тоталітарної доби творів та значним художнім і науковим
доробком діаспори.  Другий ресурс – це повноцінне українськомовне середовище
західних областей, що має сприятливий ґрунт для швидкого поширення національної
культури у розмаїтих формах її побутування – від елітарних до масових
%%%cit_comment
%%%cit_title
\citTitle{Українська мова в культурному просторі держави. Протистояння триває}, 
Лариса Масенко, www.radiosvoboda.org, 11.07.2021
%%%endcit

%%%cit
%%%cit_head
%%%cit_pic
%%%cit_text
Коли я був малим, по телевізору крутили рекламу про те, що "\emph{нас 52 мільйони}". І
хоча купа людей досі вважає мене малим, тобто з того часу не так багато води в
Тисі спливло, та \emph{нас} уже далеко не пів сотні мільйонів.  Тепер \emph{нас} на цілу
чверть менше – сорок з невеликим гаком, який ніхто не в силах підрахувати
точно. Припускаю, що тих, які живуть тут, а не перебиваються десь по закордонах
(у самообмані, що вони там лише тимчасово, щоб заробити на
квартиру/ремонт/пенсію), буде ще менше – наприклад, 37 мільйонів. Принаймні
таку цифру минулого року називав міністр Дубілет, оцінюючи перспективи
цифрового перепису
%%%cit_comment
%%%cit_title
\citTitle{Українці помирають, їдуть, зникають. Настане час, коли тут нікого не залишиться}, 
Андрій Любка, gazeta.ua, 31.07.2021
%%%endcit
