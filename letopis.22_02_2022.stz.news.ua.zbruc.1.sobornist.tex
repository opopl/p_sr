% vim: keymap=russian-jcukenwin
%%beginhead 
 
%%file 22_02_2022.stz.news.ua.zbruc.1.sobornist
%%parent 22_02_2022
 
%%url https://zbruc.eu/node/110739
 
%%author_id ljubka_andrii
%%date 
 
%%tags __feb_2022.putin.priznanie,sobornist,ukraina
%%title Ось так вона й виглядає, соборність
 
%%endhead 
 
\subsection{Ось так вона й виглядає, соборність}
\label{sec:22_02_2022.stz.news.ua.zbruc.1.sobornist}
 
\Purl{https://zbruc.eu/node/110739}
\ifcmt
 author_begin
   author_id ljubka_andrii
 author_end
\fi

Не знаю, чи в ці дні є в Україні хтось, хто може нормально і вчасно заснути. Я
– не можу. Прокидаюся вночі й перевіряю новини, прокидаюся зранку – і
насамперед читаю стрічку інформагентств. Новини все гірші.

\ii{22_02_2022.stz.news.ua.zbruc.1.sobornist.pic.1}

Працювати неможливо. Сфокусуватися на чомусь годі. Споживання новин отруює й
збільшує тривожність, але примусити себе не відкривати й не читати їх кожні
кілька хвилин не виходить. Спроби втечі – на природу, на лижі, в алкоголь – теж
не додають спокою. Тривога росте всередині, повністю підкоривши собі
повсякденне життя.

\ii{22_02_2022.stz.news.ua.zbruc.1.sobornist.pic.2}

Дехто каже й пише в ці дні: добре тобі, ти живеш в Ужгороді, тобі немає про що
хвилюватися. Частково це правда: Ужгород настільки далеко від російського
кордону, що раптова пряма атака сьогодні вночі навряд чи відбудеться.

Водночас складне й мультиетнічне Закарпаття – зручний плацдарм для провокування
хаосу, терористичних актів, що миттєво отримають міжнародний розголос, тож
сильно тут теж не розслабишся. Хоча, звісно, жити в ці дні на Закарпатті – це
не те саме, що жити на мушці в російського солдата.

Але повірте, що і в Ужгороді, на самому кордоні з ЄС, спокою нема. Від новин і
тут не сховаєшся. Після них і тут не заснеш. Хоча я уявляю, як важко тим людям,
що живуть ближче до Росії. Як тривожно їм – щодня й щогодини. Я переживаю за
них.

За Ярослава з Чернігова, за Сергія з Харкова, за Євгена з Сум, за Сергія з
Маріуполя, за Світлану з Одеси, за всіх тих, кого можу назвати знайомими й
друзями, з ким був і набувався, говорив і сперечався, за всіх тих, що й роблять
свої міста і регіони Україною.

Але не тільки за них. Я переживаю за людей, яких не знаю й ніколи не бачив, за
українців, за своїх співвітчизників. Я боюся, що в їхні будинки можуть влучити
снаряди, що до них додому увірвуться чужі озброєні люди, що їм доведеться
босими тікати через морозну ніч. Мені хочеться їх обійняти, підтримати,
заспокоїти. Хочеться, щоб вони знали, що вони не самі.

Чорт забирай, це і є те, що в патріотичних віршах називають соборністю, а в
наукових працях – уявною спільнотою, що творить націю. Відчуття гордості, коли
спортсмен з Харкова виборює медаль і піднімається на п’єдестал з українським
прапором, і відчуття тривоги, коли на мешканців Харкова скалить зуби озвіріла
істота.
 
Хай би як банально це не звучало, але країна – це організм, а її кордони –
шкіра, і куди б не вдарили – біль відчують усі. Не може бути нападу на Харків
чи Маріуполь – напад буде на всю Україну. Це війна не проти чернігівців чи
одеситів, а проти українців, України як ідеї, яка всіх нас тримає вкупі.  

Так і живемо, об’єднані тривожністю, страхом, безсонням. Тулячись поближче
плечем до плеча, ліктем до ліктя, зближуємося, щоб зігрітися і підтримати одне
одного. Слобожанщина ще ніколи не була такою близькою до Закарпаття, як нині, в
час загроз і погроз; Донбас ще ніколи не був таким близьким до Закарпаття, як
нині, коли в окопах там сидять бійці нашої Закарпатської гірсько-штурмової
бригади.

Країна, що була географічною картою в шкільному підручнику чи на екрані
телевізійного прогнозу погоди, набуває реальних обрисів людей і облич, за яких
переживаєш, яких хочеш підтримати і захистити. Безсоння оселилося у кожній хаті
від Ужгорода до Донбасу, ми не спимо разом, ми разом у цьому човні під назвою
Україна. Ось так вона й виглядає, соборність.

 
