% vim: keymap=russian-jcukenwin
%%beginhead 
 
%%file 16_11_2020.news.ua.strana.4.sadovskii_death
%%parent 16_11_2020
 
%%url https://strana.ua/news/301298-leonid-sadovskij-umer-chto-izvestno-ob-ubijstve-professora-instituta-iskusstv-v-kharkove.html
%%author 
%%author_id 
%%tags 
%%title 
 
%%endhead 
 
\subsection{Убийство профессора Садовского. Как в Харькове погиб известный режиссер и чем он знаменит}
\Purl{https://strana.ua/news/301298-leonid-sadovskij-umer-chto-izvestno-ob-ubijstve-professora-instituta-iskusstv-v-kharkove.html}

22:33, 16 ноября 2020
\index[deaths.rus]{Садовский, Леонид Викторович! 16 ноября 2020, Харьков}
\index[names.rus]{Садовский, Леонид Викторович!Режиссер, профессор (Харьков)}

\ifcmt
pic https://strana.ua/img/article/3012/98_main.jpeg
caption Леонид Садовский. Фото: num.kharkiv.ua
\fi

В понедельник, 16 ноября, в Харькове скончался известный человек - Леонид
Садовский.\Furl{https://kharkov.strana.ua/301295-v-kharkove-do-smerti-izbili-prepodavatelja-i-rezhissera-leonida-sadovskoho.html}
Он был заслуженным деятелем искусств Украины, режиссером, профессором.

71-летний Садовский был доцентом Харьковского национального университета
искусств им. Котляревского. Он возглавлял кафедру мастерства актера и был
худруком театра \enquote{Мастерская 55}.

Пожилого деятеля искусств настигла страшная смерть. На Леонида Садовского
напали, когда он возвращался вечером домой. Профессор получил тяжкие
телесные повреждения. С огромными усилиями он добрался до своей квартиры.
Спустя два дня скончался в больнице от полученных травм. 

Студенты Садовского скорбят об утрате. Среди них профессор имел репутацию
местного рок-стара. Одну из его последних постановок - \enquote{Иисус Христос -
суперзвезда} - успели представить в Харькове. 

\enquote{Страна} рассказывает, что известно о жизни и трагической смерти Леонида
Садовского. 

\subsubsection{Убийство Леонида Садовского - что известно }

Нападение на Леонида Садовского случилось в субботу, 14 ноября. Мужчина
возвращался домой через пустырь. Вдруг на него напал неизвестный. 

Профессора атаковали сзади. Его избили, в том числе по голове. На некоторое
время Садовский потерял сознание. Придя в себя, он с трудом добрался домой. Его
супруга позвонила в скорую. 

Жертву нападения госпитализировали в харьковскую БСМП. В полицию о случившемся
после 18 часов сообщили медики, которые оказывали первую помощь Садовского. 

Силовики не уточняют детали преступления. Неизвестно, обокрали Садовского или
нет, поступали ли ему угрозы. 

Леонид Садовский провел в больнице двое суток. Студенты и коллеги профессора
сообщали, что профессору была сделана операция, после чего он был помещен в
реанимацию в тяжелом состояние. На его лечение собирали деньги в соцсетях, но
спасти преподавателя на удалось.

\ifcmt
pic https://strana.ua/img/forall/u/10/91/%D0%A1%D0%BD%D0%B8%D0%BC%D0%BE%D0%BA_%D1%8D%D0%BA%D1%80%D0%B0%D0%BD%D0%B0_2020-11-16_%D0%B2_22.05_.16_.png
\fi


Сообщение о смерти преподавателя появилось вечером в понедельник, 16
ноября. 

Пока открыто уголовное дело с предварительной квалификацией по ч.1 ст. 121
(умышленное тяжкое телесное повреждение). Санкция предусматривает от 5 до 8 лет
тюрьмы. 

О задержании подозреваемых не сообщается. 

\subsubsection{Биография и творческий путь Леонида Садовского  }

Леонид Викторович Садовский - уроженец Енакиево Донецкой области. Он
окончил Донецкое училище культуры в 1969 году, после чего перебрался в
Харьков, где и пустил корни.

В первой столице получил образование в институт культуры и институте
искусств им. Котляревского, где по прошествии лет стал преподавателем.
Садовский был одним из самых известных профессоров вуза, ему он отдал
более 30 лет жизни. 

Садовский носит звание заслуженного деятеля искусств Украины. Его
последние должности - заведующий кафедрой мастерства актера драматического
театра и кино, художественный руководитель театра \enquote{Мастерская 55}. 

На годы службы Леонид Садовский ставил самые разные постановки, в том
числе по мотивам произведений Тараса Шевченко, Квитки-Основьяненко,
Довженко, Винниченко, Чехова, Достоевского, Мольера, Беккета, Олби. 

Одна из последний громких работ Садовского - \enquote{Иисус Христос - суперзвезда}
- рок-опера Эндрю Ллойда Уэббера и Тима Райса, написанная в 1970 году и
представленная на Бродвее. 

Страница Леонида Садовского была посвящена преподаванию и театру. Много
фото со студентами - с занятий, репетиций; анонсы спектаклей, поздравления
с днем театра, видео со своего творческого вечера. 

Свою политическую позицию преподаватель на показ особо не выставлял. Но,
судя по отдельным постам, профессор Садовский сочувствовал погибшим в
одесском Доме профсоюзов 2 мая 2014 года. А также критиковал попытки
ревизии достижений Красной Армии.

\ifcmt
tab_begin cols=2
	pic https://strana.ua/img/forall/u/10/91/2020-11-16_23.19_.12_.jpg
	pic https://strana.ua/img/forall/u/10/91/2020-11-16_23.19_.18_.jpg
tab_end
\fi


Сейчас его Facebook заполнен воспоминаниями и соболезнования студентов. 

\ifcmt
tab_begin cols=3
pic https://strana.ua/img/forall/u/10/91/%D0%A1%D0%BD%D0%B8%D0%BC%D0%BE%D0%BA_%D1%8D%D0%BA%D1%80%D0%B0%D0%BD%D0%B0_2020-11-16_%D0%B2_22.03_.42_.png
pic https://strana.ua/img/forall/u/10/91/%D0%A1%D0%BD%D0%B8%D0%BC%D0%BE%D0%BA_%D1%8D%D0%BA%D1%80%D0%B0%D0%BD%D0%B0_2020-11-16_%D0%B2_22.04_.18_.png
pic https://strana.ua/img/forall/u/10/91/%D0%A1%D0%BD%D0%B8%D0%BC%D0%BE%D0%BA_%D1%8D%D0%BA%D1%80%D0%B0%D0%BD%D0%B0_2020-11-16_%D0%B2_22.04_.01_.png
tab_end
\fi

Соболезнования семье профессора выразил Харьковский национальный
университет искусств им. Котляревского. 

\ifcmt
pic https://strana.ua/img/forall/u/10/91/%D0%A1%D0%BD%D0%B8%D0%BC%D0%BE%D0%BA_%D1%8D%D0%BA%D1%80%D0%B0%D0%BD%D0%B0_2020-11-16_%D0%B2_22.18_.19_.png
\fi


