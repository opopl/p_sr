% vim: keymap=russian-jcukenwin
%%beginhead 
 
%%file slova.vrag
%%parent slova
 
%%url 
 
%%author 
%%author_id 
%%author_url 
 
%%tags 
%%title 
 
%%endhead 
\chapter{Враг}
\label{sec:slova.vrag}

%%%cit
%%%cit_head
%%%cit_pic
%%%cit_text
Компашка Зеленского оставила все ту же практику Петра разделения и унижения
своих сограждан. Унижения и разделения. Каждый день. В каждом публичном
выступлении. Свои и чужие.  И это удобно. \emph{Враг} – это не оппонент. Его надо
гасить безжалостно и без аргументов. А все что не получается – это потому что
враг повсюду. В СМИ. В Раде. В мелком и среднем бизнесе. На уличном пикете.
\emph{Враг} уничтожает нашу любимую Украину и нашу единственно правильную власть.
\emph{Врага} надо бить. И правильно вычленять, чтобы затем расчеловечивать
%%%cit_comment
%%%cit_title
\citTitle{Бог Данилова - это бог расчеловечивания}, 
Игорь Лесев, strana.ua, 20.06.2021
%%%endcit

%%%cit
%%%cit_head
%%%cit_pic
%%%cit_text
Так вот, Украина считает \emph{врагом} своей безопасности Филиппа Киркорова.
КИРКОРОВ.  По мнению минкульта и правоохранителей угрожает национальной
безопасности. Так, повторим, национальной безопасности он.  Угроза.
Национальной безопасности. Шо.  Шутка «яка країна, такі теракти» теперь куда
смешнее. Помните, что о человеке можно судить по его \emph{врагам}?  \emph{Враг} нашей страны
– Киркоров.  Теперь все, кто напевает его песни, славит агрессора.  Хуже только
имя божье выговаривать впустую.  Удивительная реальность
%%%cit_comment
%%%cit_title
\citTitle{Филипп Киркоров назван угрозой национальной безопасности Украины}, Дмитрий Раимов, strana.ua, 24.06.2021
%%%endcit

%%%cit
%%%cit_head
%%%cit_pic
%%%cit_text
Принцип общего блага простой – это наличие самой Украины. Это благо. Если
кто-то торгует родиной с \emph{врагом}, то это не благо. И он должен сидеть в тюрьме
или быть расстрелянным по законам военного времени. Не существует никаких
компромиссов. Не существует. Понимаете? И в вопросе \emph{врага} тоже не существует
компромиссов. Если это \emph{враг}, то вы должны пойти на любые издержки вплоть до
бомбардировок, чтобы \emph{врага} наказать и отстоять свою землю. Вот не существует.
Как только вы втягиваетесь в обсуждение цены, то есть «а давайте не
бомбардировки, давайте будем вести годы переговоров – ну давайте». Тогда
начинаются торги. И в торгах вы теряете всю страну. Нельзя этого делать,
понимаете? Или принцип есть и тогда есть Украина или принципа нет, тогда
Украины нет
%%%cit_comment
%%%cit_title
\citTitle{Сергей Дацюк: Украина сегодня - не просто попрошайка, она на мусорнике истории}, 
Сергей Дацюк; Людмила Немыря, hvylya.net, 28.06.2021
%%%endcit

%%%cit
%%%cit_head
%%%cit_pic
%%%cit_text
Водночас, все це не відміняє того факту, що РПЦ – \emph{ворожа організація}, яка
працює в інтересах країни-агресора. Купа прикладів про поширення \emph{ворожих}
наративів, \emph{ворожої} пропаганди, я вже мовчу про періодичні схрони зі зброєю в
священнослужителів. Але з цим не мирянки в хустках мають боротися, а
правоохоронні органи та СБУ. Але щось тишина. Можливо, тому, що добра половина
політичного естеблішменту хрестила своїх дітей в РПЦ, а інша була на тих
хрестинах хресними. А український політик і державний діяч, як відомо, Бога не
боїться, але дуже переживає що батюшка скаже
%%%cit_comment
%%%cit_title
\citTitle{Не прочанки в хустках мають боротися з РПЦ, а правоохоронні органи}, 
Андрій Андрушків, gazeta.ua, 28.07.2021
%%%endcit

%%%cit
%%%cit_head
%%%cit_pic
%%%cit_text
Замість розбудови привабливого та самодостатнього українського світу, ми
часто-густо намагались побудувати маленьку кальку "русского мира" під
жовто-блакитним прапором. Так само корумповану, заздрісну та злу, яку тримає до
купи тільки наявність екзистенційного \emph{ворога}. Тільки якщо в уяві більшості
росіян таким ворогом є Америка та колективний Захід, то для українців – Росія.
І тут раптом виявилось, що 40\% українців (і навіть 44\% тих, хто народились і
вирісли у незалежній Україні) не хочуть жити в парадигмі анти-Росії
%%%cit_comment
%%%cit_title
\citTitle{Фіаско етнічної українізації}, 
Генадій Друзенко, analytics.hvylya.net, 28.07.2021
%%%endcit

