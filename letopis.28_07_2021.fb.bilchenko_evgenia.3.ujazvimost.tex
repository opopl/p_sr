% vim: keymap=russian-jcukenwin
%%beginhead 
 
%%file 28_07_2021.fb.bilchenko_evgenia.3.ujazvimost
%%parent 28_07_2021
 
%%url https://www.facebook.com/yevzhik/posts/4103097716391974
 
%%author Бильченко, Евгения
%%author_id bilchenko_evgenia
%%author_url 
 
%%tags bilchenko_evgenia,poezia
%%title БЖ. Уязвимость
 
%%endhead 
 
\subsection{БЖ. Уязвимость}
\label{sec:28_07_2021.fb.bilchenko_evgenia.3.ujazvimost}
 
\Purl{https://www.facebook.com/yevzhik/posts/4103097716391974}
\ifcmt
 author_begin
   author_id bilchenko_evgenia
 author_end
\fi

\begin{multicols}{2}
\obeycr
БЖ. Уязвимость.
\smallskip
И мы носим её у куклы под рыжими волосами,
В стареньком паровозике возим её в больницу:
Сказку, которую мы создаём сами,
Чтобы было кому присниться.
\smallskip
Человек же растёт из Бога и бурьяна,
И приносит его аист, когда стемнеет.
Больше всего на свете сказка боится одна
Остаться на белом свете, и мы за нею
\smallskip
Идём, чтобы она не боялась, идём безропотно,
И вместо неё за неё боимся.
Сказка - такая худенькая и крохотная,
Что куда там женскому коготку на слимсе.
\smallskip
Сказка слагается ежедневно и еженощно,
Одолевая смерть, безверие и опалу.
Посмотри на меня: я увижу тебя наощупь.
Не смотри на меня: я вижу тебя на память.
\smallskip
А кукла теряет руки и половины -
От испугов её не хватит у мира саек.
Но сказка не знает о том, что неосуществима.
И это - единственное, что её спасает.
\smallskip
28 июля 2021 г.
\smallskip
Фото: Аркадий Веселов.
\restorecr
\end{multicols}

\ifcmt
  pic https://scontent-lga3-2.xx.fbcdn.net/v/t1.6435-9/225316358_4103097669725312_8549475476643050315_n.jpg?_nc_cat=104&ccb=1-3&_nc_sid=8bfeb9&_nc_ohc=sLz88Pq5nlcAX9w4-24&_nc_ht=scontent-lga3-2.xx&oh=06f412087d3c77a1884a8960e3963408&oe=6128F21E
  width 0.4
\fi

\ii{28_07_2021.fb.bilchenko_evgenia.3.ujazvimost.cmt}


