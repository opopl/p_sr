% vim: keymap=russian-jcukenwin
%%beginhead 
 
%%file 13_05_2022.fb.fedosenko_pavel.harkov.newsroom.1.vopros_harkov
%%parent 13_05_2022
 
%%url https://www.facebook.com/pavelveter/posts/5190163747741658
 
%%author_id fedosenko_pavel.harkov.newsroom
%%date 
 
%%tags 
%%title Каждый день мне задают вопрос, можно ли возвращаться в Харьков
 
%%endhead 
 
\subsection{Каждый день мне задают вопрос, можно ли возвращаться в Харьков}
\label{sec:13_05_2022.fb.fedosenko_pavel.harkov.newsroom.1.vopros_harkov}
 
\Purl{https://www.facebook.com/pavelveter/posts/5190163747741658}
\ifcmt
 author_begin
   author_id fedosenko_pavel.harkov.newsroom
 author_end
\fi

Каждый день мне задают вопрос, можно ли возвращаться в Харьков.

И каждый день я говорю, что у меня нет ответа. 

Все зависит от вашего уровня тревожности.

Я видел пенсионера на улице Бучмы, которому как-то привез гуманитарку. Под
звуки взрывов он мне тогда сказал: \enquote{Та это на Северной Салтовке громко, а у нас
еще нормально}.

\ii{13_05_2022.fb.fedosenko_pavel.harkov.newsroom.1.vopros_harkov.pic.1}

Я видел людей, которые два месяца сидят в метро. Кому-то так удобно, кому-то
просто некуда идти, а кому-то - реально страшно. Как-то на \enquote{Холодной Горе}
психолог рассказал мне про местную девочку, которая, услышав однажды звуки
взрывов, отказалась после выходить на поверхность. 

Да, наши котики отодвинули орков, и уже несколько дней ствольная артиллерия не
работает по городу. Но это не гарантирует того, что немытые не ударят чем-то
посерьезнее. Я не пугаю, но, принимая решение, вы должны реально смотреть на
вещи. 

Находящиеся в городе люди могут только лишь предоставить вам факты. Да, в
Харькове стало намного тише. Нет, война не окончена.

Готовы ли вы рискнуть? А семьей? Детьми?

Никто за вас не примет это решение. 

Никто не даст правильный ответ. 

Потому что его нет.

%\ii{13_05_2022.fb.fedosenko_pavel.harkov.newsroom.1.vopros_harkov.eng}
\ii{13_05_2022.fb.fedosenko_pavel.harkov.newsroom.1.vopros_harkov.cmt}
