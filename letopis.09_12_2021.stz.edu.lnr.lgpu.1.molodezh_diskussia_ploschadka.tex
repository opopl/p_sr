% vim: keymap=russian-jcukenwin
%%beginhead 
 
%%file 09_12_2021.stz.edu.lnr.lgpu.1.molodezh_diskussia_ploschadka
%%parent 09_12_2021
 
%%url http://lgpu.org/events/molodezhnaya-diskussionnaya-ploschadka-na-temu-destruktivnyy-kult.html
 
%%author_id 
%%date 
 
%%tags Молодёжная дискуссионная площадка на тему «Деструктивный культ. Как не стать жертвой?»
%%title 
 
%%endhead 
\subsection{Молодёжная дискуссионная площадка на тему «Деструктивный культ. Как не стать жертвой?»}
\label{sec:09_12_2021.stz.edu.lnr.lgpu.1.molodezh_diskussia_ploschadka}

\Purl{http://lgpu.org/events/molodezhnaya-diskussionnaya-ploschadka-na-temu-destruktivnyy-kult.html}

\textSelect{Молодёжная дискуссионная площадка на тему «Деструктивный культ. Как не стать
жертвой?» в рамках заседания сектора «Духовно-психологические аспекты здоровья
человека»}

Дата проведения: 9 декабря, четверг;

Место проведения: г. Луганск, ул. Славянская 1, Республиканский центр развития
образования, Кабинет русского языка, истории и культуры (2 этаж, кабинет 23);

Начало: 12:40;

Контактное лицо: Богиня Юлия 072-100-95-68;

\textSelect{Содержание:}

9 декабря 2021 года Духовно-просветительский центр имени святого преподобного
Нестора Летописца проводит молодёжную дискуссионную площадку на тему
«Деструктивный культ. Как не стать жертвой?».

Целью мероприятия является забота о духовно-нравственной безопасности человека.

Задачи мероприятия:

\begin{itemize}
  \item попытка рассмотреть механизм влияния СМИ на духовно-нравственную культуру человека;
  \item понять важность правильного отношения к любой информации, умение её отфильтровать;
  \item учиться избеганию искажения традиционных ценностей, идеалов, ориентиров.
\end{itemize}

На мероприятии планируется проведение беседы о роли средств массовой информации
в жизни человека, о влиянии информационных технологий на сознание, о
положительных и отрицательных сторонах подобного воздействия. Планируется также
активная дискуссия со студентами с обсуждением видеороликов и приведением
конкретных примеров.

Организатор конференции – Духовно-просветительский центр имени святого
преподобного Нестора Летописца.

Планируемое количество участников: 10-20 человек.

Спикер:

БОГИНЯ Юлия Александровна – ведущий специалист Духовно-просветительского центра
имени святого преподобного Нестора Летописца.
