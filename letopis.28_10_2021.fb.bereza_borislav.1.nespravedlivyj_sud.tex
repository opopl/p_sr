% vim: keymap=russian-jcukenwin
%%beginhead 
 
%%file 28_10_2021.fb.bereza_borislav.1.nespravedlivyj_sud
%%parent 28_10_2021
 
%%url https://www.facebook.com/borislav.bereza/posts/7066822903343755
 
%%author_id bereza_borislav
%%date 
 
%%tags obschestvo,spravedlivost,strana,sud,sudebnaja_sistema,ukraina
%%title Несправедливый суд - это основа всех наших проблем
 
%%endhead 
 
\subsection{Несправедливый суд - это основа всех наших проблем}
\label{sec:28_10_2021.fb.bereza_borislav.1.nespravedlivyj_sud}
 
\Purl{https://www.facebook.com/borislav.bereza/posts/7066822903343755}
\ifcmt
 author_begin
   author_id bereza_borislav
 author_end
\fi

Несправедливый суд - это основа всех наших проблем. Украинцы уверены, что
справедливости и верховенства права в судах можно не искать. Это знание мы
обрели на практике сталкиваясь с примерами того, когда судьи принимают решения
в интересах того, кто заплатил или позвонил им. И таких примеров огромное
множество. Таким образом украинские судьи обеспечивают себе безбедную жизнь, но
убивают веру граждан в честный и справедливый суд.

А с судьями у нас беда. Но кто поверит, что наши судьи, понимая незаконность и
не справедливость своих решений, выносят их просто так, а не будучи
мотивированным или без давления из ОП? Поэтому нужно не только с судами
разбираться, но с той нечистью, что деньги дает. Я, как политик и гражданин,
понимаю, что суд и полиция это основа государства. Основа соблюдения закона. И
если законы не соблюдаются, а справедливый суд отсутствует, то это ведет к
ослаблению власти и краху государства. Это происходит не сразу, но финал легко
спрогнозировать. Если ситуация с судами не поменяется, то мы увидим не только
крах системы, но можем потерять государственность, чего я точноне хочу. Я
понимаю насколько суды важны для государства. Понимаю и то, что важны именно
независимые суды, которым не могут дать команду из Офиса президента. А для
этого необходимо, чтоб судьи были не на крючке у правоохранительных органов или
у олигархов. Первые, получают нужный власти результат через шантаж и угрозы, а
вторые держат судей на финансовом крючке, банально покупая решения. И этот
порочный круг необходимо разорвать. Иначе беззаконие приведет к бесправию и
повторению сценария столетней давности. Судам должны доверять, а это возможно
лишь при вынесении ими решений основанных на верховенстве права, а не на
телефонном праве, личных симпатиях или банках с долларами .

В иудаизме считается, что со времен Ноя человечеству даны семь основных
законов, которые люди должны выполнять.  

\begin{itemize}
  \item 1.Запрет идолопоклонства — вера в единого Бога.
  \item 2.Запрет богохульства — почитание Бога.
  \item 3.Запрет убийства — уважение к человеческой жизни.
  \item 4.Запрет прелюбодеяния — уважение к семье.
  \item 5.Запрет воровства — уважение к имуществу ближнего.
  \item 6.Запрет употребления в пищу плоти, отрезанной от живого животного — уважение к живым существам.
  \item 7.Обязанность создать справедливую судебную систему.
\end{itemize}

Седьмым пунктом идет справедливый суд. Понимаете? И можно рассказывать, что
угодно, но в Израиле министра превысившего скорость - штрафуют и лишают прав, а
при малейшем подозрении в коррупции открывают дело по премьеру и президенту. И
если удаётся доказать их вину, то они садятся в тюрьму. И таких примеров в
Израиле множество. А у нас депутата Трухина, близкого к Зеленскому, который
совершил ДТП в состоянии алкогольного опьянения пытается спасти вся система. У
нас судьи оправдывают судью Шестого апелляционного административного суда
города Киева Константина Бабенко, которого поймали пьяным за рулём. О каком
доверии к судам и о каком верховенстве права может идти речь, если Третья
дисциплинарная палата Высшего совета правосудия наказала судью Печерского суда
Светлану Смык выговором и лишением премии на месяц за то, что она
систематически затягивала рассмотрение дел относительно вождения в пьяном виде
и закрывала производства из-за истечения сроков. Выговор и лишение премии? Да
это же круговая порука. Потому мы живем так, а Израиль иначе. Если нет честного
и справедливого суда, то это катастрофа доя государства. Именно доверие граждан
к судам лежит в основе государства. Поэтому судьи в Израиле уважаемы и их
решениям подчиняются, а у нас они презираемы и их решениям не доверяют. А
иногда и игнорируют. Суд в Украине давно уже утратил доверие у граждан и ничего
не делает для его восстановления. 

И если государство не готово изменить эту позорную ситуацию, то граждане
изменят само государство. А оно, в современных условиях и в данной ситуации,
может это просто не пережить. И то, что просходит сейчас - это не тревожные
колокольчики, а набат к которому надо прислушаться. А времени для исправления
все меньше и меньше. Потому что если не работают суды, то граждане реализуют
свой запрос на справедливость в той форме, которую считают наиболее приемлемой
для себя. Подозреваю, что эта форма может быть близка к суду Линча. Это будет
не только маркером краха государства, но и той ситуацией, когда пьяный нардеп
или судья не смогут откупиться или решить все по звонку. Не уверен, что такого
развития ситуации хотят все те, кто сегодня толкают Украину к такому пути -
Зеленский, Ермак, Арахамия, Стефанчук и остальная прислуга. Но нарушая законы и
избегая наказания из-за своего положения они делают максимум для реализации
именно этого сценария.

