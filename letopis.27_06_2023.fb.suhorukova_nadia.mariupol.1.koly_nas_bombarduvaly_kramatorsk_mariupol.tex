%%beginhead 
 
%%file 27_06_2023.fb.suhorukova_nadia.mariupol.1.koly_nas_bombarduvaly_kramatorsk_mariupol
%%parent 27_06_2023
 
%%url https://www.facebook.com/100087641497337/posts/pfbid02xTJkAxYRGimrScQNhqWXwnzuoNkNAfxRzSbUvf1saNfBbggYYkzgCGBXEFMfHmi3l
 
%%author_id suhorukova_nadia.mariupol
%%date 27_06_2023
 
%%tags 
%%title Краматорськ - Маріуполь
 
%%endhead 

\subsection{Краматорськ - Маріуполь}
\label{sec:27_06_2023.fb.suhorukova_nadia.mariupol.1.koly_nas_bombarduvaly_kramatorsk_mariupol}

\Purl{https://www.facebook.com/100087641497337/posts/pfbid02xTJkAxYRGimrScQNhqWXwnzuoNkNAfxRzSbUvf1saNfBbggYYkzgCGBXEFMfHmi3l}
\ifcmt
 author_begin
   author_id suhorukova_nadia.mariupol
 author_end
\fi

Коли нас бомбардували, я думала: \enquote{Чому ви, суки, не спите? Ви ж і вдень, і
вночі, тварюки, спокою не даєте. Є якась межа цього кошмару? Ви ж не можете не
спати цілодобово!}

Напевно, так і було задумано: не давати нам спати. Гамселити по місту
снарядами, мінами, закидати авіабомбами. 

Отака воєнна тактика, саме так роzіяни вбивають мирних жителів.  

Живі  ридають, обличчя змарніли від страху. Вони - невипадкові жертви, їм на
голови скидають багатотонну ненависть. 

Моторошна  маріупольська історія: маленька блакитноока дівчинка загинула від
обстрілу. Мама не змогла поховати дитину, бо сама була поранена й непритомна. 

Лікарі боролися за  життя її мати, а тіло малечі пролежало в машині понад
тиждень.

Я глибоко вдихаю, коли в мене запитують: \enquote{А що, не можна було похоронити, чи
що? Потрібно ж завжди залишатися людьми. У будь-якій ситуації. Що ви за люди
такі?}

Ми залишалися людьми, але похоронити не могли. Бо нелюди не давали цього
зробити.

Спробуйте викопати яму, коли стріляють із танків і навколо падають міни. 

Наші мертві були поряд із нами. Вони лежали на землі й замерзали, їх покривала
паморозь. 

\enquote{Знаєте, як страшно копати могилу близькій людині? Та ще й кілька днів поспіль.
Навколо рвуться міни, і якщо тебе вб'є, то хто її поховає?}

Як таке взагалі можливо? Копати могилу найдорожчій людині? Чому смерть
перестала бути страшною, чому перетворилася на буденність?

Чоловікові 76 років, його дружині  - 73. Вони все життя разом. 

Вона не встигла добігти до підвалу. Уламок влучив у голову. 

Жінка була жива ще майже десять годин, її не могли доправити до лікарні:
бомбардували так, що їхня вівчарка від страху заривалася під поріг. 

Коли дружина померла, чоловік дочекався затишшя, а потім пішов копати яму на
городі. Йому допомагав друг. 

Вони копали щоранку шість днів, а мертва дружина була поряд. Він із нею
розмовляв. Перепрошував. Обіцяв, що обов'язково похоронить.

Яке полегшення було, коли це нарешті сталося! Друзі зробили дерев'я\hyp ний хрест і
табличку. Хрест через день рознесло осколками. 

Але ж головне -  похоронити. Виконати свій обов'язок. Залишитися людиною у
складній ситуації.

Мабуть, то був маленький подвиг. Чоловік цього не усвідомлював, у нього була
апатія. Досі він мав мету -  віддати кохану землі. Він досягнув мети. А що
далі? Як тепер жити? Треба знайти ще якусь мету?

Синдром того, хто вижив, -  це коли твої мертві дивляться на тебе з докором. І
ти думаєш: \enquote{Краще б я, ніж вони}

На фото Краматорськ сьогодні.  Вбивці з країни - терориста наближають кару для
себе. 

Вона буде страшніша за мільйони прокльонів і довша за нескінченність.

Аби світ не мовчав. Не звик до того, що нас убивають.

%\ii{27_06_2023.fb.suhorukova_nadia.mariupol.1.koly_nas_bombarduvaly_kramatorsk_mariupol.cmt}
