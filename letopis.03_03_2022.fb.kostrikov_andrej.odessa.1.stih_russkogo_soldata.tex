% vim: keymap=russian-jcukenwin
%%beginhead 
 
%%file 03_03_2022.fb.kostrikov_andrej.odessa.1.stih_russkogo_soldata
%%parent 03_03_2022
 
%%url https://www.facebook.com/uho.lot.1/posts/10221881550056182
 
%%author_id kostrikov_andrej.odessa
%%date 
 
%%tags 
%%title Стихотворение русского солдата
 
%%endhead 
 
\subsection{Стихотворение русского солдата}
\label{sec:03_03_2022.fb.kostrikov_andrej.odessa.1.stih_russkogo_soldata}
 
\Purl{https://www.facebook.com/uho.lot.1/posts/10221881550056182}
\ifcmt
 author_begin
   author_id kostrikov_andrej.odessa
 author_end
\fi

\headCenter{Стихотворение русского солдата}

\ii{03_03_2022.fb.kostrikov_andrej.odessa.1.stih_russkogo_soldata.pic.1}

\raggedcolumns
\begin{multicols}{2} % {
\setlength{\parindent}{0pt}

\obeycr
Стихотворение русского солдата
Мам, я в плену, но ты не плачь.
Заштопали, теперь как новый.
Меня лечил херсонский врач
Уставший, строгий и суровый.
\smallskip
Лечил меня. Ты слышишь, мам:
Я бил по городу из «Градов»,
И полбольницы просто в хлам,
Но он меня лечил: «Так надо».
\smallskip
Мам, я - чудовище, прости.
В потоках лжи мы заблудились.
Всю жизнь мне этот крест нести.
Теперь мои глаза открылись.
\smallskip
Нас провезли по тем местам,
Куда снаряды угодили.
А мы не верили глазам:
Что мы с Херсоном натворили!
\smallskip
В больницах раненых полно.
Здесь каждый русских проклинает.
Отец, белей, чем полотно,
Ребенка мертвого качает.
\smallskip
Мать, я - чудовище, палач.
И нет здесь, мама, террористов.
Здесь только стон людской и плач,
А мы для них страшней фашистов.
\smallskip
Нас, мам, послали на убой,
Не жалко было нас комбату.
Тут мне херсонец крикнул: «Стой!
Ложись, сопляк!» - и дальше матом.
\smallskip
Он не хотел в меня стрелять.
Он - Человек, а я - убийца.
Из боя вынес! Слышишь, мать,
Меня, убийцу, кровопийцу!
Мам, я в плену, но ты не плачь.
\smallskip
Заштопали, теперь как новый.
Меня лечил херсонский врач
Уставший, строгий и суровый.
\smallskip
Он выполнял врачебный долг,
А я же, от стыда сгорая,
Впервые сам подумать смог:
Кому нужна война такая?
\restorecr
\end{multicols} % }
