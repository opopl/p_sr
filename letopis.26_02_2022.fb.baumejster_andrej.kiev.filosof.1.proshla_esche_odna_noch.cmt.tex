% vim: keymap=russian-jcukenwin
%%beginhead 
 
%%file 26_02_2022.fb.baumejster_andrej.kiev.filosof.1.proshla_esche_odna_noch.cmt
%%parent 26_02_2022.fb.baumejster_andrej.kiev.filosof.1.proshla_esche_odna_noch
 
%%url 
 
%%author_id 
%%date 
 
%%tags 
%%title 
 
%%endhead 
\zzSecCmt

\begin{itemize} % {
\iusr{Sergey Titov}
Андрей, дорогой, береги себя! Ангела Хранителя

\iusr{Андрей Баумейстер}
\textbf{Sergey Titov} Серёжа, спасибо!

\iusr{Виктор Каган}
Держу кулаки, Андрей!

\iusr{Лора Чернышова}

В бомбоубежище мы разговариваем на двух языках. Никого не волнует мой русский
или русский ребят из Донецка, слава богу. Подвал - лучшее уравнивающее и
мультикультурализующее средство  @igg{fbicon.smile} 

\iusr{Лев Малхазов}

Русский язык - это язык Пушкина, а не Путина. Но я прекрасно понимаю чувства
людей, отказывающихся на нём говорить. Тем не менее, если Украина его сохранит,
а не отдаст, это будет ещё одна победа. Победа великодушия.

\begin{itemize} % {
\iusr{Наталія Єпіфанова}
\textbf{Lev Malkhazov} зберігати щось вороже це не великодушно, це нечистоплотно

\iusr{О. Олег Кобель}
\textbf{Lev Malkhazov} він збережеться, язик. В бібліотеках.
\end{itemize} % }

\iusr{Яна Прозорова}

Благодарю за ваши слова! Важные слова поддержки! Надеюсь, что так и будет. Верю
в Большую Украину. @igg{fbicon.heart.red}

\iusr{Елена Колтунович}
Мужественно! Благодарю!
Поддерживаю!
Ращу людей, способных построить большую Украину!

\iusr{Iryna Pukhta}

Я би сказала, що це Росія присвоїла собі частину української історії і на цьому
грунті збудувала собі міф \enquote{великої Росії}. А ще мені страшенно цікаво, які саме
смисли і цінності російської культури такі дорогоцінні для вас, що ви готові за
них триматися до останнього? \enquote{Тварь я дрожащая иль право имею?} - ось основний
меседж російської культури, який зараз демонструє фсбешна російська влада (не
один путін) в україні.

\iusr{Kira Savy}

Благослови вас Господь за эти слова! Мир уже не будет прежним, война в каждом
доме и в каждом сердце. Это час испытания веры, мужества,
человечности, час выбора. Мы молимся за вас неустанно.

\iusr{Татьяна Конькова}

Нет ничего более утешительного, чем то, что Вы пишете. Я стыжусь писать слова
поддержки, хотя с 2014 выходила в поддержку Украины.

Удивляюсь тому, что многие украинцы все ещё не ненавидят нас всех. Как можно
найти в сердце, полном страха и ненависти к захватчикам, место для понимания и
сострадания к нам - тем, кто ненавидит путинский режим и осуждает агрессию
против Украины? Восхищаюсь украинцами, плачу, молюсь о вашей победе и свободе.

\iusr{Лора Чернышова}
\textbf{Татьяна Конькова} спасибо

\iusr{Любовь Терехова}

Андрій Олегович, даруйте, зараз би не чекала, що час говорити про російську
мову й культуру. Стоїть питання про виживання української культури й нас як
українців. Росіяни й так, появляючись в кожному чаті, переводять все на себе й
отримають оплески як герої тільки тому, що пишуть, що вони окремо од Путіна й
проти війни і все - всі в восхіщєнії. Я два дні пишу відповідь спільній е-мейл
розсилці міжнародної олімпіади з філософії, де всі восхіщаються росіянкою, яка
написала, що вона проти війни, а україняцям вона каже \enquote{say to forgive}.
Нормально? Даже не \enquote{ask to forgive}.

І всі ой, яка молодець, ти так бідна страдаєш. Цитують Канта і Фрейда, стєнають
про долю інтелектуалів. І тільки одна людина з понад 100 у цьому чаті написала,
а як там українки? Написала в особисті, по-тихому. Бо велика і могуча руська
культура не сильно пострадає, вони дуже добре вміють робити з себе жертв, і
робиби все про себе.


\end{itemize} % }
