% vim: keymap=russian-jcukenwin
%%beginhead 
 
%%file 19_07_2021.fb.bilchenko_evgenia.1.infvojna
%%parent 19_07_2021
 
%%url https://www.facebook.com/yevzhik/posts/4074707095897703
 
%%author Бильченко, Евгения
%%author_id bilchenko_evgenia
%%author_url 
 
%%tags bilchenko_evgenia,infvojna,rusmir
%%title БЖ. Почему мы проигрываем в информвойне
 
%%endhead 
 
\subsection{БЖ. Почему мы проигрываем в информвойне}
\label{sec:19_07_2021.fb.bilchenko_evgenia.1.infvojna}
 
\Purl{https://www.facebook.com/yevzhik/posts/4074707095897703}
\ifcmt
 author_begin
   author_id bilchenko_evgenia
 author_end
\fi

БЖ. Почему мы проигрываем в информвойне.

Я могу объяснить, почему мы, увы, проигрываем в информационной войне. Когда я
говорю "мы", я условно имею в виду антиглобалистов, и я их не делю на
марксистов и консерваторов, "левых" и "правых", атеистов и православных и т.д. 

Также мое "Мы" лишено ортодоксальной корпоративной идентичности и открыто для
диалога с глобальным миром, пока последний не начинает душить наше
цивилизационное "Я", ибо душить Другого - это и есть фашизм.

\ifcmt
  pic https://scontent-cdt1-1.xx.fbcdn.net/v/t1.6435-9/218077197_4074709385897474_3076206980215928187_n.jpg?_nc_cat=109&ccb=1-5&_nc_sid=8bfeb9&_nc_ohc=tahQZoSffzsAX9okS94&_nc_ht=scontent-cdt1-1.xx&oh=c60fc38da7e19f6a203afff8b1bba280&oe=613FBEE1
  width 0.4
\fi

Мы проигрываем, и эту горькую правду следует признать, потому что мы не
работаем с инструментарием оппонента. Мы похожи на недоумевающего
Емелю-богатыря, который от стаи москитов пытается отмахнуться монолитным мечом,
а насекомые, тем временем, дробясь на токсичные модули информации, уже отожрали
ему полголовы. Мы используем ретро-риторику, безнадежно архаичную и тупо не
нужную молодому поколению, потому что общую память надо нести живо и жёстко,
используя современный язык, а не полутрупом из комода. Это надо делать, если
хотите, радикально и, не побоюсь этого слова, модно, чтобы конкурировать.

Яркий пример: украинские пляски вокруг нашумевшей статьи ВВП. Националисты не в
состоянии на нее ответить, потому что их мысли строятся на двух мемах и трёх
оппозициях. Они просто повторяются и от своих же мантр впадают в истерию.

Поэтому с националистами спорить в принципе легко, но не надо. Они не заточены
на искусство полемики. Пусть левые не обижаются, что я сравниваю, но наши наци
взяли худшее, что было в СССР по косности мышления и догматизму речи,
перекрасив только воображаемую  идентичность. А лучшее (например, диалектику
как основу мысли) забыли.

Куда интереснее играют либералы. Их игра против статьи ВВП сделана у нас
изящно. Скрепите зубы и признайте. Сама статья ВВП написана, кстати, в духе
либерально-консервативного баланса. Там есть элементы собчаковской демократии в
критике советской власти и элементы нового традиционализма в виде попытки
выстроить альтернативный глобализм, но без радикальности. Поскольку, на мой
взгляд, статья не имеет никакого отношения к Украине: здесь больше не читают
Рыбакова и Толочко, - то говорить о ее значении стоит либо для России, либо для
мира. 

Для России это может привести к политике сдерживания радикальных панславистов и
одновременно к политике ограничения непримиримых изоляционистов. Второе -
хорошо, на мой взгляд. Для мира - это жест симметричного ответа, я его значение
пока просчитать не могу. На мой взгляд, если бы под статьей не было фамилии,
туда можно было бы написать "Брайчевский", "ранний Попович" и т.д. Поздновато,
на мой взгляд. Но, видимо, так делаются долгосрочные ходы в политикуме.

Что может быть колким украинским ответом риторике русского традиционализма?

Должен быть субъект-баннер: молодой человек - не нацик, я подчёркиваю, -
желательно русскоязычный, - чтобы на нем откатывался сократовский прием:
опровержение оппонента с его же позиций. Короче, нужен модный русскоязычный
русофоб. Говорить он будет то же самое, что националисты, но более ловко и
менее провинциально. Так появляется четкая, как по нотам, реакция Светланы
Крюковой, написанная в духе транснационального американизма с подчёркнутым
якобы неприятием как этнонаци Украины, так и "российского империализма". Типа
мета-позиция. Типа центризм. Называется это, по Фуко, "перформативный дискурс".

По факту подобная "широта воззрений" является ширмой для поддержки одной из
сторон конфликта - в данном случае украинской национал-либеральной. Эти
действия давно диагностированы Аленом Бадью как "finitude" - синдром конечности
(англ.  и франц.), бег по замкнутому кругу от либералов к национал-радикалам и
обратно. На этом глобализм замыкает свою онкологическую матрицу выбора без
выбора.

Вообще, я не представляю, как русскоязычные русофобы хотят защищать русских
здесь, одновременно расписываясь в ненависти к России. Это, как жениться на
женщине, чьи ноги нравятся, а нос - нет. Так что, оттяпать жене голову?
Примерно так выглядит весьма наивный, но действенный семиотический жест
противопоставления культуры-донора и его текстов, отмеченный еще Юрием Лотманом
со времён Древней Руси в качестве признака комплексов у культуры-реципиента.
Зато под его прикрытием можно говорить любую толерантную полуправду и даже
языковой закон цветочками разрисовывать. 

Обратите внимание. Сейчас этот закон не критикует в Украине только ленивый.
Критика языкового закона стала трюизмом, маркером наличия мышления. И за это не
расправляются, как со мной. Более того: за это хвалят. А почему? Потому что
критикуют конформистски "правильно", в либеральном, а не в "пророссийским"
ключе, с соблюдением политеса. Это вызывает экзистенциально тошнотворные
ощущения. Как и все лицемерное и половинчатое, независимо от взглядов. В общем,
бьют не за язык, а за позицию. Украиноязычные русофилы страдают больше, чем
русскоязычные украинофилы. Надо дальше объяснять?

Вернёмся к информационной войне. Не надо недооценивать Крюковых. Они виртуозно
играют свою партию "нейтрального Третьего", потому что эта позиция уже
прописана в матрице глобализма. И отвечать на нее надо симметрично: используя
весь потенциал постмодерных (да, здесь не кривиться) знаний по медиа-теории.

Иначе мы смотримся ностальгично и крикливо.

\ii{19_07_2021.fb.bilchenko_evgenia.1.infvojna.cmt}
