%%beginhead 
 
%%file 07_03_2022.fb.solovjov_mikita.harkov.demsokyra.1.po_povodu_idushchikh
%%parent 07_03_2022
 
%%url https://www.facebook.com/Mikita.Solovyov/posts/pfbid0tbmVXHUH9uuv1sBednmkY4cuNYHPpyoQrdkAgJAwoip9j3eG1WhCZQDrWSwrRhuLl
 
%%author_id solovjov_mikita.harkov.demsokyra
%%date 07_03_2022
 
%%tags 
%%title По поводу идущих сейчас и любых других переговоров с Кремлем
 
%%endhead 

\subsection{По поводу идущих сейчас и любых других переговоров с Кремлем}
\label{sec:07_03_2022.fb.solovjov_mikita.harkov.demsokyra.1.po_povodu_idushchikh}

\Purl{https://www.facebook.com/Mikita.Solovyov/posts/pfbid0tbmVXHUH9uuv1sBednmkY4cuNYHPpyoQrdkAgJAwoip9j3eG1WhCZQDrWSwrRhuLl}
\ifcmt
 author_begin
   author_id solovjov_mikita.harkov.demsokyra
 author_end
\fi

По поводу идущих сейчас и любых других переговоров с Кремлем. 

Переговоры обязательно нужны. Причем очень нужно на них таки договариваться. О
гуманитарных коридорах. О прекращении огня. Об обеспечении безопасности ядерных
объектов. Не уверен в реальности, но пытаться нужно однозначно. 

Что же касается переговоров о том, что будет после войны, то я категорический
противник ВООБЩЕ любых уступок. В принципе. Нет. не потому что люблю воевать
или мне нравится сидеть под обстрелами. А потому что если Россия получит по
результатам войны хоть миллиметр, значит война является способом получить
желаемое. И в следующий раз он залижет силы и нападет снова. работает же! 

Причем ситуация на фронте совершенно не та, чтобы он мог что-то диктовать. Если
Минские 1,2 подписывались с пистолетом у виска, то сейчас такого пистолета нет.
Россия уже показала, на что она способна в военном плане. Насколько я вижу,
впечатлить ЗСУ особо не получилось. В отличие от тех ситуаций, никакой
реализуемой угрозы у России нет. И делать какие-то уступки по любым
гуманитарным и т.д. соображениям, будет просто преступлением. 

Не получится ни о чем договориться в таком случае? Да, скорее всего. Значит
вернемся к разговору через неделю, когда перемелем еще треть их армии. Не
получится опять? Значит когда будем стоять на границе РФ своими боеспособными
частями при отсутствии боеспособных частей противника. Потому что если опять
нет, то уже только на украино-китайской границе будет смысл что-то подписывать. 

Как бы не было тяжело. Мы не можем идти на мир, условия которого прямо и
существенно не ухудшат положение России. Продолжение этой войны тяжело для нас,
но значительно вреднее и опаснее для них. Да, я пишу это сидя в Харькове под
обстрелами. Мир не может выглядеть нашей почетной капитуляцией. Он должен
выглядеть их капитуляцией. Хрен с ним, пусть почетной. Например, пока оставить
за скобками Крым (не признавать ни в коем случае!), но как минимум полный
переход под наш контроль всей континентальной части Украины в международно
признанных границах. И полная компенсация всех расходов на восстановление. Я не
считаю подобные условия справедливыми, если что. Я считаю их минимальными, на
которые можно соглашаться.

Никакого языка, кроме языка силы, Россия не понимает в принципе. Даже язык
выгоды для них не слишком понятен. Значит условия мира должны быть такими,
чтобы точно не возникало идей даже о повторе. И пока ЗСУ явно наносят в разы
больше их армии урона, чем получают, пока есть такая мощная поддержка всего
мира, мы должны говорить с Россией на понятном ей языке.
