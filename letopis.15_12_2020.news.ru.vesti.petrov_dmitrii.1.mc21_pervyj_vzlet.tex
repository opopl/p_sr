% vim: keymap=russian-jcukenwin
%%beginhead 
 
%%file 15_12_2020.news.ru.vesti.petrov_dmitrii.1.mc21_pervyj_vzlet
%%parent 15_12_2020
 
%%url https://www.vesti.ru/article/2499235
 
%%author Петров, Дмитрий
%%author_id petrov_dmitrii
%%author_url 
 
%%tags mc21,aviacia,russia
%%title Первый взлет: МС-21 подняли в небо российские двигатели
 
%%endhead 
 
\subsection{Первый взлет: МС-21 подняли в небо российские двигатели}
\label{sec:15_12_2020.news.ru.vesti.petrov_dmitrii.1.mc21_pervyj_vzlet}
\Purl{https://www.vesti.ru/article/2499235}
\ifcmt
	author_begin
   author_id petrov_dmitrii
	author_end
\fi

\index[rus]{Авиация!Россия!Первый взлет, МС-21, 15.12.2020}

\ifcmt
  pic https://cdn-st1.rtr-vesti.ru/vh/pictures/xw/307/970/7.jpg
  caption Первый взлет: МС-21 подняли в небо российские двигатели, 15.12.2020
  width 0.5
  fig_env wrapfigure
\fi

\emph{Россия с триумфом возвращается в высший эшелон мировой гражданской авиации.
Новейший самолет МС-21, оснащенный отечественными двигателями, 15 декабря 2020
года совершил свой первый полет. Силовую установку впервые с 1990-х годов
собрали полностью из российских комплектующих. Как итог – низкие показатели
шума и топливная экономичность. А применение самой современной бортовой
электроники и композитных материалов позволяют лайнеру легко конкурировать с
ведущими мировыми авиапроизводителями.}

До первого взлета всего несколько минут: в кабине МС-21 летчики-испытатели Василий Севастьянов и Андрей Воропаев.

\emph{- Готовность есть к запуску.}

И вот уже самолет выруливает на взлетку Иркутского авиазавода. Сейчас борт – в наилегчайшем весе, без пассажиров, даже кресла не установлены. Топлива – на полтора часа полета, поэтому и разбег короткий.

Вид у МС-21 не парадный. Дюралевые и композитные детали покрыты только слоем технологической грунтовки. Но это и не важно. Главное, что новые двигатели подняли самолет в небо.

На борту есть еще третий – инженер-испытатель Александр Соловьев. Он контролирует работу агрегатов. В полетном задании несколько пунктов: проверка режимов работы силовой установки, устойчивости и управляемости самолета, а также функционирования всех его систем в разных режимах. Жители Иркутска стали свидетелями первого полета МС-21 на новых двигателях, под крыльями самолета проплывают городские кварталы. Через час двадцать пять минут – посадка.

Первый доклад прямо у трапа.

\begingroup
\em – Задание выполнено.

– Поздравляю!

– Как на газовке, как на пробежке, в полете показал себя очень хорошо. Впечатления очень хорошие. И от самолета – самолет ровный, сбалансированный. Полетное задание полностью даже перевыполнено.

– Поздравляем!

\endgroup

Сегодня здесь все вместе – заводчане, руководство компании "Иркут" и
Объединенной авиастроительной корпорации, "Ростеха" и представители
Минпромторга.

\begin{leftbar}
	\begingroup
		\em "Коллеги, это большой взлет отечественного двигателестроения. Теперь
				уже это, думаю, будет всемирно признанный факт. Для всего авиастроения
				в целом отечественного это такой хороший, основательный шаг. За 30 лет
				такого не было. Поэтому от всей души мы поздравляем всех тех, кто
				трудился, – конструкторов, технологов, пилотов", – отметил замминистра
				промышленности и торговли России Олег Бочаров.
	\endgroup
\end{leftbar}

Да, российские авиастроители сделали это, несмотря на то, что западные партнеры
попытались сорвать проект – отказались поставлять комплектующие. Но президент
Путин в сентябре 2020 года дал понять, что это нам не помешает.

\begin{leftbar}
	\begingroup
		\em "Ничего не имеет общего с оборонным комплексом наш новый
				среднемагистральный самолет МС-21. Взяли, и по соображениям
				недобросовестной конкуренции, партнеры наши прекратили поставку
				соответствующих композитных материалов для крыла. Ну, что это такое? Да
				это просто хамство на мировом рынке,\Furl{https://www.vesti.ru/article/2462419} с нарушением всех общепризнанных
				принципов и правил. Знают, что мы все равно сделаем, благодаря вам", –
				обратился Путин к сотрудникам ОАК.
	\endgroup
\end{leftbar}

МС-21 – лучший самолет в классе ближне- и среднемагистральных лайнеров сразу по
нескольким показателям. Доля композитов в конструкции – 35\%. И российские
инженеры, в отличие от западных коллег, научились отливать сразу
крупногабаритные детали. А это выигрыш по прочности и технологичности
изготовления. За счет аэродинамики и, конечно же, за счет новых пермских
двигателей удалось снизить расход топлива почти на четверть, вредные выбросы и
шумность.

И у МС-21 есть то, чего нет ни у кого из конкурентов – широкие кресла и салон,
в проходе свободно могут разойтись два пассажира. Российский самолет задает
новый уровень комфорта.

\begin{leftbar}
	\begingroup
		\em "Сегодня историческое событие не только для Иркутской области. Потому
				что это первый самолет, который составит конкуренцию на международном
				уровне для всех летательных аппаратов, – подчеркивает губернатор
				Иркутской области Игорь Кобзев. – И мы еще раз подтверждаем, что
				российское авиастроение находится на подъеме".
	\endgroup
\end{leftbar}

Пермские моторостроители сегодня тоже радовались успеху.

\begin{leftbar}
	\begingroup
		\em "Очень рада, это круто. Берет гордость, что я работаю в такое время, и
				приложила свои усилия к разработке нового двигателя", – признается
				инженер-конструктор ОДК "Пермские моторы" Александра Крутикова.
	\endgroup
\end{leftbar}

\begin{leftbar}
	\begingroup
		\em "Двигатель ПД-14 – это новый продукт отечественной авиационной и
				авиадвигательной строительной промышленности. Это двигатель нового
				поколения.  На основе этого двигателя будет создано целое семейство
				двигателей большей и меньшей размерностей. И в общем-то он станет
				основой будущей российской гражданской авиации и не только на ближайшие
				десятилетия", – считает главный редактор портала Avia.Ru Роман Гусаров.
	\endgroup
\end{leftbar}

Для покупателей, конечно, оставят выбор – можно заказать самолет и с
американским двигателем Pratt\&Whitney. Но, ясно, что российский ПД-14 должен
быть дешевле и при покупке, и в обслуживании.

\begin{leftbar}
	\begingroup
		\em "Создание МС-21 помогает решить важнейшую государственную задачу – это
обеспечение независимости нашей страны. Страна у нас большая, и без авиации нам
никак! Самолет получил хорошие отклики у экспертов, и у производителей, и у
покупателей. У нас сейчас уже 175 твердых заказов на эту машину, что очень
неплохо для начала", – отмечает исполнительный директор госкорпорации "Ростех"
Олег Евтушенко
	\endgroup
\end{leftbar}

В планах – выйти на производство 70 самолетов в год. А это значит, что пермякам
надо будет строить как минимум по 150 двигателей. Впереди еще много работы.
Необходимо пройти международную сертификацию. Но есть все шансы, что двигатель
ПД-14 поможет самолету МС-21 достичь поставленной цели – завоевать 10 процентов
мирового рынка ближне- и среднемагистральных перелетов.
