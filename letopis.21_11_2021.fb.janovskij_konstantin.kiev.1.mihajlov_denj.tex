% vim: keymap=russian-jcukenwin
%%beginhead 
 
%%file 21_11_2021.fb.janovskij_konstantin.kiev.1.mihajlov_denj
%%parent 21_11_2021
 
%%url https://www.facebook.com/permalink.php?story_fbid=3024056757909644&id=100009160482256
 
%%author_id janovskij_konstantin.kiev
%%date 
 
%%tags arhistratig_mihail,cerkov,pravoslavie,vera
%%title 21 ноября - Михайлов день!
 
%%endhead 
 
\subsection{21 ноября - Михайлов день!}
\label{sec:21_11_2021.fb.janovskij_konstantin.kiev.1.mihajlov_denj}
 
\Purl{https://www.facebook.com/permalink.php?story_fbid=3024056757909644&id=100009160482256}
\ifcmt
 author_begin
   author_id janovskij_konstantin.kiev
 author_end
\fi

21 ноября - Михайлов день!

Михайлов день - народное название большого церковного праздника. В церковном
календаре он называется Собором архистратига Михаила и прочих небесных сил
бесплотных.

Согласно христианскому учению, помимо видимого тварного мира, есть еще и
невидимый. Он населен ангелами (в перевод с греческого – «посланники») -
посланниками Божиими и демонами.

\ifcmt
  ig https://scontent-lga3-1.xx.fbcdn.net/v/t39.30808-6/259815453_3024056481243005_4188481853349949413_n.jpg?_nc_cat=111&ccb=1-5&_nc_sid=730e14&_nc_ohc=31LomgquuqIAX_pwRs3&_nc_ht=scontent-lga3-1.xx&oh=ba9d7a663188f61f3c99095b9609d0e3&oe=61A1A412
  @width 0.4
  %@wrap \parpic[r]
  @wrap \InsertBoxR{0}
\fi

Главные среди ангелов – это архангелы. Их в Церкви именуют высшими «ангельскими
чинами». А самый известный из них – архангел Михаил. Согласно христианскому
вероучению, он является предводителем всех ангельских сил. То есть
«главнокомандующим». По-гречески – «архистратиг».

Верующие молятся архангелу Михаилу в различных нуждах. В церковном предании
сохранились многочисленные свидетельства его помощи простым людям. Поэтому в IV
веке в его честь был установлен популярный в народе праздник, который
отмечается 21 ноября. Поскольку архангел Михаил является предводителем всех
бесплотных небесных сил, то со временем 21 ноября стали отмечать и их память.

В традиции церкви день святого Михаила называется еще «Всеобщим днем ангела».
Согласно православному вероучению, после крещения каждому человеку Бог посылает
ангела-хранителя. Тот незримым образом наставляет верующего на путь спасения и
оберегает его от злых духов.

В Михайлов день верующие читают акафисты ангелу-хранителю и архистратигу
Михаилу. Также верующие молятся им на праздничной службе, после которой
поздравляют друг друга «с днем ангела» и «с днем архангела Михаила».

\ii{21_11_2021.fb.janovskij_konstantin.kiev.1.mihajlov_denj.cmt}
