% vim: keymap=russian-jcukenwin
%%beginhead 
 
%%file 19_01_2021.fb.sergii_ivanovich.1.slova
%%parent 19_01_2021
 
%%url https://www.facebook.com/groups/394148334076653/permalink/1864129173745221/
 
%%author 
%%author_id 
%%author_url 
 
%%tags 
%%title 
 
%%endhead 
\subsection{Словник Павловського. 1818 рік - Борщ - Щи}
\label{sec:19_01_2021.fb.sergii_ivanovich.1.slova}
\Purl{https://www.facebook.com/groups/394148334076653/permalink/1864129173745221/}
\ifcmt
  author_begin
   author_id fb_sergii_ivanovich
  author_end
\fi

Тільки на початку ХІХ століття Російська імперія нарешті по-справжньому
відкрила для себе Україну, але зіштовхнулася з великою проблемою: незнанням
мови, на якій говорили українці, - пише доктор історії Тарас Чухліб. Для того,
щоб зрозуміти місцеве населення, почали складати словники-розмовники, у яких
пояснювали:

ВАРЕНИКИ - ета ж варенные пиражки!

ВАРЕНИЦІ - да ета ж варенея ляпешки!

Ну якщо вам ВЕРЗЕТЬСЯ - значить грезитца чтойто...

Словник Павловського. 1818 рік.

\ifcmt
  pic https://scontent-mad1-1.xx.fbcdn.net/v/t1.0-9/139838878_1621106838076388_6084125710510270092_n.jpg?_nc_cat=101&ccb=2&_nc_sid=730e14&_nc_ohc=7M2NuIEUECMAX-8yYDo&_nc_ht=scontent-mad1-1.xx&oh=36bb51cfd5aeda47bf78ba87606c4d0c&oe=602D5D05
  width 0.8
\fi

