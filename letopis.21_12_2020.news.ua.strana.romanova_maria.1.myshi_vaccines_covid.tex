% vim: keymap=russian-jcukenwin
%%beginhead 
 
%%file 21_12_2020.news.ua.strana.romanova_maria.1.myshi_vaccines_covid
%%parent 21_12_2020
 
%%url https://strana.ua/news/307367-astrazeneka-sanofi-kovaks-kakie-vaktsiny-ot-koronavirusa-budut-delat-v-ukraine.html
 
%%author Романова, Мария
%%author_id romanova_maria
%%author_url 
 
%%tags covid_vaccine,ukraina
%%title Испытанные на мышах с летальными исходами. Какие вакцины от коронавируса поступят в Украину первыми
 
%%endhead 
 
\subsection{Испытанные на мышах с летальными исходами. Какие вакцины от коронавируса поступят в Украину первыми}
\label{sec:21_12_2020.news.ua.strana.romanova_maria.1.myshi_vaccines_covid}
\Purl{https://strana.ua/news/307367-astrazeneka-sanofi-kovaks-kakie-vaktsiny-ot-koronavirusa-budut-delat-v-ukraine.html}
\ifcmt
	author_begin
   author_id romanova_maria
	author_end
\fi

\ifcmt
  pic https://strana.ua/img/article/3073/astrazeneka-sanofi-kovaks-67_main.jpeg
  caption Вакцина от коронавируса. Фото с сайта pixabay.com 
  width 0.5
  fig_env wrapfigure
\fi

Вчера сенсационно прозвучали заявления Владимира Зеленского в интервью The New
York Times.\Furl{https://strana.ua/news/307719-intervju-zelenskoho-the-new-york-times-20-dekabrja-hlavnye-zajavlenija.html}

Как оказалось, переговоры о поставках в страну вакцины Pfizer приостановлены.
Нет понимания и по поставкам других западных вакцин в нашу страну. При этом от
закупок российских вакцин Украина отказывается.

Это ставит под большой вопрос все предыдущие заявления украинского руководства
о том, что вакцинация в нашей стране начнется уже в ближайшее время.

Правда, при этом в правительстве говорят, что уже есть договоренности о
поставках вакцин в рамках гуманитарной программы Covax (бесплатная помощь,
которая идет по линии международных организаций).

Однако, как выяснили журналисты Bloomberg,\Furl{https://strana.ua/news/307272-kakie-vaktsiny-ot-koronavirusa-zakazala-ukraina.html} через Covax нам доставят, вопреки
ожиданиям, не те препараты, которые уже на слуху - Pfizer и Moderna. 

А куда более "сырые" AstraZeneka, Sanofi и вообще никому не известная UBI
group. 

В отличии от Pfizer, Moderna и российской вакцины, которые, согласно заявлениям
производителей, уже прошли все три стадии испытаний, вакцины от Covax только
лишь к ней подходят. У AstraZeneka обнаружились сильные побочные эффекты (и
теперь британцы начинают сотрудничать с российскими производителями "Спутника",
чтобы повысить эффективность своей вакцины).

А прочие вообще еще до третьей стадии испытаний (на большом числе людей) даже и
не дошли.

Некоторые вообще испытаны на данный момент лишь на мышах.

Естественно, что пока эти препараты не пройдут все стадии испытаний и не будут
рекомендованы ВОЗ, поставки ни в одну страну не начнутся. И пока непонятно, о
каких сроках идет речь. 

Подробно о том, какими препаратами собираются иммунизировать украинцев от
коронавируса международные организации, разбиралась "Страна". 

\subsubsection{AstraZeneca (Великобритания)}

\textbf{Название} (рабочее): AZD1222. 

\textbf{Описание:} векторная вакцина на основе аденовируса шимпанзе. 

Во время испытаний препарата AstraZeneca были зафиксированы тяжелые побочные
эффекты. Одному добровольцу - женщине - был диагностирован поперечный миелит
(воспаление спинного мозга). Второй испытуемый - 28-летний мужчина - умер. 

\textbf{Эффективность:} средняя - 70\%. 

90\% - в случае, если вводить дозу в два раза с перерывом в 28 дней. Если всю
инъекцию за раз - эффективность оценивается в 62\%. 

\textbf{Условия хранения:} от плюс 2 до плюс 8 градусов по Цельсию. 

\textbf{Этап производства:} отправлена на доработку. Вакцина была близка к завершению
третьей фазы испытаний и подала заявку на регистрацию в Великобритании. Но
из-за недостаточно высокой эффективности компания вернулась на вторую фазу
испытаний, где начала взаимодействовать с российским центром Гамалеи -
производителем российской вакцины "Спутник V" (подробнее об этом сотрудничестве
мы расскажем в следующем блоке материала). 

\textbf{Мировые поставки:} вопрос продажи за границу не актуален.

AstraZeneca работала по заказу правительства Великобритании. До 2020 года
компания обещала произвести 100 миллионов доз, но сделала лишь 4 миллиона и
прекратила производство. 

Теперь британские власти рассчитывают, что к началу лета 2021 года этой
вакциной иммунизируют 30 миллионов человек - пожилых и с хроническими
заболеваниями. Но это, понятное дело, если смогут довести вакцину до ума. 

Цена: 6-10 долларов для вакцинации одного человека (нужно две дозы; цена одной
инъекции - 3-5 долларов). 

\subsubsection{AstraZeneca + Спутник V}

В связи с тем что вакцина AstraZeneca показала недостаточно высокую
эффективность, она, как указывалось выше, была отправлена на доработку. Но
повторные испытания будут проводить уже совместно с московским центром имени
Гамалеи - производителем вакцины "Спутник V". 

Дело в том, что структура у британской и российских препаратов схожа. Они
созданы на основе аденовируса. Но у AstraZeneca это аденовирус шимпанзе, а у
"Спутника V" - аденовирус человека. 

После провала испытаний препарата британцами россияне предложили им свою
помощь. В AstraZeneca приняли предложение, о чем сообщили на своем сайте 11
декабря. Пресс-релиз британцев на двух языках прилагается ниже. 

"Научное сотрудничество компании "АстраЗенека" с Национальным исследовательским
центром эпидемиологии и микробиологии имени Н.Ф. Гамалеи имеет большое значение
не только для изучения потенциала комбинированного применения вакцин и оценки
их синергизма в отношении иммунного ответа, но и для повышения в будущем
доступности вакцинации за счет широкого портфеля препаратов", - завершается
сообщение AstraZeneca. 

Поясним суть сотрудничества британцев и россиян простыми словами. Эффективность
вакцины AstraZeneca в среднем составила 70\%, то есть иммунитет против ковида
получают 7 из 10 привившихся. Заявленная эффективность "Спутника V" - от 90\%
(и 100\% для тяжелых случаев), то есть антитела вырабатывают у 9 из 10
привившихся.  А десятый человек, если и заболеет, но перенесет коронавирус в
бессимптомной или легкой форме. 

Для того, чтобы поднять эффективность британской вакцины до уровня российской,
первый укол испытуемым будут делать инъекцией с аденовирусом шимпанзе от
AstraZeneca, второй укол - с аденовирусом человека от "Спутника V". 

\ifcmt
pic https://strana.ua/img/forall/u/10/91/%D0%A1%D0%BD%D0%B8%D0%BC%D0%BE%D0%BA_%D1%8D%D0%BA%D1%80%D0%B0%D0%BD%D0%B0_2020-12-18_%D0%B2_16.19_.48_(1).png
\fi

На днях спикер Верховной Рады Дмитрий Разумков в качестве нового аргумента,
почему не надо покупать российскую вакцину,\Furl{https://kiev.strana.ua/307345-razumkov-somnevaetsja-v-effektivnosti-rossijskoj-vaktsiny-ot-sovid-19-.html} сказал: ею не прививаются жители
Великобритании.

На сегодня там действительно делают уколы фирмы Pfizer. Но основной
потенциальный поставщик, который работает по заказу правительства, это
AstraZeneca. Если, как предполагают ученые, сотрудничество с центром Гамалеи
будет успешным, то как раз жителей Британии и будут прививать российской
вакциной.

Также нельзя исключать, что эта британско-российская вакцина в будущем пойдет
на гуманитарную помощь разным странам мира, в том числе Украине. В таком случае
Киев получит тот же "Спутник V". Но намного позже, чем мог бы это сделать
обратившись за покупкой напрямую в Москву. 

Впрочем, это произойдет, если сотрудничество британцев и россиян по вакцине
продолжится. 

\subsubsection{Sanofi и GSK (Франция и Великобритания)}

\textbf{Описание:} вакцина разработана на основе вирусных белков, в их основе -
искусственно созданные вирусы, выращенные внутри клеток насекомых. Конструкция
препарата та же, что и у уже применяемой вакцины от гриппа Flublok.

\textbf{Эффективность:} испытания на мышах показали, что вакцина эффективна. Испытания
на людях - недостаточную эффективность. У возрастной группы старше 49 лет был
зафиксирован низкий иммунный ответ. Предположительно - из-за недостаточной
концентрации антигена. Поэтому теперь вакцину будут доделывать. 

\textbf{Условия хранения:} в холодильнике; точный температурный режим производители не
назвали, но отметили, что условия более выгоднее, чем у препарат Pfizer,
который требует минус 70 градусов по Цельсию. 

\textbf{Этап производства:} отправлена на доработку. Были завершены исследования на
второй фазе (малых группах людей). Должна была начаться третья фаза. Ее
результаты хотели обнародовать в мае 2021 года, а завершить работу над вакциной
- в середине года.

Однако, так как все пошло не по плану, препарат готовы сдать не ранее конца
2021 года. Разумеется, в случае, если вновь не случится форс-мажор. Поэтому
большой вопрос, когда ее вообще начнут поставлять в Украину по механизму Covax
(начало этих поставок в Киеве наметили уже на март).

\textbf{Мировые поставки:} Евросоюз подписал контракты на поставку вакцины Sanofi-GSK в
количестве 300 миллионов доз. Еще 200 миллионов доз компания готова передать
COVAX для бесплатного распределения по миру в качестве гуманитарной помощи,
которая пойдет в том числе и украинцам. 

\textbf{Цена:} около 10 евро за одну дозу, для иммунизации одного человека надо
две дозы.

\subsubsection{UBI/UBIA Group и Covaxx (Китай и США)}

\textbf{Название (рабочее):} UB-612.

\textbf{Описание:} препарат на синтетической пептидной платформе, направленный на активацию B-лимфоцитов и Т-лимфоцитов.

Для сравнения, похожая конструкция вакцины у второй по счету российской вакцины
- "ЭпиВакКорона" производства новосибирского центра "Вектор" (в РФ на днях ей
уже начали прививать население). 

При этом UB-612 вообще малоизвестна и в СМИ практически не обсуждается. Что и
понятно: идет лишь первая фаза испытаний, при этом препарат далеко не первый на
рынке.

\textbf{Эффективность:} нет данных. 

\textbf{Условия хранения:} нет данных. Но если отталкиваться от условий хранения пептидных вакцин, то речь может идти о температурном режиме от плюс 2 и плюс 8 градусов по Цельсию. 

\textbf{Этап производства:} первая фаза испытаний - препарат сейчас исследуют на мышах. Завершить испытания, в том числе на людях, хотят 31 августа 2021 года. 

\textbf{Мировые поставки:} предварительная дата мировых поставок - конец 2021 года. Производитель взял на себя предварительные обязательства распространить по миру более 140 миллионов доз такой вакцины в рамках механизма Covax. 

\textbf{Цена:} нет данных. 

\subsubsection{Когда начнутся поставки в Украину?}

Как неоднократно заявляли украинские власти - старт поставок в рамках Covax
намечен на март-апрель 2021 года. 

Однако из статуса готовности описанных выше вакцин видно, что под этот срок
подходит лишь британская AstroZeneka, которую сейчас пытаются смешать с
российской вакциной "Спутник V".

Есть вероятность, что "Астру", которая недавно была близка к завершению
третьего этапа испытаний, к весне как раз доработают. И смогут начать
поставлять в том числе Украине. 

Британский производитель обещал распределить бесплатно 300 миллионов доз
вакцины - это самый большой вклад в механизм Covax. Но, повторимся, испытания
препарата не окончены. Поэтому с уверенностью говорить, что его поставки
начнутся в конце первого квартала - нельзя. 

Это значит, что украинские власти выдают желаемое за действительное, обещая
завоз вакцины уже через пару месяцев. Ведь она еще до конца не готова. 

А так как остальные вакцины еще менее готовы, существует вероятность, что
украинцам весь 2021 год будут поставлять в основном британский препарат,
который пока показывает эффективность куда ниже, чем Pfizer, Moderna или
российская вакцина. И то эти поставки начнутся только в случае, если
AstroZeneka решит проблемы со своими препаратам и закончит наконец-то третий
этап испытаний. Что, повторимся, пока еще не гарантировано.

И в любом случае это будут очень небольшие поставки. По расчетам Bloomberg
Украина законтрактовала доз вакцины на 5\% населения. На уровне африканских
стран.

Положение может исправить лишь массовая закупка вакцин от разных
производителей, доступных на рынке. Вне зависимости от страны происхождения.
Главное, чтобы вакцина была эффективной.
