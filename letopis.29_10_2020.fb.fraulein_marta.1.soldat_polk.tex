% vim: keymap=russian-jcukenwin
%%beginhead 
 
%%file 29_10_2020.fb.fraulein_marta.1.soldat_polk
%%parent 29_10_2020
 
%%url https://www.facebook.com/fraulein.marta/posts/3531284950299267
%%author elena shahova
%%tags 
%%title Еще один солдат Бессмертного полка
 
%%endhead 

\subsection{Еще один солдат Бессмертного полка}

\Purl{https://www.facebook.com/fraulein.marta/posts/3531284950299267}
\Pauthor{Шахова, Е. П.}

Кто такие поисковики? Сложный вопрос… Многие нас считают ненормальными,
блаженными, чокнутыми, не от мира сего. А мы и действительно не от мира сего.
Мы те, кто смертельно ранен той далекой войной, Великой Отечественной. Именно
потому многие тысячи человек, как только сходит снег, а иногда и не дожидаясь
этого, отправляются в путь, чтобы на местах великих битв, боев местного
значения и мелких боестолкновений искать останки бойцов Красной Армии, так и не
вернувшихся домой, желая отдать им последний долг памяти. Мы - последние
солдаты той войны, подхватывающие Знамя Победы из слабеющих пальцев ветеранов.
Именно мы, словно Харон, служим проводниками душ павших воинов, именно мы
провожаем их в последний путь, к сожалению, почти всегда безвестных и
безымянных, нашедших покой в братских могилах.

И мы из года в год выходим в горы, болота, леса и поля, чтобы опять делать дело
своей совести. В глазах обывателя так и читается вопрос «Зачем?», зачем тратить
свои деньги, время, свой отпуск на неблагодарное дело поиска давно умерших
людей? Им уже без разницы!!! А мы идем и ищем, потому что душа нас зовет,
потому что не можем иначе, потому что, когда первый раз перед тобой открывается
картина гибели твоего ровесника, просто человека, благодаря которому я могу
говорить и писать на РУССКОМ языке, есть только одно желание – уйти под
ближайшую елку и скулить там раненым зверем от несправедливости и
безысходности, кусая ладонь, чтоб никто не слышал твоего тихого воя, и грызть
себя мыслью: «А ты так сможешь?». И есть такая мыслишка - не смогу, сломаюсь,
сдамся…

Но, проходят минуты личной меланхолии, и возвращаешься в раскоп, а твои
товарищи делают вид, что не замечают грязных разводов на твоем лице, молча
протягивают две фляги: с компотом и некомпотом. Делаешь по глотку и иногда даже
не чувствуешь разницы. А потом встаешь на колени в жидкую грязь и расчищаешь
останки от земли и корней, стараясь не упустить ни единой мелочи, вырывая из
безвременья последние секунды жизни неизвестного тебе человека. В этот момент
менно от тебя может зависеть, обретет ли он имя и покой рядом с родственниками,
что ждали солдата с войны, узнают ли близкие о судьбе бойца или так и сгинет он
в вихре времен безвестным, оставило ли беспощадное время хоть надежду на
установление имени этого человека.  Расскажу небольшую историю. На Синявинских
высотах в Ленинградской области мы работаем уже не первый год. Именно там много
раз пытались прорвать блокадное кольцо, именно там было самое тонкое и уязвимое
место, всего 16 километров вражеских позиций, именно там проходила единственная
дорога с твердым покрытием, пригодная для передвижения техники – Путиловский
тракт, и именно за него исступленно держались оккупанты. Попыток прорыва
Блокады в этом месте было множество, вот и стоим неподалеку от бывшей деревни
Гонтова Липка, в двухстах метрах от наших окопов.

Не люблю шум лагеря, потому всегда ставлю палатку дальше всех, и вот с самого первого раза и до Вахты Памяти прошлого года, особенно, в первые дни, когда нервы просто обнажены и неприкрыты коростой ежедневности, перед сном слышала разговор двух солдат, слов не разобрать, только интонации и голоса. Один солдатик молоденький, лет 18- 20, не больше, что-то спрашивает у своего старшего товарища. Тот ему степенно объясняет, рассказывает, советует, вот сразу чувствуется опытный боец. И так несколько лет…
В прошлом году в наших траншеях аккурат за моей палаткой находим двоих бойцов: одному около 20, второму лет 35-40. Старший, по всему видно повоевал, в ботинках вместо шнурков телефонный провод, младший просто пацан, только что на фронт попавший. Они сидели в окопе, между ними взорвалась минометная мина, так их и раскидало. Так и лежали, ждали нас. У старшего бойца медальон – обгоревший и промокший. Почти год экспертиз, шаманства и танцев с бубном, но имя установили, звали его Матвеев Василий Иванович, 1911 года рождения, старший сержант, установили и место призыва: деревня Теремец Новгородского района Ленинградской области. На сегодняшний день это фактически пригород Новгорода Великого, дачный поселок, коренных жителей нет, документов того периода не сохранилось.

Не буду описывать всю цепочку поиска родственников бойца, но нашли родного
племянника Матвеева, живет он (надеюсь счастливо) в Ижевске и участие в
посмертной судьбе своего дяди принимать отказался. Ну, да не нам судить… Вот
так и лежат теперь два бойца – молодой и взрослый – оба в братской могиле на
Синявинских высотах, смотрят, как снег укрывает лес, как сыпет непроглядной
пеленой и укутывает землю, как медленно и неохотно отступает обратно в болота
белое покрывало по весне и робкие первоцветы улыбаются теплому солнышку, а
потом буквально за несколько дней поднимается буйная трава и вновь прячет от
глаз людских старые шрамы окопов и воронок.

А там все так же лежат и ждут бойцы, ждут своего поисковика, что однажды
споткнется о кочку, стукнет щупом или поймает сигнал металлодетектором и еще
одна солдатская судьба вернется к нам. Даже если останется боец безымянным, но
мы хотя бы узнаем о его последних минутах. Мы узнаем, а я расскажу вам, и будет
на одного меньше в полку пропавших без вести.

Вместо послесловия. Только благодаря кропотливой и долгой работе сотрудников
Воткинского районного архива удалось установить имя нашего бойца и его
ближайших родственников.

Сердечное спасибо всем, кто помогал распутать эту историю, в частности,
Сурковой Маргарите Александровне, а так же учителю истории Черновской школы
Порсеву Владимиру Васильевичу за огромную проделанную работу и живую отзывчивую
душу.

