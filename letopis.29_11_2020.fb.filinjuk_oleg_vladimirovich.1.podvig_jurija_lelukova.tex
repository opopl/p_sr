% vim: keymap=russian-jcukenwin
%%beginhead 
 
%%file 29_11_2020.fb.filinjuk_oleg_vladimirovich.1.podvig_jurija_lelukova
%%parent 29_11_2020
 
%%url https://www.facebook.com/filinuk.o.v/posts/2889934217893727
 
%%author Филинюк, Олег Владимирович
%%author_id filinjuk_oleg_vladimirovich
%%author_url 
 
%%tags 
%%title Четыре секунды длиною в жизнь. Подвиг Юрия Лелюкова
 
%%endhead 
 
\subsection{Четыре секунды длиною в жизнь. Подвиг Юрия Лелюкова}
\label{sec:29_11_2020.fb.filinjuk_oleg_vladimirovich.1.podvig_jurija_lelukova}
\Purl{https://www.facebook.com/filinuk.o.v/posts/2889934217893727}
\ifcmt
	author_begin
   author_id filinjuk_oleg_vladimirovich
	author_end
\fi

\index[deaths.rus]{Лелюков!Юрий Николаевич, 29.11.1983, взрыв гранаты в школе, спас детей}

29 ноября 1983 года погиб \textbf{Юрий Николаевич Лелюков}, старший лейтенант запаса,
преподаватель начальной военной подготовки школы №2 поселка Иваничи Волынской
обл. Военрук закрыл собой приведенную во время урока в действие боевую гранату,
оказавшуюся среди полученного из военкомата учебного имущества, и этим спас
жизнь двадцати шести ученикам.

\ifcmt
pic https://scontent.fiev6-1.fna.fbcdn.net/v/t1.0-9/128066867_2889933831227099_2400109727889845868_n.jpg?_nc_cat=111&ccb=2&_nc_sid=730e14&_nc_ohc=u9Ymj5mHUv0AX9aIvZs&_nc_ht=scontent.fiev6-1.fna&oh=896671f8bc8e343883f3750491faf435&oe=5FED9A18
caption Четыре секунды длиною в жизнь. Подвиг Юрия Лелюкова
fig_env wrapfigure
width 0.4
\fi

После публикации 26 февраля 1985 года в газете \enquote{Комсомольская правда} очерка
\enquote{ЧЕТЫРЕ СЕКУНДЫ ДЛИНОЮ В ЖИЗНЬ} Юрий Лелюков был награжден орденом \enquote{Знак почета}, 
кафедра НВП и физподготовки одного из педагогических институтов
добилась присвоения его имени. После его гибели учебные гранаты стали сверлить.

...День был ясный, сухой и холодный.

Учитель начальной военной подготовки второй средней школы Юрий Лелюков
посмотрел в окно и взял со стола одну из учебных гранат.

- Внимание, - сказал он двадцати шести ребятам, сидящим в классе.- Вы давно
просили показать принцип обращения с гранатой. Надеюсь, вам эта штука в жизни
не пригодится, но тем не менее: слушайте, смотрите и запоминайте!..

Значит, так: чтобы привести гранату в боевое положение, нужно прижать спусковой
рычаг и выдернуть кольцо - я это делаю. Теперь, если отпустить рычаг, у нас, в
учебной гранате, просто щелкнет, а в боевой раздастся хлопок, и пойдет дымок.
До взрыва останется четыре секунды. Тогда бросайте гранату, не мешкая. Итак, я
отпускаю рычаг...

- А если задымит? - весело спросил кто-то.
Лелюков улыбнулся. И отпустил рычаг. Раздался хлопок, и пошел дымок.

Вряд ли двадцать шесть девчонок и мальчишек поняли в то первое мгновение, что случилось.

Вряд ли понял, что случилось, в то, самое первое, мгновение и сам Лелюков.

Боевая граната - в классе?! Глупость, нелепость, абсурд!

Но он все же был военным человеком. Старшим лейтенантом запаса... Дым пошел.
Значит, четыре секунды - и все. Значит, надо спасать детей. Что делать?!
Отбросить гранату? Подальше от себя? Куда? В классе двадцать шесть пар ничего
еще не понимающих глаз. Граната. Глаза. Снова граната. И снова глаза. И
вдруг-окно! Ну, конечно же,- окно! До него было всего лишь полшага. «Ничего,
ребята, ничего. Вы только не пугайтесь...»

Внизу, в школьном дворе, гуськом, точно утята, вышагивали на обед шестилетки.

Едва не метнулся он к двери. И, должно быть, тут же понял, что потерял еще одно
мгновение: ну, конечно же, там, в коридоре за дверью кабинета, сидели дети.
Школа была тесновата, и время от времени парты выставлялись прямо в коридор. Да
и глупо было рваться к двери - не успеть...

Он повернулся к классу спиной, шагнул в угол, неловко нагнулся и крепко,
намертво прижал гранату к животу. Пытаясь что-то сказать, только с силой
выдохнул из легких воздух...

Говорят, он был хорошим учителем. Говорят, у него не было проблем с дисциплиной
в классах. Говорят, классы эти слушали его, разинув рты. И выходил он из этих
классов всегда окруженный ребятами. Собирался построить в школе тир. Собирался
организовать гандбольную команду. Он много чего собирался сделать к той минуте,
когда, взглянув в окно, взял с преподавательского стола одну из учебных гранат
и сказал: «Слушайте, смотрите и запоминайте!..»

...Взрыв оглушил класс. В оглушительной тишине, точно снег, сыпалась с потолка
побелка. В оглушительной тишине падали стенды. И в оглушительной этой тишине
ошеломленные десятиклассники бросились к двери.  Все. Двадцать шесть человек.
Живые.

- Стой, ребята! Как же Юрий Николаевич?!

Трое из них вернулись и в дыму вынесли Лелюкова из класса. Сейчас, Юрий
Николаевич, говорили они, сейчас, еще минутку...

К ним бежали учителя. Бежали врачи. Мчались «Скорые»...

Остались без отца восьмилетняя Алеся, годовалый Юрасик, стала вдовой жена Лариса.

Это произошло 29 ноября 1983 года. Ежегодно в этот день ученики тех двух
классов приходят в Иваничевскую среднюю школу №2. Заходят в класс, где свой
последний урок мужества и человечности провел их любимый учитель и где висит
его портрет. Идут на кладбище.

В школе более двадцати лет действует музей Юрия Лелюкова. Среди многочисленных
экспонатов школьный журнал, открытый на той же странице и пробитый осколками
гранаты. А также сотни писем, которые пришли тогда со всех концов Советского
Союза. Во многих из них стихи, написанные людьми, которые никогда не видели
учителя, но преклоняются перед его подвигом.

