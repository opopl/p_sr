% vim: keymap=russian-jcukenwin
%%beginhead 
 
%%file 08_06_2023.stz.news.ua.donbas24.1.mariupol_aktrisa_virshi_nimecchyna_vystupy_olga_samojlova
%%parent 08_06_2023
 
%%url https://donbas24.news/news/mariupolska-aktrisa-z-ukrayinskimi-virsami-vistupaje-v-nimeckix-mistax
 
%%author_id demidko_olga.mariupol,news.ua.donbas24
%%date 
 
%%tags 
%%title Маріупольська актриса з українськими віршами виступає в німецьких містах
 
%%endhead 
 
\subsection{Маріупольська актриса з українськими віршами виступає в німецьких містах}
\label{sec:08_06_2023.stz.news.ua.donbas24.1.mariupol_aktrisa_virshi_nimecchyna_vystupy_olga_samojlova}
 
\Purl{https://donbas24.news/news/mariupolska-aktrisa-z-ukrayinskimi-virsami-vistupaje-v-nimeckix-mistax}
\ifcmt
 author_begin
   author_id demidko_olga.mariupol,news.ua.donbas24
 author_end
\fi

\ifcmt
  %ig https://i2.paste.pics/67835a0fceaabb8bf0c46a4d118ab0be.png
	ig https://donbas24.news/storage/news/mmlsefnzhsujqu4w.jpeg
  @wrap center
  @width 0.9
\fi

\begin{center}
  \em\color{blue}\bfseries\Large
Маріупольські митці продовжують популяризувати українську культуру за
кордоном 
\end{center}

6 червня в місті Білефельд відбувся поетично-музичний україн\hyp{}ський вечір за
участю колишньої актриси Донецького академічного обласного драматичного театру
(м. Маріуполь) та режисерки і актриси Народного театру \enquote{Театроманія}
\href{https://www.facebook.com/profile.php?id=100001752890543}{Ольги
Самойлової}. Актриса прочитала вірші відомих майстрів українського слова Тараса
Шевченка, Лесі Українки, сучасних українських авторів Сергія Жадана та Катерини
Калитко і маріупольської поетеси Оксани Стоміної.

\ii{insert.read_also.demidko.donbas24.u_kievi_vidbulos_proschannja_iz_geroichnym_mariupolcem}

Організатором українського вечора стала Німецько-українська організація в місті
Білефельд (Північний Рейн Вестфалія). Ця програма проводилась для підтримки
дітей, що постраждали внаслідок російської агресії. Зібрані кошти будуть
передані в Дніпро. 

\ii{08_06_2023.stz.news.ua.donbas24.1.mariupol_aktrisa_virshi_nimecchyna_vystupy_olga_samojlova.pic.1}

\begin{leftbar}
\emph{\enquote{Я дуже багато разів вдома перед дзеркалом репетирувала з цими аркушами.
Доходжу до якогось важкого місця і не можу стримати сліз}}, — розповіла
Ольга про підготовку до заходу.
\end{leftbar}

Разом з Ольгою Самойловою виступили скрипалька з України Наіра Арзуманян,
вірменка, яка з 1995 року жила у Дніпрі. Також виступила німецька піаністка
Клаудія Кьоль та професійний актор Білефельдського театру Томас Вульф. Ольга
Самойлова читала вірші українською мовою, а Томас читав переклади німецькою
мовою.

\textbf{Читайте також:} \href{https://donbas24.news/news/zustrinemosya-na-drami-mariupolciv-rozculilo-proniklive-video-pro-ridne-misto-video}{\emph{\enquote{Зустрінемося на драмі}: маріупольцям призначили побачення в українському звільненому місті }}%
\footnote{\enquote{Зустрінемося на драмі}: маріупольцям призначили побачення в українському звільненому місті, Еліна Прокопчук, donbas24.news, 26.05.2023, \par%
\url{https://donbas24.news/news/zustrinemosya-na-drami-mariupolciv-rozculilo-proniklive-video-pro-ridne-misto-video}%
}

\ii{08_06_2023.stz.news.ua.donbas24.1.mariupol_aktrisa_virshi_nimecchyna_vystupy_olga_samojlova.pic.2}

\begin{leftbar}
\emph{\enquote{Все пройшло на такому нерві. Для мене це було досить важко, тому що
цього дня я дізналася про страшні події, які відбулися в Каховці. І
тому мені було дуже важко читати ці вірші після всього, що я почула та
побачила. Після всіх цих подій і \enquote{Заповіт} Тараса Шевченка, і \enquote{Contra
spem spero} Лесі України набули нового сенсу. Вони сьогодні дуже
актуальні. Я декілька разів зупинялася, щоб стримувати свої емоції.
Мені здавалося, що я себе віддаю всім по шматочку. Емоційно було дуже
важко}}, — поділилася актриса. 
\end{leftbar}

\textbf{Читайте також:} \href{https://donbas24.news/news/mariupolskii-muzei-im-ai-kuyindzi-vidteper-mozna-vidvidati-ne-vixodyaci-z-domu-foto}{\emph{Маріупольський музей ім. А. І. Куїнджі відтепер можна відвідати, не виходячи з дому}}%
\footnote{Маріупольський музей ім. А. І. Куїнджі відтепер можна відвідати, не виходячи з дому, Тетяна Веремєєва, donbas24.news, 18.05.2023, \par%
\url{https://donbas24.news/news/mariupolskii-muzei-im-ai-kuyindzi-vidteper-mozna-vidvidati-ne-vixodyaci-z-domu-foto}%
}

Вечір викликав справжній фурор. До актриси підходили глядачі і дякували за
такий потужний та емоційний виступ. Після виступу Томас Вульф запропонував
Ользі створити спільний проєкт. Крім того, Наіра Арзуманян та Клаудія Кьоль
домовилися з актрисою повторити цю програму, але вже для учнів
німецько-української школи.

\begin{leftbar}
\emph{\enquote{У програму для учнів ми вирішили додати ще нові вірші, більш легкі та
зрозумілі для молоді. Але всі вірші, які були прочитані 6 червня,
залишаться, тому що вони дуже виразно передають те, що сьогодні
відчувають всі українці}}, — наголосила Ольга Самойлова.
\end{leftbar}

%Раніше \href{https://donbas24.news}{Донбас24} розповідав, що \href{https://donbas24.news/news/teatr-avtorskoyi-pjesi-conception-predstaviv-poeticnu-imprezu-liniyi-zittya}{театр авторської п'єси Conception представив
%поетичну імпрезу \enquote{Лінії життя}}.

Раніше \href{https://donbas24.news}{Донбас24} розповідав, що \href{https://archive.org/details/26_04_2023.olga_demidko.donbas24.teatr_conception_linii_zhyttja}{театр авторської п'єси Conception представив
\emph{поетичну імпрезу \enquote{Лінії життя}}}.%
\footnote{Театр авторської п'єси Conception представив поетичну імпрезу \enquote{Лінії життя}, Ольга Демідко, donbas24.news, 26.04.2023, \par%
\url{https://donbas24.news/news/teatr-avtorskoyi-pjesi-conception-predstaviv-poeticnu-imprezu-liniyi-zittya}, \par%
Internet Archive: \url{https://archive.org/details/26_04_2023.olga_demidko.donbas24.teatr_conception_linii_zhyttja}%
}


Ще більше новин та найактуальніша інформація про Донецьку та Луганську області
в нашому \href{https://t.me/donbas24}{телеграм-каналі Донбас24}.

Фото: з архіву Ольги Самойлової

\ii{insert.author.demidko_olga}
%\ii{08_06_2023.stz.news.ua.donbas24.1.mariupol_aktrisa_virshi_nimecchyna_vystupy_olga_samojlova.txt}
