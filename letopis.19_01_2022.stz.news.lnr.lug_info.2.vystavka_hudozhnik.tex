% vim: keymap=russian-jcukenwin
%%beginhead 
 
%%file 19_01_2022.stz.news.lnr.lug_info.2.vystavka_hudozhnik
%%parent 19_01_2022
 
%%url https://lug-info.com/news/ubilejnaa-vystavka-zasluzennogo-hudoznika-lnr-pavla-borisenko-otkrylas-v-luganske
 
%%author_id news.lnr.lug_info
%%date 
 
%%tags donbass,hudozhnik,isskustvo,lnr,lugansk,vystavka
%%title Юбилейная выставка заслуженного художника ЛНР Павла Борисенко открылась в Луганске
 
%%endhead 
 
\subsection{Юбилейная выставка заслуженного художника ЛНР Павла Борисенко открылась в Луганске}
\label{sec:19_01_2022.stz.news.lnr.lug_info.2.vystavka_hudozhnik}
 
\Purl{https://lug-info.com/news/ubilejnaa-vystavka-zasluzennogo-hudoznika-lnr-pavla-borisenko-otkrylas-v-luganske}
\ifcmt
 author_begin
   author_id news.lnr.lug_info
 author_end
\fi

Открытие персональной выставки работ заслуженного художника ЛНР, члена Союза
художников (СХ) Республики Павла Борисенко, приуроченной к его 65-летию,
состоялось в Галерее искусств Луганского художественного музея (ЛХМ). Об этом
сообщила пресс-служба Министерства культуры, спорта и молодежи (МКСМ) ЛНР.

\enquote{На выставке Борисенко представил около 40 работ. Вернисаж посетили
представители МКСМ ЛНР, художники, работники учреждений культуры, педагоги и
студенты Луганской государственной академии культуры и искусств имени Михаила
Матусовского, где юбиляр работает доцентом кафедры станковой живописи, а также
творческая интеллигенция Луганска}, – говорится в сообщении.

Начальник управления культуры и туризма МКСМ Сергей Рожков вручил художнику
поздравительный адрес.

\ii{19_01_2022.stz.news.lnr.lug_info.2.vystavka_hudozhnik.pic.1}

\enquote{Ваш вклад в развитие искусства нашего региона поистине бесценен, а работы
знакомы далеко за пределами Луганской Народной Республики. Имея огромный опыт
творческой деятельности, вы делитесь им с молодыми талантами – своими
студентами, тем самым сохраняя и передавая лучшие традиции отечественной школы
изобразительного искусства будущим специалистам сферы культуры нашего края}, –
сказано в поздравлении.

\ii{19_01_2022.stz.news.lnr.lug_info.2.vystavka_hudozhnik.pic.2}

Декан факультета изобразительного и декоративно-прикладного искусства ЛГАКИ
Наталья Феденко отметила, что Борисенко является \enquote{очень плодовитым художником,
учителем, который с открытым сердцем передает свои знания студентам}.

\enquote{Сегодня на юбилей, на персональную выставку пришло много студентов. Павел
Николаевич очень давно работает в академии, в основном, он преподает у
графических дизайнеров. В общем-то, он график по специальности, но в душе он
художник, потому что несет свет, и этот свет – живой и яркий. Это явление света
и цвета в творчестве дорогого стоит!} – сказала Феденко.

Председатель правления СХ ЛНР Артем Фесенко пожелал юбиляру \enquote{здоровья,
творческих успехов, неиссякаемой энергии}.

\enquote{Художники нашей организации очень ценят вас и ваш вклад в развитие
изобразительного искусства в Республике. Спасибо, что всегда небезразличны и
всегда приходите нам на помощь. Радуйте нас своими прекрасными полотнами как
можно дольше, ждем новых выставок!} – обратился он к Борисенко.

Народный художник ЛНР, известный луганский скульптор, профессор кафедры
станковой живописи ЛГАКИ Александр Редькин отметил, что \enquote{выставка получилась},
поскольку на ней представлены \enquote{живые и многозначительные картины}.

Юбиляр поблагодарил коллектив Луганского художественного музея, Союз художников
и МКСМ ЛНР за помощь в организации вернисажа.

Выставка будет экспонироваться в течение двух недель. Посетить ее можно
ежедневно, кроме понедельника и вторника, с 09:00 до 17:00. Галерея искусств
ЛХМ по располагается по адресу: Луганск, улица Тараса Шевченко, 4.
Дополнительную информацию можно получить по телефону (0642) 50 17 41 или на
сайте Луганского художественного музея.

Павел Николаевич Борисенко работает в области графики с 1978 года. Первое
художественное образование получил в Луганском художественном училище, затем
поступил в Харьковский художественно-промышленный институт, где получил
специальность художника-конструктора. С 2006 года – старший преподаватель
станковой живописи факультета изобразительного искусства ЛГАКИ, с 2021 года –
доцент кафедры.

В 2020 году за высокое профессиональное мастерство и вклад в развитие культуры
и искусства Республики указом главы ЛНР Леонида Пасечника Борисенко присвоено
почетное звание \enquote{Заслуженный художник ЛНР}.
