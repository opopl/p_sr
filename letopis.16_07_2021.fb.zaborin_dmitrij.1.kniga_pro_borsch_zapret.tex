% vim: keymap=russian-jcukenwin
%%beginhead 
 
%%file 16_07_2021.fb.zaborin_dmitrij.1.kniga_pro_borsch_zapret
%%parent 16_07_2021
 
%%url https://www.facebook.com/d.zaborin/posts/4424502620903700
 
%%author_id zaborin_dmitrij
%%date 
 
%%tags borsch,cenzura,chtenie,kniga,knigoizdanie,kuhnja,rossia,rusmir,tradicii,ukraina,zapret
%%title Я стал запрещенным писателем )) - Книга про Борщ
 
%%endhead 
 
\subsection{Я стал запрещенным писателем )) - Книга про Борщ}
\label{sec:16_07_2021.fb.zaborin_dmitrij.1.kniga_pro_borsch_zapret}
 
\Purl{https://www.facebook.com/d.zaborin/posts/4424502620903700}
\ifcmt
 author_begin
   author_id zaborin_dmitrij
 author_end
\fi

Я стал запрещенным писателем )) Вообще говоря, писать книгу про борщ не
собирался, но попросило издательство "Фолио", предложившее фантастические
деньги: тысячу гривен за 40 тыс. знаков (авторский лист). За такое-то бабло я
уж постарался выжать из темы максимум, разложив борщ на атомы, перечитав груду
источников на разных языках мира и исследовав тему от Шумера и Вавилона до
"Борщевого пояса" штата Нью-Йорк. 

\ifcmt
  tab_begin cols=2

     pic https://scontent-lga3-1.xx.fbcdn.net/v/t1.6435-9/219248049_4424495720904390_8515200328899191663_n.jpg?_nc_cat=111&ccb=1-5&_nc_sid=730e14&_nc_ohc=NXtVvPHtfrsAX8N235a&_nc_ht=scontent-lga3-1.xx&oh=90a0e673d6f32f9e600402b39577a5cb&oe=6172F2B0

     pic https://scontent-lga3-1.xx.fbcdn.net/v/t1.6435-9/218435619_4424605754226720_2999778697101683191_n.jpg?_nc_cat=111&ccb=1-5&_nc_sid=730e14&_nc_ohc=ODNvuFoC23wAX9MUjGX&_nc_ht=scontent-lga3-1.xx&oh=d35ba2b0cd19dc139a2110dfb500c1eb&oe=6171AF65

  tab_end
\fi

Выводы оказались простые, но интересные: никакого уникального украинского борща
не существует. Даже еще лучше, украинские канонические рецепты сформировались
когда-то, но такого теперь попросту не варят. В ЮНЕСКО и защитить нечего.
Вообще украинская народная кухня прошлых веков для современного человека
слишком жирна и примитивна. А многие традиции, связанные с борщом, попросту
забыты и вряд ли кому-то придет в голову использовать кастрюлю еды для
вызывания дождя или кормления покойников. 

\ifcmt
  tab_begin cols=2

     pic https://scontent-lga3-1.xx.fbcdn.net/v/t1.6435-9/218095143_4424605764226719_3805413845985603726_n.jpg?_nc_cat=107&ccb=1-5&_nc_sid=730e14&_nc_ohc=57WEbjlkb6oAX_IFyPX&tn=lCYVFeHcTIAFcAzi&_nc_ht=scontent-lga3-1.xx&oh=257a962d1880cefabcb96893bdb755e5&oe=6174A160

     pic https://scontent-lga3-1.xx.fbcdn.net/v/t1.6435-9/218405227_4424605504226745_3502075286754179745_n.jpg?_nc_cat=103&ccb=1-5&_nc_sid=730e14&_nc_ohc=C2c_rSKIQTUAX9eXJjL&tn=lCYVFeHcTIAFcAzi&_nc_ht=scontent-lga3-1.xx&oh=e857b7b432357e91b3c56b1c01e2f579&oe=6171A45B

  tab_end
\fi

Кроме того, это книга не только про борщ, но и про историю. А в этой самой
истории само понятие "украинцы" как отдельный народ и "Украина" как политоним
появляются очень недавно. 

Это потрясло хрупкую психику директора Красовицкого. Он увидел здесь путинскую
пропаганду, от которой всего лишь шаг до оправдания агрессии. Опытной рукой
директор пересеял по слову весь текст и пришел к выводу, что там слишком много
слов "русский" и "Россия". Меньше чем, "украинский" и "Украина", но это
неважно. Нас ведь нелигимно присоединили и есть запрещенные слова.

Автор должен был догадаться, что борщ - наш! Что у нас - культ! Что нужно
читать расово верных авторов, а не "Домострой" XVI века. А переводы тоже брать
расово верные, где die Reussen - никакие не русские. И доказать, что враги
украли наше сокровище - украинский борщ.

Автору надо было держаться идеологической линии, про которую, впрочем, заказчик
не сообщил ему сразу. А то б он его сразу нхй и послал. 

В общем, потрясенное моей неполиткорректностью издательство сообщило о том, что
разрывает договор. В этой связи есть два вопроса к почтеннейшей публике: 

1) Имею ли я основания вжарить их по суду, поскольку рекомендаций
комсомольского комитета по душной цензуре и четкого ТЗ у меня не было, а
условия договора выполнены?

2) Каков механизм расчета стоимости допечатной подготовки и самой печати? 

Дякую и слава Украине!

\ii{16_07_2021.fb.zaborin_dmitrij.1.kniga_pro_borsch_zapret.cmt}
