% vim: keymap=russian-jcukenwin
%%beginhead 
 
%%file 14_05_2022.stz.news.ua.pravda.1.ostriv_rosia
%%parent 14_05_2022
 
%%url https://www.pravda.com.ua/articles/2022/05/14/7345856
 
%%author_id dubinjans'kij_mihajlo,news.ua.pravda
%%date 
 
%%tags 
%%title Острів Росія
 
%%endhead 
 
\subsection{Острів Росія}
\label{sec:14_05_2022.stz.news.ua.pravda.1.ostriv_rosia}
 
\Purl{https://www.pravda.com.ua/articles/2022/05/14/7345856}
\ifcmt
 author_begin
   author_id dubinjans'kij_mihajlo,news.ua.pravda
 author_end
\fi

\ifcmt
  ig https://img.pravda.com/images/doc/d/e/de8c965-dubi-ostrov-ros.jpg
  @wrap center
  @width 0.8
\fi

Пушкин, Толстой, Достоевский, Булгаков. Чаще всего украинская дискуссия о
русской литературе, ее значении и ее стигматизации после 24 февраля
ограничивается именно этими четырьмя фамилиями.

Но если уж на то пошло, самое актуальное российское сочинение в 2022 году – это
не \enquote{Война и мир}, \enquote{Преступление и наказание} или
\enquote{Мастер и Маргарита}. Это \enquote{Остров Крым} Василия Аксенова.

О нашумевшей антисоветской книге, написанной в конце 1970-х, вновь вспомнили
восемь лет назад, во время российской аннексии Крыма. Тогда журналистов
привлекла география романа и совпадение фамилии автора с фамилией
новоиспеченного крымского гауляйтера.

Однако в действительности речь в романе идет вовсе не об украинском
полуострове, который авторская фантазия превратила в остров, населила
белоэмигрантами и сделала неким аналогом Тайваня. Нет, это книга о российском
государстве и его потенциальной способности быть другим.

\begin{qqnagolos}
Читайте также: \href{https://www.pravda.com.ua/rus/articles/2022/04/30/7342793/}{Кто был нечем}
\end{qqnagolos}

В разгар брежневского застоя у Аксенова хватило выдумки, чтобы нарисовать
альтернативную Россию – зажиточную, модернизированную и наслаждающуюся всеми
благами цивилизации. С развитой рыночной экономикой и свободным передвижением
своих граждан по миру, с небоскребами и суперкарами, с именитыми западными
гостями и блистательными светскими львицами, с изобильными супермаркетами
\enquote{Елисеев и Хьюз} и бойкими репортерами телекомпании \enquote{Ти-Ви-Миг}.

И тогда же, в 1979-м, автор безошибочно предсказал, что эта альтернативная
Россия без сопротивления сдастся на милость архаичного кремлевского монстра.
Что ни большие деньги, ни бытовой комфорт, ни технологический прогресс не
спасут ее от цивилизационного самоубийства.

В финале одни обитатели благоустроенного острова радовались приходу советских
танков, а другие пытались сбежать: но ни у кого не было желания и сил защищать
свой привлекательный образ жизни.

Фактически это краткое изложение новейшей российской истории. Истории о том,
как в РФ попытались построить уютный \enquote{остров Россия}: и эта затея, поначалу
казавшаяся достаточно успешной, потерпела крах.

\begin{leftbar}
Сегодня у каждого украинца есть соблазн постфактум объявить себя потомственным
русофобом, всю жизнь видевшим на севере неисправимую Империю зла. Но в
большинство случаев это будет лукавством.
\end{leftbar}

Достаточно вспомнить о звездах украинской журналистики – от Виталия Портникова
до Павла Казарина – успевших плодотворно поработать в Москве. Вряд ли кто-то из
них считал себя секретным агентом, заброшенным в Мордор.

Вместе с продвинутой частью россиян они действительно верили в перспективу
альтернативной России: цивилизованной, глобализированной и живущей по правилам
21 века.

Основания для такой веры имелись. Было время, когда российские медиа выступали
для украинских образцом свободомыслия и профессионализма.

Было время, когда шенгенские визы выдавались россиянам охотнее, чем украинцам.
Было время, когда всякое западное новшество сначала приживалось в бывшей
метрополии, а уж потом добиралось до нас.

\enquote{Остров Россия} манил постсоветскую публику московским шиком и финансовым
благополучием, широкими возможностями для бизнеса и творчества, карьеры и
досуга. Купаясь в нефтедолларах и будучи тесно интегрирован в глобальный мир,
он выглядел непотопляемым. Но в итоге оказался совершенно бессилен перед
призраками прошлого.

\begin{leftbar}
По мере того, как российское государство погружалось в воинственную архаику,
обитатели райского острова капитулировали: каждый по-своему.
\end{leftbar}

Одни сами заражались реваншизмом, активно поддерживали Путина, обвешивали свои
новые иномарки георгиевскими ленточками и бурно радовались \enquote{Крымнашу}.

Не будучи обременены логическим мышлением, они искренне надеялись совместить
имперский ресентимент с модными западными брендами, а навязчивую апологию СССР
– с отдыхом на Лазурном берегу.

Другие осуждали происходящее и, убедившись в необратимости государственного
курса, покидали страну.

А третьи верили в возможность сосуществования с кремлевским режимом. Мол, нужно
просто оставаться \enquote{вне политики}. Просто делать свою работу и развивать
собственный бизнес.

\begin{qqnagolos}
Читайте также: \href{https://www.pravda.com.ua/rus/articles/2022/04/2/7336113/}{Если завтра мир}
\end{qqnagolos}

Просто заниматься шопингом и осваивать новые гаджеты. Просто путешествовать по
миру и выкладывать красивые фото в \enquote{Инстраграме}. Главное, не нарываться, не
лезть на рожон – и тогда в путинском государстве можно будет по-прежнему вести
жизнь человека XXI века.

Оказалось, нет, нельзя. В 2022-м \enquote{остров Россия}, изрядно потрепанный в
предыдущие годы, окончательно пошел ко дну. Он исчез вместе с \enquote{Макдональдсом} и
\enquote{Нетфликсом}, с \enquote{Майкрософтом} и \enquote{Букингом}, с \enquote{Икеей} и \enquote{Спотифаем}, а заодно
и с последними либеральными реликтами вроде \enquote{Эха Москвы} или Московского центра
Карнеги.

\begin{leftbar} 
На смену российской модернизации пришла тотальная фашизация
страны и размахивание ядерной бомбой. На смену глобализированному быту – адские
санкции и стремительное превращение в КНДР-2. 
\end{leftbar}

Сегодня воюющая Украина регулярно сталкивается с бывшими островитянами – теми
самыми альтернативными россиянами, которые осуждают Кремль, вооруженную
агрессию и международную изоляцию РФ.

Несмотря на продвинутость этой публики, продуктивного диалога с ней обычно не
получается. И даже если отбросить все эмоции, порожденные войной, на то есть
веская практическая причина.

Оппозиционно настроенные россияне – независимо от личных мотивов – стремятся к
возвращению \enquote{острова Россия}. Циники хотели бы отмотать ситуацию на несколько
лет назад, когда можно было беспрепятственно летать в Европу и без ограничений
пользоваться банковскими картами.

Идеалисты желали бы возродить условный 2000 год, когда еще не была уничтожена
политическая конкуренция и свобода слова. Но и первые, и вторые мечтают о
благополучной жизни.

И те, и другие воображают целостную и экономически успешную Российскую
Федерацию, ведущую более или менее цивилизованное существование.  

Однако для украинцев, убедившихся в несостоятельности прежнего \enquote{острова
Россия}, такое решение неприемлемо.

\begin{leftbar}
В нашем представлении гарантии украинской безопасности связаны не с эфемерной
цивилизованностью РФ, а с ее физической неспособностью к агрессии. 
\end{leftbar}

Как минимум Россия должна быть раздавлена экономически –  чтобы не думать ни о
чем, кроме повестки выживания. Как максимум она должна быть уничтожена
политически – с распадом на несколько ослабленных и враждующих между собой
субъектов.

Ни первый, ни второй вариант не предполагают российского благополучия.

Это не просто взаимное недопонимание – это объективное столкновение интересов.
Мы не можем требовать от российских противников Путина, чтобы они пожертвовали
мечтой о благоустроенном \enquote{острове Россия}.

Но оппозиционные россияне не вправе рассчитывать на украинскую поддержку в
отстаивании этого привлекательного прожекта.

Михаил Дубинянский

Теми: Росія Україна війна
