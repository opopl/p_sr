% vim: keymap=russian-jcukenwin
%%beginhead 
 
%%file 28_11_2021.yz.soullaway_soullaway.1.megapolisy_sravnenie
%%parent 28_11_2021
 
%%url https://zen.yandex.ru/media/id/5dbc544f2f1e4400aff32830/sravnenie-megapolisov-rossii-i-ukrainy-61a2507297769e40fd7bd28a
 
%%author_id yz.soullaway_soullaway
%%date 
 
%%tags dnepropetrovsk,ekaterinburg,gorod,kiev,lvov,megapolis,moskva,odessa,rossia,sravnenie,st_peterburg,ukraina
%%title Сравнение мегаполисов России и Украины
 
%%endhead 
 
\subsection{Сравнение мегаполисов России и Украины}
\label{sec:28_11_2021.yz.soullaway_soullaway.1.megapolisy_sravnenie}
 
\Purl{https://zen.yandex.ru/media/id/5dbc544f2f1e4400aff32830/sravnenie-megapolisov-rossii-i-ukrainy-61a2507297769e40fd7bd28a}
\ifcmt
 author_begin
   author_id yz.soullaway_soullaway
 author_end
\fi

Мне тут иной раз попадаются размышления о том, что на Украине всё же есть
какое-то сейчас развитие и перспектив там, в ряде городов великое множество.
Заявления доходят даже до такого абсурда, что жизнь в некоторых мегаполисах
Украины лучше, чем в мегаполисах России. Хорошо, давайте попробуем сравнить
несколько городов. Само собой Москву мы тут брать не будем. Это город из
мирового списка наряду с Парижем или Нью-Йорком. Его влияние огромно и не менее
велико население. Было бы странно пытаться сравнивать Москву с Киевом. Киев это
локальный город, который не может влиять на мировые процессы.

Такая же история и с Санкт-Петербургом. Ничего аналогичного на Украине нет. Но
справедливости ради отмечу, что и в России нет ничего подобного Киеву. Нам вот
вообще очень не хватает такой точки притяжения с населением миллиона в три.
Хорошо если когда-нибудь на Урале и в Сибири появятся подобные города, но
сейчас их нет, и пока что не предвидится. Таким образом, можно попробовать
сравнить Харьков и, допустим что? Ну, пусть будет Новосибирск. Они, в общем-то,
схожи по населению и влиянию внутри страны.

Если рассматривать эти мегаполисы с точки зрения влияния на свои страны, то они
идентичны. Правда, Новосибирск немного крупнее. Там проживает 1 миллион 600
тысяч, а в Харькове 1 миллион 400 тысяч человек. Если взять метрополитены, то
Харьков обгоняет нашу столицу Сибири. Три ветки метро и 30 станций против двух
веток и 13 станций. То есть тут Россия, как ни странно отстаёт и отстаёт
серьёзно. В плане же притока населения, учебных заведений и предприятий оба
города почти идентичны. Распространяется это и на инфраструктуру и на
привлекательность города для жителей. Фотографии соответственно верху Харькова,
а ниже Новосибирска.
