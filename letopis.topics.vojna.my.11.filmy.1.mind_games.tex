% vim: keymap=russian-jcukenwin
%%beginhead 
 
%%file topics.vojna.my.11.filmy.1.mind_games
%%parent topics.vojna.my.11.filmy
 
%%url 
 
%%author_id 
%%date 
 
%%tags 
%%title 
 
%%endhead 

Здесь я пишу о своих впечатлениях о фильме Игры Разума (Mind Games), год 2021.
Я раньше смотрел другой фильм, тоже Игры Разума по русски, а по английски
Beautiful Mind, то есть Прекрасный Разум, о гениальном математике из Принстона,
Джоне Неше (John Nash), который являлся реально живущим человеком, который
действительно был гениальным математиком, страдающим умственным расстройством
(шизофрения), но который в итоге сумел научиться жить со своим душевным
недугом, в итоге вернувшись к полноценной работе ученого. По его биографии и
был снят замечательный фильм, который был даже удостоен в свое время Оскара.
Очень интересная и поучительная жизненная история, замечательная и тонкая игра
актеров, всем рекомендую посмотреть! Так вот, находясь под впечатлением от
этого фильма, я ожидал увидеть что-то похожее в фильме 2021 года с таким же
названием, но в этом фильме мои ожидания не оправдались, потому что я увидел
совершенно другую историю, гораздо более трагичную, ужасную, довольно тяжелую
для психики, можно даже сказать, выворачивающую душу наизнанку, причем
счастливого конца (хеппи энда), как это обычно бывает в американских фильмах,
здесь нет. Признаюсь, тяжело было заставить себя досмотреть этот фильм до
конца.

Ну что ж, начнем. Фабула фильма такова. Молодой армейский психолог, Тара Мейс,
к тому же молодая, красивая, жизнерадостная девушка, что то делает у себя дома.
Внезапно раздается звонок в дверь, Тара идет открывать дверь, там стоит
мужчина, Тара говорит - спасибо, вы там что то привезли, то есть, как будто
посыльный или курьер что то привез, вроде там речь шла про пиццу. Тара говорит
ему, не ожидая подвоха, но внезапно ей в лицо пшикают чем-то, и сцена
заканчивается, очевидно, пшикают чем-то, что влияет на сознание, Тара теряет
сознание, и потом уже обнаруживает себя в фургоне, прикованной цепью. Она
обнаруживает, что ее похитили, неизвестно кто. Тара в шоке и ужасе от того, что
происходит, и пытается собраться с мыслями. Ситуация ужасная, потому что
непонятно кто, зачем и почему решил ее похитить и поместить в фургон, приковав
ее к тому же цепью.

Параллельно, идет рассказ о работы Тары, а она, как уже было сказано, работает
армейским психологом в армии США, а именно, ее основная задача -
психологическая помощь военнослужащим, особенно служащим элитных войск
(спецназа), в решении их психологических проблем. Попутно она еще встречается с
молодым парнем, и периодически есть сцены с этим парнем. А что касается работы
психологом. Сидя уже запертой в фургоне, Тара начинает мучительно вспоминать
эпизоды своей работы с военными заключенными, а эти разговоры обычно
происходили в камере, где с одной стороны сидит психолог (Тара Мейс), а с
другой - сам заключенный, которого Тара спрашивала о психологических проблемах,
и с кем Тара общалась, беседовала, тет-а-тет, за круглым столом, с одной
стороны Тара, а с другой - заключенный, сидящий за столом напротив Тары, с
руками, положенными на стол, и закованными в наручники.




