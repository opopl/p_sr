% vim: keymap=russian-jcukenwin
%%beginhead 
 
%%file 18_11_2019.stz.news.ua.mrpl_city.1.viktor_orlov_posmihajtes_ne_vpadajte_u_vidchaj
%%parent 18_11_2019
 
%%url https://mrpl.city/blogs/view/viktor-orlov-posmihajtes-i-nikoli-ne-vpadajte-u-vidchaj
 
%%author_id demidko_olga.mariupol,news.ua.mrpl_city
%%date 
 
%%tags 
%%title Віктор Орлов: "Посміхайтесь і ніколи не впадайте у відчай!"
 
%%endhead 
 
\subsection{Віктор Орлов: \enquote{Посміхайтесь і ніколи не впадайте у відчай!}}
\label{sec:18_11_2019.stz.news.ua.mrpl_city.1.viktor_orlov_posmihajtes_ne_vpadajte_u_vidchaj}
 
\Purl{https://mrpl.city/blogs/view/viktor-orlov-posmihajtes-i-nikoli-ne-vpadajte-u-vidchaj}
\ifcmt
 author_begin
   author_id demidko_olga.mariupol,news.ua.mrpl_city
 author_end
\fi

Наш наступний герой - дуже енергійна, принципова, вольова і цілеспрямована
людина. Він вміє ставити цілі і досягати мети, адже працелюбності,
наполегливості та віри у власні сили йому не позичати. Мова піде про декана
факультету № 2 Донецького юридичного інституту МВС України \textbf{Віктора
Анатолійовича Орлова}.

\ii{18_11_2019.stz.news.ua.mrpl_city.1.viktor_orlov_posmihajtes_ne_vpadajte_u_vidchaj.pic.1}

Народився Віктор у 1980 році в Маріуполі в родині працівниці громадського
харчування та металурга. Виріс у Лівобережному районі міста. Дуже полюбляв у
дитинстві насолоджуватися відпочинком на березі моря.

\textbf{Читайте також:} \emph{Молодой ювенал Алина Парамоненко: \enquote{Обожаю работать с детьми}}%
\footnote{Молодой ювенал Алина Парамоненко: \enquote{Обожаю работать с детьми}, Юлия Франжева, mrpl.city, 15.11.2019, \par%
\url{https://mrpl.city/blogs/view/molodoj-yuvenal-alina-paramonenko-obozhayu-rabotat-s-detmi}}

Після закінчення школи Віктор вступив до Донецького інституту внутрішніх справ,
зараз - Донецький юридичний інститут МВС України. В школі полюбляв фізику і
математику, але точно не міг визначитись з професією. У випускному класі, коли
один з однокласників Віктора розповів про вказаний інститут, а також про те, що
там можна отримати професію слідчого, хлопець вирішив обрати саме цей внз. На
той час йому подобалась професія слідчого (до речі, слідчі є не тільки в
міліції (зараз поліції), вони є і у СБУ, прокуратурі, Державному бюро
розслідувань, НАБУ тощо). Дійсно, в радянських фільмах тих часів слідчий був
чесним, справедливим, професійним та завзятим. Саме ці риси завжди були
близькими нашому герою. Після закінчення інституту та отримання професії
слідчого доводилося працювати на різних посадах в міліції, поліції.

\ii{18_11_2019.stz.news.ua.mrpl_city.1.viktor_orlov_posmihajtes_ne_vpadajte_u_vidchaj.pic.2}

Віктор Анатолійович з посмішкою пригадує, що коли закінчив інститут і виходив з
його воріт, то подумав, що ще повернеться. Тоді він ще точно не знав –
магістром, ад'юнктом чи деканом факультету. Коли працював довгий час
заступником начальника слідчого відділення у своєму рідному Лівобережному
районі, то був наставником у молодих слідчих, які тільки закінчували навчальні
заклади. Він передавав їм свій досвід, навчав - і йому це дуже подобалось. Тоді
виникла думка про подальше навчання та роботу в інституті в якості викладача. У
2010 році він вступив до аспірантури, закінчив її, а поєднувати роботу і
навчання було важко. Закінчив аспірантуру, але захистити підготовлену
дисертацію не встиг - почалася війна. Темп роботи та покладені на поліцію
завдання не давали можливості вийти на захист роботи. Можливість захисту
з'явилась тільки у 2017 році. Віктор Анатолійович захистив дисертацію кандидата
юридичних наук за темою муніципальної поліції в Харківському національному
університеті імені Каразіна. Тоді він вже повністю прийняв рішення займатися
науково-педагогічною діяльністю. Спочатку прийшов працювати викладачем в
Академію поліції, а коли в березні 2018 року в  Маріуполі було створено
факультет № 2 Донецького юридичного інституту МВС України, був запрошений
ректором інституту \textbf{Віктором Бесчастним} очолити його.

\ii{18_11_2019.stz.news.ua.mrpl_city.1.viktor_orlov_posmihajtes_ne_vpadajte_u_vidchaj.pic.3}
\ii{insert.read_also.demidko.balabanov}

Є відомий вислів Конфуція: \emph{\enquote{Скажіть мені - і я забуду. Покажіть мені - і я
запам'ятаю. Дайте можливість обговорити - і я зрозумію. Дайте можливість
навчити іншого - і я досягну досконалості}}. Погоджуючись з цими словами, Віктор
Анатолійович вважає, що навчання курсантів, студентів - це з одного боку
передавання їм свого досвіду, а з іншого - це власний шлях до
самовдосконалення.

Оскільки в Україні проходить дуже багато цікавих процесів реформування у всіх
сферах життя, чоловіку хочеться бути корисним. Він прагне брати учать у
вказаних процесах, застосовувати свої навички, знання та отриманий досвід. Але
поки що він розвивається на своїй посаді, вона нова для нього, тому треба ще
багато чому навчитися. Однак на майбутнє залишається багато планів. Зокрема,
отримання наукового ступеня доктора наук (зараз - кандидат юридичних наук), а в
плані служби в поліції – спеціального звання полковник (зараз - підполковник
поліції).

%\ii{18_11_2019.stz.news.ua.mrpl_city.1.viktor_orlov_posmihajtes_ne_vpadajte_u_vidchaj.pic.4}
\ii{18_11_2019.stz.news.ua.mrpl_city.1.viktor_orlov_posmihajtes_ne_vpadajte_u_vidchaj.pic.4_5}

Дружина Віктора родом з міста Донецька. Вона підтримує свого чоловіка, хоча
його робота завжди займала дуже багато особистого часу. Найбільше надихає
чотирирічна донечка \textbf{Олександра}. Весь вільний час він намагається
розподіляти між роботою і родиною. Маріуполець активно займається спортом, не
пропускає жодного тренування у тренажерному залі.

\textbf{Улюблене місце в Маріуполі:} Морський бульвар.

\textbf{Улюблена книга:} \enquote{1Q84. Тисяча невестьсот вісімдесят чотири} Харукі Муракамі.

\textbf{Улюблений фільм:} \enquote{1+1 (Недоторкані)} (2012 рік).

\textbf{Курйозний випадок з життя:} 

\begin{quote}
\em\enquote{Можу пригадати випадок, коли у далекому 2002 році, коли ще не було мобільних
телефонів, я домовився зі своїм другом зустрітися в Алушті в Криму. Він
поїхав в Алушту. Ми домовились, що він зателефонує на міський телефон
своєму дідусю в Сімферополь та розкаже про те, де ж він зупинився. Коли
я приїхав в Сімферополь, він ще з ним не зв'язався, я не став чекати і
прийняв рішення все ж поїхати в Алушту і там знайти свого друга. Від
діда я дізнався, що він відпочиває на пляжі, де є парашут. Я був
впевнений, що зможу знайти друга, але пробув на пляжі до вечора,
розглядаючи там всіх, а його не побачив. Відпочивальників було дуже
багато. Я розумів, що його вже не знайду, вже збирався на вокзал шукати
житло. Практично впав у відчай. Та тут на набережній випадково (а може
й ні) побачив свого друга (він був дуже здивований). Це було дуже
несподівано - на багатотисячній набережній Алушти все ж зустрітися. І
тоді я зрозумів, що якщо чогось дуже сильно захотіти, то це
здійсниться}. 
\end{quote}

\ii{18_11_2019.stz.news.ua.mrpl_city.1.viktor_orlov_posmihajtes_ne_vpadajte_u_vidchaj.pic.6}

\textbf{Порада маріупольцям:} 

\enquote{Ніколи не зупинятися на шляху до самовдосконалення, знаходити позитивні
моменти в житті, посміхатись і ніколи не впадати у відчай...}

\ii{insert.read_also.demidko.kyrnyckyj_policeman}
