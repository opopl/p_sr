% vim: keymap=russian-jcukenwin
%%beginhead 
 
%%file 10_11_2020.fb.georgii_protoierei.1.zolushka_film
%%parent 10_11_2020
 
%%url https://www.facebook.com/permalink.php?story_fbid=203887511151958&id=100045921323126
%%author 
%%tags 
%%title 
 
%%endhead 

\subsection{Невыдуманные истории - Золушка}
\label{sec:10_11_2020.fb.georgii_protoierei.1.zolushka_film}

\Purl{https://www.facebook.com/permalink.php?story_fbid=203887511151958&id=100045921323126}
\Pauthor{Бандура, Татьяна}

\ifcmt
pic https://scontent.fdnk6-1.fna.fbcdn.net/v/t1.0-9/124296949_203887494485293_3777928265479359741_o.jpg?_nc_cat=100&ccb=2&_nc_sid=730e14&_nc_ohc=MDXbztDcZMYAX871P_B&_nc_ht=scontent.fdnk6-1.fna&oh=3acf329ee9326c581db1dafa253de5ca&oe=5FD1B300
\fi

Режиссер ленты Надежда Кошеверова вспоминала, как в 1944-м, возвращаясь из
эвакуации, встретила Янину Жеймо на вокзале.

Та сидела в уголке --- такая маленькая, растерянная.

А ведь рост у актрисы --- 147 см.

Двумя годами ранее ее супруг, считая Янину погибшей, женился на другой.

Когда выяснилось, что актриса жива, он не стал ничего менять и не отдал ей
детей.  Женщина была на грани самоубийства.

На вокзале Надежда Кошеверова взглянула на нее и неожиданно предложила:
«Яничка, вы должны сыграть Золушку…»

Евгений Шварц написал сценарий специально для Жеймо.

29 мая 1945 года его зачитали и обсудили на худсовете «Ленфильма».

Художник Николай Акимов создал прекрасные эскизы, а также макет королевского
замка высотой 100 см.

Замок в картине мы видим благодаря комбинированным съемкам именно этого макета.
Экспозицию на экране с помощью доступной тогда технологии потом совместили с
актерами. Художник прорабатывал все декорации и костюмы для съемки на цветной
немецкой пленке. Но сняли на черно-белую.

Атмосфера Ленинграда 1945-го меньше всего подходила для сцен балов и эпизодов с
нарочитой роскошью. Декорации и костюмы шили всем миром. Платья --- из занавесок
и скатертей, антикварную мебель приносили из дома. Часть платьев были
трофейными --- привезли из оперного театра в Берлине. Сложным стал подбор
костюмов для Золушки и феи.

В 1946-м Фаина Раневская снималась в ленте «Весна» в Праге, откуда привезла
лампу в виде луны, украшенную бисером. Абажур от нее и стал головным убором
феи.  Янина Жеймо сильно похудела за годы войны. Ее направили в санаторий,
поставив задачу набрать 5 кило. Актриса не хотела и даже горько иронизировала
по поводу стройности из-за военного голода.

Многих смущал возраст Янины --- 37 лет.

Василий Меркурьев, сыгравший отца, в жизни был старше актрисы всего на 5 лет.
Пробы прошли и другие претендентки на роль Золушки, но режиссеры отстояли свою
протеже. Лишь подобрали 34-летнего Алексея Консовского на роль принца, чтобы
пара смотрелась гармонично.

Фею сыграла Варвара Мясникова, ранее известная по роли Анки-пулеметчицы в
фильме «Чапаев» (1934 год). 11-летнего Игоря Клименкова, ставшего пажом, нашли
в Ленинградском Доме пионеров. Мальчишка прошел пробы с 25 тысячами других
претендентов. Он вспоминал: «Янина Жеймо вела себя как обычная девчонка, в
перерыве между съемками мы залезали с ней в карету-тыкву и болтали. Я лузгал
семечки, Золушка курила. С ней было легко и просто, я был по-детски влюблен…»

Сложно представить другого актера в роли короля, которого сыграл Эраст Гарин.
По воспоминаниям очевидцев, иногда он появлялся в гримерной слегка подшофе, не
находил какой-нибудь части своего костюма и начинал ругаться. «Эраст, ты --- хам,
ты видишь, здесь ребенок», --- возмущалась Фаина Раневская, указывая на
мальчика-пажа, и шлепала Гарина по физиономии. Конечно же по-доброму.

Фаина Георгиевна, сыгравшая мачеху, тоже была исхудавшей и не походила на
сытого домашнего тирана. В кадре ей клеили нос, а за щеки набивали вату. Многие
хлесткие реплики Фаина Раневская придумывала сама. Со сценаристом возникали
регулярные споры. Она импровизировала: «Жалко, королевство маловато,
разгуляться мне негде! Стану королевской свахой, поссорюсь с соседями, тогда и
территорий прибавим!» Интерьеры бального зала и других помещений выстроили
осенью на «Ленфильме». Павильоны не отапливались, актеры в шикарных нарядах
укутывались в телогрейки, тулупы, сбрасывая их по команде: «Мотор!».

Завистников у авторов ленты хватало. Писали доносы.

Еще на этапе подготовки дважды хотели закрыть производство из-за
неблагонадежности всей съемочной группы. Сорежиссер Михаил Шапиро женат на
родственнице Троцкого Жанне Гаузнер. Актер Василий Меркурьев женат на дочери
уничтоженного Мейерхольда, усыновил трех детей репрессированного брата и двух
осиротевших ребят после расстрела их родителей. Художник Акимов --- ученик
белоэмигранта Анненкова. Алексей Консовский --- из семьи врагов народа,
расстреляны отец и брат. И так далее по всему списку.

После съемок в Риге 24 сентября 1946-го состоялся худсовет, где просматривали
отснятое.

Режиссер Ян Фрид метил в авторы костюмированных исторических комедий. В
режиссерах «Золушки» видел прямых конкурентов и выступил со словами:
«Дегенеративен Гарин, который никак не вызывает симпатии, неприятен принц —
рахитик с разжиженными мозгами».

Эраста Гарина не раз вызывал к себе директор «Ленфильма» и заявлял: «Вы играете
ненастоящего короля! Глава государства не может так себя вести!» Несмотря на
препоны, картину впервые показали в 1947-м в Доме кино в Ленинграде.  Премьера
была действительно мировой.

28 ноября 1947 года ленту демонстрировали в Финляндии, 23 декабря --- в Австрии,
24 марта 1948-го --- во Франции, 13 ноября 1949-го --- в Швеции, 24 марта 1951-го —
в Японии.

.....когда спустя много лет спустя Надежду Кошеверову спросили, почему она больше не снимает сказки, то она ответила: Посмотрите в глаза Золушки! В них сказка! А в глазах нынешних молодых актрис - рубли, как в кассовом аппарате...

