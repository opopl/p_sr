% vim: keymap=russian-jcukenwin
%%beginhead 
 
%%file 01_12_2020.news.ua.pravda.marinovich_miroslav.1.revolution
%%parent 01_12_2020
 
%%url https://www.pravda.com.ua/articles/2020/12/1/7275415/
 
%%author Маринович, Мирослав
%%author_id marinovich_miroslav
%%author_url 
 
%%tags 
%%title Після кожної революції український народ зростає у своїй гідності. Але це ще не стеля --- Мирослав Маринович
 
%%endhead 
 
\subsection{Після кожної революції український народ зростає у своїй гідності. Але це ще не стеля --- Мирослав Маринович}
\label{sec:01_12_2020.news.ua.pravda.marinovich_miroslav.1.revolution}
\Purl{https://www.pravda.com.ua/articles/2020/12/1/7275415/}
\ifcmt
	author_begin
   author_id marinovich_miroslav
	author_end
\fi

Ольга Кириленко --- Вівторок, 1 грудня 2020, 05:30

\begin{leftbar}
	\bfseries\em
Сім років тому, близько 4 ранку 30 листопада 2013 року силовики розігнали
студентів на Майдані Незалежності в Києві.

Вже наступного дня, 1 грудня, українці вийшли на один з наймасовіших протестів
в історії України, що стало потужним поштовхом до подій, які згодом отримали
назву Революції Гідності.

На відміну від Помаранчевої революції, яка сталася за 9 років до того, цього
разу люди не вірили політикам. Потрібні були інші авторитети, які б своїми
словами і діями могли надихнути і надати віри, що боротьба не марна.

Через тиждень під час Маршу Мільйонів зі сцени на Майдану Незалежності
ініціативна група \zqq{Першого грудня}, створена з відомих інтелектуалів і
громадських діячів, виступила зі зверненням \zqq{В Україні настав новий час}, в
якому засудила насильство та закликала берегти цілісність України.

Тоді голос моральних авторитетів був почутий і відчутний.

Однак через 7 років після Революції, анексії Криму і війни на Сході, яка триває
досі, в суспільстві відчуваються все ті ж сумніви, зневіра в політиках і у
кращому майбутньому.

Суспільство знову потребує голосу розуму, справжніх моральних авторитетів, які
своїми словами допомогли б розібратися, що ж з нами відбувається і чи є світло
в кінці цього нескінченного тунелю.  

Саме тому \zqq{Українська правда} до річниці Революції вирішила поспілкуватися з
одним з тих, хто був і лишається справжнім моральним авторитетом. 

В інтерв’ю \zqq{Українській правді} Мирослав Маринович пригадує \zqq{романтичний}
початок Революції Гідності, розмірковує про сильні українські \zqq{горизонталі},
прикладом яких став Майдан, і припускає, що ще, окрім політичних лідерів і
партій, можуть породжувати революції
\end{leftbar}

\ifcmt
pic https://img.pravda.com/images/doc/f/4/f4d1293-theuk-09.jpg
caption Маринович: \zqq{Це побиття студентів було несправедливим. Вони стояли мирно, не загрожували державі своїм протестом. А оскільки держава пішла брутальною силою проти них --- піднялося суспільство. Тоді почалася фактично Революція Гідності}
\fi

\textbf{– Після побиття студентів на Майдані Незалежності 30 листопада 2013 року, яке
згодом вилилось у хвилю опору та врешті Революцію Гідності, минуло сім років. З
якими думками, відчуттями, емоціями ви пригадуєте початок тих подій, початок
Революції?}

– Це був дуже важливий злам у Євромайдані, який перейшов у Революцію Гідності.

Я пригадую перші дні Євромайдану: було відчуття, принаймні у Львові, що студенти хотіли поберегти свій студентський характер протесту --- дуже світлий, гарний. Вони запрошували лише тих промовців, яких хотіли. Не допускали, скажімо, релігійних діячів. Один протестантський пастор просив мене заступництва, щоб я допоміг йому… 

Тобто це було таке трішечки… Я не хочу ображати, воно було дуже тепле, дуже гарне. Але трішечки романтичне.

І тут раптом --- у той романтизм увірвалася брутальна сила держави. Це перше
побиття студентів змінило атмосферу в суспільстві і включило те, що я вважаю чи
не найголовнішим механізмом протестного руху в Україні --- відчуття, що це
несправедливо. 

Це побиття студентів було несправедливим. Вони стояли мирно, не загрожували
державі своїм протестом. А оскільки держава пішла брутальною силою проти них –
піднялося суспільство. Тоді почалася фактично Революція Гідності. 

Ця зміна вплинула на подальший перебіг подій. Згодом була друга зміна --- це
прийняття отих драконівських законів, перехід Майдану на Грушевського. 

З боку влади був, фактично, насильницький характер реагування на революцію. Але
30 листопада --- це день, коли фактично почалася Революція Гідності.

\ifcmt
pic https://img.pravda.com/images/doc/6/a/6a6ebb9-128764315-856675658438014-2718621440347295112-n.jpg 
caption Мирослав Маринович: \zqq{Я називаю і Помаранчеву Революцію, і Революцію Гідності --- революціями духу. Передусім духу, а не політичними революціями}
\fi

\textbf{– Чого у ваших спогадах про 30 листопада 2013 року більше --- смутку, радості,
гордості?}

– Я маю більше радості. 

Всі ці емоції є, але вони по-різному розподіляються, вони в різних моїх файлах,
так би мовити. 

Якщо говорити про мою оцінку Революції Гідності, то вона дуже висока. І це
почуття великої гордості за народ, великої радості, що ми спромоглися на таке
піднесення духу. 

І тому я називаю і Помаранчеву Революцію, і Революцію Гідності --- революціями
духу. Передусім духу, а не політичними революціями. 

Вони не змінили, скажімо, олігархічний лад. Але вони утвердили незалежницький
дух українців. 

Усі ці три революції, включно з Революцією на граніті, не допустили реалізації
російського сценарію --- російських задумів, російських спроб повернути собі
Україну. У цьому їхня фантастична перемога. 

Те, що ми ще не реалізували своїх сподівань --- це інше. Майдани досягли, були
дуже переможними.

Смуток був. Я вже трішечки про нього забув. А от тепер, дивлячись на кадри
Революції Гідності, які показувало українське телебачення, десь у душу увійшов
неспокій, сум. Я мушу сказати, що завдячую Порошенку тим, що в часи його
президентства відступив страх за державу. 

Ми могли критикувати Порошенка, могли бути невдоволеними тим, що олігархічний лад все ще має силу в Україні. Але отой страх за державу відступив зовсім.

Пригадую, коли виграв президент Зеленський, моя дружина раптом сказала: "Тепер, здається, у мою душу знову входить страх за державу". І ми усвідомили, що всі ці 5 років його не було.

{\bfseries --- У вашу душу також повернувся цей страх?  }

– Повернувся неспокій. Я бачу дуже багато негативних сигналів і я, як би то сказати акуратно --- не знаю майбутнього. 

Ці негативні сигнали можуть нас готувати до нового духовного піднесення. Але це
можуть бути і елементи занепаду. Ніхто з нас не знає майбутнього. Тому цей
неспокій залишається.

\ifcmt
pic https://img.pravda.com/images/doc/c/e/ce8ca49-128950108-155323539652523-3461900231685604697-n.jpg
caption Мирослав Маринович: \zqq{Попри наші \zqq{американські гірки}, які страшенно болісні й дуже вимотують нерви, суспільство зростає у своєму засвоєнні свободи}
\fi

\textbf{– Пане Мирославе, ви називаєте революції, зокрема останню, революціями духу. У
суспільстві останню революцію прийнято називати Революцією Гідності. На вашу
думку, ми як суспільство, в цих подіях вибороли гідність для себе?}

– Це добре питання тому, що був цікавий феномен після завершення Революції Гідності --- смітникова люстрація.

Український народ постійно, після кожної революції утверджує свою гідність. Він
зростає у своїй гідності. Але це ще не стеля. 

І те, що в нас ще багато роботи, засвідчила, власне, смітникова люстрація. 

Давайте проаналізуємо. Під час Революції Гідності ми говорили Януковичу так
само, як говорили на Помаранчевій Революції --- що ми є люди, що він не сміє у
нас відбирати гідність, тому що гідність нам давав не він, а Господь Бог.
Гідність притаманна людині з народження, він не може відбирати те, що йому не
належить. 

І тут після завершення революції раптом люди побачили, що деякі чиновники діють
так, як діяли до революції, що на них нічого не вплинуло --- та сама корупція, те
саме зловживання владою. І відбулася гостра реакція на це --- протестні групи
почали кидати тих нерадивих чиновників у смітники. 

Але ж, вибачте, це приниження людської гідності. Ми, обурені їхніми діями, мали
право і зобов’язані були подавати на них в суд, боротися з їхніми
зловживаннями. Але ми не сміємо принижувати їхню гідність. 

Бо як ми казали на Майдані? Не ми давали їм гідність, а Господь Бог. 

Тобто обстоюючи свою гідність від влади, від нелюбого Януковича, ми легко
впадали в його пастку і принижували гідність інших людей, яких ми вважали нашим
ворогами.

А яка типова реакція людей: "Та що суд? Та він корумпований"! Але вибачте, а
чия це вина, як не нашого суспільства, що ми маємо корумповані суди? Це не
хтось винен, це ми винні. 

Тому що всі ми миримося з корупцією. Усі люди, ті, що кричать проти
корупціонерів, вони так само миряться з корупцією. 

Я пригадую, як в час Помаранчевої Революції дуже гарна була стаття одного
автора --- "не думаймо про того Януковича, Янукович сидить у нас". 

Це була блискуча стаття, яка показала, що ми делегуємо своє обурення на
Януковича, не бачачи, що зерна того непорядку є в нас самих. Угорі є тільки
екстракт того, що в наших душах нанизу.  

Можна уподібнитися збиванню яблук з дерева в надії, що наступного року
виростуть груші. Не виростуть груші. Виростуть знову яблука. Тому що дерево
залишається те саме і коріння в цього дерева в наших душах.

Крім того, також дуже боляче вдарила по мені оця історія з плівками, коли
нардеп з партії Ляшка...

\textbf{– Ігор Мосійчук.}

– Мосійчук, так. Він активний майданівець. І я щиро вірю, що він щиро боровся
на Майдані. 

А потім він собі говорить так: "Я хочу добра Україні, моя партія хоче добра
Україні. Отже, підтримка моєї партії буде на добро Україні. А не важливо, як ми
підтримаємо свою партію. Головне --- результат". І вимагає хабара. Тим самим
порушуючи майданні цінності, за які боровся.

Несторівська група сформулювала цю тезу у своєму документі кілька років тому –
ми вміємо боротися за цінності, але не вміємо жити відповідно до них. І це
почасти стосується також і гідності. Тобто ми вміємо боротися за нашу гідність,
але жити гідним життям нам не завжди вдається.

\ifcmt
pic https://img.pravda.com/images/doc/0/9/09c9027-128756410-401886934192602-799708839070950042-n.jpg
caption Мирослав Маринович: \zqq{Український народ постійно, після кожної революції утверджує свою гідність. Він зростає у своїй гідності. Але це ще не стеля}
\fi

{\bfseries 
– До цього треба дорости?
}

– Так, це доростання. Я пригадую цю фразу Блаженнішого Любомира, якою він мені
відкрив цілу філософію. Це був момент, коли миряни греко-католицької церкви
напосілися на блаженнішого Любомира, тоді ще глави церкви, кажучи: \zqq{А от в
моєму селі священник те і те недобре робить}. 
А другий: \zqq{А в моєму селі те, а в
моєму місті, у моїй парафії --- такий!}.

На нього посипалися вістки про якісь хибні дії священників. Блаженніший
намагався якось захиститися від тих докорів і звинувачень. А тоді сказав
врешті-решт таку фразу: "Дорогі мої, дайте людям вирости у свободі". Оця фраза
відкрила мені цілий світ. 

Ми йдемо з рабського суспільства. Раби не можуть бути одразу вільними. Момент
їхнього звільнення ще нічого не означає, тому що вони в душі ще залишаються
рабами. Цей перехід від рабського стану до стану усвідомленої свободи вимагає
часу, вимагає поту і крові, вимагає проб і помилок. Наші революції, власне, і є
отакими елементами зростання суспільства.

Я для студентів малюю таку криву: спалах --- Революція на граніті, потім падіння,
тоді спалах --- Помаранчева Революція, тоді знову падіння. Але падіння не до
рівня того, що був після Революції на граніті. Знову спалах --- Революція
Гідності, і знову певна деградація сьогодні. Але не до того стану, що була
перед Помаранчевою революцією. 

Тобто постійно, попри наші \zqq{американські гірки}, які страшенно болісні й дуже
вимотують нерви, суспільство зростає у своєму засвоєнні свободи.

– Сім років тому ви вірили, що Україна скине Януковича?

– Признаюся вам, що перед Помаранчевою революцією я собі сказав: Господи, ну
невже цей народ ніколи не підніметься більше?

Літо 2004 року для мене було дуже тяжке психологічно. Я бачив занепад людей,
якусь прострацію і т.д. І тому Помаранчева Революція була для мене
несподіванкою --- приємною, натхненною несподіванкою. 

Пізніше, перед Революцією Гідності, я вже такого відчуття безнадії не мав. Було
відчуття --- ні, все-таки якесь піднесення духу українського буде. І коли воно
сталося, це була реалізація того, на що я сподівався.

\ifcmt
pic https://img.pravda.com/images/doc/3/6/3642210-128913677-188595819582733-2098556834376309169-n.jpg
caption Маринович: \zqq{Майдан --- це була спільнота наметів, сотень, які жили своїм автономним життям. Але коли треба було виконати спільну дію, вони виконували} 
\fi

{\bfseries 
– Чому Майдан не народив нових політичних лідерів?
}

– З одного боку, я міг би повторити всі ці розчаровані висновки, які зробило
ледь не все суспільство --- революція нічого не дала, бо не дала якісно нової
еліти. 

Відразу після Революції Гідності я був у Вашингтоні, на засіданні правління
National Endowment for Democracy, розповідав їм про Революцію Гідності. Вони
були дуже заінтриговані тим явищем і хотіли почути мою думку. 

Під час свого виступу я сказав, зокрема: \zqq{На жаль, революція не дала нової
політичної сили, не дала партії Майдану, не дала тієї  \zqq{Солідарності}, як у
Польщі в 80-х роках. І це очевидно наша національна біда}. Я згадав їм про
отаманщину, про те, що в нас у суспільстві є щось, що заважає творенню якоїсь
сили. 

На що члени правління заусміхалися і сказали: \zqq{Ні, ні, ні, це не лише ваша
українська  проблема. Тому що рух Occupy у Нью-Йорку, парасольковий протестний
рух у Гонконзі і навіть мусульманські революції --- всі відкинули централізоване
правління, керівництво. Вони не мали одного штабу управління}. І кажуть мені
американці --- змінився характер суспільства.

У нас зараз потужно діють горизонталі, а не вертикалі. І Майдан якраз був
відображенням горизонталей. 

Бо Майдан --- це була спільнота наметів, сотень, які жили своїм автономним
життям. Але коли треба було виконати спільну дію, вони виконували.

Це є мережевий рух, мережева структура. Тоді я раптом усвідомив, що ми маємо
зміну параметрів суспільства. Щось змінилося в способі функціонування людства. 

Наші звичні уявлення, що в результаті протестного руху має бути потужна
політична сила, яка візьме владу і змінить все, ця візія починає буксувати. Я
навіть не знаю, чи я правильно формулюю це, бо це тільки-тільки формується.
Свого часу воно знайде свої термінологічні оздобини.

Але звідси випливає одна, я би сказав, напів божевільна думка. 

З релігійної літератури, з Євангеліє мені пригадується Ізраїль у часи Ісуса –
він очікував на політичного месію, царя, який дасть незалежність і змінить усе.
А коли він прийшов, обіцяний пророками --- його не впізнали! Тому що очікували
іншого. 

Може ми просто не зауважуємо іншої сили в суспільстві, яка з’являється?  

Можливо, для нинішньої України, як тоді для Ізраїлю, головними є духовні
розв’язки? Такими, що принесуть нам успіх на політичному рівні, на державному?
А духовні розв’язки і формулюються, і вирішуються не політичним шляхом, а
духовним. 

У мені раптом закрався сумнів, що мої традиційні пошуки --- Господи, коли ж
нарешті прийде і президент, і якась еліта, яка буде те і те, і те --- є ілюзіями.
А я не бачу чогось іншого --- такого, що принесе Україні успіх, але на духовному
рівні.

\ifcmt
pic https://img.pravda.com/images/doc/b/4/b419722-img-9895lvbs.jpg
caption Мирослав Маринович: \zqq{Духовне піднесення та інтелектуальна модель нової держави --- це те, що нам сьогодні потрібно}
\fi

{\bfseries 
– Що ви маєте на увазі?
}

– Дозвольте я візьму за модель слова російського дисидента Андрея Амальрика про
дисидентів.

Він дав дуже гарну дефініцію: дисиденти --- це ті люди, хто, живучи в невільній
державі, поводилися так, наче вони вільні. Тобто в рамках старої системи
з’явилися елементи нової системи, які, звичайно ж, мусили йти на жертви,
оскільки відразу їх система ламала. 

Але вони, борючись проти системи, привносили нові для того часу цінності. Вони
задавали нову норму поведінки. Таким чином вони виграли. 

Таким чином у тому уже напівпрогнилому суспільстві з’являється критична маса
людей, які задають нову норму організму. Приблизно за тою самою моделлю, я собі
уявляю, має відбутися зміна в Україні. 

Бо поки що ми уявляємо собі таким чином: от нарешті нові вибори, з’явиться
новий технологічний проект, вони запропонують цікаву програму, ми їх виберемо і
вони змінять наше суспільство! Неправильно. Спершу має змінитися низ, спершу
внизу має з’явитися політична сила. 

І як модель я можу назвати Конрада Аденауера і Християнську демократію у
повоєнній Німеччині. Це потужна суспільна сила, яка сказала: ми жили
неправильно і довели свою країну до руїни, ми маємо покласти в основу соціальну
доктрину церкви. 

Це був масовий рух, на якому виросла політична партія. Бо вони опиралися на
нові цінності. Виросла політична партія, вона прийшла до влади і почався
нормальний розвиток. 

Отже, нам потрібна нова ціннісна платформа і нам потрібен проект комплексної
мутації, комплексної реформи нашої державної системи. 

Бо неможливо змінити одне, не змінивши інше. Що дала нам реформа поліції без
реформи судової влади? Судова влада поновила практично всіх тих міліціонерів,
які нібито були очищені. 

Оці два елементи --- духовне піднесення та інтелектуальна модель нової держави –
це те, що нам сьогодні потрібно. Не нова партія під нові вибори і т.д. От у це
я вже не вірю. 

Мусить спершу з’явитися критична маса людей, яка повторить той жест, який Юрій
Орлов продемонстрував західним журналістам: вони запитали в нього, що змусило
його, абсолютно успішного фізика-теоретика піти в дисиденти і бути викинутим з
Академії наук, а він показав: \zqq{Надоєло}. 

Людям має набриднути жити в ціннісно хибному світі. Має набриднути суддям,
прокурорам, вони мають вони сказати --- все, досить! Не може нами отой Вовк
керувати.

– Але ж відбулось уже чимало подій, зокрема ті ж плівки Вовка, після яких можна
сказати \zqq{усе, досить}. Що ще має статись?

– Маєте рацію. Я щораз більше чую думки, що Зеленський скінчить тим, що
підніметься новий Майдан. 

Ну добре, якщо він підніметься, то що буде? Чи є суспільно-політична сила, яка
несе альтернативу? Ми її не маємо. 

І тому Майдани будуть. Вони відіграють свою роль, але вони не будуть
ефективними в тому сенсі, в якому ми хочемо. 

Якщо говорити про інтелектуальну складову, то я пригадую, як тішився, коли
почув інтерв’ю пана Рябошапки, колишнього генерального прокурора. Він говорив,
що зі своїми друзями хоче сформувати платформу реформування юридичної системи –
я аж підстрибнув на своєму кріслі. Дуже добре, це саме те, що потрібно!

Бо я вважаю, що зараз потрібне дисидентство професіоналів. Дисидентство
юристів, бізнесменів, освітян і т.д., по кожній професійній ланці люди мають
чітко сформулювати, як має виглядати держава, та, яку ми хочемо.

\ifcmt
pic https://img.pravda.com/images/doc/f/c/fc9d1e0-theuk-17.jpg
caption Мирослав Маринович: \zqq{Оцінки Революції Гідності можуть бути і є різні --- нема ради тут, ми з цим довго ще будемо жити}
\fi

– Про Революцію Гідності. Окрім погляду на ці події, як на революцію, яка
багато змінила як всередині країни, так і в нашому зовнішньо-політичному
векторі, серед тих, хто не сприйняв Майдан існує думка, що революція стала
причиною занепаду країни і поширення тотальної ворожнечі в суспільстві. 

Чи можна ще чимось, окрім впливу Росії, пояснити існування такої думки
всередині країни?  

– Я розумію, що оцінки Революції Гідності можуть бути і є різні --- нема ради
тут, ми з цим довго ще будемо жити. 

Так, можливо, революцію називають як чинник, який ділить. Бо це ніби тест --- чи
ти за, чи ти проти, відповідно, є поділ. Але революція проілюструвала деякі
дуже важливі речі. 

Зокрема, вона проілюструвала силу спільного ворога. Це один з найважливіших,
найсильніших клеїв, які склеюють суспільство. І тому промайданна частина була
фантастично об’єднана Майданом --- весь негатив пішов на Януковича. Усе, що було
проти Януковича, об’єдналося. 

Це привело до дивовижного єднання усіх національностей на Майдані --- були євреї,
кримські татари. Майдан молився усіма молитвами --- і християнською, і
іудейською, і мусульманською. 

Коли пішли перші жертви Небесної сотні, то я просто аж вражений --- то тобі
вірменин,то тобі білорус. Ей, а чи там є українці на Майдані, подумав я в
якийсь момент. 

– Але ж були й ті, хто стояв по ту сторону барикади.

– Так, я розумію, але оце об’єднання --- воно було. І люди, які були на Майдані,
пам’ятають це. І певний час після Майдану люди несли цю пам'ять, пам’ятали, що
вона була важлива, що вона взагалі була. 

Так, сьогодні немає тої єдності між консерваторами, традиціоналістами і
модернізаторами. Та єдність розбилася. Розбилася єдність між україномовними і
російськомовними. 

Але все-таки, якщо пошкрябати в пам'яті людей, то вони пам’ятають, що це було
можливо на Майдані. І тепер неможливо сказати, що це утопія. Ні, не утопія,
воно було можливим.

Але нам потрібно замінити формулу спільного ворога на, я би сказав, формулу
спільного Бога. Формулу тих цінностей, які нас об’єднують. А нас об’єднували
цінності гідності людини, милосердя, взаємної підтримки, взаємного розуміння.
Пробачення якихось хибних кроків...

Тому знову ж таки я вертаюся до цінностей. Бо це те, що повинно нас об’єднати.

\ifcmt
pic https://img.pravda.com/images/doc/5/b/5bbd40d-ed.uku.jpg 
caption Маринович: \zqq{Українці мають один гріх --- вони дуже часто розуміють свободу як сваволю}
\fi

{\bfseries 
– Є люди, які виходили на Майдан, і які з часом у ньому розчарувалися. Як пояснити такий феномен?
}

– Знову ж таки, це неминуче. Такою є людина. Одна людина сильна, друга людина слабка. 

Були і в наш дисидентський час люди, які потягнулися до світлого, гарного, нового, дістали один раз чи два рази обухом від КГБ, та й відступили. І розчарувалися. 

Так само і тут. Там, де є люди, ви обов’язково будете мати і феномен розчарування, але попри те, ви будете мати феномен сили.

{\bfseries 
– Після Майдану в нас переважає феномен сили?
}

– Майдан залишив у нас велику силу, яка сьогодні себе не усвідомлює. Яка розбита, розосереджена... 

Але я думаю, що в якийсь один момент --- і цей момент стається не з волі людини,
а з волі Святого Духа --- це може відновитися, усе це може піднятися. І знову ті
зв’язки між людьми поновляться.

\textbf{– Після революції в Києві на Майдані Незалежності довгий час висів плакат з
розірваними ланцюгами і написом \zqq{Свобода --- це наша релігія}. Якщо
відштовхнутися від цього вислову, то на кого ми, українці, молимося, хто є
нашим Богом?}

– Я можу прокласти місточок до християнського Бога, який і дав нам цю свободу,
дав нам високе розуміння свободи. Але що важливо: Українці мають один гріх –
вони дуже часто розуміють свободу як сваволю.

Свобода перетворюється на сваволю людей, які не терплять над собою вказівок. Чи
є це свобода у справжньому, високому розумінні? Ні. Бо свобода у високому
розумінні означає і передбачає відповідальність людей. 

Свобода і відповідальність, баланс одного і другого. Бо просто свобода, яка не
несе за собою відповідальності, приводить до великих проблем.

Якщо до гасла, про яке ви говорили, додати: \zqq{свобода + відповідальність} є
нашим Богом --- ось я би тоді радів душею. 

Так, я приймаю це гасло, я бачив його, я радію йому, бо справді свобода є дуже
важлива для людини. Але сьогодні ми розуміємо, що свобода без відповідальності
може бути навіть небезпечна.

\ifcmt
pic https://img.pravda.com/images/doc/a/b/ab2ece5-mar3.jpg 
caption Мирослав Маринович: \zqq{Кожен народ має свої проблеми. Росія має проблему гордині, яка її заводить просто в прірву, а Україна має гріх вторинності --- вона ніяк не може повірити в себе}
\fi

\textbf{– У своєму інтерв’ю у 2010 році ви казали: \zqq{Вірю, що настане час, коли нам
вдасться досягти в стосунках з росіянами того, що ми вже маємо у спілкування з
поляками}. Чи не змінили ви свою думку після 14-го року?}

– Зрозуміло, що сьогодні спілкування з поляками теж обтяжене історією. Бачте,
коли поляки й українці мають відчуття спільного ворога, вони готові
порозумітися. 

Я пригадую табір, наші емоції на інформацію про те, що в Польщі зародилася
"Солідарність", і вона бореться. Це захоплення. Ніхто тоді не згадував про
якісь польсько-українські проблеми, усі були спрямовані на майбутнє.

І коли Чорновіл, ми разом з Михайлом Горинем приїжджали до Польщі, нас радо
зустрічали! І ніхто не згадував про минулі історичні проблеми, бо мали
спільного ворога. 

Пізніше змінилася ситуація, поляки відчули себе безпечними в Європейському
союзі, Польща піднялася економічно. Історія зразу спрацювала. І тоді українці,
які би мали слухатися велику Польщу, почали мати свою думку. І тоді запрацювала
історична ворожнеча. 

Ми маємо пам’ятати, що в нас був дуже тривалий період, коли і Польща, і Україна
вміли жити в дуже комфортних, правильних умовах. 

І нам потрібно думати, як все-таки бодай змоделювати якесь позитивне майбутнє
для українсько-російських стосунків. Адже ніхто нам не змінить цього сусіда. Ми
мусимо з ним жити. Ми не вибудуємо китайську стіну між нами і ними. 

Я дуже добре розумію, що сьогодні, особливо для тих, хто втратив своїх близьких
на війні з Росією, сама думка про примирення з Росією є дуже болісною. Я шаную
їхні емоції, їхні почуття. 

Але в Європі і нацистській Німеччині знаходилися люди, котрі в час Другої
світової війни думали про те, що буде після неї, і яким чином закладати основи
нормального співжиття. Вони про це думали так ефективно, що коли закінчилася
війна, скажімо, Франція й Німеччина, які були вічними ворогами, зуміли стати
нормальними партнерами. 

Не без проблем, бо ніхто не говорить про якусь ідеальну любов між націями. Але
вони вміють жити цивілізовано поруч. Так само і ми маємо пройти ту школу. 

Кожен народ має свої проблеми. Росія має проблему гордині, яка її заводить
просто в прірву, а Україна має гріх вторинності --- вона ніяк не може повірити в
себе. Вічно дивиться, у кого би взяти модель для себе --- а вона сама повинна
створити модель для себе. Мають з’явитися люди, має  запрацювати інтелект
нації, який є фантастичний. Ми маємо стільки розумних людей!

\textbf{– Скільки, на вашу думку, має минути часу перед тим, як українці і росіяни
почнуть комунікувати --- на рівні родичів, знайомих? Мають пройти покоління?}

– Передусім мусить відбутися зміна в самій Росії. 

Скажімо, налагодження французько-німецьких стосунків не почалося відразу. Так,
були люди, які формували цю модель, але зразу в перші роки після війни діяли
емоції. І я знаю випадки, коли священики зверталися в церквах із закликом
простити німців, їх ледь не виганяли з церков!

\ifcmt
pic https://img.pravda.com/images/doc/f/f/ffc358a-mar2.jpg
caption Мирослав Маринович: \zqq{Я не хотів би нового Майдану, який не дасть нам нової альтернативи, нових еліт. Я боюся такого Майдану, він може скінчитися недобре. Зараз багато зброї на руках у людей, і він може закінчитися криваво}
\fi

{\bfseries 
– У нас буде щось схоже? 
}

– Абсолютно, у нас буде те саме, бо це людські пристрасті. Але разом з тим
Німеччина і Франція чи інші держави зуміли це зробити. Чому зуміли? Бо
змінилася Німеччина передусім. 

Росія має пережити катарсис. На Facebook я вичитав одну думку, уже забув
автора: \zqq{Россия должна шмякнуться, и больно шмякнуться. И только тогда можно
будет думать о прощении}. Я не бажаю Росії катастрофи, але бачу, що інакше вона
не виїде зі своїх проблем. 

Гріх гордині працює у цього народу --- отже, його треба обломати. Я не хочу
ламати російську державність, хай вони собі живуть як народ щасливо. Але гріх
російської гордині має бути зламаний.

\textbf{– Ви жили в Україні глибоко радянській, потім в Україні за часів перебудови, в
Україні, як тільки вона отримала незалежність, і в Україні, яка вже пройшла
декілька революцій за час своєї незалежності. А в якій Україні ви б хотіли
жити?}

– Її ще немає. 

Я можу вам щиро признатися, що після виходу з табору в мене були дещо ідеальні
уявлення про те, якою буде Україна після здобуття незалежності. Мені чомусь
уявлялося в таборі, що ми дійшли вже до дна, що будь-який розвиток після того
буде поступовим підніманням. 

В якомусь плані це воно так і є. Але ми дуже боляче б’ємося об це дно. Я не
знаю, чи це дно, чи ми його не пробиваємо до наступного дна. Оці болісні зриви
дають відчуття, що тої України, у якій я би хотів жити, ще немає. 

Але, коли я подивлюся на Україну, яка була в радянський час, і теперішню --- я
маю можливість порівняти. Я бачу ті фантастичні зміни у суспільстві. І тому я
дякую Богу, що я є свідком цих подій, що я маю надію.

Так що я не втрачаю оптимізму.

{\bfseries 
– У найближчі роки Україна ще переживе якусь із революцій, Майданів? 
}

– Я не хотів би нового Майдану, який не дасть нам нової альтернативи, нових
еліт. Я боюся такого Майдану, він може скінчитися недобре. Зараз багато зброї
на руках у людей, і він може закінчитися криваво. 

Я вірю в піднесення українського духу --- але яким чином це станеться, на жаль,
не знаю.

{\bfseries 
Ольга Кириленко, УП
}
