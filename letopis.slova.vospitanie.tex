% vim: keymap=russian-jcukenwin
%%beginhead 
 
%%file slova.vospitanie
%%parent slova
 
%%url 
 
%%author 
%%author_id 
%%author_url 
 
%%tags 
%%title 
 
%%endhead 
\chapter{Воспитание}
\label{sec:slova.vospitanie}

%%%cit
%%%cit_head
%%%cit_pic
\ifcmt
  pic https://img.strana.ua/img/article/3413/milliony-na-ukrainskuju-28_main.jpeg
	width 0.4
	caption Кабмин откровенно нарушил Уголовный кодекс. Фото: Фейсбук/ KabminUA 
\fi
%%%cit_text
Кабмин принял программу \emph{перевоспитания} \enquote{неправильных} украинцев.
Буквально накануне прошла информация о том, что правительство не может найти
деньги, чтобы, как обещалось, с 1 июля повысить на 400 гривен пенсии украинцам
старше 75 лет, а уже вчера стало известно, что то же правительство собирается
потратить более 300 млн гривен из государственного и местных бюджетов на
утверждение \enquote{украинской гражданской идентичности}. На заседании Кабмина была
утверждена государственная программа национально-патриотического \emph{воспитания} в
Украине до 2025 года.  В чем же, по мнению членов правительства, состоит суть
\enquote{национально-патриотического} \emph{воспитания}? Ключевую роль в воспитании
патриотизма они отводят украинскому языку. Оказывается, тот факт, что половина
граждан страны говорит на русском, это – \enquote{постколониальные деструктивные
последствия в создании населения} и \enquote{языково-культурная неполноценность}.
Собственно, уже за одну только формулировку насчет неполноценности на
государство можно подавать в суд
%%%cit_comment
%%%cit_title
\citTitle{Миллионы на украинскую идентичность, зрада на Евро-2020, планы Путина по Украине}, 
, strana.ua, 01.07.2021
%%%endcit

