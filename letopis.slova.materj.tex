% vim: keymap=russian-jcukenwin
%%beginhead 
 
%%file slova.materj
%%parent slova
 
%%url 
 
%%author 
%%author_id 
%%author_url 
 
%%tags 
%%title 
 
%%endhead 
\chapter{Матерь}
\label{sec:slova.materj}

%%%cit
%%%cit_head
%%%cit_pic
%%%cit_text
Майор.
Очень долго думал, стоит об этом писать или нет. Потому что есть темы, где
удобней не топтаться. Но все же напишу, потому что это касается всех нас.
На днях на Донбассе погиб украинский военный. Майор Богдан. И так вышло, что
это сын одной из лучших подруг моей \emph{мамы} тети Оли. Она его всегда называла
Бодя. И все мое подростковое детство в Гайсине прошло рядом с Богданом. Не то,
чтобы мы с ним дружили, но дружили \emph{наши мамы}. А значит все дни рождения,
дежурные походы в гости – это все проходило у нас совместно.
Когда я закончил школу – я старше Богдана на 4 года – и уехал из Гайсина, с ним
мы практически больше не виделись. Помню, Богдан как-то даже стал заочной
причиной скандала между мной и \emph{мамой}. Уже студентом приехав на пару дней в
Гайсин, я увидел, что не хватает каких-то моих машинок и солдатиков.
Оказывается \emph{мама} дала их поиграть (с возвратом и он, действительно, потом их
вернул) Богдану. А я орал, что это моя частная собственность и вообще, какого
хера
%%%cit_comment
%%%cit_title
\citTitle{Наши ребята гибнут там, где у власти нет ни стратегической, ни тактической цели}, 
Игорь Лесев, strana.ua, 08.07.2021
%%%endcit

%%%cit
%%%cit_head
%%%cit_pic
\ifcmt
  pic https://strana.ua/img/forall/u/0/36/2021-07-27_12h58_59.png
  width 0.4
\fi
%%%cit_text
Причем он рассказал, что упрекал в русскоязычности и собственную \emph{мать}.
"Один раз в садике, где \emph{мама} разговаривала на русском языке, а наши друзья из
садика разговаривали на украинском, я сказал: "\emph{Мама}, как тебе не стыдно, ты
разговариваешь на русском, а с тобой разговаривают на украинском. И с тех пор я
начал \emph{маму} перевоспитывать на украинском. \emph{Мама} сначала тоже не понимала, почему
нельзя разговаривать на русском и любить Украину", – вспомнил Михаил
%%%cit_comment
%%%cit_title
\citTitle{Песни на русском – дерьмо. Как на канале Порошенко с ребенком о национализме говорили}, 
Анна Копытько, strana.ua, 27.07.2021
%%%endcit

%%%cit
%%%cit_head
%%%cit_pic
%%%cit_text
І я подумав: треба ввійти в свідомий контакт з нею з великою \emph{матір’ю} сущого. З
її дітьми — деревами, квітами, живою тканиною планети. Ця думка далека від
інтелектуальної гри. Це — прозріння в суть життя. Коли відкриваються нові очі,
ми бачимо, як життя спалахує там, де раніше нас панувала смерть, ми чуємо, як
поверхнею Ґеї-Землі і в підземних пластах течуть буйні ріки трепетного життя
глибин. Ми починаємо розуміти що покоління наших попередників — живих істот
котр нам здаються навіки загубленими в імлі минулого, — живуть у всеожошюсті
вічності, разом з тим входячи в плоть і кров, у душі сучасників, в дух
вічноплинного потоку єдиного життя, пронизують нас, вимагають зміни... Вони — це
ми...
%%%cit_comment
%%%cit_title
\citTitle{Вогнесміх}, Олесь Бердник
%%%endcit

%%%cit
%%%cit_head
%%%cit_pic
%%%cit_text
Пахнув дід теплою землею і трохи млином. Він був письменний по-церковному і в
неділю любив урочисто читати псалтир. Ні дід, ні ми не розуміли прочитаного, і
це завжди хвилювало нас, як дивна таємниця, що надавала прочитаному особливого,
небуденного смислу.  \emph{Мати} ненавиділа діда і вважала його за
чорнокнижника. Ми не вірили \emph{матері} і захищали діда од її нападів, бо
псалтир всередині був не чорний, а білий, а товста шкіряна палітурка —
коричнева, як гречаний мед чи стара халява. Зрештою, \emph{мати} крадькома таки
знищила псалтир. Вона спалила його в печі по одному листочку, боячись палити
зразу весь, щоб він часом не вибухнув і не розніс печі
%%%cit_comment
%%%cit_title
\citTitle{Зачарована Десна}, Олександр Довженко
%%%endcit

%%%cit
%%%cit_head
%%%cit_pic
%%%cit_text
Неадекватные в своих взглядах (но при этом весьма небедные) товарищи из числа
бывших депутатов БПП и им сочувствующих уже требуют отключить вечернее
освещение городов и начать веерные отключения. Все ради того, чтоб не идти на
прямые поставки газа из России и угля с Донбасса. По факту всем нормальным
людям ясно, что мы все эти годы используем российский газ (только через
коррупционные «реверсные» схемы) и донецкий уголь (только нелегально под видом
российского, который занимал 68\% рынка). Но этих упоротых волнует сама идея
того, что Киев возвращается к нормализации отношений с Москвой.  Вот уж точно,
на зло \emph{маме} эти отморозки хотят отморозить нам уши. Проще выкинуть их в
Канаду. Там не будет ни российского газа, ни угля. Хотя даже в США больше 10\%
нефти из РФ
%%%cit_comment
%%%cit_title
\citTitle{Упоротые товарищи ради своих идей готовы заморозить страну / Лента соцсетей / Страна}, 
Олег Волошин, strana.news, 31.10.2021
%%%endcit

%%%cit
%%%cit_head
%%%cit_pic
%%%cit_text
\emph{Матуся} оцього нашого гобіта... — до речі, а хто такий гобіт? Я гадаю, гобітів
сьогодні треба якось описати, адже вони стали рідкісними!] і почали триматись
осторонь від Великого Народу, як вони називають нас. Вони є (чи були)
маленькими чоловічками, десь до половини нашого зросту, і меншими, ніж бородаті
ґноми. У гобітів борід немає. У них також мало або й зовсім немає нічого
чарівного, за винятком звичайних буденних чарів, які допомагають їм зникати
безшумно і швидко, коли дурні здоровані, як-от ви чи я, немов слони, незграбно
сунуть у їхній бік, зчиняючи шум, який можна почути за милю. Гобіти схильні до
повноти; вони вдягаються в усе яскраве (переважно в зелене та жовте); не носять
взуття, бо від природи мають на ступнях товсту підошву та густе і тепле
брунатне хутро, подібне до волосся в них на головах (тільки не кучеряве); мають
довгі вправні брунатні пальці, добро-душні обличчя і заливаються глибоким
мелодійним сміхом (особливо після обіду, який вони залюбки заживають двічі на
день, коли вдасться). Тепер ви знаєте досить, отож можна розповідати далі. Як я
почав був говорити, \emph{матусею} цього гобіта — тобто Більбо Торбина — була
знаменита Беладонна Тук, одна з трьох славних доньок Старого Тука, старійшини
гобітів, котрі жили в Заріччі — по той бік маленької річечки, що протікала біля
підніжжя Пригірка
%%%cit_comment
%%%cit_title
\citTitle{Гобіт, або Туди і звідти}, Джон Толкін
%%%endcit
