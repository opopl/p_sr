% vim: keymap=russian-jcukenwin
%%beginhead 
 
%%file 08_01_2021.fb.zharkih_jurii.1.kapitolii
%%parent 08_01_2021
 
%%url https://www.facebook.com/yuriy.zharkikh/posts/3653042988096081
 
%%author 
%%author_id 
%%author_url 
 
%%tags 
%%title 
 
%%endhead 
\subsection{События в Капитолии}
\label{sec:08_01_2021.fb.zharkih_jurii.1.kapitolii}
\Purl{https://www.facebook.com/yuriy.zharkikh/posts/3653042988096081}
\ifcmt
  author_begin
   author_id zharkih_jurii
  author_end
\fi

Вот посмотрел вчера на события в Капитолии, а потом почитал международную
прессу. Впечатлило. Поэтому захотелось поделиться с друзьями в ФБ.

Внешне, события очень напоминали те, которые не раз были в Украине. В данном
случае, собралась масса недовольных людей, которые считают, что результаты
выборов были сфальсифицированы. Этим было попрано достоинство американцев.
Таким образом создались предпосылки для Революции Достоинства (среди активистов
были даже Proud Boys).

Ну а далее, по обычному для таких Революций сценарию: революционэры захватили
Капитолий и устроили там трам таррарам. Раньше, когда похожее происходило в
Украине, «западным партнерам» это очень нравилось. Всем.

Посол США вместе с Нулонд бродили по Майдану. И Нуланд кормила революцинэров
булочками, как зверюшек в зоопарке (правда и смотрела она на них, как на
зверюшек). 

И сенатор Маккейн выступал с трибуны так, что по Кличку бегали
мурашки. И послы «западных партнеров» в едином порыве высказывали горячую
поддержку.

А то, что произошло в Америке, «западным партнерам» не понравилось. Меркель,
Макрону, фон Дер Ляйн, Трюдо, Столленбергу и другим «партнерам» очень не
понравилось. Всем.

Да и то сказать. Посол России не бродил среди революционэров. Лавров не
раздавал хот-доги. Жириновский не выступал с трибуны так, чтобы с мурашками.
Послы «восточных партнеров» не высказывали горячую поддержку. Американская
общественность называла революционэров бандитами и погромщиками, «псами хаоса
сорвавшимися с цепи».

А какие они погромщики? Смешно. Наш Президент разочарованно отметил, что уже
через два часа уборщицы навели порядок и Конгресс продолжил работу. Вот после
наших революционэров, нужен капитальный ремонт или даже перестройка здания. У
них в США революционэров сейчас арестовывают по всей стране, а полицейских
награждают. У нас круче! Полицейских ищут по всей стране и содют в тюрьму, а
революцинэров содют в Верховну Раду (ну типа в награду). Что интересно, наш
Президент. у которого весь лоб в шишках от поклонов до пола революционэрам,
американских революционэров решительно осуждает.

В заключение приведу высказывание бывшего президента США Джорджа Буша-младшего:
«Вот так оспаривают результаты выборов в банановой республике, а не в нашей
демократической стране. Я потрясен безрассудным поведением некоторых
политических лидеров после выборов и отсутствием уважения, проявленного сегодня
к нашим институтам, нашим традициям и нашим правоохранительным органам». Ну к
нам это, конечно не относится. У нас же не растут бананы.
