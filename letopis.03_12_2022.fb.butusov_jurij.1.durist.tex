% vim: keymap=russian-jcukenwin
%%beginhead 
 
%%file 03_12_2022.fb.butusov_jurij.1.durist
%%parent 03_12_2022
 
%%url https://www.facebook.com/butusov.yuriy/posts/pfbid024wmDRvJsYRnuUkpQUQ5YgubwMmZpintv3MwRBitK34Lqk77LXPG9PCskZo8ZVWqzl
 
%%author_id butusov_jurij
%%date 
 
%%tags 
%%title Дурість - набагато більш небезпечний ворог добра, ніж зло
 
%%endhead 
 
\subsection{Дурість - набагато більш небезпечний ворог добра, ніж зло}
\label{sec:03_12_2022.fb.butusov_jurij.1.durist}
 
\Purl{https://www.facebook.com/butusov.yuriy/posts/pfbid024wmDRvJsYRnuUkpQUQ5YgubwMmZpintv3MwRBitK34Lqk77LXPG9PCskZo8ZVWqzl}
\ifcmt
 author_begin
   author_id butusov_jurij
 author_end
\fi

"Дурість - набагато більш небезпечний ворог добра, ніж зло. Проти дурості ми
беззахисні. Тут нічого не добитись ні протестами, ні силою; докази тут не
допомагають; фактам, що протирічать власному судженню, просто не вірять - в
таких випадках дурень перетворюється у критика, а якщо факти неспростовні, їх
відкидають як неважливу випадковість. До того дурень, на відміну від злодія,
абсолютно задоволений собою; і навіть стає небезпечним, коли у роздратуванні
переходить у напад. Ні в якому разі не можна намагатися переконати дурня
розумними доказами, це безнадійно та небезпечно. 

Чи можемо ми впортатись з дурістю? Дурість представляється більше
соціологічною, ніж психологічною проблемою. Це реакція особистості на вплив
історичних обставин, побочне психологічне явище у певній системі зовнішніх
стосунків. При уважному розгляді виявляється, що будь-яке посилення зовнішньої
влади (політичної або релігійної) уражає значну кількість людей дурістю. 

Влада одних вимагає дурості інших. Особистість, пригнічена видовищем
всемогутньої влади, позбавляється внутрішньої самостійності і відрікається від
пошуку власної позиції у створеній ситуації. Дурість часто супроводжується
впертістю, але це не повинно вводити в оману відносно  її несамостійності.
Спілкуючись з такою людиною, відчуваєш, що говориш не з ним, не з його
особистістю, а з опанувавшими ним гаслами та закликами. Ставши знаряддям,
дурень здатен на будь-яке зло, і разом з тим він вже не здатен розпізнати зло.  

В такій ситуації очевидна марність наших зусиль збагнути, і про що думає
"народ", і чому це питання зайве по відношенню до людей, які мислять та діють в
усвідомленні власної відповідальності.  

Думки про дурість дещо втішні: вони не дозволяють вважати більшість людей дурнями за будь-яких обставин. 

В дійсності все залежить від того, на що роблять ставку правителі - на людську
дурість чи на внутрішню самостійність та розум людей".

Дітріх Бонхеффер (1906-1945), євангельський пастор, герой антинацистського
спротиву у Німеччині, учасник заколоту проти Гітлера, цитата з листа "Спротив
та покора", написаного у тюрмі перед стратою.
