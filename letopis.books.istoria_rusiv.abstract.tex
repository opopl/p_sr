% vim: keymap=russian-jcukenwin
%%beginhead 
 
%%file books.istoria_rusiv.abstract
%%parent books.istoria_rusiv
 
%%url 
 
%%author_id 
%%date 
 
%%tags 
%%title 
 
%%endhead 

\begin{zznagolos}
\raggedcolumns
\begin{multicols}{2} % {
\setlength{\parindent}{0pt}

For a long time, the Little Russians have been desiring to see \enquote{The
History of Russians, or the History of Little Russia}, a work by Georgii
Koniskii, published.

There has been much discussion, time from time, that somebody - this one or
that one - was going to publish it, even that somebody was already printing it;
however, we still don't have it! Having at hands several copies of this
history, I have chosen the best one, then made comparisons between different
variants, and then proposed to the Imperial Society of History and Antiquities
of Russia to publish; so it is being published at the moment. Time from time, I
have an intention to do the same also with other literary sources of Little
Russia, chronicles, notes, descriptions, and so on, - for example, - works by
Shafonskii, Simanovskii and others. And so I would like to ask everyone, who
has the desire to see these and similar works published, to send me them, as
the Secretary of the Society, in order to make copies, and then to place them
in the \enquote{Readings}. A sincere gratitude and recognition by the
contempory historians, as well as our descendants, will be the best award for
those who will respond to this appeal.

The time has come; yes, the Time has come to stop concealing this Treasure, and
henceforth, to ease learning and perception of the History of Southern Russia
for everyone, and especially for the Russians themselves.  It is good that
there is a certain and prompt opportunity to fullfill this sacred
responsibility of a true son of its Nation and Motherland.

Osyp Bodyanskii, Moscow, February 9, 1846. 

Давно вже малоросіяни бажали бачити друкованою «Історію Русів, або Малої Росії,
твір Преосвященного ГЕОРГІЯ КОНИСЬКОГО».

Багато слів було час від часу, що той чи другий збираються видати її, навіть
друкують; але до сих пір Ті як нема, так нема! Маючи кілька списків цієї
історії, я вибрав ліпший з-поміж них, підвів до нього з інших різночитання, а
потім запропонував Імператорському Товариству Історії і Старожитностей
Російських видати його в світ, що і здійснюється нині. Час від часу я маю намір
те саме зробити і з іншими писемними джерелами Малої Росії, літописами,
записками, описами і т. ін., напр., з Шафонським, Симановським та інш.  А тому
я просив би всіх, хто тільки має і бажає бачити за короткий час ці та подібні
до них пам'ятники надрукованими, надсилати їх мені як секретареві Товариства
для зняття з них списків і негайного поміщення в «ЧТЕНІЯХЬ». Щира вдячність і
признання сучасників, які займаються історією, а також і самого потомства, буде
найкращою нагородою тим, хто відгукнеться на цей заклик. Пора вже, давно пора
не затаювати подібного скарбу і тим, оскільки можливо, полегшити вивчення і
пізнання історії Південних Русів для всіх і кожного, особливо ж самих Руських.
Добре, що є певна і скора нагода виконати цей священний обов'язок справжнього
сина свого народу і батьківщини.

О. Бодянський\footnote{%
Йо́сип (О́сип) Макси́мович Бодя́нський (псевд. О. Бода-Варвинець, Ісько
Материнка, І. Мастак та ін.) (31 жовтня (12 листопада) 1808, за іншими
даними — 3(15) листопада 1808, Варва, Лохвицький повіт, Полтавська
губернія, Російська імперія (нині Чернігівська область, Україна) — 6
(18) вересня 1877, Москва, Російська імперія) — український
філолог-славіст, історик, фольклорист, перекладач, видавець, письменник,
член-кореспондент Петербурзької Академії наук (1854) [Википедия].
}

Лютого 9, 1846, Москва

\end{multicols} % }
\end{zznagolos}

\ii{books.istoria_rusiv.abstract.pic.1}
