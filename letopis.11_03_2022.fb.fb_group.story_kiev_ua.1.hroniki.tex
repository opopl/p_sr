% vim: keymap=russian-jcukenwin
%%beginhead 
 
%%file 11_03_2022.fb.fb_group.story_kiev_ua.1.hroniki
%%parent 11_03_2022
 
%%url https://www.facebook.com/groups/story.kiev.ua/posts/1878949935635132
 
%%author_id fb_group.story_kiev_ua,stepanov_farid
%%date 
 
%%tags 
%%title ХРОНИКИ НЕОБЪЯВЛЕННОЙ ВОЙНЫ, Или: день двенадцатый. Уроки жизни
 
%%endhead 
 
\subsection{ХРОНИКИ НЕОБЪЯВЛЕННОЙ ВОЙНЫ, Или: день двенадцатый. Уроки жизни}
\label{sec:11_03_2022.fb.fb_group.story_kiev_ua.1.hroniki}
 
\Purl{https://www.facebook.com/groups/story.kiev.ua/posts/1878949935635132}
\ifcmt
 author_begin
   author_id fb_group.story_kiev_ua,stepanov_farid
 author_end
\fi

ХРОНИКИ НЕОБЪЯВЛЕННОЙ ВОЙНЫ

Или: день двенадцатый. Уроки жизни

09:00

\ii{11_03_2022.fb.fb_group.story_kiev_ua.1.hroniki.pic.1}

- Папа, нам Авраменко (автор учебника по украинскому языку) на своей странице
разрешил писать рф и россия с маленькой буквы. - дочка в восторге. - Скоро
внесут изменение в грамматику украинского языка.

- Отлично. На самом деле, он просто юридически закрепил то, что уже двенадцать
дней назад сделали все. Но все равно приятно.

12:30

- Не знаю, что будет с этим твоим ЗНО в этом году, но не хочешь ли поговорить
об Истории? - мы идём с дочкой в супермаркет за хлебом. - Тем более, что
История сейчас происходит на твоих глазах.

- Знаешь, папа, как то слишком много Истории в моей жизни стало в последнее
время! - чувство юмора у дочки от меня.

Знаете, когда изучаешь Историю по учебникам для сдачи ЗНО, это одно
удовольствие. Когда сама жизнь заставляет тебя погрузиться в Историю, совсем
другое. И не нужно никакой внешней мотивации.

Всю дорогу до супермаркета мы обсуждали с дочкой события последних двенадцати
дней. Я рассказывал ей о революции 1917 года: о том, как Ленин, на деньги
Германии, организовывал революцию в россии и почему это было выгодно Германии,
которая хотела, чтобы россия вышла из І мировой. И что у любого народного
восстания должен быть лидер. А есть ли сегодня такой лидер в рф, сомнительно.
Поэтому, не очень верю в их народную революцию. Скорее - в заговор приближенных
к ху...лу. Объяснял ей как так получилось, что на волне реваншизма, Адольф
Гитлер пришел к власти в послевоенной Германии. И почему именно в россии путина
идея \enquote{встать с колен} нашла такой отклик в умах и сердцах масс. Говорили о том,
почему в тоталитарном государстве власть присваивает себе монополию на правду.
И что происходит с теми, кто позволяет себе роскошь иметь другую точку зрения.

Я много провел уроков Истории за свою педагогическую жизнь. Но этот - был самым
волнительным. 

Больно было слышать комментарии дочки: \enquote{А помнишь, до войны...} Эти двенадцать
дней разделили наши жизни на: \enquote{до} и \enquote{после}.

13:50

Заходим в супермаркет. 

Полупустые полки, покупатели в поиске того, что ещё вчера было само собой
разумеющимся.

- Папа, это и есть дефицит как в СССР?

- Ну что ты, это - не дефицит. Даже в Киеве, в моем детстве, не было
сегодняшнего изобилия. 

18:15

- Вот смотрю, папа, как ты заклеил окна, и понимаю, почему ты в школе любил
геометрию. - сын с интересом рассматривает заклеенные окна на кухне.

- При чем тут геометрия? Просто смотрел фильмы о ІІ мировой. Только никогда не
думал, что в жизни пригодится.

- Ну как причем. Смотри: вот тут квадрат на пересечении лент, а здесь -
шестиконечная звезда. А это - логотип Mitsubishi.

- Эх ты, ученик. Во-первых, не квадрат, а ромб. Во-вторых, такую шестиконечную
звезду называют - Звездой Давида. 

- А логотип Mitsubishi?

- Сам не знаю, как так получилось.

Продолжение следует...
