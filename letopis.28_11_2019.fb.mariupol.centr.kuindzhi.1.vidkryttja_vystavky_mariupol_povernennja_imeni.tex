%%beginhead 
 
%%file 28_11_2019.fb.mariupol.centr.kuindzhi.1.vidkryttja_vystavky_mariupol_povernennja_imeni
%%parent 28_11_2019
 
%%url https://www.facebook.com/100023161160311/posts/pfbid02psE19VQvnac5bDT4psBR684jrywhBD5jpyQk5W5He8dHmc79cp3ewEQ8Y8qXUdjfl
 
%%author_id mariupol.centr.kuindzhi
%%date 28_11_2019
 
%%tags 
%%title Відкриття виставки "Маріуполь: повернення імені"
 
%%endhead 

\subsection{Виставка \enquote{Маріуполь: повернення імені}}
\label{sec:28_11_2019.fb.mariupol.centr.kuindzhi.1.vidkryttja_vystavky_mariupol_povernennja_imeni}

\Purl{https://www.facebook.com/100023161160311/posts/pfbid02psE19VQvnac5bDT4psBR684jrywhBD5jpyQk5W5He8dHmc79cp3ewEQ8Y8qXUdjfl}
\ifcmt
 author_begin
   author_id mariupol.centr.kuindzhi
 author_end
\fi

%\begin{minipage}{0.9\textwidth}
	
30 листопада 2019 р. о 14:00 у Центрі сучасного мистецтва і культури ім. Куїнджі
(пр. Металургів, 25) відбудеться урочисте відкриття виставки \enquote{Маріуполь:
повернення імені}.

Експозиція триватиме до 8 грудня 2019 р. Вхід вільний. 

На виставці будуть представлені унікальні фотографії, видання, картини та
документальний фільм 1989 року з приватних колекцій маріупольців та
Центрального державного кінофотофоноархіву України ім. Г. Пшеничного. Деякі з
них демонструються вперше. 

На відкриття запрошуються учасники та активісти політичних, громадських та
культурних рухів, представники мерії сьогодняшнього та тогочасного Маріуполя,
журналісти, всі, кому небайдужа історія рідного міста. 

Під час відкриття відбудеться нагородження школярів та вчителів істо\hyp{}рії,
переможців конкурсу \enquote{Повернення Маріуполю його імені - очима
ма\hyp{}ріупольців}. 

Виставка проводиться одночасно з виставкою робіт Олександра Бондаренка~~
\enquote{Абстрактна геометрія}, художника-авангардиста ~з~ об'єднання\par\noindent
\enquote{Маріуполь-87}.

Виставка проводиться в рамках проекту \enquote{30 років свободному Маріуполю},
який реалізується у співпраці з Центром громадянського суспільства
\enquote{Друкарня} та за підтримки Міністерства закордонних справ Федеративної
Республіки Німеччини. 

Партнери проекту: Центр А. Куїнджі, ЦДКФФА України ім. Г. Пшеничного,
Методично-консалтинговий центр при Департаменті освіти Ма\hyp{}ріупольської міської
ради.  

Спонсори призів для конкурсу \enquote{Повернення Маріуполю його імені - очима
маріупольців}: видавничий дім \enquote{Темпора}, книгарня \enquote{Є},
рекламно-інформаційне агентство Бориса Дембицького, фотостудія Лілії
Ровенської, Клуб любителів фотографії Маріуполя. 

Ласкаво просимо!

%\end{minipage}
