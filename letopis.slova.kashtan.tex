% vim: keymap=russian-jcukenwin
%%beginhead 
 
%%file slova.kashtan
%%parent slova
 
%%url 
 
%%author_id 
%%date 
 
%%tags 
%%title 
 
%%endhead 
\chapter{Каштан}
\label{sec:slova.kashtan}

%%%cit
%%%cit_head
%%%cit_pic
\ifcmt
  tab_begin cols=4
     pic https://img.pravda.com/images/doc/1/2/123eb93-7-zibrov-soviet-khreschatyk.jpg
     pic https://img.pravda.com/images/doc/3/6/36610d1-8-tort-kaschtan-morozyvo.jpg
		 pic https://img.pravda.com/images/doc/5/9/59a1355-8-2-magazin-kaschtanka.jpg
		 pic https://avatars.mds.yandex.net/i?id=9326c82c426242b427da8e383a3562e3-3821958-images-thumbs&n=13
  tab_end
\fi
%%%cit_text
Молодий, але вже з сивиною і фірмовими вусами – у 1994-му, під час зйомок кліпу
на пісню "Хрещатик", 37-річний Павло Зібров проїхався в ретро-кабріолеті
головною вулицею столиці.  На відео видно, що ще чверть століття тому Хрещатик
був набагато зеленішим. На початку 90-х значущість \emph{київських каштанів} як
символу закріпило керівництво підземки.  З 92-го по 94-й рік в метро їздили по
металевим жетонам з мідним покриттям і зображенням листа \emph{каштана}. Потім їх
стали випускати з пластика. Сьогодні раритетні металеві жетони можна купити в
інтернеті, заплативши за один до ста гривень.  У міському просторі зображення
\emph{каштанів} почали з'являтися ще в кінці 19-го – на початку 20-го століть, коли
архітектори та скульптори переносили його на фасади будинків.  У 1956-му
кулінари продемонстрували світові легендарний "Київський торт", на якому досі
красується соковитий п’ятилистник. Пізніше з'явилося одне з найбільш популярних
і дорогих видів морозива у СРСР – \emph{"Каштан"} за 28 копійок. Емблема хокейного
клубу "Сокіл" теж відобразила цей символ Києва.  – Моя пісня "Хрещатик"
починається з трьох слів \emph{"Каштани, каштани, каштани"}...  Звичайно ж, це
символ міста! – каже автор "Хрещатика" Юрій Рибчинський.  Коли майбутній
народний артист був маленьким хлопчиком, з усіх радіоприймачів лунав "Київський
вальс": "Знову цвітуть \emph{каштани}, хвиля Дніпровська б'є. Молодість мила –
ти щастя моє"
%%%cit_comment
%%%cit_title
\citTitle{"Туйовий Хрещатик". Чому кияни сумують за каштанами та як вони стали символом столиці}, 
Євген Руденко; Ельдар Сарахман, www.pravda.com.ua, 29.05.2020
%%%endcit
