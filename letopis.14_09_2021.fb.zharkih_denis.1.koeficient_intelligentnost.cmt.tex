% vim: keymap=russian-jcukenwin
%%beginhead 
 
%%file 14_09_2021.fb.zharkih_denis.1.koeficient_intelligentnost.cmt
%%parent 14_09_2021.fb.zharkih_denis.1.koeficient_intelligentnost
 
%%url 
 
%%author_id 
%%date 
 
%%tags 
%%title 
 
%%endhead 
\subsubsection{Коментарі}

\begin{itemize} % {
\iusr{Сергей Никонов}
Так и есть, но я бы назвал это коэффициентом мышления или интеллекта.

\iusr{Наталья Синец}

Жаль, Денис, что толкнуть некому... Жаль, что за "энное" количество лет, мы
превратились в "неваляшек". Раскачиваемся туда - сюда, пытаясь не упасть, а по
сути, многие упали уже ниже плинтуса. А тем, кто сжигает книги и млеет от
свастики, гораздо эффективнее сжечь свои свидетельства о рождении и свои
паспорта! Совершить этот акт милосердия, чтобы далее не противоречить своим
существованием теории старика Дарвина!

\begin{itemize} % {

% -------------------------------------
\ii{fbauth.staliv_aleksandr.lugansk.lnr.donbass.aristokrat.istina}
% -------------------------------------
 
\textbf{Наталья Синец} 

Небеспричинно межусобное насилие называют зверством. То есть, люди звереют - в
человечьем обличье превращаются в зверей, в хищников, безжалостно истребляющих
и изничтожающих себе внешне подобных.

Жизнь в потребительских отношениях не существует. В потребительских отношениях
существует выживание. Именно звериное хищническое выживание.

И современное мировое сообщество - крайне потребительское, в нём ничтожно мала
величина чувства и сознания меры - благоразумия и благородства, добродетели.

Отсюда и лжепатриоты. Потому что нет достаточной правовой (нормативной)
естественнонаучной определённости жизни. Нет космополитического, подсистемного
в Системе Вселенной естественнонаучного подхода.

У современников нет обоснованных научно, одинаковых, однообразных убеждений
(духовного стержня), без которых всё как в басне Крылова "Лебедь, рак и щука" -
каждый норовит по-своему. А оно, "по-своему", оно ничьё, потому что не в
соответствие (не пропорционально) состоянию и действию (механизму и механике)
среды обитания, то есть Вселенной, Системы Вселенной относительно Её
средоточия, неотъемлемой и неотделимой составной частью и принадлежностью
Которой все мы являемся.

Вот о чём речь. Мы крайне недоразвиты как люди. Мы человекоподобные животные,
звери, хищники. Мы духовно и нравственно (морально) не доросли ещё до людей. И
наши отношения во многом действительно (реально) звериные.

У меня есть простое жизненное правило (норма), госпожа Наталья. Оно на
картинке.

Из этого правила (нормы) само собою естественно необходимо и следует строить и
созидать своё собственное общение, свои собственные отношения, начиная именно с
самого себя. Это главное. Всё остальное само собою естественно приложится
благоприятно, благостно, жизненно.

И без знания и исполнения этого правила (нормы) - переворот (катастрофа) в
повседневном быту.

Что мы и имеем.

Жизненности Вам, Вашим родным и близким людям, госпожа.

Жизненности Украине, всему человечеству.

С особым почтением.

\iusr{Александр Ста́лив}
\textbf{Наталья Синец}  @igg{fbicon.hand.waving}  @igg{fbicon.face.smiling.eyes.smiling}  @igg{fbicon.heart.red}

\ifcmt
  ig https://scontent-lhr8-1.xx.fbcdn.net/v/t1.6435-9/242006107_4346556362105314_1751785582859918879_n.jpg?_nc_cat=100&ccb=1-5&_nc_sid=dbeb18&_nc_ohc=GZhtco-loKkAX81x25n&_nc_ht=scontent-lhr8-1.xx&oh=f5374b46caea27110c3b32e08cc6ebcd&oe=61673042
  @width 0.3
\fi

\iusr{Александр Ста́лив}
Об этом правиле (норме) речь в моём первом комментарии. С особым почтением.

\ifcmt
  ig https://scontent-frt3-2.xx.fbcdn.net/v/t1.6435-9/241980856_4346566048771012_1731706346067528025_n.jpg?_nc_cat=101&_nc_rgb565=1&ccb=1-5&_nc_sid=dbeb18&_nc_ohc=QNKzaGdhmKgAX__1aOo&_nc_ht=scontent-frt3-2.xx&oh=42fe837001eb635eb6548dda444c0d44&oe=6166538C
  @width 0.3
\fi

\iusr{Наталья Синец}
\textbf{Александр Ста́лив} 

Согласна, Александр! Даже нечего добавить! Преамбула в том, что каждый должен
начать именно с себя! Самообразование, саморазвитие, анализ своих жизненных
приоритетов, которые должны, хоть немного совпадать с общечеловеческими. Увы!
Соотношение потребление и созидание уже давно утратило свой баланс... И на
сегодня, невероятно сложно нормальным, адекватным людям, балансировать между
безумием, которое творится в стране и заложенным багажом ценностей.

\iusr{Александр Ста́лив}
\textbf{Наталья Синец} Берегите себя. Всё остальное само собою естественно прийдёт в равновесие (баланс).
С особым почтением.
1

\iusr{Наталья Синец}
\textbf{Александр Ста́лив} Спасибо, Александр!)

\end{itemize} % }

\iusr{Олег Бондарь}

Какая разница кем считают себя провокаторы - патриотизм - это собирание
разности твоего края. А сопливые дурачки уже понятно кто. То что они не
понимают - никого не волнует. Это уголовка террор - запугивание населения в
пользу властей отнють не патриотичных.  

\iusr{Александр Ста́лив}
Всё просто. Вот уравнение (формула).

\ifcmt
  ig https://scontent-frt3-1.xx.fbcdn.net/v/t1.6435-9/241982090_4346447235449560_5573131407000780552_n.jpg?_nc_cat=104&_nc_rgb565=1&ccb=1-5&_nc_sid=dbeb18&_nc_ohc=gugZg27OOkgAX-u51nn&_nc_ht=scontent-frt3-1.xx&oh=06607b2703dfffead94a888acdce0dd2&oe=6166A955
  @width 0.3
\fi

\end{itemize} % }
