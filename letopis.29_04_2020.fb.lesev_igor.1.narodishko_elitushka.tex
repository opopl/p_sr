% vim: keymap=russian-jcukenwin
%%beginhead 
 
%%file 29_04_2020.fb.lesev_igor.1.narodishko_elitushka
%%parent 29_04_2020
 
%%url https://www.facebook.com/permalink.php?story_fbid=3166504176713997&id=100000633379839
 
%%author_id lesev_igor
%%date 
 
%%tags elita,narod,obschestvo,politika,ukraina
%%title О грязном народишке и светлой элитушке
 
%%endhead 
 
\subsection{О грязном народишке и светлой элитушке}
\label{sec:29_04_2020.fb.lesev_igor.1.narodishko_elitushka}
 
\Purl{https://www.facebook.com/permalink.php?story_fbid=3166504176713997&id=100000633379839}
\ifcmt
 author_begin
   author_id lesev_igor
 author_end
\fi

О грязном народишке и светлой элитушке.

Вы ведь знаете эти истории о блокадном Ленинграде? Нет, нам еще не нужен
метроном, и в какой-то квартирке на Троещине новая Таня Савичева вряд ли пишет
свой дневник. И мы, естественно, сидим без своего «Невского пятачка». Хотя у
кого нет машины, блокада давно уже началась.

\ifcmt
  ig https://scontent-frt3-2.xx.fbcdn.net/v/t1.6435-9/95372843_3166504003380681_7088623868830547968_n.jpg?_nc_cat=101&ccb=1-5&_nc_sid=730e14&_nc_ohc=iNctowW0dd0AX-Mxk0j&_nc_ht=scontent-frt3-2.xx&oh=e29e7e3bb0ba26444fbf1f6f7a5f9c28&oe=61A5C4A4
  @width 0.4
  %@wrap \parpic[r]
  @wrap \InsertBoxR{0}
\fi

Но вот с чем у нас полный порядок, это со шкурами и скотством. Вот все как
тогда Там, так и сейчас Здесь. Свои упитанные подонки Ромашовы из «Двух
капитанов». Для них карантин – это только сокращение операционной прибыли. И
некоторые неудобства, когда нужно сторониться от плебса-обслуги. А вдруг эти
грязные нищеброды переносят заразу? Риски, конечно же, помереть минимальные –
все-таки вип-палаты и врачи с индивидуальными сиделками уже наготове. Но все
же, зачем эти две недели зазря температурить и нервничать, распаковывая
персональный ИВЛ из гаража?

А так, почти Ленинград. Вот тут наркомовское – «Рошен», «Эпицентр», «Велюр» и
извиняющаяся Мендель для внекарантинных наци. А вот тут уже народное –
пиздюлины коммерсам за бунт под Радой и по 17 штук штрафа обнаглевшим
косметологам и пловцам на длинные дистанции.

Нас даже беда не объединяет, потому что нет никаких «нас». Какая вообще может
быть беда у железнодорожника Сережи Лещенко, который за несколько месяцев
наскреб миллион гривасиков зп чистоганом? Или у Игорька Смелянского, который за
развал Укрпочты в месяц этот самый лям вытягивает из кармана своих же
подчиненных-почтальонш?

Или какие личные проблемы есть у штатного порохобота Кулебы, который по
совместительству рулит МИДом и препятствует выезду заробитчан на сельхозработы
в Европу? Для него «не логично» организовывать чартеры для вывоза украинцев на
работу. Зато для него логично было возвращать в страну людей по двойному тарифу
за их же счет. И логично не задуматься, а с хера ли люди готовы собирать ягоды
в Польше и Финляндии за гроши, а не сидеть в чудесной кисельно-кулебовской
Украине, где кругом лещенковское раздолье и смелянское благополучие.

Давайте прямо. Карантин не сгладил различия между нами, а оголил еще больше
этот водораздел двух Украин – чиновничье-беспредельной и народно-бесправной. И
люди Тогда ведь тоже умирали не только за наркомовские 100 грамм, и не за
Родину и Сталина. Как и сейчас фельдшера и кассиры в маркетах – вот уж
настоящие подвижники – рискуют всем не только за крошечные зарплаты,
позволяющие хоть как-то выжить. Есть работа, которую никто другой больше не
сделает. Потому что если не ты, то кто тогда?

И крысы – они тоже всегда есть. В тучные часы – жирные, с толстыми портфелями и
лощенными физиономиями, создающие ореол какого-то тайного знания. В суровые
времена мене наглые, менее заметные и более испуганные, забивающиеся в трюмы и
тихо попискивающие. А у нас крысы не только не затихарились. У них элит-перушка
на капитанском мостике. Образ жизни не изменился ни на йоту. С плебсом
объединяет только наличие намордника.

Говорят, мир после коронавируса изменится. Он уже изменился. Мировое
банкротство авиаперевозчиков и гражданского авиастроения. Банкротство
туристической отрасли. Банкротство общепитов. Назревающее банкротство кинопрома
с оттоком людей из кинотеатров. И очень сомнительные перспективы большого
футбола.

Но еще коронавирус станет большим испытанием для неэффективных государственных
систем, которые не умеют скрывать главный порок всей мировой элитарности –
несправедливости.

Мир – он всегда несправедлив. Были те, кто строили в пустыне пирамиды и те,
кому пришла в голову такая мысль. А еще были те, кто сумели заставить сотни
тысяч сильных людей мобилизовать на хрен пойми что. Но пирамиды – они хотя бы
на века. Понтовые и всегда бесполезные сооружения, заставляющие задуматься о
вечности. А «эпицентры» и «рошены» - это для брюшка отдельных особей, после
которых будет пустота.

Но сейчас ведь вопрос не о вечности. Сейчас в стране двух Украин вопрос сугубо
пролетарского характера. Почему одна кучка мерзавцев паразитирует за счет всех
остальных? И как долго эта очевидная для всех обитателей скотного двора
несправедливость будет продолжаться?
