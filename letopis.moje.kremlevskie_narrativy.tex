% vim: keymap=russian-jcukenwin
%%beginhead 
 
%%file moje.kremlevskie_narrativy
%%parent moje
 
%%url 
 
%%author_id 
%%date 
 
%%tags 
%%title 
 
%%endhead 

\subsection{Кремлевские нарративы}

Кремлевские нарративы. Здесь - кремлевский нарратив, там - кремлевский
нарратив. Ущемление прав русскоязычных - Кремль. Геноцид на Донбассе -
кремлевский нарратив. Обнищание населения и гражданская война - Кремль. Высокие
тарифы - Кремль снова. Бунтуют против вакцинации - снова Кремль. И так далее.

Слушайте. Ну надоело уже. Все время несется изо всех утюгов... Кремль, Кремль,
Кремль. Кремль сделал то, Кремль сделал это. Путин мечтает о том, Путин хочет
это. Скажите вот по-честному, Вы правда влюблены в Кремль, в Путина, или что? В
каких-то кремлевских рабов превратились на ментальном уровне. Ни дня не можете
прожить без Кремля и Путина. Путин - просто какой-то бог у вас, причем бог
противоречивый. С одной стороны, это кремлевский карлик, какой-то сумасшедший,
а с другой стороны, вы все его всегда внимательно слушаете и боитесь.  А как же
Святая София, Киев, Днепр, университет Шевченко и институт философии имени
Сковороды? Где ваши собственные Киевские нарративы? Где ваше Киевское слово,
твердое, уверенное и решительное, основанное на тысячелетней мудрости и истории
Стольного Града Киева? Разве ж Киев не старше Москвы, не мудрее, не умнее
Москвы?  Почему так много Москвы в информационном пространства и так удручающе
мало Киева? Везде Москва, Москва...  а Киев... Спрятался где-то в красивых
альбомах и фейсбук-группах по истории Киева... Как же так можно, елки-палки,
обидно же! Вы же каждый день ездите не в Московском метро, а в Киевском. В
красивейшем киевском метро вы едете от Арсенальной до Гидропарка, а может быть
- от Золотых Ворот до Выдубичей. Берете Киевское такси, например, Уклон или
Болт, если надо быстро куда-то доехать, а не московское такси - как оно
называется, извините, - мы не в курсе, наверное, надо позвонить в Кремль и
узнать у Пескова; кладете ребенка в Киевскую больницу, например, Охматдет, а не
в институт Склифосовского (или какие там еще есть больницы?). Почему так мало
гордости за Киев, за Столицу, почему такое безумное преклонение перед Москвой.
Киев же уже как тридцать лет имеет собственное политическое значение, времена,
когда весь центр принятия решений находился в Москве, давно уже в прошлом, так
почему же вы не пускаете в ход наш родной Киев? Слово Киева, ценности Киева,
мудрость Киева? 
