% vim: keymap=russian-jcukenwin
%%beginhead 
 
%%file 19_06_2021.fb.fb_group.story_kiev_ua.1.most_truhanov.cmt
%%parent 19_06_2021.fb.fb_group.story_kiev_ua.1.most_truhanov
 
%%url 
 
%%author_id 
%%date 
 
%%tags 
%%title 
 
%%endhead 
\subsubsection{Коментарі}

\begin{itemize} % {
\iusr{Андрей Надиевец}
Письма и рассказы написанные от сердца,не опаздывают никогда

\iusr{Ира Зеленина}
Петюня, спасибо за хорошие воспоминания! На Труханке прошла молодость! Встречи
и прогулки с друзьями и близкими! Как же было хорошо!

\begin{itemize} % {
\iusr{Петр Кузьменко}
Ира! Прямо таки, \enquote{молодость прошла}! Не прибедняйся! Ты, Ируся, сейчас
в самом соку. Как говорится, женщина на \enquote{Ять}!  @igg{fbicon.heart.sparkling} 

\iusr{Ира Зеленина}
\textbf{Петр Кузьменко} спасибо мой друг за комплеман !

\iusr{Ира Зеленина}
Когда будете в Киеве?
\end{itemize} % }

\iusr{Ира Зеленина}

\ifcmt
  ig https://i2.paste.pics/97c15226ff6d11e62c60c337de7d0b9c.png
  @width 0.2
\fi

\iusr{Gennady Henry Sergienko}
Очень тепло написано. Спасибо Вам.


\iusr{Tatiana Goncharova}

Спасибо за воспоминания! Как же, а про салют? На Пешеходном мосту мы всегда
смотрели салют и он был всегда самый красивый!

\begin{itemize} % {
\iusr{Петр Кузьменко}
\textbf{Tatiana Goncharova} несомненно!  @igg{fbicon.hands.shake} 

\iusr{Tatiana Goncharova}
\textbf{Петр Кузьменко} а вдоль набережной до моста самые красивые каштаны, помните? @igg{fbicon.heart.red}

\iusr{Петр Кузьменко}
\textbf{Tatiana Goncharova}, конечно! И про рыбалку там уже писал. Как всё было прекрасно в детстве и юности. Помним мы и наш Подол...

\iusr{Tatiana Goncharova}
\textbf{Петр Кузьменко} каштаны, 1977 год. Тогда наш Город был действительно зелёным:

\ifcmt
  ig https://scontent-frx5-1.xx.fbcdn.net/v/t39.30808-6/187835156_2714504198695767_4783256315477310135_n.jpg?_nc_cat=100&ccb=1-5&_nc_sid=dbeb18&_nc_ohc=inwibAiXxnkAX8_MQaT&_nc_ht=scontent-frx5-1.xx&oh=d43978e58068373ae826d9ef276fcc80&oe=61BD1766
  @width 0.7
\fi

\iusr{Раиса Карчевская}
Спасибо за прекрасные, добрые, теплые и трогательные воспоминания детства

\iusr{Ирина Архипович}
Просто прекрасно!! @igg{fbicon.heart.red}

\end{itemize} % }

\iusr{Світлана Проценко}

\ifcmt
  ig https://i2.paste.pics/799f011f678a46268b4fb024b87217dc.png
  @width 0.2
\fi

\iusr{Ирина Ещенко}
Петр, спасибо за приятные воспоминания детства !

\iusr{Олексій Вєровчук}
Душевно

\iusr{Елена Любкина}

\ifcmt
  ig https://i2.paste.pics/55b7cea60d03bc5e41a2a72d830f20d8.png
  @width 0.2
\fi

\iusr{Наталия Калатозишвили}
Душевные воспоминания спасибо

\iusr{Наталия Кебкал}
Спасибо за теплые воспоминания. Словно фильм посмотрела. Они Ваши личные, но близки многим...

\iusr{Konstantin Shamin}
Спасибо.

\iusr{Валентина Луконина}
Какие чудесные воспоминания! Этот мост, несомненно, в памяти любого киевлянина. Ну а для подолянина и подавно!
Как тепло и с какой любовью Вы, Пётр, вспоминаете своих родителей, обожаемую жену Нелли - и это не может не нравиться!
Спасибо Вам!

\iusr{Петр Кузьменко}
\textbf{Валентина Луконина} благодарю!  @igg{fbicon.hands.shake} 

\iusr{Оксана Писарчук}
Мы тоже сыном так ходили

\iusr{Ирина Иванченко}
Очень здОрово, респект!

\iusr{Валентина Андрусяк}
Благодарю за ваши чудесные воспоминания.

\iusr{Наталия Свирина}
І коли було багато людей міст трохи коливався...

\iusr{Людмила Аскерова}

С большим удовольствием прочитала ваши воспоминания и нахлынули воспоминания
детства с каким чувством шла я с мамой, а потом с подругами по мосту на пляж это
было настоящее путешествие сердце замирало от ощущения свободы.......

\iusr{Георгий Овсинский}

А у меня \enquote{тяжёлые} воспоминания... Пойдем с пацанами через мост, попросим
взрослых перевести, так как малых самих не пускали, возвращаемся домой и во
дворе всех встречают кого с ремнем, а кого и с резиновым шлангом. А меня не
встречали. Мама всегда на работе. Очень \enquote{завидовал}


\iusr{Natalia Volyanska}
Спасибо.

\iusr{Калерия Ходос}

Супер, спасибо людям которые делятся воспоминаниями о старом Подоле который был
прекрасен со своими маленькими домами и калорит неповторимого Подола. Мы жили
тоже на Подоле только в районе Житнего рынка на ул. Мирная это было просто
незабываемая жизнь.

\iusr{лилия шакалова}

А я ещё помню времена, когда не было этого моста. Мы жили на Подоле, и каждое
воскресенье точно также с молодой картошкой с маслом и укропом в банке и в
такой же банке - клубника в сахаре отправлялись по Спасской улице на Спасские
причал . Там мы покупали билеты и садились на маленький катер, который
переправлял нас на Подольский пляж. Все то же самое: потрясающе вкусная и
желанная еда, крики мамы, чтобы вылезла из воды, игры с песком... А вот вечером
нужно было отстоять длиннющую очередь на последний или предпоследний катер,
отправляющийся в Киев. Помню, на этом катере нужно было стоять в очень плотной
многолюдной толпе между взрослыми - ничего не было видно вокруг и помню мой
страх, что катер сейчас утонет. Вот так было ещё ДО появления Паркового моста


\iusr{Георгий Овсинский}
\textbf{лилия шакалова} Этот маленький катер имел гордое имя - ЛАПОТЬ!

\iusr{Алла Лукьянченко}
Пешеходный мост - это дорога в детство!

\iusr{Наталя Кудря}
\textbf{Alla Lukyanchenko} романтично і правдиво...

\iusr{Владислав Городецький}
\textbf{Петр Кузьменко}
Написано тепло, но вот только раки боком не ходят.

\begin{itemize} % {
\iusr{Петр Кузьменко}
\textbf{Владислав Городецький} мне маленькому казалось, что они передвигались именно так. @igg{fbicon.laugh.rolling.floor} 

\iusr{Валентина Горковенко-Спицына}
\textbf{Владислав Городецький} Почему не ходят? В случае опасности (и др.) раки передвигаются боком.
\end{itemize} % }

\iusr{Ира Руденкова}

\ifcmt
  ig https://scontent-frx5-2.xx.fbcdn.net/v/t39.1997-6/s168x128/47270791_937342239796388_4222599360510164992_n.png?_nc_cat=1&ccb=1-5&_nc_sid=ac3552&_nc_ohc=vYZtHV62k4wAX-vidwc&_nc_ht=scontent-frx5-2.xx&oh=00_AT-v0yqgJI1DCEXuyjpI1F8te-dLlJdVPCe5kcpTwbkCfg&oe=61BD1176
  @width 0.1
\fi

\iusr{Люба Потемкина}
СПАСИБО, П. КУЗЬМЕНКО ЗА ВАШИ ПРЕКАСНЫЕ ОПИСАНИЯ КИЕВСКИХ ПРЕКРАС И ТЕПЛЫЕ СЛОВА В АДРЕС РОДИТЕЛЕЙ

\iusr{Наталья Голубенко}
Чудесно!

\ifcmt
  ig https://scontent-frx5-2.xx.fbcdn.net/v/t39.1997-6/p480x480/72232259_2363266403926026_5287256169037955072_n.png?_nc_cat=1&ccb=1-5&_nc_sid=0572db&_nc_ohc=5MinV-PjcPYAX-WH7_A&_nc_ht=scontent-frx5-2.xx&oh=00_AT_zTNJLIZMBmdt4wr3NLfbWajhRa96yHGEqjQQ2sWIcWA&oe=61BC2F70
  @width 0.2
\fi

\iusr{Marina Ganopolska}

У меня на Андреевском спуске 2 жили бабушка и дедушка. Кажется, на 6ом этаже. В
огромной коммунальной квартире

\begin{itemize} % {
\iusr{Георгий Овсинский}
\textbf{Marina Ganopolska} 

А у меня на Андреевском спуске 2 жили бабушка и дядя. Дядю могли видеть,
летчик, полковник и к тому же красавец.


\iusr{Marina Ganopolska}
К сожалению мне было 7лет когда умер дедушка и 10 когда не стало бабушки. Я никого не помню

\iusr{Георгий Овсинский}

Там ещё жили два брата по фамилии Норис. Уехали в Америку и там от них родился
народный артист Америки Чак Норис!  Сам то я родился в 15.
\end{itemize} % }

\iusr{Валентина Блащук}
Спасибо, пишите, очень интересно!
Будьте здоровы!

\iusr{Марина Тоцкая}
Благодарю за прекрасную экскурсию в детство

\iusr{Галина Гурьева}

Лилия Шакалова, и называли этот катер \enquote{лаптем}. Часто на пляж переправлялись на
нем. Можно было даже сидеть на горячих лавочках)) А вот обратно только через
мост! По вышеуказанным Вами причинам)))

\begin{itemize} % {
\iusr{лилия шакалова}
\textbf{Галина Гурьева} да, именно лапоть. Но моста тогда ещё не было - в моем детстве. Поэтому : очередь, «лапоть» и битком...0

\iusr{Polina Glinkin}
\textbf{лилия шакалова} и в моем тоже. Одно из первых воспоминаний моего детства мне было не больше 3 лет и мы по воскресеньям ездили на пляж имнно на лапте. А когда построили этот пешеходный мост, то все ходили только через мост.
\end{itemize} % }

\iusr{Светлана Степурко}
Огромное спасибо за то что окунули в детство.

\iusr{Петр Кваша}
Чудесно, сам Подолянин, жил на Жданова, детство и юность 50-60х, эх.

\iusr{Валентина Бахмацкая}

\ifcmt
  ig https://scontent-frx5-2.xx.fbcdn.net/v/t39.1997-6/s168x128/93118771_222645645734606_1705715084438798336_n.png?_nc_cat=1&ccb=1-5&_nc_sid=ac3552&_nc_ohc=lc93vluqWdkAX_TmSQY&tn=lCYVFeHcTIAFcAzi&_nc_ht=scontent-frx5-2.xx&oh=cc148aa91f5f83fe85867eae6a27a0a0&oe=61BCDCBF
  @width 0.1
\fi

\iusr{Валентина Нагорна}

\ifcmt
  ig https://scontent-frx5-2.xx.fbcdn.net/v/t39.1997-6/s168x128/93118771_222645645734606_1705715084438798336_n.png?_nc_cat=1&ccb=1-5&_nc_sid=ac3552&_nc_ohc=lc93vluqWdkAX_TmSQY&tn=lCYVFeHcTIAFcAzi&_nc_ht=scontent-frx5-2.xx&oh=cc148aa91f5f83fe85867eae6a27a0a0&oe=61BCDCBF
  @width 0.1
\fi

\iusr{Elena Veremenko}

Спасибо за рассказ! Вернулась не надолго в детство. Через пешеходный мост мы
юные пловцы ходили летом на станцию Водник на Трухановом острове. Замечательное
было время!

\iusr{Анатолий Золотушкин}

Замечательно, тепло написано. Прошёлся вместе с вами по старинному Подолу, без
которого \enquote{Киев не возможен}. И по Труханому, где прошло спортивное детство.

\iusr{лилия шакалова}
\textbf{Анатолий Золотушкин} спасибо

\iusr{Ирина Овсиенко}

В детстве всегда самые яркие впечатления, которые помнятся всю жизнь !!! Чем
дальше от того времени, тем острее ностальгия о прошлом, о своих родителях,
друзьях, о событиях того времени и т. д.

Всё окутывается романтическим ореолом, а случавшиеся неприятности стираются из
памяти, если и не совсем, то тоже вспоминаются с налетом грусти. Удивительное
качество человеческого организма - это память !!! В любой момент можем
оказаться в нужном месте прошлого, хоть и виртуально !!!

\begin{itemize} % {
\iusr{лилия шакалова}
\textbf{Ирина Овсиенко} 

вы знаете, у меня нет ностальгии, я умею жить «здесь и сейчас». Моя жизнь
сейчас также интересна, как и в прошлые годы. Я не окрашиваю прошлое в
романтические тона - очереди за мукой, бидончики с молоком, мои болезни,
болезни брата, когда одна ампула пенициллина стоила 800 рублей, а ему нужно из
было колоть каждые 4 часа... нет, я знаю и помню многое - и эти воспоминания -
просто воспоминания-констатация. Я люблю каждый день своей жизни : и в прошлом.
И в настоящем...

\begin{itemize} % {
\iusr{Ирина Овсиенко}
\textbf{лилия шакалова} 

Если Вы думаете, что я сижу у окошка, подперев лицо рукой, утираю слёзы о
событиях полувековой давности, когда у меня было детство и юность, то Вы
глубоко ошибаетесь. Вы не поняли душевного настроя ни публикации, ни моего
комментария. Моя семья жила в той же стране, что и Вы. И, поверьте, манна с
небес нам не сыпалась па голову. Было много трудностей, слез и горя. Но были и
есть радостные события. Вот о них и надо помнить. Жить я тоже умею \enquote{здесь и
сейчас}, и не живу прошлым. У каждого своя жизнь, свои воспоминания, которые
только его и не нуждаются в осуждении чужих людей.


\iusr{Elena Sivolapova}
\textbf{лилия шакалова} медицина была платная? Или дефицит всегда оплачивался отдельно?

\iusr{Elena Sivolapova}
\textbf{лилия шакалова} помню джинсовую юбку купила с кнопками, стоила больше зарплаты

\iusr{лилия шакалова}
\textbf{Elena Sivolapova} медицина была бесплатной

\iusr{лилия шакалова}
\textbf{Elena Sivolapova} Ооо, джинсы моей жизни появились намного позже

\iusr{лилия шакалова}
\textbf{Ирина Овсиенко} я рада за Вас. Любви Вам и удачи!

\iusr{Инна Белинская}

Человек пишет о своих детских воспоминаниях!!! При чем здесь медицина, джинсы и
\enquote{здесь и сейчас}? .Привет всему новомодному(ничего личного). Давайте радоваться
за других, за то что у них есть такие теплые воспоминания...


\iusr{Ирина Овсиенко}
\textbf{Инна Белинская} Спасибо за понимание!!!
\end{itemize} % }

\end{itemize} % }

\iusr{Светлана Дубински}
Благодарность,за такие похожие воспоминания детстваи славного Подола.

\iusr{Светлана Дубински}

\ifcmt
  ig https://scontent-frx5-2.xx.fbcdn.net/v/t39.1997-6/s168x128/93118771_222645645734606_1705715084438798336_n.png?_nc_cat=1&ccb=1-5&_nc_sid=ac3552&_nc_ohc=lc93vluqWdkAX_TmSQY&tn=lCYVFeHcTIAFcAzi&_nc_ht=scontent-frx5-2.xx&oh=cc148aa91f5f83fe85867eae6a27a0a0&oe=61BCDCBF
  @width 0.1
\fi

\iusr{Алла Дубровина}

Спасибо, чудесный рассказ! Он и про мои детские воспоминания. Наверное, для
нас, Подолян, этот мост значит немного больше, чем для других. У меня там
произошли первые самостоятельные шаги в жизни, на радость моим юным родителям.
Спасибо за рассказ и воспоминания.


\iusr{Павел Гурнік}
чудесно написано





\end{itemize} % }
