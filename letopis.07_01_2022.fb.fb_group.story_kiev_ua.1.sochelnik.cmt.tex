% vim: keymap=russian-jcukenwin
%%beginhead 
 
%%file 07_01_2022.fb.fb_group.story_kiev_ua.1.sochelnik.cmt
%%parent 07_01_2022.fb.fb_group.story_kiev_ua.1.sochelnik
 
%%url 
 
%%author_id 
%%date 
 
%%tags 
%%title 
 
%%endhead 
\zzSecCmt

\begin{itemize} % {
\iusr{Alena Torun}
Какой великолепный рассказ, спасибо огромное Автору

\iusr{Tetiana Desiatka}

\ifcmt
  ig https://scontent-frt3-1.xx.fbcdn.net/v/t39.30808-6/s851x315/271505257_1092432144909588_4542926611417152763_n.jpg?_nc_cat=106&ccb=1-5&_nc_sid=dbeb18&_nc_ohc=dICQ3J61x5cAX-AGDHa&tn=lCYVFeHcTIAFcAzi&_nc_ht=scontent-frt3-1.xx&oh=00_AT-BVE6HnUTwJY_XrPVxLMk0yMwoNoHHPo0xClpFxV6H2A&oe=61E160B0
  @width 0.2
\fi

\iusr{Елена Сидоренко}
Спасибо большое! С Рождеством! Мира, здоровья, благополучия! @igg{fbicon.heart.beating} 

\iusr{Надежда Владимир Федько}

Дуже тепла і інформаційно насичена розповідь!
Дякую!

\iusr{Николай Бурчин}
Отличный рассказ, ждем теперь описание Рождественских обедов)

\iusr{Ирина Заяц}

\ifcmt
  ig https://scontent-frx5-1.xx.fbcdn.net/v/t39.30808-6/271598915_1038536930344942_4531674162903906649_n.jpg?_nc_cat=105&ccb=1-5&_nc_sid=dbeb18&_nc_ohc=Vc66Nvc18BgAX_vuDWv&_nc_ht=scontent-frx5-1.xx&oh=00_AT9K4eQGmdhcqi84LPqFL-qwWcth9Nz9voj4C3y2aAfdAw&oe=61E316A5
  @width 0.2
\fi

\iusr{Муся Баранка}
Дякую! Дуже приємні різдвяні відчуття! Зі святом, Христос народився!

\iusr{Ирина Нищимная}

\ifcmt
  ig https://i2.paste.pics/b22aca2de3a326aeaf27f3f06403812e.png
  @width 0.2
\fi

\iusr{Татьяна Червоношапка}

\ifcmt
  ig https://scontent-frx5-1.xx.fbcdn.net/v/t39.30808-6/271476873_2017153198462493_1337612253975891597_n.jpg?_nc_cat=111&ccb=1-5&_nc_sid=dbeb18&_nc_ohc=w0y_zweb94AAX8i43ai&_nc_ht=scontent-frx5-1.xx&oh=00_AT_npPOsIVZpKCufB44TuxRmyPfEUClj-rZn4oR8m9Babw&oe=61E30DBD
  @width 0.2
\fi

\iusr{Irina Bentaleb}

\ifcmt
  ig https://scontent-frt3-1.xx.fbcdn.net/v/t39.30808-6/271497229_10228108774687787_3187081683695667912_n.jpg?_nc_cat=106&ccb=1-5&_nc_sid=dbeb18&_nc_ohc=Lv33ddR0E2kAX8GXfRP&_nc_ht=scontent-frt3-1.xx&oh=00_AT9XuqF1dAh9DtUwJbqWpZB2GVGXzdxsb3prP5dqc2cfJQ&oe=61E1FDB4
  @width 0.3
\fi

\iusr{Lyubov Pakholchenko}
Какое это счастье - иметь такую семью.

\iusr{Валентина Бабченко}
И счастье сохранить память и следовать традициям.

\iusr{Antonina Chepiga}
Душевно очень...

\iusr{Лариса Іванова}

\ifcmt
  ig https://scontent-frt3-1.xx.fbcdn.net/v/t39.30808-6/271280961_3110849789187176_4770204827039188203_n.jpg?_nc_cat=108&ccb=1-5&_nc_sid=dbeb18&_nc_ohc=kYC3BsdzgNkAX_hQqb5&_nc_ht=scontent-frt3-1.xx&oh=00_AT-9SeUpxZ3n2dfiRinhTtxX1dwd_5JttNmilpZcLcIIdA&oe=61E1A58D
  @width 0.2
\fi

\iusr{Людмила Власенко}

Маленькое уточнение: богатая кутя - это на Маланки. Отсюда и \enquote{щедрівки}, щедрий
вечір, потому что рождественский пост закончился.

А 6 января ещё последний день рождественского поста. На столе должны быть
постные блюда. Потому и кутя не может быть богатой. Не ели в этот вечер ни
мяса, ни колбас, ни сала

\begin{itemize} % {
\iusr{Nadiya Nadiya}
\textbf{Людмила Власенко} Багата кутя готувалася на Святвечір. Вона так називається, бо готувалося 12 пісних страв. Багато страв - багата кутя. На Маланки, у Щедрий вечір, готується Щедра кутя, бо обов'язково на столі, окрім куті ставилися м'ясо, ковбаси, шинка та інші страви. Автор все написав правильно.
\end{itemize} % }

\iusr{Nata Eremenko}
Багато прикмет даже й не чула.

\iusr{Юлия Николаевна}
\textbf{Nata Eremenko} большая часть из них язычество.

\iusr{Анна Сидоренко}
А мы, как правило безбожники, всё едим.

\iusr{Nadiya Nadiya}
Дуже душевна сімейна історія. Збереження традицій світлого свята в родині. Веселих Різдвяних свят. Спасибі автору.

\iusr{Раиса Карчевская}
Очень интересная семейная история. Счастливого Рождества
Большое спасибо за пост

\iusr{АННА Киевлянка}
Прямо кулинарная книга и

\iusr{Юлия Николаевна}
Интересный рассказ, спасибо, что поделились, единственное, что расстраивает,
что много языческих традиций соблюдали.

\begin{itemize} % {
\iusr{Вика Нурибекова}
\textbf{Юлия Николаевна} так в Киеве они тесно переплетаются.

\begin{itemize} % {
\iusr{Юлия Николаевна}
\textbf{Вика Нурибекова} да, к сожалению, со времен крещения не можем от этого избавиться.

\iusr{Ольга Писанко}
\textbf{Юлия Николаевна} а стоит ли ?

\iusr{Ольга Писанко}
\textbf{Юлия Николаевна} все «трапезные элементы» в христианстве- это приспособленные языческие

\iusr{Ольга Писанко}
\textbf{Юлия Николаевна} начиная от писанок- крашенок и заканчивая дидухом!

\iusr{Юлия Николаевна}
\textbf{Ольга Писанко} нет, это не так, вы плохо знанте христианство, а яйца крашеные это как раз христианство, по легенде у Марии Магдалины яйцо стало красным в руках.

\iusr{Юлия Николаевна}
\textbf{Ольга Писанко} конечно, христианство и язычество несовместимые вещи, т.к. служат противоположным сущностям, христиане Богу, а язычники противикам Бога.

\iusr{Ольга Писанко}
\textbf{Юлия Николаевна} и где это написано? В Новом завете? А как же з писанками?

\iusr{Ольга Писанко}
\textbf{Юлия Николаевна} ну почему же язычники противники Бога? Просто христианство- это монотеизм. Хотя есть множество святых, покровительствующих тем или иным сферам

\iusr{Inna Dm-s}
\textbf{Ольга Писанко} не путайте святых, реальным людям, верующих в Христа, с бесами-богами. Это грех. И Рождество- это рождение в мир Христа. При чём здесь Лада и другие? Это как или крестик снимите, или... Язычество и христианство не совместимы.
\end{itemize} % }

\iusr{Ольга Писанко}
А дидуха первого кто сплёл?

\begin{itemize} % {
\iusr{Юлия Николаевна}
\textbf{Ольга Писанко} Таким чином, підсумовуючи все вище сказане, можемо заявити, що дідух — це не просто українська різдвяна прикраса, це елемент української культури, який походить ще з дохристиянських часів, а тому — дуже язичницький. - из интернета., А какой язычник первым сплел не знаю.

\iusr{Ольга Писанко}
\textbf{Юлия Николаевна} а тут по фамилии и не нужно знать, имя ему- пращур!

\iusr{Inna Dm-s}
\textbf{Ольга Писанко} це ви так думаєте.
\end{itemize} % }

\iusr{Юлия Николаевна}
\textbf{Ольга Писанко} есть писание, а есть предание, церковь признает и то и другое, вы не в теме.

\iusr{Юлия Николаевна}
\textbf{Ольга Писанко} вот именно, что христиане поклоняются единому Богу, а язычники многим, т.е. бесам, поэтому это очень изживалось и считается очень большим грехом против 1 заповеди, вы наверное не слышали о языческом боге молохе, которому киевляне приносили в жертву младенцев при язычестве?

\begin{itemize} % {
\iusr{Ольга Писанко}
\textbf{Юлия Николаевна} да церкви просто некуда было деваться, пришлось предания и обряды вплетать в свою канву! Давно известный всем факт!

\iusr{Юлия Николаевна}
\textbf{Ольга Писанко} вы вообще не разбираетесь ни в теме язычества, не в теме веры, уверена, что и в церковь не ходите, и смеете что-то утверждать как истину, удивляюсь таким людям!

\iusr{Ольга Писанко}
\textbf{Юлия Николаевна} да неужели? Я тоже очень удивлена неприятием Вами очевидных и давно известных истин! Но не собираюсь ни в чем Вас переубеждать! Просто имею право, как и Вы высказать своё мнение. Как- то так!

\iusr{Юлия Николаевна}
\textbf{Ольга Писанко} к сожалению, таких людей как вы большинство, и пасхи ходят святить и перед бесами скачут на ивана купала, им все равно, а первые христиане мученики жизнь отдавали, не соглашаясь поклониться языческим идолам, им даже было достаточно, бросить немного углей на жертвенник, чтоб остаться в живых, но они предпочитали верность своему Богу.

\iusr{Ольга Писанко}
\textbf{Юлия Николаевна} никогда не скакала на Ивана Купала и не собираюсь! Просто к вере отношусь вдумчиво...

\iusr{Юлия Николаевна}
\textbf{Ольга Писанко} я образно говоря написала.

\end{itemize} % }

\end{itemize} % }

\iusr{Ольга Писанко}

А пост действительно замечательный, очень тёплый и душевный!

\iusr{Victoria Novikov}

Так чудесно написано, так красочно ! Как-будто я фильм посмотрела!
@igg{fbicon.hands.applause.yellow}{repeat=3} 

\iusr{Inna Dm-s}

6 свчня - Святвечір, 13 січня - багата кутя або щедра/ Василя або старий новий
рік, де 12 страв по кількості місяців, 18 січня - голодна кутя. Дитячі спогади
іноді бувають неточними. І ладу точно не могли кликати і сонцю поклонятись. Це
вже підміна. Бо це були конкретні дати прив’язані до певних подій, як
народження Христа. не зочу буркотіти, але якщо знали про піст, то знали і про
інше.

\end{itemize} % }
