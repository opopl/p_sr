% vim: keymap=russian-jcukenwin
%%beginhead 
 
%%file 28_11_2021.fb.skarshevskij_viktor.1.prezident_snishozhdenie
%%parent 28_11_2021
 
%%url https://www.facebook.com/Viktor.Skarshevsky/posts/2087127434774701
 
%%author_id skarshevskij_viktor
%%date 
 
%%tags press_konferencia,ukraina,zelenskii_vladimir
%%title К президенту долго относились снисходительно
 
%%endhead 
 
\subsection{К президенту долго относились снисходительно}
\label{sec:28_11_2021.fb.skarshevskij_viktor.1.prezident_snishozhdenie}
 
\Purl{https://www.facebook.com/Viktor.Skarshevsky/posts/2087127434774701}
\ifcmt
 author_begin
   author_id skarshevskij_viktor
 author_end
\fi

К президенту долго относились снисходительно, прощали ошибки и
непрофессионализм, поскольку верили в его искренность. 

Но Владимир Зеленский, будучи погруженным своим окружением в теплую
информационную ванну, не заметил, что эти времена прошли. Иначе он не рисковал
бы выходить к прессе, будучи не готовым отвечать на острые вопросы. 

Зеленский рассчитывал побить прессу апломбом и актерским мастерством. Но в
итоге все произошло все с точностью до наоборот. 

Зрители увидели косноязычного, напуганного человека, зачастую говорящего
неправду, который даже хамит журналистам и которому те не стесняются хамить в
ответ. Никогда еще в истории страны с президентом публично так не
разговаривали.

Чего стоит только объявленная президентом информация о планируемом
госперевороте на следующей неделе (1-2 декабря). 

Тот, кто принес Зеленскому эту теорию про заговор, откровенно подставил
президента. 

Мало того, что инвесторы и так обходящие Украину десятой дорогой, после слов
президента о возможном госперевороте и подавно в сторону Украины не будут
смотреть, но и сам вброс о госперевороте, может использоваться для прикрытия
внутренних корпоративных разборок. 

Например, один из потенциальных участников подготовки «госпереворота» Василий
Грицак (по информации журналиста BuzzFeed Кристофера Миллера,
\url{https://bit.ly/3cXy9fS}), является бизнес-партнером беглого бизнесмена Юрия
Сидоренко, у которых есть спорные активы ликвидированного ЕДАПС – бывший
монополист на производство бланков паспортов на полиграфкомбинате "Украина".
По информации СМИ, Сидоренко является партнером Давида Арахамии. В чью пользу
теперь разрешится корпоративный конфликт после объявления потенциального
«госпереворота» – можно додумать и самим.

Не нужно было президенту проводить этот пресс-марафон. К прошлым скандалам,
которые он хотел смягчить своей пресс-конференцией, добавились (например,
перепалки с журналистами) и могут добавиться и другие, не менее резонансные.

\ii{28_11_2021.fb.skarshevskij_viktor.1.prezident_snishozhdenie.cmt}
