% vim: keymap=russian-jcukenwin
%%beginhead 
 
%%file slova.pobeda
%%parent slova
 
%%url 
 
%%author 
%%author_id 
%%author_url 
 
%%tags 
%%title 
 
%%endhead 
\chapter{Победа}
\label{sec:slova.pobeda}

%%%cit
%%%cit_pic
%%%cit_text
Борьба за \emph{победу} действительно была острой. И противник - сборная Нидерландов -
на последних минутах забил гол и сорвал уже казалось бы железобетонную ничью.
В итоге Украина потерпела поражение, но можно убедительно сказать, что почетное
- со счетом 3:2. Чему, кстати, порадовались даже в России - показав, что многие
болельщики не повелись на попытку раздуть политическое противостояния вокруг
чемпионата. В Украине тоже многие хвалили сборную за волю к \emph{победе}. Но
политику в обсуждения внес внезапно сам Владимир Зеленский, который зачем-то
вспомнил о скандале с формой украинской сборной, поздравляя ее с игрой.
Интересно, что за украинскую сборную болели и многие россияне. В том числе и
комментаторы на госканалах. И это резко контрастирует с тем, как в Украине
некоторые восприняли недавнее поражение российской сборной от Бельгии.
\enquote{Страна} публикует отзывы на первый матч Украины в еврочемпионате
%%%cit_comment
%%%cit_title
\citTitle{Украина - Нидерланды на Евро 2020. Отзывы на матч в Украине и России}, 
Оксана Малахова, strana.ua, 14.06.2021
%%%endcit

%%%cit
%%%cit_pic
%%%cit_text
Взять хотя бы последнюю историю с формой для игры в мяч. Какая была \emph{победа},
когда УЕФА по недопониманию утвердила её. Какие словесные гопаки выплясывали
хлопцы и девчата в появившейся возможности потроллить соседа. И тут, бац, —
убрать немедленно! Зрада? Ещё бы. Вой неистовый в сторону продажного УЕФА не
слышат разве что на далёких квазарах. Но они упёртые. Где-нибудь в укромном
месте намалюют что хотят, потому что нация, потому что независимые, потому что
треба. И снова будет \emph{перемога}. А потом опять будет вой. Цикличность
%%%cit_comment
%%%cit_title
\citTitle{От \enquote{псеродактиля} в \enquote{андертальца} и далее в венец человечества: рассуждения никудышного антрополога}, 
Дмитрий Жук, zen.yandex.ru, 13.06.2021 
%%%endcit

%%%cit
%%%cit_head
%%%cit_pic
%%%cit_text
Будьмо готові!  Ми хочемо \emph{перемогти}, ми можемо \emph{перемогти}, ми
\emph{переможемо}!  P.S. Наявність чи відсутність підборів на парадних туфлях
українських жінок-військовослужбовців ніяк не впливає на боєготовність ЗСУ... Про
інше треба думати та говорити.  Врешті, інакше треба діяти перед дуже реальною
загрозою
%%%cit_comment
%%%cit_title
\citTitle{Постала загроза широкомасштабного вторгнення московської орди з РФ і Білорусі}, 
Дмитро Ярош, gazeta.ua, 02.07.2021
%%%endcit

%%%cit
%%%cit_head
%%%cit_pic
%%%cit_text
Враг при этом должен быть \emph{победим}. Не имеет смысла воевать с более сильным
врагом. Если нельзя победить врага непосредственно, можно пытаться победить
более слабых союзников врага.  Враг должен иметь важные для создающего врага
ресурсы, которые можно отобрать.  Это так или иначе рационализирует не просто
противостояние с врагом, но рационализирует саму войну на уничтожение врага и
присвоение его ресурсов.  Принцип симметрии враждебности: тот, кого вы считаете
врагом, должен считать врагом вас. А вот если ваш враг не хочет считать вас
врагом и вообще вас игнорирует, тогда как? Враг, который не считает вас врагом,
враг, подрывающий ваше достоинство.  Врага должно быть не жалко, то есть еще до
вражды враг должен быть унижен идеологически, морально, и психологически.
Создание враждебности — это отдельная большая работа. Враждебность должна стать
способом жизни и подготовки к войне
%%%cit_comment
%%%cit_title
\citTitle{Враг}, Сергей Дацюк, analytics.hvylya.net, 18.11.2021
%%%endcit
