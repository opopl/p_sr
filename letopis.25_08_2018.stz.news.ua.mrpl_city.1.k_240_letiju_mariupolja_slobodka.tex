% vim: keymap=russian-jcukenwin
%%beginhead 
 
%%file 25_08_2018.stz.news.ua.mrpl_city.1.k_240_letiju_mariupolja_slobodka
%%parent 25_08_2018
 
%%url https://mrpl.city/blogs/view/k-240-letiyu-mariupolya-slobodka
 
%%author_id burov_sergij.mariupol,news.ua.mrpl_city
%%date 
 
%%tags 
%%title К 240-летию Мариуполя: Слободка
 
%%endhead 
 
\subsection{К 240-летию Мариуполя: Слободка}
\label{sec:25_08_2018.stz.news.ua.mrpl_city.1.k_240_letiju_mariupolja_slobodka}
 
\Purl{https://mrpl.city/blogs/view/k-240-letiyu-mariupolya-slobodka}
\ifcmt
 author_begin
   author_id burov_sergij.mariupol,news.ua.mrpl_city
 author_end
\fi

Тихая улочка. Только трамвай

С грохотом вылетит из-за поворота.

Вожатый трезвонит: \enquote{Эй, не зевай!}

Прохожий отмахивается: \enquote{Чего там!..}

Цветут акации. Осядет пыль,

Которая лениво проплыла за трамваем.

А еще иногда такое бывает:

В улочку заглянет автомобиль.

\emph{Даниил Чкония}

Историк конца XIX столетия пишет: \emph{\enquote{Часть города между ним и морем известна ныне
под именем Слободки. Начало ее относят к 40-м годам XIX века.
Генерал-губернатор Воронцов позволил селиться на берегу моря отставным солдатам
и рабочим. Такое поселение горожане терпели потому, что слобожане были
работниками на гавани, и спохватились только тогда, когда образовался почти
город в четыре длинные улицы. Поднят был вопрос о выселении, но 30 марта 1859
года состоялся указ, в силу которого в Мариуполе дозволено причисляться не
грекам к обществу отдельному от греческого...}}

\ii{25_08_2018.stz.news.ua.mrpl_city.1.k_240_letiju_mariupolja_slobodka.pic.1}

Слободка построена бывалыми солдатами и моряками, отважными рыбаками,
умельцами-ремесленниками, рисковыми торговцами. Здесь умудрялись за ночь
построить хатку-землянку, накрыть ее камышовой крышей, сложить простейшую печь
и к утру разжечь ее. И ничего не могли сделать городские власти с
\enquote{самозастройщиками} - по царским законам, не позволено было разрушать
отапливаемые, а, следовательно, обитаемые жилища. Старожилы рассказывали: мол,
потому-то и невысоки старинные дома в этом славном районе нашего города.
Правда, люди, наделенные практическим складом ума, считают: слободские строения
изначально были выше, а за многие десятилетия осели по причине слабости здешних
грунтов, подтапливаемых периодически морской водой. Может, так оно и есть?

\textbf{Читайте также:} 

\href{https://mrpl.city/blogs/view/vasil-altuhov}{%
До 240-річчя Маріуполя: Василь Алтухов, Ольга Демідко, mrpl.city, 20.08.2018}

Четыре продольные улицы поселения \enquote{между городом и морем} еще до революции были
\enquote{пронумерованы}: та, что ближе к нагорной части Мариуполя, называлась \textbf{Первой
Слободкой}, а следующие за ней, соответственно, \textbf{Второй, Третьей и Четвертой}. В
советское время Первая Слободка получила имя героя гражданской войны \textbf{Григория
Котовского}, Вторая - председателя ВЦИК РСФСР \textbf{Якова Свердлова}, Третья стала
\textbf{Донецкой}, Четвертая - \textbf{Транспортной}. Последняя - теперь именуется улицей
Линника. Михаил Васильевич Линник - бывший инструктор физкультуры
Мариупольского порта, был удостоен за подвиги в годы Второй мировой войны
звания Героя Советского Союза. В ходе декоммунизации улицам Котовского и
Свердлова вернули первородные имена, а Вторая - Донецкая и Четвертая - Линника
сохранили названия, приобретенные в советское время. Теперь, если на Слободку
забредет какой-нибудь досужий приезжий, у него будет повод поразмышлять, куда
делись Первая и Третья Слободки.

\ii{25_08_2018.stz.news.ua.mrpl_city.1.k_240_letiju_mariupolja_slobodka.pic.2}

В октябре 1882 года было завершено строительство участка железнодорожной дороги
от Еленовки до Мариуполя, на той окраине поселка, что ближе к морю, соорудили
здание вокзала. С появлением в нашем городе железной дороги многие обитатели
Слободки пошли на нее работать. Кто устроился путевым обходчиком, кто -
стрелочником, кто - рабочим в депо. Некоторые выбились в паровозные машинисты,
то есть обрели должность, особо почитаемую в те далекие уже времена. 8 декабря
1905 года рабочие и служащие станции Мариуполь, среди которых было немало
потомков свободолюбивых моряков и рыбаков, основателей Слободки, присоединились
к Всероссийской стачке железнодорожников.

8 октября 1941 года наш город был захвачен гитлеровцами. В годы оккупации
железнодорожная станция многократно подвергалась налетам авиации. Пилоты
целились в воинские составы, депо, водонапорную башню. От слободских старожилов
довелось слышать воспоминания, как советский самолет, прорвавшийся сквозь
заградительный огонь немецких зенитчиков, сбросил бомбу на вражеский состав
цистерн с горючим. Цистерны горели, взрывались, пламя пожара было видно из
центра города. Но иногда, к сожалению, бомбы и снаряды попадали на дома
слободских жителей, при этом гибли люди. В сентябре 1943 года оккупанты,
покидая наш город, сожгли вокзал, взорвали пристанционные сооружения, сорвали
железнодорожные пути. Привокзальная площадь носит имя мичмана Евгения Павлова,
отважного морского десантника, неустрашимого разведчика, после войны ставшего
наставником мариупольской ребятни, в том числе и слободской. Он учил их грести
на шлюпках, управлять парусами, \enquote{семафорить} флажками, плавать под водой в
акваланге, а еще - мужеству, честности, верности слову и делу...

\textbf{Читайте также:} 

\href{https://mrpl.city/news/view/sovetskij-chelovek-ili-grazhdanin-ukrainy-kem-identifitsiruyut-sebya-zhiteli-donbassa}{%
\enquote{Советский человек} или гражданин Украины? Кем идентифицируют себя жители Донбасса, Іоанна Вишневська, mrpl.city, 21.08.2018}

Еще до революции была построена железнодорожная школа. В ней учились дети
работников Екатерининской железной дороги, теперь она называется Донецкой, со
всех станций и полустанков от Мариуполя до Волновахи. А расположенное рядом
Пушкинское начальное училище предназначалось для чад прочих обитателей
Слободки. После гражданской войны железнодорожная школа получила номер \enquote{30}, и
в ней было разрешено учиться всем детям из близлежащих улиц. Школа эта
славилась своими педагогами. Здание Пушкинского училища позже переоборудовали
под клуб, который сгорел во время войны.

На месте нынешней школы №37 возвышалась некогда Константино-Еленинская церковь.
В народе ее называли слободской. В экспозиции мариупольский Храм построили по
проекту Виктора Александровича Нильсена незадолго до первой мировой войны, а
зимой 1936 года взорвали... без всякого проекта... На месте церкви  возвели школу.
Здесь стоит упомянуть ее первого директора. Это Иван Спиридонович Журавлев. К
сожалению, его учеников остались единицы. Не прошло и трех лет, как в учебных
классах был развернут госпиталь для раненых и обмороженных бойцов Красной
Армии, участников советско-финской войны. Сюда же стали поступать командиры и
красноармейцы, получившие увечья на фронтах начавшейся Отечественной войны...
Как-то в газетах появилось сообщение, что школа-ветеран находится под угрозой
разрушения. На тревожный сигнал откликнулись предприниматели - бывшие ученики
\enquote{тридцать седьмой}, и не словом, а делом: выделили средства для ремонта
\enquote{альма-матер}. Из выпускников школы №37 стоит назвать Александра Павловича
Дубнова, видного ученого из Сибирского отделения Академии наук Российской
Федерации, экономиста и политолога, доктора экономических наук, профессора,
автора многочисленных научных трудов.

На Слободке некогда было несколько пекарен, но самую знаменитую держал Иван
Яковлевич Марапульцев, в народе известный, как Марапулец. Дом на углу
Вокзальной улицы и Первой Слободки, где она была, сохранился до сих пор. Сюда
ходили за пахучим хлебом, ароматными сдобными ватрушками даже из центральной
части города. В двадцатые-трид\hyp{}цатые годы слободское население лечил легендарный
фельдшер Александр Шеин. В любую погоду он отправлялся на вызов к больному, и
его появление в доме с зонтом и потертым докторским саквояжем было первым шагом
к исцелению больного. Его лекарские способности признавали авторитетные
мариупольские медицинские светила с дипломами столичных университетов, а уж о
пациентах и говорить не приходится.

В ту же пору, когда пользовал хворых и страждущих фельдшер Шейн, в доме на
улице Вокзальной, что и сейчас стоит под самым обрывом, в конце 20-х годов
обитал паровозный машинист Кашуба - городская знаменитость: никто лучше его не
играл на баритоне и басе в местном духовом оркестре. Значительно позже в этот
же дом вселился с семьей известный мариупольский врач, хирург, кандидат
медицинских наук Петр Деонисьевич Топалов. В сентябре 1962 года Петр
Деонисьевич выполнил уникальную операцию по приживлению отрезанных циркулярной
пилой пальцев руки рабочего-плотника. Конечно, это была не единственная
операция такого рода. Но вершиной его, его как хирурга, было решение проблемы
контрактуры Дюпюитрен – тяжелого заболевания, при котором скрюченные пальцы рук
причиняют постоянную боль.

\textbf{Читайте также:} 

\href{https://mrpl.city/news/view/mariupoltsam-iz-chastnogo-sektora-hotyat-vydelit-pomoshh-na-obogrev-zhilya-foto}{%
Мариупольцам из частного сектора хотят выделить помощь на обогрев жилья, Ярослав Герасименко, mrpl.city, 18.08.2018}

О слободских сорвиголовах ходили легенды, но они вырастали, становились
моряками и воинами, инженерами и сталеварами, врачами и художниками. Здесь
прошли детство и юность легендарного генерала Латышева. Георгий Александрович
начал свой боевой путь еще в гражданскую войну, незаслуженно подвергся
репрессиям, сражался с гитлеровцами, а когда вышел в отставку, вернулся в
родной город, выступал с лекциями и докладами, встречался с молодежью. Стоит
вспомнить и известного украинского кинооператора и режиссера Алексея
Александровича Мишурина. И для него Слободка - малая родина. В свое время
снятые им фильмы \enquote{В дальнем плавании}, \enquote{Максимка},
\enquote{Годы молодые}, \enquote{Спасите наши души}, \enquote{Королева
бензоколонки} и другие пользовались большим успехом у зрителей. И еще один
уроженец старинного мариупольского поселения - Евгений Николаевич Митько. По
его сценариям сняты киноленты \enquote{Бурьян}, \enquote{Там, вдали за рекой},
\enquote{Цыган}, \enquote{Дударики}...

Слободка, Слободка, пронизанная солеными морскими ветрами, запахами рыбы и
угля, грохотом проходящих поездов, легендами и бывальщинами...
