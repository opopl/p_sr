% vim: keymap=russian-jcukenwin
%%beginhead 
 
%%file 20_11_2020.fb.manya_naumchuk.1.juvilej_u_mamy
%%parent 20_11_2020
 
%%url https://www.facebook.com/manya.naumchuk/posts/791285631729660
%%author 
%%author_id 
%%tags 
%%title 
 
%%endhead 

\subsection{Ювiлей у Мами}
\Purl{https://www.facebook.com/manya.naumchuk/posts/791285631729660}
\Pauthor{Наумчук, Таня}
\index[writers.rus]{Наумчук, Таня}

Цей вiрш я хочу посв'ятити своiй мамi, якiй 21 листопада виповнилося б 80
рокiв.  Я знаю, вона спостерiгаэ за нашим життям з небес, та просить Бога для
нас допомоги.

\ifcmt
pic https://scontent-waw1-1.xx.fbcdn.net/v/t1.0-9/126461724_791285605062996_8507254734915773456_o.jpg?_nc_cat=102&ccb=2&_nc_sid=8bfeb9&_nc_ohc=Nfvu4VZbl2QAX-nX2uA&_nc_ht=scontent-waw1-1.xx&oh=5e596df8f36afea3ad2464a483b98f3e&oe=5FDCE670
caption Ювiлей у Мами - Таня Наумчук
\fi

Сьогоднi ювiлей у мами,
Булоб вiсiмдесят ii.
Але немаэ мами з нами,
Лиш тiльки спогади моi.

Що року ми забирались в тiснiм крузi,
Смаколики були вже на столi.
Рiднi та близькi, вiрнi друзi
Вiтання дарували iй своi.

Мама усix любила пригостити,
I xвилювалась, щоб було все до ладу.
Вона завжди умiла всiх любили
I вiдкликалась на чужу бiду.

Не можу я вже маму привiтати,
I не скажу, як я ii люблю.
Буду в молитвах Господа прохати,
Щоб поселив ii в своiм раю.

20.11.2020.г.
