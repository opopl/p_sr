% vim: keymap=russian-jcukenwin
%%beginhead 
 
%%file 02_09_2017.stz.news.ua.mrpl_city.1.istoria_morozhenoe
%%parent 02_09_2017
 
%%url https://mrpl.city/blogs/view/morozhenoe
 
%%author_id burov_sergij.mariupol,news.ua.mrpl_city
%%date 
 
%%tags 
%%title История: Мороженое
 
%%endhead 
 
\subsection{История: Мороженое}
\label{sec:02_09_2017.stz.news.ua.mrpl_city.1.istoria_morozhenoe}
 
\Purl{https://mrpl.city/blogs/view/morozhenoe}
\ifcmt
 author_begin
   author_id burov_sergij.mariupol,news.ua.mrpl_city
 author_end
\fi

Война окончилась совсем недавно. Хлеб и скудный ассортимент других продуктов в
Мариуполе, как и во всей стране, распределяются строго по карточкам. В городе
еще много \enquote{погорелок}. Но Торговая улица - от Николаевской и до старого
пивзавода - почти не пострадала от пожара. Немцы успели сжечь лишь двухэтажное
здание почтового отделения на углу улицы Митрополитской и среднюю школу № 3 —
трехэтажное здание, в котором до революции находилось частное реальное училище
Гиацинтова. Оконные и дверные проемы \enquote{погорелок} заложены кирпичом,
добытым тут же. Ребятам непостижимым способом удается проникать внутрь
\enquote{погорелок}. Одно из развлечений мальчишек — бегать по балкам на уровне
второго, а то и третьего этажа, рискуя каждую секунду сорваться на груды
обломков стен. И ведь же срывались.

Разумеется, у детворы есть и другие развлечения: у мальчишек — самодельные
самокаты, игра в лапту и футбол тряпичным мячом, \enquote{казаки-разбойники}, стрельба
из рогаток по воробьям, походы на пляж. У девчонок – потрепанные куклы,
сохранившиеся с довоенных времен, изготовление \enquote{духов} из цветов акации, игра в
\enquote{кремушки}, впрочем, самые бойкие из них участвуют и в мальчишечьих забавах.
Редко выпадающие лакомства: прежде всего, семечки подсолнечника или тыквы,
поштучно продающиеся ириски из патоки, такой же поштучный товар — карамель
\enquote{подушечки}, леденцы \enquote{петушки на палочке}. 

Но первейшее из лакомств, несомненно, мороженое. На нечетной стороне Торговой,
там, где к ней примыкают сразу две улицы – Фонтанная  и Малофонтанная, стоит
тележка — это фанерный ящик, покрытый уже кое-где облупившейся голубой краской,
к нему приторочены два колеса. Хозяйка тележки — женщина средних лет. На ней
белоснежный фартук и такие же нарукавники. Волосы, туго зачесанные, прикрыты
головным убором, изготовленным из самодельных простеньких кружев, туго
накрахмаленных. Время от времени, но не так уж часто, к тележке подбегает
кто-нибудь из местной босоногой ребятни. У каждого в плотно сжатом, вспотевшем
от напряжения кулачке - зеленого цвета \enquote{трешка} с изображением бравого
красноармейца с полной выкладкой и в каске, а иногда и синяя \enquote{пятерка} с
летчиком. 

Кулачок разжимается, и обладательница белого фартука и нарукавников берет
измятую купюру, разглаживает ее тщательно и приобщает к пачке уже заработанных
денег. Теперь она открывает крышку ящика, в недрах которого стоят бачки,
погруженные в крошево изо льда, левой рукой берет свой главный инструмент —
жестяную формочку с поршеньком, правой вкладывает в нее круглую вафельку,
ложкой, смоченной водой, со стенок бачка соскребается белая масса мороженого.
Ловкими и быстрыми движениями формочка заполняется мороженым, сверху
закрывается еще одной вафелькой. Еще одно движение — и круглый столбик
выдавливается поршеньком из формочки. Все это время маленький покупатель или
покупательница неотрывно следят за волнительным для них действом.

Наконец-то порция мороженого попадает в руки счастливца или счастливицы. Чтобы
подольше насладиться сладостью, мороженое слизывается языком. Вафельки
постепенно сближаются, язык немеет от холода. Вся эта процедура происходит под
пристальными взглядами тех, кому не удалось выпросить деньги у мамы или
бабушки. У кого-то не выдерживают нервы:

- Дай лизнуть. - На, только слегка. - И мне. - Обойдешься! - Ладно, ладно.
Завтра ты попросишь...

Вафельки сошлись. Они полностью размокли. Осталось их отправить в рот. И
постараться жевать как можно дольше, растягивая удовольствие... 

Прошли годы. Появилось в Мариуполе фруктовое мороженое и мороженое в
стаканчиках вафельных и из плотного полукартона, к которым прилагалась тонкая
деревянная пластинка, с ее помощью доставалось содержимое стаканчика. Стали
обыденностью пломбир и эскимо — облитый шоколадом и обернутый станиолью
(фольга. - Прим.) продолговатый комок сладкой холодной массы. Кажется, всех
сортов, разновидностей не счесть. Было время, когда на проспекте Республики,
ниже кинотеатра \enquote{Победа}, открылось заведение с  вывеской: \enquote{Мороженое}. Там
сладко-холодный продукт подавался в специальных чашках на подставках с
небольшой ложечкой в придачу. Горка мороженого была обязательно полита сиропом
или вареньем, иногда украшением служила ягодка клубники. Здесь же можно было
заказать ситро, минеральную воду, даже шампанское. Да, это было... 

Но вот что интересно: тем, кому довелось попробовать мороженое из голубой
тележки, кажется, что именно оно было самым вкусным, самым сладким и самым
незабываемым. 
