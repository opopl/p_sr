% vim: keymap=russian-jcukenwin
%%beginhead 
 
%%file 09_10_2017.stz.news.ua.mrpl_city.1.serce_kulturnogo_zhyttja_mariupolja
%%parent 09_10_2017
 
%%url https://mrpl.city/blogs/view/sertse-kulturnogo-zhittya-mariupolya
 
%%author_id demidko_olga.mariupol,news.ua.mrpl_city
%%date 
 
%%tags 
%%title Серце культурного життя Маріуполя
 
%%endhead 
 
\subsection{Серце культурного життя Маріуполя}
\label{sec:09_10_2017.stz.news.ua.mrpl_city.1.serce_kulturnogo_zhyttja_mariupolja}
 
\Purl{https://mrpl.city/blogs/view/sertse-kulturnogo-zhittya-mariupolya}
\ifcmt
 author_begin
   author_id demidko_olga.mariupol,news.ua.mrpl_city
 author_end
\fi

Є в Маріуполі місце, яке знає кожний мешканець міста. Часто саме тут
призначають побачення, проводять фестивалі, концерти. Сюди ж маріупольці
приходять в найбільш вишуканому одязі, щоб насолодитися театральним мистецтвом
завдяки виставам Донецького академічного обласного драматичного театру
(Маріуполь) чи приїжджим театральним колективам. Приміщення драматичного
театру завжди було прикрасою міста. У будь-яку пору року його полюбляють
фотографувати з різних ракурсів, писати з нього картини і навіть присвячувати
йому вірші. Будівля давно прикрашає більшість маріупольських листівок та
календарів. Не дивно, що приміщення драматичного театру є єдиним, що отримало
статус пам'ятки архітектури державного значення і одним з 6 пам'яток
архітектури місцевого значення.

Проте історія будівництва є непростою, адже приміщення театру в радянський час
будували лише в обласних центрах. Хоча недивно, що саме Маріуполь став
виключенням. Мабуть, все ж таки у маріупольців є свій менталітет, самобутній і
неповторний.

Наприкінці 1949 року в Маріуполі припинив діяльність професійний
музично-драматичний театр, що було пов'язано з браком коштів, відсутністю
приміщення та горезвісною постановою ЦК КП(б)У від 1946 р. \enquote{Про репертуар
драматичних і оперних театрів УРСР і заходи до його поліпшення}, у якій
піддавалася критиці діяльність українських театрів. Водночас Маріуполь
відносився до необласного міста, тому не мав права на існування постійної
театральної трупи. Отже, театр закрили.

Однак, маріупольців не полишала думка мати професійний театр. Колишній секретар
міськради Г. Лихачова розповідала: \enquote{У тому, що в нашому місті почали будувати
театр, велика заслуга самих маріупольців. Були клопотання, були скарги, були
заяви різні. Іноді навіть у дуже високі інстанції. Незважаючи на безліч
проблем, пов'язаних з промисловими підприємствами, будівництвом, комунальним
господарством, керівники міста \enquote{захворіли} разом зі своїми співгромадянами
ідеєю відродження театру}. Завдяки підтримці Донецька та колосальній роботі
Костянтина Олейниченка, який на той час очолював міськом КПУ, все ж таки
вдалося обійти директиву, згідно з якою театральне приміщення могли побудувати
лише в обласному місті. В усіх клопотаннях мова йшла про створення обласного
драматичного театру, а вже потім зазначалося, що з різних об'єктивних причин
приміщення для нової установи культури доцільно звести у прилеглому Маріуполі.
Саме з тих часів на афішах театру   з'явилася назва: \emph{\enquote{Донецький}}, а в дужках –
\emph{\enquote{місто Маріуполь}}.

\ii{09_10_2017.stz.news.ua.mrpl_city.1.serce_kulturnogo_zhyttja_mariupolja.pic.1}

 Проект будівлі був розроблений архітекторами  Київського  ДІПроміста О.
Малишенком та О. Криловою за зразком Полтавського обласного театру. Будівництво
театру почалося навесні 1956 р. У листопаді 1959 р. на сторінках міської газети
повідомлялося, що \enquote{будівельники міста форсують спорудження міського театру.
Зараз тут повним ходом йдуть внутрішні оздоблювальні роботи}.

Однак будівництво театру йшло важко. З великими труднощами діставали необхідні
матеріали, були збої з фінансуванням. Велику роль у тому, що театр все ж таки
був побудований, відіграли колективи та керівники будівельних організацій і
великих підприємств міста: заводів імені Ілліча, \enquote{Азовсталі}, коксохімічного. 

\ii{09_10_2017.stz.news.ua.mrpl_city.1.serce_kulturnogo_zhyttja_mariupolja.pic.2}

Нарешті було побудовано красиве, чудово обладнане приміщення для театру в стилі
радянського монументального класицизму з великою кількістю ліпних декоративних
елементів. Об'єм будівлі – 37 тис. м 3. Центральна фасадна частина будівлі з
портиком коринфського ордера на квадратних колонах і віконними прорізами з
напіварочним завершенням злітає вгору кілеподібним фронтоном на консолях із
гармонійною скульптурною композицією, де чільне місце відводиться металургам і
хліборобам як основним працівникам Приазов'я, хвалебну оду яким виконують
покровительки мистецтва. Глядацька зала розрахована на 800 місць. Освітлення
сцени забезпечили 5 софітових ферм, якими можна зарядити одночасно 42
декорації, тобто забезпечити резерв на дві вистави.

Зручно розташований пульт освітлення – під сценою, таким чином, що головний
освітлювач, виконуючи всі перемикання по сцені, водночас може спокійно
спостерігати за результатами своєї роботи. Під сценою розташована кімната
шумових ефектів, яка органічно пов'язана з суфлерською. Завдяки потужній
вентиляційній установці створювалася можливість забезпечувати глядацьку залу
очищеним теплим повітрям в зимовий час і свіжим повітрям – у літній.

\ii{09_10_2017.stz.news.ua.mrpl_city.1.serce_kulturnogo_zhyttja_mariupolja.pic.3}

2 листопада 1960 р. засяяли вогні нового театру – відбулося його урочисте
відкриття. Ця подія стала справжнім святом для жителів міста, які так довго
чекали цього дня. Гідним завершенням свята стала вистава за п'єсою О. Арбузова
\enquote{Іркутська історія}.

У 1985 році сталася важлива подія в театральному житті Маріуполя, була відкрита
мала зала, а 12 листопада 2007 р. наказом Міністерства культури і туризму
театру було присвоєно заслужений статус академічного.

Сьогодні наш театр не перестає приємно дивувати талановитими виставами. Адже
маріупольський театр вже давно має своє творче обличчя, він особливо ретельно
ставиться до підбору репертуару, можливо, інколи й незвичайного, але
притаманного саме йому, його смакам, творчим можливостям і пориванням.

Біля будівлі маріупольського театру святкують День міста, Новий Рік, тут
гуляють сім'ями, годують голубів і насолоджуються відпочинком. Особливо красиво
і гармонійно будівля театру виглядає в темний час доби, або взимку завдяки
добре організованому підсвічуванню органічно виглядає і поблизу, і на відстані.

Є такі місця у кожному місті, які причаровують і надихають, біля яких хочеться
відпочивати. Якимось чином ці ділянки землі так резонують з людським серцем, що
зусиль докладати не треба: все і так добре. У Маріуполі таким місцем для
більшості маріупольців, безумовно, є будівля драматичного театру, що вже давно
стала серцем культурного життя міста та умовним центром Маріуполя, орієнтиром
як для мешканців міста, так і для туристів.

Зараз приміщення готують до капітального ремонту, і вже дуже скоро маріупольці
зможуть побачити оновлену улюблену та найбільшу сцену Маріуполя і знову
відкрити для себе диво театрального дійства, що доставляє величезне задоволення
і глядачам, і артистам.

\ii{09_10_2017.stz.news.ua.mrpl_city.1.serce_kulturnogo_zhyttja_mariupolja.pic.4}

\ii{insert.author.demidko_olga}
