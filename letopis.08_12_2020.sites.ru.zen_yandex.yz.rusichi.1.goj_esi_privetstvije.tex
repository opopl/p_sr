% vim: keymap=russian-jcukenwin
%%beginhead 
 
%%file 08_12_2020.sites.ru.zen_yandex.yz.rusichi.1.goj_esi_privetstvije
%%parent 08_12_2020
 
%%url https://zen.yandex.ru/media/politinteres/goi-esi-strannoe-privetstvie-rusichei-5fc955e352642f33b9116905
 
%%author Русичи (Яндекс Zen)
%%author_id yz.rusichi
%%author_url 
 
%%tags russia,ancient,istoria
%%title "Гой еси". Странное приветствие русичей
 
%%endhead 
 
\subsection{\enquote{Гой еси}. Странное приветствие русичей}
\label{sec:08_12_2020.sites.ru.zen_yandex.yz.rusichi.1.goj_esi_privetstvije}
\Purl{https://zen.yandex.ru/media/politinteres/goi-esi-strannoe-privetstvie-rusichei-5fc955e352642f33b9116905}
\ifcmt
	author_begin
   author_id yz.rusichi
	author_end
\fi

{\bfseries 
Что оно означает?
}

\ifcmt
  pic https://avatars.mds.yandex.net/get-zen_doc/2993437/pub_5fc955e352642f33b9116905_5fca27478f8c7853ed683095/scale_1200
\fi

В письменных источниках практически никогда не упоминается о том, какие речевые
формулы использовались нашими предками в повседневных диалогах.

Возвышенные обороты, вроде \enquote{иду на вы}, обязательно в летописи вносились.
Потому что на них нужно было обратить внимание потомков и современников. А на
обычные приветствия, например, зачем внимание обращать. Все и так знают, как
они произносятся.

Из-за таких особенностей летописания историки и лингвисты очень мало знают о
живой речи русичей. Да, есть еще берестяные грамоты, где, по идее, отражена
именно разговорная лексика. Но тут тоже есть свои нюансы. Писать на бересте -
это вам не по клавиатуре стучать. Буквы приходилось процарапывать на древесной
коре, поэтому стиль письма был сжатым, чуть ли не телеграфным. Мыслью не
растечешься.

В таких условиях настоящим спасением являются сказки. Именно в сказках, былинах
и прочих устных преданиях до нас дошли разные выражения, бытовавшие у предков.

В том числе и традиционное приветствие.

\ifcmt
	tab_begin cols=2
	  pic https://avatars.mds.yandex.net/get-zen_doc/1916740/pub_5fc955e352642f33b9116905_5fca277e8f8c7853ed688e7a/scale_1200
	  pic https://avatars.mds.yandex.net/get-zen_doc/3958762/pub_5fc955e352642f33b9116905_5fca279d8f8c7853ed68c1ec/scale_1200
	tab_end
\fi

Какими словами встречают друг друга герои русских народных сказок? Правильно:
\enquote{Ох ты, гой еси, добрый молодец!} Вариации могут быть разными, но почти везде
фигурирует это крайне странное для современного уха \enquote{Гой еси}.

А что оно означает? Вернее, что означает первое слово, поскольку со вторым-то
все более-менее понятно?

Велик соблазн увидеть в этом еврейское словечко \enquote{гой}. Так в иудаизме.
напомним, называют всех, кто не иудей. На этом поле сейчас активно топчутся
разного рода неоязыческие сайты, утверждая, что так славяне сообщали друг другу
- не еврей я, мол, свой человек.

Но в реальности к иудаизму это приветствие не имеет никакого отношения.

\textbf{\enquote{Гой еси}} - это остаток очень древней праславянской речевой формулы. Было
когда-то в древности такое слово - \textbf{гоить}. Оно означало жить. Слово это
находилось в родстве с общим праиндоевропейским корнем \textbf{gi} с тем же значением.

Так что в целом фраза \enquote{Гой еси} на современный русский язык переводится
следующим образом: \textbf{\enquote{Жив будь!}} Вполне понятное и логичное приветствие.
Практически то же самое, что \enquote{Здравствуй!}.

