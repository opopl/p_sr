% vim: keymap=russian-jcukenwin
%%beginhead 
 
%%file 20_01_2022.yz.volar.1.punkt_naznachenia.cmt
%%parent 20_01_2022.yz.volar.1.punkt_naznachenia
 
%%url 
 
%%author_id 
%%date 
 
%%tags 
%%title 
 
%%endhead 
\zzSecCmt

\begin{itemize} % {
\iusr{Волар}

\enquote{Одно дело, если это незначительное вторжение. Тут не о чем ссориться. Но если
они действительно сделают то, на что способны, с силами, собранными на границе,
это будет катастрофой для России}.

Дж. Байден на пресс-конференции в Восточном зале Белого дома в Вашингтоне, 19
января 2022 года.

***

Когда вы говорите Джозеф, такое впечатление что вы бредите.

Что интересно он имел ввиду под словами \enquote{незначительное вторжение}???

\begin{itemize} % {
\iusr{Андрей К.}

Волар, ну, это как \enquote{чуть-чуть беременна} и \enquote{немножко мёртвый}.
Или, чтоб господа из Залужья поняли: \enquote{На полшишечки}!

\iusr{ЛилёК}

Волар, может быть сонный дед нам намекает дескать \enquote{Ладно, ЛЛНР можете брать. Но
не более того} Поэтому пока надо брать себе только республики. А там видно
будет.

\iusr{Наталья Фичоряк}

Волар, да это просто сдовесный шум! Типа, и нашим - и вашим - и никому...

\iusr{NoLena N}

ЛилёК, брать надо все, что нам нужно. А потом сказать, что незначительное
вторжение. Сколько там той Украины по сравнению с планетой!
\end{itemize} % }

\iusr{Евгений Мельников}

Все верно, вот только сейчас по моему это сделать без войны не получится, а я
помню своего деда и понимаю что это такое.

\iusr{moomi-mama}

Освободить - да, нужно, согласна с этим. Но присоединять - только если на
референдуме в украине результат будет как в Крыму. Если у них меньше 90\% хотят
в Россию - зачем нам такое счастье? Насильно мил не будешь.

\iusr{Ольга Киреева}

молодец, автор!!!

\iusr{Нина Павлова}

Украинцев все устраивает, почему мы должны вмешиваться? Чтобы наши ребята
получили пулю в спину? Сами, только сами.

\begin{itemize} % {
\iusr{Волар}

Нина, потому что эта страна граничит с нами. Потому что то, что они делают
влияет на нас. На нашу Ростовскую, Воронежскую, Белгородскую области. Реки не
заблокируешь. Воздух не заблокируешь. Что тут непонятного?! Я не говорю о
людях. Я говорю о территории. О земле, за которую наши деды умирали. Почему мы
должны отдавать свое? Скажите-ка мне?
\end{itemize} % }

\iusr{Ольга Истомина}

Природу, конечно, жаль. Но мы не в состоянии ее спасти на всей Земле.

Каждому дан свой участок, за который он несет ответственность и получает по
делам своим.

Под идеей спасти землю (воду и все остальное) на Украине можно увидеть идею
спасти эту землю от самих украинцев, а они - какие-никакие, но все же люди. Так
что главное здесь - люди и их личный выбор. Пока они, по согласию, только самим
себе что-то причиняют по своему хотению, другие не имеют права в это лезть.
Насильно никого не спасешь и себя погубишь.

Самое большое зло в современном мире творят США и ВБ. Зло вышло уже на самый
высокий уровень. Пока у них есть возможность влиять на других, зло будет
умножаться, распространяться все дальше. Выйдет из всяких берегов. Значит,
выход тут один - лишить их этой возможности.

Конечно, зло не исчезнет совсем. И со временем опять возрастет до запредельных
величин. И опять понадобится кому-то вырывать его вновь отросшие до гигантских
размеров ядовитые зубы.

\iusr{елена романова}

Первый раз не согласна с автором. Некого нам там спасать. Пора завязывать с
благотворительностью.

\begin{itemize} % {
\iusr{Волар}

елена,сосредоточьтесь пожалуйста. Прочитайте внимательно.

1. британо-нидерландская Royal Dutch Shell разрабатывает Юзовское месторождение
(располагается на территории Донецкой и Харьковской областей)

2. американская Chevron намерена искать сланцевый газ на Олесской площади (Западная Украина)

Все это методом фрекинга. Вы знаете ЧТО ЭТО такое??? Посмотрите фото.От добычи
сланцевого газа на Украине пострадают пограничные Брянская и Курская области
России, так как у них с Украиной единые подземные водные горизонты.

Я не говорю \enquote{кого}. Я говорю \enquote{что}.

\href{https://sevastopol-su.turbopages.org/sevastopol.su/s/news/slancevyy-genocid-razvyazan-kak-ukraina-prevrashchaet-donbass-v-pustynyu}{«Сланцевый геноцид» развязан: как Украина превращает Донбасс в пустыню, 20.01.2021%
}

У нас общая с Украиной граница. Не будет такого, что у них все отравлено, а у
нас боржоми из водопроводного крана льется. Река Северский Донец и по нашей
территории течет. Посмотрите фото последствий этого фрекинга.

\iusr{Мимокрокодила Мимокрокодила}
\textbf{Волар}, почитала про фрекинг на западе и его последствия, это ужас...

\iusr{Волар}
\textbf{Мимокрокодила}, у меня та же реакция. Захотелось узнать, а какого черта запад
так вцепился в Украину. Начинаешь читать и волосы дыбом и про фрекинг и про
\enquote{Монсанто}. Они хотят у нас на границе экологическую катастрофу создать.
Начхать мне на долбанутых нациков, но у меня дед за Украину воевал. Он напрасно
там кровь проливал, как и тысячи наших? Надо забирать. И чем быстрее, тем
лучше.

\iusr{Мимокрокодила Мимокрокодила}
\textbf{Волар}, про \enquote{забирать} не знаю, политика дело тонкое, но почему экологи молчат?!
Люди на Западе почти 20 лет бьются, столько уже накоплено материалов, столько
больных, столько загаженных водных источников и убитой земли. Я вообще о такой
катастрофе от вас впервые узнала.

\iusr{Волар}
\textbf{Мимокрокодила}, бабки. Орут о том, о чем велено орать. О чем не велено,
помалкивают. Охрана природы это одно. А экология это бизнес.

\end{itemize} % }

\iusr{André Krug/Андрій Кругляк}

Смотрите, пристально смотрите на мою любимую Украину. Всё это могло быть у вас.

\begin{itemize} % {
\iusr{Волар}
\textbf{André Krug/Андрій Кругляк}, 

могло. Мы тоже сошли с ума в 90-х годах, но не до конца. Россия западу не по
зубам. Пока они жрали наследие СССР мы очухались. Украина без нас, одна, с
западными тварями не справится.

\iusr{Наталья}
\textbf{Волар}, они и не хотят справляться. Агитация тут не уместна. Её уже давно пережевали и выплюнули.
\end{itemize} % }

\iusr{Александр Зуболенко}

Все так все верно вставйте люди русские против ангосаксоского мира мать Россия
защити сваю.непутевае.детя

\begin{itemize} % {
\iusr{Нина Павлова}
\textbf{Александр Зуболенко}, 

вот-вот. А украинцы будут сидеть по хатам и ждать, когда их освободят, чтобы
потом освободителей винить во всех бедах.

\iusr{Волар}
\textbf{Нина}, 

вы мыслите, как обиженный ребенок. Отбросьте обиду. Поднимитесь над этим и
подумайте стратегически. Нам всем аукнется разграбленая и уничтоженная Украина.
Запад ноет об экологии. У себя!!!! А у нас он что делает?

\end{itemize} % }

\iusr{Елена Елена}

Нацисты во время ВОВ эшелонами вывозили чернозем с оккупированных советских
территорий. Ничего не меняется. Снова дорвались.

\iusr{Лариса Богомолова}

Простите, но только слабоумные не умеют контролировать размеры поглощения
пищи. Мы 8 лет не можем решить проблему с территориями и людьми Донбасса и всего
лишь месяц, как наш МИД начал голос подавать, который наконец-то прорезался. А Вы
сразу с места в карьер-всю окраину захватить, нациков пересажать, бандитам
"вышку" дать, проросийскую власть на госпосты поставить, договора на миллиарды
разорвать, всех на все буквы послать и будет всем счастье. А остальное узколобое
население, которое всё устраивает, куда денете? В резервацию отправите? Как у Вас
всё лихо. Можно подумать, что у нас экономика впереди планеты всей стала, наравне
с гиперзвуковым оружием, а Россия лет 30, как гегемонит на планете Земля. Сейчас
мы всем покажем кузькину мать! Спасём украинскую землю от разрухи и загрязнения
и все бывшие союзные народы из российской казны накормим, всю инфраструктуру
экологически чистую построим, всех ворогов с исконно русских земель выгоним. У
нас ведь русский мужик-солдат десять жизней имеет, когда у украинского
хатаскраюшника всего одна и её беречь ему надобно. Знаете-ли... фантазировать и
мечтать-это, конечно не запрещается, но и тут нужно меру знать и учитывать реалии
жизни. Популизм зашкаливает. За геополитическими аппетитами необходимо тоже
следить, чтобы заворота кишок не случилось.

\begin{itemize} % {
\iusr{Волар}

Лариса, я не Шойгу. И не Путин. Я высказываю свои мысли. Решения принимают
другие люди.

Спасибо за ваше мнение.
\end{itemize} % }

\iusr{Скрипка Лиса}

Вот только у меня вопрос: а они сами то что, не видят всего этого? Неужто уж не
осталось во всей Украине здравомыслящих людей? Или как всегда: \enquote{моя хата с
краю- ничего не знаю}! Опять русский Иван должен о них позаботиться? А они за
длинным рублем и кружевными труселями подались в Гейропу! Наши сыновья и внуки,
между прочим, тоже не бессмертные и как то вот не очень то хочется, чтобы они
там головы свои сложили за этих салоедов!

\end{itemize} % }
