% vim: keymap=russian-jcukenwin
%%beginhead 
 
%%file 17_12_2020.news.ua.strana.romanova_maria.1.vaccine_ukraina_5_percent
%%parent 17_12_2020
 
%%url https://strana.ua/news/307272-kakie-vaktsiny-ot-koronavirusa-zakazala-ukraina.html
 
%%author Романова, Мария
%%author_id romanova_maria
%%author_url 
 
%%tags ukraina,covid_vaccine,vaccination
%%title Для 5% украинцев. Сколько вакцины заказал на Западе Минздрав и когда ее ждать в Украине
 
%%endhead 
 
\subsection{Для 5\% украинцев. Сколько вакцины заказал на Западе Минздрав и когда ее ждать в Украине}
\label{sec:17_12_2020.news.ua.strana.romanova_maria.1.vaccine_ukraina_5_percent}
\Purl{https://strana.ua/news/307272-kakie-vaktsiny-ot-koronavirusa-zakazala-ukraina.html}
\ifcmt
	author_begin
   author_id romanova_maria
	author_end
\fi

\ifcmt
  pic https://strana.ua/img/article/3072/kakie-vaktsiny-ot-72_main.jpeg
  caption Вакцины, которые получит Украина бесплатно, еще не прошли испытаний. Фото Gavi.com 
  width 0.5
  fig_env wrapfigure
\fi

Украинские власти заявляют о старте вакцинации населения уже в марте следующего
года. 

Правда, от Минздрава нет никакой информации по ряду важных вопросов. Первый -
какое количество доз и когда сможет купить Украина. А не только получить в
качестве гуманитарной помощи.

Второй вопрос - какие вакцины в нашу страну придут в первую очередь. Минздрав
конкретных марок не называет, утверждая, что к нам будут поставлять разные
препараты.

Но, как оказалось, в мире давно не секрет, что бесплатные поставки будут
включать вакцины, которые еще не проверены. И сами производители говорят, что с
ними есть проблемы. 

Что же до массовых платных поставок, которые уже начинаются по всему миру -
Украине на них рассчитывать даже в наступающем году, судя по всему, не стоит. 

"Страна" разбиралась в том, какие будут вакцины и когда их получат украинцы. 

\subsubsection{Что власти заявляли о сроках доставки вакцины}

"Страна" недавно разбирала планы и реальность по доставке в Украину иностранных
вакцин.\Furl{https://strana.ua/news/305828-vaktsina-ot-koronavirusa-kohda-ona-postupit-v-ukrainu-pfizer-moderna-sputnik.html}

Восемь миллионов доз, достаточных для вакцинации вдвое меньшего количества
людей, в Украину обещают передать бесплатно - в рамках международного механизма
Covax. 

В пакет войдут вакцины в основном западного производства, но какие именно
поставят Украине - точно непонятно, поскольку слать собираются то, что будет
доступно на момент поставки. 

В Минздраве ожидают начала поставок в марте. Правда, это не первый вариант
даты. Ранее звучал апрель, а еще до этого - январь. 

"Эти 8 миллионов доз начнут поставляться с конца первого квартала следующего
года. Мы рассчитываем их получить до конца первого полугодия 2021 года. Они
будут идти соответствующими траншами", - рассказал сегодня глава МОЗ Максим
Степанов.\Furl{https://strana.ua/news/307149-kohda-ukraina-poluchit-vaktsinu-ot-koronavirusa.html}

В данной цитате звучит существенная коррекция сроков. Раньше из слов чиновников
следовало, что уже все восемь миллионов доз Украина получит весной. А
оказывается, они будут идти траншами.

Впрочем, "Страна" именно это и предполагала. Мы приводили мнения экспертов,
согласно которым "переварить" сразу большой объем прививок украинская
медицинская система не сможет. В силу того, что нужна сложная логистика для
вакцин, которые нужно перевозить при сверхнизких температурах. 

То есть растянуться процесс вакцинации бесплатными прививками может и до осени.
И это речь только о четырех процентах населения. Остальных же нужно будет
прививать за счет госбюджета - то есть вакцины должно будет закупать
государство напрямую у производителей. 

Украинские власти пока не хвастаются "перемогами" на ниве заключенных
контрактов, туманно заявляя, что переговоры идут со всеми возможными
поставщиками. Кроме россиян, чей "Спутник V" Киев официально покупать
отказались. Хотя Россия была единственной страной, которая это Украине
предложила без очереди. 

Формальная причина - вакцина не проверена (хотя это на самом деле можно сказать
о любой западной вакцине - в том числе в рамках Covax, о чем мы скажем ниже).
Фактическая же - запрет от американцев.\Furl{https://strana.ua/news/295565-posolstvo-ssha-zapretilo-brat-rossijskuju-vaktsinu-ot-koronavirusa-reaktsija-seti.html}

Таким образом Киев стал в длинную очередь желающих купить европейские и
американские препараты. Правда, шансов получить там вакцину более-менее
оперативно, почти нет.

\subsubsection{Массовая вакцинация - после 2022 года}

О том, что Украина сможет купить вакцины позже многих стран, еще пишут ведущие
западные СМИ. 

На днях издание Bloomberg опубликовало данные по контрактам, которые заключила
каждая страны, чтобы вакцинировать свое население (речь о тех контрактах,
которые агентство нашло в публичной сфере). 

Есть в рейтинге и Украина. Указывается, что наша страна на данный момент
договорилась о поставка вакцин для 5\% населения или 2,1 человек. Это 4
миллиона доз. 

\ifcmt
pic https://strana.ua/img/forall/u/0/92/%D0%BA%D0%B0%D1%80%D1%82%D0%B0(24).png
\fi

Если посмотреть, кто еще в мире договорился вакцинировать всего 5\% своих
жителей, то мы окажемся в компании беднейших государств мира. 

Украина здесь на уровне Уганды, Того, Зимбабве, Танзании, Таджикистана и других
стран. 

\ifcmt
pic https://strana.ua/img/forall/u/0/92/%D1%81%D0%BF%D0%B8%D1%81%D0%BE%D0%BA(2).png
\fi

Для сравнения - Россия уже застолбила вакцины для 50\% населения, Турция - для
30\%. А богатые страны Запада и Евросоюз (зеленые зоны на карте) уже контрактуют
количество доз, превышающих население - с прицелом на ревакцинацию, которую
придется, судя по всему, проводить достаточно часто. 

Абсолютный лидер здесь Канада, которая обеспечила себе вакцин для того, чтобы
привить все свое население по четыре раза. 

Пока точно непонятно, о каких контрактах говорит Bloomberg. Но напомним, что
только в рамках Covax - то есть бесплатно - украинский Минздрав говорил о
количестве вдвое большем: не для 5\%, а для 10% населения. Как видим, в США
говорят о другой цифре. 

"Bloomberg отследил девять самых многообещающих вакцин по всему миру, от
национальных закупок до инъекций в руки пациентам. По нашим подсчетам, уже
выделено 7,95 миллиарда доз", - пишет издание. 

Итак, мы разобрались, что из этих почти 8 миллиардов доз Украина смогла
застолбить только 2,1 миллиона. То есть когда прибудут остальные - под
вопросом.

Судя по другому исследованию, ждать придется действительно долго. Еще в ноябре
журнал The Economist публиковал расчеты, согласно которым массовое
вакцинирование в Украине сможет начаться только в апреле... 2022 года. 

\ifcmt
pic https://strana.ua/img/forall/u/0/92/%D1%8D%D0%BA%D0%BE%D0%BD%D0%BE%D0%BC%D0%B8%D1%81%D1%82.jpg
\fi

\subsubsection{Какие вакцины к нам придут в рамках Covax?}

Кстати, о Covax. "Блумберг" публикует интересную информацию о том, какие
вакцины будут поставляться в рамках этого механизма (украинские власти об этом
молчат). Это британская AstraZeneka, китайско-американская UBI group и
французская Sanofi. 

\ifcmt
pic https://strana.ua/img/forall/u/0/92/%D0%BA%D0%BE%D0%B2%D0%B0%D0%BA%D1%81.png
\fi

То есть, как видим, здесь нет ни Pfizer, ни Moderna, которые считаются
первоклассными вакцинами. И уже массово закупаются во многих странах мира. 

Чего не скажешь о вакцинах, которые будут поставляться в бедные страны - в том
числе и Украину - бесплатно. Все они еще не прошли испытаний. 

Шведско-британская компания AstraZeneca разработала вакцину совместно с
Оксфордским университетом. В этой вакцине аденовирус шимпанзе доставляет
доставляет белок коронавируса. Особенности этой разработки в ее низкой
стоимости (предположительно 5 долларов за две дозы), а также, что ее можно
хранить и перевозить в обычных холодильниках. 

Однако есть и проблемы. Например, в сентябре AstraZeneca пришлось приостановить
испытания вакцины, после того, как у одного добровольца развился поперечный
миелит (воспаление на спинном мозге). После этого было начато расследование,
чтобы выяснить связан ли этот случай с вакцинацией.

Также во время испытаний в Индии были зафиксированы побочные эффекты, в рамках
нормы, но их было больше, чем в вакцине плацебо.


\ifcmt
pic https://strana.ua/img/forall/u/0/34/54565895_303.jpg
width 0.4
fig_env wrapfigure
\fi

Вакцина AstraZeneca сейчас вернулась на вторую стадию испытаний. В ЕС
предазаказали 400 млн доз, но даже в Британии пока ей не прививаются.

Французская компания Sanofi совместно с британской GSK также разработала
вакцину. Они находятся на второй стадии испытаний.

Эта вакцина против Covid-19 разработана на основе вирусных белков, созданных с
помощью искусственно созданных вирусов, выращенных внутри клеток насекомых. GSK
дополнила эти белки адъювантами, которые стимулируют иммунную систему. Вакцина
основана на той же конструкции, что и уже одобренная вакцина от гриппа Flublok.

На днях стало известно, что вакцина Sanofi станет доступна не раньше конца 2021
года. Изначально ожидалось, что это произойдет в середине следующего года.

Задержка связана с тем, что в Sanofi решили улучшить показатели иммунной
реакции среди более возрастных групп. Проведенные исследования показали, что у
вакцинированных лиц в возрасте от 18 до 49 лет вырабатывался такой же иммунный
ответ, как и у пациентов, переболевших Covid-19. Но у пациентов более старшего
возраста иммунная реакция была слабее. 

Евросоюз уже подписал контракты на поставку вакцины Sanofi-GSK на 300 млн доз.
Также 200 млн доз компания готова передать американской компании COVAX для
"справедливого распространения по всему миру".

Еще одна вакцина от коронавируса разработана UBI/UBIA Group (United Biomedical
Inc., Asia) в коллаборации с американской компанией Covaxx (дочерняя компания
UBI со штаб-квартирой в Нью-Йорке). 

Вакцина, получившая название UB-612, представляет собой препарат высокоточного
действия на синтетической пептидной платформе, направленный на активацию
B-лимфоцитов и Т-лимфоцитов.

При этом доклинические исследования проводили на морских свинках, крысах и
мышах. До 31 декабря 2020 года в Тайване будет продолжаться первая фаза
клинических испытаний. Компания также имеет соглашение с Медицинским центром
Университета Небраски на проведение второй фазы испытаний в США.

Производитель взял на себя предварительные обязательства по поставке более 100
миллионов доз UB-612 по всему миру.

Другими словами, "Ковакс" предусмотрел для Украины вакцины, которые еще очень
сырые и не годятся для массовой вакцинации.

Теперь щедрость международных партнеров понятна: украинцам будут поставлять
препараты, испытания многих из которых ко времени поставок еще могут не
завершиться. Испытают ли их на украинцах - вопрос открытый. 

Понятно, что скорее всего эти вакцины поступят в страну уже после всех фаз.
Однако, когда они закончатся, никто точно не знает - это может растянуться на
долгие месяцы. А министр Степанов обещает первые вакцины уже в марте. 

Таким образом, что именно нам привезут в бесплатном пакете - этого скорее всего
не знают даже власти Украины. 

При этом они отказываются брать российскую вакцину, которую уже
зарегистрировали и массово прививают население - мотивируя, что лекарство не
испытано и даже "не существует". Но соглашаются на препараты, которые еще
только на первой фазе испытаний. 

Отметим при этом, что Украине предлагали производить российскую вакцину на
своих мощностях. При этом никто не мешает проверить препарат в Украине и, если
он будет безопасным, начать производство прямо здесь. Это значит, что мы смогли
бы гораздо быстрее и дешевле покрыть свои потребности в вакцине. 

Теперь же этого этапа придется ждать до 2022 года. 
