% vim: keymap=russian-jcukenwin
%%beginhead 
 
%%file slova.otvetstvennost
%%parent slova
 
%%url 
 
%%author 
%%author_id 
%%author_url 
 
%%tags 
%%title 
 
%%endhead 
\chapter{Ответственность}

%%%cit
%%%cit_head
%%%cit_pic
%%%cit_text
Им все равно, кем работать - главное, ни за что не \emph{отвечать}.  За
последние десять лет, в Украине в исполнительной власти сформировался клубок
\enquote{универсальных менеджеров}, - новая номенклатура как в ссср, которым безразлично
где и кем работать, лишь бы платили большие деньги. Эти люди демонстрируют свои
западные взгляды и опираясь на неряшливые западные связи и знакомства имитируют
любую квалификацию и компетентность. Так, например, преподаватель в
американском университете может быть главным оружейником страны во время войны,
а бывший журналист, \enquote{разгребатель грязи} может быть железнодорожником. И так во
всем Кабмине Украины.  Отсюда полный отрыв власти от \emph{ответственности} за
свои действия, что страшно вредит Украине
%%%cit_comment
%%%cit_title
\citTitle{В украинской власти сформировался клубок универсальных менеджеров - как в СССР}, 
Виктор Небоженко, strana.ua, 21.06.2021
%%%endcit


%%%cit
%%%cit_head
%%%cit_pic
%%%cit_text
Статья 43 гарантирует право на своевременную выплату заработной платы.  Но по
факту это право не соблюдается. Перед шахтерами государственных шахт возникают
многомесячные задолженности. Которые ликвидируются небольшими дозами. К
\emph{ответственности} за это никого не привлекают.  Что же до зарплат в частном
секторе, то их держат в основном в тени, на что государство закрывает глаза.
Таким образом владельцы бизнеса никакой \emph{ответственности} за невыплату или
задержку зарплаты \enquote{в конверте} не несут
%%%cit_comment
%%%cit_title
\citTitle{День Конституции Украины 28 июня - какие статьи нарушаются сильнее всего}, 
Оксана Малахова; Максим Минин, strana.ua, 28.06.2021
%%%endcit

