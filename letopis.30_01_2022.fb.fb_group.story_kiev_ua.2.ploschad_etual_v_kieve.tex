% vim: keymap=russian-jcukenwin
%%beginhead 
 
%%file 30_01_2022.fb.fb_group.story_kiev_ua.2.ploschad_etual_v_kieve
%%parent 30_01_2022
 
%%url https://www.facebook.com/groups/story.kiev.ua/posts/1850962235100569
 
%%author_id fb_group.story_kiev_ua,solovjeva_tatjana.kiev
%%date 
 
%%tags francia,gorod,kiev,maidan_nezalezhnosti.kiev,paris
%%title ПЛОЩАДЬ ЭТУАЛЬ В КИЕВЕ
 
%%endhead 
 
\subsection{ПЛОЩАДЬ ЭТУАЛЬ В КИЕВЕ}
\label{sec:30_01_2022.fb.fb_group.story_kiev_ua.2.ploschad_etual_v_kieve}
 
\Purl{https://www.facebook.com/groups/story.kiev.ua/posts/1850962235100569}
\ifcmt
 author_begin
   author_id fb_group.story_kiev_ua,solovjeva_tatjana.kiev
 author_end
\fi

ПЛОЩАДЬ ЭТУАЛЬ В КИЕВЕ

Площадь Шарля де Голля в Париже называлась раньше Площадь Звезды - Этуаль по-
французски. Это историческое название она получила, потому что была местом
встречи двенадцати прямых проспектов, включая Елисейские поля. А в центре, как
всем известно, Триумфальная арка.

\ii{30_01_2022.fb.fb_group.story_kiev_ua.2.ploschad_etual_v_kieve.pic.1}

А посмотрите на наш Майдан! На нем встречаются десять улиц, включая Крещатик.

\ii{30_01_2022.fb.fb_group.story_kiev_ua.2.ploschad_etual_v_kieve.pic.2}

Улица Гринченко, переулок Шевченко, Софиевская и Малая Житомирская,
Михайловская и Костельная. Улица Архитектора Городецкого и Институтская!

А парочка на площади, у кофейни, засыпанной снегом - как из французского
фильма...

А еще в центре площади есть ворота, но они очень даже напоминают Арку.

Чем не Париж?
