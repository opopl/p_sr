% vim: keymap=russian-jcukenwin
%%beginhead 
 
%%file 31_08_2021.fb.kulik_vitalij.1.pro_movu_i_ne_tilky
%%parent 31_08_2021
 
%%url https://www.facebook.com/vitalii.kulik/posts/4543830625667847
 
%%author Кулик, Виталий
%%author_id kulik_vitalij
%%author_url 
 
%%tags jazyk,mova,obschestvo,ukraina,ukrainizacia
%%title ПРО МОВУ І НЕ ТІЛЬКИ
 
%%endhead 
 
\subsection{ПРО МОВУ І НЕ ТІЛЬКИ}
\label{sec:31_08_2021.fb.kulik_vitalij.1.pro_movu_i_ne_tilky}
 
\Purl{https://www.facebook.com/vitalii.kulik/posts/4543830625667847}
\ifcmt
 author_begin
   author_id kulik_vitalij
 author_end
\fi

ПРО МОВУ І НЕ ТІЛЬКИ.

Нещодавно \textbf{Pavel Sebastianovich} виступив з кількома постами в соцмережі, які
деякі мої друзі поспішили обвішати ярликами та захейтити. Мова йде про
\enquote{розколи в Україні, до яких доклали руки колишні і нинішні очільники
країни} та рефлексія Павла на одну з програм Романа Скрипіна про
\enquote{корінні народи}. Все зводиться до мови. 

\ifcmt
  pic https://scontent-cdg2-1.xx.fbcdn.net/v/t39.30808-6/240509146_4543821059002137_3594606493538203393_n.jpg?_nc_cat=104&ccb=1-5&_nc_sid=730e14&_nc_ohc=mUAGJZmluRQAX-XRdI-&_nc_ht=scontent-cdg2-1.xx&oh=6ea178eb11aaafd58c9032432c31c54d&oe=613627AC
  width 0.6
\fi

1) МОЯ ПОЗИЦІЯ. Відразу зазначу, що я не поділяю позицію Павла щодо
необхідності переглядати закон про мову. Більш того, я вважаю, що сучасна мовна
політика в Україні хоча і містить в собі елементи \enquote{трошкізму} та
непослідовності, все ж дозволяє поширювати українську мову в сферах, де раніше
її не було. І це добре.

Так. Я з одну державну мову, за мовні квоти на телебаченні, за пільги для
української книги, за обов’язкове використання української мови в сфері послуг,
тести на знання мови для держслужбовців тощо. 

Проте я категорично не сприймаю і засуджую спроби деяких \enquote{граманаці}
примушувати, нав’язувати, влаштовувати мовосрач на пустому місці, заперечувати
патріотизм російськомовних, право на приватне спілкування в соцмережах і т.п.
Переконаний, що фаріони зробити більше зла українізації чим Павло з його хибним
(як на мене) постом.

2) ЧОМУ МОВА ВАЖЛИВА? Американський соціолог Роджерс Брубейкер вивів таке
поняття, як "націоналізовуючий націоналізм". Це етнонаціоналізм націй, які
ведуть боротьбу за національне визволення та утвердження власної
етнонаціональної ідентичності. 

У боротьбі з російським імперіалізмом, в зіткненні з т.зв. «Русскім міром», в
Україні відбувся запуск «компенсуючого» проекту по використанню державної влади
для відстоювання певних (раніше неадекватно задоволених) інтересів корінної
нації.

Україна не завершила останнього етапу національно-визвольного процесу. Ми не
націоналізували власну державу. Вона все ще залишається успадкованою від СРСР
колоніальною системою, яка трималася на клептократії компрадорської еліти.
Війна з російською агресією на Донбасі викликали до життя потужний імпульс
націоналізовуючого націоналізму, який повинен завершити цей перехід.

На певному історичному етапі ми були приречені увійти в фазу «націоналізуючого
націоналізму». Це допомогло нам вистояти проти імперського «вітчизняного
націоналізму» Путіна в 2014 році. І нам доведеться пожинати всі неприємні
наслідки цього. Українізація дискурсу повинна завершитися і знайти закінчену
форму. 

Я був одним з тих, хто публічно критикував панування дискурсу ультраправих на
Майдані, потім декомунізацію, вказував на недоцільність перетворення
етнонаціоналізму в ідеологію правлячої партії... Проте, я розумію, що ми маємо
пройти цей етап. 

Головко Федор якось зауважив, що концепт ігнорування історичної пам'яті та
об'єднання на основі бачення майбутнього не враховує одного фактора. Єдине
майбутнє і факт того, що ти будеш з кимось ділити його - треба прийняти. Це
питання самоідентифікації і розрізнення свій-чужий.

Неможливо поважати точку зору тих, хто вважає, що вулиця Жукова повинна бути,
якщо я вважаю Жукова мерзотником. Або те ж саме з Бандерою. Змиритися з
наявністю альтернативної пам'яті, в якій вшановуються мерзотники - складно.

Головко вважає, що виходом з цієї петлі є не дві історичних пам'яті, а
вироблення базового консенсусного погляди з акцентом на Україну і заборона
мусування цих тем, щоб уникнути розколу.

Але це теж не вихід. Ми так можемо дійти до криміналізації ворожого історичного
наративу. Так в Росії вирішили з темами Другої світової війни. Але ні в Росії,
ні тим більше в Україні, подібні практики ні до чого доброго не приведуть. Рано
чи пізно офіційний міф буде зруйнований і настане історичний шок, подібний
тому, як це було в останню фазу перебудови в СРСР.

3) ЩО РОБИТИ. Єдиною адекватною позицією сьогодні є винесення мовного питання
за душки політичної дискусії та повернутися до нього тоді, коли основні
завдання по відновленню суверенітету та подаланню олігархату будуть виконані. 

Мовне питання та проблеми історичної пам’яті в Україні не можна вирішити без
\enquote{стасізу} (громадянської напруги / конфлікту) а ні в межах \enquote{націоналізуючого
націоналізму}, а ні в форматі \enquote{постнаціонального}. В цій сфері потрібна
перезбірка принципів та цінностей, новий Суспільний договір. І це складний
шлях. 

4) ПРО ПОЛІТИЧНУ ПОЗИЦІЮ РУХУ 30+. Чи пост Павла є позицією Політичного руху
30+? Ні! Це його особиста думка, право на яку ми поважаємо. Це не означає, що,
наприклад, я з цим погоджуюсь. Ми сперечаємося в колі засновників руху і
виносимо цю дискусію в публічну площину. 

В найближчий час ми проведемо серію дебатів, серед яких буде і тема мовної
політики. Переконаний, що в підсумку дебатів ми, з Павлом чи \textbf{Yuriy Romanenko},
вийдемо на спільну позицію у вирішенні мовного питання. Вона і стане офіційною
позицією Руху 30+.

\begin{verbatim}
	#мова #рух30плюс
\end{verbatim}

\ii{31_08_2021.fb.kulik_vitalij.1.pro_movu_i_ne_tilky.cmt}
