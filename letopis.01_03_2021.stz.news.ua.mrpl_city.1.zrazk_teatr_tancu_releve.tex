% vim: keymap=russian-jcukenwin
%%beginhead 
 
%%file 01_03_2021.stz.news.ua.mrpl_city.1.zrazk_teatr_tancu_releve
%%parent 01_03_2021
 
%%url https://mrpl.city/blogs/view/zrazkovij-teatr-estradnogo-tantsyu-releve-ku-tsentr-kulturi-livoberezhnij
 
%%author_id demidko_olga.mariupol,news.ua.mrpl_city
%%date 
 
%%tags 
%%title Зразковий театр естрадного танцю "Релеве" КУ "Центр культури "Лівобережний""
 
%%endhead 
 
\subsection{Зразковий театр естрадного танцю \enquote{Релеве} КУ \enquote{Центр культури \enquote{Лівобережний}}}
\label{sec:01_03_2021.stz.news.ua.mrpl_city.1.zrazk_teatr_tancu_releve}
 
\Purl{https://mrpl.city/blogs/view/zrazkovij-teatr-estradnogo-tantsyu-releve-ku-tsentr-kulturi-livoberezhnij}
\ifcmt
 author_begin
   author_id demidko_olga.mariupol,news.ua.mrpl_city
 author_end
\fi

Зовсім нещодавно завдяки рубриці \emph{\enquote{Маріуполь театральний} на Маріупольському
телебаченні} я відкрила для себе безліч нових театральних колективів, як
драматичних, музичних, так і танцювальних. Виявляється, що в Маріуполі працює
15 професійних та самодіяльних театрів (!), кожен з яких має свій репертуар та
свого глядача.

\ii{01_03_2021.stz.news.ua.mrpl_city.1.zrazk_teatr_tancu_releve.pic.1}

У культурному просторі не тільки Маріуполя, але й всього регіону важливе місце
посідає \emph{Театр естрадного танцю \enquote{Релеве}}, який  розпочав свою діяльність у
Міському Палаці культури у 1997 році. Почесне звання \enquote{Зразковий} отримав вже  у
2000 році. У 2018 році відзначив свій 20 – річний ювілей. Режисерка театру –
\emph{\textbf{Олійник Вікторія Віталіївна}} – спочатку працювала балетмейстером в театрі-клубу
\enquote{Діалог} при Палаці культури, художньою керівницею в якому була \emph{\textbf{Лідія
Анатоліївна Хаджинова}}. Саме вона підштовхнула Вікторію Віталіївну створити
власний театр. Назва колективу \enquote{Релеве} пов'язана з окремим рухом в
хореографії. У перекладі означає \emph{підйом на полупальці}. Піднімаючись на
полупальці у кожній театралізованій постановці учасниці колективу стають вище,
таким чином долучаючись до мистецтва.

\ii{01_03_2021.stz.news.ua.mrpl_city.1.zrazk_teatr_tancu_releve.pic.2}

Протягом творчого життя в колективі займається 90 учасниць – дівчат віком від
6-ти до 17-ти років (всього чотири вікові групи).  Щорічно театр здійснює 5-6
нових хореографічних постановок і одну міні-виставу.

За роки існування колективу були здійснені постановки – \enquote{Блакитне щеня},
\enquote{Снігова Королева}, \enquote{Дюймовочка}, \enquote{Вовк і семеро козенят}, \enquote{Бременські
музиканти}.

Напрями репертуарної політики колективу різноманітні – \emph{сучасна хореографія,
стилізація, фольклор}. Здебільшого театр готує сюжетну хореографію. Яскраві,
оригінальні, тісно пов'язані з сучасністю, хореографічні композиції завоювали
повагу і любов глядача. Кожний номер – театралізована дія, яка несе в собі
смислове навантаження.

\ii{01_03_2021.stz.news.ua.mrpl_city.1.zrazk_teatr_tancu_releve.pic.3}

Хореографічні композиції – \enquote{Гіркота війни}, \enquote{Відродження}, \enquote{На околиці}, \enquote{Я
малюю}, \enquote{Молитва за Україну}, \enquote{Маша і Ведмідь}, \enquote{Лісова пісня}, \enquote{Фіксікі},
\enquote{Колискова}, \enquote{Мамі}, \enquote{Незламні} – характеризуються цілеспрямованістю,
організованістю, дотриманням інноваціям у хореографічному жанрі.

Завдяки педагогічній майстерності Вікторії Віталіївни не одне покоління
зростало на танцях \enquote{Релеве} і багато його вихованців пов’язали своє життя з
професійною хореографією чи театральною справою.

Важливе місце в репертуарі колективу займає тема сьогоднішніх подій. Біль,
повага і любов до своєї Батьківщини виразно і проникливо розкрита в
хореографічних постановках – \enquote{Україно, будьмо!}, \enquote{Танцюй, юність Маріуполя},
\enquote{Відродження}, \enquote{Горниця}, \enquote{Івана Купала}, \enquote{Протистояння}.

\ii{01_03_2021.stz.news.ua.mrpl_city.1.zrazk_teatr_tancu_releve.pic.4}

Кожні три роки колектив успішно підтверджує звання \enquote{Зразковий}. \enquote{Релеве} –
активний учасник і неодноразовий призер фестивалів та конкурсів хореографічного
мистецтва Всеукраїнського і Міжнародного статусу.

Колектив – обов'язковий учасник спільних проєктів  та шоу – програм Палацу
культури, обласних, міських та районних заходів.

\ii{insert.read_also.demidko.grecka_teatralna_studia}

Неодноразово діяльність Зразкового театру естрадного танцю \enquote{Релеве} була
відзначена Обласним управлінням культури, департаментом культурно –
громадського розвитку, районною адміністрацією.

Як наголошує художня керівниця ЦК \enquote{Лівобережний}, \emph{\textbf{Тетяна Володимирівна
Живолуга}}: \emph{\enquote{У Вікторії Віталіївни є один дуже важливий життєвий принцип –
найголовніше, щоб всі учасники колективу стали гідними людьми}}. Зразковий театр
естрадного танцю \enquote{Релеве} вже давно став справжньою великою сім'єю. На звітних
концертах театру завжди аншлаги. Наразі театр відновлює спектакль мініатюр
\enquote{Дюймовочку}, який востаннє представляли на сцені 8 років тому. Водночас готує
важливий соціальний проєкт \emph{\textbf{\enquote{Стань краще}}}, присвячений дітям з нелегкими долями.
Він буде представлений у квітні цього року. Сучасна манера виконання твору
особливо близька колективу і робить театр естрадного танцю оригінальним,
неповторним, що має своє творче обличчя.

\ii{01_03_2021.stz.news.ua.mrpl_city.1.zrazk_teatr_tancu_releve.pic.5}
