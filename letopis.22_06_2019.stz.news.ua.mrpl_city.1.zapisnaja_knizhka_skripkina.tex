% vim: keymap=russian-jcukenwin
%%beginhead 
 
%%file 22_06_2019.stz.news.ua.mrpl_city.1.zapisnaja_knizhka_skripkina
%%parent 22_06_2019
 
%%url https://mrpl.city/blogs/view/zapisnaya-knizhka-skripkina
 
%%author_id burov_sergij.mariupol,news.ua.mrpl_city
%%date 
 
%%tags 
%%title Записная книжка Скрипкина
 
%%endhead 
 
\subsection{Записная книжка Скрипкина}
\label{sec:22_06_2019.stz.news.ua.mrpl_city.1.zapisnaja_knizhka_skripkina}
 
\Purl{https://mrpl.city/blogs/view/zapisnaya-knizhka-skripkina}
\ifcmt
 author_begin
   author_id burov_sergij.mariupol,news.ua.mrpl_city
 author_end
\fi

\ii{22_06_2019.stz.news.ua.mrpl_city.1.zapisnaja_knizhka_skripkina.pic.1}

Жил в Мариуполе отставной подполковник авиации, пролетавший всю Отечественную
войну на бомбардировщиках, обстрелянных вражескими зенитками и истребителями.
Интересный человек с насыщенной событиями биографией, одаренный природой и
живым характером и многими талантами, интересами и умениями. Звали его
\textbf{Константин Николаевич Скрипкин}. Он оставил после себя не только добрую память,
но и ничем непримечательную, на первый взгляд, записную книжку со следами
длительного использования. Хранила ее, как семейную реликвию дочь Константина
Николаевича - \textbf{Алина Константиновна}, музыкант и педагог. Записи в книжке выходят
далеко за пределы только семейной истории. Поскольку на ее страницах содержатся
ценные для местных краеведов сведения. Это автографы артистов, которые
гастролировали в нашем городе. Каждый из автографов отмечен датой, аккуратно
помеченной рукой Константина Николаевича.

\ii{22_06_2019.stz.news.ua.mrpl_city.1.zapisnaja_knizhka_skripkina.pic.2}

\vspace{0.5cm}
\begin{minipage}{0.9\textwidth}
\textbf{Читайте также:}

\href{https://mrpl.city/news/view/k-pop-na-publike-mariupoltsev-priglashayut-stantsevat-po-yuzhnokorejski-video}{%
K-POP на публике: мариупольцев приглашают станцевать по-южнокорейски, Анастасія Селітріннікова, mrpl.city, 22.06.2019}
\end{minipage}
\vspace{0.5cm}

Нет, он был не из тех собирателей автографов, что пробираясь сквозь толпу, суют
знаменитостям открытку или клочок бумаги, лишь бы заполучить пару-тройку вкривь
и вкось нацарапанных слов. Коллекция автографов Скрипкина была результатом его
профессиональной деятельности. Константин Николаевич был директором нашего
Парка культуры и отдыха, работал инспектором в городском отделе культуры.
Именно он организовывал гастроли артистов в Летнем театре парка и на эстраде –
раковине. И встречи со знаменитостями были исключительно деловыми. Но обаяние,
умение свести сложности до минимума, тактичность Скрипкина вызывали у
гастролеров желание как-то выразить свою признательность в виде записи в его
книжке...

Ну что же, пора заглянуть в нее. 

\textbf{\enquote{В память о концерте в Жданове. Ю. Силантьев} (Летний театр, 20.05.64 год).}

\textbf{\enquote{Жак Дувалян} (16.06.65 год, Летний театр).}

И если представлять Юрия Силантьева, композитора и главного дирижёра
Эстрадно-симфонического оркестра Центрального телевидения и Всесоюзного радио
нет нужды, то о Жаке Дуваляне стоит поговорить. Он родился в 1920 году в Сирии.
Когда ему было пять лет, семья переехала в Париж. Способности к пению
проявились очень рано. В начале 40-х годов был принят в круг известных шансонье
. Однако в 1954 году отправился в советскую Армению. С грандиозным успехом
гастролировал по городам Советского Союза. Его мягкий баритон завораживал
публику, особенно женскую половину. Но в 1965 году он покидает СССР. Было
сделано все, чтобы Жака Дуваляна в Стране советов забыли.

\textbf{\enquote{Константин Николаевич. В память о наших встречах в Жданове. Лемешев} (28.09.63 год, Летний театр).}

Сергей Яковлевич Лемешев. Выдающийся оперный певец (тенор). Народный артист
СССР, лауреат Государственной премии, кавалер высших орденов страны, солист
Большого театра в Москве, популярность его была необыкновенна. 

\textbf{\enquote{На добрую память о концерте в Жданове. С наилучшими пожеланиями. Валерий
Климов} (20 мая 64 года).}

Валерий Александрович Климов – скрипач-виртуоз, народный артист СССР, лауреат
многочисленных международных конкурсов, в том числе и 1-го Международного
конкурса им. П. И. Чайковского (1-я премия). Выступал с концертами во всемирно
известных концертных залах Киева, Москвы, Нью-Йорка, Ленинграда, Лондона, Вены,
Берлина, Лейпцига, Сиднея и многих других крупных городов мира.

\textbf{Читайте также:} 

\href{https://mrpl.city/blogs/view/mariupol-fm-nova-ukrainska-muzika-myata-ta-ivan-navi}{%
Mariupol FM. Нова українська музика: М'ята та Іван Наві, mrpl.city, 15.05.2019}

\textbf{\enquote{Константину Николаевичу на добрую память от Радмилы Караклаич} (20.05. 64 год, Летний театр).} 

Радмила Караклаич, сербская эстрадная певица и актриса, народная артистка
Югославии, пик ее популярности в Советском Союзе пришелся на 60-80-е годы. Она
часто участвовала в эстрадных шоу Центрального телевидения.

\textbf{\enquote{Константину Николаевичу. В память о моих гастролях в г. Жданове. Очень замерзла. Ляля Черная} (2 мая 1964 года).}

Ляля Черная - популярная в 40-60-х годах актриса театра и кино, Заслуженная
артистка РСФСР, прима театра \enquote{Ромэн}, танцовщица, исполнительница цыганских
песен и романсов. Нужно сказать, что эти гастроли Ляли Черной в нашем городе
были далеко не первыми.

\textbf{\enquote{Уважаемый Константин Николаевич. От всей души желаю Вам больших успехов,
счастья и радости в жизни. Галина Олейниченко} (5 февраля 1972 года, Дворец
культуры Коксохимического завода).}

Галина Олейниченко, оперная певица, лирико-колоратурное сопрано, солистка
Большого театра, преподаватель, она сотрудничала с Ансамблем скрипачей Большого
театра под управлением Юлия Реентовича. Запись эту Галина Васильевна сделала
как раз во время ее гастролей с этим коллективом в нашем городе.

\textbf{\enquote{Константин Николаевич, с дружескими пожеланиями, всего доброго на свете. Колмановский} (17.10.73 года).}

Эдуард Колмановский, композитор, народный артист СССР, лауреат Государственной
премии СССР.  Автор музыки к драматическим спектаклям, музыкальным комедиям,
музыки к кинофильмам, мультфильмам, телефильмам, радиопостановкам, автор музыки
инструментальных пьес и крупных оркестровых сочинений. Он же автор более
двухсот песен, среди них - \enquote{Я люблю тебя, жизнь}, \enquote{Бирюсинка}, \enquote{Вальс о
вальсе}, \enquote{Бежит река} и другие.

Здесь приведены не все пожелания артистов из книжки Константину Николаевичу. Не
все гастролеры оставляли в ней свои автографы. Но и то, что осталось в ней –
ценный исторический документ о музыкальной жизни в Мариуполе.

\textbf{Читайте также:} 

\href{https://archive.org/details/30_06_2018.sergij_burov.mrpl_city.ijul_zhara_gastroli}{%
Июль, жара, гастроли, Сергей Буров, mrpl.city, 30.06.2018}
