% vim: keymap=russian-jcukenwin
%%beginhead 
 
%%file 11_03_2022.fb.petrov_sergij.harkiv.1.hroniki
%%parent 11_03_2022
 
%%url https://www.facebook.com/kharkivian/posts/5236488163069162
 
%%author_id petrov_sergij.harkiv
%%date 
 
%%tags 
%%title Харків. Хроніки атаки на місто, день 16 (11.03.2022)
 
%%endhead 
 
\subsection{Харків. Хроніки атаки на місто, день 16 (11.03.2022)}
\label{sec:11_03_2022.fb.petrov_sergij.harkiv.1.hroniki}
 
\Purl{https://www.facebook.com/kharkivian/posts/5236488163069162}
\ifcmt
 author_begin
   author_id petrov_sergij.harkiv
 author_end
\fi

Харків. Хроніки атаки на місто, день 16 (11.03.2022)

День почався з повітряної тривоги о 4-30 ранку. Коротше, улюблений час орків -
після побудки почати гатити по місту. Власне. сьогодні зранку і аж до
надвечір'я вони були збуджені. Традиційно найактиніше обстрілювали Північну та
номерну мікрорайонну Салтівку. Хоча сьогодні також дуже дісталося Дергачам і
частково Олексіївці з Павловим Полем. У Дергачах прильоти по лікарні та станції
швидкої допомоги, через що пошкоджені автівки та пробиті колеса у машин швидкої
допомоги. Пошкоджено будинок культури, приватні будинки. Є загиблі та поранені.
Зокрема, загинув колишній начальник Дергачівського райвідділу ще міліції та
колишній проректор ХНУВС. За чутками, був лютим руSSкомірцем. Бомби та снаряди
руSSкава міра вбивають без розбору. Прилітало і в інші райони - П'ятихатки,
Обрій (Горизонт), Рогань, ХТЗ.

Комплекс будівель Харківського фізтеху посічено осколками снарядів. Зокрема, і
будівлю ядерної установки \enquote{Джерело нейтронів}, яке є дослідницькою
установкою, де є 37 збірок ТВЕЛів. Проте сам реактор-установка зараз у безпеці,
не пошкоджена і заглушена. Московити цілеспрямовано намагаються створити в
Україні ще одну ядерну катастрофу, як це зробили у 1986 році, бо потрібно було
робити \enquote{план}.

Після відімкнення світла на П'ятихатках, де і знаходиться фіз-тех інститут,
світло там відновили. Але котельня і мережі теплопостачання зазнали критичних
руйнувань і не можуть бути відновлені до кінця війни. Ще кілька кварталів без
тепла на невизначений час....

Зауважу, що зараз у соцмережах дотримуються загалом тиші щодо місць обстрілів,
щоби не допомагати московитським військам і не корегувати їх вогонь. Тому
приходиться збирати інформацію буквально по крупицям, зокрема, і на слух.

Міська влада готова надати мешканцям районів, де ще довго не буде тепла
проживання у школах, які не зруйновані обстрілами, та в метро. Але близько 50
шкіл вже зазнали пошкоджень через обстріли... Така невтішна статистика.

Сьогодні обстріляли селище Високий, біля Харкова, - є пошкодження будинків, є
поранені. А поблизу Чугуєва розгромили колону руснявої техніки, яка їхала на
черговий штурм передмість міста. Більше всякого різного металобрухту!

Проте найгарячішими точками стали Ізюм та Балаклійська громада. Ізюм не
здається окупанту і сьогодні бої у місті поновилися з новою силою та
активністю. При цьому прилітало і в навколишні села, які за 10 км від міста.
Так, прилетіло в церкву РПЦвУ в с. Кам'янка Ізюмського району, що спричинило
деякі руйнування будівлі. Ясна річ, що це не дало змоги ані евакуювати людей,
ані завезти продовольство в місто без світла, тепла, води, їжі та ліків. Більше
того, колона автобусів, яка мала евакуйовувати мешканців Ізюма, ледь ноги
унесла від обстрілів московитів.

Зараз артилерія та авіація московитів методично працює по селам Мілова,
Петрівське та Гусарівка Балаклійської міської громади. Є пошкодження будинків,
є загиблі та поранені. Сама Балаклія в оточенні і не може допомогти у відправці
продовольства і ліків у села громади.

На жаль, на окупованих територіях починають зникати патріотичні українці - ФСБ
та Росгвардія намагаються знищити Рух Опору, гадаючи. що це справа рух
невеликої групки "нациків". У Мелітополі зник після викрадення московитами
міський голова міста. Містяни вийшли на акцію протесту проти цього.

Попри те. що тривають постійні бої та обстріли, але загальна ситуація на
Харківщині має ознаки перелому: московитська армія застрягла усюди в позиційних
боях і не може отримати хоча би якогось успіху. Єдиний успіх - розбомблені
міста, містечка та села в області. Загалом настрій в місті бойовий і бажання
дати якнайбільших пиздюлів свинособакам.

Ну а премію дарвіна сьогодні отримують оператори, очевидно, московитського
безпілотного літального апарату, який вони відправили замість с. Ярунь
Новоград-Волинського району у район Ярун міста Загреба, столиці Хорватії. Вся
справа в тому, що координати задаються внесенням назви населеного пункту. І ці
грамотєї просто переплутали назви - їм що Jarun (Ярун), що Yarun (Ярунь) - один
хєр. Але премію з ними розділяють військові Румунії, Угорщини та Хорватії, які
не помітили БПЛА, адже територією цих країн пролетів апарат до того як впав.
Звідки запустили його - поки невідомо. Захист НАТО? Ага, щаз!

Віримо у ЗСУ, НГУ, тероборону. Підтримуємо наших військових, волонтерів,
медиків, рятувальників та комунальників. Віримо у себе, підтримуймо рідних та
близьких, і наближаємо день нашої перемоги над московитами!!

\ii{11_03_2022.fb.petrov_sergij.harkiv.1.hroniki.cmtx}
