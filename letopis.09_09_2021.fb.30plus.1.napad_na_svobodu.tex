% vim: keymap=russian-jcukenwin
%%beginhead 
 
%%file 09_09_2021.fb.30plus.1.napad_na_svobodu
%%parent 09_09_2021
 
%%url https://www.facebook.com/movement30plus/posts/112880527791621
 
%%author_id 30plus
%%date 
 
%%tags leros_geo.ukraina.deputat,napadenie,svoboda,tischenko_nikolai.deputat,ukraina,verhovna_rada
%%title НАПАД НА СВОБОДУ
 
%%endhead 
 
\subsection{НАПАД НА СВОБОДУ}
\label{sec:09_09_2021.fb.30plus.1.napad_na_svobodu}
 
\Purl{https://www.facebook.com/movement30plus/posts/112880527791621}
\ifcmt
 author_begin
   author_id 30plus
 author_end
\fi

НАПАД НА СВОБОДУ.

Сьогодні у Верховній Раді України трапилась подія, яка змушує бити на сполох.
Народний депутат Микола Тищенко силоміць перервав виступ свого колеги Гео
Лероса, виштовхавши останнього від трибуни і затуливши доступ до неї.

\ifcmt
  ig https://scontent-frx5-1.xx.fbcdn.net/v/t1.6435-9/241629774_112880494458291_6881921204739655431_n.png?_nc_cat=110&ccb=1-5&_nc_sid=730e14&_nc_ohc=6FJ0ZG0HaeoAX-2FqpP&_nc_ht=scontent-frx5-1.xx&oh=df6b3c214c23b675fe90c7180c89e33d&oe=616321DB
  @width 0.4
  @wrap \InsertBoxR{0}
\fi

Свобода слова – одна із основних підвалин демократії. Як казав Джордж
Вашингтон, «якщо в людей забрати свободу слова, їх, німих та безмовних, можна
вести, як овець до різника». Парламент за означенням – це інституція, де
відбуваються суперечки та озвучуються різні точки зору, іноді у доволі
емоційній манері. Не даремно сама назва цього органу походить від французького
слова parler, яке в перекладі означає «говорити, розмовляти». Саме тому
Конституція України містить пряму норму, яка звільняє народних депутатів від
відповідальності за «висловлювання у парламенті та його органах, за винятком
відповідальності за образу чи наклеп». А отже затикання рота парламентарям – це
постріл в серце парламентаризму.

Якщо Україна, як задекларовано в Основному Законі, дійсно зробила свій
цивілізаційний вибір на користь Західної цивілізації, можновладці повинні
поважати плюралізм та свободу висловлювань, які за висловом Європейського суду
з прав людини, «є невідокремними від демократичного суспільства та за які
впродовж століть було заплачено велику ціну». Навіть якщо такі висловлювання їм
не подобаються чи ображають їхні почуття. Бо вільна публічна дискусія щодо
суспільних проблем та можливість критикувати владу – одні з небагатьох реальних
здобутків української демократії, які докорінно відрізняють нас від сусідніх
автократій.

Врешті-решт Зеленський прийшов до влади саме завдяки свободі нещадно
критикувати українських політиків на провідних телеканалах та концертних
майданчиках країни. Зважаючи на це, особливо цинічними виглядають спроби чинної
влади запровадити цензуру у ЗМІ та максимально зменшити простір свободи слова.
Це короткозора політика, яка повертає нас в «русский мир». 

Громадсько-політичний рух «Україна 30+» закликає можновладців припинити спроби
згортання прав і свобод загалом і свободи слова зокрема, до яких не перший раз
вдається чинна влада. Ми нагадуємо українським можновладцям, що українці не раз
довели: свобода є для нас засадничою цінністю і ми готові захищати її усіма
доступними засобами, зокрема і зі зброєю в руках.

Ми також закликаємо наших співгромадян належним чином реагувати на зазіхання на
наші «одвічні права і вольності», як казали наші волелюбні предки. Бо як казав
американський суддя Хенд, «Свобода лежить в серцях чоловіків та жінок; коли
вона помирає там, жодна конституція, жоден закон, жоден суд не здатні врятувати
її...».
