% vim: keymap=russian-jcukenwin
%%beginhead 
 
%%file 29_03_2022.fb.tregub_oleksandr.kiev.1.geroizm_on_hold
%%parent 29_03_2022
 
%%url https://www.facebook.com/alexander.tregub/posts/7525906227449612
 
%%author_id tregub_oleksandr.kiev
%%date 
 
%%tags 
%%title Вже нема ніякого героїзму в тому, щоб лишатись на холді. Повернутись — от що складно
 
%%endhead 
 
\subsection{Вже нема ніякого героїзму в тому, щоб лишатись на холді. Повернутись — от що складно}
\label{sec:29_03_2022.fb.tregub_oleksandr.kiev.1.geroizm_on_hold}
 
\Purl{https://www.facebook.com/alexander.tregub/posts/7525906227449612}
\ifcmt
 author_begin
   author_id tregub_oleksandr.kiev
 author_end
\fi

Першими ламались ті, хто вірив, що скоро от от все закінчиться. 

Після них — ті, хто не вірив в те, що це колись закінчиться.

Вижили ті, хто фокусувався на своїх діях, без очікувань про те, що може чи не
може статися.

\ii{29_03_2022.fb.tregub_oleksandr.kiev.1.geroizm_on_hold.pic.1}

Це спогади Віктора Франкла, психіатра, що вижив в нацистських концтаборах під
час другої світової. Ми зараз — в третій. 

Не варто думати, що війна от от закінчиться. Ще одна атака. Ще одна
Чорнобаївка. Ще одні перемовини. Маріуполь.

Гарні новини. Страшні новини.

Ніхто не знає, коли закінчиться війна.

Найкраще, що можна зробити зараз — сфокусуватись на своїх діях, та
перетворювати тимчасове життя на нормальне.

Облаштувати тимчасово орендовані оселі. Замінити зламані чайники. Пофіксити
інтернет. Повернутись до роботи. Повернути дітей до навчання. Повернути себе в
графік. Повернути собі вихідні. Сформувати нові звички. Почати читати і
вчитись. Продовжувати волонтерити та допомагати ЗСУ.

Підприємцям визнати нову реальність. Перевинайти свої проекти. Заробляти і
платити зарплати, відновити рекламу. Заморожений бізнес тепер працює проти
перемоги. 

Вже нема ніякого героїзму в тому, щоб лишатись на холді. Повернутись — от що
складно.

Коли ми переможемо, відкриється таке вікно можливостей, що від потоку кисню
перехоплюватиме подих. Зміниться все, а післявоєнний світ буде кращим, ніж
довоєнний. 

А поки позбавляємось зайвих очікувань, включаємось в роботу, облаштовуємось,
перетворюємо тимчасове на нормальне.

Та продовжуємо працювати на перемогу.

Слава Україні! @igg{fbicon.flag.ukraina}

\ii{29_03_2022.fb.tregub_oleksandr.kiev.1.geroizm_on_hold.cmt}
