% vim: keymap=russian-jcukenwin
%%beginhead 
 
%%file 03_01_2022.fb.zharkih_denis.1.pro_banderu
%%parent 03_01_2022
 
%%url https://www.facebook.com/permalink.php?story_fbid=3157789571101105&id=100006102787780
 
%%author_id zharkih_denis
%%date 
 
%%tags bandera_stepan,banderovec,nacionalizm,ukraina
%%title Про Бандеру и бандеровцев
 
%%endhead 
 
\subsection{Про Бандеру и бандеровцев}
\label{sec:03_01_2022.fb.zharkih_denis.1.pro_banderu}
 
\Purl{https://www.facebook.com/permalink.php?story_fbid=3157789571101105&id=100006102787780}
\ifcmt
 author_begin
   author_id zharkih_denis
 author_end
\fi

Про Бандеру и бандеровцев.

Хоть по специальности я не философ, но учился на философском факультете. Это
было еще при СССР. Факультет был идеологическим, а потому был под пристальным
вниманием партии и КГБ. Уже тогда я обращал внимание, что большинству
преподавателей и студентов этого важного идеологического факультета было
совершенно наплевать, что преподавать и что учить. 

Вместо ленининзма, советской философии и психологии многие преподаватели с
таким же успехом могли преподавать Тору, Коран, \enquote{Мою борьбу} Гитлера или
уфологию. Задача научить чему-нибудь студентов у них лично не стояла. Их
интересовали квартиры, загранпоездки, премии, ученые степени, которые тешили
самолюбие и повышали статус, кого-то интересовала возможность спать со
студентками, кого-то спокойно и вдумчиво бухать, кто-то просто не хотел
возвращаться в провинцию, а тем более, работать руками. Многих студентов наука
и учебный процесс не интересовали от слова \enquote{вообще}. Большинство просто хотело
пристроиться в столице, чтобы делать то, что делают преподаватели. Впрочем, об
этом я писал выше. 

Так вот, этой системе перейти с Ленина на Бандеру было раз плюнуть. Более того,
Бандера этой системе намного ближе Ленина не потому, что сильно болел за
украинскую независимость, а потому, что в сто раз проще. 

Ленина и коммунистов породили сложные и противоречивые исторические процессы, а
Бандера прост, как угол дома. Еще до развала Союза пошла тяга к примитиву, а уж
после него она стала главной не только в науке, но и в политике вообще. Если
идеология хорошо пожрать и пристроиться с минимальными усилиями скрывалась, ее
стыдились, то вскоре ей стали гордиться. 

Если раньше возможность пристроиться давал идол Ленина, то этот идол заменили
на Бандеру, только и всего. Но ведь я пишу об условных жрецах, тех, кто по
должности должен следить за святынями, тех, на ком держится мораль и интеллект
нации, что же говорить о простом обывателе? Социальная справедливость,
классовая солидарность, развитие трудящихся, если уж жрецы объявили это
глупостью, недостойной понимания, то обыватель спешит на все это нагадить,
показать свою исключительность себе и лояльность власти. Ведь бандеровщина
самая примитивная идеология, которую можно было найти вна Украине. Украина
превыше всего и смерть врагам. Даже Берия (пусть такой, какого его рисуют наши
либералы) для этой идеологии интеллигент очкастый. Ведь все проще простого -
дайте нам независимую Украину и свалите с нашей земли. Что непонятного?
Лаврентий Павлович от такого откровения просто офигел бы. 

И главное все понятно, пипл хавает, жрецам не надо заморачиваться, опять же
понятно, кто враг. А то Горбачев совсем в ступор ввел, заявив, что Запад не
конкурент, а пример для слепого подражания. Поэтому конкурируем не с другой
общественной системой, а с соотечественниками за кормушку, что понятнее и
привычнее. И тут Бандера очень пригодится, поскольку он как раз работал над
тем, кого и за что надо казнить ради Украины. 

Можно ли построить на этом страну? Нет, конечно, только разрушить.Социальная
организация намного сложнее. Кстати, Степан Андреевич ничего и не строил. Но на
разрушении страны политики зарабатывают больше, чем на ее строительстве. И за
Крым никто не воюет, и Донбасс им уже не интересен. Зато личные карманы
наполнили под завязку. Ленин не одобрил бы, Берия - тем более, а вот Бандеру
устроила весьма укороченная и безлюдная Украина. Собственно такую и строим -
теряем территории и людей, зато вовсю идет украинизация, закручивание поясов и
гаек. Вопрос не в Бандере, а в масштабе, в хуторянстве. 

Почему, например, Россия так легко допускает образцы украинских идеологов на
свое ТВ? Да знают они, что в нужный момент продадут они того Бандеру, а,
главное. Украину и украинский народ за хороший харч. В Брюсселе и Вашингтоне
продали, значит, и в Москве продадут. И тут Бандера удобен, поскольку
идеологическую машину легко развернуть и объявить его подонком. Но не пришло
время, курс обмена убеждений еще не тот. Ну так, подождем немного и увидим, как
они все перекрашиваться будут.

2020

\ii{03_01_2022.fb.zharkih_denis.1.pro_banderu.cmt}
