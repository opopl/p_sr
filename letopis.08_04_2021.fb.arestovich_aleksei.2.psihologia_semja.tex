% vim: keymap=russian-jcukenwin
%%beginhead 
 
%%file 08_04_2021.fb.arestovich_aleksei.2.psihologia_semja
%%parent 08_04_2021
 
%%url https://www.facebook.com/alexey.arestovich/posts/4217850624945652
 
%%author 
%%author_id 
%%author_url 
 
%%tags 
%%title 
 
%%endhead 

\subsection{С кем бы Вы предпочли общаться?}
\label{sec:08_04_2021.fb.arestovich_aleksei.2.psihologia_semja}
\Purl{https://www.facebook.com/alexey.arestovich/posts/4217850624945652}

\ifcmt
  pic https://scontent-bos3-1.xx.fbcdn.net/v/t1.6435-0/p180x540/170421030_4217849638279084_830556946440885643_n.jpg?_nc_cat=104&ccb=1-3&_nc_sid=8bfeb9&_nc_ohc=7ktt6Y3zK8sAX9X1Y-U&_nc_ht=scontent-bos3-1.xx&tp=6&oh=e21eaa9054a64e5f5be7e844b5fc4c75&oe=60999656
\fi

- С кем бы Вы предпочли общаться?
С человеком, который Вас ненавидит, и честно в этом признаётся, или с тем, кто Вас терпеть не может, но притворяется, что любит?..
А если это кто-то из Вашей семьи?..
Как часто мы боимся продемонстрировать свои настоящие чувства, предлагая людям, которые нас интересуют, некое модельное сверх-Я: призванное угадать их желания или требования общества?..
Это - очень глубокая потребность, потребность изображать, переполнившая социальную жизнь и жизненные сети, установка:
- не показывай своих истинных чувств - иначе будешь осмеян и унижен.
Кто из нас не переживал ситуации или страх быть высмеянным и осуждённым, за то, чем или кем он по-настоящему увлечён? За то, что по-настоящему нам важно и дорого?  
Конфликт, который нас разрывает, состоит в том, что мы одновременно хотим и предьявить своё настоящее, и спрятать его как можно дальше, чтобы его не растоптали - вместе с нами. 
Это - дилемма, но дилемма приобретает особо острый резонанс, когда тот, перед кем ты изображаешь успех - твой отец, который помнит тебя маленьким мальчишкой и знает все твои тайные и слабые места. 
Дауни-младший изображает преуспевающего, безжалостного адвоката и плейбоя и ему это нетрудно, потому, что он - действительно преуспевающий адвокат и плейбой.
Но отец, усадивший его, родного сына, в тюрьму зелёным, 17-летним пацаном, знает его главную тайну: герой Дауни - ещё и - маленький мальчик, который в браваде наделал непоправимых, страшных вещей, искалечивших его собственную семью, потом был вытолкнут ею прочь и живет с огромной, чёрной, никогда не заживающей дырой в душе, разрываемый чувством вины, гнева и обиды на отца, которые не унять никаким успехом, реальными достижениями и силой характера. 
Выдающийся драматизм этой истории заключается в том, что обстоятельства ставят отца-судью, ровно перед тем же выбором, что и сына:
- быть или изображать?..
Судья, ставший в маленьком городке символом чести и беспристрастия, обвиняется в преступлении, обвинительный приговор по которому, грозит отправить на помойку всю его праведную жизнь, все, во что он верил и как поступал, даже сажая в тюрьму собственного ребёнка.
Сын-адвокат берётся защитить отца, применяя головоломный арсенал крутого столичного адвоката, но отец не даёт ему этого делать, последовательно делая ставку на честность и честь.
Тяжелейшие моральные выборы, которые раз за разом, приходится делать Дауни-младшему, снова столкнувшись с отцовской безжалостностью и упрямством, с каждым новым эпизодом суда над судьей, увеличивают страшную дыру в его детской душе ещё больше, доводя конфликт с отцом: родственный, профессиональный, психологический до кульминации. 
Мы будем с вами решать на конкретном материале центральную проблему отношений человека с иными людьми:
- быть или казаться?..
Разбирать - откуда взялся великий соблазн «казаться», как он возник, чего требует, и сколько воли и мужествам требуется, чтобы решиться быть.
И Вы, войдя в этот семинар, уже не выйдете из него прежними.
17 апреля.

Регистрация на кинотренинг:
 \url{https://forms.gle/aPjT2reXZSEdswL87} (українською)
 Регистрация на кинотренинг:
 \url{https://forms.gle/pqFbFxKwmKDE42ev9} (русский язык)
 Контакты администратора / Контакты администратора:
Facebook: \url{https://www.facebook.com/kate.ksiondzyk/}
Email: \url{kateksiondzyk@apeiron.school}
Телефон: \verb|+38 (063) 0680165|
