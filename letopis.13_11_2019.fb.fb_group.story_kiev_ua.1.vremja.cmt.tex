% vim: keymap=russian-jcukenwin
%%beginhead 
 
%%file 13_11_2019.fb.fb_group.story_kiev_ua.1.vremja.cmt
%%parent 13_11_2019.fb.fb_group.story_kiev_ua.1.vremja
 
%%url 
 
%%author_id 
%%date 
 
%%tags 
%%title 
 
%%endhead 
\zzSecCmt

\begin{itemize} % {
\iusr{Олег Коваль}
Спасибо, Виктория! @igg{fbicon.heart.red}

\begin{itemize} % {
\iusr{Виктория Угрюмова}
@igg{fbicon.heart.red}

\iusr{Владимир Ильенко}
Огромное спасибо, Виктория! Нет слов.......
\end{itemize} % }

\iusr{Валерій Соловйов}
Спасибо за слова, которые вдохновляют...

\iusr{Виктория Угрюмова}
Благодарю Вас

\iusr{Неонила Сваток}
Спасибо!

\iusr{Елена Малышко}
\enquote{Слезы, которым 110 лет...} Удивительный образ...

\begin{itemize} % {
\iusr{Виктория Угрюмова}
спасибо, да, странно видеть такую эфемериду, пережившую того, кто их уронил уже на полвека

\iusr{Елена Малышко}
\textbf{Виктория Угрюмова} Все, что Вы пишете о Прошлом, не просто трогает
душу, а рвет ее на части - каждое слово в точку!

\iusr{Виктория Угрюмова}
\textbf{Елена Малышко} Это высокая честь - иметь таких читателей. Благодарю @igg{fbicon.sun.with.face} 
\end{itemize} % }

\iusr{Ніна Рудік}
Бл

\iusr{Ніна Рудік}
Супер

\iusr{Oksana Shukayeva}
Спасибо за эти трогательные истории о киевлянах... нас осталось мало, но мы
понимаем лруг друга с полуслова)))

\iusr{Татьяна Гуменюк}

\ifcmt
  ig https://scontent-frx5-2.xx.fbcdn.net/v/t39.1997-6/s480x480/14050144_1775288802711824_1454378351_n.png?_nc_cat=1&ccb=1-5&_nc_sid=0572db&_nc_ohc=ZtylVc2Bi5kAX_VpXB7&_nc_ht=scontent-frx5-2.xx&oh=00_AT8OXVO1o9AiPpgXxzvN2MKF_GVCLAimKG1e-X12xcHbKw&oe=61C470D5
  @width 0.2
\fi

\iusr{Виктория Угрюмова}

\ifcmt
  ig https://scontent-frx5-2.xx.fbcdn.net/v/t39.1997-6/s168x128/16180605_157616578064658_2801258149548392448_n.png?_nc_cat=1&ccb=1-5&_nc_sid=ac3552&_nc_ohc=2F_5N_-4hOUAX8ligD1&_nc_ht=scontent-frx5-2.xx&oh=00_AT9Or2znXVQ4pc41ift6TXKY45y46_QDIfmjdohPYf84pA&oe=61C433B4
  @width 0.2
\fi

\iusr{Татьяна Панкова-Балог}
Спасибо за слова и память, за мой Город. Я была маленькой в 80 ушедшего века.
Но мне дорог мой Киев. Тот прежний...

\iusr{Владимир Мнишенко}

Спасибо, так написал бы Михаил Александрович об Эмилии Львовне, если бы сегодня
мог отправить эту открытку ...

\iusr{Виктория Угрюмова}
Благодарю, это высочайшая оценка

\iusr{Елена Романская}

Спасибо за ваши рассказы, за вашу дружбу. И отдельное спасибо Hélg Smith за
идею и создание группы. Благодаря чему я познакомилась с людьми, которые ходят
рядом, но не имеют возможности узнать друг друга.


\iusr{Олег Коваль}

Елена, Спасибо Вам!@igg{fbicon.heart.red}

\iusr{Renata Tereshchenko}

Спасибо всем. Я все ещё здесь, и мне интересно, и радостно, что страница
сохраняет своё предназначение. Благодаря вам я вспоминаю Город, тот, настоящий.

\iusr{Виктория Угрюмова}
 @igg{fbicon.bird}{repeat=7}  @igg{fbicon.heart.red} ️ 

\iusr{Renata Tereshchenko}
\textbf{Виктория Угрюмова}  @igg{fbicon.face.smiling.eyes.smiling} 

\iusr{Olena Ivanenko}

Спасибо, Виктория, за Ваши бесценные семейные истории, за Ваши старания и их
теплоту, за то, что и мы все своими открытками, фотографиями, воспоминаниями из
сундуков и шкатулок и просто рассказами, будто маленькими кирпичиками,
складываем здание нашего большого Города....

\iusr{Виктория Угрюмова}
Огромное спасибо)

\iusr{Лиля Дубинская}

Правда? Это кому-то важно? И нужно рассказать о них, любимых, дорогих, как была
уверена, только моему сердцу?

\begin{itemize} % {
\iusr{Надія Сокольчук}
Мені, наприклад, важливо  @igg{fbicon.smile}. Якщо звісно хочеш розповісти)

\iusr{Лиля Дубинская}
Записувала для сина. Не вважала цікавим ще для когось. Ризикну. Розповім.

\iusr{Виктория Угрюмова}

Певна, що ви десь в душі знаєте відпвідь - це дуже важливо. Кожному з нас, вам,
їм та світу. @igg{fbicon.heart.red} У мене є власна теорія (власне, це дуже давня та не моя
теорія)))) - все, про що ми кажемо вголос, все, що ми записуємо - воно назавжди
залишається та існує у Всесвіті, вже незалежно від нас. Розповідаючи з любов'ю,
ви \enquote{ризикуєте} поселити пам'ять про ваших дорогих у нашіх серцях та подарувати
їм нове життя - у інших спогадах. Ви знаєте - я весь час повторюю розповіі про
зовсім чужих меня людей - але тих, кого тепер, ніхто не пам'ятає, крім мене. Не
пам'ятав - до минулого року. І з практичного натхнення - так трапилось, що ці
розповіді стали статтями в багатьох енциклопедіях, на багатьох сайтах,
присвячених певним темам - і поширилися на весь такий невеликий тепер світ. І я
щаслива. Бо тепер я не заберу ці спогади з собою - вони живуть без мене.
Уявіть, яке це щастя!

\end{itemize} % }

\iusr{Лариса Ковалева}

Спасибо Вам, Виктория, за такое трогательное, доброе, пронизанное заботой
отношение к истории своей семьи, своего города. Как порадовали меня строки о
том, что если мы будем помнить о наших предках и важных событиях, датах, мы
сделаем их бессмертными, и время узнает о своем бесправии. Золотые слова! Глядя
на Сarte Postale, вспомнила об открытке своего прадеда с передовой, в Первую
мировую войну, 1914 года. Захотелось подобно Вам поделиться частицей истории
своей семьи, хоть живу и не в Киеве. Еще раз большое, искреннее Вам спасибо!

\begin{itemize} % {
\iusr{Виктория Угрюмова}

Делитесь непременно! Самое невероятное происходит в таких случаях - знали бы
вы, сколько чудес просыпалось на меня дождем с тех пор, как я стала говорить об
этом вслух - сколько подарков в жанре чуда. Знаете, я все время повторяю,
рискуя всем надоесть - когда мы вспоминаем о них, они вдруг будто бы
откликаются из невообразимой дали, и начинают говорить о себе - и всплывают
чудесным образом вещи, которые воим усердием, своим произволением не узнал бы.
Так, вообразите, вывешиваю очередную историю о своих пращурах по материнской
линии и говорю мужу - как странно и печально: этих я знаю до 9 века, сколько
историй, вал - а отцовских - никого (все расстреляны были, и отец умер, когда я
была маленькой - от бабушки и дедушки с той стороны - только два портрета,
которые отец писал по памяти); о прадедах молчу - только одна прапрабабка
легендарная чудом известна. И в тот же день ищу материал по текущей работе, а
компьютер зависает, тарахтит и выкидывает мне гугл-страницу не по запросу, а
какую-то "левую". Но я уже привыкла думать, что случайностей не бывает - если
так, надо смотреть. Смотрю - вторая ссылка - мой прадед по отцовской линии и
трое его сыновей, моих двоюродных дедов. Послужные списки - а это полностью
информация - сухая, скупая, но 12 листов на каждого. А за час до этого - вообще
ничего. Так что даже с этой точки зрения говорить и писать о них - практично,
разумно и чудесно)))

\iusr{Irena Visochan}

Виктория! Спасибо Вам большое за все, рассказанные Вами истории, пронизанные
любовью и теплом! Обязательно пишите!!! @igg{fbicon.hearts.two} 

\end{itemize} % }

\iusr{Юлія Конопко}
Надо-же. Привет Виктория!

\begin{itemize} % {
\iusr{Виктория Угрюмова}
Ух ты! Действительно - надо же!!!!! Здравствуй))))

\iusr{Юлія Конопко}
У меня сейчас очень слабый инет. Через десять дней буду дома и напрошусь в друзья)
\end{itemize} % }

\iusr{Natasha Levitskaya}

Спасибо, Виктория! Спасибо за вашу любовь к Городу, за ваши воспоминания о
ваших близких, о людях, живших рядом с вами, о временах, когда были другие
ценности и отношения людей, о прежнем Киеве... В памяти остался и очень тронул
ваш душевный рассказал про молоко, булку и трогательную пару, когда старик
остался один... С возрастом понимаешь, как важно знать историю семьи, как мало
мы расспрашивали у своих бабушек и дедушек...

У меня нет Киевской истории, я \enquote{купаюсь} в ваших воспоминаниях, но пока жива
моя мамочка - ей 93 и она в совершенно здравом уме и светлейший памяти, я
записала всё из её рассказов обо всех, о временах прошлых. Теперь рассказываю
своим детям, двоюродным и их детям-внукам. Спасибо вам и всем, кто делится
своими историями! @igg{fbicon.heart.red}

\begin{itemize} % {
\iusr{Виктория Угрюмова}

Какая вы умница - записывайте маму - эти истории на самом деле неисчерпаемы -
спрашивайте - про моду и еду, про институтские вечера, концертыи любимых
артистов, и про все-все))) это счастье, что мама есть и дай ей Бог долгих и
здоровых лет


\iusr{Natasha Levitskaya}
\textbf{Виктория Угрюмова} Спасибо, Виктория!!!

\iusr{Татьяна Гурьева}
какая Вы умница, я тоже такие истории обожаю!
\end{itemize} % }

\iusr{Журкова Татьяна}

Я думаю во многих киевских семьях хранятся такие воспоминания. Папа написал 73
года назад маме в роддом трогательное письмо. Он благодарил за дочку, а также
отчитался про то, что делают мальчики и как они управляются без нее. На карточки
на мясо давали селёдку. Это была жизнь киевлян в 46 году.

\begin{itemize} % {
\iusr{Виктория Угрюмова}

Вот это бечсценно, я думаю - и если бы вы показали нам такую записочку, это
было бы восхитительно. Я ведь о том и говорю, что у каждого есть огромная
семейная сага. И смысла в ковырнадцатый раз постить сообщение о том, что на
месте памятника - навскидку - княгине Ольге был памятник Княгине Ольге нет
смысла, есть смысл вывешивать что-то уникальное, вот такое, про селедку в 1946.
У меня есть такие записочки из 1936 года - карандан почти стерся - бумага
интербеллума и послевоенная - худшая из всех, листки разлетаются в руках, там
дедушка писал про цену на апельсины и что идет в кино. Том Сойер. в октября
1936 года в кино показывали Тома Сойера

\iusr{Журкова Татьяна}
\textbf{Виктория Угрюмова} 

я попрошу сына, чтобы он мне помог выложить это письмо. Может в воскресенье.
Только чернила на письме уже выгорели, но читать можно, и на сгибах бумага очень
ветхая.

\iusr{Татьяна Гурьева}
действительно, это и есть сокровища группы! Неслыханное богатство!
\end{itemize} % }

\iusr{Егор Красава Тебякин Егор}
Спасибо за Ваши интересные публикации.@igg{fbicon.heart.red}

\iusr{Tetyana Zerova-Lyubimova}

\ifcmt
  ig https://scontent-frx5-2.xx.fbcdn.net/v/t39.1997-6/s168x128/17527812_1652591011433966_4391041969200037888_n.png?_nc_cat=1&ccb=1-5&_nc_sid=ac3552&_nc_ohc=SxmAARtd56gAX-GdNsX&_nc_ht=scontent-frx5-2.xx&oh=00_AT_6O67Z2-RGcCu0PbN_JQWqIB0diX5aAGtxzGWaJe6eEw&oe=61C44498
  @width 0.1
\fi

\iusr{Tetyana Zerova-Lyubimova}

Як Ви гарно написали! Саме такі ж відчутт переживаю і я, коли читаю дописи. У
кожного з нас є сімейні історії, і коли ділишся ними, наче знову переживаєш,
поринаєш у ті часи. Але є ще в історіях ті суттєві моменти, які нас об’єднують
- комунальні квартири, дворики, вулиці Києва.

\begin{itemize} % {
\iusr{Виктория Угрюмова}

так, оце дуже хочу написати про землетрус 1977 - якраз переживала його на
Музйному провулку у комуналці

\iusr{Tetyana Zerova-Lyubimova}
\textbf{Виктория Угрюмова} 

чекаємо! гарно його пам'ятаю - наче вчора. Успіхів!!! Ви неймовірно пишете, з
душею!
\end{itemize} % }

\iusr{Елена Лутченко}
 @igg{fbicon.hearts.revolving} 

\iusr{Марина Соболевская}

Дорогая Викочка! Вы даже не представляете, как нужны Ваши слова. Как нужна
добрая поддержка человека, умеющего объяснить, насколько все это - важные вещи.
Иногда опускаются руки от трудностей в мире поисков и историй. А вы вдохновенно
превратили эти истории в бесценные бриллианты. Спасибо!!!)

\begin{itemize} % {
\iusr{Виктория Угрюмова}

Мариночка! Огромное спасибо! вы же знаете, главное в рассказе - это слушатели,
в книге - читатели. Если нет их восприятия, нетх их души, понимания и доброй
воли - нет ничего. Такой тонкий читатель и пониматель, как вы - это мечта
любого рассказчика

\end{itemize} % }

\iusr{Елена Лутченко}

Я познакомилась с этим прекрасным городом в сознательном возрасте, мамочка
родилась в Киеве, война началась, когда ей было 10 лет, мама в это лето гостила у
родного дяди, в Житомирской области, ее воспоминания того дня...,, Мы с подружками
прыгали на скакалке, когда налетели фашистские самолёты, все поб ежали в подвал, а
мама растерялась, самолёт летел так низко, и пулемёт строчил, что шагнуть в
сторону было невыносимо страшно, запомнился летчик и его оскал, ему доставляло
удовольствие запугивать детей... Случайно или он просто дал им шанс, дети
остались живы, но улыбка наслаждения осталась в детской памяти
навсегда... Извените, что не о Киеве, но о девочке, моей маме, которая жила в этом
удивительном городе, за который пошел воевать ее отец, а мой дорогой дедушка и
дошел до Берлина, и я горжусь этим.

\begin{itemize} % {
\iusr{Виктория Угрюмова}

Я думаю - это и есть - о Киеве, потому что все откуда-то здесь и куда-то тоже.
Я рассказываю часто о людях, которых носило по всему свету - их нельзя назвать
киевлянами формально - например, одна из моих прапрабабушек жила в Киеве и
закончила Киевский институт Благородных девиц, и уехала с мужем - а вернулась
уже очень-очень в летах... киевлян разбросало по свету, но раз мы собрались
тут, значит, киевляне

\end{itemize} % }

\iusr{Александра Пихур}
СПАСИБО

\iusr{Виктория Угрюмова}

\ifcmt
  ig https://i2.paste.pics/891f650061f770212ba0f31512046336.png
  @width 0.2
\fi


\iusr{Ольга Соловьёва}
Спасибо.

\iusr{Наталия Вовченко}

Спасибо, всегда восхищалась людьми, умеющими вдохновенно и не ординарно
высказать свои мысли!

\iusr{Irena Visochan}
Спасибо дорогие читаю до слез.

\iusr{Ковальская Татьяна}

У меня сохранились такие открытки, даже есть первая моей тети к сестре в первые
дни после блокады Ленинграда, как люди переживали первые дни! Такие письма
могут многое рассказать и вспомнить страшные дни!


\iusr{Евгения Бочковская}

Спасибо Вам, Виктория, за Ваши, такие всегда интересные, рассказы! За окунание
в дух того прошлого времени, о котором Вы так тепло всегда рассказываете... И
за ту память, которая возвращает нас в наше прошлое, когда мы читаем Ваши
рассказы...  @igg{fbicon.heart.sparkling} 

\iusr{Виктория Угрюмова}
Спасибо огромное

\iusr{Olena Klymenko}

Такие душевные и трогательные строки! Спасибо Вам, Виктория! Очень люблю читать
такие интересные КИ.@igg{fbicon.heart.red}

\iusr{Виктория Угрюмова}

\ifcmt
  ig https://scontent-frt3-1.xx.fbcdn.net/v/t39.1997-6/p480x480/67914419_1097865693746130_5344278843509178368_n.png?_nc_cat=108&ccb=1-5&_nc_sid=0572db&_nc_ohc=Uz-nTTJFbCoAX9vly9S&_nc_ht=scontent-frt3-1.xx&oh=00_AT82KzFKb4qQ3aCdanKdzDP1haWZUmSdMZIGDy0p0flFsw&oe=61C39BC0
  @width 0.2
\fi

\iusr{Татьяна Рыбалко}
Спасибо !

\ifcmt
  ig https://scontent-frx5-2.xx.fbcdn.net/v/t39.1997-6/p370x247/69142145_2155530931224624_1312636185700466688_n.png?_nc_cat=1&ccb=1-5&_nc_sid=0572db&_nc_ohc=IPkfRcrSWrEAX85mCmf&_nc_ht=scontent-frx5-2.xx&oh=00_AT_Nu5loqlu-Og1NcAy2EQak4eRIMj2hChNagtW56RhH4g&oe=61C3EBA8
  @width 0.2
\fi

\iusr{Виктория Угрюмова}

\ifcmt
  ig https://scontent-frx5-2.xx.fbcdn.net/v/t39.1997-6/s168x128/16180605_157616578064658_2801258149548392448_n.png?_nc_cat=1&ccb=1-5&_nc_sid=ac3552&_nc_ohc=2F_5N_-4hOUAX8ligD1&_nc_ht=scontent-frx5-2.xx&oh=00_AT9Or2znXVQ4pc41ift6TXKY45y46_QDIfmjdohPYf84pA&oe=61C433B4
  @width 0.2
\fi



\end{itemize} % }
