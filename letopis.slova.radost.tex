% vim: keymap=russian-jcukenwin
%%beginhead 
 
%%file slova.radost
%%parent slova
 
%%url 
 
%%author 
%%author_id 
%%author_url 
 
%%tags 
%%title 
 
%%endhead 
\chapter{Радость}
\label{sec:slova.radost}

%%%cit
%%%cit_head
Радость-Россия
%%%cit_pic
%%%cit_text
Находим и \emph{радуемся}:
\begin{itemize}
  \item 1. Сельское хозяйство. Экспорт пшеницы занимает 1 место в мире. В 2020 году впервые экспорт продовольствия за рубеж превысил импорт;
  \item 2. Наша страна является единственной, где биоресурсы не только восстанавливаются, но и растут. Это животный и растительный мир;
  \item 3. Самые большие запасы пресной воды на Земле - у нас. Это одно из основных преимуществ, и в скором времени, думаю, станет основной претензией к нам;
  \item 4. Россия обладает самыми большими месторождениями полезных ископаемых. Их разнообразие тоже впечатляет. Неуклонно растёт доля перерабатывающих предприятий, что способствует развитию экономики страны;
  \item 5. По выработке электро- и других видов энергии наша страна занимает третье место в мире;
  \item 6. В нашей стране очень развита металлургическая промышленность. По производству стали - 5 место в мире, по производству некоторых цветных металлов - первое
\end{itemize}
%%%cit_comment
%%%cit_title
  
%%%endcit

%%%cit
%%%cit_head
%%%cit_pic
%%%cit_text
Не можу \emph{нарадуватися} ефекту від повернення Нью-Йорку історичної назви.
Рідко так приємно читати палання на московських ресурсах. Безпорадна лють у них
там, бо що вони скажуть?  Звісно, спершу завели свою шарманку \enquote{краще б
пенсії підвищили}. Але це мимо, бо тим же пенсіонерам Нью-Йорку це плюс: через
екзотичну для наших широт назву зростає потік відвідувачів, а відтак більше
можна заробити на продажу товарів і послуг.  Мабуть, хотіли б представити це як
черговий прогин перед Заходом, але і це мимо. Це – історична назва, котру якраз
в угоду ідеологічним міркуванням взяли і поміняли на ісконно руску. Факти – річ
уперта
%%%cit_comment
%%%cit_title
\citTitle{Український Нью-Йорк розгромив міф про рускій Донбас}, 
Ігор Луценко, gazeta.ua, 07.07.2021
%%%endcit

%%%cit
%%%cit_head
%%%cit_pic
%%%cit_text
Все справжнє й велике — просте, як дихання. Коли я на Карпатах проводив третій
тиждень голодування... правда, це вже для мене було не те поняття... але ви
розумієте! Так ось... В якусь мить мене осяяла \emph{радість}! Безпричинна,
дивна, світозарна, потрясаюча, всепроникаюча… Її не можна віднести до
визначення насолоди, задоволення, щасливого настрою. Це щось звільняюче,
окриляюче, таке, що дає нам певність у нашій вічності, у нашій всеєдності з
Природою. Я тоді сміявся, мов дитя... Ви бачили, як у сні безхмарно всміхаються
немовлята? Є навіть народне прислів’я: «Немовля сміється — з янголами
грається».  \emph{Сміх-радість} — то ключ до замків усього вселенського життя.
І тут я збагнув...
%%%cit_comment
%%%cit_title
\citTitle{Вогнесміх}, Олесь Бердник
%%%endcit

%%%cit
%%%cit_head
%%%cit_pic
%%%cit_text
Мабуть, він один з усіх присутніх не зволив повернути голову туди, де
сперечалися прислужник з Коппенолем. І от, волею випадку, панчіш-ник з Гента,
до якого народ уже відчув палку симпатію і до якого тепер звернулись усі
погляди, сів у першому ряді на помості, саме під жебраком. Яке ж було загальне
здивування, коли фламандський посол, глянувши на цього пройдисвіта, дружньо
поплескав його по вкритому лахміттям плечу. Жебрак обернувся. Обидва
здивувались, упізнали один одного, обличчя обох засяяли від \emph{радості}, потім,
зовсім не зважаючи на глядачів, панчішник і жебрак, тримаючись за руки, почали
перешіптуватися, причому лахміття Клопена Труйльфу, розкинуте по золотавій
парчі помосту, було схоже на гусінь на апельсині.
Незвичайність цієї дивної сцени викликала такий вибух нестримних веселощів і
радості, що кардинал одразу ж звернув на це увагу; він трохи нахилився, але із
свого місця міг розгледіти тільки лахміття Труйльфу. Вирішивши, що жебрак
просить милостиню, і обурившись із такого зухвальства, кардинал крикнув
%%%cit_comment
%%%cit_title
\citTitle{Собор Паризької Богоматері}, Віктор Гюго
%%%endcit

%%%cit
%%%cit_head
%%%cit_pic
%%%cit_text
Малыш был очень \emph{рад}, что познакомился с Карлсоном. Как только Карлсон прилетал,
начинались необычайные приключения. Карлсону, должно быть, тоже было приятно
познакомиться с Малышом. Ведь что ни говори, а не очень-то уютно жить одному в
маленьком домике, да ещё в таком, о котором никто и не слышал. Грустно, если
некому крикнуть: «Привет, Карлсон!», когда ты пролетаешь мимо.
Их знакомство произошло в один из тех неудачных, дней, когда быть Малышом не
доставляло никакой \emph{радости}, хотя обычно быть Малышом чудесно. Ведь Малыш —
любимец всей семьи, и каждый балует его как только может. Но в тот день всё шло
шиворот-навыворот. Мама выругала его за то, что он опять разорвал штаны, Бетан
крикнула ему: «Вытри нос!», а папа рассердился, потому что Малыш поздно пришёл
из школы
%%%cit_comment
%%%cit_title
\citTitle{Карлсон, который живёт на крыше}, Астрид Линдгрен
%%%endcit

%%%cit
%%%cit_head
%%%cit_pic
%%%cit_text
Но даже те, кто \emph{радостно} принимает новую гуманитарную политику, не сильно от
них отличаются. Все без исключения граждане рассматриваются правящей верхушкой
как ресурс для «дойки» в своих личных интересах и в интересах международных
«партнёров». Украинцы лишь платят и платят, ничего не получая от государства
взамен, без возможности как-то на это повлиять и что-то изменить. С одной
стороны власть в виде клоунов, мародеров и клоунов-мародеров, с другой –
нанятая Ахметовым опереточная оппозиция, Порошенко, Тимошенко.  Какой именно
джавелиний нужно дать этим гражданам Украины, чтобы они пошли умирать за эту
страну?
%%%cit_comment
%%%cit_title
\citTitle{Чувствуют ли граждане Украины эту страну своей собственной? / Лента соцсетей / Страна}, 
Вячеслав Чечило, strana.news, 06.12.2021
%%%cit_url
\href{https://strana.news/opinions/365636-chuvstvujut-li-hrazhdane-ukrainy-etu-stranu-svoej.html}{link}
%%%endcit
