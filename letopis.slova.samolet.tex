% vim: keymap=russian-jcukenwin
%%beginhead 
 
%%file slova.samolet
%%parent slova
 
%%url 
 
%%author 
%%author_id 
%%author_url 
 
%%tags 
%%title 
 
%%endhead 
\chapter{Самолет}
\label{sec:slova.samolet}

%%%cit
%%%cit_head
%%%cit_pic
%%%cit_text
На ГП \enquote{Харьковское государственное \emph{авиационное} производственное
предприятие} официально работают 1580 человек, однако более тысячи из них уже
давно не появляются на заводе.  Об этом рассказал генеральный директор
предприятия Александр Кривоконь в эфире телеканала Simon.  По его словам, самой
большой проблемой является фактическое отсутствие специалистов, которые при
необходимости смогли бы собирать \emph{самолеты}.  \enquote{Другие
руководители, которые говорили, что мы можем что–то строить... Не можем мы
ничего строить сегодня. У нас нет сегодня тех людей, того количества и той
квалификации людей, которые смогут сегодня строить \emph{самолеты}}, - сказал
он
%%%cit_comment
%%%cit_title
\citTitle{ХГАПП больше не может производить самолеты}, Антон Щукин, strana.ua, 27.06.2021
%%%endcit

%%%cit
%%%cit_head
%%%cit_pic
%%%cit_text
А потом, перелет из Нью-Йорка в Вашингтон продолжается всего час. \emph{Самолет}
отправляется с нью-йоркской взлетной полосы N° 5, которая, как и все
государственные взлетные полосы, подобающим образом защищена. Выход в открытую
атмосферу происходит только по достижении скорости полета. А сядут они на
вашингтонской второй полосе, тоже укрытой.  Более того, Бейли хорошо знал, что
в \emph{самолете} не будет окон. Там хорошее освещение, приличная еда, все
необходимые удобства. Полет, управляемый по радио, пройдет гладко. Вряд ли
движение будет вообще чувствоваться, когда самолет поднимется в воздух
%%%cit_comment
%%%cit_title
\citTitle{Обнаженное солнце}, Айзек Азимов
%%%endcit


