% vim: keymap=russian-jcukenwin
%%beginhead 
 
%%file 15_12_2020.news.ua.strana.1.nikto_ne_zhdet_voz
%%parent 15_12_2020
 
%%url https://strana.ua/news/306084-sputnik-v-vaktsina-proshla-li-tretju-fazu-ispytanij-kakimi-budut-issledovanija-s-astrazeneka.html
 
%%author 
%%author_id 
%%author_url 
 
%%tags vaccine,sputnik_v,russia,ukraina
%%title Никто не ждет ВОЗ. Прошла ли российская вакцина "Спутник" третью фазу испытаний и чем она отличается от Pfizer
 
%%endhead 
 
\subsection{Никто не ждет ВОЗ. Прошла ли российская вакцина \enquote{Спутник} третью фазу испытаний и чем она отличается от Pfizer}
\label{sec:15_12_2020.news.ua.strana.1.nikto_ne_zhdet_voz}
\Purl{https://strana.ua/news/306084-sputnik-v-vaktsina-proshla-li-tretju-fazu-ispytanij-kakimi-budut-issledovanija-s-astrazeneka.html}

\ifcmt
  pic https://strana.ua/img/article/3060/sputnik-v-vaktsina-84_main.jpeg
  width 0.5
  fig_env wrapfigure
\fi

\index[rus]{Коронавирус!Украина!Вакцины Спутник V, Pfizer}

С 10 декабря по всей России началась вакцинация от коронавируса. Пока прививают
врачей, учителей, военных, социальных работников, продавцов. 

Группы риска имеют приоритет. Там, где вакцины завезли достаточно, бесплатно
иммунизируют всех желающих россиян в порядке живой очереди. 

В Украине же заявляет, что российская вакцина не прошла третью фазу испытаний,
а потому не может считаться безопасной. И вообще нельзя говорить о том, что
такая вакцина существует. Ряд украинских СМИ сами слова "российская вакцина" и
"вакцинация" (применительно к России) берут в кавычки. 

"Страна" выяснила, как обстоит дело с третьей стадией исследований "Спутника V"
на самом деле. 

\subsubsection{Степанов: \enquote{Российской вакцины не существует} }

Тезис о том, что Украина не будет покупать российскую вакцину в виду того, что
она не прошла третью стадию испытаний, глава МОЗ Степанов повторял
неоднократно.

"О российской вакцине я говорил неоднократно, что не существует на сегодняшний
день российской вакцины. Когда говорят, что зарегистрировали вакцину, которая
не прошла третью стадию, то не о чем говорить. Третья стадия - самая большая,
при которой испытывается вакцина, в том числе с точки зрения ее безопасности",
- сказал Степанов. 

Глава комитета Рады по здравоохранению Михаил Радуцкий заявил, что о
производстве в Украине российской вакцины (этот вопрос поднимался на прошлой
неделе) речь не идет. 

"Сегодня не существует российской вакцины против коронавируса. "Спутник V" -
это кандидат, и таких кандидатов в мире очень много. Для меня, и для Украины, и
для любой страны вакциной может называться только то, что имеет авторизацию ВОЗ
- вакцина, которая прошла сертификацию и регистрацию в странах, где очень
строгая регистрационная политика. Это страны ЕС, США, Япония, Швейцарская
конфедерация... Я не слышал, что в какой-то из этих стран была зарегистрирована
или авторизована российская вакцина. Скорее всего, это кандидат, но это не
вакцина, поэтому сегодня производить то, чего не существует, невозможно", -
сказал Радуцкий.

Так прошла ли российская вакцина третью - последнюю - фазу испытаний? 

\subsubsection{Что такое третья фаза испытаний и кто ее на самом деле прошел? }

Кратко опишем, что такое третья фаза испытаний вакцины. Это исследование на
широких массах населения, на десятках тысяч человек.

Организуется во всех странах по одному принципу: часть добровольцев получает
плацебо, часть - настоящий препарат. Уколы вводятся с перерывом в 21-28 дней. А
иммунный ответ - выработка антител к ковиду - случается или нет в среднем через
20 дней. 

Эти данные взвешиваются, после чего фиксируется эффективность. Чем больше
добровольцев вышли из испытаний с антителами, тем лучше препарат.

На качество влияют и другие факторы. Например, побочные эффекты. В том числе
для этого части испытуемых котят пустышку - чтобы умышленно или случайно люди
не выискивали у себя ложные симптомы после укола.

Ну, и еще два фактора: как долго антитела будут сохраняться в организме и не
отразится ли плохо вакцина на организме в дальнейшем (предположим, через
несколько месяцев)?

Все это - от первого укола тысячам добровольцев до изучения их здоровья на
протяжении полугода - и есть третьей фазой испытаний.

Россия со своим "Спутником V", США с Moderna и Pfizer, Британия с AstraZeneсa
завершили третью фазу испытаний в ноябре. По крайней мере, они об этом сами
объявили. 

Они нашли нужное число добровольцев, ввели им препараты и выявили иммунный
ответ. Например, у "Спутника" эффективность составила более 90% (и 100% для
тяжелых случаев), у Moderna - от 94,5% до 100%, у Pfizer - около 95%, у
AstraZeneсa - в среднем 70%. Повторимся - все это показатели, которые называют
сами производители вакцин.

То есть все эти вакцины находятся примерно на одном уровне доказанности их
эффекта и безопасности.

При этом, ни одну из них ВОЗ еще не рекомендовала.

Что, впрочем, не является препятствием для того, чтобы начинать вакцинацию. И
она уже идет после того, как регулирующие органы тех или иных стран дали на это
разрешение. Поэтому "Спутником" уже вакцинируют в России, Pfizer - в
Великобритании и США (Евросоюз должен выдать разрешение до конца декабря). 

Другими словами, все обозначенные выше вакцины (включая российскую) реально
существуют. И промежуточные показатели дают основания полагать, что они
безопасны и в разной степени эффективны. 

Могут ли производители вакцин подтасовывать данные об их эффективности или
скрывать негатив при испытаниях? В теории - да. На практике же, когда все
делается по международным стандартам, это почти нереально. В них входит в том
числе испытание препарата за границей.

Например, сейчас "Спутник V" исследуют на себе жители Венгрии и Китая. И, будь
вакцина плохой и с сильной побочкой, об этом было бы уже известно как минимум
от этих государств.

Кроме того, от России ждут вакцины в Индии, Аргентине, Саудовской Аравии,
Мексике, Венесуэле, Бразилии и других странах - всего более миллиарда заказов
на 2021 год.

Также использовать "Спутник" для своего препарата решили британцы из
AstraZeneсa.\Furl{https://strana.ua/news/306125-astrazeneca-ispytyvaet-vaktsinatsiju-kombinirovannuju-s-rossijskim-sputnikom-v.html}
Если работа этого союза будет успешной, появится новая вакцина -
англо-российская, которой будут прививать жителей Великобритании.

\subsubsection{Украина будет ждать ВОЗ?}

Означает ли это, что и Украина может в любой момент зарегистрировать у себя
одну из уже существующих вакцин и приступить к ее закупкам и вакцинированию?

Не совсем. У нас, как говорят эксперты, есть своя специфика.

"В Украине, в отличие от той же России, приоритетными являются международные, а
не внутренние нормативы фармацевтического регулирования, - сказал в комментарии
"Стране" директор Института мирокобиологии и иммунологии имени Мечникова НАМН
Украины Валерий Минухин. - Следовательно, чтобы наше государство могло за
бюджетный счет закупать вакцину от коронавируса - независимо от того, кто
производитель, - она должна быть рекомендована ВОЗ.

Как выдается эта рекомендация? Во-первых, производителем должны быть полностью
закончены испытания, включая третий этап, то есть испытания на ограниченном
контингенте людей. Во-вторых, он должен подготовить полное досье на вакцину с
подробным описанием самой разработки, условий, в которых она проводилась,
вплоть до типа напольного покрытия в лаборатории и марки центрифуги,
клинических испытаний. Как правило, это несколько томов. 

В-третьих, страна-потенциальный покупатель должна обратиться в ВОЗ, а ВОЗ -
назначить третью, независимую и финансово незаинтересованную сторону для
экспертизы вакцины. Ей передаются все документ и еще раз проводятся испытания
на ограниченном контингенте людей. Это занимает, минимум, несколько месяцев. И,
если все проходит успешно, ВОЗ такую вакцину рекомендует.

На данный момент ни одна из вакцин не прошла всего цикла.

Но страны могут покупать вакцины без рекомендации ВОЗ, если они не брали на
себя обратных обязательств. Но для Украины же такой вариант будет считаться
отходом от взятых на себя обязательств".

Академик Владимир Широбоков, руководитель кафедры микробиологии и вирусологии
Медицинского университета им. Богомольца, говорит, что в рамках механизма COVAX
(о распределении вакцин между странами, в котором участвует и Украина), могут
использоваться только вакцины, одобренные ВОЗ и получившие соответствующие
сертификаты.

Но пока ни одна вакцина не получила одобрения ВОЗ, хотя та же Pfizer уже подала
документы на сертификацию. 

"В Великобритании массовая вакцинация препаратом от Pfizer/BioNTech началась
после его одобрения британским независимым Агентством по контролю за
лекарственными средствами и изделиями медицинского назначения. Эту вакцину
намерена сертифицировать также Еврокомиссия до конца года. В США вакцину Pfizer
одобрило управление по вопросам качества продовольствия и медикаментов (FDA)
при Министерстве здравоохранения Штатов. Если влиятельные внутренние регуляторы
дают добро вакцине, то, как правило, ВОЗ пускает ее сертификацию уже по
упрощенной процедуре. У нас регистрация рекомендованных ВОЗ вакцины идет уже
практически автоматически", - говорит Владимир Широбоков.

То есть если российскую вакцину рекомендует ВОЗ, то никаких препятствий для
регистрации ее в Украине быть не должно. Помешать могут только чисто
политические моменты.

