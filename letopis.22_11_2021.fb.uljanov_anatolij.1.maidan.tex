% vim: keymap=russian-jcukenwin
%%beginhead 
 
%%file 22_11_2021.fb.uljanov_anatolij.1.maidan
%%parent 22_11_2021
 
%%url https://www.facebook.com/dadakinder/posts/4944761342209656
 
%%author_id uljanov_anatolij
%%date 
 
%%tags cenzura,dostoinstvo,maidan2,nasilie,presledovanie,revgidnosti,svoboda,ukraina
%%title Майдан
 
%%endhead 
 
\subsection{Майдан}
\label{sec:22_11_2021.fb.uljanov_anatolij.1.maidan}
 
\Purl{https://www.facebook.com/dadakinder/posts/4944761342209656}
\ifcmt
 author_begin
   author_id uljanov_anatolij
 author_end
\fi

Когда я читаю пост своей однокурсницы про то, что Майдан – это «ценность жизни
каждого человека, свободы высказывания и личной свободы, плюрализма в
противовес моноидеологии и единоправильности; ценность справедливости и
солидарности, уважения к правам человека, а ещё – принятия другого», я тупо не
понимаю, что это такое, для кого это написано, чья это правда? Знаю, что не
моя.

\ifcmt
  ig https://scontent-lga3-2.xx.fbcdn.net/v/t39.30808-6/259829573_4944750578877399_8662888479030106966_n.jpg?_nc_cat=105&ccb=1-5&_nc_sid=730e14&_nc_ohc=tFsu7Fy7Ue8AX91JawO&_nc_ht=scontent-lga3-2.xx&oh=cace29a2e4306f910d981bb29fc6bdc8&oe=61A115F0
  @width 0.4
  %@wrap \parpic[r]
  @wrap \InsertBoxR{0}
\fi

В результате майданов, и инспирированных ими политических процессов, я не могу
жить и работать в собственной стране, и свободно выражать свои взгляды, не
сталкиваясь, при этом, с цензурой и насилием.

Мой этнос «неправильный». Мой язык – «собачий». Культура моей семьи – это
культура «оккупантов». Её символы запрещены и караются уголовной статьей.

Вот уже более дюжины лет, я – политический беженец. Не из ГУЛАГа. Из Украины. Я
знаю, что такое потерять дом из-за взглядов. Что такое потерять друзей из-за
взглядов. Что такое потерять работу из-за взглядов. 

Мне не нужно рассказывать про репрессии в совке. Я про репрессии не из книжек
по истории узнаю. Мне в нашей с вами реальности говорят с украинских экранов,
что меня нужно сдать в СБУ, арестовать или депортировать, если я, гражданин
Украины, посмею в ней быть. 

И после этого я читаю, как люди, знающие меня лично, пишут про Майдан как про
ценности «гуманизма и достоинства… справедливого правосудия, равных правил и
равных возможностей». Уж не знаю как называют этот литературный жанр в Украине
всего хорошего, но американцы называют его «бычьим говном».  

>>> ссылка в первом комментарии<<<
