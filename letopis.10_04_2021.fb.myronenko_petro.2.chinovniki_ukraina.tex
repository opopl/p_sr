% vim: keymap=russian-jcukenwin
%%beginhead 
 
%%file 10_04_2021.fb.myronenko_petro.2.chinovniki_ukraina
%%parent 10_04_2021
 
%%url https://www.facebook.com/petro.myronenko/posts/3103373833223590
 
%%author 
%%author_id 
%%author_url 
 
%%tags 
%%title 
 
%%endhead 

\subsection{Тривожні думки - Чиновники}
\Purl{https://www.facebook.com/petro.myronenko/posts/3103373833223590}

Нещодавно спілкувався з одним високопосадовцем, який відповідає за
воєнно-політичну складову в державі.

Як досвідчений бюрократ з понад 40-річним досвідом державної служби мушу
засвідчити, що високопосадовці стали дрібними і здатні лише сліпо виконувати
рішення керівництва.

Розумне керівництво дослухається думки досвідчених працівників і саме таких
приймають на відповідальну державну службу. 

Біда наших керманичів у тому, що вони не бажають мати поруч розумних радників і
помічників. Це є ознакою комплексу неповноцінності і психологічною проблемою,
від чого не маємо поступу вперед до цивілізованих держав, а лише декларації та
замилювання очей як Українському народові, так і стратегічним партнерам у
світі, які вже зрозуміли нашу хибну політику, що за 30 років не відійшла від
радянського мислення.

Радянське мислення - це агресивно-православна спадщина Московського улусу Орди.

Прикро, що всі мої пости читають високопосадовці з верхнього щабля влади і
нікому в голову не прийде думка щодо конкретних пропозицій керівництву задля
просування держави до світу, бо не уявляють себе самостійними державними
службовцями, які складали Присягу на вірність Українському народові так само,
як і їхні керівники. 

Крісло затягує швидко і тіло, і сіру речовину.

З повагою П. В. Мироненко
