% vim: keymap=russian-jcukenwin
%%beginhead 
 
%%file 17_07_2019.stz.news.ua.mrpl_city.1.svitlana_otchenashenko_zberegty_zhyvoju_dushu
%%parent 17_07_2019
 
%%url https://mrpl.city/blogs/view/svitlana-otchenashenko-najvazhlivishezberegti-zhivoyu-dushu
 
%%author_id demidko_olga.mariupol,news.ua.mrpl_city
%%date 
 
%%tags 
%%title Світлана Отченашенко: "Найважливіше – зберегти живою душу!"
 
%%endhead 
 
\subsection{Світлана Отченашенко: \enquote{Найважливіше – зберегти живою душу!}}
\label{sec:17_07_2019.stz.news.ua.mrpl_city.1.svitlana_otchenashenko_zberegty_zhyvoju_dushu}
 
\Purl{https://mrpl.city/blogs/view/svitlana-otchenashenko-najvazhlivishezberegti-zhivoyu-dushu}
\ifcmt
 author_begin
   author_id demidko_olga.mariupol,news.ua.mrpl_city
 author_end
\fi

\ii{17_07_2019.stz.news.ua.mrpl_city.1.svitlana_otchenashenko_zberegty_zhyvoju_dushu.pic.1}

Наступний нарис присвячено дуже глибокій, талановитій, муд\hyp{}рій і дійсно
унікальній людині. Єдина народна артистка України в Маріуполі, лауреат премії
ім. Заньковецької, \textbf{Світлана Іванівна Отченашенко} пройшла довгий творчий шлях,
який викликає щирий подив і захоплення кожного, хто знайомий з біографією
видатної маріупольчанки.

\ii{insert.read_also.demidko.mariupolchanky}

% батьки
\ii{17_07_2019.stz.news.ua.mrpl_city.1.svitlana_otchenashenko_zberegty_zhyvoju_dushu.pic.2}

Народилась актриса в селі Іскивці Полтавської області. Батько мав три вищі
освіти, закінчив Вчительський інститут, Вищу партійну школу та
Сільськогосподарський інститут. Займав посаду голови колгоспу. Батьки
познайомились у військовому шпиталі, куди потрапив батько після поранення, і де
працювала медсестрою мати. Медичною сестрою мама актриси пройшла всю війну.
Винесла більше 80 поранених з поля бою. І до старості працювала операційною
сестрою. Хотілося б відзначити, що батько Світлани Іванівни справжній українець
(полтавчанин), а мама – росіянка, хоча й вона прожила 50 років в Полтаві, тобто
більшу половину життя. На малій Батьківщині Світлана Іванівна завжди говорила
українською, яку вважає рідною мовою.

Юна Світлана брала участь в шкільній самодіяльності. З 7 років почала приміряти
на себе образ актриси. Раз на два роки мати вивозила доньку із села до
Ленінграда, де водила в театр. Враження від театру й вистав з його комплексним
мистецьким впливом на свідомість стануть найсильнішими спогадами юності. Бачила
на сцені живих Ольхину та Смоктуновського у виставі \enquote{Ідіот} у Большому
драматичному театрі, що було для дівчини дарунком долі.

\ii{17_07_2019.stz.news.ua.mrpl_city.1.svitlana_otchenashenko_zberegty_zhyvoju_dushu.pic.3}

Освіта майбутньої актриси розтяглась на декілька років і пройшла у різних
містах. Вступила до Всеросійського державного інституту кінематографії у
Москві, але не витримавши напруги мегаполісом, перебралась до Києва, де
закінчила студію імені Франка. Серед викладачів Світлани Отченашенко – народний
артист УРСР Покотило Михайло Федорович та народний артист СРСР Степанков
Костянтин Петрович. По закінченні студії працювала в Полтавському театрі, але
як початківець грала дрібні ролі.

\ii{17_07_2019.stz.news.ua.mrpl_city.1.svitlana_otchenashenko_zberegty_zhyvoju_dushu.pic.4}

Коли Світлана приїхала до Маріуполя, вирішивши спробувати свої сили у місцевому
театрі, з нею трапився курйоз. Оскільки вона не знала, де знаходиться театр, їй
довелося їхати на таксі. Однак, незважаючи на невелику відстань від вокзалу до
театру, таксист вирішив показати місто вродливій дівчині і катав її 40 хвилин.
На відміну від Полтави, у Маріуполі молоду Світлану Отченашенко відразу ж ввели
в поточний репертуар театру. Вистава за п'єсою Ю. Едліса \enquote{Крапля в
морі} (1966 р.) стала першою роботою актриси. Олівцем від руки в програмку
вписано: \enquote{Віра-Отченашенко}. Після ролей Валі Анощенко
(\enquote{Російські люди} К. Симонова), Ніси (\enquote{Дурочка} Лопе де Вега) і
Тані Свєтлової (\enquote{Глєб Космачова} М. Шатрова) закріпилося положення, що
С. Отченашенко - одна з провідних актрис театру. І як підтвердження тому –
звання заслуженої артистки України, отримане в 1972 р., в 27 років. Звання
отримала в один день зі своїм Майстром, Учителем, головним режисером театру
\textbf{Олександром Кадировичем Утегановим}. Вона завжди була
\enquote{утегановською} актрисою. Світлана Іванівна – людина вдячна, а ця
якість досить рідкісна в театрі. На все життя вона зберегла подяку Олександру
Кадировичу за те, що він навчив її думати на сцені, професійно підкував, сприяв
формуванню індивідуальності. Саме про нього писала вона дипломну роботу на
театрознавчому факультеті Ленінградського державного інституту театру, музики і
кінематографії (закінчила у 1984 році).

З кожною новою роллю зростала майстерність актриси. Особливо запам'яталася С.
Отченашенко в ролі Віктоші з \enquote{Казок старого Арбату} О. Арбузова – юна дівчина,
що змогла перевернути розмірене життя двох літніх чоловіків, ролі яких
виконували \href{https://archive.org/details/13_12_2018.olga_demidko.mrpl_city.unikalnyj_aktor_borys_saburov}{Борис Сабуров}%
\footnote{Унікальний актор маріупольського театру Борис Сабуров, Ольга Демідко, mrpl.city, 13.12.2018, \par%
\url{https://mrpl.city/blogs/view/unikalnij-aktor-mariupolskogo-teatru-boris-saburov}, \par%
Internet Archive: \url{https://archive.org/details/13_12_2018.olga_demidko.mrpl_city.unikalnyj_aktor_borys_saburov}
} і \textbf{Микола Земцов}. На сцені панували такі чисті
стосунки, випромінювалося таке тепло, що ніхто з глядачів не зміг залишитися
байдужим. Світлану Отченашенко в ролі Віктоші вітав автор п'єси Олексій
Арбузов, її органічністю захоплювався Зіновій Гердт.

\ii{17_07_2019.stz.news.ua.mrpl_city.1.svitlana_otchenashenko_zberegty_zhyvoju_dushu.pic.5}

Одна з ключових ролей для Світлани Іванівни – це роль Марії Каллас
(\enquote{Майстер-клас} \enquote{Страсті по Марії}). Спочатку актриса не знала, як підійти до
цієї ролі, бо дуже хвилювалася. Разом з режисером \textbf{Костянтином Володимировичем
Добруновим} вирішили не \enquote{ліпити} зовнішньої схожості, а передати характер. І це
було правильним рішенням. Ця роль допомогла Світлані Іванівни висловити власні
переживання, власну біль. Вона спробувала не зіграти Марію Каллас, зірку, що
завдяки драматизму життя і долі стала легендою, а зіграти колегу, Актрису і
Жінку, яку дуже добре розуміє, оскільки й сама – актриса й жінка. Світлана
Іванівна вирішила прожити цю роль, що їй вдалося блискучо.

\ii{insert.read_also.demidko.kozhevnikov}
\ii{17_07_2019.stz.news.ua.mrpl_city.1.svitlana_otchenashenko_zberegty_zhyvoju_dushu.pic.6}

Загалом, про зіграних персонажів народної артистки України Світлани Отченашенко
можна сказати, що всі вони сильні, неоднозначні та незабутні. Будучи актрисою
далекого провінційного театру, вона стала відома в столичних професійних колах,
їй присвячував статті відомий журнал \enquote{Театр}. Актриса, в чиєму репертуарі
Аркадіна, Раневська, Аббі Патнем, Марія Каллас і театральний критик - в одній
особі. Статті С. Отченашенко в \enquote{Литературке}, \enquote{Советской культуре} піднімали аж
ніяк не специфічно периферійні, а загальноцехові професійні проблеми. На
рахунку С. Отченашенко цілий ряд моноспектаклів (вечори поезії програми віршів
Анни Ахматової, Марини Цвєтаєвої, літературно-музичний спектакль \enquote{Заметіль} за
О. Пушкіним і сценічна новела з вистави \enquote{Ілюзіон} по Крецу, також навіть роль
матері з драми \enquote{Божевільні}, \enquote{Цинкові хлопчики} – про хлопців, які пішли і не
повернулися з Афганістану, \enquote{Мати} за однойменним твором Олександра Довженка та
ін.). Вечори поезії та моновистави допомогли актрисі вистояти, коли не було
ролей. Правда, вистави – це одне, наголошує актриса, а читання віршів – інше.
\begin{quote}
\em\enquote{У виставі ти пов'язаний з партнерами, режисером, художником... Коли ж читаєш
вірші, тобі ніхто не допоможе, але ніхто і не завадить. Ось глядач, а далі –
Всесвіт, Бог, Совість. Яке щастя відчувати глядача, бачити, як від твоїх слів
розкриваються обличчя. Відчуваєш таку свободу! Ці програми допомагають мені
вдосконалюватися, знаходити себе через спілкування з високою поезією. А ще
вірші допомогли мені знайти свого глядача}, 
\end{quote}
- поділилася Світлана Іванівна.

\ii{17_07_2019.stz.news.ua.mrpl_city.1.svitlana_otchenashenko_zberegty_zhyvoju_dushu.pic.7}

Світлана Отченашенко дуже любить Маріуполь і вже давно вважає його найріднішим
містом. Саме тут відбувалися найважливіші події в її житті. Тут вона
познайомилася і зі своїм чоловіком \textbf{Харабетом Юхимом Вікторовичем}, скульптором і
медальєром, заслуженим діячем мистецтв України, який підтримував у всьому
кохану жінку.

\ii{17_07_2019.stz.news.ua.mrpl_city.1.svitlana_otchenashenko_zberegty_zhyvoju_dushu.pic.8}
\ii{17_07_2019.stz.news.ua.mrpl_city.1.svitlana_otchenashenko_zberegty_zhyvoju_dushu.pic.9}

Завдяки енергійній діяльності актриса змогла зберегти історичну театральну
спадщину Маріуполя, надихнувши телеканал \par\noindent\enquote{Сигма} на створення унікального
циклу, присвяченого історії та людям Маріупольського театру. Мало кому відомо,
що актриса брала активну участь в створенні бібліотеки в маріупольському театрі
разом із завідувачем трупою \textbf{Іскаковою Неллі Миколаївною}.

Світлана Іванівна Отченашенко невтомно підкорює маріупольців своєю талановитою
акторською грою, адже завжди їй вдається проживати життя своїх героїнь. Вона
каже: \enquote{\emph{Суть професії в театрі}, - тобто професійна функція акторів, на думку
Світлани Іванівни - \emph{це допомагати людям у формуванні душі. Адже, здебільшого,
людська душа сьогодні виявляється сиротою. Тілом ми займаємося: фізично
ублажаємо його завжди, а ось душа... А душа у нас забута, закинута. Є душі
нерозбуджені... Акторам слід допомогати душам залишатися живими, рости,
формуватися, знаходити себе}}.

У актриси є досвід і в кіно. Вона зіграла бабу Зою у фільмі \enquote{Зачароване
кохання} (2008 рік). Світлана Іванівна підкреслила, що \emph{\enquote{кіно вимагає загибелі
всерйоз... Все повинно йти зсередини. Не потрібно стільки міміки і жестів, як в
театрі, потрібно стежити за своєю грою уважніше}}.

\ii{17_07_2019.stz.news.ua.mrpl_city.1.svitlana_otchenashenko_zberegty_zhyvoju_dushu.pic.10}
\ii{insert.read_also.demidko.levchenko}

\textbf{Улюблені книги:} поезія Марини Цвєтаєвої, \enquote{Майстер і Маргарита} М.
Булгакова, \enquote{Град приречений} Аркадія і Бориса Стругацьких, \enquote{День восьмий}
Торнтона Уайлдера, \enquote{Сто років самотності} Ґабрієля Ґарсія Маркеса.

\textbf{Улюблений фільм:} \enquote{Гамлет} Г. Козинцева (1964 р.).

\textbf{Порада маріупольцям:} 

\begin{quote}
\em\enquote{Зберіжіть живою вашу душу. Не пускайте туди ненависть і злість, адже
зруйнована душа це набагато гірше зруйнованого тіла!}. 
\end{quote}

