% vim: keymap=russian-jcukenwin
%%beginhead 
 
%%file 27_07_2020.fb.lnr.3
%%parent 27_07_2020
 
%%endhead 
  
\clearpage
\subsection{3 --- Судьба «Большевика». Как жемчужина киевской промышленности превратилась в склад
секонд-хенда}
\url{https://www.facebook.com/groups/LNRGUMO/permalink/2880726538705568/}

\index{LNRGUMO}

Судьба «Большевика». Как жемчужина киевской промышленности превратилась в склад
секонд-хенда 

26 июля 1882 года швейцарский предприниматель Яков Гретер и немецкий инженер
Филипп Мозер основали на западной окраине Киева - Шулявке - Киевский
чугунолитейный и механический завод. В последующем он сыграл огромную роль и в
развитии города, и в его истории.

Впервые Шулявка упоминается в древнерусских летописях X-XI веков как Шелвова
борка или борок. Так наши далёкие предки называли низкорослый лесок с
множеством лужаек, который рос тогда на этом месте. В документах XVII века уже
появляется село, которое называется Шелвов Борок или Шелвово сельцо, а в XVIII
веке --- как Шулявка. 

В декабре 1881 года Гретер купил у наследников некоего полковника Леоновича
(бывшего помощника начальника киевской тюрьмы) четыре десятины земли (около 4,4
га), на которых завод начал своё развитие. Первоначально на нём работали всего
30 рабочих, которые изготавливали литьё и несложные в производстве детали.
Экономили на всём. Так, однажды Гретер закупил большую партию списанных
однозарядных винтовок-берданок, которые пошли в переплавку. 

\begin{figure}[ht]
 \centering
 \PrjPicW{27_07_2020/fb/lnr/3/1}{0.5}
 \caption{Судьба «Большевика». Как жемчужина киевской промышленности превратилась в склад
секонд-хенда}
 \label{fig:}
\end{figure}



Алексей Стаценко
  
