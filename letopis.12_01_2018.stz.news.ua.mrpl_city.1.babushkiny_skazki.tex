% vim: keymap=russian-jcukenwin
%%beginhead 
 
%%file 12_01_2018.stz.news.ua.mrpl_city.1.babushkiny_skazki
%%parent 12_01_2018
 
%%url https://mrpl.city/blogs/view/babushkiny-skazki
 
%%author_id burov_sergij.mariupol,news.ua.mrpl_city
%%date 
 
%%tags 
%%title Бабушкины сказки
 
%%endhead 
 
\subsection{Бабушкины сказки}
\label{sec:12_01_2018.stz.news.ua.mrpl_city.1.babushkiny_skazki}
 
\Purl{https://mrpl.city/blogs/view/babushkiny-skazki}
\ifcmt
 author_begin
   author_id burov_sergij.mariupol,news.ua.mrpl_city
 author_end
\fi

Тягучие зимние вечера. Электричества нет. Немцы, покидая город, взорвали
электростанцию на Малофонтанной. Поужинали, как говорит бабушка,  что Бог
послал. Дедушка уносит на веранду карбидную лампу, чтобы погасить ее до утра.
Карбид дорог – нужно экономить. От лампы, еще минуту назад освещавшей
мертвенно-голубым светом лица, обеденный стол, печь, буфет с резными утками на
дверцах, часы-ходики, остается лишь едкий запах ацетилена. Комната погрузилась
во тьму. Лишь постепенно некоторые предметы становятся чуть различимыми. Огонек
лампадки осветил лик и латунный оклад святого Пантелеймона целителя.
Раскаленная чугунная плита печки делает видимым посапывающий зеленый
эмалированный чайник, стоящий на ней, мешочек с сухарями \enquote{на черный день},
притороченный в углу. Поддувало бросает пучок света на пол.  Время от времени
световой пучок вдруг вспыхивает, становится ярким – это в пространство
поддувала провалился сквозь колосники раскаленный уголёк.

Взрослые идут спать. Завтра они встанут очень рано. Бабушка и дети остаются.
Дети устраиваются на кровати, накрытой грубым \enquote{кусачим} солдатским одеялом,
бабушка садится на низкую табуретку перед поддувалом, свет освещает крупные
черты ее лица. С десяток минут она молчит, а затем начинает один из своих
рассказов о \enquote{мирном} времени, то есть периоде жизни в Мариуполе до начала
Первой мировой войны. Как в девятьсот третьем году вместе с односельчанками
нанималась в Волонтеровке к хозяину \enquote{экономии} полоть плантации баклажан и
помидоров. Как девчат, прежде чем принять на работу, усаживали за длинный стол,
уставленный глубокими тарелками с борщом, кусками жирной баранины, \enquote{шматками}
ситного хлеба. Хозяин садился во главе стола и наблюдал за едоками. Девчата
давились, стараясь как можно больше съесть, зная, что их наниматель следует
принципу: кто мало ест, тот плохо работает.

Рассказывалось о старом базаре, где тарань и рыбец продавали кулями из рогожи,
а перед  лавками стояли открытые бочки с изюмом, мочеными яблоками и сладкими
экзотическими рожками, неведомыми для детей, вступившими в сознательное
состояние в годы страшной войны. И каждый мог подойти к этим бочкам, взять
горсть изюма или рожек, попробовать их, и приказчик не ругался, а только
расхваливал свой товар. Вспоминались \enquote{сочники}, зажаренные изделия из
мелко-мелко нарубленного ливера, лука и специй, обернутых в баранью сетку.
Бабушке так и не удалось передать словами ни вкуса, ни запаха этого шедевра
мариупольских кулинаров. И еще одна картинка из прошлого. Мещанки или
чиновницы, выбрав судака, севрюгу или чебака приличных размеров, не укладывали
рыбу себе в кошелку, да у них-то и кошелок с собой не было, а расплачивались за
покупку. И мальчишка, прислуживавший в рыбной лавке, за две-три копейки бегом
доставлял товар по указанному адресу: на Торговую, Константиновскую, а то и на
Евпаторийскую улицы.

И еще бабушкин совершенно невероятный рассказ на рыбную тему. В Мариуполе было
изобилие и разнообразие  своей, азовской рыбы: севрюга, белуга, лещ, карп,
судак, тарань, рыбец, чехонь, азовская селедка, пузанок, селява, камбала,
хамса-анчоус, всего более ста видов. От одного перечисления  дух захватывает. И
вместе с тем … в наш город завозили  дары моря со стороны. Было странно
слышать, что в магазинах, где продавали обувь, не было очередей, чтобы купить,
например, башмаки. Наоборот, приказчик торопился к покупателю или
покупательнице, которые не обязательно были жителями города, - это могли быть и
крестьяне из Мангуша, села Покровского или немецкой колонии Бергталь, -
усаживал в кресло, расставлял товар, становясь на одно колено, сам примерял
полусапожки, туфли \enquote{последней парижской моды}.

И совсем уже невероятную историю довелось как-то услышать из уст бабушки о том,
как боролись с холерой в Мариуполе и, наверное, и в других городах Российской
Империи. Начиная с весны, когда на базаре появлялись овощи и фрукты нового
урожая, среди торговых рядов появлялся городской санитарный врач, на нем был
белый халат, поверх которого надет кожаный фартук, за ним семенил служитель с
ведром, наполненным наполовину ... керосином. Когда в поле зрения врача
попадалась горка незрелых фруктов, он макал небольшую метелку в керосин и
окроплял ею подозрительные дары полей и садов Приазовья. Такая же судьба
ожидала подпорченную рыбу и неприятно попахивавшие мясные изделия. Потому то,
по мнению бабушки, в \enquote{мирное время} на мариупольских базарах и лавках крайне
редко торговали подпорченными продуктами. А может, ей только казалось? Ведь в
молодые годы все кажется хорошим, а многое и прекрасным...

Под бабушкин неспешный рассказ смыкаются веки, сон овладевает все более. И
снится длинный стол, за которым сидит наш директор школы Иван Спиридонович,
перед ним большой диск макухи, так приятно пахнущий постным маслом. Он говорит:
\enquote{Ешь, это все твое. Съешь – получишь пятерку по чистописанию}. Потом он
становится на одно колено и расшнуровывает мне ботинок. Это странно, ведь на
ногах у меня стеганые бурки – матерчатые сапоги... Еще один сон. Магазин, к
которому прикреплена наша семья. Продавщица Вера рассматривает хлебные
карточки. Вертит их туда и сюда, даже нюхает. \enquote{Нет, сегодня ваши карточки
хлебом запретили отоваривать. Бери стерлядь, иваси и устрицы}. Что это? Их
никогда не приходилось видеть. \enquote{А может, дадите на карточки тюльку?}.  \enquote{Еще
чего? Тюльку захотел, ее и без тебя все любят. Иди прочь! Позову сейчас
санитарного врача, он тебя керосином обольет}...

Утром пробуждение по протяжному гудку \enquote{Азовстали}.  Быстро умыться ледяной
водой из-под крана, перехватить вчерашней мамалыги и садиться за уроки. Анна
Дмитриевна задала на дом много примеров по арифметике, нужно еще написать
несколько предложений по русскому языку, выучить стихотворение \enquote{Село мое}, а
еще сходить в магазин, отовариться хлебом. Если будет довесок, он будет мой. Мы
учимся во вторую смену в мужской неполной средней школе, в первой смене учатся
девчонки. Скорей, скорей в школу. Забыты рассказы бабушки о \enquote{мирном времени}
больше похожие на сказки. Сегодня, как бы не получить \enquote{двойку}...

Прошли десятилетия. Давным-давно нет  ни дедушки, ни бабушки, нет и других
обитателей комнаты, где зимой  уютно пыхтел на печке зеленый чайник,
отстукивали время ходики, где стоял источенный шашелем старинный буфет с
резными утками на дверцах, а в углу мерцала лампадка перед иконой святого
Пантелеймона целителя. Да и в доме этом живут совершенно незнакомые люди. И
вдруг, совсем недавно в памяти, как на листе фотобумаги, погруженной в
проявитель, стали обозначаться сначала едва видимые контуры, а затем  четкие
картины бабушкиных рассказов, казавшихся тогда, более семидесяти лет назад,
похожими на сказки.
