% vim: keymap=russian-jcukenwin
%%beginhead 
 
%%file 01_11_2021.stz.news.ua.pravda.1.ukr_mria
%%parent 01_11_2021
 
%%url https://www.pravda.com.ua/columns/2021/11/1/7312312
 
%%author_id ljubko_deresh.pisatel.ukr
%%date 
 
%%tags mechta,ukraina
%%title Українська мрія між веснянками й Космосом
 
%%endhead 
 
\subsection{Українська мрія між веснянками й Космосом}
\label{sec:01_11_2021.stz.news.ua.pravda.1.ukr_mria}
 
\Purl{https://www.pravda.com.ua/columns/2021/11/1/7312312}
\ifcmt
 author_begin
   author_id ljubko_deresh.pisatel.ukr
 author_end
\fi

\begin{zznagolos}
Якою є українська мрія? 

Якщо набрати це словосполучення в Google, ми отримаємо, здебільшого, два
результати: інформацію про літак АН-225 \enquote{Мрія} і модульні будинки під
однойменною назвою. 

Звісно, це зовсім не означає, що в Україні ніхто й ніколи не формував візії
того, якою має бути українська мрія (читай 3/4 перспективи нашого розвитку), але
іноді складається враження, що напружене, незмінно кризове сьогодення змушує
нас фокусуватися на теперішньому.
\end{zznagolos}

А ще частіше 3/4 на минулому, тоді як саме у візії майбутнього і є наш шанс як
суспільства вирватися із зачарованого кола повторюваних помилок, яким блукає
Україна останні десятиліття.

Саме з цих міркувань в рамках проєкту \enquote{Мрійники-ХХІ} я організував форум
\enquote{Мрійники: Листи до Всесвіту}. Мета його одна і важлива: актуалізувати розмову
про те, що є українською мрією в широкому сенсі. 

Оскільки мрія 3/4 поняття межове, котре існує на межі між світом явленим і
уявним, ми вирішили не обмежуватися лише інтелектуальним аналізом української
мрії, але й додали елемент мистецького візіонерства. 

Відтак, супроводом до форуму став перформенс (точніше, \enquote{видиво} 3/4 авторський
термін на позначення оригінального синтетичного мистецького жанру) від
львівського художника і перформера Влодка Кауфмана під назвою \enquote{Листи до
Всесвіту}. 
 
Разом з колегою-співмодераторкою, історикинею Ольгою Михайловою, ми ставили
собі за мету, насамперед, дослідити поле самого поняття мрії (і, вже потім,
українського характеру цієї мрії). 

\begin{zznagolos}
Про що українська мрія? Яка її сутність? Чи має вона структуру? 

Як українська мрія дотикається до великого, глобального світу?

І чи може бути вона цікавою йому, як є цікавою американська мрія? 

А ще хотілося дізнатися й про темну сторону мрії: чи є українські мрії, які
потребують розвінчання і чи можуть мрії тягти за собою катастрофи? 

Чи є мрії, котрі ведуть до дорослішання, чи може, навпаки, штовхають назад у
інфантильність?
\end{zznagolos}

Форум складався з трьох панельних дискусій. На першій панелі свої думки про
українську мрію представили мистецький критик Костянтин Дорошенко,
кримсько-татарська історикиня Гульнара Бекірова та громадський діяч та ідеолог
Олесь Доній. 

Центральним посланням пана Дорошенка стала теза про те, що мрія може і повинна
бути виключно індивідуальною, оскільки саме європейська цивілізація, на відміну
від азійських, наділила цінністю окрему особистість. 

У цьому сенсі пан Дорошенко не бачить потреби випрацьовувати окрему українську
мрію, відмінну від європейської, з її гаслами "Рівність, свобода, братерство"
чи американської, з її націленістю на добробут, персональною успішністю,
конкуренцією та недоторканністю приватної власності, тоді як власне мрію варто
було б залишити окремій людині для її приватного життя. 

У випадку пана Дорошенка така мрія достатньо прагматична: отримати шанс на
гідну, спокійну старість у цивілізованій рідній країні, де комфортно і захищено
почувається кожен.

Пані Гульнара Бекірова наголосила на тому, що мрію, котра вже, начебто,
здобута, можна легко втратити ¾ і важкий досвід втрати державності, який
пережили кримські татари, цьому приклад. 

Вона зробила історичний екскурс у спільне минуле кримських татар, українців та
європейських монархій і окреслила тяглу мрію корінних кримчан ¾ відновлення
власної державності, втраченої через імперський вплив Радянського Союзу та
новітньої Росії, на рідній кримській землі в рамках незалежної України.

Олесь Доній, натомість, спробував дати відповідь на запитання, що стається з
мрією тоді, коли її вдається певною мірою реалізувати, як це сталося з мрією
про Незалежність після 1991-го року, коли її було здобуто. 

Пан Доній, за його словами, здійснив нахабну спробу сконструювати нову
українську мрію, виклавши її у рамках своєї концепції \enquote{25 сходинок до
суспільного щастя}. 

Головним акцентом \enquote{25 сходинок} пана Донія є переосмислення ролі
держави в житті громадянина: держава повинна бути не кінцевою метою, а засобом
здобуття щастя, бути джерелом передумов для того, аби кожен міг здійснити своє,
індивідуальне бачення щастя.

Друга панель спікерів виявилася кипучою. До розмови сіли науковиця Наталія
Шульга, філософ Сергій Дацюк та телеведучий Микола Вересень. 

Драйвером дискусії став Сергій Дацюк, котрий здійснив інтелектуальну
провокацію, заявивши про те, що, позаяк тема форуму заявлена гучно ¾
\enquote{Листи до Всесвіту} ¾ ті мрії, котрі предʼявляють сьогодні українці, на
жаль, не мають до Всесвіту жодного відношення. 

Мріяти, в розумінні пана Дацюка, ¾ означає виходити за межі часу, простору,
взагалі всього мислимого ¾ лише такі мрії можуть пройти випробування історією. 

Використовуючи елементи акторської гри, пан Дацюк сам став на позицію Всесвіту
і спробував сформулювати філософські передумови для справжньої,
понаднаціональної, понаддержавної мрії. 

Його думки виявилися співзвучними роздумам Миколи Вересня, за спостереженнями
якого українці в принципі не є нацією мрійників, а схильні радше до конкретної
боротьби проти чогось. 

Модераторка Ольга Михайлова тут же нагадала панові Миколі про письменника Олеся
Бердника з його мрією про \enquote{Українську духовну республіку} та філософа
Сергія Кримського, однак була змушена в чомусь погодитися з паном Миколою:
згідно недавніх соціологічних опитувань (\enquote{Рейтинг}, 2021), 58\%
українців мріють про хороше здоровʼя, 41\% – про підвищення зарплат та пенсій,
а 40\% – про продовження роду, тобто, цілком приземлені та зрозумілі речі. 

Наталія Шульга, натомість, зауважила, що саме через відданість конкретиці і
досягаються великі звершення, навівши як приклад авіаконструктора Олега
Антонова і його відданість власній мрії ¾ створювати найкращі літаки у світі. 

Окрім цього, пані Шульга зауважила, що осердям української мрії впровдож
останніх століть було омріювання української окремішності, самості, головною
задачею якої було дати відповідь на запитання: а чи потрібна українцям, що
перебували на той час під владою двох імперій, власна держава в принципі?

Третя панель, до якої були запрошені головна редакторка \enquote{Української правди}
Севгіль Мусаєва, директорка українського відділення \enquote{Amnesty International}
Оксана Покальчук та засновниця екологічної ініціативи \enquote{Україна без сміття}
Євгенія Аратовська, носила не стільки дискусійний, скільки емпатичний характер
¾ на ній ішлося не стільки про аналіз мрії, скільки про спів-переживання мрії. 

Євгенія Аратовська підняла тему залучення до власної мрії інших людей ¾
необхідності ненасильницького, обережного характеру цього процесу. 

Пані Покальчук поділилася досвідом того, наскільки складно є комунікувати в
суспільстві теми, повʼязані з правами людини, які, де-факто, є основою для
будь-яких мрій, а також озвучила необхідність існування у суспільстві певних
модераторів, котрі б допомагали процесу комунікації суспільних груп з різними
мріями (різними потребами, болем). 

Пані Мусаєва, натомість, як представниця медіа і очільниця одного із означених
модераторів комунікації, сайту \enquote{Українська правда}, розповіла про почуття
особистої відповідальності у процесі такого модерування, про потребу узгодження
різних дискурсів і мрій та про спроби недопущення ситуацій, за яких різні мрії
починають гасити одна одну. 

Вона звернула увагу на проєкт \enquote{Дон Кіхоти}, що існує на \enquote{Українській правді},
який розповідає про невідомих широкому загалові людей, котрі, йдучи на зустріч
власній мрії, змінюють місця та міста, в яких вони живуть.

Вкрай емоційна панель завершилася життєствердною тезою пані Мусаєвої про те, що
українцям варто шукати самодостатньості, і в цій самодостатньості дослівно
\enquote{витягувати себе щодня з болота за волосся}. Зрештою, в підсумку, ці останні
слова і є, напевне, поясненням того, чим є мрія ¾ тим, що допомагає здійснювати
неможливе.

Глибоким і емоційним виявився і мистецький епізод події ¾ видиво \enquote{Листи до
Всесвіту} від Влодка Кауфмана. 

Після завершення третьої панелі дискусії художник разом з помічником
несподівано перевернули таємничу плахту, по якій до цього прогулювалися гості
події. Нею виявилося полотно, склеєне з десятків старих і сучасних географічних
карт. 

Під спів колискової від Katya Chilly Кауфман, пригасивши в галереї світло,
ходив поміж людей з ліхтарем, вишукуючи спікерів форуму, та пропонував кожному
з них виголосити коротку, на 2-3 хвилини, особисту промову на тему власної,
глибоко особистої мрії ¾ відправити до Всесвіту власного \enquote{листа}. 

В перервах між відвертими, нерідко щемкими та пронизливими словами учасників
Кауфман клеїв шматки чорного скотчу на карти, роблячи таким чином відсилання до
власного ж проєкту 1994-го року. Тоді, у Львові, він провів мистецьку акцію під
назвою \enquote{Листи до землян, або 8-ма печать} ¾ видиво, на якому географічні карти
вже не існуючого тоді Радянського Союзу теж відігравали центральну роль. 

\enquote{Листи до Всесвіту}, завершивши форум з обговорення української мрії, таким
чином стали подією-відповіддю на \enquote{Листи до землян} початків Незалежності. 

Звуки монотонного склеювання Кауфманом карт за допомогою скотча у поєднанні з
народними, архаїчними мелодіями колискової у напівтемряві галереї створювали
відчуття одинокості людини та її мрій перед невблаганністю зовнішніх сил, однак
людські голоси, що розповідали власні історії мрій, руйнували гіпнотизм сліпої
геополітичної механіки і зміцнювали віру в людську спільність. 

Наповнившись досвідом тридцяти років самостійності, українці спробували так чи
інакше відповісти Буттю на даровану їм свободу і означили ті перспективи і
горизонти, до яких мріють рухатися далі. 

Щемка веснянка, виконана на сопілці Катею, що прозвучала перед самим
завершенням події, вселила надію на не, що в українців, та й загалом у нас,
людей, усе, врешті, буде добре.

\ii{01_11_2021.stz.news.ua.pravda.1.ukr_mria.pic.1}

Довідка: проєкт \enquote{Мрійники} ¾ багатоплатформний проєкт, реалізований за
підтримки \enquote{Українського культурного фонду}, метою якого є дослідження вимірів
української мрії. У рамках проєкту був проведений форум-видиво \enquote{Мрійники: листи
до всесвіту}, знятий документальний фільм \enquote{Мрійники-ХХІ}, присвячений ключовим
постатям української культури останніх 30-ти років, проведений онлайн-челендж
\enquote{Портрет мрійника}.

Любко Дереш
