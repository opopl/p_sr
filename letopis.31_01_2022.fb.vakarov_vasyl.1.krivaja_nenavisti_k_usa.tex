% vim: keymap=russian-jcukenwin
%%beginhead 
 
%%file 31_01_2022.fb.vakarov_vasyl.1.krivaja_nenavisti_k_usa
%%parent 31_01_2022
 
%%url https://www.facebook.com/vasylvd/posts/3177361902587493
 
%%author_id vakarov_vasyl
%%date 
 
%%tags nenavist,rossia,ukraina,usa
%%title В Украине стремительно растёт кривая ненависти к США
 
%%endhead 
 
\subsection{В Украине стремительно растёт кривая ненависти к США}
\label{sec:31_01_2022.fb.vakarov_vasyl.1.krivaja_nenavisti_k_usa}
 
\Purl{https://www.facebook.com/vasylvd/posts/3177361902587493}
\ifcmt
 author_begin
   author_id vakarov_vasyl
 author_end
\fi

В Украине стремительно растёт кривая ненависти к США.

Этот процесс начался ещё в начале апреля прошлого года, когда президент Украины
Зеленский потребовал от США немедленного принятия Украины в НАТО - как
инструмента защиты от России.

С каждым днём затяжки положительного решения, украинская власть проявляла
сначала негодование, а затем разочарование позицией Администрации Байдена.

Градус в отношениях подпортил Госсекретарь Блинкен, который с мая месяца
настойчиво потребовал (и продолжает это делать) «побороть коррупцию и провести
реформы в Украине».

Негативное отношение к новой администрации США перекочевало и в украинское
общество.

Атака американских СМИ, начатая в конце октября прошлого года («Россия
нападёт!») на головы украинцев, получила очень резкую реакцию властей и лично
Зеленского, а также неодобрение части украинского общества.

Люди стали понимать:

Америка их не защитит;

их хотят «погнать на убой» за чужие интересы, подменяя всё словами о
«ценностях».

Тенденция..
