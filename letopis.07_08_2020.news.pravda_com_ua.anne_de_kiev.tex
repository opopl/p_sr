% vim: keymap=russian-jcukenwin
%%beginhead 
 
%%file 07_08_2020.news.pravda_com_ua.anne_de_kiev
%%parent 07_08_2020
 
%%endhead 
\subsection{Нарешті, сквер Анни Київської у Києві став реальністю}
\url{https://life.pravda.com.ua/columns/2020/08/7/241909/}
  
\vspace{0.5cm}
 {\ifDEBUG\small\LaTeX~section: \verb|07_08_2020.news.pravda_com_ua.anne_de_kiev| project: \verb|letopis| rootid: \verb|p_saintrussia| \fi}
\vspace{0.5cm}

Федір Баландін засновник Міжнародного фестивалю мистецтв "Anne de Kyiv Fest"
268 1 В закладки 7 серпня 2020

Блискуче одинадцяте століття!

Пентеконтаетія Києва. Час, коли ми найкращі в усьому, коли ми – молода
дерзновенна держава.

Ярослав Мудрий – Конунг Яріслейф – і його Русь.

37 років перемог і здобутків.

Уже є з чим порівнювати. У сучасній Україні за приблизно такий самий час
новітньої історії лише президентів змінилось шестеро. І втрати, втрати, втрати…

Територій, промисловості, заощаджень громадян, іноземної власності, ядерної
зброї, шельфу, чого ще?

А головне – людей, громадян.

Чвари та брак загальновизнаних національних особистостей-символів.

Маючи великий список історичних постатей світового рівня і впливу, прославляємо
та "розмальовуємо" рідне місто іменами вельми суперечних "героїв" посереднього
масштабу і радіємо, мов діти, що змогли скрутити нову "солодку дулю" у бік
чергової "саратовської-мурманської" вулиці та внести до існуючого
урбанонімічного хаосу нову загадку для киян.

Реалізувавши 2016 року проєкт встановлення в Києві пам’ятника Анні, доньці
Ярослава Мудрого, я активно просував ідею найменування скверу на Львівській
площі її святим іменем.

Пам'ятник Анні Ярославні – королеві Франції – на Львівській площі в Києві.  Фото
vkieve.net

За три роки пройдено попередні комісій, громадські слухання, просто комісії.

І ось, нарешті, Сквер Анни Київської став реальністю.

І хоча ця назва вже давно є "народною" (верифікованою Google Maps), а маленький
пам’ятник Великій Киянці п’ятий рік "володарює" у сквері, все ж офіційне
визнання є важливим першим кроком.

Насамперед в усвідомленні громадою, що хай і не умовна тисяча років, але не
набагато менша часова відстань може бути обов’язковою умовою втручання нас,
недосконалих сучасників, у топоніміку Міста віком дві тисячі років.

Як ми з такою швидкістю можемо розглядати присвоєння умовній вулиці імен
середньостатистичних діячів-лузерів початку двадцятого століття?

Чи підозрілих і контраверсійних постатей кінця ХХ – початку ХХІ століття,
часто-густо звичайних авантюристів-ландскнехтів як з Півночі, так і з Заходу?

Тим часом на мапі столиці відсутня практично вся родина Ярослава: Інгігерда,
Добронега, Єлизавета, Агата, Анастасія… 

Ця золота доба має стати для українців тим хребтом національної єдності, якого
так бракує суспільству!

Пройшовши вулицею Володимира, повертаємо на Вал Ярослава, звідтам потрапляємо у
Сквер Анни Київської.

Це навіть не мелодія рідного міста. Це марш перемог і справжньої історії
Українського Успіху.

Символічно, що київська комісія з найменувань проголосувала за Сквер Анни
Київської, Королеви Франції одностайно.

На відміну від багатьох інших рішень.

Далі буде.

Федір Баландін, засновник Міжнародного фестивалю мистецтв "Anne de Kyiv Fest",
спеціально для УП.Життя

Титульна світлина 

Вас також може зацікавити: 

Ми хочемо тримати з вами зв'язок. Будемо раді бачитися і спілкуватися з вами на
наших сторінках у Facebook та у Twitter.

А якщо хочете бути в курсі лише новин та важливої інформації про здоров'я,
підписуйтесь на нашу Facebook-групу про здоров'я та здоровий спосіб життя.
  
