% vim: keymap=russian-jcukenwin
%%beginhead 
 
%%file 12_04_2023.stz.news.ua.donbas24.1.mrpl_teatr_conception_vystava_ternopil
%%parent 12_04_2023
 
%%url https://donbas24.news/news/mariupolskii-teatr-conception-pokazav-u-ternopoli-svoyu-vistavu
 
%%author_id demidko_olga.mariupol,news.ua.donbas24
%%date 
 
%%tags 
%%title Маріупольський театр "Conception" показав у Тернополі свою виставу
 
%%endhead 
 
\subsection{Маріупольський театр \enquote{Conception} показав у Тернополі свою виставу}
\label{sec:12_04_2023.stz.news.ua.donbas24.1.mrpl_teatr_conception_vystava_ternopil}
 
\Purl{https://donbas24.news/news/mariupolskii-teatr-conception-pokazav-u-ternopoli-svoyu-vistavu}
\ifcmt
 author_begin
   author_id demidko_olga.mariupol,news.ua.donbas24
 author_end
\fi

\ii{12_04_2023.stz.news.ua.donbas24.1.mrpl_teatr_conception_vystava_ternopil.pic.front}

\begin{center}
  \em\color{blue}\bfseries\Large
  Колектив театру авторської п'єси \enquote{Conception} виступив на тернопільській сцені
\end{center}

7 квітня маріупольський театр авторської п'єси \enquote{Conception} на сцені
Тернопільського драматичного театру імені Т. Г. Шевченка представив свою
виставу \href{https://donbas24.news/news/zittya-pereselenske-mariupolskii-teatr-pidgotuvav-novu-premjeru}{%
\emph{\enquote{Життя переселенське}}}.% 
\footnote{\enquote{Життя переселенське} — маріупольський театр підготував нову прем'єру, Ольга Демідко, donbas24.news, 28.02.2023, \par\url{https://donbas24.news/news/zittya-pereselenske-mariupolskii-teatr-pidgotuvav-novu-premjeru}}

Цей документальний спектакль про труднощі,
небезпеку та непередбачені ситуації по дорозі до вільної від агресорів
території приємно вразив глядачів, які довго не хотіли відпускати маріупольців
зі сцени.

\textbf{Читайте також:} \emph{Маріупольські актори зіграли благодійну виставу в Києві}%
\footnote{Маріупольські актори зіграли благодійну виставу в Києві, Артем Батечко, donbas24.news, 17.03.2023, \par%
\url{https://donbas24.news/news/mariupolski-aktori-zigrali-blagodiinu-vistavu-v-kijevi-foto}, \par%
Internet Archive: \url{https://archive.org/details/17_03_2023.artem_batechko.donbas24.mrpl_aktory_zigraly_blagodijnu_vystavu_v_kyevi}%
}

Ідея поїхати з виставою у Тернопіль належить директорці Департаменту
культурно-громадського розвитку Маріупольсьскої міської ради \emph{\textbf{Діані Тримі}}, яка
наголосила, що ця поїздка відбулася в рамках проєкту \enquote{Культурна деокупація},
оскільки наразі відбувається формування загального інформаційного простору для
об'єд\hyp{}нання маріупольців та однодумців. Був обраний саме спектакль \enquote{Життя
переселенське}, адже він є зрозумілим багатьом українцям, які вимушені покидати
свої домівки на окупованих територіях.

\begin{leftbar}
\emph{\enquote{Ці емоції повною мірою неможливо передати ні картинами, ні словами на
папері. Я бачила транформацію людей, які сиділи в залі. До та після
перегдяду \enquote{Життя переселенське}. Мистецтво під час війни відзеркалює
події сьогодення та не дає нам змоги забувати про важливі для нас самих
речі}}, — наголосила Діана Трима.
\end{leftbar}

\ii{insert.read_also.demidko.donbas24.u_kyevi_vidnovyly_mariupolsku_vystavu}
\ii{12_04_2023.stz.news.ua.donbas24.1.mrpl_teatr_conception_vystava_ternopil.pic.1}

За словами акторів театру, це своєрідне продовження тематики повномасштабної
війни у творчості колективу.

\begin{leftbar}
\emph{\enquote{Найскладніше для кожного з наших акторів було знову занурюватися у ті
події, які всі ми пережили, коли виїжджали з Маріуполя під обстрілами.
Знову і знову повертатися у важке минуле — це дійсно морально важко.
Але ми намагаємося впоратися, бо в нас є дуже велика мета —
продовжувати розповідати свою історію}}, — зауважила актриса театру
Марина Говорущенко.
\end{leftbar}

Художній керівник і режисер театру авторської п'єси \enquote{Conception} \textbf{\emph{Олексій Гнатюк}}
зауважив, що гастрольна діяльність важлива для колективу, оскільки це завжди
сприяє розвитку. 

\ii{12_04_2023.stz.news.ua.donbas24.1.mrpl_teatr_conception_vystava_ternopil.pic.2}

\begin{leftbar}
\emph{\enquote{Нас дуже тепло зустріли. Без особливих зусиль нам вдалось відтворити
декорації. Ми відчули, що глядачі співпереживали нашій історії. А після
спектаклю ми змогли ближче познайомитися з чудовим містом Тернопіль,
погуляли його вулицями і парками. Залишились дуже приємні спогади і
позитивні емоції}}, — поділився Олексій Гнатюк.
\end{leftbar}

\ii{insert.read_also.demidko.donbas24.chy_ljubyv_marko_kropyvnyckyj_priazovja_stvoren}
\ii{12_04_2023.stz.news.ua.donbas24.1.mrpl_teatr_conception_vystava_ternopil.pic.3}
\ii{12_04_2023.stz.news.ua.donbas24.1.mrpl_teatr_conception_vystava_ternopil.pic.5}

У глядацькій залі були присутні і маріупольці, і запрошені гості, які були
захоплені історією про важкі випробування під час переселення. Глядачі відчули
весь біль та невпевненість, які пережили маріупольські переселенці, але
водночас були вражені силою духу і мужністю, завдяки яким героям вдалося знайти
вихід з нелегких ситуацій.

\ii{12_04_2023.stz.news.ua.donbas24.1.mrpl_teatr_conception_vystava_ternopil.pic.4}

\begin{leftbar}
\emph{\enquote{Ви неймовірні! Дякую всім учасникам театру! Така віддача! Пройшли по всім
болючи точкам... Наші історії однакові. Ми пережили, ми вижили, всі
разом... ще раз}}, — поділилась емоціями одна з глядачок. 
\end{leftbar}

\ii{insert.read_also.demidko.donbas24.teatromania_rozv_pidtrym_kulturu_mrpl}
\ii{12_04_2023.stz.news.ua.donbas24.1.mrpl_teatr_conception_vystava_ternopil.pic.6}

Найближчим часом театр авторської п'єси \enquote{Conception} планує відновити виставу
\enquote{Страхи в стилі ню} — психологічну драму на одну дію, яка не залишить нікого
байдужим. Також планується підготувати документальний фільм, в основу якого
ляжуть спогади режисера і акторів про пережите в Маріуполі та важкий виїзд з
окупованого міста.

Раніше Донбас24 розповідав унікальні факти про \href{https://archive.org/details/27_03_2023.olga_demidko.donbas24.unikalni_fakty_teatr_kultura_priazovja}{театральну культуру Маріуполя}.%
\footnote{Унікальні факти про театральну культуру Приазов'я, Ольга Демідко, donbas24.news, 27.03.2023, %
\par\url{https://donbas24.news/news/unikalni-fakti-pro-teatralnu-kulturu-priazovya-do-vsesvitnyogo-dnya-teatru}, \par%
Internet Archive: \url{https://archive.org/details/27_03_2023.olga_demidko.donbas24.unikalni_fakty_teatr_kultura_priazovja}%
}

Ще більше новин та найактуальніша інформація про Донецьку та Луганську області
в нашому телеграм-каналі Донбас24.

Фото: Євгена Сосновського, Театру авторської п'єси \enquote{Conception} та з відкритих
джерел

\ii{insert.author.demidko_olga}
%\ii{12_04_2023.stz.news.ua.donbas24.1.mrpl_teatr_conception_vystava_ternopil.txt}
