% vim: keymap=russian-jcukenwin
%%beginhead 
 
%%file 04_01_2022.yz.figurka_ot_maksima_jagudina.1.chempionaty_evropy.4.budapest_2014
%%parent 04_01_2022.yz.figurka_ot_maksima_jagudina.1.chempionaty_evropy
 
%%url 
 
%%author_id 
%%date 
 
%%tags 
%%title Чемпионат Европы 2014 (Будапешт)
 
%%endhead 
\subsubsection{Чемпионат Европы 2014 (Будапешт)}
\label{sec:04_01_2022.yz.figurka_ot_maksima_jagudina.1.chempionaty_evropy.4.budapest_2014}

\ii{04_01_2022.yz.figurka_ot_maksima_jagudina.1.chempionaty_evropy.4.budapest_2014.pic.1}

Спустя 8 лет после победы Иры Слуцкой на Чемпионате Европы снова прозвучал
российский гимн в честь нашей победы в женском виде. Победу на предолимпийском
первенстве континента одержала дебютантка взрослых стартов 15-летняя Юлия
Липницкая, которая к тому моменту уже успела покорить сердца многих болельщиков
и завоевать золото и серебро на двух юниорских Чемпионатах Мира. Юля первая
чемпионка группы Этери Тутберидзе, лучшей сейчас женской группы современности.
В том сезоне фамилия Липницкой, что называется не сходила с первых полос газет.

\ii{04_01_2022.yz.figurka_ot_maksima_jagudina.1.chempionaty_evropy.4.budapest_2014.pic.2}

После золота на Европе будет олимпийское золото командника и серебро Чемпионата
Мира. При этом в Сочи Юля откатала по два раза короткую и произвольную
программы в крайне сжатые сроки и на максимальном уровне концентрации и
ответственности. Это было, как показали следующие сезоны, своеобразное
\enquote{выгорание}, следующих главных стартов в карьере Юли не будет, лишь несколько
этапов Гран-при и быстрое завершение карьеры. 
