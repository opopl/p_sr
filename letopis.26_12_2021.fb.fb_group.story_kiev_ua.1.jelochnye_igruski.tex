% vim: keymap=russian-jcukenwin
%%beginhead 
 
%%file 26_12_2021.fb.fb_group.story_kiev_ua.1.jelochnye_igruski
%%parent 26_12_2021
 
%%url https://www.facebook.com/groups/story.kiev.ua/posts/1827140537482739
 
%%author_id fb_group.story_kiev_ua,manilova_svetlana.kiev.ukraina
%%date 
 
%%tags igrushka,igrushka.jolka,jolka,kiev,novyj_god,prazdnik
%%title Наши старые  елочные игрушки. У них, как и у нас, есть свои киевские истории
 
%%endhead 
 
\subsection{Наши старые елочные игрушки. У них, как и у нас, есть свои киевские истории}
\label{sec:26_12_2021.fb.fb_group.story_kiev_ua.1.jelochnye_igruski}
 
\Purl{https://www.facebook.com/groups/story.kiev.ua/posts/1827140537482739}
\ifcmt
 author_begin
   author_id fb_group.story_kiev_ua,manilova_svetlana.kiev.ukraina
 author_end
\fi

Наши старые елочные игрушки. У них, как и у нас, есть свои киевские истории.
Они, как приветы из прошлого, радуют глаз и возвращают в те добрые и светлые
детские годы. Они так же, как и мы, ждут праздников, чтобы оказаться на
новогодней елке. Каждый год приходится расставаться с некоторыми из них. От
этого становится немного грустно. 

\ii{26_12_2021.fb.fb_group.story_kiev_ua.1.jelochnye_igruski.pic.1}

Вот, и в прошлом году ушла моя самая первая елочная игрушка- розово- сиреневый
колокольчик. Он всегда находил свое почетное место на елке, будучи уже совсем
стареньким и совсем непривлекательным. Но он прошел со мной свою жизнь, радуя
меня каждый год. Пришлось расстаться и с ним... Он проделал свой путь с
Подольского универмага в мой родной дом детства на Подоле, затем его ждал
переезд в новую квартиру на Лукьяновке, а потом он поселился в доме, в котором
я живу сейчас. Здесь колокольчик, как и мои другие елочные игрушки, переехавшие
со мной, ждали игрушки мужа, которые с удовольствием приняли их в свою
компанию, разместившись в одной коробке. 

\ii{26_12_2021.fb.fb_group.story_kiev_ua.1.jelochnye_igruski.pic.2}

Каждая игрушка имеет свою историю, помнит руки тех, кто их держал когда-то,
передавая тепло наших родных, которых уже тоже с нами нет... Ежегодно, конечно,
в доме появлялись и новые елочные украшения, но эти были как-то родней и дороже
сердцу. Скоро я достану свою коробку и снова встречусь с ними, чтобы потом,
когда праздники закончатся, спрятать игрушки вновь и сказать: \enquote{До
встречи в следующем году!}

\ii{26_12_2021.fb.fb_group.story_kiev_ua.1.jelochnye_igruski.cmt}
