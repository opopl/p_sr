% vim: keymap=russian-jcukenwin
%%beginhead 
 
%%file 19_11_2021.fb.vasiljev_sergej.moskva.1.vystavka_piter_kurakin_usadjby.cmt
%%parent 19_11_2021.fb.vasiljev_sergej.moskva.1.vystavka_piter_kurakin_usadjby
 
%%url 
 
%%author_id 
%%date 
 
%%tags 
%%title 
 
%%endhead 
\subsubsection{Коментарі}

\begin{itemize} % {
\iusr{Вера Ткаченко}
А до какого выставка?

\iusr{Сергей Васильев}
\textbf{Вера Ткаченко} вроде бы до середины января.

\iusr{Вера Ткаченко}
\textbf{Сергей Васильев} Спасибо. Мне и декабря хватит.)

\iusr{Леля Сорокина}
там Антон Батагов играл? Это вроде же та же выставка что должна скоро приехать в Москву.

\iusr{Сергей Васильев}
\textbf{Леля Сорокина} да.

\iusr{Анастасия Васильченко}

Естественно работы предложили мы, куратор про такого художника не знал) Еще
одну работу А.Б. Куракина со Степановским-Волосово, которую мы любезно
предложили куратору (и в 3 недели оформили выдачу через министрерство культуры,
госкаталог, внутр. фондовую комиссию и привезли СПб; даже она висела не стене
при монтаже), куратор недавно прислал обратно... Видимо, не уместилась на
экспозиции огромного Манежа! Кроме того про Причетникова все обьяснили, что в
Твери (из Степановского) худший варианты по живописи, чем наши из Надеждино.

\iusr{Maria Shuvalova}
\textbf{Анастасия Васильченко} 

интересно про Причетникова  @igg{fbicon.cat.weary}, нам показалось, что они
просто другие ))) те, которые наши

\iusr{Анастасия Васильченко}
\textbf{Maria Shuvalova} 

Про это писала Н. Н. Скорнякова, которая хранит нашу серию и их внимательно изучала.

\iusr{Inna Vollstaedt}
Хорошие работы и хорошо, что их сейчас выставляют.

\iusr{наталья коваленко}
Красота

\end{itemize} % }
