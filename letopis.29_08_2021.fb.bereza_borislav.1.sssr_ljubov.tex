% vim: keymap=russian-jcukenwin
%%beginhead 
 
%%file 29_08_2021.fb.bereza_borislav.1.sssr_ljubov
%%parent 29_08_2021
 
%%url https://www.facebook.com/borislav.bereza/posts/6733830853309630
 
%%author Береза, Борислав
%%author_id bereza_borislav
%%author_url 
 
%%tags ljubov,obschestvo,pamjat,sssr,zhizn
%%title Не понимаю, как можно не любить СССР?
 
%%endhead 
 
\subsection{Не понимаю, как можно не любить СССР?}
\label{sec:29_08_2021.fb.bereza_borislav.1.sssr_ljubov}
 
\Purl{https://www.facebook.com/borislav.bereza/posts/6733830853309630}
\ifcmt
 author_begin
   author_id bereza_borislav
 author_end
\fi

\begin{multicols}{2}

Не понимаю, как можно не любить СССР? Ведь в СССР было столько замечательного и
все было для людей. Вспомните, какие замечательные были продукты. Помните ту
самую полусгнившую картошку, которой пропахли все советские овощные? А
вечнозеленые помидоры, помните? А пельмени помните, которые слипались в пачке в
один пельмень, потому что были несколько раз разморожены-заморожены? И мы это
ели. С удовольствием. Ведь другого на прилавках не было. Думаете, что это
потому что при СССР качество продуктов было низким? Нет! Этим всем нас готовили
к сложным временам. Нас учили выживать. И научили. Потому что не каждый человек
может выжить, если его кормить тем, что было в советских магазинах. Это сейчас
в супермаркетах помидоры 6 сортов и хлеб 20 видов, а тогда были помидоры и
хлеб. Без сортов и видов. И люди тоскуют по тем временам. Потому что не было
этой проблемы выбора. Брали то, что было, если оно вообще было в продаже. 

А очереди? Помните какие потрясающие очереди были тогда. Стояли за всем. За
продуктами, мебелью, книгами, сапогами и одеждой. И в этих очередях нас учили
не уставать, не пропускать тех, кто хочет пролезть без очереди и потом
радоваться, что удалось купить такие дефицитные вещи, как туалетная бумага,
сигареты или зубная паста. В тех очередях советские люди знакомились, женились,
рожали детей и умирали, но не сдавались. Ведь в этих очередях нас приучали к
стойкости и выносливости. И на подкорке у советского человека, рядом со словами
"Куда по помытому?" отпечаталось и "Куды без очереди?!". Это сейчас очередь
возможна за новым iPhone, а тогда люди по 10 лет в очереди на домашний телефон
стояли. И не все дожидались.
\vspace{0.5cm}

%\columnbreak

\begin{minipage}{0.45\textwidth}
\ifcmt
  ig https://scontent-cdt1-1.xx.fbcdn.net/v/t39.30808-6/240846130_6733830749976307_8975226906869169477_n.jpg?_nc_cat=101&_nc_rgb565=1&ccb=1-5&_nc_sid=8bfeb9&_nc_ohc=-HIsjVDx4pgAX9Sqsfl&_nc_ht=scontent-cdt1-1.xx&oh=6533aefdac8324b0266516906e6b344d&oe=6134E754
\fi
\end{minipage}

\end{multicols}

А покупка продуктов? Помните этот квест? Кстати, слова "квест" мы не знали, но
все подчинялись общему правилу - найди продукты или будь голодным. Поэтому,
живя в полукружке на Дарницком бульваре, я покупал молоко в гастрономе возле
почты на Малышко, а потом бежал за ряженкой и разливной сметаной на Бойченко.
Там был  молочный магазин и в нем шанс купить молочку был выше, чем в
гастрономе. И если кто-то не может понять почему нельзя все было купить в одном
магазине, то причина была одна - всего в одном месте не было. Ведь таким
образом нас так учили хорошо ориентироваться на местности и выполняя правила
квеста приносить домой все продукты. И если нам везло, то в процессе добычи еды
мы могли получить бонус - выстоять очередь и купить синюю птицу счастья. На
ценнике было написано, что это курица, но она была реально синего цвета и это
было счастье, если её удалось купить. И мы радовались этому счастью. А сейчас
зашёл в магазин, смотришь на ассортимент и понимаешь, что добывать ничего не
надо. Нужно все сложить в корзину и идти на кассу. Но это же примитивно и убило
ежедневный квест в нашей жизни.

А помните водку при СССР? Названия были разные Пшеничная, Русская, Столичная,
Посольская, но народ их разделял на два типа с "закруткой" и "бескозырка". И
если вначале у "бескозырки" был хвостик за который дёргали и бутылка
открывалась, то потом хвостик убрали и открывать стали зубами, вилками, ножами.
Так людей учили находить выход из сложных ситуаций. И научили. Потому что
заботились о людях. Не то что сейчас - открутил и водка сама полилась. И
никакого удовольствия от преодоления трудностей. И даже в очереди за водкой
стоять не надо. А какое же удовольствие пить её без преодоления трудности по
добыче водки.

А как хорошо люди вместе проводили время? Помните регулярные партсобрания и
политинформации? На них людей превращали в единый организм, которому указывали
кого надо ненавидеть - евреев, США, диссидентов, Сахарова, писателей. Да,
идеология тогда была. Пусть она была человеконенавистнеческая, но была. И её
сейчас так не хватает потомкам вохровцев и НКВДистов, детям партработников и их
обслуги. А ещё многие из них тоскуют по словам "блат" и "дефицит". Потому что
все качественное, красивое, вкусное и нужное можно было достать только по
блату, ведь все было в дефиците. Какие времена были! И по ним скучают. 

А помните КГБ? Там работали светлые люди, которые тратили свою жизнь на то,
чтоб бороться с диссидентами и теми, кому было близко инакомыслие. А ещё в КГБ
работали над тем чтоб люди не засырали себе мозги книгами Оруэлла и Булгакова,
Аксенова и Набокова, Пастернака и Солженицина. А то прочтут и всяким
вольнодумством начнут заниматься. Кстати, КГБ занималось и борьбой с
вольнодумством, но видимо не очень успешно, потому что СССР таки сдох. Но ведь
тогда люди занимались творчеством, благодаря сотрудничеству с КГБ. Советские
люди с воодушевлением и усердием писали анонимки и кляузы на соседей, друзей,
знакомых, коллег и всех остальных. Вот это времена были! А сейчас с этим
неплохо справляются анонимные телеграм-каналы. Но без души. Нет того
творческого порыва у людей - писать доносы в КГБ. Отвыкли. 

А какие лидеры были тогда? Один Сталин чего только стоит? Какую страну
построил! Именно при нем появились ГУЛАГ и Беломорканал. Этим же гордиться
можно? А как при нем расцвела наука и культура? На каждой "зоне" были
профессора и актёры, академики и инженеры, поэты и писатели. Некоторых правда
расстреляли, но как известно лес рубят - щепки летят. И таких щепок набралось с
1930 по 1953 год немало. Было 3 778 234 осуждённых политических и из них 786
098 приговорили к расстрелу. Но это же мелочи, да? Страну какую отгрохали.
Правда народ из неё бежал при первой возможности, но это они от непонимания
своего счастья жития в СССР. Ведь СССР - это была страна счастья.

И вот глядя на все выше перечисленное не понятно, как можно не любить СССР -
страну, которая все делала для того, чтоб каждый гражданин был мелким винтиком
в общем механизме тоталитарного режима. Это же счастье жить всем вместе в такой
стране. Пусть и в стране-бараке, но вместе. И мороженое же было вкусное, хоть и
по 20 копеек, а все остальное было бесплатное. Но последнее не точно. Хотя люди
считают именно так. Потому и скучают по СССР - стране, которая все делала для
людей. Она даже сдохла, чтоб люди стали свободными и счастливыми. Но не всем
это удалось. И теперь они с грустью говорят "Какую страну проср@ли!", тоскуют
по всем прелестям СССР и ностальгируют по самому вкусному, на их взгляд,
мороженому в стаканчиках, которое тоже было тогда в дефиците. А о чем будут
тосковать сегодняшние дети, если в супермаркетах можно купить любое мороженое и
даже Haagen-Dazs со вкусом солёной карамели. И вот, как можно не любить после
всего этого СССР?



