% vim: keymap=russian-jcukenwin
%%beginhead 
 
%%file slova.vata
%%parent slova
 
%%url 
 
%%author 
%%author_id 
%%author_url 
 
%%tags 
%%title 
 
%%endhead 
\chapter{Вата}
\label{sec:slova.vata}

Коли я через декілька тижнів повернувся у Харків, застав, мабуть, останній
розгін \emph{вати}, коли містом ішов проукраїнський марш. Його перестріли \emph{ватники},
проте вони не співставили сили. Їх відгупали – і це була остаточна перемога.
Потім, у 15 році, ще був теракт, коли вибухнула міна, яка повбивала людей і
серед них був зовсім юний хлопець Даня Дідік. Я знаю його батька - він з того
часу двіжує по різних проукраїнських акціях. А ще були погрози, що типу на 9
травня теж будуть \emph{ватні} теми. Але на той час у нас вже сформувалася рота
\enquote{Східний корпус}. (Підрозділ патрульної служби поліції особливого призначення).
Вона була не дуже велика, але 9 травня вони їздили на джипах, з кулеметами з
вікон – \emph{вата}, звісно, перелякалася, 
\textbf{Доброволець Олександр Воробєй: \enquote{Війна не змінює людину, вона
просто розкриває її сутність. Якщо людина була гандоном, то після фронту вона
стане махровим п\#дарасом}},
Віка Ясинська, censor.net.ua, 02.06.2021

Коли почалися події на сході, зокрема і в моєму місті, я був у Києві. І ми
зібралися з хлопцями на так званий \enquote{Автопроїзд дружби}, який мав допомагати
ганяти \emph{вату} в Харкові, Луганську, Донецьку. Але під Чутовим (Полтавська
область) нас зупинив \enquote{Беркут} і сильно усіх відгамселив. Тоді з нами їхав мій
друг Сергій Бондар, псевдонім Тімур, а у нього була корочка помічника депутата,
\textbf{Доброволець Олександр Воробєй: \enquote{Війна не змінює людину, вона
просто розкриває її сутність. Якщо людина була гандоном, то після фронту вона
стане махровим п\#дарасом}},
Віка Ясинська, censor.net.ua, 02.06.2021

%%%cit
%%%cit_head
%%%cit_pic
%%%cit_text
Бог Данилова затопил Крым. Заслуженный ветеринар СНБО Данилов пошутил о
наводнении в Крыму. Удивило кого-то? Не больше, чем следователи удивлялись
какому-то новому эпизоду в деле Чикатило. Ага, не 52, а таки 53 трупа.  Все что
несет Данилов и подобные ему персонажи – это история из сумасшедшего 2013 года.
Именно тогда нам здесь запустили технологию расчеловечивания своих же
сограждан. Сначала появился «кровавый режим» Януковича. Затем всплыла
«беркутня» и «менты – не люди». А за них топят только «титушки». Которые из
«Даунбасса». А самые темные из «Лугандона». Ведь именно там голосуют за
«рыгов», которые хотят в «таежный союз».  Когда полилась кровь – а как она
могла не политься, когда прямой наводкой в мозг бьет из всех эфиров «Радио
тысячи холмов»? – в эпитетах уже не заморачивались. «Сепары», «самки
колорадов», «русня», \emph{«вата»}, «дедывоевали» и «там нет генов»
%%%cit_comment
%%%cit_title
\citTitle{Бог Данилова - это бог расчеловечивания}, 
Игорь Лесев, strana.ua, 20.06.2021
%%%endcit

%%%cit
%%%cit_head
%%%cit_pic
%%%cit_text
\enquote{Такі рішення стратегічно важливі, адже на сьомому році війни з Росією у нас
був пам'ятник дружби з Москвою. Звісно, зараз у росіян і \emph{ватників} бомбитиме,
про що свідчить навіть ейджизм і сексизм у депутата від ОПЖЗ, який таким чином
пробував зірвати розгляд цього рішення. Але такі рішення правильні і цього
декілька років безуспішно добивалась громада району}, - розповіла LB.ua
ініціаторка рішення, депутатка від фракції \enquote{Голос} Аліна Михайлова
%%%cit_comment
%%%cit_title
\citTitle{\enquote{Київрада вирішила демонтувати пам'ятник дружби Києва та Москви}}, 
Олександр Рудоманов, lb.ua, 08.07.2021
%%%endcit

%%%cit
%%%cit_head
%%%cit_pic
%%%cit_text
Спостерігаючи за такими специфічними проявами поведінки не більшості, а саме
активної меншості, неминуче виникає запитання: а чи існує принципова різниця
між українською і російською «\emph{ватою}»? Адже сам термін «\emph{вата}» походить від
«\emph{ватника}», одягу в'язнів ГУЛАГу. «\emph{Ватниками}» першопочатково називали людей, чия
свідомість відформатована радянською тоталітарною системою до того рівня, що
вони стали найвідданішою опорою нової авторитарної влади. В Україні, на щастя,
авторитарної влади нема – на відміну від сусідніх Білорусі та Росії. Але й
радянське щеплення від несвободи, схоже, не спрацювало. Тисячі активних
громадян готові шукати «ворогів народу», звинувачувати в усіх гріхах навіть
вигадану особу, про яку їм взагалі майже нічого не відомо, вимагати для неї
найжорстокішого покарання за навіть невідомий їм злочин. І як тут дивуватися на
тему «хто написав мільйони сталінських доносів»? Та от такі щирі активісти й
написали
%%%cit_comment
%%%cit_title
\citTitle{Червоно-чорна «вата»}, Павло Зуб'юк, zaxid.net, 03.11.2021
%%%endcit

%%%cit
%%%cit_head
%%%cit_pic
%%%cit_text
Особое бешенство у майданных горлумов вызвал флешмоб с вывешиванием в видных
местах различных украинских городов баннеров «Да, мы один народ». В течении
одной недели эти плакаты появились в Одессе, Херсоне, Николаеве, Запорожье,
Днепропетровске.  «\emph{Вата} вообще обнаглела. Еду, и на мосту висит цитата
Путина. И никто не догадался сорвать! В Днепре, рядом с наибольшим
предприятием», — возмущается днепропетровский «патриот», снявший видео о
плакате в своём городе. «Особое возмущение вызывает тот факт, что сотни
проезжающих водителей и тысячи проходящих людей до сих пор не додумались
сорвать этот провокационный баннер… Очередная провокация сепаров, которых ещё
много в Украине! Так и хочется ещё раз процитировать "Чемодан-Вокзал-Россия"»,
— беснуются в промайданном ТГ-канале «Украина Online». Который, к слову
сказать, собственную новостийную ленту ведёт на русском языке — все-таки хотят,
чтобы их читали
%%%cit_comment
%%%cit_title
\citTitle{Геть от цивилизации! Одна нация, одна мова, один гетман}, 
Константин Кеворкян, ukraina.ru, 11.11.2021
%%%endcit
