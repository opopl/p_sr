% vim: keymap=russian-jcukenwin
%%beginhead 
 
%%file 14_02_2022.fb.fb_group.story_kiev_ua.2.kievskij_hlebushek.cmt
%%parent 14_02_2022.fb.fb_group.story_kiev_ua.2.kievskij_hlebushek
 
%%url 
 
%%author_id 
%%date 
 
%%tags 
%%title 
 
%%endhead 
\zzSecCmt

\begin{itemize} % {
\iusr{Владимир Новицкий}

Супер фото, все мы так делали и правда не возможно было удержаться от этого
запаха вкуснейшего хлеба!!

\iusr{Оксана Дубинина}
\textbf{Владимир Новицкий}  

@igg{fbicon.face.happy.two.hands}  @igg{fbicon.face.smiling.hearts}  А
мальчишки еще могли и к хлебзаводу за горяченьким хлебом
сбегать) @igg{fbicon.wink}
@igg{fbicon.face.smiling.sunglasses} 

\iusr{Владимир Новицкий}
\textbf{Оксана Дубинина} да и я там был!!

\iusr{Светлана Манилова}
Оксаночка, спасибо за такие вкусные воспоминания! Руки так и потянулись за
кусочком хлеба...@igg{fbicon.heart.red}

\begin{itemize} % {
\iusr{Оксана Дубинина}
\textbf{Светлана Манилова} 

@igg{fbicon.heart.red} @igg{fbicon.face.savoring.food}
@igg{fbicon.face.grinning.smiling.eyes}  приятного аппетита!

\iusr{Євгеній Гончарук}
\textbf{Светлана Манилова} Так сегодняшний так не хрустит.

\iusr{Віктор Зайчук}
\textbf{Євгеній Гончарук} Раніше і дерева були високими, а ми - молодими...

\iusr{Олександра Мазаєва}
У нас, в сім'ї \enquote{офіційно} дозволялось обгризаний по колу Український

\iusr{Tamara Bialik}
А забули арнаутку!

\end{itemize} % }

\iusr{Ирина Иванченко}

Это преступление против человечества... человечности... сорри, человеколюбия -
аппетитный пост с такой вкуснятиной и нА ночь, Оксана !!! Придётся ставить
чайник... А ведь так спокойно все шло ко сну...

\begin{itemize} % {
\iusr{Светлана Манилова}
Ирина, я не решилась так высказаться, но поддерживаю Вас! @igg{fbicon.face.tears.of.joy} 

\iusr{Раиса Карчевская}
\textbf{Светлана Манилова} Я тоже

\iusr{Кузя Музя}
Это точно! Уже, \enquote{не до сна} !
\end{itemize} % }

\iusr{Ирина Хмара}

Корочку))) Мы, пока не открыли на Половецкой хлебный, ходили с подружками на
Нагорную. Пока дойдем каждая съест больше половины. Украинский гарячий такой
вкуснючий был. Вначале хрустящие края шли, а потом и мякушка хорошо шла. А
иногда за разговорами и случайно почти весь сьедался. Приходилось идтить за
новым хлебом.

\begin{itemize} % {
\iusr{Оксана Дубинина}
\textbf{Ирина Хмара}
Да, Ирочка, мне тоже особо «украинский» черный нравился.
На Нагорной классный был магазин и еще на Мурашко в «хлебный» горячий хлеб привозили, а еще на Лукьяновке))
Еще вспоминается магазин «Паляниця», где сейчас Макдональдс расположился, там всегда свеженький хлеб был

\iusr{Ирина Хмара}
\textbf{Оксана Дубинина} на Майдане как-то даже вкуснее вроде был. Там и выбор был больше. Паляница с корочкой тоже вкусная была

\iusr{Marina Shepeleva}
\textbf{Оксана Дубинина} а булочка по 6 и по 3 копейки - это было просто наслаждение. Сейчас вообще такого хлеба нет - жаль

\iusr{Анна Анна}
\textbf{Оксана Дубинина} , мы тоже на Мурашко ходили. С Лермонтовской

\iusr{Sasha Ty}
\textbf{Ирина Хмара}
Які ходячі на хлібах, а?  @igg{fbicon.smile} 
І паляниця в них пішла, і жопки..... 

Довелось спостерігати два хлібзаводи з середини: на Шулявці і Дарниці.  Дівчата
там хоть куди були, тому і хліб смачний але б мене придавили б як комара якщо
щось.

\iusr{Оксана Дубинина}
\textbf{Sasha Ty} 

так на таких молодичок любо-дорого дивитися)))
@igg{fbicon.face.grinning.smiling.eyes}  @igg{fbicon.wink} 

\iusr{Eva Evidze}
\textbf{Ирина Хмара} 

А ми жили на Нагорній! В будинку графа Львова в одній квартирі з його внучкой.
Хліб продавали в Гастрономі на площі - три кроки від нашого дому. Але навіть
така коротка відстань дозволяла з'їсти майже весь хліб по дорозі додому. І
внучка графа їла хліб зовсім по-пролетарські

\iusr{Оксана Дубинина}
\textbf{Eva Evidze} навіть королівські особи не втрималися б)) @igg{fbicon.wink}  @igg{fbicon.face.grinning.smiling.eyes} 
\end{itemize} % }

\iusr{Катя Бекренева}
Самая вкусная горбушка была у \enquote{Украинского} за 20 копеек), а ежели ещё и со сливочным маслом и солью...

\begin{itemize} % {
\iusr{Ирина Хмара}
\textbf{Катя Бекренева} по 20 копеек не помню украинского. По 14 и 16коп., а батон 22 коп.

\iusr{Катя Бекренева}
\textbf{Ирина Хмара} 

по 14 он был какой-то серый, по 16 самый распространённый, а по 20 он был чуть
больше, и тесто немного отличалось. У нас в Святошино его можно было купить
только в одном магазине \enquote{У бабы Раи} на Кременецкой (название народное) ))

Видно разные хлебзаводы поставляли.

\iusr{Ирина Хмара}
\textbf{Катя Бекренева} я думала, что цена от веса зависила.

\iusr{Алёна Выгорницкая}
\textbf{Катя Бекренева} 

а в Боярке был обалденный кирпичик по 20 коп, пока домой доходили, был уже без
корки хрустящей

На всю жизнь в память врезалось, и всегда тёплый, душистый, свежий,
хрустящий... эх, словами не передать... сникерс детства

\iusr{Moti Manzon}
\textbf{Катя Бекренева} 

Самая самая вкусная горбушка была у \enquote{Украинского} за 28 копеек на Крещатике,
возле остановки 16 и 18 тролейбусов  @igg{fbicon.face.smiling.hearts} ... а где сейчас выпекают в Киеве самый
вкусный хлеб ? Кто подскажет

\iusr{Лариса Борсук}
\textbf{Moti Manzon} на оболоне остался, но выкупили частники уже не то современные технологии....

\iusr{Natalie Di Giovanni}

Я только по 16 помню. Мы на левом берегу недалеко от хлебзавода жили.
Просыпались под этот запах сумасшедшей с утра... ммм


\iusr{Людмила Мозговая}
\textbf{Катя Бекренева} да помидорку с огорода.

\end{itemize} % }

\iusr{Ludmila Kilimnik}
Справді смачно було!! Чудові спогади...
@igg{fbicon.face.savoring.food}{repeat=3}  Дякую!!!

\iusr{Петр Кузьменко}

Хорошо ещё, если в хлебный магазин, а не в ларёк свежайшего хлеба при
подольском хлебзаводе. Тогда домой приносилась, в лучшем случае, обзрызанная,
но всё равно ароматная, середина хлеба  @igg{fbicon.wink}  или обожаемой мной
лично арнаутки...@igg{fbicon.heart.red}

\begin{itemize} % {
\iusr{Светлана Манилова}
Петр, и ты туда же...Люди, пощадите! @igg{fbicon.face.tears.of.joy} 

\iusr{Оксана Дубинина}
\textbf{Петр Кузьменко} дааа, мальчишки часто к хлебзаводам бегали.)) @igg{fbicon.heart.red}

\iusr{Анжела Жуковская}
\textbf{Петр Кузьменко} самый вкусный украинский пекли там, особенные печи там были - старые
\end{itemize} % }

\iusr{Елена Турчина}
Да нет, лучше уж корочку чесночком натереть @igg{fbicon.sun.with.face}{repeat=3} 

\begin{itemize} % {
\iusr{Оксана Дубинина}
\textbf{Елена Турчина} ага)) и чуть сольки посыпать))

\iusr{Ирина Хмара}
\textbf{Оксана Дубинина} тогда ещё и в ароматное масло подсолнечное макать))

\iusr{Оксана Дубинина}
\textbf{Ирина Хмара} аж слюнки текут)) @igg{fbicon.face.savoring.food}  @igg{fbicon.face.grinning.smiling.eyes}{repeat=2} 

\iusr{Ирина Хмара}
\textbf{Оксана Дубинина} ты первая всех раздразнила на ночь глядя  @igg{fbicon.cat.face} 
\end{itemize} % }

\iusr{Микола Веселий}

И белый и черный хлебушек любил я надгрызть, чтоб корочку. И булочки по 3 коп
уважал, хоть они были и без корочки)

\iusr{Алёна Выгорницкая}

Мы наш хлебушек раньше все хранили в деревянных хлебницах

А теперь в пакетах... оттого он и плесневет.

Вот работала в хлебном, и у нас был возврат... естественно оставляли в
пакете (возврат раз в неделю) так я из-за плесени не могла опознать... бородинский
или житний. Потом думаю проэксперементирую и сниму пакет, так вот он просто
засох.

Делайте выводы, пора опять прежние хлебницы, либо тканевые торбочки, но не
полиэтилен

\begin{itemize} % {
\iusr{Sasha Ty}
\textbf{Алёна Выгорницкая}
Справа в компонентах а не кулечках, хоча сирість сприяє.
В мене \enquote{хліб} по півроку чекає в пакеті, страшно відкрити, плесені ніякої.
Плесінь це показник кволоті

\iusr{Алёна Выгорницкая}
\textbf{Sasha Ty} я вам розповіла, як у мене на роботі було.
В пакеті цвілий, без пакету ні

\iusr{Klara Mezhebovsky}
\textbf{Sasha Ty} Плесень говорит о том, что в этом хлебе нет консервантов

\iusr{Татьяна Годынская}
\textbf{Алёна Выгорницкая} учтите ещё и \enquote{новые технологии}...
\end{itemize} % }

\iusr{Петр Кузьменко}

Оксана! Как удивительно Ваш прекрасный пост перекликается с моей давней
публикацией в нашей группе
\url{https://www.facebook.com/groups/story.kiev.ua/permalink/1630285413834920/}

\begin{itemize} % {
\iusr{Виталий Пидченко}
\textbf{Петр Кузьменко}
В смысле \enquote{перекликается}...

\iusr{Оксана Дубинина}
\textbf{Петр Кузьменко} потому что все мы родом из детства!))@igg{fbicon.heart.red} и так много общих киевских воспоминаний))
\end{itemize} % }

\iusr{Людмила Иванова}
Домой доносили обгрызаный со всех сторон батон!

\iusr{Євгеній Гончарук}
Меня просила мама грызть только в одном месте.

\begin{itemize} % {
\iusr{Оксана Дубинина}
\textbf{Євгеній Гончарук} а слушались и грызли только в одном месте?))

\iusr{Євгеній Гончарук}
\textbf{Оксана Дубинина} Да. У мамы был дар спокойно и аргументировано объяснять, почему стоит так делать или не стоит.
\end{itemize} % }

\iusr{Tekhkhi Shekk}
Я так кушала в детстве. И на фото вообще очень похоже на меня ))))

\iusr{Валентина Козачук}

Коли купляла паляницю в дитинстві, то її шляпка згризалася миттєво.. А хліб до
нас в магазин чомусь завжди привозили в обідню перерву. Всі школярі вже були
дома і під магазином шикувалася черга з дітей і пенсіонерів. А в руки давали
тільки одну буханку, а рогалики з повидлом привозили тільки один лоток.. І яке
це було щастя, коли тобі діставався цей жаданий рогалик, зверху припорошений
цукровою пудрою

\begin{itemize} % {
\iusr{Татьяна Коваленко}
\textbf{Валентина Козачук} 

да..., паляница это было нечто!!! Хрустящая корочка, а мякиш в дырочку, пышный.
Жаль, что сейчас паляница забыта, а ведь это исконно украинский хлеб.

\end{itemize} % }

\iusr{Натали Считалочка}

Выбор хлеба сейчас огромный, а старый, знакомый с детства какой-то уже не такой
на вкус...

В детстве, гуляя по улице и проголодавшись, заходишь в хлебный - дают бесплатно
горбушку, либо покупаешь булочку за 3 копейки, всё свежайшее, хрустящее снаружи
и мягкое внутри, с ароматом таким, что слюнки так и текут, еле успеваешь
добежать в кафетерий за коктейлем, чтоб не в сухомятку перекусывать)

\iusr{Августина Тытянчук}
\textbf{Натали Считалочка} О... а коктейль на Дорогожицкой, в молочном... это что-то...

\iusr{Ольга Дяченко}
Зараз мої онуки роблять те саме

\iusr{Katerina Slesarenko}

Поляныцю за 28 коп... Я буду вспоминать, сколько буду жить... Аромат,
неповторимый, корочка сверху, кот. съедали сразу, домой не доносили... Жаль,
что это только в памяти осталось... куда подевались эти сказочно вкусные
рецепты хлебобулочных изделий?


\iusr{Ростислав Чентемиров}

Помню, еще подростком был. Послал меня старший брат в магазин за хлебом: пол
\enquote{украинского} и батон! Я купил батон и наисвежайший, еще горячий, целый
\enquote{украинский}. Он меня начал ругать: вот... теперь сохнуть будет... выбросим...
Я возразил, что он еще горячий - влет уйдет, да я сам половину съем. Не
поверил. Я тут же съел половину. Когда уже доедал, он присоединился и сказал,
что был неправ.

\iusr{Almira Yusupov}

спасибо, давно не слышала слово паляниця, а хлеб был классный, до сих пор люблю
хлеб. я пытаюсь отказаться для диеты но не могу, без него какие-то все не так.
Мой муж тоже горбушку любит а дети смеются над нами и не понимают как можно
жопки « любить

\iusr{Вячеслав Настецкий}
песня про хлеб

\ifcmt
  tab_begin cols=3,no_fig,center,resizebox=0.7

     pic https://i2.paste.pics/fc9ca3137a86b938ef2f0effc36ef67b.png
		 pic https://i2.paste.pics/8f0eb3bb2a35e987fa6c1e296529f65a.png
		 pic https://i2.paste.pics/1632efd76264d0399ac130f039d56723.png

  tab_end
\fi


\ifcmt
  tab_begin cols=2,no_fig,center

     pic https://i2.paste.pics/a76f9173a5c27125b98a0f665fe93665.png
		 pic https://i2.paste.pics/e2f550cb90794552b9836052b7c8a803.png

  tab_end
\fi

\iusr{Вика Нурибекова}

Я с коркой украинского когда то проглотила молочный зуб. Сказала маме - такой
соленый хлеб, а оказалось, то кровь была. Жаль, вкус хлеба уже не тот. А какие
бублики по 5 коп. были. Таких вообще сейчас НЕТ. Бублики разрезали пополам
вдоль, мазали маслом (тоже масло нынче другое) и с молоком из треугольного
пакета - красота....

\begin{itemize} % {
\iusr{Alexandra Eglantier}
\textbf{Вика Нурибекова} 

Мама, а в свое время и бабушка в студенчестве любили взять молоко в треугольном
пакете и гулять с ним по пешеходному мосту, потом в юности я так делала, ходила
там же, пила это треугольное молоко и думала о них, связь поколений.

\end{itemize} % }

\iusr{Ольга Лубягина}
Вкуснятина

\iusr{Игорь Печерный}

А еще была \enquote{арнаутка}. Ну, а про \enquote{жулики} тут уже писали не раз. Мне еще
вспоминается булка, кажется \enquote{Праздничная}. Такая большая, сдобная с изюмом...

\begin{itemize} % {
\iusr{Людмила Гаценко}
\textbf{Igor Pecherny}

Арнаут, козацький, сімейний, міська булка - найкорисніші сорти, як вважають
дієтологи й гастроентерологи. Шкода, що цих сортів зараз нема.

А святковий з родзинками є. І справді смачний.


\iusr{Natalie Di Giovanni}
\textbf{Igor Pecherny} арнаутку не могу вспомнить

\begin{itemize} % {
\iusr{Игорь Печерный}
\textbf{Natalie Di Giovanni} 

Это круглая булка чуть меньше, чем паляниця, без козырька и не белая, а серая с
характерным вкусом, который ни с чем не спутаешь. Особенно вкусная, когда
свежая и с маслом ))


\iusr{Людмила Гаценко}
\textbf{Natalie Di Giovanni}

Житньо-пшеничний, але за якимось цікавим рецептом, його називали сірим. Легкий,
смачний, для здорового харчування. Особливо необхідний тим, у кого проблемний
жкт, у кого гастрити, гастродуоденіти...


\iusr{Татьяна Бригинец}
\textbf{Людмила Гаценко} це такий за формою, як маленька цеглинка? Білковий?

\iusr{Людмила Гаценко}
\textbf{Tetiana Brihinets}
Ні, це не \enquote{кірпічік}, а великий круглий житньо-пшеничний дріжджовий.

\iusr{Людмила Гаценко}
\textbf{Natalie Di Giovanni}
А я не можу забути.

\iusr{Галина Демченко}
\textbf{Игорь Печерный} А правда, куда девалась арнаутка?

\iusr{Elena Beseda}
\textbf{Галина Демченко} что, в Киеве уже нету? А у нас в Лос Анджелесе можно найти.

\iusr{Анжела Жуковская}
\textbf{Игорь Печерный} а паляниця, как фуражка?)

\iusr{Людмила Гаценко}
\textbf{Anzhela Zhukovskaya}
Точно @igg{fbicon.laugh.rolling.floor} 

\iusr{Tamara Zueva}
\textbf{Igor Pecherny} А мне кажется \enquote{днепропетровский}
\end{itemize} % }

\end{itemize} % }

\iusr{Klara Mezhebovsky}
A булочки по 3 копейки...

\iusr{Natalie Di Giovanni}
\textbf{Klara Mezhebovsky} и по 6, длинные такие

\iusr{Александра Немченко}
Были очень очень вкусные длинные булочки по 6 копеек и \enquote{городские} булочки.

\begin{itemize} % {
\iusr{Мария Маслова}
\textbf{Александра Немченко} 

Только хотела написать, но Вы уже вспомнили. А еще были с повидлом, посыпанные
пудрой, по 7 копеек, и просто круглые сдобные - желтые и очень пышные, похожие
на пампушки, только больше, с темно-коричневой глянцевой поверхностью.

\iusr{Надежда Пилипчук}
\textbf{Мария Маслова} Это были булочки по 9 коп., а ещё были сайхи - 6 коп.

\iusr{Мария Маслова}
Я их помню по 7 почему-то. Но не суть важно. Вкусные были.

\iusr{Людмила Мозговая}
\textbf{Александра Немченко} разворачивать для еды их любили.
\end{itemize} % }

\iusr{Inna Sarkisova}

Теплый хрустящий батон и украинский свежий хлеб. Ничего вкуснее не было.
Незабываемый вид, запах и вкус. Брат возил в командировки за границу в арабские
страны в качестве самого дорогого подарка вместе с банкой селёдки. Очень
просили.

\iusr{Наталия Маковская}
Да, паляниця была знатной! И очень вкусной. Жаль, что сейчас ее уже нет!

\iusr{Мария Маслова}

Замечательно вспомнили! А мне хочется добавить еще и про запах свежей выпечки.
В старших классах занималась на курсах английского на Прорезной в клубе
работников искусств. И вот выйду оттуда вечером в девять часов, голова
побаливает. Иду на остановку девятого троллейбуса неподалеку от двухэтажного
книжного и пельменной. Там и центральный хлебный рядом. Захожу – огромная
очередь: свежий хлеб привезли. Выхожу из магазина с горячим украинским и
батоном или арнауткой, или паляныцей и главное(!) без всякой головной боли –
надышалась ароматом свежего хлеба, и все как рукой сняло!

\iusr{Оксана Дубинина}
\textbf{Мария Маслова} наверное, потому что позитив @igg{fbicon.face.smiling.hearts} 

\iusr{Irina Somova}

Жила на Бассейной. В нашем доме в подвале была пекарня. Сторож дядя Сеня угощал
всех ребятам нашего двора булочками с повидлом и \enquote{городскими}. Можете не
верить, но запах этого свежеиспеченного хлеба помню и сейчас.

\iusr{Оксана Дубинина}
\textbf{Irina Somova} охотно верю!) @igg{fbicon.face.happy.two.hands} @igg{fbicon.heart.red}

\iusr{Татьяна Иванова}
Какой же вкусный был хлеб!

\iusr{Татьяна Калиновская}

Да, за \enquote{арнауткой} сердце @igg{fbicon.heart.red} особенно печалится... Чудо из чудес! И булочная на
площади Калинина... Я с Печерска часто туда добиралась!

\iusr{Dina Vainshtein}
Хочется кусочек и сейчас .........

\iusr{Tina Marsagishvili}

Лежу, читаю, 3,55 утра, а тут такие воспоминание, слюнки текуть, но лень,
кухня на первом этаже, я нахожус на втором! Арнаутка мне тоже нравилься очень.

\ifcmt
  ig https://i2.paste.pics/51ebef1ec67d2df53966f762f86b3ad1.png
  @width 0.2
\fi

\begin{itemize} % {
\iusr{Оксана Дубинина}
\textbf{Tina Marsagishvili} приятные ж эмоции @igg{fbicon.face.grinning.smiling.eyes}{repeat=2}  Время и позавтракать @igg{fbicon.wink} 

\iusr{Tina Marsagishvili}
\textbf{Оксана Дубинина} не просто приятные, а прекрасные эмоции! Вы так мастерски все рассписало что большое спасибо!
\end{itemize} % }

\iusr{Dina Vainshtein}

Моя мама много лет проработала в хлебном магазине на ул. Ленина. Выпечка была
потрясающая - батоны, чёрный хлеб особенно, жулики - чёрные булочки с изюмом,
булочки с повидлом...... можно перечислять бесконечно, приятное воспоминание
детства. Спасибо что напомнили.

\begin{itemize} % {
\iusr{Vitaliy Stepanyants}
\textbf{Dina Vainshtein} 

я подрабатывал грузчиком и резчиком хлеба в 101-м хлебном магазине на ул.
Ленина, когда учился в институте. Это было в 80х.

\iusr{Dina Vainshtein}
\textbf{Vitaliy Stepanyants} это было ещё в 70х

\iusr{Наталия Борщева}
Мы туда за хлебом ходили ..
\end{itemize} % }

\iusr{Lina Garber}

Я до сьих пор помню вкус свежего украинского хлеба (горбушки с хрустящей
корочкой) с маслом и сыром, хоть уже прошло 50 лет...

\iusr{Наталья Цымбалюк}
А рогалики по 6 коп. Мим

\iusr{Marina Lavrow}

А у нас хлебный был на первом этаже нашего дома. Меня родители отправляли за
хлебом лет с пяти. Я, по-моему, ни разу не донесла ни украинский, ни паляныцю,
домой с корочкой. Такая ностальгия!


\iusr{Пауль Володимирівський}
И хозяйственное мыло.

\end{itemize} % }
