% vim: keymap=russian-jcukenwin
%%beginhead 
 
%%file slova.pravda
%%parent slova
 
%%url 
 
%%author 
%%author_id 
%%author_url 
 
%%tags 
%%title 
 
%%endhead 
\chapter{Правда}
\label{sec:slova.pravda}

Який тільки народ не адаптував сюжет стародавньої казки про відомих українцям
братів \emph{Правду} і Кривду. Проте найкраще, здається, це вдалося зробити в
англосаксонському світі, де цю моралізаторську казку назвали «Засліплення
правди фальшивістю». І тут йдеться не тільки про маніпуляції та перебріхування.
Мова йде також про тих, хто облудою і фальшем забирає в інших здатність бачити,
чути і розуміти. Хоча у відомій казці є безліч незрозумілих для сучасної людини
поведінкових практик, а сюжет містить багато спрощень, основна мораль зрозуміла
– \emph{неправдою} щасливим не станеш. Але цього разу нам важливо зосередитися на
тому, що одному з братів за допомогою цинічних маніпуляцій вдалося виманити у
брата його коржа. Ще цікавішим є те, чому брат \emph{Правда} погодився на те, щоб його
скалічили до напівживого стану,
\textbf{Правда і Кривда}, Василь Расевич, zaxid.net, 04.06.2021

Может, хватить улучшать - и будем просто жить?  Я вот удивляюсь - почему, когда
власть берется что-то улучшить, граждан охватывает смятение? Почему все
улучшения, по итогу заканчиваются Ухудшением? Вспомните недавние «реформы». От
медицинской до судебной!  Может, прав индейский вождь: «Когда белый человек
нашёл нашу землю, на ней правили индейцы. Не было налогов, не было долгов,
буйволы и бобры были в изобилии, вода была чиста. Женщины делали домашнюю
работу, лекарь не брал денег. Мужчины тратили весь день на охоту, а всю ночь на
секс». Потом вождь улыбнулся и добавил: «Только белый человек может быть
настолько глупым, чтобы думать, что такую систему можно улучшить».  А и
\emph{правда}, может хватит «улучшать жизнь и преодолевать эпоху бедности»?,
\citTitle{Почему все наши реформы, от медицинской до судебной, заканчиваются ухудшением?}, Константин Стогний, strana.ua, 10.06.2021

%%%cit
%%%cit_head
Россия - Нация-Девчонка
%%%cit_pic
\ifcmt
	pic https://avatars.mds.yandex.net/get-zen_doc/3989805/pub_60c48c27b582164697cb2369_60c49e6e72119574b73e6391/scale_1200
  caption Девчонки с флагами. Россия вперёд!
\fi
%%%cit_text
Как сказал кто-то из западных правителей (Бисмарк, что ли, не важно):
\enquote{Россия -- это нация ртуть. Её рвёшь на куски, а она, временно
рассыпавшись, снова неумолимо собирается в одно целое, и никто, и ничто этому
помешать не сможет никогда}. Да, это \emph{Правда}. У России множество
феноменов. Главный -- это язык.  Все россияне (все и везде) говорят на
одинаковом языке, без всяких каких-то серьёзных отступлений, как в других
европейских и иных странах -- все россияне говорят на одном чистом,
литературном русском языке. Это феномен
%%%cit_comment
%%%cit_title
\citTitle{Слава России! Мы на вершине могущества и спускаться более не намерены}, 
Алексей Наст, zen.yandex.ru, 12.06.2021
%%%endcit

%%%cit
%%%cit_head
%%%cit_pic
%%%cit_text
Но, как бы россияне и русские не пытались приобрести европейские черты лица, и
стать чуть-чуть исторически старше, \emph{правда} все равно лезет наружу, разрывая
наспех сметанные исторические симулякры.  Кто же, они, русские?  Знаете, что в
СССР была милая шовинистическая поговорка про русских: \enquote{Ты почему русский?
Потому, что у тебя глаз узкий!}. Но именно она открывала всю \emph{правду} \enquote{русских}
%%%cit_comment
%%%cit_title
\citTitle{Страна без исторических корней всегда будет оккупантом}, 
Олена Степова, news.obozrevatel.com, 17.06.2021
%%%endcit

