% vim: keymap=russian-jcukenwin
%%beginhead 
 
%%file 22_05_2021.fb.fb_group.story_kiev_ua.1.belye_rozy_okonchanie
%%parent 22_05_2021
 
%%url https://www.facebook.com/groups/story.kiev.ua/permalink/1667871636742964
 
%%author Киевские Истории
%%author_id fb_group.story_kiev_ua
%%author_url 
 
%%tags 
%%title Итак. Окончание! «Белые розы и одно венское кресло»
 
%%endhead 
 
\subsection{Итак.Окончание! «Белые розы и одно венское кресло»}
\label{sec:22_05_2021.fb.fb_group.story_kiev_ua.1.belye_rozy_okonchanie}
\Purl{https://www.facebook.com/groups/story.kiev.ua/permalink/1667871636742964}
\ifcmt
 author_begin
   author_id fb_group.story_kiev_ua
 author_end
\fi

\textbf{Svetlana Kievljanka}


\ifcmt
  pic https://scontent-lga3-2.xx.fbcdn.net/v/t1.6435-9/p843x403/189600895_2929713687302274_4778804891781731851_n.jpg?_nc_cat=105&ccb=1-3&_nc_sid=825194&_nc_ohc=7oCDg4AM06EAX_dUgj0&_nc_ht=scontent-lga3-2.xx&tp=6&oh=81973bbd71fd0103447f2d165a12a8b8&oe=60D07336
\fi

Итак. Окончание! 

«Белые розы и одно венское кресло»

Впервые письмо от Женьки пришло в январе, как раз перед православным
Рождеством. Он писал о своих успехах, о том, как ему повезло, о выставках и новых
знакомствах. Все в ярких, восторженных красках. Неля перечитывала его каждый
вечер. Затем мысленно разговаривала с Женькой, радовалась его успехам. Она
старалась представить себе его нынешнюю жизнь и... не могла. Это все было в
каком-то другом измерении, не в этой жизни... Потом садилась и писала ответное
письмо. Страницы на три. Писала и выбрасывала. Снова писала и снова
выбрасывала. Мысли путались, скакали. Хотелось рассказать о каждой минуте своей
жизни, но она прям кожей чувствовала - ему это не надо. В этой, новой жизни для
неё у Женьки места не было... И снова бросала скомканный лист в мусорник...

Но однажды она увидела его. И это было не во сне. Она увидела его живьём.
Вспомнила все его письма. Такого шока она не испытывала давно...

\zzrule

На конференцию в США отбирали лучших врачей по всей стране. О том, что она летит
в составе делегации, Неля узнала несколько дней назад. Первым делом она бросилась
писать письмо. Успеет ли оно дойти - вот это вопрос. Уже стала появляться
мобильная связь, но для простого стоматолога она все ещё оставалась недоступной
роскошью... Она надеялась, она очень надеялась, что успеет.

\zzrule

Америка... Могла ли Неля представить себе ещё несколько лет назад, что увидит
то, что увидела сегодня?! Нет, конечно... Уже посадка в огромный Боинг привела
ее в неописуемый восторг. Сам полёт она пыталась не вспоминать. Вода, вода,
бесконечная вода под пузом самолета. Величественно и безумно страшно... А затем
город-мечта! И хоть поселили их в каком-то слабеньком отельчике ,видимо, в
целях экономии средств, но уже дорога к нему уже казалась сказкой. Вроде, те же
люди на улицах, автомобили, - все, как везде, но... другие люди, об автомобилях
вообще лучше молчать. Как в другой век попала. Завтра ознакомление с протоколом
визита, вечером приём, послезавтра сама конференция. И целая неделя в
городе-празднике! И... Женька! Встреча через десятилетие... Как это будет?! Как
они посмотрят с глаза друг другу?! Что будут говорить при этом?! О чем
говорить?.... Не стали ли они совершенно чужими людьми за эти годы?

Ведь столько пережито, столько дров наломано...

И все же она хотела сделать ему сюрприз. Она же женщина! А женщина с одного
взгляда поймёт, какое впечатление она произвела на мужчину. Она хотела увидеть
именно этот взгляд - случайный! И она сделает это! Как? Подумает. У неё ночь
впереди! А сюрприз удастся точно. Потому что прилетела она на две недели
раньше. Он не знает об этом! Тем трогательнее будет их встреча! 

Боже! Как?! Это что?! Это возможно??? Она будет в Нью-Йорке??? Это фантастика!
Наконец, он сможет представить Нельку Люси! Он покажет свои картины! Она
увидит, кем он стал! Она поймёт, что с ним ей будет здесь хорошо! Он уговорит ее
остаться! Фууууух... Не верю!))))) И всего через каких-то пару недель ! 

Ещё две капельки духов за ушки и на выход....

Неля шла дорожкой, ведущей к озеру Централ-парка. Именно там он любил работать по
субботам. Он писал ей об этом в каждом письме. Мол, это «место силы» , сакральное
место. То , с которого он гордо шагнул в новую, другую, счастливую жизнь! После
двух дней марафона конференции эта прогулка казалась ей сказочным
мероприятием. Народу было очень много. Кто-то просто валялся на траве, устеленной
пледом, кто-то играл в бадминтон, кто-то из стариков дремал на лавочке, уронив
голову на палочку. Вон вдалеке показалось озеро. Неля остановилась и
оглянулась. Слава Богу, что ей предложила помощь переводчица
делегации, согласившись провести этот день с Нелей. Сама бы она точно это место
не нашла. Дина, так звали девушку, показала рукой вдаль, где стояла немаленькая
кучка людей.

- Обычно они тусуются там. Подойдем поближе, посмотрим, тут ли твой ненаглядный.

Дина была из «наших». Приехала в США с родителями совсем ребёнком. Хохотунья
немыслимая! Сколько же в ней оптимизма! 

- Пошли, пока все там.

Они двинулись в сторону поляны.

Их было много - художников... Практически все писали портреты прогуливающихся по
парку! Женьку она увидела сразу. Вон он сидит. Третий слева. Господи, во что он
одет?! Что за дырявые джинсы?! Что за свитер на три размера больше? И что это
перед ним?! Шляпа?! Он работает на шляпу?! А где же выставки? Где успех и
почитание? Значит, это все в письмах - ложь? От начала до конца?! Нет! Ей все
равно! Пусть он хоть нищим будет, пусть. Но зачем он лгал? И кому? Ей? Хотел
выглядеть солиднее?.... Она ничего не понимала. Но страшно было не это. Страшно
было, что встреться они по-другому, он продолжал бы лгать ей, рассказывая о своём
успехе. А она бы уже знала, что это неправда... И как быть? Она ждала этой встречи
десять лет, он, наверное, тоже. Но подойди она сейчас к нему, и все, стеклянная
стена рухнет, похоронив под собой их ту первую, красивую и настоящую любовь...

- Я все увидела, Дин. Спасибо тебе. Могу я попросить тебя об одной вещи? Сделай
это, пожалуйста.  Неля потянулась к сумке, извлекла 100\$ ...

- Передай это вон тому парню в зелёном свитере, брось ему в шапку, если не
сложно. Я подожду тебя вон там, за той беседкой. Спасибо...  Дина ещё что-то
щебетала, называла Нелю транжирой и дурочкой, объясняла, что даже портрет не
больше двадцатки стоит, а сотня - это безумие! Но Неля уже не слышала
слова, брошенные ей в спину. Эта сотня - ерунда. Самое малое, что она могла для
него сделать. Она помогла ему избежать позора при встрече и выручила
денежкой. Сколько же ему надо тех дурацких портретов написать, чтобы прожить в
этом безумном городе... Да уж. Выставки, галереи... Женька-Женька. Как же так.

Теперь у неё была цель...

- А он изменился... Он повзрослел, был долговязым юношей с длинными руками, а стал
мужчиной. Прическа ему не идёт, зачем эти длинные волосы? И хвостик этот
дурацкий... Показалось или действительно появились морщинки? А, ну да... Весь день
на улице. Ветер, Солнце... Господи! Это ненормально! Это надо прекращать и
немедленно! Я все беру в свои руки. Я не дам пропасть его таланту. Да и ему
самому. Все же он был и остаётся дорогим человеком для меня...,- сидя в
такси, везущем ее в отель, Неля размышляла о планах на будущее. О своих
планах, главным пунктом в которых был он, ее Женька! 

Теперь у неё была цель - она вытянет его из этой убогой жизни. Она уговорит его
вернуться домой. В ее стране тоже можно работать и зарабатывать! Пусть не
больше, чем уличные художники в Нью-Йорке, но уж точно не меньше! И первым делом
она выбросит эти ужасные рваные джинсы! Она неплохо обеспечена. Жить есть, где, на
жизнь точно хватит - протянут! Главное - рядом! Главное - вместе. И
главное, чтобы он не узнал, что она его видела...

Мальчишка протянул Юджину сотенную бумажку. Только что улыбавшийся во все 32
зуба мужчина посерьезнел в минуту. Вскочил со стульчика, схватил мальчишку за
шиворот и спросил, что это? Откуда?! 

Пацанчик испугался и махнул рукой в сторону удаляющейся женщины, за которой
вприпрыжку бежала другая. Что-то знакомое показалось Юджину в той походке, что-то
неуловимо родное, но он не мог понять, что именно.

Вскоре обе женщины скрылись за поворотом. Бросив кисть на землю, он бросился
следом. Но женщин и след простыл... Загадка осталась загадкой.

Дома он сел на балконе, расстелил банкноту на столе и долго вглядывался в лицо
американского президента... Какое-то странное чувство не покидало его. И эта
женская фигура вдалеке... И эта походка. Ему казалось, что он даже запах
почувствовал! Ее запах! Но это невозможно. Она прилетит через две недели! Он уже
готовит ей программу в его любимом городе. Он познакомит ее со своими друзьями, с
Люси! Он уговорит ее переехать к нему! Сейчас уже нет тех проблем, что были в
пору их юности! Они будут вместе! Он богат и знаменит, он сможет сделать ее
жизнь счастливой! Главное - рядом! Главное - вместе! 

Неторопливо он прошёл в квартиру, сбросил с себя одежду, которую всегда надевал
на «субботники» в парке. Он любил эти обноски! Они напоминали ему о той
незабываемой встрече, которая резко поменяла его жизнь.

Сегодня он мог купить себе лучшую одежду в лучших нью-йоркских магазинах, но
рядом с брендовой одеждой в шкафу всегда лежали эти рваные (очень модные в своё
время, кстати) джинсы и этот зелёный свитер, который связала мама...

Вернувшись домой, Неля принялась осуществлять свой план. Она села писать
письмо, в котором извинялась, что не приехала, мол, так получилось... И
звала, звала, звала Женьку к себе. Хоть на недельку, лучше на месяц. Уж за месяц она
точно его уговорит. Сто процентов!!! Он ответил вскоре. Писал, что никак не может
вырваться - едет в Канаду с выставкой. Вот потом планируется Европа, а там видно
будет. А вдруг удастся и в Киеве организовать? 

Неля читала и стирала слёзы с лица... Он не знает, что Я знаю все... И не надо!
Пусть представляет себя богатым! Ну пусть! Может, ему так легче!!! Спокойнее...

Год промчался, как один день. Они переписывались, а однажды даже поговорили по
телефону. Но было очень шумно, видимо, он звонил с какой-то вечеринки. Ей
казалось, что ещё немного и она его уговорит...

Сегодня утром она проснулась в предвкушении чего-то важного. Чего-то, что должно
перевернуть ее жизнь... Но случалось все что-то незначительное. Ничего! Счастье
не терпит суеты! Подождём.

Коньяк сделал своё благое дело и Неля уснула. Ей снились дурацкие белые
розы, которые она возненавидела ещё давно, в юности. И это было единственное
плохое, что случилось с ней в этот день...

- Здравствуй, новый день! Здравствуй, Солнышко! Ну, что? Ждём чудес? - Неля
готовила кофе и любовалась утром. Кресло ждало ее на балконе. Не спеша
никуда, смакуя ароматный напиток, она готовилась к новому дню. И ждала чуда.


Долго ждать не пришлось... В дверь ее квартиры позвонили. На пороге стоял молодой
человек с огромным букетом белых роз и пакетом. Сказав несколько дежурных
фраз, он мило откланялся и исчез в парадном. Неля осталась одна.

Розы... Господи. Неужели ещё кто-то не знает, как она не любит этот цвет именно у
этих цветов?! Чья это дурацкая шутка? А пакет? В нем что? 

Оставив цветы на столе в комнате, она вернулась на балкон и стала разворачивать
блестящую упаковку. Красивая бархатная коробка. Двойная открытка. Отложила. Потом
прочитаю. Аккуратно открыв замочек, внутри коробки Неля обнаружила небольшую
рамку для фото. В ней лежала стодолларовая купюра... Открытка! Где она? Что
там??? «Это была ты!» 

Она выронила бумажку... Как?! Как он узнал? 

Встала, прошла в комнату. Надо поставить розы в воду. Не пропадать же красоте. Ну и
что, что она не в восторге от них. Пусть стоят.... В кухне... Так! Стоп! А это
что?! Ещё конверт? 

«Имею честь пригласить госпожу Нелли на выставку, которая будет представлена по
адресу..... С уважением Юджин Клин». Так?! А это ещё кто? Нет, на днях, сидя в
салоне на маникюре она слышала разговоры о каком-то сильно модном художнике из
США и его выставке. Мол, попасть нереально, с билетами проблема. А тут личное
приглашение?! С чего бы это? Кто он?! И почему приглашает именно ее???
О, Господи.... Наверное, это какой-то Женькин друг! Наверное, привёз весточку от
него!!! Ой, как же хорошо! И розы, кстати, весьма неплохи!

На работу она летела в приподнятом настроении. Оделась соответственно
моменту. Ведь он вернётся и расскажет о том, как она сегодня выглядит. Осталось
забежать в обеденный перерыв в салон, поправить причёску и она готова! Так вот
оно - чудо? Дождалась?.... Уфффф.....

Народу перед входом было много. Очень много. Медийных лиц немало. Их сегодня
селебрити сказывают. Неля не стала толкаться, отстояла в очереди и вошла в
галерею.

Тихая музыка, приятные ароматы, официанты с напитками... Интересно, где же виновник
торжества? Юджин этот... Кто-то коснулся ее талии. Нежно, мягко, но она все равно
дёрнулась. Не любила, когда подходят незаметно, ещё и сзади... Резко
обернувшись, она увидела его, своего Женьку...

И это я хотела устроить ему сюрприз?! Я??? После такого опускаю руки..

- Женя??? Ты??? Как? Откуда? Постой.Вы вместе прилетели? Ты и Юджин? Я ничего не понимаю.

- А тебе и не надо ничего понимать. Юджин - это Женя. То есть я. В
Америке... пойдём. Я хочу показать тебе кое-что. Это не для всех. Это только для
тебя. Лично.

Они прошли сквозь толпу, рассматривающую висящие на стенах работы, прошли через
небольшой коридор и вошли в комнату. Как по мановению волшебной палочки зажегся
свет. Неля ахнула... Вся комната была выстелена белыми розами, а со стен на неё
пялились те же белые розы во всех мыслимых и немыслимых ракурсах! И тут
произошло немыслимое - из динамиков откуда-то из-под потолка грянуло «Беелые
розы, беелые розы...» Неля остолбенела. Казалось, ещё минута и она грохнется в
обморок. Она замахала руками, зажала уши... Музыка стихла...

- Ну что? Удивил? Сюрприз удался?

- Ещё как удался,- прошептала ошарашенная Неля...

- Я же помню, ты обожала эту песню! И розы эти! Я писал эти картины специально
для тебя! Ты у меня на каждом полотне! У тебя будет время проверить
это, поискать! Ты рада? 

- Я?! Обожала? Песню? Розы? Да я...... я.... Ну да, ты прав. Обожала! И да! Я
очень рада. Честно.

Только сейчас она поняла, как, насколько ей не хватало его, роз этих
дурацких. Только сейчас она начала понимать, что тогда сама придумала себе
историю и поверила в неё. Только сейчас поняла - вот он! Женька ее! Рядом! И
ничего, что он Юджин. 

- «Что значит имя? Роза пахнет розой, хоть розой назови ее, хоть нет!» -
процитировала Шекспира Неля ...- вот именно! Ничего не значит! 

Добраться в Борисполь не составило труда. За месяц, который они провели
вместе, было принято решение лететь вместе. Пока туда, в его гнездо. Потом они
вернутся в ее дом. А потом.... Потом видно будет.

Казалось, что за этот месяц на всем Земном шаре не было счастливее этих
двоих. Они то сидели на балконе - она в том самом кресле, которое успела привести
в порядок в аккурат к его приезду, а он то у ее ног, свернувшись калачиком, то
рядом, на маленькой банкеточке. Но обязательно лицо в лицо! Чтобы видеть!!! 

А иногда они бродили склонами Днепра, где любили сидеть в юности. Подросли
деревья, да и Киев стал другим... А она все вспоминала его последние слова «Это
не конец! Это начало!!!» Да, они на пороге чего-то нового и неизведанного.

Они могли говорить часами, пытаясь рассказать все, что не знали друг о
друге, осторожно и деликатно обходя тему, связанную с личной жизнью. Всякое
бывало. Он откровенно признался в том, что был «звездный» период в его жизни, а
она.... Она нашла в себе силы рассказать о том, как приезжала в Нью-Йорк и видела
его... Иногда они просто сидели молча, держась за руки. Они дарили себя друг
другу. Заново.

Ее очень тепло и трогательно отпустил коллектив клиники, она пообещала
вернуться. Когда? А никто не знает. Жизнь покажет...

В такси они ехали каждый со своими переживаниями - она оставляла за спиной всю
свою прежнюю жизнь, он - с переживаниями о том, как она впишется в новую
... Ведь они едут вместе! Он обещал! Он должен сделать ее счастливой! ....

Пассажиры их рейса с умилением смотрели на двух отнюдь не молоденьких
влюблённых, которые ни на миг не разжимали рук. Потому что жизнь даётся многим. И
только любовь даётся избранным... Ее заслужить надо. Ведь оно как - главное, что
рядом! Главное, что  вместе! А остальное.... мишура, суета... Ведь так? Ведь
правда? 

( «Беелые розы, беелые розы... Беззащитны шипы...»)

Старое венское кресло, бережно укрытое простынкой, стояло в углу комнаты. Оно
только что простилось со своей хозяйкой, улетевшей в новую жизнь. Оно будет
ждать!!! И оно точно знало - дождётся... Рано или поздно она вернётся и они
снова будут вместе. А пока пусть она набирается впечатлений. Ведь оно вон
сколько повидало на своём веку, а она что видела? 

- Пусть, пусть развеется девочка. Я подожду! 

Ссылка на начало и продолжение в первых комментариях.

Спасибо всем за отклики! 

Спасибо Anna Ginzburg за иллюстрацию!

\ii{22_05_2021.fb.fb_group.story_kiev_ua.1.belye_rozy_okonchanie.cmt}


