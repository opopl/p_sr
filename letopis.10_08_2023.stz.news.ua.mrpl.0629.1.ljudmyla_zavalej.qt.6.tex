% vim: keymap=russian-jcukenwin
%%beginhead 
 
%%file 10_08_2023.stz.news.ua.mrpl.0629.1.ljudmyla_zavalej.qt.6
%%parent 10_08_2023.stz.news.ua.mrpl.0629.1.ljudmyla_zavalej
 
%%url 
 
%%author_id 
%%date 
 
%%tags 
%%title 
 
%%endhead 

\begin{quote}
\em\enquote{Насправді, я знала, що буде війна, і що вона буде не такою, як у 2014-му, теж
здогадувалась. Тому коли все почалося, ми з чоловіком одразу відправили дітей
разом із мамою подалі від Маріуполя. Але для себе було прийнято рішення
залишатись стільки скільки зможемо бути корисними.

Чоловік разом з дружньою організацією Маріупольська асоціація жінок \enquote{Берегиня}
взяв участь у евакуації  жінок, дітей та літніх людей, які були в зоні ризику
через різні підстави. Андрій – лікар, тож він мав супроводжувати евакуаційний
автобус до Польщі, а потім повернутися по наступний евакуаційний рейс. Я була
впевнена, що залишиться хоча б якісь коридор для виїзду із Маріуполя,
сподівалася, що ми, волонтери, зможемо бути корисними для наших захисників, і
для цивільних теж. Але сталося страшне.

Мені пощастило купити в одній із аптек на початку березня медичні аптечки,
досить непогані, в яких було все необхідне для надання екстреної медичної
допомоги. Я забрала всі ці аптечки і розвезла по найбільших цивільних сховищах
в Приморському районі. 

\enquote{Черема} спочатку здавались безпечними, але потім там почалось страшне. Там
взагалі була срака. Приїжджаємо, дивлюсь, а там один з будинків на Латишева,
вже не пам'ятаю номеру, прилетіло всередину. Дивлюсь, стоять пожежні,
дивляться, а воно горить. Ніхто не гасить, ніхто не метушиться. І я така, а що
ви стоїте? Вони кажуть, а води ж немає. І все, і стоять. Я кажу, може там... Ну
ти розумієш. А вони мені,  ні, ні, ні, евакуювати нікого не треба. Всі
евакуювалися, хто міг.

Біля мого будинку в квітні впав снаряд, так бетонну відмостку метрову просто
вирвало зі стіни.

Те, що Лівобережжя пережило у лютому-березні, в порту та на Кіровці розпочалось
у квітні. Розх...ли мікрорайон повністю. Біля мого будинку ще пара будиночків
вціліла, а навколо – пустеля.

Знаєш, я ж 24 лютого бігала по поселку де я мешкала, вмовляла людей
евакуюватися, а вони дивилися на мене як на божевільну. А тепер вже самі були,
як божевільні.

Я бачила стареньку жінку, яка сиділа біля свого зруйнованого вщент дому, і
причитала: \enquote{Чому мене не завалило... Хочу здохнути тут, біля свого будинку...}.
Багато такого я бачила.

Я впевнена, в Маріуполі не менше 100 тисяч загинуло. Я на власні очі бачила
тисячі тіл. Так я ж поїхала у березні, а після ще більше місяця бомбили}.
\end{quote}

