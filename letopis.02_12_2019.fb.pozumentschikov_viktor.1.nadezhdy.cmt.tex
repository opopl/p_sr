% vim: keymap=russian-jcukenwin
%%beginhead 
 
%%file 02_12_2019.fb.pozumentschikov_viktor.1.nadezhdy.cmt
%%parent 02_12_2019.fb.pozumentschikov_viktor.1.nadezhdy
 
%%url 
 
%%author_id 
%%date 
 
%%tags 
%%title 
 
%%endhead 
\subsubsection{Коментарі}

\begin{itemize} % {
\iusr{Аркадий Анин}

Эти обещания - обещания ребенка ясельного возраста завтра перестать писать мимо
горшка. Обещание ни о чем! Обещание, направленное на недоумы группы в поддержку
президента! Верните пенсионерам украденные вами у них деньги! А это, примерно,
60 миллиардов гривень в год. А потом расскажете о подъёме ВВП! Хоть на 1\%!

\iusr{Пианит Пианит}

Віктор, ти як завжди правий. Згоден на всі 100\%

\iusr{Пианит Пианит}
 @igg{fbicon.hands.shake} 

\iusr{Leonid Tieriekhov}

Та да, сложно было ожидать "прорыва" от команды дилетантов. Лишь бы не хуже
стало чем при "барыгах", хотя похоже что будет хуже и во внутренней и особенно
во внешней политике, которая однозначно "усугубит" положение и внутри страны.
Последнее видео обсуждения цен на тарифы с участием Зеленского- вообще удивило.
Зачем было людям обещать сделать то, о чём они только "приблизительно" понимают
и не знают всех механизмов и условий этой системы? Поппулизм и иначе это не
назовёшь. В общем - как обычно - надеяться на лучшее нет оснований и приходится
готовиться к худшему. Как то так.

\iusr{Максим Мельник}
Когда недопрезидент просит снизить цену на коммуналку, хотя не в курсе своих полномочий - это пиздец какой-то.

\iusr{Роман Сердюк}
ВиктОр, а уже можно говорить, что "я ж вам всем говорил" или ещё рано?

\begin{itemize} % {
\iusr{Виктор Позументщиков}
\textbf{Роман Сердюк} пока рано))), но уже будет время, как я понимаю

\iusr{Максим Мельник}
Роман, тебе - нет, ты же от зелёных шёл  @igg{fbicon.face.grinning.big.eyes} 

\iusr{Алена Тищенко}
\textbf{Роман Сердюк} УЖЕ поздно, позавчера надо было

\iusr{Tanja Gasik}
\textbf{Виктор Позументщиков}, (тримтячим голоском, пошепки) приєднуюсь до питання Романа.
PS. Сьогодні вперше за останні півроку в мене зявилось відчуття, що може й цього разу вдасться від найгіршого "отпетлять", як казав мій брат.

\iusr{Виктор Позументщиков}
\textbf{Tanja Gasik}
\end{itemize} % }

\iusr{Слава Демченко}

и раньше тоже были разные премьеры. Азаров для для "лицо в тв", а Арбузов в
"тени с кабмином". Но сегодня составчик кабмина, слов не подберу.... и никто ж
за них не голосовал, а всё-таки назначили, да ещё и порошковых оставили на 80\%
оставили...

\begin{itemize} % {
\iusr{Leonid Tieriekhov}
\textbf{Слава Демченко}- 

где ты увидел 80\% порошковых в кадмине? Кроме Маркаровой , которая
действительно спец и будет работать при любой власти и Авакова - там больше
никого не осталось. Если бы действительно были 80\% спецов- то можно было бы как
то быть спокойнее за ситуацию в стране, а так перспектив мало для того , чтобы
не стало хуже, чем при "барыгах", а про лучшее я даже и не думаю.


\iusr{Слава Демченко}
\textbf{Leonid Tieriekhov} 

Маркарова спец? я не слышал такого в её адрес. Её привела Яресько, и ты не
понимаешь для чего? Баблуху контролировать. Почитай, как у её мужа банк не
худеет, А на фоне детей с самокатами, милыми папиросками, заказами Алин,
конечно она выглядит покрасивше... а те 80\%, это все те, кто были замами
министров, советниками и прочей шелухой для подноски в обед чая Дядям и Тётям,
И не только в кабмине, везде пристроились, кроме музыкальных школ, там не тот
"бабловый аккорд" получается, тем более ещё и нужно уметь работать, а жизнь
проходит ..... Политики это ф.... ну их на.... Надоело всё это набаралово..

\iusr{Leonid Tieriekhov}
\textbf{Слава Демченко}. 

Конечно "набар" был и есть, но что мы можем делать? Только выборы был реальный
шанс что то менять, или очередные майданы - когда уже совсем становится
"тяжко", или вообще "забить" на всё и надеяться на "милость" природы и власти,
которая сама тебе определит твоё "место" в этой жизни. Как то так.

\iusr{Слава Демченко}
\textbf{Leonid Tieriekhov} 

какие майданы, все майданы, заварухи - это планы богачиков. Договорняки.
Последний пример в 14м у фирташа назначили петю вместо кличко. А нам предложили
программу в виде галочки в бюллетенях, мол, всё зависит от твоего выбора. Я,
первый раз засомневался после 91го года, когда увидел, что
камуняки-номенклатурные остались и стали до не приличия по-буржуски богатыми.
Леша, ты правильно говоришь. Согласен. Но, ну их на...

\iusr{Слава Демченко}

\href{https://www.youtube.com/watch?v=SrwlxcEZC4Q}{%
Всё уже украдено до нас ( "Операция "Ы" и другие приключения Шурика" ), youtube, 25.09.2012%
}

\end{itemize} % }

\iusr{Алена Тищенко}
Диагноз точен, В МОРГ

\iusr{Лариса Грищенко}
Странно, а на что надеялись, когда выбирали, ну да выбор у нас всегда один из худшего лучшее, а народ пока валит из страны))

\iusr{Dmitriy Ratushny}
В точку

\iusr{Borys Borysenko}
А как появятся другие при не очень самостоятельном, необразованном и недалеком президенте?

\iusr{Ivan Krasikov}
Премьер превратился в традиционного чиновника-громоотвода, которые были во все времена. Что с него теперь взять?
\end{itemize} % }

