%%beginhead 
 
%%file 30_07_2016.fb.loskutova_natalia.mariupol.1.fotokvest_rodina_kuindzhi
%%parent 30_07_2016
 
%%url https://www.facebook.com/1427894275/posts/pfbid0RULsDj3qLDDqdQT5HCKSHkPeWXvwp2io3KEdB7w6C8DZQTao39ytGRBYQcZQPJWbl
 
%%author_id loskutova_natalia.mariupol
%%date 30_07_2016
 
%%tags mariupol,mariupol.pre_war
%%title Фото-квест "Мариуполь - родина Куинджи"
 
%%endhead 

\subsection{Фото-квест \enquote{Мариуполь - родина Куинджи}}
\label{sec:30_07_2016.fb.loskutova_natalia.mariupol.1.fotokvest_rodina_kuindzhi}

\Purl{https://www.facebook.com/1427894275/posts/pfbid0RULsDj3qLDDqdQT5HCKSHkPeWXvwp2io3KEdB7w6C8DZQTao39ytGRBYQcZQPJWbl}
\ifcmt
 author_begin
   author_id loskutova_natalia.mariupol
 author_end
\fi

Давненько я так замечательно и интересно не проводила время! 

\ii{30_07_2016.fb.loskutova_natalia.mariupol.1.fotokvest_rodina_kuindzhi.pic.1}

Сегодня я участвовала в настоящем квесте, именно КВЕСТЕ, а не той жалкой
пародии на него под названием \enquote{Радужный квест}. Сегодняшнее мероприятие носило
название \enquote{Фото-квест \enquote{Мариуполь – родина Куинджи}}. Квест был посвящен
175-летию со дня рождения нашего знаменитого земляка – Архипа Ивановича
Куинджи. Организаторами этого великолепного действа выступили МНФК \enquote{Берег}, МФК
\enquote{Матрица-М} и художественный музей им. А. И. Куинджи. В этом квесте было все
продуманно, сделано с душой и с умом, так сказать, по-взрослому. 

Во-первых, при регистрации вручалось Положение о проведении этого фото-квеста,
в котором были предусмотрены малейшие детали и нюансы: от целей и задач
проведения, требований к фото и к участникам до примечаний, наподобие
следующих: \enquote{За нарушение правил общественного порядка и правил дорожного
движения во время прохождения маршрута команда дисквалифицируется} или \enquote{Лица,
не достигшие совершеннолетия, допускаются к участию только в командах с
совершеннолетними}. Вот так вот! Ну как тут не провести параллели с
недоразумением типа предыдущего квеста, где детвора носилась через дорогу,
подвергая опасности свои жизни и создавая угрозу на дороге для водителей. 

Во-вторых, были указаны организаторы и члены жюри, которым можно было задать
вопросы и предъявить претензии, в случае чего. Предъявлять, к счастью, было
нечего, поскольку организаторы и члены жюри – просто молодцы!!! 

В-третьих, содержание заданий – оно было квестовым, интересным, иногда
мозгодробительным, но оттого вызывающим еще большее желание его раскусить,
разгадать и выполнить! Например, такое задание: \enquote{Друг по жизни – противник за ...
доской}. Подсказка: бюст. В городском саду не менее десятка бюстов, поди,
угадай, чей нужно было сфотографировать. Оказалось, это великий ученый
Д. И. Менделеев, который был знаком с Куинджи и сыграл с ним не одну партию в
шахматы. Задания были разнообразными, но красной нитью через них шла связь с
жизнью и творчеством известного мариупольца. Нужно было вспомнить многое: и где
мог жить Куинджи, и кем он работал в Мариуполе, и как к нему могли обращаться,
когда он создавал свои шедевры, и чем он увлекался... Очень пригодилось то, что
организаторы на обратной стороне листа с заданиями внесли информацию из
положения о проведении фото-квеста и даже поместили карту того района, где
необходимо было вести поиски. Прекрасно были подобраны локации – они были
равноудалены друг от друга, и обойти их все за 2 часа не составило большого
труда. Даже оставалось время покреативить ) 

В-четвертых, оценивание. Оно проводилось экспертами, которые тщательно изучали
каждую фотографию и выносили справедливое решение. И делалось это не как в
прошлый раз – выбирая наугад из 20 команд. Нам показали итоговый оценочный
лист, куда жюри выставляло баллы за каждую фотографию. А если что-то было не
понятно – они любезно объясняли, где мы совершили ошибку и что нужно было
сфотографировать. 

В общем, наша команда осталась полностью довольна, и мы весь день делились
впечатлениями и снова переживали приятные эмоции. Я пишу эти хвалебные строки
не потому, что мы заняли первое место, а потому, что нам действительно
понравилось! Займи мы последнее место – я написала бы то же самое. Потому что
все было правильно, с душой организовано. Отдельное спасибо руководству и
сотрудникам музея, которое нас очень тепло и радушно встретило на территории
художественного музея им. А. И. Куинджи. Знаю, что они тоже принимали
непосредственное участие в организации мероприятия, поэтому также выражаем им
свою благодарность. Также благодарим нашу мэрию, которая оказала содействие
организации и в лице Д. Визенкова поздравила нас и вручила призы. Как результат,
у меня и у моей команды самые позитивные впечатления от сегодняшнего дня! 

Был только один минус – организаторы немного недоработали в информационном
плане и, в итоге, квест собрал небольшое число участников. А жаль! Мероприятие
получилось увлекательным и познавательным. Я уверена, что пришло бы достаточное
количество человек, но поучаствовав до этого в \enquote{Радужном квесте} у людей
надолго отбили охоту принимать участие в подобных акциях. Люди, если вы видите,
что мероприятие организовано МНФК \enquote{Берег}, МФК \enquote{Матрица-М} или художественным
музеем им. А. И. Куинджи, это изначально предполагает знак качества. Значит, все
будет грамотно организовано и в нем нужно участвовать! Уж мы – так точно будем.
Еще раз – СПАСИБО!!!
