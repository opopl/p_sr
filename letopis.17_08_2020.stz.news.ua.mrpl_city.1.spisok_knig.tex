% vim: keymap=russian-jcukenwin
%%beginhead 
 
%%file 17_08_2020.stz.news.ua.mrpl_city.1.spisok_knig
%%parent 17_08_2020
 
%%url https://mrpl.city/blogs/view/spisok-knig-yaki-varto-prochitati-mariupoltsyam
 
%%author_id demidko_olga.mariupol,news.ua.mrpl_city
%%date 
 
%%tags 
%%title Список книг, які варто прочитати маріупольцям
 
%%endhead 
 
\subsection{Список книг, які варто прочитати маріупольцям}
\label{sec:17_08_2020.stz.news.ua.mrpl_city.1.spisok_knig}
 
\Purl{https://mrpl.city/blogs/view/spisok-knig-yaki-varto-prochitati-mariupoltsyam}
\ifcmt
 author_begin
   author_id demidko_olga.mariupol,news.ua.mrpl_city
 author_end
\fi

\emph{\enquote{Що прочитати у відпустці?}} – мабуть, головне питання, яке щорічно постає перед
кожною людиною. Я вирішила скласти свій невеличкий список книг, які варто
прочитати кожному маріупольцю. А, враховуючи мою любов до історії та рідного
міста, хочу запропонувати незвичний список, що складається з книг саме
маріупольських авторів або написані про Маріуполь.

\ii{17_08_2020.stz.news.ua.mrpl_city.1.spisok_knig.pic.1}

Думаю, всім, хто дійсно любить Маріуполь і його історію варто звернути увагу на
книгу \textbf{\enquote{Однажды в СССР} \emph{Андрія Марченка}}, яка, можливо, у майбутньому буде
екранізованою. Книга переносить читачів у брежнєвські часи. Життя міста у моря
в 1970-ті не може не зачарувати. В центрі сюжету доля молодого героя, який,
образившись на тодішні реалії і панування радянської номенклатури, разом з
армійським товаришем готує пограбування, що сколихне приморське місто. В книзі
дуже точно передана атмосфера того часу. Сторінки роману дозволяють перенестися
у минуле і побачити вулиці Маріуполя 50 років тому. Особисто я отримала велике
задоволення від прочитання цього твору і разом з головними героями переживала
всі описані події. Всі охочі прочитати книгу зможуть з легкістю знайти її в
інтернеті.

Всім, хто віддає перевагу книгам, сповненим гумору, рекомендую оповідання, що
стало для мене справжньою знахідкою – \textbf{\enquote{Істо\hyp{}рія одного дня} \emph{Петра Валерійовича
Каменського}}. За один день головний герой твору Григорій Ільяшенко вирішив
провчити колишнього члена грецького суду Маріуполя Логафетова за зловживання
довіреною йому владою. Виявивши неабиякі акторські здібності, він представився
уповноваженим імператора і розпорядився \enquote{за грабежі та вбивство і взагалі за
всі зловживання позбавити (Логофетова) всіх прав стану зі засланням в алтайські
заводи у вічні працівники, а маєток його продати з публічного торгу і
задовольнити всіх боржників...}. Маріупольська влада, налякана \enquote{начальницькою}
особою, завзято виконала всі вказівки, заарештувала Логафетова і навіть
поголила йому півголови. Але, на жаль, благородна справа Г. Ільяшенка було
розкрита... Цей короткий переказ – ніщо в порівнянні з самою книгою, яка дозволяє
краще зрозуміти побут, відносини між різними народами, особливості життя в
Маріуполі наприкінці XIX ст. І, якщо вірити автору, ця неймовірна історія не
була вигадкою. Всім, хто виявить бажання, відкрити для себе Маріуполь
наприкінці XIX ст., книгу зможу відправити особисто. Для цього можете залишити
коментар до блогу з вашою електронною поштою, чи написати мені у Facebook.

Поціновувачам поезії раджу відкрити для себе унікальне і самобутнє художнє
слово маріупольських поетес \textbf{\emph{Наталі Ростової, Оксани Стоміної та Лідії
Белозьорової}}.

\ii{17_08_2020.stz.news.ua.mrpl_city.1.spisok_knig.pic.2}

А любителям досліджень, раджу ознайомитися з твором маріупольського поета,
прозаїка та есеїста \textbf{\emph{Ніколіна Анатолія Гнатовича} \enquote{Огонь и агнец}}. Це
есе-дослідження про російського поета і прозаїка, що працював у Франції –
\emph{Бокова Миколу}. Автор активно переписувався з Боковим до його смерті, а коли
дізнався, що батько Бокова ще й проживав у Маріуполі, вирішив з'ясувати всі
деталі сімейної історії письменника. З цим твором мені порадила ознайомитися
співробітниця Центральної міської бібліотеки ім. Короленка \emph{\textbf{Ольга Миколаївна
Подтинна}}. До речі, вона брала безпосередню участь в дослідженні Анатолія
Гнатовича. Стиль Ніколіна відрізняється образністю, афористичністю,
використанням свіжих метафор і дуже багатою мовою, тому викличе справжнє
естетичне задоволення у всіх, хто ознайомиться з його роботою.

\ii{17_08_2020.stz.news.ua.mrpl_city.1.spisok_knig.pic.3}

Любителям фантастики цікаво буде прочитати книги \emph{\textbf{Грома Олександра Павловича}} –
маріупольського письменника-фантаста. Автор відкритий до спілкування і з
радістю поділиться своїми книгами. А тим, хто віддає перевагу
соціально-психологічній драмі, варто звернути увагу на молодого письменника
Маріуполя – \emph{\textbf{Моргана Роттена}}. Його книга \textbf{\enquote{История Одного Андрогина}} стає все
більш обговорюваною серед молодих поціновувачів сучасної прози. Автор цього
літературного твору пропонує читачам історію, засновану на особистій драмі
героя-гермафродита, який досяг свого ідеалу, але розчарувався в ньому. Роман
наповнений життям, почуттями і драмою. Незважаючи на серйозність теми,
читається неймовірно легко і на одному диханні. Особисто мене вразило, як
молодий автор настільки грамотно і точно і водночас чуттєво і відверто передає
всі події, що відбуваються з головним героєм. Як підкреслила співробітниця
міської бібліотеки ім. Короленка, Ольга Подтинна, у цього автора велике
майбутнє. Після такого дебюту складно не погодитися з цими словами.

Ось такий вийшов список книг, якими, на мою думку можна насолодитися у
відпустці, чи у будь-який вільний для читання час. Ці книги дозволять не тільки
відкрити рідне місто з нових сторін, а ще й познайомитися з талановитими
маріупольськими письменниками та поетами. Всі, рекомендовані мною книги, можна
знайти в інтернеті (крім \enquote{Історії одного дня}, якою я з радістю поділюся).
Залишається побажати приємного і цікавого читання та незабутніх і яскравих
вражень від прочитаного!

\textbf{Читайте також:} \emph{Мариуполец в команде с MRPL.CITY издал книгу об осуществлении мечты.}%
\footnote{Мариуполец в команде с MRPL.CITY издал книгу об осуществлении мечты., Ганна Хіжнікова, mrpl.city, 11.04.2019, \par%
\url{https://mrpl.city/news/view/mariupolets-v-komande-s-mrplcity-izdal-knigu-ob-osushhestvlenii-mechty-foto}}

