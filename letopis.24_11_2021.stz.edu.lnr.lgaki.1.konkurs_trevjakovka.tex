% vim: keymap=russian-jcukenwin
%%beginhead 
 
%%file 24_11_2021.stz.edu.lnr.lgaki.1.konkurs_trevjakovka
%%parent 24_11_2021
 
%%url https://lgaki.info/novosti/podtverdili-masterstvo-student-akademii-matusovskogo-snova-v-spiske-pobeditelej-sovmestnogo-proekta-shhuki-i-tretyakovki
 
%%author_id 
%%date 
 
%%tags 
%%title Подтвердили мастерство! Студент Академии Матусовского – снова в списке победителей совместного проекта «Щуки» и Третьяковки
 
%%endhead 
\subsection{Подтвердили мастерство! Студент Академии Матусовского – снова в списке победителей совместного проекта «Щуки» и Третьяковки}
\label{sec:24_11_2021.stz.edu.lnr.lgaki.1.konkurs_trevjakovka}

\Purl{https://lgaki.info/novosti/podtverdili-masterstvo-student-akademii-matusovskogo-snova-v-spiske-pobeditelej-sovmestnogo-proekta-shhuki-i-tretyakovki}

Студент первого курса кафедры кино-, телеискусства Академии Матусовского,
будущий телеведущий Ярослав Сорокин стал лауреатом второй степени на II
Всероссийском фестивале-конкурсе чтецов «Читаем классику в Третьяковской
галерее». Этот турнир – совместный проект Театрального института имени Бориса
Щукина и Государственной Третьяковской галереи.

Ярослав читал на конкурсе «Радость» Антона Чехова и занял второе место в
номинации «Серебряный век русской литературы». В ответ на вопрос о главном
впечатлении говорит:

— Должен был прочитать лучше.

Но, на самом деле, его результат – большая победа. С которой мы поздравляем и
самого студента и его педагога по сценической и экранной речи, старшего
преподавателя кафедры кино-, телеискусства, заслуженную артистку ЛНР Наталью
Недоступ!

\ii{24_11_2021.stz.edu.lnr.lgaki.1.konkurs_trevjakovka.pic.1}

К слову, на первом конкурсе «Читаем классику в Третьяковской галерее» два года
назад Академию и Республику представлял тогда студент 4-го курса кафедры
театрального искусства нашего вуза Игорь Чепига. И он тогда стал лауреатом III
степени.

Растем! И это здорово!

«Мы уверены, что благодаря таким проектам осуществляются не только пропаганда
лучших произведений литературы и искусства, но и становится реальностью
поддержка молодых дарований, развиваются межрегиональные связи, повышается
уровень профессиональной компетенции преподавателей образовательных учреждений.

Мы благодарим ваш вуз за участие, поздравляем победителей – студентов,
преподавателей и выражаем надежду на встречу в 2023 году», — сказано в письме,
которое пришло на имя ректора Академии Матусовского Валерия Филиппова за
подписью народного артиста РФ, лауреата Государственной премии РФ, профессора,
ректора Театрального института имени Бориса Щукина Евгения Князева.

Мы начинаем готовиться!

 

Фото из личного архива студента (Ярослав — четвертый слева в нижнем ряду).
