% vim: keymap=russian-jcukenwin
%%beginhead 
 
%%file 03_06_2021.fb.maslov_evgenij.1.degradacia_nacia_slovo
%%parent 03_06_2021
 
%%url https://www.facebook.com/maslovevgeniy14/posts/1492894894385547
 
%%author Маслов, Евгений
%%author_id maslov_evgenij
%%author_url 
 
%%tags 
%%title Общая деградация нас как нации сказалась на языке прежде всего, умения обратиться друг к другу мы теряем себя как народ. Как жить без умения назвать?
 
%%endhead 
 
\subsection{Общая деградация нас как нации сказалась на языке прежде всего, умения обратиться друг к другу мы теряем себя как народ. Как жить без умения назвать?}
\label{sec:03_06_2021.fb.maslov_evgenij.1.degradacia_nacia_slovo}
\Purl{https://www.facebook.com/maslovevgeniy14/posts/1492894894385547}
\ifcmt
 author_begin
   author_id maslov_evgenij
 author_end
\fi

* * *
"... Мы страна без обращения к другому. Вот что я слышал от одного эмигранта, приезжавшего в Россию: «Вы знаете, что у вас заменило обращение к другому человеку? Слово «ну». Всегда к нам обращается экскурсовод и говорит: «Ну, пойдем...», «Ну, сейчас будем обедать...» Постоянное «ну», привычка обращаться с понуканием вошла в язык.

\ifcmt
  pic https://scontent-cdg2-1.xx.fbcdn.net/v/t1.6435-9/194945088_1492894871052216_72130961209307330_n.jpg?_nc_cat=107&ccb=1-3&_nc_sid=730e14&_nc_ohc=uHQU_QacwWMAX_M42dv&_nc_ht=scontent-cdg2-1.xx&oh=a5928f011406a21b71223586affb8523&oe=60DFBF14
\fi

... Общая деградация нас как нации сказалась на языке прежде всего, умения обратиться друг к другу мы теряем себя как народ. Как жить без умения назвать? Вообще заметить какое-нибудь явление – это дать ему имя, создать термин, поэтому в средние века наука занималась главным образом называнием, созданием терминологии. Называние уже было познанием. Когда открывали остров, ему давали название, и только тогда это было географическим открытием. Без называния открытия не было.
... У меня очень много писем по поводу мата или, как осторожнее говорили до революции, «трехэтажных выражений». Если бесстыдство быта переходит в язык, то бесстыдство языка создает ту среду, в которой бесстыдство уже привычное дело. Существует природа. Природа не терпит бесстыдства.
... Тот, кто чувствует себя свободным, не будет отвечать матом…
... В лагере на Соловках расстреливали чаще всего тех, кто НЕ РУГАЛСЯ <матом>. Они были «чужие».
... Еще сто лет назад в словаре русского языка было 287 слов, начинающихся с «благо». Почти все эти слова исчезли из нашей речи, а те, что остались, обрели более приземленный смысл. К примеру, слово «благонадежный» означало «исполненный надежды»…
... Слова исчезли вместе с явлениями. Часто ли мы слышим «милосердие», «доброжелательность»? Этого нет в жизни, поэтому нет и в языке. Или вот «порядочность». Николай Калинникович Гудзий меня всегда поражал – о ком бы я ни заговорил, он спрашивал: «А он порядочный человек?» Это означало, что человек не доносчик, не украдет из статьи своего товарища, не выступит с его разоблачением, не зачитает книгу, не обидит женщину, не нарушит слова.
... На Соловках интеллигентного, доброго Георгия Михайловича Осоргина островное начальство собиралось расстрелять и уже заключило в карцер, когда по разрешению более высокого начальства к Осоргину приехала на свидание жена, княжна Голицына. Осоргина выпустили под ЧЕСТНОЕ СЛОВО ОФИЦЕРА с условием, что он ничего не скажет жене о готовящейся ему участи. И он ничего ей не сказал.
... А «любезность»? «Вы оказали мне любезность». Это добрая услуга, не оскорбляющая своим покровительством лицо, которому оказывается. «Любезный человек». Целый ряд слов исчезли с понятиями. Скажем, «воспитанный человек». Он воспитанный человек. Это прежде всего раньше говорилось о человеке, которого хотели похвалить. Понятие воспитанности сейчас отсутствует, его даже не поймут. «Доброта» из нашей жизни уходит, как и словосочетание «добрый человек», которое в русских народных сказках характеризует вообще человека, ВСЯКОГО ЧЕЛОВЕКА.
... Я бы поставил на первое место необходимость создания словаря БУНИНА. Его язык богат не только связью с деревней и дворянской средой, но еще и тем, что в нем литературная традиция – от «Слова о полку Игореве», от летописей.
... До сих пор остается бедой русского языка то, что отменили преподавание церковно-славянского языка. Это был второй язык, близкий к русскому. Нарядный язык… Да-да, этот язык поднимает значение того, о чем идет речь в слове. Это другое совершенно высокое эмоциональное окружение. В старой гимназии древнерусской литературе уделяли больше места, больше внимания, чем на современных филологических факультетах... Исключение из школьного образования церковно-славянского и нашествие матерщины – это симметричные явления.
... Очень важно читать детям вслух. Чтобы учитель пришел на урок и сказал: «Сегодня мы будем читать «Войну и мир». Не разбирать, а читать с комментариями. Так читал нам в школе наш учитель словесности Леонид Владимирович Георг. Стихи же вообще нельзя прочитать с первого раза. Сперва нужно уловить музыку стиха, затем уже читать с этой музыкой – про себя или вслух."

Академик Дмитрий Сергеевич Лихачёв
