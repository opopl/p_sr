% vim: keymap=russian-jcukenwin
%%beginhead 
 
%%file slova.madagaskar
%%parent slova
 
%%url 
 
%%author 
%%author_id 
%%author_url 
 
%%tags 
%%title 
 
%%endhead 
\chapter{Мадагаскар}
\label{sec:slova.madagaskar}

%%%cit
%%%cit_head
%%%cit_pic
%%%cit_text
Как бы не пришлось нам есть саранчу и кактус, как на \emph{Мадагаскаре}...
Зе-зомби на Дне Конституции.  Смотрю торжественное заседание в Раде.  Речь
Александра Мороза, одного из отцов Конституции.  Говорит о новом феодализме, о
нарушении функционала социального государства.  В завершии - просит не
преврашать землю в товар. Это станет началом конца государства. Овации в зале,
провожают стоя.  У Зеленского и Разумкова не дрогнул ни оди мускул на лице. И,
вероятно, ни одна извилина.  Что-то вообще может повлиять на таких людей?
Просто зародить сомнения?  Как написал Виктор Ерофеев: сомнение - не слабость
человека, а способность к анализу и синтезу.  Такой способности у сегодняшней
власти нет
%%%cit_comment
%%%cit_title
\citTitle{Что-то может повлиять на украинскую власть? Просто зародить сомнения? / Лента соцсетей / Страна}, 
Алексей Кущ, strana.ua, 28.06.2021
%%%endcit

