% vim: keymap=russian-jcukenwin
%%beginhead 
 
%%file 08_11_2021.fb.makarevich_ekaterina.1.tochka_blago_nasilie
%%parent 08_11_2021
 
%%url https://www.facebook.com/ekmakarevich/posts/4776834155714349
 
%%author_id makarevich_ekaterina
%%date 
 
%%tags ad,blago,chelovek,gegel_georg,nasilie,psihologia
%%title Где та точка, когда благо становится насилием?
 
%%endhead 
 
\subsection{Где та точка, когда благо становится насилием?}
\label{sec:08_11_2021.fb.makarevich_ekaterina.1.tochka_blago_nasilie}
 
\Purl{https://www.facebook.com/ekmakarevich/posts/4776834155714349}
\ifcmt
 author_begin
   author_id makarevich_ekaterina
 author_end
\fi

Вы когда-нибудь задумывались о том, почему благие намерения ведут в ад? Где та
точка, когда благо становится насилием? 

Прежде всего, благо есть что-то не от мира сего, это то, что приходит как дар
сверху, одаривается, но вот как этим даром воспользуется человек, зависит
исключительно от нравственности человека. То есть последующая операция с благом
- свобода человека. 

Когда благо нисходит в земной, физический мир, то первым этапом часто приходит
соблазн или испытание. Оно заключается в том, что выходя из-за порога
небесного, благодатного, он прежде чем преобразить благим опытом, разделяется
на две части. У Гегеля - это тезис и антитезис. 

Так любое явление, возникающее в обществе, как мы все уже заметили, проходит
через противоречие в обществе. Общество делится на две части. Всегда. На уровне
внутреннего человека происходит то же, но многие этого не осознают, поскольку
смотрят лишь на внешние проявления события. А любой феномен, разделяясь в
физическом мире, и родив два варианта, создает в обществе вопрос и запрос на
то, чтобы пройти и пережить некоторый опыт. 

Это процесс, описанный сверху вниз. 

А теперь то же самое посмотрим снизу вверх, то есть взглянем на то, как человек
справляется с видимым явлением. Когда нечто происходит новое, человек всегда
хочет распознать, что же делать, как относиться к явлению. И любое явление в
себе содержит два варианта действий - обезопасить себя и одновременно найти
некий вариант, который часто выходит за пределы двух вариантов, а как бы
складывается в третье - синтез. 

Первая позиция - позиция со стороны власти, то есть обрести власть над
явлением. Власть как мотивация всегда про борьбу и подавление, чтобы
восстановить безопасность. 

Вторая позиция, по сути, является творческой. И она же есть выход и решение
ситуации. 

Вот только творческая позиция всегда оказывается в меньшинстве. Почему? Потому
что она интуитивна и в каждый конкретный момент требует собственного
уникального поступка, решения, мысли. Но именно она и является выигрышной,
потому что Бог всегда хочет, чтобы человек научился творить. 

Почему же она оказывается в меньшинстве? потому что первая позиция, которая про
безопасность, которая тоже нужна безусловно, хочет и ее в том числе побороть.
Выйдя в позицию борьбы, она уже не замечает, что борется с той интуицией,
которая единственно и может спасти положение. 

Проигрыш первой позиции заведом, потому что неверно выбран мотив и направление.
Власть - это позиция сверху. Но Бог не хочет, чтобы люди боролись друг с
другом, он хочет, чтобы каждый творил, рос и помогал другим творить, расти. Но
люди, которые выбирают позицию власти забывают, что борьбой лишь подавляют не
только худшее, что есть в обществе, какие-то пороки, но часто и самое главное,
что позволяет человеку направлять свою волю к творчеству и таким образом
подрывают само желание что-либо делать. А потом удивляются, почему в обществе
депрессия. 

И в итоге, если власть вовремя не останавливается, то ее останавливает то, что
она сама начала. Неслучайно и говорится "кто с мечом придет, от меча и
погибнет". И это не просто слова, это принцип, закон жизни. 

Но что же делать и как выйти из тупика, который власть часто неосознанно сама
же создает, подавляя и желая создать единственно верный принцип отношения к
чему-либо. И тем самым, погружая общество в континуум бесконечности. То есть
когда общество по кругу спорит и количество противоречий лишь ширится. 

Позиция власти нужна для того, чтобы подавить худшее, но не трогать лучшее,
творческое. А творческое априори про свободное. Потому что именно в творчестве
и кроется решение ответа на тот или иной вопрос и выход из противоречия.
Всегда! 

Но что делает власть часто вместо этого? Она вместо того, чтобы подавить
худшее, хочет распространить свое влияние и на творчество, тем самым, подавляет
ростки того, что когда взойдет, поможет решить вопросы. 

Другими словами, она и вбивает своим мотивом власти, то есть позиции сверху -
клин между двумя позициями. И вот тут и происходит начало конца...

Потому что вбив клин, тем самым, одну часть из двух она определяет как верное -
доброе. А вторую часть - как злое. 

И в этот момент происходит вкушение от запретного дерева познания добра и зла. 

Почему? 

А поскольку изначально она руководствуется безопасностью, что не удивительно,
это ее задача, то выбирает ту часть, которая отстаивает безопасность. И это
почему-то почти всегда оказывается те, кто слева, то есть те, которые так же
хочет побороть все одним махом и то же руководствуются желанием обрести власть
над явлением. 

Все бы ничего. Но вот те, которые оказываются справа и оказываются праведники,
которые все это время держались и пытались не делить людей на таких и сяких. По
сути, это всегда делает позиция власти - потому что ее главная задача подавить.
Давить, то есть принижать ценность чего-либо. Уничтожать. И одновременно всех
сделать одинаковыми. Такими же. 

В итоге, те, кто справа оказываются либо жертвой, либо теми самыми настоящими
Человеками, потому что пытаются докопаться до сути явления, которое поможет и
уничтожить плохое, но при это не воспользоваться насилием по отношению к
действительно ценному. То есть и сохранить свободу. 

То есть справа по сути делают внутри себя работу и за власть, потому что учатся
вычленять факты, которые приводят к плохому. Но при этом и сохранять и помогать
расти тому, что позволит справиться - творческий импульс, креативность. 

В итоге, именно праведники, поскольку они проделывают грандиозную внутреннюю и
внешнюю работу по различению, проходят от Бога урок и потому получают
творческую силу, которая впоследствии решит противоречие. 

А что значит творческое? Это и есть синтез, то есть взгляд на противоречие как
на некогда цельное явление. И уже из этого внутреннего созерцания цельности
явления, понимания, что же нужно делать. 

В итоге, те, кто справа, находят решение. Власть это присваивает как свое
изобретение. Так и не получив урок, потому что внутри себя не проделало
внутреннюю работу, ведь все время тратило на то, чтобы вернуть влияние через
восстановление безопасности. Но зато общество выдыхает. 

И все начинается сызнова. Только очки духовные в качестве кристаллизации как
человека, поскольку пережили серьезный опыт напряжения и переосмысления,
получают именно те, которые были справа и которые не поддались, потому что
слушали свою интуицию, всегда недоказуемую до поры до времени, но единственно
правильную, потому что интуиция или вера - это и есть связь с Богом.

\begin{itemize} % {
\iusr{Денис Смирнов}

Прекрасно, Катя! Задумался — и попытаюсь свести к лапидарному  @igg{fbicon.face.smiling.eyes.smiling} : «благие» они
заблаговременно, а «намерения» значит — решимость довести их до «блага». И в
этом уже пагуба. Если мы оцениваем результат вообще, а уж тем более — заранее,
мы сразу замыкаемся в области безжизненного прошлого. В творчестве же ценен
процесс, а результат становится Божией милостью, явлением чуда. Я не могу
творить, опираясь на готовую оценку: это ремесленничество, но никак не
творчество. «Се творю всё новое», и это новое всегда — милость и чудо.

Спасибо тебе!

\end{itemize} % }
