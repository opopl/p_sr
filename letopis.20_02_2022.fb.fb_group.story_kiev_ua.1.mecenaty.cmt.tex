% vim: keymap=russian-jcukenwin
%%beginhead 
 
%%file 20_02_2022.fb.fb_group.story_kiev_ua.1.mecenaty.cmt
%%parent 20_02_2022.fb.fb_group.story_kiev_ua.1.mecenaty
 
%%url 
 
%%author_id 
%%date 
 
%%tags 
%%title 
 
%%endhead 
\zzSecCmt

\begin{itemize} % {
\iusr{Валентина Тимошенко}
Дякую за прекрасну розповідь

\iusr{Елена Сидоренко}

Оксана, спасибо, напомнили мне о поездке в Радомышль! Всё это видели своими
глазами, и бумагу делали! Очень понравилась экскурсия! @igg{fbicon.heart.beating} 

\ifcmt
  ig https://scontent-lhr8-1.xx.fbcdn.net/v/t39.30808-6/274542593_2754769034817681_6436224276005913741_n.jpg?_nc_cat=106&ccb=1-5&_nc_sid=dbeb18&_nc_ohc=iMuSY05kxb8AX_yjPy6&_nc_ht=scontent-lhr8-1.xx&oh=00_AT-vjJQK82mzWxAh7IWwo7iftJ4ekjQXr0GRsFMxyehTyQ&oe=621B5206
  @width 0.3
\fi

\iusr{Ростислав Чентемиров}
Чудесное место. Побродил бы с удовольствием.

\iusr{Тетяна Саєнко}
Пані Оксано!
Дякую Вам за чудову розповідь!
Завжди цікаво читати Ваші дописи!
Окремо дякую за Українську Мову!!!

\iusr{Оксана Дубинина}
\textbf{Тетяна Саєнко} @igg{fbicon.heart.red}

\iusr{Ирина Веник}

Меценаты???? вы век не перепутали? кому это надо? жлобью, захватившему Киев? на
Бродский все меценатство закончилось, к сожалению... а за рассказ спасибо!

\begin{itemize} % {
\iusr{Оксана Дубинина}
\textbf{Ирина Веник} и Вам спасибо! Всё же хочется надеяться на чистые помыслы. Слишком много горького происходит...

\iusr{Ирина Веник}
\textbf{Оксана Дубинина} 

мне иногда хочется удалиться из группы.. так обидно за любимый Киев.. как
посмотрю, что было и что уничтожено, так нервов не хватает...

\end{itemize} % }

\iusr{Раиса Карчевская}
Спасибо большое за очень интересный пост

\iusr{Микола Веселий}
Був там проїздом - тихе містечко і люди добрі

\begin{itemize} % {
\iusr{Ирен Рум}
\textbf{Микола Веселий} проездом да, а вот бензин и вход 150грн это дороговато

\iusr{Микола Веселий}
\textbf{Ирен Рум} Ну, иногда можно)
\end{itemize} % }

\iusr{Maksym Oleynikov}

Пані Оксано, перераховані Вами у першому абзаці допису об'єкти знаходяться у
приватній власності. Меценат може дати кошти на реставрацію чогось державного,
суспільного, а не чужого приватного. Приклад Ольги Богомолець засвідчує: якщо
хочеш і можеш щось реставрувати - спочатку ним заволодій, зроби своїм.

\begin{itemize} % {
\iusr{Оксана Дубинина}
\textbf{Maksym Oleynikov}

так, Ви праві. Але ж так шкода, коли гинуть такі чудові об‘єкти історичної
спадщини. Можливо новий хазяїн зміг би зробити для всього Києва якийсь
реставраційний проект

\iusr{Вита Вовченко}
\textbf{Maksym Oleynikov} 

А будинок Терещенко на розі Коцюбинського, підкажіть, будь ласка, в чиїй
власності? Питаю, бо два роки тому давали запит і не отримали чіткої відповіді,
на чиєму балансі будинок.

\begin{itemize} % {
\iusr{Maksym Oleynikov}
\textbf{Вита Вовченко} Власник особняка на бульварі Шевченка, 34 станом на 2021 рік - ПАТ «Центрелеватормлинбуд».

\iusr{Вита Вовченко}
\textbf{Maksym Oleynikov} Дякую. Несподівано. Ходять чутки, що Хілтон хочуть добудувати до вулиці Коцюбинського.

\iusr{Maksym Oleynikov}
\textbf{Вита Вовченко} 

Ще у 2017р. була інформація, що поряд із садибою планують побудувати
17-поверховий офісно-діловий центр. До \enquote{Хілтону} ці плани стосунку не мають.


\iusr{Вита Вовченко}
\textbf{Maksym Oleynikov} 

Дякую. Можу лише додати, що ще кілька років у внутрішньому дворі знаходились
офіси, редакція газети. Будова була у нормальному стані. Зараз має жалюгідний
вигляд, вікна вибито, лівньовки пошкоджено та т. і. Стійке враження, що будуть
зносити, нажаль. @igg{fbicon.cry} 

\end{itemize} % }

\iusr{Maksym Oleynikov}

Шкодую, як і Ви. Але \enquote{реставраційний проект для усього Києва} - маніловщина
чистої води. Реставрація в рази дорожча за нове будівництво, це просто космічні
суми, від проектування і до виконання практичних робіт. Так що сподіватися на
порятунок того, що гине, можна лише в тому разі, якщо це державна чи комунальна
власніть. От до уваги до неї і варто закликати меценатів. А за розповідь про
Радомишль - дякую, може з настанням весни виникне у киян бажання поїхати і
подивитися, бо є на що.

\end{itemize} % }

\iusr{Инна Валентиновна}

Дякую за чудову екскурсію. Підбір фото чудовий, дає повне уявлення про
радомишльський замок. Ваш розпач про старі історичні будівлі Києва поділяю
повністю і підтримую.

\iusr{Татьяна Желдубовская}
К, сожалению, таких единицы

\iusr{Людмила Джулай}
Прекрасне місце для втілення високої ідеї.. Допомагала пані Ользі із промоцією. Тож знаю ціну!!!!

\iusr{Оксана Дубинина}
\textbf{Людмила Джулай} тисну Вашу руку. @igg{fbicon.heart.red} Браво!!

\iusr{Алла Васильева}

Скільки цікавого поруч, але подорожуємо в далекі місця. Дякую за чудову
інформацію і неймовірні фото  @igg{fbicon.hand.waving} 

\iusr{Alina Gaiduchenko}
Дякую, Оксанка, за допис, за душу бере. Дякую і за коментарі, які пояснюють, що не плекайте надію марно. На жаль...

\iusr{Светлана Светличная}
Не треба чекати меценатів. Саме держава мусить опікуватись своїм надбанням. треба ії до цього примушувати

\iusr{Оксана Дубинина}
\textbf{Светлана Светличная} так, власне, це обов‘язок тих, хто дуже хотів влади...

\iusr{Ольга Гуцу}
Спасибо, очень интересно!

\iusr{Владимир Новицкий}
Спасибо за чудесные фото!!

\iusr{Михаил Гаркавцев}
меня удивили цены на входные билеты к мадам Богомолец!

\begin{itemize} % {
\iusr{Оксана Дубинина}
\textbf{Михаил Гаркавцев} 

сначала удивление, а потом понимаешь, что содержание комплекса требует больших
затрат. А так чувствуется крепкая рука хозяина, добротно и чисто. цены немалые,
но и не космические.

\iusr{Михаил Гаркавцев}
\textbf{Оксана Дубинина} Почаще приходите!

\iusr{Оксана Дубинина}
\textbf{Михаил Гаркавцев} почаще не получится, но еще обязательно буду приезжать.  @igg{fbicon.face.happy.two.hands} 
\end{itemize} % }

\iusr{Ольга Еремеева}

Спасибо за чудесный очерк. Жаль, что меценатов так мало, в основном наша
\enquote{элита} не связывает свое будущее с Украиной, здесь они только обогащаются.

\iusr{Тамара Ар}

Да, были ! Да, красиво и интересно в Радомышле! Да, в Киеве разрушаются
старинные дома и усадьбы! Почему? Потому что, нет государственных денег, а
меценаты все уехали в Европу

\iusr{Анна Попова}

Большое спасибо за прекрасный рассказ! Летом обязательно семьёй посетим этот
прекрасный уголок! Благодарность Ольге Богомолец, что не очерствела душой и
занимается по мере своих сил восстановлением прошлого уголков Украины!

\iusr{Екатерина Сыч}

Бажаю щоб таких меценатів було побільше, які зберігають нашу історію. Дякую
щиро Ользі Богомолець!!!

\iusr{Татьяна Оржеховская}
Благодарю. Очень жаль, что канули в лету традиции меценацтва

\iusr{Ольга Волкова}
Очень познавательно . Спасибо.

\iusr{Наталия Евдокимова}
ЧЕСТЬ И ХВАЛА. Умница

\iusr{Тамара Нарижная}
чудесный музей я была там и природа восхитительная

\iusr{Наталия Лаврентьєва}
Колекція ікон підібрана чудові дуже професійно

\iusr{Лолита Рыжак}
Замок Річарда на Андріївському узвозі та садибу Івана Терещенка на бульварі Шевченка меценатам не віддають.

\iusr{Валя Богданец}

Была я в этом замке в Радомышле. Там очень красиво. И икон очень большое
количество. У каждой своя история.

У кого есть возможность съездить, едте, не пожалеете. Деткам тоже будет
интересно.

Лучше ехать когда потеплеет. Много экспонатов находится на улице.

\iusr{Надежда Калинская}

Дякую за історічну довідку, чудовий парк, замок, ікон старовинних дуже багато,
були сімейством на екскурсії, сподобалось!

\iusr{Людмила Ива}
Спасибо!)

\iusr{Валетина Пухкал}
Цікаво. Дякую!

\iusr{Наталия Озерова}

\ifcmt
  ig https://scontent-lhr8-1.xx.fbcdn.net/v/t39.30808-6/274228044_2605077602970335_7299632354032569016_n.jpg?_nc_cat=109&ccb=1-5&_nc_sid=dbeb18&_nc_ohc=VhG10lhcG4MAX84PyAQ&_nc_ht=scontent-lhr8-1.xx&oh=00_AT-HzLJ4clR6l7BEHrYFfeTnnefU2SNMGWBVXAu8DUXpIw&oe=621A0F57
  @width 0.2
\fi

\iusr{Мария Дидык}

Дуже ВЧАСНО Це потрібноробити ТЕРМІНОВО Ці памятки вже ЛЕДВЕ ДИХАЮТЬ Що собі
ДУМАЄ та РОБИТЬ Держкомітет по ОХОРОНІ ПАМЯТОК ЖАХ Підписуюсь під кожним словом
і ДЯКУЮ що ЗАБИЛИ ТРИВОГУ

\iusr{Анна Мартинюк}
Прекрасное место и большие вложения.

\iusr{Barbara Novokhatska}
Дякую Ользi Богомолець! Дякую, Оксано!

\iusr{Donna Stopper}
Красивый замок-крепость с профи реставрацией, все аутентично и атмосферно. Спасибо..

\iusr{Tetyana Dmytriyeva}
Страна - руина. Бессильны......

\iusr{Yuriy Pavlenko}

Там дуже гарно, колись там був, ніби занурився у стародавні часи) особливо дуже
гарна виставка ікон, зібраних з усіх куточків України))

\iusr{Нина Смоляк}

Всей душой поддерживаю крик души о реставрации памятников. Здание Ивана
Терещенко в самом центре города. Неужели не найдется истинного патриота
мецената для его восстановления. Медбиблиотека тоже не в лучшем виде. Ау.
Меценаты.

\iusr{Любовь Егорова}

Оксаночка, спасибо за рассказ и фотки! Как всегда после твоих постов, хочется
сразу отправиться в те чудесные места! Увидеть все своими глазами и насладиться
увиденным. Просвещай и надыхай!

\iusr{Оксана Дубинина}
\textbf{Любовь Егорова} спасибо @igg{fbicon.heart.red}

\end{itemize} % }
