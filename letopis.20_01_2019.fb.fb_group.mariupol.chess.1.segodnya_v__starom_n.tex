%%beginhead 
 
%%file 20_01_2019.fb.fb_group.mariupol.chess.1.segodnya_v__starom_n
%%parent 20_01_2019
 
%%url https://www.facebook.com/groups/mariupol.chess/posts/1321451621346864
 
%%author_id fb_group.mariupol.chess,jarmonov_igor.chess.mariupol
%%date 20_01_2019
 
%%tags mariupol,mariupol.pre_war,chess,chess.zadachi
%%title Сегодня в "старом-новом" клубе прошел конкурс по решению шахматных задач
 
%%endhead 

\subsection{Сегодня в \enquote{старом-новом} клубе прошел конкурс по решению шахматных задач}
\label{sec:20_01_2019.fb.fb_group.mariupol.chess.1.segodnya_v__starom_n}
 
\Purl{https://www.facebook.com/groups/mariupol.chess/posts/1321451621346864}
\ifcmt
 author_begin
   author_id fb_group.mariupol.chess,jarmonov_igor.chess.mariupol
 author_end
\fi

Сегодня в \enquote{старом-новом} клубе прошел конкурс по решению шахматных задач.
Учитывая юный возраст участников (от 6 до 14 лет), я подобрал 5 классических
задач миниатюр на мат в 2 хода. Миниатюра - позиция с количеством фигур не
более 7. На этот раз конкурс проводился по \enquote{всем} правилам, как на многих
других подобных конкурсах. Каждый из участников решал шахматные \enquote{орешки}
самостоятельно, без наводящих подсказок, и записывал свое решение на отдельном
бланке. На решение первой задачи давалось 5 минут, второй - 10 мин. и т. д. К
слову, именно вторая задачка и стала камнем преткновения для многих детей.
Вторую и четвертую не сумели разгадать 6 решателей: Прунчак Кирил, Береговой
Михаил, Журавлев Кирилл, Перцев Евгений, Трощенко Никита и Пархоменко Кирилл.
Они и разделили на равных 4-9 места - правильно решили по 3 задачи. Четыре
задачи решил правильно только один участник - перворазрядник Шанькин Иван, и
это 3-й результат! А победу в конкурсе поделили 2 юных участника. Старостенко
Диана (1 разряд) и Антон Синельников (2 разряд). Они решили абсолютно верно все
5 заданий! 100\% результат! На дополнительной задаче ошибся Антон и в итоге
занял 2 место. А корону победителя примерила на себя юная звездочка Диана.
Отличившиеся были награждены эксклюзивными призами. Все фото с двухдневного
шахматного марафона - завтра! Устал просто малость!
