% vim: keymap=russian-jcukenwin
%%beginhead 
 
%%file slova.zapad
%%parent slova
 
%%url 
 
%%author 
%%author_id 
%%author_url 
 
%%tags 
%%title 
 
%%endhead 
\chapter{Запад}
\label{sec:slova.zapad}

%%%cit
%%%cit_pic
%%%cit_text
А вот \emph{западные} русские, которые остались под властью Польши, долго искали свое
уникальное самоназвание, но так и не нашли. За неимением лучшего они приняли
польское определение этой части русской нации, так как, по сути, русскими уже
не являлись – ассимиляция русского населения в польской среде происходила
достаточно интенсивно. И в то время как русский язык в землях, принадлежавших
Польше, никак не развивался с древних времен и очень сильно пополнился
полонизмами и германизмами («папир», «файно», «накшталт», «кошт», «страва» и
прочее), менялось и самосознание \emph{западных русских}. Эти русские, с одной
стороны, уже забыли, что они русские, а с другой – все же не хотели становиться
поляками до конца
%%%cit_title
\citTitle{Почему современную Украину назвали «окраиной», а не более престижно - Киевской Русью?}, 
Исторический Понедельник, zen.yandex.ru, 22.02.2021 
%%%endcit


%%%cit
%%%cit_head
%%%cit_pic
%%%cit_text
И теперь задача \emph{Запада} - обезвредить Россию в ее новом качестве.  Московский
центр Карнеги вдруг выкатил достаточную годную статью по встрече Путина и
Байдена. Приведу отрывки. Кому интересно - вот ссылка на полную версию.
\url{https://carnegie.ru/commentary/84786} За последний год демонстративно
завершен условный горбачевско-ельцинский этап развития России. Разрыв с этим
этапом оформлен обновлением конституции. Многочисленные поправки знаменуют не
столько начало этапа, сколько переводят количество накопившихся изменений в
качество, институциализируют путинскую Россию. Путин поехал в Женеву говорить
от имени этой новой России, которая больше не развивается через строительство
или даже имитацию западных институтов. Задача снята с повестки. Россия больше
не идет суверенным путем к общей цели, теперь суверенна и сама цель
%%%cit_comment
%%%cit_title
\citTitle{Превратить Россию в часть западного мира не удалось}, 
Максим Войтенко, strana.ua, 19.06.2021
%%%endcit

%%%cit
%%%cit_head
%%%cit_pic
\ifcmt
tab_begin cols=2
  pic https://avatars.mds.yandex.net/get-zen_doc/5286518/pub_60d4599a6032b51597be0fe3_60d459d54164827f89ad3306/scale_1200
	caption Британский эсминец \enquote{Дефендер} (источник: glas.ru)
	width 0.4

	pic https://avatars.mds.yandex.net/get-zen_doc/5238996/pub_60d4599a6032b51597be0fe3_60d45c89aa9ed1749402327e/scale_1200
	caption Посмотрите в грустные глаза английского матроса. Он явно хочет идти другим маршрутом
	width 0.4
tab_end
\fi
%%%cit_text
Произошедший в Чёрном море инцидент с участием британского эсминца «Дефендер»
глубоко символичен и войдёт в учебники истории, как поворотная точка во
взаимоотношениях России и \emph{Запада}.  Это ведь не только командир корабля
береговой охраны обратился к капитану эсминца. Это Россия обратилась к НАТО его
устами, его словами - «Если не измените курс, я буду стрелять»
%%%cit_comment
%%%cit_title
\citTitle{«Если не измените курс, я буду стрелять»: как в Чёрном море британцам объяснили внешнюю политику России}, 
ИСТОРИЯ Спорный Контент, zen.yandex.ru, 24.06.2021 
%%%endcit

%%%cit
%%%cit_head
%%%cit_pic
%%%cit_text
Тем временем европейские послы впервые высказали озабоченность информационной
политикой нынешней власти. Поводом для этого стало закрытие газеты Kyiv Post.
Формально в нем нет никакой политики: газету закрыл ее нынешний владелец –
бизнесмен Аднан Киван (который говорит, что готовит ее перезапуск).  Но
комментарии послов ЕС, Германии и Швеции показывают, что никто из них в
"коммерческую" версию событий не верит. И потому дружно говорят о
"последствиях" и о том, что "свободные и независимые СМИ в Украине нуждаются в
защите на нескольких фронтах".  Наверняка они правы. Наверняка газету закрыли
потому, что она критически освещала деятельность Банковой, а также активно
писала о скандалах вокруг Зеленского (вроде офшорного).  Но в этом и есть
цинизм ситуации: когда закрывали "112", "Ньюсван" и "ЗИК", когда вводили
санкции против "Страны", никто из послов не заявлял о необходимости защиты СМИ
и свободы слова. То есть в их понимании "независимые СМИ" – это только те,
которые отражают правильную, с точки зрения \emph{Запада}, точку зрения
%%%cit_comment
%%%cit_title
\citTitle{Антирекорд смертности от ковида, 73\% лохов, "гибридная война" Бацьки. Итоги "Страны"}, 
, strana.news, 10.11.2021
%%%endcit
