% vim: keymap=russian-jcukenwin
%%beginhead 
 
%%file 18_11_2020.news.ua.strana.1.battle_ljusenka_jura_ilovajskii.butusov_i_ilovajsk
%%parent 18_11_2020.news.ua.strana.1.battle_ljusenka_jura_ilovajskii
 
%%url 
%%author 
%%tags 
%%title 
 
%%endhead 

\subsubsection{Бутусов и Иловайск}

О деталях подготовки наступления на Иловайск, в котором участвовал журналист
Юрий Бутусов, рассказывал еще командир батальона "Шахтерск" Андрей
Филоненко\Furl{https://strana.ua/articles/istorii/28775-vojna-2014-put-k-ilovajskomu-kapkanu.html}
- временной следственной комиссии, расследующей причины трагедии.

Встреча в Днепропетровской областной администрации, которой тогда
руководил Игорь Коломойский, состоялась 5 августа 2014 года. Она
продлилась 5-6 часов. На совещании в здании ОГА присутствовали Геннадий
Корбан (тогдашний вице-губернатор области), Семен Семенченко (батальон
\enquote{Донбасс}), Андрей Билецкий (\enquote{Азов}), журналист Юрий Бутусов. Позже
подъехал и командующий в секторе \enquote{Б} генерал-лейтенант Руслан Хомчак -
ныне глава украинского генштаба.

Корбан озвучил тему совещания --- взятие Иловайска и до приезда Хомчака
участники обсуждались детали операции: с какой стороны зайти, качество
связи.

Прибывший Хомчак стал вести совещание вместо Корбана. Он сказал, что
возьмет в кольцо Донецк, а для Иловайска не было времени и пехоты, поэтому
он попросил разобраться с этим батальоны: \enquote{Зайдите туда батальоны и
зачистите, там максимум 50 террористов}.

В результате договорились, что своими силами добровольческие батальоны
\enquote{Шахтерск}, \enquote{Азов} и \enquote{Днепр-1} возьмут южную половину города.

По словам Филоненко, все действия тогда согласовывали с представителями
МВД --- главой Департамента по управлению и обеспечению подразделений
милиции особого назначения Виктором Челованом и представителем МВД в штабе
АТО Владимиром Гриняком. Для выхода в зону АТО выходил соответствующий
приказ министра внутренних дел. И после совещания в ОГА он получил
согласование от координатора от МВД в штабе АТО.

Часть из этих показаний Филоненко позже подтвердили и другие участники
встречи.

Так, в разговоре со \enquote{Страной} Геннадий Корбан подтвердил, что у него в
кабинете проходила встреча, поскольку командованием было принято решение
взять Иловайск.

\enquote{Важно было скоординировать обеспечение всех войск, принимавших участие в
военной операции, начиная от батальонов террообороны, заканчивая
бригадами. Цель и задача - максимально обеспечить людей всем необходимым
для проведения этой операции и логистику} - рассказал Корбан.

На вопрос как на этой встрече оказался журналист Юрий Бутусов Корбан
ответил так: "Бутусов был все время со мной, месяцами находился у меня в
кабинете Днепропетровской администрации, поскольку там размещался основной
фронт. Все слышал и видел. Он был чуть ли не единственным журналистом, кто
регулярно освещал ряд военных действий с начала и до конца. Кроме того, он
сам неоднократно выезжал в зону АТО. Он лично проводил какие-то
тактические консультации командиров подразделений. Плюс он брал нашу
техническую помощь и несколько раз доставлял в зону АТО: телевизоры,
бинокли, планшеты и прочую техническую мелочь".

Чем закончилась история с Иловайском - всем известно: украинские войска
там попали в "котел". Правда, виновных (с украинской стороны) следствие
так и не обнаружило.

В том числе, не была расследована роль Бутусова в планировании операции,
которая закончилась самым тяжелым поражением украинских войск за все время
боевых действий на Донбассе.

