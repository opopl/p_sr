% vim: keymap=russian-jcukenwin
%%beginhead 
 
%%file 30_11_2021.fb.voloshyna_larysa.1.skandal_bloggery_golodomor.cmt
%%parent 30_11_2021.fb.voloshyna_larysa.1.skandal_bloggery_golodomor
 
%%url 
 
%%author_id 
%%date 
 
%%tags 
%%title 
 
%%endhead 
\subsubsection{Коментарі}

\begin{itemize} % {
\iusr{Sergiy Parhomenko}
ТРК Україна не далеко від них зайшло, теж розважальне шоу показувало

\iusr{Інна Назаренко}
Згодна. Теж промайнула думка, що зараз почнуть насаджувати ідею, що українське суспільство занадто вимогливе, у русскіх прощє.

\begin{itemize} % {
\iusr{Oleksandr Vyhovskyi}
\textbf{Інна Назаренко} та не прощє то нєдароси ментально расєянє

\iusr{Vladimir Pravdenko}
\textbf{Інна Назаренко} у русских они бы уже присели на пятёрку минимум

\iusr{Юрій Поплавський}
\textbf{Інна Назаренко} хай валять к рускім

\iusr{Інна Назаренко}
\textbf{Vladimir Pravdenko} але ж вони про це не знають))
\end{itemize} % }

\iusr{Ntina Ntoubrova}

Все одно це питання постане рано чи пізно. Крім того - а точно це не є
прихованою проблемою емоційної непричасності щодо жителів тих регіонів, яких
Голодомор не торкнувся? Бо ж не тільки східняки байдужі до УПА - і галичани
можуть бути прохолодні щодо Голодомору. Як-не-як співчуття та емпатія - з ними
людина не народжується, це соціально виховані якості особистості..

\begin{itemize} % {
\iusr{Лариса Волошина}
\textbf{Ntina Ntoubrova} 

Я думаю, що хохміть на поминках, жартувати про загиблих - це питання моралі, а
не регіону. У нас деморалізоване суспільство, чи і користається ворог.

\iusr{Ntina Ntoubrova}
\textbf{Лариса Волошина} 

Деморалізоване 100\%. Але там ніби не було жартів про загиблих. І справа в тому,
що так - ці блогери є нарцисічними відморозками. Але їхню аудиторію ми мусимо
знайти метод аби навчати тим, щоб вирощувати в них почуття емпатії,
співпричасності.

Якщо в них через скандал тепер з,являється почуття небайдужості до цих
історичних подій - це вже добре. Отож треба працювати на те, щоб ця
небайдужість перейшла у знак "+". Образливі слова та дорікання нічого не
дадуть.

Мені здається, це такий само механізм, як привчати наших пересічних
співвітчизників що роми або араби - це теж люди. (бо багато хто їх зневажає,
бачить нижчими за себе і ненавидить  @igg{fbicon.frown} )... Те ж саме про вмерлих людей - це не
тільки страшно дивитися і хочеться відвернутися від фотографій знурених тіл та
облич. Це також сльози і жаль і співчуття. ((

\iusr{Olga Glushenko}
\textbf{Ntina Ntoubrova} 

а чему их научила Революция Достоинства? Чему научила Революция депутатов от
Слуг, которые регулярно тусят в Трускавце? Они видят, что так можно, им это
показывают постоянно, что можно обсмеять все, их призывают относится к жизни
легко, весело и им плевать, что кто-то сейчас умирает, задыхаясь от ковида, как
когда-то умирал от голода. ведь и тогда кто-то в это время жрал в три горла и
веселился...


\iusr{Ntina Ntoubrova}
\textbf{Olga Glushenko} 

Пробачьте, але все те, що ви перераховуєте - пройшло повз них без уваги. Деякі,
що помолодше навіть про все це і не здогадуються, в них альтернативний
Інста-світ з опором на російське тло. ((

\iusr{Olga Glushenko}
\textbf{Ntina Ntoubrova} 

не могло это все пройти мимо них и не может. Они в соцсетях, информации там хоть
отбавляй, включай мозги и делай выводы! Все они знают, но демонстрируют то, чему
их учит власть:хапай все пока есть возможность и плюй на ограничения в
законах, в морали, чести и порядочности! Помните, перед выборами отовсюду
звучало: дайте дорогу молодым? И вот пришла молодая команда зеленского! и
что??? да, Вы правы, чему может научить эта молодая команда, которая впитала в
себя московскую тусу, с ее брезгливым, хамским отношением к жизни простых людей?
Теперь эту заразу власть распространяет у нас, через своих прикормленных
блогеров.


\iusr{Ntina Ntoubrova}
\textbf{Olga Glushenko} 

В російській частині соцмереж немає війни. Там є тільки те, що \enquote{україна
недолуга країна, якісь смішні претензії увесь час проти Росії висуває. Дурепки
вони. Краще бути з росіянами, у нас веселіше}.

\iusr{Olga Glushenko}
\textbf{Ntina Ntoubrova} 

я не писала, что там есть война или о войне. Наоборот, \enquote{гуляй, пока молодой} и \enquote{я
вам ничего не должен}, и эта молодая поросль так и живет. Но они или не
понимают, или не хотят понимать, что ими, их сознанием управляют умелые ручки.
Согласна с пани Ларисой.

\iusr{Ntina Ntoubrova}
\textbf{Olga Glushenko} Ну... Всі ми такі були у молодості. Але я підкреслюю - образливими словами нічого не діб,єшся.

\iusr{Olga Glushenko}
\textbf{Ntina Ntoubrova} надо говорить на понятном для оппонента языке, чтоб дошло @igg{fbicon.wink} 

\end{itemize} % }

\iusr{Олександр Михельсон}

нє, ну все може бути. хоча іноді гнилий банан - просто гнилий банан.

\iusr{Olga Glushenko}

Они устали от войны, от памяти жертв голодомора... они устали от всего, что
связано с историей Украины, не помнят даты, события... даже нет слов и сил, чтобы
комментировать это, просто противно...

\begin{itemize} % {
\iusr{Яків Миколайович}
\textbf{Olga Glushenko} 

Вони стомилися від свободи, нерозривно пов'язаної з відповідальністю за свою
долю та долю Держави. І мріють про доброго царя, який в них цю сваободу забере
і цю відповідальність з них зніме. Хай в плуг запрягає, хай голодом морить, хай
канчуками б'є - аби за них все вирішував! А поки того царя не отримають -
стогнатимуть і лаятимуть будь-яку владу - тим більше лаятимуть, чим більше
свободи та відповідальнгості отримають.


\iusr{Olga Glushenko}
\textbf{Яків Миколайович} 

какая свобода, если они в стаде и от стада зависят? они тусуются, смеются над
глупыми шутками, потому что в стаде надо смеяться, иначе будут смеяться над
тобой, они матеряться, как сапожники, потому что так модно сейчас, они хвастают кто
сколько срубил бабла и какую тачку купит. может, кто-то из них и знал, что в этот
день чтят память тех, кто не пережил голод, но как не пойти на тусу, стадо не
поймет и хозяин тусы обидется и выгонит из стада...

\iusr{Яків Миколайович}
\textbf{Olga Glushenko} 

Їх бісить відсутність зовнішніх обмежень. Саме звідси - зловживання своєю
свободою в формі порушення прав і свобод інших: вони ВИПРОШУЮТЬ по морді, щоб
хоч в такій формі отримати від інших обмеження, які нездатні встановити
самостійно.

\iusr{Olga Glushenko}
\textbf{Яків Миколайович} 

я не думаю, что они настолько мазохисты. Нет... они подражают власти, которой
плевать на законы, которая творит, что хочет. Власть заявляет: я власть, я имею
право, вот они и подражают ей, подражают своему потешному кумиру, и демонстрируют
свое право плевать на всех!


\iusr{Яків Миколайович}
\textbf{Olga Glushenko} 

Це не тільки НАШЕ явище. і навіть в нас воно не при ЗЕ з'явилося, і з ним не
піде.


\iusr{Olga Glushenko}
\textbf{Яків Миколайович} 

да, оно пришло из россии, из московии, от пыхатых \enquote{старших братьев}, перед которыми
их кумир падал на колени в экстазе, и сейчас у нас расцветает пышным цветом.
Вседозволенность идет власти, от высшей к низшей. Перед ними пример, как любой
кривляка может стать президентом, а подонок - депутатом. Рыба гниет с головы это
истина и от нее никуда не уйти!

\iusr{Оксана Ландо}
\textbf{Olga Glushenko} 

вот просто бесчт эти фразы «мы устали от войны карантина и тд» ! Такое
впечатление что они живут без света воды еды связи денег... ограничений нет ведь
никаких! Это люди которые провели на войну отцов детей братьев сестер-вот они
имеют право устать! А все остальные должны заткнуться про усталость! И даже
усталость это не повод жить такими зашоренными и ничего не знающими?! Ой, а я
не знала, ой а я не слышала-тупые отмазки, а ещё блогеры с миллионными
подписчиками-на что там подписываться???

\end{itemize} % }

\iusr{Тарас Коковський}
логічно, скидається на сплановану акцію

\iusr{Marimar Dobosh}
Я вже боюсь, що наступним кандидатом в президенти може бути якийсь такий блогер...

\iusr{Остап Стек}

Але мало хто звертає увагу на інше, що ця туса відбулася у Будинку вчених. Ось
вам і рівень вчених. Може хтось чув хоч якусь заяву вчених?

\begin{itemize} % {
\iusr{Ірина Сардига}
\textbf{Ostap Stek} є заява, сказали, що будуть більш ретельно підходити до відбору орендарів
\end{itemize} % }

\iusr{Яків Миколайович}
Поки не отримають новий голодомор - не порозумнішають.

\iusr{Олексій Гребенюк}
\textbf{Яків Миколайович} не порозумнішают.

\iusr{Оксана Лук'янчук}

І нікому ніде з 'блогерів-мільйонників'' не тиснуло глянути у календар? Ніхто
не подумав про недоречність у такий день святкувати? ''голодна туса'' теж так
собі співпало? Ну так, звичайно що вірю у ''випадковий збіг''

\end{itemize} % }
