% vim: keymap=russian-jcukenwin
%%beginhead 
 
%%file 05_12_2021.fb.fb_group.story_kiev_ua.1.priorka
%%parent 05_12_2021
 
%%url https://www.facebook.com/groups/story.kiev.ua/posts/1812346135628846
 
%%author_id fb_group.story_kiev_ua,olejnikov_maksim
%%date 
 
%%tags gorod,istoria,kiev,priorka
%%title Про Пріорку. І її окрасу - Покровську церкву на вул. Мостицькій
 
%%endhead 
 
\subsection{Про Пріорку. І її окрасу - Покровську церкву на вул. Мостицькій}
\label{sec:05_12_2021.fb.fb_group.story_kiev_ua.1.priorka}
 
\Purl{https://www.facebook.com/groups/story.kiev.ua/posts/1812346135628846}
\ifcmt
 author_begin
   author_id fb_group.story_kiev_ua,olejnikov_maksim
 author_end
\fi

В одному з коментарів до мого попереднього допису дорікнули, що я пишу про
місцевість, де сам ніколи не жив.  @igg{fbicon.face.savoring.food}  Тож зараз – про місцевість, де таки
довелося жити, хоч і не дуже довго.

Про Пріорку. І її окрасу - Покровську церкву на вул. Мостицькій.  

\begin{multicols}{2}
\ii{05_12_2021.fb.fb_group.story_kiev_ua.1.priorka.pic.1}
\ii{05_12_2021.fb.fb_group.story_kiev_ua.1.priorka.pic.2}
\ii{05_12_2021.fb.fb_group.story_kiev_ua.1.priorka.pic.3}
\ii{05_12_2021.fb.fb_group.story_kiev_ua.1.priorka.pic.4}
\ii{05_12_2021.fb.fb_group.story_kiev_ua.1.priorka.pic.5}
\ii{05_12_2021.fb.fb_group.story_kiev_ua.1.priorka.pic.6}
\ii{05_12_2021.fb.fb_group.story_kiev_ua.1.priorka.pic.7}
\ii{05_12_2021.fb.fb_group.story_kiev_ua.1.priorka.pic.8}
\ii{05_12_2021.fb.fb_group.story_kiev_ua.1.priorka.pic.9}
\end{multicols}

Назва цієї місцевості походить від слова «пріор» - так називався настоятель
католицького домініканського монастиря, чия резиденція в середині ХVIIст.
знаходилась там, а навколо неї – біля 20 сільських хат. Звідки взялися на
приміських київських землях домініканці ? - ще на початку ХІІІ століття у Києві
існував католицький монастир святої Марії, заснований ірландськими місіонерами.
Пізніше він був заселений польськими домініканцями, їх місіонер Яцек Одровонж у
1222-1226рр. проповідував у Києві католицизм і заснував поселення у цій
місцевості, яке належало монастирю і за його ім’ям називалося Яцківка. У 1630-х
рр. католицькому домініканському монастирю належала досить велика місцевість
від річки Сирець до селища Берковець включно, до річок Ірпінь і Горенка на межі
із тодішньою Вишгородською землею. Тож назву місцевості, якою володів монастир,
дав церковний чин його настоятеля, тобто пріора.

\ii{05_12_2021.fb.fb_group.story_kiev_ua.1.priorka.pic.10}

Після повстання під проводом Богдана Хмельницького ці землі у католиків
відібрали і передали у 1659р. Братському монастирю на Подолі, а у 1701р. указом
гетьмана Івана Мазепи Пріорка була передана Київському магістрату (сучасною
мовою – перейшла в комунальну власність). 

На середину XVIII ст. село Пріорка локалізувалося на перетині сучасних вулиць
Мостицької і Вишгородської. Через Пріорку проходила дорога на Вишгород і
Межигірський монастир вздовж сучасних Кирилівської і Вишгородської вулиць.
Також на Пріорці починалася дорога, що вела на Гостомель, проходячи поряд із
Синім озером, що зараз на західній околиці Виноградаря. Мешканці Пріорки
займалися городництвом, садівництвом, кустарними промислами, тримали винокурні. 

\ii{05_12_2021.fb.fb_group.story_kiev_ua.1.priorka.pic.11}

Приміська слобода Пріорка у другій половині ХІХ ст. перетворилась на дачну
місцевість Києва і у 1880р. офіційно увійшла до складу міста. Станом на
1894р. Пріорка входила до Плоської поліцейської дільниці Києва і
простягалася територіями сучасних Вишгородського і Мостицького масивів
уздовж Вишгородської та сучасних Автозаводської і Мостицької вулиць. 

Ще у ХVIIIст. на Пріорці стояла дерев’яна церква святих Георгія і Дмитра. У
1791р. її розібрали і у 1795р. на старому кам’яному фундаменті збудували
дерев’яну церкву Покрова Пресвятої Богородиці. У 1803р. до неї прибудували
окрему дерев’яну дзвіницю з шатровою покрівлею. Десь у цей же час Пріорка –
приміська слобода - стає дачною місцевістю Києва, хоча до меж міста вона
офіційно увійшла лише у 1880 році. 

\ii{05_12_2021.fb.fb_group.story_kiev_ua.1.priorka.pic.12}

Тож на початку ХХст. Пріорка вже була міською окраїною, її населення значно
виросло. Виникла потреба у новому, більшому за розмірами храмі. На той час
стан старої столітньої церкви на Мостицькій вулиці був вже фактично
аварійним, через що богослужіння довелося проводити у дзвіниці. Тому віряни
у 1902 році стару церкву знову розібрали і заклали нову. При будівництві
використали максимум з того, що залишилось від старої церкви. З розібраного
храму взяли дзвони, іконостас, жертовники і все церковне начиння. 

\begin{multicols}{2}
\ii{05_12_2021.fb.fb_group.story_kiev_ua.1.priorka.pic.13}
\ii{05_12_2021.fb.fb_group.story_kiev_ua.1.priorka.pic.14}
\ii{05_12_2021.fb.fb_group.story_kiev_ua.1.priorka.pic.15}
\end{multicols}


На місці
престолу знесеної дерев’яної Покровської церкви спорудили невеликий
пам’ятник, обнесений залізною огорожею. З матеріалу розібраної церкви
побудували два будинки на території храму, один з яких зберігся до наших
днів. Ніякої механізації тоді не було, тому під час будівництва на високий
пагорб цеглу і воду підносили вручну добровільні помічники. Так протягом
чотирьох років «усім миром» зводили великий храм. 

\ii{05_12_2021.fb.fb_group.story_kiev_ua.1.priorka.pic.16}

У 1906р. будівництво завершили і нову церкву освятили. У ті часи більшість
церков будувалися не за казенний рахунок, а за пожертви. Загальна сума витрат
на будівництво склала 57 тисяч рублів сріблом, які й були пожертвами пріорських
віруючих, духовних осіб і київського купецтва. 

Цікаво, що до будівництва храму «приклали руку» відразу троє відомих київських
архітекторів. Є. Єрмаков проектував споруду, О. Вербицький розраховував
стійкість будівлі, М.Казанський керував будівництвом. Храм Покрови Пресвятої
Богородиці вийшов у псевдоросійському стилі, з кокошниками і напівкруглими
пілястрами, 5 маківками і дзвіницею. Матеріалом для будівництва була славетна
київська світло-жовта цегла, випалена на двох цегляних заводах на Сирці. 

За радянської влади парафію церкви зареєстрували в грудні 1920р. У той час вона
мала адресу Межигірський провулок, 16. Службу тоді там правила Українська
автокефальна православна церква (УАПЦ). Потім деякий час в церкві одночасно
співіснували парафії УАПЦ і традиційної православної громади, а наприкінці
1920-х храм повністю відійшов до старославянської парафії. 

Цікаво також, що храм зумів пережити «більшовицьку реконструкцію» у 1930-х,
коли були зруйновані і розграбовані 145 київських церков. Він був зачинений
порівняно недовго, лише три роки, з лютого 1938 по вересень 1941 року, коли в
ньому знаходилась овочева база. З приходом німців у вересні 1941р. церква знову
відкрилась (як і решта київських церков), і після цього вже не зачинялась до
сьогоднішнього дня. Однак церковне начиння було втрачене, за винятком дверей
від іконостасу з іконою Архангела Михаїла (вона стоїть у правому бічному
вівтарі храму досі). А новий іконостас, що зберігся до сьогодні, був
споруджений у 1944 році зусиллями тодішнього настоятеля храму Тимофія Коваля.
Іконостас розписав відомий український художник Іван Їжакевич.  

У 1945 році на гроші вірян у вівтарі встановили новий жертовник, а у 1953 році
в храмі зробили капітальний ремонт.

У 1980-х для Пріорки і її храму почалось нове життя. Приватну забудову зносили,
місцевість забудовували багатоповерхівками, через дорогу від храму на
Мостицькій відкрили великий пологовий будинок. Церква Покрови Пресвятої
Богородиці, або Покровська церква – єдина в тому районі (найближча церква – на
Вишгородській, аж за кінотеатром ім. Шевченка), тому в храмі досить багатолюдно,
особливо на церковні свята. Крім місцевих мешканців сюди також нерідко
приїжджають люди, які після знесення одноповерхової Пріорки отримали квартири в
інших районах. До 100-річчя храму (у 2006р.) позолотили раніше зелені куполи, і
тепер їх видно із значної відстані.
