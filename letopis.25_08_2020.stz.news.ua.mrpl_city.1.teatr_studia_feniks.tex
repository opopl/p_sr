% vim: keymap=russian-jcukenwin
%%beginhead 
 
%%file 25_08_2020.stz.news.ua.mrpl_city.1.teatr_studia_feniks
%%parent 25_08_2020
 
%%url https://mrpl.city/blogs/view/teatr-studiya-feniks
 
%%author_id demidko_olga.mariupol,news.ua.mrpl_city
%%date 
 
%%tags 
%%title Театр-студія "Фенікс"
 
%%endhead 
 
\subsection{Театр-студія \enquote{Фенікс}}
\label{sec:25_08_2020.stz.news.ua.mrpl_city.1.teatr_studia_feniks}
 
\Purl{https://mrpl.city/blogs/view/teatr-studiya-feniks}
\ifcmt
 author_begin
   author_id demidko_olga.mariupol,news.ua.mrpl_city
 author_end
\fi

Після створення рубрики \emph{\enquote{Маріуполь театральний}} на Маріупольському телебаченні
я почала частіше спілкуватися з театральними колективами міста. Нещодавно
відкрила для себе ще один самодіяльний театр, який, як виявилося, знала дуже
давно. Це театр-студія \emph{\textbf{\enquote{Фенікс}}}. Його назва говорить сама за себе, адже театр
отримав друге життя. Колектив продовжує історію і традиції \emph{\textbf{Народного
драматичного театру \enquote{Азовсталі}}}, створеного 89 років тому, але через низку
обставин був змушений почати все спочатку...

\ii{25_08_2020.stz.news.ua.mrpl_city.1.teatr_studia_feniks.pic.1}

Втім насправді його історія продовжується. А починалось все в далекому 1931
році, коли в Маріуполі було створено самодіяльний театр. Нова сторінка творчої
біографії колективу розпочалась у 1956 році з приходом досвідченого режисера і
наставника \emph{\textbf{Леоніда Олександровича Бессарабова}}, який вдумливо підходив до
створення кожної нової вистави. Саме завдяки його зусиллям у 1961 році театр
отримав звання \enquote{Народного}. Колектив працював у ПК \enquote{Азовсталь} (нині – ПК
\enquote{Молодіжний}). У 2008 році він переїжджає до ПК \enquote{Чайка}. На долю театру випало
чимало випробувань, але найважче – актори пережили в 2019 році, коли вони були
вимушені покинути приміщення ПК \enquote{Чайка}, через що не встигли підтвердити звання
\enquote{Народного} театру. Тоді в місті багато журналістів та театралів вирішили, що
колектив розпався і театр перестав існувати. Звісно, багато акторів обрали для
себе новий шлях та продовжили театральну діяльність в інших колективах. Але
найбільш віддані залишалися, попри всі негаразди. Актори вирішили, що лише
разом вони зможуть відновити почесне звання Народного театру і зуміли
відродитися, ніби Фенікс, та повернулися до театральної справи з новими силами.
Втім відродилися вони швидко. Того ж 2019 року створили театр-студію. На
відміну від колишнього керівництва ПК \enquote{Чайка}, яке змусило театр покинути
приміщення, нова директорка – \emph{\textbf{Наталя Лозова}} – дозволила театру на засадах
громадського об'єднання ставити вистави та репетирувати в стінах палацу
культури. Підтримує театр і заступниця директора ПК \enquote{Чайка} \emph{\textbf{Оксана Гальченко}}.
Шкода тільки, що минулого не виправити і тепер новостворений театр-студія
\enquote{Фенікс} знову намагатиметься повернути звання \enquote{Народного} театру.

\ii{25_08_2020.stz.news.ua.mrpl_city.1.teatr_studia_feniks.pic.2}

Режисер \emph{\textbf{Ігор Курашко}} почав грати у Народному драматичному театрі Азовсталі у
1981 році, що посприяло його подальшому становленню як актора. Він з теплом
згадує  мудрого Леоніда Олександровича, наставництво якого допомогло Ігорю
потрапити до професійного театру. Насправді така практика була доволі
поширеною. Головний режисер професійного драматичного театру нашого міста
\emph{\textbf{Олександр Кадирович Утеганов}} часто приходив до Народного театру і обирав
найбільш талановитих і яскравих акторів. Так сталося і з Ігорем, який був
запрошений у 1988 році попрацювати у професійному театрі. Він і сьогодні
продовжує грати у Донецькому академічному обласному драматичному театрі (м.
Маріуполь). До речі, його дружина, \textbf{\emph{Тетяна Курашко}}, працює в театрі помічницею
режисера. А після роботи продовжує грати в театрі, де режисером є її чоловік.

\ii{25_08_2020.stz.news.ua.mrpl_city.1.teatr_studia_feniks.pic.3}

Театр-студія \enquote{Фенікс} складається з 12 учасників (4 чоловіка і 8 жінок). До
репертуару театру входять постановки за класичними творами і сучасні п'єси,
комедії, драми, казки. Як зазначила актриса театру, \emph{\textbf{Людмила Волобуєва}}, для неї
цей театр – другий дім і справжня віддушина. Цікаво, що в колективі декілька
театральних сімей, адже актори настільки віддані театру, що це не може не
надихати їхніх рідних. Зокрема, Людмила привела з собою свою доньку – Юлію.
Оскільки в колективі більше жінок, режисер віддає перевагу виставам більш
ліричним і душевним, які присвячені саме жінкам (\emph{\enquote{Звідки беруться діти}, \enquote{У
війни не жіноче ім'я}, \enquote{Обережно – жінки!}}). Найбільш тепло всі актори згадують
репетиції і гру у виставі \emph{\enquote{У війни не жіноче ім'я}}, що поставлена за
документально-нарисовою книгою білоруської письменниці, лауреата Нобелівської
премії з літератури 2015 року Світлани Алексієвич. У цій книзі зібрані
розповіді жінок, що брали участь у Другій світовій війні. Актриси театру-студії
\enquote{Фенікс}, читаючи книгу, не могли стримати емоцій. Це та вистава, коли акторам
потрібно не просто зіграти, а прожити кожен епізод на тому зламі, межі людських
сил, коли загострені і нерви, і почуття, і коли ти знаєш, що \enquote{завтра} може не
настати ніколи. Побачивши цю виставу, маріупольські ветерани звернулися до
місцевих депутатів з проханням не залишати цей театр без власного приміщення і
сприяти подальшій діяльності акторів...

Планів у театру-студії \enquote{Фенікс} ще дуже багато. Головне, щоб карантин не зірвав
прем'єри колективу і зустріч з улюбленим глядачем все ж відбулася.

\ii{25_08_2020.stz.news.ua.mrpl_city.1.teatr_studia_feniks.pic.4}
