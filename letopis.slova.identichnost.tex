% vim: keymap=russian-jcukenwin
%%beginhead 
 
%%file slova.identichnost
%%parent slova
 
%%url 
 
%%author 
%%author_id 
%%author_url 
 
%%tags 
%%title 
 
%%endhead 
\chapter{Идентичность}
\label{sec:slova.identichnost}

%%%cit
%%%cit_head
%%%cit_pic
%%%cit_text
А ведь это еще не все. Оказывается, в понятие \enquote{патриотического воспитания}
входят еще и формирование поддержки вступления в НАТО, и уважительное отношение
к \enquote{героям борьбы украинского народа за обретение независимости}.  Но самый
главный пункт в этом перечне – последний: Кабмин предлагает привлекать на
работу в органы государственной власти и местного самоуправления только граждан
со \enquote{сформированной национальной \emph{идентичностью}}. Проще говоря, если у тебя
украинский не родной, если ты не поддерживаешь вступление в НАТО и не
относишься уважительно к Бандере и Шухевичу, дорога в органы власти тебе
заказана
%%%cit_comment
%%%cit_title
\citTitle{Миллионы на украинскую идентичность, зрада на Евро-2020, планы Путина по Украине}, 
, strana.ua, 01.07.2021
%%%endcit

%%%cit
%%%cit_head
%%%cit_pic
%%%cit_text
От урока истории Юлия Мендель перешла к урокам языка.  Начала издалека – о том,
что язык в Украине стал символом борьбы против Москвы и способом получения
украинской \emph{идентичности}. Негодяи, сталкивающие украинцев, – это
пророссийские политики и националисты: \enquote{Використовуючи нашу травму
історичних заборон говорити рідною українською і російську пропаганду, спочатку
нас роз’єднували, я навіть не скажу проросійські політики, а ті, хто не міг
відірватися від вимені Кремля і не мислив Україну незалежною.  Як годиться, з
другого боку підпряглися націоналісти. На цьому гріх було не заробити політичні
бонуси}.  Мендель при этом уверена, что языки защищать вообще не нужно. Потому
что украинцы, мол, вообще друг друга не понимают на почве взаимной неприязни
%%%cit_comment
%%%cit_title
\citTitle{\enquote{Я їб...в Гітлера в зад}. Что пишет Юлия Мендель в своей книге о двух годах у Зеленского}, 
Екатерина Терехова, strana.ua, 07.07.2021
%%%endcit

%%%cit
%%%cit_head
%%%cit_pic
%%%cit_text
Не нужно большого ума, чтобы признать: \emph{идентичность} многомиллионной
страны, живущей в XXI веке в центре Европы, говорящей на разных языках,
исповедующей разные религии, чествующей разных героев и по-разному оценивающей
исторические события нельзя строить на принципах, близких лишь части населения
— НЕВОЗМОЖНО.  Второй изъян – разрыв интересов украинской власти и народа.  В.
Путин в своей статье напоминает о договоре об учреждении Советского Союза 1922
года, согласно которому «республики – учредители Союза, после того как они сами
же аннулировали Договор 1922 года, должны вернуться в те границы, в которых они
вступили в состав Союза». Приводя историю создания Советского Союза, он
говорит: «С чем пришли, с тем и уходят». А все остальное – предмет
договоренностей и дискуссий. Кстати, после обретения независимости Украины
такие договоренности были достигнуты
%%%cit_comment
%%%cit_title
\citTitle{О будущем украинского и русского народов}, 
Виктор Медведчук, strana.ua, 15.07.2021
%%%endcit

%%%cit
%%%cit_head
%%%cit_pic
%%%cit_text
Звичайно, можна по різному інтепретувати соціологічні дані. Вони промовисті,
але це – лише цифри. Хтось поспішить звинуватити в усьому Президента Володимир
Зеленський, хтось "хохлів-малоросів", які за нього проголосували. Для мене
очевидно, що ці цифри – вирок політиці української \emph{ідентичності}, яку
реалізує українська держава. В якій пафос модернізації підмінив культ
тожсамості.  Причому тожсамості не оригінальної, позитивної, а вторинної,
ресентиментної та ксенофобської.  Замість розбудови привабливого та
самодостатнього українського світу, ми часто-густо намагались побудувати
маленьку кальку "русского мира" під жовто-блакитним прапором. Так само
корумповану, заздрісну та злу, яку тримає до купи тільки наявність
екзистенційного ворога. Тільки якщо в уяві більшості росіян таким ворогом є
Америка та колективний Захід, то для українців – Росія.  І тут раптом
виявилось, що 40\% українців (і навіть 44\% тих, хто народились і вирісли у
незалежній Україні) не хочуть жити в парадигмі анти-Росії
%%%cit_comment
%%%cit_title
\citTitle{Фіаско етнічної українізації}, 
Генадій Друзенко, analytics.hvylya.net, 28.07.2021
%%%endcit

%%%cit
%%%cit_head
%%%cit_pic
%%%cit_text
Тішити себе такими ілюзіями або свідомо бути оббріхувачем можна тільки тоді,
коли не знаєш історії, фальшуєш її, не волієш знати правду, догоджаючи
чужинцям, присипляючи народ, роблячи його безвольним і заколиханим байками про
\enquote{єдинородство}.  Бо коли українські землі захоплено теперішніми
кочівниками-москвинами, восьмий рік продовжується війна і людиновбивство, Росія
намагається всіма способами знищити нашу \emph{ідентичність}, державність, а нам
продовжують утовкмачувати колонізаторську ідеологему про \enquote{єдиний народ}, то
яким же (за висловом того ж П. Штепи) бездонним ідіотизмом є бараняча віра
українських \enquote{демократів}: мовляв, московський народ є невин­ний, мовляв, він
також страждає під тиранією
%%%cit_comment
%%%cit_title
\citTitle{Правда в Рідному Слові}, Георгій Філіпчук, slovoprosvity.org, 12.07.2021
%%%endcit

