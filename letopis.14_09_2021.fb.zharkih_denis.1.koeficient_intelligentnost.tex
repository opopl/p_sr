% vim: keymap=russian-jcukenwin
%%beginhead 
 
%%file 14_09_2021.fb.zharkih_denis.1.koeficient_intelligentnost
%%parent 14_09_2021
 
%%url https://www.facebook.com/permalink.php?story_fbid=3077173702496026&id=100006102787780
 
%%author_id zharkih_denis
%%date 
 
%%tags bezumie,chelovek,intelligentnost',knigi,nacizm,obschestvo,patriotizm
%%title Коэффициент интеллигентности
 
%%endhead 
 
\subsection{Коэффициент интеллигентности}
\label{sec:14_09_2021.fb.zharkih_denis.1.koeficient_intelligentnost}
 
\Purl{https://www.facebook.com/permalink.php?story_fbid=3077173702496026&id=100006102787780}
\ifcmt
 author_begin
   author_id zharkih_denis
 author_end
\fi

Коэффициент интеллигентности

В свое время будучи студентом психологом я, начитавшись всяких тестовых
методик, в шутку придумал коэффициент интеллигентности. Формула была совершено
поста:

КоИн (Коэффициент интеллигентности)= n-1

n - количество мыслей, приходящих в голову, по одному поводу

Основная моя мысль была тогда в том, что интеллигентный, думающий, да просто
воспитанный человек не позволит одной мысли всецело завладеть собой. Сомнение
это привычка мудрых. А вдруг я не прав? А вдруг не все так просто? А вдруг я
поступаю глупо? 

Конечно, большое количество мыслей не должно парализовать волю и желание
творить и действовать, но это уже другая грань личности. В конце концов,
отсутствие воли приводит к бездействию или мысли, что думать и действовать себе
дороже, что опять описывается формулой выше (мысль или одна, или ноль). 

Почему я это вспомнил? На это навели видео доблестных патриотов, которые
терроризируют обывателей. Патриотам и в голову не приходит, что они могут
ошибаться, оказаться неправыми, что они показывают не предательство и пятую
колонну в виде инакомыслящих, а собственную глупость и низость. 

Ведь фашисты, когда сжигали книги, убивали "не те национальности", вешали
непокорных тоже ни минуты не сомневались в своей правоте. В своем мире, в своих
понятиях они были правы, и им и невдомек было, что есть иная правота, иные
ценности, иная справедливость. Им казалось, что Рейх будет всегда, что они
пришли всерьез и надолго, они думали, что таким образом укрепляют государство,
а на самом деле они его разрушали. Они ощущали себя хозяевами страны, а на деле
были ее палачами.  

Вот потому и с удовольствием позировали на фоне виселиц, горящих книг,
расстрелянных жертв. Нынешние патриоты считают себя героями, когда кучей
нападают на интеллигентов, терроризируют обывателей, пишут доносы и угрозы. И
им в голову не приходит, а вдруг все изменится? А в друг сегодняшние сильные
станут слабыми и будут осуждены и прокляты? В нашей стране нет ни одного
периода истории, который бы не был на определенном этапе проклят, так почему
нынешние времена исключение?

Я вот далек от мысли, что все понесут заслуженное наказание, кто-то переобуется
в прыжке, а некоторые из них возглавят процесс казни бывший сильных, но кару
понесут самые глупые. Вот те, кто сейчас показывает, насколько они крутые и
лояльны этой власти, завтра окажутся главными жертвами следующей власти.

Полицаи тоже были лояльны новой власти, думали, она пришла навсегда, но так не
оказалось.  Нынешние полицаи думают, что этот режим навсегда, он он уже явно
прогнил, и не обрушился только потому, что пока некому толкнуть, как следует. 

Война властью проиграна, и мудрым людям нужно думать, что будет после того, как
эта власть рухнет. Совершенно не обязательно, что будет лучше. Будет
по-другому. Все течет, все меняется, а кто этого не понимает, того уносит
поток. Вот наци жгли книги, а они не сгорели. Исчезли эти наци, но не книги.

2020

\ii{14_09_2021.fb.zharkih_denis.1.koeficient_intelligentnost.cmt}
