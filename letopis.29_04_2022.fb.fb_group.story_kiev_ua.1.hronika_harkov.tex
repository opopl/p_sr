% vim: keymap=russian-jcukenwin
%%beginhead 
 
%%file 29_04_2022.fb.fb_group.story_kiev_ua.1.hronika_harkov
%%parent 29_04_2022
 
%%url https://www.facebook.com/groups/story.kiev.ua/posts/1913099805553478
 
%%author_id fb_group.story_kiev_ua,stepanov_farid
%%date 
 
%%tags 
%%title ХРОНІКИ НЕОГОЛОШЕНОЇ ВІЙНИ. День шістдесят четвертий. Перша столиця
 
%%endhead 
 
\subsection{ХРОНІКИ НЕОГОЛОШЕНОЇ ВІЙНИ. День шістдесят четвертий. Перша столиця}
\label{sec:29_04_2022.fb.fb_group.story_kiev_ua.1.hronika_harkov}
 
\Purl{https://www.facebook.com/groups/story.kiev.ua/posts/1913099805553478}
\ifcmt
 author_begin
   author_id fb_group.story_kiev_ua,stepanov_farid
 author_end
\fi

%\ii{29_04_2022.fb.fb_group.story_kiev_ua.1.hronika_harkov.eng}

ХРОНІКИ НЕОГОЛОШЕНОЇ ВІЙНИ 

День шістдесят четвертий. Перша столиця. 

Харків - ще одна рана, що не відпускає своїм болем мене ні на хвилину. 

\ii{29_04_2022.fb.fb_group.story_kiev_ua.1.hronika_harkov.pic.1}

Харків, у нас з тобою було майже два роки. Протягом яких було різне, але ці два
роки - на все життя. 

\obeycr
\noindent 
Коли настане день,
Закінчиться війна,
Там загубив себе,
Побачив аж до дна.
\restorecr

Харків, одного разу ти став для мене справжнім откровенням. Вперше приїхавши
до тебе, я очікував побачити індустріального монстра, а неочікуванно зустрів
місто-парк.

\ii{29_04_2022.fb.fb_group.story_kiev_ua.1.hronika_harkov.pic.2}

\obeycr
\noindent 
І от моя душа
Складає зброю вниз,
Невже таки вона
Так хоче теплих сліз?
\restorecr

\ii{29_04_2022.fb.fb_group.story_kiev_ua.1.hronika_harkov.pic.3}

Харків, це не була любов з першого погляду. Сумська, Дарвіна, Пушкінська... я
відкривав тебе для себе крок за кроком. Площа Свободи, Держпром, Університет
Каразіна... ти відкривав себе мені кожного дня, кожного вечора. У всій своїй
красі. Зачаровував та надихав. 

\obeycr
\noindent
Обійми мене, обійми мене, обійми. 
І більше так не відпускай.
Обійми мене, обійми мене, обійми
Твоя весна прийде нехай.
\restorecr

Харків, так, прийде твоя весна. А потім - ще одна і ще. І жоден ворог не
поставить тебе на коліна, жоден не підкорить. 

Харків, так, у нас все ще буде. Бо це - на все життя. 

PS:

Святослав Вакарчук зняв кліп \enquote{Обійми} спеціально для Венеційської бієнале 2022.

Зйомка відбулася 14 квітня у Харкові, у дворі Палацу праці, що був зруйнований
російськими бомбардуваннями. Будівля, зведена у 1916 році, вистояла дві Світові
війни, проте зазнала значних пошкоджень під час російського вторгнення в
Україну у 2022 році.

Серед учасників струнного квартету, що допомагав Вакарчуку у запису пісні -
харківський музикант, віолончеліст Денис Караченцев, що відомий своїми
музичними перформансами серед харківських руїн. 

PPS:

Харків. 

Фото: Степанов Фарід, 2017 рік.
