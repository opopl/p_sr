% vim: keymap=russian-jcukenwin
%%beginhead 
 
%%file 25_01_2022.stz.news.lnr.lug_info.2.student_goda
%%parent 25_01_2022
 
%%url https://lug-info.com/news/nagrazhdenie-pobeditelej-konkursa-student-goda-sostoyalos-v-luganske
 
%%author_id news.lnr.lug_info
%%date 
 
%%tags donbass,konkurs,lnr,lugansk,obrazovanie,studenty
%%title Награждение победителей конкурса "Студент года" состоялось в Луганске
 
%%endhead 
 
\subsection{Награждение победителей конкурса \enquote{Студент года} состоялось в Луганске}
\label{sec:25_01_2022.stz.news.lnr.lug_info.2.student_goda}
 
\Purl{https://lug-info.com/news/nagrazhdenie-pobeditelej-konkursa-student-goda-sostoyalos-v-luganske}
\ifcmt
 author_begin
   author_id news.lnr.lug_info
 author_end
\fi

Церемония награждения победителей конкурса \enquote{Студент года} состоялась в
Луганской государственной академии культуры и искусств (ЛГАКИ) имени Михаила
Матусовского. Об этом с места события передает корреспондент ЛИЦ.

Мероприятие было посвящено Дню студента и Татьяниному дню.

\ii{25_01_2022.stz.news.lnr.lug_info.2.student_goda.pic.1}

Конкурс проводился Министерством образования и науки ЛНР, Центром развития
детского и молодежного движения и общественной организацией \enquote{Молодая гвардия}.
Отборочный тур прошел во всех семи вузах Республики, после чего анкеты лучших
студентов были направлены в адрес оргкомитета конкурса.

\ii{25_01_2022.stz.news.lnr.lug_info.2.student_goda.pic.2}

Победителем конкурса \enquote{Студент года} стала студентка Луганского государственного
медицинского университета имени Святителя Луки Надежда Мосягина.

\ii{25_01_2022.stz.news.lnr.lug_info.2.student_goda.pic.3}

Лучшими в номинациях признаны: \enquote{Общественный деятель года} - студентка
Донбасского государственного технического института Татьяна Околелова,
\enquote{Интеллектуал года} - студентка Луганского государственного университета имени
Владимира Даля Ксения Олейникова, \enquote{Волонтер года} - студентка Луганского
государственного аграрного университета Екатерина Живодробова, \enquote{Спортсмен года}
- студент Луганской академии внутренних дел имени Эдуарда Дидоренко Олег
Плехов, \enquote{Творческая личность года} - студентка ЛГАКИ Валерия Меркулова,
\enquote{Председатель студенческого самоуправления года} - студентка Луганского
государственного педагогического университета София Левова.   

Участников конкурса поздравил председатель Народного Совета ЛНР Денис
Мирошниченко.

\enquote{Студенчество - это настолько романтичный, непредсказуемый период в жизни, что
никогда не знаешь, что будет дальше. Друзья, только вперед}, - обратился он к
студентам.

Первый заместитель министра образования и науки ЛНР Ольга Долженко рассказала,
что инициатива проведения такого конкурса принадлежит молодежи.

\enquote{Наши студенты самостоятельны, они сами могут организовывать и разнообразить
свой досуг. Мы уже два года работаем в разных режимах: очно-заочно,
дистанционно. И ребята также предложили, чтобы сегодня было не просто
награждение, а именно вузы представили свои визитки. Видно, что молодежь
соскучилась за этим, им хочется показать свой вуз, прославить его}, - сказала
она.

Депутат Народного Совета ЛНР, народный артист Украины и ЛНР Михаил Голубович
поблагодарил студенческую молодежь за активную жизненную позицию и созидание.

\enquote{Спасибо вам за уважение и любовь к луганской земле. Спасибо, что вы учитесь
здесь, что думаете о будущем, несмотря на сложную ситуацию, в которой мы
оказались. Вы все-таки грызете гранит науки, и вы будете делать сильное,
крепкое государство}, - отметил он.

Церемонии награждения предшествовали спортивно-развлекательные соревнования по
боулингу, мегаболу, пинг-понгу, лазертагу, в которых приняли участие студенты
всех высших учебных заведений ЛНР.  

\enquote{Сегодняшнее мероприятие организовано для того, чтобы студенты могли ощутить,
что сегодня их праздник, могли повеселиться, наполниться новыми эмоциями и
достигать новых высот}, - сказал председатель общественной организации \enquote{Молодая
гвардия} Даниил Степанков.

Татьянин день – памятная дата в России, а также день в православном и народном
месяцеслове. Название дня произошло от имени раннехристианской мученицы Татьяны
Римской, память которой совершается в православной церкви 25 января. После
подписания в 1755 году императрицей Елизаветой Петровной указа об учреждении
Московского университета Татьянин день стал праздноваться сначала как день
рождения университета, а позднее и как праздник российского студенчества.

День студента установлен в Республике указом главы ЛНР и отмечается ежегодно 25
января.
