% vim: keymap=russian-jcukenwin
%%beginhead 
 
%%file slova.rossia
%%parent slova
 
%%url 
 
%%author 
%%author_id 
%%author_url 
 
%%tags 
%%title 
 
%%endhead 
\chapter{Россия}
\label{sec:slova.rossia}

%%%cit
%%%cit_pic
\ifcmt
  pic https://avatars.mds.yandex.net/get-zen_doc/4452915/pub_60c42adae72f9332f715ee29_60c4310b7c472712031a49e5/scale_1200
  caption Русский феникс. Изображение из открытых источников.
\fi
%%%cit_text
\begin{itemize}
\item Возникнув в 9 веке, к 10 веку \emph{Россия} стала величайшим, после Византии, государством Европы.
\item В 13 веке \emph{Россия} была почти уничтожена.
\item В 16 веке \emph{Россия} стала величайшим, после Испании и Турции, государством
Европы и Азии, гораздо более обширным и могучим, чем к моменту краха.
\item В начале 17 века \emph{Россия} была почти уничтожена.
\item В 18 веке \emph{Россия} стала величайшим, после Британии, государством мира, гораздо более обширным и могучим чем к моменту краха.
\item В начале 20 века \emph{Россия} была почти уничтожена.
\item В середине 20 века \emph{Россия} стала величайшим, наряду с США, государством
мира, гораздо более обширным и могучим чем к моменту краха.
В конце 20 века \emph{Россия} была почти уничтожена.
\item Во 2-й четверти 21 века \emph{Россия} станет величайшим, наряду с США (?) и Китаем,
государством мира, а, если всё будет нормально, во 2-й половине 21 века просто
величайшим государством мира, без всяких \enquote{рядов}, гораздо более
обширным и могучим чем когда-либо.
\item И то, что всё именно так и будет, доказывает наша многовековая привычка каждый
раз, как Феникс, возрождаться из пепла и становиться с каждым новым
возрождением больше и сильнее прежнего; доказывает сие и то, что эта привычка
оправдывается на наших глазах. \emph{Русский Феникс} снова расправляет крылья.
\end{itemize}
%%%cit_title
\citTitle{Что должен знать каждый русский}, 
Школа Экзорцистов, zen.yandex.ru, 12.06.2021
%%%endcit


Новости Крымнаша. Предатели в Крыму пришли к выводу, что \emph{россияне} — хуже
глистов, obozrevatel.com, 24.05.2021

Луганский академический театр кукол совместно с коллегами из \emph{РФ} и ДНР
провел онлайн-выставку рисунков юных зрителей, посвященную Дню славянской
письменности и культуры, lug-info.com, 24.05.2021

\enquote{Луганский академический театр кукол совместно с Брянским областным
театром кукол и Горловским городским театром кукол организовал онлайн-выставку
детских рисунков \enquote{Мы - славяне!}, посвященную Дню славянской
письменности и культуры, который ежегодно широко отмечается в \emph{России} и
других странах 24 мая}, - говорится в сообщении, lug-info.com, 24.05.2021.

Сейчас мы переживаем время не менее судьбоносное. И от того, вспомнит ли народ
свои изначальные культурные коды, зависит судьба наша в новом веке – судьба не
только \emph{России}, но и всей человеческой цивилизации, vz.ru, 24.05.2021

С этого времени \emph{Русь} ощущает себя уже не просто некоей «срединной
землей» между Востоком и Западом, Азией и Европой, но неким особым миром между
землей и небом, между началом и концом истории. И здесь, между небом и землей,
началом и концом истории, начинает она искать свои берега: «Открылась бездна,
звезд полна. Звездам числа нет, бездне – дна», vz.ru, 24.05.2021

Свободная и богатая за счет транзитной торговли, \emph{Русь} переживает бурный
рост, на пике которого и является «Слово о законе и благодати» (между 1037 и
1050 годами) митрополита Илариона – первое слово \emph{русского} самосознания,
сказанное первым этнически \emph{русским} митрополитом, vz.ru, 24.05.2021

Не просто Україна, а Україна-\emph{Русь}. Як Михайло Грушевський захищає
державні кордони?  radiosvoboda.org, 23.05.2021

А вот принципиального различия между укладами современной \emph{Россией} и
Украиной нет. Тут, как в Крыму или на Донбассе - были украинцы сверху, стали
\emph{русские}, а по сути, так примерно одни и те же люди. Народу, по большому
счету, начхать, сегодня \emph{русские} записываются в украинцы, завтра украинцы
запишутся в \emph{русские} - делов-то, \textbf{Дело Протасевича - еще один
повод раздуть Холодную войну}, Денис Жарких, strana.ua, 24.05.2021

Ця традиція іде ще від апостола Андрія – дві тисячі років тому він першим
провістив сакральну природу Києва. У ХІ столітті у знаменитій промові «Слово
про закон і благодать» перший \emph{Руський} митрополит Іларіон возвеличив Київ
як «Город Святий Всеславний», radiosvoboda.org, 09.10.2018

Добре, хоч \emph{Мордору} нічого не дали, але перед Литвою соромно! Так не
роблять! Лариса НиЦой, facebook, 23.05.2021

Я голосувала окрім деяких інших і за Литву, вони класні. А журі заангажовані,
це очевидно. І що бісить, Україна з року в рік продовжує давати бали
\emph{Русні}, комментарий (Ірина Чубарова), пост Ларисы НиЦой, facebook,
23.05.2021

Если \emph{Россия} не захочет быть мощной, имперской страной, она станет
Украиной. Само по себе майданное государство с \emph{российского} горизонта не
исчезнет, а предположение, будто украинская проблема рассосется сама по себе,
за семь лет доказало свою несостоятельность, \textbf{Надоела Украина? Ежедневное «иди и
смотри»}, Константин Кеворкян, 24.05.2021

Двери закрываются. Жителей \enquote{Л/ДНР} в \emph{России} ждёт ГУЛАГ 2.0,
obozrevatel.com, 25.05.2021

Помимо этого, как юрист, Лилия Корнилова раскрывает природу инструментария
\emph{российского} агрессора на международной арене и методы противостояния
этим провокациям, \url{www.obozrevatel.com/person/liliya-kornilova.htm}

«В \emph{России} отличное бездорожье» — как автобус не смог довезти детей до
школы, regnum.ru, 25.05.2021

Полузабытый \emph{русский} гений, которого в США считают своим — фото с
выставки, regnum.ru, 25.05.2021

\emph{Россия} лайкающая, Что на самом деле смотрят и ищут \emph{россияне} в
интернете, lenta.ru, 14.05.2021

Для поставок в Европу нефтегазового сырья, удобрений и металлов \emph{Россия}
использует не грузовые автомобили, а трубопроводы, железную дорогу и танкеры в
морских портах. \emph{Россия} в последние годы активно переводила свои грузы с
белорусской железной дороги на собственные морские порты на Балтике, vz.ru,
25.05.2021

Возмущение патриотов поддержал министр культуры и информационной политики
Александр Ткаченко, заявив, что в \emph{России} в очередной раз пытаются
присвоить себе национальное украинское блюдо, odnarodyna.org, 25.05.2021

У примітках до «Полтави» Пушкін зазначає: «Дорошенко, один з героїв давньої
Мало\emph{росії}, непримиренний ворог \emph{російського} панування»,
radiosvoboda.org, 25.05.2021

Не все, що звучить по-\emph{російськи} чи не все, що схоже на \emph{російську}
чи подібне до слова, яке існує в \emph{російській}, було сюди принесено з
\emph{російської} мови, pravda.com.ua, 25.05.2021

Якщо розбирати, яке слово де-факто прийшло з української, а яке запозичене,
вийдуть дуже прикольні речі. Вийде, що слово \enquote{щур} в українській мові скоріш за
все з \emph{російської}, а \emph{російське} слово \enquote{крыса} скоріш за все з української, просто
свого часу ми ними помінялися, pravda.com.ua, 25.05.2021

Ступка, вічна пам'ять, в кожній другій сцені, де він з'являється в ролі Тараса
Бульби, повторює про \emph{"русскую душу"}, \emph{"русскую землю"},
\emph{"русскую силу"}. Будь-яка компліментарність щодо України в цьому випадку
– це імперська, шовіністична компліментарність до меншого брата. Гоголь
спочатку був не про це, Бортко зробив його про це, і цей контекст змінює всі
повідомлення, pravda.com.ua, 25.05.2021

Сладкие парочки: \emph{Россия} и Украина, mikle1.livejournal.com, 21.11.2009

Даже я понял, что он лично виноват в том, что \emph{Россия} до сих пор не
развалилась на демократические Чечни и Запопинские республики Окрайны, Китайны
и прочая, mikle1.livejournal.com, 21.11.2009

Записки киевлянки: людей накручивают - «26 мая затмение, вот тогда
\emph{Россия} и нападет! - kp.ru, 15.04.2021

Истории Олеся Бузины: Почему рухнула Киевская \emph{Русь}, sedognya.ua, 14.06.2014

Я уже писал в одной из предыдущих статей, что название «Киевская \emph{Русь}»
придумал только в XIX веке московский историк Михаил Погодин. До него никто
даже не подозревал, что она «киевская». Современники называли эту страну просто
\emph{Русью} или \emph{Русской} землей, Олесь Бузина, segodnya.ua, 14.06.2014

Пельмени-мутанты, \emph{русский} гимн и \enquote{Кин-дза-дза} – англичанин
попал в \emph{Россию} и признался: \enquote{I was not gotov...}, tsargrad.tv,
12.04.2021

«Наглые \emph{русские}»: мир протестует против формы \emph{российских}
олимпийцев, woman.ru, 16.04.2021

«Наш слоган — \enquote{Окна в \emph{Россию} будущего}: интервью с автором \emph{русской}
кибердеревни, www.mirf.ru, 16.04.2021

Фигурное катание. \enquote{Невозможно победить \emph{Россию}. Их фигуристы на
запредельном уровне}, - Японцы высказались о КЧМ-2021, sport.ru, 16.04.2021

Первое упоминание названия \emph{Россия} - X век н.э. Константин
Багрянородный, zhenziyou.livejournal.com, 19.03.2016

Как Киевскую \emph{Русь} превратили в Украину, а потом в Анти\emph{Россию},
kp.ru, 17.06.2019

\enquote{Одержимость вот такая русофобская и навязчивая идея обвинить \emph{Россию} во всем и
вся. Наверное, уже скоро дойдет до того, что \emph{Россия} будет обвиняться просто в
факте своего существования}, - заявил Песков, strana.ua, 25.05.2021

«Даже обидно. На Украине из всех утюгов воюют с \emph{Россией}, а \emph{РФ} на нее плевать», - пишет Иосиф Эглис,
\textbf{Реакцию россиян на человека с символикой Украины сняли на видео}, riafan.ru, 13.05.2021

«А теперь попробуйте прогуляться по Львову с \emph{российской} символикой», - предлагает Татьяна Сергиенко,
\textbf{Реакцию россиян на человека с символикой Украины сняли на видео}, riafan.ru, 13.05.2021

Єдине «тішить» у цій ситуації, що й українці дали «достойну відповідь»
слов'янам. Українські глядачі лише проголосували за виступ від однієї
слов'янської країни – \emph{Росії}. Хоча \emph{Росію} (що показово!) представляла співачка,
яка не є слов'янкою. Але ж українці без «братньої \emph{Росії}» жити не можуть
– хоча й воюють із нею. Тому й дали співачці Маніжі від \emph{Росії} 4 бали.
Добре, що не 12, - \textbf{Про слов'янську солідарність?}, Петро Кралюк, day.kiev.ua,
24.05.2021

Империя воскресла: почему \emph{Россия} вычеркивает диссидентов из своей
истории, glavred.info, 24.05.2021

Ці «сайти-помийки» теж часто перетворюються на рупори \emph{російської}
пропаганди. Чи то піарячи того-таки Пальчевського з його \emph{проросійськими}
тезами, чи то поширюючи фейк Медведчука про американські біолабораторії, чи
цитуючи псевдоекспертів про фашизм в Україні, чи просто напряму даючи слово
Пєскову, \textbf{Сотні тисяч. Яка аудиторія (про)російських медіа в Україні},
texty.org.ua, 12.05.2021

Ще частіше незаконно використовують \emph{російську} мову замість української
посадовці обласного рівня, \textbf{Місцеві ради, телебачення, спорт та інше.
Хто порушує закон про мову і що робити, щоб він виконувався}, texty.org.ua,
20.05.2021

Партизанська війна? Як Зеленський готує ЗСУ до можливого масштабного наступу
\emph{Росії}, radiosvoboda.org, 26.05.2021

Може, якби наші 210 тисяч вояків були озброєні такою технікою, як Армія оборони
Ізраїлю, то, може, і \emph{Росія} тремтіла б перед нами, розумієте?» – зазначає
Жданов, radiosvoboda.org, 26.05.2021

Вице-президент Ассоциации туроператоров \emph{России} (АТОР) Дмитрий Горин
рассказал, в какие пляжные страны из тех, с которыми \emph{Россия} возобновила
авиасообщение с 25 мая, можно будет лететь отдыхать. Об этом сообщает «Москва
24», lenta.ru, 26.05.2021

Накануне создания \emph{Русского} царства в Северо-Восточной \emph{Руси} было
три великих княжества: Московское, Тверское и Рязанское, \textbf{Как немецкий
принц стал черниговским князем и русским святым}, Подумалось мне часом,
zen.yandex.ru, 24.03.2021

Великое княжество Рязанское много веков было пограничным краем, встречавшим
всех завоевателей. К югу от рязанских земель начиналось Дикое Поле, откуда
испокон веков на \emph{Русь} набегали кочевники, \textbf{Как немецкий принц
стал черниговским князем и русским святым}, Подумалось мне часом,
zen.yandex.ru, 24.03.2021

\emph{Россия} большая, сильная, она с нефтью и газом, современным вооружением,
лучшими в мире вакцинами и много чем ещё. Много кому от нас что-то надо, все
едут, просят, договариваются, Мак Сим, zen.yandex.ru, 16.05.2021

У середу, 26 травня, активісти «КримSOS» провели акцію «Ходіння по колу» біля
посольства \emph{Росії} в Києві. Це стало 58-м заходом на підтримку жертв
насильницьких зникнень в окупованому \emph{Росією} українському Криму та їхніх рідних,
\textbf{«Ходіння по колу»: посольство Росії пікетували на підтримку жертв насильницьких зникнень у Криму – фоторепортаж},
radiosvoboda.org, 26.05.2021

У 1905 році Рада міста Львова вирішила урочисто відзначити 250-річчя оборони
Львова під час облоги міста військами Богдана Хмельницького. Офіційно це
звучало як \enquote{порятунок міста від козаків та \emph{росіян}}, \textbf{Дві
історії, які нічого не вчать}, zaxid.net, 25.05.2021

Українець Феофан Прокопович назавжди змінив обличчя \emph{Росії}, перетворивши
її на імперію. Навіть відновлення патріаршества, революція, епоха радянського
атеїзму не змогли вбити його напрацювання, \textbf{Прокопович та інші українці,
що будували Росію}, Володимир Володько, zrada.org, 04.03.2011

Но остается открытым вопрос: почему страна, которая \enquote{не \emph{Россия}},
так сильно озабочена в основном \emph{российской} или, на худой конец,
бела\emph{русской} повесткой?
\textbf{Протасевич нам важнее, чем состояние собственной экономики}, Владислав Михеев, 
strana.ua, 27.05.2021

\textbf{Протасевич нам важнее, чем состояние собственной экономики}, Владислав Михеев, 
strana.ua, 27.05.2021

Массовые расстрелы в школах и других общественных местах — пугающая проблема
современного общества, с которой в последние годы столкнулась и \emph{Россия},
lenta.ru, 27.05.2021

В то же время, Пивоваров сообщил, что закрытие организации вовсе не означает,
что ее члены опустили руки. Они по прежнему готовы делать все возможное
\enquote{чтобы Россия стала свободной}. При этом он выразил уверенность в том,
что \enquote{Россия обязательно будет свободной}, Организация \enquote{Открытая
Россия} закрылась, strana.ua, 27.05.2021

Добре, що ми навчилися відкидати \enquote{\emph{русский} мир}. Погано, що ми ще
не навчилися боротися з нашими рідними терористами-популістами, і країна
залишається у них в заручниках, Сергій Фурса, gazeta.ua, 25.05.2021

Населення території цілком допускає можливість, що \emph{рускій} начальнік
повернеться, Іван Семесюк, gazeta.ua, 25.05.2021

Как сообщал OBOZREVATEL, на Донбассе вооруженные формирования \emph{Российской}
Федерации ранили украинского военного, obozrevatel.com, 27.05.2021

\enquote{Ненависть окутала их глаза}. В Кремле прокомментировали слова
президента Польши о \enquote{\emph{России}-агрессоре}, strana.ua, 27.05.2021

Ранее мы рассказывали, что Кабинет министров \emph{Российской} Федерации
утвердил список недружественных \emph{России} стран. Однако Польши в этом
перечне нет, \textbf{\enquote{Ненависть окутала их глаза}. В Кремле
прокомментировали слова президента Польши о \enquote{России-агрессоре}},
strana.ua, 27.05.2021

Судьба \emph{России} будет решаться на улицах, Виталий Портников,
obozrevatel.com, 27.05.2021

Ведь власть пытается информационно пристегнуть его к \enquote{зраде} Медведчука. И,
даже если вины лидера ОПЗЖ доказать не удастся, медийно Порошенко будут
\enquote{полоскать} и обвинять в сотрудничестве с \emph{Россией}, strana.ua, 27.05.2021

\emph{Россиянин} полез целоваться с лошадью и остался с откушенным носом, strana.ua, 10.12.2020

Как \emph{Россия} воюет на Донбассе: 30 главных расследований,
radiosvoboda.org, 27.05.2021

Князь Николай Трубецкой ещё до войны снискал себе славу одного из наиболее
проникновенных \emph{русских} историков и религиозных философов. Его занятия
привели его к мысли, что Европа в \emph{русском} обществе переоценена именно в
духовно-практическом смысле, тогда как Азия имеет непосредственное влияние на
формирование \emph{русского} менталитета, \textbf{«Евразийская концепция
\emph{русской} истории». Черниговский евразиец Петр Савицкий}, ukraina.ru,
27.05.2021

Подругу Романа Протасевича, администрировавшую канал со сливами личных данных
бело\emph{русских} силовиков и арестованную вместе с ним, зовут София Сапега,
\textbf{У подруги Протасевича историческая фамилия - Сапега}, strana.ua,
28.05.2021

У звіті Facebook сказано, що \emph{Росія} залишається світовим лідером у
виробництві дезінформації, спрямованої на втручання до інших країн. Переважно
цілями її закордонних операцій, окрім США та України, є Велика Британія, Лівія
та Судан, radiosvoboda.org, 28.05.2021

«Роды просто раздавили», Одиночество, нищета и депрессия: истории
\emph{россиянок}, которых заставили рожать нежеланных детей, lenta.ru,
28.05.2021

«Он невероятно умелый манипулятор», Как \emph{Россия} может воспользоваться
окончательным разрывом Лукашенко с Западом?, lenta.ru, 28.05.2021

\emph{Россиянка} выиграла квартиру и тут же ее лишилась, lenta.ru, 28.05.2021

Як хочеться потрапити у світ без \emph{Русской} общины Украины, \emph{Русского}
блока та їм подібних московських шавок.  Як хочеться почути живих Драгоманова
та Ключевського; тих людей, які займались історією як наукою, а не як повією,
\textbf{Забыть все = Забить на всё. Як хочеться усе забути}, zrada.org,
24.07.2010

Саме в цей день рівно 7 років тому наші герої зробили вибір на користь свободи,
навіть якщо за неї треба було віддати життя. \emph{Російські} бойовики
намагалися взяти під контроль міжнародний аеропорт Донецька, вимагали в
українських військових скласти зброю. Наші захисники здаватися не збиралися та
мужньо тримали оборону аеропорту. Підрозділи спецназу за підтримки авіації
завдали першого потужного удару по терористах. Того дня \emph{російські}
окупанти відступили, зазнавши великих втрат, \textbf{26 травня 2014 року – один
з переломних моментів війни на Сході}, bigkyiv.com.ua, 26.05.2021

Более того, \emph{российская} вакцина и разговоры о ней являются элементом
гибридной войны против Украины. В частности, таковым может быть и заявление
Лукашенко, считает Николенко, \textbf{\enquote{В русле подрывных усилий}}...
strana.ua, 28.05.2021

Более того, поскольку мы уже столько лет ведем войну с \emph{Россией}, самый
прям удачный момент отменить вместе с бензином электричество, например. И
антрацит с коксом. В рамках выбранного курса, так сказать, \textbf{Сейчас самое
время отменить вместе с бензином электричество}, Дмитрий Заборин, strana.ua,
29.05.2021

Уже давно, к сожалению, прошло то время, когда \emph{российские} и украинские патриоты
обменивались историческими шуточками и только ими могли и ограничиваться. Одна
из них, с \emph{российской} стороны, была такая:
Если вы так недовольны Москвой, то все ваши претензии – к вашему же
«украинскому» киевскому князю Юрию Долгорукому, который её основал! - zen.yandex.ru, 

Далее человек, выдающий себя за \emph{российского} оппозиционера, требует у
европейцев биткоины и подписаться на его YouTube-канал, учит делать Путину
больно и показывать голый зад \emph{российскому} посольству. Седые господа
вежливо слушают.  Связь никто не разрывает, \textbf{Наши европейские партнеры
готовы договариваться хоть с чертом лысым}, Максим Могильницкий, strana.ua,
29.05.2021

Как было не раз сказано, оная история почти повторяет разворот и обыск
\emph{белорусского} самолёта, устроенного СБУ в 2016-м. Но, как мне (примерно)
написал один из комментаторов с флагом ЕС на аватарке, \enquote{Да мне
наплевать на \emph{белорусский} самолёт. \emph{Беларусь} не член ЕС. А то был
европейкий самолёт, с европейскими гражданами на борту.} Отлично раскрыта тема
кто какого сорта, я считаю.  Но это ещё не всё, \textbf{История с Протасевичем
объемно показывает, кто чего стоит}, Роман Подолян, strana.ua, 29.05.2021

Ан-26 — самолёт со славным прошлым. Машина, засветившаяся даже в голливудском
кино — к примеру, в боевике «Неудержимые».  Но время не стоит на месте. И
скоро, в том числе и из-за провальной политики украинского руководства и
разрыва производственных связей с \emph{Россией}, господство в небе могут
захватить другие машины, \textbf{Привет «Антонову»: \emph{Россия} лишит Украину
важного рынка}, ukraina.ru, 28.05.2021

\emph{Російська} дезінформаційна кампанія в Європі проти вакцини Pfizer.
Розслідування Радіо Свобода встановило дійових осіб, radiosvoboda.org,
29.05.2021

Хоча жодні докази не свідчать про причетність \emph{російських} державних
установ чи чиновників до кампанії в соціальних мережах, їхні меседжі та способи
подачі звучать суголосно до того, що робить PR-кампанія \emph{Російського}
фонду прямих інвестицій, який фінансує і просуває на ринок вакцини Sputnik V,
\textbf{\emph{Російська} дезінформаційна кампанія в Європі проти вакцини
Pfizer. Розслідування Радіо Свобода встановило дійових осіб}, radiosvoboda.org,
29.05.2021

Прессекретар Кремля Дмитро Пєсков прокоментував ці звинувачення. «\emph{Росія}
нікого не дезінформує, \emph{Росія} з гордістю говорить про свої успіхи, і
\emph{Росія} ділиться своїми успіхами щодо першої в світі зареєстрованої
вакцини», – сказав він, \textbf{\emph{Російська} дезінформаційна кампанія в
Європі проти вакцини Pfizer. Розслідування Радіо Свобода встановило дійових
осіб}, radiosvoboda.org, 29.05.2021

Я люблю читать новости. Какими бы они ни были. Хорошими, плохими, леденящими
кровь - без разницы. Читая новости, я ощущаю биение жизни и понимаю, что от
кого бы мы ни произошли, \emph{русские} люди, обезьяны были нашими пра-пра или
инопланетяне, без разницы, наша степень психической надежности - как у
хранилища Форт Нокс, хотя мы и по эту сторону океана, \textbf{Мы - несгибаемая
нация. Нас - только дустом}, Из Питера С Любовью. Юля, zen.yandex.ru,
01.04.2021

Мы вышли из одного древнего государства, приняли одно крещение и в то время
были не просто вместе - мы были одним народом. Нам доводилось разделяться на
разные княжества, бороться между собой за первенство власти, воевать за города
и земли, но как народ мы не разделялись и никто нас не разделял. Для самих себя
и для окружающих мы все именовались \emph{Русью}, \emph{русскими}, ну а
феодальная раздробленность народов была тогда явлением распространенным. Для
нашего народа такая раздробленность добром не кончилась - от внешнего врага
отбиться не смогли, \textbf{\emph{Россия} и Украина - утраченное Содружество},
Igor Novikov, zen.yandex.ru, 29.05.2021

Как ожидание Второго Пришествия в 1492-ом году изменило курс \emph{русской}
истории, Открытая семинария, zen.yandex.ru, 20.05.2021

Пелагея, душа народа, очень тронула ее новая песня, Люблю ее всем сердцем и
душой! Достояние \emph{России}! Слушая ее, я забываю обо всем, душа летает от
такого неимоверного сильного голоса, zen.yandex.ru, 11.05.2021

Тилль из «Rammstein»: любовь к \emph{Росиии} и песни на \emph{русском} языке,
zen.yandex.ru, 26.04.2021

В глазах у девочки Васнецов увидел столько одиночества и чисто \emph{русской}
печали, что прямо ахнул, Юлия Варенцова, zen.yandex.ru, 20.03.2021

Президент Польши Анджей Дуда во время визита в Грузию назвал \emph{Россию}
\enquote{ненормальной страной}.  \enquote{Действия \emph{России} — агрессивные,
имперские, которые отбирают у людей нормальную жизнь, приводят к разрушению
государств, к военным положениям, к гибели людей. <...> \emph{Россия} ненормальная
страна. Это не то государство, которое нормально себя ведет},— сказал господин
Дуда, obozrevatel.com, 29.05.2021

Когда я подготовил в 1985 году свой первый семинар по истории \emph{русских}
интеллектуальных течений, 127 студентов подвои заявки на 12 мест.....  Я не
рассчитывал, что они будут что-то знать о \emph{России}, но считал, что человек
интересующийся историей интеллектуальных течений, должен быть знаком с
классикой мировой литературы, \textbf{Американская элита совершенно оторвана от
мировой культуры}, Игорь Заславский, strana.ua, 30.05.2021

Лихі 90-і, «Крим наш» і єдиний народ \emph{росіян}, білорусів та українців.
Викриваємо міфи, що Путін насаджує про \emph{Росію}, radiosvoboda.org,
30.05.2021

Народи України, Білорусі та \emph{Росії} – це єдина нація, radiosvoboda.org,
30.05.2021

Крим завжди був \emph{російським}, radiosvoboda.org, 30.05.2021

Крим і \emph{Росію} «пов'язує спільна історія, що йде в глиб століть – тут
прийняв хрещення князь Володимир, тут знаходяться могили \emph{російських}
солдатів, які завоювали Крим в 1783 році для \emph{російської} держави, тут
стоїть Севастополь – батьківщина \emph{російського} Чорноморського військового
флоту». «В серцях людей Крим завжди був невід'ємною частиною \emph{Росії}»,
radiosvoboda.org, 30.05.2021

\emph{Росія} і Захід «однаково погані» і варті одне одного, radiosvoboda.org, 30.05.2021

Нам потрібна нова загальноєвропейська архітектура безпеки за участю
\emph{Росії}, radiosvoboda.org, 30.05.2021

Захід повинен поліпшити відносини з \emph{Росією}, навіть якщо вона не йде на
поступки, оскільки це занадто важливо, radiosvoboda.org, 30.05.2021

Санкції щодо \emph{Росії} - це невірний підхід, radiosvoboda.org, 30.05.2021

Хоча у самій \emph{Росії} точно такі ж проблеми з інтегральними героями. Хіба
тягне на цю роль підкорювач Кавказу генерал Єрмолов? Чи може розраховувати на
статус об'єднавчої фігури герой балканських воєн і затятий націоналіст генерал
Скобелєв? \emph{Російське} суспільство готове ставити на п'єдестал усіх тих,
при кому країна «приростала територіями», але чи готові корінні народи співати
цим людям осанну? \textbf{Від Мазепи до Бандери. В України і Росії різні
герої}, Павло Казарін, radiosvoboda.org, 28.04.2021

В пятницу, 28 мая, пограничники Сумского отряда пресекли незаконное перемещение
из \emph{России} продукции военного назначения. Также был задержан 48-летний житель
города Сумы, obozrevatel.com, 29.05.2021

Организованный кафедрой социально-гуманитарных дисциплин Академии марафон
провели онлайн. В течение нескольких часов студенты и преподаватели вуза читали
стихи и прозу о \emph{русском} языке, \emph{русской} культуре,
\textbf{Литературным марафоном отметили в Академии Матусовского День славянской
письменности и культуры}, lgaki.info, 24.05.2021

Прах известного \emph{российского} стилиста Александра Шевчука захоронен в
Умани, strana.ua, 10.02.2021

Конституція Орлика — це пакт конституції прав і вольностей Війська Запорозького
1710 року. Його повинен був прийняти Карл XII у разі перемоги у Північній війні
за послуги, надані козаками у боях Швеції проти \emph{Росії}, \textbf{Названо
локацію у Києві, де покажуть оригінал Конституції Пилипа Орлика}, kyiv.media,
29.05.2021

\enquote{Нарисовали гигантское яблочко на спине Украины}. Чего ждут от встречи
Байдена и Путина на Западе и в \emph{России}, strana.ua, 30.05.2021

\enquote{\emph{Россия} - это серьезный игрок, нравится нам это или нет}, - так
кратко и четко мотивирует Washington Post, почему встреча все же состоится,
strana.ua, 30.05.2021

Как писал «Рамблер», ранее журналист Дмитрий Гордон заявил, что презирает
украинских артистов, которые продолжают выступать в \emph{России}. Он назвал их
«людьми без родины».  «Я не призываю их наказывать, репрессировать, делать им
плохо. Ни в коем случае. Они сделали свой выбор. Он абсолютно законный», -
заявил журналист, добавив, что у них либо «две извилины в голове», либо им «
застили глаза зеленые бумажки», news.rambler.ru, 27.05.2021

Одна из бед \emph{России} в том, что у нас сапожник варит щи, а кухарка мнит
себя сапожником. Вот и получается, что Юлия Волкова может стать политиком. Уже
даже не смешно. Смотрите, что получается... ЗВЕЗДУЛЬКИ, zen.yandex.ru,
02.05.2021

Певец \footnote{Юрий Лоза} и на этот раз сказанул (небезосновательно), почему
считает, что население \emph{России} тупеет. Я частично согласен с ним и
расскажу, почему (по моему мнению) это происходит и кто виноват. Поскакали,
\textbf{Почему Лоза считает, что население тупеет: а я скажу кто виноват?}
ЗВЕЗДУЛЬКИ, zen.yandex.ru, 11.04.2021

Самым странным выглядит \emph{русский} язык, на котором актёры говорят с явным
акцентом. Судя по всему, это чтобы сериал в \emph{Россию} потом продать.
Непатриотично конечно, но за деньги. Вакцину у \emph{России} брать нельзя -
непатриотично, идёт в разрез с анти\emph{российским} политическим курсом. А сериал на
\emph{русском} снимать можно, так как (опять же) за деньги.  Извиняюсь, снова в
сторону ушёл от темы. Так вот: представьте, вы живёте в стране, где
разговариваете (а значит думаете) на \emph{русском}, веселитесь (слушаете
песни) на \emph{украинском}, а в общественных учреждениях ориентируетесь при
помощи надписей на \emph{английском}, Тут волей-неволей каша в голове будет,
ЗВЕЗДУЛЬКИ, zen.yandex.ru, 01.05.2021

В \emph{России} взорвался надувной батут, на котором играли дети. Их отбросило
на рельсы. Фото, strana.ua, 30.05.2021

В \emph{России} на кладбище 6-летний мальчик сел за руль автомобиля и случайно
насмерть задавил свою мать, strana.ua, 03.05.2021

1300 курян приняли участие во Все\emph{российском} полумарафоне
\enquote{ЗаБег.РФ}, riakursk.ru, 30.05.2021

В этом году акция \enquote{Забег.рф} претендует на звание рекордсмена Гиннеса,
как самый большой в мире полумарафон с синхронным стартом. Старты были
организованы в 85 городах \emph{России}, riakursk.ru, 30.05.2021

\emph{Русско}язычная проза в Украине исчезает как Аральское море. Стоящих
авторов можно пересчитать на пальцах одной руки. Книг, достойных прочтения,
выходит крайне мало. \emph{Русско}язычного литературного процесса толком нет.
Обсуждать, по большому счёту, нечего, \textbf{Большой киевский роман},
hvylya.net, 30.05.2021

Да здравствует великая \emph{Русь}. Все \emph{русские} должны быть вместе, мы
один народ, (комментарий), \textbf{Потерять яйца в сражении с Беларусью},
Анатолий Шарий, youtube, 30.05.2021

Жителям Крыма тоже говорили: \enquote{Ааа продались за \emph{российские} зарплаты и
пенсии!!!}. Так это и есть называется - думать о своём народе. Вместо этой
конченной власти в украине, в Крыму и Беларуси будет нормальная экономика,
(комментарий), \textbf{Потерять яйца в сражении с Беларусью}, Анатолий Шарий,
youtube, 30.05.2021

18:48  Еще одна активистка - Анна Гвоздяр - вышла к микрофону с американским
флагом на рукаве. Она заявила, что уже восемь лет защищает Украину \enquote{от
всякой \emph{русни}}, \textbf{\enquote{Завтра будем требовать другими
методами}. Как в Киеве прошла акция сторонников Стерненко}, strana.ua,
30.05.2021

18:12  Активисты принесли импровизированные \enquote{двери офиса президента},
на которых воспроизвели те же надписи, которые там были на предыдущей акции за
Стерненко. Также они держат плакаты \enquote{Stop \emph{Русский} мир},
\textbf{\enquote{Завтра будем требовать другими методами}. Как в Киеве прошла
акция сторонников Стерненко}, strana.ua, 30.05.2021

Он хотел остановить катастрофу \emph{русского} народа, Владимир Станулевич, regnum.ru, 28.05.2021

Три \emph{Руси} и три \emph{России}. Что не дает покоя Польше и Прибалтике, В
Переулках Истории zen.yandex.ru, 18.05.2021

\emph{Россия} как всегда подставила плечо своему \enquote{верному} союзнику,
одолжив пол-лярда зелени и предоставив бело\emph{русскому} авиаперевозчику
\enquote{Белавиа} дополнительные рейсы в \emph{российские} города.  Также Путин
ещё раз подтвердил что первый блок БелАЭС будет запущен в эксплуатацию в
следующем месяце, в июне. Ах да, два президента ещё и в море искупались и
дежурно что-то про интеграцию говорили, \textbf{Бело\emph{русская}
многовекторность вечна, как и сам Лукашенко}, Мак Сим, zen.yandex.ru,
30.05.2021

Это благодаря ему мыслящая \emph{Россия} познакомилась с дарвинизмом - перевёл
\enquote{Происхождение видов}. Вёл огромную общественную, просветительскую
работу, за свой счёт отправлял за границу особо одарённых студентов и помогал
выжить студентам неимущим, \textbf{Неиссякаемо талантливый народ!}, Наталья
Баева, zen.yandex.ru, 06.04.2021

Массовое убийство на вечеринке в \emph{России}: главная версия и имена жертв,
glavred.info, 07.11.2020

А на додачу потрібна ефективна пропаганда у мас-медіа; зрозуміло, що на палких
адептів \enquote{русского мира} вона не подіє, але ж є чимало таких, що
коливаються між вірою у московські побрехеньки та довірою до власної держави,
Сергій Грабовський, gazeta.ua, 27.05.2021

\enquote{Заблоковані} телеканали далі працюють. Це небезпечно для України.
Сотні тисяч реальних адептів \enquote{\emph{русского} мира} із захватом
дивляться їхні програми, Сергій Грабовський, gazeta.ua, 20.05.2021

Туляки встріли в академії студентів з усієї \emph{Росії}. Великоруський синод
[6] ще попереду, ніж уряд, спостеріг ідею \emph{русифікації}, і для того він
велів в академіях мішати українців з \emph{руськими} студентами. Тим-то в
Київську академію пруть семінаристів з Костроми, Архангельська, з Волги й
Сибіру, мішаючи їх з киянами, полтавцями, одесцями й іншими і посилаючи
українських семінаристів до Москви й Петербурга, котрі, одначе, не мають охоти
туди їхати, \textbf{Хмари}, Іван Нечуй-Левицький

По розкішних алеях Братського монастиря [7], густо обсаджених усяким деревом,
гуляли студенти з усіх кінців широкого \emph{Російського} царства,
\textbf{Хмари}, Іван Нечуй-Левицький

Але, як і революціонери, що використовували нову форму Інтернету для об'єднання
та усунення ворога, \emph{Росія} тепер використовувала мережі, щоб розірвати Україну
на частини, \textbf{Війна лайків. Зброя в руках соціальних мереж}, П. В.
Сінґер, Емерсон Т. Брукінґ, Харків, 2019

Важливим прецедентом цього стала Україна. Кількість негативних
\emph{російськомовних} новин про Україну зросла вдвічі, а потім утричі. Етнічні
\emph{росіяни} всередині України невдовзі збурилися проти активістів, що
скинули \emph{проросійський} уряд. Тим часом \emph{російські} спецпризначенці
проникли до Криму, а потім на схід України, набираючи та озброюючи загони
\emph{проросійських} сепаратистів. Хвилі протестів переросли в насильство, а
потім у трагедію, \textbf{Війна лайків. Зброя в руках соціальних мереж}, П. В.
Сінґер, Емерсон Т. Брукінґ, Харків, 2019

У моделі Птолемея Землю оточували вісім обертових сфер. Кожна наступна сфера
була більша за попередню — щось на кшталт \emph{російської} матрьошки. Земля
перебувала в центрі. Що саме лежить за межами останньої сфери, ніколи не
уточнювали, але це, безперечно, було недосяжним для людського погляду. Тому
найдальшу сферу вважали чимось на кшталт кордону, вмістищем Усесвіту,
\textbf{Найкоротша історія часу}, Стівен Гокінґ і леонард млодінов, Харків,
2016

...і в результаті \emph{Росія} – країна цілком конкурентоспроможна в окремих галузях
науки – безнадійно відстала у царині молекулярної біології й генної інженерії.
Було втрачено два покоління біологів. Лисенківщина протрималася до 1964 року,
аж поки її головного ідеолога не зняли з посади після палких дискусій в
Академії наук, одній із небагатьох інституцій, яка ще зберігала відносну
автономію. Важливу роль у цьому епізоді відіграв фізик-ядерник Андрій Сахаров,
\textbf{«Світ, повний демонів. Наука як свічка у пітьмі»}, К. Э. Саган —
Книжный Клуб «Клуб Семейного Досуга», 1996

Показовий приклад – \emph{Росія}. За царату в країні буйно квітли релігійні забобони,
а наукове і скептичне мислення існувало хіба що у вузькому колі вчених, які
не мали великого впливу. При комуністах і релігію, і псевдонауку систематично
переслідували, однак на заміну їм прийшла нова державна ідеологія. Вона
видавала себе за наукову, але до науки їй було так само далеко, як і будь-якому
містичному культу. У критичному мисленні радянська влада вбачала небезпеку і
карала за його прояви. Про те, щоб викладати його у школі, не могло бути й
мови. Критична думка могла існувати тільки у вузькій герметичній сфері
фундаментальної науки, потрібної режиму,
\textbf{«Світ, повний демонів. Наука як свічка у пітьмі»}, К. Э. Саган —
Книжный Клуб «Клуб Семейного Досуга», 1996

У 1950-х роках \emph{російський} фізик Лев Ландау показав, що електричний заряд
електрона залежить від масштабу, на якому він вимірюється. З нікуди вигулькують
віртуальні частинки, тож електрони та всі інші елементарні частинки перебувають
в оточенні хмари пар віртуальних частинок й античастинок. Ці пари екранують
заряд аналогічно до екранування заряду в діелектриках. Позитивно заряджені
віртуальні частинки схильні щільно оточувати негативний заряд, тож на відстані
фізичні впливи початкового негативного заряду зменшуються, \textbf{Таємниці
походження всесвіту}, Лоуренс Краусс, Книжный Клуб «Клуб Семейного Досуга» 2017

До чого я веду? Реальної дати заснування Києва не знає ніхто. Так само, як у
випадку зі Львовом чи Москвою (ті, хто говорять про жаб, які квакали на місці
Москви у той час, як у Києві існували бібліотеки, грішать проти історичної
правди, бо археологи свідчать про наявність великого торгового центру на місці
нинішньої столиці \emph{Росії} вже у Х столітті). Одні міста ведуть свій
початок від першої згадки у писемних джерелах (тоді історія Києва велася б від
860 року).  Інші - від гіпотетичної дати заснування, \textbf{\emph{Киев} не
должен мериться древностью с другими столицами}, Константин Бондаренко,
strana.ua, 31.05.2021

Дірка замість унітазу. Шкільні туалети і велич \emph{Росії}, radiosvoboda.org,
31.05.2021

Она уточнила, что большинство отправившихся на отдых детей ни разу не были в
Крыму, а многие вообще впервые увидят море.  \enquote{То, что делает для
детворы ЛНР \emph{Российская Федерация}, это неоценимо, тем более в преддверии
Международного дня защиты детей}, - сказала уполномоченный по правам ребенка,
\textbf{Группа из 100 детей из ЛНР отправилась на отдых в крымский ДОК
\enquote{Дельфин}}, lug-info.com, 27.05.2021

Не только на Украине, но даже в \emph{России} многие не могут понять, почему на южных
\emph{русских} землях, внезапно ставших независимым государством, «что они ни делают,
не идут дела». Пока были \emph{Россией} — и «крокодилы ловились», и «кокосы
колосились», а как стали Украиной — с каждым днём становится хуже и хуже,
\textbf{Непойманный украинский крокодил}, Ростислав Ищенко, ukraina.ru, 31.05.2021

Страна, в которой господствует идеология украинского национализма (идеология
же), катится в пропасть с той же скоростью, с которой \emph{Россия} рвётся к
звёздам.  Между тем \emph{Россия} живёт без обязательной идеологии, но её
население в целом сохранило имперский дух, позволивший предкам создать самое
крупное государство на планете. Идея евразийской империи цементирует
\emph{российское} государство, даёт цель его развитию, 
\textbf{Непойманный украинский крокодил}, Ростислав Ищенко, ukraina.ru, 31.05.2021

Анатолий, езжай в \emph{Россию}, без шуток ты тут нужен и будешь иметь успех,
комментарий, \textbf{Опять Шария достали. Что дальше?} Анатолий Шарий, youtube.com, 31.05.2021

Чеські та українські експерти разом з білоруськими колегами почали аналізувати
та документувати присутність та вплив \emph{Російської Федерації} на життя та події в
Біло\emph{Русі}. Цей тристоронній проєкт надає можливість зібрати важливі факти, щоб
зрозуміти, як \emph{Росія} підтримує та допомагає підтримувати авторитарний режим у
Біло\emph{Русі},
radiosvoboda.org, \textbf{\emph{Російська} підривна діяльність, Білорусь та Крим –
серед тем другої зустрічі Українсько–чеського форуму}, 31.05.2021

...Сегодня мы работаем над несколькими сценариями остановки \enquote{Северного
потока-2}.  Необходима мобилизация всех ветвей украинской власти и четкая и
последовательная координация с нашими международными партнерами, которые
понимают опасность \emph{российской} трубы для всей Европы. Такими сейчас я
вижу приоритеты своей работы, - прокомментировала Залищук свое назначение на
своей странице в Facebook, 
\textbf{Галантерейщик и кардинал. Сцена третья. Апофеоз абсурда}, strana.ua, Валентин Землянский, 01.06.2021

Пограничники с тепловизором поймали \emph{россиянина}, который закопал свой
паспорт в лесу, strana.ua, 01.06.2021

Мужчина пробирался к границе под покровом темноты около двух часов ночи.
Пограничникам удалось обнаружить нелегала при помощи тепловизора.  После
задержания мужчина назвался гражданином Украины, но через некоторое время
признался, что является \emph{россиянином}, а в доказательство этого у него
обнаружили ксерокопию паспорта \emph{РФ}. Сам документ, по словам нарушителя,
он закопал где-то в лесу. Объясняя, зачем он изначально принялся врать
пограничникам, мужчина заявил, что \enquote{опасался за свою жизнь},
\textbf{Пограничники с тепловизором поймали \emph{россиянина}, который закопал
свой паспорт в лесу}, strana.ua, 01.06.2021

За панування \emph{Російської} Імперії на землях Речі Посполитої, як і України,
можливості отримання вищої освіти, надто для жінок, були вкрай обмеженими.
Відповіддю на це поляків стало створення підпільного Летючого університету у
Варшаві та околицях. Лекції, доступні слухачам незалежно від гендеру, читали
кращі вчені, зокрема викладачі розігнаного Віленського університету.
Відбувалося це на конспіративних квартирах. За 20 (!) років існування
Університету його випускницями стали не менше 5000 жінок. Найвідомішою серед
них є двічі лауреат Нобелівської премії Марія Складовська-Кюрі. Радіоактивний
елемент \enquote{Полоній} названо нею на честь поневоленої, в той час, \emph{росіянами}
Польщі, 
\textbf{Щодо вчорашнього указу про \enquote{Президентський університет}}, Костянтин Матвієнко, pravda.com.ua, 01.06.2021

Знову \enquote{Україна не \emph{Росія}}, проте багатьом дуже хочеться стерти цю відмінність,
Костянтин Матвієнко, pravda.com.ua, 02.02.2021

\emph{Росія} самоізолюється від світу тому, що за своєю природою не може існувати як
демократична держава, бодай із мінімальним набором громадських прав і свобод.
\emph{Росія} виникла та існує винятково як абсолютна монархія. Усі президенти України,
хіба окрім першого – Л. Кравчука, намагалися побудувати державну систему
України за \emph{російським} зразком. Усі спроби зазнали невдачі. Передостання з них
має своїм наслідком масове кровопролиття, окупацію Криму і частини Донбасу,
руйнування економіки...
\textbf{Роби те, що слід, і нехай станеться те, що статися має!}
Костянтин Матвієнко, pravda.com.ua, 13.05.2016

Може статися, що учора Україна нарешті подолала останній вододіл між радянським
абсолютизмом і модерною європейською демократією. Вистачило б часу та сил дійти
до мети, бо \emph{Росія} зубами тягтиме назад у морок самодержавства,
\textbf{Роби те, що слід, і нехай станеться те, що статися має!}
Костянтин Матвієнко, pravda.com.ua, 13.05.2016

Активно популяризувати українське в Україні почали відносно недавно – з
початком Революції Гідності та війни на Сході. Під приводом \enquote{захисту
\emph{російськомовного} населення} у 2014 році до України вторглися окупанти. Сьогодні
наші полонені на території ОРДЛО сидять у в'язницях. Вони змушені говорити
\emph{російською}, щоб мати шанс вижити та дочекатися української армії. Ось вам і
наслідки \enquote{захисту \emph{російської} мови від конкуренції},
\textbf{Чи є різниця, якою мовою говорити?} Андрій Білецький, pravda.com.ua, 22.01.2021

знати мову має кожен, хто живе, або працює в нашій країні! крапка! а особливо
державні діячи!!! це взагалі неприпустимо, бути так званою політичною елітою і
не знати державної мови. Впевнений, що більшість верховної ради з великими
зусиллями розмовляє українською, от з них і треба починати!!  Блокувати
\emph{російську} рекламу, блокувати \emph{російськомовні} канали! Гнати з країни всіляких
Медведчуків та рабіновичив, які чхати хотіли на нашу державну мову, в головах
тих людей досі сидить закорінілий совок!!!
коментар, \textbf{Чи є різниця, якою мовою говорити?} Андрій Білецький, pravda.com.ua, 22.01.2021

В 1993-ем нас было 52,2 миллиона. Через двадцать лет, - до начала войны с
\emph{Россией}, - 45,5 миллиона. Не было ни Гитлера, ни Сталина. Кто убил 6,7 миллиона
украинцев?  Политические игрища привели не только к потере территорий, но и к
дальнейшим потерям населения. В прошлом году министр Дубилет назвал
предполагаемую оценку «остатков» - чуть более 37 миллионов (2). Социологи
возмутились (3). Однако есть и более впечатляющая оценка,
\textbf{Кто убил 17 миллионов украинцев?}, Юрий Гуленок, hvylya.net, 01.06.2021

Украина 1 июня оказалась на третьем месте среди стран Европы по количеству
летальных случаев пациентов с коронавирусом за сутки. Возглавляет антирейтинг
\emph{Россия}, следом за ней идет Германия, 
\textbf{Украина вчера стала третьей в Европе по смертям от коронавируса}, Елена Вьюн, strana.ua, 02.06.2021

\enquote{Несмотря на вопли о \enquote{зраде} (а-ля нас сдали \emph{эрэфии}) мы системно работаем над
развитием отношений со стратегическим партнером №1. Только что представительная
делегация американских сенаторов в составе Джин Шахин (Демпартия), Роберта
Портмана (Республиканская партия) и Кристофера Мерфи (Демпартия) прибыла с
визитом в Украину! Завтра (2 июня - Ред.) будем говорить о том, как вместе
сделать Украину сильнее!}, - написал Евгений Енин,
\textbf{В Киев на переговоры прибыла группа сенаторов США}, Александр Максюк, strana.ua, 02.06.2021

\enquote{Украина и Грузия являются важными партнерами США в борьбе со все более
агрессивной \emph{Россией}, и очень важно, чтобы мы выразили нашу солидарность народам
Украины и Грузии и побудили политических лидеров продолжать проводить
необходимые реформы, которые укрепят их демократии}, - заявляла перед визитом
глава делегации Джин Шахин,
\textbf{В Киев на переговоры прибыла группа сенаторов США}, Александр Максюк, strana.ua, 02.06.2021

Трагедія, яка може стати спусковим гачком \emph{російського} BLM, Лилия Корнилова, obozrevatel.com, 02.06.2021

\emph{Российские} СМИ преподнесли новость о диаспоре \enquote{с изюминкой}. В
частности, написали, что \enquote{сотруднику ДПС, который случайно застрелил
19-летнего азербайджанца Векиля Абдуллаева во время задержания в Мошковском
районе Новосибирской области, пригрозили убийством}.  Якобы \enquote{в день
инцидента отдел полиции и больницу в Мошково окружили около 60 автомобилей},
\enquote{в них находились вооружённые представители азербайджанской диаспоры,
которые требовали выдать им \enquote{убийцу} и спасти раненного парня},
\enquote{пригрозили расстрелять семью инспектора},
\textbf{Трагедія, яка може стати спусковим гачком \emph{російського} BLM}, Лилия Корнилова, obozrevatel.com, 02.06.2021

І саме це є однією з причин, чому \emph{Росія} виграє інформаційно-пропагандистську
війну за розум української молоді. \emph{Росіяни} працюють системно і мають більше
досвіду. А в нас відбувається імітація виховної роботи. Через відсутність
рефлексії й усвідомлення подій та проблем наша молодь легко піддається на
маніпуляції і йде в \emph{російському} фарватері, бо змістовно для неї немає різниці.
Молоді люди не розуміють українського контексту, їх цього не вчили і не вчать,
\textbf{Союз – мертвий, комсомол – живий, Тиск замість розвитку! Що хочуть від \emph{української} молоді?},
Станіслав Безушко, zaxid.net, 31.05.2021

Через відомо які події знову зросла зацікавленість вітчизняного інформаційного
простору в \emph{БілоРусі}. І знову понеслося рефреном: «А-а-а, дурні
\emph{білоруси}, взуття знімали, а треба було бруківку жбурляти!». Я вже писав
про те, що ситуація в \emph{БілоРусі} радикально відрізняється від української,
повторюватися не буду. Тому вирішив поговорити про причинно-наслідкові зв'язки
в нашій нещодавній і не дуже нещодавній історії, які декому дуже не хочеться
помічати, 
\textbf{П'ять цікавих фактів – від \emph{УПА} до АТО}, Павло Зуб'юк, zaxid.net, 01.06.2021

Другий. Ні, наявність україномовної та націоналістичної Галичини сама собою не
була запобіжником від реалізації «\emph{білоруського} сценарію» в Україні. Суспільна
думка переважної більшості українців у році так 1994-му була \emph{проросійською}, чи
точніше – прорадянською. Воювати за незалежність від України галичани б не
стали. Якби справа була лише в цьому – \emph{проросійський} уряд у Києві з радістю дав
би автономію кільком областям або взагалі федералізував би державу. Значно
більшим запобіжником від аналога Лукашенка були міцні, зокрема регіональні,
олігархічні еліти, які вже утворилися в українській економіці. Звісно, Кучма
міг спробувати їх знищити, але Україна набагато більша і складніша за \emph{Білорусь}.
Варто поглянути на мапу: у \emph{Білорусі} тільки один меґаполіс, він же столиця, а по
радіусу від нього – приблизно однакові за розміром і значенням обласні центри.
\emph{Білорусь} природно значно централізованіша. І це, зокрема серед іншого,
посприяло її \emph{зросійщенню} в радянські роки, 
\textbf{П'ять цікавих фактів – від \emph{УПА} до АТО}, Павло Зуб'юк, zaxid.net, 01.06.2021

Третій. Ні, мова не рятує від диктатури, зокрема і \emph{проросійської}. Таджикистан –
країна таджицькомовна, \emph{російська} мова погано зрозуміла більшості таджиків.
Однак від самого проголошення незалежності ця країна є залежною від Москви
бідною державою з диктатором при владі. Чи, може, азербайджаномовний та дуже
націоналістичний Азербайджан – демократія?
\textbf{П'ять цікавих фактів – від \emph{УПА} до АТО}, Павло Зуб'юк, zaxid.net, 01.06.2021

П'ятий. Ні, Порошенко не створив армію з нуля і не врятував Україну від
\emph{російської} окупації. АТО почалася до приходу Порошенка на посаду глави держави.
Армія ще до приходу Порошенка захищала аеропорти і військові об'єкти. Перше
місто було звільнене від \emph{проросійських} бойовиків напередодні складання
Порошенком президентської присяги – це Красний Лиман. Нещодавно читав пафосний
текст одного письменника, де описується, як путінські танки вже прогрівали
мотори на кордоні, щоб іти на Київ, письменник навіть уточнює, що до вторгнення
лишалися лічені години,
\textbf{П'ять цікавих фактів – від \emph{УПА} до АТО}, Павло Зуб'юк, zaxid.net, 01.06.2021

\emph{Росія} – вічний супротивник України, Роман Кізима, zaxid.net, 28.05.2021

Українська держава, на жаль, приречена співіснувати зі своїм \emph{північно-східним
сусідом}. Враховуючи історичний досвід такого співжиття можна говорити, що
теперішня \emph{Російська Федерація} є екзистенційним ворогом/опонентом/супротивником
України. І так було завжди... Не будемо вдаватися в глибокі історико-філософські
роздуми причин такого відношення до нас з боку \emph{Росії-Московії}, але повинні
сприйняти таку реальність як данність, з якою нам жити,
\textbf{\emph{Росія} – вічний супротивник України}, Роман Кізима, zaxid.net, 28.05.2021

Частково спільна історія, певні спільні слов'янські корені лише підкреслюють
категоричну нашу різність в світоглядному та державницькому розвитку.
\emph{Росія} як один із осколків Київської \emph{Русі}, попавши під вплив
Золотої Орди, - стала її повним відображенням і продовженням. Саме тоді
відбувся екзистенційний розкол двох цивілізацій між так званим європейським,
умовно демократичний шляхом розвитку та східним, авторитарний методом
функціонування держави. В подальшому усі «фасадні» дії \emph{Московії}, як то
наприклад, фактична крадіжка слова \emph{Русь-Росія}, прийняття християнства,
реформи Петра I не відміняли «азійськості» у їх державотворенні,
\textbf{\emph{Росія} – вічний супротивник України}, Роман Кізима, zaxid.net, 28.05.2021

На превеликий жаль, така самовизначеність \emph{Росії} усі подальші періоди проходили
болюче для її сусідів, яких імперія хотіла або поглинути або підпорядкувати.
Причиною таких агресивних дій окрім банально імперських був постійний конфлікт
всередині \emph{Московії} між її елітами,
\textbf{\emph{Росія} – вічний супротивник України}, Роман Кізима, zaxid.net, 28.05.2021

Частина таких еліт вважала себе європейцями і відповідно вбачала в існуванні
більш «прозахідних» інших слов'янських цивілізацій загрозою своєї
«європейськості». Ми, українці, \emph{білоруси}, поляки та інші, ніби відтіняємо їх
своєю більшою приналежністю до Європи. Таким чином, з одного боку ми жертви
імперськості, з іншого - боротьби за спадок європейської цивілізації. Такі дії
нашого агресивного сусіда не одноразово приводили до втрати нашої державності.
Радянський період окупації України був черговим болючим етапом
\emph{російсько}-українського протистояння, який завершився винищенням мільйонів
українців, заселенням нашої території громадянами інших національностей та
\emph{русифікацією} більшості наших міст,
\textbf{\emph{Росія} – вічний супротивник України}, Роман Кізима, zaxid.net, 28.05.2021

Україно-\emph{російська} війна 2013-2021 років розглядалася і розглядається
теперішньою \emph{російською} владою як період остаточного вирішення «українського
питання». Путін неодноразово заявляв і всередині країни і західним партнерам,
що Україна не заслуговує на власну державу. Частина її, мовляв, належить
\emph{Росії}, а частину пропонувалося розділити між східно-європейськими країнами.
Звідси і план \emph{«Новоросія»}, який на щастя не вдалося реалізувати. Закінчилося
усе на даному етапі «лише» окупацією частини нашої території,
\textbf{\emph{Росія} – вічний супротивник України}, Роман Кізима, zaxid.net, 28.05.2021

Усім напевно відомі \emph{російські} дефініції, мовляв, без України не буде \emph{Російської}
імперії... Центробіжні і центровідштовхувальні тенденції, які відбувалися в
\emph{російській} імперії, впродовж її історії не міняли її, \emph{Росії}, ставлення до нашої
державності. Сподіватися на те, що демократична \emph{Росія} майбутнього змінить на
180 градусів свою політику по відношенню до своїх західних сусідів звичайно
можна. Але враховуючи остання 300 років історії такі сподівання є як мінімум
наївними, а фактично – небезпечними і навіть злочинними. Ми мали 22 роки
невеличкого перепочинку, який мали би використати на побудову власної
державності. На жаль, справилися ми зі своїм «домашнім завданням» на слабеньку
трієчку,
\textbf{\emph{Росія} – вічний супротивник України}, Роман Кізима, zaxid.net, 28.05.2021

Центр протидії дезінформації є лише маленькою частинкою такої політики. Вважаю,
що потрібно створити «інститут \emph{Росії}», як науковий центр, де би на державному
рівні вивчалася внутрішня та зовнішня політика сусіда, тенденції, які там
відбуваються, вивчення думок основних політичних та громадських лідерів.
Необхідна постійна робота по прогнозуванню різних сценаріїв розвитку \emph{Росії} та
її зовнішньої політики в Європі в короткій, середній та довгій перспективі,
\textbf{\emph{Росія} – вічний супротивник України}, Роман Кізима, zaxid.net, 28.05.2021

Итак, в 2018 году Украина вырвалась вперед всей Европы по числу разводов,
обогнав даже \emph{Россию}. Тогда на 228 411 заключенных за год браков пришлось
153 949 разводов (67,4\%). В 2019 году этот показатель улучшился до 58\%. Но
вот в прошлом 2020-м нас постигла настоящая катастрофа, вызванная резким
сокращение браков: за год украинцы сыграли всего лишь 167 974 свадьбы – это
самое низкое число с 1944 года! А ведь тогда в Украине и женихов почти не было,
всех мобилизовали, куда же они подевались теперь?
\textbf{Почему разваливается \emph{украинская} семья?}, Иван Пургин, from-ua.com, 02.06.2021

Фактически, данный документ можно назвать объявлением войны \emph{России}. Нашей
стране отказано во всём. Перечислим ряд мифов, от которых, по мнению  Chatham
House, должны избавиться западные политики и государственные деятели:
утверждение, что народы Украины, \emph{БелаРуси} и \emph{России} - это единая нация, Крым
всегда был \emph{русским}, \emph{России} обещали, что НАТО не будет расширяться, \emph{Россия} имеет
право на «оборонительный периметр» - сферу своих привилегированных интересов,
куда входят и территории других государств, санкции - это неверный подход,
\textbf{\emph{России} пора переходить к жесткому решению «британского вопроса»}, 
Александр Владимиров, voskhodinfo.su, 28.05.2021

Британский истеблишмент – давний, очень умный, опасный и на вечно враг \emph{России},
поэтому какое-либо сотрудничество с ним просто невозможно (impossible). Попытки
наладить отношения будут пустой тратой времени, сил и гарантированно будут
использованы против самой \emph{России}.  Тот факт, что Лондон облюбовали некоторые
\emph{российские} олигархи, имеет мало отношения к коренным государственным интересам
\emph{России} и \emph{её народа}. На этих меценатов британской модели также следует обратить
внимание, т.к. у них сердце там, где их деньги, т.е. не в \emph{России}. А
использование Британией их потенциала в качестве пятой колонны и против \emph{России}
вполне возможно,
\textbf{\emph{России} пора переходить к жесткому решению «британского вопроса»}, 
Александр Владимиров, voskhodinfo.su, 28.05.2021

Это даже видно на примере недавнего высказывания польского президента о \emph{России},
как «не нормальной стране». Британия задаёт тезисы и устойчивые выражения, а
остальные дружно повторяют. С учётом наличия у \emph{России} и США стратегических
ядерных сил и необходимости поддержания стратегического паритета какими бы
плохими не были отношения между странами, и американцам, и нам нужен постоянный
контакт и диалог по вопросам стратегической стабильности,
\textbf{\emph{России} пора переходить к жесткому решению «британского вопроса»}, 
Александр Владимиров, voskhodinfo.su, 28.05.2021

Тон данного документа задан в его преамбуле: «Этот доклад деконструирует 16
наиболее распространенных мифов и заблуждений, которые формируют современное
западное мышление о \emph{России}. В нём объясняется их пагубное влияние на
разработку и осуществление политики, и в каждом случае описывается, как позиции
Запада нуждаются в критическом пересмотре, чтобы обеспечить более рациональные
и эффективные ответы на действия \emph{России}. В основе нашего анализа лежит
важный аргумент о том, что, вопреки ожиданиям многих евроатлантических
политиков и государственных деятелей, в обозримом будущем у \emph{России} мало
шансов стать более конструктивным и сотрудничающим партнером для западных
правительств. Таким образом, благонамеренные усилия по «улучшению» отношений с
Кремлем, скорее всего, потерпят неудачу, поскольку стратегические цели,
ценности и понимание \emph{Россией} межгосударственных отношений безвозвратно
отличаются от западных»,
\textbf{\emph{России} пора переходить к жесткому решению «британского вопроса»}, 
Александр Владимиров, voskhodinfo.su, 28.05.2021

Справедливость и закон в \emph{России}, Андрей Медведев, voskhodinfo.su, 01.06.2021

Бацька идет на Восток. Как санкции Запада и Украины толкают Лукашенко к интеграции с \emph{Россией},

Обвиненный в \enquote{российской пропаганде} Netflix опубликовал список фильмов, попавших под цензуру в разных странах,
strana.ua, 08.02.2021

Запад никак не может определиться, отсталая \emph{Россия} или передовая, То она - \emph{Страна-бензоколонка}, то - \emph{питомник хакеров},
Максим Войтенко, strana.ua, 02.06.2021

Пришли новые вести о \emph{русских хакерах}. Только они отдохнули от взлома
SolarWinds и нефтепровода на Восточном побережье США, как Путин приказал им
ломать сетку крупнейшей в мире мясоперерабатывающей компании. Команду хакеров
усилили Петровым и Бошировым, парой ломов для взлома серверов и все заверте.
«Из-за инцидента компании пришлось отменить смены на канадском заводе и на
некоторых предприятиях в США, а в Австралии пришлось полностью остановить
работу». Белый Дом строчит обращения к Кремлю, общественность паникует и везде
ищет руку Кэй-Джи-Би. Если уж не шутить, то скажу следующее. Я бы эти новости
на месте западных и особенно украинских пропагандистов так не форсил. Потому
что у маломальски думающего человека возникает лютый диссонанс. С одной стороны
мем \emph{«Страна-бензоколонка»}, а в украинском патриотическом
медиа-пространстве вообще рассуждения о \emph{московском улусе}, варварах и
шестисотлетней борьбе (с момента падения украинского стольного града Коцюбиева)
с варварами,
\textbf{Запад никак не может определиться, отсталая \emph{Россия} или передовая, То она - \emph{Страна-бензоколонка}, то - \emph{питомник хакеров}},
Максим Войтенко, strana.ua, 02.06.2021

– А что тут вспоминать? Войну начал Киев. Дом мне разбомбила Украина. Кто же
еще? Я помню, как они стояли в Георгевке и стреляли по Роскошному. И что, надо
было мой дом бомбить? Залетел шальной снаряд. Как по мне, все очевидно – \emph{Россия}
нас защищает, а Украина стреляет. И те луганчане, которые уехали, то есть
сбежали – изменники Родины. Но и хорошо. Ведь если ты не веришь в «ЛНР», тебе
здесь делать нечего,
\textbf{«Никто нам не вернет украденные годы»: как в \emph{Луганске} вспоминают захват города},
Донбас.Реалії, Записки з окупації, radiosvoboda.org, 02.06.2021

Екатерина, учительница: идейных я уже не вижу, – Я никогда не поддерживала эту
\emph{«русскую весну»} и не верила в агрессию Украины. Для большинства населения эти
идеи \emph{русского мира} оказались близки, думаю, из-за низкого уровня
образованности. Промзона, работяги – тут всегда было не до развития
интеллектуальных способностей. Сейчас самые упертые не верят уже ни в
республику, ни в \emph{РФ}. Идейных я уже не вижу. Основное настроение здесь – люди
смирились с ситуацией. Печально все и пока не вижу выхода,
\textbf{«Никто нам не вернет украденные годы»: как в \emph{Луганске} вспоминают захват города},
Донбас.Реалії, Записки з окупації, radiosvoboda.org, 02.06.2021

«Товарищи политруки!» \emph{Росія} диктує, як розповідати про Шарія, «СС Галичину» і Зеленського,
radiosvoboda.org, 02.06.2021

Во-вторых, наивно выглядят и претензии к Западу во внешнеполитических вопросах.
Взять, например, жалобы на то, что Германия и Франция отказываются признать
\emph{Россию} стороной войны на Донбассе, т.е. признать якобы факт
\emph{российско}-украинской войны. Просто руководители этих стран, в отличие от
Зеленского, не руководствуются и не обязаны руководствоваться нарративами
украинской пропаганды. У них есть собственное, объективное понимание
происходящего на Донбассе, а вовсе не \enquote{боязнь \emph{России}}, и это их понимание надо
уважать. Украину не берут и, вероятно, не возьмут в НАТО в том числе еще и
потому, что украинское руководство не в состоянии объективно оценивать
международную стуацию, да и ситуацию в собственной стране, и находить баланс
интересов, 
\textbf{Интервью Зеленского отражает слабость украинской внешней политики,
Никому не интересна страна, которая всегда попрошайничает}, Виктор Суслов,
strana.ua, 03.06.2021

Неожиданный (хотя и не слишком удивительный) скандал произошел вокруг
американского стриммингового сервиса \enquote{Нетфликс}, который разместил у себя два
\emph{российских} фильма – \enquote{Брат} и \enquote{Брат-2} - с английскими субтитрами. Собственно,
сами субтитры и привели к скандалу. Наверное, никому не нужно объяснять, что
во втором \enquote{Брате} есть пара фраз, которые в контексте нынешней украинской
идеологии выглядят крамольными. Одна из них – звучит в диалоге между старшим
братом главного героя и представителем украинской мафии: \enquote{Слышь, земляк, а где
здесь \emph{русские} живут?} - \enquote{Москаль менi не земляк}. – \enquote{Бандеровец?},
\textbf{Операция \enquote{Антиолигарх}, антибандеровский \enquote{Нетфликс},
МИД Украины против \enquote{БелАвиа}. Итоги \enquote{Страны}}, strana.ua,
03.06.2021


Кабмін і СБУ також мають опрацювати питання щодо відкликання мережевих ресурсів
(ІР-адрес), які виділені у користування операторам (провайдерам)
телекомунікацій, що перебувають в ОРДЛО або використовуються в інформаційній
агресії \emph{Росі} проти України,
\textbf{Зеленський \enquote{активізував} дії щодо Донбасу: правосуддя, IP-адреси, розвідка},
pravda.com.ua, 03.06.2021

Я написав своїй співрозмовниці, що чекаю біля цієї сцени і почав спостерігати
за цим дійством.  На сцену виходив рабінович, він запрошував депутатів із свої
секти. Вони кричали про любов до нашої країни, землі і тому подібне. Більшість
з цих покидьків говорили \emph{російською}.  Чогось згадалося минуле століття, коли
діди цього рабіновича, медведчука, ківи, королевської - вбивали багатих
українців, називали їх куркулями і знищували заможний український клас,
\textbf{Чому партія \enquote{опзж}, морить своїх людей голодом?}, 
Михайло Ухман, censor.net.ua, 03.06.2021

Іду біля будинку парламенту, де сьогодні чорти з опзж влаштували зібрання.
Навпроти мене вийшло два московських попи і чоловік у жилетці з емблемою секти
любителів \enquote{\emph{расєї}}! Підійшовши до них, побачив, що чоловік геть \enquote{пропитий}, а
попи в брудних накидках і немитими бородами. Один з них побачив мою цікавість
ними і теж ковзнув поглядом по мені. Його очі зупинилися на кілька секунд на
моєму лівому вусі, де гордо виблискував Тризуб. Що було далі, ви навіть не
уявляєте? Мій тризуб - це як осиковий кіл для вампірів. Для всіх чортів, які
підтримують партію регіонів, медведчука, моспатріархат, він має таку ж дію,
\textbf{Чому партія \enquote{опзж}, морить своїх людей голодом?}, 
Михайло Ухман, censor.net.ua, 03.06.2021

16 травня 2015 року після 14 години, в ході перестрілки біля містечка Щастя
Луганської області, бійці 92 бригади ЗСУ, затримали кадрових офіцерів ГРУ
\emph{Російської} федерації: сержанта Олександра Олександрова і капітана Євгена
Єрофиєва з Третьої окремої бригади спецназу міста Тольятті.  В ході
перестрілки, нажаль загинув український воїн, який першим побачив окупантів.
Два вище згаданих персонажі, отримали поранення, але вижили. Один із
\emph{російських} розвідників, хотів підірвати гранату, яка була в його кишені,
але не зміг цього зробити, через поранену руку. Інший не встиг використати
автомат, який знаходився за метр від нього, адже теж лежав поранений під
прицілом українських воїнів,
\textbf{\emph{Російські} офіцери в українському полоні},
Михайло Ухман, censor.net.ua, 17.05.2021

Знаєте, що саме цікавіше - цих російських \enquote{героїв} Кремль нагороджує, як
захисників країни, хоча вони звичайні окупанти. Як потім виявляється, нікому
непотрібні: \emph{росії}, родинам, які відмовляються від них... Вони стають
викиднями...  Така доля чекає і сепаратистів, так званих вояк \enquote{днр, лнр}! Їх не
визнає \emph{росія}, світ - вони стають покидьками, котрі зрадили своїй країні. І
тепер приреченні на існування та пригноблення. Така доля чекає всіх, хто
намагається воювати проти України, 
\textbf{\emph{Російські} офіцери в українському полоні},
Михайло Ухман, censor.net.ua, 17.05.2021

Один із \emph{російських} розвідників, хотів підірвати гранату, яка була в його
кишені, але не зміг цього зробити, через поранену руку. Інший не встиг
використати автомат, який знаходився за метр від нього, адже теж лежав
поранений під прицілом українських воїнів.  Як згодом вдалося вияснити, що на
початку березня 2015 року, з \emph{росії} в Луганськ прибуло двісті чоловік
їхнього спецназу. Це були розвідники, бійці технічного резерву. Командував
всіма ними, один із генералів ФСБ \emph{московії},
\textbf{\emph{Російські} офіцери в українському полоні},
Михайло Ухман, censor.net.ua, 17.05.2021

Абсолютно необучаемый пони. Вот совершенно. \enquote{Открытую \emph{Россию} закрыли - скоро
\emph{Россия} будет свободной!}. \enquote{Рейс остановили на рулежке толпой Росгвардии - нас
большинство!} \enquote{Меня везут в Краснодар - как они нас боятся!} \enquote{Мне раздавили
яички сапогом - как Путин зассал!} \enquote{Мне вырвали все ногти - боже, какие мы
прекрасные, как мы ломим!} \enquote{Мне пробили голову прикладом на прииске Колымы и
скинули в общую яму с трупами в вечной мерзлоте - Прекрасная \emph{Россия} Будущего!}
Именно Андрею, вот конкретно ему, я несколько раз говорил в глаза в упор - что
вы делаете, идиоты, вас всех перетравят, перебьют, пересажают. Бегите, глупцы.
Неа. Прекрасная \emph{Россия} будущего, шарики, умное голосование, мы идем на выборы,
ой, нас опять наебали, ой, опять все в кутузке, ой, опять произошла какая-то
ошибка, ой, куда вы меня тащите,
\textbf{Новый тридцать седьмой и \emph{Россия} будущего}, Аркадий Бабченко, 03.06.2021

Смотрю по 1 каналу \emph{российского} телевидения (бывшее ОРТ) прямую
трансляцию из Питера – открытие Второй сцены Мариинского театра оперы и балета:
Пласидо Доминго, Анна Нетребко, Денис Мацуев, множество чрезвычайно талантливых
певцов и балерин – мировых и \emph{российских} звезд, Путин в зале, и конечно
Гергиев, Гергиев, Гергиев, которому исполнилось 60 лет.  Смотрю и, кажется,
начинаю понимать – почему в Украине ничего подобного невозможно, почему Украина
– это туша, которую уже распилили и продали, а \emph{Россия} – феодальная
страна, в которой иногда случаются интересные события – но только при
перенапряжении усилий и ресурсов и по высочайшему повелению,
\textbf{Мариинский-2 и украинская культура: Почему Украина – это фабрика человеческого мяса? (обн.)},
Ірина Славінська, pravda.com.ua, 03.05.2013

В нормальной стране высокое искусство существует, потому что существует
представление об \enquote{общем благе}, разделяемое и гражданским обществом, и властью,
а также – обеспеченная государством система экономической конкуренции и
традиция меценатства.  В \emph{России} высокое искусство существует, потому что царей
убеждают – мол, надо дать денег на высокое искусство. Искусство – это фактор
имперского престижа \emph{России}, это инструмент \emph{российского} влияния на мировую
политику (особенно любят в этом контексте вспоминать \enquote{Большой Балет}), Soft
Power и т.д. (В этом контексте следует рассматривать и награждение Гергиева
звездой Героя Труда N1.),
\textbf{Мариинский-2 и украинская культура: Почему Украина – это фабрика человеческого мяса? (обн.)},
Ірина Славінська, pravda.com.ua, 03.05.2013

Пьесы и рассказы 12-летней луганской писательницы и драматурга Фаины Савенковой
переведены на разные языки мира, они входят в шорт-листы престижных
всероссийских премий и конкурсов, включая наиболее престижный конкурс Гильдии
драматургов \emph{России} «Автора на сцену»,
\textbf{Фаина Савенкова. Верить и надеяться}, antifashist.com, 02.06.2021

В качестве инструмента воздействия на украинскую \emph{русофобию} можно создать
телеканал на \emph{русском} языке по украинской проблематике, а также
использовать такие источники, как интернет и радио, направленные против
пропаганды Незалежной, сказал в интервью изданию Украина.ру \emph{российский}
политолог Сергей Марков,
\textbf{Украинская \emph{русофобия} становится все более радикальной. Марков поделился инструментом для борьбы с ней},
ukraina.ru, 03.06.2021

Виважене і шанобливе ставлення українців до жіноцтва було поширене серед усіх
соціальних станів і за \emph{Київської Русі}, і після її Хрещення. Права жіноцтва
були вписані до \enquote{\emph{Руської} Правди} – зведення тогочасних законів
\emph{Київської Русі}, виконаного Ярославом Мудрим на початку XІ століття,
\textbf{Шанобливе ставлення до жіноцтва в \emph{Україні}},
Марія Гуцол, slovoprosvity.org, 17.05.2021

Мы все, выросшие в \emph{России} - внуки жертв и палачей. Все абсолютно, все, без
исключения. В вашей семье не было жертв? Значит были палачи. Не было палачей?
Значит были жертвы. Не было ни жертв, ни палачей? Значит...,
\textbf{Бред сумасшедшего! Как с этим жить?}, Владимир Яковлев, news24ru.net, 03.06.2021

Мне кажется, мы сильно недооцениваем влияние трагедий \emph{российского} прошлого на
психику сегодняшних поколений. Нашу с вами психику. По сей день, прощаясь, мы
говорим друг другу - \enquote{До свидания!}, не сознавая, что
\enquote{свидание} вообще-то слово тюремное. В обычной жизни бывают встречи,
свидания бывают в тюрьме.  По сей день мы легко пишем в смсках: \enquote{Я
напишу, когда освобожусь!} Когда ОСВОБОЖУСЬ...  Оценивая масштаб трагедий
российского прошлого, мы обычно считаем погибших. Но ведь для того, чтобы
оценить масштаб влияния этих трагедий на психику будущих поколений, считать
нужно не погибших, а - выживших,
\textbf{Бред сумасшедшего! Как с этим жить?}, Владимир Яковлев, news24ru.net, 03.06.2021

Погибших - десятки миллионов. Выживших - сотни миллионов. Сотни миллионов тех,
кто передал свой страх, свою боль, свое ощущение постоянной угрозы, исходящей
от внешнего мира - детям, которые, в свою очередь, добавив к этой боли
собственные страдания, передали этот страх нам. Просто статистически - сегодня
в \emph{России} нет ни одной семьи, которая так или иначе не несла бы на себе
тяжелейшиe последствия беспрецедентых по своим масштабам зверств,
продолжавшийся в стране в течение столетия,
\textbf{Бред сумасшедшего! Как с этим жить?}, Владимир Яковлев, news24ru.net, 03.06.2021

Истеричный лукашизм, или Подарок для Путина, Во избежание спекуляций как
с той, так и с другой стороны автор сразу же отметит, что, имея белорусские
корни, а также придерживаясь старой доброй советской традиции, с искренней
симпатией относится к сябрам, причем прежде всего именно к \emph{Белоруссии} и к
близкородственным нам, выросшим с нами из одного \emph{древнерусского} и советского
корня братьям-\emph{белорусам}. Поэтому вся эта \emph{антибелорусская}, совершенно идиотская,
скотская, подонковская, гнусная, ублюдочная (далее уже только нецензурные
выражения!) истерия, которая развернута в Украине в последние дни в исполнении
разного рода \enquote{актывиздов} и Зе-власти, вызывает рвотный рефлекс и классовую
ненависть,
\textbf{Европа впала в иррациональный истеричный лукашизм}, Александр Карпец,
strana.ua, 03.06.2021

Первым эту дурашливую стрельбу куда попало начал сам \emph{белорусский} бацька,
эффектно приземливший на нары некую недоросль по фамилии Протасевич и
окончательно толкнувший в изоляцию \emph{Беларусь}, создав огромные проблемы
всем ее гражданам, включая фанатов Лукашенко, которых в \enquote{рэспубліке}
остается еще очень много. Тем самым Лукашенко продолжил череду своих
\enquote{epic fail-ов}, начатую, как минимум, во время выборов прошлого года,
\textbf{Европа впала в иррациональный истеричный лукашизм}, Александр Карпец,
strana.ua, 03.06.2021

Бывший главный редактор Телеграм-канала \enquote{Nexta} Роман Протасевич дал большое
интервью программе \enquote{Марков. Ничего личного}, которая выходит на белорусском
телеканале ОНТ. В нем он выразил свое мнение о президенте Беларуси Александре
Лукашенко.  По словам Протасевича, Лукашенко \enquote{поступал как человек со стальными
яйцами}. Задержанный оппозиционер добавил, что \enquote{безусловно} уважает президента
Беларуси, 
\textbf{Протасевич заявил, что Лукашенко поступает как человек со стальными яйцами. Видео},
, strana.ua, 03.06.2021

Через атаку хакерів зупинилися м'ясозаводи у США: ФБР звинувачує \emph{Росію},
ФБР звинуватило \emph{Росію} в організації кібернападу на найбільшого світового
виробника м'яса – компанію JBS. Росія має це припинити і не надавати прихисток
злочинцям, заявив Держсекретар США.  За кібернападом на найбільшого світового
виробника м'яса, компанію JBS, стоїть пов'язана з \emph{Росією} група хакерів
Revil, повідомили в ФБР, radiosvoboda.org, 03.06.2021

\emph{Росія} має зупинити подібні кібернапади і «не надавати прихисток злочинцям,
причетним до цих нападів», заявив 3 червня Держсекретар США Ентоні Блінкен, в
інтерв'ю іспанському виданню.  «На жаль, ми бачимо новий фронт кіберзагроз і це
злочинні організації, що використовують так звані «програми-вимагачі», щоб
тримати в заручниках компанії, тримати в заручниках критичну інфраструктуру
задля фінансового зиску», – заявив Блінкен. «Обов'язок кожної країни робити
все можливе, щоб знайти ці компанії, притягнути їх до відповідальності,
включаючи питання нападу на Colonial Pipeline. Компанія, відповідальна за цей
напад, її керівники були і тепер перебувають в \emph{Росії}, тому, гадаю, \emph{Росія}
зобов'язана зробити так, щоб таке не продовжувалось», – додав посадовець,
\textbf{Через атаку хакерів зупинилися м'ясозаводи у США: ФБР звинувачує \emph{Росію}},
radiosvoboda.org, 03.06.2021

Перемога Джамали викликала значні суперечки. Пісня сама по собі була
незаперечно чудова, та критики твердили, що вона надто відверто політизована і
загрожує підірвати легковажну природу пісенного конкурсу. Та попри ці
нарікання, пісня «1944» стала великим тріумфом «м'якої сили» України. Вона
знову привернула міжнародну увагу до страждань кримських татар, які живуть у
Криму під \emph{російською} окупацією. Вона також підвищила зацікавленість до теми
радянських злочинів проти людяності і спростувала намагання Москви неправдиво
твердити, ніби Крим «завжди був \emph{російським}»,
\textbf{Чому «Євробачення» – таємна зброя «м'якої сили» України}, radiosvoboda.org, 02.06.2021

Виступ Вєрки також створив одну з найменш імовірних і найбільш пам'ятних
політичних суперечок в історії «Євробачення», коли \emph{російські} джерела
протестували проти того, що в українському номері, мовляв, є слова «\emph{Russia},
goodbye». Андрій Данилко, виконавець ролі Вєрки Сердючки, запевняв, що
насправді слова були «лаша тумбай». Це, заявляв він дещо непереконливо, нібито
фраза монгольською мовою, що означає «збиті вершки»,
\textbf{Чому «Євробачення» – таємна зброя «м'якої сили» України}, radiosvoboda.org, 02.06.2021

Відкриваючи концертну програму творчого колективу, зі словами вдячності до
гостей вечора звернулася художній керівник та диригент капели, начальник служби
з питань культури і діаспори НКЦУ, заслужений працівник культури України та
\emph{Росії}, голова РГО «Культурно-просвітницький центр українців у м. Москві»
Вікторія Скопенко.  Протягом вечора у виконанні Української народної хорової
капели Москви та її солістів прозвучала українська духовна музика, обробки
народних пісень, етнографічні та вокально-хореографічні композиції, авторські
твори, поезія Тараса Шевченка, Лесі Українки,
\textbf{Пісенне джерело натхнення}, ukrcentr.ru, 28.05.2021

\enquote{Ми продовжуємо суворо засуджувати атаки \emph{Росії} на цивільне
населення. Вони мають бути зупинені. Натомість \emph{Росії} слід зосередитись
на виконанні своєї частини безпекових положень Мінських домовленостей}, -
заявив Цимбалюк,
\textbf{Внаслідок російської агресії на сході України загинули 240 дітей, -
Україна в ОБСЄ}, day.kiev.ua, 03.06.2021

\enquote{Меня назовут предателем}. О чем исповедь Романа Протасевича, который заплакал в эфире \emph{белорусского} ТВ,
strana.ua, 03.06.2021

Но вернемся к \enquote{Нехте}. По информации Протасевича, создать канал - это была идея
журналиста Владимира Чуденцова, а не Степана Путило. В дальнейшем в адрес
Путило будет много выпадов - да Протасевич и сам признает, что у него с ним
личностный конфликт. Он говорит, что владелец \enquote{Нехты} был более высокого
уровня \enquote{посвящения}, имел охрану и контролировал деньги проекта. А также
допускался к \enquote{специалистам}. Протасевич рассказал, что было и \emph{российское}
финансирование \enquote{Нехты},
\textbf{\enquote{Меня назовут предателем}. О чем исповедь Романа Протасевича, который заплакал в эфире белорусского ТВ},
strana.ua, 03.06.2021

Протасевич говорит, что против Лукашенко действительно готовились покушения.
Одна из групп - команда Автуховича, которого ранее задержали в Беларуси.  Также
он говорит, что общался с двумя людьми, которых в \emph{Беларуси} и \emph{России} обвиняют в
подготовке покушения и госпереворота - Юрием Зенковичем и Дмитрием Щигельским.
Протасевич признался, что выступал посредником между заговорщиками,
планировавшими убить Лукашенко, и штабом Тихановской.  \enquote{Я должен был быть
связующим звеном между заговорщиками и штабами}, - говорит Протасевич. Вот
полный диалог на тему вооруженного захвата власти в \emph{Беларуси}, который якобы
планировался,
\textbf{\enquote{Меня назовут предателем}. О чем исповедь Романа Протасевича, который заплакал в эфире белорусского ТВ},
strana.ua, 03.06.2021

Исследование показало, что в \emph{России} показатель смертности занизили в 5,4
раза - 593,6 тысяч смертей из-за коронавируса против 109,3 тысяч по официальным
данным,
\textbf{В Украине от Covid-19 умерло втрое больше людей, чем указано в официальной статистике - исследование},
Юлия Супрун, strana.ua, 07.05.2021

Ну і окрема тема. Порівняння з \emph{Росією}. У нас дуже популярна тема, що в
\emph{Росії} Путін подолав олігархів, і одразу встановив автократію (ну а за
домінування Березовського та компанії там, очевидно, була демократія), тож
наших олігархів треба оберігати, захищати, і слідкувати, аби хтось їх не
образив. Думка, звісно, цікава, але вона відображає відрив інтелектуальної
еліти пострадянського простору від реалій цього простору,
\textbf{Власть \emph{олигархов} сможет преодолеть только обновленное общество},
Петр Олещук, strana.ua, 04.06.2021

Помню, как после государственного переворота и прихода к власти майдана, в
эфире ТВ рыдал бывший \enquote{регионал} Мирошниченко. Так всегда бывает: либо рыдают
те, кого иностранцы швырнули совершать переворот в своей стране, либо — те,
кого \enquote{перевернули}.  Рад за \emph{БелаРусь}, что не доведётся повторить печальный опыт
Украины, которую \enquote{цивилизованный запад} превратил в свою колонию, и только то и
делает 7 лет, что грабит свои территори, и население, живущее на них,
\textbf{Сотрудничество с властью и слезы - это выбор самого Протасевича},
Александр Скубченко, strana.ua, 04.06.2021

Даже \enquote{антироссийскость} балтов и поляков - это тоже их национальный интерес,
потому что им за это платят миллиарды евро.  Без этой \enquote{антироссийскости}, их
функция в системе коллективной безопасности НАТО или балансе сил ЕС,
существенно снизилась бы. Практически до нуля.  Если бы им за это не платили,
они тут же активизировали бы переговоры с РФ о совместных транзитных проектах
или доступе товаров на рынки друг друга,
\textbf{Америка поняла, что может добиться своего от Украины за гроши},

США готовят наступательную операцию против хакеров из \emph{России} - СМИ,
Елена Вьюн, strana.ua, 04.06.2021

Також \emph{російську} лекцію відвідали британський фотограф Дін О'Браян та французька
документалістка Анна-Лаура Боннел, які підтримують \emph{російські} наративи про
«громадянську війну на Донбасі».  Захід вів постпред \emph{Росії} при ООН Василь
Небензя,
\textbf{«Азаров та Царьов у Раді безпеки ООН»: що насправді відбулося і в чому маніпуляція?},
Донбас.Реалії, radiosvoboda.org, 03.06.2021

Україна нібито воює на сході з власними громадянами, і воювати їй допомагає
підтримка Заходу.  Що насправді: участь \emph{Росії} у війні на Донбасі доведена
десятками міжнародних розслідувань, і вже давно. Ось підбірка з трьох десятків
із них, і це ще не все. Якщо узагальнити коротко, дотепер ідентифіковані
близько 2500 \emph{російських} військовослужбовців і 101 підрозділ регулярних військ,
що брали учать в боях на території України, а також 53 види військової техніки
армії \emph{Росії}, яка не перебуває на озброєнні України,
\textbf{«Азаров та Царьов у Раді безпеки ООН»: що насправді відбулося і в чому маніпуляція?},
Донбас.Реалії, radiosvoboda.org, 03.06.2021

Як насправді: хронологія і суть подій абсолютно інша. \emph{Проросійські} акції у
містах Донбасу навесні 2014-го не стикалися із жодною протидією, а особи, які,
як пізніше стало відомо, приїхали з \emph{Росії}, підбурювали жителів Донецька та
Луганська до протиправних дій, а також брали безпосередню участь в організації
нападів на зібрання Євромайдану (аж до вбивства людини). Потім загони під
керівництвом \emph{російських} офіцерів захопили органи влади. Далі зі зброєю
штурмували українські військові частини та аеропорт. Лише у відповідь на це
влада України почала збройну відповідь,
\textbf{«Азаров та Царьов у Раді безпеки ООН»: що насправді відбулося і в чому маніпуляція?},
Донбас.Реалії, radiosvoboda.org, 03.06.2021

Представник Франції, котрий взяв участь в організованій \emph{Росією} зустрічі,
заявив, що зробив це лише для того, щоб заявити, що \emph{Росія} зловживає
своїм правом ініціювати ці неформальні зустрічі. Він додав, що Франція не
підтримує наративи \emph{Росії} та закликає Москву зупинити провокації, вивести
війська зі сходу України, а також повернути Крим.  Аналіз наративу, який
розповсюджують \emph{російські} пропагандистські ресурси як у самій \emph{Росії}, так
і на окупованих нею територіях України, а також у світі – вказує на наявність
створених Кремлем «темників», що спотворюють хронологію і зміст подій, з метою
приховати акт прямої агресії \emph{Росії} і перекласти відповідальність на
Україну, яка власне і стала жертвою цієї агресії,
\textbf{«Азаров та Царьов у Раді безпеки ООН»: що насправді відбулося і в чому маніпуляція?},
Донбас.Реалії, radiosvoboda.org, 03.06.2021

Як розповідає у своєму дослідженні Сергій Камінський, мажорський International
MXT отримав броньовані панелі на скло і захищену капсулу в кузові для кулемета.
У боях за Дебальцеве машина потрапила під потужний обстріл дружньої для сім'ї
Януковичів \emph{російської} артилерії і перетворилася на купу металобрухту,
\textbf{Шушпанцери і бандеромобілі. Героїчна історія саморобних броньовиків у війні за український Донбас},
Євген Руденко; Дмитро Ларін, pravda.com.ua, 04.06.2021

А втім, «остаточного прощавай» не сталося. На жаль, хоч українці (в переважній
більшості) і перестали споживати \emph{російський} інформаційний шлак безпосередньо та
некритично, вони не припинили дивитися на \emph{Росію} крізь різні додаткові призми.
Хтось пильнує, як висміюють «вату», когось не відпускають «незаангажовані та
аполітичні» зірки Youtube, багатьох манить музика, де вони спроможні зрозуміти
хоч якісь слова. Знову ж таки, варто пам'ятати і про опозицію режиму чи
відцентрові рухи народів \emph{федерації},
\textbf{«Вікно в Європу»}, Назар Кісь, zaxid.net, 03.06.2021

До речі, ще в доцифрову епоху ситуація, принаймні в Західній Україні, була
кращою. Можна було покрутити антену і подивитися, що там робиться. Зараз антену
крутити не потрібно, достатньо поклацати пальцями по клавіатурі. Але ж простіше
глянути, як там змагання з копання могил \emph{за Уралом}. Що сказав Дудь, наприклад,
дуже цікаво знати. Чи не залишилося в \emph{«глубінкє»} чогось доброго, світлого, не
ворожого Україні?  Так і живемо вже 30 років незалежності. Ніби і «ящик» уже
переналаштували на режим «в Україні/у світі», але запити в пошукових вперто
свідчать про те, що орієнтуємося на \emph{«1/6 частину суші»},
\textbf{«Вікно в Європу»}, Назар Кісь, zaxid.net, 03.06.2021

Страйк жінок? Свобода преси? Протистояння ПіС та ГП? Корупційні скандали?
Стосунки Варшави з Брюсселем? Це все нас не цікавить. Це вже коли людина
емігрує з України до Польщі, тоді й дізнається. Якщо вийде з «інформаційного
ґетто» україно- чи \emph{російськомовних} ресурсів, які допомагають прибулим
зорієнтуватися в нових реаліях. Тобто пишуть знову ж таки про працевлаштування,
дискримінацію та нюанси трудового законодавства. Про новини з Румунії,
Словаччини чи Угорщини і говорити годі. Це з розряду «цікавинки Океанії та
Полінезії»,
\textbf{«Вікно в Європу»}, Назар Кісь, zaxid.net, 03.06.2021

Альтернативою \emph{російському} могло б (і теоретично – так і планувалось) стати
західне інформаційне поле. І йдеться не про український дубляж на Netflix. Мова
про те, що ми й досі злочинно мало уваги приділяємо нашим найближчим адекватним
сусідам. Донедавна новини з \emph{Росії} у нас переважно йшли просто як «новини», а
для інформації зі Словаччини, Румунії, Угорщини чи Польщі існували спеціальні
рубрики,
\textbf{«Вікно в Європу»}, Назар Кісь, zaxid.net, 03.06.2021

Сборная \emph{России} по хоккею не побеждает канадцев на чемпионатах мира уже
десять лет. Давно она не была так близка к тому, чтобы прервать эту серию,
однако упустила свой шанс. В четвертьфинале мирового первенства, которое
проходит в Риге, команда Валерия Брагина уступила «кленовым листьям» в
овертайме. А ведь до стартового вбрасывания \emph{россияне} считались
фаворитами не только этой встречи, но и всего турнира. Принципиальные соперники
прибыли в Латвию не в оптимальных составах — те же канадцы едва пробились в
плей-офф. О богатом на сенсации мировом первенстве и самых ярких моментах
печального для \emph{России} матча — в материале Ленты.ру,
\textbf{«Это фантастическое позорище». Сборная России по хоккею проиграла канадцам и вылетела с чемпионата мира},
Андрей Стрельцов, lenta.ru, 04.06.2021

Ранее те же «демократические государства» поддержали кровавый госпереворот 2014
года (как в 2004 году и «оранжевую революцию»), а потом абсолютно
антиконституционное использование ВСУ против граждан Украины и кровавое
усмирение \emph{«Русской весны»},
\textbf{Казус Стерненко. Европейская «правда» как символ лжи}, 
Константин Кеворкян, ukraina.ru, 04.06.2021

\emph{Російський} реп може зруйнувати Україну,
Микола Несенюк, gazeta.ua, 02.06.2021

Діти шукають правди і щирості і знаходять їх у брудних матюках
\emph{російських} реперів. Бо ті репери справді такі. І такими виростуть наші
діти, поки їхні батьки вдягають вишиванку раз на рік і говорять рідною мовою
лише коли без цього не можна. Чи підуть нинішні школярі з тими матюками в атаку
на своїх \enquote{братьев по разуму}?  Насправді \emph{російський} реп набагато
страшніший для України, ніж був для СРСР американський рок-н-ролл півстоліття
тому. Тому що Америка не збиралася на той СРСР нападати\footnote{А як же плани
з масованих атомних ударів по містах СРСР, пане? Дякуйте Леву Ландау, Ігорю
Курчатову і Андрію Сахарову, що живі і спокійно сидите в фейсбуці, пишучи всяку
маячню!} – він впав сам під їхню музику. Нинішні наші вороги не будуть чекати,
поки Україна впаде сама під їхній реп з матюками. Вони допоможуть! Хто тоді
стане на захист від них окрім рокерів минулого століття?,
\textbf{\emph{Російський} реп може зруйнувати Україну},
Микола Несенюк, gazeta.ua, 02.06.2021

Недофюрери втратили глузд. Боронитися від таких сусідів треба кожну секунду
Наш кордон з \emph{Росією} тепер стає на тисячу кілометрів довшим. Ну що, революція,
хрєнь, про яку попереджали і яку розуміли всі притомні люди, відбувається
прямо зараз. У \emph{Білорусі} заявили про інтеграцію з \emph{Росією}. Влада устами
прем'єр-міністра тішиться, що будуватимуть \enquote{єдиний економічний простір}. А там
і до \enquote{єдиного політичного} не так далеко. Один біснуватий недофюрер
відбудовує імперію, інший біснуватий зробив однозначну ставку на тиранію і
автократію всередині майже приватизованої країни. А тим часом ми
спостерігаємо, як анексують \emph{Білорусь}. Гібридні методи сучасної геополітичної
стратегії \emph{Росії}. Дружнє поглинання братів і сестер \emph{білорусів},
\textbf{Недофюрери втратили глузд. Боронитися від таких сусідів треба кожну секунду},
Зоя Казанжи, gazeta.ua, 02.06.2021

Ми маємо в сусідах токсичних \emph{Росію} і \emph{Білорусь}. Це, звісно, не
зовсім так, як себе почуває Ізраїль. Але бути готовими боронитися від таких
сусідів треба кожну секунду. Як це робить Ізраїль. В схожій ситуації з сусідами
Литва з Польщею,
\textbf{Недофюрери втратили глузд. Боронитися від таких сусідів треба кожну секунду},
Зоя Казанжи, gazeta.ua, 02.06.2021

На встрече с представителями международного инвестиционного сообщества и
иностранных компаний-производителей «Спутника V» он назвал странными и
контрпродуктивными действия «некоторых наших партнеров, которые откровенно
пытаются не допустить любыми способами, ограничить использование \emph{российской
вакцины}». По утверждению Путина, когда речь идет о сохранении здоровья,
спасении жизни людей, нужно оставить в стороне политические и иные разногласия.
При этом, отметил \emph{российский лидер}, эффективность \emph{российской} вакцины является
одной из самых высоких в мире, а «все побочные явления — лишь повышение
температуры на пару часов». «Летальных исходов нет и, надеемся, не будет», —
сказал он,
\textbf{Путин заявил о попытках ограничить применение российской вакцины от коронавируса},
lenta.ru, 04.06.2021

Без оглядки на законность, права человека и благодушную Европу с ее ценностями.
Вплоть до событий в одесском Доме профсоюзов, которые зачастую расцениваются не
как трагическое стечение обстоятельств, но как результативное подавление
вражеской \enquote{пятой колонны}. Украинская крав-мага в 2015-м – это подрывы
ЛЭП в Херсонской области, принудившие руководство страны к энергетической
блокаде аннексированного Крыма.  Или же громкое убийство Олеся Бузины: несмотря
на версию о \enquote{кремлевской провокации} и \enquote{сакральной жертве},
многие восприняли произошедшее как эффективный метод борьбы с
\emph{пророссийской пропагандой},
\textbf{Крав-мага нашої мрії},
Михайло Дубинянський, pravda.com.ua, 05.06.2021

\emph{Зросійщеним} українцям показали справжню чайку (Фото), 
volyn.com.ua, 05.06.2021

Процитирую снова. Эпоха Порошенко сменилась на времена Зеленского, но ничего не
изменилось: \enquote{В старой \emph{России} единственная область, где
украинство, и то под сильной цензурой, разрешалось, — это театр. Все поколения
нынешних украинских деятелей воспитаны на театре, откуда пошли любовь ко всякой
театральности и увлечение не столько сущностью дела, сколько его внешней
формой}, 
\textbf{За Порошенко пришел Зеленский, но в украинской политике ничего не поменялось}, Максим Войтенко, strana.ua, 05.06.2021

\emph{Росія} здійснює глобальну інформаційну «гібридно-месіанську агресію»: як її нейтралізувати?,
radiosvoboda.org, 05.06.2021

«Контент негативу» – під такою назвою опубліковані результати дослідження,
проведеного провідними фахівцями Науково-дослідного центру Військового
інституту Київського національного університету імені Тараса Шевченка. Серед
іншого було проаналізована активність у соціальних мережах військовослужбовців
двадцяти бригад ЗСУ і те, як на них впливає інформаційна війна, яку веде
\emph{Росія}.  Про головні результати дослідження та шляхи нейтралізації
(дослідники наполягають саме на цьому терміні) нав'язуваних \emph{Росією}
наративів негативу, Радіо Свобода розпитала авторів дослідження – провідних
наукових співробітників Центру Валерія Короля та Григорія Любовця,
\textbf{\emph{Росія} здійснює глобальну інформаційну «гібридно-месіанську агресію»: як її нейтралізувати?},
radiosvoboda.org, 05.06.2021

Какое живое существо является символом \emph{России}? Большинство читателей укажут
медведя, некоторые, подкованные в геральдике, - про двуглавого орла. Кстати,
сей странный пернатый мутант, по всей видимости, до полного истребления водился
не только у нас, но и в воздушном пространстве современной Сербии, Албании,
Черногории (речь о гос. гербах). Историки находят наглядные подтверждения его
существования в Хеттском царстве, Византии и империи Габсбургов,
\citTitle{Как медведь стал символом России? Кого за это надо винить или
благодарить | Любопытный Паганель | Яндекс Дзен}, Любопытный Паганель,
zen.yandex.ru, 03.06.2021

Крымский мост является одним из великих сооружений, которое было построено в
ХХI веке. Он стал самым длинным в \emph{России} и Европе, а его цена составила
около 228 млрд рублей. Сегодня, 15 мая, Крымский мост празднует три года с
момента торжественного открытия автомобильного движения президентом РФ
Владимиром Путиным, который проехал по новому сооружению на КамАЗе.  Многие
\emph{россияне} называли этот проект «стройкой века», а на Украине и Западе не
верили, что такой мост вообще возможно построить, из-за его сложности. Однако
\emph{российские} специалисты в очередной раз всем показали высокий уровень
профессионализма, сдав 19-километровый объект на семь месяцев раньше
положенного срока.  Известно, что в возведении Крымского моста участвовало
более 10 000 рабочих, 1500 специалистов разных уровней и более 200
отечественных предприятий. Он построен с учетом всех особенностей региона,
который является сейсмически активным,
\textbf{«Крымский мост стал настоящим спасением»: три года с начала запуска движения},
riafan.ru, 15.05.2021

Кстати, когда грянула пандемия, мы ещё раз всеми руками перекрестились, что
уехали именно в \emph{Россию}. Ну и в третьих, мы не выбирали (!) какой-либо регион
сами. В отличие от ЕС, \emph{Россия} не даёт политического убежища украинцам,
единственная возможность переехать была связана с «Программой переселения
соотечественников», которой мы воспользовались. По этой программе человек не
может выбрать ни Москву, ни Питер, мало того, количество наших детей ( а на
бюрократическом языке они называются иждивенцами) давало повод  для отказа за
отказом, регионы не хотели принимать физика-ядерщика с женой и пятью детьми.
Для местных чиновников мы были не «своими русскими людьми», а камнем на шее
местного бюджета. Мы уже перестали надеяться, честно говоря, что будем спасены,
но в дело вмешались депутаты Госдумы Константин Затулин и Александра Докучаева.
Они-то и помогли нам буквально «выбить» разрешение от Ярославской области. Вот
так мы и оказались в Ярославле,
\textbf{Ярославль глазами Киевлянки},
odnarodyna.org, 03.06.2021

Второе, что меня очень сильно радовало, – здесь на каком языке говорят, на
таком и вывески, на таком и учат в школе, на таком и ведётся документация.
Отсутствие языковой шизофрении значит для душевного равновесия человека гораздо
больше, чем может показаться на первый взгляд. Третье моё открытие – это
нормальные, качественные продукты в супермаркетах.  Не нужно было знать, какое
молоко на самом деле является молоком. В Киеве я привыкла, что есть подделки,
которых надо избегать, а есть определённые марки, в которых более или менее
честно отражён состав продукта. Здесь это умение лавировать стало ненужным
рудиментом. За составом продуктов ведётся контроль, ГОСТы не упразднены,
санэпидемстанции – тоже, поэтому читай состав и думай, что тебе нужно, вот и
всё. Это кажется странным, но именно в \emph{России} я приучилась к хорошему сыру,
например, и стала безбоязненно покупать кулинарию,
\textbf{Ярославль глазами Киевлянки},
odnarodyna.org, 03.06.2021

С уверенностью могу сказать, что социальная поддержка многодетных семей в
\emph{России} – одна из лучших в мире. Не всеми «бонусами» мы смогли
воспользоваться, для этого и рожать надо было здесь, но уж как сложилось, так
сложилось.  Главное, дети в безопасности, они учат родной язык, им преподают
нормальную историю.  Главное, мы можем говорить то, что мы думаем, без угрозы
попасть за это за решётку. Как сказал мне один мой знакомый из ЕС, «в
\emph{России} сейчас настоящая свобода слова»,
\textbf{Ярославль глазами Киевлянки},
odnarodyna.org, 03.06.2021

Ну и напоследок о самом городе. Ярославль – столица Золотого кольца, город,
основанный \emph{Ярославом Мудрым} на слиянии двух рек Волги и Которосли. Его русская
красота влюбляет с первого взгляда. Если идти по набережной реки Волги от
знаменитого парка «Стрелка», то увидишь не менее десятка храмов за двадцать
минут ходьбы. Зимой здесь с ноября по апрель лежит снег, и я научилась ходить
на лыжах по сосновым борам и берёзовым рощам. Летом в центре цветут липы, мои
любимые деревья. А вообще, если бы город был холмистый, то он поразительно был
бы похож на \emph{Киев}. Те же дома, тот же русский стиль в центре и те же многоэтажки
на окраине. Да и люди здесь тоже говорят по-русски. Как в \emph{Киеве},
\textbf{Ярославль глазами Киевлянки},
odnarodyna.org, 03.06.2021

Вот вам шах и мат, борцы за украинское информационное и культурное
пространство.  Пока вы меряетесь, кто круче, кого больше показывают по
телевизору, кто больше накрутил себе просмотров на ютубе и т. д. — юные
украинцы продолжают обожать самые примитивные зацепки \emph{рускавамира}.  Кажется,
это всех устраивает, или нет? Что там думают наши родные стратеги, политики,
минкульты?,
\textbf{Что поют наши дети. Шах и мат украинизаторам (Главком, Украина)},
Анжелика Рудницкая, inosmi.ru, 01.06.2021

Галя Шимачевская (имя и фамилия из этических соображений изменены) училась со
мной в одном классе. Тогда Украина еще не была \enquote{незалэжной}, а была обычной
республикой в составе СССР. Училась она хорошо, была почти отличницей, ее папа
преподавал в нашей школе английский. Он был по происхождению поляк. А мама -
очень приятная, интеллигентная женщина, по национальности была \emph{русской}. Она и
изъяснялась на \emph{русском} языке - не могла отвыкнуть. А мы говорили на суржике -
своеобразной смеси \emph{русского} языка и \emph{малороссийского} диалекта,
\textbf{Как примерная \emph{украинская комсомолка} Галя стала стала непримиримой националисткой},
Greg, zen.yandex.ru, 06.06.2021

Ее\footnote{Светлана Пикта, Киевлянка, мать пятерых детей, бежавшая из Киева в
Ярославль - Золотое Кольцо России} посты в Фейсбук против гонений на \emph{русский}
язык, поборов в школах на оказание помощи бойцам АТО, протестное голосование
против переименования в Киеве улицы Маршала Жукова в улицу Кубанской Украины, а
проспекта генерала Ватутина в проспект имени командующего УПА (запрещенная
организация на территории \emph{РФ}) Романа Шухевича вызвали резонанс в обществе, в
том числе и среди так называемых «свидомых». Особенно преуспела в обличении
Светланы Пикты украинская детская писательница Лариса Ницой, та самая, которая
швыряла мелочь в лицо продавцам, говорившим на \emph{русском языке}, а сейчас
будоражит общественное мнение измышлениями на тему, что \emph{Россия} незаконно
называется \emph{Россией}, украв это название у Украины,
\textbf{Мужественные киевляне стали нашими земляками}, rweek.ru, 20.07.2018

Осознав, что помощи от правоохранительных органов ей не дождаться, Светлана
обратилась к представителям мониторинговых миссий ОБСЕ и ООН в Киеве. В августе
2017 года, будучи на шестом месяце своей четвертой беременности, смелая женщина
приняла участие в программе «Время покажет» на «Первом канале», где рассказала
о прессинге со стороны радикалов: – Кошмарят они меня просто так или хотят
довести дело до конца, я не знаю, – не скрывала она опасений в прямом эфире
\emph{российского} телеканала,
\textbf{Ярославщина ждет отважных киевлян}, rweek.ru, 19.01.2018

\emph{Русские} своих не бросают в беде. Узнав о судьбе многодетной киевлянки,
бросившей вызов украинскому режиму, люди в своих комментариях в Интернете
восхищались ее мужеством, поддерживали, подбадривали. Многие советовали не
искушать судьбу:
\begin{itemize}
\item - Да переезжайте вы уже всей семьей к нам в \emph{Россию}! Какой смысл беременной женщине рисковать благополучием своей семьи? Вы хотите рисковать своей жизнью и счастьем своих близких ради того, чтобы позлить этих неадекватных невменяшек??
\item - Светлана, то что вы не прогнулись под гнилую власть, уважение Вам! Уезжайте, подумайте о детях. Надеюсь, у вашей семьи будет всё хорошо.
\item - Травить женщину, у которой трое детей и четвёртый на подходе – это только наши рагули могут. Светлана молодец, удачи Вам и вашей семье.
\item - Светлана, насколько я слышала, у Вас есть родня в \emph{России}. Уезжайте уже к \emph{нам}. Понимаю, что Киев Вам родной, но нервы и семья – это дороже...
\end{itemize},
\textbf{Ярославщина ждет отважных киевлян}, rweek.ru, 19.01.2018

Окрім того, усі релігійні організації, секти, культи, які є частинами, або ж
залежні, або є в організаційному підпорядкуванні, або визнають верховенство
релігійних організацій, сект, культів \emph{країни-агресора}, повинні бути
\underline{забороненими} на території України на період такої агресії з
\underline{повною конфіскацією} активів. Після припинення агресії, конфісковані
активи \underline{поверненню не підлягають},
\textbf{Програма національного визволення та державотворення}, pravyysektor.info

Таким чином, \emph{російська} і західна культура, космополітизм і мультикультурність є
неприйнятними для Українського народу. Побудувати гідну державу здатні тільки
ті, хто усвідомлює, що фундаментом для Української Самостійної Соборної Держави
є національна консолідація і спільна командна праця, синергія всього народу
заради досягнення спільної мети,
\textbf{Програма національного визволення та державотворення}, pravyysektor.info

Слід розуміти, що як західна, так і \emph{російська} культурна експансія є формами
колонізації, при якій саме поняття «цивілізованість» застосовується тільки до
громадян, які відмовились від своїх культурних традицій, фактично –
національної ідентичності, змирилися зі своєю другорядністю і неповноцінністю,
прийняли хибне твердження, що саме культури колонізаторів є носіями прогресу,
матеріальних і духовних благ,
\textbf{Програма національного визволення та державотворення}, pravyysektor.info

Из художников — в разведчики. Впрочем, с получением высшего образования
пришлось повременить: в 1919 году Фишеры решили вернуться в \emph{Россию},
оставив, однако, гражданство Великобритании. В Москве семью постигло несчастье:
спасая тонущую девочку, погиб старший брат Вильяма Гарри.  Через некоторое
время после переезда в Москву Вильям Фишер стал студентом Высших
художественно-технических мастерских (ВХУТЕМАС), однако полтора года спустя
перевелся в Институт востоковедения,
\textbf{«Не раскрою тайн и приму смерть» Советский разведчик добывал атомные секреты США. Что спасло его от казни после провала?},
lenta.ru, 06.05.2021

Мачей Вишнёвски: «я в мучительной, душной атмосфере \emph{русофобии}»,
Светлана Пикта, odnarodyna.org, 27.04.2021

Резо Габриадзе скончался в возрасте 84 лет. Художник, режиссер театра и кино,
писатель и скульптор родился в 1936 году в Кутаиси. Он автор более 35 фильмов,
в том числе \enquote{Мимино}, \enquote{Ар даидардо}, \enquote{Шерекилеби},
\enquote{Необычная выставка}, \enquote{Трое в пути} и так далее, - сообщили
журналисты. Габриадзе был автором сценариев более чем 35 фильмов. Также он
известен как художник, скульптор и мастер книжной графики. Его работы
выставлялись в Москве, Санкт-Петербурге, Париже, Рене, Дижоне. Он был
участником выставки в Мюнхене \enquote{От Эйзенштейна до Тарковского}. Его
работы по живописи, графике и скульптуре находятся в многочисленных
государственных и частных коллекциях в США, \emph{России}, Германии, Израиле,
Франции и Японии,
\citTitle{Умер Резо Габриадзе - сценарист Кин-дза-дза и Мимино}, , strana.ua, 06.06.2021

Если мы будем просто звать людей ограничить свои потребности, ничего не выйдет,
кроме раздора. Петр кивнет на Ивана, Европа на Америку, Азия на Европу. Поворот
может дать только открытие ценности созерцания, паузы созерцания в делах, в
диалогах и дискуссиях, в развитии мысли...  Школа не может отвлечься от
сегодняшнего дня, не может не готовить программистов, юристов, менеджеров. Но
сегодняшний день скоротечен, и течение несет его к смерти. Слово \enquote{кала}
на санскрите - омоним: и время, и смерть. Культура, не нашедшая опоры в вечном,
падет под напором перемен.  Школы могут и должны учить паузе созерцания: через
искусство, через литературу. Со временем - используя телевидение, если оно
повернется к величайшей проблеме века...,
Григорий Соломонович Померанц (13 марта 1918, Вильно, Литва — 16 февраля 2013, Москва, \emph{Россия})

Бывший главарь организованной исторической группировки Вятрович, ныне депутат у
Порошенко, выразил сожаление принятием в первом чтении этого закона, назвал его
\emph{российским} и пообещал продолжить борьбу. «Майн кампф» Вятровича, так
сказать, \textbf{Вятрович назвал закон об антисемитизме российским}, Эдуард
Долинский, strana.ua, 07.06.2021

Краеугольный камень украинской пропаганды - весьмирснами. Украина - форпост на
пути азиатомосковских орд, а запад надёжный тыл незалэжной на пути
\emph{русских агрессоров},
\textbf{Очередная зрадаперемога}, Мак Сим, zen.yandex.ru, 06.06.2021

Да, в каждой шутке... А если серьезно, просто хочу говорить \emph{по-русски}.
Писать книги \emph{по-русски}. И думать. И быть в команде исторических
победителей, а не лузеров.  И я уверен, что День \emph{русского языка} станет
не только филологическим, но и геополитическим событием,
\citTitle{Победить по-русски. Почему язык стал оружием}, Дмитрий Выдрин, ukraina.ru, 07.06.2021

«Никто не знает, где Путин появится завтра. Семь лет назад против Украины
началась «крайняя волна \emph{российской агрессии}». В 2008 году это было в
Грузии, затем в Нагорном Карабахе и до этого в Приднестровье, Сирии и Абхазии.
Он «является причиной» продолжающихся военных конфликтов. В течение семи лет в
его политике не было никаких изменений», — написал Порошенко,
\citTitle{«Он везде!» — «Никто не знает, где Путин появится завтра»}, Оксана Переможенко, regnum.ru, 06.06.2021

Для США Україна, як би не боляче це було визнавати українцям, лише ситуативний
партнер, інтересами якого можна пожертвувати заради нормалізації з \emph{Росією}. Джо
Байдену який є глобалістом і за яким стоять американські транснаціональні
компанії дуже важливо, щоб \emph{російський} ринок та капітали були частиною
глобальної ринкової системи. Глобалістам важливо, щоб \emph{Росія} і Іран повернулись
до глобального ринку, де існує вільний рух капіталів, товарів і послуг.
Капіталістична \emph{Росія} не є для США ідеологічним і економічним конкурентом, на
відміну від Китаю, де були створенні державні транснаціональні корпорації, які
є конкурентами для американських ТНК,
\citTitle{Про Революцію Гідності та її наслідки}, Стефан Закревський, analytics.hvylya.net, 06.06.2021

Виявилося, що видалення акаунту Богдани пов'язане із спланованою акцією \emph{росіян}
проти інформ-ресурсів добровольчих батальонів, які брали участь в
\emph{російсько}-українській війні. У даному випадку мова йде про бататьон \enquote{Айдар}.
04.06.21 в продовження агресивної політики щодо українських патріотичних сил,
зі сторони адміністрації facebook, було ліквідовано, без попередження та яких
небудь вагомих причин, сторінку \enquote{ЛУГАНСЬКИЙ ФРОНТ БАТАЛЬОН АЙДАР} і аккаунти та
сторінки всіх редакторів цієї сторінки (Alex Che, Vadim Kandinsky, Safe
Bridges, Світлана Бевз, Алексей Онасенко, Vitally McGregor Андрей Сущенко та
Богдани Бабич),
\citTitle{Український сегмент Facebook під контролем росіян. Що з цим робити?}, Виталий Кулик, analytics.hvylya.net, 07.06.2021

Також надходять повідомлення про видалення цілого ряду акаунтів громадських
активістів, журналістів, волонтерів, добровольців тощо з різних регіонів
України. Їх нараховується вже далеко за тисячу. Єдиною провиною людей, чиї
акаунти було видалено те, що вони мали патріотичну позицію й називали війну
війною.  Колись Facebook був мережею, де не існувало політичної цензури,
мережею, яка дозволяла мобілізувати людей на спротив авторитаризму не лише в
Україні. Проте, зараз мусимо констатувати - УКРАЇНСЬКИЙ СЕГМЕНТ FACEBOOK
КОНТРОЛЮЄТЬСЯ \emph{РОСІЙСЬКИМИ СПЕЦСЛУЖБАМИ}.  Адміністрація Facebook знає про це,
але нічого з цим не робить і, як бачимо, не збирається. Тим самим власники
мережи солідаризуються з \emph{московським агресором},
\citTitle{Український сегмент Facebook під контролем росіян. Що з цим робити?}, Виталий Кулик, analytics.hvylya.net, 07.06.2021

\enquote{С огромной печалью вынуждены сообщить, что в воскресенье вечером в
ужасной автокатастрофе погибла преподаватель вокального отдела, народная
артистка \emph{России} Римма Степановна Волкова}, - заявили в учебном заведении.
Источник в экстренных службах сообщил, что авария, в которой погибла Волкова,
произошла в воскресенье, 6 июня, около 17:30 по московскому времени на трассе в
Ломоносовском районе Ленинградской области,
\citTitle{Певица Римма Волкова - В ДТП погибла народная артистка России}, Илья Кравцов, strana.ua, 07.06.2021

Василина Иванова из города Железногорска Курской области прошла в финал
\emph{всероссийского} музыкального конкурса «Новая звезда-2021». Ее выступление
в отборочном туре покорило жюри проекта, судьи аплодировали исполнительнице
стоя.  Она набрала максимальное количество баллов - 100.  23-летняя курянка,
выступающая под псевдонимом Васса Железнова, в эфире федерального телеканала
«Звезда» блестяще исполнила народную песню «Под ракитою зеленой». Представитель
жюри певица Юлия Савичева отметила, что это мировой уровень. - Мне бы очень
хотелось, чтобы такие голоса звучали как Enigma, как Deep Forest, только в
\emph{русском} прочтении, - прокомментировала артистка,
\citTitle{Курянка Василина Иванова прошла в финал всероссийского музыкального конкурса «Новая звезда-2021»}, РИА Курск, riakursk.ru, 06.06.2021

\enquote{Почта ЛНР} ввела в обращение блок марок к 280-летию открытия Русской
Америки, Государственное унитарное предприятие (ГУП) \enquote{Почта ЛНР} ввело
в обращение блок марок, посвященный 280-летию со дня открытия русскими
мореплавателями южного побережья Аляски и Алеутских островов. Об этом сообщили
на предприятии.  \enquote{4 июня 2021 года ГУП ЛНР \enquote{Почта ЛНР} введен в
обращение и считается действительным для оплаты услуг почтовой связи во всех
отделениях почтовой связи блок художественных почтовых марок \enquote{Русская
Америка 280 лет}}, - говорится в сообщении,
\citTitle{Луганский Информационный Центр — \enquote{Почта ЛНР} ввела в обращение блок марок к 280-летию открытия Русской Америки}, , lug-info.com, 07.06.2021

Юрий Федорович Лисянский (1773 – 1837) - \emph{российский офицер},
мореплаватель и путешественник, вписал свое имя в историю, совершив в качестве
командира шлюпа \enquote{Нева} первое российское кругосветное плавание.
Григорий Иванович Шелихов (1749 – 1795) - купец, мореплаватель, один из
основателей \emph{Русской} Америки.  Николай Петрович Резанов (1764 — 1807) -
\emph{русский} дипломат, путешественник, предприниматель. Стоял у истоков
\emph{Российско-американской} компании. Один из руководителей первого
\emph{русского} кругосветного плавания. Первый официальный посол \emph{России}
в Японии, составитель одного из первых \emph{русско-японских} словарей.. ,
\citTitle{Луганский Информационный Центр — \enquote{Почта ЛНР} ввела в
обращение блок марок к 280-летию открытия Русской Америки}, , lug-info.com,
07.06.2021

В \emph{России} сын православного священника решил в соцсети TikTok научить
людей правильно играть на церковных колоколах. Михаил Иванов с детства посещает
церковь и быстро прославился в соцсети благодаря своим знаниям.  На его канал в
Тик-токе \verb|ivanich_mikh| подписались уже более 100 000 человек. Об этом
передает Mash,
\citTitle{В России в TikTok прославился сын священника, который учит играть на колоколах}, Илья Кравцов, strana.ua, 08.06.2021

Из этого опуса можно сделать два взаимно противоположных вывода
\begin{itemize}
\item 1. \emph{Россия} это благо, выгодный сосед, потому что она дает деньги на финансирование нашей армии
\item 2. \emph{Россия} это агрессор, потому что она нам угрожает
\end{itemize}
Это и есть антиномия по-украински. Точнее, это абсурдопедия,
\citTitle{Зеленский - это антиномия по-украински. Точнее, абсурдопедия}, Андрей Головачев, strana.ua, 08.06.2021

В нем он не скрывает недовольства позицией США по \enquote{Северному потоку - 2},
отношениям с Европой, \emph{Россией}, а главное – Украиной. С которой американцы, как
оказалось, не советуются практически ни по каким вопросам, которые касались бы
украинских интересов.  Зеленский очень обижен тем, что Украину ни в НАТО не
берут, ни гарантий безопасности не дают, а Байден встречается с Путиным раньше,
чем с ним, с Зеленским.  Впрочем, количество лести в адрес Штатов оказалось
примерно равным по объему. Так, про Байдена Зеленский заявил \enquote{We trust in God}
– то есть верит в него, как в бога, и сравнил его с легендой баскетбола Майклом
Джорданом.  Хотя и звучат намеки, что если американцы будут и дальше так
поступать, то украинцы могут отказаться от сотрудничества с заокеанским
государством. И пойти своим путем, став \enquote{актором} мировой политики,
\citTitle{Зеленский интервью Axios – зрада по Северному потоку, НАТО и встрече с Путиным}, Максим Минин, strana.ua, 07.06.2021

Першою відзначилася Марія Захарова. Але я б її не судив суворо. Судячи з тієї
нісенітниці, яку вона з себе викинула, це говорила не вона, а настоянка глоду.
Марія, хочеться все ж порекомендувати вам вирішити свою проблему з алкоголем.
Ви як-не-як, але офіційна особа. І дуже погано, коли офіційна особа несе
маячню в стані дикої алкогольної інтоксикації. А у вас це вже система. Може,
спробуєте закодуватися? Я читав, що це може допомогти. Але в будь-якому
випадку доста бухати та брехати. До речі, два тижні тому ви обіцяли
спростувати свою брехню, але все ще не спростували. Це тому що бухнули й
забули або просто вам збрехати, як подихати, а дихайте ви постійно?  Після
спікера \emph{російського МЗС}, який сильно напився, голос подала людина з найбільшим
інтелектом в ГосДурі. Так, Микола Валуєв не тільки зміг прочитати написаний
йому текст, але навіть зміг повторити. Майже без запинки. До речі, навіть він
виглядав більш адекватно, ніж Марія Захарова. Мабуть, не п'є, як вона,
\citTitle{Нова форма збірної України з футболу - реакцію Росії аналізує Борислав Береза}, , fakty.ua, 06.07.2021

Возникновение креста с косой перекладиной.  Косую перекладину можно увидеть на
восьмиконечном кресте, он известен как \emph{русский} православный крест, и на
шестиконечном кресте, который именуется просто как \emph{русский крест}.
Восьмиконечный крест с косой нижней перекладиной появился в 6 веке, его
придумали византийцы, изображая крест на фресках и иконах. Но затем крест
закрепился в \emph{Русской православной церкви}, став отличительным символом
\emph{русского} православного христианства. Шестиконечный крест с косой нижней
перекладиной появился в \emph{России}, когда патриарх Никон решил отказаться от
восьмиконечного креста в пользу шестиконечного. Он встречается гораздо реже,
чем восьмиконечный, его можно увидеть на некоторых храмах и как декоративный
элемент гербов,
\citTitle{Почему на православном кресте перекладина косая?}, ТРИКСТЕР Философия Религия, zen.yandex.ru, 26.05.2021

Уж как мечтает \emph{Рашку} победить, И Крым вернуть, забрать Донбасс, Гоняя
мяч в который раз, Очередной, шановний тридварас,
\citComment{andrey petrov},
\citTitle{Форма не главное – главное содержание! И в спорте тоже...}, Мысли Бабы Яги, zen.yandex.ru, 07.06.2021

\enquote{В \emph{Россию} мы импортируем титановое сырье и калий, в
\emph{Беларусь} - пищевую продукцию. Это достаточно весомые процессы, которые,
несмотря на агрессию \emph{России}, будут сохраняться как можно дольше}, -
рассказал Качка.  По его словам, определённые сырьевые связи с \emph{Беларусью}
и \emph{Россией} могут привести к взаимопониманию в вопросе нецелесообразности
торговой войны.  Однако, Качка уточнил, что на данный момент наблюдается
ослабление подобного рода связей между странами,
\citTitle{Торговые связи Украины с Россией будут сохраняться}, Владислав Бовтрук, strana.ua, 08.06.2021

Поняв, что дом (\emph{Россия}) не то что разваливаться не собирается, а даже готов
прирасти новыми-старыми землями, недруги давай нашептывать жителям дома
(\emph{России}), что живут те на самом деле в нищете, что при покраске дома половины
краски было украдено, что кровля сделана не по фен-шую, а в огороде не
соблюдается севооборот, ну а Путин всё и всегда делает не так (чтобы усилить
недовольство главным по дому \enquote{партнеры}-соседи мелко пакостят санкциями). У
многих, кстати, эти идеи находят горячий отклик, потому что жизнь в доме
(стране) действительно далека от совершенства. Каждый первый из них в душе
стратег, он знает что справился бы куда лучше нынешнего президента. Эта
категория лиц напрочь забыла недавние времена, когда речь шла о сохранении дома
(страны) как такового, о том, в каком плачевном состоянии было строение, о
риске исчезнуть в небытие и стать жертвой рейдеров и грабителей.  Но сдаётся
мне что адекватных людей куда больше, чем лиц с короткой памятью, уверен в
этом. Если сравнить дом (\emph{Россию}) в 1999 и в 2021 то... То это вообще несравнимо
в принципе. Так что их довольно примитивная пропаганда на умных не действует, а
альтернативно мыслящему меньшинству закон не писан,
\citTitle{Прораб Путин}, Мак Сим, zen.yandex.ru, 08.06.2021

Вот смотрите, задаю вопрос, куда можно на планете спрятаться от \emph{русских}?
Нет такого места, где не знают русских, \emph{Русский Дух} везде, он проникает
всюду.  Вот я \emph{русский человек} и я горжусь тем, что я русский.
\emph{Русских} боятся до колик в животе, до паники, до визга, до спазмов, до
разрыва сердца. И поделом, ибо \emph{русский} – это Воин, что не горит в огне,
в воде не тонет. «Если меня с ног собьют, я буду ползти, но сражаться, если
меня вдруг убьют, я \emph{Русским Духом} приду, но буду сражаться».
\emph{Русских} любят, любят сильнее чем своих кумиров, русскими восхищаются,
хотят примером быть до боли в сердце, почему? Об этом ниже.  Так же
\emph{Россию} желают уничтожить, но как избавиться от \emph{Русского Духа}? Россия
такая большая, холодная, горячая, прекрасная, ужасная, веселая, угрюмая,
счастливая, несчастная, распятая, воскресшая... Невозможно \emph{Россию} не
понять, не осознать, она необъятная,
\citTitle{Русский Дух шагает по планете и проникает всюду}, Вестник , zen.yandex.ru, 06.06.2021

Вот этой то необъятности, непредсказуемости, неуловимости, вездесущности и
всеведущности боятся все. «Сегодня я смеюсь и тут же плачу», «люблю и ненавижу,
весь на показ и вечно прячусь». Только \emph{русский человек} плачет со слезами на
глазах. И все это \emph{Русский Дух}, как же его можно не бояться?  Восток хитрее
Запада, они не связываются с \emph{Русским Духом}, они используют его, как связь, как
силу, как пример. Для Востока \emph{Русский Дух} загадка, но и источник для
исследований. Иногда даже кажется, что Восток взращивает \emph{Русский Дух}, выводя
его на первый план,
\citTitle{Русский Дух шагает по планете и проникает всюду}, Вестник , zen.yandex.ru, 06.06.2021

И сейчас самое главное. Земля обретает новую реальность, и эта реальность
называется \emph{Святая Русь}. Не страна, не часть, а вся планета во всех мирах жизни
обретает имя, и имя — это \emph{Русь Святая}. Русский Дух не просто проникает всюду,
он преобразует ветхую реальность в Новый Мир, мир справедливости и творческих
свершений. Условие одно для всех народов – стать русскими от ныне и на веки.
Радуйтесь, \emph{русские люди}, ибо в то время, как всему миру еще только предстоит
стать русскими, мы уже \emph{русские}, мы уже представители Новой Земли и Нового Неба,
которая называется в грядущем Святая Русь. \emph{Святая Русь} – это не страна, это
Новый Мир, это вся Земля во всех трех уровнях сознания и жизни,
\citTitle{Русский Дух шагает по планете и проникает всюду}, Вестник , zen.yandex.ru, 06.06.2021

Это не \emph{Россия} шагает по планете, а Дух \emph{Святой Руси} все быстрее проникает всюду
Ярым Светом пробивает путь огнем к Навнушке - Душе \emph{России}. Вспомните, еще Иисус
сказал, что восстанет Дух Святой, проникнет всюду освободит от зла двуногих все
наРоды. Так вот он – Дух Святой, примите – это и есть Русский Дух,
\citTitle{Русский Дух шагает по планете и проникает всюду}, Вестник , zen.yandex.ru, 06.06.2021

В \emph{России} пассажирка такси выстрелила в голову водителю из-за того, что ее не
устроила цена поездки. Оружие у недовольной молодой женщины было
травматическим.  Об этом сообщает \enquote{Комсомольская правда}.  Инцидент
произошел на территории аэропорта \enquote{Большое Савино} в Перми вечером 30
апреля. На въезде в парковочную зону аэропорта между женщиной и водителем
произошла ссора из-за стоимости поездки. В итоге россиянка достала пистолет,
выстрелила в шофера, но промахнулась. 63-летний таксист отделался травмой губы,
\citTitle{В Перми пассажирка такси выстрелила в водителя из-за стоимости поездки}, Карина Вольтер, strana.ua, 02.05.2021


\citEntry{%
Весь год - в шапочках.  Мне до сих пор интересно, почему дети целый год в
шапках в \emph{России}? Зимой - понятно, тогда холодно. Летом, когда жарко, я понимаю,
лучше от солнца спрятаться в середине дня. Но зачем шапочка, когда +22 и
немного нехолодного ветра? Мне даже кажется, что шапочка не нужна, когда +10.
Но это я. Жена уже по-другому думает об этом :).  Мне кажется, как родители
реагировали - неправильно. Я понимаю, что они беспокоятся. Это понятно. Но
можно по-другому это сказать. Можно, например, спросить: \enquote{А шапочку не надо?}
или \enquote{А ветер не сильный?} или \enquote{Им не холодно без куртки?} Зачем сразу ругаться
на воспитателя?
}{%
  \citTitle{Голландец в России. Что меня изумляет в русском детсадике (на примере случая в нашем садике)}, Голландец В России, zen.yandex.ru, 09.06.2021
}

\citEntry{%
\ifcmt
  pic https://img.strana.ua/img/article/3375/strana-jua-i-23_main.jpeg
  caption Марш в честь дивизии СС Галичина в Киеве. Фото \enquote{Страны} 
\fi
Организация \enquote{Интерньюз-Украина} опубликовала материал, который обвиняет
\enquote{Страну} в распространении \enquote{\emph{российских} нарративов}.  Эта организация мало кому
известна, однако в узких кругах она имеет серьезные позиции. Ее финансируют
правительства США и Канады, а на отдельные проекты деньги дает фонд Джорджа
Сороса \enquote{Возрождение}.  Занимается указанная структура вопросами журналистики и
медиа в Украине.  Правда, сама организация похвастаться широкой аудиторией не
может. Поэтому материал, посвященный \enquote{Стране}, раскручивали через рекламную
кампанию в Facebook. Также текст перепечатали отдельные СМИ вроде \enquote{Радио
Свобода}.  Особенно болезненную реакцию получателей грантов вызвали наши тексты
о марше в честь дивизии СС \enquote{Галичина}.  Статьи объявили \enquote{\emph{российскими}
нарративами}.  Разобрали это \enquote{исследование}, а также проанализировали зачем оно
появилось
}{%
  \citTitle{Страна юа и ее статьи о нацистах вызвали гнев получателей грантов. Анализ скандала}, Максим Минин, strana.ua, 09.06.2021
}

\citEntry{%
В \emph{России} задержали популярного видеоблогера Юрия Хованского за песню о теракте
в театральном центре \enquote{Норд Ост}. В ней усмотрели публичные призывы к
терроризму. Ролик выложили в сеть еще год назад, однако силовики нагрянули к
блогеру только сейчас. Задержание было громким - с обыском и Росгвадией.
Свидетелем по делу проходит еще один популярный \emph{российский} блогер Андрей
Нефедов.  Разбирались, в чем обвиняют Юрия Хованского, а также, чем он наиболее
известен.  В чем обвинили блогера Известного в \emph{России} YouTube-блогера Юрия
Хованского задержали вечером 8 июня. В его квартире в Санкт-Петербурге до
глубокой ночи проводили обыск. После него у блогера изъяли нескольких
компьютеров, \enquote{девайс для курения} и пистолет.  В сети появились фото из
квартиры во время обыска. Видно, что в ней беспорядок, много бутылок с
алкоголем и пустых бутылок
}{%
\citTitle{Хованский - за что задержали российского блогера, видео песни}, Оксана Малахова; Анна Копытько, strana.ua, 09.06.2021
}

Президент \emph{России} Владимир Путин прокомментировал законопроект о коренных
народах Украины, по которому \emph{русские} не могут считаться таковым. Глава Кремля
назвал инициативу уродливой.  По поводу законопроекта Путин высказался 9 июня в
интервью, показанном в эфире телеканала \enquote{\emph{Россия-24}}.  Отвечая на вопрос
журналиста об отношении к законопроекту, \emph{российский} лидер сказал: \enquote{Разумеется,
отрицательно}.  \enquote{Как еще к этому можно относиться?} - задал риторический вопрос
Путин.  Он считает, что \enquote{смешно и глупо называть \emph{русских} некоренным народом
Украины}. А попытки разделить народы Украины на коренные и некоренные назвал
похожими на \enquote{идеи нацистской Германии},
\citTitle{Путин назвал уродливым законопроект Зеленского о коренных народах}, Наталья Полулях, strana.ua, 09.06.2021

Разумеется, на протяжении веков \emph{Россия} не везде и не всегда побеждала, бывало
всякое. Однако если посмотреть, просто посмотреть на карту мира станет понятно,
что такую территорию невозможно получить и главное удержать без побед.  В чем
секрет?  Скептики могут сказать, что территория, которую наши предки получили в
свои руки огромная, но малопригодная для жизни. Поэтому и особо ни кто другой
не претендовал. Это не правда, потому отметаем как достойный аргумент.
Полагаю, что секрет в \emph{русском менталитете}, которых на все накладывает свой
отпечаток, в том числе на способы ведения войн.  От обороны. С древних времен
\emph{русские} лучше воевали от обороны. Выдержать первый удар, через не могу выстоять
и постепенно измотанного порывом противника погнать.  Например, битва на
Чудском озере 1242 год. Выдержали удар, устояли под натиском, хотя и пятились
под напором псов-рыцарей и в кульминационный момент фланговый удар дружины
Александра Невского. Следует отметить, что дружина состояла из
тяжеловооруженных, профессиональных воинов, от рыцарей ничем не отличавшихся.
Это к тому, что технологическое равенство должно быть и на одном менталитете не
выехать,
\citTitle{Почему Россия непобедима? Особенности менталитета и парадоксальная стратегия}, Это История, zen.yandex.ru, 23.05.2021

\emph{Русские люди} от природы действуют не логично, так, по крайней мере,
констатируют, все кто имел неосторожность ввязаться в войну с \emph{Россией}. Наши,
казалось бы, безумные действия, редко являющиеся результатом тяжких
размышлений, на практике загоняют противника в состояние недоумения и
обескураживают. А это и есть залог победы – непредсказуемость ответных
действий.  Например, немцы приходили в недоумение в 1941 году, когда из
каких-то непроходимых болот и лесов вдруг появлялись вполне организованные
силы, с танками, пушками и нападали на беззащитные тылы. Противнику приходилось
перебрасывать войска с фронта обратно в тыл, чтобы еще раз разбить этих \emph{русских}
уже разбитых ранее, убежавших в болота и вышедших оттуда внезапно и
организованно. При этом наши просто шли, куда могли не особо задумываясь, это и
есть парадоксальная стратегия. И она у нас в крови,
\citTitle{Почему Россия непобедима? Особенности менталитета и парадоксальная стратегия}, Это История, zen.yandex.ru, 23.05.2021

Мнение украинцев о \emph{российско}-украинской войне изменилась за последние 2
года. Большинство считает конфликт на Донбассе агрессией \emph{России} против
Украины. Это продемонстрировал соцопрос, проведенный Центром Разумкова
совместно с Фондом \enquote{Демократические инициативы} им. Илька Кучерива.
Согласно данным этого исследования 53,5\% украинцев считают конфликт на востоке
страны агрессией \emph{РФ} с использованием местных боевиков.  Причем за
последние 1,5 года количество сторонников такой точки зрения возросла с 45\%.
Несмотря на то, что такая позиция более популярна на Западе и в Центре, во всех
других регионах поддержка именно такого определения конфликта существенно
увеличилось,
\citTitle{Россию виновной в войне на Донбассе считает большинство украинский}, Надежда Данищук, news.obozrevatel.com, 09.06.2021

Фигурное катание, пожалуй, самый популярный вид спорта в \emph{России} после
футбола.  Скоро начнется чемпионат Европы по футболу и это будет огромный
праздник для болельщиков. Ну а фигурное катание пока находится в спячке, однако
фигуристы уже на полном ходу готовятся к следующему сезону. Осенью начнётся
важнейший, олимпийский сезон и фигурное катание будет на первом месте в
спортивных трендах страны. Сложно предугадать кто выиграет олимпийские игры. Но
из девушек одним из главных, фаворитов является Александра Трусова - наша
\enquote{\emph{Русская} ракета}. Саша Трусова является одной из самых
популярных фигуристок \emph{России} и во всём мире. Уступает она разве что
Алине Загитовой и Жене Медведевой. Так в чём же секрет популярности Саши? Я
скажу, что есть много причин болеть за эту замечательную фигуристку. Александра
Трусова умеет удивлять болельщиков,
\citTitle{В чём секрет популярности Александры \enquote{Русской ракеты} Трусовой}, Russian Rocket Skating, zen.yandex.ru, 08.06.2021

Так уж повелось, что один раз в столетие, ну, плюс-минус несколько лет, к нам
приходят европейские гости. Но не с цветами и бутылочкой, а со стуком кулаками
в дверь - отворяй \emph{русский} мужик, не то сломаем! А иной раз и ломать начинают.
\emph{Русский мужик} отворяет, смотрит на порядком испорченную дверь, кучи навоза у
порога, злится и гонит гостей до ворот, впрочем, иногда и за воротами еще их
погоняет.  Принято считать, что первым
таким европейским гостем, был шведский король Карл XII, Северная война, в ходе
которой произошло Полтавское сражение, в 1709 году, тогда королю Карлу знатно
ввалили, да так, что бежал он аж до Турции и больше в \emph{Россию} не возвращался... ,
\citTitle{Главный европейский экстремальный аттракцион - раз в сто лет напасть на Россию}, 
Взгляд В Глубину Веков, zen.yandex.ru, 08.06.2021

Проблемы возникли из-за ограничения предложения со стороны Норвегии и \emph{России} —
основных поставщиков газа. В Норвегии начались профилактические работы на
инфраструктуре по добыче, а \enquote{Газпром} принципиально отказывается
увеличивать поставки. Транзит через Украину зафиксирован контрактом, и выкупать
дополнительные мощности компания не хочет.  Импорт сжиженного природного газа
(СПГ), который выручал Западную Европу в прошлые годы, на этот раз вынужденно
сокращен. Отмечается, что в мае поставки снизились на 29\% (семь миллионов
тонн), поскольку газовозы из-за более высокой премии направляются на азиатские
рынки,
\citTitle{Вопреки летним тенденциям, цена на газ в Европе выросли}, Валентин Землянский, strana.ua, 09.06.2021

І наостанок, варто зауважити, що цей законопроект схвалює та нав'язує
українському суспільству не лише всяких девіантів, але й захищає ліворадикалів.
Адже «ненависть на політичному ґрунті» теж каратиметься штрафом і тюрмою, а це
і протести проти \emph{проросійських} поглядів наших депутатів, і незгода з
сепаратистською діяльністю багатьох членів українського політикуму.
Каратиметься також «негативне ставлення» щодо місця проживання, мови та
етнічного походження. Що для українців фактично означатиме покарання за критику
сепарні, \emph{рускоязичних тітушок}, комуністів, прихильників путіна і
недоумкуватих політиків, які неспроможні вивчити українську.  «Правий сектор»
виступає проти антиукраїнського законопроекту 5488 і закликає все свідоме
українське суспільство не бути пасивним і байдужим, а гучно заявляти про своє
право бути християнами та консервативними українцями на своїй землі,
\citTitle{Законопроект 5488 – Уряд «ЗА» дискримінацію українців}, , pravyysektor.info, 06.06.2021

З самого ранку мене повідомили, що Інститут масової інформації включив мене до
списку експертів, «що поширюють \emph{проросійські} на антизахідні наративи». Ну
прийом давно відомий: якщо 2х2=4, і це визнається \emph{Росією}, то людина в
Україні, яка також у результаті множення двійки на двійку отримуватиме
четвірку, а не, скажімо, число п'ять або шість, вважатиметься носієм
\emph{російських наративів}. Чесно скажу - задоволений. По-перше, перебуваю в
гарній компанії достойних людей.  По-друге - дістати подібну оцінку від більш
ніж сумнівної контори, яка тривалий час діє на шкоду Україні (у цьому абсолютно
переконаний) - це найкраща нагорода,
\citTitle{Если Россия признает, что дважды два - четыре, то в Украине это должно быть три или пять}, 
Константин Бондаренко, strana.ua, 10.06.2021

А может быть, все было наоборот? Как вы отнесетесь к такой версии - варяги, по
поводу которых сломано столь копий, сами являются потомками жителей
\emph{Руси}? Это версия норвежского исследователя Тура Хейердала.  Тур Хейердал
в представлении не нуждается. Известный археолог, писатель, больше всего
известный своими путешествиями и историческими экспериментами. Если вкратце, то
суть утверждений Хейердала такова: колыбель Европы - \emph{Россия}; норвежцы -
не северная нация, они пришли с Кавказа,
\citTitle{ВАРЯГИ - ПОТОМКИ СЛАВЯН?? Версия Тура Хейердала}, 
Stories И История, zen.yandex.ru, 28.05.2021

%%%cit
\ifcmt
tab_begin cols=2
  %pic https://zen.yandex.ru/media/id/5e1deb60a1bb8700b26a71ee/variagi-potomki-slavian-versiia-tura-heierdala-6078abcc4fe8d226e656ca2a
  %pic https://avatars.mds.yandex.net/get-zen_doc/4341754/pub_6078abcc4fe8d226e656ca2a_60b0cdab0fe5492d0e7b6559/scale_2400
  pic https://avatars.mds.yandex.net/get-zen_doc/4341754/pub_6078abcc4fe8d226e656ca2a_60b0cdab0fe5492d0e7b6559/scale_1200
  caption Н. К. Рерих Заморские гости. Картина 1902 г.

  pic https://avatars.mds.yandex.net/get-zen_doc/4337106/pub_6078abcc4fe8d226e656ca2a_60b0d6a5f8cd844b4e003ffe/scale_1200
  caption Путь \enquote{из варяг в греки}
tab_end
\fi

Вновь обратимся к мнению Тура Хейердала: Я абсолютно уверен, что среди большой
группы людей, пришедших в Скандинавию вместе с Одином, были и те, чьи корни
лежат в \emph{России}. Люди в то время передвигались вдоль рек – не только на
лодках, но и пешком. Дорог тогда не было, а людям нужна была вода для жизни, и,
кроме того, путешествуя вдоль рек гораздо легче следить за направлением и
возвращаться. После окончания последнего Ледникового периода Норвегия и Швеция
могли заселяться только теми людьми, кто поднимался вверх по рекам: Дону,
Днепру, Волге... Я хочу обратить внимание всех моих оппонентов, что на
сегодняшний день антропологи просто не могут определить происхождение
норвежского народа, а любая догма - антинаучна... Кроме того, возвращаясь во
времена викингов, стоит вспомнить, что, уходя на своих кораблях на Запад,
норвежцы собирались на битвы и большой грабеж, а на Восток, в Гардарику, они
шли торговать, а нередко и искать прибежища, как многие короли-викинги, такие,
как Олаф Трюгвассон, Олаф Святой и Харальд Хардрада. Это значит, на \emph{Руси}
их ждали родственники и друзья. А почему князья Владимир и Ярослав принимали
норвежцев с распростертыми объятиями? По той же самой причине: они встречали
своих старых знакомых и близких друзей!,
\citTitle{Варяги - Потомки Славян?? Версия Тура Хейердала}, Stories И История, zen.yandex.ru, 28.05.2021
%%%endcit

Слово \enquote{изгой} мы встречаем в \enquote{\emph{Русской Правде}} - первом
памятнике \emph{древнерусского права}. И в \enquote{Краткой...}, и в
\enquote{Пространной Правде} изгой упоминается по одному разу - в самой первой
статье, где говорится о мести и штрафах за убийство свободного человека:
\enquote{Убъетъ муж (ъ) мужа, то мъститъ брату брата, или сынови отца ... аще
не будетъ кто мъстя, то 40 гривен за голову; аще будетъ \emph{русин}, любо гридин
<младший княжеский дружинник>, любо купчина, любо ябетник, любо мечник, аще
изъгои будетъ, любо словенин, то 40 гривен положити за нъ}. Изгой, как мы видим
из этой статьи, - это 1) представитель свободного населения, оказавшийся вне
своей корпорации или, другими словами, лишившийся каким-либо образом своего
социального статуса; 2) штраф за убийство свободного, оказавшегося в положении
изгоя, не изменяется, т.е. своих прав он на теряет.  Изгой предстает как
полноправный член общества и вообще стоит в одном ряду с дружинником, купцом и
даже \enquote{с \emph{русином}} (в XI веке \emph{русин} по статусу выше словенина).  Не
исключено, что добавление статьи об изгоях в текст РП произошло вследствие
восстания в Новгороде 1015 г., так что, возможно, это свидетельство иного рода
- что в XI в. изгои не занимали равного с другими свободными категориями
положения, но в обществе хорошо помнили о тех временах, они были равноправны,
\citTitle{Кто такой на Руси ИЗГОЙ?}, Stories И История, zen.yandex.ru, 08.06.2021

Не менше також і царгородські патріархи, від яких і вся \emph{Руська Країна}
прийняла святу віру, ту ж таки зверхність римської столиці святого Петра знали
немалий час і їй підлягали і від неї брали благословенство. Від неї хоча й
багаторазово відступали, але знову з нею єдналися і до послушенства їй
поверталися. А врешті на Флорентійському соборі, року Божого тисяча чотириста
тридцять восьмого, через патріарха Йосипа і цесаря царгородського Івана
Палеолога спільно до того послушенства повернулися, визнаючи, що римський папа
є отцем, та вчителем, та правителем усього християнства і правним намісником
святого Петра. На тому-ото Флорентійському соборі був і наш київський і всієї
\emph{Русі} архієпископ, митрополит Ізидор,
\textbf{Соборна грамота київського митрополита Михайла Рогози та єпископів з
іншими духовними про їхній вступ до унії 1596 року}, litopys.org.ua

Основная угроза для Украины и ЕС – политика РФ, которая пытается установить
контроль над Украиной, а также блокирует ее путь в ЕС и НАТО. Все это
происходит на фоне кризиса и дисфункции международных институтов. Однако,
невзирая на кризис, украинская дипломатия реанимирует Бухарестскую девятку,
Вышеградскую четверку, Люблинский треугольник, ГУАМ, ОЧЕС, Веймарский
треугольник и создаст ряд новых треугольников.  Украина заставит всех выполнить
Будапештский меморандум, особенно \emph{Россию}.  Отношения с РФ перейдут в режим
конфронтации, ограниченных контактов и жестких переговоров. Украина будет
добиваться замещения \emph{российского} импорта из других источников. Не указано,
каких. Украина будет противодействовать милитаризации Крыма. Интересам Украины
отвечают демократические преобразования в \emph{РФ},
\citTitle{Правительство согласовало Стратегию внешнеполитической деятельности Украины}, Елена Дьяченко, strana.ua, 10.06.2021

Дискуссия об искусственном интеллекте заметно оживилась в последние
полгода-год, причем не только в \emph{России}. В декабре прошлого года президент
Путин, выступая на международной онлайн-конференции Artificial Intelligence
Journey, заявил, что искусственный интеллект – залог технологического рывка для
всего человечества. Технологии и цифровизация были одной из главных тем на
Петербургском международном экономическом форуме, они же вскоре будут
обсуждаться на встрече «Большой семерки» в британском Корнуолле.  В конце мая
дискуссию о пределах распространения ИИ взорвала новость о том, что военный
дрон впервые убил человека. В применении дронов на войне уже нет ничего
необычного, но исключительность данного случая состоит в том, что решение об
убийстве принималось самим дроном. Эпизод произошел еще в марте 2020 года, но
известно об этом стало только сейчас, благодаря исследованию ООН, на которое
ссылается журнал New Scientist,
\citTitle{Как ограничить могущество искусственного интеллекта}, Глеб Простаков, vz.ru, 10.06.2021

Лидер ЛДПР Владимир Жириновский прокомментировал «СЭ» новую форму сборной
Украины для выступления на чемпионате Европы-2020. На футболке изображен силуэт
Крымского полуострова, а также политические лозунги. На форме сборной Украины на Евро-2020
будет карта страны с Крымом. В \emph{России} сочли это провокацией «От того, что
Украина назовет Крым своим он таковым, конечно, не станет. Крым столетиями был
частью \emph{России}, колыбелью русского православия, и последние 7 лет он, к счастью,
снова в составе нашей страны. А от того, что в Киеве кто-то будет повторять
иное — ничего не изменится. Я бы разве что предложил им раскрасить форму в
красный цвет и добавить на нее серп и молот, ведь именно коммунисты передали
Крым ими же созданному государству Украина,
\textbf{Жириновский об ответе Украине: «Я бы предложил использовать в нашей форме карту Российской империи»},
www.sport-express.ru, 08.06.2021

Нелюбові та некраси стало так багато, що не помічати цього уже неможливо.
Періодичні пошуки причин та винних у нашому складному та розділеному
суспільстві дають традиційні та очікуванні результати. \emph{Російськоорієнтовані} та
\emph{російсько-культурні} громадяни вважають, що джерелом нелюбові й некраси в
Україні є національностурбовані автохтони. Перелік претензій до них починається
зі звинувачень у зарядженості на політичну боротьбу з чужим і закінчується
доріканням у бідності, обумовленою орієнтованістю на архаїку та замкненість.
«Як може народжуватись любов у війні та злиднях?» – приблизно так міркує не
лише \emph{проросійська} сторона в Україні, але й значна частина громадян з
пострадянською ментальністю. Колись на радіо «Апельсин» навіть було таке
ранкове шоу «Злюки хохли»,
\citTitle{Нелюбов та некраса української незалежності}, Анатолій Якименко, analytics.hvylya.net, 10.06.2021

Ганебну поразку радянська пропаганда подає як перемогу. СРСР без українських
копалин та зерна потроху деградує. Після смерті Сталіна починаються безлади.
Спроби Хрущова врятувати ситуацію марні. В 1964 охоплена страйками та
повстаннями імперія розпадається.  Україна ж вступає в Союз Європейських
Держав.  Можна ще багато такого нафантазувати – було би бажання.  Але це зайве.
Ми й так живемо в подібному сюжеті. Варто лише згадати, що насправді коїлося в
1918-19 і порівняти з сьогоденням. Знову біда суне зі Сходу. Знову головною
ударною силою є регулярні \emph{російські} частини з домішками місцевих негідників.
Знову \emph{російська пропаганда} харамаркає, що \enquote{ихтамнет},
\citTitle{Якби}, Дмитро Десятерик, day.kyiv.ua, 09.06.2021

Всім відомо, що історія не знає умовного способу, що минуле взагалі краще не
чіпати окрім як з виховною чи пізнавальною метою. Всі знають і все одно
регулярно граються в альтернативну історію. Бо це, зрештою, не так про минуле,
як про сьогодення. Яким би воно могло бути, \enquote{якби} – так, це закляття
чарує хай там що, надто в суспільствах, що ніяк не виборсаються з перехідного
періоду.  Ось типова мрія про минуле: Галицько-Волинське князівство поступово
стає одним з наймогутніших східноєвропейських гравців, укладає кілька вигідних
регіональних пактів, звільняє українські землі від Орди і перетворюється на
\emph{Королівство Русь}. Місцеве самоврядування та народні вольності
посилюються, так що в середині ХІХ століття \emph{Русь} стає конституційною
монархією і входить у новітню добу цивілізованою високорозвиненою демократією,
\citTitle{Якби}, Дмитро Десятерик, day.kyiv.ua, 09.06.2021

Такая бурная реакция Москвы обусловлена тем, что затрагивается один из основных
нарративов \emph{российской} гибридной войны и \emph{российской} идеологии. Он заключается в
том, что вся территория Украины является \enquote{исконно русской землей}. А если не
признавать \emph{русских} коренным народом, то эта земля, получается, не является
таковой, как ее представляет Москва. В конце концов, это подрывает миф об
исторической \emph{Российской империи} на этих просторах – о том, что эта земля была
колыбелью \enquote{\emph{великой России}}, российской государственности, \emph{русской} нации,
\citTitle{Как законопроект о коренных народах Украины выбил почву из-под ног Москвы - Главред}, 
Григорий Перепелица, opinions.glavred.info, 10.06.2021

%%%cit
\ifcmt
  pic http://img.lug-info.com/cache/5/3/1623324903_742.jpg/1000wm.jpg
  caption танцевальный флешмоб \enquote{С \emph{Россией} в сердце}, посвященный \emph{Дню России}
\fi
Молодежные активисты Перевальска провели танцевальный флешмоб \enquote{С
\emph{Россией} в сердце}, посвященный \emph{Дню России}. Об этом сообщила
пресс-служба общественного движения (ОД) \enquote{Мир Луганщине}.  В
мероприятии приняли участие активисты проекта \enquote{Молодая гвардия} ОД,
Молодежного совета при администрации Перевальского района и Молодежного
парламента ЛНР. Они исполнили танец под песню \enquote{Мы - молодые - надежда
страны}.  \enquote{Все мы мечтаем, чтобы в Донбасс пришел мир. На протяжении
семи лет мы переживаем тяжелые времена, - сказала работник исполкома
перевальского теротделения ОД \enquote{Мир Луганщине} Анна Козлова. -
\emph{Россия} - это страна, которая поддерживала и продолжает нас поддерживать.
Мы благодарны ей за это, поэтому праздник братского государства является для
всех нас особым днем},
\citTitle{Луганский Информационный Центр — Перевальская молодежь провела танцевальный флешмоб \enquote{С Россией в сердце}}, 
, lug-info.com, 10.06.2021
%%%endcit

%%%cit
%%%cit_pic
\ifcmt
  pic https://avatars.mds.yandex.net/get-zen_doc/3965742/pub_60c25585596b103555b7d6e3_60c255d932783008ebbeebe0/scale_1200
  caption Вот эти финно-угорские племена, как утверждают некоторые историки, основали Россию и построили ее столицу Москву
\fi
%%%cit_text
Часто от наших либеральных псевдо-историков приходится слышать утверждения, что
\emph{русские}, мол – это финно-угры, которые славянами не являются, а
захватили в древности все славянские земли к северу от нынешней Украины, а
потом и вовсе всю Украину. И даже их столица Москва – и то финно-угорское
слово, а также почти половина всех городов и прочих населенных пунктов начиная
с центра нынешней \emph{России} – и те называются по-фински.  Самое интересное,
что те, кто это утверждают, не так уж и далеки от истины, но часть истины – это
не истина. Все эти историки хорошо знают историю, но, тем не менее, извращают
ее под свои меркантильные цели. Присутствие на \emph{Среднерусской} равнине
финно-угорских племен в древности никто не отрицает, еще византийский историк
Иордан знал названия этих племен не понаслышке – он довольно подробно занимался
историей готов, которые обитали в Северном Причерноморье и имели торговые связи
с северными землями
%%%cit_title
\citTitle{Как финно-угорская «чудь» 1000 лет назад захватила Русь, построила Москву и стала пугать Европу}, 
Исторический Понедельник, zen.yandex.ru, 10.06.2021
%%%endcit

%%%cit
%%%cit_pic
%%%cit_text
А вот в \emph{Северной Руси} вся «чудь» вдруг исчезла еще почти 1000 лет назад
как по мановению волшебной палочки, как только в те земли пришли русские. И где
же, спрашивается, все эти эсты, веси и меря, которые якобы основали Москву и
заселили берега Белого моря? Если они растворились в русских, то это не значит,
что сами \emph{русские} стали «чудью». А если они откочевали со своих исконных земель,
то почему, спрашивается, русские их не изгнали с новой их родины, когда
продолжили свою завоевательную экспансию позже?

Видать, новые \emph{русские завоеватели} просто не признали в них своих древних
«родственников», которых упорно им навязывают современные либеральные историки,
которые постоянно хотят смешать \emph{русских} со всеми на свете – то со
шведами в лице Рюрика, то с монголами в лице Батыя, то с «чудью» в лице
несуществующих сегодня финно-угорских народов.
%%%cit_title
\citTitle{Как финно-угорская «чудь» 1000 лет назад захватила Русь, построила Москву и стала пугать Европу}, 
Исторический Понедельник, zen.yandex.ru, 10.06.2021
%%%endcit

%%%cit
%%%cit_pic
\ifcmt
  pic https://avatars.mds.yandex.net/get-zen_doc/1590365/pub_60bb62f961893124cbe6fbb3_60bb63a761893124cbe8e771/scale_1200
  caption Украинцы почему-то забывают, что они принадлежат \enquote{русскому миру} как минимум генетически, причем все поголовно
\fi
%%%cit_text
Последние 7 лет со стороны украинских пропагандистов и части политиков слышен
такой тезис, что якобы «\emph{русский мир} – это гибель Украины». При этом
«\emph{русским миром}» пугают украинских детей, как раньше пугали Бабаем, а еще раньше
– татарами. В общем, так получается, что «\emph{русский мир}» - это нечто
плохое, и даже ужасное, и его в Украину допускать никак нельзя.

Ну, не будем вдаваться в подробности того, что украинцы понимают под понятием
«русский мир», отметим только, что сами украинцы принадлежат этому самому
«\emph{русскому миру}» уже более 1000 лет, особенно учитывая, что Киев – это
«\emph{мать городов русских}», а \emph{Древнерусское} государство ими самими
постоянно называется \emph{Киевской Русью}
%%%cit_title
\citTitle{Чем отличается «украинский мир» от «русского мира», и какой из них для Украины лучше?}, 
Исторический Понедельник, zen.yandex.ru, 05.06.2021 
%%%endcit

%%%cit
%%%cit_pic
%%%cit_text
Таким образом получается, что современный «украинский мир» - это мир польский,
а не \emph{русский}. Напомню, что западные украинцы, от которых и проистекает
вся эта \emph{русофобия} под лейблом \emph{«русский мир»} - это потомки
завоевателей-поляков. Спросите любого западного украинца – как он относится к
Киеву, \emph{«матери городов русских»}? Да он этот \emph{русский} Киев в гр*бу
видал, мягко говоря, для «щирого западенца» столицей Украины милее был бы
польский Львов (Лемберг), такие планы многие западноукраинские политики
генерируют еще с самого начала независимости Украины в 1991 году
%%%cit_title
\citTitle{Чем отличается «украинский мир» от «русского мира», и какой из них для Украины лучше?}, 
Исторический Понедельник, zen.yandex.ru, 05.06.2021 
%%%endcit

%%%cit
%%%cit_pic
%%%cit_text
Не сохранилось ни одного документа, даже самого захудалого, в котором было бы
отражено стремление украинцев к независимости со времен разгрома \emph{Киевской Руси}
монголо-татарскими войсками, а потом захвата ее западной и южной части
польско-литовскими колонизаторами. Если тщательно изучить историю Украины
вплоть до 1991 года, то оказывается, что украинцы в лице тогдашних \emph{малороссов}
никогда не собирались основывать собственное государство, чтобы сделать его
независимым.  Дело в том, что эта часть \emph{русского народа} обитала в такой
рискованной зоне, и была настолько малочисленной, что без покровительства
соседних стран просто не могло обойтись. Самое большее, что было в силах и умах
тогдашних украинцев, так это создать военизированные объединения, которые
состояли из козаков (аналог \emph{русских} казаков). Но это произошло только в XVI
веке, когда польско-литовское государство, которое владело основной частью
\emph{южнорусских} земель, начало ослабевать
%%%cit_title
\citTitle{Украина 800 лет подряд хотела отдаться кому-то в управление, этот вектор продолжается и сегодня}, 
Исторический Понедельник, zen.yandex.ru, 04.06.2021
%%%endcit

%%%cit
%%%cit_pic
%%%cit_text
Сегодня многие украинские националисты и прочие «патриоты» вовсю пытаются
раздувать миф о том, что якобы Украина немалую часть истории была колонией
\emph{России}, а потом и СССР. Однако если разобраться в этом вопросе поглубже, то
оказывается, далеко не все «спецы» из этой категории придерживаются такого
мнения. Среди националистов немало и достаточно умных людей, которые не
используют этот тезис даже в пропагандистских целях.  Но вот есть же такие
«историки» на Украине, которые вовсю хотят убедить свою паству в том, что
Украина является колонией \emph{России} даже сейчас, и при этом они приводят такие
смехотворные доводы, которые можно легко опровергнуть, но которые находят
благодатную почву в умах многих украинцев, которые думать сами не хотят.
Например, приводится такой довод, что \emph{Россия} за всю свою историю
сосуществования с Украиной оттянула из нее очень много умов, талантов и прочих
прекрасных дарований, которые в первую очередь работали на Москву и Петербург,
а не на свою родину. К таким можно отнести множество ученых, писателей,
политиков и прочих всяких «украинских умов», включая Гоголя и Королева – список
просто огромный. Кстати, в него можно включить и генсеков СССР Хрущева с
Брежневым – тоже ведь украинцы,
%%%cit_title
\citTitle{Украина никак не может выбраться из колониальной зависимости – то от России, то от Запада}, 
Исторический Понедельник, zen.yandex.ru, 01.06.2021
%%%endcit

%%%cit
%%%cit_pic
%%%cit_text
Даже при беглом прочтении книги Кучмы заметна вся ее противоречивость, которая
никак не позволяет достаточно убедительно разделить \emph{русских} и украинцев по
основным параметрам. Но я на всех этих противоречиях останавливаться не буду, а
сосредоточусь только на некоторых моментах, в которых Кучма сравнивает \emph{русских}
с украинцами, и которые демонстрируют несостоятельность теории «украинцы – не
\emph{русские}».  Так, где-то в первой четверти своей книги \emph{русский}, по сути, человек,
Кучма утверждает, что украинцы – законопослушные и признают любую власть, если
она законная. А \emph{русские}, по его словам, никогда не живут по закону, и даже
\emph{российские власти}, будь то цари, генсеки или президенты, предпочитают
действовать «по понятиям»
%%%cit_title
\citTitle{Русские живут «по понятиям», а украинцы чтут закон и любят власть.
Что тут не так?}, Исторический Понедельник, zen.yandex.ru, 29.05.2021
%%%endcit

%%%cit
%%%cit_pic
\ifcmt
  pic https://static2.gazeta.ua/img2/cache/gallery/1036/1036940_3_w_1000.jpg?v=0
\fi
%%%cit_text
Народних депутатів від партії \enquote{Слуга народу} Микиту Потураєва, Євгенію Кравчук
і Максима Бужанського намалювали в карикатурі. Малюнок художника Олексія
Кустовського опублікувало \enquote{Радіо Свобода}. \enquote{Слуги народу} тримають \enquote{\emph{русский}
мир}, який трощиться об закон про мову. Також тримають \emph{російський} триколор.
Зверху написана репліка: \enquote{Він заважає кіно, Кремлю...ой зайве сказала}
%%%cit_title
\citTitle{Слуг народу з русским миром намалювали у карикатурі}, , gazeta.ua, 11.06.2021
%%%endcit

%%%cit
%%%cit_pic
%%%cit_text
Другая особенность этой писанины – это клиническое противостояние отдельных
предложений и абзацев. Тут целые страницы ведут друг с другом беспощадную войну
на умственное уничтожение. В одном месте здесь напишут, что Украина является
удобным энергетическим и логистическим хабом. А уже в другом, что РФ блокирует
торговые потоки между Украиной и странами Средней Азии. Здесь пишут, что
Украина остается надежным партнером ЕС по транзиту газа, а через страничку
рассуждают о репарациях, которые непременно заплатит \emph{Россия}.  Вообще, пацанва
не очень заморачивалась со смыслами. За основу бралась матрица с ключевым
словосочетанием «отражение \emph{российской агрессии}» и ее лепили в каждом
придорожном сортире. В итоге, с отражением \emph{российской} агрессии дотрахались до
самой Африки. Вот вам пример трэш-эпистолярного имбецилизма – «Україна докладе
зусиль для отримання з боку африканських держав підтримки... зокрема у питаннях
підтримки протидії \emph{російської агресії}». Вот это у них там такое написано.
Бурунди и Малави передают привет и выражают озабоченность
%%%cit_title
\citTitle{Украинская власть разродилась стратегией внешнеполитической деятельности}, Игорь Лесев, strana.ua, 11.06.2021 
%%%endcit

%%%cit
%%%cit_pic
%%%cit_text
Есть в этой стратегии еще один забавный пассаж. \emph{Россия}, оказывается, проводит
насильственную ассимиляцию украинцев на своей территории. Это очень задорное
обвинение для власти, которая здесь закрыла все \emph{ру-школы}, ввела штрафы за
использования \emph{русского языка} и приняли закон о коренных народах, где \emph{русские} в
Украине становятся не совсем кошерным этносом. Но зато в Мордоре хоббитов
насильственно \emph{русифицируют}.  И да, друзья, всего-то два года прошло, а
попробуйте теперь додумать, под каким пунктом в этой «Стратегии» не подписался
бы Порошенко? Мы ведь ради этого его и поменяли, чтобы было то же самое, только
с зелеными рюшечками
%%%cit_title
\citTitle{Украинская власть разродилась стратегией внешнеполитической деятельности}, Игорь Лесев, strana.ua, 11.06.2021 
%%%endcit

%%%cit
%%%cit_pic
%%%cit_text
\enquote{Вы, защищаете \enquote{Слава Украине} и одновременно слушаете
русскоязычные песни, смотрите \emph{русскоязычные} ютуб, сериалы, фильмы.. вы
подписываете себя в соцсетях московским языком..Вы, защищая \enquote{Слава
Украине}, являетесь одновременно распространителями московской экспансии в
Украину. Не видите вы некоторую нестыковку в этом?} - вопрошает Ницой.  По
мнению писательницы, у все в Украине есть посттравматическое расстройство и его
можно преодолеть двумя путями: освобождать свое пространство от всего, что есть
на \emph{русском языке}, не писать на нем и не слушать музыку. И делать все тоже
самое, но уже на государственном уровне
%%%cit_title
\citTitle{Писательница Ларица Ницой назвала украинских футболистов моковитами}, Эллина Либцис, strana.ua, 11.06.2021
%%%endcit

%%%cit
%%%cit_pic
%%%cit_text
Началом и источником всего этого скандала является исключительно агрессивная
\emph{российская} политика. И идти на поводу у \emph{российской агрессивной политики} то ли
Украине, то ли УЕФА абсолютно неуместно, ведь \emph{Россия} никогда не остановится и
будет постоянно углублять и расширять свои требования. Самым разумным решением
является игнорирование требований Москвы, потому что они – неуместны и не
обоснованы. К тому же, даже если бы на форме не было надписи \enquote{Героям слава!}
\emph{Россия} устроила бы скандал по другому поводу, например, из-за того, что карта
Украины изображена вместе с Крымом, который \emph{РФ} считает своим
%%%cit_title
\citTitle{Почему взбесилась Россия из-за надписи \enquote{Героям слава!} на форме украинской сборной - Главред}, 
Владимир Вятрович, opinions.glavred.info, 11.06.2021
%%%endcit

%%%cit
%%%cit_pic
\ifcmt
tab_begin cols=2

  pic https://avatars.mds.yandex.net/get-zen_doc/4395091/pub_60c2a0e7badbe26f3b18cd96_60c2a1c20f49c61cbcb1eaa0/scale_1200
  caption А украинские борцуны, как обезьяны в джунглях, сразу забросали соцсети УЕФА возмущенными криками, ну, что можно посоветовать в таком случае футболистам жовто-блакитной страны

  pic https://avatars.mds.yandex.net/get-zen_doc/1591747/pub_60c2a0e7badbe26f3b18cd96_60c2a2c14ab53641294e3f36/scale_1200
  caption Кулачок-то поставленный. Бойтесь, бриты... головые; англичане – бойтесь, русские вышли из леса, идут на стадион, болеть за своих.

tab_end
\fi
%%%cit_text
А украинские борцуны, как обезьяны в джунглях, сразу забросали соцсети УЕФА
возмущенными криками, ну, что можно посоветовать в таком случае футболистам
жовто-блакитной страны. Кроме формы спортсмен надевает на себя трусы и гольфы,
этот лозунг, по совету одного комментатора, можно расположить на гульфике
трусов или на нижнюю часть гольфов, там ему самое место.  А еще, как всегда,
порадовали англичане, которые в свое время были крайне напуганы, можно сказать,
очень-очень были удивлены, что \emph{русские} болельщики умеют за себя постоять
и не будут разбегаться врассыпную перед всех пугающими своей агрессивностью
англичанами. После того как британские болельщики были посрамлены на Евро-2016
во Франции, англичане добавили к длинному списку новую страшилку про
\emph{Россию}.  У нас ходят медведи с балалайками по улицам, пьют водку,
которую выпускают на заводах, в которых разливают «Новичок», который придумали
русские болельщики, которые ходят в лес участвовать в рукопашных боях, которыми
руководят Боширов и Петров
%%%cit_title
\citTitle{Ни героев, ни славы. УЕФА запретил лозунги на форме футболистов Украины}, 
Моя Поляна, zen.yandex.ru, 11.06.2021
%%%endcit

%%%cit
%%%cit_pic
\ifcmt
  pic https://avatars.mds.yandex.net/get-zen_doc/3866587/pub_60b49d56b86a7b140d750d16_60b4ad0a0d73310cf0cb72a4/scale_1200
  caption Ось, який я гарный хлопець!
\fi
%%%cit_text
Ну, и, конечно, остались брательники, которые доворуют все остальное, а есть
те, которые не воруют, те занимаются идеологическим фронтом. Такие как еще один
Володимир, бывший директор Украинского института национальной памяти. Идейный
вдохновитель украинства, Зеленского и гопкомпании - Владимир Михайлович
Вятрович, персона нон-грата даже в такой лояльной к Украине Польше. Работал в
США, СБУ, его жена в 2000-е руководила грантовыми проектами фонда Сороса. Я
наткнулась на его очередной опус в ИНОСМИ, в котором нынешний депутат Рады без
устали рассказывает украинцам, как надо сплотиться и вокруг чего: «\emph{Русские}
должны понять, что имеют дело с теми, кто не отречется от своего, кто умеет его
защитить и будет это делать бескомпромиссно... Чтобы \emph{русские} поняли, что украинцы
готовы защищать свое до последнего, первыми сами украинцы должны осознать, что
у них есть это свое, оно отличается от \emph{русского}, уникальное и стоит защиты»
%%%cit_title
\citTitle{Застенчивая улыбка Зеленского похожа на улыбку \enquote{голубого воришки} Альхена. 
\enquote{Ты кому Украину продал, жулик?!}}, Моя Поляна, zen.yandex.ru, 01.06.2021
%%%endcit

%%%cit
%%%cit_pic
%%%cit_text
Естественно ни \emph{Россия}, ни Украина, ни Белорусь не являются никакими
приемниками \emph{Киевской Руси}. Они - дети или даже внуки Киевской
\emph{Руси} и их наследие может быть только в плане одного из блоков их
идентичности по типу: \enquote{Наша национальная идентичность имеет корни вот в
этой Киевской Руси}, без всяких там претензий на \enquote{воссоединение земель
русских Москвой} или на \enquote{воссоздание Киевской Руси со столицей в
Киеве}. Напоминаю, что мы живем в 2020 году, во время Инстаграма, планов
колонизации Марса, начала квантового интернета и осознания важности сохранения
экологического балланса. А всякая \emph{украино-российская} фаллометрия на
тему: \enquote{Я - настоящий наследник \emph{Киевской Руси}/Нет - я настоящий,
ты украл этот титул} это - обыкновенное подростково-мальчуковое мерянье сами
знаете чем, кое широко распространено в странах бывшего Соц. Лагеря: сербы с
хорватами тоже так же меряются кто из них более канонiчный, болгары утверждают
что македонцы - не более чем компьютерная графика CGI, а македонцы
соответственно говорят что это - они кошерные, от самого Александра
Македонского.  Странно, женщины этим занимаются гораздо меньше, у них наверное
голова настроена на решение более реальных проблем, а не проблем тысячелетней
давности
%%%cit_title
\citTitle{Киевская Русь - это Россия, которая перенесла столицу и поменяла
название или другое государство? Какие мнения на этот счет? Какие аргументы?},
Anton Tkachyov, Яндекс Кью, 12.03.2016
%%%endcit

%%%cit
%%%cit_pic
%%%cit_text
Одним блокуванням каналів Медведчука \emph{російська} пропаганда не зникла.
Медійники та \enquote{балакаючі голови} перейшли на канали Наш та Україна-24.
А от з онлайн-медіа ситуація наступна.  Лідером з поширення \emph{просійських}
наративів була і залишається \enquote{Страна}.  Якщо порахувати усю аудиторію
\emph{проросійських} онлайн-медіа до та після рішення РНБО по медведчуківським
телеканалам, то вона зменшилася лише на чверть.  Основна аудиорія
\enquote{сидить} на \enquote{Стране}. Але її чіпати не можна. Бо фанати та
захисники \enquote{Страны} сидять в Офісі президента
%%%cit_title
\citTitle{\enquote{Страна} – лідер з поширення проросійських наративів}, 
Олексій Братущак, blogs.pravda.com.ua, 10.06.2021
%%%endcit

%%%cit
%%%cit_pic
%%%cit_text
Фашистів із кремля під суд нового Нюренберга!  Називайте речі своїми іменами:
\emph{паРаша} - це фашистська держава, Путін - фюрер і убивця, ФСБ - \emph{рос-гестапо}, а
ОПЗЖ - холуї фашистської \emph{росії} і зрадники!  Поширюйте слова принца Чарльза,
який назвав Путіна Гітлером публічно ще в 2014 році і заяву американського
сенатора Рона Уайдена від штату Орегон, який назвав \emph{російську} владу
фашистською. Холуям \emph{фашистської росії} це точно не сподобається!
%%%cit_title
Валерий, \citLink{t.me/ua_patrioty/2969}, ПАТРІОТИ, telegram, 10.05.2021
%%%endcit

%%%cit
%%%cit_pic
\ifcmt
  pic https://avatars.mds.yandex.net/get-zen_doc/4079337/pub_60c3819603088a3a1a36d8ea_60c381c103088a3a1a375e35/scale_1200
  caption Малороссийская женщина и черт. Кадр из фильма, снятого по повести великороссийско-малороссийского писателя Николая Васильевича Гоголя \enquote{Вечера на хуторе близ Диканьки}
\fi
%%%cit_text
Очень часто можно услышать от украинских политиков, историков и прочих деятелей
фразу, что, мол, «украинцы – это не \emph{русские}». Ну, с этим можно было бы
поспорить, а можно было бы и не спорить, но разобраться в этом надо
обязательно.  Известно, что \emph{русские} – это триединый народ, который вышел из
одного корня. Это \emph{великороссы}, \emph{малороссы} и \emph{белорусы}. Проживал этот народ на
обширной территории \emph{Среднерусской} равнины – от среднего течения Днепра на юге
до юго-западного побережья Балтийского моря на севере, от Карпатских гор на
западе и до междуречья Волги и Оки на востоке
%%%cit_title
\citTitle{Почему великороссы стали такими сильными, а малороссы остались такими слабыми?}, 
Исторический Понедельник, zen.yandex.ru, 11.06.2021
%%%endcit

%%%cit
%%%cit_pic
\ifcmt
  pic https://avatars.mds.yandex.net/get-zen_doc/912331/pub_60c3819603088a3a1a36d8ea_60c38a3e1d6b9f4413b10537/scale_1200
  caption Приблизительные границы Древнерусского государства на переломе I-го и II-го тысячелетий, как раз в момент дробления его на феодальные уделы. Многие русские с Южной Руси переселялись на север и север-восток
\fi
%%%cit_text
Так вот, феодальная раздробленность обычно влечет за собой всякие разборки
между феодалами, от которых в первую очередь страдает простое население, как
городское, так и сельское. Начиная с XI столетия раздробившаяся в результате
междоусобиц местных князей \emph{Русь} начала терпеть экономические трудности. Города
грабились, разорялись, выгорали от бесконечных пожаров, и чем слабее был
феодал, тем больше всяких лишений терпели его подданные. Естественно, простой
народ стал мигрировать из разоренных земель в земли более сильных князей,
княжества которых разорению подвергались меньше всего
%%%cit_title
\citTitle{Почему великороссы стали такими сильными, а малороссы остались такими слабыми?}, 
Исторический Понедельник, zen.yandex.ru, 11.06.2021
%%%endcit

%%%cit
%%%cit_pic
%%%cit_text
С тех пор прошло около 350 лет, но \emph{малороссы} по-прежнему остаются совсем другим
народом по отношению к \emph{великороссам}, которые некогда покинули эти земли и стали
совсем другими людьми, наделенными совершенно более высокими качествами. И это
неудивительно, потому что северная часть разделившихся \emph{русских} выковали в себе
совершенно иной характер, позволивший им выжить в очень непростых условиях
Севера
%%%cit_title
\citTitle{Почему великороссы стали такими сильными, а малороссы остались такими слабыми?}, 
Исторический Понедельник, zen.yandex.ru, 11.06.2021
%%%endcit

%%%cit
%%%cit_pic
\ifcmt
  pic https://avatars.mds.yandex.net/get-zen_doc/4581585/pub_60c3819603088a3a1a36d8ea_60c38fbce72f9332f7edc53e/scale_1200
  caption Как только в Южную Русь вернулись с севера великороссы, они изгнали всех врагов своей прародины и превратили Киев в один из самых цветущих городов Великой Руси
\fi
%%%cit_text
Вот отсюда и начинаются все разговоры о том, что якобы «украинцы – это не
\emph{русские}». Естественно, запустить свою страну и ничего не делать для ее развития
несколько сотен лет – это еще уметь надо. Но \emph{малороссам} уметь ничего и не надо
было, это не они виноваты, а обстоятельства. Виноват теплый климат, который не
позволял развить самые ценные для гомо сапиенс качества, виноваты постоянные
нашествия – то монголов с востока, то поляков с запада, то татар с юга.
Виновата малочисленность населения огромной страны, которая не позволяла
концентрировать все силы на одном направлении.  Всего этого не было у северных
\emph{русских}, которые не зря себя назвали \emph{великороссами}. Вот потому великороссов
сегодня боится весь мир, а кто в этом мире знает, что есть под боком у них
такой народ, как украинцы, которые на самом деле являются \emph{малороссами}?
%%%cit_title
\citTitle{Почему великороссы стали такими сильными, а малороссы остались такими слабыми?}, 
Исторический Понедельник, zen.yandex.ru, 11.06.2021
%%%endcit

%%%cit
%%%cit_pic
%%%cit_text
После того, как многие земли \emph{Киевской Руси} вошли в состав Великого Княжества
Литовского, а потом Речи Посполитой, то эти земли, населенные чуждыми полякам и
литовцам народами, и имевшие статус колониального владения, по старому \emph{русскому}
образцу также стали называться поляками «украиной», то есть «окрайной».
Интересный момент – когда \emph{Россия} присоединила восточную часть Украины к
\emph{Русскому царству}, то «украиной» русские стали называть западные области страны,
оставшиеся под властью Польши, в основном это были Волынь и Галичина.
Официально же «русская Украина» стала называться \emph{Малороссией}, и это вполне
правильно, так как украинцев как нации не существовало даже тогда, они даже
сами себя считали русскими, а считать себя «окраинцами» гордость не позволяла.
\emph{«Малороссы»} - это все же лучше, чем «окраинцы»
%%%cit_title
\citTitle{Почему современную Украину назвали «окраиной», а не более престижно - Киевской Русью?}, 
Исторический Понедельник, zen.yandex.ru, 22.02.2021 
%%%endcit

%%%cit
%%%cit_pic
%%%cit_text
А в то время, когда западнорусские националисты вознамерились создать свое
собственное государство, они почему-то выбрали для него самое неудачное
название из всех, которые только можно было придумать. Ну, \emph{Русское воеводство}
давно кануло в лету, \emph{Киевская Русь} тоже как-то не коррелировалось с
существовавшей политической обстановкой, когда следовало противопоставить себя
\emph{России} во всем, в том числе и в названии новой страны. Потому в спешке, чтобы
не изобретать ничего нового, и был выбран этот неудачный полонизм – Украина
%%%cit_title
\citTitle{Почему современную Украину назвали «окраиной», а не более престижно - Киевской Русью?}, 
Исторический Понедельник, zen.yandex.ru, 22.02.2021 
%%%endcit

%%%cit
%%%cit_pic
%%%cit_text
Не лишним будет напомнить, что «\emph{всея Русь}» - это была именно \emph{ВСЯ РУСЬ}, а не
только «Московия» или что-то там еще. Таким образом Иван I Рюрикович, ближайший
потомок всех великих киевских князей, получил право на \emph{ВСЮ РУСЬ}, от Киева и до
Новгорода, от Смоленская и до Твери, от Черного моря до самого Ледовитого
океана. Это право было подтверждено перемещением из разоренного
монголо-татарами Киева резиденции митрополита киевского, который в конечном
итоге стал митрополитом московским. Соответственно в Москве сосредоточилась и
вся духовная власть, а не только политическая
%%%cit_title
\citTitle{Кто основал Москву? Вы даже удивитесь - ее основал великий киевский князь в XII веке и заселил ее украинцами}, 
Исторический Понедельник, zen.yandex.ru, 18.02.2021
%%%endcit

%%%cit
%%%cit_pic
%%%cit_text
Бывшие пределы \emph{Киевской Руси} оставались в таком запустении очень долго, пока их
не присоединило к себе Великое Княжество Литовское. Только тогда на них снова
начали селиться люди, но это были в основном не славяне, а поляки, литовцы,
татары-земледельцы и потомки гуннов – севрюки. Чисто славянское население
появилось в этих местах только после присоединения Левобережной Украины к
\emph{Русскому Царству}. Но это уже совсем другая история
%%%cit_title
\citTitle{Кто основал Москву? Вы даже удивитесь - ее основал великий киевский князь в XII веке и заселил ее украинцами}, 
Исторический Понедельник, zen.yandex.ru, 18.02.2021
%%%endcit

%%%cit
%%%cit_pic
%%%cit_text
Алексей, Извините Вы кто? Историк философ? Все собрали в кучу! Я не понимаю
Вас, почему придераетесь так болезнено была не была, да еще шовиниски на
строены? Какой был язык 10-11 век? Пример бияши скакаши на коняши, и Вы
приводите примеры 17-18 веков. А почему не придератесь к Казахстану,
Узбекистану и т.д, тоже не было таких стран и языков. И был тяжелый период, как
для \emph{Руси} и как Украины, слаянских народов. Мы \emph{русские} хотим отмазаться от всего
этого, И за придурков бандеровцев мазать всю Украину! А языки формировались
веками! А Вы, что точно знаете, каким был \emph{русский язык}? Не врите, в нем много слов
разных народов! Вы меня не убедили!
%%%cit_comment
Vladimir S
%%%cit_title
\citTitle{Кто основал Москву? Вы даже удивитесь - ее основал великий киевский князь в XII веке и заселил ее украинцами}, 
Исторический Понедельник, zen.yandex.ru, 18.02.2021
%%%endcit

%%%cit
%%%cit_pic
%%%cit_text
Между тем, опыт финляндизации нужно изучать досконально.
\enquote{Контрреволюционные вещи вы говорите, Филипп Филиппович...} Многолетний
опыт знакомства с доминирующим дискурсом позволяет говорить о том, что вопрос о
фактическом содержании финского опыта взаимодействия с \enquote{\emph{русским}
медведем} едва ли не табуирован
%%%cit_title
\citTitle{Опыт финского взаимодействия с Россией в Украине почти табуирован / Лента соцсетей / Страна}, 
Роман Химич, strana.ua, 12.06.2021
%%%endcit

%%%cit
%%%cit_pic
%%%cit_text
Короче говоря, \emph{Россия} всюду – с севера, юга, запада и востока. Настоящий
киевский кошмарный сон :=) Хотя, если серьезно, \emph{Россия} для Украины, конечно же,
сосед ВОСТОЧНЫЙ. Ну, от силы СЕВЕРО-ВОСТОЧНЫЙ.  Причислять же себя к нашим
южным соседям – это несколько неграмотно. Настоящий северный сосед для них –
это \emph{Белоруссия}. Хотя... Может Зеленский ее и имел виду? :=)
%%%cit_title
\citTitle{Ошибка в речи Зеленского. Почему Россия – вовсе не \enquote{Северный сосед} Украины}, 
Объясняю На Пальцах, zen.yandex.ru, 22.04.2021
%%%endcit

%%%cit
%%%cit_pic
\ifcmt
  pic https://img.championat.com/i/r/v/1622373875440285371.jpg
  caption Алина Загитова
\fi
%%%cit_text
«Ребята, наша сборная! Весь сезон вы дарили нам радость своей игрой, а сейчас
вы на главном турнире года. Мы знаем, как долго вы к этому шли, несмотря ни на
что, вопреки всему. Мы уверены, что вы сильнее всех, быстрее всех и точнее
всех. Мы знаем, на что вы способны, и верим, что вы победите всех соперников и
завоюете золотые медали.  Вы не одни, вместе с вами вся наша страна, миллионы
болельщиков: от Калининграда до Владивостока, от Мурманска до Сочи. Мы будем
переживать за вас и радоваться вашим победам. Желаем вам только успеха и
победы. Мы — \emph{Россия}, мы — «Красная машина». Мы победим!» — сказала Загитова
%%%cit_title
\citTitle{Как Алина Загитова связана с хоккеем: отец — хоккеист, номер с клюшками, горячая поддержка сборной России - Чемпионат}, 
Яна Левхина, www.championat.com, 30.05.2021
%%%endcit

%%%cit
%%%cit_pic
%%%cit_text
Мы - \emph{русские}, пели, поём и будем петь прекрасные украинские песни. В
наших сердцах всегда остаётся только хорошее
%%%cit_comment
Сергей Селиверстов
%%%cit_title
\citTitle{5 задушевных украинских песен, которые пели наши родители, а теперь поем мы}, 
Кино Вояж И Не Только, zen.yandex.ru, 07.06.2021
%%%endcit

%%%cit
%%%cit_pic
\ifcmt
  pic https://avatars.mds.yandex.net/get-zen_doc/4719986/pub_60be8c24942ceb0598e0db47_60be93a4bc58736775579a95/scale_1200
  caption Уже 7 лет как Крым в составе России. Источник: Яндекс.Картинки
\fi
%%%cit_text
В этой форме они будут играть на Евро 2020 (21). Форму для их сборной выпускает
испанская компания Joma.  Очень странной оказалась реакция из \emph{России},
особенно от представителей власти. Ведь с самого начала стало ясно что это
провокация со украинской стороны. А наши политики стали комментировать это.
Вопрос принадлежности Крыма для \emph{России} уже давно решен, еще с 2014 года.
То что многие страны его не признали, это их дело. Ждать признания с их стороны
дело
неблагодарное
%%%cit_comment
%%%cit_title
\citTitle{Какая разница на каких футболках Крым, главное он в составе России}, 
Добрый Человек, zen.yandex.ru, 11.06.2021
%%%endcit

%%%cit
%%%cit_pic
%%%cit_text
За каждый гол бельгийцев в ворота сборной \emph{России} пришедшим обещают 100
бокалов пива бесплатно. «Вы же рискуете обанкротиться», — написал в
комментариях один из пользователей «Инстаграма». Но многие оценили такую акцию
и похвалили организаторов фестиваля
%%%cit_title
\citTitle{Обзор \#ЕВРО2020. Матч открытия в Риме. Россия потеряла Мостового. А на Украине будут праздновать голы в ворота России}, 
Пластырь Тв, zen.yandex.ru, 11.06.2021
%%%endcit

%%%cit
%%%cit_pic
%%%cit_text
В субботу, 12 июня, в Санкт-Петербурге перед стартовым свистком в матче
Чемпионата Европы по футболу 2020 бельгийская сборная преклонила колени в знак
поддержки движения Black Lives Matter. За это их освистали \emph{российские}
болельщики. Об этом сообщается Тwitter-аккаунте \enquote{Евроспорта}.
Футболисты сборной команды по футболу \emph{РФ} не стали преклонять колени.
Игроки бельгийской сборной стали на одно колено перед стартовым свистком и были
освистаны болельщиками
%%%cit_title
\citTitle{Футболисты российской сборной не преклонили колено в поддержку Black Lives Matter}, 
Эллина Либцис, strana.ua, 12.06.2021
%%%endcit

%%%cit
%%%cit_pic
%%%cit_text
Спасибо за такое количество людей для моей квартиры, за более трехсот
просмотров в час, за русскую и украинскую подлинную солидарность и отдельно за
экзорцистский эффект у бесов-фаши. Они думали, я уже прячусь в сибирской
тундре. \emph{Россия}, спасибо отдельно
%%%cit_title
Евгения Бильченко, facebook, 12.06.2021
%%%endcit

%%%cit
%%%cit_pic
%%%cit_text
Это тот случай когда спортивный результат не столь важен, когда честь дороже.
Лично я бы просто не стал смотреть матч \emph{Россия}-Бельгия после коленопреклонных
\emph{русских}. Из принципа. Нашей стране не за что просить прощения у негров, у
\emph{России} не было владений в Африке, в \emph{России} не было черных рабов, а редкие
чернокожие, кто всё же попадал в нашу страну, запросто могли сделать блестящую
карьеру. Всё зависело от личных качеств, не от цвета кожи
%%%cit_title
\citTitle{Первая победа наших парней: сборная России отказалась встать на колени}, 
Мак Сим, zen.yandex.ru, 12.06.2021
%%%endcit

%%%cit
%%%cit_pic
%%%cit_text
И, конечно, чтобы желающие спеть украинскую песню \emph{русские кумовья}, всегда
смогли бы спеть её на \emph{русском языке} так же легко и красиво, как это делается на
украинском языке. И чтобы праздник у обоих кумовьёв по этой причине никогда бы
не был испорчен. Ниже привожу тесты некоторых украинских песен пересказанных
мною на \emph{русском языке}
%%%cit_comment
%%%cit_title
\citTitle{Украинские Песни Русскими Словами}, 
БРАТИНА, zen.yandex.ru, 15.12.2020
%%%endcit

%%%cit
%%%cit_pic
%%%cit_text
В хвалебных тонах гол Ярмоленко прокомментировали \emph{российские телекомментаторы} с
Первого канала: \enquote{Ох, какой гол! Вот оно мастерство. Как хорошо, что некем
менять Андрея Ярмоленко, потому что не надо. Потому что он потрясающий гол
забивает, и Украина начинает догонять. Роскошный удар! Таких голов на этом
чемпионате еще не было. Идеальная траектория под перекладину. Ох, какая
красота}
%%%cit_comment
%%%cit_title
\citTitle{Украина - Нидерланды на Евро 2020. Отзывы на матч в Украине и России}, 
Оксана Малахова, strana.ua, 14.06.2021
%%%endcit

%%%cit
%%%cit_pic
%%%cit_text
\enquote{Таких голов на Евро еще не было}. Как россияне за Украину болели.
Среди \emph{российских} комментаторов действительно вчера царило единодушие по поводу
игры сборной Украины - многие откровенно болели против Нидерландов.  Но не
только среди них - простые болельщики тоже выступали за Украину.  В сети
появилось видео, как в Санкт-Петербурге на улице болельщики празднуют гол нашей
сборной. Украинцы забили два мяча в ворота голландцев против трех со стороны
Амстердама.  Видно, как россияне радуются второму голу сборной Украины.  \enquote{Вот
так радовались \emph{россияне} второму голу нашей сборной голландцам. Спорт все равно
победит политику!}, - прокомментировал этот ролик украинский телеведущий
Дмитрий Спивак
%%%cit_comment
%%%cit_title
\citTitle{Украина - Нидерланды на Евро 2020. Отзывы на матч в Украине и России}, 
Оксана Малахова, strana.ua, 14.06.2021
%%%endcit

%%%cit
%%%cit_head
Я россиянин! Россия моя страна!
%%%cit_pic
\ifcmt
  pic https://avatars.mds.yandex.net/get-zen_doc/3380298/pub_60c48c27b582164697cb2369_60c49b3f4a43ce56d5ae16fb/scale_1200
  caption Памятник Минину и Пожарскому.
\fi
%%%cit_text
Мы с Богом, а они забыли его. Что ж, они выбрали свой путь, а у нас свой,
светлый путь справедливости. И идём в правильном направлении.  Обожаю, как
каждый \emph{русский} и каждый \emph{истинный россиянин}, свою \emph{Россию},
нашу \emph{Россию}, любимую и суровую \emph{Матушку Русь}... А как стать
\emph{истинным россиянином}? Надо себе в мыслях сказать: \enquote{Я россиянин!
\emph{Россия моя страна!}}, и всё, ты наш уже навсегда, уже никак не отмоешься перед
Богом -- твои мысли услышаны и запечатаны. Ты \emph{россиянин}. Всё очень просто и
прозаично, потому народ наш всегда полнится и крепчает.  Да, многое пока
(Пока!) не складывается, как хотелось бы, но предпосылки есть, все предпосылки
есть к успеху! И нас он ждёт, потому что Высший разум за нас и наш общий,
коллективный разум \emph{российского народа} говорит нам, что мы на правильном пути
процветания и прогресса. Мы единственный народ (\emph{россияне}), кто, где бы
из нас не жил, всё время волнуется, переживает за \emph{Россию} и тоскует о
ней. Как она, родимая?
%%%cit_comment
%%%cit_title
\citTitle{Слава России! Мы на вершине могущества и спускаться более не намерены}, 
Алексей Наст, zen.yandex.ru, 12.06.2021
%%%endcit

%%%cit
%%%cit_head
%%%cit_pic
%%%cit_text
Смех смехом, а такое только у нашего многомиллионного, состоящего из сотен
народностей и наций, народа. Какая-то общая энергетическая положительная аура.
Мы все относимся друг к другу положительно, а они всегда злобятся, враждуют и
воинствуют. У нас не так -- мы один народ, \emph{россияне}, нам между собой делить
нечего, только помогаем. \emph{Россияне}!  Феномен: наша дружба и общая изумительная
взаимовыручка. Мы все, независимо от нации, хозяева \emph{нашей России}, мы любим и
обожаем нашу красивую, замечательную и огромную, могучую и несравненную Родину!
\emph{Феномен России} необъясним для Западной науки. Нам он понятен без объяснений, а
для них загадка. Как шмель, который летать по всем законам современной
аэродинамики летать не может, летает миллионы лет и летает более чем хорошо и
сбавлять оборотов не собирается, так и наша \emph{многонациональная замечательная
страна} живёт и, проходя сквозь годы лихолетий, только набирает силу и мощь!
%%%cit_comment
%%%cit_title
\citTitle{Слава России! Мы на вершине могущества и спускаться более не намерены}, 
Алексей Наст, zen.yandex.ru, 12.06.2021
%%%endcit

%%%cit
%%%cit_head
%%%cit_pic
\ifcmt
  pic https://strana.ua/img/forall/u/0/36/QIP_Shot_-_Screen_7372.png
\fi
%%%cit_text
\enquote{При этом весь трейлер на \emph{русском}. Теперь на Западе игру считают
\emph{российской}, а \emph{русня} бегает по сети и кричит про \enquote{отечественный} геймдев
}, - пишет пользователь Даниил Максименко.  \enquote{Потому что Чернобыль - это
Украина, и название должно быть украинское}, - считает еще один пользователь
соцсети
%%%cit_comment
%%%cit_title
\citTitle{Сталкер 2 Сердце Чернобыля - почему хейтят видеоигру украинских разработчиков}, 
Екатерина Терехова, strana.ua, 16.06.2021
%%%endcit

%%%cit
%%%cit_head
%%%cit_pic
%%%cit_text
Это сказано о \emph{Российской империи}. Это сказано о нас.  Начал читать
(слушать) книгу \enquote{Империя должна умереть} Михаила Зыгаря. Многие цитаты
можно прямо применить к сегодняшней Украине. И встаёт вопрос - а чем мы
отличаемся от \emph{Российской империи} конца XIX - начала XX века?  "Я начинаю
книгу на рубеже XIX и XX веков. Это очень интересное время. Многие молодые
столичные интеллектуалы – поколение нулевых – еще аполитичны... Они считают
политику чем-то вчерашним, неинтересным и немодным. Но политика резко
вторгается в их жизнь, власти вмешиваются в свободу творчества, запрещая и
закрывая все, что их не устраивает. Так постепенно начинаются первые в
российской истории массовые митинги протеста – и то, как их подавляют,
привлекает к ним все больше внимания. В течение нескольких лет в \emph{России}
возникает гражданское общество – активное, требовательное и сознательное.
... Креативный класс требует всеобщих выборов, создания парламента, свободы
слова, равенства перед законом – и уверен, что добьется своего... В
\emph{России} «закручивают гайки», многие считают, что пора валить
%%%cit_comment
%%%cit_title
\citTitle{Страной управляет стая жадных, пронырливых, безнравственных чиновников}, 
Павел Себастьянович, strana.ua, 18.06.2021
%%%endcit

%%%cit
%%%cit_head
%%%cit_pic
\ifcmt
  pic https://avatars.mds.yandex.net/get-zen_doc/2369622/pub_60d1a80b5fc3481f3f16c7af_60d2f31c9e2a48409aa0bae2/scale_1200
  width 0.4
\fi
%%%cit_text
Мы называем себя \emph{Россией}, это же название используют и большинство наших
иностранных партнёров.  Ничего необычного: \emph{Russia, Russie, Rússia, Russland} и
прочие вариации. Даже на арабском всё понятно (на слух) - روسيا (\emph{\enquote{Русия}}). Все
эти слова - переработанное самоназвание \enquote{Русь}.
%%%cit_comment
%%%cit_title
\citTitle{Загадочное слово \enquote{Россия}}, 
Языковедьма, zen.yandex.ru, 23.06.2021
%%%endcit

%%%cit
%%%cit_head
%%%cit_pic
%%%cit_text
Как бы то ни было, большинство ученых считают, что термин \emph{\enquote{рус}} - не
славянское. Вернадский, например, выводит это слово из иранского \enquote{рухс}
(\enquote{светлый, белый}), и полагает, что скифы, сарматы и аланы так назвали своих
соседей, славян.  В западных языках видят родство \emph{\enquote{Руси}} с известным корнем,
обозначающим красноту или рыжину (\enquote{rouge}, \enquote{red}, \enquote{roux}). Эта версия
подтверждается тем, что арабский путешественник Аль-Масуди в X веке писал: есть
некий народ, который византийцы называют \enquote{русие}. Его \enquote{коллега} Ибн-Фадлан
лично видел их, и тоже упоминает красноту. Но что это было, румянец, загар или
краснота от холода? И тот ли это народ?
%%%cit_comment
%%%cit_title
\citTitle{Загадочное слово \enquote{Россия}}, 
Языковедьма, zen.yandex.ru, 23.06.2021
%%%endcit

%%%cit
%%%cit_head
%%%cit_pic
%%%cit_text
Ну и ещё немного о том, как перевирают наше название :) Венгры называют \emph{Россию}
Оросорсаг. По-венгерски \enquote{орсаг} это \enquote{страна}, ну а в части \enquote{орос} вполне
узнаётся наш корень. Кстати, себя они называют \enquote{Мадьярорсаг}, то есть \enquote{страна
мадьяров}. Так что мы их название намного сильнее переврали.  Монголы раньше
называли нас Урус. Сейчас называют Орос. Китай, кстати, понабрался от них, но
не смог выговорить \enquote{р}, и получилось Элосы (俄国), что при переводе иероглифов
означает \enquote{внезапная страна}.  Если в Монголии нас нас назвать \enquote{РФ}, то
получится ОХУ, а с суффиксом принадлежности \enquote{ын} - ОХУ-ын
%%%cit_comment
%%%cit_title
\citTitle{Загадочное слово \enquote{Россия}}, 
Языковедьма, zen.yandex.ru, 23.06.2021
%%%endcit

%%%cit
%%%cit_head
%%%cit_pic

\ifcmt
  tab_begin cols=2
  width 0.3

     pic https://avatars.mds.yandex.net/get-zen_doc/1880741/pub_60cb23ffa497d93fa903a37a_60cb28341b56a81aefb68a26/scale_1200
     caption Что это: белокаменные рельефы Дмитриевского собора во Владимире, 1194-1197 годы. Съёмка, коллаж — У\&З (извините)

     pic https://avatars.mds.yandex.net/get-zen_doc/2377661/pub_60cb23ffa497d93fa903a37a_60cb2491a7be3029d48322d0/scale_1200
     caption Случай с Александром Македонским, Дмитриевский собор, Владимир

  tab_end
\fi

%%%cit_text
На высоком берегу Клязьмы, в 150 метрах к небу над Владимирским Опольем, парит
белокаменное чудо — самый красивый, и, пожалуй, самый сложный по сюжету храм
\emph{России}.  Съездил поцеловать камни Дмитриевского собора, отчитываюсь
%%%cit_comment
%%%cit_title
\citTitle{Македонский всея Руси: древнейший наш памятник пропаганды}, 
Увидел Зацепило, zen.yandex.ru, 19.06.2021
%%%endcit

%%%cit
%%%cit_head
%%%cit_pic
%%%cit_text
\enquote{Намеренно зашел в воды Крыма}.
При этом корреспондент Би-би-си по вопросам обороны Джонатан Бил, который
находился на борту \enquote{Дефендера}, рассказал, что заявления Британии и \emph{России} не
вполне соответствуют тому, что он сам наблюдал.  По его словам, он насчитал в
небе над эсминцем не менее 20 \emph{российских} военных летательных аппаратов, а два
российских патрульных катера в некоторые моменты приближались к нему на
расстояние примерно в сотню метров.  В своем репортаже Бил описывает
воинственные предупреждения, передаваемые по радиосвязи с \emph{российских катеров} в
тот момент, когда команда британского корабля готовилась к возможной
конфронтации.  \enquote{Если вы пересечете границу, я открою огонь, если вы пересечете
границу, я буду стрелять, вы меня слышите?} - предостерегают \emph{российские военные}
экипаж эсминца
%%%cit_comment
%%%cit_title
\citTitle{Что пишет об инциденте с кораблем \enquote{Дефендер} британские СМИ}, Александра Харченко, strana.ua, 24.06.2021
%%%endcit

%%%cit
%%%cit_head
%%%cit_pic
%%%cit_text
В списке \emph{российские военные}, в том числе члены ЧВК \enquote{Вагнер}, воевавшие на
Донбассе (в частности командир \enquote{Вагнера} Уткин). \emph{Российские банкиры} Андрей
Акимов (Газпром-банк), Андрей Костин (ВТБ банк) и др.  В санкционный список
также попали \emph{российские} журналисты и главные редакторы изданий - Маргарита
Симоньян (\emph{Russia Today}), Константин Эрнст (первый канал), Виталий Лейбин
(\emph{Русский репортер}), Павел Гусев (МК)
%%%cit_comment
%%%cit_title
\citTitle{Полный список лиц, против которых ввел санкции Зеленский}, Екатерина Терехова; Оксана Малахова, 
strana.ua, 24.06.2021
%%%endcit

%%%cit
%%%cit_head
%%%cit_pic
%%%cit_text
Правда, \enquote{всестороннее развитие и функционирование украинского языка во всех
сферах общественной жизни на всей территории Украины} не значит, что ТОЛЬКО
украинского и путём запрета \emph{русского}. Украинский язык вполне может
функционировать вместе с \emph{русским}. Потому что та же статья 10 Конституции
Украины гарантирует свободное использование, развитие и защиту \emph{русского языка} в
Украине. Но мудакам плевать на Конституцию, а развитие украинского для них =
запрет \emph{русского}
%%%cit_comment
%%%cit_title
\citTitle{Защищать украинский язык можно и без угнетения русского / Лента соцсетей / Страна}, 
Александр Скубченко, strana.ua, 28.06.2021
%%%endcit


%%%cit
%%%cit_head
%%%cit_pic
%%%cit_text
Доба \emph{Русі}. Тисячоліття після класичної Античності Основні закони на території
України якщо і ухвалювалися, то ми про них нічого не знаємо. А от з Х століття
відомості у нас надійні.  Як не дивно, \emph{Русь} приєдналася до міжнародного права
раніше, ніж отримала державне. 911 року між Києвом і Константинополем був
підписаний перший договір (\enquote{договір} 907 року, як і одночасний \enquote{похід},
помилково датовані), який визначав взаємини двох держав і порядок судочинства
наших купців у Візантії. Потім норми цієї та пізніших угод з Візантією сильно
впливали на право \emph{Русі}
%%%cit_comment
%%%cit_title
\citTitle{Сергій Громенко: Українській Конституції – 2300 років}, 
, gazeta.ua, 26.06.2021
%%%endcit


%%%cit
%%%cit_head
%%%cit_pic
%%%cit_text
Литовсько-польська доба. У складні XIII-XIV століття на більшій частині
України, яка перебувала під литовським контролем, продовжувало діяти
\emph{руське право}. Правителі сповідували принцип: \enquote{старого не рушимо,
нового не вводимо}.  Лише в окремих містах з'являється Магдебурзьке право, яке
забезпечувало внутрішнє самоврядування жителів і виборність місцевої влади.
Остаточний перехід від \emph{руського} до власне литовського права почався із
введенням 1468 року судебника великого князя Казимира IV, який визначав нові
норми судочинства.  Згодом були видані ще три таких судебника, але вищої сили,
яка належала Литовським статутам, вони не мали
%%%cit_comment
%%%cit_title
\citTitle{Сергій Громенко: Українській Конституції – 2300 років}, 
, gazeta.ua, 26.06.2021
%%%endcit


%%%cit
%%%cit_head
%%%cit_pic
%%%cit_text
Під \emph{Росією}.  У період остаточного поглинання України \emph{Росією} правознавці зробили
кілька спроб кодифікації власне українського права – щоб якщо не в політичній,
то хоч в юридичній площині відстояти свої позиції. \enquote{Права, за якими судиться
\emph{малоросійський народ}} 1743 року, \enquote{Суд і розправа в правах \emph{малоросійських}} 1750
року, \enquote{Екстракт \emph{малоросійських прав}} 1767 року і \enquote{Екстракт з указів, інструкцій
і установлень} 1786 року так і не стали офіційно визнаними, але широко
використовувалися у діловодстві
%%%cit_comment
%%%cit_title
\citTitle{Українській Конституції – 2300 років}, 
Сергій Громенко, gazeta.ua, 26.06.2021
%%%endcit

%%%cit
%%%cit_head
%%%cit_pic
%%%cit_text
В этой связи есть несколько вопросов. Почему латиноязычную копию называют
\enquote{оригинал}? Оригинал на \emph{русском языке} (его еще стыдливо называют
\enquote{староукраинским}) хранится в \emph{Российском} государственном архиве древних актов и
вроде бы никуда не собирается.  Что будем делать с преамбулой документа, ведь
там есть преступно нецензурный термин \enquote{всего народу \emph{малороссийского}}?
И, наконец, как себя будет чувствовать Владимир Александрович, когда узнает,
что \enquote{проживать иноверцам на Украине, а особливо зловерью иудейскому, не
дозволять}?
%%%cit_comment
%%%cit_title
\citTitle{Знает ли Зеленский суровую правду о конституции Пилипа Орлика? / Лента соцсетей / Страна}, 
Дмитрий Заборин, strana.ua, 29.06.2021
%%%endcit

%%%cit
%%%cit_head
%%%cit_pic
%%%cit_text
Сколько людей получили гражданство \emph{РФ} ДО майдана сказать сложно, только
вот костяк моего класса устроил вечер встреч выпускников в... в Химках в 2012
году, ведь 16 из 24 бывших одноклассников были \emph{россиянами} на тот момент,
ещё двое имели вид на жительство.  Я жил и учился на востоке Украины, наверняка
западенцы точно так же по возможности легализуются в Европе. Формально
оставаясь украинцами, встречи одноклассников организовывают в предместьях
Варшавы и Праги. Не знаю, но не удивлюсь...
%%%cit_comment
%%%cit_title
\citTitle{Страшная тайна демографии Украины}, 
Мак Сим, zen.yandex.ru, 28.06.2021
%%%endcit

%%%cit
%%%cit_head
%%%cit_pic
%%%cit_text
\enquote{Hpaвитcя вaм этo или нeт, нo \emph{Poccия} являeтcя кpyпным игpoкoм нa
миpoвoй apeнe, и она увеличила свое политическое присутствие во многих частях
мира, в том числе в странах, где на карту поставлены интересы ЕС. Есть также
глобальные проблемы, в решении которых в наших интересах привлечь \emph{Россию},
потому что не решение этих проблем затронет всех нас}, — заявил Верховный
представитель ЕС по иностранным делам и политике безопасности Жозеп Боррель
%%%cit_comment
%%%cit_title
\citTitle{У Зеленского уже поняли, что курс Киева и Запада совпадает все меньше / Лента соцсетей / Страна}, 
Олег Волошин, strana.ua, 29.06.2021
%%%endcit

%%%cit
%%%cit_head
%%%cit_pic

\ifcmt
  tab_begin cols=3

     pic https://strana.ua/img/forall/u/10/88/E5FWiu3WEA0aOjA.jpg
     width 0.39

     pic https://strana.ua/img/forall/u/10/88/E5FWi9nWEAcHVt7.jpg
     width 0.2

     pic https://strana.ua/img/forall/u/10/88/E5FWjKJXEAYAtpx.jpg
     width 0.2

     caption На матче Украина - Швеция российскому болельщику порвали футболку

  tab_end
\fi

%%%cit_text
Болельщику, который пришел с \emph{российским триколором}, в шапке с советской
символикой на матч Украина - Швеция и обнимался с украинским фанатом, порвали
футболку.  Снимки \emph{россиянина} до и после нападения публикует Ofnews в
Twitter.  Инцидент произошел на трибуне с украинскими фанатами. Сначала мужчина
с триколором обнимался с болельщиком с украинским флагом. После этого, судя по
всему, ему порвали футболку. В сети появились снимки, на которых мужчина стоит
в уже разорванной одежде и держится рукой за лицо
%%%cit_comment
%%%cit_title
\citTitle{На матче Украина - Швеция российскому болельщику порвали футболку}, 
Игорь Рец, strana.ua, 30.06.2021
%%%endcit

%%%cit
%%%cit_head
%%%cit_pic
\ifcmt
  pic https://strana.ua/img/forall/u/11/33/1608286068-513.jpg
  width 0.3
\fi
%%%cit_text
Одновременно с этим \emph{президент России} начертил красные линии для такого
обсуждения: прекращение провокаций вокруг \enquote{\emph{российского Крыма}} (то есть
признание де факто его аннексии), прекращение ущемления \emph{русского языка} в
Украине.  Однако и это еще не все. Довольно пространный ответ относительного
\enquote{\emph{единого русского народа}}, в который входят и \emph{россияне}, и украинцы – это уже
задел на более широкую и длительную программу, конечной целью которой является
если не единое государство, то, по крайней мере, единое геополитическое
пространство наподобие Евросоюза. А в ближайшей перспективе это – сигнал
неназванным при обсуждении белорусам. Ведь, согласно классическому определению,
\enquote{триединый русский народ} включает \emph{великороссов}, украинцев и \emph{белорусов}. То есть
последние – это тоже, в логике Путина, единый народ с \emph{русскими}
%%%cit_comment
%%%cit_title
\citTitle{Миллионы на украинскую идентичность, зрада на Евро-2020, планы Путина по Украине}, 
, strana.ua, 01.07.2021
%%%endcit

%%%cit
%%%cit_head
%%%cit_pic
%%%cit_text
Мне уже приходиломсь не соглашаться с тезисом В. Путина о том, что украинцы и
русские - один народ. Во-первых, понятие \enquote{народ} слишком многозначно и не
случайно сегодня предпочитают говорить о \enquote{нации}, а слову \enquote{народ}, оставляют
смысл \enquote{граждане одного государства} (см. многонациональный русский народ,
\enquote{россияне}). Во-вторых, как можно говорить, о том, что русские и украинцы (в
смысле - граждане России и Украины) один народ, если даже \enquote{украинцы} - не один
народ, а, как минимум - два. Про один из них (жителей Новороссии, и с большой
натяжкой - часть жителей Гетьманщины) можно сказать, что они и русские - один
народ. А про жителей Западной Украины этого сказать нельзя и вряд ли кто-то
станет с этим спорить
%%%cit_comment
%%%cit_title
\citTitle{Украина остается лоскутным одеялом, расколотой страной / Лента соцсетей / Страна}, 
Михаил Погребинский, strana.ua, 01.07.2021
%%%endcit

%%%cit
%%%cit_head
%%%cit_pic

\ifcmt
  pic https://odnarodyna.org/sites/default/files/styles/adaptive/public/article/2021/%D0%A1%D0%BD%D0%B8%D0%BC%D0%BE%D0%BA%20%D1%8D%D0%BA%D1%80%D0%B0%D0%BD%D0%B0%202021-06-30%20%D0%B2%2016.31.58.png?itok=YJmCHhd2
	width 0.4
\fi

%%%cit_text
Развитие \emph{русского языка} на Украине – процесс естественный и нормальный. А вот
попытки запретить его – явление противоестественное. О том, что это так, помимо
воли свидетельствует опубликованная газетой украинцев США «Свобода» статья под
громким названием «Язык и построение государства». Её автор, постоянный
сотрудник медиакорпорации Конгресса США доцент-филолог Львовского
государственного университета Олег Романчук, жалуется на то, что в украинский
язык якобы проникает всё больше и больше \emph{русизмов} – заимствований из \emph{русского}.
«В Украине государственный язык – это прежде всего категория политическая и
идеологическая, – пафосно утверждает Романчук. – О языковых реалиях в Украине
сегодня можно судить по выступлениям на радио и телевидении, газетным и
журнальным публикациям, где привычные украинские лексемы день за днём, месяц за
месяцем, год за годом оказываются вне языкового закона, а ненормативные \emph{русизмы}
в общественно-политической лексике, в быту, к сожалению, уже стали нормой»
%%%cit_comment
%%%cit_title
\citTitle{Украинским националистам не нравится близость украинского языка к русскому}, Сергей Бондаренко, odnarodyna.org, 30.06.2021
%%%endcit

%%%cit
%%%cit_head
%%%cit_pic
%%%cit_text
Данные заявления были не единичными. Мнение о \enquote{братских народах}
проходит красной линией в других интервью Зеленского. Он также был уверен:
говорить \emph{по-русски} и любить Украину можно.  \enquote{Если на востоке и в Крыму
люди хотят говорить \emph{по-русски}, отцепитесь, отстаньте от них. На законном
основании дайте им говорить \emph{по-русски}. Язык никогда не будет делить нашу родную
страну. У меня еврейская кровь, я говорю \emph{по-русски}, но я гражданин Украины и
люблю эту страну. \emph{Россия} и Украина – действительно братские народы}, - говорил
Зе в 2014-м в эфире \enquote{1+1}
%%%cit_comment
%%%cit_title
\citTitle{Зеленский о русских и русском языке. Что будущий президент говорил в 2014 году}, 
Екатерина Терехова, strana.ua, 03.07.2021
%%%endcit

%%%cit
%%%cit_head
%%%cit_pic
\ifcmt
  pic https://strana.ua/img/forall/u/0/36/2021-07-03_14h28_05.png
	width 0.4
\fi
%%%cit_text
\enquote{Абсолютно нормально, они же играли международные все. Да хоть на английском. А
вы хотели только на украинском?}, - задает нам встречный вопрос женщина. И
подлавливает нас на том, что мы задаем вопрос тоже на \emph{русском}.  \enquote{А вы мне
задаете вопрос тоже на \emph{русском} - и я вам отвечаю на \emph{русском}! Мы же в центре
столицы. Все нормально. Чем больше знаешь языков, тем лучше. Это никакого
отношения не имеет к конфликту нашей страны и противоположной нашей соседки.
Это все по-доброму. Это все хорошо... Они же-то за Украину играли, они же не
играли за \emph{Россию}. Ну, такое это поколение. Хотите разговаривать только на
украинском, подождите вы, пока пройдет наше поколение. И будете разговаривать
только на украинском. Все это впереди}, - аргументирует участница опроса свой
ответ
%%%cit_comment
%%%cit_title
\citTitle{Что говорят украинцы о пресс-конференциях футболистов на русском языке. Опрос Страны}, 
Антонина Белоглазова, strana.ua, 03.07.2021
%%%endcit

%%%cit
%%%cit_head
%%%cit_pic
\ifcmt
  pic https://strana.ua/img/forall/u/0/36/2021-07-03_15h17_58.png
	width 0.5
\fi
%%%cit_text
\enquote{\emph{Русский язык} все понимают}.  \enquote{Я поддерживаю. Всем нужно говорить на \emph{русском
языке}. \emph{Русский язык} все понимают. Пусть критикуют те, кто не по-украински
нормально говорить не могут, не \emph{по-русски}. Вот они и критикуют. Всю жизнь
говорили \emph{на русском}, никто не умирал}, - была лаконична одна из опрошенных.
%%%cit_comment
%%%cit_title
\citTitle{Что говорят украинцы о пресс-конференциях футболистов на русском языке. Опрос Страны}, 
Антонина Белоглазова, strana.ua, 03.07.2021
%%%endcit

%%%cit
%%%cit_head
%%%cit_pic
%%%cit_text
І хоча важко сказати на чому ґрунтується така вперта прихильність канцлера
Меркель до \emph{російського} диктатора, проте, коли вона вперто пропонує
\enquote{перезавантаження відносин} з \emph{Росією}, то це означає одне, в Берліні
готові обнулити всі попередні злочини Путіна, і цим надати йому індульгенції на
усі злочини наступні.  Забувши про загарбання \emph{Росією} українського Криму
і окупацію частини Донбасу, про те, що Москва направляла агентів для вбивств
своїх противників по всій Європі, використовуючи в європейських країнах хімічну
зброю, вчинила терористичні акти, знищивши військові об'єкти в Чехії та
Болгарії
%%%cit_comment
%%%cit_title
\citTitle{Європа сподівається, що Путін буде задоволений вбивством сусідів і далі не піде}, 
Віктор Каспрук, gazeta.ua, 29.06.2021
%%%endcit

%%%cit
%%%cit_head
%%%cit_pic
%%%cit_text
У той час, як міжнародна увага зосереджена на західних антиросійських санкціях,
Україна і \emph{Росія} замкнуті в своєму циклі все більш жорстких санкцій, які грають
важливу роль в триваючій семирічній гібридній війні між двома країнами.
Зазвичай такі обміни ініціює \emph{Росія}, а Україна відповідає.  24 червня президент
Зеленський підписав розширені санкції проти \emph{Росії}, спрямовані проти 538 осіб і
540 організацій. У той же день він видав черговий указ про введення санкцій
щодо 55 \emph{російських} державних фінансових установ. Ці кроки є найсуворішими, які
будь-яка країна ввела проти \emph{Росії} з 2014 року, - пише Андерс Аслунд для
Atlantic Council
%%%cit_comment
%%%cit_title
\citTitle{Україна завдала потужного удару по Кремлю. Захід має взяти приклад}, 
Андерс Аслунд, gazeta.ua, 30.06.2021
%%%endcit

%%%cit
%%%cit_head
%%%cit_pic
\ifcmt
  pic https://gdb.rferl.org/64F111FC-CE00-4A42-8FC1-671FE37331C2_w650_r0_s.jpg
	caption Гасло: «Тюркські народи, українці, поляки, єднайтеся у боротьбі проти спільного ворога – путінської агресії. Польща. Україна. Татарстан» Нафіс Кашапов, активіст із Татарстану, під час демонстрації у Варшаві, 23 листопада 2014 рок
	width 0.5
\fi
%%%cit_text
Вершина \emph{російської} бюрократичної дурості – це коли Державна дума \emph{Росії}
забороняє татарському народу використовувати латиницю замість кирилиці. Тому що
в усіх мовах народів \emph{Росії} може використовуватися тільки кириличний алфавіт – і
крапка! І виходить, що татарська мова, одна з найбільш високорозвинених і
древніх тюркських мов, відірвана від інших мов своєї мовної групи тільки тому,
що так вирішили у Москві. І це не обмеження прав, ні в якому разі. А обмеження
прав в «розумінні Кремля» – це відсутність \emph{росіян} у переліку корінних народів
України
%%%cit_comment
%%%cit_title
\citTitle{Корінні народи в Україні та Росії. Чим обурений Путін?}, 
Віталій Портников, www.radiosvoboda.org, 03.07.2021
%%%endcit

%%%cit
%%%cit_head
%%%cit_pic
\ifcmt
  pic https://strana.ua/img/forall/u/0/25/%D0%A1%D0%BD%D0%B8%D0%BC%D0%BE%D0%BA_%D1%8D%D0%BA%D1%80%D0%B0%D0%BD%D0%B0_2021-06-30_%D0%B2_19.55_.30_.png
  width 0.4
\fi
%%%cit_text
На самом деле \emph{российские звезды} даже после начала войны на Донбассе
продолжали выступать в Украине.  \enquote{\emph{Российские артисты} в Украине
всегда были, есть и будут. Выступают регулярно, не впервые, вне зависимости от
того, какие тут поднимают \enquote{волны}. Дебютов уже давно нет, организаторы
возят некоторых исполнителей уже по десятому кругу}, – говорит \enquote{Стране}
Дмитрий Чинь, совладелец и директор по маркетингу билетного оператора
Concert.ua.  Почему же в этом сезоне в Украине выступает так много
\emph{российских звезд}?  Ответ одновременно и прост, и сложен: потому что их
хотят видеть украинские зрители.  Опрошенные \enquote{Страной} участники
концертной индустрии считают, что украинцы слушают ту музыку, которая им
нравится, на понятном им языке. А из какой страны артист, зрителей интересует в
последнюю очередь
%%%cit_comment
%%%cit_title
\citTitle{Поющий реванш. В Украине этим летом выступят десятки российских артистов}, 
Анастасия Товт; Полина Пронина, strana.ua, 01.07.2021
%%%endcit


%%%cit
%%%cit_head
%%%cit_pic
%%%cit_text
В обещании притеснений \emph{русскоговорящих} он не видит проблемы.  Президент Украины
Владимир Зеленский, 26 апреля 2021года: «Я считаю, что это продолжение этого
информационного нарратива о притеснениях \emph{русскоязычных людей}. Я здесь не вижу
никакой проблемы».  Народный депутат Украины от партии «Слуга народа» Олег
Семинский, 2 июля 2021 года: «\emph{Россияне} на законодательном уровне – не коренной
народ, потому не будут иметь возможности в полной мере владеть всеми правами
человека и основоположными свободами, определенными нормами международного
права, а также предусмотренными в Конституции и законах Украины»
%%%cit_comment
%%%cit_title
\citTitle{Дискриминационные заявления Слуг не тревожат Зеленского / Лента соцсетей / Страна}, 
Максим Могильницкий, strana.ua, 03.07.2021
%%%endcit

%%%cit
%%%cit_head
%%%cit_pic
%%%cit_text
Отвечу свидомым, что лично я, мои знакомые болельщики, комментаторы на ТВ и
журналисты, то есть реально \emph{ВСЯ РОССИЯ} радовалась вашим футбольным свершениям.
Молодцы!!!
%%%cit_comment
%%%cit_title
\citTitle{Мова, политика и сборная Украины по футболу}, Мак Сим,
zen.yandex.ru, 04.07.2021
%%%endcit

%%%cit
%%%cit_head
%%%cit_pic
%%%cit_text
Но самое главное, что Путина не так беспокоит тема эсминца, как военное
освоение Украины, которая создает для \emph{России} \enquote{существенные проблемы в сфере
безопасности} и \enquote{касается реально уже жизненных интересов \emph{Российской Федерации}
и народа \emph{России}}. И при этом \emph{президент России} неоднократно подчеркивал, что
\emph{русские} и украинцы - один народ, при этом украинцы сами по себе дружественно
настроенный по отношению к \emph{русским}, а вот киевское руководство, выполняющее
полученные в Вашингтоне и иногда в Берлине и Париже указания, – нет
%%%cit_comment
%%%cit_title
\citTitle{Луганский Информационный Центр – НЕДЕЛЯ ГЛАЗАМИ ЭКСПЕРТА: 
Пеленг дозволенного, штампы внешнего управления и выходной бастард}, , lug-info.com, 04.07.2021
%%%endcit
