% vim: keymap=russian-jcukenwin
%%beginhead 
 
%%file 23_12_2022.fb.dedova_natalia.mariupol.2.r_i_p__22_grudnya_20.cmt
%%parent 23_12_2022.fb.dedova_natalia.mariupol.2.r_i_p__22_grudnya_20
 
%%url 
 
%%author_id 
%%date 
 
%%tags 
%%title 
 
%%endhead 

\qqSecCmt

\iusr{Татьяна Гирина}

Вспоминаю 18 февраля 2022 года, мы были в театре. Спектакль с Горянским.
Долго смотрела на люстру, потом мысль, а вдруг она упадет...
😞

\iusr{Janina Skaskiewicz}

У меня много лет хранился стеклярус с театральной люстры, Леонид Григорьевич
подарил мне наверно в 1979 году. Держала этот кристаллик в руках наверно в 2015
году. Театр - стержень Мариуполя. 😥

\iusr{Anastasiya Svetlichnaya}

Наташ, не могу остановиться, час назад увидела видео😭 для меня театр с весны был
уже Мемориалом, могилой братской и памятью... все снесли... разрывает от всего
этого

\iusr{Svetlana Kudrina}

Всегда любила эту люстру. Теперь любимый театр только на фотографиях

\iusr{Oksana Galchenko}

Як же боляче!!! Слава Богу Архіпов не побачив цього жихіття, для якого театр
був всім його життям...💔 Дякую, Наталочко, за відео минулих років.

\iusr{Наталья Авлиякулова}

А я, поражаюсь людям......🤦, которые поставили "Класс" данной публикации‼️ Если что, ТАМ есть другие СМАЙЛЫ 🙄😏‼️‼️

\iusr{Ксения Гармаш}

У меня тоже есть фото этой люстры, сделанное мной. Я даже публикацию в инстаграме делала о люстре.

\iusr{Anna Vitvitskay}

Так больно 😪 всю ночь сегодня снился наш театр ...

\iusr{Наталя Рижкова}

Я плачу...

\iusr{Альона Кондур}

Біль ...розпач, що нічого вже не змінити.... біль за загиблими...

\iusr{Натали Кина}

Так боляче за наш прекрасний рідний театр, ніби варвари добили пораненого, який вижив... Не можу... 💔💔💔

\iusr{Нина Муравченко}

В детстве, когда в театр приходили, эта люстра казалась какой-то волшебной.

Когда в зале гас свет - всегда смотрела не на сцену, а на люстру. А потом на
эту волшебную люстру так же смотрели мои дети.

Мы бывали и в других театрах, в других городах. Заходили в зал, а люстры там
нет

И уже - разочарование. Без люстры театр какой-то не настоящий.

\begin{itemize} % {
\iusr{Natalya Dedova}
\textbf{Нина Муравченко} Сашка в детстве называл люстру \enquote{пирожное}.
\end{itemize} % }

\iusr{Євдокія Фрушичева}

У меня тоже только память в виде фото осталась 😔

\ifcmt
  igc https://scontent-fra5-2.xx.fbcdn.net/v/t39.30808-6/321411321_1133286800888871_8234133201159398273_n.jpg?_nc_cat=109&ccb=1-7&_nc_sid=dbeb18&_nc_ohc=F-aHR7PLM1YAX_Bzj-k&_nc_ht=scontent-fra5-2.xx&oh=00_AfAmmaZ2Tg1B4BCfSTqI_Tg2ALN9CLDb8V21i9zbTGcTXg&oe=6412CA4D
	@width 0.4
\fi

\iusr{Dina Nichkalo}

Покидьки!

\iusr{Tetyana Dzhurik Levitska}

Все в душе сжалось. Вспомнила, как мы бежали туда в надежде спастись. Как мне
поначалу там стало спокойно и безопасно, столько людей, военные приезжают,
говорят обнадеживающие слова. От Киева отошли, от Харькова вышли на госграницу
. До сих пор не знаю, все ли это правда, или они просто нас успокаивали.
Помню, как люди хлопали и радовались этим новостям. Говорили, что Херсон не
сдан. Мы думали, что нас не сдадут тоже. А потом ночью по проспекту пролетел
огненный вихрь, такое летело, что мы остолбенели все. Тишина была, все дыхание
затаили. А оно не долетело и взорвалось где-то где переход, или ближе. И я
поняла, надо уходить. Его разобьют.

\begin{itemize} % {
\iusr{Natalya Dedova}
\textbf{Tetyana Dzhurik Levitska} расскажите мне свою и трию для проекта Голоса мирных. 💙💛

\iusr{Tetyana Dzhurik Levitska}
\textbf{Natalya Dedova} я хочу рассказать. Но никак не могу собираться с мыслями.

\iusr{Natalya Dedova}
\textbf{Tetyana Dzhurik Levitska} я помогу. Давайте. С понедельника. Начнем собираться. Ага?

\iusr{Tetyana Dzhurik Levitska}
\textbf{Natalya Dedova} хорошо

\iusr{Natalya Dedova}
\textbf{Tetyana Dzhurik Levitska} на зв'язку
\end{itemize} % }

\iusr{Hanna Krushynska}

А мы ничего не знали, сидели безвылазно у себя в хате в чс Черёмушки, весь ужас
узнала уже, когда появился нормальный интернет, 28 апреля, в Эстонии.

\iusr{Gala Isaeva}

А я помню запах кулис.....

\iusr{Екатерина Кибкало}

У меня ребенку 12 лет, она держалась все время пока были в блокаде в Мариуполе, и
после.... Первый раз сильно расплакалась, когда узнала, что попали в любименький
Драм Театр, второй раз, когда Стеллу перекрасили.

\iusr{Ольга Боглевська}

\enquote{Зачистка территории}, замітання слідів тяжкого воєнного злочину... Суки...

\iusr{Володимир Коскін}

Поставити на суд кожного, від працівників на тракторі до тих, хто віддавав накази.

Жодна мразота не повинна піти непокараною!

\iusr{Лилия Савицкая}

Боже, как можно стоять и спокойно смотреть. Я иду с остановки и стараюсь, чуть
ли не бегом пробежать, сердце сжимается так, что дышать невозможно

\iusr{Анна Бугаевская Лебедева}

Тварюкі
