% vim: keymap=russian-jcukenwin
%%beginhead 
 
%%file 13_02_2022.tg.lesev_igor.1.kuda_rossia_sprjatala
%%parent 13_02_2022
 
%%url https://t.me/Lesev_Igor/308
 
%%author_id lesev_igor
%%date 
 
%%tags napadenie,rossia,ugroza,ukraina
%%title Куда Россия спрятала 100 тысяч солдат на границе с Украиной?
 
%%endhead 
 
\subsection{Куда Россия спрятала 100 тысяч солдат на границе с Украиной?}
\label{sec:13_02_2022.tg.lesev_igor.1.kuda_rossia_sprjatala}
 
\Purl{https://t.me/Lesev_Igor/308}
\ifcmt
 author_begin
   author_id lesev_igor
 author_end
\fi

Куда Россия спрятала 100 тысяч солдат на границе с Украиной?

Россия сконцентрировала на границе с Украиной 100 тысячную группировку и
вот-вот готовится начать широкомасштабное вторжение в соседнюю страну.
Вторжение начнется, возможно, уже завтра, может на следующей неделе, в крайнем
случае в конце месяца или в начале следующего... Именно так, даже без
прилагательного «приблизительно», ежедневно звучат новостные, и что особенно
показательно, официально-должностные заявления из Вашингтона и его
стран-сателлитов.

Ситуация особенно идиотская, если учитывать, что первые сообщения «о неминуемом
вторжении» России на Украину появились в начале ноября прошлого года. В
дальнейшем интенсивность только возрастала, а с нового года переросла в
театрально-постановочные флешмобы. Сначала публиковались подробные «карты
вторжения», затем появились списки «оккупационного правительства», трогательные
легенды о создании «концентрационных лагерей для нелояльных украинцев» и,
наконец, последовала «эвакуация посольств» из Киева. Эвакуация оказалась тоже
специфической. Они как бы «эвакуировались», но при этом продолжают работать в
штатном режиме.

Оставим в стороне политическую часть вопроса, а именно, какие задачи решила бы
Россия, надумав захватить часть или даже всю Украину. Если следовать
пространным умозаключениям западных экспертов, можно докатиться до размышлений,
а вдруг Россия надумает захватить Монголию? Интересна в данном случае цифра в
«100 тысяч военных на границе с Украиной».

Для начала, сама цифра взята от фонаря, и не подтверждена вообще ничем. Если
следует заявление о «концентрации войск» в точке А, то эти войска по логике
должны были убыть с точки Б. Особенно, когда на Западе постоянно ссылаются «на
данные разведки». В солидном доказательном плане это должно выглядеть
приблизительно так. Такая-то бригада из Улан-Удэ прибыла в такой-то район
Ростовской области, а такой-то полк из-под Ижевска передислоцирован в
Белгородскую область.

Но западный политикум и СМИ не заморачиваются подобными доказательствами. Самым
громким «доказательством» с претензией на конкретность была историей с
концентрацией военной техники под Ельней. В начале декабря прошлого года
американское издание Politico показало «сенсационные снимки» скопления военной
техники, снабдив это комментарием «на границе с Украиной». Потом, правда, вышел
конфуз. Американским журналистам пояснили, что Ельня находится в Смоленской
области, которая даже не приграничная с Украиной. А расстояние от Ельни до
границы с Украиной почти 400 километров. Даже у Джейн Псаки, которая одна на
всю планету знает, где находится «Белорусское море», подобное расстояние может
вызвать некие сомнения.

Поэтому, с тех пор конкретика прекратилась. Западные СМИ стали ограничиваться
кадрами с военной техникой на железных дорогах России и просто военными
колоннами. Если где-то перевозят танки или едут военные грузовики, то ясен
пень, едут они исключительно к границе Украины. А куда ж еще, если Россия
вот-вот готовится вторгнуться на Украину?

И, тем не менее, Россия действительно наращивает военное строительство. И не
только на украинском направлении. Оставаясь ключевым геополитическим игроком на
международной арене, Россия свою исключительность в первую, в основную и в
главную очередь утверждает через наличие мощной армии. Вооруженные силы России
– это базовый фактор влияния страны в мире.

После крымских событий Россия кратно увеличила инфраструктуру военного
строительства. Открываются новые военные городки, новые базы в Арктике, на
Курилах и Камчатке, формируются или воссоздаются общевойсковые армии и дивизии.
Если мы говорим конкретно о приграничных областях с Украиной, то за последние 5
лет там появились три новые мотострелковые дивизии – 144-я в Брянской, 3-я в
Белгородской и 150-я в Ростовской области. Всего же в Южном военном округе за
последние 8 лет были сформированы 13 новых дивизий и бригад. При этом подобный
процесс происходит во всех военных округах, а приоритетным в создании новых
подразделений отдан даже не Южному, а Западному военному округу.
