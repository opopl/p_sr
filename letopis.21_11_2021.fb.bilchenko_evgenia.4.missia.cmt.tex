% vim: keymap=russian-jcukenwin
%%beginhead 
 
%%file 21_11_2021.fb.bilchenko_evgenia.4.missia.cmt
%%parent 21_11_2021.fb.bilchenko_evgenia.4.missia
 
%%url 
 
%%author_id 
%%date 
 
%%tags 
%%title 
 
%%endhead 
\subsubsection{Коментарі}

\begin{itemize} % {
\iusr{Борис Никифоров}
Совсем по Питерски...) Стародёжь?)) Круть

\begin{itemize} % {
\iusr{Евгения Бильченко}
\textbf{Борис Никифоров} просто люди, живущие Макаревичем 90-х, считают себя адептами молодого бунта, это потешно, когда либерализм и традиция поменялись местами: первый стал властью, по инерции имитирующей контркультуру, а второй - контркультурой. Сейчас в православной косынке в их клуб придти - мега-контркультура.

\iusr{Борис Никифоров}
\textbf{Евгения Бильченко} 

и это уж не первый год) Про молодёжь сказать ничего не могу-наслаждаюсь слепой
надеждой) А за стародёжь-это точно... Трагикомичное зрелище... на пути в
"Дядюшкин сон"))

\iusr{Евгения Бильченко}
\textbf{Борис Никифоров} да. Молодняк - это пофигизм и тик-ток, а стародёжь - это Макар, Быков, оседланный Бродский, Яблоко.

\iusr{Борис Никифоров}
\textbf{Евгения Бильченко} Осёдланный Бродский?)


\iusr{Евгения Бильченко}
\textbf{Борис Никифоров} да, Бродский - antimordor - без ампира и текста про Украину. Наполовину Иосиф - наполовину ИлонМаск

\iusr{Борис Никифоров}
\textbf{Евгения Бильченко} 

а.... понял... суженный))"Не меньше чем коммунистов я не люблю
антикоммунистов" Иосиф Свет Александрович не дожил до торжества
либерализма)) Хотя не трудно предположить его реакцию....

\iusr{Евгения Бильченко}
\textbf{Борис Никифоров} нетрудно, учитывая суицид Ильенкова после заявления его аспиранта: "Я отдам всего вашего Маркса за один абзац Бердяева". Не спорю, что это может быть легенда-подводка, но показательно.

\iusr{Борис Никифоров}
\textbf{Евгения Бильченко} 

Помню сто лет назад как только появилась возможность, купил в Питере его \enquote{Смысл
истории} тоненькую - чуть ли не на папиросной бумаге..) Когда у таких людей
находишь совпадения со своими догадками.... это конечно нечто

\end{itemize} % }

\iusr{Анастасия Рыжова}
Мне сначала показалось, что на крыше собака

\begin{itemize} % {
\iusr{Евгения Бильченко}
\textbf{Анастасия Рыжова} это руки-крюки фотографа, меня, грешного, плюс дб-увеличка на смарте. Прощенья просим)

\iusr{Анастасия Рыжова}
\textbf{Евгения Бильченко} да ничего страшного. Особенно, учитывая то, что собак я люблю. А тут я углядела весьма симпатичного лабрадора:))


\iusr{Евгения Бильченко}
\textbf{Анастасия Рыжова} ого, фантазия

\iusr{Анастасия Рыжова}
\textbf{Евгения Бильченко} Да, фантазия у меня развита хорошо:))


\iusr{Евгения Бильченко}
\textbf{Анастасия Рыжова} я заметила. Ты, наверно, облака видишь - самые разные по форме...

\iusr{Анастасия Рыжова}
\textbf{Евгения Бильченко} вижу в облаках разные образы. И знаю, что умными словами оно называется "парейдолии")
Но всё равно это очень интересно))

\iusr{Евгения Бильченко}
\textbf{Анастасия Рыжова} что ты видишь?

\ifcmt
  ig https://scontent-lga3-1.xx.fbcdn.net/v/t39.30808-6/259797088_4463675157000893_6571650284964476976_n.jpg?_nc_cat=101&ccb=1-5&_nc_sid=dbeb18&_nc_aid=0&_nc_ohc=edxqGxU9hsQAX_pLUto&_nc_ht=scontent-lga3-1.xx&oh=5a6aa8b46297296cb9c4a71a28e08841&oe=61A0FE82
  @width 0.4
\fi

\iusr{Анастасия Рыжова}
\textbf{Евгения Бильченко} вижу всадника (или всадников) на белых конях. Всадник в рогатом шлеме и с колчаном за спиной.
А тёмное небо - это силуэт единорога, или краба, если смотреть под другим углом)))


\iusr{Евгения Бильченко}
\textbf{Анастасия Рыжова} них\#се... Ну, ты мистериум ходячий. У меня мурахи

\iusr{Анастасия Рыжова}
\textbf{Евгения Бильченко} ну, я всего лишь описываю то, что вижу))

\end{itemize} % }

\iusr{Алексей Костромин}
C праздником!  @igg{fbicon.smile} 

\iusr{Helga Soloha}
Это великолепно!

\end{itemize} % }
