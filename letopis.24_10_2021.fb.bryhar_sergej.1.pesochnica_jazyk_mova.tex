% vim: keymap=russian-jcukenwin
%%beginhead 
 
%%file 24_10_2021.fb.bryhar_sergej.1.pesochnica_jazyk_mova
%%parent 24_10_2021
 
%%url https://www.facebook.com/serhiibryhar/posts/1817402848459659
 
%%author_id bryhar_sergej
%%date 
 
%%tags deti,jazyk,mova,ukraina
%%title Пісочничні хроніки (продовження)
 
%%endhead 
 
\subsection{Пісочничні хроніки (продовження)}
\label{sec:24_10_2021.fb.bryhar_sergej.1.pesochnica_jazyk_mova}
 
\Purl{https://www.facebook.com/serhiibryhar/posts/1817402848459659}
\ifcmt
 author_begin
   author_id bryhar_sergej
 author_end
\fi

\obeycr
Пісочничні хроніки (продовження).
Сьогодні в тій самій пісочниці зустріли того ж самого хлопця, що вчора говорив, що не розуміє українську (старшого за Назара на 2-3 роки).
Перше, що я почув ще на відстані кількох метрів: "ти чьо сєло?"
Хлопець, якому було адресоване питання трохи задумався, і видав:
- Нє. Я нє сєло.
- А чьо тада тупіш?
Поняття не маю, чого саме стосувалася суперечка.
- Я нє туплю.
- Ти тупіш, і нє спорь со мной.
- Нє.
- Даааа. Как сєлооо!
\restorecr
\ifcmt
  pic https://scontent-frx5-1.xx.fbcdn.net/v/t39.30808-6/248533133_1817402801792997_8188637352984581614_n.jpg?_nc_cat=100&ccb=1-5&_nc_sid=8bfeb9&_nc_ohc=nTkubenTbqwAX_xgwrW&_nc_ht=scontent-frx5-1.xx&oh=85002f42040202965d19ddbd886040d6&oe=617A4E95
  @width 0.8
\fi
\obeycr
Заходимо. Дістаємо іграшки. Назар починає коментувати свої дії...
- Назар, что-что ти сказал?  
- Я роблю паркан і хідник біля дороги.
- Что-что?
- Паркан роблю.
- Нє понімаю. Говорі нормально.
Звертається до мене:
- А он умєєт говоріть по-русскі?
- Ні. Він розмовляє українською.
Моя відповідь співрозмовника вочевидь не задовільняє. Він знову звертається до Назара:
- Скажи что-то по-русскі. Как я.
- Ніііііііі!
- Почєму?
- Бо ні. Я не буду.
- Почєму?
- Тому що я вже дорослий, і я кажу, що не буду, і я не буду - і все!
* * *
Тут варто пояснити: він у мене "бик" травневий, і впертий багато в чому. 
Звісно, я вчив його: "ми розмовляємо українською, бо ми українці, і на російську не переходимо". Але ця "наука", як то кажуть, зайшла на благодатний ґрунт. 
Іноді його впертість створює мені проблеми.
Але в багатьох випадках мене ця риса дуже навіть радує. От як зараз.
PS. Я не думаю, що отак от і жити нормально. Ця перманентна боротьба забирає безліч сил, які можна було би спрямувати у творчий конструктив. "Ти ж вистояв", - скажете ви. - "Так, я вистояв, але за це довелося платити серйозну ціну... і бажати своїм дітям того ж самого - якось не дуже чесно"...
Однак мати достатньо рішучості, щоб вміти дати відсіч, знати, що сказати, й ні на кого при цьому не оглядатися - обов'язково!
\restorecr

\begin{cmtfront}
\uzr{Yaroslav Lytovchenko}

Сергію, я реально захоплений Вашою витримкою і боротьбою в тому ватному
гадючнику на своїй рідній землі. І насолоджуюся Вашими постами. Побільше б
таких родин, як Ваша, і все буде добре в Україні. Дякую Вам!

\uzr{Микола Савка}

Треба було вуха понакручувати пі.дюку малому!

\uzr{Svitlana Chub-Krywuzka}

Та вони насправді жалюгідні, оті московитики.  Я ось читала цю розповідь,
уявляла, і мені стало смішно.  Які ж вони жалюгідненькі. Примітивщина. ''Ти
чьо, сєло, гаварі нармальна'' - ааааахахах. Оце і все, на що ви спроможні,
окупантські правнучата?)

\uzr{Людмила Демянчук}

Незавжди, діти дуже жорстокі, а ще от таке московське потороччя !  Добре, що
батько поряд, а так не знати , які б були дії малого москворотого шакала, який
навряд чи змінить своє відношення до українців.  Це той варіант, коли
презирство і ненависть до українського, це щеня всмоктало з молоком мамаші

\uzr{Viktor Voznyak}

Цей "узькоязикий рагулізм" не закінчиться доти , допоки на державному рівні не
будуть присікати розповсюдження "общєпанятного лайна" у всіх сферах діяльност і
потрібно діяти саме так, як це колись робили проти української мови !
	
\end{cmtfront}

\ii{24_10_2021.fb.bryhar_sergej.1.pesochnica_jazyk_mova.cmt}
