% vim: keymap=russian-jcukenwin
%%beginhead 
 
%%file 09_01_2022.fb.zharkih_denis.1.smena_gegemonov
%%parent 09_01_2022
 
%%url https://www.facebook.com/permalink.php?story_fbid=3162589343954461&id=100006102787780
 
%%author_id zharkih_denis
%%date 
 
%%tags informacia,infvojna,obschestvo
%%title Смена гегемонов
 
%%endhead 
 
\subsection{Смена гегемонов}
\label{sec:09_01_2022.fb.zharkih_denis.1.smena_gegemonov}
 
\Purl{https://www.facebook.com/permalink.php?story_fbid=3162589343954461&id=100006102787780}
\ifcmt
 author_begin
   author_id zharkih_denis
 author_end
\fi

Смена гегемонов.

Можно смело сказать, что европейской демократии, какой она была в 1945-1991
годах, больше нет. Украина, Россия, Беларусь, Казахстан, любая страна не может
ее взять себе за эталон, даже если все граждане разом этого захотят. Более
того, в самой ЕС, Британии и США эти рецепты больше не работают, и все это
заполняется другими правилами и обстоятельствами. 

Пролетариат давно больше не гегемон, да и уже в послевоенное время (после 1945
года) он перестал быть, собственно, пролетариатом. Рабочий, имеющий свое жилье,
авто и счет в банке (вклад), уже, собственно, и не пролетариат. И это, что в
СССР, что на Западе. Разница, конечно, есть, но не такая, какую ее рисовала
пропаганда во время перестройки и парада суверенитетов. Советский рабочий был
формальным совладельцем предприятия, то есть, возможно, большим собственником,
чем рабочий Запада. Но рабочий Запада мог жить и лучше, чем советский рабочий,
не будучи членом трудового коллектива и прочих признаков социализма. 

Однако в мире рос новый класс, который в СССР точно не смогли осмыслить.
Названий у этого класса много, но я назову его интеллигенцией, хотя, пожалуй,
это будет не совсем в точку.  Это тот самый класс, который, вроде как,
занимался умственной деятельностью, и рос, как на дрожжах. Этот класс был
освобожден от производства материальных продуктов, что привлекало в него
огромные массы народа. В советской реальности это были всевозможные управленцы,
ИТР-цы, ученые и масса других профессий, которые не подлежали строгому
контролю, поскольку непонятно, что нужно было контролировать. Вспомним
\enquote{Размышления в постели} Райкина, или \enquote{Понедельник начинается в субботу}
Стругацких. 

Кстати, написанная в 1965 году повесть "Понедельник начинается в субботу"
предрекла появление нового мира для интеллигенции, который не особенно-то и
связан с реальностью. Похожие процессы происходили на Западе, но там от
реальности отрывала реклама, массовая культура и телевиденье. Естественно, что
столкновение мира советских интеллигентов с западной виртуальной реальностью ни
оставила от от этого мира практически ничего. Советская интеллигенция, в
большинстве своем, оказалась завербованной и поглощенной Западным миром.  

А теперь перейдем к главному тезису. Сегодня победа определяется не количеством
пушек и прочей военной техники, которую делают рабочие на заводах, а качеством
виртуальной реальности, которую производит интеллигенция. Мы перешли в
информационную эпоху, и тут роль играет уже информационная единица. Теперь
гражданин не просто слушает радио, смотрит ТВ, он заражается и заражает других
в сетях. В этой ситуации принципы свободы слова, да и любой свободы,
невозможны. Ведь можно создать сколько угодно ботов, то есть, нереальных
мнений, а можно задавить мнение миллионов граждан, вырубив их из информационной
системы. Бог с ними, оппозиционными каналами в Украине, вы посмотрите, что
сделали с действующим президентом США Трампом. 

Но это не все - виртуальная реальность делает ненужным вторжение в чужие
страны. Граждане сами заберут власть у своего правительства, и отдадут, кому
скажут. В этом они продолжают политику банановых республик. Почему именно их?
Климат давал пищу многим людям без определенных занятий, а вот в Европе,
Северной Америке (не только в Украине!) голод был страшным бедствием, вплоть до
50-х годов прошлого века.  Поэтому революционеров было меньше, но они понимали,
если они возьмут власть, то придется кормить народ. 

Нынешние революционеры, что в Украине, что в Грузии, что Казахстане, что в США,
о народе не думают. Если население этих стран уменьшится в 10 раз ничего не
произойдет. Потому человеческая жизнь уже не имеет такой цены. Не нужно много
солдат, рабочих, а умников, явно перебор. Фактически ни о каком свободном
соревновании идей, мнений речь уже не может идти. Когда доводы кончаются одна
из сторон хватается за оружие. И это считается нормальным. Ну и какая после
этого европейская демократия? Ее просто нет. Кто первым выстрелил, тот и
демократ. Естественно стрельба должна быть оправданна информационно, иначе
никак. По сторонникам Трампа стрелять можно, по сторонникам Байдена - нет. По
русским стрелять можно, по казахам - нет. 

Информационное поле решает, что можно, что нельзя. Естественно, оно у каждого
свое. У каждого свое можно и нельзя. А раз теперь так, то какая, к черту,
демократия. Точнее, это античная демократия, которая на рабов не
распространяется. Так и при нацизме закон не распространялся на низшие расы,
которые обречены на порабощение и уничтожение. Теперь имеем дело с
информационным нацизмом, который намного сложнее.  И это вызов современной
истории.

\ii{09_01_2022.fb.zharkih_denis.1.smena_gegemonov.cmt}
