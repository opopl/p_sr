% vim: keymap=russian-jcukenwin
%%beginhead 
 
%%file slova.rekord
%%parent slova
 
%%url 
 
%%author_id 
%%date 
 
%%tags 
%%title 
 
%%endhead 
\chapter{Рекорд}
\label{sec:slova.rekord}

%%%cit
%%%cit_head
%%%cit_pic
\ifcmt
  tab_begin cols=2
     pic https://avatars.mds.yandex.net/get-zen_doc/3855260/pub_617f0c9834f9a03edcc0877b_617f0da1090cfe5e631b57eb/scale_1200
     pic https://avatars.mds.yandex.net/get-zen_doc/1595998/pub_617f0c9834f9a03edcc0877b_617f0d9893121f356caa5072/scale_1200
  tab_end
\fi
%%%cit_text
И, да, есть еще один важный, но уже позитивный момент. Благодаря победе на
этапе Гран-при Камила Валиева стала одной из главных претенденток на участие в
Олимпийских играх в Пекине, которые пройдут в феврале следующего года. Если
Валиева сохранит форму и продолжит выступать так же уверенно, то у нее есть
очень серьезные шансы стать олимпийской чемпионкой.  К слову, на этом этапе
Гран-при Камила Валиева обновила собственный мировой \emph{рекорд}, набрав
265,08 балла по сумме программ
%%%cit_comment
%%%cit_title
\citTitle{Перенервничала. Во время награждения российских фигуристок произошел курьезный инцидент}, 
53 новости, zen.yandex.ru, 01.11.2021
%%%endcit

%%%cit
%%%cit_head
%%%cit_pic
\ifcmt
  tab_begin cols=3
     pic https://img.strana.news/img/article/3654/kievljanin-s-dtsp-32_main.jpeg
     pic https://strana.news/img/forall/u/0/0/%D0%B8%D0%B7%D0%BE%D0%B1%D1%80%D0%B0%D0%B6%D0%B5%D0%BD%D0%B8%D0%B5_2021-12-04_152719.png
		 pic https://strana.news/img/forall/u/0/0/%D0%B8%D0%B7%D0%BE%D0%B1%D1%80%D0%B0%D0%B6%D0%B5%D0%BD%D0%B8%D0%B5_2021-12-04_152730.png
  tab_end
\fi
%%%cit_text
32-летний киевлянин Михаил Горбаток с диагнозом \enquote{детский церебральный паралич}
(ДЦП) установил \emph{национальный рекорд}. Мужчина собственноручно создал крупнейшую
в Украине композицию из спичек с украшением природными материалами. Таким
образом он стал рекордсменом в категории \enquote{Сила воли}.  Об этом сообщила
представитель Национального реестра рекордов Украины Лана Ветрова.  В
экспозиции Михаил воспроизвел украинский дом со двором, на котором расположен
колодец-журавль и пасека.  Общее количество использованных спичек – почти 13
тыс. Для плетня вокруг двора мужчина использовал лозу, а вот крыша дома, сарая
и ульев покрыта соломой. Дверь и ставня на окнах открываются
%%%cit_comment
%%%cit_title
\citTitle{Киевлянин с ДЦП установил рекорд, создав крупнейшую композицию из спичек. Фото}, 
Полина Пронина, strana.news, 04.12.2021
%%%cit_url
\href{https://strana.news/news/365432-kievljanin-s-dtsp-ustanovil-rekord-sozdav-krupnejshuju-v-ukraine-ekspozitsiju-iz-spichek.html}{link}
%%%endcit
