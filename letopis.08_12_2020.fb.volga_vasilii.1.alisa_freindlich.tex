% vim: keymap=russian-jcukenwin
%%beginhead 
 
%%file 08_12_2020.fb.volga_vasilii.1.alisa_freindlich
%%parent 08_12_2020
 
%%url https://www.facebook.com/Vasiliy.volga/posts/2772695993047815
 
%%author Волга, Василий Александрович
%%author_id volga_vasilii
%%author_url 
 
%%tags 
%%title АЛИСА ФРЕЙНДЛИХ
 
%%endhead 
 
\subsection{АЛИСА ФРЕЙНДЛИХ}
\label{sec:08_12_2020.fb.volga_vasilii.1.alisa_freindlich}
\Purl{https://www.facebook.com/Vasiliy.volga/posts/2772695993047815}
\ifcmt
	author_begin
   author_id volga_vasilii
	author_end
\fi
\index[names.rus]{Фрейндлих, Алиса!Актриса, день рождения (85 лет), 08.12.2020}

Почти полвека тому назад на экраны наших черно-белых тогда ещё телевизоров
вышел фильм «Служебный роман», в котором Алиса Фрейндлих не только играла
главную роль начальницы управления статистики, не только пела главные песни
этого фильма, но показала то чудо преображения, на которое способна женщина,
полюбившая мужчину и вдруг раскрывшаяся ему всей силой своего женского
естества. Причём, чувственность эта явлена без пошлости, глубоко, откровенно,
ярко.

\ifcmt
pic https://scontent-waw1-1.xx.fbcdn.net/v/t1.0-9/130100951_2772695963047818_7751469093300191307_n.jpg?_nc_cat=109&ccb=2&_nc_sid=730e14&_nc_ohc=I3ZluQOMMfUAX8Ma1yP&_nc_ht=scontent-waw1-1.xx&oh=eb829cddd4357462e8d927ae9cfe6e33&oe=5FF4B2A8
caption Алиса Фрейндлих
\fi

Мне тогда даже казалось, что этот фильм как-то крайне «не комсомольский»,
как-то недопустимо откровенный, как-то слишком сильно являлась в этом фильме
женщина, и даже казалось, что это не фильм, что нет там камер, режиссера,
операторов, что это происходит с каждым из нас и как будто каждый смотрит
историю своей собственной первой любви.

Сегодня у Алисы Фрейндлих День Рождения.  Сегодня ей 85. И я сегодня уже
несколько раз за утро прослушал песню, которую она исполнила в этом фильме:

\begingroup
\obeycr\noindent\large\color{blue}
«У природы нет плохой погоды,
Ход времен нельзя остановить.
Осень жизни, как и осень года,
Надо не скорбя благословить».
\restorecr
\endgroup

Алиса Бруновна, примите, пожалуйста наши из Украины поздравления.

Нам запретили смотреть фильмы с Вашим участием, но мы плевать хотели на их запреты.

Мы любим Вас. Мы благодарны Вам. Живите еще сто лет. А образы, созданные Вами в
кино, сильнее любой статистики напоминают нам о том, что ещё не так уж и давно
все мы были настоящими.
