% vim: keymap=russian-jcukenwin
%%beginhead 
 
%%file 26_11_2020.fb.semesjuk_ivan.2.markov_film
%%parent 26_11_2020
 
%%url https://www.facebook.com/ivan.semesyuk/posts/3845349082182297
 
%%author Семесюк,Іван
%%author_id semesjuk_ivan
%%author_url 
 
%%tags 
%%title Марков - как живет русская провинция / вДудь
 
%%endhead 
 
\subsection{Марков - как живет русская провинция / вДудь}
\label{sec:26_11_2020.fb.semesjuk_ivan.2.markov_film}
\Purl{https://www.facebook.com/ivan.semesyuk/posts/3845349082182297}
\ifcmt
	begin_author
   author_id semesjuk_ivan
	end_author
\fi

До речі. Якщо ви хочете зазирнути в очі людям які цілком щиро їздили убивати
нас на Донбас маючи при цьому своїх адових проблем вище голови, відчути які
вони насправді і чим живуть, і чому врешті \enquote{це не ми переможемо, а це вони
програють}, раджу взяти і подивитися \enquote{Марков - как живет русская провинция /
вДудь}. Хоча фільм не лише про таких людей, але й про них теж.

Це журналістика дуже високого рівня, світового, але подивитися все ж варто не
через це, а тому, що воно прямо стосується нас і нашого майбутнього, як не
дивно. 

Тамтешнє суспільство нині перебуває на межі смислового колапсу, і тримається
лише на \enquote{робимо вигляд, що все добре, бо страшно втратити навіть цей пиздець}.
Це дуже добре тут проартикульовано, чітко. Вони бояться самі себе. Але вічно
такі стани тривати не можуть. Трансформація станеться і їм доведеться
повернутися фейсом до свого локального пекла. 

В тому стані в якому там все перебуває зараз, в майбутнє проскочити неможливо.
Це кінець. При цьому для того, щоби це все трансформувати у щось придатне,
треба тупо похитнути основи основ аж до фундаментальних установок. До самих
ідентичностей. Їм доведеться це зробити.

Хоча навіть війни з сусідами уже не створюють для них особливих смислів, на диво. 

Зазирніть їм в очі, подивіться цей фільм. Їм дуже погано і дуже безсмислено,
цим людям, саме тому вони такі які є, здебільшого. Серед них є реально святі,
які віддають себе милосердній допомозі ближньому, але при цьому вони ж цілком
готові мочити нас на Донбасі. 

Це пов'язано, гадаю, з відсутністю культури співіснування з сісідами, тобто
культури взаємодії з ЗОВНІШНІМ та ІНАКШИМ, бо на тисячу кілометрів навколо них
лише вони ж і живуть --- такі самі страждальці в п'яной ізбушкє. 

Я бачу це так --- для них все, що існує за цією межею, все чого вони не розуміють
і до чого торкаються --- це тупо неприйнятна дикість і щось ненормальне. Схоже на
те, що вони не сприймають інакшість інших в принципі і тому просто
дегуманізують їх. Питання \enquote{чо ти как нє русскій?} в чесному перекладі означає
\enquote{а чому ти не людина?}.

Це останнє спостереження, ясне діло, в фільмі не артикулюється, але такий
висновок можна зробити з побаченого і відчутого під час перегляду. Якщо
дивитися саме крізь нашу українську оптику.
