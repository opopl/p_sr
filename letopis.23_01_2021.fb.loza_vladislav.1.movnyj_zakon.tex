% vim: keymap=russian-jcukenwin
%%beginhead 
 
%%file 23_01_2021.fb.loza_vladislav.1.movnyj_zakon
%%parent 23_01_2021
 
%%url https://www.facebook.com/vladyslavloza/posts/1637849356419520
 
%%author Лоза, Владислав
%%author_id loza_vladislav
%%author_url 
 
%%tags jazyk,mova,ukraina,ukrainizacia,zakon,zakonoprojekt
%%title КІЛЬКА СЛІВ ЩОДО ТАК ЗВАНОГО “МОВНОГО ЗАКОНУ”. Читати до кінця або не читати взагалі
 
%%endhead 
 
\subsection{КІЛЬКА СЛІВ ЩОДО ТАК ЗВАНОГО \enquote{МОВНОГО ЗАКОНУ}. Читати до кінця або не читати взагалі}
\label{sec:23_01_2021.fb.loza_vladislav.1.movnyj_zakon}
\Purl{https://www.facebook.com/vladyslavloza/posts/1637849356419520}
\ifcmt
 author_begin
   author_id loza_vladislav
 author_end
\fi

КІЛЬКА СЛІВ ЩОДО ТАК ЗВАНОГО \enquote{МОВНОГО ЗАКОНУ}. Читати до кінця або не читати взагалі.

Шановне панство, про що взагалі мова?

\subsubsection{Закон – антиправовий і з правом несумісний}

1. Закон – антиправовий і з правом несумісний. Він нахабно і безсоромно порушує:
ст.1, ст. 8, ст. 10 ч.3, ст. 11, ст. 53 ч. 5, ст. 21, ст. 22, ст. 24, ст. 32,
ст. 64 Конституції України. 

Щодо статті 10, яка є епіцентром дискусії, нагадаю
про існування поганенького позитивістського офіційного тлумачення її
Конституційним судом (справа № 1-6/99). 

Так ось, навіть це поганеньке
тлумачення 1999 року (яке досі в силі і на яке досі повинен зважати
законодавець), справедливо розкритиковане суддею Мироненком в окремій думці,
визначає сферою обов'язкового вживання української мови державні органи та
органи місцевого самоврядування (щоправда, сором'язливо згадуючи про можливість
подальшого розширення цієї сфери окремими законами на підставі ч. 5 тої ж
статті 10 КУ). 

Законодавець, схоже, вирішив скористатись цим \enquote{пробілом} (який,
на думку судді Мироненка, пробілом не є – оскільки неможливо поширити
обов'язкову регламентацію мови на інші сфери, окрім зазначених, не порушивши
при цьому ряд вищеперелічених фундаментальних статей, що лежать в основі
українського конституціоналізму). Також закон несумісний із: ч.1 ст. 2
Декларації ООН про права осіб, що належать до національних або етнічних,
релігійних або мовних меншин; ст. 3 ч.2, ст. 4, ст. 5, ст. 6,ст. 10, ст. 16
Рамкової конвенції про захист національних меншин; ст. 7 ч. 2, 3, 4, ст. 10 ч.
3 Європейської хартії регіональних мов; він суперечить  програмним документам
ЮНЕСКО та безлічі інших декларацій, листів, хартій рекомендаційного характеру,
що так чи інакше зачіпають аспекти захисту культурних спільнот та залучених в
них осіб. Тобто він прямо, нахабно, по-варварськи підриває вельми тендітну і
непросту міжнародну правову архітектуру культурного співробітництва, що
створювалась великими зусиллями переважно у другій половині XX ст., в т.ч. й за
участі України. 

Усі наведені документи Україна ратифікувала. Згідно з ст. 19 ч.
2 ЗУ \enquote{Про міжнародні договори}, за наявності колізії між національною та
міжнародною нормою перевага надається останній. Себто, тим, хто скаже, що увесь
цей міжнародний jus cogens нічого не важить порівняно з місцевою варварською
\enquote{правдою} – відповідаю: ще й як важить, і ця локальна \enquote{правда}, як її не крути,
не витримує жодної критики. З юридичної точки зору – повна, цілковита і ганебна
катастрофа. Наведені статті навмисно детально не розписую: можливо, це
простимулює у випадкового читача зацікавлення й пошуковий імпульс, який за
наявності мінімуму аналітичного розуму мав би викликати спалах громадянської
солідарності з тими, кого цей закон безпосередньо зачіпає.

\subsubsection{Пряма антидержавна диверсія}

2. Закон – недолугий і схожий на пряму антидержавну диверсію: він не дає
ефективного механізму \enquote{покарання} його \enquote{порушників} ( ще не вистачало). За ним
громадянина майже неможливо притягнути до  юридичної відповідальності (на цю
тему вже було багато хорошої аналітики): єдине, що може зробити стурбований
активіст – подати скаргу і чекати її розгляду. У відповідь на скаргу приходить
(максимум!) попередження, а механізм притягнення до відповідальності у разі
невиконання настільки недолугий і громіздкий, що важко уявити, ніби колись
хтось його використає до кінця. Застосування поліції при реагуванні на такі
скарги не передбачене. Тобто закон – \enquote{пустий}, це чистий медіа-вірус: він
написаний в інтересах радикалів, точніше їхніх хазяїв, яким потрібна подальша
дестабілізація ситуації в країні. Єдине, що він зробить: розв'яже руки
хуліганам, які законів не читають і про свої вкрай вузькі \enquote{повноваження} чути
не захочуть, створить сотні і тисячі ситуацій дрібного психологічного та
фізичного насильства, спричинить ескалацію ненависті та інтенсифікацію
ворожнечі, проведе міріади невидимих розділень. \enquote{Мовне питання} він не вирішить
аж ніяк.

\subsubsection{Закон – антисоціальний}

3. Менш помітний аспект: закон – антисоціальний. Він поглиблює незахищеність
працівників сфери послуг, перетворюючи їх на пряму мішень. З одного боку,
роботодавців – які, по-перше, отримали додатковий ідеологічний привід для суто
економічного тиску на \enquote{неугодних} працівників, а по-друге, тепер спокушені в
умовах неоліберальної ковідної економіки компенсувати дисциплінарними засобами
на кшталт \enquote{вирахування із зарплати} можливі збитки від \enquote{штрафів}, якщо такі все
ж будуть відбуватись. По-друге, з боку вуличних маргінальних ідеологічних банд,
які в умовах безкарності, правового нігілізму та поліцейської апатії отримають
зайву можливість виміщати свої примітивні колоніальні комплекси на пролетаріаті
сфери обслуговування, постаючи при цьому в масовій свідомості \enquote{борцями за
законність} та \enquote{порядними громадянами}. Себто, по суті, \enquote{вульгарна} сторона
закону – все те саме посилення економічної експлуатації працівників чи не
єдиної сфери, яка в умовах локдауну лишилась на плаву і фактично \enquote{несе} на собі
знищену економіку. І це єдина його \enquote{раціональна} сторона, яку можна пояснити
зрозумілими прагматичними стратегіями – все інше належить до сфери
ідеологічного психозу. Що цікаво: ось таким варварським способом, через
вульгарний ідеологічний симптом, Україна згадує, що вона – \enquote{соціальна держава}.
Виявляється, ідеологія \enquote{атлантів} у нас до пори до часу, а втручатись в
економіку все ж припустимо – тільки не для забезпечення соціальних гарантій чи
встановлення справедливості, бо тоді вас назвуть \enquote{комуністом} і знищать.
Капіталізм перестає бути  недоторканним \enquote{золотим тільцем} тоді, коли йому треба
винайти чергову стратегію власного збереження. Будь-який фашизм проголошує
припустимість грубого вторгнення в капіталістичну економіку – аби ця сама
капіталістична економіка існувала вічно.

\subsubsection{Закон - антиукраїнський}

3. Нарешті, закон антиукраїнський. Бо Україні потрібні не закони проти
продавщиць у кіосках, а добре відкоментовані переклади Томи Аквінського,
Пруста, Шекспіра. Високе кіно, незаангажовані поетичні фестивалі, авангардна
академічна музика. Усе – за державної підтримки, і це – єдиний засіб захисту
української мови.  Все це традиційно створювалось і буде створюватись
ініціативами \enquote{знизу} радше всупереч, аніж завдяки. Натомість, усе, до чого
призвели останні дискримінаційні закони – це поява огидної \enquote{своєї} попси і
низькоякісного пропагандистського кіно. Якщо позитивна динаміка й є, то не
варто приписувати її законодавцю – це ініціативи конкретних людей, зазвичай
важкі й неоплачувані. Правову інфраструктуру, яка задовольняла би вимоги
унікального мовно-культурного ландшафту України, у нас ніхто конструювати не
збирається. Створити можливості для вільного еволюційного проростання
української мови у зв'язках, перетинах, зустрічах з іншими мовами та традиціями
– безмежно важко, а от зарегулювати \enquote{зверху} надзвичайно складну сферу, яка до
регулювання майже не надається, сукупністю примітивних норм імперативного
характеру й зробити вигляд, ніби проблему вирішено – надзвичайно просто. Обидві
риторики – націоналістична, яка посилається на досвід держав із монолінгвальною
моделлю (\enquote{В Німеччині вам касир не відповість російською}) – та
нігілістично-обивательська (\enquote{Мова не має значення, на першому місці – економіка
і т.д.}) – жалюгідні й не витримують критики. Остання зі зрозумілих причин:
навіть у \enquote{вульгарному} марксизмі, який все редукував до економіки, національне
й мовне питання займали значне місце. Бо економічний розвиток і економічний
розподіл все ж відбуваються у групах людей, які є носіями певних символічних
систем та культурних архетипів, і обидва фактори вельми сильно пов'язані. Перша
– тому, що не існує ніякого ідеального Іншого, який задав один раз і назавжди
приклад регулювання подібних речей. Українська ситуація унікальна: ми НЕ
моноетнічні й водночас НЕ колоніальні; частини, з яких зліплена наша держава,
колись належали до різних імперій та цивілізаційних просторів, проте ніколи не
мали в них очевидно пригнобленого статусу: підросійська частина України
належала до імпероутворюючого ядра, подобається це комусь чи ні; західні
регіони максимум претендували на статус культурної периферії Європи, але не
\enquote{колонії} в тому сенсі, в якому колоніями Британії були африканські чи
індійські території. Тож прямо запозичувати чийсь досвід нам не можна,
щонайбільше – зважати і уважно вивчати. Проте це – інша розмова, довга і важка.

\subsubsection{Дискримінація}

4. І наостанок: закон спрямований на дискримінацію однієї з найбільших
культурно-історичних спільнот, що проживає на території України. Саме так: це
не \enquote{меншина}, а культурно-історична спільнота, чия спадщина є невід'ємною
українською спадщиною. В основі закону – ненависть, штучно сконструйований
колоніальний ресентимент. Ідентичність \enquote{жертви}, якій надано дозвіл на помсту.
Ця помста зачіпає широкий спектр груп. Це і \enquote{російськомовні українці}, часто -
обивателі із комплексом роздвоєної ідентичності, які, інтериоризуючи
дискримінаційну установку, несуть свою російськомовність як \enquote{тягар} і можуть
під впливом своєрідної садомазохістської симптоматики навіть вітати найбільш
драконівські заходи, спрямовані проти мови їхнього шоденного спілкування. І
етнічні росіяни, які потрапляють під визначення \enquote{національної меншини} та під
відповідний правовий захист. І росіяни за \enquote{цивілізаційним вибором}, себто
громадяни України, що ототожнюють себе не з наявною національною державою, а з
більш широким цілим (колись таких називали \enquote{малороси}, проте даний означник за
багато років націоналістичного монопольного мовлення був сильно
стигматизований, хоча до того іноді вживався з гордістю, зважаючи на його
шановану етимологію). На рахунок всіх цих людей скажу наступне. Я – український
поет, проте російська для мене – перша рідна. В українську мову я прийшов, щоб
служити їй, бути у ній Dasein. Російською мовою написана левова частка моїх
віршів, і це – не просто так: моє посольство у цій мові значне і приростає.

\subsubsection{Російська - імперська мова українців}

Російська мова – не (лише) національна мова росіян і не монополія емпіричної
Росії як держави із зрозумілими геополітичними амбіціями. Це імперська мова
самих українців, яка віками визрівала в наших монастирях, колегіях, братствах,
університетах. Це наша регіональна латина, яку ми як спадкоємці імперського
простору розділяємо з рядом інших спільнот. Мова Куліша, Нечуя-Левицького,
Купріна, Буніна і Булгакова. 

Імперія, в свою чергу, не гіпертрофована держава
однієї нації, що зводить інші культури до статусу локальних, а простір
спілкування культур, скріплений воєдино спільними етико-правовими або
релігійними універсаліями. 

Спадщину цієї Імперії неможливо зрозуміти без
України, і навпаки – Україна немислима без цієї спадщини. Щоправда, Імперія
була завжди зсередини розколота – на власне Імперію як універсальний простір
зустрічі, і на \enquote{Росію} з її специфічними етнонаціональними зазіханнями, що мали
локальний характер і постійно загрожували підмінити собою ідею Імперії. 

Власне, це протиріччя – між універсалізмом Імперії та партикуляризмом
російської національної ідеї –  поступово перетворювало величний проект Петра і
Олександра на циклопічну за розмірами національну державу росіян, з колоніями
та периферіями. Воно було двигуном для розвитку локальних сепаратизмів, що з
часом перетворювались у буржуазні національно-визвольні рухи, які мали на меті
проголошення модерних націй. Потрібно повернути російській мові її
\enquote{імперський}, тобто – універсальний статус, але для цього необхідно
зрозуміти, чим є для нас – Імперія: місцем зустрічі чи колоніальним гегемоном,
що роздмухує травму. 

\subsubsection{Порозуміння з російською цивілізаційною спадщиною}

Потрібно порозумітися з російською цивілізаційною спадщиною, усвідомивши її
внутрішню розколотість: на універсальну – до якої ми маємо прямий історичний та
дієвий стосунок – та локальну, з якою ми, не ототожнюючись, повинні перебувати
у напруженому діалозі, як локальність із локальністю. Необхідне включення
\enquote{імперського етосу} у горизонт української ідентичності. Саме так: не
українську ідентичність включити в більш широке \enquote{цивілізаційне} ціле, а
навпаки – імперську спадщину включити в українську ідентичність. Для цього –
припинити ототожнення російської мови та культури з \enquote{геополітичною
Росією}, яка у своїх партикуляристських зазіханнях часто чинить спротив
звільненню універсального потенціалу російської мови, перетворює останню на
монопольну \enquote{власність} замкненої національно-ідентитарної групи.

Що ж до самих цих геополітичних зазіхань – так, вони існують, аналогічно з
геополітичними зазіханнями західних країн. Тій частині російськомовної
спільноти України, яка виправдовує закон і дискримінацію або вважає ці речі
несуттєвими, потрібно зрозуміти: ви опиняєтесь між двох вогнів. З одного боку,
геополітична Росія завжди готова говорити \enquote{від вашого імені}, репрезентуючи вас
так, як не хотілося би вам самим. З іншого боку, цей факт репрезентації робить
вас мішенню для українських націоналістів. Тож здобудьте свій голос,
пам'ятаючи: у нас, громадян, що скуті кордонами 1991 року – одна історична
доля, і Україна або постане багатомовною, у єдності своїх імперських спадщин,
всупереч обом геополітичним хижакам – або згине навіки. Натомість тій частині
російськомовної спільноти України, яка досі пов'язує певні надії з
геополітичною Росією, з порятунком \enquote{ззовні}, з відновленням єдиного політичного
простору, потрібно усвідомити: цим ви робите ваші переслідування якщо не
законними, то легітимними, як би жорстоко це не звучало. Ви перекладаєте
відповідальність за свою мову, культуру, цивілізацію – на Іншого, який прийде і
врятує. Нехай цей закон буде для вас можливістю усвідомити свою \enquote{бездомність} –
не як покинутість дитини, а як повноліття, як незалежність від геополітичного
Батька-рятівника, якого не існує. 

Ви не повинні виступати проти України як
політичної ідеї, які темні сили би не говорили від її імені. Ця ідея – не про
зіткнення убогих культурних периферій. Вона – про периферію як місце зустрічі:
мов, культур, етосів, онтологій, моделей існування. Ви у себе вдома, в усіх
можливих сенсах: але це і ваш політичний дім. Зараз йому загрожують варвари. Як
зовнішні – геополітичні – так і внутрішні, що під личиною національного почуття
ховають цілком зоологічні мотиви. Станьте його порятунком. Ототожніть себе з
історичною долею цієї конкретної спільноти, не шукаючи ширших ідентифікацій:
нікому, окрім України, ви не потрібні, ніхто і нізвідки вас не врятує. Ваша
задача як громадян – зробити внесок у розбудову національної спільноти, в якій
ваша спадщина зайняла би місце серед інших, стала би місцем зустрічі
різноманітних траекторій – культурних, біографічних, історичних. 

\subsubsection{Точка перезбирання української ідентичності}

Ваш розрив із \enquote{природним середовищем} втраченої Імперії нехай не
підштовхує вас шукати уламки цієї імперії серед навколишніх привидів та
симулякрів. Зате цей розрив міг би стати \enquote{точкою перезбирання} самої
української ідентичності – тим місцем, де ви, зрікаючись імперської ностальгії,
приймаєте відповідальність за свою країну тут-і-зараз і вступаєте у боротьбу за
її вдосконалення: за те, щоб ваша мова, культура і гідність були визнані як
невід'ємна частина загального блага.

\subsubsection{Комментарии}

\iusr{Виталий Помазан}

Исчерпывающе. Давно не хватало такого текста - как будто прочел собственные
соображения по этому вопросу, только умнее сказано)

\iusr{Влад Лоза}

Виталий Помазан дякую, радий!

\iusr{Евгения Бильченко}
\url{https://www.facebook.com/yevzhik}

А мене в понеділок звільняють за виступ проти закону. Пирогова, 9. 12.00. На
допиті я не відступлю. Ось і вся школа з марксизму

\iusr{Макар Авдеев}

Дякую, було цікаво побачити таку точку зору. А то многие бросаются либо в одну
крайность, либо в другую: либо \enquote{розмовляти тільки українською} и \enquote{російська –
мова ворога}, либо дружба навек с Россией и целоваться в десна с Путиным. Мне,
как россиянину, с одной стороны понятны антироссийские настроения в Украине на
фоне внешней и внутренней политики Кремля за последние двадцать лет. С другой
стороны, конечно, русский язык – это не только политика, это и огромный пласт
литературы и культуры (не только агрессивно-империалистической направленности).
Поэтому подобные законы, регулирующие "языковой вопрос", нужно принимать с
огромной осторожностью.

\begin{itemize}

\iusr{Viktor Ostafeychuk}
\url{https://www.facebook.com/viktor.ostafeychuk}

а при чому тут російська культура до сфери послуг в Україні? А ви у Росії
українську літературу та культуру вивчаєте? Про історію навіть не питаю, бо
Путін показав наочно всім як пишеться російська історія

\iusr{Макар Авдеев}
\url{https://www.facebook.com/profile.php?id=100022846511234}

\textbf{Viktor Ostafeychuk} отвечаю по пунктам.

1. Не российская культура, а украинская культура. Культура - это не только
искусство, это в том числе сфера повседневного общения. Русский язык является
частью украинской культуры, так как для многих людей, проживающих в Украине, он
является родным. В посте Влада, кстати, как раз на этом акцентируется внимание.

2. В России я не изучал украинскую литературу и культуру. На данный момент я
проживаю в Украине, учу украинский язык, начинаю погружаться в украинскую
литературу и культуру. Думаю, что если бы в России существовала бы значительная
территория, где жили бы люди, которые говорили преимущественно на украинском -
в этом регионе бы изучали украинскую историю и культуру, так же как в России в
регионах с неродным русским языком сейчас определённое место в школьной
программе уделено специализированной \enquote{местной} литературе и культуре.

3. Историю Украины я изучал на первом курсе (я учусь в Николаеве), у меня было,
кажется, 82 балла по ней в конце семестра. Узнал много интересных моментов для
себя. Как связаны Путин и история? Путин - это не история. История будет
существовать сама по себе независимо от попыток отдельных личностей её
переписать.

\emph{Влад Лоза}
\url{https://www.facebook.com/vladyslavloza}

\textbf{Макар Авдеев} Макаре, дякую за думку. 

Антиросійські настрої в Україні не мають прямого зв'язку із зовнішньою
політикою Кремля. Пряма геополітична експансія Росії як проблема існує
(окупації Криму і того удару, який Росія завдала своїм геополітичним нігілізмом
міжнародному правопорядку, не скасуєш). Проте немає гіршого способу реагування
на загрозу, аніж запуск аутоімунної реакції ненависті проти величезної частини
власних громадян шляхом розкрутки маховика дискримінації. Ці настрої не є
обов'язковими для захисту суверенітету - їх розкручують ззовні сили, яким
український суверенітет якраз не потрібен. Тобто антиросійський ресентимент в
Україні - парадоксально абсолютно антиукраїнське явище в усіх сенсах. Він
нічого спільного не має із громадянським почуттям, із захистом інтересів своєї
національної спільноти, навіть шкодить їм. Не меншою проблемою є також
геополітичні амбіції західних країн євроатлантичного блоку, яким на даний
момент підпорядковується неоліберальний маріонетковий київський уряд. Єдине, що
мене в цій ситуації непокоїть - як повернути Україні суверенність і
суб'єктність хоча б в тому обсязі, який був до 2014-15 рр.  Ви не зобов'язані
погоджуватись. Я вдячний Вам за читання і дискусію.

\emph{Влад Лоза}

\textbf{Viktor Ostafeychuk} 

всі коментарі, де Ви або підтримуєте ганебне цькування моїх вчителів, або
переходите на особистості щодо мене, або вмикаєте мову ненависті, я видалив. Ще
один подібний плебейський жест чи зоологічний вигук - опинитесь в бані. На моїй
сторінці дозволена винятково мова раціональної аргументації. Надалі, якщо
хочете тут відрізнитись, використовуйте саме її. 

Тепер по суті. На всі Ваші питання у пості дана розлога відповідь. По пунктах.
Для Вас примітивно розжую.

До чого тут російська мова: бо закон всупереч Конституції поширює вжиток
державної мови на непублічні сфери в країні, де російськомовні є однією з
найбільших спільнот. Сфера обслуговування, за винятком державного
обслуговування, яке і так україномовне, юридично є сукупністю приватних актів,
а не публічною сферою - читайте офіційні тлумачення статей Конституції і
Цивільний кодекс. Не можна просто так розширяти сферу регламентації, не
порушуючи ряд фундаментальних статей, що лежать в основі всієї конституційної
споруди. 

Щодо \enquote{а чого в Росії не вчать, а у нас вчать} - розучіться, нарешті,
кивати на Великого Татка і його недоліки. Те, що у нас до пори до часу
забезпечувались свободи і можливості, які не забезпечувались аналогічним чином
в інших країнах, тільки робило нам честь. Чому десь не роблять так, як потрібно
- питання до тих, хто не робить. Ми ж повинні подати приклад, а не запозичувати
найгірший досвід, керуючись почуттям помсти. Пора бути повнолітніми і припиняти
діяти за логікою \enquote{як десь, так і у нас}. 

Врешті-решт, примітивний колоніальний комплекс все зводить до недолугої
імітації дій колонізатора - як приймати, слухати, розуміти і заліковувати рани,
він не знає, бо тупий і ображений. Що російська мова для Вас - це винятково
спадщина Росії як держави, і що Ви з легкістю готові під дією безкультур'я й
ненависті віддати цей скарб хижакам - тут вже позначається Ваша освіченість і
рівень культури. 

Те, що вона тисячі років формувалась на українських теренах, Вас, схоже, не
цікавить. Про це теж в тексті є. Що закон може бути дискримінаційним і водночас
погано прописаним, нікчемним з точки зору правотворчості, не містити прозорого
механізму правозастосування і, відповідно, бути націленим лише на те, щоб
зайвий раз розв'язати руки бандитам і провокаторам і створити тисячу-другу
ситуацій ненависті і насильства - це людина з розвинутою на рівні п'ятикласника
логікою може зрозуміти. 

Сподіваюсь, на все відповів. Постарайтесь тепер проявити апофеоз інтелекту. А
ще пам'ятайте, що я заборонив вступати у дискусію людям, які не прочитали
статтю до кінця, в заголовку. У самому тексті є відповіді на всі Ваші питання,
лише увімкніть ті рештки уважності, які у Вас лишились після того, як Ви
відмовились від мислення на користь кліпового миттєвого сприйняття.


\end{itemize}
