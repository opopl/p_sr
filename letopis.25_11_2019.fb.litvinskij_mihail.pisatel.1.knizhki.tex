% vim: keymap=russian-jcukenwin
%%beginhead 
 
%%file 25_11_2019.fb.litvinskij_mihail.pisatel.1.knizhki
%%parent 25_11_2019
 
%%url https://www.facebook.com/permalink.php?story_fbid=2387267634869738&id=100007595737396
 
%%author_id litvinskij_mihail.pisatel
%%date 
 
%%tags bulgakov_mihail,kniga,kultura,literatura,vysotskii_vladimir
%%title Читайте мои книжки. Там много всякого и разного, вплоть до событий из параллельного мира
 
%%endhead 
 
\subsection{Читайте мои книжки. Там много всякого и разного, вплоть до событий из параллельного мира}
\label{sec:25_11_2019.fb.litvinskij_mihail.pisatel.1.knizhki}
 
\Purl{https://www.facebook.com/permalink.php?story_fbid=2387267634869738&id=100007595737396}
\ifcmt
 author_begin
   author_id litvinskij_mihail.pisatel
 author_end
\fi

Уважаемые друзья, к тому списку книг, которые представлены здесь мной, мои
редакторы работают ещё с тремя рукописями.

1) «Возвращение мастера»

Это был не легкий труд для меня. Я попытался сравнить судьбы двух гениев:
Михаила Булгакова и Владимира Высоцкого. Вам покажется это очень смелым с моей
стороны, но в качестве оправдания хочу сказать: сам Булгаков, когда писал
письмо Иосифу Сталину и просил отпустить его: «Я не нужен стране и чувствую
себя затравленным волком!». У Владимира Высоцкого образ волка (охота на волков)
пересекается с тем о чём я уже сказал. А дальше, если Вам интересно, книга
будет называться «Возвращение мастера».

\ii{25_11_2019.fb.litvinskij_mihail.pisatel.1.knizhki.pic.1}

2) «Американский аукцион»

Это страницы жизни автора с тех пор, как он сел впервые за руль автомашины.
Много событий сменяют одно другим в России и в Америке, где автор живёт
(1993г). Всё, что с ним происходило в Америке близко по содержанию с аукционом
машин, когда он купил свою блестящую \enquote{Volvo cross country-70} и испортил себе
жизнь на несколько лет. Прочтите. Вам будет интересно.

\ii{25_11_2019.fb.litvinskij_mihail.pisatel.1.knizhki.pic.2}

3) «Запоздалый круиз»

Сердцевина этой книги взята из классического Российского фильма «Зимний вечер в
Гаграх». Автор подобно герою фильма в конце своей жизни оказывается никому не
нужным: ни своей жене, ни своей дочери. Но он помнит совет, который давал ему
один из героев фильма «Кураж! Громче, кураж!» , но сил уже почти не осталось. И
вот тот случай, сводит его с молодой женщиной, с которой они решили поехать на
корабле в круиз. Круиз оказался запоздалым, но это всё же был кураж! Я думаю
многим из вас эта книга, если понравится окажется руководством к действию. 

Читайте мои книжки. Там много всякого и разного, вплоть до событий из
параллельного мира. С уважением, Михаил Литвинский.

mlitvinskiy@yahoo.com

\ii{25_11_2019.fb.litvinskij_mihail.pisatel.1.knizhki.pic.3}
