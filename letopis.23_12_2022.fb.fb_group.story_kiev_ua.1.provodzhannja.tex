% vim: keymap=russian-jcukenwin
%%beginhead 
 
%%file 23_12_2022.fb.fb_group.story_kiev_ua.1.provodzhannja
%%parent 23_12_2022
 
%%url https://www.facebook.com/groups/story.kiev.ua/posts/2098831606980296
 
%%author_id fb_group.story_kiev_ua,olejnikov_maksim
%%date 
 
%%tags 
%%title Проводжання Василя Стуса, Юрія Литвина і Олексу Тихого
 
%%endhead 
 
\subsection{Проводжання Василя Стуса, Юрія Литвина і Олексу Тихого}
\label{sec:23_12_2022.fb.fb_group.story_kiev_ua.1.provodzhannja}
 
\Purl{https://www.facebook.com/groups/story.kiev.ua/posts/2098831606980296}
\ifcmt
 author_begin
   author_id fb_group.story_kiev_ua,olejnikov_maksim
 author_end
\fi

19 листопада 1989 року у Києві десятки тисяч людей провели в останню путь
загиблих в радянських таборах українських діячів Василя Стуса, Юрія Литвина і
Олексу Тихого. Їх свого часу засудили на різні терміни за «антирадянську
агітацію і пропаганду».

Радянські закони забороняли забирати тіла в'язнів до завершення терміну їх
ув'язнення – і померлі, вони повинні були «відбувати» свій строк... Тож певний
час Олекса Тихий був похований у Пермі на кладовищі «Севєрноє», а Юрий Литвин і
Василь Стус – поблизу Кучина, в селі Борисові.

Окрім того, ексгумацію не дозволяли проводити і через нібито загрозу погіршення
санітарно-епідеміологічної ситуації.

Активна робота щодо перепоховання загиблих у радянських тюрмах тривала у тому ж
1989р., рідні вели тривалі і виснажливі переговори з владою, а громадськість
збирала кошти. Врешті 18 листопада о 20год 30хв літак із трунами Стуса, Литвина
і Тихого приземлився у «Борисполі». Його зустрічали там сотні людей із свічками
і корогвами в руках. На рідну землю повернулися «особливо небезпечні
рецидивісти»:  Олекса Тихий відбув 17 років у неволі, Юрій Литвин – 22 роки.
Василь Стус – 12 років...

Відспівали померлих у Свято-Покровській церкві на Куренівці. Потім труни з
тілами Стуса, Литвина і Тихого перевезли на Софійську площу, де їх зустрічали
дуже багато людей, які, попри острах бути розігнаними силовиками, прийшли
провести в останню путь загиблих у радянських тюрмах українських дисидентів.
Люди ховалися у дворах і прилеглих вулицях, та як тільки три автобуси з
домовинами приїхали, площа відразу заповнилась. Люди хотіли нести труни на
руках, але КДБ наполягло, щоб везли у автобусах.

З Софійської площі рушили до пам'ятника Т.Шевченку. А вже від нього всупереч
забороні несли труни на руках аж до Байкового кладовища. Фактично це була
масова хода центром Києва і далі до Байкового, з неймовірною на той час
кількістю синьо-жовтих знамен. Крім учасників самої ходи ще тисячі людей стояли
обабіч вулиць, якими йшла жалобна процесія. Багато хто з тих, що стояли на
тротуарах, тримали запалені свічки.

Стояв тоді і я на розі Володимирської і Толстого. Одне з найсильніших вражень
того дня, що залишилось у пам'яті назавжди і про яке я не втомлююсь розповідати
– простий дядько біля «будинку Мороза» із саморобним плакатом, звичайним листом
ватману. А написано на ньому було чи то фломастером, чи плакатним пером:
«Обдурені усіх країн, роздурюйтесь!». Усю глибину цього гасла до кінця зрозуміє
зараз лише той, хто виріс під дещо співзвучні слова, що десятиліттями
«прикрашали» перші сторінки усіх без винятку радянських газет...

Різні джерела називають різну кількість людей, що вийшли того дня віддати
останню шану мученикам, закатованим радянським режимом – до 100 тисяч. У
службовій довідці МВС фігурували 90 тисяч. За даними КДБ УРСР, 19 листопада
було від 90 до 110 тисяч людей – рахували впродовж цілого дня, в різні моменти.

Міліція до того всіляко перешкоджала використанню на масових заходах
національної символіки, але така кількість синьо-жовтих знамен, яка була у той
день над головами десятків тисяч людей, стала важливим поступом у відновленні
права на життя національного українського прапора. Люди не боялися йти з на той
час забороненими прапорами. Їх було стільки, прапорів і людей, що ані КДБ, ані
міліція вже не5 хапали, не ламали, не забирали. Що вже казати, якщо у голові
синьо-жовтої колони, що рухалась по Володимирській повз КДБ, де й починався
хресний шлях Стуса, Литвина і Тихого, майорів великий червоно-чорний прапор...

Василя Стуса, Юрія Литвина і Олексу Тихого поховали на 33-й ділянці Байкового
кладовища, встановивши однакові дубові хрести. У березні 1990-го ці хрести
підпалили невідомі. Винних тоді, звісно, не знайшли... А у 1993 році, вже за
незалежної України, на могилах загиблих дисидентів встановили козацькі хрести
із сірого піщанику.

\ii{23_12_2022.fb.fb_group.story_kiev_ua.1.provodzhannja.orig}
\ii{23_12_2022.fb.fb_group.story_kiev_ua.1.provodzhannja.cmtx}
