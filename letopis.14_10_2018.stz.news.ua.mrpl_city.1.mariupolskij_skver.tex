% vim: keymap=russian-jcukenwin
%%beginhead 
 
%%file 14_10_2018.stz.news.ua.mrpl_city.1.mariupolskij_skver
%%parent 14_10_2018
 
%%url https://mrpl.city/blogs/view/mariupolskij-skver
 
%%author_id burov_sergij.mariupol,news.ua.mrpl_city
%%date 
 
%%tags 
%%title Мариупольский сквер
 
%%endhead 
 
\subsection{Мариупольский сквер}
\label{sec:14_10_2018.stz.news.ua.mrpl_city.1.mariupolskij_skver}
 
\Purl{https://mrpl.city/blogs/view/mariupolskij-skver}
\ifcmt
 author_begin
   author_id burov_sergij.mariupol,news.ua.mrpl_city
 author_end
\fi

Прежде чем приступить к рассказу о  нашем  Сквере, стоит обратиться к книге
\enquote{Мариуполь и его окрестности}, изданной в 1892 году иждивением Давида
Александровича Хараджаева. Там в разделе \enquote{Православные храмы}  написано
следующее: \emph{\enquote{Губернатор Азовской губернии Чертков, желая украсить
новооткрывающийся  город Павловск, заложил  в нем по собственному  плану
церковь во имя Марии Магдалины в честь супруги Павла Петровича Марии
Федоровны}}. Чтобы не отягощать дальнейший текст цитатами, нужно отметить, что
крещение в православии будущей императрицы состоялось в день святой
равноапостольной Марии Магдалины.

\textbf{Читайте также:} 

\href{https://archive.org/details/04_08_2018.sergij_burov.mrpl_city.v_chest_kogo_nazvan_nash_gorod}{%
В честь кого назван наш город?, Сергей Буров, mrpl.city, 04.08.2018}

Уже были возведены стены будущего храма, как ситуация резко изменилась, весьма
немногочисленное население города Павловска было выселено, а на его место
пришли греки-переселенцы из Крыма. Было это летом  1780 года. Грекам предложили
за \enquote{свой кошт} достроить церковь. Не углубляясь в причины, почему церковь
оставалась недостроенной, можно отметить, что жившие в округе украинцы
прапорщик Павел Горленский, Петр Пилипенко, Петр Велегура, Иван Головко,
Емельян Дейнека и другие просили передать ее им. Обосновывалось это тем, что
около двух тысяч украинцев, живущих на косах Азовского моря и вблизи города
\enquote{часто лишены христианских треб}. Просьба была удовлетворена и за очень
короткий срок украинская громада достроила церковь. 4 июня 1791 года в ней
открылось богослужение и священнодействие.

Еще одна цитата из упомянутой  выше книги: \emph{\enquote{Марии-Магдалиновская церковь как по
мысли губернатора Черткова, так и потому, что она была построена в честь и во
имя Высочайшей особы, имением которой назван город, должна была и заняла
первенствующее положение}}. Но к 1891 году церковь настолько обветшала, что 1
сентября того же года была запечатана и служение в ней закрыто. Находилась она
на том месте, где теперь Греческая улица пересекает проспект Мира. А причем
здесь Сквер?

\textbf{Читайте также:} 

\href{https://mrpl.city/news/view/v-mariupole-ustanovyat-bronzovyj-maket-hrama-marii-magdaliny-foto}{%
В Мариуполе установят бронзовый макет храма Марии Магдалины, Роман Катріч, mrpl.city, 08.10.2018}

% 1 - Александровский сквер и Церковь Св. Марии-Магдалины
\ii{14_10_2018.stz.news.ua.mrpl_city.1.mariupolskij_skver.pic.1}

Зеленый островок в центре нашего города в разные времена именовался то
\textbf{Александровской площадью}, то \textbf{площадью Свободы}, то \textbf{Центральным}, то \textbf{Театральным
сквером}. На планах Мариуполя 1888, 1892 годов хорошо видно, что пространство,
где он сейчас находится, специально было оставлено незастроенным среди
кварталов жилых домов. Ради чего? Почитаем строки из  книги \enquote{Мариуполь и его
окрестности}: \emph{\enquote{Ветхость Марии-Магдалининской церкви и недостаточность ее
помещений сознаны давно, и уже на плане г. Мариуполя 1811 г. указано место для
постройки новой на Александровской площади. В 1862 г. заложена новая церковь,
большая, трехпрестольная во имя св. Марии Магдалины, св. Иоанна Крестителя и
Покрова Пресвятой Богородицы. Тогда же выведен фундамент, но дело на этом и
остановилось}}.

% 2 - В. Остапенко Александровский сквер
\ii{14_10_2018.stz.news.ua.mrpl_city.1.mariupolskij_skver.pic.2}

Церковь строили долго и с перерывами почти тридцать пять лет. Освятили ее
только в 1897 году. Зато сооружение получилось величественное. Купола
Марии-Магдалин\hyp{}инской церкви в ясную погоду были хорошо видны с моря, поэтому
рыбаки использовали храм, как маяк.  Церковь на самой высокой точке города,
колокольня Харлампиевского собора и нехитрое приспособление позволяли рыбакам
определять расстояние до берега. Так что храм окармливал людей не только
духовной пищей, но и оказался полезным и в делах мирских. Талантливый
мариупольский художник Валерий Остапенко написал картину – историческую
реконструкцию и Марии-Магдалининской церкви, и людей одетых по моде 1913 года,
и буйной растительности. И хотя это полотно яркое, пронизанное светом, веет от
него почему-то  печалью, печалью о загубленных душах, о низвергнутых храмах.

В 1896 году под руководством Георгия Григорьевича Псалти, который обучался
садоводству во Франции, был разбит обширный сквер с тщательно подобранными
породами  деревьев и кустарников. Заведующая отделом природы краеведческого
музея Ольга Александровна Шакула в свое время рассказывала, что, по ее данным,
систематическая посадка деревьев и кустарников на Александровской площади
началась в 1897 году, когда ­городская дума приняла постановление об
обязательном озеленении улиц и площадей Мариуполя. Именно тогда на территории
будущего сквера появились первые саженцы акации, клена, граба, ясеня, сирени.
Конечно, большинство из них не дожило до наших дней, но могучий вяз в южном
секторе сквера, по мнению О. А. Шакулы, принадлежал  к той, еще первой посадке.
К слову скажем, в сквере произрастали несколько видов акаций и кленов, крымские
сосны и голубые ели, казацкий можжевельник и гледичия колючая, спирея и
вечнозеленый самшит, березы и липы, шелковица, сирень китайская и обыкновенная,
дуб и множество других видов растений.

\textbf{Читайте также:} \href{https://archive.org/details/25_03_2018.sergij_burov.mrpl_city.belaja_akacia}{%
Белая акация, Сергей Буров, mrpl.city, 25.08.2018}

Однако вернемся к истории. В начале XX века, вскоре после сооружения
гражданским инженером Виктором Александровичем Нильсеном водопровода в
Мариуполе,  в центре сквера был устроен фонтан. Он представлял собой чашу со
скульптурной группой из двух мальчиков, прикрывавшей трубу, из которой вверх
била струя воды. Под чашей располагались бронзовые фигуры четырех  девушек,
олицетворявших времена года. По воспоминаниям известного художника, почетного
гражданина города Леля Николаевича  Кузьминкова бронзовые скульптуры и сама
чаша фонтана исчезли на второй день оккупации города гитле­ровцами, начавшейся
8 октября 1941 года. Судя по всему, захватчики поспешили вывезти городскую
достопримечательность в Германию...

% 3 - Фонтан в сквере. Фото из фондов МКМ
\ii{14_10_2018.stz.news.ua.mrpl_city.1.mariupolskij_skver.pic.3}

В сквере над братской могилой был установлен безымянный памятник, хотя имена
погребенных здесь бойцов-апатовцев, погибших в 1919 году в боях под Решетилово,
продармейцев, тела которых предали земле в 1921 году, известны. По воле
обстоятельств, с этой могилы началось постепенное превращение места отдыха
горожан в кладбище. В 1996 году ныне покойный Петр Дионисович Топалов,
известный в городе врач-хирург, опубликовал в газете \enquote{Азовсталец} заметку, в
которой рассказано, как немцы, только что захватившие Мариуполь, с почестями
хоронили в сквере красноармейца, погибшего в бою при защите города. То была,
конечно, показная демонстрация уважения к героизму врага.

Вскоре среди стволов кленов и акаций, кустов роз и сирени начали рыть одну за
другой могилы. Теперь оккупанты хоронили своих солдат и офицеров, умерших от
ран и болезней в военных госпиталях, устроенных в мариупольских больницах и
школах. Через короткое время в северном секторе сквера образовалось кладбище со
стройными, по шнурку расположенными рядами сначала деревянных, а позже вместо
них установленных чугунных крестов с литыми табличками с именами погребенных.
После изгнания оккупантов из Мариуполя в сентябре 1943 года кресты были
извлечены и отправлены на переплавку, а могильные холмики сровняли с землей.
Отливали чугунные кресты, - а они представляли собой увеличенную копию немецкой
военной награды \enquote{Железный крест}, - в одном из литейных цехов завода
имени Ильича. Интересно, что в 50-х годах прошлого века в архиве
проектно-конструкторского отдела этого предприятия хранились чертежи для
изготовления этих крестов.

В конце сентября того же сорок третьего года в сквере вновь состоялись
похороны. На этот раз с соблюдением всех воинских почестей земле предавали тело
Героя Советского Союза Владимира Григорьевича Семенишина, погибшего в бою 29
сентября. Майор В. Г. Семенишин был штурманом 298-го истребительного
авиационного полка 219-й бомбардировочной дивизии. В марте 1944 года его прах
был перенесен в Городской сад. Там, в братской могиле над обрывом, он нашел
вечное упокоение вместе с Героем Советского Союза Н. Е. Лавицким.

\textbf{Читайте также:} 

\href{https://mrpl.city/news/view/v-mariupole-zavershili-raskopki-po-e-ksgumatsii-nemetskih-i-rumynskih-soldat}{%
В Мариуполе завершили раскопки по эксгумации немецких и румынских солдат, Ярослав Герасименко, mrpl.city, 26.05.2018}

% single
%После окончания войны на остатках фундамента Марии-Магдалинин\hyp{}ской церкви был
%построен круглый павильон, как бы объединяющий не\hyp{}сколько ларьков. Обойдя по

После окончания войны на остатках фундамента Марии-Магдалининской церкви был
построен круглый павильон, как бы объединяющий несколько ларьков. Обойдя по
кругу ларьки павильона, можно было купить папиросы, пирожное или мороженое,
кому что по вкусу, выпить газированной воды с сиропом или без него, кружку пива
и, если не изменяет память, что-нибудь и погорячее. Особенно многолюдно было у
павильона в дни праздничных демонстраций - 1 мая и 7 ноября. Мариупольцы,
пройдя в колоннах мимо трибуны - ее устанавливали на том месте, где недавно
возвели модернистскую пристройку к зданию банка на главном проспекте города, -
устремлялись к скверу, чтобы на лавочках отдохнуть от долгого стояния и
хождения, утолить жажду и заморить червячка пирожком, булочкой или пирожным у
павильона, о котором рассказано выше...

% 4 - Мариуполь. Вход в сквер. 1964 год
\ii{14_10_2018.stz.news.ua.mrpl_city.1.mariupolskij_skver.pic.4}

В 1948 году жители Мариуполя неожиданно для себя узнали, что умерший недавно в
Москве партийный и государственный деятель Андрей Александрович Жданов - их
земляк и в его честь город отныне будет носить его имя. Не так уж много утекло
воды в Кальмиусе после того, как у восточного входа в сквер водрузили бетонную
фигуру вождя - земляка во весь рост и призывно поднятой рукой. Со временем,
несмотря на ежегодное покрытие поверхности фигуры то \enquote{под бронзу}, то масляной
краской белого или кремо­вого цвета, монумент стал кое-где разрушаться. И в
60-х годах его место занял более скромный гранитный бюст А. А. Жданова. В 1989
году под давлением общественности он был демонтирован.

Почти одновременно с появлением памятника Жданову вход в сквер с
противоположной стороны был украшен аркой. Снимок с ее изображением, сделанный
мариуполь­скими фотохудожниками Федором Науменко и Львом Сосной, был
растиражирован на почтовых открытках. В середине 50-х годов часть сквера за
аркой была огорожена дощатым забором. Снесли круглый павильон и начали рыть
огромный котлован. Когда он был готов, установили несколько башенных кранов.
Принялись строить здание. То был  театр. К весне 1960 года оно было готово.
Последним \enquote{штрихом} стало разрушение арки. А осенью того же года театр, стены
которого были облицованы белоснежным крымским песчаником, был полностью готов.
2 ноября состоялось его торжественное открытие.

Через несколько лет после окончания Отечественной войны был построен фонтан с
чашей из железобетона. Он запечатлен на почтовых открытках. Но со временем он
начал постепенно разрушаться. В 1970 году мариупольским художникам Виктору
Пономареву, Олегу Ковалеву и Евгению Скорлупину поручили создать новое
оформление фонтана. Оформление получилось оригинальным и выразительным. Не
будем рассуждать почему, но районными властями было принято решение переделать
оформление фонтана.

% 5 - Оформление фонтана в сквере. 1970 год. Авторы: В. Пономарев, О. Ковалев, Е. Скорлупин
\ii{14_10_2018.stz.news.ua.mrpl_city.1.mariupolskij_skver.pic.5}

В последние годы старинный сквер подвергся коренной реконструкции. Пешеходные
дорожки и площадки, не занятые растительно­стью, украсились узорчатым покрытием
из плиток, а в центре его устроили новый фонтан, струи которого в вечернее
время подсвечивались меняющими свой цвет фонарями. Сквер с давних пор был
местом, где малыши делали свои первые шаги, где влюбленные назначали свидания,
куда приходили горожане в первые дни после выхода на заслуженный отдых.

И вот совсем недавно сквер снова реконструировали. Рассказывать каким он стал,
нет необходимости. Каждый может пройтись в зеленый островок в центре Мариуполя.

\textbf{Читайте также:} 

\href{https://mrpl.city/news/view/novyj-teatralnyj-skver-stal-lyubimym-mestom-otdyha-mariupoltsev-foto}{%
Новый Театральный сквер стал любимым местом отдыха мариупольцев, Роман Катріч, mrpl.city, 08.10.2018}.
