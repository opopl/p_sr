% vim: keymap=russian-jcukenwin
%%beginhead 
 
%%file slova.kommunizm
%%parent slova
 
%%url 
 
%%author 
%%author_id 
%%author_url 
 
%%tags 
%%title 
 
%%endhead 
\chapter{Коммунизм}
\label{sec:slova.kommunizm}

%%%cit
%%%cit_head
%%%cit_pic
%%%cit_text
Во-первых, латиноамериканцы плохо ассимилируются. Их дети, в отличие от
потомков русских, ментально останутся мексиканцами. И второе, от чего я орнул.
В Латинской Америке популярен \emph{коммунизм}. И нелегалы приносят его с собой
в Штаты, говорят критики. Так вот. Правда, по поводу \emph{коммунизма} говорят
республиканцы. У них тут своя проблема. Компартия в США может до сих пор и
есть, но не слишком популярна. А вот демократическая партия представлена везде.
В итоге латиноамериканцы топят на выборах за демов
%%%cit_comment
%%%cit_title
\citTitle{В США испугались массового наплыва коммунистов}, 
Максим Войтенко, strana.ua, 21.06.2021
%%%endcit


%%%cit
%%%cit_head
%%%cit_pic
%%%cit_text
\emph{Китайские коммунисты} к тому времени окончательно поссорились с советскими. И
американцы увидели в этом шанс открыть против Москвы \enquote{второй фронт} - сделать
Китай своим неофициальным союзником в холодной войне против СССР. Это резко
ухудшило геополитическую ситуацию для Советского Союза, так как пришлось
тратить огромные ресурсы для укрепления обороны Дальнего Востока.  С конца 70-х
годов, уже после смерти Мао, в Китая начались рыночные реформы, страна
открылась для внешних инвестиций. И позитивное отношение к этому американцев,
которые были заинтересованы в то время в укреплении КНР для противовеса
Советскому Союзу, сыграло в начавшемся инвестиционном буме не последнюю роль.
Но с конца 80-х годов отношение начало меняться
%%%cit_comment
%%%cit_title
\citTitle{Коммунистической партии Китая сто лет - Как появилась КПК}, 
Дмитрий Коротков, strana.ua, 28.06.2021
%%%endcit
