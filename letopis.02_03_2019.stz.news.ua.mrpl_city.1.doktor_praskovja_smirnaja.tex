% vim: keymap=russian-jcukenwin
%%beginhead 
 
%%file 02_03_2019.stz.news.ua.mrpl_city.1.doktor_praskovja_smirnaja
%%parent 02_03_2019
 
%%url https://mrpl.city/blogs/view/doktor-praskovya-smirnaya
 
%%author_id burov_sergij.mariupol,news.ua.mrpl_city
%%date 
 
%%tags 
%%title Доктор Прасковья Смирная
 
%%endhead 
 
\subsection{Доктор Прасковья Смирная}
\label{sec:02_03_2019.stz.news.ua.mrpl_city.1.doktor_praskovja_smirnaja}
 
\Purl{https://mrpl.city/blogs/view/doktor-praskovya-smirnaya}
\ifcmt
 author_begin
   author_id burov_sergij.mariupol,news.ua.mrpl_city
 author_end
\fi

\ii{02_03_2019.stz.news.ua.mrpl_city.1.doktor_praskovja_smirnaja.pic.1}

Прасковья Георгиевна Смирная не занимала крупных постов, не имела
государственных наград, если не считать медаль \enquote{За доблестный труд в Великой
Отечественной войне 1941-1945 гг.}. Не была удостоена ученых степеней и
почетных званий, но вот прошло немало лет после того, как она ушла в мир иной,
а ее с уважением и благодарностью вспоминали и ее коллеги, и ее пациенты...

Паша Смирная родилась 27 октября 1899 г. в многодетной крестьянской семье в
селе Ново-Петровское под Бердянском. В юные годы ей пришлось батрачить на
богатых соседей. Свершилась Октябрьская революция - и наступили новые времена.
В 1918 году Паше доверили заведование детским домом. А было ей тогда всего
девятнадцать лет. Несмотря на молодость, ей удалось в голодные годы и
свирепствовавшие эпидемии сохранить жизнь всем своим подопечным. В 1926 году
заведующую детдомом, у которой всего-то образования было четыре класса сельской
церковно-приходской школы, направляют учиться в Харьков на рабочий факультет
при медицинском институте. Следующим этапом ее образования стал педиатрический
факультет того же вуза. В 1935 г. Прасковья Георгиевна Смирная получила диплом
врача-педиатра и одновременно с ним - направление в Мариуполь, в больницу
Правого берега, где ей предстояло избавлять детей от страшного недуга -
туберкулеза.

\textbf{Читайте также:} 

\href{https://mrpl.city/blogs/view/do-vsesvitnogo-dnya-borotbi-z-tuberkulozom-situatsiya-v-ukraini-ta-mariupoli}{%
До Всесвітнього дня боротьби з туберкульозом: ситуація в Україні та Маріуполі, Тетяна Гололобова, mrpl.city, 24.03.2018}

22 июня 1941 года гитлеровцы напали на Советский Союз. 8 октября того же года
они захватили Мариуполь. Случилось так, что Прасковья Георгиевна осталась в
оккупированном Мариуполе и ей довелось спасать людей от гибели не только как
врачу, но и как патриотке. Работая терапевтом, она выдала десяткам заведомо
здоровых молодых людей справки, что они больны туберкулезом, тем самым
предотвратив их угон в Германию. За каждую такую справку доктор Смирная могла
поплатиться жизнью...

В плен к немцам попал советский полевой госпиталь. Его медперсонал и раненые
бойцы были помещены в правобережную больницу, территорию которой обнесли
колючей проволокой, поставили охрану. Прасковья Георгиевна, другие врачи,
медсестры и нянечки делали все возможное, чтобы облегчить участь несчастных
узников. Однажды трое из них собрались бежать из плена. Нужны были \enquote{аусвайсы} -
удостоверения личности гражданских лиц, выданные немецкими властями. Без этих
документов первый же немецкий патруль отправил бы их в полицию, а там едва
зажившие раны сразу же выдали бы их с головой. Пленные обратились за помощью к
Прасковье Георгиевне. Что делать? Она вспомнила, что до войны ей удалось
буквально вырвать из смертельных лап туберкулеза скромного тогда учителя
математики по фамилии Бердичевский. Теперь Бердичевский наводил ужас на
мариупольцев своей жестокостью, оккупанты назначили его начальником
вспомогательной криминальной полиции и службы безопасности, и этот предатель
старался кровавыми делами угодить своим хозяевам. Доктор Смирная пошла к
Бердичевскому, он довольно любезно принял свою спасительницу:

- Что привело вас ко мне?

- Петр Николаевич, мне нужны \enquote{аусвайсы} для трех моих пациентов. Это мирные
люди, они были ранены при бомбежке, документы их сгорели, они хотят вернуться к
себе домой, на родину. И тут Прасковья Георгиевна всем своим существом ощутила,
что начальник полиции догадался, кому в действительности нужны документы, более
того, он понял, что его собеседница уже знает: ложь ее раскрыта. Но, видимо, и
у негодяев в глубинах души сохраняется чувство благодарности: после небольшой
паузы он распорядился выдать требуемые документы...

\textbf{Читайте также:} 

\href{https://mrpl.city/blogs/view/mariupol-ne-zalishaetsya-ostron-problem-ohoroni-zdorovya}{Маріуполь не залишається осторонь проблем охорони здоров'я, Тетяна Гололобова, mrpl.city, 07.04.2018}

Через некоторое время к дому на Правом берегу, где жила Прасковья Георгиевна,
подъехала линейка, с нее спрыгнул полицай.

- Вы доктор Смирная?

- Я...

- Берите инструменты, вас вызывают к господину Бердичевскому.

\enquote{Все, конец, - пронзила мозг мысль, - мне устроили ловушку, беглецов с липовыми
\enquote{аусвайсами} поймали}. Но что делать? Она уселась на линейку, рядом с ней -
полицай. Поехали. Прасковья Георгиевна, чуть успокоившись, заметила, что
возница направил лошадь не к центру города, где была полиция, а в сторону
противоположную - на Левый берег. Потом все выяснилось: у Бердичевского заболел
один из двух его малолетних сыновей и понадобилась ее помощь как врача...

\textbf{Читайте также:} 

\href{https://mrpl.city/blogs/view/tuberkuloz-v-mariupolskij-shkoli-yak-batkam-zahistiti-ditej}{Туберкульоз в маріупольській школі: як батькам захистити дітей, Тетяна Гололобова, mrpl.city, 23.01.2019}

В августе 1944 года П. Г. Смирную перевели в тубдиспансер, где она лечила детей
вплоть до ухода на пенсию в 1971 году. Коллеги отмечали ее мягкость и
деликатность, ее эрудицию, умение точно установить диагноз, ее готовность
оказать помощь врачу в затруднительных ситуациях, дать консультацию. Большое
внимание она уделяла профилактике туберкулеза. К середине 70-х годов прошлого
уже века детский туберкулез в нашем городе был побежден. Он встречался лишь в
единичных случаях среди подростков. Конечно, такие успехи являлись результатом
работы многих врачей, но немалая заслуга в этом принадлежит и Прасковье
Георгиевне.

Для многих доктор Смирная была спасительницей, для коллег - добрым другом и
наставником, а для врача-фтизиатра Ленины Климентьевны Тельбизовой - еще и
мамой.

- Какой Прасковья Георгиевна была дома?

- Человеком с открытым сердцем, гостеприимной хозяйкой. Для наших родственников
побывать у нее в гостях было радостью. Много читала, и не только медицинской
литературы. Свои отпуска посвящала путешествиям по стране. Последний раз
отправилась в турпоход в Карпаты, когда ей перевалило за семьдесят. Была
азартной болельщицей хоккея.

\textbf{Читайте также:} 

\href{https://mrpl.city/news/view/kuts-ne-nameren-podpisyvat-dogovor-na-poluchenie-bebi-boksov-dlya-donetchiny}{Куць не намерен подписывать договор на получение \enquote{беби-боксов} для Донетчины, Яна Іванова, mrpl.city, 01.03.2019}
