% vim: keymap=russian-jcukenwin
%%beginhead 
 
%%file 06_04_2021.fb.kuchinskij_sergej.1.liderstvo
%%parent 06_04_2021
 
%%url https://www.facebook.com/sergey.kuchinsky.712/posts/503134404032294
 
%%author Кучинский, Сергей
%%author_id kuchinskij_sergej
%%author_url 
 
%%tags biznes,liderstvo
%%title Всеобщее лидерство / Total Leadership
 
%%endhead 
 
\subsection{Всеобщее лидерство / Total Leadership}
\label{sec:06_04_2021.fb.kuchinskij_sergej.1.liderstvo}
 
\Purl{https://www.facebook.com/sergey.kuchinsky.712/posts/503134404032294}
\ifcmt
 author_begin
   author_id kuchinskij_sergej
 author_end
\fi

Всеобщее лидерство / Total Leadership. 

Добрый день, коллеги! Cегодня я написал коммент о лидерстве к посту одного
известного учителя. В заглавном тексте говорилось, что сегодня продвинутые
специалисты востребованы больше, чем профессиональные лидеры. Мне кажется, что
мой «школьный» коммент может быть информативным не только в контексте школы.
Потому что в нем, во-первых, «подсвечивается» инструментальная, а не
божественная природа лидерства. Во-вторых, упоминается связь лидерства с
«этажами» Полной Модели Бизнеса(Whole Business Model). Картинку прошу всерьез
не воспринимать, потому что это юмор ))

\ifcmt
  pic https://scontent-cdt1-1.xx.fbcdn.net/v/t1.6435-9/169436988_503133680699033_7448328810528493008_n.jpg?_nc_cat=109&ccb=1-4&_nc_sid=730e14&_nc_ohc=qRWpdktCYoAAX_eRqL7&_nc_ht=scontent-cdt1-1.xx&oh=fb01dc14a9370ab33b7eb5651a676ee0&oe=6137AB0A
  width 0.4
\fi

=============================

«Лидерство» похоже на (школьную)математику. Как и таблица умножения,
«лидерство" это инструмент для решения (более)сложных задач - набор типовых
знаний и навыков, как вести и направлять людей, объединять их в команды для
решения возникающих сложных задач. Можно сказать, что «лидерство» это набор
приемов и навыков «умножения и сложения людей» для решения возникающих задач. 

.
Лидерство, как и математика, нравится не всем. Не все становятся
математиками-профессионалами. И не все любят применять математику в своей
деятельности. Но, при этом, все выпускники школ могут "под задачу" использовать
свои базовые знания и навыки математики - умножение, сложение и т.п. Или не
использовать. Так же и с лидерством. Если у кого-то появляется своя или чужая
сложная задача, которая требует объединить и направить людей на ее решение, то
человек может это сделать сам по мере способностей и базовых навыков. А может
сказать, "это не мое - я специалист, а не лидер". Есть еще вариант, когда
кто-то хочет быть директором по жизни. Ради статуса, власти и заработка.
Причем, где угодно, а не для решения той или иной конкретной задачи. 

.
Базовые приемы и навыки лидерства можно и нужно преподавать/тренировать в
школе, как таблицу умножения. Давить детям ощущение, что «если понадобится, то
и я могу это делать». А дальше каждый сам себе выберет, пользоваться этими
навыками «умножения и сложения людей» или не пользоваться и для чего. 

.
Всеобщая способность к ситуационному лидерству может стать особенно важной,
если оправдаются прогнозы о приходе сетевого предпринимательского общества. С
необходимым динамическим лидерством в узлах предпринимательской сети. 

.

PS. И, конечно, нужны разные типы лидеров для трех основных типов задач(и
команд) – операционных задач, задач развития и инновационно-предпринимательской
деятельности. Об этих трех типах директоров и предпринимателей есть раздел в
«Хорошем/плохом бизнесе», «Очевидном бизнесе» и «Двенадцати моделях».

\ii{06_04_2021.fb.kuchinskij_sergej.1.liderstvo.cmt}
