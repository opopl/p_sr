% vim: keymap=russian-jcukenwin
%%beginhead 
 
%%file 29_05_2022.fb.lonska_tetjana.1.den_kyiv
%%parent 29_05_2022
 
%%url https://www.facebook.com/tatiana.lonskaja/posts/pfbid02kNPzCP5JUSipF6pbn7YJxFdTLsqjYurZfSQVUrpjbjVDvNGTCb7V9XASyxh45px1l
 
%%author_id lonska_tetjana
%%date 
 
%%tags 
%%title З днем народження, любе місто! Ти незламне і вільне! Допоки ти живе, жива й Україна
 
%%endhead 
 
\subsection{З днем народження, любе місто! Ти незламне і вільне! Допоки ти живе, жива й Україна}
\label{sec:29_05_2022.fb.lonska_tetjana.1.den_kyiv}
 
\Purl{https://www.facebook.com/tatiana.lonskaja/posts/pfbid02kNPzCP5JUSipF6pbn7YJxFdTLsqjYurZfSQVUrpjbjVDvNGTCb7V9XASyxh45px1l}
\ifcmt
 author_begin
   author_id lonska_tetjana
 author_end
\fi

Коли я дивлюся на мапу «повітряних тривог», то чітко бачу на ній ділянку у
формі серця. Це Київ. Одна за одною, області зафарбовуються червоним кольором,
і мені здається, що вся Україна – ніби єдиний організм, у якого болить, то тут,
то там...  І маленьке серденько, яке так часто стає червоним, безперестанку
турбується за всю країну. Завмирає, ниє, шалено б'ється...  І безперервно болить...

Нашому найдорожчому, найкращому, найгарнішому місту на світі сьогодні 1540
років. Ми пишаємося його величчю, історією, красою.  Але сьогодні я хочу
привітати не тільки місто, я хочу привітати людей, і головне – подякувати
їм.

Вклоняюся із вдячністю військовим, які мужньо захистили його від ворога і
далі пильно охороняють. Вклоняюся киянам, які, попри небезпеку і облогу,
продовжили працювати задля життя, пекти хліб, освітлювати вулиці й
будинки, прибирати, висаджувати квіти. Вклоняюся лікарям, які продовжили
лікувати людей й допомагати з'являтися на світ немовлятам.  Київ вистояв,
не здався, не помер тільки завдяки людям.

Я дякую мужнім містам і селищам Київщини,  які не відступили, стримали
окупантів, заплативши за це подекуди своїми будинками, а подекуди –
своїми життями. Найдорожча і найстрашніша ціна... Але завдяки ній ворога не
пропустили до столиці, і сьогодні вона жива.  Вічна пам'ять загиблим...
Низький уклін тим, хто вижив... Сьогодні це і ваше свято.

Я дякую тим родинам, чиї чоловіки, діти, онуки тримали і продовжують
тримати оборону Києва. Сьогодні це і ваше свято також. Я дякую тим, хто
вивіз зі столиці дітей – це наше майбутнє, яким неможна ризикувати. Дякую
тим, хто вже повернувся, і тим, хто ще повернеться. А найбільше тим, хто
залишався в рідному місті всі три місяці цієї страшної війни. За світло у
ваших вікнах, яке стало справжнім маяком для тих, хто далеко від дому.

Сьогодні свято з присмаком гіркоти... Немає гучної музики, веселих дітей,
славнозвісних пробігів під каштанами, вишуканого вальсу на площі... Проте із
гучномовців голосно виє сирена...Немає кольорових феєрверків, яскравої
ілюмінації... Проте з'явилися обгоріли чорні будинки з вибитими вікнами і
потрощеними ракетними ударами стінами... Але музика все одно співає в наших
душах безсмертною мелодією, і ми підспівуємо їй: «Як тебе не любити»...

І все одно Київ продовжує жити, щосили намагаючись довести – це неправда, що ми
залишись у лютому! Погляньте навколо! Весна змінюється літом. І біло-рожевий
каштановий цвіт на тлі золотих куполів ніби благословляє всю країну на життя.

З днем народження, любе місто! Ти незламне і вільне! Допоки ти живе, жива
й Україна. І ми зробимо все, аби твоє змучене серденько перестало
червоніти від болю, а, нарешті, сповнилося радісного, мирного спокою...

Тетяна Лонська Tatiana Lonskaja

\ii{29_05_2022.fb.lonska_tetjana.1.den_kyiv.orig}
\ii{29_05_2022.fb.lonska_tetjana.1.den_kyiv.cmtx}
