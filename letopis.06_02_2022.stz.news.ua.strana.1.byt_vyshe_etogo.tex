% vim: keymap=russian-jcukenwin
%%beginhead 
 
%%file 06_02_2022.stz.news.ua.strana.1.byt_vyshe_etogo
%%parent 06_02_2022
 
%%url https://strana.news/news/375315-fihuristka-butyrskaja-otsenila-reaktsiju-ukraintsev-na-vystuplenie-valievoj.html
 
%%author_id pronina_polina
%%date 
 
%%tags figurnoje_katanie,rossia,sport,ukraina,valieva_kamila.rossia.figuristka
%%title "Надо быть выше этого". В России отреагировали на поведение украинцев после выступления Валиевой на Олимпиаде
 
%%endhead 
 
\subsection{\enquote{Надо быть выше этого}. В России отреагировали на поведение украинцев после выступления Валиевой на Олимпиаде}
\label{sec:06_02_2022.stz.news.ua.strana.1.byt_vyshe_etogo}
 
\Purl{https://strana.news/news/375315-fihuristka-butyrskaja-otsenila-reaktsiju-ukraintsev-na-vystuplenie-valievoj.html}
\ifcmt
 author_begin
   author_id pronina_polina
 author_end
\fi

Российская фигуристка, чемпионка мира, трехкратная чемпионка Европы Мария
Бутырская прокомментировала тот факт, что после проката Камилы Валиевой на
Олимпиаде аплодировали стоя все соперники из других стран, за исключением
Украины.

\ii{06_02_2022.stz.news.ua.strana.1.byt_vyshe_etogo.pic.1}

Об этом пишет \enquote{Спорт-экспресс}. 

\enquote{Это не совсем правильно. Я понимаю, что каждый переживает и болеет за свою
страну. Но там ты приобретаешь столько знакомых, друзей. Если человек классно
катается, то народ всегда поддерживал. Неважно, из какой страны был спортсмен.
И американцы русских поддерживали. Нам просто надо быть выше этого и не
обращать внимания. Тем более, на Украину}, — сказала Бутырская изданию.

Напомним, на Олимпиаде в Пекине состоялось выступление женщин в короткой
программе по фигурному катанию (в рамках командных соревнований).
Фантастический результат — 90.18 баллов — показала 15-летняя россиянка Камила
Валиева. По итогам проката ей аплодировали стоя главные конкуренты — США и
Япония, а также остальные сборные: Канада, Чехия, Германия, Китай и другие.
Все, кроме Украины.

Ранее мы писали, что Министерство спорта Украины порекомендовало украинским
спортсменам на Олимпиаде-2022 в Пекине не общаться с прессой на русском языке.
Кроме того - избегать совместных фото с россиянами.

Также мы сообщали о медальных итогах первого дня зимних Игр в Пекине.
