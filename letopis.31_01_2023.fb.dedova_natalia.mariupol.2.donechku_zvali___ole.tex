%%beginhead 
 
%%file 31_01_2023.fb.dedova_natalia.mariupol.2.donechku_zvali___ole
%%parent 31_01_2023
 
%%url https://www.facebook.com/permalink.php?story_fbid=pfbid02pLk6bGiKuKAPAcmzwKgsXTydbYFXcPqz7mdJ98z6fVDdHtcXgHAVsY9F5paRAZ4fl&id=100007662284921
 
%%author_id dedova_natalia.mariupol
%%date 31_01_2023
 
%%tags mariupol.golosy_myrnyh,mariupol,dity
%%title Донечку звали - Олександрина
 
%%endhead 

\subsection{Донечку звали - Олександрина}
\label{sec:31_01_2023.fb.dedova_natalia.mariupol.2.donechku_zvali___ole}

\Purl{https://www.facebook.com/permalink.php?story_fbid=pfbid02pLk6bGiKuKAPAcmzwKgsXTydbYFXcPqz7mdJ98z6fVDdHtcXgHAVsY9F5paRAZ4fl&id=100007662284921}
\ifcmt
 author_begin
   author_id dedova_natalia.mariupol
 author_end
\fi

\obeycr
\#міймаріуполь
\#голосимирних
\#олександрина
\#шобвоноздохло
\restorecr

▪️Коли вони її вбили, їй було всього/лише/тільки - п'ять. І їй назавжди -
стільки. Більше вона - не дорослішатиме. І більше - не візьме батька за руку.
Адже його вони також - вбили. І йому назавжди - 49. 

▪️Донечку звали - Олександрина. 

▪️Батька - Володимир.

▪️Мати - Валерія. Вона випадково залишилася в живих.

▪️Бабуся та дідусь Саші виїжджати - не захотіли. А Валерія не змогла покинути
батьків - одних. Тому разом з донечкою та чоловіком - залишилася в Маріуполі. 

▪️А 1 квітня танк поцілив саме в кімнату маленької Саші. Один постріл. І мінус
одне життя. 

Розповідає \href{https://www.facebook.com/anna.dyomina.90}{Anna Dyomina}, жінка брата Валерії:

"Утром мужу скидывают информацию. В дом родителей попал снаряд. И Саши,
племянницы, больше нет. Она погибла. Может, это ошибка? А вдруг просто
ранили.... Я пыталась отрицать. Но соседи подтвердили: \enquote{Да, это так. Вся в
крови. Прямое попадание танка. В квартиру. В их комнату}. 

С вечера сильно бахали. Говорили Лере, будьте в подвале. Она: \enquote{Боюсь, что будет
приступ (Саша - алергик). Мы успеем убежать}. Но они - не успели. Осколок попал
в сонную артерию. Сашенька - не мучилась. А Леру - ранили. Муж повез ее в
больницу. Но, когда возвращался домой, попал под обстрел градами. Вова умер в
один день с доченькой, только на несколько часов позже. Его тело так и не
нашли. 

А Лера - так и не дождалась мужа. В тот же день больницу обстреляли. Ей быстро
оказали помощь, разорвана нога. И она побежала домой к родителям. Во дворе
кричала: \enquote{Мама, папа}. Один из подъездов стал гореть. Грады. Лера упала.
Осколки обрезали волосы. Стала бежать в сторону частного сектора. Ранение в
живот, много осколочных. Отвели в дом. Оказали помощь"... ... ... 

▪️Бабушка с дедушкой хоронили собственную внучку в собственном дворе. Сами. 

▪️Через три месяца - перезахоронили. За 20 тысяч рублей. 

🖤🖤🖤🖤🖤🖤💔💔🖤🖤🖤🖤🖤🖤

▪️Двое соседей готовили еду. Снаряд попал в костер. Один раненный. Сам шел в
больницу. Второго несли на покрывале. Не знали, как сказать жене и двум детям,
что их папы больше нет. Это было первое понимание, что сегодня человек есть, а
завтра его нет. 

▪️Второе страшное воспоминание. Пошли в больницу заряжать телефоны. Муж встретил
одноклассницу. \enquote{Как дела?}. Она расстёгивает куртку, а у нее по локоть нет
руки. Ампутировали. Отец погиб. Мина, частный сектор, готовили есть.

▪️Спали все вместе. Молилась о двух вещах: не истекать кровью, чтобы медленно не
умирать. И чтобы не остаться единственной живой из близких.

▪️Ты смотришь фильм, в котором ты оказался по ошибке. 

▪️Обстрел Пентагона. Бабушка, с одной дамской сумочкой. "Все, что у меня
осталось, это сумочка, пенсионное удостоверение и вещи, которые на мне". 

▪️Страшно было от осознания того, что никто не приедет: скорые, пожарные. Только
ты можешь помочь, но не хватает ресурсов. Истерика от безысходности. 

▪️Мы понимали, что город ранен. Но то, что он не ранен, а почти убит, поняли,
когда стали выезжать. 

▪️Я хочу, чтобы о нас не забывали... ... ... 

❤️🩹 Повне інтерв'ю з Анною вже незабаром. 

🇺🇦 Друзі, якщо Ви готові розповісти свою історію, пишіть у приват або в
коментарях. Світ має почути кожного! 

Більше інтерв'ю 👇

👇
\url{https://civilvoicesmuseum.org/}

❣️ Аню, я Вас обіймаю! І дякую, що попри біль Ви розповіли цю історію!
