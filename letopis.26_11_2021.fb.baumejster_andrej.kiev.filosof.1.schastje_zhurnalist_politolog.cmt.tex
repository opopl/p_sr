% vim: keymap=russian-jcukenwin
%%beginhead 
 
%%file 26_11_2021.fb.baumejster_andrej.kiev.filosof.1.schastje_zhurnalist_politolog.cmt
%%parent 26_11_2021.fb.baumejster_andrej.kiev.filosof.1.schastje_zhurnalist_politolog
 
%%url 
 
%%author_id 
%%date 
 
%%tags 
%%title 
 
%%endhead 
\subsubsection{Коментарі}

\begin{itemize} % {
\iusr{Татьяна Белоус}
А я даже не смотрела. Но, мысли похожие. Как мы попали в этот мес? Вопрос риторический

\begin{itemize} % {
\iusr{Андрей Баумейстер}
\textbf{Татьяна Белоус} да, это правда. Но вопросы всё равно нужно озвучивать

\iusr{Анатолий Шлепак}
\textbf{Татьяна Белоус} а міжна я теж не буду дивитися то всьо ? мені достатньо вже мюслєвисновків Баумейстера
\end{itemize} % }

\iusr{Svitlana Azarova}
весь мир сейчас в "глубоком" пике.....

\iusr{Мила Садовская}

Те же чувства. Отъявленная разнузданность и хамство. Наши журналисты извратили все, даже понятие демократии

\iusr{Igor Solomadin}

Впервые в независимой Украине не было свободной аккредитации журналистов на
пресс-конференцию президента. Отбирали «элиту». Журналистов с явных
оппозиционных президенту каналов и близко не подпустили.

Но, несмотря на тщательный отбор, «тёплой ванны» не получилось.

Вы почему-то не заметили залихватский тон Зеленского по отношению к Шустеру,
обвинённому в «дестабилизации» ситуации. И это обвинение прозвучало в контексте
какого-то загадочного «переворота», намеченного на 1 декабря. Даже тщательно
отобранные журналисты не смогли проглотить поток бессвязных «откровений»
Зеленского и стали задавать ему неудобные вопросы. Это Вы называете «хамством»?

Недавно журналист ВВС провёл интервью с Лукашенко и также ставил ему в довольно
резкой форме неудобные вопросы. Он тоже «хам» и «раб»?

Зеленский и его окружение в последнее время делали довольно много заявлений,
противоречащих одно другому, это вызвало соответствующую реакцию в обществе.

Неужели лучше было бы, если бы журналисты чинно и соблюдением политеса вели
беседу с президентом, который, мягко говоря, «путается в показаниях»?

\iusr{Сергій Чаплигін}

\ifcmt
  ig https://scontent-frx5-2.xx.fbcdn.net/v/t39.1997-6/p480x480/105941685_953860581742966_1572841152382279834_n.png?_nc_cat=1&ccb=1-5&_nc_sid=0572db&_nc_ohc=dMmZHIhxBS8AX9N3DeF&_nc_ht=scontent-frx5-2.xx&oh=3bf0f2d83392832373c1f281957ecef6&oe=61A82B4B
  @width 0.1
\fi

\iusr{Alexey Burov}
Померанц говорил, что цивилизованность — это прежде всего стиль, и что ангел, на губах которого появилась злая пена — уже пал.

\iusr{Руслан Воронка}
А что значит, понимать природу власти ?

\begin{itemize} % {
\iusr{Максим Колісник}
\textbf{Ruslan Voronka} думаю в цьому випадку це значить "комусь заплатили за інтелектуальний пост на захист Зеленського". Сподіваюсь що хоч заплатили бо тратити свій час і писати про це безкоштовно це абсолютна дурість

\iusr{Gleb Dolianovskiy}
\textbf{Максим Колісник}  @igg{fbicon.man.facepalming} 
\end{itemize} % }

\iusr{Павел Себастьянович}

Слово \enquote{свободный} может быть уместно в несвободной стране?

\iusr{Max Kuzmenko}

Дякую. Сьогодні відбувся сеанс мракобісся та самовикриття від людей, які
покликані піклуватися про інформаційну гігієну. А насправді виявилися не
журналістами, а примітивними пропагандистами та піарниками.

І ці люди регулярно розповідають українцям про політику та економіку.

\iusr{Мария Мартыненко}

Прекрасный текст. По поводу подчинения... да, вопрос трактовок. Мне нравится
Антонио Менегетти. Он сказал, лидер, это тот, кто умеет служить. И тоже
согласна с тем, что фиатный мир совсем претит своей откровенной пошлостью.

\iusr{Людмила Полищук}

Так, как некоторые журналисты сегодня позволяли себе говорить с президентом
Зеленским (Бутусов и Ко), - это за гранью. Президент может быть неправ. Он
может ошибаться. Он может у некоторых вызывать различные, даже недобрые
чувства.

Но свободный человек должен вести себя благородно. - да.. да да!
вседозволеннность, ненаказуемость, віпендреж, неуважение....... хаос... во
всем. и здесь - в том числе. всі розумні і всі все знають .... можна багато
говорити.....

\iusr{Maria Lypych}

Ті хто пише, що президент реагував не як правитель, що серйозні лідери держав
не повинні так психувати від запитань журналістів... по-перше, кожна фахова
людина повинна розуміти що таке професіоналізм, що Зеленський не професійний
політик, і що за два роки неможливо виробити навики досконалої блискучої
поведінки і вмінь політика, по-друге, в такій ситуації!! як в Україні, і з
такою риторикою журналістів, словоблуддям, хаосом, я не знаю ким треба бути,
щоб могти нормально відповідати.

Всім у світі правлять гроші, які створюють страх і тиск, а в Україні з жлобною
і примітивною ментальністю це виливається у ось таке - тиск верхньостоящих,
страх, напруга, і хаос, ігнор, і так по колу.

Тим хто знецінює Зеленського, і дуже поважає Порошенка - цього року!!! до Дня
Незалежності з Трускавця везли автобусами людей в Київ стояти за Порошенка!,
були знайомі моїх знайомих, вони не знали що так буде, думали просто так, а їх
заставляли бути весь час, цілий день, ніде не відійти в сторону, і кричати по
команді ПОРОШЕНКО. Вони в стиду від безвиході стояли. Я була в шоці........
Революція Гідності! Браво! Тільки матюкнутися хочеться...  @igg{fbicon.face.eyes.crossed.out}  @igg{fbicon.dizzy} 

\iusr{Michael Baskin}
Добрий ранок. А що це за програма була? Я тепер подивлюся

\iusr{Mykhailo Havryliuk}
Спасибо!

\iusr{Ната Руст}
как продвигается работа над книгой о природе власти?  @igg{fbicon.face.smiling.sunglasses} 

\iusr{Анатолий Шлепак}

Андрію, а якби якийсь прогресивний мільярдер викупив в Україні клаптик землі
розміром з невеличке містечко і збудував там ідеальне середовище для обраного
населення, де існував би певний інтелектуально-духовний ценз на відповідність,
ви б туди переїхали жити ?

\iusr{Дмитрий Тароян}
Атмосфера такая, что недалеко и до украинского Шабтая Цви? Разочарование потрясающее...

\iusr{Volodymyr Matkovskyi}

Ви оцінюєєте якось дуже вибірково. Почитайте пости Арестовича, як \enquote{благородно}
він висловлювався про цю \enquote{священну корову} (президента), перед тим як він
перевзувся в повітрі і почав брехати про вагнергейт разом з Зеленським. А
Бутусов розмовляв з Зеленським просто на рівних. Він не винен, що президент -
хамло. Надіюсь нікого не зачепив і не образив.

\iusr{Юрий Чернявский}

Озвученная программа на 100 млрд в год по термомодернизации всего
многоквартирного жилья - очень важная и очень амбициозная цель.

\iusr{Владимир Ивлев}
Бутусов вообще в погоне за популярностью переступил закон.

\begin{itemize} % {
\iusr{Yuri Sam}
\textbf{Vladimir Ivlev} какой?

\iusr{Владимир Ивлев}
\textbf{Yuri Sam} про державну таємницю

\iusr{Yuri Sam}
\textbf{Vladimir Ivlev} а він держслужбовець чи військовий?

\iusr{Владимир Ивлев}
\textbf{Yuri Sam} хватит того, что он журналист
\end{itemize} % }

\iusr{Michael Baskin}

Там є цікавий момент коли Зеленський звинувачує Цензор. у зливі інформації,
взагалі я б сказав що режисер той самий, що показував нам "Стадіон так Стадіон"

\iusr{Gleb Nemo}

Я бы ещё добавил, но сейчас как-то в унисон с Зеленским о нормах сообщества
начали проявлять заботу Фейсбук с Ютубом. Наверное догадываются, о чем люди
должны заговорить ). Чувствуется, что они торопятся, потому что у них мало
времени. Из-за этого их деяния становятся видимыми.

\iusr{Мартин Тихолаз}

Андрей, но так точно можно воскликнуть \enquote{какое счастье разбираться в возможных
инновациях в сфере образования} , подозреваю, экономистам тоже \enquote{радость
беспрестанная}... Девальвация уже не забавна, не где-то, не секторальна, а ещё
все меньше тех, кто может увидеть, что это именно девальвация.

\iusr{Gleb Usakovsky}

Самое время заняться живописью! Художник философствует кистью. Проблемы
администрации всегда будут. Посмотрите картины Марка Ротко!

\iusr{Олег Хомяк}

Вне всякой сомнения Бутусов вышел за рамки приличия. Но что вы скажете об этике
коммуникации Президента? Откровенное хамство, обвинения, ложь. Разве это
достойно символа государства? Ведь имеет смысл обсуждать как цельную ситуацию:
этика коммуникации обоих сторон. И можно ли кого -то считать причиной?

\iusr{Helen Lysenko}

Андрей Олегович, даже если по самому худшему сценарию не следовать проф этике,
то не дословно, но Президент Украины взывал : Вы можете хотябы просто человека
во мне увидеть!???

\iusr{Майя Плющева}

Риторика по отношению к действующему президенту, которой наелся Зеленский, была
впервые инициирована ним самим по отношению к Порошенко. Во время предвыборной
гонки и на дебатах.

Если он сам показал, как можно с президентом, то ему странно было рассчитывать,
что с ним будут по-другому.

Воплощенное хамство \enquote{внезапно} узнало, что такое диалектика раба и господина?
 @igg{fbicon.shrug} 

Этого следовало ожидать. Случилось самое логичное и последовательное, что
только могло случиться.

\iusr{Dmytro Bintsarovskyi}

Еще большее хамство Зеленский себе позволял по отношению к тогда еще
действующему президенту Порошенко. Тот знаменитый диалог на \enquote{Плюсах} и был
апофеозом неприкрытого и публичного хамства в украинской политике. Если
следовать Вашей логике, тот выпад Зеленского и был выпадом раба. Теперь раб
вознесен на престол и требует почестей.

\url{https://www.youtube.com/watch?v=bONMAoX6sPo}

\begin{itemize} % {
\iusr{Саша Курский}
\textbf{Dmytro Bintsarovskyi} хамство бубочки не оправдывает хамства Бутусова. Вам пишут про "понимание власти", а вы про "сам дурак".

\iusr{Майя Плющева}
\textbf{Саша Курский} 

речь о том, что сам Зеленский легитимизировал хамство по отношению к
действующему президенту. В предвыборной гонке в 2019-м. Именно он запустил этот
процесс.

Это не оправдывает институт журналистики, но делает понятным и логичным
процесс, происходящий с институтом президента.

А у вас получается, что когда первичное, я бы даже сказала, первопричинное,
хамло вдруг услышало хамство в свой адрес, то внезапно обиделось. Ну в смысле?

А с кого журналистам пример брать, как не с действующего президента?! Раз даже
он себе такое позволяет, значит, это теперь - норма.

\iusr{Евгений Мартын}
\textbf{Саша Курский} 100\%  @igg{fbicon.thumb.up.yellow} 
\end{itemize} % }

\iusr{Александр Конев}

Многие сейчас увидели, что он психологически крошится под давлением. Серьёзные
лидеры государств не должны так психовать от вопросов журналистов. Это говорит
о том, что ранее настоящего давления на него не было. А отсюда уже можно
сделать много важных выводов о реальной ситуации в управлении страной.

\begin{itemize} % {
\iusr{Павел Недашковский}
\textbf{Александр Конев} отче, возможен ещё один вариант — это отыгрывание «откровенного и неформального разговора», что несколько раз повторил Президент. Об этом говорит и место проведения, и формат, и специфическое, «кухонное», арго. Мне (при этом я вообще монархист) это видится, как полная девальвация иерархии и институтов власти. Вопрос о содержательной стороне вопросов и ответов при этом вообще выношу за скобки
\end{itemize} % }

\iusr{Viacheslav Oriekhov}
Бутусову семками на базаре торговать с такой риторикой. Чван и хам

\iusr{Edgar Leitan}

Очень трудно и нереалистично даже считать себя "свободным" человеком, когда
кольцо власти сжимается удавкой, когда у нас в Австрии, например, под угрозой
тюрьмы заставляют вакцинироваться небезопасным чёртичем, когда практически
снова запрещены передвижения по миру, когда власти контролируют любое публичное
проявление даже частных мнений (в нашей же ах какой "свободной" Австрии
министерша юстиции из партии Зелёных хотела бы сажать за "распространение
заведомо ложных измышлений" о ... прививках, климате, и т. д. под предлогом
"борьбы с ненавистью в соцсетях и распространением ложной информации"). Всё это
очень относительно, и "свобода" - это всего лишь благородная риторика.

Нынешняя наша реальная свобода в ЕС сузилась, как и в СССР, до свободы критики
идиотов у власти в узких компаниях. Но в СССР не было запрета собираться
компаниями на частных квартирах. А в Австрии это теперь запрещено под угрозой
огромных штрафов и даже тюрьмы. 

А в полностью социалистической Шотландии запрещены даже любые неполиткорректные
высказывания (offensive speech) в частных группах на частных квартирах, за
которые положена тюрьма, если соседи или стукачи донесут. Да, недавно приняли
такой закон! И я думаю, что это лишь начало.

\begin{itemize} % {
\iusr{Edgar Leitan}

Мы не свободны, мы рабы так называемых "элит", их дойные коровы! Как в Австрии
выяснилось недавно, эти "элиты" (начиная с Курца и других членов его партии) в
своей переписке презрительно называют своих избирателей "плебсом" (Pöbel). Надо
этого не стыдится, но и не считать себя свободным. Надо снова восстать, как
восставали рабы, и развесить.... (далее вымарано цензурой). А потом уже всё это
отрефлексировать и пофилософствовать (горькая шутка, пардон).


\iusr{Елена Донец}
\textbf{Edgar Leitan} По-моему, это процесс, характерный для всего мира. У нас эпидемия абсурда!

\iusr{Edgar Leitan}
\textbf{Elena Donets} 

Да, именно. И если в СССР те, для кого удушливая атмосфера Совдепии была
невыносима, могли хотя бы мечтать вырваться в \enquote{свободный мир}, то теперь
никакого \enquote{свободного мира} не осталось вообще. И я боюсь, что мы живём лишь при
самом начале процесса. В наше время тот, кто считает себя подлинно свободным,
должен быть готовым заплатить за это самую высокую цену. Сесть в тюрьму или
быть ликвидированным \enquote{стражами глобалистской революции}. Тогда я поверю в
\enquote{свободу} этого конкретного человека.

\end{itemize} % }

\iusr{Александр Шрамко}
Зеленский был так жалок и одновременно хамовит, что напрашивался.

\iusr{Anna Makievska}

Увы, президент реагировал как раб, а не как правитель. Путин на его фоне
выглядит просто президентом мечты, так как не опускается до открытого
проявления негативных эмоций и хамства в ответ на хамство. Страна сделала
неверный выбор, произошёл неудачный найм, при этом система настолько
прогнившая, что не может эту ошибку никак компенсировать. Теперь нам важно всем
признать свою ответственность и найти способ уволить неудачного правителя без
жертв со стороны населения.

 · 1 д.
Віталій Козаренко
презедент не должен вести себя так безпардонно чтоб не нариваться на ненависть масс
 · 1 д.
Максим Колісник
Виходячи з ваших думок президент це раб з рабів бо стільки скільки хамив він не дозволяв собі ще жоден з президентів України.


\end{itemize} % }
