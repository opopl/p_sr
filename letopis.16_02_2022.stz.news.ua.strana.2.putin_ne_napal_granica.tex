% vim: keymap=russian-jcukenwin
%%beginhead 
 
%%file 16_02_2022.stz.news.ua.strana.2.putin_ne_napal_granica
%%parent 16_02_2022
 
%%url https://strana.news/news/377056-vtorzhenie-rossii-16-fevralja-ne-nachalos-kak-ob-etom-zajavljali-smi-britanii.html
 
%%author_id bovtruk_vlad
%%date 
 
%%tags napadenie,rossia,ugroza,ukraina
%%title Путин так и не напал. Видео "Страны" с границы в ночь "вторжения России"
 
%%endhead 
 
\subsection{Путин так и не напал. Видео \enquote{Страны} с границы в ночь \enquote{вторжения России}}
\label{sec:16_02_2022.stz.news.ua.strana.2.putin_ne_napal_granica}
 
\Purl{https://strana.news/news/377056-vtorzhenie-rossii-16-fevralja-ne-nachalos-kak-ob-etom-zajavljali-smi-britanii.html}
\ifcmt
 author_begin
   author_id bovtruk_vlad
 author_end
\fi

Ночью 16 февраля российское \enquote{вторжение} в Украину не началось в
анонсированное британскими СМИ время.

Напомним, что сразу два британских таблоида The Sun и Mirror написали о
\enquote{вторжении России} в ночь с 15 на 16 февраля и даже указали конкретное
время - час ночи (три часа ночи по киевскому времени).

The Sun описал, что все должно начаться с массированного ракетного удара и
атаки двухсоттысячной группировки. Удары якобы будут наноситься по военным и
правительственным объектам.

\enquote{Вторжение}, по данным The Sun, должно было начаться со всех сторон - севера,
востока и юга. 

Издание Mirror кроме того добавило, что Россия также высадит десант на
украинское побережье Черного моря.

Корреспондент \enquote{Страны} Влад Бовтрук отправился проследить за \enquote{вторжением} прямо
на украино-российский кордон. 

Однако на КПП \enquote{Бачевск} в Сумской области пока ничего тревожного не происходит
вообще. Пограничники, с которыми мы пообщалась, подтвердили, что обстановка
спокойная.

По их словам, контрольно-пропускной пункт в \enquote{ночь вторжения} работает в штатном
режиме без какого-либо усиления. Также не видно какой-либо военной техники с
украинской стороны.

Как и самолетов с танками от россиян. А вот простые люди проезжают спокойно.

\url{https://t.me/stranaua/24929}

Ранее \enquote{Страна} проанализировала, кто сообщал о \enquote{вторжении} 16 февраля и как на
это реагировали в Украине и России.

Напомним, что Владимир Зеленский объявил 16 февраля в Украине Днем единения и
призвал надеть сине-желтые ленточки и вывесить флаги, чтобы \enquote{показать миру
единство Украины}.
