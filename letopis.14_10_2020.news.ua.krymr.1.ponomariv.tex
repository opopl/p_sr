% vim: keymap=russian-jcukenwin
%%beginhead 
 
%%file 14_10_2020.news.ua.krymr.1.ponomariv
%%parent 14_10_2020
%%url https://ua.krymr.com/a/news-pomer-oleksand-ponomariv/30892823.html
%%tags oleksandr ponomariv,death,mova,ukraine
 
%%endhead 

\subsection{Помер український філолог Олександр Пономарів}
\label{sec:14_10_2020.news.ua.krymr.1.ponomariv}

\url{https://ua.krymr.com/a/news-pomer-oleksand-ponomariv/30892823.html}

У віці 84 років помер доктор філологічних наук та колишній професор кафедри
мови та стилістики Інституту журналістики Київського національного університету
імені Тараса Шевченка Олександр Пономарів, інформує Радіо Свобода.

Народний депутат від «Європейської солідарності» та випускник Інституту
журналістики КНУ Микола Княжицький повідомив у Facebook, що Пономарева
ховатимуть у суботу, 17 жовтня. Цього дня професору мало виповнитися 85 років.

«Сьогодні на Покрови помер Олександр Данилович Пономарів, видатний мовознавець
і викладач. Олександр Данилович викладав у мене українську мову. Вічная пам'ять
і співчуття близьким. Ховатимуть Олександра Даниловича у суботу», --- написав
Княжицький.

Олександр Пономарів народився 17 жовтня 1935 року. У 1961 році закінчив
філологічний факультет Київського університету. Протягом 1961-1976 років він
працював в Інституті мовознавства імені О. Потебні АН України, був редактором у
видавництві. У 1979-2019 роках викладав мовностилістичні дисципліни в Інституті
журналістики КНУ імені Тараса Шевченка. У 1988-2001 роках очолював профільну
кафедру.

Пономарів --- автор понад 250 наукових, науково-популярних і
науково-публіцистичних праць, присвячених проблемам стилістики, культури та
статусу української мови, українського правопису, нормативності мови ЗМК,
українсько-грецьких мовно-літературних зв’язків, теорії й практики перекладу.

У його доробку --- монографія «Лексика грецького походження в українській мові»
(2005), підручники й навчальні посібники «Стилістика сучасної української мови»
(2000), «Культура слова: Мовностилістичні поради» (2011), «Українське слово для
всіх і для кожного» (2013, 2017), за його редакцією та у співавторстві побачили
світ чотири видання підручника «Сучасна українська мова».

Пономарів був одним з укладачів «Етимологічного словника української мови» в
семи томах (1982-2012), «Словника античної мітології» (2006),
новогрецько-українського (2005) та українсько-новогрецького (2008) словників.
