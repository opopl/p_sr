% vim: keymap=russian-jcukenwin
%%beginhead 
 
%%file 04_02_2022.tg.kotsnews.1.blokirovka_ldnr_youtube
%%parent 04_02_2022
 
%%url https://t.me/sashakots/28268
 
%%author_id tg.kotsnews
%%date 
 
%%tags cenzura,dnr,donbass,lnr,ukraina,youtube
%%title История с блокировкой каналов ЛДНР на Youtube
 
%%endhead 
 
\subsection{История с блокировкой каналов ЛДНР на Youtube}
\label{sec:04_02_2022.tg.kotsnews.1.blokirovka_ldnr_youtube}
 
\Purl{https://t.me/sashakots/28268}
\ifcmt
 author_begin
   author_id tg.kotsnews
 author_end
\fi

История с блокировкой каналов ЛДНР на Youtube не сулит ничего хорошего и,
конечно, ее нельзя рассматривать как что-то отдельное, вне контекста общей
медиаистерии. Это, на самом деле, часть комплексной спецоперации. Сейчас
поясню. 

Мне все это напоминает 2014 год, когда Киев старательно зачищал информационную
поляну от инакомыслящих перед тем, как развязать войну в Донбассе. Сначала
отрубили российские каналы, затем запретили въезд в страну (а кто не помнит,
граница ЛДНР долгое время была под контролем Киева) российским журналистам, ну
а потом администрации соцсетей начали банить всех, кто дает отличную от
официальной Украины картинку о бойне в Донбассе. Я из банов просто не вылезал.
У меня было три аккаунта в Фейсбуке, два из которых постоянно были
заблокированы. А потом два и вовсе удалили без объяснения причин.

Любопытно, что вместе с информационными площадками ЛДНР Youtube удалил
украинские каналы условно Медведчука - «Перший незалежний» и UkrLive, созданные
журналистами заблокированных ещё год назад ZIK, 112, NewsOne. Они конечно не
вещали в поддержку республик, но в определенной ситуации могли бы сыграть
против Киева.

То есть, блокировкой в Youtube Украина со своими западными партнерами устраняет
внутреннюю информационную угрозу. Самых крикливых в соцсетях забанят
по-старинке. А как же российские журналисты, съехавшиеся в Донецк, спросите вы?

А в отношении них уже развязана кампания по расчеловечиванию. Думаете случайно
аккурат перед этими блокировками The New York Times, The Washington Post и
Reuters опубликовали одинаковые залипухи о том, как Россия готовит «красночный
пропагандистский ролик, в котором будут трупы и актеры, изображающие скорбящих,
а также изображения разрушенных районов для оправдания вторжения». Теперь любой
российский репортаж из зоны боевых действий можно выдавать за инсценировку.

В сухом остатке видим комплексную гибридную операцию по зачистке информполя при
участии западных соцсетей и транснациональных СМИ. Зачистка перед бурей.
