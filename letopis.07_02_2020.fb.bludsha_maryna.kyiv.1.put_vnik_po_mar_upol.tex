%%beginhead 
 
%%file 07_02_2020.fb.bludsha_maryna.kyiv.1.put_vnik_po_mar_upol
%%parent 07_02_2020
 
%%url https://www.facebook.com/maryna.bludsha/posts/pfbid02E2RD8jPFpUWwChtQ6cjB2e9MEMQYrau92iZimpi5szs8fZNrEoVMhzkuxBxDzh4bl
 
%%author_id bludsha_maryna.kyiv
%%date 07_02_2020
 
%%tags mariupol,mariupol.pre_war,putivnyk
%%title Путівник по Маріуполю. Pictures inside us
 
%%endhead 

\subsection{Путівник по Маріуполю. Pictures inside us}
\label{sec:07_02_2020.fb.bludsha_maryna.kyiv.1.put_vnik_po_mar_upol}

\Purl{https://www.facebook.com/maryna.bludsha/posts/pfbid02E2RD8jPFpUWwChtQ6cjB2e9MEMQYrau92iZimpi5szs8fZNrEoVMhzkuxBxDzh4bl}
\ifcmt
 author_begin
   author_id bludsha_maryna.kyiv
 author_end
\fi

Досі не можу повірити, що це сталося, але дивіться самі!

Я зробила «Путівник по Маріуполю. Pictures inside us». Дякую всім, хто був зі
мною на цьому довгому шляху:

😍Студії живої історії та Інша Освіта / Insha Osvita за підтримку на всіх
стадіях - від ідеї до презентації

😍усім менторам і запрошеним лекторам за віру в ідею і знання про історію,
основи драматургії і навіть тайм-менеджмент, окреме дякую Taras Grytsiuk i
Ярослав Переходько за нагадування і зустрічі

😍усім учасникам і учасницям Студій живої історії за розмови, поради і їхні
дослідження, які надихали мене

😍 моїм колегам по U-Cycle, які вірили в мене і підтримували цей божевільний
проект на всіх етапах

😍 команді Port.agency, завдяки якій відбулась перша публічна презентація в
Port Creative Hub у Києві

😍 фотографу і журналісту Иван Станиславский, у якого знайшлись фотокартинки до
всіх місць з путівника 

😍 ілюстраторці Марина Склодовська, яка намалювала за один вечір прекрасну
обкладинку

😍 усім, хто погодився розповісти свої історії і намалювати місця почуттів у
Маріуполі - вважаю їх усіх співавторами

(і тих, кого можна називати, і тих кого не можна:)

😍моїм батькам і друзям, які підтримували мене на всіх етапах - морально,
матеріально і просто добрим словом.

Путівник не відбувся б без усіх цих людей. Якщо я когось забула, то пробачте -
ще не відійшла від безсонних ночей останніх днів.

«Путівник по Маріуполю. Pictures inside us» складається з історій-замальовок
про місця в Маріуполі, які викликають у місцевих різні емоції: подив, щастя,
пригнічення, гордість тощо.

Це дає змогу побачити не лише місто на східному кордоні України, але й картину
світу його жителів, відчути атмосферу і почути одне одного, не їдучи через
півкраїни.

Але обережніше - після віртуального знайомства з містом вам може захотітись
побачити його на власні очі🤩

\url{http://mariupol-pictures.tilda.ws}

p.s. дякую всім, хто дійшов на презентацію путівника в Києві - за ваші очі,
інтерес, питання і фідбек!

і буду вдячна всім читачам путівника за ваші враження😇
