% vim: keymap=russian-jcukenwin
%%beginhead 
 
%%file 05_12_2021.fb.fb_group.story_kiev_ua.2.1994_maidan_foto.cmt
%%parent 05_12_2021.fb.fb_group.story_kiev_ua.2.1994_maidan_foto
 
%%url 
 
%%author_id 
%%date 
 
%%tags 
%%title 
 
%%endhead 
\zzSecCmt

\begin{itemize} % {
\iusr{Таня Бокий}
Моя любимая детская фотка с мамой,сделана на Полароид))

\iusr{Александр Герасько}
Цікава історія

\iusr{Надежда Владимир Федько}

Осінь 1994-го... Іду із знайомою балериною випити в \enquote{трубі} кави після
репетиції... На майдані зустрічаю друзів-фотографів. Вже не пам'ятаю, що
послужило причиною, але вирішуємо сфотографуватися на пам'ять...

Наталі за камерою... Ми позуємо (у мене в руках її кульочок...).

\ifcmt
  ig https://scontent-frx5-1.xx.fbcdn.net/v/t39.30808-6/263423179_4766915380034549_2558348543862489470_n.jpg?_nc_cat=105&ccb=1-5&_nc_sid=dbeb18&_nc_ohc=_1ERTu5yWfcAX_xqTmS&_nc_ht=scontent-frx5-1.xx&oh=00_AT9X2M9f5_NxjdKrD7kpmQFZmKvbKOnK_PGzQmCmy7AEFw&oe=61C8A1E6
  @width 0.4
\fi

\begin{itemize} % {
\iusr{Тарас Єрмашов}
То відомий фотограф Є. Лукацький скраю, здається?

\iusr{Надежда Владимир Федько}
\textbf{Тарас Єрмашов} Так, це Єфрем))

\iusr{Тарас Єрмашов}
\textbf{Надежда Владимир Федько} А за ним, з вусами, схожий на мого френда С. Волошина...

\iusr{Тарас Єрмашов}

В \enquote{Поляроїді} я розчарувався, коли побачив, як швидко облущується зображення на
його фотографіях, наче фарба на старих рамах...

До речі, на з-ді \enquote{Арсенал} з середини 90-х, у моєму цеху, робили до них на
основу дюралеві пластини (для Бразилії, здається, де їх збирали - якщо точно
пригадую).


\iusr{Надежда Владимир Федько}
\textbf{Тарас Єрмашов} 

Всьому свій час. Якщо врахувати, що перші фотографії з'явилися вже у 1948
році... Пік популярності прийшовся на 80-ті роки. Фактично у 1994-му Поляроїд"
доживав останні дні.


\iusr{Надежда Владимир Федько}
\textbf{Тарас Єрмашов} 

У одного мого приятеля був \enquote{Хассельблад}. (Це той фотоапарат з якого зробили
копію - \enquote{Салют}). Так крім касет для плівки \enquote{Рольфільм 120}, у нього була ще
касета для комплекту \enquote{Поляроїд}.


\iusr{Тарас Єрмашов}
\textbf{Надежда Владимир Федько} 

Ну, не знаю - за моїми спостереженнями, він став модним в Україні з середини
90-х і спокійно дожив до поч. 2000-х (та й завод виконував замовлення до цих
часів, до 2000 року точно). У 1997 один родич на Донбасі нас фотографував ним,
дорогою на той час іграшкою. Зараз у молоді, бачу, знову на нього мода,
продають успішно на різних заходах.


\iusr{Надежда Владимир Федько}
\textbf{Тарас Єрмашов} 

На той час основною тематикою моїх фотозйомок були балет, портрет, скульптура,
живопис і графіка. \enquote{Поляроїд} не підходив для цих жанрів фотографії. Тому я
задовольнив свою цікавість, погрався і поміняв іграшку на іншу.


\iusr{Тарас Єрмашов}
\textbf{Надежда Владимир Федько} 

Та звісно - я, хоч і не \enquote{технар}, і то це розумію (щоправда, обмежуюся, як
аматор, недорогою бездзеркалкою - до 5 тис., спершу \enquote{Ніконом}, зараз \enquote{Кеноном}.
Іноді непогано наче виходить).


\iusr{Надежда Владимир Федько}
\textbf{Тарас Єрмашов} Фотографія спочатку народжується у мізках. Якщо людина \enquote{не бачить} прекрасного, то жодна техніка їй не допоможе.
\end{itemize} % }

\iusr{Марія Голуб}
Атмосфера мого дитинства...
Цікаво читати ваші дописи, дякую!
Вибачте за питання, у вас з дружиною є діти?

\begin{itemize} % {
\iusr{Надежда Владимир Федько}
\textbf{Марія Голуб} 

Дивне запитання? Добре, відповім... Донька і шестеро онуків)) До речі, донька ж є на фотографіях...

\iusr{Марія Голуб}
\textbf{Nadegda Volodymyr Fedko} 

я уже дочитала) дякую за відповідь! До цього читала ваші історії у \enquote{Старих
фотографіях}, там про доньку нема, чи я пропустила.

Ви багатий дідусь! @igg{fbicon.heart.red}

\iusr{Надежда Владимир Федько}
\textbf{Марія Голуб} Фотопортрети доньки є у моїй публікації в \enquote{Киевские истории}.
\end{itemize} % }

\iusr{Галина Тучина}

У нас тоже есть Поляроид. Очень классно фотографировать им и, главное, не надо
никаких навыков. До сих пор сохранился, очень жаль, что нет для него кассет. Они
стояли дороже самого фотоаппарата. Да и фирма, выпускавшая, Поляроиды
разорилась. Очень жаль.

\iusr{Надежда Владимир Федько}
\textbf{Галина Тучина} Невже смартфон гірший?

\iusr{Юрий Панчук}

І зараз є люди, які принципово продовжують знімати на плівку. Якось одна людина
сказала \enquote{фото на плівку - це мистецтво, цифрове фото - це вже тираж}. І
справді, раніше сім раз подумаєшь перед тим, як витратити один з 36-ти кадрів,
а зараз наклацал пару сотен, потім переглянув, 70\% видалив. Є диваки, які
сьогодні захоплюються саме моментальними фото на Полароїд, знаходять в ньому
якийсь особливий кайф. Подружка доньки нещодавно просто переповнена щастям
придбала Полароїд. Напевно незважаючи на банкрутство, продукція все рівно
виробляється. Дивно, що перший апарат для моментальних фото на папері був
створений ще в 1948 році! Зараз такі новинки вже не дивують, але ще у 80-х це
було просто неймовірно!

\begin{itemize} % {
\iusr{Надежда Владимир Федько}
\textbf{Юрий Панчук} 

Я фотографую з 8 років. Коли мені було 9 років, то брат, офіцер військовий
топограф, подарував мені ФЭД. На той час (1956-й) це було велике багатство для
мене.

У плівковій фотографії є магія...

Фотографуєш... проявляєш плівку... Потім сідаєш друкувати фотографії. Червоне
світло... Проектуєш негаттив, кадруєш, підкладаєш фотопапір, експозиція...
Папір не змінився. Ти кидаєш його у проявник і... з'являється зображення. Про
це можна роман написати...

***

У 2000 - 2005 роках я працював менеджером в \enquote{Юг-Контракт}. Наша фірма
торгувала фотоапаратурою, плівками і т.д. За ці роки я протестував сотні
фотокамер.  Починаючи від \enquote{мильниць} і закінчуючи цифровими
\enquote{Nikon}, \enquote{Canon}, \enquote{Sony}...

***

Для пересічного фотоаматора цілком досить смартфона.

\iusr{Юрий Панчук}

Так, я ще застав цю магію. Фотографував на Зеніт 3м з Індустар 50 моєї мами.
Проявник, фіксаж, проявник для плівки, папір. Потім вже в нульових придбав за
копійки старий Зеніт ТТЛ з Геліосом. Заряджав кольорову плівку, фотографії
виходили просто суперові, з новими мильницями, які тоді коштували дорожче ніяк
не порівняти. Перші чотири роки доньки засняті на той ТТЛ! Потім в Зеніта щось
із шторкою трапилося, я вже не став ремонтувати, перейшов на цифру.

\begin{itemize} % {
\iusr{Надежда Владимир Федько}
\textbf{Юрий Панчук} 

В період 1974 - 1980-й я багато фотографував на слайдову плівку Орвохром. Майже
до 2008-го фотографував на кольорову і друкував у наших фотолабораторіях. Потім
перейшов на цифру.

\iusr{Юрий Панчук}
\textbf{Надежда Владимир Федько} 

Майже все моє дитинство відснято на ГДРівську Орвохром. Нещодавно відсканував
ці слайди, відцифрував. Досить непогано получилося.

\iusr{Надежда Владимир Федько}
\textbf{Юрий Панчук} Я теж зараз оцифровую свій слайд-архів...

\iusr{Надежда Владимир Федько}
\textbf{Юрий Панчук} Пане Юрію, а на чому Ви оцифровуєте слайди? Я на сканері UMAX Astra 3450 (з слайд-приставкою 10х15).

\iusr{Юрий Панчук}
\textbf{Надежда Владимир Федько} Відверто кажучи, сканував чоловік сестри  @igg{fbicon.smile} . Можу перепитати.

\iusr{Юрий Панчук}
Це я ))

\ifcmt
  ig https://scontent-frt3-1.xx.fbcdn.net/v/t39.30808-6/265384670_1799435596908322_8878969106650234441_n.jpg?_nc_cat=106&ccb=1-5&_nc_sid=dbeb18&_nc_ohc=Z_kZ20iOPdkAX92L7Hd&_nc_ht=scontent-frt3-1.xx&oh=00_AT8W_D6ko8q2a_F29_T0_EKpkKi4UmAcb2Z_jCLU26yAIA&oe=61C82C85
  @width 0.4
\fi

\iusr{Надежда Владимир Федько}
\textbf{Юрий Панчук} Якість чудова. Нормальна різкість і передача кольорів.

\iusr{Юрий Панчук}
\textbf{Надежда Владимир Федько} 

От він відповів HP scanjet 5500c+TMA C9911A. Слайдадаптер. Каже, що сканер
достатньо старий і працює під Windows 7. Наскільки я розумію, ці послуги зараз
недорогі і можна зробити на замовлення.


\iusr{Надежда Владимир Федько}
\textbf{Юрий Панчук} Дякую)

\end{itemize} % }

\end{itemize} % }

\end{itemize} % }
