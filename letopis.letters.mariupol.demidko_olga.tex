% vim: keymap=russian-jcukenwin
%%beginhead 
 
%%file letters.mariupol.demidko_olga
%%parent letters
 
%%url 
 
%%author_id 
%%date 
 
%%tags 
%%title 
 
%%endhead 

дякую, буду щиро вдячний! Збираюсь робити (і вже роблю) тематичні збірки
(альбоми) по різним темам та подіям... Як-от,
Маріуполь-Травень-Великдень-Писанки-Кролі-2019, або Шахове Життя Маріуполя,
короче багато цікавих тем!  На жаль, є така проблема (тема). Тема називається
Цифрова Спадщина Маріуполя та Важливість її запису та збереження, та її
важливість у справі Культурно-Духовного Відродження Маріуполя.  І Цифрова Спадщина ця, -   
з однієї  сторони, просто величезна - на даний момент, це тисячі, десятки
тисяч, мільйони фото, публікацій в фб, і т.д. А з іншої сторони, все це зараз
перебуває у дуже підвішеному розпорошеному незв'язаному стані, під загрозою
зникнення... Ось наприклад, оте фото зверху Капсули Часу. Я от ніколи не був в
Маріуполі, і взагалі із Маріуполем не був ніяк раніше пов'язаний.  То як я її
дістав? А ось так. Поставив точку в гугл-мапах, заліз, потім в результаті
вишукування по Кролям та Писанкам таки дібрався до цого фото і зберіг.  А поки
я її не зберіг - вона може зникнути в будь-який момент. Сьогодні фото є - а
завтра його вже немає, і назавжди немає... А гугл або фейсбук потім скаже -
вибачте, а оцього фото (сайту, посту, альбому тощо) вже нема, сорян!  І як ти
потім доведеш комусь, що саме оцей офігезний Кроль (я вже визначив його автора,
до речі) таки був у Маріуполі у травні 2019 року? І це насправді вже
відбувається. Маріуполь зникає поступово інформаційно. Як отой Чеширський Кіт -
щасливе безтурботне Місто посміхається, співає, кличе, у ньому кожен день в
минулому відбувається безліч цікавих подій, зустрічей, тощо але... Посмішка
того щасливого Міста кожен Божий день стає все меншою і меншою...  Процес цей
доволі повільний і непомітний для багатьох, але це факт, мирне довоєнне Місто
зникає інформаційно...  Величезний обсяг, просто неймовірно величезний,  -
тобто, все те, чим жило Місто до 24 лютого, поступово зникає з Інтернету.  Є
вже купа сайтів (як-от той vidsebe.media або сайт Маріупольські Новини або ще
десяток - я вже цілий список склав в процесі своїх досліджень), які вже
просто-напросто не існують (публікація Тетяни, яку я виклав недавно, - це вже
мій результат години роботи по ретельному витягуванню змісту з Інтернет-Архіву,
і потім склеювання докупи із максимально можливим збереженням початкового
вигляду і наповнення). А навіть коли залазиш в Інтернет-Архів і щось читаєш,
буває так, що є новина, але фото вже немає.  Наприклад. Є новина в
Маріупольських Новинах, що писанки в Маріуполі було повалені вітром, але
встояли(!) - в Києві на Софійській Площі до речі було теж саме, але писанки
повалились ))) Новина є - а фото вже немає, як оті писанки та кролі встояли
проти вітру... І нічого вже не зробиш, окрім як напевне питати вже конкретних людей,
слухай, а ти пам'ятаєш і може маєш фотки, - як воно саме було, що писанки
встояли проти вітру під час тої травневої виставки?  Ну ок...  тема дуже
важлива, але складно тут в коментарях все тут описати, як треба, а писати
детальний пост про все це, я ще з думками не зібрався...


%13:07:43 24-04-23
дякую! До речі, щодо пошуку... як це виглядає у мене тута на компі -
траверсування вниз в минуле по точці або сторінці місця. Ставиш точку і
починаєш гортати, гортати, гортати... І поки знайдеш хоч щось що тебе цікавить
зараз (тобто Кролі або Писанки), побачиш безліч різних цікавинок, як-то
Пряники Маріуполя тощо. І до речі! деякі Кролики (або навіть Кроликині! От яке
слово походу само собі вилізло), які (точно!) були в Маріуполі, зараз
знаходяться в музеї міста Києва, Богдана Хмельницького 7 (+ там ще писанки в
холі на першому поверсі стоять). Два кролика (по моєму, один з них точно був в
Маріуполі, але треба звірити по відповідним фото з Маріуполя...) стоять на
вході. І купа кроликів (два або три з них точно були в Маріуполі) зараз
одиноко стоять на четвертому поверсі музею. Там нікого немає, немає виставок.
Тільки ліфт, кролики, ну і вікна, і все. Самі собі кролики стоять, сумують...
Чекають... може хто прийде і побалакає з ними... І не знаю ще скільки вони там
будуть... Може завтра їх вже запихнуть в тісну темну кімнату на зберігання і
вони будуть цілий рік чекати аж до наступного Великодня, поки їх знову
дістануть, хто зна?....

%11:04:35 27-04-23
Доброго ранку, пані Ольга! Чудово, що будете екскурсію проводити... так...
можливо, Ви все це вже знаєте, що я нижче напишу, і навіть знаєте вже те, чого
я не знаю... ))) я дивився відео старе, де Ви проводите екскурсію (відео Сергій
Чабар, потяг Труханівська Січ)) ) () просто поділюсь тим, що в голову зараз
прийшло відносно вашої запланованою екскурсії... Там є (1) будинок Сікорського,
печальне видовище, насправді (2) на вулиці також є будинок, де жила Леся
Українка, (3) поруч, на вулиці Олеся Гончара, музей шістдесятників та
Кабінет-музей В'ячеслава Чорновола (4) недалеко є Нестерівський провулок, там
жив Іван Франко (5) !! в кінці вулиці, там, де вже Львівська площа, знаходиться
універ Карпенко-Карого, там у них є навчальний театр! вистави просто агонь!!
свого часу я напевне на вистав десять сходив, одна за одну )) (6) так... ще там
є будинок, де жив Віктор Грушков, патріарх української кібернетики... ну, ця
тема близька до мене... до речі, перший комп'ютер в континентальній Європі
зробили в Києві, на території монастирських будівель в Феофанії в 1950 році...
також, там є в окрузі декілька цікавих муралів + книгарні Академкнига, Абзац.
Ну і звісно кафе Ярославна з булочками )) і ще там є... ем, чи є зараз,
незрозуміло, студія репера Ярмак, я от дивлюсь на їх сторінку, там немає зараз
адреси, можливо через війну приховали (Ярмак зараз в ЗСУ), чи переїхали, не
знаю... але свого часу я слухав багато їх музики, цікавився... про Київ вони
багато пісень записали, як-то ось ця
https://www.youtube.com/watch?v=bF0lV0Xtqjk (напевне, моя найулюбленіша пісня
про Київ, поруч із Гімном Києва, звісно... ). Короче, дуже добре, що Ви робите
екскурсію ))) 

;
06:50:10 02-05-23

Доброго ранку! Знаєте, я теж не бачив... донедавна... Але... Короче. Спочатку
треба зробити невеличке пояснення. Ееем... Пояснення може розтягнутись в
принципі на декілька сторінок, але я сподіваюсь все ж таки втиснути в якісь
пристойні рамки. Справа в тому... що ми всі живемо (1) в одному великому світі,
ну тобто ми всі живемо (а) в Україні (б) на Планеті Земля (в) у Всесвіті, якщо
брати вверх, і... ну тобто, у світі, де для всіх нас світить сонце, віє вітер,
і т.д. однаковим чином. Це є така собі велика сукупна Реальність. І є...
реальність персональна, тобто, певні речі, події, які, будучи влаштованими
однаковим чином у реальному просторі або ж будучи зробленими однаковим чином в
інформаційному просторі, як наприклад, фото певних обьєктів в фізичному
просторі, ну тобто, наприклад, фото якихось будівель, міст, і т.д. тим не менш,
мають абсолютно різне смислове і психологічне навантаження. Це один аспект мого
вступу. Інший аспект, це.. важливість того, як ми пишемо Слова, і відношення
цього до війни з московією. От дійсно, ми не пишемо зараз рос-федерація або ж
росія (з великої літери), ми пишемо росія або московія з малої літери. Більш
того, це вже стало питанням державного значення, як саме нам називати росію..
ну тобто у Верховній Раді було підняте таке питання, наскільки я знаю з
новин... І більш того, на мою думку, особливість цієї війни, і також причина
того, чого ця війна є настільки страшною, жорстокою, чому настільки велике
психологічне і емоційне навантаження від неї, є в тому, що ця війна не є чисто
війною тільки за території. Це війна, яка охоплює багато сфер одночасно, і що
важливо тут відмітити, сферу когнітивно-метафізичну, яка зачіпає самі основи
нашого буття, та нашої історії від давніх часів. Ну тобто війна з московією -
це також війна метафізично-когнітивна, війна за те, чий спосіб мислення, чиї
метафізичні конструкції, чиє розуміння історії і т.д. і т.д. переможуть в
кінці. Війна Слів, Війна Смислів, Війна Символів. І тому така жорстокість і
безумство зі сторони росії, тому що вони дійсно вірять в той брєд, що ллється з
іх ресурсів, а вірять через те, що вся ця пропаганда грунтується на досить таки
глибокій промивці мізків через історію. культуру і т.д. От бачите, скільки вже
написав... а я відчуваю... що я навіть ще не починав ))) ну добре... Щодо того
написання київ чи Київ чи КИЇВ. В інтернеті у орків знаєте, є така картинка.
Якщо треба, я знайду і надішлю для ілюстрації в персональном порядку... Не
хочеться таке безобразіє виставляти тут... Називається у них - "робота над
помилками". Росіяни намалювали. Там гарний вид Хрещатику. Потім - надпис -
Київ. Потім. Буква ї - ЗАКРЕСЛЕНА. Замість неї, вставлена буква Е. Написано -
робота над помилками. + кольори російського прапору. Розумієте. От за цю букву
Е вони реально готові вбивати, реально їдуть сюда воювати, як... не знаю, яка
аналогія є... Їх натаскали психологічно на оцю букву ї настільки, що коли десь
в чатах воюєш з ними, то вони зазвичай кажуть не Киев, а "Куев", або ж
"кукуев", або ж @ев, або ж ... тобто з презирством таким. Це напевне також йде
від англійського написання як KYIV. Їм незрозуміло оця ї, це їх приводить
просто в бєшєнство. А чому оця штука з Києвом. Ну а тому що в їх міфології і
свідомості Київ - це виток їхньої державності, і тому все що повьязано з
Києвом, настільки їх збуджує... То вони беруть Київ за три дня, то вони оруть
благім матом, що знесуть Київ ядерним ударом, і їм іже нічого і нікого не
жалко, бо тут саме питання існування росії, і як вони самі собі кажуть,

    Reply
    2m

Іван Іван

нащо потрібен той світ, якщо в ньому не буде росії, то вони мріють щоби Київ
весь захлинувся у нечистотах, щоби у отих клятих @@влян не було навіть чим
змити воду в туалеті, то навпаки опаришна сімоньянша влізає і починає
заспокоювати, ну що ви, ну що ви, не хвилюйтесь, ніхто по Києву ядерного удару
наносити не буде, тут же Лавра і мать-городам-русским + владімір владімірович
поважає Лавру, так що нехай Київ живе! разрешаєм-с... і сімоньянша починає
також пускати сльозки, ах... от як я хочу попасти в Київ, де вареники, Оксани,
парубки, і де всі тобі посміхаються... Ну реально вони просто тупо божевільні
там у себе. Така собі психушка на 100 мільйонів душ і 1/6 частину планети... Це
щодо Києва. А щодо буковок. От ви напевне памьятаєте те фото, коли орки міняють
таблички на в'їзд до Маріуполя або виїзд. Для Вас наприклад це має просто
неймовірно величезне емоційне навантаження, і зрозуміло чому. Тому що це Ваше
рідне місто, його зруйнували, забрали його у Вас, і тепер орки там хазяйнічают
і міняють все як їм заманеться, витирають, знищують нашу культуру там. Бачите,
одна літера, а стільки нескінченно важливого смислу!

Який зв'язок між

(1) оцим Кроликом
(2) вулицею Коперника у Львові
(3) одним із найвидатнішим гітарістом 
світу Річі Блекмором

Здоров'я та низький уклін Ігорю!!! Дуже добре, що знаходите на це час!! Я зі
свого боку - інформаційної сторони, теж займаюсь цим, тобто, записую,
систематизую інформацію про наших Героїв, особливо полеглих... бо. знаєте, їх
фейсбук акаунти... вони можуть зникнути в будь-який момент, а це ж жива
пам'ять... і все, що вони писали..., повинно бути збережено, це абсолютно все
безцінно... ось, якраз зараз виклав збірку, присвячену волонтеру та воїну ЗСУ
Денису Галушко, він загинув під Бахмутом 8 січня цього року... а вступив до лав
ЗСУ теж десь так у березні 2022, сам він киянин, мав кафе до війни... зараз
якраз пишу пост про це... https://archive.org/details/geroi.denys_galushko

\ii{letters.mariupol.demidko_olga.1}
\ii{letters.mariupol.demidko_olga.2.ohmatdyt}
\ii{letters.mariupol.demidko_olga.3.kartoplja}

%06:13:19 01-10-23
\ii{letters.mariupol.demidko_olga.4}

\ii{letters.mariupol.demidko_olga.5}
