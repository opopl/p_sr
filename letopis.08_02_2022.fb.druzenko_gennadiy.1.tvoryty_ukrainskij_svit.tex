% vim: keymap=russian-jcukenwin
%%beginhead 
 
%%file 08_02_2022.fb.druzenko_gennadiy.1.tvoryty_ukrainskij_svit
%%parent 08_02_2022
 
%%url https://www.facebook.com/gennadiy.druzenko/posts/10158754264748412
 
%%author_id druzenko_gennadiy
%%date 
 
%%tags tvorchestvo,ukraina
%%title ТВОРИТИ УКРАЇНСЬКИЙ СВІТ!
 
%%endhead 
 
\subsection{ТВОРИТИ УКРАЇНСЬКИЙ СВІТ!}
\label{sec:08_02_2022.fb.druzenko_gennadiy.1.tvoryty_ukrainskij_svit}
 
\Purl{https://www.facebook.com/gennadiy.druzenko/posts/10158754264748412}
\ifcmt
 author_begin
   author_id druzenko_gennadiy
 author_end
\fi

ТВОРИТИ УКРАЇНСЬКИЙ СВІТ!

Коли путінська орда лязкає зброєю біля наших кордонів, я вирішив запросити до
себе в студію Тарас Компаніченко, який для мене є камертоном та втіленням
української культури: \url{https://youtu.be/PBssTkmVW9E}

\ifcmt
  ig https://scontent-mxp1-1.xx.fbcdn.net/v/t39.30808-6/273546782_10158754240893412_4714174418571913185_n.jpg?_nc_cat=108&ccb=1-5&_nc_sid=730e14&_nc_ohc=o1blRi8avPMAX8UhAdN&_nc_ht=scontent-mxp1-1.xx&oh=00_AT92NU2wWi4KmCnswqY8SBorHIMumjE7ThDOme3xnYWv3w&oe=620FAA6B
  @wrap center
  @width 0.8
\fi

Колись Борис Пастернак написав: \enquote{1913-й був останнім роком, в якому любити було
легше, аніж ненавидіти}. Ми почали нашу дружню бесіду з питання: як нам не
розучитися любити в цей важкий час? Як не заразитися від нашого ворога недугою
ресентименту? Як уникнути спокуси стати його маленькою копією?

Від рефлексій про дотримання моральної правди як запоруки нашої перемоги ми
перейшли до дуже цікавого питання, що вважати українською культурою та
українством. Я часто аргументую, що ми не маємо права зрікатися наших
російськомовних геніїв: Сковороди, Гоголя, Короленка. Але Тарас продовжив та
розвинув мою думку: ми також маємо памʼятати наших латиномовних та
польськомовних мислителів та поетів. Навколо україномовного ядра має обертатися
безліч культурних супутників, які проросли на нашій землі іншими мовами:
латиною, польською, ідишем, німецькою, російською...

Врешті-решт ми перейшли до Сковороди. Як залишатися вільним у часи, коли
хочеться збитись у зграю і відкинути як непотріб будь-які етичні умовності? Як
залишитись людиною, коли час спонукає стати звірем?

Дружні роздуми про всі ці непрості питання – у нашій розмові з українським
лірником, бандуристом, кобзарем Taras Kompanichenko. І звичайно його геніальні
пісні! Все це ми вмістили в нашу 53-хвилинну розмову. Насолоджуйтесь!
