% vim: keymap=russian-jcukenwin
%%beginhead 
 
%%file 16_10_2021.fb.bilchenko_evgenia.2.poezia_ljubov_chekanova_margarita.cmt
%%parent 16_10_2021.fb.bilchenko_evgenia.2.poezia_ljubov_chekanova_margarita
 
%%url 
 
%%author_id 
%%date 
 
%%tags 
%%title 
 
%%endhead 
\subsubsection{Коментарі}

\begin{itemize} % {
\iusr{Максим Иващенко}
Вначале нас научили купаться в плавках, потом - заниматься любовью в презервативах, а теперь - дышать в масках. Что дальше?

\begin{itemize} % {
\iusr{Евгения Бильченко}
\textbf{Максим Иващенко} 

картина жизнеутверждающая, стих классически русский, а мы не в палеолите. Муж
болел. Нам не понравилось.


\iusr{Евгения Бильченко}
\textbf{Максим Иващенко} 

нельзя защищать русский Логос, собирая весь обскурантизм в округе. Блин, учишь
вас, учишь собирать нормально Калашников, а вы опять за свое...


\iusr{Евгения Бильченко}
\textbf{Максим Иващенко} 

я не нудист, я православная, да, вот такая, с рэпом, постмодерное и сленгом.

\iusr{Алексей Бажан}
\textbf{Евгения Бильченко} 

Сколько могу судить, продвинутая молодежь о Майке, Цое, Окуджаве и Высоцком или
не слышала, или их продукция служит темой для шуток. Зачем тогда к этому
апеллировать?

\iusr{Евгения Бильченко}
\textbf{Алексей Бажан} 

Наверно, чтобы знали. Вон дева у Собчак считает, что "Отцы и дети" Толстой
написал. Так что мне теперь? На заваленке сидеть и ныть в сети? Нет, не для
занудства, ностальгии и ретроградства мать русская поэзия меня рожала.

\iusr{Алексей Бажан}
\textbf{Евгения Бильченко} 

Был в русской культуре такой роман Что делать и поэт Надсон. Звезда великого
поэта быстро закатилась после 1917-ого; величие Чернышевского продержалось до
перестройки, хотя уже Розанову было непонятно, как такое вообще можно читать.
Аналогия понятна? Цой сейчас, как и Чернышевский в советское время, явно
занимает не свое место - для наглядности песня эпохи Цоя

\href{https://www.youtube.com/watch?v=Bq1CgKbye24}{%
Вежливый отказ - Помощник, Vadim Neronov, youtube, 24.01.2015%
}

\end{itemize} % }

\iusr{Алексей Бажан}

Справедливости ради, первоисточник не русский (у Майка, кажется, вообще не было
собственных песен), а perpetuum mobile взялся как бы не из припева
непосредственно.

\href{https://www.youtube.com/watch?v=3kh6K_-a0c4}{%
Bob Dylan - Stuck Inside of Mobile with the Memphis Blues Again (Official Audio), youtube, 12.03.2019%
}

\begin{itemize} % {
\iusr{Евгения Бильченко}
\textbf{Алексей Бажан} вечный двигатель как первообраз взялся из философии 16 века вообще-то.

\iusr{Алексей Бажан}
\textbf{Евгения Бильченко} Что поет Майк, я, слава Богу, давно забыл, но "переведенная" им песня называется Торчу в Мобиле с мемфисским блюзом в голове.
\end{itemize} % }

\end{itemize} % }
