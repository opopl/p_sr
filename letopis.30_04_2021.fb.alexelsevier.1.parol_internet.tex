% vim: keymap=russian-jcukenwin
%%beginhead 
 
%%file 30_04_2021.fb.alexelsevier.1.parol_internet
%%parent 30_04_2021
 
%%url https://www.facebook.com/alexelsevier/posts/1483201038691873
 
%%author 
%%author_id 
%%author_url 
 
%%tags 
%%title 
 
%%endhead 
\subsection{Инакомыслящие пишут, что спецслужбы незаконно получают доступы к их файлам с паролями аккаунтам в соцсетях}
\Purl{https://www.facebook.com/alexelsevier/posts/1483201038691873}

\ifcmt
  pic https://external-bos3-1.xx.fbcdn.net/safe_image.php?d=AQF-Ai4odnytPRpp&w=500&h=261&url=https%3A%2F%2Fwww.stickypassword.com%2Fpictures%2Fshare-ru.jpg&cfs=1&ext=jpg&ccb=3-5&_nc_hash=AQFtiC_a56MZJxMt
\fi


Инакомыслящие пишут, что спецслужбы незаконно получают доступы  к их файлам с
паролями аккаунтам в соцсетях. Пару слов о технической стороне безопасности
паролей, есть ещё и человеческая, и юридическая. О чем знаю, о том и пишу.  Что
ж, силовые органы действуют по своей логике. Неугодным для власти людям это
приносит как минимум существенные затруднения. Хороший рецепт от таких действий
следующий: либо вообще не хранить пароли в каких-либо файлах, либо обеспечивать
безопасность пароля одним или несколькими из этих способов: 1. Самый надежный.
Пользуйтесь специальными программы для хранения паролей. Вроде Sticky password.
Вам нужно помнить только главный пароль, остальные она хранит в зашифрованном
виде. Вам не нужно их вводить при входе на сайт. Программа войдет сама. Помните
только пароль от программы Ее часто раздают безплатно 2. Для паролей на сайтах
используйте двухфакторную идентификацию. То есть, Вам не достаточно ввести
пароль. После успешного ввода пароля, на Ваш телефон вам пришлют СМС с кодом
подтверждения. Вводите верный код и заходите. Применяют и в Контакте, в Гугле,
в Фейсбуке. 3. Поставьте пароль на файл с паролями. 4. Заархивируйте этот файл
любой программой архивации с парольной защитой. 5. Храните пароли в файле
контейнере. Подробнее о таких узнаете по одному из ключевых слов: Veracrypt
(бесплатный), Bestcrypt, Steganos, Kaspersky total security, Truecrypt (для
старых операционных систем и ПК, разработчики заменили его на Veracrypt), иные,
Все эти рецепты годятся и для смартфонов. Sticky password надежнее других
потому, что с помощью современного шпионского ПО и обыск порой не нужен. Если
программы которые отслеживают нажатия клавиш на реальной клавиатуре,
фотографируют экран монитора, взламывают пароли и т.п. Берите Sticky password и
вводите пароль программы с помощью виртуальной клавиатуры (имитация реальной),
храните пароли в зашифрованном виде (на взлом могут уйти столетия), применяйте
двойной вход в программу (помимо пароля нужен код, отображаемый на Вашем
смартфоне). Чтобы понять смотрите материал по ссылке ниже. 
