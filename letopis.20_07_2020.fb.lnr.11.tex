% vim: keymap=russian-jcukenwin
%%beginhead 
 
%%file 20_07_2020.fb.lnr.11
%%parent 20_07_2020
 
%%endhead 

\subsection{Пострадавшие от ВСУ жители Золотого-5 записали видеопоказания для МУС и ЕСПЧ }
\url{https://www.facebook.com/groups/LNRGUMO/permalink/2861122283999327/}
  
\vspace{0.5cm}
{\small\LaTeX~section: \verb|20_07_2020.fb.lnr.11| project: \verb|letopis| rootid: \verb|p_saintrussia|}
\vspace{0.5cm}

Пострадавшие от военной агрессии Киева жители Золотого-5 при поддержке
общественной организации (ОО) "Мемориал" записали видеообращения, которые
станут очередными свидетельскими показаниями в Международном уголовном суде
(МУС) и Европейском суде по правам человека (ЕСПЧ). Об этом сообщил член
правления общественной организации "Мемориал" Сергей Белов. 

"Сегодня мы приехали в Золотое-5 для видеофиксации обращений, а если быть
точным – свидетельских показаний жителей, пострадавших в результате украинской
агрессии. Эти показания в дальнейшем будут в качестве материалов переданы в
такие международные инстанции, как Международный уголовный суд и Европейский
суд по правам человека", - сообщил он, уточнив, что на встречу пришли около 30
местных жителей, 10 из которых дали показания. 

Белов напомнил, что общественная организация "Мемориал" уже продолжительное
время оказывает помощь жителям ЛНР в составлении исков по факту военных
преступлений ВСУ и причинения материального ущерба в результате обстрелов. 

"Теперь к тому формату, к формуляру иска, который заполнялся согласно
нормативам МУС и ЕСПЧ, мы стараемся записать и видеопоказания людей для того,
чтобы дать максимум информации, чтобы было видно человека, который пострадал и,
самое главное, мы, насколько это возможно, стараемся делать видео
непосредственно в месте, где произошла та или иная трагедия", - рассказал
общественник. 

Он подчеркнул, что благодаря содействию "Мемориала" жители ЛНР вместе с
юристами из России составили около 200 исковых формуляров, каждый из которых
содержит несколько эпизодов преступлений ВСУ. 

"В целом мы уже рассказали Европе, и надеемся, что она нас все-таки услышит, о
более чем 500 эпизодах. Они должны начать реагировать на происходящее, и
надеемся, что, в конце концов, во всех международных отчетах мы увидим
информацию, которую передавали и отсюда", - добавил Белов. 

В двухэтажный жилой дом в центре населенного пункта, где проживал Анатолий
Кислица с семьей, попали два реактивных снаряда системы залпового огня (РСЗО)
"Град": один – в основание дома, второй – в квартиру соседей. 

"11 февраля 2015 года, около часа дня, вдруг все как посыпалось! Я сидел на
диване с ребенком, жена была на кухне. Ребенка я буквально вытолкнул в коридор
от падающих конструкций в комнате; жена отошла к холодильнику, когда со стены
сорвало кухонный шкаф. Все чудом остались живы", - вспоминает житель
Золотого-5. 

Дом пострадал настолько, что всем жильцам пришлось съехать. У семьи Кислицы был
только один вариант переезда – в частный сектор у самой линии разграничения. 

"Я на самой линии сейчас живу, до украинских позиций метров четыреста. На нашей
улице никого больше нет, мы одни. Стреляют каждый день: утром, днем, вечером –
когда угодно. Минометами, зушками (зенитная установка калибра 23 мм ЗУ-23-2 –
примечание ЛИЦ), пулеметами, гранатометами. Сволочи они, что еще можно сказать,
если они даже детей убивают спокойно…", - рассказал хозяин. 

Он отметил, что с приходом к власти на Украине президента Владимира Зеленского
военная обстановка, вопреки ожиданиям, не изменилась. 

"До этого (президент Украины Петр) Порошенко лупил по нам, теперь Зеленский", -
констатировал Кислица. 

Юрий Булавко уже два года живет с 79-летним отцом на съемной квартире в центре
города. Его домостроение, расположенное на окраине, регулярно обстреливается,
превратившись после двух попаданий РСЗО "Град" практически в нежилое. От
нервного напряжения мужчина начал лысеть. 

"Мы уже терпели, терпели, но после 3 октября 2018 года, когда в наш дом был
второй прилет "Града", вынуждены были уехать. Первый раз "Град к нам прилетел"
7 декабря 2014 года. Мы ушли, а следом за нами все жители нашей улицы, там
сейчас вроде только два человека остались", - сказалз он. 

Галина Великая, имевшая в довоенный период опыт работы в Луганском
энергетическом объединении, сама скручивает поврежденные обстрелами провода
линии электропередачи на приграничных улицах. 

"Находимся мы от украинской армии в двухстах метрах, обстрелы каждый день как
"доброе утро" и "спокойной ночи". У всех в поселке стекла в окнах побиты, у
всех разбиты крыши. Как жители поселка мы хотим обратиться в международный суд,
хочется верить, что это поможет повлиять на Киев, и он перестанет нас убивать",
- выразила она надежду. 



ЛуганскИнформЦентр
  
