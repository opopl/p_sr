% vim: keymap=russian-jcukenwin
%%beginhead 
 
%%file 10_11_2020.fb.denis_zharkih.3.moi_principy
%%parent 10_11_2020
 
%%url https://www.facebook.com/permalink.php?story_fbid=2851721598374572&id=100006102787780
%%author 
%%tags 
%%title 
 
%%endhead 

\subsection{Серия "Мои принципы" Принцип шестой. "Ради какой женщины стоит жить?"}
\label{sec:10_11_2020.fb.denis_zharkih.3.moi_principy}
\Purl{https://www.facebook.com/permalink.php?story_fbid=2851721598374572&id=100006102787780}
\Pauthor{Жарких, Денис}

Для этого женщина должна обладать следующими качествами:

\begin{itemize}
\item 1. Быть благодарной.

Каждая женщина хочет чтобы ее мужчина:

\begin{itemize}
  \item - совершал подвиги;
  \item - зарабатывал золото;
  \item - возился с детьми;
  \item - делал успехи на их ложе любви;
  \item - обсуждал с ней прошедший день;
  \item - работал по хозяйству;
  \item - многое другое....
\end{itemize}

И только благодарная женщина понимает, что он не может все это делать одновременно.

\item 2. Быть спокойной.

Каждую женщину посещают следующие мысли:

\begin{itemize}
  \item - А достоин ли он меня?
  \item - А достаточно ли он уделяет мне внимания?
  \item - А не слишком ли  дешево я ему досталась?
  \item - А осталась ли у нас любовь или у нас уже привычка?
  \item - А не слишком ли он пристально смотрит за другими женщинами?
  \item - многое другое...
\end{itemize}

И только спокойная женщина отмахнется от этих мыслей и скажет: "Да ладно! Живем же хорошо, чего Бога гневить?"

\item 3. Полюбить - так королеву!

Все женщины в определенном возрасте принцессы. Они ждут своих рыцарей в замках
своих душ. Но не все после того, как их оттуда освободят становятся королевами.
Кто-то ведьмами. Кстати, третьей роли нет. Принцесса с возрастом становится
либо королевой, либо ведьмой. Так вот, королева относится к своему мужчине, как
к королю. Не как раба, не как подданная, а как королева, равная. И делает из
рядового рыцаря короля. Пусть империю они не покорят, но в своем доме он вполне
может быть королем, а она королевой. И жить вместе в уважении и достоинстве.
Для этого нужны первые два качества. И если мужчина будет чувствовать у себя в
доме королем, да еще не в одиночестве, а с королевой, то он посвятит этому всю
свою жизнь.
	
\end{itemize}

Думаю, что эти качества необходимы для счастливой совместной жизни. Но они и
достаточны. Их же всего три!  2013
