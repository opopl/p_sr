% vim: keymap=russian-jcukenwin
%%beginhead 
 
%%file slova.jug
%%parent slova
 
%%url 
 
%%author 
%%author_id 
%%author_url 
 
%%tags 
%%title 
 
%%endhead 
\chapter{Юг}

%%%cit
%%%cit_pic
%%%cit_text
\emph{Южные русские}, которые стали называться малороссами, никогда не
испытывали таких трудностей, они не зависели от погоды, потому и имели характер
чересчур расслабленный. Постепенно они бросали занятия не прибыльным сельским
хозяйством, предпочитая поступать на воинскую службу или к полякам, или к своим
феодалам, или образовывать разбойничьи отряды и даже армии. Одним из таких
подразделений, жившим за счет набегов на татар, турок, поляков и прочих соседей
была Запорожская Сечь. Когда в середине XVII века, после Переяславской Рады,
северные русские пришли на \emph{Южную Русь}, которая называлась Гетманщиной,
они увидели полное разорение
%%%cit_title
\citTitle{Почему великороссы стали такими сильными, а малороссы остались такими слабыми?}, 
Исторический Понедельник, zen.yandex.ru, 11.06.2021
%%%endcit

