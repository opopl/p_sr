% vim: keymap=russian-jcukenwin
%%beginhead 
 
%%file 04_08_2020.fb.fb_group.story_kiev_ua.1.166_let_zanjkoveckaja.cmt
%%parent 04_08_2020.fb.fb_group.story_kiev_ua.1.166_let_zanjkoveckaja
 
%%url 
 
%%author_id 
%%date 
 
%%tags 
%%title 
 
%%endhead 
\subsubsection{Коментарі}

\begin{itemize} % {
\iusr{Григорий Теплюк}
Читаєм
Історію України. Українські
Знаменитості.

\iusr{Ирина Петрова}
\textbf{Григорий Теплюк} це гарна справа!

\iusr{Ирина Петрова}
\textbf{Марію Костянтинівну} поховано на центральній алеї Байкового кладовища, поряд з Миколою Садовським.

\iusr{Любовь Бодня}

Забыли написать, что Дом Музей Заньковецкой заставила отстроить и вернуть все
как было при жизни актрисы - Раиса Горбачева !!!

\begin{itemize} % {
\iusr{Ирина Петрова}
можна попросити у Вас посилання на першоджерело такої інформації? Дякую.

\iusr{Любовь Бодня}
\textbf{Ирина Петрова} 

Можно !!! Сходите в Музей, а не торчите в Интернете и вам там расскажут. А для
таких как вы "особо одаренных" отвечаю !!! Дом в котором жила Заньковецкая был
снесён так как был близко к проезжей части ул Красноармейской, вот тогда Р.
Горбачева дала указание по чертежам давше в скверике восстановить дом
Заньковецкой, в котором и находится Музей в котором сохранены ее вещи и
предметы обихода, портреты с друзьями!!! Вопросы ещё есть ???

\iusr{Ирина Петрова}
\textbf{Любовь Бодня} 

про-перше, тон можна знизити. По-друге - я товаришую з усіма співробітниками
музею вже понад 16 років, по-третє... але, не роблю собі честі відповідати на
таке поводження. А ось вам рекомендую у музеях вчитись чемному веденню бесіди.
Вже побачила вашу сторінку - в мене немає жодного питання... тільки співчуття до
хворих святошинського закладу....

\iusr{Виктория Цьопа}
\textbf{Ирина Петрова} о! То це троляка ватна. Там і одноглазнікі, і вдв, і сєвастопаль. От зовсім не здивована таким тоном коментаря.

\iusr{Alik Perlov}
\textbf{Ирина Петрова} Правильно Ирино!!! Треба Вчити таких знатокив Хабальних((( Тьфу на неи)))))

\iusr{Ирина Петрова}
\textbf{Виктория Цьопа} так ото ж тому одразу перервала спілкування... що такі шукають у нашій темах??? Не можу збагнути...

\iusr{Ирина Петрова}
\textbf{Alik Perlov} ти ж мій друже, @igg{fbicon.heart.eyes} 

\end{itemize} % }

\iusr{Ирина Петрова}
Народна артиска України Лариса Кадирова в ролі Марії Заньковецької

\ifcmt
  ig https://scontent-frx5-1.xx.fbcdn.net/v/t1.6435-9/117196343_3475918752441707_1710945375551257958_n.jpg?_nc_cat=110&ccb=1-5&_nc_sid=dbeb18&_nc_ohc=LS5yCpyQ54AAX8d8ce_&_nc_ht=scontent-frx5-1.xx&oh=60033e9f2978f9a8e2df72f2b586b66c&oe=61B776F6
  @width 0.3
\fi

\begin{itemize} % {
\iusr{Людмила Шевченко}

Я хочу сказать об актрисе Ларисе Кадыровой , исполнявшей роль Марии
Заньковецкой. Года полтора назад была в театре Франка, смотрела моноспектакль
"Не плачте за мною ніколи". Грустная пьеса. Играла Лариса Кадырова. Ведь она
уже в годах, а исполнение - выше всех похвал. Талант + опыт театральной игры =
блестящий результат. Может пьеса и не всем понравится (Автор Мария Матиос), но
если вы хотите увидеть НАСТОЯЩУЮ ТЕАТРАЛЬНУЮ ИГРУ, настоящий спектакль - не
пожалейте времени. денег и пойдите. Кстати, родом она из Узбекистана, а
владеет украинским языком непревзойденно.

\end{itemize} % }

\iusr{Светик Мороз}

Сегодня на канале ИНТЕР об этой роли Р. Горбачевой было упомянуто. Молодец!!! И
все благодаря тому, что отец Р. Горбачевой был родом с Черниговской области,
т. е. земляк Марии Заньковецкой.

\begin{itemize} % {
\iusr{Ирина Петрова}

цей канал не дивлюсь. Я обов'язково спитаю про це у співробітників музею, з
якими товаришуємо вже понад 16 років... але таке твердження чую вперше. Зараз,
до впевненності у цьому питанні, не коментую. Але, чесно кажучі, канал той для
мене - не авторитетне джерело(

\end{itemize} % }

\iusr{Людмила Краснюк}

Возмущает перевод текста на русский язык. В самом первом абзаце: вместо
«корифей украинской сцены» написано «корановирус украинской сцены» ганьба!!!

\begin{itemize} % {
\iusr{Ирина Петрова}
про який переклад іде мова? поясніть, будь ласка!

\iusr{Людмила Краснюк}
\textbf{Ирина Петрова} прочитайте уважно!

\ifcmt
  ig https://scontent-frt3-2.xx.fbcdn.net/v/t1.6435-9/117090580_2954893977973052_8912708888378719282_n.jpg?_nc_cat=103&ccb=1-5&_nc_sid=dbeb18&_nc_ohc=QZ0EYMzdymUAX8TotSw&_nc_ht=scontent-frt3-2.xx&oh=1b526e0d3ed7237c932c72423b659664&oe=61B499BB
  @width 0.3
\fi

\iusr{Светлана Манилова}
\textbf{Людмила Краснюк}, автор поста к этому переводу не имеет никакого отношения...

\iusr{Олег Коваль}
\textbf{Людмила}, эту претензию Вам нужно предъявить администрации фейсбука, а не
автору. Автор статьи к переводу текста не имеет никакого отношения.

\iusr{Игорь Мезецкий}

Якщо на сторінці у Вашому браузері стоїть "мова оригіналу", - текст буде
відтворюватися в оригінальному вигляді. Якщо Ви робите переклад на БУДЬ-ЯКУ
іншу мову, Ви повинні змиритися з тим, що дуже часто перекладений текст може
досить сильно відрізнятися від оригінального. Що стосується перекладу з
української на російську мову, то у мене взагалі складається думка, що
перекладачі не просто навіть не намагаються знайти підходяще за змістом слово,
а навпаки прикладають усі сили, щоб викривити, спаплюжити зміст
російськомовного перекладу. А зважаючи на те що зараз переважну більшість
миттєвих перекладів здійснюють програми-перекладачі, напрошується висновок, що
в ці програми спеціально закладений неправильний алгоритм. Сподіваюсь, що я
достатньо зрозуміло пояснив і, в даному випадку, будь-які претензії до автора
просто безпідставні...

\iusr{Людмила Краснюк}
\textbf{Игорь Мезецкий} Дякую! Саме це я мала на увазі.

\iusr{Ирина Петрова}

В мене ніякого перекладу видно не було. Коли я хочу писати російською - пишу,
коли українською - також. Переклад роблять роботи, до них претензії висувати
марно) все пояснили пан Олег, пані Світлана, пан Ігор, за що щиро вдячна)

\end{itemize} % }

\iusr{Светик Мороз}
Ваши проблеми

\iusr{Ирина Петрова}
це не проблеми, я відповіла на ваш комент. В групі вітається шановне поводження.

\iusr{Михайло Савченкл}

З діяльністю М. К. Заньковецької пов'язаний Ніжин на Чернігівщині, де в місцевому
театрі грала легендарна актриса, й де М. Заньковецькій є пам'ятник, а селі. Заньки
- музей славетної землячки...

\iusr{Михайло Савченкл}

\ifcmt
  ig https://scontent-frx5-2.xx.fbcdn.net/v/t39.1997-6/s480x480/51165322_379255722625305_729354346726162432_n.png?_nc_cat=1&ccb=1-5&_nc_sid=0572db&_nc_ohc=dIGZvlAx-q0AX9F9bWD&_nc_ht=scontent-frx5-2.xx&oh=27d9b09ea40ecd3b98a6ec1cd19253ba&oe=61966BFE
  @width 0.2
\fi

\iusr{Наталия Трач}
навік поруч

\ifcmt
  ig https://scontent-frt3-1.xx.fbcdn.net/v/t1.6435-9/117113931_3212882762080766_2091186567189868227_n.jpg?_nc_cat=106&ccb=1-5&_nc_sid=dbeb18&_nc_ohc=oP3uqpHFetIAX8r-N_F&_nc_ht=scontent-frt3-1.xx&oh=11395f83d4a4078e856d94b26912903b&oe=61B7A0CB
  @width 0.3
\fi

\begin{itemize} % {
\iusr{Ирина Петрова}
\textbf{Наталия Трач} дуже дякую за світлину! @igg{fbicon. heart. eyes}  так... смерть знов звела цих непересічних чоловіка та жінку, Акторів, митців... хай спочивають з миром...
\end{itemize} % }

\iusr{Olena Klymenko}

\ifcmt
  ig https://scontent-frx5-2.xx.fbcdn.net/v/t39.1997-6/p370x247/69142145_2155530931224624_1312636185700466688_n.png?_nc_cat=1&ccb=1-5&_nc_sid=0572db&_nc_ohc=SbmcEmwW5LMAX9lDREM&tn=lCYVFeHcTIAFcAzi&_nc_ht=scontent-frx5-2.xx&oh=8edfeb5e7dca47fb6cd05ee63a789132&oe=61966FE8
  @width 0.15
\fi

\iusr{Катерина Макарова}

Листівка із зображенням Марії Заньковецької, видана у Стокгольмі в 1911 році.

\ifcmt
  ig https://scontent-frx5-2.xx.fbcdn.net/v/t1.6435-9/117107424_2722323858050194_966948911708362704_n.jpg?_nc_cat=109&ccb=1-5&_nc_sid=dbeb18&_nc_ohc=g2-z2Wau57MAX9Mt4UT&_nc_ht=scontent-frx5-2.xx&oh=89a8c1757eb29845b4c09f989a5fca5f&oe=61B608EF
  @width 0.3
\fi

\begin{itemize} % {
\iusr{Ирина Петрова}
\textbf{Kateryna Makarova} дякую!!!! Рідкісна листівка, я такої не бачила! Чи можна собі взяти до архіву? Буду вдячна.

\iusr{Катерина Макарова}
\textbf{Ирина Петрова} так, звичайно. Я взяла її на сторінці діаспори

\iusr{Ирина Петрова}
\textbf{Kateryna Makarova} щиро дякую!
\end{itemize} % }

\iusr{Oksana Shukayeva}

\ifcmt
  ig https://scontent-frx5-2.xx.fbcdn.net/v/t39.1997-6/s168x128/93025159_222645582401279_8207007965157261312_n.png?_nc_cat=1&ccb=1-5&_nc_sid=ac3552&_nc_ohc=cx_4kk5_uMYAX9UuFn2&_nc_ht=scontent-frx5-2.xx&oh=8ee9579d045329a7efe36dbc26880996&oe=6195CC82
  @width 0.1
\fi

\iusr{Volodimir Obolonsky}

Пам'ятаю я колимив у 80-х, iхав по Червноармiйськiй, будинок горiв. Потiм
думали що гебня пiдпалила.

\iusr{Ирина Петрова}

\ifcmt
  ig https://scontent-frt3-2.xx.fbcdn.net/v/t1.6435-9/117295662_3477125052321077_3147414866155394601_n.jpg?_nc_cat=101&ccb=1-5&_nc_sid=dbeb18&_nc_ohc=kj_FKjRtkWkAX_Kh8_X&_nc_ht=scontent-frt3-2.xx&oh=6778c671bc618ea632e0dd1a6359bb0d&oe=61B601DD
  @width 0.4
\fi

\iusr{Ирина Петрова}

Вистава "Марія" (режисер Збігнєв Хшановський). В ролі Марії Заньковецької
Народна артистка України, лауреат Шевченківської премії Лариса Кадирова. Сцена
фіналу вистави.

\begin{itemize} % {
\iusr{Любовь Бодня}

Украинский писатель Юрий Михайлович Хорунжий, написал очень интересную книгу о
творчестве М. Заньковецкой и о большой любви Марии и Николая Садовского, о
Киевском Театре на ул. Саксаганского. Книга называется "Садовский садит сад "
Прочтите на досуге !!!

\end{itemize} % }

\iusr{Ирина Слободенюк}

\ifcmt
  ig https://scontent-frx5-2.xx.fbcdn.net/v/t39.1997-6/s168x128/16781161_1341101952618574_7704631035023065088_n.png?_nc_cat=1&ccb=1-5&_nc_sid=ac3552&_nc_ohc=7VVbYisllDsAX9tnMTI&_nc_ht=scontent-frx5-2.xx&oh=3b8f94f2f3c31915504b6fa47591d770&oe=619666AB
  @width 0.1
\fi

\end{itemize} % }
