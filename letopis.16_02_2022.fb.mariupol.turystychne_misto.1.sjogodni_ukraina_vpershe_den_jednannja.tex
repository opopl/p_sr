%%beginhead 
 
%%file 16_02_2022.fb.mariupol.turystychne_misto.1.sjogodni_ukraina_vpershe_den_jednannja
%%parent 16_02_2022
 
%%url https://www.facebook.com/mistoMarii/posts/pfbid0eszZ4jHMmnwEe7y6yB2rGCwLSa8mJEiEqRgtttfjapvmcwApss6jNYkxu4nxVb7Pl
 
%%author_id mariupol.turystychne_misto
%%date 16_02_2022
 
%%tags 
%%title Сьогодні Україна вперше відзначає День єднання
 
%%endhead 

\subsection{Сьогодні Україна вперше відзначає День єднання}
\label{sec:16_02_2022.fb.mariupol.turystychne_misto.1.sjogodni_ukraina_vpershe_den_jednannja}

\Purl{https://www.facebook.com/mistoMarii/posts/pfbid0eszZ4jHMmnwEe7y6yB2rGCwLSa8mJEiEqRgtttfjapvmcwApss6jNYkxu4nxVb7Pl}
\ifcmt
 author_begin
   author_id mariupol.turystychne_misto
 author_end
\fi

Сьогодні Україна вперше відзначає День єднання.

Маріуполь за певних обставин став містом, яке близько торкнулося непривабливих
реалій подій після 2014-го року й, виборовши свою свободу та незалежність у
складі України, отримало друге життя, яке жителі по-справжньому цінують разом з
миром.

Єднання та обмін досвідом стали одною з цілей, якої прагне Маріуполь. Кілька
років тому культурно туристичний центр \enquote{Вежа} започаткував проєкт \#ТутВарто
дружити містами, де ми порівнювали, здавалося б, полярні міста України, шукали
й знаходили спільне, захоплювалося відмінностями, які надихають.

2021-й рік для Маріуполя також видався знаковим, адже упродовж року в рамках
проєкту \enquote{Велика культурна столиця} за підтримки Українського культурного фонду
було реалізовано безліч подій й фестивалей. 

Вони показали та \enquote{притягнули} до нас іншу культуру від сучасних діячів України
(й не тільки) та дозволили розкритися у своїх задумах, а також бути почутими на
широку публіку талановитим маріупольцям. З усієї країни з'їжджалася зацікавлені
аби подивитися чи послухати перформанси від українців для українців. Так місто
Марії стало не тільки культурною столицею, а й взірцем позитивних змін у єдиній
країні.

Новий проєкт, який покликаний \enquote{зшити} країну -  \enquote{Маріуполь Україні. Україна
Маріуполю}. Це історія про культурний обмін та власне єднання різних міст. Було
складено перелік тринадцяти міст різних регіонів України, із якими потенційно
було б цікаво стати ближче у культурній площині. До списку увійшли: Вінниця,
Умань, Івано-Франківськ, Полтава, Біла Церква, Луцьк, Харків, Бердичів, Острог,
Одеса, Чернівці, Ужгород та місто Лева.

Кожен з вас може взяти учать у виборах п'ятірки кращих претендентів на цей рік,
проголосувавши на сайті Маріупольської міської ради за посиланням. Переможці
голосування візьмуть участь у спільних подіях навколо розташування
арт-об'єктів, які інтерактивно та у незвичайний спосіб розповідатимуть про своє
місто у інших та навпаки.

Віддати свій голос можна тут: \url{https://bit.ly/3gSnULO}
