% vim: keymap=russian-jcukenwin
%%beginhead 
 
%%file 14_07_2021.fb.krjukova_svetlana.1.statja_putina_mnenie
%%parent 14_07_2021
 
%%url https://www.facebook.com/kryukova/posts/10159535705273064
 
%%author Крюкова, Светлана
%%author_id krjukova_svetlana
%%author_url 
 
%%tags __jul_2021.putin.statja,future,istoria,malorossia,narod,odin_narod,rossia,strana,ukraina
%%title Крюкова - Мнение - Статья Путина
 
%%endhead 
 
\subsection{Крюкова - Мнение - Статья Путина}
\label{sec:14_07_2021.fb.krjukova_svetlana.1.statja_putina_mnenie}
\Purl{https://www.facebook.com/kryukova/posts/10159535705273064}
\ifcmt
 author_begin
   author_id krjukova_svetlana
 author_end
\fi
 
В громкой дискуссии о личной статье Путина о «едином народе» - выскажу и свое
скромное мнение на этот счет.

Я честно и внимательно его прочла, борясь со сном и спотыкаясь о философские
обороты. А потом честно и терпеливо ждала ответ от Зеленского. Но нет. 

Позволю своё, женское мнение на этот скучный опус кремлевских придворных
историков, махровую пропаганду, зачем-то подписанный фамилией «Путин».

Разложим по сути тезисов. 

Очень понравился пассаж историков Путина про то, что сосед в равной мере с нами
является наследниками Древней Руси. В то время как Киевские князья уже давно
устанавливали дани для Новгорода и Пскова, Мурома и северного Кавказа,
роднились с королевскими династиями Франции, Германии, Венгрии (нынешнего
Евросоюза, который про те союзы и продолжительные эпизоды совместной истории
как-то нынче не сильно и вспоминает), Норвегии, Византии, строили
величественную Святую Софию и Золотые ворота, Михайловский Златоверхий
монастырь, Десятинную церковь и Лавру, писали книги и создавали библиотеку –
все это время столица нынешнего «величия» была сплошным непроходимым болотом и
топями. Которые начали превращаться в город и укрепляться после переноса
кафедры Киевского митрополита Петра из Владимира в Москву в 1325 году. И при
этом гордые «величием» – не забывали регулярно платить дань Золотой Орде, а
Крымским ханам – вплоть до 1700 года.

Про «культурную столицу» на территориях, отжатых у скандинавов, вообще молчу.
Ей до рождения оставалось еще несколько столетий.

Путин записал в ряды Руси всех, кого смог только, запретендовал на монополию на
православие, попенял «ополячивание», «латинизацию» и Святой Престол. А нас,
украинцев», обьявил пограничниками.

Отдельного внимания стоит образчик кондового советского стиля, списанный из
учебников по истории КПСС – «реставрировались старые порядки, усилился
социальный и религиозный гнёт».

Сколько неприкрытого презрения во фразе – «Малороссы во многом и созидали
большую общую страну, её государственность, культуру, науку. Участвовали в
освоении и развитии Урала, Сибири, Кавказа, Дальнего Востока. Кстати, и в
советский период уроженцы Украины занимали самые значимые, в том числе высшие
посты в руководстве единого государства. Достаточно сказать, что в общей
сложности без малого 30 лет КПСС возглавляли Н. Хрущёв и Л. Брежнев, чья
партийная биография была самым тесным образом связана с Украиной».


И зависти, откровенной зависти и неготовности признать, что пока они грезили
захватами территорий и очередными кровавыми походами, а в Советском Союзе не
были способны оформить хотя бы некоторые элементы «квази-автономности», которые
были и Украинской и еще 14 советских республик, мы придерживались той
концепции, которой тысячелетиями придерживался и придерживается древний и
нынешний Китай – не «прирастать территориями», а заниматься своими, строить
мирные и дружеские отношения с соседями, думать о будущем и работать над ним.

Путин и сегодня живет в парадигме «делить и приватизировать» историческое,
культурное пространство. В нем, или в его писарях, до сих пор живет гигантское
ностальжи по СССР, где были космос, наука и промышленность. Но там была и
насильственная коллективизация и голод, десятилетиями процветала тотальная
бедность, селяне (а их было до 40\% населения СССР) до августа 1974 года вообще
не имели паспортов и более полстолетия «народной власти трудящихся» были
фактически крепостными рабами при колхозах, работая за «трудодни», где была
недоступна свобода путешествий по миру. А отсутствие партийного билета сразу
закрывало перед тобой практически все карьерные перспективы. Где был народ (а
точнее сотни народов со своей собственной самоидентичностью), а была
партийно-государственная номенклатура, которая имела все, что хотела. Много
чего там было, хорошего и нехорошего.

Главный путинский тезис, ради которого и затевался этот длинный
исторически-философский очерк – «современная Украина – целиком и полностью
детище советской эпохи. Мы знаем и помним, что в значительной степени она
создавалась за счёт исторической России. Очевидно одно: Россия фактически была
ограблена… Другими словами – (государства, образованные на базе союзных
республик) уходите с тем, с чем пришли. С такой логикой трудно спорить».

Сегодня он пугает мир, что мы, украинцы, готовы были делать этнические чистки,
что мы –  «неонацисты», готовые чинить расправы в Крыму, Севастополе, Донецке и
Луганске, и только и ждем своего часа – но не дождемся.

Это завуалированное повторение уже не раз звучавших красных линий для Украины.

Но нас это не должно пугать. Да, у нас множество проблем, многие из которых
будет сложно решить на скорую руку в ближайшие годы.

Но у нас есть ЕДИНЫЙ НАРОД, люди разного происхождения, разного
вероисповедания, представители сотен национальностей, приехавших уже в
независимую Украину из десятков стран мира. Избравшим ее местом, в котором они
добровольно захотели поселиться, жить, учиться, рожать детей, любя и уважая
своих соседей. И у нас среди них не было и нет этнических конфликтов даже в тех
случаях, когда на своей исторической родине или в стране происхождения между
ними доходило до геноцида и гражданских войн.

Мы не издаем указы с «крайними сроками» для мигрантов, не разгоняем торговые
ряды дубинками «росгвардии», не навязываем свои паспорта как условия для
пользования общечеловеческими правами и свободами.

Будучи недовольны властью, мы научились ее менять и мы точно не дадим никому не
приклеиться к президенсткому стулу на четверть века, превратив некогда
уважаемую демократическую и порядочную страну в государство-изгоя.

Мы должны все вместе дружно работать, чтобы сделать нашу страну успешной. Не
полагаясь на власть, а прежде всего на свои собственные силы. Мы обязательно
станем лучше, если каждый из нас на своем рабочем месте, будь то сталевар или
хлебороб, учитель или медик, журналист или спортсмен каждый день по шагу будет
работать над собой и двигаться по пути достижения профессионального и
личностного совершенства. Это единственный путь - благосостояние и
профессионализм, образование и безопастности, здесь и начиная с сегодняшнего
дня. 

И это не какой-то наш уникальный рецепт. За примерами далеко ходить не надо.
Где был всего полстолетия Китай, про который в те времена говорили, что он весь
работает за горстку риса? Сегодня – наряду с США – крупнейшая экономика мира.
Еще полстолетия назад Южная Корея была беднейшей аграрной страной, где местные
женщины обязаны были сдавать свои волосы, чтобы наполнять бюджет? Сегодня –
также одна из ведущих экономик мира. Сингапур на заре независимости был
скоплением бедных рыболовецких поселений. Посмотрите на фотографии аравийского
полуострова полустолетней давности…

Но они не хотели оставлять все как есть. Упорно трудились и достигли наконец
желанного результата.

У нас сегодня нет выбора. Вариантов два. Первый – реализовать «голубую мечту»
Путина и сдаться, но это не наш путь. 

Второй – и единственно возможный – идти вперед, зная, что у нас всегда будет
тот, кто при каждом удобном случае будет вставлять палки в колеса. И не только
снаружи, но и внутри страны, рассчитывая на свои определенные политические
дивиденды.

Мы не строим проект «анти-Россия». Мы строим проект «Украина».

И мы понимаем, насколько высоки ставки писарей от Путина, чтобы нам помешать.
Им страшно потому, что проект «Успешная Украина» похоронит проект «нынешняя
Россия». Примерно так, как проект «успешная Европа / Америка» развалил проект
нищего и отсталого, но при этом громко бряцающего ядерным и другим оружием
СССР.

И в этом случае обязательно наступит время, когда в самой России начнут
задавать вопросы к тем, кто сегодня учит истории и жизни нас. Почему есть
купающиеся в золоте Москва и Питер, а есть все остальные регионы? 

Почему «алмазная Якутия» и «нефтегазовая Сибирь» не живут и близко как
государства Аравийского полуострова, хотя нами тоже практически единолично
управляет почти монарх? 

Почему уборщица «Газпрома» ездит на Mitsubishi Outlander и не расстраивается
краже сумочки Dior за 2 миллиона рублей, а средняя зарплата в Кемерово и
Новосибирске не дотягивает и до 18 тысяч рублей? 

Зачем российская власть время от времени давит бульдозерами турецкие помидоры и
французские сыры? Скульптура петуха из навоза – это все, что мы можем
предложить в области современного искусства? 

Почему когда весь мир скорбит от жертвах мировых войн мы должны устраивать
приступы «победобесия» и «может повторить»?

Когда это произойдет, это станет началом конца для собирателей земель и их ныне
плохо скрываемых имперских амбиций. 

И обязательно «вернутся в родную гавань» финская Карелия и германский
Кенигсберг, молдавское Приднестровье и украинский Крым, грузинские Абхазия и
Северная Осетия, казахская Оренбург и Астрахань, китайские Приамурье и
Приморье, японские Карафуто (Сахалин) и Курильские острова.

