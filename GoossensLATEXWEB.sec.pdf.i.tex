%%page page_47                                                  <<<---3
 
\isec{Portable_Document_Format}{Portable Document Format}

For some applications, the methods employed to disseminate information
across the World Wide Web are unacceptable. This is because they leave
the rendering of the ``page'' to the reader's software, not to the
author's software. Even if pure HTML and Cascading Style Sheets are
used, the author does not know where line breaks will occur, and, of
course, there is no concept of ``page breaks.'' Graphics are often
presented as low-resolution bitmaps with unreliable colors; table layout
may be radically different. The author cannot even be sure which font
will be seen by the reader, or whether some unsuitable symbols might be
used in mathematics, for example. Finally, and perhaps most important,
the current generation of Web browsers is not very sophisticated at
typesetting and page makeup; the result of hitting the Print icon from a
browser does not produce a high-quality result. 

Who cares about these issues? On the one hand, lawyers may regard it vital that 
an electronic document is \emph{exactly} the same as the traditional printed copy, down to 
the line breaks. On the other hand, it might simply be that an author has spent 
a lot of time making a beautiful page and wants it to be seen as such. In between 
are applications where HTML output is simply not very ``nice,'' such as for very 
complex forms, tables, and mathematical material. 

This chapter, set up in three parts, describes a solution --- Portable
Document Format (PDF). In the first part we take a general look at what
PDF is, and what are the issues in creating it using \ \TeX. In the
second part, we describe a special \LaTeX\ package (\verb|hyperref|) that
enables you to make enhanced PDF documents from \LaTeX\  source, using a
variety of back-end drivers and a high-level interface to hypertext
commands.  
%%page page_48                                                  <<<---3

The final part of the chapter describes, in some detail, a special
version of \ \TeX\ that generates PDF. Many readers, perhaps most, will
not need to understand the new PDF primitives added to this version of 
\TeX, since packages like hyperref provide a familiar \LaTeX\ interface.
However, \pdfTEX\ offers tremendous possibilities for producing advanced
interactive electronic documents, and confident \ \TeX\ programmers will
want to understand what is going on under the hood. 

\iii{2_1_What_is_PDF}
\iii{2_2_Generating_PDF_from_TeX}
\iii{2_3_Rich_PDF_with_LaTeX_The_hyperref_package}
\iii{2_4_Generating_PDF_directly_from_TeX}
