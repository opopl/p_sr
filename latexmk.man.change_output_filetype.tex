% vim: keymap=russian-jcukenwin
%%beginhead 
 
%%file man.change_output_filetype
%%parent man.errors
 
%%endhead 
\section{ALLOWING FOR CHANGE OF OUTPUT FILE TYPE}
  
\vspace{0.5cm}
 {\ifDEBUG\small\LaTeX~section: \verb|man.change_output_filetype| project: \verb|latexmk| rootid: \verb|p_saintrussia| \fi}
\vspace{0.5cm}

When one of the latex engines is run, the usual situation is that latex
produces  a .dvi file, while pdflatex and lualatex produce a .pdf file.  For
xelatex the default is to produce a .pdf file, but to optimize processing time
latexmk  runs xelatex its -no-pdf option so that it produces an .xdv file.
Further processing by  latexmk  takes  this  as  a starting point.

However, the actual output file may differ from the normal expectation; and
then latexmk can adjust its processing to accommodate  this  situa- tion.   The
difference in output file type can happen for two reasons: One is that for
latex, pdflatex and lualatex the  document  itself  can override  the defaults.
The other is that there may be a configuration, or misconfiguration, such that
the program that latexmk invokes to com- pile  the  document is not the
expected one, or is given options incom- patible with what latexmk initially
expects.  (E.g.,  the  -output-for- mat=...   option  could  be used with
lualatex, or xelatex gets invoked without the -no-pdf option.)

Under latex and pdflatex, control of the output format by the  document is
done by setting the \verb|\pdfoutput| macro.  Under lualatex, the
\verb|\outputmode| macro is used instead.

One example of an important use-case for document control of the output format
is  a document that uses the psfrag package to insert graphical elements in the
output file. The \verb|psfrag| package achieves its effects by inserting
postscript code in the output of the compilation of the document.  This entails
the use of compilation to a .dvi file, followed  by the  use  of  conversion
to  a postscript file (either directly, as by dvips or implicitly, as an
intermediate step by dvipdf).   Then  it  is useful  to  force output to be of
the .dvi format by inserting \verb|\pdfoutput=0| in the preamble of the
document.

Another example is where the document uses graphics file of  the  .pdf, .jpg,
and  png types.  With the default setting for the graphicx package, these can
be processed in compilation to \verb|.pdf| but not with  compilation  to  .dvi.  In
this case, it is useful to insert \verb|\pdfoutput=1| in the preamble of the document
to force compilation to .pdf  output  for- mat.

In all of these cases, it is needed that latexmk has to adjust its pro-
cessing to deal with a mismatch between the actual output  format  (out
of  .pdf,  .dvi,  .xdv) and the initially expected output, if possible.
Latexmk does this provided the following conditions are met.

The first is that latexmk's \verb|$allow_switch| configuration variable is set to  a
non-zero value as it is by default.  If this variable is zero, a mismatch of
filetypes in the compilation results in an erro.

The second condition for latexmk to be able to handle a change of  output  type
is that no explicit requests for .dvi or .ps output files are made.   Explicit
requests  are  by  the  -dvi  and  -ps,   -print=dvi, -print=ps,  -view=dvi,
and -view=ps options, and by corresponding settings of the \verb|$dvi_mode|,
\verb|$postscript_mode|, \verb|$print_type|, and  \verb|$view|  configuration
variables.   The print-type and view-type restrictions only apply when printing
and viewing are explicitly requested, respectively.  For  this  purpose,  the
use of the -pdfdvi and -pdfps options (and the corresponding setting of the
\verb|$pdf_mode| variable) does not count  as  an explicit  request  for the
.dvi and .ps files; they are merely regarded as a request for making a .pdf
file together with an initial  proposal for the processing route to make it.

Note that when accommodating a change in output file type, there is involved a
substantial change in the network of rules that  latexmk  uses in  its
actions.  The second condition applied to accommodate a change is to avoid
situations where the change in the rule network is too radical to be readily
handled automatically.
  
