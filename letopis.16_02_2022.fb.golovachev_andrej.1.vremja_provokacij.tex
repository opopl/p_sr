% vim: keymap=russian-jcukenwin
%%beginhead 
 
%%file 16_02_2022.fb.golovachev_andrej.1.vremja_provokacij
%%parent 16_02_2022
 
%%url https://www.facebook.com/permalink.php?story_fbid=2804118296564597&id=100008993618796
 
%%author_id golovachev_andrej
%%date 
 
%%tags napadenie,provokacia,rossia,ugroza,ukraina,usa
%%title Время провокаций
 
%%endhead 
 
\subsection{Время провокаций}
\label{sec:16_02_2022.fb.golovachev_andrej.1.vremja_provokacij}
 
\Purl{https://www.facebook.com/permalink.php?story_fbid=2804118296564597&id=100008993618796}
\ifcmt
 author_begin
   author_id golovachev_andrej
 author_end
\fi

Время провокаций

Для меня совершенно очевидно, что масштабная истерия, которую устроили США и
Англия, указание ими  точной даты и  даже часа (!!) начала агрессии и
показательное бегство посольств из Киева одной из целей ( но  далеко не
единственной) имело принуждение Зеленского к объявлению всеобщей мобилизации и
введению военного положения. Безусловно это была масштабная провокация со
стороны США, потому что Россия могла использовать всеобщую мобилизацию как
казус Белли.

К чести Зеленского  он   не поддался на провокацию и даже потроллил
Байдена, пригласив его в Украину именно на 16.02.

Байден, конечно, это Зеленскому не простит.

Многие ошибочно считают, что причиной войн является желание стран захватить
экономические ресурсы. Это неверно. Большинство войн начинаются по причине
внутренних кризисов. И Россия и Западный мир испытывают сейчас серьезные
внутренние противоречия.

В России подходит к концу период легитимности режима Путина и лучшим выходом из
кризиса, по мнению правящего чекистского режима, был бы впечатляющий разгром
Украины  с последующим присоединением ее к союзу славянских государств.   Но
необходимо создать впечатление у российского общества, что война эта
справедливая и вынужденная, то есть Россию вынудили воевать, что война это
единственное условие выживания  

России.

Запад, в свою очередь вступил в период массовых восстаний против локдаунов,
против диктата корпораций и против так называемого "нового мирового порядка".
То есть против цифрового фашизма. Самые важные события, которые, возможно,
определят, будущее мира, сейчас происходят в Канаде и Австралии, где бушуют
опасные стихийные восстания. На подходе аналогичные восстания в США,
Франции,Австрии, Италии и т.д. В США  вообще внутренний раскол грозит перерасти
а открытую гражданскую войну.

В таких условиях США также не против с помощью ограниченной войны сбросить
протестный потенциал на экспорт.

Интересы сошлись. А значит это вопрос времени, когда война начнется. Не  факт,
что она начнется  именно в Украине, но  у нас самые большие шансы, так как мы
ничего не делаем, чтобы увернутся от войны. Напротив, мы, как всегда, с
удовольствием вошли в свой любимый образ жертвы, потому что он позволяет
получать кредиты, разворовывать их, вывозить капитал на Запад  и не делать
реформы. Ну, какие реформы, если война на носу!
