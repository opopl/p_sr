% vim: keymap=russian-jcukenwin
%%beginhead 
 
%%file 21_04_2022.fb.solovjov_mikita.harkov.demsokyra.2.evakuacia.cmt
%%parent 21_04_2022.fb.solovjov_mikita.harkov.demsokyra.2.evakuacia
 
%%url 
 
%%author_id 
%%date 
 
%%tags 
%%title 
 
%%endhead 
\zzSecCmt

\begin{itemize} % {
\iusr{Артем Коротенко}

Та блин, а в Киеве - \enquote{смотрите, город живет, проведем благотворительный
концерт, метро ездит, кафе открываются и сто, посольства возвращаются, но не
возвращайтесь пожалуйста без разницы даже если вы живете в школе в спортзале
хрен знает где второй месяц, тут очень опасно!}

\begin{itemize} % {
\iusr{Ольга Нехаева}
\textbf{Артем Коротенко} , 

Киев отлично себя чувствует, пусть сидит ровно на заднице и лишний раз
заткнется. И не знает по-настоящему, как работает ПВО, не слышит, как летят
мины и накрывает взрывной волной. У жителей Северной Салтовки спросите лишний
раз, ценой чего Киев живёт. Пусть спасибо скажет Харькову и Киевской области,
что мы сохраняем ему жизнь. Салтовка приветствует.

\iusr{Артем Коротенко}
\textbf{Ольга Нехаева} 

зачем столько агрессии, тем более к кому? У меня есть вопросы к городскому
руководству по коммуникации - пост автора о таких же вопросах, только в другом
городе. А люди в любом мирном городе не виноваты что их война затронула меньше.
Тернополь не виноват перед Киевом, Киев перед Харьковом, а Харьков перед Бучей
и Мариуполем.

\iusr{Ольга Нехаева}
\textbf{Артем Коротенко} , 

вам не понять. Будьте здоровы. Если вы не в курсе, кто не был у нас на
Северной, думает, что это Мариуполь. Это не агрессия, а нормальная реакция, а
то после войны хватит наглости некоторым особям рассказывать, какой Киев герой.
Харькову и в подмётки не годится.

\iusr{Людмила Волочай}
\textbf{Артем Коротенко} в Киеве иллюминация такая! Что аж страшно! Проезжали на поезде, из Харькова! Как кроты в темноте полтора месяца жили. А там..... Все в огнях! Вот и почувствуйте разницу!

\iusr{Артем Коротенко}
\textbf{Ludmila Volozay} 

Я не очень понимаю что вы хотите доказать. Да, города страдают по-разному,
Харьков принимает на себя страшный удар, Киев принял ужасный удар своими
пригородами с северо-западной стороны. Вряд ли горе и разрушения это то чем
стоит меряться.

Спасибо нашей армии за то что у нас все еще есть города которые могут жить, за
то что Буча не повторилась в масштабах Киева. А города которые не затронуты
активной войной сейчас должны жить, чтобы люди зарабатывали деньги и кормили
эту армию, иначе тогда зачем вообще это все?

\iusr{Микита Соловйов}
Слушайте, а о чем вы вот здесь все спорите, а?

\iusr{Артем Коротенко}
\textbf{Микита Соловйов} 

мой коммент был только о том что отсутствие стратегии в отношение
жизни-эвакуации-переселения и как следствие коммуникации свойственно сейчас не
только Харькову. Может он был как-то неправильно понят

\iusr{Оксана Масалітіна}
\textbf{Артем Коротенко} мне показалось, что он был не только вообще не понят, а даже не прочитан.

\iusr{Оксана Масалітіна}
\textbf{Артем Коротенко} 

а \enquote{...но не возвращайтесь} говорится просто на всякий случай явно, \enquote{а то вдруг что и скажете потом, что я виноват}

\iusr{Татьяна Чекалова}
\textbf{Артем Коротенко} в Киеве не лостреливает Арта.

\iusr{Sergiy Ryabykin}
\textbf{Ольга Нехаева} 

Скажіть спасибі ППО, придбаному Порошекно для Київа. Для Харкова Кернес придбав
Зоопарк та атракції в парк білочки, теж непогано захищають наше небо. Хто за що
голосував...

\iusr{Anja Annie}
\textbf{Sergiy Ryabykin} Як у вас все просто, прямо заздрю)) Тобто, як тільки Харкову придбають ППО, то у Харкові почнеться знов мирне життя?

\iusr{Sergiy Ryabykin}
\textbf{Anja Annie} 

В Київі збивається біля 80\% ракет. Рахуйте в попаданнях. Переведіть в
кількість загиблих. Подумайте перед виборами.

\end{itemize} % }

\iusr{Roman Frolov}

Начать легко - остановиться трудно.

Первое заявление - для руководства страной (видите какие мы молодцы, удерживаем
город и людей от новой большой волны беженцев), и тут скорее правда - все
быстро ремонтируют что можно, завоз продуктов и гуманитарки есть.

Второе для местного населения - кто не спрятался мы не виноваты и перед вами не
отвечаем - вроде тоже правда, прилетает везде, просто где-то чаще, где-то реже.

По очень пострадавшим районам может быть один не очень хороший момент (я тут
могу быть не прав) - если всех эвакуировать и ничего там не чинить, то беда
может начаться дальше, в тех местах где еще теппимо и все кое-как работает.

\begin{itemize} % {
\iusr{Микита Соловйов}
\textbf{Roman Frolov} Нет, не может. Потому что бьют куда достреливают. И удерживает от продвижения уж точно не мирняк, а ЗСУ и ТрО.

\iusr{Roman Frolov}
\textbf{Микита Соловйов} Кстати вот только что увидел - \enquote{Мы готовы вас эвакуировать}, – Терехов обратился к жителям Северной Салтовки и Пятихаток 

\url{https://www.newsroom.kh.ua/news/my-gotovy-vas-evakuirovat-terehov-obratilsya-k-zhitelyam-severnoy-saltovki-i-pyatihatok}

\iusr{Микита Підлісняк}
\textbf{Roman Frolov} нарешті
\end{itemize} % }

\iusr{Анатолий Ман}
Это касается не только харьковчан, вообще всех украинцев. \enquote{Власть} упорно пытается относиться к ним как к совковым рабам.

\iusr{Sergiy Ryabykin}
\textbf{Anatoly Man} 

Да їм просто пофіг. Влада зазаз думає про вибори та попил коштів, що надходитимуть \enquote{після війни}.

В тих випадках, коли йдеться про життя, треба і примусити. Наприклад, вивезти
всіх дітей треба незалежно від інфантильності батьків.

\iusr{Alisa Bey}
У них в голове цветочки и фонтаны... Полуавтоматы какие-то.

\begin{itemize} % {
\iusr{Sergiy Ryabykin}
\textbf{Alisa Bey} Яскраве проявлення неспроможності ефективно працювати. Вивезти дітей, відселити дорослих з найбільш небезпечних районів. Налагодити економіку та транспорт міста, логістику допомоги армії.
- А ми квіточки посадимо!

\iusr{Alisa Bey}
\textbf{Sergiy Ryabykin} план робіт, мабуть, затверджено на 2022-й,за ним і йдуть. Далі фестивалі та свято дитячого малюнка. Сарказм.
\end{itemize} % }

\iusr{Alexander Veprik}

В останньому абзаці, на жаль, протиріччя.

Якщо це люди, які відповідальні за свої дії, то вони самі можуть вирішувати, чи
треба їм евакуюватись. Абсолютна більшість таких вже вирішили, що їм робити, і
навіть централізована евакуація для них нічого не змінить - вони або ВЖЕ
поїхали, або залишились / повернулись.

Або навпаки, їх треба евакуювати централізовано з командою від міського
керівництва, але це тому, що вони не \enquote{дорослі люди що самі можуть вирішувати і
брати відповідальність за свої дії}, і саме тому їх треба як в дитсадочку,
ставити в пари за ручки і вести до автобуса...

\begin{itemize} % {
\iusr{Chacha Shushuridze}
\textbf{Alexander Veprik} так буває коли ліберали хочуть і лібералами лишитися, і вирішити за інших що їм робити і куди поїхати  @igg{fbicon.smile} 

\iusr{Марианна Маркова}
\textbf{Олександр Веприк} 

дуже багато розгублених людей, це не про несамостійність, це про реакцію на
стресову ситуацію (й ні, зовсім не всі можуть звикнути й діяти адекватно,
оосбливо якщо оточення та інформація неадекватні)

\iusr{Alexander Veprik}
\textbf{Chacha Shushuridze} Ви це намагаєтесь розповісти лібералу?  @igg{fbicon.smile} 
\end{itemize} % }

\iusr{Валерия Величко}

Начать эвакуацию это посеять панику, конкретики не может быть по одной причине
каждые день по несколько боёв и предвидеть результат ни кто не может, те кто
хотел и мог уехали, остальные выживают, но руководство городо предупреждает об
опасности обстрелов, так что живём как можем, гарантий безопасности не будет)

\iusr{Микита Соловйов}
\textbf{Валерия Величко} 

причем тут паника? Вернее даже не так, какая именно паника? По нам могут
попасть? Так могут. Но или нужно говорить о том, что риски для большинства
районов невелеки, вылезайте из убежищ и страайтесь заняться делом. Или что
риски большие, давайте-ка лучше в эвакуацию. Или третий вариант, который я и
предложил. Риски есть, они слишком большие, чтобы просто сидеть, но далеко не
запредельные по сравнению с пользой, которую можно принести.

\iusr{Lucy Zaglada}
Централизованно возможно стоит вывозить стариков и детей.

\iusr{Юлия Денисюк}
\textbf{Lucy Zaglada} 

Вы предлагаете детей вывозить отдельно от родителей?..Да и старики, если у них
есть хоть как-то то родственники, сами не поедут

\iusr{Dmitry Miller}

здесь ещё вопрос, что подразумевается под термином эвакуация. просто вывоз или
описание и соблюдение условий проживания на новом месте.

сколько разговоров с предложением выезда оканчивалось - и где я там буду жить?

\begin{itemize} % {
\iusr{Катерина Гонтаренко}
\textbf{Dmitry Miller} згодна, питання: де жити? за що жити? як заробляти? - стають першими.

\iusr{Владимир Завгородний}
\textbf{Katerina Gontarenko} в отличие от того, чтобы сидеть в метро и не давать ему работать, например.
Там как раз кристальная ясность, где работать и как зарабатывать.

\iusr{Микита Соловйов}
\textbf{Dmitry Miller} 

Что ты понимаешь под \enquote{соблюдением условий проживания}? Если соблюдение
санитарных норм и т.д., то естественно нет. Если возможность где-то лечь
поспать, то этого и так море. Условно комфорт или возможность физиологически
выжить? Второе обеспечивается, а первое не должно в условиях войны относиться в
приоритетам государства.

\iusr{Катерина Гонтаренко}
\textbf{Володимир Завгородній} 

ну, особисто я сиджу вдома і працюю віддалено. Виїду - втрачу дохід. Бо ноута
не маю, а стаціонарний комп з собою не забереш.

В нас в родині з 4-х людей працює 2. Батько в лікарні, він точно нікуди не поїде (хоч і не медик, а електрик)

А для тих, хто в метро - не знаю, спробую спроектувати на себе... Сидять в
рідному місті, а раптом завтра війна скінчиться - повернуться до домівок (якщо
вціліли), якось годують/забезпечують. А їхати - то невідомість. І ризик. І
втрата того мінімального відчуття безпеки, що вони мають.

Но то лише мої придумки, просто намагалась влізти в їхню шкуру.

\iusr{Elena Sichkaruk}
\textbf{Катерина Гонтаренко} 

Можна говорити про об'єктивні причини: можуть люди просто не витримати
тривалого переїзду, не мати сили облаштуватися на новому місті, забезпечувати
себе, не мати грошей зовсім, просто на їжу. А в метро їжу приносять, воду
дають, допомагають, тут вже відомо, до кого звернутися. Уявіть людину похилого
віку, яка жодного разу, або років 50, нікуди не переїжджала, їй страшно.
Можливо, якщо запропонувати тимчасове проживання, показати, де можна отримати
допомогу, їжу, і це у межах Харкова, просто в іншому районі, там, де відносно
безпечно. Але це важка робота, й підготовча, й матеріальна, й психологічна. Хто
нею буде займатися?

\iusr{Катерина Гонтаренко}
\textbf{Elena Sichkaruk} 

абсолютно згодна. Я й кажу, що люди чепляється за свій крихкий обламок
стабільності, чи то у вигляді рідних стін, чи то відносної безпеки метро.

Їх за це ніхто не засуджує.

Мені теж страшно, чесно кажучи. І наразі мені страшніше було б їхати внікуди,
ніж залишатися вдома. Бо я теж чепляюся за уламки свого звичного світу. Хоча в
маєму районі відносно безпечно було до останнього часу.

\iusr{Марианна Маркова}
\textbf{Катерина Гонтаренко} 

ми виїзжали з 2-ма стаціонарними компами, 1-м монітором, 2-ма собаками, 4-ма
котами й ще з дівчиною, яка ледь могла рухатися на той час...

\iusr{Dmitry Miller}
\textbf{Микита Соловйов} 

простое описание условий куда предлагается человека эвакуировать и их
соблюдение. затем предупреждение, что волонтёрская поддержка может сокращаться.
(тоже был момент, предлагаю свою помощь отнести продукты родителям хороших
знакомых. - нет, тебя мы не можем просить. как будто образный волонтёр имеет
другую защиту)

когда человек не имеет картинки перед глазами, какой минимум ему предлагается -
конечно остались те, кто сидит на попе ровно до того момента, пока не вылетели
стекла или чуть дольше. не все пожилые готовы "прыгать в никуда" пока не
припекло, да и кругом столько молодежи заботливой.

\iusr{Катерина Гонтаренко}
\textbf{Marianna Markova} 

а моїм сусідам волонтери, що їх на вокзал відвозили, дозволили взяти лише по 1
рюкзаку на людину. Вони навіть кішку нам залишили через це.

За умови відсутності власного авто я не уявляю навіть теоретичної можливості
брати будь що, що я не зможу дотягти на собі без втрати мобільності.

\iusr{Микита Соловйов}
\textbf{Катерина Гонтаренко} 

Вот именно. \enquote{Как-то кормят/обеспечивают}. То есть в условиях войны из-за
призрачной надежды что все скоро закончится, люди месяцами требуют себя
обеспечивать во фронтовом городе. Причем еще и блокируя единственный
относительно безопасный транспортный ресурс для всего города.

\iusr{Катерина Гонтаренко}
\textbf{Микита Соловйов} та я цілком з Вами згодна.
Просто щось робити треба на міському рівні, централізовано.
Не виганяти ж людей з метро... Бо людей теж можна зрозуміти, їхній світ рухнув. І вони продовжують чеплятися за будь-який осередок відносної стабільності та прихистку, яке наразі метпо й уособлює. Не всі мають гнучку психіку для сприйняття нової реальності та ресурс щось робити в цій реальності...

\end{itemize} % }

\iusr{Shepa アレックス}
Закрытая граница одна из причин для молодых. А стариков даже в мирное время с места не сдвинуть.

\iusr{Микита Соловйов}
\textbf{Shepa アレックス} 

Так не вопрос. Оставайтесь здесь и занимайтесь чем хотите. На свой страх и
риск. Но если остались, то постарайтесь не обременять своими потребностями
власти. У них есть значительно более важные задачи.

\iusr{Деметрий Синайский}
О каких организованных действиях по эвакуации можно говорить, если такого опыта с 1942 года нет.

\iusr{Андрей Балычев}
Русня розконсервовує ФАБ-3000, походу щоб нищити станції метро з людьми там

\iusr{Roman Frolov}
\textbf{Andriy Baliychev} они вроде такое уже по Чернигову применяли без всякого метро, или я ошибаюсь?

\iusr{Людмила Волочай}

Просто золотые слова! Власти прислушайтесь к гласу народа! Займитесь звакуацией
населения! Не подвергайте своим бездействием, ещё большему риску людей!

\begin{itemize} % {
\iusr{Микита Соловйов}
\textbf{Людмила Волочай} Это точно не мой призыв  @igg{fbicon.smile} 
Хотите - уезжайте. Не хотите - оставайтесь. Но и то, и другое, на свой страх и риск.

\iusr{Людмила Волочай}
\textbf{Микита Соловйов} я давно за эвакуацию семей с детьми, это просто жизненно необходимо. Дети не должны находиться в подвалах!

\iusr{Микита Соловйов}
\textbf{Людмила Волочай} Так теперь я включу обратный режим. А что сейчас мешает эвакуироваться семьям с детьми?

\iusr{Юлия Старко}
\textbf{Людмила Волочай} кому надо тот ищет варианты и едет

\iusr{Людмила Волочай}
\textbf{Микита Соловйов} возможно пинок под зад. Как бы ценично это не звучало.
\end{itemize} % }


\iusr{Сергей Давыдко}
но почему же сами граждане обстреливаемых районов такие тупые?!

\begin{itemize} % {
\iusr{Людмила Колесник}
\textbf{Сергей Давыдко} им деваться некуда, не знают куда поедут и денег нету. Это люди 60+

\iusr{Сергей Давыдко}
\textbf{Людмила Колесник} прикинь, мне тоже 60+, но с головой всё в порядке

\iusr{Nata Vladi}
\textbf{Людмила Колесник} А если им предложить эвакуироваться централизованно- они сразу согласятся?

\iusr{Людмила Колесник}
\textbf{Nata Vladi} думаю да!

\iusr{Nata Vladi}
\textbf{Людмила Колесник} Не уверена. Деньги ж у них не появятся от этого

\iusr{Виктория Болотова}
Это разные люди, с разными причинами за и против. Вы тупо жили когда-нибудь под обстрелами, под реальными обстрелами в своем доме?

\iusr{Людмила Колесник}
\textbf{Виктория Болотова} это мне вопрос?

\iusr{Людмила Колесник}
\textbf{Nata Vladi} деньги нет, но крыша над головой и еда должны быть, тогда поедут

\iusr{Виктория Болотова}
\textbf{Людмила Колесник} нет, не вам

\iusr{Виктория Болотова}
\textbf{Людмила Колесник} желаю вам когда-нибудь бросить свой дом и иметь еду и крышу над головой

\iusr{Людмила Колесник}
\textbf{Виктория Болотова} Понимаю, время трудное, но не надо перегреваться
\end{itemize} % }

\iusr{Елена Вас}

Высадка цветочков наше все! Как ещё не предложили построить фонтаны в метро для
тех, кто там прячется @igg{fbicon.face.screaming.in.fear} 

\iusr{Светлана Белоус}

Елена Вас не поверите, бюветы, минифонтанчики у них есть. ехала в последнее
утро работы метро, сама видела.

\iusr{Нестеренко Анастасия}

Тут другая проблема. Куда эвакуироваться то? Очень много уже всего занято.
Сидеть в привычном аду или уехать в никуда....

\iusr{Марианна Маркова}
\textbf{Nesterenko Anastasiia} 

Закарпаття приймає. Так, це не окремі кімнати на кодну людину, це дітсадочки чи
інші соціальні приміщення, але -- затишно у порівнянні з Харковом, є де митися,
є де перевести дух та пошукати щось більш влаштовуюче.

\iusr{Борис Шалайкин}
По перше: тоді нахріна увесь штат, що сидить у більшості на \enquote{удальонці}?

Светлана Белоус

нет слов.... бла, бла, бла.... ни одного дня не сидела в подвале- некогда. с первого
дня мотаюсь через полгорода, пешком и автостопом. люди у нас остались
душевные, подвозят. какие разные люди попадаются, и все классные. разве им не
страшно под прилетами? об этом большинству думать некогда. город задыхается без
транспорта, нужно метро. а рфлексии наслушалась за два месяца- ой
сколько. пипец, граждане....

\iusr{Геннадий Козаренко}
Cui bono? Cui prodest?

\iusr{Александр Андрущенко}
Вам же ОН сказал: ну оккупируют Харьков, ну и что?

\iusr{Ольга Белецкая}
\textbf{Олександр Андрущенко} диствительна...
А потом на шашлыки звал.
А до этого всю границе разминировал.
Ну оккупируют. Ну Харьков.... бывает....

\iusr{Александр Торохов}

Слишком большой дар ответственности ( долго с английского переводил) Вы хотите
от власти. В большинстве своём это одноклеточные: проплатил, избрали, украл,
отмазался и по кругу. Им похуй на всех кроме себя и мозгов нет . И наша Война
именно за это. Чтобы их никогда к власти не подпускать, ибо результат мы видим
сейчас.

\iusr{Виктория Болотова}

Ох, демагогия. Эвакуироваться по всем пунктам, куда? В спортзал, на матрас? Но
вы же топите за полноценную жизнь. Я вчера уехала только потому, что после
выбитых окон в центре у меня не осталось сил и оптимизма, и мне мои друзья
настоятельно предложили пожить мне и кошке в их квартире в Киеве, с интернетом,
где я могу работать и т.д. и т.п.

\begin{itemize} % {
\iusr{Марианна Маркова}
\textbf{Viktoriya Bolotova} спортзал з матрацем й безпека поганіше, ніж метро й обстріли?

\iusr{Виктория Болотова}
\textbf{Марианна Маркова} спортзал з матрасом значно поганіше, чим власний дім з умовно звичайним життям, навіть під обстрілами для моїх батьків. Там є ще багато нюансів, но те вас не турбує

\iusr{Марианна Маркова}
\textbf{Viktoriya Bolotova} мова зара йде про тих, хто живе у метро. Власний дом конче ліпше, але якщо людина доби проводить у метро, то про який власний дом й умовно звичне життя Ви говорите?

\iusr{Виктория Болотова}
\textbf{Марианна Маркова} Мова йде не тільки хто в метро

\iusr{Микита Соловйов}
\textbf{Виктория Болотова} 

Так а кто против того, чтобы человек у себя в доме жил? Уж точно не я. Но тогда
не нужно ему рассказывать «не выходите из укрытий». Как-то очень странно
большинство читает этот и не только мой пост. Я упорно говорю о том, что есть
два варианта для каждого. Или принять риски и жить в домах. Или ехать в
эвакуацию. Причём каждый пусть для себя из них выбирает. А вот жить месяцами в
убежище это не вариант.

Ну и отдельно о тех пяти микрорайонах, естественно.

\iusr{Виктория Болотова}
\textbf{Микита Соловйов} сори, за излишние неконструктивные эмоции. Но вопрос эвакуироваться куда? остается

\iusr{Aleksey Aksyonov}
\textbf{Виктория Болотова} Погодите... Это эмигрируют \enquote{куда}. Переезжают \enquote{куда}. А эвакуируются \enquote{откуда}. Туда, куда можно...

\iusr{Sergiy Ryabykin}
\textbf{Viktoriya Bolotova} Будинки відпочинку в більш спокійних областях

\end{itemize} % }

\iusr{Andrey Khavryuchenko}

Тому що оффі завжди будуть ставитись до свого харчового ресурсу, як шо малих
дітей (якщо вони в хорошому гуморі). Приблизно як сільський ґадза любить і
доглядає свою свиню, перед тим як пустити на сало.

\iusr{Марианна Маркова}
ну, то про обл та міськраду здорової людини... на жаль ми таких обрати не спромоглися

\iusr{Сергей Пузырьков}
А о каких 5 микрорайонах речь?

\iusr{Люда Колчева}
\textbf{Сергей Пузырьков} Думаю те которые по краям города.

\iusr{Елена Ермошенко-Левицкая}

Не нагнетайте обстановку, люди и так на нервах и осталось не долго, некому и
нечем уже воевать, все скоро закончится!!!!

\begin{itemize} % {
\iusr{Марианна Маркова}
\textbf{Елена Ермошенко-Левицкая} ой... а кому це \enquote{некому и нечем}? аж цікаво стало

\iusr{Ларшина Елена}
\textbf{Елена Ермошенко-Левицкая} 

я из дома выезжала на неделю. В итоге уже два месяца живу вне дома. А
американцы готовятся до конца года поддерживать Украину оружием. Только
арестовичь этого не скажет

\iusr{Елена Ермошенко-Левицкая}
\textbf{Марианна Маркова} рашистам естественно!!!

\iusr{Марианна Маркова}
\textbf{Елена Ермошенко-Левицкая} на жаль поки ні(((.

\iusr{Ольга Белецкая}
\textbf{Ларшина Елена} и что же вы предлагаете, а?
Просто перестать стрелять?
Вы там за два месяца кукухой поехали?

\iusr{Микита Соловйов}
\textbf{Елена Ермошенко-Левицкая} 

Хочу вас разочаровать. Им еще достаточно надолго хватит ресурсов для войны. В
Украине по данным на позавчера было 76 русских БТГр, еще 12 на границах.
Снарядов для ствольной арты и РСЗО у них вообще еще как у дурня фантиков.
Потому не нужно вот этих шапкозакидательских настроений. Нужно готовиться к
тому, что война еще надолго.

\iusr{Sergiy Ryabykin}
\textbf{Микита Соловйов} 

Є й інші обмежуючі фактори: зменшення кількості літаків та танків, також
пілотів та водіїв танків/БМП, неспроможність платити зарплатню навіть
військовим, зменшення бажання вмирати за шизи пуйла. Навіть номінальні БТГ з
втратами понад 25\% стають неспроможними ефективно воювати, а таких все більше.

Тож є надії, що ситуація переломиться за декілька тижднів. Хоча згоден,
готуватись треба до найгіршого сценарію.

Вивозити треба всіх дітей взагалі, навіть примусово. Бачів фотку дітей, що
грають у футбол. Дуже не хочеться думати, що з ними станеться, якщо на поле
влучить снаряд.

\iusr{Елена Ермошенко-Левицкая}
\textbf{Микита Соловйов} не согласна с вами, все скоро закончится!!!

\end{itemize} % }

\iusr{Ларшина Елена}
Я с Вами согласна полностью!

\iusr{Tetyana Sfandex}

«Харьковский городской голова Игорь Терехов сегодня, 21 апреля, с просьбой
эвакуироваться обратился к жителям Северной Салтовки, Пятихаток и других
районов, которые находятся под постоянными обстрелами оккупантов.

«Да, это нелегко, но мы готовы вас эвакуировать. Очень прошу, примите решение
сохранить свою жизнь. Мы вас вывезем и поселим в нормальные условия», -
обратился он к харьковчанам.

Мэр сообщил, что сегодня посетил один из интернатов, куда эвакуируют харьковчан
из горячих точек города.

«Я встречался с харьковчанами, которые последние восемь недель не видели
ничего, кроме бомбежек своих домов. Сегодня наши службы создали для них
нормальные условия проживания. Они пережили ужасные дни, но у них есть дух и
огромная вера в нашу победу! Победу, которую Украина обязательно добудет!» -
подчеркнул Игорь Терехов.

По его словам, враг продолжает жестоко обстреливать жилые районы города, в
частности, сегодня были разрушены два рынка. Поэтому Игорь Терехов призвал
харьковчан без необходимости не покидать укрытиях, убежища и станции
метрополитена.»

Тільки що в телеграм-каналі міської ради

Ви достукалися)))))))

\iusr{Маша Бахтігозіна}

Поддерживаю. Удивляло меня в 2014/2015, что ни городские, ни госвласти ничего
не сделали для беженцев из Донбасса. Тогда вообще был мрак, когда сотрудники
соцуправления тупо \enquote{повесили} этот вопрос на \enquote{Станцию Харьков}. Сейчас удивляет
то, что область/государство никаких возможностей для эвакуации и расселения
людей не предоставляет. Если у человека разбомбили квартиру, и он остался без
сбережений, он/она будет сидеть до последнего в метро. И Вы абсолютно правы,
что человек там бесполезен, более того, на него должны потратить время, рискуя
жизнью волонтеры. Но он будет сидеть, если нет централизованной эвакуации и
поселения  @igg{fbicon.frown} 

\iusr{Максим Попков}

Тяжко таке пісати але роки цкувань зі сторрони чіновникив породили в людей
зневіру. Це моя особиста думка я зараз в Харкові. Я опісував процесс отріманя
ІД картки. Не буду повторуватися але в кінци в мене попросіли копию старого
паспорта. Коли я його вже здав. И в мене вона була. Пощастило.

\iusr{Sergiy Ryabykin}

Відповіді на поверхні: а) керівникам міста начхати на людей та б) про
ефективність будь-чого вони й думки не мають - такі вже наші партійні
\enquote{менеджери}, працюють виключно на кишеню та кар'єру.

Війна ще більше висвітила проблему, що в керівництві міста необхідні думаючі
люди, а не ігруни в популізм та кар'єризм з сер'йозними пиками, за якими ані
думки, ані дії.



\end{itemize} % }
