% vim: keymap=russian-jcukenwin
%%beginhead 
 
%%file 12_12_2016.fb.uljanov_anatolij.1.travlja_kabanov
%%parent 12_12_2016
 
%%url https://www.facebook.com/dadakinder/posts/1376780182341141
 
%%author_id uljanov_anatolij
%%date 
 
%%tags kabanov_aleksandr.poet.kiev,literatura,obschestvo,travlja,ukraina
%%title ПО ПОВОДУ ТРАВЛИ ПОЭТА КАБАНОВА
 
%%endhead 
 
\subsection{ПО ПОВОДУ ТРАВЛИ ПОЭТА КАБАНОВА}
\label{sec:12_12_2016.fb.uljanov_anatolij.1.travlja_kabanov}
 
\Purl{https://www.facebook.com/dadakinder/posts/1376780182341141}
\ifcmt
 author_begin
   author_id uljanov_anatolij
 author_end
\fi

ПО ПОВОДУ ТРАВЛИ ПОЭТА КАБАНОВА

Литература традиционно является самой консервативной частью украинской
культуры. На то есть ряд объективных причин. Народ, только недавно обретший
свою государственность, сталкивается с необходимостью само-идентифицироваться.
В Украине этот процесс монополизирован сторонниками этнического национализма,
для которых язык – это не просто одно из средств общения, но фундамент
национального сознания. Такое отношение к языку определяет характер украинской
литературы, и утверждает нетерпимость в рядах её консервативных функционеров. 

\ifcmt
  ig https://scontent-frt3-1.xx.fbcdn.net/v/t1.18169-9/15380545_1376779655674527_2043467048244528324_n.jpg?_nc_cat=108&ccb=1-5&_nc_sid=730e14&_nc_ohc=OiA5TXtoB10AX-LpKCs&_nc_ht=scontent-frt3-1.xx&oh=4fa316cd25a0037d8e1481a008c576be&oe=61D05942
  @width 0.4
  %@wrap \parpic[r]
  @wrap \InsertBoxR{0}
\fi

Травля моего бывшего главреда
\href{https://www.facebook.com/kabanovsho}{Alexandr Kabanov}, вызванная его
высказываниями в отношении переименования очередного украинского проспекта в
проспект Бандеры, хоть и является удручающей, но не является неожиданной.
Националисты не любят других точек зрения. Речь, впрочем, не о националистах. И
не о Бандере. Речь об Украине, которая здесь и сейчас. 

Какими бы намерениями не было продиктовано желание отдельной части украинского
общества превратить все проспекты Ленина в проспекты Бандеры, нужно понимать,
что измученный украинский народ сегодня разобщён. Поэтому хорошо бы
воздержаться от спорных переименований, которые это разобщение только
усугубляют. А вот от чего не следует воздерживаться, так это от культуры –
понимания, что она включает в себя цветное разнообразие элементов, которые
невозможно свести к тому или иному региональному изводу.  Кабанов, в этом
смысле, фигура важнейшая, – он годами строил культурные мосты как внутри, так и
вне Украины. Травить его за это могут только отъявленные украинофобы.

(IMAGE: Snyders, Frans (Flemish, 1579-1657) – Загнанный кабан, 98 см x 101 см,
холст, масло)

\ii{12_12_2016.fb.uljanov_anatolij.1.travlja_kabanov.cmt}
