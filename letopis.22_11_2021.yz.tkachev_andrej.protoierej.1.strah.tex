% vim: keymap=russian-jcukenwin
%%beginhead 
 
%%file 22_11_2021.yz.tkachev_andrej.protoierej.1.strah
%%parent 22_11_2021
 
%%url https://zen.yandex.ru/media/andreytkachev__official/ia-ochen-mnogogo-boius-6197a71bc79525319bb08f0f
 
%%author_id 
%%date 
 
%%tags 
%%title Я очень многого боюсь...
 
%%endhead 
\subsection{Я очень многого боюсь...}
\label{sec:22_11_2021.yz.tkachev_andrej.protoierej.1.strah}

\Purl{https://zen.yandex.ru/media/andreytkachev__official/ia-ochen-mnogogo-boius-6197a71bc79525319bb08f0f}

– Вы испытывали в своей жизни чувство страха? Если – да, то, как Вы его
побороли?

Отец Андрей:

– Я его испытывал, и я его не поборол. Страх сопровождает меня постоянно. Я не
поборол страх. Вы думаете, что мы – «Кибальчиши» такие, «Супермены» такие. Я
боюсь потерять родных. Я боюсь многого. Болезни я, в принципе, не шибко боюсь.
Но нестерпимой боли – боюсь. Я боюсь безденежья. Как бы я ни был не
сребролюбив. Но я – боюсь безденежья.

\ii{22_11_2021.yz.tkachev_andrej.protoierej.1.strah.pic.1}

Я боюсь страшных потрясений во вселенной. Когда все с ума сходят. Я видел это.
Все с ума посходили. И ты не можешь ни на кого влиять. Были люди – и вдруг,
взяли и сошли с ума. Все кругом. И ты говоришь: «Эй! Эй!» А мне: «Отстань от
меня. Новая жизнь началась!» Бант на грудь и – побежали. Я это видел. И мне это
очень не нравится. И я этого боюсь. Я ненавижу революции. И я боюсь революций.
Ели бы мне дали право, я бы задушил всех революционеров (если бы мне за это
ничего не было). Потому что они разрушают жизнь в самом основании. Они не дают
жить простым людям. Они раздражают их больную незрелую душу и портят жизнь на
многие столетия вперед.

Я – боюсь. Много чего боюсь. Я смерти боюсь. Иногда я ее не боюсь. Отслужишь
литургию – и думаешь: «Вот, умереть бы сейчас!» Но: «Стоп! Жена есть. Дети
есть. Замуж не повыходили... Стоп... Стоп... Извиняюсь. Я поспешил. Я –
извиняюсь. Я очень извиняюсь».

Как снежная королева тебе в морду подышит. «Можно меня еще потерпеть чуть-чуть?
Пожалуйста. Денек. Два. Три. А лучше – больше».

Я очень многого боюсь. И это нормально, по-моему.

Боишься бедности, боишься старости, боишься одиночества, боишься смерти
близких. Боишься того, боишься сего. Так надоело бояться, Господи Иисусе!
Сколько можно бояться? «В страхе есть мучение». Так в Библии написано
(1Ин.4:18). Господи Иисусе, Дай не бояться! И хоть мы с вами не богатыри и не
силачи, но бояться не надо. Потому что Христос говорит нам: «Не бойтесь. Отец
Мой хочет дать вам Царство».

Я боюсь людей, которые ничего не боятся, между прочим. Очень странно видеть
человека, который ничего не боится. По-моему, это и не человек совсем. У него
есть мама, папа, любимая, любимое дело, любимая страна. Если он ничего не
боится, значит он никого не любит. Значит, у него и за душой ничего нет.

Но – бояться и преодолевать страх – это разные вещи. Одно дело бояться, другое
дело – бояться и вопреки страху делать то, что нужно делать.

Дай Бог, чтобы у нас хватило духа делать то, что нужно, когда очень страшно.
Страшно выйти вперед, когда: «А кто здесь такой православный, иди сюда!»
Конечно, хочется спрятаться. Но: «Я!» И робкий такой шаг, на полусогнутых: «А
что вы хотели?» У меня нет иллюзий про себя. И я не знаю, как я себя поведу.
Может быть, я побегу в атаку на танк. А может быть я сяду в угол и скажу: «Не
трогайте меня!»

Всякое в жизни бывало. Как у каждого человека. Бывали моменты трусости и
слабости. А бывали моменты смелости. Когда собой потом гордишься. Всякое
бывало.

Но – не знаю...
