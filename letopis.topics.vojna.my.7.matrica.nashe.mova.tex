% vim: keymap=russian-jcukenwin
%%beginhead 
 
%%file topics.vojna.my.7.matrica.nashe.mova
%%parent topics.vojna.my.7.matrica.nashe
 
%%url 
 
%%author_id 
%%date 
 
%%tags 
%%title 
 
%%endhead 

\paragraph{22:33:57 20-08-22 Валерий Мартыненкоreplied to ღВера}
%img Screenshot from 2022-08-20 22-33-18.png

УКРАЇНСЬКА МОВА - в 1934 році була визнана на конкурсі краси мов в Парижі, як найкраща, наймилозвучніша й найбагатша мова світу, і зайняла друге місце після італійської.

В 2012 році занесена в книгу рекордів Гіннеса, і віднині на першому місці, встановивши рекорд за тривалістю музичного телемарафону національної пісні.

Лише українські композиції лунали без перерви на рекламу в прямому ефірі 110 годин. Попередній рекорд був встановлений Італією в 2010 році у місті Пезаро — 103 години 9 хвилин 26 секунд.

УКРАЇНЦІ — співуча нація, яка створила найбільшу кількість народних пісень у світі. Разом нараховано близько 200 тисяч українських народних пісень. Ні одна нація за всю історію немає такої кількості пісень, яку створив український народ самостійно.

В ЮНЕСКО зібрана дивовижна фонотека народних пісень усього світу. У фонді України знаходиться 15,5 тисяч пісень.

УКРАЇНСЬКА МОВА -

МОВА СОЛОВ’ЇНА
