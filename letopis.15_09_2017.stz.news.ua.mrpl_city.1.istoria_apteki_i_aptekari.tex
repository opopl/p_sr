% vim: keymap=russian-jcukenwin
%%beginhead 
 
%%file 15_09_2017.stz.news.ua.mrpl_city.1.istoria_apteki_i_aptekari
%%parent 15_09_2017
 
%%url https://mrpl.city/blogs/view/istoriya-apteki-i-aptekari
 
%%author_id burov_sergij.mariupol,news.ua.mrpl_city
%%date 
 
%%tags 
%%title История: Аптеки и аптекари
 
%%endhead 
 
\subsection{История: Аптеки и аптекари}
\label{sec:15_09_2017.stz.news.ua.mrpl_city.1.istoria_apteki_i_aptekari}
 
\Purl{https://mrpl.city/blogs/view/istoriya-apteki-i-aptekari}
\ifcmt
 author_begin
   author_id burov_sergij.mariupol,news.ua.mrpl_city
 author_end
\fi

\ii{15_09_2017.stz.news.ua.mrpl_city.1.istoria_apteki_i_aptekari.pic.1}

Чего-чего, а аптек в Мариуполе предостаточно. В каждой больнице и в каждой
поликлинике их по нескольку. Ну и на улицах немало. Издалека видны кресты,
образованные зелеными светодиодными лампочками. Чуть ли не каждая аптека
обещает покупателям то низкие цены, то бонусы различные, то дисконтные карты.
Конкуренция – ничего не поделаешь. Но цены на лекарства остаются высокими.
Правда, некоторые лечебные препараты отечественного производства дешевле
зарубежных, но, покупая свое родное, нет уверенности, что вместо аспирина,
например, не подсунет изготовитель якобы лечебного препарата таблетку,
спрессованную из мела. Иногда приходит на ум крамольная мысль: а ведь сейчас у
нас вовсе и не аптеки, а магазины по продаже готовых лекарственных форм.

В чем же отличие аптеки от аптекарского магазина. В аптеках лекарства
приготавливались по рецептам врачей, а в аптекарских магазинах продавались
таблетки, микстуры, мази и тому подобное, произведенное на фармацевтических
фабриках и специальным способом упакованное, люди старшего поколения называли
их патентованными средствами. Воспоминания о мариупольских аптеках образца
начала 40 – середины 50-х годов прошлого века в классическом понимании этого
слова. Допустим, не дай бог, кто-то в семье заболел. Доктор выписывает рецепт,
где указано название лекарства и его количество для конкретного больного. Если
это порошок, - проще, другое дело, когда речь идет о микстуре или мази. Тогда
нужно искать какую-нибудь бутылочку или баночку. Аптеки, как правило, не
располагали в достаточном количестве емкостями для своих произведений.

Итак, рецепт есть, флакончик найден, заходим в аптеку. Фармацевт подсчитывает
стоимость лекарственной формы и выдает талончик с указанием цены и даты
готовности препарата. В кассе расплачиваемся и идем восвояси. Кстати, в каждой
аптеке была отдельно стоящая касса, только кассир прикасался к деньгам, которые
изобилуют возбудителями всех болезней. Лекарство могло быть готовым и через
несколько часов, и через день. А если оно нужно тяжелобольному человеку? Тогда
врач писал на рецепте слово \enquote{cito}, что означает  на латинском языке -
\enquote{срочно}. Если же это волшебное слово помечено еще и восклицательным знаком, то
это значит  - \enquote{срочнее не бывает}. В таком случае аптекари просили  немножко
подождать. В аптеках, конечно же,  можно было купить  \enquote{патентованные средства}
- фабричные таблетки, мази, растворы. Там же можно было, говоря канцелярским
языком, приобрести предметы санитарии и гигиены.

А теперь несколько абзацев о прошлом аптечного дела в нашем городе. Самое
раннее упоминание о мариупольских аптеках удалось найти в известном справочнике
\enquote{Адрес-календарь. Весь Мариуполь и его уезд. 1910 г.}.  К моменту подготовки к
печати  этого издания в нашем городе  было шесть аптек. Две аптеки  - на
Екатерининской улице, две – на Торговой улице, одна - на заводе \enquote{Русский
провиданс}, одна – на Слободке. Кроме  того,  имелось шесть аптекарских
магазинов. Три магазина  были  на Екатерининской улице,  два - на Торговой, по
одному – на Таганрогской  и  Базарной улицах. К этому времени в нашем городе
проживало около 50 тысяч человек. В \enquote{Адрес-календаре Екатеринославской губернии
на 1916 год} упоминается всего пять мариупольских аптек без указания их места
расположения. Четыре из них – частные, а одна принадлежала Мариупольскому
городскому общественному управлению.

После установления советской власти, а точнее, после свертывания НЭПа, все
частные аптеки и аптечные магазины были национализированы. Некоторые из
владельцев аптек стали провизорами в них, кое-кто был назначен управляющим. Но
были и такие аптекари, которых репрессировали. Как значительное событие в
истории нашего города было отмечено открытие 3 апреля 1924 года мариупольским
представительством Общества Красного Креста магазина \enquote{Санитарии и гигиены} на
углу Торговой улицы  и проспекта Республики (ныне проспект Мира). Анонсировался
большой  выбор аптекарских, парфюмерных, хозяйственных товаров, оптики.  В
конце 30-х годов прошлого века в первом этаже дома № 45 была открыта большая
аптека, оборудованная по самым высоким требованиям того времени. Гитлеровцы,
покидая город в сентябре 1943 года, сожгли практически все многоэтажные дома, в
том числе и дом № 45 с аптекой. Когда после освобождения дом этот
восстанавливали, не забыли и аптеку, известную во всем городе, как 16-я. Она
была интересна тем, что часть ее зала была отделена перегородкой, верх которой
был застеклен. И вот во время ожидания выполнения заказанного лекарства было
интересно смотреть, как \enquote{колдуют} над малюсенькими весами фармацевты, как они
крошечными совочками подсыпают в стаканы с водой порошки...

3 апреля 1953 года в газете \enquote{Приазовский рабочий} было напечатано объявление:
\enquote{Аптека № 21 переведена в новое помещение по адресу ул. Артема, 47}.  Ее
управляющим был назначен Семен Ильич Брук. До войны он работал в системе
аптекоуправления. С началом войны был мобилизован в армию. И демобилизован в
1945 году,  в то время ему уже было сорок пять лет. Конечно, Семен Ильич
безукоризненно владел всеми тонкостями фармации, но у него была еще и струнка
изобретателя. Во время службы в армии он придумал мазь из красного стрептоцида
и рыбьего жира, которая способствовала заживлению ожогов воинов. Уже в мирное
время \enquote{узким местом} был недостаток дистиллированной воды для приготовления
микстур. Семен Ильич разработал портативный перегонный куб для получения
дистиллята прямо в аптеке. Брука знали все ведущие врачи города. Именно от него
узнавали о новых лекарствах и препаратах...

Вот на этом и окончим пестрые заметки об аптеках и аптекарях Мариуполя.
