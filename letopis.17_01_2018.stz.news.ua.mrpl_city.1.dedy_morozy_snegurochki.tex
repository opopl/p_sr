% vim: keymap=russian-jcukenwin
%%beginhead 
 
%%file 17_01_2018.stz.news.ua.mrpl_city.1.dedy_morozy_snegurochki
%%parent 17_01_2018
 
%%url https://mrpl.city/news/view/kak-dedy-morozy-i-snegurochki-hodili-v-gosti-k-mariupoltsam-i-veselili-ih-na-ploshhadyah-foto
 
%%author_id ivanova_jana.mariupol,news.ua.mrpl_city
%%date 
 
%%tags 
%%title Как Деды Морозы и Снегурочки ходили в гости к мариупольцам и веселили их на площадях (ФОТО)
 
%%endhead 
 
\subsection{Как Деды Морозы и Снегурочки ходили в гости к мариупольцам и веселили их на площадях (ФОТО)}
\label{sec:17_01_2018.stz.news.ua.mrpl_city.1.dedy_morozy_snegurochki}
 
\Purl{https://mrpl.city/news/view/kak-dedy-morozy-i-snegurochki-hodili-v-gosti-k-mariupoltsam-i-veselili-ih-na-ploshhadyah-foto}
\ifcmt
 author_begin
   author_id ivanova_jana.mariupol,news.ua.mrpl_city
 author_end
\fi

В зимние праздники в Мариуполе Деды Морозы и Снегурочки веселили жителей в
общественном транспорте, во дворах и на площадях. Дарение счастья и волшебства
со Дня святого Николая настигало горожан внезапно и с музыкой. Увлекательные
флешмобы десанта Дедов Морозов и Снегурочек были организованы участниками
\href{https://mrpl.city/news/view/v-mariupole-budut-gotovit-universalnyh-prazdnichnyh-iventerov}{Event-школы \enquote{ЛюдиПраздники}}%
\footnote{В Мариуполе будут готовить универсальных праздничных ивентеров, Яна Іванова, mrpl.city, 06.07.2017, \par%
\url{https://mrpl.city/news/view/v-mariupole-budut-gotovit-universalnyh-prazdnichnyh-iventerov}
}
ДК \enquote{Молодежный}, передает MRPL.CITY.

Как рассказала руководитель школы Татьяна Живолуга, подготовка к Новому году
началась еще в ноябре на слете будущих Дедов Морозов и Снегурочек. Именно тогда
участники праздничного десанта извлекли новогодние запасы резиденции Деда
Мороза.

\begin{quote}
\em\enquote{Вместе мы готовили реквизит для обновленных \enquote{апартаментов}
Дедушки. Также молодежь сама готовила сценарии и разрабатывала маршруты для
проведения флешмобов и сюрпризов жителям Мариуполя. Все это нами впервые было
представлено 18 декабря на областном празднике для 300 детей-сирот в
Краматорске, а уже 21-го с такой же \href{https://mrpl.city/news/view/v-mariupole-ustroili-prazdnik-dlya-detej-iz-quotkrizisnyh-quot-semej-foto-1}{анимационной программой}%
\footnote{В Мариуполе устроили праздник для детей из \enquote{кризисных} семей, Анастасія Селітріннікова, mrpl.city, 31.12.2017, \par%
\url{https://mrpl.city/news/view/v-mariupole-ustroili-prazdnik-dlya-detej-iz-quotkrizisnyh-quot-semej-foto-1}
}

мы выступили в фойе
нашего Дворца}, 
\end{quote}
- сказала Татьяна Живолуга.

\ii{17_01_2018.stz.news.ua.mrpl_city.1.dedy_morozy_snegurochki.pic.1}

Она отметила, что участники еvent-школы были не только в образе Дедов Морозов и
Снегурочек – молодежь радовала мариупольцев в нарядах эльфов, фонариков и
других костюмах ростовых кукол. Хитом людей-праздников Мариуполя стала песня
\enquote{Пусть будет светло}, под которую они устраивали танцевальные флешмобы на
улицах города.

\begin{quote}
\em\enquote{Самым необычным в эти зимние праздники стали выступления молодежи во \href{https://mrpl.city/news/view/v-mariupole-novogodnij-desant-vysadilsya-vozle-pamyatnika-vysotskomu-fotofakt}{%
дворах многоэтажек, общественном транспорте, площадях}%
\footnote{В Мариуполе \enquote{новогодний десант} высадился возле памятника Высоцкому, Ярослав Герасименко, mrpl.city, 30.12.2017, \par%
\url{https://mrpl.city/news/view/v-mariupole-novogodnij-desant-vysadilsya-vozle-pamyatnika-vysotskomu-fotofakt}}
и у крупных ТРЦ. Специально были
разучены правила проведения хороводов с простыми и запоминающимися движениями
для большого количества людей. Кроме того, наши эльфы стали первыми пассажирами
светящегося \href{https://mrpl.city/news/view/na-mariupolskih-dorogah-poyavitsya-svetyashhijsya-trollejbus-foto-plusvideo}{новогоднего троллейбуса},%
\footnote{На мариупольских дорогах появится светящийся троллейбус, Яна Іванова, mrpl.city, 18.12.2017, \par%
\url{https://mrpl.city/news/view/na-mariupolskih-dorogah-poyavitsya-svetyashhijsya-trollejbus-foto-plusvideo}
}
прибывшего к \href{https://mrpl.city/news/view/v-tsentre-mariupolya-raznotsvetnymi-ogonkami-zasverkala-glavnaya-elka-goroda-foto}{открытию городской елки}},%
\footnote{В центре Мариуполя разноцветными огоньками засверкала главная елка города, Ярослав Герасименко, mrpl.city, 19.12.2017, \par%
\url{https://mrpl.city/news/view/v-tsentre-mariupolya-raznotsvetnymi-ogonkami-zasverkala-glavnaya-elka-goroda-foto}%
}
\end{quote}
- добавила Татьяна Живолуга.

\ii{17_01_2018.stz.news.ua.mrpl_city.1.dedy_morozy_snegurochki.pic.2}

Таким образом, подготовка к Новому году обеспечила мариупольским Дедам Морозам
и Снегурочкам успех среди зрителей. Выступления во дворах многоэтажек появились
в интернете, а горожане благодарили участников еvent-школы по телефону и даже
через контакт-центр Мариупольского городского совета.

Как отметила заведующая информационного отдела ДК \enquote{Молодежный} Мария Кутнянова,
выступления Дедов Морозов и Снегурочек в общественном транспорте также
сопровождались музыкой и танцами.

\ii{17_01_2018.stz.news.ua.mrpl_city.1.dedy_morozy_snegurochki.pic.3}

\begin{quote}
\em\enquote{Вместе с переносной bluetooth-колонкой они садились в троллейбусы и маршрутки
и пели песни. На остановках молодежь также радовала пассажиров флешмобами.
Таких вояжей было несколько. Также была поездка на площадь Победы Левобережного
района}, 
\end{quote}

- сказала Мария Кутнякова.

По ее словам, такое необычное внимание Дедов Морозов и Снегурочек мариупольцам
очень понравилось.

\ii{17_01_2018.stz.news.ua.mrpl_city.1.dedy_morozy_snegurochki.pic.4}

\begin{quote}
\em\enquote{Встречаясь с жителями, мы также просили пожелать что-то и нам. Многие желали,
чтобы Дед Мороз дарил им подарки, а затем, поняв, что сказочные персонажи тоже
нуждаются в добрых словах в свой адрес, они пожелали денег, чтобы Дед Мороз
смог на них купить мариупольцам подарки. Кроме флешмобов на улицах, молодежь
отправилась во дворы больниц. В этом году наше посещение таких учреждений очень
порадовало и осчастливило пациентов и медиков. Мы не заходили в отделения, но
привлекали внимание из окон. И, завидев нас, они приглашали внутрь лечебных
учреждений}, 
\end{quote}
- подчеркнула Татьяна Живолуга.

Параллельно с флешмобами на улицах Мариуполя, на Театральной площади работала
\href{https://mrpl.city/news/view/v-mariupole-ded-moroz-v-svoih-apartamentah-razdaet-detyam-podarki-fotofakt}{Резиденция Деда Мороза},%
\footnote{В Мариуполе Дед Мороз в своих \enquote{апартаментах} раздает детям подарки, Ярослав Герасименко, mrpl.city, 19.12.2017, \par%
\url{https://mrpl.city/news/view/v-mariupole-ded-moroz-v-svoih-apartamentah-razdaet-detyam-podarki-fotofakt}
} %
где были задействованы не только актеры \enquote{Театромании}
ДК \enquote{Молодежный}, но сотрудники ДК \enquote{Чайка}, пос. Каменск и других учреждений
культуры.

\begin{quote}
\em\enquote{Мариупольцы написали Деду Морозу порядка 500 писем. При этом старшее поколение
предпочло ограничиться отправкой письма в ящик, а не вручением при личной
встрече со сказочным персонажем. В связи с большой популярностью Резиденции у
жителей, она работала в две смены}, 
\end{quote}
- добавила Мария Кутнякова.

\ii{17_01_2018.stz.news.ua.mrpl_city.1.dedy_morozy_snegurochki.pic.5}

Также с вождением самых больших хороводов в Мариуполе состоялся \href{https://mrpl.city/news/view/v-mariupole-skazochnye-personazhi-sojdutsya-v-batle-dedov-morozov}{баттл}%
\footnote{В Мариуполе сказочные персонажи сойдутся в \enquote{Баттле Дедов Морозов}, Анастасія Селітріннікова, mrpl.city, 26.12.2017, \par%
\url{https://mrpl.city/news/view/v-mariupole-skazochnye-personazhi-sojdutsya-v-batle-dedov-morozov}}
Дед Морозов и Снегурочек, в котором победила дружба. С колядками участники
еvent-школы выступили в \href{https://mrpl.city/news/view/rozhdestvenskij-vertep-sobral-na-teatralnoj-ploshhadi-desyatki-mariupoltsev-fotofakt-1}{Рождественском вертепе}%
\footnote{Рождественский вертеп собрал на Театральной площади десятки мариупольцев, Анастасія Селітріннікова, mrpl.city, 08.01.2018, \par%
\url{https://mrpl.city/news/view/rozhdestvenskij-vertep-sobral-na-teatralnoj-ploshhadi-desyatki-mariupoltsev-fotofakt-1}
}
7 января на Театральной площади.

Помимо десантов Дедов Морозов, еvent-школа \enquote{ЛюдиПраздники} провела немало
других мероприятий – одним из таковых был \href{https://mrpl.city/news/view/v-preddverii-goda-sobaki-v-mariupole-proshel-gav-parad-foto}{\enquote{ГАВ-парад}}%
\footnote{В преддверии года Собаки в Мариуполе прошел \enquote{ГАВ-парад}, Ярослав Герасименко, mrpl.city, 24.12.2017, \par%
\url{https://mrpl.city/news/view/v-preddverii-goda-sobaki-v-mariupole-proshel-gav-parad-foto}
}
в центре города. При этом команда ивентеров обещает радовать мариупольцев в
других массовых праздничных мероприятиях.

Напомним, что в Мариуполе городскую елку закрыли в день \href{https://mrpl.city/news/view/v-mariupole-pod-pesni-molodezhi-i-tantsy-snegurochek-zakryli-glavnuyu-elku-foto}{Старого Нового года}.%
\footnote{В Мариуполе под песни молодежи и танцы Снегурочек закрыли главную елку, Анастасія Селітріннікова, mrpl.city, 14.01.2018, \par%
\url{https://mrpl.city/news/view/v-mariupole-pod-pesni-molodezhi-i-tantsy-snegurochek-zakryli-glavnuyu-elku-foto}
}
