% vim: keymap=russian-jcukenwin
%%beginhead 
 
%%file slova.zvezda
%%parent slova
 
%%url 
 
%%author 
%%author_id 
%%author_url 
 
%%tags 
%%title 
 
%%endhead 
\chapter{Звезда}
\label{sec:slova.zvezda}

%%%cit
%%%cit_head
%%%cit_pic
\ifcmt
  pic https://www.nkj.ru/upload/iblock/c66/c66c72df52373083c6433a2c57fd873d.jpg
	caption Красноватый дневной свет нарастал. Бетельгейзе поднималась тихо и величественно... 
	width 0.4
\fi
%%%cit_text
Наблюдая за \emph{звёздами}, астрономы, как правило, не видят их диска. Им
крайне редко удаётся следить за какими-либо процессами на поверхности и в
окрестности небесных светил, за исключением Солнца. Отсюда огромный интерес к
тем немногим \emph{звёздам}, которые выглядят не просто как яркая точка. И
когда в декабре 2019 года одна из таких \emph{звёзд} — Бетельгейзе — неожиданно
на глазах начала сильно терять свою яркость, это сразу привлекло внимание, тем
более что \emph{звезда} находится на той стадии эволюции, когда она может
превратиться в сверхновую. Возможность такого сценария сильно подогрела интерес
к поведению звезды, породив вал публикаций в прессе c предположениями, что
\emph{звезда} вот-вот взорвётся.  Разберёмся с причинами глубокого «обморока»
\emph{звезды} и вспомним, что мы знаем о Бетельгейзе и её загадках
%%%cit_comment
%%%cit_title
\url{https://www.nkj.ru/archive/articles/38479/}
%%%endcit

