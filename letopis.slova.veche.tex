% vim: keymap=russian-jcukenwin
%%beginhead 
 
%%file slova.veche
%%parent slova
 
%%url 
 
%%author_id 
%%date 
 
%%tags 
%%title 
 
%%endhead 
\chapter{Вече}

%%%cit
%%%cit_head
%%%cit_pic

\ifcmt
  tab_begin cols=2

     pic https://avatars.mds.yandex.net/get-zen_doc/3518430/pub_61695c9b8ac46f6d0f2f558c_6169644fffdef07a4ab68e64/scale_1200

     pic https://avatars.mds.yandex.net/get-zen_doc/108047/pub_61695c9b8ac46f6d0f2f558c_616970163a2c7849d523e1e0/scale_1200

  tab_end
\fi
%%%cit_text
Система же власти в Новгороде была устроена так. Высшей властью обладало \emph{вече} -
по сути народный сход. Обсуждением насущных проблем руководил посадник - аналог
премьер министра, а в случае с \emph{вече} еще и спикера. Военными делами, причем не
только ведением непосредственно боевых действий, но и формированием,
содержанием и обеспечением дружины, заведовал тысяцкий. Очень важно отметить,
что очень большой властью в Новгороде обладал местный архиепископ. Причем не
только духовной, но и светской. Он например был главной новгородского аналога
"палаты мер и весов", а также председательствовал на суде
%%%cit_comment
%%%cit_title
\citTitle{Господин Великий Новгород}, Илья Duke, zen.yandex.ru, 20.10.2021
%%%endcit
