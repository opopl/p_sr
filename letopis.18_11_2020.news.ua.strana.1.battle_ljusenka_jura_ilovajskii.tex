% vim: keymap=russian-jcukenwin
%%beginhead 
 
%%file 18_11_2020.news.ua.strana.1.battle_ljusenka_jura_ilovajskii
%%parent 18_11_2020
 
%%url https://strana.ua/news/301611-arestovich-i-butusov-o-chem-battl-dvukh-adeptov-partii-vojny.html
%%author 
%%tags 
%%title 
 
%%endhead 
\Purl{https://strana.ua/news/301611-arestovich-i-butusov-o-chem-battl-dvukh-adeptov-partii-vojny.html}


\subsection{Битва "Люсеньки" с "Юрой Иловайским". Как и почему "фельдмаршал" Бутусов и Арестович устроили публичный баттл}

\Pauthor{Венк, Виктория}
\par 18:10, сегодня

\ifcmt
pic https://strana.ua/img/article/3016/11_main.jpeg
caption Юрий Бутусов и Алексей Арестович. Коллаж "Страны"
\fi

В Сети уже несколько дней развивается конфликт двух известных адептов
"партии войны" - главреда сайта "Цензор" Юрия Бутусова и советника
украинской делегации в ТКГ Алексея Арестовича.

Последнего недавно туда назначили,\Furl{https://strana.ua/articles/interview/298040-intervju-spikera-tkh-alekseja-arestovicha-o-tom-kak-zakonchit-vojnu-na-donbasse-.html} чтобы поддерживать информационную
повестку Офиса президента по вопросу Донбасса. И убеждать общественность,
что "зрады" нет. 

Это вступило в противоречие с позицией Бутусова, который в последнее время
критикует Зеленского и обвиняет его окружение даже в госизмене по делу
"вагнеровцев" - а также в провале оборонной политики.\Furl{https://strana.ua/articles/analysis/285543-vahnerovtsy-tolko-povod-kto-i-pochemu-atakuet-hlavu-op-andreja-ermaka.html}

В итоге Арестович - бывший порохобот, который сегодня защищает Офис
президента - позвал своего оппонента на баттл. И в ходе из сетевой
перепалки они обменялись интересными претензиями. 

"Страна" проследила за этой заочной, но весьма показательной баталией.

\subsubsection{О чем поспорили Бутусов и Арестович}

Все началось с поста главреда "Цензора" от 11 ноября, в котором тот
раскритиковал Минобороны за закупку гаубиц "Дана". Они, по мнению
Бутусова, устарели.

\ifcmt
pic https://strana.ua/img/forall/u/0/92/%D0%B4%D0%B0%D0%BD%D0%B0.png
\fi

На претензию внезапно ответил Арестович - хотя она его никак не касалась.
Но явно затрагивала парафию Зеленского, что и вынудило Алексея начать
гибридную войну в соцсетях. 

Он назвал Бутусова "фельдмаршалом" (то есть намекнул на некомпетентность
журналиста в военных вопросах) и упрекнул в том, что тот "сеет записную
зраду". А также напомнил, что при Порошенко армия закупала вооружение
производства 70-х годов. 

\ifcmt
pic https://strana.ua/img/forall/u/0/92/%D0%BE%D1%82%D0%B2%D0%B5%D1%82_%D0%BD%D0%B0_%D0%B4%D0%B0%D0%BD%D1%83.png
\fi

На следующий день советник Украины в ТКГ вызвал Бутусова на баттл.

"Юра, тут народ хочет баттла.

Или как у нас иногда говорят, "дізнатися істину".

По "Вагнеру", по военной экспертизе, по штурму тюрем, по Иловайску, по
"Данам", про то, кто, где служил и многому другому.

Придёшь?"

\ifcmt
pic https://strana.ua/img/forall/u/0/92/%D0%B2%D1%8B%D0%B7%D0%BE%D0%B2.png
\fi

Через три дня Бутусов ответил, что на баттл при посредниках не согласен,
но может записать с Арестовичем интервью. И предложил локацию - Офис
президента. Намекая этим, что военный эксперт транслирует месседжи из ОП. 

При этом главред "Цензора" обозначил круг вопросов, которые прозвучат:
так, он усомнился в том, что Арестович имеет звание майора и 33 раза ходил
за линию фронта, служа в разведке ВСУ (о чем Алексей когда-то признавался
сам). 

"Заявленное число "боевых выходов" Арестовича в "тыл врага" в 18-19-м году
в то время, когда активных действий не велось, а фронт был стабилен,
звучит просто удивительно, поскольку все эти многочисленные рейды в тыл
врага Арестович совершал мимоходом, в перерывах между частыми поездками в
Киеве и многочисленными интервью на телеканалах", - написал журналист. 

Также Бутусов заявил, что спросит, как Арестович из "порохобота"
превратился в "зелебота". И напомнил, что в прошлом Арестович - актер,
игравший травести "Люсеньку". Тем самым намекая, что вся его военная
биография - фарс. 

\ifcmt
pic https://strana.ua/img/forall/u/0/92/%D0%B1%D1%83%D1%82%D1%83%D1%81%D0%BE%D0%B2(20).png
\fi

Арестович ответил в комментариях и назвал Бутусова "черным
пропагандистом", а не журналистом. И от интервью с ним отказался,
настаивая на дебатах.

\ifcmt
pic https://strana.ua/img/forall/u/0/92/%D0%B0%D1%80%D0%B5%D1%81%D1%821.png
\fi

Сегодня же Арестович написал пост о том, что уже согласен на интервью с
Бутусовым, но теперь сам главред "Цензора" от него отказывается. Также
спикер ТКГ опубликовал фото наградных книжек от Минобороны, которые должны
подтверждать военное прошлое Арестовича. 

Далее Арестович снова назвал Бутусова "Юрой Иловайским" и "фельдмаршалом"
- намекая, что он, несмотря на свои амбиции военного эксперта, никаких
оснований считать себя таковым не имеет. И делая отсылку к истории с
участием Бутусова в планировании провальной для Украины иловайской
операции.

\subsubsection{Бутусов и Иловайск}

О деталях подготовки наступления на Иловайск, в котором участвовал журналист
Юрий Бутусов, рассказывал еще командир батальона "Шахтерск" Андрей
Филоненко\Furl{https://strana.ua/articles/istorii/28775-vojna-2014-put-k-ilovajskomu-kapkanu.html}
- временной следственной комиссии, расследующей причины трагедии.

Встреча в Днепропетровской областной администрации, которой тогда
руководил Игорь Коломойский, состоялась 5 августа 2014 года. Она
продлилась 5-6 часов. На совещании в здании ОГА присутствовали Геннадий
Корбан (тогдашний вице-губернатор области), Семен Семенченко (батальон
"Донбасс"), Андрей Билецкий ("Азов"), журналист Юрий Бутусов. Позже
подъехал и командующий в секторе "Б" генерал-лейтенант Руслан Хомчак -
ныне глава украинского генштаба.

Корбан озвучил тему совещания — взятие Иловайска и до приезда Хомчака
участники обсуждались детали операции: с какой стороны зайти, качество
связи.

Прибывший Хомчак стал вести совещание вместо Корбана. Он сказал, что
возьмет в кольцо Донецк, а для Иловайска не было времени и пехоты, поэтому
он попросил разобраться с этим батальоны: "Зайдите туда батальоны и
зачистите, там максимум 50 террористов".

В результате договорились, что своими силами добровольческие батальоны
"Шахтерск", "Азов" и "Днепр-1" возьмут южную половину города.

По словам Филоненко, все действия тогда согласовывали с представителями
МВД — главой Департамента по управлению и обеспечению подразделений
милиции особого назначения Виктором Челованом и представителем МВД в штабе
АТО Владимиром Гриняком. Для выхода в зону АТО выходил соответствующий
приказ министра внутренних дел. И после совещания в ОГА он получил
согласование от координатора от МВД в штабе АТО.

Часть из этих показаний Филоненко позже подтвердили и другие участники
встречи.

Так, в разговоре со "Страной" Геннадий Корбан подтвердил, что у него в
кабинете проходила встреча, поскольку командованием было принято решение
взять Иловайск.

"Важно было скоординировать обеспечение всех войск, принимавших участие в
военной операции, начиная от батальонов террообороны, заканчивая
бригадами. Цель и задача - максимально обеспечить людей всем необходимым
для проведения этой операции и логистику" - рассказал Корбан.



На вопрос как на этой встрече оказался журналист Юрий Бутусов Корбан
ответил так: "Бутусов был все время со мной, месяцами находился у меня в
кабинете Днепропетровской администрации, поскольку там размещался основной
фронт. Все слышал и видел. Он был чуть ли не единственным журналистом, кто
регулярно освещал ряд военных действий с начала и до конца. Кроме того, он
сам неоднократно выезжал в зону АТО. Он лично проводил какие-то
тактические консультации командиров подразделений. Плюс он брал нашу
техническую помощь и несколько раз доставлял в зону АТО: телевизоры,
бинокли, планшеты и прочую техническую мелочь".

Чем закончилась история с Иловайском - всем известно: украинские войска
там попали в "котел". Правда, виновных (с украинской стороны) следствие
так и не обнаружило.

В том числе, не была расследована роль Бутусова в планировании операции,
которая закончилась самым тяжелым поражением украинских войск за все время
боевых действий на Донбассе.

\subsubsection{В чем смысл баттла?}

Конфликт Арестовича и Бутусова - эхо войны, которая идет между Офисом
президента и командой Турчинова-Порошенко. У бывшего гаранта таким образом
консолидируют свой электорат, раскачивая тему "зрады" Зеленского.

И, надо сказать, тактика дала свой эффект: на местных выборах
"Евросолидарность" перетянула на себя часть националистически настроенных
избирателей "Слуги народа". Активно в этой истории поучаствовал и Бутусов,
раскручивая, к примеру, громкую историю о госизмене Ермака по делу
"вагнеровцев". 

Но вернемся к Арестовичу. Его взяли на работу в ТКГ вскоре после
увольнения умеренного экс-премьера Витольда Фокина.\Furl{https://strana.ua/news/282441-vitold-fokin-chto-izvestno-o-pervom-premere-ukrainy-kotoroho-mohut-otpravit-na-donbass.html} Того в свою очередь
пригласили накануне местных выборов, чтобы создать иллюзию прогресса в
минском процессе. И остановить отток юго-восточных избирателей к
"Оппозиционной платформе". 

Однако после того, как Фокин в интервью "Стране" заявил,\Furl{https://strana.ua/news/282441-vitold-fokin-chto-izvestno-o-pervom-premere-ukrainy-kotoroho-mohut-otpravit-na-donbass.html} что нужно
исполнять Минские соглашения во всей полноте - его убрали. А через
какое-то время поставили Арестовича, который до этого считался один из
главных ЛОМов условной "партии войны". Что, видимо, по задумке Зеленского
и Ермака, должно было накал "зрады" снизить. 

Однако, как показала ситуация с Бутусовым, "партии войны" абсолютно без
разницы, какие декорации на Банковой возводят вокруг темы Донбасса. Атака
идет в принципе на весь процесс отношений с РФ в рамках минской площадки.

Цель при этом двоякая.

Во-первых - загнать президента за флажки, "красные линии", прочерченные
Порошенко и Ко, ликвидировав даже теоретическую возможность каких-либо
компромиссов с Россией и "республиками". 

Во-вторых - дискредитировать Зеленского среди промайданной части
избирателей, представив его едва ли не агентом Кремля. Хотя по факту его
политика по "Минску" мало чем отличается от времен Порошенко. У
последнего, впрочем, можно найти еще больше "зрады" - он не только
подписал Минские соглашения, но и продавил в Раде принятие закона об
особом статусе Донбасса. 

Поэтому - если Офис президента наймет для своего пиара по Донбассу даже не
Арестовича, а, к примеру, самого Парубия, отношение "партии войны" к
переговорам и к самому Зеленскому это никак не изменит.
