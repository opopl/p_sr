% vim: keymap=russian-jcukenwin
%%beginhead 
 
%%file 12_04_2022.fb.juzych_jurij.1.plast_krest_nagrada
%%parent 12_04_2022
 
%%url https://www.facebook.com/serdukus/posts/4925270310843835
 
%%author_id juzych_jurij
%%date 
 
%%tags 
%%title Бронзовий пластовий хрест \enquote{За значне діло}
 
%%endhead 
 
\subsection{Бронзовий пластовий хрест \enquote{За значне діло}}
\label{sec:12_04_2022.fb.juzych_jurij.1.plast_krest_nagrada}
 
\Purl{https://www.facebook.com/serdukus/posts/4925270310843835}
\ifcmt
 author_begin
   author_id juzych_jurij
 author_end
\fi

Сьогодні на першій пластовій ватрі
\href{https://www.facebook.com/Kyiv.Plast}{Київський Пласт} в цьому році
\href{https://www.facebook.com/ivan.fedko.1238}{Ivan Fedko} отримав Бронзовий
пластовий хрест \enquote{За значне діло}. Цю відзнаку ініціював особисто
основоположник скаутського руху британський генерал Бейден Поуел. Як подяку тим
скаутами, які врятували життя. 

\ii{12_04_2022.fb.juzych_jurij.1.plast_krest_nagrada.pic.1}

Ця відзнака у
\href{https://www.facebook.com/PlastUA}{Пласт - український скаутинг}
запроваджена в далекому 1922 році, тобто рівно сто років тому.  Символічно, що
сьогодні її вручено в Києві вихованцю підготовчого куреня ім.  Ярослава Стецька
за врятоване життя завдяки навикам, отриманими у Пласті (вчасно зупинив
кровотечу). Юнаку, який з перших днів московської війни вступив добровольцем до
\href{https://www.facebook.com/112btro}{112 Бригада територіальної оборони
міста Києва} СКОБ! Слава Україні!

Фото \href{https://www.facebook.com/andriy.kovalov}{Andriy Kovalov}

\begin{itemize} % {
\iusr{Нелля Даниленко}
Героям Слава!

\iusr{Oksana Jatskivska}
Пласт Києва має ким пишатися!

\iusr{Liubov Tsop}
Дякую!

\iusr{Володимир Міщенко}
Сильно! Іван надихає

\iusr{Yuriy Kostyuk}
Героям Слава!

\iusr{Andriy Kovalov}
\textbf{Юрій Юзич}, Хто фоткав?  @igg{fbicon.smile} 

\iusr{Юрій Юзич}
\textbf{Andriy Kovalov} вже додав. Дякую за фото, Андрію

\end{itemize} % }
