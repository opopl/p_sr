% vim: keymap=russian-jcukenwin
%%beginhead 
 
%%file songs.tin_soncja.pisnja_chugajstera.rus
%%parent songs.tin_soncja.pisnja_chugajstera
 
%%endhead 

\subsubsection{Русский}

В чащах вековых, где болотный цвет не тлеет,
Где полегла чёрно-золотая конница Батыя
Временами, в белой рубахе, бродит заколдованый
Леший-Чугайстар и читает стихотворение:

Если можешь, заглядывай ко мне ты,
Если хочешь, испей моей водицы,
Если веришь, расскажи мне,
О чём мечтаешь ты, когда живёшь в одиночестве.

Ворота паутины от сосны и до берёзы —
Дежурные ожидания — Судьбы сломанные весы.
Пламенная рута животворно расцветает
Но никто не видит! Никто того не знает!

Стонет хищным плачем трясина на Полесье,
Пеплом укрыты листья в древлянском лесу.
В безнадёге квёлый тополь склонился,
Дует тревожный ветер с отравленного поля.

Всё, что приснилось — больше не вернется —
Сказочные мечты в небытии.
Слишком рано остановилось сердце:
Мы умерли слишком молодыми.

Мой голос с неба чёрный ворон принесет,
Проплывёт по Припяти венок.
Ждите весточки под Хмурым Солнцем,
В сиянии жалобных звёзд.
