% vim: keymap=russian-jcukenwin
%%beginhead 
 
%%file 21_09_2021.fb.bodrjagova_olga.1.jazyk_mova_odnoklassnica
%%parent 21_09_2021
 
%%url https://www.facebook.com/olga.bodriagova/posts/4500095743381287
 
%%author_id bodrjagova_olga
%%date 
 
%%tags jazyk,mova,odnoklassnik,shkola,udivlenie,ukraina,vstrecha
%%title о, привет, Оля, ты теперь уже полностью говориш на укрАинском?! Вот это Да!
 
%%endhead 
 
\subsection{О, привет, Оля, ты теперь уже полностью говориш на укрАинском?! Вот это Да!}
\label{sec:21_09_2021.fb.bodrjagova_olga.1.jazyk_mova_odnoklassnica}
 
\Purl{https://www.facebook.com/olga.bodriagova/posts/4500095743381287}
\ifcmt
 author_begin
   author_id bodrjagova_olga
 author_end
\fi

Розмова з однокласницею, після перерви в 38 років:

-(рос мова, Ліля) о, привет, Оля, ты теперь уже полностью говориш на
укрАинском?! Вот это Да!

-так, Ліля, а чому ти так здивувалась? Я українка, живу в Україні, розмовляю
українською. Мене більше дивує твоє питання- це все одно, що спитати Китайця в
Китаї: «о, а ти говориш китайською?!»

- (Ліля, рос мова) нєт, ну ты просто в детстве разговаривала по русски

- так, звісно, пам’ятаєш нашу вчительку рос мови? (Примітка: Заслужена
вчителька, нагороджена орденом Леніна). Вона нам завжди повторювала: «вы должны
говорить на чистом литературном русском языке, а не на этом вашем «рыбасивском»
жаргоне; если, конечно, вы не хотите на базаре торговать семечками». Ми були
зросійщені, всі, хоча моя мама все життя була україномовна. Саме цю мову, мову
моєї мами з незначними домішками суржику, тоді обізвали «рибасівським
жаргоном.» Нам, маленьким дітям, вкладали в голови важливість і пріоритетність
російської мови.

- (Ліля, рос мова): а знаєш, Оля, я тебе даже завидую. Так, правда! Я бы тоже
так хотела, наверное…А я вот сейчас живу в Луцке, и все очень хорошо переходят
на русский, и так хорошо говорят на русском! Вот молодцы просто! Очень
толерантные все!

\ii{21_09_2021.fb.bodrjagova_olga.1.jazyk_mova_odnoklassnica.pic}

-так, я помітила це і знаю. Перехід на рос мову завжди сприймається позитивно,
на ура. А ось навпаки - відразу викликає здивування і шквал запитань!

За ці останні роки, коли я свідомо перейшла на українську, я почула не менше
сотні запитань. Від друзів, від колег, від однокласників,  від таксистів. Всі
ці фрази нижче - оригінальна мова, запам’ятались мені дослівно, бо завжди
лунали з таким подивом, ніби в мене на голові виросли рога, раптом  @igg{fbicon.smile}  

перші три питання - топові, найчастіші. Особливо тяжко мені було в перший рік
переходу. Я після цих запитань почувалась дуже некомфортно, ніби зі мною щось
не так..

хочете дізнатись як це? Спробуйте перейти на українську, після російської…

- а что, ты теперь всегда так разговариваешь?

- а кто у вас в семье говорит на укрАинском?

- и давно ты разговариваешь на укрАинском?

- я тоже умею свободно на укрАинском, просто на русском удобнее

- нет нет, все нормально, все так, просто странно как то это слышать…

-ну, укрАинский это как то непривычно от тебя…

- а дети твои тоже говорят на укрАинском? 

- а муж твой на каком языке разговаривает? А, ну да, у тебя же муж бандеровец..

- я не знаю укрАинского, потому что из Кривого Рога. Вы тоже из Кривого рога?
Странно, там не говорят на укрАинском 

- а  вы откуда приехали? Из Киева? Я был в Киеве, там все на русском
разговаривают. 

- я вообще считаю, что вопросы языка - это не важно, это второстепенно, вот не
надо это трогать вообще

- а вообще, какая разница на каком языке разговаривать?

Останнє - ключове. Оскільки немає різниці для російськомовних, але є така
різниця для україномовних, то цілком логічно, що має бути лише українська! Всі
ті, для кого немає різниці, переходять на мову тих, для кого різниця таки є.  @igg{fbicon.smile}  

Але чомусь це так не працює…

В храмі Св Сімейства в Барселоні на воротах слави є фрагмент молитви «отче
наш», на 50 мовах. Наша мова теж є серед цих 50! На фото нижче саме цей
фрагмент. І ще там є аудіогайд Українською мовою!

\ii{21_09_2021.fb.bodrjagova_olga.1.jazyk_mova_odnoklassnica.cmt}
