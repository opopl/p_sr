% vim: keymap=russian-jcukenwin
%%beginhead 
 
%%file 15_02_2022.fb.pogrebinskij_mihail.1.usa_putin_ukraina
%%parent 15_02_2022
 
%%url https://www.facebook.com/permalink.php?story_fbid=1571443629886672&id=100010631505344
 
%%author_id pogrebinskij_mihail
%%date 
 
%%tags napadenie,putin_vladimir,ugroza,ukraina,usa,vtorzhenie
%%title США пытаются доказать, что Путин нападет на Украину
 
%%endhead 
 
\subsection{США пытаются доказать, что Путин нападет на Украину}
\label{sec:15_02_2022.fb.pogrebinskij_mihail.1.usa_putin_ukraina}
 
\Purl{https://www.facebook.com/permalink.php?story_fbid=1571443629886672&id=100010631505344}
\ifcmt
 author_begin
   author_id pogrebinskij_mihail
 author_end
\fi

США инсценировали эвакуацию дипломатов и олигархов из Украины, чтобы в мире
поверили, что Путин нападет на нее

США пытаются доказать, что Путин нападет на Украину, но этого не произойдет, в
связи с чем они окажутся в критически невыгодной ситуации на мировой арене. Об
этом в беседе с Telegram-каналом «Радиоточка НСН» заявил директор Киевского
центра политических исследований и конфликтологии Михаил Погребинский.

На фоне заявлений западных СМИ о готовящемся «российском вторжении», которое
якобы произойдет 14 февраля, более 30 стран рекомендовали своим гражданам
покинуть Украину. Одновременно с этим страну покидают олигархи и бизнесмены.
Погребинский подтвердил, что предприниматели уезжают сами и вывозят свои семьи
из страны.

«Дипломаты тоже уезжают. Но посольства не прекращают работать, просто
сокращается персонал. Я объясняю это очень просто: вот эта фантасмагорическая
беспрецедентная компания «Путин нападет» запустила механизм, который
предусматривает определенный протокол действий. В соответствии с ним надо
вывозить граждан и предлагать им уехать. Просто если этого не сделать, то тогда
будет понятно, что это все мистификация. А американцы заинтересованы в том,
чтобы как можно больше людей поверили в то, что Путин нападет», - отметил он.

При этом, по его мнению, США «таким образом выстраивают беспроигрышную
ситуацию».

«Если Путин нападет – они были правы и все, что могли сделали. Украину защищать
США не собираются, и они об этом предупреждали, а своих граждан и посольство –
да. А если Путин не нападет, то они скажут: «Мы его напугали, поэтому он не
напал». Но, по-моему, они слишком далеко зашли, и теперь, если Путин не
нападет, то, конечно, американцы окажутся в самой настоящей критической ж***. И
какие дураки найдутся для того, чтобы поверить им в следующий раз – не знаю», -
заключил Погребинский.

\ii{15_02_2022.fb.pogrebinskij_mihail.1.usa_putin_ukraina.cmt}
