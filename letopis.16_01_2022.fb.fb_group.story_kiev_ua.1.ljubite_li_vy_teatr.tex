% vim: keymap=russian-jcukenwin
%%beginhead 
 
%%file 16_01_2022.fb.fb_group.story_kiev_ua.1.ljubite_li_vy_teatr
%%parent 16_01_2022
 
%%url https://www.facebook.com/groups/story.kiev.ua/posts/1841775492685910
 
%%author_id fb_group.story_kiev_ua,petrova_irina.kiev
%%date 
 
%%tags kiev,kultura,teatr
%%title "Любите ли вы театр?" (с)
 
%%endhead 
 
\subsection{\enquote{Любите ли вы театр?} (с)}
\label{sec:16_01_2022.fb.fb_group.story_kiev_ua.1.ljubite_li_vy_teatr}
 
\Purl{https://www.facebook.com/groups/story.kiev.ua/posts/1841775492685910}
\ifcmt
 author_begin
   author_id fb_group.story_kiev_ua,petrova_irina.kiev
 author_end
\fi

\enquote{Любите ли вы театр?} (с). 

Для меня это не вопрос с трехлетнего возраста. Первая роль Мышки положила
начало огромной любви к Театру. может, этому способствовало нечто сакральное -
я родилась, росла и жила на улице. названной в честь первой Народной артистки
Украины Марии Заньковецкой. В театре русской драмы работала очень близкая
подруга мамы, и, понятно, я пересмотрела много спектаклей легендарного театра,
слонялась за кулисами. Папа некоторое время работал в инженерной службе театра
Франко. Детский садик был рядом с театром.

\raggedcolumns
\begin{multicols}{3} % {
\setlength{\parindent}{0pt}

\ii{16_01_2022.fb.fb_group.story_kiev_ua.1.ljubite_li_vy_teatr.pic.1}
\ii{16_01_2022.fb.fb_group.story_kiev_ua.1.ljubite_li_vy_teatr.pic.1.cmt}

\ii{16_01_2022.fb.fb_group.story_kiev_ua.1.ljubite_li_vy_teatr.pic.2}

\ii{16_01_2022.fb.fb_group.story_kiev_ua.1.ljubite_li_vy_teatr.pic.3}
\ii{16_01_2022.fb.fb_group.story_kiev_ua.1.ljubite_li_vy_teatr.pic.3.cmt}

\ii{16_01_2022.fb.fb_group.story_kiev_ua.1.ljubite_li_vy_teatr.pic.4}
\ii{16_01_2022.fb.fb_group.story_kiev_ua.1.ljubite_li_vy_teatr.pic.5}

\end{multicols} % }

Садик, школьная сцена, приветствия на всяких съездах, конкурсы. слёты. смотры,
студенческий театр, вечера, фестивали в Институте гидромеханики АН
УССР... сотрудник института, фанат театра Вадим Базилевич, светлая ему память,
и привел меня в Народный театр драмы и комедии Киевского
трамвайно-троллейбусного управления. 

Согласна, звучит, как  \enquote{ансамбль скрипачей  имени листопрокатного цеха}.

\ii{16_01_2022.fb.fb_group.story_kiev_ua.1.ljubite_li_vy_teatr.pic.6}

Сейчас это уже далекое прошлое, 80-е годы 20-го века, замшелая древность.
Возможно, кто-то и помнит, что такое \enquote{народные театры}. Самодеятельность? Да.
\enquote{А Театр-то ненастоящий}? Возможно.  В веке 21-м, просвещённом, продвинутом.
технологичном. цифровом подобие самодеятельности - блогерство, нервый тик
Тик-Тока.  Вовсе не осуждаю. не насмехаюсь. Обещала милейшей нашей Светлана
Манилова немного вспомнить байки нашего театрика. Выполняю.

Киевлянам известен \enquote{пряничный домик} - Лукьяновский народный дом, построенный
в самом начале 20-го века на Лукьяновке (теперешняя Дегтяревская). История
этого дома большая, сложная. 

\raggedcolumns
\begin{multicols}{2} % {
\setlength{\parindent}{0pt}

\ii{16_01_2022.fb.fb_group.story_kiev_ua.1.ljubite_li_vy_teatr.pic.7}
\ii{16_01_2022.fb.fb_group.story_kiev_ua.1.ljubite_li_vy_teatr.pic.7.cmt}

\ii{16_01_2022.fb.fb_group.story_kiev_ua.1.ljubite_li_vy_teatr.pic.8}

\end{multicols} % }

После 1945 года  и до начала 2000-х здесь находился Дом культуры КТТУ. В числе
прочих коллективов был создан и народный театр. С  1936 года руководил
народный артист Гранатов А.Б.,  в 70-х  главрежем театра  стала народная
артистка Украины Вера Леонидовна Предаевич.  В конце 70-х в театр, волею
судьбы. пришла я. Зная уже Веру Леонидовну, её мужа Анурова Александра
Герасимовича, как актеров театра русской драмы с некоторым трепетом пришла я на
знакомство. Первая картинка - Вера Леонидовна сидит на стульчике посреди
большой репетиционной комнаты, на шее много низок бус, на запястье - несколько
браслетов, на пальцах - крупные перстни. В руках - сигарета. Глубоким, чуть
хрипловатым, голосом говорит: \enquote{И что Вас привело?} Моё несмелое блеяние: имею.
мол опыт сценический - садик, школа, стихи и спектакли, КВНы, в институте -
студтеатр. короче - звИзда . талант. 

\raggedcolumns
\begin{multicols}{2} % {
\setlength{\parindent}{0pt}

\ii{16_01_2022.fb.fb_group.story_kiev_ua.1.ljubite_li_vy_teatr.pic.9}
\ii{16_01_2022.fb.fb_group.story_kiev_ua.1.ljubite_li_vy_teatr.pic.10}

\ii{16_01_2022.fb.fb_group.story_kiev_ua.1.ljubite_li_vy_teatr.pic.11}
\ii{16_01_2022.fb.fb_group.story_kiev_ua.1.ljubite_li_vy_teatr.pic.12}

\ii{16_01_2022.fb.fb_group.story_kiev_ua.1.ljubite_li_vy_teatr.pic.13}
\ii{16_01_2022.fb.fb_group.story_kiev_ua.1.ljubite_li_vy_teatr.pic.14}

\end{multicols} % }

Вера Леонидовна так прищурилась, может, и
от дымка сигареты и просит: " Произнесите слова \enquote{приезжать}, \enquote{язык} и \enquote{к
кормушке}. Выпаливаю просимое, с чувством, звонко, радостно!  \enquote{Деточка, сначала
- в студию. К Изабелле Ильиничне!} Изабелла Ильинична Павлова, тоже актриса
театра русдрамы, супруга знаменитого Николая Николаевича Рушковского.  Она
просит повторить тот же магический набор слов. вердикт -я не умею говорить! Я!
НЕ УМЕЮ! ГОВОРИТЬ! Самолюбие. было, мотнуло меня в сторону двери. Но, иногда
разум подключался вовремя. 

\ii{16_01_2022.fb.fb_group.story_kiev_ua.1.ljubite_li_vy_teatr.pic.15}

Как-то я поделилась с Изабеллой Ильиничной заветной. несбывшейся мечтой - я
струсила, не стала даже пробовать поступать в театральный. \enquote{И, правильно} -
сказала наша Бэллочка  \enquote{У тебя не очень сильный голосовой аппарат. Не твоё.
Эмоции, нерв, разумение - в наличии. Но...}   Тогда еще с гарнитурой
драматические актёры не выступали.

\raggedcolumns
\begin{multicols}{3} % {
\setlength{\parindent}{0pt}

\ii{16_01_2022.fb.fb_group.story_kiev_ua.1.ljubite_li_vy_teatr.pic.16}
\ii{16_01_2022.fb.fb_group.story_kiev_ua.1.ljubite_li_vy_teatr.pic.17}

\ii{16_01_2022.fb.fb_group.story_kiev_ua.1.ljubite_li_vy_teatr.pic.18}

\end{multicols} % }

После вышкола студии сценречи и сцендвижа начались роли. На базе нашего театра
ставили свои  дипломные спектакли выпускники киевского ин-та Карпенко-Карого.
Я пришла как раз во время репетиций спектакля по пьесе А.Дударева \enquote{Порог}.
Сюжет - о заблудшей душке алкоголика и его возрождении. «Научите собаку думать,
и собака запьёт» - девиз-оправдание героя Буслая.  Главную роль играл мой
будущий муж, Саша Гаврилов. Режиссер - Мириам Александрович-Краско, жена актера
Андрея Краско. И Андрей, и Саша уже не с нами... Мариам, Мышка, как мы её
звали, тогда была в положении. с жуткой аллергией. токсикозом. Спектакль
сложный, нервный, Саша, Мышка  ездили в клинику на Фрунзе, ознакомиться с
поведенческими признаками больных Delirium tremens. 

\ii{16_01_2022.fb.fb_group.story_kiev_ua.1.ljubite_li_vy_teatr.pic.19_20}

Мне досталась
роль... милиционера! Участковая дама-милиционер, не вредная. не чинуша, даже
человечная. Роль мне тогда очень понравилась.  Система Константина Сергеевича,
проживание в предлагаемых обстоятельствах, мастерство поставленной сценречи -
всё пошло в ход. Спектакль был готов, но, защита диплома Мышки оказалась под
угрозой. Главлит, репертуарные комиссии - теперь редко кто помнит эти
сакральные словечки. В конце 70-х цензуре подвергались. по-моему, даже сценарии
детских утренников с обязательным идейным соответствием Зайки, Мышки и пятого
гриба в третьем ряду. что уж говорить о большой сцене Народного театра!

\raggedcolumns
\begin{multicols}{2} % {
\setlength{\parindent}{0pt}

\ii{16_01_2022.fb.fb_group.story_kiev_ua.1.ljubite_li_vy_teatr.pic.21}
\ii{16_01_2022.fb.fb_group.story_kiev_ua.1.ljubite_li_vy_teatr.pic.21.cmt}

\ii{16_01_2022.fb.fb_group.story_kiev_ua.1.ljubite_li_vy_teatr.pic.22}

\end{multicols} % }


Запороли дяди-тёти в мохеровых беретах на раз и без крика нашего Сашу ( роль
Буслая-алкоголика он изображал мастерски!), был там еще персонаж. по роли
какбэ функционер чиновничий! Тип мерзкий. О! \enquote{запрещать-не пущать- всё
переделать}. Ага, щаз! Мышке уже вот-вот в родзал, Андрюша Краско \enquote{лечится} в
осветительной со \enquote{светлячками}, Вера Леонидовна, бледная от гнева, сжимает
губы и молчит. Дяди-тёти в мохберетах уходят, вслед летит крепкое словцо из уст
Народной артистки. И мы играем спектакль через три дня. Потом Верочка своими
связями как-то \enquote{отмазала} и директора Дома культуры, и Мышку (диплом ей зачли с
отличием),  впоследствии мы убрали самые сакраментальные фразы. Играли этот
спектакль  очень много раз. Мне довелось играть и жену Буслая, у нас ещё не
было романтических отношений, мне надо было рыдать около фотографии мужа,
который, якобы, замерз в сугробе. Это далось несложно.

\ii{16_01_2022.fb.fb_group.story_kiev_ua.1.ljubite_li_vy_teatr.pic.23_25}

«Киевская тетрадь» Вадима Собко, пьеса о киевских подпольщиках, о Раисе
Окипной. По сути-то пьесы и не было, а так, очерк, а роли - функции. Очень
человечный спектакль на сцене театра Леси Украинки поставил тогдашний ведущий
режиссер, народный артист Украины Николай Алексеевич Соколов.  (à propos,
родной племянник Вадима Собко, писатель и кинорежиссёр Гелий Снегирёв,
известный в семидесятые годы диссидент.)

\ii{16_01_2022.fb.fb_group.story_kiev_ua.1.ljubite_li_vy_teatr.pic.26_27}

В нашем спектакле \enquote{Киевская тетрадь} у меня была роль рассказчика, ведущего,
\enquote{от лица автора}. Раису Окипную играла красавица Олечка Ороховская.

Как-то поставили спектакль в афишу ко дню освобождения Киева. Олечка
заболевает, отмена немыслима - срыв идейного мероприятия (ужос-ужос) со всеми
вытекающими. Что делать? Ведущим, рассказчиком. \enquote{от автора} вводим
парнишку-студийца. Он читает с листа с глубокомысленным выражением Автора. Меня
вприпрыжку вводят на роль Раисы Окипной. По иронии судьбы, я сейчас живу на
улице её имени! Роль я знаю, мизанцены - тоже. да вот незадача. Раиса Окипная -
оперная певица, по ходу спектакля поет романс, театрально опираясь на пианино.
Вооот. Но! У меня драматический талант. Петь я не умею от слова \enquote{вообще}.
Внутренний голос и чудный слух наружу не показываются. Вера Леонидовна меняет
мизансцену - рояль с центра сцены придвигают вплотную ко второй кулисе, там
стоит девочка с хорошим голоском, мне \enquote{аккомпанирует} гауляйтер. я тщательно, с
душою открываю рот \enquote{Утро тумаааное, уууутро седооое...} - фонограмма? Да! И не
стыдно. Один раз в жизни можно ж. Правда. билеты на спектакль продавали.
Простите, зрители.

\raggedcolumns
\begin{multicols}{3} % {
\setlength{\parindent}{0pt}

\ii{16_01_2022.fb.fb_group.story_kiev_ua.1.ljubite_li_vy_teatr.pic.28}

\ii{16_01_2022.fb.fb_group.story_kiev_ua.1.ljubite_li_vy_teatr.pic.29}
\ii{16_01_2022.fb.fb_group.story_kiev_ua.1.ljubite_li_vy_teatr.pic.29.cmt}

\ii{16_01_2022.fb.fb_group.story_kiev_ua.1.ljubite_li_vy_teatr.pic.30}

\ii{16_01_2022.fb.fb_group.story_kiev_ua.1.ljubite_li_vy_teatr.pic.31}
\ii{16_01_2022.fb.fb_group.story_kiev_ua.1.ljubite_li_vy_teatr.pic.31.cmt}

\end{multicols} % }

Еще один дипломант - Наварэтэ Давила Иван Е. Боливия. Как мы его звали?
правильно! Ванечка. 

Эдуардо де Филиппо. \enquote{Призраки}. Роли у меня тут не было, я была муз.редактором.
Помните ленточные магнитофоны? Вооот. Склеенные фрагменты музыки включаются на
реплики. И, ЕСТЕСТВЕННО, во время спектакля предательски рвётся, зажевывается
лента. Теряются корды (цветные кусочки скотча-вставки). Но, это была сущая
ерунда. Ведь именно именно под песню  Toto Cutugno \enquote{Italiano}, помните:

Лашате ми кантаре

кон ла китарра ин мано

лашатеми кантаре

уна канцоне пиано пиано

Лашатеми кантаре

перке не соно фьеро

соно ун италиано

ун италиано веро - (звучало из каждого утюга в начале 80-х.) \enquote{залетали
бабочки} в душе, в роли красавца Альфреда я \enquote{увидела} Сашу! Мы играли
рядом уже три года, и вот... надо же... Художником спектакля был Борис Ройтер
(Краснов), светлая память... сцендвиж ставила Ирочка Шведова

\ii{16_01_2022.fb.fb_group.story_kiev_ua.1.ljubite_li_vy_teatr.pic.32_34}

Театральные байки - это особый вид творчества. Их можно рассказывать, как
Шехерезада, да еще и в трёх экземплярах. Ролей было немало, идейные,
мелодраматические, гротескные, бурлесковые. Это были чудные, незабываемые
годы. На сцене мы играли с будущим мужем, моими будущими свёкрами - вся семья
на подмостках.

\ii{16_01_2022.fb.fb_group.story_kiev_ua.1.ljubite_li_vy_teatr.pic.35}

Роль - это же маленькая жизнь. И прожито их немало. В гостинных, виллах, на
свалке, хате Савватия Гуски. О! Это был феерический, танцевально-дурашливый,
но, с очень мудрой \enquote{подмыслью} спектакль. За п'єсою Миколи Куліша \enquote{Отак загинув
Гуска}.   Мою героїню звали Пистонька (Євпистимія).

\ii{16_01_2022.fb.fb_group.story_kiev_ua.1.ljubite_li_vy_teatr.pic.36_37}

Как-то пошла я в театр, давали спектакль по пьесе Димитриса Псафаса
\enquote{Требуется лжец}. Мы играли его, я была Жожо! Смешнючая комедия
положений, зал стонет от смеха, мои соседи по ложе с недоумением взирают на мой
зарёванное лицо, подшмыгивающий нос - неадекватное восприятие. Мне ли им
рассказывать, что вот смотрю - и никого уже из тех, кто был Ферекисом и
Паралосом, Врасидасом и Агисом, Пататьясом нет с нами на этой земле... Они живы
в моей памяти.

\ii{16_01_2022.fb.fb_group.story_kiev_ua.1.ljubite_li_vy_teatr.pic.38_40}

Шикарная дама приходит на свалку за взбунтовавшимся мужем (\enquote{И был день} по
пьесе А. Дударева \enquote{Свалка}). Перестройка, задник оформлен покосившимся,
вывернутым наизнанку  кумачовым лозунгом \enquote{Мы придем к победе коммунистического
труда!}. Пришли...

Изнеженная жена партийного чинуши не приемлет радостной встречи оккупантов...
(\enquote{Одна ночь} по пьесе Б. Горбатова).

Дама высшего общества  Люся при встрече с первой. горькой любовью оказывается
способной на Поступок... Мой будущий  муж играл \enquote{первую любовь} -
Костю.  Как он пел песенку на стихи Павла Когана:

И немножко жутко, 

И немножко странно, 

Что казалось шуткой, 

Оказалось раной. 

Что казалось раной, 

Оказалось шуткой... 

И немножко странно, 

И немножко жутко. 

(\enquote{Ревизия} по пьесе Бориса Рабкина) Эту роль мы играли 28 апреля 1986 года(!),
на сцене я была тогда с предполагаемым сыночком...

Были гастроли в Молдавию, были поездки в дома отдыха, госпитали ветеранов. мы
были семьей - бухгалтер и студенты, кандидат технических наук и грузчик,
инженер и воспитатель детского садика, журналист мой свёкр и моя любимая
свекровь... Я помню всех по именам, они - часть, одна из лучших частей моей
жизни. Есть в группе КИ  и актриса нашего театра.  Розалия ГольдецкиРоза, мы с
тобой всех помним! 

\raggedcolumns
\begin{multicols}{2} % {
\setlength{\parindent}{0pt}

% roditeli
\ii{16_01_2022.fb.fb_group.story_kiev_ua.1.ljubite_li_vy_teatr.pic.41}
\ii{16_01_2022.fb.fb_group.story_kiev_ua.1.ljubite_li_vy_teatr.pic.41.cmt}

\end{multicols} % }

в конце фотографий есть малюсенькое видео. Я рассказываю участникам фестиваля
\enquote{Мария} во время экскурсии по Киеву историю первой фонограммы. Огромное спасибо
за видео режиссеру из Испании Пати Доменику  и актрисе Марии Видаль. 

К фотографиям публикации я сделаю комменты, к каждой роли. Возможно, как это
бывало не раз в группе - найдутся зрители тех лет. присмотритесь, пожалуйста,
может, и правда найдутся знакомые.

Есть маленькая просьба - не пускаться в дискуссии о тематике спектаклей, об
образах персонажей, о реалиях того времени. Всё уже пережевано до нас и еще
будет пережевываться стотыщпицот раз. Этот рассказик - просто театральные байки
и поиск возможных зрителей и участников того кусочка частной жизни. Всем
спасибо за понимание.

\ii{16_01_2022.fb.fb_group.story_kiev_ua.1.ljubite_li_vy_teatr.scr.1}

\ii{16_01_2022.fb.fb_group.story_kiev_ua.1.ljubite_li_vy_teatr.cmt}
