% vim: keymap=russian-jcukenwin
%%beginhead 
 
%%file 08_12_2021.fb.lesev_igor.1.strategicheskaja_pobeda_zelenskii.cmt
%%parent 08_12_2021.fb.lesev_igor.1.strategicheskaja_pobeda_zelenskii
 
%%url 
 
%%author_id 
%%date 
 
%%tags 
%%title 
 
%%endhead 
\subsubsection{Коментарі}

\begin{itemize} % {
\iusr{Максим Гольдарб}

\ifcmt
  ig https://i2.paste.pics/bef50a1dc0e5522a2cd251c2ac02b32e.png
  @width 0.25
\fi

\iusr{Энрике Менендес}

Всё верно с одной поправкой - ЛДНР это супер-бедное продолжение Ростовской
области. Без какой-либо экономической интеграции, кроме трудово-мигрантской (((
Получил паспорт - ищи работу в РФ за з/п меньше чем у россиянина первого сорта
раза в три

\begin{itemize} % {
\iusr{Игорь Лесев}
\textbf{Энрике Менендес} знаю не понаслышке, какая там жопа. Но эта жопа теперь там и очень сильно вряд ли, что когда-то будет здесь. Это их теперь уже проблемы, их жизнь и они сами будут планировать, что и как там обустраивать.

\iusr{Сергий Федорынчык}
\textbf{Энрике Анатольевич Менендес} Тут якісь дивні у Вас нотки - майже русофобські! На просторах Сибіру, Далекого Сходу та Крайньої Півночі трудящих донбасівців чекає багато робочих місць з високими зарплатами. І ніхто не набридатиме українською мовою!

\iusr{Энрике Менендес}
\textbf{Сергий Федорынчык} Я не живу в чёрно-белом мире. Работа ждёт везде. Но это не дома

\iusr{Сергий Федорынчык}
\textbf{Энрике Анатольевич Менендес} Рідний Дім - це справді висока вартість, як і Мир. Але чи не варто певним жителям Донбасу заради цих вартостей визнати, що вони припустились деяких серйозних помилок, зокрема - брали участь у незаконних референдумах?

\iusr{Алексей Иванов}
\textbf{Сергий Федорынчык} да в 1991 был незаконный референдум. И это была ошибка.

\iusr{Алексей Иванов}
\textbf{Сергий Федорынчык} люди бегавшие с флагами , значками и ленточками с символикой ЕС , могут свалить в ЕС.

\iusr{Сергий Федорынчык}
\textbf{Alexey Ivanov} Ви нагадуєте даму, яка вважала, що увесь полк крокує неправильно, крім її сина. Референдум 1991 р був визнаний всіма країнами світу, а референдуми в Донецькій та Луганській областях - ні.

\iusr{Елена Джабраилова}
\textbf{Игорь Лесев} но в этой жопе минпенсия на 1 тыс грн больше с 1 января и жкх в 5-10 раз ниже, чем в европейской Украине....Да и печи доменные начинают задувать)) Так надо разобраться какая жопа больше

\iusr{Алексей Иванов}
\textbf{Энрике Менендес} тут нужно признать один факт. До 2020 года в России
были железно уверенны, что Донбасс нужно будет отдавать, сейчас или потом, но
отдавать. По этому считали, что развивать и как то вкладываться, как в Крым не
нужно. Теперь пришло осознание, что таки ЛДНР это уже де -факто часть России.
По этому года через три посмотрим. И да, расчитывать Украине на ЛДНР не стоит.

\iusr{Юлия Коломойцева}
\textbf{Энрике Менендес} , не беднее Таганрога, как оказалось. Ростовская область тоже разная.
(Деградацию экономики на неподконтрольных Украине территориях я не отрицаю.)

\iusr{Юлия Коломойцева}
\textbf{Сергий Федорынчык} , просто с некоторых пор государство Украина перестало быть родным для этих территорий, поэтому не нужно обвинять жителей в этих проблемах, явно не они их создали на референдумах. Референдум был в мае, а отряды "добровольцев" начали собирать на майдане в марте-апреле.

\iusr{Юлия Коломойцева}
\textbf{Сергий Федорынчык} , "весь світ" всегда признает то, что ему выгодно, и не признает то, что ему невыгодно. Вся легитимизация.

\iusr{Сергий Федорынчык}
\textbf{Alexey Ivanov} 

Україні не потрібні ЛДНР, їй потрібно повернути ОРДЛО. Ваші побажання
прибічникам ЄС в Україні переадресуйте прибічникам Росії на Донбасі та в Криму.
Всі, кому подобається Росія більше ніж Україна, можуть туди їхати зі своїми
валізами, але не з українською територією. Варто згадати приклад судетських
німців. А соціологічні опитування показують, що більшість населення України
підтримує євроінтеграцію України, і 48\% - за вступ до НАТО. Завдяки діям Росії
ці показники лише зростатимуть.

\iusr{Алексей Иванов}
\textbf{Сергий Федорынчык} 

какая такая украинская территория, есть украденная территория
Донецко-Криворожской ре юспублики присоединенная Лениным и Сталиным к Украине.
Ленина и Сталина вы считаете преступниками, сносите им памятники. Так вот
крупнейший им памятник это границы Украины.

\iusr{Сергий Федорынчык}
\textbf{Юлия Коломойцева} 

А чому увесь світ має бути дурним і не розуміти власної вигоди? Україна- не
Росія, і має право на самостійне існування. Попри купу негараздів, з часом
Україна стане країною благополучною, а Донбас буде найбільш квітучою її
частиною. Як у Австралію колись зсилали каторжників з Британії, а нині вона
серед найуспішніших країн.

\end{itemize} % }

\iusr{Василий Январев}

Ну это окно возможностей так себе - Украине выгоднее потерять ОРДЛО, чем
выполнить Минск.

\begin{itemize} % {
\iusr{Алексей Иванов}
\textbf{Василий Январев} пусть теряет.

\iusr{Сергий Федорынчык}
\textbf{Василий Январев} 

Україна не буде втрачати ОРДЛО, бо "Найкращі українські націоналісти - з
Донбасу!" (історик руху опору, радянський політв'язень Василь Овсієнко). В
майбутньому Донбас стане найбільш квітучим регіоном України. Мінські угоди були
підписані представниками України під тиском російської зброї біля Іловайську та
Дебальцева, але написані, начебто на Донбасі - внутрішній конфлікт. Вони
брехливі за формою і за змістом, сама Росія не збирається їх виконувати своїми
маріонетками, а себе ліцемірно проголошувалає, начебто не є стороною конфлікту.
З моменту видачі десятків тисяч російських паспортів "ополченцам" це є прямою
брехнею.

\end{itemize} % }

\iusr{Игорь Шевченко}
Игорь, это случайно или целенаправленно Вы сравниваете таких \enquote{великих
стратегов} как Жуков(судя по фото) и Зеленского!?

\begin{itemize} % {
\iusr{Игорь Лесев}
\textbf{Игорь Шевченко} Жуков моль в сравнении с нашим зеленым Боунопарте

\iusr{Игорь Сидоренко}
Игорь, Джулио Чезаре, я бы сказал, так как может делать одновременно делать несколько дел- сажать миллиард деревьев, строить университет будущего и расчищать колумбийский снег! @igg{fbicon.face.smirking} 
\end{itemize} % }

\iusr{Станислав Шкурат}

И эти люди решают нашу судьбу?!! \enquote{Во время светской беседы на приеме с
герцогиней Корнуоллской 78-летний американский президент «пустил ветры»,
заставив супругу принца Чарльза покраснеть. Герцогиня Камилла была в шоке,
когда Байден «пустил ветры», при этом продолжил вести светские беседы.} /
Издание Mail on Sunday./


\iusr{Любовь Чуб}
И аналитик прекрасный, и словом владеете суперово!
Спасибо!

P.S. А договорённости там, конечно же, были, о своём, о \enquote{девичьем})). И
это совсем не об Украине, о которой два часа говорили))

\iusr{Михаил Омский}
Как всегда исчерпывающе. Перепост.

\iusr{Oleg Ivanov}
По поводу \enquote{третьего} - полагаете они искренни в своих удивлениях?

\iusr{Юрий Зыбин}

Прелестно Шедеврально Серьезно и Глубоко !!! Я то ждал другого Думаю посмотрит
Игорёк Кузичева и Скабееву да и ударит сарказмом по нашим ушам Когда Солидно и
Глубоко Браво Вот кто должен быть Советником нашему а не как их там черепица и
хламидия !!! Для такой статьи - бриллианта нужна дорогая оправа - наши
комментарии со знаком качества Не судите старика строго Но начну я со стихов
народного творчества Опана да опана Закололи кабана Пили - ели Веселились
Подсчитали - Прослезились !!! Как раздували щёки Да весь Мир с нами А сухом
остатке Громко пукал дедушка на нас с высокой колокольни И зелёный Свет
Вонючему Газу И приверженность Минским соглашениям Я нисколько не злорадствую Я
в глубоком трауре Всё это отразится на всех нас И на титульных и на обрезанных
и на некоренных Будем скоро сидеть и в холоде и в темноте Вот сижу kurwa шью
себе носки с рукавов старого свитера СПАСИБО НАШЕМУ КОРМЧЕМУ ЗА ЗАБОТУ О ЛЮДЯХ
ПОВАЖНОГО ВИКУ СУГС 3 раза

\iusr{Андрей Ювченко}
Фотка с Халкин-Гола зачетная.
Как раз про наши дела  @igg{fbicon.smile} 

\iusr{Александр Белинский}

По-моему это было просто \enquote{протокольное мероприятие}, которое ни на что не
повлияло. Исходя из бессмысленного, но громкого шума вокруг него.

\iusr{Вячеслав Савченко}

анализ отличный, спасибо. Как всегда - удовольствие от чтения. Одного вы упорно
не желаете признавать, что \enquote{принципиально другой} Украина стала в 10-м, а 14-й
- то уже следствие 10-го. Именно с 10-го по 13-й производная развития страны
стала отрицательной, людей, которые еще что-то могли и умели стали выводить на
маргинес и таки вывели окончательно и... И понеслась.

Тогдашний сЦарь одно попытался сделать верно - отменить \enquote{Ассоциацию}, но было
уже поздно: вор таки должен сидеть в тюрьме, а самостоятельную политику ему
проводить никто не даст.

\end{itemize} % }
