%%beginhead 
 
%%file 14_04_2021.fb.fb_group.mariupol.nekropol.1.otchet_volonter_rabota_10_aprelja
%%parent 14_04_2021
 
%%url https://www.facebook.com/groups/278185963354519/posts/502564810916632
 
%%author_id fb_group.mariupol.nekropol,arximisto
%%date 14_04_2021
 
%%tags 
%%title Отчет о волонтерской работе в Некрополе 10 апреля 2021 года
 
%%endhead 

\subsection{Отчет о волонтерской работе в Некрополе 10 апреля 2021 года}
\label{sec:14_04_2021.fb.fb_group.mariupol.nekropol.1.otchet_volonter_rabota_10_aprelja}
 
\Purl{https://www.facebook.com/groups/278185963354519/posts/502564810916632}
\ifcmt
 author_begin
   author_id fb_group.mariupol.nekropol,arximisto
 author_end
\fi

\vspace{0.5cm}
\textbf{Отчет о волонтерской работе в Некрополе 10 апреля 2021 года}

В прошедшую субботу волонтеры возобновили регулярные работы по благоустройству
и исследованию Некрополя.

\textbf{Открытия и находки}

В самом начале древней аллеи обнажился неизвестный фундамент – то ли
разрушенной церкви Всех святых, то ли склепов. Загадка, которую постараемся
разгадать...

На этом же участке искали место для посадки платанов – наткнулись на
известняковые камни. Их предназначение неизвестно...

Елена Сугак сделала десятки фото надгробий на участке, который был расчищен
Комбинатом коммунальных предприятий в прошлом году (см. фото и связанные с ними
обсуждения в волонтерской группе \enquote{Мариупольский Некрополь: убираем, исследуем,
восстанавливаем}).

И – самое печальное открытие. Неделю назад Иван Таушан, внучатый племянник
Виктора Михайловича Арнаутова, показал нам могилу Аделаиды Арнаутовой, мамы
знаменитого художника-муралиста. Его супруга Нона Талепоровская захоронила в
ней прах Арнаутова после его смерти и кремации в Ленинграде в 1979 году.

Могила находится недалеко от усыпальницы Найденовых на участке, расчищенном
Комбинатом. Это почти незаметный холмик без таблички, под битым кирпичом
посреди шприцов и бутылок. Мы смонтируем, опубликуем видео в отдельном посте.
Стыд и позор...:(

\textbf{Благоустройство}

Мы приступили к очистке от зарослей пространства между памятниками Александра
Хараджаева, городского головы в 1860-х, и Николая Фененко, историка,
руководителя отдела прессы мариупольского провода ОУН во время нацистской
оккупации города.

Цель этой работы – полностью благоустроить этот древний участок Некрополя
(уборка, высадка деревьев, прокладывание дорожек и т.д. – на основе детального
обследования участка).

Как случалось и в прошлом году, под кучами мусора обнажились остатки старинных
блоков или плит. Их исследование – задача следующих экспедиций.

Илья Луковенко расчистил от земли плиту Сарры Гоф (1788-1858), мариупольской
купчихи, бабушки Гавриила Гофа, владельца дома, в котором сейчас находится
редакция \enquote{Приазовского рабочего}. Как и большинство древних плит, она каждый
год \enquote{заплывает} землей. Их нужно поднимать над землей и ставить на прочное
основание, чтобы не потерять...

На расчищенных местах мы посадили саженцы клена красного и дуба \enquote{Quercus
rubra}, которые были закуплены благодаря пожертвованиям. Таким образом, на
сегодня мы высадили на месте вырубленных зарослей и сухостоя десять деревьев.
Из восьми кленов, подаренных прошлой осенью ландшафтным дизайнером @Stanislav
Kosonogov (Березовским), успешно пережили зиму семь саженцев.

Голубые мускари, посаженные благодаря пожертвованиям вокруг склепа Спиридона
Гофа, уже взошли. Ждем цветов 🙂

\textbf{Использование пожертвований}

Вскоре мы опубликуем специальный отчет об оставшихся пожертвованиях и закупках,
которые мы сделали накануне запуска нового волонтерского сезона в Некрополе.

Большое спасибо всем волонтерам-участникам! И - нашим благотворителям!

Друзья, на следующих выходных мы снова собираемся в 12:00 в субботу и
воскресенье (анонс опубликуем позже). Мы планируем убрать мусор вокруг могилы
Аделаиды Арнаутовой... Присоединяйтесь!

\#mariupol\_necropolis\_report
