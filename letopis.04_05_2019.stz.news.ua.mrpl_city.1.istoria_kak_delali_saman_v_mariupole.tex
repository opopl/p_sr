% vim: keymap=russian-jcukenwin
%%beginhead 
 
%%file 04_05_2019.stz.news.ua.mrpl_city.1.istoria_kak_delali_saman_v_mariupole
%%parent 04_05_2019
 
%%url https://mrpl.city/blogs/view/istoriya-kak-delali-saman-v-mariupole
 
%%author_id burov_sergij.mariupol,news.ua.mrpl_city
%%date 
 
%%tags 
%%title История: как делали саман в Мариуполе
 
%%endhead 
 
\subsection{История: как делали саман в Мариуполе}
\label{sec:04_05_2019.stz.news.ua.mrpl_city.1.istoria_kak_delali_saman_v_mariupole}
 
\Purl{https://mrpl.city/blogs/view/istoriya-kak-delali-saman-v-mariupole}
\ifcmt
 author_begin
   author_id burov_sergij.mariupol,news.ua.mrpl_city
 author_end
\fi

\ii{04_05_2019.stz.news.ua.mrpl_city.1.istoria_kak_delali_saman_v_mariupole.pic.1}

В первые послевоенные годы на Малофонтанной можно было увидеть такую картину:
прямо на улице, там, где она расширяется близ электростанции, вырыта неглубокая
яма, две-три женщины, надвинув чуть ли не на самые глаза туго повязанные на
головах косынки, в подоткнутых высоко юбках, впрочем, значительно ниже, чем
принято сейчас носить у некоторых современных модниц, топчутся, погружая
молочной белизны ноги до середины икр в коричневую густую жижу, которая жирно
чавкает с каждым шагом женщин.

Их товарка время от времени подсыпает в яму то желтоватый суглинок, то полову
из огромного мешка; полова, сверкая под солнечными лучами золотыми искорками,
озорно рассеивается ветром вокруг, прилипает к раскрасневшимся от жары и работы
потным лицам женщин. Когда чавканье под ступнями становится тягучим и вязким, в
яму подливают воду из шланга, змеящегося откуда-то из глубины двора.

\vspace{0.5cm}
\begin{minipage}{0.9\textwidth}
\textbf{Читайте также:}

\href{https://archive.org/details/24_11_2018.sergij_burov.mrpl_city.stirka_na_torgovoj_ulice_mariupolja}{%
Стирка на Торговой улице Мариуполя, Сергей Буров, mrpl.city, 24.11.2018}
\end{minipage}
\vspace{0.5cm}

Но вот смесь суглинка и половы достигла нужной однородности и консистенции,
женщины наполняют ею ведра и поднимают их из ямы наверх. Теперь они,
прихлопывая ладошками, заполняют остропахнущей массой деревянные формы на три
или четыре будущих кирпича, а затем дощечкой подравнивают верхнюю поверхность и
неуловимо ловким движением снимают форму. Ряд уже сформованных изделий
дополняется новыми. Некоторое время они блестят влагой боковых граней, но потом
влага испаряется, и только что сработанные кирпичи становятся незаметными среди
других.

Все эти работы происходят в декорациях, образованных \enquote{клетками}, сложенными из
сохнущих кирпичей. Знающие люди уже, без сомнения, догадались, что здесь
описано производство самана — строительного материала, из которого в свое время
была сооружена половина, если не больше, жилых строений Мариуполя. Правда, как
правило, саман изготавливали прямо на строительстве будущего дома, а на
Малофонтанной делали его на продажу. Таков был промысел у здешних жительниц.
Это были молодые вдовы и девушки, чьи мужья и женихи погибли на полях сражений,
в концлагерях врага.

\vspace{0.5cm}
\begin{minipage}{0.9\textwidth}
\textbf{Читайте также:}

\href{https://archive.org/details/15_12_2018.sergij_burov.mrpl_city.kak_otaplivali_doma_v_mariupole_v_byloe_vremja}{%
Как отапливали дома в Мариуполе в былое время, Сергей Буров, mrpl.city, 15.12.2018}
\end{minipage}
\vspace{0.5cm}
