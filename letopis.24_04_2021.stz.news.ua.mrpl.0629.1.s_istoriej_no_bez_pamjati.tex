% vim: keymap=russian-jcukenwin
%%beginhead 
 
%%file 24_04_2021.stz.news.ua.mrpl.0629.1.s_istoriej_no_bez_pamjati
%%parent 24_04_2021
 
%%url https://www.0629.com.ua/news/3085392/s-istoriej-no-bez-pamati-kak-mariupolu-vernut-staryj-gorod-foto
 
%%author_id news.ua.mrpl.0629
%%date 
 
%%tags 
%%title С историей, но без памяти. Как Мариуполю вернуть "Старый Город", - ФОТО
 
%%endhead 
 
\subsection{С историей, но без памяти. Как Мариуполю вернуть \enquote{Старый Город}, - ФОТО}
\label{sec:24_04_2021.stz.news.ua.mrpl.0629.1.s_istoriej_no_bez_pamjati}
 
\Purl{https://www.0629.com.ua/news/3085392/s-istoriej-no-bez-pamati-kak-mariupolu-vernut-staryj-gorod-foto}
\ifcmt
 author_begin
   author_id news.ua.mrpl.0629
 author_end
\fi

\ii{24_04_2021.stz.news.ua.mrpl.0629.1.s_istoriej_no_bez_pamjati.pic.1}

\begin{quote}
\em
История Мариуполя насчитывает, как минимум, 243 года (на самом деле – более 300
лет). Но в городе фактически нет того, что можно называть его старой частью. То
есть Старый Мариуполь, конечно, еще есть, но он никогда не считался какой-то
особенной ценностью. Его перестраивали, перекраивали, перепродавали. И
разрушали!  Скандальная ситуация с застройкой Городского сада подтолкнула и
общественность, и власть, наконец, начать дискуссию о том, кто мы есть, кто
наши предки, чем мы гордимся и дорожим, и как сохраняем своё. 

\enquote{Мы должны вынести вопрос об исторической части Мариуполя в отдельную
дискуссию, - заявил городской голова Вадим Бойченко. – Проблемы, которые были
подняты на встрече с краеведами и историками, не должны быть забыты только
потому, что Метинвест изменил планы по застройке. Мы должны начать широкое
обсуждение этой проблемы, чтобы выйти на разработку стратегии сохранения
исторического наследия: что нужно сохранить, какие символы превозносить, как
решать проблемы с частной собственностью}.
\end{quote}

\subsubsection{Где мы находимся сейчас}

В вопросе сохранения исторического наследия Мариуполь находится почти в нулевой
точке. Все, что делалось до этого момента, - делалось историками и краеведами
на общественных началах. Никакой системной работы властными органами по
спасению объектов, которые могли бы получить статус исторического или
культурного памятника, не велось.

В Мариуполе никогда не было раньше и не существует сейчас ни отдела, ни
управления в исполкоме, который бы занимался охраной объектов исторического
наследия. Правда, когда-то у нас существовал Градостроительный совет при
управлении архитектуры. И в него входили архитекторы, дизайнеры, художники и
чиновники. Они в числе прочего контролировали, чтобы любые перестройки фасадов
зданий в центре города осуществлялись бизнесом только после утверждения
Градостроительным советом. Однако такой орган  мешал строить \enquote{сумки} и
\enquote{шаурмы} посреди тротуаров, зашивать исторические фасады пластиком и
пр., поэтому очень скоро совет прекратил свое существование. И архитектурная
вакханалия продолжилась.

На сегодняшний день в городе нет даже реестра объектов – домов, деревьев,
возможно, кирпичей (старинной кладки), люков или оконных решеток, возраст
которых 100 и более лет,  которые могли бы представлять  историческую ценность.

Результат такой политики мы видим сейчас. В Мариуполе (за единичными
исключениями) все старинные здания находятся в частной собственности, а их
состояние граничит с критическим.

\subsubsection{Что делать дальше}

Мы обсуждали этот вопрос с мариупольскими архитекторами, краеведами,
историками. И вот что они советуют.

Нужно создать-таки при управлении архитектуры отдел сохранения исторического
наследия. Он должен заняться составлением реестра всего, что можно и нужно
сохранять. Составить паспорта таких объектов с полной информацией: состояние,
степень разрушения, потребность в реставрации, кто собственник и так далее. 

После этого начать планомерную работу по смене статуса этих объектов. Их нужно
включить в список объектов исторического наследия местного значения. И это дает
право требовать от собственника приведения его в порядок. Если собственник не
исполняет требования Закона Украины об охране культурного наследия,
муниципалитет вправе выкупить такой объект у собственника по остаточной
стоимости. 

Собственно, такой подход наш муниципалитет уже применил по отношению к пяти
старинным зданиям: бывший винный склад (ликероводочный завод), дом Нильсена,
синагога, дом со львами и дом Гампера.

И если всё получится, по такому же принципу нужно будет действовать системно,
создавая целый исторический ареал.

\subsubsection{А что конкретно спасать?} 

Этот вопрос, на самом деле, намного сложнее, чем может показаться на первый
взгляд. Если спросить  об этом у любого прохожего, то он, скорее всего, сразу
же назовет улицы Митрополитскую, Итальянскую, Земскую и другие улицы в центре.
Но историк \textbf{Ольга Демидко} говорит, что улицы, имеющие историческую ценность,
есть в каждом районе города.

\begin{quote}
\em\enquote{Історична частина Маріуполя обмежена з півдня Азовським морем, зі сходу -
річкою Кальміус. З північного заходу кордон старого міста визначає проспект
Металургів, до революції - Бахмутська дорога, яка починається у самого моря.
Північносхідна межа проходить по проспекту Шевченка, початок - вул.
Бахчисарайської (зараз - бульвар Шевченка). 

Насправді, визначити всю історичну частину складно. Слід враховувати час
заснування вулиці, архітектурну та культурну спадщину. Наразі у кожному районі
міста є вулиці,  які є історично важливими, зберігають свої легенди, загадки і
таємниці. Звісно, всім одразу приходять на думку вулиці Митрополитська,
Георгіївська, Харлампієвська, Фонтанна,  Італійська, Грецька, Земська, Миру,
Нільсена, Куїнджі... Але, на мою думку, краще створити спеціальну комісію, чи
окреме засідання, на якому б історики, музейні працівники та краєзнавці
визначили б всі вулиці в кожному районі, які є історично важливими.  Зокрема, в
Кальміуському районі це проспект Нікопольський,  в Приморському - вулиця
Львівська. Слід враховувати і території, які є парковими, але мають багату
історію (Міський сад) чи археологічними артефактами (Курган ДІД, місце
Кальміуської фортеці за ДТСААФом). Також важливо не забувати про історичну
частину населених пунктів, що входять до Маріуполя (Сартана, Старий Крим тощо).
На мою думку, це питання потребує більш докладного вивчення і дуже серйозного
підходу}.
\end{quote}

С ней согласен и \textbf{Андрей Марусов}, глава общественной организации \enquote{Архи-Город}.
Тема действительно требует серьезного подхода. Но он подчеркивает: исторический
ареал в Мариуполе  уже давно определен.

\enquote{Исторический ареал Мариуполя был определен харьковскими архитекторами с
участием городских историков еще в 1990-х годах. В проекте
историко-архитектурного опорного плана, разработанном по заказу мэрии в
2018-2020 гг киевским Институтом культурного наследия, он был уточнен}, -
говорит Андрей Марусов.

\enquote{Что такое \enquote{исторический ареал поселения}?  Это часть поселения, которая
сохранила объекты культурного наследия и связанные с ними планировку и форму
застройки, которые происходят из предыдущих этапов развития, типичные для
определенных культур или периодов развития} (ст.1 ЗУ \enquote{Об охране культурного
наследия}). 

\enquote{В упомянутом проекте исторический ареал имеет следующие границы (см. карту
ниже): улица Нильсена – улица Фонтанная – улица Харлампиевская – улица
Евпаторийская – улица Торговая – улица Малофонтанная – улица Итальянская –
улица Торговая – улица Семенишина (до пересечения с улицей Нильсена)}.

\ii{24_04_2021.stz.news.ua.mrpl.0629.1.s_istoriej_no_bez_pamjati.pic.2}

Однако на карте зонирования Мариуполя, размещенной на сайте горсовета, зона
охраны памятников архитектуры намного меньше. То есть необходимо менять карту
зонирования.

\ii{24_04_2021.stz.news.ua.mrpl.0629.1.s_istoriej_no_bez_pamjati.pic.3}

Но и это только половина проблемы. Потому что, как утверждает Марусов, сам по
себе исторический ареал не защищает здания от разрушения. Необходимо
присваивать каждому исторически ценному объекту статус памятника архитектуры
или культурного памятника местного значения.

\enquote{Статус памятки,  - поясняет Андрей Марусов, - это охранная грамота, которая
защищает историческое здание от разрушения или изменения. Только официальный
статус памятки является такой гарантией. На сегодня в городе всего 9 зданий
имеют статус памятки/объекта культурного наследия местного значения. Между тем,
авторы проекта опорного плана рекомендовали предоставить такой статус 74
зданиям}. 

И это далеко не полный перечень. Чтобы сохранить эти и другие объекты, нужно подать заявку. 

\enquote{Сделать это могут департамент культурно-общественного развития Мариупольского
горсовета или общественные организации, среди уставных задач которых значится
\enquote{охрана культурного наследия} (например, \enquote{Архи-Город} инициирует присвоение
статуса памятки Первой гимназии, бывшей Мариинской женской гимназии)}, -
говорит  Марусов.

Вся процедура оформления документации – недешевое удовольствие. 

Например, согласно тендеру департамента на разработку учетной документации для
синагоги на Георгиевской, дома Нильсена и винного акцизного склада, средняя
стоимость работ составляет около 29 тыс. грн. 

\enquote{Это означает, что защита тех же 74 исторических зданий обойдется горожанам,
плательщикам налогов, в сумму около 2,1 млн. грн}, - говорит лидер
\enquote{Архи-Города}. Именно поэтому он предложил для начала мэрии предоставить статус
памятников архитектуры хотя бы 30 историческим зданиям. А приоритетность
определять в зависимости от  уникальности архитектуры, включенности зданий в
существующие туристические маршруты (например, здания по \enquote{музейной}
Георгиевской улице), близости к уже действующим культурным магнитам города  или
из-за высокого риска разрушения,  утраты исторического облика. Прежде всего,
это касается усадеб и особняков, которые ныне пребывают в частной
собственности.

\subsubsection{Как спасать}

Главная проблема, в которую может упереться город на пути спасения своего
культурного и исторического наследия,   - частные собственники.

Подавляющая часть исторических зданий в Мариуполе находится в частной
собственности. Что делают частники со своими старинными домами? Ну, некоторые
просто бросают их и исчезают из Украины. Это совсем критический случай.

\ii{24_04_2021.stz.news.ua.mrpl.0629.1.s_istoriej_no_bez_pamjati.pic.4}

Другие здесь, на месте, поступают с культурным и архитектурным наследием, как
манкурты.

Посмотрите на это здание по ул. Георгиевской, 16, рядом с синагогой и
краеведческим музеем. Никто не помешал собственнику изуродовать его, достроить
отвратительную крышу и долепить крыло.

\ii{24_04_2021.stz.news.ua.mrpl.0629.1.s_istoriej_no_bez_pamjati.pic.5}

Здания перекрашивают в разные цвета (и это полбеды).

\ii{24_04_2021.stz.news.ua.mrpl.0629.1.s_istoriej_no_bez_pamjati.pic.6}

Хуже, когда старинные фасады заклеивают дешевой плиткой, или просто зашивают
безобразными пластиковыми или металлическими листами.

\ii{24_04_2021.stz.news.ua.mrpl.0629.1.s_istoriej_no_bez_pamjati.pic.7}

Демонтируют оригинальные балкончики со старой лепниной, заменяя на железо.

\ii{24_04_2021.stz.news.ua.mrpl.0629.1.s_istoriej_no_bez_pamjati.pic.8}

К старинным домам достраивают пристройки из ДСП, старой фанеры и досок,
\enquote{пришивают} кирпичные пристройки прямо на центральный фасад и прочее, прочее.

\ii{24_04_2021.stz.news.ua.mrpl.0629.1.s_istoriej_no_bez_pamjati.pic.9_11}

Пройдитесь по Греческой, Земской, Пушкина... Если увидите кусок старой арки, знайте, это старая постройка 19 века.

\ii{24_04_2021.stz.news.ua.mrpl.0629.1.s_istoriej_no_bez_pamjati.pic.12}

Загляните во дворик – и вы увидите остатки домов 19 века, построенных греками.
В некоторых местах сохранилась и кладка, и песчаник.

\ii{24_04_2021.stz.news.ua.mrpl.0629.1.s_istoriej_no_bez_pamjati.pic.13}
\ii{24_04_2021.stz.news.ua.mrpl.0629.1.s_istoriej_no_bez_pamjati.pic.14}

И это на самом деле не менее ценно, чем дорогая лепнина в Одессе или Киеве. И
это надо охранять. 

Но как заставить собственника не рушить? (На фото ниже – пример правильного
отношения владельца к реставрации).

\ii{24_04_2021.stz.news.ua.mrpl.0629.1.s_istoriej_no_bez_pamjati.pic.15}

Когда-то известный мариупольский архитектор Людмила Малокуцко рассказывала
0629, что работает над историческим восстановлением улиц и домов Старого
Мариуполя. Она говорила, что занимается тем, что по документам, сохранившимся
описаниям, воспоминаниям людей, по срезу старой штукатурки определяет, какими
были дома в исторической части Мариуполя, вплоть до того, в какой цвет они были
покрашены. Она мечтала, что когда-то придет время, и ее работа воплотится в
жизнь. И горожане смогут увидеть кусочек старого купеческого Мариуполя таким,
каким он был в 19 – начале 20 века.

Восстанавливать исторический ареал можно секторами. Разделить его на кварталы.
Сделать для примера хотя бы один. Посмотрев на результат, не исключено, что
найдутся желающие инвестировать в реновацию Мариуполя. Ведь хотели же в конце
90-х итальянцы вложить деньги в реконструкцию части улицы Итальянской в
Мариуполе. Тогда они не нашли поддержки в мэрии Юрия Хотлубея. Но ведь сейчас
ситуация в городе другая.

\enquote{Надо определить наиболее ценный с туристической точки зрения квартал и
начать работы. Не обязательно реконструировать всё. Можно делать внешний
периметр. При таком подходе интересы людей, живущих в эти домах, не пострадают.
И права собственности не будут ущемлены. Можно в некоторых случаях оставлять
внешний исторический фасад, создавая внутри дворов уютные домики для жизни
людей. Можно сохранять не всю кладку или здание, а только часть. Так делают во
многих европейских странах. Решение надо находить в каждом индивидуальном
случае}, - говорят мариупольские архитекторы.

Главное – начать эту работу сейчас, пока еще есть что сохранять.

\begin{leftbar}
\em
\enquote{Историческая зона Мариуполя научно определена в 1989 году. С того времени
ничего не было сделано, чтобы сохранить историческое лицо города. Ряд домов,
входящих в зону, разрушены или изуродованы пристройками}.
\bigskip

\textbf{Сергей Буров, краевед и публицист}
\end{leftbar}
