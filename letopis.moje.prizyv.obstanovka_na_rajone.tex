% vim: keymap=russian-jcukenwin
%%beginhead 
 
%%file moje.prizyv.obstanovka_na_rajone
%%parent moje.prizyv
 
%%url 
 
%%author_id 
%%date 
 
%%tags 
%%title 
 
%%endhead 

\subsubsection{Обстановка вокруг на районе}

У нас относительно тихо на данный момент,
и слава Богу. Есть электричество, водопроводная вода, у нас есть запасы
продуктов на несколько недель, так что мы пока что еще не умираем с голоду, и
вообще живы. Просто живы, и это уже круто.  В наш дом не попала ракета, как в
высотный дом по Лобановского 6 (называвшийся раньше проспект Краснозвездный),
вокруг пока что не свистят пули, и знаете, такое ощущение, что вообще почти
ничего не происходит, если не читать сообщения из интернета и телеграма.  Днем
светит солнышко над нашим Вечным Городом, Городом Ярослава Мудрого и Владимира
Крестителя... Городом Булгакова и Пушкина, Гоголя и Высоцкого, Анны Ахматовой и
Николая Амосова... Тысячелетним Городом Киевом, в котором столько всего
происходило на протяжении прошлых веков и столетий, Городом... который столько раз был в
огне, который столько раз подвергался разорению и разграблению, и тем не менее,
он каждый раз возрождался. Возрождался, чтобы стать еще красивее. Возрождался... 
просто потому что это - Киев, и этим уже все сказано. В 1240 году
этот Город был сожжен дотла нашествием Хана Батыя... а сейчас... кто помнит о
татаро-монголах?  Да никто...  потому что Город возродился, и стал еще краше, и
намного больше, чем во времена Батыя.  Никто не помнит... кроме пожелтевших
страниц летописей, и кроме самого Города, потому что Город помнит все... Дело
давно минувших дней, преданья старины глубокой, как писал когда-то Пушкин... И
несмотря на войну, мы надеемся, он останется таким же красивым, как и раньше.
Что его не успеют уничтожить в нынешней войне, и что разум восторжествует над
безумием.  И кстати, Киев - это Город, который во многом строила Российская
Империя, если вы помните историю, - так, например, Андреевскую церковь в свое
время построил архитектор Б. Ф. Растрелли по приказу Императрицы Елизаветы
Петровны в 1749-1754 годах на том месте, где, по преданию, апостол Андрей
Первозванный во время его путешествия воздвиг крест. Как вы может быть знаете,
об этом есть запись в Повести Временных Лет, которую давным-давно написал наш
преподобный Нестор Летописец, чьи мощи на протяжении тысячелетий лежат в
ближних пещерах Киево-Печерской Лавры, и куда мы с удовольствием периодически
ходим, держа свечки в руках. Вы думали, что Нестор Летописец мифический
персонаж, выдумка? Так нет, он лежит себе, лежит, там где надо.  И Может быть,
знаете, еще в этих пещерах лежат богатырь Илья Муромец, а также Агапит
Печерский, врач безмездный, и много-много других святых, которые в свое время
всю жизнь проводили в пещерах и затворах, молясь Богу за других людей и за
спасение своих душ.

\ifcmt
  tab_begin cols=3,no_fig,center
     pic https://avatars.mds.yandex.net/i?id=f6d73333003dd92e4f1326c33f8df84c-5693613-images-thumbs&n=13
		 @caption Андреевская церковь, город Киев

		 pic https://medru.su/wp-content/uploads/2019/09/41_6_crm.jpg
		 @caption Императрица Елизавета Петровна

		 pic https://www.culture.ru/storage/images/f4014625-370f-5b3d-bc08-197aebf308a8
		 @caption Франческо Бартоломео Растрелли
  tab_end
\fi

Так вот. Снаружи поют птички, предвещая наступления скорой весны... как будто
жизнь идет своим чередом до начала войны, как будто вообще ничего не
происходило. Кроме того, что улицы пустынны, что наш район опустел, нам
кажется, что как будто все такое же самое. Кроме того, что... вдалеке воет
сирена, и слышны звуки взрывов.  И если не брать во внимание сирену, которая
призывает горожан спускаться в бомбоубежище, то может показаться... что это
просто обычный салют. У нас обычно в Киеве на праздники делают классные салюты,
ну например, на День Киева, или же когда мы празднуем победу футбольного клуба
Динамо Киев. Кроме того, у нас также есть гостиница Салют, у нее немного
вычурная архитектура, и она находится возле Дворца Детей и Юношества (Київський
палац дітей та юнацтва), возле Площади Славы, там, где также есть Аллея Славы и
Вечный Огонь Неизвестному Солдату. Чуть далее находится Музей Голодомора, - а
еще чуть дальше, - если проехать по улице Лаврской, - будет Киево-Печерская
Лавра и музей Великой Отечественной Войны, где стоит высоченный монумент
Родина-Мать, которую в народе...  называют по-всякому. Ну а если спуститься вниз по дороге,
вы увидите памятник Андрею Первозванному, и также церковь Андрея Первозванного, рядом же 
парк Аскольдова Могила.

\ifcmt
  tab_begin cols=3,no_fig,center
     %pic https://reservehall.com/images/restourants/initially/Sallut_17_1551951024.png
		 pic https://novate.ru/files/u34692/TheHotelSalutinKiev.jpg
		 @caption Гостиница Салют, город Киев

		 pic https://sun9-68.userapi.com/c636322/v636322902/24dc8/r1qc77wD26c.jpg
		 @caption Дворец Детей и Юношества, город Киев

		 %pic https://horde.me/uploads/temp/fb/fb3e135a21f45e05b89bad0e0fb43392.jpeg
		 pic https://horde.me/uploads//temp/fd/fd0c17c465ccc7f87980b0cd6b4de269.jpeg
		 @caption Парк Славы, памятник Ивану Никитовичу Кожедубу, город Киев

  tab_end
\fi

