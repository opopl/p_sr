% vim: keymap=russian-jcukenwin
%%beginhead 
 
%%file 15_04_2022.fb.kernytska_kristina.1.jdyt_najkraschi
%%parent 15_04_2022
 
%%url https://www.facebook.com/kernitskaya.kristina/posts/3120571314874446
 
%%author_id kernytska_kristina
%%date 
 
%%tags 
%%title Йдуть найкращі...
 
%%endhead 
 
\subsection{Йдуть найкращі...}
\label{sec:15_04_2022.fb.kernytska_kristina.1.jdyt_najkraschi}
 
\Purl{https://www.facebook.com/kernitskaya.kristina/posts/3120571314874446}
\ifcmt
 author_begin
   author_id kernytska_kristina
 author_end
\fi

Сьогодні дізналася про смерть свого друга Андрія Кравченка, хочу вам розповісти
про цю дивовижну людину.

Одного разу Андрій запросив у свій коледж нас з
\href{https://www.facebook.com/alex.smirnov.5201}{Alex Smirnov}, щоб ми
показали виставу Театру Науки, я ніколи не забуду з якою любов'ю до Андрія
зверталися його учні, якими очима вони на нього дивилися. Він був не просто
вчителем візики для дітей, а й старшим братом, другом, наставником.

\ii{15_04_2022.fb.kernytska_kristina.1.jdyt_najkraschi.pic.1}

Андрій був люблячим батьком, пам'ятаю, як він разом з донькою дотепно жартували
на грі Розумна Родина, а його дружина уважно контролювала, щоб всі
науково-популярні експерименти були продемонстровані якнайкраще.

А ще Андрій Кравченко офіційно \enquote{Народний герой України}, учасник боїв біля
Дебальцево. Ось що про ті події писав заступник начальника штабу групи
«Північ», підполковник Віктор Кевлюк:

«Українська артилерія прикривала відхід до останнього, застосовувався досить
складний тактичний прийом «вогневий контроль». Один з корегувальників
ар’єргарду виконував завдання. Під час бою отримав осколкове поранення та
контузію. Незважаючи на це, він ще кілька годин корегував вогонь. Запитую в
офіцерів штабу 25-го окремого мотопіхотного батальйону: «Корегувальник –
офіцер?». Відповідь: «Ні. Він рядовий. Науковець». Громадо! Хто вважає
українських науковців «ботанами», хай знає, що це найгероїчніші «ботани» у
світі!».

\ii{15_04_2022.fb.kernytska_kristina.1.jdyt_najkraschi.pic.2}

Так, Андрій Кравченко ще й науковець, кандидат хімічних наук, розробляв разом з
колегами Інституту хімії поверхні імені О.О. Чуйка НАН України недорогі
гемостатики.

Андрій не зміг просто сидіти вдома, коли почалася повномасштабна війна з
Росією. 3 квітня він виїхав на автомобілі до Броварського району та зник.
Виявилося, що він підірвався на міні. 

Йдуть найкращі...

\ii{15_04_2022.fb.kernytska_kristina.1.jdyt_najkraschi.cmt}
