% vim: keymap=russian-jcukenwin
%%beginhead 
 
%%file 23_04_2022.fb.filatov_boris.1.tjazheloje.cmt
%%parent 23_04_2022.fb.filatov_boris.1.tjazheloje
 
%%url 
 
%%author_id 
%%date 
 
%%tags 
%%title 
 
%%endhead 
\zzSecCmt

\begin{itemize} % {
\iusr{Pavlo Kovalenko}

Думаю, если бы вы вышли из машины и показательно, но вежливо, пристыдили бы
кого-то в таком плюшевом костюме. И это все сняли на камеру, то эффект был-бы
еще более продуктивный, чем от поста. Но в целом, конечно, и пост понятный и
правильный. Наверное это хорошее задание для региональных журналистов в тыловых
городах, что бы сделать репортажи про эту проблему.

\begin{itemize} % {
\iusr{Borys Filatov}
\textbf{Pavlo Kovalenko} Можно без бессмысленных советов?
Хотите сделать репортаж ? Пойдите и сделайте.
А не тратьте время на комментарии в ФБ

\iusr{Pavlo Kovalenko}

Нет скилов чтобы сделать самостоятельно. Поэтому подумал, что среди вашей
аудитории могут быть журналисты из тех городов, где принимают беженцев и кого
заинтересует поднятая вами тема, для создания видеосюжета.

\end{itemize} % }

\iusr{Тихолаз Ігор}
Гадаю, що правильно.

\iusr{Boris Fel}
Добавлю к этим верным словам четкий плюс из Берлина.

\iusr{Anna Khripunkova}

Да. Да!!! Да!!!! А то я, видите ли, \enquote{просто завидую}, \enquote{просто ничего не
понимаю}, \enquote{люди поднимают экономику Украины}. Ну да, не то шо мы, сетки плетем,
фигня такая! Вот если бы мы айтишниками были, тогда да, а так фигня.

\begin{itemize} % {
\iusr{Yelyzaveta Zaitseva}
\textbf{Anna Khripunkova} 

мои друзья в Черновцах два дня плели сетки, потом им сказали больше не нужно,
на следующий день мы вместе провели благотворительный ивент для детей
переселенцев и после посидели в кафе в большой компании. Кто-то из них был в
плюшевом костюме. И что, будете и нас обсуждать? Все такие умные судить
человека по первому взгляду

\iusr{Anna Khripunkova}
\textbf{Yelyzaveta Zaitseva} 

странно, что больше не нужно, мы плетем сейчас для черновицких ребят. Но то
ладно, мы сплетем. У меня тоже костюм, только не плюшевый. Я тоже аж два раза
сидела в кафе. Я осуждаю за другое, думаю, понятно, за что.

\iusr{Yelyzaveta Zaitseva}
\textbf{Anna Khripunkova} так вы осуждаете кого не понимаю? Вы вот точно поименно знаете тех, кто не плетёт и сидит в кафе?

\iusr{Anna Khripunkova}
\textbf{Елизавета Зайцева} Да) Многих поименно знаю. Если вдруг у вас остались нитки для сеток, будем благодарны за них.

\iusr{Yelyzaveta Zaitseva}
\textbf{Anna Khripunkova} 

я бы ещё предложила посмотреть под другим углом на это: не у всех есть ресурс
вытягивать это все. У кого-то банально опускаются руки. Это конечно не хорошо,
но и осуждать за такое тоже некорректно. Каждый делает то, что может по мере
своих сил. Ниток для плетения нет, тут своих личных вещей раз два и обчелся

\iusr{Anna Khripunkova}
\textbf{Yelyzaveta Zaitseva} да, ниток нигде нет, увы... я с надеждой, а вдруг. У меня тоже нет ресурса вытягивать это все, вообще ноль, ещё после Донецка.
\end{itemize} % }

\iusr{Yelyzaveta Zaitseva}

Не люблю такие посты, если честно. Откуда вы взяли, что парень с ноутбуком не
перечисляет регулярно деньги на ВСУ? Или не бомбит в очередной раз сайт рашки?

Да и вообще очень часто волонтеры уже не нужны: мои друзья в Черновцах два дня
с утра до ночи плели сетки, а потом им сказали, что людей слишком много, до
свидания.

\iusr{Vlada Tymoschuk}
Однозначно согласна!!!!

\iusr{Alexandr Davydov}
Обмен опытом. Садовой: \enquote{У Вас 24 часа встать на воинский учёт!}.

\iusr{Arthur Berner}

Я таких называю налоговые агенты, они просто участвуют в экономике, так сказать
\enquote{аполитичные люди}, они не украинцы, такие себе бомжи мира.

\iusr{Viktoria Stepanenko}
Очень жаль, что так..

\iusr{Paul Angelsky}
Все по суті на жаль... дежавю 2014

\iusr{Акпер Гасанов}

Это правда. в Червоной Слободе, это рядом с Черкассами, тоже куча беженцев. И
многие- с фантастическими претензиями. Словно это черкасчане их в беженцев
превратили, а не Россия. При этом, впереди ждут еще большие проблемы. У тех,
кто сдал квартиры и дома беженцам . Они потому могут услышать о нежелании
покидать место ВРЕМЕННОГО проживания, пока их не обеспечат новым жильем. Там,
где они хотят. Такое, как им нравится. Я не сгущаю. Это реалии любой войны

\begin{itemize} % {
\iusr{Акпер Гасанов}

Но это, увы, еще не все За последние дни переговорил сразу с несколькими своими
родственниками, друзьями, знакомыми, которые работают или живут в Турции. Они
в шоке от происходящего! Россияне и украинцы изнасиловали турецкий рынок
недвижимости- аренды и продажи. Просто толпами валят. Особенно, в давно
облюбованные туристические регионы- Анталия, Алания, Бодрум, Кемер, меньше в
Стамбул и Измир. Самое забавное, что много россиян и из Сочи, из
Краснодарского, сука, края. Это как-бы не Сахалин. Это типа тоже, сука,
курорты. Но прут россияне в ненавистную, судя по телевизионным речам
пропагандонов. Турцию! ОК, с россиянами понятно, у них запрета на выезд из
страны нет. Но как на курорты Турции пробираются украинцы?! Нет, уточняю, это
не только женщины и дети. До хрена мужчин среднего, то есть лет 35-50,
возраста. А как же , мать вашу, запрет на выезд из страны тем, кому меньше 60?!
Причем, приезжают украинцы с НАЛИЧНЫМИ. \enquote{Куча набитых сумок с баблом}, это
цитата. Приезжают и просекают на дорогущих авто, стоимостью от 100 000 евро.
Как они их вывезли?! То есть, хотите увидеть украинскую коррупцию в действии-
езжайте на курорты Турции. Увидите кучу интересного. В том числе и совместное
распитие алкоголя в Махмутларе между украинцами и россиянами. Да, сейчас
услышу, что в семье не без уродов. ОК, согласен. Но на хрен тогда нужны все эти
правоохранительные, проверяющие органы?! Почему не снят, не привлечен к
уголовной ответственности глава погранслужбы Украины, глава таможни? Без их
согласия не смогли бы вывезти из Украины столько денег, не смогли бы убежать от
войны столько мерзавцев! И отсутствие наказания за очевидные факты коррупции на
крови, отталкивают даже тех, кто сочувствует Украине. Какого хрена, может
подумать условный азербайджанец, я буду отрывать от себя последнее, покупать
гумпомощь народу Украины, нести ее к посольству Украины, если тысячи и тысячи
украинцев покупают квартиры и виллы в Турции?! Остановите весь этот поток ,
потрясите своих гребаных олигархов и не будет необходимости в стоянии на
паперти у государства! Отсутствие такого рода шагов со стороны власти
обесценивают ее усилия по борьбе с российскими оккупантами. Я уже не говорю о
том, что рядовые турки, которые привыкли приезжать на лето в курортные регионы
СВРЕЙ страны для работы в отелях, кафе, ресторанах- они стонут. Они проклинают
россиян и украинцев одинаково. Потому, то по хрену, чьей вины больше в том, что
цены на аренду жилья выросли в два с лишним раза! Турки просто рвут и мечут от
того, что им теперь не по карману снять квартиру в СВОЕЙ стране, потому, что
Ваня и Тарас приехали с мешками бабла и все скупили, сняли, подняв цены до
небес. Таковы реалии. Позорные, замечу, реалии.

\iusr{Helena Aliieva}
\textbf{Акпер Гасанов} да ощущение, что эта Война не про вас. Братья мародеры

\iusr{Акпер Гасанов}
\textbf{Helena Aliieva} 

простите, а я тут при чем? Это не я деньги из страны вывожу. Я наоборот помогаю
, по мере сил, украинской армии. Но вам же не правда интересует, а повод для
спора

\iusr{Helena Aliieva}
\textbf{Акпер Гасанов} 

не надо обобщать. Я ехала в эвакуационном поезде из Харькова. Через многие
центры волонтёрские. Я видела этих людей из под бомбёжек. А здесь пытаются
представить всех переселенцев кончеными ублюдками. Что ж вы за люди? Цены
дерете за жильё и обсуждаете, что комфорт нарушили привычный. Когда прилетит
вам, тогда прозреете

\iusr{Акпер Гасанов}
\textbf{Helena Aliieva} 

кто дерет?! Вы в своем уме? Я не в Турции и я не сдаю квартиры. Моя семья их
Черкасс. Мы проехали через три границы- румынскую, болгарскую и турецкую.
При5хали в Баку. Я тоже все видел и все знаю. И я не обобщаю. Прлмто пишу, что
есть и такая проблема. Или вы ее отрицать будете? Ваше право. Но перед тем, как
писать о ком то, хотя бы зайдите на его страницу и попытайтесь понять, кто тот,
кого вы осуждаеете.

\iusr{Helena Aliieva}
\textbf{Акпер Гасанов} 

проблема в том, что не надо обобщать и нужно относиться к людям с милосердием.
В комментариях сплошной хейт. А в Черкассах активные боевые действия?

\iusr{Акпер Гасанов}
\textbf{Helena Aliieva} 

дорогая моя, я не обобщаю. Я просто указываю, что есть и такая проблема тоже.
Не пишу же я, что вы обобщаете, рассказывая о том, что все стали беженцами
налегке. Осевидно же, сто в Турции дома покупают те, кто уехал с деньгами. В
Черкассах тревожно, хотя боев не было и дай Бог не будет. Мы вернемся через два
месяца. Украина победит. И все восстановит со временем

\iusr{Helena Aliieva}
\textbf{Акпер Гасанов} 

Вы просто не пережив этого ужаса, рассуждаете из другой реальности. И сам пост
Филатова о плохих переселенцах и хороших местных. Вы поддержали этот мессидж. Я
просто в шоке от многих комментариев. Ваш стал случайным, на который
споткнулась

\iusr{Акпер Гасанов}
\textbf{Helena Aliieva} 

мы этот ужас пережили давно. Погуглите Агдам- Карабахская Хиросима. Почти
тридцать лет азербайджанские земли были под армянской оккупацией. Миллион
беженцев хлынуло в Баку в начале 90-ых. У Азербайджана не было времени для
строительства армии, экономики, государства, когда российско-армянские войска
оккупировали Карабах. Сорри, но у Украины было почти три десятка лет, чтобы
стать сильным государством с сильной армией.

\iusr{Helena Aliieva}
\textbf{Акпер Гасанов} 

Вы уже съехали на претензии к Украине. Я какое то отношение к этому имею? И
вообще начинаете бред писать. Латентное украинофобство.

\iusr{Акпер Гасанов}
\textbf{Helena Aliieva} а вы уже хамите незнакомому человеку. Жаль. Хорошего вечера. Все будет Украина!

\iusr{Ralph Kent}
\textbf{Акпер Гасанов} ...а в это время Зеленский выпрашивает у всего мира деньги на каски и бронежилеты и ругает правительства разных стран за прижимистость?

\iusr{Акпер Гасанов}
\textbf{Ralph Kent} 

о том и речь. но поди и докажи большинству очевидное. Путин- враг. Россия-
тоже. В любом состоянии и всегда. Но зная все это, Украина три десятка лет не
создавала сильную армию, позволяя обогащаться кучке олигархов. Но виноват во
всем этом, конечно, я. не надо было писать такой пост

\iusr{Софія Староконь}
\textbf{Акпер Гасанов} они пищат \enquote{мыатвайныбижалианиатбеднасти}.

\end{itemize} % }

\iusr{Віктор Векленко}

Підтримую повністю. Але на сьогодні ми маємо близько половнини населення в
якості громадян, відсотків 30 - таке собі болото, а 20 - або латентні, або
активні орки. Тож тримаймося разом і працюймо далі. Дякую за працю, Борисе
Альбертовичу!

\iusr{Вячеслав Ветлянский}

Точно. Каждое слово- правда! Все так. И от себя. В прошлую субботу много ходил
по Городу пешком. Во всех кафе и ресторанах забито \enquote{беженцами}. В Термах нет
места на стоянке и вокруг от АХ, ВВ, АН, ВИ и прочих. Впрочем, как и в Артисте
и Репортере. И во многих заведениях. Если машина припаркована с вызывающим
нарушением правил, то 400 процентов, что номера начинаются не с АЕ. Я понимаю,
что бежали от войны. Но на в Репортер, Артист или Термы! И это все после Бучи!
Это во время того, что творится в Мариуполе! Говорят, что нельзя затевать срач
во время войны. Согласен. В единстве - сила. Но куда убрать \enquote{этих}?!?

\begin{itemize} % {
\iusr{Yelyzaveta Zaitseva}
\textbf{Вячеслав Ветлянский} 

а что делать с бизнесом собственнику Репортера и других таких заведений? Вы
предлагаете убить остатки экономики, чтоб кому-то не было так обидно? Очень
странно решать как кому тратить свои деньги

\end{itemize} % }

\iusr{Hauke Olha}

А ще ці плюшеві костюми, з плюшеввими мізками ганьблять своєю поведінкою нашу
країну і поза межами кордону! Дуже соромно, коли люди з усього світу нам
допомагають, а ці...як останні вигрібають все до власної кишені і не думають
про майбутнє рідної країни.

\iusr{Yelyzaveta Zaitseva}
\textbf{Hauke Olha} що вигрібають?

\iusr{Anita Gezo}

Черновцы - прекрасный город с прекрасными людьми, спасибо им, меня тоже приняли
на неделю, но действительно очень переполненный, поэтому я уехала дальше, чтоб
уступить также место тем, кто далеко ехать не в силах... В стрессе думать о
других иногда тяжело, но очень надо!! Вы очень правы!

\iusr{Valentina Kovalchuk}
Нарешті це озвучено. Так само у Польщі, Німеччині  @igg{fbicon.face.confused} 

\iusr{Svitlana Busilkova}

Передали все правильно. Люди є різні. Є ті, хто шукає волонтерські центри і
день в день працює:плете сітки, шиє балаклави, вантажить, сортує гуманітарну
допомогу. І міцно зціплює щелепи, і очі темніють від новин про Волноваху,
Маріуполь, Харків, Рубіжне. Там залишився їх дім, хтось з родини.

І є ті, що стовідсотково підпадають під Ваш опис. Я одразу вирізняю вимушених
гостей нашого міста.

\iusr{Yelyzaveta Zaitseva}
\textbf{Svitlana Busilkova} 

а звідки ви взяли, що описані в пості люди вчора так само не робили сітки
думаючи про Маріуполь, Харків і інші міста?

\iusr{Irina Sirenko}

Люди - вони такі, ... людські. Дуже різні. Поведение человека зависит не от его
статуса (беженец или нет), и не от его кошелька. От его культуры. Есть беженцы,
которые, приехав, и работают удаленно, и волонтерят, и помогают местным. Есть
беженцы, которые, приехав, не только нагло себя ведут, но ещё и находятся в
позе обиженных, которым все должны. Есть местные, которые помогают всем, чем
могут, теснятся, чтобы поселить людей, кормят, делятся чем могут. Есть местные,
которые за аренду жилья подняли цены в три-четыре раза, бо
\enquote{заробляють}, да ещё и попробуйте поселиться с детьми или с животными -
нет, \enquote{такого мені тут не треба}. У кого-то война, у кого-то отпуск, у
кого-то зароботки... И в Украине, и за ее пределами. Це ж люди. Такі вони,
різні...

\iusr{Gennadii Polyakov}

\obeycr
Можна зрозуміти кожного.
І напевно варто спробувати це зробити.
Але, чи варто на це ( цих) витрачати час.
Нам своє робити.
Свій до Свого по Своє.
Разом до Перемоги!
\restorecr

\iusr{Volodymyr Korobko}

Так и есть

Но у меня перед глазами девочка, лет 10 с маленьким йорком на поводке, сзади
мама с рюкзаком, и ещё девочка лет 7-8 с картонной коробкой, такие здесь на
границе раздают волонтеры - это разовые переноски для кошек... Эти трое шли в
Румынию, вернее пятеро... Там их встретят другие волонтеры, накормят, дадут
переноску для кота. И полная неизвестность впереди!

\iusr{Artem Pavlov}

Суть в том, что плюшевые жопы и работающие с ноутбуками в заведениях имели
такие же привычки и у себя дома. Но это не значит, что такое же поведение ок в
городе, который им дал безопасность. Переселенцы должны не забывать, что они
прибыли в гости с главной целью - получить безопасность, а не сервис. Поэтому
нужно уважать местные правила. Не важно, в Черновцах, Ужгороде или городах
Европы.

\iusr{Artem Pavlov}

И если вы можете купить себе еду - не стойте, блять, в очереди за бесплатной!
Оставьте ее тем, кто в ней нуждается!

\iusr{Valentina Kovalchuk}
Абсолютно підтримую.

\iusr{Carina Hrynyk}

Такі папіки з «золотими цепками» стояли і дивились як місцеві дівчата / жінки у
центрі прийому біженців волокли їм матраси і готували їм їжу, а ті навіть за
собою посуд помити не хотіли у Львівській області

\iusr{Roman Kremena}
Самое циничное, когда эти люди из Днепра или других мирных городов.

\iusr{Александр Курбатов}
Підписуюсь під кожним словом

\iusr{Kollin Koss}

Мне попадались представители отдельного биологического вида беженцев. Таких,
которые пережидают в Кишиневе, когда можно будет вернуться в захваченную
рашистами Украину. Вот это мрак.

Конечно, они в меньшинстве, и уплывают за кораблем при первом проявлении
сучности.

\iusr{Olha Skorokhod}
\textbf{Kollin Koss} про таких треба повідомляти СЬУ.

\iusr{Inna Horbenko}
В Тернополе такая же ситуация. И да, местные немного расстроены постоянными гулянками гостей города.

\iusr{Max Goldin}

В Дніпрі теж саме. Та ж сама поведінка с бухлом, ПДР та вимогами. Є випадки
коли прїзджають з ордло, перепочивають (пробухивають) і їдуть назад працювати
на орків...

\begin{itemize} % {
\iusr{Gevorg Hamoyan}
\textbf{Max Goldin} это как?

\iusr{Max Goldin}
\textbf{Gevorg Hamoyan} ось так. Перетинають як біженці кордон, оримають безкоштовне житло і харчування, ті гроші що мали з собою пропивають і повертаються назад.

\iusr{Gevorg Hamoyan}
\textbf{Max Goldin} из ордло местных не выпускают, мужчин даже в Россию женщин и детей фильтруют сильно и не всех выпускают

\iusr{Max Goldin}
\textbf{Gevorg Hamoyan} Ви зараз в ордло?

\iusr{Gevorg Hamoyan}
\textbf{Max Goldin} 

\obeycr
нет я в Харькове на передке почти
У меня есть там друзья которые и в подвалах мгб отсидели
Что-то из них никто не может выехать
Видимо вы знаете схему расскажите
\restorecr

\iusr{Max Goldin}
\textbf{Gevorg Hamoyan} як наступний такий приїде до шелтеру, спитаю.

\iusr{Gevorg Hamoyan}
\textbf{Max Goldin} что такое шелтер?

\iusr{Max Goldin}
\textbf{Gevorg Hamoyan} притулок де приймають біженців.

\iusr{Gevorg Hamoyan}
\textbf{Max Goldin} шелтер это приют для женщин (с детьми), ставших жертвами сексуального насилия или подвергающихся насилию дома

\iusr{Gevorg Hamoyan}
\textbf{Max Goldin} так так і треба писать притулок для біженців

\end{itemize} % }

\iusr{Арман Матевосян}

Нет плохих народов, но есть плохие люди.... К сожалению.( Пол беды, когда эти
индивиды на западе Украины ведут себя неадекватно и нагло, полный трэш, когда
заграницей. Они не понимают, что этим самим портят имидж страны и заставляют
задуматься обычных жителей: Польши, Германии, Болгарии, Молдовы.... \enquote{а зачем мы
им помогаем}?

\iusr{Yelyzaveta Zaitseva}
\textbf{Арман Матевосян} 

в ЕС люди часто более осознанны и понимают, что вот такая наглость со стороны
беженцев - шоковое состояние после круглосуточных бомбежек

\iusr{Yurii Brahin}
А як вони себе поводять за кордоном - зовсім мрак  @igg{fbicon.frown} 

\iusr{Alexandra Teslenko}
Проїхала такий самий маршрут. Але зустрічала лише людей які без сну і переживають та допомагають.

\iusr{Олег Серга}
Правильно.

\iusr{Инна Ляхова}
100\% поддерживаю!!!!!!

\iusr{Viktor Novak}

Я коли сім'ю вивозив їхав цим же маршрутом і далі на на захід, то така ж
ситуація була в усіх містах. Готель орендувати неможливо, ночівля по школам та
садочкам. Всі місцеві молодці, дуже тепло приймали. Тільки Яремче засмутило, де
поголовно у всіх великі і гарні садиби - громада заможна, а от дві школи, які
ми бачили - це обшарпані старі будівлі, на яких осипана побілка фасаду місцями
оголювала глиняні стіни, стара покрівля і всередині совок. Тіки вікна були м/п.
Найгірші школи, що трапилися нам.

\iusr{Oleg Chyzh}
Забыл про оставшиеся блок посты за 1000 верст от зоны боевых... оч важные...

\iusr{Мирослав Бровко}
Дуже, дуже стримано.

\iusr{Роман Плавицький}
правильно что написал. это не только там, везде где нет фронта такое есть.

\iusr{Мороз Владимир}

Поки кожен з такої категорії (біженців) не переживуть на своїй шкурі обстріли,
бомбардування, голод, спрагу......!

Вони це не зрозуміють!!!((

\iusr{Stas Nasobin}
 @igg{fbicon.thumb.up.yellow}  @igg{fbicon.hands.shake} 

\iusr{Андрій Тимченко}
 @igg{fbicon.100.percent} 

\iusr{Roman Goldman}
Согласен!

\iusr{Елена Маленкова}
Согласна.

\iusr{As Sia}
 @igg{fbicon.100.percent} 

\iusr{Igor Volvach}
Всё правильно.
Об этом нужно писать...

\iusr{Alex Perkin}

Киев тоже \enquote{расцвел}. В последние дни возле блокпоста оядом со мной хипстеры
врубили музон в кафе и побухивают. А рядом пацаны по делу стоят. Ну вот, я
тоже, вроде понимаю, но...

Хотя, не все хипстеры одинаково полезны. Мой сосед с бородой и в подстреленых
штанах мотался весь этот период волонтерил. И в Бучу съездил в т.ч.

Но да...

\iusr{Танетко Роман}
Нам своє робити! Разом до Перемоги!

\iusr{Алена Мазур}

Все правильно написали! Приезжих очень сильно видно внешне (не только по
номерам машины) По городу ходят такиииие важные папкины защитники в
солнцезащитных очках, узких джинсах и кожанках, шо страшно как бы не пришлось
ещё его защищать  @igg{fbicon.face.tears.of.joy}  но мы все равно защитим!

\iusr{Igor Yevtushenko}
Браво!!!

\iusr{Vladimir Psunenko}
Я с этим столкнулся даже внутри своей семьи. ☹️

\iusr{Pavel Moseychuk}

Чтоб понять как необходима помощь нашим бойцам, надо просто свозить этих людей
поближе к войне, чтоб они посмотрели, что их ждет если сюда придет рускамиръ,
каждый раз возвращаясь оттуда я не могу успокоится минимум сутки и судорожно
ищу возможность помочь тем кто защищает нас от этих руZZких тварей. А потом
максимально быстро собираю все необходимое и еду снова. Мне просто не налазит
на голову как можно быть к этому равнодушным!!!

\iusr{Helena Aliieva}

Реально хочется матюкаться почитав коменты. От \enquote{добрых} соотечественников.
Реально надо, чтоб каждый посидел под обстрелами и бомбами. Тогда может дойдёт,
что это война всей Украины! В Европе люди милосерднее на порядки. А вам комфорт
обломили. Надо теперь всех переселенцев хейтить. По свински это

\iusr{Helena Aliieva}

Передерг. На хорошие и плохие города. Давайте поговорим и о заоблачных ценах на
жильё в этих областях. О том, как пофигистически относятся в ЦНАПах львовских.
Это не отменяет хороших людей. Для чего эти обобщения? Днепр не был в шкуре
людей под многодневными обстрелами. Для чего этот разжигания ненависти пост?

\iusr{Irina Marszałek}

8 лет живу в Варшаве. У меня были похожие, смешанные ощущения, когда приезжала
в Киев, на востоке у нас война, а в столице дорогущие рестораны, самые
последние тюннингованные тачки, разодетые в топовые бренды (или их подделки)
барышни, и вообще хай-лайф. И тут же обдёртые хрущёвки, старые копейки, бабушки
лет 90 в переходах продающие цветочки, разваленные дороги... я тоже все понимаю,
но этот контраст уж очень колит (колол) глаза, у многих наших людей нет ни
смотрения ни покоры ни скромности.

\iusr{Anton Zhupan}
А как удивляется Европа..

\end{itemize} % }
