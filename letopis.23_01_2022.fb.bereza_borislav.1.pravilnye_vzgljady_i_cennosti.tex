% vim: keymap=russian-jcukenwin
%%beginhead 
 
%%file 23_01_2022.fb.bereza_borislav.1.pravilnye_vzgljady_i_cennosti
%%parent 23_01_2022
 
%%url https://www.facebook.com/borislav.bereza/posts/7620474514645255
 
%%author_id bereza_borislav
%%date 
 
%%tags cennosti,chelovek,obschestvo,vospitanie
%%title Правильные взгляды и ценности
 
%%endhead 
 
\subsection{Правильные взгляды и ценности}
\label{sec:23_01_2022.fb.bereza_borislav.1.pravilnye_vzgljady_i_cennosti}
 
\Purl{https://www.facebook.com/borislav.bereza/posts/7620474514645255}
\ifcmt
 author_begin
   author_id bereza_borislav
 author_end
\fi

Я искренне верю в то, что если человек не социопат, не законченный мерзавец и
не фрик, то нарушения он совершает по незнанию, глупости или потому что в его
голову не вложили в детстве правильные взгляды и ценности. 

Например, возле Житного рынка на парковочное место для людей с инвалидностью
стал красивый чистый белый лексус. Подошёл, постучал в окошко, объяснил для
кого это место. Услышал, что всё поняли и сейчас же уезжают. Сказали, что не
заметили знак. Только попросили не фоткать и не выкладывать. Мой подписчик
оказался. 

Или стоит возле Киевфудмаркета мужчина и докурив сигарету бросает окурок себе
под ноги. Вежливо говорю ему, что это неправильно и разве ему бы понравилось,
если бы в его квартире на пол бросали бы окурки его же друзья? Он сначала хотел
что-то сказать, а потом по мимике лица стало понятно, что он понял свою
неправоту. Поднял окурок. Посмотрел по сторонам. Мусорника и вправду рядом не
было. И он об этом сказал. Я указал ему где есть мусорник. Он пошёл к нему. Не
ругался. Видимо человек представил,  как бы он реагировал на окурок брошенный
на пол в его доме. Я очень хочу верить, что все именно так было в тот момент в
его голове. 

Когда малолетка бежит через кучу полос проспекта Бандеры и добегая счастливо
выдыхает, мол смог невозможное, то мой приятель, который успел резко
затормозить чтоб не сбить этого экстримала, не поленился остановиться, выйти из
машины и без мата объяснить чем это могло закончиться. А потом, видя абсолютное
безразличие в глазах этого малолетки, приятель достает телефон и показывает
видео с телами таких же малолеток, которым не так повезло в попытке перебежать
дорогу сквозь поток машин. И вот видео с телами, которые лежат на асфальте, как
тряпичные куклы, меняет выражение лица малолетки с безразличного, на
озабоченное. Не факт, что он сразу поменяет свое мировоззрение и отношение к
бегу по дорогам, но шанс уже есть. 

И если президент публично говорит абсолютную чушь о государственном перевороте
или что-то в стиле, мол раз никто не погиб из-за Трухина, то значит это мелочь,
а не ДТП, то стоит ему на это указать. Не факт, что услышит или поймёт, но вода
камень точит. А значит есть шанс, что рано или поздно он поймёт, что он не
должен плохо играть роль президента и бесконечно факапить, а должен хотя бы
пытаться быть президентом и понимать величину ответственности, которая
возложена на него. Хотя этот шанс столь мал, что даже я теряю надежду пробить
эту стену из некомпетентности, необоснованного самомнения и детских комплексов.
И все же я не молчу, а каждый раз обоснованно критикую, чтоб был шанс
прекратить бесконечный испанский стыд за все, что делает наша власть. 

Подведя итог, могу сказать, что если у вас есть возможность поменять чьё-то
отношение к общепринятым стандартам, то пробуйте. Объясняйте и вкладывайте в
голову людям правильное мировоззрение, правильные ценности и взгляды, а главное
- правильное отношение к окружающим и миру. Это шанс исправить мир к лучшему. Я
так живу.

PS Но у вас может возникнуть вопрос, а что такое правильные взгляды и ценности?
Если у вас все же возник этот вопрос, то откройте учебник по этике и в
интернете найдите основные принципы цивилизованного общества. И найдёте ответы.
Но грустно, что до сих пор это было вам не знакомо.
