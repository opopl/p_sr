% vim: keymap=russian-jcukenwin
%%beginhead 
 
%%file 16_07_2020.stz.news.ua.mrpl_city.1.10_faktiv_kuindzhi
%%parent 16_07_2020
 
%%url https://mrpl.city/blogs/view/desyat-najtsikavishih-faktiv-pro-zhittya-i-tvorchist-arhipa-ivanovicha-kuindzhi
 
%%author_id demidko_olga.mariupol,news.ua.mrpl_city
%%date 
 
%%tags 
%%title Десять найцікавіших фактів про життя і творчість Архипа Івановича Куїнджі
 
%%endhead 
 
\subsection{Десять найцікавіших фактів про життя і творчість Архипа Івановича Куїнджі}
\label{sec:16_07_2020.stz.news.ua.mrpl_city.1.10_faktiv_kuindzhi}
 
\Purl{https://mrpl.city/blogs/view/desyat-najtsikavishih-faktiv-pro-zhittya-i-tvorchist-arhipa-ivanovicha-kuindzhi}
\ifcmt
 author_begin
   author_id demidko_olga.mariupol,news.ua.mrpl_city
 author_end
\fi

110 років назад – 11(24) липня 1910 року – пішов з життя один з найбільш
яскравих художників українського живопису – \textbf{Архип Іванович Куїнджі.} Автор
всесвітньо відомої картини \enquote{Місячна ніч на Дніпрі} пройшов тернистий творчий
шлях від художника-самоучки до художника-вчителя. Життя і діяльність видатного
уродженця Маріуполя продовжують досліджувати сучасні мистецтвознавці, адже ім’я
та творчість Архипа Куїнджі сповнені загадок. Колористичні прийоми Куїнджі
виявилися свого роду відкриттям для сучасників. Незвичайна та ефектна передача
сонячного і місячного світла, активні колірні контрасти, композиційна
декоративність полотен Куїнджі ламали старі мальовничі принципи. Я вирішила
зупинитися на 10 найцікавіших фактах і дискусійних питаннях, які зможуть краще
розкрити особистість талановитого художника.

1. Цікаво що досі не встановлена точна дата народження Куїнджі. У різних
джерелах – різні дати 1840, 1841, 1842.

2. Історики й донині сперечаються, що означає прізвище художника, яке має
декілька варіантів. З турецької \enquote{Куїнджі} – срібних справ майстер, з румейської
– \enquote{людина, що копає колодязь}. У метриці він записаний як \enquote{Еменджі}, що в
перекладі з татарської означає \enquote{працююча людина}.

\ii{16_07_2020.stz.news.ua.mrpl_city.1.10_faktiv_kuindzhi.pic.1}

3. Потяг до живопису у нього з'явився в той час, як він тільки почав ходити,
тобто із самісінького дитинства. Він малював, де тільки міг - на клаптиках
паперу, стінах, парканах. Освіта Архипа обмежилася кількома класами міської
початкової школи. Архип Іванович був самоуком. Підлітком намагався навчатися у
Айвазовського, але знаменитий мариніст, як свідчить легенда, доручив йому
фарбувати паркан.

4. До спогадів про рідний Маріуполь Куїнджі звертається протягом усього життя.
У 1870 році він напише \enquote{Вид річки Кальчик в Катеринославській губернії}. Ще
через п'ять років – \enquote{Чумацький тракт в Маріуполі}. Тоді ж, приїхавши в рідне
місто, щоб одружитися з донькою маріупольського купця Вірою Леонтіївніою
Шапо\hyp{}валовою-Кечеджи, напише \enquote{Степ} і \enquote{Степ навесні}. Також саме в Маріуполі,
він задумав картину \enquote{Ніч}. Через рік цей початковий задум був здійснений і
з'явилася \enquote{Українська ніч}, яка демонструвалася на Всесвітній виставці в
Парижі.

5. У залі Товариства заохочення художників виставляється лише одна робота
Архипа Куїнджі – \enquote{Місячна ніч на Дніпрі}. Публіка була вражена незвичайними
\enquote{фосфоросцируючими} фарбами. Картина експонувалася при закритих вікнах і
особливим чином підсвічувалася лампою. У ті дні газети писали: \enquote{Виставкова зала
не вміщала публіки, утворилася черга, і екіпажі тяглися по всій Морській
вулиці}.

\ii{16_07_2020.stz.news.ua.mrpl_city.1.10_faktiv_kuindzhi.pic.2}

6. Куїнджі був неймовірно популярний і багатий. Наприклад, великий князь
Костянтин Костянтинович купив \enquote{Місячну ніч на Дніпрі} за шалену на ті часи суму
– 5 тис. руб. А київський мільйонер Терещенко придбав авторський повтор
\enquote{Березового гаю} за 7 тис. рублів. Для порівняння: кращі портрети роботи
І. Крамського оцінювалися в 800–900 рублів. \enquote{Сяючі} полотна Куїнджі любили
Ф. Достоєвський і І. Тургенєв, зображення \enquote{Березового гаю} передруковували тисячі
разів, включаючи радянські підручники. \enquote{Куїнджі – це гроші!}, – говорив його
колега Павло Чистяков.

7. До живопису він, здається, ставився як до математичного рівняння: композиції
\enquote{вирішував}, спрощуючи і концентруючись на палітрі і світловому ефекті,
вибираючи теми заходу, сходу і, звичайно, місячної ночі в будь-який час року.
Склад фарб тримав у таємниці і дуже негативно ставився до копіювання. Для
Куїнджі було важливо не тільки написати картину, але й грамотно презентувати
її. Як правило, його фірмові місячні ночі з великим успіхом демонструвалися в
затемненому залі, при світлі тьмяних ламп. І.  Шишкін називав його або \enquote{хитрим
греком}, або \enquote{чарівником}.

8. У 1882 році Куїнджі раптово замовкає і в цьому ще одна загадка великого
майстра. Протягом 30 років він виставився всього лише один раз. При цьому
писати він не перестав. За час мовчання він створив близько 500 робіт. Всі
вони, за винятком чотирьох, стали відомі вже після смерті Архипа Івановича.

9. У 1894 році його обрали членом ради Академії мистецтв, присвоїли звання
професора живопису. Він очолив власну майстерню і виявився чудовим педагогом.
Особливою програми і системи в навчанні не було. Але він був чуйним і уважним
до учнів, ніколи не нав'язував своєї думки і своєї техніки, цінував свободу в
творчості. Куїнджі вчив, що художник повинен творити, а не копіювати.

10. На даху свого будинку на Васильєвському острові художник виростив маленький
сад. Кожен день опівдні сюди зліталися птахи з усього Санкт-Петербурга. Архип
Іванович не тільки годував їх, а й лікував, і навіть робив операції. Коли він
помер, птиці летіли за труною до Смоленського кладовища...

\ii{insert.read_also.demidko.5_misc_priazovja}
