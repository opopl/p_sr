% vim: keymap=russian-jcukenwin
%%beginhead 
 
%%file 21_01_2020.fb.fb_group.story_kiev_ua.1.elegii_prodolzhenie
%%parent 21_01_2020
 
%%url https://www.facebook.com/groups/story.kiev.ua/posts/1254759088054223/
 
%%author_id fb_group.story_kiev_ua,petrova_irina.kiev
%%date 
 
%%tags detstvo,gorod,kiev,zhizn
%%title Из ЭЛЕГИИ ДЕТСТВА (продолжение историй дома № 7)
 
%%endhead 
 
\subsection{Из ЭЛЕГИИ ДЕТСТВА (продолжение историй дома № 7)}
\label{sec:21_01_2020.fb.fb_group.story_kiev_ua.1.elegii_prodolzhenie}
 
\Purl{https://www.facebook.com/groups/story.kiev.ua/posts/1254759088054223/}
\ifcmt
 author_begin
   author_id fb_group.story_kiev_ua,petrova_irina.kiev
 author_end
\fi

Из ЭЛЕГИИ ДЕТСТВА

(продолжение историй дома № 7)

Тихая семейная жизнь, с четким распорядком трапез и снов, не спасла Наума
Львовича от тяжелого инсульта. Болел он недолго, ушел, оставив Иринушку вдовой.
Туго приходилось без зарплаты Наумчика, тем более, что толком Иринушка нигде и
никогда не работала. Ощутимым подспорьем для маленькой, чудом выхлопотанной
пенсии, стала выручка от продажи множества дефицитных хрустальных вещиц,
ковров, отрезов габардина и шевиота, нажитых запасливым тружеником нефтебазы.
Когда запасы истощились, Иринушка устроилась вахтершей в Министерство
просвещения, располагавшееся по соседству, на Карла Маркса. Гордо именуя свою
должность «ночным диспетчером». Возвращалась некая стабильность быта, утихло
неглубокое горе от утраты, шестидесятилетняя вдова вновь окунулась в Жизнь. А
что за Жизнь у Женщины без Мужчины?

\ifcmt
  tab_begin cols=2

     pic https://scontent-frt3-1.xx.fbcdn.net/v/t1.6435-9/83404930_2941250869241834_646817062550765568_n.jpg?_nc_cat=104&ccb=1-5&_nc_sid=b9115d&_nc_ohc=3f-78zVZsfsAX_ID4hj&_nc_ht=scontent-frt3-1.xx&oh=d5246ed76e5f3d4d5f1d2920d8f396b7&oe=61B480D5

     pic https://scontent-frt3-2.xx.fbcdn.net/v/t1.6435-9/83416317_2941252462575008_2716668631214718976_n.jpg?_nc_cat=103&ccb=1-5&_nc_sid=b9115d&_nc_ohc=z_k6qQO_QLAAX-IwV69&_nc_ht=scontent-frt3-2.xx&oh=fd7553e6ff6efe034e08d3806d907cd4&oe=61B337BF

  tab_end
\fi

Особо светскую жизнь Иринушка не вела, компаний развеселых не посещала, сайтов
знакомств не было еще и в зародыше. Что делать? И вновь банальнейшая история –
сосед по квартире стал предметом женского обожания. Мужчина он был еще не
пожилой, полигамная природа мужеского пола требовала своё, на стороне знакомств
искать было некогда – работа, дом, дети, жена, тёща, да что там говорить... А
рядом, в соседней комнате, душа, ищущая тепла и участия, и их же предлагающая
взамен за некоторые нехитрые знаки внимания. Так всё и началось.

Роман потек по многолетним правилам подобных взаимоотношений. Встречи украдкой,
тщательная предосторожность, холодный обмен приветствиями на кухне в
присутствии соседей. Когда угомонились первые восторги, возникла привычка и
обыденность. Мужчина, немного насытившись, стал утрачивать интерес. Да и
слишком грузная ноша для хрупких мужских плеч: семья, работа, дача, соседка –
устал... Становился невнимательным, манкировал обязанностями. А что делать Ей?
Скандал не устроишь, громко кричать нельзя, ибо подумают: не тронулась ли
соседка умом – кричит одинокая женщина неведомо на кого. А без крика, что за
скандал? Душу не облегчит.

И вот пришла на ум Иринушке древнейшая женская уловка – вызвать ревность. Как?
Объект мужского пола найти нереально, звонки мужским голосом на общеквартирный
телефон в коридоре тоже невозможны из-за отсутствия такового голоса. Озарением
возникла гениальная, по своей простоте, мысль. Теперь это назвали бы
виртуальным партнером. Но, в те годы такие слова еще не были в ходу, зато
фантазии Иринушки хватило бы на двоих.

Уже упомянутая семья соседки по лестничной клетке Валюши, была большая.
Муж-майор, боевой офицер, сын и дочка, мама Валюши Антонина Эдуардовна жили в
двух просторных светлых, как говорила сама Валюша, «царских» комнатах. Валюшу
Иринушка считала подружкой, поверяя ей все перипетии нехитрого романа, жалуясь
на обиды и хвастаясь маленькими победами. Беседовали они в огромной,
тридцатишестиметровой комнате Валюши в уголочке, полушепотом, а вот у дочки,
учившей уроки в другом углу, уши разворачивались на полный оборот, как локаторы
боевого корабля. Может, потому и сохранились эти эпизоды для истории.

Вот гардеробом мужа Володи и было придумано воспользоваться для пробуждения
дикой ревности.

В первом акте у Валюши одалживался макинтош Володи, его велюровая шляпа цвета
горького шоколада, завершенность этому наряду придавали зонт и галоши покойного
Наумчика. Вы спросите, почему не использовалась одежда безвременно ушедшего?
Ответ прост – вся более-менее приличная мужнина одежда уже была продана на
толкучке в Беличах. Зонт и галоши не продались из-за необходимости в домашнем
хозяйстве первого, и абсолютной никомуненужности вторых.

Итак, сцена первая. Дождливым вечером, тихонько выйдя на лестницу, Иринушка
звонит в свой же звонок два раза (это сигнал для её семьи, к соседям – один
звонок). Быстро и тихо входит в квартиру, а уж теперь, пошаркав перед дверью,
звеня дверной цепочкой и скрипя дверьми создаётся иллюзия Прихода Его. Потом в
коридор, на вешалку вывешиваются макинтош и шляпа соседа Володи, ставится
открытый зонт, щедро политый в комнате водой из кружки, и, намоченные таким же
способом, галоши. Все, кто проходит по коридору, видят эту картину, убеждающую
в совершенно порочном поведении соседки. Видит Он, ему эта картинка и так душу
царапает, да еще и теща с женой обсуждают, негодуют, злорадствуют – действуют
ему на нервы. Спать тогда ложились рано, телевидение было недолгим, фейсбук
появится нескоро.

Выждав, пока все утолкутся, Иринушка пару раз тяжело шаркнув в коридоре, громко
открывала и закрывала входную дверь, иногда, для вящего убеждения, чмокнув
воздух. Затем маскарад до утра убирался с вешалки, за ночь макинтош и шляпа
высыхали и возвращались Валюше. Отсутствие одежды Володей не могло быть
замечено, так как на службу майор ходил в форме.

Однако, постоянные визиты одного и того же мужчины должны были бы, по
целомудренной логике тех времен, привести к законному браку. Поэтому было
решено «сменить партнера». Как раз в это время соседу Володе удачно выделили
санаторную путевку в Судак почти на месяц. Шинель, в которой он ходил на
службу, одиноко повисла на вешалке в коридоре, поджидая хозяина. Чего же
простаивать нужной вещи?

Действие второе: после двукратного звонка, скрипения двери, шарканья шагов, на
вешалке в коридоре появлялась майорская шинель. Осень стояла дождливая,
военному человеку с зонтиком ходить не пристало. Теперь вся вода из кружки
выливалась на многострадальные шинель и форменную фуражку. Галоши и зонт в
спектакле не участвовали, покорно ожидая своего выхода за кулисами.

Ближе к концу отпуска, непросыхающую шинель и фуражку пришлось вернуть Валюше,
дабы вещи высохли к встрече хозяина.

А к Иринушке зачастил «Некий Штатский» в коричневом габардиновом пальто и серой
фетровой шляпе. Догадываетесь, в чьих?

Сосед, выбитый из психологической колеи, по степени накала эмоций приблизился к
лучшим образам ревнивцев мировой литературы. Теперь уже он устроил ночной
скандал грозным шепотом, все «соперники» были изгнаны, роман продолжался…

Пальто, макинтош, шинель, фуражка и шляпы дремали в коридоре Валюши. Им снились
свет рампы, гром оваций, как всем тщеславным актерам, сошедшим с подмостков.

На фотографиях - Актеры домашнего театра - шляпа и пальто. Демонстрируют мои
папочка Володя и мамочка Валюша.
