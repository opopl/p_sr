% vim: keymap=russian-jcukenwin
%%beginhead 
 
%%file 11_10_2019.stz.news.ua.mrpl_city.1.makarenko_ljubov_vasylivna_najgolov_samopovaga
%%parent 11_10_2019
 
%%url https://mrpl.city/blogs/view/makarenko-lyubov-vasilivna-najgolovnishetse-samopovaga
 
%%author_id demidko_olga.mariupol,news.ua.mrpl_city
%%date 
 
%%tags 
%%title Макаренко Любов Василівна: "Найголовніше – це самоповага!"
 
%%endhead 
 
\subsection{Макаренко Любов Василівна: \enquote{Найголовніше – це самоповага!}}
\label{sec:11_10_2019.stz.news.ua.mrpl_city.1.makarenko_ljubov_vasylivna_najgolov_samopovaga}
 
\Purl{https://mrpl.city/blogs/view/makarenko-lyubov-vasilivna-najgolovnishetse-samopovaga}
\ifcmt
 author_begin
   author_id demidko_olga.mariupol,news.ua.mrpl_city
 author_end
\fi

\ii{11_10_2019.stz.news.ua.mrpl_city.1.makarenko_ljubov_vasylivna_najgolov_samopovaga.pic.1}

Дорогі читачі, 13 жовтня в Україні відзначатимуть \emph{День художника}. Не забудьте
привітати всіх, хто причетний до цього свята. Приємно, що в Маріуполі набирає
обертів розвиток мистецького життя. З нагоди професійного свята я вирішила
написати нарис про дуже світлу, чуйну, добру, чисту і талановиту людину –
\textbf{Макаренко Любов Василівну} – викладачку Маріупольської школи мистецтв. Вона
підготувала не одне покоління художників і невпинно сприяє розвитку
образотворчого мистецтва не тільки в Маріуполі, але й Україні загалом.

Наша героїня народилась у Ташкенті в родині військовослужбовців. З дитинства
була різнобічною та дуже талановитою дівчиною. Встигла спробувати власні сили і
в балеті, і в музиці, і в спорті. Завдяки наполегливості і цілеспрямованості
дівчинка усюди робила успіхи. Її сім'ї завжди було притаманне особливе відчуття
кольору. Брат дуже любив малювати, а мати бездоганно розбиралася у кольорах і
створювала неймовірні композиції у вишивках. До речі, саме мама помітила у
доньці хист до художньої справи.

Дитинство Люба провела у Чорткові. Дуже любила грати в лісах Тернопільщини.
Ніякого почуття небезпеки у маленької дівчинки не було, навпаки вона
насолоджувалася неповторною атмосферою свободи і безтурботними веселощами.
Яскраві спогади залишилися і про участь в обрядах народних свят.

\ii{insert.read_also.demidko.kateryna_karaben_fortun}

Дівчина з дитинства була відповідальною та дуже старанною і зовсім не
помилилася, вирішивши пов'язати своє життя з творчою професією. Навчалася
моделюванню в Одеському театрально-ху\hyp{}дожньому училищі та Львівському
декоративно-прикладному інституті, а потім працювала за цією спеціальністю в
Донецькому будинку моделей і на Маріупольської швейній фабриці. У студентські
роки Люба проходила практику в глибинці Західної України, де збирала разом з
іншими студентами етнографічний матеріал. Саме тоді вона зрозуміла, що чим далі
від цивілізації, тим більше можна знайти добра. Згадує, як розбили наметове
містечко. Найбільшим сюрпризом було щоранку знаходити під наметами багато їжі.
Люди з селищ допомагали, хто чим зможе. Така підтримка - безкорислива,
альтруїстична та водночас анонімна - нескінченно вражала. Любов Василівна
впевнена, що \emph{якщо ставлення до жінки, до дитини, до старого буде добрішим, тоді
весь світ стане кращим}. У гуцульському селі Криворівня, де вони проходили
практику, ходила одна жінка – \textbf{Параска} – і писала краєвиди та вірші. Вона жила
лише творчістю, ніде не працювала, виживала завдяки селу, яке її годувало. Ці
приклади безкорисливої допомоги один одному завжди надихали нашу героїню. А
безмежно гарні та дивовижні краєвиди, особливо у Верховині, запам'яталися на
все життя.

Батька демобілізували, і в 1965 році сім'я Люби приїхала до Маріуполя. Оскільки
місце проживання в дитинстві часто змінювалося, малої Батьківщини жінка назвати
не може. Але приїзд до Маріуполя започаткував новий період в житті – найбільш
яскравий і насичений. Себе Любов Василівна вважає корінною маріупольчанкою. Не
всім маріупольцям відомо, що Любов Макаренко мала відношення і до нашого
драмтеатру: вона для захисту дипломної роботи у вузі моделювала костюми до
бенефісної вистави \href{https://archive.org/details/13_12_2018.olga_demidko.mrpl_city.unikalnyj_aktor_borys_saburov}{Бориса Олександровича Сабурова}%
\footnote{Унікальний актор маріупольського театру Борис Сабуров, Ольга Демідко, mrpl.city, 13.12.2018, \par%
\url{https://mrpl.city/blogs/view/unikalnij-aktor-mariupolskogo-teatru-boris-saburov}, \par%
Internet Archive: \url{https://archive.org/details/13_12_2018.olga_demidko.mrpl_city.unikalnyj_aktor_borys_saburov}
}

\enquote{Дама-невидимка}. З 2001 року
Любов Василівна почала займатися справою свого життя – працювати викладачем у
Маріупольській школі мистецтв. Спочатку вона викладала малюнок, живопис,
композицію і історію мистецтв. Потім керівництво, помітивши не тільки її
педагогічний талант, а й організаторські здібності, довірило їй ціле художнє
відділення.

\ii{11_10_2019.stz.news.ua.mrpl_city.1.makarenko_ljubov_vasylivna_najgolov_samopovaga.pic.2}

Любов Василівна працює на декількох роботах, але завжди знаходить час для своєї
сім'ї. Доньки – \textbf{Марія та Олександра} – успадкували мамине відчуття прекрасного.
Олександра навіть закінчила художньо-графічне відділення Харківського
педагогічного уні\hyp{}верситету. А ось старша донька Марія закінчила дуже серйозний
факультет комп'ютерної кібернетики Херсонського політехнічного університету. І
Марія, і Олександра зробили головний подарунок для своєї матусі – повернули
Любов Василівну у світ пелюшок та іграшок. Тепер вона ще й щаслива бабуся.

\ii{11_10_2019.stz.news.ua.mrpl_city.1.makarenko_ljubov_vasylivna_najgolov_samopovaga.pic.3}

Улюбленим місцем у Маріуполі для жінки є саме Школа мистецтв. Особливо, якщо
піднятися на третій поверх, де відкривається чарівний краєвид – видно море,
обидва будинки зі шпилями, театральний сквер, приголомшливі хмари...

\ii{insert.read_also.demidko.ukrainceva}

Вражає, що відносини з учнями у Любов Василівни завжди були далеко не
навчального рівня. З роками і досвідом прийшла і власна педагогічна техніка, в
якій найголовніше місце посідають підтримка і взаємоповага, довіра і
відвертість. Багато колишніх учнів навчаються сьогодні у престижних художніх
вузах і академіях Європи (Німеччина, Польща, Ізраїль, Франція). З багатьма вона
підтримує постійний зв'язок. Загалом, учні та учениці Любов Василівні досягли
незвичайних висот і сьогодні прославляють не тільки Маріуполь, але й всю
Україну. Зокрема, \textbf{Очкас Оля} працює у дуже відомого стиліста, \textbf{Ліза Дьяченко}
стала працівником відомих київських та європейських журналів, \textbf{Володя
Пономаренко} відкрив свою архітектурну школу. Зі своїм вчителем учні часто
виходять на зв'язок, звертаються за порадами, а коли є можливість, приходять в
гості. Важливо, що деякі учні продовжують справу вчителя. Так, учениці Наталя
\textbf{Головенкіна і Карина Галецька} стали дуже креативними і творчими викладачами
Школи мистецтв.

\ii{11_10_2019.stz.news.ua.mrpl_city.1.makarenko_ljubov_vasylivna_najgolov_samopovaga.pic.4}

Наразі Любов Василівна працює старшим викладачем у ПДТУ на кафедрі
\enquote{Архітектура} та вже розпочала роботу у філії Національної академії
образотворчого мистецтва і архітектури. Працювати на декількох роботах для
маріупольчанки явище звичне. Поєднувати викладацьку діяльність у вищих
навчальних закладах та подальшу роботу в Школі мистецтв буде нелегко, але
працелюбності і завзятості нашій героїні не позичати. Викладачка сподівається,
що незабаром познайомиться з новими талановитими і яскравими абітурієнтами, які
продовжуватимуть навчання саме в Маріуполі.

\begingroup
\color{blue}
Залишається лише привітати енергійну і талановиту художницю, викладачку і
просто світлу Людину з прийдешнім професійним святом і побажати міцного
здоров'я та нових здобутків, невичерпної енергії і творчої наснаги в
примноженні славних культурних традицій рідного міста, краю, нашої України! 
\endgroup

\ii{11_10_2019.stz.news.ua.mrpl_city.1.makarenko_ljubov_vasylivna_najgolov_samopovaga.pic.5}

\textbf{Улюблена книга:} \enquote{Майстер і Маргарита} М. Булгакова, \enquote{100 років самотності}
Ґабрієля Ґарсія Маркеса.

\textbf{Улюблений фільм:} \enquote{Дерево бажання} (1976 р.) режисера Тенгіза Абуладзе.

\textbf{Хобі:} збирати книги улюблених художників-ілюстраторів. Найбільше книг
київського видавництва \enquote{Час майстрів}, адже директор – Олег Симоненко – дуже
ретельно і тонко підбирає ілюстраторів до своїх книг. Також є ціла колекція
праць української худож\hyp{}ниці-графіка, ілюстраторки Вікторії Володимирівни
Ковальчук.

\textbf{Порада маріупольцям:} 

\begin{quote}
\em\enquote{Потрібно все починати з малого, не чинити так, щоб потім було гірко. Головне –
зберігати самоповагу в усьому: у вчинках, у думках!}
\end{quote}

\ii{insert.read_also.demidko.cherepchenko}
