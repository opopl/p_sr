% vim: keymap=russian-jcukenwin
%%beginhead 
 
%%file 04_01_2019.stz.news.ua.lb.1.arhitekturnyj_atlas_dorevoljucijnogo_mariupolja.15.kazennyj_vinnyj_sklad
%%parent 04_01_2019.stz.news.ua.lb.1.arhitekturnyj_atlas_dorevoljucijnogo_mariupolja
 
%%url 
 
%%author_id 
%%date 
 
%%tags 
%%title 
 
%%endhead 

\subsubsection{Казенный винный склад}

В самом центре Мариуполя находится хороший пример, иллюстрирующий отношение
современников к наследству предков. Горожанам объект известен как мариупольский
ликероводочный завод. Этот интересный образец архитектуры начала ХХ века был
построен для открывшегося в 1902 году седьмого казённого винного склада. На
стыке ХIХ и ХХ веков такие склады массово создавались в рамках государственной
монополии на производство и продажу спиртного, введённой С. Витте в 1896 году.

\ii{04_01_2019.stz.news.ua.lb.1.arhitekturnyj_atlas_dorevoljucijnogo_mariupolja.15.kazennyj_vinnyj_sklad.pic.1}

Строительство велось по типовому проекту, и хотя старых фотографий
мариупольского склада не сохранилось, мы можем представить, каким он был по
открытке из города Кургана. Производство горячительных напитков в этом здании
началось в советское время, а до этого происходила только расфасовка и продажа.
После антиалкогольной компании и перестройки, в девяностых годах, ЛВЗ начал
приходить в упадок и вскоре закрылся. Многие мариупольцы по сей день закатывают
глаза, вспоминая фирменные крепкие напитки этого завода. В эпоху свободных
рыночных отношений помещения в центре не могли остаться невостребованными, но
их перепрофилирование удачным не назовёшь. К первому этажу пристроили типичную
стеклянную галерею, полностью закрыв, а скорее, уничтожив, весь центральный
фасад первого этажа здания. При том, два верхних этажа остались заброшенными. В
стекляшке разместились торговцы вещами. Вместо потенциальной
достопримечательности, центр Мариуполя \enquote{украшают} пёстрые трусы и штаны с
лампасами, гордо выставленные в витринах.

\ii{04_01_2019.stz.news.ua.lb.1.arhitekturnyj_atlas_dorevoljucijnogo_mariupolja.15.kazennyj_vinnyj_sklad.pic.2}
\ii{04_01_2019.stz.news.ua.lb.1.arhitekturnyj_atlas_dorevoljucijnogo_mariupolja.15.kazennyj_vinnyj_sklad.pic.3}
\ii{04_01_2019.stz.news.ua.lb.1.arhitekturnyj_atlas_dorevoljucijnogo_mariupolja.15.kazennyj_vinnyj_sklad.pic.4}
