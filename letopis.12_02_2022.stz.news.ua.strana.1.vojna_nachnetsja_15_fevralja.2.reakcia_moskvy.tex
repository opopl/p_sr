% vim: keymap=russian-jcukenwin
%%beginhead 
 
%%file 12_02_2022.stz.news.ua.strana.1.vojna_nachnetsja_15_fevralja.2.reakcia_moskvy
%%parent 12_02_2022.stz.news.ua.strana.1.vojna_nachnetsja_15_fevralja
 
%%url 
 
%%author_id 
%%date 
 
%%tags 
%%title 
 
%%endhead 

\subsubsection{Реакция Москвы}
\label{sec:12_02_2022.stz.news.ua.strana.1.vojna_nachnetsja_15_fevralja.2.reakcia_moskvy}

В Москве уже назвали происходящее сегодня \enquote{масштабной дезинформационной
кампанией против России}.  

\enquote{Можно говорить о сговоре властей западных стран и СМИ в целях нагнетания
искусственной напряжённости вокруг Украины путем массированного и
скоординированного вброса ложной информации в геополитических интересах, в
частности, чтобы отвлечь внимание от собственных агрессивных действий}, -
заявил МИД РФ.

Ведомство опубликовал коллаж из обложек западных медиа с материалами о
\enquote{вторжении}, поставив на них надпись \enquote{фейк}. 

Тем не менее, количество таких публикаций в Европе и США, конечно, впечатляет.

\ii{12_02_2022.stz.news.ua.strana.1.vojna_nachnetsja_15_fevralja.pic.2}

Глава МИД Сергей Лавров предположил, что Запад готовит военную провокацию в
Украине.

\enquote{Когда эвакуируются сотрудники посольств, когда граждане
англо-саксонских стран призываются к тому, чтобы поскорее с Украины уехать. Мы
уже задумались сами: может быть, именно англо-саксы что-то готовят, если они
эвакуируют своих сотрудников? Я не знаю, какие задумки у наших англо-саксонских
коллег} - заявил он.

