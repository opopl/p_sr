% vim: keymap=russian-jcukenwin
%%beginhead 
 
%%file 09_09_2021.fb.bielskaja_elizaveta.1.jazyk_moshenniki
%%parent 09_09_2021
 
%%url https://www.facebook.com/elizabeth.bielska/posts/4328250840543715
 
%%author_id bielskaja_elizaveta
%%date 
 
%%tags jazyk,mova,obman,obschestvo,ukraina
%%title РОСІЙСЬКА - МОВА ШАХРАЇВ
 
%%endhead 
 
\subsection{РОСІЙСЬКА - МОВА ШАХРАЇВ}
\label{sec:09_09_2021.fb.bielskaja_elizaveta.1.jazyk_moshenniki}
 
\Purl{https://www.facebook.com/elizabeth.bielska/posts/4328250840543715}
\ifcmt
 author_begin
   author_id bielskaja_elizaveta
 author_end
\fi

РОСІЙСЬКА - МОВА ШАХРАЇВ 

\ii{09_09_2021.fb.bielskaja_elizaveta.1.jazyk_moshenniki.pic}

Днями моїй 87-річній бабусі зателефонували й сповістили, що мама (її донька)
потрапила під авто. У неї поламані руки й ноги. Терміново потрібна операція
вартістю 6000 доларів. Після слів про те, що таких коштів бабуся не має і зараз
подзвонить онуці (тобто мені), вони випитали в неї домашню адресу, куди
під'їхати. Мені дзвонити заборонили, але люб'язно погодилися на 4000 готівкою,
бо ж оперувати треба негайно. Натомість сказали, що дадуть слухавку мамі. 

Голос "мами" бабуся не впізнала. На що та відповіла, що їй боляче, тому голос
змінився. Єдине, що порятувало бабусю від зустрічі з бандитами, це згадка, що
вони з моєю мамою вдома говорять рідною мовою, і слова: "Доню, поговори зі мною
українською", на які "доня" відповіла: "Тєбє што, нє панятна, што у мєня губа
разбіта, я нє магу с табой на укрАінском разгаварівать".

Мова для українців зараз стає індикатором свій-чужий. "Шляхетні люди
розмовляють українською" - браслет із таким написом отримають усі, хто до 1
жовтня передзамовить нашу з Антін Мухарський спільну книжку "Як перейти на
українську", що побачить світ на початку листопада. Зробити це можна на сайті
UKRIDEABOOK. Усі книжки, замовлені заздалегідь, матимуть іменний автограф.

Будуймо український світ разом.

\ii{09_09_2021.fb.bielskaja_elizaveta.1.jazyk_moshenniki.cmt}
