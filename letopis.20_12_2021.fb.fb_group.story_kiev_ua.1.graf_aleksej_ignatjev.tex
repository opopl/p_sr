% vim: keymap=russian-jcukenwin
%%beginhead 
 
%%file 20_12_2021.fb.fb_group.story_kiev_ua.1.graf_aleksej_ignatjev
%%parent 20_12_2021
 
%%url https://www.facebook.com/groups/story.kiev.ua/posts/1822922424571217
 
%%author_id fb_group.story_kiev_ua,poljakova_galina.kiev
%%date 
 
%%tags ignatjev_aleksej.graf.kiev,istoria,kiev
%%title Историческая миниатюра о человеке чести и доблести. Граф Алексей Игнатьев
 
%%endhead 
 
\subsection{Историческая миниатюра о человеке чести и доблести. Граф Алексей Игнатьев}
\label{sec:20_12_2021.fb.fb_group.story_kiev_ua.1.graf_aleksej_ignatjev}
 
\Purl{https://www.facebook.com/groups/story.kiev.ua/posts/1822922424571217}
\ifcmt
 author_begin
   author_id fb_group.story_kiev_ua,poljakova_galina.kiev
 author_end
\fi

Предлагаю историческую миниатюру о человеке чести и доблести. Граф Алексей
Игнатьев учился и жил в Киеве. Детство балерины Натальи Трухановой тоже прошло
в Киеве. Но встретились они в Париже. 

Граф. Кавалергард.  Дипломат. 

\begin{multicols}{2} % {
\setlength{\parindent}{0pt}

\end{multicols} % }

Генерал-губернаторский дом в Киеве не сохранился, ибо был взорван во время боев
в 1920 году. Он находился на улице Институтской, примерно там, где сейчас стоят
дома №18 и №20/8. А вот здание по адресу Воздухофлотский проспект, 6 стоит и
поныне. Оно было построено архитектором Штроммом специально для кадетского
корпуса. Мы вспомнили о них потому, что в 1891-94 годах юный Алексей Игнатьев,
сын Киевского, Подольского и Волынского генерал-губернатора был киевским
кадетом. Вообще-то, мальчика готовили к обучению в Пажеском Корпусе, но отец, с
разрешения Александра III, вместо общих классов Пажеского корпуса, направил
сына в Киевский Владимирский кадетский корпус. Дело в том, что Киевский
кадетский корпус считался лучшим в России, и его выпускники без труда поступали
в такие уважаемые институты как Горный и Технологический. 

Как кадетом, так и пажом, Алексей Игнатьев был первым учеником. В старшем
классе Пажеского Корпуса он стал камер-юнкером императрицы, ибо отвечал всем
требованиям сурового этикета: он был первым учеником, поразительно красивым
юношей очень высокого роста и принадлежал к знатному роду. По отцу – потомок
черниговских бояр Игнатьевых, по матери – князей Мещерских. Игнатьевы гордились
тем, что честно и много служили, а Мещерские – наоборот, тем, что «никогда и
никому не служили». Кстати, дед нашего героя был Председателем совета министров
при Александре II и при Александре III, а отец его погиб от пули эсера, будучи
членом Государственного Совета. 

Но вот экзамены сданы, и «царь, теребя перчатку, произнес слова, открывавшие
перед нами целый мир: «Поздравляю вас офицерами!». Алексей Алексеевич Игнатьев
пришел служить в кавалергардский полк, в казармах которого он и родился, когда
полком командовал его отец.  

Еще ребенком Алексей обещал отцу закончить Академию Генерального штаба. И он ее
закончил. Как всегда – первым. Перед ним действительно были открыты
невообразимые возможности. Но вспыхнула русско-японская война 1904-05 года.
Кавалергардов, конечно, на Дальний Восток никто не посылал, но разве можно было
остаться в стороне? Отправляя сына на войну, Алексей Игнатьев-старший
напутствовал: «Будь чем хочешь, только не личным адъютантом!». В лазарете,
расположенном в китайской фанзе, Алексей встретил Катю Игнатьеву, свою кузину.
Молодая графиня работала сестрой милосердия. Она-то и сообщила ему о гибели
всех моряков их семьи при Цусиме. Что привело этих блестяще образованных и
богатых молодых людей из столицы на самый край земли, где их встретили пули,
вши, кровь, страдания, смерть? Стремление служить родине. Это не патетика. Это
правда.    

В неполные 30 лет граф Игнатьев стал военным атташе стазу в трех странах:
Швеции, Дании и Норвегии. А затем был назначен на ту же должность во Франции.
Военный атташе – это как «два в одном»: и дипломат, и разведчик. Накатывалась
Первая мировая война. Компетентность и порядочность российского атташе в Париже
располагает к нему не только французское, но и британское правительство.
Кропотливый и филигранный труд дипломата трудно проследить и трудно оценить.

В Париже граф женился на удивительной женщине – балерине Наталье Трухановой,
уроженке Киева. Наташу называли «Бриллиантовой Трухановой». Специально для нее
писали музыку Морис Равель и Жорж Масснэ. На премьерах они сами дирижировали
своими сочинениями. Композитор Поль Дюка посвятил ей хореографическую поэму
«Пери», а Жорж Масснэ - балет «Эспада». Костюмы и декорации для ее балетов
создавал Леон Бакст. Об умении этой балерины «умирать на сцене» слагали
легенды. Камиль Сен-Санс, увидев, как мадемуазель танцует под его симфоническую
поэму «Пляска смерти», воскликнул: «Она сама не понимает, что она сделала!» 

Революция в России лишила Алексея Алексеевича Игнатьева имений, доходов,
статуса, работы. Абсолютно всего! Из роскошного особняка Игнатьев с женой и
тещей переселяется в крохотный домишко. Красавец кавалергард, граф и дипломат
выращивает в подвале шампиньоны и отвозит их на знаменитый базар «Чрево
Парижа». Рука об руку с ним на грядках роется одна из лучших танцовщиц своего
времени. Приехавшая в гости Анна Павлова совершенно искренне восклицает: «Как
же ты счастлива, Наташа!». 

А между тем в нескольких французских банках хранятся 325 миллионов золотых
российских рублей. Все счета оформлены на имя Игнатьева. Деньги государственные
и предназначались для военных закупок. Но он к ним не прикасается. У него
накоплен богатейший профессиональный архив, за который правительство Франции
предлагает кругленькую сумму. Но Алексей Алексеевич считает, что и архив, и
деньги принадлежат не ему лично, а России. Вряд ли можно заподозрить Игнатьева
в увлечении большевизмом. Скорее всего, он считал это явление временным. В
России разруха и голод. Страшный голод. Об этом он знал и в 1921 году написал в
Москву письмо, предлагая сохраненные деньги. Кстати, на эти миллионы
претендовали и многочисленные эмигрантские организации, и даже Франция,
ссудившая царскому правительству немалую сумму. Поступали заманчивые
коммерческие предложения от крупных фирм. Но все покушения на это золото
разбивались, как о каменную скалу о непреклонность Игнатьева. А Москва молчала.
Полулегально он едет в Берлин в посольство Советской России и снова предлагает:
возьмите! А Москва молчала. И только в 1925 согласилась принять золото от
неуемного графа. 

Алексей Алексеевич долго работал в Париже в советском торгпредстве. А перед
самой войной все же добился разрешения вернуться на родину. Служил в системе
военного образования. Разработал концепцию военных учебных заведений для
мальчиков по типу кадетских корпусов. Хотя признать авторство бывшего графа
военное руководство никак не могло, но целесообразность таких учебных заведений
была очевидна, и в 1943 году в стране были открыты Суворовские училища. Об
Игнатьеве скромно умолчали. 

Алексей Алексеевич написал потрясающую книгу мемуаров «Пятьдесят лет в строю»,
ставшую бестселлером на многие годы. Как забаву, выпустил изысканную кулинарную
книгу, ценимую знатоками. Все, что он делал, он делал исключительно хорошо. Его
жена шутила, что если бы он был дворником, то его улица была бы самой чистой.
