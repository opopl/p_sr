% vim: keymap=russian-jcukenwin
%%beginhead 
 
%%file 19_11_2020.news.ua.5ua.1.knorozov
%%parent 19_11_2020
 
%%url https://www.5.ua/svit/zhyv-priamo-v-muzei-iak-kharkivianyn-pershym-u-sviti-zmih-prochytaty-rukopysy-maiia-229611.html
%%author 
%%author_id 
%%tags 
%%title 
 
%%endhead 

\subsection{Жив прямо в музеї: як харків'янин першим у світі зміг прочитати рукописи майя}
\label{sec:19_11_2020.news.ua.5ua.1.knorozov}
\Purl{https://www.5.ua/svit/zhyv-priamo-v-muzei-iak-kharkivianyn-pershym-u-sviti-zmih-prochytaty-rukopysy-maiia-229611.html}
\href{https://www.5.ua/spetstemy/154}{Спецтема: Видатні українці}

\index[names.rus]{Кнорозов, Юрий}

\ifcmt
pic https://www.5.ua/media/pictures/1140x641/201085.jpg
\fi

\lettrine[lines=3]{Т}{е, що міфічні істоти з гримасами}, зображені на стінах пірамід майя --- це
код, розуміли лінгвісти у різні століття. Розшифрувати його намагалися з І
пол. XIX ст., коли мсьє Шампольон зумів розгадати єгипетські ієрогліфи.
Але він мав змогу зіставити один і той же напис трьома мовами. Знаки ж,
залишені майя, порівнювати було ні з чим. Завдання стояло важке, і під
силу воно стало лише синові харківського інженера.

\begin{leftbar}
  \bfseries
  \enquote{Те, що створено одним людським розумом, не може не бути розгадано
  іншим. З цієї точки зору нерозв'язних проблем не існує і не може
  існувати ні в одній галузі науки}, --- сказав колись українець Юрій
  Кнорозов.
\end{leftbar}

Він народився 19 листопада 1922 р. в родині харківського інженера. Однак
сам стверджував, що насправді з'явився на світ 31 серпня.

Подейкують, що Юрій мав здібності до гіпнозу. Підлітком лікував сусідів
прикладанням руки до хворого місця, а дорослим нібито проникав у сни
племінниці, перетворюючи їх на жахи.

У 1930 р. хлопчик пішов до школи. Мав феноменальну пам'ять, яка дозволяла
йому сторінками цитувати улюблені книги.

Після школи вступив на історичний факультет Харківського державного
університету, але через війну встиг закінчити тільки два курси. Під час
війни волею долі опинився на території РСФРР і до Харкова вже не
повернувся... Після війни Кнорозов вирушив до Москви, де вступив на
кафедру етнографії історичного факультету МДУ.

\begin{leftbar}
  \bfseries
Офіційно темою наукового дослідження Кнорозова в університеті було
шаманство. Але справжньою пристрастю стали письмена майя.
\end{leftbar}

\ifcmt
pic https://www.5.ua/media/pictures/820x546/201115.jpg
caption Юрій Кнорозов, UA/TV
\fi

У тодішній Ленінській бібліотеці студент Юрій Кнорозов знайшов серед книг
два рідкісних томи --- \enquote{Кодекси майя}, видані в Гватемалі, і 
\enquote{Повідомлення
про справи в Юкатані} іспанського єпископа Дієго де Ланди. Імовірно, книги
були вивезені в Союз із Німеччини після капітуляції останньої. Дієго де
Ланда занотував три десятки ієрогліфів майя як літери алфавіту. Але
дослідники схилялися до думки, що ієрогліфи не могли бути буквами --- адже
їх забагато. Ці фоліанти стали для Кнорозова основою в розшифровці
оригінальних рукописів майя, доступ до яких він отримав у тій же
бібліотеці.

\begin{leftbar}
  \bfseries
Вважається, що в основу книги де Ландо лягли праці індіанця з
європейською освітою на ім'я Гаспар Антоніо Чи. Кнорозов здогадався, що
індіанець записував знаками майя не звуки, а назви іспанських букв, і що
алфавіт з 29 знаків --- ключ до розшифровки незрозумілих письмен.
\end{leftbar}

На той час дешифровка письма майя вважалася неможливою --- надто мало
збереглося писемних артефактів. У 1562 році єпископ Юкатана Дієго де Ланда
спалив грандіозну кількість книг та інших пам'яток цієї культури --- тож
базувати дослідження доводилося на зовсім нечисленних прикладах.

Пірнувши в море майяністики, Кнорозов натрапив на різке погіршення
ставлення до нього з боку завкафедрою професора Толстова. Той відмовився
навіть надати студентові формальну рекомендацію в аспірантуру. На щастя,
там же, на кафедрі етнографії, працював професор Токарєв, який підтримав
опального дипломника.

Утім, як згадував Кнорозов, його новий науковий керівник абсолютно не
вірив в успіх дешифрування письмен майя, оскільки, як і його американські
колеги, вважав, що це письмо не є фонетичним. Користуючись власним впливом
у науковому світі, Токарєв влаштував учня молодшим науковим співробітником
в інститут етнографії АН СРСР (Кунсткамера) --- спочатку в секторі народів
Америки, пізніше --- як голову групи етнічної семіотики.

\ifcmt
pic https://www.5.ua/media/pictures/820x546/201119.jpg
caption https://www.5.ua/media/pictures/820x546/201119.jpg
\fi

\begin{leftbar}
\bfseries
Оселився Кнорозов у самому музеї --- у довгій, як пенал, кімнаті. Від
підлоги до стелі приміщення було забите книгами, на стінах висіли
намальовані ієрогліфи майя. З меблів --- тільки письмовий стіл і
солдатське ліжко.
\end{leftbar}

Почавши свої дослідження зі збереженого гватемальського \enquote{Кодексу майя},
Кнорозов склав каталог ієрогліфів і встановив фонетичне звучання деяких з
них. П'ять років Юрій, не виходячи з кабінету, працював над дешифруванням
і не дарма --- він зробив те, що ще не вдавалося нікому. Розшифровка
писемності майя Кнорозовим розтягнулася на кілька років і була завершена
аж на початку 1950-х.

Першу статтю з попередніми результатами дешифрування ієрогліфіки майя
(\enquote{Давня писемність Центральної Америки}) Кнорозов опублікував в 1952 р. у
віці 30 років. Переклад праці Дієго де Ланди \enquote{Повідомлення про справи в
Юкатані} і коментар до нього були опубліковані ще через три роки --- в 1955
р.

\begin{leftbar}
\bfseries
Одночасно з Кнорозовим писемність майя намагалися розшифрувати і в США.
От тільки чільник американської школи маяністики Ерік Томпсон пішов
хибним шляхом, до того ж заборонив займатися розшифровкою всім іншим.
\end{leftbar}

Зі своїми розробками він мав виступити на захисті кандидатської. Але деякі
його тези суперечили теоріям Енгельса, за що Кнорозова могли посадити.

Вранці 29 березня 1955 р. уродженець міста Південне Харківської області
Юрій Кнорозов ішов на захист кандидатської на тему 
\enquote{\enquote{Повідомлення про справи в Юкатані} 
Дієго де Ланди як етно-історичне джерело} і не знав, чим
усе закінчиться. Його могли звинуватити в тому, що він піддає сумніву
марксизму, і заарештувати. Справа в тому, що Фрідріх Енгельс стверджував,
ніби в доколумбовій Америці держав не існувало. А згідно з цією ж догмою,
фонетичне письмо могло існувати тільки у класових державних утвореннях.
Заява ж про наявність в індійців-майя фонетичного письма автоматично
спростовувала відразу два положення \enquote{основоположника}.

\ifcmt
pic https://www.5.ua/media/pictures/820x546/201116.jpg
\fi

\begin{leftbar}
\bfseries
Кнорозов же у своєму дослідженні показував, як він розшифрував
писемність майя: ключовою ідеєю стало те, що символи, які збереглися, –
це не літери і не окремі слова, а склади. Доповідь вченого тривала
всього 3,5 хвилини. Наступного дня захист став легендою. 33-річному Юрію
Кнорозову на вченій раді присвоїли звання не кандидата, а одразу доктора
історичних наук, чого в гуманітарних науках практично не трапляється.
\end{leftbar}

Відтоді історія дешифрування стародавніх систем письма розділилася між
двома іменами: Шампольона, французького єгиптолога, який розшифрував
давньоєгипетські ієрогліфи, і Кнорозова, який зміг прочитати і створити
систему розшифровки письмен майя. Але якщо Шампольон, крім як ієрогліфами,
мав той самий текст іще й давньогрецьким діалектом койне (яке було
прочитати досить легко), то в Кнорозова не було нічого. Жодного тексту для
порівняння і дуже мало пам'яток самих майя.

\ifcmt
pic https://www.5.ua/media/pictures/820x546/201113.jpg
\fi

Та світове визнання доктор наук здобув не тільки завдяки цьому відкриттю,
а й через використання математичних методів при вивченні нерозшифрованих
абеток. У середині 1950-х років він почав дослідження писемності на
острові Пасхи, а в 1960-му запропонував власний варіант дешифрування
писемності долини Інду, хоча його досі ще не вважають загальноприйнятим.

Юрій Кнорозов, окрім звання доктора історичних наук, здобув звання
кавалера Ордена Ацтекського орла (Мексика) та Великої золотої медалі
(Гватемала).

\begin{leftbar}
\bfseries
У 1999 році науковець помер від інсульту, однак у Мексиці й Гватемалі
його досі пам'ятають. До речі, саме завдяки Кнорозову писемність майя
отримала офіційну підтримку уряду Мексики і викладається в університетах
та середніх школах у районах поширення майя.
\end{leftbar}

2012 року в мексиканському місті Канкун, що на півострові Юкатан,
встановили перший пам'ятник науковцю, а в березні 2018 --- у Меріді, столиці
Юкатану --- другий. Монумент стоїть поруч із Великим музеєм світу майя, його
автор --- скульптор Рейнальдо Боліо Суарес. На триметровому пам'ятнику
зображений Юрій Кнорозов з улюбленою кішкою Асею на руках.

\ifcmt
pic https://www.5.ua/media/pictures/820x546/201114.jpg
\fi

У 2016 році в рамках декомунізації іменем Юрія Кнорозова назвали вулицю в
Харкові.

\begin{leftbar}
\bfseries
Дешифрування Кнорозова дозволило розпочати серйозні дослідження мови і
культури давніх майя, навіть скласти словник давньої мови і перекласти
їхні джерела. У 1960-80 рр. Кнорозов розшифрував і переклав не тільки
рукописи майя 12-14 ст., а й десятки написів на пам'ятниках, поминальних
судинах, статуетках, історичні хроніки, міфологічні, пророчі і ритуальні
тексти 16-18 ст. з т. зв. \enquote{Книги пророка Ягуара}. Переклади рукописів і
написів на судинах і монументах він супроводжував докладними науковими
коментарями. Тоді ж з'явилися праці з історії мезоамериканських
цивілізацій і історії Юкатана 10-16 ст. Він навіть долучився до
перекладу кількох пісень 17-18 ст. зі збірки \enquote{Пісні з Ц'ітбальче}.
\end{leftbar}

Кнорозова дуже цікавили релігія, міфи й обряди давнього народу. Він навіть
зміг показати, як виникали релігійні концепції і використовувалися в
політичній боротьбі за владу (наприклад, почергове правління чотирьох
богів --- відображення реально існуючого на стадії формування держав
інституту зміни при владі вождів. Боги-розпорядники, що стежили за зміною
влади у богів, слугували для вченого ключем до відтворення структури влади
в ранніх державах майя).

\ifcmt
pic https://www.5.ua/media/pictures/820x546/201117.jpg
caption Піраміда Чичен-Іца
\fi

У 1973-95 рр. Кнорозов перекладав і трактував написи і сцени на
поминальній кераміці і пам'ятниках. Переклад багатьох десятків
династичних, переможних, ритуальних, міфологічних, поминальних, пророчих,
календарних і астрономічних текстів на статуетках і монументах дали змогу
поглянути на культуру майя з раніше абсолютно невідомої сторони. Переклад
Кнорозовим кільцевих написів на судинах --- стандартного гімну на честь
покійного, дозволив зрозуміти призначення парадної кераміки: судини і
статуетки були атрибутом поминального (не плутати з поховальним) ритуалу.

Завдяки перекладам Кнорозова вдалося отримати унікальні відомості про
структуру підземного пантеону, титули і функції богів, уявлення про душу,
обряди віщування під дією наркотиків, людські жертвоприношення, обряди
ініціації, поминальні бенкети, новорічні церемонії, перші відомості про
храмове землеволодіння, структуру влади, жрецької та військової
організації, імена і титули правителів і воєначальників майя.

У статтях про мезоамериканські календарі, що грали величезну роль у
створенні й існуванні всіх цивілізацій регіону, Кнорозов детально описав
зміни календарів у давній Америці.

\begin{leftbar}
        \bfseries
Пояснюючи свій успіх у дешифруванні, Юрій Кнорозов дуже серйозно
розповідав: \enquote{Коли мені було не більше п'яти років, брати стукнули мені
по лобі крокетною кулею ... Зір відновили, хоча і з труднощами. Мабуть,
це була своєрідна \enquote{чаклунська травма}. Можу дати рекомендацію: майбутніх
дешифровщиків бити по голові, тільки не зрозуміло, як. Можна для
експерименту взяти контрольну групу, --- а якщо хто кінці віддасть, тому
так і треба!} --- і посміхався.
\end{leftbar}

\emph{Інформація з відкритих джерел}

\paragraph{Читайте також:}

Студенти проти комуністичної і нацистської диктатур --- унікальне відео революцій, що змінили Європу.

\paragraph{Читайте за темою}

\begin{itemize}
\item Без нього \enquote{Титанік} став би іншим: як українець здійснив технічну революцію в
Голлівуді

\item \enquote{Анекдот} від Мустафи Джемілєва: як правозахисник відповів на питання \enquote{чи
сьогодні важче, ніж було в минулому?}

\item Син Франка і жінка-хорунжий: українці Першої світової війни, яких Росія
        викинула з нашої історії
\end{itemize}

