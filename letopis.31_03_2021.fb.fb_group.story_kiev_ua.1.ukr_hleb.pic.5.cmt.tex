% vim: keymap=russian-jcukenwin
%%beginhead 
 
%%file 31_03_2021.fb.fb_group.story_kiev_ua.1.ukr_hleb.pic.5.cmt
%%parent 31_03_2021.fb.fb_group.story_kiev_ua.1.ukr_hleb
 
%%url 
 
%%author_id 
%%date 
 
%%tags 
%%title 
 
%%endhead 

\iusr{Semyon Belenkiy}
Подол, Нижний Вал

\iusr{Алла Поповская}
Нижний Вал 21 @igg{fbicon.heart.red}{repeat=3}

\iusr{Serge Serge}
Он родненький.

\iusr{Stella Stavitsky}
спасибо за фотографии, прекрасно помню этот хлебный, тут хожено, перехожено кажется, что вернулась в юность

\iusr{Яков Любомирский}
Я и мой покойный отец, очень любили этот магазин.

\iusr{Валентин Степаненко}
Знайоме місце на Нижньому Валу. Інструменти теж купував

\iusr{Vadim Tkach}
Хлібний магазин, Нижній Вал 21, навпроти, на Верхньому Валу 18 такий же Хлібний. Верхнє фото - хлібний магазин на вул Сагайдачного. - Обжиті місця.

\iusr{Юлий Жигалов}

Справа магазин, где я купил первый раскладной нож... Эдак, в 1983-84.. Так
гордился))). А запах советского черного \enquote{Украинского} ни с чем не спутаешь...
Особенный вкусный запах!)))

\iusr{Людмила Обелец}


\ifcmt
  ig https://i2.paste.pics/9fd074c803c1be6e2d1db21dcc641966.png
  @width 0.2
\fi

\iusr{Лилия Гончарова-Данченко}

На Фрунзе, вначале по ходу от базара слева, был маленький магазинчик, (сейчас
продлили речучилище) продавали \enquote{жулики}. Типа сайки только чёрный и с изюмом

\iusr{Natali Gudyrenko}
Мой родненький Подол!

