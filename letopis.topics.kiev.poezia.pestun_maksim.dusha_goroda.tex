% vim: keymap=russian-jcukenwin
%%beginhead 
 
%%file topics.kiev.poezia.pestun_maksim.dusha_goroda
%%parent topics.kiev.poezia
 
%%url 
 
%%author_id 
%%date 
 
%%tags 
%%title 
 
%%endhead 

\url{https://stihi.ru/2017/06/21/6513}

Душа города
Максим Пестун

Я в древний город свой влюблен,
И в дух веков, что в нем теплится.
Знаком мне с детства каждый дом
И часто-часто ночью снится.

Домов немало нет уже,
А те, что отданы на гибель,
Стоят покорные судьбе,
Готовясь пасть в застроек тигель.

Нас очень мало киевлян...
Давно разбросаны по свету
Не можем выйти на майдан
И стать стеной у горсовета.

Осталась горстка тех людей,
Что продолжают бой неравный
За право памяти своей
И защищают город славный.
 
Ведь в каждом доме есть душа,
И в каждом камне  - дух историй.
Их создавали на века,
Не для захвата территорий!

Батый палил в недобрый час,
Большевики взрывали храмы...
Но возрождался всякий раз
Великий Город вечной Славы!
