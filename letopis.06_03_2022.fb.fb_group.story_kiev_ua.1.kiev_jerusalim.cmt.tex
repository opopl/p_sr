% vim: keymap=russian-jcukenwin
%%beginhead 
 
%%file 06_03_2022.fb.fb_group.story_kiev_ua.1.kiev_jerusalim.cmt
%%parent 06_03_2022.fb.fb_group.story_kiev_ua.1.kiev_jerusalim
 
%%url 
 
%%author_id 
%%date 
 
%%tags 
%%title 
 
%%endhead 
\zzSecCmt

\begin{itemize} % {
\iusr{Виктор Кошевенко}

Дякую, Сергію, що пишеш!  @igg{fbicon.hands.pray}  Дякую за співчуття!  @igg{fbicon.hands.pray}  Я біля Бучі, Гостомеля,
Ірпіня... Там пекло... Усе горить і чорне!  @igg{fbicon.cry} А нас \enquote{просто} трясе, будинок
хитае, скло дружить!.. Живемо... Обнімаю  @igg{fbicon.heart.sparkling}{repeat=2}  @igg{fbicon.hands.pray}{repeat=2} 

\begin{itemize} % {
\iusr{Anna Altman}
\textbf{Виктор Кошевенко} а где именно вы ? Моя родня говорит нету света , они в Клавдиево тарасовом на даче  @igg{fbicon.face.sleepy}{repeat=2}  я не могу с ними связаться

\iusr{Виктор Кошевенко}
\textbf{Anna Altman} Я в Киеве
\end{itemize} % }

\iusr{Наталя Кудря}

зараз про волонтеріва і допомогу - кому що потрібно актуально ...прогавили
меморіал - купол Ізрайль пошкодував

\iusr{Tanya Bokun}

@igg{fbicon.hands.pray} @igg{fbicon.heart.red} @igg{fbicon.heart.blue}
@igg{fbicon.heart.yellow} @igg{fbicon.flag.ukraina}

\iusr{Marfa Pylypenko}
\textbf{\#standwithukriane}

\iusr{Светлана Писаренко}
Спаси і Сохрани, Господи, Україну

\iusr{Сергей Векслер}

Не можу мовчати, біль і гнів переповнює всього від того, що бачу, чую,
відчуваю. Не можна повірити, що на таке здатне продовження (нібито продовження)
тих, які перебороли фашизм. Нащадки, які зловживають не своєю славою і
заслугами...

Росія знищує Україну??? Як таке взагалі можливо?..

Щойно побачив маленьких діточок, яких екстрено евакуюють з Бучі, з Ірпеня.

І згадав про одну історію, яку вчасно згадати зараз. Може, це комусь надасть
сил... Маю написати про це - і поширити, саме у ці дні.

Ці події відбувалися у Польщі, в Варшаві. Всіх євреїв Варшави гнали у гетто.
Маленька дівчинка з мамою йшла разом з усіма. І мати побачила раптом ляльку,
яку підібрала для своєї маленької донечки. Ця лялька, якій вже у гетто жінка
сшила з лоскутків сукню, і стала ця лялька для її мейделе донькою.

Якось жили, те, що залишилося з цінного, міняли на хліб, який крадучись
приносив у гетто польській юнак. У нього був лаз, яким він користувався для
того, щоб потрапити у гетто.

По гетто почала розповсюджуватися чутка, що ось-ось має бути акція і мають
стратити багато людей. Мати приготувала саме цінне, що зберігала на самий
чорний день,- золоте кільце і сережки, щоб віддати цей скарб польському юнаку і
упросити хлопчика унести маленьку дівчинку з гетто: будь що буде, але ж не
смерть для її донечки.

Він прийшов – і поклав дівчинку у мешок, проліз з нею на арійську частину
Варшави.

І раптом дівчинка почала бити хлопця по плечах скрізь мешок.

– Що трапилося? - спитав він.

– Нам треба повернутися, - відповіла вона. - Там Зося...

- Ти що, здуріла?

- Зося - це моя донька. Мати не може залишити свою дитину...

Юнак сам був дитиною - і тому прислухався до дівчинки.

Знову гетто, знову шлях назад. Поверталися вони вже втрьох: хлопець, дівчинка і
... лялька Зося.

Польська родина врятувала дівчинку. Імені її не знаю. У 1948 році вона
потрапила до Ізраїлю. З нею сюди прибула і Зося. Дівчина вже в Ізраїлі взяла
собі нове ім'я: Яель. У похилому віці постарівша Яель передала свою Зосю до Яд
Вашем.

Ось вона, Зося.

Історія людяності, вірності, дитячої чистоти і віри - і зрятованого життя...

Раніше уявити собі не міг, що розкажу знову цю історію українською мовою, у
зв'язку з подіями, що відбуваються у ХХІ столітті - поруч з моїм Києвом...

Вибачте за припіску. Не хочеться в це вірити: поганішого за це жахливе, що
робиться зараз, уявити собі було неможливо.

Отже, варшавянка лялька Зося. Яд Вашем, Єрусалим... Не сьогоднішня світлина.
Світ в віконці, який обіцяє надію.

\ifcmt
  ig https://scontent-cdt1-1.xx.fbcdn.net/v/t39.30808-6/275224611_10225252476552705_8155032270791462702_n.jpg?_nc_cat=110&ccb=1-5&_nc_sid=dbeb18&_nc_ohc=82v4_CKuNn8AX-pF_Dw&_nc_ht=scontent-cdt1-1.xx&oh=00_AT863ZvpzieiWihu-OUzAOL2AzzQaTTfA6-uOJ94gpU2Fg&oe=622B1FD7
  @width 0.3
\fi

\iusr{Володимир Цибульський}
Дикі люди замаскувались під антифа.

\end{itemize} % }
