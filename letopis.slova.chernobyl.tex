% vim: keymap=russian-jcukenwin
%%beginhead 
 
%%file slova.chernobyl
%%parent slova
 
%%url 
 
%%author 
%%author_id 
%%author_url 
 
%%tags 
%%title 
 
%%endhead 
\chapter{Чернобыль}

Поспішаю повідомити, що \enquote{\emph{Чорнобиль} 1986} Даніли Козловського
зараз на 1 місці в українському Нетфліксі. Це та сама друга сільнєйша атвєточка
росіян на \enquote{Чорнобиль} від HBO.  Розчарована, бо чекала на розвагу. А
воно якось і без трешняка. А я ж поціновувачка такого. Фаєрмен у виконанні
самого Козловського акі Джон Макклейн усіх спаса. В основному весь екшн
крутиться навколо групи водолазів, які ідуть під реактор відкривати шлюз із
водою. Решта - лав сторі.  Показу притрушеної совєцької системи, як ви
здогадуєтеся, по мінімуму. Ну обмовилися про бюрократію, про демонстрації на 1
травня, побідкалися, що треба було краще будувати реактор. Все. Про жахливе
замовчування і причини вибуху шановним глядачам не розповідають. Акцент - на
героїзм простих радянських людей.  Чего и следовало ожидать. Продюсер, до речі,
Олександр Роднянський.  Хоч фільм і дуже прохідний та дуже лояльний до СРСР,
але ним росіяни хоч оминули справжнє позоріщє. У 2019 канал НТВ, одразу після
показу серіалу від HBO, почав тикати всім, що почав знімати атвєт. На ютубі
негайно з'явився трейлер. У ньому герой Ігоря Петренка (радянський
контррозвідник) приїздить у \emph{Чорнобиль}, щоб боротися проти аґента ЦРУ,
який і влаштував аварію,
\citTitle{Російський фільм про Чорнобиль лідирує в українському Нетфліксі}, 
Лєна Чиченіна, gazeta.ua, 09.06.2021



%%%cit
%%%cit_head
%%%cit_pic
%%%cit_text
Наша традиционная рубрика \enquote{Ни дня без зрады}. Патриотическую истерику
можно придумать буквально из ничего. Даже по случаю выхода нового украинского
продукта.  \enquote{Многие украинские поклонники негативно восприняли
российскую транслитерацию слова Chernobyl в названии игры. Так как по правилам
украинский город Чорнобиль разработчики должны были перевести, как Chornobyl},
– говорится в сообщении.  Из-за неправильной транслитерации и трейлера,
выпущенного на русском языке, многие западные игроки решили, что игру делают
российские разработчики. В результате в Twitter появился хештег
\verb|#ChornobylNotChernobyl|. Также на русском языке говорят и персонажи, что
также не понравилось некоторым поклонникам
%%%cit_comment
%%%cit_title
\citTitle{Украинские патриоты устроили истерику по поводу русскоязычного названия игра Chernobyl}, 
Андрей Манчук, strana.ua, 16.06.2021
%%%endcit


