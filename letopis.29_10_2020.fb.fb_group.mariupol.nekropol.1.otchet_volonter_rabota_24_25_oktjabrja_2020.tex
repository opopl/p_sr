%%beginhead 
 
%%file 29_10_2020.fb.fb_group.mariupol.nekropol.1.otchet_volonter_rabota_24_25_oktjabrja_2020
%%parent 29_10_2020
 
%%url https://www.facebook.com/groups/278185963354519/posts/398989414607506
 
%%author_id fb_group.mariupol.nekropol,arximisto
%%date 29_10_2020
 
%%tags 
%%title Отчет о волонтерской работе в Некрополе 24-25 октября 2020
 
%%endhead 

\subsection{Отчет о волонтерской работе в Некрополе 24-25 октября 2020}
\label{sec:29_10_2020.fb.fb_group.mariupol.nekropol.1.otchet_volonter_rabota_24_25_oktjabrja_2020}
 
\Purl{https://www.facebook.com/groups/278185963354519/posts/398989414607506}
\ifcmt
 author_begin
   author_id fb_group.mariupol.nekropol,arximisto
 author_end
\fi

\bigskip
\textbf{Отчет о волонтерской работе в Некрополе 24-25 октября 2020}

\textbf{Находки и открытия}
\bigskip

Сенсацией прошедших выходных стало обнаружение Илья Луковенко внутри ограды
склепа Спиридона Гофа – еще одного склепа! По крайней мере, находка выглядит
как верхушка небольшого склепа. Прямоугольная кирпичная кладка, которая вроде
бы уходит в землю.

Мы не успели ее \enquote{докопать}, поэтому это – сплошная загадка! Кладка явно
выглядит дореволюционной. Понятно, что внутри ограды Спиридона Гофа могли
похоронить только кого-то из семейства Гофов. Но зачем их хоронить отдельно,
если уже есть большой склеп?! Или именно этот обнаруженный склеп является самым
древним?

Стоит напомнить, что до начала восстановления и уборки вся территория вокруг
Спиридона Гофа была мусорной свалкой из бутылок, шприцов, хвороста, обломков
кирпичей и камней. Среди них мы нашли и обломки плиты, но – без надписей.
Возможно, она лежала сверху этого новооткрытого склепа....

... а мы уже собирались засыпАть исследованное основание склепа Гофа и высаживать цветы...

Помогите разобраться с этой загадкой!

В ту же субботу Елена Сугак и Maryna Holovnova обнаружили загадочное
известняковое основание рядом с оградой Александра Хараджаева. Работу пришлось
приостановить – склеп Хараджаева тоже был масштабной мусорной свалкой. И нужно
сначала убрать беспорядочно разбросанные бетонные обломки советских памятников,
чтобы продолжить расчистку...

Наконец, благодаря уборке Марины удалось вернуть из забвения Константина
Эберлинга (см. подробнее \footnote{А кем был Константин Эберлинг (Konstantin Eberling)?, Andrei Marusov, 26.10.2020, \par%
\url{https://www.facebook.com/groups/278185963354519/posts/396501234856324}}%
)

\textbf{Благоустройство}

Мы спилили последние сухостойные деревья на древнем участке Гофов-Хараджаева.
Не закончили распилку – бензопила отказалась работать. Ремонтируем.

Насобирали шесть мешков мусора. И это только на древнем участке...

Начали разгребать \enquote{холмы} вдоль тропинки, ведущей к древнему участку.
Десятилетиями родственники умерших скидывали в это место ветки, листву... В
результате получилась перегнойная земля. Используем ее для высадки цветов и
деревьев. Надеялись найти под ней плиты – пока безрезультатно.

\textbf{О волонтерах}

За последние недели к волонтерской команде присоединились Елена Сугак и Марина
Головнова! И сразу же сделали несколько находок. Так держать! J И – огромное
спасибо за самоотверженную работу!

\textbf{Планы на выходные 31 октября – 1 ноября}

Мы изменили время сбора участников – теперь стартуем в 10 утра и в субботу, и в
воскресенье. Место сбора неизменно – у белого Памятного креста в центре
Некрополя. Приходите и после старта - контактный телефон 096 463 69 88.

Что планируем сделать? Исследовать новооткрытый склеп у Спиридона Гофа,
известняковое основание у Александра Хараджаева; расчистить \enquote{холмы}; разметить
весь древний участок (где находятся могилы, где можно проложить дорожки, где
посадить цветы, деревья...) и – продолжить возвращать из забвения имена
создателей Мариуполя!

Присоединяйтесь, друзья!

p.s. Обязательно берите свои перчатки и маски!

Пока что благодарные предки нас берегут...

\#mariupol\_necropolis\_report
