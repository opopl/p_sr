% vim: keymap=russian-jcukenwin
%%beginhead 
 
%%file 13_08_2021.fb.maljar_ganna.1.maguchih_vstrecha
%%parent 13_08_2021
 
%%url https://www.facebook.com/ganna.maliar/posts/1949340971891602
 
%%author Маляр, Анна
%%author_id maljar_ganna
%%author_url 
 
%%tags __aug_2021.maguchih.foto.olimpiada.lasickene,maguchih_jaroslava,olimpiada.tokio,ukraina,vsu
%%title Тепер, як і обіцяла, інформую про зустріч з Ярослава Магучих
 
%%endhead 
 
\subsection{Тепер, як і обіцяла, інформую про зустріч з Ярослава Магучих}
\label{sec:13_08_2021.fb.maljar_ganna.1.maguchih_vstrecha}
 
\Purl{https://www.facebook.com/ganna.maliar/posts/1949340971891602}
\ifcmt
 author_begin
   author_id maljar_ganna
 author_end
\fi

10 з 19 олімпійських медалей на літніх Олімпійських іграх у Токіо для України
завоювали спортсмени Збройних сил України! Кожна з них - це величезний привід
для гордості.  

Тепер, як і обіцяла, інформую про зустріч з Ярослава Магучих 

Ми зустрілись. Килима там не було(цей вкид з російського інтернет-сегменту
добре мобілізував їхні інформаційні гармати, але як показала ця ситуація,
українці давно навчились розрізняти хто є хто).

Нагадаю, я зверталась до всіх з проханням припинити цькування допоки Ярослава
сама не пояснить ситуацію.

Пояснення є: фото обіймів зі спортсменкою РОК - це наслідок емоцій. Ярослава
знає, що в Україні триває російсько-українська війна, розуміє, що відбулось.
Сумує з приводу того, як використали її фото в антиукраїнській пропаганді. Вона
вважає за честь належати до лав військовослужбовців ЗСУ та вірна Україні. 

Тепер, що стосується "букви закону". 

Принципи публічного спілкування у Збройних силах України визначає Доктрина
публічного спілкування.

Згідно з п 1.4 Доктрини  -  відкриті публічні заходи є однією з форм публічного
спілкування військовослужбовця.

У п. 1.6 Доктрини встановлюється наступне: організація та контроль результатів
публічного спілкування  покладається на безпосередніх керівників  та визначених
посадових осіб органів військового управління, військових частин, установ та
організацій.

Відповідно до п.2.1 згаданої Доктрини "кожен військовослужбовець в очах
суспільства є уособленням військової організації держави... Дії або
бездіяльність всіх військових посадових осіб, що призвели до іміджевих втрат
для ЗСУ, є неприпустимими". 

Тож,  наразі, згідно чинного законодавства, проводиться робота з безпосереднім
керівником. 

Для того, щоб поставити крапку у цьому питанні повідомляю - недоопрацювання
буде усунуте,  спортсмени-військовослужбовці  у подальшому будуть належним
чином ознайомлені з Доктриною публічного спілкування МОУ.

\ii{13_08_2021.fb.maljar_ganna.1.maguchih_vstrecha.cmt}
