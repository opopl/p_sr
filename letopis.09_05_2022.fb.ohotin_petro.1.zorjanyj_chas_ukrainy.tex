% vim: keymap=russian-jcukenwin
%%beginhead 
 
%%file 09_05_2022.fb.ohotin_petro.1.zorjanyj_chas_ukrainy
%%parent 09_05_2022
 
%%url https://www.facebook.com/petro.okhotin/posts/7453400054702455
 
%%author_id ohotin_petro
%%date 
 
%%tags 
%%title Ми спостерігаємо зоряний час України
 
%%endhead 
 
\subsection{Ми спостерігаємо зоряний час України}
\label{sec:09_05_2022.fb.ohotin_petro.1.zorjanyj_chas_ukrainy}
 
\Purl{https://www.facebook.com/petro.okhotin/posts/7453400054702455}
\ifcmt
 author_begin
   author_id ohotin_petro
 author_end
\fi

Ми спостерігаємо зоряний час України. І ним потрібно скористатися для того, щоб
перевести геополітичний флірт у геоекономічні довгі відносини. 

Боно, Джолі, Джил Байден в українських містах. Зеленського ставлять на
обкладинки світові ЗМІ. Українськими ЗСУ захоплюється західний світ. 

Стоп! Цукерково-букетний період не буває вічним. 

Війська Росії в Україні  з квітня очолює генерал Дворніков. З досвідом ведення
бойових дій у Сирії. 

Російська спецоперація мовою цифр зі статті Сергія Шойгу в газеті «Красная
звезда». З 2015 року по 2020 Росія зробила 44 тисячі бойових вильотів. Авіацією
та крилатими ракетами було знищено 133 542  об'єктів інфраструктури, 400
нафтопереробних заводів, 4100 паливних заправок. Росія застосовувала особистий
терор проти лідерів опозиції — їх знищено було 865, за даними Шойгу. 

Іншими словами, росіяни використовують не методичку з Грузії. Грають вдовгу. Ви
це бачите з реальності, без могу досвіду. 

\href{https://www.facebook.com/victor.andrusiv}{Victor Andrusiv} нещодавно
написав пост, звідки я виділяю два головні меседжі. 

Перше, Росії потрібно дотягнути до листопада. Початок опалювального сезону і
продовольча криза дозволить Росії домовлятися зі світом.

По-друге, ми не можемо залити умовний Захід грішми. Проте можемо виграти емоціі
виборців — елітам потрібні голоси. 

До роздумів Віктора я би  додав ще один фактор. 

Потрібно більше доносити стратегічні цілі оборони України у тому числі тим, щоб
пояснювати ситуацію і контекст. У своєму підрозділі цю задачу я вирішував,
наприклад, запросивши Ivan Patrylak з лекцією з історії України. 

Оборона України — це не лише збройний захист. 

Також це відновлення визволених населених пунктів. 66\% відновленого
газопостачання в Ірпені чи відбудована Охтирська ТЕЦ — така ж перемога як
збитий літак ворога.

Оборона України — це і повернення інтелектуальних дискусій про роль України. Ми
не можемо виїжджати на суто антиросійській риториці. Кремль, наприклад,
жонглює багатополярністю і антилібералізмом. Наше завдання - виробити місію
України.

І гроші. Потрібно не соромитися того, що нам потрібна модернізація і полегшення
роботи для активного населення. З грошей фінансується оборона. Гроші
допомагають робити кампанії.

Війна зруйнувала пострадянську стару модель. І максимум до жовтня 2022 року ми
маємо народити нову діючу модель України.

\ii{09_05_2022.fb.ohotin_petro.1.zorjanyj_chas_ukrainy.cmt}
\ii{09_05_2022.fb.ohotin_petro.1.zorjanyj_chas_ukrainy.cmtx}
