% vim: keymap=russian-jcukenwin
%%beginhead 
 
%%file 08_11_2020.fb.evgeniy_maslov.1.mir_bez_menja
%%parent 08_11_2020
 
%%url https://www.facebook.com/maslovevgeniy14/posts/1342077099467328
%%author 
%%tags 
%%title 
 
%%endhead 

\subsection{Я видел мир, в котором нет меня…}
\Purl{https://www.facebook.com/maslovevgeniy14/posts/1342077099467328}

\Pauthor{Михейшин, Владимир}

Я видел мир, в котором нет меня…
В нем было все по прежнему теченью.
Размеренно день считывал мгновенья
При этом равнодушие храня
К тому, что больше не было меня. 
С востока Солнце двигалось, маня
Величием вселенского значенья,
На запад. Мир ловил его свеченье,
Считая это окончаньем дня.
Но, в этом Мире не было меня.
Парили птицы, разбивая высь,
И дождь с небес не шел, скорее капал.
Облезлый кот акацию царапал.
Как и при мне: прохожие плелись,
По лужам перешагивая жизнь.
Все было так же, только без меня.
И город суетою жил привычной,
Как и при мне, ну может чуть приличней,
На Греческой, гитарою звеня
Пел музыкант прохожим… Без меня.
И понял я существованья суть,
Когда тревожно я под утро ожил:
Важнее нет вопроса – как ты прожил?
И кто тебя захочет помянуть,
Когда вдруг оборвется жизни путь?
Что будет в мире, где нас больше нет?
Забвенье, пустота из многоточий,
Иль из отстроенных душою строчек,
В которых столько радостей и бед,
Останется в чужой душе – поэт,
Несущий слово, а не просто – прочерк?
