% vim: keymap=russian-jcukenwin
%%beginhead 
 
%%file 13_08_2022.stz.news.ua.donbas24.3.medbrat_z_mrpl
%%parent 13_08_2022
 
%%url https://donbas24.news/news/medbrat-z-mariupolya-shhodnya-mi-priimali-do-150-poranenix-pomirali-diti-foto
 
%%author_id ivanova_jana.mariupol,news.ua.donbas24
%%date 
 
%%tags 
%%title Медбрат з Маріуполя: "Щодня ми приймали до 150 поранених, помирали діти"
 
%%endhead 
 
\subsection{Медбрат з Маріуполя: \enquote{Щодня ми приймали до 150 поранених, помирали діти}}
\label{sec:13_08_2022.stz.news.ua.donbas24.3.medbrat_z_mrpl}
 
\Purl{https://donbas24.news/news/medbrat-z-mariupolya-shhodnya-mi-priimali-do-150-poranenix-pomirali-diti-foto}
\ifcmt
 author_begin
   author_id ivanova_jana.mariupol,news.ua.donbas24
 author_end
\fi

\ii{13_08_2022.stz.news.ua.donbas24.3.medbrat_z_mrpl.pic.front}
\begin{center}
  \em\bfseries\Large
Історія медпрацівника обласної лікарні Маріуполя про те, як лікарі рятували
поранених під обстрілами
\end{center}

Обласна лікарня інтенсивного лікування у Маріуполі з перших днів
повномасштабної війни приймала до 150-ти поранених. Медики рятували мешканців
не тільки міста, але й прилеглих селищ району. Лікарі робили все, аби зберегти
життя людей.

Медбрат \textbf{Дмитро Гавро} розповів у \href{https://www.facebook.com/100023195386675/posts/pfbid02C4w3raC4wBUodVC12NKKmgip1N9ngWwvFfPnMZt48spDaGWqeWGXzT9sg4i3Uru9l}{соцмережах}%
\footnote{\url{https://www.facebook.com/100023195386675/posts/pfbid02C4w3raC4wBUodVC12NKKmgip1N9ngWwvFfPnMZt48spDaGWqeWGXzT9sg4i3Uru9l}}
про операції без світла та під \enquote{прильотами} поряд з лікарнею, як рятували вагітних після скинутої бомби на
пологовий будинок та дітей.

\ifcmt
  ig https://i2.paste.pics/PTC8F.png?trs=1142e84a8812893e619f828af22a1d084584f26ffb97dd2bb11c85495ee994c5
  @wrap center
  @width 0.9
\fi

\ifcmt
  ig https://i2.paste.pics/PTC8K.png?trs=1142e84a8812893e619f828af22a1d084584f26ffb97dd2bb11c85495ee994c5
  @wrap center
  @width 0.9
\fi

Донбас24 публікує цю історію без скорочень з його слів.

\ii{13_08_2022.stz.news.ua.donbas24.3.medbrat_z_mrpl.pic.1}
\ii{13_08_2022.stz.news.ua.donbas24.3.medbrat_z_mrpl.pic.2}

\subsubsection{Ми падали на підлогу під бомбами і молилися}

Я — медбрат приймального відділення Маріупольської обласної лікарні. Вранці
24-го лютого вже думав йти до лав ЗСУ, коли мені написали: \enquote{Хлопці, виходьте
всі на роботу, дуже багато поранених}. І я рвонув у лікарню. Я був потрібніший
там.

Першого ж дня ми прийняли понад 100 людей. Із околиць міста і селищ поблизу.
Наша обласна лікарня першою відчула весь страх війни. Допомагали,
перев’язували, оперували. Але й не здогадувалися, що від цього дня ніхто з
лікарів уже не зможе повернутися додому.

Днями не спиш, бо привозять і привозять поранених. Спини не відчуваєш, бо
електрики немає, і носиш пацієнтів на руках. Розриваєшся від болю, але не
фізичного. Бо ще болючіше — це страх за своїх рідних. Щоразу, як падали
обличчям у підлогу… Так, ви правильно зрозуміли, ми везли пацієнта коридором,
коли над головою починали скидати бомби. І коли ми просто падали на підлогу і
лежали, я просив тільки, щоб мої близькі залишились цілими.

Того дня, після всього, що ми пережили у лікарні, я вирішив зробити пропозицію
своїй дівчині. Щоправда, обручку пообіцяв після війни. Але почув \enquote{Я згодна} —
і, щасливий, знову полетів у лікарню. Там на мене чекали ще кілька тижнів
жахіть.

\subsubsection{На моїх очах помирали діти}

Допомагали всім, стояли по лікті у крові, перев'язували. Зривали спини, бо
носили пацієнтів на руках. Закривали очі дітям, які загинули. А коли починалися
обстріли, падали обличчям у підлогу просто в коридорі та лежали під стінами, що
трусилися. Було страшенно важко. Це було наше пекло.

Щодня ми приймали понад 120−150 поранених. Оперували всюди, де було світло. По
лікті у крові, падаючи від обстрілів, серед стін, що трусилися вже кожні 30
хвилин. Було дуже страшно. День у день. Лише наша лікарня прийняла до 10
березня 742 пацієнти. 742...

Серед них були і дівчата з пологового, яких привезли після бомбардування 9
березня. Зустрічали, сортували їх за тяжкістю поранень. Одна дівчина була на 40
тижні вагітності. У неї були множинні поранення ніг і тулуба. Ні її, ні дитину
врятувати не вдалося.

За ці дні на моїх очах померли шестеро дітей. Приймаєш, починаєш проводити
реанімацію. Як описати словами те, що відчуваєш у цей момент? Неймовірно важко
дивитись на дитяче обличчя, чекаючи реакції організму... а бачити очі, що
закотилися. Це були дні пекла.

Та тоді я відчув дещо важливе. Що ніколи не покину своєї професії. Я — медик, і
я маю стояти непорушною перепоною між життям і смертю. Це моя мрія, і вона тоді
здійснилася.

Нагадаємо, що \href{https://donbas24.news/news/xlopec-z-mariupolya-otrimav-vidznaku-prezidenta-za-spasinnya-lyudei-pid-obstrilami-video}{хлопець з Маріуполя нагороджений президентом за спасіння людей
під обстрілами}.%
\footnote{Хлопець з Маріуполя нагороджений президентом за спасіння людей під обстрілами, Яна Іванова, donbas24.news, 12.08.2022, \par\url{https://donbas24.news/news/xlopec-z-mariupolya-otrimav-vidznaku-prezidenta-za-spasinnya-lyudei-pid-obstrilami-video}}

Ще більше новин та найактуальніша інформація про Донецьку та Луганську області
в нашому телеграм-каналі Донбас24.

ФОТО: Дмитро Гавро
%\ii{13_08_2022.stz.news.ua.donbas24.3.medbrat_z_mrpl.txt}
