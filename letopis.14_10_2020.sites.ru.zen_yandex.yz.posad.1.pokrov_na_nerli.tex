% vim: keymap=russian-jcukenwin
%%beginhead 
 
%%file 14_10_2020.sites.ru.zen_yandex.yz.posad.1.pokrov_na_nerli
%%parent 14_10_2020
 
%%url https://zen.yandex.ru/media/posad/posetili-pokrov-na-nerli-tainstvennyi-hram-polnyi-zagadok-5f86be503940476c66de883b
 
%%author 
%%author_id yz.posad
%%author_url 
 
%%tags russia,istoria,pokrov_na_nerli
%%title Посетили Покров на Нерли. Таинственный храм, полный загадок
 
%%endhead 
 
\subsection{Посетили Покров на Нерли. Таинственный храм, полный загадок}
\label{sec:14_10_2020.sites.ru.zen_yandex.yz.posad.1.pokrov_na_nerli}
\Purl{https://zen.yandex.ru/media/posad/posetili-pokrov-na-nerli-tainstvennyi-hram-polnyi-zagadok-5f86be503940476c66de883b}
\ifcmt
  author_begin
   author_id yz.posad
  author_end
\fi

\index[rus]{Русь!Храмы!Покров на Нерли, 14.10.2020}

Когда Батый привёл свои орды на Русь и вышли они в чисто поле, взгляду
«царя-Батыги» предстал храм. Стоял он один в чистом поле. Батый был настолько
очарован прекрасным храмом, что повелел выставить вокруг него стражу, дабы
воины его не разрушили.

\ifcmt
  pic https://avatars.mds.yandex.net/get-zen_doc/1222384/pub_5f86be503940476c66de883b_5f86cd943940476c66008b22/scale_1200
  caption Взятие Владимира.
\fi

Речь здесь идёт об одном из прекраснейших храмов не только земли Владимирской,
но и всей Средней России, храме Покрова на Нерли. Храм таит в себе множество
загадок, которые тянутся за ним сквозь толщу веков.

Правда, или нет про Батыя, не знаю. Но храм стоит в чистом поле с домонгольских
времён. Его внешнее и внутреннее убранство сохранялось до XIX века (об этом
позже). Другие же храмы Руси пострадали во время ига довольно сильно.

\ifcmt
  pic https://avatars.mds.yandex.net/get-zen_doc/1908497/pub_5f86be503940476c66de883b_5f86c7cd3940476c66f3fbeb/scale_1200
  width 0.4
\fi

На школьных каникулах мы целенаправленно отправились в Боголюбово. Во-первых,
потому, что грех не видеть своими глазами красоты Родины, а во-вторых, для
изучения истории увидеть эту самую историю своими глазами весьма полезно.

Нам повезло. Несмотря на октябрь стояла чудная погода загостившегося бабьего
лета. Посетителей (паломников?) было немного. Всего лишь один туристический
автобус и человек 10 путешествующих самостоятельно. И, главное, никаких
вездесущих китайцев!

\ifcmt
  pic https://avatars.mds.yandex.net/get-zen_doc/1878571/pub_5f86be503940476c66de883b_5f86c7e2ae6a9712bf3ad78e/scale_1200
  caption Эта дорога к храму - единственная, других путей нет. В половодье до него можно добраться на лодке. Нам повезло. Несмотря на погожий день, народу было мало.
  width 0.4
\fi

Если захотите увидеть храм без толчеи и суеты (обычно желающих прийти к храму
очень много), лучше выбирать для похода будний день. И лучше с утра. Но, если
вы хотите сделать потрясающие фотографии, то лучше идти к храму на рассвете.

К храму попасть не просто. Как в русских сказках, нужно преодолеть три
препятствия.

\ifcmt
  pic https://avatars.mds.yandex.net/get-zen_doc/3514290/pub_5f86be503940476c66de883b_5f86c7fc01c3532acccef296/scale_1200
  width 0.4
\fi

Первое. Не прозевать поворот дороги. Поворот на храм очень незаметный и…
стремительный. Чуть зазеваешься, и пролетишь мимо. Поэтому ориентиром возьмём
Боголюбский Рождественский монастырь. Его нельзя не заметить. Если вы
двигаетесь со стороны Москвы, то, как только проедете монастырь, смотрите
внимательно: справа будет указатель. Поворачиваете по нему направо и едете до
железнодорожной станции.

\ifcmt
  pic https://avatars.mds.yandex.net/get-zen_doc/3892121/pub_5f86be503940476c66de883b_5f86c812ae6a9712bf3b3c25/scale_1200
  width 0.4
\fi

Второе. Подъезда к храму Покрова на Нерли нет. От железнодорожной станции нужно
преодолеть много ступенек (перейти мост). Мост сам по себе красивый, с него
открываются потрясающие виды и на монастырь (с одной стороны), и на Храм (с
другой). К мосту пристроен даже лифт для инвалидов. Но его двери по старой
русской традиции наглухо заколочены. Судя по всему, он не работал с момента
постройки перехода (но, может быть, я и ошибаюсь).

\ifcmt
tab_begin cols=2
  pic https://avatars.mds.yandex.net/get-zen_doc/3956291/pub_5f86be503940476c66de883b_5f86c827ae6a9712bf3b67aa/scale_1200
  width 0.4

  pic https://avatars.mds.yandex.net/get-zen_doc/1657335/pub_5f86be503940476c66de883b_5f86cd6a3940476c66002ed4/scale_2400
  width 0.4
tab_end
\fi

После того, как вы преодолеете мост, вас ждёт следующее испытание – дорога к
Храму. Чуть больше километра пути по заливному лугу. От красоты среднерусской
природы дух захватывает. Чистейший воздух, напоённый ароматом трав. Но на пути
нет ни одной лавочки. Отдохнуть можно только «на чём стоишь».


И вот он, Покров. Неожиданно. Потрясающе. Я прекрасно понимаю Батыя. Храм
удивляет не только красотой и совершенством линий, сколько странной резьбой по
камню. Это если приглядеться. Дальше начинаются загадки.

\ifcmt
tab_begin cols=3
  caption Храм Покрова на Нерли

  pic https://avatars.mds.yandex.net/get-zen_doc/3719229/pub_5f86be503940476c66de883b_5f86c8639cd6237d30445e69/scale_1200
  width 0.3
  caption Мне понравилась личность (не знаю, как по-другому назвать) справа и особенно её тень. Христианский символ...

  pic https://avatars.mds.yandex.net/get-zen_doc/1244179/pub_5f86be503940476c66de883b_5f86c8c03940476c66f602f8/scale_1200
  width 0.3
  caption В центре - кони?, пожирающие свои хвосты?

  pic https://avatars.mds.yandex.net/get-zen_doc/3986710/pub_5f86be503940476c66de883b_5f86cd493940476c66ffe6ee/scale_1200
  width 0.3
  caption Реки за прошедшие 900 лет изменили свои русла. Теперь перед храмом старица (старое русло Нерли).
tab_end
\fi

Храм Покрова был построен менее, чем через 200 лет после насильственного
крещения (можно сказать, «огнём и мечом») Руси князем Владимиром. Поверьте, для
времён, не знавших ни интернета, ни даже телеграфа, это немного. По данным
разных историков, то ли в 1165, то ли в 1158 году.

Но все едины в том, что храм был построен за один год «летом единым», а вернее
за один сезон (зимой не строили). Обычно на возведение храма уходило 3 – 4
года.

Подземная часть храма едва ли не больше надземной (храм уходит вглубь земли
почти на 5 метров, до материковых глин).

Храм построен на насыпном холме на заливном лугу, на стрелке Клязьмы и Нерли.
Обе реки ежегодно прилагали большие усилия по затоплению окружающих земель и
храм оказывался на острове. Зачем нужно было столько труда, если можно было
построить его в более удобном месте (нет, я понимаю, что холопы у князя
несчитанные, но всё же)?


\ifcmt
tab_begin cols=3
  caption Храм Покрова на Нерли, резьба

  pic https://avatars.mds.yandex.net/get-zen_doc/3725294/pub_5f86be503940476c66de883b_5f86c8fc9cd6237d3045bd12/scale_1200
  caption Ещё "красавец". 
  width 0.3

  pic https://avatars.mds.yandex.net/get-zen_doc/1721884/pub_5f86be503940476c66de883b_5f86c946ae6a9712bf3de2a5/scale_1200
  caption Центральный фронтон. Резьба.
  width 0.3

  pic https://avatars.mds.yandex.net/get-zen_doc/3645545/pub_5f86be503940476c66de883b_5f86c967ae6a9712bf3e292a/scale_1200
  caption Такая резьба по камню по всему периметру храма. Дверь, ведущая в никуда, или в небо?
  width 0.3
tab_end
\fi

По свидетельству Лаврентьевской летописи храм строила международная бригада «из
всех земель мастеры». Есть легенда, что бригаду эту прислал Андрею Боголюбскому
Фридрих Первый, император Римской империи (тот самый Барбаросса, в честь
которого Гитлер потом назвал план по захвату Советского Союза). Как
договаривались князь с императором, история умалчивает.

В 1877 году храм подвергся фантастически непрофессиональной реставрации,
благодаря которой были утрачены все фрески, прежде украшавшие стены храма.
Пострадали и наборные полы. Сейчас в храме голые стены. Кому понадобилось
подобное «рукопопие» и для чего? Загадка.

Более-менее полноценные раскопки на территории храма проводились только в 1954
году.

        
\ifcmt
tab_begin cols=2
  caption Храм Покрова на Нерли, авторские фото

  pic https://avatars.mds.yandex.net/get-zen_doc/3518430/pub_5f86be503940476c66de883b_5f86c9a33940476c66f7f477/scale_1200
  caption Все фотографии в статье - авторские.
  width 0.3

  pic https://avatars.mds.yandex.net/get-zen_doc/3993525/pub_5f86be503940476c66de883b_5f86c9c1ae6a9712bf3ef1e6/scale_1200
  width 0.3
  caption Внутрь храма можно войти. Постоять, помолиться, зажечь свечку. Фотографировать - не благословляется. Мы не стали нарушать запрет. Но, поверьте, стены храма внутри абсолютно "голые", все фрески утрачены. 
tab_end
\fi

С 1992 года храм входит во всемирное наследие ЮНЕСКО. С одной стороны, это
хорошо. На лугу нет ни одной дачки, ни одного зАмка в стиле «фиг знает что»,
вдоль дороги, ведущей к храму нет ларьков, уродующих панораму. Но с другой
стороны, для ведения раскопок теперь нужно разрешение этой международной
организации. Храм Покрова по-прежнему хранит свои тайны.

Храм открыт, можно зайти в него, осмотреть, поставить свечу. Службы в нём
проводятся только по двунадесятым праздникам.

Координаты храма 56.19631, 40.56128.

\ii{14_10_2020.sites.ru.zen_yandex.yz.posad.1.pokrov_na_nerli.comments}
