% vim: keymap=russian-jcukenwin
%%beginhead 
 
%%file slova.sistema
%%parent slova
 
%%url 
 
%%author 
%%author_id 
%%author_url 
 
%%tags 
%%title 
 
%%endhead 
\chapter{Система}
\label{sec:slova.sistema}

%%%cit
%%%cit_head
%%%cit_pic
%%%cit_text
Сюрпризы начались сразу же: выяснилось, что на станции не открылась крышка
отсека научной аппаратуры и отказали шесть из восьми бортовых вентиляторов
станции. Это ставило под угрозу большинство научных экспериментов и могло
вызвать серьезные осложнения в работе \emph{системы жизнеобеспечения
космонавтов}.  В первопроходцы выбрали самый опытный экипаж. За плечами
командира Владимира Шаталова и бортинженера Алексея Елисеева было уже по два
полета. Исключение составлял лишь инженер-исследователь Николай Рукавишников,
который еще не бывал в космосе. Среди основных задач миссии значились стыковка
и доставка первых космонавтов
%%%cit_comment
%%%cit_title
\citTitle{«Они были обречены» 50 лет назад погиб экипаж «Союза-11». Кто
виноват в главной трагедии советской космонавтики?: Космос: Наука и техника:
Lenta.ru}, Сергей Варшавчик, lenta.ru, 30.06.2021
%%%endcit

