% vim: keymap=russian-jcukenwin
%%beginhead 
 
%%file 13_11_2019.fb.fb_group.story_kiev_ua.1.vremja
%%parent 13_11_2019
 
%%url https://www.facebook.com/groups/story.kiev.ua/posts/1188649904665142
 
%%author_id fb_group.story_kiev_ua,ugrjumova_viktoria.kiev.pisatel
%%date 
 
%%tags kiev,pamjat,pismo,vremja
%%title ВРЕМЯ, СТОЛКНУВШИСЬ С ПАМЯТЬЮ, УЗНАЕТ О СВОЕМ БЕСПРАВИИ
 
%%endhead 
 
\subsection{ВРЕМЯ, СТОЛКНУВШИСЬ С ПАМЯТЬЮ, УЗНАЕТ О СВОЕМ БЕСПРАВИИ}
\label{sec:13_11_2019.fb.fb_group.story_kiev_ua.1.vremja}
 
\Purl{https://www.facebook.com/groups/story.kiev.ua/posts/1188649904665142}
\ifcmt
 author_begin
   author_id fb_group.story_kiev_ua,ugrjumova_viktoria.kiev.pisatel
 author_end
\fi

В десятом часу вечера 25 апреля 2018 года - а кажется, что это было гораздо
раньше - я впервые сама отправила текст в неизвестную мне группу. До этого я
только и делала, что откуда-то выходила, обнаружив себя членом неведомой
команды. А тут увидела публикацию, удивительным образом отличавшуюся от прочих
постов о Киеве - не созданную удивительным способом \enquote{просто добавь воды} -
простите, копи-пастом) Не проиллюстрированную знакомой с детства все той же
открыткой все в том же ракурсе, а живой и яркий рассказ очевидца, впечатления
участника событий, хранителя семейной памяти. Эти носители заемной памяти,
голоса, говорящие за ушедших, особенно трогают мою душу. Пускай мы помним не
так много, как хотелось бы, и в наших семейных альбомах хранятся редкие кадры -
а это еще если очень повезло - у многих вообще ничего не хранится от их бабушек
и дедушек, потому что история - это то, что происходит с каждым из нас, а не
то, о чем потом пишут историки и сухо уведомляет Википедия. И в ходе этой
истории мы приобретаем прошлое и память и теряем материальные свидетельства
ушедшей эпохи. 

\ii{13_11_2019.fb.fb_group.story_kiev_ua.1.vremja.pic.1}

И вот рассказы тех, кто, как и я, по крупицам находит сведения о своих и чужих,
ставших за время этих поисков своими, светится от счастья, получив в свое
распоряжение пожелтевшую квитанцию, дряхлое письмо на линованной когда-то
бумаге - где фиолетовые строки стали от времени коричневыми, а буквы раслылись
от пролитых на них слез - слез, которым в этом году исполнилось, скажем, 110
лет; когда вдруг становится ясна подоплека шутки, выуженная из добытой
переписки, или - огромное, редчайшее везение - ты однажды получаешь едва
видную, маленькую, но все-таки фотографию того, кого знал всю свою жизнь, но не
видел никогда, - и на этой фотографии живут только глаза... Впрочем, это
главное, глаза... словом, эти рассказы оказались сокровищем. 

Они не просто интересны и познавательны, они не просто трогательны и прекрасны,
не просто веселы или печальны, - они еще рисуют картину мира. того самого мира,
в котором жили наши родители и наши пращуры. Они ходили по одним улицам и,
вполне вероятно, приветливо кивали друг другу при встрече. они с высокой
степенью вероятности встречались в опере и на симфонических, или стояли в
очереди в Кулинарке, или ели котлеты по-киевски в Столичном, или пили коньячок
на Кукушке. И уж точно - портвейн в Петушке. Они сидели в Мариинском и ходилив
Цирк Крутикова, смотрели Фантомаса в Шанцере... Мы связаны уникальным родством
- такой невидимой нитью, она называется - Город. Так принято писать, говоря о
Киеве. Я не любитель пафоса, но в данном случае немного пафоса, вроде, даже
уместно.

Дайте себе труд пролистать иллюстрации с подписями, отыщите среди этих
публикаций те, что относятся ко времени \enquote{Золотого века} КИ - и вы узнаете, как
выглядел Йозеф Швейк - в группе вывешивали историю его пребывания в Киеве, и
подробный и уникальный рассказ о Гашеке; почитайте историю кинотеатра Днипро -
до того, как он стал стерео; послушайте, что расскажет \enquote{Девочка с Евбаза}. Я
намеренно не ставлю ссылок и не называю имен: поиск сокровищ - занятие
увлекательное, его надо осуществлять самому. 

В тот вечер я вывесила в группе крохотную историю о своем любимце - дяде Мите -
моем двоюродном прадеде, и счастлива, что теперь его любят еще несколько
человек. Я верю в чудеса, и верю в то, что когда я стала рассказывать о своих
предках людям неравнодушным, влюбленным в свою семейную историю и потому
понимающим таких же безумцев, архивариусов и помнителей, что-то сдвинулось в
пространстве, во Вселенной.

Потому что, пока мы говорим о ком-то, мы тем самым говорим за них перед Богом.
Поа мы их помним, они живут - и это единственный абсолютно гарантироанный
способ бессмертия, не зависящий от того, во что каждый из нас верит.

В преддверии Нового Года очень хотелось бы прочитать на страницах КИ семейные
истории - рассказы о новогоднем волшебстве, о поисках и находках, семейные
анекдоты и рассказы о любви. То, что добавит красок в нашу картину мира.

Делитесь собственными историями, делитесь сбственными мыслями. КИ - уникальный
музей трогательных семейных сокровищ, архив эксклюзивных воспоминаний,
сообщество, в котором когда-то живо откликнулись на простую идею - ВРЕМЯ,
СТОЛКНУВШИСЬ С ПАМЯТЬЮ, УЗНАЕТ О СВОЕМ БЕСПРАВИИ

(этим слезам исполнилось на Пасху 104 года)

\#челенджкиевскиеистории

\ii{13_11_2019.fb.fb_group.story_kiev_ua.1.vremja.cmt}
