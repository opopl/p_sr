% vim: keymap=russian-jcukenwin
%%beginhead 
 
%%file 14_02_2019.news.ru.ukraina_ru.chemakin_anton.1.professor_sikorskii
%%parent 14_02_2019
 
%%url https://ukraina.ru/history/20190214/1022667672.html
 
%%author 
%%author_id chemakin_anton
%%author_url 
 
%%tags sikorskii_ivan
%%title «Мы должны решительно сказать: мы — русские, и Киев — наш»: к 100-летию со дня смерти профессора Сикорского
 
%%endhead 
 
\subsection{«Мы должны решительно сказать: мы — русские, и Киев — наш»: к 100-летию со дня смерти профессора Сикорского}
\label{sec:14_02_2019.news.ru.ukraina_ru.chemakin_anton.1.professor_sikorskii}
\Purl{https://ukraina.ru/history/20190214/1022667672.html}
\ifcmt
  author_begin
   author_id chemakin_anton
  author_end
\fi

\index[names.rus]{Сикорский, Иван Алексеевич!Отец Игоря Сикорского, 26.05.1842 - 14.02.1919}

\ifcmt
pic https://cdn1.img.ukraina.ru/images/102267/03/1022670303.jpg
\fi

\begin{leftbar}
  \begingroup
    \em\Large\bfseries\color{blue}
Авиаконструктор Игорь Сикорский является одним из самых известных уроженцев
Киева. Его имя носят Киевский политехнический институт, аэропорт «Жуляны»,
Авиакосмический лицей, улица. Его отец, психиатр Иван Сикорский, сегодня
уступающий в известности сыну, в начале XX в. был весьма заметной фигурой в
научной и общественно-политической жизни России
  \endgroup
\end{leftbar}

\subsubsection{«Казус Сикорского»}

Если обратиться к «Энциклопедии украинознавства», выпускавшейся украинскими
эмигрантами во второй половине XX в., то мы увидим, что статьи про Сикорских
написаны вполне корректно: старший обозначен просто как «психиатр и психолог»,
младший — как «авиаконструктор родом из Киева, сын Ивана Сикорского», причем
отмечено, что Игорь Сикорский проявлял активность в «русской общине» в США.
Мания записывать в украинцы всех подряд — «и мертвых, и живых, и нерожденных» —
явление последних десятилетий, но еще 50 лет назад украинским националистам,
издававшим «Энциклопедию украинознавства», это и в голову не приходило.

Сегодня же в украинских СМИ все чаще можно наблюдать попытки представить Игоря
Сикорского украинским ученым, украинским гением. В украинской «Википедии»
обозначена и его национальность — конечно же, он «украинец», кто же еще. Но при
этом национальность его отца Ивана вообще никак не обозначена, а сам он
охарактеризован как «украинофоб». Впрочем, и Ивана Сикорского иногда зачисляют
в украинцы, но значительно реже, чем его сына. Это и понятно: монархические и
русские националистические взгляды профессора Сикорского хорошо всем известны,
да и репутация у него крайне неоднозначная, так что какой-то «ценности», в
отличие от сына Игоря, для современных украинских деятелей он не представляет.

Вот и получается, согласно этим теориям, такое интересное семейство:
отец-«украинофоб» и сын-«украинец».


\subsubsection{Детство и юность Ивана Сикорского}

Иван Алексеевич Сикорский родился 26 мая 1842 г. в семье священника
Благовещенской церкви с. Антонова Сквирского уезда Киевской губернии Алексея
Ивановича Сикорского и его супруги Елены Матвеевны. Несмотря на фамилию,
никаких польских корней у Ивана не прослеживается. 

\emph{«Семья была большая — шесть сыновей и шесть дочерей, и Иван Алексеевич был
шестым,} — писал в 1930-е гг. про И.А. Сикорского его биограф (вероятно, сын
Игорь). — \emph{С самого раннего детства он отличался вдумчивостью, был серьезен,
мало принимал участия в играх братьев и сверстников. Очень рано, почти
самостоятельно, научился читать и читал по целым дням. Книг в то время, в
особенности в деревне, в бедной семье священника, было немного, и Иван
Алексеевич читал все книги, какие мог найти: молитвенники, календари, учебники.
Любимой его книгой была — "Басни Крылова", которыми он увлекался, читал вслух
братьям и сестрам, оттеняя мораль басни»}.

В детстве Иван учился сначала в Киево-Софиевском духовном училище, а затем в
Киевской духовной семинарии. Он сразу же выделился среди сверстников глубокими
познаниями и интересом к наукам, как естественным, так и гуманитарным. Поняв,
что духовное поприще — это не его призвание, он оставил семинарию и сдал
экстерном экзамены в Первой киевской гимназии. Сикорский поступил на
естественный факультет Киевского университета, но вскоре перевелся на
медицинский.

\subsubsection{Иван Сикорский — известный ученый-психиатр}

После окончания университета он был оставлен при нем для подготовки к получению
профессорского звания. В 1872 г. защитил докторскую диссертацию «О
лимфатических сосудах легких». С 1873 по 1885 г. жил в Санкт-Петербурге,
работая в клинике душевных болезней при Военной медико-хирургической академии,
а также состоя чиновником по особым поручениям при начальнике Главного
управления военно-учебных заведений. В 1885 г., после создания в Киевском
университете кафедры систематического и клинического учения о нервных и
душевных болезнях, Сикорский вернулся на малую родину и вскоре стал профессором
на этой кафедре.

Иван Сикорский является одним из основоположников детской психологии и
психопатологии. В своих научных трудах он изучал влияние алкоголя на здоровье и
психику, наблюдал за умственной работой учащихся в школах. Он одним из первых в
отечественной науке обратился к проблемам детей с отклонениями в развитии,
выпустил фундаментальный труд «О заикании» и разработал метод его преодоления.

\emph{«Из всего обширного учения о здоровой и больной душе, которому Иван Алексеевич
посвятил себя, его в частности глубоко интересовала детская душа, еще мало
изученная и непонимаемая в то время, и Иван Алексеевич с жаром и энергией
отдался этой животрепещущей отрасли»}, — писал его биограф. В 1904 г. Сикорский
основал в Киеве Врачебно-педагогический институт для умственно отсталых детей,
а в 1912 г. — первый в мире Институт детской психопатологии.

\ifcmt
pic https://cdn1.img.ukraina.ru/images/102267/05/1022670506.jpg
caption Семья Сикорских. Игорь Сикорский (справа) с братом Сергеем и сёстрами Ольгой и Еленой
\fi

Многие труды Сикорского были переведены на иностранные языки — немецкий,
французский, английский, венгерский, чешский, болгарский, сербский. Он был
избран почетным членом множества российских и зарубежных научных обществ.

Сикорский собрал огромную частную библиотеку — она сохранилась и доныне и
насчитывает почти 13 000 томов на русском, французском, английском и латыни.
Иван Алексеевич нередко отказывал себе в самом необходимом, лишь бы приобрести
для своей библиотеки какую-нибудь редкую и ценную книгу. Незадолго до смерти он
передал все свое книжное собрание в дар Киевскому университету.

Сикорский отличался поразительным трудолюбием, \emph{«чрезвычайная воздержанность
была отличительной чертой Ивана Алексеевича. Он не пил, не курил, не играл в
карты, не сидел на мягкой мебели, питался самыми простыми и незатейливыми
кушаньями. <…> Про него говорили, что у него 36 часов в сутках, так много он
успевал сделать»}.

Профессор Сикорский был популярным человеком в Киеве, дружил со многими
деятелями науки и культуры. Так, например, художник \textbf{Виктор Васнецов},
расписывавший киевский Владимирский собор, придал образу евангелиста Иоанна
Богослова черты лица Сикорского (речь идет именно об апостоле Иоанне Богослове,
а не о святителе Иоанне Златоусте, как ошибочно сообщается в большинстве
интернет-источников, в том числе и в русской «Википедии»).

\subsubsection{Иван Сикорский — русский националист}

После революции 1905—1907 гг. Иван Сикорский принял активное участие в
монархическом движении, сотрудничал с «Союзом русского народа», вступил в
Киевский клуб русских националистов, был избран почетным членом Патриотического
общества молодежи «Двуглавый орел».

9 ноября 1910 г. на собрании русских избирателей Старокиевского участка,
проводившемся в преддверии выборов в Киевскую городскую думу, Сикорский
произнес примечательную речь:

\begin{leftbar}
  \begingroup
        \em\Large\color{blue}\bfseries
«На Киев смотрит вся Россия. Петербург никогда не имел значения руководителя
        национальной жизни России. Москва после 1905 года также утратила
        нравственный авторитет в глазах национально-русского общества. Значение
        центра русской национальной жизни начинает переходить к Киеву, и на
        киевлянах лежит высокий долг перед городом и родиной: мы должны
        укрепить возникшую здесь русскую твердыню. Пора нам сказать: мы — сыны
        великого народа, и здесь, в историческом Киеве, хозяева — мы! Городское
        управление матери городов русских должно быть русским. Вот почему на
        выборах мы должны идти твердо и прямо, никому не кланяясь и ни у кого
        не заискивая. Мы должны решительно сказать: мы — русские, и Киев —
        наш».
  \endgroup
\end{leftbar}

Сикорский написал несколько работ о национализме, правда, стоит сказать, что
особой научной ценности, тем более сегодня, они не представляют (в отличие от
его работ по «профильной» специальности). Иван Алексеевич был сторонником
модных в то время биологизаторских теорий, и поэтому феномен нации он
рассматривал с расовых позиций. Не скрывал Сикорский и своего антисемитизма.

\ifcmt
pic https://cdn1.img.ukraina.ru/images/102267/03/1022670373.jpg
\fi

Именно с этой стороной его деятельности связан наиболее спорный эпизод
биографии ученого — участие в печально известном деле Бейлиса. Сикорскому было
поручено прокуратурой составить психиатрическую экспертизу обстоятельств смерти
мальчика Андрея Ющинского. Он пришел к выводу, что это ритуальное убийство,
совершенное евреями. Экспертиза Сикорского вызвала волну возмущения в научном
сообществе. Критика его экспертного заключения — в принципе, вполне
справедливая — вскоре переросла в настоящую травлю пожилого профессора со
стороны либеральной прессы.

Очевидно, что в случае с экспертизой по делу Бейлиса Сикорский поставил
политическую необходимость выше научной беспристрастности, что и привело к
печальным результатам, в первую очередь для него самого. Участие Сикорского в
деле Бейлиса серьезно подорвало его научную репутацию, ранее казавшуюся
незыблемой.

\subsubsection{Иван Сикорский и украинский вопрос}

По некоторым данным, в юности, во время учебы в Киевской духовной семинарии,
Сикорский одно время увлекался украинофильством и даже был читателем
украинского журнала «Основа».

Но это увлечение вскоре прошло, и вновь коснулся украинского вопроса Сикорский
только в 1913 г., когда он подготовил для заседания Клуба русских националистов
доклад «Русские и украинцы», посвященный критике «украинства» с точки зрения
антропологии, лингвистики и истории (из-за болезни он сам прийти на заседание
не смог, поэтому текст доклада был зачитан по его просьбе доктором Селецким).
Доклад пришелся по душе слушателям, и в том же году Клуб русских националистов
издал его в виде небольшой книжки.

Полемизируя с лидером украинского движения \textbf{М. С. Грушевским}, Сикорский приходил
к следующему выводу: 

\begin{leftbar}
  \begingroup
    \em\Large\color{blue}\bfseries
«От Архангельска до Таганрога и от Люблинского Холма до
Саратова и Тамани живет одна и та же (в главных чертах) русская народность.
Дробление на великоруссов, малороссов и белоруссов связано с несущественными и
второстепенными, притом скорее лингвистическими, чем антропологическими
особенностями, которые притом нередко и отсутствуют.

В малорусском (по Костомарову — южнорусском) населении — тот же племенной
состав, что и в великорусском, с незначительным только перевесом славянского
элемента над финским. Этим антропологически, т. е. по своей породе и природе
исчерпывается все русское население европейской России. 

Украинцев здесь нет! Их нет ни в живущих экземплярах, ни в кладбищном
населении: нет ни на земле, ни под землей. Поэтому, если за исходное основание
для суждений и выводов взять физический состав населения, его породу и природу,
то на Украине нет такого населения, которое обладает особой породой: здесь то
же, что существует и за пределами Украины. Отсюда — естественный вывод, что
"Украина" и "украинцы" — это термин скорее географический и политический, но не
антропологический или этнический».
  \endgroup
\end{leftbar}

Формирование русского литературного языка, по мнению Сикорского, произошло на
основе великорусского разговорного языка потому, что именно на нем писали
четыре гениальных человека (Ломоносов, Пушкин, Гоголь и Лермонтов), несколько
талантливых (Жуковский, Тургенев, Аксаковы) и целая плеяда второстепенных.

Хотя прежде, в XVI—XVII вв., обе русские письменности (великорусская и
южнорусская) обладали примерно равными шансами на первенство, в XVIII и XIX вв.
великороссы вырвались вперед, «и появление украинского (южнорусского) языка на
этническом поле России около столетия тому назад уже не могло изменить судеб
даже в тот момент, когда на горизонте засветилась яркая звезда \textbf{Тараса
Шевченко}».

Но в этом «поражении» южнорусской речи Сикорский не видел ничего трагического,
так как северорусский и южнорусский говоры чрезвычайно близки:

\begin{leftbar}
  \begingroup
    \em\Large\color{blue}\bfseries
«Сравнивая язык русский и украинский, легко усмотреть почти полное тождество психологии этих двух языков и лежащую в основе их совершенную близость душевных и умственных процессов, воззрений и приемов мысли. Это показывает с очевидностью, что русский и украинский языки — это не два языка, а один язык; в крайнем случае можно говорить о двух наречиях одного праязыка, но это было бы почти логической тавтологией. Различие между русским и украинским языками — не психологическое, а фонетическое или звуковое, следовательно, различие не внутреннее — глубокое, а внешнее — кажущееся: звуками они разнятся, но их психология тождественна».
  \endgroup
\end{leftbar}

Исторические аргументы, по мнению Сикорского, также говорят не в пользу
Грушевского. Цитируя «Слово о полку Игореве», он напоминает, что его герои
называли себя «русскими», а не «украинцами», и пали «за русскую землю», а не за
«Украину», как это пытается представить Грушевский.

Сикорский делал вывод, что «этнографический термин "украинцы", за отсутствием
самого объекта, т.е. этнографически особого народа, не имеет основания
существовать, а обозначение территории именем "Украины" потеряло свою
первоначальную административную надобность, а потому самый термин
представляется бесполезным, подобно наименованию "Священной Римской империи"
или "Московского государства".

Соответственно, не имеет смысла не только термин «украинцы», но и другие
дробления — «великоросс», «белорус», «малоросс», «южнорусс» — также лишены
антропологического или этнического основания и представляют собой
территориальные обозначения традиционного, но уже устаревшего типа. «Простое
имя "русский" достаточно», — заключал Сикорский.

\subsubsection{Последние годы}

В сентябре 1914 г., практически сразу после начала Первой мировой войны, погиб
при взрыве крейсера «Паллада» сын Сикорского Сергей, военный моряк. Возможно,
эта тяжелая утрата сказалась на самочувствии профессора, и так имевшего
серьезные проблемы со здоровьем уже с начала 1910-х гг. В конце 1915 — начале
1916 г. Сикорский окончательно слег.

\emph{«Тяжкую болезнь, поразившую его за три года до смерти и приковавшую его к
постели, переносил с величайшим терпением, ясно понимая всю картину болезни,
течение ее и близкий конец, — писал его биограф. — Он говорил, что для него
было большим несчастьем заболеть той самой болезнью, от которой он лечил
других. Несмотря на тяжкий физический недуг, дух его до самой смерти продолжал
оставаться бодрым, мысль продолжала свою работу. Иван Алексеевич жил высшими
духовными интересами, слушал чтение, диктовал, стремясь сохранить для других
результаты своих глубоких познаний и глубокого опыта: заканчивал редактирование
"Психологической Хрестоматии". Спокойно и кротко готовился к исполнению, как он
говорил, "последнего долга человека на земле" — к смерти»}. Не до конца ясно,
что за болезнь поразила Сикорского, но, возможно, речь идет о тяжелом
невралгическом заболевании. 

Скончался Иван Алексеевич Сикорский на 77-м году жизни в 5 часов утра 14
февраля 1919 г., в разгар гражданской войны. Он пережил и свержение монархии, и
первый захват Киева большевиками, и власть гетмана и немецких оккупантов, и
приход петлюровских войск.

5 февраля 1919 г. петлюровцы бежали, и Киев был вторично занят большевиками.
Вероятно, Сикорский успел узнать об этом, так как «сознание его помрачилось не
более как за 12 часов до кончины». 16 февраля 1919 г., после отпевания в
университетской церкви, Сикорский был похоронен на главной аллее Байкова
кладбища, на его безымянной могиле поставили скромный железный крест (табличка
с фамилией появилась значительно позже).

Видимо, у родственников не нашлось ни времени, ни средств на более достойную
организацию похорон. Все «буржуазные» газеты были закрыты большевиками, и
некролог, оповещающий киевлян о смерти профессора Сикорского, был помещен в
газете «Борьба», органе Центрального комитета Украинской партии левых
социалистов-революционеров. То ли украинские левые эсеры не проявили классовой
бдительности, напечатав на страницах своего официального органа некролог
«реакционера» и «украинофоба», то ли подошли к этому вопросу с практической
точки зрения и просто решили не отказываться от платы, полагаемой за размещение
объявления.

Можно сказать, что Сикорскому повезло — он умер в своей постели. Несколько
месяцев спустя многие члены Киевского клуба русских националистов, в том числе
друзья и коллеги Сикорского, были расстреляны киевскими чекистами. Учитывая его
роль в деле Бейлиса, Сикорский вряд ли бы избежал их участи.

\ifcmt
pic https://cdn1.img.ukraina.ru/images/102267/10/1022671059.jpg
\fi

\subsubsection{«Яблоко от яблони…»}

Но, может быть, монархические и «украинофобские» взгляды Ивана Сикорского никак
не повлияли на его младшего сына, и тот вырос «щирым украинцем»?

Ничего подобного — Игорь Иванович Сикорский унаследовал от отца не только
потрясающую работоспособность, но и его политические воззрения (как бы к ним ни
относиться).

За год до смерти отца Игорь покинул Россию и обосновался в США, где и добился
выдающихся научных результатов и получил всемирную славу. Несмотря на это, он
всегда считал себя русским и поддерживал тесные контакты с русскими
эмигрантскими организациями, причем тяготел именно к структурам правого и
монархического толка.

\ifcmt
pic https://cdn1.img.ukraina.ru/images/102267/11/1022671106.jpg
\fi

Так, например, в 1923 г. он писал для «национально-монархического легитимного
органа» «Державная Русь», выходившего в Нью-Йорке. В статье «К убийству
Государя и Его Семьи» Игорь Сикорский, отмечая величие духа Николая II и его
кристальную верность России, делал вывод: \emph{«Пусть мученическая кровь Государя и
Его Семьи заставит русских людей одуматься, сбросить со своих глаз ловко
наброшенную пелену обмана. Пусть поможет нам всем познать истину, объединиться
и дать отпор проклятым силам зла и лжи, доныне угнетающим Россию».}

В 1930-е гг. он писал статьи (в том числе воспоминания о Киеве начала 1910-х
гг.) для другой монархической газеты — «Воскресение России», выходившей под
девизами «За Веру, Царя и Отечество!» и «Мы — русские! С нами Бог!» В 1950-е
гг. Сикорский состоял в Российском политическом комитете в Нью-Йорке, в
программе которого отмечалось, что «государственное единство России, в ее
естественных, исторических границах, составляет одну из главных основ
российского национального сознания».

Деятельность Игоря Сикорского не ограничивалась только публицистикой — в 1927
г. он предложил эмигрантской монархической организации «Союз Русских
Государевых Людей» план высадки вооруженного десанта на территории СССР.
Согласно плану, предполагалось перебросить из Болгарии на Дон на 25 воздушных
кораблях системы Сикорского 3000 монархистов, которые должны были поднять
казачество на восстание. Итогом должно было стать свержение большевиков и
восстановление Российской империи.

На реализацию проекта требовалось от 3 до 5 миллионов долларов, но найти их так
и не удалось, и от идеи десанта пришлось отказаться. Таким образом, в плане
мировоззренческом Игорь Сикорский оказался в полной мере наследником своего
отца, профессора Ивана Сикорского.

\index[names.rus]{Сикорский, Игорь Иванович!Авиаконструктор}
