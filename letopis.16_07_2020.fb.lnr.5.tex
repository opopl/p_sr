% vim: keymap=russian-jcukenwin
%%beginhead 
 
%%file 16_07_2020.fb.lnr.5
%%parent 16_07_2020
 
%%endhead 
  
\subsection{Ты знаешь, как пахнет война?}
\vspace{0.5cm}
{\small\LaTeX~section: \verb|16_07_2020.fb.lnr.5| project: \verb|letopis| rootid: \verb|p_saintrussia|}
\vspace{0.5cm}
  
\url{https://www.facebook.com/groups/lnrgumo/permalink/2850162068428682/}

В последние недели и дни, все чаще слышны призывы к войне, все чаще пишут о
том, что война близко и к ней надо готовиться. Все так, война очень близко и
готовиться надо, но надо понимать - война безвозвратно все меняет. Я, как и
любой человек, побывавший в боевых действиях, не люблю войну, не вижу в ней
необходимости, кроме одного варианта: ЗАЩИТА.

Если ты защищаешь слабых, если на тебя или твоих близких нападают, то
действовать просто необходимо. Причем не обращать внимание на то, что это
жестоко, что можно и мягче. Чем мягче ты будешь действовать, тем большие
зверства жди от агрессора.

Тем не менее, война – это всегда упущенные возможности и вот именно о них я
сожалею. Я не испытываю жалости к врагу, мне жаль, что сама ситуация так
сложилась, что вместо того, чтобы встречаться с девчонками, заводить детей,
строить и развиваться, тот парень решил, что ему надо меня или кого-то другого
убить. В этот момент его судьба пошла по другому пути. Мне жаль, что так
случилось, но я сделаю все от меня зависящее, чтобы он уже не вернулся домой.
Это жестоко? Возможно. Но только так заканчиваются войны – победой!

Дальше я хочу, чтобы вы прочитали мою заметку, которую написал в середине
апреля 2015-го года. Донесите по возможности эти слова до матерей украинских
солдат, до пацанов, которые считают, что мы враги и идут в ВСУ, чтобы сражаться
с нами. Не мы враги, а те, кто посылает вас на верную гибель. Даже если вы
сумеете вырваться живыми с Донбасса, вы навсегда погибнете как человек. Вы
убьете свой шанс на нормальную жизнь, на семью, на возможность быть
счастливыми. Над вами всегда будет висеть вопрос: «ЗАЧЕМ?»…

Запах войны

Тем, кто, сидя за монитором компьютера, радуется подбитому танку с «укропами»,
те, кто насмехается над сожжёнными заживо «колорадами», те, кто не успел
окончить школу или просто не понял ничего в жизни, хотя, возможно, успел стать
не только отцом, но и дедом, вряд ли это поймут. То, что в войне всегда обе
стороны проигравшие. Побеждают политики и бизнесмены-твари, которые
зарабатывают на этом политические очки и состояния. Солдаты и мирное население
всегда в проигрыше.

Я сталкивался с оголтелой, полной ненависти толпой мудаков, которые говорили
слова, за которые в реале бил бы очень долго, но я понимаю, что в этом виноваты
не только они и их родители, но и пресса, и государство. И еще я понимаю, что
все эти майданы, войны и санкции не нужны никому кроме тех, кто пытается
растащить некогда самую могущественную страну по кускам, чтоб потом растерзать
её своей ненасытной пастью.

Читай и думай, если есть чем. Что-то нарыл в интернете, что-то вспомнил свое...

Ты знаешь, как пахнет война?

Нет?

Я расскажу тебе.

Война - это кислый запах кордита, щекочущий ноздри, запах полыхающих соляры и
резины, запах плавящегося от нестерпимого жара металла.

Это приторный, сладкий запах горящего жира. Человеческого жира.

Это запах неделю не стиранных портянок, вонь дерьма и мочи из отхожего места, в
которое сперва бегали, стыдливо прикрываясь фанеркой или куском маскировочной
сетки, а потом на это стало просто наплевать.

Война - это запах разлагающегося на солнцепёке трупа с тёмно-зелёной гнилостной
венозной сетью и выклеванными птицами глазами, валяющегося у подошвы
простреливаемой высотки, трупа, до которому никому нет дела.

Война - это запах дыма дрянных сигарет без фильтра, замешанный на перегаре от
самодельного пойла, которым поминали ребят, не вернувшихся со вчерашней
захлебнувшейся атаки.

Её запах - это запах печёных яблок, висящих на обугленной яблоне возле руин
дома, в котором уже никто и никогда не будет жить. Она пахнет сгоревшими на
корню налитыми колосьями хлеба.

Запах Войны - это спиртово-гангренозная вонь санбатов, окровавленных бинтов,
втоптанных в грязь окопа пополам с гильзами, бинтов, которые не пригодились
тому смешному рыжему пареньку из позавчерашнего пополнения - он просто умер от
зазубренного осколка в живот.

Запах Войны - это запах дешевого одеколона, которым освежился твой сосед по
нарам, побрившись перед своим последним боем.

Её запах - запах страха и лютой ненависти, на ментальном уровне.

И над всем этим - ароматы дорогой туалетной воды и сигар в кабинете политика,
перебивающие запах типографской краски свежеотпечатанных денежных купюр.

Тебе нравится запах Войны, правда?

Тогда иди ТУДА.

А еще это запах крови и адреналина , и ненависть, ненависть, ненависть ко всему
что что там за леском...

Землей еще пахнет. Свежевскопанной.

А в иных местах запах мокрой глины повсюду.

Характерного такого говнолина ох@енной липкости.

Когда на войне и ты чувствуешь запах, то ты живой, не ранен и даже не контужен.

А самый тяжелый запах - это запах безысходности (холодного пота) собственного
тела, после того как ты выжил, а от тебя ничего не зависело и ничего не мог
сделать сам.

Кисло-приторный запах тротила от взорвавшихся снарядов.

А еще любимый запах - это запах дымящейся сигареты и запах только что
разогретой на костре каши из сухого пайка, которая в этих местах кажется
вкуснее всяких деликатесов

Холодная каменная пыль, кислятина пота, табачный перегар, ружейка, адреналин с
кровью, запах горячего песка с ветром по лицу... Вот нахрена я это с утра
вспоминал!? И так в голове в уголке сидит и не хочет забываться...

Друг-медик говорит: Кровью, гноем пахнет. Кровь свежая, кровь засохшая, кровь
стоялая, вся пахнет по-разному. Грязные бинты. Чистые бинты. Бинты из
автоклава. Ссаные простыни, порезанные на бинты из сухожара или автоклава.
Кетгут пахнет кислым. Шёлк пахнет соломой.

Резиной пахнет. Трубки-перчатки. Первомур. Спирт. Фенол. Хлорка. И гной. Кал,
слизь, моча. Горелая одежда. Пороховая гарь. Но её мало. Горелая резина. Пот.
Пот лихорадящих. Пот тех кому страшно. Пот от того, что ты уже не успеешь, а
боец ещё в сознании...

Друг-луганчанин говорит: Война!? У нас за окном второй год война. Это когда
земля трясется от близкого взрыва, и жена бежит прятаться в ванную, а ты устал
и просто лениво констатируешь: недолет. Это когда окна вылетают в квартиру,
которую только убрали. Это когда у тебя над головой самоподрывается ракета с
иглы потому что до самолета не достала, а ты вместо того чтобы остановиться с
остервенением жмешь на газ при скорости 120 км по пустынному городу. Запах
сгоревшей взрывчатки и сгоревшего дома соседа. Это когда сына хоронишь в 27, а
сам продолжаешь жить. Вот что для меня эта проклятая война.

Это целая жизнь, пролетевшая за доли секунды в момент, когда ты успел накрыть
собой жену и проговорил «мимо»

Хочешь почувствовать ЧТО ТАКОЕ ВОЙНА?

Война, ты чувствуешь? Пульс стал реже, кончики пальцев онемели и задрожали,
взор обострился. Ты чувствуешь прилив сил? Теперь закрой глаза и глубоко вдохни
через нос. Воздух, обжигающий и приятный уже только из-за того, что он
существует, наполнен запахами. Миллионами запахов. Выделяется один. Или тебе
просто кажется, что выделяется, на самом деле ты просто думаешь, что можешь его
почувствовать.

Наслаждайся. Ты знаешь, чей он. Теперь, как натасканный зверь ты будешь
охотиться за ним. Это не любовь, не первый секс, не присяга на трон, не
вождение дьявольски быстрой машины, не желание убить. Нет!

Это война. Никакой романтики, просто война. Просто пули, свистящие над головой,
просто снаряды, взрывающиеся без всяких на то имеющихся у них причин. Просто
пятидесятитонные«лёгкие» танки.

Просто стены огня вместо деревень, сёл. Просто кровь на стенах каменных и
кирпичных домов, чья-то кровь. Просто трупы людей около каждой воронки, на пути
каждой десятой пули. Просто окровавленные калеки, зовущие на помощь, жаждущие
скорейшего успокоения или смерти равнозначно. Просто чьи-то судьбы, как
колосья, срезанные чьим-то серпом войны. А ты - следующий, кого ждёт она.

Запах войны... Это запах тех самых свежесрезанных колосьев. Бывало, в детстве,
лежишь в поле на только что скошенном сене, аромат опьяняет, усыпляет,
успокаивает. Счастье. Всё наоборот, точнее, тогда всё было нормально, это
сейчас всё наоборот. Всё не так, неправильно. Война вам не наркотик, она не
успокаивает, а встряхивает тебя. И если усыпляет, то навсегда. И колосья
срезает не красивая деревенская девушка с песней на устах, а психопат с серпом
в руках.

Тебе уже ничего не хочется. Ни к маме домой, ведь ты не ребёнок уже давно. Ни
богатства, за эти несколько лет ты усвоил, что деньги - самое страшное оружие и
будет пострашнее водородных бомб, землю застилающих ковров из отстрелянных
патронов калибра 7.62 мм.

Страшнее пучка чьих-то окровавленных волос на самом краю гусениц танка. Тебе не
хочется курить, не хочется выпить водки и забыться, не хочется гарема красивых
девушек, не хочется даже окончания самой войны. Ты не хочешь ничего. Ни-че-го!

Зачем ты убил десятки людей, зачем остановил их горячие сердца, жаждущие крови
и разрушений? Зачем ты ночами напролёт сидел в своём штабе, готовил задания,
зачем посылал на эти задания людей с такими же горячими сердцами, как у тех?
Зачем всё это? Каков был смысл во всём этом?

Это смысл твоей жизни. А каков он, ты узнаешь через пару секунд. Выдыхай
медленно. Очень медленно. Открой глаза. Чуть склонив вперёд голову, чуть
прищурив глаза и улыбнувшись надвигающимся переменам, можешь идти в бой. Тебе
скоро предстоит упасть в мягкие объятия земли и перестать называться
человеком...

Чувствуешь? Ты все еще хочешь на войну?

Читать полностью: 
