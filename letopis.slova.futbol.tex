% vim: keymap=russian-jcukenwin
%%beginhead 
 
%%file slova.futbol
%%parent slova
 
%%url 
 
%%author 
%%author_id 
%%author_url 
 
%%tags 
%%title 
 
%%endhead 
\chapter{Футбол}
\label{sec:slova.futbol}

%%%cit
%%%cit_pic
%%%cit_text
Украинские патриоты просто не могут признавать ошибки.  Долпоепы тулят на
\emph{футбольную форму} лозунг ОУН – весьма неоднозначной организации, не имеющей
никакого отношения к \emph{футболу}. УЕФА вполне логично указывает долпоепам, что они
долпоепы и предписывает лозунг снять. Рассказы из серии «это не свастика, а
древний солярный символ» Европе побоку. И любой нормальный человек в данной
ситуации, даже будучи патриотом, признает правоту УЕФА. Делу конец.  Но это же
Украина – тут все немного по-другому. Всех, кто вчера не брызгал слюной и не
требовал занести УЕФА на Миротворец, уже объявили пятой колонной и агентами
Москвы. Т.е. по итогу все, кто против долпоепов, в очередной раз враги. Смешно
и грустно
%%%cit_title
\citTitle{Всех, кто не требует занести УЕФА на Миротворец, уже причислили к пятой колонне}, 
Вячеслав Чечило, strana.ua, 12.06.2021
%%%endcit

%%%cit
%%%cit_pic
%%%cit_text
Спортивные результаты нашей команды их интересуют слабо. По поводу спортивной
формы. Если бы проводился чемпионат мира по национализму, Украина заняла бы
там первое место, переиграв остальных всухую. А вот на футбольном поле
результаты будут гораздо скромнее.  Впрочем, это мало заботит патриотов - им
важно, чтобы страну представляла сборная ОУН
%%%cit_title
\citTitle{Украинским патриотам важно, чтобы страну представляла сборная ОУН}, 
Андрей Манчук, strana.ua, 12.06.2021
%%%endcit

%%%cit
%%%cit_pic
%%%cit_text
Поражение России от Бельгии в игре Евро-2020 (0:3) вызвало бурную реакцию со
стороны украинских любителей \emph{футбола} и политиков в соцсетях. Еще по ходу
матча стали появляться саркастические комментарии относительно игры хозяев на
стадионе в Санкт-Петербурге. Болельщики насхемались над российским защитником
Андреем Семеновым, который «отдал» голевой пас на Ромелу Лукаку. Последнего же,
напротив, назвали козаком и одели в форму Украины.  Предлагаем вашему вниманию
подборку украинских комментариев и мемов
%%%cit_comment
%%%cit_title
\citTitle{«Лукаку — казак!» Что пишут на Украине о разгромном поражении России от Бельгии},
sport-express.ru, 13.06.2021
%%%endcit

%%%cit
%%%cit_head
%%%cit_pic
\ifcmt
  pic https://strana.ua/img/forall/u/0/36/2021-07-03_14h33_07.png
	width 0.5
\fi
%%%cit_text
Другой парень нормально относится к тому, что \emph{футболисты} говорят на русском
языке. А ситуация, которая происходит в стране, не должна влиять на язык. 
\enquote{То есть им проще отвечать на русском. Это их право. Они же выступают под нашим
флагом, за нашу страну. Тут нет ничего такого}, - делает выводы юноша
%%%cit_comment
%%%cit_title
\citTitle{Что говорят украинцы о пресс-конференциях футболистов на русском языке. Опрос Страны}, 
Антонина Белоглазова, strana.ua, 03.07.2021
%%%endcit
