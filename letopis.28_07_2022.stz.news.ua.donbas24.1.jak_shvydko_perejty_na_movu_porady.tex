% vim: keymap=russian-jcukenwin
%%beginhead 
 
%%file 28_07_2022.stz.news.ua.donbas24.1.jak_shvydko_perejty_na_movu_porady
%%parent 28_07_2022
 
%%url https://donbas24.news/news/yak-svidko-pereiti-na-ukrayinsku-movu-korisni-poradi
 
%%author_id demidko_olga.mariupol,news.ua.donbas24
%%date 
 
%%tags 
%%title Як швидко перейти на українську мову: корисні поради
 
%%endhead 
 
\subsection{Як швидко перейти на українську мову: корисні поради}
\label{sec:28_07_2022.stz.news.ua.donbas24.1.jak_shvydko_perejty_na_movu_porady}
 
\Purl{https://donbas24.news/news/yak-svidko-pereiti-na-ukrayinsku-movu-korisni-poradi}
\ifcmt
 author_begin
   author_id demidko_olga.mariupol,news.ua.donbas24
 author_end
\fi

\ii{28_07_2022.stz.news.ua.donbas24.1.jak_shvydko_perejty_na_movu_porady.pic.front}
\begin{center}
  \em\color{blue}\bfseries\Large
За час війни все більше українців вирішують розмовляти суто державною мовою
\end{center}

Наслідком повномасштабного вторгнення росії в Україну стали не лише людські
втрати та нищівні руйнування, з кожним днем все більше мешканців України
вирішують розмовляти тільки україн\hyp{}ською. Ця тенденція наразі прослідковується
на всій території України, а особливо на Сході: Донеччині, Луганщині,
Харківщині. Питання мови стає одним з найгостріших, адже розмовляти мовою
окупанта відмовляються як молодь, так і люди похилого віку.

Переможець премії Global Teacher Prize Ukraine 2021, вчитель української мови
Артур Пройдаков надав декілька корисних порад, які допоможуть швидше заговорити
державною мовою. Артур уродженець російськомовної Луганщини. Через війну йому
довелося двічі переїжджати: у 2014 році — з Луганської області на Сумщину, у
2022 — на Прикарпаття. У Коломиї чоловік організував уроки української мови для
переселенців.

\ii{28_07_2022.stz.news.ua.donbas24.1.jak_shvydko_perejty_na_movu_porady.pic.1}

\begin{leftbar}
\emph{\enquote{У кожного своя історія переходу на українську. Комусь достатньо кілька
місяців, комусь — кілька років. Як на мене, процес поліпшення
української — це тривала історія на все життя. Я щодня намагаюся
покращувати свій рівень знання мови. Люди, які оточують мене, фільми,
які я переглядаю, книжки, які я читаю, — усе це наповнює мене колоритом
мови та робить сильнішим}}, — наголосив Пройдаков.
\end{leftbar}

1. Українська мова на гаджетах

Телефони, планшети, комп'ютери — це наші постійні супутники. А щоденна
взаємодія з українським інтерфейсом — це вже велика допомога у вивченні мови.
Встановити українську можна в налаштуваннях буквально за дві хвилини, а ефект
буде разючим.  

2. Перехід на державну в соцмережах

Це дуже ефективний та дієвий інструмент. Почніть писати дописи та коментарі
українською, листуйтеся нею в месенджерах. А якщо у вас стоїть україномовний
інтерфейс, то Facebook вам часто пропонуватиме навіть написані російською
дописи автоматично перекласти українською. З часом ви й не помітите, як звично
тепер користуватися солов'їною в інтернеті.  

3. Українськомовні блогери або Telegram-канали

Дуже важливо оточити себе українським інформаційним простором, тому час
підписатися на вітчизняних блогерів.

4. Українські подкасти та фільми і серіали українською

Віддавайте перевагу всьому українському! Саме так вдасться швидше звикнути до
розмовної мови, запам'ятовувати наголоси та коректну вимову слів.

5. Онлайн-курси з української мови, пізнавальні відео та сайти

Наразі в інтернеті можна знайти купу важливих і корисних ресурсів, які
допоможуть покращити вашу вимову та почати спілкуватися українською в
повсякденному житті. Одними з найбільш популярних залишаються освітні проєкти
Мова ДНК Нації та безкоштовні Експрес-уроки української від Олександра
Авраменка.

6. Власний словничок

Слів щодня ставатиме більше, тому словничок має бути завжди під рукою. Він може
бути як в нотатках у телефоні, так і в окремому блокноті, куди можна виписувати
нові слова або цілі вирази. Намагайтеся поповнювати словниковий запас. А потім
обов'язково інтегруйте нові для себе слова в повсякденний ужиток, щоби вони
стали звичними.

7. Читання українською вголос

Це дуже допоможе практикувати ваше мовлення та значно покращить вимову.

8. Перехід на державну в громадських місцях

Не думайте і не хвилюйтесь про можливі лексичні помилки під час спілкування.
Помилятися — це нормально. Гірше — так і не спробувати. Тому візьміть собі за
правило брати із собою українську, коли виходите з дому.

9. Спілкування українською з дітьми

Вдома теж потрібно говорити українською. Вона з вами усюди. Почніть говорити з
дитиною державною. Вчіть мову разом. Для дитини важлива ваша співучасть та
партнерство в процесі навчання. Доречно разом переглядати відео, читати разом
однакові книжки, а потім їх обговорювати. Спробуйте перетворити цей спільний
час на певний challenge чи сімейну розвагу. І тоді діти зможуть щиро долучитися
до вас та ваших ініціатив.

\begin{leftbar}
\emph{\enquote{Хист до мови присутній у кожного з нас. Просто ми не завжди готові відкривати
його в собі. Ми ж не говоримо про якісь академічні досягнення в галузі
мовознавства. Нам ідеться про комунікацію та розуміння. Тому тут варто
докладати зусиль і якомога більше практикуватися попри все. Процес навчання має
перетворитися на простір експериментів та комфорту. Для мене актуальним є те,
щоб українську сприймали крізь мову соцмереж, музики, сучасного мистецтва. Тоді
учні усвідомлюють, що мова — це не про підручники чи правила, це про те, що
навколо, про наше сьогоднішнє життя та про майбутнє}}, — підкреслив Артур
Продайков.
\end{leftbar}

\ii{28_07_2022.stz.news.ua.donbas24.1.jak_shvydko_perejty_na_movu_porady.pic.2}

Нагадаємо, раніше Донбас24 розповідав, як приготувати булгур на плиті.

ФОТО: з відкритих джерел

\ii{insert.author.demidko_olga}
%\ii{28_07_2022.stz.news.ua.donbas24.1.jak_shvydko_perejty_na_movu_porady.txt}
