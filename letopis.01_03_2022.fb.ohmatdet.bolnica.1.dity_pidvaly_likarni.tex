%%beginhead 
 
%%file 01_03_2022.fb.ohmatdet.bolnica.1.dity_pidvaly_likarni
%%parent 01_03_2022
 
%%url https://www.facebook.com/ndslohmatdyt/posts/pfbid022ExBxSknzQXxYb3sGE8r8myPCWoQgnfdDwsoQzmW8Z6LVedZvuEnAuNbH8byMpkgl
 
%%author_id ohmatdet.bolnica
%%date 01_03_2022
 
%%tags 
%%title Через війну важкохворі діти Охматдиту змушені ховатися від російських атак у підвалах лікарні
 
%%endhead 

\subsection{Через війну важкохворі діти Охматдиту змушені ховатися від російських атак у підвалах лікарні}
\label{sec:01_03_2022.fb.ohmatdet.bolnica.1.dity_pidvaly_likarni}

\Purl{https://www.facebook.com/ndslohmatdyt/posts/pfbid022ExBxSknzQXxYb3sGE8r8myPCWoQgnfdDwsoQzmW8Z6LVedZvuEnAuNbH8byMpkgl}
\ifcmt
 author_begin
   author_id ohmatdet.bolnica
 author_end
\fi

Через війну важкохворі діти Охматдиту змушені ховатися від російських атак у
підвалах лікарні💔

🔻Замість лікарняних боксів — підвальні приміщення. Тут вже шість день поспіль
перебувають пацієнти, ховаючись від ворожих обстрілів. Серед них діти з
орфанними захворюваннями, пацієнти після нейрохірургічних операцій, діти з
онкологією, новонароджені та багато інших важких пацієнтів.

🔻Охматдит — найбільша дитяча лікарня України. Це ціле містечко з різних
корпусів, де свою роботу провадять фахівці різних напрямків. І якщо медичну
допомогу на \enquote{довоєнному рівні} в такій великій системі забезпечити ми можемо,
то умови в укриттях занадто сильно залежать від стану підвальних приміщень. У
кожного корпусу вони свої. І ніхто з нас не здогадувався, що тут ми
розгортатимемо імпровізовані операційні та палати...

🔻Нині лікарня стала не лише місцем, де лікують, а й укриттям та тимчасовим
домом для мам з дітьми, які застрягли тут під час та після лікування. Тут є
батьки та діти, яких поклали в лікарню за день до війни. Родини, які втратили
звʼязок з рідними. Сьогодні всі вони змушені ночувати на підлозі та матрацах,
під звуки вибухів та сирен тривоги. Дітки не до кінця розуміють, що
відбувається, адже вони оточені неймовірною кількістю любові та турботи, а от
від мам ми чуємо лиш одне питання: \enquote{Коли це все закінчиться?}, а в очах читаємо
—  \enquote{змучена}.

🔻Завдяки небайдужим громадянам та волонтерам до нас привозять їжу, воду,
необхідні речі. Наші медики та персонал цілодобово перебувають в Охматдиті,
піклуються про кожного пацієнта та мешканця лікарні.

Ми дякуємо ЗСУ та підрозділам територіальної оборони, які бережуть наш крихкий
спокій, захищають нас від ворожих нападів та дозволяють нам рятувати життя🇺🇦

Колектив Охматдиту звертається до усього світу: агресора потрібно зупинити!
Наші діти не мають спати у підвалах, а лікарі — проводити складні операції в
бомбосховищах. Ми, звичайно, вистоїмо. Тому що у нас дивовижні всі: армія,
громадяни, лікарі, батьки та діти. Та цей жах треба зупинити якнайшвидше❗️

English below⬇️

The seriously sick children of Okhmatdyt have to hide from Russian attacks in
the hospital basements because of the war💔

🔻There are basement spaces instead of the hospital boxes. The patients have
been here for 6 days in a row, hiding from the enemy fire. Among the children
there are those with orphan diseases, patients after neurosurgical operations,
children with oncology, newborn babies and other patients.

🔻 Okhmatdyt is the biggest children's hospital in Ukraine. It's a whole town
consisting of various buildings, where specialists of different directions do
their job. While we are able to provide medical care at a "pre-war level" in
such a large system, the shelter conditions greatly depend on the state of
basements. None of us had been assuming that we would be transforming them into
operating theatres and wards...

🔻 Currently the hospital has become not only a place where people are treated,
but also a shelter and a temporary home for mothers with their children who
have been stuck here during and after treatment. There are parents and children
who were admitted to the hospital one day before the war started. Families that
lost contact with their relatives. Today they have no choice but to spend
nights on the floor and mattresses to the sounds of air raid sirens. The
children don't fully realise what is happening as they're surrounded by an
incredible amount of love and care, but the only question we hear from the
mothers is \enquote{When will it be over?}, and what we can read in their eyes is \enquote{I'm
exhausted}.

🔻Thanks to caring citizens and volunteers we have food, water and necessary
supplies delivered. Our doctors and staff stay at Okhmatdyt day and night and
take care about every patient at the hospital.

We thank the Armed forces of Ukraine and units of teretorial defence who keep
our fragile peace, protect us from enemy attacks and let us save lives 🇺🇦

The staff of Okhmadyt address the whole world: the aggressor needs to be
stopped! Our children mustn't sleep in basements, doctors mustn't perform
surgeries in bomb shelters. We will withstand because our nation is amazing:
citizens, doctors, parents, children, everyone. But this horror has to stop
now.
