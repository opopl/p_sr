% vim: keymap=russian-jcukenwin
%%beginhead 
 
%%file 25_06_2019.stz.news.ua.mrpl_city.1.tetjana_minjajlo_zminy_treba_pochynaty_z_sebe
%%parent 25_06_2019
 
%%url https://mrpl.city/blogs/view/tetyana-minyajlo-zmini-treba-pochinati-z-sebe
 
%%author_id demidko_olga.mariupol,news.ua.mrpl_city
%%date 
 
%%tags 
%%title Тетяна Міняйло: "Зміни треба починати з себе!"
 
%%endhead 
 
\subsection{Тетяна Міняйло: \enquote{Зміни треба починати з себе!}}
\label{sec:25_06_2019.stz.news.ua.mrpl_city.1.tetjana_minjajlo_zminy_treba_pochynaty_z_sebe}
 
\Purl{https://mrpl.city/blogs/view/tetyana-minyajlo-zmini-treba-pochinati-z-sebe}
\ifcmt
 author_begin
   author_id demidko_olga.mariupol,news.ua.mrpl_city
 author_end
\fi

\ii{25_06_2019.stz.news.ua.mrpl_city.1.tetjana_minjajlo_zminy_treba_pochynaty_z_sebe.pic.1}

Вона смілива, креативна, обдарована, завзята і дуже наполеглива на шляху до
мети... На долю нашої героїні випало чимало випробувань, але її внутрішня сила
і розуміння, що треба діяти саме зараз, постійна готовність до змін стали її
особистою формулою успіху. Мова піде про \textbf{Тетяну Міняйло} – засновницю і
керівницю кондитерської студії \enquote{Пряник House}, тренера курсу корекції ваги
\enquote{ЖрусикиNET} та тренера і викладача в Освітньому центрі \enquote{Місто}. Вражає скільки
відкрита і гармонійна Тетяна увібрала в собі вмінь і унікальних навичок... У неї
є чому повчитись, а її досвід може стати корисним для багатьох українців.

Народилася Тетяна в Донецьку. 12 років пропрацювала в школі вчителем
української мови та літератури, з них – 7 років – за сумісництвом завучем з
виховної роботи. Згадує, що добиратися до школи було складно, жила вона на
околиці, за 25 км від школи. Але класне керівництво у випускному класі не
дозволяло відповідальній вчительці залишити свою роботу. Після народження
синочка вона вирішила все ж таки покинути школу, занадто багато часу забирала
робота.

\textbf{Читайте також:} \emph{Мариупольцы создадут книжный клуб для обсуждения новинок литературы}%
\footnote{Мариупольцы создадут книжный клуб для обсуждения новинок литературы, Роман Катріч, mrpl.city, 25.06.2019, \par%
\url{https://mrpl.city/news/view/mariupoltsy-sozdadut-knizhnyj-klub-dlya-obsuzhdeniya-novinok-literatury}
}

У 2011 році їй вдалося відкрити \textbf{Приватну творчу студію для особливих діток
(аутисти, діти з синдромом Дауна).} У школі вона часто проводила індивідуальні
заняття з дітьми з ДЦП, могла знайти до них індивідуальний підхід. Цей досвід
знадобився їй і при відкритті центру. Загалом Тетяна підкреслює, що діти з
якоюсь вадою завжди відрізняються особливою чуйністю, щирістю, мають в чомусь
унікальний талант, з ними не може бути нецікаво.

Також у дворі Тетяни жила дівчинка, яка довгий час не говорила. З нею теж
займалася Таня. Спочатку Тетяна створила гурток, в якому впроваджувала техніку
квілінгу. Вона вважає, що моторика пальців (створення аплікацій, робота з
папером) сприяє розумовому і творчому розвитку діток. Їй довелося багато бігати
по дитячих садках, поки не наважилася на створення окремого центру. Таня
проводила виставки, різні заходи з дитячими виробами. До неї приходили діти з
усього району. Їй вдалося навіть надрукувати кілька цінних методичок.

Але воєнні події не дали Тетяні продовжувати цю соціально важливу справу.
Наскільки було важко жінці залишитися з двома синами, яким на той час
виповнилося 3,5 роки та 9 місяців, складно уявити. Їхній будинок був повністю
знищений, тому довелося шукати нове помешкання. Довгий час довелося жити в
Юр'ївці, прямо на узбережжі Азовського моря.

\textbf{Читайте також:} \emph{Данька. История о чуде}%
\footnote{Данька. История о чуде, Фонд Рината Ахметова, mrpl.city, 21.06.2019, \par%
\url{https://mrpl.city/blogs/view/danka-istoriya-o-chude}
}

Пізніше, переїхавши до Маріуполя, вона не могла знайти постійну домівку у місті
і намагалася пристосуватися разом з дітьми до існуючих умов. Про всі труднощі і
переживання нашої героїні можна прочитати в її віршах, адже вона ще й
неймовірно глибока і чуттєва поетеса.

Тетяна згадує, що дуже допомогли дівчата-волонтери з \textbf{Фонду допомоги загиблим і
пораненим.} Вони допомогли і з речами, і з кімнатою. Щоб якось завдячити їм на
Вербну неділю Тетяна приготувала великий коржик. Він так всім сподобався, що їй
запропонували приготувати смаколики на ярмарки, які проводив Фонд. Тоді вона не
здогадувалася, що це початок її майбутньої, найбільшої в житті справи. Все літо
Тетяна простояла на ярмарках з пряниками.

\ii{25_06_2019.stz.news.ua.mrpl_city.1.tetjana_minjajlo_zminy_treba_pochynaty_z_sebe.pic.2}

Сьогодні вона є засновницею кондитерської студії \enquote{Пряник House}, яка стала
відомою далеко за межами міста. Її пряники замовляють у Буковелі, Львові,
Харкові, Києві, Одесі, у польських містах – Вроцлаві та Гдянську. З 2016 року
завдяки громадській активістці \textbf{Аліні Пономарьовій} Тетяна починає проводити
майстер-класи з розпису пряників. Сьогодні вона відвідує всі міські заходи, де
її смаколики є головною прикрасою. Їй вдалося виготовити власний рецепт,
завдяки якому пряник залишається довгий час дуже м'яким. Вона має безліч ідей,
але найчастіше у неї замовляють пряники з персонажами з мультиків. Мало хто в
місті сьогодні не знає про кондитерську Тетяни. До того ж, пряниковий будиночок
Тані отримав навіть папа Римський. Сьогодні у неї дуже багато замовлень.
Найбільш прибутковий місяць – грудень, адже всі поспішають придбати унікальні
смаколики до новорічних свят. Сім'я її підтримує і пишається нею. Але сама
Тетяна й гадки не мала, що колись в своєму житті стане кондитеркою.

\ii{25_06_2019.stz.news.ua.mrpl_city.1.tetjana_minjajlo_zminy_treba_pochynaty_z_sebe.pic.3}

Разом з тим Тетяна є тренером і викладачем в \textbf{Освітньому центрі \enquote{Місто}}, де
системно навчає маленьких маріупольців IT-технологіям. І це ще не все.
Виявляється, що у нашої героїні вистачає часу на ще одну корисну і цікаву
справу. Вона отримала сертифікат від міжнародної школи фітнесу, тому,
започаткувавши курси дієтолога та корекції ваги \enquote{ЖрусикиNET}, вирішила навчати
інших. При цьому вона ділиться і власним досвідом Їй вдалося схуднути з 92 до
69 кг. Наразі у неї 13 учнів. Чотири – з Маріуполя, решта – з усієї України.
Вона наголошує, що \emph{\enquote{худнути потрібно для себе. Спочатку схуднути слід
внутрішньо, а вже потім ззовні!}}.

\ii{25_06_2019.stz.news.ua.mrpl_city.1.tetjana_minjajlo_zminy_treba_pochynaty_z_sebe.pic.4}

\ii{insert.read_also.demidko.ponomareva_feja}
\ii{25_06_2019.stz.news.ua.mrpl_city.1.tetjana_minjajlo_zminy_treba_pochynaty_z_sebe.pic.5}

Сьогодні Тетяна – мама двох чудових синочків. Вдома на неї називають \emph{\enquote{хороший
пацан Танюха}}. Її сини - \enquote{ходячі совісті}. Вони знають, що за собою треба
прибирати і дивуються, коли на їхніх очах хтось смітить на вулицях міста. Часто
на подвір'ї вони гуляють у футбол, катаються на велосипедах. Сім'я Тетяни –
палкі шанувальники футболу, постійно підтримують \enquote{Шахтар}. А ще Тетяна вишиває
цілі картини і портрети окремих людей. Ну що тут скажеш: талановита людина –
дійсно талановита у всьому.

\ii{25_06_2019.stz.news.ua.mrpl_city.1.tetjana_minjajlo_zminy_treba_pochynaty_z_sebe.pic.6}

\textbf{Улюблений фільм:} \enquote{Куди приводять мрії}.

\textbf{Книги:} \enquote{Майстер і Маргарита}, всі твори Олександра Виженка.

\textbf{Порада маріупольцям:} \emph{\enquote{Всі чекають змін, але не змінюються. Потрібно вміти змінювати внутрішній світ і починати зміни з себе!}.}

\ii{25_06_2019.stz.news.ua.mrpl_city.1.tetjana_minjajlo_zminy_treba_pochynaty_z_sebe.pic.7}
\ii{25_06_2019.stz.news.ua.mrpl_city.1.tetjana_minjajlo_zminy_treba_pochynaty_z_sebe.pic.8}
