% vim: keymap=russian-jcukenwin
%%beginhead 
 
%%file 13_10_2021.fb.menendes_enrike.1.donbass_identichnost.cmt
%%parent 13_10_2021.fb.menendes_enrike.1.donbass_identichnost
 
%%url 
 
%%author_id 
%%date 
 
%%tags 
%%title 
 
%%endhead 
\subsubsection{Коментарі}

\begin{itemize} % {
\iusr{Энрике Менендес}
Картинка вам, чтобы не быть голословным

\ifcmt
  ig https://scontent-frt3-1.xx.fbcdn.net/v/t39.30808-6/244752603_6278243732218000_2500055396869254737_n.jpg?_nc_cat=104&ccb=1-5&_nc_sid=dbeb18&_nc_ohc=Cz4rBIg737gAX8d6tOA&_nc_ht=scontent-frt3-1.xx&oh=2add01d9cb111c3fb665e4d888927dc8&oe=61A47260
  @width 0.4
\fi

\begin{itemize} % {
\iusr{Иван Шашлов}
Красивое...

\iusr{Энрике Менендес}
\textbf{Иван Шашлов} Да, но НЕ НАШЕ!

\iusr{Энрике Менендес}
\textbf{Иван Шашлов} Я за то, чтобы в Ивано-Франковске художник из Авдеевки нарисовал трубы кокосхима, только врядли такое произойдёт.

\iusr{Stanislav Tsykalovskyi}
Красивый мурал. Побольше бы таких.

\iusr{Иван Шашлов}
\textbf{Энрике Менендес} ну, трубы - это б-гомерзко же и не лепо.
Их и здесь же остаться не дОлжно, только кавьярни и барбершопы.

\iusr{Rostyslav Pelekhovych}

Энрике, а яке ваше? як виглядали селянки на "Донбасі"?
> Я за то, чтобы в Ивано-Франковске художник из Авдеевки нарисовал трубы кокосхима
це було б релевантно, якби в ІФ ходили в селянському одязі. Але ні

\iusr{Ism Vitaly}
\textbf{Энрике Менендес} Ну да. Коксохим то он аутентичнее будет.
Пусть он отравляет не только воздух но и радует глаз на мурале.
И написать "Товарищ! Твой завод - твоя гордость"
 · Ответить · 6 нед.
\iusr{Olha Vesnianka}
\textbf{Ism Vitaly} Жесть. Або відстійник зобразити

\iusr{Анатолий Наседкин}
\textbf{Rostyslav Pelekhovych} як даунбаській "іспанєць" тільки з "длінимі валасамі" )))

\iusr{Лида Радкевич}
\textbf{Энрике Менендес} что красивого в трубах коксохима?

\iusr{Анатолий Наседкин}
\textbf{Энрике Анатольевич Менендес} звісно ж не ваше. Ваше це якийсь "моторолєр" з баяном чи "ґіві" з пляшкою піва-Дєсант ))).


\iusr{Энрике Менендес}
\textbf{Stanislav Tsykalovskyi} Да и терриконы во Франковске тоже было бы круто нарисовать. Как думаешь, будет красиво?

\iusr{Stanislav Tsykalovskyi}
\textbf{Энрике Менендес} я не люблю терриконы.
Степь люблю, байбаков люблю. Если байбака зафигачить - это будет наш, донбасский стиль?

\iusr{Тая Дунаева}
\textbf{Энрике Менендес} красиво.. но я не уверенна даже о слобожаночке.. это разве их узоры?

\iusr{Rostyslav Pelekhovych}
Энрике, ну так що саме на муралі, окрім слова «слобожанка», не відповідає ідентичності мешканців науколишніх сел ?

\iusr{Евгений Афендиков}
Там некоторые и не знают что такое террикон ы, были приезжие спрашивали - а что за горы у вас.

\iusr{Vitalii Ovcharenko}
\textbf{Rostyslav Pelekhovych} В робе - фуфайке с номером на спине надо було нарисовать, вот тогда для Енріке було б хорошо))))

\iusr{Игорь Рязнов}
да какая это слобожанка))) на узбечку больше похожа))))

\iusr{Владимир Кабасин}
\textbf{Энрике Анатольевич Менендес}, ВАШЕ! Ты в Украине живёшь! Прекрати её делить!

\iusr{Aleksej Vasiliev}
\textbf{Энрике Менендес} Девушка таки лучше труб кокосхима.

\iusr{Aleksej Vasiliev}
\textbf{Евгений Афендиков} Терриконы не тянут на предмет гордости.

\iusr{Aleksej Vasiliev}
\textbf{Энрике Менендес} Вы трубу за девушку? Странный обмен любезностями.))

\iusr{Городецкая Марина}
\textbf{Евгений Афендиков},спи*дел? а так хочется, чтобы выглядело, как взаправду?

\iusr{Городецкая Марина}
Мандес, ты ватный ублюдок всячески стараешься, чтобы пуйлу понравиться?

\iusr{Евгений Афендиков}
\textbf{Aleksej Vasiliev} 

терриконы это часть нашего окружения, у когото гималаи и карпаты, а у нас
терреконы, кстати у некоторых есть названия. и как люди лазят на говерлу, мы
пацанами лазили на терриконы.


\iusr{Александр Балдынюк}
А тысячи солдат росармии и рос танки на Донбассе - это ваше?

\iusr{Сергей Бондаренко}
\textbf{Энрике Анатольевич Менендес} - автор явно на подтанцовках у россиянских пропагандонов. "народ дамбаса", "дамбасянская идентичность"..
Явно подзабыл сей "испанец", как голодом выморили украинцев на этой территории, и навезли туда эту "идентичность" из северных краёв.
\end{itemize} % }


\iusr{Энрике Менендес}
Делитесь новостью. Пишите тэг \textbf{\#Я\_это\_Донбасс}

\begin{itemize} % {
\iusr{Ievgen Nefediev}
\textbf{Энрике Анатольевич Менендес} \textbf{\#ПішовНахуйМенендес}

\iusr{Илья Петиченко}
\textbf{Энрике Анатольевич Менендес} 

тегнули два человека. Дончане не думают о своей идентичности. Те дончане
которые поддержали псевдореферендум думали не о создании новой республики, а
том чтобы с Донбассом сделали как с Крымом - примазали к нефтяной экономике в
качестве младшего родственника. Вы занимаетесь ерундой.

\end{itemize} % }

\iusr{Энрике Менендес}
\textbf{Денис Казанский} 

давно тебя хотел спросить - почему ты так ненавидишь Донбасс? Это же явная
атака на нашу региональную идентичность, в которой нет ничего плохого. Она есть
в любом другом регионе страны - где-то больше, где-то меньше

\begin{itemize} % {
\iusr{Денис Казанский}
\textbf{Энрике Менендес} не могу понять, это стеб, или у тебя крыша потекла?

\iusr{Юрий Лукшиц}

Казанский просто хочет украинизироваться. Он не понимает, что для националистов
и "патриотов" он никогда своим не станет. И служить он будет любой власти.

\iusr{Serg Kosinov}

Ленин называл Горького "наш путаник". Энрике милый путаник. Но когда количество
этих милых путаников превышает некую критическую величину, начинается война.

\iusr{Dima Schors}

Энрике, Казанский проявил себя еще в самом начале - в тексте про мертвого бойца
ЗСУ на проводах. К чему эти ваши, Энрике, экивоки?)))

\iusr{Анатолий Наседкин}
\textbf{Юрий Лукшиц} 

ДБЙБ, просто будь-яка притомна людина розуміє що "труби коксохіма" не можуть
бути будь-якою ідентичністю. Може ви себе й ідентифікуєте як "мєталурґі" чи
"шахтьори", але навіть притирене керівництво донбабве намагається замінити
українську ідентичність на "раіскую" ))). Балалайка і водка зараз в тренді на
даунбасі, звісно коли "герої" вийдуть з цехів якогось "коксохіму". Ферштейн?

\iusr{Энрике Менендес}
\textbf{Анатолий Наседкин} А что может быть идентичностью, любезный Вы наш? Борщ?

\iusr{Энрике Менендес}
\textbf{Денис Казанский} Это прямой вопрос, от которого ты ушёл. С крышей у меня всё в порядке во всех смыслах.
Кстати, я заметил твой милый стук в СБУ. Скриншоты твоих показаний мне прислали )

\iusr{Сергей Шабашов}
\textbf{Энрике Менендес} странно, войска РФ вас не смущают, а мурал в Авдеевке вас смущает.

\iusr{Энрике Менендес}
\textbf{Анатолий Наседкин} Ещё раз говоришь слово "даунбас" и уходишь в бан. ДБЙБ.

\iusr{Анатолий Наседкин}
\textbf{Энрике Анатольевич Менендес} Вам як майже "европецю" та справжньому донбасцю, скажу - якби ж ви вчились так як треба то не писали б дурні. Весь ваш "допис" то суцільна демонстрація невігластва.

\iusr{Анатолий Наседкин}
\textbf{Энрике Анатольевич Менендес} звісно, будь яка національна кухня є ознакою та складовою ідентичності. Що ви пишете? Не грузіть ні себе ні людей, зупиніться, не робіть з себе посміховисько.

\iusr{Александр Лапин}
\textbf{Энрике Менендес}, есть люди, у которых место сидения определяет точку зрения. Было бы странно требовать от блоггера на содержании власти критики политики этой самой власти...

\iusr{Венди Белова}
\textbf{Денис Казанский} а раньше не текла?

\iusr{Игорь Рязнов}
у казанского просто работа такая))) деньги он на этом зарабатывает!!!!

\iusr{Константин Исиков}
\textbf{Денис Казанский} это у тебя крыша потекла, ЧМ0

\iusr{Олег Платонов}
\textbf{Энрике Анатольевич Менендес} 

нет никакой региональной идентичности!, Особенно Донбасса... Донбасс это
географический регион и не нужно сочинять. Большинство людей свезли с разных
регионов для проведения индустриализации, а вы про идентичность... Не знаю под
чем вы, но из-за таких как вы в нашей стране война и каждый день гибнут люди..

\iusr{Галина Борисова}
\textbf{Олег Платонов} если вы не из Донбасса, то вы просто не знаете, о чём говорите. Как сказал мне один человек: "Донецкий - это бренд! "

\iusr{Олег Платонов}
\textbf{Галина Борисова} ну-ну, и где ваш Донбасс? Доидентичелись? Считаете себя особенными... Так каждый человек может сказать, что он особенный - и будет прав, только у него всё так же две руки, две ноги и кушать хочется..

\iusr{Natasha Semergey}
\textbf{Энрике Менендес} 37 й год форева ! Времена меняются, а сущность людей...

\iusr{Любовь Головко}
\textbf{Энрике Менендес}  @igg{fbicon.thinking.face}{repeat=3}  а в чем проявляется лично ваша любовь к Донбассу, начиная с довоенного периода?

\iusr{Игорь Афендиков}
\textbf{Галина Борисова} 

Вы правы. Сама фраза несёт глубокий смысл. "Донецкий" отвечает на вопрос чей?
Донецкий уголь, донецкий коксохим, донецкое пиво или донецкая мафия. В данном
выражении имеется ввиду донецкий раб? Или холоп? Принадлежность чего-то, не
одушевлённого, кому-то. Дончанин это более верный вариант.


\iusr{Rostyslav Pelekhovych}
> Как сказал мне один человек: "Донецкий - это бренд! "
Це був Янукович, Галина?

\iusr{Сергей Голован}

да не ненавидят они Донбасс, бесполезно таким обьяснять про любовь и ненависть,
люди за деньги работают, завтра заплатят будут горячо любить

\iusr{Энрике Менендес}
\textbf{Любовь Головко} 

Мне легко ответить на этот вопрос - хотя я много раз уже рассказывал.

1) Вы наверное знаете, что в Донецке у меня был бизнес - агентство по
интернет-рекламе. В её миссии ещё с 2010 года было записано "развивать
маркетинг в Донецке". Меня цепляло, что в таком крупном городе нет ни одного
партнёра Гугла или Яндекса. Я задался целью и в 2012 году моё агентство стало
первым в Донецке сертифицированным партнёром Гугл и Яндекс.

2) Я платил налоги. Моя фирма работала в белую.

3) Я развивал своих сотрудников. Обучал их за счёт компании. Сейчас почти все они переселенцы и благодарят меня за полученные знания, что помогло им найти хорошую работу.

4) Я дважды выступал спонсором и соорганизатором донецкого TEDx. Один раз выступал как спикер - без гонорара.

5) Я вёл для студентов курс по саморазвитию в ДонНУ. Бесплатно. Правда успел выступить только раз - это была зима 13-го ((

6) Выступал как лектор в донецком клубе предпринимателей. Бесплатно.

7) Моя жена была активным членом сообщества мам kroha.dn.ua. Мы постоянно занимались благотворительностью.

\iusr{Городецкая Марина}
\textbf{Serg Kosinov}, он не "путаник", он ватник и коллаборант. Работает на пуйла, однозначно...

\iusr{Галина Борисова}
\textbf{Rostyslav Pelekhovych} нет, это был человек, который под обстрелами ездил по линии разграничения и кормил стариков.

\iusr{Любовь Головко}
\textbf{Энрике Анатольевич Менендес} 

не могу принять этот список, как доказательство вашей особой любви к Донбассу.
Многие из живущих в Донецке (как и в других городах, поселках, селах Донбасса)
могут написать аналогичный список. Денис Казанский в том числе....

Вы еще и компьютерной техникой занимались, не так ли?  @igg{fbicon.wink} 

Да и некоторые другие вехи вашей деятельности вы в перечень "любви к Донбассу"
не включили... Понимаю ваше желание дистанцироваться от некоторых моментов...

А Донбасс, он разный, у его жителей разные взгляды, мнения и приоритеты и...
своя любовь малой родине... И никто не может претендовать на исключительность
своей любви...

p.s. относительно ваших возмущений красивым муралом...

Почитайте книгу "История Харьковского Слободского козачьего полка 1651-1765" Е.
А. Альбовского. В сети есть сканы издания 1895 или 93 года...

К слову, Юзовка (и значительная часть Донецкого кряжа) долгое время входила в
состав Екатеринославской губернии )))

\iusr{Olga Alunci}
\textbf{Олег Платонов} есть региональная идентичность.

\iusr{Вадим Щетинин}
\textbf{Денис Казанский}, разрешите Энрике выложить в чат скрины Вашего "стука" в СБУ.
"Кстати, я заметил твой милый стук в СБУ. Скриншоты твоих показаний мне прислали )"

\iusr{Mykola Malukha}
\textbf{Энрике Анатольевич Менендес}, 

так чего слился с ответа по поводу идентичности Донбасса? В чем идентичность
Донбасса? Нытье "мы кормим всю Украину", говорить на безграмотном русском
языке, работать шахтером, пьянствовать после смен, молиться на завод и не
мыслить своей жизни вне завода.... что составляет идентичность Донбасса?

\iusr{Олег Платонов}
\textbf{Olga Alunci} 

где-то есть(Волынь, закарпатье, но не на Донбассе.. менталитет - согласен, но
идентичность??? Нет такой народности, нет языка, даже диалекта( не считая
фени), нет обычаев и нет традиций... Так о какой идентичности идёт речь? О том
что все выросли у терриконов и у всех есть в семье шахтеры? Я понимаю, что с
Украиной вы себя не ассоциируете и судя по всему с Россией тоже, не смотря даже
на пропаганду.. я понимаю, что вы ищите идентичность, но ее нет, скажите
спасибо СССР и рос пропаганде, ну и конечно же властям в Украине за всё время
независимости, вас так и не интегрировали ментально...

\iusr{Любовь Головко}
\textbf{Oleg Platonov} 

не во всех в семьях есть шахтеры. @igg{fbicon.smile}  И терриконы есть не везде... И жители
Донбасса очень разные по национальности, вере, ментальности, культурным
предпочтениям, уровню образования и т.д.

Огромная вина за то, что произошло - на ПР. Сейчас большинство из регионалов,
благодаря которым все и случилось, - в подконтрольной Украине, и очень
комфортно себя там чувствуют. У них есть деньги, связи... Их дети-внуки
спокойно учатся в украинских школах и вузах... И плевать им на войну, на
проблемы региона, на семьи, которые оказались по разные стороны линии раздела,
на горе и боль своих земляков...

Вот и вся идентификация... @igg{fbicon.frown} 

\iusr{Олег Платонов}
\textbf{Lyubov Golovko} согласен с вами.. у меня тоже есть корни с Донбасса и семья жены - переселенцы

\iusr{Валерий Антонов}
\textbf{Олег Платонов} вы пытаетесь говорить о том, чего не знаете. Вы вообще были в тех краях о которых идет разговор ? Думаю что, нет..

\iusr{Сергий Федорынчык}

Енрике, на питання - в чому саме виявляється ідентичність? - Ви мусите дати
власну відповідь, бо підняли цю тему, а не запитувати у інших.

\iusr{Сергий Федорынчык}
\textbf{Энрике Анатольевич Менендес} 

Прочитати це цікаво і корисно для кращого розуміння Вашої особистості, бо під
час нашої короткої зустрічі Ви дуже мало про це згадували. Разом з тим, Ваші
позитивні якості можуть виявитись надто недостатніми для неймовірно складного
завдання - досягнення миру на Донбасі. І спроба створення політичної партії
скоріше віддалить Вас від мети, ніж наблизить. Я ще раз нагадую про Вашу
обіцянку - написати цілісний виклад Ваших поглядів на проблеми Донбасу та шляхи
їх розв'язання. Це абсолютно необхідно для розвитку громадського руху за
порозуміння та мир на Донбасі. Без порозуміння не буде й миру, навіть якщо
війна набридне жителям ОРДЛО. Бо Кремль від цього лише перестане прикриватись
думкою цих людей у своїх дальніх намірах відновити Російську Імперію. Ви
стільки разів дорікали Україні, що вона не чує Донбасу (зокрема, у Вашому
озвученні). Але що саме Ви хочете сказати Україні - досі так і неясно, якщо
написати цього Ви не в змозі. Цікаво, чи чують Вас у Кремлі? - принаймні,
кремлянам Ви не дорікаєте! Ще простіше запитання - чи чують Вас на Донбасі,
зокрема в ОРДЛО, крім Ваших друзів, знайомих та читачів Вашої сторінки у
Фейсбуці?

\iusr{Валерий Харламов}
\textbf{Энрике Анатольевич Менендес}
Ну, а теперь зачем занимаешься херней?

\iusr{Борис Зленко}
\textbf{Oleg Platonov} 

язвк русский. Диалект своеобразный - надо быть родом из Луганска, чтобы понять.
У нас есть русские особые, слова которые не употребляются в той же Полтаве
например. Словарь русского языка Владимир Даль написал уроженец Донбасса.
Традиции у нас есть. Отлтчающиеся от Львовских.

\end{itemize} % }

\iusr{Энрике Менендес}
\textbf{Сергей Сивохо} 

а ты как к такому относишься? Мы долго ещё будем молчать, пока убивают нашу
идентичность?

\begin{itemize} % {
\iusr{Владимир Кабасин}
\textbf{Энрике Анатольевич Менендес}, ты, я вижу, такой борец за мир и единство в Украине, что тебе иззавидуется любой «ополченец».
КТО?
УБИВАЕТ?
ТВОЮ?
ИДЕНТИЧНОСТЬ?
Кроме тебя самого.

\iusr{Марк Зоряний}
\textbf{Владимир Кабасин} он фашист на зарплате РФ


\iusr{Владимир Кабасин}
\textbf{Марк Зоряний}, 

вот с этим я не согласен. То, что полезный для целей Кремля дурачок —
соглашусь. То, что заблудившийся в реальности, склонный выгораживать
землячество человек — соглашусь. Тем, кстати, Кремлю и полезен, потому что так
удобнее же делить государство — Донбасс отдельно, а Слобожанщина — отдельно.
Львов — вообще Польша, не наше это, типа.

Энрике, когда про самобытность «народа Донбасса» говорят, то что конкретно
имеют ввиду? Есть национальная кухня Донбасса? Национальная одежда? Может быть
есть народные песни Донбасса (пару советских песен из репертуара Кобзона и
Бернеса не предлагать)? Или танцы?

Какую именно идентичность можно на Донбассе «убить»?

\iusr{Сергей Перевозчиков}
\textbf{Энрике Менендес} вашу идентичность убили в 2014 году.

\iusr{Андрій Винниченко}
\textbf{Володимир Кабасін} Он стопроцентный сепаратист, у него сепаратистская идентичность. Потому его и в ДНР считают предателем, поскольку деятели ДНР отоджествляют себя с Россией, считают себя русскими патриотами, а ДНР частью Великой России.

\iusr{Olga Alunci}
\textbf{Сергей Перевозчиков} и ее продолжают убивать в Украине. Мендес все правильно написал.

\iusr{Андрій Винниченко}
\textbf{Марк Зоряний} Фашист не может быть на зарплате. Фашисты были идейными как и коммунисты.

\iusr{Марк Зоряний}
\textbf{Андрій Винниченко} а он фашист за гроші, які не пахнуть. Завтра заплатять він буде прихильником визвольного руху ОУН -УПА. Він по духу поліцай

\iusr{Светлана Перепелкина}
Сергей, они убили свою идентичность в 14году, призвав путина в Украину.
\end{itemize} % }

\iusr{Энрике Менендес}
\textbf{Андрій Єрмолаєв} ты только сегодня писал про Донбасс. Полюбуйся. Может я с ума сошёл, как некоторые пишут? И нет никаких поводов для возмущения?

\begin{itemize} % {
\iusr{Андрій Єрмолаєв}

Одна из целей когнитивной войны, как составляющей т.н. "многосферной стратегии"
или "операции", - смещение и трансформация смыслов, привычных и значимых для
социального уклада. Донбасс - это уклад жизни, сложившийся в регионе в
определенном историческом процессе и контексте. А не территория с терриконами и
не "масса населения". Разрушение Донбасса, как и его "рождение" в связи с
возникшим "сопротивлением" - связано с этим укладом и реакцией на попытки его
разрушить. Помнишь попытки переучить "шахтеров - в предприниматели?" А
использование реформы угольной отрасли как способа "закрыть неэффективное" и
"перепрофилировать"? А бандитские времена, которые более десятилетия как бы "не
замечали", и которые оставили неиздадимый след на истории, имидже и стереотипах
о Донбассе? Думаешь, случайно? А я думаю, нет. Уже тогда болялись. Особенно
после событий 1989-1993гг. Вот тогда еще все началось. А сейчас - это уже
попытки закрыть тему окончательно. Донбасс, превращенный просто в обеднеывшее и
дезориентированное население, - это и есть цель войны. А вот население как
ресурс можно делить, "мигрировать", переобустраивать и т.д. Так что тут все три
стороны в игре (Киев, Москва, Большой Запад). Никому из них не нужен
амбициозный промышленный регион, с пролетарскими замашками, с запросом на
индустриальный уклад, образование и т.д. и т.п. Ну а вся эта сбежавшая,
набившая карманы на Донбассе "рег-элита" - случайные люди, игроки в казино,
просто проходимцы. И как видно теперь, без корней (настоящих корней) и без
совести. Но с самим Донбассом еще страница не перевернута...

\iusr{Алексей Зубов}
\textbf{Андрій Єрмолаєв} , о да-да: мы такие ох@енные, а нас никто не любит и «не слышит». А откуда все эти «рег-элиты», януковичи, бандиты? А хрен знает, «случайные люди». Мы за них, конечно голосовали килотоннами, но вообще их рептилоиды подбросили, да. «Стереотипы» и вот это вот всё.
То есть «донбасская идентичность» никаких рефлексий не предполагает. Мы услышали наконец.
Спасибо жителям Донбасса, так сказать, дуже Дякую

\iusr{Олег Платонов}
\textbf{Андрій Єрмолаєв} поплава


\iusr{Энрике Менендес}
\textbf{Андрій Єрмолаєв} Шикарный комментарий, Андрей. Спасибо!

\iusr{Ольга Сидорова Сидорова}
\textbf{Андрій Єрмолаєв} ты примитивный прокремлевский пропагандист расскажи про хунту и про фашистов и про мураевский пропагандонский украинофобский канал НАШ который спокойновоняет русским миром и никто этих Беляшей не трогает

\iusr{Сергий Федорынчык}

Ай-яй-яй, Андрію! Цілей не буває без цілепокладання та іх носіів. Хто, де і коли
ставив цілі руйнаціі ідентичності Донбасу?

\iusr{Владимир Владимиров}
\textbf{Андрій Єрмолаєв} "Рождение Донбасса в связи с возникшим сопротивлением" - получается, что это внебрачный, да ещё и брошенный ребенок Бородая и Стрелкова?

\iusr{Оксана Возова}
\textbf{Энрике Анатольевич Менендес} два "українолюба".

\end{itemize} % }

\iusr{Энрике Менендес}
\textbf{Илья Кононов} - Вы ведь тоже об этом недавно писали

\begin{itemize} % {
\iusr{Илья Кононов}

Конечно. У части нашего правящего класса и обслуживающей его интеллигенции есть
желание изменить идентичность нашего региона. Донбассу навязывают Слобожанщину
как замену его самого. Они исторически связаны, но по многим характеристикам
отличаются. Слобожанщина в культуре не дала украинского-русского синтеза, а для
Донбасса - это основа его культурной идентичности. Само самоназвание нашего
региона то пытались заменить на безликое и никого не затрагивающее Подонцовье,
то объявляют вообще "нарративом врага".

\iusr{Илья Кононов}

\href{http://www.ostrovok.lg.ua/knigohranilishhe/nauka/kononov-if-etnos-cinnosti-komunikaciya-donbas-v-etnokulturnih-koordinatah-ukrayini-monografiya}{%
Кононов І.Ф. Етнос. Цінності. Комунікація (Донбас в етнокультурних координатах України). Монографія, %
ostrovok.lg.ua%
}

\iusr{Илья Кононов}

\href{http://geopolitika.am/dir/wp-content/blogs.dir/1/files/2021_3_21_33.pdf}{%
Донбасс в региональной системе Украины, geopolitika.am%
}

\iusr{Станислав Печенкин}

Та нема там ніякого російсько-українського синтезу. Від слова "зовсім". Це
звичайний совок. Густий совок.

\iusr{Андрій Винниченко}
\textbf{Stanislav Pechenkin} 

Совок це люди з радянськими поглядами? Якщо ви маєте на увазі радянських людей,
то у радянських людей не було ненависті до українців та України, Україну вони
бачили УРСР. А цей синтез з Пушиліними, Плотницькими, Захарченками, Моторолами,
Гіві на чолі ненавидить українців і все українське.


\iusr{Halyna Pry}
\textbf{Илья Кононов} йдіть до біса зі Слобожанщини з вашою "культурою' та синтезом. Відповідь з Харкова.

\end{itemize} % }

\iusr{Энрике Менендес}
Людям, у которых нет идентичности, видимо, уже ничем не помочь (

\begin{itemize} % {
\iusr{Anatoly Vlasyuk}
\textbf{Энрике Анатольевич Менендес} Чому б вам не написати про те, що твориться у вас під носом? Я маю на увазі так звану "Ізоляцію" в Донецьку. Оце тема для з'ясування ідентичності на тлі убивств і катувань! Але ж ви не посмієте про це написати, бо самі станете в'язнем "Ізоляції". А всі ваші пости - це для того, щоб розхитувати Україну задля російської окупації.

\iusr{Serg Kosinov}
\textbf{Энрике Менендес} Автор сам не понял, насколько он прав.

\iusr{Алексей Зубов}
\textbf{Энрике Менендес} , вы на вопрос не ответили: условный янукович - это часть вашей идентичности, или, как изволит выражаться Ермолаев, «случайные люди»?

\iusr{Сергей Перевозчиков}
\textbf{Энрике Менендес} да, Вам уже ничем не помочь.

\iusr{Rostyslav Pelekhovych}
Ви пропустили слово, має бути «Людям у которых нет донбасской идентичности ничем не помочь»

\iusr{Владимир Кабасин}
\textbf{Энрике Анатольевич Менендес}, 

погоди, это ты сейчас отстаиваешь вот прямо в этом посте, что у тебя нет
украинской идентичности. Что Украина отдельно, со всякими своими
Слобожанщинами, но Авдеевка — уже нет. Там уже не Украина.

И что это необходимо отстаивать. Сивохо даже на помощь позвал.

Мы все видели «идентичность Донбасса». Итогом правления твоих земляков был
Майдан. Дважды. А потом, не устану повторять, Донбасс выстрелил себе в ноги,
типа, не услышали его. Услышали, в том и дело. Зачем же себе самострел делать,
вот что непонятно. И как теперь выкарабкиваться из этой ситуации, когда часть
Донбасса агрессор использует в собственных интересах и ни при каких условиях,
кроме полного подчинения, добровольно его не отдаст.

Энрике, не будь глупцом.

\iusr{Сергей Основин}
\textbf{Энрике Менендес} суть в том, что людям, у которых нет идентичности, помощь не нужна. Тем более, помощь от тех, у кого идентичность есть.  @igg{fbicon.wink} 

\iusr{Вадим Щетинин}
\textbf{Энрике Менендес} ее, этой идентичности и до 2014 не было, результат на лицо.

\iusr{Вадим Щетинин}
\textbf{Энрике Менендес}, эта тема по муралу просто никакашка.
Попробуйте, что-то боле серьезно, если потяните.
Например такие вопросы, которые поднимались на "Студресрублике".
Вот о чем нужно сподвигнуть думать молодежь.

\href{https://studrespublika.com/buty-razom/}{%
Навіщо нам, таким різним, бути в Україні разом?, studrespublika.com, 06.10.2021%
}

\iusr{Андрей Колосов}
\textbf{Владимир Кабасин} 

Ну, ни о каком полном подчинении никого никому речи нет. Минские соглашения
закладывают некие гарантии субъектам ОРДЛО, поскольку уже было беспрецедентное
решение - АТО.

\iusr{Андрей Колосов}

Энрике, все же о региональной идентичности - нельзя так категорично: хочешь,
имей ее, не хочешь - не имей. Сейчас она обострена нашим положением "все в
одной лодке". А до событий 2014? Насколько она была значима? Ее значение было
ничтожным. Обычный региональный патриотизм: "Шахтер", луганские тепловозы,
донецкие розы и т.п. Такое же во всех регионах имеет место....

\iusr{Марк Зоряний}
\textbf{Anatoly Vlasyuk} 

Ви праві, він за свою ідентичність уявну збирається воювати, але скоріше
підбурювати інших, бо саме чамороше тільки подряпати нігтіками може

\iusr{Владимир Кабасин}
\textbf{Андрей Колосов}, 

ОРДиЛО не субъекты, а объекты, начнём с этого. И, конечно, некая особость этих
частей двух областей Украины будет закреплена законодательно. На период
нормализации. А после будет снята.

Никакого, абсолютно, повода для автономизации у Донбасса нет. Вообще.

\end{itemize} % }

\iusr{Tatyana Malyarenko}

имхо, слобожвнская идентичность еще слабее, чем донбасская  @igg{fbicon.smile}  я бы начинала
беспокоиться, когда девушку в половецком костюме на муралах будут рисовать:)

\begin{itemize} % {
\iusr{Энрике Менендес}
\textbf{Tatyana Malyarenko} Слобожанской идентичности нет вообще. Я ни разу не видел человека с татуировкой "Слобожанщина". А с татуировкой "Донбасс" видел сотни )

\iusr{А. Полтавцев}
\textbf{Энрике Анатольевич Менендес} Так чего ж тогда так переживаете-то?  @igg{fbicon.smile}  Татуировка же круче мурала  @igg{fbicon.smile} 

\iusr{А. Полтавцев}
\textbf{Tatyana Malyarenko} 

Скажите, а что точно не приемлемо для Донбасса: девушка в половецком костюме
или та же самая девушка вообще без костюма? Социологическое исследование
провожу, о хрупкости донбасской идентичности через разрушительное воздействие
изобразительного искусства. А если будет девушка с веслом? А если будет шахтёр
в женском половецком платье? В чём истинная опасность: гендер или одежда, как
считаете?

\iusr{Ольга Сидорова Сидорова}
\textbf{Энрике Анатольевич Менендес} а разве православные ходят с наколками? Скрепненько так)))))

\iusr{Александр Лапин}
\textbf{Энрике Менендес} , насчёт отсутствия слобожанской идентичности, это ты зря. Харьковчане обидятся  @igg{fbicon.wink} 

\iusr{Городецкая Марина}
\textbf{Энрике Менендес},так называемый народ - "донбассята"? это донецкий этнос?

\iusr{Serhiy Yaryhin}
\textbf{Энрике Анатольевич Менендес}
Татуювання - це сильна ознака ідентичності  @igg{fbicon.wink} 
Може ще купола згадаємо?

\iusr{Сергий Федорынчык}

Ах, Енрике! Для людей справді культурних, а такі є у всіх краях Украіни, і на
Донбасі також, татуювання є проявом кримінальноі субкультури.

\iusr{Игорь Рязнов}
\textbf{Сергий Федорынчык} тото сейчас все в США в наколках ходят)))

\iusr{Игорь Рязнов}
\textbf{Городецкая Марина} 

ровно 100 лет назад так же смеялись над украинским и белорусским этносом. А
коммунисты создали Союз включили в состав УССР и БССР в паспорта написали
национальности и уже никто не смеётся, а считают себя щирими украинцами. Так
что история развивается по спирали

\iusr{Aleksej Vasiliev}
\textbf{Энрике Менендес} А я помню "Все будет Донбасс". И умиления это не вызывало.

\iusr{Dmytro Rybakov}
\textbf{Энрике Анатольевич Менендес} тобто дівчину треба було щ татухою в качалці намалювати і все було б "ідентично"?

\iusr{Halyna Pry}
\textbf{Энрике Анатольевич Менендес} а, так і запишемо: наколки на тілі - це ідентичність. Слава Богу, в Слобожанщині дійсно такої ідентичності ніколи не було, бгг.

\iusr{Ievgen Nefediev}
\textbf{Энрике Анатольевич Менендес} 

єбать ти баклан. Якщо вся твоя донбаська ідентичність - це татуювання у зеків,
то твій діагноз зрозумілий. І для довідки щодо ідентичності слобожан -
харків'яни і досі пам'ятають, що Харків був столицею України, а їхній
університет було засновано на початку 19 століття, коли Юз ще не народився,
тому завали. Який пиздець у твоїй голові.

\end{itemize} % }

\iusr{Сергей Прокопенко}
На самом деле мурал должен называться: "Гуцулка Ксеня шукає 10 відмінностей териконів від лисих Карпат..."

\begin{itemize} % {
\iusr{Сергей Сидоров}
\textbf{Сергей Прокопенко} тож і є саме терікони від ч. моря певно.

\iusr{Svitlana Kovalova}
\textbf{Сергей Прокопенко} та где там Западной до авдеевских красот

\iusr{Сергей Прокопенко}
\textbf{Svitlana Kovalova} та конечно... особенно сердце кровью обливается, когда вспоминаешь, как паньство в щахту спускается...

\iusr{Сергий Федорынчык}
В Червоноградв та Нововолинську панство в шахти спускається, в Бориславі та Дашаві нафту та газ видобуває.

\iusr{Сергей Прокопенко}
\textbf{Сергий Федорынчык} і шо там, є мурали донбаських жіночок?) чи 'носіі етнокультури' так справляються?
\end{itemize} % }

\iusr{Сергій Борис}

Яка крихка, однако, ця донбаська ідентичність, якщо мурал "Слобожаночка" здатен
її знищити.

\begin{itemize} % {
\iusr{Энрике Менендес}
\textbf{Сергій Борис} Какая хрупкая украинская идентичность, что признание русского вторым государственным способно её уничтожить

\iusr{Сергій Борис}
\textbf{Энрике Анатольевич Менендес} Давайте розглянемо таку можливість, чому ні. Одразу після звільнення ерефією українських територій, сплати репарацій, публічного каяття за геноцид, репресії і окупацію та офіційного засудження й відмови від злочинної імперської ідеології "руського міра".

\iusr{А. Полтавцев}
\textbf{Энрике Анатольевич Менендес} т.е. донбасская идентичность на самом деле рузкая?  @igg{fbicon.smile}  Труба коксохима это более правильный символ чем медведь с балалайкой? А что Вы ещё знаете о шизофрении?

\iusr{Ольга Сидорова Сидорова}
\textbf{Энрике Анатольевич Менендес} о украинофобиия поперла

\iusr{Станислав Печенкин}

Крихка? Так крихка. А чому крихка? Хто доклав до цього руку? Та той ( чи та
) хто і окупував частину Донбасу. Тобто Росія. Тому нашу мову та культуру
треба відроджувати і всіляко підтримувати. А визнання другою державною
російської мови цьому процесу сприяти аж ніяк не буде.


\iusr{Ольга Сидорова Сидорова}
\textbf{Andriy Poltavtsev} налейте ещё \textbf{Энрике Анатольевич Менендес} стакан водки и он раскроется по полной, начнёт рассказывать как в австрийском генштабе придумали украинский язык, как граф Потоцкий придумал украинцев.

\iusr{Сергий Федорынчык}

Так, Енрике, усі культури та мови в усіх частинах колишньоі Російськоі Імперіі
крихкі та вразливі перед натиском брутального "русского мира" з матюками. Для
жителів Бельгіі дві державних мови означають обовязковість володіння іншою, ніж
своя. Валони володіють фламандською, фламандці - французькою. Практично всі
бельгійці також володіють німецькою та англйською, а культурна людина в Бельгіі
починається з 5-оі мови. Для носіів "русского мира" вимога російськоі як другоі
державноі потрібна лише для того, щоб не вчити інших мов і тихо вважати іх
носіів унтерменшами -"чурками", "хохлами", "бульбашами", "татарвою",
"нациками". На жаль, більшість жителів Донбасу, особливо тих, що залишились на
непідконтрольних територіях, недостатньо знайомі ні з украінською мовою та
культурою, ні з російською. Натомість надзвичайно поширені татуювання та
матюки. Взагалі, Енрике, прошу вибачити за різкість, поки Ви не оволоділи
іспанською, Вам, на мою думку, було б краще не зачіпати делікатну тему
ідентичності. Вважаю, що Вам зараз варто згорнути обговорення, яке найбільше
розбурхує саме тих людей, які найменше до нього готові.

\iusr{Aleksej Vasiliev}
\textbf{Энрике Менендес} Украинская идентичность и правда хрупкая при русском втором. Это трудно не понимать. Вы неудачно ввязались в тему, явно, на мой взгляд, для вас проигрышную. И вам же вредную.

\end{itemize} % }

\iusr{Михайлов Олег}
Энрике, Вы умничка!

\begin{itemize} % {
\iusr{Михайлов Олег}
Сначала люди смеются, потом говорят - может быть, а потом - так это общеизвестно)

\iusr{Марк Зоряний}
\textbf{Михайлов Олег} хуюмничка он, разжигатель розни в обществе. Такие как этот глист в обмороке и стали причиной войны , предавши Украину и звавшие войска Хуйла

\iusr{Михайлов Олег}
\textbf{Марк Зоряний}
Ваш мат вызывает рознь между интеллигентными людьми и ...

\iusr{Городецкая Марина}
\textbf{Михайлов Олег},а ты недоумок

\iusr{Михайлов Олег}
\textbf{Городецкая Марина} Да уж, достойный уровень общения(
\end{itemize} % }

\iusr{Евгений Николаевич}

Это называется "долбаебам нет покоя!!!", в городе эпидемия, нет врачей... Очень
жаль мой город...

\begin{itemize} % {
\iusr{Svitlana Kovalova}
\textbf{Евгений Николаевич} а местная власть есть?

\iusr{Ievgen Nefediev}
\textbf{Евгений Николаевич} Правильно! А замість того, щоб писати про це, Менендес питає у Магомедова, чого мурал цей намалювали.
\end{itemize} % }

\iusr{Дмитрий Золотухин}

Вы бы так свой родной регион от захватчиков защищали в 2014-м  @igg{fbicon.smile} 
Цены б вам не было

\begin{itemize} % {
\iusr{Энрике Менендес}
\textbf{Дмитрий Золотухин} Я защищал. Где Вы были в этот момент не ясно

\iusr{Olga Kosse}
\textbf{Дмитрий Золотухин}, о, диванные защитники подъехали. Шо там по перепостам? Всех победили?)

\iusr{Vladislav Baturskiy}
\textbf{Olga Kosse}, сами бы попробовали всех победить с этих диванов! Только критиковать горазды...

\iusr{Viktor Taran}
\textbf{Энрике Анатольевич Менендес} як конкретно ви захищали? ви пішли добровольцем в армію? якщо так - в яку?

\iusr{Алексей Павлов}
Петух Димочка Золотухин умасливал интимные отверстия Пороха и ко. Ему некогда было. Стыдно, Энрике, такое спрашивать!

\iusr{Olga Kosse}
\textbf{Viktor Taran}, а защищают только через армию? И что у вас по защите - такой гладенький ГОшник. Добробат? Контракт? Пицца для военных раз в месяц и откупились?))

\iusr{Андрей Колосов}
\textbf{Viktor Taran} 

Можно ответить за коллегу, поскольку я был в точно таком же положении в
Луганске. Мы вели проукраинский майдан с 30 ноября 2013 по конец апреля 2014.
Но уже с марта руководство страны пошли на сдачу наших городов, 6 апреля - СБУ
Луганска с оружием, потом управление милиции Славянска и аж...до 5 июля- сдача
Донецка. Сдавали и офицеры, как вы пишите, армии, не выполнявшие требований
Уставов несения военной службы - без боя сдавали военные городки.


\iusr{Olga Kosse}
\textbf{Vladislav Baturskiy}, я безусловно критикую всех, кто рассказывает о "защите родины" с точки зрения армии и войны. Все эти адепты войны откупаются постиками поддержки военным в Facebook. Ну, пиццу им оплачивают раз в месяц, поздравляют в праздники. Все лишь для того, чтобы им было комфортно каждый день возвращаться в свои тёплые кроватки, да по ресторанам ходить, а не в акопах жить.

\iusr{Юрий Лукшиц}
Не нужно было сносить неконституционно власть в Киеве. Глядишь, и Крым с Донбассом не нужно было бы защищать тогда.

\iusr{Сергій Борис}
Ще чого, тільки від муралів і художників)

\iusr{Michail Vidiassov}
\textbf{Энрике Менендес} 

Так разве не ровно за это Вы стояли с майдановцами ? Были бы Вы с местными
регионалами, пытавшимися фрондировать и торговаться с Киевом, то была бы
понятна позиция: боролись, проиграли, теперь критикуете установленные
победителями порядки. Но Вы же сделали сознательный и сравнительно
информированный выбор когда пришёл "момент истины" и что-то действительно
решалось кулаками. Не логично ли Вам "лупить себя по затылку"?

\iusr{коршунова татьяна}
\textbf{Дмитрий Золотухин} а вы, надеюсь, защищали? или как во времена Киевской Руси-каждый только свой родной регион должен защищать?

\iusr{Энрике Менендес}
\textbf{Viktor Taran} не вижу Ваш комментарий (

\iusr{Olga Kosse}
\textbf{Viktor Taran}, пропал вместе со своим комментарием. Видимо, донор не согласовал месседж.

\iusr{Viktor Taran}

\textbf{Энрике Анатольевич Менендес} тобто? він є

\iusr{Viktor Taran}
\textbf{Энрике Анатольевич Менендес}

\ifcmt
  ig https://scontent-frt3-1.xx.fbcdn.net/v/t39.30808-6/245038643_10158670196646859_1663325792435579103_n.jpg?_nc_cat=108&ccb=1-5&_nc_sid=dbeb18&_nc_ohc=sjkbVOPpoSMAX9g3mv3&_nc_ht=scontent-frt3-1.xx&oh=13f06b75e72bd1cd88cb7ea0e7bfbcd1&oe=61A52A50
  @width 0.3
\fi

\iusr{Настя Степанова}
\textbf{Дмитро Золотухін} , это очень смищно, прям очень))) Безотносительно Энрике. Семь лет прошло, уже можно было не только начальную школу-то закончить, а после седьмого класса уже можно быть способным и на элементарную аналитику. Но это я, конечно, по своим детям сужу)))

\iusr{Сергей Перевозчиков}
\textbf{Энрике Менендес} Да, где Вы были с 7 до 11?

\iusr{Марк Зоряний}
\textbf{Энрике Менендес} да ты там вонял на оккупированной территории , но тебя на подвал посадили, где фсб и завербовало

\iusr{Андрей Николаев}
\textbf{Дмитро Золотухін} Они их туда, звали

\iusr{Городецкая Марина}
\textbf{Viktor Taran},ответа не будет, не ждите...

\iusr{Вадим Щетинин}
\textbf{коршунова татьяна}, типа сейчас не так. Ну прям увесь Бандерстан встал под ружжо.

\iusr{Андрій Винниченко}
\textbf{Viktor Taran} Если он считает, что Украина на него напала, чтобы лишить его идентичности, то очевидно, что он в 2014 году защищал свою страну Донбасс, от злых украинских захватчиков. Как и сегодня защищает.

\iusr{Елена Кальненко}
\textbf{Настя Степанова} За 7 лет можно грамоте научиться, чтобы не было ,,смищно,, и безотносительно.

\iusr{Настя Степанова}
\textbf{Елена Кальненко} , ахахахах)))))))

\iusr{Halyna Pry}
\textbf{Энрике Анатольевич Менендес} ви не захищали. ви сиділи в ахметівських структурах та "гуділи", а потім корчили з себе "відповідальних громадян", якими ніколи не були. Давайте назад, гудіть в Донецьку, не в Україні.
Якщо у вас клепки не вистачає навіть на те, аби розумітися в ментальностях людей, що поряд за сотню км (Харків, Суми, північ Луганщини, та сама "слобожанська ідентичність", якої за вами нема), то що ви взагалі знаєте про Україну?

\end{itemize} % }

\iusr{Роман Вовченко}
"Слобожанка" на Рианну очень похожа

\iusr{Иван Шашлов}

У нас еще в городе полно "политических украинцев" и "приазовских сепаратистов",
которые любят подрассказать, что "Мариуполь - не Донбасс". Ну в некоторых
случаях еще пытаются приазовских греков на свою сторону привлечь. Это тогда,
когда они с греками за Мариуполь не воюют, рассказывая, что "вас тут не стояло,
вы пришли на готовый город, крепость Домаха, Кальмиусская паланка".

\begin{itemize} % {
\iusr{Николай Анфалов}
\textbf{Иван Шашлов} просто некоторые "товарищи" пытаются оправдать ту линию разграничения, которая есть. Причём пытались это сделать задолго (!) до войны. А потом и в непризнанных республиках начались идеи о том, чтоб оправдать эту же линию разграничения, что, мол, в Краматорск и Мариуполь - это" свидомая бандеровская Украина"... Как же эти господа работают по обе стороны на раскол, чтоб только удержать сегодняшнее положение дел в головах людей ((((

Иван Шашлов
Николай Анфалов ну это же просто очередные следы работы по вытравливанию донбасса и донбасской идентичности, откалывание по куску и создание множественных линий раздела и резмежевания, антагонизмов. Разделяй и властвуй.

Николай Анфалов
Иван Шашлов ну вот пару часов назад в одном из обсуждений россиянин упрекнул меня в том, что я живу в "свидомом краматорске"

\iusr{Иван Шашлов}
\textbf{Николай Анфалов} а он бы хотел, чтобы вы где жили?
Один мой действительно свидомый знакомый-порохобот мечтал жить в Веллингтоне, например. Как сейчас - не знаю, вроде по прежнему в ватном Мариуполе живёт.

\iusr{Николай Анфалов}
\textbf{Иван Шашлов} ну так он посчитал меня предателем, что не поверил в счастливую жизнь в ДНР и не терплю все тяготы и невзгоды ради России. А все, кому не нравится ДНР и кто не молится на Путина, для него - "свидомые".

\iusr{Иван Шашлов}
\textbf{Николай Анфалов} честных поехавших и людей не зарплате хватает везде, к сожалению

\iusr{Николай Анфалов}
\textbf{Иван Шашлов} этот из поехавших. Он сказал, что все знает о ситуации на Донбассе из форумов и начал рассказывать что и как у нас происходит. Хотя сам дале географию Донбасса не знает. Но герой и "патриот"

\iusr{Иван Шашлов}
\textbf{Николай Анфалов} нуштош. Вообще я верю, что кое-что можно реально знать и из форумов. Как и в то, что на форумах могут быть специально обученные люди, как раз чтобы таких вот правдой накачивать  @igg{fbicon.smile} 

\iusr{Николай Анфалов}
\textbf{Иван Шашлов} просто есть люди, сильно подверженые пропаганде. Они, сидя в РФ, даже не могут себе представить, как можно уехать из Донецка в Краматорск жить и работать.

\iusr{Иван Шашлов}
\textbf{Николай Анфалов} ну глупые, что сказать. Таких хватает.

\iusr{Николай Анфалов}
\textbf{Иван Шашлов} они не глупые, они просто реальности ситуации сете не представляют вообще.

\iusr{Анатолий Наседкин}
\textbf{Ivan Shashlov} взагалі то "приазовські греки, історик ви наш, це депортовані з Криму християни, депортовані ще навіть ДО анексії Криму Катькою-2. І це історичний факт, що так їх переселили саме на землі Запоріжжя.

\iusr{Евгений Афендиков}
Так верните крым грекам, чтоб не кому не достался

\iusr{Halyna Pry}
А відколи Маріуполь став Донбасом, там є шахти і вугілля? Донецька область - так, але це не вугільний Донбас.
\end{itemize} % }

\iusr{Марія Подибайло}

ну так, і якийсь Данілов, і Стяжкіна, і Казанський і багато інших не мають
відношення до регіону, мабуть. А от якби на муралі була галичаночка, то це
означало б, що хтось намагається "насадити галицьку ідентичність"?

і так, нашу Донеччину реально "руйнують з двох сторін"? Енріке, ви нічого не
плутаєте? Чому тоді неокуповані території краще розвиваються, ніж до війни?
Думаю, ви чітко розумієте, що руйнує ЛИШЕ РОСІЯ З ЇЇ КОЛАБОРАНТАМИ, РОЗВ'язавши
війну! А ЗСУ обороняють свою територію від російської агресії!

Впевнена, що ви точно розумієте й те, що якби не було російської агресії, ЗСУ
не було б необхідності бути присутніми там, де вони є.

Ми ж, загалом, всі прекрасно розуміємо куди ви хилите своїми дописами та
інтерв'ю. Вас дурнем не назвеш. Тут багато інших визначень доречні...

\end{itemize} % }
