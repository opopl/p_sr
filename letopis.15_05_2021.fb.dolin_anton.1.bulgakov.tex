% vim: keymap=russian-jcukenwin
%%beginhead 
 
%%file 15_05_2021.fb.dolin_anton.1.bulgakov
%%parent 15_05_2021
 
%%url https://www.facebook.com/adolin3/posts/10224234518066862
 
%%author 
%%author_id 
%%author_url 
 
%%tags 
%%title 
 
%%endhead 

\subsection{БУЛГАКОВ 130}
\label{sec:15_05_2021.fb.dolin_anton.1.bulgakov}
\Purl{https://www.facebook.com/adolin3/posts/10224234518066862}

Многие решили отметить(ся), все-таки дата. Напишу свои пару строк. 

Ко мне в тинейджерстве - мои 11-12-13 лет - пришли три писателя, которых я
сразу записал в лучшие и любимые: Булгаков, Набоков, Платонов. Так и осталось
на всю жизнь (ну, по части русской литературы ХХ века). 

Выбрать из них \enquote{лучшего} никогда бы не смог. Но. Набоков все-таки - литература
для эстетов, за исключением одного популярного и мощного мифа, \enquote{Лолиты}; с
годами я сильно к нему охладел, хотя по сей день мало что знаю мощнее романа
\enquote{Подвиг}. Платонов - несравнимый, кристальный гений языка и почти фольклорного
мышления, но его экзистенция непереводима и потому на свой лад герметична. 

А вот Булгаков потрясает свободой литературного гения, которая, среди прочего,
в отсутствии той безупречной органики, что есть у Платонова, и той
бриллиантовой выделки, которой славен Набоков. Когда я взялся вновь за \enquote{Мастера
и Маргариту} лет 10 назад, был в шоке от того, как много там \enquote{неправильно},
\enquote{плохо} написанных строк и страниц. Будто невнимательный редактор просмотрел.
Но сила его письма такова, что кажется, будто он щеголяет своей небрежностью,
может себе ее позволить. Поскольку другие метафоры, образы, герои, сюжетные
повороты бьют наповал и заставляют пропустить неудачное (как недавно я и у
Толстого нашел просто ошибку, а-ля "проезжая станцию, с меня слетела шляпа", но
хмыкнул и пошел дальше - ему можно).

И персонажи Булгакова живые. Практически всегда и все. Все-таки Лужин, Гумберт
или Мартын Эдельвейс - или Саша Дванов с Вощевым - это бесплотные тени в
сравнении не только с Мастером, Коровьевым или Пилатом, но даже каким-нибудь
Степой Лиходеевым или Фридой с ее платком. Лариосик живой. Хлудов живой.
Профессор Преображенский на бумаге гораздо более живой, чем он же в исполнении
Евстигнеева. 

Булгаков для меня это прежде всего \enquote{Собачье сердце}, \enquote{Дни Турбиных}, \enquote{Бег},
\enquote{Жизнь господина де Мольера} и, понятное дело, \enquote{Мастер и Маргарита}. Думаю,
главный русский роман ХХ века. Самый читаемый (хотя так долго запрещенный),
самый долго-читаемый. И лучший роман о Москве, о самой ее природе. Булгаков
сделал для Москвы то, что Гоголь, Достоевский и Белый - для Петербурга. 

Этой книгой сформирован мой взгляд на литературу (прозу), в которой обязано
быть рядом низменное и возвышенное, бредовое и узнаваемо-реальное, мистическое
и будничное. Таков на всю жизнь мой эталон, в любую сторону, будь то
Рабле-Сервантес-Стерн-Гоголь, будь то Сорокин-Тартт-Уэлш-Мур (Алан). 

Презираю популярный взгляд на переоцененность Булгакова и его \enquote{Мастера}, в
точности как аналогичные \enquote{разоблачения} Довлатова. Говорящие и пишущие такое не
понимают самой сути литературы как искусства (по меньшей части, какой-то очень
важной ее части). 

С 130-летием всех нас.

Dmitry Front

именно эти три писателя у меня тоже так и остались любимыми..

Игорь Слуцкий

Полностью согласен по Булгакову. \enquote{Мастера} знаю почти наизусть, считаю его
великим романом. Почти согласен по Набокову. Но \enquote{Лолита} это тоже эстетская
литература (\enquote{Подвиг} не читал).

Платонова читать не могу. Очень вязкий язык, не могу через него продраться.

А тот, кого нельзя называть во избежание большого холивара, и который считает
переоцененным Довлатова, а \enquote{Мастера} написанным для тирана, по-моему просто
выпендривается. Он слишком много читал, и у него слишком хорошо подвешен язык.
Простим ему его слабости, у кого их нет.

Marina Konstantinova

Процентов на 90 от этого текста - мои солидарные мысли и чувства.

Больше, чем гений- дальновидный пророк, знавший о нас больше нас самих.

Великий. Любимый. Навсегда.

Анна Кифенко

Помню, как вкушала просто - таки \enquote{Мастера и Маргариту} ночью, укрывшись с
головой одеялом, чтоб не застукали Антон, а как думаете, почему зацепило всех
- всех, детей, стариков, слесарей и настройщиков фортепиано? Что такого этим
романом сделал Булгаков?

Andrei Sen-Senkov

Про то, что он самый московский, абсолютная правда. Не было ни одного знакомого
иностранца, у которого не было бы пожелания номер 1 погулять по
мастеримаргаритовской Москве

Alisa Zavenyagina

Как прочитала «Мастера» в 16 лет, так он и остаётся главной книгой в моей
жизни, и я не боюсь показаться ограниченной. С днём рождения любимого писателя
нас!

Галина Комболина

Подписываюсь. Это был настоящий шок, обвал, когда впервые прочла Мастера.
Правда, Довлатова, которого знала читала в рукописи еще ( дружила с ним)
перечитываю только с большими перерывами. Иначе не входит...

Katerina Katerina

Недавно слушала Собачье сердце (перечитывать не всегда есть время) ... с каждым
годом опять все актуальнее и актуальнее... производство шариковых и швондеров
поставлено на конвейер

Natalia de Heer

Прекрасный пост! Антон, вас всегда так замечательно читать и слушать. Огромное
спасибо, что напомнили про 130 лет любимому Михаилу Булгакову

Andrey Ovchinnikov

И снова мы по вкусам совпали, как и с \enquote{Бесами} и с Кирой Муратовой и еще
чего-то, не помню уже. Может поэтому все рекомендованные вами фильмы так мне
\enquote{зашли}.

Konstantin Knude

Господи, как же я обожаю Михаила Афанасьевича Регулярно перечитываю
Театральный Роман, и выражаясь его же собственными словами \enquote{от наслаждения
выступает на лбу мелкий пот.}

Roman Mitsyuk

Да, это точно, морщить нос при упоминании \enquote{Мастера и Маргариты} - это какой-то
распространенный шик. Непонятно только, для снобов или не очень образованных
людей.

Читал несколько раз взахлёб ещё в школьные годы. Без всяких рекомендаций со стороны и без последующей бравады.

Каждый там может что-то своё найти. Шикарно! Присоединяюсь ко всему сказанному.

Oleg Kruchinin

Все созвучно про Булгакова. Он ошарашил просто в 11 классе и далее уже, его
можно открывать и открывать снова и так бесконечно. Например, попался мне
сборник \enquote{Князь тьмы} совсем недавно, и там все рассказы и повести \enquote{Черный маг},
\enquote{ Копыто инжинера}, \enquote{Великий канцлер}, \enquote{Золотое копье}, \enquote{ Князь тьмы}...и опять
ты видишь, как шла эволюция мысли, как образы создавались...очень интерсно, что
конечно снова приводит к желанию перечитать \enquote{Мастера и Маргариту}

Zherebtsova Polina

Написала Вам в личку, посмотрите) Булгакова тоже прочитала в 10 лет! Люблю все
его тексты.

Leonid Mazor

Есть у Булгакова одна цитата, в которой, собственно и заключена и его жизнь, и его отношение к творчеству...

\enquote{На широком поле словесности российской в СССР я был один-единственный
литературный волк. Мне советовали выкрасить шкуру. Нелепый совет. Крашеный ли
волк, стриженый ли волк, он все равно не похож на пуделя. Со мной и поступили
как с волком. И несколько лет гнали меня по правилам литературной садки в
огороженном дворе. Злобы я не имею, но я очень устал. Ведь и зверь может
устать. Зверь заявил, что он более не волк, не литератор. Отказывается от своей
профессии. Умолкает. Это, скажем прямо, малодушие. Нет такого писателя, чтоб он
замолчал. Если замолчал, значит, был не настоящий. А если настоящий замолчал -
погибнет...}
