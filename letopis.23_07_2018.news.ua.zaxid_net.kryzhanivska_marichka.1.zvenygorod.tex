% vim: keymap=russian-jcukenwin
%%beginhead 
 
%%file 23_07_2018.news.ua.zaxid_net.kryzhanivska_marichka.1.zvenygorod
%%parent 23_07_2018
 
%%url https://zaxid.net/pid_lvovom_vidkriyut_istoriko_kulturniy_park_drevniy_zvenigorod_na_mistsi_knyazhoyi_stolitsi_n1462089
 
%%author Крижанівська, Марічка
%%author_id kryzhanivska_marichka
%%author_url 
 
%%tags zvenigorod,ukraina,russia
%%title Під Львовом відкриють історико-культурний парк «Древній Звенигород» на місці княжої столиці
 
%%endhead 
 
\subsection{Під Львовом відкриють історико-культурний парк «Древній Звенигород» на місці княжої столиці}
\label{sec:23_07_2018.news.ua.zaxid_net.kryzhanivska_marichka.1.zvenygorod}
\Purl{https://zaxid.net/pid_lvovom_vidkriyut_istoriko_kulturniy_park_drevniy_zvenigorod_na_mistsi_knyazhoyi_stolitsi_n1462089}
\ifcmt
  author_begin
   author_id kryzhanivska_marichka
  author_end
\fi

\index[cities.rus]{Звенигород!Южная Русь!У селі Звенигороді облаштують історико-культурний парк «Древній Звенигород», 23.07.2018}

\emph{У Звенигороді також відкрили виставку викопаних за 100 років скарбів}

\ifcmt
pic https://zaxid.net/resources/photos/news/605x340_DIR/201807/1462089.jpg?201810171631
\fi

У селі Звенигороді облаштують історико-культурний парк «Древній Звенигород».
Колись княжа столиця, а нині село під Львовом хоче привернути туриста і
відновити оборонні місця. Проект виграв грант Євросоюзу.

Русь, XI століття, Ростиславичі утверджуються на західних землях. Старший син
Рюрик сів у Перемишлі, молодший — Василько у Теребовлі, середущий — Володар — у
Звенигороді, де й заснував столицю однойменного князівства. Найбільший розквіт
міста припав на часи Володимирка, батька Ярослава Осмомисла. Звенигород
припинив своє існування у 1241 році, коли монголи спалили укріплення дотла.

Нині про величну історію нагадують артефакти, які лежать просто під ногами, і
вали. Тому науковці та місцева громада надумали відкрити тут
Історико-культурний парк «Древній Звенигород».

«Дуже важко показати, що зникло, але, дякуючи долі, ця територія має сліди тої
колишньої слави, це є оборонні вали, це є замкові бастіони початку 18
сторіччя», ‒ розповідає архітектор-реставратор, доцент «Львівської політехніки»
Василь Петрик.

«Ми запропонуємо туристам пройти лінією валів, долиною ріки Білки, де колись
були майстерні ювелірні, господарські будівлі, ця територія активно заселялася,
а всередині валів ‒ територія княжого міста», ‒ додає Василь Петрик.

Стрімкою стежкою уже нині є можливість піднятися на бастіони, княжий дитинець,
де стояли палати князя, храм із цвинтарем, господарські будівлі і оборонні
укріплення.

Ідею проекту такого парку подала сільська рада, а уже область написала і
виграла грант від Євросоюзу ‒ 12 мільйонів гривень. За ці гроші розчистять
вали, місце замку, облаштують доріжки і навколишню інфраструктуру у селі: місця
для відпочинку, освітлення, паркінг.

«Ми дуже чекаємо на те, щоб прийшли кошти і почали робити. Цей проект
розрахований на два роки, кошти підуть траншами. А для нашого Звенигорода це
взагалі буде чудово, тому що це є робочі місця, це є популяризація нашого
села», ‒ коментує голова Звенигородської сільради Наталія Толопко.

\ifcmt
  tab_begin cols=3
    caption Виставка «Стольний град Звенигород ‒ повернення із небуття» (фото Віталія Грабара)

    pic https://zaxid.net/resources/photos/news/201807/1462089_1517094.jpg?201807231834&fit=cover&h=540&q=85
    pic https://zaxid.net/resources/photos/news/201807/1462089_1517095.jpg?201807231834&fit=cover&h=540&q=85
    pic https://zaxid.net/resources/photos/news/201807/1462089_1517096.jpg?201807231834&fit=cover&h=540&q=85

    pic https://zaxid.net/resources/photos/news/201807/1462089_1517097.jpg?201807231834&fit=cover&h=540&q=85
    pic https://zaxid.net/resources/photos/news/201807/1462089_1517098.jpg?201807231834&fit=cover&h=540&q=85
    pic https://zaxid.net/resources/photos/news/201807/1462089_1517099.jpg?201807231834&fit=cover&h=540&q=85

    pic https://zaxid.net/resources/photos/news/201807/1462089_1517100.jpg?201807231834&fit=cover&h=540&q=85
    pic https://zaxid.net/resources/photos/news/201807/1462089_1517101.jpg?201807231834&fit=cover&h=540&q=85
  tab_end
\fi

А наразі у місцевому музеї відремонтували один зал і відкрили цілком нову
тимчасову виставку із скарбів, викопаних у Звенигороді упродовж 100 років.
Показують могутність княжої столиці зброєю, прикрасами, посудом.

«Ми вибирали на виставку речі, які би свідчили про те, наскільки була розвинута
матеріальна культура. На одній із вітрин ‒ біжутерія, величезна кількість
прикрас, зокрема, зі скла, а скло в ті часи це було досить дороге задоволення»,
‒ розповідає наукова співробітниця Рятівної археологічної служби Наталія
Войціщук.

Завдяки болотистій місцевості, збереглося чимало елементів дерев'яної забудови

А ще фахівці із «Львівської політехніки» створили 3D-модель замку і княжого
Звенигорода.
