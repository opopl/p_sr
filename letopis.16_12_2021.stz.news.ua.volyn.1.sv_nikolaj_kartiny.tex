% vim: keymap=russian-jcukenwin
%%beginhead 
 
%%file 16_12_2021.stz.news.ua.volyn.1.sv_nikolaj_kartiny
%%parent 16_12_2021
 
%%url 
 
%%author_id 
%%date 
 
%%tags 
%%title На Волині створили понад 200 картин до Дня Святого Миколая (Фото, Відео) 
 
%%endhead 
\subsection{На Волині створили понад 200 картин до Дня Святого Миколая (Фото, Відео)}
\label{sec:16_12_2021.stz.news.ua.volyn.1.sv_nikolaj_kartiny}

\begin{multicols}{2} % {
\setlength{\parindent}{0pt}

\ii{16_12_2021.stz.news.ua.volyn.1.sv_nikolaj_kartiny.pic.1.header}

У Луцькій художній школі організували конкурс дитячого малюнку «Місто Святого
Миколая».

Діти у своїх роботах зображали не лише чудотворця, а й загалом зимові свята,
національні традиції, українську родину, дитячі мрії та побажання, Луцьк і
лучан.

Загальноміський конкурс «Місто Святого Миколая» – традиційний. Його
організовують третій рік поспіль напередодні 19 грудня – перед святом
покровителя обласного центру. 

– Маємо прекрасну нагоду поєднати свято Миколая з нашим містом. Те, що він є
патроном Луцька, покладає певну відповідальність на дітей і додає родзинку до
творчого процесу, – зазначив директор художньої школи Іван Гаврилюк. 

\begin{zznagolos}
Кожна робота особлива по-своєму  і ніколи не повторюється.	
\end{zznagolos}

Усі малюнки на виставці – різноманітні, кожен – особливий, з унікальним
сюжетом.

– Як кожна мить неповторна, так і кожне дійство, і робота. Тим паче, що це
свято незвичайне. Воно – наше, національне, близьке для дітей, вони цілий рік
чекають його і ми повинні відповідально до цього ставитися, – додав Іван
Миколайович.

\end{multicols} % }

\href{https://www.youtube.com/watch?v=-sNTIiYoIRg}{%
Святий Миколай завітав до Луцької художньої школи, youtube, 16.12.2021%
}

\ii{16_12_2021.stz.news.ua.volyn.1.sv_nikolaj_kartiny.scr.1}

\begin{multicols}{2} % {
\setlength{\parindent}{0pt}

\begin{zznagolos}
На малюнках зображені дитячі мрії, Луцьк – так, як уявляють і мріють про те,
яким має бути наше місто. 	
\end{zznagolos}

Учасниця конкурсу Анна Осадча подала на конкурс роботу «Прогулянка з Миколаєм».
Дівчина розповіла, що до написання картининадихнула її пригода з друзями, коли
вони випадково потрапили на свято, і там побачили Святого Миколая.

– Ми біля нього бігали, дуже тішилися. Я вирішила це зобразити у своїй роботі,
– поділилася Анна. 

Учні школи зображають Луцьк таким, яким хочуть його бачити.  

– Конкурс спонукає до мислення про місто майбутнього, місто Святого Миколая. Це
такий наш бренд, – зазначила керівниця департаменту культури Тетяна Гнатів.

У творчому змаганні взяли участь учні художньої школи, загальноосвітніх
закладів Луцька та Княгининівської школи мистецтв. 16 грудня журі підбило
підсумки і визначило переможців, яких нагородили дипломами та призами у вікових
категоріях 7-9, 10-13, 14-17 років. Крім того, було створене учнівське журі,
яке також визначало переможців. 

На конкурсі було представлено 210 робіт. Роботи переможців експонуватимуться з
17 грудня в арт-галереї «Луцьк».

\end{multicols} % }

\ii{16_12_2021.stz.news.ua.volyn.1.sv_nikolaj_kartiny.pic.2}
\ii{16_12_2021.stz.news.ua.volyn.1.sv_nikolaj_kartiny.pic.3}
\ii{16_12_2021.stz.news.ua.volyn.1.sv_nikolaj_kartiny.pic.4}
\ii{16_12_2021.stz.news.ua.volyn.1.sv_nikolaj_kartiny.pic.5}
