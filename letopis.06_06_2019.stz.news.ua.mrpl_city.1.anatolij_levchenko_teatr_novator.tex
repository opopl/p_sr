% vim: keymap=russian-jcukenwin
%%beginhead 
 
%%file 06_06_2019.stz.news.ua.mrpl_city.1.anatolij_levchenko_teatr_novator
%%parent 06_06_2019
 
%%url https://mrpl.city/blogs/view/anatolij-levchenkoteatralnij-novator-mariupolya
 
%%author_id demidko_olga.mariupol,news.ua.mrpl_city
%%date 
 
%%tags 
%%title Анатолій Левченко – театральний новатор Маріуполя
 
%%endhead 
 
\subsection{Анатолій Левченко – театральний новатор Маріуполя}
\label{sec:06_06_2019.stz.news.ua.mrpl_city.1.anatolij_levchenko_teatr_novator}
 
\Purl{https://mrpl.city/blogs/view/anatolij-levchenkoteatralnij-novator-mariupolya}
\ifcmt
 author_begin
   author_id demidko_olga.mariupol,news.ua.mrpl_city
 author_end
\fi

\ii{06_06_2019.stz.news.ua.mrpl_city.1.anatolij_levchenko_teatr_novator.pic.1}

Режисера і сценографа \textbf{Анатолія Миколайовича Левченка} можна вважати справжнім
театральним реформатором Маріуполя, адже він завжди прагне до нового,
незвіданого, не боїться руйнувати стереотипи і вибудовувати нову художню
реальність - відкриту, чесну, безкомпромісну. Завдяки творчості Анатолія
Миколайовича театральна культура Маріуполя отримала потрібний поштовх для
розвитку, чому посприяли як українські вистави режисера, так і його сміливі
спроби у виставах знайти відповіді на гострі та актуальні запитання.

\ii{insert.read_also.demidko.zabavin}

Народився Анатолій в м. Семипалатинськ (Казахстан). Його дід був розкуркулений
і засланий туди з Кіровоградської області ще в довоєнні роки. Батьки режисера
мають українські коріння. Завдяки матері, яка працювала викладачем літератури у
педагогічному інституті, Толя з братом мали можливість багато читати і всебічно
розвиватися. Анатолій Миколайович підкреслив, що його мама була \enquote{прихованим
дисидентом}, читала багато самвидаву і виписувала безліч журналів (виходило
десь на 500 рублів на рік, в той час як її зарплатня на місяць складала 110
рублів). Анатолію з дитинства була властива деяка артистичність. Закінчив
середню та музичну (скрипка) школи у 1986 році. У роки служби на Балтійському
та Північному флоті у нього вже сформувалася тверда впевненість в тому, що
після служби він вступатиме до театрального інституту.

\ii{06_06_2019.stz.news.ua.mrpl_city.1.anatolij_levchenko_teatr_novator.pic.2}

У 1989 році Анатолій Миколайович вступив до Київського національного
університету театру, кіно і телебачення імені І. К. Кар\hyp{}пенка-Карого та в 1994
році закінчив його з присвоєнням кваліфікації спеціаліста за спеціальністю
\enquote{режисура}. Цікаво, що в цьому ж інституті Анатолій додатково навчався в
лабораторії сценографії \textbf{\emph{Михайла Френкеля}} – заслуженого художника України, тоді
– головного художника в Театрі імені Лесі Українки. Викладання Михайла
Адольфовича надихнуло режисера на докладне вивчення сценографії, адже він
вийшов за межі звичайної програми, що, безумовно, знадобилося йому в
майбутньому. Анатолій почав ставити вистави з 3 курсу, адже був дуже
обдарованим і здібним студентом. Під час навчання Анатолій поїздив по різним
фестивалям, стажувався в Миколаївському, Луганському та Чернівецькому театрах.
Часто їздив переглядати кращі вистави до Москви.

\ii{06_06_2019.stz.news.ua.mrpl_city.1.anatolij_levchenko_teatr_novator.pic.3}

У Маріуполі залишився завдяки особливій симпатії до міста, особливо сподобалося
море. Також на рішення залишитися вплинуло бажання \enquote{молодої крові} тогочасного
директора театру \emph{\textbf{Котова Олександра Миколайовича}}. Так, Анатолій Миколайович у
1994 році почав працювати режисером і сценографом в Донецькому академічному
обласному драматичному театрі (м. Маріуполь). Здається, творча енергія Анатолія
Левченка ніколи не вичерпається. За час своєї діяльності він вже встиг
поставити 75 вистав. Найвизначніші його постановки мають насичену творчу
енергію, авторську режисуру, чітке визначення художнього потенціалу, на
маріупольській сцені він створив цілу галерею різноманітних вистав сучасного та
класичного репертуару, серед яких: \enquote{Трьохгрошова опера} Б. Брехта, \enquote{Вишневий
сад} А. Чехова, \enquote{Лоліта} Е. Олбі, В. Набокова, \enquote{Маленький принц} А. Де-Сент
Экзюперi, \enquote{Аліса в Країні Див} Л. Керрола, \enquote{Пейзаж} Г. Пінтера, \enquote{Нiч перед
Рiздвом} М. Гоголя, \enquote{За зачиненими дверима} Ж.-П. Сартра, \enquote{Острів скарбів} за
Р.\ Стівенсоном, \enquote{Останній подвиг Ланцелота} за Є. Шварцем та багато інших.

\ii{06_06_2019.stz.news.ua.mrpl_city.1.anatolij_levchenko_teatr_novator.pic.4}

Водночас сценографія Анатолія Миколайовича відрізняється вишуканістю в фарбах,
формах, грою світла і миттєвою трансформацією декорацій. Наш герой неодноразово
поєднував основну діяльність з роботою художника-постановника і завідувача
ху\hyp{}дожньо-постановочної частини у театрі. Як художній керівник курсу і педагог з
акторської майстерності з 2010 року веде курси в Першій Маріупольській
театральній школі-студії.

\textbf{Читайте також:} \emph{Мариуполь украсят на манер Парижа и Бордо?}%
\footnote{Мариуполь украсят на манер Парижа и Бордо?, mrpl.city, 04.06.2019, \par%
\url{https://mrpl.city/news/view/mariupol-ukrasyat-na-maner-parizha-i-bordo-foto}%
}

Лауреат і дипломант міжнародних і регіональних театральних фестивалів
\enquote{Херсонеські ігри}, \enquote{Мельпомена Таврії}, \enquote{Дні Чехова в
Ялті}, \enquote{Театральний Донбас}, \enquote{Homoludens},
\enquote{Андріївськийфест}. Сценограф став учасником Другої обласної виставки
художників театру \enquote{LaternaMagica} (Донецьк, 1997 р.). Також представив
персональну виставку \enquote{Портрет сценографа Анатолія Левченка}, що
проходила у 2009 році у виставковому залі сучасного мистецтва ім. А. Куїнджі.

\ii{06_06_2019.stz.news.ua.mrpl_city.1.anatolij_levchenko_teatr_novator.pic.5}

Маріупольці знають Анатолія Миколайовича як відповідального та свідомого
громадянина, справжнього патріота своєї країни. Він - учасник
\enquote{Революції на граніті} в буремних 90-х. Не стояв осторонь і на обох
маріупольських майданах - в 2004-му і 2014-му. У 2017 році був нагороджений
медаллю \enquote{За жертовність та любов до України} УПЦ (КП). У тому ж році
(до вересня 2018 року) виграв конкурс та працював Головним режисером в
Донецькому академічному обласному драматичному театрі (м. Маріуполь).

\ii{06_06_2019.stz.news.ua.mrpl_city.1.anatolij_levchenko_teatr_novator.pic.6}

Двічі одружений. З першою дружиною, актрисою театру і кіно \emph{\textbf{Еллоною Чернявською}}
розлучилися добрими друзями. Цікаво, що це саме вона надихнула режисера на
українські теми, хоча сама сьогодні живе в Москві. У режисера два сина: \emph{\textbf{Андрій
і Артем}}. Старший працює звукорежисером в московському театрі. Дружина Анатолія
\emph{\textbf{Анна}} цілком і повністю підтримує його у всьому. Режисер підкреслює, що без
підтримки і розуміння коханої дружини не зміг би реалізувати все заплановане. У
Маріуполі Анатолій Миколайович любить місця, сповнені живою історією:
Гамперовський спуск, Слобідка, закинуті парки. Любить гуляти і в Приморському
парку. Надихає читання. Для Анатолія Левченка професія – це і хобі, і
задоволення. Вважає, що сучасний театр в Маріуполі – це театр соціального
звучання, що вимагає більш відповідального ставлення до всіх постановок.

Відомо, що авторитет режисера завойовується не проголошеними гаслами, а творчою
працею. Своїми постановками Анатолій Миколайович вже давно заслужив звання
режисера-експеримента\hyp{}тора. Восени цього року митець планує відкрити свій театр
у Маріуполі – \textbf{\emph{\enquote{Terraincognita: свій театр. для своїх}}}, який стане новим
майданчиком для найбільш унікальних і оригінальних рішень режисера. Також
відбудеться виставка сценографічних робіт і представлений буклет
Левченка-сценографа.

\ii{06_06_2019.stz.news.ua.mrpl_city.1.anatolij_levchenko_teatr_novator.pic.7}

\textbf{Улюблена книга:} Читає дуже багато, віддає перевагу розумній поезії,
зокрема любить вірші Юрія Андруховича, Ліни Костенко, Йосипа Бродського, Бориса
Пастернака. Окреме місце в читанні займають Володимир Набоков, Альбер Камю,
Франц Кафка, Микола Гоголь.

\ii{06_06_2019.stz.news.ua.mrpl_city.1.anatolij_levchenko_teatr_novator.pic.8}

\textbf{Улюблені фільми:} \enquote{Форест Гамп}, \enquote{Куди приводять мрії}.

\textbf{Курйозний випадок з життя:} 

У 1992 році Анатолій Миколайович проходив стажування у чернівецькому театрі,
адже дуже хотів поставити вистави українською мовою. І все ж завдяки неабиякій
наполегливості та цілеспрямованості у нього вийшло потрапити до Чернівців.
Ставив класику – Михайла Старицького. Але один актор, вже літнього віку, Вадим
Іванович Нечипоренко, приходив до молодого режисера з великою книгою і неначе
слідкував, щоб режисер не відходив від оригінального тексту. Анатолій
Миколайович через це дуже нервував. Так продовжувалося кожну репетицію. На
останній – молодий режисер не витримав і подякував акторові, що той так
ретельно стежить за його роботою. Але коли звертався до актора, не отримав
жодної відповіді. Як виявилося, Вадим Іванович дуже втомлювався перед
репетиціями і після своїх слів міцно і солодко засинав, а книга закривала його
від очей режисера та інших акторів.

\ii{06_06_2019.stz.news.ua.mrpl_city.1.anatolij_levchenko_teatr_novator.pic.9}

\textbf{Порада маріупольцям:} 

\begin{quote}
\em\enquote{Частіше думайте про себе! Не жалійте себе, а саме – думайте! Людині властиво
ставити крапку, але ніколи не ставте крапки, адже ніяких крапок не існує. Життя
весь час триває! Якщо щось закінчилося погано, то нічого не закінчилося.
Пам’ятайте, закінчується завжди все добре}.
\end{quote}

\ii{insert.read_also.demidko.kozhevnikov}

\ii{06_06_2019.stz.news.ua.mrpl_city.1.anatolij_levchenko_teatr_novator.pic.10}
