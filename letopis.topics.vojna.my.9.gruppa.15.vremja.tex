% vim: keymap=russian-jcukenwin
%%beginhead 
 
%%file topics.vojna.my.9.gruppa.15.vremja
%%parent topics.vojna.my.9.gruppa
 
%%url 
 
%%author_id 
%%date 
 
%%tags 
%%title 
 
%%endhead 

СИЛА СЛОВА 
ВРЕМЯ

... Продолжение темы о технике общения в чатах. Понимаете, вот вы пишете,
кидаете картинки, думаете, что это как то влияет на орков. Тут есть такая
проблема, что этого мало. Слишком мало. Какие бы вы картинки не кидали, какую
бы информацию вы не кидали, она не влияет сильно. А почему. А потому что
процесс контакта в чатах - ЭТО ДИНАМИЧЕСКИЙ ПРОЦЕСС ВО ВРЕМЕНИ. И кроме того,
здесь орки не ведут себя как Люди, они ведут себя как ЗОМБИ, ИНФОРМАЦИОННЫЕ
НАРКОМАНЫ. Понимаете разницу? Мы - Люди, обладающие Интеллектом, Разумом,
Волей, они - информационные зомби, информационные наркоманы, которых непрерывно
накачивают злобой, ненавистью, агрессией, которые есть информационные рабы, и
которых так просто не пробьешь простыми дискуссиями или аргументами. Понимаете.
Картинки, фразы, комментарии любой может кидать. И мы кидаем, и они. А как же
выяснить, кто именно зомби, а кто Человек? А только конкретно в процессе
общения, когда ты занимаешь чатик не только своей информацией - ну то есть,
пришел, кинул и ушел - это не влияет особо - потому что тут же прибежали твари
и затерли вашу информацию - а когда ты конкретно занимаешь чатик на протяжении
какого то времени. Ну вот, взял, пришел, например, и занял чатик на 10 минут.
Все, я пришел в этот чатик - на 10 минут я здесь. Поэтому, если вы хотите на
самом деле научиться побеждать орков в чатах, то нужно понимать, что важна не
только ИНФОРМАЦИЯ, но важны также ИНФОРМАЦИОННОЕ ПРОСТРАНСТВО, и важно ВРЕМЯ.
Понимаете, в этой информационной войне важны

- Информация, которые вы посылаете, ее наполнение, смыслы, эмоциональная сила, насыщенность, мощность, психическое воздействие

- Информационное Пространство, то есть, насколько много вы занимаете
информационного пространства, насколько много вы таким образом отжимаете чатика у орков. Потому что Чатик - это Территория, это Ресурс, за которые непрерывно идет борьба.

- ВРЕМЯ, на протяжении которого вы пишете в чатиках

Ну вот, смотрите. Они все лезут, и лезут.
ВОПРОС: Как же выгонять их нахер с чатиков?
ОТВЕТ: психологически-эмоциональные методы

Например, такой метод. Я прихожу в чатик, и начинаю быстро-быстро просто что то писать, что в голову взбредет. Просто тупо пишу, очень очень быстро, пару предложений - и отправил. 
