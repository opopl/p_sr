% vim: keymap=russian-jcukenwin
%%beginhead 
 
%%file 10_04_2022.fb.koshmarchenko_jurij.1.meta_peremoga
%%parent 10_04_2022
 
%%url https://www.facebook.com/jumma/posts/10228098714159237
 
%%author_id koshmarchenko_jurij
%%date 
 
%%tags 
%%title Є лише одна глобальна мета — перемога у війні
 
%%endhead 
 
\subsection{Є лише одна глобальна мета — перемога у війні}
\label{sec:10_04_2022.fb.koshmarchenko_jurij.1.meta_peremoga}
 
\Purl{https://www.facebook.com/jumma/posts/10228098714159237}
\ifcmt
 author_begin
   author_id koshmarchenko_jurij
 author_end
\fi

Нещодавно у нас вийшов текст, де з точки зору психології розбиралося те, що ми
зараз відчуваємо. І там була одна, як мені здається, важлива штука. Що б ми
зараз не робили, це може здатися недостатнім чи малозначущим. Тому що є лише
одна глобальна мета — перемога у війні. І поки що жодні зусилля жодного з нас
не призводять до того, що цю задачу виконано. Тому є певне розчарування від
себе.

Але штука в тому, що ця величезна перемога складається з купи дрібних
переможеньок. Із влучного пострілу, з доставленої гуманітарки, з загашеної
пожежі, з вилікованої людини, зі скарги на пропагандистський канал, з робочої
зміни на заводі, з перекинутих на правильний рахунок 100 грн, із купленого для
сусідки батона. І без цих малих успіхів нічого великого не буде.

Мені здається, зараз важливо робити навіть просто те, що вмієш. Тому що на
війні це також може бути перемогою.

Вчора у Сквоті був концерт. Гурт Фолкулака вийшов на сцену і заспівав. Було
багато людей і, здається, для всіх це було важливо — побувати на концерті в ще
недавно напівоточеному Києві. У Києві, де досі регулярно виють сирени.

А ще на вході стояла банка для пожертв — збирали на спецавтомобіль для
медиків-госпітальєрів. Після концерту все підрахували — виявилося, що люди
накидали та надіслали понад 50 тис. грн. А потім згадали, що є ще пожертви на
PayPal і ще якісь штуки, порахували заново — і вийшло, що насправді зібрали 153
тис. грн. За пару годин концерту.

Все тому, що люди просто зробили те, що вміють найкраще. Музиканти — виступали.
Бармени — наливали чаї та кави. Таксисти — привозили та відвозили людей,
аналізуючи по дорозі геополітику.

Якщо ти вмієш щось, що радикально наближає перемогу, і маєш на це сили — роби.
Якщо ні, то роби те, що можеш. Якщо нічого не можеш — не страшно, війна все ж
таки. Тоді просто не заважай тим, хто щось робить.

І не дорікай. Після перемоги можемо призначити особливу дату, коли всі
пересваримося. Цікавий буде день.

А поки давайте просто влаштуємо багато-багато дрібних перемог.

І одну дуже-дуже велику.

\ii{10_04_2022.fb.koshmarchenko_jurij.1.meta_peremoga.cmt}
