% vim: keymap=russian-jcukenwin
%%beginhead 
 
%%file 12_10_2021.fb.fb_group.story_kiev_ua.1.kievljanin_brezhnev_okonchanie
%%parent 12_10_2021
 
%%url https://www.facebook.com/groups/story.kiev.ua/posts/1772966462900147
 
%%author_id fb_group.story_kiev_ua,majorenko_georgij.kiev
%%date 
 
%%tags brezhnev_leonid,cukanov_georgij.sssr,istoria,kiev,kievljane,sssr
%%title КИЕВЛЯНИН РЯДОМ С БРЕЖНЕВЫМ. окончание
 
%%endhead 
 
\subsection{КИЕВЛЯНИН РЯДОМ С БРЕЖНЕВЫМ. окончание}
\label{sec:12_10_2021.fb.fb_group.story_kiev_ua.1.kievljanin_brezhnev_okonchanie}
 
\Purl{https://www.facebook.com/groups/story.kiev.ua/posts/1772966462900147}
\ifcmt
 author_begin
   author_id fb_group.story_kiev_ua,majorenko_georgij.kiev
 author_end
\fi

КИЕВЛЯНИН РЯДОМ С БРЕЖНЕВЫМ. 

окончание.

История о Георгии Эммануиловиче Цуканове - моем дяде и единственном киевлянине
в руководстве СССР эпохи Брежнева.

ВОЙНА ВСЕХ ПРОТИВ ВСЕХ

Что мы видим, говорит,

кроме телевидения?

В.Высоцкий

По телевидению мы видели монолитную глыбу ЦК КПСС и мало кто знал, какая
жесточайшая подковерная борьба тогда кипела в руководстве СССР.

БИТВА ЗА ИЛЬИЧА

Земля тряслась, как наши груди

Смешались в кучу кони, люди...

М.Лермонтов

Днепропетровский клан сражался с кланом Молдавским, КГБ толкалось с МВД,
консерваторы сцепились с либералами. А сверху, аки мудрый Зевс, наблюдал за
этой чехардой Леонид Ильич Брежнев.

ЗЕМЛЯЧКИ

Главный герой нашей истории Георгий Эммануилович Цуканов принадлежал к
Днепропетровскому клану - самому родному и близкому для Леонида Ильича. Этот
клан был ответвлением так называемого "украинского землячества", пустившего
глубокие корни в Москве. Но взаимоотношения между "московскими - украинскими"
были весьма специфическими.

\ifcmt
  tab_begin cols=2

     pic https://scontent-frx5-1.xx.fbcdn.net/v/t39.30808-6/244990286_2749287888703768_401270457989241274_n.jpg?_nc_cat=105&ccb=1-5&_nc_sid=b9115d&_nc_ohc=EisZTqk7Ek0AX-71p9E&_nc_ht=scontent-frx5-1.xx&oh=bb89ddc567d8e1e510cc9b1602146022&oe=619308C5

     pic https://scontent-frx5-1.xx.fbcdn.net/v/t39.30808-6/244768356_2749287965370427_3520160165843432040_n.jpg?_nc_cat=105&ccb=1-5&_nc_sid=b9115d&_nc_ohc=ctx3yd-Irv0AX_j5tql&_nc_ht=scontent-frx5-1.xx&oh=ef657aea6a49e923c036f2909b90c3ff&oe=61929547

  tab_end
\fi

Допустим, пригрел донецкий шахтер Хрущев днепродзержинского металлурга
Брежнева, но мы то знаем, куда потом Леонид Ильич отправил Никиту Сергеевича.
Такие вот землячки!

ТЕНЬ БРЕЖНЕВА

Георгий Эммануилович Цуканов не мелькал на телеэкранах, его изображение редко
появлялось на страницах газет, но в высших сферах СССР Цуканова из-за близости
к Брежневу называли "Серым кардиналом". А ещё у Георгия Эммануиловича среди
руководства страны была кличка Хоттабыч. Видимо, благодаря его широчайшим
возможностям. У известного журналиста и публициста Федора Бурлацкого есть
рассказ, как на отдыхе Гаграх, руководители республик Советского Союза
становились в очередь, чтобы засвидетельствовать почтение Георгию Эммануиловичу
Цуканову. Понимали, кто сможет решить вопрос на самом высоком уровне!

Собирался ли Георгий Эммануилович  сыграть в свою игру, выйти из тени и стать
преемником Брежнева? А почему бы нет? Технократ, интеллектуал, человек,
держащий руку на пульсе экономики страны. 

Но произошло трагическое событие, которое поставило крест на амбициях Георгия
Цуканова.

ЛИЧНАЯ ТРАГЕДИЯ

Был у Георгия Эммануиловича сын Михаил - самый родной и близкий для него
человек. Михаил Цуканов окончил Институт Международных отношений и работал
дипломатом в Австралии. Однажды после какой-то официальной встречи работников
посольства он погиб в автомобильной катастрофе. И это была не обычная авария -
Михаилу помогли уйти из жизни. Обстоятельства его гибели покрыты мраком. Не
исключено, он был свидетелем разговора, который не предназначался для
посторонних ушей, и его ликвидировали спецслужбы. Причем, это могли сделать как
иностранные спецслужбы, так и наши.

ЧЕРНАЯ МЕТКА

Очень похоже, что это была "черная метка" его отцу - на тот момент - самому
близкому для Брежнева человеку. И враги Цуканова, зная его любовь и
привязанность к сыну, нанесли сокрушительный удар.

Георгий Эммануилович Цуканов после гибели сына стал замкнутым, подавленным и,
как писал его друг журналист - международник Бовин - склонным к компромиссам.

И тут недруги Цуканова воспользовались ситуацией по-полной и оттеснили убитого
горем Георгия Эммануиловича от Брежнева.

КРАХ АМБИЦИЙ

Вот так были похоронены планы и амбиции Цуканова, который был выведен из
большой политической игры. Хотя позиции Днепропетровского клана в руководстве
СССР были сильны - премьером страны оставался Тихонов, МВД удерживал Щелоков,
на Украине правил Щербицкий, но Брежнева уже плотно опекали Черненко и
Андропов.

ШАНС ЩЕРБИЦКОГО

Незадолго до кончины Брежнев планировал передать бразды правления СССР
Щербицкому, но не успел. Вот при Шербицком мог наступить звездный час Цуканова!
Щербицкий был своим человеком.

РАЗГРОМ ДНЕПРОПЕТРОВСКОГО КЛАНА

Для возглавившего страну Андропова - "днепропетровские" были чужаками и первым
делом последовала расправа со знатным "днепропетровцем" министром МВД
Щелоковым.

Особенно свирепствовал новый руководитель КГБ СССР Федорчук,приехавший с
Украины и возглавивший Комитет после Андропова. Он и провел операцию
"Ликвидация Щелкова".

Вот такой землячок!

ЧЕРНЕНКО

Константин Устинович Черненко был представителем Молдавского клана. Как
известно, Брежнев некоторое

время руководил Молдавией и многих "молдавских" забрал в Москву в свою команду.
Естественно, Черненко рассматривал Цуканова, как чужака и конкурента. И
насколько Константин Устинович Черненко приближался к Брежневу, настолько
отходил на задний план от Леонида Ильича Цуканов. Черненко импонировал Брежневу
умением замять конфликтные ситуации и сгладить острые углы. Цуканов же наоборот
был суровым технократом и резал "правду-матку", а угасающему Брежневу рядом был
необходим человек - убаюкивающий его сладкими речами.

ДУШЕВНЫЕ РАНЫ

И стало влияние Георгия Эммануиловича уменьшаться и сдулось, как пробитый
гвоздем футбольный мяч.

Он по-прежнему был вхож к Брежневу, приносил сводки, доклады, обсуждал с
Леонидом Ильичем текущие события. Но это уже было просто дежурным официозом.
Охладел Генсек к Цуканову - былой дружбе пришел конец!

Конечно же, Георгий Эммануилович очень тяжело переживал свою опалу и "топил
грусть" в компании интеллектуалов из отдела консультантов и референтов. Цуканов
являлся создателем и шефом сего отдела. А работали в этом отделе бывшие
подопечные Андропова.

АРИСТОКРАТЫ ДУХА

Конда Андропов находился на должности Заведующего Отделом ЦК КПСС по связям с
коммунистическими и рабочими партиями социалистических стран, то создал при
отделе группу консультантов, в которой собрал интеллектуалов из
научно-журналистской среды - вольнодумцев и карбонариев. Средь них - академики
Арбатов и Шахназаров, журналисты Бурлацкий и Бовин, политолог Лев Делюсин.
Андропов называл их "аристократами духа", обсуждал с ними стратегию и тактику
отдела и, благодаря этим людям, знал, чем дышит либеральная Москва. Ведь этот
народ плотно общался с Высоцким, Любимовым, Окуджавой, Вознесенским, Аллой
Пугачевой и даже с Сахаровым.

Когда же Андропов возглавил КГБ СССР то эта информация ему очень пригодилась. А
вот потребность в консультантах- вольнодумцах отпала. У КГБ с карбонариями было
общение в виде профилактических бесед. А особо буйные направлялись в психушки и
лагеря.

Бесхозных же консультантов Андропова "подобрал и обогрел" Георгий Эммануилович
Цуканов.

РЕЧЕПИСЦЫ, ОКУНИТЕ ВАШИ ПЕРЬЯ!

В 1989 году в одном серьезнои журнале я наткнулся на статью Леонида Макаровича
Кравчука о том, какие беды нас ждут в случае развала Союза. Но через год
Кравчук утверждал противоположное и сладко пел о молочных реках и кисельных
берегах. Когда я однажды спросил известных украинских политиков, что они думают
по этому поводу, был ответ:

-А ты думаешь сам Кравчук читал эту статью?

-Как так?

-Поверь, он ее не только не читал, но даже не писал. Это за него сделали
помощники-консультанты!

Так обстояли дела и с Брежневым. Помню детский стишок:

Это что за Бармалей,

который влез на Мавзолей?

Брови черные - густые,

речи длинные - пустые...

Речи, доклады и статьи - писали Брежневу консультанты из отдела Цуканова.

И "на манеже" были все те же - аристократы духа! Среди лидеров - академики
Арбатов и Иноземцев, знаменитый журналист-международник Бовин и др. Вот в их
обществе Цуканов отдыхал душой.

СМЕРТЬ БРЕЖНЕВА. ПЕРЕСТРОЙКА.

Со смертью Брежнева Цуканов формально числился в команде Андропова, а затем
Черненко, но от его былого величия не осталось и следа.

А с приходом Горбачева настали иные времена. Вернулся из Канады (где был
послом) Александр Николаевич Яковлев, ставший "отцом русской демократии" и
идеологом перемен. И уже под крылом Яковлева нашли приют речеписцы-консультанты
из отдела Цуканова. Только теперь они величались "Прорабами перстройки".

А могучий технократ Георгий Цуканов - оказался за бортом. Настало время не
практиков - производственников, а краснобаев и баломутов:

По России мчится тройка -

Рая, Миша, перестройка!

ЭПИЛОГ

И опечалился Георгий Эммануилович, заперся в кабинете и засел за мемуары.
Журналистам интервью не давал и родственникам крпепко-накрепко запретил
делиться с прессой какой-либо информацией. Очень надеюсь, что с благословения
родни и божей помощью, когда-нибудь удастся опубликовать воспоминания Георгия
Эммануиловича Цуканова.

Похоронен дядя Жора в Москве на Новодевичьем кладбище. 

Но частичка его сердца осталось в Киеве - городе, где он родился, учился, где
жили его родственники и друзья...

Ведь был он киевлянином в нескольких поколениях. Ещё тех дореволюционных!

И хотя давно нет старинного дома на Шулявке, где проживали Цукановы, но древние
тополя и клены хранят память о Жорже - мальчишке с Борщаговской, поднявшемся на
недосягаемые высоты власти и оставившем яркий след в истории Родины.

КОНЕЦ.

\ii{12_10_2021.fb.fb_group.story_kiev_ua.1.kievljanin_brezhnev_okonchanie.cmt}
