%%beginhead 
 
%%file 28_02_2023.fb.kipcharskij_viktor.mariupol.1.r_k_tomu__bulo_take_
%%parent 28_02_2023
 
%%url https://www.facebook.com/permalink.php?story_fbid=pfbid026PaaZcjA538gMf89BQPNEdbpYvjUDsQr4GhdMdpKNfWbETfYakmKLdbn4NWqTgAtl&id=100006830107904
 
%%author_id kipcharskij_viktor.mariupol
%%date 28_02_2023
 
%%tags mariupol,mariupol.war,dnevnik,28.02.2022
%%title Рік тому  було таке:  День 5 - 28.02.22. Понеділок
 
%%endhead 

\subsection{Рік тому  було таке:  День 5 - 28.02.22. Понеділок}
\label{sec:28_02_2023.fb.kipcharskij_viktor.mariupol.1.r_k_tomu__bulo_take_}

\Purl{https://www.facebook.com/permalink.php?story_fbid=pfbid026PaaZcjA538gMf89BQPNEdbpYvjUDsQr4GhdMdpKNfWbETfYakmKLdbn4NWqTgAtl&id=100006830107904}
\ifcmt
 author_begin
   author_id kipcharskij_viktor.mariupol
 author_end
\fi

Рік тому  було таке: 
День 5 - 28.02.22. Понеділок.
Відключили опалення. Світло, зв'язок, інтернет, вода і газ є. Кілька тривог за день.
Телефонував брат Микола: в них спокійно, сини працюють, люди патрулюють село.
Я зателефонував колишньому студенту В.З. - він вже у Львові, у Дніпрі підібрали сина, який елетричками виїхав з Харкова з одногрупниками. 
У місті відкрито пункти видачі "соціальної" води для районів де нема центрального водопостачання (перелік адрес по районах) - з 8:00  до 17:00 - до комендантської години.
У ТерраСпорті вже більше тисячі людей, які покинули свої домівки - просять допомоги.
Страшне відео: з Сартани швидка привезла поранену дитину - лікарі плачуть, бо не змогли врятувати дитині життя...
Збили літак, що чотири рази скидав бомби на Маріуполь.
Новини "заколисують"... Ще ось-ось і...
В той же час, ще напередодні, вранці 27-го лютого рашисти захопили Бердянськ...
Навздогін: в зараз дехто розповідає про те, як пробирався з Мангушу в Маріуполь по слідах "можливо російської техніки". 
Ще раз: дехто вже знав, що Маріуполь оточено. Зв'язок і інтернет ще працювали. Ще кілька днів працювали. Вони "не хотіли сіяти паніку"!!!
Два гуманітарних поїзди 24 в один 25-го лютого поїхали з Маріуполя напівпорожніми, аби "не було паніки"...
 Не знаю, що б ми робили, дізнавшись про реальний стан речей, але то вже інша справа: чому хтось привласнив собі право вирішувати, що саме можна знати людям??? Зараз легко розмірковувати, про те, що треба було робити. Як кажуть в Одесі: "Щоб я був такий розумний, як моя жінка потім!".
Але! Міській владі і її опонентам треба було звернутись до людей, організувати захист магазинів та складів з їжею та ліками, бензозаправок з паливом, налагодити систему оповіщення і багато чого зробити, аби уникнути втрати продовольства, ліків, палива... Чому ж я, "такий розумний", цього не зробив? Тому, що хтось вирішив, що мені можна знати, а чого - не можна!!!
Отаке тоді написалося мені 
(дякувати Богу,  хлопці із Зміїного живі, на сьогодні декого з них повернули з полону)
Как в страшном сне, но наяву
Я в день сегодняшний живу
И что бы завтра не случилось,
Я жив сегодня - эту милость
Я с благодарностью приму.
Пройдет гроза, исчезнут тучи
И смоет чистою водой
Всё то, в что верить и не хочешь
Расстрелянные дни и ночи,
Как будто было не со мной.
Я не прошу богатства, мудрость,
Земной рай или не земной -
Вот только б мальчики-солдаты
Закрывшие меня собой,
Вернулись бы к своим, домой...
А те тринадцать - на Змеином, 
Пославшие корабль эрэф, -
Стоят в строю, уже не зримом,
В когорте витязей былинных
И охраняют нас с небес...
