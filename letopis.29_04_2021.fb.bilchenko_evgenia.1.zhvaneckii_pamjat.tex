% vim: keymap=russian-jcukenwin
%%beginhead 
 
%%file 29_04_2021.fb.bilchenko_evgenia.1.zhvaneckii_pamjat
%%parent 29_04_2021
 
%%url https://www.facebook.com/yevzhik/posts/3846821838686231
 
%%author 
%%author_id 
%%author_url 
 
%%tags 
%%title 
 
%%endhead 

\subsection{БЖ. Памяти Жванецкого посвящается}
\Purl{https://www.facebook.com/yevzhik/posts/3846821838686231}


\ifcmt
  pic https://scontent-bos3-1.xx.fbcdn.net/v/t1.6435-9/180311457_3846813855353696_1878442240768509348_n.jpg?_nc_cat=103&ccb=1-3&_nc_sid=730e14&_nc_ohc=eL0iRWtSDkMAX8FuZtK&_nc_ht=scontent-bos3-1.xx&oh=9989e5d1eb47a5e094943926383e7f64&oe=60B4DF43
\fi


Сегодня после отправки на кафедру моего научного отчета за месяц получила
интересный вопрос и написала развернутый ответ. Дальше не комментирую. 

"Доброго дня, Євгеніє Віталіївно! Кафедра просить, щоб Ви надіслали, згідно
звіту за квітень 2021 року, надруковані тексти статей і тез, які вийшли, а
також текст монографії".

Я: Дорогие коллеги по партии! 

С Вашего позволения как беспартийная в письме буду говорить на своем родном
русском языке (ст. 10, 161 КУ): благо, работаю как ученый сейчас, в основном,
за границей, хотя моё украиноязычное письмо перевешивает по числу авторских
листов, к моему, кстати, большому сожалению: горизонта не хватает, знаете ли).
Присылаю вам по Вашей странной просьбе уже сверстанный текст моей книги и
опубликованные тексты моих статей только за один месяц апрель с повторными
ссылками на всё (они все уже были в посланном Вам отчете, с полными
библиографическими описаниями в таблице для каждой публикации и даже со
ссылками URL, по которым всё это было легко проверить, но план, видимо, Вами
либо не был прочитан, либо, в силу его объема, Вы ему не поверили: проекцию
понимаю, не все служат науке, некоторые консультируют своих аспирантов одним
абзацем в ФБ).  

Кстати: как марксист - американистам: коли вдруг, не дай Бог, линк в документе
теряет активность, все глобалисты, даже не продвинутые в дигитальных
технологиях, понимают: если нажать на кнопочку Ctrl  на клавиатуре компьютера
одновременно с удержанием мышки на линке, то ссылка в Word "синеет", как небо
над Москвой, то есть становится активной, и по ней всё можно без труда
проверить и перепроверить. 

В общем, рискую и высылаю Вам опубликованные статьи еще и в формате Word, с
которого легко копировать текст (не приведи Бог, найду оттуда копипаст),
верстку монографии в PDF (оттуда тоже можно копировать) и рецензии на нее же,
дублируя также все библиографические описания и ссылки. 

Поскольку меня с точки зрения авторского права тревожит присутствие моих работ
в Word, я дублирую это письмо в ФБ и своему адвокату, заодно задав ему вопрос:
совместимы ли такие объемы работы с трудовым кодексом? Также прошу лично
передать все документы заведующему кафедрой. 

Еще шесть моих поданных за месяц статей, указанных во второй части отчета,
включая три Web of Science, в целях защиты своих прав и прав международных
журналов я не высылаю. 

Третья часть отчета - методическая - Вами прокомментирована не была, потому что
содержала консультативную научную помощь отнятому у меня аспиранту Павлу
Волкову, чей аккаунт забанен на Фейсбуке за международное выступление в Европе,
и гражданам (ученым и аспирантам) соседнего государства РФ, по запрещению
научных контактов с которым я прошу предъявить мне законодательные акты. 

Помощь моим бывшим магистрам, которых у меня забрали негласно (телефонная
запись сохранена), в отчет включена не была с целью сохранения их от невроза и
шантажа. 

По вопросам моего голосового участия в двух международных онлайн-конференциях,
подробные описания которых также даны в отчете, обращаться к организаторам:
лично к профессору Возняку Владимиру Степановичу (конференция по Андрею
Рублеву) и в: Міжнародний культурний центр «Сяйво» и Міжфракційне депутатське
об’єднання Верховної Ради України «Платформа "Розвиток України» (конференция по
феминизму).

Увидели свет в апреле:

1. Монография: Верстка. Задник. Передник. Рецензии трех докторов наук. Більченко Є. Символічні діалекти діалогу: мова ненависті, мова толерантності, мова солідарності: Монографія / Більченко Євгенія Віталіївна. – К.: Ін-т культурології НАМ України, 2021. –  480 с.
2. Вашингтон, текст статьи: Бильченко Е.В. Кибербуллинг в измерениях культурологии // VIII Международная научно-практическая конференция «Challenges in Science of Nowadays». – Вашингтон, США, 4-5 апреля 2021 г. – С. 625-634. URL: https://ojs.ukrlogos.in.ua/.../interconf/article/view/10968
3. ИК НАМ, статья ВАК: Більченко Є.В. Метафізика та діалектика Третього в діалозі культур: аксіологічний та психоаналітичний виміри // // Культурологічна думка. - 2021. - № 19. – URL: https://www.culturology.academy/metafizika-ta-dialektika... (ВАК)
4. Международная статья: Бильченко Е.В. Поэзия, философия, технология в свете культурологии: стратегия диалога / Eurasian Union of Scientists. Международный научно-исследовательский журнал.  2021. Vol. 4 No. 3(84). – С. 26-32. URL:
https://archive.euroasia-science.ru/.../issue/view/112....
