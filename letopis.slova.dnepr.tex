% vim: keymap=russian-jcukenwin
%%beginhead 
 
%%file slova.dnepr
%%parent slova
 
%%url 
 
%%author 
%%author_id 
%%author_url 
 
%%tags 
%%title 
 
%%endhead 
\chapter{Днепр}

Когда тонешь на середине \emph{Днепра} - хватаешься за проплывающие пластиковые
бутылки, Так сейчас барахтается и экономика Украины, Александр Гончаров,
strana.ua, 30.05.2021

Дамба на Северо-Крымском канале, построенная в Херсонской области, начала
разрушаться, из-за этого вода из \emph{Днепра} начала утекать на полуостров. Об
этом идёт речь в сюжете «5 канала», опубликованном на YouTube. Временную
плотину возвели в 2014 году, затем в 2015 году начали сооружать новую дамбу, но
работы так и не завершили. Недавно местные жители обнаружили, что плотина
перестала выдерживать напор воды из \emph{Днепра} и та утекает в Крым.
Конструкцию возвели из мешков с песком и металлоконструкций. Спустя почти семь
лет временная плотина перестала удерживать воду из Днепра на украинской
стороне,
\citTitle{Вода из Днепра начала поступать в Северо-Крымский канал}, , sharij.net, 07.06.2021

