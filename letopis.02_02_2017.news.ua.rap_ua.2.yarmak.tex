% vim: keymap=russian-jcukenwin
%%beginhead 
 
%%file 02_02_2017.news.ua.rap_ua.2.yarmak
%%parent 02_02_2017
%%url https://rap.ua/ya-pytalsya-sdelat-vsyo-chtoby-sohranit-stolnyy-grad-intervyu-s-yarmakom-chast-2
 
%%endhead 

\subsection{«Я пытался сделать всё, чтобы сохранить Стольный Град»: интервью с Ярмаком. Часть 2}
\url{https://rap.ua/ya-pytalsya-sdelat-vsyo-chtoby-sohranit-stolnyy-grad-intervyu-s-yarmakom-chast-2/}

\ifcmt
pic https://rap.ua/wp-content/uploads/2019/08/yarmak_intervyu_rap_ua_stolnyi.jpg
\fi

... Меня интересует только Оксимирон. Меня не интересует хайп, мне просто есть
что ему сказать. Я знаю, что многие его вызывали, Артём Лоик, например.
Касательно Тёмы – я не понял, почему в нем играет эта то ли зависть, то ли
юношеский максимализм – при этом мы нормально общаемся, но до меня доходит
разная информация, у меня в Полтаве много друзей. Меня это не оскорбляет,
просто я не понимаю такой позиции, может, он психологически не дорос.

Что у Тёмы пошло не так, как думаешь? Он же отличный МС.

Проблема в его психологии. Я ему могу это открыто сказать – со стороны-то
виднее. Тем более, я не буду выделываться – но я протоптал дорогу шире, чем он,
и ошибок насовершал больше, и во многом разбираюсь лучше. У всех: спортсменов,
бизнесменов, музыкантов – всё в нашей голове. Вот когда он решит для себя
определенные вещи, когда будет больше работать над собой, больше от себя
требовать, когда из Полтавы переедет в Киев – это первое! Я был очень удивлен,
когда он заработал на «Україна має талант» 300 000 гривен, по нему пошла
нереальная волна, а он просто купил квартиру, вместо того, чтобы вложить в
себя, а потом купить 10 квартир, купить Полтаву! Это вообще, как по мне,
неправильный подход.

Если бы он вложил деньги в себя – я уверен, что у него все намного лучше бы
сложилось. Но он почему-то вот такой путь выбрал, самый примитивный, вот зона
комфорта, она меня устраивает.

Я каждый день специально выхожу из зоны комфорта, это меня прокачивает.

В чем заключается твой комфорт?

Если говорить про жизненные вызовы, то  у меня их было о-о-очень много. Я не
хочу рассказывать детали о Стольном Граде, но я мог бы вот так вот и плыть по
течению, но я решил, что нужно выйти из зоны комфорта, и включить не пятую, а
десятую передачу, и постараться выйти из середнячка в топ-артисты. Для этого
нужно нереально над собой работать, и я это делаю. Я раньше как – написал
куплет, и готово. А сейчас я работаю над каждой строчкой, перечитываю,
переделываю, уделяю намного больше времени этому аспекту.

Может, парни просто не могли включить такие же скорости, как ты?

Они сами это знают – не могли, ну, они другие немного.

В чём все-таки причина распада?

Причин много и все больше психологические. Главная в том, что когда мы
собрались еще давно-давно, я говорю «Пацаны, давайте делать так?» «Давайте!». А
со временем оказалось, что кого-то не устраивает общая идеология, каждый хочет
делать по-своему. Я за свободу, я сказал: давайте делать так, как хочется
каждому. Нет смысла ограничивать человека, только в свободном состоянии можно
делать шедевры. Хотя пацанов, никто никогда ни в чем не ограничивал, у них была
лучшая рэп-студия в Украине, возможность снимать клипы, делать так, как хочется
им, но они сказали, что видят это все по-другому, а я сказал «Я вижу
по-своему». Не было никаких ссор, мы хорошо общаемся, но уже давно не проводим
время вместе, у каждого свои интересы и взгляды, которые за годы стали слишком
разными. Я уверен, что такой исход пойдет всем артистам лейбла только на
пользу, мы также продолжим радовать наших фанатов своим творчеством и, надеюсь,
теперь его станет еще больше и оно будет еще более свободней!

Пока я готовился к интервью, перелопатил кучу информации о тебе, и мне
показалось, что ты в каком-то переломном моменте сейчас. В чем он заключается,
кроме распада Стольного Града?

Да, так и есть. В распаде нет ничего страшного, к этому всё шло, мы не как
группа Centr, участники которой не поделили бабки, нет – просто то, как мы
построили наши взаимоотношения – это не сработало.

А как в Стольном всё построено?

Всё построено на моих заработках с концертов, вот и всё (смеётся).

Участники Стольного подписывали какие-либо контракты?

Несколько лет мы не подписывали ни одной бумажки. Потом нам пришлось это
сделать, подписать договора о передаче прав на песни, иначе мы не могли
представлять их интересы и продавать их музыку. Это чисто юридическая волокита.

А с кем они подписывали эти договора?

С компанией «ЯрмакМюзик», соучередители – я и два моих партнера, с которыми мы
начинали.

То есть на 33 процента участники Стольного подписали договора с тобой?

Да. И я от них ничего не требую, при том что я вложил очень большие деньги. Мы
подписывали некоторые бумаги не для того, чтобы требовать, а для того, чтобы
больше давать артисту, продаж на ITunes и т.д. Я потратил кучу денег и времени,
но я не разочарован, я очень многому научился. Я осознал принципы работы
индустрии, очень прокачался в плане психологии. Самая большая проблема в нашей
стране – в психологии, мы никогда не заживем хорошо, пока не изменим что-то в
наших мозгах.

Ты говоришь о правильном окружении – а распад Стольного, случайно, не твоя
попытка сменить окружение, которое ты считаешь неправильным?

На самом деле нет. Мы все дружим, все друг друга любим, в Стольном все люди
уникальные, но некоторые участники, к сожалению, далеко не мои единомышленники.

Ты говоришь, что вы сели, как муж с женой, состоялся разговор. Его инициатором
был ты?

Да. Проблемы были внутренними, они заключались в непонимании того, куда мы
идём. Я пытался их решить всеми возможными путями – ребята даже ходили на курсы
по развитию личности к моим друзьям.

Я пытался все сделать, чтобы сохранить Стольный Град, но у меня не получилось.
Я всегда предъявляю претензии только к себе.

Какую главную претензию ты можешь предъявить себе по поводу распада Стольного?

Что я мало времени уделял ребятам.

Ты правда как про жену говоришь.

Всем необходимы внимание, любовь, похвала. Я в какой-то момент всё это упустил.
Когда мы создавали первый альбом – была любовь, было уважение, и первый альбом
намного душевнее, чем второй. Я не смог отдавать Стольному много внимания – на
тот момент у меня появилась Аня, она как и любая нормальная девушка, тоже
требовала внимания, пришлось уже время делить.

Пацаны не могут тебе предъявить, что ты променял их на девушку?

Нет. Это будет неправдой.

Ну, вы сидели, делали одно дело, делали рэп. Тут появляется красивая девушка –
и ты говоришь, мол, всё, парни, мне с ней интереснее, и уходишь.

Суть в том, что я никогда не уходил, вообще никогда. Я работал днем и ночью,
чтобы это не повлияло на Стольный. Например, на первом альбоме из 15 песен я
спродюсировал 10, написал куплеты с припевами, куда все вписались, на втором
альбоме из 15 песен я также написал 10 куплетов с припевами, куда опять-таки
все вписались. Как вам статистика? Ко мне могут быть вопросы? У пацанов тоже
были девушки и когда они с ними расходились, вот это реально влияло на процесс,
кого-то я мог месяцами не видеть на студии.

Кстати, мы не закончили про формулу счастья.

Любимый человек, любимое дело, правильное окружение и путешествия! Многие люди
ошибаются, когда говорят, что у них нет денег, чтобы куда-то поехать. Важно
сменить картинку, происходит перезагрузка организма.

Если у вас есть все эти 4 компонента – значит, все у вас будет круто, и деньги,
и все.

А ты ведь последний раз в Дубае был в путешествии?

Ну да, и потом еще когда в Европу ездили. Но это отдыхом нельзя назвать – я
ебашил, пока люди в Лувр ходили.  Жара 38 градусов, а я камеру и штатив ношу.

Так ты в Лувр так и не попал?

Нет, не попал! Братан, у меня просто не было времени, я не отдыхал, все
отдыхали –  а я нет. Я снимал клип и еще многим помогал. Я привык, что я отец,
я должен всем помочь, всем должно быть хорошо.

Чувствуешь ответственность?

Конечно!

А чувствуешь ответственность за то, что ты самый популярный рэпер в Украине?

Конечно! Потому я делаю все для того, чтобы поднять хип-хоп в Украине, а не
жить в иллюзии, что я, мол, крутой, потому что собираю из рэперов пока больше
всех. Это может быть недолго. Задача – не останавливаться, я просто боюсь
отдыхать и не хочу!

Сань, это же болезнь.

Это болезнь, это реальная болезнь, я больной, я как робот! Я сейчас пишу
альбом: сажусь кушать – думаю про строчки, рифмую. Аня говорит – пожалуйста,
побудь со мной! А я ей – да я с тобой, я дома! А она – нет, ты дома, но вообще
не со мной! Я утром просыпаюсь и до самого позднего вечера работаю. Может, я
раб своей мечты, да – но я кайфую, отвечаю! Я с музыкой – и я кайфую!

У тебя был когда-нибудь продюсерский контракт?

Это была просто бумажка, на одном листике. У нас были такие отношения, ну прям
полнейшей свободы. Просто я Скорпион и меня не устраивали какие-то нюансы…

Для тебя знаки Зодиака что-то значат?

Конечно, я убежден в том, что это имеет значение.

С кем был твой первый контракт?

С моими товарищами, которые потом стали инвесторами Стольного. Мы с ними
работали все эти годы, но наши отношения также закончились вместе с закрытием
лейбла.

А они вкладывали в тебя?

Да, но очень не большие деньги. Мы их быстро отбили и начали вместе создавать
«Стольный». Вообще я прошел эту школу еще в КВН, команда, которая имела
стабильное финансирование, играла намного лучше, просто потому что ей не
приходилось думать, где взять денег на поездку, чем заплатить взнос за игру и
что вообще кушать. Я побывал по обе стороны баррикад, Сборную НАУ взяли сразу в
Высшую Украинскую Лигу, а с командой «Ботаники» нам пришлось пройти все круги
ада, зарабатывать ведущими на корпоративах, чтобы оплатить взнос, ездить без
денег, кушать «Мивину». Как-то в Одессе на полуфинале Первой Украинской Лиги у
нас вообще закончились все деньги и мы весь день кушали сахар из банки на
съемной квартире. Потом я нашел в кармане 12 гривен и купил пачку макарон и это
было настоящее счастье.  Так вот к чему я веду – механизмы везде одинаковые.
Что в спорте, что в музыке, что в телевидении! Решают не бабки, решает
гармония, когда есть все! Есть талант, есть финансирование – появляется
продукт. Посмотрите, сколько футбольных клубов в Украине поднимались с самых
низов, таких как «Металист», «Днепр» и где оказались после отсутствия
финансирования. А есть «Динамо» Киев, когда вроде все условия, но почему-то не
бегут, так вот ситуация в «Стольном» очень напоминает ситуацию в «Динамо». Я
как-то раскритиковал Суркиса и понял, что у нас проблемы очень схожие. Я пришел
к своим партнерам и говорю «Пацаны, надо кардинально менять подход», но у всех
были другие проблемы.

На планете есть тысячи бойцов сильнее МакГрегора или Майвезера, но этих бойцов
знают все, потому что у них есть большой талант, работоспособность и такое же
большое финансирование. Но все находится в идеальном равновесии.

Украинский музыкальный пример – это группа Грибы. Парни сами по себе читали
рэп, снимали иногда клипы, которые набирали по 100 тысяч просмотров, но как
только в них влили реальные деньги и поменяли подход, сразу все полетело.

Вывод. Хотите быть звездами? Работайте над собой и ищите себе инвестора. У всех
были инвестора: у Басты, у Гуфа, у Эминема, у Jay-Z.

Как найти инвестора? В нашей стране все очень плохо с этим, зарабатывают только
безидейные проекты, которым пофиг, где, перед кем и что петь. Потому попса у
нас так стремительно развивается. В музыку улиц в нашей стране, практически
никто не вкладывает, долго ждать возврата, потому инвестора нужно искать самому
среди знакомых.

Отличные советы. Но мы так и не договорили с тобой про альбом.

Альбом будет называться «RESTART». Не в том смысле, что я заново все начинаю, я
просто устал и нажал на компьютере «RESTART». Сейчас он перезагрузится, он
больше не перегревается, процессор в хорошем состоянии, и он продолжает свою
работу, но намного качественнее и лучше. Я прокачал себя во всех направлениях –
в плане техники, построения рифм, качества продукта, музло будет в разы
качественнее.

Ты недавно записал первую украиноязычную песню – почему ты делаешь её именно с
Ваней из «Сальто Назад»?

Это не рэп будет – это песня про любовь, она очень красивая и нежная. Я написал
на украинском, а незадолго до этого мы с Ваней встретились и он сказал, что
хочет мне что-то спродюсировать, но у него есть обязательное условие – чтобы
песня была на украинском. И когда я эту песню написал, показал её Ване –
говорю, Ванёчек, послушай, а то там только гитара есть.  

Мне как-то парень, который в Стереоплазе всегда с нами выступал, гитарист
группы «Нервы», сыграл на реп-базе прикольную мелодию и прислал мне, кусочек
маленький, сэмпл, можно сказать.

Ты поёшь там?

Это нельзя назвать пением, я называю это интонированием, петь я не умею и не
учился, но, кстати, буду учиться. Я раньше сам себе делал примитивные биты
какие-то – на «Пидарасию», на «Маленькие города», например – у меня кое-какие
знания есть, я очень люблю Ableton. Так что я нарезал эту гитару,  скачал
барабанный луп, расставил по структуре, презентовал Ване, он говорит – классная
песня, я тебя помогу, и мы поехали, записали пока только демку, потому что там
много работы – бас прописывать нужно, трубы. Это песня, за которую мне не
стыдно, она так гармонично звучит! Мне очень нравится моя песня «Чёрное
золото», это, наверное, лучшая моя песня.

Эта украиноязычная песня будет на альбоме?

Да. Одна. А потом в сентябре я хочу такой хип-хоповский, украиноязычный альбом
сделать. Почему так? Потому что всё-таки нашу культуру надо развивать – это
наша уникальность, душа,  то, чем мы отличаемся. А сколько у нас уникальных
сэмплов, которые можно юзать! Я вообще фанат украинской культуры, я хочу в
будущем сделать коллекцию одежды с украинскими орнаментами, хочу себе тело
забить татухами с козаками, не хочу жить чужой культурой, бить себе какие-то
египетские или японские знаки. Зачем? У нас такая богатая культура
древнеславянская, неязыческая. Невероятно богата культура времен козаков, для
меня все это безумно родное, я все детство на этом рос, читал книги только по
истории. Мне настолько нравились наши воины, они для меня были лучшими в мире
супергероями! Я даже знал, сколько в каком бою козаки потеряли, сколько поляки,
сколько турки.

Ты говоришь про книги – очень многие прицепились к твоей фразе…

Что я не читаю книги!

Да, и что знания берешь из космоса.

Я не скрываю того, что меня ведет космос и вселенная, как и всех. И не скрываю
того, что не читаю художественную литературу.

Как происходит процесс познания? Жизнь меня сама приводит к нужной литературе и
нужным статьям. Когда я их читаю, на уровне интуиции чувствую – это мне родное
или нет. Зачастую, когда читаю нужную информацию, у меня идут мурашки по коже,
как будто вселенная мне подсказывает «Да, чувак, это твое!» Также я все поддаю
нереальному анализу, слова, действия и потом делаю выводы, зачастую они на 99\%
верны.

Помню, ты про Ганди недавно читал.

Да, я очень люблю книги об исторических персонажах. Потому что это реальная
жизнь, пережитые ситуации, там есть что почерпнуть!

Чем тебе Ганди понравился?

Это просто уникальный чувак, который изменил историю Индии и проповедовал
мирные методы борьбы за права!

Я, например, огромный фанат Теслы, для меня это величайший человек, который
руководствовался знаниями, наукой и интуицией. Для меня это Зевс современности!

Тесла для тебя круче, чем Эдисон?

Да, потому что про Эдисона разные вещи говорят – я про него, правда, меньше
читал. Когда я что-то читаю – оно мне интуитивно либо нравится, либо нет. Я
думаю, что интуиция – это опыт прошлых жизней, то есть что-то ты, возможно,
знал тогда, и когда ты натыкаешься на эти вещи сейчас, они кажутся тебе
родными.

Что именно ты имел в виду, когда сказал, что черпаешь знания из космоса?

Все люди, которые хоть немного разбираются в энергии, знают, что такое
информационный банк данных, это сгустки энергии в космосе.

Как брать эти сгустки?

Люди, которые в этом не разбираются, смеются над этим, хотя ученые давно
доказали, что у нас есть аура, есть энергетическое поле, что есть нити,
которыми мы связаны с космосом. 

Я много читал различных материалов, но мой мозг, к сожалению, не успевает всё
запоминать. Я читаю, мне что-то интуитивно кажется родным, и я это воспринимаю.

Например, формула силы мысли.

Почему христианство, церковь, взяла истинные знания и облекла в какие-то
понятные человеку формы? Жили, знали и пользовались этими знаниями и до них. Я
недавно с родителями спорил про Крещение – еще до рождения Христа люди знали,
что в этот день молекулы воды на Земле меняют структуру, перезапускаются, такой
вот природный процесс происходит по всей планете. И это знали уже очень давно,
а церковь это использовала в своих целях. И таких примеров много – например, в
чем суть свячения куличей? Мы на 70 процентов состоим из воды, которая имеет
свою структуру. На Пасху воду намаливают, молитва структурирует молекулы воды,
и этой водой святят людей, пищу. Вода впитывается через поры, и также ты ее
вместе с едой употребляешь – структурированные молекулы попадают в тебя, таким
образом происходит процесс очищения.

То же самое и на Крещение – в этот день вода природным образом меняет свою
структуру, люди купаются, через поры эта вода попадает в тело человека –
доказано, что эти люди живут дольше в итоге.

Когда я учился в десятом классе, я посмотрел фильм «Секрет». И это изменило мою
жизнь, я поверил в то, о чем там шла речь. Рекомендую всем! Я как посмотрел его
тогда – с тех пор не пересматривал. Все люди, которым я раньше его советовал, а
они смеялись надо мной – все они теперь живут по законам, которые описываются в
этом фильме. Чем раньше каждый посмотрит этот фильм, серьёзно воспримет то, о
чем там говорится, будет жить согласно советам, которые даются там – тем лучше
он проживёт эту жизнь. Очень многие люди замечают – подумал о ком-то – и
человек тебе звонит. А как так происходит? А это и есть сила мысли – мы все
связаны этими ниточками, невидимыми проводами, и в моей жизни было очень много
примеров, подтверждающих существование этой связи. Подумал о ком-то – и
встречаешь человека.

Но это же могло быть совпадение?

Нет!  Когда я осознал этот процесс – то начал просто его юзать, как телефоном
пользуюсь.

Но если я сейчас попрошу тебя связаться с кем-то, чтобы он набрал тебя – ты
вряд ли сможешь это сделать.

Для этого нужно знать формулу. Вот все думают, что нужно только подумать. Закон
имеет три составляющих. Первое – думать, чётко представлять конечный результат.
Второе – говорить, проговаривать вслух. И третье – делать. Вот все люди из-за
своей лени думают, говорят, и ничего не делают для достижения целей.

В 2011-м году мы едем с ребятами кататься на лыжах, и я им говорю: пацаны, я в
течение ближайших двух лет выступлю с Бастой на одной сцене! А я на тот момент
не то что начинающий рэпер – я суперначинающий рэпер! И все мне: да ну Сань, ну
ладно, да что ты гонишь! Баста для нас был нереальным человеком из другого
мира. Проходит полгода – и я выступаю у него на разогреве. А дело в том, что я
не просто думал об этом и говорил, а что-то делал. Писал песни, искал связи –
но я делал! Можно это назвать удачей, но я в удачу не очень верю. Удача – это
следствие концентрации энергии человека. Я раньше был неудачливым человеком, а
сейчас я сконцентрирован, можем сыграть с тобой на ю-зе-фа, я могу проиграть –
но я всё-таки верю в победу (смеётся).

История о том, как я попал тогда на разогрев к Басте – это парадокс!

Это тогда в Ялте, когда Фэйм отказался идти на Газгольдер?

Да. Приведу пример, чтобы было понятно, как это работает. Я пишу песни на
университетской студии, выступаю в НАУ на разных конкурсах – и тут меня просят
спеть песню с девочкой, я как раз написал, называется «В НАУ весна». Я выступаю
и забываю оба куплета. Просто стою и мелю какую-то фигню типа «Форсаж, подняли
руки!», и жду, пока начнется припев. Это было ужасно! После выступления ко мне
подходит Рома, мой нынешний партнер по бизнесу – Рома занимается вообще-то
пластмассовыми трубами, но он такой, искренний фанат творчества, спонсирует
разные конкурсы, всем помогает. Он подходит ко мне и говорит – чувак, это было
круто! Я говорю – ты что, гонишь, я из двух куплетов забыл два куплета! А он
мне: какая разница, от тебя идёт энергия! И продолжает: у меня есть кент,
который знает Басту! Я засомневался – каким образом, Рома ж занимается
пластмассовыми трубами, ну бред же какой-то! Выяснилось, что Рома знает Валика,
а Валик учился в НАУ, тоже играл в КВН, сам он из Магадана, но делает концерты
Басты в Украине.

Мы встретились, я передал им диск с демо. Они меня закинули в конкурс на
разогрев – я, понятно, попросил друзей проголосовать, выиграл и поехал к Басте
на разогрев. Ещё полгода назад мне казалось, что сидеть рядом с Бастой в одной
гримерке – это космос! Это всё равно если б мне сейчас сказали, что я через
полгода буду в одной гримерке с Эминемом.

Важно: все, что я говорю, это мои личные умозаключение, которым время дает
поправки. Истина это или нет, но я в это искренне верю!

Насколько я знаю, тема спорта, единоборств не чужда тебе?

Мне она очень близка! У меня  достаточно неплохо всё получается, мне не хватает
физухи, но в плане техники всё нормально. Когда есть возможность – я хожу на
тайский бокс.

Когда ты в последний раз бил человека?

Давно не бил, только на тренировках. Недавно такая дурная ситуация была –
спасли мы собаку в гаражном кооперативе. Я прихожу – а собаки нет, начальник
тамошний её выгнал. Я пришёл и по-человечески начал – если не устраивает тебя
эта собака, я её заберу, я её лечил дважды, потратил 10 000 гривен, чтоб она не
сдохла, а ты её выгнал. Не устраивает она вас как охранник – я её заберу.

Прибегает тут как раз эта собака, а этот начальник давай орать на неё: пошла
нахуй отсюда! Я ему говорю – я тебе сейчас просто дам пизды! А он здоровый,
крупный мужик, не знаю, лет 55, наверное. И он начал: ты – мне?! Да я сидел! А
я ему говорю – мне все равно, где ты сидел, я тебя ёбну раз, и ты на жопу
сядешь. Мне его бить не хотелось, хотя я вообще не сомневаюсь в своих силах, но
мне было очень неприятно ругаться. Я всегда стараюсь избежать конфликта.

При всех своих возможностях и связях я мог бы хейтеров вывозить в лес пачками,
но я считаю, что это пережитки 90-х и надо от этого избавляться, это не то, к
чему я стремлюсь.

Я хочу писать песни, ездить в детские дома, снимать фильмы, и тратить время
только на это. Но все-таки иногда хочется кому-то сломать нос!(улыбается)

Кстати, у вас в этом году должна выйти криминальная комедия?

Да, я хочу снять фильм – криминально-авантюрная комедия, в которой я предложил
сняться Киевстонеру. Я ему рассказал идею – ему очень понравилось, история
реально очень смешная и босяцкая, что-то в жанре фильма «Такси».

Сценарий твой?

Да, я хочу его целиком написать, просто времени не хватает – нужно же ещё
альбом писать, куча других дел. Даже тёще моей понравилось – хотя наш сериал
ей, например, не по душе пришелся.

Многие говорят, что ты коньюктурщик. Ты видишь грань между «быть актуальным» и
«быть коньюктурщиком»?

Ее нет. Как сказал Эйнштейн – «Все относительно». У всех людей разное
восприятие, и о чем бы ты ни написал, разные люди это всегда воспримут
по-своему. Доказывать кому-то что-то – пустая трата времени. Люди из моего
окружения знают, как я фанатично верил в Майдан, как я хотел что-то изменить в
этой стране. Сколько я времени там провел, находясь очень часто на самой
передовой в маске, чтобы никто даже не знал, что это я. Самая большая проблема,
что желания и стремления было много, а как это сделать – я не знал.

Сейчас знаешь?

Да. Потому я езжу по школам и университетам, им мы еще можем что-то донести,
что-то поменять. Если посмотреть глобально, страна – это человек. Если у
человека психологические проблемы – он никогда не будет богатым, ты хоть
миллион ему дай – он все равно его прогуляет, пропьет, проиграет в карты. Я не
мог поверить, что у нас настолько воровская страна, я как-то этого не замечал.
Я оглянулся – даже вокруг меня все на мутках, даже те, кто был со мной на
Майдане, все на мутках, оправдывая это тем, что нужно выживать. А я так никогда
не делал, я всегда верен своим словам – мне в 10 раз тяжелее, я работаю в 10
раз больше, чем другие, но зато у меня совесть чиста. 

Одна из самых больших ошибок нашего народа – я сам рос с такими мыслями – что
все богатые люди плохие, что богатство это плохо. Те богатые люди плохие, кто
получил деньги нечестным путем, а я знаю таких, кто не воровал, кто работал
сутками напролет. Да, тех, кто честно заработал своё состояние – их пять
процентов. Но я хочу быть шестым процентом, а вы можете стать 7-м, 10-м, 30-м!
Деньги – это не плохо, если ты их честно заработал. Почему я должен стесняться
того, что я зарабатываю? Я что, кого-то обманул или ставлю у входа в клуб
напёрсточников? Хотите покупать одежду – покупайте, хотите ходить на концерты –
ходите.

Помню, ты расстраивался из-за распада Легенды Про, говорил, что многие ребята
не дотянули до вершины. Как по-твоему, кто в Украине не дотянул?

Знаешь, я очень боюсь упустить момент. Потому что это такая тонкая грань, когда
ты просто можешь стать неинтересным. Мне кажется, так случилось когда-то у
«Тартака» – это была одна из самых любимых групп у меня в детстве,   я очень
люблю Саню Положинского, дружу с ним, но я вижу, что у них когда-то так
случилось. Если сравнить «Бумбокс» и «Тартак» – то «Бумбокс» всё-таки большего
достиг. У меня действительно сейчас очень сложный период, можно оказаться
наверху, либо упасть и не факт, что уже получится подняться. У многих так и
случилось, я общался с ребятами из ВУЗВ, которые в конце 90-х собирали стадионы
по Украине, с VovaZiLvova, с ТНМК и все они в один голос говорят: «Мы просто
однажды проебались!». Я не хочу наступать на чужие грабли, потому мне
приходится много работать.

Расскажи о премии М1 и том, как вам предложили спеть с Зибровым.

У клипа «Хулиганом» были хорошие рейтинги, хотя я вообще такие песни буквально
штамповать могу. «Хулиганом» – это та же «Пидарасия», только без матов, более
форматная, не такая скандальная.

На М1 мы должны были идти в номинации «Альтернатива» – там то ли мы, то ли
«Грибы», что-то такое, непонятно. Нам позвонили и предложили спеть с Зибровым –
а сейчас какая-то такая тема пошла, Зибров хайпит, над ним все смеются, старый
дед шутит сам над собой, ги-ги, га-га, но я не хочу, чтобы рэп воспринимали как
«ги-ги, га-га». Я считаю, что если уж появляться на М1 – то чтоб вышли лучшие
би-бои страны, чтоб мы использовали всю базу, что есть в украинском хип-хопе по
разным направлениям, чтоб мы вышли и ёбнули так, чтоб все охуели и поняли, что
хип-хоп это очень серьёзно, и что эти люди настолько талантливы, что им не
нужно с Потапом соревноваться.

Грубо говоря, ты не согласился, потому что тебе за культуру обидно?

Да, мне вообще за неё очень обидно, потому что культура дала мне этот шанс в
жизни и я ей обязан. Я хочу сделать фестиваль, видеопроект для YouTube, в
котором все будут соревноваться при помощи песен, а не баттла, что-то вроде
«Битвы за респект». У меня в голове есть определенная концепция – она не до
конца сложена, организуем, как только я найду финансирование.

Что за фестиваль?

Хочу собрать лучших, тех, кто переживает за культуру, тех, кто что-то для неё
сделал, хочу всех сдружить. Проблема нашего хип-хопа – никто ничего не достиг и
считает, что он самый крутой, а на самом деле нет. Единственный путь –
объединение, в России точно так же всё было, кто больше работал – тот и вылез.
А у нас все сидят по углам, у всех гордыня, у всех эго – а это просто
комплексы. Если ты крут – возьми майк и докажи это. Я мечтаю всех объединить,
потому что от этого каждый что-то получит. Я, как человек, который протоптал
тропу, который знает механизмы, имеет ресурс, имеет определенные связи – я
готов этим делиться, давайте друг у друга учиться, вместе что-то делать! Я это
делаю потому, что не хочу жить в иллюзии, я не любитель баловать своё эго. Мне
некомфортно жить в таком культурном обществе – это главная причина. Мне не с
кем делиться, некем вдохновляться, некого вдохновлять.  

Ты говорил, что как раз начал строить своё новое окружение.

Ну да, но я только начал!

А кто в твоём новом окружении?

Fame, Smek, один из лучших райтеров в Украине, мы дружим с DJ Scream и DJ Vag.
Когда я в 2011-м выступал на разогреве у Басты, там проходил фестиваль Ялта
Summer Jam, я смотрел на этих людей, как на богов, и запомнил, что они
нереальные авторитеты!

Но что у нас в стране знают про Скрима и Вага? Ничего. Я выложил фотку – мол,
смотрите, у меня на студии лучшие хип-хоп персонажи этой страны, а все пишут: а
кто это? Вот такая, к сожалению, наша реальность. Моя задача, чтобы об этих
людях узнала вся страна и чтобы все с ними фотографировались, как с
футболистами. А для этого нужно построить индустрию, нужно всем собраться,
пробить эту стену, чтобы нас зауважали, чтобы восприняли, чтобы постановки
танцевальные делал не Коляденко, который боком ходит, а какой-то наш известный
би-бой.  

Мне хочется, чтобы культура шла в массы – мы на самом деле в сто раз сильнее,
чем все они! Но каждый сидит – я крутой, зачем мне это доказывать? А нужно
доказывать, как ни крути! Создавать школы, создавать лейблы, подтягивать людей. 

Вот, Лоик уже открыл школу рэпа.

Отлично! Он научит каким-то основам – но есть же куча других нюансов, помимо
того, как рифмовать. Я готов его поддерживать, готов приехать, передать детям
свои знания.  

Я открыт для всех адекватных созидательных людей. Я не собираюсь с кем-то
читать про падики – потому что эти падики плохо закончатся, потому что это не
творчество, это обида на жизнь. Задача – всех объединять, обнимать, дарить
добро, делать музыку. Почему я этим занимаюсь? Потому что я от этого кайфую,
потому что это делает меня счастливым.

Либо мы все вместе начинаем работать и вылезем из этой ямы, либо мы навсегда
останемся в ней. Некоторые начинают: вот, вы Майдан затеяли! Но не я голосовал
за Порошенко, Ляшко или Кличко, которые приходили на Грушевского – а их там
просто посылали. Вы же сами не подготовились к выборам, или вообще не сходили
на них и наголосовали такого, что страдают теперь все.

Мне некомфортно жить тут, мне некомфортно ходить по улицам, где все
озлобленные, несчастные – я преследую эгоистическую цель, я хочу жить
счастливо, среди радостных людей. Я попадаю в негативное энергетическое  поле у
себя на районе, где дети в 9 утра выходят на площадку играть, а там мужики
сидят, водочку пьют. Я им – мужики, вы что, охуели? И начинается рамс.

Ты думал когда-нибудь об эмиграции?

О полной эмиграции я не думаю – но я не вижу смысла прожить жизнь, сидя в одной
стране. Я хочу поездить по миру, попутешествовать, пожить где-то, я хочу
попробовать свои силы на Западе.

В планах на ближайшие пять лет у меня расписано дважды съездить пожить в Штаты
на 3-6 месяцев. А потом хочу на год-два съездить туда пожить и поработать,
попробовать свои силы там. Почему никто из наших не стрелял на Западе? Там
уникальные культурные вещи их народа – если ты всё это не будешь использовать,
тебя как родного там не воспримут. Чтобы там стрельнуть, нужно прочувствовать,
полюбить их культуру. Я там еще не был – но уверен, что мне там понравится, у
нас летом планируется тур по Штатам, вот, впервые слетаю, посмотрю.

Повторюсь – главная причина того, что я делаю – мне некомфортно жить.  Я не
знаю, как живется всем этим депутатам, а мне некомфортно – я постоянно вижу
женщин, которые плачут, бедных пенсионеров, недовольных людей, и мне самому
становится плохо, а я хочу, чтоб мне было круто, чтобы я радовался, улыбался, и
танцевал посреди улицы. 

Андрей Freel Шалимов
