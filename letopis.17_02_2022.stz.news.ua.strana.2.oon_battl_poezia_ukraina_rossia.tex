% vim: keymap=russian-jcukenwin
%%beginhead 
 
%%file 17_02_2022.stz.news.ua.strana.2.oon_battl_poezia_ukraina_rossia
%%parent 17_02_2022
 
%%url https://strana.news/news/377439-kislitsa-i-vershini-ustroili-v-oon-diskussiju-s-pomoshchju-stikhov.html
 
%%author_id poluljah_natalja
%%date 
 
%%tags evtushenko_evgenij,krasota,oon,poezia,rossia,stalin_iosif,stihi,ukraina,zabolockij_nikolaj.poet
%%title Представители Украины и России устроили в ООН поэтический баттл, процитировав Евтушенко и Заболоцкого
 
%%endhead 
 
\subsection{Представители Украины и России устроили в ООН поэтический баттл, процитировав Евтушенко и Заболоцкого}
\label{sec:17_02_2022.stz.news.ua.strana.2.oon_battl_poezia_ukraina_rossia}
 
\Purl{https://strana.news/news/377439-kislitsa-i-vershini-ustroili-v-oon-diskussiju-s-pomoshchju-stikhov.html}
\ifcmt
 author_begin
   author_id poluljah_natalja
 author_end
\fi

\ii{17_02_2022.stz.news.ua.strana.2.oon_battl_poezia_ukraina_rossia.pic.1}

На заседании Совбеза ООН, посвященному ситуации вокруг Украины, состоялся
своеобразный стихотворный баттл. В нем приняли участие представители Украины
Сергей Кислица и России Сергей Вершинин.

Кислица процитировал строки стихотворения \enquote{Наследники Сталина} (1961)
советского поэта Евгения Евтушенко.

\enquote{Пусть мне говорят: \enquote{Успокойся...} - спокойным я быть не сумею.
Покуда наследники Сталина живы еще на земле, мне будет казаться, что Сталин -
еще в мавзолее}, - продекламировал Кислица.

Вершинин ответил словами Заболоцкого из шедевра \enquote{Некрасивая девочка} (1955).

\enquote{Я вот очень люблю стихи про красоту Николая Заболоцкого. Помните это
четверостишие? \enquote{И если так, то что же красота? И отчего ее обожествляют
люди?  Сосуд она, в котором красота, или огонь, мерцающий в сосуде?} } -
прочитал Вершинин отрывок из произведения.

Он добавил, что сравнил бы красоту с мудростью.

\enquote{Надо, чтобы у нас принимали мудрые решения, в том числе, что касается
урегулирования на востоке Украины на основе Минского комплекса мер}, -
резюмировал дипломат из РФ.

Ранее \enquote{Страна} сообщала, что Вершинин в зале заседания ООН констатировал, что
Украина отказывается выполнять Минские соглашения.

А заместитель генсека организации Розмари Дикарло на этом же мероприятии
заявила, что напряженность в Украине достигла пика с 2014 года. В связи с этим
она призывала все стороны к сдержанности.
