% vim: keymap=russian-jcukenwin
%%beginhead 
 
%%file slova.priroda
%%parent slova
 
%%url 
 
%%author 
%%author_id 
%%author_url 
 
%%tags 
%%title 
 
%%endhead 
\chapter{Природа}
\label{sec:slova.priroda}

%%%cit
%%%cit_head
%%%cit_pic

\ifcmt
  tab_begin cols=1
     caption Когда-то здесь жили почти триста человек. Фото: Facebook/zapovidnyk2018
     width 0.4

     pic https://img.strana.ua/img/article/3415/chernobylskij-zapovednik-pokazal-11_main.jpeg
     width 0.35

     pic https://strana.ua/img/forall/u/11/33/%D1%87%D0%B5%D1%80%D0%BD(1).png
     width 0.7

  tab_end
\fi

%%%cit_text
Чернобыльский радиационно-экологический биосферный заповедник показал,
как \emph{природа} и время без следа уничтожают то, что когда-то создал человек.  В
Фейсбуке пресс-служба заповедника опубликовала фото села Новоселки, в котором
некогда, до аварии на Чернобыльской АЭС, жили почти три сотни людей. А сейчас
на их присутствие почти ничего не указывает. \enquote{Теперь там - только
полуразрушенные временем дома и тишина. Деревянные домики окутывает паутина
времени, которая каждый сезон маскируется в разные цвета и формы. Летом она
пышная и зеленая. Издалека даже может показаться, что ничего не изменилось за
несколько десятилетий. Но изменилось. Здесь свое забирает \emph{природа}...} -
говорится в посте заповедника
%%%cit_comment
%%%cit_title
\citTitle{Чернобыльский заповедник показал фото, как природа побеждает цивилизацию}, 
Наталья Полулях, strana.ua, 01.07.2021
%%%endcit

%%%cit
%%%cit_head
%%%cit_pic
%%%cit_text
Посеред тих звалищ величної \emph{природи} виднілися подекуди медведячі стежки,
протоптані від давніх-давен, вузькі, але добре втоптані, густо засіяні
вибіленими кістками баранів, оленів та всякої іншої звірини. Максим держався
тепер позаду бояр; він раз за разом обходив усі становища, осмотрював сліди,
щоб догадатися, чи вони свіжі, чи ні, підпомагав, заохочував утомлених,— і
тільки на нім однім не знати було ніякої втоми. З подивом позирала на нього
Мирослава, коли він переходив коло неї, і хоч багато досі видала вона молодців
і сильних, і смілих, але такого, як Максим, що сполучав би в собі всі прикмети
сильного робітника, рицаря і начальника,— такого їй досі не траплялося бачити
%%%cit_comment
%%%cit_title
\citTitle{Захар Беркут}, Іван Франко
%%%endcit

%%%cit
%%%cit_head
%%%cit_pic
%%%cit_text
ПреЗЕдент Украины на этой неделе явно с удовольствием вернулся к своему
любимому делу шоуменства, выступив на климатическом саммите в Глазго. Но годы,
проведенные в кресле главы "державы", явно дали о себе знать: выступал он
уныло.  Да и текст экс-игрецу на рояле писали, похоже, примерно тем же, чем он
на нем играл. Все действо напоминало анекдот про студента, который сумел
выучить лишь билет про строение блохи и пытался на этой блохе вытянуть весь
экзамен, избегая ответов на конкретные вопросы и делая невероятные допущения.
В речи Зеленского такой "блохой" стала "российская агрессия". Подстроив изрядно
всем поднадоевший жупел под тематику мероприятия, в Офисе президента сочинили
страшилку про "экобомбы", которые "агрессоры" якобы заложили в Крыму и
Донбассе. В частности, по словам Зеленского, уникальные флора и фауна
полуострова находятся под угрозой из-за размещения российской военно-морской
базы, а в Народных Республиках происходят затопления шахт и деградация почв. О
том, почему \emph{природе} Крыма не угрожала украинская база и как украинские
власти задолго до войны закрыли шахты в Донбассе, Владимир Александрович
почему-то умолчал
%%%cit_comment
%%%cit_title
\citTitle{НЕДЕЛЯ ГЛАЗАМИ ЭКСПЕРТА: Ротация блокираторов, экобомбист и комедия Гайдая}, 
, lug-info.com, 07.11.2021
%%%endcit
