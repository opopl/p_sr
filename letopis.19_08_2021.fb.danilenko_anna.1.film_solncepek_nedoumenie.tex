% vim: keymap=russian-jcukenwin
%%beginhead 
 
%%file 19_08_2021.fb.danilenko_anna.1.film_solncepek_nedoumenie
%%parent 19_08_2021
 
%%url https://www.facebook.com/anna.danilenko.144/posts/4488443317843677
 
%%author_id danilenko_anna
%%date 
 
%%tags donbass,film,film.solncepek.donbass,kino,kultura,ukraina,vojna
%%title Фильм "Солнцепек" - чувство недоумения
 
%%endhead 
 
\subsection{Фильм \enquote{Солнцепек} - чувство недоумения}
\label{sec:19_08_2021.fb.danilenko_anna.1.film_solncepek_nedoumenie}
 
\Purl{https://www.facebook.com/anna.danilenko.144/posts/4488443317843677}
\ifcmt
 author_begin
   author_id danilenko_anna
 author_end
\fi

Когда сербы рассказывают о войне, в первую очередь они говорят про «не ожидали
от соседей». «Соседи» тут - не в фигуральном смысле, а в прямом. Когда люди,
жившие рядом с тобой, в соседней квартире, с которыми вместе отмечали
праздники, ходили друг к другу в гости, пришли к тебе без приглашения, ночью и
с оружием.

\ifcmt
  ig https://avatars.mds.yandex.net/get-kinopoisk-image/4774061/07adcbdf-8b42-420b-ac0e-c3d01cc5d23d/300x450?fbclid=IwAR1o3PKfrrkO4tZ6S2R9ueEJrQR93OUncSCgBDZ5XsbpONPgQVDEfsD22VY
  @width 0.4
  @wrap \parpic[r]
\fi

«Солнцепёк» - фильм именно об этом. Только не про Югославию, а про Украину.

В нем много всего: трусость, садизм, жестокость, дурь, хатаскрайничество, элементы героизма...

Основное - почему люди идут воевать. Против тех, к кому ещё вчера ходили в гости.

И воевать они идут по-разному: кто-то в составе садистского батальона и под
наркотой, кто-то - по дури, не понимая, куда и зачем, кто-то просто
предпочитает войну мирной жизни, кто-то - защищает своё. Показали практически
все варианты.

Особое внимание уделили нацбольной швали со всего мира. 

В целом, фильм нельзя назвать просто жестоким, он - жуткий.

Мне лично не понравилось. Совсем. Фильм получился размытым и кровавым. Насчёт
анонса, звучащего как «Донбасс. Рассекреченные факты» - да какие они
«рассекреченные»? Обо всем показанном трубят все СМИ года так с 2014, начиная с
логики этой войны и заканчивая хатаскрайничеством. И жуткими кадрами гибели
мирных жителей, включая детей.

Искренне удивляют на полном серьезе сравнивающие его с «Иди и смотри» Климова.
По мне, так вообще ничего общего. Климов берет именно психологически, здесь же
- сплошная кровавая картинка с не всегда логичными и частенько наигранными
эмоциями героев. 

Зачем снимали фильм, тоже непонятно: взрослые, интересующиеся украинской
тематикой, и так в курсе всего показанного, детям же или подросткам, с целью
понимания происходящего или, лет через 10, произошедшего, ради понимания
реальной истории, этот фильм нельзя показывать просто из этических соображений,
слишком много зверств и крови, одно начало чего стоит. 

К слову, первые 15-20 минут фильма - самые жуткие, впервые радовалась перерывам
на рекламу. 

Единственно, что получилось действительно здОрово - это показать нацбольный
интернационал, включая выходца из России, причём, буквально несколькими
кадрами. Кстати, этот самый выходец - единственный идейный во всем фильме. Вот
освещение факта наличия этого самого интернационала - реально полезная вещь,
учётом того, что в СМИ обычно об этом не говорят, а так, вскользь упоминают.
Вполне логичный акцент любой современной, да и не только, войны - наёмники и
садисты со всего мира едут в любую горячую точку. Увы, и восток Украины не
избежал этого. 

Странно, что не упомянули об объявлениях, предлагающих услугу «охоты на людей»
в этом регионе для иностранцев. Стоило.

И ещё. Если про реалии.

То, что на территории ДНР/ЛНР бродили садисты плюс нацболы со всего мира -
действительно факт. Тот самый случай, когда доказательств немеряно. И искренне
непонятно, как мужчины из этих регионов могли вместо того, чтобы самим защищать
свои семьи, ехать работать в Россию, оставив родных и близких на родине. Ещё и
требовать от России ввести войска.

Мир перевернулся. 

В общем, фильм оставил какое-то чувство недоумения, что ли...
