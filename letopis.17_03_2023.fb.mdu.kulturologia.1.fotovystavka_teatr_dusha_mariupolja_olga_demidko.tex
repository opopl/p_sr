%%beginhead 
 
%%file 17_03_2023.fb.mdu.kulturologia.1.fotovystavka_teatr_dusha_mariupolja_olga_demidko
%%parent 17_03_2023
 
%%url https://www.facebook.com/kulturologiyaMDU/posts/pfbid0wdTqXmWsvt6zm8jUqrUcJAXfgLFJTNbsWckwKmgtanJDSU5JHCggVpqixbCrAymdl
 
%%author_id mdu.kulturologia
%%date 17_03_2023
 
%%tags mariupol,kiev,teatr.kiev.franka,olga_demidko,vystavka
%%title Фотовиставка "Театр. Душа Маріуполя" - Ольга Демідко
 
%%endhead 

\subsection{Фотовиставка \enquote{Театр. Душа Маріуполя} - Ольга Демідко}
\label{sec:17_03_2023.fb.mdu.kulturologia.1.fotovystavka_teatr_dusha_mariupolja_olga_demidko}

\Purl{https://www.facebook.com/kulturologiyaMDU/posts/pfbid0wdTqXmWsvt6zm8jUqrUcJAXfgLFJTNbsWckwKmgtanJDSU5JHCggVpqixbCrAymdl}
\ifcmt
 author_begin
   author_id mdu.kulturologia
 author_end
\fi

Доцентка кафедри культурології, кандидат історичних наук Olga Demidko  15 та 16
березня взяла участь у низці важливих заходів.👏 Зокрема, 15 березня вона
долучилася до публічного обговорення \enquote{Театр. Культурний код Маріуполя}. 

%\ii{17_03_2023.fb.mdu.kulturologia.1.fotovystavka_teatr_dusha_mariupolja_olga_demidko.pic.1}

З прямою трансляцію діалогової платформи можна ознайомитися за посиланням:
\url{https://www.youtube.com/live/285OaaUgLec?feature=share}. А 16 березня Ольга
Демідко промовою  відкрила у Києві фотовиставку \enquote{Театр. Душа Маріуполя},
підготовлену Департаментом культури маріупольської міської ради до річниці
жахливої трагедії. Освітянка розповіла і про історію будівництва приміщення
драмтеатру і про жахливу трагедію, яка сталася рік тому.💔 На фотовиставці були
представлені світлини з архівів Донецького академічного обласного драматичного
театру (м.Маріуполь), з фондів Маріупольського краєзнавчого музею та особистого
архіву народної артистки України Світлани Отченашенко, які дослідниця збирала з
2014 року і змогла вивезти з Маріуполя.  

Також були представлені фото Маріупольської міської ради, m. EHUB,
маріупольських фотографів Льва Сандалова, Євгена Сосновського, художниці з
освітлення драматичного театру Євгенії Забогонської та з відкритих джерел.
Ольга Демідко наголосила, що \enquote{попри численні намагання окупантів знищити
культурну спадщину Маріуполя, ми мусимо зберегти нашу пам'ять, культуру та
історію, яка створювалася не одне століття}.🙏🇺🇦 Виставка працюватиме
протягом місяця у Національному академічному драматичному театрі ім. Франка
(Київ, площа Івана Франка, 3).

❗ Також доцентка перед Національним академічним театром опери та балету
України ім. Тараса Шевченка  виступила на мирній акції вшанування всіх загиблих
маріупольців у драмтеатрі. 

🕯\#РодинаМДУ  \#Україна \#Маріуполь \#Київ

\#futurestartswithyou \#memories \#mdu\_cultural\_studies \#МДУ  \#нашівикладачі

\#наука  📝👨💻😊
