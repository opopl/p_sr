% vim: keymap=russian-jcukenwin
%%beginhead 
 
%%file 26_09_2021.fb.zagrebelnyj_igor.1.malorossia.cmt
%%parent 26_09_2021.fb.zagrebelnyj_igor.1.malorossia
 
%%url 
 
%%author_id 
%%date 
 
%%tags 
%%title 
 
%%endhead 
\subsubsection{Коментарі}

\begin{itemize} % {
\iusr{Artem Dorotenko}
Вживання терміну «малорос» оцими «ліцензованими патріотами» - це така збочена форма самоприниження. Тільки «ліцензовані» цього не усвідомлюють.

\iusr{Andrii Tolstoi}
Те ж саме зі словом "руський", яке деякі використовують як антонім до "український".

\iusr{Сергій Єрмоленко}

"В ім'я Отця і Сина і Святого Духа, Бога в Трійці Святій славімого.

Нехай залишиться на вікопомну славу і пам'ятку Війська Запорізького і всього
народу малоросійського." 

\url{http://gska2.rada.gov.ua/site/const/istoriya/1710.html}

\iusr{Ілля Камишанський}
Історію - вивчати!
Героїв - шанувати!
Україну - любити!
Нарід - цінувати!

\iusr{Andrii Tolstoi}
Ігоре, що ти думаєш про гіпотезу Кураєва з цього питання?

\href{https://www.youtube.com/watch?v=8ehI9YcuRoU}{%
Протодиакон Андрей Кураев об Украине, youtube, 08.05.2016%
}

\begin{itemize} % {
\iusr{Ігор Загребельний}
Звичайна російська антипольська дурня.

\iusr{Andrii Tolstoi}
\textbf{Ігор Загребельний} а як щодо Греції. Чи справді є підстави проводити
паралель між мікро- і макро- Елладою та мало- і велико- Росією (Руссю)?

\iusr{Ruslan Kusyk}
\textbf{Андрій Толстой} 

\href{https://youtu.be/5Zna0WTzFtY}{%
Як Русь стала Україною / Історія Русі / Історія України, імені Т.Г. Шевченка, youtube, 23.08.2020%
}

Тут згадуються ці паралелі.

\iusr{Павел Недашковский}
\textbf{Андрей Толстой} 

Μεγάλη Ελλάς — это итальянские колонии греков. Применима ли эта аналогия к паре
«Малая Россия» (митрополия) и «Великая Россия» (колонии) или речь идёт о паре
«Великороссия» (вся митрополия) и «Малороссия» (часть митрополии) — имеет место
дискуссия

\iusr{Павел Недашковский}
\textbf{Андрей Толстой} 

Μεγάλη Ρωσία (Великая Русь) использовалось византийской канцелярией для
обозначения юрисдикции митрополита Киевского т.е. как обозначение Ρωσία
(Россия) отличая ее:

\begin{itemize}
  \item a) От обособившейся Νέα Ρωσία (новой России) т.е. епархии Чернигова
  \item b) От обособившейся Μικρά Ρωσία (Малой России) т.е. епархий Галича
\end{itemize}

\end{itemize} % }

\iusr{Alexander Steppenwind}

Справедливості заради трохи розставлю крапки над і.

Ну по-перше не слобожанці, а слобожани.

По-друге, коли це "У/Вкраїна прийшли з заходу коли назва ця згадуються ще до
ляшсько-литовської унії.

По-третє колонізація донського Дикого Поля була не лише оралом та потом, але і
шаблею і кров'ю бо через Слобідську Україну пролягав Муравський шлях з його
Ізюмським і Кальміуським відгалуженнями. Це до нібито абсолютно спокійного
життя слобідських "зрадників"

В-четвертих українські слобідські козаки як і запорозькі городові, не одразу
стали лояльними. В 1670 році перший і найсхідніший слобідський полк (і на той
момент одірваний не лише од усієї Малої Ро́сії але і решти слобідських полків) -
Острогозький, в усобленні його полковника Івана Дзиньковського та старшини,
перейшов на бік повсталих донських козаків Степана Разіна.

Тому можу з упевненістю сказати, шо Слобожанщина/Слобідщина - органічне
продовження Наддніпрянщини.

До терміна Малоросія то тут так само як з Руссю - вузьке та широке розуміння. В
широкому нас так само називали "малорусским/малороссийским краем".

І да, я теж вороже ставлюся до негативного забарвлення наших історичних,
традиційних офіційних імен, назв та екзоетнонимів.

\begin{itemize} % {
\iusr{Ігор Загребельний}

1) Назагал погоджуюся. Втім, я недолюблюю філологів і вважаю суфіксальний
плюралізм досить легітимним.

2) Ідеться саме про ранньомодерну добу з її своєрідною реконкістою. В ході цієї
реконкісти поняття України зміщувалося на схід в міру освоєння земель. Спершу
Україною могло вважатися навіть Поділля (яке опісля їй виразно
протиставлялося), потім - Лівобережжя (яскравий приклад -
Україна-Вишневеччина), потім - Слобожанщина. PS. Нагадаю, що я прихильник
"річпосполитського центризму" у розумінні ранньомодерної історії, що, водночас,
не змушує мене абсолютним чином очорнювати ті сторінки нашої історії, які
пройшли під скіпетром московських/російських монархів (що видно хоч би з цього
допису).

3) Не зрозумів, який це має стосунок до допису.

4) Тут ідеться не про історичні факти, а про їхнє пізніше сприйняття
слобідськими елітами.


\iusr{Alexander Steppenwind}
\textbf{Ігор Загребельний} 

2) Мав на увазі саме походження "Вкраїна" яке було розповсюджено "знизу", а не
"згори", як Мала Ро́сія.

Щодо доби та Лівобережжя - зрозумів.

P.S. Я пам'ятаю і важко це не помітити, бо є прихильником козакоруського
центризму, як нащадка києворуського, і ніхто нікого не очорнює, а лише говоримо
та наводимо історичні факти які є задокументованими.

3) Певно надто сфокусувався на вірші твого земляка з Полтавщини (чи
Вишневеччини?))

4) Як на мене це вже особисті розбірки в залежності від рівня інкорпорування в
московський імперський проект.

\iusr{Ігор Загребельний}
\textbf{Alexander Steppenwind} "Певно надто сфокусувався на вірші твого земляка з Полтавщини". Я припускав, що це реакція "місцевого патріотизму"))

\iusr{Alexander Steppenwind}
\textbf{Ігор Загребельний} ну-с, своєрідна форма.

\iusr{Павел Недашковский}
\textbf{Alexander Steppenwind} могу ошибаться, но:

1. «Украина» происходит от «Оукраина» — см., например, Шевелёв (его в
великодержавном шовинизме обличить достаточно сложно)

2. «Μικράς Ρωσίας» — впервые употребляется византийской церковной канцелярией
для обозначения западных епархий, территории нынешней Украины, в отличие от
Киевской митрополии обозначаемой как «Μεγάλη Ρωσία» (Великая Россия), отсюда
титул митрополита Киевского, как «πάσης Ρωσίας» (всея России)

3.1 Дезуниты в своих полимических текстах начинают обозначать территории
киевской митрополии принявшие Унию, как «Малую Россию», и как часть «Великой
России» т.е. территории Московского Царства

3.2 Униональные авторы, как митрополит Киевский Иосиф Рутский, напротив, пишут
о том, что «вся Русь» или просто «Русь» приняла польскую Корону обособляя от
территории Московского Царства

\iusr{Alexander Steppenwind}
\textbf{Павел Недашковский} 

Від простої мови "Вкраїна". Але могло зростися з Украйной.

Я вільний від ресентиментів.

Генезу Малой Ро́сії можеш не розписувати, гадаю усі присутні в курсі.

\iusr{Павел Недашковский}
\textbf{Alexander Steppenwind} Шевелёв:

«Вірогіднішим здається варіянт, що власна назва Україна розвинулася
безпосередньо з первісного значення \enquote{межівна земля}, а не з вторинного значення
\enquote{країна}, \enquote{місцевість}»

См. Teasers and appeasers: Essays and Studies on Themes of Slavic Philology,
München, 1971, pp 199-20; цитата по украинскому изданию, перевод. Вакуленко,
Київ, Видавничий дім Києво-Могилянська Академія, 2009, стр 421-431

\ifcmt
  ig https://scontent-mxp1-1.xx.fbcdn.net/v/t1.6435-9/243149506_909506339694463_926311158643708870_n.jpg?_nc_cat=104&ccb=1-5&_nc_sid=dbeb18&_nc_ohc=6gq8r8gnu_sAX_HS3Qo&_nc_ht=scontent-mxp1-1.xx&oh=ddf222038b105790c5f564b64404c5ae&oe=61935EB8
  @width 0.4
\fi

\iusr{Alexander Steppenwind}
\textbf{Павел Недашковский} та тіки Вкраїна в значенні рідний край було і у донських козаків.)
Кажу ж, скоріше Вкраїна і Украйна поєдналися.

\iusr{Павел Недашковский}
\textbf{Alexander Steppenwind} согласное мнение крупнейших славистов, включая Шевелёва, что «Оукраина» в первичном значении – это «пограничье»

\iusr{Alexander Steppenwind}
\textbf{Павел Недашковский} можливо, попри те шо скептично ставлюся до совкових/російських славістів.
Але ж я написав, навіть якшо і так, то ссув наголоса доволі ріже слух, тому наймовірніше воно зрослося з побутовим рідним краєм - "Вкраїною", з того ж побуту, якого вийшли і козаки.

\iusr{Павел Недашковский}
\textbf{Alexander Steppenwind} Вам привести западных славистов или Шевелёв стал «совковым»? Я не филолог, все, что могу – это прибегнуть к аргументу ad auctoritatem

Хотя за светил советской славистики, всемирно признанных научным сообществом,
тоже обидно стало

\iusr{Павел Недашковский}
\textbf{Alexander Steppenwind} то, что со временем Оукраина обрела вторичное значение родной земли для казаков, с этим никто не спорит

\iusr{Alexander Steppenwind}
\textbf{Павел Недашковский} 

а чим Вас не влаштовує поєднання "Украйна" і "Вкраїна" коли мови йде за
фонетику? Я ж гадаю уявляєте як звучать ці слова в україномовному середовищі.

А щодо обидно, то епізод в "Город Зеро" дуже добре показав багатьох радянських
вчених.

\iusr{Павел Недашковский}
\textbf{Alexander Steppenwind} 

что там режет слух см. того же Шевелёва A Historical Phonology of the Ukrainian
Language – это крупнейшая на данный момент монография по исторической фонологии
украинского языка. Именно благодаря его изысканиям он для украинских диалектов
сделал то, что Зализняк для новгородских диалектов: вывел их непосредственно из
праславянского уровня, минуя концепцию «общерусского языка». Он, Шевелёв, в
этом эксперт и его мнение насчет топонима «Украина» см. выше  @igg{fbicon.smile} 

\iusr{Павел Недашковский}
\textbf{Alexander Steppenwind} речь о том, что «вкраїна» в значении «inland» вторично по отношению к «оукраїна» в значении «borderland»

\iusr{Павел Недашковский}
\textbf{Alexander Steppenwind} 

несравнимо финансирование Института украинского языка при советах и последние
30 лет. Не верите мне, поговорите с филологами. От этого советы не становятся
«хорошими» а третья украинская республика «плохой» – это просто данность. И
местные светила славистики и украинстики взрощены советами

\iusr{Alexander Steppenwind}
\textbf{Павел Недашковский} 

Вы походу не понимаете украинский язык, потому перейду на понятный для Вас.
Если бы Вы были носителем украинского языка, либо постоянно пребывали в среде
носителей и были знакомы с фонетикой, то Вам бы резал слух ударения в "Укра́йна"
в отличии от Вкраи́на. Потому я склоняюсь к синтезу этих двух словах.

>Зализняк

Вы серьезно?!) Это он то всемирно признан?! На которого правда не ссылаются
западные лингвисты и его работы не переведены на западные языки, мде))0)

Пример с новгородской поголовной грамотностью приводить будете? Либо с его
берестяными грамотами коих аж свыше тысячи?!)

Зализняк такой же авторитетный лингвист как Рыбаков - историк.

\iusr{Alexander Steppenwind}
\textbf{Павел Недашковский} ну да, впрочем как и упомянутый Вами социалист Грушевский.

\iusr{Павел Недашковский}
\textbf{Alexander Steppenwind} украинский вне сомнения не мой родной, однако я в достаточной мере им владею, устно и письменно, в последнем случае используя Скрипиновку. «Режет слух» — это из разряда «кухонной филологии»

\iusr{Павел Недашковский}
\textbf{Alexander Steppenwind} ожидаю ссылку на более крупных современных специалистов по древним псковским и новгородским диалектным формам

\iusr{Павел Недашковский}
\textbf{Alexander Steppenwind} работы крупнейшего украинского диалектолога, Курило — также, насколько мне известно, не переведены. От это ее вклад в украинистику стал меньше?

\iusr{Андрій Дідух}
\textbf{Павел Недашковский} 

Зализняк это фрик не имеющей никакого отношения к науке) К примеру он
высказывался о том, что в Новгородской Республике якобы более 80\% населения
были грамотными, что является абсолютной чушью (даже во Флоренции и Риме
столько не набиралось), а в самой Российской Империи к концу 19 века было более
80\% неграмотных. Да и показатели 80\%+ это показатели некоторых регионов
Западной Европы начала 19 века.

У Зализняка лошак=лошадь. Погуглите про это или вообще разборы от других
лингвистов его лекций. Также на его работы нет ссылок у западных исследователей

\iusr{Андрій Дідух}

Если присмотреться , то гуманитарные науки СССР строились таким образом, чтобы
противопоставить древние княжества Руси и Российскую Империю во всём. Тот же
миф о прогрессивном европейском Новгороде и славянское неоязычество это всё
детище совка хрущёвского периода. К чему приложил руку Зализняк и прочие фрики

\iusr{Alexander Steppenwind}
\textbf{Андрій Дідух} ну так в цьому плані совок займався знищенням русской (в широкому сенсі) ідентичності.

\iusr{Павел Недашковский}
\textbf{Андрій Дідух} пример приведенный Вами не имеет отношения к предмету т.е. к диалектным формам. Кого из значимых исследователей исторических диалектных форм Пскова и Новгорода Вы можете назвать?

\iusr{Павел Недашковский}
\textbf{Alexander Steppenwind} повторяю свой вопрос: Шевелёв «совок» или автор фундаментальной монографии по исторической фонологии украинского языка не в достаточной мере владел украинской фонетикой?

\iusr{Alexander Steppenwind}
\textbf{Павел Недашковский} >повторяю свой вопрос:
Товарищ майор, перелогинься.

\end{itemize} % }

\iusr{Mykola Ostrovskyi}
Шатость малороссийская

\begin{itemize} % {
\iusr{Alexander Steppenwind}
\textbf{Mykola Ostrovskyi} за яку згадували, коли шла мова за наші слобідські краї.
\end{itemize} % }

\iusr{Victor Datsko}
Так їм

\iusr{Олег Гуцуляк}
Ну, Сковорода ж писав "Мати моя - Малоросія, тітка моя - Україна"

\begin{itemize} % {
\iusr{Ruslan Kusyk}
\textbf{Oleg Gutsulyak} Але Малоросія і Великоросія це церковні терміни, які перенесли у світське, ширше використання.
\end{itemize} % }

\iusr{Ruslan Kusyk}

А знати і шанувати нашу історію допоможе ютубканал ім. Т. Г. Шевченка:

\href{https://youtu.be/5Zna0WTzFtY}{%
Як Русь стала Україною / Історія Русі / Історія України, youtube, 23.08.2020%
}

\begin{itemize} % {
\iusr{Павел Недашковский}
\textbf{Ruslan Kusyk} подарите им исследования Шевелёва, чтобы бред не несли

\url{https://uk.m.wikipedia.org/wiki/Шевельов_Юрій_Володимирович}

\end{itemize} % }

\iusr{Mykhailo Sokulski}

Терміни мають здатність трансформуватися. З кінця 19-го - поч. 20 століття
слово "малорос" в українському дискурсі має негативну конотацію, тому не бачу
тут нічого дивного.

\begin{itemize} % {

\end{itemize} % }

\end{itemize} % }

