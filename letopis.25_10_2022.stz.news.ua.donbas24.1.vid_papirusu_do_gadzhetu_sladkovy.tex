% vim: keymap=russian-jcukenwin
%%beginhead 
 
%%file 25_10_2022.stz.news.ua.donbas24.1.vid_papirusu_do_gadzhetu_sladkovy
%%parent 25_10_2022
 
%%url https://donbas24.news/news/vid-papirusu-do-gadzetu-novii-projekt-mariyi-ta-oleksandra-sladkovix
 
%%author_id demidko_olga.mariupol,news.ua.donbas24
%%date 
 
%%tags 
%%title "Від папірусу до гаджету" — новий проєкт Марії та Олександра Сладкових
 
%%endhead 
 
\subsection{\enquote{Від папірусу до гаджету} — новий проєкт Марії та Олександра Сладкових}
\label{sec:25_10_2022.stz.news.ua.donbas24.1.vid_papirusu_do_gadzhetu_sladkovy}
 
\Purl{https://donbas24.news/news/vid-papirusu-do-gadzetu-novii-projekt-mariyi-ta-oleksandra-sladkovix}
\ifcmt
 author_begin
   author_id demidko_olga.mariupol,news.ua.donbas24
 author_end
\fi

%\ii{25_10_2022.stz.news.ua.donbas24.1.vid_papirusu_do_gadzhetu_sladkovy.txt}
\ii{25_10_2022.stz.news.ua.donbas24.1.vid_papirusu_do_gadzhetu_sladkovy.pic.front}

\begin{center}
  \em\color{blue}\bfseries\Large
Творче подружжя Маріуполя — Олександр та Марія Сладкови почали роботу над авторською виставою 
\end{center}

Проєкти Марії та Олександра Сладкових — найбільш творчого подружжя Маріуполя —
вражають самобутністю та оригінальністю рішення. Вони неодноразово надихали
\href{https://donbas24.news/news/i-human-istoriya-muzicnogo-projektu-z-mariupolya-intervyu}{\emph{маріупольську молодь}},%
\footnote{I-Human: історія музичного проєкту з Маріуполя, Алевтина Швецова, donbas24.news, 13.10.2022, \par\url{https://donbas24.news/news/i-human-istoriya-muzicnogo-projektu-z-mariupolya-intervyu}}

адже творчий тандем використовує тільки свій авторський
матеріал. Це — сценарії, вірші, пісні з аранжуванням і потім, з готовою
фонограмою, розробка відео і створення готового відеоконтенту, створення
світлового рішення. Все це вони роблять вдвох з 2016 року. \href{https://www.facebook.com/profile.php?id=100007993115763}{Марія та Олександр}%
\footnote{\url{https://www.facebook.com/profile.php?id=100007993115763}} %
планували у 2022 році поставити свою першу виставу в приміщенні Центральної
бібліотеки ім. В. Г. Короленка у Маріуполі, проте планам завадила війна...

\textbf{Читайте також:} \emph{Почуйте голос Маріуполя: історії людей, яким пощастило евакуюватися з блокадного міста}%
\footnote{Почуйте голос Маріуполя: історії людей, яким пощастило евакуюватися з блокадного міста, Алевтина Швецова, %
donbas24.news, 21.09.2022, \par%
\url{https://donbas24.news/news/pocuite-golos-mariupolya-istoriyi-lyudei-yakim-poshhastilo-evakuyuvatisya-z-blokadnogo-mista}%
}

\subsubsection{Де тепер працює творчий тандем?}

\ii{25_10_2022.stz.news.ua.donbas24.1.vid_papirusu_do_gadzhetu_sladkovy.pic.1}

Через поранення Олександра подружжю довелося перебувати в окупованому місті.
Загалом після 24 лютого вони провели 120 довгих днів під обстрілами і в
постійній небезпеці в зруйнованому та потім окупованому Маріуполі. Але все ж їм
вдалося виїхати і навіть вивезти деяку техніку завдяки небайдужим людям.
Спочатку вони дуже хотіли виїхати до Ужгорода, але через подорожчання оренди
Сладкови вирішили зупинитися у Павлограді. Подружжя, попри пережитий негативний
досвід, налаштовано дуже оптимістично, має безліч ідей, які готове втілювати в
життя.

\begin{leftbar}
\emph{\enquote{Всі наші проєкти мають одне єдине гасло — \enquote{З любов'ю до України}}}, — наголосила Марія.
\end{leftbar}

Олександр та Марія є засновниками та керівниками \textbf{творчого хабу \enquote{Сучасна
Україна}} і в умовах війни вони розуміють, що дуже важливо, щоб глядач через
культурні проєкти отримував своє\hyp{}рідну арт-терапію.

\textbf{Читайте також:} \emph{Героям маріупольського гарнізону присвятили нову пісню}%
\footnote{Героям маріупольського гарнізону присвятили нову пісню, Яна Іванова, donbas24.news, 14.06.2022, \par%
\url{https://donbas24.news/news/geroyam-mariupolskogo-garnizonu-prisvyatili-novu-pisnyu-video}
}

\begin{leftbar}
\emph{\enquote{Наразі дуже важливе креативне мислення. Всі наші роботи стануть
арт-терапією для глядачів. Звісно, що ми мусимо пам'ятати про мертвих,
але треба і рухатися далі. Потрібно продовжувати жити}}, — підкреслила
Марія.
\end{leftbar}

\ii{25_10_2022.stz.news.ua.donbas24.1.vid_papirusu_do_gadzhetu_sladkovy.pic.2}

У Павлограді творчий тандем вразив міських діячів культури, які з радістю
погодилися підтримати проєкти Марії та Олександра. Оптимістичні та оригінальні
проєкти Сладкових надихнули і режисера Хмельницького театру ім. Старицького,
Миколу Бабина. Він навіть запросив їх на свою інтерактивну виставу \enquote{Маріуполь.
Надія на світанок}, в якій Олександр та Марія виступлять одночасно і
експертами, і учасниками. Водночас Марія та Олександр вирішили продовжити
навчання і цього року вступили до Маріупольського державного університету в
магістратуру на спеціальність \enquote{Культурологія}.

\textbf{Читайте також:} \emph{Маріуполю присвятили поштові листівки з віршами на румейській мові}%
\footnote{Маріуполю присвятили поштові листівки з віршами на румейській мові, Яна Іванова, donbas24.news, 07.07.2022, \par%
\url{https://donbas24.news/news/mariupolyu-prisvyatili-postovi-listivki-z-virsami-na-rumeiskii-movi-foto}%
}

\subsubsection{Про нову виставу}

До війни подружжя виграло грант на 100 тис. грн від Міського бюджету Маріуполя.
Вони планували поставити свою першу виставу. Виграні гроші повинні були
спрямувати на закупівлю обладнання. Але після повномасштабного вторгнення рф в
Україну довелося скоригувати багато планів. І хоча гроші так і не були
отримані, подружжя вирішило ставити виставу власними зусиллями. Марія є
авторкою багатьох чутливих та щирих віршів, але п'єс у жінки ще не було. Вона
почала її писати в Маріуполі і завершила в Павлограді. Це філософська п'єса з
елементами комедії.

\begin{leftbar}
\emph{\enquote{Дія відбуватиметься в сучасній бібліотеці. Будуть присутні і елементи містики з задіянням цифрових технологій}}, — розповіла Марія.
\end{leftbar}

Режисеркою вистави стала художня керівниця Фольклорно-об\hyp{}рядового театру \enquote{Вертеп
Марії} Марія Кравченко. У виставі гратимуть 3 актори цього ж театру на чолі з
режисеркою, яка також гратиме. Актори вже вчать свої тексти. За технічну
частину та музичне оформлення відповідатиме Олександр Сладков. 

\ii{25_10_2022.stz.news.ua.donbas24.1.vid_papirusu_do_gadzhetu_sladkovy.pic.3}

\textbf{Читайте також:} \emph{Відновлення книжкового фонду Маріуполя — коли відкриється нова бібліотека}%
\footnote{Відновлення книжкового фонду Маріуполя — коли відкриється нова бібліотека, Ольга Демідко, donbas24.news, 20.10.2022, \par%
\url{https://donbas24.news/news/vidnovlennya-knizkovogo-fondu-mariupolya-koli-vidkrijetsya-nova-biblioteka}%
}

\begin{leftbar}
\emph{\enquote{П'єсу довелося змінювати, адже через війну у нас тепер немає можливостей, які ми мали в Маріуполі. Досить все важко. Доведеться збирати апаратуру у всьому Павлограді. Але я впевнений, що все вдасться}}, — наголосив Олександр.
\end{leftbar}

Виставу планують поставити до Нового року у Культурному центрі Павлограду.

\subsubsection{Вірші Марії}

Олександр часто говорив Марії, що дуже хотів би, щоб вона почала писати вірші і
пісенні тексти. Мабуть, чоловік відчував хист коханої дружини краще за неї. У
2014 році Марія все ж таки почала писати вірші. Її поезію відрізняє чистота,
благородство, м'який ліризм та неабияка щирість. Багато віршів поетеси
присвячені і воєнній тематиці.

\ii{25_10_2022.stz.news.ua.donbas24.1.vid_papirusu_do_gadzhetu_sladkovy.pic.4}

\textbf{Читайте також:} \emph{Українським дітям — українська книга: як скористатися соціальною програмою}%
\footnote{Українським дітям — українська книга: як скористатися соціальною програмою, Алевтина Швецова, donbas24.news, 20.09.2022, \par%
\url{https://donbas24.news/news/ukrayinskim-dityam-ukrayinska-kniga-yak-skoristatisya-socialnoyu-programoyu}%
}

\begin{quote}
\textbf{ВІЙНА НАРАНОК}

Війна на ранок... Тиша не чекала, В своїх обіймах колихала нас... Життя
зненацька навпіл розірвало На \enquote{до} і \enquote{після} раптом, водночас.
\bigskip

Змарніле небо біллю застогнало, Щомиті стало сірим і смурним, Земля від жаху
сумно заридала, І горе увійшло у кожен дім.
\bigskip

Душа в огні і серце гучно плаче, І чорна хмара сонце закрива, Як тіні ходять
люди і неначе, День закінчився, світла скрізь нема.
\bigskip

Хрести повсюди, пагорби чорніють, Ворожа лють, нестерпний сморід, дим, Тварини
й люди, наче, скаженіють, Крик матерів, мольба у пеклі тим.
\bigskip

Ікона, стіл... Горить, не гасне свічка, О, Господи, нас, горе, омини, Сьогодні
тиха видалася нічка, Неначебто немає вже війни...
\bigskip

Війна на ранок... Вистає країна, Здолає, пройде свій тернистий шлях, Все буде,
без сумніву, Україна З блакитним небом, сонцем в небесах!
\bigskip

\emph{МАРІЯ СЛАДКОВА ПАВЛОГРАД 16 липня 2022р.}
\end{quote}

Раніше Донбас24 розповідав, як \href{https://donbas24.news/news/oksana-stomina-cerez-svoyi-virsi-ta-spogadi-znaiomit-nimciv-z-mariupolem}{Оксана Стоміна через свої вірші та спогади
знайомить німців з Маріуполем}.%
\footnote{Оксана Стоміна через свої вірші та спогади знайомить німців з Маріуполем, Ольга Демідко, donbas24.news, 26.09.2022, %
\par\url{https://donbas24.news/news/oksana-stomina-cerez-svoyi-virsi-ta-spogadi-znaiomit-nimciv-z-mariupolem}}

Ще більше новин та найактуальніша інформація про Донецьку та Луганську області
в нашому телеграм-каналі Донбас24.

ФОТО: з відкритих джерел.

\ii{insert.author.demidko_olga}
