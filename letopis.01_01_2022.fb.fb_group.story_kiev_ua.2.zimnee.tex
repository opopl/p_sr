% vim: keymap=russian-jcukenwin
%%beginhead 
 
%%file 01_01_2022.fb.fb_group.story_kiev_ua.2.zimnee
%%parent 01_01_2022
 
%%url https://www.facebook.com/groups/story.kiev.ua/posts/1831180863745373
 
%%author_id fb_group.story_kiev_ua
%%date 
 
%%tags jolka,kiev,novyj_god,prazdnik,sssr
%%title Еще зимнее, новогоднее и елочное...
 
%%endhead 
 
\subsection{Еще зимнее, новогоднее и елочное...}
\label{sec:01_01_2022.fb.fb_group.story_kiev_ua.2.zimnee}
 
\Purl{https://www.facebook.com/groups/story.kiev.ua/posts/1831180863745373}
\ifcmt
 author_begin
   author_id fb_group.story_kiev_ua
 author_end
\fi

Еще зимнее, новогоднее и елочное... 

Новогодние елки в Октябрьском дворце.

Мои родители впервые увидели сверкающий после реконструкции Октябрьский дворец
на открытии - 24 декабря 1957 года (см. фото), на пышной Новогодней елке, где
их веселая компания прекрасно встретила наступающий 1958-й год, кстати – год
моего рождения.

А через 12 лет плотно познакомился с дворцом и я.

Осенью 1969г. половина нашего 5-го класса 83-й борщаговской школы записались на
зимние каникулы работать массовиками на главной елке Украины в Октябрьском
дворце. Именно работать, т.к. за каждую проведенную новогоднюю елку полагалась
оплата в один рубль, и мы честно отрабатывали этот рубль, поднимаясь каждое
каникулярное! зимнее утро в половине седьмого, чтобы доехать с Борщаговки на
17-м трамвае до «Большевика», а дальше на метро до Октябрьского дворца. И
только на через год, когда открыли станцию метро «Святошино», стало добираться
полегче, хотя поди дождись того 13-го автобуса...

Нашей основной задачей, на время 3-х ежесуточных представлений, было в
красивых, индивидуально пошитых костюмах, подыгрывать ведущим и организованными
хороводами водить малышню вокруг Главной елки Украины, выставленной в
полукруглом бальном зале Дворца. В зале и на сцене выступало много
замечательных артистов, но нас сразу увлек жонглер Аркадий Пупон, который стал
нашим другом, а многим и учителем.

Мне, обычно, доставались достойные игровые роли в хороших костюмах, которые нам
шили в театральном ателье на Шота Руставели. Мой костюм вьетнамского партизана,
а тогда вьетнамская война была в самом разгаре, состоял из голубого шелкового
камзола и нонлы - настоящей конической большой вьетнамской шляпы с
подбородочным ремнем.

Очень сильно мы вертелись вокруг одного старого фокусника Арнольда Викторовича.
Фокусы его были простые, но была у него одна штука... Когда фокусник вызывал
мальчишку из зала и здоровался с ним за руку, раздавался сильный взрыв с белым
дымом! Это старый обманщик щелкал пальцами, на которые был нанесен волшебный
порошок, называемый «недотрога». Арнольд Викторович изготовлял этот порошок
сам. Представляете, так поздороваться в классе, или намазать его на классную
доску под мел учителя? Мечта пятиклассника! Мы готовы были за эту «недотрогу»
отдать все.  И вот, когда у Арнольда, кто-то украл с его волшебного столика
стопку юбилейных рублей, которые фокусник, в ходе своего выступления, доставал
с грохотом из носов и ушей юных завороженных зрителей, старый обманщик заявил,
что будет давать нам по половине спичечного коробка «недотроги» только за
каждую принесенную ему небольшую серенькую монетку, которая грохочет даже
громче металлических рублей. И показал нам какую, и рассказал, что обычно эти
монетки валяются где-то в квартире и что родителям лучше не говорить про
монетки и заветную «недотрогу». И учил нас громко щелкать пальцами и технике
безопасности с порошком, и живописно рассказывал, как хлопнул себя однажды по
карману, в котором лежал коробок «недотроги» и остался без половины «хозяйства»
...  Тот еще был старый развратник.

Вот так и перекочевали к Арнольду Викторовичу десятка полтора серебряных
советских полтинников 1924г., по 9 г. чистого серебра каждый, подлинную
стоимость которых нам позже объяснили разъяренные отцы. 

Но зато у нас была «недотрога» и нам завидовала вся школа.

Раскрою один страшный секрет Октябрьского дворца.

Многие, наверное, помнят Знайку - большого робота, стоявшего возле эваковыхода
на втором этаже дворца, который всегда точно и размеренно, глухим металлическим
голосом, отвечал на коварные вопросы малышни. И главное, что всегда правильно!
И только мы знали, что в нескольких метрах от робота, за закрытой дверью, на
площадке эвакуационной лестницы, сидит за столом, обложенный различными
тетрадями, усталый и взлохмаченный дядька с микрофоном, который и отвечал
невозмутимо на все, а взрывался только от постоянных ехидных вопросов Знайке от
наших охламонов, вроде такого: - А кто это сидит за столом слева за дверью?
Откупиться от нашей назойливости он мог только конфетами, что нас вполне и
устраивало.

За 4 года работы на Елках мы досконально изучили Октябрьский дворец, от чердака
до глубоких подвалов, которые, как потом выяснилось, были когда-то
расстрельными. Мы настолько освоились, что перестали ездить каждый день с
Борщаговки и назад, и в зимы 1971,72гг. поселились на все время каникул
(бесплатно!) в бездонных подвалах гостиницы «Москва» (старых подвалах
небоскреба Гинзбурга), из которых нас выпускали: наверх - на новогодние
гульбища в рестораны гостиницы, где нам удавалось легко наколядовывать
тяжеленные мешки конфет, фруктов и прочей выпивки; и вниз - на вечерние
прогулки Крещатиком, через прямой выход из подвалов на Карла Маркса, в тылы
кинотеатра «Украина». 

А 9 января, в последний день каникул устраивался капустник. В наше распоряжение
отдавалась большая сцена и зал, в котором сидели работники дворца, артисты и
родители. Мы, с другом Ромкой, изображали, как-то, сценку победного боя двух
вьетнамских партизан с «Фантомом». На мне был шелковый голубой,  «вьетнамский»
костюм, на ремне через плечо тяжеленный, хотя и бутафорский пулемет, и нонла,
крепко притянутая к подбородку, т.к. была огромная. На Ромку костюма не было,
поэтому он, в чем попало, нес большую коробку с патронной лентой. Стрелять
должен был я, но, когда по фонограмме приближающего «Фантома», мы начали
готовиться к бою и снимать пулемет, то оказалось, что стащить его через голову
невозможно – не дает огромная и на узел завязанная шляпа. А «Фантом»
приближается и сейчас по фонограмме пойдут пулеметные очереди…  В итоге, я
кидаюсь на пол, Ромка перетаскивает пулемет мне на живот и пристроившись между
моих скрюченных ног, пытается поднять ствол выше огромной шляпы и, позабыв про
ленту с патронами, начинает судорожно дергаться под звук фонограммных
выстрелов. Потом звук пике «Фантома» и сильный взрыв. Мы победили! 

Сказать, что зал умирал от смеха – ничего не сказать.

Вот такие были у нас, борщаговских школьников, зимние каникулы 1969-1972гг. 

П.С. Конечно борщаговским школьникам не с неба упала такая увлекательная
работа. Я уже писал про отряд «Орленок» и его создательницу и бессменного
руководителя – Веру Матвеевну Рогожникову, бывшую детдомовку, ученицу
знаменитого Калабалина (Карабанова из «Педагогической поэмы»), благодаря
неуемной энергии и напору которой и происходили эти новогодние чудеса.

П.П.С. К сожалению, мы тогда делали мало фотографий, больше увлекаясь
киносъемкой. Но 8мм. кинопленка почти не дожила до оцифровки. В тех тенях, что
уцелели, можно различить знакомые лица во время отрядной линейки и во время
капустника на сцене Октябрьского дворца. Я жонглирую (учился у Аркадия Пупона),
а Ромка крутит обруч.

П.П.П.С. На приложенных фото, снимки с различных Елок, на которых можно
разглядеть и Аркадия Пупона, и всю нашу братию.

П.П.П.П.С. Если кто-то думает, что я ностальгирую по Совку, то он очень
ошибается.
