% vim: keymap=russian-jcukenwin
%%beginhead 
 
%%file 27_02_2023.fb.miroshnichenko_mihail.mariupol.1.27_02_22_4_den__vosk.cmt
%%parent 27_02_2023.fb.miroshnichenko_mihail.mariupol.1.27_02_22_4_den__vosk
 
%%url 
 
%%author_id 
%%date 
 
%%tags 
%%title 
 
%%endhead 

\qqSecCmt

\iusr{Татьяна Кулагина}

Ты молодец, что вёл дневник, а у меня была такая мысль, но почему то рука не
поднималась. Было очень страшно, а жить хотелось больше страха. Настраивали
себя, горячо молились, верили что скоро закончится. Сила духа и высшие силы
уберегли, ценности поменялись. Продолжаем жить, и я проснувшись первым делом
говорю: Спасибо, что я живу и благодарю всех кто находился рядом физически и на
расстоянии, за всю заботу и молитвы за всех и за Мариуполь.

\iusr{Сергей Ковалев}

Здравствуй Миша Дневник это память пусть даже печальная. В зинцевой балке стояли
подразделения В С У и расстреливали город и инфраструктуру до прихода ДНР.

\begin{itemize} % {
\iusr{Яна Михейко}
\textbf{Сергей Ковалев} а чому ЗСУ не розстрілювали місто ДО 24.02.2022 ви не замислювалися?

\iusr{Сергей Ковалев}
\textbf{Яна Українка} Донецком занимались. Готовились.

\iusr{Яна Михейко}

Мені вас шкода. Чекали? Насолоджуйтесь!!!

\iusr{Сергей Ковалев}
\textbf{Яна Українка} А мне вас нет.Отстроим как деды И будем помнить.

\iusr{Яна Михейко}
\textbf{Сергей Ковалев} сильно не отстраивайте, а то ВСУ уже близко)))

\iusr{Яна Михейко}

И лыжи на рашу готовьте

\iusr{Сергей Ковалев}
\textbf{Яна Українка} Чей Крым?

\iusr{Яна Михейко}
\textbf{Сергей Ковалев} КРИМ - ЦЕ УКРАЇНА. МАРІУПОЛЬ- УКРАЇНА

\iusr{Сергей Ковалев}
\textbf{Яна Українка} И опять вас не жалко. Решают люди. А не марионетки. Меньжуетесь не зная на каком языке писать. Потеряете все.

\iusr{Яна Михейко}
\textbf{Сергей Ковалев} 

ми вже все втратили через таких, як Ви. Тепер ваша черга втрачати. Відплата
буде жорсткою. Ви вірно кажете, що вирішують ЛЮДИ, а де ви на сосії бачили
людей???

\iusr{Сергей Ковалев}
\textbf{Яна Українка} Служите мужу это единственная польза. Вас ждёт тяжёлый путь.

\iusr{Михаил Мирошниченко}
\textbf{Яна Українка}

\ifcmt
  igc https://scontent-fra3-1.xx.fbcdn.net/v/t39.30808-6/332852760_760619712239653_8880580053975046906_n.jpg?_nc_cat=101&ccb=1-7&_nc_sid=dbeb18&_nc_ohc=ZThlvmI4VaoAX8fpVXe&_nc_ht=scontent-fra3-1.xx&oh=00_AfByBZ0w_yl3d8odsrBxIT069V7NorjgeEugJhFfr3epyQ&oe=641E65E4
	@width 0.4
\fi

\iusr{Яна Михейко}
\textbf{Сергей Ковалев} служат собаки. А я мужа люблю. Но вам не понять...

\iusr{Яна Михейко}
\textbf{Михаил Мирошниченко} ничего не понятно, но очень интересно)

\iusr{Михаил Мирошниченко}
\textbf{Яна Українка} "Мне жаль небосвод этот синий,
Жаль землю и неба осколки
Мне страшно, что сытые свиньи
Страшней, чем голодные волки.

\end{itemize} % }

\iusr{Светлана Шаулис}

Я каждую ночь ложилась с мыслью, что тоже буду описывать каждый день, но снова
и снова , утро начиналось с растопки костра(мужчин не было), смастерили с
кирпичей костер., далее готовка пищи и надо побыстрее, чтобы все успели, далее
поход за бумагой, дровами, водой. И так каждый день более 90 дней. Дневник
вёлся только в мыслях и на календаре вначале, когда что отключали, а потом
просто черные круги. Молились, когда дом ходуном ходил, думали пусть сразу и
наповал. Закрывала подушками окно, когда рядом с окном пытались сбить
беспилотник и стреляли без разбора, куда попадут, понимаю, что подушка это
ничто. Господи, иногда кажется, что мы просто участвовали в съёмках кино, но
увы реальность говорит обратное. А дневник, это хорошо, это история, это
ощущение.

\iusr{Тетяна Вішинська Грузіна}

Каждый видит то, что понимает.

Я понимала, что стреляли люди пришедшие в Украину с оружием.

Принимая в учёт факт, того, что пока их небыло, вооруженные люди жили рядом с
нами и некоторых я знала лично, как вполне адекватных, и не желающих менять
границы государства, которые считали и своими границами личного пространства
для жизни по правилам системы которая называется Украина.

В городе я лично видела локацию украинских военных.

Обнаружила ее идя к подруге на блины.

Тогда ещё был интернет и ее пост
\href{https://www.facebook.com/profile.php?id=100023852258613}{Hanna
Solomatina} разместила свою историю о блинах, что побудило желание прийти к
ней.

Я ей позвонила и сообщила, что у меня есть ряженка к блинам и мы решили, что
пора мне наконец-то сходить к ней в гости, а то сколько раз хотели и не
реализовывали.

Все, что происходит со мной, ведёт меня в Свет и Любовь, пришла со
настроенность в этом намерении и я почувствовала радость от походов в гости,
которую испытывала в своём детстве, когда родители брали меня с собой туда, где
тоже были дети, и это означало возможность придумать свою игру.

Город рычал и выл от взрывных раскатов и сотрясался будто раненый зверь от
жестоких ударов.

Я ободрила шаг, и позвонила маме, идя через дворы, мимо школы в которой
училась, когда-то, казалось в другой жизни, там на момент моего шествия мимо, во
дворе школы уже стояла военная техника и на входе, обозначенным воротами
железного забора находился молодой парень с автоматом в светло пятнистой форме,
которую я знала как Форму украинской армии, на нем был тяжёлый бронижелет и
громоздкий шлем. 
