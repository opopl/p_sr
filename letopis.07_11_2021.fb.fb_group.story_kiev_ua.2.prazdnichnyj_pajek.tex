% vim: keymap=russian-jcukenwin
%%beginhead 
 
%%file 07_11_2021.fb.fb_group.story_kiev_ua.2.prazdnichnyj_pajek
%%parent 07_11_2021
 
%%url https://www.facebook.com/groups/story.kiev.ua/posts/1792069094323217
 
%%author_id fb_group.story_kiev_ua
%%date 
 
%%tags chelovek,kiev,rasskaz,sssr,zhizn
%%title ПРАЗДНИЧНЫЙ ПАЁК
 
%%endhead 
 
\subsection{ПРАЗДНИЧНЫЙ ПАЁК}
\label{sec:07_11_2021.fb.fb_group.story_kiev_ua.2.prazdnichnyj_pajek}
 
\Purl{https://www.facebook.com/groups/story.kiev.ua/posts/1792069094323217}
\ifcmt
 author_begin
   author_id fb_group.story_kiev_ua
 author_end
\fi

ПРАЗДНИЧНЫЙ ПАЁК

История из прежней жизни

Празднику Великого Октября посвящается

Степан Петрович возвращался домой на бровях. Нет, повод, конечно, был
серьёзный: как-никак завтра седьмое ноября, любимый праздник. Однако нарезаться
так в цеху с мужиками всё же не следовало.

Степан Петрович выполз из троллейбуса номер восемнадцать, едва держась на
ногах. Благо, дом его был совсем рядом. О, дом этот был не простой, жили здесь
сплошь начальники. Степан Петрович получил свою квартиру просто чудом – нужно
было разбавить стопроцентное начальство кем-нибудь из народа. А тут под руку
попался участник войны, ветеран труда, мастер цеха, да ещё и член партии. Вот
ему счастье и привалило.

\ifcmt
  ig https://scontent-mxp1-1.xx.fbcdn.net/v/t1.6435-9/254345106_1357200921363152_813640665327967457_n.jpg?_nc_cat=102&ccb=1-5&_nc_sid=825194&_nc_ohc=3mfJ5efzjy8AX9BDDb_&_nc_ht=scontent-mxp1-1.xx&oh=97fe0e338421e35a6705de111fd10a24&oe=61AF30C5
  @width 0.8
\fi

В руках Степан Петрович держал авоську. В ней головой вниз лежала завёрнутая в
коричневую бумагу курица по рупь шестьдесят за кило. Из бумаги торчали только
жёлтые куриные ноги. В народе таких кур называли "синенькими". Судя по
комплекции, курица эта умерла голодной смертью, а, судя по скрюченным пальцам с
когтями, смерть эта была мучительной. Однако достать и такую доходягу в эпоху
не в меру развитого социализма было непросто. Добыта она была не в честном бою
в магазине, а получена на родном предприятии в качестве праздничного пайка.
(Здесь, пожалуй, мне стоит остановиться и сделать некоторые пояснения.
Молодёжь, видимо, не знает, что в то время на предприятиях выдавали к празднику
немного еды, называемой "продуктовым набором" или попросту "пайком". Вот такой
"набор" в виде курицы и достался нашему герою).

Пока мы углублялись в исторический экскурс, Степан Петрович добрался до своего
парадного. Несколько минут он безуспешно давил на кнопку вызова лифта, пока не
уразумел, что на табличке, висевшей прямо напротив его носа, было написано
"Лифт временно не работает". Делать нечего, Степан Петрович, сжав в кулак
покидающие его силы, потащился на свой четвёртый этаж.

Путь был долгим и изнурительным. Наконец Степан Петрович ввалился к себе. В
квартире стоял красноватый полумрак. Человек трезвый задумался, с чего бы это.
Но Степан Петрович по этому поводу сильно не заморачивался, прикинув, что это
заходящие лучи солнца создают такой эффект. Тем более, что мозг его пронзило
воспоминание гораздо более важное. Вчера, когда Степан Петрович ужинал на
кухне, его старенький холодильник "Днепр" заурчал, затрясся, забился в истерике
и, наконец, умолк. Степан Петрович, будучи в техническом отношении весьма
продвинутым, сразу смекнул: "Компрессор накрылся. Нужен мастер". Да где же
этого мастера взять перед праздниками? Курицу нужно было спасать. По мнению
Степана Петровича, у курицы ещё сохранялась некая субстанция между кожей м
костями, которая при комнатной температуре могла быстро прийти в негодность.

Спасительная идея не заставила себя ждать. На дворе стояла осень, начинались
заморозки. Решено было вывесить курицу за окно. Степан Петрович открыл
форточку, взял курицу в авоське за ноги и стал проталкивать её наружу. Курица
сопротивлялась. Степан Петрович настаивал. Его настойчивость не оказалась
напрасной – курица поддалась и с лёгким треском ушла за окно. Победившему
человеку оставалось только зацепить ручки авоськи за форточку и прикрыть её.

Борьба с курицей окончательно лишила Степана Петровича сил, он рухнул на диван
не раздеваясь и уснул тяжёлым сном.

Ночью Степану Петровичу приснился странный сон. Будто бы он снова сидит в
окопе, а над ним проносятся немецкие бомбардировщики. Но швыряют они в Степана
Петровича не бомбы, а общипанных кур по рупь шестьдесят за кило. Те с диким
воем летят головой вниз и падают всё ближе и ближе к Степану Петровичу,
производя жуткий грохот. Он уже было отрыл рот, чтобы с криком "Врёшь, не
возьмёшь!" кинуться в атаку, как грохнуло совсем рядам. Степан Петрович
проснулся. В дверь кто-то отчаянно колотил. В комнате был такой же полумрак,
как и вчера вечером.

Степан Петрович открыл дверь. В комнату влетел участковый.

- Ты что наделал, гад? Антисоветчик, диссидент хренов! - Дальше что-то
ненормативное. - А ну пошли!

Участковый схватил за шиворот ничего не понимающего Степана Петровича и потащил
вниз по лестнице. На тротуаре стоял дворник и не отрываясь смотрел в одну
точку. Степан Петрович проследил за его взглядом и обомлел. На фасаде дома
красовался огромный транспарант с изображением вождя мирового пролетариата.
Взгляд Ильича был с до боли знакомым прищуром, но по правой щеке его в аккурат
там, где за кумачом находилось окно Степана Петровича, сползала одинокая слеза.
Казалось, что Ленин был опечален увиденным. Нет, это была не слеза. Это была
КУРИЦА!

Степан Петрович охнул и стал медленно оседать на асфальт.

* * *

По мнению Степана Петровича всё закончилось более или менее благополучно.
Перенесённый инфаркт – это мелочи. Главное, что участковый, пожалев, не передал
дело в суд. Ну, телегу на завод он, правда, послал. А там вкатили строгача с
занесением за непрерывное пьянство. Но, главное, из партии не выперли. А куда
ему без партии?
