% vim: keymap=russian-jcukenwin
%%beginhead 
 
%%file poetry.rus.olesja_gavryshko.ljudy
%%parent poetry.rus.olesja_gavryshko
 
%%url http://maysterni.com/publication.php?id=3470
%%author 
%%tags 
%%title 
 
%%endhead 

\subsubsection{Люди - найвищі із істот}
\Purl{http://maysterni.com/publication.php?id=3470}

Так сталося, нічого вже не повернути,
Але не знаю чи простити, чи тебе забути.

Чи камнем в прірву полетіть?
Якби все знати заздалегідь.

Хіба могло так статись, що ця казка
Була моя й твоя поразка?

Була б щасливою для нас обох,
Про це міг знати лиш Всевишній Бог.

Та йде життя і має воно в собі,
Білий вельон і хустку у жалобі.

Як чорне й біле, ніч і день,
Зіграє доля різних нам пісень.

Хоч люди є найвищі із земних істот,
Та часто піддаються марноті-марнот.

Такі ми є і так заведено у нас
Усе розставе і покаже час.

Погляньте лиш на милих голубів,
Їх воркутіння - це небесний спів.

А лебеді, їх вірність лебедина.
Хіба уміє так любить людина?!

Кажуть, що кохання - квітучий сад душі,
Та потребує він плекання на землі.

Так часто об скарб безцінний ноги витираєм,
Як жити в світі і самі не знаєм.

Багатство, вроду ми цінуєм,
Серця ж власного не чуєм.
