% vim: keymap=russian-jcukenwin
%%beginhead 
 
%%file 11_11_2020.news.ua.pravda.1.rusynstvo.moskovity_love
%%parent 11_11_2020.news.ua.pravda.1.rusynstvo
 
%%url 
%%author 
%%tags 
%%title 
 
%%endhead 

\subsubsection{Традиція любити московитів}

Я мешкаю на вулиці Корзо, у самому центрі Ужгорода, на перехресті біля
католицького собору, яке місцеві називають "Хрест".

Вдень на "Хресті" працюють вуличні музиканти. Спочатку літній чоловік у
капелюсі грає на акордеоні "Лібертанго" та "Гуцулку Ксеню".

Потім приходить черга довговолосого молодика, який спеціалізується на
репертуарі гурту "Кіно".

Потім юна скрипалька почала грати "Бєлой акациї грозді душистиє". Я одразу
нагострив вуха, але "білогвардійський романс" швидко перетік у "Ніч яка
місячна", а потім й зовсім поринув у якусь "вівальдівщину" з присмаком "сім
сорок".

Судячи з усього, єдиним "радикальним ідейним русином" був той хлопець, що їхав
зі мною до Мукачева. Авжеж, є ще ангажований Москвою та ОПЗЖ отець Сидор. Але,
по-перше, він вже кілька років відмовляється спілкуватись з журналістами. А
по-друге, здається, він нікому не цікавий навіть тут, в Ужгороді.

Переходжу через міст та йду до пам’ятника Олександру Духновичу, що стоїть перед
Закарпатським музично-драматичним театром. Можливо, хоч там хтось з русинів
марширує під триколором?

\ifcmt
img_begin 
	url https://img.pravda.com/images/doc/8/7/8775861-dscf1065.jpg
	caption Олександр Духнович для закарпатських русинів – це як Шевченко для українців або Міцкевич для поляків
	width 0.7
img_end
\fi

У пошуках екшну для репортажу починаю трохи сумувати за часами, коли деякі
русинські лідери наймали для охорони своїх збіговиськ бойовиків зі Сходу – до
2014 року з Одеси сюди приїздили навіть активісти проросійської партії
"Родина".

У ті роки багато було розмов, що Росія створює на Закарпатті "інтерфронт".
Інколи сценарії таких "інтерфронтів" справді вистрілювали у мультикультурних
регіонах – спочатку Придністров’я, згодом Крим.

На Закарпатті також були спроби реалізувати щось подібне. Але русинська тема не
спрацювала, і Москва почала розробляти набагато успішніший "угорський
сценарій". Достатньо згадати скандал, що виник через видачу угорських паспортів
мешканцям Закарпаття, та заяви угорських ультраправих з партії "Йоббік" щодо
надання автономії угорській етнічній меншині у Берегово…

Територія навколо музично-драматичного театру й справді нагадує місце бойових
дій. Але не тому, що тут відбуваються якісь драматичні події – просто жахливий
занепад, від якого хочеться триматися подалі.

Бронзовий Олександр Духнович зосереджено спостерігає, як на протилежному березі
річки Уж діти годують лебедів та качок.

Цей русинський Будитель ненавидів мадяр, але дуже любив московитів, писав оди
на честь царя Миколи Першого та російської армії з нагоди перемоги над турками.
Вважав російську мову справжньою спадкоємницею церковнослов’янської та
слав’яноруської мов. І мріяв, що у майбутньому літературна мова закарпатських
русинів буде засновуватися не на русиньскому підгрунті, а саме на російській
мові.

До речі, проект створення самостійної русинської нації був для нього чужим –
тобто він був не сепаратистом у сучасному розумінні, а навпаки – соборником. Бо
мріяв, що одного разу русини Закарпаття та Галичини стануть єдиним народом з
московітами.

Така собі парадигма "русского міра" 19-го століття, але у нинішньому контексті
шестирічної війни з Росією, це сприймається якось зовсім не актуально.

В кафе на вулиці Льва Толстого зустрічаюся з Владиславом Товтином, популярним
ужгородським екскурсоводом.

\ifcmt
img_begin 
    url https://img.pravda.com/images/doc/c/b/cb432ed-dscf1079.jpg
    caption Владислав Товтин: "Галичани не дуже люблять згадувати, що ще 100 років тому вони також були русинами. Я їм кажу: "У вас по хатам портрет найвідомішого русина – Івана Франко. Навіть Степан Бандера по документам – русин"
        width 0.7
img_end
\fi

Про те, що він русин, Владислав дізнався від діда, коли був ще зовсім малим.
Але він вважає, що русини – це давня назва всіх українців, а у неорусинському
руху, що був ініційований Полом Магочієм, занадто багато постмодернізму та
"неадекватів".

– Взагалі, чому люди починають себе ідентифікувати з русинами, а не з
українцями? – питає Владислав і сам собі відповідає. – У їхніх очах Україна як
держава – це невдаха у політичному та економічному сенсі. З цих причин на Сході
стають росіянами, а тут стають русинами. Наприклад, русини у Словаччині не
бажають себе ідентифікувати з українцями, бо українець – то бідний сусід зі
Сходу.

Питаю про нову русинську генерацію на Закарпатті – де вона? Владислав сміється
і відповідає – у соцмережах. До того ж ця молодь наполягає, що займається
виключно русинською культурою, а не політикою.

– Вони не розуміють, якщо ти використовуєш русинську символіку, то ти вже у
політиці. Взагалі, у цієї нової генерації є декілька цікавих проєктів, але у
більшості випадків… починаєш копати – і раптом виповзає якесь "дрімуче
православіє" або москвофілія.

Якщо розглядати русинське питання крізь призму історії, воно так або інакше
завжди впирається в москвофільство.

– В Ужгороді багато хто з русинів спілкувався російською мовою ще при
Чехословаччині, – каже Владіслав. – Зараз, наприклад, ми п’ємо каву на вулиці
Льва Толстого, яка так називалась ще у 20-х роках ХХ століття. Можемо, до речі,
прогулятися до "Снека" – там є багато цікавих експонатів того періоду.

"Снек" – маленьке кафе, що має офіційну назву "Під замком", але так його
називають лише туристи. Власник "Снеку" на ім’я Дьюри – румун, який вважає себе
угорцем, але крім румунської, угорської, української, російської та русинської
добре спілкується ще й італійською. Як справжній добрий газда він миттєво
приносить нам з Владиславом по чарці домашньої сливовиці.

Цей колоритний заклад дещо нагадує кафе "У Ріка" з фільму "Касабланка". Тут
також за сусідніми столиками можна інколи побачити людей з протилежних
політичних таборів, хлопців з "Карпатської Січі", угорських націоналістів.

Але сьогодні тут пусто. Відсутність людей мене дивує – Закарпаття оголошено
"червоною зоною", і вже завтра всі кафе, бари і ресторани мають бути зачинені
на два тижні. Але, судячи з усього, в Ужгороді немає звички "нагулятися
заздалегідь".

У "Снеку" зібрано багато речей австро-угорського та чехословацького періодів, у
тому числі й таблички з державних та громадських установ або зі старими назвами
ужгородських вулиць. І роздивляючись цей антикваріат, усвідомлюєш, що російська
мова на Закарпатті у 1920-х роках зустрічалась доволі часто. Та й взагалі до
1946 року всередині русинського руху були течії "укрофілов", "русофілов",
"мадярофілов", "чехофілов".

\ifcmt
img_begin 
        url https://img.pravda.com/images/doc/1/4/14ec600-kollazh.jpg
        caption Вивчати історію Закарпаття краще за все, розглядаючи інтер'єр ужгородського кафе "Снек". Саме тут розумієш, чому так зване русинське "москвофільство" – не сучасна вигадка, а історична традиція
        width 0.7
img_end
\fi

Це природно – бездержавний народ завжди буде шукати собі "парасольку".

Я питаю Владислава, у чому полягає проблема визнання Україною підкарпатських
русинів окремим народом, а їхню "бесіду" – окремою мовою. Чи не логічніше
вибити цей козир з рук Москви, Будапешта і Праги?

– Ти вибиваєш цей козир з рук Росії, але тим самим даєш іншій, – вважає
Владислав. – Чому Закарпаття українське? Тому що радянська влада його приєднала
до України. На яких підставах? Тому що на Закарпатті проживають етнічні
українці. І якщо зараз Україна погоджується, що русини – то не українці, таким
чином Закарпаття перетворюється на територію, що безпідставно та несправедливо
була колись приєднана до України.

Якщо русини не українці, то й договір про передачу Закарпаття до складу СРСР
недійсний. Бо офіційне формулювання – возз'єднання українського народу з
частиною українського народу, яка просто називаються інакше – русини. 

А тоді ми назавжди втрачаємо Крим. Тоді приєднання Криму до України в 1954 році
– також незаконне. Бо на той час українців в Криму була меншість від загального
населення. За цією логікою, приєднання Закарпаття до України – це обман, а
приєднання Криму – обман ще більший. Ви розумієте? Це політика, а не
етнографія, тут треба усе враховувати на пару кроків вперед.

Владислав запевняє мене, що русинське питання і проблеми, пов’язані з ним –
явище не унікальне. Це звичайна проблема розмитості ідентифікації на
прикордонні.

У Чехії, наприклад, ті самі проблеми з мораванами – Словаччина їх визнає
окремою нацією, а чехи – ні. Тому коли хтось з закарпатських русинів їздить до
Земана, а потім розповідає, що президент Чехії обіцяв допомогти "натиснути на
українську владу", той або бреше, або нічого не розуміє – ніколи Земан не стане
на захист русинів. Тому що йому у ту ж мить прилетить з Моравії.

\ifcmt
img_begin 
        url https://img.pravda.com/images/doc/5/c/5ce1d41-dscf1108.jpg
        caption Владислав Товтин: "У чому специфіка закарпатської мови? От кажуть – у вас тут багато чого намішано. Так, запозичення всюди є. Але в нас тут ще є і дуже багато архаїзмів. На Закарпатті як периферії збереглася та мова, якою писали літописи. Навіть не церковнослов’янська, а саме літописна".
        width 0.7
img_end
\fi

– Ще з часів Духновича та Фенцика існує русинська традиція любити московитів, –
пояснює Владислав. – Навіть русини, які проживають у інших державах – завзяті
москвофіли. У них всюди портрети Путіна, світлини Красної площі, російські
прапори.

Коли до них приїжджає Пол Магочій, він кожен раз питає – ви що, здуріли, навіщо
ця маячня? Але навіть його слова не діють. Тому що традиція. І Москва працює з
цією традицією. А Україна не працює навіть з закарпатськими русинами. Зараз
наша держава намагається вибудувати таку струнку красиву українську
ідентичність, але історія набагато складніша за ідеологічні канони.

