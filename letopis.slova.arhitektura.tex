% vim: keymap=russian-jcukenwin
%%beginhead 
 
%%file slova.arhitektura
%%parent slova
 
%%url 
 
%%author 
%%author_id 
%%author_url 
 
%%tags 
%%title 
 
%%endhead 
\chapter{Архитектура}

%%%cit
%%%cit_head
%%%cit_pic

\ifcmt
  tab_begin cols=2
     pic https://bigkyiv.com.ua/wp-content/uploads/2021/06/000-5-800x600.jpg
     pic https://bigkyiv.com.ua/wp-content/uploads/2021/06/001.-4.jpg
  tab_end
\fi

%%%cit_text
Андрій Меленський – той, хто зробив Київ цілісним.  Роки життя: 1766-1833.
Стиль: ампір.
Нажаль сьогодні про \emph{архітектора} Андрія Меленського мало хто знає, хоч ми і
послуговуємося його творчою спадщиною. Справа в тому, що саме Андрій Іванович
зробив Київ таким, яким сьогодні ми його знаємо і любимо.
У 1799 році Меленський\footnote{
Меленский Андрей Иванович. Родился в Москве в 1766 году, умер 1 января 1833
года в Киеве в возрасте 63 года. Русско-украинский архитектор, являлся первым
главным архитектором Киева (1799-1829гг).  Учился в Москве и С.-Петербурге, в
том числе у Джакомо Кваренги под руководством которого принимал участие в
построении дворцов в С.-Петербурге. Большое влияние на мастерство и
мировозрение в творчестве Меленского оказали В. Баженов и М. Казаков,  
\url{https://navoprosotveta.net/melenskij-andrej-ivanovich.html}
} приїздить до столиці України та обіймає новостворену
посаду \emph{головного архітектора}. В цей час Київ нагадує не стільки цілісне місто,
скільки набір шматків землі, розділених рівчаками. Ні про який цілісний
\emph{архітектурний} ансамбль тут і мови немає
%%%cit_comment
%%%cit_title
\citTitle{До Дня архітектури: легендарні зодчі Києва}, 
Тетяна Ніжинська, bigkyiv.com.ua, 01.07.2021
%%%endcit

