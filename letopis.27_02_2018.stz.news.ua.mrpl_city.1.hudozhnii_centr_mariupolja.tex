% vim: keymap=russian-jcukenwin
%%beginhead 
 
%%file 27_02_2018.stz.news.ua.mrpl_city.1.hudozhnii_centr_mariupolja
%%parent 27_02_2018
 
%%url https://mrpl.city/blogs/view/hudozhnij-tsentr-mariupolya
 
%%author_id demidko_olga.mariupol,news.ua.mrpl_city
%%date 
 
%%tags 
%%title Художній центр Маріуполя
 
%%endhead 
 
\subsection{Художній центр Маріуполя}
\label{sec:27_02_2018.stz.news.ua.mrpl_city.1.hudozhnii_centr_mariupolja}
 
\Purl{https://mrpl.city/blogs/view/hudozhnij-tsentr-mariupolya}
\ifcmt
 author_begin
   author_id demidko_olga.mariupol,news.ua.mrpl_city
 author_end
\fi

Якщо маріупольці хочуть насолодитися живописом і поринути у світ творчості та
натхнення, то краще за все завітати до Маріупольського художнього музею імені
Куїнджі, який з легкістю розширить кругозір і допоможе відчути та побачити
прекрасне. Маріупольці пишаються своїм земляком – Архипом Івановичем Куїнджі,
ім'ям якого і названо музей.

\ii{27_02_2018.stz.news.ua.mrpl_city.1.hudozhnii_centr_mariupolja.pic.1}

Архип Іванович Куїнджі – унікальний пейзажист другої половини XIX – початку ХХ
століття. Його приваблювали виняткові й прекрасні ситуації в житті природи,
пов’язані з незвичайним місячним світлом або сонячним освітленням. Серйозне
захоплення астрономією, фізикою кольору і світла знайшло своє відображення в
живописі художника. Те, що в Маріуполі діє художній музей ім. Куїнджі, який має
власну унікальну історію і по праву може вважатися художнім та мистецьким
центром міста, безперечно, є важливим культурним здобутком маріупольців.

\ii{27_02_2018.stz.news.ua.mrpl_city.1.hudozhnii_centr_mariupolja.pic.2}

Будівля, в якій розташовано музей, – старовинний особняк 1902 року, виконаний
в стилі північний модерн. Цей особняк – весільний подарунок голови
Маріупольської земської управи Олександра Гозадінова з нагоди одруження його
дочки з Гіацінтовим Василем Івановичем, засновником Маріупольського реального
училища. Після революції будинок було націоналізовано, в ньому розташовувалася
історична бібліотека та партійний архів.

Збереглися відомості, що за особняком був розкішний сад. Звідти будинок був ще
красивішим, ніж з вулиці. На жаль, особняк почали грабувати в Громадянську
війну, не берегли й потім, а довершили німці під час окупації. Тут були
рідкісної краси музейні складальні паркети, дивовижні ліпні прикраси. Витвором
мистецтва були різьблені дубові двері, вікна закривалися мідними віконницями.
Їх німці й вкрали в першу чергу, до того ж при відступі спалили будинок.
Будівля згоріла, а двері якимось дивом вціліли. Пізніше вони кудись зникли.

Після відновлення будинку в ньому знаходився аптечний склад. Цікаво, що ще в
липні 1914 року в Маріупольську міську думу прийшов лист з Москви від
Московського відділення Товариства художників імені А. І. Куїнджі, в якому
місту пропонувалося десять полотен великого майстра. Однак місто не мало
приміщення для розміщення картин. Незабаром почалася Перша світова війна.
Рішення про відкриття художнього музею імені А. І. Куїнджі відкладалося
протягом усього XX століття.

У 1966 році Маріуполь отримав картини свого знаменитого уродженця. Міністерство
культури СРСР прийняло рішення про дарування трьох робіт Куїнджі: етюдів
\enquote{Ельбрус}, \enquote{Осінь. Крим} та ескізу \enquote{Червоний захід}. Проте побачити роботи
судилося пізніше. Адже лише в 1997  році Маріупольська міська рада приймає
рішення про передачу будівлі в комунальну власність краєзнавчого музею під
майбутню філію. Після реставрації, яка затягнулася на 13 років, в будівлі 30
жовтня 2010 року було відкрито Художній музей імені Куїнджі (через 100 років
після смерті великого живописця).

\ii{27_02_2018.stz.news.ua.mrpl_city.1.hudozhnii_centr_mariupolja.pic.3}
\ii{27_02_2018.stz.news.ua.mrpl_city.1.hudozhnii_centr_mariupolja.pic.4}

На першому поверсі в трьох залах музею розмістилися експозиції, що розповідають
про життя і творчість Архипа Івановича Куїнджі: світлини, документи, листи,
предмети меблів того часу. У музеї знаходиться купіль з церкви Різдва Пресвятої
Богородиці, в якій хрестили майбутнього художника. Експозиція музею показує всі
етапи життя видатного пейзажиста. У музеї представлені копії робіт художника в
натуральну величину.

\ii{27_02_2018.stz.news.ua.mrpl_city.1.hudozhnii_centr_mariupolja.pic.5}

У четвертому залі знаходяться живописні та графічні роботи сучасників А. І.
Куїнджі: І. К. Айвазовського, А. П. Боголюбова, В. В. Верещагіна, Н. Н.
Дубівського, Л. Ф. Лагоріо, І. І. Шишкіна. Тут же знаходиться портрет А.
Куїнджі, виконаний його учнем Г. О. Калмиковим. У експозиційних залах другого
поверху періодично змінюється тематика: проводяться виставки живопису, графіки,
фотовиставки, виставки декоративно-прикладного мистецтва.

\ii{27_02_2018.stz.news.ua.mrpl_city.1.hudozhnii_centr_mariupolja.pic.6}

Художня колекція музею містить понад дві тисячі експонатів. Музей має
унікальну колекцію картин талановитих художників другої половини XX століття:
Миколи Глущенка, Михайла Дерегуса, Тетяни Яблонської, Олексія Грицая,
подарованим місту після проведення першої республіканської виставки пейзажу
\enquote{Меморіал А. І. Куїнджі} в 1973 році. У фондах музею – роботи маріупольських
художників, народного художника України Віктора Пономарьова, заслужених
художників України Н. Бендрика, С. Баранніка, О. Ковальова, В. Кота, В.
Кофанова, Є. Скорлупина картини Л. Кузьменкова та В. Константинова.

Незважаючи на те що процес відкриття художнього музею розтягнувся в часі,
сьогодні кожен маріуполець знає, де розташовано найпрестижніший виставковий
простір у місті й куди слід завітати, щоб отримати естетичне задоволення.

\ii{27_02_2018.stz.news.ua.mrpl_city.1.hudozhnii_centr_mariupolja.pic.7}
\ii{27_02_2018.stz.news.ua.mrpl_city.1.hudozhnii_centr_mariupolja.pic.8}
\ii{27_02_2018.stz.news.ua.mrpl_city.1.hudozhnii_centr_mariupolja.pic.9}
