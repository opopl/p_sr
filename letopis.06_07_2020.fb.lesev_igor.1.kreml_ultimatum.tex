% vim: keymap=russian-jcukenwin
%%beginhead 
 
%%file 06_07_2020.fb.lesev_igor.1.kreml_ultimatum
%%parent 06_07_2020
 
%%url https://www.facebook.com/permalink.php?story_fbid=3340760522621694&id=100000633379839
 
%%author_id lesev_igor
%%date 
 
%%tags dnr,donbass,geopolitika,kreml,lnr,minsk_dogovor,rossia,ukraina,ultimatum,vojna
%%title Об ультиматуме Кремля
 
%%endhead 
 
\subsection{Об ультиматуме Кремля}
\label{sec:06_07_2020.fb.lesev_igor.1.kreml_ultimatum}
 
\Purl{https://www.facebook.com/permalink.php?story_fbid=3340760522621694&id=100000633379839}
\ifcmt
 author_begin
   author_id lesev_igor
 author_end
\fi

Об ультиматуме Кремля

Сразу скажу, у меня нет прямого рабочего телефона ни Путина, ни Дмитрия Козака.
Поэтому предполагаю действия России исходя из своих умозаключений.

Для начала, конечно же, никакого «ультиматума» Козака или Кремля не существует.
Пацанва пытается максимально детализировать отношения в рамках Минского
процесса. Позиция правильная и ставящая именно Киев в дурацкое положение. И это
ставит на растяжку даже не Украину, а в первую очередь Францию и Германию. Они
ведь тоже все видят, и видят уже 5 лет. Но есть установка от Большого Брата –
расписываться за постмайданную Украину несмотря ни на что.

\ifcmt
  ig https://scontent-frt3-2.xx.fbcdn.net/v/t1.6435-9/107104989_3340759289288484_8845620105834256873_n.jpg?_nc_cat=101&ccb=1-5&_nc_sid=730e14&_nc_ohc=QqK5LvcpbY8AX-fF-4-&_nc_ht=scontent-frt3-2.xx&oh=42c8081df07cac81c45a6948c0a782b7&oe=61A43B18
  @width 0.4
	@wrap \InsertBoxR{0}
\fi

Но вот сейчас максимально удобный момент для Москвы чуток надавить на Берлин и
Париж в плане буквоедства. Американцы быкуют с «Северным потоком-2» уже по
беспределу. Да, та же Германия и так имеет условный геополитический
суверенитет. Но ведь и на дворе не 1946 год, чтобы потрахивать подавленных
немцев в любой экзотической позе. Какое-никакое национальное достоинство у
немцев имеется. Особенно, когда речь заходит о коммерции, где вроде бы не было
особых ограничений даже во времена «холодной войны».

Поэтому формализация по Минску – это очень грамотно. Это как строительство
Вавилонской башни. Все понимают, что чушь и никто в реальности не собирается
строить. Но никто первым не собирается признать нелепость, а тут еще и требуют
подробный архитектурный план объекта.

Из подобной риторики Кремля выходит еще один очевидный вывод.
Самопровозглашенные республики под любым предлогам втюхивать в Большую Украину
там не собираются. И это характеризует не столько сложившиеся
украинско-российские отношения – думаю, обе стороны толком не понимают их
особенности – сколько о тотальном недоверии Москвы к любой глобальной сделке с
Западом.

Это значит, что ЛДНР быть. Они уже существуют дольше, чем УНР, Директория и
прочие ЗУНР о которых в наших школьных учебниках написано непропорционально и
до дикости много.

Из закидона упомянутого Козака ориентировочно видится следующий поэтапный план
Кремля:

а) выбить признание Берлина и Парижа, что Минск-2 окончательно сорван
исключительно по вине Украины;

б) раз это так, запустить процесс отмены санкций Евросоюза;

в) узаконить некий новый статус ЛДНР, но уже с привлечением международного
сообщества или (если это не удастся) в одностороннем порядке Россией;

г) «принуждение к миру» на линии разграничения.

А что же Украина? В Украине все тупо и бесперспективно, а иначе и не может
быть, когда страна сначала отдала свою субъектность, а затем даже не пытается
ее вернуть.

У команды Зеленского по Донбассу нет ни плана А, ни плана Б. Да и вообще
непонятно, существует ли у Зеленского какая-то команда.

МИД у нас возглавляет Кулеба – абсолютно проспиртованный порохобот уровня
Климкина. Да и не в Кулебе даже дело. Возьми любого клерка из МИДа и попадешь в
медицинский раздел «Заболевания головного мозга». Думаю, там уборщицы между
собой, когда здороваются, показывают короткую зигу. Легче весь МИД сжечь и
набрать прохожих с улицы, нежели перестроить это заведение под те 73\%, которые
голосовали за Зеленского.

Поэтому от МИДа нам ожидать нечего. Он порошенковский, а значит конченный. Но и
через Офис ничего протащить не удается, потому что под любую инициативу
президента должна быть сформирована силовая вертикаль. А ее нет. Зато есть
Аваков, который рассказывает о «красных линиях» и недопустимости разведения
войск на Донбассе.

Ну и в информационном плане Зе тянут на дно. Истории о том, как Украина
готовится к обороне Херсонской области – это уже клиника времен
Порошенко-Турчинова. Только сейчас вроде бы Зе президент. Но об угрозе «захвата
Херсона» рассказывает действующий командующий ВМС Украины.

В общем, битву за неподконтрольный осколок Донбасса Украина даже не
проигрывает, а проиграла. Нет ни плана, ни инициатив, ни воли, ни желания.
Пилят деньги на канале «Дом», которые даже умалишенные смотреть не смогут.
Играют в профанации с кадрами, вроде Гармаша и Казанского. И уже начались
«угрозы полномасштабного вторжения России». Все, это конец. 100\%-я
порошенковская повестка в действии.

Мы пришли к той истории, когда будущее ЛДНР решается без участия Украины. Хотя,
возможно, Зеленский именно такой мир на востоке и имел ввиду. Когда оно само по
себе все рассосется и не надо напрягаться, чтобы переть через толпу и плыть
против течения. Уютная конформистская позиция президента безсубъектной страны.
