% vim: keymap=russian-jcukenwin
%%beginhead 
 
%%file 21_11_2020.fb.bevz_sergii.1.den_gidnosti
%%parent 21_11_2020
 
%%url https://www.facebook.com/spollloh/posts/209524753957543
 
%%author Бевз, Сергій
%%author_id bevz_sergii
%%author_url 
 
%%tags 
%%title Сьогодні, 21 листопада, ми відзначаємо День Гідності та Свободи
 
%%endhead 
 
\subsection{Сьогодні, 21 листопада, ми відзначаємо День Гідності та Свободи}
\label{sec:21_11_2020.fb.bevz_sergii.1.den_gidnosti}
\Purl{https://www.facebook.com/spollloh/posts/209524753957543}
\ifcmt
	begin_author
   author_id bevz_sergii
	end_author
\fi

Сьогодні, 21 листопада, ми відзначаємо День Гідності та Свободи. 7 років тому
на Майдані почалися перші протести проти політики режиму Януковича. Сотні киян
вийшли на вулицю і відмовились розходитись. Масові виступи проти дій
кримінальної влади відбулись в кількох інших великих містах. Ці події стали
передумовою Революції Гідності, яка назавжди змінила Україну. 

\ifcmt
pic https://scontent.fiev6-1.fna.fbcdn.net/v/t1.0-9/126942168_209524730624212_1491274899506654082_o.jpg?_nc_cat=110&ccb=2&_nc_sid=730e14&_nc_ohc=S520fyqBfoUAX84B-m0&_nc_ht=scontent.fiev6-1.fna&oh=07dfbc7c29a5294771b3abf5a6329756&oe=5FE4D76E
caption День Гідності та Свободи, 21.11.2020
\fi

Революційні процеси 2013-2014 років зміцнили та загартували Націю. Українці
знову продемонстрували, що національна свобода та гідність --- це засадничі
цінності, які нікому не дозволено топтати. 

Сьогодні ми віддаємо належне усім, хто виявив мужність в дні революційної
боротьби проти клептократії. Насамперед, Героям Небесної сотні, які поклали
життя за краще майбутнє Батьківщини. Їхня справа живе. Боротьба триває. Саме
від нас залежить чим вона завершиться: тріумфом антиукраїнської нечисті чи
побудовою нової України.

\paragraph{Yuliya Doktorova}
Одне боляче,що дивлячись зараз на все це, скільки Україна втратила своїх синів
і дочок,нинішня влада пускає все у безодню 
%((((

\paragraph{Симон Мельничук}
Тріумф нечисті навіть у думках не маємо допускати.
