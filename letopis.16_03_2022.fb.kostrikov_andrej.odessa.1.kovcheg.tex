% vim: keymap=russian-jcukenwin
%%beginhead 
 
%%file 16_03_2022.fb.kostrikov_andrej.odessa.1.kovcheg
%%parent 16_03_2022
 
%%url https://www.facebook.com/uho.lot.1/posts/10221937527655587
 
%%author_id kostrikov_andrej.odessa
%%date 
 
%%tags 
%%title КОВЧЕГ (утопическое стихотворение пятилетней давности)
 
%%endhead 
 
\subsection{КОВЧЕГ (утопическое стихотворение пятилетней давности)}
\label{sec:16_03_2022.fb.kostrikov_andrej.odessa.1.kovcheg}
 
\Purl{https://www.facebook.com/uho.lot.1/posts/10221937527655587}
\ifcmt
 author_begin
   author_id kostrikov_andrej.odessa
 author_end
\fi

Когда-то давно я не очень, чтобы твёрдо, но, скажем так, краем глаза верил в
то, что на севере чудом сработает \enquote{аварийный тормоз} в 2010-2012 году. А краем
глаза, потому что от потомства вертухаев и их так и не уехавшей за 20 лет
обслуги хрен дождёшься солидарности. Логика там такая сверху донизу. Это будет
внутренний монолог. Как курсив у Стивена Кинга.

Снова ссу. Чёрт! Надо чем-то срочно заняться с деловым видом. Обмануть лоха
какого-нибудь что ли опять, чтобы набрать побольше бабла? А пока счёт
увеличивается, вот вам цветочки с ангелочками в ленту, френды. Полетели первые
ангелочки и ухоженные детишки к френдам. Полетели в ответ цветочки с котиками.
Этот консенсус означает: \enquote{Пока у нас есть духовные силы на лицемерие, не ссым,
а ищем товар дальше, товарищи!}

Мало-помалу льдину с товарищами размывают тёплые течения. На отколовшихся
кусках уплывают плюющие в лицо напоследок родственники и друзья. Множатся
чёрные рамки вокруг иных фотографий. Иудушки грустят. Но недолго.

Ведь как устроена жизнь в забое у мыши? Где грусть, там и мышли. Где мышли, там
и страх. Где страх, там и жадность. Где жадность, там и лицемерие. По этому
кругу и носится привычно каждая клетка отмороженной двуглавой курицы с двумя
отрубленными головами. Без голов же клетками движет уже исключительно
вегетативная привычка.

\enquote{Привычка свыше нам дана: замена счастию она!} А. С. Пушкин 

Но то была присказка, а сказка впереди.

\ifcmt
  ig https://scontent-mxp1-1.xx.fbcdn.net/v/t39.30808-6/275861825_10221937527215576_2459245791791449037_n.jpg?_nc_cat=101&ccb=1-5&_nc_sid=730e14&_nc_ohc=A2uuv4hl-KoAX-3SkxK&_nc_ht=scontent-mxp1-1.xx&oh=00_AT9cxqpAIfvBriDy4VHIv0Rg1y8d-rzw2bFC4SaoldXLdQ&oe=62665DB6
  @wrap center
  @width 0.7
\fi

КОВЧЕГ (утопическое стихотворение пятилетней давности)

\raggedcolumns
\begin{multicols}{2} % {
\setlength{\parindent}{0pt}

\obeycr
Привязаны к мачте (и с рацией в ухе),
Ковчег сторожат старики и старухи.
Матросы поют. Юнги тащат пожрать.
Сегодня пора кое-что отмечать.
Принёс нынче ветку сюда голубок.
\smallskip
Дождь стих. Скоро суша. И вроде итог.
Но тучи стоят ещё плотной стеной,
Волна как сирена манит за собой.
Шипят все мобилы. Потеющим ухом
Сержант понимает, что в будущем глухо.
\smallskip
Он хочет куда-то в Паттайю слетать.
Он хочет уволиться, чтобы поспать.
Его зае*ало, и нас зае*ало
Совать вновь приёмники под одеяло,
Бояться гулять и смеяться в столице.
\smallskip
Мента зае*ало быть бледнолицым.
Он знает все шифры. Готов показать.
Лишь только утихните, дайте поспать!
Но гомон на улицах не утихает.
То тут, то вон там что-то кто-то взрывает.
\smallskip
То там, то прям тут кто-то тащит кого-то.
Осталась без шлемов еще одна рота.
Коттеджи горят. Все помчались к Рублёвке.
\smallskip
И где ж этот Вовка? А нету тут Вовки!
А где же Димон? Где е*учий Димон?
Димон испарился шо твой чемпион.
На площади Красной мерцают костры...
\smallskip
Там некто Иван. Там стоят три сестры.
Там раста и панк. Там и зек, и десант.
Студент и учитель. Бомжиха и франт.
Там бывший нефтяник из дальней тайги.
\smallskip
Там песни, и дым, и не видно ни зги.
Мы жарим шашлык и плюём в Ильича.
К нам питерцы едут уже, хохоча.
Они к нам везут, в местный наш зоосад
Своих депутатов и прочих утят...
\restorecr
\end{multicols} % }
