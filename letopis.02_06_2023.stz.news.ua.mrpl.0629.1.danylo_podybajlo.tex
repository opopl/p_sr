% vim: keymap=russian-jcukenwin
%%beginhead 
 
%%file 02_06_2023.stz.news.ua.mrpl.0629.1.danylo_podybajlo
%%parent 02_06_2023
 
%%url https://www.0629.com.ua/news/3606235/v-boah-za-bahmut-zaginuv-danilo-podibajlo-poet-i-muzikant-foto-video
 
%%author_id news.ua.mrpl.0629
%%date 
 
%%tags danylo_podybajlo
%%title В боях за Бахмут загинув Данило Подибайло, поет і музикант, - ФОТО, ВІДЕО
 
%%endhead 
 
\subsection{В боях за Бахмут загинув Данило Подибайло, поет і музикант, - ФОТО, ВІДЕО}
\label{sec:02_06_2023.stz.news.ua.mrpl.0629.1.danylo_podybajlo}
 
\Purl{https://www.0629.com.ua/news/3606235/v-boah-za-bahmut-zaginuv-danilo-podibajlo-poet-i-muzikant-foto-video}
\ifcmt
 author_begin
   author_id news.ua.mrpl.0629
 author_end
\fi

\begin{qqquote}
\enquote{Привіт! Як ви?} Це було останнє повідомлення, яке Данило Подибайло відправив
своїй мамі з фронту. Він загинув в боях під Бахмутом, в тій самій Кліщіївці,
про яку так багато розповідають військові експерти.
\end{qqquote}

\enquote{Вибили ворога з Кліщіївки}, \enquote{повибивали тили}, \enquote{створили умови для атаки на
фланги}... Це так технічно звучить, поки безособові слова з військових зведень не
наповнюються обличчями людей, які віддали життя за реалізацію цих військових
завдань...

\ii{02_06_2023.stz.news.ua.mrpl.0629.1.danylo_podybajlo.pic.1}

Данила Подибайла  в Маріуполі називали сином полку з 2014-го, бо він практично
зростав не дома, а в громадському центрі \enquote{Новий Маріуполь}, який створила та
очолила його відома мама, Марія Подибайло. Зростав з Україною в серці.

Пам'ятаєте юнака з гітарою, який палко співав на проукраїнських мітингах в
Маріуполі? Це був Данило. Він не вчився музиці, мабуть, вона в ньому жила
завжди.

Він був музикантом, співаком, поетом, письменником. Мріяв видати книги, глибокі
за змістом, як і він сам. Данило був одним з кращих в місті майстром з
освітлення сцени і звукорежисером.

\ii{02_06_2023.stz.news.ua.mrpl.0629.1.danylo_podybajlo.pic.2}

А ще він вивчав IT і мріяв після війни запустити щось круте, незвичайне, власне
– були у нього ідеї щодо стартапу.

Коли почалась повномасштабна війна, він залишився в Маріуполі і допомагав
військовим. Його командир (прізвище просили не називати) розповідає, що завдяки
Данилу Подибайло світ дізнався про перші авіаційні обстріли Маріуполя. 

\enquote{Він добре розумівся на техніці, і ці його знання допомагали нам виконувати
завдання в оточеному Маріуполі}, - розповідає командир.

\ii{02_06_2023.stz.news.ua.mrpl.0629.1.danylo_podybajlo.pic.3}

А ще Данило волонтерів, розвозив їжу, воду людям, допомагав в
\enquote{Халабуді}.

В березні він разом з іншими зміг вирватись із блокади. Виключно завдяки своїй
винахідливості йому вдалося пройти всі допити і численні ворожі блокпости.

Коли дістався вільної України, сидіти в тилу не став, а вступив до лав
територіальної оборони. Стояв на захисті Києва, потім брав участь у звільненні
Харкова. І от останній його фронт – Бахмут, 128 батальйон 112 бригади ТрО,
підрозділ Адама. 

Друзі кажуть, що він геройствував. Просто неймовірної відваги хлопець. Разом із
побратимом Растою просто дива творив. Це вони з Растою повибивали тили і
резерви кацапів в Кліщіівці. Чим і створили умови для просування флангів. Разом
з Растою і загинули.

Ось відео роботи їхнього підрозділу під Бахмутом. Музика, яка лунає як фон для
нищення ворогів, - це музика Данила Подибайла.

\ii{02_06_2023.stz.news.ua.mrpl.0629.1.danylo_podybajlo.pic.4.video}

\ii{02_06_2023.stz.news.ua.mrpl.0629.1.danylo_podybajlo.slova_materi}

Це слова матері героя, яка прощається із сином.

\ii{02_06_2023.stz.news.ua.mrpl.0629.1.danylo_podybajlo.pic.5}

За останню добу сотні людей, хто знав Данила та його матір, не втримують сліз.

Олена Іванців: "Несамовито боляче... В моїх спогадах Данило такий світлий
патлатий юнак, гордість матері та всієї родини. Мав би ще довго жити, співати,
кохати... Але Росія його вбила. 

Дякую тобі, Данило, що захищав нас і нашу державу до останнього подиху. Дякую
матері Марія Подибайло за сина - героя. Світла й вічна пам'ять герою.

Переможемо й назвемо одну з вулиць Маріуполя твоїм ім'ям..."

\ii{02_06_2023.stz.news.ua.mrpl.0629.1.danylo_podybajlo.pic.6}

Світлана Шурда: "Ми разом виїжджали з Маріуполя в травні 2022. То була ще та
пригода - 17 годин до Бердянська - на кожному блокпосту догляд питання, а потім
в Бердянську затримання і допит, але завдяки кмітливості Даню відпустили. Потім
ще 2 доби до Запоріжжя.

Але ми віїхали з того мороку. ДАНЯ це дуже талановита, творча, мужня і мудра
ЛЮДИНА - ЧОЛОВІК і надійний друг. 

Нажаль дуже тяжкі втрати. Дуже велика ціна".

\ii{02_06_2023.stz.news.ua.mrpl.0629.1.danylo_podybajlo.pic.7}

\textbf{Надія Умріш:} «І знову нестерпний біль і через сльози не бачу букв...
Ридання не зменшують болю... Марія Подибайло, Марічко, сестричко наша,
подруго... Як розділити твій біль?! Рідненька моя... Всі слова зникли у
несамовитому крику: "Ні!!!! Тільки не Данилко!!!"

Хлопче... Маріуполь тебе пам'ятатиме завжди. Твій голос, твою усмішку, твої
пісні під гітару... Пам'ятатимуть тебе і воїни, перед якими ти виступав, і
побратими, з якими бив ту нечисть, що товчеться по рідній землі, і учасники
фестивалю "Червона рута" у Маріуполі.

....Що ж я пишу?... Не відаю... Плачу....

Марічко, серденько, ти завжди пишалась своїм сином. Гідний Син України. Він
мав кого наслідувати... Мав приклад — маму. Ти виховала справжнього
чоловіка, воїна".

\ii{02_06_2023.stz.news.ua.mrpl.0629.1.danylo_podybajlo.pic.8}

\textbf{Ірина Нагорна:} "На фронті під Бахмутом загинув маріуполець Данило
Подибайло... Чудовий, талановитий молодий хлопець. Патріот України, патріот
нашого міста..

Це син моєї доброї приятельки Марічки Подибайло. Пам'ятаю його з 2004 року, ще
маленьким хлопчиком... а потім серпень 2014 року і це вже юнак, який під час
мітингу на площі Свободи співає патріотичні пісні для тисяч маріупольців та
присягає на вірність Україні, коли в декількох кілометрах від міста йдуть бої і
місто здригається від вибухів... І зараз він став на захист Батьківщини...

Україна втрачає найкращих... Але Герої не вмирають! Світла пам'ять, Слава та
Честь Воїну - Герою Данилові! 

Щирі співчуття матері Марія Подибайло та рідним Героя... Немає слів..."

\textbf{Іван Синєпалов:} "Ми Данилом познайомилися без малого 10 років тому на
епічному концерті Океану Ельзи у Харкові. Йому було 16.

А потім Майдан, а потім війна. У Новому Маріуполі він був як син полку. Хоч і
малий, а цікавий. І навіть пісні його – то діло другорядне, хоча вони мені
справді подобаються (от і на цьому фото, яке я знайшов на його сторінці, він
щось співає у нашій студії). Просто з ним завжди можна було добре провести час.

\ii{02_06_2023.stz.news.ua.mrpl.0629.1.danylo_podybajlo.pic.9}

Тоді в мене ще вирувала дурна думка екранізувати \enquote{Поклоніння ящірці} і я собі
уявляв його в ролі головного героя. Підходящий типаж.

Вічно ганяв його за те, що він встановлював Steam на наш робочий комп.

Не скажу, що ми прямо продовжували підтримувати контакт, коли Новий Маріуполь
розійшовся, але все одно завжди був радий випадковим зустрічам.

Переживав за нього, коли він сидів у блокадному Маріуполі. Коли виїхав – я
видихнув з полегшенням.

Останнім часом, дивлячись на його сторіз із Бахмута, постійно ловив себе на
думці, що, бляха, це ж той самий 16-річний чувак, із яким ми вривалися під
Вакарчука, абсолютно неагресивна і миролюбна людина, – і от тепер він косить
рсн в найгарячішій точці, поки я відсиджуюсь у тилу.

А тепер його нема.

Не вкладається в голові. Пишу і не вірю.

\enquote{Нехай стане зброєю серце, наповнене любов'ю}, – так було в його \enquote{Синьо-жовтій
балаклаві}. А любові в нього вистачило на всіх, бо ніхто більшої любові не має
над ту, як хто свою душу поклав би за друзів своїх.

Спочивай із миром.

P.S. Послухайте його пісню. Вона хороша. 

Хай звучить музика, в якій Даня  - живе. 

\url{https://www.youtube.com/watch?v=yT4gMua_j_c}

\ii{02_06_2023.stz.news.ua.mrpl.0629.1.danylo_podybajlo.pic.10}
