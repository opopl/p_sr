% vim: keymap=russian-jcukenwin
%%beginhead 
 
%%file slova.sinoptik
%%parent slova
 
%%url 
 
%%author_id 
%%date 
 
%%tags 
%%title 
 
%%endhead 
\chapter{Синоптик}

%%%cit
%%%cit_head
%%%cit_pic
%%%cit_text
Украина. Тут внимательно смотрим на оптимизм \emph{синоптиков}. Вне зависимости от
того, что Беларусь прекращает поставки электроэнергии на Украину, укрТВ бодро
сообщает, что 19 ноября на Западной Украине будет дождливая и прохладная
погода. Температура прогреется до +3...+10 градусов. Ночью потеплеет – будет
0...+5 градусов.  В центральном, южном, северном и восточном регионах погода
также будет дождливой. Днем – до +4...+9 градусов, ночью – от +2...+4 градусов
в Одесской области и до -4...-6 градусов на Донбассе.  Мокрый снег – не везде.
Он будет выпадать на Сумщине, местами в Черниговской, Киевской, Полтавской,
Харьковской областях и в Карпатах.  \emph{Синоптик} также рассказала, что в Киеве
атмосфера активизируется – пройдут дожди. Температура воздуха ночью будет
колебаться около нуля. Днем ожидают до +5 градусов. Заморозков не было и нет.
мокрый снег и дожди - значит ещё тепло ...
%%%cit_comment
%%%cit_title
\citTitle{Очевидно, не смотрели шведы советский мультик "Приключения капитана Врунгеля" с его незабвенным: 
"Как вы яхту назовете"...}, Красная Книга Человечества..., zen.yandex.ru, 19.11.2021
%%%endcit
