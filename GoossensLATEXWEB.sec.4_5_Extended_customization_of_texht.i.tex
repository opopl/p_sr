
\isubsec{4_5_Extended_customization_of_texht}{Extended customization of \texht}

Most \ \LaTeX\  users should require very little background information,
if any, in addition to that already covered in this chapter. But \texht 
is a large system with many facets to explore. A taste of that world is
provided in the following sections, which deal with some aspects of
customizing and running the system. 

Most readers will probably be best served by quickly skimming over the
following sections to get a general impression of the topics addressed.
The details have little bearing on the How of content in these sections,
and they can be ignored until they are needed for handling specific
requirements. 

\isubsubsec{4_5_1_Configuration_files}{Configuration files}

\texht introduces intermediate interfaces of its own, located between
the interfaces used by \ \LaTeX\  within the source files, and those
used by HTML within the output files. It does this by placing ``hooks''
in the \ \LaTeX\  style files; these are commands that the user can
redefine to get the effect they want. The intermediate interfaces
separate themselves from the style concerns of \ \LaTeX, on one hand,
and offer structures similar to those of HTML, on the other. As a
result, users can quite easily tailor a different outcome just by
defining the different hooks to produce appropriate HTML code. 

%%page page_192                                                  <<<---3
The conventional wisdom of placing definitions together in separate
customization files applies also to the configuration commands of
\texht. However, an additional motivation for the configuration files stems from the need to direct 
the output to fit the structural requirements of HTML files. 

Specifically an HTML file consists of a header and a body, each part expecting 
a different type of content. The configuration files identify these parts and provide 
the content for the header. 

\ipar{Implicit_and_explicit_files}{Implicit and explicit files}

If the first package option is not the name of a configuration file, a
default configuration is used. It looks like this: 

\begin{lstlisting}

\Preamble{options} 
\begin{document} 
\EndPreamble 

\end{lstlisting}

However, if the first option \emph{does} refer to a configuration file, then the configuration 
file must have the following structure: 

\begin{lstlisting}

early definitions 
\Preamble{options} 
definitions
\begin{document} 
insertions into the header of the HTML file 
\EndPreamble 

\end{lstlisting}

\reffig{4-8} shows an example of source and configuration files, as well as the HTML 
and CSS files they produce. 

Upon reaching the \verb|\usepackage| command, the file \verb|tex4ht.sty| is partially 
loaded to scan a few definitions. Then the configuration file is read until the 
\verb|\Preamble| command is encountered. The remainder of the style file is read and 
acted upon when the \verb|\begin{document}| command in the source document is 
reached. 

With the exception of a package option standing for the name of a
configuration file, the distribution of the other options between the
\verb|\usepackage| and the \verb|\Preamble| command is unimportant.
Moreover, unlike the case for the first option of the
\verb|\usepackage|, no restriction is made on the type of the option
that appears first in the \verb|\Preamble| command. 

\ipar{Embedded_configuration}{Embedded configuration}

The configuration file can also be embedded directly within the source
document, instead of being indirectly incorporated through a
\verb|\usepackage| command. Such an approach might make the placement of
the \verb|\begin{document}| instruction in the configuration file
clearer. However, to preserve the authoring style promoted by \ \LaTeX\, 
users are highly discouraged from employing this approach. 
 
%%page page_193                                                  <<<---3

\iii{vb_page_193}

The embedding can be achieved by replacing the instruction 
\verb|\usepackage[|\emph{options}\verb|]{tex4ht}| with an \verb|\input{tex4h1.sty}| command, and 
substituting the \verb|\begin{document}| in the source document with the contents of 
the configuration file. In such a case, the options of the \verb|\usepackage| command 
should migrate into the list of options of the \verb|\Preamble| command. 

The following source document is the source of \reffig{4-8}(a), with the
configuration file of \reffig{4-8}(b) embedded in it. 

\begin{lstlisting}

\documentclass{article} 
  \input{tex4ht.sty} 
  \Preamble{htm1} 
    \Css{UL { border : solid 1px; }} 
\begin{document} 
    \HCode{<META NAME="description" CONTENT="examp1e">} 
  \EndPreamble 
  \begin{itemize} 
    \item First 
    \item Second 
  \end{itemize} 

\end{document} 

\end{lstlisting}
\isubsubsec{4_5_2_Tables_of_contents}{Tables of contents }

The \verb|\tableofcontents| command of \ \LaTeX\  is enriched with new
features in \texht. To allow for easy control over the kind of entries
it includes, the way it 
 
%%page page_194                                                  <<<---3
 
is presented, and the locations where it can be included, some of these
features are indirectly activated by the package options 1, 2, 3, and 4
(see \refsec{4_1_1_Package_options} on 15 6). 

Remember that because one \ \LaTeX\  file will often generate many HTML
files, each output may have its own table of contents. Hence, we use the
term ``tables of contents'' instead of the usual ``table of contents.''

\ipar{Choice_of_entries}{Choice of entries }

The kinds of entries to be included in the tables of contents are
determined by \ \LaTeX\  in the usual way. As an alternative, \texht adds a
variant command, \verb|\tableofcontents[|\emph{units}\verb|]|, in which
the kind of entries need to be explicitly specified. The parameter units
is a comma-separated list of names of sectioning commands (without
backslashes). Starred versions (ending in *) are replaced by names with
like prefixes, and appendixes are requested by the word appendix. Thus 

\begin{verbatim}
\tableofcontents[chapter,appendix,section,likesection] 
\end{verbatim}

requests a table of contents with entries pointing to the logical units
created by the \verb|\chapter|, \verb|\section|, and \verb|\section*|
commands. 

\ipar{Local_tables_of_contents}{Local tables of contents }

The command 

\begin{lstlisting}
\TocAt{units} 
\end{lstlisting}

requests a table of contents at the start of each logical unit of the specified type. 
The parameter \emph{units} is a comma-separated list similar to that offered for the new 
variant of the \verb|\tableofcontents| command; the first name in the list specifies the 
unit that is to have a local table of contents. The other names specify the kind of 
entries to be included in the table. If they are preceded with a slash, they specify 
\emph{termination points} for the tables of contents. Thus 

\begin{lstlisting}
\TocAt[chapter,section,/likesection] 
\end{lstlisting}

requests a table of contents at the start of each chapter. The entries
corresponding to the \verb|\section| commands that follow should be
included, but the list terminates upon reaching a \verb|\section*| or
the next \verb|\chapter|. 

The tables requested by the command \verb|\TocAt{|\emph{units}\verb|}| appear immediately after 
the titles of the section units. The variant 

\begin{lstlisting}
\TocAt*{units} 
\end{lstlisting}

produces similar tables that follow the preambles of the units instead
of immediately following the titles. 
 
%%page page_195                                                  <<<---3
 
\ipar{Configuring_the_entries}{Configuring the entries}

Each entry in the table of contents is derived from three fields: a
mark, a title, and a number. Typically the mark is the number of a
section (or it is simply empty), and the page number is of little
significance in this context. 

The following command can be used to determine how the contents entries 
for the logical units of type \emph{unit} are to be created. The unit names follow the same 
conventions as those used in the parameter of the enhanced \verb|\tableofcontents| 
command. 

\begin{lstlisting}
\ConfigureToc{unit}{mark}{title}{page}{end} 
\end{lstlisting}

Each contents entry will be composed of the mark parameter followed by the mark 
field, the title parameter followed by the title, and the page parameter followed 
by the page number. Finally comes the end parameter. If any parameter is empty, 
the corresponding field is omitted in the table of contents. The effect is shown in 
\reffig{4-9}. 

\ipar{4_5_2_4_Configuring_the_tables}{Configuring the tables}

Tables of contents have hooks before their entry points, after their exit points, after 
their last entries, at the start of each nonindented paragraph, and at the start of each 
indented paragraph. The following command configures the hooks: 

\begin{lstlisting}
\Configure{tableofcontents}{before}{end}{after}{n-par}{i-par} 
\end{lstlisting}

The hook after the last entry is processed within the environments of
the table of contents. The hook after the exit point of the table of
contents is processed within the environment around the tables.  Local
tables of contents can be further configured in a similar manner with
the 
%%page page_196                                                  <<<---3
commands: 

\begin{lstlisting}
\Configure{TocAt}{before}{after} 
\Configure{TocAt*}{before}{after} 
\end{lstlisting}

The new configurations must be supplied before the \verb|\TocAt| and \verb|\TocAt*| commands. 

\isubsubsec{4_5_3_Parts_chapters_sections_and_so_on}{Parts, chapters, sections, and so on}

Sectioning commands determine the underlying structures of documents and they
quite often guide the partitioning of hypertext documents into files. This
subsection shows how such entities can be customized. 

\ipar{4_5_3_1_Configuring_the_boundary_points_and_titles}{Configuring the boundary points and titles}

The sectioning commands produce logical units characterized by their starting
and ending points, as well as by their titles. The following command
contributes content to such units; these are included at the start of the
units, at the end of the units, before their titles, and after their titles,
respectively. 

\begin{lstlisting}
\Configure{unit}{top}{bottom}{before}{after} 
\end{lstlisting}

If the \emph{top} and \emph{bottom} parameters are both empty, that part of the
configuration command is ignored and the old values remain in effect. The same
applies to the parameters before and after. The effect is shown in \reffig{4-10}. 

The unit names follow the same conventions as those provided for the parameters
of the enhanced \verb|\tableofcontents| command. 
 
%%page page_197                                                  <<<---3

\ipar{4_5_3_2_Partitioning_into_files}{Partitioning into files}

The package options 1, 2, 3, and 4 (see \refsec{4_1_1_Package_options} on page 156) implicitly activate 
the command 

\begin{lstlisting}
\CutAt{units} 
\end{lstlisting}

for partitioning the documents into files. The parameter units is a comma-separated 
list of unit names. The first name in the list specifies the logical units for which 
separate hypertext pages are requested. The hypertext pages extend until a unit 
whose name appears in the rest of the list is encountered. Thus 

\begin{lstlisting}
\CutAt{chapter,likechapter,appendix,part,likepart} 
\end{lstlisting}

requests hypertext pages for the logical units defined by \verb|\chapter|. Furthermore, 
the command says that starred chapters, appendixes, parts, and starred parts are 
logical units that should not be included within chapters. 

Typically the \lstinline|\CutAt{units}| command is used along with a
table of contents whose entries provide links to the hypertext pages
that are requested by the command. A possible alternative is to employ
the variant command \lstinline|\CutAt{+units}|, which, besides the
hypertext pages, also creates links to the pages. The effect is shown in
\reffig{4-11}. 

The links to the pages are enclosed between delimiters that are
configurable by the command 
 
%%page page_198                                                  <<<---3
 
\begin{lstlisting}
\Configure{+CutAt}{unit}{ldel}{rdel} 
\end{lstlisting}

Thus 

\begin{lstlisting}
\Configure{+CutAt}{section}{*~}{} 
\end{lstlisting}

requests left delimiters \verb|*~| and no right delimiters for the links to pages that result 
from the \verb|\section| command. 

\ipar{4_5_3_3_Setting_boundary_points}{Setting boundary points}

The \lstinline|\CutAt{units}| and \lstinline|\CutAt{+units}| commands
work out the end points of the sections that are placed in separate
files. The end points of the other sections should be specified using
the following command: 

\begin{lstlisting}
\Configure{endunit}{units} 
\end{lstlisting}

The endmzit stands for a unit name prefixed by an end, and the parameter units is a 
comma-separated list of unit names. 

\ipar{4_5_3_4_Customizing_the_navigation_buttons}{Customizing the navigation buttons}

The hypertext pages of the sections include panels of navigation buttons both at 
the top and the bottom of each page. Each of the buttons is embedded between a 
left and a right delimiter. The buttons point to the next hypertext page, the front 
and the tail of the previous page, the front and the tail of the current page, and 
the parent page. The following command allows you to customize the buttons and 
their delimiters, and the effect is shown in \reffig{4-12}. 

\begin{lstlisting}
\Configure{crosslinks}{ldel}{rdel}{next}{prev}{prev-tail}{front}{tail}{up} 
\end{lstlisting}

The next command deals with the panels themselves. It specifies content to be 
included before and after the front and tail panels, respectively: 

\begin{lstlisting}
\Configure{crosslinks+}{before-front}{after-front}{before-tail}{after-tail} 
\end{lstlisting}

\isubsubsec{4_5_4_Defining_sectioning_commands}{Defining sectioning commands}

New sectioning commands can be introduced through instructions of the form 

\begin{lstlisting}
\NewSection{cmdname}{marker} 
\end{lstlisting}

for the \texht package to be aware of their existence and to offer its
standard services. The parameter marker specifies the markers to be
submitted with the titles to the tables of contents. Typically such
markers either are empty or consist of the sequence number of the
current section unit. An example is shown in \reffig{4-13}. 
 
%%page page_199                                                  <<<---3

\isubsubsec{4_5_5_Lists}{Lists}

The \verb|list| and \verb|trivlist| environments are basic structures of \ \LaTeX, on
top of which quite a few environments are defined. Some of these are
themselves variants of listing environments, for instance, the
\verb|description|, \verb|itemize|, \verb|enumerate|, and \verb|thebibliography| environments. 

Other environments are display-oriented in nature, relying on
empty-label, single-item lists just for their typesetting
characteristics, for instance, the \verb|center|, \verb|flushleft|,
\verb|flushright|, \verb|quotation|, \verb|quote|, \verb|verbatim|, and
\verb|verse| environments.  The theoremlike environments, defined by the
\verb|\newtheorem| command, are also single-item lists, but their titles
are offered as labels of \verb|\item| commands.  The following command
provides content to be included before the lists, after the lists,
before the labels of the items, and after the labels of the items. The
effect is shown in \reffig{4-14}. 

%%page page_200                                                  <<<---3
 
\begin{lstlisting}
\ConfigureList{name}{pre-list}{post-list}{pre-label}{post-label} 
\end{lstlisting}

VVhen list environments are defined in terms of other list environments,
the contribution of \verb|\ConfigureList| applies only to the lists in
the top layer. Since \ \LaTeX\  defines the description environment in
terms of the \verb|list| environment, and \texht configures both of them
with the \verb|\ConfigureList| command, the configuration given to the
\verb|list| environment does not show within the description lists. 

\isubsubsec{4_5_6_Environments}{Environments}

\LaTeX\  environments constructs are customizable by commands invoked at the entry 
and exit points: 

\begin{lstlisting}
\ConfigureEnv{name}{before-env}{after-env}{before-list}{after-list} 
\end{lstlisting}
 
%%page page_201                                                  <<<---3
 
The parameters \emph{before-environment} and \emph{after-environment} specify material
to be placed before and after the named environment; if both parameters
are empty, they are ignored. An example is show in \reffig{4-14}. 

Similarly, if at least one of the parameters \emph{before-list} or
\emph{after-list} is not empty, the environment is assumed to be
realized in terms of a list-making environment.  A call is then made to
\lstinline|\ConfigureList{name}{before-list}{after-list}{}{}| for
configuring the underlying list-making commands. 

\isubsubsec{4_5_7_Tables}{Tables}

\texht goes a long way toward offering satisfactory representations for
tables, but it does not provide a complete solution. Sometimes it fails,
and special configurations or pictures might be called for. 

\ipar{4_5_7_1_The_array_and_tabular_environments}{The array and tabular environments}

The array and tabular environments differ only in that the first is handled in 
math mode and the second is processed in normal mode. The following command 
customizes these environments, before the tables, after the table, before each row, 
after each row, before each entry, and after each entry. 

\begin{lstlisting}
\Configure{table}{pre-tbl}{post-tbl}{pre-row}{post-row}{pre-entry}{post-entry} 
\end{lstlisting}

To help configure the tables, the \verb|\HRow|, \verb|\HCol|, and \verb|\ALIGN| macros can be used. 
\pindx{HRow} \pindx{HCol} \pindx{ALIGN}

The first pair of macros produces the row and column numbers in which
the commands appear; the third macro produces an encoding for the
alignment information of the table, as we show here: 

\begin{verbatim}

\Configure{tabular} {}{} 
  {\HRow: }{\HCode{<BR>}} 
  {}{(\HCol)} 
\begin{tabular}{ccc} 
A&B&C\\ D&E&F 
\end{tabular} 

\end{verbatim}

For a centered column, \verb|\ALIGN| gives a triplet made up of the digit 0, the column number, and the minus character \verb|-|. For left-aligned, right-aligned, and paragraph columns, similar triplets are produced. The only difference is that the characters \verb|<|, \verb|>|, and \verb|p|, respectively, are used instead of \verb|-|. 

Tables with \verb|\multicolumn| entries need a few \ \LaTeX\
compilations to stabilize; \texht slowly learns about the dimension of
the spanning from information provided in earlier compilations. In
configuring contributions to entries of tables, the \verb|\MULTISPAN|
macro may be tested to determine the number of columns spanned by the
entries. 
 
%%page page_202                                                  <<<---3
 
Consider the following source code: 

\begin{lstlisting}

\Configure{tabular} {\HCode{<TABLE>}} {\HCode{</TABLE>}} 
{\HCode{<TR>}} {\HCode{</TR>}} 
{\HCode{<TD \ifnum \MULTISPAN>1 COLSPAN="\MULTISPAN"\fi>}} 
{\HCode{</TD>}} 
\begin{tabu1ar}{lr} \multicolumn{2}{c}{merge}\\ 
first & second \end{tabular} 

\end{lstlisting}

The output for this is the following: 

\begin{lstlisting}

<TABLE><TR> <TD COLSPAN="2">merge</TD> </TR> 
<TR> <TD>first</TD> <TD>second</TD> </TR></TABLE> 

\end{lstlisting}

The package options \verb|pic-array| and \verb|pic-tabular| (Section
4.1.1 on page 156) request a picture version of all the array and
tabular tables, respectively. 

The \verb|\\| command is treated as a row separator. To avoid
undesirable empty rows at the end of the tables, the\verb| \\| should
not be inserted after the last row. On the other hand, the character
\verb|~| may be used to introduce invisible content for empty cells.
This will allow for the possibility of empty and nonempty cells being
treated differently by browsers. 

\ipar{4_5_7_2_The_eqnarray_environment_and_the_like}{The eqnarray environment and the like}

The variants of the eqnarray environment are configurable by \verb|\Configure|
commands similar to those used for the \verb|array| and \verb|tabular| environments.
Alternatively a picture version may be requested with the \verb|pic-eqnarray|
option. 

\ipar{4_5_7_3_The_tabbing_environment}{The tabbing environment}

The following command specifies contributions to be included before and after the 
rows, and before and after the entries of the tabbing environment. In addition, 
the command allows for a decimal number to specify a magnification factor for the 
widths of the entries. 

\begin{lstlisting}
\Configure{tabbing}[mag]{pre-row}{post-row}{pre-ent1y}{post-entry} 
\end{lstlisting}

The component \[\emph{mag}\] is optional when no change in magnification
is desired. The contributions offered by the parameters \emph{pre-row},
\emph{post-row}, \emph{pre-entry}, and \emph{post-entry} are ignored
when all of these parameters are empty. The command \verb|\TABBING| may
be used to set the widths of all the entries, where entries with no
bound on their width have a zero for their specified widths. The
trailing entries of the rows have this feature. 

Reconfiguring tables without compromising their properties is probably a
task requiring more knowledge of raw \ \TeX\  programming than most
users possess. 
 
%%page page_203                                                  <<<---3
 
However, the amount of \ \TeX\  code to be written is typically quite small in size. 

\begin{lstlisting}

\newcount\c 
\def\Width#1//{\gdef\TABBING{#1}% 
\ifnum\c>O \HCode{ WIDTH="\the\c"}\fi} 
  \ConfigureEnv{tabbing}{}{}{\Configure{HtmlPar}{}{}{}{}}{} 
  \Configure{tabbing} 
  {\HCode{<TABLE><TR>}} {\HCode{</TR></TABLE>}} 
  {\HCode{<TD}\afterassignment\Width\c\TABBING//\HCode{>}} 
  {\HCode{</TD>}} 
\begin{tabbing} 
LaTeX: \=tabbing\\ 
TeX: \>settabs 
\end{tabbing} 

\end{lstlisting}

The above fragment of \ \LaTeX\  source translates to the following HTML code. 

\begin{lstlisting}

<TABLE><TR><TD WIDTH="71">LaTeX:</TD> 
            <TD>tabbing</TD>        </TR></TABLE> 
<TABLE><TR><TD WIDTH="71">TeX:</TD> 
            <TD>settabs</TD>        </TR></TABLE> 

\end{lstlisting}

A picture version may be requested with the package option
\verb|pic-tabbing|. The variant \verb|pic-tabbing'| applies only to
those instances employing the \verb|\'|  directive of the tabbing
environment. That directive is not fully supported by \texht. 

\isubsubsec{4_5_8_Small_details}{Small details}

Most of the features described so far are tied to specific constructs of
\ \LaTeX, and they are of little use elsewhere. The following features,
of a more general-purpose nature, deal with basic issues. 

\ipar{4_5_8_1_File_names}{File names}

HTML files may result from requests made through the package options 1,
2, 3, and 4 and from \verb|\CutAt| and \verb|\HPage| commands. In such
cases unless the users offer names of their own, the filenames are
automatically created by the system.  The \verb|\FileName| command can
be used to find out the name of the current file.  On the other hand,
the command 

\begin{lstlisting}
\NextFile{filename} 
\end{lstlisting}

may be used to suggest a name for the next HTML file. 

\ipar{4_5_8_2_Conditional_code}{Conditional code}

The command 
%%page page_204                                                  <<<---3
 
\begin{lstlisting}
\ifHtml true-part\else false-part\fi 
\end{lstlisting}

enables us to choose content based on whether the html package option is used. 

\ipar{4_5_8_3_Environments_for_scripts}{Environments for scripts}

The \verb|\HCode| command allows the user to write small fragments of raw code into 
the HTML file. The command 

\begin{lstlisting}
\ScriptEnv{name}{prefix}{suffix} 
\end{lstlisting}

provides the means of defining environments for including larger
fragments of raw code. 

\begin{verbatim}
                           \ScriptEnv{css} 
                            {\HCode{<STYLE TYPE="text/css"> 
<STYLE TYPE="text/css">       \Hnewline<!--}\Hnewline} 
<!--                          {\HCode{-->\Hnewline</STYLE>}} 
  UL { border : solid 1px; }      \begin{css} 
  H1 { color: green }                 UL { border : solid 1px; } 
-->                                   H1 { color: green; } 
</STYLE>                          \end{css} 

\end{verbatim}

\ipar{4_5_8_4_Content_for_paragraph_breaks}{Content for paragraph breaks}

The following command allows you to specify what material is to be inserted at 
the start of a paragraph and what is to be saved in \verb|\EndP| at this point. There are 
separate parameters for when the first line of the paragraph is indented or not. 

\begin{lstlisting}
\Configure{HtmlPar}{noindent-P}{indent-P}{noindent-save}{indent-save} 
\end{lstlisting}

The task of \verb|\EndP| is typically to deliver code from the start of a paragraph to its 
end. 

\begin{verbatim}

                              \Configure{HtmlPar} 
														    {\HCode{<P><H2>}} 
														    {\EndP \Hcode{<P>}} 
														    {\HCode{</H2>}} {} 
<P><H2>Head </H2><P> Body     \noindent Head \par Body 

\end{verbatim}

There are extra commands to enable finer local control over the contributions 
at the start of paragraphs. The command \verb|\IgnorePar| ignores the contribution of 
content at the start of the next paragraph, and the command \verb|\ShowPar| provides 
content at the start of the next paragraph. Similarly the \verb|\NoIndent| command says 
that the first line of the next paragraph should not be indented, and the \verb|\Indent| 
command says the first line of the next paragraph \emph{should} be indented. 
 
%%page page_205                                                  <<<---3
 
\ipar{4_5_8_5_Creating_new_hooks_for_texht}{Creating new hooks for \texht}

The core of \texht is programmed to deal with general situations created
by the underlying machinery of \ \TeX, and in general it does a good job
there. However, the underlying features typically have very little to do
with \emph{structural} properties of commands defined in private and
public style files. To capture such properties, the definitions must be
extended to include \emph{hooks}. To maximize the benefit of the hooks,
they should be configurable. The command 

\begin{lstlisting}
\NewConfigure{name}[digit]{assignments} 
\end{lstlisting}

is designed for this purpose. 

This command introduces a hook configurable by a \verb|\Configure| command. 
The \emph{digit} specifies the number of configurable fields the \verb|\Configure| command 
will need. These fields are accessible with the \verb|\NewConfigure| command through 
the parameter names \verb|#1|, \verb|#2|, and so forth. An example is shown in \reffig{4-15}. 

\texht provides hooks with initial values for the commands in the style
files of \ \LaTeX, the plain file of \TeX, style files of AMS-\LaTeX and
AMS-\TeX, and other commonly used packages. 
