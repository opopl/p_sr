% vim: keymap=russian-jcukenwin
%%beginhead 
 
%%file 07_03_2023.fb.fb_group.mariupol.pre_war.2.istoriya_shkoli_na_k.cmt
%%parent 07_03_2023.fb.fb_group.mariupol.pre_war.2.istoriya_shkoli_na_k
 
%%url 
 
%%author_id 
%%date 
 
%%tags 
%%title 
 
%%endhead 

\qqSecCmt

\iusr{Rada Vashenko}

Моя школа 😞💔

\iusr{Світлана Єлісєєва}

моя школа

\iusr{Мария Кот}

🥺

\iusr{Maggy Mi}

Спасибо, очень интересно. На фото школа выглядит очень неухоженно, в последние
годы она функционировала, там были дети? До 2022 года?

\begin{itemize} % {
\iusr{Виктория Терпан}
\textbf{Maggy Mi}

Нет, последние годы она не работала. Закрылась.

\iusr{Олена Сугак}
\textbf{Maggy Mi} нет. Даже не знаю когда закрылась. Слышала что признавали аварийной...

\iusr{Anzhelika Drotianko}
\textbf{Олена Сугак} как раз 2018 помню там была тероборона, и достаточно долго, ее накануне войны перенесли на Левый. Поэтому ещё одна цель была...

\iusr{Светлана Водзянская-Живогляд}
\textbf{Maggy Mi} ні школу закрили 2016-2018 можу помилятися, бо вона була аварійна,
трамвайна колія поруч не сприяла її міцності. А від себе хочу додать що і учнів
там було не багато 😞 якось так склалося що там лишилися не найкращі вчителі і
батьки не хотіли віддавати туди дітей.

\end{itemize} % }

\iusr{Татьяна Юденкова}

Это мо ШКОЛА. А историческую экскурсию сделал нам краеведение Мариуполя Буров к
60 летнему юбилею школы. Была издана его книга о нашей школе. Вдохновителем
проекта была наша выпускница Долгова О.А. школу закрыли под мед.институт.

\begin{itemize} % {
\iusr{Олена Сугак}
\textbf{Татьяна Юденкова} да. Есть видео на ютубе.
\end{itemize} % }

\iusr{Zina Tabachnik}

И я здесь училась......

\iusr{Елена Денисенко}

Це моя школа!!!! І я поступила в 1965 році!!! Моя англійська школа!!! В ті
далекі часи там вже був лінгафонний кабінет, уявляєте? Дітей туда возили з усіх
районів, навіть з лівого берега!!! А я просто жила поруч, за 2 квартали.
Пощастило мені! Англійську люблю з тієї самої пори. Закінчила иняз. Зараз я у
Київі, ВПО, але продовжую викладати англійську. Дякую тобі, моя школо!!!

\iusr{Ольга Дзецко}

І мені пощастило вчитися в цій англійській школі, де був, дійсно, високий
рівень навчання. А англійську всі добре знали, і багатьом по життю вона дуже
знадобилося. Цікаво було прочитати історію школи. Дякую

\iusr{Marina Wilby}

А ще до того, це була єдина українська школа ! Це все, що вони нам залишили в
радянськи часи. Всі інші школи в Маріуполі були з викладанням російською!

\begin{itemize} % {
\iusr{Олена Сугак}
\textbf{Marina Wilby} Долгова (яку пом'янули у коментарях ,) цього не написала! Цікаво, вона коли вчилася, викладали українською?

\begin{itemize} % {
\iusr{Затія Марійченко}
\textbf{Олена Сугак} Пані Долгова була активною рашистською активісткою Маріуполя, тому й не згадала про український статус школи.

\iusr{Олена Сугак}
\textbf{Затія Марійченко} такою є й понині! Живе у Маріуполі радіє окупантам. Думає що расєя там навсєгда...

\iusr{Marina Wilby}
\textbf{Олена Сугак} Дурень думкою багатіє!
\end{itemize} % }

\iusr{Татьяна Юденкова}
\textbf{Marina Wilby} 

Это не так. Она вбыла первой английской школой, да. Там преподавал анг. язык
Зайчик Э. Я. прекрасный человек и учитель. Мы потеряли её координаты. Может, кто
знает!

\begin{itemize} % {
\iusr{Maksim Bondarenko}
\textbf{Татьяна Юденкова} Эмму Яковлевну помню.( Хотя и преподавала не в моём классе.
\end{itemize} % }

\end{itemize} % }

\iusr{Затія Марійченко}

Мені пощастило працювати в 11школі з 1976 до літа 1980р. Гарний, людяний,
високопрофесійний колектив! Так, Е.М.Зайчик, Т.Р.Рудаловська, О.М.Шкриль,
Г.Думбур, В.Чорна, Г.Петрига, М.К.Буданова, Галина Григорівна Сиротко, пані
Сердюкова, З.В. Гонца та багато інших, прізвища яких пам'ять не зберегла.
Пишаюся, тут навчала мови дітей, які стали відомими в Маріуполі викладачами
вузів, лікарями, спеціалістами інших професій. Серед них перший український
чемпіон Літніх олімпійських ігор в Атланті В'ячеслав Олійник. Тут навчалися
діти священника Марковського, а райком партії регулярно вимагав звіти про
антирелігійну роботу в школі. Школу закрили перш за все через появу 66-ої, яка
з'явилася в мікрорайоні поряд. Лише після того, як 66-та стала т.зв. опорною,
11-та була остаточно закрита.

\iusr{Nina Zisser}

11 школа?

\iusr{Таня Филко}

Я закончила 11 школу. Светлые воспоминания!

