% vim: keymap=russian-jcukenwin
%%beginhead 
 
%%file 13_01_2017.stz.news.ua.mrpl_city.1.staryj_novyj_god_v_mariupole
%%parent 13_01_2017
 
%%url https://mrpl.city/blogs/view/staryj-novyj-god-v-mariupole
 
%%author_id burov_sergij.mariupol,news.ua.mrpl_city
%%date 
 
%%tags 
%%title Старый Новый год в Мариуполе: история
 
%%endhead 
 
\subsection{Старый Новый год в Мариуполе: история}
\label{sec:13_01_2017.stz.news.ua.mrpl_city.1.staryj_novyj_god_v_mariupole}
 
\Purl{https://mrpl.city/blogs/view/staryj-novyj-god-v-mariupole}
\ifcmt
 author_begin
   author_id burov_sergij.mariupol,news.ua.mrpl_city
 author_end
\fi

Как-то вскоре после новогодних праздников Фатхи, аспирант из Египта (между
прочим, очень прилично владевший русским языком), спросил смущаясь:

- Скажи, что означает выражение \enquote{старый новый год}? Как может быть год сразу и
старым, и новым?

Пришлось объяснить любознательному арабу то, о чем написано дальше.

Римский консул Гай Юлий Цезарь  распорядился ввести в действие новый календарь
1 января  45 года до новой эры. Календарь был назван в честь  Цезаря
\enquote{юлианским}. Полторы тысячи лет никто не смог предложить что-нибудь лучше
юлианского календаря. Но со временем обнаружилась его неточность.

К концу ХVI века в странах Западной Европы возникла необходимость заменить
юлианский календарь на более совершенный. И эта замена произошла. 4 октября
1582 года от Рождества Христова. Папа Римский Григорий XIII ввел в католических
странах взамен юлианского новый, уточненный, так называемый  григорианский
календарь.

В это время в России летоисчисление шло от Сотворения мира, год начинался с 1
сентября и календарем пользовались юлианским. И лишь 19 декабря 7208 года Петр
I подписал указ, согласно которому новый год начинался с 1 января, а
летоисчисление вместо \enquote{от Сотворения мира} велось \enquote{от Рождества Христова}. В
России начался 1700 год от Рождества Христова...

24 января 1918 года декретом Совета Народных Комиссаров РСФСР в стране был
введен григорианский календарь. В соответствии с положениями этого декрета
была предусмотрена  поправка в 13 суток. После 31 января 1918 года в России
наступил день 14 февраля. Но Русская православная церковь осталась верна
юлианскому календарю и остается такой до сих пор. По этому календарю
определяются даты православных праздников.

С приходом к власти большевиков в 1917 году началась непримиримая борьба  с
религией.  Почти анекдотичный факт – вместе с религиозными православными
праздниками был упразднен и новогодний - 1 января по григорианскому календарю,
абсолютно светский праздник. Но на пространстве бывшего Советского Союза люди
праздновали тайно, а теперь празднуют явно два Новых года: 1 января, как велел
Совнарком, - по григорианскому календарю, и через тринадцать дней - 1 января по
календарю юлианскому, как повелел царь Петр Алексеевич.

Во время и сразу после Отечественной войны давление на православную церковь
ослабло. Стали возрождаться народные обычаи.  На Маланку, в канун 1 января по
юлианскому календарю,  дети ходили по домам щедровать, а на следующее утро
поздравляли хозяев с Новым годом – посевали жилье зерном, сопровождая обряд
пением  колядок.

К середине 50-х годов прошлого века на улицах Мариуполя  практически исчезли
\enquote{погорелки}. Часть  домов, сожженных гитлеровцами, были восстановлены.  А на
месте тех, что сильно пострадали от огня и не было смысла их восстанавливать,
были построены новые здания. Жизнь постепенно налаживалась. К  Новому году в
сквере  начали ставить елку. В Ильичевском районе елки устанавливали на улице
Левченко, против двора пожарной команды. После открытия здания драматического
театра место для главного новогоднего дерева нашли перед его фасадом.

В 60-е годы, а может, и позже,  новогодние елки, находящиеся в общественных
местах,  должны были убираться до 14 января, - 1 января по юлианскому
календарю, то есть до старого Нового года. Кто-то в \enquote{верхах} вспомнил, что в
канун старого Нового года, 13 января по старому стилю, то есть по юлианскому
календарю, после захода солнца начинался Щедрый вечер. По церковному календарю
– это день поминовения преподобной Мелании Римлянки.Так боролись с религиозными
предрассудками. Странно, но такой \enquote{обычай} у нас в городе соблюдается и
сегодня.
