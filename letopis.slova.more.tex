% vim: keymap=russian-jcukenwin
%%beginhead 
 
%%file slova.more
%%parent slova
 
%%url 
 
%%author_id 
%%date 
 
%%tags 
%%title 
 
%%endhead 
\chapter{Море}
\label{sec:slova.more}

%%%cit
%%%cit_head
%%%cit_pic
%%%cit_text
Блакитніє \emph{море}. Дитячим щебетом починається ранок на одному з мальовничих
півостровів, що по-тутешньому зветься просто кут. На куті великий старовинний
парк, один з тих парків, що їх колись насаджували та поливали в таврійських
маєтках заробітчани та місцеві степовики, ті, чиї внуки та правнуки зараз ціле
літо щебечуть у затінках цього парку та смаглявіють біля моря до мулатної
смаглості. "Комуна" зветься цей кут і цей парк, тому що тут ще в двадцяті роки
справді була комуна демобілізованих червонофлотців, і хоч комуни давно нема,
але ймення від неї ще й досі зосталось. Восени та навесні, доки діти у школі, в
Комуні проводяться наради районного масштабу, форуми чабанів або
кукурудзоводів, сюди ж їдуть відзначати й Першотравень, а потім на ціле літо —
рясносонячне, степове,— влада тут переходить до рук піонерії, хлопчаків та
дівчаток, що їх з усіх усюд звозять сюди, і їхнім житлом стають цупкі
профспілкові намети, а єдиним начальством — виховательки та вожаті
%%%cit_comment
%%%cit_title
\citTitle{Залізний острів}, Олесь Гончар
%%%endcit
