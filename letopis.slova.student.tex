% vim: keymap=russian-jcukenwin
%%beginhead 
 
%%file slova.student
%%parent slova
 
%%url 
 
%%author 
%%author_id 
%%author_url 
 
%%tags 
%%title 
 
%%endhead 
\chapter{Студент}
\label{sec:slova.student}

%%%cit
%%%cit_head
%%%cit_pic
%%%cit_text
Скандал разгорелся после того, как харьковская \emph{студентка} Маргарита возмутилась
тем, что ей нужно писать курсовую работу на украинском языке.  Она заявила, что
не понимает, как от этого изменится смысл работы и что часть материалов она
перевела с английского, а теперь ей придется делать еще один перевод.  \enquote{Пишу
курсовую работу на е@чем украинском языке. Вот понимаю, что украинская власть –
она меня ущемляет, ущемляет мои права. Что значит – я не могу писать свою
научную работу, в которой мои заключения, мои выводы, в которой описано, как я
проводила эту работу, почему я не могу писать ее на русском языке. Что, на
украинском это будет более информативно, развернуто? Так нет.  Материалы я
брала вообще на английском языке. Я их переводила. И сейчас я должна переводить
на украинский? Нравится украинский язык? Переводите, пожалуйста. Почему вы
насилуете меня, мою психику и мои нервы}, – заявила девушка на видео, которое
обнародовала в \enquote{Инстаграме}
%%%cit_comment
%%%cit_title
\citTitle{Языковой скандал с харьковской студенткой Маргаритой Мальцевой – все подробности}, 
Екатерина Терехова; Юлия Корзун, strana.ua, 18.06.2021
%%%endcit

%%%cit
%%%cit_head
%%%cit_pic
\ifcmt
  pic https://lgaki.info/wp-content/uploads/2021/10/11.jpg
  @width 0.4
\fi
%%%cit_text
Сегодня стало известно, что \emph{студенты} ЛГАКИ заняли первые места на Международном
конкурсе рисунков по мотивам творчества Ф. М. Достоевского, посвященном
200-летию со дня рождения писателя, который проходил в Гжельском
государственном университете.  \emph{Студентка} 3-го курса кафедры графического
дизайна факультета изобразительного и декоративно-прикладного искусства
Академии Матусовского Елена Васильева победила, участвуя в номинации «Сюжетная
композиция» (ее работа использована в качестве иллюстрации к этому сообщению).
— На конкурс свои работы отправили 12 \emph{студентов} кафедры, — рассказала доцент,
кандидат педагогических наук Татьяна Малая. — Все они – дипломанты, а Елена – в
списке победителей.  В нем же и \emph{студентки} отделения изобразительного искусства
колледжа нашей Академии. В номинации «Живопись» на этом же международном
турнире лучшими признаны работы четверокурсницы Алены Лысенко (преподаватель –
Анжела Лукавецкая-Радченко) и третьекурсницы Дарьи Папирной (ее педагог –
Сергей Неколов).  Поздравляем всех!
%%%cit_comment
%%%cit_title
\citTitle{Студенты Академии Матусовского – в списке победителей Международного конкурса рисунков, 
посвященного 200-летию со дня рождения Достоевского}, , lgaki.info, 19.10.2021
%%%endcit

%%%cit
%%%cit_head
%%%cit_pic
\ifcmt
  tab_begin cols=2

		 % жуковский
     pic https://library.vladimir.ru/wp-content/uploads/2020/05/zcukovsky.jpg

		 % туполев
		 pic https://osamolete.ru/wp-content/uploads/2018/10/853eb88e2775ff7aaac8ae1339ec2a30.jpg

  tab_end
\fi
%%%cit_text
Осенью 1909 года в ИМТУ начиналось чтение нового курса – аэродинамики. Читал
предмет Николай Егорович Жуковский. Спустя несколько месяцев под руководством
Жуковского было сформировано в училище новое отделение. Уже в апреле 1910 года
в стенах ИМТУ прошла первая воздухоплавательная выставка, организованная
\emph{студентами}. Среди экспонатов были масштабная модель построенного во
Франции биплана «Антуанетт» и одна из первых аэродинамических труб,
спроектированная и построенная Туполевым. Конструкция, получившая название
«плоской аэродинамической трубы», предназначалась для предполетных испытаний
новых разработок. Несколько позднее Туполев разработал аэродинамическую трубу
круглого сечения, создание которой положило начало аэродинамической лаборатории
при ИМТУ. Аэродинамические трубы Туполева эксплуатировались в училище вплоть до
1923 года
%%%cit_comment
%%%cit_title
\citTitle{А. Н. Туполев – человек и его самолеты}, Пол Даффи, Андрей Кандалов
%%%endcit

%%%cit
%%%cit_head
%%%cit_pic
\ifcmt
  pic https://img.strana.news/img/article/3653/letnjaja-kievljanka-v-15_main.jpeg
  @width 0.4
\fi
%%%cit_text
17-летняя \emph{студентка} из Киева \enquote{прославилась} в соцсетях своими
размышлениями о войне на Донбассе и комментариями о возможном вторжении РФ в
Украину. Также девушка заявила, что поддерживает самопровозглашенные
\enquote{ЛНР и ДНР}, считает из \enquote{крутыми} и вообще не понимает смысла
войны на Донбассе. Скандальные видео киевлянка размещала в TikTok.
\enquote{Кому-то просто нужны земли и сухопутный выход к морю... Ну правильно,
это очень важно... Но честное слово, если мы не можем противостоять, мы не
можем сохранить свои земли, то почему мы не можем их отдать? Честно, ребята,
\enquote{ДНР, ЛНР}, я за вас руками и ногами. Вы крутые ребята, я знаю... Но
зачем Украине эти земли, где их не ждут, не принимают и не хотят. Зачем?} -
говорит тиктокерша.  Также \emph{студентка} эмоционально заявила, что ей
\enquote{чертовски страшно оттого, что в феврале-месяце может случиться
непоправимое} и все могут умереть, поэтому она хочет \enquote{писать завещание}
%%%cit_comment
%%%cit_title
\citTitle{\enquote{Ребята, вы крутые}. 17-летняя киевлянка в TikTok заявила о поддержке \enquote{ЛДНР}, 
Юлия Супрун, strana.news, 03.12.2021
%%%cit_url
\href{https://strana.news/news/365315-17-letnjaja-kievljanka-v-sotsseti-zajavila-o-podderzhke-ldnr.html}{link}
%%%endcit
