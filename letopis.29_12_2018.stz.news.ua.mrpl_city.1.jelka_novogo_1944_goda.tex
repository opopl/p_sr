% vim: keymap=russian-jcukenwin
%%beginhead 
 
%%file 29_12_2018.stz.news.ua.mrpl_city.1.jelka_novogo_1944_goda
%%parent 29_12_2018
 
%%url https://mrpl.city/blogs/view/elka-novogo-1944-goda
 
%%author_id burov_sergij.mariupol,news.ua.mrpl_city
%%date 
 
%%tags 
%%title Елка нового 1944 года
 
%%endhead 
 
\subsection{Елка нового 1944 года}
\label{sec:29_12_2018.stz.news.ua.mrpl_city.1.jelka_novogo_1944_goda}
 
\Purl{https://mrpl.city/blogs/view/elka-novogo-1944-goda}
\ifcmt
 author_begin
   author_id burov_sergij.mariupol,news.ua.mrpl_city
 author_end
\fi

\ii{29_12_2018.stz.news.ua.mrpl_city.1.jelka_novogo_1944_goda.pic.1}

Сколько было за долгую жизнь новогодних праздников, сколько было непременных их
атрибутов – елок? А много ли из них запомнилось? Может, два-три. Только
намертво врезались в память предпраздничные и праздничные события в дедушкином
доме в канун 1944 года. К тому времени Мариуполь без малого четыре месяца был
свободен от оккупантов. Уже почти развеялся запах пожарищ, но электричества еще
не было. Не было еще очень, очень многого, но стоит ли здесь об этом
вспоминать? Кухня, куда собирались вечерами обитатели дома, освещалась
\href{https://archive.org/details/08_12_2018.sergij_burov.mrpl_city.svetilniki}{карбидной
лампой}. Она превращала окружающее пространство в безжизненно-голубоватую
сказочную картину. В один из таких вечеров тетя - мамина младшая сестра -
предложила устроить елку. Последний раз в доме ее ставили накануне 1941 года.
Затем началась война, потом оккупация. До елки ли было? Предложение, как ни
странно, было поддержано.

\textbf{Читайте также:} 

\href{https://archive.org/details/03_01_2018.sergij_burov.mrpl_city.jelochnyj_shar}{%
Елочный шар (Быль, услышанная от одноклассника), Сергей Буров, mrpl.city, 03.01.2018}

Подготовка началась на следующий день. С чердака спустили желтую еще довоенную
шляпную коробку, свернутую из тонкого листа фанеры. В ней обнаружилось огромное
для детей богатство. Сначала извлекли гирлянду с цветными лампочками. Правда,
этой находке радовались не очень долго - электричества-то не было. Потом
принялись вытаскивать из ваты стеклянные шары, дирижабль с надписью на борту
\enquote{СССР}, бусы из стеклянных трубочек. Под слоем ваты лежали картонные, оклеенные
станиолем и чуть подкрашенные рыбка, петушок, собака, верблюд и лошадка.
Найдены были сделанные из ваты яблоко, груша, снегурочка, красноармеец-часовой
в тулупе до пят и крохотные негритенок и негритяночка, облаченные в яркие
одежды. Да, еще была гирлянда разноцветных флажков. Как будто все есть.

Но тетя сказала, что нужно добавить игрушек на елку и делать мы будем их сами.
Началось с изготовления цепей. Материалом для этого послужили обложки старых
тетрадей и давние почтовые открытки. Из них были нарезаны ленточки, которые
склеивали в кольца-звенья будущей цепи. Благо с довоенных времен остался
пузырек с клеем-гуммиарабиком. Нашлось применение и вате. Ее клочки были
нанизаны на нитки. Так получился \enquote{снег}. Из лоскутков сшили нечто, отдаленно
напоминающее цветы...

\textbf{Читайте также:} 

\href{https://mrpl.city/news/view/v-mariupole-detyam-voennosluzhashhih-podarili-novogodnij-prazdnik-foto}{%
В Мариуполе детям военнослужащих подарили новогодний праздник, Олена Онєгіна, mrpl.city, 28.12.2018}

Итак, вроде все было готово. Оставалось достать елку. Но где и как? Враг еще в
Крыму, не снята блокада Ленинграда. Все мариупольские заводы, фабрики и порт
разрушены. Сожжены почти все школы, больницы, многие жилые дома, кинотеатры и
драматический театр. На проспекте Республики остались лишь считанные здания,
которые пощадил огонь. На улице Семашко в Ильичевском районе, например, от
многоэтажных домов, построенных перед войной, остались только остовы с пустыми
проемами окон. Пострадали и другие городские магистрали. Продукты и все прочее
– по карточкам. Хлеб – из непросеянной муки. Да и его вдоволь не было.

Но у тети, оказывается, уже в то время, когда она предложила устраивать елку, в
голове был план. Роль елки она отвела китайской розе. Это растение, взращенное
усилиями двух поколений женщин семьи, достигло почти полутораметровой высоты и
стояло в углу самой большой комнаты дома – зала. Утром 1 января 1944 года роза,
обвешанная игрушками, ждала маленьких обитателей дома и их гостей. Гостей было
двое – семилетняя дочка маминых приятелей Лёся и ее ровесник соседский
мальчонка Виталик. Они пришли с подарками. Лёся преподнесла листик с
переводными картинками с изображением кораблей, Виталик преподнес видавшую виды
книжку-раскладушку \enquote{Пожар} Маршака. Но самым желанным стал подарок от мамы. В
мешочке (его мама сшила из лоскута кумача) лежало полдюжины цветных карандашей.
Они все были очинены, и все были разной длины. Только много позже стало
понятно, что мама покупала их по одному, по два на толчке, своеобразном
\enquote{промтоварном} филиале главного мариупольского базара. В ту пору приобрести
коробочку с цветными карандашами было мечтой совершенно неосуществимой. Среди
подарков от взрослых были две дореволюционные открытки с изображениями
кремлевских памятников русского литейного искусства – Царь-колокол и
Царь-пушка, леденец \enquote{петушок} и альбомчик для рисования из грубой оберточной
бумаги.

\textbf{Читайте также:} 

\href{https://mrpl.city/news/view/miss-elka-2018-mariupoltsy-mogut-podderzhat-gorodskuyu-novogodnyuyu-krasavitsu}{%
\enquote{Мисс Елка-2018}: мариупольцы могут поддержать городскую новогоднюю красавицу, Анастасія Папуш, mrpl.city, 28.12.2018}

Было, конечно, и угощение. Бабушка испекла из кукурузной муки пирог – малай, в
котором было много-много сушеных вишен, заготовленных еще летом. Еще подались
коржики из пресного теста, посыпанные маком, кисель из вишен. Тетя из тыквы
вырезала разные фигурки, запекла их духовке и сдобрила патокой. Новогодним
напитком был густой узвар, сваренный из даров садов Приазовья. Угощением все
остались довольны. Те, кто просил добавки, ее получили.

Заключительной частью новогоднего торжества были выступления у\hyp{}частников.
Виталик прочел с выразительными жестами стихотворение Маршака \enquote{Мой веселый
звонкий мяч}, юному чтецу долго хлопали. Лёся начала было декламировать
\enquote{Федорино горе} Чуковского, но сбилась, застеснялась и уже готова была
расплакаться, но присутствующие одарили ее аплодисментами и она успокоилась.
Один из малолетних хозяев спел песенку про елочку. А другой, который пока не
мог ни стишок рассказать, ни петь, просто сидел на руках у бабушки и хлопал в
ладоши. Финалом праздника стал веселый хоровод, который вела тетя.

Новогодний праздник 1 января 1944 года в дедушкином доме удался.

\vspace{0.5cm}
\begin{minipage}{0.9\textwidth}
\textbf{Читайте также:} 
\href{https://mrpl.city/news/view/glavnaya-elka-mariupolya-dopolnilas-grustnymi-ukrasheniyami-foto}{%
Главная елка Мариуполя дополнилась грустными украшениями, mrpl.city, 28.12.2018}
\end{minipage}
\vspace{0.5cm}
