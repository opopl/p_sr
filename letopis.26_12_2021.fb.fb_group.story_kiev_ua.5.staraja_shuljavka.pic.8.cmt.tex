% vim: keymap=russian-jcukenwin
%%beginhead 
 
%%file 26_12_2021.fb.fb_group.story_kiev_ua.5.staraja_shuljavka.pic.8.cmt
%%parent 26_12_2021.fb.fb_group.story_kiev_ua.5.staraja_shuljavka
 
%%url 
 
%%author_id 
%%date 
 
%%tags 
%%title 
 
%%endhead 

\iusr{Ірина Бондарєва}

Наверное, недалеко от нашего дома Борщаговская 21, где жила наша семья и я до
своих 13 лет. Мой прадед Иван Архипенко до революции приехал в Киев и поселился
с женой на Борщаговской, родили троих детей. Он держал аптеку и все знали его,
как Ивана - аптекаря. Бабушка моя, вышла замуж за Севрука Василия, родилось у
них четверо детей. Я дочь старшей Раисы.

Тридцатый номер я помню, ходили туда с бабушкой зачем-то... и, кажется, он был
проходным...

\begin{itemize} % {
\iusr{Valeriy Rukhman}
\textbf{Ірина Бондарєва} Видать напротив \enquote{Юбилейного} гастранома

\begin{itemize} % {
\iusr{Ірина Бондарєва}
Ну где-то так... хотя гастроном появился гораздо позже...
Прямо напротив нашего дома был Борщаговский переулок

\iusr{Valeriy Rukhman}
\textbf{Ірина Бондарєва} да, да а я жил на ул. Политехнической 5 с 1968г по 1977г.

\iusr{Ірина Бондарєва}
\textbf{Valeriy Rukhman} 

Здорово! В школу 142 ходили? Я училась в ней с первого по восьмой класс. На
ул. Полевой недалеко от школы жила моя бабушка, мать папы...

\iusr{Андрей Борзяк}
\textbf{Ірина Бондарєва} я в 142 учился с 1976 по 1986

\iusr{Георгий Майоренко}
\textbf{Valeriy Rukhman} 

Да, на другой стороне Брест-Литовского потом появился \enquote{Юбилейный}. А
Борщаговская 30 находилась недалеко от того места, где потом построили
\enquote{Военторг}. Но ближе к Лыбеди, к железной дороге.

\end{itemize} % }

\iusr{Георгий Майоренко}
\textbf{Ірина Бондарєва} Здорово! Соседи!

\begin{itemize} % {
\iusr{Ірина Бондарєва}
\textbf{Георгий Майоренко} 

Да! Я тоже порадовалась! Спасибо за очень интересную публикацию, факты и теплые эмоции! @igg{fbicon.smile} 

\iusr{Георгий Майоренко}
\textbf{Ірина Бондарєва} 

По поводу 30 номера. Почему-то уже в конце 50-х изменили нумерацию домов и 30
стал 24. Это был двухэтажный дом среди одноэтажек. Сейчас на его месте пустырь
рядом с Борщаговской 10. А рядом - скоростной трамвай.

\iusr{Ірина Бондарєва}
\textbf{Георгий Майоренко} 

Скоростной проходит прямо по нашим дворам... У 30 номера, помню, была дурная
слава... какие-то там нехорошие компании собирались, толи там притон был, как
мелькало в разговорах взрослых... но я была слишком мала, чтоб понимать что там
именно было... но это было в конце 50-х, начале 60-х, наверное...

\iusr{Георгий Майоренко}

Вот как раз в это время (конец 50-х, начало 60-х) дом получил номер 24. Мой
отец нарисовал схему дома и описал фамилии всех жильцов квартир. Надо будет эту
схему обработать в компе и опубликовать. Возможно, кто-то узнает своих. После
сноса старых домов многие жители Шулявки получили квартиры на Воскресенке.


\iusr{Ірина Бондарєва}
\textbf{Георгий Майоренко} 

Мы с мамой получили квартиру от ее работы в Дарнице на Тампере и уехали раньше
наших родственников, а они почти все получили квартиры на Никольской Борщаговке


\iusr{Георгий Майоренко}

А Борщаговская 21 это была усадьба?

\end{itemize} % }

\iusr{Георгий Майоренко}

А остались ли фотографии ваших шулявских предков? Если есть, опубликуйте! Это
же очень интересно! Тем более, если дед держал аптеку! А где она находилась?

\begin{itemize} % {
\iusr{Ірина Бондарєва}
\textbf{Георгий Майоренко} 

В нашем же доме - Борщаговская 21, но до 50-х годов
Фотографии разберу и попозже опубликую... у многих качество уже очень плачевное...

\iusr{Георгий Майоренко}
\textbf{Ірина Бондарєва} Будем ждать! Это очень интересно!

\end{itemize} % }

\end{itemize} % }

\iusr{Нина Опольская}
Молодые девчата... Память...

\iusr{Вікторія Кокошинська}
Мой райончик)))@igg{fbicon.heart.red}

\begin{itemize} % {
\iusr{Георгий Майоренко}
\textbf{Вікторія Кокошинська} Вот это совпадение!!! Получается, не только ул. Юрковская, но и Шулявка ваши родные места! Вот это совпадение! Вот на Юрковской ещё долго сохранялись старые дома, как память о старинном Подоле!

\iusr{Вікторія Кокошинська}
\textbf{Георгий Майоренко} не даром говорят, что мир тесен! Так и есть)))  @igg{fbicon.face.happy.two.hands} 

\iusr{Milena Uleshchenko}
\textbf{Вікторія Кокошинська}, и мой @igg{fbicon.face.smiling}
\end{itemize} % }

\iusr{Олена Андурова}

такая узкая улочка! В 76 году на территории КПИ были остатки частного сектора,
потом его быстренько снесли, построили новые корпуса.
