% vim: keymap=russian-jcukenwin
%%beginhead 
 
%%file 29_12_2021.fb.fb_group.story_kiev_ua.3.zimnjaja_bajka
%%parent 29_12_2021
 
%%url https://www.facebook.com/groups/story.kiev.ua/posts/1828853597311433
 
%%author_id fb_group.story_kiev_ua,bartjuk_dmitrij.kiev
%%date 
 
%%tags kiev,zima
%%title Киевский джем.  Порция 1. Зимняя байка
 
%%endhead 
 
\subsection{Киевский джем.  Порция 1. Зимняя байка}
\label{sec:29_12_2021.fb.fb_group.story_kiev_ua.3.zimnjaja_bajka}
 
\Purl{https://www.facebook.com/groups/story.kiev.ua/posts/1828853597311433}
\ifcmt
 author_begin
   author_id fb_group.story_kiev_ua,bartjuk_dmitrij.kiev
 author_end
\fi

Киевский джем.  Порция 1. Зимняя байка.

Шапка оказалась великовата - она то и дело норовила сползти на глаза. И от
этого создавалось впечатление, что Инесса Захаровна смотрит на мир чуточку
свысока. И ещё шикарная обновка была жутко дорогой - часть денег пришлось взять
из кассы взаимопомощи. 

Но Инесса Захаровна ни о чём не жалела. Когда Ванда Ивановна из планового
отдела предложила купить это чудо, Инесса Захаровна не колебалась ни секунды.
Белый песец, а-ля «ведро», точно такая же, как у очаровательной обманутой
Галочки из «Иронии судьбы». Именно то, о чём мечтала Инесса Захаровна. Матёрые
из матёрых модницы «Укрсоветкурорта» в подавляющем своём большинстве
категорически одобрили приобретение. Да что там «Укрсоветкурорт», девочки из
родственного «Укрсовпрофа», что по соседству, на Ленина, и те несколько раз
забегали позавидовать. 

\ii{29_12_2021.fb.fb_group.story_kiev_ua.3.zimnjaja_bajka.pic.1}

Инесса Захаровна торжествовала. Это успех! И надо же, как вовремя - на завтра
назначено знакомство с родителями Светы (Светой звали невесту единственного
сына Инессы Захаровны, Игоря). Договорились встретиться в час дня у ресторана
«Олимпиада-80». Вроде бы неплохой, хоть и в Дарнице. Игорь упомянул вскользь,
что отец тоже должен быть. Замечательно. Пусть увидит, чего он лишился,
предпочтя ей, великолепной и эффектной Инессе Захаровне ту молоденькую брюнетку
из своей задрипанной лаборатории. Ничтожество.

Инесса Захаровна уже предвкушала триумф.  Тем более, что погода способствовала
- вечерний час пик, сверкающие в огнях неона снежинки, предпраздничная суета
ЦУМа, ёлочные игрушки в витринах - короче говоря, «морозной пылью серебрится
его бобровый воротник». Ну хорошо - не его, а её. И не бобровый, а
синтетический. Но шапка! Шапка компенсировала и это, и многое-многое другое.
Как сказали бы сейчас, шапка решала. Ну, или рулила. Казалось, все смотрят
только на неё - в смысле на Инессу Захаровну. В шапке. 

\ii{29_12_2021.fb.fb_group.story_kiev_ua.3.zimnjaja_bajka.pic.2}

Всю дорогу от Пушкинской до метро «Крещатик» Инесса Захаровна ощущала себя
Снежной Королевой - красивой и неприступной.

И ничуть не удивилась, когда в вагоне какой-то симпатичный мужчина почтительно
уступил ей место (что в общем-то, в Киеве не принято). Всё сегодня благоволило
Инессе Захаровне. Сев, она, едва ли не в первый раз за последние несколько лет
почувствовала покой и умиротворённость. 

Мерное покачивание вагона, тепло и приятные мысли убаюкивали.
Убаюкивали-убаюкивали и убаюкали. Голова женщины плавно склонилась на грудь.

... «Станція «Політехнічний інститут», пасажири  прискорюйте висадку та
посадку!» - Что такое?! Отчего эта внезапная лёгкость и прохлада в висках?
Вскинувшись ото сна, Инесса Захаровна лихорадочно ощупала макушку, затылок,
затем, неизвестно зачем, снова макушку - и не обнаружила шапки. И тут же
увидела её на какой-то дебелой даме, которая энергично продиралась к выходу.
Взывать о помощи не хватало ни времени, ни голоса. Инесса Захаровна тигрицей
бросилась вслед и в последний момент, когда уже закрывались двери успела
сорвать своё сокровище с головы злоумышленницы. Злоумышленница обожгла её
страшным взглядом и осталась где-то позади, на перроне. А поезд двинулся
дальше. И, что интересно - в общей сутолоке никто даже ничего не заметил. А
когда проехали «Большевик», толчея более менее рассеялась. И Инесса Захаровна
смогла пробраться к своему месту. И что ещё интереснее - там на полу вагона она
увидела свою шапку. 

Вот такой вот старый киевский анекдотец. Правда, некоторые утверждают, что это
ещё не всё. Мол, назавтра, у Олимпиады-80, в Светиной маме Инесса Захаровна
узнала ту самую злоумышленницу. А та  соответственно её. Вряд ли, конечно.
Хотя, как знать - в Киеве случается и не такое. Особенно, под Новый Год. С
Наступающим.

P.S. Предыдущую порцию «Киевского джема» см. в Киевских историях за
понедельник, 27|12|2021. Называется «93-й год».

\ii{29_12_2021.fb.fb_group.story_kiev_ua.3.zimnjaja_bajka.cmt}
