% vim: keymap=russian-jcukenwin
%%beginhead 
 
%%file 19_07_2020.fb.lnr.14
%%parent 19_07_2020
 
%%endhead 
  
\subsection{Эстония подтвердила гибель своего гражданина в Донбассе.}
\label{sec:19_07_2020.fb.lnr.14}
\url{https://www.facebook.com/groups/LNRGUMO/permalink/2857648114346744/}

Военный "медик" оказался наёмником и бывшим гражданином Белоруссии

Эстония подтвердила, что Микола Ильин, погибший в районе Зайцево, был именно её
гражданином.  Он оказался тем самым военным "медиком", о гибели которого так
пафосно объявляла Украина.

\index[names.rus]{Микола Ильин}

Микола (Николай) является бывшим гражданином Белоруссии, воевавшим в рядах ВСУ
как НАЁМНИК.

Министр иностранных дел Эстонии Урмас Рейнсалу подтвердил гибель гражданина
страны Миколы Ильина во время боевых действий в Донбассе.  Как сообщил министр
порталу ERR, погибший получил гражданство Эстонии в 2016 году в порядке
натурализации, отказавшись от гражданства Белоруссии.

«Подробности происшествия уточняет министерство иностранных дел Украины.

В Эстонии действует принцип, что граждане Эстонии могут находиться в военных
частях других стран по разрешению правительства», --- подчеркнул Рейнсалу.

Однако, акцентировал он, в данное время ни одного такого разрешения выдано не
было.  А это значит, что военный медик Микола Ильин был наёмником, воевавшим
за деньги в составе украинской армии против Донбасса.

«Выражаем искренние соболезнования семьям погибших. Эт о непостижимая трагедия
для них, для нас, для страны. Эта трагедия ещё раз возвращает нас к пониманию
наших фундаментальных ценностей. Украина должна жёстко охранять свои
территории. Украина должна жёстко отвечать на подлые действия боевиков. Украина
должна чётко понимать риски, которые несёт существование серых уголовных
анклавов. Украина должна всегда знать цену «обещаниям» той стороны», —
прочувствованно заявил 14 июля президент Владимир Зеленский, забыв упомянуть,
что речь идёт об эстонском наёмнике, а не гражданине Украины.

Тело Ильина было передано украинской стороне вчера, 17 июля.  В районе семи
часов вечера 17 июля военнослужащими ДНР было обнаружено тело третьего
погибшего диверсанта.

«Передача тела украинской стороне находится в процессе согласования», —
сообщили в представительстве ДНР в СЦКК.

Напомним, 13 июля в направлении позиций ДНР в посёлке Зайцево выдвинулась
диверсионная группа ВСУ из шести человек.  Они вышли на минное поле, где
сработали заряды.  Погибли трое военнослужащих.
