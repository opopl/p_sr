% vim: keymap=russian-jcukenwin
%%beginhead 
 
%%file 07_10_2020.sites.ru.zen_yandex.yz.pozitiv_krasok_daria_orlova.1.russkii_duh_zhivopis_andrei_shishkin
%%parent 07_10_2020
 
%%url https://zen.yandex.ru/media/pozitiv_krasok/zdes-russkii-duh-zdes-rusiu-pahnet-jivopis-andreia-shishkina-5f7d5297b516ad3bd8effd5c
 
%%author 
%%author_id yz.pozitiv_krasok_daria_orlova
%%author_url 
 
%%tags art,russia,istoria
%%title "Здесь русский дух, здесь Русью пахнет..." Живопись Андрея Шишкина
 
%%endhead 
 
\subsection{\enquote{Здесь русский дух, здесь Русью пахнет...} Живопись Андрея Шишкина}
\label{sec:07_10_2020.sites.ru.zen_yandex.yz.pozitiv_krasok_daria_orlova.1.russkii_duh_zhivopis_andrei_shishkin}
\Purl{https://zen.yandex.ru/media/pozitiv_krasok/zdes-russkii-duh-zdes-rusiu-pahnet-jivopis-andreia-shishkina-5f7d5297b516ad3bd8effd5c}
\ifcmt
	author_begin
   author_id yz.pozitiv_krasok_daria_orlova
	author_end
\fi

\ifcmt
  pic https://avatars.mds.yandex.net/get-zen_doc/1583807/pub_5f7d5297b516ad3bd8effd5c_5f7d564e7a731030315c1d99/scale_1200
	caption Художник Андрей Шишкин
	width 0.5
\fi

\index[names.rus]{Шишкин, Андрей!Художник!Здесь русский дух, здесь Русью пахнет..., 07.10.2020}
\index[rus]{Русь!Живопись!Шишкин, Андрей, Здесь русский дух, здесь Русью пахнет..., 07.10.2020}

Приветствую, дорогие читатели! Сегодня мы отправимся в завораживающий мир нашей
истории, в славное прошлое предков. А проводником нам послужат полотна
современного художника Андрея Шишкина.

Сразу скажу, что его картины - это не только славянские образы, мастер
многогранен в своем проявлении. В его многочисленных трудах можно встретить
исторические портреты разных эпох и географий, невероятные по своей глубине
пейзажи и даже шуточные жанровые картины, всего за один раз и не перечесть.

\ifcmt
  pic https://avatars.mds.yandex.net/get-zen_doc/1583807/pub_5f7d5297b516ad3bd8effd5c_5f7d57a97a731030315dcd7b/scale_1200
	caption Художник Андрей Шишкин "Былина"
	width 0.4
\fi

Я же остановлю ваше внимание на славянских образах, ибо только в этом
направлении у художника великое множество полотен, в которые хочется до
бесконечности долго всматриваться...

\ifcmt
  pic https://avatars.mds.yandex.net/get-zen_doc/1591100/pub_5f7d5297b516ad3bd8effd5c_5f7d580c7a731030315e4e15/scale_1200
	caption Художник Андрей Шишкин "Девушка в русском наряде" 
	width 0.4
\fi

Тема славянства, древних былин, сказок, быта - занимает особое место в
творчестве этого художника. Чистые образы, невыдуманные истории читаются
зрителем даже не глазами...душой...

Представляете, как много нужно знать, а может быть и помнить где-то в глубинах
сознания, чтобы создавать такие образы, которые откликнутся какой-то особой
радостью в душе зрителей.

\ifcmt
  pic https://avatars.mds.yandex.net/get-zen_doc/1639101/pub_5f7d5297b516ad3bd8effd5c_5f7d59527a731030315fff89/scale_1200
	caption "Семья" художник Андрей Шишкин
	width 0.4
\fi

Мы будто смотрим фильм, попадаем в далекое прошлое к пращурам и время
останавливается... Даже через фотографии картин, не говоря уже о подлинниках,
передается невероятно сильная энергетика, которой наделил эти образы автор. Кто
бы что не говорил, верил бы или нет, но это действительно чувствуется от
полотен Андрея Шишкина.

\ifcmt
tab_begin cols=2
	caption Художник Андрей Шишкин, 07.10.2020, За ягодами, Последний Защитник

  pic https://avatars.mds.yandex.net/get-zen_doc/48747/pub_5f7d5297b516ad3bd8effd5c_5f7d5a20b516ad3bd8f9417a/scale_1200
	caption "За ягодами" художник Андрей Шишкин

	pic https://avatars.mds.yandex.net/get-zen_doc/3822405/pub_5f7d5297b516ad3bd8effd5c_5f7d5b51b516ad3bd8fadd05/scale_1200
	caption "Последний защитник", художник Андрей Шишкин
tab_end
\fi

Удивительно, как в этой живописи переданы настроения, как в них пульсирует
жизнь... Гусли старца в картине "Былина" звучат, стоит ему только коснуться
струн, а от хаты из картины "Семья" веет теплом и уютом простого быта. И как
сжимается сердце при взгляде на сюжет "Последний защитник", здесь мы осознаем
всю трагичность и глубину этой сцены.


\ifcmt
  pic https://avatars.mds.yandex.net/get-zen_doc/1714257/pub_5f7d5297b516ad3bd8effd5c_5f7d5b20f8625710929fecb9/scale_1200
	caption "Велес" художник Андрей Шишкин
	width 0.4
\fi

Особо можно выделить и серию картин Андрея Шишкина, на которых он воссоздал
образы древних славянских богов. Как хотелось бы, чтобы в учебниках по
мифологии мы в школе учили не только имена древнегреческих и римских богов, но
и знали свою историю, рассматривали бы вот такие яркие образы пантеона
славянских богов.


\ifcmt
tab_begin cols=3
	caption Художник Андрей Шишкин, 07.10.2020, Стрибог, Даждьбог, Заря-Зарница

  pic https://avatars.mds.yandex.net/get-zen_doc/3724792/pub_5f7d5297b516ad3bd8effd5c_5f7d5cc27a7310303164aefa/scale_1200
	caption "Стрибог - хранитель ветров" художник Андрей Шишкин

	pic https://avatars.mds.yandex.net/get-zen_doc/3828082/pub_5f7d5297b516ad3bd8effd5c_5f7d5ce2398ab5384bc622a4/scale_1200
	caption "Даждьбог" Художник Андрей Шишкин

	pic https://avatars.mds.yandex.net/get-zen_doc/1711517/pub_5f7d5297b516ad3bd8effd5c_5f7d5d1bb516ad3bd8fd5177/scale_1200
	caption "Заря-Зарница" Художник Андрей Шишкин
tab_end
\fi

\ifcmt
  pic https://avatars.mds.yandex.net/get-zen_doc/3927246/pub_5f7d5297b516ad3bd8effd5c_5f7d5d3d7a73103031655cc1/scale_1200
	caption "Среча" Художник Андрей Шишкин
	width 0.4
\fi

Радует, что существуют мастера, которые достойно пишут историю современной
живописи. Лично я с огромным удовольствием периодически возвращаюсь к полотнам
Андрея Шишкина, потому, что с каждым разом открываются все новые и новые грани
в его творчестве.

Благодарю вас, уважаемые читатели, что уделили время. Буду рада вашим отзывам
на эту статью.

Приглашаю всех, кто любит творчество на свой рисовальный канал "Позитив
красок", где я публикую простые мастер классы для деток и взрослых. До новых
встреч!


\ii{07_10_2020.sites.ru.zen_yandex.yz.pozitiv_krasok_daria_orlova.1.russkii_duh_zhivopis_andrei_shishkin.comments}
