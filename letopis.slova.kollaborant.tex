% vim: keymap=russian-jcukenwin
%%beginhead 
 
%%file slova.kollaborant
%%parent slova
 
%%url 
 
%%author_id 
%%date 
 
%%tags 
%%title 
 
%%endhead 
\chapter{Коллаборант}
\label{sec:slova.kollaborant}

%%%cit
%%%cit_head
%%%cit_pic
\ifcmt
  pic https://avatars.mds.yandex.net/get-zen_doc/4350071/pub_618312bce8eeee6665ae858c_618312f737011142658acd1c/scale_1200
  @width 0.4
\fi
%%%cit_text
Здесь все категорично и однозначно. Хотя.… Смутила меня, признаюсь, семейная
история, история дедов братьев. Вернее даже не сама история, а то, как они её
рассказывали на презентации романа «Забудь-річка» в Николаеве. В основе романа
именно эта самая история их дедов.  Один дед служил пулеметчиком в Красной
армии, прошел всю войну, освобождал Севастополь. Другой же дед «отказался
воевать за Сталина», предпочел тюрьму, отсидел восемь лет. И дальше, внимание,
прямая речь: «В советские времена выходило так, что одним дедом мы должны были
гордиться, а вторым – нет. Сегодня все наоборот: мы должны гордиться тем дедом,
который отсидел 8 лет. А тот, который был красноармейцем, сегодня – оккупант и
\emph{коллаборант}».  Странно это как-то, не находите? А кем они будут гордиться
завтра?
%%%cit_comment
%%%cit_title
\citTitle{Украинские писатели-братья}, Украинский русский, zen.yandex.ru, 04.11.2021
%%%endcit
