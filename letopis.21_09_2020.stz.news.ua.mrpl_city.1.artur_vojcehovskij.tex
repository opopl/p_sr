% vim: keymap=russian-jcukenwin
%%beginhead 
 
%%file 21_09_2020.stz.news.ua.mrpl_city.1.artur_vojcehovskij
%%parent 21_09_2020
 
%%url https://mrpl.city/blogs/view/artur-vojtsehovskij-krashhe-odrazu-zajmatis-tim-shho-bilshe-do-dushi
 
%%author_id demidko_olga.mariupol,news.ua.mrpl_city
%%date 
 
%%tags 
%%title Артур Войцеховський: "Краще одразу займатись тим, що більше до душі"
 
%%endhead 
 
\subsection{Артур Войцеховський: \enquote{Краще одразу займатись тим, що більше до душі}}
\label{sec:21_09_2020.stz.news.ua.mrpl_city.1.artur_vojcehovskij}
 
\Purl{https://mrpl.city/blogs/view/artur-vojtsehovskij-krashhe-odrazu-zajmatis-tim-shho-bilshe-do-dushi}
\ifcmt
 author_begin
   author_id demidko_olga.mariupol,news.ua.mrpl_city
 author_end
\fi

\ii{21_09_2020.stz.news.ua.mrpl_city.1.artur_vojcehovskij.pic.1}

Багатьох маріупольських глядачів зачаровує своєю грою яскравий і  талановитий
артист Донецького академічного обласного драматичного театру (м. Маріуполь) –
\emph{\textbf{Артур Войцеховський}}. На сцені театру він грає довгих 14 років. За його плечима
чимало головних ролей. Здається, що Артуру під силу зіграти будь-який образ,
адже в кожній ролі він органічний і правдивий...

Артур народився і виріс в Маріуполі. Актором мріяв стати з дитинства.
Здебільшого це сталося завдяки любові до кіномистец\hyp{}тва. Вже закінчуючи школу,
хлопець вирішив, що потрібно спробувати себе в акторській професії. Навіть
хотів спочатку освоїти спеціальність психолога і з цього боку розкрити для себе
акторську майстерність. Проте за проханням батьків  він вступив до
Приазовського державного технічного університету. Втім участь у студентському
театрі \enquote{СТЕМ} допомогла зрозуміти, що йому слід змінити спеціальність. Наразі
він магістр театрального мистецтва Луганської академії культури і мистецтв.
Проте навчатися акторській майстерності почав у Театральній студії Костянтина
Добрунова, якого вважає своїм вчителем. До речі, саме Костянтин Володимирович
запросив Артура грати у  Маріупольському драматичному театрі. Актору приємно,
що батьки підтримали його вибір і не заважали будувати акторську кар'єру. Хоча
вони не приховували своїх хвилювань, адже син не закінчив навчання в ПДТУ,
вважали, що спочатку треба освіту отримати, а потім займатися тим, що більше до
душі. Але Артур вирішив не гаяти часу і одразу йти за покликом серця, при цьому
і освіту отримав, і зарекомендував себе як талановитий артист.

\ii{21_09_2020.stz.news.ua.mrpl_city.1.artur_vojcehovskij.pic.2}

За 14 років роботи на сцені для актора найбільш знаковою є роль імператора
Нерона у виставі \emph{\enquote{Театр часів Нерона і Сенеки}} (постановка Костянтина
Добрунова). Для гри у цій виставі потрібно було знайти в собі багато нових рис,
з якими раніше Артур не стикався. Деякі риси він намагався в собі розвинути,
адже імператор Нерон був особистістю доволі неоднозначною. Артур вважає, що ця
вистава створена для того,  щоб звернути увагу на природу людської особистості
та застерегти суспільство від помилок, яких часто припускається людство.
Головна помилка імператора Нерона в тому, що він шукав винних не в собі, не в
своєму спотвореному житті, він хотів звинуватити когось іншого. Зокрема, у
виставі імператор звинувачує свого вчителя Сенеку. Найбільше актора вразила
словесна дуель між головними героями, яка стала для нього важливим досвідом для
подальшого зростання у професії. Артур наголошує, що не можна щось робити
заради задоволення власних корисливих цілей, при цьому використовувати лозунги
філософії. Все повинно бути просякнуте серцем добрим, якщо серця немає, то все
буде спотворено. Після цієї ролі Артур знайшов багато нового для себе.

Для Артура всі ролі важливі, але більш цікаво грати негативних персонажів. На
думку нашого героя, у негативних героїв більше можливостей та ситуативних
ресурсів, різних обставин для внутрішнього конфлікту, для тих ситуацій, де з
якихось причин їм доводиться робити вибір, хоча, можливо, не завжди правильний
і справедливий. Їхні душевні пориви і терзання штовхають їх на злі вчинки.
Позитивні герої найчастіше позбавлені таких обставин, вони з самого початку
роблять все правильно. Негативні персонажі більш вільні на сцені, тому їх грати
завжди цікаво. У майбутньому актор хотів би зіграти Гамлета. Він сприймає цю
роль як одну з найбільш потужних і привабливих для себе, адже в ній поєднані і
негативний, і позитивний герой. У Гамлета маса внутрішніх конфліктів,
неймовірних обставин, але у нього добре серце і він намагається знайти вихід.
Це справжня \emph{поліфонія Баха для актора}.

\ii{21_09_2020.stz.news.ua.mrpl_city.1.artur_vojcehovskij.pic.3}

Планів на майбутнє у Артура багато. Зокрема, він хотів би спробувати себе в
ролі режисера та зняти кіно за власним сценарієм. У Артура є власні задумки і
бачення вистави, які він хотів би реалізувати.

У Маріуполі найбільше любить море. Проте загалом полюбляє ходити різними
вулицями міста, яке є для нашого героя найріднішим.  

\begingroup
\em

\textbf{Хобі:} театр, любительські відео, поезія. Є власний канал на YouTube, де
представлені роботи Артура Войцеховського.

\textbf{Улюблена книга:} \enquote{Третє відкриття сили} Андрія Сідерського. На думку Артура, ця
книга це справжній інструментарій для роботи зі свідомістю.

\textbf{Улюблений фільм:} \enquote{Запах жінки} (1992 рік), \enquote{До зірок} (2019 рік), \enquote{На Місяці}
(2019 рік).

\textbf{Побажання для маріупольців:} 

\begin{quote}
\enquote{Я бажаю нашим маріупольцям ніколи не дивитися ні на кого, не дивитися на Захід
чи Схід, дивитися в очі своїх рідних і бачити в них свій будинок... Я бажаю всім,
щоб ви виходили в місто і відчували в своєму місті сім'ю, якусь ступінь сім'ї
без фанатизму, звичайно. Бажаю всім натхнення та радості! Нехай кожен день
стане чимось новим для вас!}.
\end{quote}

\endgroup
