% vim: keymap=russian-jcukenwin
%%beginhead 
 
%%file slova.pamjat
%%parent slova
 
%%url 
 
%%author 
%%author_id 
%%author_url 
 
%%tags 
%%title 
 
%%endhead 
\chapter{Память}
\label{sec:slova.pamjat}

%%%cit
%%%cit_head
%%%cit_pic
%%%cit_text
В субботу, 19 июня, в Киеве открыли \emph{памятник}, установленный в честь
медицинских сотрудников, которые скончались во время борьбы с эпидемией
коронавируса.  Монумент установили возле городской клинической больницы №4 в
Соломенском районе столицы.  Об этом сообщается на странице больнице в
Facebook.  \enquote{Врачи - главные герои последних двух лет во всем мире. Весь
2020 год ежедневно врачи боролись с маленьким вирусом, который повлиял на ход
жизни многих людей. И в этой борьбе, к сожалению, у нас были не только победы,
но и поражения - мы теряли жизни... Жизни пациентов и жизни врачей, которые
жертвовали собой в борьбе за жизнь других. Поэтому врачи - реальные герои этого
времен}, - говорится в сообщении.  В церемонии открытия участвовали министр
здравоохранения Украины Виктор Ляшко, депутаты Верховной Рады Украины, депутаты
Киевского городского совета и председатель Соломенской РГА.
\enquote{Медицинским работникам, которые пожертвовали своей жизнью в борьбе с
пандемией Covid-19 в Украине}, - написано на \emph{памятнике}
%%%cit_comment
%%%cit_title
\citTitle{В Киеве появился памятник медикам, которые отдали свои жизни в борьбе с Covid-19}, 
Эллина Либцис, kiev.strana.ua, 19.06.2021
%%%endcit


%%%cit
%%%cit_head
%%%cit_pic
%%%cit_text
Цікава штука, ця глибина \emph{пам'яти} і тяглість. Може, звісно, я всі ці роки дуже
погано дивився, але наша історична \emph{пам'ять} як певний суцільний сюжет
закінчується приблизно на українській революції. Перша світова. Далі в минуле
тяглости ніби немає, є лише якісь фрагменти спогадів: козаки,
політично-культурна буча зламу століть, а між тим - тиша. Дев'ятнадцяте
століття з нашої історії ніби випадає, а все що перед ним - узагалі час
старозавітніх пророків і патріархів, тільки десь посереди первісного бульйону
дев'ятнадцятого століття бродять Шевченко, Франко і Українка
%%%cit_comment
%%%cit_title
\citTitle{У нас досі радянська пам'ять. Світ нібито почався в 1917-му, до того був морок}, 
Остап Українець, gazeta.ua, 25.06.2021
%%%endcit

%%%cit
%%%cit_head
%%%cit_pic
%%%cit_text
Правда, в сети ходит непроверенная информация о том, что в 1963 году, будучи в
Кривом Роге милицейским начальником, дед Зеленского якобы был замешан в
массовом расстреле толпы во время беспорядков, вызванных ментовским беспредом.
После чего якобы пошел на повышение, то есть был такой себе «небольшой
Новочеркасск в Кривом Рогу». Но, еще раз повторим, что эта информация не
проверенная.  Так неужели же Зеленский, внук «деда, который воевал», не в
состоянии «закрыть рот» подчиненному ему напрямую «зооветеринарному СНБО» и
разным там «центрам по дезинформации» при СНБО, чтобы те не топтались по \emph{памяти}
погибших и победивших в той войне?!
%%%cit_comment
%%%cit_title
\citTitle{Краткий словарь грантоедов под редакцией СНБО / Лента соцсетей / Страна}, 
Александр Карпец, strana.news, 26.10.2021
%%%endcit
