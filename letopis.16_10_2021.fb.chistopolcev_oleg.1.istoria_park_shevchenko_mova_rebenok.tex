% vim: keymap=russian-jcukenwin
%%beginhead 
 
%%file 16_10_2021.fb.chistopolcev_oleg.1.istoria_park_shevchenko_mova_rebenok
%%parent 16_10_2021
 
%%url https://www.facebook.com/chistopoltsev/posts/2007628852739979
 
%%author_id chistopolcev_oleg
%%date 
 
%%tags den.zaschitnika.ukrainy,deti,dnepropetrovsk,jazyk,mova,ukraina
%%title Реальна історія. м.Дніпро. 14.10.2021р. Парк Шевченко
 
%%endhead 
 
\subsection{Реальна історія. м.Дніпро. 14.10.2021р. Парк Шевченко}
\label{sec:16_10_2021.fb.chistopolcev_oleg.1.istoria_park_shevchenko_mova_rebenok}
 
\Purl{https://www.facebook.com/chistopoltsev/posts/2007628852739979}
\ifcmt
 author_begin
   author_id chistopolcev_oleg
 author_end
\fi

Реальна історія. м.Дніпро. 14.10.2021р. Парк Шевченко.

\ifcmt
  ig https://scontent-mxp1-1.xx.fbcdn.net/v/t1.6435-9/246311557_2007628826073315_4310889726499047752_n.jpg?_nc_cat=102&ccb=1-5&_nc_sid=8bfeb9&_nc_ohc=mJ3HKH4Oo6EAX8qQmbK&_nc_ht=scontent-mxp1-1.xx&oh=b057f8dc050957b0dc6cfa144a2da345&oe=6191F837
  @width 0.4
  %@wrap \parpic[r]
  @wrap \InsertBoxR{0}
\fi

- А почему у военньіх машин такой большой руль? - запитав хлопець віком рочків
так 8-10, залізши в військову броньовану автівку "Козак", яку виставили в парку
Шевченка в День захисників та захисниць України. Так як її водій відповідав на
запитання інших людей, хлопцю взявся відповідати я.

 - Це через те, що як правило військові машини важкі, то таке кермо роблять для
 того, щоби ними легше було керувати.

- А вьі можете говорить на руском? Я по украински плохо понимаю

- Ти не з України? Звідки приїхав? З росії?

- Не, я из Украиньі. Я тут живу.

Сказавши це він виліз з цієї автівки та побіг з приятелем до іншої. І от стоїмо
ми з Sergiy Sergio та Олександр Сергійович Сємакін, які були свідками цього
діалогу і бачимо та усвідомлюємо проблему - українська дитина в Дніпрі просить
говорити російскою, через те, що погано розуміє українську. Ппц.

Ні, ми звісно розуміємо, що така кількість російської в Україні це результат
довготривалої колонізаційної/окупаційної політики. Але це також результат
інформаційної, культурної, освітньої політики української влади незалежної
України - як центральної, так і місцевої. Для більшості представників якої
українська мова є чужою, є формальністю. Це ми чуємо та бачимо з того, як вони
спілкуються офрекордз (наприклад, коли оприлюднюють записи розмов засідань
вищого керівного складу країни - президентів, уряду та т.п.), як вони пишуть та
спілкуються в побуті, в соцмережах. Тому маємо такий результат. Тому маємо
обирати до влади тих, хто не на словах, а ділом підтримує розвиток всього
українського в Україні, маємо тиснути на теперішню владу, щодо такого розвитку.

Ну і самі маємо бути мовно стійкими.

Чим більше української в Україні, тим краще.

\ii{16_10_2021.fb.chistopolcev_oleg.1.istoria_park_shevchenko_mova_rebenok.cmt}
