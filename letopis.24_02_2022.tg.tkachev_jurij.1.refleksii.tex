% vim: keymap=russian-jcukenwin
%%beginhead 
 
%%file 24_02_2022.tg.tkachev_jurij.1.refleksii
%%parent 24_02_2022
 
%%url https://t.me/dadzibao/5557
 
%%author_id tkachev_jurij
%%date 
 
%%tags __feb_2022.vtorzhenie,ukraina
%%title На рефлексии, если честно, нет времени
 
%%endhead 
 
\subsection{На рефлексии, если честно, нет времени}
\label{sec:24_02_2022.tg.tkachev_jurij.1.refleksii}
 
\Purl{https://t.me/dadzibao/5557}
\ifcmt
 author_begin
   author_id tkachev_jurij
 author_end
\fi

На рефлексии, если честно, нет времени, да и эмоциональных сил: если честно,
переполняет тупая чёрная ненависть к тем, кто до этого довёл (и я сейчас не
только о Зеленском/Порошенко/Байдене, но давайте это обсудим после). Пока хочу
сказать следующее.

Дичайше бесят люди в комментариях, вопящие \enquote{ура-ура, вперёд, вали
хохлов} и так далее. Такие эмоции ещё более ли менее простительны для солдат на
фронте, у которых адреналин и вот это вот всё. С людьми, восторженно верещащими
с дивана в такой ситуации, на мой взгляд, довольно сильно что-то не так. И по
сути на мой взгляд они мало отличаются от тех, кто аналогичным образом вопил
\enquote{вали сепаров} в 2014-м. Радует, что таких сравнительно немного и - по
крайней мере у меня в комментариях - преобладают сочувствие и сострадание.

Не меньше, а то и больше бесят те, кто вопит \enquote{это фейк}, \enquote{это
вброс} и \enquote{вывсёврете} на каждую публикацию о гибели/ранениях
гражданских или о потерях наступающих войск. Я таких, снова-таки, насмотрелся в
2014-2015 с их \enquote{кондиционерами} и \enquote{самисебя}. Мерзость в обоих
случаях.

Война это горе, смерть, разрушения, сломанные судьбы. Хороших и полезных войн
не бывает вообще. Любая война - это проёб политиков, не сумевших решить вопросы
иными способами. Но об этом мы, повторюсь, с вами порассуждаем позже, а пока
хочется надеяться, что весь этот длящийся уже 8 лет кошмар поскорее закончится.
