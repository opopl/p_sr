% vim: keymap=russian-jcukenwin
%%beginhead 
 
%%file 29_09_2021.stz.edu.lnr.lgpu.1.konkurs_novosti_donbassa
%%parent 29_09_2021
 
%%url http://lgpu.org/news/8449-predstaviteli-pedagogicheskogo-universiteta-stali-pobeditelyami-konkursa-horoshie-novosti-donbassa.html
 
%%author_id 
%%date 
 
%%tags 
%%title Представители педагогического университета стали победителями конкурса «Хорошие новости Донбасса»
 
%%endhead 
\subsection{Представители педагогического университета стали победителями конкурса «Хорошие новости Донбасса»}
\label{sec:29_09_2021.stz.edu.lnr.lgpu.1.konkurs_novosti_donbassa}

\Purl{http://lgpu.org/news/8449-predstaviteli-pedagogicheskogo-universiteta-stali-pobeditelyami-konkursa-horoshie-novosti-donbassa.html}

Итоги конкурса «Хорошие новости Донбасса», организованного общественным
движением «Мир Луганщине», подвели 29 сентября на базе гостиничного корпуса
«Дружба». 

\ii{29_09_2021.stz.edu.lnr.lgpu.1.konkurs_novosti_donbassa.pic.1}

Конкурс  «Хорошие новости Донбасса» проводится для поддержки молодых,
начинающих журналистов и блогеров. Наравне с молодыми в нем участвуют и опытные
мастера.

Одно из требований для конкурсных работ –  нести позитивные новости о регионе в
информационное пространство. Сам же конкурс призван поощрять творческий
потенциал журналистов.

Конкурсантов оценивало компетентное жюри, которое возглавил председатель
Народного Совета Луганской Народной Республики Денис Мирошниченко. В состав
судейской коллегии также вошла и.о. заведующего кафедрой журналистики и
издательского дела филологического факультета ЛГПУ Алена Дроздова.

Открывая церемонию награждения, Денис Мирошниченко сказал:

\begin{zznagolos}
– Все участники конкурса понимают, какая ответственность сейчас на них
возложена: именно они пишут историю нашей молодой Республики. Ведь годы спустя
именно по вашим статьям, фотографиям и видеоматериалам, будут оценивать нашу
эпоху. Я уверен: тот смысл, который вы заложили в работы, покажет всему миру,
что Донбасс стоит ассоциировать не только с военными действиями, что жизнь на
территории нашего государства продолжается, и здесь происходит множество ярких,
позитивных событий! 	
\end{zznagolos}

Многие представители Луганского государственного педагогического университета
приняли активное участие в журналистском состязании. Победителем номинации
«Наши достижения» стал сотрудник отдела по связям с общественностью Луганского
государственного педагогического университета Станислав Шило, третье место
заняла педагог кафедры журналистики и издательского дела Марина Оселедько. А
победителем в номинации «Герои современности» стала студент кафедры
журналистики и издательского дела Дария Беспалова. Грамотами за участие в
конкурсе, творческий потенциал, креатив и личный вклад в продвижение
конструктивной и позитивной повестки в информационное пространство были
награждены ведущие специалисты отдела по связям с общественностью ЛГПУ Михаил
Ермишкин и Алина Шило.

\ii{29_09_2021.stz.edu.lnr.lgpu.1.konkurs_novosti_donbassa.pic.2}

Победителям и участникам конкурса вручили ценные призы и памятные подарки.

\begin{zznagolos}
– В своей видеоработе я осветил деятельность молодежных трудовых отрядов
нашего университета, – рассказал Станислав Шило. – МТО – это одна из визитных
карточек нашего университета. Это  и  первый шаг  на пути к обретению профессии
для многих студентов. Такие  конкурсы  очень важны для развития журналистики в
Республике. Они стимулируют  трудиться и развиваться, находить в новостях
больше позитива. Ведь хорошие новости – это то, чего всем нам порой очень не
хватает. Моя победа была бы невозможной без поддержки администрации вуза, а
также моих коллег – сотрудников отдела по связям с общественностью и учебной
телестудии ЛГПУ.
\end{zznagolos}

Организаторы конкурса «Хорошие новости Донбасса» отметили, что планируют
сделать его проведение традиционным и приглашают всех желающих принять участие!

Пресс-центр университета, фото Михаила Ермишкина
