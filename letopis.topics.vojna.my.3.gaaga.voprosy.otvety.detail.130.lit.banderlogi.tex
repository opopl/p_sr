% vim: keymap=russian-jcukenwin
%%beginhead 
 
%%file topics.vojna.my.3.gaaga.voprosy.otvety.detail.130.lit.banderlogi
%%parent topics.vojna.my.3.gaaga.voprosy.otvety.detail
 
%%url 
 
%%author_id 
%%date 
 
%%tags 
%%title 
 
%%endhead 

\paragraph{22:03:27 27-08-22 Loraks 🎀replied to Иван}

На фоне происходящих на Украине процессов, на ум приходит отрывок из произведения, написанного в 1894 году:
"Послушай, человеческий детёныш, - сказал медведь Балу, - я учил тебя Закону Джунглей, но у бандерлогов нет Закона. Бандерлоги - отверженные. У них нет собственного наречия, они пользуются украденными словами. У них не наши обычаи. У них нет памяти. Они уверяют, что они великий народ, но падает орех, и они всё забывают об этом. Бандерлогов много, они злы, грязны, не имеют стыда, и если у них есть какое-либо желание, то именно стремление, чтобы в джунглях их заметили. Они всё собираются избрать себе вожака, составить собственные законы, придумать обычаи, но никогда не выполняют задуманного. Мы не пьём там, где пьют бандерлоги, не двигаемся по их дорогам, не охотимся там, где они, не умираем там, где умирают бандерлоги.
Мы велики! Мы свободны! Мы достойны восхищения! Достойны восхищения, как ни один народ в джунглях! Мы все так говорим - значит, это правда! -- кричали они.
...
"Правда все то, что Балу говорил о Бандерлогах, - подумал Маугли про себя. - У них нет ни Закона, ни Охотничьего Клича, ни вожаков -- ничего, кроме глупых слов и цепких воровских лап".
