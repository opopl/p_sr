% vim: keymap=russian-jcukenwin
%%beginhead 
 
%%file 30_03_2021.fb.galickij_sergej.1.mova_marshrutka_dnepropetrovsk
%%parent 30_03_2021
 
%%url https://www.facebook.com/schwarzer.friedrich/posts/286917759668714
 
%%author Галицкий, Сергей
%%author_id galickij_sergej
%%author_url 
 
%%tags dnepropetrovsk,jazyk,marshrutka,mova,ukraina
%%title А вы всё время на украинском разговариваете?
 
%%endhead 
 
\subsection{А вы всё время на украинском разговариваете?}
\label{sec:30_03_2021.fb.galickij_sergej.1.mova_marshrutka_dnepropetrovsk}
 
\Purl{https://www.facebook.com/schwarzer.friedrich/posts/286917759668714}
\ifcmt
 author_begin
   author_id galickij_sergej
 author_end
\fi

- А вы всё время на украинском разговариваете?

- Так, звісно. Я ж українець! І розмовляю українською.

- А откуда вы?

- З міста Дніпро. Народився і виріс тут.

- А чого ж тогда на украинском разговариваете?

- Так Дніпро ж українське місто! Тому і розмовляю я українською. А ж вам казав,
що я українець. Ви забули?

- Та нет. Ну, просто как-то странно. 

- А що дивного, пані? На мій погляд, все логічно. Крізь навколо мене зараз
Україна. Я - українець. Дніпро - українське місто. Відповідно до цього всього,
якою мовою я повинен розмовляти? Звичайно ж, українською. Чи не так?

- Ну, в нашем же городе все говорят по-русски! И вы русский знаете. Зачем же
тогда надо это \verb|#ВЫКАБЛУЧИВАНИЕ|? Говорите нормально!

- Так я, пані, і розмовляю нормальною українською мовою. Тобто, мовою нашої з
вами країни. Я громадянин України, пані. А ви?

- Я укрАинка и по паспорту я гражданка Украины! Я родилась в Днепропетровске и
живу тут с самого рождения.

- О, добре! Добре, пані, що ви це згадали. З ваших слів, ми з вами громадяни
однієї країни. Так?

- Да!

- Йдемо далі. І я по національності українець, і ви по національності українка.
Так?

-Да!

- Я розмовляю українською мовою. Зверніть увагу на логічний ланцюжок: Україна,
українець, українська мова. Тепер ваша черга. Я уважно слідкую разом з вами.

- Ой, не начинайте! Город говорит на русском і я говорю на русском. А вам
просто \verb|#ДЕЛАТЬ_НЕЧЕГО_В_ЭТОЙ_ЖИЗНИ|!

- Добре, пані. Мені на наступній зупинці виходити. Піду далі "нічого не робити
в цьому житті". Вам всього найкращого. Слава Україні!

- Ой, та идите уже! Задрали уже своим этим "слава украине"! Мы тут все в гавне
по самые уши, а они: "Слава Украине"! Никак не наорутся!

Я вийшов з 73-ї маршрутки і попрямував далі "ВЫКАБЛУЧИВАТЬСЯ". На серці було
спокійно і, навіть, весело. А все тому, що мені багато ще "ДЕЛАТЬ НЕЧЕГО В
ЭТОЙ ЖИЗНИ". 

\begin{verbatim}
	#Україна🇺🇦
	#Дніпро💦
	#Сергій_Галицький😃
	#укрАинка_из_Днепропетровска😡
	#маршрутка_73🚌
\end{verbatim}

\ifcmt
  pic https://scontent-lga3-1.xx.fbcdn.net/v/t1.6435-9/166970391_286917729668717_4062975753269337160_n.jpg?_nc_cat=101&ccb=1-3&_nc_sid=8bfeb9&_nc_ohc=NpgLCIW5hBIAX9SYNfm&_nc_ht=scontent-lga3-1.xx&oh=e7def4eb88f1f1ae8e023f5dab3a822f&oe=61240FCE
  width 0.4
\fi

\ii{30_03_2021.fb.galickij_sergej.1.mova_marshrutka_dnepropetrovsk.cmt}


