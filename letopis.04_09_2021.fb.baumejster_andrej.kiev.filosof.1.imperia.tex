% vim: keymap=russian-jcukenwin
%%beginhead 
 
%%file 04_09_2021.fb.baumejster_andrej.kiev.filosof.1.imperia
%%parent 04_09_2021
 
%%url https://www.facebook.com/andriibaumeister/posts/4216432981811562
 
%%author_id baumejster_andrej.kiev.filosof
%%date 
 
%%tags gosudarstvo,imperia,istoria
%%title Так что же такое империя?
 
%%endhead 
 
\subsection{Так что же такое империя?}
\label{sec:04_09_2021.fb.baumejster_andrej.kiev.filosof.1.imperia}
 
\Purl{https://www.facebook.com/andriibaumeister/posts/4216432981811562}
\ifcmt
 author_begin
   author_id baumejster_andrej.kiev.filosof
 author_end
\fi

Субботний просмотр. Очень скоро))) Сегодня при слове "империя" у многих сразу
включаются сильные эмоции: гнев, страх, агрессия. А когда включаются сильные
эмоции, уже невозможно включать разум и рациональные аргументы. 

\parbox{\linewidth}{
{\color{blue}
\href{https://www.youtube.com/watch?v=HkBKLGiwkks}{%
Так что же такое империя? Критика еще одной "черной легенды", Andrii Baumeister, youtube, 04.09.2021%
}}
}

\ifcmt
  pic https://external-frt3-2.xx.fbcdn.net/safe_image.php?d=AQFsxedpSsFjbr48&w=500&h=261&url=https%3A%2F%2Fi.ytimg.com%2Fvi%2FHkBKLGiwkks%2Fmaxresdefault.jpg&cfs=1&ext=jpg&_nc_oe=6ea37&_nc_sid=06c271&ccb=3-5&_nc_hash=AQE0VaDXL6wWdvWl
  @width 0.8
\fi

Так что же такое империя? Большинство империй прошлого (Римская, Священная
Римская империя немецкой нации, Британская империя) были "минимальными
государствами". Во многом речь шла о символическом единстве в сочетании с
политическим, языковым и культурным многообразием. Насильственная унификация
чаще всего была инструментом национальных государств. Пример: Балканы до Первой
мировой и национальные государства между мировыми войнами. 

Если мы обратимся к литературе, то увидим, что определить что такое "империя"
очень не просто. Большинство встречающихся определений (в основном, в
популярной литературе) неполны, ошибочны или просто ложны. 

Мне кажется, лучше всего определяет "империю" Thomas A. Allsen: Empire: "a
political unit of large extent controlling a number of territories and peoples
under single sovereign authority" (Geschichte..., s.18). Можно перевести так:
Империя - "политическое объединение с высокой степенью контроля над
значительным количеством территорий и народов под единой суверенной властью".
По определению знаменитого британского юриста XIX века Джона Остина, суверен
может быть как одним лицом, так и группой лиц (в сущности, он повторил Гоббса).
Но тогда и ЕС это империя. И, конечно, США. 

Кстати, многие упрекали меня, что я называю США империей. Ну. во-первых, так ее
в своем тексте назвал Нил Фергюсон. А, во-вторых, это так и есть. Хотите знать
почему? Тогда посмотрите это видео. 

Литература: 

\begin{itemize}
  \item 1. Маргарет Тэтчер. Искусство управлять государством. - М.: Альпина Паблишер, 2003.
  \item 2. Ніл Фергюсон. Імперія. Як Британія вплинула на сучасний світ. - К.: Наш формат, 2020. 
  \item 3. Мартин Гилберт. Черчилль. Биография. - М.: Колибри, 2018.
  \item 4. Роберт Римини. Краткая история США. - М.: Колибри, 2020.
  \item 5. Christopher Clark. Die Schlafwandler. Wie Europa in den Ersten Weltkrieg zog. - Berlin: Pantheon, 2015. 
  \item 6. Iriye, Akira / Osterhammel, Jürgen (Hrsg.). Geschichte der Welt. Weltreiche und Weltmeere 1350-1750. - München: Beck, 2014.
\end{itemize}
