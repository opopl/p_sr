% vim: keymap=russian-jcukenwin
%%beginhead 
 
%%file 09_12_2021.stz.kiev.bigkyiv.1.bomboubezhische
%%parent 09_12_2021
 
%%url https://bigkyiv.com.ua/20-hvylyn-shhob-shovatysya-yaki-shovyshha-ta-ukryttya-gotovi-pryjmaty-kyyan-u-razi-povitryanoyi-ataky/
 
%%author_id moskalenko_lesja
%%date 
 
%%tags 
%%title 20 хвилин, щоб сховатися. Які сховища та укриття готові приймати киян у разі повітряної атаки
 
%%endhead 
\subsection{20 хвилин, щоб сховатися. Які сховища та укриття готові приймати киян у разі повітряної атаки}
\label{sec:09_12_2021.stz.kiev.bigkyiv.1.bomboubezhische}
\Purl{https://bigkyiv.com.ua/20-hvylyn-shhob-shovatysya-yaki-shovyshha-ta-ukryttya-gotovi-pryjmaty-kyyan-u-razi-povitryanoyi-ataky/}

\ifcmt
 author_begin
   author_id moskalenko_lesja
 author_end
\fi

\textSelect{Якщо прозвучить сирена – сигнал повітряної тривоги, кожен мешканець столиці, де
б він не був, повинен за 20 хвилин знайти укриття. Зупиняється транспорт і
випускає пасажирів назовні. Але що робити далі і куди бігти?}

\ii{09_12_2021.stz.kiev.bigkyiv.1.bomboubezhische.pic.1}

КМДА влаштувала для преси  спеціальний тур бомбосховищами та укриттями, аби
кияни отримали вичерпну інформацію про те, як діяти у разі так званого
«особливого періоду», а простіше кажучи – у разі, якщо Росія розпочне
масштабний наступ на Україну і буде обстрілювати Київ. 

І ось ми минаємо знак «Сховище №103988» у дворі будинку на вулиці Богдана
Хмельницького і спускаємося у підземелля.  Це навіть не підземелля, а майже
бункер – сховище другого класу. Споруда герметична, тут є запас питної води, а
також генератор, вентиляційні установки з фільтрами, зв’язок, санвузли та
достатня кількість приміщень. Тут можна укритися і працювати кілька днів і
більше, за умови, що вистачить води та їжі.

Останнє таке спеціальне сховище було збудоване у Києві 1986 році. Більше вони у
місті  не будувалися. Таких спеціальних захисних споруд з автономними системами
життєзабезпечення у Фонді захисних споруд цивільного захисту Києва 514.
Найменші з них можуть вмістити 40-50 осіб, середні – 220-240, стандартні – 320,
а є такі, що можуть вмістити до чотирьох тисяч осіб. Яка кількість людей може
вміститися у цих сховищах нам не сказали, мовляв, ця цифра не розголошується.

\ii{09_12_2021.stz.kiev.bigkyiv.1.bomboubezhische.pic.2}

Але не поспішайте шукати таке сховище, вони не призначені для укриття
населення. Як пояснив Роман Ткачук, директор Департаменту муніципальної безпеки
Києва, такі спеціальні захисні споруди розташовані на підприємствах критичної
інфраструктури і слугують для захисту максимально працюючої зміни на самому
підприємстві. Іншими словами, там укриються і продовжуватимуть працювати
структури, які забезпечують життєдіяльність міста, працівники комунальних
служб, районних адміністрацій, тощо. Там же будуть штаби, які керуватимуть
різними службами.

А куди ж сховатися нам із вами, звичайним киянам? Для нас передбачені так звані
«об’єкти подвійного використання», яких у місті близько 5000. Це підвальні і
напівпідвальні приміщення, заглиблені станції метрополітену, підземні паркінги
та переходи. Як повідомив Роман Ткачук, в таких укриттях у Києві одночасно
можуть перебувати 2,8 мільйонів осіб. Вимоги до «об’єктів подвійного
використання» зовсім інші, аніж до сховищ. Тут має бути система вентиляції,
укриття не має бути затопленим чи засміченим, мають бути лави, освітлення,
іноді – туалет. Наприклад, на сучасному підземному паркінгу під новобудовою
може бути і туалет і сучасна вентиляційна система. А на глибоких станціях
метрополітену, таких, як Хрещатик, Театральна, Арсенальна, Печерська,
Лук’янівська, тощо, можна перебувати досить довго, але все одно, орієнтовним
часом перебування людей в укриттях є 2-3 години.  Далі приймається рішення про
подальше укриття населення, або його евакуацію чи відселення, залежно від
обставин.

\ii{09_12_2021.stz.kiev.bigkyiv.1.bomboubezhische.pic.3}

Ми побували також у сховищі другого класу під будівлею Солом’янської РДА на
Повітрофлотському проспекті 41. Це сучасний паркінг із додатковими підземними
приміщеннями, де розташовані робочі місця на 60 осіб. Сховище має два виходи,
три туалети і може вмістити 600 осіб. У сховищі є радіозв’язок, вода та
зручності для осіб з обмеженими можливостями.  У разі «особливої ситуації» всі
автівки з паркінгу мають бути прибрані за 6-7 хвилин.

\ii{09_12_2021.stz.kiev.bigkyiv.1.bomboubezhische.pic.4}

Ще один тип укриття – підвали житлового фонду 60-70-х років. Практично кожна
«хрущовка» має такий підвал. Ці підвали від початку будувалися для укриття
громадян. Вони мали статус сховищ 5-го класу, але 1995 році всі такі сховища
були переведені в звичайні підвальні приміщення. А обладнання, яке там було –
демонтували. Лишилися лише системи подачі повітря. Нам продемонстрували такий
підвал на вул. Волинській 28, опікується ним керуюча компанія Солом’янського
району.  Підвал вміщує до 250 осіб. Отже, мешканцям старих районів знайти
укриття буде значно простіше, аніж жителям новобудов, особливо таких, де немає
підземних паркінгів та станцій метро поблизу.

\ii{09_12_2021.stz.kiev.bigkyiv.1.bomboubezhische.pic.5}

За стан укриттів несуть відповідальність балансоутримувачі, це стосується і тих
об’єктів, які перебувають у приватній власності. Усі укриття регулярно
проходять перевірку. Перевірка об’єктів цивільного захисту проводиться планово
і постійно службою ДСНС, представниками Департаменту житлово-комунальної
інфраструктури та відділами цивільного захисту РДА районів.  

Однак, як пояснив Олег Стоволос, заступник начальника Управління цивільного
захисту ГУ  ДСНС в Києві, згідно із новим законодавством, суб’єкт
господарювання може самостійно перевірити об’єкт і надати ДСНС акт, а самих
представників не допустити до перевірки. Із 5000 об’єктів, які можуть слугувати
укриттям для населення, 19\% – це неготові приміщення. За словами Олега
Стоволоса, вони або занедбані та в аварійному стані, або ж там не працює
система подачі повітря, тощо.

І якщо із державними та комунальними приміщеннями місту порозумітися простіше,
то із тими, які перебувають у приватній формі власності буває складно. Навіть,
якщо ці суб’єкти господарювання не виконують норми по цивільному захисту, не
існує законного способу притягнути їх до відповідальності. Наприклад, при
зведенні житлових новобудов забудовник має передбачити укриття для населення,
та нерідко ця норма просто ігнорується.

\ii{09_12_2021.stz.kiev.bigkyiv.1.bomboubezhische.pic.6}

Також, законодавством України передбачено, що всі об’єкти, які можуть слугувати
укриттям і знаходяться у приватній власності, у разі тривоги повинні бути
доступними для населення. Якщо прозвучала сирена, всі магазини, кав’ярні,
майстерні у підвальних та напівпідвальних приміщеннях, повинні відкрити двері і
надати доступ людям. Отже, якщо до станції метро чи підземного паркінгу далеко,
можна сховатися у таких приміщеннях, а їх у Києві дуже багато.

Після сигналу цивільної тривоги транспорт зупиниться, щоб знайти бодай
найпростіше укриття, у людей буде 20 хвилин. Через 20 хвилин після сигналу
зачиниться метро. Споруди подвійного призначення (паркінги, підвали) будуть
відкриті доти, доки не вмістять максимально можливу кількість осіб. 

Мешканці приватного сектора укриваються в підвальних приміщеннях своїх
будинків.

Дізнатися і перевірити, які укриття є біля вашого дому або роботи, краще
заздалегідь. Побачити адреси найближчих укриттів можна на спеціальній мапі на
сайті КМДА. Мапа \href{https://www.google.com/maps/d/viewer?mid=1GoP1MtW6Zpmn_0-VXnleUBPa-NXBqGS_&ll=50.43993412053319%2C30.45604198386075&z=11}{тут}. Також можна телефонувати на гарячу лінію за номером 1551.

\ii{09_12_2021.stz.kiev.bigkyiv.1.bomboubezhische.pic.map}

Всі сховища та укриття повинні бути промарковані. На стіні будинку має бути
табличка, або стрілочка і напис фарбою «Укриття». А на дверях об’єктів
подвійного використання має бути напис, що це об’єкт цивільного захисту, а
також – прізвище та контакти відповідальної особи. Збережіть собі той контакт,
бо це саме та особа, у якої знаходяться ключі від укриття.

Якщо у Києві розпочнуться бойові дії, буде прийнято рішення про евакуацію
населення. Якщо ж раптом зникне світло, зв’язок та інтернет, тоді всі
рятувальні машини, поліція, автомобілі МЗС, навіть машини медичної допомоги
будуть курсувати по місту і надавати киянам інформацію.
