% vim: keymap=russian-jcukenwin
%%beginhead 
 
%%file 27_09_2021.fb.blazhnova_rekuha_julia.1.mysli_posle_boja_usik.cmt
%%parent 27_09_2021.fb.blazhnova_rekuha_julia.1.mysli_posle_boja_usik
 
%%url 
 
%%author_id 
%%date 
 
%%tags 
%%title 
 
%%endhead 
\subsubsection{Коментарі}
\label{sec:27_09_2021.fb.blazhnova_rekuha_julia.1.mysli_posle_boja_usik.cmt}

\begin{itemize} % {
\iusr{Шарапиденова Дана}
Живи Украина! @igg{fbicon.trophy} 

\iusr{Андрей Коваленко}
 @igg{fbicon.hands.applause.yellow} 

\iusr{Людмила Конвисарова}
Ежедневный труд простых пацанов, которые догребли до вершины Олимпа - лучше не скажешь!

\iusr{Анна Ермолаева}
 @igg{fbicon.100.percent} 

\iusr{Анна Жарова}
Как хорошо сказано

\iusr{Наталія Василівна Криворученко}

\ifcmt
  ig https://scontent-yyz1-1.xx.fbcdn.net/v/t39.1997-6/s168x128/17633073_1652591054767295_6333333619058147328_n.png?_nc_cat=1&ccb=1-5&_nc_sid=ac3552&_nc_ohc=9l4KPN4DSSMAX8sHGKl&_nc_ht=scontent-yyz1-1.xx&oh=cc48d6277aea6c7675afdf5b09c0a6d1&oe=61586D3C
  @width 0.1
\fi

\iusr{Olga Belousova}
Замечательный пост. Одновременно практически безоценочный, но расставляющий акценты. Спасибо.

\iusr{Жандос Уап}
Пост @igg{fbicon.thumb.up.yellow}{repeat=3} . На сколько могуч русский язык!

\iusr{Olga Brown}

Не зовсім так. Англійці так само реагують на політичні, расові і релігійні
погляди людей у всіх сферах суспільства. Просто в них не прийнято обговорювати
це в компанії щоби не зіпсувати людям настрій. Це нажаль не нова тема коли
талановиті люди є носіями сумнівних поглядів. Найвідоміший приклад Вагнера,
музика якого була заборонена в Ізраїлі. Усика я вітаю з перемогою, поглядів
його я не поділяю .

\begin{itemize} % {
\iusr{Oleg Zabielin}
Вагнер был нацистом? )) Или вы другие его взгляды считаете сомнительными?

\iusr{Olga Brown}
\textbf{Oleg Zabielin} почитайте о нем. Там все есть... Период Гитлера, его статья...

\iusr{Oleg Zabielin}
Я прошу прощения, он умер ещё до рождения Гитлера. О чем вы?

\iusr{Olga Brown}
\textbf{Oleg Zabielin} він написав свою відому статтю про те що євреям не дано творити мистецтво. Гітлер вважав його найважливішим німецьким композитором, він був заборонений в Ізраїлі і здається першим хто наважився виконувати його музику в Ізраїлі був Баренбойм.
\end{itemize} % }

\iusr{Елена Маленкова}

Юля, спасибо за этот пост.

\iusr{Жандос Уап}
Для Кличко большой привет из станции Жангиз-Тобе.Казахстан

\iusr{Иван Бавыкин}
Золотые слова!

\iusr{Ульяна Лысак}
до мурашок...

\iusr{Евгения Кучма}

Британцам просто уже много лет спокойнее, чем нам, они свое с соседями
отвоевали. А когда-то ведь и фамилию королевской семьи пришлось новую
изобретать, и Баттенбергов на Маунтбаттенов менять. Было бы наверное и нам
легче жить вот так по-современному, не заморачиваться этим, но это не только
выбор Украины, к сожалению. Немного как про жертву насилия говорить - а чего ты
грустная такая, и по подворотням боишься ходить.

\begin{itemize} % {
\iusr{Ольга Смаль}
\textbf{Евгения Кучма} вот писала то же самое практически. Вообще разные контексты.

\iusr{Евгения Кучма}
\textbf{Ольга Смаль} да, нас к сожалению, нельзя сравнивать с пост-империями, нужно с пост-колониальными странами. В той же Индии, думаю, часто "не те" взгляды вызывали проблемы, и сейчас тоже. На Западе этого часто не понимают - "как вы можете ощущать то же, вы же белые".

\iusr{Ольга Смаль}
\textbf{Евгения Кучма} да я бы вообще скорее сравнивала нас не с пост-колониальными, а скорее с постсоветскими странами просто. И да, мы одна из немногих из этих стран (не считая прибалтийских), где в принципе живы политические процессы, регулярно меняются президенты и есть активная часть неаморфного гражданского актива, которая готова даже на жертвы ради свободы, свободных выборов и тд.
\end{itemize} % }

% -------------------------------------
\ii{fbauth.smalj_olga.ukraina}
% -------------------------------------

Вообще разные исторические отрезки становления у государств, определения своей
идентичности + настоящая, кровавая, действующая война с Россией откладывают
отпечаток. Люди в Великобритании могут позволить себе роскошь быть изи и ни о
чем эдаком не думать. В Украине — нет. Тут постоянно гибнут военные, у каждого
есть знакомый раненный, погибший, а агрессор РФ ведет с 2014 года жесткую
гибридную войну (в том числе информационную), и любое слово любого публичного
украинца может быть использовано не в пользу Украины. Нас буквально пытаются
стереть, растворить в «русском мире», регулярно рассказывают что мы —
недогосударство, у нас недорусский язык и прочую ахинею, давят политически,
экономически и по закону противодействия мы втройне упираемся. Поэтому градус
накала страстей, агрессии по сопряженным темам в разы выше: все элементарно
пропитано горечью о потерях. И автоматом ответственность на публичных людях
лежит тройная — такие времена. Усик — великолепный спортсмен, спасибо ему за
победу.

\begin{itemize} % {

% -------------------------------------
\ii{fbauth.ivanov_vladimir.kovrov.vladimir_obl.rossia}
% -------------------------------------

\textbf{Ольга Смаль} из 14 тысяч погибших в конфликте на востоке Украины только 4 тысячи украинских военных, остальные жители непризнанных республик

\iusr{Ольга Смаль}
\textbf{Владимир Иванов} 

не надо мне рассказывать кто где погиб и что там происходит: у меня в Донецке
дядя и двоюродный брат с женой. Я могу вам сама рассказать как российское
дерьмо вроде Гиркина, Гиви и прочей швали и отбросов на российские деньги
повалило в Украину и тыкало рос флаги на наши администрации. Как мои друзья
видели колонны военной техники из Рф. Как другие друзья находились при штурме
картогрграфического центра в Крыму, где уроды из Росии убили нашего военного. И
как Рф до этого много лет на деньги рос налогоплательщиков кормила Януковича и
говеную Партию регионов, состоящую на 80\% из донбасских бандюков и накачивала
Восток и юг своими деньгами, полит консультантами, своими смыслами и
раскачивала по разным осям примерно как и Севастополь. Украина отстаивает свою
территориальную целостность: ее поставили в положение, когда она вынуждена это
делать. Напомню еще вам сколько погибло мирных жителей в Чечне, когда россиюшка
отчего-то не дала ей самоопределиться (отчего же?): только в Первой чеченской
войне по данным правозащитных организаций погибло 30-40 тыс мирного населения.

\iusr{Владимир Иванов}
\textbf{Ольга Смаль} перечислите сотни и тысячи российских военных, которых вы обменяли, одних украинцев на других меняете все восемь лет

\iusr{Ольга Смаль}
\textbf{Владимир Иванов} так что по убитым мирным чеченским жителям и свободной республике Ичкерия?

\iusr{Владимир Иванов}
\textbf{Ольга Смаль} 

это то мирное, которое взрывало Буйнакск, или может те миряне которые детишек с
родителями захватывали в Беслане, или те мирные, обвязанные поясами шахидов,
взявшие в заложники посетителей мюзикла "Норд-ост" на дубровке в Москве, а
может те бородачи мирные, что вторглись в Дагестан в 1999? Спонсировались
саудитами и США, а воевал весь сброд, те арабы-наемники мирные или дудаевцы? В
глаза мне смотри, это Донбасс приехал в вам в Киев, Фраковск, Львов на танках,
может это те мародеры из добробатов и ВСУ, которые отправляли награбленные веши
"Новой поштой"? В глаза мне смотри!

\iusr{Kata Turner}
\textbf{Olga} підтримую вас, сил вам  @igg{fbicon.hands.pray}  а на іншого коментатора не звертайте увагу. То громадянин расєї, якому звичайно краще видно через рашатв події в нашій країні.

\iusr{Ольга Смаль}
\textbf{Kata Turner} та я його забанила, вже не бачу:).

\end{itemize} % }

\iusr{Катерина Шиянова}
Отличный пост, спасибо  @igg{fbicon.hands.applause.yellow} 

\iusr{Людмила Кондакова}
Юля, спасибо.

\iusr{Мирослава Павловская}

У UK свои проблемы. Нельзя объективно сравнить Британию и Украину. Если
сравнивать, то наша история ближе к истории Ирландии. Ну, и про маленькие
города: Шевченко вырос в Киеве на Оболони, Кличко выросли в Киеве.

А победа Усика на ринге в Лондоне - это победа украинца, это - победа Украины.

\begin{itemize} % {
\iusr{Yuliya Blazhnova-Rekukha}
\textbf{Miroslava Pavlovskaya}

\ifcmt
  ig https://scontent-yyz1-1.xx.fbcdn.net/v/t1.6435-9/243286973_2933849746853800_6113287000771658697_n.jpg?_nc_cat=101&_nc_rgb565=1&ccb=1-5&_nc_sid=dbeb18&_nc_ohc=Jut9SZnWM3IAX8JUU5g&_nc_ht=scontent-yyz1-1.xx&oh=5adf4789c58255fad07e633260606a63&oe=6179FF2F
  @width 0.3
\fi

\iusr{Мирослава Павловская}
\textbf{Yuliya Blazhnova-Rekukha} , слухайте, народився, це - не жив. Не пишіть казок і не вводьте в оману людей

\iusr{Ольга Смаль}
\textbf{Yuliya Blazhnova-Rekukha} можно при скрине чуть больше захватить:)

\ifcmt
  ig https://scontent-yyz1-1.xx.fbcdn.net/v/t1.6435-9/243260606_10159044244481939_2843035842572056997_n.jpg?_nc_cat=101&_nc_rgb565=1&ccb=1-5&_nc_sid=dbeb18&_nc_ohc=nsOPK-zedh8AX9mvsz7&_nc_ht=scontent-yyz1-1.xx&oh=1db5e921c0dfddeaae698760aca07f6e&oe=61769F33
  @width 0.3
\fi

\iusr{Yuliya Blazhnova-Rekukha}
\textbf{Olga Glushchenko Smal} це все міняє і Андрій Шевченко нездара?)

\iusr{Мирослава Павловская}
\textbf{Yuliya Blazhnova-Rekukha} , це Ви щось собі нафантазували. Я поважаю Андрія Шевченка і горджусь його перемогами та досягненнями для моєї Батьківщини.

\iusr{Yuliya Blazhnova-Rekukha}
\textbf{Miroslava Pavlovskaya} попередній коментар не до вас. Ви мені не цікава

\iusr{Ольга Смаль}
\textbf{Yuliya Blazhnova-Rekukha} 

я думала, ви журналіст, а не маніпулятор.  @igg{fbicon.smile}  вам мирослава
вказала на фактологічну неточність (яка малює портрет, має давити на емоції,
жаліти маленького Андрійка, який «попри усе» з маленького містечка (столиця
України Київ, приблизно 3 мільйони мешканців) йшов до своєї футбольної мрії). А
ви у відоповідь: Так що? Шевченко нездара?:) Маніпуляція чистої води. Звісно,
Андрій —найталановитіша людина, яка багато працювала і йшла до успіху ціною
здоров’я.

У тексті посту взагалі у цілому порівнюються реалії соціуму сталої розвинутої
держави з багаторічною історією, колишньої очільниці імперії і з іншого боку
молодої країни, що розвивається — з 30-річною історією державності, яка саме
зараз проходить процеси самоідентифікації, відокремлення від культурного,
політичного і інших полей Росії; держави, що знаходиться під тиском країни, яка
сама була імперією і зараз не хоче «відпускати», а навпаки займається процесом
«собирания земель русски», як вони самі вважають. (Почитайте останні статті
Путіна, там чітко усе сказано). 

Для них Україна — це частина Росії. Вони вже проковтнули Біларусь, а ми ще
спираємося і як заноза у сідниці. Цей опір не може не нести наслідків у
суспілстві. Великобританія і Україна — взагалі зараз різні світи, різні
суспільні явища, різні культурологічні, політичні проблеми і процеси, виклики.
Ви порівнююте непорівнювальне. 

І повторю: в нас натурально йде війна з конкретною державою. А ви така ніби не
розумієте: «чего это украинцы такие агрессивные? Цепляются вот это вот,
какие-то вопросы задают, недовольны чего-то? С чего б это? Можно ведь просто
чилить и радоваться жизни!  Более позитивными быть надо!»просто ніби не
помічаючи усього цього болю і бекграунду і ігноруючи контекст, у якому усе це
відбувається. 

Можливо із Лондона це якось інакше усе відчувається, але коли допомогаєш довго
волонтерам, коли хтось зі знайомих служив у ВСУ, а у самої двоюрідний брат і
дядя у Донецьку і ти наслишана про «підвали», дивишся на усе зовсім інакше. це
тільки поверхово у країні поза ОРДЛО мирно, зримі і незримі нитки війни усюди.
У військовому госпіталі у Києви, де робили операцію моєму татові пару років
тому ти це бачиш наживо: молоді/немолоді ранені хлопці «звідти» нагадують. 

І українці мають право на свою злість, агресивну реакцію до усього, що
пов‘язано з державою-агресором, РФ. А публічні особи мають бути більш
обережними у висловах і діях під час війни, тому що усе це гостро, усе це
болить. Коли усе закінчиться має пройти ще багато років щоб відболіло, а градус
емоцій знизився.  Тоді у когось виникне моральне право запитувати аналогічні
питання і щось там порівнювати.

\end{itemize} % }

\iusr{Iryna Dmytrova}

Юля, пост Ваш прекрасный, справедливый, душевный. Вы не только талантливая
журналист, но и очень чуткая, чувственная, эмоциональная женщина, как я
воспринимаю Вас по Вашим репортажам.

Мы часто в разговорах о жизни у них и у нас жалуемся на Украину, например:«а у
нас пенсионеры не могут даже себя прокормить на пенсию, не то, что в круиз
поехать..» и т.п.

Но это же МЫ! Это У НАС! Что же мы живем и все жалуемся? И никак не выберемся
из постсоветской нищеты духа? А, может, мы на пути к нашей мечте?

\iusr{Ирина Свирипа}
Юля, вы профессионал! За фото отдельное спасибо!

\iusr{Sasha Kot}
Должно пройти время. И Украина будет жить спокойно. Когда никто не будет посягать на независимость и границы.

\iusr{Марина Кищенко}
Бесподобно написано! Да, эти ребята сделали себя сами и именно с их именами ассоциируют Украину.

\begin{itemize} % {
\iusr{Татьяна Федорченко}
\textbf{Марина Кищенко} они все результат работы Тренера @igg{fbicon.heart.exclamation}, от мала до велика @igg{fbicon.hands.raising} 

\iusr{Марина Кищенко}
\textbf{Татьяна Федорченко} Заслуги тренера никто не умоляет, просто из миллионов мальчишек именно они стали символами современной Украины.
\end{itemize} % }

\iusr{Vasil Kalynchuk}
Усик молодець. Претензії на адресу України не зрозумілі.

\iusr{Александр Гладкий}
Троє найвідоміших у Європі українців

\iusr{Liliya Tutaj}
Юля, абсолютно с вами согласна. Я прослезилась читая ваш пост. Фото - просто гениальное.

\iusr{Valentina Valentina}
 @igg{fbicon.hands.applause.yellow}{repeat=5} 

\iusr{Sergey Sheltushev}
І Лондона відніше

\iusr{Игорь Иванюкович}
Такой бред любит писать откровенная вата. Которым какая разница.

\begin{itemize} % {
\iusr{Yuliya Blazhnova-Rekukha}
\textbf{Igor Ivanyukovich} не тот случай

\iusr{Олександр Пшенік}
\textbf{Yuliya Blazhnova-Rekukha} саме той

\iusr{Ирина Бордюг}
\textbf{Игорь Иванюкович} как вы надоели кастрюлеголовые,,лжепатриоты.сам то для прославления страны ,что сделал?

\iusr{Игорь Иванюкович}
\textbf{Ирина Бордюг} , Мои заслуги знают многие, но дело не в этом, Вы сторонник русского мира и этим всё сказано.
\end{itemize} % }

\iusr{Sergey Sheltushev}
Картина "Троє колаборантив"

\iusr{Vitalii Vitalii}
Особливо в Лондоні поважають Вільяма Воллеса

\iusr{Брати Грім}

Цих людей ідентифікуюють з Україною не тільки по їх спортивних досягненнях і
нагородах, але й по їх вчинках, словах і діях. Вони невільно стали публічними
особами зі всіма наслідками. Цей допис був би доречний після нашої Перемоги,
розвалу Вставалосколєнії і масових арештів агентів кгб по всьому світу. Одним
словом - на сьогоднішній день це допис вати про вату.

\iusr{Олександр Коваленко}
Зараз розплачусь. А як там обнімашки із силовиками країни-агресора під час війни?

\iusr{Татьяна Федорченко}
Не только в Историю Украины, но и в международные Залы Славы  @igg{fbicon.love.letter} 

\iusr{Дмитрий Загуменный}
 @igg{fbicon.hands.applause.yellow}{repeat=3} 

\iusr{Serhiy Mazur}
Очень глубоко и правильно. Спасибо

\iusr{Саша Лепесій}

\ifcmt
  ig https://scontent-yyz1-1.xx.fbcdn.net/v/t39.1997-6/s168x128/17640308_1652591141433953_2515677274297073664_n.png?_nc_cat=1&ccb=1-5&_nc_sid=ac3552&_nc_ohc=_gvTY4f3ofcAX-LalWi&_nc_ht=scontent-yyz1-1.xx&oh=617a107bffb8b8760c825eba26388f81&oe=6158A22E
  @width 0.1
\fi

\iusr{Валерій Преподобний}

Було б дуже добре, щоб ті двоє що зліва на фото не пішли в політику, як той, що
справа на фото, бо Україна цього не переживе.

\begin{itemize} % {
\iusr{Petro Chernii}
\textbf{Валерій Преподобний} та нормальний мер/політик, кращий за відсотків 80 українських політиків

\iusr{Валерій Преподобний}
\textbf{Petro Chernii} ну якщо порівнювати з льонею космосом, то напевно і кращий. @igg{fbicon.face.smiling.eyes.smiling} 
\end{itemize} % }

\iusr{Олександр Пшенік}

Британці не ведуть війну з агресором, війська, церкву та народ якого прославляє
Джошуа. Нахєр хай іде той тупорилий Усик

\begin{itemize} % {
\iusr{Александр Гичко}
\textbf{Олександр Пшенік} больной?

\iusr{Олександр Пшенік}
\textbf{Александр Гичко} невиліковно

\iusr{Александр Гичко}
\textbf{Олександр Пшенік} заметно

\iusr{Олександр Пшенік}
\textbf{Александр Гичко} як одразу помітно й те, що ти тупорило ватний

\ifcmt
  ig https://scontent-yyz1-1.xx.fbcdn.net/v/t39.30808-6/243194400_663140071329286_1817719314775148003_n.jpg?_nc_cat=105&_nc_rgb565=1&ccb=1-5&_nc_sid=dbeb18&_nc_ohc=nfxxYvHIFc0AX91zORP&_nc_ht=scontent-yyz1-1.xx&oh=1081050869be36d604154951d89c8501&oe=61583C0D
  @width 0.3
\fi

\iusr{Александр Гичко}
\textbf{Олександр Пшенік} ну по тебе видно, что ты нацист и бЭндэровец.

\iusr{Олександр Пшенік}
\textbf{Александр Гичко} так, я Бандерівець, і пишаюсь цим. А нацист - саме ти і такі як ти

\end{itemize} % }

\iusr{Lesya El}

Люди всегда добрее в стране, где нет коррупции. Честные правила для всех,
налоги идут на стариков, больницы, детей-сирот... А у нас Каждая власть
покрывает коррупцию, сами имеем и вам даём... А старых и слабых - в пропасть,
нагрузка для бюджета. Аморальность всегда вызывает злость, и никогда не
приносит счастья

\iusr{Сергей Гордиенко}
Забавно, жалко что неправда и манипуляция) но попытка была)

\begin{itemize} % {
\iusr{Yuliya Blazhnova-Rekukha}
\textbf{Sergii Gordiienko} конкретно, что неправда и в чём манипуляция?
\end{itemize} % }

\iusr{Petro Chernii}

Який гарний якісний пропогандистсько-маніпулятивний допис!

Звучить як "Такакаяразніца, якщо спортсмен успішний і золото взяв!?"

\begin{itemize} % {
\iusr{Aleksandr Logvinenko}
\textbf{Petro Chernii} он не под украинским флагом его взял? Он террорист? Не такие люди прославляють страну? Или такие "умы нации" как ты, коверкающие язык сделали больше?) достали вы, клоуны.

\iusr{Сергей Чуев}
\textbf{Aleksandr Logvinenko}  @igg{fbicon.hands.applause.yellow}{repeat=4}  @igg{fbicon.wink}  @igg{fbicon.trophy} 
\end{itemize} % }

\iusr{Oleksandr Sukhanov}
Ви просто дуже далеко від українських реалій.

\begin{itemize} % {
\iusr{Yuliya Blazhnova-Rekukha}
\textbf{Oleksandr Sukhanov} як це філософське зауваження пов'язано з моїми думками у тексті?)

\iusr{Volodymyr Doshchechkyn}
\textbf{Yuliya Blazhnova-Rekukha} хіба ж це дійсно були думки? Так деякі елементи ідеологічної війни проти держави України. Використання перемоги українця з метою посилення проросійських настроїв для поглиблення розколу українського суспільства в часи окупації частини України. Та ще й завернуте в папірець з надписом "Мир, жвачка, дружба". Я зрозуміло все написав?  @igg{fbicon.wink} 
\end{itemize} % }

% -------------------------------------
\ii{fbauth.kravchuk_vladimir.kiev.ukraina.akter.dikij_teatr}
% -------------------------------------

Англія не втрачає свої синів і дочок тисячами вже 7 років, а Джошуа не
заперечує цього факту. Йому вистачає мізків залишатись боксером!

\begin{itemize} % {

% -------------------------------------
\ii{fbauth.makaruha_viktor.kiev.ukraina.mvd.major}
% -------------------------------------

\textbf{Вова Кравчук} 

самі шановні, самі створили таку ситуацію, на яку ворог чекав близько 20 років.
Деякі експерти попереджали, що в цьому штучному Майдані прихований план, який
обернеться плачевно для України, але організатори і керівники не звертали на це
уваги, вони йшли своїм шляхом - це ВЛАДА @igg{fbicon.index.pointing.up} ️
@igg{fbicon.exclamation.mark} А ворог просто скористався тією ситуацією... А
наші "скакуни" були під ейфорією, та велися на заклики тих, хто потім зайняв
провладні кабінети, а хто крісла в парламенті....

\iusr{Валерий Лукашов}
\textbf{Viktor Makaruha} Зависть правит миром...(не любовь).

\iusr{Вова Кравчук}
\textbf{Viktor Makaruha} не дивно, що ви в МВС з таклю позицією працюєте!
\end{itemize} % }

\iusr{Вадим Лысый}
Грамотно написанные мысли  @igg{fbicon.thought.baloon} 

\iusr{Геннадий Гращенко}
?школы

\iusr{Andriy Borodavko}
Показова мова написання поста)

\begin{itemize} % {
\iusr{Yuliya Blazhnova-Rekukha}
\textbf{Andriy Borodavko} краще так категорично не коментувати пости незнайомих людей, щоб не втрапити у халепу, як ви зараз. Я цей текст подумки перекладала з української. І люди, які мене знають, знають і чого я так роблю

\iusr{Ольга Горлатая}
\textbf{Andriy Borodavko} Кто о чем, а голый о бане....

\iusr{Галина Хахалева}
\textbf{Andriy Borodavko} Легче стало? Гав!!! И в будку. И фото соответствует @igg{fbicon.face.sleepy}{repeat=4} 

\iusr{Семён Мушер}
\textbf{Andriy Borodavko} Тебе шо то не нравится?

\iusr{Евгения Жинкина}
\textbf{Andriy Borodavko} Украину не мовой славят, а делами!

\end{itemize} % }

\iusr{Igor Chornokulskiy}
 @igg{fbicon.biceps.flexed}{repeat=3}  @igg{fbicon.thumb.up.yellow} 

\iusr{Iryna Moroz-Pyrtko}
До глубины , до мурашек  @igg{fbicon.sparkles}  как правильно вы все подметили  @igg{fbicon.heart.blue}  @igg{fbicon.heart.yellow} 

\iusr{Olga Bodriagova}

Абсолютно не слідкую за боксом, але здивована таким маніпулятивним постом.
Порівняння абсолютно некоректне і недоречне. В UK немає війни 8й рік поспіль,
на англійську мову в UK ніхто не зазіхає, іншої другої мови в них не було і
немає. В них немає 15 тисяч втрачених життів і сотень тисяч зламаних доль
внаслідок дій сусіда агресора. В Британії наразі - мир, в Україні - війна. Але
згадайте хоча б протести в Британії після терактів - чи такі вони були
«спокійні» та «без ненависти», коли виходили на масові протести? І це лише
десятки втрачених життів. В Україні рахунок іде на тисячи.

Право на відстоювання своєї мови і своєї церкви - це не агресія окремих
украінців, це боротьба за виживання. Або вони нас остаточно знищать, або ми
таки виборемо право на свою ідентичність. Спортсмени, які представляють свою
країну, не є окремими людьми з власними досягненнями, вони виступають від
України, під її прапором. Має бути елементарна повага до своєї історії, своєї
спадщини, свого народу, який досі кров’ю (і це, на жаль, реальність, а не
пафос) виборює свої права- бути Українцями. Прикро, що молоді талановиті
журналісти пишуть дописи в наративі російського «какая разница».

\begin{itemize} % {

% -------------------------------------
\ii{fbauth.kozhina_tatiana}
% -------------------------------------

\textbf{Olga Bodriagova} На Украине гражданская война. Украина убивает в 14-том начала убивать граждан Украины и продолжает эту бойню.

\iusr{Yuliya Blazhnova-Rekukha}
\textbf{Olga Bodriagova} не знаю, що пишуть молоді талановиті журналісти, а я пишу свої думки, яких ви не побачили і не зрозуміли. "Какая разница" это мимо

\iusr{Olga Bodriagova}
\textbf{Yuliya Blazhnova-Rekukha} 

ваші думки викладено дуже чітко, зрозуміло для всіх. Ви порівнюєте супер
толерантних британських вболівальників з «агресивними», на вашу думку,
українцями. Але ж ті самі британці були ох які нетолерантні лише декілька років
тому після масових терактів в країні. І то лише кілька жертв! Навіть не сотні і
не тисячи, а мільйони британців вийшли з протестом. Британцям до украінськоі
толерантності (що не дивно, бо наша толерантність є продуктом репресій і
голодоморів) ох як ще далеко! Якщо цікавитись історією двох народів, таке
порівняння навіть не спаде на думку...

\iusr{Yuliya Blazhnova-Rekukha}
\textbf{Olga Bodriagova} 

ні, ви не зрозуміли посил мого тексту. Нав'язувати мені сенси, яких немає в
тексті не треба. Ви навіть не можете визначитися, як відноситися до моїх
коментарів. Лайки то ставите, то витираєте

\ifcmt
  ig https://scontent-yyz1-1.xx.fbcdn.net/v/t1.6435-9/243448951_2934099620162146_3747633075550901545_n.jpg?_nc_cat=105&_nc_rgb565=1&ccb=1-5&_nc_sid=dbeb18&_nc_ohc=WnT-r7Nh_68AX_H3VZr&_nc_ht=scontent-yyz1-1.xx&oh=bb0511394ac22dc278eca731ea8257ba&oe=617797ED
  @width 0.4
\fi

\iusr{Olga Bodriagova}
@Yuliya Blazhnova-Rekukha Я пишу з телефону, не завжди можу влучно потрапити в
потрібне поле:) але ваша увага до моєї скромної персони (навіть зі скріншотом!
Вау!) мене потішила  @igg{fbicon.smile} 

\iusr{Olga Brown}
\textbf{Tanya Kozhina} 

Це ви про що? Це Україна вдерлася в Росію? Це Україна вкрала в Росії кавалок
території? Яка громадянська війна? Якби ви сиділи у себе вдома і займалися
своїми справами ніхто би не загинув.

\iusr{Yuliya Blazhnova-Rekukha}
\textbf{Olga Bodriagova} 

так, звернула увагу, бо працюю з медичними директорами і звикла, що в них
критичне мислення на високому рівні розвинене, тому дивуюсь вашим коментарям

\iusr{Olga Bodriagova}
\textbf{Yuliya Blazhnova-Rekukha} 

пані Юлія, ви зробили мій день! Навіть на мою сторінку зазирнули! Подвійне вау!
Дякую за ваш час!

\iusr{Tanya Kozhina}
\textbf{Olga Brown} Я ,в отличии от вас, живу дома, а вот вы "сидите"в Лондоне, вот и не возникайте.

\iusr{Olga Brown}
\textbf{Tanya Kozhina} і що ж ви можете ще сказати? Я сижу в Лондоні і що? Юлія
Блажнова також сидить в Лондоні і також вазникаєт. І орди російських олігархів
тут сидять і вазникают. А ви де сидите? А вазникаєте про Україну... Сидіть там
у себе і не вазникайте. І брьісь из Украиньі.

\end{itemize} % }

\iusr{Вячеслав Павленко}
Все четко, даже мягко написано
...

\iusr{Alona Petkevicius}

Як завжди до мурашок \textbf{Yuliya Blazhnova-Rekukha}.

Дякую що пишеш про це, і дякую , що так! Негатив висказувань останньої доби
заполонив радість та гордість перемоги , шкода, що такий успіх перетворюється у
сірий бруд ...« дай за щось зачепитися»...

Я була щасливою бачити бій, там були : і повага ,і сила ,і Віра , і гопак
@igg{fbicon.heart.blue}  @igg{fbicon.heart.yellow} !

\iusr{Виталий Аникин}
Какие правильные слова!  @igg{fbicon.hands.applause.yellow} 

\iusr{Юрій Кузьмич}
Зараз розплачусь.

\iusr{Сергей Ивон}
Велике дякую українському гамноТБ, ні один канал не транслював бій.

\iusr{Katia Bodnarchuk}
Надо бы съездить за попкорном, чтобы веселее коменты было читать. Ваши посты перечитываю и разбирают как горячие пирожки  @igg{fbicon.grin} 
Блин, с топливом беда  @igg{fbicon.beaming.face.smiling.eyes} 
Тогда не буду читать комментарии вечно не довольных людей и диванных экспертов.
Спасибо, что делитесь мыслями  @igg{fbicon.face.blowing.kiss} 

% -------------------------------------
\ii{fbauth.skubenko_natalia.krivoj_rog.ukraina}
% -------------------------------------

Навчитися бути людиною в будь-якому випадку, буди щасливим не зважаючи на
національність, поважати інших людей, так багато ще треба зробити на цій
планеті нам, людям) всім щасти)

\iusr{Ольга Окара}
\textbf{Yuliya Blazhnova-Rekukha} , глубоко со смыслом, очень красиво. Спасибо!

\iusr{Yurii Kravchenko}

я б подивився, щоб брити зробили з Джошуа, якби він бодай заїкнувся, що вони з
французами один народ! в тому то і суть, що політична доктрина спортсменів в
цивілізованому світі повністю співпадає з політичною доктриною країни, що її
вони представляють!

\begin{itemize} % {
\iusr{Yuliya Blazhnova-Rekukha}
\textbf{Yurii Kravchenko} батько прем'єр-міністра Великої Британії французького походження) торік отримав громадянство Франції

\iusr{Alexandra Balitskaya}
\textbf{Yurii Kravchenko} не ідеалізуйте британців. тут ніхто нічого б не зробив- brexit приклад, добре наробив

\iusr{Yuliya Blazhnova-Rekukha}
\textbf{Alexandra Balitskaya} я не потребую порад, кого мені ідеалізувати

\iusr{Emma Braun}
\textbf{Yuliya Blazhnova-Rekukha} і? Для чого йому громадянство Франції, як думаєте?

\iusr{Yuliya Blazhnova-Rekukha}
\textbf{Emma Braun} думаю, через любовь до сиру камембер)
\href{https://www.facebook.com/100006864096210/posts/2747022728869837/}{%
Нас  закрыли на третий национальный локдаун. Пока до марта, с вероятностью продления опять, %
Yuliya Blazhnova-Rekukha, facebook, 05.01.2021}

\iusr{Emma Braun}
\textbf{Yuliya Blazhnova-Rekukha} а все таки ? )

\iusr{Yuliya Blazhnova-Rekukha}
\textbf{Emma Braun} через позицію по Brexit

\iusr{Светлана Черненко}
\textbf{Yurii Kravchenko} они вообще аполитичны

\iusr{Emma Braun}
\textbf{Yuliya Blazhnova-Rekukha} щоб внуки мали можливість також отримати французьке громадянство і насолоджуватися камембером)

\iusr{Alexandra Balitskaya}
\textbf{Yuliya Blazhnova-Rekukha} то було не вам. Я вашу думку розділяю - британцям в голову не прийде відмовлятись від переможця бо не 'дуже британець'.

\iusr{Yuliya Blazhnova-Rekukha}
\textbf{Alexandra Balitskaya} вибачте, тепер бачу, що не мені

\end{itemize} % }

% -------------------------------------
\ii{fbauth.kondratenko_aleksandr.kiev.ukraina}
% -------------------------------------

Но Британия не воюет восьмой год и ежедневно ее сыновья не гибнут и не стают
калеками на войне. И огромная часть ее территории не захвачена
соседом-террористом с атомной дубиной за поясом!

\begin{itemize} % {
\iusr{Yuliya Blazhnova-Rekukha}
\textbf{Олександр Кондратенко} как ваш комментарий связан с моим текстом?

\iusr{Олександр Кондратенко}
\textbf{Yuliya Blazhnova-Rekukha} Конечно связан ! РПЦ это вражеская агентура и любые контакты с ними известного человека наносят вред нашему государству!

\iusr{Yuliya Blazhnova-Rekukha}
\textbf{Олександр Кондратенко} при чём тут мой текст?)

\iusr{Олександр Кондратенко}
\textbf{Yuliya Blazhnova-Rekukha} 

Почитайте внимательно свой же текст, где упоминается РПЦ. И вы знаете о
постоянных контактах Усыка и Ломаченко с РПЦ. Они оба выдающиеся личности,
кумиры миллионов и ответственность на них несоизмеримо больше чем на обычном
гражданине нашей страны.

\iusr{Yuliya Blazhnova-Rekukha}
\textbf{Олександр Кондратенко} и вывод какой в моём тексте?

\iusr{Vera Pavel}
\textbf{Yuliya Blazhnova-Rekukha} Здравствуйте. Не обращайте внимания, он как плохой танцор, которому мешают я**а(простите).

\iusr{Сергей Гущин}
\textbf{Олександр Кондратенко} уёб.ок

\iusr{Viktor Koniakov}
\textbf{Олександр Кондратенко} и вы не воюйте

\iusr{Юрий Ян}

Вот же уродец моральный! Ему закурила голову политика и киевская хунта, что
захватила почти всю Украину своим нацистским бредом, что убивает своих граждан
в Донбассе и Луганске и убивала бы в Крыму, если бы Россия не вмешалась! Какая
тебе разница, ты такой злой, что готов убивать всех несогласных, вместо общения
и объяснения своей позиции

\end{itemize} % }

\iusr{Слипушкин Андрей}
 @igg{fbicon.thinking.face}  @igg{fbicon.thumb.up.yellow}  @igg{fbicon.hands.applause.yellow}{repeat=3}  @igg{fbicon.hands.shake} 

\iusr{Volodymyr Doshchechkyn}

Бідний Сашко. Він навіть і не здогадується, як його користують в гібридної
війні проти його ж коханої країни. Вже два дні читаю дописи від тролів і просто
корисних путєнських ідійотів, "... як за нього вся Расея" ... Узбагойтеся,
тролики. Все НАТО вчора вболівало за Сашу!  @igg{fbicon.wink}  ))

\begin{itemize} % {
\iusr{Oleg Zabielin}
\textbf{Volodymyr Doshchechkyn} как здорово что Вы не идиот, в отличие от них всех, правда? )
\end{itemize} % }

\iusr{Svitlana Kovalova}
Им это было нужно, но веком раньше. Нам это нужно сейчас.

\iusr{Родомир Степной}

Як файно написано, як раз для "м@лоросів". Герой той, хто віддав частину себе
не за гроші, пояси чи славу. Забрав частину життя в мами, тата, дружини.

А це просто бізнес. І на фото не герої, п люди, які заробляли грощі своєю
працею, як і шахтарі, металурги чи інші люди. Правда, їх ніхто не героями не
називає.

Усік зрадив Україну ще у 2014, коли окупантом Аксьоновим відкрив турнів з боксу
в окупованому Криму. Розповідати далі сенсу не має.

\begin{itemize} % {
\iusr{Yuliya Blazhnova-Rekukha}
\textbf{Родомир Степной} до речі, Усик і Джошуа так і сказали в Лондоні "це наша робота, і ми її зробимо"

\iusr{Родомир Степной}
\textbf{Yuliya Blazhnova-Rekukha} святі люди:)
\end{itemize} % }

\iusr{Валентина Елисеева}

Чудесная публикация, золотые слова! Время расставит приоритеты и покажет кто
был прав. Но то, что в нашей неблагодарной стране вырастают Олимпийские герои и
становятся легендами спорта мы все восхищаемся! Спасибо автору, спасибо парни,
что остаётесь украинцами несмотря ни на что


\iusr{Дмитрий Волянский}

На самом деле.. все особи на фото довольно лживие люди которые хотят казаться
лучше чем есть.. это было бы нормально если бы они не были публичными людьми на
которых равняются..


\iusr{Володимир Тихий}
Сука, АПД все испортило. Уже думал вот нормальный пост, но нет... @igg{fbicon.thinking.face} 

\iusr{Світлана Мороз}
Неблагодарные люди были, есть и будут((...

\iusr{Сергей Чепалда}
Все так ... Но и завзятых патриотов можно и нужно понять и принять . Война таки

\iusr{Инна Рябова}
От души написано) спасибо  @igg{fbicon.smile} 

\iusr{Сергій Височин}
Прекрасно. Но натянуто, тема раскрыта не полностью в идее патріотизму

\begin{itemize} % {
\iusr{Yuliya Blazhnova-Rekukha}
\textbf{Сергій Височин} раскрывайте)

\iusr{Сергей Пиковский}
\textbf{Сергій Височин} а какая разница?  @igg{fbicon.smile} 
\end{itemize} % }

\iusr{Малхази Гогинадзе}

Вот я, ешо раз хочу подчеркнуть то огромное значение этой блестящей победы Саши
Усика, в Лондоне, перед всем миром, поднятием украинского флага и исполнением
украинского гимна, вот это действо по значению больше, чем все заседания т.н.
слуг народа верховной рады за всё время существовании!!

\iusr{Olga Malchevska}

Юліє, не зовсім так про ВВС - ВВС Sport транслювали поєдинок наживо -
радіо-подкастом і текстом, доступним із будь-якого куточка Британії. То ж
платники податків мали можливість стежити за боєм. А хто вже хотів відео -
Скай.


\begin{itemize} % {
\iusr{Yuliya Blazhnova-Rekukha}
\textbf{Olga Malchevska} 

слідкувати за боксом такого рівня у текстовому режимі, або по радіо?) Це було б
круто, коли BBC з'явилося у 1922 році, але не сьогодні) А тут мій пост про
обов'язкову сплату за BBC та про інше
\url{https://www.facebook.com/100006864096210/posts/2747734592131984/} А Sky повністю
комерційний


\iusr{Olga Malchevska}

Важливо не підмінювати поняття - права на трансляцію відео боксу дуже дорогі, а
платники податків англійські платять ВВС не за розваги, а за інформування
насамперед - і це, будт ласка, текст і подкаст, і новини -
відео-репортажі-прев’ю (під час знімання якого ми з Вами бачилися). (Я стежила,
до речі, за подкастом). А ще ВВС чесно звітує за кожну копійку. І мусить
обирати, куди вкласти бюджет і віддає перевагу дитячим навчальним програмам,
наприклад, а не відео-трансляції боксу. Я просто за чесність.


\iusr{Yuliya Blazhnova-Rekukha}
\textbf{Olga Malchevska} 

по лінку, що вище🔝я розповідаю і про дитячі навчальні програми. Немає
необхідності доводити мені, що відсутність трансляції боксу по BBC, це
нормально. Саме про це спокійно і зазначено в моєму тексті. До того, для мене
логічно сплатити Sky 5 фунтів за перегляд події. Бокс по радіо - це не для
мене) А ось з чесністю проблем в мене ніколи не було)

\end{itemize} % }

\iusr{Александр Карасик}
Супер пост!!!
Благодарю  @igg{fbicon.hands.pray} 

\iusr{Andrey Ignatenko}

А при чем здесь колебания общественного мнения? Важны поступки. Он набоксировал
много и наговорил много. Отличный боксёр, ватный гражданин. Когда-то Лондон
Немцы здорово бомбили и нация тогда встала одним монолитом. А сейчас у них
такой необходимости нет.


\iusr{Hanna Poliakova}

Ані бєз нєнавісті балєют, а вот укрАінци - С ненавістью!.. @igg{fbicon.thinking.face} Дєйствітєльно! С
чєго би?! Откуда ета нєнавість взялась ваабщє?!

\begin{itemize} % {
\iusr{Yuliya Blazhnova-Rekukha}
\textbf{Hanna Poliakova} представляю, как отвратительно звучит ваше кривляние) Только посыл текста вы не поняли

\iusr{Сергей Удовик}
\textbf{Yuliya Blazhnova-Rekukha} она не кривляется, она так мыслит

\iusr{Hanna Poliakova}
\textbf{Yuliya Blazhnova-Rekukha} це ви нє понялі і прєдставілі мій ератив, як крівляніє. А атвратітєльно - це те, що ви понаписували, порівнявши історію України з історією Англії, натякнувши на расову та інші види дискримінацій з боку українців, просунувши «какую разніцу» та ін. дурню.

\iusr{Hanna Poliakova}
\textbf{Сергей Удовик} який розумний дядя

\iusr{Yuliya Blazhnova-Rekukha}
\textbf{Hanna Poliakova} в мене ні порівнянь історії, ні натяків) а ще в мене не буде для вас психологічної підтримки різкого загострення комплексу меншовартості
\end{itemize} % }

\iusr{Сергей Удовик}

Показательно, что делят людей на ватников и вышиватников примитивные люди,
которых Маркузе описал в Одномерном человеке, а реально они являются
кастрюлеголовыми


\iusr{Елена Гавва}
Благодарю, шикарный пост!!!

\iusr{Viktor Vysotsky}
Глубокие смыслы  @igg{fbicon.hands.pray} ☀️
Юлия , выражаю вам своё почтение @igg{fbicon.flag.ukraina} @igg{fbicon.face.smiling.hearts} 

Благодарю.
Нищеброды-хейтеры, которые нашли способ поднять самооценку через критику президента страны в отсутствии бесплатной трансляции и к вашему посту придрались - не стоит обращать даже внимание на них )
Продолжайте сиять  @igg{fbicon.face.eyes.star}  @igg{fbicon.hands.pray} 

\iusr{Светлана Нагорнюк}

Нажаль, пост не тільки про какаяразніца (не здалося). Майжев кожному абзаці зневага і маніпуляція.

З цікавістю слідкую давно. Навіщо вам був потрібен цей пост? Незрозуміло....

\begin{itemize} % {
\iusr{Yuliya Blazhnova-Rekukha}
\textbf{Светлана Нагорнюк} ви розумієте, що ваші вигадки і таврування образили мене, як автора?
\end{itemize} % }

\iusr{Михаил Пал}
А сказано добре

\iusr{Людмила Буглак}
Саша Усик! Симферополец ! Крымчанин! Вот Ты ещё раз доказал, что Мы народ единый ! Политики уйдут, а мы останемся !!

\begin{itemize} % {
\iusr{Yuliya Blazhnova-Rekukha}
\textbf{Людмила Буглак} как доказал?
\end{itemize} % }

\iusr{Людмила Буглак}
Россия + Украина = навсегда!!

\begin{itemize} % {
\iusr{Yuliya Blazhnova-Rekukha}
\textbf{Людмила Буглак} это бред

\iusr{Vital Ribak}
\textbf{Людмила Буглак} после обстрела градами с ростовской области?

\iusr{Анатолий Берлинер}
\textbf{Людмила Буглак} при чём тут расия нарисовалась?
\end{itemize} % }

\iusr{Владимир Костинский}

\ifcmt
  ig https://scontent-yyz1-1.xx.fbcdn.net/v/t39.1997-6/s168x128/47270791_937342239796388_4222599360510164992_n.png?_nc_cat=1&ccb=1-5&_nc_sid=ac3552&_nc_ohc=H2aA562hcy8AX_KxMeM&_nc_ht=scontent-yyz1-1.xx&oh=7e936ce9ad908c9f7170de4640bcfb2b&oe=615836B6
  @width 0.1
\fi

\iusr{Владимир Костинский}

\ifcmt
  ig https://scontent-yyz1-1.xx.fbcdn.net/v/t39.1997-6/s168x128/47270791_937342239796388_4222599360510164992_n.png?_nc_cat=1&ccb=1-5&_nc_sid=ac3552&_nc_ohc=H2aA562hcy8AX_KxMeM&_nc_ht=scontent-yyz1-1.xx&oh=7e936ce9ad908c9f7170de4640bcfb2b&oe=615836B6
  @width 0.1
\fi

\iusr{Yuri Umansky}

В Крыму, в Симферополе, проявили солидарность с Украиной и тоже не
транслировали поединок. )) Интересный факт: Усик и Яна Клочкова выросли как
спортсмены в Симферополе в одном здание спорткомплекса.


\iusr{Татьяна Мизенина}
А виталя за кого болел?

\begin{itemize} % {
\iusr{Николаенко Людмила}
\textbf{Татьяна Мизенина} А кто тут виталя? А танька?

\iusr{Татьяна Мизенина}
\textbf{Николаенко Людмила} это тот, который виталька
\end{itemize} % }

\iusr{Лариса Тарасенко}

\ifcmt
  ig https://scontent.xx.fbcdn.net/v/t39.1997-6/s480x480/47472862_1846683478763869_5271603212267290624_n.png?_nc_cat=1&ccb=1-5&_nc_sid=0572db&_nc_ohc=hGvBFZgyivwAX-cuTdf&_nc_ad=z-m&_nc_cid=0&_nc_ht=scontent.xx&oh=563fcc93d5de634bfd4713309229d820&oe=6157A8E7
  @width 0.1
\fi


\end{itemize} % }
