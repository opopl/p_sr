% vim: keymap=russian-jcukenwin
%%beginhead 
 
%%file 25_12_2020.news.ua.obozrevatel.vjatrovich_volodymyr.1.leontovych_nkvd
%%parent 25_12_2020
 
%%url https://www.obozrevatel.com/ukr/society/pisnya-yaku-vzhe-nihto-ne-pochue-yak-zaginuv-avtor-vsesvitno-vidomogo-schedrika.htm
 
%%author 
%%author_id vjatrovich_volodymyr
%%author_url 
 
%%tags nkvd,leontovych_mykola,schedryk
%%title Пісня, яку вже ніхто не почує. Як загинув автор всесвітньо відомого "Щедрика"
 
%%endhead 
 
\subsection{Пісня, яку вже ніхто не почує. Як загинув автор всесвітньо відомого \enquote{Щедрика}}
\label{sec:25_12_2020.news.ua.obozrevatel.vjatrovich_volodymyr.1.leontovych_nkvd}
\Purl{https://www.obozrevatel.com/ukr/society/pisnya-yaku-vzhe-nihto-ne-pochue-yak-zaginuv-avtor-vsesvitno-vidomogo-schedrika.htm}
\ifcmt
  author_begin
   author_id vjatrovich_volodymyr
  author_end
\fi

\index[names.rus]{Леонтович, Микола!Автор Щедрика!Пісня, яку вже ніхто не почує, 25.12.2020}

\ifcmt
pic https://i.obozrevatel.com/news/2020/12/25/000a48df2.jpg?size=972x462
\fi

Нині ця пісня Миколи Леонтовича звучатиме по всьому світу. Цей текст про його
твір, який ніхто не почує.

Пісня. 

Вже ранок? То він таки заснув. Попри вечірню тривогу, сон був глибокий і
приємний. Багато світла і музики. Хор знов співав "Щедрика", зривав захоплені
овації публіки. А потім була ще одна пісня і зала ошелешено мовчала. В очах
кожного з присутніх світився захват. Саме про неї він мріяв, відколи почав
творити. Вона була прекрасна і він чітко пам'ятає її, кожен рядок, кожну ноту.
Ця великодня веснянка досі звучить в його голові. Він може хоч зараз її
записати. Треба зробити це негайно, щоб подарувати цю красу світові. О, він
певен: пісня затьмарить навіть "Щедрика"...

– Вставай давай! – перебив захоплені думки грубий голос.

Кричав чоловік, вирячивши на нього сірі очі. Страшнішим за шалений погляд було
хіба дуло рушниці, спрямоване йому в груди.

Сон миттєво зник, пригадалася причина вчорашньої тривоги. Нею був саме цей
чоловік, який увечері постукав у двері батькового дому. Господар, в якого досі
сила виховання переважувала звички, які гарантували безпеку, впустив незнайомця
в хату. Той був напідпитку, хатнє тепло після зимового вечора налаштувало гостя
на балакучий настрій.

Від почутого тривога охопила вже не лише його, а усіх - батька, мати, сестру. В
нього все похололо всередині, коли почув оте незнайомцеве "я в ЧК работаю". Той
тим часом вів далі про свої подвиги у боротьбі з "петлюрівськими бандами", про
завдання із нищенням їх отаманів. "Значить не по мене, – з полегшенням подумав
тоді він, – принаймні цього разу".

Ніхто з присутніх не підтримував розмови, та це не заважало балакучому чекісту
не змовкати кілька годин. Врешті батько, помітивши позіхання непроханого гостя,
запропонував вкладатися спати.

– Ляжете з Миколою в кімнаті, мати постелить.

На щастя там чекіст не продовжив балаканини, а майже миттєво заснув. Микола
почав заспокоюватися і міг тепер думати про "Русалчин Великдень" – нову оперу,
над якою працював. А за кілька хвилин, попри набридливий храп чекіста і
попередні переживання, провалився в глибокий сон, який протривав до ранку.

– Вставай, кому сказал! – повторив чекіст. Микола остаточно розплющив очі,
натяг на ніс окуляри і зміг тепер краще роздивитися його. Схоже той був ще
більш п'яний як звечора. Коли і де він встиг ще напитися?

– Што, думал, пранєсло! Ан нєт, контра. Я тєбя уже давно вєду. Сабірайся, тєбя ждьот справєдлівая кара.

– Яка кара, про що Ви?

– Пра твою котрреволюционную деятельность

– Це якась помилка, я не займаюся політикою. Я музикант, композитор...

– Я знаю, кто ти, контра – Лєонтовіч. І прішол імєнно за табой. 

 – Я лише музику пишу, я не роблю нічого проти влади.

– "Нічого"? Да твая музика протів власті! Понял!

Чекіст кричав дедалі голосніше, розмахуючи рушницею спрямованою на Миколу, палець залишався на гачку зброї.

– Я творю пісні, – пробував заспокоїти його Микола.

– "Пісні"! А должен пісать пєсні! І нє на поповскую тєматіку, а нашу пролетарскую. 

 – Це не попівська тематика, це оборобки народної...

– І народ ваш тоже вєсь контра! – кричав далі чекіст – банди кругом, атамани, всєх перестрєлять!

– Да знаєм ми всьо пра тєбя, – продовжив він вже спокійніше, – і рабату твою на
правітєльство Петлюри, і пра хор твой, каторий по міру калєсіт с
контрреволюционнимі пєснямі.

– Це народні пісні, – спробував перебити Микола.

Але краще б він цього не робив – чекіст знову зірвався крик, знов почав
розмахувати рушницею.

– Я здєсь народ! Понял, контра! І народним будєт то что я скажу! Кара моя
народная!

Останні слова чекіста перервав гучний хлопок. Рушниця, яка гуляла в руках ще
хвилю перед тим, завмерла спрямована в груди Миколи. З дула здіймався маленький
хвостик диму. Його було замало для того, щоб заповнити кімнату, проте все
довкола раптом почало зникати залите туманом.

За хвилю чи вічність він побачив перелякане обличчя батька, що пробилося крізь
цей туман. Татові руки чомусь були залиті кров'ю. Здається він намагався
підняти його. Чи то він сам почав підніматися над землею, так легко, як уві
сні. Батько кричав щось крізь сльози, що заливали обличчя. Але Микола не чув
його. У вухах лунала прекрасна веснянка зі сну. Пісня, яку вже ніхто не почує.
