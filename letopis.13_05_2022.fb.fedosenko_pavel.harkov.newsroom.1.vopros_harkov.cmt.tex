% vim: keymap=russian-jcukenwin
%%beginhead 
 
%%file 13_05_2022.fb.fedosenko_pavel.harkov.newsroom.1.vopros_harkov.cmt
%%parent 13_05_2022.fb.fedosenko_pavel.harkov.newsroom.1.vopros_harkov
 
%%url 
 
%%author_id 
%%date 
 
%%tags 
%%title 
 
%%endhead 
\qqSecCmt

\begin{itemize} % {
\iusr{Yulia Yarmolenko}

Ми, звичайно, не в метро живемо, але й на голові родичів сидіти набридло.
Будемо повертатися. Поки що без дитини

\iusr{Maryna Khoroshenko}
Ми вирішили почекати ще, десь тижнів шість, ближче до липня будемо повертатися скоріше за всього.

\iusr{Олексій Жабін}
в ще громадський транспорт не працює

\iusr{Pavel Fedosenko}
\textbf{Олексій Жабін} с понедельника планируют запустить

\iusr{Iva Stishun}

Моё любимое место в Ха. Спасибо. И за фото и за посыл. Так и есть. Все мы
разные и по разному реагируем и действуем. Мне кажется никто не знает, что
будет, если сейчас многие вернутся и возникнет новое напряжение на разбитую
инфраструктуру и уставшую психику. Поэтому продолжается время сложных
выборов/решений и ответственности за них...

\iusr{Andrey Poyarkov}
Милый дом

\iusr{Igor Saldyga}
Нескорений Харків чекає на своїх мешканців

\ifcmt
  ig https://scontent-mxp1-1.xx.fbcdn.net/v/t39.30808-6/279740210_5256014951123908_2812767106461061280_n.jpg?_nc_cat=111&ccb=1-6&_nc_sid=dbeb18&_nc_ohc=SDRjQe18YUkAX-ScYkt&_nc_ht=scontent-mxp1-1.xx&oh=00_AT_rS6xn-KKn3BB-szuKr7LwLMHOpFDKbcjEK2R-kGDCfg&oe=628680CC
  @width 0.3
\fi

\iusr{Willy White}
100\%

\iusr{Олег Федоров}
саме так!

\iusr{Людмила Анненко}

Так, транспорт запускають, але не метро(, доїхати з Салтівки на Холодну Гору на
роботу треба буде дивитися з пересадками як(

\begin{itemize} % {
\iusr{Pavel Fedosenko}
\textbf{Людмила Анненко} метро запустят еще очень не скоро

\iusr{Людмила Анненко}
\textbf{Pavel Fedosenko} 

так, то ясно треба думати як добиратися, і біля нас на Салтівці магазини та
ринки порозбивали(, думаємо про повернення, бо викликають на роботу, але ще
дали відпустку до кінця травня, будемо дивитися як далі? Або на ровері їздити.

\iusr{Oksana Yarosh}
\textbf{Людмила Анненко} 

враховуючи, що впродовж перших двох тижнів обстрілювали надземний міст, саме
Салтівську лінію запустять останньою(

\iusr{Ihor Palchykivskyi}
\textbf{Павел Федосенко} судя по новостям, что скоро из метро выселяют - то к августу запустят

\iusr{Алёна Иванова}
\textbf{Людмила Анненко} велосипед всему голова

\iusr{Pavel Fedosenko}
\textbf{Ihor Palchykivskyi} если выселят в ближайшее время и если реально успели подготовить все за три месяца

\iusr{Oksana Yarosh}
\textbf{Ihor Palchykivskyi} 

ахах)) та ясно, спочатку всіх нажахати, що \enquote{неможливо} запустити метро раніше,
ніж за три місяці, а потім героїчно запустити його раніше озвучених строків
(йопта, харків'янчики, дивіться, яка у вас міська влада, п'ятирічку за три
роки!")

\iusr{Ihor Palchykivsky}
\textbf{Павел Федосенко} та там то 2 месяца то 3  @igg{fbicon.face.grinning.sweat} 

\iusr{Лариса Гнатченко}
Реально, все зависит от внутренней планки тревожности - ну, и от местоположения дома.
\end{itemize} % }

\iusr{Ekaterina Ivanitskaya}
Дом милый дом  @igg{fbicon.house.with.garden}  @igg{fbicon.heart.blue}  @igg{fbicon.heart.yellow} 

\iusr{Татьяна Прохорова}
Сумно але так і є((

\iusr{Анна Ерошкина}

Если совсем невыносимо там, где вы сейчас находитесь, то можно вернуться, жить
сможете. Но если вы в комфортных условиях для себя и детей, лучше подождать.

\iusr{Валерия Величко}
Одним постом ответил на все мои вопросы по поводу возвращения. Спасибо)

\iusr{Микита Соловйов}
Я немного иначе отвечаю. МОжно возравщаться (можно было и не уезжать), если понимаешь зачем.

\begin{itemize} % {
\iusr{Pavel Fedosenko}
\textbf{Микита Соловйов} ну, это несколько другая плоскость вопроса)
\end{itemize} % }

\iusr{Ум Карім}

Ми вирішили з дітьми поки не ризикувати. До того ж, школи не працюватимуть хто
зна скільки ще @igg{fbicon.face.sad.but.relieved}{repeat=3}. Подивимося до кінця літа. Як би не хотілося їхати,
почекаємо

\iusr{Vlad Rockway}
Виїжджаю в неділю в твою сторону)) обов’язково зустрінемось на пивко))

\iusr{Галина Чуйко}

Я вчера по рабочим вопросам вернулась в Харьков(громко сказано, тк сегодня уже,
точнее прямо сейчас еду в поезде уже обратно). Мы и так планировали
возвращаться как у детей школа закончится, те в начале июня, а посетив родной
город решила, что точно возвращаемся. Надеюсь что ситуация с бомбежками и
обстрелами сойдёт на нет за эти несколько недель. Очень хочется в это верить.
Но вот прямо сейчас я б ещё не стала и транспорт не функцилнирует(а наземный
общественный на мой взгляд это из серии треш, тк скопление людей на остановках,
вся поездка с загрузкой людьми, если не дай бог прилетит....). Только метро-и
безопасно и быстро и надёжно. Не понимаю, почему в Киеве его смогли быстро
запустить (пусть и не все ветки), а нашим нужно 3 месяца после окончания войны..
Бред же..

\begin{itemize} % {
\iusr{Таня Филоненко}
\textbf{Галина Чуйко} Я теж вважаю, що метро \textemdash самий безпечний транспорт, однозначно
\end{itemize} % }

\iusr{Boris Sevastyanov}

\obeycr
Тут ещё вопрос - зачем вернуться?
Если работать - то где и кем?
Есть ли машина?
Гортранспорт не работает.
Можешь ли оплачивать такси или будешь кататься на велике?
Готов ли морально, что может снова прилететь. И что, может быть даже бежать (если ядерная).
Отдельно на счёт детей.
Никто не знает как будут и будут ли работать садики, школы. Много из них разбиты или повреждены.
Будет ли штат учителей, будут ли сформированы классы и т.д.
Этого просто тупо никто не знает. И за любые деньги мира вам никто не даст ответ.
С детьми я бы повременил до осени.
Потому что один раз, 24 февраля, большинство людей говорили - да какая война в Харькове?
Ехать сейчас с уверенностью, что все закончилось и мы уже победили - это та самая недооценка противника.
Переможемо!
\restorecr

\begin{itemize} % {
\iusr{Oleksandr Sukhanov}
\textbf{Борис Севастьянов} Транспорт +- до кінця травня вже поїде. Звичайно не весь, але щось буде.

\iusr{Pavel Fedosenko}
\textbf{Борис Севастьянов} думаю, это все люди обдумывают в первую очередь

\iusr{Владимир Рожков}
\textbf{Борис Севастьянов} Борь, все излагаемое логично, но с легким оттенком пораженчества.
Реальная жизнь такая и есть - трава сквозь асфальт

\iusr{Boris Sevastyanov}
\textbf{Владимир Рожков} мне видится оттенок реализма

\iusr{Владимир Рожков}
\textbf{Борис Севастьянов} мы разные смыслы вкладываем в похожие звуки. Так придумал Бог

\iusr{Алексей Басакин}
И как бы первый транспорт начнут запускать с понедельника:)

\iusr{Nataliia Rieznikova}
\textbf{Борис Севастьянов}, 

полностью согласна с Вами. Еще от себя добавлю, пожилые люди. На своем примере,
и на примере друзей. К нам мамы приехали, ко мне в Краков, к подруге в
Дортмунд. Понимаем, что хотят домой. Но, раньше все проблемы решались за оплату.
Теперь, нет. Мой аргумент, и так очень много осталось пожилых без
родственников, и не нужно нагружать волонтеров. Да, деньги можно заработать,
взяв лишние заказы, и перевести волонтерам. Но ездить под обстрелами, это
совсем разные вещи. С огромной благодарностью, к тем, кто оставался и все это
делал.

\end{itemize} % }

\iusr{Владимир Рожков}
Все так, все так.

\iusr{Tatka Leetwin}
А какие службы такси работают? И насколько безумные цены?

\begin{itemize} % {
\iusr{Pavel Fedosenko}
\textbf{Tatka Leetwin} в апреле были, как до войны. Сейчас чуть выше, потому что бенз подорожал. Но уже не 10к с Салтовки до ЮЖД

\iusr{Iryna Goncharova Bagalij}
\textbf{Tatka Leetwin} їхала сьогодні з вокзалу за 76грн у центр

\iusr{Олег Кук}
\textbf{Tatka Leetwin} с Салтовки до м. Научная, 130 грн. Норм цены

\iusr{Ірина Бурла}
\textbf{Tatka Leetwin} Вчера ехала ЮЖД-АЛЕКСЕЕВКА 127 грн.

\end{itemize} % }

\iusr{Дубовик Екатерина}
Красота! Пятница 13.....

\iusr{Pavel Fedosenko}
\textbf{Дубовик Екатерина} это чуть раньше) но сегодня тоже было ок

\iusr{Alexander Kuleshov}
Упрощу для тех кто сомневается: ПОКА РАНО

\iusr{Vlasta Egina}
Пост очень правильный.
Фото суперское!!! Красота!

\iusr{Ленчик Десюшка}
Храни нас всіх Всевишній!!!

\iusr{Vlasta Egina}
Народ, выехавший из Харькова, не спешите возвращаться, подождите!!!

\iusr{Iluhis Pshenichny}
В район 23 августа приехал человек из Малой Даниловки и звонит : - ну вы там скоро? А то у вас тут стрёмно!  @igg{fbicon.beaming.face.smiling.eyes} 
Да у нас вообще тихо, в сравнении с Малой Даниловкой! Просто он неделю как домой вернулся из заграницы.

\iusr{Евгения Петрина}
Скорее бы наступил уже мир!!!
Слава Украине! Героям слава!!! @igg{fbicon.flag.ukraina}

\end{itemize} % }
