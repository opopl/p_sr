% vim: keymap=russian-jcukenwin
%%beginhead 
 
%%file slova.sssr
%%parent slova
 
%%url 
 
%%author 
%%author_id 
%%author_url 
 
%%tags 
%%title 
 
%%endhead 
\chapter{СССР}
\label{sec:slova.sssr}

%%%cit
%%%cit_pic
%%%cit_text
\enquote{Червону руту} играли и пели в концертных залах, ресторанах, на танцплощадках
по всему \emph{СССР} и за его пределами. Белорусский композитор Эдуард Ханок в одном
интервью признался, что под влиянием \enquote{Червоной руты} он написал другую
популярную песню на украинском языке - \enquote{Верба} на стихи Юрия Рыбчинского
%%%cit_comment
%%%cit_title
\citTitle{Какая легенда вдохновила Владимира Ивасюка написать \enquote{Червону руту} 
(украинскому шлягеру - 50)}, 
Татьяна Кроп, zen.yandex.ru, 07.12.2020
%%%endcit

%%%cit
%%%cit_pic
%%%cit_text
Вскоре на телеэкраны вышел первый в \emph{советской Украине} музыкальный фильм
\enquote{Червона рута}, в котором в главных ролях вместе с названными Зинкевичем и
Яремчуком снялась еще одна восходящая звезда - молодая преподавательница
культпросветучилища, очаровательная София Ротару.  На долгие годы украсила ее
репертуар \enquote{Червона рута}, которую София Михайловна любит исполнять в своих
сольных концертах. Во многом благодаря именно этой песне сформировался
сценический образ Софии Ротару - романтической, загадочной, умеющей любить и
влюблять в себя красавицы
%%%cit_comment
%%%cit_title
\citTitle{Какая легенда вдохновила Владимира Ивасюка написать \enquote{Червону руту} 
(украинскому шлягеру - 50)}, 
Татьяна Кроп, zen.yandex.ru, 07.12.2020
%%%endcit

bunyod yunusov
Мой адресс \emph{Советский Союз}, приезжайте в гости братья, сестры, друзья. Я Вас всех ждууу

Кирик Лонгинов
Мне 45 и я горжусь что из \emph{СССР}. Вроде мужик но хочется аж плакать,
старею чтоли, в детство тянет 

Владимир Гусев
Дорогой друг, не стыдись слёз о прекрасном... Ты не одинок в своём душевном
смятении. Мне седьмой десяток и печалюсь также. Эта композиция должна быть
вечной, но к сожалению \enquote{ни что не вечно}. Поверь, что моральные уроды,
воспитанные в \emph{СССР} и, сделавшие многое для утилизации \emph{Страны} и её искусства,
под эту песню тоже пускают сопли.... 

Lord БЭКЕТ'ов
2021 год! Мне 46, и я плачу! Поддержите! Хочу вернуться назад! 05.02.2021г.

Marat Mamadaliev
Мне 43 года и я рад что у меня было \emph{советское дедство}. 

Erbolat Saduov
Пацаны! А мне 22 мая исполнилось 64 года... я тоже рад, что \emph{советский} и тоже... 
реветь охота. 

TheTarantulll
Дедство у вас еще не началось)) 38, тоже из \emph{Советского детства}) 

Alex Porter
\emph{Советское детство}, самое светлое!!!! 

Валерия Ланская \enquote{Прекрасное далёко}
Хор мальчиков Московского хорового училища имени А.В.Свешникова
Музыка: Е. Крылатов, Слова: Ю. Энтин
к\ф \enquote{Гостья из будущего},  режиссер: П. Арсенов
Концерт из цикла \enquote{Живая музыка экрана} - \enquote{Мастера. Поэты Леонид Дербенёв и Юрий
Энтин}.  Концертный зал им. П.И. Чайковского
Январь, 2019 г.

%%%cit
%%%cit_pic
%%%cit_text
Формально названный \enquote{Обращением фюрера к германскому народу} текст в
05.30 по берлинскому времени (в 07.30 по Москве) зачитал третий человек в
Третьем Рейхе Йозеф Геббельс. Заявление \emph{советского} правительства - в
12.15 по московскому времени - второй человек в \emph{СССР} Вячеслав Молотов.
Именно в этих двух речах прозвучала основа версий причин германского нападения
на \emph{Советский Союз}, ставшего началом Великой Отечественной войны.
Нацистская версия состояла в том, что Москва якобы готовилась к наступлению и
сосредоточила на германской границе \enquote{все свои вооруженные силы, готовые
к броску}, в связи с чем Гитлер отдал приказ о наступлении, которое должно
спасти Европу от \enquote{еврейско-большевистских орд}
%%%cit_comment
%%%cit_title
\citTitle{Как начиналась война Германии против СССР 22 июня 1941 года}, 
Дмитрий Коротков, strana.ua, 21.06.2021
%%%endcit

%%%cit
%%%cit_head
%%%cit_pic
%%%cit_text
Рівно 80 років тому, 22 червня 1941 року з повідомлень радіостанцій громадяни
\emph{СРСР} дізналися про напад гітлерівської Німеччини та початок війни, яку пізніше
назвуть \enquote{Великою Вітчизняною}.  Для Сталіна і Червоної армії битва з колишнім
союзником Гітлером стартує провально. Мільйони вбитих і взятих у полон
\emph{радянських} солдатів не обіцяли \emph{Радянському Союзу} нічого доброго. Але перелом
все ж відбувся – Гітлер програв.  Як зазначає публіцист, письменник і історик
Марк Солонін, до перемоги \emph{СРСР} привело декілька основних причин і ще багато
різних за масштабом подій, які визначили перебіг війни
%%%cit_comment
%%%cit_title
\citTitle{Мовчання Сталіна, помилки Гітлера, окупація Києва і провалений бліцкриг – Марк Солонін про війну СРСР і Німеччини}, 
Євген Руденко, www.pravda.com.ua, 22.06.2021
%%%endcit


%%%cit
%%%cit_head
%%%cit_pic
%%%cit_text
Минули часи. \emph{Радянський Союз} розвалився. «Соціалістичний табір» розпався. У
Росії та Україні партійна номенклатура перекувала свою недавню владу на потужні
капітали, у цих незалежних державах оформилися кланово-олігархічні системи. З
тією різницею, що в Росії з часом владу перехопили вчорашні кадебісти. А в
Україні ніким не контрольовані свіжоспечені олігархи. Для захисту узурпованої
влади і накраденого добра їм так само, як і старій \emph{радянській} партноменклатурі,
довелося придумати цілу систему загроз для національних та історичних
цінностей. Систему гордощів за невідомо що
%%%cit_comment
%%%cit_title
\citTitle{Подвійний ворог України - ZAXID.NET}, 
Василь Расевич, zaxid.net, 18.06.2021
%%%endcit


%%%cit
%%%cit_head
%%%cit_pic
%%%cit_text
«Люк не открывать и ждать группу поиска».  В роковую ночь на 30 июня 1971 года
ситуация поначалу развивалась штатно. 29 июня в 21 час 25 минут по московскому
времени «Союз-11» отделился от орбитальной станции «Салют-1». Бортинженер
Волков доложил, что визуально наблюдается расхождение: «Станция пошла слева от
нас, с разворотом». В ответ Земля дала условия посадки: над территорией \emph{СССР}
малооблачно — три-четыре балла. В районе приземления ясная погода, видимость
составляет десять километров, температура — 16 градусов выше ноля, ветер —
два-три метра в секунду, давление у земли — 720 миллиметров ртутного столба
%%%cit_comment
%%%cit_title
\citTitle{«Они были обречены» 50 лет назад погиб экипаж «Союза-11». Кто
виноват в главной трагедии советской космонавтики?: Космос: Наука и техника:
Lenta.ru}, Сергей Варшавчик, lenta.ru, 30.06.2021
%%%endcit


%%%cit
%%%cit_head
%%%cit_pic
%%%cit_text
Пытки манной кашей, ношение колгот, отлучение от телевизора...  Юлия Мендель
выпустила книгу, в которой, судя по анонсу, она описала о гнете \emph{СССР} и
России, который она пережила!  Мы с Юлией одного года рождения - 1986 Я тоже
решил рассказать о тоталитарном давлении \emph{Страны Советов} на меня. Хватит
молчать!  Ведь пережили мы в свои 5 лет многое, не пожелаешь врагу. Было все!
%%%cit_comment
%%%cit_title
\citTitle{Пятилетняя Юлия Мендель пережила жутчайшее давление советского режима / Лента соцсетей / Страна}, 
Василий Апасов, strana.ua, 03.07.2021
%%%endcit

%%%cit
%%%cit_head
%%%cit_pic
%%%cit_text
Победа большевистской революция в России и приход к власти в ней грузинского
семинариста-недоучки, об колено сломавшего планы мирового интернационала,
вместе с их носителями, ни в какие планы не вписывались.  Ровно та же ситуация
и с пресловутым «планом Алена Даллеса» по развалу уничтожению \emph{Советского Союза},
который оказывается детально изложен \emph{советским} писателем Анатолием Ивановым во
втором томе романа «Вечный зов» 1973 года
%%%cit_comment
%%%cit_title
\citTitle{Революция Духа – единственный путь спасения России}, 
Юрий Барбашов, voskhodinfo.su, 30.06.2021
%%%endcit

%%%cit
%%%cit_head
%%%cit_pic
%%%cit_text
Однако культурная вестернизация не прекратилась. Как и во времена Брежнева, она
идет помимо воли властей и в нее втягиваются все более широкие слои населения:
молодежь, средний класс крупных городов, интеллигенция.
Культурная вестернизация и путинский режим - несовместимы. Она – бомба
замедленного действия, которая рано или поздно разнесет его в клочья, как, в
прошлом, \emph{СССР}. Что, впрочем, абсолютно не гарантирует успех последующих
демократических реформ
%%%cit_comment
%%%cit_title
\citTitle{Как разведенный диктатор-отравитель собирается защищать \enquote{традиционные ценности}}, 
Игорь Эйдман, opinions.glavred.info, 06.07.2021
%%%endcit

%%%cit
%%%cit_head
%%%cit_pic
%%%cit_text
До речі, той же феномен відстежується в більшості інших колонізованих
\emph{Радянським Союзом} республік. Можливо, лише в Білорусі це поки що не так
очевидно (але щось мені підказує, що Білорусь своє слово ще скаже). Так навіщо
наступати на граблі, які вже \enquote{перевірила} \emph{радянська влада}?!  Яка
є альтернатива? На мій погляд, альтернатива очевидна – вивчати українську
добровільно
%%%cit_comment
%%%cit_title
\citTitle{Чи можна примусити збірну України розмовляти державною мовою?}, 
Євген Лапін, www.pravda.com.ua, 06.07.2021
%%%endcit

%%%cit
%%%cit_head
%%%cit_pic
%%%cit_text
В общем, \enquote{легендарный} Бандера — её кумир с рождения, а все люди в
\emph{СССР}, включая и её шефа Зеленского, \enquote{ходили не в ту школу и выполняли
не то домашнее задание}. Непонятно только, откуда у Мендель, прожившей в \emph{СССР}
лишь первые пять лет своей жизни, столько \enquote{глубоких} познаний о \emph{Союзе}.
От не очень глубокого ума или с пелёнок — сразу под пресс КГБ. \enquote{Но
позвольте, как же он служил в очистке?}, и кто тот Швондер, что назначил её на
такую высокую должность — единственное, что приходит на ум
%%%cit_comment
%%%cit_title
\citTitle{Откуда у Юлии Мендель столько глубоких познаний о Советском Союзе?}, 
Александр Скубченко, strana.ua, 08.07.2021
%%%endcit


%%%cit
%%%cit_head
%%%cit_pic
%%%cit_text
Вплив совєтизованої масової свідомості населення на результати виборів 2019
року.  Нинішню індивідуальну й національну ідентичність українського населення
спотворило тривале перебування в складі комуністичної імперії.  Десятиліття
існування в системі постійної брехні, підміни понять і сфальшованих смислів не
минули марно. З лабет імперії українська спільнота вийшли зі значними
ментальними ушкодженнями.  \emph{Совєтськість} – тяжка хвороба, яка нині виявляє й
ознаки спадковості. Позбутися її нелегко – і це яскраво засвідчують останні
президентські й парламентські вибори.  В особі Володимира Зеленського наш
електорат обрав, попри його молодість, людину вчорашнього дня, людину з
\emph{совєтського} минулого. Про це свідчить і стиль його діяльності, і його ставлення
до Росії та Володимира Путіна, і схильність до авторитарного типу управління
%%%cit_comment
%%%cit_title
\citTitle{Українська мова в культурному просторі держави. Протистояння триває}, 
Лариса Масенко, www.radiosvoboda.org, 11.07.2021
%%%endcit

%%%cit
%%%cit_head
%%%cit_pic
%%%cit_text
В этот день 1885 года появился на свет Борис Збарский. Ключевой организатор
бальзамирования тела «вождя мирового пролетариата», в дальнейшем на протяжении
нескольких десятилетий следивший за его состоянием. Это позволило Збарскому
войти в \emph{советскую элиту}, но не смогло полностью избавить от репрессий
%%%cit_comment
%%%cit_title
\citTitle{День в истории. 26 июля: в Каменце-Подольском родился хранитель тела Ленина}, 
, ukraina.ru, 26.07.2021
%%%endcit

%%%cit
%%%cit_head
%%%cit_pic
\ifcmt
  pic https://strana.news/img/forall/u/0/0/%D0%A1%D0%BD%D0%B8%D0%BC%D0%BE%D0%BA_%D1%8D%D0%BA%D1%80%D0%B0%D0%BD%D0%B0_2021-10-27_%D0%B2_15.03_.50_.png
  @width 0.4
\fi
%%%cit_text
Эксперты призывают не сравнивать ситуацию с \emph{СССР}.  "Мы привыкли
ассоциировать пустые полки с \emph{Советским Союзом}. Но сегодняшние пустые
полки стали результатом максимально оптимизированной гипер-глобализированной
капиталистической модели, заточенной под доставку точно в срок", - пишет
Financial Times.  Между тем в Америке, как и в Британии, владельцы магазинов
прибегают к различным ухищрениям, чтобы скрыть пустые полки в торговых залах. В
части магазинов пустоты прикрывают картинками еды, в одном супермаркете вдоль
пустых полок поставили рядами яркие складные садовые стулья.  "Мы на краю
пропасти. Новый аспект этой загадки - притворяться, что у нас все хорошо, -
вызывает ужас", - написал автор видео со складными стульями в Twitter
%%%cit_comment
%%%cit_title
\citTitle{Картинки вместо еды. Как в Британии и США возник дефицит товаров и почему там вспоминают СССР}, 
Александра Харченко, strana.news, 27.10.2021
%%%endcit

%%%cit
%%%cit_head
%%%cit_pic
%%%cit_text
\emph{СССР} получил самое мощное в мире ядерное устройство, но едва ли мог применить
его против какой-либо из стран. По мнению одного из руководителей \emph{советского}
атомного проекта Юлия Харитона, громоздкую конструкцию нельзя было нормально
разместить в самолете, чтобы потом сбросить на землю. Академик считал
«Царь-бомбу» в большей степени демонстрацией, чем началом применения мощных
ядерных взрывных устройств. В том же ключе рассуждал Сахаров: «После испытания
"большого" изделия меня беспокоило, что для него не существует хорошего
носителя (бомбардировщики не в счет, их легко сбить) — то есть в военном смысле
мы работали впустую. Я решил, что таким носителем может явиться большая
торпеда, запускаемая с подводной лодки. Конечно, разрушение портов — как
надводным взрывом "выскочившей" из воды торпеды со 100-мегатонным зарядом, так
и подводным взрывом — неизбежно сопряжено с очень большими человеческими
жертвами»
%%%cit_comment
%%%cit_title
\citTitle{«Дом будто ножом срезало» 60 лет назад СССР взорвал «Царь-бомбу» — самую мощную в истории. Что помнят о взрыве очевидцы?: Общество: Россия: Lenta.ru}, Дмитрий Окунев, lenta.ru, 30.10.2021
%%%endcit
