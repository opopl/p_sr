% vim: keymap=russian-jcukenwin
%%beginhead 
 
%%file 28_01_2021.fb.soshnikov_andrej.1.recenzia_romanova_bilchenko_egiptologia
%%parent 28_01_2021
 
%%url https://www.facebook.com/permalink.php?story_fbid=2853986724920335&id=100009271069833
 
%%author 
%%author_id soshnikov_andrej
%%author_url 
 
%%tags bilchenko_evgenia,egiptologia,recenzia,ukraina
%%title Мої друзі прислали мені своєрідну рецензію єгиптолога Олени Романової
 
%%endhead 
 
\subsection{Мої друзі прислали мені своєрідну рецензію єгиптолога Олени Романової}
\label{sec:28_01_2021.fb.soshnikov_andrej.1.recenzia_romanova_bilchenko_egiptologia}
\Purl{https://www.facebook.com/permalink.php?story_fbid=2853986724920335&id=100009271069833}
\ifcmt
 author_begin
   author_id soshnikov_andrej
 author_end
\fi

Мої друзі прислали мені своєрідну рецензію єгиптолога Олени Романової
\url{https://www.facebook.com/olena.romanova.39/posts/2878976852374726} на
лекцію доктора культурології Євгенії Більченко про Стародавній Єгипет 

\url{https://www.youtube.com/watch?v=On014QVrshY}

З Оленою Романовою особисто я не знайомий, але, як музеєзнавець, торкаючись
теми єгипетської спадщини в музеях України, часто користувався її статтями по
цій темі, і вони справили на мене найпозитивніше враження. Я не єгиптолог, але
за освітою досить добре знаю культуру Стародавнього Єгипту, навіть трохи читаю
староєгипетською мовою  (враховуючи в принципі специфіку її передачі в
ієрогліфах). Чесно кажучи, (каюсь!) про кукурудзу, що тихо шелестить  на
єгипетських ланах і чорношкірих африканських гіксосів з 25 династії я не
повірив. 

І вирішив знайти цю лекцію на ютубі. Знайшов. Все те, що я почув (до кінця
прослухати не вистачило сил - видихався) - це суцільний треш. Вся лекція
складається з цілком художніх фантазій Євгенії, які побудовані під натягування
давно віджилих цивілізацій на сучасний світ, ну і на ідеалізацію
\enquote{деспотії} як нібито соціально гуманої спільноти. Євгенії як
культурологу, слід було б розуміти, що побут і світогляд народів давнини
приблизно так само співвідносяться з нашими, як побут і світогляд інопланетян,
тим-то далека старовина і є для нас і цікавою, унікальною, що абсолютно не
схожа на нас. 

А Євгенія пробує втягувати у лекцію про старовину свою сучасність (як вона їх
бачить) для критики цієї сучасності. Так, вона підводить студентів до думки, що
давньоєгипетська «деспотія» є більш соціально гуманнішою того суспільства, в
якому ми живемо. 

Причому доводи її вигадані від слова повністю. Наприклад, з чого вона взяла, що
єгипетські лікарі лікували когось безкоштовно? А самі вони тоді на які кошти
існували? Потім, що у розумінні Євгенії є \enquote{безкоштовно}?

Адже справа в тому, що в Стародавньому Єгипті довгий час взагалі не існувало
грошей в нашому розумінні й оплата видавалася пайком з продуктів, тканин тощо,
а дебен, що з'явився досить пізно - це по суті не гроші, а міра золота чи
срібла. Тому знову-таки - які 75 ще доларів за день бальзамування! 

Навздогін до вже поміченого поважною Оленою Романовою хотілося б поцікавитися у
Євгенії, як їй прийшло взагалі в голову на повному серйозі передавати студентам
анекдот Геродота про те, як фараони примушували дочок займатися проституцією,
нести повну нісенітницю про те, що нубийские фараони не змогли побудувати
піраміди (це мабуть ті, які чудово показують туристам в сучасному Судані?),
американська кукурудза на ланах стародавных єгиптян і чорношкірі гіксоси 25
династії (змішались до купи азіати-гіксоси з території сучасної Палестини, що
утворили 15 династію близько 1700 р. до н. е. та нубійци-негроїди з території
сучасного Судану, що утворили 25 династію близько 700 р. до н. е.) мене вбили
морально взагалі, так само, як і оповідання про красуню-хірургиню Пісочет
(насправді Песешет акушерка, якщо її можно взагалі назвати медиком - враховуючи
справжній доволі низький рівень староєгипетського лікування їй більше підійде
\enquote{повивальниця}) з медичним трактатом Ебуха - взагалі-то Еббота та з
операціями єгиптян на серці (як це, може маються на увазі трупи? :) ) -
насправді єгипетскі лікарі всього лише визначали стан хворого за ударами серця,
солоні озера переплутані з содовим розчином для бальзамування, та й душа і тіло
у єгиптян - це не одне і теж, є складна градація між анхом, Ба, Ка. Все, що я
почув, може було б доречно на якийсь сходці, але не на лекції - багато емоцій,
суцільні домисли, нуль компетенції.

