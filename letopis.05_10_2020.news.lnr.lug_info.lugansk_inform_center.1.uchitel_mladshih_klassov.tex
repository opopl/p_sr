% vim: keymap=russian-jcukenwin
%%beginhead 
 
%%file 05_10_2020.news.lnr.lug_info.lugansk_inform_center.1.uchitel_mladshih_klassov
%%parent 05_10_2020
 
%%url http://lug-info.com/news/one/uchitel-mladshikh-klassov-iz-perevalska-pobedila-v-konkurse-pedagog-goda-luganschiny-foto-60650
 
%%author ЛуганскИнформЦентр
%%author_id lugansk_inform_center
%%author_url 
 
%%tags lnr
%%title Учитель младших классов из Перевальска победила в конкурсе "Педагог года Луганщины" (ФОТО)
 
%%endhead 
 
\subsection{Учитель младших классов из Перевальска победила в конкурсе \enquote{Педагог года Луганщины} (ФОТО)}
\label{sec:05_10_2020.news.lnr.lug_info.lugansk_inform_center.1.uchitel_mladshih_klassov}
\Purl{http://lug-info.com/news/one/uchitel-mladshikh-klassov-iz-perevalska-pobedila-v-konkurse-pedagog-goda-luganschiny-foto-60650}
\ifcmt
	author_begin
   author_id lugansk_inform_center
	author_end
\fi

\index[names.rus]{Дыченко, Инна!ЛНР!Учитель Года, 05.10.2020}

\ifcmt
tab_begin cols=3
	caption Учитель младших классов из Перевальска победила в конкурсе \enquote{Педагог года Луганщины} Фото: Марина Сулименко / ЛИЦ
pic http://img.lug-info.com/cache/d/5/(165)_15.jpg/1000wm.jpg
pic http://img.lug-info.com/cache/e/9/(199)_14.jpg/1000wm.jpg
pic http://img.lug-info.com/cache/9/4/(210)_13.jpg/1000wm.jpg
pic http://img.lug-info.com/cache/6/0/(305)_11.jpg/1000wm.jpg
pic http://img.lug-info.com/cache/1/3/(538)_6.jpg/1000wm.jpg
tab_end
\fi

Церемония награждения победителей и призеров республиканского конкурса \enquote{Педагог
года Луганщины} прошла в здании Федерации профсоюзов ЛНР. Об этом с места
события передает корреспондент ЛИЦ.

Третий (заключительный) этап республиканского конкурса стартовал в Луганске 15
сентября. Организаторами конкурса, целью которого являлось раскрытие
творческого потенциала педагогов, создание условий для их самореализации и
повышение престижа педагогической профессии в Республике, выступили
Министерство образования и науки ЛНР и Республиканский центр развития
образования.

\enquote{В этот праздничный день приятно подвести итоги конкурса 
\enquote{Педагог года Луганщины}. К моей радости педагогом года в этом году стал человек из плеяды
молодых педагогов. Несмотря на то, что у этого педагога за плечами всего лишь
несколько лет педагогического стажа, но, тем не менее, этот человек снискал
заслуженного уважения среди родителей, детей и конкурсного жюри}, - сказал
министр образования и науки ЛНР Сергей Цемкало.

Он вручил нагрудный знак \enquote{Почетный работник образования ЛНР}, диплом и знак
\enquote{ Победитель республиканского конкурса \enquote{Педагог года Луганщины} } учителю
начальных классов Перевальской гимназии № 1 Ксении Збицкой.

В номинации \enquote{Учитель года} победу одержала учитель географии Луганского
общеобразовательного учреждения – специализированной школы № 54 Инна Дыченко,
на втором месте – учитель английского языка Бобриковской средней школы имени
Татаренко Юлия Андриенко, на третьем – учитель начальных классов Ровеньковской
специализированной школы № 3 имени Бурды Ольга Гузикова.

В номинации \enquote{Воспитатель года} победила воспитатель Луганского дошкольного
образовательного учреждения – ясли-сада комбинированного вида № 117 "Ладушки"
Елена Ильченко, второе место заняла воспитатель Стахановского ясли-сада
комбинированного вида \enquote{Сказка} Анна Калиниченко, третье – воспитатель ясли-сада
\enquote{Юбилейный} из Перевальского района Татьяна Батчикова.

В номинации \enquote{Педагогический дебют} лучшей признана Збицкая, на втором месте –
учитель географии Лутугинского учебно-воспитательного комплекса школы-лицея
Анна Чванова, на третьем – учитель начальных классов Луганского
общеобразовательного учреждения – средней общеобразовательной школы № 38 имени
маршала Ворошилова Анастасия Салманова.

В номинации \enquote{Практический психолог} победу одержала практический психолог
Свердловского лицея № 1 имени сестер-подпольщиц Бабарицких Елена Бородина,
второе место заняла практический психолог Луганского общеобразовательного
учреждения – учебно-реабилитационного центра № 135 Ольга Плямина, третье –
практический психолог Алчевской гимназии имени Героя Советского Союза Липовенко
Юлия Панина.

\enquote{Для нас сегодня самой важной задачей является защита социально-экономических
интересов работников образования, мы все для этого делаем и будем делать. Я
думаю, в ближайшее время мы выйдем на то, чтобы работники образовательной сферы
стали элитой в нашем обществе}, - сказал заместитель председателя Народного
Совета ЛНР, председатель Республиканского профсоюза работников образования и
науки ЛНР Олег Коваль.

Он вручил отличившимся учителям почетные знаки и грамоты и денежные сертификаты
педагогам, занявшим призовые места в республиканском конкурсе.

\enquote{Мечта стать учителем у меня была еще со второго класса, мама моя была уверена
кем я стану. Участием в конкурсе я вышла за свои рамки, у меня получилось, я
думаю, это будет не первая моя победа}, - рассказала абсолютный победитель
конкурса \enquote{Педагог года Луганщины} Збицкая.

День работника образования установлен указом главы ЛНР и, начиная с 2017 года, отмечается в Республике ежегодно 5 октября. Ранее День работника образования отмечался в первое воскресенье октября.

\textbf{ЛуганскИнформЦентр — 05 октября — Луганск}

