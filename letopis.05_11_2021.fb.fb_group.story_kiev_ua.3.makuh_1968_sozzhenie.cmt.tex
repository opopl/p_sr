% vim: keymap=russian-jcukenwin
%%beginhead 
 
%%file 05_11_2021.fb.fb_group.story_kiev_ua.3.makuh_1968_sozzhenie.cmt
%%parent 05_11_2021.fb.fb_group.story_kiev_ua.3.makuh_1968_sozzhenie
 
%%url 
 
%%author_id 
%%date 
 
%%tags 
%%title 
 
%%endhead 
\subsubsection{Коментарі}

\begin{itemize} % {
\iusr{Inna Maistrouk}
Да и в Праге были жертвы от самосoжжения ..окупация...

\iusr{Олена Кучугурна}
Царство Небесне і вічна пам'ять видатному українцю патроіоту Василю Макуху!

\iusr{Алекс Борисенко}
Читаєш і ком у горлі... Неймовірна любов і самопожертва до/заради своєї Країни!
Але... ж були діти, і дружина, які ще і постраждали...
Чи варто так вмирати? Чи не краще розклеювати листівки, вести просвітницько-підривну діяльність тощо? Я розумію смерть, коли, наприклад, захищаєш тілом побратимів від кулі, коли з собою забираєш якогось нелюда, підірвавшись гранатою...
Р.s. сподіваюсь у всіх вистачить почути, а не накинутися з лайками...

\begin{itemize} % {
\iusr{Ковальская Татьяна}
\textbf{Алекс Борисенко} А по іншому не можна було! Для тих, хто розклеювал листівки а ще щось , бросали за грати та замовчуванням ці вчинки...

\begin{itemize} % {
\iusr{Алекс Борисенко}
\textbf{Ковальская Татьяна} але ж і про цю самопожертву практично ніхто тоді не знав, бо все засекретили.

\iusr{Ковальская Татьяна}
\textbf{Алекс Борисенко} знаю! Теж багато бачила та отримала від леніносталінохрущева горбачого влади!
\end{itemize} % }

\end{itemize} % }

\iusr{Tatiana Desiatka}
Він наш Герой.

\ifcmt
  ig https://scontent-frx5-2.xx.fbcdn.net/v/t39.1997-6/s168x128/851586_126361977548599_392107290_n.png?_nc_cat=1&ccb=1-5&_nc_sid=ac3552&_nc_ohc=Jb2FQWwoJ_oAX8gHPrT&_nc_ht=scontent-frx5-2.xx&oh=1416112ae026c151ccf6876ecb53d24a&oe=618B4571
  @width 0.2
\fi

\iusr{валерий волошин}

Були герої-патріоти, є і будуть, та на привеликий жаль не меньше в
покидьків-зрадників України, які себе почувають героями, Бо Таку владу обирає
собі народ. То може варто просипатись?...

\iusr{Влад Васькин}
В чём подвиг ?

\begin{itemize} % {
\iusr{Darya Chertkova}
\textbf{Vladislav Vaskin} если не доходит , то вам уже никто не объяснит....

\iusr{Влад Васькин}
\textbf{Darya Chertkova} вы готовы к такому "подвигу"? Одобряете?

\iusr{Михайло Наместник}
\textbf{Влад Васькин} рабам этого не понять.
\end{itemize} % }

\iusr{Сергей Пацкин}
Все правильно сделал!

\begin{itemize} % {
\iusr{Влад Васькин}
\textbf{Сергей Пацкин} чаще бы так.. У "патриотов", да ?

\iusr{Михайло Наместник}
\textbf{Влад Васькин} что ты здесь забыл, Васькин? Гуляй себе за поребриком
\end{itemize} % }

\iusr{Янина Ромова}
Распорядиться так своей жизнью, это безумие.

\begin{itemize} % {
\iusr{Сергей Пацкин}
\textbf{Янина Ромова} А он и был безумен, человек с диагнозом, о чем автор стыдливо умалчивает.

\begin{itemize} % {
\iusr{Tatyana Mazerati}
\textbf{Сергей Пацкин} ujas

\iusr{Maksim Pestun}
\textbf{Сергей Пацкин} у вас есть данные о том, что Макуха был психически нездоров? Есть справки с диагнозом или другие документы, на которые вы ссылаетесь в своем утверждении. Или вы считаете любовь к Родине психическим отклонением по определению?

\iusr{Сергей Пацкин}
\textbf{Maksim Pestun} Естественно, уже многократно разбирали.

\iusr{Михайло Наместник}
\textbf{Сергей Пацкин} -бот
\end{itemize} % }

\iusr{валерий волошин}
\textbf{Янина Ромова} главное вкусно пожрать и посмотреть сериал про голобородька. Да?

\iusr{Янина Ромова}
\textbf{валерий волошин} это предел ваших мечтаний?

\end{itemize} % }

\iusr{Мелания Буткова}
Жаль человека, даже 5сли он иди0т

\iusr{Ольга Гавриш}
Оставил детей без отца - это не правильно. А за мужество - низкий поклон ему!

\iusr{Maksim Pestun}

Человек не мог выдержать несправедливость и отдал самое дорогое, что у него
было, свою жизнь. Конечно, можно было идти и с оружием в руках бороться с
системой, как это делали проклинаемые до сих пор пропагандой захватчика
повстанцы во всех республиках под гнетом ссср. Но он выбрал свой страшный путь.

Те, кто считает его больным, никогда не поймут, что для кого-то есть вещи
поважнее их сытого мирка. Это те, кто приспосабливается к любой власти, к любым
унижениям, лишь бы их не трогали. Они и в конвоиры, и в полицаи, и в
заградотряды шли без зазрений совести

\iusr{Tatyana Mazerati}

Obyasnite mne pochemu v Ukraine, a ne na Ukraine, kak ranshe i pochemu etot vopros stal ocen delikatnim?

\begin{itemize} % {
\iusr{Вадим Горбов}
\textbf{Tatyana Mazerati} маркер как Шиболет в Библии

\begin{itemize} % {
\iusr{Tatyana Mazerati}
\textbf{Вадим Горбов} характерную речевую особенность, по которой можно опознать группу-vi eto imeli v vidu?

\iusr{Вадим Горбов}
\textbf{Tatyana Mazerati} типа того

\iusr{Tatyana Mazerati}
\textbf{Вадим Горбов} spasibo, te, chto davno uehali prodoljayut govorit s predlogom na

\iusr{Вадим Горбов}
\textbf{Tatyana Mazerati} так и здесь половина граждан)

\iusr{Tatyana Mazerati}
\textbf{Вадим Горбов} eto ya ponimayu klyuchevoy vopros ,liniya razdela separatistov i ostalnih.

\iusr{Вадим Горбов}
\textbf{Tatyana Mazerati} нет. Ключевой: чей Крым?
\end{itemize} % }

\iusr{Maksim Pestun}
потому что "в Украину" можно приехать. В гости, например. А "на Украину" можно только напасть...

\begin{itemize} % {
\iusr{Tatyana Mazerati}
\textbf{Maksim Pestun} tut uje kakaya to filosofiya mne neponyatnaya

\iusr{Maksim Pestun}
"На" употребляется обычно в отношении субъекта, а не объекта. На Россию, на Белорусь, На Канаду, на Францию, на Австралию... не очень звучит, правда?...

\iusr{Яр Мик}
\textbf{Maksim Pestun} ,, як їхав милий на Україну дала хустину сльози втирать,, . ,, ..серед степу широкого на Вкраїні милій..,, і пісня і Кобзар є ще багато прикладів. А це ,,в,, ,,на,, вилізло з кремля.

\iusr{Maksim Pestun}
\textbf{Яр Мик} это на каком языке написано?...
\end{itemize} % }

\iusr{Darya Chertkova}
\textbf{Tatyana Mazerati} учите орфографию, посмотрите правила .

\iusr{Tatyana Mazerati}
\textbf{Darya Chertkova} u menya s etim nikogda ne bilo problem, vsegda drugih ispravlyala

\iusr{Volodymyr Nekrasov}
\textbf{Tatyana Mazerati} Тетяно, НА Україні - це від Сяну до Дону і до Кубані, а В Україні - це в адмінстративних межах держави Україна.

\iusr{Tatyana Mazerati}
a chto eto Syan?

\end{itemize} % }

\iusr{Арт Юрковская}
Хороший пример для патриотов. Побольше бы таких людей сейчас. Батько наш Бандееееерааааа....

\iusr{Иван Тесленко}
«Чехословакия» оценила?

\iusr{Maksim Pestun}

Пост - xороший тест для определения человеческих ценностей и общего уровня развития

\begin{itemize} % {
\iusr{Александр Асатуров}
\textbf{Maksim Pestun} Максим, прямой вопрос. Этот пост для КИ?

\iusr{Tatyana Mazerati}
nado otdat soziologam na analiz mnogie iz KI postov

\iusr{Maksim Pestun}
Вопрос сложный. КИ не площадка для политических баталий. Но с другой стороны, это 100\% киевская история.

\begin{itemize} % {
\iusr{Александр Асатуров}
\textbf{Maksim Pestun} 

знаю достаточное количество авторов, которые перестали публиковать свои 99\%
истории из-за вот таких 100\%. И не лукавьте пож. Все архивы охранки, кгб, сбу
тоже киевские. Публикуем все дерьмо? Насколько я знаком с основателем группы,
концепт у нее совсем другой.

\iusr{Maksim Pestun}

Я не принимал участие в публикации этого поста, но не считаю кого-то вправе
определять, что есть хорошо (дерьмо по вашему), а что плохо. Есть правила
страницы. Если публикация не нарушает их, то я не вижу причин ее не
публиковать. Каждый киевлянин имеет право на свои воспоминания и свою точку
зрения, высказанную в культурной форме! Ностальгическая любовь части читателей
к своей юности, которая дивным образом трансформируется в любовь к власти того
времени, не дает им право осуждать тех, кто относится к покойному строю иначе.
По крайней мере, я как один из модераторов группы, считаю именно так.

\iusr{Александр Асатуров}
\textbf{Maksim Pestun} 

абсолютно правы в части имеет право, любовь нелюбовь к покойному
строю.... миллион групп подобного толка как и личных страниц.... там и
публикуйтесь.... а вот насчёт кого то определять что хорошо и что плохо тут
явная заминка... это не цензорнет... это киевские истории.... а не персональный
политический блог Полуэктова

\end{itemize} % }

\iusr{Сергей Пацкин}
\textbf{Maksim Pestun} 

Макуха был психически нездоровый человек(почему Полуэктов умолчал?). Делать из
такого человека символ неэтично, а использовать как повод для плевка в прошлое,
время куда более светлое для подавляющего большинства - мерзость. Автор текста
- мразь.

\begin{itemize} % {
\iusr{Maksim Pestun}
\textbf{Сергей Пацкин} я придерживаюсь прямо противоположного мнения !

\iusr{Maksim Pestun}
\textbf{Сергей Пацкин} на мой взгляд, плевок в прошлое, это героизация самого подлого и циничного периода жизни нашей страны. Какая ностальгия по молодости бы не терзала...

\iusr{Сергей Пацкин}
\textbf{Maksim Pestun} Вы так говорите, будто сейчас иначе. Уровень цинизма и манипулирования смыслами ничуть не меньше, для подавляющего большинства, для простых Макух.

\iusr{Maksim Pestun}
\textbf{Сергей Пацкин} да, но сейчас не расстреливают тех, кто это понимает! Пока, по крайней мере...
\end{itemize} % }

\iusr{Ludmila Vorobiova}

Срез общества. Какая разница - это большинство. Пожрать и по.... ать.
Нравственная убогость.

\end{itemize} % }

\iusr{Maksim Pestun}
\obeycr
Каждый выбирает по себе - женщину, религию, дорогу,
Дьяволу служить или пророку - каждый выбирает по себе.
Каждый выбирает по себе - шпагу для дуэли, меч для битвы.
Слово для любви или молитвы, каждый выбирает по себе.
Каждый выбирает для себя - щит и латы, посох и заплаты,
Меру окончательной расплаты, каждый выбирает для себя.
Юрий Левитанский
\restorecr

\iusr{Людмила Бойко}

Прикро те, що мразі, які знущалися з рідних Василя вже після його смерті
прожили сите життя, виховали «по своїй подобі» негідників-мишебратьєв @igg{fbicon.cry} 

\iusr{Михаил Алексинский}
Вышиванка тоже сгорела? Психически здоровый человек такого сделать не мог.

\begin{itemize} % {
\iusr{Сергій Михайлович}
\textbf{Михаил Алексинский} а ви наскільки психічно здоровий питати за вишиванку ?
\end{itemize} % }

\iusr{Nata Maki}
Справжній патріот і герой

\iusr{Alexey Novozhylov}
Дуже дякую за справжню київську історію!!!  @igg{fbicon.flag.ukraina}{repeat=2}   @igg{fbicon.thumb.up.yellow} 

\iusr{Vladimir Karasovskiy}
Помощь психолога могла спасти человеку жизнь.

\iusr{Klara Mezhebovsky}
A о своей семье он не не подумал: почему они должны были страдать?

\begin{itemize} % {
\iusr{Alexey Novozhylov}
\textbf{Klara Mezhebovsky} мабуть пасіонарною був людиною, а нє жєлудочніком.
\end{itemize} % }

\iusr{Тарас Єрмашов}

У Донецьку в 2000-х рр. було створено музей «Смолоскип», присвячений В. Макуху.
Нині окупанти з РФ повністю знищили всі його матеріали...

На "Ютубі" можна переглянути д/ф "Василь Макух. Смолоскип".

"... Борці за Україну по-різному гинули: у бойових акціях ОУН-УПА, були
розстріляні чекістами у підвалах большевицьких катівень, на шибенецях, доведені
хворобами і голодом до крайнього виснаження. Василь Макух сам собі вибрав
смерть, протестуючи самоспаленням, і тим увійшов у безсмертя. Слава Героєві
України!" (П. Дужий, референт ОУН, письменник).

Василь Стус присвятив акту самопожертви В. Макуха вірш (цей момент показано в
ігровому фільмі про поета "Заборонений"):

\obeycr
Напередодні свята,
коли люди метнулися по крамницях,
виносячи звідти шпроти, смажену рибу,
шинку і горілку з перцем,
якийсь дивак, обутий в модні черевики
(такі тиждень тому були викинули
в універмазі \enquote{Україна} — двадцять два
п’ятдесят з навантаженням — дитячі штанці
вісімнадцятого розміру), облився чортівнею
і підпалив себе.
О, він горів, як порося, смажене примусом,—
налетів на людей, що культурно собі стояли
в черзі за цитринами.
Порозбігалися усі як один:
від нього так несло смаленим —
носа було навернути ніяк.
На щастя узялося кілька міліціонерів,
одразу вкинули його в машину
і помчали в бік Лук’янівки.
А черги ми таки достоялись. Аякже:
що то за святковий стіл без цитрин?
\restorecr

\iusr{Olena Kharkevich}
Перехоплює подих! Вічна слава Героям!

\iusr{Oleksiy Rozhkov}
Герою Слава

\iusr{Георгий Майоренко}

Депутат Лилия Григорович пыталась себя сжечь в Верховной Раде в знак поддержки
Ющенко. Но ей скрутили руки и отобрали спички. Кажется, депутат Кендзер помешал
ее самосожжению.

\iusr{Volodymyr Nekrasov}
Дякую за публікацію!  @igg{fbicon.hands.pray} 

\iusr{Татьяна Горовенко}

Людина була націлена на ідею незалежності та справедливості. Це його героїчний
протест на свавілля імперії. Більше не міг терпіти і змиритись... Вічна пам'ять
героям!!! Вдячна, що опублікували цей факт; я тоді закінчила школу, в душі
засуджувала придушення свободовиявлення Чехословаччини радянською державою, але
не знала про протест Василя Макуха.

\iusr{Денис Гнатюк}

Дякую за допис. Вату аж плющить від такого. Скаженіють коли тюрму народів
показують як є.

\iusr{Шлома Кац}

интересно, сколько сегодня подобных героев суицидников готовых себя спалить при
преступниках у сегодняшней власти.

\end{itemize} % }
