% vim: keymap=russian-jcukenwin
%%beginhead 
 
%%file 22_12_2022.fb.fb_group.mariupol.pre_war.1.dom_gampera.cmt
%%parent 22_12_2022.fb.fb_group.mariupol.pre_war.1.dom_gampera
 
%%url 
 
%%author_id 
%%date 
 
%%tags 
%%title 
 
%%endhead 

\qqSecCmt

\iusr{Вадим Коробка}

Три маріупольські лікарі. Сергій Федорович Гампер (1887р), його батько, Федір
Васильович Гампер (ймовірно, Фрідріх Вільгельмович) Герман Вільгельмович
Гампер, можливо, брат Федора Васильовича – маріупольський міський (городовий)
лікар, надворний радник. Кому з них належав будинок? Невідомо! Десь в
маріупольський газеті (1899-1900 рр.) я зустрічав таке – «бывший дом Гампера».

Був ще Микола Сергійович Гампер, молодий офіцер білої Добровольчої армії, який
загинув 1918 р. Похований в Маріуполі. Його мати – О.А. Гампер, вочевидь вдова
померлого 1911 р. Сергія Гампера.

Колись фейсбучанин (не пам'ятаю хто) писав, що був ще один Гампер, який
з'явився в Маріуполі під час націстської окупації.

\iusr{Вадим Коробка}

Чудові світлини!

\begin{itemize} % {
\iusr{Олена Сугак}
\textbf{Вадим Коробка} дякую. місто у нас чудове... Так хочеться додому...
\end{itemize} % }

\iusr{Viktoria Zaytseva}

А что с домом сейчас? Кто в курсе?

\begin{itemize} % {
\iusr{Олена Сугак}
\textbf{Виктория Зайцева} ранен...

\iusr{Олена Сугак}
\textbf{Виктория Зайцева} только скрин... если не снесут, то может хоть как память, оставить..

\ifcmt
  igc https://scontent-fra5-2.xx.fbcdn.net/v/t39.30808-6/321489466_511972804331449_5398020514335615794_n.jpg?_nc_cat=106&ccb=1-7&_nc_sid=dbeb18&_nc_ohc=XDc37t3ZnDIAX-9mOHT&_nc_ht=scontent-fra5-2.xx&oh=00_AfCcMlvrQWFTUaAFlWpfbcqkkAroYTyo7rlA-U6ewxXi2g&oe=6429A876
	@width 0.4
\fi

\begin{itemize} % {
\iusr{Наталия Павлова}
\textbf{Елена Сугак} и до него добрались

\iusr{Viktoria Zaytseva}
\textbf{Елена Сугак} Теперь только останется на наших фото и в памяти..

\end{itemize} % }
\end{itemize} % }

\ifcmt
  tab_begin cols=3,no_fig,center,separate

     pic https://scontent-frt3-2.xx.fbcdn.net/v/t39.30808-6/321552434_2167408556774552_7410150747692026842_n.jpg?_nc_cat=110&ccb=1-7&_nc_sid=dbeb18&_nc_ohc=hZIUbuDB6-cAX9fuKrU&_nc_ht=scontent-frt3-2.xx&oh=00_AfBe-20txGQxyv28PHTNdlLzbnTB1GZYe7vmsEy2dDvRrA&oe=64295125

		 pic https://scontent-fra3-1.xx.fbcdn.net/v/t39.30808-6/321383444_400611575583736_71550278106052460_n.jpg?_nc_cat=104&ccb=1-7&_nc_sid=dbeb18&_nc_ohc=snS2SDOn9DgAX9jMVis&_nc_ht=scontent-fra3-1.xx&oh=00_AfD7RDmVgVjF5Yp5fpZU2p7bHKdhHVrzf6tq-C5iUISetw&oe=6429DBBD

		 pic https://scontent-fra5-1.xx.fbcdn.net/v/t39.30808-6/321606109_468165225498569_8128426880022282047_n.jpg?_nc_cat=102&ccb=1-7&_nc_sid=dbeb18&_nc_ohc=9oZ7_Hrv1A4AX-5Z9R-&_nc_ht=scontent-fra5-1.xx&oh=00_AfBGhLb6aj0Q1BdGu1vb5Xj7_RDMCBSxQsXoiDZ88jOLpg&oe=6429A185

  tab_end
\fi

\iusr{Andrii Bekhter}

Это, наверное, мое самое любимое здание было.

\begin{itemize} % {
\iusr{Viktoria Zaytseva}
\textbf{Бехтер Андрей} Я тоже его очень любила. Одно из самых интересных сооружений
\end{itemize} % }

