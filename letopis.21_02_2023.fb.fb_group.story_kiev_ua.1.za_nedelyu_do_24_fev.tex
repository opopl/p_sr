%%beginhead 
 
%%file 21_02_2023.fb.fb_group.story_kiev_ua.1.za_nedelyu_do_24_fev
%%parent 21_02_2023
 
%%url https://www.facebook.com/groups/story.kiev.ua/posts/2143937149136408
 
%%author_id fb_group.story_kiev_ua,drobot_tatjana.kiev
%%date 21_02_2023
 
%%tags kiev,24.02.2022,vojna.2022
%%title За неделю до 24 февраля друзья   предлагают сходить в баню
 
%%endhead 

\subsection{За неделю до 24 февраля друзья   предлагают сходить в баню}
\label{sec:21_02_2023.fb.fb_group.story_kiev_ua.1.za_nedelyu_do_24_fev}
 
\Purl{https://www.facebook.com/groups/story.kiev.ua/posts/2143937149136408}
\ifcmt
 author_begin
   author_id fb_group.story_kiev_ua,drobot_tatjana.kiev
 author_end
\fi

За неделю до 24 февраля друзья предлагают сходить в баню. 

Предвкушаем совместную встречу, давно не виделись, купила новое аромамасло для
тела и травяной чай для души. 

В воздухе тревога, которую я отгоняю, как назойливую муху, мысли о сдаче
большого заказа кейтеринга 1 марта, закупка продуктов, уговоры мужа убрать,
наконец, новогоднюю елку. 

За шесть дней до 24 февраля - 

нужно оплатить спортзал на март - таки решила, что буду ходить, четырехчасовой
семинар по клиентскому сервису, уговоры мужа убрать, наконец, новогоднюю елку. 

Соседка с верхнего этажа осторожно поинтересовалась, приготовила ли я тревожный
чемоданчик. 

- Вы шутите?! Какая война? Мы в 21 веке живем...

За пять дней до 24 февраля - наконец, привезли новые подставки для кейтеринга,
целый день не могу наглядеться, предлагаю друзьям в первых числах марта слетать
к морю на уикенд...

В воздухе тревога, которую я отгоняю, как соседку с верхнего этажа. 

За четыре дня до 24 февраля - неожиданно приехала сестра с оккупированного
Луганска. 

- А я ждала тебя на Пасху...

- Таня, война! У нас полная эвакуация.. будет что-то страшное... давай в храм
сходим.. тревожно..

- Ну что ты меня пугаешь! Ну какая война, мы в современном мире, в
цивилизованной стране...

В душе противный комок, от которого я не могу избавиться. 

В салоне разговоры о графике на март, планах на выходные, и о… тревожных
чемоданчиках. 

Вечером после работы я записалась на курсы неотложной медпомощи. Первое занятие
24 февраля, это четверг.

За три дня до 24 февраля - провожаю сестру на дачу под Киевом, встречаемся с
подружкой, гуляем по Подолу, заглядываем на Житний рынок, накупили смаколикив
на предстоящий Великий пост.

В воздухе плотный туман из тревоги, предсказаний и где-то мельком встречаются
заметки о кopoнoвиpyce. Спасибо, уже неактуально.

В ФБ новостные посты - триггеры, все на одну тему. 

За два дня до 24 февраля - 

высаживаю на рассаду семена перца (ну врут же про войну, правда?), проверяю,
что у меня в тревожном рюкзаке, прошу маму собрать необходимые ей лекарства и
документы на всякий случай. 

За день до вторжения 24 февраля - очень высокопоставленная клиентка с одного
министерства на мой вопрос о войне:

- Никакой войны не будет. Им (paшиcтaм) нужна разного рода секретная
информация, мы же в 21 веке живем. Будут ложить наши сервера, бoмбить
компьютерные технологии, то есть так, как мы представляем себе войну с танками,
снарядами и автоматами - так не будет...

В воздухе еще не скомканные надежды, еще целые хрустальные мечты.. 

24 февраля. 5 утра.

Сонные глаза, теплая постель..

- Это гром? Прикрой окно, пожалуйста, это ж надо грозе случиться в феврале..
Совсем природа кукукнулась...

Повторный гром... 

Затряслись окна..

- Таня, нет, это взрыв... Война..

Трехсекундная мысль тревоги, непонимания, гнева, отрицания перерастает во
внутренний трехэтажный мат.

Мечусь по квартире. 

Не замечаю новогоднюю елку.

Третий раз за 15 минут туалет.

В вайбере строчу в рабочей салонной группе: 

- Визажистам выходить сегодня на работу или нет? Салон будет работать?

Параллельно возвращаю клиенту деньги за кейтеринг: 

- Я вернула вам предоплату, проверьте, пожалуйста! Как, где я? Дома, конечно. А
вы? На трассе из Киева в пробке? Все стоит? Ну ангела-хранителя в дорогу!

- Мама, как ты?! Громко?! Мама, я еду за тобой, я успею.. 

Через пару дней 24 февраля.

Год спустя.

Длинный бесконечный разрушающий год. 

Год, который выпал из бытия, который растоптала пpoклятaя pyccкaя opда
беспардонно вторгшись в наши дома в расчете указывать нам, как надо жить.

Десятки тысяч жизней, которые безвозвратно отобрала у украинцев соседняя
страна, жизни, которые никогда не вернуть и не воскресить. 

Сотни тысяч разбитых украинских семей, разломанных судеб.

И миллионы. 

Миллионы украинцев, у которых в сердце слово poccия навсегда будет
ассоциироваться со cмepтью, болью, бpeзгливocтью и пpeзpeниeм.

Я вспоминаю, как за несколько дней до вторжения посольства вывозили своих
сотрудников, предсказывая падение Киева. 

Киев стоит твердо, как скала.

Бeзyмнaя россия до сих пор хочет стереть Украину с лица земли. 

А Украина стоит и будет стоять.

Я искренне обнимаю все страны, кто протянул руку помощи нашей незламнiй нації. 

Поддержка со всего мира укрепила и без того внутренний стержень, дала сил и
стойкости. 

Пусть ничто, ни наши расшатанные эмоции, ни цвет лица у нас, всех оттенков
горгонзоллы, - не вытеснят веру в приближение победы. 

ЗСУ и волонтеры - это наш незыблемый фундамент, наш бесценный ресурс, наша
невероятная мегасила, благодаря которой мы все еще живем в Украине. 

Украина - наша земля, наше небо, солнце, корни, наша жизнь, наш дом.

Я в очередной раз говорю всем, кто меня окружает: пусть с вами будут те, кто
опора, стержень и надежда. Будьте сами опорой, стержнем и надеждой для
окружающих.

Нам еще донатить и донатить, поддерживать, терпеть, ради кого надо вытаскивать
себя за шкирку и топать дальше, ради кого все это имеет смысл. 

Берегите этих людей.

Они на нуле тоже вытаскивают себя за шкирку ради нас.

Избавляйтесь от токсичных людей, субъектов с гнильцой. 

Не деньги, власть, слава и зависть, а человеческое тепло, доверие, взаимопомощь
и сострадание. Прошедший год нам ясно указал на эти реальные ценности. 

Ну и о будущем. 

Чем ближе крах рос.империи, тем безумнее её действия. 

Тем яростней ее агония. 

А мы выстояли.

Мы закалились и сплотились.

Мы идем дальше. 

Звон рюмок будет громким.

Емоційний пост сьогодення заряджений на перемогу.

Єднаймося, українці.

Найтемніша ніч перед світанком.

Усім вам моя щира підтримка.

Обіймаю❤️
