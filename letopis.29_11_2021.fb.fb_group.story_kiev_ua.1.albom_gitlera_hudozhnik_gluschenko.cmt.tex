% vim: keymap=russian-jcukenwin
%%beginhead 
 
%%file 29_11_2021.fb.fb_group.story_kiev_ua.1.albom_gitlera_hudozhnik_gluschenko.cmt
%%parent 29_11_2021.fb.fb_group.story_kiev_ua.1.albom_gitlera_hudozhnik_gluschenko
 
%%url 
 
%%author_id 
%%date 
 
%%tags 
%%title 
 
%%endhead 
\zzSecCmt

\begin{itemize} % {
\iusr{Андрей Кужеев}
Класс! Красиво!

\iusr{Oleg Ivanenko}

\ifcmt
  ig@ name=scr.hands.applause
  @width 0.2
\fi

\iusr{Yuriy Bubnov}

Из служебной характеристики «Яремы»:

«...Во время пребывания во Франции и Германии выполнил ряд сложных заданий по
добыванию научно-технической информации оборонного характера. В результате
советская разведка получила полностью секретные чертежи на 205 видов военной
техники, в частности авиационные моторы для истребителей...»


\iusr{Оксана Ходорич}
Шок! У кожного свої ідеали

\begin{itemize} % {
\iusr{Надежда Владимир Федько}
\textbf{Оксана Ходорич} 

У Вас буде ще більший шок, коли Ви дізнаєтесь, що до вербовки Глущенко був
причетний Олександр Довженко.

У мене була велика публікація про життя М. Глущенко. Там я наводив дані і про
вербовку, і про конкретну роботу Глущенко у Франції, і чому він терміново втік
з Франції...

\begin{itemize} % {
\iusr{Оксана Ходорич}
\textbf{Надежда Владимир Федько} хай Бог береже всіх нас у час, коли акварелі Гітлера - це красиво

\iusr{Надежда Владимир Федько}
\textbf{Оксана Ходорич} Мистецтво стоїть поза часом і політикою!

\iusr{Оксана Ходорич}
\textbf{Надежда Владимир Федько} 

ой, це просто сентенція, тим більше, що ці акварелі не дуже великий витвір
мистецтва як на мене, а так, да, чому б не побачити в каті загубленого митця.
До речі, здається, у Роберта Шеклі (не пам'ятаю) є оповідання про альтернативну
історію, в якій Гітлер став художником.


\iusr{Анна Бабаяга}
\textbf{Надежда Владимир Федько} а можете дати посилання на статтю? Цікаво почитати.

\iusr{Надежда Владимир Федько}
\textbf{Анна Бабаяга} Правилами заборонено давати посилання. Ця стаття 21-го листопада якраз за посилання була відхилена.

\iusr{Надежда Владимир Федько}
\textbf{Оксана Ходорич} 

Гітлер не більший негідник, ніж Троцький, Ленін, Сталін, Хрущов, Каганович,
Рузвельт, Черчілль, та інші європейські лідери.

\iusr{Анна Бабаяга}
\textbf{Надежда Владимир Федько} пробачте, на зауважила на це(

\iusr{Надежда Владимир Федько}
\textbf{Анна Бабаяга} 

Стаття про Глущенко була надрукована в журналах \enquote{Новая медицина
тысячелетия} и \enquote{Организатор}. Сейчас сайтов этих журналов уже нет.

\end{itemize} % }

\iusr{Оксана Ходорич}
\textbf{Надежда Владимир Федько} 

це якась Ваша особиста провокація, я про це не дискутую, до побачення.
Насолоджуйтеся! Дай Бог, Вам горя не знати, заробляючи на подібних
нісенітницях. Сподіваюся, що Вони принесуть Вам задоволення


\iusr{Надежда Владимир Федько}

Прощавайте)

\end{itemize} % }

\iusr{Yuriy Bubnov}

1960 год. Друзья. В центре группы, без головного убора - сын Н.П. Глущенко -
Александр (Шурик, как называли его родные и друзья). Часто собирались у него
дома, слушали пластинки, которые привозил из загранкомандировок Николай
Петрович.

\ifcmt
  ig https://scontent-frx5-1.xx.fbcdn.net/v/t39.30808-6/262244921_1065097324282381_1890951796905082817_n.jpg?_nc_cat=111&ccb=1-5&_nc_sid=dbeb18&_nc_ohc=kw5b7Ij5zyUAX_VLbgc&_nc_ht=scontent-frx5-1.xx&oh=00_AT9bq5PzqXtGyLDljtvA7dFwgWX9tBed3DPUO3ZPhPl77g&oe=61C8D576
  @width 0.4
\fi

\begin{itemize} % {
\iusr{Надежда Владимир Федько}
\textbf{Yuriy Bubnov} Я був з ним знайомий...

\iusr{Леся Фандралюк}
\textbf{Yuriy Bubnov} Стиляги))
\end{itemize} % }

\iusr{Willy Poindexter}

Сэмуэль Моргенштерн, австро-венгерский предприниматель и деловой партнёр
Гитлера в его венский период жизни, купил некоторые из ранних картин Гитлера.
По словам Моргенштерна, Гитлер впервые пришёл к нему в начале 1910-х годов — в
1911 или 1912. Когда Гитлер в первый раз пришёл в магазин Моргенштерна, где тот
торговал стеклом, то якобы предложил ему купить три картины. Моргенштерн вёл
базу данных по своим клиентам, с помощью которой было можно искать покупателей
для ранних картин Гитлера. Установлено, что большинство покупателей его картин
были евреями. Так, важный клиент Моргенштерна, адвокат по имени Йозеф
Фейнгольд, еврей по национальности, купил целую серию картин Гитлера,
изображающую виды старой Вены.


\iusr{Willy Poindexter}

Альбом - даже если он и существовал, не может стоить « миллионы». Сами
картины рейхсканцлера не оченьдороги. По международным меркам. В 2006 году пять
из девятнадцати работ, приписываемых Гитлеру, на аукционе Jefferys (Уэльс) были
приобретены оставшимся неизвестным русским коллекционеро. В 2009 году
аукционный дом Маллок в Шропшире продал пятнадцать картин Гитлера на общую
сумму в 120 тысяч долларов. тогда как на аукционе Ладлоу в Шропшире было
продано тринадцать его картин на общую сумму более 100 тысяч евро. В 2012 году
одна картина Гитлера была продана на аукционе в Словакии за 42300 долларов. 22
июня 2015 года на аукционе в Германии 14 картин, написанных Адольфом Гитлером,
были проданы за 400 000 €[.

\begin{itemize} % {
\iusr{Надежда Владимир Федько}
\textbf{Willy Poindexter} Ви мабуть звернули увагу на ці слова з моїх спогадів:

\enquote{На двох стінах – почорнілі полотна фламандських і нідерландських
живописців, а також німецьких і французьких імпресіоністів; пожовклі від часу
гравюри, виконані в традиціях Клода Лоррена і Джованні Тьєполо. Старовинний
малюнок оголеної натурниці, кілька керамічних плиток українського народного
майстра Бахматюка.}

***

У спогадах про зустрічі з Миколою Петровичем мистецтвознавця з Прибалтики (у
1977 році) ці картини і гравюри також згадуються... Ну, я ділетант, можуть
сказати, що помилився. А вона професійний мистецтвознавець, помилитися не
могла.

Але факт, що картини і гравюри зникли!

\end{itemize} % }

\iusr{Willy Poindexter}

Конечно купить картины рейхсканцлера Германии не очень просто. При этом
специалисты называют объективные трудности, препятствующие продаже работ
Гитлера на аукционах: во-первых, в некоторых европейских странах подобные
продажи могут быть приравнены к пропаганде нацизма, ввиду чего подпадают под
законодательные запреты, во-вторых, владельцами многих аукционных домов
являются евреи, которые отказываются принимать на торги произведения фюрера по
принципиальным соображениям.

\iusr{Willy Poindexter}


\ifcmt
  %tab_begin cols=2,no_fig,center,no_height,width=0.3
  tab_begin cols=2,no_fig,center

     pic https://scontent-frt3-2.xx.fbcdn.net/v/t39.30808-6/260973218_1065184527552146_7825584302520124614_n.jpg?_nc_cat=103&ccb=1-5&_nc_sid=dbeb18&_nc_ohc=GUKs4ddOx6EAX8_6RBz&_nc_ht=scontent-frt3-2.xx&oh=00_AT_adCHwvTIY7SkzGEiJvgpjrNtDU1-hkirC6ECuBl2WIg&oe=61C97997

		 pic https://scontent-frx5-1.xx.fbcdn.net/v/t39.30808-6/261475104_1065184597552139_6230435290236436012_n.jpg?_nc_cat=111&ccb=1-5&_nc_sid=dbeb18&_nc_ohc=if9attXJAxcAX-ZaptE&_nc_ht=scontent-frx5-1.xx&oh=00_AT9L-4Ek6tXArjo0E4Up1D48KHH7iYegJdD2II2Ei_dDuQ&oe=61C87309

  tab_end
\fi

\iusr{Мария Константиновская}
Дааа... страшненькая информация. И фото холодят душу... @igg{fbicon.face.pensive} 

\iusr{Надежда Владимир Федько}

М. П. Глущенко, 1976-й рік...

\ifcmt
  ig https://scontent-frx5-2.xx.fbcdn.net/v/t39.30808-6/262603585_4746288498763904_26952508807377748_n.jpg?_nc_cat=109&ccb=1-5&_nc_sid=dbeb18&_nc_ohc=-tX7WK5Eq4wAX-Kf3Na&_nc_ht=scontent-frx5-2.xx&oh=00_AT-4XBInnDU0tL_mOc4RW48BI-4kRiQb0eCZkw62SGwxMg&oe=61CA3E13
  @width 0.4
\fi

\iusr{Yuriy Bubnov}

Ещё одна фотография, 1960 г. Та же компания, крайний справа - Александр
Глущенко, рядом с ним - мой брат, Валентин. В те времена их называли \enquote{стиляги}.

\ifcmt
  ig https://scontent-frx5-2.xx.fbcdn.net/v/t39.30808-6/262017335_1065174437608003_1517065892392965721_n.jpg?_nc_cat=109&ccb=1-5&_nc_sid=dbeb18&_nc_ohc=NhORQ6eU8UoAX9hfSGT&_nc_ht=scontent-frx5-2.xx&oh=00_AT-aBh0WLrJ4k17JwQMWLUgFB9oNbhYx1D-CwRrNaM344g&oe=61CA1A02
  @width 0.4
\fi

\begin{itemize} % {
\iusr{Надежда Владимир Федько}
Яку музику Ви тоді слухали на \enquote{плитах}, які привозив Микола Петрович?

\begin{itemize} % {
\iusr{Yuriy Bubnov}
\textbf{Надежда Владимир Федько} 

Это было в конце 50-х - начале 60-х. Я был ещё совсем юнцом, и в эту компанию
попал благодаря моему старшему брату Валентину. Преимущественно слушали
джазовую музыку и появившийся недавно рок-н-ролл. Там впервые я увидел
пластинки на 45 об/мин. или, как мы их называли \enquote{сорокопятки}. Мама и отец
Александра, хозяева квартиры в доме на углу Б. Житомирской и Владимирской были
очень гостеприимны. Приходили мы в гости днём и уходили поздним вечером.
Садились за большой стол, пили чай с бутербродами, которыми угощала мама
Александра, выпивали немного вина, которое приносили с собой. Слушали музыку,
танцевали (с нами были две-три девушки), всё было очень прилично. Николай
Петрович сидел в кресле, которое стояло в углу комнаты и некоторое время с
интересом наблюдал за нами, а затем, куда-то уходил. Кстати, с тех пор, я
навсегда полюбил джаз. Многие из этих пластинок (\enquote{плит}), я переписывал на свой
магнитофон \enquote{Днепр-10}, подарок отца в честь окончания школы. Именно он на
фотографии - предмет зависти многих...

\ifcmt
  ig https://scontent-frx5-1.xx.fbcdn.net/v/t39.30808-6/261773515_1065203864271727_1171305254317485854_n.jpg?_nc_cat=105&ccb=1-5&_nc_sid=dbeb18&_nc_ohc=UArrmSMdfNAAX9WIA9_&_nc_ht=scontent-frx5-1.xx&oh=00_AT9T_NLRYwV9ONTBscLliOaB-d9lRG3qq3XkZ39s4OH_kQ&oe=61C90998
  @width 0.4
\fi

\iusr{Надежда Владимир Федько}
\textbf{Yuriy Bubnov} Дякую!

\iusr{Надежда Владимир Федько}
\textbf{Yuriy Bubnov} Пане Юрію, а фотографій Миколи Петровича у Вашому цікавому архіві випадково немає?

\iusr{Yuriy Bubnov}
\textbf{Надежда Владимир Федько} На жаль, немає.

\end{itemize} % }

\end{itemize} % }

\iusr{Владимир Шипов}
Ого, крутая судьба и непростая, но интересная жизнь!!
@igg{fbicon.hands.applause.yellow}  @igg{fbicon.hand.ok}
@igg{fbicon.thumb.up.yellow} 

\begin{itemize} % {
\iusr{Константин Кострыкин}
\textbf{Владимир Шипов} Настоящий патриот. Не в иммиграции остался жить а вернулся в Украину.

\iusr{Надежда Владимир Федько}
\textbf{Константин Кострыкин} 

Глущенко повернувся в СССР не за власним бажанням, а тому, що французька
контррозвідка провела широкомасштабну операцію по викриттю і розгрому
радянської розвідувальної мережі у Франції. Глущенно висів на волосинці від
викриття і арешту.

(Про діяльність радянської розвідки у Франції та розгром агентурної мережі
можна прочитати в книзі:

Вольтон Т. КГБ во Франции: Пер. с фр. - Издательская группа \enquote{Прогресс}, 1993.

Персонально Глущенко там не згадується, але досить детально описано діяльність
резидентури, до якої він належав).

\end{itemize} % }

\iusr{Людмила Даниленко}

Этот альбом экспонировался на выставке работ Николая Петровича в помещени
архива Софии Киевской. Это было лет 40 назад. Я его видела, но там было мало
работ.

\begin{itemize} % {
\iusr{Надежда Владимир Федько}
\textbf{Людмила Даниленко} На першій сторінці альбому був автограф Гітлера: Ніколасу Глущенко Адольф Гітлер (зрозуміло, що німецькою мовою).
\end{itemize} % }

\iusr{Людмила Даниленко}

Глущенко Николай Петрович, его жена Мария Давыдовна и сын Шура похоронены на
центральной аллее Байкового кладбища напротив церкви, в Киеве

\begin{itemize} % {
\iusr{Надежда Владимир Федько}
\textbf{Людмила Даниленко} 

Я знаю. А далі була дуже сумна історія...

Я дружив з художником Олександром Масиком, сином графіка і художника В. І.
Масика. Ми одногодки і часто зустрічалися, як на виставках, так і в Спілці
художників та в його майстерні.

Десь у 2000-х ми зустрілися у фірмі, де я працював (Юг-Контракт), коли він
приїхав поміняти гарантійний фотоапарат, що поламався. Поки сервісний інженер
оформляв заміну по гарантії, я запропонував випити кави в нашій офісній кухні.
За кавою Олександр розповів, що квартиру Глущенко продали... Нові хазяї
викинули на смітник дещо з меблів, старий поламаний мольберт, палітри, фарби та
купу недописаних робіт Миколи Петровича. Олександр зовсім випадково побачив це
художнє майно, що валялося під сміттєзбірником, і забрав собі мольберт,
палітри, фарби та декілька робіт.

\end{itemize} % }

\iusr{Сергей Пацкин}
Веселая байка от киевского алкоголика(Глущенко)

\iusr{Bob Voronkov}
\textbf{Сергей Пацкин} В дзеркало подививсь? ... і по собі судиш ... Шкода тебе, паця ...

\iusr{Bob Voronkov}
Десь на початку 90-х бачив той альбом, тримав в руках ...

\begin{itemize} % {
\iusr{Надежда Владимир Федько}
\textbf{Bob Voronkov} 

А при яких обставинах і як виглядав альбом? Чи бачили ви автограф Гітлера?

(Запитую, бо деякі очевидці, говорять, що автографу на альбомі не було... Також
є розходження щодо кількості факсимільних репродукції в альбомі. Одні говорять
\enquote{багато}, інші - \enquote{мало}.

\iusr{Bob Voronkov}

В СБУ оформлював виставку М. Глущенка ... Були його твори і якісь речі ... І
анотація про його розвіддіяльність... Альбом Гітлера десь з палець товщиною,
акварелі, як такі, і якість репродукцій гарна /ліпша, ніж у нас, в 80-90 -х
було для загалу/, - був вражений цією стороною, бо не знав на той час ... А що
було підписано, вже не пам"ятаю , - 30 літ минуло / ... були ще складні аркуші
, і перекладені сторінки ,,калькою" ... Тож зверніться в СБУ, можливо щось
підкажуть ще ...

\begin{itemize} % {
\iusr{Надежда Владимир Федько}
\textbf{Bob Voronkov} 

Щиро дякую за розповідь! Я працював в архіві СБУ з документами у 2005-2006
роках. Яле якось мені в голову не прийшло поцікавитися альбомом...


\iusr{Надежда Владимир Федько}
\textbf{Bob Voronkov} 

А чи не було це у 1992 році?

Саме тоді з'явилася перша публікація про розвідувальну діяльність Глущенко.
Стаття називалася «Ательє на вулиці Волонтерів». Я тоді не міг зрозуміти причин
розсекречення інформації. Трохи пізніше, здається у 1996-му, коли ми створили
«Журнал історії розвідки і контррозвідки», я запитував про причини
розсекречення діяльності Глущенко в розвідці у генерал-майора Ковтуна, Георгія
Кириловича, який в радянські части був одним із заступників Голови КДБ України
і керував ПГУ... Він сказав, що таке розпорядження поступило з Москви.


\iusr{Bob Voronkov}
\textbf{Nadegda Volodymyr Fedko} 

Точно рік не пам"ятаю ... але було за Марчука ... в музеї СБУ.. але там
експозиції були не для загалу ... Поцікавтесь керівниками музею тих часів /були
3 точно/... можливо щось скажуть ще ...


\iusr{Надежда Владимир Федько}
\textbf{Bob Voronkov} Ще раз дякую Вам! Обов'язково займуся цим... Цілком можливо, що альбом зберігається в архіві СБУ.
Так, нібито випадково, і виясняється доля \enquote{Альбому Гітлера}.
\end{itemize} % }

\end{itemize} % }

\iusr{Надежда Владимир Федько}

Из статьи В. Федько «Триумф и Трагедия Мастера (Часть I. Покорить Париж! И
умереть?!)

«В 1932 году французская полиция начинает третью широкомасштабную операцию по
ликвидации советской разведывательной сети во Франции. Начиная с 1928 года,
полиции удалось идентифицировать более 250 советских агентов, восстановленной
после предыдущих разгромов, разведсети. Руководили сетью агенты ОМС Коминтерна
во Франции (и агенты советской разведки) Клод Лиожье («Филипп») и Исайя Бир
(«Фантомас»). Также были арестованы ответственные работники Французской
компартии Морис Гандуэн и Андрэ Куату. Руководителю подпольного аппарата ФКП
Жаку Дюкло, которого во время следствия заподозрили в причастности к делу,
удалось бежать в Берлин, где он под псевдонимом Лауэр работал со знаменитым
Димитровым в Бюро Западной Европы Коминтерна.

5 декабря 1932 года «Фантомаса» приговорили к трём годам тюрьмы!

Глущенко (оперативный псевдоним «Ярема»), завербованный советской разведкой ещё
в 1926 году, и благополучно переживший предыдущий разгром (Бернштейн – сеть
Креме), живёт в постоянном напряжении, в непрерывном ожидании ареста, в
постоянном страхе за семью и себя. Живёт и работает на грани нервного срыва.
Неизвестность гнетёт его всё больше, в любом посетителе ателье он подозревает
тайного сотрудника полиции, и в 1934 году Художник не выдерживает и уезжает из
неспокойной Франции в Испанию, где много работает, участвует в групповых
выставках. После поездки Глущенко пишет серию пейзажей Болеарских островов и
Майорки. Его лучшие работы того времени – «Корсиканский пейзаж»,
«Провансальский пейзаж», «Ивы», «Южный пейзаж».

Из сообщений парижской резидентуры ГУГБ НКВД в Москву становится ясно, что
Глущенко неоднократно просил разрешения о возвращении на родину. В одном из них
отмечается: «Из письма № 7 параграф 14, Центру: «Ярема» настойчиво требует
позволить ему вернуться на Украину. Мы пытаемся доказать, что для завершения
важной разведывательной акции нужно остаться в Париже ещё на один год. Он очень
неудовлетворен, говорит, что больше двух месяцев не выдержит. Резидент»

Очень показательным есть также то, что в этот критический момент наступает
разрыв отношений Глущенко с Винниченко и со всей украинской эмиграцией. В
последнем письме художнику от 19 июля 1935-го Винниченко отказывает Глущенко в
праве называться украинцем, и упрекает его в «меркантильности» и «отсутствии
убеждений».

А парижская критика и эстетствующая публика, ничего не подозревающие о
разыгрывающейся драме Художника, продолжает восхищаться: «Глущенко с прекрасным
торсом, весь из мускулов – не только художник, а и чемпион по лёгкой атлетике.
Как пловец, он удивляет даже выросших на море болеарцев. Он заплывает так
далеко, что с обратной дорогой расстояние достигает восьми километров. А
вечером надев смокинг, он превращается в светского человека на обедах в
международном обществе в своей гостинице».

\begin{itemize} % {
\iusr{Калиниченко Марина}
\textbf{Надежда Владимир Федько}, 

вы упомянули о сети советской разведки во Франции и сразу вспомнился фильм
\enquote{Красная капелла} - жаль, что этот фильм оказался малоизвестным, очень немногие
люди его посмотрели....

\iusr{Надежда Владимир Федько}
\textbf{Калиниченко Марина} Цікавий фільм... Він є на DVD.

\ifcmt
  ig https://scontent-frx5-1.xx.fbcdn.net/v/t39.30808-6/263112247_4755709654488455_7051341329777038799_n.jpg?_nc_cat=105&ccb=1-5&_nc_sid=dbeb18&_nc_ohc=j11ojEY0gYIAX_FDAIA&_nc_ht=scontent-frx5-1.xx&oh=00_AT_HTBbRGhviGsMvmN8Tc2y9wovaudyjYPPzjM8rLgCA9g&oe=61C90FE0
  @width 0.4
\fi

\end{itemize} % }

\end{itemize} % }
