% vim: keymap=russian-jcukenwin
%%beginhead 
 
%%file slova.teoria
%%parent slova
 
%%url 
 
%%author 
%%author_id 
%%author_url 
 
%%tags 
%%title 
 
%%endhead 
\chapter{Теория}

%%%cit
%%%cit_pic
\ifcmt
tab_begin cols=2
  pic https://avatars.mds.yandex.net/get-zen_doc/4055701/pub_60c4a1a87c4727120350b555_60c4a49792d1652ea77b089c/scale_1200
	width 0.4

	pic https://avatars.mds.yandex.net/get-zen_doc/3006682/pub_60c4a1a87c4727120350b555_60c4a4ea72119574b751162a/scale_1200
	width 0.4
tab_end
\fi
%%%cit_text
Вернёмся лучше к не андертальцам. Вырисовывается очень \emph{стройная теория}.
Вот смотрите. Они якобы вымерли только потому, что имели великолепное свойство
подставляться, безграничное самомнение в маленьком мозгу, в котором уже не было
места для самосохранения и тем более логического мышления. Что очень ярко
прослеживается в прямых, чудом выживших, потомках. Это видно на примерах восьми
последних лет. Перемога-зрада — не останавливающиеся качели. Причём настолько
быстрые в движении туда и обратно, что голова кругом
%%%cit_comment
%%%cit_title
\citTitle{От \enquote{псеродактиля} в \enquote{андертальца} и далее в венец человечества: рассуждения никудышного антрополога}, 
Дмитрий Жук, zen.yandex.ru, 13.06.2021 
%%%endcit

