% vim: keymap=russian-jcukenwin
%%beginhead 
 
%%file 08_03_2021.stz.news.ua.mrpl_city.1.mentalni_osoblyvosti_spravzhnih_zhinok_ukrainy
%%parent 08_03_2021
 
%%url https://mrpl.city/blogs/view/mentalni-osoblivosti-spravzhnih-zhinok-ukraini
 
%%author_id demidko_olga.mariupol,news.ua.mrpl_city
%%date 
 
%%tags 
%%title Ментальні особливості справжніх жінок України
 
%%endhead 
 
\subsection{Ментальні особливості справжніх жінок Ук\hyp{}раїни}
\label{sec:08_03_2021.stz.news.ua.mrpl_city.1.mentalni_osoblyvosti_spravzhnih_zhinok_ukrainy}
 
\Purl{https://mrpl.city/blogs/view/mentalni-osoblivosti-spravzhnih-zhinok-ukraini}
\ifcmt
 author_begin
   author_id demidko_olga.mariupol,news.ua.mrpl_city
 author_end
\fi

\ii{08_03_2021.stz.news.ua.mrpl_city.1.mentalni_osoblyvosti_spravzhnih_zhinok_ukrainy.pic.1}

З нагоди 8 березня хочеться привітати всіх жінок світу, України і Маріуполя і
наголосити, що дівчата кожної країни, безперечно, є особливими. Декілька днів я
перечитувала дослідження етнографів, істориків, соціологів та психологів і була
вражена висновками фахівців. Підсумовуючи всі знайдені матеріали, я вирішила
виокремити риси, які відрізняють українок і маріупольчанок від інших жінок з
усього світу. Звісно, можна говорити про особливості характеру, але не варто
забувати про менталітет (\emph{спосіб мислення, загальна духовна налаштованість,
установка індивіда або соціальної групи}).

\textbf{\emph{1. Сильні, рішучі, сміливі, незалежні.}}

Українкам дійсно притаманні рішучість, здатність не коритися обставинам і брати
долю у свої руки. Це засвідчено історією та фольклором. Прикладом можуть
виступати Анна Ярославна або Роксолана. Досить згадати спостереження
французького інженера Гійома Левассера де Боплана XVII ст., який зафіксував
існування в Україні незвичайної традиції: \emph{дівчата самі сватали хлопців, яких
пригледіли, і вміли переконати батьків свого обранця дати згоду на шлюб}. Це
свідчить про неабияку рішучість українок.

Етнопсихологи пов'язують \enquote{жорсткі} риси характеру українок з тривалим
пануванням матріархату. Насправді звичка наших жінок до незалежності і
здатність самостійно братися за \enquote{нежіночі} справи обумовлена ще й  історичними
обставинами. Чоловіки часто гинули в боях, вмирали на чужині або поверталися
додому покалічені – і весь вантаж домашнього господарства та виховання дітей
лягав саме на жінку. А ще й землю треба було обробляти, врожай збирати, щоб
прогодувати сім'ю; в разі необхідності –  захищати своє до кінця.

\emph{Кандидат психологічних наук, президент Асоціації психологів України Олександр
Вакуленко}, досліджуючи психологічний портрет українських жінок, підкреслив, що
жінки нашої країни дійсно унікальні, адже стільки сили в інших знайти складно.

\emph{\textbf{2. Красиві, жіночні, яскраві}}

Наші жінки красиві. Вони завжди вміли бути такими, з часів \enquote{намиста} і
\enquote{хустинок}, до часів сучасних ділових костюмів. Як зазначила \emph{керівниця
проєкту Profi Fashion Уляна Кулікова} українкам завжди вдавалося залишатися
жіночними і яскравими. Квіткова символіка переважала в українських образах з XV
ст. і сьогодні копіюється знаменитими фешн кутюр'є і кінодівами. Українка – це
справжня квіткова революція!

\textbf{3. Працьовиті, енергійні, завзяті.}

Українським жінкам все під силу: дбати про рідних, налагоджувати побут та,
навіть, виконувати дещо чоловічу роботу. В нашій країні з давніх часів поважали
жінку саме за її працьовитість. \emph{Фольклористка й письменниця, кандидатка
історичних наук Ірина Ігнатенко} провела низку досліджень щодо ролі жінок в
українському суспільстві і прийшла до висновку, що краса була бажаною, але
далеко не вирішальною. Старше покоління орієнтувало своїх дорослих дітей
вибирати пару, яка не цурається роботи і в майбутньому стане гарною
господаркою. За свідченнями етнографів поляки та росіяни звертали свою уваги
частіше на українок саме через їхню працьовитість.

\textbf{\emph{4. Музичні, артистичні, креативні та талановиті}}

У своєму блозі особисто я переконалася наскільки талановиті і унікальні жінки
Маріуполя. Вони і художниці, і балерини, і режисерки, і артистки, і
письменниці, і науковиці, і співачки. При цьому маріупольчанки встигають не
тільки реалізувати свою материнську функцію, але й всьому місту влаштувати
незабутнє свято. Яскравим прикладом є \emph{Тетяна Живолуга}. Звісно, українські жінки
і дуже музичні. Втіленням пісенної душі України стала Маруся Чурай. Як
наголосила Катерина Ластенко, директорка агентства особливих подій Hakuna
Matata, бізнес-тренерка: \enquote{Ми шалено артистичні і креативні. Ми вміємо
створювати простір, де кожен почуватиметься на висоті. Наша сила в тому – що ми
вміємо показати, наскільки унікальний світ навколо нас. І ніколи не забуваємо
про свої етнічні корені}.

\textbf{\emph{5. Турботливі, м'які, ніжні}}

\emph{Українська психологиня Дар'я Селіванова}, що \enquote{українські жінки по-східному
чудові і м'які. Ми з легкістю віддамо чоловікові все права, якщо він цього
попросить. Але в нас завжди житиме велика загадка. Таємниця нашої жіночої душі,
яку ми залишимо для себе – бути завжди бажаними і цікавими...}.

На мою думку, українські жінки дуже універсальні. Ми об'єдну\hyp{}ємо в собі безліч
якостей і намагаємося не тільки зберегти їх але й примножити. З півдня на
північ, від сходу до заходу. Унікальні українські жінки з Києва і Маріуполя,
Бердянська і Львова, Рівного та Запоріжжя, з Полтави та Одеси – ми любимо свою
країну та своє місто і залишаємося українками в душі і серці. Ми можемо бути
зовсім різними, але всіх нас об'єднує не тільки наш менталітет, але й здатність
надихати чоловіків на нові звершення та перемоги.
