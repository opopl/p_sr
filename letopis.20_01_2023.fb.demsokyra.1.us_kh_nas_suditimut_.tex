%%beginhead 
 
%%file 20_01_2023.fb.demsokyra.1.us_kh_nas_suditimut_
%%parent 20_01_2023
 
%%url https://www.facebook.com/sokyra.space/posts/pfbid02W4rVpKhGABfB6cu9nB4yxZUQoctbVQaUdR9bpCNLTJRvL79RNcefgXUWd3N6sPLdl
 
%%author_id demsokyra
%%date 20_01_2023
 
%%tags horvatia,otvetstvennost',future,sud
%%title Усіх нас судитимуть. Але є і гарні новини: наша промова на суді вже готова
 
%%endhead 

\subsection{Усіх нас судитимуть. Але є і гарні новини: наша промова на суді вже готова}
\label{sec:20_01_2023.fb.demsokyra.1.us_kh_nas_suditimut_}

\Purl{https://www.facebook.com/sokyra.space/posts/pfbid02W4rVpKhGABfB6cu9nB4yxZUQoctbVQaUdR9bpCNLTJRvL79RNcefgXUWd3N6sPLdl}
\ifcmt
 author_begin
   author_id demsokyra
 author_end
\fi

Усіх нас судитимуть. Але є і гарні новини: наша промова на суді вже готова.

Після початку війни почав я слухати пісні хорватів. Як і ми, вони важко
виборювали свою незалежність під час своєї Вітчизняної війни, яка зветься
хорватською «Domovinski rat».

Є такий популярний хорватський співак Марко Перкович, більш відомий як Томпсон.
Томпсон — це його позивний. Патлатий хорват-рокер пішов воювати в 1991 році,
узяв у руки Томпсон і прямо на фронті записав кілька кліпів. До речі, згодом
став мемом в американських соцмережах через свій вкрай романтичний вигляд.
Важко зараз впізнати в цьому сивому правому консерваторі (дехто вважає його
навіть фашистом) молодого рокера.

Хорвати відстояли Батьківщину, а згодом і повернули землі під час операції
«Буря». Але генерали, герої війни постали перед судом у Гаазі, бо того вимагали
росія та Сербія. І не змогли тоді хорвати відстояти своїх героїв, а змушені
були піти на це, бо залежали від Європи багато в чому. Й отримали генерали свої
вироки, і сиділи роки в тюрмах, поки все ж не визволили їх: хорвати не
відступилися і продовжили домагатися звільнення своїх героїв, врешті-решт
досягнувши їхнього виправдання.

І після того Томпсон разом з іншим хорватським патріотичним співаком і
політиком Мирославом Шкоро (автор тексту) записали пісню, приспів якої я вільно
перекладу українською:

\obeycr
Судять мене
За те, що своє любив,
Любив понад усе,
Що я боронив
Своє найдорожче.
Судять мене
Вороги, люба моя,
Але не знають,
Що є істина:
Вода глибока.
\restorecr

І коли вони разом співають ту пісню на концертах, то сльози в них на очах, і
слухачі плачуть, бо знають: така правда. Так і кожен із нас постане перед судом
за свої вчинки — хтось за підбиті танки (куди подів БК?), хтось за волонтерку
(куди на картку гроші збирав?), хтось за участь у спеціальних операціях (хто
дозвіл дав?), хтось за закупівлі зброї (яке право мали?), а хтось і за виграші
арбітражів у Стокгольмі, і за те, що майже 3 мільярди доларів опинилися в
українському бюджеті.

Так, судити будуть усіх. І Порошенка, і Зеленського, і Ріфмастера, й Авакова, і
Залужного, і Буданова, і Рєзнікова, і Сирського, і ветеранів, і волонтерів, і
мене, і вас, дорогі читачі. І не за помилки будуть судити — а за те, що чинили
правильні речі, за те, що мали сміливість і діяли не за правилами, на свій
ризик. І лише від нашої поведінки і від вдячності людей залежить, чи дозволимо
ми виносити вироки за те, за що треба нагороджувати.

Та я вірю, що правда своє візьме.

А нам все одно своє робити — підтримувати ЗСУ:

🇺🇦 Гривня, валюта і крипта — \url{https://bit.ly/33x63od}

🇺🇦 Карта — 5169 3351 0047 5223 

🇺🇦 Монобанка — \url{https://send.monobank.ua/jar/4RPbRHJBC4}

🇺🇦 PayPal — donate@sokyra.space 

(Обирайте в «Payment type» опцію «For friends and family»)

\href{https://www.facebook.com/profile.php?id=100013467931016}{Игорь Щедрин}, боєць ЗСУ, до 24.02.2022 член політради Демократичної Сокири
