% vim: keymap=russian-jcukenwin
%%beginhead 
 
%%file slova.bolelschik
%%parent slova
 
%%url 
 
%%author 
%%author_id 
%%author_url 
 
%%tags 
%%title 
 
%%endhead 
\chapter{Болельщик}
\label{sec:slova.bolelschik}

%%%cit
%%%cit_pic
%%%cit_text
«З дитинства за Бельгію» або «Наші хлопці, віримо в вас». Так українські
\emph{вболівальники} підтримували бельгійську команду в матчі проти російської
збірної на груповому етапі Євро-2020. Матч в Санкт-Петербурзі завершився із
рахунком 3:0 на користь Бельгії. Голами відзначився Ромелу Лукаку та Тома
Меньє. Зібрали найцікавіші реакції українців на бельгійську перемогу.
Напередодні матчу у Львові анонсували, що в одній із фан-зон міста наливатимуть
100 бокалів пива за кожен забитий гол Бельгії у ворота Росії. У соціальних
мережах впевнені, після другого забитого гола фан-зона Львова виглядає так,
%%%cit_comment
%%%cit_title
\citTitle{«Слава Бельгії! – Героям слава!» Як українці відреагували на поразку росіян у футболі}, Ольга Модіна, www.radiosvoboda.org, 12.06.2021
%%%endcit

%%%cit
%%%cit_head
%%%cit_pic
\ifcmt
  pic https://strana.ua/img/forall/u/11/33/205690732_4392724120780400_7557187098677902487_n.jpg
	width 0.4
\fi
%%%cit_text
Однако пока нормальные \emph{болельщики} радовались, отдельные индивидуумы выискивали
повод для \enquote{зрады}. И – нашли.  Зачин сделала скандально известная писательница
Лариса Ницой, которая облила грязью автора победного гола сборной Украины
Артема Довбика за то, что тот дал послематчевый комментарий на русском языке. И
она оказалась не одинока в своих претензиях к герою матча со шведами: в
соцсетях ревнительницу \enquote{национальной идентичности} поддержали довольно большое
количество людей
%%%cit_comment
%%%cit_title
\citTitle{Миллионы на украинскую идентичность, зрада на Евро-2020, планы Путина по Украине}, 
, strana.ua, 01.07.2021
%%%endcit
