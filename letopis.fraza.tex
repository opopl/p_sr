% vim: keymap=russian-jcukenwin
%%beginhead 
 
%%file fraza
%%parent videos
 
%%url 
 
%%author 
%%author_id 
%%author_url 
 
%%tags 
%%title 
 
%%endhead 

%https://zen.yandex.ru/media/id/5e9700f8986c916d56eea882/rossiia-rvet-vse-za-chto-by-ne-bralas-zarubejnye-hity-ot-rossiiskih-ispolnitelei-i-kommentarii-inostrancev-607727b28fb1c273057c33f5

%"Россия рвет все, за что бы не бралась"! Зарубежные хиты от российских
%исполнителей и комментарии иностранцев

%https://www.youtube.com/watch?v=EVBCh_jtsTY
%Шарий: России плевать! Они знают, что сделать, чтобы Америка еще пожалела!

Владимир Владимирович просто молчал, и чудо - США передумали посылать эсминцы в
Черное море, Эрдоган взял слова обратно и заявил, что Турция придерживается
невмешательства, Урсула заявила, что Европа исключительно за мирный диалог,
Меркель не приехала на встречу с Зеленским, Макрон послал Зеленского выполнять
минские соглашения...  Сила - в молчании

Да уж воистину когда Путин говорит можно заслушатся, а когда молчит то можно
обосратся...

Когда у тебя есть ядерный гиперзвук, можно и помолчать.

Россия это сумашедший которого выпустили навестить родственников.
Я уже и незнаю хорошо или плохо что Гитлер проиграл.

Амермка это женщина, на неё лучшая реакция это молчать

Просто молчание не работает. Работает мышечная масса, сбитые костяшки и ствол в руке - вкупе с угрожающей мимикой.

не фамилия славит человека, а человек фамилию. Украина плюнула на могилы своих
предков, в первую очередь  на могилы " героев СССР "

Вы правы: НЕ БОЙСЯ ПУТИНА ГОВОРЯЩЕГО--- БОЙСЯ ПУТИНА  МОЛЧАЩЕГО!

как вы,Чудик, не живя в России, можете судить о Путине! Он наш Президент. Мы уважаем его. И не смотря на все трудности и проблемы сегодня в России, нас устраивает наш Президент. А ,если есть наша вражеская оппозиция, то она готовится СМи и ЦРУ США. 
НЕ РАССУЖДАЙТЕ,ЧУДИКИ, О ТОМ, О ЧЕМ ВЫ НЕ ИМЕЕТЕ ПОНЯТИЯ.
ОБРАТИТЕ СВОЕ ВНИМАНИЕ НА СЕБЯ!

Украине раньше чем самому надежному, братскому народу - Сербии предложили
производить вакцину. Это надо было оценить. Нет. Плюнули в лицо. Теперь сами.

Жаль, что таких людей как Анатолий записывают в пророссийски настроенных, во
врагов. А человек говорит просто правду,такая какая она есть.

Жаль что дикторы не по Русски говорят...ни чего не понимаю в их речах....если
так и пойдет дальше смотреть шнягу эту не буду....Шария смотреть буду, человек
здравомыслящий и говорит складно и понятно для меня....Шарий красава!!!

Если бы ещё Шарий перестал утверждать, что Крым аннексирован Россией, а больше
бы прислушивался к мнению крымчан, то я  перед ним сняла бы шляпу.)))

Зелю позвали косточки обглодать после макарона. И указали место у будки и приказали не гавкать.

%https://censor.net/ru/resonance/3255117/morph_van_kostenko_treba_divitisya_pravd_u_vch_vyina_zachepila_duje_neveliku_klkst_narodu_vona_ne_stala

"ВАНЕЧКА! КАКАЯ МОРСКАЯ ПЕХОТА?"
Я Крим, в принципі, не люблю. Ані Крим, ані людей, які там мешкають: через намагання заробити грошей вони себе ведуть абсолютно неадекватно... Але це все одно моя земля. Источник: https://censor.net/ru/r3255117

%КАРАСЬОВ: Росія може піти на війну, щоб не допустити вступу України в НАТО / "ВАЖЛИВЕ" НАШ 15.04.21
%https://www.youtube.com/watch?v=rFnc7hBrmiI

А кто позволит Украине разрабатывать ядерное оружие? Это даже не анекдот. Уровень цирка...

Не, ну Латвия с Литвой в качестве пiдтримки - это конечно грозаааа..., полей и огородов 

Какая там война, нам уже показали в 2014 победный конец, мало что ли, тот же
Хомчак своих солдат бросил на поле боя и сматывался куда глаза глядят, сколько
можно все это терпеть?

Так мы же уже воюем 7 лет... Или нет?

Карасев искажает последовательность событий. Обострение началось с того, что
ВСУ стали возить эшелоны с открытой техникой туда-сюда. А теперь говорят "А нас
за шо?"

В. В. Путин сказал, дернитесь потеряете государственность.  Пацан  зря слова на
ветер не бросает.

Каждая мама хочет, чтобы сын стал ее надеждой, опорой в старости,  Для нее ее
Ваня или Магомед (не суть важно) самый лучший. Так вот не надо эту жизнь
прерывать пулей или осколком...

Стареет,, карась,,! Плывёт, изворачивается, буксует! Ни хрена не понимает
Россию! Ссмех да и только.... 

%https://www.youtube.com/watch?v=R8uEaA9rqBM
%На свете нет ужаснее напасти, 
%чем клоун оказавшийся у власти, 
%полезней труп в одежде короля, 
%чем глупый шут стоящий у руля.

%Что происходит с людьми, которые выбирают лицедеев и аферистов, как метод и
%возможность развития общества и государства?! Неужели уже не важны элементарный
%анализ фактов, жизнедеятельности, перспектива деятельности?! Очнитесь, если вы
%люди, которые думают о своих детях и внуках и окружающей среде! Правительство
%полностью соответствует примитивному, потребительскому отношению народа к
%самому себе! И лишь социальная справедливость и национализация экономики вернёт
%вас к себе!

%Город, в котором никто не выходит. Рэй Брэдбери

%Шире открой глаза, живи так жадно, как будто через десять секунд умрешь.
%Старайся увидеть мир. Он прекраснее любой мечты, созданной на фабрике и
%оплаченной деньгами. Не проси гарантий, не ищи покоя — такого зверя нет на
%свете.  Брэдбери

% from Arestovich Baumeister https://www.youtube.com/watch?v=zYgBnzL9d38&t=173s
%когда родину заносит ее сыновья падают в братские могилы цитата
%Мы запутались в понимании Свободы 38:05 - Что есть Рабство, что есть Свобода,
%и где Земля Обетованная 
%десакрализация власти, окаянные дни Бунина
Расползание по частям Мира, сказочный украинский проект
Кравчук, ссср, онтология, трактовать по-своему,
одна колея с Западом, предали свою онтологию, по западным правилам,
теория конвергенция, андропов, две системы, хельсинки 1975,
когда произошел слом, сменить онтологию, моисей,
позиция отрицания онтологии фараонов
война за справедливость, азербайджан - в едином порыве, 1918 - 1994 знамя,
солдат, алиев, обет молчания пока идет война турция эрдоган - мир несправедлив
достоинство

%https://strana.ua/opinions/329959-ukraina-zamykaet-pervuju-sotnju-stran-v-rejtinhe-svobody-slova.html
прорубь посреди Вселенной

%https://strana.ua/opinions/329938-ukrainskij-pravjashchij-klass-vyvel-unikalnuju-formulu-svobody.html
%Вести себя как хам и свинья, настаивая при этом на своем ангельском достоинстве!

%Формула украинский свободы

%Допустим, "старший брат" с Востока - демонический хам и свинья.

%https://strana.ua/opinions/329921-sut-ofitsialnoj-publichnoj-ukrainskoj-politiki-odnim-slovom-eto-isterichnost.html
Суть публичной украинской политики одним словом - это истеричность
А тремя - это лицемерная театральная истеричность

Арестович - Баумейстер
	разлом между онтологиями, попытка подняться из разлома на равнину,
	выпали из времени - базовых категорий,
	выпали из времени - не находимся в пространства и времени
	гуляй поле Поле
	Нет Сущности
	Какие Свойства Нашей Сущности
	Купина Неопалимая
	МЫ потерялись

Якщо довго бити в одну точку, то стіна трісне". Чому адвокація прав є важливою складовою безпечної реінтеграції 
%https://life.pravda.com.ua/projects/bezpechna-reintehratsiya/2021/04/22/244636/

УТЕШЕНИЕ
 Встретились былинка и пылинка.
 - И никому-то я больше не нужна! – пожаловалась былинка.
 - А я и вовсе никогда никому не нужна была! – вздохнула пылинка.
 Мимо шел человек.
 Услыхал он этот разговор и сказал:
 - У Бога даже лист случайно не падает с дерева. Все в Его воле и нет ничего в этом мире забытого и тем более лишнего.
 Ахнула от неожиданности пылинка.
 Просияла от радости былинка.
 Посмотрели они с уважением на человека.
 И радостные разошлись по своим делам!

Монах Варнава (Санин)

%den_pobedy
Какому-то русскому гениальному человеку, у нас их много, пришла идея о создании
«Бессмертного полка». Это совершенно удивительно. Это шествие забыть
невозможно. Я за свою жизнь видел много чего интересного и очень
впечатлительного, но ничего такого более грандиозного и важного в жизни увидеть
не довелось. Один офицер мне прислал поздравление и сказал, что если бы все
воины, которые погибли во время этой отечественной войны, восстали бы и пришли
на парад, то чтобы им пройти, понадобилось бы две недели, а чтобы пройти всем,
кто погиб за эти страшные четыре года, потребовалось бы полтора месяца
непрестанного движения. И вот, ещё раз повторю, этот воистину крестный ход,
когда внуки, правнуки, дети тех наших героев и не героев, а просто страдавших
людей, вышли с их портретами. Потому что хуже, чем война в том смысле, как
человек страдает, нет ничего на земле. Война – это настоящий дьявольский пир,
когда находятся люди, которые ни во что не ценят человеческую жизнь и готовы
ради своих политических и экономических интересов уничтожать других людей. 
 
И вот это шествие – это посильнее танков и самолётов. Это действительно то, что
внушает огромную надежду. Потому что без всякой специальной организации люди
сами дома наклеили фотографии, перед этим нужно было их как-то увеличить,
многие сделали рамки, некоторые в стекле, насадили на какие-то там палочки,
кругленькие, квадратненькие, некоторые несли в руках. Представляете, сколько
нужно нести! Причём людей там было много и пожилых. И вот, они идут. Какая
цель? Они идут не просто на какую-то демонстрацию, чтобы что-то
продемонстрировать. Нет, это было желание соединиться со своим родом. Потому
что ведь каждый из нас не просто из никуда, из бездны космоса вдруг появился на
земле. Нет, у каждого есть отец и мать, у каждого есть два дедушки и две
бабушки, у каждого есть четыре прадедушки и четыре прабабушки. У многих есть
тёти, дяди. И эта вот людская река людей, которые свою жизнь поставили ни во
что, чтобы справиться с этим злом. 
 
Страшно даже себе представить, хотя такие рассуждения бывают, что если бы немцы
нас победили, мы бы жили лучше. Ну, может быть, с точки зрения немецких свиней,
мы бы жили на таком уровне. Немецкие свиньи, а так ведь они и называли нас:
“руссиш швайн”, они чистенькие, ухоженные, они накормленные, их лечат, они
тучные. Вот в этом смысле мы, может быть, жили бы и лучше. Но свинья, бедная,
она может создать только свиную рульку, но она не может написать песню
«Вставай, страна огромная!», она не может даже построить себе хлев, она ни на
что великое не способна. И вот в этом шествии наш народ явил как раз величие. 
 
Конечно, будет ещё и 80 лет Победы, и вообще, каждый год мы празднуем этот
день, и бесполезно говорить, что его не надо забывать. Такое не забудешь! Если
уж Куликовскую битву мы никак не можем забыть, в которой погибло несколько
десятков тысяч человек, если считать с обеих сторон. И даже спорят ещё до сих
пор учёные, а где, собственно, оно, Куликово поле-то было точно. Но это
совершенно не важно, где было географически, главное, что оно в сердце. Так же
и эта страшная битва. Потому что вряд ли силы зла и все вот эти наши партнёры
когда-нибудь успокоятся. И это противостояние вот этой злобы, зависти, гнусной
подлости, совершенно беспардонной и подлейшей политики. Что, вот эта абсолютно
полная нечестность, несправедливость вдруг куда-то исчезнет и люди вообще
станут хорошие, добрые? Нет, это невозможно. Это всё равно что чтобы Англия
стала по-доброму относиться к России. Только тогда, когда здесь не останется ни
одного человека, только тогда. Это всё равно что дьявол покается и станет
добрым, это невозможно в принципе. Здесь только речь идёт о том, каков будет
экономический баланс. Они всё время считают, смотрят: ну что, пора, не пора,
сумеем, не сумеем или ещё у нас пока кишка тонка или не тонка. Только это их
ограничивает – тонкость кишки. Ну вот, пока не могут. Если смогут, начнут
опять. И надежду нам дают не только те астрономические ресурсы, которые у нас
есть все, кроме людских. Вот этого главного ресурса у нас маловато, потому что
до сих пор у нас мозги наших детей таким устроены образом, что они так же
относятся к своим детям, как к нашим детям относятся те же англичане, французы,
немцы, венгры, поляки, кто угодно. На нас только смотрят, как на источник
наживы. Больше никак это невозможно, чтобы как-то иначе. Ничего не сделаешь. 
 
Но так случилось, что Бог, видимо, за эту нашу возможность отдать жизнь за
добро противу зла и даёт нам пока ещё жизнь на земле. Это главная миссия
России, в которой мы все живём, которую населяют 200 народов – сохранить
человечество. Потому что страшно только подумать, что будет с миром, что будет
с людьми, что будет с детьми, когда вот это зло победит. И так во многих
государствах так называемой европейской цивилизации воспитание детей
превратилось просто в доведение их до полного сумасшествия, к сожалению. Видно,
что те, кто этим занимается – это просто сотрудники самого дьявола, больше
ничьи, потому что нормальному человеку это в голову не придёт, это что-то у
него в голове очень сильно повредилось. 
 
Духовное здоровье в народе ещё в какой-то степени сохраняется. И вот, эту
надежду протягивают нам те, которые отдали свои жизни, своё здоровье, своё
имущество часто, а некоторые отдали вообще ещё и всю свою семью на алтарь
победы. Они нам эту надежду вручают. Ну а нам, ныне живущим, что остаётся?
Сколько есть силы, постараться не посрамить эту надежду. 
 
Протоиерей Димитрий Смирнов

%deti
Потратьте время на детей, хоть пять минут. 
Не убегайте, а присядьте тихо рядом… 
Прижмитесь к волосам, вдохнув любовь... 
Стук сердца мамы, что еще им надо. 

Не убегайте, не спешите, отвернув 
От малышей взор сердца и вниманье. 
Им хочется так много рассказать, 
И получить своих заслуг признанье. 

Не убегайте, сядьте ... Просто так... 
Радар сердечка детского настроен 
На то, чтоб чутко очень уловить 
Любви и нежности хоть крохотные, волны. 

И может быть ...весь час, и день, всю ночь 
В душе ребенка будет петь, как птичка, 
То чувство, что навеки осветит всю жизнь его. 
Он – мамина частичка... 

Моника Масгеди.

%https://vesti.ua/lite/hi-tech/utechku-mozgov-eshhe-mozhno-ostanovit-kak-uchenyj-iz-ukrainy-rabotaet-v-ssha
Чтобы разбросанные в космосе камни собрались в планету, нужна гравитация, а
чтобы народ стал нацией, нужен коллективный интеллект. 

- Российский гимн не будет звучать на Чемпионате мира по фигурному катанию!
- Значит, и никакой другой звучать не будет, - сказали российске фигуристки и заняли весь пьедестал под Первый концерт Чайковского.

Ядерная реакция в разрушенном 4-м энергоблоке ЧАЭС: ученые бьют тревогу
читайте подробнее на сайте "Диалог.UA": https://news.dialog.ua/229070_1620762952

%https://forpost.media/novosti/v-zaporizkykh-shkolakh-mozhut-posylyty-zakhody-bezpeky-pislia-strilianyny-v-inshchiy-kraini.html
%kazan2021
В Запорізьких школах можуть посилити заходи безпеки після стрілянини в інщій країні

ОБЩАЯ ВСЕЧЕЛОВЕЧЕСКАЯ БЕДА 

Источник самых болезненных терзаний человечества – живущие в нас страсти. Диких
зверей, суровую природу и огромные пространства покорил человек. Но зеленая
зависть и глупая обидчивость, но наглая ложь и подлое предательство, но
холодная месть и бесовское злорадство не дают места счастью. 

Во всех часовых поясах и на всех континентах страсти терзают человеческие
сердца, а терзаемые страстями люди мучают друг друга. Людей терзает блуд,
отнимающий у юности радость, а у старости – ум, разрушающий семьи, поселяющий
гниль в костях. Людей манит власть, превращая в лютых врагов тех, кто вчера еще
ел с одной миски. И люди умножают знания, умножая одновременно печаль, а бесов
делают попросту безработными, поскольку сами, по степени изощренности во зле, с
бесами сравниваются. 

Так разве не под силу нам почувствовать эту общую всечеловеческую беду как нашу
личную беду, чтобы молиться: «Господи, Очисти грехи наша. Владыко, прости
беззакония наша, Святый посети и исцели немощи наша, имени Твоего ради»?!

Протоиерей Андрей Ткачёв

Заходите в "Мир Души" в Viber: https://invite.viber.com/?g2=AQBPAk42xobroEg9ff2loYeIncVnx3njZLXYXkHwnrFkud6BYxIAUfVapYbRTzhe

%chudo
На Кубани многие знают об этом чуде, так как много было свидетелей его, а эти свидетели рассказывали о нем своим родственникам и своим знакомым.
Девять монахинь на подводах привезли на расстрел. Поставили их плечом к плечу над обрывом с таким расчетом, чтобы потом не хоронить их, не делать лишней работы.
Комсомольцы приготовились открыть огонь, а монахини громко начали петь какой-то псалом.
И вот тут-то и произошло чудо. Внезапно засиял ослепительный свет, и появились три Ангела. Два Ангела опустились и стали слева и справа от монахинь, а третий Ангел опустился и стал перед комсомольцами, которые сразу все упали на землю и потеряли сознание.
От подвод прибежали ездовые, увидели лежащих, как трупы, комсомольцев, поющих живых и невредимых монахинь и поскакали в село «за подмогой». «Подмога» в растерянности и страхе стала грузить на подводы комсомольцев. – А что нам делать? – спросила одна монахиня. – Разбегайтесь, – последовала команда, которую они и выполнили с молитвами.
Все комсомольцы-конвоиры очень быстро умерли, а вот их секретарь до смерти лежал парализованный (целый год). Он худел с каждым днем все быстрее и быстрее. И вот когда остались только кости да кожа, появилось множество червей. Не успевала мать сгрести с него большую кучу, как они снова появлялись неизвестно откуда.
Мучился он страшно и все кричал, просил: – Молитесь обо мне!
Молиться о нем было некому, так как кругом жили люди неверующие, не знающие молитв.
А вот те, которые могли бы ему помочь, все разбежались.

Иеромонах Трифон. "Чудеса последнего времени".

%https://strana.ua/opinions/332948-bez-oshchushchenija-strakha-nasha-vlast-ne-budet-reshat-problemu-donbassa.html
Радует одно – новые выборы уже не за горами. Будет новый день и новый шанс на перемены.
Их требуют наши сердца.

ЗАЖИВО ОТПЕТЫЙ
Памяти протоиерея-фронтовика Петра Бахтина

Во время Великой Отечественной Петр Бахтин командовал дивизионом «Катюш». Когда шёл бой за Кенигсберг, он был тяжёло ранен, потерял сознание и остался на поле боя.

В части его после боя не обнаруживают. Шлют домой его маме «похоронку»: «Пропал без вести. Пал смертью храбрых».Мама – глубоко верующий человек – сына в храме заочно отпевает.Через некоторое время, после того как в каком-то госпитале он прошёл лечение, он снова обретается в своей части, маме шлют письмо: ваш сын жив.Через некоторое время эта ситуация повторяется, она опять получает «похоронку». Опять его отпевает. Потом снова приходит известие, что произошла ошибка, «ваш сын жив»…

Представляете, какие амплитуды должно было выдержать материнское сердце? Сначала – горе, потеря единственного сына. Потом, оказывается, рано его похоронили, он всё это время был жив. Но не успела мама порадоваться – и опять вынуждена хоронить. А потом оказывается, что опять был живой… Кто способен выдержать такое?
Так состоялись два отпевания Петра Бахтина при жизни, остальные случились уже в советские годы, но об этом чуть позже.

У отца Петра Бахтина было несколько военных орденов и множество медалей. Ордена, как рассказывал он, во время Великой Отечественной было заслужить не так просто. Один орден он получил за оборону высоты, когда был командиром дивизиона «Катюш». С бойцами они удерживал высоту, и уже им нечем было обороняться, закончились боеприпасы. А фашисты ползут со всех сторон. И тогда он решается на последнюю меру: вызывает по рации огонь своих «Катюш», которые стояли в стороне, на себя.

Реактивные снаряды перелопатили всю эту высоту, всю местность. Фрицы почти все были истреблены, но он и горстка его боевых товарищей остались в живых.

За оборону высоты ему вручили первый орден.
Второй орден он получил за взятие одного из населенных пунктов.

После боя он отправился прогуляться по этому поселку – и вдруг заметил прикопанный немецкий танк. «Тигр». Он уже был не на ходу, его использовали как ДОТ, в качестве пулеметной точки.И вдруг из-за угла дома, из-за хаты, которая была рядом с этим «Тигром», выходит один гитлеровец, вооруженный автоматом, за ним – второй, третий, четвёртый… помните, как в фильме «А зори здесь тихие» герои считали фашистских диверсантов: «15… 16… 17…». Петр Бахтин насчитал 14 вооруженных фашистов. И когда они его заметили, почему-то испугались и бросились бежать, хотя у них было явное численное преимущество.

Непонятная реакция. Возможно, они находились ещё под впечатлением проигранного боя, может быть, у них еще не прошла паника. А может быть, они решили, что сейчас из-за угла за этим русским тоже выйдет отряд и подкрепление.Во всяком случае, они бросились от него наутёк. А он, вместо того чтобы побежать в обратную сторону, устремился за ними.

Я его спрашиваю:
– Батюшка, как же вы не боялись? Любой бы из них развернулся, полоснул бы очередью из автомата, и вас бы больше на этом свете не было.
Он объясняет:
– А в конце войны уже перестаешь бояться. Идёшь на поле боя: пули вокруг свистят, снаряды рвутся, осколки летят, думаешь: «А, всё равно убьют», – и даже не пригибаешься.

И вот он догоняет одного отставшего молодого фрица. Хотел его пустить в расход, расстрелять. А тот заплакал, закричал:
– Киндер, киндер!.. – и указывает куда-то рукой на запад.Пётр понял, что у немца там остался дома ребёнок. Пожалел он фашиста, притащил живым в штаб.

Гитлеровец оказался не из простых, оказался носителем секретной информации, сообщил о месте нахождения фашистских боевых частей, указал, где у них – тяжёлая артиллерия, где огневые точки.И по его наводкам пошли вперёд наши танки, полетели самолёты, пошла в атаку пехота.На этом участке фронта состоялось успешное наступление.
И когда один из тех, кто возглавлял эту наступательную операцию, спросил:
– Кто добыл «языка»?

Ему ответили: командир дивизиона Бахтин.
– Представить к Звезде Героя Советского Союза! – приказал командующий.

Но во время оформления документов выяснилось, что отец у Петра Бахтина был «кулак». Звезду Героя в тех политических реалиях ему дать было невозможно, и её заменили на очередной орден.

Третий свой орден Пётр Бахтин получил в последние дни войны, за бои под Прагой.

Чтобы поберечь своих солдат, он сам пополз на разведку, для того чтобы выяснить, где у фашистов находятся пулемётные гнёзда, ДОТы, ДЗОТы, орудия, укрепления, где расположены огневые цели, чтобы потом по ним ударить огнём дивизиона «Катюш».

Когда он занимался занесением целей на временную карту, фашисты его обнаружили и открыли по нему пулемётный и минометный огонь. Он успел только прыгнуть, скатиться в одну из воронок, и там, раненый, потерял сознание.

Когда через три дня Пётр Бахтин пришёл в себя, открыл глаза, врач-чех, который над ним склонился, промолвил:
– Ну, слава Богу, жив. Не зря, значит, я за тебя Богу молился.
Бахтин недоумевает:
– Какому Богу? Я, – говорит, – неверующий.

Врач:
– А я видел – у тебя крестик под гимнастёркой приколот на груди.
– А это, – объясняет Пётр, – мне мама дала, когда я уходил на фронт. Мама этим крестиком благословила.
Врач махнул рукой:
– А, на, разбирайся сам, – и протянул ему Библию.

И отец Пётр Бахтин признаётся, что с тех пор для него вопрос: «Есть Бог или нет?» стал самым главным вопросом.

После окончания войны наши воинские части стояли в Европе, и нельзя было просто так демобилизоваться и уехать домой. Для того чтобы это сделать, нужно было вступить в партию.Он вступил в партию и в 1947-м году вернулся в свою Караганду, к маме.
И мама ему говорит:
– Сынок, из ста человек, ушедших на войну из Караганды, живым вернулся только ты один. Тебе нужно поблагодарить Бога. Поступай в семинарию.

А он думает: ну, правда, где ещё лучше узнаешь: есть Бог или нет, как не в семинарии, где как раз этому учат?
Он отправляется в Сергиев Посад (в то время он назывался Загорском) из своей Караганды, в семинарию.

А ему там говорят:
– А мы партийных не берём.
Тогда он возвращается обратно в Караганду. И заявляет в парткоме:
– Я хочу сдать свой партбилет.

Партийное начальство сначала решило, что он шутит. Потому что тогда такого практически не случалось, чтобы человек добровольно выходил из Коммунистической партии, это могло даже равняться просьбе: «Хочу, чтобы меня посадили или даже расстреляли».

.....И вот он возвращается обратно в семинарию, в Сергиев Посад, сдавать вступительные экзамены.На первом экзамене его попросили прочитать церковно-славянский текст. По нынешнем меркам это были не самые высокие требования для экзаменов. Несложные экзамены. Но по тем временам, очевидно, это было не так просто.

А вы, наверное, знаете, что вместо пропущенных букв в церковнославянском шрифте над словом указывается титл. И по смыслу или по пропущенному знаку нужно вставлять пропущенную букву. Бог пишется, как Б~г, Богородице, как Б~це.

%https://regnum.ru/news/accidents/3268209.html
«Опухоль мозга — диагноз общества» — Башкирия обсуждает трагедию в Казани 

%https://zen.yandex.ru/media/id/5f7659c81ca11a0cea25f92f/chto-ia-viju-o-nastoiascem-i-buduscem-rossii-i-chelovechestva--aleksandra-barvickaia-609c354eb3e6cf0944176aa9
Что я вижу о настоящем и будущем России и человечества | Александра Барвицкая


%bulgakov
На свете существует только две силы: доллары и литература. «Зойкина квартира» 

Все пройдет. Страдания, муки, кровь, голод и мор. Меч исчезнет, а вот звезды
останутся, когда и тени наших тел и дел не останется на земле. Нет ни одного
человека, который бы этого не знал. Так почему же мы не хотим обратить свой
взгляд на них? Почему? «Белая гвардия» 

%https://strana.ua/opinions/333177-khorosho-by-nashej-natsii-nauchitsja-u-majja-ne-derzhat-obid-na-ves-mir.html
— Это и есть Мудрость, — ответил вождь. — Каждый раз, когда ты злишься, в твоём
сердце, как в этой торбе, вонючая тяжесть. Её становится всё больше, она тебя
угнетает и мешает воевать и отдыхать, отравляет любую радость. И вся жизнь
превращается в сплошное тягучее болото.

Поэтому наша нация не держит обид на весь мир. Мы растим детей, бережем
Природу, дорожим Миром и просто радуемся жизни. Хорошо бы и вам, европейцам,
научиться тому же!

%remark_erih_maria
Я вижу, что кто-то натравливает один народ на другой и люди убивают друг друга, в безумном ослеплении покоряясь чужой воле, не ведая, что творят, не зная за собой вины.

Я вижу, что лучшие умы человечества изобретают оружие, чтобы продлить этот кошмар, и находят слова, чтобы ещё более утонченно оправдать его.

==========================
Эрих Мария Ремарк. «На Западном фронте без перемен». 1929 год.

%https://strana.ua/news/333089-pochemu-v-ssha-voennye-vsled-za-frantsuzami-napisali-pismo-s-pretenzijami-k-bajdenu.html
"Нация в опасности". Вслед за Францией военные выступили против власти в США. Что это значит?

%https://www.pravda.com.ua/articles/2021/05/14/7292466/
Борхес проти Кернеса. Про Харків здорової людини і людину, якій більше за всіх треба

%Patsan.TV - 18+ Обращение упоротого киевлянина (2016)
%https://www.youtube.com/watch?v=HDyB4IA6b2Q

%https://www.radiosvoboda.org/a/29866457.html
БЕЗ ІМЕНІ

Викрали моє ім’я

(не штани ж – можна і без нього жити!)
І тепер мене звуть
той, у кого ім’я викрадено.

Я вмію сіяти і мурувати білі стіни,
і коли я посію, то всі дізнаються, що сіяв той,
у кого ім’я викрадено,
а на білих стінах я завжди пишу:

стіну вибудував той, у кого ім’я викрадено.
Привітальні телеграми і листи
ідуть уже на моє нове ім’я,
на ім’я того, у кого ім’я викрадено.

Уже всі примирилися (бо ж і сам давно)
із моїм новим ім’ям.
Дружина теж звикла.
Тільки от не знаю, як бути дітям,
як їх кликатимуть по батькові?

І дай Боже Вам (і нам усім...) дожити до перемоги – до Повернення Вкраденого.

%https://news.obozrevatel.com/sport/football/podtashnivalo-a-nogi-byili-vatnyie-blohin-rasskazal-o-pobede-dinamo-v-kubke-kubkov.htm
Громогласным возгласом "Победа, Кубок наш!" встречали нас в Борисполе сотни
киевских болельщиков, когда по трапу с Кубком Кубков в руках спускался наш
капитан Виктор Колотов, а за ним все мы – самые счастливые в тот день в мире
футбольные парни и их тренеры".


%7. Игорь Ольгович
%https://proza.ru/2014/06/17/1492

КИЕВ - РУССКИЙ ГОРОД!⁣⁣53:15⠀- стих
%https://www.youtube.com/watch?v=0PyQtY5pneA

%Горите в аду, кровавые украинские палачи! Молодежь ЛНР попрощалась с 14-й бригадой ВСУ
%https://www.youtube.com/watch?v=yc6UCKwLIkQ

 %Вслед за газовой атакой — электрический удар
%Население ждет критическое повышение коммунальных тарифов 
%https://strana.ua/opinions/333508-vsled-za-hazovoj-atakoj-elektricheskij-udar-.html

%https://strana.ua/opinions/333533-nasha-rodina-nakhoditsja-na-samom-dne-jamy-pozora-i-bedstvija-chto-zhe-dalshe.html
%Наша родина находится на самом дне ямы позора и бедствия. Что же дальше?

%«500 років знадобилося, щоб повернутися до передмонгольської чисельності населення» – дослідник про завоювання Києва Батиєм 
%https://www.radiosvoboda.org/a/istorychna-svoboda-kyivska-rus-batyj/30986656.html

%640-річчя Куликовської битви: коли волинський князь воював за московського 
%https://www.radiosvoboda.org/a/kulykovska--bytva/30836440.html

%В Кремле объяснили ненависть Украины к России "дефицитом суверенитета" Киева
%https://strana.ua/news/333707-dmitrij-peskov-zajavil-o-defitsite-suvereniteta-v-ukraine.html

%Какие вопросы к Зеленскому? Опрос Антонины Белоглазовой | Страна.ua
%https://www.youtube.com/watch?v=eXWKyBZ_j24

%«Лично мне немного обидно, что прекрасный национальный наряд стал признаком слабоумия. Возможно подобная ситуация когда-нибудь изменится. Но для начала нужно перестать мериться патриотизмом посредством вышитой рубахи», — пишет об этом киевлянка Диана Ванади.

%Безусловно, она права.

%Читать далее: https://ukraina.ru/opinion/20210520/1031421481.html

%Это всё та самая «воля» — идеал мироустройства, отложившийся в западнорусских генах. Украина требует учитывать её интересы от России, хоть считает её государством-агрессором. Украина требует от ЕС учитывать её интересы, хоть сама не планирует оказывать Евросоюзу аналогичную любезность. Это «воля»: все должны мне (иначе где же я буду харчеваться), а я никому.

%Читать далее: https://ukraina.ru/opinion/20210520/1031408295.html

«Стрыйщина — земля обетованная для каждого украинца, — пишет Лариса Ницой на своей странице в социальной сети. — Стрыйские дети не верят, что в Киеве ученики, учителя с учениками и люди на улицах говорят на русском.

— Что, действительно?! — переспрашивали по нескольку раз.

Читать далее: https://ukraina.ru/opinion/20210520/1031420001.html

https://matchtv.ru/figure-skating/matchtvnews_NI1331500_Velikije_russkije_devochki_Kak_Shherbakova_Tuktamysheva_i_Trusova_vzali_vse_medali_chempionata_mira?utm_referrer=https%3A%2F%2Fzen.yandex.com
rtyu8765432
около 2 месяцев назад

Такого не было фурора
Всё трио русское стоит
Хоть поднят флаг не триколора
Не свой победный гимн звучит

У них есть мужество отвага
У наших всех - троих девчат
Они без гимна и без флага
На льду историю творят

Забрали вы у нас штандарты
Но всё ж имейте вы в виду
Летят над миром наши гранды
Как революция на льду

Сквозь рой чинуш и их хлыстов
Наш ледокол вперёд идет
Круша победами козлов
Переворачивая лёд

Вся бесполезна ваша спешка
Унизить гордость и страну
В ответ лишь Анина усмешка
При награждении на льду


%https://www.youtube.com/watch?v=LcdciITcux4


Майдан творить історію. 2000 – 2012
Спочатку було слово. І слово було «Проти»
\url{https://maidan.org.ua/aboutmaidan/}

Украину нужно лишить права оценивать выступления участников от страны-агрессора: это аморально
%https://news.obozrevatel.com/show/people/ukrainu-nuzhno-lishit-prava-otsenivat-vyistupleniya-uchastnikov-ot-stranyi-agressora-eto-amoralno.htm

Пресс-конференция Зеленского с точки зрения психиатрии
https://racurs.ua/b205-press-konferenciya-zelenskogo-s-tochki-zreniya-psihiatrii.html

Запад нам не поможет - нас презирают, от нас устали
Неужели впереди - третий Майдан? Неужели такова наша судьба? 
https://strana.ua/opinions/334883-zapad-nam-ne-pomozhet-nas-prezirajut-ot-nas-ustali.html


Нынешняя власть существует в формате поддержки двух сил, как встречных векторов
давления, с запада и востока, которые как подпорки удерживают это "ветхое
здание" в положении "стоя".
Если одна из сил ослабнет, "здание" накренится в противоположную сторону.
А если исчезнут обе - то "хибара" рухнет внутрь...
https://strana.ua/opinions/335167-zelenskij-javno-stremitsja-k-ustanovleniju-lichnoj-diktatury.html


%История о мальчике Бобби
%https://www.youtube.com/watch?v=wkT1e_TUMVk
