%%beginhead 
 
%%file 10_06_2022.fb.vorotilova_oleksandra.mariupol.1.skoro_zroblyu_kom_ks
%%parent 10_06_2022
 
%%url https://www.facebook.com/permalink.php?story_fbid=pfbid0253MxyUwKpWcPdpSn2gGaeMb1qmakshPzujicEUQZKnEqxGAfM5XFVqKw5r3Rn6oCl&id=100010830875187
 
%%author_id vorotilova_oleksandra.mariupol
%%date 10_06_2022
 
%%tags komiks,mariupol,mariupol.war
%%title Скоро зроблю комікс про один день мого життя в окупації Маріуполя
 
%%endhead 

\subsection{Скоро зроблю комікс про один день мого життя в окупації Маріуполя}
\label{sec:10_06_2022.fb.vorotilova_oleksandra.mariupol.1.skoro_zroblyu_kom_ks}

\Purl{https://www.facebook.com/permalink.php?story_fbid=pfbid0253MxyUwKpWcPdpSn2gGaeMb1qmakshPzujicEUQZKnEqxGAfM5XFVqKw5r3Rn6oCl&id=100010830875187}
\ifcmt
 author_begin
   author_id vorotilova_oleksandra.mariupol
 author_end
\fi

Скоро зроблю комікс про один день мого життя в окупації Маріуполя. 

9 березня я пішла у гості до своїх друзів, але мене дуже тягнуло додому, я
повернутись не змогла через обстріли. Мій дім був цілий 9 березня. Я
поверталась додому і побачила, що моя вулиця була засипана вікнами, склом,
плити лежали на дорозі та на деревах. Сказати, що я була а шоці, це нічого не
сказати. Ми йшли, але летіли гради та міни прямо в нас. Я сказала друзям, що ми
повинні сховатися у під'їзді поруч, але у якому під'їзді можно було сховатись,
якщо все виглядало приблизно так. Ми підійшли близько до мого двору , вже горів
мій будинок з південної та півночної сторони. Російський літак скинув дві бомби
один на мій дім, а іншу в магазин поруч. Я побачила вогонь та дім і почала
плакати та кричати ні, тільки не моя сім'я. Прохід у двір завалило біля мого
будинку. Ми сховались у магазині. Я почала  плакати та казати, що я залишилась
одна, всі померли, там не можна було вижити. Плакала дуже довго. Під обстрілами
ми зайшли у двір, я почали кричати тільки не це. Я  підійшла і почала рахувати
вікна від цієї діри. Я дорахувалася до трьох поверхів, це було вічністю, а
потім дорахували до свого балкону і почали кричати і задихатись. Я побачила
підвал, залізні двері, просто вибило. Я підходила ближче , і змогла сказати,
задихаючись ім'я моєї 4 річної племінниці, а сестра сказала, що вони удвох у
підвалі. Мою сім'ю врятувала міна, яка прилетіла увечері і розбила всі вікна.
Тому моя сім'я ночували в під'їзді. Діти з мого будинку плакали, а моя
племінниця хотіла гратись з цуциками. Я заспокоїлась, а потім спитала де мама.
Сестра розповіла, що вона залишилась у будинку. Я набралась сміливості йти у
мою квартиру та витягувати речі. Я була смілива шукати мертву маму. Я прекрасно
розуміла, що її тіло могло  розірвати. Я забігла на 3 поверх, 9 поверх вже
палав, я витягла речі, документи, два дипломи про закінчення університету, всі
сертифікати Erasmus. Я забігла,  почала кричати ім'я своєї мами. Я розуміла, що
у своїй квартирі, я можу померти. Я спустилась на перших поверх, побачивш
незнайомих людей, я почала кричати, де моя мама. Я знайшла її живу. Я зайду у
квартиру, де просто немає двох стін другий раз, але вже палав 5 поверх.

Ми були між двома арміями, у наказі військових не написано про цивільних.
Маріуполь - це жертва для спасіння всіє України. Як історик, я можу зрозуміти
це, але як людина, яка йде шукати 4 річну племінницю мертву ніколи. Це провина
української влади, яка здала Херсон, Бердянськ, Мелітополь, частину Запорізької
області. Нахуй така влада взагалі потрібна, яка знає, що ми вже з одного боку
оточені ДНР.
