% vim: keymap=russian-jcukenwin
%%beginhead 
 
%%file slova.kosmonavt
%%parent slova
 
%%url 
 
%%author 
%%author_id 
%%author_url 
 
%%tags 
%%title 
 
%%endhead 
\chapter{Космонавт}
\label{sec:slova.kosmonavt}

%%%cit
%%%cit_head
%%%cit_pic
%%%cit_text
В 00 часов 16 минут 30 июня состоялся еще один сеанс связи с кораблем.
Организатор и руководитель подготовки \emph{космонавтов} генерал Николай Каманин
напомнил о необходимости при спуске вести постоянный репортаж на коротких и
ультракоротких волнах. А после посадки действовать строго по инструкции: резко
не двигаться, люк не открывать и ждать группу поиска.  «Желаю мягкой посадки.
До скорой встречи на Земле!» — подытожил Николай Петрович. В ответ командир
экипажа Добровольский сказал: «Вас понял, условия посадки отличные. На борту
все в порядке, самочувствие экипажа отличное. Благодарим за заботу и добрые
пожелания»
%%%cit_comment
%%%cit_title
\citTitle{«Они были обречены» 50 лет назад погиб экипаж «Союза-11». Кто
виноват в главной трагедии советской космонавтики?: Космос: Наука и техника:
Lenta.ru}, Сергей Варшавчик, lenta.ru, 30.06.2021
%%%endcit

