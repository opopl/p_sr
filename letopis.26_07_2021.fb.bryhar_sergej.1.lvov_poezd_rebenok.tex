% vim: keymap=russian-jcukenwin
%%beginhead 
 
%%file 26_07_2021.fb.bryhar_sergej.1.lvov_poezd_rebenok
%%parent 26_07_2021
 
%%url https://www.facebook.com/serhiibryhar/posts/1749591708574107
 
%%author Бригар, Сергей
%%author_id bryhar_sergej
%%author_url 
 
%%tags jazyk,mova,ukraina
%%title Львов - Язик - Мова
 
%%endhead 
 
\subsection{Львов - Язик - Мова}
\label{sec:26_07_2021.fb.bryhar_sergej.1.lvov_poezd_rebenok}
 
\Purl{https://www.facebook.com/serhiibryhar/posts/1749591708574107}
\ifcmt
 author_begin
   author_id bryhar_sergej
 author_end
\fi

Мій старший син впродовж чотирьох місяців питав: "Тату, а коли ми поїдемо до
Львова?" Фактично це триває з того дня, коли я почав активно показувати йому
фільми, програми, альбоми, картини, присвячені цьому місту (і не лише йому). Це
можна назвати невеличкою програмою географічної освіти для "пташат".

Але з поїздкою все не складалося... І ось, нарешті вийшло. Ми викупили собі
купе в обидва кінці, за що довелося віддати... страшно навіть подумати, яку
суму, й вирушили в подорож усім дружним сімейством...

Трохи, звісно, хвилювалися, але вийшло дуже добре. По-перше, діти майже зовсім чудово впоралися з дорогою. Але було і дещо важливіше...

От уявіть собі: 4-річна дитина, яка завжди виходила на вулицю, і відразу ж чула
іншу, погано зрозумілу мову, що тотально домінує в її місті, раптом опиняється
у світі, де все інакше: друзі та знайомі розмовляють українською; працівники
сфери послуг і чиновники - українською; діти на майданчику, звісно ж, -
українською; випадкові перехожі в абсолютній більшості - теж. 

Ми вже давно не туристи, і цікавлять нас тепер не історичні будівлі чи Високий
замок, а люди - і свої, і різні... Цікавить звичайне життя, порівняння укладів,
особливостей.

Здається, вже кількох годин перебування в іншому просторі - (до речі, не
виключаю і особливої ролі нашої ментальної сумісності саме з тими людьми, бо
хоч і не можу зараз чітко сформулювати, в чому проявляються відмінності між
мешканцями півдня і заходу, виразно відчуваю, що вони є) вистачило, щоб його
поведінка суттєво змінилася. 

Я не можу назвати його компанійською дитиною. Триматися осторонь - це його
норма. Більше спостерігати, гратися самому, підпускати до себе лише обраних -
звична справа. В тому, що це не патологія, ми пересвідчилися давно. Просто воно
якось так... Мені здавалося, що це природна інтровертивність, і так має бути -
такі особливості. Але у Львові він почав непогано контактувати фактично з усіма
дітьми, які виявляли бажання контактувати з ним. Виглядало майже фантастично...

Тепер уже й не беруся говорити про інтровертивність. Чи може
"інтровертивність". Не знаю... Загалом я добре ставлюся до інтровертів. Вони
класні. 

\ifcmt
  pic https://scontent-cdt1-1.xx.fbcdn.net/v/t39.30808-6/225617666_1749574485242496_5727848714395951506_n.jpg?_nc_cat=106&ccb=1-3&_nc_sid=8bfeb9&_nc_ohc=FBnu4GCnTd0AX90D2mC&_nc_ht=scontent-cdt1-1.xx&oh=e12f69e158a50eea00fd85b6ed71736a&oe=610801E2
  width 0.4
\fi

Просто не факт, що це про нього. Можливо, дитині просто не підходять існуючі
умови.

Думаю, не дуже контактним він став через нас, батьків, які навіть на
підсвідомому рівні прагнуть уникнути спілкування з "мєстнимі" "мамамі-папамі",
"бабулямі-дєдудямі", які часто дивляться на нас як на інопланетян... 

Особливо складно з сусідами, з якими буває приблизно так: "Ви шо ето єго
русскому язику вообщє нє учітє? - Ні. - В Одєссє бєз нєго нікак. - Ми в
Україні, і державна мова тут українська. - Ето во Львове, а у нас іначє. - Якщо
так, то тут небезпечно. - Ой, вот только нє надо про "небезпечно". В СССР било
отнюдь нє плохо...". Звісно, якось контактуємо, але це  неприємно...

Знаєте, про що найперше він спитав по приїзді додому (от просто після майже
12-годинної поїздки денним поїздом, що втомлює навіть нас, дорослих)?

\begin{itemize}
  \item - Мама, тато, а коли ми знову поїдемо до Львова?
  \item - Та скоро, синку, - відповідаю.
  \item - Я хочу вже.
  \item - Ну то ти хоч відпочинь, а потім уже поговоримо...
\end{itemize}

Наразі ми дуже задоволені отриманими результатами.

\ii{26_07_2021.fb.bryhar_sergej.1.lvov_poezd_rebenok.cmt}
