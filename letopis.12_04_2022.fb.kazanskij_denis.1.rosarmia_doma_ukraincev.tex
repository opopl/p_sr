% vim: keymap=russian-jcukenwin
%%beginhead 
 
%%file 12_04_2022.fb.kazanskij_denis.1.rosarmia_doma_ukraincev
%%parent 12_04_2022
 
%%url https://www.facebook.com/den.kazansky/posts/5069950543084417
 
%%author_id kazanskij_denis
%%date 
 
%%tags 
%%title Последствия пребывания российской армии в домах украинцев
 
%%endhead 
 
\subsection{Последствия пребывания российской армии в домах украинцев}
\label{sec:12_04_2022.fb.kazanskij_denis.1.rosarmia_doma_ukraincev}
 
\Purl{https://www.facebook.com/den.kazansky/posts/5069950543084417}
\ifcmt
 author_begin
   author_id kazanskij_denis
 author_end
\fi

Российский художник Юрий Анненков (1889 -1974) в своих дневниках делал
зарисовки о Гражданской войне. Вот как он описывал свою дачу в Финляндии после
того, как там побывали красноармейцы. 

\ii{12_04_2022.fb.kazanskij_denis.1.rosarmia_doma_ukraincev.pic.1}

«В 1918 году, после бегства красной гвардии из Финляндии, я пробрался в
Куокаллу (это еще было возможно), чтобы взглянуть на мой дом. Была зима. В
горностаевой снеговой пышности торчал на его месте жалкий урод — бревенчатый
сруб с развороченной крышей, с выбитыми окнами, с черными дырами вместо дверей.
Обледенелые горы человеческих испражнений покрывали пол. По стенам почти до
потолка замерзшими струями желтела моча, и еще не стерлись пометки углем: 2
арш. 2 верш., 2 арш. 5 верш., 2 арш. 10 верш.... Победителем в этом
своеобразном чемпионате красногвардейцев оказался пулеметчик Матвей Глушков: он
достиг 2 арш. 12 верш, в высоту.

Вырванная с мясом из потолка висячая лампа была втоптана в кучу испражнений.
Возле лампы — записка:

«Спасибо тебе за лампу, буржуй, хорошо нам светила».

% 2-3
\ii{12_04_2022.fb.kazanskij_denis.1.rosarmia_doma_ukraincev.pic.2}
\ii{12_04_2022.fb.kazanskij_denis.1.rosarmia_doma_ukraincev.pic.3}

Половицы расщеплены топором, обои сорваны, пробиты пулями, железные кровати
сведены смертельной судорогой, голубые сервизы обращены в осколки,
металлическая посуда — кастрюли, сковородки, чайники — до верху заполнены
испражнениями. Непостижимо обильно испражнялись повсюду: во всех этажах на
полу, на лестницах — сглаживая ступени, на столах, в ящиках столов, на стульях,
на матрасах, швыряли кусками испражнений в потолок. Вот еще записка:

«Понюхай нашава гавна ладно ваняит».

В третьем этаже — единственная уцелевшая комната. На двери записка:

«Тов. Камандир».

% 4-5
\ii{12_04_2022.fb.kazanskij_denis.1.rosarmia_doma_ukraincev.pic.4}

На столе — ночной горшок с недоеденной гречневой кашей и воткнутой в нее
ложкой...»

Как будто о сегодняшней Буче писал, за 100 лет вообще ничего не изменилось. 

На фото последствия пребывания российской армии в домах украинцев.

% 6
\ii{12_04_2022.fb.kazanskij_denis.1.rosarmia_doma_ukraincev.pic.5}

% 7-9
\ii{12_04_2022.fb.kazanskij_denis.1.rosarmia_doma_ukraincev.pic.7_9}

\ii{12_04_2022.fb.kazanskij_denis.1.rosarmia_doma_ukraincev.pic.10}
\ii{12_04_2022.fb.kazanskij_denis.1.rosarmia_doma_ukraincev.pic.11_12}

\ii{12_04_2022.fb.kazanskij_denis.1.rosarmia_doma_ukraincev.cmt}
