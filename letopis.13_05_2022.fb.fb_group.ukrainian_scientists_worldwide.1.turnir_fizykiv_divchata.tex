% vim: keymap=russian-jcukenwin
%%beginhead 
 
%%file 13_05_2022.fb.fb_group.ukrainian_scientists_worldwide.1.turnir_fizykiv_divchata
%%parent 13_05_2022
 
%%url https://www.facebook.com/groups/314070194112/posts/10160758252179113
 
%%author_id fb_group.ukrainian_scientists_worldwide,golubov_oleksij.harkiv.astronom
%%date 
 
%%tags 
%%title На Міжнародному турнірі фізиків команда українських дівчат посіла 6 місце серед 15 команд
 
%%endhead 
 
\subsection{На Міжнародному турнірі фізиків команда українських дівчат посіла 6 місце серед 15 команд}
\label{sec:13_05_2022.fb.fb_group.ukrainian_scientists_worldwide.1.turnir_fizykiv_divchata}
 
\Purl{https://www.facebook.com/groups/314070194112/posts/10160758252179113}
\ifcmt
 author_begin
   author_id fb_group.ukrainian_scientists_worldwide,golubov_oleksij.harkiv.astronom
 author_end
\fi

На Міжнародному турнірі фізиків команда українських дівчат посіла 6 місце серед
15 команд, потрапивши до півфіналу, але не дійшовши до фіналу.

Дівчата включились в підготовку до турніру, холи стало зрозуміло, що команда
хлопців, переможці Всеукраїнського студентського турніру фізиків (5-9 лютого
2022), не зможе виїхати з країни на турнір через війну. 

Готувалися лише онлайн, без доступу до лабораторій, з перервами на вибухи та
вимкнення електрики, і всього за 1,5 місяці зробили роботу, яку зазвичай
роблять за 6-8 місяців. 

Імена наших героїнь - Дарина Сич, Анастасія Демидченко,
\href{https://www.facebook.com/PolinaDimitriieva}{Polina Dimitriieva} (КНУШ),
Анастасія Романчук, Влада Фандєєва (ХНУ ім. Каразіна),
\href{https://www.facebook.com/profile.php?id=100004338379209}{Яна Калисецкая}
(КПІ ім. Сікорського).

Керівниці команди - Kseniia Minakova і Anastasia Gaieva.
