% vim: keymap=russian-jcukenwin
%%beginhead 
 
%%file 24_08_2021.fb.starchenko_natalia.kiev.istorik.1.den_nezalezhnosti
%%parent 24_08_2021
 
%%url https://www.facebook.com/natalya.starchenko.1/posts/3913657665405895
 
%%author_id starchenko_natalia.kiev.istorik
%%date 
 
%%tags nezalezhnist,ukraina
%%title День Незалежності
 
%%endhead 
 
\subsection{День Незалежності}
\label{sec:24_08_2021.fb.starchenko_natalia.kiev.istorik.1.den_nezalezhnosti}
 
\Purl{https://www.facebook.com/natalya.starchenko.1/posts/3913657665405895}
\ifcmt
 author_begin
   author_id starchenko_natalia.kiev.istorik
 author_end
\fi

День Незалежності - це не про те, що нічого не було, а потім раптом нам хтось
його на тарілочці, як думає частина наших співвітчизників. Це про шанс, який ми
готові були використати, коли він з’явився. Але за цією готовністю насправді
дуже багато не лише жертв, а передусім маленьких і великих перемог. І крові,
яка неминуча з нашим непростим минулим і не менш складним сьогоденням, і в разі
перемог теж. Але шанс використовують переможці, а не жертви обставин.

Щоб ми використали свій шанс, ми мусили його створити. Мусило бути дієве
\enquote{прозріння} після Чорнобиля і люди, здатні  заблокували кроки Москви по
створенню наступних \enquote{чорнобилів}. Мусило бути розуміння біди з мовою (близько
20 \% українських шкіл у Києві, 40 \% - по Україні), а значить - критична маса
людей, яким \enquote{не какая разніца}, здатних на її захист (Товариство української
мови імені Тараса Шевченка). Ми мусили почути голос тих, хто повертався з
таборів (Українська гельсінська спілки), і мусили заповнити вулиці густим
потоком, нарешті віддаючи своїй землі їй належне - великих українців Василя
Стуса, Юрія Литвина, Олекси Тихого. Мав виникнути Народний Рух України за
перебудову, який збирав багатотисячні мітинги, і мала відбутися Революція на
граніті з відчайдушною акцією голодування студентів, головною вимогою якої було
- не допустити підписання нового союзного договору. Мала бути проголошена
Декларація про державний суверенітет України 16 липня 1990, де було записано -
оголосити надалі цей день Днем незалежності України. І саме Декларація стане
підставою для провалених українцями новоогарьовських перемовин про дальше
існування СССР.

Я щиро вдячна сьогодні не лише тим, хто був на гребені тих подій, а й тим 80\%
українців, тоді ще громадян УРСР, які в березні 1991 віддали свій голос за
питання: «Чи згодні ви, щоб Україна була у складі Союзу Суверенних Республік на
підставі Декларації про суверенітет України». Не знаю, чи розуміли вони до
кінця, що своїми голосами звалили Колосса. Очевидно, що ні, але вони це
зробили. Я безмежно вдячна усім тим 90 \% українців, які 1 грудня 1991 року
відповіли ствердно на питання: «Чи підтверджуєте ви Акт проголошення
незалежності України?». І сьогодні мені навіть не важить, чим вони керувалися -
потребою незалежної України як простору буття для себе й своїх дітей, без якого
біда, а чи прагматичнішими міркуваннями. Бо тоді ці 90 \% вирішили бути народом
суверенної України, а значить - заклали свій камінчик у фундамент дня
сьогоднішнього.

Так, наш рух не був прямим і переможним. Ми петляли, здавали позиції, але
водночас, хай повзком і з надзусиллями, але ми нарощували свою готовність до
шансу. Як сказала під час Помаранчевої революції моя студентка з Луганщини:
\enquote{Нам говорили в школі, що ми живемо в незалежній Україні, що наша столиця -
Київ. Вони може в це й не вірили, а ми взяли й повірили}. Сьогодні тим дітям,
які вийшли тоді на Майдан, і які формувалися на спогадах, якою красивою була
наша Помаранчева революція, за тридцять. І коли восени 2013 у нас спробували
відібрати наш шанс, ми його вже не здали, бо знали - справа про нашу гідність,
кожного й усього народу; своє \enquote{бути} кращі з нас готові були захищати ціною
життя, і їх виявилося досить, аби вихопити Україну вже з прірви.

Вчора у подвір’ї музею Івана Гончара Тарас Компаніченко виконував стару відому
пісню про козака, який не може піти в похід, бо його “коняку турки вбили, //
Ляхи шаблю пощербили, // І рушниця поламалась, // І дівчина відцуралась”. Ті,
хто стояв на Майдані з дерев'яними щитами і хто пішов на війну в кедах і без
зброї весною 2014 року остаточно поламали цей код українців як жертви, бо свій
шанс виборюють насправді ті, у кого за плечима перемоги.

Тому для мене сьогоднішній день - День нашої перемоги, жертовної, красивої,
складної. Хай він буде саме таким - гарним і світлим, бо у нас попереду ще дуже
багато різного і нам ще довго нарощувати нашу готовність до шансів, великих і
малих.  

З Днем незалежності, друзі мої. З найбільшим нашим Святом, бо іншого такого
просто немає. Слава Україні. Вічна слава героям-переможцям.
