% vim: keymap=russian-jcukenwin
%%beginhead 
 
%%file 01_12_2021.fb.potemkin_aleksandr.kiev.1.125_let_georgij_zhukov
%%parent 01_12_2021
 
%%url https://www.facebook.com/potemich/posts/2148819025258754
 
%%author_id potemkin_aleksandr.kiev
%%date 
 
%%tags godovschina,istoria,pamjat,zhukov_georgii
%%title Сегодня 125 лет со дня рождения Георгия Константиновича Жукова
 
%%endhead 
 
\subsection{Сегодня 125 лет со дня рождения Георгия Константиновича Жукова}
\label{sec:01_12_2021.fb.potemkin_aleksandr.kiev.1.125_let_georgij_zhukov}
 
\Purl{https://www.facebook.com/potemich/posts/2148819025258754}
\ifcmt
 author_begin
   author_id potemkin_aleksandr.kiev
 author_end
\fi

Сегодня 125 лет со дня рождения Георгия Константиновича Жукова. 1 декабря 1896
года. Моя детская любимая книга была "Воспоминания и размышления". Для меня
Жуков прежде всего это герой освободитель. 

\ifcmt
  ig https://scontent-frx5-1.xx.fbcdn.net/v/t39.30808-6/262264036_2148819001925423_5888593478735760842_n.jpg?_nc_cat=105&ccb=1-5&_nc_sid=8bfeb9&_nc_ohc=pBUTtui4pbgAX8jIh4a&_nc_ht=scontent-frx5-1.xx&oh=bed3d7a5e18c342034de8892f76f9c3b&oe=61ADBFA7
  @width 0.4
  %@wrap \parpic[r]
  @wrap \InsertBoxR{0}
\fi

Сегодня наша дружная команда ( Титаренко Любовь, Руслан Коцаба, Виталий, Денис
Жарких, Екатерина Жарких, Татьяна Мазурова ) приехали в гости к знаменитым
киевским бабушкам с портретом Жукова, которые все пришедшие 9 мая к вечному
огню могли видеть. К сожалению одна бабушка умерла недавно. Говорили, что у них
протекала крыша, приехали посмотреть чем можем помочь. Оказалось что крышу им
недавно сделали новую, но полностью упал старый забор. Решили как юные
Тимуровцы сделать забор. Огромная благодарность нашим дедам и бабушкам, которые
на своих плечах пронесли все тяготы страшной войны. Вечная память героям! Пока
живы наши ветераны, будет жить память в наших сердцах. Спасибо всем ветеранам! 

Жуков — фигура в истории исключительная, человек редкого военного таланта.
Жизнь то поднимала его до небес, то низвергала. Его боялись такие люди как
Сталин и Хрущев, ненавидел Брежнев. Обладая громадной властью, они унижали его,
передвигая на второстепенные должности, или вовсе оставляли не у дел, но при
всем желании не могли лишить его боевой славы, всемирной известности. Не будем
умалять достоинств других полководцев. Каждый из тех, кто вел народ к победе,
достоин безграничной признательности. Но пальму первенства надо все-таки отдать
Г.К. Жукову, ибо только за ним закрепилось звание, не предусмотренное «табелью
о рангах» — Народный Маршал.
