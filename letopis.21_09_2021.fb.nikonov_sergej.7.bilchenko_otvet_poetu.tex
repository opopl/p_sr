% vim: keymap=russian-jcukenwin
%%beginhead 
 
%%file 21_09_2021.fb.nikonov_sergej.7.bilchenko_otvet_poetu
%%parent 21_09_2021
 
%%url https://www.facebook.com/alexelsevier/posts/1586078495070793
 
%%author_id nikonov_sergej,bilchenko_evgenia
%%date 
 
%%tags 02.05.2014,bilchenko_evgenia,lenin_vladimir,odessa,okean,poezia,rusmir,st_peterburg,tragedia.odessa.2_maj.2014
%%title БЖ. Ответ поэту
 
%%endhead 
 
\subsection{БЖ. Ответ поэту}
\label{sec:21_09_2021.fb.nikonov_sergej.7.bilchenko_otvet_poetu}
 
\Purl{https://www.facebook.com/alexelsevier/posts/1586078495070793}
\ifcmt
 author_begin
   author_id nikonov_sergej,bilchenko_evgenia
 author_end
\fi

Новое стихотворение. Как ответ другому поэту. Без моих реакций, мнений, комментов.
БЖ. Ответ поэту.

\ifcmt
  ig https://scontent-yyz1-1.xx.fbcdn.net/v/t1.6435-9/242356547_1586076825070960_2371736423413814611_n.jpg?_nc_cat=105&ccb=1-5&_nc_sid=730e14&_nc_ohc=Enhx98-RbD8AX_lHACR&_nc_ht=scontent-yyz1-1.xx&oh=4780f62bfaaffb3166f11cc114c60cb9&oe=6170BB83
  @width 0.4
  %@wrap \parpic[r]
  @wrap \InsertBoxR{0}
\fi

Уходя, петербуржец становится каплей Невы,
Не желая сливаться с толпой в мировом океане... Сергей Адамский
Дорогой ненавистник родной толпы, когда вас сожгут в Одессе,
А через час, чисто по Цою, из вас прорастёт трава,
Вы забудете о деликатности, дипломатии, политесе,
О либеральной борьбе за свои автозаковские права.
Атомарный гений из постмодерна, считающий себя асом
Классической метрики, критики Сталина иже с нею,
Если бы вы побыли на месте детей Донбасса,
Вы бы поэзию объявили девичьей ахинеей.
Конечно, я не имею право учить вас русскому миру, -
Скажете вы: я же столько против него грешила.
Я - аборт его, я - дитя его, я - пушка его, я - лира.
Я - зашившийся его Шива, я - в его попе шило.
Дорогой господин мизантроп и сноб, не любящий скоморошьего,
Ярмарочного, азартного, ярого, ополченского...
Русский мир - это бублики, дырочки, крестики, крошки, рожки...
Это - готовность к смерти: не по-творчески, нет, по-честному.
Это - Ржев без подмоги. Это - разведчик в Праге.
Это - училка в Горловке. Это - и рэп, и мат.
Севастополь его любви не вмещается на бумаге,
Его тоника Маяковская не причешется в стройный ряд.
Русский мир - это дворник. Русский мир - это продавщица:
Взбесившая вас толпа Ленина выберет, единороссов и, блин, Захара.
Мир мой русенький на кортах: "С нами Бог!", - от водочки протащился,
И я тащусь, потому что есть Семеныч, СашБаш, гитара.
Я ведь люблю военных: красивых и здоровенных.
Я ведь люблю бухгалтера и добровольца, мама!
Нельзя стать каплей Невы, нельзя: все нивы впадают в вены,
Вены - в реки, реки - в моря, моря - в океаны, в храмы
Все они вместе впадают: неразрывно и неслиянно.
Ибо - Святая Троица - иконы златой экран.
Не хочу быть пражцем ли, петербуржцем ли, москвичем-смутьяном...
В океане русского мира я - капля,
И я же в ней - океан.
21 сентября 2021 г.
