%%beginhead 
 
%%file 17_04_2022.fb.shubenko_kateryna.mariupol.1.13_marta_13_mikrorajon
%%parent 17_04_2022
 
%%url https://www.facebook.com/katy.svistun/posts/pfbid0dxP37JZa3vCmJaaCNzpocU7bLixnB8Hvb4WS3sr9CZozLyDEqgrznptcCz3pXrJHl
 
%%author_id shubenko_kateryna.mariupol
%%date 17_04_2022
 
%%tags 13.03.2022,mariupol.war,mariupol,dnevnik
%%title 13 марта в 13-ом микрорайоне
 
%%endhead 

\subsection{13 марта в 13-ом микрорайоне}
\label{sec:17_04_2022.fb.shubenko_kateryna.mariupol.1.13_marta_13_mikrorajon}

\Purl{https://www.facebook.com/katy.svistun/posts/pfbid0dxP37JZa3vCmJaaCNzpocU7bLixnB8Hvb4WS3sr9CZozLyDEqgrznptcCz3pXrJHl}
\ifcmt
 author_begin
   author_id shubenko_kateryna.mariupol
 author_end
\fi

У каждого есть свой  день в обстреливаемом Мариуполе. Тот, который никогда не
забыть. Для меня это 13 марта в 13-ом микрорайоне. Символично, если не так
трагично. В этот день я могла погибнуть три раза. Один раз под завалами почти
такого же дома, что на фото. 

ЯДОВИТЫЙ ГАЗ - СМЕРТЬ ВО СНЕ

В ту ночь мне снилась мама, которая ушла из жизни три года назад. И Мариуполь в
страшных огнях. Она вела меня в безопасное место - в храм Святогорска. И я
резко проснулась.... Я не могла больше уснуть. Мне, казалось, что это плохое
знамение. 

Наверное, благодаря этому я в 4 утра почувствовала неприятный запах. Вместе со
мной проснулся бывший спасатель Саша. Он прошел в соседнюю секцию и увидел
густое облако едкого дыма. В магазине "Жигули", что над нами - из-за высокой
температуры прямых попаданий начали взрываться банки лакокрасочных материалов.
Ядовитый и удушающий газ начал спускаться по вентиляции и быстро
распространяться смертельными щупальцами. Саша быстро поднял тревогу, нужно
было собраться за минуту - иначе мы погибнем. Нужно будить людей, иначе они не
проснутся. Эту тревогу подняли во всем укрытии. Мы выстроились в коридоре и
выдвинулись к выходу. Толкаясь и в панике. Дым был очень плотным и дышать было
невозможно. Часть людей выбежала на улицу в комендантский час.... Среди них - я
с мужем, а в руках собака.

ОГНИ МИНОМЕТА - ЖИВАЯ МИШЕНЬ

Мы выбежали на улицу. Темно, белый дым, ничего не видно. Нас приняли за
диверсионно-разведовательную группу, которая выбежали из дымовой завесы
атаковать. По нам открыли огонь. 

\ii{17_04_2022.fb.shubenko_kateryna.mariupol.1.13_marta_13_mikrorajon.pic.1}

На этот момент мы с мужем наощуп вышли к середине двора и тут погнала
минометная очередь. Мы побежали к пристройке во дворе, а в нас летели яркие
огни. Я помню этот момент стоп-кадрами. Будто ты в фильме о войне. Муж обнимает
меня за плечо и впервые за все это время говорит: "Катенька, я не знаю, что
делать" и тут мы понимаем, что умрем.... Вот так, во дворе и нас разнесет по
кусочкам. 1-3 секунды и мы видим соседний дом, и дверь в подъезд. Миномет нам
осветил дорогу. Мы со всех ног мчим туда. Железная дверь закрыта. Что делать?
Мы изо всех сил стучим и душераздирающе кричим: "ПОМОГИТЕ!" Не открывают,
собираемся бежать к другому подъезду. Но к нам присоединяется ещё пару человек,
бьют кулаком в дверь: "Мишаня, открывай". И Мишаня открыл. В этот момент
миномет попадает где-то рядом, да так рядом, что мы взрывной волной просто
влетели и впечатались в подъезд. Следующий взрыв застал нас на лестничной
площадке, когда мы с неизвестными людьми обнимаемся и прикрываем лица, чтоб в
нас не полетели оставшиеся стекла. Из-за третьего взрыва, я упала в подвал
лицом вниз. Мы в безопасности, мы уцелели....

РУХНУВШИЙ ПОДЪЕЗД - ПОХОРОНЕННЫЕ ЗАЖИВО

В подвале соседнего дома мы не могли принять решение - оставаться тут или
вернуться в свое укрытие. Тут был свежий воздух, но очень холодно. И все же это
было не укрытие, а обычный подвал. Было слышно, как военные проходят возле
дома, а в стене щели, через которые могут залететь шальные пули. Мы просидели
полдня в раздумьях. "Это русская рулетка. Решай!" - сказал муж. После его
разведки, мы  побежали через двор обратно. Мы летели под звуки стрельбы, она
была где-то рядом, но уже не по нам. Я залетела в укрытие и начала задыхаться
от нервного истощения. Чтобы привести меня в чувства, добрые люди дали выпить
стакан каньяка. Успокоили, стало легче.... 

Если мы приняли такое решение, то другая семья, что была в нашей секции во
главе с тем же Сашей - обратное. Они решили перебраться в тот соседний дом. Все
потому, что воздух там лучше. В нашем укрытии невозможно было дышать, пыль
стояла столбом и вообще оно было залито наполовину фекалиями. Мы все тут
кашляли от этого и чувствовали себя шахтерами. Саша принял такое решение ради
своей 10-летней дочери Сонечки. Ей было здесь плохо дышать. Они перетащили
оставшиеся вещи, обустроили там комнату для ночлега.

Спустя несколько часов наш дом затрещал от двух мощных взрывов. Мы не поняли,
что это было. Наш дом зашатался, загудел, посыпалась штукатурка. Это не было
похоже на Град, мы думали попало в наш дом и выбежали в коридор с вещами, боясь
что он рухнет. Но... Вместо этого к нам начали бежать люди из соседнего
дома.... В крови, слезах, с криками.... Все они бежали к нам. Упал второй
подъезд целиком. После первого удара часть людей успела выбежать, после второго
- на них легла бетонная плита. Замуровав навсегда.... Говорят, их голоса были
слышны, но они могли через несколько минут задохнуться от поднявшейся пыли....
Завалы невозможно было разгребсти в ручную, только спецтехникой. А ее не
было....

"Мама, папа! Где мои родители? Они живы?" К нам в секцию влетела та самая
Сонечка. Она рыдала, кричала, ее тело тряслось. Мы не могли ее успокоить.
Появилась мама. Ее лицо было залито кровью, на голову упало что-то тяжёлое.
"Мама, где папа? Скажи, что он живой? Он выжил? Мама, он выжил? Он должен
выжить, ведь он спас много людей. Он у нас спасатель. Боженька, не мог так с
ним поступить! Он добрый, всегда всем помогал. Мама, папа жив?" Эти слова она
повторяла долго - несколько дней и горько плакала. Плакали и мы с ней,  и тоже
хотели верить, что ее папа жив. Каждый из нас молился за это в душе.... Но к
сожалению... Война забрала ещё одного хорошего человека, мужа, отца, сына....

Тогда многие искали у нас в подвале своих детей, дети - родителей. Не все друг
друга нашли. Один мужчина добежал в наше укрытие и упал замертво. Не от ран -
сердце не выдержало. Говорили, что не успели выбраться девушка с двумя детьми -
грудничок и пятилетний мальчик, который всех смешил и рассказывал стишки.
Хочеться верить, что они побежали в другой дом. Хочеться верить в лучшее....

Это страшный день для меня и не только.   Вряд ли смогла передать те самые
эмоции хоть на 5\%. Также я знаю, что многие испытали более худшие моменты и...
потери....

\#ЯзМаріуполя

п.с.: завал, о котором я пишу, произошел в доме \#108 по улице Митрополитской
(его фото можно найти в переписках к посту). На фото другой дом, но с того же
микрорайона. К сожалению, сейчас все дома в Мариуполе не узнать и выглядят они
похоже....
