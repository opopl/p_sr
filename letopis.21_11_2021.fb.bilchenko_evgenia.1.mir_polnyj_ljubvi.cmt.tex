% vim: keymap=russian-jcukenwin
%%beginhead 
 
%%file 21_11_2021.fb.bilchenko_evgenia.1.mir_polnyj_ljubvi.cmt
%%parent 21_11_2021.fb.bilchenko_evgenia.1.mir_polnyj_ljubvi
 
%%url 
 
%%author_id 
%%date 
 
%%tags 
%%title 
 
%%endhead 
\subsubsection{Коментарі}

\begin{itemize} % {
\iusr{Тим Печеніг}

Евгения Витальевна, мне почему то выражение на подвал, которое увы практикуется
на моем родном Донбассе по отношению к каждому хоть в чем вызывающим подозрение
тамашних правил напомнило

\begin{itemize} % {
\iusr{Евгения Бильченко}
\textbf{Тим Печеніг} 

Уж не знаю, что оно вам напоминает, но стихотворению - десять лет, оно победило
на всех мыслимых и немыслимых фестивалях и писалось чисто о рокешном
андеграунде. В свое время, время хиппианской моей юности. Когда же
контркультура стала рабом вашего мира welfare, я перестала читать этот текст.
Теперь я ее вижу в русском бревне: мы, русское бревно, лежащее обрубком духа
поперек дороги триумфального шествия наслаждения. Каждому свое. И мне очень
приятно, что мое давнее киевское стихотворение о рокерах напомнило вам Донбасс,
не родной он вам. Он мне больше родной, раз я вам его напомнила стихами,
написанными в другое время, в другом месте, о других людях. Значит, я всегда
была Донбасс. Только мне нужно было время: покаяться и понять. Я всегда с
детства Егора Летова слушала. Вы правы: вы увидели во мне неприятный для вас
архетип. Думали: легко можно будет подшить девочку, которую обидели наци, под
либеральный мир, а девочка сама, кого надо, сделает. И, да, на фото - парни из
Донбасса есть тоже. Елизаров вообще из Харькова. А девочка оказалась мальчиком.
Вот такая нескладуха. Но мне от вашего разочарования, как бы сказать
вежливее... "Пренебречь - вальсируем". Выражение понятно, откуда?

\iusr{Тим Печеніг}
\textbf{Евгения Бильченко} , позвольте , Евгения Витальевна , отчего вы решили что Донбасс мне не родной ?

\iusr{Евгения Бильченко}
\textbf{Тим Печеніг} Сердцем чую. Иначе вы бы там были.

\iusr{Тим Печеніг}

Евгения Витальевна! Голубушка! Окститесь, ей богу, из Донбасса вынуждено выехала
масса народу, по тем или иным причинам этот регион и до последних событий
некоторые выехали и до этого, как ваш покорный слуга например. Да и вы тоже,
Евгения Витальевна, тепериче не на своей малой родине, матери городов русских
то

\iusr{Евгения Бильченко}
\textbf{Тим Печеніг} 

я на Родине. Я не делю Россию и Украину. Именно потому я не в Европе (отказ от
статуса беженца в Чехии и должности помощника депутата ЕС, отказ от беженства в
Хельсинки, гранты левых), не в Израиле (отказ от репатриации по бабушке). Я на
Родине. Это понятно? Для меня Киев-Питер-Донецк: одна мать, земля русская. А к
вам Я за вас съезжу, благо, ради этого я и здесь, кроме лечения. Мне надо на
Донбасс. Очень надо.

\iusr{Тим Печеніг}
\textbf{Евгения Бильченко} 

непременно поезжайте, был там давече, сказать что весь этот довольно обширный и
по площади и по разнообразию населения край Киев Питер Донецк, одним светом
Мазан, это ваше дело конечно, но по мне так мягко говоря большое упрощение

\iusr{Тим Печеніг}
Надо ещё успеть вернуться, в конкурсе участвую

\ifcmt
  ig https://scontent-frx5-1.xx.fbcdn.net/v/t39.30808-6/258092238_10220307127537363_6981585535587079322_n.jpg?_nc_cat=110&ccb=1-5&_nc_sid=dbeb18&_nc_ohc=vqCQX4hhmWIAX-GNfQM&_nc_ht=scontent-frx5-1.xx&oh=7a183cf1a03be89e6d29aed501449f58&oe=619F9B5F
  @width 0.4
\fi

\end{itemize} % }

\end{itemize} % }
