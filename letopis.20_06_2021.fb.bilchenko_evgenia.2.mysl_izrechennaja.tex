% vim: keymap=russian-jcukenwin
%%beginhead 
 
%%file 20_06_2021.fb.bilchenko_evgenia.2.mysl_izrechennaja
%%parent 20_06_2021
 
%%url https://www.facebook.com/yevzhik/posts/3994703783898035
 
%%author Бильченко, Евгения
%%author_id bilchenko_evgenia
%%author_url 
 
%%tags bilchenko_evgenia,izrechenie,literatura,myshlenie,poezia
%%title БЖ. Мысль изречённая
 
%%endhead 
 
\subsection{БЖ. Мысль изречённая}
\label{sec:20_06_2021.fb.bilchenko_evgenia.2.mysl_izrechennaja}
\Purl{https://www.facebook.com/yevzhik/posts/3994703783898035}
\ifcmt
 author_begin
   author_id bilchenko_evgenia
 author_end
\fi

БЖ. Мысль изречённая.

И сказать нечего, кроме трюизма дебильного в стиле: \enquote{Тю, ты чего?}
Любое высказыванье - бестактно, если границы Тютчева,
В которые он заключил сердце (\enquote{Как сердцу высказать себя}, - да, а как?),
Превращаются в пошлое и кичовое кухонное поддакивание.

Я знаю, что грех гордыни - прилипчивое и пыльное
Паутинное чмо в уголке души. Но знаю, что быть терпилами 
Тоже не полагается, когда изменяют жене своей,
Когда разрывает детей на куски по двору, а спросить-то не с кого.

Есть с кого. Просто боязно. Они говорят: \enquote{Политика}.
Школа молчит, и жена молчит, и церковь молчит, и критика.
А это - мертвый ребёнок и баба, слезу свою кровоточиной
В ванне пускающая... Мне чуждо пацифистское миротворчество.

Пассивная толерантность, грозящая репрессивною стать:
Один шаг до коллаборанта - от дидактика от красивого.
Не \enquote{возвышенности}, а Выси Библия учит...  ДОжили!
Никогда конформисты и пацифисты не станут сынами Божиими.

Посему, если женщина будет плакать от мужа, по хатам шастающего,
Я не буду сидеть, и думать, и перед ним расшаркиваться.
А если фашистская бомба снова в детсад угодит, что с сосками?..
Как вы думаете, толстовцы, что бы сейчас Христос сказал?

20 июня 2021 г.

Илл.: Борис Ольшанский. 2000 г. \enquote{Русский реквием}.

\ifcmt
  pic https://scontent-frt3-2.xx.fbcdn.net/v/t1.6435-9/201624004_3994703730564707_1943082552633479333_n.jpg?_nc_cat=101&ccb=1-3&_nc_sid=8bfeb9&_nc_ohc=5-CgnQ_H858AX8RCYy6&_nc_ht=scontent-frt3-2.xx&oh=cf23d4cf11fd37fe1a1bea5b83473e28&oe=60D4D223
\fi

\emph{Юрий Николаевич Тявин}

Евангелие от Матфея
...
5:9. Блаженны миротворцы, ибо они будут наречены сынами Божиими.
5:10. Блаженны изгнанные за правду, ибо их есть Царство Небесное.
5:11. Блаженны вы, когда будут поносить вас и гнать
и всячески неправедно злословить за Меня.

\emph{Евгения Бильченко}

Юрий Николаевич Тявин да.

\emph{Евгения Бильченко}
Злословить за меня... А если адепт говорит: \enquote{Я не знаю этого человека?} - и так каждый день.

\emph{Юрий Николаевич Тявин}
Адепт

Аде́пт (лат. adeptus «достигший, помогающий») — последователь, обычно ревностный
приверженец какого-либо учения, идеи, знания, помогающий понять идею и знание.
Понятием может определяться не только отношение к учению, личности или
организации, но и степень этого отношения. Адепт Бога?

\emph{Евгения Бильченко}
Юрий Николаевич Тявин я сейчас говорю в целом: истины, честности, ценностей. Назовите \enquote{ученик} или \enquote{активист}. Собрат. Соратник.
