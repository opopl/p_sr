%%beginhead 
 
%%file 31_03_2023.fb.vlasjuk_anatolij.ua.zhurnalist.writer.1.yur_i_knorozov__1922
%%parent 31_03_2023
 
%%url https://www.facebook.com/anatolij.vlasjuk.9/posts/pfbid0ZGfJuUkC2rWHoPuoR8Z42octZL8C7bZbgp7gKuBUJ5TBJYtsGN8xZc2vnB5FPJZPl
 
%%author_id vlasjuk_anatolij.ua.zhurnalist.writer
%%date 31_03_2023
 
%%tags knorozov_jurij.maja.nauka,civilizacia.maja,pismennost',nauka,1999,1922
%%title Юрій Кнорозов (1922-1999)
 
%%endhead 

\subsection{Юрій Кнорозов (1922-1999)}
\label{sec:31_03_2023.fb.vlasjuk_anatolij.ua.zhurnalist.writer.1.yur_i_knorozov__1922}

\Purl{https://www.facebook.com/anatolij.vlasjuk.9/posts/pfbid0ZGfJuUkC2rWHoPuoR8Z42octZL8C7bZbgp7gKuBUJ5TBJYtsGN8xZc2vnB5FPJZPl}
\ifcmt
 author_begin
   author_id vlasjuk_anatolij.ua.zhurnalist.writer
 author_end
\fi

Його пам'ятають і надзвичайно шанують у Мексиці й Гватемалі. Він випередив
найавторитетніших британських і американських вчених, які були змушені визнати
хибність своїх багаторічних наукових пошуків. Проте на Батьківщині майже забули
про одного з найбільших геніїв XX століття, який розшифрував писемність давніх
майя і в житті якого були дві головні пристрасті - наука і... коти.

\ii{31_03_2023.fb.vlasjuk_anatolij.ua.zhurnalist.writer.1.yur_i_knorozov__1922.pic.1}

31 березня день пам'яті Юрія Кнорозова (1922-1999).
