% vim: keymap=russian-jcukenwin
%%beginhead 
 
%%file 23_07_2021.fb.fb_group.story_kiev_ua.2.skulptor_igor_lysenko.pic.4.cmt
%%parent 23_07_2021.fb.fb_group.story_kiev_ua.2.skulptor_igor_lysenko
 
%%url 
 
%%author_id 
%%date 
 
%%tags 
%%title 
 
%%endhead 

\iusr{Жанна Донец}

Я его и его дочку хорошо знала, ходила на ее экскурсии по Киеву, восхищалась её
знаниями о Киеве, а ее отец был известным режиссером в капеле \enquote{Думка}, шле пела
моя родная тетя 25лет, она им восхищалась, иногда я ходила на концерты, где она
пела, а он режессировал, было это очень давно, но я, как ни странно помню.


\iusr{Георгий Майоренко}
Спасибо. Замечательное воспоминание.

\iusr{Татьяна Гурьева}
Вся семья исключительная. Для Киева много сделали, бесценный культурный вклад.

\begin{itemize} % {
\iusr{Георгий Майоренко}
\textbf{Татьяна Гурьева} 

Семья уникальная! Где, словно знамя, несли традиции киевской интеллигенции.
Совесть, благородство. Увы, таких людей все меньше и меньше, но это люди нашего
круга. Мы сегодня с мамой говорили на эту тему. Она тоже помнила и Игоря, как
моего одноклассника, и его маму. На концерты Натана Рахлина мама ходила
студенткой КПИ.

\iusr{Татьяна Гурьева}
\textbf{Георгий Майоренко} 

я так и поняла, что вы дружили. Ходила на экскурсии и восхищалась ее знаниями.
Моя мама ходила на концерты. Все изменилось.

\iusr{Георгий Майоренко}
\textbf{Татьяна Гурьева} Интересно, что Игорь вначале был в моем классе, а потом перешёл в класс роман чудновский
\end{itemize} % }

\iusr{Борис Кесельман}
На Франка, около Минкульта вроде.
