% vim: keymap=russian-jcukenwin
%%beginhead 
 
%%file 06_11_2021.fb.fb_group.story_kiev_ua.1.bereznjaki_aktery_buchma_brondukov_mykolajchuk.cmt
%%parent 06_11_2021.fb.fb_group.story_kiev_ua.1.bereznjaki_aktery_buchma_brondukov_mykolajchuk
 
%%url 
 
%%author_id 
%%date 
 
%%tags 
%%title 
 
%%endhead 
\subsubsection{Коментарі}

\begin{itemize} % {
\iusr{Елена Сидоренко}

Знаменитые актёры, великие и талантливые люди, жили в обыкновенных
квартирах....

\begin{itemize} % {
\iusr{Наталья Клочкова}
\textbf{Елена Сидоренко} останні роки свого життя Борислав Брондуков проживав в будинку на розі Інститутської і Шовковичноі

\begin{itemize} % {
\iusr{Татьяна Сирота}
\textbf{Наталья Клочкова} В останні роки він проживав у Биківні.Там і помер...

\iusr{Наталья Клочкова}
\textbf{Татьяна Сирота} на будинку меморіальна табличка

\iusr{Татьяна Сирота}
\textbf{Наталья Клочкова}  @igg{fbicon.thinking.face}{repeat=3} 

\iusr{Наталья Клочкова}

\ifcmt
  ig https://scontent-frx5-1.xx.fbcdn.net/v/t39.30808-6/253711637_1774458712753584_7682488256257529163_n.jpg?_nc_cat=111&ccb=1-5&_nc_sid=dbeb18&_nc_ohc=xjVDNV-gyfwAX_M-w_R&_nc_ht=scontent-frx5-1.xx&oh=e05cb4c7888af276664e4b2939097091&oe=61BA512D
  @width 0.5
\fi

\iusr{Елена Сидоренко}
\textbf{Наталья Клочкова} 

я знаю, потом квартиру сдавали, и жили за городом. Жена очень хорошая, и дети
заботились... всё-таки лучше в частном доме...

\iusr{Наталья Клочкова}
\textbf{Елена Сидоренко} так, я знайома з біографією Борислава Брондукова. Великий артист!

\iusr{Татьяна Сирота}
\textbf{Наталья Клочкова} 

Практически полностью парализованный, актер был прикован к кровати последние
шесть лет жизни. Работать он уже, конечно, не мог, и семья оказалась в полной
нищете. Семья вынуждена была переехать в небольшой частный домик в Быковне, а
свою четырехкомнатную квартиру в центре сдать в аренду (возможно это квартира в
том доме, о котором говорите Вы).

Из Википедии:
Дата смерти -
10 марта 2004 г.
Место смерти -
Быковня, Киев.

\end{itemize} % }

\iusr{Elena Kravchenko}
\textbf{Елена Сидоренко} улица Бучмы, но сам актёр жил вовсе не на Берещнякпх, а на пересечении Владимирской и БЖ
\end{itemize} % }

\iusr{Елена Заславская}
17 этажка! И Криницкая жила на 17 а мы с родителями на 9

\begin{itemize} % {
\iusr{Татьяна Сирота}
\textbf{Елена Заславская}  @igg{fbicon.thinking.face}{repeat=3} 

\iusr{Игорь Новус}
\textbf{Татьяна Сирота} 

При всем уважении, вы забыли Алима Федоринского... живёт по Бучмы5а... часто его
вижу... а в молодые годы часто видел Брондукова с ребенком, гуляли на озере...


\iusr{Татьяна Сирота}
\textbf{Игорь Новус} 
Я не забыла! Я об этом не знала ☺ ️ 
Спасибо за дополнение к публикации!

\iusr{Ness Maria}
\textbf{Елена Заславская}
А мы жили на 3- ем этаже.
\end{itemize} % }

\iusr{Ирина Нищимная}
Моя родина, здесь прошло моё детство и юность,,, люблю Березняки,, это особый
рай, родом из детства,,,

\begin{itemize} % {
\iusr{Tamila Leychkis}
\textbf{Ирина Нищимная} и мое тоже. Серафимовича 6

\iusr{Наталия Сорока}
\textbf{Ирина Нищимная} привет, Ирочка! Да, Березняки это наше детство и юность)

\iusr{Ирина Нищимная}
\textbf{Наталия Сорока} Приветик, где ещё встретиться, очень рада, сколько лет не живём, а родина есть родина!
\end{itemize} % }

\iusr{Ол Ткаченко}

Район забудовувався у 70-х роках і ліфти з того часу не мінялися. В 90-х роках
вони часто не працювали, виходили з ладу. На схилі віку в Маргарити Криніциної
ускладнилися проблеми із ногами і вона старенька, живучи під дахом не могла ані
спуститися ані піднятися додому і сиділа днями як ув’язнена у своїй квартирі.

\iusr{Татьяна Сирота}
\textbf{Ол Ткаченко} Да... я про это читала. @igg{fbicon.face.pensive}{repeat=3} 

\iusr{Larisa Lappo}
Спасибо, Таня!

\iusr{Татьяна Зубко Маркина}

Спасибо. Давно, много лет не была. Несколько лет была квартира возле озера.
Какие артисты. Миколайчук, глыба, талант и его трепетная, любящая жена

\begin{itemize} % {
\iusr{Татьяна Сирота}
\textbf{Татьяна Васильевна Зубко} Маркина Она до сих пор живёт в этом доме.

\iusr{Татьяна Зубко Маркина}
\textbf{Татьяна Сирота} прекрасная женщина
\end{itemize} % }

\iusr{Vadim Vadim}
Вышел летом из гостей и упал в Тельбин))!

\begin{itemize} % {
\iusr{Татьяна Сирота}
\textbf{Vadim Vadim} И такое может быть @igg{fbicon.face.upside.down}{repeat=3} 

\iusr{Vadim Vadim}
\textbf{Татьяна Сирота} лето, ночь, жара))

\iusr{Татьяна Сирота}
\textbf{Vadim Vadim} Ну да...\enquote{Тополиный пух, жара, июль...}😉
\end{itemize} % }

\iusr{Вахтанг Кварелашвили}
Лучший район Киева.

\iusr{Татьяна Сирота}
\textbf{Вахтанг Кварелашвили} Мне тоже нравится ☺ ️ 

\iusr{Петр Кузьменко}

Прекрасный пост рассказ о знаковых для Киева и Украины замечательных и любимых
актёрах! И великолепные фотографии уютного района родного Города! Благодарю!
 @igg{fbicon.hands.applause.yellow}  @igg{fbicon.heart.sparkling} 

\iusr{Анна Сидоренко}
Спасибо.

\iusr{Сергій Савелій}

Бучмы, 8, квартира 108, милости просим... Так говорили, коли там жив Боря
Савлохов... На жаль, нині це й місце суму... Адже у березні 2000- го саме біля
цього будинку було вбито Тимура Савлохова, світла їм пам'ять...

\begin{itemize} % {
\iusr{Татьяна Сирота}
\textbf{Сергій Савелій} Да...лихие 90-е.

\iusr{Мария Иванова}
\textbf{Сергій Савелій} , хорошие бандиты были, у богатых, наверное отнимали и бедным раздавали. Робин Гуды

\iusr{Сергій Савелій}
Маріє, Борис - заслужений тренер СРСР, Тимур - заслужений тренер України. Так,
наша Асоціація спортивної боротьби України, президентом якої був Борис
Сосланович, допомагала дитячим будинкам, школам - інтернатам для дітей-
сиріт...

\begin{itemize} % {
\iusr{Мария Иванова}
\textbf{Сергій Савелій} , что не мешало им создать одну из самых больших ОПГ.

\iusr{Сергій Савелій}
\textbf{Mariia Ivanova} Мені здається, що нині ОТГ справжні ОПГ... А те, що пишуть і говорять... То переважно ті, хто і знать не знав ні Боріка, ні Тимура...

\iusr{Мария Иванова}
\textbf{Сергій Савелій} , ну как не знал. Вся коммерция Киева знала, кому платят. Кто Савлоху, кто Киселю, кто Авдышеву.
\end{itemize} % }

\iusr{Андрей Гаврилко}
\textbf{Сергій Савелій} ,живу всю жизнь в этом доме, такого фольклора не слышал))

\begin{itemize} % {
\iusr{Мария Иванова}
\textbf{Андрей Гаврилко} , здрасти, каждой год цветы лежат в месте убийства. Напротив второго подъезда.

\iusr{Андрей Гаврилко}
\textbf{Мария Иванова} ,я про поговорку

\iusr{Сергій Савелій}
\textbf{Андрей Гаврилко} Так говорили борці між собою...

\iusr{Любовь Кулакова}
\textbf{Андрей Гаврилко} 

Я не пам'ятаю який це був день, але пам'ятаю цю ніч. Коли його розстрілювали, а це
було десь о першій ночі, то я прокинулася від грохоту салюту. Тоб то так все
було сплановано. Я Савлохова старшого бачила, його жінку, двоє хлопців. Знала, що
тренер, більш нічого. Це вже пізніше, коли його посадили, почали писати багато чого
про його життя.

Так, в нашому районі проживало і проживає багато знаменитих людей, крім тих що
згадали, ще Могилевська, Бужинська, Зевйо, а ще хлопці з відомих гуртів, але забула
їх назву вже. Може ще хтось згадає про когось.
\end{itemize} % }

\iusr{Ольга Патлашенко}
\textbf{Сергій Савелій} Рада приветствовать, Сергей! Присказка про Савлохова очень понравилась.
А еще на Березняках жил хорошо нам с вами знакомый Леня Вайнерман...

\begin{itemize} % {
\iusr{Сергій Савелій}
\textbf{Ольга Патлашенко} 

Навзаєм! Льоня любив баскетболістів і борців. Завжди проводжав і зустрічав нас
в аеропорту, щоб все було за вищим розрядом!...

\end{itemize} % }

\end{itemize} % }

\iusr{Любовь Добровольская}

Дружина Івана Миколайчука, Марія Евгенівна Миколайчук, теж відома акторка(в
місті Чернівці, на площі біля театру, на алеї зірок, є зірка і на її честь), і
зараз мешкає в тій квартирі. На жаль вона дуже хвора і вже не виходить з
квартири. А раніше, коли мала сили, то виходила і казала, що йде посидіти до
каменюки і має надію дочекатись на пам'ятник (це камінь в сквері, який помпезно
відкрили років 6 тому, і встановили камінь з написом, що незабаром тут буде
пам'ятник Івану Миколайчуку).

Мабуть з такими темпами та Марія Евгенівна того пам' ятника може і не
дочекатись.

А ще трохи і забудовники які мають владу в Київі і сквер забудують.  @igg{fbicon.cry} 

\begin{itemize} % {
\iusr{Людмила Ива}
\textbf{Любовь Добровольская} очень печально...

\iusr{Татьяна Сирота}
\textbf{Любовь Добровольская} Хочеться сподіватися, що пам'ятника дочекається...
\end{itemize} % }

\iusr{Людмила Ива}
Еще здесь проживает знаменитая украинская певица Алла Кудлай!))

\begin{itemize} % {
\iusr{Татьяна Сирота}
\textbf{Людмила Ива} Ух ты!!! Не знала.А где именно?

\iusr{Людмила Ива}
\textbf{Татьяна Сирота} 

на Днепровской набережной, в доме, где раньше размещался, на первом этаже,
Яготинский магазин. Очень часто встречала ее на Днепре, в районе, не доходя до
общего пляжа. Очень приятный человек в общении.)

\end{itemize} % }

\iusr{Лариса Сидоренко}

Жила на П. Тычины 1975-1980гг - новые 16-этажные дома, песок, ветер... очень не
уютно. Мы молодые, нам на работу на правый берег - переезд трамваем через Днепр до
метро - целое дело: конец света! Автобус упал с моста Патона, аварии машин... Мечтали
переехать в центр - переехали. Постарели-какой замечательный зелёный красивый
стал массив! И Днепр рядом! Как бы было хорошо в нем сейчас жить: курорт!!! С
возрастом меняются вкусы и желания.

\begin{itemize} % {
\iusr{Татьяна Зубко Маркина}
\textbf{Лариса Сидоренко} и я прожила несколько лет возле озера, Березняковская, З0, уехавши с Леси..Чудесно, плавай прямо с парадного. Страдала за центром. Теперь Оболонь, не любила. А как теперь радуюсь. Все рядом и озера метро,познаем в сравнении. Удачи
\end{itemize} % }

\iusr{Фарид Степанов}

Моя малая родина: Тычины 3.

Учился в 191 школе. Ходил в кинотеатр \enquote{Десна}. С пацанами бегали купаться на
Тельбин. Когда открывали музей ВОВ, стояли напротив моста Платона и кричали
\enquote{Ура!}.

\begin{itemize} % {
\iusr{Маргарита Рассказова}
\textbf{Фарид Степанов} абрес Оз. Тельбин придумал знаменитый Киевский Архитектор Вадим Михайлович Гечина ,один из авторов Украинского дома ...))

\iusr{Валентина Горковенко-Спицына}
\textbf{Маргарита Рассказова} Гречина В.М. И, наверное, абрИс? Я об этом не знала. Хотя долго работала с В.М. и долго жила на Березняках.
\end{itemize} % }

\iusr{Михайлина Голуб}

Во-первых, огромное спасибо, уважаемая Татьяна, за рассказ о моих Березняках. Я
прожила тут 29 лет, дольше, чем где-либо в своей жизни. Более того, я жила на
Русановке с самого начала ее существования. Мы переехали сюда 14 февраля 1964
года. А училась в первом выпуске первой русановской 182-ой школе. И начиная с
9-го класса у нас учился целый класс с Кухмистерской слободки - деревне в черте
города на месте современныз Березняков. И Березняки появились и росли на моих
глазах. Помню я и печально знаменитое противостояние между молодежью Русановки
и Березняков. Эти драки, заканчивающиеся часто серьезными трав. мами.  И
замечательные березняковские 2 маленьких кинотеатра, такие уютные. Сейчас, увы,
"Десны" уже много лет нет. Осталась только остановка автобуса "Кинотеатр
"Десна". Судьбу "Десны" разделили и гастроном "Харьков", и замечательный
книжный магазин. Дольше всех сопротивлялся магазин "Культовары". Сначала его
все урезали и урезали, а потом и вовсе заменили супермаркетом "АТБ". Зато на
Березняках есть и свой ВУЗ. Вначале был институт усовершенствования учителей, а
потом в этом же здании разместили часть пединститута им. Б.Гринченко.

О Березняках можно говорить еще много и много. Еще раз спасибо, Татьяна, за ваш
пост.

\end{itemize} % }
