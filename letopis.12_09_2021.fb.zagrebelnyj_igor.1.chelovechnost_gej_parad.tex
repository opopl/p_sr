% vim: keymap=russian-jcukenwin
%%beginhead 
 
%%file 12_09_2021.fb.zagrebelnyj_igor.1.chelovechnost_gej_parad
%%parent 12_09_2021
 
%%url https://www.facebook.com/permalink.php?story_fbid=3102981273359838&id=100009439885823
 
%%author_id zagrebelnyj_igor
%%date 
 
%%tags chelovechnost,chelovek,gej_parad,lbgt,ukraina
%%title Цікаве фото з гей-параду у Харкові: з плакатом "Людяність личить усім"
 
%%endhead 
 
\subsection{Цікаве фото з гей-параду у Харкові: з плакатом \enquote{Людяність личить усім}}
\label{sec:12_09_2021.fb.zagrebelnyj_igor.1.chelovechnost_gej_parad}
 
\Purl{https://www.facebook.com/permalink.php?story_fbid=3102981273359838&id=100009439885823}
\ifcmt
 author_begin
   author_id zagrebelnyj_igor
 author_end
\fi

Цікаве фото з гей-параду у Харкові: з плакатом "Людяність личить усім".

Якраз сьогодні думав, чи можна називати панівний ідеологічний дискурс
сучасності антилюдським. З одного боку - так. Бо на прикладі тієї ж
ЛГБТ-ідеології ми бачимо тенденцію до скасування, відміни людини. З іншого
боку, хіба цей дискурс не є продовженням гуманістичної/антропоцентричної
програми? Хіба усе божевілля сучасності не випливає з ідеї "людини як міри всіх
речей"? 

\ifcmt
  ig https://scontent-mxp1-1.xx.fbcdn.net/v/t1.6435-9/241971611_3102950676696231_2216406739998938464_n.jpg?_nc_cat=103&ccb=1-5&_nc_sid=730e14&_nc_ohc=iDWjbsWTBqgAX80F3kk&_nc_ht=scontent-mxp1-1.xx&oh=d3a508e40814f6aa09e2e1f16d7b6ff1&oe=619221EA
  @width 0.4
  %@wrap \parpic[r]
  @wrap \InsertBoxR{0}
\fi

Навіть на гей-парадах, як бачимо, послуговуються безглуздим поняттям людяності.
А що таке "бути людяним"? Що таке "людяність"? Людяність - це, буквально,
якість, властива явищу людини. Але ж люди бувають різними. І якщо відкидати
існування певного арбітра, котрий є вищим за людину, об'єктивної людської
сутності поза за цим різноманіттям не існує. Моральні нікчеми, криваві тирани,
звихнені маніяки - всі вони люди. Та й "пересічні" люди є сумнівним матеріалом,
аби виводити з нього якість-ідеал. Проте уже декілька століть, відкинувши
надійні надлюдські орієнтири, людство прямує за цим оманливим компасом.

Сьогодення рухається в напрямку скасування людини, одночасно продовжуючи
розмахувати прапорцем гуманізму. З точки зору християнської есхатології
воцарювання Антихриста буде тріумфом гуманізму. Але справа в тому, що тоді,
коли Антихрист як утілення людської самодостатності "в Божому храмі сяде як Бог
і за Бога себе видаватиме" (2 Сол. 2:4), від людини може залишитися не так уже
й багато.

\ii{12_09_2021.fb.zagrebelnyj_igor.1.chelovechnost_gej_parad.cmt}
