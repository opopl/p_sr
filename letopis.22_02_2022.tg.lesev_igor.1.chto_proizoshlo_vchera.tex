% vim: keymap=russian-jcukenwin
%%beginhead 
 
%%file 22_02_2022.tg.lesev_igor.1.chto_proizoshlo_vchera
%%parent 22_02_2022
 
%%url https://t.me/Lesev_Igor/321
 
%%author_id lesev_igor
%%date 
 
%%tags minsk_dogovor,putin_vladimir,ukraina
%%title Что произошло вчера
 
%%endhead 
 
\subsection{Что произошло вчера}
\label{sec:22_02_2022.tg.lesev_igor.1.chto_proizoshlo_vchera}
 
\Purl{https://t.me/Lesev_Igor/321}
\ifcmt
 author_begin
   author_id lesev_igor
 author_end
\fi

Что произошло вчера.

Знаете, вчера слушал Путина о ленинской Украине, о декоммунизации
отца-основателя и вот создавалось впечатление, что у ВВП почитывают мою
тележку. Иллюзия, конечно. Но теперь уже не получится сказать, что я озвучиваю
«нарративы Кремля». Хрен вам в попу! Я придумываю нарративы Кремля.

Теперь чуть серьезней. Вчера произошел геополитический слом, который даже
важнее событий февраля-марта 2014 года. Крым – это все-таки эмоциональное
решение. Путина «западные партнеры» кинули в февраля с гарантиями Януковичу и
согласованными досрочными выборами. И случился Крым. Пройди выборы на Украине
«не позднее декабря 2014 года», как и было согласовано сторонами, и сохранись
до этого времени в Киеве Янукович, никакой «Крымской весны» и «возвращении в
родную гавань» не случилось бы. Вы тут все не дураки и сами это понимаете.

К тому же сам по себе Крым всегда был СПОРНЫМ. История его «не такой» передачи
от 1954 года, автономия в составе Украины, мешковская конституция 1992 года,
Черноморский флот, Харьковские соглашения… Крым все 23 года в составе Украины
был пусть и не антиУкраиной, но точно НЕ совсем Украиной.

Другое дело, что «крымская весна» зажгла юго-восточную Украину. Всё это было не
во времена царя Гороха, вы и сами помните, а кто был на местах, еще и знаете
лучше меня, как полыхали Харьков, Запорожье и как жгли в буквальном смысле
Одессу. А Донбасс не просто полыхал, он ревел и метал.

И будь пиздот-элитка с Майдана пусть не семи, но хотя бы одной пяди во лбу, там
должно было родиться понимание, что для бурлящей массы нужно разыграть хотя бы
ВИДИМОСТЬ компромисса. Да, все мы мудрецы задним числом, но на то оно и заднее,
чтобы смотреть панорамно. И я скажу, чего бы хватило, чтобы от Луганска до
Львова сейчас не лежали пацаны в гробах. Не трогайте языковой закон
Кивалова-Колесниченко, не догоняйте «беркутят», не разгоняйте Партию регионов.
Всё. Этого было достаточно.

И что же эти долбоебы сделали? Всё именно наоборот. Они первым делом отменили
языковой закон, тем самым переведя весь конфликт в этнический. Они начали
арестовывать сотрудников «Беркута», и в итоге первыми профессиональными
боевиками в Крыму и на Донбассе стали профессиональные бойцы именно этого
подразделения МВД, которым терять уже было нечего. И они разогнали самую
массовую партию юго-востока, тем самым уничтожив прослойку для мирного выхлопа,
и теперь уже люди от Харькова до Одессы бились не за свою партию и даже не за
свою страну, а размахивали соседским триколором.

А что же в это время Путин? Пацанва в Москве рассчитывала после оскорбительного
плевка с кидком-госпереворотом, рассчитаться Крымом и начать с амерами всё с
чистого листа. Вспоминайте, вспоминайте. Все в открытом доступе. Признание
Петруши в мае легитимным президентом UA. Публичный призыв к Донецку и Луганску
не проводить референдум о независимости. Ведь все это было. Москва готова была
удовлетвориться границами РСФСР до 1954 года и жить с Западом, как жили. Нам
тут Владимир Владимирович вчера рассказывал, что он в 2000-м предлагал Биллу,
разрешившему соснуть Монике, рассмотреть заявку на вступление в НАТО. Так этот
же Владимир Владимирович предлагал и в 14-м забыть «крымский прецедент» и
продолжить жить по-старому.

Но помните, как в «Крестном отце»? «Но ты не просишь с уважением». Путин, по
мнению Вашингтона, просил без уважения, и так появился Минск-1, а потом и
Минск-2, когда русским исходя уже из репутационных соображения таки пришлось
вписаться за ЛДНР. И ведь история появления донбасских республик очень схожа с
появлением Приднестровья. Сначала появилось русское сопротивление снизу, и
только потом уже вынужденно вписался Кремль.
