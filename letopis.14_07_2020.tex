% vim: keymap=russian-jcukenwin
%%beginhead 
 
%%file 14_07_2020
%%parent articles
 
%%endhead 
\subsection{Из школы в армию. Военкомы Украины охотятся на потенциальных абитуриентов }
\url{https://www.facebook.com/groups/LNRGUMO/permalink/2846013405510215/}
  
\vspace{0.5cm}
{\small\LaTeX~section: \verb|14_07_2020| project: \verb|letopis| rootid: \verb|p_saintrussia|}
\vspace{0.5cm}
  
Послекарантинный призыв в армию на Украине оказался скандальным. Военкомы пошли
на хитрость, решив загнать вчерашних школьников в армию и не дав им шанса
получить отсрочку – справку об учебе. При этом потенциальные служивые даже не
могут сбежать за рубеж – границы закрыты 

В ходе весеннего призыва, который в этом году перенесли из-за коронавируса, на
срочную службу призовут почти 16,5 тысяч человек. В Вооруженные силы Украины
(ВСУ) — 9 тысяч, в Нацгвардию — 5460, в Госпогранслужбу — 1300, в
Государственную спецслужбу транспорта — 700 человек. 

Призывная кампания стартовала в мае и продлится до конца июля. Несмотря на то,
что численность призыва сократили более чем на две тысячи человек: пришла
разнарядка из погранслужбы — уменьшить количество. 

Тем временем большого желания идти служить у молодежи не появилось. Хотя в этом
году и увеличили возрастные критерии: теперь в армию кроме призывников 20-27
лет, которых установил еще президент Порошенко, могут по желанию отправляться и
18-19 летние юноши. 

Это и дало военкомам карты в руки: они стали массово призывать вчерашних
школьников служить даже без желания последних. 

\subsubsection{В армию загребают со школьной скамьи }

В связи с карантином весенний призыв в армию перенесли. Соответствующее
распоряжение Правительство приняло еще 26 февраля 2020. Пунктом 5 настоящего
документа предусмотрено, что граждане в возрасте 18-19 лет могут быть призваны
на службу только по собственному желанию и письменному заявлению. 

Но вот уже несколько недель в МОН поступают многочисленные обращения
абитуриентов в ВУЗы и их родителей о том, что их детей военкомы загоняют в
армию, рассылают повестки. 

Военкомы решили воспользоваться тем, что вступительная кампания тоже была
перенесена на август, а весенний призыв продлен до конца июля. То есть те, кто
сегодня закончил школу и кому стукнуло 18 лет, а также выпускники вузов,
получавшие диплом бакалавра, в этот период как бы «зависли в воздухе».
Студентами не стали. И им всем надо было отметиться в военкомате — к великой
радости военкомов. 

16 января 2020 года президент Владимир Зеленский издал указ, по которому
призывать в армию можно граждан с 18 лет. О письменном заявлении на согласие
ничего не сказано. 

В указе говорится, что призыву подлежат те, кто не имеет права на увольнение
или отсрочку (а причиной для таковой, по законодательству, является в том числе
учеба в вузе). Такая же формулировка содержится в законе «О военной обязанности
и военной службе». 

В феврале 2020 года в распоряжении Кабмина было сказано, что призыв граждан в
возрасте 18-19 лет проводить по их желанию (по письменному заявлению). 

По сути, получилась законодательная коллизия. В законе указано, что призыв для
18-летних допускается и ничего нет о письменном согласии, а в распоряжении
Кабмина отмечается, что в армию 18-19-летних будут брать только по их
заявлению, согласию. Поэтому абитуриентам начали приходить повестки до того,
как они смогут подать документы в вузы. 

\subsubsection{МОН в защиту абитуриентов }

Собственно, эта путаница только на руку военкомам, потому что появился шанс
подзаработать. В Черновцах задержали сотрудника военкомата, который за 2700
долларов обещал решить вопрос о непрохождении срочной военной службы
призывником. Первую часть он получил, а когда принесли вторую, его задержали. 

Чем больше перепуганных призывников из числа вчерашних школьников, тем больше
шанс заработать, потому что ими и их родителями легко манипулировать.
Минобразования и науки (МОН) Украины вынуждено было обратиться к Минобороны с
просьбой урегулировать вопрос предоставления выпускникам отсрочки от весеннего
призыва. 

Этот вопрос взял на контроль и Кабинет министров. Премьер-министр Денис Шмыгаль
из-за ситуации с массовым военным призывом выпускников школ обратился в
Министерство обороны и Министерства образования и поручил скоординировать
коммуникацию по этому вопросу. Премьер также подчеркнул, что 18-летние ребята
готовы вступить в высшие учебные заведения, должны такую возможность получить. 

«Им, наверное, не надо вручать срочно повестки Детям надо дать возможность
спокойно подготовиться к поступлению… включитесь в процесс, важно доказать на
места военным комиссарам, чтобы в этом плане была координация и покой», —
заявил глава правительства. 

Из-за переноса сроков вступительной и призывной кампаний Генштаб ВСУ
распорядился организовать взаимодействие комиссариатов с местными органами
образования по призыву. Также военкоматы должны изучить дела призывников и
включить в них заявления о продлении учебы и соответствующие справки из учебных
заведений. 

Местным военкоматам рекомендовано предоставлять отсрочки для 18-19-летних до 1
октября этого года по их заявлениям. 

При этом, как отмечает военный эксперт Сергей Олийнык, военкомы всегда ищут
легкий путь выполнить призыв. 

«Каждый год накануне призыва потенциальные призывники бегут из страны за
границу. В этом году они все сидят по домам из-за карантина. Военкомам нужно
только поработать, пройтись по адресам, разнести повестки. 

Но в то же время к ним стали приходить выпускники колледжей и вузов, которые по
окончании учебы должны явиться в военкомат и получить их печать на обходном
листе. То есть, нежданно-негаданно, к ним [военкомам] в руки сама приплыла
добыча. 

К тому же из-за карантина сбились сроки, что дает возможность манипулировать
законом. Формально, так как призывник не числиться студентом, отсрочка
аннулируется и он подлежит призыву, но фактически, он еще может поступить и
возможно поступит, но известно это будет только в сентябре. А призыв-то идет.
Хотя действительно разумно подождать до осени, но у военкомов планы «горят», —
рассказывает Олийнык. 

На днях Министерство образования позволило приемным комиссиям выдавать
предварительную регистрацию поступающих. Теперь поступающим будут выдавать
документ о намерениях учиться в ВУЗах. Эта справка даст возможность получить в
военных комиссариатах отсрочку до 1 октября 2020 года. 

Но пока ее смогут получить только студенты-бакалавры, решившие учиться дальше
на магистратуре. Школьники и выпускники колледжей еще сдают тесты.

Елена Громова
