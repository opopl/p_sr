% vim: keymap=russian-jcukenwin
%%beginhead 
 
%%file 06_02_2022.stz.news.lnr.lug_info.1.marochko_nedelja
%%parent 06_02_2022
 
%%url https://lug-info.com/news/nedel-ya-glazami-eksperta-voennye-pesni-ze-otgovorki-k-peregovoram-i-novyj-trend-youtebe
 
%%author_id news.lnr.lug_info
%%date 
 
%%tags cenzura,donbass,lnr,marochko_andrej.nm_lnr,nm_lnr,ukraina,youtube,zelenskii_vladimir
%%title НЕДЕЛЯ ГЛАЗАМИ ЭКСПЕРТА: Военные песни Зе, отговорки к переговорам и новый тренд Youtube
 
%%endhead 
 
\subsection{НЕДЕЛЯ ГЛАЗАМИ ЭКСПЕРТА: Военные песни Зе, отговорки к переговорам и новый тренд Youtube}
\label{sec:06_02_2022.stz.news.lnr.lug_info.1.marochko_nedelja}
 
\Purl{https://lug-info.com/news/nedel-ya-glazami-eksperta-voennye-pesni-ze-otgovorki-k-peregovoram-i-novyj-trend-youtebe}
\ifcmt
 author_begin
   author_id news.lnr.lug_info
 author_end
\fi

Своим видением событий прошедшей недели, так или иначе связанных с Луганской
Народной Республикой, с ЛИЦ делится военный эксперт, общественный деятель,
подполковник запаса Народной милиции ЛНР Андрей Марочко.

\ii{06_02_2022.stz.news.lnr.lug_info.1.marochko_nedelja.pic.1}

\subsubsection{НА ЛИНИИ СОПРИКОСНОВЕНИЯ}

Минувшая неделя, несмотря на отсутствие зафиксированных наблюдателями
представительства ЛНР в СЦКК фактов нарушения режима прекращения огня, для
жителей Луганской Народной Республики прошла неспокойно: обстановка на линии
соприкосновения по-прежнему оставалась стабильно напряженной.

Всего с 00:00 часов 21 июля 2019 года режим всеобъемлющего устойчивого и
бессрочного прекращения огня со стороны вооруженных формирований Украины был
нарушен 1021 раз, из них 687 раз – после вступления в силу допмер; 44 защитника
Республики погибли, 24 получили ранения. Среди мирного населения 5 человек
погибли, 38 получили ранения, повреждены 194 объекта гражданской
инфраструктуры.

\subsubsection{ЛОЖЬ ПО КОНТРАКТУ}

Во вторник, 1 февраля, преЗЕдент Украины Зеленский выступил на открытии седьмой
сессии Верховной рады. В ходе этого спича было произнесено много дежурных фраз
и озвучено неизменных тезисов, но, помимо всего прочего, он еще и «порадовал»
некоторыми инсайдами. Например, было озвучено что «украинский верховный
главнокомандующий» подписал якобы дежурный указ, в котором говорится об
улучшении денежного довольствия военных (не ниже трех минимальных зарплат),
увеличении срока службы контрактников, перехода в ближайшей перспективе к
полностью контрактной армии. Также Зеленский приказал за три года увеличить
численность украинский армии на 100 тысяч человек, а в добавок к имеющимся
создать еще 20 бригад.

На сегодняшний день численность украинской армии, по разным оценкам, составляет
порядка 255 тыс. человек. Согласно бюджетному документу, на нужды армии в 2022
году предусмотрено выделение 9,7 млрд долларов. Зеленский планирует за три года
увеличить численность войск примерно на 35 \%, следовательно, военный бюджет
также нужно увеличивать, причем не на 35\%, а намного больше, поскольку
запланировано повышение денежного довольствия. Да еще пан «Зе» хочет полностью
перейти на контрактную службу, отказавшись от дешевой рабсилы в виде срочников.
Их, к слову, сейчас больше половины. Гипотетическая сумма расходов может
достичь более 30 млрд долларов, и эта цифра даже при самых оптимистических
прогнозах в экономике является неподъемной. Допустим, так называемый
коллективный Запад скинется и даст деньги, но где взять материально-технические
средства? Ведь нужно оборудовать ППД, построить казармы, обеспечить жильем
семьи военнослужащих, оснастить всем необходимым подразделения и еще уйма чего.
А где взять 100 000 человек, если даже с нынешним составом укомплектованность
во многих частях не превышает 60\%? Или они думают, когда предполагаемая сумма
денежного довольствия составит 700 американских Джорджей Вашингтонов, все прям
ринутся служить? Да те же «заробитчане», моя туалеты, зарабатывают больше!   

Можно приводить еще уйму доводов в пользу того, что подобные инициативы
скомороха Зеленского воспринимать всерьез не стоит, да и сам инициатор
прекрасно об этом осведомлен. Так зачем это сотрясание воздуха? По моему
мнению, есть несколько причин. Прежде всего это личное обогащение украинского
миллиардера Зеленского, ведь все мы помним, сколько было украдено на стене
Яценюка, а тут возможности несоизмеримо больше. Не следует забывать и о
фактически стартовавшей предвыборной кампании, а «конфетка» в виде контрактной
армии, которой можно заманить избирателя, всегда безотказно работала как на
Украине, так и в других постсоветских странах. Фактор запугивания мнимого
агрессора со щитов тоже сбрасывать нельзя, ведь уже всем очевидно, что помощи
от североатлантического альянса не будет. В общем причин для вранья Зеленского
много, но ложь - это мелкая монета, с ней долго не проживёшь.

\subsubsection{ОТГОВОРКИ ПЕРЕД ПЕРЕГОВОРАМИ}

Уже на следующей неделе состоится очередной раунд переговоров Минской
контактной группы, а также встреча в Берлине советников Н4. В преддверии данных
переговоров официальные лица на Украине сделали ряд заявлений, которые уже
позволяют в некоторой степени спрогнозировать результаты будущих встреч. 

Так, секретарь Совета нацбезопасности и обороны Украины Алексей Данилов в
интервью агентству Associated Press заявил, что попытка реализации Минских
соглашений может привести к беспорядкам в стране. Он добавил, что западные
страны не должны давить на Киев, побуждая выполнить обязательства в рамках этих
договоренностей. По его мнению, это спровоцирует нестабильность и хаос, вплоть
до разрушения Украины. Секретарь СНБО считает, что общество может не принять
эти соглашения, тогда это приведёт к очень сложной внутренней ситуации, а
Москва якобы рассчитывает на это. В связи с этим Данилов призвал начать
переговоры о новом документе.

ПреЗЕдент нэзалежной в ходе встречи с премьер-министром Великобритании Борисом
Джонсоном прокомментировал выполнение Украиной Минских соглашений, заявив, что
Украина ответственно к ним относится, однако у Киева и Москвы есть серьезные
разночтения по отдельным пунктам. \enquote{Мы по-разному относимся к порядку выполнения
пунктов. Но мы взрослые мальчики, мы должны делать что-то, что может
деоккупировать наши территории, что-то защищающее наше государство в том или
ином виде}, - сказал Зеленский.

Министр иностранных дел Украины Дмитрий Кулеба, находясь в Варшаве, дал
интервью польскому изданию Rzeczpospolita, в котором на вопрос о готовности
Украины предоставить особый статус Донецкой и Луганской областям ответил: \enquote{Ни
один регион Украины не получит право вето на какие-либо решения, касающиеся
всей страны. Это написано в камне! Таким образом, не будет никакого особого
статуса, как это себе представляет Россия, не будет права вето}.

Суммируя все сказанное, становится очевидным тот факт, что Украина без зазрения
совести на официальном уровне заявляет, что даже не начинала выполнять
подписанные ими соглашения, которые, к слову, имеют статус международного
документа, поскольку одобрены в Совбезе ООН. Таким образом, это дает полное
право утверждать о несоблюдении Киевом взятых на себя международных
обязательств. За это, согласно нормам международного права, нарушителя должны
привлечь к ответственности, дабы побудить его выполнить обязательства
надлежащим образом, а потерпевшей стороне, в данном случае республикам
Донбасса, должно быть предусмотрено возмещение за причиненный ущерб. Чем дольше
международные арбитры будут игнорировать очевидные вещи, тем сложнее им будет
сохранить свое реноме. Если на следующей неделе не последует никаких действий
за деструктивные разговоры, а этот факт очевиден уже сейчас, то нужно думать о
других способах заставить Киев выполнить свои обязательства, либо вообще
денонсировать соглашения и отпустить Донбасс в свободное плаванье.

\subsubsection{НЕ ЮТУБОМ ЕДИНЫМ}

На этой неделе американская компания Google, а точнее видеохостинг YouTube,
который позволяет загружать и просматривать видео в интернете, «грохнул»
основные ресурсы Донбасса. В одночасье весь видеоряд, который был собран за все
время противостояния был удален. Блокировка крупнейших информационных ресурсов
ЛНР и ДНР на платформе YouTube в очередной раз доказал, что демократия с точки
зрения американцев весьма своеобразное понятие, а практика двойных стандартов
приобрела более выраженную форму. Так широко распространенная США доктрина о
независимых СМИ, которая позволяет любому человеку или учреждению, работающему
в сфере массовой информации иметь свое собственное мнение и право на
независимую оценку происходящего, работает, как оказалось, только под внешним
контролем и влиянием Запада. До этого инцидента у некоторых представителей СМИ
еще были иллюзии о независимости YouTube, и они считали хостинг основной
платформой, которая позволяла доносить правду о происходящем в Донбассе до
всего мира. Теперь все встало на свои места.

Для жителей Донбасса катастрофы, конечно, не случилось. После того, как на
Украине начали блокировать независимые средства массовой информации, которые
освещают события в Донбассе, мы поняли, что в ближайшей перспективе надо ждать
атаку и на информационные ресурсы республик. Многие начали делать «бэкапы»
материалов на видеохостингах, и, как оказалось, не зря. Но тут дело не в
удалении контента, а в причине почему это было сделано. Можно констатировать,
что начата активная фаза информационной войны. Поскольку американцам
практически уже никто не верит, в Вашингтоне решили зачистить информационное
поле для того, чтобы никто не смог подвергнуть сомнению официальную позицию
Белого дома. Также не следует забывать о том, что в условиях гибридной войны
информационная блокада района боевых действий - это одна из ключевых и
завершающих фаз приготовления к военной операции. Простыми словами: попытка
заблокировать распространение объективной информации о происходящем в Донбассе
нужна не только, чтобы заткнуть нам рот, но и прикрыть свои коварные планы.

Однако в США не учли нынешние реалии. Блокировка ведущих информационных центров
Донбасса на видеохостинге YouTube при всем желании не сможет заглушить их
голос. Уже существует масса альтернативных площадок, а также неподконтрольных
американцам соцсетей. Именно на этих платформах начали работать все те, кого
забанили на YouTube. При желании любой желающий может ознакомится с материалом,
а также указать на него ссылку. Так что человечество уже не замыкается на
американском контенте, имеет возможность свободы выбора, и для США данный факт
станет тем гвоздем, который вобьют в крышку гроба.
