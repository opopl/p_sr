% vim: keymap=russian-jcukenwin
%%beginhead 
 
%%file 27_12_2021.fb.fb_group.story_kiev_ua.2.novyj_god_malchik.cmt
%%parent 27_12_2021.fb.fb_group.story_kiev_ua.2.novyj_god_malchik
 
%%url 
 
%%author_id 
%%date 
 
%%tags 
%%title 
 
%%endhead 
\zzSecCmt

\begin{itemize} % {
\iusr{Альбина Орел}
Помню, что мама сама мне шила пачку снежинки и белые тапочки к утреннику)))

\iusr{Наталия Луц}
\textbf{Альбина Орел} мне тоже

\iusr{Наталия Луц}
Мне тоже

\iusr{Lora Berdichevsky}
\textbf{Наталия Луц} и мне @igg{fbicon.face.grinning.smiling.eyes} 

\iusr{Катя Бекренева}
Маленький мальчик символизировал Новый год), к концу года он превращался в Деда Мороза)

\begin{itemize} % {
\iusr{Maksym Oleynikov}
\textbf{Катя Бекренева} 

Такого не пам'ятаю ні тоді, ні пізніше. Було таке, що Дід Мороз із Снігуркою
забавляють дітей, аж тут звляється хлопчик - Новий рік, і це символізувало
очікуваний прихід нового року, і Дід Мороз прощався.... Мабуть, різноманітні
були сценарії.

\end{itemize} % }

\iusr{Ирина Иванченко}
Приємні ваші спогади, до душі, дякую.

Ніколи не користувалася \enquote{казенними}, у дитинстві новорічні костюми шили
самостійно, завжди вигадували щось оригінальне, не як у всіх.


\iusr{Татьяна Кривицкая}

Гарний петрушка. Трохи пізніше Схоже і в мене було. Я була сніжинкою, а
\enquote{чешки} мамуля пошила за ніч, з білоі тканини, плаття прикрасили з мішури
сніжинками. Я була найщасливіша і найкраща сніжинка! А, саме головне, в наступному
році був Мир і молоді батьки.

\iusr{Анна Доманская}

Новорічні спогади з дитинства самі теплі і чудові. Саморобні новорічні костюми і
прикраси на ялинку, все було казкове в дитинстві.

\iusr{Elena Marijchuk}

А мне мама тапочки - балетки ВЯЗАЛА КРЮЧКОМ. Получались САМЫЕ КРАСИВЫЕ.

\iusr{Inna Radzinsky}

Дякую за приємні спогади. Мені того ж 1959 року в дитячому садку на вулиці Анрі
Барбюса довелося бути Снігуронькою. Незважаючи на те, що я була бодай
найменшого зросту в групі. Мене вповноважили на цю посаду тому що я була здатна
запам'ятати і проголосити досить довгу новорічну промову. Біло-блакитний
кожушок і такий самий капелюшок були оздоблені ватою з безліччю блискучих
плямок з потовчених кольорових кульок. Десять хвилин акторської слави в
невимовно красивому одязі - один з найяскравіших спогадів дитинства.

\begin{itemize} % {
\iusr{Maksym Oleynikov}
\textbf{Inna Radzinsky} 

Інночко, ти й у дорослому віці була аж ніяк не дилдою....
@igg{fbicon.face.smiling.eyes.smiling} 

\iusr{Наталя Будзинська}
А мене ось так відягнули в садку в 1958

\ifcmt
  ig https://scontent-frt3-1.xx.fbcdn.net/v/t39.30808-6/270317983_4523673071075751_8336366262052736821_n.jpg?_nc_cat=104&ccb=1-5&_nc_sid=dbeb18&_nc_ohc=eiBh2RBsqqwAX-QsWH-&_nc_ht=scontent-frt3-1.xx&oh=00_AT8C6q4AXmaypwDKHyRWpP1U5mBppYUcSe3mWNmtBnpjYQ&oe=61D3C636
  @width 0.3
\fi

\end{itemize} % }

\end{itemize} % }
