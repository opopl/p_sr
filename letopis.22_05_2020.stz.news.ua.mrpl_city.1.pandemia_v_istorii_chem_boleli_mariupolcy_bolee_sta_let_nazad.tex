% vim: keymap=russian-jcukenwin
%%beginhead 
 
%%file 22_05_2020.stz.news.ua.mrpl_city.1.pandemia_v_istorii_chem_boleli_mariupolcy_bolee_sta_let_nazad
%%parent 22_05_2020
 
%%url https://mrpl.city/news/view/pandemiya-v-istorii-chem-boleli-mariupoltsy-bolee-sta-let-nazadputi-zarazheniya-i-sposoby-vyzdorovleniya
 
%%author_id nevska_oleksandra.mariupol,news.ua.mrpl_city
%%date 
 
%%tags 
%%title Пандемия в истории: чем болели мариупольцы более ста лет назад – пути заражения и способы выздоровления
 
%%endhead 
 
\subsection{Пандемия в истории: чем болели мариупольцы более ста лет назад – пути заражения и способы выздоровления}
\label{sec:22_05_2020.stz.news.ua.mrpl_city.1.pandemia_v_istorii_chem_boleli_mariupolcy_bolee_sta_let_nazad}
 
\Purl{https://mrpl.city/news/view/pandemiya-v-istorii-chem-boleli-mariupoltsy-bolee-sta-let-nazadputi-zarazheniya-i-sposoby-vyzdorovleniya}
\ifcmt
 author_begin
   author_id nevska_oleksandra.mariupol,news.ua.mrpl_city
 author_end
\fi

Олександра Невська | 22 травня 2020 в 09:00

Сегодня во всем мире наиболее актуальными остается вопрос борьбы с пандемией
коронавируса и возможные способы преодоления болезни. \href{\urlDemidkoIA}{Историк и преподаватель
Ольга Демидко} выяснила, какие заболевания \enquote{ударили} по мариупольцам в XIX и XX
веке и поделилась эффективными советами того времени, о том, как сохранить и
укрепить свое здоровье.

\enquote{Благодаря данным журнала \enquote{Северный архив} и \enquote{Межевой книги}, частью которой
стало \enquote{Камеральное описание города Мариуполя}, удалось узнать численность
населения города в 1823 (3354 человека) и 1826 (2998 человек) годах. То есть за
эти три года мужчин и женщин умерло больше, чем родилось на 356 человек. Такой
высокий уровень смертности в городе был результатом различных болезней,
носивших эпидемический характер}, – рассказала историк.

По ее словам, на высокую смертность мариупольцев в первой половине XIX века
повлияли также смена климата и воды, условия труда и быта, особенности питания
и жилищных условий.

Как отметил мариупольский краевед Николай Григорьевич Руденко, причиной многих
заболеваний у мариупольцев были и некоторые семейные обычаи. К примеру,
распространенным был обычай дарить одежду умершего ребенка другим детям без
учета того, от какой болезни он умер. Такой обычай нередко мог поспособствовать
возникновению эпидемий кори, скарлатины, дифтерии.

\enquote{Не следует забывать и о длительном отсутствии медицинской помощи в Мариуполе,
что повлияло на тяжелое протекание наиболее распространенных на то время
заболевнаий, таких как холера, брюшной и сыпной тифы, лихорадка и туберкулез.
Но несмотря на появление профессиональных врачей и открытие в 1874 году земской
больницы, эпидемии продолжали свирепствовать}, – подчеркнула \href{\urlDemidkoIA}{Ольга Демидко}.

Она уточнила, что в 1879 году эпидемия дифтерии унесла почти тысячу жизней.
Мариупольцы, чтобы спастись от смерти часто прибегали к  помощи знахарей и
бабок, молились перед своей святыней –  иконой  Божьей Матери или же  в храме,
построенном в честь Харлампия – целителя от повальных болезней. С открытием
земской больницы ее персонал начал вести активную пропаганду правил санитарии и
гигиены в городе, что повлияло на уменьшение эпидемий.

Отмечается, что наиболее часто повторялась эпидемия холеры, что подтолкнуло
городские власти к созданию первого в городе так называемого \enquote{заразного}, то
есть инфекционного отделения в больнице, которая находилась при заводе
\enquote{Никополь}. Позднее холерный барак был постороен и отдельно. Когда эпидемии
холеры повторились в 1907-м и  в 1910-м годах, многих мариупольцев спасло
наличие холерного барака. В это время заболело почти 3 700 человек, из которых
1 900 умерли.

По словам историка, холера, дифтерия и тиф отнимали жизни у мариупольцев только
во время  эпидемий. В свою очередь, туберкулез стал зловещим спутником местных
жителей, начиная с первых лет существования города. Многие историки и краеведы
считали его главной болезнью Приазовья. Врачи тоже акцентировали внимание на
особую распространенность туберкулеза среди греков.

В годы гражданской войны заболевание туберкулезом едва не перешагнуло
эпидемический порог. Но в 50-е годы в Мариуполе начинают вести системную и
организованную борьбу с болезнью: открылись два туберкулезных диспансера и
специализированная туберкулезная больница, в которой оказывали помощь
высококвалифицированные специалисты.

В 1960–1980 годах заслуженную славу уникального хирурга – фтизиатра высочайшего
класса приобрел главврач туберкулезной больницы – кандидат медицинских наук
Петр Михайлович Твердохлеб, на операции к которому приезжали жители других
стран.

\enquote{В Адрес-Календаре \enquote{Весь Мариуполь и его уезд} от 1910 года (издание С. А.
Копкина) приводятся медицинские сведения, в которых содержатся наставления, как
мариупольцам сохранить свое здоровье. Так, \enquote{самым верным способом сохранения
здоровья во всякое время, а особенно во время различных эпидемий считался
правильный образ жизни}}, – подчеркнула \href{\urlDemidkoIA}{Ольга Демидко}.

Авторы наставлений, которые были написаны 110 лет назад, когда в городе
свирепствовали эпидемии тифа и холеры, подчеркнули, что \enquote{очень важно не бояться
заболеть и сохранять бодрое настроение}.

\emph{\textbf{Кроме этого главного правила, врачи того времени давали такие наставления:}}

\begingroup
\em

1. Не следует резко изменять привычный образ жизни.

2. Не следует выходить из дома утром, не позавтракав и оставаться долго без еды.

3. Следует сохранять умеренность в пище и питье, при этом соблюдать небольшую
осторожность в отношении доброкачественности пищи.

4. Избегать жирной, недостаточно проваренной, слишком холодной пищи.

5. Умеренно употреблять овощи, зелень, фрукты.

6. Не следует работать до чрезмерной усталости как физической, так и
умственной, ложиться спать пораньше и спать не менее 8 часов.

7. Не гулять в сырых местах, особенно после заката солнца.

8. Следует соблюдать чистоту тела.

9. Необходимо всегда относиться с особым вниманием к своему здоровью.

\endgroup
