% vim: keymap=russian-jcukenwin
%%beginhead 
 
%%file 21_06_2011.stz.mrpl.old_mariupol.1.semnadcat_let_spustja
%%parent 21_06_2011
 
%%url http://old-mariupol.com/semnadcat-let-spustya
 
%%author_id mrpl.old_mariupol,jaruckij_lev.mariupol
%%date 
 
%%tags 
%%title СЕМНАДЦАТЬ ЛЕТ СПУСТЯ
 
%%endhead 
 
\subsection{Семнадцать лет спустя}
\label{sec:21_06_2011.stz.mrpl.old_mariupol.1.semnadcat_let_spustja}
 
\Purl{http://old-mariupol.com/semnadcat-let-spustya}
\ifcmt
 author_begin
   author_id mrpl.old_mariupol,jaruckij_lev.mariupol
 author_end
\fi

\begin{quote}
\em Через семнадцать лет после посещения Пушкиным Мариуполя здесь побывал другой
русский поэт — Василий Андреевич Жуковский, тот самый, который молодому автору
\enquote{Руслана и Людмилы} подарил свой портрет с надписью: \enquote{Победителю ученику от
побежденного учителя}. 	
\end{quote}

Недавно в Москве вышел солидный труд \enquote{Жуковский} с выразительным
подзаголовком \enquote{Книга о великом русском поэте}.

Эпитет \enquote{великий} в применении к Жуковскому может нашему современнику
показаться чрезмерным, но огромен его вклад в отечественную поэзию и, по слову
Белинского, велико его значение в русской литературе.

Беспримерный успех выпал на долю его \enquote{Певца во стане русских воинов},
наиболее яркого поэтического отклика на Отечественную войну 1812 года. В те
дни, когда Пушкин учился в начальных классах Лицея, а Лермонтова еще не было на
свете, Жуковский совершенно справедливо считался первым поэтом России. Это и
предопределило придворную карьеру поэта, без рассказа о которой невозможно
объяснить, каким образом Жуковский попал в Мариуполь.

Сначала Василия Андреевича пригласили ко двору в качестве чтеца Марии Федоровны
— вдовы Павла I, затем он стал учителем русского языка великой княгини
Александры Фе­доровны (прусской принцессы Шарлоты) — жены великого князя
Николая Павловича. Через несколько лет Николай Павлович стал Николаем I, а
когда старшему сыну его исполнилось 8 лет, Жуковского назначили наставником
наследника престола — будущего Александра II.

Десять лет руководил Василий Андреевич воспитанием и обучением своего питомца,
а когда цесаревич достиг совершеннолетия, был устроен экзамен, на котором
присутствовал государь император. Венценосный папа остался доволен успехами
сына и выразил пожелание, чтобы будущий хозяин земли русской совершил
путешествие по стране, посмотрел Россию и себя показал.

Запрягали долго — целых два года: цесаревич в сопровождении Жуковского и свиты
выехал из Петербурга только 2 мая 1837 года. Но в течение этих двух лет были
отданы многочисленные распоряжения и к встрече наследника готовились с особым
тщанием. Маршрут, разумеется, был разработан заранее — самим Жуковским, и
таким образом, стало известно, что путешественники побывают и в Мариуполе.

По этому случаю у таганрогского градоначальника (в административном подчинении
которого находился в то время Мариуполь) волнений и хлопот было предостаточно.
Он даже послал в подведомственный ему город своего полицейского чиновника
Сесемана для при­ведения в порядок дороги и тротуаров, для наблюдений \enquote{за
сохранением чистоты и опрятности}. Донесения Сесемана из Мариуполя рисуют нам
весьма неприглядную картину:

\begin{quote}
\em\enquote{Церковная ограда вокруг собора обвалилась, в ограде нечистоты и неопрятности.
Ули­цы имеют рытвины и выбоины, а в иных лес и прочее недолжное, близ собора
площадка в ямах и буграх, дома во многих местах требуют починки, крыши ветхи,
по крайней мере необмазаны, трубы развалились и нигде не побелены, также во
многих домах нет стекол, а в других забиты дощечками или залеплены бумагой,
заборы каменные обвалились и необмазаны, деревянные разрушились или редко где
сделаны через доску и около обросли бурьяном, во многих местах нет ворот, а в
других, хотя и есть, изломаны и ничто не окрашено. Ряд требует починки,
штукатурки и побелки и в конце оных ветхие лавчонки угрожают падением, а около
оных поставлены бочки с дегтем, с дручками и кучами лубья и прочая нечистота и
неопрятность}.
\end{quote}

Кроме того, дотошный Сесеман, заботясь о безопасности цесаревича, проверил
также переправу через Кальмиус и установил, что у парома нет цельного каната:
он составлен из связанных кусков. Город наспех привели в порядок. И вот 17
октября 1837 года путешественники подъезжают к Мариуполю. В карете цесаревича
ехал и Жуковский.

Василию Андреевичу в то время было уже за пятьдесят. Как выглядел он в 1837
году, когда посетил Мариуполь, мы знаем по знаменитому портрету Карла Брюллова.
Это тот самый портрет, который Василий Андреевич, высоко ценя талант и личность
Т. Г. Шевченко и желая помочь Тарасу Григорьевичу, предложил написать Брюллову
с целью разыграть его в лотерею в царской семье. На вырученные деньги и был
выкуплен из крепостной неволи великий украинский поэт. Даже если бы Жуковский
ничего больше не совершил в своей жизни, одним только своим участием в судьбе
Тараса Григорьевича он заслужил бы нашу вечную благодарность и добрую память.

Но он совершил еще многое, обессмертившее его имя и как поэт (\enquote{Его стихов
пленительная сладость пройдет веков завистливую даль} — Пушкин), и как
царедворец, использовавший свое влияние, чтобы смягчить участь декабристов,
опального Герцена и многих других деятелей русской культуры. Это он в 1820
году, когда Пушкину грозила ссылка в Соловецкий монастырь или даже в Сибирь,
уговорил Александра I заменить ее службой в южных губерниях. И, следовательно,
ему мы обязаны тем, что 29 мая 1820 года судьба забросила Пушкина в Мариуполь и
этот город, таким образом, стал пушкинским местом страны.

И правильно сделали мариупольцы, что поэта Жуковского 17 октября 1837 года на
руках носили. Да-да, если мы скажем так: \enquote{на руках носили}, то будем
очень недалеки от истины. Правда, как и 29 мая 1820 года, жители города,
встречая прославленного героя Отечественной войны генерала Раевского, и не
подозревали, что юноша в одной из колясок — Александр Сергеевич Пушкин, точно
так же 17 октября 1837 года, бурно выражая свой восторг по случаю приезда в их
город наследника престола, они не знали, что человек, сидящий в одной карете с
цесаревичем, — выдающийся поэт России.

Между тем в Мариуполе греческий суд в полном составе во главе со своим
председателем Чентуковым, купцом третьей гильдии, и толпа празднично одетых
обывателей ждали высокого гостя у въезда в город (то есть у нынешнего здания
жилсоцбанка, ибо в то время здесь, у церкви Марии Магдалины, была западная
граница города). Когда вдали показался поезд карет, сопровождаемый длинным
шлейфом пыли, мариупольцы отрепетированно закричали \enquote{ура}, а как только
передний дормез (спальная карета), поравнявшись с толпой, остановился,
Чентуков, чтобы быть услышанным в восторженном реве толпы, в самое ухо шепнул
что-то царедворцу, сидевшему рядом с цесаревичем (это был Жуковский), и,
получив разрешающий кивок, подал условный сигнал. Тотчас же встречающие
бросились к лошадям и, суетясь и мешая друг другу, распрягли их. Затем в порыве
усердия приподняли было первую карету, чтобы нести ее на руках, но тяжел был
огромный дормез, Чентуков запрещающе замахал руками. Тогда карету снова
опустили на мягкую дорогу (Екатерининскую улицу замостят только через 44 года)
и осторожно покатили под уклон к Базарной площади.

На ней в то время строился грандиозный собор. Он будет освящен только через
восемь лет и назван Харлампьевским, а пока Харлампьевским собором была
маленькая неказистая церковка. Вот к ней-то и подкатили дормез цесаревича
мариупольцы, осторожно одерживая (как строго наказывал на репетициях Чентуков),
чтобы не дай Бог не раскатать ее на спуске. Цесаревич в сопровождении
Жуковского и всей свиты немедленно отправился в церковь служить эктинию (по
словарю: \enquote{заздравное молений о государе и доме его}), а Чентуков
поспешил к себе на Торговую, где накрывали торжественный завтрак.

Отслушав эктинию, цесаревич, по совету Жуковского, узнавшего предварительно,
что до дома председателя греческого суда рукой подать, отказался от кареты и
решил пешком прогуляться по городу и таким образом, как сказано в старинном
документе, \enquote{прибыл в на­значенную его высочеству в доме председателя
сего суда 3-й гильдии купца Чентукова квартиру, где встречен хозяином дома и
гражданами города с хлебом-солью, изволил тут завтракать и пожаловал хозяину
дома золотую табакерку...}.

Такой подарок оставался в каждом городе, который посещал Цесаревич, но в
Мариуполе Александр Николаевич, по совету Жуковского, дал еще для бедных людей
200 рублей, \enquote{да особо приказал — раздать отставным солдатам и прочим бедным до
150 рублей ассигнациями}.

Пока цесаревич со своей свитой завтракал у Чентукова, в кареты перезапрягли
лошадей (на лугу у Кальмиуса на тот случай держали наготове целый табун отлично
подкованных умельцами с Кузнечной улицы лошадей) и подали на Торговую.
Путешественники спешно покатили дальше, так как условлено было, что Александр
должен в назначенный срок прибыть в станицу Аксайскую, чтобы встретить там
своего венценосноного родителя, возвращавшегося из поездки по Кавказу.

Несмотря на многотрудную деятельность Сесемана, найти новый канат для парома
так и не удалось, но все обошлось благополучно, и вереница роскошных карет без
приключений добралась до Успеновки, казачьего хутора на левом берегу
Кальмиуса. Далее кортеж покатил по земле области Войска Донского, атаманом
которого числился цесаревич, и Жуковский деликатно напоминал своему
воспитаннику историю мест, которые мелькали за окном дормеза.

* * *

Отразилось ли посещение Мариуполя и Приазовья в творчестве Жуковского?

Василий Андреевич вел дорожный дневник. Записи делал не развернуто, а
отрывочно, конспективно. Вот несколько строчек:

\enquote{17 (октября). Переезд из Орехова в Таганрог. Завтракали в Мариуполе.
Греческая живость лица женщин. Приезд весьма поздний в Таганрог. Заблуждение.
Мой дом у головы прекрасный.

18 (октября). Осмотр Таганрога. Дворец Александров (зажженные днем свечи).
Больница. Гимназия. Выставка (сафьян, сапоги...). Переезд из Таганрога в
Аксайскую станицу. Таганрог, начавший расцветать город и могущий пасть. Две
стороны Воронцова. (Из-за конкуренции только что основанного графом Бердянска.
— Л. Я.) Взгляд на Нахичевань, армянский город по проезде через крепость св.
Дмитрия. Прелестные головки армянок в окнах. Кавалькада из греков вплоть до
Аксая}.

Стихов он к тому времени уже не писал, но поэтом оставался, и не заметить
\enquote{прелестные головки и красоту мариупольчанок с их \enquote{греческой
живостью лица}} Василий Андреевич, конечно, не мог.

\textbf{Лев Яруцкий.}

\textbf{\enquote{Мариупольская старина}}
