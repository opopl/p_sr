% vim: keymap=russian-jcukenwin
%%beginhead 
 
%%file 15_12_2021.stz.news.ua.den.1.schedryk_vystavka
%%parent 15_12_2021
 
%%url https://day.kyiv.ua/uk/news/151221-vidoma-mystkynya-zakoduvala-shchedryk-u-vytynankah
 
%%author_id palamarchuk_pavlo
%%date 
 
%%tags 
%%title Відома мисткиня закодувала «Щедрик» у витинанках
 
%%endhead 
\subsection{Відома мисткиня закодувала «Щедрик» у витинанках}
\label{sec:15_12_2021.stz.news.ua.den.1.schedryk_vystavka}

\Purl{https://day.kyiv.ua/uk/news/151221-vidoma-mystkynya-zakoduvala-shchedryk-u-vytynankah}

\ifcmt
 author_begin
   author_id palamarchuk_pavlo
 author_end
\fi

В Національному музеї у Львові імені Андрея Шептицького  відкрилася різдвяна
виставка витинанок Дарії Альошкіної «Щедрик». В кількаметрових ажурних паннах
художниця передала усю магію Різдва навіяну всесвітньовідомою колядкою Миколи
Леонтовича.

\ii{15_12_2021.stz.news.ua.den.1.schedryk_vystavka.pic.1}

Витинанки художниці вражають як своєю величиною, так і деталізацією. Свої
витинанки майстриня робить винятково вручну. Особливість її техніки  –
симетрія: так само як колись складали аркуш паперу та вирізали фігури й узори,
вона складає величезні полотна навпіл і витинає орнамент. Потім розгортає
полотно, і воно, як фантастична сніжинка, перетворюється на філігранне
мереживо.

Дарія Альошкіна першою почала створювати витинанки величезних розмірів.
Мисткині-новаторці вдалося не лише відродити давнє українське ремесло
витинанки, але й перетворити його на сучасне мистецтво і представити цілому
світу. Ці ажурні панно стали точкою тяжіння безлічі виставок і арт-проєктів у
Польщі, Німеччині, Франції, Південній Кореї, Канаді, США, Бельгії, Австрії,
Швейцарії, Данії. Усі орнаменти – авторські до останнього штриха.

\ii{15_12_2021.stz.news.ua.den.1.schedryk_vystavka.pic.2}

За свою творчу біографію мисткиня створила сотні квадратних метрів полотен. Усі
візерунки витинанок – неповторні, бо виконані на основі «живого»
імпровізованого малюнка-ескізу. Вона першою звернулася до експериментів зі
сучасними матеріалами, що дало змогу формувати з декоративних полотен справжні
інсталяції, театральні декорації та прикрашати ними публічний простір.

Варто зазначити, що цього року Дарію Альошкіну внесли у міжнародний гід від
фонду The Michelangelo Foundation for Creativity and Craftsmanship. Це
міжнародна некомерційна організація, метою якої є належне визнання  ручної
праці та підтримка тих, хто робить це найкраще.

Оглянути виставку витинанок «Щедрик» можна буде до 13 січня.

Павло Паламарчук
