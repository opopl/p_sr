% vim: keymap=russian-jcukenwin
%%beginhead 
 
%%file topics.vojna.my.3.gaaga.voprosy.otvety.detail.99.luzheckii
%%parent topics.vojna.my.3.gaaga.voprosy.otvety.detail
 
%%url 
 
%%author_id 
%%date 
 
%%tags 
%%title 
 
%%endhead 

\paragraph{22:16:43 25-08-22 Анатолий Лужецкийreplied to ivan}

Очень эмоциональные, честные и откровенные слова.   Несколько раз от начала и
до конца прочитал твои посты. Если бы я не знал, что ты украинец, подумал бы
что это писал донбассовец. Только если кое-что подправить и поставить с головы
да на ноги. Ну во первых... никто не собирается уничтожать Украину и украинцев.
Никто  ни тебя, ни твоих родственников никогда не отправит ни в газовую камеру,
ни в концлагерь. Никто не отберёт у вас украинский язык и вашу культуру. Ты
напрасно думаешь, что мы не знаем и не ценим ваших классиков. "Щедрик",
которого ты упомянул - это хоровое произведение украiнського хорового митця
Миколи Леонтовича. В основі цього твору лежить проста українська народна
колядка. Изумительное произведение поистине мирового значения. У него есть ещё
"Зашумiла лiщинонька"," Праля", "Козака несуть" та багато чого... Мы знаем и
Николая Лысенко, и Лесю Украинку и т.д.  Памятник  Тарасу Шевченко стоит в
Донецке. Его именем назван бульвар, рядом с которым находится школа номер 22,
которую вы недавно расхерачили американскими снарядами, убили учителей. Всё как
70 лет назад. Моя крёстная, которая родилась в Ровно и детство которой прошло в
одной из ровенских деревень, рассказывала как в их селе бандеровцы убили
учительницу. Ничего не поменялось... И наука и культура и всё прочее - это да,
это никуда не денешь. Только появилось одно "Но"... Я слушать украинские песни
самые раскрасивые и самые расдушевные уже  не могу. Я вырос на них, но слушать
сейчас не могу... Это так же, наверно, как тот дедушка, который единственный
выживший в беларусской Хатыни, не мог переносить немецкую речь, уходил, когда в
Хатынь приезжали немецкие туристы. У меня просто в голове и в душе не
совмещаются эти невероятно красивые и душевные украинские песни со зверствами
украинских националистов на Донбассе, ваших "героев". Времени и места здесь не
хватит всё рассказать. Если бы их матери знали,  что их детки здесь творили и
сейчас творят, они бы их прокляли... Поэтому, Иван, если ты говоришь об
украинской душе, то в этом плане у меня большие сомнения, что она у вас есть.
Или осталась.
