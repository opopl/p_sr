% vim: keymap=russian-jcukenwin
%%beginhead 
 
%%file 26_11_2021.fb.medvedeva_olesja.2.prezident_ludi
%%parent 26_11_2021
 
%%url https://www.facebook.com/olesia.medvedieva/posts/1747442985446816
 
%%author_id medvedeva_olesja
%%date 
 
%%tags chelovek,rnbo,sankcii,ukraina,zelenskii_vladimir
%%title "Вы их людьми называете, да?", - спросил президент
 
%%endhead 
 
\subsection{\enquote{Вы их людьми называете, да?}, - спросил президент}
\label{sec:26_11_2021.fb.medvedeva_olesja.2.prezident_ludi}
 
\Purl{https://www.facebook.com/olesia.medvedieva/posts/1747442985446816}
\ifcmt
 author_begin
   author_id medvedeva_olesja
 author_end
\fi

Зеленский усомнился в том, что подсанкционные (люди которые попали под санкции
СНБО) - это вообще люди.

"Вы их людьми называете, да?", - спросил президент. И пустился в такие
рассуждения:

"Есть представители людей, не все представители людей являются людьми. Есть
особи. Я так считаю. Я не обо всех говорю. Да, это мое личное мнение. Потому
что, если я не прав, то не существовало бы правоохранительных органов, а все
люди жили бы в мире".

Он заявил, что раньше Совбез был "пассивным" органом, а сегодня это "мощный
орган".

По словам Зеленского, санкции вводились, к примеру, против тех, кто "нарушил
правила на границе". Хотя таких оснований в законе о санкциях нет (санкции
могут применяться только к тем гражданам Украины, которые причастны к
терроризму, а не а контрабанде или ещё к чему-то).

Также он прямо признал, что санкции вводятся без решений судов, где была бы
доказана вина людей. Потому что, мол, суды это долго.

То есть фактически он прямо расписался в преступлении - узурпации власти.
Потому что никто не может ограничивать права граждан Украины без решения суда.

Это п\#\#здец

\ii{26_11_2021.fb.medvedeva_olesja.2.prezident_ludi.cmt}
