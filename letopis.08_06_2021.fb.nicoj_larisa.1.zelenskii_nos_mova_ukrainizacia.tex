 vim: keymap=russian-jcukenwin
%%beginhead 
 
%%file 08_06_2021.fb.nicoj_larisa.1.zelenskii_nos_mova_ukrainizacia
%%parent 08_06_2021
 
%%url https://www.facebook.com/nitsoi.larysa/posts/930196510880055
 
%%author Ницой, Лариса
%%author_id nicoj_larisa
%%author_url 
 
%%tags jazyk,mova,nicoi_larisa,tkachenko_aleksandr,ukraina,ukrainizacia,zelenskii_vladimir
%%title Пане Зеленський, уся країна спостерігає, як ваша політична сила має вас в носі
 
%%endhead 
 
\subsection{Пане Зеленський, уся країна спостерігає, як ваша політична сила має вас в носі}
\label{sec:08_06_2021.fb.nicoj_larisa.1.zelenskii_nos_mova_ukrainizacia}
\Purl{https://www.facebook.com/nitsoi.larysa/posts/930196510880055}
\ifcmt
 author_begin
   author_id nicoj_larisa
 author_end
\fi

Пане Зеленський, уся країна спостерігає, як ваша політична сила має вас в носі.
Ви обіцяли винести питаня мови за дужки. Натомість, за останні два роки ваша
політична сила, всупереч вашій обіцянці, лише те й робить, що штурмує Верховну
Раду мовними питаннями. Ви, пане Зеленський, або брехун, або зовсім не керуєте
своєю політичною силою.

Ваш міністр культури України ткаченко і ваш голова гуманітарного комітету ВРУ
потураєв хочуть захистити і створити комфортні умови бикам, які зневажають
державну мову. І міністрові, і голові комітету, обом, чужа і далека біда
українськомовних - бути щодня дискримінованими в Україні. 

Ваші два державні високопосадовці (а ще є ваші бужанські, разумкови, тощо)
педалюють скасування/відтермінування штрафів  за українську мову, але, давайте
розберемося, що таке ці штрафи?

Українськомовний громадянин в Україні має право на отримання послуг ДЕРЖАВНОЮ
мовою, але часто наштовхується на грубість, відмову і зневажливе ставлення. 

Штрафи запроваджуються не проти російськомовних, а  стосуються ЛИШЕ конфліктних
агресивних людей, які зневажають українську мову і українськомовних українців.

Штраф - це механізм, який убезпечує  українськомовних від дискримінації. Штрафи
захищають українськомовних від зневажливого ставлення і упереджують конфлікти.

Той, хто проти такого механізму захисту - той ЗА дискримінацію, ЗА конфлікти,
ЗА зневажливе ставлення до державної мови...  

Я особисто дискутувала і з Разумковим, і з Бужанським. А остання моя розмова
була з вашим міністром культури та з вашим головою гуманітарного комітету.  

- Пані Ларисо, прийдіть на ефір ТБ, - дзвонить редактор Наташі Влащенко.
- Ні, дякую.
- Будуть міністр культури ткаченко і голова комітету ВРУ гуманітарної політики потураєв.
- Тоді прийду.
Ефір. Безліч разів я  зверталася до ткаченка і потураєва:
- Якщо ви, ткаченко і потураєв, доб'єтеся відтермінування штрафів - як ви збираєтеся убезпечити українськомовних від дискримінації?
Ткаченко:
- Ми не повинні примушувати російськомовних...
- Згодна, ми не будемо їх примушувати, але питання не про НИХ, а про українськомовних. Як ви збираєтеся убезпечувати українськомовних від дискримінації?
Потураєв:
- Ми повинні створити умови для російськомовних вивчити мову.
- Згодна, але питання не про російськомовних. Поки вони вчитимуть мову, як Ви будете убезпечувати українськомовних від дискримінації?
Ткаченко:
- Але російськомовні такі самі громадяни України, як і ви, пані Ларисо.
- Згодна, вони громадяни, але питання не про російськомовних. Як ви будете убезпечувати українськомовних від дискримінації?
- Держава повинна забезпечити російськомовних...
- Ви тільки те  робите, що дбаєте про російськомовних, але як ви будете убезпечувати українськомовних від дискримінації?
Відповіді, як захистити українськомовних громадян, яких більшість в Україні, від дискримінації, я так і не отримала.
Пане Зеленський. Ви руками своїх посадовців продовжуєте дискримінувати українськомовних громадян в Україні. Янукович уже дискримінував. І Петро Олексійович зрозумів важливість українського питання під кінець каденції, та було вже пізно. Як гадаєте, якими будуть наслідки дискримінації українців для вас?
