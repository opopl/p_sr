% vim: keymap=russian-jcukenwin
%%beginhead 
 
%%file 15_10_2020.fb.lnr.1
%%parent 15_10_2020
%%url https://www.facebook.com/groups/LNRGUMO/permalink/3122384544539765/
 
%%endhead 

\subsection{Бандеровцы вывели бронетехнику на улицы Львова: Грозят Днем УПА в Донецке и Луганске }

\url{https://www.facebook.com/groups/LNRGUMO/permalink/3122384544539765/}
\index[authors.rus]{Тарас Стрельцов}

14 октября во Львове прошли мероприятия в честь «Дня защитника Отечества»,
празднование которого после победы Евромайдана на Украине было перенесено с 23
февраля на годовщину создания УПА, сотрудничавшей с гитлеровцами в годы Великой
Отечественной войны.

По центру города проехала колонна военной техники 30-х годов и состоялся марш
ряженых самостийников под украинскими и бандеровскими флагами, передаёт
корреспондент «ПолитНавигатора».  

Люди, переодетые в форму украинских
коллаборационистов, под дробь барабанов прошли мимо памятника Шевченко, где
были запланированы торжественные мероприятия. Возле импровизированной сцены под
памятником, помимо действующих украинских боевиков АТО, стояли люди, наряженные
в стилистике разных времён – от Киевской Руси и казачества, до петлюровского и
бандеровского периода.

«В современно украинско-российской войне это актуально как никогда и очень
нужно, чтобы люди знали, кто их защитник», – сказал один из боевиков.

Кроме того, женщина, переодетая в соответственную форму, исполняла песни,
прославляющие УПА. Также выступали дети в шароварах с «боевым гопаком».

«Слава Богу, что этот праздник стал и праздником защитника Украины, – говорил
наряженный в форму капитана 2 ранга экс-боевик нацистского батальона «Айдар»,
меджлисовец Тимур Баротов. – Потому что раньше нам навязывали чужие идеологии,
чужие истории и чужие праздники».

Он рассказал, что украинская история уходит далеко вглубь веков – «трипольская
культура, скифы, сарматы, а потом княжеский период и казацкий период», где так
же, «как в демократических странах избирали гетмана», и могли его свергнуть,
«если что-то не так».

«Сначала нам надо победить в этой войне с заклятым врагом, с которым мы
боролись все эпохи и все времена – с Московской империей. И, когда эта победа
настанет, а я в этом не сомневаюсь, тогда я смогу всем пожелать мира, потому
что мира без победы не бывает», – добавил Баротов.

Между выступлениями дети дарили ораторам, среди которых были и современные
украинские боевики, цветы.  «Очень сегодня приятно и радостно. И хотелось бы
видеть, чтобы мы праздновали не только во Львове, не только в Киеве, но и в
Донецке, и в Луганске», – сказал участвовавший в бомбёжках Донбасса комбат 16-й
отдельной бригады армейской авиации «Броды» подполковник Виктор Андрийчук.

