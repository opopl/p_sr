% vim: keymap=russian-jcukenwin
%%beginhead 
 
%%file 25_12_2021.fb.fb_group.story_kiev_ua.2.pediatria.cmt
%%parent 25_12_2021.fb.fb_group.story_kiev_ua.2.pediatria
 
%%url 
 
%%author_id 
%%date 
 
%%tags 
%%title 
 
%%endhead 
\zzSecCmt

\begin{itemize} % {
\iusr{Алла Парадня}

Такие люди есть всегда, но чем выше достаток, тем меньше сочувствия и
понимания. \enquote{Страшно далеки они от народа}, как говорил когда-то ещё
Ленин.  Деньги большая проблема для человека и когда их нет, и когда они есть.
Много ли их было, таких, как Бродский.

\begin{itemize} % {
\iusr{Ирине Вильчинская}
\textbf{Alla Paradnya} 

в Киеве - много. Когда-то Виталий Ковалинский, который один из первых поднял
тему благотворительности и меценатства в Киеве, проработал огромные массивы
архивных материалов, рассказывал на лекциях, что в Киеве КАЖДЫЙ 7-й его житель
был БЛАГОТВОРИТЕЛЕМ. Это не значит, что всем им были под силу такие масштабные
\enquote{проекты}, как сказочно богатым семьям Бродских, Терещенко, Ханенко,
Могилевцеву, Дехтяреву, Галаганам и другим, но каждый вносил свою, посильную
лепту в дело милосердия и благотворительности. Киев был городом
купеческо-коммерческим, у многих были свои \enquote{бизнесы} и считалось делом чести
помочь тем, кто не так удачлив, кому не так \enquote{фартит}. Сам Л. Бродский, отвечая
на вопрос, почему он столько тратит на благотворительность, отвечал так. \enquote{Я
делаю это не потому, что сам хочу, а потому, что я знаю, что это делать нужно}.
Кроме всего прочего, в обществе царила мораль и преступать ее было неприлично.
Это, как раз, в продолжение темы о социальных элитах, которые задают тон в
обществе своим примером, делами, а не пустыми лозунгами и враньем. Тогда и
общество, в целом, здорово...

\begin{itemize} % {
\iusr{Тетяна Мариношенко}
\textbf{Irina Vilchinskaya} 

так і книга вельмишановного В. Ковалінського \enquote{Меценати Києва} просто
неймовірна. Та і взагалі, його дослідження, присвячені меценатству безцінні.

\iusr{Ирине Вильчинская}
\textbf{Tetyana Marinoshenko} 

Так, я прослухала усі його лекції (ще до виходу книги), була просто у захваті
від його праці, наснаги, терпіння та бажання донести до людей майже втрачені
знання своєї історії, повернути забуті імена, події і сам дух того часу - з
його складностями, несподіваними поворотами доль людей і шляхетними чеснотами,
які люди зберігали та цінували більше, ніж матеріальні статки. А наостанок, на
останній лекції вже була готова його книга \enquote{Меценати Києва} і я
отримала його власноручний підпис на примірнику. Дуже шкода, що така людина вже
відійшла у засвіти. Нехай там зустрінеться з усіма своїми героями! А ще його
праця та мій особистий досвід роботи у \enquote{Шоколадному будиночку} привели
мене до зовсім іншого сприйняття \enquote{купечества}, ніж ми звикли за часів
радянських. Нам показували \enquote{чернуху}, зворотній бік на прикладах
\enquote{дозволених} цензурою літературних героїв Островського- неосвічених,
темних, деспотичних \enquote{глитаїв}, які висотували кров з народу та
влаштовували \enquote{домострой} у своїх родинах. Та сором язливо замовчували
усі ті здобутки, які вносили ця працелюбна, підприємлива частина суспільства,
яка прийшла на зміну досить поблажливої та зніженої частини дворянства.
Фактично, купецтво стало тим \enquote{середнім класом}, який підніс економіку
тогочасної імперії до незрівняних висот. Окрім того, отримавши певні привілеї,
діставшись до можливостей отримувати знання,освіту,вони вчилися самі та вчили
своїх дітей, переймаючи найкращі зразки культурного досвіду Європи. Таквони не
тільки зміцнюють економіку держави, вони створюють чудові зразки матеріальної
культури - ми зараз милуємося залишками тієї краси, потрапляючи до дивом
збережених особняків. Вони стають меценатами, які підтримують вітчизняних
митців- художників, музикантів, співаків. А згодом, саме з їхнього середовища
виходять вже \enquote{технократи}-капіталісти, які будують не тільки заводи і
фабрики, але й навчальні заклади для підготовки своїх, вітчизняних, кадрів для
них!От такий ланцюжок вимальовується!

\iusr{Тетяна Мариношенко}
\textbf{Irina Vilchinskaya} 

а я ще у1992 на його статтях будувала уроки історії для своїх дітей і писала
роботу для конкурсу \enquote{Учитель року-2000}. І завжди включала матеріал про
меценатів в уроки, навіть, коли про це ще ніхто не говорив. І усе завдяки пану
Віталію Ковалинському. Була вражена цим матеріалом на початку 90-х. Доземний
уклін цьому науковцю. На жаль, його немає вже з нами. І з його спадщиною
неприємності трапилися

\end{itemize} % }

\iusr{Ольга Морозова}
\textbf{Алла Парадня} Было много - погуглите.

\iusr{Алла Парадня}
\textbf{Ольга Морозова} было

\end{itemize} % }

\iusr{Ольга Морозова}
Спасибо и вечный покой тем, кто когда-то творил добро. Сейчас, к сожалению, таких нет.

\iusr{Ірина Дебкалюк}

Дякую за допис! Гарну справу робите, пані Галина! «Мы ленивы и нелюбопытны»
А. С. Пушкин... Дізнатись щось нове з історії свого міста, людей, що мали до
нього причетність чудово!


\iusr{Віка Виктория}

Дааа, среди сегодняшних ново-миллионеров - нет \enquote{бродских}.... А за наши деньги
налогоплательщиков - и то толком ничего не строят, только воруют ....

\iusr{Лариса Кушниренко}
Может нужно назвать улицу его именем?

\begin{itemize} % {
\iusr{Галина Полякова}
\textbf{Лариса Кушниренко} Ви праві. Родина Бродських заслуговує на нашу вдячну пам'ять. Вони зробили чимало для киян!
\end{itemize} % }

\iusr{Мария Константиновская}
Спасибо! Напомнили... @igg{fbicon.thumb.up.yellow}  @igg{fbicon.sunflower}
@igg{fbicon.musical.notes} 

\iusr{Татьяна Першина}

Нижний Вал,33, Подол:

«...благотворительные организации, содержащиеся на средства подольских купцов:
бесплатный родильный приют с курсами бабок-повитух Общества оказания
медицинской и материальной помощи бедным роженицам (1905–1914 гг.) и Первое
отделение \enquote{Капель молока}. В последнем неимущим матерям бесплатно выдавалось
молоко для детей до двух лет. А если мать как следует присматривала за
ребенком, ей могли продлить услуги.»


\iusr{Anatoliye Anatoliy}
Thanks for yours publication

\iusr{Анна Мазманян}

\ifcmt
  ig https://i2.paste.pics/d6985d166d31c2ee6beb6c2c3af63946.png
  @width 0.2
\fi

\iusr{Алла Сыч}

\ifcmt
  ig@ name=scr.hands.applause
  @width 0.2
\fi


\end{itemize} % }
