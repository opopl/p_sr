% vim: keymap=russian-jcukenwin
%%beginhead 
 
%%file 04_01_2019.stz.news.ua.lb.1.arhitekturnyj_atlas_dorevoljucijnogo_mariupolja.16.vodonapornaja_bashnja
%%parent 04_01_2019.stz.news.ua.lb.1.arhitekturnyj_atlas_dorevoljucijnogo_mariupolja
 
%%url 
 
%%author_id 
%%date 
 
%%tags 
%%title 
 
%%endhead 

\subsubsection{Водонапорная башня}

Одним из главных символов Мариуполя является водонапорная башня. Строительство
тридцатиметровой башни со смотровой площадкой для пожарных, по проекту Виктора
Нильсена завершилось в 1910 году. Но уже в 30-х годах она утратила свои
первоначальные функции. В эпоху индустриализации был сооружён новый водопровод
и насосная станция. В 60-е город стал сильно расти вверх, и наблюдательный пост
пожарников тоже стал бесполезным. Во время войны башня была повреждена, но
капитального ремонта дождалась только в середине восьмидесятых, когда возникла
идея расположить в ней отделение краеведческого музея. Идея эта так и не
осуществилась, но постепенно водонапорная башня превратилась в узнаваемый
символ Мариуполя. С 1996 по 2013 г.г. в ней располагалось отделение
Приватбанка, а в 2016 году в башню ненадолго подселили отделы муниципалитета.
Наконец-то к 240-му юбилею Мариуполя здание отреставрировали и придумали
достойное назначение. Сейчас здесь функционирует креативное публичное
пространство. Интересно отметить, что, несмотря на богатейшее индустриальное
прошлое, водонапорная башня единственный в Мариуполе узнаваемый архитектурный
объект инженерного назначения.

\ii{04_01_2019.stz.news.ua.lb.1.arhitekturnyj_atlas_dorevoljucijnogo_mariupolja.16.vodonapornaja_bashnja.pic.1}
\ii{04_01_2019.stz.news.ua.lb.1.arhitekturnyj_atlas_dorevoljucijnogo_mariupolja.16.vodonapornaja_bashnja.pic.2}
\ii{04_01_2019.stz.news.ua.lb.1.arhitekturnyj_atlas_dorevoljucijnogo_mariupolja.16.vodonapornaja_bashnja.pic.3}
\ii{04_01_2019.stz.news.ua.lb.1.arhitekturnyj_atlas_dorevoljucijnogo_mariupolja.16.vodonapornaja_bashnja.pic.4}
\ii{04_01_2019.stz.news.ua.lb.1.arhitekturnyj_atlas_dorevoljucijnogo_mariupolja.16.vodonapornaja_bashnja.pic.5}
\ii{04_01_2019.stz.news.ua.lb.1.arhitekturnyj_atlas_dorevoljucijnogo_mariupolja.16.vodonapornaja_bashnja.pic.6}

