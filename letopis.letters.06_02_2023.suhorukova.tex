% vim: keymap=russian-jcukenwin
%%beginhead 
 
%%file letters.06_02_2023.suhorukova
%%parent letters.06_02_2023
 
%%url 
 
%%author_id 
%%date 
 
%%tags 
%%title 
 
%%endhead 

Добрый день, Надежда!

Очень здорово, что Вы вернулись в фейсбук!

Простите, не представился раньше. Меня зовут Иван, я программист из Киева.
Недавно Вы в чате написали, чтобы подписчики накидали красивых фоток Мариуполя.
У меня к сожалению нет моих собственных фоток, потому что я никогда сам не был
в Мариуполе.  Тем не менее, я бы тоже хотел поделиться с Вами своими работами
по Мариуполю.

Что я имею в виду?

Кроме моей основной работы, я также на волонтерских началах уже довольно
длительное время занимаюсь разработкой методов, а также программного
обеспечения для записи по датам, авторам и категориям самых разных публикаций в
интернете. А зачем это нужно вообще? - Мне всегда нравилось читать, читать
много, книжки, статьи, публикации - и я в какой то момент решил для себя, что
не нужно просто читать, а нужно также записывать то, то есть, сохранять файлы,
что мне интересно, чтобы оно не пропало, и всегда можно было вернуться назад и
прочитать снова.

Когда началось вторжение... я решил записывать, то, что происходило и происходит
доныне - в первую очередь, не только мои впечатления, которые и так у меня в
голове - а именно то, что пишут другие люди. В общем, получился проект Летописи
Войны, которым я занимаюсь в свободное от работы и других дел время. Пока что
он еще в довольно начальной своей фазе, потому что объем информации по войне
просто невообразимо огромен, а я пока что еще занимаюсь этим сам по себе, когда
есть время.  Технология, которую я использую, называется LaTeX (произносится
Л-а-т-е-х). Это фактически отдельный язык программирования документов. Это
довольно старая технология, ей уже более 30 лет, но она повсеместно
используется в научном мире учеными (физиками, математиками) для записи и
публикации своих статей.  Преимущество LaTeX над, например, Microsoft Word, в
том, что вы в принципе можете записать в один документ сколько информации
пожелаете, хоть сто страниц, хоть тысячу. А как это работает в научном мире, вы
можете увидеть например, здесь https://www.arxiv.org - это архив научных
препринтов, там миллионы статей по самым разным областям науки.

И что касательно Мариуполя. Я сейчас собираю, записываю, разные посты о
Мариуполе (и не только о Мариуполе). Это нужно делать, особенно с учетом
ужасной политики фейсбука.  Недавно я записал полностью пост Натальи Дедовой от
10.01.2023 по списку погибших (130 страниц получилось). Также, практически
близка к завершению работа по посту об историях (тоже Натальи Дедовой).  Это
важные нужные посты о трагических событиях в Мариуполе, но они требуют больше
времени для обработки и проверки практически каждой строчки. Также, к счастью,
в свое время я натолкнулся на Ваш пост в фейсбук от 28 мая 2022, и я его тоже
успел записать в свое время.  Это единственная из Ваших публикаций из Вашего
аккаунта, которую мне удалось сохранить (уже выложена в телеграм-канале, со
всеми комментариями и фото).

А здесь я бы хотел поделиться небольшими записанными постами о довоенном
Мариуполе, прикладываются ниже.

И замечу - что ни фейсбук, ни телеграм, ни инстаграм, не являются надежными
хранилищами информации.  То, что пишется, нужно записывать, записывать, и еще раз
записывать, сохранять.

А касательно трагической судьбы Мариуполя... Знаете, я из Киева.  Я тут
родился, вырос, учился... И скажу как киевлянин... что нет на Земле Города...
более трагичного, и более прекрасного, чем Киев... Его столько раз жгли,
разоряли, уничтожали...  Судьба киевлян, обретших последнее пристанище под
куполами Десятинной Церкви, вскоре упавшими на их головы от последнего натиска
дикой монгольской Орды, крушившей все вокруг...  в далеком 1240 году, в чем-то
так схожа с трагичной судьбой мариупольцев, погребенных под завалами
Драмтеатра... Киев много раз переживал то, что пришлось уже в наше время
пережить Мариуполю...  Но, несмотря на все, Киев снова возрождался... И снова,
и снова... как птица-феникс, восставал из пепла, каждый раз становясь все краше
и краше... Чудо-Город на берегах Днепра... Мариуполь... Місто Марії - Город
Богородицы... Киев тоже имеет право носить это звание, ведь уже почти 10 веков
Оранта смотрит на нас с высот Софиевского Собора... Так что... я думаю... все
будет хорошо. Мариуполь воскреснет, засияет и вновь и снова, своими новогодними
трамваями, муралами и фонтанами...  И будет еще краше, еще чудеснее, чем был до
вторжения этой безумной злой дикой и лживой современной орды...

В телеграм у меня есть канал с ссылками @kyiv_fortress_1 - там намного больше
всего, постепенно добавляю туда новые записи.

С уважением,

Иван.

%07_12_2021.fb.vyshedska_masha.bahmut.hudozhnyk.illjustrator.1.mariupol__chast_2__v.1_4
%17_01_2023.fb.fb_group.mariupol.pre_war.2.mar_upol____stor_ya_.1_3
%20_01_2023.fb.fb_group.mariupol.pre_war.4.mar_upol___murali__g.1_13
%29_12_2022.fb.wishnevskaja_helen.mariupol.1.a_pomnite_etot_tikhi.1_6
