% vim: keymap=russian-jcukenwin
%%beginhead 
 
%%file 09_02_2022.stz.news.lnr.lug_info.1.konkurs_tehnika_molodezh.pic.7
%%parent 09_02_2022.stz.news.lnr.lug_info.1.konkurs_tehnika_molodezh
 
%%url 
 
%%author_id 
%%date 
 
%%tags 
%%title 
 
%%endhead 


\ifcmt
  tab_begin cols=2,no_fig,center

     pic https://storage.lug-info.com/cache/2/b/05e01685-f756-427f-aaa6-654efb290a45.jpg/w1000h616%7Cwm
     pic https://storage.lug-info.com/cache/7/0/adf138a5-4508-4d9c-a554-24cfbf868233.jpg/w1000h616%7Cwm

  tab_end
\fi
