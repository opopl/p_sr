% vim: keymap=russian-jcukenwin
%%beginhead 
 
%%file slova.banderovec
%%parent slova
 
%%url 
 
%%author 
%%author_id 
%%author_url 
 
%%tags 
%%title 
 
%%endhead 
\chapter{Бандеровец}

Почему-то современные \emph{бандеровцы}, которым принадлежит вся власть в западных
регионах страны, очень стесняются собственной истории. Надписи на памятниках
жертвам \enquote{буржуазных националистов} замазывались цементом, целые страницы
прошлого вырваны с мясом, чтобы напоказ гордиться незначительными эпизодами.
Например, про период бурной любви с немцами и деятельное участие националистов
во всех мероприятиях по укреплению оккупационной власти принято отнекиваться и
яростно доказывать, что такого не было никогда. Но так же яростно убеждать в
войне \enquote{на два фронта}, для которой примеров как-то маловато,
\textbf{Современные бандеровцы любят говорить, что их предки воевали на два фронта},
Дмитрий Заборин, strana.ua, 06.05.2021

Если только не считать войной такие эпизоды, как убийство пяти сержантов 1-й
гв. танковой бригады в с. Иваниковка Богородчанского района. Дочь хозяина хаты,
где спали бойцы, навела \emph{бандеровцев}, которые застрелили четверых, а пятого
ранили. Он пытался бежать, но его догнали и добили лопатами. После всех кинули
в скотомогильник и забросали землей.  За это гражданку Ганну Ковтык
\enquote{отправили в Сибирь} - так в документе 1985 года. Правда, к тому
моменту она оттуда уже вернулась, как ни странно при лютой жестокости совитив,
о которой тоже модно ввернуть что-нибудь эдакое,
\textbf{Современные бандеровцы любят говорить, что их предки воевали на два фронта},
Дмитрий Заборин, strana.ua, 06.05.2021


