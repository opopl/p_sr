% vim: keymap=russian-jcukenwin
%%beginhead 
 
%%file 26_11_2022.fb.trymbach_sergij.1.smert_jak_voskresinnja
%%parent 26_11_2022
 
%%url https://www.facebook.com/sergiy.trymbach/posts/pfbid0VSN5Cd8B8YrhbQSsFH4soZjadj24oULLSS8XNhDJiE2GF9HvEY8iMCisggq3xUkSl
 
%%author_id trymbach_sergij
%%date 
 
%%tags dovzhenko_oleksandr,istoria,kino,kultura,ukraina
%%title СМЕРТЬ ЯК ВОСКРЕСІННЯ
 
%%endhead 
 
\subsection{СМЕРТЬ ЯК ВОСКРЕСІННЯ}
\label{sec:26_11_2022.fb.trymbach_sergij.1.smert_jak_voskresinnja}
 
\Purl{https://www.facebook.com/sergiy.trymbach/posts/pfbid0VSN5Cd8B8YrhbQSsFH4soZjadj24oULLSS8XNhDJiE2GF9HvEY8iMCisggq3xUkSl}
\ifcmt
 author_begin
   author_id trymbach_sergij
 author_end
\fi

СМЕРТЬ ЯК ВОСКРЕСІННЯ

66 літ тому, 25 листопада 1956-го, в Москві пішов з життя ОЛЕКСАНДР ДОВЖЕНКО.

Смерть була несподіваною... І чи ж випадковою? Через два тижні Довженко мав
летіти до Парижа, куди його запросили (серед найбільших кіногеніїв світу -
Чарлі Чапліна, Жоржа Ренуара, Роберто Росселіні...) на святкування ювілею
Французької синематеки. Упродовж 26 років (від 1930-го) великого митця не
випускали за кордон. А тут... Є версія, яку я уперше почув від племінника
Довженка Тараса Дудка (Царство Небесне!): є щось підозріле у тому, як поводили
себе у той трагічний листопадовий вечір лікарі, та й сама Юлія Солнцева,
дружина. Адже бездіяльність у момент сильного серцевого нападу зрозуміло чим
закінчується. Чи ж не інспірована та бездіяльність спецорганами, яким так не
хотілось випускати Довженка у Францію. Мало що він там наговорить... 

Чи так було насправді ми вже ніколи не дізнаємось. Подібні лиходійні акції
документальних слідів не полишають. Та все ж маємо право на версії, припущення...  

Той трагічний листопадовий день 1956-го дружина митця Юлія Солнцева
згадувала так. Це була неділя, вони знаходились на дачі, у підмосковному
Передєлкіно. У Довженка боліло серце, він поривався їхати до Москви, адже
наступного дня мали відбутись акторські проби до фільму «Поема про море»
(його потім зніме сама Солнцева, а Микита Хрущов накаже присудити Довженку
– посмертно – найвищу тоді нагороду, Ленінську премію). 

«Мы сели в машину. Молчали. Ему было нехорошо. Вошли в подъезд. «Сейчас будет
лифт», — предложила лифтерша. — «Не надо. Я не могу ждать». Сашко тяжело
поднялся по лестнице. Войдя в дом, он сейчас же лег на кровать. Ему
становилось все хуже и хуже. Я вызвала лечащего врача...» 

«Это был плохой врач...»

\enquote{...Я вызвала лечащего врача. Это был плохой врач, тридцатилетняя девица, ничего
не видящая вокруг себя}. 

Уколи не допомогли. \enquote{Дівиця} заходилася викликати професора, одначе той
відмовив. Тоді Солнцева викликала  іншого лікаря, який тривалий час у минулому
лікував Довженка. Той приїхав. Тільки тоді була викликана \enquote{швидка допомога}.
Іван Андрійович, так звали лікаря, \enquote{пришел недовольный и раздраженный. Таким я
его не видела  давно. Спросил врача, что было сделано за это время, посмотрел
лекарства и взял руку Сашко, чтобы найти пульс.  Пульса не  было... минута
молчания, и он бросил эту руку на грудь бедного Довженко. Бросил, не положил
[...] Поняв, что Довженко уже не стало, он быстро вышел из нашей квартиры, не
сказав никому ни слова}

В цій оповіді найдивніше те, що пов'язано з лікарями.  Солнцева пише, що не раз
виводила Довженка із ситуацій украй загрозливих для здоров'я, а тут ніяк не
скажеш, що подібний досвід у неї був. Знала, що лікар з \enquote{дівиці} слабенький й
однак покладалася на неї? У такій-от ситуації? Адже пише: лікарка по телефону
говорила професорові, що той вмирає... Хоч правдою є й те, що серце Довженкове
було зношене і змучене багатолітніми хворобами.

Усе ж забагато дивних подробиць тих останніх годин страдницького
Довженкового життя. Можна зрозуміти, що йдеться про жінку, в якої помирав
чоловік. Та можна зрозуміти й сестру Довженка, Поліну Дудко і її  родину, де
усі були лікарями (окрім  Поліни Петрівни і її чоловіка, Миколи Дудка,
лікарями стали й обидва їх сини, Тарас і Олександр). Багато разів
вислуховуючи розповідь про смерть Довженка, вони, у підсумку, не могли не
запідозрити чогось лихого. Ну, надто вже плутано. Що само по собі не означає
якогось злого умислу. Просто, якщо справді було все так, як описується, то
слід визнати: кваліфікованої, дієвої допомоги хворий не дістав.

„А что было раньше?    Он лежал с закрытыми глазами. Розовое пятно на
левой щеке исчезло. Исчезла последняя надежда.  Кровь перестала двигаться
даже по этим сосудам. Довженко лежал с закрытыми глазами. Очевидно, это
конец. Но неожиданно он заметался и открыл глаза. Остановил свой взгляд на
мне. Он пытался что-то сказать, но не мог, и опять попытался, и наконец
тихо произнес: «Юля, ты так много сделала мне в жизни». Это были его
последние слова, и он сам уже понимал, что умирает. Поняла и я. Он опять
заметался и закрыл навсегда свои добрые глаза». 

Чи справді такими саме були останні Довженкові слова – хто знає? Усі ми
грішні і всім хочеться історію, принаймні власну, бодай трохи прикрасити
якимись красивими подробицями. 

За тим було те, що буває у подібних випадках. Почали сходитись люди...
„Уже наша квартира заполнялась людьми: Борис Андреев, Каплуновский
(художник і режисер.- С.Т.), Авдеенко (режисер-документаліст, працював з
Довженком і Солнцевою над документальними стрічками.- С.Т.), Марьямов
(кінодраматург і критик, близько знав Довженка ще замолоду.- С.Т). Я
ничего не видела и не слышала, только потом, гораздо позднее, Борис
Федорович Андреев рассказал мне, как мы готовили безудержного Сашко к
вечному покою: сняли пижаму и одели его единственный костюм (другого не
было), а потом уложили на стол. «Я держал его за голову,- говорил
Андреев.- Голова у него была еще теплая. Я прижал его к себе, и мне
показалось, что эта уходящая теплота Довженко идет по всему телу». Что
было потом, мне трудно представить.  «Сашко! Сашко! – кричало все во мне.-
Я одна!»

На рахунку в ощадбанку у Довженка було 32 рублі

Коли наступного дня  заступник міністра культури РРФСР В.Сурін запитав у
Солнцевої про її побажання, вона, за її ж власним свідченням, сказала, що
варто назвати Київську кіностудію Довженковим іменем. Й одразу „вспомнила,
что у нас нет денег и на что же мы будем хоронить Сашко. На его
сберегательной книжке за №... стояло 32 рубля (...) Эту сберегательную
книжку  я оставила себе на память. В свое время я забыла о ней и только
через 20 лет нашла ее в папках. Немного для великого художника».  

Солнцева зателефонувала до Києва – вона хотіла виконати волю свого Сашка бути
похованим в Україні. Миколи Бажана, тоді голови Спілки письменників, не було
вдома, вона передала своє прохання і моління подзвонити їй. Не подзвонив
ніхто... Солнцева потому складала провину за невиконання заповіту на Бажана та
Корнійчука. Хоча навряд чи вони могли піти проти Москви – саме остання не
бажала бачити «націоналіста» у Києві. Втім, і керівництву України такий
головний «біль» не був потрібен так само. 

Похоронний обряд відбувався у Домі літераторів по вулиці Воровського, 50 –
27-28 листопада. Продовжився на Новодівичому... З України приїхала невелика
делегація на чолі з письменником Василем Минком. Із крупних діячів культури не
приїхав ніхто. Це правда, на жаль. Якою бути панахиді вирішував, головним
чином, великий співак і Довженків друг Іван Козловський. Він заспівав улюблене,
на Шевченкового вірша, \enquote{Мені однаково, чи буду Я жить в Україні, чи
ні...} та \enquote{Чуєш, брате мій...} На скрипці грав уже знаменитий тоді
Леонід Коган...

Українські письменники привезли сніп жита, вузлик рідної землі і яблука –
все це вони поклали в ноги небіжчику. На цвинтарі Козловський висипав ту
жменьку землі в могилу, на гріб, і мовив: \enquote{Земля, по якій твої ноги ходили,
нині теплом тебе приймає}.  Від українців говорив Василь Минко. У спогадах
він напише потім, що промову склали утрьох – він, Василь Козаченко та
Олександр Підсуха. \enquote{Ми, - промовляв Минко над розритою могилою,- твої
близькі друзі і земляки, привезли тобі найсердечніші почуття  любові і
великого смутку. Сьогодні, разом з усім радянським народом над твоєю могилою
сумують мати-Україна, твоя улюблена Десна, де ти народився, уперше побачив
світ сонця, тужать Славута-Дніпро, Київ і Харків, Запоріжжя й Каховка [...]
Земно кланяючись, ми привезли тобі найщирішу подяку за все, що  зробив ти
для великого українського народу, для його культури й мистецтва. Наш народ
безмежно любить, шанує й цінує тебе як художника-живописця, зачинателя
українського кіно, визначного письменника-новатора. Наш народ пишається
тобою за те, що ти уславив Україну на весь Радянський Союз, на весь
світ[(...] Прощай, соколе ясний, товаришу вірний}. Знайшли потрібні слова...

„Довженко,- зацитую ще раз спогади Солнцевої,- хоронили за государственный
счет из-за отсутствия денег в нашем доме. Было именно такое постановление,
а также постановление о памятнике, который должен был быть поставлен на
Новодевичьем кладбище и который стоит там сейчас. Делала этот памятник
большой друг Сашко  Вера Игнатьевна Мухина (один із найвідоміших
радянських скульпторів, автор знаменитої скульптурної групи „Робітник і
колгоспниця”.- С.Т.). Я как-то сначала не могла привыкнуть к этому
памятнику, мне всегда казалось, что Довженко другой – и лоб, и глаза, а
главное – выражение. Но впоследствии я свыклась. Делала Вера Игнатьевна
бюст Довженко еще при жизни Александра Петровича, но умерла она раньше
Довженко (у 1953 р.- С.Т.).[...] Теперь мне кажется, что  Довженко был
именно таким, каким его видела В.Мухина. Рабочая гипсовая скульптура стоит
сейчас в комнате Сашко в нашей квартире. Она уже потемнела от времени, на
ней остались следы моих слез и тихих поцелуев. А иногда я вижу следы
губной помады, которые стараюсь поскорее стереть, чтобы никто не заметил».

«Я сьогодні ранком полетів на Україну...»

Довженко вмирав довго – ще від кінця 1943-го, як тільки стало зрозумілим,
що за кіноповість «Україна в огні» його просто розіпнуть на
більшовицько-сталінському хресті. «16. 12. 43. Мені хочеться вмерти. Мені
здається, що я прожив уже все своє життя, пережив усі свої радощі і
виплакав усі сльози [...] О боже мій, боже мій, боже мій! Отак зітхаючи весь
день, мов схоплений за горло залізною мертвою рукою, хожу як неприкаяний.
Що мені робити, куди подітись? Умерти мені, та як же вмерти [...] 

Хочеться плакати, кричати, жалітися. Кому? І куди не повернуся, і до чого не
прислухаюся, скрізь: – не їдь на Вкраїну, загинеш. Там тебе ненавидять, там
тебе не хотять, там плямують твоє ім'я, щоб потім знищити тебе. Там ти помреш.
Там не хотять, щоб ти бачив свій народ. Твій талант і твоє серце там не
потрібні. Дай простор Микиті (Хрущову), геть з України». 

І так до самої смерті – усе видавалося, що смерть поруч, що  осьдечки ж вона,
стоїть, лагідно посміхаючись, зі своєю правічною косою. 

І запис 1945-го: «Я сьогодні ранком полетів на Вкраїну. Обламалися крила і я
упав. У мене вельми заболіло в грудях. І я заплакав, упавши. Я спробував ще раз
летіти, я заспівав початок думи, і од жалібного голосу свого знову заплакав.

Учора було два приступи стенокардії.

Я свідомий свого стану. Мене вбито повільним вбивством і вже мені не
воскресати». Одначе на надгробку Довженковому на Новодівичому кладовищі у
Москві напис: «Умер в воскресенье». 

P.S. Учора ввечері не працював Інтернет... О 22.30 я запалив свічку і поставив
її на вікно. Кажуть, на вогонь такої свічки прилітає душа небіжчика. І вона
прилетіла... О 23.00, тієї миті, коли Довженкова душа прощалась із цим світом.
Не назавжди, не назавжди. Свічку цього вечора завжди ставила Тетяна Дерев'янко,
яка багато зробила для увічнення пам'яті великого митця. Пом'янув і її...

На фото: Довженко на хресті (походження світлини і її автора зараз не можу
пригадати, може хто підкаже). Під час зйомок.


\ii{26_11_2022.fb.trymbach_sergij.1.smert_jak_voskresinnja.orig}
\ii{26_11_2022.fb.trymbach_sergij.1.smert_jak_voskresinnja.cmtx}
