% vim: keymap=russian-jcukenwin
%%beginhead 
 
%%file slova.rossia
%%parent slova
 
%%url 
 
%%author 
%%author_id 
%%author_url 
 
%%tags 
%%title 
 
%%endhead 
\chapter{Россия}

Новости Крымнаша. Предатели в Крыму пришли к выводу, что \emph{россияне} — хуже
глистов, obozrevatel.com, 24.05.2021

Луганский академический театр кукол совместно с коллегами из \emph{РФ} и ДНР
провел онлайн-выставку рисунков юных зрителей, посвященную Дню славянской
письменности и культуры, lug-info.com, 24.05.2021

\enquote{Луганский академический театр кукол совместно с Брянским областным
театром кукол и Горловским городским театром кукол организовал онлайн-выставку
детских рисунков \enquote{Мы - славяне!}, посвященную Дню славянской
письменности и культуры, который ежегодно широко отмечается в \emph{России} и
других странах 24 мая}, - говорится в сообщении, lug-info.com, 24.05.2021.

Сейчас мы переживаем время не менее судьбоносное. И от того, вспомнит ли народ
свои изначальные культурные коды, зависит судьба наша в новом веке – судьба не
только \emph{России}, но и всей человеческой цивилизации, vz.ru, 24.05.2021

С этого времени \emph{Русь} ощущает себя уже не просто некоей «срединной
землей» между Востоком и Западом, Азией и Европой, но неким особым миром между
землей и небом, между началом и концом истории. И здесь, между небом и землей,
началом и концом истории, начинает она искать свои берега: «Открылась бездна,
звезд полна. Звездам числа нет, бездне – дна», vz.ru, 24.05.2021

Свободная и богатая за счет транзитной торговли, \emph{Русь} переживает бурный
рост, на пике которого и является «Слово о законе и благодати» (между 1037 и
1050 годами) митрополита Илариона – первое слово \emph{русского} самосознания,
сказанное первым этнически \emph{русским} митрополитом, vz.ru, 24.05.2021

Не просто Україна, а Україна-\emph{Русь}. Як Михайло Грушевський захищає
державні кордони?  radiosvoboda.org, 23.05.2021

А вот принципиального различия между укладами современной \emph{Россией} и
Украиной нет. Тут, как в Крыму или на Донбассе - были украинцы сверху, стали
\emph{русские}, а по сути, так примерно одни и те же люди. Народу, по большому
счету, начхать, сегодня \emph{русские} записываются в украинцы, завтра украинцы
запишутся в \emph{русские} - делов-то, \textbf{Дело Протасевича - еще один
повод раздуть Холодную войну}, Денис Жарких, strana.ua, 24.05.2021

Ця традиція іде ще від апостола Андрія – дві тисячі років тому він першим
провістив сакральну природу Києва. У ХІ столітті у знаменитій промові «Слово
про закон і благодать» перший \emph{Руський} митрополит Іларіон возвеличив Київ
як «Город Святий Всеславний», radiosvoboda.org, 09.10.2018

Добре, хоч \emph{Мордору} нічого не дали, але перед Литвою соромно! Так не
роблять! Лариса НиЦой, facebook, 23.05.2021

Я голосувала окрім деяких інших і за Литву, вони класні. А журі заангажовані,
це очевидно. І що бісить, Україна з року в рік продовжує давати бали
\emph{Русні}, комментарий (Ірина Чубарова), пост Ларисы НиЦой, facebook,
23.05.2021

Если \emph{Россия} не захочет быть мощной, имперской страной, она станет
Украиной. Само по себе майданное государство с \emph{российского} горизонта не
исчезнет, а предположение, будто украинская проблема рассосется сама по себе,
за семь лет доказало свою несостоятельность, \textbf{Надоела Украина? Ежедневное «иди и
смотри»}, Константин Кеворкян, 24.05.2021

Двери закрываются. Жителей \enquote{Л/ДНР} в \emph{России} ждёт ГУЛАГ 2.0,
obozrevatel.com, 25.05.2021

Помимо этого, как юрист, Лилия Корнилова раскрывает природу инструментария
\emph{российского} агрессора на международной арене и методы противостояния
этим провокациям, \url{www.obozrevatel.com/person/liliya-kornilova.htm}

«В \emph{России} отличное бездорожье» — как автобус не смог довезти детей до
школы, regnum.ru, 25.05.2021

Полузабытый \emph{русский} гений, которого в США считают своим — фото с
выставки, regnum.ru, 25.05.2021

\emph{Россия} лайкающая, Что на самом деле смотрят и ищут \emph{россияне} в
интернете, lenta.ru, 14.05.2021

Для поставок в Европу нефтегазового сырья, удобрений и металлов \emph{Россия}
использует не грузовые автомобили, а трубопроводы, железную дорогу и танкеры в
морских портах. \emph{Россия} в последние годы активно переводила свои грузы с
белорусской железной дороги на собственные морские порты на Балтике, vz.ru,
25.05.2021

Возмущение патриотов поддержал министр культуры и информационной политики
Александр Ткаченко, заявив, что в \emph{России} в очередной раз пытаются
присвоить себе национальное украинское блюдо, odnarodyna.org, 25.05.2021

У примітках до «Полтави» Пушкін зазначає: «Дорошенко, один з героїв давньої
Мало\emph{росії}, непримиренний ворог \emph{російського} панування»,
radiosvoboda.org, 25.05.2021

Не все, що звучить по-\emph{російськи} чи не все, що схоже на \emph{російську}
чи подібне до слова, яке існує в \emph{російській}, було сюди принесено з
\emph{російської} мови, pravda.com.ua, 25.05.2021

Якщо розбирати, яке слово де-факто прийшло з української, а яке запозичене,
вийдуть дуже прикольні речі. Вийде, що слово \enquote{щур} в українській мові скоріш за
все з \emph{російської}, а \emph{російське} слово \enquote{крыса} скоріш за все з української, просто
свого часу ми ними помінялися, pravda.com.ua, 25.05.2021

Ступка, вічна пам'ять, в кожній другій сцені, де він з'являється в ролі Тараса
Бульби, повторює про \emph{"русскую душу"}, \emph{"русскую землю"},
\emph{"русскую силу"}. Будь-яка компліментарність щодо України в цьому випадку
– це імперська, шовіністична компліментарність до меншого брата. Гоголь
спочатку був не про це, Бортко зробив його про це, і цей контекст змінює всі
повідомлення, pravda.com.ua, 25.05.2021

Сладкие парочки: \emph{Россия} и Украина, mikle1.livejournal.com, 21.11.2009

Даже я понял, что он лично виноват в том, что \emph{Россия} до сих пор не
развалилась на демократические Чечни и Запопинские республики Окрайны, Китайны
и прочая, mikle1.livejournal.com, 21.11.2009

Записки киевлянки: людей накручивают - «26 мая затмение, вот тогда
\emph{Россия} и нападет! - kp.ru, 15.04.2021

Истории Олеся Бузины: Почему рухнула Киевская \emph{Русь}, sedognya.ua, 14.06.2014

Я уже писал в одной из предыдущих статей, что название «Киевская \emph{Русь}»
придумал только в XIX веке московский историк Михаил Погодин. До него никто
даже не подозревал, что она «киевская». Современники называли эту страну просто
\emph{Русью} или \emph{Русской} землей, Олесь Бузина, segodnya.ua, 14.06.2014

Пельмени-мутанты, \emph{русский} гимн и \enquote{Кин-дза-дза} – англичанин
попал в \emph{Россию} и признался: \enquote{I was not gotov...}, tsargrad.tv,
12.04.2021

«Наглые \emph{русские}»: мир протестует против формы \emph{российских}
олимпийцев, woman.ru, 16.04.2021

«Наш слоган — \enquote{Окна в \emph{Россию} будущего}: интервью с автором \emph{русской}
кибердеревни, www.mirf.ru, 16.04.2021

Фигурное катание. \enquote{Невозможно победить \emph{Россию}. Их фигуристы на
запредельном уровне}, - Японцы высказались о КЧМ-2021, sport.ru, 16.04.2021

Первое упоминание названия \emph{Россия} - X век н.э. Константин
Багрянородный, zhenziyou.livejournal.com, 19.03.2016

Как Киевскую \emph{Русь} превратили в Украину, а потом в Анти\emph{Россию},
kp.ru, 17.06.2019

\enquote{Одержимость вот такая русофобская и навязчивая идея обвинить \emph{Россию} во всем и
вся. Наверное, уже скоро дойдет до того, что \emph{Россия} будет обвиняться просто в
факте своего существования}, - заявил Песков, strana.ua, 25.05.2021

«Даже обидно. На Украине из всех утюгов воюют с \emph{Россией}, а \emph{РФ} на нее плевать», - пишет Иосиф Эглис,
\textbf{Реакцию россиян на человека с символикой Украины сняли на видео}, riafan.ru, 13.05.2021

«А теперь попробуйте прогуляться по Львову с \emph{российской} символикой», - предлагает Татьяна Сергиенко,
\textbf{Реакцию россиян на человека с символикой Украины сняли на видео}, riafan.ru, 13.05.2021

Єдине «тішить» у цій ситуації, що й українці дали «достойну відповідь»
слов'янам. Українські глядачі лише проголосували за виступ від однієї
слов'янської країни – \emph{Росії}. Хоча \emph{Росію} (що показово!) представляла співачка,
яка не є слов'янкою. Але ж українці без «братньої \emph{Росії}» жити не можуть
– хоча й воюють із нею. Тому й дали співачці Маніжі від \emph{Росії} 4 бали.
Добре, що не 12, - \textbf{Про слов'янську солідарність?}, Петро Кралюк, day.kiev.ua,
24.05.2021

Империя воскресла: почему \emph{Россия} вычеркивает диссидентов из своей
истории, glavred.info, 24.05.2021

Ці «сайти-помийки» теж часто перетворюються на рупори \emph{російської}
пропаганди. Чи то піарячи того-таки Пальчевського з його \emph{проросійськими}
тезами, чи то поширюючи фейк Медведчука про американські біолабораторії, чи
цитуючи псевдоекспертів про фашизм в Україні, чи просто напряму даючи слово
Пєскову, \textbf{Сотні тисяч. Яка аудиторія (про)російських медіа в Україні},
texty.org.ua, 12.05.2021

Ще частіше незаконно використовують \emph{російську} мову замість української
посадовці обласного рівня, \textbf{Місцеві ради, телебачення, спорт та інше.
Хто порушує закон про мову і що робити, щоб він виконувався}, texty.org.ua,
20.05.2021

Партизанська війна? Як Зеленський готує ЗСУ до можливого масштабного наступу
\emph{Росії}, radiosvoboda.org, 26.05.2021

Може, якби наші 210 тисяч вояків були озброєні такою технікою, як Армія оборони
Ізраїлю, то, може, і \emph{Росія} тремтіла б перед нами, розумієте?» – зазначає
Жданов, radiosvoboda.org, 26.05.2021

Вице-президент Ассоциации туроператоров \emph{России} (АТОР) Дмитрий Горин
рассказал, в какие пляжные страны из тех, с которыми \emph{Россия} возобновила
авиасообщение с 25 мая, можно будет лететь отдыхать. Об этом сообщает «Москва
24», lenta.ru, 26.05.2021

Накануне создания \emph{Русского} царства в Северо-Восточной \emph{Руси} было
три великих княжества: Московское, Тверское и Рязанское, \textbf{Как немецкий
принц стал черниговским князем и русским святым}, Подумалось мне часом,
zen.yandex.ru, 24.03.2021

Великое княжество Рязанское много веков было пограничным краем, встречавшим
всех завоевателей. К югу от рязанских земель начиналось Дикое Поле, откуда
испокон веков на \emph{Русь} набегали кочевники, \textbf{Как немецкий принц
стал черниговским князем и русским святым}, Подумалось мне часом,
zen.yandex.ru, 24.03.2021

\emph{Россия} большая, сильная, она с нефтью и газом, современным вооружением,
лучшими в мире вакцинами и много чем ещё. Много кому от нас что-то надо, все
едут, просят, договариваются, Мак Сим, zen.yandex.ru, 16.05.2021

У середу, 26 травня, активісти «КримSOS» провели акцію «Ходіння по колу» біля
посольства \emph{Росії} в Києві. Це стало 58-м заходом на підтримку жертв
насильницьких зникнень в окупованому \emph{Росією} українському Криму та їхніх рідних,
\textbf{«Ходіння по колу»: посольство Росії пікетували на підтримку жертв насильницьких зникнень у Криму – фоторепортаж},
radiosvoboda.org, 26.05.2021

У 1905 році Рада міста Львова вирішила урочисто відзначити 250-річчя оборони
Львова під час облоги міста військами Богдана Хмельницького. Офіційно це
звучало як \enquote{порятунок міста від козаків та \emph{росіян}}, \textbf{Дві
історії, які нічого не вчать}, zaxid.net, 25.05.2021

Українець Феофан Прокопович назавжди змінив обличчя \emph{Росії}, перетворивши
її на імперію. Навіть відновлення патріаршества, революція, епоха радянського
атеїзму не змогли вбити його напрацювання, \textbf{Прокопович та інші українці,
що будували Росію}, Володимир Володько, zrada.org, 04.03.2011

Но остается открытым вопрос: почему страна, которая \enquote{не \emph{Россия}},
так сильно озабочена в основном \emph{российской} или, на худой конец,
бела\emph{русской} повесткой?
\textbf{Протасевич нам важнее, чем состояние собственной экономики}, Владислав Михеев, 
strana.ua, 27.05.2021

\textbf{Протасевич нам важнее, чем состояние собственной экономики}, Владислав Михеев, 
strana.ua, 27.05.2021

Массовые расстрелы в школах и других общественных местах — пугающая проблема
современного общества, с которой в последние годы столкнулась и \emph{Россия},
lenta.ru, 27.05.2021

В то же время, Пивоваров сообщил, что закрытие организации вовсе не означает,
что ее члены опустили руки. Они по прежнему готовы делать все возможное
\enquote{чтобы Россия стала свободной}. При этом он выразил уверенность в том,
что \enquote{Россия обязательно будет свободной}, Организация \enquote{Открытая
Россия} закрылась, strana.ua, 27.05.2021

Добре, що ми навчилися відкидати \enquote{\emph{русский} мир}. Погано, що ми ще
не навчилися боротися з нашими рідними терористами-популістами, і країна
залишається у них в заручниках, Сергій Фурса, gazeta.ua, 25.05.2021

Населення території цілком допускає можливість, що \emph{рускій} начальнік
повернеться, Іван Семесюк, gazeta.ua, 25.05.2021

Как сообщал OBOZREVATEL, на Донбассе вооруженные формирования \emph{Российской}
Федерации ранили украинского военного, obozrevatel.com, 27.05.2021

\enquote{Ненависть окутала их глаза}. В Кремле прокомментировали слова
президента Польши о \enquote{\emph{России}-агрессоре}, strana.ua, 27.05.2021

Ранее мы рассказывали, что Кабинет министров \emph{Российской} Федерации
утвердил список недружественных \emph{России} стран. Однако Польши в этом
перечне нет, \textbf{\enquote{Ненависть окутала их глаза}. В Кремле
прокомментировали слова президента Польши о \enquote{России-агрессоре}},
strana.ua, 27.05.2021

Судьба \emph{России} будет решаться на улицах, Виталий Портников,
obozrevatel.com, 27.05.2021

Ведь власть пытается информационно пристегнуть его к \enquote{зраде} Медведчука. И,
даже если вины лидера ОПЗЖ доказать не удастся, медийно Порошенко будут
\enquote{полоскать} и обвинять в сотрудничестве с \emph{Россией}, strana.ua, 27.05.2021

\emph{Россиянин} полез целоваться с лошадью и остался с откушенным носом, strana.ua, 10.12.2020

Как \emph{Россия} воюет на Донбассе: 30 главных расследований,
radiosvoboda.org, 27.05.2021

Князь Николай Трубецкой ещё до войны снискал себе славу одного из наиболее
проникновенных \emph{русских} историков и религиозных философов. Его занятия
привели его к мысли, что Европа в \emph{русском} обществе переоценена именно в
духовно-практическом смысле, тогда как Азия имеет непосредственное влияние на
формирование \emph{русского} менталитета, \textbf{«Евразийская концепция
\emph{русской} истории». Черниговский евразиец Петр Савицкий}, ukraina.ru,
27.05.2021

Подругу Романа Протасевича, администрировавшую канал со сливами личных данных
бело\emph{русских} силовиков и арестованную вместе с ним, зовут София Сапега,
\textbf{У подруги Протасевича историческая фамилия - Сапега}, strana.ua,
28.05.2021

У звіті Facebook сказано, що \emph{Росія} залишається світовим лідером у
виробництві дезінформації, спрямованої на втручання до інших країн. Переважно
цілями її закордонних операцій, окрім США та України, є Велика Британія, Лівія
та Судан, radiosvoboda.org, 28.05.2021

«Роды просто раздавили», Одиночество, нищета и депрессия: истории
\emph{россиянок}, которых заставили рожать нежеланных детей, lenta.ru,
28.05.2021

«Он невероятно умелый манипулятор», Как \emph{Россия} может воспользоваться
окончательным разрывом Лукашенко с Западом?, lenta.ru, 28.05.2021

\emph{Россиянка} выиграла квартиру и тут же ее лишилась, lenta.ru, 28.05.2021

Як хочеться потрапити у світ без \emph{Русской} общины Украины, \emph{Русского}
блока та їм подібних московських шавок.  Як хочеться почути живих Драгоманова
та Ключевського; тих людей, які займались історією як наукою, а не як повією,
\textbf{Забыть все = Забить на всё. Як хочеться усе забути}, zrada.org,
24.07.2010

Саме в цей день рівно 7 років тому наші герої зробили вибір на користь свободи,
навіть якщо за неї треба було віддати життя. \emph{Російські} бойовики
намагалися взяти під контроль міжнародний аеропорт Донецька, вимагали в
українських військових скласти зброю. Наші захисники здаватися не збиралися та
мужньо тримали оборону аеропорту. Підрозділи спецназу за підтримки авіації
завдали першого потужного удару по терористах. Того дня \emph{російські}
окупанти відступили, зазнавши великих втрат, \textbf{26 травня 2014 року – один
з переломних моментів війни на Сході}, bigkyiv.com.ua, 26.05.2021

Более того, \emph{российская} вакцина и разговоры о ней являются элементом
гибридной войны против Украины. В частности, таковым может быть и заявление
Лукашенко, считает Николенко, \textbf{\enquote{В русле подрывных усилий}}...
strana.ua, 28.05.2021

Более того, поскольку мы уже столько лет ведем войну с \emph{Россией}, самый
прям удачный момент отменить вместе с бензином электричество, например. И
антрацит с коксом. В рамках выбранного курса, так сказать, \textbf{Сейчас самое
время отменить вместе с бензином электричество}, Дмитрий Заборин, strana.ua,
29.05.2021

Уже давно, к сожалению, прошло то время, когда \emph{российские} и украинские патриоты
обменивались историческими шуточками и только ими могли и ограничиваться. Одна
из них, с \emph{российской} стороны, была такая:
Если вы так недовольны Москвой, то все ваши претензии – к вашему же
«украинскому» киевскому князю Юрию Долгорукому, который её основал! - zen.yandex.ru, 

Далее человек, выдающий себя за \emph{российского} оппозиционера, требует у
европейцев биткоины и подписаться на его YouTube-канал, учит делать Путину
больно и показывать голый зад \emph{российскому} посольству. Седые господа
вежливо слушают.  Связь никто не разрывает, \textbf{Наши европейские партнеры
готовы договариваться хоть с чертом лысым}, Максим Могильницкий, strana.ua,
29.05.2021

Как было не раз сказано, оная история почти повторяет разворот и обыск
\emph{белорусского} самолёта, устроенного СБУ в 2016-м. Но, как мне (примерно)
написал один из комментаторов с флагом ЕС на аватарке, \enquote{Да мне
наплевать на \emph{белорусский} самолёт. \emph{Беларусь} не член ЕС. А то был
европейкий самолёт, с европейскими гражданами на борту.} Отлично раскрыта тема
кто какого сорта, я считаю.  Но это ещё не всё, \textbf{История с Протасевичем
объемно показывает, кто чего стоит}, Роман Подолян, strana.ua, 29.05.2021

Ан-26 — самолёт со славным прошлым. Машина, засветившаяся даже в голливудском
кино — к примеру, в боевике «Неудержимые».  Но время не стоит на месте. И
скоро, в том числе и из-за провальной политики украинского руководства и
разрыва производственных связей с \emph{Россией}, господство в небе могут
захватить другие машины, \textbf{Привет «Антонову»: \emph{Россия} лишит Украину
важного рынка}, ukraina.ru, 28.05.2021

\emph{Російська} дезінформаційна кампанія в Європі проти вакцини Pfizer.
Розслідування Радіо Свобода встановило дійових осіб, radiosvoboda.org,
29.05.2021

Хоча жодні докази не свідчать про причетність \emph{російських} державних
установ чи чиновників до кампанії в соціальних мережах, їхні меседжі та способи
подачі звучать суголосно до того, що робить PR-кампанія \emph{Російського}
фонду прямих інвестицій, який фінансує і просуває на ринок вакцини Sputnik V,
\textbf{\emph{Російська} дезінформаційна кампанія в Європі проти вакцини
Pfizer. Розслідування Радіо Свобода встановило дійових осіб}, radiosvoboda.org,
29.05.2021

Прессекретар Кремля Дмитро Пєсков прокоментував ці звинувачення. «\emph{Росія}
нікого не дезінформує, \emph{Росія} з гордістю говорить про свої успіхи, і
\emph{Росія} ділиться своїми успіхами щодо першої в світі зареєстрованої
вакцини», – сказав він, \textbf{\emph{Російська} дезінформаційна кампанія в
Європі проти вакцини Pfizer. Розслідування Радіо Свобода встановило дійових
осіб}, radiosvoboda.org, 29.05.2021

Я люблю читать новости. Какими бы они ни были. Хорошими, плохими, леденящими
кровь - без разницы. Читая новости, я ощущаю биение жизни и понимаю, что от
кого бы мы ни произошли, \emph{русские} люди, обезьяны были нашими пра-пра или
инопланетяне, без разницы, наша степень психической надежности - как у
хранилища Форт Нокс, хотя мы и по эту сторону океана, \textbf{Мы - несгибаемая
нация. Нас - только дустом}, Из Питера С Любовью. Юля, zen.yandex.ru,
01.04.2021

Мы вышли из одного древнего государства, приняли одно крещение и в то время
были не просто вместе - мы были одним народом. Нам доводилось разделяться на
разные княжества, бороться между собой за первенство власти, воевать за города
и земли, но как народ мы не разделялись и никто нас не разделял. Для самих себя
и для окружающих мы все именовались \emph{Русью}, \emph{русскими}, ну а
феодальная раздробленность народов была тогда явлением распространенным. Для
нашего народа такая раздробленность добром не кончилась - от внешнего врага
отбиться не смогли, \textbf{\emph{Россия} и Украина - утраченное Содружество},
Igor Novikov, zen.yandex.ru, 29.05.2021

Как ожидание Второго Пришествия в 1492-ом году изменило курс \emph{русской}
истории, Открытая семинария, zen.yandex.ru, 20.05.2021

Пелагея, душа народа, очень тронула ее новая песня, Люблю ее всем сердцем и
душой! Достояние \emph{России}! Слушая ее, я забываю обо всем, душа летает от
такого неимоверного сильного голоса, zen.yandex.ru, 11.05.2021

Тилль из «Rammstein»: любовь к \emph{Росиии} и песни на \emph{русском} языке,
zen.yandex.ru, 26.04.2021

В глазах у девочки Васнецов увидел столько одиночества и чисто \emph{русской}
печали, что прямо ахнул, Юлия Варенцова, zen.yandex.ru, 20.03.2021

Президент Польши Анджей Дуда во время визита в Грузию назвал \emph{Россию}
\enquote{ненормальной страной}.  \enquote{Действия \emph{России} — агрессивные,
имперские, которые отбирают у людей нормальную жизнь, приводят к разрушению
государств, к военным положениям, к гибели людей. <...> \emph{Россия} ненормальная
страна. Это не то государство, которое нормально себя ведет},— сказал господин
Дуда, obozrevatel.com, 29.05.2021

Когда я подготовил в 1985 году свой первый семинар по истории \emph{русских}
интеллектуальных течений, 127 студентов подвои заявки на 12 мест.....  Я не
рассчитывал, что они будут что-то знать о \emph{России}, но считал, что человек
интересующийся историей интеллектуальных течений, должен быть знаком с
классикой мировой литературы, \textbf{Американская элита совершенно оторвана от
мировой культуры}, Игорь Заславский, strana.ua, 30.05.2021

Лихі 90-і, «Крим наш» і єдиний народ \emph{росіян}, білорусів та українців.
Викриваємо міфи, що Путін насаджує про \emph{Росію}, radiosvoboda.org,
30.05.2021

Народи України, Білорусі та \emph{Росії} – це єдина нація, radiosvoboda.org,
30.05.2021

Крим завжди був \emph{російським}, radiosvoboda.org, 30.05.2021

Крим і \emph{Росію} «пов'язує спільна історія, що йде в глиб століть – тут
прийняв хрещення князь Володимир, тут знаходяться могили \emph{російських}
солдатів, які завоювали Крим в 1783 році для \emph{російської} держави, тут
стоїть Севастополь – батьківщина \emph{російського} Чорноморського військового
флоту». «В серцях людей Крим завжди був невід'ємною частиною \emph{Росії}»,
radiosvoboda.org, 30.05.2021

\emph{Росія} і Захід «однаково погані» і варті одне одного, radiosvoboda.org, 30.05.2021

Нам потрібна нова загальноєвропейська архітектура безпеки за участю
\emph{Росії}, radiosvoboda.org, 30.05.2021

Захід повинен поліпшити відносини з \emph{Росією}, навіть якщо вона не йде на
поступки, оскільки це занадто важливо, radiosvoboda.org, 30.05.2021

Санкції щодо \emph{Росії} - це невірний підхід, radiosvoboda.org, 30.05.2021

Хоча у самій \emph{Росії} точно такі ж проблеми з інтегральними героями. Хіба
тягне на цю роль підкорювач Кавказу генерал Єрмолов? Чи може розраховувати на
статус об'єднавчої фігури герой балканських воєн і затятий націоналіст генерал
Скобелєв? \emph{Російське} суспільство готове ставити на п'єдестал усіх тих,
при кому країна «приростала територіями», але чи готові корінні народи співати
цим людям осанну? \textbf{Від Мазепи до Бандери. В України і Росії різні
герої}, Павло Казарін, radiosvoboda.org, 28.04.2021

В пятницу, 28 мая, пограничники Сумского отряда пресекли незаконное перемещение
из \emph{России} продукции военного назначения. Также был задержан 48-летний житель
города Сумы, obozrevatel.com, 29.05.2021

Организованный кафедрой социально-гуманитарных дисциплин Академии марафон
провели онлайн. В течение нескольких часов студенты и преподаватели вуза читали
стихи и прозу о \emph{русском} языке, \emph{русской} культуре,
\textbf{Литературным марафоном отметили в Академии Матусовского День славянской
письменности и культуры}, lgaki.info, 24.05.2021

Прах известного \emph{российского} стилиста Александра Шевчука захоронен в
Умани, strana.ua, 10.02.2021

Конституція Орлика — це пакт конституції прав і вольностей Війська Запорозького
1710 року. Його повинен був прийняти Карл XII у разі перемоги у Північній війні
за послуги, надані козаками у боях Швеції проти \emph{Росії}, \textbf{Названо
локацію у Києві, де покажуть оригінал Конституції Пилипа Орлика}, kyiv.media,
29.05.2021

\enquote{Нарисовали гигантское яблочко на спине Украины}. Чего ждут от встречи
Байдена и Путина на Западе и в \emph{России}, strana.ua, 30.05.2021

\enquote{\emph{Россия} - это серьезный игрок, нравится нам это или нет}, - так
кратко и четко мотивирует Washington Post, почему встреча все же состоится,
strana.ua, 30.05.2021

Как писал «Рамблер», ранее журналист Дмитрий Гордон заявил, что презирает
украинских артистов, которые продолжают выступать в \emph{России}. Он назвал их
«людьми без родины».  «Я не призываю их наказывать, репрессировать, делать им
плохо. Ни в коем случае. Они сделали свой выбор. Он абсолютно законный», -
заявил журналист, добавив, что у них либо «две извилины в голове», либо им «
застили глаза зеленые бумажки», news.rambler.ru, 27.05.2021

Одна из бед \emph{России} в том, что у нас сапожник варит щи, а кухарка мнит
себя сапожником. Вот и получается, что Юлия Волкова может стать политиком. Уже
даже не смешно. Смотрите, что получается... ЗВЕЗДУЛЬКИ, zen.yandex.ru,
02.05.2021

Певец \footnote{Юрий Лоза} и на этот раз сказанул (небезосновательно), почему
считает, что население \emph{России} тупеет. Я частично согласен с ним и
расскажу, почему (по моему мнению) это происходит и кто виноват. Поскакали,
\textbf{Почему Лоза считает, что население тупеет: а я скажу кто виноват?}
ЗВЕЗДУЛЬКИ, zen.yandex.ru, 11.04.2021

Самым странным выглядит \emph{русский} язык, на котором актёры говорят с явным
акцентом. Судя по всему, это чтобы сериал в \emph{Россию} потом продать.
Непатриотично конечно, но за деньги. Вакцину у \emph{России} брать нельзя -
непатриотично, идёт в разрез с анти\emph{российским} политическим курсом. А сериал на
\emph{русском} снимать можно, так как (опять же) за деньги.  Извиняюсь, снова в
сторону ушёл от темы. Так вот: представьте, вы живёте в стране, где
разговариваете (а значит думаете) на \emph{русском}, веселитесь (слушаете
песни) на \emph{украинском}, а в общественных учреждениях ориентируетесь при
помощи надписей на \emph{английском}, Тут волей-неволей каша в голове будет,
ЗВЕЗДУЛЬКИ, zen.yandex.ru, 01.05.2021

В \emph{России} взорвался надувной батут, на котором играли дети. Их отбросило
на рельсы. Фото, strana.ua, 30.05.2021

В \emph{России} на кладбище 6-летний мальчик сел за руль автомобиля и случайно
насмерть задавил свою мать, strana.ua, 03.05.2021

1300 курян приняли участие во Все\emph{российском} полумарафоне
\enquote{ЗаБег.РФ}, riakursk.ru, 30.05.2021

В этом году акция \enquote{Забег.рф} претендует на звание рекордсмена Гиннеса,
как самый большой в мире полумарафон с синхронным стартом. Старты были
организованы в 85 городах \emph{России}, riakursk.ru, 30.05.2021

\emph{Русско}язычная проза в Украине исчезает как Аральское море. Стоящих
авторов можно пересчитать на пальцах одной руки. Книг, достойных прочтения,
выходит крайне мало. \emph{Русско}язычного литературного процесса толком нет.
Обсуждать, по большому счёту, нечего, \textbf{Большой киевский роман},
hvylya.net, 30.05.2021

Да здравствует великая \emph{Русь}. Все \emph{русские} должны быть вместе, мы
один народ, (комментарий), \textbf{Потерять яйца в сражении с Беларусью},
Анатолий Шарий, youtube, 30.05.2021

Жителям Крыма тоже говорили: \enquote{Ааа продались за \emph{российские} зарплаты и
пенсии!!!}. Так это и есть называется - думать о своём народе. Вместо этой
конченной власти в украине, в Крыму и Беларуси будет нормальная экономика,
(комментарий), \textbf{Потерять яйца в сражении с Беларусью}, Анатолий Шарий,
youtube, 30.05.2021

18:48  Еще одна активистка - Анна Гвоздяр - вышла к микрофону с американским
флагом на рукаве. Она заявила, что уже восемь лет защищает Украину \enquote{от
всякой \emph{русни}}, \textbf{\enquote{Завтра будем требовать другими
методами}. Как в Киеве прошла акция сторонников Стерненко}, strana.ua,
30.05.2021

18:12  Активисты принесли импровизированные \enquote{двери офиса президента},
на которых воспроизвели те же надписи, которые там были на предыдущей акции за
Стерненко. Также они держат плакаты \enquote{Stop \emph{Русский} мир},
\textbf{\enquote{Завтра будем требовать другими методами}. Как в Киеве прошла
акция сторонников Стерненко}, strana.ua, 30.05.2021

Он хотел остановить катастрофу \emph{русского} народа, Владимир Станулевич, regnum.ru, 28.05.2021

Три \emph{Руси} и три \emph{России}. Что не дает покоя Польше и Прибалтике, В
Переулках Истории zen.yandex.ru, 18.05.2021

\emph{Россия} как всегда подставила плечо своему \enquote{верному} союзнику,
одолжив пол-лярда зелени и предоставив бело\emph{русскому} авиаперевозчику
\enquote{Белавиа} дополнительные рейсы в \emph{российские} города.  Также Путин
ещё раз подтвердил что первый блок БелАЭС будет запущен в эксплуатацию в
следующем месяце, в июне. Ах да, два президента ещё и в море искупались и
дежурно что-то про интеграцию говорили, \textbf{Бело\emph{русская}
многовекторность вечна, как и сам Лукашенко}, Мак Сим, zen.yandex.ru,
30.05.2021

Это благодаря ему мыслящая \emph{Россия} познакомилась с дарвинизмом - перевёл
\enquote{Происхождение видов}. Вёл огромную общественную, просветительскую
работу, за свой счёт отправлял за границу особо одарённых студентов и помогал
выжить студентам неимущим, \textbf{Неиссякаемо талантливый народ!}, Наталья
Баева, zen.yandex.ru, 06.04.2021

Массовое убийство на вечеринке в \emph{России}: главная версия и имена жертв,
glavred.info, 07.11.2020

А на додачу потрібна ефективна пропаганда у мас-медіа; зрозуміло, що на палких
адептів \enquote{русского мира} вона не подіє, але ж є чимало таких, що
коливаються між вірою у московські побрехеньки та довірою до власної держави,
Сергій Грабовський, gazeta.ua, 27.05.2021

\enquote{Заблоковані} телеканали далі працюють. Це небезпечно для України.
Сотні тисяч реальних адептів \enquote{\emph{русского} мира} із захватом
дивляться їхні програми, Сергій Грабовський, gazeta.ua, 20.05.2021

Туляки встріли в академії студентів з усієї \emph{Росії}. Великоруський синод
[6] ще попереду, ніж уряд, спостеріг ідею \emph{русифікації}, і для того він
велів в академіях мішати українців з \emph{руськими} студентами. Тим-то в
Київську академію пруть семінаристів з Костроми, Архангельська, з Волги й
Сибіру, мішаючи їх з киянами, полтавцями, одесцями й іншими і посилаючи
українських семінаристів до Москви й Петербурга, котрі, одначе, не мають охоти
туди їхати, \textbf{Хмари}, Іван Нечуй-Левицький

По розкішних алеях Братського монастиря [7], густо обсаджених усяким деревом,
гуляли студенти з усіх кінців широкого \emph{Російського} царства,
\textbf{Хмари}, Іван Нечуй-Левицький

Але, як і революціонери, що використовували нову форму Інтернету для об'єднання
та усунення ворога, \emph{Росія} тепер використовувала мережі, щоб розірвати Україну
на частини, \textbf{Війна лайків. Зброя в руках соціальних мереж}, П. В.
Сінґер, Емерсон Т. Брукінґ, Харків, 2019

Важливим прецедентом цього стала Україна. Кількість негативних
\emph{російськомовних} новин про Україну зросла вдвічі, а потім утричі. Етнічні
\emph{росіяни} всередині України невдовзі збурилися проти активістів, що
скинули \emph{проросійський} уряд. Тим часом \emph{російські} спецпризначенці
проникли до Криму, а потім на схід України, набираючи та озброюючи загони
\emph{проросійських} сепаратистів. Хвилі протестів переросли в насильство, а
потім у трагедію, \textbf{Війна лайків. Зброя в руках соціальних мереж}, П. В.
Сінґер, Емерсон Т. Брукінґ, Харків, 2019

У моделі Птолемея Землю оточували вісім обертових сфер. Кожна наступна сфера
була більша за попередню — щось на кшталт \emph{російської} матрьошки. Земля
перебувала в центрі. Що саме лежить за межами останньої сфери, ніколи не
уточнювали, але це, безперечно, було недосяжним для людського погляду. Тому
найдальшу сферу вважали чимось на кшталт кордону, вмістищем Усесвіту,
\textbf{Найкоротша історія часу}, Стівен Гокінґ і леонард млодінов, Харків,
2016

...і в результаті \emph{Росія} – країна цілком конкурентоспроможна в окремих галузях
науки – безнадійно відстала у царині молекулярної біології й генної інженерії.
Було втрачено два покоління біологів. Лисенківщина протрималася до 1964 року,
аж поки її головного ідеолога не зняли з посади після палких дискусій в
Академії наук, одній із небагатьох інституцій, яка ще зберігала відносну
автономію. Важливу роль у цьому епізоді відіграв фізик-ядерник Андрій Сахаров,
\textbf{«Світ, повний демонів. Наука як свічка у пітьмі»}, К. Э. Саган —
Книжный Клуб «Клуб Семейного Досуга», 1996

Показовий приклад – \emph{Росія}. За царату в країні буйно квітли релігійні забобони,
а наукове і скептичне мислення існувало хіба що у вузькому колі вчених, які
не мали великого впливу. При комуністах і релігію, і псевдонауку систематично
переслідували, однак на заміну їм прийшла нова державна ідеологія. Вона
видавала себе за наукову, але до науки їй було так само далеко, як і будь-якому
містичному культу. У критичному мисленні радянська влада вбачала небезпеку і
карала за його прояви. Про те, щоб викладати його у школі, не могло бути й
мови. Критична думка могла існувати тільки у вузькій герметичній сфері
фундаментальної науки, потрібної режиму,
\textbf{«Світ, повний демонів. Наука як свічка у пітьмі»}, К. Э. Саган —
Книжный Клуб «Клуб Семейного Досуга», 1996

У 1950-х роках \emph{російський} фізик Лев Ландау показав, що електричний заряд
електрона залежить від масштабу, на якому він вимірюється. З нікуди вигулькують
віртуальні частинки, тож електрони та всі інші елементарні частинки перебувають
в оточенні хмари пар віртуальних частинок й античастинок. Ці пари екранують
заряд аналогічно до екранування заряду в діелектриках. Позитивно заряджені
віртуальні частинки схильні щільно оточувати негативний заряд, тож на відстані
фізичні впливи початкового негативного заряду зменшуються, \textbf{Таємниці
походження всесвіту}, Лоуренс Краусс, Книжный Клуб «Клуб Семейного Досуга» 2017

До чого я веду? Реальної дати заснування Києва не знає ніхто. Так само, як у
випадку зі Львовом чи Москвою (ті, хто говорять про жаб, які квакали на місці
Москви у той час, як у Києві існували бібліотеки, грішать проти історичної
правди, бо археологи свідчать про наявність великого торгового центру на місці
нинішньої столиці \emph{Росії} вже у Х столітті). Одні міста ведуть свій
початок від першої згадки у писемних джерелах (тоді історія Києва велася б від
860 року).  Інші - від гіпотетичної дати заснування, \textbf{\emph{Киев} не
должен мериться древностью с другими столицами}, Константин Бондаренко,
strana.ua, 31.05.2021

Дірка замість унітазу. Шкільні туалети і велич \emph{Росії}, radiosvoboda.org,
31.05.2021

Она уточнила, что большинство отправившихся на отдых детей ни разу не были в
Крыму, а многие вообще впервые увидят море.  \enquote{То, что делает для
детворы ЛНР \emph{Российская Федерация}, это неоценимо, тем более в преддверии
Международного дня защиты детей}, - сказала уполномоченный по правам ребенка,
\textbf{Группа из 100 детей из ЛНР отправилась на отдых в крымский ДОК
\enquote{Дельфин}}, lug-info.com, 27.05.2021

Не только на Украине, но даже в \emph{России} многие не могут понять, почему на южных
\emph{русских} землях, внезапно ставших независимым государством, «что они ни делают,
не идут дела». Пока были \emph{Россией} — и «крокодилы ловились», и «кокосы
колосились», а как стали Украиной — с каждым днём становится хуже и хуже,
\textbf{Непойманный украинский крокодил}, Ростислав Ищенко, ukraina.ru, 31.05.2021

Страна, в которой господствует идеология украинского национализма (идеология
же), катится в пропасть с той же скоростью, с которой \emph{Россия} рвётся к
звёздам.  Между тем \emph{Россия} живёт без обязательной идеологии, но её
население в целом сохранило имперский дух, позволивший предкам создать самое
крупное государство на планете. Идея евразийской империи цементирует
\emph{российское} государство, даёт цель его развитию, 
\textbf{Непойманный украинский крокодил}, Ростислав Ищенко, ukraina.ru, 31.05.2021

Анатолий, езжай в \emph{Россию}, без шуток ты тут нужен и будешь иметь успех,
комментарий, \textbf{Опять Шария достали. Что дальше?} Анатолий Шарий, youtube.com, 31.05.2021

Чеські та українські експерти разом з білоруськими колегами почали аналізувати
та документувати присутність та вплив \emph{Російської Федерації} на життя та події в
Біло\emph{Русі}. Цей тристоронній проєкт надає можливість зібрати важливі факти, щоб
зрозуміти, як \emph{Росія} підтримує та допомагає підтримувати авторитарний режим у
Біло\emph{Русі},
radiosvoboda.org, \textbf{\emph{Російська} підривна діяльність, Білорусь та Крим –
серед тем другої зустрічі Українсько–чеського форуму}, 31.05.2021

...Сегодня мы работаем над несколькими сценариями остановки \enquote{Северного
потока-2}.  Необходима мобилизация всех ветвей украинской власти и четкая и
последовательная координация с нашими международными партнерами, которые
понимают опасность \emph{российской} трубы для всей Европы. Такими сейчас я
вижу приоритеты своей работы, - прокомментировала Залищук свое назначение на
своей странице в Facebook, 
\textbf{Галантерейщик и кардинал. Сцена третья. Апофеоз абсурда}, strana.ua, Валентин Землянский, 01.06.2021

Пограничники с тепловизором поймали \emph{россиянина}, который закопал свой
паспорт в лесу, strana.ua, 01.06.2021

Мужчина пробирался к границе под покровом темноты около двух часов ночи.
Пограничникам удалось обнаружить нелегала при помощи тепловизора.  После
задержания мужчина назвался гражданином Украины, но через некоторое время
признался, что является \emph{россиянином}, а в доказательство этого у него
обнаружили ксерокопию паспорта \emph{РФ}. Сам документ, по словам нарушителя,
он закопал где-то в лесу. Объясняя, зачем он изначально принялся врать
пограничникам, мужчина заявил, что \enquote{опасался за свою жизнь},
\textbf{Пограничники с тепловизором поймали \emph{россиянина}, который закопал
свой паспорт в лесу}, strana.ua, 01.06.2021

За панування \emph{Російської} Імперії на землях Речі Посполитої, як і України,
можливості отримання вищої освіти, надто для жінок, були вкрай обмеженими.
Відповіддю на це поляків стало створення підпільного Летючого університету у
Варшаві та околицях. Лекції, доступні слухачам незалежно від гендеру, читали
кращі вчені, зокрема викладачі розігнаного Віленського університету.
Відбувалося це на конспіративних квартирах. За 20 (!) років існування
Університету його випускницями стали не менше 5000 жінок. Найвідомішою серед
них є двічі лауреат Нобелівської премії Марія Складовська-Кюрі. Радіоактивний
елемент \enquote{Полоній} названо нею на честь поневоленої, в той час, \emph{росіянами}
Польщі, 
\textbf{Щодо вчорашнього указу про \enquote{Президентський університет}}, Костянтин Матвієнко, pravda.com.ua, 01.06.2021

Знову \enquote{Україна не \emph{Росія}}, проте багатьом дуже хочеться стерти цю відмінність,
Костянтин Матвієнко, pravda.com.ua, 02.02.2021

\emph{Росія} самоізолюється від світу тому, що за своєю природою не може існувати як
демократична держава, бодай із мінімальним набором громадських прав і свобод.
\emph{Росія} виникла та існує винятково як абсолютна монархія. Усі президенти України,
хіба окрім першого – Л. Кравчука, намагалися побудувати державну систему
України за \emph{російським} зразком. Усі спроби зазнали невдачі. Передостання з них
має своїм наслідком масове кровопролиття, окупацію Криму і частини Донбасу,
руйнування економіки...
\textbf{Роби те, що слід, і нехай станеться те, що статися має!}
Костянтин Матвієнко, pravda.com.ua, 13.05.2016

Може статися, що учора Україна нарешті подолала останній вододіл між радянським
абсолютизмом і модерною європейською демократією. Вистачило б часу та сил дійти
до мети, бо \emph{Росія} зубами тягтиме назад у морок самодержавства,
\textbf{Роби те, що слід, і нехай станеться те, що статися має!}
Костянтин Матвієнко, pravda.com.ua, 13.05.2016

Активно популяризувати українське в Україні почали відносно недавно – з
початком Революції Гідності та війни на Сході. Під приводом \enquote{захисту
\emph{російськомовного} населення} у 2014 році до України вторглися окупанти. Сьогодні
наші полонені на території ОРДЛО сидять у в'язницях. Вони змушені говорити
\emph{російською}, щоб мати шанс вижити та дочекатися української армії. Ось вам і
наслідки \enquote{захисту \emph{російської} мови від конкуренції},
\textbf{Чи є різниця, якою мовою говорити?} Андрій Білецький, pravda.com.ua, 22.01.2021

знати мову має кожен, хто живе, або працює в нашій країні! крапка! а особливо
державні діячи!!! це взагалі неприпустимо, бути так званою політичною елітою і
не знати державної мови. Впевнений, що більшість верховної ради з великими
зусиллями розмовляє українською, от з них і треба починати!!  Блокувати
\emph{російську} рекламу, блокувати \emph{російськомовні} канали! Гнати з країни всіляких
Медведчуків та рабіновичив, які чхати хотіли на нашу державну мову, в головах
тих людей досі сидить закорінілий совок!!!
коментар, \textbf{Чи є різниця, якою мовою говорити?} Андрій Білецький, pravda.com.ua, 22.01.2021

В 1993-ем нас было 52,2 миллиона. Через двадцать лет, - до начала войны с
\emph{Россией}, - 45,5 миллиона. Не было ни Гитлера, ни Сталина. Кто убил 6,7 миллиона
украинцев?  Политические игрища привели не только к потере территорий, но и к
дальнейшим потерям населения. В прошлом году министр Дубилет назвал
предполагаемую оценку «остатков» - чуть более 37 миллионов (2). Социологи
возмутились (3). Однако есть и более впечатляющая оценка,
\textbf{Кто убил 17 миллионов украинцев?}, Юрий Гуленок, hvylya.net, 01.06.2021

Украина 1 июня оказалась на третьем месте среди стран Европы по количеству
летальных случаев пациентов с коронавирусом за сутки. Возглавляет антирейтинг
\emph{Россия}, следом за ней идет Германия, 
\textbf{Украина вчера стала третьей в Европе по смертям от коронавируса}, Елена Вьюн, strana.ua, 02.06.2021

\enquote{Несмотря на вопли о \enquote{зраде} (а-ля нас сдали \emph{эрэфии}) мы системно работаем над
развитием отношений со стратегическим партнером №1. Только что представительная
делегация американских сенаторов в составе Джин Шахин (Демпартия), Роберта
Портмана (Республиканская партия) и Кристофера Мерфи (Демпартия) прибыла с
визитом в Украину! Завтра (2 июня - Ред.) будем говорить о том, как вместе
сделать Украину сильнее!}, - написал Евгений Енин,
\textbf{В Киев на переговоры прибыла группа сенаторов США}, Александр Максюк, strana.ua, 02.06.2021

\enquote{Украина и Грузия являются важными партнерами США в борьбе со все более
агрессивной \emph{Россией}, и очень важно, чтобы мы выразили нашу солидарность народам
Украины и Грузии и побудили политических лидеров продолжать проводить
необходимые реформы, которые укрепят их демократии}, - заявляла перед визитом
глава делегации Джин Шахин,
\textbf{В Киев на переговоры прибыла группа сенаторов США}, Александр Максюк, strana.ua, 02.06.2021

Трагедія, яка може стати спусковим гачком \emph{російського} BLM, Лилия Корнилова, obozrevatel.com, 02.06.2021

\emph{Российские} СМИ преподнесли новость о диаспоре \enquote{с изюминкой}. В
частности, написали, что \enquote{сотруднику ДПС, который случайно застрелил
19-летнего азербайджанца Векиля Абдуллаева во время задержания в Мошковском
районе Новосибирской области, пригрозили убийством}.  Якобы \enquote{в день
инцидента отдел полиции и больницу в Мошково окружили около 60 автомобилей},
\enquote{в них находились вооружённые представители азербайджанской диаспоры,
которые требовали выдать им \enquote{убийцу} и спасти раненного парня},
\enquote{пригрозили расстрелять семью инспектора},
\textbf{Трагедія, яка може стати спусковим гачком \emph{російського} BLM}, Лилия Корнилова, obozrevatel.com, 02.06.2021

І саме це є однією з причин, чому \emph{Росія} виграє інформаційно-пропагандистську
війну за розум української молоді. \emph{Росіяни} працюють системно і мають більше
досвіду. А в нас відбувається імітація виховної роботи. Через відсутність
рефлексії й усвідомлення подій та проблем наша молодь легко піддається на
маніпуляції і йде в \emph{російському} фарватері, бо змістовно для неї немає різниці.
Молоді люди не розуміють українського контексту, їх цього не вчили і не вчать,
\textbf{Союз – мертвий, комсомол – живий, Тиск замість розвитку! Що хочуть від \emph{української} молоді?},
Станіслав Безушко, zaxid.net, 31.05.2021

Через відомо які події знову зросла зацікавленість вітчизняного інформаційного
простору в \emph{БілоРусі}. І знову понеслося рефреном: «А-а-а, дурні
\emph{білоруси}, взуття знімали, а треба було бруківку жбурляти!». Я вже писав
про те, що ситуація в \emph{БілоРусі} радикально відрізняється від української,
повторюватися не буду. Тому вирішив поговорити про причинно-наслідкові зв'язки
в нашій нещодавній і не дуже нещодавній історії, які декому дуже не хочеться
помічати, 
\textbf{П'ять цікавих фактів – від \emph{УПА} до АТО}, Павло Зуб'юк, zaxid.net, 01.06.2021

Другий. Ні, наявність україномовної та націоналістичної Галичини сама собою не
була запобіжником від реалізації «\emph{білоруського} сценарію» в Україні. Суспільна
думка переважної більшості українців у році так 1994-му була \emph{проросійською}, чи
точніше – прорадянською. Воювати за незалежність від України галичани б не
стали. Якби справа була лише в цьому – \emph{проросійський} уряд у Києві з радістю дав
би автономію кільком областям або взагалі федералізував би державу. Значно
більшим запобіжником від аналога Лукашенка були міцні, зокрема регіональні,
олігархічні еліти, які вже утворилися в українській економіці. Звісно, Кучма
міг спробувати їх знищити, але Україна набагато більша і складніша за \emph{Білорусь}.
Варто поглянути на мапу: у \emph{Білорусі} тільки один меґаполіс, він же столиця, а по
радіусу від нього – приблизно однакові за розміром і значенням обласні центри.
\emph{Білорусь} природно значно централізованіша. І це, зокрема серед іншого,
посприяло її \emph{зросійщенню} в радянські роки, 
\textbf{П'ять цікавих фактів – від \emph{УПА} до АТО}, Павло Зуб'юк, zaxid.net, 01.06.2021

Третій. Ні, мова не рятує від диктатури, зокрема і \emph{проросійської}. Таджикистан –
країна таджицькомовна, \emph{російська} мова погано зрозуміла більшості таджиків.
Однак від самого проголошення незалежності ця країна є залежною від Москви
бідною державою з диктатором при владі. Чи, може, азербайджаномовний та дуже
націоналістичний Азербайджан – демократія?
\textbf{П'ять цікавих фактів – від \emph{УПА} до АТО}, Павло Зуб'юк, zaxid.net, 01.06.2021

П'ятий. Ні, Порошенко не створив армію з нуля і не врятував Україну від
\emph{російської} окупації. АТО почалася до приходу Порошенка на посаду глави держави.
Армія ще до приходу Порошенка захищала аеропорти і військові об'єкти. Перше
місто було звільнене від \emph{проросійських} бойовиків напередодні складання
Порошенком президентської присяги – це Красний Лиман. Нещодавно читав пафосний
текст одного письменника, де описується, як путінські танки вже прогрівали
мотори на кордоні, щоб іти на Київ, письменник навіть уточнює, що до вторгнення
лишалися лічені години,
\textbf{П'ять цікавих фактів – від \emph{УПА} до АТО}, Павло Зуб'юк, zaxid.net, 01.06.2021

\emph{Росія} – вічний супротивник України, Роман Кізима, zaxid.net, 28.05.2021

Українська держава, на жаль, приречена співіснувати зі своїм \emph{північно-східним
сусідом}. Враховуючи історичний досвід такого співжиття можна говорити, що
теперішня \emph{Російська Федерація} є екзистенційним ворогом/опонентом/супротивником
України. І так було завжди... Не будемо вдаватися в глибокі історико-філософські
роздуми причин такого відношення до нас з боку \emph{Росії-Московії}, але повинні
сприйняти таку реальність як данність, з якою нам жити,
\textbf{\emph{Росія} – вічний супротивник України}, Роман Кізима, zaxid.net, 28.05.2021

Частково спільна історія, певні спільні слов'янські корені лише підкреслюють
категоричну нашу різність в світоглядному та державницькому розвитку.
\emph{Росія} як один із осколків Київської \emph{Русі}, попавши під вплив
Золотої Орди, - стала її повним відображенням і продовженням. Саме тоді
відбувся екзистенційний розкол двох цивілізацій між так званим європейським,
умовно демократичний шляхом розвитку та східним, авторитарний методом
функціонування держави. В подальшому усі «фасадні» дії \emph{Московії}, як то
наприклад, фактична крадіжка слова \emph{Русь-Росія}, прийняття християнства,
реформи Петра I не відміняли «азійськості» у їх державотворенні,
\textbf{\emph{Росія} – вічний супротивник України}, Роман Кізима, zaxid.net, 28.05.2021

На превеликий жаль, така самовизначеність \emph{Росії} усі подальші періоди проходили
болюче для її сусідів, яких імперія хотіла або поглинути або підпорядкувати.
Причиною таких агресивних дій окрім банально імперських був постійний конфлікт
всередині \emph{Московії} між її елітами,
\textbf{\emph{Росія} – вічний супротивник України}, Роман Кізима, zaxid.net, 28.05.2021

Частина таких еліт вважала себе європейцями і відповідно вбачала в існуванні
більш «прозахідних» інших слов'янських цивілізацій загрозою своєї
«європейськості». Ми, українці, \emph{білоруси}, поляки та інші, ніби відтіняємо їх
своєю більшою приналежністю до Європи. Таким чином, з одного боку ми жертви
імперськості, з іншого - боротьби за спадок європейської цивілізації. Такі дії
нашого агресивного сусіда не одноразово приводили до втрати нашої державності.
Радянський період окупації України був черговим болючим етапом
\emph{російсько}-українського протистояння, який завершився винищенням мільйонів
українців, заселенням нашої території громадянами інших національностей та
\emph{русифікацією} більшості наших міст,
\textbf{\emph{Росія} – вічний супротивник України}, Роман Кізима, zaxid.net, 28.05.2021

Україно-\emph{російська} війна 2013-2021 років розглядалася і розглядається
теперішньою \emph{російською} владою як період остаточного вирішення «українського
питання». Путін неодноразово заявляв і всередині країни і західним партнерам,
що Україна не заслуговує на власну державу. Частина її, мовляв, належить
\emph{Росії}, а частину пропонувалося розділити між східно-європейськими країнами.
Звідси і план \emph{«Новоросія»}, який на щастя не вдалося реалізувати. Закінчилося
усе на даному етапі «лише» окупацією частини нашої території,
\textbf{\emph{Росія} – вічний супротивник України}, Роман Кізима, zaxid.net, 28.05.2021

Усім напевно відомі \emph{російські} дефініції, мовляв, без України не буде \emph{Російської}
імперії... Центробіжні і центровідштовхувальні тенденції, які відбувалися в
\emph{російській} імперії, впродовж її історії не міняли її, \emph{Росії}, ставлення до нашої
державності. Сподіватися на те, що демократична \emph{Росія} майбутнього змінить на
180 градусів свою політику по відношенню до своїх західних сусідів звичайно
можна. Але враховуючи остання 300 років історії такі сподівання є як мінімум
наївними, а фактично – небезпечними і навіть злочинними. Ми мали 22 роки
невеличкого перепочинку, який мали би використати на побудову власної
державності. На жаль, справилися ми зі своїм «домашнім завданням» на слабеньку
трієчку,
\textbf{\emph{Росія} – вічний супротивник України}, Роман Кізима, zaxid.net, 28.05.2021

Центр протидії дезінформації є лише маленькою частинкою такої політики. Вважаю,
що потрібно створити «інститут \emph{Росії}», як науковий центр, де би на державному
рівні вивчалася внутрішня та зовнішня політика сусіда, тенденції, які там
відбуваються, вивчення думок основних політичних та громадських лідерів.
Необхідна постійна робота по прогнозуванню різних сценаріїв розвитку \emph{Росії} та
її зовнішньої політики в Європі в короткій, середній та довгій перспективі,
\textbf{\emph{Росія} – вічний супротивник України}, Роман Кізима, zaxid.net, 28.05.2021

Итак, в 2018 году Украина вырвалась вперед всей Европы по числу разводов,
обогнав даже \emph{Россию}. Тогда на 228 411 заключенных за год браков пришлось
153 949 разводов (67,4\%). В 2019 году этот показатель улучшился до 58\%. Но
вот в прошлом 2020-м нас постигла настоящая катастрофа, вызванная резким
сокращение браков: за год украинцы сыграли всего лишь 167 974 свадьбы – это
самое низкое число с 1944 года! А ведь тогда в Украине и женихов почти не было,
всех мобилизовали, куда же они подевались теперь?
\textbf{Почему разваливается \emph{украинская} семья?}, Иван Пургин, from-ua.com, 02.06.2021

Фактически, данный документ можно назвать объявлением войны \emph{России}. Нашей
стране отказано во всём. Перечислим ряд мифов, от которых, по мнению  Chatham
House, должны избавиться западные политики и государственные деятели:
утверждение, что народы Украины, \emph{БелаРуси} и \emph{России} - это единая нация, Крым
всегда был \emph{русским}, \emph{России} обещали, что НАТО не будет расширяться, \emph{Россия} имеет
право на «оборонительный периметр» - сферу своих привилегированных интересов,
куда входят и территории других государств, санкции - это неверный подход,
\textbf{\emph{России} пора переходить к жесткому решению «британского вопроса»}, 
Александр Владимиров, voskhodinfo.su, 28.05.2021

Британский истеблишмент – давний, очень умный, опасный и на вечно враг \emph{России},
поэтому какое-либо сотрудничество с ним просто невозможно (impossible). Попытки
наладить отношения будут пустой тратой времени, сил и гарантированно будут
использованы против самой \emph{России}.  Тот факт, что Лондон облюбовали некоторые
\emph{российские} олигархи, имеет мало отношения к коренным государственным интересам
\emph{России} и \emph{её народа}. На этих меценатов британской модели также следует обратить
внимание, т.к. у них сердце там, где их деньги, т.е. не в \emph{России}. А
использование Британией их потенциала в качестве пятой колонны и против \emph{России}
вполне возможно,
\textbf{\emph{России} пора переходить к жесткому решению «британского вопроса»}, 
Александр Владимиров, voskhodinfo.su, 28.05.2021

Это даже видно на примере недавнего высказывания польского президента о \emph{России},
как «не нормальной стране». Британия задаёт тезисы и устойчивые выражения, а
остальные дружно повторяют. С учётом наличия у \emph{России} и США стратегических
ядерных сил и необходимости поддержания стратегического паритета какими бы
плохими не были отношения между странами, и американцам, и нам нужен постоянный
контакт и диалог по вопросам стратегической стабильности,
\textbf{\emph{России} пора переходить к жесткому решению «британского вопроса»}, 
Александр Владимиров, voskhodinfo.su, 28.05.2021

Тон данного документа задан в его преамбуле: «Этот доклад деконструирует 16
наиболее распространенных мифов и заблуждений, которые формируют современное
западное мышление о \emph{России}. В нём объясняется их пагубное влияние на
разработку и осуществление политики, и в каждом случае описывается, как позиции
Запада нуждаются в критическом пересмотре, чтобы обеспечить более рациональные
и эффективные ответы на действия \emph{России}. В основе нашего анализа лежит
важный аргумент о том, что, вопреки ожиданиям многих евроатлантических
политиков и государственных деятелей, в обозримом будущем у \emph{России} мало
шансов стать более конструктивным и сотрудничающим партнером для западных
правительств. Таким образом, благонамеренные усилия по «улучшению» отношений с
Кремлем, скорее всего, потерпят неудачу, поскольку стратегические цели,
ценности и понимание \emph{Россией} межгосударственных отношений безвозвратно
отличаются от западных»,
\textbf{\emph{России} пора переходить к жесткому решению «британского вопроса»}, 
Александр Владимиров, voskhodinfo.su, 28.05.2021

Справедливость и закон в \emph{России}, Андрей Медведев, voskhodinfo.su, 01.06.2021

Бацька идет на Восток. Как санкции Запада и Украины толкают Лукашенко к интеграции с \emph{Россией},

Обвиненный в \enquote{российской пропаганде} Netflix опубликовал список фильмов, попавших под цензуру в разных странах,
strana.ua, 08.02.2021

Запад никак не может определиться, отсталая \emph{Россия} или передовая, То она - \emph{Страна-бензоколонка}, то - \emph{питомник хакеров},
Максим Войтенко, strana.ua, 02.06.2021

Пришли новые вести о \emph{русских хакерах}. Только они отдохнули от взлома
SolarWinds и нефтепровода на Восточном побережье США, как Путин приказал им
ломать сетку крупнейшей в мире мясоперерабатывающей компании. Команду хакеров
усилили Петровым и Бошировым, парой ломов для взлома серверов и все заверте.
«Из-за инцидента компании пришлось отменить смены на канадском заводе и на
некоторых предприятиях в США, а в Австралии пришлось полностью остановить
работу». Белый Дом строчит обращения к Кремлю, общественность паникует и везде
ищет руку Кэй-Джи-Би. Если уж не шутить, то скажу следующее. Я бы эти новости
на месте западных и особенно украинских пропагандистов так не форсил. Потому
что у маломальски думающего человека возникает лютый диссонанс. С одной стороны
мем \emph{«Страна-бензоколонка»}, а в украинском патриотическом
медиа-пространстве вообще рассуждения о \emph{московском улусе}, варварах и
шестисотлетней борьбе (с момента падения украинского стольного града Коцюбиева)
с варварами,
\textbf{Запад никак не может определиться, отсталая \emph{Россия} или передовая, То она - \emph{Страна-бензоколонка}, то - \emph{питомник хакеров}},
Максим Войтенко, strana.ua, 02.06.2021

– А что тут вспоминать? Войну начал Киев. Дом мне разбомбила Украина. Кто же
еще? Я помню, как они стояли в Георгевке и стреляли по Роскошному. И что, надо
было мой дом бомбить? Залетел шальной снаряд. Как по мне, все очевидно – \emph{Россия}
нас защищает, а Украина стреляет. И те луганчане, которые уехали, то есть
сбежали – изменники Родины. Но и хорошо. Ведь если ты не веришь в «ЛНР», тебе
здесь делать нечего,
\textbf{«Никто нам не вернет украденные годы»: как в \emph{Луганске} вспоминают захват города},
Донбас.Реалії, Записки з окупації, radiosvoboda.org, 02.06.2021

Екатерина, учительница: идейных я уже не вижу, – Я никогда не поддерживала эту
\emph{«русскую весну»} и не верила в агрессию Украины. Для большинства населения эти
идеи \emph{русского мира} оказались близки, думаю, из-за низкого уровня
образованности. Промзона, работяги – тут всегда было не до развития
интеллектуальных способностей. Сейчас самые упертые не верят уже ни в
республику, ни в \emph{РФ}. Идейных я уже не вижу. Основное настроение здесь – люди
смирились с ситуацией. Печально все и пока не вижу выхода,
\textbf{«Никто нам не вернет украденные годы»: как в \emph{Луганске} вспоминают захват города},
Донбас.Реалії, Записки з окупації, radiosvoboda.org, 02.06.2021

«Товарищи политруки!» \emph{Росія} диктує, як розповідати про Шарія, «СС Галичину» і Зеленського,
radiosvoboda.org, 02.06.2021

Во-вторых, наивно выглядят и претензии к Западу во внешнеполитических вопросах.
Взять, например, жалобы на то, что Германия и Франция отказываются признать
\emph{Россию} стороной войны на Донбассе, т.е. признать якобы факт
\emph{российско}-украинской войны. Просто руководители этих стран, в отличие от
Зеленского, не руководствуются и не обязаны руководствоваться нарративами
украинской пропаганды. У них есть собственное, объективное понимание
происходящего на Донбассе, а вовсе не \enquote{боязнь \emph{России}}, и это их понимание надо
уважать. Украину не берут и, вероятно, не возьмут в НАТО в том числе еще и
потому, что украинское руководство не в состоянии объективно оценивать
международную стуацию, да и ситуацию в собственной стране, и находить баланс
интересов, 
\textbf{Интервью Зеленского отражает слабость украинской внешней политики,
Никому не интересна страна, которая всегда попрошайничает}, Виктор Суслов,
strana.ua, 03.06.2021

Неожиданный (хотя и не слишком удивительный) скандал произошел вокруг
американского стриммингового сервиса \enquote{Нетфликс}, который разместил у себя два
\emph{российских} фильма – \enquote{Брат} и \enquote{Брат-2} - с английскими субтитрами. Собственно,
сами субтитры и привели к скандалу. Наверное, никому не нужно объяснять, что
во втором \enquote{Брате} есть пара фраз, которые в контексте нынешней украинской
идеологии выглядят крамольными. Одна из них – звучит в диалоге между старшим
братом главного героя и представителем украинской мафии: \enquote{Слышь, земляк, а где
здесь \emph{русские} живут?} - \enquote{Москаль менi не земляк}. – \enquote{Бандеровец?},
\textbf{Операция \enquote{Антиолигарх}, антибандеровский \enquote{Нетфликс},
МИД Украины против \enquote{БелАвиа}. Итоги \enquote{Страны}}, strana.ua,
03.06.2021


Кабмін і СБУ також мають опрацювати питання щодо відкликання мережевих ресурсів
(ІР-адрес), які виділені у користування операторам (провайдерам)
телекомунікацій, що перебувають в ОРДЛО або використовуються в інформаційній
агресії \emph{Росі} проти України,
\textbf{Зеленський \enquote{активізував} дії щодо Донбасу: правосуддя, IP-адреси, розвідка},
pravda.com.ua, 03.06.2021

Я написав своїй співрозмовниці, що чекаю біля цієї сцени і почав спостерігати
за цим дійством.  На сцену виходив рабінович, він запрошував депутатів із свої
секти. Вони кричали про любов до нашої країни, землі і тому подібне. Більшість
з цих покидьків говорили \emph{російською}.  Чогось згадалося минуле століття, коли
діди цього рабіновича, медведчука, ківи, королевської - вбивали багатих
українців, називали їх куркулями і знищували заможний український клас,
\textbf{Чому партія \enquote{опзж}, морить своїх людей голодом?}, 
Михайло Ухман, censor.net.ua, 03.06.2021

Іду біля будинку парламенту, де сьогодні чорти з опзж влаштували зібрання.
Навпроти мене вийшло два московських попи і чоловік у жилетці з емблемою секти
любителів \enquote{\emph{расєї}}! Підійшовши до них, побачив, що чоловік геть \enquote{пропитий}, а
попи в брудних накидках і немитими бородами. Один з них побачив мою цікавість
ними і теж ковзнув поглядом по мені. Його очі зупинилися на кілька секунд на
моєму лівому вусі, де гордо виблискував Тризуб. Що було далі, ви навіть не
уявляєте? Мій тризуб - це як осиковий кіл для вампірів. Для всіх чортів, які
підтримують партію регіонів, медведчука, моспатріархат, він має таку ж дію,
\textbf{Чому партія \enquote{опзж}, морить своїх людей голодом?}, 
Михайло Ухман, censor.net.ua, 03.06.2021

16 травня 2015 року після 14 години, в ході перестрілки біля містечка Щастя
Луганської області, бійці 92 бригади ЗСУ, затримали кадрових офіцерів ГРУ
\emph{Російської} федерації: сержанта Олександра Олександрова і капітана Євгена
Єрофиєва з Третьої окремої бригади спецназу міста Тольятті.  В ході
перестрілки, нажаль загинув український воїн, який першим побачив окупантів.
Два вище згаданих персонажі, отримали поранення, але вижили. Один із
\emph{російських} розвідників, хотів підірвати гранату, яка була в його кишені,
але не зміг цього зробити, через поранену руку. Інший не встиг використати
автомат, який знаходився за метр від нього, адже теж лежав поранений під
прицілом українських воїнів,
\textbf{\emph{Російські} офіцери в українському полоні},
Михайло Ухман, censor.net.ua, 17.05.2021

Знаєте, що саме цікавіше - цих російських \enquote{героїв} Кремль нагороджує, як
захисників країни, хоча вони звичайні окупанти. Як потім виявляється, нікому
непотрібні: \emph{росії}, родинам, які відмовляються від них... Вони стають
викиднями...  Така доля чекає і сепаратистів, так званих вояк \enquote{днр, лнр}! Їх не
визнає \emph{росія}, світ - вони стають покидьками, котрі зрадили своїй країні. І
тепер приреченні на існування та пригноблення. Така доля чекає всіх, хто
намагається воювати проти України, 
\textbf{\emph{Російські} офіцери в українському полоні},
Михайло Ухман, censor.net.ua, 17.05.2021

Один із \emph{російських} розвідників, хотів підірвати гранату, яка була в його
кишені, але не зміг цього зробити, через поранену руку. Інший не встиг
використати автомат, який знаходився за метр від нього, адже теж лежав
поранений під прицілом українських воїнів.  Як згодом вдалося вияснити, що на
початку березня 2015 року, з \emph{росії} в Луганськ прибуло двісті чоловік
їхнього спецназу. Це були розвідники, бійці технічного резерву. Командував
всіма ними, один із генералів ФСБ \emph{московії},
\textbf{\emph{Російські} офіцери в українському полоні},
Михайло Ухман, censor.net.ua, 17.05.2021

Абсолютно необучаемый пони. Вот совершенно. \enquote{Открытую \emph{Россию} закрыли - скоро
\emph{Россия} будет свободной!}. \enquote{Рейс остановили на рулежке толпой Росгвардии - нас
большинство!} \enquote{Меня везут в Краснодар - как они нас боятся!} \enquote{Мне раздавили
яички сапогом - как Путин зассал!} \enquote{Мне вырвали все ногти - боже, какие мы
прекрасные, как мы ломим!} \enquote{Мне пробили голову прикладом на прииске Колымы и
скинули в общую яму с трупами в вечной мерзлоте - Прекрасная \emph{Россия} Будущего!}
Именно Андрею, вот конкретно ему, я несколько раз говорил в глаза в упор - что
вы делаете, идиоты, вас всех перетравят, перебьют, пересажают. Бегите, глупцы.
Неа. Прекрасная \emph{Россия} будущего, шарики, умное голосование, мы идем на выборы,
ой, нас опять наебали, ой, опять все в кутузке, ой, опять произошла какая-то
ошибка, ой, куда вы меня тащите,
\textbf{Новый тридцать седьмой и \emph{Россия} будущего}, Аркадий Бабченко, 03.06.2021

Смотрю по 1 каналу \emph{российского} телевидения (бывшее ОРТ) прямую
трансляцию из Питера – открытие Второй сцены Мариинского театра оперы и балета:
Пласидо Доминго, Анна Нетребко, Денис Мацуев, множество чрезвычайно талантливых
певцов и балерин – мировых и \emph{российских} звезд, Путин в зале, и конечно
Гергиев, Гергиев, Гергиев, которому исполнилось 60 лет.  Смотрю и, кажется,
начинаю понимать – почему в Украине ничего подобного невозможно, почему Украина
– это туша, которую уже распилили и продали, а \emph{Россия} – феодальная
страна, в которой иногда случаются интересные события – но только при
перенапряжении усилий и ресурсов и по высочайшему повелению,
\textbf{Мариинский-2 и украинская культура: Почему Украина – это фабрика человеческого мяса? (обн.)},
Ірина Славінська, pravda.com.ua, 03.05.2013

В нормальной стране высокое искусство существует, потому что существует
представление об \enquote{общем благе}, разделяемое и гражданским обществом, и властью,
а также – обеспеченная государством система экономической конкуренции и
традиция меценатства.  В \emph{России} высокое искусство существует, потому что царей
убеждают – мол, надо дать денег на высокое искусство. Искусство – это фактор
имперского престижа \emph{России}, это инструмент \emph{российского} влияния на мировую
политику (особенно любят в этом контексте вспоминать \enquote{Большой Балет}), Soft
Power и т.д. (В этом контексте следует рассматривать и награждение Гергиева
звездой Героя Труда N1.),
\textbf{Мариинский-2 и украинская культура: Почему Украина – это фабрика человеческого мяса? (обн.)},
Ірина Славінська, pravda.com.ua, 03.05.2013

Пьесы и рассказы 12-летней луганской писательницы и драматурга Фаины Савенковой
переведены на разные языки мира, они входят в шорт-листы престижных
всероссийских премий и конкурсов, включая наиболее престижный конкурс Гильдии
драматургов \emph{России} «Автора на сцену»,
\textbf{Фаина Савенкова. Верить и надеяться}, antifashist.com, 02.06.2021

В качестве инструмента воздействия на украинскую \emph{русофобию} можно создать
телеканал на \emph{русском} языке по украинской проблематике, а также
использовать такие источники, как интернет и радио, направленные против
пропаганды Незалежной, сказал в интервью изданию Украина.ру \emph{российский}
политолог Сергей Марков,
\textbf{Украинская \emph{русофобия} становится все более радикальной. Марков поделился инструментом для борьбы с ней},
ukraina.ru, 03.06.2021

Виважене і шанобливе ставлення українців до жіноцтва було поширене серед усіх
соціальних станів і за \emph{Київської Русі}, і після її Хрещення. Права жіноцтва
були вписані до \enquote{\emph{Руської} Правди} – зведення тогочасних законів
\emph{Київської Русі}, виконаного Ярославом Мудрим на початку XІ століття,
\textbf{Шанобливе ставлення до жіноцтва в \emph{Україні}},
Марія Гуцол, slovoprosvity.org, 17.05.2021

Мы все, выросшие в \emph{России} - внуки жертв и палачей. Все абсолютно, все, без
исключения. В вашей семье не было жертв? Значит были палачи. Не было палачей?
Значит были жертвы. Не было ни жертв, ни палачей? Значит...,
\textbf{Бред сумасшедшего! Как с этим жить?}, Владимир Яковлев, news24ru.net, 03.06.2021

Мне кажется, мы сильно недооцениваем влияние трагедий \emph{российского} прошлого на
психику сегодняшних поколений. Нашу с вами психику. По сей день, прощаясь, мы
говорим друг другу - \enquote{До свидания!}, не сознавая, что
\enquote{свидание} вообще-то слово тюремное. В обычной жизни бывают встречи,
свидания бывают в тюрьме.  По сей день мы легко пишем в смсках: \enquote{Я
напишу, когда освобожусь!} Когда ОСВОБОЖУСЬ...  Оценивая масштаб трагедий
российского прошлого, мы обычно считаем погибших. Но ведь для того, чтобы
оценить масштаб влияния этих трагедий на психику будущих поколений, считать
нужно не погибших, а - выживших,
\textbf{Бред сумасшедшего! Как с этим жить?}, Владимир Яковлев, news24ru.net, 03.06.2021

Погибших - десятки миллионов. Выживших - сотни миллионов. Сотни миллионов тех,
кто передал свой страх, свою боль, свое ощущение постоянной угрозы, исходящей
от внешнего мира - детям, которые, в свою очередь, добавив к этой боли
собственные страдания, передали этот страх нам. Просто статистически - сегодня
в \emph{России} нет ни одной семьи, которая так или иначе не несла бы на себе
тяжелейшиe последствия беспрецедентых по своим масштабам зверств,
продолжавшийся в стране в течение столетия,
\textbf{Бред сумасшедшего! Как с этим жить?}, Владимир Яковлев, news24ru.net, 03.06.2021

Истеричный лукашизм, или Подарок для Путина, Во избежание спекуляций как
с той, так и с другой стороны автор сразу же отметит, что, имея белорусские
корни, а также придерживаясь старой доброй советской традиции, с искренней
симпатией относится к сябрам, причем прежде всего именно к \emph{Белоруссии} и к
близкородственным нам, выросшим с нами из одного \emph{древнерусского} и советского
корня братьям-\emph{белорусам}. Поэтому вся эта \emph{антибелорусская}, совершенно идиотская,
скотская, подонковская, гнусная, ублюдочная (далее уже только нецензурные
выражения!) истерия, которая развернута в Украине в последние дни в исполнении
разного рода \enquote{актывиздов} и Зе-власти, вызывает рвотный рефлекс и классовую
ненависть,
\textbf{Европа впала в иррациональный истеричный лукашизм}, Александр Карпец,
strana.ua, 03.06.2021

Первым эту дурашливую стрельбу куда попало начал сам \emph{белорусский} бацька,
эффектно приземливший на нары некую недоросль по фамилии Протасевич и
окончательно толкнувший в изоляцию \emph{Беларусь}, создав огромные проблемы
всем ее гражданам, включая фанатов Лукашенко, которых в \enquote{рэспубліке}
остается еще очень много. Тем самым Лукашенко продолжил череду своих
\enquote{epic fail-ов}, начатую, как минимум, во время выборов прошлого года,
\textbf{Европа впала в иррациональный истеричный лукашизм}, Александр Карпец,
strana.ua, 03.06.2021

Бывший главный редактор Телеграм-канала \enquote{Nexta} Роман Протасевич дал большое
интервью программе \enquote{Марков. Ничего личного}, которая выходит на белорусском
телеканале ОНТ. В нем он выразил свое мнение о президенте Беларуси Александре
Лукашенко.  По словам Протасевича, Лукашенко \enquote{поступал как человек со стальными
яйцами}. Задержанный оппозиционер добавил, что \enquote{безусловно} уважает президента
Беларуси, 
\textbf{Протасевич заявил, что Лукашенко поступает как человек со стальными яйцами. Видео},
, strana.ua, 03.06.2021

Через атаку хакерів зупинилися м'ясозаводи у США: ФБР звинувачує \emph{Росію},
ФБР звинуватило \emph{Росію} в організації кібернападу на найбільшого світового
виробника м'яса – компанію JBS. Росія має це припинити і не надавати прихисток
злочинцям, заявив Держсекретар США.  За кібернападом на найбільшого світового
виробника м'яса, компанію JBS, стоїть пов'язана з \emph{Росією} група хакерів
Revil, повідомили в ФБР, radiosvoboda.org, 03.06.2021

\emph{Росія} має зупинити подібні кібернапади і «не надавати прихисток злочинцям,
причетним до цих нападів», заявив 3 червня Держсекретар США Ентоні Блінкен, в
інтерв'ю іспанському виданню.  «На жаль, ми бачимо новий фронт кіберзагроз і це
злочинні організації, що використовують так звані «програми-вимагачі», щоб
тримати в заручниках компанії, тримати в заручниках критичну інфраструктуру
задля фінансового зиску», – заявив Блінкен. «Обов'язок кожної країни робити
все можливе, щоб знайти ці компанії, притягнути їх до відповідальності,
включаючи питання нападу на Colonial Pipeline. Компанія, відповідальна за цей
напад, її керівники були і тепер перебувають в \emph{Росії}, тому, гадаю, \emph{Росія}
зобов'язана зробити так, щоб таке не продовжувалось», – додав посадовець,
\textbf{Через атаку хакерів зупинилися м'ясозаводи у США: ФБР звинувачує \emph{Росію}},
radiosvoboda.org, 03.06.2021

Перемога Джамали викликала значні суперечки. Пісня сама по собі була
незаперечно чудова, та критики твердили, що вона надто відверто політизована і
загрожує підірвати легковажну природу пісенного конкурсу. Та попри ці
нарікання, пісня «1944» стала великим тріумфом «м'якої сили» України. Вона
знову привернула міжнародну увагу до страждань кримських татар, які живуть у
Криму під \emph{російською} окупацією. Вона також підвищила зацікавленість до теми
радянських злочинів проти людяності і спростувала намагання Москви неправдиво
твердити, ніби Крим «завжди був \emph{російським}»,
\textbf{Чому «Євробачення» – таємна зброя «м'якої сили» України}, radiosvoboda.org, 02.06.2021

Виступ Вєрки також створив одну з найменш імовірних і найбільш пам'ятних
політичних суперечок в історії «Євробачення», коли \emph{російські} джерела
протестували проти того, що в українському номері, мовляв, є слова «\emph{Russia},
goodbye». Андрій Данилко, виконавець ролі Вєрки Сердючки, запевняв, що
насправді слова були «лаша тумбай». Це, заявляв він дещо непереконливо, нібито
фраза монгольською мовою, що означає «збиті вершки»,
\textbf{Чому «Євробачення» – таємна зброя «м'якої сили» України}, radiosvoboda.org, 02.06.2021

Відкриваючи концертну програму творчого колективу, зі словами вдячності до
гостей вечора звернулася художній керівник та диригент капели, начальник служби
з питань культури і діаспори НКЦУ, заслужений працівник культури України та
\emph{Росії}, голова РГО «Культурно-просвітницький центр українців у м. Москві»
Вікторія Скопенко.  Протягом вечора у виконанні Української народної хорової
капели Москви та її солістів прозвучала українська духовна музика, обробки
народних пісень, етнографічні та вокально-хореографічні композиції, авторські
твори, поезія Тараса Шевченка, Лесі Українки,
\textbf{Пісенне джерело натхнення}, ukrcentr.ru, 28.05.2021

\enquote{Ми продовжуємо суворо засуджувати атаки \emph{Росії} на цивільне
населення. Вони мають бути зупинені. Натомість \emph{Росії} слід зосередитись
на виконанні своєї частини безпекових положень Мінських домовленостей}, -
заявив Цимбалюк,
\textbf{Внаслідок російської агресії на сході України загинули 240 дітей, -
Україна в ОБСЄ}, day.kiev.ua, 03.06.2021

\enquote{Меня назовут предателем}. О чем исповедь Романа Протасевича, который заплакал в эфире \emph{белорусского} ТВ,
strana.ua, 03.06.2021

Но вернемся к \enquote{Нехте}. По информации Протасевича, создать канал - это была идея
журналиста Владимира Чуденцова, а не Степана Путило. В дальнейшем в адрес
Путило будет много выпадов - да Протасевич и сам признает, что у него с ним
личностный конфликт. Он говорит, что владелец \enquote{Нехты} был более высокого
уровня \enquote{посвящения}, имел охрану и контролировал деньги проекта. А также
допускался к \enquote{специалистам}. Протасевич рассказал, что было и \emph{российское}
финансирование \enquote{Нехты},
\textbf{\enquote{Меня назовут предателем}. О чем исповедь Романа Протасевича, который заплакал в эфире белорусского ТВ},
strana.ua, 03.06.2021

Протасевич говорит, что против Лукашенко действительно готовились покушения.
Одна из групп - команда Автуховича, которого ранее задержали в Беларуси.  Также
он говорит, что общался с двумя людьми, которых в \emph{Беларуси} и \emph{России} обвиняют в
подготовке покушения и госпереворота - Юрием Зенковичем и Дмитрием Щигельским.
Протасевич признался, что выступал посредником между заговорщиками,
планировавшими убить Лукашенко, и штабом Тихановской.  \enquote{Я должен был быть
связующим звеном между заговорщиками и штабами}, - говорит Протасевич. Вот
полный диалог на тему вооруженного захвата власти в \emph{Беларуси}, который якобы
планировался,
\textbf{\enquote{Меня назовут предателем}. О чем исповедь Романа Протасевича, который заплакал в эфире белорусского ТВ},
strana.ua, 03.06.2021

Исследование показало, что в \emph{России} показатель смертности занизили в 5,4
раза - 593,6 тысяч смертей из-за коронавируса против 109,3 тысяч по официальным
данным,
\textbf{В Украине от Covid-19 умерло втрое больше людей, чем указано в официальной статистике - исследование},
Юлия Супрун, strana.ua, 07.05.2021

Ну і окрема тема. Порівняння з \emph{Росією}. У нас дуже популярна тема, що в
\emph{Росії} Путін подолав олігархів, і одразу встановив автократію (ну а за
домінування Березовського та компанії там, очевидно, була демократія), тож
наших олігархів треба оберігати, захищати, і слідкувати, аби хтось їх не
образив. Думка, звісно, цікава, але вона відображає відрив інтелектуальної
еліти пострадянського простору від реалій цього простору,
\textbf{Власть \emph{олигархов} сможет преодолеть только обновленное общество},
Петр Олещук, strana.ua, 04.06.2021

Помню, как после государственного переворота и прихода к власти майдана, в
эфире ТВ рыдал бывший \enquote{регионал} Мирошниченко. Так всегда бывает: либо рыдают
те, кого иностранцы швырнули совершать переворот в своей стране, либо — те,
кого \enquote{перевернули}.  Рад за \emph{БелаРусь}, что не доведётся повторить печальный опыт
Украины, которую \enquote{цивилизованный запад} превратил в свою колонию, и только то и
делает 7 лет, что грабит свои территори, и население, живущее на них,
\textbf{Сотрудничество с властью и слезы - это выбор самого Протасевича},
Александр Скубченко, strana.ua, 04.06.2021

Даже \enquote{антироссийскость} балтов и поляков - это тоже их национальный интерес,
потому что им за это платят миллиарды евро.  Без этой \enquote{антироссийскости}, их
функция в системе коллективной безопасности НАТО или балансе сил ЕС,
существенно снизилась бы. Практически до нуля.  Если бы им за это не платили,
они тут же активизировали бы переговоры с РФ о совместных транзитных проектах
или доступе товаров на рынки друг друга,
\textbf{Америка поняла, что может добиться своего от Украины за гроши},

США готовят наступательную операцию против хакеров из \emph{России} - СМИ,
Елена Вьюн, strana.ua, 04.06.2021

Також \emph{російську} лекцію відвідали британський фотограф Дін О'Браян та французька
документалістка Анна-Лаура Боннел, які підтримують \emph{російські} наративи про
«громадянську війну на Донбасі».  Захід вів постпред \emph{Росії} при ООН Василь
Небензя,
\textbf{«Азаров та Царьов у Раді безпеки ООН»: що насправді відбулося і в чому маніпуляція?},
Донбас.Реалії, radiosvoboda.org, 03.06.2021

Україна нібито воює на сході з власними громадянами, і воювати їй допомагає
підтримка Заходу.  Що насправді: участь \emph{Росії} у війні на Донбасі доведена
десятками міжнародних розслідувань, і вже давно. Ось підбірка з трьох десятків
із них, і це ще не все. Якщо узагальнити коротко, дотепер ідентифіковані
близько 2500 \emph{російських} військовослужбовців і 101 підрозділ регулярних військ,
що брали учать в боях на території України, а також 53 види військової техніки
армії \emph{Росії}, яка не перебуває на озброєнні України,
\textbf{«Азаров та Царьов у Раді безпеки ООН»: що насправді відбулося і в чому маніпуляція?},
Донбас.Реалії, radiosvoboda.org, 03.06.2021

Як насправді: хронологія і суть подій абсолютно інша. \emph{Проросійські} акції у
містах Донбасу навесні 2014-го не стикалися із жодною протидією, а особи, які,
як пізніше стало відомо, приїхали з \emph{Росії}, підбурювали жителів Донецька та
Луганська до протиправних дій, а також брали безпосередню участь в організації
нападів на зібрання Євромайдану (аж до вбивства людини). Потім загони під
керівництвом \emph{російських} офіцерів захопили органи влади. Далі зі зброєю
штурмували українські військові частини та аеропорт. Лише у відповідь на це
влада України почала збройну відповідь,
\textbf{«Азаров та Царьов у Раді безпеки ООН»: що насправді відбулося і в чому маніпуляція?},
Донбас.Реалії, radiosvoboda.org, 03.06.2021

Представник Франції, котрий взяв участь в організованій \emph{Росією} зустрічі,
заявив, що зробив це лише для того, щоб заявити, що \emph{Росія} зловживає
своїм правом ініціювати ці неформальні зустрічі. Він додав, що Франція не
підтримує наративи \emph{Росії} та закликає Москву зупинити провокації, вивести
війська зі сходу України, а також повернути Крим.  Аналіз наративу, який
розповсюджують \emph{російські} пропагандистські ресурси як у самій \emph{Росії}, так
і на окупованих нею територіях України, а також у світі – вказує на наявність
створених Кремлем «темників», що спотворюють хронологію і зміст подій, з метою
приховати акт прямої агресії \emph{Росії} і перекласти відповідальність на
Україну, яка власне і стала жертвою цієї агресії,
\textbf{«Азаров та Царьов у Раді безпеки ООН»: що насправді відбулося і в чому маніпуляція?},
Донбас.Реалії, radiosvoboda.org, 03.06.2021

Як розповідає у своєму дослідженні Сергій Камінський, мажорський International
MXT отримав броньовані панелі на скло і захищену капсулу в кузові для кулемета.
У боях за Дебальцеве машина потрапила під потужний обстріл дружньої для сім'ї
Януковичів \emph{російської} артилерії і перетворилася на купу металобрухту,
\textbf{Шушпанцери і бандеромобілі. Героїчна історія саморобних броньовиків у війні за український Донбас},
Євген Руденко; Дмитро Ларін, pravda.com.ua, 04.06.2021

