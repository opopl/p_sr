% vim: keymap=russian-jcukenwin
%%beginhead 
 
%%file slova.dnevnik
%%parent slova
 
%%url 
 
%%author 
%%author_id 
%%author_url 
 
%%tags 
%%title 
 
%%endhead 
\chapter{Дневник}
\label{sec:slova.dnevnik}

%%%cit
%%%cit_head
%%%cit_pic
%%%cit_text
Еще в своих \emph{дневниках}, которые датировались 20-ми и 30-ми годами (в
Германию он переехал в конце 1918 года), Розенберг рисовал примерно такую
картину: следовало создать на просторах СССР марионеточные квазигосударства. Из
них ключевая роль отводилась бы Украине.  Идеолог нацистов считал, что украинцы
смогут стать противовесом полякам и русским, которые должны как раз полностью
утратить государственность. По мнению Розенберга, Украина могла получить
ограниченную автономию под немецким протекторатом.  Часть российских губерний
при этом он планировал передать под управление этой полностью лояльной немцам
Украины. Чтобы ее население составляло по итогу порядка 60 миллионов человек.
\enquote{Разбудить украинское национальное самосознание} было одной из
центральных идей Розенберга.  В общем, достаточно \enquote{сладкие} планы для
уха нынешних украинских националистов
%%%cit_comment
%%%cit_title
\citTitle{22 июня - 80 лет нападения на СССР. Что немцы готовили для украинцев}, 
Максим Минин, strana.ua, 22.06.2021
%%%endcit

