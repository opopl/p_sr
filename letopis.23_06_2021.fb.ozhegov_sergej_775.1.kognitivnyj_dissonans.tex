% vim: keymap=russian-jcukenwin
%%beginhead 
 
%%file 23_06_2021.fb.ozhegov_sergej_775.1.kognitivnyj_dissonans
%%parent 23_06_2021
 
%%url https://www.facebook.com/sergey.ozhegov.775/posts/500178344536478
 
%%author Ожегов, Сергей
%%author_id ozhegov_sergej_775
%%author_url 
 
%%tags diskussia,germania,istoria,kiev,kognitivnyj_dissonans,nacizm,obschestvo,pesnja,psihologia,sssr,ukraina,vojna,vov,vov.22.06.1941
%%title О когнитивном диссонансе
 
%%endhead 
 
\subsection{О когнитивном диссонансе}
\label{sec:23_06_2021.fb.ozhegov_sergej_775.1.kognitivnyj_dissonans}
\Purl{https://www.facebook.com/sergey.ozhegov.775/posts/500178344536478}
\ifcmt
 author_begin
   author_id ozhegov_sergej_775
 author_end
\fi

О когнитивном диссонансе.

Читая комментарии на странице одного из популярных блоггеров обратил внимание
на такой вот комментарий. Прочтёт этот вброс какой нибудь читатель, отложит у
себя в голове и при случае транслирует дальше. 

На это и расчёт - посеять сомнение в том, что было тогда. 

А было ли это вообще?

\ifcmt
	width 0.4
  pic https://scontent-lga3-2.xx.fbcdn.net/v/t1.6435-9/204450180_500185031202476_7622350799086115356_n.jpg?_nc_cat=104&ccb=1-3&_nc_sid=8bfeb9&_nc_ohc=RL2fP-IbTLgAX9QWcpV&_nc_ht=scontent-lga3-2.xx&oh=5a02bddfb8c68aa6cbe3d7b6058167f4&oe=60D799F0

	pic https://scontent-lga3-2.xx.fbcdn.net/v/t1.6435-9/206679884_500185064535806_3695424348874900445_n.jpg?_nc_cat=104&ccb=1-3&_nc_sid=8bfeb9&_nc_ohc=eS-A4mgBIGgAX8Vd56u&tn=ntrKbsW_7ChXu3v-&_nc_ht=scontent-lga3-2.xx&oh=c75bf943594db011eab00e5839fde4e0&oe=60D7C81D
\fi

Вдруг Украина была оккупирована большевистской Московией, а Запад в 1941 пришёл нас не убивать, а освобождать?

Не зря же украинцы встречали гитлеровцев хлебом солью, в вышиванках. Вдруг
бомбёжки Украины - это просто советская пропаганда, которая даже песню такую
придумала, \enquote{22 июня, ровно в 4 часа, Киев бомбили, нам объявили, что началась
война}. Причём слова наложили на мелодию песни \enquote{Синий платочек}.

Только вот песня эта от 1941 года и была широко известна в народе. Невозможно
было так врать народу, ведь масса киевлян такие бомбёжки пережила лично.

Возможно ли было такое подвергать сомнению всего то 5-10 лет назад?

Нет.

Этот комментарий - лишь крупинка, мелкая капля пропаганды, которая продвигается
в головы украинцев. Но вода, как известно, точит даже камень.

И сегодня, мы видим, как солдаты Президентского полка уже торжественно хоронят
бывшего эсэсовца. А спустя неделю принимают участие в мероприятии, посвящённому
80-летию начала Великой Отечественной войны - когда эти эсэсовцы под знамёна и
Гитлера пришли убивать наших предков.

Казалось бы, страна в некоем  когнитивном диссонансе - когда человек попадает в
состояние психического дискомфорта, вызванное столкновением в его сознании
конфликтующих представлений: идей, верований, ценностей или эмоциональных
реакций.

Но только этот дискомфорт ненадолго. Человек ведь стремится жить в
эмоциональном комфорте. И под влиянием неослабевающей пропаганды, он вскоре
примет реальность, которую ему регулярно преподносят извне. 

И тогда через год-два Президентский полк уже не будет задействован 22 июня. 

А там и до возрождения идеи чистоты расы рукой подать.

Поэтому противодействуйте. Распространяйте правду.

\url{t.me/ozhegoVS}
