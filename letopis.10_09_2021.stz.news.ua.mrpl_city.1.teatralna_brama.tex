% vim: keymap=russian-jcukenwin
%%beginhead 
 
%%file 10_09_2021.stz.news.ua.mrpl_city.1.teatralna_brama
%%parent 10_09_2021
 
%%url https://mrpl.city/blogs/view/teatralna-brama
 
%%author_id demidko_olga.mariupol,news.ua.mrpl_city
%%date 
 
%%tags 
%%title "Театральна брама"
 
%%endhead 
 
\subsection{\enquote{Театральна брама}}
\label{sec:10_09_2021.stz.news.ua.mrpl_city.1.teatralna_brama}
 
\Purl{https://mrpl.city/blogs/view/teatralna-brama}
\ifcmt
 author_begin
   author_id demidko_olga.mariupol,news.ua.mrpl_city
 author_end
\fi

\ii{10_09_2021.stz.news.ua.mrpl_city.1.teatralna_brama.pic.1}

11–19 вересня 2021 року на сцені Донецького академічного обласного
драматичного театру (м. Маріуполь) вдруге відбудеться обласний відкритий
фестиваль театрального мистецтва \enquote{Театральна брама}. Цей фестиваль має на меті
показати глядачу унікальний всесвіт з традиційними цінностями та новими
підходами, інноваціями, мистецьким проривом, а також налагодити співпрацю між
театральними колективами для подолання стереотипів та єднання між регіонами.
10 професійних театральних колективів з різних регіонів України покажуть
маріупольцям та гостям міста свої вистави.

Генеральний директор Донецького академічного обласного драматичного театру (м.
Маріуполь)  \textbf{Володимир Кожевніков} поділився, що представлені будуть не зовсім
великі вистави, розраховані на малу сцену. Останнім часом театри приділяють
багато уваги саме цьому формату, тому що це, як кажуть, \enquote{на носу у глядача}.
Водночас всі спектаклі дуже різнопланові, драматургія непроста. Володимир
Володимирович зазначив, що вистави, які везуться на фестивалі – некасові, вони
розраховані на обізнаного глядача, на людей, які дійсно полюбляють театр, які
знають різні жанри та різні театральні школи. Керівництво театру сподівається
на розуміння глядачів і теплий прийом такого потужного фестивалю в місті.

Оцінювати всі представлені на фестивалі вистави буде високоповажне журі.
Генеральний директор підкреслив, що для режисерів і акторів Донецького
академічного обласного драматичного театру (м. Маріуполь) дуже важлива оцінка
провідних театрознавців країни, адже необхідно розуміти, чи в правильному
напрямі рухається театр. До складу журі увійшов \textbf{Дмитро Дощинський} – відомий
театральний критик, театрознавець, який оцінює вистави й XXIII міжнародного
фестивалю \enquote{Мельпомена Таврії}. Також серед журі буде і \textbf{Олена Либо} –
театрознавиця, кураторка Міжнародних і Всеукраїнських театрально-мистецьких
проєктів, кураторка театральних фестивалів, старша викладачка кафедри
театрознавства Харківського національного університету мистецтв ім.
І. П. Котляревського, членкиня Міжнародної Спілки діячів театру ляльок
\enquote{UNIMA-Україна}, членкиня національної спілки театральних діячів України,
(м.Харків). Представлятимуть журі і педагог \textbf{Валерій Гувін} (готує професійних
режисерів) заступниця начальника обланого управління культури та туризму
\textbf{Світлана Шмельова} та відома маріупольчанка, народна артистка України, лауреатка
премії ім. М. Заньковецької, театрознавця \textbf{Світлана Отченашенко}. В обговоренні
вистав Донецького академічного обласного драматичного театру (м. Маріуполь)
Світлана Іванівна брати участь не буде.

Серед учасників конкурсної програми – театри з Києва, Миколаєва, Дніпра, Сум,
Львова, Кропивницького, Харкова, Херсона: Відкриватиме фестиваль маріупольський
драматичний театр своєю яскравою виставою \enquote{Біла ворона}. \emph{\textbf{Володимир Кожевніков}}
наголосив, що \enquote{Біла ворона} – одна з наймасштабніших робіт маріупольського
драматичного театру, але на фестивалі ще не потрапляла, тому було вирішено не
тільки включити її, а й відкрити нею фестиваль. Режисерка-постановниця вистави
\emph{\textbf{Анжеліка Добрунова}}, зауважила, що буде дуже цікаво, відповідально і хвилююче
дізнатися думку журі про те, наскільки цікаво та точно їй та акторам вдалося
відтворити історію Жанни д'Арк. Вона радіє, що саме ця вистава відкриватиме
фестиваль, адже в ній задіяна майже вся трупа. Анжеліка Арганівна дуже
розраховує на підтримку маріупольських глядачів, як емоційну так і
психологічну.

Якщо на першому фестивалі \enquote{Театральна брама}, що проходив теж  у Маріуполі,
було представлено один національний театр, то цього року їх аж три. Це
Київський національний академічний театр російської драми імені Лесі Українки,
Миколаївський національний академічний український театр драми та музичної
комедії і Сумський національний академічний театр драми та музичної комедії
імені М. С. Щепкіна.

Донецький академічний обласний драматичний театр (м. Маріуполь) на фестивалі
покаже ще одну виставу – \enquote{Кохання дона Перлімпліна}. Режисерка-постановниця
спектаклю Людмила Колосович зазначила, що цей фестиваль – це дуже важлива подія
для міста, оскільки маріпольські актори нечасто гастролюють, цей досвід є для
них дуже цінним і важливим. На думку Людмили Леонідівни, всі вистави, що будуть
представлені на фестивалі, є дуже вдалими.

Після кожної вистави відбуватиметься обговорення з режисерами та акторами
театрів, представлених на фестивалі. За результатами голосування журі будуть
визначені переможці в таких номінаціях як: \emph{\enquote{найкраща вистава}}, \emph{\enquote{найкраща
режисура}}, \emph{\enquote{найкраща жіноча роль}}, \emph{\enquote{найкраща чоловіча роль}}, \emph{\enquote{найкраще музичне
оформлення}} \emph{\enquote{найкраща сценографія}} тощо. Втім не виключено, що під час
фестивалю журі додасть й нові номінації.

Підготовкою урочистої церемонії відкриття займається\par\noindent режисерка-постановниця
\textbf{Анжеліка Добрунова}. Вона розповіла, що будуть задіяні декілька міських
колективів, які допоможуть зробити це дійство ще більш яскравим. На урочистому
відкритті глядачі побачать гістріонів, вершників, церемоніймейстера і навіть
королеву  театральної держави.

Захопливий дев'ятиденний театральний марафон стартує 11 вересня о 17.30
яскравою урочистою церемонією відкриття, а далі будуть представлені добірні
вистави знаних театральних колективів країни.

\href{https://archive.org/details/video.03_09_2021.dram_teatr_mrpl.anons_teatr_festivalu_teatralna_brama_2021}{%
Відео: Анонс театрального фестивалю Театральна брама 2021, Драматичний Театр Маріуполь, 03.09.2021}%
\footnote{%
Відео: Анонс театрального фестивалю Театральна брама 2021, Драматичний Театр Маріуполь, 03.09.2021, %
\par\url{https://www.youtube.com/watch?v=ToYc2dNLEoc}, \par%
Internet Archive: \url{https://archive.org/details/video.03_09_2021.dram_teatr_mrpl.anons_teatr_festivalu_teatralna_brama_2021}
}

\ifcmt
  ig https://i2.paste.pics/PP4CH.png?trs=1142e84a8812893e619f828af22a1d084584f26ffb97dd2bb11c85495ee994c5
  @wrap center
  @width 0.9
\fi

\ifcmt
  tab_begin cols=1,no_fig,center,separate,no_numbering
     pic https://i2.paste.pics/PP4GT.png?trs=1142e84a8812893e619f828af22a1d084584f26ffb97dd2bb11c85495ee994c5
     pic https://i2.paste.pics/PP4HK.png?trs=1142e84a8812893e619f828af22a1d084584f26ffb97dd2bb11c85495ee994c5
     pic https://i2.paste.pics/PP4HU.png?trs=1142e84a8812893e619f828af22a1d084584f26ffb97dd2bb11c85495ee994c5
     pic https://i2.paste.pics/PP4KP.png?trs=1142e84a8812893e619f828af22a1d084584f26ffb97dd2bb11c85495ee994c5
     pic https://i2.paste.pics/PP4KW.png?trs=1142e84a8812893e619f828af22a1d084584f26ffb97dd2bb11c85495ee994c5
     pic https://i2.paste.pics/PP4LG.png?trs=1142e84a8812893e619f828af22a1d084584f26ffb97dd2bb11c85495ee994c5
     pic https://i2.paste.pics/PP4LP.png?trs=1142e84a8812893e619f828af22a1d084584f26ffb97dd2bb11c85495ee994c5
     pic https://i2.paste.pics/PP4MA.png?trs=1142e84a8812893e619f828af22a1d084584f26ffb97dd2bb11c85495ee994c5
     pic https://i2.paste.pics/PP4NV.png?trs=1142e84a8812893e619f828af22a1d084584f26ffb97dd2bb11c85495ee994c5
     pic https://i2.paste.pics/PP4O3.png?trs=1142e84a8812893e619f828af22a1d084584f26ffb97dd2bb11c85495ee994c5
     pic https://i2.paste.pics/PP4OB.png?trs=1142e84a8812893e619f828af22a1d084584f26ffb97dd2bb11c85495ee994c5
     pic https://i2.paste.pics/PP4OM.png?trs=1142e84a8812893e619f828af22a1d084584f26ffb97dd2bb11c85495ee994c5
     pic https://i2.paste.pics/PP4PJ.png?trs=1142e84a8812893e619f828af22a1d084584f26ffb97dd2bb11c85495ee994c5
     pic https://i2.paste.pics/PP4PU.png?trs=1142e84a8812893e619f828af22a1d084584f26ffb97dd2bb11c85495ee994c5
     pic https://i2.paste.pics/PP4YH.png?trs=1142e84a8812893e619f828af22a1d084584f26ffb97dd2bb11c85495ee994c5
     pic https://i2.paste.pics/PP4YO.png?trs=1142e84a8812893e619f828af22a1d084584f26ffb97dd2bb11c85495ee994c5
     pic https://i2.paste.pics/PP7V4.png?trs=1142e84a8812893e619f828af22a1d084584f26ffb97dd2bb11c85495ee994c5
     pic https://i2.paste.pics/PP7VC.png?trs=1142e84a8812893e619f828af22a1d084584f26ffb97dd2bb11c85495ee994c5
  tab_end
\fi
