% vim: keymap=russian-jcukenwin
%%beginhead 
 
%%file 15_05_2021.fb.luzina_lada.2.bulgakov_130_let_kiev
%%parent 15_05_2021
 
%%url https://www.facebook.com/lada.luzina/posts/4094399760640381
 
%%author 
%%author_id 
%%author_url 
 
%%tags 
%%title 
 
%%endhead 
\subsection{В Киеве родился только один великий писатель}
\Purl{https://www.facebook.com/lada.luzina/posts/4094399760640381}

В Киеве родился только один великий писатель, верно, поэтому за тысячелетия
существования Великого Города, о нем написан всего один гениальный роман -
«Белая гвардия» Михаила Булгакова. 

\ifcmt
  pic https://scontent-frt3-2.xx.fbcdn.net/v/t1.6435-0/s600x600/186481061_4094367627310261_2177459568758294914_n.jpg?_nc_cat=101&ccb=1-3&_nc_sid=730e14&_nc_ohc=iAjatZaGQ74AX_Y1jxs&_nc_ht=scontent-frt3-2.xx&tp=7&oh=b23469efd888e4695ccc18b8e0ab2c43&oe=60C6E8C4

	pic https://scontent-frt3-1.xx.fbcdn.net/v/t1.6435-9/186367243_4094368037310220_3315234785158443071_n.jpg?_nc_cat=107&ccb=1-3&_nc_sid=730e14&_nc_ohc=sSDRNIGtXr8AX_Rii4t&_nc_ht=scontent-frt3-1.xx&oh=3d4d1247bb523d727c5691a61ab45c19&oe=60C71AF1

	pic https://scontent-frx5-1.xx.fbcdn.net/v/t1.6435-9/186459728_4094368630643494_5884172877119475187_n.jpg?_nc_cat=111&ccb=1-3&_nc_sid=730e14&_nc_ohc=Axs2f3b7ACsAX-SY-uv&tn=DvKZIYNFfQZfOhcY&_nc_ht=scontent-frx5-1.xx&oh=0a45ed5e54a9edd28d75f78c1d8919a6&oe=60C4A856
\fi

Но и у гениального Мастера тоже был только один Город - Киев!

На первый взгляд это кажется преувеличением, ведь все его произведения, включая
легендарные «Мастер и Маргарита» написаны в Москве и о Москве... Но то кажется,
а стоит вглядеться: писатель этот сугубо киевский, впитавший в себя все
противоречия Великого Города, его тайны и мистику, его боль.

«Нет красивее города на свете, чем Киев» - сказал М.Б.  И этим было сказано все. 

Владимирская горка для Булгакова – «лучшее место в мире»!

Самая фантастическая улица в мире - Мало-Провальная (Малоподвальная).

Сам Киев – «самый прекрасный город нашей Родины». 

Никто и никогда ни до, ни после него не писал о Киеве с такой
безапелляционной любовью «правоверного киевлянина». Этот Город, который
Булгаков писал с большой буквы почти не знал его писателям. Но именно
он породил на свет Мастера, который, взявшись за перо, знать не желал,
что в мире существуют и более прекрасные города.

Особую киевоцентричность булгаковского мышления первым открыл нам Мирон
Петровский. Согласно его книге «Мастер и Город» булгаковский Ершалаим стоял не
где-нибудь, а на Владимирской горке. На киевскую Лысую прилетела пред балом
ведьма Маргарита. Берлиоз попал под трамвай на киевской Европейской площади
(или как минимум провел там генеральную репетицию собственной смерти). Воланд,
впервые мелькнувший в «Белой гвардии» под псевдонимом Шполянский, учился с
Мишей в Первой киевской гимназии.  И даже Вечный дом на том свете имеет вполне
конкретный киевский адрес – добавлю уже я… 

Дом № 13 на Андреевском спуске. 

И потому памятник рядом с этим домом будет украшен сегодня тревожными желтыми
цветами от многочисленных киевских \enquote{Маргарит}.

С Днем Рождения, наш киевский Мастер. 

Сегодня Булгакову исполняется 130 лет!
