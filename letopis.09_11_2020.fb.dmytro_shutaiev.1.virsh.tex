% vim: keymap=russian-jcukenwin
%%beginhead 
 
%%file 09_11_2020.fb.dmytro_shutaiev.1.virsh
%%parent 09_11_2020
 
%%url https://www.facebook.com/DmytroShutaiev/posts/1250381115338234
%%author 
%%tags 
%%title 
 
%%endhead 

\subsection{Вірш - День Писемності і Мови}
\Purl{https://www.facebook.com/DmytroShutaiev/posts/1250381115338234}
\Pauthor{Шутаєв, Дмитро}

Сьогодні День української писемності та мови, тобто чудовий привід, щоб
поділитися моїм улюбленим віршем авторства одного занадто відомого поета:

\ifcmt
pic https://scontent.fiev6-1.fna.fbcdn.net/v/t1.0-9/124669206_1250378735338472_4357271876731502945_o.jpg?_nc_cat=108&ccb=2&_nc_sid=730e14&_nc_ohc=zYhcrQkaP1cAX8Z_Pmd&_nc_ht=scontent.fiev6-1.fna&oh=7eb6affc0e1e3bd058bf41436875b5b3&oe=5FDA708A
\fi

\obeycr
Їй п’ятнадцять і вона торгує квітами на вокзалі.
Кисень за шахтами солодкий від сонця та ягід.
Потяги завмирають на мить і рушають далі.

Військові їдуть на Схід, військові їдуть на Захід.
Ніхто не зупиняється в її місті.
Ніхто не хоче забрати її з собою.

Вона думає, стоячи зранку на своєму місці,
що навіть ця територія, виявляється, може бути бажаною і дорогою.
Що її, виявляється, не хочеться лишати надовго,

що за неї, виявляється, хочеться чіплятись зубами,
що для любові, виявляється, достатньо цього вокзалу старого
і літньої порожньої панорами.

Ніхто не пояснює їй, у чому причина.
Ніхто не приносить квіти на могилу її старшому брату.
Крізь сон чути, як у темряві формується батьківщина,

ніби хребет у підлітка з інтернату.
Формуються світло й темрява, складаючись разом.
Літнє сонце перетікає в зими.

Все, що діється нині з ними всіма, називається часом.
Головне розуміти, що все це діється саме з ними.
Формується її пам’ять, формується втіха.

В цьому місті народилися всі, кого вона знає.
Засинаючи, вона згадує кожного, хто звідси поїхав.
Коли згадувати немає кого, вона засинає.
\restorecr
