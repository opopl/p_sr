% vim: keymap=russian-jcukenwin
%%beginhead 
 
%%file 17_06_2017.stz.news.ua.mrpl_city.1.istoria_vizit_vracha
%%parent 17_06_2017
 
%%url https://mrpl.city/blogs/view/vizit-vracha
 
%%author_id burov_sergij.mariupol,news.ua.mrpl_city
%%date 
 
%%tags 
%%title История: Визит врача
 
%%endhead 
 
\subsection{История: Визит врача}
\label{sec:17_06_2017.stz.news.ua.mrpl_city.1.istoria_vizit_vracha}
 
\Purl{https://mrpl.city/blogs/view/vizit-vracha}
\ifcmt
 author_begin
   author_id burov_sergij.mariupol,news.ua.mrpl_city
 author_end
\fi

У доктора Позе в Мариуполе была прекрасная репутация. И как врача, который
успешно лечил взрослых и детей, и как человека отзывчивого, с чувством тонкого
юмора. Он обладал большими познаниями в областях, казалось бы, далеко отстоящих
от его профессии, например, в художественной литературе и отечественной
истории. В русский язык со второй половины XIX века вошло понятие \enquote{земский
врач}, включающее в себя ученого лекаря широкого профиля, посвятившего всего
себя делу попечения о народном здоровье, вплоть до самопожертвования. Так вот,
Георгий Георгиевич Позе был и оставался земским врачом. Не по должности,
поскольку она после революции семнадцатого года была повсеместно упразднена, но
по сути. 

%\ii{17_06_2017.stz.news.ua.mrpl_city.1.istoria_vizit_vracha.pic.1}

\enquote{Георгий Георгиевич пришел!} И все домочадцы, за исключением разве что лежачего
больного, спешили к нему навстречу. Кто с тазиком, кто с кувшином с теплой
водой, кто с мылом, кто с чистым, только что вынутым из комода полотенцем.
Доктор мыл руки очень тщательно, несколько раз намыливал их и столько же раз
смывал пену, омовению подвергался каждый палец, каждая его фаланга в
отдельности. С не меньшим тщанием вытирались руки. Пока шли эти процедуры,
Георгий Георгиевич как бы между прочим спрашивал: \enquote{Когда  обнаружилась высокая
температура? Была ли рвота? Что ел больной накануне?} Наконец он садился на
край постели, где лежал страждущий. Осматривал горло, прижав язык черенком
чайной ложки. Измерял частоту пульса, поглядывая на свои карманные
часы-\enquote{луковицу}. Затем в ход шел стетоскоп - раскладная деревянная трубка, с
помощью которой  Георгий Георгиевич прослушивал легкие и сердце. Стоит ли
описывать другие манипуляции?  Они знакомы каждому, кто побывал на приеме у
добросовестного врача-терапевта. Наконец доктор доставал из жилетного кармана
стопочку узких полосок бумаги, проштемпелеванных печатью \enquote{Врач Г. Г. Позе}. На
одной из них он мельчайшими латинскими буковками писал рецепт. Иногда рецепт
сопровождался короткой надписью: \enquote{Сito!}. Много лет спустя, уже в зрелом
возрасте, довелось узнать, что по-латыни это значит \enquote{срочно!}. Эта надпись
делалась не для того, кто должен был стремглав бежать в аптеку за лекарством, а
для провизора. В то время, когда доктор Позе врачевал мариупольцев, в аптеках
было относительно мало готовых лекарственных форм, разве что таблетки аспирина,
пирамидона, цитрамона, касторка в бутылочках, лечебные травы в картонных
коробочках и тому подобные снадобья, которые не могли принести большого вреда
при самолечении. Все остальное аптекари приготавливали сами по рецептам врача,
а для этого, естественно, нужно было время. Поэтому лекарство оплачивалось в
кассе, а взамен давалась квитанция, на которой указывалось время, когда нужно
прийти за готовым препаратом. А вот когда на рецепте было начертано: \enquote{Сito!}, к
приготовлению порошка ли, микстуры ли приступали немедленно,  в присутствии
заказчика. Ведь для провизоров рецепт врача имел силу закона...  

\ii{17_06_2017.stz.news.ua.mrpl_city.1.istoria_vizit_vracha.pic.1}

С какого времени Георгий Георгиевич Позе начал лечить представителей нашего
немалого тогда семейства, никто уже не скажет. Во всяком  случае, до войны –
это уж точно. В памяти четырехлетнего мальчонки  на многие годы запечатлелась
череда картинок из довоенной жизни. Мы идем с мамой по Большой улице... Подходим
к угловому дому на Советской... Крутая лестница, ведущая на второй этаж...  В углу
за дверью - сноп тростей с разнообразными набалдашниками, длинные зонтики с
загнутыми ручками... Седоватый сухонький человек с веселым прищуром глаз, со
щеточкой усов... И мамин полушепот мне на ухо: \enquote{Это доктор Позе!}...

Следующие воспоминания о Георгии Георгиевиче относятся к первым послевоенным
годам, когда он, возвратившись из Еревана, где служил в военном госпитале,
непродолжительное  время квартировал в доме дедушки. Здание, в котором он жил
до войны, было сожжено немцами.  Вечерами все, в том числе и доктор Позе,
собирались в комнате, служившей одновременно и кухней, и столовой. Под тихое
сопение чайника, стоявшего на плите, присутствующие с необыкновенным вниманием
слушали Георгия Георгиевича. Он делился  о пережитом. Иногда он мог рассказать
остроумный анекдот. Малолетних обитателей нашего жилища Георгий Георгиевич
потешал байками, вроде той, как некая дама объелась гречневой кашей, и у нее
лопнул живот, да так, что его пришлось зашивать суровыми нитками. Мы это
принимали за чистую монету. Запомнились некоторые его высказывания. \enquote{Врач
должен выглядеть опрятным}. И все видели, как он каждое утро чистил щеткой свой
пиджак и пальто, он всегда был тщательно выбрит. \enquote{Тот, кто лечит, не может себе
позволить иметь болезненный вид, а тем более говорить о своих болезнях, иначе
кто же ему поверит, что он может вылечить других}. Доктор Позе всегда был бодр
и энергичен. \enquote{Врач  не имеет права быть брезгливым. При осмотре больного он
должен присесть на край его постели, даже если она не совсем чиста. Этим
подчеркивается, что лечащий человек не боится болезни}.

То, что осталось в памяти из его рассказов и подкрепленное знаниями,
почерпнутыми из прочитанных книг, позволяет в какой-то мере воссоздать его
жизнеописание. Будущий врач родился в городе Вознесенске Херсонской губернии,
население которого составляло чуть больше пятнадцати тысяч жителей. Сведений о
том, где закончил Георгий Позе гимназию, нет. Во всяком случае, не в
Вознесенске, потому как такового учебного заведения там не было. Но то, что он
получил медицинское образование в Одессе, в  Новороссийском университете –
теперь он называется Одесским  –  известно доподлинно, поскольку не раз и не
два рассказывал доктор  о своей альма матер, о курьезах, связанных с
профессорами, о порядках, царивших там. Получив диплом лекаря (таково было в
царской России официальное название профессии врача, прослушавшего полный курс
медицинского факультета университета и успешно сдавшего все положенные
экзамены),  Георгий Георгиевич был приглашен земскими деятелями родного города
на вакансию в местную больницу. Но приглашение... не было принято.

Вопреки воле отца и к ужасу матери, новоиспеченный лекарь вместе с несколькими
сокурсниками отправился на ликвидацию вспышки холерной эпидемии в одну из южных
губерний России. Только после того, как положение дел с холерой исправилось,
доктор Позе вернулся в родные пенаты и занял ранее предложенную должность.
Будучи земским врачом, ему пришлось лечить взрослых и детей от всех болезней,
делать собственноручно прививки от инфекционных болезней и несложные
хирургические операции, вправлять вывихнутые конечности и даже принимать
осложненные роды, с обычными справлялись местные акушерки и бабки-повитухи.
Георгий Георгиевич не однажды подчеркивал, что работа  земского врача была
необыкновенно полезной практикой, пригодившейся ему в последующей работе. 

Однако работать в земской больнице ему пришлось работать не так уж долго.
Началась Первая мировая война. Доктор Позе был призван в действующую армию.
Сразу или через какое-то время был назначен начальником прифронтового лазарета.
В одном из сражений к расположению этого лазарета просочился сквозь линию
фронта австро-венгерский отряд. Пленение или даже гибель персонала лазарета и
раненых, казалась бы, была неминуема. Но военный врач Позе не растерялся,
схватил наган одного из унтер-офицеров охраны и с криком \enquote{Ура!} увлек за собой
в атаку всех, кто в тот момент мог держать в руках оружие. Неприятельский отряд
отступил. За этот подвиг он был награжден георгиевским крестом. Наверное, он
был единственным врачом во всей царской армии, удостоенным этой награды, так
как по статуту георгиевский крест получали только за заслуги боевые.

В годы гражданской войны Георгий Георгиевич был мобилизован в Красную Армию.
Где, когда и при каких обстоятельствах это произошло – неизвестно. Неизвестно
также, в каком году он перебрался в Мариуполь. В послевоенные годы, вплоть до
своей кончины в начале октября 1956 года, доктор Позе работал в городской
поликлинике, которая размещалась в двухэтажном особняке у сквера.

Страшно подумать, прошло более полувека со дня кончины Георгия Георгиевича
Позе, а перед внутренним взором нет-нет да и возникнет образ сухощавого
человека среднего роста в черном пальто с узким воротником, покрытым по моде
начала XX века черной бархоткой, шагающего с саквояжем в руке куда-нибудь на
Парковый поселок или на Торговую улицу, где с надеждой и нетерпением ждут его
больной  и его близкие.
