% vim: keymap=russian-jcukenwin
%%beginhead 
 
%%file 28_12_2021.fb.fb_group.story_kiev_ua.1.babushka_lelja
%%parent 28_12_2021
 
%%url https://www.facebook.com/groups/story.kiev.ua/posts/1828459464017513
 
%%author_id fb_group.story_kiev_ua,debourdeau_ljudmila
%%date 
 
%%tags babushka,kiev,semja
%%title Бабушка Лёля
 
%%endhead 
 
\subsection{Бабушка Лёля}
\label{sec:28_12_2021.fb.fb_group.story_kiev_ua.1.babushka_lelja}
 
\Purl{https://www.facebook.com/groups/story.kiev.ua/posts/1828459464017513}
\ifcmt
 author_begin
   author_id fb_group.story_kiev_ua,debourdeau_ljudmila
 author_end
\fi

Сегодня в преддверии Нового года хочу вам рассказать о моей замечательной
красавице бабушке Ольге Алексеевне Овсянниковой (бабушке Лёле), от которой мне
остался на память этот фарфоровый Дед Мороз (я думаю годов 50х). 

Лёле повезло родится в в прекрасной любящей семье в 1912 году в Киеве.
Прадедушка Алексей Овсянников-Овсиенко, коренной киевлянин, профессиональный
военный, дослужился до чина Штабс-капитан царской армии (из семьи Алексея
Карповича Овсиенко). Он встретил прабабушку Людмилу Акаёмову, потомственную
дворянку из города Симбирска, в Варшаве, где служил. Вскоре они поженились.
Появились на свет Лёля и её брат Жорж. О них я писала в моей прошлогодней
публикации. 

\raggedcolumns
\begin{multicols}{3} % {
\setlength{\parindent}{0pt}

\ii{28_12_2021.fb.fb_group.story_kiev_ua.1.babushka_lelja.pic.1}
\ii{28_12_2021.fb.fb_group.story_kiev_ua.1.babushka_lelja.pic.1.cmt}

% 1914 год
\ii{28_12_2021.fb.fb_group.story_kiev_ua.1.babushka_lelja.pic.2}
\ii{28_12_2021.fb.fb_group.story_kiev_ua.1.babushka_lelja.pic.2.cmt}

% 1915 год
\ii{28_12_2021.fb.fb_group.story_kiev_ua.1.babushka_lelja.pic.3}
\ii{28_12_2021.fb.fb_group.story_kiev_ua.1.babushka_lelja.pic.3.cmt}

\end{multicols} % }


\raggedcolumns
\begin{multicols}{2} % {
\setlength{\parindent}{0pt}

\ii{28_12_2021.fb.fb_group.story_kiev_ua.1.babushka_lelja.pic.4}
\ii{28_12_2021.fb.fb_group.story_kiev_ua.1.babushka_lelja.pic.4.cmt}

% 1913 год
\ii{28_12_2021.fb.fb_group.story_kiev_ua.1.babushka_lelja.pic.5}
\ii{28_12_2021.fb.fb_group.story_kiev_ua.1.babushka_lelja.pic.5.cmt}

\end{multicols} % }

\zzrule

После скитаний по гарнизонам семья обосновалась в Киеве. Вскоре началась Первая
Мировая война, Алексей воевал два года, затем после тяжелого ранения вернулся в
Киев, долго лечился, остался инвалидом, но выжил. Затем наступили неспокойные
голодные годы революции и Гражданской войны. Семье пришлось очень тяжело, еле
выживали. Прабабушка Людмила шила и продавала вещи, прадед Алексей работал
билетёром в кинотеатре.

\zzrule

\raggedcolumns
\begin{multicols}{3} % {
\setlength{\parindent}{0pt}

% 1936 год у дома на Гончарной улице с моей маленькой тётей Музой.
\ii{28_12_2021.fb.fb_group.story_kiev_ua.1.babushka_lelja.pic.6}
\ii{28_12_2021.fb.fb_group.story_kiev_ua.1.babushka_lelja.pic.6.cmt}

% С подругами в санатории, бабушка вторая снизу, 30е годы
\ii{28_12_2021.fb.fb_group.story_kiev_ua.1.babushka_lelja.pic.7}
\ii{28_12_2021.fb.fb_group.story_kiev_ua.1.babushka_lelja.pic.7.cmt}

% 1930 год
\ii{28_12_2021.fb.fb_group.story_kiev_ua.1.babushka_lelja.pic.8}
\ii{28_12_2021.fb.fb_group.story_kiev_ua.1.babushka_lelja.pic.8.cmt}

% 1936 год, Севастополь, бабушка с дедушкой
\ii{28_12_2021.fb.fb_group.story_kiev_ua.1.babushka_lelja.pic.9}
\ii{28_12_2021.fb.fb_group.story_kiev_ua.1.babushka_lelja.pic.9.cmt}

% 1936 год, бабушка и дедушка Наум
\ii{28_12_2021.fb.fb_group.story_kiev_ua.1.babushka_lelja.pic.10}
\ii{28_12_2021.fb.fb_group.story_kiev_ua.1.babushka_lelja.pic.10.cmt}

% 1940 год
\ii{28_12_2021.fb.fb_group.story_kiev_ua.1.babushka_lelja.pic.11}
\ii{28_12_2021.fb.fb_group.story_kiev_ua.1.babushka_lelja.pic.11.cmt}

% 1942 год, в эвакуации в городе Чкалове, бабушка с моей маленькой мамой и тётей
\ii{28_12_2021.fb.fb_group.story_kiev_ua.1.babushka_lelja.pic.12}
\ii{28_12_2021.fb.fb_group.story_kiev_ua.1.babushka_lelja.pic.12.cmt}

\end{multicols} % }

\zzrule

Но в 1931 году Алексея арестовали по сфабрикованному делу «Весна» и
репрессировали. Бабушке было всего 18 лет. Так она стала дочерью врага народа. 

Лёля выучилась на бухгалтера и стала кормить мать, брата, а вскоре и маленькую
дочку, после короткого и неудачного брака.

В середине 30х она встретила моего дедушку Наума, в 1940 появилась на свет моя
мама Лора. А затем была война. Лёля вывезла всю семью и ближайших родственников
в эвакуацию в город Чкалов. Дедушка был на фронте. Слава Богу все выжили и
вернулись в Киев.

После войны Лёля очень много работала, имела ответственный пост, главного
бухгалтера треста столовых города Киева. Она вырастила дочерей, а затем и
внуков (меня  @igg{fbicon.smile} ). Меня она обожала. Мы были очень близки. Бабушка жила с нами с
моего рождения. В конце жизни она очень сильно болела и умерла на руках у моей
мамы. Осталось много фотографий, стихов, этот чудесный Дед Мороз и прекрасные
воспоминания.

Поздравляю всех с наступающими праздниками, Новым годом и Рождеством Христовым!
