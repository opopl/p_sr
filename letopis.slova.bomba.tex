% vim: keymap=russian-jcukenwin
%%beginhead 
 
%%file slova.bomba
%%parent slova
 
%%url 
 
%%author 
%%author_id 
%%author_url 
 
%%tags 
%%title 
 
%%endhead 
\chapter{Бомба}
\label{sec:slova.bomba}

%%%cit
%%%cit_head
%%%cit_pic
%%%cit_text
\emph{Бомбардировщик} Су-24М выполнил предупредительное \emph{бомбометание} по курсу движения
британского эсминца Defender, нарушившего российскую границу в Черном море. Об
этом сообщило Минобороны России, передает ТАСС.  Уточняется, что эсминец
Военно-морских сил (ВМС) Великобритании пересек государственную границу России
и вошел в территориальное море в районе мыса Фиолент на три километра.
Черноморский флот совместно с Погранслужбой ФСБ предупредили экипаж корабля о
применении оружия, однако тот никак не отреагировал. Тогда самолет Су-24М
выполнил предупредительное \emph{бомбометание} по курсу движения Defender
%%%cit_comment
%%%cit_title
\citTitle{Российский Су-24 сбросил бомбы по курсу британского эсминца в Черном море}, 
Варвара Кошечкина, lenta.ru, 23.06.2021
%%%endcit

%%%cit
%%%cit_head
%%%cit_pic
%%%cit_text
Старшина сверхсрочной службы Талгат Аюпов наблюдал за взрывом термоядерной
\emph{бомбы} 30 октября 1961 года из поселка Белушья Губа в юго-западной части
Южного острова — это крупнейший населенный пункт на всем архипелаге. Он
запомнил клокочущий ядерный шар и вспышки.  «До нас дошла сейсмическая, а сразу
за ней и воздушная ударная волна, — рассказал Аюпов «Ленте.ру». — Следом
громоподобные мощные звуки. Всех нас предварительно вывели из казарм. В поселке
никаких разрушений не было».  В начале 1962 года, когда уровень радиации в
эпицентре должен был снизиться, специальная комиссия решила оценить ущерб,
нанесенный испытанием «\emph{Царь-бомбы}».  Вертолет совершил посадку в поселке
Северный у пролива Маточкин Шар, примерно в 50 километрах от места, куда ее
сбросили. По словам Аюпова, увиденное его поразило
%%%cit_comment
%%%cit_title
\citTitle{«Дом будто ножом срезало» 60 лет назад СССР взорвал «Царь-бомбу» — самую мощную в истории. Что помнят о взрыве очевидцы?: Общество: Россия: Lenta.ru}, Дмитрий Окунев, lenta.ru, 30.10.2021
%%%endcit
