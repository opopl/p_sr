% vim: keymap=russian-jcukenwin
%%beginhead 
 
%%file 16_01_2022.stz.news.ua.mrpl_city.1.vystavka_uchniv_abstrakcia
%%parent 16_01_2022
 
%%url https://mrpl.city/blogs/view/vistavka-uchniv-hudozhnogo-viddilennya-mariupolskoi-shkoli-mistetstv-abstraktsiya
 
%%author_id demidko_olga.mariupol,news.ua.mrpl_city
%%date 
 
%%tags 
%%title Виставка учнів художнього відділення Маріупольської школи мистецтв "Абстракція"
 
%%endhead 
 
\subsection{Виставка учнів художнього відділення Маріупольської школи мистецтв \enquote{Абстракція}}
\label{sec:16_01_2022.stz.news.ua.mrpl_city.1.vystavka_uchniv_abstrakcia}
 
\Purl{https://mrpl.city/blogs/view/vistavka-uchniv-hudozhnogo-viddilennya-mariupolskoi-shkoli-mistetstv-abstraktsiya}
\ifcmt
 author_begin
   author_id demidko_olga.mariupol,news.ua.mrpl_city
 author_end
\fi

Початок 2022 року для мене ознаменувався знайомством з юними маріупольськими
художниками. Спочатку мені пощастило відвідати виставку робіт студентів та
викладачів Маріупольської філії Національній академії образотворчого мистецтва
і архітектури \enquote{Мистецька імпреза}, про яку я розповідала нещодавно. А тепер
хочу познайомити читачів ще з однією унікальною виставкою учнів художнього
відділення Маріупольської школи мистецтв \emph{\textbf{\enquote{Абстракція}}}, що працює з кінця грудня
в BEETROOT IT HUB, і на яку ще є час потрапити. Насправді всі роботи, що
представлені на цій виставці були створені в рамках справжнього творчого
експерименту.

\ii{16_01_2022.stz.news.ua.mrpl_city.1.vystavka_uchniv_abstrakcia.pic.1}

Організаторка виставки, викладачка Маріупольської школи мистецтв \textbf{Олена Сидорова}
наголосила, що 
\begin{quote}
\enquote{абстракція – це такий напрям, який допомагає висловити почуття
та розкритися. Діти дуже закомплексовані і дуже важко відкриваються, тому
робота в цьому напрямі для них і важлива, і корисна водночас}. 
\end{quote}
Викладачка почала цікаві вправи, завдяки яким учням дозволялося абсолютно все:
бризкати, руками розводити фарби на картині, не було ніяких обмежень. Всі свої
емоції можна було залишати на полотні, тому викладачка вирішила познайомити
своїх учениць з абстракціонізмом. Головна умова для дітей – не закриватися та
не соромитися. Роботи, що представлені на виставці \enquote{Абстракція} були створені
досить незвично. До учнів прийшов музикант. Він імпровізував на фортепіано, а
вони імпровізували на полотні. Спочатку це було складно, адже діти боялися
зробити щось не так. Але результати були досить вражаючими. На виставці
представлені роботи не лише дітей які малювали під музику, а ще й учнів, які
намагалися через фарби і лінії зобразити конкретний образ. Учні вважають, що
малювати під музику часом навіть необхідно.

\ii{16_01_2022.stz.news.ua.mrpl_city.1.vystavka_uchniv_abstrakcia.pic.2}

В абстракціонізмі художники  за допомогою плями, лінії, крапки та фарб могли
показати сум чи радість. Для деяких авторів  створення картин в такому напрямі
є незвичним але цікавим заняттям. Зокрема, учениця художнього відділення
Маріупольської школи мистецтв \textbf{Олена Варовей} зазначила, що це не зовсім той
напрям, в якому вона готова працювати постійно, але вважає, що це цікавий
досвід. Дівчина під музику намагалася зобразити хвилі почуттів. Її картина
нагадує море.

\ii{16_01_2022.stz.news.ua.mrpl_city.1.vystavka_uchniv_abstrakcia.pic.3}

А інша учениця \textbf{Владислава Хатуєва} вирішила пов'язати свою подальшу творчість
саме з напрямом абстракціонізму, який є найближчим для неї як художниці.
Художниця вважає, що працюючи над абстракціями можна не тільки висловити свої
почуття, а й позбутися якихось негативних емоцій. Це своєрідна арт-терапія.
Вона зобразила на своєму полотні чоловіка, який не може впоратися зі своїми
емоціями та втрачає контроль над собою та власним життям.

\ii{16_01_2022.stz.news.ua.mrpl_city.1.vystavka_uchniv_abstrakcia.pic.4}

Приємно вражає, що у учениць вже є чітке бажання наслідувати видатних
художників. Зокрема, \textbf{Софія Каракурчи} хотіла б працювати у техніці відомого
американського художника, одного з найвизначніших представників абстрактного
експресіонізму Джексона Поллока.

Для юних художниць це цікавий експеримент, який викликав лише позитивні емоції.
Учениця художнього відділення Маріупольської школи мистецтв \textbf{Анастасія Лисак}
зауважила, що робота в цьому напрямі допомагає художникам відкрити внутрішній
світ, зрозуміти самого себе, а також трохи розслабитися, адже під час створення
картини не відчуваєш жодних обмежень.

\ii{16_01_2022.stz.news.ua.mrpl_city.1.vystavka_uchniv_abstrakcia.pic.5}

Всім авторам приємно, що їхні роботи можуть подивитися мешканці міста, які
побачать в представлених картинах щось своє, адже абстракціонізм спрямований на
те, щоб викликати у людини певні емоції та асоціації. Всі форми та колірні
поєднання, які знаходяться в периметрі зображення, мають ідею, своє вираження
та смислове навантаження. Якби не здавалося глядачеві, дивлячись на картину, де
крім ліній і плям нічого немає, все в абстракції підпорядковується певним
правилам вираження, та званої \enquote{абстрактної композиції}.

Цікаво, що  для учениць Маріупольської школи мистецтв це не перша виставка в
BEETROOT IT HUB. Їхня викладачка \textbf{Олена Сидорова} намагається постійно надихати
учениць на нові унікальні виставки. Головна мета виставок – познайомити жителів
міста з різними мистецькими  напрямами та художниками і загалом різними видами
мистецтва. Ця ідея виникла, тому що на уроках в Маріупольській школі мистецтв
учні вивчають зовсім різні напрями і намагаються створювати за мотивами відомих
робіт власні картини. Викладачка вирішила проводити виставки, адже це дуже
стимулює учнів. Перша виставка \emph{\enquote{Кубізм}} була представлена за мотивами
робіт Пабло Пікассо. Яскраві картини учнів схвалила адміністрація BEETROOT IT
HUB, після чого була представлена друга виставка. Це були \emph{\enquote{Художники XX
століття і сюрреалізм}}. Учні представили свої роботи, які були створені за
мотивами відомих картин Сальвадора Далі, Фріди Кало, Рене Магрітта. Третя
виставка – \emph{\enquote{Попарт}} – завдяки своїй унікальності та незвичноті
привернула до себе найбільшу увагу.

\ii{16_01_2022.stz.news.ua.mrpl_city.1.vystavka_uchniv_abstrakcia.pic.6}

Виставка \enquote{Абстракція} триватиме до 18 січня. Втім впевнена, що це не
остання виставка  і вже незабаром можна буде побачити нові роботи учнів
художнього відділення Маріупольської школи мистецтв.

