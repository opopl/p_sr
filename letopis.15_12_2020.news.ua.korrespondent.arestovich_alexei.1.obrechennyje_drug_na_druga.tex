% vim: keymap=russian-jcukenwin
%%beginhead 
 
%%file 15_12_2020.news.ua.korrespondent.arestovich_alexei.1.obrechennyje_drug_na_druga
%%parent 15_12_2020
 
%%url https://blogs.korrespondent.net/blog/politics/4306601/
 
%%author Арестович, Алексей
%%author_id arestovich_alexei
%%author_url 
 
%%tags ukraina,mysli,nacia
%%title Обреченные друг на друга
 
%%endhead 
 
\subsection{Обреченные друг на друга}
\label{sec:15_12_2020.news.ua.korrespondent.arestovich_alexei.1.obrechennyje_drug_na_druga}
\Purl{https://blogs.korrespondent.net/blog/politics/4306601/}
\ifcmt
	author_begin
   author_id arestovich_alexei
	author_end
\fi

\ifcmt
  pic https://kor.ill.in.ua/m/610x385/2573162.jpg
  width 0.5
  fig_env wrapfigure
\fi

\begin{leftbar}
	\begingroup
		\em\large\bfseries
Если чему и учит жизнь в Украине, так это вообще не обращать внимание на чьё-то
				мнение.
	\endgroup
\end{leftbar}

Мнение девальвированы полностью - потому, что не стоят выеденного яйца.

А не стоят потому, что мало кто их берётся, хоть как-нибудь обосновать хоть на
сколько-нибудь реальной основе.

Некогда. Все комментируют чужие мнения, основываясь на мнениях чужих.

Этой лавиной комментариев на комментарии, этим бесконечным митингом,
бессмысленным и беспощадным, мы пытаемся обмануть свою бесконечную тревогу
неопределенности, вызванную нашей неспособностью дать общий ответ на вопросы:

\begin{itemize}
  \item - кто мы?
  \item - где мы?
  \item - в каких отношениях мы находимся?
  \item - и зачем это все?
\end{itemize}

Мы ненавидим, подозреваем и выталкиваем друг друга, потому, что не знаем, зачем
мы друг другу даны и что нам с друг другом делать?

Украинец для украинца - сверхраздражитель, и все мы живем в худшем из возможных
для нас аду - обречённые друг на друга.

При большом желании из этого раздражения можно было бы сделать рай - только мы
не умеем желать.

Больные люди, скрученные волчком от ненависти к самим себе, беззвучно кричащие
друг на друга, потому, что давно оглохли в этой звонкой пустоте.

Недолюбленные дети ненавидящих взрослых, верящие в то, что, если громко и долго
кричать, им удастся выкричать это проклятие.

Это было чужое для вас мнение. 

