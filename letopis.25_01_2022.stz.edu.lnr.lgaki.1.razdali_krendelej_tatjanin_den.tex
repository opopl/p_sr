% vim: keymap=russian-jcukenwin
%%beginhead 
 
%%file 25_01_2022.stz.edu.lnr.lgaki.1.razdali_krendelej_tatjanin_den
%%parent 25_01_2022
 
%%url https://lgaki.info/novosti/razdali-krendelej-ministr-kultury-sporta-i-molodezhi-lnr-i-rektor-pozdravili-studentov-akademii-s-tatyaninym-dnem
 
%%author_id lgaki
%%date 
 
%%tags lgaki,lnr,lugansk,pozdravlenie,prazdnik,prazdnik.tatjanin_den,studenty
%%title «Раздали кренделей»: министр культуры, спорта и молодежи ЛНР и ректор поздравили студентов Академии с Татьяниным днем
 
%%endhead 
 
\subsection{«Раздали кренделей»: министр культуры, спорта и молодежи ЛНР и ректор поздравили студентов Академии с Татьяниным днем}
\label{sec:25_01_2022.stz.edu.lnr.lgaki.1.razdali_krendelej_tatjanin_den}
 
\Purl{https://lgaki.info/novosti/razdali-krendelej-ministr-kultury-sporta-i-molodezhi-lnr-i-rektor-pozdravili-studentov-akademii-s-tatyaninym-dnem}
\ifcmt
 author_begin
   author_id lgaki
 author_end
\fi

В День студента в главном творческом вузе Донбасса наградили лауреатов премии
«Ты – лучший» (традицию учредили год назад). В списке – студенты всех
факультетов Академии Матусовского и отделений колледжа ЛГАКИ, особенно
отличившиеся в учебе и научной деятельности, в творчестве и самоуправлении.

Засвидетельствовавшие успех грамоты обладателям премии вручили министр
культуры, спорта и молодежи ЛНР Дмитрий Сидоров и ректор вуза Валерий Филиппов.
А саму церемонию провели председатель Студсовета Академии Елена Мосьпан и ее
заместитель Владислав Кисленко.

— От вас, от того, насколько тщательно вы овладеваете своей профессией сегодня
в стенах вуза, зависит настроение граждан нашей Республики через несколько лет,
— напомнил Дмитрий Сидоров.

— Студенчество — самый прекрасный период в жизни. Успевайте и учиться и жить,
не жалейте себя! — напутствовал студентов Валерий Филиппов.

После торжественной церемонии пришло время угощенья — студентам раздали
кренделя. Их, к слову, по одной из версий, еще в седьмом веке придумали
итальянские монахи как раз для того, чтобы поощрить детвору за усердие в учебе
– за то, что выучили молитвы.

А студентам Академии их раздали в честь Всероссийского дня студента, в честь
студенческой дружбы, в честь их успехов и достижений, в честь грядущего
20-летия Академии!

Фото – Марина Машевски.
