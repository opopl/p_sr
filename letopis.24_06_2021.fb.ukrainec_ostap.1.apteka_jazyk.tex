% vim: keymap=russian-jcukenwin
%%beginhead 
 
%%file 24_06_2021.fb.ukrainec_ostap.1.apteka_jazyk
%%parent 24_06_2021
 
%%url https://www.facebook.com/permalink.php?story_fbid=4252002088193586&id=100001515094776
 
%%author 
%%author_id ukrainec_ostap
%%author_url 
 
%%tags apteka,jazyk,mova,ukraina,zhizn
%%title Слухайте, це щелепа їм так тисне на мозок, чи в чому справа?
 
%%endhead 
 
\subsection{Слухайте, це щелепа їм так тисне на мозок, чи в чому справа?}
\label{sec:24_06_2021.fb.ukrainec_ostap.1.apteka_jazyk}
 
\Purl{https://www.facebook.com/permalink.php?story_fbid=4252002088193586&id=100001515094776}
\ifcmt
 author_begin
   author_id ukrainec_ostap
 author_end
\fi

Аптека.

\begin{itemize}
  \item - Доброго дня, дайте, будь ласка, ібупрофен.
  \item - *нерозбірлива фраза російською*
  \item - Що, перепрошую?
  \item - *нерозбірлива фраза російською*
  \item - Що?
  \item - *те саме*
  \item - Перепрошую, чому ви відповідаєте мені російською?
  \item - А ви що, хотіли, щоб я вам українською відповідала?
\end{itemize}

Слухайте, це щелепа їм так тисне на мозок, чи в чому справа?

Коли що – я не білінґв. Я знаю російську, але перейти на неї мені не легше, ніж
на польську. Тобто потрібне свідоме зусилля. До якого не схиляє ні мігрень, ні
факт перебування в Україні, бо я цілком логічно чекаю від співрозмовника
кооперації в комунікативному процесі.

Тому правильна відповідь на питання "якщо ти потрапиш у лікарню і з тобою лікар
говоритиме російською" – так, матері вашій ковінька, мені є різниця. 

Бо в стані шоку я теж маю всі шанси просто не зрозуміти російськомовного
лікаря. Так працює вивчена, а не рідна мова.

\ii{24_06_2021.fb.ukrainec_ostap.1.apteka_jazyk.cmt}
