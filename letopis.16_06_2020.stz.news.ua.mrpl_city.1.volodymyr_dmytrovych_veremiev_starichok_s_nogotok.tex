% vim: keymap=russian-jcukenwin
%%beginhead 
 
%%file 16_06_2020.stz.news.ua.mrpl_city.1.volodymyr_dmytrovych_veremiev_starichok_s_nogotok
%%parent 16_06_2020
 
%%url https://mrpl.city/blogs/view/volodimir-dmitrovich-veremiev-mariupolskij-starichok-s-nogotok
 
%%author_id demidko_olga.mariupol,news.ua.mrpl_city
%%date 
 
%%tags 
%%title Володимир Дмитрович Веремієв: маріупольський "старичок с ноготок"
 
%%endhead 
 
\subsection{Володимир Дмитрович Веремієв: маріуполь\hyp{}ський \enquote{старичок с ноготок}}
\label{sec:16_06_2020.stz.news.ua.mrpl_city.1.volodymyr_dmytrovych_veremiev_starichok_s_nogotok}
 
\Purl{https://mrpl.city/blogs/view/volodimir-dmitrovich-veremiev-mariupolskij-starichok-s-nogotok}
\ifcmt
 author_begin
   author_id demidko_olga.mariupol,news.ua.mrpl_city
 author_end
\fi

\ii{16_06_2020.stz.news.ua.mrpl_city.1.volodymyr_dmytrovych_veremiev_starichok_s_nogotok.pic.1}

Дуже хочеться розповісти про людину, що не втомлюється дарувати радість іншим і
вже давно стала позитивним символом Марі\hyp{}уполя. Маріупольці \emph{\textbf{Володимира
Дмитровича Веремієва}} називають по-різному: \enquote{дідусь Вова}, \enquote{маленький Вова},
\enquote{Вовочка}, \enquote{Дід Мороз} але він би хотів, щоб його називали \enquote{старичок с
ноготок}. 74-річний Володимир – неймовірно глибока і цікава Людина. За його
святковою і цирковою маскою приховано безліч мудрих думок. Жонглюючи унікальним
саморобним цирковим реквізитом біля підземного переходу чи тисячі дрібниць, він
намагається розвеселити вибагливу маріупольську публіку. Його програма
складається з різноманітних трюків з обмотаною різнобарвною тканиною палицею,
мотузкою, обв'язаною пакетиками, обручем та різнокольоровим шапками. Костюмів у
Володимира Дмитровича багато, адже виступає круглий рік. Спочатку купляв, потім
почав сам шити.

Досі не всі маріупольці сприймають дідуся позитивно. Але робити щось всупереч
усьому і всім він звик з дитинства. Тому навчився не звертати увагу і не
ображатися. На щастя, кількість тих маріупольців, кому позитивний дідусь
подобається, з кожним роком лише зростає. Сьогодні він вже є справжньою зіркою
міста і вважається одним з найбільш відомих маріупольців. Виступати
подобається, завжди мріяв стати видатним спортсменом чи артистом. Позитивний
дідусь вважає, що його виступи потрібні більше тим маріупольцям, які бояться
бути сміливими, щирими і справжніми. Каже, що виглядати смішно не страшно,
страшно – не намагатися займатися тим, що приносить справжнє задоволення.

%\ii{16_06_2020.stz.news.ua.mrpl_city.1.volodymyr_dmytrovych_veremiev_starichok_s_nogotok.pic.1_2}
\ii{16_06_2020.stz.news.ua.mrpl_city.1.volodymyr_dmytrovych_veremiev_starichok_s_nogotok.pic.2}

Розповідаючи про своє дитинство, Володимир Дмитрович не приховував сліз. Він
зізнався, що однолітки його сильно ображали: через маленький ріст глузували,
часто били, знущались. Проте хлопчик вирішив не падати духом і пообіцяв собі
довести всьому світу, що маленький ріст не завадить йому стати щасливим. Щоб
навчитися себе захищати з юних років займався різними видами спорту: карате,
дзюдо, гімнастика. Найбільше маленькому Володі подобалося грати у футбол. Коли
Володимиру було 20 років, він служив в армії у місті Загорську Московської
області. Направили його в будівельний батальйон, де він: штукатурив, клеїв
шпалери. З юних років намагався багато працювати, не боявся труднощів. 25 років
пропрацював на аглофабриці комбінату ім. Ілліча. Був визнаний ветераном праці.

\enquote{Старичок с ноготок} вже більше 50 років живе з дружиною в 21-му мікрорайоні.
Свою жінку дуже поважає. Наголошує, що саме завдяки витримці і терпінню дружини
вдалося створити міцну і щасливу родину. Вважає, що в сім'ї все завжди
тримається на жінці, на її мудрості і щоденній праці. Чоловік повинен
забезпечувати сім'ю, а дружина створювати затишок і комфорт. Разом вони
виховали двох дітей і зараз доглядають трьох онуків. Незважаючи на свої щоденні
виступи намагається встигати виконувати і домашні справи: вранці на ринок за
продуктами, ввечері – на джерело за водою.

Попри поважний вік, дід Володя продовжує підтримувати себе у чудовій фізичній
формі. Обов'язковими залишаються пробіжка і вправи з гантелями. Впевнений, що
секрет довголіття – у здоровому способі життя, адже сам ніколи не курив і не
пив. Лікуватися намагається народними методами. Найпозитивніший маріуполець
вважає, що коронавірус, як і будь-яка інша хвороба, нестрашний для веселих,
добрих і спортивних людей. Вражає, що Володимир Дмитрович спить зовсім мало.
Години сну йому вистачає, щоб відчувати себе бадьорим протягом всього дня.

У Маріуполі найбільше любить проводити час саме біля підземного переходу. На
його думку, там зосереджена найкраща енергія в місті, тому, проводячи час в
цьому місці, можна отримати потужний заряд позитиву (особливо з 12 до 14 годині
дня). Наприкінці бесіди \enquote{старичок с ноготок} вклонився всій планеті та всім
маріупольцям і побажав міцного здоров'я та святкового настрою.

\emph{Улюблена книга:} \enquote{Буквар першокласника}.\par
\bigskip
\emph{Улюблений фільм:} \enquote{Чапаєв} (1934 рік).\par
\bigskip

\textbf{\emph{*Команда MRPL.CITY уважает и любит Владимира Дмитриевича, предлагаем вам посмотреть несколько видео с ним:}}\par
\bigskip

\emph{Мариуполец дедушка Вова раскрыл несколько фактов из своей жизни (ВИДЕО)}%
\footnote{Мариуполец дедушка Вова раскрыл несколько фактов из своей жизни (ВИДЕО), Олена Онєгіна, mrpl.city, 20.10.2018, \par%
\url{https://mrpl.city/news/view/mariupolets-dedushka-vova-raskryl-neskolko-faktov-iz-svoej-zhizni-video}
}

\emph{Мариупольский дедушка Вова рассказал о спорте и личном \enquote{эликсире молодости} (ВИДЕО)}%
\footnote{Мариупольский дедушка Вова рассказал о спорте и личном \enquote{эликсире молодости} (ВИДЕО), Олена Онєгіна, %
mrpl.city, 19.10.2018, \par%
\url{https://mrpl.city/news/view/mariupolskij-dedushka-vova-rasskazal-o-sporte-i-lichnom-e-liksire-molodosti-video}
}
