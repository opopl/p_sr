% vim: keymap=russian-jcukenwin
%%beginhead 
 
%%file slova.chelovechestvo
%%parent slova
 
%%url 
 
%%author_id 
%%date 
 
%%tags 
%%title 
 
%%endhead 
\chapter{Человечество}

%%%cit
%%%cit_head
%%%cit_pic
%%%cit_text
Есть мнение, что цифровой мир меняет человека чуть ли не биологически. Может
быть когда-нибудь это и произойдет. Но пока человек обживает цифровое
пространство также, как новую пещеру или жилище из костей мамонта в палеолите.
Для начала он просто населяет его цифровыми духами. Германия готовится провести
весною масштабную выставку "Кибершаманизм". С духами, оказывается, можно
общаться через смартфон и научится этому при желании. А дальше, будьте добры,
ознакомьтесь с "Манифестом киберведьм", чтобы овладеть кибермагией.  С
нетерпением жду появления манифеста киберинквизиции и сожжения этих
инфернальных тварей на киберкострах...  Естественно, все это именуется цифровой
духовностью и духовными киберпрактиками. Психосоциальную потребность в инобытии
надо же как-то удовлетворять.  Цифровые оптимисты считают, что технологии ведут
\emph{человечество} по пути прогресса. Но, даже если дубину в руках первобытного
человека заменить на смартфон, его сознание, судя по всему, будет проходить те
же стадии религиозно- мифологической трансформации
%%%cit_comment
%%%cit_title
\citTitle{Мир постепенно погружается в цифровую магию и мистику / Лента соцсетей / Страна}, 
Владислав Михеев, strana.news, 02.12.2021
%%%cit_url
\href{https://strana.news/opinions/365165-mir-pohruzhaetsja-v-tsifrovuju-mahiju-i-mistiku.html}{link}
%%%endcit
