% vim: keymap=russian-jcukenwin
%%beginhead 
 
%%file 21_01_2022.yz.tkachev_andrej.protoierej.1.isskustvo.cmt
%%parent 21_01_2022.yz.tkachev_andrej.protoierej.1.isskustvo
 
%%url 
 
%%author_id 
%%date 
 
%%tags 
%%title 
 
%%endhead 
\zzSecCmt

\begin{itemize} % {
\iusr{Александр Шилов}

 @igg{fbicon.thumb.up.yellow}  @igg{fbicon.hands.applause.yellow}{repeat=3} 

О целебном действии классической музыки говорилось давно.

Я думаю, что в отличие от рока или попсы это так и есть.

\begin{itemize} % {
\iusr{Канис Александр}

Александр, Рамштайн чудесно помогает от уныния...
\iusr{Александр Шилов}

Канис, на мой взгляд, от уныния лучше помогут AS/DS или QUEEN.

Но классика, считается более полезной для здоровья.
\iusr{Eugen Nikiforov}

Александр, почему именно классической? Любая музыка в классическом
инструментальном исполнение приобретает очень приятные и лечебные свойства.
Ведь дело в самих звуках и манере исполнения, а какая там мелодия это уже на
вкус и цвет)
\end{itemize} % }

\iusr{Теперь местный}

Нужно, конечно нужно! Но все должно быть разумно, от души и безо всякого,
прости Господи, хайпа!!!

\begin{itemize} % {
\iusr{Eugen Nikiforov}

Теперь местный, вот этого к слову и не хватает, а на поверхности как всегда
всякий мусор плавает, самые золотые залежи где-то на дне океана)
\end{itemize} % }

\iusr{Вадим Алекперов}

Ура! Наконец-то созвучный мне пост! Спасибо!

\iusr{Людмила Калинина}

Тема сродни вопросам - нужен ли человеку воздух? Или вода? И простые ответы
-смотря каким воздухом дышать и какую воду пить... Так и искусство.

\iusr{Иван Иванов}

Младшему внуку четыре года... Когда везу его в машине, всё время говорит мне -
дед, пой! Пой, как в храме! Еду и пою, а он сидит слушает, и о чём то думает...
Ребёнка не обманешь. И ещё. Я руководитель русского народного хора...
Собираются мои артисты на репетицию. Приходит одна женщина, устала - шла через
метель... Говорит: \enquote{Иду и думаю, - за хлебом не пошла бы!} Вот и рассуждайте -
нужно ли нам искусство?

\iusr{Анна М.}

Подскажите, пожалуйста, где простому народу послушать по радио классическую
музыку? Есть ли такое радио и на каких широтах? Интернет не предлагать, он
слабый, на диски денег нет, на филармонию - нет и денег и времени.

\iusr{Alex Solace}

Искусство дано нам для того, чтобы мы не умерли от правды (мира сего).

Правда Божия дана нам для того, чтобы мы не умерли от искусства (человеческого,
слишком человеческого).

\iusr{Александр Андреев}

Исскуство это очень сильная вещь и обладает очень большим воздействием на
людей. Оно увлекает их за собой. Композиторы, поэты, скульпторы, художники,
драматургии, хореографы и прочие люди творческих мрофессий. должны понимать
какую ответственность они несут и какое влияние их произведения оказывают на
человека. Композитор может написать прекрасную музыку, на неё ложаться странные
стихи и исполняет песню Шура. Или например песня \enquote{Секшн революшн} где хорошая
музыка внедряет в голову наивной молодёжи мерзкие мысли. Нужно понимать, для
чего будет использоваться, твоё творение. И чем гениальный твоё творение, тем
большее воздействие оно окажет на людей и нужно приложить все усилия, чтобы
влияние было не со знаком \enquote{-}. Часто человеку даётся дар Свыше и он
злоупотребляя им неожиданно для себя его лишается. Но \enquote{дар} от демонов не
исчезает, он овдадевает человеком и от этой страсти трудно избавиться. Культура
это инструмент и очень важно в чьих руках он находится.

\end{itemize} % }
