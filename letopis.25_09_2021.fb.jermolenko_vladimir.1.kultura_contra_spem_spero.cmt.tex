% vim: keymap=russian-jcukenwin
%%beginhead 
 
%%file 25_09_2021.fb.jermolenko_vladimir.1.kultura_contra_spem_spero.cmt
%%parent 25_09_2021.fb.jermolenko_vladimir.1.kultura_contra_spem_spero
 
%%url 
 
%%author_id 
%%date 
 
%%tags 
%%title 
 
%%endhead 
\subsubsection{Коментарі}

\begin{itemize} % {
\iusr{Serhiy Bochechka}
Ви краще скажіть, чи є майбутнє в кафедри літературознавства і хто нею наразі керує  @igg{fbicon.wink} 

\iusr{Vasyl Babych}
\textbf{Volodymyr}, 

дякую за цей допис. Часто задумуюся і шукаю причини складності для українців/ок
працювати з майбутнім і горизонтом планування. І водночас є суб'єктивне
відчуття, що у покоління теперішніх 20річних (народжених після 2000р.) з цим
точно краще через стійке відчуття, що Україна була, є і буде no matter what.


\iusr{Igor Holfman}
только избавившись от "исторической судьбы", оторвавшись от этой зловонной "заданности", можно войти в светлое будущее..

\iusr{Liudmyla Taran}
як мудро й вичерпно. дякую

\iusr{Olga Кotsiuruba}

Завдяки ним все і є, мабуть було це відчуття «якщо не я то хто», що без їх
вкладу майбутнього точно не буде, тому треба пробувати зі всіх сил.

\iusr{Тамара Ящук}

\ifcmt
  ig https://scontent-mxp1-1.xx.fbcdn.net/v/t39.1997-6/p240x240/14146891_1775284826045555_695465285_n.png?_nc_cat=1&ccb=1-5&_nc_sid=0572db&_nc_ohc=J-6axyc9DS0AX9xhE_8&_nc_ht=scontent-mxp1-1.xx&oh=f8291ae2ebfadb8ac3e8b05dc9da59c1&oe=61732FD6
  @width 0.1
\fi

\iusr{Christina Vasylkiv}

щира правда. Люблю вас не лише читати, а й слухати. Дуже би хотіла почитит
більше про наших класиків, зокрема Михайла Драгоманова та багатьох з
Розстріляного Відродження - це, як на мене, дуже потужні феномени підвалин
нашої культури.

\begin{itemize} % {
\iusr{Volodymyr Yermolenko}
\textbf{Christina Vasylkiv} це буде на Kult: Podcast , але трохи пізніше: в кінці 2021 чи на початку 2022)

\iusr{Iuliia Zharikova}
\textbf{Volodymyr Yermolenko} чудово! Це цікава тема

\iusr{Christina Vasylkiv}
\textbf{Volodymyr Yermolenko} дякую:))))дуууже чекаю @igg{fbicon.hibiscus} 
\end{itemize} % }

\iusr{Ірина Соловей}

Вони моя опора. Завжди, коли збирається хмарою відчуття, що складно і важко,
згадую їх по одному, і віра відновлюється, і миттєво з‘являється промінчик і
вітер життя розганяє ту хмару. Останнім часом правда спадають на думку наші
військові, що копають окопи, щоб видити у війні. Отак вони мене бережуть, а я
можу старатись, щоб було що берегти)

\iusr{Mariana Sadovska}

Яку важливі для мене слова саме в ці дні. Дуже б хотілося могти з вами коротко
порадитися. Напишу у приват, добре?

\begin{itemize} % {
\iusr{Volodymyr Yermolenko}
\textbf{Mariana Sadovska} звісно, завжди радий)
\end{itemize} % }

\end{itemize} % }
