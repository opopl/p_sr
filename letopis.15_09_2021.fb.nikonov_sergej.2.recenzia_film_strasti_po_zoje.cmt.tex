% vim: keymap=russian-jcukenwin
%%beginhead 
 
%%file 15_09_2021.fb.nikonov_sergej.2.recenzia_film_strasti_po_zoje.cmt
%%parent 15_09_2021.fb.nikonov_sergej.2.recenzia_film_strasti_po_zoje
 
%%url 
 
%%author_id 
%%date 
 
%%tags 
%%title 
 
%%endhead 
\subsubsection{Коментарі}
\label{sec:15_09_2021.fb.nikonov_sergej.2.recenzia_film_strasti_po_zoje.cmt}

\begin{itemize} % {
\iusr{Мария Лелека}

Возможно моё мнение нах никому не надо, но раз уж меня отметили в этом посте, я
хочу один раз в жизни написать всё, что считаю важным на эту тему. Простите,
если жестко, но я так думаю. Это тебе, Женя:

1. «Они меня не прощают…». Бля, и что с того? Нахуй тебе вообще их прощение? Ты
христианка, за тебя лично умер Христос. Все твои грехи уже искуплены его
кровью. Его жертва – идеальна. Ты не можешь к ней своими страданиями добавить
решительно ничего! За это кстати я не люблю православие, я протестантка. Бесит
эта православная традиция к покаянию перед людьми и самобичеванию. Женя, каятся
нужно перед Богом, и только перед ним. Люди переменчивы, как флюгер, их мнение
вообще не имеет особого значения. У них сегодня одно, завтра другое. Раскайся
тихо, молча один раз – перед Господом Богом. И всё. Покончи с этим навсегда. Не
прощают какие-то десять людей – в ЧС их. Не прощают сто – сотню в ЧС. Их
непрощение – это их личная проблема. Их, а не твоя. Да и по сути, не прощать
тебя особо и не за что. Ты никого не убивала, ты не политик, ты просто поэт –
жила как умела, как понимала. Бог тебя простил, ты уже спасена) Выдохни и
расслабься наконец. Потому, что читать это всё реально противно уже. Хватит.
Серьёзно. Ты всё искупила, Бог это видел. Переступи и твори во славу Его. Иначе
увязнешь в этом как в болоте. А нахрена?

2. «Про русский мир». Я русская, но рискну предположить, что даже с тобой мы
разное понимаем под этим словом. Плюс его сильно политизировали в последнее
время. Да и твои хейтеры, я уверена, понимают под «русским миром» не то же
самое, что ты. Для них «русмир» - это война и реки крови на востоке, для тебя,
если я тебя правильно понимаю «русский мир» - это духовное единство славян на
фоне традиции, русского языка и тд. Ну и славно – значит нужно развивать его, а
не убивать себя бессмысленными страданиями. Цветаеву погубила её гордыня.
Потому, что только полная дура может покончить с собой, когда у тебя есть дочь,
ради которой ты обязана жить! Суицид – это почти всегда гордыня, человеку,
который убивает себя уже класть на всё, он уже проиграл. Победить твоих
хейтеров можно, хочешь я скажу как: полным равнодушием к ним и презрением.
Люди, которые набрасываются на человека, которому плохо – это гиены, халдеи.
Единственный язык который они понимают – это язык силы. Потому, нужно стать
очень сильной. Запастись терпением. Тебе всего 40 лет. Да, уволили – это
ужасно. Но! В твоих силах взять себя в руки и освоить другую литературную
профессию, которая не помешает тебе писать стихи – главный редактор какого-то
медиа или издательства, пишущий журналист, возможно – лектор в другом
институте, где к тебе не так предвзяты, по твоей специальности или
специальности смежной с культурологией. Да, в 40 лет найти такую работу трудно.
Но! Молись Господу об этом ежедневно, стучись и тебе откроют. Не бойся быть
частью дебильной системы – ведь систему можно разрушить только изнутри. Плюс –
это социальный статус и свобода писать стихи, не думая о хлебе насущном, когда
перебиваешся случайными заработками. Если придется ради этого уехать из Украины
– не страшно. Пусть хейтеры радуются – это их дело, ты – это ты, где бы ты не
жила. И да – это трудно, но каммон – я всю жизнь так живу. И многие другие
писатели тоже. Работа (конформистская) отнимает уйму времени, но другого пути
нет. Зато свободное время ценится еще сильнее. Даже Леонардо да Винчи работал
придворным художником. Это при его то гениальности… А ты не Леонардо) Отож) Так
что постоянная работа нужна в приоритете, даже если ее поиск затянется на
долго.


\iusr{Мария Лелека}

3. И еще – страдание деструктивно. И бессмысленно. Лучше подумай, что
конструктивного ты можешь противопоставить всему этому безумию вокруг. Напиши,
например, роман о всём что с тобой случилось. Чем не новая «Белая Гвардия»?
Уверена, на эту книгу будет огромный спрос и ты войдешь в историю. Это явно
получше постов в фб. Плюс – писать прозу – это вызов самой себе. Но ты же не из
тех людей, кто отступает. Вот любимый тобой Прилепин (которого я терпеть не
могу)) пишет же. И ты можешь! Или сделай еще что-то выдающееся – выздоровей
всем назло! Всем кто желает твоей смерти. Восстанови своё здоровье настолько,
чтоб они были в ахуе. Живи ВОПРЕКИ им. Живи, потому, что они хотят твоей
смерти. Возобнови литературную студию в библиотеке – не правда, что люди не
придут! Ты слишком зациклилась на своих хейтерах, и забыла что полно нормальных
людей, поэтов. Придут, поверь! И даже больше, чем ты ожидаешь. Я лично приду,
блин. И не потому, что я люблю русский мир – я его не люблю. Но я люблю тебя) И
я готова тебя поддержать) Чтоб конформистская работа ради заработка бабла не
казалась тупой и бессмысленной - поставь себе цель, зачем тебе нужны деньги –
например, усынови ребёнка. Вот где настоящее мужество, героизм! Воспитай его
так, чтоб он был настоящим гражданином Украины без фашизма и идиотизма. Сделай
что-то такое ГЛОБАЛЬНОЕ, ради чего тебе стоит жить! Чтоб ты могла сама себя
уважать и все, кто тебя обижает навсегда закрыли ебальники. Если русские
действительно НЕ СДАЮТСЯ – возьми себя в руки и докажи это! Потому, что читать
твои посты просто нет сил, ну бля – заводная пластинка по кругу. Даже если с
правильной мелодией. Не хотела тебя обидеть, но если обидела – это даже хорошо.
Может в тебе проснется злость что-то менять! А не просто констатировать, что
всё хуево. Жень, мне тебя не жалко. Потому, что жалось – это презрение к
человеку. Жалость убивает. Жалость – для слабых. А ты сильная – возьми себя в
руки и докажи всем, чего ты стоишь!!! И никогда не говори плохо о своих
друзьях, Женя. И что тебя кто-то кинул. Если ты глянешь дальше своей гордыни,
то увидишь что овердохуя людей поддерживали тебя несмотря ни на что, и
продолжают поддерживать. И я – одна из них. Потому, что я знаю, что ты хороший
человек. Восстань из пепла своего прошлого! Докажи всем, чего ты стоишь! Хотя
бы ради своего дедушки, если ты внучка победителя – победи всех. Я тебя люблю и
верю в тебя, ты это знаешь.

Только не тегайте меня больше никогда в таких постах, блядь – я это больше читать не могу и не буду.


\iusr{Микола Осадчий}
\textbf{Мария Лелека} "блядь – я это больше читать не могу и не буду" - ГЕНИАЛЬНО !!! Пешите исчо ...
\end{itemize} % }
