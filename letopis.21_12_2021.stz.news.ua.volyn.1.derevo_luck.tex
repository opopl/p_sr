% vim: keymap=russian-jcukenwin
%%beginhead 
 
%%file 21_12_2021.stz.news.ua.volyn.1.derevo_luck
%%parent 21_12_2021
 
%%url https://www.volyn.com.ua/news/201368-zhylo-bulo-derevo-v-lutsku-abo-shcho-prykrashaie-vulytsiu-bohdana-khmelnytskoho-video
 
%%author_id 
%%date 
 
%%tags 
%%title Жило-було дерево в Луцьку, або На що муніципали витратили 10 тисяч гривень (Відео) 
 
%%endhead 
\subsection{Жило-було дерево в Луцьку, або На що муніципали витратили 10 тисяч гривень (Відео)}
\label{sec:21_12_2021.stz.news.ua.volyn.1.derevo_luck}

\Purl{https://www.volyn.com.ua/news/201368-zhylo-bulo-derevo-v-lutsku-abo-shcho-prykrashaie-vulytsiu-bohdana-khmelnytskoho-video}

На вулиці Богдана Хмельницького в обласному центрі Волині напередодні свят
з’явилася цікава локація. Там до Нового року оригінально прикрасили липу,
вбравши її у великі кольорові кулі, які неможливо не помітити навіть здалека. А
як дерево сяє ввечері! Такої ілюмінації нема на жодній вулиці Луцька.

Це справа рук працівників департаменту муніципальної варти Луцької міськради.
На оздоблення новорічного дерева витратили майже 10 тисяч гривень з міської
скарбниці. 

– Ініціатива належить міській раді, аби місто було гарним на Новорічно-Різдвяні
свята, – запевнила директорка департаменту Юлія Сиротинська.

\href{https://www.youtube.com/watch?v=h9WP7jlec0M}{%
Жило-було дерево в Луцьку, Газета Волинь, youtube, 20.12.2021%
}

\ii{21_12_2021.stz.news.ua.volyn.1.derevo_luck.scr.1}

Юлія Сиротинська розповіла, що роки поспіль беруть активну участь в тому, аби
на зимові свята додати позитивного настрою городянам, тим паче, що в цей час
майже не буває снігу. А позаяк біля установи немає ялинки, то вирішили
прикрасити липу. Це роблять уже четвертий рік. Кулі різного кольору, фасону і
розміру закупили заздалегідь в місцевому гіпермаркеті. Цьогоріч дерево
доповнили яскравою підсвіткою, яка стала цікавою окрасою в цій частині міста.
Лучанам, які проходять повз, подобається. Вважають, що на такі речі можна й
витратитися, якщо це викличе хороші емоції чи додасть настрою.

\begin{zznagolos}
\enquote{Лучанам, які проходять повз, подобається. Вважають, що на такі речі можна й
витратитися, якщо це викличе хороші емоції чи додасть настрою.}
\end{zznagolos}

– У нас час зараз й так багато негативу: і коронавірус, і війна. Повинно бути
щось святкове для віри в диво, і в те, що все буде добре, – сказала пані Оксана
з Луцька.

Згодні з тим, що варто витрачати кошти на оформлення міста і лучанки Дарина та
Анна:

– Це створює новорічний настрій, приваблює туристів, які приїздять на свята
подивитися наше місто. Гарно виглянути ввечері в вікно, коли все світиться. 10
тисяч гривень не така велика сума. Є сфери, на які витрачають значно більше і
невиправдано. І почасту цього не видно, а тут бачимо, на що потратили, і
любуємось, – вважають дівчата.

\ii{21_12_2021.stz.news.ua.volyn.1.derevo_luck.pic.1}

Читайте також:

\href{https://www.volyn.com.ua/news/201019-u-lutskii-khudozhnii-shkoli-vidkryly-vystavku-do-dnia-sviatoho-mykolaia-foto-video}{%
На Волині створили понад 200 картин до Дня Святого Миколая (Фото, Відео), volyn.com.ua, 16.12.2021%
}
