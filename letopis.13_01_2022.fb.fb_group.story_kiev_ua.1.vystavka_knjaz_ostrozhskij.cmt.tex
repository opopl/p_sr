% vim: keymap=russian-jcukenwin
%%beginhead 
 
%%file 13_01_2022.fb.fb_group.story_kiev_ua.1.vystavka_knjaz_ostrozhskij.cmt
%%parent 13_01_2022.fb.fb_group.story_kiev_ua.1.vystavka_knjaz_ostrozhskij
 
%%url 
 
%%author_id 
%%date 
 
%%tags 
%%title 
 
%%endhead 
\zzSecCmt

\begin{itemize} % {
\iusr{Ольга Клюева}
Дуже цікаво, дякую!

\iusr{Микола Крижанівський}

У цьому наша сила. На жаль не піднята до належного державного пошанування та
вивчення українськими дітками для подальшого наслідування і гордощів.


\iusr{Kateryna Vinn}
А чи не пора нам \enquote{подякувати} ПЦРосії за \enquote{гостинність} і
звільнити наші святині від їх присутності та зберегти для нащадків України.

\iusr{Yelena Fesik}

Дякую, надзвичайно цікаво. Буквально вчора на ютьюбі слухала цікавезну
розповідь про князя Остожського. Це знак часу - повільно, занадто повільно, але
правда проростає.

\iusr{Людмила Чикалова}
А Зе треба розбудити Волю Муромця

\iusr{Татьяна Третьяченко}
\textbf{Людмила Чикалова}
Думаю, що його мощі вже вивезли на росію.

\iusr{Igor Poluektov}

\ifcmt
  ig https://scontent-frt3-1.xx.fbcdn.net/v/t39.30808-6/271788047_4722246821215562_7062546219664087708_n.jpg?_nc_cat=106&ccb=1-5&_nc_sid=dbeb18&_nc_ohc=mSboWm1jxosAX-uPHH5&_nc_oc=AQmenkBGcRo7JgiLC-KdXtIYgIXeP_nOQOVyyMn36Mt2ivWOwnZFNPR7L3zUnuT1OgQ&_nc_ht=scontent-frt3-1.xx&oh=00_AT-t23ZWV9PCWtZ6_5xu876kvEkXHLWY8QGb0TiYl2hZiQ&oe=61E498B7
  @width 0.3
\fi

\iusr{Евгения Кожемякина}
Нашій Україні зараз така людина дуже потрібна, алеж де вона?

\iusr{Игорь Семенов}

Он по моему в католичество перешёл иль предки, ну настоящий украинец, дак Украина
то что- папистский удел, Тарас Бульба бы ему все кости переломал бы

\begin{itemize} % {
\iusr{Вероника Руснак}
\textbf{Игорь Семенов} побольше бы таких Тарасо Бульб в наше время, чтобы не смотрели на католиков, а были верны православной вере

\iusr{Emiliia Vasianovych}
\textbf{Вероника Руснак}, так католики - це ж теж християни! Ви загрузили в середньовіччі!)

\iusr{Вероника Руснак}
\textbf{Emiliia Vasianovych} Рассказала бы ты это Тарасу Бульбе и козакам в то время. Они бы тебе объяснили быстро

\iusr{Андрей Лопата}
\textbf{Игорь Семенов} вера православная, вера католическая, иудаизм - суть заработок. Не может вера как то назьіваться. А так да. Библия изначально говорила, что найдутся 1000, что именем моим будут прикрьіваться...

\iusr{Emiliia Vasianovych}
\textbf{Игорь Семенов}, тільки московит може протиставляти християн, «західних» (католиків) і «східних» (православних)!

\iusr{Mykhaylo Losytskyy}
\textbf{Игорь Семенов} - точно? Насправді в католицтво перейшов його внук, Януш. А син заснував православну академію і видав ту саму Острозьку біблію (чи теж не чули?), на нього Іван Федоров працював.

\iusr{Игорь Семенов}

Ага, только православие и католичество это не просто вера, а ещё и ментальность
в поколениях, это то козаки хорошо понимали, наш - не наш, а то что у Ляхов
православные в холопах почитались, дак это вам и не \enquote{москали} расскажут

\end{itemize} % }

\iusr{Александр Даниленко}
Спасибо интересно!!!!
По поводу пекторали вы конечно загнули...)))
А ничего что пектораль сейчас находится в Голландии и Украина судится с рашей за неё.

\begin{itemize} % {
\iusr{Igor Poluektov}
\textbf{Александр Даниленко} Нідерланди вже повернули Україні і Золоту пектораль з Товстої могили і ще десятки інших прикрас того часу, які також представлені на виставці в Музеї коштовностей  @igg{fbicon.smile} 
На честь цієї події і виставку ж організували. Унікальна можливість побачити саме оригінали.

\iusr{Александр Даниленко}
Да ладно))
Украина выиграла только первый суд. Россия пода апелляцию.
И решающий суд нам ещё предстоит.
И в новостях не где не сообщалось что реликвии нам вернули.
Может показать ссылку в официальном источнике о том что это те самые оригиналы которые нам вернули????
А не копии которые и были в лавре.

\begin{itemize} % {
\iusr{Igor Poluektov}
\textbf{Александр Даниленко} так можу. Але мене заблокують, бо тут заборонені посилання на стронні ресурси  @igg{fbicon.smile} 
Про це повідомляли і провідні змі і сам Музей коштовностей Києво-Печерської лаври. На сайті Музею прямо вказано, що це оригінал Золотої пекторалі (НЕ копія).
Ось наприклад ВВС повідомляло:

\ifcmt
  ig https://scontent-frx5-1.xx.fbcdn.net/v/t39.30808-6/271780570_4722416494531928_4153805593825993442_n.jpg?_nc_cat=105&ccb=1-5&_nc_sid=dbeb18&_nc_ohc=x46C8KE1K30AX_cA9ey&_nc_ht=scontent-frx5-1.xx&oh=00_AT91mRVCiMvvfxLtEvkKxQeKXSrDH_ZcgLGsvIRvxskNYQ&oe=61E554F2
  @width 0.2
\fi

\iusr{Ніна Миронюк}
\textbf{Александр Даниленко}, дивно, що Ви не чули про цю подію. Здається усі ЗМІ про повернення скарбу повідомляли і про реакцію московії на це)

\iusr{Igor Poluektov}
\textbf{Nina Myronyuk} для цього і цей мій пост.
Багато людей дійсно не чують про важливі події з життя України, бо інформаційний ефір захаращений чимось іншим.
Люди ж які підписані на сторінки музеїв, наукових установ, відомих істориків, освітян тощо - вони в курсі всіх культурних подій життя столиці і України.

\iusr{Ніна Миронюк}
\textbf{Ігор Полуектов}, просвітництво в наш час надважливе. Інформаційна боротьба за майбутнє нації. Ви робите важливу справу.

\iusr{Александр Даниленко}
Спасибо огромное.
Извините что сомневался!!!

\iusr{Igor Poluektov}
\textbf{Nina Myronyuk} дякую!

\iusr{Igor Poluektov}
\textbf{Александр Даниленко} все нормально.
Не можна знати всього.
Для цього і ця група, щоб розповісти щось цікаве про життя міста, щоб і інші дізнались, долучились.
Виставка коштовностей буде відкрита ще два тижні.
Потім її знову сховають і невідомо коли так просто можна буде прийти та подивитись.
Так що поспішіть відвідати.

\end{itemize} % }

\end{itemize} % }

\iusr{Татьяна Третьяченко}
Дякую за цікавий і корисний для України допис!

\iusr{Людмила Терпелюк}
Дякую. Дуже цікаво, наше минуле, яке приховувалося.

\iusr{Ірина Дебкалюк}
ДЯКУЮ!

\iusr{Таня Головина}
Чудовий історичний виклад! Усвідомлено патріотично, дякую

\iusr{Тамара Ар}
Зараз надосліджують,,,,,

\iusr{Pavlo Kyjeslav Bliznichenko}
\textbf{Toma Ar} це раніше надосліджували так, що тепер те все розгрібати треба(

\iusr{Ніна Миронюк}
Дякую!

\iusr{Юрій Луцький}
руський князь переміг руських...

\begin{itemize} % {
\iusr{Надія Копачовець}
\textbf{George Lutskyi} руський князь переміг московитів

\iusr{Юрій Луцький}
Де ж було його Руське князівство?

\begin{itemize} % {
\iusr{Юрій Луцький}

Все прекрасно і дуже пафосно, але тільки підтверджує вислів, що ІНОДІ, історія
це всього лише політика повернута у минуле і замість справжнього розуміння
суспільства породжує лише додаткові конфлікти і труднощі в сьогоденні.


\iusr{Nata Romaniuk}
\textbf{Юрій Луцький} То не створюйте \enquote{додаткові конфлікти і труднощі в сьогоденні......}

\iusr{Юрій Луцький}

Автор посту це робить, ототожнюючи середньовічні феодальні конфлікти із
нинішнім українством. Якщо хоч трохи сумлінно повивчати усі тодішні постаті -
буде ясно, що для феодалів та їхньої челяді не існувало нічого святого, крім
власного герба та вотчини і вони міняли віри, армії та мови по десять разів на
день. Грунтувати патріотизм на цій хиткій основі можна лише насипавши поверх
наративу пафос , зверхність та зневагу до критики. Так можна виховати лише
обмежених, зашорених і заляканих послідовників. Усі інші вирвуться на свободу і
сміятимуться. Меньше давайте оцінок - більше обґрунтованих зв'язків і пафос не
знадобиться.

\end{itemize} % }

\end{itemize} % }

\iusr{Александр Даниленко}
Автор, может организуйте экскурсию в это воскресенье.
Посмотреть экспонати это одно а объективно и интересно послушать это совсем другое.

\begin{itemize} % {
\iusr{Igor Poluektov}
\textbf{Александр Даниленко} на території лаври забороняють проводити екскурсії без спеціальної акредитації

\iusr{Александр Даниленко}
\textbf{Igor Poluektov} а ихние экскурсоводы явно не так интересно расскажут и многого не доскажут...
Жаль
\end{itemize} % }

\end{itemize} % }
