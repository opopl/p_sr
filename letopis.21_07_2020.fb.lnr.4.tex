% vim: keymap=russian-jcukenwin
%%beginhead 
 
%%file 21_07_2020.fb.lnr.4
%%parent 21_07_2020
 
%%endhead 
\subsection{Один человек из числа пассажиров захваченного автобуса в Луцке ранен, внутри также находится беременная и ребенок}
\label{sec:21_07_2020.fb.lnr.4}
\url{https://www.facebook.com/groups/LNRGUMO/permalink/2865693703542185/}

\index[cities.rus]{Луцк!Террорист, 21.07.2020}
  
Один человек из числа пассажиров захваченного автобуса в Луцке ранен, внутри
также находится беременная и ребенок.  Об этом заявил сам террорист в
телефонном разговоре с журналистами львовского телеканала "НТА":

"Заложники в плохом состоянии, полный автобус.  Один человек ранен, одна
беременная, один ребенок", - утверждает он.

В ходе разговора террорист заявил, что полиция "не идет навстречу, не сделала
ни одного шага", добавив, что все действия правоохранителей считает
провокацией.

Захватчик сообщил, что все его требования, сделанные ранее, остаются в силе и
он ожидает их выполнения.

"Это было сделать нетрудно", - утверждает террорист, заверив, что отпустит
заложников после выполнения требований.

Злоумышленник также заявил, что считает, что привлечение большого количества
журналистов к освещению происшествия "поможет урегулировать конфликт".

"Приезжайте, если успеете, потому что можете и не успеть", - сказал он
журналистам телеканала.

Свои требования террорист опубликовал в Twitter, он хочет, чтобы главы
украинских судов, министерств, прокуратуры, парламента, церквей - всего 24
человека - записали и выложили в YouTube заявления о том, что они - террористы
в законе.

От президента Украины Владимира Зеленского он также потребовал опубликовать
видеообращение.

"Правда из уст 24 спасет жизни сотен", - написал террорист.

Ранее сообщалось, что представитель правоохранительных органов Украины после
того, как ему удалось передать воду пассажирам автобуса, попытался вновь
вступить в переговоры с террористом, но тот сделал одиночный выстрел.
Украинское издание "Страна" распространило видео с места происшествия, на
котором видно, как полицейский во второй раз подошел к автобусу с заложниками,
чтобы вступить в переговоры.

Ссылаясь на информацию местных журналистов, "Страна" сообщает, что "полицейский
переговорщик пытался договориться, чтобы из автобуса выпустили ребенка". Однако
после нескольких секунд переговоров злоумышленник выстрелом отогнал
переговорщика.

При этом замминистра внутренних дел Украины Антон Геращенко в эфире телеканала
"112 Украина" назвал переговоры лучшим вариантом для освобождения заложников.

Также, по его словам, злоумышленник, захвативший людей, ведет себя нервно,
несогласованно.  Геращенко подтвердил, что террорист выстрелом пытался сбить
дрон
