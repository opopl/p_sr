% vim: keymap=russian-jcukenwin
%%beginhead 
 
%%file topics.vojna.my.3.gaaga.voprosy.otvety.detail.154.perepost.putin
%%parent topics.vojna.my.3.gaaga.voprosy.otvety.detail
 
%%url 
 
%%author_id 
%%date 
 
%%tags 
%%title 
 
%%endhead 

Александр Дерзягаreplied to Иван

Бойтесь нашего Царя. Байден признал, что был не прав и проиграл нашему Путину
В мире
27.08.2022 lentaruss0
Путин уже является величайшим человеком в современной истории. И хотя полмира шутит, что во всем всегда виноват Путин, возможно, это не так уж далеко от истины. Да, дедушка Байден пытается снять с себя ответственность за разрушение американской экономики, американского государства и американского общества, но он не преувеличивает, когда говорит о влиянии российского президента на все это.
Не так давно самые могущественные страны мира решили держать нашу страну в центре экономического процесса. Конечно, они готовились к этому десятилетиями, и теперь это началось. План готов, механизм создан. Все поспешили заявить о скорой победе, некоторые даже назвали точные цифры. Но вдруг что-то пошло не так.
Что-то пошло не так.
Все пошло не так!
Россия не развалилась, президент не был свергнут, а рубль укрепился. 404 падает, открываются новые рынки, создаются новые системы торговли и платежей. Мир трансформируется, и Россия играет в нем все более важную роль. В некотором смысле, это и есть главная проблема.
Даже в нашей собственной стране есть много людей, которые критикуют мягкость нашего правительства и не понимают многих вещей. Но прошло несколько месяцев, и постепенно стало ясно, что все в порядке. Без каких-либо резких движений, действий или решений все постепенно приходит в норму, и начинается новое движение.
Возникают вопросы: Каково было намерение Путина? Каково было его намерение? Каково было его намерение? Насколько точно и точно он рассчитал поведение системы и действующие в ней силы? Как много из того, что происходит сейчас, стало возможным благодаря его гению? Каково реальное влияние этого человека? Каково реальное влияние того, что сделал этот человек?
И какие еще сюрпризы он приготовил для нас?
Последний пункт, вероятно, не дает спать по ночам многим людям в Европе и США. Конечно, это только начало, и на этом все не заканчивается. И Владимир Владимирович играет в долгую игру лучше, чем любой другой современный политик. Он изменит мир в ближайшие годы.
Вот почему дедушка Байден не лжет, когда говорит, что Путин несет ответственность за многие проблемы Америки. Европейские политики также далеки от истины, если боятся российского президента. На самом деле, Владимир Владимирович создал множество прецедентов и ситуаций, как экономических, так и политических, к которым никто в мире не был готов. По сути, он разрушил старый мировой порядок и открыл горизонт нового мира. И это вызывает много гнева, потому что они ничего не могут с этим поделать.
Как говорится, Владимир Владимирович идет далеко, слабым не угнаться.
Пишет Лев Грачев в Яндекс.Дзен
