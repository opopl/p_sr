% vim: keymap=russian-jcukenwin
%%beginhead 
 
%%file 25_05_2021.fb.litinskij_svjatoslav.1.mova_harkov_sud_jazyk
%%parent 25_05_2021
 
%%url https://www.facebook.com/svyatoslav.litynskyy/posts/10158987863680617
 
%%author Литинский, Святослав
%%author_id litinskij_svjatoslav
%%author_url 
 
%%tags 
%%title УРАА!!! Чотири роки судів, три судових провадження і це сталося!!! У місті Харкові скасовано регіональну мову!!!
 
%%endhead 
 
\subsection{УРАА!!! Чотири роки судів, три судових провадження і це сталося!!! У місті Харкові скасовано регіональну мову!!!}
\label{sec:25_05_2021.fb.litinskij_svjatoslav.1.mova_harkov_sud_jazyk}
\Purl{https://www.facebook.com/svyatoslav.litynskyy/posts/10158987863680617}
\ifcmt
 author_begin
   author_id litinskij_svjatoslav
 author_end
\fi

УРАА!!!
Чотири роки судів, три судових провадження і це сталося!!!
У місті Харкові скасовано регіональну мову!!!
Відповідне рішення виніс Харківський окружний адміністративний суд. 
Відтепер усі посилання на рішення міської ради про впровадження регіональної мови є нікчемними.
Чотири роки знадобилося Харківська обласна прокуратура  щоб виконати наші клопотання і численні рішення судів та врешті подати відповідний позов.

За юридичну допомогу дякую Українська Галицька Партія 

Повна історія:
1. Вперше ми звернулися до Харківської міської ради у серпні 2018 року з проханням скасувати своє рішення про запровадження регіональної мови у місті.
Харківська міська рада навіть не надала конкретної відповіді.
Тому ми звернулися до суду і виграли дві інстанції.
ХМР надала відповідь, що нічого вони скасовувати не будуть.
2. Із відмовою Харківської міськради ми звернулися у прокуратуру, аби та у судовому порядку оскаржила регіональну мову у Харкові.
Прокуратура проігнорувала наше звернення і не надала жодної відповіді.
Ми звернулися до суду ,виграли дві інстанції.
Отримали відповідь прокуратури, що вона нічого скасовувати не може, а отже і не буде.
3. Ми оскаржили відмову прокуратури щодо бездіяльності у оскарженні регіональної мови у Харкові.
Пройшли дві інстанції (третю інстанцію прокуратурі не відкрили) і виграли суди.
Прокуратура і на цьому не здалася та пів року не виконувала рішення суду.
Ми порушили щодо них кримінальне провадження за фактом невиконання судового рішення.
Тоді вже прокуратура таки виконала рішення і подала позов про скасування рег. мови у Харкові.

\ifcmt
  pic https://scontent-bos3-1.xx.fbcdn.net/v/t1.6435-0/p526x296/191743015_10158987863640617_7024146409922915297_n.jpg?_nc_cat=109&ccb=1-3&_nc_sid=8bfeb9&_nc_ohc=sqEGsCmJ1MAAX_kdMvv&_nc_ht=scontent-bos3-1.xx&tp=6&oh=dfcc99f25814351bb93e60bb357eb6a7&oe=60D096C8
\fi

