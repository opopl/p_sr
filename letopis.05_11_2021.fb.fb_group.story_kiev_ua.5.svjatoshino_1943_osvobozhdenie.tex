% vim: keymap=russian-jcukenwin
%%beginhead 
 
%%file 05_11_2021.fb.fb_group.story_kiev_ua.5.svjatoshino_1943_osvobozhdenie
%%parent 05_11_2021
 
%%url https://www.facebook.com/groups/story.kiev.ua/posts/1790802007783259/
 
%%author_id fb_group.story_kiev_ua,jeremenko_evgenia
%%date 
 
%%tags 1943,istoria,kiev,nacizm,okupacia,osvobozhdenie,rkka,sssr,ukraina,vov
%%title 5 ноября 1943 года ‒ ДЕНЬ ОСВОБОЖДЕНИЯ СВЯТОШИНО от немецко-фашистских захватчиков
 
%%endhead 
 
\subsection{5 ноября 1943 года ‒ ДЕНЬ ОСВОБОЖДЕНИЯ СВЯТОШИНО от немецко-фашистских захватчиков}
\label{sec:05_11_2021.fb.fb_group.story_kiev_ua.5.svjatoshino_1943_osvobozhdenie}
 
\Purl{https://www.facebook.com/groups/story.kiev.ua/posts/1790802007783259/}
\ifcmt
 author_begin
   author_id fb_group.story_kiev_ua,jeremenko_evgenia
 author_end
\fi

5 ноября 1943 года ‒ ДЕНЬ ОСВОБОЖДЕНИЯ СВЯТОШИНО от немецко-фашистских
захватчиков

Сейчас, глядя на живой, полноценно функционирующий район, сложно представить,
какая огненная мясорубка была на наших улицах и переулках в такой же осенний
день 78 лет назад. Сколько молодых жизней было прервано, столько отцов и дедов
не вернулось в свои семьи. Сколько воинов остались калеками ради того, чтобы мы
жили на своей земле хозяевами. С благодарностью ‒ совсем коротко ‒ вспомним,
листая странички документов и воспоминаний очевидцев, что происходило в эти дни
на западной окраине города, в Святошино. Подходили к концу 778 дней оккупации
города. День же 5 ноября можно назвать переломным в освобождении Киева. 

\ifcmt
  tab_begin cols=3

     pic https://scontent-frx5-2.xx.fbcdn.net/v/t1.6435-9/253234139_1551703158502369_9137577161781726361_n.jpg?_nc_cat=109&ccb=1-5&_nc_sid=b9115d&_nc_ohc=3Mf4_zKFWSIAX-B-JS1&_nc_ht=scontent-frx5-2.xx&oh=83f4bd2ff7939fbc76e8bad85afbbc4f&oe=61ACE187

     pic https://scontent-frx5-1.xx.fbcdn.net/v/t1.6435-9/253174752_1551703198502365_4282739346037477461_n.jpg?_nc_cat=111&ccb=1-5&_nc_sid=b9115d&_nc_ohc=kWExWvTh4F8AX-SMLR5&_nc_ht=scontent-frx5-1.xx&oh=9744bfc68e6d5468edcb656ea3598bc1&oe=61A9BA50

     pic https://scontent-frx5-2.xx.fbcdn.net/v/t1.6435-9/252645479_1551703411835677_7631236219689263665_n.jpg?_nc_cat=109&ccb=1-5&_nc_sid=b9115d&_nc_ohc=WDoAvbTPcdoAX8YYzj9&_nc_ht=scontent-frx5-2.xx&oh=860b9d1eeab767e48a53f685074e02f9&oe=61ABAF11

  tab_end
\fi

Поздним вечером 3 ноября ВГК издало Директиву о сроках освобождения Киева. Вот
она:

ДИРЕКТИВА СТАВКИ ВГК № 30236 КОМАНДУЮЩЕМУ ВОЙСКАМИ 1-го УКРАИНСКОГО ФРОНТА,
ПРЕДСТАВИТЕЛЮ СТАВКИ (прим. - т.е. Г. К. Жукову) О СРОКАХ КИЕВСКОЙ
НАСТУПАТЕЛЬНОЙ ОПЕРАЦИИ

3 ноября 1943 г. 20 ч 30 мин

Операцию, начатую на правом крыле фронта, не затягивать, т. к. каждый лишний
день дает только преимущество противнику, позволяя ему сосредоточивать сюда
свои силы, пользуясь хорошими дорогами. Тогда как у нас разрушенные противником
дороги за-трудняют и ограничивают маневр. Исходя из этого Ставка Верховного
Главнокомандования приказывает:

1. Не позже 5.11 перерезать железную дорогу Киев - Коростень восточнее или
западнее р. Ирпень в зависимости от обстановки и не позже 5—6.11 овладеть
городом Киев.

2. Не позже 9—10.11 перерезать железную дорогу Киев - Фастов.

3. Решая эту задачу пойти на жертвы, имея в виду, что эти жертвы будут во много
раз меньше тех, которые придется затратить, если эта операция затянется.

4. Учесть, что киевский плацдарм является важнейшим и наивыгоднейшим плацдармом
на правом берегу р. Днепр, имеющим исторически важное значение для изгнания
немцев из Правобережной Украины.

5. О получении и отданных распоряжениях донести.

Ставка Верховного Главнокомандования И. СТАЛИН А. АНТОНОВ

(Русский архив: Великая Отечественная. Ставка ВГК: Документы и материалы. 1943
год)

\ifcmt
  tab_begin cols=2

     pic https://scontent-frt3-1.xx.fbcdn.net/v/t1.6435-9/252444897_1551703491835669_4543094671556326176_n.jpg?_nc_cat=108&ccb=1-5&_nc_sid=b9115d&_nc_ohc=cGjSeBOZsJ8AX804oum&tn=lCYVFeHcTIAFcAzi&_nc_ht=scontent-frt3-1.xx&oh=a8298e30fed6b498649b58cf8a15b39c&oe=61AAE467

     pic https://scontent-frt3-1.xx.fbcdn.net/v/t1.6435-9/253403458_1551703318502353_8537000523279447994_n.jpg?_nc_cat=107&ccb=1-5&_nc_sid=b9115d&_nc_ohc=2vPRrilcWsQAX8rZksh&_nc_ht=scontent-frt3-1.xx&oh=79f294700281fde65db1351713d37525&oe=61AB618F

  tab_end
\fi

4 ноября 1943 года особенно упорные бои развернулись в районе Святошино. Здесь,
по шоссе Киев-Житомир, проходил последний оборонительный рубеж гитлеровцев,
прикрывавший подступы к Киеву. Захват этого района советскими войсками создавал
для врага угрозу глубокого охвата и имел исключительно важное значение не
только для скорейшего освобождения Киева, но и для разгрома всей киевской
группировки неприятеля. Перерезав основное шоссе, связывавшее немцев с тылами,
советские войска, с одной стороны, прекратили бы снабжение немецких войск
резервами и материальными средствами, а с другой — перекрыли противнику пути
отступления на запад. Понимая важность удержания в своих руках этого последнего
рубежа обороны, враг оказывал яростное сопротивление, и бросил в бой все силы. 

\ifcmt
     pic https://scontent-frt3-1.xx.fbcdn.net/v/t1.6435-9/252455093_1551703251835693_3224324046101148835_n.jpg?_nc_cat=108&ccb=1-5&_nc_sid=b9115d&_nc_ohc=TQZJnwWeLfgAX-Ad1on&_nc_ht=scontent-frt3-1.xx&oh=0a57dbdb7a4e03ff6b52620afad94d99&oe=61A9D32C
     @width 0.8
\fi

Ночью 4 ноября 1943 года взвод гвардии лейтенанта Н.К.Кругликова действовал в
разведке и шел впереди основных сил бригады. Шоссе Киев-Житомир было перекрыто
гитлеровскими танками и артиллерией. Решив перехватить дорогу, Кругликов
прорвался через оборону противника, а чтобы парализовать движение на главной
магистрали, оставил у дороги два танка. Дорога на Святошино была открыта, и
Кругликов смог внезапно для врага ворваться на окраину села, где уничтожил 5
противотанковых орудий, 2 автомашины с грузами, 3 миномета, до 150 фашистских
солдат и офицеров. Его стремительные действия способствовали быстрому и
беспрепятственному продвижению танковой бригады на Святошино.

Войдя в поселок, танкисты заняли оборону. Но гитлеровцы контратаковали и
блокировали нашу танковую бригаду. И без того тяжелое положение осложнялось еще
и тем, что доложить об этом в штаб корпуса командир бригады не смог. Решено
было послать донесение со связным. Лейтенант Кругликов вызвался лично его
доставить. Преодолевая сильный артиллерийский огонь, танк Кругликова с боем
ворвался в боевые порядки врага. На пути он уничтожил вражескую артиллерийскую
батарею, препятствующую выходу из окружения. В этом неравном бою танк
Кругликова был подбит, и его пришлось оставить. С автоматом в руках отважный
гвардеец прорывался через укрепленный узел противника, и в конечном итоге смог
доставить донесение в срок. Однако в штабе Кругликов не остался — его путь
лежал обратно, через вражескую оборону. Добравшись до своего танка, он
восстановил машину и прибыл в часть.

Всю ночь здесь шли ожесточенные бои. Согласованными ударами пехоты, танков,
артиллерии и авиации советские войска освободили Святошино. К утру наши войска
перерезали железную и шоссейную дороги Киев-Житомир, основную коммуникационную
линию киевской группировки противника.

5 ноября 1943 г. в журнале боевых действий Верховного Командования Вермахта
(ОКВ) отмечен прорыв позиций 4-й танковой армии северо-западнее Киева,
«противник ударил танковыми силами на юг через шоссе Киев-Житомир и ворвался в
городские кварталы Киева».

Первым из тех, кто выскочил на шоссе, перекрыв дорогу отступающим немецким
частям, был командир взвода 306-го танкового батальона 53-й гв. танковой
бригады, гвардии лейтенант Алексей Фалин, отвоевавший Зимнюю войну и все три
года Великой Отечественной.

На своем Т-34 он ворвался в с. Святошино (теперь оно в черте города), подбил
два штурмовых орудия, четыре бронетранспортера, и сам был подбит. Звание Героя
Советского Союза присвоено 3 июня 1944 г. посмертно, а похоронен лейтенант на
Святошинском кладбище.

Преследуя врага, части 7-го гвардейского танкового корпуса вышли к северной
окраине Святошино, перекрыв ведущую в Житомир дорогу. Танки зажгли фары,
включили сирены и вместе с пехотой пошли в ночную атаку. Одним из первых
достигла шоссе и вышла к населенному пункту Святошино тридцатьчетверка гвардии
старшего  сержанта Ива-на Васильевича Антонова из 425 танкового батальона 56-ой
гвардейской танковой бригады. От Святошино танк Антонова двинулся на юг, где
располагались станция Жуляны и село Глеваха. В ночных условиях Антонов вел свою
машину на вражеские пушки и технику. Ни полученное ранение, ни повреждение
машины не заставили механика Антонова покинуть поле боя. Под огнем противника
раненый Антонов сумел восстановить свою машину. Хотя подбитый танк и лишился
пушки, отважный водитель продолжил бой, давя врага гусеницами, пока вражеский
снаряд не поразил его Т-34.

Под траками танка механика-водителя Антонова в боях под Святошино были
уничтожены: 12 немецких автомашин и 13 повозок с военными грузами, 2
шестиствольных миномета с расчетом, 5 противотанковых пушек и 3 станковых
пулемета противника. За мужество и отвагу гвардии старшему сержанту
И.В.Антонову было присвоено звание Героя Советского Союза посмертно.

Немало других танкистов отличились в боях за Святошино. Так, командир танка
гвардии лейтенант Николай Андреевич Жихарев быстро и решительно ворвался в
Святошино, устремившись вглубь вражеской обороны, и перерезав важную магистраль
в тылу немцев. Ему удалось уничтожить на этом участке 20 автомашин, 40 повозок
противника с грузами и солдатами, 50 солдат, а также захватить исправный
немецкий танк, который противник в панике оставил. Не обошлось без ранения,
которое, впрочем, не остановило Жихарева — он оставался на посту и продолжал
вести огонь по неприятелю до тех пор, пока наши основные силы не вышли на
подступы к Киеву.

В боях за Святошино сотни советских воинов проявили мужество, чувство долга,
верность присяге, любовь к Родине. Многие из них позже были удостоены звания
Героя Советского Союза за это сражение. Среди них были и наши с вами родные. 

Если у кого-нибудь в семейном архиве сохранились фотографии тех дней,
присоедините их, пожалуйста.

Мира всем!

\ii{05_11_2021.fb.fb_group.story_kiev_ua.5.svjatoshino_1943_osvobozhdenie.cmt}
