% vim: keymap=russian-jcukenwin
%%beginhead 
 
%%file 11_02_2022.tg.lesev_igor.1.po_minskim
%%parent 11_02_2022
 
%%url https://t.me/Lesev_Igor/304
 
%%author_id lesev_igor
%%date 
 
%%tags donbass,minsk_dogovor,ukraina,vojna
%%title По Минским
 
%%endhead 
 
\subsection{По Минским}
\label{sec:11_02_2022.tg.lesev_igor.1.po_minskim}
 
\Purl{https://t.me/Lesev_Igor/304}
\ifcmt
 author_begin
   author_id lesev_igor
 author_end
\fi

По Минским.

Казалось бы, еще одна бессмысленная сходка советников Путина и Зеленского с
примкнувшей мебелью от Германии и Франции. Последние двое очень пекутся «о мире
в Европе», но как только доходит дело до конкретики, тут же понятно становится,
чем геополитические игроки отличаются от набитых деньгами сморчков.
Поразительно, но даже нищебродская Украина в плане протягивания своей линии,
смотрится куда солиднее.

Но по итогам. Формально Россия во главе с Козаком облажались. По крайней мере,
именно так все смотрится со стороны. Не достигнуто договоренностей вообще ни по
одной позиции. И казалось бы, ру-сторона должна теперь что-то предпринять
радикальное, чтобы не смотреться слабаком. Типа, надавить силой, или чего-то
там признать. Да и вообще, все ведь любят стремительную кавалерийскую атаку.
Чапаев смотрится изящнее, чем те, кто просто стояли на Угре.

Но все-таки итоги есть. Они просто не стремительные, а скорее, это как история
болезни. Тяжелой, которую вспять повернуть уже нельзя, просто каюк будет не
прямо сейчас, а завтра-послезавтра. И каюк этот взаправдашний, в отличии от
«широкомасштабного российского вторжения на Украину».

Итак, первое. Украина защитила право на бандеровскую модель своего
существования. И если совсем резануть битвой Оккама, то эта модель пиздеца
самой Украины. Бандеровщина – это всегда о ненависти внутри своего же
сообщества. Если три бандеровца в Тернополе будут петь гимн Украины, и вокруг
них никого больше не будет, они все равно посрутся, потому что кто-то из них
будет обвинен в неискреннем пении. В такой стране никогда ничего не будет
построено и в скором времени борьба за внутренний рынок Украины сменится только
борьбой за пространство, которое занимает Украина. Нищая страна со стремительно
сокращающимся населением в среднесрочной перспективе будет интересна ТНК не
больше, чем Боливия или Мозамбик. Бандеровской Украине суждено о себе говорить
на международной арене только через географию. Ничем другим, включая
патентованную русофобию, на международных рынках она интересна не будет.

Второе. Украина практически юридически отказалась от той части Донбасса, где
Донецк и Луганск. Все эти имитационные переговоры от украинской делегации, в
которых избегается любая конкретика, любые обязательства со стороны Украины,
избегается даже публичное цитирование самих Минских соглашений – вот это всё
видят в континентальной Европе. Это происходит на глазах у немцев и французов.
Англосаксов оставим в стороне. Там особо тупые. Я вот не могу представить,
чтобы какой-то министр иностранных дел РФ, Союза, царской России, начиная
где-то с середины XVIII века не мог не знать, что Ливерпуль и Бирмингем – это в
Англии, а не в Нидерландах. А у британцев в XXI веке глава МИД – эксперимент по
скрещиванию Псаки с Безуглой.

Но вот это петляние Украины в континентальной Европе не спрячешь. В силу
отсутствия геополитической субъектности, немчура и франки мычат на переговорах,
вынужденно в Москве и Киеве говорить абсолютно противоположные вещи, и каждый
раз соглашаясь с двумя антагонистами. Но даже когда нет возможности внятно
говорить, а приходится только мычать, мнение все равно формируется. Вот как во
время просмотра какого-то судебного фильма, где зритель для себя определяет
подонка и жертву. В хорошем кино изначальная жертва обязательно станет
подонком. А у нас тут очень хорошее кино, и «жертвенность Украины» сегодня
принимают только англосаксы, поляки и прибалты по причине врожденной русофобии.

Третье. Любой длительный тупик превращается в новую реальность. Отказавшись в
613-й раз возвратить ЛДНР на условиях Минских, Украина не получает взамен
ничего. Вообще, никакой альтернативы. Никто Киеву не даст ни плана «А», ни
плана «Б», ни плана «Я». Украина живет в своей бандеровской реальности, у ЛДНР
выстраивается свой местный русский мир. Они не пересекаются, потому что обе
модели даже на дискуссионном уровне несовместимы. Тупик-разрыв на Донбассе,
который генерирует две параллельные реальности. И Украина не имеет никаких
«особых» механизмов его разорвать. «Бесконечная война» на Донбассе, которую
выбрала бандеровская Украина – это не только о космическом воровстве. Это еще и
бесконечное разорение Украины, и ее бесконечная деградация.

P.S. Минимальная пенсия в ДНР с января текущего года 7844 рубля. Это 2940
гривен. У моей мамы пенсия в Украине 2100. И это даже не минимальная. Будем
живы, посмотрим еще через 8 лет на Большую Украину и ЛДНР. Думаю, будет очень
богатая почва для сравнения.
