% vim: keymap=russian-jcukenwin
%%beginhead 
 
%%file 11_05_2019.stz.news.ua.mrpl_city.1.junost_kotoroj_ne_bylo_vospominania_vojna
%%parent 11_05_2019
 
%%url https://mrpl.city/blogs/view/yunost-kotoroj-ne-bylo-vospominaniya-mariupoltsev-o-voennom-vremeni
 
%%author_id burov_sergij.mariupol,news.ua.mrpl_city
%%date 
 
%%tags 
%%title Юность, которой не было: воспоминания мариупольцев о военном времени
 
%%endhead 
 
\subsection{Юность, которой не было: воспоминания мариупольцев о военном времени}
\label{sec:11_05_2019.stz.news.ua.mrpl_city.1.junost_kotoroj_ne_bylo_vospominania_vojna}
 
\Purl{https://mrpl.city/blogs/view/yunost-kotoroj-ne-bylo-vospominaniya-mariupoltsev-o-voennom-vremeni}
\ifcmt
 author_begin
   author_id burov_sergij.mariupol,news.ua.mrpl_city
 author_end
\fi

\ii{11_05_2019.stz.news.ua.mrpl_city.1.junost_kotoroj_ne_bylo_vospominania_vojna.pic.1}

Эти люди долгие годы предпочитали молчать о своей юности. Юности, которая
прошла на чужбине, в фашистской Германии, куда они попали в четырнадцати -
пятнадцатилетнем возрасте. Да и была ли у них юность вообще? Вот фрагменты
воспоминаний этих людей о том ужасном времени.

\textbf{Валентина Петровна Войцеховская (Тимченко), угнана в Германию, когда ей было
четырнадцать с половиной лет:}

- В 42 году меня и моих сверстников угоняли в Германию. Весь перрон был
запружен людьми, вокруг гул, рев, крики, слезы матерей. Во время трехлетнего
пребывания в Германии я видела перед глазами маму, которую поддерживали брат и
отец, чтобы она не потеряла сознание. Еще помню, что кто-то играл на баяне и
мальчик семи-восьми лет пел какую-то печальную песню. В Германии я работала на
заводе, приходилось делать самую тяжелую работу. Часто после основного рабочего
дня нас посылали на выгрузку железнодорожных вагонов. Освободили нас
американцы, и случилось это в день моего совершеннолетия. Большего счастья, чем
в тот день, я не испытывала никогда в жизни.

\textbf{Галина Андреевна Шепель, угнана в Германию четырнадцатилетней девочкой:}

- На фабрике, куда я попала, работали шесть \enquote{остарбайтеров}, все малолетки,
была даже двенадцатилетняя девчушка. А жили мы в лагере, там находились человек
двести, в основном люди из Дебальцева, Чистякова и Мариуполя. Никто не принимал
во внимание, что мы малые дети. Мы выполняли ту же работу, что и взрослые
немцы-рабочие.

\textbf{Николай Васильевич Шепель:}

- В Германии я оказался семнадцатилетним юношей. Добывал уголь в шахте на
глубине 780 метров на пласте толщиной 1,2 метра. За двенадцатичасовой день
каждый из нас обязан был добыть не менее 12 тонн. Даже сейчас вспоминать об
этом страшно.

\textbf{Вера Константиновна Евтушевская (Гудзь), отправлена на принудительные работы в
Германию в шестнадцатилетнем возрасте:}

- Из Мариуполя нас увозили в товарных вагонах. Людей туда натолкали так много,
что каждый мечтал лишь об одном: вдохнуть бы глоток свежего воздуха... В
Германии я попала в крестьянское хозяйство, к бауэру. Поднимали нас, батраков,
очень рано. Труд был непосильный: приходилось ухаживать за многочисленными
коровами, свиньями, работать в поле. За то, что я не справлялась со своей
работой, меня все время хозяйка била. И однажды я убежала. Но вскоре полиция
задержала и отправила меня в Бохум, в тюрьму. Оттуда я попала в штраф-лагерь на
шахту. Нас, девочек, заставляли возить вагонетки с породой. Были такие большие
куски породы, что мы, голодные, не в силах были их поднять. Жизнь во вражеском
плену была сплошным кошмаром. Но нужно сказать, что среди немцев находились
добрые души. Случалось, что немецкие рабочие делились с нами кусочком
бутерброда или другой более чем скромной едой.

\textbf{Валентина Федоровна Контарева, угнана в Германию в семнадцать лет:}

- Увозили нас в конце сентября 42-го года. Это была третья или четвертая
отправка из Мариуполя. В вагоне, в котором я оказалась, были все больные, все
после нескольких медкомиссий. Привезли нас под Берлин и направили всех в
домработницы. Из-за нашей слабости на заводы нас брать не захотели, отказались
от нас и бауэры. Однако примерно через год всех позабирали на фабрику. Работа
была тяжелая, в последнее время работали только ночью по двенадцать часов: с 8
вечера до 8 утра. Кормили, как и везде всех: брюква, отварная картошка,
какая-то коричневая подлива. Бомбежки страшные переносили. Бегали в подвалы,
ночью работали, днем не отдохнешь, опять же - бежали под фабрику, чтобы
спастись.

\textbf{Виталий Никитич Макарец, попал в Германию четырнадцатилетним мальчуганом:}

- В концлагере Дахау я чудом избежал смерти. Земляки-мариупольцы затолкали меня
в толпу людей, которых отправляли на работы в северные районы Германии. Там
рядом с нашим лагерем располагался лагерь военнопленных французских летчиков.
Кое с кем из них я завязал знакомство. К тому времени я уже мог изъясняться на
немецком языке. И когда французы создали подпольную организацию сопротивления,
мои знания пригодились. Стал я выполнять поручения связного.

\textbf{Николай Григорьевич Володин, угнан в Германию с матерью, когда ему едва минуло пять лет:}

- Несмотря на свой детский возраст, я на всю оставшуюся жизнь запомнил, что
такое брюква, что такое турнепс - основные продукты питания \enquote{остарбайтеров}. И
все же, вопреки бытовавшему мнению, что все немцы - фашисты, могу утверждать:
были среди них хорошие люди. Мне на всю жизнь запомнилось имя Отто... Это был
обер-мастер на фабрике, где мы находились. Он почти ежедневно приносил
маленький бутербродик и подкармливал меня. Может, благодаря нему я и выжил.

\textbf{Надежда Александровна Лишова (Подобина):}

- В феврале 42-го года гестапо арестовало моего отца. Нас у матери осталось
трое. Я - самая старшая. Тогда мне было всего пятнадцать лет. В апреле получила
повестку, что должна явиться на биржу. Когда я пришла туда, то меня,
естественно, отправили в Германию. Попала в город Фельберт на военный завод,
работала первый год в литейном цехе. Пришлось отбивать пневмомолотом приливы на
деталях. Тяжело заболела, домой возвратилась после лечения в госпитале. Еще
целый год дома пришлось лечиться в то тяжелое время, пока не стала на ноги.

\textbf{Как же встретила Родина своих детей, вернувшихся из чужеземного плена,
испытавших в самом начале жизни унижения, холод и голод, непосильный труд и
болезни?}

\textbf{Надежда Александровна Лишова (Подобина):}

- У нас тогда как лечили? Получала пайку хлеба 400 г и должна была из этой
пайки половину продать на Ильичевском рынке, чтобы купить бинт, потому что у
меня была рана до костного мозга и воспаление надкостницы (это после двух
операций), а бинтовать было нечем. Потому-то вместо сносного питания
приходилось продавать часть своего хлеба и покупать бинты.

\textbf{Николай Васильевич Шепель:}

- Вернувшись домой в Мариуполь, я уже на вокзале увидел плакат: \emph{\enquote{Все
репатрианты, прибывающие из Германии, обязаны в течение 24 часов пройти
проверку и зарегистрироваться на ул. 1 Мая, 63}}. Пошел туда. Сняли отпечатки
пальцев и выдали бумажку, некое подобие паспорта, с которой я должен был
регулярно отмечаться в НКВД.

\textbf{Вера Константиновна Евтушевская (Гудзь):}

- Когда мы приехали домой, в Мариуполь, нас считали изменниками Родины, врагами...

\vspace{0.5cm}
\begin{flushright}
\textbf{Записал С. Буров 30 ноября 1996 года.}
	
\end{flushright}
\vspace{0.5cm}

\clearpage
\textbf{Читайте также:}

\href{https://mrpl.city/news/view/sozhzhennoe-okkupantami-dvazhdy-o-rasstrele-zdaniya-gorupravleniya-militsii-9-maya-2014-goda-foto-plusvideo-360}{Сожженное оккупантами дважды: о расстреле здания Горуправления милиции 9 мая 2014 года (ФОТО+ВИДЕО-360), Ігор Романов, mrpl.city, 09.05.2019}
