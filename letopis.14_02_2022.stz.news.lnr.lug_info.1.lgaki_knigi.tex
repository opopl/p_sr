% vim: keymap=russian-jcukenwin
%%beginhead 
 
%%file 14_02_2022.stz.news.lnr.lug_info.1.lgaki_knigi
%%parent 14_02_2022
 
%%url https://lug-info.com/news/kollektiv-lgaki-imeni-matusovskogo-v-hode-akcii-knigodareniya-razdal-250-knig
 
%%author_id news.lnr.lug_info
%%date 
 
%%tags donbass,kniga,lgaki,lnr,lugansk
%%title Коллектив ЛГАКИ имени Матусовского в ходе акции книгодарения раздал 250 книг
 
%%endhead 
 
\subsection{Коллектив ЛГАКИ имени Матусовского в ходе акции книгодарения раздал 250 книг}
\label{sec:14_02_2022.stz.news.lnr.lug_info.1.lgaki_knigi}
 
\Purl{https://lug-info.com/news/kollektiv-lgaki-imeni-matusovskogo-v-hode-akcii-knigodareniya-razdal-250-knig}
\ifcmt
 author_begin
   author_id news.lnr.lug_info
 author_end
\fi

Акция книгодарения, в ходе которой 250 книг нашли новых владельцев, прошла в
Луганской государственной академии культуры и искусств (ЛГАКИ) имени Михаила
Матусовского. Об этом сообщила пресс-служба вуза.

\ii{14_02_2022.stz.news.lnr.lug_info.1.lgaki_knigi.pic.1}

Кафедра межкультурной коммуникации и иностранных языков ЛГАКИ проводит в вузе
акцию книгодарения третий год подряд. Все желающие смогли поучаствовать в
небольшой викторине и выбрать по две книги из 250 предложенных томов
художественной литературы на русском и иностранных языках, книг по искусству,
справочных изданий из личных библиотек педагогов.

\ii{14_02_2022.stz.news.lnr.lug_info.1.lgaki_knigi.pic.2}

\enquote{Цель ее (акции) простая – дарить книжки тем, кто любит читать}, –
рассказал заведующий кафедрой межкультурной коммуникации и иностранных языков
Валерий Унукович.

После того как все книги были розданы, студенты оставили записки с названиями
книг, которые они хотят прочесть, а также исписали лист ватмана словами
благодарности организаторам мероприятия.

Международный день книгодарения отмечается ежегодно 14 февраля, начиная с 2012
года.
