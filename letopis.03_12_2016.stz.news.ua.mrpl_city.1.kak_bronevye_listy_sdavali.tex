% vim: keymap=russian-jcukenwin
%%beginhead 
 
%%file 03_12_2016.stz.news.ua.mrpl_city.1.kak_bronevye_listy_sdavali
%%parent 03_12_2016
 
%%url https://mrpl.city/blogs/view/kak-bronevye-listy-sdavali
 
%%author_id burov_sergij.mariupol,news.ua.mrpl_city
%%date 
 
%%tags 
%%title Как броневые листы сдавали
 
%%endhead 
 
\subsection{Как броневые листы сдавали}
\label{sec:03_12_2016.stz.news.ua.mrpl_city.1.kak_bronevye_listy_sdavali}
 
\Purl{https://mrpl.city/blogs/view/kak-bronevye-listy-sdavali}
\ifcmt
 author_begin
   author_id burov_sergij.mariupol,news.ua.mrpl_city
 author_end
\fi

\ii{03_12_2016.stz.news.ua.mrpl_city.1.kak_bronevye_listy_sdavali.pic.1}

Как известно, Мариупольский металлургический комбинат имени Ильича ведет свою
родословную от слитых воедино национализированных после Октябрьской революции
заводов Никополь-Мариупольского общества (его краткости ради называли
\enquote{Никополь}) и \enquote{Русский Провиданс}. Об этом прославленном предприятии написано
много и в газетах, и в журналах, и в специально посвященных ему книгах. Но,
кроме этого, существует о нем своеобразный фольклор: были и легенды,
передающиеся из уст в уста не одним поколением ильичевцев. Фольклор этот
расцвечивает сухую схему исторических исследований, очеловечивает их бытовыми
подробностями. Одно из таких изустных преданий предлагается вниманию читателей...

В сентябре 1911 года на завод \enquote{Никополь} пришел заказ от Морского министерства
на изготовление броневых листов для постройки боевых кораблей. Заказ был
особенно желанным, так как несколько лет до этого многие цеха завода
простаивали из-за отсутствия работы. Хозяева \enquote{Никополя} времени попусту не
теряли: нашли нужные средства и быстренько приступили к строительству
прокатного стана, нагревательных и термических печей и другого оборудования,
необходимого для производства броневых листов.

Прошло не так уж много времени, цех был сооружен, отслужили молебен по случаю
его открытия, немного воды утекло в Кальчике, и новая для завода продукция была
освоена. Вот тогда-то в цехах завода периодически стали появляться офицеры,
уполномоченные соответствующим ведомством принимать продукцию, предназначенную
для флота. Были ли они специально назначены для этой процедуры или совмещали с
основной службой в Мариупольском воинском присутствии (так в царское время
называли знакомые нам военкоматы) сказать трудно.

Жили офицеры не в прилегающем к заводу поселке, а где-то в центре Мариуполя.
Поэтому, когда по мере накопления готовой продукции приходилось предъявлять ее
военному приемщику, за ним посылали лучший конный экипаж из имеющихся на заводе
\enquote{Никополь}. Приемку листов офицеры осуществляли всегда только по одному, а
предъявлял их обер-мастер цеха. Разные офицеры вели себя по-разному. Одни
строго придерживались буквы закона и не отступали ни на шаг от установленных
инструкциями требований к качеству поверхности листов, к химическому составу.
Других можно было \enquote{уговорить}, если листы, мягко выражаясь, не совсем отвечали
нужным кондициям. Как уговаривать, обер-мастер знал.

Происходило это примерно таким образом. Обер-мастер подводил офицера, как он
знал, достаточно \enquote{сговорчивого} к листам, предварительно помещенным на
специальные подложки. Строгий страж интересов державы в погонах, заложив руки
за спину, медленным шагом обходил листы, брезгливо осматривая их поверхность.
Если в поле его зрения попадалась хотя бы маленькая неровность, царапина или
щербинка на поверхности стального листа, он останавливался и твердым голосом
изрекал: \enquote{Не годится}. Обер-мастер услужливо бочком подбегал к листу и ставил
заранее приготовленным куском мела крестик. Такие отметки могли появиться на
одном или нескольких листах.

Закончив осмотр, офицер делал вид, что уходит. Тогда-то и пускал в ход свое
искусство поднаторевший в таких делах обер-мастер. \enquote{Ваше высокоблагородие,
окажите любезность, задержитесь. Тут ваше высокоблагородие велели пометочку
сделать, что тут на листе окалина вкатана, малюсенький закат получился. Так это
ведь не страшно. Поставят его внутрь корабля. Зато другая сторона, клянусь
Богом, просто великолепная. Извольте, ваше высокоблагородие, своим пальчиком
потрогать. Там лист гладенький, извиняюсь, как щечка молоденькой девицы}.

Высокий гость цеха не спеша снимал с правой руки перчатку. Шарил рукой снизу
листа. Наконец под его пальцем оказывалось нечто. Через секунду это нечто
оказывалось в кармане форменных брюк. Сдатчик и приемщик переходили к
следующему листу, помеченному крестиком. Все описанное выше повторялось в той
же последовательности.

Завершалась приемка тем, что появлялся рабочий c чистым полотенцем и с
кувшином, наполненным теплой водой. Пока офицер мыл руки, обер-мастер успевал
принести из конторки документы, чернильницу и ручку. Приемщик подписью с
многочисленными росчерками и завитушками удостоверял годность листов для флота
Его Императорского Величества... Тут появлялся небольшой поднос, прикрытый
белоснежной накрахмаленной до хруста салфеткой. Салфетка поднималась, под ней
оказывалась рюмка с прозрачной жидкостью, крохотный кусочек ржаного хлеба и
малосольный огурчик. Офицер привычным движением опрокидывал себе в рот
содержимое рюмки, нюхал хлеб и, вернув его на поднос не тронутым, закусывал
хрустящим огурчиком. Изображая всем своим видом, что долг исполнен, он твердым
шагом направлялся к выходу из цеха. Обер-мастер, семеня, сопровождал его до
самого экипажа...

Как гласит легенда, секрет \enquote{сговорчивости} приемщика заключался в том, что
обер-мастер заранее к внутренней стороне листа сомнительного качества
подклеивал золотую монету царской чеканки достоинством десять рублей. Ее-то \enquote{не
очень принципиальный} страж качества военной продукции нащупывал, а затем с
ловкостью фокусника отправлял себе в карман.

Так ли все это было? Свидетели и участники пересказанного здесь давно ушли в
мир иной. Переспросить не у кого.
