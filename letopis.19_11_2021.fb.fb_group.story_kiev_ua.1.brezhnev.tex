% vim: keymap=russian-jcukenwin
%%beginhead 
 
%%file 19_11_2021.fb.fb_group.story_kiev_ua.1.brezhnev
%%parent 19_11_2021
 
%%url https://www.facebook.com/groups/story.kiev.ua/posts/1801121290084664
 
%%author_id fb_group.story_kiev_ua,fedjko_vladimir.kiev
%%date 
 
%%tags 1981,brezhnev_leonid,kiev,sssr
%%title Декабрь 1981-го... На улице мороз и метель...
 
%%endhead 
 
\subsection{Декабрь 1981-го... На улице мороз и метель...}
\label{sec:19_11_2021.fb.fb_group.story_kiev_ua.1.brezhnev}
 
\Purl{https://www.facebook.com/groups/story.kiev.ua/posts/1801121290084664}
\ifcmt
 author_begin
   author_id fb_group.story_kiev_ua,fedjko_vladimir.kiev
 author_end
\fi

«Приказываю выпить со мной по случаю юбилейной даты – 75-летия нашего горячо
любимого генерального секретаря ЦК КПСС Леонида Ильича Брежнева!»

***

Декабрь 1981-го... На улице мороз и метель... А я в конце рабочего дня заглянул к
жене на работу. В то время она работала в одной из организаций Укрхудожпрома на
Тургеневской.  Мужское и женское интеллектуальное сословие искусствоведов
наслаждается кофе с сигаретой в большой умывальной комнате – вестибюле перед
туалетами. Дым сигарет с ментолом смешивается с ароматами трубочного табака
(Черная вишня), который курит искусствовед А., мой приятель. Я не курю, но
вышел с ним, чтобы не прерывать нашу беседу о постимпрессионизме и его влиянии
на европейское искусство.

\ii{19_11_2021.fb.fb_group.story_kiev_ua.1.brezhnev.pic.1}

Напротив входа в умывальную комнату дверь кабинета начальника Первого отдела,
которая всегда закрыта. Да и самого начальника никогда не видно. Он общается
только с руководством, которое приходит к нему в кабинет. Начальник Первого
отдела – старый пенсионер, офицер «действующего резерва КГБ». 

А. выколачивает трубку, и мы выходим в коридор. На пороге полуоткрытой двери в
кабинет стоит начальник Первого отдела... По его розовому лицу видно, что он уже
слегка «накатил»... Да и лёгким перегаром попахивает.

Начальник показывает пальцем на нас и говорит: «Быстро! Зашли в кабинет!» 

Мы заходим... Мне интересно поглядеть, что скрывается в этом таинственном
кабинете. Алексей же здесь работает, и отказывать могущественному начальнику не
желает.

Особист закрывает за нами дверь на ключ, внимательно оглядывает нас и
спрашивает: «Какой сегодня день?»

А. отвечает числом...

- Неправильно!

Я отвечаю днём недели...

- Неправильно!

Мы в недоумении!

Особист торжественным голосом говорит: «Сегодня день рождения нашего горячо
любимого генерального секретаря ЦК КПСС Леонида Ильича Брежнева – 75-летие!»
Переводит дух и строгим командирским голосом продолжает: «Приказываю выпить со
мной по случаю этой юбилейной даты!»

Ситуация становится интересной...

Особист достаёт из холодильника двухлитровую стеклянную банку со слегка мутной
жидкостью, банку с квашеной капустой, банку с квашеными огурцами и помидорами...
большой шмат сала и кольцо полукопчёной колбасы "Краковская"... Из шкафчика на
стол появляется круглая буханка «Украинского»... Не хилая "поляна"
накрывается...

Стоя мы выслушиваем заздравный тост за здоровье Ильича и опрокидываем по треть
стакана. Чёрт подери, это же слегка разбавленный спирт! Мы плюхаемся на стулья
вокруг стола и наслаждаемся закуской. Особист разливает по второй... по третьей...
Энергично жуём и слушаем болтовню особиста, изредка вставляя что-то своё. За
трапезой и болтовнёй проходит часа полтора...

После возлияния моих собутыльников постоянно тянет курить... Особист курит
какие-то импортные сигареты, а А. – трубку... Кабинет маленький и мгновенно
наполняется дымом. Открытая форточка с вентиляцией явно не справляется. Рабочий
день уже закончился, все сотрудники разошлись по домам... Особист открыл настежь
дверь в коридор, чтобы проветрить кабинет... 

Воспользовавшись моментом я, сославшись на необходимость сходить в туалет,
сбежал. Да, банально сбежал, ибо почувствовал, что дело может закончиться плохо
или очень плохо. 

Как оказалось, чуйка меня не подвела!

Вечером следующего дня жена рассказала мне забавную историю о финале нашего
застолья.

Утром пришла уборщица, открыла входную дверь, а в коридоре валяется А.... Чуть
далее, на пороге своего кабинета, валяется особист... А на ступеньках, ведущих на
второй этаж, валяется ночной охранник, которого особист втянув в застолье
взамен сбежавшего меня. Уборщица в полуобморочном состоянии побежала к уличному
таксофону и вызвала милицию. Приехали патрульный наряд милиции и скорая помощь,
которые и выяснили, что все трое мертвецки пьяны! Спирт был отменный!

Сотрудников отправили в вытрезвитель, а начальника Первого отдела отвезли
домой. КГБ было неприкасаемым.

***

\ii{19_11_2021.fb.fb_group.story_kiev_ua.1.brezhnev.cmt}
