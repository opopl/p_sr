% vim: keymap=russian-jcukenwin
%%beginhead 
 
%%file 19_05_2019.stz.news.ua.mrpl_city.1.gromad_activistka_larysa_zhyvko
%%parent 19_05_2019
 
%%url https://mrpl.city/blogs/view/mariupolska-gromadska-aktivistka-larisa-zhivko
 
%%author_id demidko_olga.mariupol,news.ua.mrpl_city
%%date 
 
%%tags 
%%title Маріупольська громадська активістка Лариса Живко
 
%%endhead 
 
\subsection{Маріупольська громадська активістка Лариса Живко}
\label{sec:19_05_2019.stz.news.ua.mrpl_city.1.gromad_activistka_larysa_zhyvko}
 
\Purl{https://mrpl.city/blogs/view/mariupolska-gromadska-aktivistka-larisa-zhivko}
\ifcmt
 author_begin
   author_id demidko_olga.mariupol,news.ua.mrpl_city
 author_end
\fi

\ii{19_05_2019.stz.news.ua.mrpl_city.1.gromad_activistka_larysa_zhyvko.pic.1}

В житті нашої наступної героїні було багато курйозних випадків та у будь-якій
ситуації її рятував гумор. Вона переконана, що якщо ставитися до життя з
гумором, то стане набагато легше і веселіше жити. Пропоную вам познайомитися з
маріупольською громадською активісткою \textbf{Ларисою Живко}! Вона має дуже
легку вдачу, випромінює суцільну любов і впевненість. Спілкування з нею
заряджає позитивом і гарним настроєм. Кількість реалізованих Ларисою проектів у
Маріуполі вражає. Вона - справжній лідер! Знає, куди йде і вміє повести за
собою інших, горить сама і \enquote{запалює} оточуючих.

Народилася Лариса в Маріуполі в дружній сім'ї. Її батько – \textbf{Жебровський Микола}
(на жаль, рано пішов з життя) – з молодості займав активну громадянську
позицію. Йому довелося працювати кіномеханіком і організатором культурних
заходів. У зрілі роки і до пенсії працював токарем-розточувальником
судноремонтного заводу. Став почесним працівником і ветераном праці. Все своє
життя він присвятив сім'ї та безкорисливій допомозі людям. До сих пір люди
згадують батька Лариси з великою повагою. Саме пам'ять про батька, його добре і
безкорисливе ставлення до людей, чуйне серце і щедрість надихають Ларису
найбільше. Соціальний проект \textbf{\emph{\enquote{Диво-ярмарок до дня св. Миколая}}} вона
створила на його честь. Мама – \textbf{Людмила} – також ветеран праці Маріупольського
морського порту і, головне, берегиня домашнього вогнища. У Лариси є молодша
сестра, \textbf{Юлія}, з якою вона підтримує дуже теплі відносини.

\ii{insert.read_also.burov.kazancev}

Маріупольчанка закохана у своє місто, яке теж її надихає. У дитинстві їй дуже
подобалося гуляти в Приморському парку та в Міському саду. Жовті кульбаби по
всьому місту завжди радували око. Також Лариса полюбляє проводити час на морі,
де відпочиває душею. Міський сад вважає найкращою локацією для своїх проектів.
Вже встигла реалізувати п'ять проектів саме на території парку.

\ii{19_05_2019.stz.news.ua.mrpl_city.1.gromad_activistka_larysa_zhyvko.pic.2}

Першу вищу освіту Лариса отримала в Бердянському університеті менеджменту і
бізнесу (спеціальність \enquote{Облік і аудит}) та близько 10 років працювала в галузі
економіки і бухгалтерії. Ця спеціальність була в її родині традиційною. Але
вроджений хист і життєвий досвід привели нашу героїню до практичної та медичної
психології. Так вона отримала другу вищу освіту в Міжрегіональній Академії
управління персоналом у Маріуполі, спеціалізація медичний психолог. Зараз
працює в Благодійному фонді \enquote{Карітас} психологом мобільної бригади, до якої
також входять лікарі і соціальні працівники. Вони виїжджають в сіру (буферну)
зону, в села і селища і працюють там з місцевими жителями. Проект розрахований
на допомогу людям, які перебувають у важких умовах проживання. Лариса з
колегами надає психологічну, медичну та соціальну допомогу.

Сьогодні активістка радіє, що їй вдалося виховати порядну, цілеспрямовану
доньку \textbf{Тетяну} та зберегти справжні почуття з коханим чоловіком \textbf{Віталієм}. У
багатьох проектах Лариси брали участь як волонтери і донька, і чоловік. Вони –
її тил і опора.

\ii{19_05_2019.stz.news.ua.mrpl_city.1.gromad_activistka_larysa_zhyvko.pic.3}

2016 рік став для маріупольчанки новою віхою в житті. Ситуація в країні не
залишила її байдужою. І як патріот свого міста і країни вона швидко включилася
в громадське життя міста. Перший проект, в якому вона взяла участь як куратор,
– фестиваль вуличної їжі \emph{\textbf{\enquote{Люди змінюють людей}}}. Для Лариси цей слоган означав
зміни на краще та конструктивні взаємодії з людьми. Команда Лариси навчалася
взаємодіяти один з одним і з учасниками фестивалю. Було цікаво, хвилююче,
важко, іноді опускалися руки від нерозуміння, але все ж таки проект був успішно
реалізований.

Після цього проекту Лариса з іншими ініціативними маріупольцями виграла грант
на новий проект. Цей проект вони назвали \emph{\textbf{\enquote{Стадіон 57 FOREWER}}}. Коли був
обстріл мікрорайону \enquote{Східний}, найбільше постраждала школа №57.
Організатори проекту вирішили допомогти дітям і побудували спортивний майданчик
для школярів.  Відкриття майданчика було неймовірно цікавим. Сам спортмайданчик
представляли відомі спортсмени міста, був рок-концерт місцевої музичної групи,
проводилися розіграші призів і подарунків серед активних спортивних сімей,
квести і, звичайно ж, щасливі очі дітей. Ініціативна група, що реалізовувала
проект, показала громадськості, що кожен з нас може бути не байдужий до чужої
біди. Особливо якщо в біді діти.

\ii{insert.read_also.demidko.cherepchenko}

\ii{19_05_2019.stz.news.ua.mrpl_city.1.gromad_activistka_larysa_zhyvko.pic.4}

Наступним кроком для Лариси став Благодійний \textbf{\emph{\enquote{Диво-ярмарок до св. Миколая}}}.
Вона є ідейним натхненником і організатором цього заходу, який вже став
щорічним. Протягом 3-х років ініціативна група проводить це свято для дітей. У
перший рік основною метою проекту був збір грошей для хворого хлопчика. У
рамках \enquote{Диво-ярмарку} пройшов звітний концерт творчих дитячих колективів, був
ярмарок-продаж новорічних сувенірів, фуд-корт і анімаційна програма. На другий
і третій рік вони відбирали дітей, що особливо потребують уваги і турботи.
Спонсори, які хотіли допомогти дітям, знаходили Ларису самі. Вони пропонували
свою допомогу в проведенні \enquote{Диво-ярмарку}. Формат проведення заходу щороку
змінюється. Все залежить від цільової аудиторії. Команда Лариси підлаштовується
під запити дітей. На жаль, в нашому житті завжди знайдуться діти, яким потрібна
турбота і увага. Але і бажаючих стати помічниками Святого Миколая і
підключитися до цього важливого соціального проекту – теж вистачає. Таким
чином, \enquote{Диво-ярмарок} став платформою для якісної взаємодії людей.

У 2017 році активістка стала куратором локальної команди Ма\hyp{}ріуполя проекту
\emph{\textbf{\enquote{Маріуполь туристичний}}} в рамках регіонального фестивалю \enquote{Обличчя міст}. Метою
цього фестивалю було залучення громадськості міста, бізнес-структур, державних
структур в наше місто як у туристичний центр Приазов'я. Це було склад\hyp{}не
завдання. Однак команда ініціативних людей не побоялася заявити, що Маріуполь,
все ж таки, туристичний, незважаючи на екологію, промисловість і байдужість
людей.

\ii{19_05_2019.stz.news.ua.mrpl_city.1.gromad_activistka_larysa_zhyvko.pic.5}

Водночас Лариса бере участь в соціальних проектах інших організацій як
запрошений фахівець (психолог). Це такі проекти як: \emph{\textbf{\enquote{Ти не сама, або приховані
проблеми жінок у соціумі} БФ \enquote{СХІД SOS} та проект \enquote{MrpLife: Знайомимось,
гуртуємось, розвиваємось} ЦБ Короленка.} }

Маріупольська громадська активістка підкреслює, що Маріуполь багатий на
активних та ініціативних людей. У неї дуже багато однодумців.

Як тільки народжується ідея реалізувати будь-який проект, завжди знайдуться
люди, яким ця ідея небайдужа. Зокрема, у 2017 році утворилася Громадська
організація \emph{\textbf{Студія соціальної взаємодії \enquote{Ми Поруч}}}, в якій вона є
співзасновницею. Цілі організації: створення сприятливих умов для формування
суспільства з новим форматом мислення, заснованого на високих духовних,
моральних, культурних принципах; об'єднання людей, зацікавлених у відродженні,
розвитку і популяризації духовних, історичних і культурних цінностей в
суспільстві; надання сприяння соціальному прогресу; розвиток позитивного
творчого потенціалу молоді та сприяння встановленню дружніх відносин між
народами, популяризація здорового способу життя. У 2017 році громадська
організація запустила проект \textbf{\emph{\enquote{Лікарні, в яких одужують швидше}}}. Команда
розробила заходи, пов'язані з турботою про хворих (як дітей, так і дорослих), в
першу чергу, стану їхнього духу, наприклад, пограти з дітьми або почитати
книжки,чи провести творчий майстер-клас, щоб підняти їм настрій і переключити
на позитивний лад, адже, як відомо, при здорових думках – здоровий дух, а при
здоровому дусі – здорове тіло.

\ii{19_05_2019.stz.news.ua.mrpl_city.1.gromad_activistka_larysa_zhyvko.pic.6}

Незважаючи на велику кількість реалізованих проектів, Лариса не збирається
зупинятися. В розробці у неї є проект, спрямований на формування у молоді
цінностей психічного здоров'я і особистісного розвитку. Зі свого досвіду
психолога і співпраці з молоддю в проектній діяльності вона зрозуміла,
наскільки важливе формування цих цінностей. Адже здорова молодь – здорове
суспільство і здорове майбутнє.

\textbf{Улюблені книги Лариси:} \enquote{Майстер і Маргарита} Михайла Булгакова, \enquote{Мудрість людства} серія книг Віктора Лавського.

\textbf{Улюблені фільми:} \enquote{Гладіатор} (2000 р.), \enquote{Управління гнівом} (2003 р.).

\textbf{Хобі:} \emph{\enquote{Я люблю малювати, цей процес свого роду медитація. Люблю збирати гриби в лісі. Можу ходити годинами по лісу і не відчувати часу. Завжди чекаю осені і грибного сезону. Люблю подорожувати і насолоджуватися красою світу}}.

\textbf{Порада маріупольцям:} \emph{\enquote{Бажаю небайдужим маріупольцям, об’єднання. Разом ми сила! Разом ми зможемо багато чого зробити для свого улюбленого міста!}}.

\ii{insert.read_also.demidko.sosnovskij}
