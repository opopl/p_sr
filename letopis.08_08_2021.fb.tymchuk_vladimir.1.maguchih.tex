% vim: keymap=russian-jcukenwin
%%beginhead 
 
%%file 08_08_2021.fb.tymchuk_vladimir.1.maguchih
%%parent 08_08_2021
 
%%url https://www.facebook.com/volodymyr.tymchuk.1/posts/2971124693140818
 
%%author Тымчук, Владимир
%%author_id tymchuk_vladimir
%%author_url 
 
%%tags maguchih_jaroslava,olimpiada.tokio,skandal,ukraina
%%title Магучіх могла повернутися в пошані
 
%%endhead 
 
\subsection{Магучіх могла повернутися в пошані}
\label{sec:08_08_2021.fb.tymchuk_vladimir.1.maguchih}
 
\Purl{https://www.facebook.com/volodymyr.tymchuk.1/posts/2971124693140818}
\ifcmt
 author_begin
   author_id tymchuk_vladimir
 author_end
\fi

Магучіх могла повернутися в пошані, але після такого "прольоту" її дорога вже
встелена крижаними висотами, здолати які буде набагато важче, ніж світові
рекорди. Своїм "вирваним контекстом" вона, правда, призвела до неминучого та
запізнілого просвітлення великого спорту, що пихато себе іменує "поза
політикою" та який три десятиліття зацементувався в проросійському (від
основної мови) мороку. Відчуваю, що до "братання" з окупантом майбутні
лавроносці вже не вдаватимуться. А у Збройних силах України, що переймають, хай
і дещо сповільнено, натівські підходи до кадрового менеджменту, високе звання
сержанта присвоюватимуть не за старти і фініші, а відповідно до програм
підготовки до наступного щабля (звання), виконання яких дозволяє
військовослужбовцю зрозуміти хто проти кого воює і за що.

Upd. Вже молодшого лейтенанта у 19 юних зрілих літ.
