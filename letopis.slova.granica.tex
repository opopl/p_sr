% vim: keymap=russian-jcukenwin
%%beginhead 
 
%%file slova.granica
%%parent slova
 
%%url 
 
%%author 
%%author_id 
%%author_url 
 
%%tags 
%%title 
 
%%endhead 
\chapter{Граница}
\label{sec:slova.granica}

%%%cit
%%%cit_head
%%%cit_pic
%%%cit_text
23 июня британский эсминец Defender пересек \emph{российскую границу} в 11:52 и на три
километра вошел в территориальные воды у мыса Фиолент. По данным российских
военных, эсминец предупредили о применении оружия в случае нарушения \emph{границы}.
Российский \emph{пограничный} сторожевой корабль дважды выполнил предупредительную
стрельбу, Су-24М провел предупредительное бомбометание по курсу движения
эсминца. В 12:23 Defender покинул российские воды.  Министерство обороны
Великобритании заявило, что эсминец совершил мирный проход в территориальных
водах Украины в соответствии с международным правом. Также в ведомстве
опровергли информацию о том, что российской стороной были произведены
предупредительные выстрелы
%%%cit_comment
%%%cit_title
\citTitle{Попавший на борт британского эсминца журналист описал инцидент в Черном море}, 
Варвара Кошечкина, lenta.ru, 23.06.2021
%%%endcit

%%%cit
%%%cit_head
%%%cit_pic
%%%cit_text
Но этого не понимают современные украинские политики. Ведь, разорвав с Россией
международные договоры, они сами поставили вопросы о \emph{границах}, которые
эти договоры закрепляли. Советский Союз, приращивая территории УССР, не ставил
задачей создание этнической республики. А если теперь руководство Украины
поставило задачу создать моноэтническое государство, то это автоматически
подрывает основы  обретения Украиной независимости, поскольку эта независимость
подразумевала многонациональное украинское государство, а не то, в котором
многомиллионное русское и русскоязычное население в таком случае обретает
статус колониального народа.  Ведь если УССР не было этапом становления
украинской государственности, а всего только колонией, как утверждают идеологи
современной Украины, то ей, избавляясь от колониального прошлого, по совету
Собчака нужно отдать то, что подарила метрополия, а оставить только то, с чем
ее «колонизировали». Так что выбор между «колониальным» прошлым и общей
историей несет за собой большую ответственность, чем представляет нынешняя
власть
%%%cit_comment
%%%cit_title
\citTitle{О будущем украинского и русского народов}, 
Виктор Медведчук, strana.ua, 15.07.2021
%%%endcit
