% vim: keymap=russian-jcukenwin
%%beginhead 
 
%%file 27_04_2018.fb.fb_group.750_hudozhnykiv.1.folk_ua_uzhe_nachinajet_podgotovku_k_sledujuschemu_festivalju.cmt
%%parent 27_04_2018.fb.fb_group.750_hudozhnykiv.1.folk_ua_uzhe_nachinajet_podgotovku_k_sledujuschemu_festivalju
 
%%url 
 
%%author_id 
%%date 
 
%%tags 
%%title 
 
%%endhead 

\qqSecCmt

\iusr{Людмила Остапец}

Особенно хотелось бы больше внимания к коллективному детскому
творчеству!)))😉😊

\iusr{Елизавета Крестьянкина}

Поэтому и хотим получить финансирование, чтобы была такая возможность)

\iusr{Galyna Yatsyshyna}

Идея: семейные герои. Например, зайчик большой и зайчик маленький. Большого
расписывает взрослый, маленького - ребенок.

\iusr{Василь Беженар}

Дитяча творчість має бути присутньоє на даному фестивалі. Діти - наше
майбутнє!!))))

\iusr{Елизавета Крестьянкина}

Для этого мы хотим чтобы впервые за весь период существования проекта получить
финансирование из общественного бюджета Киева. Чтобы реализовать проект, нужно
чтобы художники из Киева голосовали за проект на сайте. Я потом детально напишу
информацию.

\iusr{Людмила Остапец}

В новом проекте будут котики и птички. Предложение - котика расписывает
взрослый, птичек - дети!)

\iusr{Тетяна Горбенко}

Це чудово, дорога командо 💛💙 Folk Ukraine ! Юна обдарованність Мирослава
Горбенко запрошує приєднатись до нашого родинного мистецького проекту 4-х
поколінь, який уже відкрито 10 квітня 2018 року в Ржищеві! Добрих друзів
запрошуємо до 10 липня відвідати Ржищівський археолого-краєзнавчий музей, адже
все в Україні починається з археології, краєзнавства, мистецтва і культури. Це
ж здорово, коли правнуки третього тисячоліття відновлюють, цінують і шанують
творчість свого прадіда-художника, родом з другого тисячоліття!

\ifcmt
  igc https://scontent-frt3-2.xx.fbcdn.net/v/t1.6435-9/31326814_1504577266338044_8467347566718091264_n.jpg?_nc_cat=110&ccb=1-7&_nc_sid=dbeb18&_nc_ohc=UCXtCRNhfigAX_6anHa&_nc_ht=scontent-frt3-2.xx&oh=00_AfD6wp_wBtxNDC4Qol8WWujA829fxI-h7lndwR-kFABJPg&oe=646EC606
	@width 0.5
\fi

\begin{itemize} % {
\iusr{Тетяна Горбенко}

На виставці експонується Мирослави альбом \enquote{Київ 11-21століть} ... от там
закладено дитячу ідею проведення 9-го фестивалю писанок на Софії!

\iusr{Тетяна Горбенко}

Ось тут є ідея проведення 9-го фестивалю)) запрошуємо!!!

\ifcmt
  igc https://scontent-fra3-1.xx.fbcdn.net/v/t1.6435-9/31347602_1504592973003140_2715738238219190272_n.jpg?_nc_cat=101&ccb=1-7&_nc_sid=dbeb18&_nc_ohc=D0mktruihaEAX-EdUX9&_nc_ht=scontent-fra3-1.xx&oh=00_AfCtUk9jxhi9aMl1DKk48v-522MP4F1fVLNDiD9hS48GYg&oe=646ED7B7
	@width 0.6
\fi

\end{itemize} % }

\iusr{Elena Riedick}

А я могу с вами поучаствовать? можем сделать проект и в Австрии.

\iusr{Елизавета Крестьянкина}

Думаю да)

\begin{itemize} % {
\iusr{Elena Riedick}

Они очень любят кроликов, а дети в восторге.

\end{itemize} % }

\iusr{Anya Makarova}

Желаю Успехов! ❤️ Верю все получится и поддерживаю!

\begin{itemize} % {
\iusr{Елизавета Крестьянкина}

Если есть друзья в Киеве, то они могут голосовать за наш проект. Если мы
наберем голоса поддержки необходимые, то сможем получить финансирование !

\iusr{Anya Makarova}

Есть, будем ждать новостей и обязательно голосовать!

\end{itemize} % }

\iusr{Тетяна Горбенко}

Все залежить від місцезнаходження... це буде парк, чи інше...?

\begin{itemize} % {
\iusr{Елизавета Крестьянкина}

Это будет фестиваль 2019. Локация - Софийская площадь. Для этого мы хотим чтобы
впервые за весь период существования проекта получить финансирование из
общественного бюджета Киева.

\iusr{Тетяна Горбенко}
\textbf{Елизавета Крестьянкина} ясно)) орієнтуємось на Софію! Подумаємо🌝

\iusr{Елизавета Крестьянкина}
\textbf{Таня Шевенок-Горбенко} Нужно будет голосовать за наш проект на сайте!

\iusr{Maria Shemotiuk}
\textbf{Елизавета Крестьянкина} проголосуем ❤️❤️❤️

\iusr{Тетяна Горбенко}
\textbf{Елизавета Крестьянкина} проголосуємо))
\end{itemize} % }

\iusr{Adam Alrubaye Olga}

О. У меня как раз есть идея, но я незнала зачем она пришла ко мне в голову.
Куда писать идеи?) Сюда прям?

\begin{itemize} % {
\iusr{Елизавета Крестьянкина}

Сама идея уже есть. Просто раньше мы реализовывали проект за собственные
средства. А теперь мы хотим чтобы впервые за весь период существования проекта
получить финансирование из общественного бюджета Киева. Для этого нам нужно
набрать голоса поддержки. Тогда у нас будет возможность воплотить задуманное.
Фестиваль проходит уже 3 года подряд. Каждый год мы делаем что-то новое.

\iusr{Тетяна Горбенко}

Озвучте ідею, Лізо!

\iusr{Елизавета Крестьянкина}
\textbf{Таня Шевенок-Горбенко} Провести фестиваль пысанок в 2019 году.
\end{itemize} % }

\iusr{Julia Julia}

Молодці! Успіхів Вам! 💕 і вже чекаю наступного року щоб знову прийняти
участь!))

\begin{itemize} % {
\iusr{Елизавета Крестьянкина}

мы тоже ждем, но нам будет нужна помощь в этот раз))) Чтобы проект прошел,
нужно будет получить на сайте определенное количество голосов. )))

\iusr{Julia Julia}
\textbf{Елизавета Крестьянкина} чекаю на ссилку)

\iusr{Елизавета Крестьянкина}
\textbf{Julia Lyshanets} Голосование будет открыто в конце мая приблизительно.
\end{itemize} % }

\iusr{Anna Kshanovska-Orlova}

Дякую за все! Підтримаю вас по можливості в наступних проектах!

\iusr{Тетяна Горбенко}

Пропозиція ось така🌝Софіївську площу оформити в дусі Київської Русі Ярослава
Мудрого... там же розмістити і кроликів, і пташки, і писанки на брамах
міні-Києва 10-13 століть... буде поєднання стародавнього 11 століття і
сучасного 21-століття Києва!

\begin{itemize} % {
\iusr{Елизавета Крестьянкина}

Чтобы это все реализовать нужно получить голоса на сайте, и тогда уже полет
фантазии. А если не пройдем, то не получится(

\iusr{Тетяна Горбенко}
\textbf{Елизавета Крестьянкина} ясненько) вранці гроші, а ввечері стільці...🤩

\iusr{Nata Rasp}

Буду голосовать за вас!!! Удачи !!! Уже жду Фестиваль2019 !

\iusr{Елизавета Крестьянкина}
\textbf{Nata Rasp} Спасибо, вы всегда с нами)))

\iusr{Nata Rasp}

За 3 года стали родными ! 🙂

\iusr{Елизавета Крестьянкина}
\textbf{Ирина Попдякуник} Дякуємо за підтримку!

\end{itemize} % }

\iusr{Marianna Maslova}

Супер! Буду голосовать за вас! В посте не совсем ясно, что это конкурс проектов
фирм. Я тоже сначала подумала, что вы хотите найти идеи и авторов-исполнителей

\iusr{Елизавета Крестьянкина}

Это я подготавливаю почву)))

\iusr{Liliya Solomko}

Замечательно!) от успеха к успеху!)

\iusr{Василиса Киевская}

мужайтесь))))

\iusr{Таня Золотухина}

Очень здорово!!! Ждём!!!

\iusr{Елизавета Крестьянкина}

А мы будем ждать Ваши голоса. Только так у нас получиться реализовать проект
вместе в 2019 году!

\iusr{Таня Золотухина}

Присоединим и родителей деток

\begin{itemize} % {
\iusr{Елизавета Крестьянкина}
Это было бы здорово)
\end{itemize} % }

\iusr{Olga Mashevskaya}

Думаю, что нужно отделить названиями \enquote{Фестиваль писанок} и \enquote{наши арт объекты
}, очень накидываются пысанкари на наши яйца, зайцев, говорят, то уничтожаем
понимание \enquote{писанка}. Поэтому многие пысанкари и не принимают участие, а мы
должны знать свои сокральные и народные сокровища. Думаю, что многие со мной
согласны. Нужна дружба и разделённые понятия для художников и писанкарей, иначе
будет бунт, и будут только художественные произведения, а про народное забудем.

\begin{itemize} % {
\iusr{Елизавета Крестьянкина}

У нас и так оно разделяется) Специально есть разные направления. И все, что не
касается именно пысанкарства - это арт-перфоманс. Об этом мы говорим во всех
интервью, постах и пресс-релизах.

\iusr{Елизавета Крестьянкина}

И в этом году пысанки были только в традиционном стиле))) Но сейчас, что
развить все направления в 2019 году, нужна ваша поддержка. На первом этапе -
это голосование за наш проект, когда мы запустим его в ход)

\iusr{Olga Mashevskaya}
\textbf{Елизавета Крестьянкина} всегда готовы прийти на помощь!

\iusr{Olga Mashevskaya}
\textbf{Елизавета Крестьянкина} 

ну, у нас есть очень умные и образованные))), им всегда в радость погрызть
художника. Продолжайте это писать и оглашать!!! Спасибо за ответ!

\iusr{Елизавета Крестьянкина}
\textbf{Ольга Машевская} Вот я поэтому и начала заранее)))

\iusr{Елизавета Крестьянкина}
\textbf{Ольга Машевская} спасибо Вам)

\iusr{Olga Mashevskaya}
\textbf{Елизавета Крестьянкина} респект!!! Спасибо

\iusr{Ірина Надюкова}

У цьому році не всі писанки були традиційні, традиційних було небагато, а інші
дійсно арт-об'єкти. На мою думку має бути і традиційна і арт-об'єкти.
Традиційну можуть створити тільки писанкарі, в писанкарстві є свої символи і
традиції. Але це фестиваль і хай писанкарі розписують традиційну писанку, а
художники мають право виразити свою думку. Я за дружбу художників і писанкарів.
І дякую за організацію такого гарного фестивалю.

А голосувати можуть тільки київляни?

\iusr{Елизавета Крестьянкина}
\textbf{Ірина Надюкова} да, потому что это общественный бюджет Киева.

\end{itemize} % }

\iusr{Lena Olefirenko}

Желаю удачи!!! Мы всегда поддержим ваши проекты!!

\begin{itemize} % {
\iusr{Елизавета Крестьянкина}

Если есть в друзья в Киеве, пусть готовятся голосовать)

\iusr{Lena Olefirenko}
\textbf{Елизавета Крестьянкина} Есть!!! Хорошо!!! Обязательно!!!!
\end{itemize} % }

\iusr{Yana Khachikyan}

Мы-то проголосуем, конечно. Но если они не захотят финансировать, то фестиваля
не будет в следующем году?

\begin{itemize} % {
\iusr{Елизавета Крестьянкина}

будем решать вопросы по мере поступления) Если мы все дружно проголосуем, то все будет окей)
\end{itemize} % }

\iusr{Олеся Токарська}

Із задоволенням підтримаю! Де і коли голосувати, буду чекати інформації.

\begin{itemize} % {
\iusr{Елизавета Крестьянкина}

Спасибо) мы обязательно все напишем - как и что) пока готовимся) Но решили, что вы должны быть в курсе)
\end{itemize} % }

\iusr{Наталия Рабинович}

Ваши проекты нужны нам ! Хотя я не Киевлянка ! Участвовать в них -СУПЕР !
Обязательно будем голосовать !

\iusr{Yuliya Zherlitsyna}

В гб-3 участвовать будете?)))

\begin{itemize} % {
\iusr{Елизавета Крестьянкина}

Да)
\end{itemize} % }

\iusr{Елизавета Крестьянкина}

Друзья!\par
Ура! Ура! Ура!\par
Наш проект прошел модерацию на сайте общественного бюджета Киева!\par
Давайте вместе поможем ему реализоваться на новом уровне! Очень ждем Ваши голоса!\par
\url{https://gb.kyivcity.gov.ua/projects/138}\par
Голосовать могут только Киевляне. Сделать очень просто:\par
1. Можно идентифицировать себя через Приват24\par
2. Через KYIV ID\par

\ifcmt
  igc https://i2.paste.pics/a1b75fc7060feda3f91ef113fff38879.png
	@width 0.4
\fi

\iusr{Olga Mashevskaya}

Через приват можно войти и проголосовать?

\begin{itemize} % {
\iusr{Елизавета Крестьянкина}

да) я так заходила

\iusr{Елизавета Крестьянкина}

\ifcmt
  igc https://scontent-fra3-1.xx.fbcdn.net/v/t1.6435-9/34343690_1732380090150615_4737911860261552128_n.jpg?_nc_cat=101&ccb=1-7&_nc_sid=dbeb18&_nc_ohc=U96fmasUNxAAX8jMBbw&_nc_ht=scontent-fra3-1.xx&oh=00_AfCPU4nmUX9paYnfjryci2q3gNPQLUNXgh4oWVf-pTLOkA&oe=646ECA74
	@width 0.5
\fi

\end{itemize} % }

\iusr{Olga Mashevskaya}

\ifcmt
  igc https://scontent-frt3-2.xx.fbcdn.net/v/t1.6435-9/34459216_535347730195940_7995863034968932352_n.jpg?_nc_cat=108&ccb=1-7&_nc_sid=dbeb18&_nc_ohc=3SYJhGx1I04AX_T5VMa&_nc_ht=scontent-frt3-2.xx&oh=00_AfCm5CRJEo_tcUFuBmlSQFqYWh1RBpEqBDlXZP7n-TUU7A&oe=646ECA3D
	@width 0.5
\fi

\begin{itemize} % {
\iusr{Елизавета Крестьянкина}

\ifcmt
  igc https://i2.paste.pics/82058a7faf26f8127885077f2d4d1c04.png
	@width 0.4
\fi

\iusr{Елизавета Крестьянкина}

спасибо!!!!!!!!!!!

\end{itemize} % }

\iusr{Калина Калин}

Ми з друзями хотіли б додати свою маленьку пропозицію і замість зайців,
запропонувати до розпису маленьких ягнят. )

\begin{itemize} % {
\iusr{Елизавета Крестьянкина}

Это здорово) Пока что планируем делать птиц вместо зайчиков) Но если не наберем
300 голосов.... Но верю, что наберем!

\iusr{Елизавета Крестьянкина}

Наша команда будет признательна получить от Вас и ваших друзей несколько
голосов на сайте Общественного бюджета за проект
\url{https://gb.kyivcity.gov.ua/projects/138}

\ifcmt
  igc https://i2.paste.pics/45f41f67f2ef6b1771096f2f5567a0a3.png
	@width 0.6
\fi

\iusr{Калина Калин}

Обов'язково!

\iusr{Елизавета Крестьянкина}
\textbf{Катерина Калина} это будет здорово! Ждем!!

\iusr{Наталия Рабинович}

Голосовать должны опять только Киевляни?

\iusr{Елизавета Крестьянкина}
\textbf{Наталия Рабинович} пока мы в режиме ожидания. Вы комментируйте старый пост)

\end{itemize} % }

\iusr{Яна Власенко-Бернацька}

Урааааа!

\iusr{Lena Olefirenko}

Ура !!! Супер !!!!

\iusr{Пан Олександр Фурман}

Коли буде вислан альбом-каталог за 2017 рік? Розписували 2 писанки- тільки
дитячий диплом получили і все! Як далі приймати участь? Не чесно! Адресу
багаторазово висилали. Глухо. 😡

\begin{itemize} % {
\iusr{Елизавета Крестьянкина}

Всем отправляем. Я вижу ваш адрес и вам все придёт. Иногда в комментариях
теряются адреса. Я сегодня все проверю и отправлю. Извините за неудобства.
Участников 750, а Менеджер один. Хорошего творческого вам дня 🌸🌸🌸

\iusr{Пан Олександр Фурман}
Дякуєм!
\end{itemize} % }

\iusr{Natalia Onishenko}

Я тоже ещё не получила ни грамоты, ни каталог. Уже несколько раз писала адрес

\begin{itemize} % {
\iusr{Елизавета Крестьянкина}

Пожалуйста пишите информацию под соответствующим постом, иначе информация может
потеряться при большом количестве комментариев. Тем более Ваш каталог уже
отправлен. Номер накладной указан. В соответствующем посте. Очень сложно по
всей группе собрать всю информацию. Будьте пожалуйста внимательнее. Хорошего
дня!

\end{itemize} % }

\iusr{Наталия Рабинович}

Спасибо! диплом получила!

\iusr{Natalia Onishenko}

Большое спасибо за каталог

\iusr{Alena Yuhimenko}

Ми теж бажаємо приймати участь у наступному проекті такого чуда. Чекаємо з
нетерпінням !!!

\iusr{Karina Petrova}

Что нужно?

\iusr{Karina Petrova}

Поддерживаем

\iusr{Надія Колесник}

А кто из художников в Киеве готов проводить развлекательные мастер- классы для
детей и взрослых?

\begin{itemize} % {
\iusr{Dei Gratia}
\textbf{Надежда Колесник} есть много таких)) Можно Ваше предложение в личку))

\iusr{Яна Власенко-Бернацька}

Я регулярно таким занимаюсь. Мастеркрвссы для детей и взрослых.

\iusr{Elena Orestova}
\textbf{Надежда Колесник} напишите в дочку, есть опыт и желание)

\end{itemize} % }

\iusr{Taisiya Dmitrievna}

Я провожу МК по витражной росписи для детей.

\iusr{Maksym Oleshchenko}

Ребята! Мы хотим сплагиатить Вашу идею у нас в Днепре. Надеюсь вы не против. Но
если подсоветуете, будем признательны.

\begin{itemize} % {
\iusr{Елизавета Крестьянкина}

Добрый день, мы можем вам помочь с изготовлением макетов. Можете позвонить нам,
мы про консультации +38 068 527 16 58

\iusr{Maksym Oleshchenko}
\textbf{Елизавета Крестьянкина} ок. завтра наберем
\end{itemize} % }

\iusr{Anna Toka}

Здравствуйте хотела бы получить каталог участник pysanka art fest 2018. Спасибо

\begin{itemize} % {
\iusr{Елизавета Крестьянкина}
\textbf{Аня Тока} добрый день. Пока что каталога нет. Мы уже писали ранее в группе об этом. Как получится его сделать - обязательно сообщим.

\iusr{Nataliya Savchenyuk Ostapchenko}
\textbf{Елизавета Крестьянкина}, чекаю з нетерпінням також)))

\iusr{Елизавета Крестьянкина}
\textbf{Наталия Савченюк}

\ifcmt
  igc https://i2.paste.pics/e88ba90d6cf3bd6ebc76a60a3a14b6b5.png
	@width 0.3
\fi

\iusr{Violett Obryvko}
\textbf{Наталия Савченюк} в очірєдь))))

\iusr{Анна Столярова}

Как говорится... Про каталог помним, ждём)

\end{itemize} % }

\iusr{Оксана Вовк}

як прийняти участь?

\iusr{Olga Mashevskaya}

Мы с вами!

\iusr{Dei Gratia}

💖

\iusr{Yelahina Nataliia}

супер! надеемся на расширение финансирования! ждем условий участия, очень хочу!

\iusr{Natalia Tryfonova}

Спасибо, будем ждать новостей, мы с детьми очень хотим поучаствовать.

\iusr{Светлана Мешкова-Давиденко}

Интересно.. В этом году будет еще подобный фестиваль?

\iusr{Anael Sunny}

О дааа

\iusr{Сергей Редченко}

Теж хочу зі своїм художнім рельєфом проявити себе)

\iusr{Natasha Podtykanova}

С нетерпением жду новых проэктов 🙂

\iusr{Александра Янченко}

Тоже очень хочется принять участие в каком нибудь творческом проекте.

\iusr{Olena Kolotova}

Супер!!! Поздравляю Вас!!! Буду ждать новостей!!!

\iusr{Алла Виленская}

ВІТАЮ! Буду чекати новин!

\iusr{Tamara Fialka}

Дякую, будемо чекати новин.

\iusr{Reginaldo Pereira}

\ifcmt
  igc https://i2.paste.pics/16aee193abede70064d0505a525b2606.png
	@width 0.3
\fi
