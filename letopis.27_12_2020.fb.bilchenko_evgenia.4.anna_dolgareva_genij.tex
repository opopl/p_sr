% vim: keymap=russian-jcukenwin
%%beginhead 
 
%%file 27_12_2020.fb.bilchenko_evgenia.4.anna_dolgareva_genij
%%parent 27_12_2020
 
%%url https://www.facebook.com/yevzhik/posts/3509088649126220
 
%%author Бильченко, Евгения
%%author_id bilchenko_evgenia
%%author_url 
 
%%tags anna_dolgareva.poet.rus,bilchenko_evgenia,genij,poezia
%%title Анна Долгарева - Гений
 
%%endhead 
 
\subsection{Анна Долгарева - Гений}
\label{sec:27_12_2020.fb.bilchenko_evgenia.4.anna_dolgareva_genij}
\Purl{https://www.facebook.com/yevzhik/posts/3509088649126220}
\ifcmt
 author_begin
   author_id bilchenko_evgenia
 author_end
\fi

Совершенно потрясающий гений - Анна Долгарева. И совершенно потрясающий текст,
который буквально держит на плаву, когда совсем невмоготу. 

\ifcmt
  pic https://scontent-lga3-2.xx.fbcdn.net/v/t1.6435-9/133129953_3509083522460066_6383629768333966103_n.jpg?_nc_cat=110&ccb=1-3&_nc_sid=730e14&_nc_ohc=9mLtxeaBeFEAX8ET9OJ&_nc_ht=scontent-lga3-2.xx&oh=a1529f32cd339016bc132a0659f4edc9&oe=60CB44DC
	caption Анна Долгарева
\fi

***
Когда я пишу,
Как видела мертвых людей,
Разбитые окна, сожжённые крыши,
Как стреляли и жгли,
И не было правды среди земли,
И казалось, что нет и выше,
Как над зимней степью кричали вороны,
Как пальцы прилаживались к патронам,
То это я пишу о любви.
Когда я пишу,
Как подыхали мы в девяностых,
Как полыхали кресты на погостах,
И черные с головы до пят проходили женщины,
И черным становилось все на земле,
То это я пишу как умею,
Пишу вслепую во мгле,
Это я пишу о любви.
И о чем бы я ни писала,
И кричала бы ни о чем,
Она была посреди всего -
Ангел с горящим мечом,
Когда хоронили,
Когда расставались,
Когда кольцо надевала на мертвый палец,
Она была среди черных,
Изломанных, сгоревших людей,
Она была ветер степной,
И звон колокольный.
И я говорю о ней.
Все, что я есть, - отмечено ею.
По большому счёту,
Я вообще ни о чем не умею,
Кроме любви.
\verb|#АннаДолгарева|

\emph{Григорий Кудрявцев}
Голос Стивена Крейна

\emph{Tanya Ponomareva}
Очень круто! Аня, за книжкой зайдите уже!

\emph{Анна Долгарева}

Таня Пономарева о! Вы есть в фб) отлично, а то у меня телефонофобия, щас напишу

\emph{Евгения Бильченко}
Спасибо, девчонки.

\emph{Александр Апальков}
Такие страницы надо собирать в книги, настоящие.

\emph{Евгения Бильченко}
Александр Апальков У нее много книг)

\emph{Любовь Новикова}
\textbf{Евгения Бильченко} напиши о ней эссе и с библиографией её книг/текстов. Очень нужно.

\emph{Евгения Бильченко}
\textbf{Любовь Новикова} Обязательно сделаю эссе или нечто вроде развернутого
философского анализа творчества. Аня поможет с библиографией. Аня очень полно
воплощает наш Логос в его самом лучшем поэтическом и духовном смысле слова и
передает то, что зачастую не хватает традиционализму: универсальную силу любви,
заложенную во \enquote{всемирную отзывчивость Пушкина} (Ф. Достоевский). Практически -
уже начало эссе)

\emph{Любовь Новикова}
\textbf{Евгения Бильченко}, да, это начало эссе! То, что надо!))

\emph{Анна Долгарева}

Спасибо, милая!

\emph{Irina Rudneva}
«Всё невозможно, кроме...»

\emph{Александр Кужукин}
Что уже за сейчас. О любви как умею. Круто.
