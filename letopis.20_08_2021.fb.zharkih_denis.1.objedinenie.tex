% vim: keymap=russian-jcukenwin
%%beginhead 
 
%%file 20_08_2021.fb.zharkih_denis.1.objedinenie
%%parent 20_08_2021
 
%%url https://www.facebook.com/permalink.php?story_fbid=3059145560965507&id=100006102787780
 
%%author 
%%author_id zharkih_denis
%%author_url 
 
%%tags chelovechnost,objedinenie,ukraincy
%%title Об объединении украинцев.
 
%%endhead 
 
\subsection{Об объединении украинцев.}
\label{sec:20_08_2021.fb.zharkih_denis.1.objedinenie}
 
\Purl{https://www.facebook.com/permalink.php?story_fbid=3059145560965507&id=100006102787780}
\ifcmt
 author_begin
   author_id zharkih_denis
 author_end
\fi

Об объединении украинцев.

У меня всегда искреннее восхищение вызывают люди труда, какой бы национальности
и политических взглядов они не были и на каком бы языке не говорили. Украинские
шахтеры для меня герои, а не "лишние люди". Они каждый день обнимаются со
смертью, они знают цену братству, цену мужеству, цену свежему воздуху, наконец. 

Западноукраинские труженики вызывают у меня не меньшие положительные эмоции.
Вот недавно стоматолог из Западной Украины, Наталія Богданівна, прекрасно
сделала мне пломбы. И строго так, по-галицийски, говорит:

-Денисе, я не дозволяю вам копирсатися там зубочисткою. 

Делаю вид двоечника и вздыхаю:

- Ну раз Вы не разрешаете, то не буду...

Железная женщина (мало того, что зубной врач, так еще и с Галичины) смягчается.
Теперь она похожа на отличницу, которой этого самого двоечника жалко:

- Так я те... того... нікому цього не дозволяю ... От!

Ну и как тут не умилиться? Люди труда, люди, которые реализовали себя через
труд, творчество, созидание всегда будут основой счастливого процветающего
государства.

Почему мы сделали героями негодяев? Запад нам это велел? Да ничего подобного!
Мы сами унизили и отодвинули порядочных людей от власти, а потом и от жизни.
Сами, своими руками. И вернуть это все не так просто, как кажется. Нам нужно
опять поднимать культуру, нам нужно вернуть уважение людям труда и, значит,
гнать в шею приспособленцев и негодяев. Сможем? Это не просто моральный вопрос,
а вопрос выживания нас, как людей.

2018
