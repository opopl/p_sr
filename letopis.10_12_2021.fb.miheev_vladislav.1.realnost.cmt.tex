% vim: keymap=russian-jcukenwin
%%beginhead 
 
%%file 10_12_2021.fb.miheev_vladislav.1.realnost.cmt
%%parent 10_12_2021.fb.miheev_vladislav.1.realnost
 
%%url 
 
%%author_id 
%%date 
 
%%tags 
%%title 
 
%%endhead 
\zzSecCmt

\begin{itemize} % {
\iusr{Елена Скачко}
Или сделать героем и совестью нации профессионального подонка....

\iusr{Анатолий Басов}

Толпа-это собрание людей, живущих по преданию и рассуждающих по авторитету.
Историческая память толпы ограничена настоящим, плюс минус две недели.

(В. Белинский)

\begin{itemize} % {
\iusr{German Viktorov}
\textbf{Анатолий Басов} 

Это ведь определение толпы, а не общества. Далеко не всякое ведь общество
состоит в основном из толпы, как в нашем случае...

\iusr{Анатолий Басов}

Толпа, толпа. Не обольщайтесь.

\enquote{Думающее меньшинство}, составляет всего лишь 5\%. Оставшееся
\enquote{стадо} бессознательно следует за лидерами.

Основное качество толпы — нежелание и неумение самостоятельно думать и
приходить ко мнениям, соответствующим реальному положению дел и направленности
течения событий.
\end{itemize} % }

\iusr{German Viktorov}

Всё-таки, наверное разумнее анализировать любое мнение, включая и общественное,
чем по его носителям определять своё к нему отношение ) Нельзя же ведь смотреть
на весь мир через призму Украины, хотя в ней действительно, на практике :
\enquote{...наиболее адекватным ситуации может быть только \enquote{движение неприсоединения} -
к любому мнению, которое считается общественным}.


\iusr{Alexander Klymenko}
Да, Владислав, блестящее наблюдение. В современном мире именно так всё и делается.

\iusr{Вячеслав Бутко}

Почему-то вспомнил, как систематизировал свои мысли после прочтения основных
методологов науки (Поппер, Кун, Лакатос, Файерабенд и др):

1. Ученые не думают, а следуют неким предписанным сообществами правилам –
парадигмам (прокрустово ложе). Что не попадает в эти правила - отсекается. Это
Кун.

2. Поппер считал, что это неправильно. Что-то есть конечно (без предвзятости
нельзя) но истина ЕСТЬ как говаривал Тарски. АБСОЛЮТНАЯ истина и ее можно
достичь.

3. Лакатос считал - фиг его знает, за истину не могу сказать, но про метод Куна
могу сказать, что он не вполне прав, т.к. парадигму не меняют резко одну на
другую, а постоянно по чуть-чуть меняют. ИС (исследовательская программа
Лакатоса) - это твердое ядро и мягкая периферия. Когда эксперимент не попадает
в прокрустово ложе, ИС меняют мягкую периферию. и только когда после
НЕОДНОКРАТНЫХ экспериментов ИС ВСЯ не работает, только тогда меняют парадигму.
Но бывает это КРАЙНЕ редко.

4. Фейерабенд считал – ВСЕ, что пишут Кун, Лакатос, и Поппер в сущности фигня.
Потому, что он знал примеры, когда у ученых, которые делали открытия, ВООБЩЕ не
было никаких внятных методов. Поэтому – сгодится все.

А прочитал Вашу публикацию и думаю – а не к Файерабенду близка современная
эпоха  @igg{fbicon.smile} ?

\iusr{Владислав Михеев}
\textbf{Вячеслав Бутко} похоже  @igg{fbicon.smile} 

\iusr{Олег Климов}
Зато \enquote{перед забором все равны} (\enquote{Бригада С})

\iusr{Александр Тонконогов}
Подождите, подождите.

Мы следуя логике поста, так \enquote{неприсоединимся}, что договоримся, что один дядя
не huylo.

\begin{itemize} % {
\iusr{Татьяна Корелякова}
\textbf{Александр Тонконогов} 

так договорились же, что huylo  @igg{fbicon.smile} ....отказавшись от множества важных нюансов
фактологических, концептуальных и элементарных логических. Транслируете активно
и успешно из каждого утюга. Или \enquote{мы не понимаем - это другое}  @igg{fbicon.smile}  ?

\end{itemize} % }


\iusr{Геннадий Беляев}

\enquote{Каждый живет в собственном вымышленном мире, но большинство людей этого не
понимают. Никто не знает подлинного мира. Каждый называет Истиной свои личные
фантазии. Я отличаюсь тем, что знаю: я живу в мире грез. Мне это нравится, и я
не терплю, когда мне в этом мешают. Фантазии — единственная реальность.}

Федерико Феллини

\iusr{Татьяна Корелякова}
королевство кривых зеркал

\iusr{Sergey Kuchinsky}

Согласен полностью.

В этом контексте (искусственности так называемого общественного мнения в
Украине) только что прошла интересная инициатива. Лютый майдановец (3-го
эшелона) Юра Романенко предложил еще одну контру Минским соглашениям – провести
всенародный референдум о полном и окончательном отделении Донбасса. Это
предложение поступило от хитроумного майдановца всего через несколько недель
после того, как СНБО напрочь запретил населению даже произносить вслух слова
Донбасс, повстанцы (rebels) или ДНР-ЛНР. И всего через пару месяцев после
запрета лживых оппозиционных телеканалов и сайтов. И в условиях 7-летней
неподсудности постмайдановских затыкателей рта - террористов на улицах
Украины(от Бубенчика с Медведькой до Карася со Стерненкой).

Надо этот пост отослать таким наивным (или хитрым?) майдановцам. Вдруг лютые,
но наивные майдановцы не догадываются, что сегодняшнее так называемое
общественное мнение «всіх українців», это технологический результат
постмайдановского монопольного промывания мозгов, террора и цензуры?

\url{https://www.youtube.com/watch?v=vXN5-9iDzww}

\end{itemize} % }
