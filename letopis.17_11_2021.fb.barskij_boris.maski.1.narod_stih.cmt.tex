% vim: keymap=russian-jcukenwin
%%beginhead 
 
%%file 17_11_2021.fb.barskij_boris.maski.1.narod_stih.cmt
%%parent 17_11_2021.fb.barskij_boris.maski.1.narod_stih
 
%%url 
 
%%author_id 
%%date 
 
%%tags 
%%title 
 
%%endhead 
\subsubsection{Коментарі}

\begin{itemize} % {
\iusr{Boris Barsky}

\ifcmt
  ig https://scontent-frx5-1.xx.fbcdn.net/v/t39.30808-6/257480486_4650906828288758_2617990959334233216_n.jpg?_nc_cat=111&ccb=1-5&_nc_sid=dbeb18&_nc_ohc=a5sz8B6e8M8AX9PwBx9&_nc_ht=scontent-frx5-1.xx&oh=33b5efbe3a7a35c5aae92da8978ed2b8&oe=61B19F96
  @width 0.4
\fi

\iusr{Галина Барышевская}
\textbf{Boris Barsky} Да за что ?)за правду ?) @igg{fbicon.hands.raising}  @igg{fbicon.face.happy.two.hands} 

\iusr{Андрей Кондратов}

- Она была суровой, совсем не ласковой с виду. Не гламурной. Не приторно
любезной. У неё не было на это времени. Да и желания не было. И происхождение
подкачало. Простой она была.

Всю жизнь, сколько помню, она работала. Много. Очень много. Занималась всем
сразу. И прежде всего — нами, оболтусами.

Кормила, как могла. Не трюфелями, не лангустами, не пармезаном с моцареллой.
Кормила простым сыром, простой колбасой, завёрнутой в грубую серую обёрточную
бумагу.

Учила. Совала под нос книги, запихивала в кружки и спортивные секции, водила в
кино на детские утренники по 10 копеек за билет.

В кукольные театры, в ТЮЗ. Позже — в драму, оперу и балет.

Учила думать. Учила делать выводы. Сомневаться и добиваться. И мы старались,
как умели. И капризничали. И воротили носы.

И взрослели, умнели, мудрели, получали степени, ордена и звания. И ничего не
понимали. Хотя думали, что понимаем всё.

А она снова и снова отправляла нас в институты и университеты. В НИИ. На заводы
и на стадионы. В колхозы. В стройотряды. На далёкие стройки. В космос. Она всё
время куда-то нацеливала нас. Даже против нашей воли. Брала за руку и вела.
Тихонько подталкивала сзади. Потом махала рукой и уходила дальше, наблюдая за
нами со стороны. Издалека.

Она не была благодушно-показной и нарочито щедрой. Она была экономной.
Бережливой. Не баловала бесконечным разнообразием заморских благ. Предпочитала
своё, домашнее. Но иногда вдруг нечаянно дарила американские фильмы,
французские духи, немецкие ботинки или финские куртки. Нечасто и немного. Зато
все они были отменного качества — и кинокартины, и одежда, и косметика, и
детские игрушки. Как и положено быть подаркам, сделанным близкими людьми

Мы дрались за ними в очереди. Шумно и совсем по-детски восхищались. А она
вздыхала. Молча. Она не могла дать больше. И потому молчала. И снова работала.
Строила. Возводила. Запускала. Изобретала. И кормила. И учила.

Нам не хватало. И мы роптали. Избалованные дети, ещё не знающие горя. Мы
ворчали, мы жаловались. Мы были недовольны. Нам было мало.

И однажды мы возмутились. Громко. Всерьёз.

Она не удивилась. Она всё понимала. И потому ничего не сказала. Тяжело
вздохнула и ушла. Совсем. Навсегда.

Она не обиделась. За свою долгую трудную жизнь она ко всему привыкла.

Она не была идеальной и сама это понимала. Она была живой и потому ошибалась.
Иногда серьёзно. Но чаще трагически. В нашу пользу. Она просто слишком любила
нас. Хотя и старалась особенно это не показывать. Она слишком хорошо думала о
нас. Лучше, чем мы были на самом деле. И берегла нас, как могла. От всего
дурного. Мы думали, что мы выросли давно. Мы были уверены что вполне проживём
без её заботы и без её присмотра.

Мы были уверены в этом. Мы ошибались. А она — нет.

Она оказалась права и на этот раз. Как и почти всегда. Но, выслушав наши
упрёки, спорить не стала.

И ушла. Не выстрелив. Не пролив крови. Не хлопнув дверью. Не оскорбив нас на
прощанье. Ушла, оставив нас жить так, как мы хотели тогда.

Вот так и живём с тех пор.

Зато теперь мы знаем всё. И что такое изобилие. И что такое горе. Вдоволь.

Счастливы мы?

Не знаю.

Но точно знаю, какие слова многие из нас так и не сказали ей тогда.

Мы заплатили сполна за своё подростковое нахальство. Теперь мы поняли всё, чего
никак не могли осознать незрелым умом в те годы нашего безмятежного
избалованного детства.

Спасибо тебе! Не поминай нас плохо. И прости. За всё! Советская Родина»...

М. Жванецкий

\begin{itemize} % {
\iusr{Alla Kaminska}
\textbf{Андрей Кондратов} 

я когда читала Борины стихи, вспомнила и подумала именно о Жванецкому. Спасибо
Вам наши мысли совпали. Просто удивительным образом!

\iusr{Семен Калика}
\textbf{Андрей Кондратов} мудро...
\end{itemize} % }

% -------------------------------------
\ii{fbauth.semenjuk_viktor.chernomorsk.ukraina.odessa.poet}
% -------------------------------------

\begin{multicols}{2}
\obeycr
Что случилось, скажите, люди,
Как единство нам сохранить,
Где проходит раздела нить,
Как с таким настроением жить,
\smallskip
Что со внуками завтра будет?
Невесёлый сегодня вечер,
В небе месяца рог повис,
И народ, словно парус в бриз,
\smallskip
Безразлично, безвольно вниз
Опустил и глаза, и плечи...
С головы загнивает рыба,
А затем и мозги - труха,
\smallskip
Пахнут нечистью потроха,
Разлагаются в прах верха,
И готовится к пиру дыба...
Может солнце идёт к закату,
\smallskip
Может облако на восток,
Мир лукавый порой жесток,
Вот вам истина между строк:
Брат становится волком брату.
\smallskip
Кто вбивает столбы осины?
Вразуми, Отче наш, мирян...
Может воздух от злобы пьян,
Может кормчий безмерно рьян
\smallskip
И не видит нутра трясины...
Встрепенись же, Отчизна-Нэнька.
Супостатов стряхни с себя
И свободу, как жизнь любя,
\smallskip
Душу больше не бередя,
Выбирай, чтоб по шапке Сенька.
\smallskip
Виктор Семенюк
\restorecr
\end{multicols}


\iusr{Георгий Делиев}

Боренька. Это ярко, точно, правдиво, красиво и смешно... ну и оптимистично,
конечно, как все у тебя.

\begin{itemize} % {
\iusr{Максим Чеботаренко}
\textbf{Георгий Делиев} Высокие отношения!

\iusr{Юлия Шестопалова}
\textbf{Георгий Делиев} Где вы тут что-то смешное увидели?? Да и оптимизма тут не ноль, а минус. Очевидно, профессиональная деформация.
\end{itemize} % }

\iusr{Владимир Лушев}

Барский в своем репертуаре если зачерпнет так с самого дна, чтоб проняло до
самой печёнки, настоящий Одессит.


\iusr{Савченко Татьяна}
Не хочется вообще в дерьме стоять....

\iusr{Inara Zarichanskaya}
Сильно! Спасибо! Не хочется стоять по ноздри в дерьме...

\iusr{Анна Аллахвердиева}
Супер! Точнее и не скажешь!!! Благодарю!!!  @igg{fbicon.beaming.face.smiling.eyes}  @igg{fbicon.kiss.mark} 

\iusr{Alla Kaminska}

Спасибо большое Боря! Читая твои стихи думала, как жаль что так мало людей их
читают, вот бы огромные залы и твои стихи, как Михаил Михайлович Жванецкий
собирать! Каждая строка и в сердце и к уму ступенька!!! Боря -это гениально
!!!@igg{fbicon.heart.red}


\iusr{Elena Vikke}
Какой Вы замечательный!

\iusr{Наталия Толоманенко}
Браво!

\iusr{Sergei Solodov}
"Страну духовных великанов
Вели к могиле лилипуты." - ГЕНИАЛЬНЫЕ СТРОКИ!!!

\iusr{Kolia Terentiev}
Даже лайк как- то неловко ставить. Но поставлю потому что талантлив о!

\iusr{Алёна Зелинская}
Правда(

\iusr{Инна Трубчанинова}
Браво, Борис!...Браво!

\iusr{Лариса Гейчун}
Браво! Гениально! Спасибо!  @igg{fbicon.face.happy.two.hands} 

\iusr{Александр Сесмий}
Гениально!

\iusr{Сергей Джиоев}
Актуально как-то  @igg{fbicon.cry} 
Браво Боря  @igg{fbicon.thumb.up.yellow} 

\iusr{Александр Петренко}
"Страну духовных великанов вели к могиле лилипуты" Лучше и не скажешь (

\iusr{Дмитрий Стрельников}
Вся правда за нашу жизнь! Гениально!

\iusr{Margo Art}
Спасибо за ваше творчество, вызвало смешанные чувства! @igg{fbicon.hands.applause.yellow}{repeat=3} 

\iusr{Денис Иванов}
Дядь Борь ты лучший одессит.кайф

\iusr{Кирилл Дубинский}
Тонко  @igg{fbicon.thumb.up.yellow} 
Спасибо, за глубокий смысл. Сегодня очень актуальны эти строки  @igg{fbicon.thumb.up.yellow}{repeat=2} 

\iusr{Tais Helaf}
Спасибо!

Андрей Шашков
Все так

\iusr{Андрей Шашков}
Борис как верно сказано. Так и есть  @igg{fbicon.hands.applause.yellow}{repeat=2} 

\iusr{Oleg Borysyuk}
Одесса?

\iusr{Алена Асмила}
Китай!

\iusr{Сергей Давий}
Браво, Боря !!!
Как жаль, что это про нас ...

\iusr{Roraima Orion}
Брависсимо!!!!!!!
Спасибо вам огромное!!!!

\iusr{Ellie Charming}
Очень тонко!!!  @igg{fbicon.hands.applause.yellow}{repeat=3} 

\iusr{Саня Ромашкин}

\obeycr
И шо? Как быть?
Тонуть не падать...
Набрали воздух и в говно...
И даже научившись плавать
Шо там, шо там - сплошное дно
\restorecr

\iusr{Михаил Волошин}
Прекрасно, Супер!

\iusr{Vovker Vovs}
Пессимистично

\iusr{Игорь Чагирь}
Красиво! Но жаль, что про нашу жизнь!

\iusr{Елена Мартынюк}
Боря, гениально!!!

\iusr{Леонид Волох}
Актуальное!  @igg{fbicon.thumb.up.yellow} И изумительное!

\iusr{Семен Калика}
Вот такая вот х...я, «малятки»...

\iusr{Сашка Крыл}
Очень сильно и жизненно как никогда! @igg{fbicon.thumb.up.yellow} 

\iusr{Яра Кот}
Борис..." СТРАНУ ДУХОВНЫХ ВЕЛИКАНОВ
ВЕЛИ К МОГИЛЕ ЛИЛЕПУТЫ..." это бомба
...это в точку...даже не бровь, не в глаз, а в лоб @igg{fbicon.hand.ok} 
С Вашего разрешения забиру себе на память. Если всё же пройдем это .... будет что вспомнить и потомкам рассказать в стихотворной форме @igg{fbicon.wink} 

\iusr{Sergey Fadeev}
Да, лучше и не скажешь. Браво, Борис!

\iusr{Елена Кибальчич}
 @igg{fbicon.hands.applause.yellow}  браво

\iusr{Лариса Лавренюк}
В точку.

\iusr{Александр Барошин}
Для социальной группы вата, достаточно первой строчки

\iusr{Наталья Лапина}
Это просто супер класс

\iusr{леон гукасов}
Решительно СО-ГЛА-СЕН!
( не слова матом а как ёмко, ...ть)

\iusr{Игорь Нагорный}
Хорошо

\iusr{Юлия Шестопалова}

"Закат затягивала Вата", как много можно сказать в одной фразе! Боря, вы
супер-художник с рифмой на уровне Ф.Г.Лорки! И Зелёная игуана - таки да,
лыбится именно придурковато!

\iusr{Михаил Зайдель}
Универсальный стих, что характерно...

Дороги Меняют Цвет - ДМЦ

Это очень!

\iusr{Саша Касьяненко}
супер

\iusr{Inna Muzychuk}
@igg{fbicon.heart.red}{repeat=3}

\iusr{Коля Ступак}
@igg{fbicon.check.mark} ️  @igg{fbicon.flame}  @igg{fbicon.hands.applause.yellow} 

\iusr{Николай Коротких}
Браво Боря @igg{fbicon.thumb.up.yellow}  Точнее и не скажешь @igg{fbicon.hand.ok} 

\iusr{Ирина Леус}
Страну духовных великанов вели к могиле лилипуты... Сильно!

\iusr{Света Зигида}
Супер

\iusr{Наталья Тимофеева}
Как Вы правы!!!!!!?

\iusr{Татьяна Коновалова}

\ifcmt
  ig https://scontent-frt3-1.xx.fbcdn.net/v/t39.30808-6/258445688_660256971808329_295670151897874502_n.jpg?_nc_cat=108&ccb=1-5&_nc_sid=dbeb18&_nc_ohc=aX7cQK_8iNQAX_TdXu3&_nc_ht=scontent-frt3-1.xx&oh=9adb1e071c0f6384068c1bae85d9c613&oe=61B0CE67
  @width 0.4
\fi



\end{itemize} % }
