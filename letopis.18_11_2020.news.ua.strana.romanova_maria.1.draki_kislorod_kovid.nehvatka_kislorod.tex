% vim: keymap=russian-jcukenwin
%%beginhead 
 
%%file 18_11_2020.news.ua.strana.romanova_maria.1.draki_kislorod_kovid.nehvatka_kislorod
%%parent 18_11_2020.news.ua.strana.romanova_maria.1.draki_kislorod_kovid
 
%%url 
 
%%author 
%%author_id 
%%author_url 
 
%%tags 
%%title 
 
%%endhead 

\subsubsection{Нехватка кислорода, роды на ступеньках}
\label{sec:18_11_2020.news.ua.strana.romanova_maria.1.draki_kislorod_kovid.nehvatka_kislorod}

Разберем наиболее резонансные случаи в больницах, связанные с ковидом. 

15 ноября преподаватель Житомирского государственного технологического
университета Ольга Грабарь попала в реанимацию с ковидом. Оттуда украинка
пожаловалась мэру на отсутствие кислорода для пациентов. 

\enquote{Я знаю, вы в этом городе можете все! Я не знаю, как нам сейчас пережить ночь,
а на утро кислорода нет вообще! Я сейчас сама тут лежу, кислород отключали за
сегодня уже 8 раз! Бедные врачи в реанимации не знают, что делать!} - написала
Грабарь.

\ifcmt
pic https://strana.ua/img/forall/u/10/91/%D0%B3%D1%80%D0%B0%D0%B1%D0%B0%D1%80%D1%8C_1.png
\fi

В комментариях под этим постом женщине написала заммэра Мария Мисюрова. Она
заявила, что по состоянию на 22:00 15 ноября \enquote{кислород есть, все отделения
работают}. 

\ifcmt
pic https://strana.ua/img/forall/u/10/91/%D0%B3%D1%80%D0%B0%D0%B1%D0%B0%D1%80%D1%8C_2.png
\fi

Однако жизненно важного кислорода преподаватель не дождалась. Спустя два дня,
17 ноября, она скончалась в городской больнице Житомира №1. Ольге Грабарь было
39 лет. 

\index[rus]{Коронавирус!Смерть!Ольга Грабарь, Гор. больница Житомира 1}

О втором подобном прецеденте  - в Киевской области - сообщила волонтер Дана
Яровая.

По словам Яровой, у ее знакомого отец и сын попали в больницу. Им давали
кислород по 15 минут в сутки.

\enquote{Отец отказался от своих 15 минут в день в пользу сына. Отец умер, сына
продолжают лечить. Только вот не знаю, за умершего отца ему дают еще 15 минут
кислорода, или уже нет}, - написала Яровая.

Позже в комментарии \enquote{Стране} она уточнила, что трагедия случилась в больнице №3
Белой Церкви. 

\ifcmt
pic https://strana.ua/img/forall/u/10/91/%D1%8F%D1%80%D0%BE%D0%B2%D0%B0%D1%8F1.jpg
\fi

В Киевской ОГА инцидент отрицают. Чиновники ссылаются на главного врача
больницы Татьяну Дидыч, которая заявила: \enquote{В подчиненном мне медучреждении
такого случая не было}. 

Не только кислород становится причиной драматических инцидентов. 

В середине ноября на пороге роддома в Черкассах ребенка родила Евгения Киркичи.
Дальше порога роженицу с позитивным тестом на коронавирус не пустили. 

Муж Александр Киркичи пытался достучаться к врачам, чтобы они оказали помощь,
но к ним никто не вышел. Схватки продолжались. Мужчина вынужден был снять
куртку, на которую легла жена. Роды он принял на улице самостоятельно. 

Несмотря на чудовищность случая, эта история со счастливым концом - с мамой и
малышом все хорошо. В подробностях об \enquote{оригинальных} родах молодой папа
рассказал в интервью \enquote{Страны}.\Furl{https://strana.ua/articles/interview/300808-aleksandr-kirkicha-v-cherkassakh-prinjal-rody-u-zheny-na-ulitse-pod-vishnej-v-bolnitsu-ikh-ne-vpustili-iz-za-kovida.html}

Во Львовской области в начале ноября трагически оборвалась жизнь целой семьи
врачей. 

79-летняя Анна Городенчук была заведующей гинекологическим отделением, а ее муж
- 80-летний Богдан Городенчук - завлабораторией и гастроэнтерологом в
Сокальской центральной районной больнице.

Несмотря на преклонный возраст, супруги до последнего продолжали принимать
пациентов. В итоге Горденчуки заразились ковидом и были госпитализированы во
львовскую больницу. Скончались они в разницей в день. Муж - 10 ноября, жена -
11-го. 

В конце октября осиротел ребенок другой пары врачей. Семья проживала в городе
Тараще под Киевом. 39-летняя Галина Степченко работала медсестрой, 51-летний
Валентин - врачом-анестезиологом в местной больнице.

Вместе со своим 13-летним сыном они заразились коронавирусом и попали в
реанимацию. Через неделю после госпитализации женщина скончалась. Супруг, узнав
об этом, умер в тот же день. Теперь ребенок остался один, он продолжал лечение
на аппарате ИВЛ.
