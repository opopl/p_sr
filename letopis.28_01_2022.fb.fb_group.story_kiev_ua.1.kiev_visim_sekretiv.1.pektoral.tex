% vim: keymap=russian-jcukenwin
%%beginhead 
 
%%file 28_01_2022.fb.fb_group.story_kiev_ua.1.kiev_visim_sekretiv.1.pektoral
%%parent 28_01_2022.fb.fb_group.story_kiev_ua.1.kiev_visim_sekretiv
 
%%url 
 
%%author_id 
%%date 
 
%%tags 
%%title 
 
%%endhead 
\clearpage
\subsubsection{1. ЗОЛОТА ПЕКТОРАЛЬ з Товстої могили}

\ii{28_01_2022.fb.fb_group.story_kiev_ua.1.kiev_visim_sekretiv.pic.1}

1. Залишилося зовсім мало часу, щоб побачити головний український скарб -
ЗОЛОТУ ПЕКТОРАЛЬ з Товстої могили!

Лише до 30.01.2022 року в \enquote{Музеї історичних коштовностей України}
(вул. Лаврська, 9) можна побачити оригінал золотої пекторалі, а також і низку
інших красивих експонатів. 

Виставка відкрита в 2021 році до 50-річчя з визначної дати в історії
української археології, коли Борис Мозолевський досліджував курган поблизу
м. Покров і відкрив одну з найбільш цінних знахідок.

Особливу увагу до виставки привернули події із міжнародним судом, де російська
федерація намагається відсудити у України належне останній скіфське золото.
Україна перемагає в європейських судах і перші 19 експонатів вже повернуті та
презентовані на цій виставці.

Поспішіть, оригінал пекторалі буде доступний для огляду лише кілька днів! 

Але інші прекрасні експонати музею можна буде подивитись в будь-який час.

