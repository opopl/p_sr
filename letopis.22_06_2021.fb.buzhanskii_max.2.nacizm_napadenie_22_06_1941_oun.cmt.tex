% vim: keymap=russian-jcukenwin
%%beginhead 
 
%%file 22_06_2021.fb.buzhanskii_max.2.nacizm_napadenie_22_06_1941_oun.cmt
%%parent 22_06_2021.fb.buzhanskii_max.2.nacizm_napadenie_22_06_1941_oun
 
%%url 
 
%%author 
%%author_id 
%%author_url 
 
%%tags 
%%title 
 
%%endhead 
\subsubsection{Коментарі}

\begin{itemize}
\iusr{Evgeniy Akunin}
Да, миллионы погибших украинцев, русских, белорусов, евреев и др, на счету оуновцев

\begin{itemize}
\iusr{Макс Бужанский}
\textbf{Evgeniy Akunin}, миллионы? Щедро вы.

\iusr{Andrey Kadkalenko}
\textbf{Evgeniy Akunin} судя по действиям 14 дивизии, эффективность коллаборантов Вы завысили весьма по-недетски)

\iusr{Evgeniy Akunin}
\textbf{Макс Бужанский} сегодня по ТСН писали, что от вторжения захватчиков
погибло 8-10 млн украинцев. Сколько конкретно ОУН убила я не знаю. Но раз они
воевали в составе немецких войск, то и ответственность должны нести
коллективную, наравне с нацистами. Не бывает же такого, что СС убивали, а
Галичина вся в белом рядом стояла.

Не просто же так Порошенко, будучи президентом, ездил в кнессет и просил
прощения за преступления оун-упа. Значит виноваты в холокосте и убийствах
миллионов.

\iusr{Роман Романовский}
\textbf{Evgeniy Akunin} Порошенко за деньги сделает всё что угодно и не угодно. Мразь.
\end{itemize}

\iusr{Maxim Dzhezhelo}
Лицемерие!

\iusr{Эдуард Татарчук}

А что тут решать??? Тут или чёрное или белое, без оттенков серого. Нельзя быть
чуть-чуть беременным. Пеленать чёрную, свою, сущность в белые одежды
ПРАВЕДНИКОВ.  НЕ ЗА ЭТО НАШИ ДЕДЫ ВОЕВАЛИ!

\iusr{Андрей Буров}
Слава Украине! Героям слава! Какие могут быть претензии?

\iusr{Viktoria Svitlova}

Полагаю, концепция звучит примерно так - «всех жалко»

\iusr{Бойко Алексей}

Какого бывшего эсесовца?  Как оуновцы попали в эти батальоны, на каком языке
говорили, гражданами какой страны были?  (я в этом не очень; ответы хотелось бы
знать)

\iusr{Ihor Chvankin}

\textbf{Бойко Алексей}
Большинство из этих оуновцев родились в одной стране с Гитлером- в
Австро-Венгрии. Большинство были гражданами так кстати ликвидированной Польши.

\iusr{Бойко Алексей}
\textbf{Ihor Chvankin} спасибо

\iusr{Alexander Bergelson}
И присягнувших врагу!..
\enquote{Каждый подонок, надевший нацистскую форму, сдохнет.} Это из \enquote{Бесславных ублюдков}...

\iusr{Egor Trofimoff}
Увы, не каждый.

\iusr{Alexander Bergelson}
Egor Trofimoff Пока - не каждый.

\iusr{Egor Trofimoff}
Сдохнут все. 80 лет есть 80 лет. Что досадно - своей смертью.

\iusr{Андрей Блинов}

Стоит прямо писать, что Роман Шухевич перешёл границу утром 22 июня. И это было
не ОУН, не УПА. Подразделение «Бранденбург 800» переправилось через реку Сан
возле села Воловое, двигаясь в боевых порядках 49-го горно-стрелкового корпуса
17-й немецкой армии группы армий «Юг». Затем в течение недели батальон двигался
с минимальными боями через Яворов — Янов к Львову

\begin{itemize}
\iusr{Макс Бужанский}
\textbf{Андрей Блинов}, да, но вопрос же не лично в Шухевиче, а в подходе

\iusr{Александр Мордкович}
\textbf{Андрей Блинов} это была как раз ОУН. приданные Абверу части ДУН - боевое крыло ОУН

\iusr{Станислав Яшин}
\textbf{Андрей Блинов} и это \enquote{герой}???

\iusr{Андрей Блинов}
\textbf{Александр Мордкович} в сортах абвера не силён
\end{itemize}

\iusr{Oleg Gorbenko}
Полностью согласен. Государство двойных стандартов.

\iusr{Андрей Кащеев}
Для адекватных людей это отбросы и позорная страница нашей истории, заслужили только осуждения и презрения.

\iusr{Елена Хорошевская}

Мы уже давно живем в каком-то виртуальном мире и казалось, что трудно чему-либо
удивиться, но Президентским воякам таки удалось. Сначала думала фейк поверить
не могла.

\iusr{Alexander Abend}

\enquote{какие претензии к немецким бомбардировщикам?} Исходя их текста \enquote{Акта відновлення...} претензий не было

\iusr{Георгий Слащев}

Они всеядные
\enquote{Он играл на похоронах и свадьбах}

\iusr{Александр Мордкович}

в сотый раз) это не части Абвера. не брали папуасов в вермахт. приданные
туземные войска.  по версии Бандеры, союзническая Гитлеру украинская армия.

\begin{itemize}
\iusr{Макс Бужанский}

\textbf{Александр Мордкович}, абверовское кураторство, а паслись отдельно, на
коротком поводке) Версию Степана не комментирую из соображений гуманизма)

\iusr{Александр Мордкович}
да) вышкил проходили в Краковской разведшколе Абвера.

\end{itemize}

\iusr{Sergey Grinchak}
Один из центральных проспектов Киева назвать именем Шухевича - гауптмана абвера нацистской Германии это вообще сюр какой-то...

\iusr{Александр Шибанов}
\textbf{Сергей Гринчак} Уверен, не надолго. Местные говорят Ватутина, а не Шухевича.

\iusr{Oleg Turner}

Из сегодняшнего поста господина Зеленского в FB -

\enquote{Мине майже два роки, і нацистів з Києва буде вигнано...
Мине 80 років, а рани, які залишила Друга світова на серці нашої країни та столиці,
все ще болітимуть...}

Даже не нужно писать какие-либо ответы на это... Там и так их достаточно... Поэтому, просто, скопирую ответ одного из парней -
Мине 80 років і... -зигующие молодчики будут безнаказанно ветеранов унижать,
мине 80 років - будут на 9 мая бабушек/дедушек разгонять за пение военных
песен, мине 80 років - будут могилы красноармейцев осквернять безнаказанно,
мине 80 років - будут с почестями пособников фашистов хоронить, мине 80
років-будут улицы и проспекты в Украине и Киеве называть именами тех, кто
присягал гитлеру и фашисткой нечисти, мине 80 років, і нам взагалі буде
заборонено називати війну \enquote{Великой Отечественной}, а день победы 9 Мая
перестанет быть праздником..." Собственно спорить не с чем...😎

\iusr{Алексей Александров}
\textbf{Oleg Turner} так вы с этим миритесь. Протестантов не видно...

\iusr{Ирена Березник}
\textbf{Алексей Александров}, а кто хочет быть избитым «врагом народа»? И хорошо,если просто избитым...

\iusr{Марина Клевцова}
Эти попытки и нашим и вашим, вызывают только чувство брезгливости....Господь им судья....

\iusr{Елизавета Бойкова.}

\textbf{Марина Клевцова} не, до Господа люди должны своё слово сказать и своё дело
сделать, если они люди. Нечего всю ответственность за творящееся и творимое на
Господа сваливать

\iusr{Vergola Alexander}
Я не знаю, кто для сегодняшней власти, в т.ч. для президента \enquote{наши}. Думаю, что у меня и у них \enquote{наши} разные.

\iusr{Evgeniy Tunic}

Если кто-то решил свернуть шею большевикам, так что ж в этом плохого ? Но
погубило то, что кто-то решил, что его порода в этом мире более элитная...

\begin{itemize}
\iusr{Макс Бужанский}

\textbf{Evgeniy Tunic}, да ничего плохого, а ваши родители кем были,
оккупантами, коллаборантами или рабами? Лично ваши. Родители родителей, если
так удобней.

\iusr{Evgeniy Tunic}
\textbf{Макс Бужанский}, ближе к рабам ((

\iusr{Макс Бужанский}
\textbf{Evgeniy Tunic}, так это ваша личная семейная проблема, комплексы, которые стоит держать при себе.
У всех остальных родители рабами не были, жили в своей стране.

\iusr{Иван Бодрийяров}
\textbf{Evgeniy Tunic} \enquote{Но погубило то, что кто-то решил, что его порода в этом мире более элитная} - жалеете?

\iusr{Макс Бужанский}
\textbf{Иван Бодрийяров}, никак не найду этот ваш замечательный пост про допрос, когда
допрашиваемый вдруг начал блеять в оправдание, а вот вы, вот вы.

\iusr{Evgeniy Tunic}

Макс Бужанский, до поры до времени. И где теперь \enquote{своя} страна ? 😉

\iusr{Evgeniy Tunic}
Макс Бужанский, а допрос Миляги ? ))

\iusr{Бойко Алексей}
\textbf{Evgeniy Tunic} 😅🤣😂

Вообще ничего плохого не находите?  Так то чужую территорию пришли
завоёвывать.. С убийствами там, членовредительством..  Норм?

\iusr{Макс Бужанский}

\textbf{Evgeniy Tunic}, теперь я в ней живу, в своей стране. А вы в чужой, ваши то рабами были, откуда ж своему взяться.

\iusr{Evgeniy Tunic}
\textbf{Бойко Алексей}, ну, если большевикам подобное позволительно ! Вот, кто-то и поставил их себе в пример )))

\iusr{Evgeniy Tunic}
\textbf{Макс Бужанский}, да, вы такой не первый ))

\iusr{Макс Бужанский}
\textbf{Evgeniy Tunic}, не первый, тут таких под 40 миллионов)

\iusr{Бойко Алексей}

\textbf{Evgeniy Tunic}, так себе пример и сравнение с последующим
\enquote{выводом} так себе.. Не решение вопроса и не вариант вообще.  В топку.

\iusr{Evgeniy Tunic}

\textbf{Макс Бужанский}, да, так Милягу к гражданству какой страны можно отнести ? 😉

\iusr{Макс Бужанский}

\textbf{Evgeniy Tunic}, да мне наплевать на Милягу, вы написали бред у меня в
комментариях, я вам ответил.  Потомок рабов, о чем тут спорить, вы с другой
планеты.

\iusr{Evgeniy Tunic}

Макс Бужанский, ну, вы ж сами такой малый выбор предоставили )) И я не ответил
утвердительно )). А себя-то к каким оставшимся двум категориям (оккупантам или
коллаборантам) определите ? ))

\iusr{Иван Бодрийяров}

\enquote{да, так Милягу к гражданству какой страны можно отнести ? } - вы его сами
придумали и сами теперь спрашиваете.  речь изначально о совершенно другом
допросе вообще

\iusr{Макс Бужанский}

\textbf{Иван Бодрийяров}, это он Чонкина цитирует, ему кажется, что это смешно.

\iusr{Макс Бужанский}
\textbf{Evgeniy Tunic}, а я вне этих категорий, изначально вне их.

\iusr{Evgeniy Tunic}
\textbf{Иван Бодрийяров}, понимаю, но, просто, вспомнилось )

\iusr{Evgeniy Tunic}
\textbf{Макс Бужанский}, вовсе не смешно. Поражаюсь вашей осведомлённости о том, что мне кажется ! ))

\iusr{Mark Shamis}
\textbf{Evgeniy Tunic}

А \enquote{шею сворачивали} только большевикам или по ходу всем подряд...?🤔

\iusr{Иван Чернов}
\textbf{Evgeniy Tunic} такой взрослый и такой глупый. (

\iusr{Evgeniy Tunic}
\textbf{Mark Shamis}, а смысл всем подряд ? Кто ж тогда, после полной победы, за победителями дерьмо будет убирать ?

\iusr{Evgeniy Tunic}
\textbf{Иван Чернов}, да, в атноклассники мне тоже такое говорили ))

\iusr{Макс Бужанский}
\textbf{Evgeniy Tunic}, выражения выбирайте, ещё раз услышу что то похожее на слово вата, и вас тут не будет.

\iusr{Иван Чернов}
\textbf{Evgeniy Tunic} но кастрюля все равно не даёт это понять? ))

\iusr{Evgeniy Tunic}
\textbf{Макс Бужанский}, я там поправил орфографическую ашипку )) И, в свою
защиту и оправдание, могу предъявить собственный профиль, где упомянутый вами
термин не использую, точно так же, как и другие популярные среди наших
сограждан ! ))

\iusr{Evgeniy Tunic}
\textbf{Иван Чернов}, обиделся, да ? ))

\iusr{Александр Твардовский}
\textbf{Evgeniy Tunic} плохого ....ничего...разве что могло бы и вас не быть,
просто потому что кого то из ваших родственников, не дай бог конечно..могли бы
просто сжечь.

\iusr{Александр Твардовский}
\textbf{Бойко Алексей} сжигание деревень...с жителями...

\iusr{Evgeniy Tunic}
\textbf{Александр Твардовский}, равно, как и отправить безвозвратно строить Светлое Будущее ((

\iusr{Сергей Бабенко}

Пришлось общаться с таким строителелем \enquote{светлого будущего}, 10 лет лагерей.
Ответ меня поразил, в нем вся мудрость народа.  Он сказал:\enquote{ВРЕМЯ было такре. По
другому было нельзя. Иначе стране КАПЕЦ.}

\iusr{Mark Shamis}
\textbf{Evgeniy Tunic}
Об одном жалею - что победитель хреново вычистил \enquote{дерьмо}, после полной победы.

\iusr{Нико Соловей}
\textbf{Evgeniy Tunic} жертва аборта, недоносок 😀

\iusr{Александр Твардовский}

\textbf{Evgeniy Tunic} равно как и то что я знаю что если бы немец все же
бросил гранату в блиндаж где была моя бабушка.. не было бы ни моей матери ни
меня...., Говоря о том что кто то решил убрать большевиков подумайте о том что
цели и задачи были как раз в установлении своего режима, а это совсем не то что
вы говорите, уничтожение жителей и военнопленных это отнюдь не особождение от
другого строя, говоря подобные вещи вы просто показываете себя Не с лучшей
стороны....

\iusr{Mila Bondar}

\textbf{Mark Shamis} дипломатично.

\iusr{Наталья Лучкевич}
\textbf{Mark Shamis} 👍👍👍

\iusr{Станислав С.}

\textbf{Evgeniy Tunic} потрудитесь изучить цели плана Барбаросса

\iusr{Konstantin Konstantinovich}
\textbf{Evgeniy Tunic} жили в землянке, ложились спать голодными, не могли
оплатить обучение в ВУЗе....

\iusr{Олег Морев}
Мак для чего вообще с такими вести диалог, они найдут проблему и в пустом,чистом стокане.

\iusr{Vladimir Makarov}
\textbf{Evgeniy Tunic}. От части соглашусь с вами, большевики построили полное
политическое дерьмо, и не строй а целую чёрт знает какую цивилизацию, но есть
одно НО, иностранных освободителей от жизни в ту страну ни кто не звал, и те
кто с ними пришли тоже ино... See More

\iusr{Evgeniy Tunic}
\textbf{Александр Твардовский}, но немец не бросил..., странно... Исходя из
логики большевицких фильмов, должен был бросить. Как же так ?!  

\iusr{Evgeniy Tunic}
\textbf{Mark Shamis}, кого вы считаете победителем ? Указкой на карте покажите ! ))

\iusr{Evgeniy Tunic}
\textbf{Константин Константинович}, я другой такой страны не знаю...!

\iusr{Evgeniy Tunic}
\textbf{Станислав С.}, я нацистов и не оправдываю. А чтоб вы им предложили
альтернативное ? Или считаете, что те не видели, как красные подтягиваются к
восточной границе ?

\iusr{Александр Твардовский}
\textbf{Evgeniy Tunic} дядя. Ты серьезно дурак? Похоже это диагноз.

\iusr{Evgeniy Tunic}
\textbf{Александр Твардовский}, скачи уже, будёновец !

\iusr{Вадим Бродский}
\textbf{Evgeniy Tunic} лечись, слабоумный

\iusr{Станислав С.}
\textbf{Evgeniy Tunic} к восточной границе чего?

Война Германии была никак не против Большевиков или коммунистической идеологии,
абсолютно захватническая кампания, с целью господства.

К тому времени Германия уже с теми же целями напала на вполне себе даже более капиталистические государства.

Не перекручивайте историю.

Можете ознакомиться с историей этой войны из германских современных источников.
Поверьте с германской стороны эта война рассматривается как история мирового
позора, и там нет и толики намёка на то, что это была война с Большевиками, или
освободительная кампания ради независимости Украины или украинского народа.

Захватив советские земли территории современной Украины, Германия установила
Рейскомиссариат, а никак не суверенное государство.

\iusr{Evgeniy Tunic}
\textbf{Станислав С.}, но, если бы Германия не напала первой, удалось бы Европе избежать \enquote{освободительного} похода красных ?

\iusr{Александр Твардовский}
\textbf{Evgeniy Tunic} скройся в схроне, прихвостень.

\iusr{марина пешня}
\textbf{Макс Бужанский} а он без роду без племени

\iusr{Константин Исиков}
\textbf{Evgeniy Tunic} удалось, ибо красные не планировали никакого
\enquote{освободительного похода} для этого им нужно было иметь экономический потенциал
хотя бы равный своим будущем противникам, а СССР уступал совокупному
антисоветскому союзу раз в 10.

\iusr{Константин Исиков}
\textbf{Evgeniy Tunic} и кстати, как тут правильно заметили, это не была война
против большевизма, это была война за \enquote{жизненное пространство} согласно Майн
Кампфу, где надлежало выселить всё славянские население за Урал, а на
европейской части СССР оставить два десятка миллионов для обслуживания
инфраструктуры. Все эти земли должны были заселить немецкие колонисты.

\iusr{Станислав С.}

\textbf{Evgeniy Tunic} 

почитайте историю внимательней. Возьмите немецкие источники. Их уж точно нельзя
считать пророссийскими или просоветскими. Там и намёка нет на то, что нападение
было превентивным.

До подобной чуши, кроме вас ещё додумался Гёббельс, и то только с целью
пропаганды коллаборационизма и привлечению молодых трудоспособных недоумков на
работы в Германию.

\iusr{Mark Shamis}

\textbf{Evgeniy Tunic}

Победителем я считаю своего отца, его родного брата, мужа моей родной тети....

Если бы не они, у меня бы не было шанса родиться.

И да, вам полезно иногда смотреть на карту Мира. Не многим странам удалось,
пройдя сквозь века, войны, революции, майданы, сохранить себя, свою
аутентичность, свой суверенитет.

В эти \enquote{шахматы} знаете ли, играют сегодня всего 2 игрока и играют они фигурами
разного достоинства, но с одним недостатком - игрок может \enquote{съесть} или
пожертвовать любую из этих фигур по своему усмотрению и исходя из соображений
защиты своих национальных интересов.

И вопрос вам, любителю карт: какая страна веками владеет самой большой
территорией, которую так и не смогли захватить ни Наполеон, ни Антанта, ни
Гитлер?

Трижды за 200 лет Россия противостояла объединенным усилиям Запада её завоевать
и всё ещё остаётся самой большой страной мира и самой большой занозой для
Запада. Пожалуй такое под силу только победителю.

Не правда ли?

\iusr{Konstantin Konstantinovich}
\textbf{Evgeniy Tunic} не понял ????

\iusr{Evgeniy Tunic}
\textbf{Mark Shamis}, про территории согласен, а что на счет ВВП на душу населения победителей и побеждённых ?

\iusr{Alex Naz}
\textbf{Evgeniy Tunic} дБ.

\iusr{Alex Naz}
Суверинитет пишется как суверенитет. Мыслители хреновы.

\iusr{Alex Naz}
Где вы, бдь, этих слов берете!?)

\iusr{Evgeniy Tunic}
\textbf{Константин Исиков}, как можно отнять у народа любимое занятие ?!

\ifcmt
  pic https://scontent-lga3-2.xx.fbcdn.net/v/t1.6435-9/205467983_4488266864518696_3856663065068370980_n.jpg?_nc_cat=104&ccb=1-3&_nc_sid=dbeb18&_nc_ohc=atebfam3x7cAX_JdaJG&_nc_ht=scontent-lga3-2.xx&oh=f94c41aa71b028620efda33a9ec6a868&oe=60D7F10C
  width 0.7
\fi

\iusr{Константин Исиков}
\textbf{Evgeniy Tunic} а что там ещё на заборе написано? Какие любимые слова про дятлов,?

\iusr{Mark Shamis}
\textbf{Evgeniy Tunic}

Вы, часом, не из числа сторонников теории: \enquote{сдались бы в 1941-м Гитлеру - пили бы сегодня Баварское}?

Примите во внимание кровавые и разрушительные события, произошедшие в России с
1900 года: 2 революции, 1-я мировая, Гражданская война, ВОВ, \enquote{холодная
война} развал СССР и всё это происходило не без помощи \enquote{заклятых
друзей} России...🧐

Как по-вашему, каков был бы ВВП России, если бы всего этого не произошло?

О холодной войне с 1947 с санкциями и о печатном станке я уже молчу... 🤫

Не всё в нашей жизни измеряется с помощью ВВП.

Те, кого ВВП не устраивает могут эмигрировать и гордиться чужим ВВП или
остаться и работать, работать, работать... 😉

\iusr{Evgeniy Tunic}
\url{https://www.facebook.com/profile.php?id=100000062810456}

\textbf{Константин Исиков}, в те времена, за такие слова, те, кого вы пытаетесь
оправдать, к тому забору и поставили бы ! Работать - это конечно ! ))) Разве
Светлое Будущее кто-то в этой стране отменил ? 😉

\iusr{Alexey Pugachev}

\textbf{Evgeniy Tunic} если кто-то решил подставить жопку Карлу, потом Петру,
потом Шикельгрюберу, потом хрущеву, потом, потом, потом... Жопка ещё не болит?

\iusr{Vladimir Sklyarov}
Очень тупой. Гений тупости.

\iusr{Наталья Бень}
\textbf{Evgeniy Tunic} что это сейчас было?? Свернуть шею? оказывается это было идеей или
чем то еще, этой бойни? ну бред сивой кобылы, нет других слов, тьфу

\iusr{Evgeniy Tunic}
Наталья Бень, а вы знаете, сто Слава КПСС больше не мужчина ?

\iusr{Станислав С.}
\textbf{Владимир Макаров} вы в школе изучали при каких обстоятельствах в 1917
году Большевики пришли к власти? Про Февральскую и Октябрьскую революции
слышали? Про первую мировую?

\iusr{Vladimir Makarov}
\textbf{Станислав С.} Не поверите даже, историю КПСС \enquote{, изучал} или преподавали, вы не юлите, суть вопроса то в чём?!)

\iusr{Станислав С.}
\textbf{Владимир Макаров} это комментарий на вопрос что построили большевики,
дерьмо по вашему.  В чем дерьмо правда не уточнили, главное пнуть мертвого
льва.  Я вообще поражаюсь украинскому мировоззрению, во всех прошлых бедах
виновны большевики или коммунисты,
во всех теперешних Путин.

\iusr{Vladimir Makarov}
\textbf{Станислав С.} Там если внимательно читать написано, \enquote{политическое
дерьмо} но не индустриальное, и знаете не настолько я молод, что бы моё
мировозрение было сформированно при нэзалэжной.

\end{itemize}

\iusr{Тамара Карпенко}
ДА .....

\iusr{Александр Шибанов}

Человек вступивший в сговор с нацистами,даже ради своих великих
целей,признаётся пособником и соучастником убийств евреев,славян,ромов..

\iusr{Fedor Limarev}
Щас вам ответят, что вы ничего не понимаете и что это другое....

\iusr{Анна Герман}
\textbf{Фёдор Лимарев} наративы Кремля

\iusr{Прошкин Виктор}
Это другое 🙈

\iusr{Andriy Hluhenkiy}

Це спекуляція на емоціях. Наші предки воювали по один бік з енкаведистами, які
масово катували, розстрілювали й репресували їх рідних. Чомусь у вас це не
викликає обурення. Реальна історія складніша за пропагандистські інтерпретації
й трибунні спекуляції хайпожерів.

\begin{itemize}
\iusr{Макс Бужанский}
\textbf{Andriy Hluhenkiy}, серьезно?) Так это как то оправдывает вторжение в страну в рядах нацизма, да?)

\iusr{Andriy Hluhenkiy}
\textbf{Макс Бужанский} ні. Немає сенсу когось виправдовувати. Тоталітарні
режими з обох боків лишали мало простору для маневрів звичайним людям. Кожен
дійсно повинен визначатись зі своїм ставленням сам, але бажано маючи у
розпорядженні якомога більший обсяг фактів, а не лише подані патетично
радянські кліше.

\iusr{Evgeniy Akunin}
\textbf{Андрій Глухенький} а реальная история - это та, которую по 5 каналу и ТСН рассказывают украинцам😂

\iusr{Макс Бужанский}
\textbf{Andriy Hluhenkiy}, да нет, не надо рассуждать о тоталитарных режимах, тут это ни к чему.
Одни завели людей в газовые камеры, вторые их оттуда вывели, вот и все, даже если заменить советские клише на антисоветские.

\iusr{Крохин Дмитрий}
\textbf{Макс Бужанский} да в совке убивали миллионы по другому не в газовых камерах)))

\iusr{Andriy Hluhenkiy}
\textbf{Макс Бужанский} так може для повної чесності не варто забувати, як люди
потрапляли з нациських таборів в радянські? Чи ви вбачаєте принципово якісну
різницю між вбивствами газом й розстрілами чи доведенням до голодної смерті?

\iusr{Макс Бужанский}
\textbf{Крохин Дмитрий}, везде убивали по всякому, только газовая камера- единственное место, в котором шансов нет никаких никогда.

\iusr{Макс Бужанский}

\textbf{Andriy Hluhenkiy}, а расскажите, как люди попадали из нацистских
лагерей в советские.  И как вы вдруг приравняли концентрационные лагеря и
лагеря смерти к фильтрационным.

\iusr{Крохин Дмитрий}
\textbf{Макс Бужанский} у двадцати миллионов и без газовой камеры шансов не было)

\iusr{Макс Бужанский}
\textbf{Крохин Дмитрий}, не знаю, о каких двадцати миллионах идёт речь, не готов комментировать.

\iusr{Крохин Дмитрий}
\textbf{Макс Бужанский} жаль

\iusr{Макс Бужанский}
\textbf{Крохин Дмитрий}, скажите, о чем речь, обсудим.

\iusr{Крохин Дмитрий}
речь о уничтоженных коммунистами миллионах, по этому я понимаю людей,которые с оружием в руках боролись с коммунистами)

\iusr{Макс Бужанский}
\textbf{Крохин Дмитрий}, абсурдная цифра абсолютно, космическая.  Кстати, до
появления коммунистов, они благополучно уничтожали мирных украинцев в Западной
Украине, просто не могли себе позволить подобные масштабы.

\iusr{Evgeniy Akunin}
\textbf{Крохин Дмитрий} ваших предков убили или они убивали?

\iusr{Павел Харченко}
\textbf{Крохин Дмитрий} - і багатьох Ваших рідних у совку стратили? Як же Ви, сердешний, на світ з'явилися?

\iusr{Evgeniy Akunin}
\textbf{Крохин Дмитрий} с коммунистами всегда борятся фашисты. Вы понимаете
фашистов?👏кстати, где захоронения этих уничтоженных миллионах? Есть
доказательства этому, кроме бреда по 5 каналу?

\iusr{Konstantin Konstantinovich}
\textbf{Крохин Дмитрий} вы лично растительные списки пересчитывали?

\iusr{Vitaliy Pantin}
\textbf{Макс Бужанский} супер сказано

\iusr{Vitaliy Pantin}
\textbf{Андрій Глухенький} НКВД,ГПУ,КГБ конечно те еще сучьи дети. Но это наши
сучьи дети, а не фашистские. Поэтому, дядя, не надо делать ква.

\iusr{Andriy Hluhenkiy}
\textbf{Vitaliy Pantin} ваші. Мені нема різниці між гестапівцем та нквдистом, які вбивютт расово чи класово чужих їм елементів та неблагонадійних.

\iusr{Vitaliy Pantin}
Дима, возьми ружжо, стань мужчиной, плаксель ты наш.

\iusr{Andriy Hluhenkiy}

\textbf{Evgeniy Akunin} з комуністами боролись капіталісти. А фашисти взагалі
майже пацифіст на цій ниві, на тлі палкої боротьби з совєтами їх братів по духу
націонал-соціалістів. Вчіть матчастину

\iusr{Evgeniy Akunin}

\textbf{Андрій Глухенький} начните с себя и начните учить матчасть. В
капиталистических странах коммунистов не запрещали в отличии от фашистских
Германии, Италии, Испании. Ты оправдал фашизм😁 браво👏. Хорошо рассказывать
про хороший фашизм сидя в полученной квартире от коммунистов и когда мама и
папа в партии состояли, правда ведь?😂

\iusr{Валентин Станев}

\textbf{Андрій Глухенький} вы думаете что эти ваши предки убивали только
нквдистов ?а я смотрел хроники немецкие ! Так что не надо ля ля. Я против
действий НКВД осуждаю Ежова, но и тех кто был на стороне нацисткой Германии

\iusr{Vitaliy Pantin}
\textbf{Андрій Глухенький} отвечу просто и доступно для тебя. АНДРЮХА, ТЫ ПРОСТО ДЛБ.

\iusr{Nikolay Dyumin}
\textbf{Валентин Станев} во время ВОВ был Берия

\iusr{Andriy Hluhenkiy}
\textbf{Валентин Станев} мої предки воювали в Червоній армії й ненавиділи
комуністів. Вони добре пам'ятали як встановлювалась радянська влада, що таке
колективізація і Голодомор.

\iusr{Vitaliy Pantin}

\textbf{Андрій Глухенький} живи теперь с этим. Всегда мало кто любит власть. А
думаешь моему деду, мобилизованому с Межевского района Днепропетровской области
, ужк взрослому мужчине было всласть воевать в два захода? Но это был ВРАГ!
Поэтому шли и воевали. Против ВРАГА

\iusr{Andriy Hluhenkiy}
\textbf{Vitaliy Pantin} нацисти й були ворогом, але й кремлівська влада теж
була ворогом, в якого до того ж були в заручниках рідні і Батьківщина. А як
червоні вчиняли з заручниками, яких таврували родичами ворогів народу, всі
добре знали.

\iusr{Валентин Станев}
\textbf{Николай Дюмин} я знаю

\iusr{Валентин Станев}
\textbf{Андрій Глухенький} но Макс не пишет об бойцах красной армии которые не
любили власть таких много было, речь не об этом. Не путайте это

\iusr{Andriy Hluhenkiy}
\textbf{Валентин Станев} я уточнив, що мільйонам доводилось воювати за одного ворога проти другого на обох сторонах.

\iusr{Валентин Станев}
\textbf{Николай Дюмин} я просто не хотел писать всех кого осуждаю

\iusr{Валентин Станев}
Но были и честные храбрые генералы которые так же попали под каток репрессий

\iusr{Andriy Hluhenkiy}
\textbf{Валентин Станев} теж з обох боків? )

\iusr{Vitaliy Pantin}
\textbf{Андрій Глухенький} у СССР в пределах границ на то время был один враг.
Напавший первым. Все мои роздственники пострадали от него. Что не так я
объясняю, сударь? Все воевавшие против страны, в границах которой жили обычные
люди, являлись на тот момент врагом. А вы мне пытаетесь рассказать, что
миллионы воевали на стороне фашистов чисто францюватые хлопцы? Ок. Они для меня
лично враги. Потому что Дед мой погиб...живите с этим

\iusr{Vitaliy Pantin}
\textbf{Андрій Глухенький} за одного врага по обе стороны воюют только черти.

\iusr{Валентин Станев}
\textbf{Андрій Глухенький} с каких обоих)) враг это нациская Германия точка. Все остальное личное каждого

\iusr{Andriy Hluhenkiy}
\url{https://www.facebook.com/andriy.hluhenkiy}

\textbf{Vitaliy Pantin} це ваш особистий родинний досвід і висновки. Вони ніяк
не є єдино правильними для всіх інших людей. Бо це й може бути лише
індивідуально особистим, а не колективно урочистим побєдобєсієм. Щодо ворогів,
то для СРСР це була не лише Німеччина, до того були ще поляки, фіни, балтійські
народи, румуни, а головне - світова буржуазія. І дії щодо цих ворогів були
далеко не гуманними й оборонними. До речі, РФ зараз красномовно наслідує цю
агресивну й загарбницьку параною в стилі \enquote{кругом враги} з
\enquote{захисними} нападами на сусідів.

\iusr{Vitaliy Pantin}
\textbf{Андрій Глухенький} короче. Иди на йух

\iusr{Александр Шибанов}
\textbf{Vitaliy Pantin} Ну или так)))

\iusr{Константин Исиков}
\textbf{Андрій Глухенький} да хватит уже свои пропагандистские сказки
рассказывать про ужосы СССР 🙂 Я вам в другой ветке приводил цитату из
докладной записки СД и комментарии украинских националистов , что подавляющее
большинство считало себя советскими людьми, не поддерживали идею Незалежности и
не испытали не себе ужасы репрессий и пр.

\iusr{Alex Naz}
\textbf{Андрій Глухенький} слипенький уеб..н

\iusr{Andriy Hluhenkiy}
\textbf{Константин Исиков} я нормальним читачам там надав посилання на
документи, й вам, здавалось би, дохідливо пояснив, що з русскомірцами, які
підтримують агресію Кремля проти України, словесні дискусії з 2014 року
закінчені. З ворогами під час війни спілкуються інакше

\iusr{Константин Исиков}

\textbf{Андрій Глухенький} ну я другого ответа и не ожидал услышать от
демагогов. В отличие от таких как вы, я оперирую фактами, документами и
ссылками, этому меня научили в Мюнхене за два года. А вся это политическая
лирика \enquote{ворогів народу} удел пропагандистов. Я являюсь учёным,
исследователем. Каждый свой тезис я могу подтвердить не пустыми словами.

Ну и на последок, вороги Украины, это такие как вы, продавшие страну за чужие
геополитические интересы, втянув Украину а опасное и заведомо проигрышное
противостояние с соседом.

Всё это было спрогнозировано ещё 20 лет назад, и каждая сторона отыграла свою
домашнюю заготовку.

Рано или поздно эти события получат свою должную оценку историков и поставит
вас в один ряд с коллаборантами и вертухаями.

\iusr{Andriy Hluhenkiy}

\textbf{Константин Исиков} знову самопіар дешевого пропагандиста. Подальша доля РФ
очевидна. Ви, як і Кремль, не можете запропонувати оточуючим нічого вартого
уваги, крім понтів, насильства й брехні. Просто прийміть цю неприємну
реальність і від...ся від нас зі своєю третьосортною недоімперійкою. Вам же
краще буде)

\iusr{Денис Лазаренко}
\textbf{Андрій Глухенький} НКВД были русские или они украинцы????????

\iusr{Andriy Hluhenkiy}
\textbf{Денис Лазаренко} загалом різнонаціональний склад. Що це повинно змінити в оцінці їх діяльності?

\iusr{Денис Лазаренко}
\textbf{Андрій Глухенький} ты понял что написал?? Также на Донбассе украинцев убивают украинци но виноват Путин !!!!!

\iusr{Валера Фрейд}
\textbf{Андрій Глухенький}. А как тебе 5 миллионов доноров, в органы НКВД с территории Украины в 1936-1939 годах.

\iusr{Константин Исиков}

\textbf{Андрій Глухенький} видно кто из нас дешёвый пропагандист, ибо вы не
читали недавней эпохальной речи Макрона к правительству где он в который раз
возвращается к идеи великого Де Голля о единой Европе от Ла-Манша до
Владивостока потому что Росси есть что предложить Европе в духовном плане. Эта
речь легко гуглится. Так же как и легко гуглятся статьи, о том, что до вашего
никчемного майдана, в Европе в кругах старой элиты начала возрождаться идея о
единой Европе от Ла-Манша до Владивостока, что готовились огромные политические
и экономические эпохальные программы, которые вызвали опасения ваших
вашингтонских хозяев, именно поэтому они и устроили всю эту давно
спрогнозировнную и обновленную всеми сторонами войну в Украине. И США потом с
вас же и посмеялись.

Видно что ваш уровень знаний на примитивном уране пропагандиста, тогда как мой
опирается на реальные факты, которые я могу цитатами и ссылками подтвердить,
но боюсь тут места не хватит поместить весь материал.

\iusr{Andriy Hluhenkiy}

\textbf{Денис Лазаренко} Керує вбивствами українців Кремль. Воюють в 1і 2 АК,
що підпорядковуються командуванню ЗС РФ Південного військового округу,
іхтамнєти без шевронів і колаборанти, озброєння вишкіл і командування
російські. Що ж тут може бути не зрозумілого на восьмий то рік війни? Від того,
що у Французькому легіоні служить якась кількість росіян він же не стає
російським, і якщо французи воювали б з РФ це ж не було б громадянською війною)

Колаборанти - такі ж вороги, як і російські військові. Якщо ж говорити про пост
Бужанського, то іронія, що люди про яких він пише переважно навіть громадянами
СРСР побувати не встигли за два роки після німецько-радянської агресії проти
Польщі)

\iusr{Денис Лазаренко}
\textbf{Андрій Глухенький} а нахрена Америка нам помощь заморозила ??¿?

\iusr{Владимир Сумкин}
\textbf{Андрій Глухенький} в фаворе! Его слушают и дискутируют! Вы его поддерживаете как
чайный гриб: пить противно, а выкинуть жалко

\iusr{Andriy Hluhenkiy}

Константин Исиков Франція не РФ, там демократія, політики швидко змінюються, а їх факапам у підсумку дається належна оцінка. Де зараз попередник Макрона нагадувати, думаю, нікому не треба? ) А про російську духовність дуже смішно було, визнаю)

\iusr{Константин Исиков}
\urlFriend{https://www.facebook.com/profile.php?id=100057402014042}

\textbf{Андрій Глухенький} кстати, \enquote{Русский мир} с которого вы
иронизируете придумали два украинца преподаватели киево-могилянской академии
киевлянин Прокопович и львовянин Яворский и ещё 5 лет уговаривали Петра 1
принять эту концепцию. Целью этой триединой братской концепции было совместными
усилиями отвоевать западные, южные, восточные земли Киевской Руси, а также
экспансия на Восток, в этом всём \enquote{коварном завоевании чужих земель}
украинцы принимали самое активное участие и соответсвенно занимали высочайшие
должности сначала в Российской империи а затем в СССР.

А теперь вы скулите про недоимперию лишившись добровольно всего, что
завоёвывали наши прадеды своей кровью и превратились в отсталую аграрную
недореспублику, жалкий хутор торгующий сырьём.

Вы даже этого простейшего исторического материала не знаете.

\iusr{Andriy Hluhenkiy}

\textbf{Константин Исиков} я в курсі хто нафантазував цей міф, як і про те, що потім
його активно розвивали німці для німецьких же правителів РІ. Але ж в підсумку
погано вийшло, забрехались теоретики, а практики облажались. Помилки треба
визнавати, а безперспективний проект закривати. Ресурсоємний, непродуктивний,
забагато багів від самих основ, щоб зараз сподіватись щось виправити. Вже ж не
взлетіло під різними соусами. РІ проіснувала 200 років, поки світ змінювався не
так швидко, СРСР - 70, не складно зрозуміти динаміку і усвідомити, що тепер
чекає на імперіалістичну, традиційно відсталу РФ в осяжному майбутньому.

\iusr{Татьяна Владимировна Ващенко}

\textbf{Макс Бужанский}, на самом деле, у них адское чувство вины. Все они
понимают. И тут либо принять позор, либо искать повод для оправдания. Отсюда и
вот это про «нужно было выбрать среди тиранов, в Сталина я не люблю, про него
Путин говорил». Неприятно, что вместо того, чтобы дать им возможность принять
позорные страницы своего рода, их используют как инструмент для раздора в
стране. Вопрос вины и принятия это не решает. Иначе тема бы давно заглохла.

\iusr{Татьяна Владимировна Ващенко}
\textbf{Крохин Дмитрий}, какие 20 миллионов? Откуда? Всего было около 600 тысяч приговоров. За все годы. Откуда 20 миллионов? Вы себе эту цифру представляете?

\iusr{Людмила Антонишина}
Макс Бужанский Макс! На кого вы тратите время и эмоции? Я прямо вам удивляюсь!

\iusr{Татьяна Владимировна Ващенко}

\textbf{Андрій Глухенький}, простите, не отказала себе в удовольствии сделать скрин с Вашей страницы. Так Вы определитесь-Вы за кровавый режим или против?)))
А то в одном посте крестик, в другом-трусы надо снимать...

\ifcmt
  pic https://scontent-lga3-2.xx.fbcdn.net/v/t1.6435-9/205474815_4265531890171292_2603170889512040347_n.jpg?_nc_cat=111&ccb=1-3&_nc_sid=dbeb18&_nc_ohc=mS2q6TJxsm8AX-XVdGg&tn=ntrKbsW_7ChXu3v-&_nc_ht=scontent-lga3-2.xx&oh=0331202a53bd79c12a8b614c118d566e&oe=60D8E0AC
  width 0.2
\fi

\iusr{Валера Фрейд}

\textbf{Андрій Глухенький}. Правильно. Была агрессия. Союз анексиравал
Галычыну. Отдайте её в зад Польше. Вы против пакта Молотова? Наверное Вы правы.
А Ьандера ,Шухевич и прочая нечисть, воевали против Украины потому, что они
были подданными Австро Венгрии. Потом Польши.

\iusr{Andriy Hluhenkiy}
\textbf{Татьяна Владимировна Ващенко} не думав, що це складно, але поясню. Я за
невідворотність покарання за злочин, що веде до розуміння відповідальності, з
чим в нас поки очевидні проблеми. І проти канібалістичних режимів. Спробуйте
зрозуміти різницю.

\iusr{Татьяна Владимировна Ващенко}
\textbf{Андрій Глухенький}, да никакой разницы))) Вы думаете, хоть одна из
диктатур не за справедливость? Серьезно? Просто справедливость-это
номинализация, в ней может быть очень много смыслов.  Верну вопрос-поняли
ответ, любитель крови?

\iusr{Andriy Hluhenkiy}
\textbf{Валера Фрейд} по реваншизму й ревізіонізму з далеку видно експертів з
історії. Поцікавтесь операцією Вісла й іншими процесами, що слідували за вище
згаданими роками. Що ніяк не заважає називати події 1939 року своїми іменами,
тобто військову агресію - військовою агресією.

\iusr{Валера Фрейд}
\textbf{Андрій Глухенький} Так и я о том же. Была агрессия против Польши,
забрали Галицию и посадили на шею Украине. Так отдайте на зад.  Двойные
стандарты.

\iusr{Константин Исиков}

\textbf{Андрій Глухенький} нет никакого немецкого мифа, это и есть миф о
немецком мифе, на самом деле это историографический факт закреплённый даже в
энциклопедии Британика, американский национальной энциклопедии и пр.
общепризнанных авторитетных источников. Только невигласы рассказывает про
\enquote{немецкий миф}

Обьединение славянских народов неизбежно перед лицом очередной кластеризации
мировой экономики. Киевская Русь исчезла потом возродилась в Русской империи,
империя развалилась и возрадилась ещё в более могущественном СССР. Следующее
возрождение также неизбежно как и исчезновение хуторянской недоукраины

\iusr{Константин Исиков}
\textbf{Татьяна Владимировна Ващенко} шикарный скрин!
А потом скулят, а нас то за шо?!!!

\iusr{Татьяна Владимировна Ващенко}
\textbf{Валера Фрейд}, мне больше всего нравится, когда вот такие Андрій Глухенький и слепенькие раздают советы, чем поинтересоваться. Сами почему не интересуются? Кто отнял право пользоваться библиотекой?)))

\iusr{Татьяна Владимировна Ващенко}
\urlFriend{https://www.facebook.com/profile.php?id=100001435692017}
\textbf{Константин Исиков}, да они только об одном жалеют-о невозможности топить в крови.

\iusr{Andriy Hluhenkiy}
\textbf{Валера Фрейд} Зроблю невеличкий лікнеп: територіальний спір між
українцями й поляками за ці землі існував значно раніше. В результаті агресії
СРСР отримав ці землі, потім, в результаті розвалу совка, вони на законних
підставах відійшли Українській державі. Та й навіть з точки зору міжнародного
права такий ревізіонізм зараз не можливий. Інакше буде, як у РФ з Кримом.
Очевидно ж, що для неї історія з цією окупацією ще далеко не завершена.

\iusr{Andriy Hluhenkiy}

\textbf{Татьяна Владимировна Ващенко} так в Україні, на відміну від РФ всі радянські
архіви відкриті. Йдіть і читайте. Це Кремлю вочевидь є що приховувати, якщо при
своїх голосних пропагандистських інтерпретаціях вони не хочуть знімати
секретність з багатьох паперів скоро столітньої давнини. Не здогадуєтесь чому?
😉

\iusr{Виктор Медведчиков}

\textbf{Андрій Глухенький} кроме смеха и отвращения подобный ваш спич не вызывает других эмоций.

\iusr{Константин Исиков}

\textbf{Андрій Глухенький} вы опять манипулируете, если бы ОУН- овцы
(нациствующие фашисты ) не терроризировали местное польское население, никто бы
операцию Висла не проводил. Вина за это расселение украинцев и поляков лежит на
украинских националистов (фашистов )

\iusr{Татьяна Владимировна Ващенко}
\textbf{Андрій Глухенький}, и что из архивов Вы лично читали? И где?

\iusr{Andriy Hluhenkiy}

\textbf{Виктор Медведчиков} ваші емоції - виключно ваші проблеми. Випийте
заспокійливе, зробіть фізичні вправи ) Але поясніть, чому влада РФ не хоче
показувати людям документи, щоб розкрити повну, а не вибіркову історію того
періоду?

\iusr{Валера Фрейд}
\textbf{Андрій Глухенький} А ты поинтересуйся львовскими погромами 30.06.1941.
Это были не немцы,это была банда в немецкой форме под командованием
гауптштурмфюрер войск SS, Шухевича. Будь проклято Вс- Вышним имя его и имена
последователей.

\iusr{Andriy Hluhenkiy}

\textbf{Валера Фрейд} помітьте, я ще ні слова не сказав тут на захист націоналістів.
Просто уточнив щодо підлеглих вусатого кремлівського бандита) Тому до чого тут
аргументи в стилі \enquote{а у ніх нгров лінчуют}?

\iusr{Виктор Медведчиков}
\textbf{Андрій Глухенький} у меня проблем нет. Для меня как раз все ясно. И
дела мне до власти РФ нет, когда я живу в Украине, в которой героизируют
нацистских прихвостней и хоронят их с почестями, а ветераны ВОВ умирают в
нищете в построенной ими из руин стране. Преступники преступниками и останутся.
Уже весь мир говорит, что в Украине процветает нацизм при поддержке власти, а у
нас зигующих ублюдков в упор не видят. Не о чем мне с такими как вы говорить.
Мировоззрение у нас совершенно разное.

\iusr{Andriy Hluhenkiy}

Виктор Медведчиков ви забули уточнити що це весь русскій мір гаваріт, і деякі
підгодовані їм ліваки на Заході. \enquote{Не читайте радянських газет} )

\iusr{Виктор Медведчиков}

\textbf{Андрій Глухенький} да я сам разберусь, что я читаю и смотрю. Без ваших
рекомендаций. Искажение языка вызывает ещё большее отвращение к вам. Режет
русский язык ухо, да? Немецкий милее душе? Вопрос риторический. Издевательство
над русским языком с претензией на уважение к украинскому выглядит глупо.

\iusr{Andriy Hluhenkiy}
\textbf{Виктор Медведчиков} ага, ще шнапс полюбляю, шванкі співаю, губна гармоніка, пиво й білявки подобаються. Що ви верзете? )

\iusr{Татьяна Владимировна Ващенко}

\textbf{Андрій Глухенький}, про львовские погромы говорит только Кремль? Я не
знаю, почему Вы только Кремль слушаете, но про эти погромы в Гугле на многих
языках мира, не только русском.

\iusr{Сергей Орёл}

\textbf{Андрій Глухенький} только почему-то тех, кто воевал на одной стороне с
энкаведистами мировое сообщество не признало военными преступниками и не
судили.. выходит с сообществом что-то не так?

\iusr{Виктор Медведчиков}

\textbf{Андрій Глухенький} даже не сомневаюсь. Я пишу, а не то, что вы
сочинили. Посмеялся вдоволь. Уж сколько застенчивых нациков на просторах ФБ
повидал, которые не зигуют, а от солнца глаза закрывают, не тату с \enquote{мёртвой
головой}, а любители пиратов, и непременно втайне томик произведения
австрийского художника под подушкой, что со всех смеяться здоровья не хватит.
То же самое и со сказочниками, вещающими о войне, развязанной двумя режимами.
Спустя столько лет винить во всех своих бедах РФ и все время на кремль кивать -
чушь несусветная. На сегодня достаточно. Да и на будущее, пожалуй, тоже.

\iusr{Andriy Hluhenkiy}

\textbf{Сергей Орёл} так в результаті ж Союз опинився по інший бік барикад,
заплатив десятками мільйонів життів за спільну перемогу й зайняв Східну Європу.
Хто ж судитиме союзників, яким активно допомогали воювати? Але окупацію СРСРом
країн Балтії, до прикладу, ніхто так і не визнав. Однак, ви праві, відсутність
нюрнберга над радянським режимом вже дає свої негативні результати - окупація
Кремлем частин Молдови, Грузії й України. Ситуація явно ще не на фінальній
стадії, і з сьогоднішніми гітлерчиками швидше буде як з Мілошевичем та Ратко
Младичем. Хоч не виключений, звісно і варіант Каддафі чи Чаушеску.

\iusr{Elena Grechishkina}

\textbf{Андрій Глухенький} блин, не всех, я гляжу расстреляли и скатували.

\iusr{Сергей Орёл}

\textbf{Андрій Глухенький} ничего не будет, так же как и не было с руководством
СССР. Но вы конечно можете думать как вам угодно и удобно. Никто не против
плюрализма... почти никто

\iusr{Andriy Hluhenkiy}
\textbf{Сергей Орёл} ну, для початку, варіант, як з розвалом СРСР теж може підійти. А там побачимо)

\iusr{Сергей Орёл}
\textbf{Андрій Глухенький} сначала Украину сохраните в целости, потом ждите что другие разваляться)))

\iusr{Andriy Hluhenkiy}
\textbf{Сергей Орёл} неодмінно. Більш того, очевидно, що це напряму
взаємопов'язані процеси - цілісність України й дефрагментація недорозваленої
імперії.

\iusr{Сергей Орёл}
\textbf{Андрій Глухенький} пока недорозваленная империя развалится, от её осколка уже ничего не останется. Что не успеют разворовать и продать, само на кусочки рампадется. К сожалению

\iusr{Михаил Шадхан}
\textbf{Валера Фрейд} и без формы тоже. В костюмах на вид - интелигенция.

\iusr{Юрий Юрьев}
\textbf{Андрій Глухенький} перекрашенный прихвостень нацистов

\iusr{Andriy Hluhenkiy}

\textbf{Юрий Юрьев} Як прокинетесь, починайте читати книжки, обов'язково різні.
Світ стане складнішим, але й цікавішим) Двоколірний світогляд обкрадає життя

\iusr{Юрий Юрьев}

\textbf{Андрій Глухенький} читал перечитал уже сотни книг и белых и черных но
никак не могу понять с каких пор сс в украине \enquote{херои}, каратели прислужники
фашистов \enquote{херои}, а те кто освободил нашу страну от нацистов стали оккупантами.

\iusr{Andriy Hluhenkiy}

\textbf{Юрий Юрьев} а ви не за кольором обирайте) Щодо згаданих дивізій, як я
розумію, вафен СС, то зараз не складно знайти і критичні, і нейтральні, й
позитивні книжки про них. Після прочитання проаналізувати наведені там
документи й аргументи та зробити більш зважений висновок щодо цих військових
з'єднань. А не лише на старій радянській міфології й фейсбучних дописах. До
якого висновку прийдете - ваша власна справа. Бо принцип вибірковості джерел за
ідеологічним критерієм і ставлення до решти типу \enquote{не читал, но осуждаю}, м'яко
кажучи, не продуктивний.

\iusr{Andriy Hluhenkiy}
\textbf{Татьяна Владимировна Ващенко} ок, уточню, на відміну від совєтів та
нацистів, мова йде про рівну відповідальність, яка вимірюється масштабом
порушення, а не вибіркову стигматизацію за класовими, чи етнічними ознаками. І
точно не владно-майновий ценз з кастою недоторканних, як в одного
північно-східного сусіда.

\iusr{Татьяна Владимировна Ващенко}
\textbf{Андрій Глухенький}, Вы прям необольшевик))) Троцкизмом увлекаетесь,
судя по речам? Вы каким мерилом это мерять будете? Какими стандартами,
категориями? Вот эту шляпу аккурат все диктатуры и продвигали, в курсе?Скажите,
Вы что-то о социологии, обществоведении слышали? Об истории не спрашиваю,
прочла комментарии, и так видно-изучали по бульварной прессе и высерам
блохеров.  А чего Вы в каждом комменте соседнюю страну упоминаете? Они Вам за
рекламу доплачивают или это обязательное условие трудоустройства?)

\iusr{Татьяна Владимировна Ващенко}
\textbf{Андрій Глухенький}, а ещё очень смешно читать про то, как Вы СССР и
Третий рейх объединили «социалистическими государствами». Вы реально, на
взрослой волне, не знаете разницу между этими двумя странами, идеологией? Или
для Вас если диктатор-то все, это все одно и то же? Я не верю, что Вы настолько
безграмотны. Даже поклонник бульварной литературы и фейколюб такие элементарные
вещи знает. Или я просто о Вас лучшего мнения и Вы все-таки не знаете? Гугл же
есть.

\iusr{Oksana Perova}
\textbf{Андрій Глухенький}, а от чего же Ваши предки в червонной армии воевали,
а не присоединились с оружием в руках к \enquote{освободительной армии УОН УПА}????
Может от того, что они то как раз видели различия между коммунистами и
фашистами?

\iusr{Константин Исиков}
\textbf{Андрій Глухенький} касательно какой-то там военной агрессии в 1939 г
так её не зафиксировала ни одна страна Даже Лига наций промолчала, более того,
нет приказа польского командования об отражении советский агрессии. В немецкой
Вики написано что СССР вернул себе свои исторические ранее утраченные
территори.

\iusr{Vladimir Sklyarov}
\textbf{Дмитрий Крохин} появился на свет от непорочного зачатия.

\iusr{Александр Молчанов}
\textbf{Андрій Глухенький} можете рассказать историю, о именно ваших (близких родственниках) предках, кто пострадал, если да, то за что?

\iusr{Константин Исиков}

\textbf{Андрій Глухенький} У вас какая-то пропагандистская каша в голове. В
одном месте вы пишите, что демократия это хорошо, а а диктатура плохо, при этом
у себя на страничке вы пишите о том что Майдан не должен быть гуманным, при
том, что Майдан это не демократия а охлократия - власть толпы, диктатура толпы.

Но вы даже не понимаете, что демократия хороша в мирное время, а в годы
испытаний нужна диктатура или тоталитаризм (тот же Ковид продемонстрировал )
причём диктатуры бывают разные, к примеру диктатура Пак Чон Хи сотворила то
самое "Южнокорейское чудо" за 20 лет переделав отсталую феодальную аграрную
страну в передовую технологическую. Вы предлагаете мир рассматривать не как
чёрно-белый, а с разных точек зрения, при этом только там где вам выгодно, как
в случае с СС Галичина, вы предлагаете рассматривать с разных сторон, но в
случае с СССР вы категорично видите его в радикально чёрном свете, не смотря на
то, существует бесконечное число книг, научных статей, воспоминаний
современников и историков о высокой роли СССР в мировом развитии. Но вы как
муха предпочитаете видеть там только гавно.

Продолжая грязные пропагандистские манипуляции вы пишите что СССР и Германия
социалистические страны совершенно игнорируя тот элементарный факт, что
социализм по определению это общественная собственность на средства
производства.

Но в Германии была и частная собственность на средства производства, и богатые
промышленники и олигархи. Сам Гитлер писал, что \enquote{Мы бы могли называться
либерально-демократической партией, но назвались социалистической }. В самой же
НСДАП рабочих было около 30\%.

Ничего социалистического кроме названия партии в Германии не было.

Т.е. вы во всех этих вопросах откровенно врёте и манипулируете.

\iusr{Mikha Staryj}
\textbf{Андрій Глухенький}, чим молодший українець , тим тяжче йому жилося в часи голодомору і сталінських репресій ...

\iusr{Александр Диденко}
\textbf{Андрій Глухенький} как-то странно они освобождали нашу страну! Убивая настоящих освободите лей!

\iusr{Andriy Hluhenkiy}
\textbf{Mikha Staryj} а мені пощастило - в мене ще живі родичі, які вціліли в
Голодомор. Тому з перших рук ще в дитинстві й дізнавався від рідних про
Голодомор, колективізацію, репресії, війну, окупацію, остарбайтерів і
сіросвитників. Тож маю уявлення про ту державу не лише з від дитячих
пропагандистських брошурок 80 років типу \enquote{Ленін і пєчнік}.

\iusr{Юрий Юрьев}

\textbf{Андрій Глухенький} но увы надо проводить новый нюрнбергский трибунал
который признал все СС преступным. Теперь надо провести оправдательное
заседание. И признать что сс убивало не со злостью а с добротой и улыбкой. А
это кардинально меняет все.

\iusr{Andriy Hluhenkiy}

\textbf{Константин Исиков} думаючим нагадаю, що вони спочатку й Голодомор
\enquote{не помічали}, бо так було зручно. Але з рештою як так вийшло, що війни
з Польщею не було, а десятки тисяч військовополонених поляків були? Яких потім
розстріляли. 

Найвідоміша зараз історія Катині. Документ, який частково пояснює
мовчання заходу, приведу ще тут

\url{https://web.archive.org/web/20070927002738/http://katyn.codis.ru/shelepin.htm}

Оригінал

\url{https://web.archive.org/web/20070927002738/http://katyn.codis.ru/shelepin.htm}

А тут записка Берії
Сталіну щодо таборів військово полонених і пропозиції вищої міри для їх
мешканців 

\url{https://web.archive.org/web/20070927002738/http://katyn.codis.ru/shelepin.htm}

\iusr{Andriy Hluhenkiy}

\textbf{Юрий Юрьев} взагалі то Нюрнберг осудив \enquote{групи осіб} в організаціях, а не
організації загалом. Так як юристам демократичних країн не близька норма
колективної відповідальності. Але радянська пропаганда, як завжди, дещо
підкорегувала результати процесу для власних громадян. Трошки спрощений варіант
\url{https://www.istpravda.com.ua/articles/2010/11/5/2083/}

\iusr{Константин Исиков}
\textbf{Андрій Глухенький} опять это бред про голодомор ... Э Уже надоело это
бред опровергать.  Был голод, который наложился на коллективизацию и саботаж
крестьян

\iusr{Валентин Станев}

\textbf{Андрій Глухенький} ну мы в этот омут входим, а что мы с этого будем
иметь ? В про СССР что там все враги и т.д а у США доктрина была и есть СССР и
Россия враг. Хотя это им не мешает иметь русские кварталы в стране, читать
русскую классику и снимать фильмы по ней, и как Байдон говорит жевать жуйку. 

Я учился ещё при СССР и у нас мова и язык по часам имели знак равенства и это
было круто я так думаю. Сейчас дети могут в начальных классах даже читать на
русском языке Пушкина Тургенева Лермонтова и т.д и мне это очень не нравится.

Но Путин мудак который влез в чужой монастырь и отобрал Крым. Просто Андрей
нужно отличать где чёрное, а где белое, где Королев, а где Берия и т.д.

\iusr{Aleksandr Chervonny}
\textbf{Андрій Глухенький} 89 лет с момента голодомора прошло. Сколько лет
Вашим родственникам которые помнят? И ещё вопрос. Какую модель Украины
предлагаете Вы. Спасибо.

\iusr{Vitalua Lysenko}
\textbf{Андрій Глухенький} мало вас катували!

\iusr{Andriy Hluhenkiy}

\textbf{Aleksandr Chervonny} 96 років. Тобто, в 33 було 8. Правова,
громадянська держава. Мені близький політичний націоналізм (без шовіністичних
проявів) з ліберальною ринковою економікою.

\iusr{Artem Bodriy}
\textbf{Крохин Дмитрий} миллиарды убивали! Странно как столько катрюль уцелело?
Посмотрите статистику численности населения СССР.

\iusr{Artem Bodriy}
\textbf{Крохин Дмитрий} птенцы Солженицына!
Да вы историю по голливудским фильмам только и знаете.

\iusr{Andriy Hluhenkiy}
\textbf{Artem Bodriy} статистика тоді теж була політичною. Приклад
\url{https://ru.m.wikipedia.org/wiki/%D0%9F%D0%B5%D1%80%D0%B5%D0%BF%D0%B8%D1%81%D1%8C_%D0%BD%D0%B0%D1%81%D0%B5%D0%BB%D0%B5%D0%BD%D0%B8%D1%8F_%D0%A1%D0%A1%D0%A1%D0%A0_(1937)}

\iusr{Artem Bodriy}
\textbf{Андрій Глухенький} вот так аргумент. А кто автор статьи этой?
Посольство Штатов?

\iusr{Елена Родышевцева}
Хороший вопрос.

\iusr{Пелюшенко Ярослав}

\enquote{напали на нашу страну} - а нагадайте якісь хоч частково вільні вибори у ті
часи, по яким можна було б вважати, що та система влади/курсу країни/ідеології
/ прав/ хоч чогось, з чого можна зробити висновок, що це була саме наша країн.
Громадянство тих чи інших діячів мало про що свідчить

\begin{itemize}
\iusr{Макс Бужанский}
\textbf{Пелюшенко Ярослав}, а с каких пор свободные выборы и прочие признаки демократии стали показателем чего то?)))

\iusr{Пелюшенко Ярослав}
\textbf{Макс Бужанский} Ок, тоді на чому базуються слова \enquote{наша країна}?
Географічне місце перебування людей, думки яких нікого не цікавили, з
мінімальним рівнем свободи, можливістю щось змінити, зрештою, можливість просто
висловлювати свою думку? Це швидше географічна батьківщина

\iusr{Макс Бужанский}
\textbf{Пелюшенко Ярослав}, ну для вас географическая, для меня нет.
Если знаете ещё примеры, когда люди за \enquote{географическую} страну, а не за свою,
ложатся под танк, подрывают себя гранатой, идут на таран, значит вы знаете
больше меня, я таких не знаю.

\iusr{Пелюшенко Ярослав}

\textbf{Макс Бужанский} Та загалом, багато прикладів найманців і фанатиків,
ІДІЛ це взагалі на звичне діле перетворив, і питання не про смерть, та і не про
індивідуальність. Питання більш глобальне - чому вважати СРСР \enquote{нашою страною},
: ми її обрали? ми мали змогу вільно щось змінити? мали право вільно міняти той
чи інший режим? мали право вільної політичної діяльності?мали вплив на кадрову
політику ( не окремі люди, а шляхом виборів). І таких нюансів багато. Якщо ні,
тоді у чому саме ( особливо у політико-правовому плані) виражається ця нашість?
Я розумію індивідуальність, спогади, сімейні історії, традиції, память
поколінь, тут все зрозуміло, чого наша, нічого дивного.

\iusr{Станислав Стасевский}

\textbf{Макс Бужанский} блестяще!

\iusr{Oleg Raznochenko} 

\textbf{Пелюшенко Ярослав} все таки табаки повылазило.
А то я думаю что так тихо. Даже зе выступил флаги висят перекрасился наверное.
А табаки отвечу так. За географичную местность как вы говорите. Сколько было
заявлений в военкоматы. Сколько дивизий народного ополчения. Подвиг 300 батареи
форта Сталина. Немцы очень удивлялись что там оказалось всего несколько
коммунистов. Остальные беспарттйные и сражались. До последнего снаряда. . У вас
хоть какая то совесть есть они сражались за вас.

\iusr{Макс Бужанский}
\textbf{Пелюшенко Ярослав}, вы это называете фанатиками, а весь мир воспринял как патриотизм и героизм.
Да считайте, чем хотите, мне то что)

\iusr{Пелюшенко Ярослав}
\textbf{Макс Бужанский} 

Простіше кажучи, ніякої відповіді, крім відсилки до смертей та емоцій у вас
немає. Хоч якогось, хоч натягнутого розуміння, у чому нашість,, крім спільних
могил, виховання, агітації, історії.

Скажіть, а гибель на фронтах першої світової, та і загалом усіх колоніальних
війнах, індійських та інших солдат Британії, часом теж героїчна робить Британію
для них \enquote{нашою державою}? Просто ніде не зустрічав, у кращому випадку - вони
воювали за Корону чи Імперію, у складі якої ми були.

\iusr{Макс Бужанский}
\textbf{Пелюшенко Ярослав}, вы путаете собственную страну и страну колониальную, как Индия, бывшая Британской колонией.

\iusr{Пелюшенко Ярослав}
\textbf{Макс Бужанский} Хочете сказати, що житель СРСР, особливо 30-40 років, мав більше прав і свобод, ніж індус в ті ж часи?

\iusr{Алексей Заровский}
\textbf{Макс Бужанский} юноша очень обиделся за Нахтигаль, но прямо об этом говорить стесняется. Поэтому несёт всякий бессвязный бред, про свободные выборы, Первую мировую, Индию, Британию и, почему-то, Игил.

\iusr{Александр Шибанов}
\textbf{Алексей Заровский} Ну,а где ему ещё вежливо и доходчиво объяснят?А тут целый Бужанский дискутирует.Хотя юноша уже потерян!

\iusr{Макс Бужанский}
\textbf{Пелюшенко Ярослав}, ну как найдёте индуса-академика, или индуса- командующего британской армией, сразу дайте знать.

\iusr{Алексей Заровский}
\textbf{Александр Шибанов} ангельское терпение у господина Бужанского, мне бы такое.

\iusr{Александр Шибанов}
\textbf{Алексей Заровский} Да,а мог бы ударить)))

\iusr{Vitaliy Pantin}
\textbf{Пелюшенко Ярослав} зараз дуже кого цікавлять? Пустлбрехам зараз кругом шлях широкий

\iusr{Александр Шибанов}
\textbf{Макс Бужанский} А так же доверить крупный ядерный потенциал!

\iusr{Evgeniy Akunin}

\textbf{Пелюшенко Ярослав} это образование украинцев начиная с села, школы,
институты. При Украине не построен не один институт.Это лучшая бесплатная
медицина абсолютно всем. Сейчас вы заболеете и будете по ТВ собирать деньги на
лечение. Бесплатное жилье для семей. Сейчас вы никогда себе не заработаете на
жилье. 8 часовой рабочий день, 24 дня отпуска. Уменьшение пенсионного возраста
и стажа. Сейчас вы до пенсии не доживёте. И вы можете выехать на заработки на
польские поля😂

У вас сейчас есть выбор? За какого из слуг олигархов голосовать? У вас свобода
слова? Да, на кухне, повозмущаетесь очередным ростом налогов и молча пойдете
платить олигарху. Иначе он выкинет вас из квартиры, которая досталась вам от
СССР и заблокирует вам счета

\iusr{Aleksandr Chervonny}
\textbf{Пелюшенко Ярослав} Сравнить мальчишек в 41 приписывавших себе года для отправки на фронт с фанатиками ИГИЛ.

\end{itemize}

\iusr{Валентин Станев}

Тот хто забивал ребенка прикладом, стрелял в мирных жителей, вещал по классовой
ненависти, не может быть героем!!! кто говорит что они боролись с НКВД таким
образом такие же самые. Скажете брехня так факты на плёнках ни чем не
исправить.

\iusr{Sergey Gres}

Макс.. такими постами в ФБ вы страну не сошьете. Уже 17 лет негодяи занимаются разделением.

\iusr{Tatiana Kornilova}
\textbf{Sergey Gres} А как можно сшить правду с неправдой?

\iusr{Alex Naz}
\textbf{Sergey Gres} не буду с нациками сшиваться

\iusr{Vitalua Lysenko}
\textbf{Sergey Gres} а сшивать ничего и не надо, вернуть все до 39 года, нах!

\iusr{Лина Сахацкая}
Спасибо, что помните!

\iusr{Олег Резник}

\ifcmt
  pic https://scontent-lga3-2.xx.fbcdn.net/v/t1.6435-9/204018663_963482244439784_1146137958497162994_n.jpg?_nc_cat=111&ccb=1-3&_nc_sid=dbeb18&_nc_ohc=5R6-iegLe-UAX9Xtw55&tn=ntrKbsW_7ChXu3v-&_nc_ht=scontent-lga3-2.xx&oh=b191753158ecb2bb0829837c02d850e4&oe=60D842B9
  width 0.2
\fi

\iusr{Виталий Гаркуша}
Спасибо, говорить нужно и только вслух. Молчать, значит быть соучастником неонацистов сегодня.

\iusr{Наталья Сажнева}

\ifcmt
  pic https://scontent-lga3-2.xx.fbcdn.net/v/t1.6435-9/202625144_3053395091583605_1060183271890832836_n.jpg?_nc_cat=100&ccb=1-3&_nc_sid=dbeb18&_nc_ohc=syg45jALJ8MAX_ydPUq&tn=ntrKbsW_7ChXu3v-&_nc_ht=scontent-lga3-2.xx&oh=f8825a5e750a2e11725011a91b3edc3c&oe=60D757DD
  width 0.2
\fi

\iusr{Владимир Снисарь}
Михальчишина спросите на очередных дебатах кем считать пришедших в рядах врага.

\iusr{Mikhail Potapovich}

\enquote{Кем считать людей, пришедших в рядах врага, в форме врага, с оружием врага и под командованием врага}

Конечно, участниками \enquote{визвольних змагань} 🙂

Вы забываете, что это были враги для украинцев, а не для бандеро-галичан, которые ныне управляют Украиной.

А для них... У них даже лозунги вс оттуда. \enquote{Дойчланд юбер аллес} -
\enquote{Україна понад усе}. \enquote{Хайль Гитлер -- Зиг хайль} -
\enquote{Слава Українi -- Героям слава}...

\iusr{Elena Poylisher}
\textbf{Макс Бужанский}, Вы писали, что собираетесь узнать, кто дал
распоряжение хоронить этого эсэсовца с президентским полком. Сообщите
пожалуйста о результатах.

\begin{itemize}
\iusr{Макс Бужанский}
\textbf{Elena Poylisher}, ещё нет результатов, разослал запросы, жду ответы.

\iusr{Elena Poylisher}
\textbf{Макс Бужанский} спасибо

\iusr{Иван Божков}
\textbf{Макс Бужанский} ждем с интересом. Большим.

\iusr{Алексей Шимановский}
\textbf{Иван Божков} соглашусь с вами, таких людей нужно знать и помнить. Чтобы при случае воздать должное.
\end{itemize}

\iusr{Владимир Михайловский}
Не стоит удивляться, все очень логично проистекает из продекларированного принципа \enquote{какая разница}.

\iusr{Vladimir Tatarnikov}
Нет ощущения!! Что есть ощущение президента.

\iusr{Roman Gorgiy}
Жёсто, но чертовски верно. Совсем оскотинели некоторые либерасты от желания служить двум богам...

\iusr{Eugene Tsypin}

\obeycr
От дома отчего - стена
солдату русскому осталась,
пришел назад, а тут весна,
сирень живою оказалась
и нежным цветом расцвела
над обстановкою плачевной,
и над разрушенной деревней
вечерняя сгущалась мгла.
И уголь взял солдат, и вот
нарисовал окно, комод,
и угол, и в углу иконы,
и кошку - на стене беленой,
и на шинель под нею лег
на левый недырявый бок.
И снилась русскому солдату
вся в белом молодая мать,
ее, забитую прикладом,
он не сумел нарисовать.
Д. Мельников.
\restorecr

\iusr{Petr Samoylov}

Не хочу никого обидеть, но и похороны и памятные мероприятия очень легко
совмещаются: по-научному это называется \enquote{выученная беспомощность}. Т.е., когда
люди выполняют даже не приказ, а любое пожелание своих \enquote{вождей}. Конрад
Лоренц, Нобелевская премия аж за 1973 год за изучение поведения. И никакой
логики тут нет, одна биология-матушка.

\iusr{Дмитрий Голиков}

Помнится, года три назад, в Киеве, какая то малолетняя соска в майке с
эмблемами Нахтигаль бдила 9-го мая что бы никто советскую символику не достал
🤦

\iusr{Михаил Варшавский}
Макс ты четко очертил грань ,решать самим ,что тут решать нацистская нечисть не достойна памяти ...

\iusr{Игорь Игорь}
Дебелизм крепчает

\iusr{Роман Романовский}
Вот эти похороны и есть одно.

\iusr{Людмила Резнікова}
\enquote{І будуть люди..} Український телесеріал.)

\iusr{Вячеслав Якушев}
Браво! Спасибо Макс и уважуха.

\iusr{Нина Абросимова}

Юрий Бутусов, блогер помойки \enquote{Цензор нет}, где одни порохоботы и дякулы, с
восторгом все это освещал, устроил целую фотосессию. Радует, что Фейсбук его за
это на неделю забанил.

\iusr{Елена Щербина}

Только что посмотрела Ваш прямой эфир(по Скайпу). Спасибо за исторические
факты, которые вы пытались донести до историка Черновола, но диалога не
получилось, прямо как в анекдоте - \enquote{шифровка до Штирлица не дошла}

\iusr{Тамара Камышан}

\textbf{Елена Щербина} Я смотрела!!! Браво, Максим!

\iusr{Владимир Иванеев}

Корректнее было бы не обсуждать отдельные личности с их судьбами, а идеологию
государства Германия и государства СССР на тот момент да и в предшествующие
войне годы. А судьбы у людей разные и их миллионы.

\begin{itemize}
\iusr{Виталий Рублев}
\textbf{Владимир Иванеев} В чем проблема? Вы у себя,корректно рассуждайте,об идеологии,кто вам мешает?

\iusr{Владимир Иванеев}
\textbf{Виталий Рублев}, а Вам, смотрю, не хочется. Ну да, тема скользкая.
Проще, Гришка хороший, а Ванька плохой. Хотя, если Гришка служил в
заградотряде..... Но об этом здесь нельзя.

\iusr{Виталий Рублев}
\textbf{Владимир Иванеев} Ну да,ну да,в СС»Галичина»,или РОА,это другое дело.

\iusr{Владимир Иванеев}

Виталий Рублев, вопрос на засыпку. Живёт некий Ваня на Западной Украине, никого
не трогает. Приходят туда большевички по сговору с Алоизовичем. Насилуют и
убивают Ванину жену и дочь-подростка, хату экспроприируют, хозяйство. А тут
война, подлый Алоизович кинул друзей-большевичков. Вы бы на месте Вани на чьей
стороне воевать пошли? Это я к тому, что судьбы у людей бывают разные. Так на
чьей?

\iusr{Oksana Perova}
\textbf{Владимир Иванеев}, ну, и до кучи ещё соседям своим евреям, полякам на
пути своём героическом раздаёт. У него же травма психоологическая. Ему можно.
Это красной армии на оккупированной территории нельзя мстить мирному населению.
А героям можно. У них трагедия. Личная такая судьба.

\iusr{Константин Андрюхин}

\textbf{Владимир Иванеев} Да отчего же нельзя про загрядотряды? Извольте:
\url{https://militaryexp.com/zagradotryady-vermahta-osobennosti-raboty-polevyh-zhandarmov}

\iusr{Владимир Иванеев}
\textbf{Оксана Перова}, а Вы на чьей стороне пошли бы воевать? На стороне
захватчиков-нацистов, для которых вы человек низшего сорта, или на стороне
захватчиков-убийц своей семьи?

\iusr{Владимир Иванеев}
\textbf{Константин Андрюхин}, про эти заградотряды здесь можно, про советские нельзя.

\iusr{Константин Андрюхин}
\textbf{Владимир Иванеев} вы не могли бы указать разницу между ними?

\iusr{Владимир Иванеев}
\textbf{Константин Андрюхин}, Вы мой коммент читали или сразу в бой? Он не про
тонкости трактования исторических фактов. Он о разных судьбах простых людей во
времена той страшной мясорубки, включая годы репрессий.

\iusr{Irina Posmetnaya}
\textbf{Владимир Иванеев} то есть у Вани поехала крыша? Он пошёл убивать народ
своей страны, детей, сжигать деревни своей страны, тясячами убивать соседей??
Ну такое объяснение и оправдание Вани 😏👈

\iusr{Владимир Иванеев}
\textbf{Irina Posmetnaya}, Вы в школу ходили? Читать научились? Ваня из
Западной Украины. Для него что Германия захватчик, что СССР. Вот только СССР
изнасиловал и убил его семью, а Германия пока ничего натворить не успела, так
как только вторглась. Теперь ставьте себя на место Вани и отвечайте на простой
вопрос. Хотя, кому я это пишу.....

\iusr{Irina Posmetnaya}

\obeycr
Владимир Иванеев в смысле \enquote{пока что ничего не успела}??????. Вы в курсе как нас упала Германия??? Вот эти приколы про школу оставьте при себе.
Вы считаете, что армия Вермахта шла и такая \enquote{ой, извините, му тут поговорить хотим} ?????
Армия Вермахта пришла вот так 👇👇👇и Ваня ВИДЕЛ И ПОНИМАЛ ЧТО ЭТО!!!
Вы СЛАБО понимаете что такое война.
Репрессии - это не война. Да, это было, но у Вани по всей видимости какая
разница о херня в голове была. Как и сейчас : нацисты Украины борются с мирным
населением, а с нацистами у власти - нет. Сейчас националисты это не НКВД???
Полностью одно и тоже!! Но почему-то многие считают это нормой. Так и тогда.
Ваня убивал мирное население и называть это борьбой против НКВД - это
шизофрения.
\restorecr

\iusr{Владимир Иванеев}
\textbf{Irina Posmetnaya}, шизофрения - это называть нацистом еврея, которого
остальные \enquote{нацисты} подавляющим большинством выбрали себе в президенты. Сложно
для Вас, наверное. Выражусь попроще. Будь мы здесь все нацистами, как же мы
могли еврея выбрать? Вы из России, что ли? По профилю непонятно. Недалёкая
какая-то.

\end{itemize}

\iusr{Роман Ходукин}

Говорят, Киев 22го не бомбили

\iusr{Vadik Kulish}
Кто забудет прошлую войну, получит новую. К этому все идет.

\begin{itemize}
\iusr{Nataliia Ivanska}
Спочатку навчіться розмовляти на українській мові, а потім обговорюйте її питання

\iusr{Олег Ермоленко}
... сказала та, у которой одни репосты на русском...

\iusr{Александр Шибанов}
\textbf{Nataliia Ivanska} Что б не веселить людей с утра глупостями надо почитать законы про употребление русского языка в быту и при общении))

\iusr{Александр Лукьяненко}
\textbf{Nataliia Ivanska} Это поменяет суть изложенного?

\iusr{Ілля Сандирєв}
\textbf{Nataliia Ivanska} правильно говорити не \enquote{на українській мові}, а \enquote{українською мовою}. \enquote{На українській мові} то русизм.

\iusr{Natalia Zadorozhnaya}
\textbf{Nataliia Ivanska}, с какой стати??? Я не в Тернополе прожила всю жизнь,
вокруг меня мои соотечественники говорили и говорят на русском! И? А вот для
страны они сделали много. Пока вылупки не подтянулись зигующие.

Эта страна в количестве 64\% её населения (официальная статистика) говорит на
русском. А это примерно (перепись не ведут осознанно - цифры удручающие будут)
примерно так это 27 миллионов человек.

Так понятнее, в цифрах, патриотка страны по языковому признаку?

\iusr{Nataliia Ivanska}

От поки так будуть люди на сході думати як Ви, то війна ніколи не закінчиться.
Ви, росіяни, не є українці, ви живете в українських хатах, власників яких
вивезли в сибір. Їдьте в свою Росію і там так думайте, але не на українській
землі.

\end{itemize}

\iusr{Sergey Ozhegov}
\urlFriend{https://www.facebook.com/sergey.ozhegov.775}
Страна когнитивного диссонанса

\iusr{Тамара Зажирило}
Плеваться хочеться!

\iusr{Владислав Чернецов}
\textbf{Тамара Зажирило}
БЛЕВАТЬ!!!

\end{itemize}

\end{itemize}
