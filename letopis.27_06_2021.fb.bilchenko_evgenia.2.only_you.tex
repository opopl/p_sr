% vim: keymap=russian-jcukenwin
%%beginhead 
 
%%file 27_06_2021.fb.bilchenko_evgenia.2.only_you
%%parent 27_06_2021
 
%%url https://www.facebook.com/yevzhik/posts/4013729205328826
 
%%author Бильченко, Евгения
%%author_id bilchenko_evgenia
%%author_url 
 
%%tags bilchenko_evgenia,poezia
%%title БЖ. Онли ю
 
%%endhead 
 
\subsection{БЖ. Онли ю}
\label{sec:27_06_2021.fb.bilchenko_evgenia.2.only_you}
\Purl{https://www.facebook.com/yevzhik/posts/4013729205328826}
\ifcmt
 author_begin
   author_id bilchenko_evgenia
 author_end
\fi

\begin{center}
\large
\textbf{БЖ. Онли ю.}
\end{center}

\begin{multicols}{2}
\obeycr
\noindent 
\lettrine[findent=2pt]{\textbf{Д}}{}ядя Толя был дворником на проспекте Победы, дом сорок восемь.
Он верил в Бога, а в черта не верил вовсе.
Молился он так: \enquote{Если Ты, Отец, - сплошное добро,
То нечистый собран в Твой мир, как мусор, - в моё ведро}.
\smallskip
А, значит, некого попрекать за житьё-бытьё,
За окурки, что мимо урны, и всё это ё моё.
Дядя Толя был дворником. Он не читал ни Жижека, ни Бадью.
Но он точно знал, что, если выпить пиваса и кваса бадью,
\smallskip
Можно чуть поднапрячься и спеть чёт типа \enquote{Only  you},
Но зачем ему онли-я, если есть семья, а семью свою
Дядя Толя любил из последних сил
И на кладбище к ней исправно цветы носил.
\smallskip
Дядя Толя был дворником, но убирался ещё и там,
Где стоит университет, - огромный такой вигвам,
Напичканный половиной дебильных кадров,
Но ваще дядя Толя был добрым и Божьей кары
\smallskip
Никому не желал, даже тем дуракам с медалями,
Которые в урны бычками не попадали.
Иногда дядя Толя просил у меня курить.
Курить или зажигалку: по настроению.
\smallskip
Странно, что у меня. Вот у доцента Карины
Он и снега зимой не спрашивал, еле-еле
Шевеля метлой, пока она проходила.
\restorecr
\end{multicols}

\ii{27_06_2021.fb.bilchenko_evgenia.2.only_you.pic}

\begin{multicols}{2}
\obeycr
\noindent 
\smallskip
За дядею Толей даже лоху ложь победившая сила
Видна была так явно, что люди обходили его заведомо.
А потом перестали пересекать даже проспект Победы.
Окружными петляли, плескались по Гавела, по бульвару, как караси:
Такой был у дяди Толи персональный, блин, Апокалипсис.
\smallskip
Пустынная улица стала чище, чем Дольче Вита:
Никакого трафика, никакого тебе ковида.
Дядя Толя был дворником. У него была даже рация -
Такая мобила на кнопках, огромная, старая: разобраться
В ней очень просто, братцы, как хиппану - в виниле.
\smallskip
Единственное что: дяде Толе почти уже не звонили.
Но на всякий случай он утром подзаряжался:
Телефон дряхлел, хотя бы к нему из жалости.
Разве что с кладбища иногда набирала его жена.
\smallskip
И он пел ей песню свою \enquote{Only you}, хлебнув до рожна \verb|го#вна|.
Онли-ю, онли-я, онли-мы, онли-все, онли-он,
Неон: он освещал, зачищал проспект, как киллер такой, Леон.
А потом дядя Толя умер и дождь убирает: Там у каждого каплет ум с лица -
На проспекте Победы, небо седьмое...
а народ вернулся назад на улицу.
\restorecr
\end{multicols}

27 июня 2021 г.
