% vim: keymap=russian-jcukenwin
%%beginhead 
 
%%file 06_10_2022.stz.news.ua.donbas24.1.spilka_arheologiv_ukr_zasudyla_pretenzii_rf.txt
%%parent 06_10_2022.stz.news.ua.donbas24.1.spilka_arheologiv_ukr_zasudyla_pretenzii_rf
 
%%url 
 
%%author_id 
%%date 
 
%%tags 
%%title 
 
%%endhead 

Ольга Демідко (Маріуполь)
06_10_2022.olga_demidko.donbas24.spilka_arheologiv_ukr_zasudyla_pretenzii_rf
Україна,Археологія,06_10_2022

Спілка археологів України засудила претензії рф на українську культурну
спадщину

Українські археологи закликали до ігнорування російської культури

ВГО «Спілка археологів України» засуджує претензії російської федерації на
українську культурну спадщину і закликає міжнародну спільноту долучитися до
ігнорування та стигматизації усієї російської культури.

«Від 2014 р. росія веде гібридну війну проти України й українства,
використовуючи зокрема й методи та інструменти гуманітарної агресії: руйнуючи,
знищуючи, розкрадаючи й привласнюючи об'єкти історико-культурної та природної
спадщини України», — наголосили археологи України.

Читайте також: Історія життя видатного археолога Володимира Кульбаки

Які саме дії засуджують археологи?

Представники ВГО «Спілка археологів України» наголошують, що компромісний
меседж від російських інтелектуалів про науку/культуру поза політикою є лише
прикриттям імперського (неоколоніального) менталітету. Після незаконної
окупації російською федерацією територій України представники російської
академічної науки систематично відвідують ці регіони. Зокрема, 2 жовтня 2022
року після оголошення путіним нікчемних указів про анексію чотирьох українських
регіонів, відбувся візит делегації російського історичного товариства (ріт) до
тимчасово окупованої частини Донецької та Запорізької областей. До складу
делегації увійшли голова правління ріт Костянтин Могилевський, член президії
ріт, віце-президент російської академії наук, директор інституту археології ран
Микола Макаров та голова ради ріт в «республіці Крим», директор центрального
музею тавриди Андрій Мальгін.

Читайте також: Кам’яні половецькі баби на Донеччині оберігають жителів від
бомбардувань (ФОТО)

Візит цих осіб мав на меті «інтегрувати» до російського культурного і наукового
простору тимчасово окуповані території України, створити відповідні
організаційні структури та залучити для цього колаборантів. Одним із напрямів
їхніх цинічних дій стало анонсування відновлення Маріупольского краєзнавчого
музею, безжалісно знищеного росіянами. До того ж російські археологи вже
голосно заявляють про свої права на культурну спадщину і пам’ятки України.
Зокрема, директор інституту археології ран Микола Макаров, який вже внесений до
переліку фізичних осіб, що причетні до незаконних археологічних розкопок,
досліджень об'єктів культурної спадщини, самовільного розкопування військових
поховань на тимчасово окупованій території Автономної Республіки Крим та м.
Севастополя, надав неоднозначний коментар.

«Археологов всегда интересовали Приазовские степи: здесь находится знаменитый
Мариупольский могильник эпохи бронзы, памятники Средневековья и многое другое.
Однако начинать надо не с археологических раскопок, а с восстановления музея и
создания научной инфраструктуры».

Українські археологи підкреслили, що такі слова, вчинки та дії представника
російської наукової спільноти є прямим відображенням «путніської імперської
політики» й «русского мира».

Читайте також: Археологічні експедиції Маріупольського державного університету
поставлено на паузу (ФОТО)

«Якщо 24 лютого росія оприявнила цивілізованому світові своє справжнє обличчя
країни-терориста, то 2 жовтня — й російська археологія остаточно скинула маски
й перестала приховувати, що є лише інструментом гуманітарної агресії росії
проти України та виконавцем впровадження політичної ідеології».

До яких дій закликає Спілка археологів України?

ВГО «Спілка археологів України» закликає відповідні державні органи України:

1. Накласти санкції на осіб, що сприяють російському режиму, а саме:

• Костянтина Могилевського, голову правління російського історичного товариства;
• Андрія Мальгіна, так званого «директора центрального музею тавриди» (м.Сімферополь, АРК, Україна).

2. Відкрити кримінальні провадження проти громадян російської федерації та
колаборантів, причетних до руйнування і розграбування об'єктів культурної
спадщини, організації та проведення незаконних археологічних розкопок на
тимчасово окупованих територіях і подати цих осіб у міжнародний розшук.

Раніше Донбас24 розповідав про злочини росіян проти культурної спадщини Донбасу.

Ще більше новин та найактуальніша інформація про Донецьку та Луганську області в нашому телеграм-каналі Донбас24.

ФОТО: з відкритих джерел.
