%%beginhead 
 
%%file 15_06_2021.fb.fb_group.mariupol.nekropol.1.chentukovy
%%parent 15_06_2021
 
%%url https://www.facebook.com/groups/278185963354519/permalink/539096630596783
 
%%author_id fb_group.mariupol.nekropol,marusov_andrij.mariupol
%%date 15_06_2021
 
%%tags 
%%title Чентуковы: У них гостили Айвазовский и император Александр II
 
%%endhead 

\subsection{Чентуковы: У них гостили Айвазовский и император Александр II}
\label{sec:15_06_2021.fb.fb_group.mariupol.nekropol.1.chentukovy}
 
\Purl{https://www.facebook.com/groups/278185963354519/permalink/539096630596783}
\ifcmt
 author_begin
   author_id fb_group.mariupol.nekropol,marusov_andrij.mariupol
 author_end
\fi

\vspace{0.5cm}
\textbf{Чентуковы: У них гостили Айвазовский и император Александр II}

\#люди\_мариупольского\_некрополя

Заросшие мхом, эти плиты почти 150 лет лежат забытые возле роскошной
усыпальницы Найденовых. Похороненные под ними люди лично знали
императора-реформатора Александра II и его брата-адмирала Константина,
художника Ивана Айвазовского, поэта Василия Жуковского, автора имперского гимна
\enquote{Боже, царя храни!}, и исследователя Арктики графа Литке... 

\textbf{Авраам Константинович Чентуков} (1819-1883) и его супруга \textbf{Марина Фотиевна}
(1821-1895; в девичестве Логофетова) были представителями одного из богатейших
купеческих родов Мариуполя XIX века. 

Дед Авраама – грек \textbf{Иван Константинович Чентух} – прибыл в Мариуполь в 1780 году
из крымского Инкермана. Известны еще два рода Чентухов – в Старом Крыму и
мариупольской Марьинке. Только потомкам Ивана Чентуха удалось выбиться в
деловую и политическую элиту города в XIX веке.

Незадолго до смерти Иван становится купцом. Его сын Константин (род. 1791 г.)
возглавляет Мариупольский греческий суд, орган самоуправления греческой общины,
и быстро становится одним из самых уважаемых и богатых людей города.

В октябре 1837 года через Мариуполь проезжает \textbf{цесаревич Александр Николаевич},
будущий император-освободитель крестьян. Его сопровождает \textbf{Василий Жуковский},
воспитатель Александра, основоположник романтизма в русской поэзии и наставник
А. Пушкина.

Цесаревич \emph{\enquote{отправился прямо в соборную Харлампиевскую церковь и отслушав
эктению прибыл в назначенную Его Высочеству в доме 3-й гильдии купца Чентукова
квартиру, где встречен хозяином дома и гражданами города с хлебом-солью}}.

Так опишут эту встречу через полвека преподаватели Александровской мужской
гимназии, авторы книги \enquote{Мариуполь и его окрестности}. Александр подарил
Константину Чентукову золотую табакерку. Без сомнения, в этой встрече
участвовал и 19-летний Авраам Чентуков.

Кроме завтрака в Мариуполе, романтик Жуковский в своем дневнике особо отметил
\enquote{греческую живость лица женщин}...

Больше отмечать было нечего. В городе жило всего около трех тысяч человек.
Накануне приезда наследника престола таганрогский градоначальник отрядил в
Мариуполь чиновника Сесемана – чтоб удостовериться, всё ли в порядке.

Всё оказалось в полном беспорядке:

\begin{leftbar}
\em\enquote{Церковная ограда вокруг собора развалилась, в ограде всякие нечистоты и
неопрятности. Улицы имеют рытвины и выбоины... дома во многих местах требуют
починки, крыши ветхи, по крайней мере, не обмазаны, трубы развалились и нигде
не побелены... во многих домах нет стекол, а в других забиты дощечками или
залеплены бумагой, заборы каменные развалились и не обмазаны, деревянные
обрушились...}
\end{leftbar}

Сесеман очень беспокоился, что канат на пароме через Кальмиус оборвется – он
состоял \emph{\enquote{из кусков}}. Но все обошлось, и цесаревича со свитой не унесло в
Азовское море...

В 1845 году Мариуполь посетил \textbf{великий князь Константин Николаевич Романов, брат
Александра II}. К тому времени 26-летний Авраам Чентуков уже женился на Марине
Фотиевне Логофетовой, выпускнице Харьковского института благородных девиц.

Возможно, Марина смогла блеснуть своим знанием французского перед великим
князем – как и его брат, Константин Романов остановился у Чентуковых. В
результате они стали обладателями второй золотой табакерки \enquote{от царя}.

Кстати, именно к приезду Константина были уничтожены остатки Кальмиусской
крепости вокруг Харлампиевского собора и базарной площади.

Свиту Константина возглавлял его воспитатель \textbf{граф Федор Литке} (1797-1882),
мореплаватель, легендарный участник нескольких кругосветок, исследователь
Арктики, президент Академии Наук... В свиту входил и молодой художник-маринист
\textbf{Иван Айвазовский}, согласно исследованиям мариупольского краеведа Льва Яруцкого.

Великий князь Константин (1827-1892) последний раз посетил Мариуполь в 1872
году. Он посадил в городском саду два диковинных дерева \enquote{эвкалиптуса} и
полюбовался фруктовыми садами на Кальчике. Именно в его честь в городе возникла
улица Константиновская (ныне - архитектора Нильсена). Встречался ли он с
Чентуковыми – мы не знаем.

Но мы точно знаем, что через восемь лет уже его сын, \textbf{великий князь Константин
Константинович Романов}, купит за баснословные пять тысяч рублей у мариупольца
\textbf{Архипа Куинджи} гениальную картину – \enquote{Лунная ночь на Днепре}! Он даже возьмет ее
в плавание по Средиземному морю и покажет в Париже... 

Возможно, отец и сын Романовы в семейном кругу удивлялись – откуда в
провинциальном Мариуполе появился такой талантливый художник, сравнимый с
Айвазовским...

А что же Чентуковы? В 1864 году Авраам - заседатель Мариупольского греческого
суда. Он богат. У него с Мариной десять детей! Через год родится последний
ребенок – сын Александр...

Правда, до зрелых лет доживет половина из них.

\textbf{Спиридон Аврамович Чентуков} (1847-1919) будет купцом, землевладельцем, гласным
городской думы, председателем городского Сиротского суда. Он женится на
итальянской дворянке Анастасии Деросси. У них будет два сына – Анатолий и
Антоний. Спиридон умрет своей смертью в революционное лихолетье (могила
неизвестна).

Самый младший ребенок Чентуковых, \textbf{Александр Аврамович Чентуков} (1865-1920),
попытается бежать от революции, но не успеет на корабли белой Русской Армии,
уходившие в Стамбул из Крыма. Большевики расстреляют его в Керчи 7 декабря 1920
года...

Дочери Чентуковых удачно выйдут замуж. Самая старшая \textbf{Екатерина Аврамовна}
(1839-1898) получит фамилию мужа Георгия – \textbf{Палеолог}. Два ее сына – Андрей и
Леонид – будут потомственными почетными гражданами Мариуполя. Андрей станет
гласным городской думы в 1916 году. 

\textbf{Варвара Аврамовна} (род. 1856) выйдет замуж за Фому Лаврентьевича \textbf{Бялковского},
католика, коллежского регистратора.

\textbf{Василися Аврамовна} (род. 1844) станет супругой Исидора
\textbf{Корради}, купца, таганрогского нотариуса…

... В 1990-х краевед Лев Яруцкий общался с потомками купцов Чентуковых, но ничего
не написал об их судьбе. Если вам что-то известно о них – пожалуйста,
отзовитесь!

==============================

Если вам интересны истории людей Мариупольского Некрополя, приглашаем на наши
экскурсии по Некрополю.

Если вы хотите своими руками исследовать и восстанавливать Некрополь –
приглашаем на регулярные волонтерские экспедиции на Старое городское кладбище!

Анонсы экскурсий и экспедиций публикуются на странице Архі-Місто
\textbackslash~ Архи-Город

P.S. Огромное спасибо волонтерам Александру и Наталия Шпотаковская за расчистку
плит Чентуковых на прошедших выходных!

Источники:

\begin{itemize} % {
\item \enquote{Мариуполь и его окрестности}, раздел \enquote{о посещении Мариуполя высочайшими Особами}
				\footnote{\url{http://old-mariupol.com/vtoroj-den-ekskursii-15-marta}}
				\footnote{Internet Archive: \url{https://archive.org/details/25_03_2011.old_mariupol.vtoroj_den_ekskursii_8_marta}}
\item Ревизские сказки и родословные приазовских греков см. здесь \url{https://www.azovgreeks.com}
\item Метрические книги церквей Мариуполя см. \url{https://tsdea.archives.gov.ua/ua/metric-books}
\item Статья Льва Яруцкого о посещении Мариуполя вел.князем Константином, Федором Литке и Иваном Айвазовским см. 
%\href{http://old-mariupol.com.ua/znamenitye-gosti-goroda}{}
\href{https://archive.org/details/14_07_2011.lev_jaruckij.old_mariupol.znamenitye_gosti_goroda}{%
				ЗНАМЕНИТЫЕ ГОСТИ ГОРОДА, Лев Яруцкий, Старый Мариуполь, 14.07.2011}
\footnote{\url{http://old-mariupol.com.ua/znamenitye-gosti-goroda}}
\footnote{Internet Archive: \url{https://archive.org/details/14_07_2011.lev_jaruckij.old_mariupol.znamenitye_gosti_goroda}}

\item Статья Льва Яруцкого о посещении Мариуполя цесаревичем Александром и поэтом Василием Жуковским 
\footnote{Семнадцать лет спустя, Лев Яруцкий, Старый Мариуполь, 21.06.2011, \url{http://old-mariupol.com/semnadcat-let-spustya}} %
\footnote{Internet Archive: \url{https://archive.org/details/21_06_2011.lev_jaruckij.old_mariupol.semnadcat_let_spustja}}
\end{itemize} % }

\#авраам\_чентуков

\#марина\_чентукова

\#чентуков
