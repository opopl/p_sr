% vim: keymap=russian-jcukenwin
%%beginhead 
 
%%file 20_09_2021.fb.moja_movna_stijkistj.1.nestle_sud_jazyk
%%parent 20_09_2021
 
%%url https://www.facebook.com/mmstiykist/posts/382441623463988
 
%%author_id moja_movna_stijkistj
%%date 
 
%%tags jazyk,kompania.nestle,mova,sud,ukraina,ukrainizacia
%%title Запрошуємо всіх на перше засідання Справа Нестле про примушування використовувати у роботі російську мову
 
%%endhead 
 
\subsection{Запрошуємо всіх на перше засідання Справа Нестле про примушування використовувати у роботі російську мову}
\label{sec:20_09_2021.fb.moja_movna_stijkistj.1.nestle_sud_jazyk}
 
\Purl{https://www.facebook.com/mmstiykist/posts/382441623463988}
\ifcmt
 author_begin
   author_id moja_movna_stijkistj
 author_end
\fi

Запрошуємо всіх на перше засідання Справа Нестле про примушування
використовувати у роботі російську мову.

Історія почалась у 2019 році, коли працівник українського офісу Нестле Артем
Кравченко, зіштовхнувся з дискримінацією на мовному ґрунті. А сьогодні адвокат
Максим Васильєв веде цю справу в суді.

\ifcmt
  pic https://scontent-frt3-2.xx.fbcdn.net/v/t39.30808-6/c0.26.1920.1002a/s843x403/242133150_107073961730693_4056661737542356148_n.jpg?_nc_cat=101&_nc_rgb565=1&ccb=1-5&_nc_sid=340051&_nc_ohc=8CNwywhCux0AX-wm7Nz&_nc_ht=scontent-frt3-2.xx&oh=b4e132e49dc24b791c4e0b7c02d47c9b&oe=6150DF3A
  @width 0.5
\fi

Цей судовий процес може стати унікальним прецедентом у випадках порушення
законодавства про мову. Але для цього нам потрібна ваша підтримка і розголос.

Півтори години вашого часу 30 вересня або поширення цього допису зробить
великий вклад у захист української мови.

✔️Час засідання: 30 вересня, четвер, 10:00

Адреса: Подільський районний суд, вул. Волоська 6/14, корпус Б.

Перед засіданням, о 09:30, відбудеться прес-конференція ініціативної групи по
справі Нестле.

Контакт: \url{mmstiykist@gmail.com}

Посилання на подію та хронологія за посиланням: 

\url{https://fb.me/e/17POxFZpq}

Ваша присутність на акції чи поширення інформації про Справу Нестле
продемонструє іншим корпораціям, що насадження своїм працівникам мови
країни-окупанта в Україні є неприйнятним та несе репутаційні втрати.
