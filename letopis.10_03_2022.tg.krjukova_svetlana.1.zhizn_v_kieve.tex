% vim: keymap=russian-jcukenwin
%%beginhead 
 
%%file 10_03_2022.tg.krjukova_svetlana.1.zhizn_v_kieve
%%parent 10_03_2022
 
%%url https://t.me/kryuchoktv/3884
 
%%author_id krjukova_svetlana
%%date 
 
%%tags 
%%title Жизнь в Киеве продолжается, принимая свои новые формы
 
%%endhead 
 
\subsection{Жизнь в Киеве продолжается, принимая свои новые формы}
\label{sec:10_03_2022.tg.krjukova_svetlana.1.zhizn_v_kieve}
 
\Purl{https://t.me/kryuchoktv/3884}
\ifcmt
 author_begin
   author_id krjukova_svetlana
 author_end
\fi

На сотый вопрос к моей пятисотлетней бабушке «Почему ты не хочешь уехать из
Киева?» сегодня получила ёмкий ответ: «Света, я прожила долгую жизнь. Лучше
умру от ракеты, чем от болезни где-то не в своей постели».

И что ей ещё нужно дочитать Теккерея - вторую часть. И сделать генеральную
уборку. И завести мне ключи, а то вдруг она умрет от инфаркта. Ну что с ней
делать?

Сегодня проснулась не от сирен. Моя психика включила игнор на страх. Две недели
назад этот звук вызывал тошноту и мороз в сердце, но уже сегодня - что сирены,
что вороны каркают - пофиг. Их кстати, ворон горластых, в Киеве дофига.
Голгочут как базарные бабы - роются по урнам, чувствуют себя коренными
киевлянами, крадут выброшенное, хазяйничают по урнам. 

Говорят, в Киеве осталось 2 млн людей. Любопытно, это те самые коренные
киевляне? Ещё 2 млн - это вороны. 

Я отвлеклась. Так вот утром разбудил меня дворник. Из местных. Он чистил снег.

В столице похолодало и это прекрасная новость - русском солдатам жутко холодно
в танках. Украинский чернозём, подмёрзлой сверху и вязкий чуть ниже от
поверхности это капкан. Наши воины партизанским движением отстреливают их по
лесам и полям, и это выглядит так круто, что мой сын сегодня сделал
торжественное заявление - учиться больше необязательно, он пойдёт в армию, и
наверное станет летчиком, или этим, как его, с микро-пушкой в руках, которая
обстреливает танки. Ему нравится как выглядят солдаты ВСУ - оснащённые,
крепкие, одухотворенные идеей, в отличие от потерянных русских - наши защищают
свой дом. Русские - приехали ловить летучих мышей, заражённых бактерией новой
версии коронавируса и очутились в братских могилах. Сотни трупов солдат. 

Украинские дети... 

Дети...

Дворник методично, с любовью и расстановкой расчищает снег. 

- Ваша работа успокаивает, - говорю я ему. 

- Он улыбается, и его улыбка в призме Киевского солнца, на фоне снега и стен
любимого города действует как сто психотерапий одновременно. 

- Вы знаете, я делаю три вещи, чтобы не сойти с ума. Работа, трудовая рутина,
труд, который могу делать только я - это важно и нужно, и это - отвлекает.
Второе - доброе дело. Сегодня я помог человеку. Он рад. 

- А третье? 

- Я строю планы на будущее. Хорошие и простые планы. Я постеснялась спросить
дальше. 

Завтра спрошу, о чем он мечтает, что планирует. Я же, после победы, куплю себе
новый чайник, этот, сука, ревет как ракета, - соседи в ужасе. Открою пекарню.
Буду печь дешёвый и вкусный хлеб. Нарисую кучу картин. Нарожаю кучу детей. И
заведу собаку. Здоровую, дурную, и верную.

После победы!
