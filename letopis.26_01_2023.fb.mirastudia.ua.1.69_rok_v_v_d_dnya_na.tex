%%beginhead 
 
%%file 26_01_2023.fb.mirastudia.ua.1.69_rok_v_v_d_dnya_na
%%parent 26_01_2023
 
%%url https://www.facebook.com/mirastudia.ua/posts/pfbid0Wk5Kdq1jk9zruS3ndPAuCwnVRTapYr6Lrct6VsLcxbjjoEBrDXV4v9j6xsvP9Qusl
 
%%author_id mirastudia.ua
%%date 26_01_2023
 
%%tags hudozhnik,isskustvo,kartina,birthday,kovalchuk_viktoria.ua.hudozhnik.1954_2021
%%title 69 років від дня народження української художниці Вікторії Ковальчук
 
%%endhead 

\subsection{69 років від дня народження української художниці Вікторії Ковальчук}
\label{sec:26_01_2023.fb.mirastudia.ua.1.69_rok_v_v_d_dnya_na}

\Purl{https://www.facebook.com/mirastudia.ua/posts/pfbid0Wk5Kdq1jk9zruS3ndPAuCwnVRTapYr6Lrct6VsLcxbjjoEBrDXV4v9j6xsvP9Qusl}
\ifcmt
 author_begin
   author_id mirastudia.ua
 author_end
\fi

26 січня виповнилося б 69 років від дня народження Вікторії Ковальчук - відомої
української художниці-графіка, ілюстраторки, літераторки, дизайнерки, лауреатки
премії імені Л.Українки і літературно-мистецької премії імені О.Пчілки,
дипломантки конкурсів \enquote{Мистецтво книги} і численних міжнародних художніх
виставок (1954-2021).

Її творчість називали міфічним реалізмом.

\enquote{Мистецтво - це сфера, яка межує з духовним світом і є проявом нашого
спілкування з невидимим. Мені дуже близьке тлумачення середньовічних схоластів:
мистецтво - це відображення Божественного в людській реальності, свідомості.},
- вважала Вікторія Ковальчук.

\enquote{Художник - вісник, а не садівник власної пихи.}, - було девізом мисткині. 

Вікторія Ковальчук проілюструвала майже 200 книжок, з яких 30 - дитячих. 

\enquote{Прекрасний дар і диво велике: в усьому, що творить пензлем художниця, живе
душа, тому придивіться до її малюнків. Вони рухаються в просторі, промовляють
музикою і веселяться співом, дивуючи несподіваністю кольорів, вражаючи
художньою метафорою. А дає життя фантастичному світові малюнка національний
рух, який проміниться світлом з кожної клітини мистецького твору, являється і
дітям, і дорослим як жива казкова істота, що була на цій землі завжди.}, -
сказав про творість Вікторії Ковальчук колишній директор видавництва \enquote{Веселка},
письменник Ярема Гоян. 

Народилася майбутня художнииця на Волині в м.Ковелі. Змалку любила малювти, з
12-ти років відвідувала вечірню художню школу. Навчалася у Львівській
спеціалізованій математичній школі. 

Захоплювалася з дитинства не лише образотворчим мистецтовм, а й літературою,
відвідувала літературний гурток.

Закінчила Львівський поліграфічний інститут імені І. Федорова (нині Львівська
академія друкарства). Після закінчення інституту деякий час працювала в
Українському поліграфічному інституті, у видавництівах \enquote{Вища школа} і
\enquote{Каменяр}, в Українській академії дизайну. Але більшу частину життя була
вільним художником, працювала з вітчизняними і зарубіжними видавництвами. 

Більшу частину життя Вікторія Ковальчук прожила у Львові.

Великою мірою стиль і образну структуру її робіт сформувало глибоке захоплення
фольклором, особливо традиціями, звичаями, віруваннями гуцулів, бойків, лемків
- етнічними групами Українських Карпат.

У 2000 р. вона розробила оригінальне оформлення українського Букваря, який був
названий найкращою книгою року, отримавши Диплом І-го ступеня на
Всеукраїнському конкурсі \enquote{Мистецтво книги}. Буквар витримав дев'ять! перевидань
загальним накладом понад 200 тис примірників. 

\enquote{Книга для дитини - дороговказ, а не розвага. Тому так важливо, аби вона була
світлою, якісною, мудрою. Художник дитячих книг - той, що зберіг дитячість і не
цурається її, адже багато що можна побачити очима серця.}, - була переконана
Вікторія Ковальчук. 

Мистецтвознавці умовно поділяють творчість художниці на кілька періодів:
етноперіод з елементами сюрреалізму, \enquote{східний} період, \enquote{міфічний реалізм},
спроба візуалізувати властивості людської душі і \enquote{приколізм} - своєрідний
протест цивілізації споживання, культу наживи і насильства.

Чотри роки Вікторія Ковальчук ретельно працювала над книжкою, присвяченою
народному вбранню \enquote{Український стрій}, збираючи \enquote{польовий} матеріал по всій
країні. \enquote{Свого} видавця книжка шукала шість років і вийшла у 2000 р.
ексклюзивним накладом.

Її ілюстрації прикрашають твори, зокрема Богдана-Ігоря Антонича, Івана
Світличного, Ігоря Калинця, Наталі Забіли, Дмитра Павличка, Ліни Костенко,
Богдана Стельмаха,  Галини Кирпи. Видання, проілюстровані Вікторією Ковальчук,
часто перемагали на книжкових виставках. 

\enquote{Я художник-символіст. Так, я люблю писати натюрморти, пейзажні етюди, але це
не головне. Найбільший кайф і задоволення я отримую тогді, коли роздумую над
чистим аркушем.}, - Вікторія Ковальчук.

А ще Вікторія Ковальчук писала казки, новели, повісті: \enquote{Про фею Дорофею},
\enquote{Подорож з Нічним постояльцем} (Бучач очима Шмуеля Агнона), \enquote{Гусінь Мотя, війна
і любов - Правдива казочка}, \enquote{Казка про ворону, яка хотіла лишитися чорною}.

У 1999 р. Львівське телебачення зняло фільм \enquote{Поетичний театр Вікторії Ковальчук}.

Підготовлено за матеріалами з відкритих джерел.
