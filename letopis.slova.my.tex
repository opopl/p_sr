% vim: keymap=russian-jcukenwin
%%beginhead 
 
%%file slova.my
%%parent slova
 
%%url 
 
%%author 
%%author_id 
%%author_url 
 
%%tags 
%%title 
 
%%endhead 
\chapter{Мы}

%%%cit
%%%cit_pic
%%%cit_text
\emph{Мы} ещё умеем дружить и не прощаем предательства. \emph{Мы} ещё помним,
что любить человека – отнюдь не синоним слова «спать» с ним. \emph{Мы}
пережили развал страны, 90-е, парочку мировых кризисов и несколько воин
(особенно внутренних).  До \emph{нас}, наконец, дошло, что наши родители делали
для \emph{нас} всё, что могли. Даже если не делали ничего.  \emph{Мы} теперь
уже точно осознали, что «круто» – это не клубы каждые выходные, не градусы в
стакане, а здоровый цвет лица, крепкий сон и дорогие люди рядом.  \emph{Мы}
вручную рисовали «поля» в тетрадях, дрались исключительно «до первой крови» и
уже многих похоронили. \emph{Мы} – самое закаленное и, не в обиду другим, самое
прогрессивное поколение. \emph{Нам} повезло родиться ещё до того, как у детей
отобрали свободу и право выбора. \emph{Мы} привыкли никого не слушать, ни на
кого не надеяться и ни от кого ничего не ждать.  \emph{Мы}, дети 60-80-х, всё
привыкли делать сами – потому что лучше \emph{нас} никто не сделает...» Автор -
Галина Ляшевская
%%%cit_comment
%%%cit_title
\citTitle{Советскому поколению повезло родиться до того, как у детей отобрали право выбора}, 
Елена Лукаш, strana.ua, 13.06.2021
%%%endcit

%%%cit
%%%cit_pic
%%%cit_text
Фактически \emph{мы} живем в двух непересекающихся сообществах. Есть правящее. У него
«героям слава», проспекты Бандеры и Шухевича, годовщины Майдана и постоянная
гордость за какую-то иррациональную хрень, вроде украинского борща. И есть
второе сообщество – наблюдателей и терпящих. Оно тоже неоднородное, пестрое и
со своими тараканьими представлениями об этом мире
%%%cit_comment
%%%cit_title
\citTitle{Бандеровцы при власти создали в Украине уютненькую Уганду}, 
Игорь Лесев, strana.ua, 13.06.2021
%%%endcit

