% vim: keymap=russian-jcukenwin
%%beginhead 
 
%%file slova.chinovnik
%%parent slova
 
%%url 
 
%%author 
%%author_id 
%%author_url 
 
%%tags 
%%title 
 
%%endhead 
\chapter{Чиновник}
\label{sec:slova.chinovnik}

%%%cit
%%%cit_head
%%%cit_pic
%%%cit_text
\emph{Чиновники} и политики, за наши с вами деньги, нарушили все ключевые положения
Конституции - от бесплатной медицины, защиты частной собственности и личной
неприкосновенности, нарушения прав меньшинств, запрета на вмешательства в
религиозные дела, создание государственной идеологии, до отсутствия
референдумов и принципа разделения ветвей власти, разрушения независимости
судебной системы.  Мы пережили уже 5 периодов \enquote{конституционного моделирования}:
советско-парламентский (1991-1996), президентско-парламентский (1996-2004),
парламенско-президентский (2004-2010), президентско-парламентский (2010-2014),
парламентско-президентский / по сути президентский (2014-2018)
%%%cit_comment
%%%cit_title
\citTitle{Из 161 статьи Конституции Украины системно нарушаются 125 / Лента соцсетей / Страна}, 
Руслан Бортник, strana.ua, 28.06.2021
%%%endcit

%%%cit
%%%cit_head
%%%cit_pic

\ifcmt
  tab_begin cols=3
     width 0.3

     pic https://img.strana.ua/img/article/3424/julija-mendel-kniha-57_main.jpeg
     width 0.5

     pic https://strana.ua/img/forall/u/10/91/%D0%A1%D0%BD%D0%B8%D0%BC%D0%BE%D0%BA_%D1%8D%D0%BA%D1%80%D0%B0%D0%BD%D0%B0_2021-07-07_%D0%B2_17.25_.46_.png
     width 0.2

     pic https://strana.ua/img/forall/u/10/91/%D0%A1%D0%BD%D0%B8%D0%BC%D0%BE%D0%BA_%D1%8D%D0%BA%D1%80%D0%B0%D0%BD%D0%B0_2021-07-07_%D0%B2_17.25_.46_.png
     width 0.2

  tab_end
\fi

%%%cit_text
Суть того, что хотела сказать этим Юлия, понять можно с трудом.
Тем не менее отношение Юлии Мендель к солдатам Красной армии донесено.
\emph{Чиновница} описывает собирательный образ советского военного, победившего в
Великой Отечественной войне. Войдя в Берлин, он принуждал к сексу и избивал
немку. А после возвращения с фронта беспросветно пил водку, отталкивал семью и
орал на родную мать. В общем, был насильником, алкоголиком и деспотом. Высказал
свое отвращение к нацизму он, считает Мендель, лишь раз – когда написал на
стене Рейхстага: \enquote{Я @бу Гитлера в ж@пу}.
Для сохранения стиля автора мы публикуем эту часть текста полностью
%%%cit_comment
%%%cit_title
\citTitle{\enquote{Я їб...в Гітлера в зад}. Что пишет Юлия Мендель в своей книге о двух годах у Зеленского}, 
Екатерина Терехова, strana.ua, 07.07.2021
%%%endcit

