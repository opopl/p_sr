% vim: keymap=russian-jcukenwin
%%beginhead 
 
%%file 27_08_2018.stz.news.ua.mrpl_city.1.240_rokiv_mrpl_oleksandr_uteganov
%%parent 27_08_2018
 
%%url https://mrpl.city/blogs/view/do-240-richchya-mariupolya-oleksandr-uteganov
 
%%author_id demidko_olga.mariupol,news.ua.mrpl_city
%%date 
 
%%tags 
%%title До 240-річчя Маріуполя: Олександр Утеганов
 
%%endhead 
 
\subsection{До 240-річчя Маріуполя: Олександр Утеганов}
\label{sec:27_08_2018.stz.news.ua.mrpl_city.1.240_rokiv_mrpl_oleksandr_uteganov}
 
\Purl{https://mrpl.city/blogs/view/do-240-richchya-mariupolya-oleksandr-uteganov}
\ifcmt
 author_begin
   author_id demidko_olga.mariupol,news.ua.mrpl_city
 author_end
\fi

2018 рік став ювілейним як для Маріуполя, так і для Донецького академічного
обласного драматичного театру (м. Маріуполь), якому виповнюється 140 років.
Маріуполь дійсно відрізняється унікальною театральною культурою, яку створювали
талановиті і яскраві режисери, актори, художники та вдячні глядачі. До
140-річчя театру хочу розпочати цикл статей, присвячених важливим і знаковим
подіям та діячам, що вплинули на становлення і розвиток театрального мистецтва
Маріуполя.

1970-ті роки стали \enquote{золотим} десятиліттям в житті Маріупольського театру та
однією з найбільш яскравих епох в історії театрального мистецтва Приазов'я.
Маріупольський театр впродовж 1970-х років набуває неофіційного статусу одного
з кращих драматичних театрів України, чому сприяли гастрольні маршрути
колективу та акторський склад трупи. Однак найбільший внесок у розвиток
театральної справи цього періоду зробив режисер, народний артист УРСР \textbf{Олександр
Кадирович Утеганов (1968 – 1986 роки).}

\ii{27_08_2018.stz.news.ua.mrpl_city.1.240_rokiv_mrpl_oleksandr_uteganov.pic.1}

У грудні 1968 року О. Утеганов був призначений головним режисером
маріупольського театру, в якому на той час склалася досить напружена ситуація.
Після смерті заслуженого артиста УРСР М. Шейка трупа майже півтора року
існувала методом \enquote{самоврядування}, вистави розвалювалися, актори гарячково вели
переписку з іншими театрами. Це була яскрава ілюстрація до відомого в театрі
правила: створювати хороший театр доводиться десятиліттями, а зруйнувати можна
й за один рік. Перед О. Утегановим постало завдання відтворити театр заново. За
короткий термін переорієнтувати збірний творчий склад у стабільну творчу
єдність із високими вимогами професійної культури було надзвичайно складно.
Однак поступово, наполегливо і послідовно О. Утеганов привів трупу в дієздатний
стан. Вже перші його вистави дали можливість акторам відчути перспективу роботи
з цікавим режисером, і через кілька років маріупольський театр став одним з
провідних колективів у республіці.

\textbf{Читайте також:} \emph{Большая реконструкция. Мариуполь спешит преобразиться к своему 240-летию}%
\footnote{Большая реконструкция. Мариуполь спешит преобразиться к своему 240-летию (ФОТО), Ярослав Герасименко, mrpl.city, 25.08.2018, \url{https://mrpl.city/news/view/bolshaya-rekonstruktsiya-mariupol-speshit-preobrazitsya-k-svoemu-240-letiyu-foto}}

Олександр Утеганов навчався режисурі у видатного майстра українського театру М.
Крушельницького, що тяжів до піднесено-романтичного стилю поетичного театру.
Після інституту молодий режисер працював у театрах Ізмаїла, Петрозаводська,
Якутська, Алма-Ати. Кілька років – в Петрозаводську та Алма-Аті – разом з
відомим у Росії режисером М. Сулімовим, майстром театру психологічного. Саме М.
Сулімова О. Утеганов вважав своїм другим вчителем. Для своєї першої постановки
у місті Жданов він обрав п'єсу О. Окулевича \enquote{Джордано Бруно}.

Незважаючи на слабку драматургію, схематичні, позбавлені психологічної глибини
характери персонажів, режисерові вдалося змусити акторів по-новому поглянути на
свої творчі можливості та згуртувати колектив.

Поява в репертуарі \enquote{Піднятої цілини} М. Шолохова (1969 рік) в постановці О.
Утеганова дала підставу для серйозної розмови про нового головного режисера.
Об'ємні, повноцінні характери шолоховських героїв дозволили О. Утеганову
проявити одне з головних своїх обдарувань – вміння працювати з акторами.

Гастролі в Москві та Києві, Ленінграді та Баку, в інших містах країни стали, з
одного боку, перевіркою сил, а з другого – надали колективу творчої впевненості
для пошуків, для розширення обріїв. Мінськ, Гомель, Рига, Таллін, Вільнюс,
Ленінград, Харків Полтава, Чернівці, Псков, Калуга, Тула, Севастополь – літні
маршрути маріупольців. Під час московських і ризьких гастролей частими гостями
на спектаклях були драматурги – автори п'єс, включені у гастрольну афішу, – О.
Арбузов, Б. Васильєв, Г. Горін.

\ii{27_08_2018.stz.news.ua.mrpl_city.1.240_rokiv_mrpl_oleksandr_uteganov.pic.2}

Відомий московський критик Ю. Рибаков у своїй статті зазначав, що \emph{\enquote{театр, яким
керує головний режисер О. Утеганов, йде шляхом психологічного мистецтва: в
характерах героїв, в моральних конфліктах він шукає відображення суттєвих явищ
і процесів соціальної дійсності. Режисура уважна до духовного світу людини,
енергійна в утвердженні добра, гідності людини, ідеалів часу. Кращі роботи
ждановського театру свідчать про зрілість, самостійність творчого мислення}}. За
час роботи в маріупольському театрі режисером, народним артистом УРСР О.
Утегановим було поставлено понад 50 вистав, сформована повноцінна трупа – з 50
творчих працівників 14 з них мали почесні звання. \textbf{Народні артисти України: О.
Утеганов, Б. Сабуров, Н. Юргенс, Н. Білецька, заслужені артисти України М.
Земцов, В. Митрофанов, М. Алютова, В. Бугайов, С. Отченашенко, заслужені
артисти РРФСР: Г. Лєсніков, А. Сергєєв, заслужений артист Казахської РСР В.
Ахрамєєв, заслужений артист МаРСР П. Реп'єв, заслужений художник України М.
Ковальчук.} Завдяки цьому творчому надбанню, акторський цех театру нарешті став
повністю укомплектований.

\textbf{Читайте також:} \emph{К 240-летию Мариуполя: Слободка}%
\footnote{К 240-летию Мариуполя: Слободка, Сергей Буров, mrpl.city, 25.08.2018, \url{https://mrpl.city/blogs/view/k-240-letiyu-mariupolya-slobodka}}

\ii{27_08_2018.stz.news.ua.mrpl_city.1.240_rokiv_mrpl_oleksandr_uteganov.pic.3}

Для О. Утеганова не було акторів бездарних, були тільки ті, що не до кінця
розкрилися. Володіючи великим педагогічним чуттям і тактом, він умів знайти
індивідуальний підхід до кожного. Одним з найцікавіших моментів в його роботі з
постановочним колективом був застільний період аналізу п'єси, вибудовування
дієвої лінії, розтин підтексту і глибинних психологічних мотивувань кожного
образу – тe, що в театрі іменується \enquote{закваскою}. Вистави О. Утеганова завдяки
цій \enquote{заквасці}, відмінно закладеному фундаменту протягом 5-6 років не тільки
зберігалися в прем'єрному стані, але й росли, ставали кращими й більш
злагодженими. Олександр Кадирович був впевнений, що кожному працівнику –
режисерові, акторові, художнику треба ставити дуже складні творчі завдання –
часом навіть непосильні. Саме тоді люди будуть працювати з повною віддачею і
зроблять більше того, що можуть.

Від часу, коли колектив очолив Олександр Утеганов, творча репутація його
піднеслася і стабілізувалася. Немає такого республіканського огляду, на якому
вистави маріупольського театру не були б відзначені чи то в цілому, чи за
окремими компонентами. В архіві театру зберігається Почесна грамота Президії
Верховної Ради Азербайджану за підсумками гастролей в Баку в 1973 р. Колектив
нагороджений Почесними грамотами міністерства культури Латвії та Білорусії
після гастролей в столицях республік – Ризі та Мінську. О. Утеганов
наголошував, що театральний колектив, його життєздатність визначають єдність
творчих поглядів і спільність етичних норм. Відсутність етичної культури
негативно позначається й на творчості.

За художнього керівництва О. Утеганова театр зміг поліпшити зацікавленість
глядачів. Роками вироблялось у колективу уміння відчувати пульс часу,
готовність винести на суд глядача найбільш злободенні проблеми; роками
вдосконалювалась, шліфувалась акторська майстерність. Саме ці якості театру
проклали шлях до сердець глядачів. О. Утеганов зумів створити творчий колектив
високої театральної культури, виховав, навчив чимало справжніх акторів, які
зуміли гідно втілити його режисерські задуми. Цінуючи значення індивідуальної
майстерності актора, митець піклувався про міцний акторський ансамбль.

\textbf{Читайте також:} \emph{Мариупольцы в День города смогут прогуляться по обновленному Театральному скверу}%
\footnote{Мариупольцы в День города смогут прогуляться по обновленному Театральному скверу, Ярослав Герасименко, mrpl.city, 22.08.2018, \url{https://mrpl.city/news/view/mariupoltsy-v-den-goroda-smogut-progulyatsya-po-obnovlennomu-teatralnomu-skveru}}

Провідний художник театру Михайло Сидорович Ковальчук О. Утеганова вважав своїм
вчителем. Він зазначав, що \emph{\enquote{для актора Олександр Кадирович був чудовим
педагогом, дуже дбайливо підходив до кожного, нікого ніколи не травмував своїми
зауваженнями, пояснював, надихав, запалював. Багато працював з молоддю, в тому
числі з режисерами-дипломниками, які практично щороку ставили в маріупольському
театрі свої випускні спектаклі. З любов'ю ставився до них, в роботу не
втручався, але направляв}}.

\ii{27_08_2018.stz.news.ua.mrpl_city.1.240_rokiv_mrpl_oleksandr_uteganov.pic.4}

Коли Олександру Кадировичу Утеганову виповнилося шістдесят, він сказав: \emph{\enquote{Треба
вчасно піти, але встигнути виховати нового, молодого головного режисера}}. На
жаль, його надії на гідну зміну не цілком виправдалися. У театрі почали
виникати залаштункові турніри. Вразливість, невміння захиститися від нахабності
змусили його піти. Останні роки життя він працював у Палаці культури комбінату
\enquote{Азовсталь}, там його й досі шанобливо згадують. Станіслав Непряхін, яскравий
режисер і актор театру (на жаль, помер у 2017 році), зауважив, що \emph{\enquote{після
відставки О. Утеганова театр став здавати позиції, але творчий багаж, який був
накопичений його працею, залишається у нас і зараз. Потрібно намагатися
зберегти і не розгубити його}}.

Не кількістю часу вимірювалася цінність спілкування з цією людиною і режисером,
а широтою інтересів, глибиною знань, чим він до себе вабив, та якістю того, що
він давав і чим допомагав театральному колективу Маріуполя.
