% vim: keymap=russian-jcukenwin
%%beginhead 
 
%%file 07_12_2021.stz.kiev.bigkyiv.1.kyanka_peredovaja_vojna
%%parent 07_12_2021
 
%%url https://bigkyiv.com.ua/voyuye-vyshyvaye-i-pyshe-virshi-kyyanka-rozpovila-pro-svoye-zhyttya-na-peredovij
 
%%author_id moskalenko_lesja
%%date 
 
%%tags vojna,donbass,kiev,ukraina,kievljane,chelovek
%%title Воює, вишиває і пише вірші. Киянка розповіла про своє життя на передовій
 
%%endhead 
\subsection{Воює, вишиває і пише вірші. Киянка розповіла про своє життя на передовій}
\label{sec:07_12_2021.stz.kiev.bigkyiv.1.kyanka_peredovaja_vojna}

\Purl{https://bigkyiv.com.ua/voyuye-vyshyvaye-i-pyshe-virshi-kyyanka-rozpovila-pro-svoye-zhyttya-na-peredovij}

\ifcmt
 author_begin
   author_id moskalenko_lesja
 author_end
\fi

Зараз Юлії 35 років, війна для неї почалася зразу після Майдану. Спочатку вона
волонтерила у батальйоні «Айдар», возила для бійців допомогу – одяг, ліки,
спорядження. У 2014 році працювала у військовій лікарні.

\ii{07_12_2021.stz.kiev.bigkyiv.1.kyanka_peredovaja_vojna.pic.1}

\begin{zzquote}
«А вам запахло вже Різдвом?
Ялинкою із мерехтливими вогнями.
Отим святковим, радісним теплом.
Що ланцюжками тягнеться між нами?»
\end{zzquote}

Ці рядки свого вірша кілька днів тому опублікувала у Фейсбуці киянка Юлія
Євдокімова, яка служить у лавах Збройних сил України бойовим медиком.

«Великий Київ» запитав у Юлії, чим пахне Різдво на передовій, чи очікує бойова
родина приходу свята і як наші захисники будуть готуватися до нього.

«Ми готуємося, звісно!, – каже Юля, – ось поїдемо до магазину, накупимо всякого
дощику. На свята наготуємо різних традиційних страв, будемо, як всі українці!
На попередньому місці служби у мене була коробка з новорічними іграшками і я,
коли перебиралися кудись, завжди возила її за собою. Зараз ця святкова коробка
їздить по локаціях із моїми побратимами. До свят нам неодмінно передадуть
подарунки. Дуже приємно отримувати листи і малюнки від різних незнайомих діток.
Минулого року ми всі сиділи, читали їхні листи і писали їм відповіді».

Наша з Юлею розмова переривалася вигуками – «Дай ліхтарик! Подай запальничку!»

«Ти борщу хочеш? Борщ отам, і сметани візьми!» – відповідає Юля якомусь
невидимому співрозмовнику. І пояснює, що, на щастя, поранення зараз бувають
рідко, тому у вільний час вона готує їжу хлопцям. А вони ж не їдять всі
одночасно. Хтось іде в наряд, хтось – повернувся, тому і вказівки де шукати
борщ, треба давати всім по черзі.

Зараз Юлії 35 років, війна для неї почалася зразу після Майдану. Спочатку вона
волонтерила у батальйоні «Айдар», возила для бійців допомогу – одяг, ліки,
спорядження. У 2014 році працювала у військовій лікарні. Потім деякий час – у
Києві на цивільній роботі, влаштувалася арт-директором до кафе. Та поєднувати
волонтерську діяльність із роботою арт-директора було непросто. Тож у 2016
знову подалася на фронт. Служила у 72 бригаді, потім  59, зараз служить у
Маріупольській бригаді. Юля розповідає, що 56 бригада для неї, як сім’я, адже
там зібралися побратими, яких вона знає ще з Майдану.

\ii{07_12_2021.stz.kiev.bigkyiv.1.kyanka_peredovaja_vojna.pic.3}

«Коли я пішла на фронт, моєму синові було 8 років, зараз йому виповнилося 16,
він вже вчиться у інституті» – каже Юлія. На запитання, скільки ще збирається
служити, відповідає, що по контракту ще 2,5 років. І раптом каже «Ой, це ж мама
прочитає, а я їй кажу, що скоро повернуся, але нехай! Думаю, вона все і так
розуміє».

Зараз армія забезпе6чена усім необхідним, докуповувати доводиться хіба що
техніку і оснащення – дрони, автомобілі. А це немалі гроші, зібрати їх буває
складно. Тож Юля вишиває неймовірні картини, виставляє їх на аукціон, а
виручені кошти витрачає на потреби свого пірозділу. На восьмому році війни
просити грошей незручно, а продати річ ручної роботи, це зовсім інша справа,
вважає вона.

«Останню роботу, яку я виставила на продаж якраз на день ЗСУ, я вишивала на
ротації пів-року, – розповідає Юля, – зветься картина «Материнська любов».
Аукціон триватиме до 11 грудня, а гроші з продажу роботи підуть у скарбничку
підрозділу, ми збираємо на купівлю «бойової конячки». Хочемо придбати якусь
просту але витривалу машину, наприклад «Ниву» чи «УАЗік».

Ветеранка зізнається, що берегла цю картину, наче відчувала, що прийде час і
вона зможе допомогти у досягненні мети. І ось тепер вона та її побратими дуже
чекають на те, що знайдуться люди, які придбають їхній лот і це буде найкращим
подарунком нашим захисникам до Різдва!

Аукціон триватиме до 11 грудня, долучитися до нього можна
\href{https://www.facebook.com/100004442853474/posts/2056104344547641/}{тут}.
