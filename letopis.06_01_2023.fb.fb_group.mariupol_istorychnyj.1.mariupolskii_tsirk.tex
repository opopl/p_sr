%%beginhead 
 
%%file 06_01_2023.fb.fb_group.mariupol_istorychnyj.1.mariupolskii_tsirk
%%parent 06_01_2023
 
%%url https://www.facebook.com/groups/933164170541571/posts/1455195671671749
 
%%author_id fb_group.mariupol_istorychnyj,melnik_vadim.mariupol
%%date 06_01_2023
 
%%tags mariupol,mariupol.istoria,istoria,cirk,1918
%%title Мариупольский цирк
 
%%endhead 

\subsection{Мариупольский цирк}
\label{sec:06_01_2023.fb.fb_group.mariupol_istorychnyj.1.mariupolskii_tsirk}
 
\Purl{https://www.facebook.com/groups/933164170541571/posts/1455195671671749}
\ifcmt
 author_begin
   author_id fb_group.mariupol_istorychnyj,melnik_vadim.mariupol
 author_end
\fi

Давным-давно в группе выходила публикация про Мариупольский цирк. Недавно
попалась на глаза, захотелось повторить без изменений публикации автора...

Цирк Братьев Яковенко, Мариуполь. В этом цирке 25-го октября 1918-го года была
с единственной гастролью знаменитая актриса своего времени Вера Холодная, она
приехала на единственную гастроль с Осипом Руничем. Вера читала стихи и
танцевала танго с Руничем. Как это в тему сегодняшней пандемии - жизнь молодой
актрисы оборвала эпидемия гриппа \enquote{Испанка}, она умерла 16 февраля 1919-го года
в Одессе. Осипа Рунича судьба забросила в Йоханнесбург ЮАР, где он скончался
6-го апреля 1947-го года. Здание цирка было утрачено в 1943м, о былом цирке
напоминает Цирковый переулок который до сих пор есть на карте Мариуполя, но увы
цирка там давно нет.

%\ii{06_01_2023.fb.fb_group.mariupol_istorychnyj.1.mariupolskii_tsirk.cmt}
