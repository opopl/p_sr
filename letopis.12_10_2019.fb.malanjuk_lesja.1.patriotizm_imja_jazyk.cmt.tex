% vim: keymap=russian-jcukenwin
%%beginhead 
 
%%file 12_10_2019.fb.malanjuk_lesja.1.patriotizm_imja_jazyk.cmt
%%parent 12_10_2019.fb.malanjuk_lesja.1.patriotizm_imja_jazyk
 
%%url 
 
%%author_id 
%%date 
 
%%tags 
%%title 
 
%%endhead 
\subsubsection{Коментарі}

\begin{itemize} % {
\iusr{Pavlo Slavynskyy}
Теоретично можливі винятки - коли батьки назвали російським ім'ям, а людина свідома, тільки паспорт не хоче змінювати.
Але в 99\% це саме так.

\begin{itemize} % {
\iusr{Kyjeslav Borščahiveć}
\textbf{Павло Славинський} мене обурює 3 сторінка паспорта, де моє ім'я російською передають як "Павел". Питання, чого західнослов'янські імена не адаптують, а українські та білоруські перекладають

\iusr{Igor Noga}
Не існує ніяких "російських імен" Тому що самі узкіє це фікція. Діагноз я б навіть сказав  @igg{fbicon.smile}  Назвіть хоч одне.

\iusr{Ольга Когнітенко}
\textbf{Павло-Києслав Блізніченко} , та поміняйте вже той паспорт на ІД картку  @igg{fbicon.face.smiling.eyes.smiling} . Бачи ли б ви як у флюрографістки око сіпається, коли нама в паспорті росіянського варіанту  @igg{fbicon.laugh.rolling.floor} .
\end{itemize} % }

\iusr{Сергій Денісенко}

От я дуже давно створив акаунт, ще коли спілкувався переважно російською. Тепер
і прибрав би це написання, та не можу знайти, як саме. Створювати новий акаунт
не хочеться. Виходить, кожен бачить ім'я тією мовою, що налаштована в нього.

\begin{itemize} % {
\iusr{Леся Маланюк}
Я бачу «Сергій Денісенко».)

\iusr{Сергій Денісенко}

Але в розмовах іноді, коли відповідають, вилізає посилання на "Сергей ...".
Залежить від налаштувань. Одразу видно мову інтерфейсу в людини, навіть якщо
пише українською. Своєрідний індикатор

 @igg{fbicon.smile} 

\iusr{Pavlo Slavynskyy}
\textbf{Сергій Денісенко} Я це свідомо використовую. У мене ім'я для українців - одне, для іноземців - інше. По відповідях одразу бачу, як налаштований інтерфейс у співрозмовника.

\iusr{Олеся Гапанюк}
\textbf{Леся Маланюк} я також

\iusr{Костянтин Собіченко}
\textbf{Сергій Денісенко}, 

але ж в налаштуваннях, наскільки пам'ятаю, можна змінити ім'я повністю. У вас
там фігурує десь "Сергей"? Бо я, наприклад, бачу лише "Сергій Денісенко". А на
вашій сторінці - "Сергій Денісенко (Serhii Denisenko)". Звідки ж тоді береться
"Сергей"?... Цікава тема. Також неодноразово помічав, що фб-імена людей інколи
набувають інакшого вигляду. Приміром, людина підписана як Іван Петренко, а в
коментарях, коли хтось відповідає на його репліки, активізується чомусь вже
варіант Ivan Petrenko... або й того гірше - Иван.)


\iusr{Kyjeslav Borščahiveć}
\textbf{Павло Славинський} я своє ім'я налаштував лише українською та латинкою, тож всі люди з російським інтерфейсом бачать ім'я латинкою

\iusr{Сергій Денісенко}
\textbf{Костянтин Собіченко} десь "сидить" те, що було введено при створенні акаунту. Як до нього дістатися, не знайшов. Українською та латиною - то "альтернативне написання".

\iusr{Костянтин Собіченко}
\textbf{Павло-Києслав Блізніченко}, а як робиться оце налаштування додаткове, латинкою, наприклад?

\iusr{Людмила Батишкіна}
\textbf{Сергій Денісенко} А чому ДенІсенко, а не ДенИсенко.

\iusr{Kyjeslav Borščahiveć}
\textbf{Костянтин Собіченко} в налаштуваннях можна вибрати "підпис иншими мовами"

\iusr{Сергій Денісенко}
\textbf{Людмила Батишкіна} наслідки совкової транслітерації. Перший паспорт отримував у сересері, там "первинною" була рос.мова, українською лише дублювали.

\iusr{Людмила Батишкіна}
\textbf{Сергій Денісенко} Та це зрозуміло. Тепер переоформити проблематично. Багато возні. Документів багато переоформляти треба
Знаю по зятю.

\end{itemize} % }

\iusr{Yura Yakubovsky}
То все Християнські Імена. І правильно має бути так: Александер, Надія, Інґвар, Катеріна.

\begin{itemize} % {
\iusr{Леся Маланюк}
Тут не про імена, а про те, якою мовою вони написані.

\iusr{Yura Yakubovsky}
\textbf{Леся Маланюк} ми не знаємо ЯК людина записана в паспорті. Може то справді у неї таке імя.

\iusr{Леся Маланюк}
Та невже? Є такі імена Надежда чи Игорь?

\iusr{Yura Yakubovsky}
\textbf{Леся Маланюк} я їм в паспорти не заглядав. Що таке імя? Це показник психологічного здоровя батьків дитини.

\iusr{Василь Іванович}
Що Ви вкладаєте в "християнські імена"?

\iusr{Yura Yakubovsky}
\textbf{Василь Іванович} 

те і вкладаю. Більшість імен сучасних українців - це Імена, дані на честь
Християнських Святих. Вони у свою чергу поділяються на Загальнохристиянські
(зустрічаються серед усіх християн), окремо Католицькі імена (зустрічаються в
основному серед католиків) і окремо православні (зустрічаються в основному
серед православних народів). Наприклад ваше імя Василій - це Християнське імя,
проте в основному Православне. Також в Українському народі присутні язичницькі
словянські імена (тобто зустрічаються лише серед словян) та інші імена.


\iusr{Василь Іванович}

Я не Василій, а Василь... Якщо москвописка "братія" не вміє вимовити це слово,
то хто їм дохтор?.. Я Вас лиш прошу не протиставляти сучасну форму і зміст
нашої віри її попереднім формам і змістам, які всі є нашими, і більш нічиїми,
на цій Мідгард-Землі...

\iusr{Yura Yakubovsky}
\textbf{Василь Іванович} ви саме Василій. Так це імя пишеться в оригінальній грецькій формі. Василь і Вася - то лише скорочення. Чоловічі Християнські імена всі в основному закінчуються на -ій.

\iusr{Леся Маланюк}
\textbf{Юра Якубовський} (\textbf{Yura Yakubovsky}) А якщо людина пише у Фб не Василій, а Василий?

\iusr{Василь Іванович}
Якій ще грецькій мові! Ви маєте на увазі "старогрецьку", тобто - геленську, яка є мовою орійською, санскритом, українською?

\iusr{Василь Іванович}
\textbf{Леся Маланюк} От і його ім'я - це Ур (Ор, Ар, Ір) з йотизованим "Й" -
Йур, Юр. Зменшувально-ласкальна форма від нього - Юрко. Отак ми кличемо його.

\iusr{Yura Yakubovsky}
\textbf{Леся Маланюк} треба дивитися як в грецькому ориґіналі.

\iusr{Yura Yakubovsky}
\textbf{Василь Іванович} яка у вас освіта?

\iusr{Василь Іванович}
\textbf{Юра Якубовський} філологічна, дипломатична і історично-самоучна (що найцікавіше)... Років 40 займаюся історіографією, етнолігнгвістикою і іншим таким цікавеньким - Білінський аддихаєд.

\iusr{Yura Yakubovsky}
\textbf{Василь Іванович} імя Юрко завжди ненавидів. Моє імя Юрій. Походження цього імені цікаве, але це довго розписувати.

\iusr{Василь Іванович}
Дарма. Юрій - це ж Орій, який списом (арімійського-китайського) дракона убиває nach!

\iusr{Yura Yakubovsky}
\textbf{Василь Іванович} ні. Юрій - це поєднання імен Ярослав і Георгій.

\iusr{Василь Іванович}
\textbf{Юра Якубовський} 

Ось-ось! Це ОДНЕ Й ТЕ Ж. Ур-Юр, Ор-Йор, Ар-Яр, Ір-Їр. Jor - прочитайте
англійською (Джор), іспанською (Хор), французькою (Жор)... Це просто Сонце,
Сонячний, ... - і ще одно, але я думаю, Ви поки-що не готовий то сприйняти...

\iusr{Vitalik Stanislav}
\textbf{Василь Іванович} - Віталій ,розпишіть будьте добрий

\iusr{Завгородній Микола-пономар}
\textbf{Юра Якубовський} Інгвар-це скандинавське ім'я.

\iusr{Ivan Prylutskyj}

Що за маячня. Лише одне з них - Катерина - можна назвати можливо християнським.
Олександр - дохристичнське грецьке. Ігор взагалі нехристиянське варязьке, Надія
- питомо наше, дохристиянське...


\iusr{Ivan Prylutskyj}

Всі народи , беручи чужинські імена, спотворюють їх до невпізнання згідно своєї
вимови. І лише х...ли-чухраїнці намагаються витримати всі канони чужинської
мови!

\iusr{Zoya Jakivna}
\textbf{Ivan Prylutskyj} не підтримую, випадково натиснути

\iusr{Микола Романченко}
\textbf{Yura Yakubovsky} Олександр

\iusr{Mykola Krushnitski}
\textbf{Юра Якубовський} грецькою буде Васілеос, а ви перекручуєте

\iusr{Костянтин Собіченко}

Я щось також Інгваром здивований - скандинавське ж нібито. А ще більше - оцими
християнськими історіями. Можна подумати, що грецьке - то все християнське.

\iusr{Yura Yakubovsky}
\textbf{Ivan Prylutskyj} 

Катеріна - Християнське. Були аж дві Святі з таким іменем. Свята Катеріна
Римська і Свята Катеріна Сієнська. Александер - одночасно грецьке імя
Александра Великого (і двох його попередників) і християнське імя (бо було
кілька римських пап з таким іменем). Інґвар/Ігор - Скандинавське, але і
християнське, бо був такий наш правитель князь Ігор ІІ, і його Церква
канонізувала. Надія - поняття, на честь якого Християнська Церква дозволила
називати дітей і адаптовувати це імя відповідно до своєї мови. На Заході теж є
це імя - Хоуп. Поняття Надія завжди було популярне серед християн в усьому
Світі, бо це сутність самого Ісуса Хреста.

\iusr{Yura Yakubovsky}
\textbf{Завгородній Микола-пономар} Ні. Скандинавське імя - Інґвар. Записується, як Ingvar. Через G. Такого імені як Інгвар (Inhvar) - не існує.

\iusr{Yura Yakubovsky}
\textbf{Ivan Prylutskyj} хіба??? В УСІХ народів є імя Александер. І лише українці чомусь спотворюють це імя до Олександр. Хоча там і звуку такого немає.

\iusr{Yura Yakubovsky}
\textbf{Микола Романченко} Александер. Досить псувати прекрасне імя. Там навіть звуку О немає. Лише А.

\iusr{Юрій Іщук}
\textbf{Yura Yakubovsky} 

ага, Катеріна. Літературною українською мовою є мова Тараса Григоровича
Шевченка! І твір його безсмертний називається "КатерИна"! А Ваша КатерІна- то
десь у Італію! І Олександр , а не Александр! Не викручуйте мову під себе...

\end{itemize} % }

\iusr{Володимир Герасименко}
Ви праві

\iusr{Руслан Бащенко}

О, згоден на 100\%. Теж не розумію, навіщо українці пишуть свої імена
російською. Вірніше, розумію - малоросійська звичка, інерція, "так нада"...
Сумно

\iusr{Беляев Алексей}

Не согласен. Меня зовут Алексей , но это не мешает мне защищать нашу Украину. А
котось звати Тарас і т.і. так тільки кричати можуть "ганьба" а як зброю взяти в
руки так і всееееее......

\begin{itemize} % {
\iusr{Леся Маланюк}
А що тобі заважає назватися Олексієм?  @igg{fbicon.smile} 

\iusr{Костянтин Собіченко}
\textbf{Беляев Алексей}, якби ви вважали Україну своєю, то були б Олексієм. А так ви просто найманець, москвин на службі в Україні. Хоча й це, звісно, непогано вже.

\iusr{Pavlo Slavynskyy}
\textbf{Беляев Алексей} А що вам заважає писати українською?

\iusr{Костянтин Собіченко}
\textbf{Павло Славинський}, русскіє корні.)

\iusr{Ольга Храмова}
\textbf{Беляев Алексей} Олексію, здоров*я вам і Ангела Охоронця у вашій важкій праці захищати рідну Україну, повертайтеся живим. Прізвище своє пишіть як хочете, а ім*я пишіть українською всім ворогам назло, Олексій в перекладі Син Божий.

\iusr{Ivan Matviyiv}
\textbf{Беляев Алексей} 

ще один язікій захисник. Жити на землі народу і не знати його мову, або не
хотіти спілкуватися, краще помовчи, так буде чесніше.

\iusr{Беляев Алексей}
\textbf{Костянтин Собіченко} 

такий найманець як я був мій дід Петр який в роки Другої Світової на Первом
Украинском фронте первым вошёл в занятый фашистами Львов , та це йому не
заважало спилкуватись оба мовами

\iusr{Беляев Алексей}

А розповідати хто тут патріот а хто ні в контексті мови я вважаю це
розпалювання війни. Из -за этих разногласий ненужных погибло уже много хороших
людей та справжніх патриотів

\iusr{Костянтин Собіченко}
\textbf{Беляев Алексей}, краще б він туди не заходив.)

Розпалювання війни - це продовжувати утверджувати в Україні москвинський язик,
будь-де, в будь-якій формі. Навіть підписуючи ним свій фб-профіль.

\iusr{Орелі Боян}
\textbf{Беляев Алексей} 

захищать, так. Але і перетворювати Україну в НеоМатрьошкостан, теж. Україна, це
не лише територія. Це і мова, і культура, і традиції.

Ви язиком вбиваєте мову і весь духовний спадок українців. І замінюєте його на
спадок матрьошок.

Тобто такі собі внутрішні хуйлята. Великому Хуйлу не віддамо, самі правити
будемо. А українці і так паймут. Я вас правильно зрозуміла?

\iusr{Орелі Боян}
\textbf{Беляев Алексей} 

краще б не заходив до Львова. Те, що робили московські окупанти у Львові,
німчура і в кошмарах не бачила. Серед двох окупантів з німцями можна було жити.
А з московитами лише вмирати. Тож, не бачу приводу для гордості. Дід-окупант.
Внук досі не українець.

\iusr{Беляев Алексей}

А Вашими взглядами Донбасс лучше не освобождать.....

\iusr{Оксана Петришина}
\textbf{Беляев Алексей} 

Справа не тільки в комунікуванні чи написанні імені. Ви говорите російською,
думаєте нею ж, читаєте кого? Пушкіна чи Шевченка? Ахматову чи Костенко? Увесь
ваш світ - це російський світ. І ви захищаєте Україну, ризикуєте життям, але ви
будуєте в Україні другу Московщину.

\iusr{Орелі Боян}
\textbf{Беляев Алексей} я звільняю, але не Донбас, українців. В кожного своя війна.

\iusr{Pavlo Slavynskyy}
\textbf{Беляев Алексей} Львів німці залишили за 3 дні до підходу радянської армії, там ще ціла історія зі змагання ОУН та АК за прапор була. Куди ваш дідусь входив, ще раз?

\iusr{Беляев Алексей}
Краше прочитати

% -------------------------------------
\ii{fbauth.beljajev_aleksej.kramatorsk.ukraina.gorlovka.policia.mvd}
% -------------------------------------

\ifcmt
  ig https://scontent-frt3-1.xx.fbcdn.net/v/t1.6435-9/72642024_968725403504663_5722446105422069760_n.jpg?_nc_cat=102&ccb=1-5&_nc_sid=dbeb18&_nc_ohc=1tflxS3YmH0AX_wTTmY&_nc_ht=scontent-frt3-1.xx&oh=d40790092a5152bb14fa1ffb77820874&oe=618B5AA3
  @width 0.4
\fi

\iusr{Беляев Алексей}
Как-то так!!!!

\iusr{Andriy Borodavko}
\textbf{Беляев Алексей}
І - що?

\iusr{Руслана Курах}
такі, як Алєксєй, просто псують патрони українцям: і на фронті, і дірявлять тил

\iusr{Роман Волошинюк}

Російська мова в Україні і її носії це в першу чергу символ російської окупації
а носії це представники окупантів!!! І пам"ятаємо, що в Криму і лугандонії саме
"російськомовні" зрадили Україну, що в принципі не дивує!

\iusr{Ivan Matviyiv}
\textbf{Беляев Алексей} , краще б твій дід не заходив у Львів. Знаємо про радянську армію все. Не треба було нас визволяти, ніхто про це їх не просив.

\iusr{Inna Shch}
Ненароком потрапила на цей допис.
Це треба так вміти цькувати людину, яка розмовляє російською, але воює за Україну. Ще такої дурні, як тут написано не читала. Українці, агов, а чи то не люди, які вдягають вишиванку і розмовляють українською в раді вже 28років сидять і щодня зраджують Україну? Чи то не українці, їздять на заробітки до росії. Чи то не українські артисти їздять до росії на гастролі. Може вже досить? Так мовне питання болюче. Але крім того, що треба хотіти розмовляти українською, ще треба і в серці мати Україну.

\iusr{Беляев Алексей}
Дякую Вам велике за цю промову!!!

\iusr{Беляев Алексей}
\textbf{Inna Shchukina} Вони зрозуміють цю проблему коли їх дитина поїде воювати и побачать що разом з нею буде стояти побратим котрий скаже: " Привет я Саня . Я твой второй номер". "Слава Україні".

\iusr{Орелі Боян}
\textbf{Inna Shchukina} бачу аналізувати то теж не ваше. Людина вперто вважає, що москвояз в Україні - норма. А те, що захищає - індульгенція на вживання язика.
Тепер аналіз.
Якщо нащадки росіян захищають Україну, вони мають право на те, щоб тут був москвояз?
Чи захищають Україну і розуміють, що це країна українців і дотримуються мови і культури, і традицій ?
Якщо я воюю і імєю права, то питання, за що воює ? За свою нову Московію?
Друге, якби в Україні були лише українці та ті, хто поважає Україну і її право на саморозвиток, проблеми були б?
Проблема в Україні не українці, а нащадки окупантів, які вважають, що імєют права і не хочуть зрозуміти, що їх права на язик по той бік кордону, а тут їх права нав'язані кров' б, голодом і терором.
У Вас Стокгольмський синдром?
Чи не розумієте про що мова?
Крім етноменшини
московитів є і інші, але ні одна не вважає, що їх мова має бути в Україні на тих же правах, що і українська, або і замість української. Лише ці. Бо їм на підкірку прописали "імєєм права".

\iusr{Орелі Боян}
\textbf{Беляев Алексей} за що дякуєте? Що вас захистили від українців? За що воюєте?

\iusr{Орелі Боян}
\textbf{Inna Shchukina} а тепер по факту. Ви, або втратили коріння, або така ж московка, як і пан Бєляєв.

\iusr{Inna Shch}
\textbf{Орелі Боян} 

ви смішна справді))) то може приїдете до Одеси і навчите як має бути? Може до
ради будете виходити щодня і розповідати, що люди, які все життя розмовляли
російською повинні з наступної секунди перестати так говорити? Може у вас вийде
викорінити те, що впроваджувалось століттями серед українців.

\iusr{Inna Shch}
\textbf{Орелі Боян} мені все одно ваша думка. Але ви якраз добре працюєте заради Кремля, розповідаючи яка ви щира українка, а інші бруд під ногами. На все добре, з шовіністами не веду дискусій.

\iusr{Орелі Боян}
\textbf{Inna Shchukina} а Адєса . Бачила ваших. Місто манкуртів. Асобєнниє гражданє. І не українці, і не росіяни. Не пойми хто. По суті манкурти. Ви з Беляєвим знайшли одне одного.

\iusr{Орелі Боян}
\textbf{Inna Shchukina} бруд під ногами? На Кремль? З таким вінегретом до того, хто це зможе розібрати і пояснити. Не маю ні бажання, ні часу. Схоже чула. 10 жовтня 14 годин вєщало.

\iusr{Елена Норенко}
\textbf{Беляев Алексей} Вас звати ОЛЕКСІЙ! Лесик, Олесь. Така традиція вживання імені є більш притаманя для українців. Так само, якщо жінка відчуває себе українкою, то вона Надія, а не Надєжда. Ось про що йшлося)))

\iusr{Беляев Алексей}
Дірявлять таких як Ви , а я державу захищаю

\iusr{Люда Люда}
\textbf{Беляев Алексей} продаючи її москалям

\iusr{Костянтин Собіченко}
\textbf{Inna Shchukina}, 

за те що воює - цю людину не цькують. Її, якщо й цькують (хоча цього ніхто
насправді не робить), - то лише за нерозуміння того, за що, власне, вона воює.
Вона, людина ця, однією рукою відмежовується від окупантів, а іншою - їхні ж
ідеї й поширює. Як і ви, зрештою. Отака роздвоєність також ні до чого доброго
не призводить.

\iusr{Ігор Кушнірчук}
\textbf{Беляев Алексей}, таваріщь, ви захищаєте руцкій мір в Україні, а не Україну. Ви п'ята колона хуйла. І якщо ви цього не розуміє, то не означає, що це не так.

\iusr{Ігор Кушнірчук}
\textbf{Беляев Алексей}, от твій дід окупант, а ти внук окупанта й внутрішній ворог України.

\iusr{Сергій Жижко}
Пане Олексію.

За Вашу участь у боях за Україну - повага і подячність.

Але це не знімає необхідність бути патріотом України не тільки у бойових діях.

Людина, яка готова віддати життя, померти за Україну, яка бачила причини і
трагедії війни і не прийшла до висновку, що необхідно перемогти себе у
важливому - прийняти українськість у всьому, така людина ще не стала вповні
українцем.

А війна тому, що русскій мір воює за себе і міткою, знаком цього міра є русскій
язик у тому числі як жаргон совка.

\end{itemize} % }

\iusr{Беляев Алексей}
Так ділити нас не треба. ИЗ ЗА ЭТОГО И ВОЙНА У НАС.....

\begin{itemize} % {
\iusr{Леся Маланюк}
Не через московію?

\iusr{Беляев Алексей}
\textbf{Леся Маланюк} через наш розум

\iusr{Костянтин Собіченко}
\textbf{Беляев Алексей},

війна через недолугість малоросійську, через "какую
разніцу на каком язикє", чим так вміло завжди користувалась Московія. От вона й
вклала у ваші вуста оті смисли про "не треба ділити", хоча сама цим постійно і
займалась якраз, нав'язуючи українцям свій язик, щоби ті стали інакшими:
спочатку трішки не такими українцями, а потім взагалі неукраїнцями. А потім -
таке стадо вже не являє собою ніякої загрози, можна брати голими руками.

\iusr{Беляев Алексей}

Я сам з Горлівки та не був там вже 5 років так як я там "предатель", я розумію
що вже ніколи туди не повернусь , але там пройшло моє дитинство, 35 років мого
життя , там жили мої друзі ,там поховані мої родичі. І мені тепер жити з цим
все моє життя. А Ви погодились би втратити все : житло,друзів, родичів, не мати
змоги поховати рідних - заради України......

\iusr{Беляев Алексей}
А я зробив свій вибір!!! У 2014 році взяв у руки зброю і почав вирішувати чиєсь мовне питання!!!!

\iusr{Костянтин Собіченко}
\textbf{Беляев Алексей}, це добре, що взяли зброю. Але мовне питання в першу чергу треба вирішити своє.

\iusr{Орелі Боян}
\textbf{Беляев Алексей} у вас є два варіанти уникнути поділу. Перший говорити українською і нарешті зрозуміти, що тут Україна, а не НеоМатрьошкостан. Другий емігрувати з України.

\iusr{Беляев Алексей}
Хорошо судить когда спишь на своей кровати у себя дома!!!!

\iusr{Орелі Боян}
\textbf{Беляев Алексей} 

якби ви говорили українською і вся Горлівка, Хуйло б не прийшло. Ви вважаєте,
що лише постраждали? А ми всі живемо і радіємо? У всіх нас вкрадено 5 років
життя. У всіх нас гинули друзі, рідня, знайомі. Всі ми замість жити своє життя
маємо воювати. Кожен на своєму фронті. І в кожного нема спокою. Бо колись
емігранти вирішили що з України можна зробити Московію, замість зрозуміти, що
тут Україна.

\iusr{Орелі Боян}
\textbf{Беляев Алексей} 

ви знаєте чим кожен з нас мав жертвувати? А наші роди яку ціну заплатили за ваш
язик? Це для вас війна в 2014 почалася. А для нас триста з гаком років тому. А
останні сто років ми мали пекло від ваших. Тому так, ми спимо в своїх ліжках,
поки що.

\iusr{Оксана Петришина}
\textbf{Беляев Алексей} У нас війна, бо ви не вивчили українську мову!!!

\iusr{Руслана Курах}
\textbf{Беляев Алексей} 

"Хорошо судить когда спишь на своей кровати у себя дома!!!!" - ти диви, яке
нахабне.. так це через вас, защемлених стільки жервт, стільки кровопролиття, тож
платіть, паскуди, і не скигліть! воно ще й розмахує, замість посмертного
відчуття провини... тварюка така.. пішов на родіну, в гробу я бачила таких
"защітнікаф"! потвора кінчена.

\iusr{Беляев Алексей}
Паскуди твої діти якщо ти так говориш

\iusr{Тетяна Бевська}
\textbf{Беляев Алексей} Ви пішли захищати Батьківщину (нехай і малу - Горлівку) від російського окупанта, а не мовне питання вирішувати.

\iusr{Беляев Алексей}
Саме так!!!!

\iusr{Орелі Боян}
\textbf{Беляев Алексей} 

а хто від таких як ви захистить Україну? Адже таких, як ви-багато. І ви своїми
руками вбиваєте Україну. Очі відкрийте, нарешті. Ви захищаєте московський
гібридний світ і вбиваєте Україну. Нема України москвоязичної, це-колонія
московитів. Україна-українська. Все інша окупована Україна. На даний момент
окупували майже всю територію. І такі як ви навіть не усвідомлюєте, що нащадки
вбивць українців, а не освободители. І доки ви це не зрозумієте, доти Ви проти
України.

\iusr{Орелі Боян}
\textbf{Тетяна Бевська} а він який окупант? Не російський? Просто він - внутрішній окупант. А то зовнішні. А все це разом - гібридна війна.

\iusr{Люда Люда}
\textbf{Беляев Алексей} не із-за цього, подумайте, згадайте..

\iusr{Ігор Кушнірчук}
\textbf{Беляев Алексей}, о московитські методички пішли в рух. Хто б сумнівався, що ви споживач московитського лайна.

\iusr{Yuriy Kozik}
\textbf{Беляев Алексей} тоді зміни своє імя на фб, бо для мене це написано ворожою москальською мовою. Чи ти саме так себе позиціонуєш?

\iusr{Беляев Алексей}
Саме так для Вас шановні!!!

\iusr{Беляев Алексей}
И писать в ФБ свое имя я буду как свободный человек !!!! От маразма и дебилов...

\iusr{Ігор Кушнірчук}
\textbf{Беляев Алексей}, от ти й проявив себе московите. Ти і є ворог України.

\iusr{Ігор Гримайло}
\textbf{Беляев Алексей}
Війна не через мову, а через Путіна, через його найманців, через його поставки зброї та боєприпасів в Донбас.
Невже це для вас новина?

\iusr{Галя Волошин}
\textbf{Фрейр Бинтмакерь} а ваше ім'я якою написане?

\iusr{Галя Волошин}
\textbf{Беляев Алексей} 

мовним питанням путін тільки прикривається для виправдіння воєнної агресії. І
ще мовне питання використовують, щоб розсварити нас і довести до громаддянської
війни.

Насправді плани у нього великі і це зовсім не захист язика))).

Біда в тому, що ви "російськомовні українці" ненавидите Україну і українську
мову більше за того самого путіна. Бо як інакше пояснити той несамовитий
спротив переходу на українську мову? Варіантів лише два:

\begin{itemize}
  \item 1- ненависть до України
  \item 2- нездатність вивчити мову, тупість. @igg{fbicon.shrug} 
\end{itemize}

\iusr{Тетяна Бевська}
\textbf{Беляев Алексей} 

а коли станете захисником Батьківщини - України, то зникне питання мови. В
Польщі - польська, в Чехії - чеська, в Угорщині - угорська, а в Україні тільки
українська.

\iusr{Ігор Кушнірчук}
\textbf{Галя Волошин}, моє ім'я українці бачать, а решті не обов'язково.

\iusr{Галя Волошин}
\textbf{Фрейр Бинтмакерь} я українка. Без ваших викрутасів проживу і українкою бути не перестану

\iusr{Сергій Тростинський}
\textbf{Беляев Алексей} 

війна не через те що путін гіркіна привів? Через те що всетаки ділять нас? А ви
не діліться, об'єднайтесь навколо державної мови, і все, згідно вашої теорії війна
закінчиться.

\end{itemize} % }

\iusr{Беляев Алексей}
Усі ми діти ГОСПОДНІ!!!

\begin{itemize} % {
\iusr{Орелі Боян}
\textbf{Беляев Алексей} українці діти Божі. Над ними нема господина. Лише Бог.
\end{itemize} % }

\iusr{Беляев Алексей}
Кому як зручніше....

\iusr{Беляев Алексей}
Власно мені подобається і так і так...

\begin{itemize} % {
\iusr{Руслана Курах}
пішов за порєбрік. ВОРОГ

\iusr{Микола Владзімірський}
та звичайне троленятко воно

\iusr{Беляев Алексей}
Из-за таких как Вы война идёт и люди гибнут каждый день. Писаки

\iusr{Беляев Алексей}
Приїдь та подивись ворогу в очі....

\iusr{Беляев Алексей}
Писать все мастики а когда 152 рядом упадет обсераются все одинаково

\iusr{Ірина Мельник-Демченко}
\textbf{Микола Владзімірський} цікаво, їх у тролошколі всіх вчать такі дурнуваті ави ставити?

\iusr{Надія Синявська}
\textbf{Беляев Алексей} оце сів путін, почитав фб, і -,,ти диви, які писаки!,, а піду но я на них війною, ато бач, пишуть! так, Олексію?

\iusr{Костянтин Собіченко}
\textbf{Беляев Алексей}, і тут урівняв - знову какая разніца. А чому москвин, коли впаде 152, обсирається лише своєю мовою? Йому чомусь обирати не доводиться. Чому?
\end{itemize} % }

\iusr{Олекс Музичко}

Натомість відбувається гірше: навіть українки соромляться імені ГАННА -
поголівно Анни. Чоловіки вже соромляться імен Євген - поголівно Євгенії.
Соромляться імені Микита. Не теоретизую, а бачу по студентах. Тотальне
зросійщення. Ця нещасна недодаержава немає жодних шансів. Ну то може турки,
вєтнамці, китайці краще впораються на цій благодатній землі. Бо мають гідність.

\begin{itemize} % {
\iusr{Анна Брикова}
\textbf{Музичко Олександр} 

ну, по-перше, Анна і Ганна - це фонетичні варіанти одного і того самого імені,
і жоден з них не суперечить українському правопису. По-друге, чому Ви вирішили,
що це сором? Я, наприклад, переконана, що сором - це називати свою Батьківщину
"недодержавою".

\iusr{Орелі Боян}
\textbf{Музичко Олександр} 

дивлячись де. Західна Україна завжди Анни. Підозрюю, що причина в
католицизмі...

\end{itemize} % }

\end{itemize} % }
