% vim: keymap=russian-jcukenwin
%%beginhead 
 
%%file 15_04_2021.fb.respublikalnr.4.nabat
%%parent 15_04_2021
 
%%url 
 
%%author 
%%author_id 
%%author_url 
 
%%tags 
%%title 
 
%%endhead 

\subsection{ГАЗЕТА "РЕСПУБЛИКА" (№15, 2021г).  СКОРБНАЯ ДАТА.  ПАМЯТИ НАБАТ}
\label{sec:15_04_2021.fb.respublikalnr.4.nabat}

\ifcmt
  pic https://scontent-mxp1-2.xx.fbcdn.net/v/t1.6435-9/173539825_122565626590212_7098454849306185995_n.jpg?_nc_cat=100&ccb=1-3&_nc_sid=b9115d&_nc_ohc=7OxFbK5uK6wAX9D6ahH&_nc_ht=scontent-mxp1-2.xx&oh=b68965c69c807ccc22a17553c96d836c&oe=609BF9A7

	pic https://scontent-mxp1-2.xx.fbcdn.net/v/t1.6435-9/174689995_122565579923550_3346666868770520063_n.jpg?_nc_cat=103&ccb=1-3&_nc_sid=b9115d&_nc_ohc=MOPjJBQaGtAAX_L3CHy&_nc_ht=scontent-mxp1-2.xx&oh=d65615f83891ba006db64e193105fbf6&oe=609FA809

	pic https://scontent-mxp1-2.xx.fbcdn.net/v/t1.6435-9/174085052_122565589923549_7459144936619885741_n.jpg?_nc_cat=104&ccb=1-3&_nc_sid=b9115d&_nc_ohc=SI-D2Ppri2UAX8Z9Eu2&_nc_ht=scontent-mxp1-2.xx&oh=452f3f35cb6ef148626cef25cb3c142c&oe=609EE89F
	
	pic https://scontent-mxp1-2.xx.fbcdn.net/v/t1.6435-9/173539825_122565573256884_6726944483010712166_n.jpg?_nc_cat=110&ccb=1-3&_nc_sid=b9115d&_nc_ohc=OSRJP6Qn60wAX8aKfM3&_nc_ht=scontent-mxp1-2.xx&oh=4bc955541547c1ec406eca745e55d152&oe=609EE2B9
\fi


Для Донбасса 14 апреля – черная дата. В этот день семь лет назад киевский режим дал отмашку жестокому, кровавому, беззаконному наступлению вооруженных нацистских бандитов на мирное население, на войну против тогда еще своих сограждан, цинично прикрываемую формулировкой «антитеррористическая операция». В результате бесчеловечной агрессии, проходящей под покровительством Запада, сотни тысяч судеб нарушены, более двух тысяч жизней наших земляков – прерваны. В том числе 35 детей.
В память о них, невинно убиенных, а также сложивших головы защитниках Республики 14 апреля в ЛНР установлен как День жертв украинской агрессии. В преддверии этой даты в рамках официальной поездки в Стаханов Леонид Пасечник почтил память жителей города, сложивших головы в этой войне. Вместе с Главой Республики к Мемориалу погибшим бойцам ополчения и мирным жителям, ставшим жертвами украинской агрессии, возложили цветы представители руководства Республики и города, члены Общественной палаты ЛНР, активисты молодежных организаций. В Почетный караул у мемориала выстроились участники военно-патриотического движения «Молодая гвардия – Юнармия».
В седьмую годовщину начала украинской военной агрессии в столице Луганской Народной Республики митинг-реквием состоялся у Мемориального комплекса «Не забудем! Не простим!» в поселке Видное – на месте массового захоронения жителей Луганска, погибших блокадным летом 2014 года. 
«Каждая утрата невосполнима. И люди, которые в этом виновны, особенно в смерти невинных детей, обязаны понести наказание. Эти люди никогда не будут спать спокойно. В верховной раде Украины зарегистрирован законопроект, в котором утверждается, что Россия убивала детей Донбасса. Это говорит о том, что те, кто на самом деле виновен в этом, понимают, что время расплаты все равно придет, и пытаются грубо, цинично оправдаться. Но это не оправдание – это страх перед неизбежным возмездием», – констатировала советник Главы ЛНР Марина Филиппова.
Была объявлена минута молчания, затем корзину с цветами от Главы ЛНР установили к подножию памятника «Не забудем! Не простим!» военнослужащие роты почетного караула Народной милиции ЛНР. 
«Это наша общая боль, боль, которую мы переживаем по сей день. Ее невозможно забыть. Это то, что мы видим с вами своими глазами, это то, что мы чувствуем внутри и переживаем. За все эти годы то, что творит Украина, нельзя назвать человечным, и наша задача ту правду, которую мы наблюдаем, передавать из поколения в поколение», – подчерк-нул Председатель Народного Совета ЛНР Денис Мирошниченко.
В дань памяти обо всех погибших к мемориалу возложили цветы представители Народного Совета ЛНР, Правительства Луганской Народной Республики, Народной милиции ЛНР, Общественной палаты, общественных движений, молодежных и ветеранских организаций.
«Хотелось бы высказать слова соболезнования семьям всех погибших, в том числе однополчанам, которые лежат в этой братской могиле. Это ужасная боль, но, тем не менее, сегодня семь лет нашей Республике, она продолжает развиваться и процветать. Мы видим настроение народа, в том числе и на подконтрольной Киеву территории: все ждут, когда придет Донбасс и освободит их от этого режима», – сказал депутат Народного Совета ЛНР, президент мотоклуба «Ночные волки» отделение Донбасс Виталий Кишкинов.
Памятные мероприятия состоялись и в столичном парке имени Щорса. От имени и по поручению Леонида Пасечника к памятному знаку «Погибшим детям Луганщины» была установлена корзина с цветами. Почтили память самых юных жертв украинской агрессии представители республиканского руководства, общественности, активисты общественной организации «Патриотическая ассоциация Донбасса».
«Сегодня 2557-й день войны, войны настоящей, страшной, кровопролитной, в течение которой гибнут старики, женщины, безвременно обрываются детские жизни, – отметила заместитель Министра иностранных дел ЛНР, председатель Специальной комиссии при Главе ЛНР по сбору и фиксации военных преступлений военно-политического режима Украины против населения ЛНР Анна Сорока. – В этих условиях тяжело говорить о справедливости, но мы не имеем права сдаваться. Мы здесь, мы обязаны бороться, делать жизнь в Республике лучше и говорить на весь мир о тех преступлениях, которые совершает Украина против нас, жителей Донбасса».
Как и ежегодно, в парке у памятного знака собрались семьи с детьми, малыши-детсадовцы с воспитателями, учащиеся ближайшей школы со своими учителями. Они принесли свои книги и игрушки тем, кто уже никогда не сможет воспользоваться ими... 
У мемориала жителям Луганщины, павшим от рук карателей-националистов из ОУН-УПА, также было в этот день многолюдно. Памятную акцию здесь провели активисты проектов «Мы помним!», «Волонтер», «Доброволец», «Молодая гвардия» ОД «МЛ» совместно с коллективом Института гражданской защиты ЛГУ имени Владимира Даля и Луганского казачьего кадетского корпуса имени маршала авиации Александра Ефимова. Представители республиканского и городского руководства, Федерации профсоюзов ЛНР, общественных объединений возложили цветы и почтили минутой молчания память всех погибших... 
Светлана ЮРОВА, 
фото Ксении ЩЕРБИНЫ
ГАЗЕТА "РЕСПУБЛИКА" (№15, 2021г).
#газета #республика #жертвы_украинской_агрессии #скорбная_дата
