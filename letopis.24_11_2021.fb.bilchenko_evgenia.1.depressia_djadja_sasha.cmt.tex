% vim: keymap=russian-jcukenwin
%%beginhead 
 
%%file 24_11_2021.fb.bilchenko_evgenia.1.depressia_djadja_sasha.cmt
%%parent 24_11_2021.fb.bilchenko_evgenia.1.depressia_djadja_sasha
 
%%url 
 
%%author_id 
%%date 
 
%%tags 
%%title 
 
%%endhead 
\subsubsection{Коментарі}

\begin{itemize} % {
\iusr{Анатолий Миронов}

\headCenter{Лютая графомань}

\begin{multicols}{2}
\obeycr
Нет - место встречи изменить нельзя,
как на духу об этом заявляю.
Тот ноунейм, кому вы, как ферзя,
сбежавшая из "приторного рая".
\smallskip
Нельзя никак, вся жизнь пустяк...
Об этом знал когда-то дядя Саша.
(о падших женщинах, о том, что мир бардак),
что правда далеко увы не наша.
\smallskip
О безразличной безнадёге спин,
о гиблом мельтешении по кругу.
Как в песенке забытой группы Queen
"Show must go on"... " не открывай фрамугу".
\smallskip
Простите мне невольный, злой сарказм,
ноябрь надежд остатки добирает.
Но будет срок и будет даже час -
когда мелькнёт удачи нить иная.
\smallskip
И дядя Саша улыбнётся вдруг,
несвоственной мгновению улыбкой.
Так, будто скажет: "Что же вы мой друг?
Дерзайте! Будущность так зыбка..."
\smallskip
И кто бы что теперь не говорил,
и вопреки недобрым, едким взглядам.
На реплику глумящейся судьбы
ответите - спасибо, но не надо!
\smallskip
Пока же неизбывная стезя
влечёт вальсировать по кругу.
То место встречи - изменить нельзя.
Не совершай... не открывай фрамугу.
\smallskip
( буковки символизирующие грустную улыбку)
\restorecr
\end{multicols}

\begin{itemize} % {
\iusr{Евгения Бильченко}
\textbf{Анатолий Миронов} про фрамугу круть.

\iusr{Анатолий Миронов}
\textbf{Евгения Бильченко} дядя Миша (Вэй) учил нас бездарей, что окончание текста должно быть чуть более ярким, нежели основная ткань изложения - "зарубите это себе на носу!" Я это очень хорошо запомнил. Да и дядя Иосиф сам не чурался подобных приёмов... Думаю, что он мне простит этот вольный недоплагиат. На фоне того, как полоскали его "Не выходи из комнаты" - я ангел и душка))
Понимание "как сделать круть" и сделать "эту самую круть" - две большие разницы. ))


\iusr{Евгения Бильченко}
\textbf{Анатолий Миронов} дядя Миша вот ничего нового не открыл: о недопустимости смазывания концовок меня лет в 17 Марина Ивановна учила на примере критики Анны Андреевны поздней. Там бабы сrались не на шутку, прости, Господи. Все мои неудачи в текстах - из-за сдувания к концу. Или уже доводи сдувания до полного издоха: тогда тоже работает.

\iusr{Анатолий Миронов}
\textbf{Евгения Бильченко} 

"Поставили на вид - уйти и застрелиться... вдох-выдох ***ть осечка"

(два с половиной года назад меня учили считать количество гласных в строке)

P.s.

Я не знаю что у фб за фокусы с доставкой сообщений. Потому оставлю это здесь.
Срок жизни у подобных текстов крайне непродолжителен.

\headCenter{\enquote{на правах спама ч. 2}}

\begin{multicols}{2}
\obeycr
Деревянная лошадка,
сабелька из пропилена.
Пустота в сухом остатке
критики неоткровенной.
\smallskip
Мельниц высится громада
над болотами цинизма.
Это в общем - то, что надо,
будет отповедь и схизма.
\smallskip
Нитью днк-цепочки
самаю себя продела,
чёрных точек резкий прочерк,
за пределы передела.
\smallskip
В ту купель Любви и Света,
где прибудет Мира Логос.
Дерзновение поэта
с серебром ровняет волос.
\smallskip
Явственней тысячекратно,
чувственней ежесекундно.
Неизбывная потрата
нас влекущая подспудно.
\smallskip
Несмешной бурлеск упадка,
буквиц вечная дилемма.
Деревянная лошадка,
сабелька из пропилена.
\smallskip
Для Жени Б.
\restorecr
\end{multicols}

\end{itemize} % }

\iusr{Алексей Костромин}

Не такой уж он и «дядя», т.к. ему было всего 37, но и просто «Сашей» не
назовёшь)

\begin{itemize} % {
\iusr{Евгения Бильченко}
\textbf{Алексей Костромин} он "дядя Саша", это я безжалостно у \textbf{Александр Кабанов} сперла, как "родню" у \textbf{Захар Прилепин}

\iusr{Захар Прилепин}
\textbf{Евгения Бильченко} что-то мне подсказывает, что и «дядю» Кабанов у меня спер. В Киеве в гостях у Кабанова я всегда звал так покойного Сашу Абрамыча Кабакова. «Дядь Саш». Кабанов всё слышит. Всё сечёт.

\iusr{Александр Кабанов}
\textbf{Евгения Бильченко} на свете нет меня виновней.


\iusr{Евгения Бильченко}
\textbf{Александр Кабанов} у тебя же гениальный текст про дядю Сашу. Вот не поленюсь...

\iusr{Евгения Бильченко}
\textbf{Захар Прилепин} Он в заложниках, вызывай спецгруппу.

\iusr{Евгения Бильченко}

Саша, без шуток, текст про груди и задницу у девушки оной, от которого
протащился даже мой муж (и я его к тебе ревновала): не могу найти его в сети

\iusr{Евгения Бильченко}

Пока вот этот АК: 

\begin{multicols}{2}
\obeycr
Говорят, что смерть — боится щекотки,
потому и прячет свои костлявые пятки:
то в смешные шлёпанцы и колготки,
то в мои ошибки и опечатки.
\smallskip
Нет, не все поэты — п*здострадальцы, —
думал я, забираясь к смерти под одеяльце:
эх, защекочу, пока не сыграет в ящик,
отомщу за всех под луной скорбящих —
у меня ведь такие длииинные пальцы,
ох*енно длинные и нежные пальцы!
\smallskip
Но, когда я увидел, что бедра её — медовы,
грудь — подобна мускатным холмам Кордовы,
отключил мобильник, поспешно задёрнул шторы,
засадил я смерти — по самые помидоры.
\smallskip
...Где-то на Ukraine, у вишнёвом садочку —
понесла она от меня сына и дочку,
в колыбельных вёдрах, через народы,
через фрукты-овощи, через соки-воды...
\smallskip
Говорят, что осенью — Лета впадает в Припять,
там открыт сельмаг, предлагая поесть и выпить,
и торгуют в нём — не жиды, ни хохлы, не йети,
не кацапы, не зомби, а светловолосые дети:
\smallskip
у девчонки — самые длинные в мире пальцы,
у мальчишки — самые крепкие в мире яйцы,
вместо сдачи, они повторяют одну и ту же фразу:
\smallskip
«Смерти — нет, смерти — нет,
наша мама ушла на базу...»
\restorecr
\end{multicols}

\iusr{Александр Кабанов}
\textbf{Евгения Бильченко} я понял о чём ты. Да, есть такой текст.

\iusr{Евгения Бильченко}
Видишь, груди у меня, дядя Саша (с) - не помню дальше, но, бил же ж ,классно же ж

\iusr{Александр Кабанов}
\textbf{Евгения Бильченко} а кто спорит?)

\iusr{Евгения Бильченко}
\textbf{Александр Кабанов} 

Сашенька, ты можешь мне его дать, а? Меня он поднимает вместо гидазепама.

\iusr{Александр Кабанов}
\textbf{Евгения Бильченко}

*****

\begin{multicols}{2}
\obeycr
Замерзает за окном простокваша,
это что там у тебя, юность наша:
снизу - попка у меня, дядя саша,
сверху - груди у меня, дядя саша.
\smallskip
При честном, но очень лживом народе,
я сидел и я стоял на природе,
весь в смирительной на совесть гавайке,
шмыгал носом, а елдой ставил лайки.
\smallskip
Надо мной расцвел грибок из брезента,
подо мной садились пчелы на стебли,
я хотел быть двойником президента,
серединою: от воблы до ебли.
\smallskip
Ночью днепр вставляет баржи в петлицы,
замирают над волной по эвклиду,
а затем, безмолвно падают птицы,
образуя под водой пирамиду.
\smallskip
Слева - облако, а справа - фазенда,
и закат неотличим от восхода,
вот и стал я двойником президента,
а народ мой - двойником у народа.
\smallskip
12.02.2020
\restorecr
\end{multicols}

\iusr{Евгения Бильченко}
\textbf{Александр Кабанов} вот оно, вот оно, на тую намотано. Обожаю, спасибо.
\textbf{Алексей Костромин} \textbf{Захар Прилепин}

\end{itemize} % }

\iusr{Владислав Сушков}
Итальянская площадь ))) хорошо )))

\begin{itemize} % {
\iusr{Евгения Бильченко}
\textbf{Владислав Сушков} искусств, к ней улица идёт Итальянская

\iusr{Владислав Сушков}
А, да ))) склероз )))
\end{itemize} % }

\iusr{Тереза Славович}
А, что там у Стуса нужно было переводить?

\begin{itemize} % {
\iusr{Евгения Бильченко}
\textbf{Тереза Славович} то, что на семинаре, а с семинара я сбежала с Коровиным в БГУ стихи жарить и водку пить.

\iusr{Тереза Славович}
\textbf{Евгения Бильченко})) коровин, это нормальный выбор))


\iusr{Евгения Бильченко}
\textbf{Тереза Славович} да, я тоже так подумала, отпив с ним ночь
\end{itemize} % }

\iusr{Александр Лашманкин}
дядя был Вася



\end{itemize} % }
