% vim: keymap=russian-jcukenwin
%%beginhead 
 
%%file 19_11_2018.fb.lesev_igor.1.prochj_ot_moskvy
%%parent 19_11_2018
 
%%url https://www.facebook.com/permalink.php?story_fbid=2179572508740507&id=100000633379839
 
%%author_id lesev_igor
%%date 
 
%%tags moskva,politika,poroshenko_petr,rossia,ukraina
%%title Прочь от Москвы
 
%%endhead 
 
\subsection{Прочь от Москвы}
\label{sec:19_11_2018.fb.lesev_igor.1.prochj_ot_moskvy}
 
\Purl{https://www.facebook.com/permalink.php?story_fbid=2179572508740507&id=100000633379839}
\ifcmt
 author_begin
   author_id lesev_igor
 author_end
\fi

Прочь от Москвы.

Это не просто тупой предвыборный лозунг Петра Порошенко. Это еще одной клеймо
его политической команды, которая непременно проиграет выборы, даже если в них
будет участвовать только один Порошенко.

Вообще, предлагать подобный лозунг в независимой стране – члену ОНН признанному
абсолютно всеми правительствами на планете – это комплекс потусторонней
неполноценности.

\ifcmt
  ig https://scontent-frt3-1.xx.fbcdn.net/v/t1.6435-9/46447631_2179572448740513_8705604005516017664_n.jpg?_nc_cat=104&ccb=1-5&_nc_sid=730e14&_nc_ohc=3PkHGkcRXqQAX-F2xks&_nc_ht=scontent-frt3-1.xx&oh=5c9419f6b9d2dccd0fafe73202a3000c&oe=61B6D407
  @width 0.4
  %@wrap \parpic[r]
  @wrap \InsertBoxR{0}
\fi

Скажем, когда Египет и Сирия объединились в Объединенную Арабскую Республику,
теоретически в Дамаске мог звучать (возможно, даже и звучал) лозунг «Прочь от
Каира». Подобное может звучать в йеменском Адене, который вдруг опять бы
захотел стать независимым от Саны. Такое на выборах может звучать где-то на
Тайване или на недавнем референдуме в Новой Каледонии (прочь от Парижа), или
даже в Пуэрто-Рико (прочь от Вашингтона).

Украина с 1991 года бодро шагает прочь от Москвы и продавать очевидную
очевидность настолько тупо, что этого никогда не додумались делать откровенно
националистические конторы, вроде «Свободы». Это все равно, как ставить борды
«Пусть вода будет мокрой» или «Табак – основа сигарет».

Но это так, частность, показывающая хрестоматийную тупость команды президента.
Мне даже интересно, как человек, добившийся грандиозных успехов в бизнесе, что
априори подразумевает умение блестяще подбирать исполнителей и руководить
крупными коллективами, так бездарно окружает себя самыми фантастическими
идиотами.

Хотя тут нужно еще учитывать особенности «работы с кадрами» самого Порошенко.
Человек не терпит критики, даже очень ласковой. Вроде, вы сегодня не гениально
выступили, а только восхитительно. А ведь в обойме Порошенко есть люди, которые
достаточно успешно себя реализовывали ДО контакта с первым лицом в самых разных
направлениях. Но все вместе они превращаются в коллективное бессознательное,
генерируя из всех вариантов действий – самые проигрышные.

Возьмем для примера ту же избирательную стратегию команды Порошенко. Она уже
абсолютно понятна и вызывает только недоумение. Это как два гроссмейстера
играют партию, но один начинает ходить как шестиклашка – знает только как
передвигать фигуры. Сначала это у оппонента вызывает оторопь. Что это, мать
его? Защита Филидора или вариант Чигорина? Но нет же, просто оппонент третий
раз в жизни играет в шахматы.

«Гроссмейстеры» президента почему-то внушили ему, что победу можно добыть,
натянув на себя одеяло ультра-наци. Вера, мова и почему-то армия вместо
классической нации. В общем-то, это нормальные лозунги для домайданной Украины,
в которой всегда выборы проходили по водоразделу условно «про-российскости» и
«про-европейскости». Толпа жрала эту пустоту и партии/кандидаты особо ничего не
предлагая предметно-осязаемого, впаривали от выборов к выборам одно и то же
говно. «Мы защитим украинский язык»! «А мы – русский»! Главное нарезать округа,
согласовать «правильную оппозицию», и все – вперед за победой.

Но Порошенко своими же руками эту халяву и разрушил. Весь условный
«про-российский» политико-бомонд он загнал в маргинез. Это при том, что реально
пророссийские политики у нас и при Януковиче были в маргинезе. Сидел в тюрьме
Марков. «Русское единство» Аксенова в Крыму (в Крыму!) набрало в 2010-м аж 4\%.
Ру-посольство в Украине вообще нихера никогда не делало. Это даже не 10\% от
того, как ведет себя в Украине пацанва с Танковой.

У Порошенко могло быть две успешные стратегии. Первая – самая очевидная. Чуть
полиберальничать, дать эфирную свободу «реваншистам» с востока. Открыть эфиры
для Бойко/Новинского/Мураева, создав иллюзию, что победа с востока возможна. Но
на Банковой реванша боятся настолько, что даже играть в эту партию на
управляемо-теоретическом уровне, зассали.

Вторая стратегия – самому Порошенко на фоне буром идущей Юли, стать чуть-чуть
ближе к востоку. На совсем ничего не обязывающую херню. В Харькове где-то
перейти на русский. Веночек павшим воинам ВОВ где-то положить. Чуть дворовых
наци публично приструнить. Ну и мотивировать того же восточного кандидата, не
попадающего во второй тур, формально/не формально поддержать действующего
президента. Иными словами, донести, что я миндальный к вам – второму сорту
русскоязычных соотечественников – но за мной идут совсем звери.

Учитывая специфичность всей команды Порошенко, уверен, что и эта стратегия в их
исполнении хрен бы сработала. Но эта лотерея хотя бы при купленном билете. А
так у Порошенко выстроили кампанию, которая упирается в 10-15\% отмороженного
электората. В агрессивное, пассионарное, деятельное, но меньшинство. Которое
еще и в своей массе самого Порошенко за своего не признает.

И все это на фоне социально-экономической и коррупционной жопы. Что особо
смешно, беспрецедентно тупейшая кампания президента создает теперь неожиданные
проблемы и главному фавориту гонки – Тимошенко. Потому что, если Порошенко не
попадает во второй тур, там уже появляются не самые приятные варианты для самой
Юли.

\ii{19_11_2018.fb.lesev_igor.1.prochj_ot_moskvy.cmt}
