% vim: keymap=russian-jcukenwin
%%beginhead 
 
%%file 08_12_2022.fb.ohmatdet.bolnica.1.reanimacia_pisok_dity
%%parent 08_12_2022
 
%%url https://www.facebook.com/ndslohmatdyt/posts/pfbid0JwQNtxgkJk7urgzHKkPyvZ72CDmd3SS8NNAbGG9EszXhubUQakA85QK1jsJ53cmtl
 
%%author_id ohmatdet.bolnica
%%date 
 
%%tags 
%%title Вікна реанімації затулені мішками з піском: як під час війни лікують недоношених дітей в Охматдиті
 
%%endhead 
 
\subsection{Вікна реанімації затулені мішками з піском: як під час війни лікують недоношених дітей в Охматдиті}
\label{sec:08_12_2022.fb.ohmatdet.bolnica.1.reanimacia_pisok_dity}
 
\Purl{https://www.facebook.com/ndslohmatdyt/posts/pfbid0JwQNtxgkJk7urgzHKkPyvZ72CDmd3SS8NNAbGG9EszXhubUQakA85QK1jsJ53cmtl}
\ifcmt
 author_begin
   author_id ohmatdet.bolnica
 author_end
\fi

Вікна реанімації затулені мішками з піском: як під час війни лікують
недоношених дітей в Охматдиті💔

Від початку війни неонатологічна служба Охматдиту не зупиняла свою роботу. 

\ii{08_12_2022.fb.ohmatdet.bolnica.1.reanimacia_pisok_dity.pic.1}

До нас доставляли дітей з пологових будинків столиці, Київської області та
сусідніх регіонів. До Охматдиту прибували також діти з територій Чернігівщини,
Сумщини, Київщини, які тимчасово знаходилися під окупацією. Лише за перші два
місяці війни неонатологічна служба прийняла понад 100 нових пацієнтів. Усім їм
була надана необхідна висококваліфікована медична допомога. Зараз лікарі
продовжують рятувати малят та роблять все можливе, щоб вони були у безпеці🙏🏻

\ii{08_12_2022.fb.ohmatdet.bolnica.1.reanimacia_pisok_dity.pic.2}

\enquote{У нас лікуються новонароджені дітки з усієї України. Під час війни ми не
зупиняли свою роботу та не відмовили в жодній з госпіталізацій. Наші хірурги
продовжували оперувати дітей, команда лікарні працювала як злагоджений
механізм. Ми надаємо медичну допомогу недоношеним діткам. Найменший наш пацієнт
важив 650 грамів. Зараз ми організували свою роботу так, щоб максимально
убезпечити дітей та медичний персонал}, — зауважила Ніконова Лариса Василівна,
завідувачка відділення інтенсивної терапії новонароджених НДСЛ \enquote{Охматдит}. 🙌

\ii{08_12_2022.fb.ohmatdet.bolnica.1.reanimacia_pisok_dity.pic.3}

Всі діти мають проживати дитинство у безпеці, з вільним доступом до медичної
допомоги. А поки в нашій країні війна, кожен з нас робить все можливе на своєму
фронті.🇺🇦

Фото: \href{https://www.facebook.com/gettyimages/}{Getty Images}

%\ii{08_12_2022.fb.ohmatdet.bolnica.1.reanimacia_pisok_dity.orig}
%\ii{08_12_2022.fb.ohmatdet.bolnica.1.reanimacia_pisok_dity.cmtx}
