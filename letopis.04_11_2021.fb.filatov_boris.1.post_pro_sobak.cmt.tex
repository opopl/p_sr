% vim: keymap=russian-jcukenwin
%%beginhead 
 
%%file 04_11_2021.fb.filatov_boris.1.post_pro_sobak.cmt
%%parent 04_11_2021.fb.filatov_boris.1.post_pro_sobak
 
%%url 
 
%%author_id 
%%date 
 
%%tags 
%%title 
 
%%endhead 
\subsubsection{Коментарі}
\label{sec:04_11_2021.fb.filatov_boris.1.post_pro_sobak.cmt}

\begin{itemize} % {
\iusr{Алексей Воронец}

Уважаемый Борис Филатов, пожалуйста, вразумите коммунальщиков. На улице +19, а
отопление и не думают отключать. ПРошу, защитите нас от космических платежек за
ненужную, насильно навязанную услугу. Особенно с учетом чревзычайной ситуации с
газом в Европе. Экономить нужно каждый куб.

\begin{itemize} % {
\iusr{Borys Filatov}
\textbf{Алексей Воронец} 

Скажите, вы нормальные? Нормальные или нет?

Вы ж все бегали и еще неделю назад кричали: «Включи отопление! Мы мёрзнем !»

Я ж об этом только тут ТРИ раза уже писал, на каждом углу говорил и в каждом
телевизоре: «Потерпите немного, люди дорогие, носочки оденьте, свитера
потеплее, цена на газ запредельная»

Читать тут: 

\href{https://www.facebook.com/100002157183088/posts/4479461452135717/?d=n}{%
Ничего не меняется. Фейсбук напомнил - Экзистенциальная бессмысленность бытия, Borys Filatov, facebook, %
27.10.2021%
}

Или тут: 

\href{https://www.facebook.com/100002157183088/posts/4459358090812720/?d=n}{%
У меня самая лучшая работа. Она не позволяет утратить веру в человечество, %
Borys Filatov, facebook, %
21.10.2021%
}

А теперь «выключи отопление, на улице тепло»?

Оно так не работает.

НЕ РАБОТАЕТ, бл@ть!!!

Централизованное отопление это система! Гребанная СИСТЕМА, которую нельзя
ВКЛ/ВЫКЛ!!!

Она один раз запускается, развоздушивается, подаётся давление и пр.пр. и также
ОДИН РАЗ выключается.

Вы издеваетесь? Издеваетесь, да?!!!

\iusr{Юрий Нефёдов}
\textbf{Borys Filatov} Вечером сыну обьяснял как сейчас патриотично не ходить в школу 4ю неделю сидят дома.
с. Дубовики Васильковский р-н.
Днепр. обл.
котельная школы на газу.
Говорю что дорого топить школу. сидите дома.
а тут

\iusr{Victoria P. Tkalich}
\textbf{Borys Filatov}, как же я Вас понимаю. Каждый раз объяснять элементарные вещи это жесть.

\iusr{Тетяна Барабаш}
\textbf{Borys Filatov} Борис Альбертович, можно газ выключить, а насосы будут гонять холодную/тёплую воду и не нужно потом развоздушивать систему при включении газа. Ваши подчинённые, почему-то, от Вас это скрывают.

\iusr{Александра Новак}
\textbf{Алексей Воронец} Потерпите, на следующей неделе похолодает..  @igg{fbicon.face.smirking} 

\iusr{Borys Filatov}
\textbf{Тетяна Барабаш} пожалуйста, умоляю, давайте без ненужных советов!!!
Просто умоляю. Мы без Вас знаем как прикручивать газ на минимум и гонять воду в системе

\iusr{Инна Чередниченко}
\textbf{Oleksiy Voronets} Пи.,ец! Вы правда думаете, что мэр сидит с кнопками играется? Кому включить, а кому выключить?

\iusr{Инна Чередниченко}
\textbf{Oleksiy Voronets} Де світлини? Ви хто?

\iusr{Irina Kolesova}
\textbf{Алексей Воронец} 

если в вашем доме есть счетчик, просто прикрутите его. сами создаете проблему. с
понедельника идет похолодание и лето заканчивается. мозги включите-вы, что
думаете-захотел вкл захотел выкл., маразм полный...


\iusr{Irina Kolesova}
\textbf{Borys Filatov} не обращайте внимание,идиотов в стране хватает

\iusr{Инна Чередниченко}
У нас в больнице тепло приезжайте не вакцинированные!!!!!

\iusr{Irina Kolesova}
\textbf{Тетяна Барабаш} вы в своем уме?

\iusr{Андрей Федорук}
\textbf{Алексей Воронец} организуйте ОСМД, ставьте прибор учёта и крутите себе сами как хотите

\iusr{Galina Garaschenko}
\textbf{Borys Filatov} Не знаю, кому жарко, а в нашем доме до сих пор никакие пуско-наладочные работы так и не начинались. И так каждый год. На стенах уже иней. На следующей неделе уже будет минусовая температура на улице. Ул. Ю. Савченко, д.6Б. Управляющая компания Экология Д.

\iusr{Віктор Боговін}
\textbf{Oleksiy Voronets} насильно навязанную  @igg{fbicon.laugh.rolling.floor}{repeat=3} 
Да... О боги...

\iusr{Юлия Заика}
\textbf{Oleksiy Voronets} о, а вот и он @igg{fbicon.laugh.rolling.floor} советчик  @igg{fbicon.laugh.rolling.floor}{repeat=3} 

\iusr{Ірина Костун}
\textbf{Galina Garaschenko} в одних плюс 19 на вулиці, в других іній на стінах.Яке цікаве місто.

\iusr{Светлана Ставницер}
\textbf{Тетяна Барабаш} очень рентабельно гонять холодную воду.А за электричество платить не надо?

\iusr{Елена Карповская}
\textbf{Алексей Воронец} о, началось) и так всю зиму - включите - нам холодно, теперь выключите - нам жарко. Люди, лучше головы включите и не выключайте!)

\iusr{Татьяна Антипова}

Я не знаю кому ОЧЕНЬ ЖАРКО, но ул МАГДАЛИНОВСКАЯ и ЗАПАСНАЯ сидят без
отопления. Ето примерно 10 домов, и как-то ето никого не смущает!!!

\iusr{Віталій Хайло}
\textbf{Borys Filatov} 

люди не понимают что система огромная и невключается/выключается одной кнопкой
и за один день, то что никто не ставит у себя в квартире на батареии перемычки
для выключения своей батареии чтоб уменьшить общий счёт на дом. \enquote{я же за это
заплатил я просто открою окно}

\iusr{Алла Черная}
\textbf{Borys Filatov} самый здравомыслящий человек  @igg{fbicon.biceps.flexed} 

\iusr{Ирина Порохня}
\textbf{Юрий Нефёдов} о да.... для нас так много делает заботливое госудаоство.....

\iusr{Galina Garaschenko}
\textbf{Татьяна Антипова} а власти пофиг. Мы же не поем дифирамбы в комментах! Наши не рассматриваются.

\iusr{Ольга Коростелева}
\textbf{Oleksiy Voronets} поставте счетчик и прикручивайте тепло как хотите.

\iusr{Лускань Алина}

Улица Звонкая включили отопление только в одном доме, а остальные дома сидят
без тепла как это понимать. Писали что в этом году будут все с отоплением а
получилось как всегда.

\iusr{Владислав Врублевский}
\textbf{Алексей Воронец} Вы думаете, что включить и отключить отопление так же легко как выключатель света у себя дома?

\iusr{Valerii Shepeliev}
\textbf{Oleksiy Voronets} Опалення - не лошадь на "тпрууу" не зупиняється. Це технологічний процес з великою інерцією.

\iusr{Влад Балицкий}
\textbf{Андрей Федорук} система управления в зависимости от температуры
окружающей среды на дом стоит на минуточку около 3 млн грн под ключ, это вам не
просто прибор теплоучета

\iusr{Анна Алєксєєва}
\textbf{Татьяна Антипова}, ул. Криворожская тоже без отопления((( люди дома в шубах и шерстяных носках ходят...

\iusr{Шурик Оленебой}
\textbf{Borys Filatov} терпения Вам!
P. S. У меня автономка. Я доволен!

\iusr{Саня Бреннер}
\textbf{Borys Filatov} бомбануло!) Вот делай людям добро)

\iusr{Alex Gulak}
\textbf{Irina Kolesova} счетчик не прикручивают, а уменьшают расход теплоносителя. Проблема в том, что тепловики это не разрешают.

\iusr{Evgeniy Tunic}
\textbf{Алексей Воронец} , а замороженных спросили по этому поводу ?

\iusr{Alina Kartashova}
\textbf{Oleksiy Voronets} Одесса, вчера только в школы дали отопление. В домах тишина, но мы не особо настаиваем, пока терпимо, на улице +15.

\iusr{Максим Дьяченко}
\textbf{Borys Filatov} 

граждане всегда дают особо ценные советы:

Из воспоминаний о Петербурге XIX века: крестьянин весной, игнорируя запрет,
начал переходить реку по льду, несмотря на свистки городового, провалился под
лёд, городовой его вытащил, а крестьянин начал орать «чего же ты, собака, меня
не остановил?» - «так я тебе свистел!» - «свистел он... надо было в морду
дать!».

\iusr{Вадим Олегович}
\textbf{Borys Filatov} 

Хочу чтобы окна мышью закрывались @igg{fbicon.face.grinning.sweat} молодец мер,
и такие 2,5 года мне назад говорили, ну и что что клоун, он научится, сидим
теперь с этими долбое... ми обучаем.

\iusr{Куценко Станислав}
\textbf{Алексей Воронец} у нас його і не включали  @igg{fbicon.laugh.rolling.floor}{repeat=3}  батареї холодні

\iusr{Куценко Станислав}
\textbf{Елена Карповская} це як з віконцем в тролейбусі і сварливу бабцю ))) відкрийте-закрийте

\iusr{Fedir Iermolaiev}
\textbf{Borys Filatov} 

удивительно. Наверное я тоже получу похожий ответ. Но, видимо нам очень далеко
до Европы, если даже такое понятие как вкл/выкл отопления у нас работает не
так... Почему в Европе отключают централизовано отопление, если температура
выше определенного значения, и так же централизовано включают, если температура
ниже. Тут с середины сентября начинали подтапливать, когда температура была
ниже 10°. Это я уже не говорю, про индивидуальные счетчики за отопление и
регуляторы температуры в каждой квартире на каждой батарее.

Это не притензия конкретно к Вам. Лишь мысли в слух о том, как мы застряли во
времени.

\iusr{Таня Омельченко}
\textbf{Borys Filatov} ..Извините, что здесь обращаюсь... помогите пожалуйста.... написала Вам в личку... Спасибо

\end{itemize} % }

\iusr{Валерия Таран}
Один терьер из той троицы все-таки есть?

\begin{itemize} % {
\iusr{Borys Filatov}
\textbf{Валерия Таран} да, 17 лет без трех месяцев.

\iusr{Александр В. Шульга}
\textbf{Borys Filatov} здоровья оставшемуся инопланетянину!
\end{itemize} % }

\iusr{Віктор Іванченко}
А у кого американцы украли ФАУ2?

\begin{itemize} % {
\iusr{Borys Filatov}
\textbf{Віктор Іванченко} У вас.

\iusr{Алим Галушка}
\textbf{Віктор Іванченко} 

Что значит \enquote{украли}??? К ним пошел на ПМЖ автор - Фон Браун. Который не то что
ФАУ2, а и на Луну сделал полет. А вот ссср действительно скопировал ФАУ2,
причем руководил процессом лично Королев (на ютубе есть документальное кино об
этом).

\iusr{Ірина Манюк}
\textbf{Алим Галушка} а ссылку можно?

\iusr{Алим Галушка}
\textbf{Ірина Манюк} 

Я уже не помню, но погуглите по \enquote{ФАУ-2}. Там будет много всякого и в том числе
советский документальный фильм об испытаниях советской копии этой ракеты. Это
был учебный фильм для ракетчиков (наверняка секретный в то время). И в фильме
присутствует комиссия из проводящих эти испытания ракетчиков, где кроме
Королева есть еще несколько известных фамилий. Хотя я все это знал и без фильма
от непосредственных участников).


\iusr{Ірина Манюк}
\textbf{Алим Галушка} очень уважаю Королёва. Он не из тех, кто присвоит свое @igg{fbicon.face.weary} ...

\iusr{Алим Галушка}
\textbf{Ірина Манюк} 

Это был начальный этап. Надо было пройти то, что прошел Браун, а не пытаться
изобрести велосипед. Возможно вы также не в курсе, что первый полет в космос
тоже совершила ФАУ2 зимой 43-го года? Единственное отличие, что это был не
орбитальный полет.

\iusr{Сергей Воевода}
\textbf{Ірина Манюк} 

А вы сугубо чувствами мыслите, факты поискать уважение к Королёву не позволяет?
 @igg{fbicon.wink} 

"Специалистами ОКБ-1 на основе узлов и агрегатов немецкой 
\enquote{Фау-2} была собрана
баллистическая ракета и 18 октября 1947 года произведен ее запуск. Копия
\enquote{Фау-2} пролетела 247 км, поднявшись на высоту 86 км.

Конструкторским бюро Королева из отечественных материалов на базе немецкой
ракеты была разработана ракета Р-1. С 1950 года, изучив все недостатки немецкой
конструкции, коллектив Королева, с участием таких ученых, как Валентин Глушко,
Николай Пилюгин и др., занялся ее коренной переработкой. В 1949 году начались
испытания Р-2, чья дальность увеличилась с 300 до 600 км. В 1955 году впервые
стартовала советская стратегическая баллистическая ракета Р-5М (8К51), а в 1957
году - межконтинентальная Р-7 (8К71)...".

\iusr{Инна Тарнавская}
Почитайте \enquote{Ракеты и люди} Чертока.

\iusr{Віктор Іванченко}
\textbf{Borys Filatov}

"Правда ракета у американцев была краденная.
Знаменитая ФАУ-2."
Ваша цитата.
Повторяю вопрос:
"У кого американцы украли фау? у немцев у советов?

\iusr{Oleg Levenets}
\textbf{Віктор Іванченко} 

а щє, американці, які рибачили в Аденській затоці, вкрали якорь, з сухогруза
\enquote{Фаіна}, з метою здати його на брухт. В результаті цих підступних дій
\enquote{піндосів} сухогруз був захоплений сомалійськими піратами....

Пане Філатов-респект Вам @igg{fbicon.biceps.flexed} 

\end{itemize} % }

\iusr{Anton Drobovych}
Дуже зворушливий пост, хоча і багато історії. Хай пси процвітають і господарі поміж ними)

\iusr{Юра Черных}
У Вас был прекрасный пост "про консервацию", про банки в 90х...

\iusr{Ruslan Gorovyi}

Коли я народився, в бабусі був собака Руслан. Сусіди шушукалися мовляв дитину
як собаку назвали))) Руслан мене обожнював. До старості проводжав до трамвая,
потім біг пять зупинок за ним, проводжав до квартири і вертався до баби


\iusr{Геннадий Сульдин}

6 лет назад мы поехали в питомник выбирать щенка. Пока мы с Милой рассматривали
собак и умничали, один щенок быстро растолкал других, уселся рядом с 3-х летней
Дашей, лизнул её в щеку и тявкнул на других маленьких зенненхундов. И потом не
отходил от Дашки ни на шаг. И так по сей день. Бегемоша любит всю семью, но
только Дашке можно все. Даже забрать кость из пасти

\ifcmt
  ig https://scontent-frx5-1.xx.fbcdn.net/v/t39.30808-6/251576169_4765613006811606_5792811870177366922_n.jpg?_nc_cat=105&ccb=1-5&_nc_sid=dbeb18&_nc_ohc=FVqLdzisEdAAX9Ts3he&_nc_ht=scontent-frx5-1.xx&oh=9d6092199c2ae8dfe68dde6c01fa7cbc&oe=61A89E49
  @width 0.25
\fi

\begin{itemize} % {
\iusr{Татьяна Березина}
\textbf{Геннадий Сульдин} какая славная мордаха, умные собачьи глаза у Бегемоши. Будь здоров, пес! @igg{fbicon.heart.red}
\end{itemize} % }

\iusr{Irina Zemlyanskaya}
\textbf{Borys Filatov}, 

доброе утро, пожалуйста, найдите время издать сборник \enquote{Избранное}, очень
интересно читать. Удачи!


\iusr{Светлана Бобошко}

Моя охрана и мои собеседники. Они меня слушают и, гла́вное, не спорят. Они
принимают моё настроение, желание общаться, а встречают, как будто сто лет не
виделись

\ifcmt
  ig https://scontent-frt3-2.xx.fbcdn.net/v/t39.30808-6/251513767_2092598027560712_6936718967289394021_n.jpg?_nc_cat=101&ccb=1-5&_nc_sid=dbeb18&_nc_ohc=bvEqw9Lb9qcAX88pU3P&_nc_ht=scontent-frt3-2.xx&oh=dfa19cfdb03368071c4ffbf0ebf3f3eb&oe=61A78E68
  @width 0.4
\fi

\iusr{Ольга Черненко}
Ну как можно их не любить? )))

\ifcmt
  ig https://scontent-frt3-1.xx.fbcdn.net/v/t39.30808-6/252911407_4732640193463288_4960636009185874502_n.jpg?_nc_cat=106&ccb=1-5&_nc_sid=dbeb18&_nc_ohc=bmjsuJzUwuoAX_nc90v&_nc_ht=scontent-frt3-1.xx&oh=b80ba0e363b811557b894c5e3c05624d&oe=61A7C9F5
  @width 0.4
\fi

\iusr{Любовь Мазуренко}

Борис Альбертович,
ещё несколько лет назад, я Вас тоже ...
не очень любила.(
Простите за откровенность!
А сейчас, всё чаще восхищаюсь ☺ ️ 
Чтоб Вы были нам здоровы!  @igg{fbicon.heart.red}

\iusr{Елена Елена}
Пронзительно

\iusr{Ольга Романенко}
Спасибо, читаю Вас и всякий раз восторгаюсь! Браво! Мир дому, где собак и лошадей любят!!!!

\iusr{Виктория Жидкова}
до слез... нам до них как до луны..

\iusr{ирина терентьева}

Когда-то на севере подобрали лайку)), она потом отблагодарила, спасла жизнь
всей семье при возгорании ночью. Потом приехала в Днепр с семьёй и прожила
долгую и счастливую жизнь))

\iusr{Irina Verlyuk}

Завжди було дуже шкода до сліз Лайку(( собак не питали чи хочуть вони у космос,
просто так хотіли люди((( а ще крім Лайки величезна кількість собак, які
загинули при різних космічних випробуваннях...((((


\iusr{Елена Немиченко}
Прочитала...
Скажу одно: хорошо, когда сердце работает по назначению @igg{fbicon.heart.red}.

\iusr{Лилия Свиридюк}
...на одном дыхании @igg{fbicon.heart.suit}
И познавательно, и поучительно, и очень трогательно!

\iusr{Свєтік Свєтік}
Доброго раночку Борис Альбертович! Гарного дня! Люди, які люблять собак, ви - найкращі!  @igg{fbicon.hearts.two}{repeat=3} 

\iusr{Александра Новак}

Собака - не просто друг человека, она самый старый друг человека и возможно это
космосом предусмотрено, чтобы человек никогда не был одинок. С собакой можно
поделиться любой тайной, она никому не расскажет, собака никогда не предаст и в
случае угрозы будет биться за хозяина, до конца. Даже через века, собака
выполняет все функции заложенные ей, а вот человек иногда даёт сбой, чаще всего
люди предают своих питомцев, бросают их, перестают любить, особо поражает когда
выбрасывают маленьких щенков, люди даже не осознают, что это тоже дети и выбросив
щенка у обочины или в посадке, люди подвергают его опасности или обрекают на
верную смерть. Борис Альбертович, ваш пост очень классный может он заставит
людей задуматься и быть добрее, не пройти, а хотя бы накормить бродяжек. Ваша
семья, пример для многих, спасибо Вам, за ваше \enquote{большое сердце}. Приветик, всем
вашим \enquote{иномланетным} друзьям меньшим, особый привет Кабыздоху Марте. Как вы эти
клички, так прикольно придумываете.. супер @igg{fbicon.wink}  @igg{fbicon.heart.purple} 

\iusr{Станислав Пономаренко}

Ну почему же, Борис Альбертович, читали, помним. И очень печально и грустно
становится за то что мы называем себя гордо - люди.

\iusr{Станислав Пономаренко}

Ну почему же, Борис Альбертович, читали, помним. И очень печально и грустно
становится за то что мы называем себя гордо - люди.

\iusr{Ірина Дячишин}

Я так не можу повз котів пройти..( Маю 4 куплених «шотландців», а ще 10
підібрала з вулиці... Ще хаскі знайшла, блукав містом виснажений і (як потім
виявилось), з піропразмосом(( Вилікувала, хоча прогнози були кепські. І так, я
люблю тварин більше, ніж людей! Бо кожен-перший може сказати шо то «пєрєбор» і
«шо, там у вас кошкін дом?!», а тварини просто мене люблять @igg{fbicon.heart.red}

\iusr{Наталья Видишева Слива}
Обожаю мою Берту!!!

\ifcmt
  ig https://scontent-frt3-1.xx.fbcdn.net/v/t39.30808-6/253172909_1482175312175947_9106681895932050188_n.jpg?_nc_cat=108&ccb=1-5&_nc_sid=dbeb18&_nc_ohc=CBa-jqg68qIAX8yR4h8&_nc_ht=scontent-frt3-1.xx&oh=133db42e42f07f6ad73a20fbbd44dd9f&oe=61A8CB6C
  @width 0.4
\fi

\iusr{Любовь Бушмакова}

Очень тёплый пост. Наша девочка Евдокия - Дуся, тоже нас сама выбрала, когда в
приют ездили

\ifcmt
  ig https://scontent-frx5-1.xx.fbcdn.net/v/t39.30808-6/253199204_654055262253117_1985736628552392835_n.jpg?_nc_cat=105&ccb=1-5&_nc_sid=dbeb18&_nc_ohc=A1drCJK5U8UAX9QZUsf&_nc_ht=scontent-frx5-1.xx&oh=3711a2c94927660d44b89db87dfe6e2d&oe=61A90B7A
  @width 0.4
\fi

\iusr{Люда Донев}

Я обожаю собак! - и этим всё сказано. Для меня только этого было бы достаточно,
чтобы искренне и уважительно интересоваться Филатовым... И мне даже неважно,
стали вы Мэром чи ни  @igg{fbicon.wink}  ))) Я тоже обожаю собак и кажется я
собачница на генетическом уровне (и они это тоже чуют) А вам, искреннее
СПАСИБО, за всё, но за собачника - отдельный
поклон.@igg{fbicon.index.pointing.up} @igg{fbicon.palms.up.together}
@igg{fbicon.face.smiling.hearts} Обняла (не фамильярничаю ни разу, то от души)
виртуально -то можно?! - и у меня все три вакцины есть в лучшем виде)))

\iusr{Вячеслав Савтюк}

Мне кажется что Коты даны в противоположность нам - землянам в НАКАЗАНЬЕ
@igg{fbicon.face.squinting.tongue} 

\begin{itemize} % {
\iusr{Татьяна Березина}
\textbf{Вячеслав Савтюк} я кошатница, но что-то не поняла смысл. Мои котейки не наказание, а радость и позитив. 
@igg{fbicon.cat.heart.eyes} 
\end{itemize} % }

\iusr{Anna Hudoteplaya}

Собаки, да и вообще животные, лучшие \enquote{люди} на земле!!! Работая в большом
городе с большим количеством \enquote{разномастных} человечков, я предпочитаю своих
собак! Мой сын учится на ветеринара и мы мечтаем о своей маленькой
клинике-приюте для животных))), откуда здоровые и счастливые \enquote{зайки} будут
уезжать только к \enquote{здоровым} на голову, любящим животных семьям!!! Спасибо,
Борис Альбертович, за этот пост!!!  @igg{fbicon.hands.applause.yellow} 

\iusr{Виктория Алексеенко}

У меня была любимая собака, доберман Веста.. Я была на кухне и забыла выключить
свой старый утюг( в 90 не было возможности купить новый).. Он так нагрелся, что
кусок раскаленного железа отвалился и упал на пол, могла быть беда. Но моя
собака, поняла, что что то пошло не так, начала громко и агрессивно
лаять.. Вообщем, я успела, благодаря ей... Она прожила с нами 12 лет ..было
онко, не уследили, пришлось усыплять, чтобы она не мучилась... Снится мне
иногда... пишу, и слезы на глазах..

\iusr{Лариса Бабич}
Семь собак и у меня, все подобранные на улице. Все разные по характеру, всех люблю.

\iusr{Людмила Солодка}
И познавательно, и душевно, и грустно...

\iusr{Светлана Голубицкая}
Великолепно написано! Спасибо!

\iusr{Оксана Стельмах}

Я \enquote{подумай}?????????????????? Это Вы приютам начинайте помогать уже... Я -
плачу налоги и хочу, чтобы в МОЁМ городе на улицах не было брошенных собак... Я
не могу помочь всем (жаль!)... Но спасаю многих... (Приют \enquote{Друг}. г.Днепр)

\iusr{Марина Нємцева}
Я б вже лише за це за вас голосувала, якби жила в Дніпрі.

\iusr{Tetiana Zheliezna}
Спасибо! Огромное количество брошенных животных. И это не их вина, а человека
Если хоть один человек с Ваших подписчиков, возьмет себе с приюта или с улицы животное, это +1 спасенная жизнь

\iusr{Максим Сыч}
Борис Альбертович, у вас очень крутые собаки  @igg{fbicon.dog} 
Обожаю смотреть их фото)

\iusr{Дмитрий Пластун}
Все таки не перестаю убеждаться во мнении о положительных человеческих качествах людей, которые любят животных.

\iusr{Сервер Фахриев}
я не мог не заскринить.
Момент, когда у этого поста был 31 лайк.

\ifcmt
  ig https://scontent-frx5-1.xx.fbcdn.net/v/t39.30808-6/252981773_6374853082556900_1674058135560051490_n.jpg?_nc_cat=100&ccb=1-5&_nc_sid=dbeb18&_nc_ohc=tB_dXOadP-oAX-QwAIp&_nc_ht=scontent-frx5-1.xx&oh=ec1eafa0f5edc9aca192af8f568e8f59&oe=61A94BC3
  @width 0.4
\fi

\iusr{Алена Заусалина}
Да собаки нас учат добру и верности

\iusr{Olga Noskova}
\textbf{Borys Filatov} 

а как же произошла метаморфоза от ровного отношения к любви?, хочу понять что
внутри включается?, может некоторые мои знакомые тоже еще \enquote{не потеряны} для
любви к животному и этот тумблер включится

\iusr{Инна Тарнавская}
Пронзительный текст. До слез.
Спасибо.

\iusr{Ksusha Ponomarenko}
Наш любимчик.

\ifcmt
  ig https://scontent-frx5-1.xx.fbcdn.net/v/t39.30808-6/253206448_2142402139233003_7881619255572320096_n.jpg?_nc_cat=100&ccb=1-5&_nc_sid=dbeb18&_nc_ohc=tqRuqCEPtjYAX-zM4xd&_nc_ht=scontent-frx5-1.xx&oh=e19caa2e3c90e22549f970c54478e1db&oe=61A886A1
  @width 0.25
\fi

\iusr{Sofiya Melnyk}
Спаибо, Очень много интересных фактов. И спасибо Вам и Вашей семье за то, что вдохновляете землян)

\iusr{Валентина Попеня}
Собаки последние ангелы на земле...

\iusr{Наталія Коцюк}
Прекрасний пост! Як же я Вас поважаю! Правда!

\iusr{Prykhodko Lena}
Дуже гарно написано!!! І пізнавально!!! Люблю, обожнюю тварин!!! Наша киця- наша принцеса!!!

\iusr{Тамара Рогачева}
Люблю читати Ваші пости про тварин. А з фото ще більш люблю.
Мої теж самі мене знаходять. І собаки і коти.
Добре, що я заходжу їм місце і в будинку і в серці.
Зараз 3 собаки і 2 коти.

\iusr{Daria Dementieva}

Да, а тем кто на последних этажах долбись с этим отоплением чтобы запустить. И
тут говорят отключите после запуска.

\iusr{Andrii Kryhan}

Про Малыша так и не полюбившего людей напомнило роман Джека Лондона. Про
собак-смертников с билетом в один конец ужас...

Ещё слышал, что при отборе собак для полёта в космос выбрали собаку по кличке
Смелый и она оказалась не только самой смелой, но и хитрой - сбежав со
стартовой площадки перед началом полёта.

\iusr{Ірина Манюк}

Знаю Вашу любовь к животным, обращаюсь с просьбой: помочь организовать
муниципальный приют для брошенных животных. Нет сил читать, сердце рвётся от
сообщений в группах: брошена, брошенный @igg{fbicon.face.crying.loudly}{repeat=4} 

\begin{itemize} % {
\iusr{Александра Новак}
\textbf{Ірина Манюк} 

Да, да, было бы не плохо, сил нет смотреть на бездомных собачек, особенно на
щенков, просто пачками бросают, очень тяжело найти каждому семью, особенно
девочкам. Только нужно, что бы наши люди, тоже были немного ответственей и добрее
и брали в семью не породистых щеночков, а то у нас частные приюты переполнены, а
животные кушать хотят. Знаю один приют, хозяйка просто молит о помощи. Жалко
их. @igg{fbicon.face.pleading} 

\iusr{Ірина Манюк}
\textbf{Александра Новак} 

смотрю часто Олега Новикова из Израиля. Крошечная страна, окружённая врагами...
У них другое отношение к животным, просто зависть берёт @igg{fbicon.face.crying.loudly}{repeat=4} .

\end{itemize} % }

\iusr{Анна Белогур}

Когда умер муж меньше месяца назад я решила взять собаку из приюта. До этого
жил питбуль. Добрейший парень прожил 16 лет. Теперь замечательный парень живёт
со мной. Благодарю его за то что есть в моей жизни

\iusr{Таня Буряк}
У меня на руках умирала моя собака. От возраста. Єто не возможно передать. Прошло почти 10 лет, все помню и боль не отпускает...

\iusr{Елена Тарабаненко}
Собаки-это ангелы. Они не предают и не бросают в трудную минуту. Они лучше нас, людей.

\iusr{Светлана Гуськова}
Обожнюю собак і поважаю лише тих людей, які поважають тварин

\iusr{Анна Стафійчук}
Плачу

\end{itemize} % }
