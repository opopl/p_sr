% vim: keymap=russian-jcukenwin
%%beginhead 
 
%%file 13_11_2018.stz.news.ua.mrpl_city.1.blagodijna_dijalnist_teatr_pracivnyk_mrpl
%%parent 13_11_2018
 
%%url https://mrpl.city/blogs/view/blagodijna-diyalnist-teatralnih-pratsivnikiv-mariupolya
 
%%author_id demidko_olga.mariupol,news.ua.mrpl_city
%%date 
 
%%tags 
%%title Благодійна діяльність театральних працівників Маріуполя
 
%%endhead 
 
\subsection{Благодійна діяльність театральних працівників Маріуполя}
\label{sec:13_11_2018.stz.news.ua.mrpl_city.1.blagodijna_dijalnist_teatr_pracivnyk_mrpl}
 
\Purl{https://mrpl.city/blogs/view/blagodijna-diyalnist-teatralnih-pratsivnikiv-mariupolya}
\ifcmt
 author_begin
   author_id demidko_olga.mariupol,news.ua.mrpl_city
 author_end
\fi

Театр завжди займав особливе місце в історії розвитку людства, адже він здатен
зберегти звичаї, традиції, ціннісні пріоритети того чи іншого періоду,
об'єднати всі види мистецтва, залучити всі рівні людського існування. У нових
умовах економічного та політичного життя України, коли постала необхідність
вдосконалення соціального захисту малозабезпечених верств населення, нового
осмислення потребує вивчення благодійної діяльності театральних установ.

\ii{13_11_2018.stz.news.ua.mrpl_city.1.blagodijna_dijalnist_teatr_pracivnyk_mrpl.pic.1}

Вивчення проблеми здійснювалося здебільшого засобами масової інформації на
шпальтах місцевої періодики \enquote{Мариупольская жизнь} та \enquote{Мариупольские известия}.

Складні умови історичного та соціально-економічного життя мешканців Північного
Приазов'я кінця XIX – поч. XX ст. вплинули на формування театрального мистецтва
в регіоні. Довгий час не було жодного стаціонарного чи мандрівного
(пересувного) театру. У 1878 р. видатний антрепренер, талановитий артист,
режисер та вихователь театрального колективу В. Шаповалов створив у Маріуполі
першу професійну трупу. З цього часу у місті починають діяти професійні та
аматорські драматичні гуртки. Місцева громадськість розглядала театр як
виховний засіб, здатний прищепити моральні норми та естетичні погляди до
літератури і драматичного мистецтва.

\textbf{Читайте також:} \emph{Муниципальная няня не помешает мариупольчанкам уйти в декрет}%
\footnote{Муниципальная няня не помешает мариупольчанкам уйти в декрет, Анастасія Папуш, mrpl.city, 12.11.2018, \url{https://mrpl.city/news/view/munitsipalnaya-nyanya-ne-pomeshaet-mariupolchankam-ujti-v-dekret}}

Наприкінці XIX ст. у місті функціонувало декілька благодійних товариств:
\textbf{Маріупольське Благодійне Товариство}, \textbf{Товариство допомоги вихованцям
Маріупольської Олександрійської гімназії}, \textbf{Товариство взаємної допомоги},
\textbf{Товариство допомоги бідним}. Водночас багато любителів сценічного мистецтва
давали вистави з благодійною метою і, завдяки своєму старанному відношенню до
постановки п'єс, пожинали лаври і мали матеріальний успіх. Звичайно, місцева
публіка, охоче відвідуючи вистави любителів драматичного театру, завжди
розуміла, що перед нею – не \enquote{присяжні} актори, а \enquote{добровольці}
мистецтва, що полегшують життя найбідніших верств населення міста.

На думку українського вченого О. Ільченка, зростання благодійного руху було
викликано розвитком капіталістичних відносин і формуванням підприємницького
прошарку. Разом з тим завдяки пожертвам та різним видам допомоги як окремі
громадяни, так і культурні установи, отримували певне визнання в суспільстві,
підвищували свій статус, включалися до життя верхівки.

\vspace{0.5cm}
\begin{minipage}{0.9\textwidth}
Важливою подією стало заснування Маріупольського драматично-музичного
товариства у 1884 р. Члени товариства ставили аматорські вистави, влаштовували
концерти, сприяли естетичному вихованню мешканців міста. Завдяки організатору
товариства Е. Батієвському на благодійних засадах було поставлено безліч
вистав, концертів.
\end{minipage}
\vspace{0.5cm}

\textbf{Читайте також:}

\begin{itemize}
\item \emph{Маріупольський театр в роки німецької окупації. Частина 1., Ольга Демідко, mrpl.city, 07.10.2018}
\item \emph{Маріупольський театр в роки німецької окупації. Частина 2., Ольга Демідко, mrpl.city, 15.10.2018}
\end{itemize}

\ii{13_11_2018.stz.news.ua.mrpl_city.1.blagodijna_dijalnist_teatr_pracivnyk_mrpl.pic.2}

На початку XX ст. у Маріуполі вистави йшли на кількох сценах. Якщо в театрі
Уварова завжди працювала російська трупа, то в цирку-театрі братів Яковенків –
українська. Драматичні спектаклі, концерти оперних співаків влаштовувалися
також у концертній залі готелю \enquote{Континенталь}, в театрі
Олександрівського парку, в Народному будинку, на заводі \enquote{Нікополь}, в
залі Маріупольського громадського зібрання.

Згідно з відомостями періодичної преси, на початку XX ст. відвідування театру
заохочувалося. Так, 15 жовтня 1911 р. любителями драматичного мистецтва була
представлена вистава \enquote{Журба}, драма в п'яти діях, автором якої був І.
Шпажінський. У зв'язку з тим що вистава проходила у приміщенні зборів
службового маріупольського порту, до якого добиратися було незручно,
організатори вистави замовили спеціальний потяг з міста в порт для міських
відвідувачів. Проїзд на потязі був абсолютно безкоштовний. \textbf{\em Отже, маріупольські
театрали сприяли розвитку культури мешканців міста, піклувалися про доступ всіх
верств населення, особливо малозабезпечених, до художньої творчості та
мистецьких цінностей.}

\ii{13_11_2018.stz.news.ua.mrpl_city.1.blagodijna_dijalnist_teatr_pracivnyk_mrpl.pic.3}

У 1911 році у місті було проведено ряд благодійних спектаклів. Так, 16
листопада 1911 р. в театрі Уварова пройшов благодійний спектакль за п'єсою А.
Чехова \enquote{Дядя Ваня}. Збір з вистави пішов на посилення засобів товариства
піклування про дітей. А 19 листопада в театрі Уварова пройшов спектакль, збір з
якого було надіслано на користь талмуд-тори. У міській газеті \enquote{Мариупольская
жизнь} напередодні благодійного спектаклю повідомлялося:

\begin{quote}
\em\enquote{Талмуд-тора} – початкова школа для єврейських дітей, переважно незаможних
батьків. Школа відчуває крайню потребу у фінансуванні. У ній навчається
понад 200 дітей, переважно єврейської бідноти. Талмуд-тора не тільки
дає початкове навчання дітям, але й забезпечує останніх посібниками, а
багатьом дітям постачає одяг і взуття. Кожний, хто відвідає виставу 19
листопада, зробить добру справу, надавши допомогу талмуд-торі, як
філантропічній установі. Відбудуться дві п'єси: \enquote{Столичне повітря} –
комедія в чотирьох діях і \enquote{Поцілунок – перший і останній} - етюд в
одній дії.
\end{quote}

\textbf{Читайте також:} \emph{Легко ли попасть постороннему в мариупольскую школу?}%
\footnote{Легко ли попасть постороннему в мариупольскую школу?, Олена Онєгіна, mrpl.city, 10.11.2018, \url{https://mrpl.city/news/view/legko-li-popast-postoronnemu-v-mariupolskuyu-shkolu}}

Протягом 1912 р. поціновувачі драматичного мистецтва ставили вистави в
приміщенні зборів службового маріупольського порту. 3 листопада 1912 р. у
приміщенні відбувся благодійний спектакль на користь вдів і сиріт чорногорців,
убитих у війні з Туреччиною. Любителями драматичного мистецтва представлена
була нова п'єса \enquote{Убога і ошатна} – драма в трьох діях, написана С.
Бєлою.  Важливим є те, що організатори подібних вистав не забували піклуватися
про маріупольців і продовжували замовляти спеціальний поїзд, який безкоштовно
підвозив міських відвідувачів.

Початок XX ст. позначений розквітом аматорських театральних колективів та більш
широкою благодійною діяльністю. 4 березня 1913 р. в маріупольському театрі
Уварова пройшов перший спектакль музичного драматичного гуртка
\enquote{Товариства взаємної допомоги}. Весь чистий збір від виконаної п’єси
\enquote{Приватна справа} при участі хору і оркестру гуртка надійшов на користь
курсів для дорослих при зазначеному товаристві. 24 березня 1913 р. відбувся
перший дитячий спектакль, який приніс неочікуваний матеріальний успіх.
\enquote{Товариство взаємної допомоги} вирішило системно щонеділі й надалі
ставити дитячі спектаклі за доступними цінами.

На сцені маріупольського театру продовжували ставити благодійні вистави. Так, у
лютому 1914 р. в театрі братів Яковенків відбулася вистава на користь
незабезпечених учнів маріупольського реального училища В. Гіацинтова.

У лютому 1919 р. за ініціативою спілки Трудящих жінок на заводі Нікополь
пройшов благодійний спектакль. Вистава ставилася групою молодих любителів
театрального мистецтва. Після спектаклю були влаштовані танці. Кореспондент
газети повідомляв, що \emph{\enquote{Чистий збір в сумі 116 руб. 30 коп. було передано в руки
голови Союзу Трудящих жінок... для розподілу між бідними прошарками населення}.}

\textbf{Читайте також:} \emph{Дети из интернатов и детских домов: какие подарки они ждут на праздники?}%
\footnote{Дети из интернатов и детских домов: какие подарки они ждут на праздники?, Дарья Касьянова, mrpl.city, 10.11.2018, \url{https://mrpl.city/blogs/view/deti-iz-internatov-i-detskih-domov-kakie-podarki-oni-zhdut-na-prazdniki}}

Таким чином, благодійна діяльність театральних працівників, як професійних, так
і аматорських, наприкінці XIX – початку XX ст. здійснювалася у наступних
напрямах: \textbf{\em поліпшення матеріального становища малозабезпечених мешканців міста,
надання матеріальної допомоги талановитим та малозабезпеченим ді\hyp{}тям, піклування
про доступ всіх верств населення до театрального мистецтва.} Професійні актори
та аматори виконували свою роботу, спираючись на високоморальні засади
милосердя та гуманізму. Разом з тим благодійна діяльність маріупольських
театралів була не тільки прикладом громадської активності та соціальної
творчості, але й сприяла покращенню свого авторитету та створенню власного
іміджу.


\clearpage
