% vim: keymap=russian-jcukenwin
%%beginhead 
 
%%file 18_03_2023.fb.rassadina_iryna.harkiv.1.poperedu_buv_mar_upo.cmt
%%parent 18_03_2023.fb.rassadina_iryna.harkiv.1.poperedu_buv_mar_upo
 
%%url 
 
%%author_id 
%%date 
 
%%tags 
%%title 
 
%%endhead 

\qqSecCmt

\iusr{Natalya Seletskaya}

росії немає бути.

\iusr{Dina Voronkova}

Ира, хоть я умом понимаю, что ты пишешь это сейчас, находясь в безопасности, у
меня чуть сердце не выскочило, когда я читала.

Всё, что ты рассказываешь, фактически материалы для суда

\begin{itemize} % {
\iusr{Iryna Rassadina}
\textbf{Дина Воронкова} как это доказать?
Я рассказываю и буду рассказывать
Готова и в суде...

\iusr{Dina Voronkova}
\textbf{Ірина Рассадіна} главное, чтобы он состоялся

\iusr{Iryna Rassadina}
\textbf{Дина Воронкова} та уберут плешивого свои же, скоро.
Вчера метку черную выдали в Гааге
Знак всему окружению, что договариваться не будут.
\end{itemize} % }

\iusr{Viktoria Stepanenko}

\obeycr
Згадала, як рік тому читала про цю поїздку і плакала.
І зараз плачу.
Я з Дніпра.
В нас і зараз багато Маріупольців.
А подруга з Харкова увесь час там.
Не схотіла нікуди їхати.
Мама в неї хвора і три котики.
\restorecr

\iusr{Oksana Boychun}

Я читала твій пост і мене не покидало вже знайоме відчуття - там не працюють
ніякі закони! Ніякі! Взагалі!! Там працює тільки інтуїція і вдача!! Повезе / не
повезе- ось єдиний закон там. І ніхто не знає чи будеш ти тим самим
щасливчиком. Коли мої друзі хотіли їхати за нами, я нікому не дозволила!! Я
сказала, що виїду сама і тільки тоді, коли буду відчувати що можна. Так і сталося
- ми пройшли через свій ад і залишились живими!!

\iusr{Альона Кондур}

Читала і плакала....
Тоді, рік тому, вирішувалась моя доля і доля моєї родини...
Ми не були першими в списку порятунку.
Ви могли забрати Юлію і зразу виїхати
Могли б забрати інших людей....
...так, ми чекали і не чекали... бо ніхто нічого гарантувати не міг....
Дякую Богові, дякую Вам, Ірина, за порятунок 🙏🏾

\begin{itemize} % {
\iusr{Iryna Rassadina}
\textbf{Альона Кондур} мені здавалося чесним дивитися адреси у порядку їх попадання до мене...
Я не знала як ще зробити цей чортів мій вибір, кому жити, а кому залишитися і, вірогідно, вмерти, бо я не приїхала...
Збіг обставин вирішив за мене.
Хитрість зберегла автівку на в'їзді й кмітливість дозволила не втратити на виїзді...
Ми з вами вижили тоді.
І ми тут.

\iusr{Viktoria Stepanenko}
\textbf{Альона Кондур} 

Сподіваюся, що далі ви більш - менш нормально змогли дістатися безпечного місця.

Дай Боже!

\iusr{Альона Кондур}
\textbf{Виктория Степаненко} дякую велике.
Ми були в Бердянську. Потім виїхали до Запоріжжя де знаходимось і по сьогодні.
Будинок з квартирою у Маріуполі знесла окупаційна влада 25 січня 2023 року...

\iusr{Viktoria Stepanenko}
\textbf{Альона Кондур}

\obeycr
Запоріжжя моє улюблене місце, багато з ним пов'язано.
Тому кожен удар ворогів по ньому сприймаю, як і коли вони по моєму Дніпру б'ють.
Так хочеться, щоб нам усім вцілити в перемогти!
Тримаймося!
Обнімаю!
\restorecr

\end{itemize} % }

\iusr{Tetiana Ostrovska}

Ірино. Ридаю, читаючи це. Ви неймовірна

\iusr{Olga Matsuka}

Читаю ваш допис і забуваю дихати, буквально провалююсь у ті події. Дочитала, піднімаю голову і не розумію, де я взагалі.

Без усякого пафосу кажу: ви Героїня.

Я народилася і виросла в Маріуполі, хоча все життя живу в Києві. Я не знаю до
чого це, але хочеться саме вам про це сказати. Дякую за все і обіймаю❤️

\begin{itemize} % {
\iusr{Iryna Rassadina}
\textbf{Ольга Мацука} 

ви взнаєте фото порта?

Знайомі місця? Я не мала часу зробити більше, й боялася, й спішила, але може
щось колихнеться з цих крихти у вашій пам'яті? 🌼

\iusr{Olga Matsuka}
\textbf{Iryna Rassadina} 

порт, на жаль, не пригадую. Але моє дитинство пройшло в самому центрі міста,
саме там, звідки ви забирали людей. І той театр був моїм першим театром, і в
тому пологовому я народилася. З жахом думаю, що сталося із більшістю моїх
однокласників і друзів дитинства.

\end{itemize} % }

\iusr{Dmitryi Romanyuk}

Читаєш на одному подиху... маю надію що колись ви напишете книгу...

\begin{itemize} % {
\iusr{Iryna Rassadina}
\textbf{Дмитрий Романюк} 

я частково маю...

Й ще одну, іншу книгу, про мою евакуацію, з моїх слів набрала моя подруга, журналіст

Теж цікаво.

Я б дуже хотіла написати...
\end{itemize} % }

\iusr{Iryna Rassadina}

Згадала ще.

Я замалювала на мапі ту форточку через заміновані поля, і як мені потім сказали
волонтери з Бердянську, вони вивезли у наступні кілька днів багато людей так,
по тій стежці...

\begin{itemize} % {
\iusr{Viktoria Stepanenko}
\textbf{Ірина Рассадіна} Окрема подяка за це!
\end{itemize} % }

\iusr{Дмитрий Чикуров}

Треш жизни

\iusr{Дмитрий Вулканов}

Це дуже героїчний вчинок. Дуже.

\iusr{Лариса Овчаренко}

В мене нема слів, Іра...Героїня.

\iusr{Виктория Павлова}

Це якісь кадри зі страшного фільму. І все ж таки, яка Ви мужня !!! Чекаємо на книгу !

\begin{itemize} % {
\iusr{Iryna Rassadina}
\textbf{Виктория Павлова} ага
Я так і назвала це, коли давала інтерв'ю тут, у Швеції
Хоррор і я головна героїня у ньому...
\end{itemize} % }

\iusr{Lucy Karpova}

Пиши ще, читаю і плачу. Яка ж ти крута. Як же хочеться, щоб не було приводів
для геройських вчинків, щоб жити і плакати лише від радості. Обіймаю, нехай
буде книга і бестселлер, адже це правда життя 🙏

\iusr{Залина Малахова}

Я тоже реву...

\iusr{Sergiy Podnos}

Дякую.

\ifcmt
  igc https://scontent-fra5-2.xx.fbcdn.net/v/t39.30808-6/336788991_172870428940280_84229575073564749_n.jpg?_nc_cat=107&ccb=1-7&_nc_sid=dbeb18&_nc_ohc=SWDEoPZAbjYAX9TQXkD&_nc_ht=scontent-fra5-2.xx&oh=00_AfCcns1GAh6nUn5kzIuyrzfh3nAhH_FAG-7DFuH567HCsg&oe=642AF1B9
	@width 0.5
\fi

\begin{itemize} % {
\iusr{Iryna Rassadina}
\textbf{Sergiy Podnos} блін, я й забула про це фото! 🌼

\iusr{Iryna Rassadina}
\textbf{Sergiy Podnos} здається я нарешті виплакала це все...
Моя психолог сказала, що ми все це ще з нею проговоримо, але схоже, що я справилася

\iusr{Oksana Boychun}
\textbf{Iryna Rassadina} ти точно справилась!!!! Обіймаю!!
\end{itemize} % }

\iusr{Елена Семёнова}

\ifcmt
  igc https://scontent-frt3-2.xx.fbcdn.net/v/t39.30808-6/336786368_772641667304349_133123533348037964_n.jpg?_nc_cat=108&ccb=1-7&_nc_sid=dbeb18&_nc_ohc=7Fyi3HsWdgAAX9rnBha&_nc_ht=scontent-frt3-2.xx&oh=00_AfB63VzGWPTqW5AlkOj3pDChiCAotilCX2Kf-Yw0NvBagQ&oe=6429ECFA
	@width 0.5
\fi

\begin{itemize} % {
\iusr{Iryna Rassadina}
\textbf{Елена Семёнова} саме так

\iusr{Елена Семёнова}
\textbf{Ірина Рассадіна} своє згадую... Дякую вам за тих людей!
\end{itemize} % }

\iusr{Kharchenko Dmirty}

Досі не вірю, що я знаю особисто цю героїчну дівчину!

\iusr{Olga Sushko}

Ира, обіймаю міцно, сил тобі та відновлення, дякую за все, що ти зробила.

\iusr{Natalia Poliakh}

Знаєте, от зараз я плАчу😥 Наче стоїть перед очима все це( Я такого не
переживала, лише сирени, але таке враження, що це зі мною було😥 За що нам все
це?((

\begin{itemize} % {
\iusr{Iryna Rassadina}
\textbf{Natalia Poliakh} я коли писала цей текст, окунулася в усі ті спогади...
Здається навіть той сморід відчула.
Я думаю то був сморід мертвих тіл
Якийсь трохи солодкий й блювотно-незабутній.
\end{itemize} % }

\iusr{Вика Радченко}

Пиши нам. Пиши.

\iusr{Олена Ступак}

Плачу

\iusr{Людмила Панина}

Ви дуже сильна жінка, обіймаю❤️.

\iusr{Iryna Rassadina}

Посилання на тогорічний допис

Я тоді ще писала російською

Мене завалили коментарями й у особисті, щоб хтось вивіз їх рідних. Сотні, сотні людей...

Я боялась відкривати Месенджер 🙄

\url{https://m.facebook.com/story.php?story_fbid=3032553610339754&id=100007554331518}

\iusr{Natalya Nata}

С 14 года думала, что меня не удивишь. Ирина, я была в таком шоке, что Вы
вытворяли. В шоке от того, что могут обычные люди.  Я стараюсь всегда помочь.
Но такое б я не смогла.

\begin{itemize} % {
\iusr{Iryna Rassadina}
\textbf{Natalya Nata} 

\obeycr
я не впевнена, що я б змогла знову
Так склались обставини.
Я тоді сказала Вікінгу, що не зможу себе поважати, якщо не поїду.
Бо я весь час, все своє життя щось робила для людей. А тут я на два+ тижні застрягла, розумієте? Я ховалась у підвалі і потіла від жаху. Я бачила, як люди допомагають, розбирають завали, щось роблять, а я не могла! Не могла зрушити з місця, тільки найближчий магазин і короткі прогулянки з собакою.
І коли прозвучало: \enquote{Я не маю права тебе просити, але...}
Я зрозуміла, що це шанс реабілітували весь цей просраний час у підвалі.
Почати поважати себе знову.
От я і поїхала
\restorecr
\end{itemize} % }

\iusr{Станіслав Чепурко}

Неймовірно...

\iusr{Ірина Мудрик}

Іра, я в онімінні. Ти якась сталева, неймовірна

\iusr{Володимир Скорик}

Охох. Не знаю, як коментувати. Це набагато краще за сюжети голівудських фільмів. Це дуже сильно.

\begin{itemize} % {
\iusr{Iryna Rassadina}
\textbf{Володимир Скорик} я б не проти сценарій з цього зробити)
Головною акторкою можна було б Анджеліну 🤩
\end{itemize} % }

\iusr{Крапива Анна}

Ох, Іро, читати важко... Уявляти важко(( Яке пекло, який жах... Дякую, що пишете! Дякую, що вивезли людей! Дякую!

\iusr{Natalya Nevmerzhitskaya}

Господи, за що ?! Плачу

\iusr{Olena Levytska}

Дякую вам.

\iusr{Iryna Vare}

Господи, я давно так не ревла. Дай Вам Боже здоров'я, Ірино!!! На таких, як Ви,
тримається Україна!!! Все в нас буде добре!!!!

\iusr{Milena Kovalska}

Я навіть сама відчула той сморід. Тобі обов'язково треба писати. Я ще не читала
такого тексту про ті події, на одному диханні. Ну і, звісно, знімаю капелюха. Я
небагато знаю людей, які б на таке відважилися. Я точно ні, к моєму сорому. Це
справжній Героїчний вчинок. Ти неймовірна ✊🏻🇺🇦.

\begin{itemize} % {
\iusr{Iryna Rassadina}
\textbf{Мілена Ковальська} 

колись, коли я буду старою і вся у татухах, я буду сидіти на своїй веранді і
приймати сонячні ванни, та примружившись роздивлятися свій новий сліпучо-білий
Харлі.

А в цей час чіп, вживлений мені за вухо, буле зчитувати і нотувати мої спогади,
зкладаючи оповідання.

Мою 126 книжку 😏

\iusr{Milena Kovalska}
\textbf{Iryna Rassadina} Бажаю тобі того Харлі, але радше трохи раніше, бо хочеться ще сприймати інфу 🤣.
\end{itemize} % }

\iusr{Oleg Shimko}

Якби у нас був титул Праведника за спасіння українців, як в Ізраїлі за спасіння
євреїв, то це саме той випадок. Власне, суть саме така і без офіційного титулу

\begin{itemize} % {
\iusr{Iryna Rassadina}
\textbf{Олег Шимко} мені навіть ніяково...

\iusr{Oleg Shimko}
\textbf{Iryna Rassadina} але це правда

\iusr{Олена Щербань}
\textbf{Oleg Shimko} так!!

\iusr{Олена Щербань}
Дякую Вам, Іро!!!! Дякую!!!
\end{itemize} % }

\iusr{Юлія Гіренко}

Сижу, читаю, ридаю і захоплююся тобою. Я і раніше тебе лю, а зараз просто
схиляюся ... Це ..ну я навіть не знаю ... я б сказала про сталеві яйця але ж ти
дівчинка ... а ще дуже за тобою скучила і хотілося б побачитися ; (

\begin{itemize} % {
\iusr{Iryna Rassadina}
\textbf{Юлія Гіренко} та приїду я...
ХЗ тільки коли, але обов'язково у цьому році

\iusr{Юлія Гіренко}
\textbf{Iryna Rassadina} і Денчику привіт від нас з Олегом передавай. Обіймашечки вам і Стетіку. Я вам завжди рада, чекаю !
\end{itemize} % }

\iusr{Catherine Lytvyn}

А я сидю і мене колотить. Я поки не можу висловити свої емоції

\iusr{Василий Маленко}

Немає слів...

\iusr{Tetiana Kharlan}

Я не знаю, який ставити смайл. Я читаю це, як роман. Я читаю і думаю, що так
мало зробила порівняно з тобою. Бо ти врятувала людей.

Ти просто навіжена ❤️ Я так пишаюсь, що знайома з тобою!

\iusr{Ольга Ісакова}

Реву...Іро, дякую !! Обіймаю міцно !❤

\iusr{Юлія Сергіївна}

Читаю з жахом і тремтінням...

\iusr{Інна Штонда}

Ти неймовірна!

\iusr{Elena Shalik}

Дякую. Головна нагорода нашого буття - зустріч з Людиною.

\iusr{Ольга Ахмедова}

Восхищаюсь тобой!!!! Мега сильная девчуля! Сильная духом!!! Неимоверная!!! Обнимаю

\begin{itemize} % {
\iusr{Iryna Rassadina}
\textbf{Ольга Ахмедова} и я тебя обнимаю 🌼
\end{itemize} % }

\iusr{Марина Тарасюк}

Дякую Вам.

\iusr{Виктория Мезенцева}

До сліз .дякую

\iusr{Olena Sokolynska}

пробирає до кісток ваша розповідь. дякую. хочу спитати - чи не збираєтеся ви
оприлюднити ваші розповіді у вигляді книги чи збірки? як на мене - це безцінні
свідоцтва. і про те, що \enquote{Украіни больше нєт} це все важливі момннти для пам'яті

\begin{itemize} % {
\iusr{Iryna Rassadina}
\textbf{Олена Соколинська} 

я б дуже не проти

Й є ідея зробити це, є матеріали, але треба, щоб зацікавилися ті, хто допоміг
би з процесами, бо я ніколи не робила нічого такого...

\iusr{Olena Sokolynska}
\textbf{Iryna Rassadina} 

якщо ви це зробили для фб, то це вже досвід, як на мене) також, якщо у вас є
матеріал для книги, то є два шляхи (як я бачу). можна знайти зацікавленого
видавця і відписати книгу за договром та термінами. можна спершу відписати, а
потім знайти зацікавлену особу. видавців на фб досить багато, один з них Антон
\textbf{Антон Санченко} - наприклад. інша справа, що війна йде, і хотілося б, мабуть,
мати такі свідоцтва по закінченні. але ж це можуть бути роки. як на мене, то
важливо зараз писати, що пам'ятаєте, і потім зібрати у видання. ці тексти дуже
важливі

\iusr{Iryna Rassadina}
\textbf{Олена Соколинська} дякую, в мене є ідея, яку я б хотіла з видавцем обсудити
Я пробувала знайти в Швеції (я тут живу тепер), але тут все занадто важко і довго, мене футболять безкінечно, хоча є що показати...
\end{itemize} % }

\iusr{Ліля Дубінська}

Ірино, не вміщається ні в голову, ні в душу усе, про що ви пишете. Надлюдське
все - горе, розпач, страх, підлістьь тих на НЛО, втрата надії і сенсу до
застиглості, невтрачена надія і жага рятувати життя, і ваша, Ірино, сміливість,
інтуіція, людяність.

Іро, збережіть "в хмарі" ваші записи і фото (я пам"ятаю ваши полум"яніючи фото
з Норвегії). Бо я вже знаю, як в мить втрачаєш ціли роки записів, пам"яті,
фото, написану книгу, які були на залишеному в іншому житті вінчестері.
Збережіть.

Вдячність вам безмежна.

\begin{itemize} % {
\iusr{Iryna Rassadina}
\textbf{Ліля Дубінська} 

я цю історію рік тому написала для одного японського видання

Вони збирали історії про Маріуполь в якийсь, наче альманах, про українських
Самураїв.

Здається його видали вже

Так що десь ця історія існує...

Я б хотіла книгу, або сценарій зробити з цього матеріалу, якби знала як...

\iusr{Ліля Дубінська}
\textbf{Iryna Rassadina} 

Треба шукати тих, хто знає і схоче допомогти. Марина Козлова, може, ви когось порадите?

\iusr{Марина Козлова}
\textbf{Ліля Дубінська} 

тут два питання, вони різні. 1. Як зробити книгу чи сценарій. 2. Як/де
опублікувати книгу або гідно прилаштувати сценарій. По першому пункту з приводу
книги можу проконсультувати я, з приводу сценарія - мій син, він кіносценарист.
Про другий пункт поговоримо також. Якщо автор не проти.

\iusr{Ліля Дубінська}
\textbf{Марина Козлова} Дякую, Марино.

\iusr{Tetyana Alekseyeva}
\textbf{Iryna Rassadina} це був би крутий фільм!

\iusr{Iryna Rassadina}
\textbf{Марина Козлова} 

я за

І в мене є ідеї, які я частково пропрацювала вже 😌

Бо є ще одна історія, що тісно зв'язана з цією поїздкою

Це моя особиста евакуація з Харкова
\end{itemize} % }

\iusr{Marinka Gritsik}

Ты невероятная!

\iusr{Andrei Peresichanski}

Ви дуже мужня жінка. Я би не зміг ще раз повернутись на окуповану територію... Виїжджав, пам'ятаю ці відчуття.

\begin{itemize} % {
\iusr{Iryna Rassadina}
\textbf{Андрей Пересичанский} це незабутній досвід...
\end{itemize} % }

\iusr{Рая Ковальчук}

Дякую Вам! Божих благословінь.

\iusr{Yuliia Pimenova}

Дякую!!!

\iusr{Светлана Кадученко}

Дякую Вам.

\iusr{Svitlank Ensoleiile}

Сльози на очах.

Дякую вам

\iusr{Axel Foly}

Читав, і мало у самого адреналін з вух не полився... В кінці не витримав,
заплакав. Не знаю, що ще можна сказати, щаслив, що живу з Вами в одній країні
🤗

\iusr{Наталья Доманская}

Неймовірна дівчина! Пишаюся, що в нас є такі ЛЮДИ.

\iusr{Olena Cherednychenko}
\textbf{Iryna Rassadina}, 

ви неймовірна, смілива жінка, коли прочитала вашу історію, забракло слів хоч
щось написати в коментарі, просто розшарила. Це мої рідні міста - Бердянськ,
який приймав людей з Маріуполя і місцеві були просто в шоці від їхнього
вигляду, і пенсионери несли по 100 гривень на їжу і одяг, бо неможливо було
уявити, через що довелося пройти цим людям. І Харків..

Уявляю собі дорогу, бо багато разів їздила з Маріка до Бердянську і назад. Досі
немає слів, самі емоції. Дякую вам

\begin{itemize} % {
\iusr{Iryna Rassadina}
\textbf{Olena Cherednychenko} нажаль я не роздивилась Бердянськ, може вдасться ще
Марік, з тих місць де я була і що бачила, нажаль зруйновано, знищено в нуль. Повністю, разом з деревами(

\iusr{Olena Cherednychenko}
\textbf{Iryna Rassadina}, головне - люди. Кожен врятований - це безцінно

\end{itemize} % }

\iusr{Мария Станиславовна}

Це неможливо читати без сліз. Дякую Вам!

\iusr{Людмила Борина}

Дякую Вам.

\iusr{Ярослава Куколка}

Дякую Вам ❤️ Пишаємось Вами, пишу, а сльози котяться

\iusr{Марианна Карпинская}

Дякую вам за сміливість та збережені життя.

\iusr{Anna Vitvitskay}

Це неймовірно ...плачу та і все ... коли читаєшь і розумієшь кожне слово, бо
сам сидів там в Маріуполі та не знав чи виживешь, чи сможешь виїхати взагалі.
Дякую вам за те шо ви така є 🤗

\iusr{Tatyana Sadkovskaya}

Читала и внутри все сжималось от волнения. Спасибо Вам за спасение людей!🙏

\iusr{Hanna Krushynska}

Расплакалась

\begin{itemize} % {
\iusr{Iryna Rassadina}
\textbf{Hanna Krushynska} обіймаю
Бо теж плачу, третій день..
\end{itemize} % }

\iusr{Сергій Моргун}

Неймовірна, смілива подорож.

Дякую

\iusr{Aleksey Voloshchuk}

Дякую за пронизливий текст. 18 березня теж їхав через Василівку, і нам теж
дорогу показував хлопець на велосипеді. Кожного інструктувпв, як їхати щоб на
міни не попасти. Він нам тоді казався янголом, який рятує. Від мене отримав
пачку Скетчесів

\begin{itemize} % {
\iusr{Iryna Rassadina}
\textbf{Aleksey Voloshchuk} про ангела точно!)
може ми їхали одне за одним 😆
\end{itemize} % }

\iusr{Alena Hriy}

Дякую. Ревіла весь час поки читала..

\begin{itemize} % {
\iusr{Iryna Rassadina}
\textbf{Alena Hriy} а я - поки писала...
\end{itemize} % }

\iusr{Oksana Sinitsyna}

Читаю і серце встає...ти неймовірна героїня, справжня людина!!! Що ж наробили
ці потвори з риб'ячими очима в жаб'ячій формі?! Вони перетворили звичайних
людей, чоловіків, жінок і дітей в справжніх воїнів, велетнів в повному сенсі
слова!!! Народився народ який не зламати і не здолати!!! Пишаюся тобою!!! Слава
Україні!!! Слава героям і героїням!!!

\begin{itemize} % {
\iusr{Iryna Rassadina}
\textbf{Оксана Синицына} дякую, люба 🌼😌
\end{itemize} % }

\iusr{Tatyana Yar-co}

Плачу... Велика ДЯКА за всіх. Боже бережи Вас і Ваших близьких ❤️❤️❤️

\iusr{Marina Mirnenko}

Плачу. Благодарность. Ангела-Хранителя! Он вел Вас тогда.

\iusr{Olga Selezneva}

Низький уклін Вам, Ви - неймовірна, пишіть книгу, про те, що відбувалося і
відбувається, повинен читати весь світ.
