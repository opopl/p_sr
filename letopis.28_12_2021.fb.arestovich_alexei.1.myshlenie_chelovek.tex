% vim: keymap=russian-jcukenwin
%%beginhead 
 
%%file 28_12_2021.fb.arestovich_alexei.1.myshlenie_chelovek
%%parent 28_12_2021
 
%%url https://www.facebook.com/alexey.arestovich/posts/5039451112785595
 
%%author_id arestovich_alexei
%%date 
 
%%tags chelovek,filosofia,myshlenie
%%title Що робить людину людиною? Мислення
 
%%endhead 
 
\subsection{Що робить людину людиною? Мислення}
\label{sec:28_12_2021.fb.arestovich_alexei.1.myshlenie_chelovek}
 
\Purl{https://www.facebook.com/alexey.arestovich/posts/5039451112785595}
\ifcmt
 author_begin
   author_id arestovich_alexei
 author_end
\fi

- Що робить людину людиною? Мислення. 

Якби його не було, ми б освоювали світ тільки практично і нічим не відрізнялися
від тварин.

За Арістотелем мислення є частиною душі, яка здатна осягати думкою світ та саму
себе. І саме тому мислення так тісно взаємопов'язане  з такими поняттями як
«буття», «ідея», «образ», «знання»,  «матерія», «людина»...

\ii{28_12_2021.fb.arestovich_alexei.1.myshlenie_chelovek.pic.1}

А найбільшою сферою  інтенсивного мислення є філософія.  Саме вона розповідає
нам, як людина мислить.  Вона - велика, безмежна, неймовірно різноманітна та
надзвичайно цікава. У ній тисячі різних поглядів на будь-що. І тільки одне
завдання – дарувати неймовірну насолоду тому хто мислить. 

А якщо ми ще навчимося правильно користуватися нею - вона нам допоможе в
будь-чому.

Чи можна осягнути філософію в поп-форматі?

Безумовно.

Адже популяризація - це вид творчості. Коли складне пояснюється більш простою і
зрозумілою для широкої аудиторії мовою, за умови, що зміст викладається, а не
спотворюється.

Саме тому \href{https://www.facebook.com/S.M.Chaplygin}{Сергій Чаплигін} зараз
працює над створенням цікавого навчального продукту, який допоможе зрозуміти
філософію. 

Адже інформації по філософії багато, а потрібно відібрати те, що працює. 

Наприклад. Актуальність філософії в умовах сучасності та цифрової гегемонії;
методики саморозвитку людини - від стоїків до чільних філософів XXI століття;
рідкісні книги, фільми та статті та багато іншого.

Тримайте посилання-запрошення та не втратьте його:

\url{https://chaplygin.customer.smartsender.eu/lp/xEZXOC8t}

Я особисто підписався.)

\ii{28_12_2021.fb.arestovich_alexei.1.myshlenie_chelovek.cmt}
