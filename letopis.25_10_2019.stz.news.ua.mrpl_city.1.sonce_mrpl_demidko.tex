% vim: keymap=russian-jcukenwin
%%beginhead 
 
%%file 25_10_2019.stz.news.ua.mrpl_city.1.sonce_mrpl_demidko
%%parent 25_10_2019
 
%%url https://mrpl.city/blogs/view/sontse-mariupolya-istorik-aktivistka-bloger-olga-demidko
 
%%author_id hizhnikova_ganna.mariupol,news.ua.mrpl_city
%%date 
 
%%tags 
%%title Сонце Маріуполя — історик, активістка, блогер Ольга Демідко
 
%%endhead 
 
\subsection{Сонце Маріуполя — історик, активістка, блогер Ольга Демідко}
\label{sec:25_10_2019.stz.news.ua.mrpl_city.1.sonce_mrpl_demidko}
 
\Purl{https://mrpl.city/blogs/view/sontse-mariupolya-istorik-aktivistka-bloger-olga-demidko}
\ifcmt
 author_begin
   author_id hizhnikova_ganna.mariupol,news.ua.mrpl_city
 author_end
\fi

\ii{25_10_2019.stz.news.ua.mrpl_city.1.sonce_mrpl_demidko.pic.1}

Якщо Ви захочете дізнатися про історію Маріуполя, легенди Азовського моря або
біографії видатних містян — я знаю людину, яка зможе пролити світло на ці
факти. Якщо Ви захочете поспілкуватися зі світлою, усміхненою та активною
дівчиною — я знаю таку. Давайте знайомитись з нею. Героїня нашого блогу —
сонячна \textbf{Ольга Демідко}, історик, викладач, екскурсовод, громадська діячка,
блогер,\footnote{\url{https://mrpl.city/blogs/olga-demidko}} молода мама і просто приємна людина.

\ii{25_10_2019.stz.news.ua.mrpl_city.1.sonce_mrpl_demidko.pic.2}

Ольга — корінна маріупольчанка, народилася в нашому місті в 1991 році. Вже в
дитинстві була активною дівчинкою. Вона зі старшим на 3 роки братом росла в
приватному будинку. На їхній вулиці було багато дітвори, тому пустощі були
завжди. Одного разу, щоб заробити грошей на цукерки, за ініціативою брата діти
винесли з дому одяг мами і тата, посуд і книги та влаштували розпродаж,
горлаючи на всю вулицю: \emph{\enquote{Продається!}}. А ще Ольга згадує: \emph{\enquote{Вдома завжди було
багато тварин. Я шалено любила і люблю собак і котиків. Якщо когось бачила на
вулиці, маленького і безпритульного, одразу несла додому. Мама спочатку лаяла,
але потім залишала тварину. Вона у мене завжди була дуже доброю}}. А ще мама Олі
\textbf{Ірина Борисівна} — неймовірно чуйна та щедра людина. На всі Дні народження сина
та доньки вона влаштовувала величезні свята. Приходило багато друзів, ласували
солодощами. А мама влаштовувала різні конкурси, і гості йшли додому теж з
подарунками.

\ii{25_10_2019.stz.news.ua.mrpl_city.1.sonce_mrpl_demidko.pic.3}

Варто зазначити, що мама для нашої героїні - приклад життєвої незламності.
\emph{\enquote{Мене надихає моя мама, яка неймовірно сильна. Я це і раніше знала, але
останнім часом черпаю натхнення завдяки її силі. Мама не ходить з 2008 року.
Але її бажання боротися і жити не можуть не захоплювати}}, - ділиться Ольга. 

Після закінчення школи наша героїня зрозуміла, що хоче працювати вчителем. 

\begin{quote}
\em\enquote{Я у
свої 17 років усвідомлювала, що завдяки своїй вчительці історії я вирішила щось
змінювати навколо себе. Тому відразу вирішила, що хочу теж викладати історію.
Але, навчаючись в університеті, я дуже захопилася науковою діяльністю,
зрозуміла, що це дійсно моє і почала брати участь у різноманітних конференціях,
семінарах. Потім – аспірантура}, 
\end{quote}
- так коротко Ольга розповідає про своє
навчання, додаючи, що ще встигла й один рік попрацювати в школі.

Зараз молода дівчина працює не тільки викладачем МДУ, а й телеведучою і
літературним коректором на МТБ, блогером, екскурсоводом.

\ii{25_10_2019.stz.news.ua.mrpl_city.1.sonce_mrpl_demidko.pic.4}

\begin{quote}
\em\enquote{Загалом я помітила,
що мало втомлююся і після роботи готова займатися господарською роботою,
енергії дуже багато. Рідні дивуються і кажуть, що після роботи треба
відпочивати, а мені інколи здається, що часто мені вдається відпочити саме на
роботі. Спілкування з допитливими і цікавими студентами, зустрічі з
талановитими людьми і створення чогось корисного для свого рідного міста не
можуть втомлювати}, 
\end{quote}
- з сонячною посмішкою додає наша героїня.

Якщо перераховувати всі проекти, в яких брала участь Ольга Демідко, - голова
піде обертом. 

\begin{quote}
\em\enquote{Проектів багато. Складно виділити один улюблений, всі вони
допомогли мені усвідомити важливість роботи з маріупольцями і зрозуміти, що все
можливе, просто для неможливого треба більше часу},
\end{quote}
- каже активістка.

\ii{25_10_2019.stz.news.ua.mrpl_city.1.sonce_mrpl_demidko.pic.5}

\begin{itemize} % {
\item У травні – червні 2016 року Ольга виступала розробником та координатором
ініціативної групи \enquote{Разом} проекту \enquote{Маріуполь – це Україна!}. Його мета – зміна
уявлень мешканців та гостей міста про його історію і життя, розкриття нових
невідомих широкому загалу фактів, знайомство з його різноманіттям, пробудження
у маріупольців інтересу до спадщини міста і майбутнього його розвитку.
\end{itemize} % }

\ii{25_10_2019.stz.news.ua.mrpl_city.1.sonce_mrpl_demidko.pic.6}

\begin{itemize} % {
\item У березні – листопаді 2016 року під егідою Української миротворчої школи за
фінансової підтримки Посольства Великої Британії в Україні на волонтерських
засадах Ольга Демідко виконувала обов'язки координатора проекту \enquote{Малюємо заради
миру}.
\end{itemize} % }

\ii{25_10_2019.stz.news.ua.mrpl_city.1.sonce_mrpl_demidko.pic.7}
\ii{25_10_2019.stz.news.ua.mrpl_city.1.sonce_mrpl_demidko.pic.8}
\ii{25_10_2019.stz.news.ua.mrpl_city.1.sonce_mrpl_demidko.pic.9}

\begin{itemize} % {
\item З квітня 2016 року по 2018 рік виступала разом з \textbf{Олександрою Ваксман},
\textbf{Анастасією Пономарьовою}, \textbf{Тетяною Петрушкіною} організатором читацького клубу
\enquote{Маріуполь читає дітям}.

\item Червень 2016 року – на волонтерських засадах розробила разом з художницею
Анастасією Пономарьовою розмальовки \enquote{Розфарбуй місто}, в яких
маріупольська архітектура представлена в антистресовому
варіанті. Біля кожної споруди на сторінках розмальовки можна
знайти історичну довідку. Тираж: 350 екз.

\item Грудень 2016 р. – розробила та завдяки підтримці маріупольських волонтерів
випустила збірку творчих робіт маріупольців \enquote{Мелодії незримих почуттів}. Тираж
- 300 примірників.

\item З 20 січня 2018 року і по теперішній час – керівник арт-центру \enquote{Театр всередині
тебе}, який було відкрито при Маріупольському державному університеті за
фінансової підтримки Британської ради в Україні. На сьогодні в арт-центрі
проводяться творчі зустрічі з акторами, майстер-класи від
ре\hyp{}жисерів-постановників, обговорення вистав, працюють експозиції, присвячені
театральній історії міста.

\item Лютий – березень 2018 року - виступала одним з організаторів фестивалю
\enquote{Обличчя міст} в Маріуполі. На волонтерських засадах виконувала
обов'язки координатора історичної локації (проводила екскурсії,
квести).

\item Також Ольга є автором оригінального проекту \enquote{360 градусів},	 про який
розповідає: 

\begin{quote}
\em\enquote{Ця ідея виникла у \textbf{Володимира Селіванова}, головного
редактора сайту mrpl.city, після нашого знайомства. Він сказав,
що на сайті є камера 360°, але немає ведучого, який розповідав
би про архітектуру Маріуполя. Ми вирішили спробувати - і це
стало новим проектом \enquote{Маріупольського телебачення}. Всього
підготовлено 50 турів. Окремо 10 серій вийшли для міського
сайту Маріуполя. Віртуальні тури присвячені як маріупольським
особнякам XIX ст., так і архітектурі XX ст. Також сюжети
розкривають таємниці міського саду Маріуполя, старовинного
цвинтаря, археологічних артефактів міста тощо}.
\end{quote}

\end{itemize} % }

\ii{25_10_2019.stz.news.ua.mrpl_city.1.sonce_mrpl_demidko.pic.10}

Мало хто знає, але попри велике навантаження на роботі та в громадській
діяльності, Ольга ще займається кулінарією, зокрема, робить чудову випічку. На
питання, як їй все вдається встигнути, дівчина відповідає: 

\begin{quote}
\em\enquote{Я насправді багато
всього не встигаю. Але любов до тортиків у мене від мами теж. Вона мене
привчила, що жінка повинна робити все сама}.
\end{quote}

\ii{25_10_2019.stz.news.ua.mrpl_city.1.sonce_mrpl_demidko.pic.11}
\ii{25_10_2019.stz.news.ua.mrpl_city.1.sonce_mrpl_demidko.pic.12}

Цього року Ольга народила маленького сина \textbf{Данила}. \emph{\enquote{Він мене змінив багато в
чому. З ним я знайшла справжню гармонію}}, - ділиться молода мама. Хлопчик у
свої 6 місяців і сидить, і повзає, і \enquote{розмовляє} з матусею. А ще він дуже
любить танцювати і співати. Наша героїня з натхненням розповідає, що хоче
навчити сина багато чому. Але головне - щоб він знайшов себе і виріс гідною
Людиною, чуйною, доброю та шляхетною.

Планів на майбутнє у Ольги Демідко дуже багато. На початку 2019 року вона
захистила дисертацію, присвячену театру Приазов'я.

\ii{25_10_2019.stz.news.ua.mrpl_city.1.sonce_mrpl_demidko.pic.13}

\begin{quote}
\em\enquote{Мені б дуже хотілося продовжити своє дослідження і відкрити в
Маріуполі інтерактивний музей театральної культури, оскільки на Сході
України немає жодного подібного музею.  А ще планую випустити книгу,
присвячену моєму найулюбленішому і унікальному місту - Маріуполю.
Сподіваюся, що зможу здивувати маріупольців новим видом екскурсії з
елементами перформансу і квесту}, 
\end{quote}
- розкриває плани історик.

Спілкування з усміхненою, енергійною та сонячною Ольгою Демідко надихає. Як
шанувальник її творчості та редактор блогів, я завжди уявляю, що чудові тексти
про легенди Азовського моря, про видатних сучасників, про пам'ятки архітектури
звучать її мелодійним голосом. Так я дізналася багато фактів про Маріуполь.
Тепер історія міста для мене — не просто слова в текстах, це пісня, це
цікавинка, це натхнення. Погодитеся зі мною?

\ii{25_10_2019.stz.news.ua.mrpl_city.1.sonce_mrpl_demidko.pic.14}

\textbf{Улюблене місце:} У Маріуполі люблю стару частину міста, Приморський парк, ну і, звичайно ж, море.

\textbf{Улюблена маріупольська легенда:} Вона пов'язана з морем, \href{https://archive.org/details/23_06_2018.olga_demidko.istoria_i_legendy_azovskogo}{дуже романтична і присвячена красуні Азі},
\footnote{Історія і легенди Азовського моря, Ольга Демідко, mrpl.city, 23.06.2018, \par%
\url{https://mrpl.city/blogs/view/istoriya-i-legendi-azovskogo-morya}, \par%
Internet Archive: \url{https://archive.org/details/23_06_2018.olga_demidko.istoria_i_legendy_azovskogo/mode/2up}
}
на честь якої і названо наше море Азовським.

\textbf{Улюблені книги:} Таких дуже багато. Вони у мене змінюються з кожним роком.
Читати намагаюся постійно. Можу назвати книги, які сподобалися найбільше
останнім часом - \enquote{Друге життя Уве} Фредрік Бакман, \enquote{Антирадянський роман} Оун
Метьюз, \enquote{Сміх крізь сльози} Фаїна Раневська.

\textbf{Улюблені фільми:} \enquote{Дівчата}, \enquote{Невидима стіна}, \enquote{В гонитві за щастям},
\enquote{Достукатися до небес}. Ще дуже любила серіал \enquote{Пагорб одного дерева}.

\textbf{Хобі:} Збираю статуетки сов та янголів. Ще збираю капці, фото Азовського моря і
люблю в кожному місті купувати нові сережки.

\ii{25_10_2019.stz.news.ua.mrpl_city.1.sonce_mrpl_demidko.pic.15}

\textbf{Побажання маріупольцям:} 

\begin{quote}
\em\enquote{Вірити в свої сили, в своє місто і свою країну.
Всупереч всім життєвим обставинам залишатися Людьми. Головне - не зачерствіти
душею і зберегти людську подобу. Ще бажаю підтримувати своє рідне місто і не
кидати його в біді, як би банально це не звучало. У нас, маріупольців, є свій,
властивий тільки нам менталітет. І у нас є величезна кількість переваг, але ми
звикли робити все самі. Бажаю нам навчитися об'єдну\hyp{}ватися заради однієї
благородної мети}.
\end{quote}

Фото зі сторінки Ольги Демідко в Facebook.
