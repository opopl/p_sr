% vim: keymap=russian-jcukenwin
%%beginhead 
 
%%file 11_02_2022.stz.news.ua.hvylya.1.vse_zlo_ot_ressentimenta
%%parent 11_02_2022
 
%%url https://hvylya.net/analytics/246999-vse-zlo-ot-resentimenta
 
%%author_id dacjuk_sergij,news.ua.hvylya
%%date 
 
%%tags chelovek,ressentiment,zlo
%%title Все зло - от Ресентимента
 
%%endhead 
 
\subsection{Все зло - от Ресентимента}
\label{sec:11_02_2022.stz.news.ua.hvylya.1.vse_zlo_ot_ressentimenta}
 
\Purl{https://hvylya.net/analytics/246999-vse-zlo-ot-resentimenta}
\ifcmt
 author_begin
   author_id dacjuk_sergij,news.ua.hvylya
 author_end
\fi

\begin{zznagolos}
У зла есть две причины: неосознанный эгоизм, не учитывающий интересы других, и Ресентимент.
\end{zznagolos}

\ifcmt
  ig https://hvylya.net/crops/545218/834x0/1/0/2022/01/13/2u9ECnPjMntIadDNux0bsJPPw6PI6hMkmFgsT4qV.png
  @caption Сергей Дацюк. Романенко. Беседы
  @wrap center
  @width 0.8
\fi

У зла есть две причины: неосознанный эгоизм, не учитывающий интересы других, и
Ресентимент. Неосознанный эгоизм, как правило, корректируется и ограничивается
или вообще преодолевается воспитанием, образованием и правоохранительными
институтами. В то же время Ресентимент не только не ограничивается социально,
но и продуцируется, поддерживается и легитимируется в обществе. Именно поэтому
Ресентимент является столь массовой и легитимной причиной зла, уже почти
вытеснившей зло от несоциализированного эгоизма. (
\href{https://blogs.pravda.com.ua/authors/datsuk/6206047ca6715/}{Этот текст на
украинском}).

\ii{11_02_2022.stz.news.ua.hvylya.1.vse_zlo_ot_ressentimenta.1.chto_takoe_ressentiment}
\ii{11_02_2022.stz.news.ua.hvylya.1.vse_zlo_ot_ressentimenta.2.istoricheskij_koll_pamjati}
\ii{11_02_2022.stz.news.ua.hvylya.1.vse_zlo_ot_ressentimenta.3.totalnost}
\ii{11_02_2022.stz.news.ua.hvylya.1.vse_zlo_ot_ressentimenta.4.identichnost_ressentimenta}
\ii{11_02_2022.stz.news.ua.hvylya.1.vse_zlo_ot_ressentimenta.5.preodolenie_ressentimenta}


