% vim: keymap=russian-jcukenwin
%%beginhead 
 
%%file slova.solnce
%%parent slova
 
%%url 
 
%%author 
%%author_id 
%%author_url 
 
%%tags 
%%title 
 
%%endhead 
\chapter{Солнце}
\label{sec:slova.solnce}

%%%cit
%%%cit_pic
%%%cit_text
Села птаха белокрылая на тополь. Слова Нины Матвиенко. Села птаха белокрылая на
тополь, Село \emph{Солнце}, красным заревом крича. Полюбила, полюбила я до
боли, Молодого, молодого скрипача. Я влюбилась, околдованная струною,
Заблудилась та мелодия в лесу
%%%cit_comment
%%%cit_title
\citTitle{Украинские Песни Русскими Словами}, 
БРАТИНА, zen.yandex.ru, 15.12.2020
%%%endcit

%%%cit
%%%cit_head
%%%cit_pic
\ifcmt
  pic https://strana.ua/img/forall/u/11/52/1625582748-930.jpg
	width 0.4
\fi
%%%cit_text
Когда вспышки на \emph{Солнце} такого класса направлены непосредственно на Землю,
самые сильные из них могут угрожать жизням астронавтов и работе спутников в
космическом пространстве, а также создавать помехи электрических сетей на
Земле.  \emph{Солнечные вспышки} классифицируются латинскими буквами А, В, С, М или Х
в зависимости от величины пика интенсивности рентгеновского излучения вспышки
(А - самая маленькая, Х - самая большая). За буквой указывается также индекс,
который уточняет значение вспышки и может быть от 1,0 до 9,9.  Ученые
утверждают, что пятно AR2838, вызвавшее вспышку Х1,59, является новой активной
областью на \emph{Солнце}
%%%cit_comment
%%%cit_title
\citTitle{На Солнце произошла самая мощная вспышка за последние годы}, 
Анна Копытько, strana.ua, 06.07.2021
%%%endcit

%%%cit
%%%cit_head
%%%cit_pic

\ifcmt
  tab_begin cols=3
     pic https://regnum.ru/uploads/pictures/news/2021/10/01/regnum_picture_16330825827663862_normal.JPG
     pic https://regnum.ru/uploads/pictures/news/2021/10/01/regnum_picture_16330826013161964_normal.JPG
		 pic https://regnum.ru/uploads/pictures/news/2021/10/01/regnum_picture_16330826384000816_normal.JPG
  tab_end
\fi
%%%cit_text
В основном в октябре будет облачно с пояснениями, во вторую неделю возможны
осадки, но есть вероятность, что установится \emph{солнечный} период. Дневная
температура во второй половине октября опустится до 5 градусов, ночью до 2
градусов тепла
%%%cit_comment
%%%cit_title
\citTitle{В Москве ждут бабьего лета-2021 — фоторепортаж}, 
Наталья Стрельцова, regnum.ru, 01.10.2021
%%%endcit

%%%cit
%%%cit_head
%%%cit_pic
%%%cit_text
Джиб из Болот встал до восхода \emph{солнца}. Он всегда вставал рано, но сегодня у
него было очень много дел. Именно сегодня гномы велели ему прийти за новым
топором: лезвие старого, изношенное и стершееся, уже невозможно было наточить
как следует.  Обычно по утрам в это время года болото затягивал низкий туман,
но сегодня утро было ясное. Несколько клочьев слоистого тумана висело над
островом, где добывались дрова, но в целом тумана не было. На восток и на юг
тянулось плоское болото, коричневое и серебряное, поросшее тростником и
травами. Утки плескались в прудах поблизости, мускусная крыса плыла по полоске
воды, оставляя за собой аккуратный разбегающийся след в виде буквы «V». Где-то
далеко крикнула цапля. К западу и северу на фоне неба поднимались холмы,
заросшие лесом — дубами и кленами, и некоторые из них уже были тронуты
огненными красками осени
%%%cit_comment
%%%cit_title
\citTitle{Зачарованное паломничество}, Клиффорд Саймак
%%%endcit

%%%cit
%%%cit_head
%%%cit_pic
%%%cit_text
На світанку я виходив зустрічати \emph{сонце} і єднався з ним. Так, так! Це не
абстракція! Я відчував себе його часткою, його атомом. Разом з тим я відчував
себе самим \emph{сонцем}, самим світлом, що відділило свій згусток, свій флюїд,
свою хвилю для мандрування в глибинних світах. Що з того, що я нібито
непрозорий, не вогняний? Що з того, що я звик спати, повзати по планеті,
одягатися, їсти? Моя промениста сутність заснула, одяглася в скафандр
тримірності. А тепер я хочу повернутися у безмірну домівку, я прокидаюся! Мені
не потрібно їжі, не потрібно вживання чужої енергії. Я сам — енергія, вічна
динаміка вселенського руху!  Друже мій! Ви навіть не можете уявити собі, що
творить воля. Згадайте, як з маленької сніжки блискавично нагромаджується
снігова баба. Або як відбувається ядерна ланцюгова реакція. Вибух втаємничених
сил, про які ми раніше й не підозрювали. Закон той же для деградації й для
саморозкриття. Неухильність волі — ключ від всіх замків: у науці, в любові, в
подвигу, в пошуку нових можливостей!
%%%cit_comment
%%%cit_title
\citTitle{Вогнесміх}, Олесь Бердник
%%%endcit

%%%cit
%%%cit_head
%%%cit_pic
%%%cit_text
Затока виявилась ширшою, ніж видавалась з берега, вже й \emph{сонце} вишнево пірнуло
за степовим берегом, а Тоні, нажаленій медузами, ще далеко було добуватись до
берега. Море не лякало її своєю глибиною, бо вона знала, що комунівські жінки,
які працюють по той бік на птахофермі, щоранку перебродять його, отак
підібравши спідниці... Жінкам з птахоферми тутешні броди, видно, були краще
відомі, а Тоню це забите водоростями Саргасове море піймало так, що насилу
виплуталась, а в другому місці зненацька шугнула на глибоке — опинилась у воді
по шию. Проте й це не вибило її з радісного настрою, тим паче, що далі море
знову мілкішало, ставало під руки, стало по пояс, і Тоня побрела швидше. Жахало
її тільки, коли медузи раз у раз торкались у воді її голого тіла, жалили ноги,
тіло від них щеміло, мов від кропиви, але не так самого жаління Тоня боялась,—
просто були їй нестерпні слизькі доторки цих морських потвор
%%%cit_comment
%%%cit_title
\citTitle{Залізний острів}, Олесь Гончар
%%%endcit

%%%cit
%%%cit_head
%%%cit_pic
%%%cit_text
Ще \emph{сонечко} не зараз мало сходити, а вже ловецьке товариство вирушило з табору.
Глибока тиша стояла над горами; нічні сумерки дрімали під темно-зеленими
коронами смерек; на густім, чепіргатім листю папороті висіли краплі роси;
повзучі зелені поясники вилися попід ноги, плуталися поміж корінням величезних
вивертів, спліталися в непрохідні клебуки з корчами гнучкої, колючої ожини та з
сплетами дикого, пнучого хмелю. З пропадистих, чорних, мов горла безодні,
дебрів піднімалася сивими туманами пара — знак, що на дні тих дебрів плили
невеличкі лісові потоки. Повітря в лісі напоєне було тою парою й запахом
живиці; воно захоплювало дух, немовбито ширших грудей треба було, щоб дихати
ним свобідно
%%%cit_comment
%%%cit_title
\citTitle{Захар Беркут}, Іван Франко
%%%endcit

%%%cit
%%%cit_head
%%%cit_pic
%%%cit_text
Материні гроші танули, як сніг на \emph{сонці}, Гоголь метався туди й сюди, пробував
сили в чиновництві (не пішов далі титулярного радника), в акторстві (не взяли
через погану дикцію, хоч згодом сучасники в один голос твердили про його
геніальну здібність читання!), в художничестві ("вольноопределяющимся"
відвідував уроки малюнка в Академії художеств), у педагогіці (наставником
сановних недоростків) і навіть у професорстві (двічі — професором історії, і
все марно), а тим часом за ним стояли цілі віки історії й незвичайної
талановитості, велика душа народу рідного, його неповторно барвисте, вільне,
широке, доброзичливе бачення і сприйняття світу, стояв народ, що віки цілі
промовляв до світу своїми піснями і думами, плачем і сміхом, дивуючись
безмірно, чому ж не чує його той великий світ, уперто й з надією ждучи, коли
прийде той, хто переллє його сміх і тугу, його жалі й надії, його неповторний
голос у свій голос, який стане чутний повсюди і розлунюватиметься віки цілі
%%%cit_comment
%%%cit_title
\citTitle{Три долі. Гоголь, Шевченко, Чехов}, Павло Загребельний
%%%endcit
