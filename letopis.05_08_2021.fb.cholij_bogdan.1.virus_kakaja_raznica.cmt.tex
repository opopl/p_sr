% vim: keymap=russian-jcukenwin
%%beginhead 
 
%%file 05_08_2021.fb.cholij_bogdan.1.virus_kakaja_raznica.cmt
%%parent 05_08_2021.fb.cholij_bogdan.1.virus_kakaja_raznica
 
%%url 
 
%%author 
%%author_id 
%%author_url 
 
%%tags 
%%title 
 
%%endhead 
\subsubsection{Коментарі}

\begin{itemize}
\emph{Oleksa Zodchiy}
Егоїзм - оце найзгубніший вірус людства

\begin{itemize}
\emph{Oksana Cha Chadaieva}
\textbf{Oleksa Zodchiy} егоцентризм. Егоїзм - це добре.
 · Ответить · 1 д.
\emph{Iryna Melnychuk}
\textbf{Oleksa Zodchiy}
І байдужість
 · Ответить · 22 ч.
\emph{Oleksa Zodchiy}
\textbf{Iryna Melnychuk} так, але це вже похідна від егоїзму та відсутності емпатії... Взагалі, в людства багато пороків і пора астероїда😂😂😂
\end{itemize}

\emph{Petro Hotych}
Згідний з кожним словом.

\emph{Liudmyla Levcheniuk}

дуже узагальнено про "стадо". "усі, хто не думають так, як я," - "стадо"? чи
всі, хто не мислять критично - "стадо"? тоді як визначаєш, що людина не мислить
критично? бо вона думає інакше? упередження та стереотипи є основою геноцидів і
диктатури. це вже науково та історично доведено. протилежністю до толерантності
та емпатії є егоїзм та інші деструктивні -ізми. зрештою, природа толерантності
також різна: людина пройшла навчання, навчилася розуміти іншого/людина
емпатійна/ людина байдужа/людина не думає/людина пливе за течією. тому це
небезпечне узагальнення "стадо". однодумці тебе підтримають)

\begin{itemize}
\emph{Petro Hotych}
\textbf{Liudmyla Levcheniuk} не шукайте глибоко те, що знаходяться на поверхні.
 · Ответить · 1 д.
\emph{Liudmyla Levcheniuk}
ну так, згідна, залежить, як аналізувати написане.. тому не люблю узагальнень, бо вони багатозначні.
 · Ответить · 1 д.
\emph{Bohdan Cholii}
\textbf{Liudmyla Levcheniuk} Стадо, це ж метафора, ти повинна це розуміти)
\end{itemize}

\emph{Галина Агафонова}

Це правда. Але люди не стають такими відразу. Все починається з того, коли
суспільство чи особистість відвертаються від свого Творця. А далі лише справа
часу…

«А що вони не вважали за потрібне мати Бога в пізнанні, видав їх Бог на розум
перевернений, щоб чинили непристойне. Вони повні всякої неправди, лукавства,
зажерливости, злоби, повні заздрости, убивства, суперечки, омани, лихих
звичаїв, обмовники, наклепники, богоненавидники, напасники, чваньки, пишні,
винахідники зла, неслухняні батькам, нерозумні, зрадники, нелюбовні,
немилостиві. Вони знають присуд Божий, що ті, хто чинить таке, варті смерти, а
проте не тільки самі чинять, але й хвалять тих, хто робить таке.»

До римлян1 : 28-32

\begin{itemize}
\emph{Ксенія Педаш}
\textbf{Galina Agafonova} ось тільки для християн було нормою вбивство
іновірців і переслідування інакомислячих, а так нічого. "Добрі" християни
мільйони людей вбили, покалічили, зламали, нав'язували решті народів Землі свою
релігію. Каються тепер, але вбитих не повернути.
 · Ответить · 1 д.
\emph{Галина Агафонова}
\textbf{Ксенія Педаш} , 

якщо люди роблять зло, прикриваючись Богом чи Біблією - то їх це не
виправдовує, і їм це зійде з рук. Це людські суди повні беззаконня, але Божий
суд завжди справедливий. Тому не керуйтесь емоціями чи людськими переданнями.
Якщо вас дійсно цікавить Істина, як вона є, то читайте Біблію. До того ж, в Ній
знайдете відповіді на питання, які хвилюють.

Хоча те, як ви описали, більше схоже на діла мусульман, а не християн (навіть у найтемніші часи).
 · Ответить · 22 ч.
\emph{Olena Karavaeva}
\textbf{Ксенія Педаш} Найбільшим християнином, мабуть, був Чингісхан))))
\end{itemize}

\emph{Teresa Zena}
Кожне слово підтримую 100\%
 · Ответить · 1 д.
\emph{Iryna Melnychuk}

Щось ви все в одну купу накидали. Шкода, бо це - контр-продуктивно. Ідеї
демократії, перемішані з гомофобією Свободи та інших ультра-правих партій, - не
найкращий аргумент для переконання. Власне так і працює пропаганда. А ви не
лише піддалися її впливу, а ще й перетворилися на її носія (((


\end{itemize}
