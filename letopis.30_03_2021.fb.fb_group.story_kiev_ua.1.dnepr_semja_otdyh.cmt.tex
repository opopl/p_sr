% vim: keymap=russian-jcukenwin
%%beginhead 
 
%%file 30_03_2021.fb.fb_group.story_kiev_ua.1.dnepr_semja_otdyh.cmt
%%parent 30_03_2021.fb.fb_group.story_kiev_ua.1.dnepr_semja_otdyh
 
%%url 
 
%%author_id 
%%date 
 
%%tags 
%%title 
 
%%endhead 
\subsubsection{Коментарі}

\begin{itemize} % {
\iusr{Светлана Манилова}
И я сегодня вспоминала! Была на Подоле, на Набережной. Тянет меня туда,
особенно весной. Ты, видимо, почувствовал и написал...@igg{fbicon.heart.red}

\iusr{Nataliia Samusenko}
Спасибі за приємні спогади @igg{fbicon.hearts.revolving} 
У дитинстві також з батьками їздили, ракетою, по Дніпру на відпочинок у Прохорівку, путівки від АНТК \enquote{Антонов}
Кожне літо катались на катері, зі своїми дітьми також така традиція в мене, кожне літо корабликом по Дніпру обов'язково @igg{fbicon.hearts.revolving} 

\iusr{Александр Киевлянин}
\textbf{Nataliia Samusenko} база отдыха \enquote{Солнечная}))

\iusr{Воробей Виталий}
Если не к концу апреля...

\iusr{Костянтин Черній}
Апрєль - не квітень (цитата Л.Д. Кучми)

\iusr{Ольга Лубягина}
Очень приятные воспоминания детства

\iusr{Мария Константиновская}
Такие дорогие сердцу воспоминания!

\iusr{Наталия Водяницкая}

Какие сказочно приятные воспоминания из детства! Я даже запах всего этого
почувствовала. И речного воздуха, ветра - на ракете; и тихой воды, запаха
весенней земли, речной рыбки - на берегу; и детского счастья - когда родители
молодые и вместе!.. Супер!  @igg{fbicon.smile} @igg{fbicon.heart.red}

\iusr{Раиса Карчевская}
Какие прекрасные воспоминания детства. Спасибо большое

\iusr{Владимир Николаев}
И кому это все помешало?

\begin{itemize} % {
\iusr{Iryna Saltanova}
\textbf{Владимир Николаев} никому не помешало. Река больше не судоходная, Днепр умирает. Причины как естественные, так и техногенные. В общем, экологи называют этот процесс естественным следствием постройки каскада ГЭС и водохранилищ по Днепру.

\iusr{вита шаповал}
\textbf{Владимир Николаев} не рентабельно

\iusr{Владимир Николаев}
\textbf{вита шаповал} я думаю что причина в неправильной приватизации Главречфлота и его инфраструктуры как в прочем на других стратегических объектах экономики.
\end{itemize} % }

\iusr{Людмила Краснюк}

Ах, какой приятный пост!  @igg{fbicon.heart.sparkling} . Очень тёплые, яркие
воспоминания ... читала, вспоминала ... тоже с учащённым
биением сердца. Спасибо большое! @igg{fbicon.heart.red}

\iusr{Юрий Вишенский}

\obeycr
Была когда-то красота!
Плыли - \enquote{Ракеты}, \enquote{Метеоры}.
Ходили баржи по Днепру!
И килем резали просторы...
\restorecr

\begin{itemize} % {
\iusr{Victor Gregory}

Да, было. Киевская Товарная биржа ходила с 1869-го по революцию и потом с
перерывами до 1922-го. Сейчас ходит Киевская Универсальная биржа. И вообще, на
август прошлого года было зарегистрировано 11 бирж.

\end{itemize} % }

\iusr{Андрей Надиевец}
Настоящий рассказ настоящего киевлянина

\iusr{Lora Litvin}

\ifcmt
  ig https://scontent-frx5-1.xx.fbcdn.net/v/t39.1997-6/s228x119/64706006_1211446515709593_6967343699017596928_n.png?_nc_cat=111&ccb=1-5&_nc_sid=ac3552&_nc_ohc=dpRvkxd4q5YAX9UTQUm&_nc_ht=scontent-frx5-1.xx&oh=00_AT9tF_oOswCwHw2S_YUoxBg_8qf_FxaFhnJOlwfcL4KTKg&oe=61BDB48C
  @width 0.1
\fi

\iusr{Валерий Стрельцов}
Станция речного такси. До Пирново 1,5 часа. Теплоход Турист 6 часов.

\iusr{Victor Gregory}
\textbf{Валерий Стрельцов}, обратно по течению шёл 3.

\iusr{Игорь Лысенко}
мне папа уговаривал за денежку порулить

\iusr{Анатолий Макаренко}

Дорогой Друг, спасибо за замечательный рассказ. Провели громкую читку в Семье.
Очень тепло, душевно и солнечно-ностальгично.

\begin{itemize} % {
\iusr{Петр Кузьменко}
Благодарю за лестный отзыв и внимание к моим эпистолярным пробам. @igg{fbicon.hands.shake} 
\end{itemize} % }

\iusr{Галина Буртолик}

\ifcmt
  ig https://scontent-frx5-2.xx.fbcdn.net/v/t39.1997-6/s168x128/93027172_222645632401274_7176243611145601024_n.png?_nc_cat=1&ccb=1-5&_nc_sid=ac3552&_nc_ohc=5bx1DLdOCosAX8NNX4d&_nc_ht=scontent-frx5-2.xx&oh=00_AT8Sa4aezLqwImEmpvaei8DWAPTRqU3PnZHlng6ieN3gAA&oe=61BE3A70
  @width 0.1
\fi

\iusr{Алексей Мась}

Катера эти реставрируются и ходят. Чайка на крыльях. Ходил с Минской на таком
на Десну два года назад, разговорился с капитаном. По идее это его катер на
последнем фото (а у него целая линейка их)

\begin{itemize} % {
\iusr{Вячеслав Бурдюков}
\textbf{Алексей Мась} \enquote{Стрела} назывались они вроде) имели движок газ, а некоторые ЗИЛ v8

\iusr{Вадим Реминский}
\textbf{Алексей Мась} ракеты и метеоры давно обслуживают людей в турцыи, а речные трамвайчики ходят по России, а унас в речпорту стоит ржавый, старый и никому ненужный водный транспорт... До хазiйнувались.

\begin{itemize} % {
\iusr{Алексей Мась}
\textbf{Вадим Реминский} я про катера на подводных крыльях (фото 2 и фото 3) это 

\href{https://volga-2.uaprom.net/p16213183-kater-strela-volga.html}{%
Катер Стрела (Волга) на подводных крыльях, volga-2.uaprom.net%
}

\ifcmt
  ig https://images.ua.prom.st/28539450_kater-strela-volga.jpg
  @width 0.4
\fi

\iusr{Алексей Мась}
Вот \url{https://katerplaza.io.ua}

\iusr{Алексей Мась}
И даже вот: 

\href{https://www.youtube.com/watch?v=tmY16Rq1rqE}{%
Катер Стрела Волга на подводных крыльях, 08.10.2016%
}, 

я с ним катался. У него то ли 5 то ли 3 катера, он их реставрирует и катает
людей


\ifcmt
  tab_begin cols=3,no_fig,center

     pic https://i2.paste.pics/7e4fa2aeef1ea10c9ed8b68881675e71.png
		 pic https://i2.paste.pics/3a14f4e1fd25a95e759d313bb7c205bc.png
		 pic https://i2.paste.pics/dc96fe4a98d15ea17d3b08f0eea3f878.png

  tab_end
\fi

\end{itemize} % }

\end{itemize} % }

\iusr{Раиса Лузан}

Читая этот рассказ о \enquote{прекрасном далёко}, я перенеслась в детские годы и
вспомнила , как мы отдыхали на Десне в деревянных домиках Транссигнала. Еду
готовили на кирогазе и костре, посуду мыли прямо в реке. Вода была чистая и
теплая. В Десне было много йода и на ногтях оставался йодный педикюр. Большое
спасибо автору за вспыхнувшие воспоминания.....

\iusr{Света Амеличева}

Пётр, очень приятно читать твои публикации. Они возвращают в радостное детство
и родной Подол. Спасибо большое.

\iusr{Любовь Лукасевич}

Дякую за спогади але болить душа наша держава спочатку все розвалить що таким
трудом простих людей створювалось а потім реставрує

\iusr{Larysa Lytvyn}

\ifcmt
  ig https://scontent-frx5-2.xx.fbcdn.net/v/t39.1997-6/s168x128/118211260_659965514618313_1559572347261946985_n.png?_nc_cat=1&ccb=1-5&_nc_sid=ac3552&_nc_ohc=s0iht4SPC8QAX_-cvYf&_nc_ht=scontent-frx5-2.xx&oh=00_AT_GrHOnTry5--q7qQ3PjFPi_NXv-NyL2mqCXUGX69G-6w&oe=61BDED77
  @width 0.1
\fi

\iusr{Marina Ganopolska}

Как здорово было просто покататься на катере до Осокорков или до Русановских
садов! Этот запах воды, этот вид на зелёные холмы и купола! Какой потрясающий
город!

А ещё мы ездили \enquote{Ракетой} к родственникам в Черкассы. Там был лес и
широченный Днепр.

\begin{itemize} % {
\iusr{Lara Ilich}
\textbf{Marina Ganopolska} 

а помнишь Никольскую Пустынь(Хотяновку)? - тоже туда катером добирались. Ракеты
шныряли по Десне. И причал -\enquote{кричал}, как 2-х летняя Сонечка называла


\iusr{Marina Ganopolska}
\textbf{Lara Ilich} ну как не помнить! Прекрасно было! Мы молодые, дети маленькие....

\iusr{Марина Парфентиева}
в Никольскую Пустынь на «лапте» подушке, и конечно на стреле 1-50 с носа ( очень дорого)

\iusr{Мария Константиновская}
\textbf{Lara Ilich} 

ооо!!! В Никольскую Пустынь добирались на \enquote{лапте} в студенческий лагерь
\enquote{химтурист}, о котором писали даже в журнале \enquote{Кругозор}!!! Родные, любимые
места...

\end{itemize} % }

\iusr{вита шаповал}

Спасибо. Стараюсь летом проехаться по Днепру, хоть и на прогулочном катере.
Беру с собой вино и курочку гриль, иду на нос корабля и блаженствую


\iusr{Александр Венге}
Очень вкусно написали! Спасибо большое.

\iusr{Валентина Луконина}
Как всегда, восхитительно!

\iusr{Петр Кузьменко}
\textbf{Валентина Луконина} Благодарю!

\iusr{Нина Опольская}
Великолепные воспоминания, спасибо!

\iusr{Павел Пауль}

Однажды на стреле лихие мужчины во главе с начальником пароходства в ноябре
мотанулись на охоту на устье Припяти. Шторм со снегом. Вывезла!

\iusr{Dimitri Statnikov}

Когда я учился в школе ходила крытая ракета в Канев на могилу Тараса Шевченко
несколько раз ездил, чудесные воспоминания..

\iusr{Ирина Архипович}

Замечательно рассказали!!! И свои воспоминания нахлынули!!
@igg{fbicon.hands.applause.yellow}  @igg{fbicon.face.happy.two.hands}
@igg{fbicon.hearts.two} 

\iusr{Мария Дидык}

Дуже ьцікава розповідь Пригадується й мені в 1968-69 рр - поїздки на теплоході
до Канева Було чудово Дякую

\iusr{Оксана Стефановская}

\ifcmt
  ig https://scontent-frt3-1.xx.fbcdn.net/v/t39.1997-6/s228x119/64848591_1211446745709570_6848216441691308032_n.png?_nc_cat=104&ccb=1-5&_nc_sid=ac3552&_nc_ohc=gDXW_AIJ8LYAX81g7jr&tn=lCYVFeHcTIAFcAzi&_nc_ht=scontent-frt3-1.xx&oh=00_AT-zrZodlUZKRQZLYfAac2SwmeIBWLfvB6eilo3AOFhU9w&oe=61BC7E86
  @width 0.1
\fi

\iusr{Валерий Владимирович}

\ifcmt
  ig https://i2.paste.pics/ffc7f2bbab6f56e6892211ce40d31cd1.png
  @width 0.1
\fi

\iusr{Natali Kornienko}

Спасибо. Интересно вспомнить и узнать подробности того времени. Тогда казалось
что это будет всегда и ещё лучше...

\ifcmt
  ig https://scontent-frx5-2.xx.fbcdn.net/v/t39.1997-6/s130x130/64706890_1211449942375917_4560497275343732736_n.png?_nc_cat=1&ccb=1-5&_nc_sid=ac3552&_nc_ohc=Orl1toJ1Xc0AX9gfKqR&tn=lCYVFeHcTIAFcAzi&_nc_ht=scontent-frx5-2.xx&oh=00_AT-ELSRepbIZUkaN9fTeJRFTOlcuvWO6NlhGY80e-DAfRw&oe=61BC9546
  @width 0.1
\fi

\iusr{Chyzhyshyn Oksana}
Наша остання поїздка в Канів з вітерцем на Метеорі.
Кінець травня, цвіли акації і всі береги радували око.
Їх садили якраз вздовж Дніпра, для укріплення берегів.

\iusr{Евгений Клопов}

Меня в детстве на таком катали. Руль от автомобиля Волга, шикарный красный
салон, длинный белый капот... воспоминания и впечатления на всю жизнь!

\begin{itemize} % {
\iusr{Алексей Мась}
\textbf{Евгений Клопов} именно такой в такой комплектации с хромированными шильдиками сейчас и ходит

\iusr{Евгений Клопов}
\textbf{Алексей Мась} а где ходит? Хочу своих мальчишек покатать...

\iusr{Алексей Мась}
\textbf{Евгений Клопов} возле церкви на набережной на Минской
\end{itemize} % }

\iusr{Татьяна Крышевич}

Петя, до мурашек по коже трогают твои воспоминания о детстве, о родителях.
Окунаешься в то время, время счастья и солнечной родительской любви, время
нашего детства. Вот умеешь! Моряк ли, писатель ли, поэт... Во всех ипостасях -
талантлив. Спасибо!

\begin{itemize} % {
\iusr{Петр Кузьменко}

Танечка! От всей души блгодарю тебя за высокий и тёплый отзыв о моих постах в
нашей замечательной группе! Вместе с одногруппниками ждём твоих прекрасных
публикаций в прозе и стихах!

\iusr{Дмитрий Антипов}
\textbf{Петр Кузьменко} 

А после того как вы добирались до места назначения, «Стрела» уходила и
возвращалась на договорённое время?

\iusr{Петр Кузьменко}
\textbf{Дмитрий Антипов} 

нет. Рулевого угощали припасами. Расслаблялись, загорали, отдыхали. Так было
удобнее. Таксометра в катере не было. Расплачивались по времени аренды.

\iusr{Татьяна Крышевич}
\textbf{Петр Кузьменко} Простите друзья, пока нет вдохновения
\end{itemize} % }

\iusr{Valentyna Prokhorenko}

\ifcmt
  ig https://scontent-frx5-1.xx.fbcdn.net/v/t39.1997-6/p240x240/67777252_2313400932112931_6724811104880427008_n.png?_nc_cat=111&ccb=1-5&_nc_sid=0572db&_nc_ohc=0FFqN4FXh1QAX-imXEa&_nc_ht=scontent-frx5-1.xx&oh=00_AT9tv49IVjiypCLUvB5CS-rNNnL8GbY-5OH9JjmuApS5Yg&oe=61BDE239
  @width 0.2
\fi

\iusr{Инна Шполянская-Кисиленко}

А вот я никогда не забуду выпускной 1981 года который нам устроили тоже на
карабле. В той самой 100 школе. Для двух выпускных классов сняли речной корабль
и устроили гульку на реке. Ещё до утра мы ходили по набережной на Подоле и
никак не хотели расставаться. Но всё равно почти все разъехались по разным
странам.

\begin{itemize} % {
\iusr{Петр Кузьменко}

Инна, спустя год по вашему примеру нам тоже устроили такой же праздник на воде
на речном трамвайчике \enquote{Эверест}. Инна, только это не корабль. Кораблём
называется только вооружённое судно. Однако впечатления от такого
увеселительного речного выпускного путешествия были незабываемы и остались в
памяти навсегда.

\iusr{Инна Шполянская-Кисиленко}
\textbf{Петр Кузьменко} 

Петя, ты прав речной трамвайчик. Спасибо за такие классные воспоминания. Уже
приезжая со своим младшим сыном, который родился в Израиле я его тоже катала по
Днепру не речном трамвайчике. @igg{fbicon.face.blowing.kiss} 


\iusr{Serge Serge}
У нас тоже был параходик в 81ом.
124ая школа.

\iusr{Инна Шполянская-Кисиленко}
\textbf{Serge Serge}

\ifcmt
  ig https://i2.paste.pics/333c42c3ce751df221b77d6a529cc00e.png
  @width 0.2
\fi

\end{itemize} % }

\end{itemize} % }
