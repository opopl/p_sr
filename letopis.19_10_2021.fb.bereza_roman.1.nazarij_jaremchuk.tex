% vim: keymap=russian-jcukenwin
%%beginhead 
 
%%file 19_10_2021.fb.bereza_roman.1.nazarij_jaremchuk
%%parent 19_10_2021
 
%%url https://www.facebook.com/permalink.php?story_fbid=10222288683980850&id=1078646188
 
%%author_id bereza_roman
%%date 
 
%%tags jaremchuk_nazarij.ukraina.pevec,kiev,kultura,muzyka,pesnja,sssr,ukraina,ussr
%%title НАЗАРІЙ ЯРЕМЧУК
 
%%endhead 
 
\subsection{НАЗАРІЙ ЯРЕМЧУК}
\label{sec:19_10_2021.fb.bereza_roman.1.nazarij_jaremchuk}
 
\Purl{https://www.facebook.com/permalink.php?story_fbid=10222288683980850&id=1078646188}
\ifcmt
 author_begin
   author_id bereza_roman
 author_end
\fi

\begin{verbatim}
	#українапишається #знайнаших 
\end{verbatim}

НАЗАРІЙ ЯРЕМЧУК. Двадцять восьмого травня 1995 року над Києвом сонце виборювало
у хмар право бути володарем у піднебессі. В цей день столиця України урочисто
відзначала своє свято – День міста Києва. На Співоче поле з’їхалося багато
артистів та творчих колективів, які в численних гримерках  та спеціально
відведених позаду велетенської сцени наметах наполегливо шукали для себе місце,
щоби гідно підготуватися до виступу. Багатотисячна публіка у наростаючих
оплесках вже відчутно знемагала, жадаючи від кожного із прибулих артистів
справжнього дива, ім’я якому – мистецтво, творчість, блиск і натхнення. 

\ifcmt
  ig https://scontent-lga3-2.xx.fbcdn.net/v/t1.6435-9/246362560_10222288666500413_1170427915509060464_n.jpg?_nc_cat=104&ccb=1-5&_nc_sid=730e14&_nc_ohc=Pzd-LPrD5QwAX-U-0C2&_nc_ht=scontent-lga3-2.xx&oh=5a9e2b0df64ffc59ddeada49f1a34b7b&oe=61941378
  @width 0.4
  %@wrap \parpic[r]
  @wrap \InsertBoxR{0}
\fi

З-поміж усього цього артистичного люду виділялася постать надмірно худого, не
високого на зріст чоловіка із густою чорною, педантично укладеною шевелюрою,
елегантно зодягнутого у світлу маринарку, розкішний сценічний багатий темний
галстук, який прикрашав білосніжну сорочку. Тільки уважно придивившись до
постаті цього чоловіка, багато-хто не відразу, але впізнавав у ньому колишнього
статечного красеня, що мав опінію правдивого символу української естради
Назарія Яремчука.

Коли ж концертне дійство нарешті розпочалося, а згодом сягнуло свого апогею, то
хтось із помічників режисера концерту ввічливо підійшов до Назарія і тихо
промовив: «до Вашого виступу хвилин 30-40, прошу готуватися.»

-Так, так, – ледь чутно відповів на це артист, я лишень зовсім на коротко
приляжу на ось цій лаві, бо щось знову той клятий біль у шлунку розгулявся і
так нестерпно діймає мене…

Прилігши на лаві, артист заплющив свої очі. 

30-40 хвилин… Як це багато, щоби у цьому відрізку позбутися того болю, який
розпирає мене із середини і стає для мене, немов якоюсь покутою за страшний
гріх, – подумалось чоловікові.

Коли ж біль й справді дещо втамувався, то на зміну тривозі думками Назарія
опанувало відчуття якогось солодкого марення, котре вже давно не відвідувало
його. 

Щойно він заплющив очі, чомусь у цей теплий травневий київський вечір йому
згадався холодний і засніжений останній день листопада 1951 року, про який йому
часто розповідали його батьки. Адже це був день, коли село Рівня, що поблизу
містечка Вижниці, у славному буковинському краю, разом із першими заметілями
пустилося в танок із принишклими від ще кволого морозу вічнозеленими горами,
які, мов велетенські витязі оточували село з усіх сторін, очікуючи перший подих
зими. У цей останній день осені ці гори, хоч і не бажали прощатися зі споминами
про таке ніжне і тепле карпатське літо, яке щороку ласкаво обіймає усі тутешні
верхів’я, плаї і полонини та своєю благодатною присутністю вселяє у них віру,
не даючи змоги розуміти, що на зміну йому прийде барвиста осінь, а вже після
неї, звиклі до п’янких пестощів тепла шпилі та хребти цих звитяжних гір, попри
усе їхнє небажання, таки вкриються пухкою веретою снігу, яку для них із кожним
новим ранком невтомно прястиме зима.

Я був четвертою дитиною в сім’ї мого няня Назарія та мами Марії, – роздумував
чоловік. Цікаво, але на час мого народження моєму батькові вже виповнилося 64
роки. 

Пам’ятається, як після щоденної нелегкої селянської праці, у дні свят та
празників завжди гостинну для численних відвідувачів нашу хату наповнював
лункий тенор батька, який у супроводі мандоліни, що іскрометно вигравала в
руках моєї мами, задавав тон, якому підтягувало усе товариство, котре зібралося
в домівці.  

Сумно, але недовго довелося звучати тій батьківській пісні. Бо коли мені
заледве виповнилося дванадцять років, батька не стало. А задовго до того, коли
першого вересня 1959 року я переступив поріг нашої сільської школи, мені
здавалося, що світ науки і рай, який панував у нашій хаті буде, якщо не вічним,
то бодай ще довго сущим і безхмарним. 

Вже після смерті батька мама перевела мене на навчання у Вижницьку
школу-інтернат, де я вперше самостійно розпочав давати собі раду поза домом.
Здобуття своєї першої освіти я завершив у Вижницькій середній школі № 1 у 1969
році. 

Пригадую, коли отримав із рук директора школи атестат про середню освіту, то
відчув, що далі у всьому мушу торувати для себе свій шлях самостійно. А, разом
і з цим, зрозумів, що зазвучав у мені вроджений від батька талант до співу.

Не без смутку пригадую, як після закінчення навчання в школі моя перша спроба
вступу на географічний факультет Чернівецького державного університету
виявилася невдалою через неспроможність подолати конкурс. Відтак, розпочалася
моя самостійна робота у складі геологів та науковців Західноукраїнської
геологорозвідувальної партії. 

А за якийсь час після цього, у 1969 році, моя співоча доля таки привела мене за
лаштунки сцени. 

Пригадую, як у вільний від роботи час я не знав іншої дороги, як та, яка вела
до Вижницького Будинку культури. Адже в його стінах під час репетицій лунала
чарівна музика у виконанні недавно створеного і ще зовсім невідомого для
широкого загалу вокально-інструментального ансамблю «Смерічка». Пригадую, як
під час цих репетицій, які іноді закінчувалися аж під ранок, завжди вимогливий
і вічно невдоволений досягнутим результатом Левко Дутківський раптом звернув на
мене увагу. А після цього запропонував вийти на сцену і в супроводі ансамблю
заспівати соло в пісні Ігоря Поклада на вірші Ігоря Бараха «Кохана», над якою
саме у цей вечір працював колектив. 

Сором’язливий і дещо розгублений тоді я вперше постав на кону сцени перед
«Смерічкою». Пізніше смеречани, пригадуючи цю мить мого першого сходження на
сцену, зізнавалися, що сприйняли мене не більше, як симпатичного з виду юнака з
тонкими рисами обличчя, з очей якого струменіла дивовижна енергетика. І,
мабуть, саме через це, я звернув на себе увагу Левка Дутківського – вдумливого
музиканта та композитора, а у майбутньому справжнього метра української
естради, під очільництвом якого у 1966 році при Вижницькому Будинку культури і
був створений вокально-інструментальний ансамбль «Смерічка». Не залишився
байдужим тоді ще дуже недосконалий мій спів для усіх співаків та музикантів,
які становили творчу основу ансамблю. 

На цей час у колективі вже працювали такі обдаровані співаки, як Василь
Зінкевич, Лідія Шевченко, Віталій Середа, Володимир Михайлюк, Володимир
Матвієвський, Марія Ісак, а поряд із ними такі талановиті музиканти, як Олексій
Гончарук, Олександр Шкляр, Юрій Шорін, Віктор Музичко, Валерій Бурмич, Леонід
Сіренко та інші. Тож до усіх них у той вечір доєднався і я, щоби разом
продовжити творити розпочату ними славну історію «Смерічки». 

Попри те, що вже в період із 1970 по 1975 роки творчий склад «Смерічки»
постійно піддавався кадровим ротаціям, наш дует у парі з Василем Зінкевичем у
неповторному тембральному злитті наших «золотих голосів», як їх часто називала
критика, приніс колективу та й усій молодій українській естраді небувалий успіх
та популярність. 

Знаковим для ансамблю став 1971 рік, коли на екрани СРСР вийшов музичний фільм,
знятий відомим режисером Романом Олексівим за сценарієм знаного львівського
тележурналіста Мирослава Скочиляса «Червона рута». Тодішній студент
Чернівецького медичного інституту, безмежно талановитий Володимир Івасюк
долучився до нас, запропонувавши свої неповторні пісні: «Червону руту»,
«Водограй», «Мила моя», «На швидких поїздах», «Залишені квіти» (Валерій
Громцев), які у нашому з Василем Зінкевичем виконанні, а також, завдяки
неповторному голосу Софії Ротару, стали правдивими хітами часу. Відразу після
виходу фільму солісти та «Смерічку» стали справжніми героями всенародної любові
і популярності.

У 1970 році, з другої спроби я таки вступив на навчання до Чернівецького
державного університету. Втім, вже у 1973 році, коли «Смерічку» було переведено
із Вижниці в штат Чернівецької обласної філармонії і стрімкість темпу
концертних виступів та гастролей ансамблю почала набирати шаленої амплітуди, я
змушений був перевестися на заочну форму навчання, щоби нарешті закінчити свої
вищі студії.

Відомо, що у кожної людини у житті бувають найщасливіші миті, хвилини, дні,
місяці й роки. Усі вони є найщасливішими, бо відчуття, пережиті в їхньому
бігові неможливо забути ніколи. І у мене був саме такий – найщасливіший рік.
Ним став 1972, коли в приміщенні Чернівецької філармонії Левко Дутківський
познайомив мене із Оленкою Шевченко. 

Мить, пережита під час цієї пам’ятної зустрічі, переросла для нас обох у палке
кохання, яке справжніми стожарами пристрасті і взаємного почуття зігрівало нас
цілих вісімнадцять років. 

Пригадую, як перший день 1975 року у мальовничому, рідному Оленчиному селі
Пилипець, що на Закарпатті, ознаменував собою не тільки початок нового року для
землян, а й став першим днем нашого спільного з Оленкою пошлюбного життя. Син
Дмитро, народжений у 1976 році, і другий син Назар, котрий прийшов на білий
світ роком пізніше, стали вінцем нашого з дружиною сімейного щастя. Ось тільки
гармонія нашої спільної пісні любові припинила своє благотворне звучання у 1990
році, враз перетворившись у справжнє потрясіння для мене, дітей та наших з
Оленкою родин. Сталося так, як сталося. На зміну нашій, здавалося, непорушній
любові увірвався неочікуваний для усіх вибір дружини на користь іншого… Попри
усі намагання зберегти сім’ю нам довелося пристати на вибір дружини,
підкоритися йому і крізь внутрішній опір, врешті-решт, таки його прийняти.   

Гідно пережити усі мої, дітей та рідних страждання допомогла, як завжди, пісня. 

Хоча і з нею – моєю незрадливою рятівницею і розрадницею теж не відбувалося усе
так однозначно. Адже, сягнувши свого апогею на найпрестижніших всеукраїнських
та всесоюзних конкурсах та небувалої знаності й широкої популярності в країнах
радянського зарубіжжя, куди  потоками переправлялися записи з виступами
«Смерічки», усе ж на початку сімдесятих років набираючий обертів морок
брежнєвсько-маланчуківської ідеологічної машини не оминув скреготом
переслідувань та цькувань усіх тих, хто наважився хоча б у чомусь мислити,
діяти і виявляти свою духовну самодостатність інакше, як це було прописано в
постановах, рішеннях і резолюціях з’їздів, конференцій та зборів правлячої
комуністичної партії. Серед цих «інших» не могли не опинитися й ми – талановиті
буковинці. Як результат, затиснуті в лещата творчої безвиході, у 1975 році,
один за одним, «Смерічку» покидають Василь Зінкевич, а згодом і сам Левко
Дутківський. Ще за якийсь час відчув і я, що перспектива мого артистичного
майбутнього, день за днем, перетворюється на вельми примарну реальність. 

І тут відбувся ще один нищівний удар долі. 

В травні 1979 року зі Львова приходить жахлива звістка. У лісі, що проліг між
Львовом і селищем Брюховичі, знайшли повішаним понівечене тіло Володі Івасюка. 

Довідавшись про мій намір їхати на похорон друга, мене затягали по кабінетах
відповідних спецслужб, наполегливо переконуючи і прозоро натякаючи, що ця
поїздка може коштувати мені подальшої артистичної кар’єри і неминуче внесе
суттєві корективи у ставлення влади до мене.

Втім, незважаючи на усі ці перестороги, 22 травня я, разом із Левком
Дутківським, тримаючи величавий вінок із білих квітів, стояв на чолі
багатотисячної жалобної ходи, в якій люди почергово змінюючи один одного несли
на своїх плечах труну, в котрій лежало сплюндроване тіло української пісні,
навічно поєднаної у цей трагічний день з великим іменем Володимира Івасюка,
щоби у цьому єднанні вже ніколи й нізащо не бути розлученим із ним аж до віку
усієї своєї майбутньої історії і свого буття. 

Згадується, як подальший розвиток подій, оповитий мовчанкою влади, став для
мене, зрештою, як і для багатьох, свідченням того, що ота  всесильна, на
переший погляд, влада, попри усе, боїться правди і здатна відступитися, коли
відчуває справжню силу неприхованого гніву, обурення і осуду, які нуртували у
думках, виступах і судженнях свідомої частини суспільства тоді при прощанні із
Володею і стали  відчутними для багатьох  найвищих чиновників, що засідали у
затишних партійних кабінетах.

Непросто переживши усі ці події і, зрештою, оговтавшись, я зрозумів, що моя
розлука із «Смерічкою», рівно ж, як і Левка Дутківського, на щастя, виявилася
не такою вже й тривалою. Бо вже у 1979 році ми знову працювали поруч на сцені,
відроджуючи колишню  популярну славу нашої «Смерічки». Це був час, коли
«Смерічка» отримала друге дихання, яким на повні груди на усьому просторі
України, республік СРСР, а також Німеччини, Чехословаччини, інших зарубіжних
країн знову зазвучала наша величава українська пісня.

У 1978 році мені було присвоєно почесне мистецьке звання заслуженого, а
дев’ятьма роками потому й народного артиста УРСР. З одного боку, це була велика
радість від усвідомлення того, що твій доробок став гідно оцінений державою. З
іншого ж, я завжди відчував, шо найвищою нагородою за мою працю є завжди
переповнені концертні зали, а у них тисячі широко розплющених очей слухачів, в
зіницях яких сяяло тільки одне – бажання пісенного дива.. І це було головним
завданням, від  якого проймалося усе моє єство і було ще більше присутнім
розуміння того, що мій голос не мав права допуститися навіть найменшої хиби.

А після цього ще один щемливий спомин вкрадається у мою пам'ять. У 1982 році,
після переходу Левка Дутківського на посаду художнього керівника Чернівецької
обласної філармонії, очільником «Смерічки» було запропоновано стати мені.
Погодившись і розпочавши тісно співпрацювати із такими відомими музикантами, як
Олександр Соколов, Володимир Прокопик, Віктор Морозов, співак Павло Дворський,
«Смерічка» знову почала промовляти до численних слухачів голосом своєї пісенної
правди, здобуваючи для себе все ширшу аудиторію нових шанувальників та щирих
поціновувачів.

Обійнявши таку відповідальну посаду, я розумів, що поза вже немалим здобутим
мною сценічним досвідом, мені катастрофічно бракує знань саме, як для керівника
колективу і постановника у його виконанні майбутніх програм. Тому, у середині
вісімдесятих років моє заочне навчання  на факультеті театральної режисури
Київського державного інституту культури ім. О. Корнійчука стало кроком у
заповненні цієї прогалини.

У 1990 році, після розлучення з Оленою, в моє життя знову увійшло кохання.
Дарина була вдовою і виховувала доньку. Проживала жінка в селі Тюдові, яке, хоч
і містилося в сусідній із Чернівецькою - Івано-Франківській області, а все ж
було сусіднім до моєї рідної Рівні. Другого лютого 1991 року в серці Гуцульщини
– містечку Косові, в церкві Святого Іоанна Хрестителя був освячений наш із
Дариною шлюб. А другого березня 1993 року у нас народилася донечка, названа на
честь моєї мами – Марією.

Пригадую, як невдовзі після весілля ми з Дариною придбали в Чернівцях
потребуючий глобального ремонту будинок, що на вулиці Інтернаціональній (нині
вулиця імені Назарія Яремчука. Авт.). Тоді вірилося, що ця занедбана будівля
колись перетвориться у наше затишне сімейне гніздечко. Та сталося інакше…

Радість нової віхи сімейного щастя, що розпочалася у нашому з Дариною житті
підсилилася ще однією знаменною подією. 

На початку дев’яностих років, з’їздивши кілька разів з концертами на
американський континент, я зустрів свого зведеного брата Дмитра, народженого
від першого шлюбу мого батька. Про нього до того часу я знав тільки те, про що
нишком, під великою таємницею, міг почути від своїх рідних. Зокрема, мені було
відомо, що у вирі непростих подій, які розгорнулися на Буковині наприкінці
тридцятих – початку сорокових років Дмитро примкнув до мельниківського крила
ОУН. Коли ж поразка руху опору стала неминучою, Дмитро, щоб уникнути
переслідувань, був змушений емігрувати до Канади. Саме там, за роки, ми й
зустрілися. 

Пам’ятаю, як тоді при зустрічі, очі, на двадцять сім років старшого від мене
Дмитра світилися радістю і сльози щастя котилися по його щоках. Бо сталося те,
про що він мріяв чи не увесь період свого заокеанського життя –  перемога над
невблаганною розлукою, яка, попри усе, стала доконечною.  

А після повернення в Україну на  мене знову чекав невгаваючий вир концертного
життя. Сотні і тисячі отих концертів та окремих виступів. Чи можливо згадати
або фрагментарно відтворити їх усіх у своїй пам’яті? Але один, серед усіх
інших, таки пам’ятатиму завжди…

Взимку 1993 року, під час виступів «Смерічки» разом із композитором Остапом
Гавришем та співаком Павлом Дворським, що відбувалися у Путивльському та
Верховинському районах Івано-Франківщини, перед одним із концертів в мою
гримерку, казна як, пробрався зашарілий від морозу хлопчина і мовчки, із
присоромленим виглядом, вручив мені конверт. Розпакувавши його, я прочитав
короткий і лаконічний текст, виведений розкиданим почерком, рукою, вочевидь,
одного із однолітків прийдешнього незнайомця. В листі йшлося про наступне:
«Слава Ісусу Христу! Вельмишановний, Назарію Яремчук, дуже просимо Вас пустити
нас на Ваш концерт. Ми будемо сидіти тихо. До нас рідко приїжджають артисти, а
Ви  –вперше. Ми без грошей, бо весь тиждень живемо у пришкільному інтернаті.
Дуже любимо пісню «Батько і мати» у Вашому виконанні. Будемо пам'ятати Вашу
доброту. Діти з пришкільного інтернату: Ромко, Женя, Таня, Володя, Сергій,
Василь, Марія, Світланка, Марія, Василь, Іван та Олександр».

Звісно дітваків, за моїм наполяганням, попри те, що усі квитки були продані,
пропустили на концерт і усіх їх посадили на спеціально доставлений ряд крісел.
Та от тільки початок того концерту не забуду ніколи. Бо, споглядаючи з-за
лаштунків за ними – принишклими і зачарованими в очікуванні так жаданого ними
концертного дійства, я плакав навзрид. Бо промайнуло у ці хвилини перед моїми
очами й моє дитинство, яке було теж, мабуть, як і їхнє, таким небагатим на
розкоші яскравих вражень і манливих сподівань.

А незадовго після цього найпам’ятнішого з усіх моїх виступів, в моє життя
холодним вихорем увірвалася ще одна тривога. Щоправда, на цей раз вона
виявилася набагато страшнішою з усіх, які мені доводилося переживати досі. Я
довідався про те, що я хворий... І хворобі моїй не здатен зарадити жоден лік із
тих, які за усю свою історію понапридумувало людство, рівно ж, як і не міг
спинити її перебігу жоден із лікарів, навіть той, якого називали із «золотими
руками». Відтак, хірургічне втручання навіть зарубіжних лікарів, яке стало
спробою, хоча б якось відтягнути мою передчасну кончину, зійшло намарно… 

І зараз, лежучи на цій широкій лаві і маючи її за єдиний порятунок від
нестерпного болю у шлунку, я подумки прогорнув, ніби книгу, сторінку за
сторінкою, усе своє насичене і яскраве життя, кожну мить якого я, не
задумуючись, присвятив рідній пісні і Україні, що завжди були з’єднаними для
мене воєдино.

У цю мить роздуми артиста перервав неголосний оклик помічника режисера
концерту, 

-Назаріє Назаровичу, за три хвилини ваш вихід на сцену. Прошу приготуватися.

«Назаріє Назаровичу, за три хвилини ваш вихід на сцену. Прошу приготуватися» –
чомусь знову і знову  пролунали у думках співака ці слова помічника режисера. 

Як часто впродовж усього мого артистичного життя мені доводилося чути цю фразу…
Мабуть й не злічити… 

Ну, то що ж, Назарій… Тебе ж батьки нарекли саме цим іменем, яке в перекладі з
давніх мов означає – «той, що присвятив себе Богу». Якщо ж це справді так, то
нехай в цю хвилину мого чергового виходу на сцену Бог, який завжди мене
супроводжував досі, не покине мене й зараз…

Після цієї думки Яремчук повільно підвівся з лави. По враз перекошеному обличчю
було видно, що гострий біль знову наздогнав його після нетривалої перерви.
Втім, чоловік випростався і, втамовуючи нестерпний біль, налаштував вираз свого
обличчя на таку вкрай необхідну для цієї миті сценічну посмішку. А після цього,
почувши добре знайомі акорди вступу до пісні «Родина» композитора Олександра
Злотника та поета Вадима Крищенка, написаної напередодні спеціально для
першовиконання саме ним, попрямував на сцену, де його довгоочікувану з’яву
зустрів вир оплесків чільної публіки, яка не вгавала від того, що нарешті
побачила свого улюбленця.

Може в житті хтось принаду підкине

У чарівничих, звабливих очах.

Тільки родина, як зірка єдина,

Твій порятунок – надійний причал.

Мій голос звучить не так свіжо і не так впевнено, як завжди, – навперебій
роїлися думки в голові Назарія під час співу. Біль, цей клятий біль… Чи
облишить він колись моє вже до краю змордоване ним тіло? Якщо ж ні, то бодай
хоча б зараз, доки я докінчу свій виступ… Нехай хоча б на мить він – цей клятий
біль відступить… 

Але ж сонце… Яке яскраве сонце над Києвом, а значить над усією Україною… Воно –
це вселенське сонце осяває усміхнені і радісні обличчя всіх цих людей, які
зараз слухають мій спів і підтримують мене своїми оплесками… Тож я мушу, я
повинен співати для них,..  я маю линути своїм голосом назустріч кожному
промінчику цього ласкавого сонця, яке прагне додати мені віри, надії і любові
до усіх цих людей, до світу, до України…

При цих уривчастих думках голос Назарія із кожним тактом набирав все більшої
снаги. У кожній наступній ноті цей голос прагнув злетіти все вище і вище, аж у
саме піднебесся... 

Родина, родина від батька до сина,

Від матері доні добро передам.

Родина, родина – це вся Україна

З глибоким корінням, з високим гіллям.

І на цих словах пісня, розправивши свої могутні крила, злетіла-зазвучала над
Києвом, над Україною, над усім світом, несучи у своєму леті велику правду,
безмежну любов і невгасиму пам’ять…

30 червня 1995 року великого буковинця, талановитого українця Назарія Яремчука
не стало. 

Навічно змовк голос, який красою своїх яскравих фарб подарував для нетлінної
пам’яті України дивовижний пісенний розмай, що став немеркнучим сузір’ям у небі
над Україною. І кожна з цих зір не перестає сяяти своїми незабутніми назвами:
«Стожари», «Гай, зелений гай», «Гей, ви, козаченьки», «Чуєш, мамо», «Горить
горобина», «Батько і мати», «Запроси мене у сни»… усіх не перелічити.

Похорон співака, що відбувся на міському кладовищі у Чернівцях, став свідченням
велелюдної любові до митця, який у віці всього сорока чотирьох років так
передчасно зронив яскравий квіт свого пісенного цвіту, щоб проростав він у
майбутньому новими талантами, на які завжди була багатою наша рідна Україна. 

…Коли на землю на зміну дню спускається вечір, то людину огортають роздуми, що
доброго я зробив за цей день? Коли небо закривають густі дощові хмари, то
людське серце охоплює тривога, чи бува, стихія не стане на заваді усьому
задуманому і омріяному. Коли ж у вирій відлітає молоде життя, забираючи з собою
голос, в якому, мов у барвистому розмаю вигравала веселка почуттів, віри і
любові, то сиротіють  люди, бо без пісні їм несила жити…

Назарій Яремчук – незабутня пісня України, що перейшла у її вічну пам'ять і
завжди живу славу… 

Роман Береза.
