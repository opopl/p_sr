% vim: keymap=russian-jcukenwin
%%beginhead 
 
%%file 08_12_2020.news.ru.lenta_ru.2.planeta
%%parent 08_12_2020
 
%%url https://lenta.ru/photo/2020/12/08/reuters
 
%%author 
%%author_id 
%%author_url 
 
%%tags 
%%title Планета не очистилась
 
%%endhead 
 
\subsection{Планета не очистилась}
\label{sec:08_12_2020.news.ru.lenta_ru.2.planeta}
\Purl{https://lenta.ru/photo/2020/12/08/reuters}

\begin{leftbar}
	\bfseries
	Хаос, насилие и желание бежать: самый тяжелый год глазами фотографов со всего мира
\end{leftbar}

2020 год оказался особенным. Настолько особенным, что многим хотелось бы просто
вычеркнуть его из памяти. Не позволят это сделать фотографы агентства Reuters,
которые весь год, несмотря на вспыхнувшую пандемию и мировую изоляцию,
продолжали скрупулезно документировать каждое событие на планете. Reuters
представило лучшие снимки своих фотографов, сделанные в 2020-м, «Лента.ру»
публикует наиболее яркие из них.

\ifcmt
pic https://icdn.lenta.ru/images/2020/12/02/18/20201202181520916/preview_e0053036f6ac3bb12095901665942d74.jpg
fig_env wrapfigure
\fi

Cупружеская пара из американского Сент-Луиса, штат Миссури, с оружием в руках
прогоняет сторонников движения Black Lives Matter от своего дома. По словам
автора снимка Лоуренса Брайанта, в тот день 29 июня он испытал беспокойство,
когда женщина вышла к протестующим и направила пистолет в их сторону.
«Множество фотографий, сделанных тогда, в первую очередь фокусировались на том,
как они держат оружие, но для меня это не вся история. Я хотел показать, что
там были и мирно протестовавшие люди», — добавил фотограф.
