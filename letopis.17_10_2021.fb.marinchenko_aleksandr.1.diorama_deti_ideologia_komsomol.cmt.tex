% vim: keymap=russian-jcukenwin
%%beginhead 
 
%%file 17_10_2021.fb.marinchenko_aleksandr.1.diorama_deti_ideologia_komsomol.cmt
%%parent 17_10_2021.fb.marinchenko_aleksandr.1.diorama_deti_ideologia_komsomol
 
%%url 
 
%%author_id 
%%date 
 
%%tags 
%%title 
 
%%endhead 
\subsubsection{Коментарі}

\begin{itemize} % {
\iusr{Serge Lunin}

"Петлюра захищав Вітчизну від червоних банд; бійці Червоної армії зламали
хребет фашистським загарбникам..."

... з якими колаборувала певна кількість уенерівців (петлюрівців).

\begin{itemize} % {
\iusr{Alex Marinchenko}
\textbf{Serge Lunin} Незручні факти завжди можна опустити)))
\end{itemize} % }

\iusr{Viacheslav Anissimov}

Все відповідно до чинних ідеологічних настанов УІНП. В кожній середній школі
наполегливо культивують цей патрійотичний абсурд.

\begin{itemize} % {
\iusr{Alex Marinchenko}
\textbf{Viacheslav Anissimov} 

Всі наші, всі герої. Коли на початку 2000-х почав вивчати російську
історіографію/публіцистику, то звернув увагу як деякі автори намагаються
вигадати таке собі безконфліктне минуле - і червоноармійці гуд гайз, і власівці
нічого так собі, і Деникін молодець, і Фрунзе теж. "В усіх своя правда!!!"
Просто обожнюю цей вислів. А в минулому там теж всі наші гарні правителі,
просто вони по-різному називалися. Князі-царі-імператори-генсеки-президенти...

\iusr{Новосельский Андрей}
\textbf{Alex Marinchenko} 

дозвольте спитати, як би Ви переробили спiтч вчителькi, тобто хто, на вашу
думку, гуд гайс? I чому дiвчата хочуть стати вiйсковими (мiлiтаризм)?


\iusr{Alex Marinchenko}
\textbf{Новосельский Андрей} 

Я на такий спітч просто не здатен, бо немає таких акторських талантів + в моєму
публічному історичному наративі немає наших/ваших, гарних/поганих,
героїв/антигероїв. В мене є особисті уподобання, але поки про це не питають, я
з ними нав'язливо не лізу до інших людей

\iusr{Макс Бужанский}
\textbf{Alex Marinchenko} , 

если прекратить подходить исходя из примитивного хорошо- плохо, и заменить его
концепцией- выдающийся, то появится хоть какой то смысл, давно безнадёжно
потерянный)

\iusr{Новосельский Андрей}
\textbf{Alex Marinchenko} 

припустимо, Ви вчитель - программа вимагаэ, дата зобов'язуэ, дiти чекають -
кого в приклад?

\iusr{Valeriy Kasyanenko}
\textbf{Alex Marinchenko} 

є таке в росіян) в них еклектика ще є: бути сталіністом, олігархом та
православним одночасно...

але як би нам сприймати українців в лавах ЧА і УПА? Хто, на ваш погляд, гуд, а
хто не гуд?

\iusr{Alex Marinchenko}
\textbf{Valeriy Kasyanenko} 

Тут справа не у тому, щоб дати якусь однозначну й остаточну відповідь, а
розповісти дітлахам про складність вибору (або його відсутність) в
екстремальних умовах війни. Тут не може бути якоїсь математично правильної
відповіді. + ми все ж таки не можемо відкидати регіональні відмінності. Обидва
мої діди з Дніпропетровської області. Які в них були шанси потрапити до лав
УПА, а які до лав РСЧА? Мабуть не випадково вони опинилися саме в РСЧА. А якщо
б вони були родом з Тернопільської області?

\iusr{Viacheslav Anissimov}
\textbf{Новосельский Андрей} 

А чому вчителі чогось чекають? Зараз у середньої школи є академічна автономія.
Вони можуть самі розробляти програми, у тому числі експериментальні, і
працювати за своїми програмами. Всі ж навчалися в педагогічних інститутах ))

\iusr{Alex Marinchenko}
\textbf{Viacheslav Anissimov} 

До речі, якщо називати речі своїми іменами: я за 5 років викладання якогось
прямого тиску "зверху" не відчув на собі, хоча з офіційними підручниками ніколи
не працював і всі про це знали. Тут справа особистого вибору - пристосуванство
і флюгерство, або власна позиція, яка може відрізнятися від офіційної лінії. А
діти не тупі, щоб не розуміти різницю. Хоча деякі вчителі чомусь впевнені у
тому, що цим "недолугим карапузам" можна вішати будь-яку лапшу на вуха.

\iusr{Новосельский Андрей}
\textbf{Вячеслав Анисимов} питання теж саме, кого включати в програми в парадигмi зрадоперемоги, тобто ставити у приклади на свято (як в постi)?

\iusr{Alex Marinchenko}
\textbf{Новосельский Андрей} Відмовитися від самої парадигми)

\iusr{Новосельский Андрей}
\textbf{Alex Marinchenko} 

як вiдмовитись? наприклад, Ви кажете, "дiти, берiть приклад з воiнiв Червоноi
Армii, тим паче ваших родичiв", тут включаются "послiдовники парадигми" -
окупанти, "пили б баварське", "нiмокгвалтували"...

I таких прикладiв "тьма" (або тумен  @igg{fbicon.smile} 

\iusr{Serge Lunin}

Выход один: прекратить преподавать историю в школе. Это всегда служило лишь
одной цели — воспитанию психопата из заурядного человека (а к востоку от Одера
служит и до сих пор).

\iusr{Лариса Бойко}
\textbf{Новосельский Андрей} 

"Беріть приклад" - то совєтчина. До сих пір тіпає від того, що заставляли
писати твори чого я повинна брати приклад з Зої Космодем'янської. Мені до
вподоби було брати приклад з своїх батьків.

\iusr{Новосельский Андрей}
\textbf{Лариса Бойко} тобто, на день захисника приклади захисника виключити?

\iusr{Лариса Бойко}
\textbf{Новосельский Андрей} 

Звісно. В мене багато друзів-АТОвців. Але це пуста фраза - "беріть приклад з
Петра Петренка". Я захоплююсь своїми друзями. Але сказати діткам -"Будьте як
Петренко" не можу. Бо в кожного свій шлях.

\iusr{Valeriy Kasyanenko}
\textbf{Alex Marinchenko} 

я з вами згодний, звісно. Сам такої думки. Скажу більше, в Старожитній Кам'янці
(АНД район), яку я трохи досліджую як аматор, та на Амурі був осередок ОУН. І
немало місцевих хлопців та дівчат було залучено. Цю б замовчану довгий час тему
теж би дослідити історикам.

Отже, питання як же показати цю складність та розмаїття? Все рівно прийдеться
творити загальнонаціональний наратив і говорити про героїв УПА і РСЧА,
захисників України, хоч і з різним кутом бачення ідеальної України. І в дітей
може виникнути амбівалентні відчуття. Чи не вийде знову ж таки еклектика на
практиці? Чи реально показати всю складність війни в шкільній програмі? Ви
правильні речі кажете, але я не зовсім розумію, як пояснити дітям антагонізм
цих процесів...

\iusr{Новосельский Андрей}
\textbf{Лариса Бойко} це сучаснiсть, мова про бiльш давню iсторiю

\iusr{Viacheslav Anissimov}
\textbf{Новосельский Андрей} 

що недавня, що давня історія - проблема одна. Якщо метою є не осмислення
складних речей, а пишання 'нашими славетними героями' відповідно до кон'юнктури
ідеологічного ринку, то така 'наука' не потрібна.

\end{itemize} % }

\iusr{Andrey Nikitin}
А вдруг я славный потомок половцев. Обидно, понимаешь

\begin{itemize} % {
\iusr{Alex Marinchenko}
\textbf{Andrey Nikitin} Мои ноги растут прямиком из Триполья!

\iusr{Serge Lunin}
\textbf{Andrey}, в таком случае вы не называли бы кипчаков аллоэтниномом.
\end{itemize} % }

\iusr{Алексей Браточкин}

Так хорошо пост начался, что я вдруг чего- то с утра хохотать начал. Спасибо:)
У нас тут есть Музей современной белорусской государственности, местный
идеологический проект , где жизнь и история начинается только после 1994 года,
а не раньше) Я там к одной группе школьников пристроился экскурсию послушать
пару лет назад. И когда начал задавать вопросы- а где нарратив и люди из конца
80-х начала 90-х, кроме Лукашенко, политики? Так на меня набросились две
учительницы, которые сопровождали группу. Довольно агрессивно)) Смотри- у нас
из истории много кого исключают, у вас всех релятивистски включают:)))))))))

\begin{itemize} % {
\iusr{Alex Marinchenko}
\textbf{Алексей Браточкин} У нас инклюзивная версия!)))

\iusr{Алексей Браточкин}
\textbf{Alex Marinchenko} именно:)
\end{itemize} % }

\iusr{Катерина Саченко}
Тихою сапою... таки хочуть запрягти українців у збудування АЗІОПИ.

\begin{itemize} % {
\iusr{Serge Lunin}
\textbf{Катерина}, Україна завжди була й буде Азіопою. Не тому що хтось щось накидає, а тому що більшість українців — пристосуванці та невігласи. Європа — це лише самовиховання. А в нас домінує попит на істерику, ненависть і брехню.
\end{itemize} % }

\iusr{Olga Stasiuk}

good job, Alex, dear. keep going, розбивайте ті маразми своїм світлим
розумом:).

\iusr{Andrey Nedzelnitsky}

Интересно, если Украина когда нибудь на законодательном уровне воплотит Минские
соглашения, следствием чего станет совершенно другая культурная политика.
Интересно, что будут говорить тогда. А в принципе конечно печально когда все в
кучу. Вот когда смотрю западные программы, то там да акценты делаются на
неудобных страницах, а потом да поясняя что все начиналось с живых людей.
Добавлю, что самое интересное, и к полякам и татарам совершенно другое
отношение, в пику Москве, так что это ещё с Союза осталось

\iusr{Andrey Nedzelnitsky}

Ну и на Востоке аж никак не мир и спокойствие сейчас, как бы не описывали этот
конфликт, что признают все. Даже в гуманитарном разрезе, не то что военном

\iusr{Алексей Шимановский}

Ну, "все наши, всё молодцы" - это не самая плохая концепция на свете. Это
концепция, во-первых, не злая, а во-вторых, она работает на внутреннее единство
общества, а не против.

Вот честно, бывает значительно хуже.

\begin{itemize} % {
\iusr{Andrey Nedzelnitsky}
Только не одновременно

\iusr{Алексей Шимановский}
\textbf{Андрей Недзельницкий} история вообще штука тяжёлая. Самый лучший вариант - гордиться не прошлым, а настоящим.

\iusr{Andrey Nedzelnitsky}

С настоящим у нас честно пока не очень. Вопрос то ещё в том что и с историей у
нас тоже грустно. Просто чтобы оправдать жуткие провалы сейчас, рассказывают
как плохо было когдась. Поскольку по банльным социально экономическис
показателям страна уступает даже не совсем развитым соседям. Вот это проблема

\iusr{Алексей Шимановский}
\textbf{Андрей Недзельницкий} К сожалению.

\end{itemize} % }

\iusr{Yuriy Korogodskyy}

Заплутався. Червоні банди зламали хребет хазарам. Але хазари не стали жертвами
Голокосту. Петлюра у дитинстві бачив повстання Гонти. А гроші на діораму про
Кучму на Тузлі дав Приватбанк. Десь так)))

\iusr{Лариса Бойко}

Ну свято ж так і називається "захисників та захисниць". А вчителька а-ля совок.
Приємно, що в другому класі є розумні дівчатка.

\begin{itemize} % {
\iusr{Новосельский Андрей}
\textbf{Лариса Бойко} чомусь забули трансгендерів, вони що чужі на цьому святі...
\end{itemize} % }

\iusr{Olga Yadlovska}
Добре, що діти адекватно реагують.

\iusr{Соловйов Віталій}

Ну це просто бажання все поєднати в українському політикумі.... Друга світова
це трагедія і ніяка не героїчна доба. Люди просто неуважно дивляться фільм в
Діорамі і не розуміють трагедію чорносвитників.....

\iusr{Tamara Paslavska}

Цікаво, якби на місці вже не молодої вчительки була молода .. Що за розмова була
б.. Слово є небезпечним, бо може змінити реальність. Але чомусь мені здається, що
це не той випадок. Тут вже є діагноз.

\iusr{Dmytro Lukin}

Петлюра боронив вітчизну від червоних чоловічих банд, червоні банди боронили
вітчизну від фашистських банд

\begin{itemize} % {
\iusr{Dmytro Lukin}
А дівчата потішили
\end{itemize} % }

\iusr{Andriy Liuty-Liutenko}
Шкода, що в цій промові не знайшлося місця славетним воякам вермахту та Waffen SS

\iusr{Yevgen Luniak}

Давно пора встановити якийсь Міжнародний чоловічий день, і буде всім щастя!

\begin{itemize} % {
\iusr{Alex Marinchenko}
\textbf{Yevgen Luniak} Слушна думка!)

\iusr{Лариса Бойко}

Міжнародний чоловічий день відзначається 19 листопада. Вже зовсім скоро. Та
совок нєпобєдім. 23 лютого то наше всьо. Отета мужской день у багатьох і
головах. Включно моїх співробітників і їх дітей, які в совєцькій армії не
служили, але "23 февраля".

\end{itemize} % }

\iusr{Sofia Grachova}

Я в дитинсиві через таку пропаганду зненавиділа все, що стосувалося Великої
вітчизняної. Дуже відчувався нещирий офіціоз, тому тема війни поділилася на дві
цілком окремі. Був дід, який воював, але ніколи не виголошував патетичних
промов, і взагалі про війну не говорив. Бцв прадід, який обороняв Севастополь,
а потім провів трироки в полоні. Він теж не говорив про війну, якщо його не
питали, а жив своїм життям, у сучасності. А з другого боку був всілякий
шкільний офіціоз, розмови поо Молоду Гвардію (як рольові моделі, хоча це лише
страшна історія про молодих людей, які трагічно загинули), купа всяких фільмів
по телевізору, частіше поганих, ніж добрих. І от на все, що належало до другої
категорії, вже років у вісім в мене виробилася стійка алергія. І цілком
імовірно, що в нинішніх дітей, від таких промов, як у пості, теж буде алергія і
на сучасну війну, і на воєнну історію

\begin{itemize} % {
\iusr{Лариса Бойко}
\textbf{Sofia Grachova} 

Мій дід зовсім нічого не розповідав про війну. Показував тільки медаль "За
отвагу". З гордістю її іноді діставав і роздивлявся. Знаю тільки, що був
артиллеристом.

\iusr{Sofia Grachova}
\textbf{Лариса Бойко}\textbf{Лариса Бойко} 

Мого діда забрали у військо при визволенні Луганщини. Йому тоді ще 17 років не
було. Дійшов до Німеччини і повернувся живим. Але про війну зовсім не любив
говорити. Тільки коли я одного разу щось його спитала, то відповів: ти бачила,
як людина без голови на танку їде? А я бачив!

\end{itemize} % }

\iusr{Анжела Беляк}

@igg{fbicon.face.tears.of.joy}{repeat=3} перенесли 23 лютого на жовтень


\end{itemize} % }
