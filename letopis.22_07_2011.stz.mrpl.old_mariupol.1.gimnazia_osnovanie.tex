% vim: keymap=russian-jcukenwin
%%beginhead 
 
%%file 22_07_2011.stz.mrpl.old_mariupol.1.gimnazia_osnovanie
%%parent 22_07_2011
 
%%url http://old-mariupol.com.ua/gimnaziya-osnovanie
 
%%author_id mrpl.old_mariupol,jaruckij_lev.mariupol
%%date 
 
%%tags 
%%title Гимназия: Основание
 
%%endhead 
 
\subsection{Гимназия: Основание}
\label{sec:22_07_2011.stz.mrpl.old_mariupol.1.gimnazia_osnovanie}
 
\Purl{http://old-mariupol.com.ua/gimnaziya-osnovanie}
\ifcmt
 author_begin
   author_id mrpl.old_mariupol,jaruckij_lev.mariupol
 author_end
\fi

Пятница, Июль 22nd, 2011

2. Основание гимназий

\begin{quote}
\em\bfseries В начале июля 1874 года Хартахай подал мариупольскому голове заявление, в
котором обосновал необходимость создания в городе частной прогимназии, с тем
чтобы позднее пре­образовать ее в гимназию. Содержать училище Хартахай
обязывался на свои средства \enquote{при пособии городского общества}.
\end{quote}

\ii{22_07_2011.stz.mrpl.old_mariupol.1.gimnazia_osnovanie.pic.1}

Прижимистые мариупольские купцы, медлительные, когда требовалось потратиться на
общественные нужды, на этот раз раскошелились сравнительно быстро: доводы
Хартахая, что теперь им не придется нести расходы по содержанию своих детей в
гимназиях Таганрога, Бердянска, Харькова, что сынки их, завершив среднее
образование, получат льготы при отбывании воинской повинности, оказались
убедительными. К тому же Мариуполь только что стал центром обширного уезда,
отделившись от Александровского, и не иметь в городе собственной гимназии
становилось просто неприличным.

Тем не менее в управе шли горячие споры, избирались комиссии, вырабатывались
пра­вила и условия договора с Хартахаем, куда, между прочим, — на всякий случай
— внесли и такой пункт: 

\begin{quote}
\em\enquote{Город оставляет за собой право наблюдать через своих
уполномоченное за состоянием заведения и, в случае замеченных вредных для
учащихся недостатков, отказать в выдаче субсидий}. 
\end{quote}

И наконец большинством в 16 голосов против восьми постановили: 

\begin{quote}
\em\enquote{Поручить городской управе внести в смету
сего года для выдачи господину Хартахаю 2000 рублей на устройство прогимназии}.
\end{quote}

Открыв ее осенью 1875 года, Хартахай усиленно хлопотал о том, чтобы на этой
основе организовать гимназию. Он обивает пороги различных канцелярий в
Таганроге, Харькове, Екатеринославе, \enquote{положив гордость в карман}, пишет
верноподданнические письма высокопоставленным сановникам. Когда Д. А. Толстой,
министр просвещения, следуя морем из Таганрога в Бердянск, остановился на
мариупольском рейде, Хартахай во главе сколоченной им делегации городской думы
добрался к нему на пароход и употребил все свое красноречие, чтобы убедить
сиятельного графа: городу нужна гимназия.

В конце концов последовало высочайшее повеление: учредить в Мариуполе мужскую и
женскую гимназии.

Немало усилий приложил Хартахай к тому, чтобы обеспечить город достойными
педагогами. Сделать это было совсем не просто, потому что в тогдашнем
Мариуполе, по свидетельству современников, учителя зарабатывали меньше, чем
кучеры, дворники, домашняя прислуга.

Возникает естественный вопрос: что заставило самого Хартахая, успешно
работавшего (или служившего, как тогда говорили) в одной из лучших гимназий
Варшавы, прекрасного европейского города, потянуться в маленький, глухой,
захолустный Мариуполь и принять на себя многотрудные хлопоты, успех которых
никак невозможно было заранее гарантировать?

Ностальгия, тоска по родине? Возможно, но только ли это?

Может быть, желание улучшить свое материальное положение? В заявлении
мариупольскому городскому голове Хартахай писал: 

\begin{quote}
\em\enquote{Моя личная выгода заключается лишь в том, что я желаю на родине получать такое вознаграждение,
какое я получаю на месте моего служения, т. е. в Варшаве}.
\end{quote}

Может быть, им руководили чувства националистического характера?

Ни в коем случае.

Грек по национальности, он был русским интеллигентом, человеком широких
взглядов, передового мировоззрения, о чем свидетельствуют и его близость к
революционным демократам 60-х годов, и его печатные труды, где он, прямо или
косвенно, отстаивает право на развитие самобытной национальной культуры и
русских, и украинцев, и поляков, и евреев, и татар. В то же время он считал,
что изучение русского языка всеми народами, населяющими Россию, поможет им
приобщиться к великой русской культуре и прогрессивным идеям века. Отличный
знаток и пламенный пропагандист языка и культуры русского народа, Хартахай
неоднократно писал, что создание в Мариуполе гимназии объединит разноплеменное
население Приазовья, \enquote{свяжет его с отечеством неразрывной нитью единства
языка, умственного развития и общегосударственного прогрессивного
направления}.

Более вероятной представляется другая причина. В 70-е годы наметился новый
подъем революционного движения, началось \enquote{хождение в народ}. Многие
интеллигенты — учителя, врачи — оставляли в городах обжитые места,
переселялись в глушь и в служении народу видели счастье и смысл жизни.

Вполне логично, что Хартахай, который в 60-е годы исповедовал и проповедовал
идеи Чернышевского и Добролюбова, в \enquote{народнические} 70-е посчитал своим долгом
\enquote{сеять разумное, доброе, вечное} не в столичной Варшаве, а в родном Приазовье.
Не случайно, думается, именно в 70-е годы он издает и переиздает уже
упомянутый \enquote{Букварь} свой, \enquote{Книгу для чтения в народных школах}, а также
\enquote{Русские прописи для народных школ}.

Из Варшавы Хартахай приехал не один: он привез своего друга, с которым вместе
рабо­тал во 2-й гимназии, Ивана Эдуардовича Александровича, воспитанника
Киевского университета, и учительницу Петербургской Петровской гимназии
Александру Александровну Генглез. Эти люди вместе с Хартахаем стояли у истоков
среднего образования в Мариуполе. Они, очевидно, тоже разделяли народнические
взгляды, потому что иначе трудно объяснить их решение поменять Петербург и
Варшаву на Мариуполь.

Эти три человека были первыми интеллигентами Мариуполя, от них пошла местная
демократическая интеллигенция. И если сегодня мы встречаем на страницах
Большой Советской Энциклопедии и других авторитетных справочников имена
воспитанников мариупольских гимназий, внесших заметный вклад в отечественную
культуру, то следует при этом с благодарностью вспомнить и имена тех, кто стоял
у колыбели средних учебных заведений в городе.

\ii{22_07_2011.stz.mrpl.old_mariupol.1.gimnazia_osnovanie.pic.2}

Александра Александровна Генглез позднее стала директором женской гимназии
(Мариинской), попечителем которой был И. Э. Александрович. Иван Эдуардович
поработал год в прогимназии Хартахая, затем принял приглашение занять должность
секретаря Мариупольской уездной земской управы. Многие русские интеллигенты
видели в земствах возможность способствовать развитию края в интересах
населения. Многолетняя деятельность И. Э. Александровича на посту секретаря
земства была плодотворной: он занимался открытием народных школ и больниц,
прокладыванием дорог, ему, в частности, принадлежит книга \enquote{Краткий очерк
Мариупольского уезда}, вышедшая двумя изданиями — в 1884 и 1897 гг. О ее авторе
в \enquote{Мариуполе и его окрестностях} сказано следующее: 

\begin{quote}
\em\enquote{Он был главным советником
Хартахая и направлял, так сказать, его деятельность, как в деле открытия
гимназий, так и в устройстве их. С этой именно целью Ф. А. Хартахай, имея в
виду открытие гимназий в Мариуполе, пригласил И. Э. Александровича, состоявшего
преподавателем 2-й гимназии и юнкерского училища в Варшаве}.
\end{quote}

После кончины Феоктиста Авраамовича (он умер 25 марта 1880 года в возрасте 46
лет) по ходатайству местных властей было получено разрешение правительства
вывесить в гимназии портрет их основателя, Министерство просвещения учредило
стипендии имени Хартахая. Не забыли и заслуги его друга: была учреждена
стипендия имени И. Э. Александровича \enquote{для недостаточных учащихся Мариупольской
Александровской гимназии}.

В 1962 году, разыскивая людей, которые могли знать писателя Александра
Серафимовича в бытность его в Мариуполе в конце XIX века, встретился я с А. И.
Александровичем, сыном Ивана Эдуардовича. Направили меня к нему его бывшие
ученицы. Это были женщины весьма почтенного возраста, но о своем учителе
говорили они с юной восторженностью и благоговением. Они говорили мне, что
Александр Иванович, преподававший в женской гимназии русский язык и
словесность, был светлой личностью в Мариуполе. Закончив с золотой медалью
Александровскую гимназию, в основании которой принимал участие его отец, он
поступил в Нежинский лицей имени графа Безбородко, где некогда учился Гоголь.
После блестящего окончания первого курса А. И. Александрович, как лучший
студент, получил право спать в общежитии лицея на кровати, которую в свое имя
занимал, как гласила табличка на ней, Н. В. Гоголь-Яновский.

Лицей Александр Иванович окончил кандидатом, то есть с отличием, затем защитил
диссертацию и стал магистром, или, говоря по-современному, кандидатом
философских наук. Будучи человеком энциклопедических знаний, владея 17 языками,
он мог стать видным египтологом: его оставляли в лицее для подготовки к
профессорскому званию и предлагали научную командировку в Каир. Но он был
казеннокоштным студентом, учился, в частности, на средства Мариупольского
земства, и, хотя земство не возражало против поездки в Египет, Александр
Иванович посчитал своим нравственным долгом вернуться в родной город и служить
делу культурного развития Приазовья.

Мне рассказали еще вот какой любопытный случай из жизни Александра Ивановича.
Однажды в Министерство просвещения поступила бумага от учителей Мариинской
гимназии, которые жаловались на то, что Александрович их \enquote{заслоняет}.
Разбираться приехал то­варищ (то есть заместитель) министра. Он побывал на
уроках Александра Ивановича, подолгу беседовал с ним, был поражен кругозором
провинциального учителя и, уезжая, сказал
жалобщикам: \enquote{Да, вы правы, он вас заслоняет}.

***

Когда в 1962 году встретился я с Александром Ивановичем, ему было 90 лет. Да,
конечно, он хорошо помнит журналиста Александра Попова, который приходил в
гимназию. Она тогда уже располагалась в центре города у Хараджаева (сейчас на
этом месте дом со шпилем), было ужасно тесно, о чем Попов и написал в
\enquote{Приазовский край}. Он подписывался \enquote{Серафимович}, весь город читал, и это
подхлестнуло строительство собственного здания гимназии, то, которое и сейчас
стоит на Георгиевской (\enquote{На улице Первого мая}, — поправила Александра Ивановича
присутствовавшая при нашей беседе его жена).

При советской власти А. И. Александрович работал в педучилище, затем — в школе
красных командиров, которая располагалась в здании бывшего епархиального
училища. В том самом здании, где сейчас располагается Мариупольский
металлургический институт, ректором которого является профессор, доктор
технических наук Игорь Владимирович Жежеленко, внук А. И. Александровича,
правнук Ивана Эдуардовича.

— Я очень многим обязан своему деду, — сказал мне Игорь Владимирович.

Он познакомил меня со своей матерью, Екатерине Александровной, дочерью А. И.
Александровича.

— Моей крестной матерью была Анна Феоктистовна, младшая дочь Хартахая.

Надо ли говорить о том, какое впечатление произвела на меня эта фраза?! Изучая
жизнь и деятельность Хартахая, я имел дело с событиями полуторавековой —
вековой давности, рылся в архивах, вчитывался в редкие старинные издания, и мне
в голову никогда, конечно, не приходило, что о людях того бесконечно далекого,
как мне казалось, времени могу услышать от живого человека, нашего
современника.

После безвременной смерти Феоктиста Авраамовича И. Э. Александрович взял к себе
на воспитание младшую дочь своего друга Анну. Анна Феоктистовна вышла замуж за
преподавателя гимназии Владимира Викторовича Рудевича. Живя в Мариуполе, с ним,
зятем Хартахая, был дружен Серафимович.

Рудевичи — одна из самых интеллигентных семей старого Мариуполя. Один из
братьев Владимира Викторовича был известным архитектором в Вильно (Вильнюсе),
другой стал юристом. В том же 1962 году, когда я еще не знал, что в истории
Мариуполя был такой интересный человек, как Феоктист Авраамович Хартахай, меня
познакомили с сестрой В. В. Рудевича — Валентиной Викторовной Рудевич. При
встрече она сказала:

— Дамы обычно скрывают свой возраст, но я этого делать не стану, мне 87 лет.

Валентина Викторовна рассказала мне, как в конце прошлого (то есть XIX) века
она танцевала на благотворительных вечерах с Александром Поповым, или Серафймчиком,
как они его звали, каталась с ним на лодке. Он был вхож в их дом.

К истории среднего образования в Мариуполе имеет отношение не только Хартахай,
но и обе его дочери, которые преподавали в основанных их отцом гимназиях.
Преподавала французский язык и внучка Хартахая, дочь В. В. Рудевича.

От Екатерины Александровны Жежеленко я узнал еще одну любопытнейшую деталь: в
Варшаве Хартахай и Иван Эдуардович Александрович дружили с учителем физики и
субинспектором Владиславом Склодовским. Они вместе работали в гимназии на
Новолипской. Когда познакомились, дочери Склодовских Мане было полтора года.
Иван Эдуардович в 1907 году, за три года до своей кончины, был счастлив узнать,
что членом-корреспондентом Петербургской Академии наук избрана Мария
Склодовская-Кюри, ученый с мировым именем, дочь его друга Владислава, та самая
Маня, которую он в Варшаве знал ребенком.

Когда я прохожу по улице Карла Маркса мимо школы № 1, неизменно вспоминаю
Хартахая. Здесь открыл он первую в Мариуполе женскую гимназию. Так что школа
не случайно носит номер 1. Но думаю, что она могла бы еще быть и именной.

А когда прохожу мимо книжного магазина \enquote{Светоч}, думаю, что могла бы у его
входа быть установлена мемориальная доска. Ведь на этом месте стоял дом
мариупольского миллионера А. Д. Хараджаева, и город более двадцати лет, до
1899 года, арендовал здание под мужскую гимназию. В его стенах получили
образование многие люди, достойные, чтобы мы сохранили о них благодарную
память, но сейчас мы ограничимся перечислением (в алфавитном порядке) лишь
тех, о ком Большая Советская Энциклопедия поместила на своих страницах
биографические справки.
\bigskip

МИХАИЛ ИОСИФОВИЧ АВЕРБАХ (БСЭ поместила также его портрет), советский
офтальмолог, академик АН СССР, лауреат Государственной премии СССР - Основатель и
первый директор Центрального офтальмологического института имени Гельмгольца.
Участвовал в лечении В. И. Ленина. Премия имени М. И. Авербаха ежегодно
присуждается Академией медицинских наук СССР за наиболее выдающиеся заслуги в
области офтальмологии. В Москве установлен памятник И. Авербаху. Серебряный
медалист Мариупольской мужской гимназии.
\bigskip

ДМИТРИЙ ВЛАСЬЕВИЧ АЙНАЛОВ, советский историк искусства, профессор,
член-корреспондент Академии наук СССР, автор фундаментальных трудов, не
утративших своего значения и поныне.
\bigskip

РУДОЛЬФ ЛАЗАРЕВИЧ САМОЙЛОВИЧ, полярник с мировым именем, профессор, доктор
географических наук, \enquote{директор Арктики}, как его называли, руководитель
экспедиции на \enquote{Красине} по спасению Нобиле. В честь Самойловича названы пролив
и ледниковый купол на Земле Франца-Иосифа, бухта на Новой Земле, гора и
полуостров в Антарктиде.
\bigskip

ГЕОРГИЙ ИВАНОВИЧ ЧЕЛПАНОВ, русский психолог и логик. Вместе с Айналовым был
среди учеников Хартахая. Первый золотой медалист Мариупольской мужской
гимна­зии. Основатель и первый директор Московского психологического института
— крупного центра экспериментальной психологии. Автор многократно
переиздававшихся учебников психологии и логики, по которым училось не одно
поколение в России. О своей учебе у профессора Челпанова вспоминают Владимир
Маяковский, Мариэтта Шагинян, Вениамин Каверин, академик Н. М. Дружинин и
многие другие.
\bigskip

Мариупольскую мужскую гимназию окончили также академик АМН СССР и АН УССР,
лауреат Государственной премии СССР терапевт Вадим Николаевич Иванов и академик
ВАСХНИЛ, Герой Социалистического Труда, лауреат Ленинской премии Михаил
Иванович Хаджинов, но они учились уже в новом здании, на Георгиевской, о
строительстве которого расскажем позднее.

***

Преемником Феоктиста Авраамовича Хартахая на посту директора гимназии стал Г.
Нейкирх. Его имя избежало полного забвения потому, что он издал брошюру
\enquote{Краткая история Мариупольской гимназии}. Это издание представляет интерес
потому, что его автор использовал архивные документы, не дошедшие до нашего
времени.

Ничего примечательного не случилось за восемь лет директорства Н. Броницкого
(1883- 1891), зато долгую память о себе оставил

\bigskip
\emph{\textbf{(Продолжение следует)}}\par

\bigskip
\textbf{Лев Яруцкий}\par

\bigskip
\textbf{\enquote{Мариупольская старина}.}

%\ii{22_07_2011.stz.mrpl.old_mariupol.1.gimnazia_osnovanie.cmt}
