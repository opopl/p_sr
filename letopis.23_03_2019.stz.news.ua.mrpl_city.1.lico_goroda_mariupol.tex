% vim: keymap=russian-jcukenwin
%%beginhead 
 
%%file 23_03_2019.stz.news.ua.mrpl_city.1.lico_goroda_mariupol
%%parent 23_03_2019
 
%%url https://mrpl.city/blogs/view/litso-goroda-mariupol
 
%%author_id burov_sergij.mariupol,news.ua.mrpl_city
%%date 
 
%%tags 
%%title Лицо города Мариуполь
 
%%endhead 
 
\subsection{Лицо города Мариуполь}
\label{sec:23_03_2019.stz.news.ua.mrpl_city.1.lico_goroda_mariupol}
 
\Purl{https://mrpl.city/blogs/view/litso-goroda-mariupol}
\ifcmt
 author_begin
   author_id burov_sergij.mariupol,news.ua.mrpl_city
 author_end
\fi

\ii{23_03_2019.stz.news.ua.mrpl_city.1.lico_goroda_mariupol.pic.1}

Александр Викторович Нильсен рассказывал, что его отец Виктор Александрович –
главный архитектор Мариуполя – в свое время собственноручно нарисовал, не
пропустив ни одной детали, фасады всех мариупольских строений, выходящих на
улицы города. Они были изображены не порознь, а по кварталам, стоящими друг за
дружкой. Эту объемную работу Виктор Александрович выполнил не ради своего
удовольствия и, тем более, не для удовлетворения собственного тщеславия, а с
целью сугубо практической. Когда ему приносили на обязательное согласование
проект нового дома или реконструкции дома старого, он смотрел, как он будет
выглядеть в окружении уже существующих строений, привлекая для этого рисунки, о
которых сказано выше. К сожалению, эти рисунки вместе с квартирой В. А.
Нильсена погибли при сожжении Мариуполя гитлеровцами перед отступлением в
сентябре 1943 года. Александр Викторович, как мы видим, старался сохранить
своеобразие лица города.

\textbf{Читайте также:} 

\href{https://mrpl.city/news/view/trashtag-mariupolskij-park-do-i-posle-vsemirnogo-chellendzha-foto}{\#Trashtag: Мариупольский парк ДО и ПОСЛЕ всемирного челленджа, Кіра Булгакова, mrpl.city, 22.03.2019}

Следует отметить, что с самого его основания Мариуполь застраивался отнюдь не
хаотично. Имеется план Мариуполя 1782 года, на котором видны не только
расположенные в строгом порядке уличные кварталы, но и заранее распланированные
участки под усадьбы горожан. Городские власти строго следили за соблюдением
плана при строительстве, и особенно за тем, чтобы строения или их части не
выходили за \enquote{красную линию}. Скорее всего, и сами дома сооружались,
подчиняясь некоему подобию стандартов. Во всяком случае, известно, что
мариупольцы ездили в Таганрог, - мы бы сказали, - за типовыми проектами домов.
Вероятно, это было в тот период, когда Мариуполь с 1807 по 1859 года находился
в составе Таганрогского градоначальства.

Как известно, в сентябре 1943 года более 60\% жилых и общественных зданий
Мариуполя были уничтожены оккупантами. Еще шла война, но некоторые жители
начали обустраивать, как могли, погорелки. Оставшиеся бесхозными коробки домов
восстанавливали предприятия. Своим сотрудникам они выдавали кредиты,
строительные материалы, а уж все остальное было делом рук или денег будущих
обитателей. Как правило, первоначальный внешний вид строений сохранялся. В этот
период успешно работали архитекторы Иван Иванович Лехтер и Николай Иосифович
Никаро-Карпенко.

Вместе с тем, руководители крупных заводов настаивали на сносе разрушенных
зданий в старой части города и предлагали построить один новый город на левом
берегу реки Кальмиус, обосновывая это экономической целесообразностью. Против
этого решительно выступил главный архитектор города Александр Митрофанович
Веселов, назначенный на этот пост Советом Министров УССР в 1946 году. Он убедил
городское начальство, что такие предложения ошибочны. Но для восстановления
жилых домов, школ, больниц нужны были деньги из государственного бюджета. И
тогда председатель горисполкома Назар Львович Кудрявцев выехал в Москву
\enquote{выбивать} ассигнования. Его хлопоты дали результат. В 1950 году вышло
постановление Совета Министров СССР о выделении первых 120 миллионов рублей на
восстановление Мариуполя. К освоению этих денег приступили немедля,
финансировали возрождение исторической части города и заводы \enquote{Азовсталь},
Коксохим, имени Ильича.

\textbf{Читайте также:} 

\href{https://mrpl.city/news/view/stala-izvestna-stavka-turisticheskogo-sbora-v-mariupole}{%
Стала известна ставка туристического сбора в Мариуполе, Олена Онєгіна, mrpl.city, 21.03.2019}

В 1954 – 1956 годах на проспекте Республики (ныне пр.Мира) на месте
разрушенных появились дома № 13, № 10/20, № 4, дома со шпилями, открывающие
северную часть улицы Артема и зданий, продолжающих их, в большей степени, по
четной стороне, и в меньшей – по нечетной. Этот ансамбль создали архитекторы
под руководством известного украинского зодчего Льва Александровича Яновицкого.
В работе активное участие принимал А. М. Веселов. Кстати, некоторые «горячие
головы» настаивали ради экономии не устанавливать на зданиях украшения, в том
числе и шпили. Александр Митрофанович решительно выступил против этого  - и
добился своего. С выходом постановления о борьбе с излишествами в архитектуре,
инициированное Н. С. Хрущевым, осуществить застройку улицы Артема по проекту
мастерской Л. А. Яновицкого не удалось.

В 60 – 70 годы прошлого столетия в нашем городе, как и во всей стране,
развернулось невиданное ранее строительство жилых микрорайонов. Но
застраивались они типовыми пятиэтажными домами. Бывший первый секретарь горкома
партии Владимир Михайлович Цыбулько добился в Совмине и Госстрое УССР
разрешения возводить в Мариуполе жилые дома повышенной этажности в девять,
двенадцать и даже в пятнадцать этажей. Целью было придать силуэту города
своеобразие.

В 1989 году Главное управление архитектуры и градостроительства Мариупольского
горисполкома совместно с научными сотрудниками Мариупольского краеведческого
музея разработали проектные предложения по реконструкции и реставрации
исторической застройки города Мариуполя. Было проинвентаризировано ни мало, ни
много - две сотни объектов. Этот фонд был распределен на шесть следующих
категорий: памятники архитектуры местного значения; здания, которые должны быть
взяты на учет, как памятники архитектуры; здания, имеющие архитектурную и
художественную ценность, являющиеся элементами архитектурного ансамбля; здания,
имеющие архитектурную и художественную ценность; здания, не имеющие
эстетической ценности, но являющиеся элементом ансамбля и, наконец, ветхие
здания, подлежащие сносу. Все осталось, увы, на бумаге. Реализация нужного дела
уперлась в отсутствие финансирования. Подводя итог, можно сказать, что в
Мариуполе стремились соблюсти единство и индивидуальность облика города.
Правда, не всегда это удавалось по причинам как объективным, так и
субъективным.

\textbf{Читайте также:} 

\href{https://mrpl.city/news/view/yadovitye-tsveta-i-vyveski-na-polfasada-pochemu-v-mariupole-nelzya-fotografirovat-istoricheskie-zdaniya-foto}{Ядовитые цвета и вывески на полфасада: Почему в Мариуполе нельзя фотографировать исторические здания?, Яна Іванова, mrpl.city, 30.07.2018}

Но вернемся из исторического экскурса в наши дни. В прежние времена выход
строения за \enquote{красную линию} рассматривался почти как преступление. А что
сейчас? Предприниматели без всяких сомнений делают пристройки к фасадам, при
этом каждый на свой лад, иногда до минимума сокращая тротуары. И дома
приобретают вид комода в комнате нерадивой хозяйки, ящики которого выдвинуты на
разную длину, а из них торчит разноцветное тряпье. Такое можно увидеть на
проспекте Строителей близ магазина \enquote{1000 мелочей}. Дом № 24 по проспекту Мира,
его цокольный этаж полностью утратил первоначальный вид. Здесь каждый владелец,
оформляя вход в свое заведение, в полной мере проявил меру своей безвкусицы.
Нелепо выглядит и новая облицовка фасада отделения Ощадбанка, совершенно
негармонирующая с видом всего здания.

Участок проспекта Мира по нечетной стороне от улицы Казанцева до улицы
Леваневского должен был стать парадным пространством города. Он таким и был
вначале, его украшали монументальные художественные произведения местных
мастеров, предполагалось установить еще и барельефы с портретами выдающихся
мариупольцев. Теперь здесь царит безвкусица и откровенный вандализм.
Большинство мозаичных панно, выполненных художниками П. А. Котом, О. Б.
Ковалевым и В. С. Андилахаем, уничтожены, а оставшиеся осыпаются, да и к тому
же скрыты от зрителя. Рельефы В. К. Константинова и Д. Н. Кузьминкова, в свое
время получившие высокую оценку столичных искусствоведов, сейчас мало того, что
разрушаются, но еще совершенно теряются в окружении разномастных вывесок. А
один рельеф, изображающий женщину, безвкусно размалеван разными красками.

\textbf{Читайте также:} 

\href{https://archive.org/details/12_01_2019.sergij_burov.mrpl_city.istoria_hatka_na_torgovoj_ulice_mariupolja}{%
История: Хатка на Торговой улице Мариуполя, Сергей Буров, mrpl.city, 12.01.2019}

В последние десятилетия особенно пострадал внешний вид старой части города.
Без зазрения совести намеренно уничтожаются архитектурные детали, стены
перекрашиваются в дикие неестественные цвета, срубаются элементы фигурной
кладки, а вместо них появляются безликие пластиковые панели, выбрасываются
узорчатые кованые решетки прошлого века. Зачастую новые постройки совершенно
не согласуются с соседствующими строениями. Создается впечатление, что местные
архитекторы разрабатывают новый проект не для конкретного городского
окружения, а для пустого пространства.

Спору нет, возвращение исторического облика Мариуполю потребовало бы огромных
капиталовложений. Но выход мог бы быть найден, если бы кто-нибудь
воспользовался опытом других городов. Помнится, в середине 60-х годов в одном
из номеров журнала \enquote{Архитектура и строительство} была опубликована большая
статья о сохранении облика старого Таллина. Что там сделали городские власти?
Они обязали владельцев старинных зданий произвести реставрацию фасадов домов,
придав им первоначальный вид. На этот счет могут быть такого рода возражения.
Тогда были советские времена: горком сказал – и все под козырек. А сейчас –
капитализм, каждый хозяин действует по принципу: что хочу, то и ворочу. Но в
дореволюционном провинциальном Мариуполе при разгуле, так сказать, капитализма
без утверждения главного архитектора нельзя было не то чтобы построить дом,
даже вывеску установить запрещали!

Справедливости ради, надо отметить, что в Мариуполе есть еще люди, бережно
относящиеся к лицу города. Это Галина Ильинична Бургу, которая восстановила
первоначальный облик возглавляемого ею кинотеатра \enquote{Победа}. По инициативе Веры
Николаевны Черемных был расчищен от многолетних наслоений фасад редакции газеты
\enquote{Приазовский рабочий}, и вновь окрашен в благородный бирюзовый цвет. Также
большую работу провели владельцы домов № 35 и № 37 на проспекте Мира, чтобы
вернуть им тот внешний вид, который они имели в начале ХХ века. Есть, к
счастью, и некоторые другие примеры тактичного отношения к старым строениям.

\textbf{Читайте также:} 

\href{https://mrpl.city/news/view/ischezayushhuyu-arhitekturu-mariupolya-sobirayut-po-krupitsam-na-pamyatfoto-plusvideo}{%
Исчезающую архитектуру Мариуполя собирают по крупицам на память, mrpl.city, 15.10.2018}
