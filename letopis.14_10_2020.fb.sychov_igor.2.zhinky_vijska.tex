% vim: keymap=russian-jcukenwin
%%beginhead 
 
%%file 14_10_2020.fb.sychov_igor.2.zhinky_vijska
%%parent 14_10_2020
 
%%url https://www.facebook.com/groups/167555693733355/permalink/921043488384568/
 
%%author 
%%author_id 
%%author_url 
 
%%tags 
%%title 
 
%%endhead 

\subsection{Жінкам - військовослужбовцям}
\Purl{https://www.facebook.com/groups/167555693733355/permalink/921043488384568/}

\ifcmt
  author_begin
   author_id sychov_igor
  author_end
\fi

\begin{multicols}{2}
	\obeycr
Неважливо, грішні ви чи праведні. Як палає бій
Мужньо ви рятуєте поранених попри страх та біль.
Бачив сам – з передової лінії ведете вогонь.
Зрозумів – від докорів сумління не врятує бронь.

Бо страшніш можливості загинути почуття вини
Перед вами – квіти з поля битви, ластівки війни.
Стали ви у час журби й випробувань до солдатських лав,
Довго у окопах я виборював рівність наших прав,

Рівність наших статей та обов’язків в полі й бліндажах.
І нехай під обстрілами боязко, був ще більший страх.
Лютий страх дивитись в ваші очі та лишитись сну
Знаючи – за спинами жіночими пережив війну.

Недостатньо сипати порадами там де бій тривав,
І кричати по тилах: «нас зрадили!» – то пусті слова.
Важко захищати рідну сторону від страшних атак.

Я – служив; тепер мені – не соромно. Тож скажу вам так:
Хай і не розвіялись примарами лихоліття дні,
Хоч біда висить густою хмарою, ви у цій війні
Переможете! Зарано вас хоронять! Тверда ваша стать!

Тільки жаль, що найміцніші «броніки» вас не захистять
Від лихого наклепу та ницості, й заздрісних думок…
Ластівки війни! Святі та грішниці! Бережи вас Бог!

Автор – Ігор Сичов.
	\restorecr
\end{multicols}

\ifcmt
pic https://scontent.fiev6-1.fna.fbcdn.net/v/t1.0-9/121373932_995944327591185_3161585080950265352_o.jpg?_nc_cat=109&ccb=2&_nc_sid=825194&_nc_ohc=HMeoYoxdQV4AX_w-WYn&_nc_ht=scontent.fiev6-1.fna&oh=c56e59a97d6a2728565696f5bc0eddaa&oe=5FF4CB56
caption Художник - Беата Куркуль. На картині - ветеран АТО Дарина Рудик. Картина створена на основі записів з її щоденника.
\fi
