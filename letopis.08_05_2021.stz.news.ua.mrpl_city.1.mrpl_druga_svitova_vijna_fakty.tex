% vim: keymap=russian-jcukenwin
%%beginhead 
 
%%file 08_05_2021.stz.news.ua.mrpl_city.1.mrpl_druga_svitova_vijna_fakty
%%parent 08_05_2021
 
%%url https://mrpl.city/blogs/view/mariupol-u-drugij-svitovij-vijni-vrazhayuchi-fakti
 
%%author_id demidko_olga.mariupol,news.ua.mrpl_city
%%date 
 
%%tags 
%%title Маріуполь у Другій світовій війні: вражаючі факти
 
%%endhead 
 
\subsection{Маріуполь у Другій світовій війні: вражаючі факти}
\label{sec:08_05_2021.stz.news.ua.mrpl_city.1.mrpl_druga_svitova_vijna_fakty}
 
\Purl{https://mrpl.city/blogs/view/mariupol-u-drugij-svitovij-vijni-vrazhayuchi-fakti}
\ifcmt
 author_begin
   author_id demidko_olga.mariupol,news.ua.mrpl_city
 author_end
\fi

\ii{08_05_2021.stz.news.ua.mrpl_city.1.mrpl_druga_svitova_vijna_fakty.pic.1}

8 та 9 травня Україна, разом з усім світом, відзначає День пам'яті та
примирення, День перемоги над нацизмом у Другій світовій війні та 76-ту річницю
вигнання нацистів з України. У Другій світовій війні загинув кожен п'ятий
українець. Ті страшні події міцно закарбувалися в пам'яті поколінь  Роль і
значення Маріуполя в роки Другої світової війни, його звільнення від німецьких
окупантів досліджувало багато істориків, музейних працівників та краєзнавців.
Зокрема, цікавими стали дослідження професора, доктора історичних наук
Володимира Романцова, історикині-краєзнавиці Валентини Зінов'євої, краєзнавців
Вадима Джуваги та Олексія Попова і музейного працівника Антона Гузя. 

Хочеться розповісти про декілька цікавих фактів щодо історії Маріуполя в роки
Другої світової війни, які варті того, щоб про них дізнатися і запам'ятати. 

У 2012 році вийшов фільм, присвячений футбольному \enquote{матчу смерті} між київськими
футболістами і збірною зенітників-люфт\hyp{}вафе в окупованому Києві влітку 1942
року. Цікаво, що і в Маріуполі був свій \enquote{матч смерті}, який відбувся 31 жовтня
1941 року в Маріуполі між солдатами танкових частин вермахту, що захопили
місто, і маріупольськими футболістами. Крім розваги, матч ще вважав своєю метою
показати \enquote{лояльність нового порядку} до мешканців міста, і звичайно,
спритність, силу і перевагу \enquote{істинних арійців} над місцевими.

Маріупольський \enquote{Матч смерті} проходив на заводському стадіоні в Іллічівському
районі (тепер пустир в районі вулиці та провулку стадіону). Містяни поспішно
зібрали команду здебільшого з тих \enquote{іллічівців}, хто мав заводську \enquote{бронь} від
призову на фронт, але не встиг евакуюватися з виробництвом на Урал. Склад,
зрозуміло, був вкрай ослабленим.

Німці, за спогадами маріупольських гравців, грали дуже грубо, навіть жорстоко –
показували свою вседозволеність. Як проходило суддівство в Маріуполі, не
вдалося дізнатися, але можна припустити. Приблизно в ті ж дні грали \enquote{Матч
смерті} в Києві, і суддя цього матчу чіплявся до кожного руху київських
\enquote{динамівців}.

\ii{08_05_2021.stz.news.ua.mrpl_city.1.mrpl_druga_svitova_vijna_fakty.pic.2}

Цікаво, що деякі маріупольські учасники \enquote{матчу смерті} благополучно пережили
війну, і згодом ще грали в післявоєнних командах. Але, правда і те, що
лояльності до переможців нацисти не виявляли, а навпаки, \enquote{брали їх на замітку}.
До чого це часто призводило, пояснювати не потрібно.

У маріупольській команді того дня за перемогу боролися футболісти Поповиченко,
Судаков, Каракаш, Машкін, Шаповаленко, Ковальов, Чуриков, Зінченко та інші
містяни, прізвища яких залишилися невідомі. Маріупольський \enquote{Матч смерті},
незважаючи ні на що, закінчився з рахунком 3: 1 на користь місцевих жителів.

Вражає, що за 701 день нові \enquote{господарі} так і не змогли налагодити роботу
металургійних заводів, випуск боєприпасів, і не отримали ні грама
азовстальського металу. Цьому завадили 29 підпільних груп і сотні патріотів
Маріуполя. Найбільш розгалуженою та організованою була підпільна група під
керівництвом В. І. Шипіцина. Ця група діяла у порту, на судноремонтному заводі
та залізничній станції. Але у зв'язку з масовими доносами багато членів
підпільної групи були заарештовані. 24 липня 1943 р. членів підпільної групи на
чолі з її керівником В. І. Шипіциним розстріляли біля селища Агробаза.

Також існували й інші підпільні групи, керівниками яких були: Є. М. Штанько
(діяла на заводі \enquote{Азовсталь}), основними діями були диверсії, але за доносом
членів підпільно-патріотичної групи Є. М. Штанько арештували та 24 липня 1943 р.
розстріляли. Неабияку мужність та сміливість проявила і Єлизавета Пилипівна
Бірюкова, яка перетворила рибний комбінат Маріуполя на табір спасіння.

Особливе місце в боротьбі проти німецьких загарбників займають патріотичні
групи медичних працівників. Коли з таборів військовополонених направляли на
лікування в міськлікарні, медпрацівники встановлювали їм \enquote{страшні діагнози} і
навіть робили помилкові медоперації. А потім ховали поранених бійців і
забезпечували їх фіктивними документами. Групою лікарів тільки Іллічівської
лікарні, яку очолив Б. Т. Гнилицький за такими довідками було переправлено через
лінію фронту більше 200 осіб. Окрім дорослих до боротьби з окупантами
приєднувалась і молодь міста. Зокрема, відомою була підпільна група на чолі з
Д. Ломизовим, яка розповсюджувала листівки, збирала речі для військовополонених
та закликала населення до боротьби з нацистами. За доносами зрадників  багатьох
членів групи було заарештовано та розстріляно 4 вересня 1943 р. 

Цікаво, що за статистикою, кожен другий танк, задіяний на фронті, був
виготовлений з металу, який був прокачаний на іллічівському стані \enquote{4500}. У
своїх мемуарах німецькі воєначальники визнавали Т-34 найкращим танком у світі.

\ii{insert.read_also.demidko.elizaveta_birjukova}

Для німців Маріуполь завжди мав важливе стратегічне значення. Звідси вони
підтримували зв'язок з Кримом, тут отримували військові вантажі, які прибувають
з портів Румунії та Болгарії. Відступаючи під натиском наших військ, німецьке
командування наказало зруйнувати Маріуполь і спалити. Чотири дні горіли
пожарища, на яких згоріли дві тисячі мирних жителів. 10 вересня 1943
воїни-визволителі нашого міста побачили руїни і попіл. На мітингу 11 вересня
маріупольці пообіцяли відновити місто.

На початку 1944 року в Маріуполі широко розгорнувся рух зі збору коштів до
Фонду оборони Батьківщини. Ініціатором виступив трудовий колектив рибкомбінату,
який вирішив зібрати кошти на будівництво підводного човна. Незабаром про
патріотизм рибалок Маріуполя дізналася вся країна. Всього маріупольці внесли до
Фонду 7 млн 58 тис рублів, з використанням яких були виготовлені і створені
танкові колони – \enquote{Відповідь іллічівців}, \enquote{Маріуполь мстить}.
Вносили гроші і на створення літаків ескадрильї \enquote{Звільнений Донбас}.

Історичні документи засвідчують інформацію про те, що Маріупольський російський
драматичний театр силами колективу акторів провів 20 безкоштовних шефських
концертів, з них 14 – у шпиталях для поранених бійців та офіцерів. Концерти в
шпиталях були прерогативою культурного шефства працівників мистецтва над
пораненими бійцями Червоної Армії. Актори здійснювали і матеріальні пожертви
для перемоги над ворогом. Художній керівник театру, заслужений артист РРФСР А.
Ходирєв особисто вніс у фонд оборони країни 25 тис. крб. Загалом колектив
театру зробив благодійний внесок у розмірі 34500 крб. у фонд допомоги дітям
фронтовиків. 17 лютого 1945 р. Маріупольський державний драматичний театр
отримав листа, в якому І. Сталін особисто висловлював подяку співробітникам
театру за збір 5000 крб. у фонд допомоги сім'ям фронтовиків і дітям-сиротам.

Місто понесло великі матеріальні, людські втрати, але завдяки мужності та
героїзму простих людей, як і вся країна, Маріуполь вистояв та зробив свій
великий внесок у перемогу над німецькими загарбниками.

\ii{insert.read_also.demidko.zhinoche_oblycchja_vijny}
