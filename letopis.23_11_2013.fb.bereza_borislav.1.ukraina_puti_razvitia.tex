% vim: keymap=russian-jcukenwin
%%beginhead 
 
%%file 23_11_2013.fb.bereza_borislav.1.ukraina_puti_razvitia
%%parent 23_11_2013
 
%%url https://www.facebook.com/borislav.bereza/posts/764900010202774
 
%%author_id bereza_borislav
%%date 
 
%%tags es,evropa,geopolitika,kreml,rossia,ukraina,zapad
%%title Украина - пути развития
 
%%endhead 
 
\subsection{Украина - пути развития}
\label{sec:23_11_2013.fb.bereza_borislav.1.ukraina_puti_razvitia}
 
\Purl{https://www.facebook.com/borislav.bereza/posts/764900010202774}
\ifcmt
 author_begin
   author_id bereza_borislav
 author_end
\fi

Крайне тяжело выжить в современном мире, без сильного партнера, такой стране,
как Украина. Потенциально таких партнеров сейчас два. Россия и Европа.
Присоединение к одному и к другому, с точки зрения краткосрочных перспектив,
несет абсолютно разные последствия. При вляпывании в ТС мы получаем
относительно дешевый газ и временно сохраняем рынки сбыта нашей продукции. При
ассоциации с ЕС ничего практически не получаем, кроме проблем с восточным
соседом и  началом телодвижений в сторону членства ЕС. Повторяю, это на
начальном этапе. Затем будет не все так оптимистично. Как это "дружить" с
Кремлем хорошо знают уже в Белоруссии. Да и казахи не в восторге. Зато Большой
Пу и Кремль довольны. Сохранены зоны влияния, идут подвижки к реинкарнации
Империи и очередной щелчок Западу. Ну а о дополнительных возможностях для
экспансии российской олигархии можно и не говорить. Это по умолчанию. 

Ассоциация с ЕС на начальном этапе может дать гораздо меньше. Она может даже
ничего и не дать, кроме возобновления сотрудничества с МВФ и получением
относительно дешевых денег. Но! Это возможность выхода из-под влияния Кремля.
Это возможность обозначить начала прощания с совковым менталитетом. Это первый
этап начала возникновения государственности и возможность приобщения к
европейским ценностям. Это не легкий путь. Это не путь в РАЙ, и не панацея от
тяжелой жизни, но это путь в нормальное общество. Это как возможность переезда
из трущоб в новостройку. Нужно будет и вещи перевезти и ремонт сделать и с
соседями перезнакомится, найти новую работу и свое место в жизни . И надо будет
найти деньги на все эти телодвижения. Но какой, нормальный, человек откажется
от такой возможности? И не только для своих детей, но и для себя?

Именно поэтому я за ассоциацию с Европой. Именно поэтому я считаю, что мы,
Украина, не имеем право упустить этот шанс. И именно поэтому я надеюсь, до
последнего, что у нас все получится! Я исхожу из долгосрочных перспектив.
