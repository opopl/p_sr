% vim: keymap=russian-jcukenwin
%%beginhead 
 
%%file 20_10_2020.news.ru.lenta.1.sovok_nebesnii
%%parent 20_10_2020
 
%%endhead 

\subsection{Совок небесный}

Россияне мечтают возродить СССР. Кому выгоден миф о коммунистическом рае?

«Нерушимый» Советский Союз распался почти три десятилетия назад. Россияне
хорошо помнят времена закрытых границ, пустых полок и дефицитной мебели из
стран Восточного блока. Но, несмотря на весь негатив, жители страны добрым
словом вспоминают советскую эпоху и хотят в нее вернуться. Это подтверждают
социологические исследования. Кто и почему ностальгирует по рухнувшей стране?
Кто заставил россиян поверить в миф о величии Советского Союза и полюбить
Совок? А главное — кому выгодна легенда об ушедшей прекрасной и сытной жизни? В
рамках спецпроекта «Мифы о России» «Лента.ру» отвечает на эти вопросы вместе с
историками и социологами.

\subsubsection{«Фантастический райский мир»}

Российские социологи отмечают странный парадокс: о желании вернуться в недавнее
прошлое все чаще говорят совсем молодые россияне — им еще не исполнилось 30
лет, они родились уже в новой России, выросли при Путине, привыкли к гаджетам,
«Тиктоку» и «Тиндеру», а жить хотят в эпоху Советского Союза.

Социологи констатируют, что молодые люди оказались крайне восприимчивы к мифу
об идеальной стране. Им греют душу рассказы о счастливом детстве, самом вкусном
мороженом по 13 копеек, пионерских лагерях, магазине «Балатон», старых
«Волгах», пляжах Крыма, кавказских курортах и народной дружбе, а главное —
социальных гарантиях: дешевой еде, стабильной зарплате, достойной пенсии и
бесплатных квартирах.

\ifcmt
pic https://icdn.lenta.ru/images/2020/10/02/18/20201002182416360/preview_10f5c78e2142e905395d0af567162752.png
\fi

То, что «развитой социализм», помимо всех вышеперечисленных благ, имел и
серьезные негативные стороны, из национальной памяти уже как-то стерлось.

Как, например, необходимость ни свет ни заря ехать на другой конец города,
потому что там продают молоко из бочки, или бежать сломя голову в магазин, где
выкинули (выпустили в продажу) обычные женские колготки. Наконец, знаменитая
загадка советских времен: «Длинная, зеленая, пахнет колбасой». Вряд ли многие
сегодняшние молодые люди с ходу скажут, что правильный ответ — «электричка из
Москвы». Ведь часто жители регионов действительно ездили в столицу за такими
обычными и общедоступными сейчас продуктами, как колбасные изделия, кофе и
апельсины из Марокко.

Как подчеркивает историк Николай Сванидзе, современный молодой человек, который
говорит о том, что хотел бы жить во времена Советского Союза, попросту не
отдает себе отчета в том, с какими бытовыми трудностями, кажущимися сейчас
невероятными, приходилось тогда сталкиваться большинству советских граждан.

\ifcmt
pic https://icdn.lenta.ru/images/2020/10/01/14/20201001143313275/preview_3472c111a9f2c3de6187a0e7b0ec9d66.png
\fi

«Это как — нет в магазине мяса? Значит, в соседнем есть. Как это — нет десяти
сортов колбасы или сыра? Значит, в соседнем есть. Поэтому когда им говорят, что
было все поровну, была справедливость, они хотят этот фантастический райский
мир. Вот и все», — объясняет историк.

До умов современной молодежи информация о минусах «социалистического образа
жизни» попросту не доходит.

Это подтверждает и социолог Денис Волков, неоднократно убеждавшийся в этом при
работе с фокус-группами.

Молодые люди говорят, как хорошо было жить при Брежневе. А когда им возражаешь
— отвечают, что им мама рассказывала

Денис Волков социолог
