% vim: keymap=russian-jcukenwin
%%beginhead 
 
%%file 01_12_2021.fb.fb_group.story_kiev_ua.1.familnaja_lozhka.cmt
%%parent 01_12_2021.fb.fb_group.story_kiev_ua.1.familnaja_lozhka
 
%%url 
 
%%author_id 
%%date 
 
%%tags 
%%title 
 
%%endhead 
\zzSecCmt

\begin{itemize} % {
\iusr{Оксана Дубинина}

Георгий, ничего себе! Каждый абзац Вашей публикации тянет на отдельный роман о
судьбе человека. Очень интересно, а еще чудо, как Вы сохраняете историю своего
рода. Спасибо!


\iusr{Георгий Майоренко}

Благодарю. Ещё очень важно встретить добрых людей на пути, которые подскажут и
помогут в поиске. И некоторые из этих добрых людей в группе \enquote{КИ}. Здесь я даже
родственников нашел. Так-что отдельная благодарность создателям группы и
модераторам. И с первым днём зимы!

\iusr{Татьяна Гурьева}

Блеск) как интересно читать о таких людях, поднимающих империю. Вот именно род
укрепляет человека, многое было завязано на этом. Славная история)

\begin{itemize} % {
\iusr{Георгий Майоренко}
\textbf{Татьяна Гурьева} 

Спасибо. Сейчас читаю воспоминания, переписку некоторых из тех, кого упомянул.
И таки да, интеллектуалы были высшей пробы! Причем, без айфонов, интернета.
Сейчас это сложно даже представить!

\iusr{Татьяна Гурьева}
\textbf{Георгий Майоренко} это называется порода)

\begin{itemize} % {
\iusr{Георгий Майоренко}
\textbf{Татьяна Гурьева} И колоссальный труд. Ведь многие аристократы жизнь \enquote{прожигали}. И такое было.
\end{itemize} % }

\iusr{Татьяна Гурьева}
\textbf{Георгий Майоренко} да всяко было. Главное, что в сухом остатке)

\iusr{Кретов Андрей}

Спасибо огромное за такой интереснейший материал!!!

О В. Шульгине знал, читал его мемуары и книгу, написанную после его
насильственного возвращения в Союз. Знал и что его племянник был в Раде, а вот
остальное для меня откровение. О богатейших Оболонских знал, а вот о
Жлобах... Спасибо!

\begin{itemize} % {
\iusr{Георгий Майоренко}
\textbf{Кретов Андрей} Да, слово жлобы звучит грубо. Но из песни слов не выкинешь! ))

\iusr{Rimma Turovskaya}
\textbf{Георгий Майоренко} Может быть, это сейчас звучит грубо, а в те времена это слово имело совсем другой смысл.

\iusr{Георгий Майоренко}
\textbf{Rimma Turovskaya} 

Есть версия, что в польском варианте это звучало не Жлобы - Погорельские, а
Злобы - Погоржецкие. Так что, возможно, эти люди были либо прижимистыми, либо
злобными. Хотя иногда употребляют слово жлоб по отношению к здоровяку - амбалу.

Выходили из избы
здоровенные жлобы...

В. Высоцкий

\iusr{Rimma Turovskaya}
\textbf{Георгий Майоренко} Да, вариантов много.

\iusr{Rimma Turovskaya}
\textbf{Георгий Майоренко} 

Георгий, я хотела спросить, неужели все эти материалы о своих предках Вы нашли
в архивах? Наверно, это потому, что среди них много выдающихся личностей?

\iusr{Георгий Майоренко}
\textbf{Rimma Turovskaya} 

Дело в том, что мне достался по наследству неплохой домашний архив. Да, так
совпало, что некоторые родственники были людьми незаурядными. Но эту тему я
начал подробно исследовать два года назад. До этого - все не было времени, не
доходили руки.


\iusr{Георгий Майоренко}
\textbf{Rimma Turovskaya} 

А с архивами (Киева и не только) сотрудничаю достаточно активно. А ещё в сборе
материала помогают многие замечательные люди. В том числе и из группы КИ. Под
Новый год подниму за них бокал.

\end{itemize} % }

\end{itemize} % }

\iusr{Natasha Levitskaya}

Георгий! Вам браво и восхищение! Какой огромный труд найти вот такую историю
своего рода! Интереснейшая история!

Спасибо вам за ваши публикации, за такие настоящие Киевские истории о людях,
которые эту историю делали. Такие публикации украшение нашей группы! @igg{fbicon.hands.applause.yellow} 

\begin{itemize} % {
\iusr{Георгий Майоренко}
\textbf{Natasha Levitskaya} Благодарю за высокую оценку. Ищем, стараемся. Находим - делимся. С первым зимним днём!

\iusr{Natasha Levitskaya}
\textbf{Георгий Майоренко}
Спасибо, взаимно! @igg{fbicon.snowflake}

\iusr{Светлана Манилова}
\textbf{Георгий}, после Наташи уже добавить нечего. Спасибо!

\iusr{Георгий Майоренко}
\textbf{Светлана Манилова} Весьма признателен, Светлана! Удачи и добра!
\end{itemize} % }

\iusr{Вадим Перегуда}

Дуже вдячний за публікацію, ви приклад для наслідування. Цікаво, що мій власний
рід пов'язаний також з Погорельськими, тільки Перегуда - Погорельськими гербу
Сас.

\begin{itemize} % {
\iusr{Георгий Майоренко}
\textbf{Вадим Перегуда} Спасибо! Очень интересно! А где ваши Перегуды - Погорельские проживали? Жлобы (Бублики) - Погорельские с Черниговщины.

\iusr{Георгий Майоренко}
\textbf{Вадим Перегуда} Вадим, возможно Погорельские где-то в прошлом и пересекались? Вот схема наших.

\ifcmt
  ig https://scontent-mxp2-1.xx.fbcdn.net/v/t39.30808-6/263155342_2785142971784926_6980182949990922893_n.jpg?_nc_cat=102&ccb=1-5&_nc_sid=dbeb18&_nc_ohc=OcS7JqWGYhUAX-COsip&_nc_ht=scontent-mxp2-1.xx&oh=00_AT-br_kfiOibYeyfjyw6weqiWSXz3bZPuZeXb4FFp5wWEA&oe=61DA07DD
  @width 0.6
\fi

\end{itemize} % }

\iusr{Наталья Писная}

Интерес к историям семьи возникает с нашим взрослением, когда задумываешься, а
что после нас? Но не каждому дано открыть тайны прошедших лет и столетий..

Вам удалось ! Спасибо!

\begin{itemize} % {
\iusr{Георгий Майоренко}
\textbf{Наталья Писная} 

Благодарю. Думаю, никогда не поздно заняться поисками. Просто нужно время и
знание источников. Хотя и в домашних фотальбомах можно найти ключи от многих
тайн прошлого.

\iusr{Наталья Писная}
\textbf{Георгий Майоренко} 

Спасибо за подсказку. Мои корни ( бабушка и дед) жили в Киеве, учились и
работали, но годы репресий не прошли мимо - семью рассорили, все потерялись на
долгие годы... Где только не оказались родные... А мне вот суждено было вернуться
в Киев! И очень этому рада.

\iusr{Георгий Майоренко}
\textbf{Наталья Писная} 

Надо выяснить, где предки жили до революции. И тогда уже выяснять детали.
Хорошо, если они обитали в одном районе и были прихожанами одной церкви. Тогда
в церковных книгах о них вся информация. Тогда церковные книги заменяли архивы
ЗАГСов и программу \enquote{Дiя}.

\iusr{Наталья Писная}
\textbf{Георгий Майоренко} Спасибо.
\end{itemize} % }

\iusr{Ірина Кравець}
Потрясающе!!! Как впечатляет. Вот уж история. А как написано, живо, легко...

\iusr{Георгий Майоренко}
\textbf{Ірина Кравець} Спасибо на добром слове. Удачи и добра!

\iusr{Лора Лора}

Браво, Георгий! Это огромный труд - по крупицам собирать историю своего рода!

\begin{itemize} % {
\iusr{Георгий Майоренко}
\textbf{Лора Лора} 

Спасибо, дорогая сестричка! Ты ведь тоже к этой линии имеешь некоторое
отношение! Кстати, всем нашим привет от тети Нины из Питера. Там сейчас
отмечается 30-летие степ-театра Винниченко. Вот это уже твоя близкая родня.
Удачи и здоровья!


\iusr{Alyona Artobalevska}
\textbf{Георгий Майоренко} я прочитала с огромным удовольствием, как буд-то бы побывала там, в той эпохе! Спасибо за увлекательные открытия нашего родового древа!

\iusr{Георгий Майоренко}
\textbf{Alyona Artobalevska} О, родня подтянулась! Всех наших обнимаю!
\end{itemize} % }

\iusr{Ольга Писанко}
Ооочень интересно! Особенно о Василии Шульгине!

\begin{itemize} % {
\iusr{Георгий Майоренко}
\textbf{Ольга Писанко} 

О Василии Шульгине - интереснейшая тема! Но она требует отдельного рассказа. У
него ведь и отец был известнейшим человеком в Киеве, и отчим...

\end{itemize} % }

\iusr{Раиса Карчевская}

Георгий!

Спасибо большое за Ваши потрясающе интересные воспоминания

\begin{itemize} % {
\iusr{Георгий Майоренко}
\textbf{Раиса Карчевская} Благодарю, Раиса. Успехов и благополучия!
\end{itemize} % }

\iusr{роман чудновский}
Знатно @igg{fbicon.hands.applause.yellow}  @igg{fbicon.sun.with.face}  @igg{fbicon.smile} 

\iusr{Георгий Майоренко}
\textbf{роман чудновский} Та да @igg{fbicon.face.tears.of.joy} 

\iusr{Svitlana Plieshch}

Спасибо вам за историю. Так много событий, что можно книгу писать. Вдохновляете
на изучение истории своего рода.

\begin{itemize} % {
\iusr{Георгий Майоренко}
\textbf{Svitlana Plieshch} 

Благодарю. Жаль, что мало документов передается от поколения к поколению. Вот
потом и ломай голову, что это были за люди? Но иногда удается по крупицам
собрать материал.

\iusr{Svitlana Plieshch}
\textbf{Георгий Майоренко} ощущение было такое, что вы о них знаете почти все.

\iusr{Георгий Майоренко}
\textbf{Svitlana Plieshch} 

О некоторых - немало. Но белых пятен - уйма! Понятно, что нельзя объять
необъятное. Стараюсь ориентироваться на минимум. Главное, что есть на свете
добрые люди, которые могут помочь полезными советами.

\iusr{Svitlana Plieshch}
\textbf{Георгий Майоренко} удачи в поисках информации, вдохновения и новых, интересных историй. Ждем
\end{itemize} % }

\iusr{Марина Соболевская}

Пишете замечательно! Биографии - таким легким и увлекательным слогом!
Изумительно.)

\begin{itemize} % {
\iusr{Георгий Майоренко}
\textbf{Марина Соболевская} 

Спасибо, Марина за добрые слова! Это краткое изложение. У некоторых личностей
из сюжета интереснейшие судьбы, рассказ о них потянет на целую книгу. А о
некоторых известна самая малость...

\begin{itemize} % {
\iusr{Марина Соболевская}
\textbf{Георгий Майоренко}, 

написать родословные книги увлекательно- непростая задача.)) У самой такая
цель. Но еще не уложились мысли - как совместить цифры, биографии с семейными
историями. У вас это чудесно получилось.)

\iusr{Георгий Майоренко}
\textbf{Марина Соболевская} 

Цель благородная! Удачи! Если будут вопросы в техническом плане - обращайтесь.
Мне самому помогают люди, которые в этих вопросах - \enquote{собаку съели}!

\iusr{Марина Соболевская}
\textbf{Георгий Майоренко} , спасибо!) Напишу вам в личку.)
\end{itemize} % }

\end{itemize} % }

\iusr{Inna Maistrouk}
Василид на чьей стороне был ?

\begin{itemize} % {
\iusr{Георгий Майоренко}
\textbf{Inna Maistrouk} На стороне Скоропадского. Белая гвардия.

\iusr{Елена Абдаллах Матющенко}
\textbf{Inna Maistrouk} Думаю, на стороне своих убеждений. Сильные и смелые люди!
\end{itemize} % }

\iusr{Леся Фандралюк}
С нетерпением жду продолжения!)

\iusr{Oleg Berezhinskiy}

В. В. Шульгін був затятий нацист, монархіст, антисиміт та українофоб. Його
\enquote{киянином} мабуть, надихався штрейхер при створенні свого \enquote{штюрмера}...

\begin{itemize} % {
\iusr{Георгий Майоренко}
\textbf{Oleg Berezhinskiy} Очень резкая характеристика. Даже его племянник Александр Шульгин (один из ярких представителей Центральной Рады) с ним полемизировал, но до грубости не опускался.
\end{itemize} % }

\iusr{Bogdan Mishchenko}

Только не \enquote{КИЕВЛЯНИНЪ}, а \enquote{КІЕВЛЯНИНЪ}... Потому что
\enquote{І} идёт перед гласными \enquote{Е}, \enquote{Я} ... \enquote{ЕВГЕНІЙ},
\enquote{ЕВГЕНІЯ} - как ещё один пример...

\begin{itemize} % {
\iusr{Георгий Майоренко}
\textbf{Bogdan Mishchenko} Огромное спасибо. Исправлю.
\end{itemize} % }

\iusr{Надежда Владимир Федько}
Дуже цікава історія!

\iusr{Георгий Майоренко}
\textbf{Надежда Владимир Федько} Спасибо, Надежда Владимир! Добра и удачи!

\iusr{Наталия Варавина}
Ваше исследование просто замечательное! Вот так, \enquote{изнутри}, узнаешь историю

\begin{itemize} % {
\iusr{Георгий Майоренко}
\textbf{Наталия Варавина} 

Благодарю. Это я вкратце написал. В интернет-варианте. А в биографиях этих
людей были такие сюжетные повороты... Может, когда-то напишу подробнее. Удачи
вам и добра!

\iusr{Арт Юрковская}
\textbf{Георгий Майоренко} Напиши! Можно книгу целую уже составить. Например под названием \enquote{Судьба одной киевской семьи}.

\iusr{Наталия Варавина}
\textbf{Георгий Майоренко} а Вам вдохновения!
\end{itemize} % }

\iusr{Надежда Владимир Федько}

У 1961 році Бобков був заступником начальника 2-го Головного управління КДБ при
РМ СРСР (до 1967-го). Заступником Андропова Бобков став у лютому 1982 року.

У 1961 році Андропов був завідувачем відділу по зв'язках із соціалістичними
країнами та комуністичними партіями цих країн (з 1957-го), з 1962 року —
секретар ЦК КПРС. В КДБ він прийшов у 1967-му.

\begin{itemize} % {
\iusr{Георгий Майоренко}
\textbf{Надежда Владимир Федько} Кстати, Бобков родом с Кировоградщины (Елисоветградский уезд). А с городом Елисоветградом у семьи Шульгиных многое связано.

\iusr{Надежда Владимир Федько}
\textbf{Георгий Майоренко} А Ви у фондах ДАКО працювали?

\iusr{Георгий Майоренко}
Ой, там шикарные фонды!!!
\end{itemize} % }

\iusr{Zinaida Mamontova}

Спасибо, пожалуйста, напишите книгу о своей семье, которая жила чаяниями и
волнениями своего времени. Это так интересно.

\begin{itemize} % {
\iusr{Георгий Майоренко}
\textbf{Zinaida Mamontova} Благодарю, как только сумею скомпоновать и обработать материалы, то, думаю, и книжечка получится. Удачи вам и благополучия!
\end{itemize} % }

\iusr{Татьяна Кец}
Спасибо. Очень интересно! Просто как выдержки из исторического романа.

\begin{itemize} % {
\iusr{Георгий Майоренко}
\textbf{Татьяна Кец} 

Благодарю за добрые слова. Так жизнь порой и есть исторический роман. Опять же,
это я рассказал в двух словах. А по этим линиям было ещё множество интересных
сюжетов.


\iusr{Татьяна Кец}
\textbf{Георгий Майоренко} с удовольствием познакомимся со следующими сюжетами. Ждём продолжения Ваших исследований.
\end{itemize} % }

\iusr{Вадим Горбов}

Киевлянинъ Шульгин заставил государя императора отречься от престола. И
возглавлял антисоветское подполье после премии чекистами и самоубийства Бориса
Савинкова. И интересные книги писал.

\begin{itemize} % {
\iusr{Георгий Майоренко}
\textbf{Вадим Горбов} Я думаю, кто-то напишет о В.В. Шульгине подробнее. Я не потяну. Слишком масштабная историческая фигура Василий Витальевич.

\iusr{Вадим Горбов}
\textbf{Георгий Майоренко} Да уж. Вес бурный ХХ век в его биографии. Титан.
\end{itemize} % }

\iusr{Арт Юрковская}

Спасибо! Огромное удовольствие тебя читать. Ты уже набрал материал для книги. В
публикациях приходится многое опускать. Жалко такой материал пускать только на
фейсбучные посты. Это все реальные истории и повороты - головокружительные - и
они есть история нашей страны - большой и такой разнообразной. Семья с такими
яркими биографиями!

\begin{itemize} % {
\iusr{Георгий Майоренко}
\textbf{Арт Юрковская} 

Благодарю. Ну, видите, в старое время благородные семьи искали для своих пары
из своего круга. Потому и такие интересные линии. Хотя, допустим, тетя Мотя,
что до революции жила в том доме, где нынче вы обитаете, была из простых. Зато
мужа нашла из высшего света и переселилась на Львовскую 14.

\begin{itemize} % {
\iusr{Арт Юрковская}
\textbf{Георгий Майоренко} Мы вроде на \enquote{ты} были.

\iusr{Георгий Майоренко}
Просто пост о благородных людях и я \enquote{автоматом} перешёл на вы.
\end{itemize} % }

\end{itemize} % }

\iusr{Irena Visochan}

Уже несколько раз сегодня перечитала Ваш пост. Биография всей семьи
восхищает! Надеюсь, что будет продолжение.@igg{fbicon.heart.suit}

\begin{itemize} % {
\iusr{Георгий Майоренко}
\textbf{Irena Visochan} Благодарю. По поводу продолжения, время покажет.
\end{itemize} % }

\iusr{Slawa Rotmann}

Здравствуйте. Огромное удовольствие читать Вас. Скажите, а о семье Слинько Вы
не знаете подробно? Наш дедуля Анатолий Платонович Дубенко был племянником
Слинько. Мы к сожалению, не знаем кто именно был дядей, так как дедуля умер 40
лет назад. С уважением Слава.

\begin{itemize} % {
\iusr{Георгий Майоренко}
\textbf{Irena Visochan} Благодарю. По поводу продолжения, время покажет.
\end{itemize} % }

\iusr{Slawa Rotmann}

Здравствуйте. Огромное удовольствие читать Вас. Скажите, а о семье Слинько Вы
не знаете подробно? Наш дедуля Анатолий Платонович Дубенко был племянником
Слинько. Мы к сожалению, не знаем кто именно был дядей, так как дедуля умер 40
лет назад. С уважением Слава.

\begin{itemize} % {
\iusr{Георгий Майоренко}
\textbf{Slawa Rotmann} 

Мне самому интересна эта линия. Насколько мне известно, тот Слинько, что жил до
революции, имел не только торговлю, но ещё и доходные дома. Конечно же, я
постараюсь выяснить подробности.

\begin{itemize} % {
\iusr{Slawa Rotmann}
\textbf{Георгий Майоренко} 

Спасибо Вам. Родственники со стороны бабушки были немецкого происхождения из
Кулдиги. Там мы довольно хорошо расследовали. Их звали Агафоновы Едельман. Они
занимались производством уксуса. Знаем где жили и чем занимались. Всю жизнь
думали, что ещё до революции уехали в Германию, но потом все выяснили, что нет.
Это очень интересная тема - читать документы 100-130 летней давности и узнавать
о своих корнях.

\iusr{Георгий Майоренко}
\textbf{Slawa Rotmann} Слава, вот схема родословной Жлобов - Бубликов. Иван Степанович Слинько в самом низу.

\ifcmt
  ig https://scontent-mxp2-1.xx.fbcdn.net/v/t39.30808-6/260829467_2785139215118635_8198946949329063532_n.jpg?_nc_cat=103&ccb=1-5&_nc_sid=dbeb18&_nc_ohc=9bkC1TKtz6UAX8nu2yy&_nc_ht=scontent-mxp2-1.xx&oh=00_AT_jcOu2ex3qFdlrpZmz0PcS7jUeEMC7F8Ufet0m1BUT2w&oe=61DAB920
  @width 0.4
\fi

\end{itemize} % }

\iusr{Георгий Майоренко}

Такие знатные люди обычно фигурировали в справочниках, размещали рекламу. Была
прекрасная серия дореволюционных справочников \enquote{Весь Киевъ}. Там списки жителей
города, владельцев домов, усадеб. Как только выдастся свободное время,
обязательно постараюсь ещё раз просмотреть. Там уникальнейшая информация. В
свою очередь, если вам удастся что-то выяснить по этой линии, то обязательно
сообщите.

\end{itemize} % }

\iusr{Maryna Chemerys}

Дуже цікаво, навть дивовижно" Дякую за розповідь! На Лук"янівському бачила
могилу Володимира Шульгіна

\begin{itemize} % {
\iusr{Георгий Майоренко}
\textbf{Maryna Chemerys} Он там похоронен со своим другом Владимиром Наумовичем.

\iusr{Maryna Chemerys}
\textbf{Георгий Майоренко} Так
\end{itemize} % }

\iusr{Татьяна Иванова}
Спасибо.

\iusr{Георгий Майоренко}
\textbf{Татьяна Иванова} Благодарю.

\iusr{Владимир Дубровский}
Надо же, фамилия Жлоб, а что Бублик лучше?

\iusr{Георгий Майоренко}
\textbf{Владимир Дубровский} Ну, жлоб это что-то прижимистое - скупердяйское. А бублик - сьедобное.

\iusr{Блюминов Алексей}

Василий Шульгин не только в тюрьме побывал в СССР. Он потом из нее вышел, и жил
вполне обычной жизнью советского пенсионера - даже пенсию получал- вместе с
женой во Владимире вплоть до 1976 год, давал интервью советской прессе и даже
снялся в главной роли в ищвестном докфильме Перед судом истории где признал
историческую правоту красных.

\begin{itemize} % {
\iusr{Георгий Майоренко}
\textbf{Блюминов Алексей} Да. Василий Шульгин - личность уникальная!

\iusr{Елена Абдаллах Матющенко}
\textbf{Блюминов Алексей} Был умным и человеком. И честным патриотом. Не ряженым
\end{itemize} % }

\iusr{Владимир Новицкий}
Спасибо, очень интересно. Мы тоже храним реликвии моих предков дворян Новицких!!

\begin{itemize} % {
\iusr{Георгий Майоренко}
\textbf{Владимир Новицкий} С удовольствием читаю ваши сюжеты о родне и старом Киеве! Наверняка ваши предки где-то с нашими пересекались.

\iusr{Георгий Майоренко}
\textbf{Владимир Новицкий} А Новицкие были приписаны в Дворянство Киевской губернии? Судя по фамилии - польские корни?

\begin{itemize} % {
\iusr{Владимир Новицкий}
\textbf{Георгий Майоренко} Да, были и польские корни.

\iusr{Георгий Майоренко}
\textbf{Владимир Новицкий} Здорово! А Новицкие какой церкви были прихожанами? Вроде самая близкая к вам это Златоустовская (на Евбазе). Там мою бабушку крестили в 1909 году.
\end{itemize} % }

\iusr{Владимир Новицкий}
\textbf{Георгий Майоренко} Да, Вы правы!!

\end{itemize} % }

\end{itemize} % }
