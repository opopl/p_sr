% vim: keymap=russian-jcukenwin
%%beginhead 
 
%%file 18_12_2017.stz.news.ua.mrpl_city.1.den_svjatogo_mykolaja
%%parent 18_12_2017
 
%%url https://mrpl.city/blogs/view/den-svyatogo-mikolaya
 
%%author_id demidko_olga.mariupol,news.ua.mrpl_city
%%date 
 
%%tags 
%%title День Святого Миколая
 
%%endhead 
 
\subsection{День Святого Миколая}
\label{sec:18_12_2017.stz.news.ua.mrpl_city.1.den_svjatogo_mykolaja}
 
\Purl{https://mrpl.city/blogs/view/den-svyatogo-mikolaya}
\ifcmt
 author_begin
   author_id demidko_olga.mariupol,news.ua.mrpl_city
 author_end
\fi

Одне з найулюбленіших свят, про яке завжди залишаються найсвітліші спогади –
День святого Миколая. Його відзначають у всьому світі, правда відрізняються
традиції, але діти точно знають, що в ніч з 18 на 19 грудня Святий Миколай
приносить подарунок і залишає його під подушкою.

Завдяки місцевій періодиці можна зробити припущення, що в XIX – на початку XX
ст. в День святого Миколая маріупольці вітали тих, хто особливо потребував
уваги або опинився за межею бідності. Суспільство допомоги бідним міста
Маріуполя (під головуванням Д. Хараджаєва) займалося збором коштів для допомоги
нужденним. У його правління міг прийти кожен містянин і залишити пожертвування.

Водночас у День святого Миколая маріупольці вітали своїх дітлахів і поспішали
придбати подарунки: дитячі іграшки, дитячі освітні ігри, гравюри, картини,
книги. Актуальним подарунком був календар на наступний рік. Відділення дитячих
іграшок було відкрито спеціально напередодні свят. Його господар, пан
Гольденштейн, запевняв містян в самих помірних цінах.

На початку XX ст. напередодні Дня святого Миколая особливо популярними були
різні зимові розваги. Наприклад, у грудні 1910 р. у Маріуполі працювала льодова
ковзанка на озері Домаха біля біржі, яка була захищена від холоду високим
парканом. Тут можна було взяти ковзани напрокат. Плата для дорослих становила
15 копійок, для дітей – 10 копійок.

Маріупольці до зимових свят завжди готувалися заздалегідь. На жаль не
збереглося відомостей, коли в місті була встановлена перша публічна ялинка.
Однак у XIX ст. напередодні Дня святого Миколая маріупольці відправлялися до
магазинів по вул. Георгіївській, на розі вул. Торгової та Італійської, щоб
придбати ялинку. Вартість хвойної красуні була від 20 копійок. Ялинку в той час
прикрашали яблуками, мандаринами, сухими фруктами, квітами, стрічками,
позолоченими горіхами, а також гірляндами, паперовими іграшками, бенгальськими
вогнями, хлопавками, пряниковими фігурками, восковими свічками та цукерками.
Наприкінці ХІХ століття у продажу з'явилися новорічні іграшки німецького і
російського виробництва. Вінчала різдвяну красуню Віфлеємська зірка.

Маріупольцям в різних торгових точках пропонувався великий асортимент ялинкових
прикрас. Вони були досить дорогими. Одним з найбільш рекламованих у місцевій
газеті був магазин П. Чебаненка, там ціна набору ялинкових іграшок становила 1
рубль, 1,5 рубля, 2, 3, 5 рублів і дорожче. Від 1 до 25 рублів коштувала
колекція ялинкових прикрас у \enquote{Харківській кондитерській і булочній}. Там же
продавалися ялинки, \enquote{святкові сюрпризи}, іграшки новітніх винаходів і
маскарадні маски.

Невідомо, чи писали маленькі маріупольці листи Святому Миколаю у XIX і XX ст. і
як кожна сім’я відзначала це чарівне свято раніше. Сьогодні до Дня святого
Миколая проводять благодійні акції, відкриваються фабрики і резиденції Святого
Миколая.

Вражає, що святий Миколай має різні імена: Міколай, Міклаш, Йолупукі, Сейнт
Ніколаус, Сантаклос, Фадер Крісмас, Пер Ноель, Сінтер Клаас. На відміну від
більшості святих, яких впізнаємо за повір’ями чи легендами, цей був історичною
постаттю. Народився він у 270 році нашої ери в Патарі, найбагатшому місті
півострова Лікії, що у Середземному морі. Батьки хлопчика – Феофан і Нона, жили
заможно та глибоко вірили в Бога, відтак із перших днів виховували своє маля у
християнському дусі. Принаймні, одразу його охрестили, а це тоді було рідкістю.
Відтоді Миколай ріс із глибоким переконанням, що найстрашнішим у людському
житті є вчинити гріх та стати на сторону зла, зійшовши зі стежини світлого та
істинного.

Проте ледь досягнувши повноліття, юнак втратив своїх маму й тата, а спадок,
який вони йому залишили, згодом віддав нужденним. Зокрема, вдівцеві, який жив
поблизу нього. Той мав єдину доньку, коханий якої був із багатої сім’ї.
Зрозуміло, що родина хлопця і чути не хотіла про безприданницю, яка згодом
прийде до них у дім. Тому Миколай вирішив віддати своє золото тим, хто
страждає. І в один з вечорів він непомітно поклав мішечок із монетами у вікно
спальні майбутньої нареченої. Тоді містом ходили чутки, що закоханим начебто
допоміг ангел. І, почувши такі розповіді, Миколай вперше відчув себе щасливим.

Однак на цьому він не зупинився і у будинки знедолених час від часу приносив
теплі речі, іграшки та їжу. За легендами Миколай врятував місто Світи від
голоду, допоміг оклеветанним воєначальникам уникнути смерті, а те, скільки він
допоміг бідним як на суші, так і на морі і порахувати складно.  І все ж таки
незважаючи на те, що Миколай намагався все тримати в таємниці, якось юнакові
таки не вдалося втаїти своїх добрих діянь. Мешканці вистежили Миколая і
здивувалися, що цей сором'язливий хлопець має настільки щире серце. І за такі
милосердні справи згодом містяни обрали його своїм єпископом. А у 325 році
Миколай відвідав Перший Вселенський Собор у Нікеї.

В Україну культ Миколая-чудотворця прийшов разом із християнством. Про це
свідчать церкви та храми, які збудували на честь цього святого. Зокрема, в XI
столітті такий звели на Аскольдовій горі у Києві, а через два століття – у
Львові, під горою Будельниця, тепер – Княжою.

Напередодні свята діти пишуть до нього листи зі своїми побажаннями і вкидають
їх у поштову скриньку або кладуть за вікно і моляться до нього, просячи
передусім здоров'я собі та батькам. У день перед святом згадують всі свої добрі
і злі вчинки, зважують: чого більше. Чи буде подарунок, а чи, можливо, різка?
Бо чемні діточки обов'язково знайдуть під подушкою подарунок, а неслухняні –
прутик. Ця різочка є своєрідним попередженням дитині, що час задуматися над
своєю поведінкою і виправитися.

Під покров Миколая потрапляють не лише дітлахи, які у ніч на 19 грудня терпляче
чекають від святого дарунків. Угодник Бога ділиться своєю опікою також зі
студентами, мандрівниками, водіями, торговцями та моряками. До слова, останні
часто беруть образ чудотворця з собою на риболовлю. У народі ще кажуть, що
святий Миколай береже людей і від стихійного лиха, особливо на воді. Згадуємо
рядки найпопулярнішої пісні про нього: \enquote{...хто спішить у твої двори, того ти на
землі й морі все хорониш від напасти}.

\clearpage
