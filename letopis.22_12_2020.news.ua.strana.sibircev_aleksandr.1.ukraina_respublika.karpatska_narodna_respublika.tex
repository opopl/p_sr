% vim: keymap=russian-jcukenwin
%%beginhead 
 
%%file 22_12_2020.news.ua.strana.sibircev_aleksandr.1.ukraina_respublika.karpatska_narodna_respublika
%%parent 22_12_2020.news.ua.strana.sibircev_aleksandr.1.ukraina_respublika
 
%%url 
 
%%author 
%%author_id 
%%author_url 
 
%%tags 
%%title 
 
%%endhead 
\subsubsection{\enquote{Карпатская народная республика}}

За последние недели на Западной Украине прошло несколько собраний
самопровозглашенных советов. 

Самый громкий случай произошел в Карпатах, в селе Верхняя Рожанка
Львовской области, которое относится к Славской объединенной
территориальной громаде и по планам должно стать новым горнолыжным
курортом, соперником знаменитого "Буковеля".

Селяне, а это тысяча с лишним человек, давно были недовольны тем, что у
них нет своего сельсовета и что их "приписали" в Славское ОТГ.

Часть из них протестует против попыток власти отобрать у них земли под
постройку в следующем году горнолыжного курорта (проект львовского
предпринимателя, владельца сели автозаправок Виталия Антонова, на который
обещают потратить 500 млн долларов). Даже перекрывали дорогу в знак
протеста.

И вот теперь группа горцев решила создать свой орган самоуправления и не
признавать решений официальной власти - собрались еще 7 декабря в одной из
хат и провозгласили собственную "сильраду", "сельским головой" выбрали
местную активистку Богдану Шкелебей (она этим летом активно выступала
против передачи земель под курорт) и даже свой флаг утвердили -
перевёрнутый национальный - желто-синий, а не сине-желтый. По их мнению
это и есть настоящий украинский стяг. Выбрали и "депутатов".

"Уже несколько лет мы добиваемся создания своего сельсовета, а нас все
футболят. Мы не хотим, чтобы нами управляли из Славского, а хотим быть
хозяевами на своей земле", - заявила "Стране" активистка Оксана Стасив,
которую выбрали "депутатом" нового "сельсовета".    

В руководстве Славской ОТГ заявили, что "не допустят повторения сценария,
который в свое время был реализован на юге и востоке Украины". А в
соцсетях уже в шутку пишут о создании "Карпатской народной республики"
(КНР). Другие иронизируют - мол, скоро в Карпатах введут визовый режим и
отдельный паспорт будут выдавать.

"Сценарий ДНР/ЛНР в Славской ОТО ждали? И мы нет. В то время как наши
парни и девушки воюют против сепаратизма на Донбассе, 15 жителей Верхней
Рожанки тут, на Львовщине, создали свой "сельсовет". Заодно они выбрали
себе "голову и депутатов". Первую сессию провели прямо дома, на кухне и
перевернули государственный флаг: сделали  его неконституционного цвета -
желто-голубым", - пишут в Фейсбуке на странице "Я люблю Славську ОТО". 

\ifcmt
pic https://strana.ua/img/forall/u/10/85/%D0%A1%D0%BD%D0%B8%D0%BC%D0%BE%D0%BA_%D1%8D%D0%BA%D1%80%D0%B0%D0%BD%D0%B0_2020-12-21_%D0%B2_22.30_.39_.png
\fi


