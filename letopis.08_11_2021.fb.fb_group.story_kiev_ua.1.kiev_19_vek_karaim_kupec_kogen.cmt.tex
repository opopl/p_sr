% vim: keymap=russian-jcukenwin
%%beginhead 
 
%%file 08_11_2021.fb.fb_group.story_kiev_ua.1.kiev_19_vek_karaim_kupec_kogen.cmt
%%parent 08_11_2021.fb.fb_group.story_kiev_ua.1.kiev_19_vek_karaim_kupec_kogen
 
%%url 
 
%%author_id 
%%date 
 
%%tags 
%%title 
 
%%endhead 
\subsubsection{Коментарі}

\begin{itemize} % {
\iusr{Lyudmila Edelweiss}
А я знаю про караимов - была на экскурсии в Крыму  @igg{fbicon.smile} 

\begin{itemize} % {
\iusr{Оксана Денисова}
\textbf{Lyudmila Edelweiss} В Евпатории очень красивая Кенасса!

\iusr{Lyudmila Edelweiss}
\textbf{Оксана Денисова} да, именно там, в Евпатории  @igg{fbicon.face.smiling.eyes.smiling} 
\end{itemize} % }

\iusr{Елена Мельникова}
Благодарю за интереснейшую информацию.

\begin{itemize} % {
\iusr{Оксана Денисова}
\textbf{Елена Мельникова} Спасибо Вам!
\end{itemize} % }

\iusr{Елена Сидоренко}

Очень интересно, спасибо. Была на Вашей экскурсии, Оксана, на улице Пушкинской и
всё видела своими глазами. @igg{fbicon.heart.sparkling}  @igg{fbicon.hands.applause.yellow} 

\begin{itemize} % {
\iusr{Оксана Денисова}
\textbf{Елена Сидоренко} Спасибо Вам, что ходите на мои экскурсии, «живьём» увидеть всегда интереснее @igg{fbicon.grin} 
\end{itemize} % }

\iusr{Andrei Kolinichenko}
Про караимов сегодняшних ни разу не слышал. Где они?

\begin{itemize} % {
\iusr{Natali Grebenik}
\textbf{Andrei Kolinichenko} в Крыму живут , только в Евпатории их около 2000 чел

\iusr{Оксана Денисова}
\textbf{Andrei Kolinichenko} Они есть, но их очень мало осталось.

\iusr{Ирина Алексеенко}

Живут караимы и в литовском городе Тракай. Несколько столетий назад один из
литовских князей вывез в Тракай большую группу караимов - искусных строителей и
ремесленников.

\iusr{Юлия Тертица}
Много караимов было в Литве, в Тракай и сейчас большая община

\iusr{Alexander Bronstein}
\textbf{Andrei Kolinichenko} 

Я общался с ккараимами в Литве, в Тракае. Там у них кенесса есть. Много веков
назад местный князь привёз их из Крыма в свою дружину.

\iusr{Вікторія Максимчук}

Приятно познакомиться! Во мне течет караимская кровь. Нас мало. В каждом
областном центре есть небольшое караимское общество. Центр караимской культуры
перенесен после аннексии Крыма в Мелитополь. В моем родном Николаеве караимов
человек 20-25 не больше.

Караимская кенасса в Киеве так и не возвращена караимскому обществу. В ней был
дом актера. Что сейчас - не знаю. Но такое чудесное здание должно быть
однозначно историческим памятником и не в частных руках.

\begin{itemize} % {
\iusr{Julia Ablamska}
\textbf{Вікторія Максимчук} в ней по-прежнему дом актёра.
\end{itemize} % }

\iusr{Alexey Paschenko}
\textbf{Andrei Kolinichenko} В Украине они компактно жили на Волыни, в Луцке


\ifcmt
  tab_begin cols=2,no_fig,width=0.2

     pic https://scontent-lga3-1.xx.fbcdn.net/v/t1.6435-9/255061607_5089103611099765_7920922044466814686_n.jpg?_nc_cat=104&ccb=1-5&_nc_sid=dbeb18&_nc_ohc=gShz-fUPlI8AX9ZEily&_nc_ht=scontent-lga3-1.xx&oh=a6d2b9ad611928b5b07c18404ecf7e12&oe=61AE887D

     pic https://scontent-lga3-1.xx.fbcdn.net/v/t1.6435-9/254795948_5089104114433048_776356214594163573_n.jpg?_nc_cat=105&ccb=1-5&_nc_sid=dbeb18&_nc_ohc=ksjJHBZC3iQAX8p49Yu&_nc_ht=scontent-lga3-1.xx&oh=37c8abd5eddc23080c6678e2afa3458d&oe=61AE8542

  tab_end
\fi

\end{itemize} % }

\iusr{Лилия Момотюк}

Спасибо! Вообще впервые слышу эту информацию к своему стыду

\begin{itemize} % {
\iusr{Оксана Денисова}
\textbf{Лилия Момотюк} Главное, что теперь Вы знаете!

\iusr{Лилия Момотюк}
\textbf{Оксана Денисова} это да. Благодарю Вас
\end{itemize} % }

\iusr{Nina Cygankova}
Спасибо, очень интересно, здание мне знакомо но истории его не знала, спасибо

\begin{itemize} % {
\iusr{Оксана Денисова}
\textbf{Nina Cygankova} Спасибо Вам!
\end{itemize} % }

\iusr{Всеволод Цымбал}

Всё это прекрасно, память для города. Но схемы ухода от налогов работали уже
тогда. Так называемая благотворительность

\begin{itemize} % {
\iusr{Оксана Денисова}
\textbf{Всеволод Цымбал} Вот не согласна с Вами! Не обижайте Киевских меценатов, они очень много для Киева делали. И они Вам не могут ответить!

\begin{itemize} % {
\iusr{Ирине Вильчинская}
\textbf{Оксана Денисова} 

Нынешним нашим согражданам, к сожалению, не всегда понятно различие таких
понятий, как современное спонсорство и тогдашнее -меценатство и
благотворительность...


\iusr{Всеволод Цымбал}
\textbf{Оксана Денисова} 

Я ни в коем случае не хотел обидеть людей, которые создавали историю нашего
Города. Но благотворительность испокон веку была деятельностью с двойным дном.
Нынешние бизнесмены используют опыт коллег из прошлых времён на 200 процентов.
Какие-то храмы аляповатые, которые только лишь портят городские виды, помогают
попавшим в беду людям на копейку, а отмывают на этом миллионы... Меценаты из
прошлого, конечно, тоже "химичили" в этих вопросах (налоговые службы во все
времена пытались успешных людей обобрать по максимуму), но после них
действительно остались архитектурные шедевры, которые украшают Город уже не
один век

\end{itemize} % }

\iusr{Олена Бородатова}
\textbf{Всеволод Цымбал} 

для киевских купцов, промышленников, учёных конца 19-начала 20 века
благотворительность не была "так называемой". Мы до сих пор с благодарностью
пользуемся профинансированными ими зданиями, учреждениями образования, культуры
и здравоохранения.

\begin{itemize} % {
\iusr{Лидия Гончарук}
\textbf{Олена Бородатова} А КПИ? Разве можно сравнивать с сегодняшним днем

\iusr{Всеволод Цымбал}
\textbf{Олена Бородатова} 

Вот только не надо придираться к словам. Купец в прошлые времена, и нынешний
бизнесмен в первую очередь думают о своём обогащении при минимальных затратах.
И меценатство никогда не было главным мотиватором. Рассказывать о
благотворительности красиво умели всегда. Послушаешь, почитаешь - да купцы и
фабриканты просто в убыток себе работали, чтобы только город украсить. Да
ничего подобного. Шедевры архитектуры, церкви, больницы и прочие деяния
сохранились, за это предкам спасибо. Но не всё так просто. Никто в убыток себе
никогда ничего не делал. В каждом акте меценатства на первом месте стоял личный
интерес. И в нынешние времена ничего не изменилось. Только лишь усугубилось,
причём уродливо

\iusr{Оксана Денисова}
\textbf{Всеволод Цымбал} 

В семье Терещенко было правило, которое было установлено первым из них- Николой
Терещенко, 70\% от прибыли отдавать на благотворительность. И никто из его
потомков ни разу это правило не нарушил. И где здесь расчёт ? В чем? Вы такое
сейчас можете представить?

\end{itemize} % }

\iusr{Rimma Turovskaya}
В то время благотворительность была еще и богоугодным делом.

\end{itemize} % }

\iusr{Nina NinaNina}
Реклама табака Когена в Киеве

\ifcmt
  ig https://scontent-lga3-1.xx.fbcdn.net/v/t1.6435-9/254934336_6331160823625920_4916227171599738599_n.jpg?_nc_cat=105&ccb=1-5&_nc_sid=dbeb18&_nc_ohc=nsyrRYv22OkAX_ont3o&_nc_ht=scontent-lga3-1.xx&oh=752406ff576d713afdc500df6f68eeef&oe=61AFAB98
  @width 0.3
\fi

\begin{itemize} % {
\iusr{Оксана Денисова}
\textbf{Nina NinaNina}

\ifcmt
  ig https://scontent-lga3-1.xx.fbcdn.net/v/t39.30808-6/254498595_10158786935792198_9073828387409100192_n.jpg?_nc_cat=101&ccb=1-5&_nc_sid=dbeb18&_nc_ohc=98r59e9IhhkAX8jDUR0&_nc_ht=scontent-lga3-1.xx&oh=b3fee1fdb75d012bc935da2ae2162bc8&oe=618FC2D4
  @width 0.3
\fi

\iusr{Nina NinaNina}
\textbf{Оксана Денисова}

\ifcmt
  ig https://scontent-lga3-1.xx.fbcdn.net/v/t1.6435-9/254553095_6331166640292005_4133227406899968140_n.jpg?_nc_cat=108&ccb=1-5&_nc_sid=dbeb18&_nc_ohc=qEHMZybNolUAX9xgcps&_nc_ht=scontent-lga3-1.xx&oh=ccde9bd0501b5457487e83bc48205747&oe=61B0A434
  @width 0.3
\fi

\end{itemize} % }
\iusr{Olena Babak}

А теперь дом актера..

\begin{itemize} % {
\iusr{Оксана Денисова}
\textbf{Olena Babak} Да, но уже без купола, к сожалению @igg{fbicon.face.unamused} 
\end{itemize} % }

\iusr{Журавель Павло}

Караимы любят говорить, что они тюрки, хотя в анамнезе семиты, как и евреи.
Фамилия "коген" говорит о носителе, что он из потомков коенов. Служителей
Иерусалимского Храма. Наличие коенских фамилий сообщает о еврейском
происхождении этноса.

\begin{itemize} % {
\iusr{Оксана Денисова}
\textbf{Журавель Павло} Спасибо Вам большое, очень интересное разъяснение!

\iusr{Вікторія Максимчук}
\textbf{Журавель Павло} Вы заблуждаетесь относительно происхождения караимов. Историческое место проживания, язык, обычаи, культовые строения - имеют тюркское, а не иудейское происхождение.

\begin{itemize} % {
\iusr{Журавель Павло}
\textbf{Вікторія Максимчук} 

Это долгий разговор. Да, последние сто лет они доказывают тюркское
происхождение. Началась эта история с последнего гахама С. Шапшала. Это он
"тюркизировал" караимов. Есть дореволюционная открытка на которой на
евпаторийской кенассе, на фасаде имеется шестиконечная звезда. В целом все
еврейские субэтносы похожи на народы среди которых они обитают: горские евреи
на кавказцев, сефарды на испанцев и тд.

\iusr{Журавель Павло}
\textbf{Вікторія Максимчук} Впрочем, если Вы мне приведёте серьёзное исследование на эту тему буду Вам благодарен.

\iusr{Edward Koren}
\textbf{Вікторія Максимчук} 

Караимы происходят от еврейской группы, которая в Египте в 10 веке откололась
от талмудического иудаизма ( раббанитов, по терминологии караимов). Всегда и
везде видели себя частью еврейского народа, по израильским законам имеют право
на репатриацию. Лишь в конце 19 века появилась "тюркская" теория, для того
чтобы антиеврейские законы Российской Империи обошли караимов стороной. Уже
упоминаемый Шапшал довел при совке эту теорию до маразма, придумав "священные
дубы" и прочую чушь. Караимы - часть еарейского народа. Хотя т они этого или
нет.

\iusr{Irena Shvartsburd}
\textbf{Журавель Павло} вы правы на 100\%

\end{itemize} % }

\iusr{Ksenia Sereda}
\textbf{Журавель Павло}, \textbf{Edward Koren} 

в Евпатории на экскурсии ещё в 2007-м году рассказывали, что они евреи. И
раньше знала. А сейчас удивилась, что кто-то этого не знает и более того
отрицается их происхождение в пользу тюркского.

\begin{itemize} % {
\iusr{Журавель Павло}
\textbf{Ksenia Sereda} Бытует "хазарская теория". Довольно популярна среди караимов и крымчаков.

\iusr{Ksenia Sereda}
\textbf{Журавель Павло}, интересно. Спасибо, что написали комментарий и возникла дискуссия. Всегда знала, что они "тюркские евреи". А теперь уже почитала чуток))

\iusr{Журавель Павло}
\textbf{Ksenia Sereda} Они потомки переднеазиатских караимов. Но некоторым нравятся романтические теории.

\iusr{Irena Shvartsburd}
\textbf{Журавель Павло} а в чем их отличие от крымчаков? Всегда считала что это один народ. Если вы знаете, буду благодарна за объяснение.

\iusr{Edward Koren}
\textbf{Irena Shvartsburd} 

Крымчаки - это просто евреи, но крымские, использовали в быту
еврейско-татарский язык, но в религии такие же,,как и все евреи, талмудические.
Караимы - отрицают Талмуд и признают только Тору. Но в быту использовали все
тот же еврейско-татарский язык, ныне мертвый. Крымчаков нацисты уничтожали,
караимов - нет.

\iusr{Irena Shvartsburd}
\textbf{Edward Koren} спасибо.

\iusr{Журавель Павло}
\textbf{Edward Koren} 

Уточню. Этнолект крымчаков нельзя назвать полноценным языком в отличие от идиша
или ладино. Это крымско-татарский с некоторым заимствованием из
древнееврейского. Крымчаки, судя по фамилиям (Ламброзо, Пиастро, Измерли,
Лехно, Варшавский и тд.), смешанная группа куда влились сефарды, восточные
евреи, ашкенази и тд. В течение нескольких веков они сформировались, как
отдельный субэтнос переняв татарский язык, одежду, кулинарию, образ жизни.

\end{itemize} % }

\end{itemize} % }

\iusr{Elena Tsirulnik}
Очень интересно, спасибо! Кенаса случайно была не там, где кинотеатр "Заря" когда то ?

\begin{itemize} % {
\iusr{Оксана Денисова}
\textbf{Elena Tsirulnik} Да, одно время там был кинотеатр «Заря», у меня даже фото где- то есть !

\begin{itemize} % {
\iusr{Elena Tsirulnik}
\textbf{Оксана Денисова} Просто я жила на Рейтерской в 70-ых годах, и всегда говорили что там ,где "Заря"-была синагога,а оказалось была кенаса.Бегали в этот кинотеатр повторных фильмов  @igg{fbicon.grin} 

\iusr{Оксана Денисова}
\textbf{Elena Tsirulnik} Хороший, мне рассказывали, был кинотеатр!

\iusr{Татьяна Петрачек}
\textbf{Оксана Денисова} Да,я там была не раз, очень красиво внутри и снаружи здание очень красивое,жалко,что сейчас никому нет дела,реставрировать бы его.

\iusr{Оксана Денисова}
\textbf{Татьяна Петрачек} Там и сейчас внутри красиво, но реставрировать обязательно надо!

\iusr{Elena Tsirulnik}
\textbf{Оксана Денисова} Да, было время @igg{fbicon.face.grinning.smiling.eyes} 
\end{itemize} % }

\iusr{Zoya Pshenichny}
\textbf{Elena Tsirulnik} Да ,Леночка, мой любимый и родной кинотеатр Заря.

\begin{itemize} % {
\iusr{Elena Tsirulnik}
\textbf{Zoya Pshenichny} Ага, бегали туда постоянно в кино. @igg{fbicon.face.zany} 
\end{itemize} % }

\end{itemize} % }

\iusr{Петр Кузьменко}
Очень интересный и великолепно изложенный материал. Благодарю!  @igg{fbicon.hands.applause.yellow} 

\begin{itemize} % {
\iusr{Оксана Денисова}
\textbf{Петр Кузьменко} Спасибо большое!

\iusr{Rozana Kruchinina}
\textbf{Петр Кузьменко} спасибо большое
\end{itemize} % }

\iusr{Раиса Карчевская}
Оксана!
Большое спасибо за очень интересный пост и фотографии

\begin{itemize} % {
\iusr{Оксана Денисова}
\textbf{Раиса Карчевская} Спасибо Вам!
\end{itemize} % }

\iusr{Владимир Ходоровский}

В этом здании одно время был Кукольный театр, а затем кинотеатр "Заря".Если
будете в Крыму (Бахчисарай) - посетите Чуфут-Кале! А если занесёт в Литву, то
познакомьтесь с Тракаем (под Вильнюсом) - столицей караимов. Эти места значительно
расширят познания о караимах!

Несмотря на иудейские корни, Гитлер не трогал караимов(В отличие от евреев).

P. S. Я уже соскучился за нашими встречами на Лиге экскурсоводов Киева(?!)

\begin{itemize} % {
\iusr{Оксана Денисова}
\textbf{Владимир Ходоровский} В Крыму была в Чуфут- Кале, а вот в Тракае нет, надо будет побывать! Спасибо!

\iusr{Roman Sytnyk}
\textbf{Владимир Ходоровский} За Галич забули.

\begin{itemize} % {
\iusr{Сергей Пацкин}
\textbf{Roman Sytnyk} Какой из Галчей, наш или уркаинской?

\iusr{Roman Sytnyk}
\textbf{Сергей Пацкин} Український. Там велика колонія караїмів була у 1920-30-х роках.

\iusr{Вікторія Максимчук}
\textbf{Roman Sytnyk} Насколько мне известно караимов в Галиче не осталось. Но есть музей караимской культуры.
\end{itemize} % }

\end{itemize} % }

\iusr{Aleksandr Mitryaev}

Мало кто знает, что во время наибольшего расцвета польско-литовского
государства, литовский князь Витовт, то примерно 1380 год, вывез из Чуфут-Кале
200 семей караимов и они поселились вблизи замка Тракай в Литве. До сих пор по
Вильнюсу и Каунасу ходят люди, имеющие паспорт Литвы, Но их лица так и не стали
славянскими, литовскими: они круглы и эти люди гордятся своим происхождением,
берегут свою идентичность ....

Во времена СССР, такие факты тщательно замалчивались, так как русские
приписывали себя к славянам, а не к финно-угорским племенам. Но поляки 380 лет
после татаро-монгольского нашествия владели киевскими землями, Киев был
окраинным городом Польско-литовской империи. На Щекавице в Киевестоял замок
польского Киселя и так далее.

Вплоть до того, что земли южнее Полесья заселялись украинскими потомками.
народа Киевской Руси.

Вплоть до фальсификации истории. Во время татаро-монгольского нашествия в 1240
году никакого перетока людей из киевских территорий на север не было, а в Киеве
во время наибольшего расцвета Киевской Руси имел всего 50 000 жителей.

Вплоть до того, что начало Москвы не 1147 год, а на 200 лет позже.

\begin{itemize} % {
\iusr{Оксана Денисова}
\textbf{Aleksandr Mitryaev} спасибо Вам за такой подробный экскурс в историю!

\iusr{Елена Сидоренко}
\textbf{Aleksandr Mitryaev} скажите, это караимов называли "литваки" или литовских евреев за акцент?

\iusr{Aleksandr Mitryaev}

Это, Елена Сидоренко, я не понимаю. Тупой. Караимы не литваки. Караимы, тюркский народ, а по религии одна из веток иудаизма. Не путайте евреев и их религией. В Киеве кеншаса караимов была одна (Кинотеатр Заря, повторного кино, Дом актера), Синагог было центральных три. Это при Советах Ляльковый театр, вторую вспомню, скажу. А третью переделали, надстроив ещё этаж в Дом автомобилиста, там был кинотеатр, Дом детского творчества ...
\end{itemize} % }

\iusr{Анна Сидоренко}

Спасибо вам большое, я не знала про купца и табачный завод, да ещё и на
Крещатике.

\begin{itemize} % {
\iusr{Оксана Денисова}
\textbf{Анна Сидоренко} Крещатик, 17 - это был адрес его фабрики

\iusr{Анна Сидоренко}
Спасибо Оксаночка , не люблю вилять хвостом, если я не знаю то я прямо и пишу об этом.

\begin{itemize} % {
\iusr{Оксана Денисова}
\textbf{Анна Сидоренко} Всего знать невозможно, тем интереснее узнавать что- то новое!!!

\iusr{Анна Сидоренко}
Согласна!
\end{itemize} % }

\end{itemize} % }

\iusr{Нина Гордийчук}
Спасибо за интересную информацию.

\iusr{Наташа Котляренко}

Караимы были верной гвардией литовского князя. И они жили в тракае. Кстати, во
времена СССР, когда в паспорте была графа национальность, в стране было около
100 человек караимов в Литве. Хотя Сталин их выслал в Сибирь.

\iusr{Марианна Смакова}
Огромное спасибо, осень интересно.

\begin{itemize} % {
\iusr{Оксана Денисова}
\textbf{Марианна Смакова} Спасибо Вам!
\end{itemize} % }

\iusr{Владимир Ходоровский}

Караимы, как и теперь швейцарцы, охраняющие папскую резиденцию в Ватикане, были
на службе в охране замка литовский королей!

\begin{itemize} % {
\iusr{Оксана Денисова}
\textbf{Владимир Ходоровский} Интересно!

\iusr{Юрий Панчук}
\textbf{Владимир Ходоровский} И Крымского хана

\iusr{Юрий Панчук}
\textbf{Владимир Ходоровский} Считались самыми неподкупными стражами

\iusr{Юрий Панчук}

Не знаю, писал ли кто то про табачку Когенов на бульваре Шевченко, а после
1985г в начале проспекта Победы. Ещё несколько лет назад она стояла.

\begin{itemize} % {
\iusr{Оксана Денисова}
\textbf{Юрий Панчук} Табачка попала в руки компании Реемтсма- Империал Тобакко, но потом фабрику закрыли. Даже не знаю , стоит ли здание.

\iusr{Татьяна Годынская}
\textbf{Юрий Панчук} нет там уже фабрики... @igg{fbicon.face.rolling.eyes} 

\iusr{Ирина Петрова}
\textbf{Оксана Денисова} нет, здание уже разрушено.
\end{itemize} % }

\iusr{Людмила Дзюбенко}
\textbf{Владимир Ходоровский} была в. Литве в 1987 году на экскурсии, возили на улицу, где проживают караимы. Их осталось очень мало.

\iusr{Аркадий Шухман}
\textbf{Владимир Ходоровский} В городе Тракай в Литве есть музей караимов, был там лет сорок назад. Очень интересно, надеюсь, что он существует и сегодня. Караимы охраняли в свое время Тракайский замок.

\end{itemize} % }

\iusr{Валентина Юшко}
Спасибо. Киевлянка, но не знала об этом факте. Побольше бы таких публикаций. Еще раз спасибо.

\begin{itemize} % {
\iusr{Оксана Денисова}
\textbf{Валентина Юшко} Спасибо Вам!
\end{itemize} % }

\iusr{Светлана Гаврилко}
Спасибо за интересную статью!!!!!

\begin{itemize} % {
\iusr{Оксана Денисова}
\textbf{Светлана Гаврилко} Спасибо Вам!
\end{itemize} % }

\iusr{Irena Visochan}
Спасибо, большое, я впервые слышу эту историю, очень интересно.

\begin{itemize} % {
\iusr{Оксана Денисова}
\textbf{Irena Visochan} Спасибо Вам!
\end{itemize} % }

\iusr{Liudmila Kabanova}

Кенаса прекрасна! Каждый день мимо неё прохожу. Там реставрация аж "плачет"!
Сверху распадаются фрагменты лепки.

\begin{itemize} % {
\iusr{Оксана Денисова}
\textbf{Liudmila Kabanova} Да, реставрировать ее очень нужно! Если бы ещё купол восстановить !

\iusr{Наталия Ковалева}
\textbf{Liudmila Kabanova} А рядом находился "Еврейский театр", несколько раз была на постановках,очень понравилось!
\end{itemize} % }

\iusr{Татьяна Васильевна Зубко Маркина}
В Крыму видела город караимов. Спасибо за интересный рассказ и всем за комментарии

\iusr{Галина Ливанская}
это здание сохранилось?

\begin{itemize} % {
\iusr{Оксана Денисова}
\textbf{Галина Ливанская} Кенасса? Конечно, сохранилась, Ярославов Вал 7. Она теперь только без купола.

\ifcmt
  ig https://scontent-lga3-2.xx.fbcdn.net/v/t39.30808-6/254710291_10158787029927198_6240391643539576128_n.jpg?_nc_cat=102&ccb=1-5&_nc_sid=dbeb18&_nc_ohc=t-UOz2RVkJQAX86ayjx&_nc_ht=scontent-lga3-2.xx&oh=bc63f1852461f4ff93b0b7210139264c&oe=618FAF89
  @width 0.4
\fi

\begin{itemize} % {
\iusr{Анна Слонина}
\textbf{Галина Ливанская} да сохранилось.

\iusr{Галина Ливанская}
\textbf{Анна Слонина} спс

\iusr{Аркадий Израилевский}
\textbf{Галина Ливанская} бывший кинотеатр "Зоря"
\end{itemize} % }

\end{itemize} % }

\iusr{Виктория Логинова}

Про караимов слышала, была в Бахчисарае и Чуфут-Кале. Но, что они были в Киеве
и даже построили кенасу (к сожалению, почему-то совсем её не помню), не знала.
Очень интересно. Спасибо огромное, Оксана!

\begin{itemize} % {
\iusr{Оксана Денисова}
\textbf{Виктория Логинова} Спасибо Вам!

\iusr{Mykhaylo Losytskyy}
Дім актора на Ярославовому Валу, близько до Золотих воріт.

\begin{itemize} % {
\iusr{Виктория Логинова}
\textbf{Mykhaylo Losytskyy} спасибо большое. Дом актёра знаю, но не поняла, что речь об этом здании. Очень красивое, хоть и требует реставрации.

\iusr{Sem Zilman}
\textbf{Mykhaylo Losytskyy} Yes! Я там бывал очень часто, так мой приятель был в Доме Актёра зам.директора...

\iusr{Валентина Осокина}
\textbf{Mykhaylo Losytskyy} Бывший кинотеатр Заря
\end{itemize} % }

\iusr{Ирина Ярмак}
\textbf{Виктория Логинова} есть в Литве город Тракай, недалеко от Вильнюса. Там живут караимы, которые были вывезены князем Витаутосом время его военного похода на Крым.

\begin{itemize} % {
\iusr{Виктория Логинова}
\textbf{Iryna Yarmak} спасибо. К сожалению, ещё там не была.

\iusr{Наталья Корниенко}
\textbf{Виктория Логинова Тракай} - это must see, волшебный мир, замок на острове! Два года назад поезки в Вильнюс были доступны
\end{itemize} % }

\iusr{Roman Sorokhtei}
\textbf{Виктория Логинова} Караїми з Криму на заклик і по запрошенню короля Данила збудували Галич!
\end{itemize} % }

\iusr{Наталия Ковалева}

У мужа бабушка, киевлянка, рассказывала, что девушек, продающих папиросы
называли "табакрошками", прямо провели экскурсию, еще раз пройдусь по этим
местам, очень интересно, спасибо!!! @igg{fbicon.maple.leaf}  @igg{fbicon.fallen.leaf}  @igg{fbicon.top.hat} 

\begin{itemize} % {
\iusr{Оксана Денисова}
\textbf{Наталия Ковалева} Рада, что Вам было интересно!
\end{itemize} % }

\iusr{Инна Иванова}
Очень красивое здание. И интересный рассказ. Спасибо.

\begin{itemize} % {
\iusr{Оксана Денисова}
\textbf{Инна Иванова} Спасибо Вам!
\end{itemize} % }

\iusr{Лисенко Олег}
В переписі 1913 Караїмських Купців Києва є дві сімї .
ДАКО 280 о174д спр 354 1913 рік

\ifcmt
  ig https://scontent-lga3-2.xx.fbcdn.net/v/t1.6435-9/254434687_5079241545461074_4185500126847090033_n.jpg?_nc_cat=103&ccb=1-5&_nc_sid=dbeb18&_nc_ohc=lUvvcAEwXBQAX-7d-g0&_nc_ht=scontent-lga3-2.xx&oh=753d455ec2129793bb37fb6b90990638&oe=61AF5B43
  @width 0.4
\fi

\iusr{Лисенко Олег}
Купці 2 гільдії

\ifcmt
  ig https://scontent-lga3-2.xx.fbcdn.net/v/t1.6435-9/255120800_5079245275460701_2883269256251764371_n.jpg?_nc_cat=109&ccb=1-5&_nc_sid=dbeb18&_nc_ohc=Sdncx-m_i_oAX887aDn&_nc_ht=scontent-lga3-2.xx&oh=ef73dfb14ed6919f2f74fe9f7cf1897e&oe=61B1C9D5
  @width 0.4
\fi

\begin{itemize} % {
\iusr{Оксана Денисова}
\textbf{Лисенко Олег} Ух ты, как интересно! Спасибо!
\end{itemize} % }

\iusr{Ирина Камнева}
 @igg{fbicon.face.happy.two.hands} Спасибо

\iusr{Лисенко Олег}
Один жив Хрещатик 29 другий брат Лютеранська 4

\iusr{Лисенко Олег}
А всього в 1913 жило 9 сімей купців Караїмів.

\iusr{Наталія Кузенкова}
Спасибо! Очень интересно! @igg{fbicon.face.smiling.eyes.smiling} 

\iusr{Александр Ермоленко}
Спасибо большое, дорогая Оксаночка!! Блеск!!! И нищета нынешних хозяев жизни!

\begin{itemize} % {
\iusr{Оксана Денисова}
\textbf{Александр Ермоленко} Спасибо Вам!

\iusr{Наталья Корниенко}
\textbf{Александр Ермоленко} СПАСИБО!
\end{itemize} % }

\iusr{Елена Хмельницкая}

Вот люди не только богатели, но и строили для города ! Благодарность и
уважение!

\begin{itemize} % {
\iusr{Оксана Денисова}
\textbf{Елена Хмельницкая} Это правда, другие были люди!

\iusr{Ирина Петрова}
\textbf{Elena Khmelnytska} строили культовые сооружения. И сейчас строят храмы .
\end{itemize} % }

\iusr{Iryna Dashkovska}

В Тракае пол Вильнюсом ( Литва ) есть целые улицы караимов , и дома у них в три
окна ...

\ifcmt
  ig https://scontent-lga3-1.xx.fbcdn.net/v/t39.30808-6/254738131_4596056687097481_8157668283457935296_n.jpg?_nc_cat=110&ccb=1-5&_nc_sid=dbeb18&_nc_ohc=ly9zPWQje54AX9BCh9e&_nc_ht=scontent-lga3-1.xx&oh=5fe7efb022b2eb5fc89552d7fa5f1390&oe=618ED446
  @width 0.3
\fi

\begin{itemize} % {
\iusr{Оксана Денисова}
\textbf{Iryna Dashkovska} Спасибо, как интересно!

\iusr{Юрий Руденко}
\textbf{Iryna Dashkovska} почему именно в три окна?

\iusr{Iryna Dashkovska}
\textbf{Юрий Руденко} такие у них традиции ... они что-то символизируют, Но я не помню точно что ...

\iusr{Александр Ермоленко}
Был в Тракае и все видел , более чем впечатляет!! Не говоря о природе и замке!!
\end{itemize} % }

\iusr{Iryna Dashkovska}

\ifcmt
  ig https://scontent-lga3-1.xx.fbcdn.net/v/t39.30808-6/253567935_4596057680430715_6860652113739840499_n.jpg?_nc_cat=102&ccb=1-5&_nc_sid=dbeb18&_nc_ohc=dWK8uswOXKcAX8Ytb7P&_nc_ht=scontent-lga3-1.xx&oh=329381152cd26d6a6398b36252fb0f04&oe=618F5172
  @width 0.3

	ig https://scontent-lga3-1.xx.fbcdn.net/v/t39.30808-6/255370745_4596058683763948_7998275388871206200_n.jpg?_nc_cat=109&ccb=1-5&_nc_sid=dbeb18&_nc_ohc=q-lSxsuhBCsAX_Dpdcm&_nc_ht=scontent-lga3-1.xx&oh=d1a30a7a6cf79afd3bacab5c2f77f649&oe=618F2A74
  @width 0.3
\fi

\begin{itemize} % {
\iusr{Оксана Денисова}
\textbf{Iryna Dashkovska} Какая красота!!!

\iusr{Людмила Дзюбенко}
\textbf{Iryna Dashkovska} дв, это типичный дом караимов в Трокае, всего одна улица односторонняя была...
\end{itemize} % }

\iusr{Iryna Dashkovska}
Тракай

\ifcmt
  ig https://scontent-lga3-1.xx.fbcdn.net/v/t39.30808-6/255137769_4596059593763857_8132180671922548910_n.jpg?_nc_cat=102&ccb=1-5&_nc_sid=dbeb18&_nc_ohc=MXMhfH1OVcoAX_cLUQe&_nc_ht=scontent-lga3-1.xx&oh=3be0978449f80faa441f9449159ca8a2&oe=618F5F1E
  @width 0.3
\fi

\begin{itemize} % {
\iusr{Светлана Куричко}
\textbf{Iryna Dashkovska} ,была там. Очень красиво! И в кафе ели блюда крымско- татарской кухни.

\iusr{Iryna Dashkovska}
\textbf{Светлана Куричко} караимской кухни . Я тоже была в Тракае осенью в 16-м году , зашли в караимский ресторан и дегустировались их национальные блюда и настойки ...Было очень вкусно!
\end{itemize} % }

\iusr{Надія Строй}
Дуже красива Кенасса в Євпаторії.
Так трапилось, що я була там 2012р.
Яка доля її на сьогодні???
Дякую пОксано за гарну
Оповідь.
Київська кенасса закрита?

\begin{itemize} % {
\iusr{Оксана Денисова}
\textbf{Надія Строй} В кенассе сейчас Театр киноактера, туда можно попасть . А в Евпатории кенасса очень красивая, я тоже там была!

\iusr{Надія Строй}
\textbf{Оксана Денисова} цей театр я люблю, часто там буваю.
Чомусь думала ,що кенасса поряд.. .

\iusr{Надія Строй}
Оксана Денисова це певно БУДИНОК актора.

\iusr{Оксана Денисова}
\textbf{Надія Строй} Да , я неправильно написала, дом актера.

\iusr{Viktoriya Horbach}
\textbf{Надія Строй} там будинок актора

\end{itemize} % }

\iusr{Игорь Кокарев}

Ох, неужто и вправду "в середине XIX века Киев был ещё тихим провинциальным
городом" из-за табака-самосада? И даже наличие в городе Университета и Лавры не
помогло)?

\begin{itemize} % {
\iusr{Andrey Smotritsky}
\textbf{Игорь Кокарев} не верите, но это так.

\iusr{Оксана Денисова}
\textbf{Игорь Кокарев} Да, это было так. Рост начался в 70-е годы 19 века, когда появились сахарные заводы и железная дорога, тогда произошёл скачок!

\iusr{Игорь Кокарев}
Придется поверить, раз вы так дружно
об этом заявляете  @igg{fbicon.smile} 

\iusr{Оксана Денисова}
\textbf{Игорь Кокарев} 1850 год, Козьеболотский переулок - нынешний переулок Шевченко в двух шагах от Майдана

\ifcmt
  ig https://scontent-lga3-1.xx.fbcdn.net/v/t39.30808-6/253310372_10158787108577198_6994792468067404044_n.jpg?_nc_cat=105&ccb=1-5&_nc_sid=dbeb18&_nc_ohc=oyGxtj4ZsYEAX82KKV4&_nc_ht=scontent-lga3-1.xx&oh=2e3aeab8291c856b2eceeae9802ee07e&oe=61905724
  @width 0.4
\fi

\end{itemize} % }

\iusr{Liudmila Baliasna}
Боагодарю . , Интерсно...

\iusr{Tiana Larina}
Да-да.. кенаса.. а потом кинотеатр Зоря) .. затем - Дом Актера.. - родные места

\iusr{Sem Zilman}
WOW@! This is VERY INTERSTING STORY!!

\begin{itemize} % {
\iusr{Оксана Денисова}
\textbf{Sem Zilman} Thanks!
\end{itemize} % }

\iusr{Tetiana Samoilovych}
Красивое здание построил за здоровье многих поколений местных жителей

\iusr{Игорь Мачулин}
А что счас в этом здании)?

\begin{itemize} % {
\iusr{Оксана Денисова}
\textbf{Игорь Мачулин} Дом Актера

\iusr{Игорь Мачулин}
\textbf{Оксана Денисова} Это продолжает традиции - любой храм в чом то и театр)!
\end{itemize} % }

\iusr{Наталия Слободян}
познавательно  @igg{fbicon.thumb.up.yellow} @igg{fbicon.exclamation.mark.double}

\iusr{Надежда Ухина}
Дякую, пані Оксано! Цікаво)

\begin{itemize} % {
\iusr{Оксана Денисова}
\textbf{Надежда Ухина} Дякую!
\end{itemize} % }

\iusr{Elena Dubovenko}
Дякую за цікаву розповідь @igg{fbicon.sparkles} 

\begin{itemize} % {
\iusr{Оксана Денисова}
\textbf{Elena Dubovenko} Дякую!
\end{itemize} % }

\iusr{Sasha Jampolsky}
Геній дзюдо та Лимонадний Джо все що пам'ятаю про Зорю

\iusr{Ирина Иванченко}

Любимая фраза из обожаемого фильма:,, Я думала, у вас шкворчить в грудях, а то
шкворчала ваша папироска..." Любимая кенасса Городецкого... Спасибо вам ,
Оксана, за милые сердцу встречи, как и всегда , информативно и тепло о нашем
Городе.

\begin{itemize} % {
\iusr{Оксана Денисова}
\textbf{Ирина Иванченко} Спасибо Вам большое!
\end{itemize} % }

\iusr{Тамара Ар}

Театр теперь в караимской кенассе. Была как то, не могу сказать, что спектакли
увлекательные по сравнению с репертуарами Театра на Подоле, или Театра драмы и
комедии на Левом берегу. А здание интересное.

\begin{itemize} % {
\iusr{Оксана Денисова}
\textbf{Тамара Ар} Согласна, театр так себе, но зато можно здание изнутри посмотреть @igg{fbicon.grin} 

\iusr{Тамара Ар}
\textbf{Оксана Денисова} здание интересное, как и история
\end{itemize} % }

\iusr{Elena Dubovenko}

Кенасса наверное от слова кнессет - дом. Предполагаю.  красивое здание, с
детства любовалась.

\begin{itemize} % {
\iusr{Оксана Денисова}
\textbf{Elena Dubovenko} Да , кнессет и кенасса от одного корня!

\iusr{Elena Dubovenko}
\textbf{Оксана Денисова} спасибо Вам большое, всегда с удовольствием читаю ваши посты в КИ!

\iusr{Оксана Денисова}
\textbf{Elena Dubovenko} Спасибо Вам большое!

\iusr{Lada Samoylov}
\textbf{Elena Dubovenko} кенасса, как и кнессет скорее всего от слова "книса" - вход.  @igg{fbicon.face.smiling.halo} 
\end{itemize} % }

\iusr{Надія Чуй}
Дякую!!! Кенасса...!

\iusr{Борис Лушельский}
Даже фашисты не трогали караимов.
Среди военначальников, караим дослужился до маршала Советского Союза.

\begin{itemize} % {
\iusr{Оксана Денисова}
\textbf{Борис Лушельский} Кто?

\iusr{Борис Лушельский}
\textbf{Оксана Денисова}
Радион Малиновский.

\iusr{Виктория Угрюмова}
\textbf{Оксана Денисова} 

самая известная история про караимов того периода - это истррия
офицера-караима, героя 1905 года, он положил один 16 солдат и офицеров японцев,
режде чем его зарубили. Они его палаш скопировали, он до сих пор хранится в
музее. Тело передали с почестями обратно. Как сказано не мной - и имя его
знает, разумеется, каждый школьник. Разумеется - каждый ЯПОНСКИЙ школьник.

\iusr{Оксана Денисова}
\textbf{Виктория Угрюмова} Да, я что-то такое читала, но не помню где! Я не японский школьник, мне стыдно  @igg{fbicon.face.unamused} 

\iusr{Andrew Ragev}
\textbf{Борис Лушельский} Радион Малиновский - Маршалл и министр обороны при Хрущёве @igg{fbicon.index.pointing.up}

\begin{itemize} % {
\iusr{Оксана Денисова}
\textbf{Andrew Ragev} Спасибо, не знал, что он был караим!

\iusr{Борис Лушельский}
\textbf{Andrew Ragev}
Да. Именно, Малиновский.

\iusr{Оксана Денисова}
\textbf{Борис Лушельский} Спасибо большое за интересную информацию!!!

\iusr{Татьяна Грицай}
\textbf{Andrew Ragev}, не знала, что Малиновский из караимов. Ходила информация, что он польских кровей.

\iusr{Борис Лушельский}
\textbf{Оксана Денисова}
Не за что, Оксана.

\iusr{Геннадий Дудко}
\textbf{Борис Лушельский} Извините, но откуда эта информация?

\iusr{Борис Лушельский}
\textbf{Геннадий Дудко}
Какая именно?

\iusr{Геннадий Дудко}
\textbf{Борис Лушельский} Что маршал Малиновский был караим?

\iusr{Геннадий Дудко}
\textbf{Татьяна Грицай} поляком был маршал К. К. Рокоссовский.

\iusr{Борис Лушельский}
\textbf{Геннадий Дудко}
Читал в его автобиографии.

\iusr{Геннадий Дудко}
\textbf{Борис Лушельский} Я так понимаю, что это была автобиография?

\iusr{Борис Лушельский}
\textbf{Геннадий Дудко}

Была книга под названием "Фельдмаршалы России и Маршалы Советского Союза." Там
я прочел. В официальной биографии Малиновского, как, собственно, и большинства
выдающихся людей, очень многое скрыто и не опубликовано. Так же об этом есть
сведения в Большой Военной энциклопедии.

\iusr{Геннадий Дудко}
\textbf{Борис Лушельский} Спасибо!

\iusr{Борис Лушельский}
\textbf{Геннадий Дудко}
Не за что.

\end{itemize} % }

\iusr{Борис Лушельский}

По большому счету, караимы - выходцы из иудейской среды. Скажем, течение.

У нас в Израиле есть национальность - друзы. Они - выходцы из Ливана. Они те же
арабы, но отошедшие от основных арабских канонов. Они, как и караимы, не имеют п
сменного Святого писания. Всё, что им нужно передать, они передают в устной форме
.

Кроме них, у нас есть течение Бахаев. Это - бывшие персы, выходцы из Ирана.
Благодаря своему основателю Бабе, они стали проповедовать религию, о личную от
трёх основных, объединив их вместе.

\begin{itemize} % {
\iusr{Мастер Массажа Алекс Городин}
\textbf{Борис Лушельский} а не выходцы ли друзы из Сирии? Ведь они живут на голанских высотах, где раньше Сирия занимала

\iusr{Борис Лушельский}
\textbf{Мастер Массажа Алекс Городин}
Нет, Саша. Они выходцы из Ливана. Просто,Ливан- бывшее Сирийское побережье. Они ,так же,есть и в Иране,и в Турции.

\iusr{Мастер Массажа Алекс Городин}
\textbf{Борис Лушельский} если Ливан бывшее сирийское побережье, то изначально из Сирии получается
\end{itemize} % }

\iusr{Борис Лушельский}
\textbf{Мастер Массажа Алекс Городин}
Как бы Ливант, был, но сирийцы всегда считали Ливан своей территорией.

\end{itemize} % }

\iusr{Александр Семенище}
Спасибо

\iusr{Svitlana Eden}
Есть у меня жестяная коробочка ФАБРИКА ТУЦКАГО ТАБАКУ СОЛОМОНА КОГЕНА.

\begin{itemize} % {
\iusr{Оксана Денисова}
\textbf{Svitlana Eden} Ой, а можете сфотографировать и показать , интересно!

\iusr{Svitlana Eden}
\textbf{Оксана Денисова} вот прямо сейчас попробую
\end{itemize} % }

\iusr{Натали Булах}
Спасибо за интересный рассказ!!

\begin{itemize} % {
\iusr{Оксана Денисова}
\textbf{Натали Булах} спасибо!
\end{itemize} % }

\iusr{Nik Bitkin}

Ша всем, любящих эту ветвь человеков. На папиросной фабрике работал 13 часов
рабочий, работница. Вы так сможете в табаке. Мой прадед тоже, но он кондуктор
трамвая... С 19-го века.

\iusr{Александр Длугач}

Вообще-то караимы, это пещерный народ, к которому пренадлежала Дева Мария...
Известен тем, что раввины спасли их от Гитлера не признав евреями, в отличии от
тех же крымчаков... Не признали их евреями не по причине непризнания ими
Талмуда, а по причине не признания ими устной Торы, лично я слабо в этом
разберусь, так что вполне может быть, что это одно и тоже! Опять таки, нужно
понимать, в паспортах Российской Империи, "национальность" еврей, значилась в
паспортах в графе вероисповедание...

\begin{itemize} % {
\iusr{Оксана Денисова}
\textbf{Александр Длугач} Это все очень запутано, я не сильна в этом  @igg{fbicon.grin} 

\iusr{Александр Длугач}
\textbf{Оксана Денисова} 

так ни кто не силён! В наше время о таком, как бы и говорить особо было не
принято! Вообще, судя по всему, караимы это Предтече Хазарского Каганата,
прямыми наследниками которого, согласно Конституции Пилипа Орлика, являются
"козаки" Войска Запорожского... @igg{fbicon.index.pointing.up} 
@igg{fbicon.monkey.see.no.evil}  @igg{fbicon.monkey.hear.no.evil}  @igg{fbicon.monkey.speak.no.evil} 

\iusr{Оксана Денисова}
\textbf{Александр Длугач} в общем, мы все караимы @igg{fbicon.grin} 

\iusr{Александр Длугач}
\textbf{Оксана Денисова} 

ну да, где то так... Есть два варианта, либо русины, либо, в зависимости от
признания или не признания "Устной Торы"!!! Осталось только определиться что
это такое и в чем собственно разница!!! @igg{fbicon.index.pointing.up} @igg{fbicon.face.wink.tongue}  @igg{fbicon.laugh.rolling.floor}{repeat=3}  @igg{fbicon.monkey.see.no.evil}  @igg{fbicon.monkey.hear.no.evil}  @igg{fbicon.monkey.speak.no.evil} 


\iusr{Александр Длугач}
\textbf{Оксана Денисова} 

прикол в том, что я не смеюсь!!! Дале фрагмент Конституции Пилипа Орлика с
официального сайта архива государства Украина! Написано на "староукраинском",
который идентичен "староруському" и "старорусинмкому"...

1710 р., квітня 5.

З-під Бендер. – Конституційні пакти Пилипа орлика, новообраного гетьмана
Війська Запорзького, з його присягою договоры и постаnовлεnѧ правъ и волностεй
войсковыхъ мεжи яснεвεл можнымъ εго милостю паномъ Филиппомъ орликомъ
новоизбраннымъ войска Zапорожского гεтманомъ, и мεжи εнεральними особами,
полковниками и тимъ жε войском Zапорожскимъ с полною з обоихъ сторонъ обрадою
утвεржεnnые и при волной εлεкції формалною присягою ωт того жъ яснεвεлможного
гεтмана потверженnые року ωт Рождεства Христова αψί, м[εся]ца априля дня ε. //
во имя тца и с[ы]на и с[вя]того д[у]ха, Б[о]га во тройцы с[вя]той славимого.
Nεхай станεтся на вѣкопомную войска Zапорожского и всεго народу малороссийского
славу и паматку. дивный и нεпостижимый в сүдбахъ своих Б[о]гъ, милосεрдный в
долготεрпѣнії, пр[а]вεдный в казни, яко всεгда от початку видимого сεго свѣта
на пр[а]вдномъ правосүдія свого мирилѣ, εдны панства и Nароды возвышаεтъ,
другіε за грѣхи и бεззаконія смираεтъ, εдны порабощаεт, другіε свобождаεтъ,
εдны возноситъ, другіε низвεргаεтъ, такъ и nародъ валεчный стародавный
козацкий, прεждε сεго имεнованый козарскій, пεршъ прεвознεслъ былъ славою
нεсмεртεлною, обширнымъ владѣніεмъ и ωтвагами рицεрскими, которими нε тылко
окрεстным nародомъ, лεчь и самому восточному панству на морѣ и на зεмли
страшεнъ былъ такъ далεцε, жε цεсар восточный, хотячи оный сεбѣ вѣчнε
примирити, сопряглъ малжεнскимъ союзомъ с[ы]нови своεму дочку кагана, то εст
князя козарского."

\iusr{Оксана Денисова}
\textbf{Александр Длугач} Что касается каганата, я Вам верю, я это читала! Осталось разобраться с Торой @igg{fbicon.grin} 

\iusr{Александр Длугач}
\textbf{Оксана Денисова} 

с этим сложности! В реальности есть только два закрытых для пришлых
вероисповедания, это иудаизм и униатство... Но то что караимов и каганат в
иудаизме не признавались и не признаются иудеями по причине проблем с принятием
именно "Устной Торы", это факт! А вот каким образом вероисповедание влияет на
"национальность", это вообще не объяснимо!!! @igg{fbicon.index.pointing.up}  @igg{fbicon.face.wink.tongue}  @igg{fbicon.face.happy.two.hands}  @igg{fbicon.face.smiling.halo} 

\end{itemize} % }	

\iusr{Владимир Дубровский}

Обманщики, просто не читали всю Тору, а то что им выгодно, жили в 9 -м веке
ленивые, учиться не хотели, вот и сделали отдельную нацию.

\begin{itemize} % {
\iusr{Николай Солонин}
\textbf{Владимир Дубровский} наборот, оказались умнее

\iusr{Борис Лушельский}
\textbf{Владимир Дубровский}
У нас в Израиле много таких наций. Например: самаритяне, ассирийцы. Они, до сих пор существуют, но в очень малых количествах.

\begin{itemize} % {
\iusr{Lora Berdichevsky}
\textbf{Борис Лушельский} 

Самаритяне проповедуют Тору, говорят на арабском и иврите. Единственная разница в
датах празднования. Все праздники у самаритян сдвинуты на месяц назад.

\iusr{Lora Berdichevsky}

Очень интересно. Спасибо за историю. Осмелюсь выдвинуть предположение по поводу
названия :"кенасса".На иврите существуют однокоренные слова
:Книса-вход, кенес-съезд, слет, кнессет-парламент, кнессия-церковь, бэйт
кнессет-синагога. Т. е. это здание где собираются люди для обсуждения светских
или религиозных дел.


\iusr{Борис Лушельский}
\textbf{Lora Berdichevsky}

На сколько мне известно, самаритяне- первые христиане. После них идут армяне.
Дальше, все остальные. На сегодняшний день, в Израиле, самаритян очень мало и, Вы
правы, они ,в основном, говорят на арабском.

\iusr{Lora Berdichevsky}
\textbf{Борис Лушельский} у нас в школе работает один. Есть небольшое поселение в Шхеме и здесь в Холоне

\iusr{Борис Лушельский}
\textbf{Lora Berdichevsky}
Ясно. Очень много их живёт в Афуле и Дженине. Я их видел в арабской деревень Бакка.
\end{itemize} % }

\iusr{Alexey Paschenko}
\textbf{Владимир Дубровский} 

Караимы - потомки тех евреев, которые выехали из иудеи задолго до изгнания.

Возле Евпатории есть мацевы (могильные надгробия), которым больше 2000 лет. То
есть там жили евреи, выехавшие до нашей эры. Вот потому они и признают только
Тору, а не признают Талмуд, который был создан позже.

\iusr{Ludmila Marchenko}
\textbf{Alexey Paschenko} А какова была у караимов причина покинуть Иудею?

\iusr{Ludmila Marchenko}
\textbf{Владимир Дубровский} 

Чтобы "выбрать выгодное", нужно как раз прочитать книгу от корки до корки. Вы
по жизни зануда или нездоровится?

\end{itemize} % }

\iusr{Вика Нурибекова}

Говооят, что с тех поо и пояаились в Киеве папиросницы, мундштуки и пепельницы.
У бабушки был янтарный мундштук и серебрянный портсигар с резинками внутри..

А часом наследство Когенов не досталось ли совместному украинско-немецкому
предприятию «Реемтсма-Киев табачная фабрика»?

\begin{itemize} % {
\iusr{Оксана Денисова}
\textbf{Вика Нурибекова} Да, их фабрика на бульваре Шевченко досталась Реемтсме-Империал Тобакко сейчас.
\end{itemize} % }

\iusr{Svitlana Eden}

\ifcmt
  ig https://scontent-lga3-2.xx.fbcdn.net/v/t39.30808-6/255340323_1046773516074706_8526889372586732609_n.jpg?_nc_cat=108&ccb=1-5&_nc_sid=dbeb18&_nc_ohc=bIE7AEDamKQAX9nvb3Z&_nc_ht=scontent-lga3-2.xx&oh=07cba62a96001290fdbfe0123d2ee83d&oe=618FAD73
  @width 0.4
\fi

\begin{itemize} % {
\iusr{Natalie Gora}

Коробочку продать не желаете! У меня в паспорте в СССР было написано «караимка»
по национальности моего папы. Есть много семейных историй, книг, словарей и
вырезок из печати про караимов, которые бережно собирали и хранили мои
родители. Теперь всё это храню я. Надеюсь мои дети этим когда-то
поинтересуются. Ваша коробка была бы чудесным приложением...а если кого-то в
Киеве интересует караимский вопрос, советую посетить караимский музей в Тракае.
Ресторан там тоже есть с прекрасной караимской кухней.

\end{itemize} % }

\iusr{Svitlana Eden}

\ifcmt
  ig https://scontent-lga3-2.xx.fbcdn.net/v/t39.30808-6/254745581_1046773549408036_5452161416218834364_n.jpg?_nc_cat=111&ccb=1-5&_nc_sid=dbeb18&_nc_ohc=Lq8JabPUcCUAX8tAjMx&_nc_ht=scontent-lga3-2.xx&oh=9b58f83e106998cfef984ecfc750e05d&oe=618F8ADC
  @width 0.3

	ig https://scontent-lga3-2.xx.fbcdn.net/v/t39.30808-6/253279817_1046773632741361_6274359545520901916_n.jpg?_nc_cat=103&ccb=1-5&_nc_sid=dbeb18&_nc_ohc=34DyI6zgJtIAX9DT-lh&_nc_ht=scontent-lga3-2.xx&oh=2d17e5f5b755b5138489e1abb7c35313&oe=618F4564
  @width 0.3

	ig https://scontent-lga3-2.xx.fbcdn.net/v/t39.30808-6/254552769_1046773699408021_1031158769464919198_n.jpg?_nc_cat=104&ccb=1-5&_nc_sid=dbeb18&_nc_ohc=Yf-FAmEAJuEAX_L4zWu&_nc_ht=scontent-lga3-2.xx&oh=a010dec9fb8155983d4ccfcb745a8256&oe=6190336F
  @width 0.3
\fi

\begin{itemize} % {
\iusr{Оксана Денисова}
\textbf{Svitlana Eden} Вы мне просто подарок сделали, спасибо Вам огромное @igg{fbicon.heart.red}{repeat=3}

\iusr{Svitlana Eden}
\textbf{Оксана Денисова} и мне было очень интересно прочитать Вашу публикацию  @igg{fbicon.smile} 

\iusr{Оксана Денисова}
\textbf{Svitlana Eden} Спасибо Вам@igg{fbicon.heart.red}{repeat=3}
\end{itemize} % }

\iusr{Svitlana Eden}
Храню старые фотографии  @igg{fbicon.smile} 

\ifcmt
  ig https://scontent-lga3-2.xx.fbcdn.net/v/t39.30808-6/253328516_1046773852741339_8147541875787448932_n.jpg?_nc_cat=111&ccb=1-5&_nc_sid=dbeb18&_nc_ohc=FEbGCdZ7JzkAX8GcHlV&_nc_ht=scontent-lga3-2.xx&oh=41818dcd007ef5496c56d8a16178795c&oe=6190BF4B
  @width 0.4
\fi

\iusr{Сергей Липавский}
Интересно, спасибо. Хотелось бы прочитать еще что-нибудь

\iusr{Аркадий Шухман}

В скором времени в Украине, как и во всех странах Восточной Европы и Прибалтики
будет принят закон о реституции, и так называемый " Дом Актера" перейдет к
своим законным хозяевам - караимской общине Украины. Надеюсь, справедливость
восторжествует.

\begin{itemize} % {
\iusr{Оксана Денисова}
\textbf{Аркадий Шухман} Хотелось бы, и купол бы восстановили!

\begin{itemize} % {
\iusr{Аркадий Шухман}

Когда-то на экскурсии, которую проводила Е. Н. Рахлина, я узнал, что караимская
община Киева почти в полном составе эмигрировала в США сразу после революции
1917 года. Через несколько лет, в20х годах они обратились к властям Киева с
предложением : за большие деньги разобрать Кенассу и вывезти ее в Штаты, а там
вновь собрать. Киевсовет после обсуждения им отказал, и, чтобы окончательно
закрыть вопрос, открыл там Клуб рабочей молодёжи. После войны в здании долго
работал кинотеатр повторного фильма "Заря", который был очень популярен среди
киевлян-киноманов.

\iusr{Оксана Денисова}
\textbf{Аркадий Шухман} Спасибо Вам большое, не знала этой информации!
\end{itemize} % }

\iusr{Геннадий Дудко}
\textbf{Аркадий Шухман} Очень бы хотел бы посмотреть на это. Очень.

\iusr{Наталья Твердохлеб}
\textbf{Аркадий Шухман} вы в это верите?!

\begin{itemize} % {
\iusr{Аркадий Шухман}
\textbf{Наталья Твердохлеб} Нет, конечно! Иначе тут такое начнется, что пером не описать... Это я пофантазировал слегка.

\iusr{Наталья Твердохлеб}
\textbf{Аркадий Шухман} А я-то думала, что нашла наконец единомышленника) Кстати, а что должно начаться, если вдруг приедет собственник, реставрирует, например, свой особняк и распорядится им по своему усмотрению?)
\end{itemize} % }

\iusr{Soloviova Nataly}
\textbf{Аркадий Шухман} Цього вже не буде. Нерухомість змінила багато власників з радянських часів.

\begin{itemize} % {
\iusr{Аркадий Шухман}
\textbf{Soloviova Nataly} 

Но в Прибалтике, когда такой закон заработал, это привело к множеству трагедий.
Например, в Латвии народную артистку Вию Артмане выкинули на улицу из ее
особняка, когда пришли бывшие законные владельцы.И у нас, например, во Львове,
многим пришлось бы убираться прочь из незаконно занятых во время немецкой
оккупации квартир. Но таков закон.Правда, в настоящее время по понятным
причинам это нереально. Но придет время, и никуда от этого деться - иначе
Евросоюза Украине не видать, как своих ушей. И никто тут исключений для нашей
страны делать не будет, уверяю вас.

\iusr{Soloviova Nataly}
\textbf{Аркадий Шухман} 

зазвичай такі закони приймають в перші роки після закінчення диктатур, чи не
приймають. У нас вже майже не лишилось будівль, які колись були безоплатно
передані. Їх вже багато разів перепродали. А коли власники купили щось за
гроші, то вже у них не відбереш.

\end{itemize} % }

\iusr{Ирина Петрова}
\textbf{Аркадий Шухман} 

а сколько сейчас караимов в Украине? И есть ли у них финансовая возможность
реконструировать здание кенассы, поддерживать его жизнедеятельность и т.пр.?
Сколько людей в Киеве будут посещать кенассу, как конфессионное заведение? Не
говорю, что сейчас здание в отличном состоянии, вовсе нет. Есть ли где-то
данные по численности караимов, интересно?

\begin{itemize} % {
\iusr{Аркадий Шухман}
\textbf{Ирина Петрова} 

О численности караимов в Украине можно ответить только после переписи
населения, которой не проводилось уже 20 лет. Но исходя из того факта, что
недавно Верховна Рада признала караимов одни из коренных народов в Украине, они
в стране живут. Думаю, непосредственно в столице их единицы, зато в Крыму община
есть, культурная и религиозная жизнь там поддерживается. Слава богу, их не
уничтожали во время немецкой оккупации, как крымчаков, евреев и цыган. Будет ли
в Киеве Кенасса работать полноценно, сказать трудно. Все выяснится после того,
как заработает закон о реституции. И дело тут не в финансах, спонсоры найдутся.

\iusr{Аркадий Шухман}
\textbf{Ирина Петрова} Даже если в Киеве осталось несколько караимов, их права на Кенассу неоспоримы.
\end{itemize} % }

\iusr{Ирина Петрова}
\textbf{Аркадий Шухман} 

спасибо . Кстати, о реституции. Дом бабушки в Святошине был как раз на том
месте, где сейчас корпус университета "Украина", (где преподавал мой сын).
Жаль, что нашей семье этот закон полезным не окажется. Сорри за эгоизм.

\iusr{Ирина Петрова}
\textbf{Аркадий Шухман} согласна.

\end{itemize} % }

\iusr{Natalia Pigulevska}
Интересно, а что случилось с ними после революции... Выжили они или нет?

\begin{itemize} % {
\iusr{Оксана Денисова}
\textbf{Natalia Pigulevska} 

Соломон Коген умер ещё в 1900 году , до революции фабриками и фирмой управлял
его старший сын Моисей. После революции все фабрики национализировали, а вот
что с Моисеем было дальше, честно , не знаю!

\iusr{Аркадий Шухман}
\textbf{Natalia Pigulevska} Практически вся община переехала в США.

\begin{itemize} % {
\iusr{Оксана Денисова}
\textbf{Аркадий Шухман} Вот, теперь я тоже знаю!

\iusr{Natalia Pigulevska}
\textbf{Аркадий Шухман} фух, спасибо!

\iusr{Эвелина Яковенко}
\textbf{Аркадий Шухман} Умницы! Женщины-караимки очень красивые!
\end{itemize} % }

\end{itemize} % }

\iusr{Ольга Чапала}
Курили люльки, а не трубочки

\iusr{Константин Грищенко}
Познавательно)

\iusr{Надія Шейна}
Спасибо, очень интересная история ...

\iusr{Костя Федотов}
кинотеатр "Зоря"
Таким, я его запомнил)

\begin{itemize} % {
\iusr{Аркадий Лопухов}
\textbf{Костя Федотов} Уважаемый Константин , с помню , до кинотеатра там был кукольный театр , который я посещал с родителями .

\begin{itemize} % {
\iusr{Костя Федотов}
\textbf{Аркадий Лопухов} 

Уважаемый Аркадий! Когда улица ещё называлась Большой Подвальной, а не
Ярославов вал, я жил в доме \#14, в коммунальной квартире \#91. В те времена...
Кукольный театр находился в здании бывшей, сейчас настоящей Центральной
Синагоги Киева, на ул. Шота Руставели 13  @igg{fbicon.wink} 

\iusr{Костя Федотов}

...а, ещё помню, напротив кинотеатра Зоря, через дорогу, где сейчас посольство
Польши, находились бараки, одноэтажные деревянные сооружения

\iusr{Аркадий Лопухов}
\textbf{Костя Федотов} Уважаемый Константин , Вы родились позже , поэтому кукольный театр на Большой Подвальной не застали .
\end{itemize} % }

\end{itemize} % }

\iusr{Ольга Чапала}

Караїмів лишилось дуже мало. Наприкінці існування СРСР їх було всього 8 тисяч і
мешкали вони невеликими громадами у країнах Балтії

\begin{itemize} % {
\iusr{Александр Длугач}
\textbf{Ольга Чапала} караимов значительно больше, просто далеко не все сохранили идентификацию народа, более того знают о том, кто они есть на самом деле...

\iusr{Ковальская Татьяна}
\textbf{Ольга Чапала} За 2 років тому була в Тракаї і в Вільнюсі, є багато караїмів, є караїмьська школа, ресторан караїмської кухні і ще багато чого, сподіваюсь, що є і чистокровних караїми.

\iusr{Оксана Денисова}
\textbf{Ковальская Татьяна} Я теперь мечтаю попасть в Тракай! Обязательно съезжу, посмотрю ! Спасибо Вам!

\iusr{Илья Калинин}
Я на четверть караим. И горжусь этим!

\begin{itemize} % {
\iusr{Ольга Чапала}
\textbf{Илья Калинин} Чудово, коли люди не забувають своє коріння

\iusr{Илья Калинин}
\textbf{Ольга Чапала} Только так! Более того был снят фильм о караимах в котором есть небольшой сюжет о моем кафе, где моя мама представляет несколько блюд караимской кухни. А мои караимские пирожки обожают по всему миру.
\end{itemize} % }

\iusr{Ковальская Татьяна}
\textbf{Оксана Денисова} Все возможно!

\end{itemize} % }

\iusr{Ковальская Татьяна}

Мои дедушка и бабушка были чистокровными караимами и были хорошо знакомы с
Коганом, я ещё помню, как в кенассе караимы устраивали вечера, ещё были молодые
исполнители караимских песен, мне приходилось бывать там. Конечно хотелось бы
чтобы кенасса ьыла передана караимской общине, но где эта община? ,да и
караимов в Украине почти не осталось, есть немного от смешаных браков, Израиль
признает караимов, как иудеев.

\begin{itemize} % {
\iusr{Александр Длугач}
\textbf{Ковальская Татьяна} проблема именно в том, что не признавал и не признает, не путайте караимов с крымчаками...

\begin{itemize} % {
\iusr{Ковальская Татьяна}
\textbf{Александр Длугач} Мне ли путать ?

\iusr{Александр Длугач}
\textbf{Ковальская Татьяна} 

судя по комментарию, то действительно путаете! В чем разница в вероисповедании
караимов и крымчаков знаете? Народ как бы один, а вот в вероисповедании разница
ощутима... Кроме того среди крымских татар то же большое количество из
караимов, правда, с вероисповеданием проблема...

\iusr{Daniel Zak}
\textbf{Александр Длугач} 

Караимы и Крымчаки это одни люди. Израиль признаёт Караимов и там самая большая
община. На ютюбе есть интересный фильм об этом. И кстати Караимы это прямые
потомки Хазаров, которые исповедовали свою интерпретацию Иудаизма. Есть много
доказательств того что Князь Владимир, принёсший Христианство на Русь был
наполовину Хазаром (по матери).

\iusr{Александр Длугач}
\textbf{Daniel Zak} 

я уже писал ранее о том, что караимы и крымчаки это один народ, но исповедующий
отличное вероисповедание, кроме этого я писал о том, что караимы это основа
Хазарского Каганата, а не выходцы из него и о том, что согласно Конституции
Пилипа Орлика Казаки Войска Запорожского как раз являются представителями
прямой ветви выходцев из основы Каганата... Что касается караимов, то это так
называемый "пещерный народ", принадлежащий к Роду Девы Марии... Кроме этого, я
писал, что Иудеи, не признают караимов, как собственно и хазар иудеями,
поскольку последние не признают Устную Тору... Что касается Израиля то на его
территории проживают люди исповедующие массу религий, более чем вероятно, что
среди них существуют и караимы!!! Что касается Владимира, то то, что его мать
могла принадлежать к населению Хазарского Каганата, не говорит о том, что
Владимир исповедовал иудаизм... Поскольку, Иудаизм, это вероисповедание и
"национальность" из него сделал ни кто иной, как Первый Нарком по делам
национальностей, Товарищ Сталин!!! @igg{fbicon.index.pointing.up}
@igg{fbicon.wink}  @igg{fbicon.face.wink.tongue}
@igg{fbicon.face.rolling.eyes}  @igg{fbicon.laugh.rolling.floor}{repeat=3}
Самое сложное во всей этой истории определить кто есть кто, к примеру, кто по
национальности тот же "бухгалтер из Полтавы, выкрест в униатство" и
одновременно герой современной Украины, застреленный евреем за геноцид евреев,
некто Симон Петлюра?!! @igg{fbicon.index.pointing.up}
@igg{fbicon.face.rolling.eyes}  @igg{fbicon.face.head.bandage} 

\iusr{Daniel Zak}
\textbf{Александр Длугач} 

я не думаю что это формат в котором нам необходимо продолжать такого рода
дискуссию. Ведь сама история совершенно не о том. Но могу вас заверить что
история остаётся историей несмотря на то как ее меняют и приспосабливают. В
моей интерпретации половина того что вы сказали это... скажем так, не совсем
правильно и совершенно беспочвенно. Но мы оба останемся при своём мнении и
поэтому обсуждение (даже если оно очень цивилизованное) все равно ни к чему не
приведёт. Спасибо.

\end{itemize} % }

\iusr{Наталья Твердохлеб}
\textbf{Ковальская Татьяна} 

Было бы желание передать... А караимы бы нашлись, есть еще. Но разве у нас хоть
что-то чужое возвращали? Хоть кто-то искал законных наследников, потомков
владельцев чудесных домов? Не хочется приводить примеры и писать, во что их
превратили новые «хозяева». Собственно, кенасса — яркий пример. Грустное
зрелище, не правда ли?

\begin{itemize} % {
\iusr{Ковальская Татьяна}
\textbf{Наталья Твердохлеб} Может и случится, но будут ли средства у той горстки караимов привести в порядок, отремонтировать кенасса, ведь за годы хозяйствования все пришло в аварийное состояние, спасибо, что хоть не взорвали!

\iusr{Наталья Твердохлеб}
\textbf{Ковальская Татьяна} Как "хозяйствует" культурная "элита" Киева - результат говорит сам за себя(
\end{itemize} % }

\iusr{Soloviova Nataly}
\textbf{Ковальская Татьяна} 

Якщо ваші бабуся з дідусем були караїмами, ви знаєте, що тільки таких вони і
визнають (від чистих шлюбів). Я була в Євпаторії ще до окупації і на той момент
в книзі Караїмов (родовід народу) було щось біля 1000 караїмов і вони розуміли,
що вимирають.

\begin{itemize} % {
\iusr{Ковальская Татьяна}
\textbf{Soloviova Nataly} Караїмами теж визнають, якщо матір караїмка, в мене так!

\iusr{Soloviova Nataly}
\textbf{Ковальская Татьяна} Я за освітою історик і спілкувалась з їх архівістом, він жалівся, що так не зробили. То з його слів. На жаль, зв'язки втрачені, тому підтвердити зараз не можу.
\end{itemize} % }

\iusr{Оксана Космина}
\textbf{Ковальская Татьяна} і правильно робить

\iusr{Оксана Денисова}
\textbf{Ковальская Татьяна} 

Татьяна Адольфовна, как интересно то, что Вы рассказываете о караимах! Напишите
о них побольше, расскажите всем, это так интересно!!! И спасибо Вам большое!!!

\begin{itemize} % {
\iusr{Ковальская Татьяна}
\textbf{Оксана Денисова} 

Особенно много мне не рассказать, при жизни моих родных я была слишком мала и
не любознательна, к сожалению, однако немного освоила караимскую кухню, кроме
мамы и бабушки мне помогла книга евпоторианской караимки , Караимская кухня,
пользуюсь и сейчас, просто читаю, т к уже не готовлю. Спасибо Вам, ваши
информации неоценимы!


\iusr{Оксана Денисова}
\textbf{Ковальская Татьяна} Спасибо Вам огромное за Вашу любовь к Киеву, к караимам!!!

\iusr{Ковальская Татьяна}
\textbf{Оксана Денисова} Во время оккупации немцы освидетовали караимов и выдавали справки записью: ...вона( він ) є дійсно караімка и на немецком тот же текст!

\iusr{Оксана Денисова}
\textbf{Ковальская Татьяна} Как интересно! Спасибо !

\iusr{Александр Длугач}
\textbf{Ковальская Татьяна} правильно, именно так и было, а причина тому, кроется в том, что на запрос Гитлера к Раввинам, на тему, "кто есть кто", последние ответили, что караимы не евреи, т.е. не признали их иудеями ...

\iusr{Ковальская Татьяна}
\textbf{Александр Длугач} Могу вам сказать: караимы молодцы, ещё умнее евреев! Горжусь!

\iusr{Александр Длугач}
\textbf{Ковальская Татьяна} без всякого сомнения!!!
\end{itemize} % }

\end{itemize} % }

\iusr{Alexey Paschenko}

Сохранилось здание, в котором находилась кенасса до того, как было построено
здание на Большой Подвальной (в настоящее время - Ярославов Вал)

\ifcmt
  ig https://scontent-lga3-1.xx.fbcdn.net/v/t1.6435-9/254677719_5089055754437884_8968879308245791049_n.jpg?_nc_cat=104&ccb=1-5&_nc_sid=dbeb18&_nc_ohc=OhuCtVva91QAX8VP_xo&_nc_ht=scontent-lga3-1.xx&oh=4af0c9b87da7dd25d39ca3e12b1c946f&oe=61AF3CB6
  @width 0.4
\fi

\begin{itemize} % {
\iusr{Оксана Денисова}
\textbf{Alexey Paschenko} Это Михайловский переулок 8, правильно?

\begin{itemize} % {
\iusr{Alexey Paschenko}
\textbf{Оксана Денисова} да. Гугл-карты даёт фото до ремонта

\ifcmt
  ig https://scontent-lga3-2.xx.fbcdn.net/v/t1.6435-9/254957371_5090268250983301_189411551443909351_n.jpg?_nc_cat=105&ccb=1-5&_nc_sid=dbeb18&_nc_ohc=SWnt6CYoCTYAX_5sd0A&_nc_ht=scontent-lga3-2.xx&oh=1cc17463e6f5077c20cecee67eb483ab&oe=61AFD071
  @width 0.4
\fi

\iusr{Оксана Денисова}
\textbf{Alexey Paschenko} Спасибо большое!

\end{itemize} % }

\end{itemize} % }

\iusr{Alexey Paschenko}

\ifcmt
  ig https://scontent-lga3-2.xx.fbcdn.net/v/t1.6435-9/254916365_5089055984437861_9133982255505053274_n.jpg?_nc_cat=107&ccb=1-5&_nc_sid=dbeb18&_nc_ohc=L1xbLVQuuZAAX9H3G2N&_nc_oc=AQmqQe-aBgQCyjuk6WLbZCFxja8QH9qbYN8rJAIT9q2r0BpGB1DAHPcvK9ErTcwHnVM&_nc_ht=scontent-lga3-2.xx&oh=042f63b4c961d2f8b3ff81f3eaf054ae&oe=61AE9A46
  @width 0.4
\fi

\iusr{Alena Tkachenko}
Благодарю за чудесный рассказ  @igg{fbicon.face.smiling.hearts} 

\iusr{Євгенія Туринська}
Дякую. Дуже цікаво.

\iusr{Tanya Orlov}
Такие имена надо не забывать !

\iusr{Alexey Paschenko}

\ifcmt
  ig https://scontent-lga3-2.xx.fbcdn.net/v/t1.6435-9/254352300_5089105837766209_2337266171602858261_n.jpg?_nc_cat=111&ccb=1-5&_nc_sid=dbeb18&_nc_ohc=jNSFsytrfQsAX_mbsuM&_nc_ht=scontent-lga3-2.xx&oh=1f25e27972d58f23bc7b3ca9eab5c505&oe=61B0A8AA
  @width 0.3

	ig https://scontent-lga3-2.xx.fbcdn.net/v/t1.6435-9/255184702_5089106037766189_4890122599480747432_n.jpg?_nc_cat=100&ccb=1-5&_nc_sid=dbeb18&_nc_ohc=VGsQ4FZP6HoAX9O3YiH&_nc_ht=scontent-lga3-2.xx&oh=679c2e7778d23a6463da5f18e6ab2878&oe=61B0E681
  @width 0.3
\fi

\iusr{Alexey Paschenko}

Караимы - потомки тех евреев, которые выехали из Иудеи задолго до изгнания.

Возле Евпатории есть мацевы (могильные надгробия), которым больше 2000 лет. То
есть там жили евреи, выехавшие до нашей эры. Вот потому они и признают только
Тору, а не признают Талмуд, который был создан позже.

\begin{itemize} % {
\iusr{Нечипоренко Надежда}
\textbf{Alexey Paschenko} Как интересно!

\iusr{Alexey Paschenko}
\textbf{Нечипоренко Надежда} из Википедии:

Открытие А. Фирковичем и исследование древнейших памятников родового кладбища
караимов в Иосафатовой долине[10] стало научной сенсацией, вызвало глубокий
интерес и получило высокую оценку еще при жизни ученого такими гебраистами XIX
века, как Б. Штерн[11], С. Пинс. [12], Р. Рабинович[13].

на кладбище Чуфут-Кале

На основании этих находок Фиркович сделал заявление о том, что караимы жили в
Крыму до Рождества Христова и потому не должны нести ответственность за
распятие Христа. В 1855 году караимы Тракая обратились к российским властям с
просьбой: на основании фундаментальных находок Фирковича перестать называть их
«евреями-караимами» и называть просто «караимами». Фиркович в 1857 году
присоединяется к этой просьбе. Царское правительство направило положительный
ответ.

\iusr{Рафаэль Назаров}
\textbf{Alexey Paschenko}  @igg{fbicon.thumb.up.yellow} 

\iusr{Оксана Денисова}
\textbf{Alexey Paschenko} Как интересно, спасибо Вам!!!

\iusr{Валерия Радченко-Стецула}
\textbf{Alexey Paschenko} очень интересно!.. спасибо за информацию.
\end{itemize} % }

\iusr{Alexey Paschenko}
Есть хорошая книга: "Караим Авраам Фиркович. Еврейские рукописи. История. Путешествия" .
Автор Всеволод Вихнович

\iusr{Вера Ярошенко}
Очень интересно. Спасибо

\iusr{Vadym Shvydkiy}
співав там у часи, коли там був будинок Актора

\begin{itemize} % {
\iusr{Оксана Денисова}
\textbf{Vadym Shvydkiy} Класс! Там же очень хорошая акустика !

\iusr{Vadym Shvydkiy}
\textbf{Оксана Денисова}  @igg{fbicon.grin} 
\end{itemize} % }

\iusr{Anna Kievatz}
Как хорошо, не курили, а нюхали! А теперь выходят на балкон, даже среди ночи; а нюхают, аж угорают, не курящие  @igg{fbicon.face.disappointed} 

\iusr{Наталия Пантелеенко}
Спасибо! Интересно @igg{fbicon.exclamation.mark} @igg{fbicon.smile} 

\iusr{Анатолий Пузырь}

Караимы жили не только, в основном, в Крыму, но и на Кавказе их еще
больше. И, вообще, когда евреев изгнали из Эдема, они разбрелись, кто куда.

\begin{itemize} % {
\iusr{Нина Белоусова}
\textbf{Анатолий Пузырь} изгнали из Эдема Еву с Адамом, евреев изгнали из Египта.

\iusr{Анатолий Пузырь}
\textbf{Нина Белоусова} Из Эдема, район Персидского залива(их Родина).
А через 1000 лет тех, кто ,,рванул" в Египет, левитов, изгнали еще и оттуда.
Эти факты мало рекламируют.
Историю пишут,, ПОБЕДИТЕЛИ".

\iusr{Анатолий Пузырь}
\textbf{Нина Белоусова} А Адама и Еву изгнали из Рая(Эдема).

\iusr{Анатолий Пузырь}
\textbf{Нина Белоусова} Кому то удобно историей манипулировать.
\end{itemize} % }

\iusr{Татьяна Годынская}
Очень познавательно. Как же разнообразна и богата история нашего города!

\begin{itemize} % {
\iusr{Оксана Денисова}
\textbf{Татьяна Годынская} Спасибо!
\end{itemize} % }

\iusr{Antonina Sheiko}
жаль, что советские уроды сняли с нее купол(

\begin{itemize} % {
\iusr{Оксана Денисова}
\textbf{Antonina Sheiko} Очень жаль, изуродовали красоту @igg{fbicon.face.pensive} 
\end{itemize} % }

\iusr{Наталия Хохлова}
Дякую за розповідь, дуже цікаво

\begin{itemize} % {
\iusr{Оксана Денисова}
\textbf{Наталия Хохлова} Дякую!
\end{itemize} % }

\iusr{Людмила Гаврилюк}
Спасибо, очень интересно
Я много лет работала на Ярославом Валу прямо рядом с этим строением и не знала эту историю...

\begin{itemize} % {
\iusr{Оксана Денисова}
\textbf{Людмила Гаврилюк} Спасибо!
\end{itemize} % }

\iusr{Валентина Нагорна}
Дуже вдячна за цікаву , пізнавальну розповідь!

\begin{itemize} % {
\iusr{Оксана Денисова}
\textbf{Валентина Нагорна} Дякую!
\end{itemize} % }

\iusr{Натали Спивак}

\ifcmt
  ig https://scontent-lga3-1.xx.fbcdn.net/v/t39.1997-6/s168x128/198679738_1581651558693954_1103589833244933008_n.png?_nc_cat=1&ccb=1-5&_nc_sid=ac3552&_nc_ohc=l0Di8ofzBCIAX9_wjps&tn=lCYVFeHcTIAFcAzi&_nc_ht=scontent-lga3-1.xx&oh=7a92eac47f3d3a253b4e3cfd6c512e1c&oe=618FB937
  @width 0.1
\fi

\iusr{Наташа Костылева}
А потом история преподнесла олигархам сюрприз и теперь всё это национализировано, а в кенасе Дом актера. Быть добру

\iusr{Альона Андрєєва}
Очень увлекательно!

\iusr{Алина Савина}

Большое спасибо за интересный рассказ! Кстати, путешествуя Украиной нынешним
летом, узнала о караимах то, что они также большими сообществами проживали на
Западной и в Польше. И меня ещё больше удивило, куда же делся этот такой
большой народ, о котором лично я впервые как раз и услышала благодаря нашей
киевской Кенасе...

\begin{itemize} % {
\iusr{Оксана Денисова}
\textbf{Алина Савина} Спасибо!

\iusr{Sima Maliuga}
\textbf{Алина Савина} Караимы жили и в Литве!. Друзья нам показали улочку с деревянными домами, покрашенные в разные цвета.
\end{itemize} % }

\iusr{Наталка Бондаренко}
Дякую, дуже цікава пізнавальна розповідь.

\begin{itemize} % {
\iusr{Оксана Денисова}
\textbf{Наталка Бондаренко} Дякую!
\end{itemize} % }

\iusr{Мария Бутковская}
Благодарю за интересную информацию!

\begin{itemize} % {
\iusr{Оксана Денисова}
\textbf{Мария Бутковская} Спасибо!
\end{itemize} % }

\iusr{Люала Пилипейко}

\ifcmt
  ig https://scontent-lga3-1.xx.fbcdn.net/v/t39.1997-6/s168x128/11891339_897114570361979_1916032859_n.png?_nc_cat=1&ccb=1-5&_nc_sid=ac3552&_nc_ohc=P3x2VopvSg4AX_TT-mX&_nc_ht=scontent-lga3-1.xx&oh=ada6d65b4f603745426c031067bacf58&oe=618F81B2
  @width 0.15
\fi

\iusr{Ольга Лубягина}
Спасибо. Очень познавательно

\begin{itemize} % {
\iusr{Оксана Денисова}
\textbf{Ольга Лубягина} Спасибо!
\end{itemize} % }

\iusr{Оксана Космина}
Караїми – це в першу чергу релігія, а не окремий народ!

\iusr{Alexey Paschenko}
\textbf{Oksana Kosmina} ні, релігія в них - іудаїзм

\iusr{Наташа Коваленко}
Дуже цікаво, дякую!

\iusr{Оксана Денисова}
\textbf{Наташа Коваленко} Дякую!

\iusr{Ирина Марголина}
Цікаво, дякую!

\iusr{Оксана Денисова}
\textbf{Ирина Марголина} Дякую !

\iusr{Елена Пака}
Да уж... И спаивали,
и скуривали.

\iusr{Валентин Бабич}
Да, интересный рассказик. 

@igg{fbicon.hands.applause.yellow}{repeat=3}  @igg{fbicon.thumb.up.yellow}{repeat=3} 

\iusr{Оксана Денисова}
\textbf{Валентин Бабич} Спасибо!

\iusr{Лина Панчук}

Интересненько. Вот только сдается мне, что эти караимы Соломон, Моисей да еще
и Коганы, да с такими бизнесовыми замашками, обыкновенные - евреи.

\begin{itemize} % {
\iusr{Оксана Денисова}
\textbf{Лина Панчук} Вы у нас в группе человек новый, у нас не принята такая нетерпимость на грани хамства.

\begin{itemize} % {
\iusr{Лина Панчук}
\textbf{Оксана Денисова} а в чем хамство?
И кто вам сказал , что я плохо отношусь к еареям ? У меня много знакомых еврев , и у меня были названные дедушка с бабушкой - евреи.
Просто надо называть вещи своими именами.

\iusr{Олександр Длігач}
\textbf{Лина Панчук} чё ж ня называяте "своимя имянами"?
Мабуть, вивчали з повагою історію, хоч одного етносу? Хоч свого???
У ті прадавні часи Великий князь Вітовт вивів плем'я караїмів з...
Та, це вже зовсім інша історія...

\iusr{Mark Gutman}
\textbf{Лина Панчук} нацисты убивали их как евреев. В советском паспорте для отдела кадров- караим почти еврей  @igg{fbicon.smile}. Израиль их тоже принимал как репотриантов. Мы называли это здание караимской синагогой. Но различия есть.
\end{itemize} % }

\iusr{Евгений Раидун}
\textbf{Лина Панчук} "Необыкновеные"

\end{itemize} % }

\iusr{Людмила Коваленко}
Спасибо за чудесную историю

\iusr{Сергей Хромешкин}

\ifcmt
  ig https://scontent-lga3-1.xx.fbcdn.net/v/t39.1997-6/p480x480/105941685_953860581742966_1572841152382279834_n.png?_nc_cat=1&ccb=1-5&_nc_sid=0572db&_nc_ohc=KWw8Fgmk2-kAX9qvJ0i&_nc_ht=scontent-lga3-1.xx&oh=cfc4bee3008ee23fea69782f0725388a&oe=6190704B
  @width 0.1
\fi

\iusr{Светлана Андреева}
Спасибо за увлекательный рассказ.

\begin{itemize} % {
\iusr{Оксана Денисова}
\textbf{Светлана Андреева} Спасибо!
\end{itemize} % }

\iusr{Елена Бурд}

\end{itemize} % }
