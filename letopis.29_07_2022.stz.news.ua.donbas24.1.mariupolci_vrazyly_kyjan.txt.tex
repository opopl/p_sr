% vim: keymap=russian-jcukenwin
%%beginhead 
 
%%file 29_07_2022.stz.news.ua.donbas24.1.mariupolci_vrazyly_kyjan.txt
%%parent 29_07_2022.stz.news.ua.donbas24.1.mariupolci_vrazyly_kyjan
 
%%url 
 
%%author_id 
%%date 
 
%%tags 
%%title 
 
%%endhead 

Маріупольці вразили своєю прем'єрою киян

У Театрі на Подолі 28 липня відбулася важлива прем'єра Театру авторської п'єси «Conception», присвячена Маріуполю та маріупольцям

28 липня Театр авторської п'єси «Conception» представив прем'єрну виставу
«Обличчя кольору війна» за документальними подіями. Маріупольські актори вперше
після довгої перерви виступали на великій сцені. При цьому, це одна з найбільш
сучасних сцен України. Глядацька зала була переповнена. Автор сценарію та
постановник — режисер і художній керівник «Conception» Олексій Гнатюк майстерно
висвітлив тему війни, виживання і людяності. Він дуже задоволений результатом і
вважає, що все задумане вдалося реалізувати.

«Вважаю, що і актори, і вся команда виклалася на всі 100%. Непідробні емоції
глядачів є яскравою ілюстрацією того, що все у нас вийшло», — наголосив
режисер. 

Глядачів заворожувала і самобутність оформлення сцени, і влучний світловий та
музичний супровід. Проте особливу увагу привертала майстерність акторської гри
кожного учасника. Від пронизливих монологів героїв у глядачів завмирало серце,
адже актори проживали кожне слово, кожну емоцію, а разом з ними ці емоції
передавалися і глядачам. Всі вони розповідали власні історії, те, що самі
пережили і що бачили на власні очі... Як на початку війни намагалися жити і
займатися звичними справами. Однак «прильотів» ставало все більше, на п'ятий
день вимкнули світло. Були змушені під обстрілами готувати їжу. Як ховалися у
підвалах і втрачали знайомих. Після всього пережитого актори грали ще більш
майстерно. Хоча вони зовсім не грали, кожну репетицію і в день прем'єрного
показу артисти проживали свою роль знову і знову. Актор Дмитро Гриценко та
актриса Катерина Калмикова вважають, що завдяки виставі вдалося передати все
те, що відчували і відчувають маріупольці зараз.

«За особистими відчуттями, думаю, що ми змогли донести та передати глядачу весь
той біль, страх, надії, все, що пережили і переживають маріупольці. Думаю, що
все вийшло. Я відчував енергетику своїх партнерів по сцені та розумів, що для
всіх нас, на жаль, це не гра і ми дійсно знову і знову переживаємо ті події.
Здається, у нас вийшло достукатись до глядачів», — поділився Дмитро.

«До мене після вистави підходило багато киян, які говорили, що навіть не
уявляли, який жах насправді пережили маріупольці. Багато хто питав, чи дійсно
це правда, чи можливо, все ж вигадана історія… Доводилося пояснювати, що все це
відбувалося з усіма маріупольцями, яким не вдалося одразу виїхати. Я вважаю, що
у нас все вийшло і дуже задоволена результатом. Ми змогли донести цей матеріал
до глядача і будемо розповідати нашу історію й надалі», — розповіла Катерина.

Відомий маріупольський фотограф і актор Євген Сосновський вважає, що з кожним
виступом вистава ставатиме ще більш цілісною.

«Не можу сказати, що мені все вдалося, про свої власні недоліки я знаю. Але ми
зробили все, що могли і віддавалися на повну силу. Я дуже вдячний режисеру
вистави Олексію Гнатюку за все, що він робить. У цьому колективі почав
працювати нещодавно, але мені дуже комфорно працювати з усіма акторами. Вважаю,
що з кожним показом вистава ставатиме ще кращою», — підкреслив актор Євген
Сосновський. 

А юна актриса Катерина Салієва наголосила, що для кожного з акторів вихід на
таку велику сцену є дуже важливим і потрібним.

«Ця прем'єра для нас дуже багато значить, тому що вихід на сцену після такої
довгої перерви був дуже хвилюючим. Навіть найбільш досвідчені актори дуже
переживали. Складно розповідати свої історії, доносити до глядачів, адже
хочеться, щоб все зрозуміли саме так, як ми хочемо це донести. Це дійсно
трагедія, яка покалічила велику кількість людей. І коли ти чуєш монологи інших
акторів, неможливо не плакати, хоча я їх слухала більше 15 разів», — зазначила
Катерина.

Завдяки сильній режисурі та блискучій грі акторів глядачі перенеслися в
Маріуполь у найбільш важкий його період. Дуже складно було втримати власні
емоції.

«Дивитися без сліз було неможливо. Я спочатку подумала, що це київські актори
ставлять виставу про Маріуполь. Але потім почула, що це маріупольські актори. Я
під глибоким враженням. Дивилася крізь сльози. Дуже сильна гра. Дякую, що
створили таку виставу. Я б її ще раз подивилася, але трохи згодом», — розповіла
глядачка з Києва Катерина.

«Цю виставу я прийшов подивитися з друзями. Я ніколи не був у Маріуполі, але
історія, яку розповіли актори, не може не вразити. Ми перенеслися в місто, коли
воно було під бомбардуваннями, коли там виживали і гинули люди. Цей спектакль
дуже важливий і потрібний. Мене завоювали актори. Сподіваюся побачити й інші
вистави цього театру», — поділився враженнями киянин Михайло.

На виставу прийшли і маріупольці, які наразі перебувають у Києві. Зокрема,
маріупольський театрал Євген Сокирко вважає, що виставу «Обличчя кольору війна»
наразі потрібно показувати щодня.

«Життя у підвалі, під обстрілами, пошуки води та їжі, згорілі будинки, трупи і
могили у дворі — в цьому довелося всім нам, маріупольцям, виживати. Дивитись на
це на сцені ще щемливіше. Разом з дуже точною музикою і відеорядом — болісно.
Примітивні малюнки під час монологів героїв відображають, яким примітивним
стало наше життя завдяки нахабному і безглуздому, кровожерливому ворогу. Цю
виставу потрібно у тримільйонному Києві показувати кожного дня, за квитками з
мінімальною ціною. Бо багато людей не відчуває, що таке насправді війна. Що це
безлад, бомби, втрата рідних назавжди», — прокоментував Євген Сокирко.

Актори продемонстрували не лише акторську майстерність, високий рівень
художнього слова, а й вокальні та хореографічні вміння. Доречно використані і
сучасні технічні можливості. Автором візуальної частини став куратор багатьох
театральних програм Андрій Палатний. Відеоряд, який транслювався на екрані
(фото Євгена Сосновського, малюнки Даниїла Немировського) та створені наживо
ілюстрації (Маріанна Кондратенко) дуже органічно взаємодіяли з грою акторів.
Для присутніх година промайнула як мить. Вдячна публіка довго аплодувала
акторам та режисеру, дякуючи за пережиті емоції. 

Після прем'єри колективу необхідно вирішити багато важливих питань, щоб
продовжити працювати задля розвитку культури Маріуполя.

«Ми дуже вдячні Театру на Подолі, в якому ми представили нашу прем'єру. Проте в
нас і досі немає власного приміщення, де б ми могли проводити репетиції.
Гострим питанням залишається відсутність житла у акторів. Це наші головні
проблеми наразі, які вимагають термінового вирішення», — наголосив режисер
Театру авторської п'єси «Conception» Олексій Гнатюк.

Художній керівник і режисер театру буде намагатися вирішити всі важливі питання
найближчим часом, щоб якомога швидше продовжити роботу театру та реалізовувати
нові ідеї. У планах міської влади Маріуполя показати виставу «Обличчя кольору
війна» містами як України, так і Європи.

Нагадаємо, раніше Донбас24 розповідав про Жіночий футбольний клуб «Маріуполь».

ФОТО: Олександра Євмененка, Галини Бабій та з особистого архіву Ольги Демідко.
