% vim: keymap=russian-jcukenwin
%%beginhead 
 
%%file 24_12_2021.stz.news.ru.kp.1.doneck_jolka
%%parent 24_12_2021
 
%%url https://www.donetsk.kp.ru/daily/28374.5/4524263
 
%%author_id makarenkov_nikita,hanarin_pavel
%%date 
 
%%tags 
%%title Главная елка Донецка 2021-2022 зажгла свои огни: фоторепортаж

%%endhead 
\subsection{Главная елка Донецка 2021-2022 зажгла свои огни: фоторепортаж}
\label{sec:24_12_2021.stz.news.ru.kp.1.doneck_jolka}

\Purl{https://www.donetsk.kp.ru/daily/28374.5/4524263}
\ifcmt
 author_begin
   author_id makarenkov_nikita,hanarin_pavel
 author_end
\fi

На площади Ленина собрались сотни дончан.

\ii{24_12_2021.stz.news.ru.kp.1.doneck_jolka.pic.1}

24 декабря в Донецке открыли главную елку Республики. Чтобы увидеть мгновенье,
когда «зеленая красавица» засияет по-новому, на площади собралось немало
горожан. Многие, несмотря на морозную погоду, пришли с детьми. Специально,
чтобы согреть народ, недалеко от елки раздавали бесплатный чай, сделанный в
огромном самоваре.

Уже перед открытием десятки людей облюбовали фотозоны, которые расположились на
самой площади и прилегающей к ней территории. Многие композиции в этом году
стали сюрпризом для жителей: огромный белый медведь, трон около главной сцены,
инсталляция «2022» и множество мини-елок.

\ii{24_12_2021.stz.news.ru.kp.1.doneck_jolka.pic.2}

- В этом году много мест для фотографий! Очень хочется сфотографироваться
везде, но иногда приходится выстоять в очереди из таких же желающих, - говорит
студентка Ирина, которая пришла на площадь с компанией друзей и фотоаппаратом
на шее.

Несколько часов жителей развлекали коллективы. Дончане и сами не уступали
артистам в вокальном исполнении новогодних песен, исполняя их всей площадью
хором, пританцовывая при этом то ли от холода, то ли от праздничного
настроения.

\begin{multicols}{2} % {
\setlength{\parindent}{0pt}

\ii{24_12_2021.stz.news.ru.kp.1.doneck_jolka.pic.3}

- Для нас по-настоящему важно почувствовать дух праздника, видеть улыбки на
лицах у окружающих, - говорит дончанка Ирина Сергеевна. - Ежедневные обстрелы,
сводки с фронта, постоянное напряжения из-за коронавируса... все это настолько
нагнетает, что нам всем просто необходима разгрузка. Да и дети должны видеть
мирные отблески, развлекаться и верить в чудеса.

По традиции перед тем, как на главной елке Республики зажглись огни, к жителям
обратился глава города Алексей Кулемзин.

- Для всех нас новогодние праздники - это сказка, когда сбываются самые
сокровенные мечты и свершаются чудеса. Пусть наступающий год будет радостным,
сказочным, мирным, добрым, здоровым и благополучным, - пожелал градоначальник.

В этом году главную елку Республики впервые украсили светодиодной гжелью
(разновидность русской народной росписи). Ее дополняют гирлянда и шары с
колокольчиками. Использованы огни в трех цветах – синем, красном и белом.

\end{multicols} % }

\ii{24_12_2021.stz.news.ru.kp.1.doneck_jolka.pic.4}

Дончанам понравилась не только елка, но и общая световая композиция вокруг нее.
На столбах по периметру площади расположены установки, с которых раздаются
мощные лучи, то освещающие все вокруг, то соединяющиеся на праздничном дереве.

Народные гулянья по случаю новогодних праздников в Донецке продлятся с 24
декабря по 7 января.

%\ii{24_12_2021.stz.news.ru.kp.1.doneck_jolka.pic.5}

Читайте также:

Дворец молодежи «Юность» станет настоящим дворцом с лифтом и фонтанами. Вопрос
один – когда?

Вопреки всему, разрушенный Украиной дворец, продолжает жить (\href{https://www.donetsk.kp.ru/daily/28374.5/4523761}{подробнее})
Подписывайтесь на наш Telegram и Viber, там самая оперативная информация.
