% vim: keymap=russian-jcukenwin
%%beginhead 
 
%%file 11_09_2022.stz.news.ua.donbas24.1.potjag_do_peremogy.1.geroi_azovstal
%%parent 11_09_2022.stz.news.ua.donbas24.1.potjag_do_peremogy
 
%%url 
 
%%author_id 
%%date 
 
%%tags 
%%title 
 
%%endhead 

\subsubsection{Донеччина. Герої: захисники \enquote{Азовсталі}}

У майбутньому про \enquote{Азовсталь} писатимуть у підручниках з історії. Тижнями
захисники та захисниці заводу мужньо відбивали атаки росіян та були щитом для
українського Маріуполя. Героїзм Азовців на вагоні \enquote{Потягу до перемоги} — подяка
всієї країни за вклад захисників у нашу майбутню перемогу.

Митець \textbf{Дмитро Касянюк} різнобарв'ям фарб поєднав металеву гаму заводу з
кольорами українського прапора, а також зобразив легендарних оборонців
\enquote{Азовсталі}: Редіса, Ореста, Пташку, які зараз перебувають в полоні, з хештегом
\#Маріуполь.

\ii{11_09_2022.stz.news.ua.donbas24.1.potjag_do_peremogy.pic.3}

Дмитро Касянюк, художник: — \enquote{Азовсталь} став могутнім символом героїзму
українського народу. Два місяці тяжкої оборони в повній осаді наші воїни
тримали оборону. Дуже тяжко прийшлося людям, які перебували кожен день під
масивними обстрілами з усієї зброї, яку має ворог. Але вони не здалися, вони
трималися до кінця, вони вистояли. Тяжко уявити те, що пережили мешканці
Маріуполя та військові. Вічна слава всім тим людям, котрі забули про страх,
тримали оборону та боролися за перемогу України.

