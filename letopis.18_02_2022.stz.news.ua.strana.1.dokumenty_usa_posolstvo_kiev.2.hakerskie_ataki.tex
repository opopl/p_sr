% vim: keymap=russian-jcukenwin
%%beginhead 
 
%%file 18_02_2022.stz.news.ua.strana.1.dokumenty_usa_posolstvo_kiev.2.hakerskie_ataki
%%parent 18_02_2022.stz.news.ua.strana.1.dokumenty_usa_posolstvo_kiev
 
%%url 
 
%%author_id 
%%date 
 
%%tags 
%%title 
 
%%endhead 

\subsubsection{Хакерские атаки из Украины}
\label{sec:18_02_2022.stz.news.ua.strana.1.dokumenty_usa_posolstvo_kiev.2.hakerskie_ataki}

Также могли уничтожить документы и сервера, связанные с возможным
вмешательством Демпартии в выборы в США на стороне Клинтон в 2016 году - с
помощью Украины.

Трамп упоминал о них во время скандального телефонного разговора с Зеленским,
который чуть не довел экс-президента до импичмента. По версии его окружения,
украинцы, а не русские стояли за хакерской атакой на сервера Демпартии.

\enquote{В Crowdstrike говорят (компания в США, которая специализируется на
противодействии кибератакам, - Ред.)... Речь об одном из ваших богатых людей.
Сервер, как они говорят, в Украине}, – сказал Трамп.

Crowdstrike - компания, которая помогала расследовать проникновения хакеров в
сети Национального демократического комитета США на прошлых президентских
выборах, откуда информацию слили в WikiLeaks. Речь идет о секретной электронной
переписке Клинтон, обнародование которой пошатнуло ее рейтинг.

В ноябре 2018 года агентство Reuters писало, что хакеры из группировки Cozy
Bear, которую связывают с российскими спецслужбами, рассылали американским
чиновникам фишинговые письма, прикрываясь именами представителей
Государственного департамента США. Информацией об этом с агентством поделились
представители компаний CrowdStrike и FireEye Inc., специализирующиеся на
кибербезопасности.

\enquote{Есть версия, что хакерские сервера на самом деле не в России, а в Украине, у
друзей Демократической партии. И что демократы сами подстроили взлом, чтобы
таким образом скомпрометировать Трампа в якобы сговоре с Россией, а также
оправдать проигрыш на выборах. Демократы, естественно, все отрицают}, - говорит
\enquote{Стране} политический эксперт из Вашингтона Питер Стоун.  

Эту версию косвенно ранее продвигал и сам Трамп.

\enquote{Почему Национальный демократический комитет отказался передать свой сервер
ФБР? Это большая афера демократов и оправдание проигрыша на выборах!} - написал
Трамп в Twitter в июне 2017 года.

В свою очередь в Crowdstrike уверяют, что передали сервера ФБР.

\enquote{Мы предоставили всю криминалистическую экспертизу и анализ ФБР}, - заявила New
York Times пресс-секретарь CrowdStrike Илина Кашиола.

По информации Forbes, русский эмигрант Альперович вместе с Джорджем Курцом (оба
бывшие руководители McAfee) основали Crowdstrike в 2011 году. 

Наши источники в дипломатических кругах говорят, что помимо дела Байдена, у
Трампа хотят расследования на тему кибератак на сервера Демпартии, которые
могли исходить из Украины в сговоре с местной продемократической элитой.

Мол, демократы надеялись убить двух зайцев - обвинить Россию во вмешательстве,
а Трампа - в сговоре с ней. 

\enquote{Среди уничтоженных файлов могли также быть документы, связанные и с хакерскими
атаками}, - говорит источник.
