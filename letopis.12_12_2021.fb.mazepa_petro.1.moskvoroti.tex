% vim: keymap=russian-jcukenwin
%%beginhead 
 
%%file 12_12_2021.fb.mazepa_petro.1.moskvoroti
%%parent 12_12_2021
 
%%url https://www.facebook.com/permalink.php?story_fbid=1338241579979592&id=100013812876387
 
%%author_id mazepa_petro
%%date 
 
%%tags jazyk,mova,ukraina
%%title Завжди ставлю на місце московиторотих, які наївно вважають, що вони вправі щось вимагати
 
%%endhead 
 
\subsection{Завжди ставлю на місце московиторотих, які наївно вважають, що вони вправі щось вимагати}
\label{sec:12_12_2021.fb.mazepa_petro.1.moskvoroti}
 
\Purl{https://www.facebook.com/permalink.php?story_fbid=1338241579979592&id=100013812876387}
\ifcmt
 author_begin
   author_id mazepa_petro
 author_end
\fi

\#мовапромову 

Зайшов нещодавно дядько Петро (далі — ДП) в один із столичних закладів
харчування.

\ii{12_12_2021.fb.mazepa_petro.1.moskvoroti.pic.1}

Стою на лінії роздачі. Роблю замовлення. Продавці, звісно, обслуговують
державною мовою. За мною, на певній відстані, стоять чоловік та жінка (далі МІ
— московиторота істота). Вона робить замовлення російською. Продавчиня (далі —
П) обслуговує державною (переходити на іноземну вона не зобов'язана). 

Процес роздачі-отримання їжі минає без проблем, аж тут... Чи то я пропустив
початок. Чи то продавчиня не зрозуміла (чи не хотіла розуміти) узького язика.
Чи, що більш вірогідно, московиторота істота почала вимагати, щоб її
обслуговували на узькому. Одним словом, початок я прослухав, але потім вийшла
ось така розмова:

П — говоріть державною мовою;

МІ — ета ваші праблєми што ві нє понімаєтє рускава язіка. Ета ваши проблєми. Ви
далжни панімать руский язік;

ДП — Ні, це якраз ваші проблеми. Бо російська мова в нашій державі є іноземною.
Її кожен має право розуміти або не розуміти. Право, а не обов'язок. Російська
мова не має жодного офіційного статусу в Україні. Тому це винятково ваші
проблеми, а не проблеми П.

МІ — ооо, апять начілось.

Розмова відбувалася на підвищених тонах. Позбігалися охорона. У результаті МІ
швидко заткнулася і пішла мовчки цямкати свої українські страви. Мужик, який
був з нею, рота не розкривав взагалі. Можливо, він німий або теж московиторотий
(навіть не знаю, що гірше в наші дні).

ДП поївши повернувся на лінію роздачі та подякував за те, що було смачно та за
мовну стійкість. Продавчині нагадав, що згідно чинного законодавства, вона мала
рацію на  @igg{fbicon.100.percent}. Хоча, мабуть, вона це й так знала. 

\ii{12_12_2021.fb.mazepa_petro.1.moskvoroti.pic.2}

Наприкінці поспілкувався з охоронниками, які мене теж підтримали. Як виявилося,
ми спільні у думках з приводу того, що надто довго надміру толерантно ставилися
до чужої та ворожої мови на нашій землі. За це заплатили (і продовжуємо
платити) високу ціну.

P.S. Завжди ставлю на місце московиторотих, які наївно вважають, що вони вправі
щось вимагати (ті часи уже в минулому). Маю ще декілька історій у запасі, але
то вже не сьогодні.

P.P.S. Не забуваймо що перемагає той, у кого сильніший мовний стрижень. І
взагалі московиту \#маюправонерозуміти.

\#мовамаєзначення \#мованачасі \#моваважлива \#моваднкнації \#нашамова
\#мовнийпатруль \#законпромову \#державнамова \#українськамова \#українська
\#сфераобслуговування \#київ \#неоднаковомені \#різницяє \#немовчи \#російськевбиває
\#росіяворог \#російськамова

\ii{12_12_2021.fb.mazepa_petro.1.moskvoroti.pic.3}
