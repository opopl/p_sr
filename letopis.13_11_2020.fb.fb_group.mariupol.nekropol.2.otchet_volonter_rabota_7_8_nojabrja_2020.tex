%%beginhead 
 
%%file 13_11_2020.fb.fb_group.mariupol.nekropol.2.otchet_volonter_rabota_7_8_nojabrja_2020
%%parent 13_11_2020
 
%%url https://www.facebook.com/groups/278185963354519/posts/411011970071917
 
%%author_id fb_group.mariupol.nekropol,arximisto
%%date 13_11_2020
 
%%tags 
%%title Отчет о волонтерской работе в Некрополе 7-8 ноября 2020
 
%%endhead 

\subsection{Отчет о волонтерской работе в Некрополе 7-8 ноября 2020}
\label{sec:13_11_2020.fb.fb_group.mariupol.nekropol.2.otchet_volonter_rabota_7_8_nojabrja_2020}
 
\Purl{https://www.facebook.com/groups/278185963354519/posts/411011970071917}
\ifcmt
 author_begin
   author_id fb_group.mariupol.nekropol,arximisto
 author_end
\fi

\textbf{Отчет о волонтерской работе в Некрополе 7-8 ноября 2020}

\textbf{Открытия и находки}

Илья Луковенко завершил расчистку неизвестного склепа у подножия склепа
Спиридона Гофа. Правда, это не приблизило нас к разгадке назначения этого
сооружения из кирпича. Склеп? Подножие для памятника?

Елена Сугак и Maryna Holovnova сделали сенсационное открытие – оказывается, у
склепа А. Хараджаева располагается целая усыпальница! Они расчистили
известняковое основание для кованой ограды. Нашли обломки плиты с надписями
\enquote{Екатерина}, \enquote{август}... А все началось с уборки мусорной кучи у ограды
Александра Хараджаева...

Еще летом волонтеры обнаружили на этом участке прямоугольные известняковые
основания, а также обломки плит с датами и надписью \enquote{Хараджаев}. Похоже, в
усыпальнице были похоронены представители рода Хараджаевых. Только технология
надгробий оказалась неудачной – на известняковые основания клали тонкую плиту с
именами и датами, которая впоследствии разрушилась...

\textbf{Благоустройство}

Благодаря пожертвованиям благотворителей%
\footnote{\url{https://archive.org/details/13_11_2020.fb.arximisto.mrpl_nekropol.pozhertvovania_otchet_1}}
мы закупили и посадили клубни мускарей вокруг склепа Спиридона Гофа, а также
начали покраску оград и крестов безымянных захоронений и подготовку ям для
посадки саженцев клена.

Большое спасибо всем волонтерам за самоотверженную работу!

Планы на выходные 14-15 ноября

На этих выходных мы планируем:

\begin{itemize}
  \item Покрасить оградки \textbackslash\ кресты на древнем участке.
  \item Выкопать ямы для деревьев и, возможно, их посадить.
  \item Восстановить разваленные советские памятники.
  \item Продолжить разбор мусорных куч возле склепа Александра Хараджаева, городского головы в 1860-64 гг.
  \item Если успеем – сделаем разметку для прокладки дорожек из агроволокна и гравия.
\end{itemize}

Мы приглашаем всех желающих присоединяться к волонтерам! Мы работаем в субботу
и воскресенье с 10 утра. Встреча – в центре Некрополя, у белого Памятного
креста. Возьмите обязательно маски и перчатки. Контактный телефон – 096 463 69
88.

Мы не публикуем отдельный анонс о волонтерской экспедиции на этих выходных
из-за введения \enquote{локдауна выходного дня}. Надеемся, что вскоре ограничения будут
четко понятны...

До встречи в Некрополе, друзья!

\#mariupol\_necropolis\_report
