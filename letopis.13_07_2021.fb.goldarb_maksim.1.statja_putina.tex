% vim: keymap=russian-jcukenwin
%%beginhead 
 
%%file 13_07_2021.fb.goldarb_maksim.1.statja_putina
%%parent 13_07_2021
 
%%url https://www.facebook.com/maksim.goldarb/posts/2669927809983072
 
%%author Гольдарб, Максим
%%author_id goldarb_maksim
%%author_url 
 
%%tags __jul_2021.putin.statja,odin_narod,rossia,rusmir,ukraina
%%title СТАТЬЯ РОССИЙСКОГО ПРЕЗИДЕНТА: очевидные спектры
 
%%endhead 
 
\subsection{СТАТЬЯ РОССИЙСКОГО ПРЕЗИДЕНТА: очевидные спектры}
\label{sec:13_07_2021.fb.goldarb_maksim.1.statja_putina}
 
\Purl{https://www.facebook.com/maksim.goldarb/posts/2669927809983072}
\ifcmt
 author_begin
   author_id goldarb_maksim
 author_end
\fi

СТАТЬЯ РОССИЙСКОГО ПРЕЗИДЕНТА: очевидные спектры.

1. Не лирика, не эмоции, не крик души, а понятно обозначенная и
аргументированная государственная программа, касающаяся защиты национальных
интересов.

2. Как всё, что он делает, появилась в свой срок, в своё время (в тч с
необходимо выдержанной классической паузой после встречи с американским
коллегой).

3. В который раз очерчивает одну из ранее озвученных им «красных линий» -
Украина не должна быть использована как угроза интересам России. А именно так
её сейчас воспринимают, и созданная модель - «анти-Россия» - неприемлема и
должна быть аннулирована (о механизмах позже): «курс … на формирование
…государства, агрессивно настроенного к России, по своим последствиям сравним с
применением против нас оружия массового поражения».

4. Для этого, в тч, оснований для пересмотра «волюнтаристских территориальных
манипуляций», результатов «произвольной перекройки границ» - более чем
достаточно.

5. Исторических, духовных, глубинных, экономических, опытных, выгодных
оснований для сближения многократно больше, чем для разрыва и вражды. Сближение
- естественно и выгодно для народов, разрыв - искусственно создан и явно
противоречит их исконным интересам. 

6. Отношение к украинскому коллеге: не интересен, не вызывает уважения и
симпатии и желания встречаться и общаться. В тч из-за лжи («обещания оказались
враньём»), полной зависимости и отсутствия субъектности. 

7. Цели и задачи власти и народа у нас - диаметрально противоположные. Именно
власть «растранжирила, пустила по ветру достижения многих поколений».

8. Нескрываемо жёсткое, отрицательное отношение к украинским олигархам, как к
врагам страны, «ограбившим народ Украины, …, готовым продать мать родную, чтобы
сохранить капиталы». 

9. Для запада Украина всегда (и в ретроспективе, и сегодня) - лишь инструмент
создания проблем для России плюс ресурсы. Люди неинтересны совершенно. Так
было, так есть, так, не дай Бог, и будет. До последнего …ресурса. 

10. В отличие от запада, «подлинная суверенность Украины возможна именно в
партнёрстве с Россией». 

11. Окончательные выбор и решение будут за украинским народом, не за властью.
Судя по лейтмотиву статьи, в обозримо близкое время. И механизм и право выбора
ему обеспечат. 

Ps: специально для провокаторов, кликуш, лжепатриотов, «зелёных слуг»,
спецслужб - приведена выжимка основных очевидных месседжей. Для особо
непонятливых - с цитатами

\ii{13_07_2021.fb.goldarb_maksim.1.statja_putina.cmt}
