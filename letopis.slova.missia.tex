% vim: keymap=russian-jcukenwin
%%beginhead 
 
%%file slova.missia
%%parent slova
 
%%url 
 
%%author 
%%author_id 
%%author_url 
 
%%tags 
%%title 
 
%%endhead 
\chapter{Миссия}

%%%cit
%%%cit_pic
%%%cit_text
Однако надо понимать, что эти воззрения у Розенберга появились еще когда он в
нацистском госаппарате не работал, а был только функционером НСДАП. То есть
тотальная война еще не была в повестке дня. Когда же он возглавил министерство
по делам восточных территорий нацистской Германии, эти планы даже близко не
подошли к своей реализации.  Причина была в том, что сам Адольф Гитлер никаких
марионеточных государств на территориях Украины создавать не собирался. Он
видел свою \emph{миссию} в прямом завоевании украинских территорий и заселении их
немцами. 
%%%cit_comment
%%%cit_title
\citTitle{22 июня - 80 лет нападения на СССР. Что немцы готовили для украинцев}, 
Максим Минин, strana.ua, 22.06.2021
%%%endcit

%%%cit
%%%cit_head
%%%cit_pic
%%%cit_text
Этот проект — одна из немногих возможностей заглянуть на окраины Вселенной, уж
насколько их можно так назвать. Пусть снимки чёрно-белые и низкого качества —
но можете ли вы себе представить объект в миллиардах световых лет от нас? Лично
мне Луна уже кажется чем-то недостижимым. А ведь до неё свет добирается всего
за 1,3 секунды. Неизвестно, долетит ли когда-нибудь человечество до самых
удалённых галактик, но полюбоваться ими можно уже сейчас.  И потом не стоит
забывать про амбициозную \emph{миссию} Dark Energy Explorers. Каждый волонтёр проекта
— это исследователь космоса, который помогает составить самую масштабную карту
Вселенной. Причём для этого даже не нужно покидать пределы своей комнаты.
Получается, когда учёные всё-таки разгадают природу тёмной энергии, каждый
участник Dark Energy Explorers сможет сказать: «В этом есть и моя заслуга». И
это будет чистой правдой
%%%cit_comment
%%%cit_title
\citTitle{В поисках тёмной энергии | Наука и жизнь}, 
Владислав Цой, www.nkj.ru, 03.06.2021
%%%endcit

