% vim: keymap=russian-jcukenwin
%%beginhead 
 
%%file 24_08_2021.fb.zaharova_maria.mid_rf.1.pvrvotu
%%parent 24_08_2021
 
%%url https://www.facebook.com/maria.zakharova.167/posts/10226720042696882
 
%%author Захарова, Мария (Россия)
%%author_id zaharova_maria.mid_rf
%%author_url 
 
%%tags kiev,krym,krymskaja_platforma,politika,rossia,ukraina
%%title ПВРВОТУ
 
%%endhead 
 
\subsection{ПВРВОТУ}
\label{sec:24_08_2021.fb.zaharova_maria.mid_rf.1.pvrvotu}
 
\Purl{https://www.facebook.com/maria.zakharova.167/posts/10226720042696882}
\ifcmt
 author_begin
   author_id zaharova_maria.mid_rf
 author_end
\fi

ПВРВОТУ.

В своем извращенном понимании национальной политики нынешний киевский режим
пошел в каком-то смысле даже дальше тех, кому сейчас устанавливает памятники.

Прикрываясь словами о защите народов, он поразил в правах целые национальности
и народности. Речь о законе «О коренных народах Украины», одном из самых
вопиющих актов этногражданской сегрегации со времен законов Джима Кроу.

Поставив свою визу под текстом документа, Зеленский юридически лишил испокон
веков населявших Украину русских, венгров, греков, евреев, белорусов и т.д.
целой массы прав, традиционно предоставляемых коренным народам: создавать свои
представительные органы, использовать национальную символику, иметь свои СМИ,
получать помощь от государства и из иностранных источников, участвовать в
международной деятельности. Все потому, что в Киеве решили наказать ни в чем не
повинных людей за ошибки предыдущих украинских правителей. Законодательная
власть абсурдно ссылается при этом на то, что у коренных народов Украины якобы
не может быть своего национального государства. Откуда это взяли теоретики
национализма по-киевски? Ни в Декларации ООН о правах коренных народов 2007 г.,
ни в других международно признанных документах ничего такого нет.

Впрочем, ответ очевиден: цель закона – дать ручным, финансово пригретым рупорам
проукраинской пропаганды возможность и дальше официально поддакивать Киеву по
вопросу о территориальной принадлежности Крыма: всем этим сидящим на грантах
псевдогуманитарщикам. Вторая задача - дальше не замечать, как не на словах, а
на деле защищаются и развиваются культуры различных национальностей в
российском Крыму. 

Очевиден и тот факт, почему закон появился именно сейчас – на носу ведь был
главный международно-политический шабаш года – «Крымская платформа». Что у
киевских идеологов к нему было готово, кроме заезженных лозунгов и нежелания
выполнять международные обязательства? Теперь же вроде как появился повод
похвалиться перед патронами - «защитой коренных народов Украины». 

Интересно, что в законе «коренные народы», о которых Киев теперь так печется –
прямо перечислены, и все они живут в Крыму. Что мешало выражать о них заботу,
хотя бы формально ранее, неизвестно. Видимо, до изобретения «платформы» тема не
была столь животрепещущей, чтобы обращать на неё внимание Рады и Президента.
Возможно, только потеряв территории по итогам волеизъявления граждан,
политическая элита Украины почувствовала некий дискомфорт и решительно
принялась за дело, лишь усугубляя ситуацию. Ведь для исправления ошибок,
совершенных в национальной политике, надо было гармонизировать отношения между
народами, а не заряжать катапульту очередным гигантским булыжником
национального раздора, целясь по и без того затравленным гражданам. 

Глупо и наивно полагать, что очередная подписанная украинским Президентом
националистическая береста (которая к тому же противоречит и международной
правозащитной практике) сможет что-то изменить в лучшую сторону. Ну да,
заморских гостей ею почетно встретили, а после того как все разъедутся, Украине
с этим жить. Так же, как и со всеми бесконечными реляциями и заявлениями
Министерства с сюрреалистичным названием – «по вопросам реинтеграции временно
оккупированных территорий Украины». С другой стороны, аббревиатура у ведомства
подходящая - ПВРВОТУ. 

Пока в Киеве не поймут, что «никакие связи не помогут сделать ножку маленькой,
а сердце большим», ничего не изменится. Пора выучить урок: Крым - российский, а
населяющие Украину национальности и народности надо любить одинаково, не деля
на коренные и с государствами.  

А то, что делают сейчас на Украине только отталкивает подавляющее большинство
граждан и укрепляет болезненное мировоззрение ультранационалистов. Хотя,
казалось бы, после того, что было сделано, куда уж больше.

И по-братски один совет ЗеПрезиденту: хватит уже позориться с проведением
мероприятий в честь зарубежных населённых пунктов, типа Ялтинской конференции и
«Крымской платформы».

Что дальше? В Киеве пройдут заседание Севастопольской обсерватории по правам
человека, Алуштинский симпозиум коренных народов, Коктебельская вечеринка в
честь джазового фестиваля в Коктебеле или Бахчисарайский конгресс фонтанов им.
Олександра Пушкина? Лучше развивайте мероприятия в городах Украины с
одноимёнными названиями, например, в Одессе или Харькове. Хотя, теперь тоже не
очень звучит - их же основали не коренные, как выясняется.
