% vim: keymap=russian-jcukenwin
%%beginhead 
 
%%file slova.komanda
%%parent slova
 
%%url 
 
%%author 
%%author_id 
%%author_url 
 
%%tags 
%%title 
 
%%endhead 
\chapter{Команда}
\label{sec:slova.komanda}

%%%cit
%%%cit_head
%%%cit_pic

\ifcmt
  tab_begin cols=3
   	 width 0.3

     pic https://storage.lug-info.com/cache/5/e/583deb92-0e9e-4833-a8ab-5a80b96584d9.jpg/w1000h616
		 width 0.24

     pic https://storage.lug-info.com/cache/4/4/b0618e81-14da-4138-9ff1-1a1e5507f58c.jpg/w1000h616
		 width 0.28

		 pic https://storage.lug-info.com/cache/3/4/cbcfc12d-37a4-45ff-bfdb-3b82835fca07.jpg/w1000h616
		 width 0.2

  tab_end
\fi

%%%cit_text
\emph{Команда} из Ровеньков стала призером военно-исторического форума
\enquote{Александровский стяг}, который состоялся в Санкт-Петербурге. Об этом
ЛИЦ сообщил председатель общественной организации \enquote{Патриотическая
ассоциация Донбасса} (ПАД) Алексей Селиванов \enquote{В Санкт-Петербурге
завершился ежегодный, одиннадцатый по счету, Всероссийский молодежный
военно-исторический форум \enquote{Александровский Стяг} имени святого
благоверного князя Александра Невского. Традиционно в форуме приняли участие
команды Народных Республик Донбасса, ЛНР представили \emph{команды} ПАД и
\enquote{Гвардия} Ровеньковской гимназии №1 имени Шонина}, - сказал Селиванов
%%%cit_comment
%%%cit_title
\citTitle{Луганский Информационный Центр – Команда из Ровеньков стала призером
военно-исторического форума \enquote{Александровский стяг}}, , lug-info.com,
04.07.2021
%%%endcit

%%%cit
%%%cit_head
%%%cit_pic
\ifcmt
tab_begin cols=2
	width 0.4
  pic https://strana.ua/img/forall/u/0/92/%D1%80%D0%B0%D0%B7%D0%B3%D1%80%D1%83%D0%B7%D0%BA%D0%B0(1).jpg

	pic https://strana.ua/img/forall/u/0/34/207941474_2503568363110874_3886265068086152222_n.jpg
tab_end
\fi
%%%cit_text
Выйдя из поезда Петр Алексеевич весь в черном и \emph{команда} направились в
новоиспеченный Нью-Йорк (переименованный поселок Новгородское). По дороге он
заехал в штаб 10-й горно-штурмовой бригады и передал бригадный комплекс, сетку
и пять генераторов, упомянув, что купили их в \enquote{Розетке} с большой скидкой.  На
месте лидер \enquote{Евросолидарности} показал, как умеет организовывать работу и стал
командовать выгрузкой свертков и коробок из машины
%%%cit_comment
%%%cit_title
\citTitle{Гетьман в Нью-Йорке. Как Порошенко устроил черешня-тур на Донбасс}, 
Оксана Малахова, strana.ua, 06.07.2021
%%%endcit

%%%cit
%%%cit_head
%%%cit_pic
%%%cit_text
І в цьому головна роль належить найбільш авторитетній людині в \emph{команді} –
її головному тренеру Андрієві Шевченку. Якщо би він повірив, що спільне
прийняте рішення (наприклад, щодо вивчення української мови) може якісно
збільшити згуртованість та ефективність \emph{команди}, та прийняв відповідне
рішення, ми б мали якісно кращу \emph{команду} та якісно кращий футбол на
чемпіонатах, причому вже за декілька місяців!  Залишається побажати
\emph{команді} прислухатись до цієї пропозиції! Ну, й зараз якнайшвидше
відновитись після виснажливого марафону, проаналізувати отриманий досвід,
провести роботу над помилками і почати готуватись до нових викликів!
%%%cit_comment
%%%cit_title
\citTitle{Чи можна примусити збірну України розмовляти державною мовою?}, 
Євген Лапін, www.pravda.com.ua, 06.07.2021
%%%endcit

%%%cit
%%%cit_head
%%%cit_pic
%%%cit_text
Диктор: На поле выходят \emph{команды}-соперники.  В черных футболках с
двуглавым орлом, синих трусах и красных гетрах - \emph{команда} ДНР.  В
желто-синих вышиванках, красных шароварах и черных сапогах гармошкой -
\emph{команда} Украины. На их футболках изображён глобус Украины. У каждого
"жовто-блакитного" в руках нагайка, на лице - ухмылочка.  На поле появляется
главный судья в белой каске с надписью "ОБСЕ".  Зрители волнуются
%%%cit_comment
%%%cit_title
\citTitle{Футбольный матч ДНР - Украина: предполагаемая модель игры команд},
Музей города N (Некропоткина), zen.yandex.ru, 11.06.2021
%%%endcit

%%%cit
%%%cit_head
%%%cit_pic
%%%cit_text
Когда я учился в старших классах, в нашей школе была \emph{команда} по алгебре,
которая состояла из пятерых ребят. Мы ездили в разные школы и участвовали в
математических конкурсах. Мы садились в один ряд, другая \emph{команда} - в
другой.  Учительница, проводившая конкурс, доставала конверт, на котором было
написано "сорок пять секунд". Она открывает его, пишет задачу на доске и
говорит: "Начали!", так что на самом деле времени было больше, потому что можно
было думать, пока она пишет. Игра заключалась в следующем. У Вас есть лист
бумаги, на котором Вы можете написать все, что угодно, и сделать все, что
угодно.  Считался только ответ. Если ответ был "шесть книг". Вы должны были
написать "6", и обвести цифру в кружочек. Если цифра в кружочке правильная. Вы
выигрывали; если нет - проигрывали. В одном можно не сомневаться. Не было
никакой возможности решить задачу простым традиционным способом, приняв,
например, что "А - это количество красных книг, В - количество голубых книг",
ширк, ширк, ширк, пока не получится "шесть книг". На это вас ушло бы пятьдесят
секунд, потому что люди, которые назначали время на решение этих задач, всегда
давали немного меньше времени, чем требуется. Так что приходилось думать: "Есть
ли какой-то способ увидеть решение?" Иногда это получалось в мгновение ока,
иногда приходилось искать другой путь и максимально быстро выполнять
алгебраические действия. Это была изумительная практика, у меня получалось все
лучше и лучше, и в конце концов я возглавил \emph{команду}. Вот так я научился
очень быстро решать алгебраические задачки, и в колледже алгебра давалась мне
легко. Когда бы мы не встречались с задачкой на вычисление, я очень быстро мог
увидеть, к чему идет дело, и выполнить нужные алгебраические операции - просто
моментально
%%%cit_comment
%%%cit_title
\citTitle{Вы, конечно, шутите, мистер Фейнман!}, Ричард Фейнман
%%%endcit

%%%cit
%%%cit_head
%%%cit_pic
%%%cit_text
Вихованка туристичного гуртка Ощівської гімназії Анна Дмитрук у складі
\emph{команди} Горохівської територіальної громади зайняла ІІ місце на ХХІ
обласних особисто-\emph{командних} змаганнях зі спортивного орієнтування серед
учнівської та студентської молоді «Волинська осінь», що відбуваються під егідою
Волинського центру національно-патріотичного виховання, туризму і краєзнавства,
повідомляє сайт Горохівської міської громади. Змагання проходили у Ковельському
районі біля села Городилець.  За повідомленням керівника туристичного гуртка,
тренера \emph{команди} Віталія Новосада \emph{команда} Горохівської громади загалом зайняла 8
сходинку серед сімнадцятьох \emph{команд} Волинської області
%%%cit_comment
%%%cit_title
\citTitle{Юна горохівчанка привезла нагороду зі спортивних змагань}, 
, www.volyn.com.ua, 30.10.2021
%%%endcit

%%%cit
%%%cit_head
%%%cit_pic
\ifcmt
   pic https://avatars.mds.yandex.net/get-zen_doc/1244179/pub_61765bdfb5e72065df124fa2_61766cf4b5e72065df2695c7/scale_1200
	 @width 0.4
\fi
%%%cit_text
Мир разделился на две \emph{команды}. Это не сейчас произошло. Мы перетягиваем этот
канат влияния Цивилизаций из века в век. Участвуем в совместной деятельности,
так сказать, с переменным успехом. Партнерствуем.  На одной стороне
\emph{команда} Запада - "Люди "Пи". \emph{Команда} не имеет сплочённости.
Каждый из игроков может подставить другого в любой момент. Но канат тянуть
надо, потому что иначе они свалятся в пропасть. А там, в пропасти, с их точки
зрения, нарезает круги голодный и злой медведь. Зубами клац-клац... И вот так
упадешь к нему в когтистые лапы и конец толерантным "ценностям". И всему конец.
Что-то страшное произойдет. Что произойдет они и сами толком не знают. Не могут
сформулировать, а чем им так плоха Россия, как друг. Просто экзистенциальные
Страх и Трепет. И вековая Ненависть.  На другой стороне этого каната
\emph{команда России}. И в ней тоже нет сплочённости. К тому же она, к
сожалению, малочисленна. Кто в нее входит? Беларусь да Сербия? И те и другие
могут плюнуть на все и уйти в любой момент. Одна в подполье, вторая на ту
сторону
%%%cit_comment
%%%cit_title
\citTitle{Люди "Пи"}, Волар, zen.yandex.ru, 25.10.2021
%%%endcit

%%%cit
%%%cit_head
%%%cit_pic
%%%cit_text
Сборной Украине по шахматам, которая недавно победила на Чемпионате Европы в
Словении, власти не оплатили проживание.  Наша \emph{команда} оказалась
единственной такой на чемпионате, который, несмотря на «подставу» она выиграла,
завоевав золотые медали, Об этом рассказал украинский гроссмейстер Антон
Коробов.  «Наше родное министерство сделало просто гениальный ход конем.
Считаю, за это нужно сразу вручать звание Героя Украины. Украина – единственная
\emph{команда}, которая не смогла заплатить за проживание. Мы находились в
подвешенном состоянии: в любой момент нас могли выгнать на улицу. Когда ты не
уверен, будет ли у тебя завтрак, будет ли крыша над головой, то чувствуешь себя
совершенно расслабленно», - с иронией рассказывает Коробов в интервью
YouTube-каналу My.ua
%%%cit_comment
%%%cit_title
\citTitle{Родная власть сделала бездомной сборную Украины по шахматам / Лента соцсетей / Страна}, 
Евгений Хурсин, strana.news, 30.11.2021
%%%endcit
