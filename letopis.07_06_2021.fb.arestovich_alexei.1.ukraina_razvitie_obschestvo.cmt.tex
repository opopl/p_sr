% vim: keymap=russian-jcukenwin
%%beginhead 
 
%%file 07_06_2021.fb.arestovich_alexei.1.ukraina_razvitie_obschestvo.cmt
%%parent 07_06_2021.fb.arestovich_alexei.1.ukraina_razvitie_obschestvo
 
%%url 
 
%%author 
%%author_id 
%%author_url 
 
%%tags 
%%title 
 
%%endhead 
\subsubsection{Комментарии}
\label{sec:07_06_2021.fb.arestovich_alexei.1.ukraina_razvitie_obschestvo.cmt}
\begin{itemize}

\iusr{Юрий Ряснянский}

\enquote{шоб жёсткая политика и принуждение, но меня не коснулось})

\iusr{Dmitriy Black}

Мало найдётся реформаторов которые работали в то время как над ними нависала
огромная внешняя угроза... Которой еще и симпатизирует минимум треть населения.

\iusr{Сергій Чаплигін}

Люблю соцреалізм. Як же нам не вистачає естетичного вираження усвідомленої
концепції світу і людини, обумовленою епохою боротьби за встановлення і
творення нового суспільства.

\iusr{Lev Malkhazov}

Политика иррациональна по единственной причине. Она призвана обслуживать
интересы коллективного бессознательного (КБ. Но это статистически. Появляется
личность, которая обладает достаточной мощью индивидуального сознательного
(ИС), чтобы провести в жизнь рациональные идеи. Но со временем масса КБ
начинает выдавливать их силой своей инерции. После условного Гавела приходит
условный Земан, грубо говоря. Порочный круг разорвать можно, но это долгая
кропотливая история массового развития культуры, образования, науки и
приобщения к искусству. Для этого тоже нужен мощный субъект, который выстроит
программу замены в каждом человеке КБ на ИС.

\iusr{Paschynskaya Irina}

Налево пойдешь, диктаттра найдешь. Направо пойдешь, диктатурой золотого тельца
поглощен будешь. Прямо хоть иди, хоть беги левые или правые своими мощными
катками раздавят.  И все, кто предлагают коврижку, на самом деле норовят
накинуть узду и приставить к попе руль.  Что-то вкорне не так. Не про человека
и человечность.

\iusr{Скрипка Вера}
\textbf{Ирина Пащинская} Политика это про власть и контроль.

\iusr{Sergii Strikhar}

Как говорится, пока толстый похудеет - худой помрет. Может получится так, что
тем изменений, котрые удастся синхронизировать с темпом приспособленности, не
будет адекватен вызовам сегодняшнего дня, и темпу развития других. Вот и
получится (и уже получается по факту) - вроде и движемся вперед, но при этом
все больше и больше отстаем.

\iusr{Олексій Арестович}

\textbf{Sergii Strikhar} поэтому нужно определить точки роста + критические
факторы, и там темпы изменений должны быть не плановыми, а экстраординарными.

\iusr{Sergii Strikhar}

\textbf{Олексій Арестович} Но ведь по факту это будет то самое навязывание
повестки - а точки роста - только геометрически маленькие, на самом деле речь
идет ведь о глобальных вещах. Вы упоминали патернализм - но это как раз и есть
одна из точек роста. Население стареет и убывает - а значит у государства все
меньше и меньше ресурсов для поддержки левых идей. Поэтому - даже в точках
выбора, возможно, довольно болезненное навязывание политики. Другое дело, что с
этим нужно работать, а не шашкой махать.

\iusr{Олексій Арестович}
\textbf{Sergii Strikhar}

Но не будет провала в авторитаризм.

\iusr{Иван Кожевников}

\textbf{Алексей Арестович} Допустим, фундаментальная либерализация налогового
законодательства.

\iusr{Кирилл Подгорный}

\textbf{Иван Кожевников} и так никто нихрена не платит, какая либерализация, не смешите.

\iusr{Иван Кожевников}

\textbf{Кирилл Подгорный} вот по этому никто нихрена и не платит.

\iusr{Кирилл Подгорный}
\textbf{Иван Кожевников} не поэтому.

\iusr{Кирилл Подгорный}
\textbf{Иван Кожевников} Нет, конечно

\iusr{Georgy Polo}

\textbf{Алексей Арестович} //Но не будет провала в авторитаризм// Это вы кого
сейчас обманываете, нас или себя. Вы же уже запустили авторитаризм, но решили
его завуалировать не под одного человека, а под группу людей(СНБО).  Большие
реформы в стране и в обществе, которые на дне, всегда возможны только благодаря
авторитаризму

\iusr{Иван Кожевников}

\textbf{Кирилл Подгорный} чем меньше налоговая ставка тем больше поступлений в
бюджет.  Изучи кривую Лаффера и опыт стран с низким налогообложением.

\iusr{Діжечко Максим}
Класс

\iusr{Юлія Максименко}

Не только политика... но и в принципе вся жизнь иррациональна в своей основе ))))

\begin{itemize}
\iusr{Олексій Арестович}
Юлія Максименко в отношении реформирования собственной жизни, у народа и интеллектуалов, все то же самое.)

\iusr{Юлія Максименко}
Олексій Арестович что в малом то и в большом типа?)

\iusr{Mike Wind}

Олексій Арестович колись на вступних екзаменах, професор казав мені, юному:
\enquote{п'ятірки не самоціль, головне, юначе, тримайте ритм, щоби не випасти з сідла}.
я послухав і таки поступив

\iusr{Ольга Олхадеси}
\textbf{Алексей Арестович} Чтобы носить очки, мало быть умным, надо ещё и плохо видеть.

\iusr{Dmitriy Grabina}
\textbf{Ольга Кондратюк} Мало плохо видеть - необходимо испортить зрение чтением....

Мало испортить зрение чтением - надо натренировать мысль размышлениями.
Мало натренировать ... .и.т.д.

\iusr{Igor Yevtushenko}
\textbf{Юлія Максименко}
Жизнь многогранна и многолика.)
При определенных сочетаниях иррационального и рационального, она, жизнь, имеет 50 оттенков гармонии.
Взгляды на гармонию меняются в процессе жизни и зависят от:
1. самомотивации;
2. возраста;
3. от позиции:
живущий,
наблюдающий за живущим,
наблюдающий за наблюдающим за живущим)).
\end{itemize}

\iusr{Andy Vedder}

Демократия хороша только в осознанном и сознательном обществе, во всех других
вариантах Демократия - зло, тормоз развития. Поэтому согласен с Алексеем, на
первых порах только диктатура, а затем когда уровень народонаселения подойдет к
осознанности, можно и в Демократию играться.

\iusr{Andriy Malyarenko}

\enquote{Нынешняя молодежь мало борется, мало уделяет внимания борьбе, нет у нее
стремления бороться больше, бороться за то, чтобы борьба по-настоящему стала
главной, первоочередной задачей всей борьбы, а ведь если она, наша чудесная,
талантливая молодежь, и дальше будет так мало бороться, то в этой борьбе у нее
останется немного шансов стать настоящей борющейся молодежью, всегда занятой
борьбой за то, чтобы сделаться настоящим борцом, который борется за то, чтобы
борьба...} 

\iusr{Panzerkampf Süd-Ukrainischer}

Не могли бы ли вы, если вам это будет угодно, раскрыть тему \enquote{открытого
общества}? Что именно ВЫ вкладываете в это понятие?

\iusr{Наталка Бойко}

Тот случай, когда всю жизнь топишь за демократию равенство и братство, а
получив наконец полноту власти понимаешь, что НИКОГДА не будешь мил всем.  И
выслушивая мнения того самого народа, которому по конституции принадлежит
власть - понимаешь, что демократия - зло)) и принимать важные и полезные
решения только мешает.

\iusr{Слава Сухой}

Точки роста надо не только определять, но так же и давать им появляться самим
по себе, инициативно. И эти, которые появятся сами, удивят всех.

\iusr{Георгій Тіто}

А почему Вы считаете что приучать нужно народ и меняться должен народ?,а
менеджеры которых этот народ выбрал меняться и приучаться работать на страну и
на народ не желает?, а экономику эти менеджеры не желают взять и пустить на
благо страны и тогда может коллективное бессознательное и менять не придется не
по каким принципам?

\iusr{Игорь Бордун}

Подкину такую идею-следующая цивилизация должна быть под Новыми открытиями в
науке , в сфере религии , то есть - теократия как в Израиле было ,до первого
царя , народ не захотел быть под властью Бога и попросил \enquote{дай нам царя}
,теперь пойдет от обратного!

\iusr{Georgy Polo}

Весь мир идёт к лево-патерналистской картине политики и экономики, а Украина,
как и СССР в своё время, своей дорогой идёт, только теперь на право. Посмотрим
какой возможно построить либеральный капитализм в экономике, которая
автоматизируется и постоянно лишает граждан работы, которую найти(нормальную)
всё труднее и труднее. Любой нормальный бизнесмен, как только у него появиться
возможность заменить людей на технику, сразу это сделает. Что просто так, уже
вон даже в России заговорили о базовом доходе и четырехдневной рабочей недели.
Интересно как и как долго в стране будет существовать правый либеральный
капитализм, при большом количестве безработных, где государство людям ничего не
должно.

\ifcmt
  pic https://scontent-cdt1-1.xx.fbcdn.net/v/t1.6435-9/194360360_3518001954967040_6571866939610289233_n.jpg?_nc_cat=106&ccb=1-3&_nc_sid=dbeb18&_nc_ohc=mecbtqFCrLcAX_nM0f_&_nc_ht=scontent-cdt1-1.xx&oh=4a5d9bef2abc4c1f71835dd88156676a&oe=60E44FE7
	width 0.3
\fi

\iusr{Roman Perekhrest}

\enquote{Так что Вы выбираете?}

Любопытный вышел бы опросник, с любопытными результатами.

\iusr{Юлія Максименко}

\enquote{- выигрывает тот, кому удалось синхронизировать темп изменений с темпом
приспособления к ним. Если темп изменений низкий, вы отстаете. Если темп
изменений слишком высокий, а темп приспособления низкий, то у вас происходит
революция или иные катаклизмы.}!!!

\iusr{Слава Сухой}

Рациональность это способность просчитывать причинно-следственные связи и с
помощью расчета принимать те решения, которые приведут к желанному исходу. Но
то, что то, что видится желанным на три шага вперед, может на шестом обернуться
ужасом, а на восьмом опять окажется, что все хорошо и так далее. И чем дальше
по цепочке, тем сильнее расчет отличается от реального хода событий, просто в
силу их хаотичности. Поэтому, надо планировать на два-три шага вперед, не
дальше, после каждого шага пересчитывать последствия и вносить корректировки.
Или, если Вам по нраву ссылки на авторитеты, \enquote{планы - ни что, планирование -
все} (Д. Эйзенхауэр).

\iusr{Georgy Polo}

Перепрограммирование народа, возможно силой и страхом и пропагандой, а на
банальный подкуп у власти просто нет денег. Чем лучше и обширней пропаганда,
тем меньше надо будет силы и страха. Для этого надо взять под контроль все
основные сми, легче и быстрее нагнуть журналистов в правильном направлении, чем
народ. Тем более Украинской свободой слова, пользуются от силы 2\% в стране и
то в основном для личного обогащения, а для остальных она выражается в том,
чтобы просто кого то громко оскорблять и смотреть как другие, друг друга
оскорбляют на тв.

\ifcmt
  pic https://scontent-cdg2-1.xx.fbcdn.net/v/t1.6435-9/194238353_3517924574974778_638259040369246596_n.jpg?_nc_cat=108&ccb=1-3&_nc_sid=dbeb18&_nc_ohc=4I6sgNTc--QAX92wpkR&_nc_ht=scontent-cdg2-1.xx&oh=97ec453d58c90198af74bb5f4d2241de&oe=60E3BB99
	width 0.2
\fi

\iusr{Павел Морьев}

...приучение народа к открытости общества - звучит парадоксально!

\iusr{Игорь Бордун}

Вы шутите ? у нас есть выбор? Все придем к жесткому электронному контролю за
личностью и разделению на бедных и богатых, только соотношене будет меняться
от независящих от меня причин

\iusr{Elena Mudretskaya}

Бытие определяет сознание. Коллективное Сознание формирует пространство, которое управляет людьми…
Что делать? Вечный вопрос

\iusr{Yury Lyubitsky}

Если вы не читаете английские журнал \enquote{Ланцет}, ну хотя бы а переводе на
украинский или русский то информативность картинки вам достаточно; айм сорри)

Алексей, не сочтите за сарказм, но вы основной тригер внимания к общественной
жизни на сегодняшний день! Огромное Вам спасибо!

\iusr{Скрипка Вера}

Интересно.

\iusr{Iuri Shcherbak}

И с высоты моей вчёности люди кажутся как тее миши, пардоньте, крисы!

\iusr{Ольга Олхадеси}

\enquote{Иванушка-дурачок за заслуги перед отечеством был признан умным! Посмертно...}

\iusr{Ignat Kotko}

Не могли бы вы объяснить ему, что у НАТО есть такой документ как Устав, что
ответственность за оборону страны несёт именно он, а не Германия, не Франция,
не Байден, что работать надо уметь, что надо было учиться думать, что надо
трезво оценивать свои силы. Впрочем, это бесполезно. Артист периферийной
самдеятельности за пределы этой оболочки не выйдет. Страну, конечно, жалко.

\iusr{Наталья Хохлачева}

К кто ж навяжет повестку дня пиначетам.... СБУ, ГПУ.....НП, ВАКсам, и таке
иншим.......  Решения в местных органах власти на соответствие Конституции и
законам Украины не пробовали аудировать и контролировать ? Или заявления о
преступлениях в НП ?  Проверьте, может и пиначет не понадобиться, или хотя бы
будете знать кого пиначетить.........

\iusr{Юрій Ярошенко}

Ну вже доведено демократії з елементами прямої, більш успішні. більш раціональні, то навіщо нам йти повільними темпами ? Може лише тому що ми дуже дурні ? Ні не лише народ а і влада, котру він обирає і годує.

\iusr{Громадянин Марифорнии}

Сравнили Зеленского и Ли Куан Ю :)  Помнится, Ю даже свою родню прессовал, в то
время как Зеля всю свою родню во все структуры и ведомства распихал.

\iusr{Natalia Omelianchuk}

В перекладі - це ми до зеленої шмарклі не доросли?

\iusr{Макс Сидоренко}

Як поминять? Гроши і власть ..Це ж пиздец таможники вся сімья і родичи..судді
вся сімья і родичи..і комісії в виборчи таке саме дінастіїї на різних фаміліях
но рахують на тих самих дільницах вони живуть взятками от виборив до
виборив.люди прости тут непричом

\iusr{Alexander Komarenko}

Да патому шо, этих советчиков развелось, как собак нерезанных, один другого умнее

\iusr{Макс Сидоренко}

Нащо придумувать пока елита не схоче щоб їх внукі жили в Україні ,то нещо не
зупинить...мінять вирощувать нову еліту?уже революцій скільки було і ті
самі?тому що елиту сформував кучма-могилевич це порошенка банда медведчука
банда януковича і яценюка банда а також тимршенко це всі мафіозі ...вони тільки
красти бюджет і вміють.а також клани Суддів -сімей .таможників -сімей. І
генералів-сімей .фу бридкі вори...кучму порвать і усіх аферистів-репортерів і
карманну опозицію...нема Конкуренції .нема кадрів....нехто не сів в тюрму..у
них наказанія це Отправка на пенсію в Ізраїль..що прости люди тупи?передай
зеленськом що так уже всі говорять .хай знае..

\iusr{Alexey Shershnev}

Ну выборы ж выиграны в 2019 именно на таком нарративе (якобы отставания) - так
шо теперь кому то надо притормаживать, кому-то ускоряться )))

\iusr{Yevhen Sereda}

Цикличное проигрывание и хождение по кругу, за все тридцать лет независимости,
показало что люди , в большинстве , в принципе не думают, и реагируют только
эмоционально. И это не меняется. Пока во главе всего инертность,
безответственность и безрассудность, ничего глобально позитивного ждать не
приходится.

\begin{itemize}
\iusr{Dmytro Ponomarov}
\textbf{Евгений Середа} Может всё-таки не столько эмоционально, сколько рационально..

\iusr{Yevhen Sereda}
\textbf{Дмитрий Пономарёв} нет, рационально мыслят в Украине единицы. Двадцать
восемь лет выборов аферистов и бандитов во власть, этому подтверждение.

\iusr{Dmytro Ponomarov}
\textbf{Евгений Середа} единицы, это больше к иррационально.. //афёра, выборы
заточены на быстрый результат ,более рациональный подход.. условно в виде
оперативного планирования. Имхо конечно..
\end{itemize}

\iusr{Орест Стоколоса}

Яка різниця, хто що пропонує.?. Ліберально-равликовий метод чи
радикально-реформаторський, коли жодна влада в Україні не досягла (апріорі не
хотіла досягнути) рівності перед законом серед усіх верств населення, що не
залежить від посади, професії, статусу, чи фінансового становища. Саме
законність та непідкупність є базою, чи нулем від якого можна відштовхуватись,
коли будуєш стратегію розвитку держави. По барабану, яка модель, коли за пару
ре, якись районний суддя може пригальмувати всі реформаторські потуги))) О
так.. з тим суддьою, прокурором, чи інспектором будь-чого потім будуть
розбиратись.. Можливо навіть антикорсуд припакує його на шість років з
конфіскацією задекларованої двушки в хрущовці)) Але в офшорах його вже буде
чекати золота пенсія.. І ось, той маленький крок з кілометрового шляху займе
рік-два часу.. І похєр, що цю реформу підтримало 35 з 40-ка лямів українців.. і
похєр, хто ту реформу впроваджував, ліберал чи радикал. Нема від чого
відштовхнутися, коли нема від чого відштовхнутия... Як-би банально чи
тавтологічно це не звучало.. Ось така в нас срана селяві)) рьєбьята)).

\iusr{Yuriy Chernyavskyy}

Я за гибридный режим. Авторитарная Княжая монархия сверху и вечевая структура снизу. И интегральный федерализм. Сочетающий интегральный национализм вместо либеральный институтов с федерацией ретро-нео княжеств вместо муниципалитетов.

\iusr{Илья Слесеренко}

Украине нужен рывок, других вариантов нет

\iusr{Иван Кожевников}

Абсолютно поддерживаю политику Саакашвили!

\end{itemize}
