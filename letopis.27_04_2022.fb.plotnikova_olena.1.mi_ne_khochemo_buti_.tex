%%beginhead 
 
%%file 27_04_2022.fb.plotnikova_olena.1.mi_ne_khochemo_buti_
%%parent 27_04_2022
 
%%url https://www.facebook.com/helen.plotnikova/posts/pfbid0WFfeBcK7W5zCQnKrb2Q4uVtA3cbHReCZg6oxVLALHf4RWhr1NnTG3oVkg9oK9Hs1l
 
%%author_id plotnikova_olena
%%date 27_04_2022
 
%%tags mariupol
%%title Ми не хочемо бути  героями-мучениками,  ми хочемо жити
 
%%endhead 

\subsection{Ми не хочемо бути  героями-мучениками,  ми хочемо жити}
\label{sec:27_04_2022.fb.plotnikova_olena.1.mi_ne_khochemo_buti_}

\Purl{https://www.facebook.com/helen.plotnikova/posts/pfbid0WFfeBcK7W5zCQnKrb2Q4uVtA3cbHReCZg6oxVLALHf4RWhr1NnTG3oVkg9oK9Hs1l}
\ifcmt
 author_begin
   author_id plotnikova_olena
 author_end
\fi

" Ми не хочемо бути  героями- мучениками,  ми хочемо жити"- ці слова
повторювали хлопці- прикордонники в Маріуполі. Кожен день  згадую  21 березня,
коли чоловік  сказав:" Тобі більше  не  потрібно  тут залишатися, йди на вихід
з міста". Він приніс мені на дорогу  сухарів,  води та дві інжирини. Багато
було сліз, я залишала в місті найріднішу  людину. Ми попрощалися. Він пішов у
порт, де базувалася його частина, а я пішки,  під обстрілами,  виходила з
міста. Сльози , як краплі дощу, котилися по обличчю...Так я залишила в
Маріуполі  своє серце. Дорогою мене підібрали  прикордонники,  які втікали з
міста.Їх була ціла колона...До сьогодні  я так і не зрозуміла  для кого був
наказ про вихід, чому частина виїхала, частина прикордонників  залишилася. Та
це нехай буде  на їх совісті . Як таке можливо під час воєнного стану?! ( з
головою  держприкордонслужби  України  неможливо зв'язатися ,як і з генштабом ,
з президентом,  з будь- яким впливовим  українським журналістом. Неможливо. Для
простої людини це є недосяжно ). 

Я обривала всі  гарячі лінії,  щоб якось хлопцям допомогти. Але коли на іншому
кінці "дроту" сидить оператор  і каже:" Залишіть  заявку..." Я запитую :" Яку
заявку, на що заявку?"  Зрозуміла  одне, в Україні  громадянин держави не може
в екстренних випадках ні з ким зв'язатися,  ти просто нікому не потрібен.

Увесь час болить і кричить душа :" ДОПОМОЖІТЬ ВІЙСЬКОВИМ У МАРІУПОЛІ!!! ВОНИ -
ЛЮДИ, ТАКОЖ ЛЮДИ, ЯКІ ХОЧУТЬ ЖИТИ!" ДЛЯ НАС ВОНИ УЖЕ ГЕРОЇ. Але схоже влада
окрім безкінечних обіцянок про прориви , деокупації та слова 

" ТРИМАЙТЕСЯ" нічого іншого сказати не може. Я запитую у голови
держприкордонслужби  України , що ви зробили  для прикордонників  Маріуполі,
перерахуйте 1, 2, 3..схоже, що НІЧОГО. Часто лунають  по телебаченню слова :"
ВОНИ військові,  не здадуть  місто,  будуть стояти до кінця" . А ви запитали,
що вони хочуть?!  Вони хочуть жити, а не бути героями- мучениками. Таким чином
виправдовує  себе генштаб. Бо ,напевно ,краще залишити  їх помирати,  аніж
діяти. Місто контролюється  рос.військовими, у хлопців  закінчується вода та
їжа, вони виснажені,  багато поранених... будь-які ресурси вичерпуються...Чому
я повинна  благати про допомогу? Чому військові Маріуполя повинні відповідати
за прорахунки  генштабу? Чи це норма усвідомлено  залишати  їх помирати?! 

Я звертаюся до всіх , хто приймає  рішення про життя людей  на Азовсталі ( і
про військових  в тому числі, бо для пані Верещук люди- це цивільні,  а
військові  нехай здихають і гниють далі, бо такі правила війни):" Допоможіть
військовим!" ВОНИ хочуть жити. Пане Президенте,  це ваш прямий обов'язок! 

Якщо комусь  важко прийняти рішення, тоді я вас запрошую  на Азовсталь,
(висадіться десантом)на одну добу. Повірте, що рішення  прийметься дуже швидко.
Я не військовий експерт,  але розвиток  подій навколо військових   передбачала
на початку  березня . На що чоловік  відповідав,  що нам допоможуть,  буде
прорив. Вони вірили генштабу,  який їх зрадив. 

Я готова бути зеленим коридором  для свого чоловіка,  але поряд хочу бачити
голову держприкордонслужби  та увесь генштаб. 

P.S. Не повинні  обіймати державні  посади  люди,  які призвели до
катастрофічної ситуації  цивільних  та військових в Маріуполі. Це державна
зрада.

Вікторія, дружина воїна - прикордонника з Маріуполя
