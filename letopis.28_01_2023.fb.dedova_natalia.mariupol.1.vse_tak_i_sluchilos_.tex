%%beginhead 
 
%%file 28_01_2023.fb.dedova_natalia.mariupol.1.vse_tak_i_sluchilos_
%%parent 28_01_2023
 
%%url https://www.facebook.com/permalink.php?story_fbid=pfbid02qa4bQrFKLvGec32ezWMTAAy4zTukZUneXV7siJgRneJ2AEFkRQxQXYsL2piqvgul&id=100007662284921
 
%%author_id dedova_natalia.mariupol,suhorukova_nadia.mariupol
%%date 28_01_2023
 
%%tags mariupol
%%title Все так и случилось? Мариуполь - город, о котором перестали говорить?
 
%%endhead 

\subsection{Все так и случилось? Мариуполь - город, о котором перестали говорить?}
\label{sec:28_01_2023.fb.dedova_natalia.mariupol.1.vse_tak_i_sluchilos_}
 
\Purl{https://www.facebook.com/permalink.php?story_fbid=pfbid02qa4bQrFKLvGec32ezWMTAAy4zTukZUneXV7siJgRneJ2AEFkRQxQXYsL2piqvgul&id=100007662284921}
\ifcmt
 author_begin
   author_id dedova_natalia.mariupol,suhorukova_nadia.mariupol
 author_end
\fi

\#міймаріуполь
\#голосимирних
\#таня
\#Мариуполь \#Надежда

\obeycr
Все так и случилось? 
Мариуполь - город, о котором перестали говорить?
Он остался в середине марта. 
Одинокий и несчастный. 
Как брошенный ребенок. 
Его забыли среди чужих и злобных существ и не забирают.
Он не может понять: почему? 
Ему больно и страшно. 
Он хочет домой. 
Сегодня прочитала у Наташи Дедовой историю девушки Тани. 
Она погибла в Мариуполе. 
Ей было двадцать. 
Она была влюблена. 
Когда бомбили и стреляли, Тане было очень страшно. 
Она хотела жить, но страх отнимал надежду. 
Она писала об этом любимому человеку, пока была связь. 
А потом связи не стало. 
Не стало тепла, воды, еды, города, привычной жизни. 
Почти нечем было дышать. 
В голове ничего не укладывалось. 
Только ужас. 
Не осталось ничего, кроме бомб и обречённости. 
Очень больно в двадцать лет чувствовать обречённость. 
Каждую минуту могут убить. 
А хочется закинуть голову и посмотреть в небо. 
И чтобы оно было небом, как раньше. 
А не мрачным и пугающим рвом над головой. 
Откуда каждые десять минут выглядывает смерть. 
И заставляет играть по ее правилам.
А ещё хочется похохотать на лавочке с котенком.
Беззаботно. 
Двадцать лет - это же сплошное счастье. 
Он маленький и смешной. 
И тоже обречён, как и Таня. 
А жить дальше - такое наслаждение. 
Просто жить, как обычно, как раньше. 
Выйти из дома и пойти пешком в центр города. 
Встретиться с друзьями у башни Нильсена. 
Купить круассан во французской булочной и покормить возле театра наглых голубей. 
Но у Тани этого больше не было. 
И не будет. 
За нее все решил российский лётчик. 
Перечеркнул Танину жизнь. 
Поставил черную метку на  крышу Таниного дома. 
Приговорил к смерти. 
Наказал ни за что. 
И Таня осталась в своем городе.
Навсегда.  
Вместе с близкими. 
А  вместо неба и голубей табличка с именем  и фамилией.  
Весной под табличкой расцвели нарциссы. 
Такие нежные и веселые, как Таня. 
Это так странно. 
Тани нет, а жизнь продолжается. 
Многих больше нет.
А Мариуполь есть. 
Его назвали городом - героем. 
А герои не умирают. 
Пока о них помнят.
Надежда Сухорукова ❤️🩹
\restorecr

\url{https://m.facebook.com/story.php?story_fbid=3317841738481203&id=100007662284921}
