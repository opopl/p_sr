% vim: keymap=russian-jcukenwin
%%beginhead 
 
%%file 18_10_2020.news.ua.strana.1.lomachenko_nazis
%%parent 18_10_2020
%%url https://strana.ua/news/295671-lomachenko-lopes-zasudili-li-ukraintsa-v-las-vehase.html
 
%%endhead 

\subsection{"Видно было, что Василий выиграл". Справедливо ли присудили поражение Ломаченко и чему радуются националисты}

\url{https://strana.ua/news/295671-lomachenko-lopes-zasudili-li-ukraintsa-v-las-vehase.html}

\ifcmt
img_begin 
	tags lomachenko
	url https://strana.ua/img/article/2956/71_main.jpeg
	caption Бой Ломаченко-Лопес. Кадр из видео
	width 0.7
img_end
\fi

Сегодня утром украинский боксер Василий Ломаченко проиграл чемпионский титул в
легком весе американцу Теофимо Лопесу. 

Это поражение стало одним из крупнейших в истории украинского современного
бокса. Поэтому его активно обсуждают в Украине и не только. 

Тем для обсуджений две - что украинца явно засудили американские судьи в
Лас-Вегасе, чтобы отдать победу своему спортсмену.

Вторая - по традиции, националисты, вопреки тому, что приграла вообще-то
Украина, стали радоваться поражению Ломаченко. Объясняя, что он поддерживает
УПЦ Московского патриархата. 

Собрали реакцию на бой Ломаченко с Лопесом. 

\subsubsection{Засудили или нет?}

О том, что решение судей несправедливое, после поединка заявил сам Ломаченко.
Это сообщение у него вышло в Facebook. 

"Я не знаю. Честно говоря, я думал, что половина поединка была равной. Может
быть, он где-то выигрывал на 1-2 раунда, но во второй половине я включился. В
принципе, я считаю, что вторую половину я забрал.

Я хочу приехать домой и пересмотреть поединок, но я не согласен с решением
(судей – Ред.). Я думал, что выиграю, но результат есть результат. Я с ним
спорить сейчас не собираюсь.

Сейчас я поеду домой. Потом я поговорю со своим промоутером и менеджером. Мы
все обсудим и решим, что делать дальше", - написал он.

\ifcmt
pic https://strana.ua/img/forall/u/0/92/image_2020-10-18_135058.png
\fi

Как видимо, Ломаченко и сам признал, что в первой половине боя был менее
активен и только потом "включился". Да и было видно, что в начальных раундах
боксер больше защищался, чем атаковал. Лопес же в те моменты прессинговал и
постоянно занимал центр ринга. 

При этом во второй части поединка Василий начал серьезно теснить противника и
"пробивать" его комбинациями ударов.

По статистике боя, Ломаченко сделал в два раза меньше ударов, чем Лопес (321
против 659). Зато процент попаданий украинца был вдвое выше: 44 против 28 у
американца. То есть Лопес чаще бил мимо или в блок. Но все равно получал очки. 

Учитывая, что бой закончился без нокаута, картина неочевидна с точки зрения
победы или поражения Ломаченко. Но у американских судей все оказалось иначе:
они присудили победу Лопесу, причем один из рефери посчитал, что украинец
проиграл вообще все раунды. 

Речь о судье Джули Ледерман (ее судейская карта со счетом по центру).

\ifcmt
pic https://strana.ua/img/forall/u/0/92/%D1%81%D1%83%D0%B4%D1%8C%D0%B8(1).jpg
\fi

Видимо, поэтому и сам Ломаченко, и многие его коллеги-боксеры раскритиковали
судейство. 

Генеральный секретарь Федерации бокса России Умар Кремлев считает, что
Ломаченко выиграл бой. 

"Лично мое мнение - очень печально, когда так судьи судят. Видно было, что
Василий выиграл. Не надо жить американской мечтой, надо добиваться своей цели.
Там надо досрочно выигрывать, очень грубое судейское решение", - сказал
Кремлев.

Он отметил, что в профессиональном боксе решения судей не пересматриваются,
потому Ломаченко следует готовиться к реваншу.

"Напомнило мне бой Головкина и Альвареса. У Василия еще все впереди, но в
реванше, если не досрочно, будет тяжело победить. Очень печальное судейство,
очень расстроился", - добавил Кремлев.

В свою очередь российский коллега Ломаченко по рингу Александр Беспутин
поддержал украинца в своем Instagram.

"Вася чемпион! Ты лучший, брат", - написал спортсмен.

\ifcmt
pic https://strana.ua/img/forall/u/0/92/FireShot_Capture_3145_-_Alexander_Besputin_%D0%B2_Instagram__%C2%AB%D0%92%D0%B0%D1%81%D1%8F_@lomachenkovasiliy_%D1%87%D0%B5%D0%BC%D0%BF%D0%B8%D0%BE%D0%BD!_%D0%A2%D1%8B__-_www.instagram__.com_[1]_.png
\fi

\subsubsection{Игры "патриотов"}

Но у американских судей внезапно нашлась группа поддержки в Украине.

Поражению Ломаченко возрадовались украинские националисты - за то, что
спортсмен, как и его коллега Александр Усик, является прихожанином УПЦ и
сторонником нормализации отношений с Россией.

"Лопес чемпион. Пророссийский коллаборационист Ломаченко проиграл. Александр
Усик, теперь твоя очередь, грязь подмосковская. Мы все помним", - написал
экс-глава одесского "Правого сектора" Сергей Стерненко. 

\ifcmt
pic https://strana.ua/img/forall/u/0/92/%D1%81%D1%82%D0%B5%D1%80%D0%BD%D0%B5%D0%BD%D0%BA%D0%BE(10).png
\fi

"Не понимаю, как это могло произойти. Его же благословил Онуфрий", -
сыронизировал представитель Украины в Трехсторонней контактной группе Денис
Казанский. 

\ifcmt
pic https://strana.ua/img/forall/u/0/92/image_2020-10-18_141300.png
\fi

"Ломаченко скрепы не помогли. А Усик там как? Держится, молится?", - издевается
волонтер Роман Доник. 

\ifcmt
pic https://strana.ua/img/forall/u/0/92/image_2020-10-18_141450.png
\fi

То есть развернулась травля Ломаченко. При этом многие встали на защиту
украинского боксера. 

"Боюсь, что против Усика Стерненко уже из тюрьмы болеть будет", - написал
нардеп Макс Бужанский, говоря о том, что праворадикала сейчас судят за
умышленное убийство человека 
\footnote{\url{https://strana.ua/articles/analysis/150795-krov-sternenko-na-nozhe-dominiruet.html}}

\ifcmt
pic https://strana.ua/img/forall/u/0/92/%D0%B1%D1%83%D0%B6(6).png
\fi

"Слежу за социальными сетями и травлей Ломаченко. Помните старую шутку: "тут
что ни строй- обком кпсс выходит". Строили Незалежную Украину- а получился
опять Советский Союз со сладострастной травлей инакомыслящих. Жду "письма
писателей" с призывом лишить партбилета или гражданства", - пишет телеведущая
Наталья Влащенко.

\ifcmt
pic https://strana.ua/img/forall/u/0/92/image_2020-10-18_141926.png
\fi

"Ночью смотрел бой Ломы, расстроился.

А утром увидел радостные коменты – так ему и надо, манкруту и ватнику.

Не поленился – заблокировал ручками больше 100 чел.

А тем, кто знает, кто такой манкрут, образованным, написал, что они х...сосы.

Как думаете, догадаются, о чем я?", - пишет бизнесмен Гарик Корогодский. 

\ifcmt
pic https://strana.ua/img/forall/u/0/92/image_2020-10-18_142148.png
\fi

"Ломаченко в любом случае - боец и личность. Человек.

Молодец.

Те же, кто сейчас шакалят по вопросу его веры, сами того не понимая, льют воду
на мельницу тех, кто видит в присуждении победы американцу дискриминацию и
политический подтекст.

И я склонна с этим согласиться. Потому как подобное в избытке видела в нашей
стране после майдана. Травля, убийства, аресты, липовые рейтинги, фейковые
победители и виртуальные герои - все это признаки деградации и разложения", -
считает экс-нардеп Елена Бондаренко.

\ifcmt
pic https://strana.ua/img/forall/u/0/92/image_2020-10-18_142247.png
\fi

А вот бизнесмен Евгений Черняк считает, что Лопес победил заслуженно. Но
плевков от националистов Ломаченко за это не заслужил.

"Против Лома есть приёмы. Гондурас сегодня победил. Лопес из Гондураса, я об
этом, если что.

И дело не только в судьях, конечно.

Нужно было пять раундов пешком не ходить, а включать скорость уже в первом
раунде.

Но каждый мнит себя стратегом видя бой издалека.

Лопес очень не простой парень и ему только 23.

Те, кто скачет по пабликам захлёбываясь от ненависти к Ломаченко, просто хочу
напомнить, что он один сделал для Украины больше, чем вы все вместе взятые, со
своими ботофермами.

Благодаря Василию звучит Гимн страны, сегодня звучал в Вегасе.

А благодаря вам, профессиональные патриоты, о стране знают только то, что это
коррупция, отсутствие законов и уличные беспорядки.

Ничего хорошего, конечно, вы для своей страны не сделали.

Максимум на что способны- это тупые комментарии в Фейсбуке, разжигающие
национальную и религиозную ненависть и разламывающие пополам страну.

И бесконечное воровство с рассказами о том, что нужно всё быстрее разграбить,
пока не напал никто.

А Василий ещё вернётся и победит, увидите и в очередной раз добавит позитивного
промо своей стране".

\ifcmt
pic https://strana.ua/img/forall/u/0/92/image_2020-10-18_142519.png
\fi
