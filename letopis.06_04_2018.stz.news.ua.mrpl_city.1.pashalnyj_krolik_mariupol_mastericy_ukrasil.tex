% vim: keymap=russian-jcukenwin
%%beginhead 
 
%%file 06_04_2018.stz.news.ua.mrpl_city.1.pashalnyj_krolik_mariupol_mastericy_ukrasil
%%parent 06_04_2018
 
%%url https://mrpl.city/news/view/pashalnyj-krolik-ot-mariupolskih-masterits-ukrasil-ploshhad-v-stolitse-foto
 
%%author_id news.ua.mrpl_city,selitrinnikova_anastasia.mariupol
%%date 
 
%%tags fb.hashtag.#фестивальписанок,mariupol,velykden,velykden.krolik
%%title Пасхальный кролик от мариупольских мастериц украсил площадь в столице (ФОТО)
 
%%endhead 
 
\subsection{Пасхальный кролик от мариупольских мастериц украсил площадь в столице (ФОТО)}
\label{sec:06_04_2018.stz.news.ua.mrpl_city.1.pashalnyj_krolik_mariupol_mastericy_ukrasil}
 
\Purl{https://mrpl.city/news/view/pashalnyj-krolik-ot-mariupolskih-masterits-ukrasil-ploshhad-v-stolitse-foto}
\ifcmt
 author_begin
   author_id news.ua.mrpl_city,selitrinnikova_anastasia.mariupol
 author_end
\fi

Мариупольские мастерицы отправили расписного кролика на крупнейший в Украине
фестиваль писанкарского искусства, приуроченный к празднику Светлого
Воскресения Христова.

Об этом журналисту MRPL.CITY рассказала руководитель клуба \enquote{Макошь} Светлана
Мешкова-Давиденко.

Вчера в Киеве на Софийской площади стартовал восьмой по счету Всеукраинский
фестиваль писанки: центр столицы украсили большие писанки и 374 расписных
кролика, один из которых – дело рук мариупольских умелиц.

\ii{06_04_2018.stz.news.ua.mrpl_city.1.pashalnyj_krolik_mariupol_mastericy_ukrasil.pic.1}

\enquote{Мы участвуем в этом фестивале уже третий год. Это крупнейшее мероприятие, куда
отбирают работы самых лучших мастеров, прошедших жесткий отбор. От Мариуполя
подавали заявки множество мастериц, но приняли только наш клуб, чему мы очень
рады}, - говорит Светлана Мешкова-Давиденко.

\textbf{Читайте по теме:} \href{https://mrpl.city/news/view/pashalnye-vyhodnye-stoit-li-mariupoltsam-dostavat-zontiki}{%
Пасхальные выходные: стоит ли мариупольцам доставать зонтики?, Анастасія Селітріннікова, mrpl.city, 06.04.2018%
}

По ее словам, раньше рукодельницам приходилось расписывать яйца, а в этом году
прислали фигурку пасхального кролика. Кроме того, на фестиваль отправились и
классические писанки, украшенные мариупольчанками.

\ii{06_04_2018.stz.news.ua.mrpl_city.1.pashalnyj_krolik_mariupol_mastericy_ukrasil.pic.2}

Мастерицы также планируют после Пасхи посетить столицу, где в рамках
Всеукраинского фестиваля писанки также пройдет награждение участников и ряд
мастер-классов, где участницы \enquote{Макоши} планируют научиться чему-то новому и
\enquote{повысить квалификацию}. Сейчас они ищут спонсоров, которые смогут оплатить
поездку.

Полюбоваться шедеврами лучших художников и мастеров писанкарского искусства в
сердце столицы можно будет не только в пасхальную неделю: выставка-фестиваль
продлится до 22 апреля.
