% vim: keymap=russian-jcukenwin
%%beginhead 
 
%%file 27_05_2021.fb.taksjur_jan.1.pushkin_kiev_xxi_vek_ierusalim
%%parent 27_05_2021
 
%%url https://www.facebook.com/taksyur/posts/4228070303905760
 
%%author Таксюр, Ян
%%author_id taksjur_jan
%%author_url 
 
%%tags 
%%title Нема вже вашого Пушкина! 21-й век. Киев. Мать городов русских. Русский Иерусалим.
 
%%endhead 
 
\subsection{Нема вже вашого Пушкина! - 21-й век. Киев. Мать городов русских. Русский Иерусалим}
\label{sec:27_05_2021.fb.taksjur_jan.1.pushkin_kiev_xxi_vek_ierusalim}
\Purl{https://www.facebook.com/taksyur/posts/4228070303905760}
\ifcmt
 author_begin
   author_id taksjur_jan
 author_end
\fi

Моя жена регент церковного хора. Утром она едет в храм на маршрутке. Наш храм
расположен в парке, носящем имя великого поэта. Там же и памятник ему. Голос у
жены высокий и звонкий. Подъезжая к остановке, она громко, чуть по-детски,
просит водителя в утренней тишине:

\obeycr
- Парк Пушкина!
Водитель некоторое время угрюмо молчит. А потом говорит с раздражением:
- Нема вже вашого Пушкина!
21-й век. Киев. Мать городов русских. Русский Иерусалим.
\restorecr

Феофанов Максим

Во первых профессия извозчика, всегда была не легким хлебом.
А ще цьому мужчінчіку із усіх відомих і близьких його серцю каналів кажуть, що російська мова, російська культура—це ворог
А нема в нього часу в цьому розібратись
А приїде він в депо
І там його керівник
Азовець, Упашнік чи СС овець щось ідеологічне знов зарядить
А дома дружина і діти
А в дружини щось подібне у голові бо вона взагалі до амереканських протестантів ходить, а дитина в школі тільки що пожурбила по тим хто загинув на майдані.
Та родине—це ще добре
А якщо той гордий укр, взагалі спить в свої маршрутці, чи в хостелі-пролітаріку...
Це журба
Суцільна
І хто цих людей буде жаліти
Молитися за них
Співчувати
Якщо і навіть його мати чи батько живі, то тоже його цькують на русский мир...
Вопиет земля к небу о беззакониях в стране нашей происходящих
Может же и будет выжженная земля
Скорбь, боль и черное беззаконие. И есть уже. И будет умножение.
Пушкина нет
Да и Гоголя нет
И Сковороды нет
Нет их в сознании.
А как они появятся, если слов этих не было слышно.
Ни от куда.
Скоро нужно будет переводить их, что бы все таки прикоснуться к уму современного украинца светлыми смыслами замечательных людей.
Грустно это все.
А с другой стороны хороший стимул к умному деланию.
Хоть и крайне сложный.
Но все же без стремления к сочувствию к современным "свободным", не будет у православных будущего в этой стране...
Пока водитель за камни или ружье опять не взялся, нужно постараться пожалеть его и помолиться за него...
Не знаю...
Больно это все

Татьяна Горлач

Лукоморья больше нет, от дубов простыл и след, дуб годится на паркет так ведь
нет. Выходили из избы здоровенные жлобы, порубили все дубы на гробы, ты уймись
тоска у меня в груди это только присказка сказка впереди. Ты уймись тоска душу
мне не рань, если это присказка, значит дело дрянь...

Andrey Vorobyov

Любую, даже самую гениальную затею можно довести до полного абсурда. Вот если
бы остановка была, к примеру, сквер Оффенбаха? Как отреагировал бы водитель?
Скорее всего, он не в курсе, что Оффенбах немецкий француз, или французский
немец, а фамилия очень спорная, мягко говоря:-) Вот просто жутко представить,
что в голове у него( водителя)

Татьяна Дьяченко

О Киев-град, где с верою святою
Зажглася жизнь в краю у нас родном,
Где светлый крест с Печерскою главою
Горит звездой на небе голубом,
Где стелются зеленой пеленою
Поля твои в раздольи золотом,
И Днепр-река, под древними стенами,
Кипит, шумит пенистыми волнами!
Как часто я душой к тебе летаю,
О светлый град, по сердцу мне родной!
Как часто я в мечтах мой взор пленяю
Священною твоею красотой!
У Лаврских стен земное забываю,
И над Днепром брожу во тьме ночной:
В очах моих все русское прямое —
Прекрасное, великое, святое.
Уж месяц встал; Печерская сияет;
Главы ее в волнах реки горят;
Она душе века напоминает;
Небесные там в подземелье спят;
Над нею тень Владимира летает;
Зубцы ее о славе говорят.
Смотрю ли вдаль — везде мечта со мною,
И милою всё дышит стариною.
Там витязи сражались удалые,
Могучие, за родину в полях;
Красою здесь цвели княжны младые,
Стыдливые, в высоких теремах,
И пел Баян им битвы роковые,
И тайный жар таился в их сердцах.
Но полночь бьет, звук меди умирает;
К минувшим дням еще день улетает.
Где ж смелые, которые сражались,
Чей острый меч, как молния, сверкал?
Где та краса, которой все пленялись,
Чей милый взгляд свободу отнимал?
Где тот певец, чьим пеньем восхищались,
Ах, вещий бой на всё мне отвечал!
И ты один под башнями святыми
Шумишь, о Днепр, волнами вековыми!

Козлов Иван

Очень тоскую по родному городу...
