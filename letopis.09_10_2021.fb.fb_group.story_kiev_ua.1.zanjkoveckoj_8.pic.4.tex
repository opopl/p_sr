% vim: keymap=russian-jcukenwin
%%beginhead 
 
%%file 09_10_2021.fb.fb_group.story_kiev_ua.1.zanjkoveckoj_8.pic.4
%%parent 09_10_2021.fb.fb_group.story_kiev_ua.1.zanjkoveckoj_8
 
%%url 
 
%%author_id 
%%date 
 
%%tags 
%%title 
 
%%endhead 

\ifcmt
  ig https://scontent-mxp1-1.xx.fbcdn.net/v/t39.30808-6/244587035_3017871988468515_4030896064864514050_n.jpg?_nc_cat=104&ccb=1-5&_nc_sid=b9115d&_nc_ohc=3QOy_z8CuAYAX84K7cp&_nc_oc=AQnPxtfLDXUFWWqrXJCi1qnHTwv5OaVa0dP7V8YR1n7N_z22mefaSN16EPpq1o_Idjs&_nc_ht=scontent-mxp1-1.xx&oh=b27619594812b97f64fcd247aa146a0a&oe=6195184B
  @width 0.4
\fi

\iusr{Анатолий Борозенец}

Тильна сторона будинку з вул. Заньковецької.

\iusr{Victoria Ivaniy Stein}

На жаль, засклення балконів повністю руйнує вигляд будинку. Не розумію чому це
так стало популярно. Там же і транспортного потоку такого немає і чому
дозволяють це робити в таких будинках. Невже нікому не хочеться сісти на
стільчик і поряд маленький столик і зранку випити кави, подивитись навкруги, ну
наприклад.

\iusr{Лариса Кушниренко}
\textbf{Victoria Ivaniy Stein} 

А куди ховати запаси на зиму і старий мотлох до переїзду на дачу? Квартирки
маленькі, місця не вистачає. Деякі вважають, що це захист від шуму та пилу. А
на те, що виглядає жахливо, неохайно, спотворює будинок, незаконно і
небезпечно, як кажуть-до лампочки.

\iusr{Victoria Ivaniy Stein}
\textbf{Лариса} от дуже шкода! А те що "до лампочки" це тому що немає в Україні
чи Києві норм і їх дотримання, які існують в інших цивілізованих країнах щодо
архітектури міста. Сараї висять на кожному будинку- жах! А навіщо старий мотлох
тримати? А запаси на зиму які? Що не можна купити зараз в магазині те, що
потрібно?
