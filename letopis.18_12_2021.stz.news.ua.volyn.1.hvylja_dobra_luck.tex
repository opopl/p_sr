% vim: keymap=russian-jcukenwin
%%beginhead 
 
%%file 18_12_2021.stz.news.ua.volyn.1.hvylja_dobra_luck
%%parent 18_12_2021
 
%%url https://www.volyn.com.ua/news/201146-khvylia-dobra-prokotylasia-lutskom-foto
 
%%author_id 
%%date 
 
%%tags 
%%title «Хвиля добра» прокотилася Луцьком (Фото) 
 
%%endhead 
\subsection{«Хвиля добра» прокотилася Луцьком (Фото) }
\label{sec:18_12_2021.stz.news.ua.volyn.1.hvylja_dobra_luck}

\Purl{https://www.volyn.com.ua/news/201146-khvylia-dobra-prokotylasia-lutskom-foto}

\ifcmt
  ig https://i2.paste.pics/eaa33956b7b2491ec1b61de45d154f72.png
  @width 0.7
  %@wrap \parpic[r]
  %@wrap \InsertBoxR{0}
\fi

\begingroup
\Large\em\color{blue}
Не дивлячись на примхи погоди, наші люди довели, що вони добрі, співчутливі,
активні та творчі. Близько двох місяців тому ми розпочали підготовку до другої
Хвилі добра, адже були впевнені, що це саме той проєкт, який точно має «жити»
далі
\endgroup

У Луцьку відбувся другий благодійний ярмарок «Хвиля добра». організували його:
Творчий калейдоскоп, Сонячні діти Волині-ВОГО «Даун-синдром»,
Кав'ярня-кондитерія Старе Місто та департамент молоді та спорту. Відкривав
творчу програму відомий музикант та співак - Сергій Скулинець, який завжди радо
долучається до благодійних акцій. Неперевершеним та зворушливим був танок від
колективу «Гармонія» із сонячними вихованцями Сонячні діти Волині-ВОГО
«Даун-синдром» та пісня у виконанні Маргарита Ярута. А куди ж без Святого
Миколая?

\ii{18_12_2021.stz.news.ua.volyn.1.hvylja_dobra_luck.pic.1}

Цього дня до ярмарку долучилися десятки волонтерів із різних навчальних
закладів Луцька та Волині. Адже продавати вироби у не таку вже й літню погоду -
той ще виклик!

Направду ж, випробування тільки роблять нас сильнішими. Адже ми знали якою є
наша мета - допомогти Нікіті та його батькам! І ми це зробили.

\begin{zznagolos}
Якщо Ви не змогли відвідати ярмарок сьогодні, 17 грудня, кав’ярня-кондитерія,
яка знаходиться за адресою: м. Луцьк, вул. Драгоманова, 11, радо прийме Вас у
найближчий тиждень. Ще є багатий вибір різноманітних товарів.	
\end{zznagolos}

Після 16:00 (час завершення ярмарку цього дня) ми розпочали процедуру
підрахунку коштів. Цей процес детально знімкувався, сторони здійснили підписи у
акті вилучення коштів з благодійних скриньок, спостерігачами були представники
з Волинський обласний ліцей з посиленою військово-фізичною підготовкою.

\ii{18_12_2021.stz.news.ua.volyn.1.hvylja_dobra_luck.pic.2}

Отже, ми готові оголосити результати - цього дня ми спільно із Вами зібрали
суму у 19 147 грн.. Протягом тижня Ви можете збільшити суму, придбавши товари у
ка'ярні. Після цього ми перерахуємо кошти батькам Нікіти, про що обов'язково
прозвітуємось перед Вами.

У 2020 році під час ярмарку збирали кошти для дівчини з Волині, у 2021 році -
для маленького хлопчика з Києва. Наша акція також має на меті показати, що для
нас не має значення, де живе людина, адже добро не має меж та кордонів.

%\ii{18_12_2021.stz.news.ua.volyn.1.hvylja_dobra_luck.pic.3}

Нікітка, ці всі люди понад два місяці творили заради тебе. Кожного вівторка та
середу вони малювали картини, розмальовували еко-торбинки, власними руками
створювали новорічні іграшки. А інші люди та організації привозили свої
подарунки (іграшки, варення, мед, книжки...). Те, що мають; те, що вміють
робити; те, чим хочуть та можуть поділитися.

ВАЖЛИВО:

Якщо Ви не змогли відвідати ярмарок сьогодні, 17 грудня, кав’ярня-кондитерія,
яка знаходиться за адресою: м. Луцьк, вул. Драгоманова, 11, радо прийме Вас у
найближчий тиждень. Ще є багатий вибір різноманітних товарів.

Наш традиційний захід «Хвиля добра» йде на невелику перерву. Але ж наступного
року ми обов'язково зустрінемося і знову будемо доводити, що наші люди -
найкращі на світі. Адже допомоги потребує ще багато людей. Напередодні
новорічних свят відбувається безліч інших благодійних заходів - закликаємо Вас
у разі можливостей та бажання долучатися і до них! Адже чим більше нас, тим
більше добра! А саме добро врятує світ!

Дякуємо щиро усім, хто долучився!!!

Лена Кольцова
