% vim: keymap=russian-jcukenwin
%%beginhead 
 
%%file 04_12_2021.fb.tregubov_viktor.1.napad_rf_ukraina.cmt
%%parent 04_12_2021.fb.tregubov_viktor.1.napad_rf_ukraina
 
%%url 
 
%%author_id 
%%date 
 
%%tags 
%%title 
 
%%endhead 
\subsubsection{Коментарі}
\label{sec:04_12_2021.fb.tregubov_viktor.1.napad_rf_ukraina.cmt}

\begin{itemize} % {
\iusr{Alex Derkach}
всього-то проблема - кожен українець має вальнути трьох росіянців і одного малороса))

\begin{itemize} % {
\iusr{Родіон Шевченко}
\textbf{Alex Derkach} зїбс - роботи на пів дня, а на другу можна і шашлик організувати )

\iusr{Гилли Шервуд}
\textbf{Alex Derkach} Угу. При цьому, хто саме буде назначений малоросом, буде вирішувати той, хто голосніше вищатиме, що саме він і є найсправжнісінькій, найчистісінький, з діда-прадіда, українець.

\iusr{Гавриленко Олег}
\textbf{Гилли Шервуд} добре, що по першому питанню заперечень немає

\iusr{Гилли Шервуд}
\textbf{Гавриленко Олег} які можуть бути питання, свинопси - не люди.

\iusr{Alex Derkach}
\textbf{Гилли Шервуд} хм... я думав, того малороса всі на хвамілію знають)

\iusr{Oleksii Divnyi}
\textbf{Alex Derkach} та і трьох росіян навкруги нього - теж
\end{itemize} % }

\iusr{Jake Ferguson}

Хтось із відомих єврейських інтелектуалів сказав: «Якщо вам кажуть, що вас
хочуть вбити, обов’язково сприймайте то всерйоз».

\begin{itemize} % {
\iusr{Alex Derkach}
\textbf{Jake Ferguson} "-ваня пашлі укроп косіть! -а єслі оні нас? -а нас-то за чьто?"

\iusr{Андрій Кулешов}
Голда Меїр?

\ifcmt
  ig https://scontent-frt3-2.xx.fbcdn.net/v/t39.30808-6/262789643_1839200759612734_1260939116377090818_n.jpg?_nc_cat=103&ccb=1-5&_nc_sid=dbeb18&_nc_ohc=STbVQr68bbgAX_xnaEo&_nc_ht=scontent-frt3-2.xx&oh=8f141a4a2478f605a2bd20f6fb140ec0&oe=61B26B4E
  @width 0.4
\fi

\iusr{Jake Ferguson}
\textbf{Андрій Кулешов}
Це інший вираз

\end{itemize} % }

\iusr{Alex Nikishin}

Це може бути зливом з російських спецслужб, заради шантажу та залякування.

Однак порівняно навіть з 2014 роком є суттєве погіршення. Зараз влада яка
здасть Україну в разі війни, вона не буде боротися, це не ті люди.

Вони швидко підпишуть капітуляцію, до якої їх спонукатимуть і західні держави,
а потім їх не можна буде скинути, бо їхнє існування при владі буде гарантією
подальшого ненаступу Росії. Власне це все на Вірменії відроблено вже.

\begin{itemize} % {
\iusr{Иван Иванович Чай}
\textbf{Alex Nikishin} влада Москвою поставлена. Тому вона не лише підпише капітуляцію, але й вже 3й рік розвалює державу та армію

\iusr{Alex Nikishin}
У цій владі багато російських агентів. Але функція її буде виконана, коли вона здасться Росії. Заради цього тих людей вводили у владу.

\iusr{Иван Иванович Чай}
\textbf{Alex Nikishin} не лише заради цього - це планується лише як фінал демонтажу української державності та армії
\end{itemize} % }

\iusr{Liana Solomatina}

Есть одно существенное отличие от 2014 года - Верховный главнокомандующий
Украины. Под которым уже зашаталось кресло. Лучшей возможности для Кремля может
никогда не представится.


\iusr{Sergiy Kostin}

Мене більш турбує, коли, гіпотетично, полететять оті літаки з десантом, і треба
буде їх збивати, наш фельдмаршал віддасть наказ стріляти, чи пропаде на
тиждень? А потім вилізе і вже губернатор малоросіі.

\begin{itemize} % {
\iusr{Наталья Панина}

Це найстрашніше. Є всі шанси, що у випадку такого загострення ситуації,
Зеленський скоріш за все буде тихо і швидко пакувати свої валізи.


\iusr{Victor Tregubov}
\textbf{Sergiy Kostin} я вірю, що коли політять ті літаки, наказ нашого сельдмаршала про відкриття вогню буде виконано незалежно від того, чи він його віддасть.

\iusr{Andrew Dmitrenko}
\textbf{Наталья Панина} жувати галстук
\end{itemize} % }

\iusr{Вячеслав Гор}

\ifcmt
  ig https://scontent-frt3-1.xx.fbcdn.net/v/t39.30808-6/262329729_4827460487288615_6969508978869279725_n.jpg?_nc_cat=106&ccb=1-5&_nc_sid=dbeb18&_nc_ohc=tisB9nZaxDMAX-ddk42&_nc_ht=scontent-frt3-1.xx&oh=881b3367085434a822303cfca06e33b8&oe=61B13289
  @width 0.4
\fi

\iusr{Аліна Барановська}
Плетемо

\iusr{Тарас Князєв}

Я думаю що головна проблема Путіна в тому що час йде а нічого не рухається. Ми
не нападаємо, не відбиваємо. А нього все з рук сиплеться не так як він
запланував. Тому думаю що він не може вічно чекати і хоче якось проштовхнути.

Звісно йому треба щоб не він а ми напали а вони просто оборонялися. То не
виключені провокації.

Так чи інакше щось повинно бути.

\begin{itemize} % {
\iusr{Иван Иванович Чай}
\textbf{Тарас Князєв} Казус Беллі не є проблемою і його буде вчинено у разі прийняття рішення про повномасштабне
\end{itemize} % }

\iusr{Інна Накопюк}
Може то створюють підстави, щоб Бубочка біг персонально домовлятися і заглядати в очі?

\iusr{Valeriy Butenko}

Знаєте, особисто у мене було б набагато менше тривожності з приводу
сьогоднішньої безпекової ситуації навколо України - якби Верховним
Головнокомандуючим ЗСУ, головою РНБО та очільниками інших відповідальних посад
в нашій країні були ті люди, за яких я проголосував у 2019-му. Але \enquote{мудрий,
простий нарід (с)} проголосував інакше (((, і в мене є відчуття, що цей
\enquote{нарід-акакаяразніца} вже подумки змінив прапори на адмінбудівлях та власних
домівках, і з самозаспокійливим: \enquote{Так вєдь тоже можна жіть (с)} - ніяких
потенційно трагічних змін в своєму житті не бачить.


\iusr{Андрій Копецький}
План простий. Можеш - йди в ЗСУ. Не можеш в ЗСУ? Є ТРо.

\iusr{Vladyslav Synyahovsky}
Істину каже. Але краще все ж таки купити хоча б гладкоствол. Все ж більше стволів.

\iusr{Danila Volo}
ці плани виникли після коментів люсі арестовича дощу про те як би він наступав.
у мене є ще більш детальні плани, причому від 2005 року

\iusr{Світлана Коширець}
Я вірю в збройні сили. Але чи є достатньо ЗСУ на білоруському напрямку, наприклад?

\begin{itemize} % {
\iusr{Олександр Макаренко}
\textbf{Світлана Коширець} у нас достатньо сил на усіх кордонах. В кожному місті та селі. І кожен наш солдат - мотивований воїн, який знає за що буде воювати, якщо доведеться! З іншого боку - стадо деморалізованого м’яса, яке гонять в Україну боротись з Америкою. Головне бути готовими
\end{itemize} % }

\iusr{Dmytro Prokhorov}

Здається мені, що під приводом \enquote{раптової} війни рф проти України, США
пропустять пп2 начебто в обмін на мир в Україні

\begin{itemize} % {
\iusr{Valeriy Butenko}
\textbf{Дмитро Прохоров} , 

так там внутрішні розбірки між демократами та республіканцями з приводу санкцій
щодо ПП-2. Республіканці в Сенаті блокують оборонний бюджет на 2022 рік,
вимагаючи включити до нього жорсткі санкції щодо газопроводу. Демократи, яким
важливо зберігати добрі відносини з Німеччиною - не настільки категоричні.

Тому, \enquote{обміняти} вторгнення в Україну на ПП-2 у Байдена не вийде вже
через американські внутрішньополітичні причини.

\end{itemize} % }

\iusr{Natalia Paulus}
Вірю у Сбройни Сили @igg{fbicon.heart.blue}  @igg{fbicon.heart.yellow} 

\iusr{Сашко Клочан}

\obeycr
У фб вже майже не заходжу.
Читаю лише авторитетні джерела інформації.
Інфа, яка надходить - від розвідки США і розвідки НАТО. Інших не читав.
Інфа однакова. І вона насторожує.
І такої інфи не було ні в 2014-му, ні будь коли раніше.
Панікувати - нічого не дає.
Займатись самообманом і самозаспокоюванням - звісно , психологічно легше.
А як буде - побачимо.
\restorecr

\iusr{Лана Швецова}

За ці сім років склалось враження, що вони дуже хочуть нас захопити, але
неочікувано, щоб бліцкрігом, а потім знов втихомирювати захід, що так само
вийшло.

І щоразу їх зупиняє саме алярм і розкриття планів.

Цього разу тільки одна різниця - алярм від наших очільників затягнувся, а
раніше він був першим.

\iusr{Dmitry Shevchuk}
Бутусов нас спасьот!

\iusr{Lyudmila Yemchenko}
Якщо ці плани вивалили у публічний простір, то точно вже так не буде.
Таке враження, що мацкву дражнять, щоб подивитись, як далеко вона сама себе зажене у д\#*пу, у спробі поділити світ під себе.

\iusr{Олена Сидорчук}

Порошенко в 2018 ввів воєнний стан, бо рашисти накопичились на кордонах. Хоча і
тоді не було так страшно, як зараз із тим опосумом на троні. Тож цей аларм не
просто так.

\begin{itemize} % {
\iusr{Роман Курбатов}
Он ввел военное положение после атаки на наши корабли и их захвата.

\iusr{Людмила Михайловская}
\textbf{Олена Сидорчук}, 

ви забули, чи не знали чого Порошенко ввів в деяких областях України військовий
стан - за напад на наші катери

\iusr{Олена Сидорчук}
\textbf{Людмила Михайловская} Не тільки

\ifcmt
  ig https://scontent-frx5-2.xx.fbcdn.net/v/t39.30808-6/263653776_1497628687288366_5147989533779766977_n.jpg?_nc_cat=109&ccb=1-5&_nc_sid=dbeb18&_nc_ohc=D90g8_veCdIAX_ftNua&_nc_ht=scontent-frx5-2.xx&oh=2cad7267de49c5eb4e1d545ad612b34c&oe=61B23B2C
  @width 0.4
\fi

\end{itemize} % }

\iusr{Ірина Гаєва}

Саме так. Знаю з особистого досвіду 14го року, що найкращий засіб проти усіх
тих істерик- реальне волонтерство. По-перше робишь хорошу справу й допомагаєшь
армії/переселенцям/хворим/дітям...... По- друге нема часу читати усю ту фігню в
інети. Й спишь хорошо ніччю.

\iusr{Halyna Lipatova}
В нас старі батьки і нема авто, щоб швиденько кудись поїхати якщо що. Та й батьків не кинемо.
Залишається допомагати волонтерам і вірити в ЗСУ.
Єдине, що справді лякає- це зелений клован з єрмаком на чолі держави...

\iusr{Dmitry Balobin}

в мене додаткове цікаве заняття - дивитись, хто з професійних медійників, тих
хто не може не усвідомлювати своїх дій, роздмухує істерію. вони часто співають
хором. набивши статистику можна оперувати поняттями "хор імені такого-то співає
отаке для того-то"


\iusr{Анна Харьков}
В ЗСУ вірю, в говнокомандуючого -ні. Він всіх сдасть

\begin{itemize} % {
\iusr{Сергей Лидинг}
\textbf{Анна Харьков} Ви мали на увазi верховного говнокомандувача? Бо це рiзне.. (с) :)))
\end{itemize} % }

\iusr{Сергій Даценко}

Для країн заходу є сенс говорити про високу імовірність вторгнення для того,
щоб пояснити як правильно вони роблять, що не хочуть нас брати в НАТО. А для
пуйла зараз багато що змінилося: в Україні клоуни-кроти, Білорусь дозріла до
поглинання, в Німеччині - зміна влади, в США старенький біленький обама,
брекзит, макрон і тп. В Ердогана інфляція, яка ризикує призвести взагалі до
зміни влади в Туреччині... Коли ще як не зараз?

\iusr{Енвер Кадиров}
Ну, нарешті..

\iusr{Василий Тарасенко}

++ саме правильне рішення обходити такі \enquote{новини} які неможливо перевірити як
лайно і йти собі спокійно дали.

\end{itemize} % }
