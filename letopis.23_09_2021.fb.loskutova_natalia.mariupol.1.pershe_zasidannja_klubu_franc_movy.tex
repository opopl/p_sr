%%beginhead 
 
%%file 23_09_2021.fb.loskutova_natalia.mariupol.1.pershe_zasidannja_klubu_franc_movy
%%parent 23_09_2021
 
%%url https://www.facebook.com/1427894275/posts/pfbid02t8ktGHCPcYypxrB4pukwMoQYqjyrNPSNDCWZJByNVJTD8o7syVMaUSachcEPmmssl
 
%%author_id loskutova_natalia.mariupol
%%date 23_09_2021
 
%%tags mariupol,mariupol.pre_war,obrazovanie,language.french
%%title Спілкування французькою триває - 23 вересня 2021 у Центрі 1991 Mariupol відбулося перше засідання Клубу французької мови
 
%%endhead 

\subsection{Спілкування французькою триває - 23 вересня 2021 у Центрі 1991 Mariupol відбулося перше засідання Клубу французької мови}
\label{sec:23_09_2021.fb.loskutova_natalia.mariupol.1.pershe_zasidannja_klubu_franc_movy}

\Purl{https://www.facebook.com/1427894275/posts/pfbid02t8ktGHCPcYypxrB4pukwMoQYqjyrNPSNDCWZJByNVJTD8o7syVMaUSachcEPmmssl}
\ifcmt
 author_begin
   author_id loskutova_natalia.mariupol
 author_end
\fi

Спілкування французькою триває

23 вересня 2021 у Центрі 1991 Mariupol відбулося перше засідання Клубу
французької мови.

\ii{23_09_2021.fb.loskutova_natalia.mariupol.1.pershe_zasidannja_klubu_franc_movy.pic.1}

Організатор і керівник Клубу – Сергій Шитченко, письменник,
викладач, перекладач, директор мовної школи BrAmS. У засіданні взяли участь
носії французької мови (Антоні Тунда, конголезець, студент 3 курсу
спеціальності \enquote{Міжнародні відносини} Маріупольського державного університету),
а також аматори та прихильники мови Рабле й Гюго. Не могли залишитися осторонь
цієї події студенти МДУ, які вивчають французьку мову як основну іноземну.
Студентки другого курсу спеціальностей \enquote{Середня освіта. Французька мова і
література} та \enquote{Філологія. Романські мови та літератури (переклад включно),
перша – французька} разом з доценткою кафедри німецької та французької
філології Наталією Лоскутовою залюбки відвідали цей захід. Протягом півтори
години у залі лунала мелодійна французька мова: учасники знайомилися,
розповідали про себе, про те, чому їх захопила ця гарна мова. Засідання Клубу
будуть проходити щосереди о 16.00, а у студентів з'явилася додаткова можливість
поспілкуватися французькою з франкофонами та однодумцями.
