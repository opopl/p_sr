%%beginhead 
 
%%file 07_08_2023.fb.mariupol_neskorenyj.1.mariupol_pejzazhist_vasyl_korenchuk
%%parent 07_08_2023
 
%%url https://www.facebook.com/100066312837201/posts/pfbid02Hy5t9Y1o3c2Y8UsZ6H2rSiBjTzpzJbaN38PZvYxpkw4rTUzm4TpNcJVoQ1KU4xWPl
 
%%author_id mariupol_neskorenyj
%%date 07_08_2023
 
%%tags 
%%title Маріупольський пейзажист Василь Коренчук
 
%%endhead 

\subsection{Маріупольський пейзажист Василь Коренчук}
\label{sec:07_08_2023.fb.mariupol_neskorenyj.1.mariupol_pejzazhist_vasyl_korenchuk}

\Purl{https://www.facebook.com/100066312837201/posts/pfbid02Hy5t9Y1o3c2Y8UsZ6H2rSiBjTzpzJbaN38PZvYxpkw4rTUzm4TpNcJVoQ1KU4xWPl}
\ifcmt
 author_begin
   author_id mariupol_neskorenyj
 author_end
\fi

📎 Учасники проєкту \enquote{Маріуполь нескорений} - маріупольський пейзажист Василь Коренчук

✅️ Під час першого відкриття в Києві виставки, яка в подальшому переродилась у
грандіозний проєкт \enquote{Маріуполь нескорений}, один з її організаторів та учасників
Костянтин Чернявський поділився спогадами: \enquote{На все життя в мене залишиться
квиток на поїзд Київ – Маріуполь на 24 лютого 2022 року, оскільки 25 лютого в
місті Маріуполь в музеї імені Архипа Куїнджі мала відкритись персональна
виставка нашого колеги Василя Коренчука}. Роботи Василя Михайловича Коренчука,
зокрема \enquote{Маріупольський порт}, написана ще у 2021 році, як і картини десятків
митців зі всієї України, увійшли в збірку колекції \enquote{Маріуполь нескорений}.

Художника Василя Коренчука маріупольці знають як енергійну людину та
талановитого пейзажиста, який майже кожен рік радував шанувальників мистецтва
новими картинами. Його роботи, пронизані любов'ю до України та її природи,
вражають своєю відкритістю. До виставки, презентація якої очікувалась в
маріупольському художньому музеї ім. А. І. Куїнджі 25 лютого 2022 року, вже були
підготовлені афіші, запрошені гості, преса. Але відкриття не відбулося.
Завадила страшна війна.😪

@igg{fbicon.heart.white.middle} Працівники музею згадують, як 24 лютого зранку Василь Михайлович допомагав
ховати в укриття картини Куїнджі, Айвазовського, Дубовського, Калмикова.
Сподівалися, що так можна зберегти шедеври. Але у квітні стало відомо, що ворог
їх цинічно вкрав і перевіз до Донецька, в так звану \enquote{днр}. Картини самого
Василя Михайловича сховати не змогли. У художньому музеї не було фондосховища.
Вони так і залишились в експозиційних залах. І, на жаль, також частково були
вкрадені окупантами. На відео, яке з'явилось у соцмережах, можна було побачити,
як співробітники Донецького краєзнавчого музею приймають де-які з них на
зберігання. Доля інших робіт на цей час невідома.

Втрата творчого спадку, це дуже велике випробування для художника. У бесідах з
митцем чуються і гіркі нотки про втрачений творчий спадок, і світла надія на
Перемогу, і віра в те, що після нашої Перемоги картини повернуться до
художника. 

👏Василь Михайлович знайшов у собі творчі сили знову почати працювати,
створювати нові, дуже душевні пейзажі і натюрморти. Тепер вже у Львові, де він
зараз знаходиться. 18 травня 2023 року, у Міжнародний день музеїв, він
подарував три свої нові картини Львівському центру \enquote{ЯМаріуполь}. Це полотна,
створені маріупольським художником за останні півтора року життя у місті Лева –
\enquote{Підірваний міст біля \enquote{Азовсталі}}, \enquote{Краєвид на Маріупольський порт} та
\enquote{Весняний настрій}.🖼✨️

📍Біографічна довідка:

Василь Михайлович Коренчук народився у 1950 р. на Хмельниччині. Про свою малу
Батьківщину живописець з посмішкою каже, що той, хто народився і виріс у цій
мальовничій частині України, приречений стати художником, або хоча б любителем
споглядання природи.

У 1970-х роках В. Коренчук переїхав до Маріуполя, закінчив Університет
художників-оформлювачів. До 2000 р. працював художником-оформ\hyp{}лювачем на заводі
\enquote{Азовсталь}. З 2006 р. є членом Національної Спілки художників України. У
творчості Василя Михайловича переважає живопис у реалістичному стилі – пейзажі,
натюрморти, портрети. Художник неодноразово приймав участь не лише у міських, а
й в обласних, всеукраїнських та міжнародних мистецьких виставках. У 2007 р.
отримав Медаль Меморіалу А. І. Куїнджі. Його твори зайняли почесне місце у
Донецькому, Маріупольському, Бердянському художніх музеях.

\#Маріупольнескорений \#виставка \#культурнадеокупація \#Маріуполь \#Київ \#художники
