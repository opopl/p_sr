% vim: keymap=russian-jcukenwin
%%beginhead 
 
%%file slova.imperia
%%parent slova
 
%%url 
 
%%author 
%%author_id 
%%author_url 
 
%%tags 
%%title 
 
%%endhead 
\chapter{Империя}
\label{sec:slova.imperia}

%%%cit
%%%cit_head
%%%cit_pic
%%%cit_text
Может, стоит больше внимания обращать на примеры нашего величия?
\emph{Имперскости} нам не хватает. Тут в связи с мовосрачами, поднятыми
Ларисой и Вахтангом, вот что хочу сказать.  \emph{Имперскойсти} нам не хватает.
Величие собственное не осознаём. Всё нам кажется, что кто-то нас использовал.
Что не мы строили СССР, а какие-то марсиане строили его, а Украину
использовали. Что не мы были \emph{Российской империей}, Речью Посполитою и
Княжеством Литовским, а эти \emph{империи} строили какие-то чужинци, а
украинцев порабощали и пили кровь ихнюю.  А как зайдёшь в какой-то музей
польский или литовский - то там вся украинская знать в одном ряду с польской и
литовской знатью. Как-то не похожи князья наши стародавние на подчинённых и
порабощенных... А первыхе лица \emph{Российской империи} и СССР - ну украинцев
больше, чем литовцев, казахов и таджиков
%%%cit_comment
%%%cit_title
\citTitle{Почему нам все время кажется, что нас кто-то использовал? / Лента соцсетей / Страна}, 
Павел Себастьянович, strana.ua, 01.07.2021
%%%endcit

%%%cit
%%%cit_head
%%%cit_pic
%%%cit_text
Така \enquote{віра} в російську \enquote{демократію} завжди породжувала нескінченні трагедії
для українців. Це вже траплялося і в часи новітньої історії, насаджуючи таку
маріонеткову не \enquote{українську} владу (на кшталт білоруської), яка б забезпечила
поглинання України цією варварською \emph{імперією}.  Відкриті спроби завершувалися
Майданами, оскільки дух свободи і віри в непереможність ідеї \enquote{своєї правди в
своїй хаті} був і залишається найвищою цінністю Українського народу. В Україні
і поза її межами мають знати всі, що ніщо і ніхто не може позбавити Народ цього
священного і суверенного права. Будь-яка українська влада є життєздатною, коли
відчуватиме цю спраглість народної Волі, не зраджуватиме Україну, сповідуватиме
у Слові і Чині Українську Ідею. Усе, що поза цими цінностями, є брехнею,
\enquote{блекотою}, приводячи таких правителів лише до забуття та народного осуду.
Культурна, політична, економічна інтеграція з тими, хто успадкувавши найгірші
риси ординського \enquote{Кипчака}, де завжди панує хан-диктатор і суспільний
менталітет рабської залежності, не приніс житейських благ жодному народу
%%%cit_comment
%%%cit_title
\citTitle{Правда в Рідному Слові}, Георгій Філіпчук, slovoprosvity.org, 12.07.2021
%%%endcit

%%%cit
%%%cit_head
%%%cit_pic
%%%cit_text
А ви – малороси? Навіть якщо ваші предки – етнічні росіяни, то ви мали би вже
давно зукраїнізуватися в Україні, а не російщити українців. Це не природно. І
настільки не природно, що вас треба називати незукраїнізованими росія­ нами
(малоросами), а не «російськомовними» українцями. Ви обурюєтеся, що я
категорично не даю вам права почувати себе «російськомовними українцями». Це
не я не даю права. Просто такого не існує. Птах чоловічої статі з обручкою
орнітолога на нозі не стає завдяки обручці одруженою людиною. Щур, який
народився у хліві, не стане вепром завдяки місцю народження. Баба з базару, яка
торгує яблуками, виноградом, бананами та гранатами не стане військовим, бо вона
має кавказькі гранати. Скрипалем не станеш, бо вмієш посмикати себе по
причандалах, тому що скрипка має смичок. Якщо ви хочете, аби вас називали
українцями, то заговоріть українською а російську покиньте назавжди, бо
\emph{Імперія Зла} вбиває тепер щодня українців а частину території – окупує
%%%cit_comment
%%%cit_title
\citTitle{НЕ ІСНУЄ РОСІЙСЬКОМОВНИХ УКРАЇНЦІВ. КРАПКА}, Przemysław Lis-Markiewicz Profil Prywatny, %
facebook, 22.10.2021
%%%endcit

%%%cit
%%%cit_head
%%%cit_pic
%%%cit_text
Первая галактическая \emph{империя} рухнула. Она разлагалась в течении
столетий, и только один человек полностью осознал этот факт.  Это был Хари
Селдон, последний великий ученый Первой \emph{Империи}, который
усовершенствовал психоисторию – науку, описывающую человеческое поведение
математическими уравнениями.  Индивидуальный человек – существо
непредсказуемое, но реакции человеческой массы, считал Селдон, могли быть
вычислены статистически.  Чем больше масса, тем большей точности можно
добиться. А объем человеческой массы, с которой работал Селдон, был не меньше,
чем население всех миллионов обитаемых миров Галактики
%%%cit_comment
%%%cit_title
\citTitle{Кризис основания}, Айзек Азимов
%%%endcit

%%%cit
%%%cit_head
%%%cit_pic
%%%cit_text
І все ж настав той день, коли, плачучи й ридаючи, сіла царівна в ромейську
неоковирну кувару, щоб попливти до темних варварів, хижих умів, тупих душ, в
дикість і безвір’я. Але то тільки так мовиться, що сіла в кувару й попливла.
Бо вже перед тим послано в Рим, щоб повідомити \emph{імператора} й папу про те,
скільки пресвітерів і слуг віри супроводжує Анну, щоб хрестити Русь, яка
дружина супроводжує сестру \emph{імператорів}, щоб захистити при потребі віру й
честь, які дарунки відправлено київському князеві: мечі, кольчуги, лати,
золоті ланцюги, срібні й золоті тканини, фібули, сосуди.
Все це пливло на ромейських незграбних кораблях-куварах, волочилося по морю
повільно й тяжко, ми насилу стримували свої меткі легенькі ладьї, щоб не
вирватися наперед і не завдати образи високій особі. Берегом, не відстаючи від
морського походу, супроводжувало нас кінне ромейське військо з повозами,
шатрами, припасами на всі забаганки Анни, коли набридне їй море і зволить вона
зійти на берег для спочинку
%%%cit_comment
%%%cit_title
\citTitle{Тисячолітній Миколай}, Павло Загребельний 
%%%endcit

%%%cit
%%%cit_head
%%%cit_pic
%%%cit_text
Тут, щоправда, є деякі питання до Путіна з Лавровим. Зрозуміло, що ринок зброї
побудований на цинізмі, але невже не можна було використати безліч спільних
тем, які пов'язують ердоганівську Туреччину з путінською Росією, щоб не
допустити продажу такої ефективної зброї Україні, проти якої Путін веде війну?
Ці запитання могли б поставити прихильники зміцнення \emph{імперії}, якби у
фашистській державі, побудованій Путіним, диктатору та його найближчому
оточенню можна було б ставити безсторонні запитання. Утім, добре, що цей режим
сам обрубав усі канали комунікації, що прирікає його на постійні помилки та
загальну неадекватність і, зрештою, робить неминучою його загибель
%%%cit_comment
%%%cit_title
\citTitle{Росія отримала два щигля по носі від України}, 
Ігор Яковенко, gazeta.ua, 29.10.2021
%%%endcit

%%%cit
%%%cit_head
%%%cit_pic
%%%cit_text
Давайте еще раз начнем с азов. Степан Андреевич Бандера, "Третья мировая война
и освободительное движение", журнал "Сурма", Германия, 1950 год: "Это все было
бы невозможно, если бы в западном мире не принялась и не распространилась
иллюзия, что русский народ и большевистский \emph{империализм} – это две отдельные
вещи, что русский народ не является носителем, а только жертвой последнего, что
к борьбе с большевизмом можно привлечь часть русского народа, и даже большую
часть, как союзника. Это заблуждение распространяется на западе и россиянами,
целой политической работой русской эмиграции Россия есть только одна —
\emph{империалистическая}, и так будет до тех пор, пока российский \emph{империализм} НЕ
будет опустошен, разбит, а российский народ не вылечится от него через
познание, что его \emph{империализм} приносит ему самому больше всего бедствия —
жертв, страданий и падения"
%%%cit_comment
%%%cit_title
\citTitle{Россия. Березки. Фашизм}, 
Аркадий Бабченко, news.obozrevatel.com, 27.10.2021
%%%endcit

%%%cit
%%%cit_head
%%%cit_pic
%%%cit_text
Генеральна лінія російської \emph{імперії} в усі її часи і в усіх її форматах
залишалася незмінною: українське має дорівнювати неповноцінному, марґінальному,
примітивному, архаїчному, сміховинному, карикатурному, неперспективному,
неспроможному й відірваному від будь-якого, умовно кажучи, Novum (cума
доданків, що її Юрій Шевельов узагальнив поняттям "Картагена нашої
провінційности"). Але утримування "другої серед рівних" республіки в такому
протиприродно затиснутому статусі вимагало від \emph{імперії} дедалі більшого
напруження всіх її сил і виняткової непроникності ззовні. Кожна чергова
всесоюзна кампанія з придушення й викорінення чергового ворожого "ізму" в
українських реаліях набирала значно більш драконівського характеру, ніж у
середньому по есесесер. "Якщо в Москві стрижуть нігті, то в Україні відрубують
пальці", – свідчила популярна в мистецьких колах приказка, і станом на середину
1980-х вона не втрачала актуальності. Радянська Україна була й лишалася
республікою під особливим наглядом
%%%cit_comment
%%%cit_title
\citTitle{Українську культуру довелося витягати з дна і перестворювати з уламків}, 
Юрій Андрухович, gazeta.ua, 29.10.2021
%%%endcit

%%%cit
%%%cit_head
%%%cit_pic
\ifcmt
  tab_begin cols=3
     pic https://w-dog.ru/wallpapers/0/66/424797810259420/galaktika-planety-zvezdy.jpg
     pic https://avatars.mds.yandex.net/get-zen_doc/1246934/pub_5e0b902e73034b00b0037c26_5e0b903e6f5f6f00ae0274ff/scale_1200
		 pic https://w-dog.ru/wallpapers/3/3/328733121541359/tumannost-zvezdy-planety-galaktika.jpg
  tab_end
\fi
%%%cit_text
А затем остались лишь звездолет, большой и сверкающий, совершеннейшая продукция
двенадцатитысячелетнего развития \emph{Империи}, и он сам, Гаал Дорник,
новоиспеченный доктор наук, получивший приглашение на Трантор от самого Хари
Сэлдона. Молодому ученому предстояло принять участие в каком-то загадочном
сэлдоновском проекте.  После разочарования, испытанного при гиперпрыжке, Гаал с
нетерпением ждал, когда же наконец появится Трантор. Он ни на минуту не покидал
комнаты общего обзора. В заданное время стальные покрышки иллюминаторов
откатывались назад, и Гаал всматривался в звезды, наслаждаясь их россыпями,
похожими на рой застывших на месте светлячков. Когда заслонки откатились в
очередной раз, Дорник увидел примерно в пяти световых годах от звездолета
газовое скопление, которое проскользнуло мимо иллюминатора, как выплеснутое
молоко. На какой-то миг оно наполнило комнату обзора леденящим светом, исчезнув
из вида после нового прыжка.  Сначала солнце Трантора показалось маленькой
точкой, затерянной среди миллиардов других, — здесь, в центре Галактики,
скопления звезд были густы. С каждым прыжком эта звездочка сияла все ярче и
ярче, заслоняя собой остальные
%%%cit_comment
%%%cit_title
\citTitle{Основание}, Айзек Азимов
%%%endcit

%%%cit
%%%cit_head
%%%cit_pic
\ifcmt
  tab_begin cols=2
     pic https://avatars.mds.yandex.net/get-zen_doc/5335957/pub_6180d6d880f2655b868c60cf_6180de944ad86220098d68c8/scale_1200
     pic https://avatars.mds.yandex.net/get-zen_doc/50335/pub_6180d6d880f2655b868c60cf_6180e1b9d51b313bac75ca90/scale_1200
  tab_end
\fi
%%%cit_text
Самое вкусное мороженное где производилось? Не улыбайтесь… но в Жданове. А
самая элитная черешня для всемирных и всесоюзных выставок? В Мелитополе. Лучшее
советское пиво для дипломатического корпуса? В Краматорске. Ну да… это были
колониальные замашки, знаменитое «тёмное бархатное» до последнего литра
отправлялось в Москву и Ленинград.  Можно многое перечислять, куда вкладывались
непропорционально (по сравнению с РСФСР) огромные средства. Футбол, бокс,
русский классический театр... и тот перевезли в Киев. Где была сама большая
сеть санаториев, профилакториев, домов отдыха и поселковых — «культуры и
досуга»? На Украине.  Самые высокие зарплаты индустриального сектора? Самые
внушительные сбережения граждан на сберкнижках? Наибольшее число
колхозов-миллионеров? Самый современный и многочисленный торговый и речной
флот? Именно в \emph{Империи}. Под названием «Украинская ССР». Где это всё сегодня?
%%%cit_comment
%%%cit_title
\citTitle{Как Украина была колонией СССР... Не наоборот ли?}, 
Исторические напёрстки, zen.yandex.ru, 02.11.2021
%%%endcit

%%%cit
%%%cit_head
%%%cit_pic
\ifcmt
  tab_begin cols=4
		 pic https://polit.ru/media/photolib/2013/06/11/thumbs/nikolay-1-06_1370948651.jpg.600x450_q85.jpg
     pic https://bigkyiv.com.ua/wp-content/uploads/2021/10/003.-2.jpg
		 pic https://alyoshin.ru/Photo/butnik-siversky/02.jpg
     pic https://bigkyiv.com.ua/wp-content/uploads/2021/10/002.-2.jpg
  tab_end
\fi
%%%cit_text
Коли вже будувати університет, то грандіозний – так, вочевидь, вирішив
архітектор Вікентій Беретті. Проєкт зодчого щодо зведення Університету Святого
Володимира (нині відомий Червоний корпус) був затверджений Миколою І. Але коли
\emph{імператор} дізнався, що будівництво встане у суму в 8 млн. рублів, то
ледь не втратив свідомість.  Скандал вгамували тим, що київські забудовники
знизили розцінки у двічі. Саме ця поблажливість і дала змогу Вікентію Беретті
створити свій шедевр у стилі класицизм. Будівництво стартувало в 1837 році, а
вже за п’ять років у стінах Червоного корпусу пройшли перші лекції.  Паралельно
з будівництвом університету Вікентій Беретті розробив проєкт і ще однієї
легендарної споруди Києва – Інституту шляхетних дівчат (нині Жовтневий палац).
Але помилуватися творінням архітектор не встиг, продовжувати справу батька був
вимушений Олександр Беретті
%%%cit_comment
%%%cit_title
\citTitle{Місто «на стилі»: українське бароко, неоготика, сталінський ампір і не тільки...}, 
Тетяна Ніжинська, bigkyiv.com.ua, 23.10.2021
%%%endcit

%%%cit
%%%cit_head
%%%cit_pic
%%%cit_text
Занимают, как у нас нынче говорят, активную проукраинскую позицию, члены
националистической организации «Тризуб имени Степана Бандеры». Так что они люди
убежденные и идейные. Противостоят всему русскому и борются активно со всем
русским и \emph{имперским}. В украинском культурном пространстве не должно быть места
всяким там гоголям, рихтерам, шаляпиным (я об этом рассказывала, «Неукраинский
Гоголь»)
%%%cit_comment
%%%cit_title
\citTitle{Украинские писатели-братья}, Украинский русский, zen.yandex.ru, 04.11.2021
%%%endcit

%%%cit
%%%cit_head
%%%cit_pic
\ifcmt
  tab_begin cols=3
     pic https://rgnp.ru/wp-content/uploads/8/3/a/83ad1b787a2ed614832b926963ec806a.jpeg
     pic https://avatars.mds.yandex.net/i?id=d8a37732c6e5a28d88a1dbffd18c837a-5345552-images-thumbs&n=13
		 %pic https://s0.tchkcdn.com/g-YAc8Px3c6PrefEdO642ZHg/14/118383/660x480/f/0/89058864.jpg
		 pic https://lh6.googleusercontent.com/QLiIoLB4vB09fbhs8lAj6iWlX2lYTz2Wg52GVzM92iHDd9jeG4CFlp1UTdIa9nDkJciNVASgh4xT-jbSisI7FEI1_7xAL4DocYIQWn6Dc5gEThoorpluqvYnH0LCCS-5od5qv_9t
  tab_end
\fi
%%%cit_text
У книзі багато інформації, яка є важливою для переосмислення історії, зокрема
про персоналій, які у своїх епохах були лідерами суспільної думки: Ярослав
Мудрий, Богдан Хмельницький, Іван Мазепа, рід Розумовських, Іван Котляревський,
Федір Достоєвський, Адам Міцкевич, Петро Григоренко, Микола Скрипник, Михайло
Горбачов, — усе це потрібно читати і переосмислювати, формувати на цій основі
фактично новий світогляд, нове уявлення про Україну та її місце у світі.
Оглядаючи історію Росії та України, можна сміливо стверджувати, що українці
добряче долучилися до творення \emph{імперської Росії}. Настільки широко і глибоко, що
без українців та України — починаючи від Києво-Могилянської академії! — навряд
чи ця \emph{імперія} могла б існувати в такому вигляді, в якому вона є. Отже, настав
час поставити все на свої місця, забрати додому своє і сказати: далі Росія без
нас. Серед тем, висвітлюваних «Днем», можна назвати матеріали про участь
українців у становленні \emph{Російської імперії}. Це матеріали про український
генералітет у \emph{Російській імперії}, про внесок українців у радянсько-німецьку
війну під час Другої світової й нарешті — про те, «ким були наші в Політбюро»,
та інші статті
%%%cit_comment
%%%cit_title
\citTitle{Чому історію потрібно переписувати і переосмислювати?}, 
Микола Семена, day.kyiv.ua, 04.11.2021
%%%endcit
