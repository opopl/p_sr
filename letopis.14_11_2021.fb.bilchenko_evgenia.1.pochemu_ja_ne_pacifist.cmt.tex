% vim: keymap=russian-jcukenwin
%%beginhead 
 
%%file 14_11_2021.fb.bilchenko_evgenia.1.pochemu_ja_ne_pacifist.cmt
%%parent 14_11_2021.fb.bilchenko_evgenia.1.pochemu_ja_ne_pacifist
 
%%url 
 
%%author_id 
%%date 
 
%%tags 
%%title 
 
%%endhead 
\subsubsection{Коментарі}

\begin{itemize} % {
\iusr{Таня Пономарева}

Аничка с котиком не видят эту картиночку! Им бы очень понравилось!
@igg{fbicon.face.smiling.eyes.smiling}  (Извини, что я не по существу...
Сплин...)

\iusr{Евгения Бильченко}
\textbf{Таня Пономарева} \textbf{Анна Долгарева}

\iusr{Алексей Бажан}

Флоренскому приписывается доклад о Блоке, где много сказано о блоковском
бесовидении, кульминирующем в явлении "страдающего бога" в 12-и. Потом тот же
"бог" посетил Чевенгур, о чем у АГД в Магическом большевизме Андрея Платонова,
евпочя. Сам Гельич не чужд тех же посещений, отразившихся в его древней песне
"84-ый ближе, ближе, ближе" (песню мало кто слышал, записей в сети не встречал,
но текст обнародован). Давайте мы без Логоса как-нибудь, а то постоянно
приходят не те, кого вроде бы звали. Как к Блоку.

\href{http://blok.lit-info.ru/blok/kritika-o-bloke/florenskij-pavel-o-bloke.htm}{%
Флоренский Павел: О Блоке, blok.lit-info.ru%
}

\begin{itemize} % {
\iusr{Евгения Бильченко}
\textbf{Алексей Бажан} как всегда, интересно

\iusr{Алексей Бажан}
\textbf{Евгения Бильченко} 
//Полчеловечества в речку
Полчеловечества в печку.
А остальные в вечность.//

\url{http://lj.rossia.org/users/udod99/693106.html}

\iusr{Алексей Бажан}
\textbf{Евгения Бильченко} 

Да, еще Вам Дугина (текст его, сам вроде не пел), последнюю строфу ему до сих
пор припоминают. 

\href{https://www.youtube.com/watch?v=KKG5MgWcG5I}{%
Банда четырех - Абсолютный рассвет, Gordei Petrik, youtube, 23.11.2018%
}

\end{itemize} % }

\iusr{Настя Бузиашвили}
Круто! Интересные мысли

\iusr{Борис Никифоров}
Полностью поддерживаю докладчика. От пацифистов жутко пахнет серой)

\iusr{Татьяна Пьянченко}
...как сложно и интересно Вас читать! Спасибо.

\end{itemize} % }
