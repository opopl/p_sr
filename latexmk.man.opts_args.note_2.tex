% vim: keymap=russian-jcukenwin
%%beginhead 
 
%%file man.opts_args.note_2
%%parent man.opts_args
 
%%endhead 

\subsection{Note 2}
  
\vspace{0.5cm}
 {\ifDEBUG\small\LaTeX~section: \verb|man.opts_args.note_2| project: \verb|latexmk| rootid: \verb|p_saintrussia| \fi}
\vspace{0.5cm}
  
In this documentation, the program pdflatex is often referred to.

Users of programs like lualatex and xelatex should know that  from latexmk's
point  of view, these other programs behave very like pdflatex, i.e., they
make a pdf file from a  tex  file,  etc.   So  whenever pdflatex is mentioned
without mention of the other programs, the statements apply equally to
lualatex, xelatex, and any  other  similar  programs.  Latexmk can be easily
configured to use whichever of these programs is needed.  See the
documentation  for  the  following  options: -pdflua,  -pdfxe,  -lualatex, and
-xelatex, and also see the documentation for the \verb|$pdflatex|,
\verb|$lualatex|,  and  \verb|$xelatex|  configuration  variables.   At
present  latexmk  does not do automatic detection of which program is to be
used.

