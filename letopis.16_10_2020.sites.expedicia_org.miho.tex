% vim: keymap=russian-jcukenwin
%%beginhead 
 
%%file 16_10_2020.sites.expedicia_org.miho
%%parent 16_10_2020
%%url https://expedicia.org/pamyati-oleksandra-mikho/
%%tags oleksandr miho
 
%%endhead 

\subsection{Пам’яті Олександра Міхо}
\label{sec:16_10_2020.sites.expedicia_org.miho}

\url{https://expedicia.org/pamyati-oleksandra-mikho/}

\ifcmt
pic https://expedicia.org/wp-content/uploads/2020/10/119703703_10222555341368270_4413560364325926306_o.jpg
\fi

Великий смуток і жаль викликала звістка про те, що після боротьби з важкою
хворобою пішов з життя легендарний полярник, учасник багатьох арктичних та
антарктичних експедицій Олександр Міхо.

Радист, фотограф, художник та письменник рівненчанин Олександр Міхо був
учасником 4-ї та 7-ї Українських антарктичних експедицій. Менеджер зв´язку на
антарктичній станції «Академік Вернадський» він був наставником і для інших
радистів.

Редакція часопису «Експедиція ХХІ» висловлює щирі співчуття близьким і друзям
Олександра Міхо. Світла пам’ять про нього залишиться в історії українських
полярних досліджень.

Зовсім недавно ми публікували чудове післяслово Олександра Міхо з книги
«Антарктида: рай і пекло», в якому автор передає любов до Антарктики.
Пригадаємо цей текст ще раз.

\subsubsection{Післяслово --- Олександр Міхо}

Антарктида зустрічає всіх однаково. Континент дихає то спокоєм, то бурями.
Важливим є те, як сама людина, що вперше потрапила на шостий континент матінки
Землі, сприйме його. 

Як стверджує Месснер, протягом усієї подорожі через льодовий континент його не
полишали картини «Божественної комедії» Данте. Картини пекла --- під час
споглядання гостроверхих гірських вершин, оточених льодовиками, і рай земний –
коли сонце радісно грало в переливах легкої поземки, й та тиша і спокій, які
він зміг відчути тільки тут. Але наприкінці подорожі автор полюбив цю холодну й
безлюдну місцевість, дійшовши думки, що Антарктида є доказом того, що світ
колись був раєм. Тільки коли людина почала там подорожувати і ділити території,
вона винайшла пекло. 

Перетинання антарктичного материка без використання будь-яких технічних засобів
– мотосаней і всюдиходів --- це спосіб спілкування Месснера з незайманою
природою. Написана книга стала закликом до людей дбайливо ставитися до цього,
майже ще незайманого, оазису людства. Подорожувати «чистіше», ніж
першопрохідці, які не дуже переймалися питаннями екології, --- ось принцип
учасників унікальної експедиції автора.

Перші дослідники Антарктиди, прагнучи до підкорення нескореного, притягли за
собою на материк техніку і паливо. Відстрілювали тюленів і пінгвінів заради
заготівлі м’яса на зимівлю. Сьогодні ми ставимося до таких вчинків як до
злочинного варварства, але пробачимо першопрохідців і вкотре захопимося їх
подвигами, перемогами і відкриттями. В ім’я їх самопожертви та з повагою до їх
трагедій, що розігралися в ті далекі роки. Імена Амундсена і Скотта, Шеклтона і
Берда назавжди залишаться в історії освоєння Антарктиди.

Будучи учасником полярних експедицій, я неодноразово зимував на Українській
антарктичній станції «Академік Вернадський».

Антарктида і зараз така, якою вона була сотні років тому. Варто лише вийти
зоряної антарктичної ночі з дому і відійти настільки далеко, щоб не чути звуків
працюючих дизелів. І коли скрип снігу під ногами зникне, і коли ти піднімеш
голову, і над тобою буде тільки небо… ти відчуєш себе піщинкою на величезній
планеті. Я все це бачив та відчував, що саме це небо з нескінченним туманом
Чумацького Шляху, а також Південний Хрест, гори, освітлені яскравим, але
холодним місяцем, дивляться на мене, як дивилися на Роберта Скотта під час його
останньої подорожі до полюса, як дивилися на експедицію Шарко, що застрягла на
дерев’яному вітрильнику в льодовому полоні біля Антарктичного півострова. І
якщо гарненько прислухатися, то напевне можна почути голоси тих людей, які
побували тут багато років тому. Для Антарктиди сто, двісті років --- це не
термін. Вона зберігає пам’ять про події та таємниці багатьох, хто хоч раз
побував тут.

Хлопці, які вперше потрапляють на станцію в Антарктиді, захоплюються видами
гірських вершин і проток, дивуються тваринам і птахам, які живуть у цих краях.
Багато хто не встигає за рік насолодитися цим експедиційним життям і не
приховує сліз, коли на прощання махає рукою новій зміні, що залишилася на
скелястому березі, повертаючись додому «до суєти міст і потоків машин»… 

Що ми тут знаходимо? Чому так солодко усвідомлювати, що ти --- свій в Антарктиді?
Чому нас так тягне сюди знову і знову? Чому ми заздримо, з нотками ревнощів
душі, тим, хто залишився після нас на наступну експедицію? Напевне, потрібно
про це розпитати у кожного… Загальної формули немає.

Месснер знайшов тут рай. І в своїй захопливій книзі він називає себе не
підкорювачем і не географом. Він уважає себе людиною, яка намагається ужитися з
дикою природою, щоб показати іншим те, чого вони не мають можливості побачити
на власні очі і здійснити самі. Мета його подорожі недоторканими куточками
природи --- стати її очевидцем, навчити мільйони людей поважати її, любити і
захищати.

«Ми не потрібні природі. Природа потрібна нам!» --- ось кредо автора, яке він
хоче донести своїм читачам. 

Олександр Міхо,

полярник, учасник п’яти антарктичних експедицій
