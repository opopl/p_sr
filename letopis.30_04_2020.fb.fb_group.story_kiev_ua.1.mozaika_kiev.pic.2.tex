% vim: keymap=russian-jcukenwin
%%beginhead 
 
%%file 30_04_2020.fb.fb_group.story_kiev_ua.1.mozaika_kiev.pic.2
%%parent 30_04_2020.fb.fb_group.story_kiev_ua.1.mozaika_kiev
 
%%url 
 
%%author_id 
%%date 
 
%%tags 
%%title 
 
%%endhead 

\begin{fminipage}{0.45\textwidth}

\ifcmt
	ig https://scontent-frt3-1.xx.fbcdn.net/v/t1.6435-9/94979444_3174550545911864_3650300221063168000_n.jpg?_nc_cat=108&ccb=1-5&_nc_sid=b9115d&_nc_ohc=QSCiuFk4lYcAX-WRKva&_nc_ht=scontent-frt3-1.xx&oh=51983367466c588afde9034666ea8060&oe=61B6445D
	@width 1.0
\fi

\iusr{Наталья Крутенко}
Подземный переход на Майдане. Автор - Александр Ворона

\iusr{Ирина Петрова}

Звісно! Це вони! А чи пам'ятаєте Ви ще чекані роботи у маленьких кафе , такі
були чотири, зараз в цих приміщеннях теж в одному чи двох кафе, сувенірні
авочки. Ті чеканки вже зникли. Але одна була дуже популярна.)))

\iusr{Наталья Крутенко}
\textbf{Ирина Петрова} увы... если их автор также Ворона, найдите в ФБ его дочь Ольгу, поинтересуйтесь у нее

\iusr{Ирина Петрова}
Може, й вона, а можливо іншій автор, бо то ж були чеканки... ала спробую, дякую!

\iusr{Виктория Цьопа}
Дякую! З дитинства люблю ці мозаїки, майоліку, чеканку.

\end{fminipage}
