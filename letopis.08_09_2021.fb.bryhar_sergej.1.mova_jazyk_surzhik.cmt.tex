% vim: keymap=russian-jcukenwin
%%beginhead 
 
%%file 08_09_2021.fb.bryhar_sergej.1.mova_jazyk_surzhik.cmt
%%parent 08_09_2021.fb.bryhar_sergej.1.mova_jazyk_surzhik
 
%%url 
 
%%author_id 
%%date 
 
%%tags 
%%title 
 
%%endhead 
\subsubsection{Коментарі}

\begin{itemize}

% -------------------------------------
\ii{fbauth.bryn_vladimir.lvov.ukraina}
% -------------------------------------
 
Повага за цю боротьбу. Де "моя хата скраю - першим в ворога стріляю."

\ifcmt
  ig https://scontent-frx5-2.xx.fbcdn.net/v/t39.1997-6/s480x480/12385791_539003332919315_2129379260_n.png?_nc_cat=1&ccb=1-5&_nc_sid=0572db&_nc_ohc=QS3HA4n3sTEAX-oO9lU&_nc_ht=scontent-frx5-2.xx&oh=545f7c328cece8f50c0078ba30a6a435&oe=613FFFD4
  @width 0.2
\fi

%%%fbauth
%%%fbauth_name
\iusr{Лиза Ребар}
%%%fbauth_url
%%%fbauth_place
%%%fbauth_id
%%%fbauth_front
%%%fbauth_desc
%%%fbauth_www
%%%fbauth_pic
%%%fbauth_pic portrait
%%%fbauth_pic background
%%%fbauth_pic other
%%%fbauth_tags
%%%fbauth_pubs
%%%endfbauth
 

Звісно, можна повчитися на українській філології та отримати другу вищу освіту.
Навіть для себе, але нема часу, бо вдома треба здобувати другу середню освіту.
Вас теж це чекає. З міжнародним днем грамотності!

\begin{itemize}
%%%fbauth
%%%fbauth_name
\iusr{Serhii Bryhar}
%%%fbauth_url
%%%fbauth_place
%%%fbauth_id
%%%fbauth_front
%%%fbauth_desc
%%%fbauth_www
%%%fbauth_pic
%%%fbauth_pic portrait
%%%fbauth_pic background
%%%fbauth_pic other
%%%fbauth_tags
%%%fbauth_pubs
%%%endfbauth
 
\textbf{Ліза Ребар} Дякую! Навзаєм \Smiley[1.0][yellow]! Гарна дата для гарного допису))).
\end{itemize}

%%%fbauth
%%%fbauth_name
\iusr{Prymara Makeeva}
%%%fbauth_url
%%%fbauth_place
%%%fbauth_id
%%%fbauth_front
%%%fbauth_desc
%%%fbauth_www
%%%fbauth_pic
%%%fbauth_pic portrait
%%%fbauth_pic background
%%%fbauth_pic other
%%%fbauth_tags
%%%fbauth_pubs
%%%endfbauth
 
узкоязичні мовознавці... на яких філфаках вчили мову? проффєсура, курвочкаряба...

%%%fbauth
%%%fbauth_name
\iusr{Сергій Лащенко}
%%%fbauth_url
%%%fbauth_place
%%%fbauth_id
%%%fbauth_front
%%%fbauth_desc
%%%fbauth_www
%%%fbauth_pic
%%%fbauth_pic portrait
%%%fbauth_pic background
%%%fbauth_pic other
%%%fbauth_tags
%%%fbauth_pubs
%%%endfbauth
 
Остання фраза варта популяризації: будь яка українська краща, ніж мова окупанта. Невже будуть заперечення?

%%%fbauth
%%%fbauth_name
\iusr{Наталія Квасницька}
%%%fbauth_url
%%%fbauth_place
%%%fbauth_id
%%%fbauth_front
%%%fbauth_desc
%%%fbauth_www
%%%fbauth_pic
%%%fbauth_pic portrait
%%%fbauth_pic background
%%%fbauth_pic other
%%%fbauth_tags
%%%fbauth_pubs
%%%endfbauth
 
@igg{fbicon.thumb.up.yellow}{repeat=3}

%%%fbauth
%%%fbauth_name
\iusr{Oleksandr Chumak}
%%%fbauth_url
%%%fbauth_place
%%%fbauth_id
%%%fbauth_front
%%%fbauth_desc
%%%fbauth_www
%%%fbauth_pic
%%%fbauth_pic portrait
%%%fbauth_pic background
%%%fbauth_pic other
%%%fbauth_tags
%%%fbauth_pubs
%%%endfbauth
 

Я упоротим завжди кажу, що російська це висока і вишукана мова, тому російською
треба або говорити тільки літературною, або не говорити нею взагалі.

%%%fbauth
%%%fbauth_name
\iusr{Ntina Ntoubrova}
%%%fbauth_url
%%%fbauth_place
%%%fbauth_id
%%%fbauth_front
%%%fbauth_desc
%%%fbauth_www
%%%fbauth_pic
%%%fbauth_pic portrait
%%%fbauth_pic background
%%%fbauth_pic other
%%%fbauth_tags
%%%fbauth_pubs
%%%endfbauth
 

Справа в тому, що певна кількість наших співвітчизників дуже зрусифікована. І в
них дійсно побутує якась дивна ілюзія, що україномовні люди "придурюються",
"вони неправі". Такі є люди, зрусифіковані в 2-3му поколінні, зокрема ті, які в
маленькому віці були перевезені до міста або народилися у подружжі таких
недавно переїхавших, що вже втратили контакти із сільською ріднею.

І ще цікавий момент з моїх спостережень... Якщо україномовні співвітчизники
завжди сприймали та до війни 2014го року поблажливо з російськомовними
спілкувалися та відчували їх рідними "бо в нас один паспорт! бо в нас єдина
батьківщина! бо ми з тобою змалечку знайомі та дружимо!". То зрусифіковані
відчувають себе в україномовному середовищі як справжні імігранти, що потрапили
до іншої країни і їм досить незручно і неприємно в цьому знаходитися. А нюанс
міститься у тому, що тема іміграції в нашій совєцькій літературі, що
російської, що української, ніколи в пострадянських школах не розбиралася з
позитивної та проактивної точки зору, тільки через осуд або співчуття. По ТБ та
в ЗМІ про неї не говорять інакше як через симпатію та підтримку людини, що не
може знайти своє місце в новому для себе оточенні. Дуже нечасно наші
співвітчизники передивляючись фільми на тему страждань імігрантів із невмінням
влитися в нове суспільство - відчуватимуть бажання обуритися та заперечити: "ну
навіщо ти приїхав/ла тоді? або вертай назад - або змінюйся, бо ти тепер вже
маєш тут жити! чому хтось має під тебе підлаштовуватися??"..

Тому зросійщені вже ментально і через встановлений таким культурний світогляд
"заточені" на те, щоб відчайдушно відстоювати свою ідентичність від зазіхань. І
саме через такі "внутрішньо-імігрантські" відчуття в 2014му році кримчани у
першу чергу відчули любов до окупанта.

Якби була в нас представлена масово література про те, як імігрант стає "своїм"
в новому середовищі (а вона є!!) - може, зрусифікованим було б легше влитися до
загальноукраїнського моря. Хоча б на рівні відчуття себе "одним із нас",
замість протистояння "я і вони".

\begin{itemize}
%%%fbauth
%%%fbauth_name
\iusr{Roman Romanow}
%%%fbauth_url
%%%fbauth_place
%%%fbauth_id
%%%fbauth_front
%%%fbauth_desc
%%%fbauth_www
%%%fbauth_pic
%%%fbauth_pic portrait
%%%fbauth_pic background
%%%fbauth_pic other
%%%fbauth_tags
%%%fbauth_pubs
%%%endfbauth
 
\textbf{Ntina Ntoubrova} ну це один із пазлів. але важливий да!
\end{itemize}

%%%fbauth
%%%fbauth_name
\iusr{Світлана Троцюк}
%%%fbauth_url
%%%fbauth_place
%%%fbauth_id
%%%fbauth_front
%%%fbauth_desc
%%%fbauth_www
%%%fbauth_pic
%%%fbauth_pic portrait
%%%fbauth_pic background
%%%fbauth_pic other
%%%fbauth_tags
%%%fbauth_pubs
%%%endfbauth

\ifcmt
  ig https://scontent-frt3-1.xx.fbcdn.net/v/t39.1997-6/s180x540/104088995_259375198670368_6998486030201420399_n.png?_nc_cat=106&ccb=1-5&_nc_sid=ac3552&_nc_ohc=ore_2vKtSgcAX-Fl9J7&tn=lCYVFeHcTIAFcAzi&_nc_ht=scontent-frt3-1.xx&oh=b97fd08e6abd82b0cb92919ab474bee0&oe=613EA8FA
  @width 0.3
\fi

% -------------------------------------
\ii{fbauth.bilenjkij_nazarij.kiev.ukraina}
% -------------------------------------


 

Цікава позиція) Але, знаючи скільки ще цієї бидло-маси і скільки нових стають
дорослими, то я веду більш жорстку політику з ними, відразу на..й, якщо клепки
трохи є саме зрозуміє, якщо немає, то не варто і вступати в дискусію! @igg{fbicon.wink} 
\end{itemize}

