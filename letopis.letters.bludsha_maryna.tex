% vim: keymap=russian-jcukenwin
%%beginhead 
 
%%file letters.bludsha_maryna
%%parent letters
 
%%url 
 
%%author_id 
%%date 
 
%%tags 
%%title 
 
%%endhead 

Доброго дня, пані Марино! Пишу Вам з приводу Вашого путівника Путівник по
Маріуполю. Pictures inside us. Сьогодні натрапив на посилання в одному пості.
Як я розумію, він був знищений через те, що Тільда - роскомпанія, і це звісно
правильно, що не треба підтримувати роскомпанії під час війни (я побачив
повідомлення на адресі сайту, що цей сайт був знищений). Але, на жаль, я не
зміг знайти ніде копію сайту, і це дуже неймовірно сумно, оскільки наскільки я
бачу, це була просто неймовірно красива робота по Маріуполю. Щодо мене - мене
звати Іван, я киянин, програміст, живу і працюю в Києві. У мене є проєкт
Літопису Війни, де я займаюсь систематизацією та архивацією різних публікацій,
і взагалі кажучи, не тільки про війну. Останнім часом я активно займаюсь
Маріуполем... Як все це виглядає, можна подивитись ось тут
https://t.me/kyiv_fortress_1 або ж тут
https://archive.org/details/@kyiv_chronicler (викладено лише насправді мала
частина архівованих матеріалів). Якщо у Вас або Ваших друзів/колег залишились
матеріали сайту, я думаю, було б добре перенести сайт на якусь іншу платформу
або взагалі переписати заново, із тими самими матеріалами, бо задумка дійсно
чудово. І це особливо важливо, оскільки Маріуполь, на мій погляд, потребує не
тільки чисто військового звільнення, а також культурно-духовного відродження.
Так, фізично він був знищений, але в духовному просторі він залишився, у
вигляді спогадів, фото, книжок, постів на фб. Але... якщо не займатись
систематично збереженням спадщини Маріуполя, то існує, на мій погляд, велика
загроза в тому, що Маріуполь буде забутий в культурно-духовній площині також, і
це буде означати фактично духовну смерть Маріуполя. І зокрема, як одна ознак
цього процесу... я майже не бачу книжок про Маріуполь в Києві. Оббігав купу
книгарень - книжок про Маріуполь або немає зовсім, або одна-дві назва
(невеличкі книжки Сухорукової та Шишацького - про трагічні події), а про
красивий, гарний, мирний Маріуполь - зовсім немає. Тому я і займаюсь зараз
Маріуполем, записую, систематизую. Маріуполь - неймовірно гарне і красиве
місто, справжня Перлина біля Моря! - яке неодмінно знову воскресне і засяє
новими барвами! Але щоб цього досягти, треба ще дуже багато роботи попереду...
А щодо сайту, буду дуже радий, якщо первісні матеріали збереглися, і якщо є
можливість відтворити сайт на новій платформі на новій адресі, або ж хоча б
заархівувати та опублікувати матеріали (не в формі сайту, а в формі книжки - я
це так роблю ). З повагою, Іван.
