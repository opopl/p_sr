% vim: keymap=russian-jcukenwin
%%beginhead 
 
%%file 07_06_2021.fb.arestovich_alexei.3.myshlenie_organizacia_pobeda
%%parent 07_06_2021
 
%%url https://www.facebook.com/alexey.arestovich/posts/4397816976949015
 
%%author Арестович, Алексей
%%author_id arestovich_alexei
%%author_url 
 
%%tags 
%%title Побеждает мышление, организованное выше, чем предыдущее
 
%%endhead 
 
\subsection{Побеждает мышление, организованное выше, чем предыдущее}
\label{sec:07_06_2021.fb.arestovich_alexei.3.myshlenie_organizacia_pobeda}
\Purl{https://www.facebook.com/alexey.arestovich/posts/4397816976949015}
\ifcmt
 author_begin
   author_id arestovich_alexei
 author_end
\fi

- Побеждает мышление, организованное выше, чем предыдущее. 
Сложное мышление позволяет организовывать более сложные действия. 
Более сложные действия ведут к высоким результатам.
Система образования застряла в модерне и имеет своим идеалом, людей, которые на одинаковые вопросы дают одинаковые ответы - винтики большого механизма большого производства.

\ifcmt
  pic https://scontent-cdg2-1.xx.fbcdn.net/v/t1.6435-9/197909956_4397811090282937_271380226913815814_n.jpg?_nc_cat=104&ccb=1-3&_nc_sid=8bfeb9&_nc_ohc=2q-nNSD25MwAX9kMuIA&_nc_ht=scontent-cdg2-1.xx&oh=6bb466608f2a85e5a8433e56ffd6c6eb&oe=60E58B0B
\fi

И как не зови «думать за пределами коробки», выйдя из коробки, ты обнаруживаешь себя в том же цеху.
Мыслить - это получать в результате мышления новое знание, которого не было у Вас до начала мышления.
Следует различать «думание» и «мышление».
И то, и другое есть выведение одних умозаключений из других и сведение их к возможно простому, но думанье - это рекомбинация в своём сознании уже известных Вам знаний, а мышление - получение для себя нового знания. 
Большинство людей думает, но мыслить не умеет, что технически легко заметно по тому, что результаты их умственных упражнений, не бывают для них неожиданными: результат процесса известен заранее и является подгонкой фактов и логики под заранее сформированную установку (интеллектуальную или эмоциональную). 
Признаком успешного мышления является получение неожиданного результата для себя самого - нового знания, которого Вы не ожидали. 
Вот этим и займёмся.
Начнём с того, чего, увы, практически не найдёшь в нашей культуре - понимания, что такое - предмет мышления. Только осознавая предмет и осознанно работая с ним, мышление становится управляемым и не рискует провалиться в думание.
Далее, модуль состоит из пяти занятий.
Мы будем изучать с Вами теорию и практику по следующим темам:
- искусство наблюдать и делать выводы: наблюдение - суждение - оценка - вывод,
- формула поиска наилучшего объяснения, 
- субъектно-объектная матрица: субъект действия, объект, связь, процесс, фон, условия: раскладываем ситуацию на игроков, их связи и мотивы,
- умение задавать вопросы и некоторые другие техники. 
В результате Вы научитесь правильно отбирать золото из атакующего нас информационного хаоса, структурировать его и собирать в слитки нового знания, сами организовывать свой интеллектуальный поиск, пользоваться своим сознанием, как целевым, организованным инструментом. 
————-
Модуль стартует 19 июня, будет проходить по субботам, раз в две недели, с 17.00 до 20.00, с домашними заданиями и обратной связью.
Записаться:
рус:
https://go.apeiron.school/modul-IM-1906-ru-fb
укр:
https://go.apeiron.school/modul-MM-1906-ua-fb
Администратор — Татьяна Кондратенко
почта kondratenko@apeiron.school
https://www.facebook.com/kondratenkotanya
+380509948403
