% vim: keymap=russian-jcukenwin
%%beginhead 
 
%%file 30_01_2014.fb.dykyj_evgen.1.para_bellum
%%parent 30_01_2014
 
%%url https://www.facebook.com/evgen.dykyj/posts/10152128663248808
 
%%author_id dykyj_evgen
%%date 
 
%%tags janukovich_viktor,maidan2,ukraina
%%title PARA BELLUM
 
%%endhead 
 
\subsection{PARA BELLUM}
\label{sec:30_01_2014.fb.dykyj_evgen.1.para_bellum}
 
\Purl{https://www.facebook.com/evgen.dykyj/posts/10152128663248808}
\ifcmt
 author_begin
   author_id dykyj_evgen
 author_end
\fi

PARA BELLUM

Отже, Овоч обрав війну. Це доконаний факт, і тепер не так важить на що
розраховував він, чому його послухались \enquote{риги} та все інше бла-бла.

Є факт: Янукович показав, що абсолютно ні фіга не зрозумів та не розуміє в тому
що відбувається в Україні. Він досі вважає, що сила за ним і у нього, і мабуть
вважає що і так пішов на офігенні поступки нам.

Війна тепер неминуча та невідворотна. Її вибрали не ми, але нам її вести та
вигравати.

Дуже добре, що Хам зайняв настільки категорично-дебільну позицію, що навіть
найслабші з оппо не змогли піти на угоду з ним. Отже, воюємо єдиним фронтом, що
значно краще.

Чи виграємо? Залежить лише від нас. Ресурси режиму добре зрозумілі,
прораховуються на "раз-два" і на щастя дуже не безмежні. Направду їхньої
фізичної сили весь цей тиждень ледь вистачало на утримання оборони Печерську та
репресії по окремих областях, де наших активістів досить багато щоб створити
проблему, але ще не так багато щоб утримати позиції в бою. Ці сили режиму
виснажені, і навряд чи здатні на активну атаку, а також на тривале
протистояння. Але на спробу останнього ривка в два етапа - спершу зачистка
центральних областей (на Західну поки не попхаються, стрьомно) і потому
відчайдушна спроба придушити Київ - ще здатні.

А от наш ресурс дуже невизначений, бо виключно добровільний. Ми можемо інколи
виставити півмільйона, а інколи заледве збираємо пару тисяч, і наперед не
знаємо скільки \enquote{підірветься} кожного разу. Наша сила - величезний
мобілізаційний ресурс, наша слабкість - його неорганізованість та
непередбачуваність.

Перебіг громадянської війни (а вже пора змиритись з тим, що йдеться саме про
неї) повністю залежить від того, як швидко та як масово ми мобілізуємось в ці
визначальні дні.

Якщо завтра (вибачте - вже кілька хвилин як \enquote{сьогодні}) попри дуже невчасний
мороз на вулицях Києва будуть кількадесят тисяч розлючених громадян, у пятницю
такі само розлючені громадяни підірвуться зі своїх сіл та містечок на свої
обласні центри в центральних областях, а із Західної на вихідні масово приїдуть
підтримати киян - тоді вже у суботу чи неділю півмільйонне чи більше віче зможе
спокійно оголосити режим поваленим, кількатисячна колона демонстрантів зайде до
Міжгіря та прийме відставку Хряка. Це пройде швидко і майже без крові, бо проти
десятків та сотен тисяч людей менти безсилі, добре це знають і особливо не
рипатимуться. Громадянська війна почнеться у четвер та закінчиться до
понеділка.

Якщо ж мороз та втома завадить масовій мобілізації - тоді начуваймось. Тоді
героям Самооборони та Правого сектору довдеться прийняти справжні жорстокі бої
з переважаючим технікою та озброєнням ворогом, а у областях замість швидкого
захоплення влади доведеться партизанити, з величезними жертвами.

Так що і тривалість боротьби, і кількість можливих жертв зараз визначає кожний
з нас персонально. Тим, вийде він ТУТ І ЗАРАЗ, чи чекатиме ще невідомо чого та
доколи.

Готуймось до обох варіантів, зокрема готуймо нашу адекватну відповідь на спробу
застосувати бойову зброю (гадаю, вже нікого не треба переконувати, що Овоч
віддасть такий наказ? чи ще у когось лишились ілюзії?) - відповідь країни
мисливців, відповідь країни любителів коктейльних вечірок, відповідь країни яка
знає що у кожному захудалому райвідідлку лежить автоматика по кількості
працівників, а міліцейські патрулі на вулицях здуру озброїли пістолями -
коротше, готуймось вигравати повноцінну війну. Не тому що ми її хочемо, а саме
тому що ми хочемо миру - справжнього миру, при якому нам не треба боятись
ментів та бандитів, платити хабарі, працювати на кримінальну корумповану кодлу.
Нам потрібний лише ТАКИЙ мир, і саме за нього ми готові воювати.

Обєднуємось у боївки з тими, кого добре особисто знаємо, друзями, родичами,
товаришами, готуймось діяти автономно, лише орієнтуючись на загальний перебіг
подій, якщо не буде наказів Штабу. Формуємо загони Національної Гвардії у
кожному селі - містечку - місті - районі - районі у місті, і виконуємо всі
накази Штабу Гвардії. Жарти скінчились, це вже не Євромайдан і навіть не
Революційний Майдан - це громадянська війна, війна громадян проти злочинного
режиму заколотників та узурпатора.

Але памятаємо - ця війна ще може бути майже безкровною, якщо нас вийде
достатньо багато. Чим більше нас буде на вулиці в ці вирішальні морозні дні -
тим менше генералів ризикне наказати \enquote{вогонь!}, і ще тим менше солдатів будуть
готові такий наказ виконати. Пяти тисячам доведеться вести страшний нерівний
бій, пятидесяти - піти на відчайдушний прорив з тяжкими втратами, але з
гарантією перемоги, проти пятисот тисяч навіть не рипнуться.

Так що Овоч зробив СВІЙ вибір, але поки що це не вирок всім нам. Ми зараз самі
обираємо, якою буде ця війна, як скоро та якою ціною дістанеться НАШ мир.

Вперед. До зброї. Мобілізація.
