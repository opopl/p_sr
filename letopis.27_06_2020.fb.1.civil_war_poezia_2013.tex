% vim: keymap=russian-jcukenwin
%%beginhead 
 
%%file 27_06_2020.fb.1.civil_war_poezia_2013
%%parent 27_06_2020
 
%%url https://www.facebook.com/kate.zharkih.5/posts/4571748719517863
 
%%author 
%%author_id 
%%author_url 
 
%%tags 
%%title 
 
%%endhead 

\subsection{На пороге гражданской войны}
\url{https://www.facebook.com/kate.zharkih.5/posts/4571748719517863}

Мы на пороге гражданской войны.
Мы приближаемся к бездне.
Перед глазами серой стены 
Слушаем серые песни.
Их же пытаемся и повторить,
Хоть голосов не имеем.
Пафосной логики тонкую нить
Лепим к речам быстрым клеем.
Мы на пороге гражданской войны.
Чуть ли не трем воском сабли.
Нам, почему-то, уже не страшны
Слез и другой жижи капли.
Нам, почему-то, уже все равно,
Что будет дальше со всеми.
Ведь в головах у нас лишь одно - 
Чести и гордости бремя.
Ведь, очерствев до немой слепоты,
Мы заигрались в серьезность.
Дальше – мгновение до пустоты,
Ну а назад – невозможно.
Словно на разных концах молний мы
Носимся, хаос лелея.
Сумели создать новый вид сверх тюрьмы.
Психика – узник модели.
Мы на пороге гражданской войны.
В будущем ждут еще войны.
Мы никому ничего не должны.
И ничего не достойны.
Декабрь, 2013
Сегодня у моего лучшего друга и автора этого стиха Gabriel Gabrilchuk
 День Рождения. Ценю дружбу с тобой и горжусь тобой. Развивай свой талант. Будь счастлив!
