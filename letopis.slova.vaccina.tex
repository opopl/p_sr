% vim: keymap=russian-jcukenwin
%%beginhead 
 
%%file slova.vaccina
%%parent slova
 
%%url 
 
%%author 
%%author_id 
%%author_url 
 
%%tags 
%%title 
 
%%endhead 
\chapter{Вакцина}
\label{sec:slova.vaccina}

%%%cit
%%%cit_head
%%%cit_pic
%%%cit_text
Результаты такого подхода возвращаются сейчас бумерангом. Нищая перенаселенная
Индия стала идеальным инкубатором для формирования новых, более заразных
штаммов, которые уже пришли в Европу и США, провоцируя новую волну пандемии.
Очевидно, что такая же угроза потенциально исходит с территории других бедных
стран, с высоким уровнем населения и низкой доступностью медицинского
обслуживания – как в государствах Африки, Юго-Восточной Азии, Ближнего Востока
и Латинской Америки. Для победы над коронавирусом необходимо делать упор на
\emph{вакцинировании} жителей Третьего мира, предоставив им свободный доступ к
качественным \emph{вакцинам} – вместо того, чтобы делать из этого привилегию для
европейцев, американцев и их ближайших союзников
%%%cit_comment
%%%cit_title
\citTitle{Для победы над коронавирусом надо было вакцинировать Третий мир / Лента соцсетей / Страна}, 
Андрей Манчук, strana.ua, 24.06.2021
%%%endcit

%%%cit
%%%cit_head
%%%cit_pic
%%%cit_text
Для реализации этой схемы фармацевтические корпорации вытесняли с
международного рынка конкурентов, в лице \emph{вакцины} «Спутник V» и препаратов из
КНР – причем, эту дискриминацию зачастую объясняют причинами политического
характера. В то же время, Европа и США отказывались приостановить действие
патентов на свои \emph{идеологически правильные вакцины} – чтобы их можно было
свободно производить в Третьем мире. Прежде всего, потому, что это лишило бы
корпоративный менеджмент запланированных доходов
%%%cit_comment
%%%cit_title
\citTitle{Для победы над коронавирусом надо было вакцинировать Третий мир / Лента соцсетей / Страна}, 
Андрей Манчук, strana.ua, 24.06.2021
%%%endcit

%%%cit
%%%cit_head
%%%cit_pic
%%%cit_text
\enquote{Страна} разбиралась в первой смерти после \enquote{Пфайзера}, которая
произошла в Украине. И напоминает, какие риски может нести в себе эта
\emph{вакцина} - и как их избежать.  Что сообщил Минздрав.  Вчера, 2 июля,
после \emph{вакцинации} препаратом Pfizer умер житель Винницкой области.
Мужчине было 47 лет.  Хронология событий такая. В 13:45 пациент получил первый
укол американской вакцины. И в течение часа после него никаких жалоб на
самочувствие не поступало.  Но в 17:00 - то есть через три с лишним часа -
мужчина почувствовал тошноту и потерял сознание. Приехавшая \enquote{скорая}
пыталась реанимировать человека, но к 18 часам он скончался
%%%cit_comment
%%%cit_title
\citTitle{Pfizer - первая смерть в Украине после вакцинации. Что об этом известно}, 
Максим Минин, strana.ua, 03.07.2021
%%%endcit

%%%cit
%%%cit_head
%%%cit_pic
%%%cit_text
Стосувалася варіоляція виключно віспи. Ніхто навіть не думав, що щось подібне
можна застосувати до чогось іншого. Ми знаємо про варіоляцію тому, що приблизно
в 1719 році англійці завезли її з Туреччини до себе. У Великобританії щепився
король Георг І та його діти. Потім вона потрапила до Франції, звідти до Росії.
Відомо, що Катерина ІІ щепилася сама і щепила свого сина Павла і всіх своїх
онуків. Більше того, Катерина ІІ створила віспяні дома, куди можна було прийти
будь-кому і щепитися таким чином. Такий дім був відкритий і в Києві. У Києві за
кожне щеплення держава платила 1 срібний рубль. Не той, хто щепився, мав
платити, а йому платили за те, що він прийшов.
– А це ж пристойні гроші, як на другу половину XVIII століття!
– Це велика сума! Люди за 3 рублі працювали цілий рік. А тут отримуєш рубль
просто за те, що щепився проти віспи. Проте, повторюю, це була дуже небезпечна
процедура. І щойно винайшли альтернатива, від цієї процедури відмовилися.
Альтернативою була власне \emph{вакцинація}. Тобто щеплення матеріалом від хворих на
коров’ячу віспу. Це теж було винайдено не одномоментно. Кілька людей у цьому
приймали участь
%%%cit_comment
%%%cit_title
\citTitle{Історія вакцинації: від віспи до коронавірусу}, 
Дмитро Шурхало, www.radiosvoboda.org, 30.10.2021
%%%endcit

%%%cit
%%%cit_head
%%%cit_pic
%%%cit_text
Интересно, а когда разработают вакцину от жажды наживы?  Ведь практически весь
наш олигархат заражен этой опасной болезнью, которая \enquote{вынуждает} их строить
схемы грабежа людей, приумножать капиталы, которые даже их правнуки не смогут
потратить, уничтожать природу для извлечения прибыли и подменять
государственные интересы, своими шкурными договорняками.  Очевидно, что
\enquote{вакцинация} от 1917-го года, уже утратила \enquote{потенциал антител}, срочно нужна
\enquote{вторая доза}...
%%%cit_comment
%%%cit_title
\citTitle{Интересно, а когда разработают вакцину от жажды наживы?  / Лента соцсетей / Страна}, 
Дмитрий Василец, strana.news, 08.11.2021
%%%endcit
