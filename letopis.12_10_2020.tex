% vim: keymap=russian-jcukenwin
%%beginhead 
 
%%file 12_10_2020
%%url https://www.pravda.com.ua/articles/2020/10/12/7269382/
%%parent articles
 
%%endhead 

Андрій Федорів: Україна – це свобода, вольниця, бардак, хаос

\url{https://www.pravda.com.ua/articles/2020/10/12/7269382/}

Кажуть, він різкий і категоричний. Якщо вважає, що проект – лайно,
співпрацювати не буде.

За це його не люблять.

"Якщо ти думаєш, що світ – це місце, де живуть виключно твої конкуренти, ти не
розвиваєшся сам. І не розвиваєш ринок", – повторює він.

Те, що він має талант змінювати середовище навколо себе, визнають навіть ті,
хто його не любить.

Перелік того, що створив Андрій Федорів може бути задовгим. Він – засновник
креативної інвестиційної компанії Fedoriv Group, маркетингової агенції Fedoriv
Agеncy, Фундації Дарини Жолдак, бізнес-простору Kooperativ, каналу Fedoriv Vlog
на YouTube, автор навчального курсу для власників бізнесу BrandFather.

У 2019 році спільно з Міністерством цифрової трансформації України його агенція
створила бренд цифрової держави "Дія", метою якого є оцифровка та автоматизація
всіх державних послуг.

Федорів – один з найкращих в Україні фахівців з брендінгу. А ще він візіонер,
здатний відповісти не лише на питання "як?", але й на питання "куди?" і
"навіщо?".

Саме тому "Українська правда" запропонувала Андрію Федоріву розповісти, який
ребрендінг потрібен Україні і в чому наша конкурентна перевага.

В інтерв’ю УП Федорів розповів про Україну як стартап.

Про сильні ідеї, яким не будуть плескати в долоні, і які потрібно виборювати.

Про те, чому наші факапи є нашими можливостями.

Про те, чому шкодять порівняння на кшталт "Київ – це новий Берлін".

Про прозорі стіни сучасного світу і про те, чому ребрендінг України не має
сенсу, допоки "Беркут" б’є людей.

Про своє бачення перемоги і фотографії майбутнього.

Обережно: спойлер. 

Під час розмови жодного разу не прозвучало слово "зрада". 


