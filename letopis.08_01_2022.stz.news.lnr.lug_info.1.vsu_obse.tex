% vim: keymap=russian-jcukenwin
%%beginhead 
 
%%file 08_01_2022.stz.news.lnr.lug_info.1.vsu_obse
%%parent 08_01_2022
 
%%url https://lug-info.com/news/vsu-prepyatstvovali-rabote-bpla-obse-v-rajone-nizhnej-ol-hovoj-narodnaya-miliciya
 
%%author_id news.lnr.lug_info
%%date 
 
%%tags vsu,donbass,ukraina,vojna,obse,bpla,nm_lnr
%%title ВСУ препятствовали работе БПЛА ОБСЕ в районе Нижней Ольховой – Народная милиция
 
%%endhead 
\subsection{ВСУ препятствовали работе БПЛА ОБСЕ в районе Нижней Ольховой – Народная милиция}
\label{sec:08_01_2022.stz.news.lnr.lug_info.1.vsu_obse}

\Purl{https://lug-info.com/news/vsu-prepyatstvovali-rabote-bpla-obse-v-rajone-nizhnej-ol-hovoj-narodnaya-miliciya}
\ifcmt
 author_begin
   author_id news.lnr.lug_info
 author_end
\fi

ВСУ использовали средства радиоэлектронной борьбы (РЭБ) для препятствования
работе беспилотных летательных аппаратов (БПЛА) Специальной мониторинговой
миссии ОБСЕ вблизи села Нижняя Ольховая Станично-Луганского района. Об этом
сообщила пресс-служба управления Народной милиции ЛНР.

\enquote{В районе населенного пункта Нижняя Ольховая мобильная группа 20-го батальона
РЭБ с помощью станции \enquote{Буковель-АД} осуществляла подавление сигналов каналов
управления БПЛА миссии ОБСЕ в целях недопущения вскрытия наблюдателями
размещенной в данном районе военной техники ВСУ}, - говорится в сообщении.

Власти Украины начали силовую операцию против Донбасса в апреле 2014 года.
Урегулирование конфликта базируется на Комплексе мер по выполнению Минских
соглашений, подписанном 12 февраля 2015 года в белорусской столице участниками
Контактной группы и согласованном с главами стран - участниц \enquote{нормандской
четверки} (Россия, Германия, Франция и Украина). Документ, в частности,
предусматривает прекращение огня и отвод тяжелых вооружений от линии
соприкосновения.
