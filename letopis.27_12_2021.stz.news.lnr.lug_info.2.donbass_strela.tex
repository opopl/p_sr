% vim: keymap=russian-jcukenwin
%%beginhead 
 
%%file 27_12_2021.stz.news.lnr.lug_info.2.donbass_strela
%%parent 27_12_2021
 
%%url https://lug-info.com/news/novogodnij-blic-politolog-aleksandr-procenko-donbass-strela-letyashaya-v-cel
 
%%author_id 
%%date 
 
%%tags 
%%title Новогодний блиц. Политолог Александр Проценко: "Донбасс - стрела, летящая в цель"
 
%%endhead 
\subsection{Новогодний блиц. Политолог Александр Проценко: \enquote{Донбасс - стрела, летящая в цель}}
\label{sec:27_12_2021.stz.news.lnr.lug_info.2.donbass_strela}

\Purl{https://lug-info.com/news/novogodnij-blic-politolog-aleksandr-procenko-donbass-strela-letyashaya-v-cel}

Политолог, кандидат политических наук, доцент кафедры государственной политики
Луганского государственного университета имени Владимира Даля, член
Общественной палаты ЛНР, руководитель Центра социологических исследований
\enquote{Особый статус} Александр Проценко в традиционном новогоднем блиц-интервью ЛИЦ
делится своим видением итогов уходящего года и прогнозом на год грядущий.

\ii{27_12_2021.stz.news.lnr.lug_info.2.donbass_strela.pic.1}

- Как вы оцениваете положение и ситуацию в Республиках Донбасса к началу 2022
года?

- На мой взгляд, положение и ситуация в Республиках на начало 2022 года
достаточно серьезно отличается внутренне от того, что было год назад, и эти
отличия в лучшую сторону. В первую очередь это касается экономики, где
наконец-то произошли назревавшие годами изменения и фактически были разрублены
главные узлы, сдерживавшие нормальное функционирование и развитие нашей
промышленности.

Если говорить коротко и тезисно, то за год существенно оживилась ситуация в
экономике, пошел поток, что называется, свежего воздуха, и появились конкретные
перспективы улучшения ситуации.

Ну, и, конечно, магистральные процессы по росту количества российских граждан в
ЛНР и ДНР вместе с параллельным ростом их благосостояния также идут стабильно
приличными темпами.

- Что вы считаете наиболее значимыми достижениями Луганской и Донецкой Народных
Республик и каковы были самые серьезные проблемы в уходящем году?

- Событий знаковых было, достаточно много. Среди главных я выделил бы,
во-первых, прохождение на территории Республик полноценного голосования на
парламентских выборах в Госдуму РФ, что означает фактическое перемещение
Донбасса во внутриполитическую повестку страны. Во-вторых, это заход нового
инвестора в сферу крупной промышленности, формирование новой структуры – ЮГМК
(общества с ограниченной ответственностью \enquote{Южный горно-металлургический
комплекс}). В-третьих, это подписание договора о тесной экономической и
правовой интеграции ЛНР и ДНР, что уже привело к ликвидации таможни между нами.
Ну, и, в-четвёртых, под конец года очень нужный и важный указ о допуске нашей
продукции на рынки РФ подписал (президент Российской Федерации) Владимир Путин,
чего долгие годы ждал средний и крупный бизнес.

Если говорить о проблемах, то они остались привычные: возможность обострения со
стороны Киева в случае науськивания из-за границы или же вследствие внутренних
катаклизмов. Но, думаю, Донбасс к этому готов.

- Ваш прогноз развития ситуации вокруг Донбасса в 2022 году?

- Скажу образно и коротко. У китайцев есть такая пословица: стрела, летящая в
цель, на пейзаж не отвлекается. Так вот, Республики Донбасса сейчас как раз
представляют собой такую стрелу, уверенно летящую туда куда надо, и никакие
посторонние события, обстоятельств в новом году уже не повлияют на нашу
траекторию.

Года три назад в таком же блиц-опросе я говорил о том, что в обозримом будущем
мы должны демонстрировать привлекательность выбранного нами пути, стать
своеобразной точкой социальной аттракции для русских на Украине. Это касалось,
безусловно, и социально-экономического благосостояния людей. Мне кажется, за
эти три года многое у нас поменялось и в лучшую сторону в этом отношении, и в
новом году, думаю, эти изменения станут ещё более заметны.
