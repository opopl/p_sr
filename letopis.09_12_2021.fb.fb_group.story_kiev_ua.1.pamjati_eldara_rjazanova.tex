% vim: keymap=russian-jcukenwin
%%beginhead 
 
%%file 09_12_2021.fb.fb_group.story_kiev_ua.1.pamjati_eldara_rjazanova
%%parent 09_12_2021
 
%%url https://www.facebook.com/groups/story.kiev.ua/posts/1814822962047830
 
%%author_id fb_group.story_kiev_ua,zolotushkin_anatolij.hajfa
%%date 
 
%%tags kiev,pamjat,rjazanov_eldar
%%title Памяти Эльдара Рязанова
 
%%endhead 
 
\subsection{Памяти Эльдара Рязанова}
\label{sec:09_12_2021.fb.fb_group.story_kiev_ua.1.pamjati_eldara_rjazanova}
 
\Purl{https://www.facebook.com/groups/story.kiev.ua/posts/1814822962047830}
\ifcmt
 author_begin
   author_id fb_group.story_kiev_ua,zolotushkin_anatolij.hajfa
 author_end
\fi

Памяти Эльдара Рязанова.

Многое осталось в прошлой жизни. Плохое и хорошее. В том числе и походы в кино.
За 25 лет в Израиле я был в кинотеатре два раза и оба раза на \enquote{Титанике}. Так
получилось. Теперь набираешь название в интернете и смотри, что хочешь.

Одним из первых фильмов, который я посмотрел с мамой в кино была \enquote{Карнавальная
ночь}.  Как прекрасна была Гурченко!

\ii{09_12_2021.fb.fb_group.story_kiev_ua.1.pamjati_eldara_rjazanova.pic.1}

Новый год был в то время ещё настоящий, со стеклянными игрушками, ватным Дедом
Морозом и живой ёлкой.

Мы смотрели это волшебство в кинотеатре \enquote{Днiпро}. Он утопал в зелени парка на
днепровском склоне, а на наружной стене был нарисован павлин.

Какой это был праздник - поход с мамой в кино!

Мы шли через Крещатик, проходили мимо Пассажа, мама угощала меня мороженым в
галерее на втором этаже. 

Я гордился своей мамой - она была самая красивая. У неё даже была новая шляпка
с вуалеткой, по моде того времени.

Потом там же мы смотрели \enquote{Гусарскую балладу}

 - Корнет, вы женщина...

Примерно через 10 лет вышел главный советский хит - \enquote{Ирония судьбы}

Как созвучен он оказался всем!

В баню, правда,  я не ходил, но так  хотелось чуда любви. И его сотворил
великий мастер. 

Мы вместе с Рязановым переживали \enquote{Служебные романы},  мечтали  замолвить слово
за бедного гусара. Но жизнь оказалась \enquote{Жестоким романсом}.

Как вам там теперь на \enquote{Небесах обетованных}?

Снимаете ли вы там кино?

Мы придем, посмотрим.

\ii{09_12_2021.fb.fb_group.story_kiev_ua.1.pamjati_eldara_rjazanova.cmt}
