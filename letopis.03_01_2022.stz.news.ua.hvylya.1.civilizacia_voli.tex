% vim: keymap=russian-jcukenwin
%%beginhead 
 
%%file 03_01_2022.stz.news.ua.hvylya.1.civilizacia_voli
%%parent 03_01_2022
 
%%url https://hvylya.net/analytics/244706-civilizaciya-voli
 
%%author_id novahatskij_oleksandr,news.ua.hvylya
%%date 
 
%%tags volja,ukraina,civilizacia,ideologia,obschestvo,future,nacidea
%%title Цивілізація волі
 
%%endhead 
\subsection{Цивілізація волі}
\label{sec:03_01_2022.stz.news.ua.hvylya.1.civilizacia_voli}

\Purl{https://hvylya.net/analytics/244706-civilizaciya-voli}
\ifcmt
 author_begin
   author_id novahatskij_oleksandr,news.ua.hvylya
 author_end
\fi

\begin{zznagolos}
Ми виросли з «коротких штанців» державно-національних проектів, які задають
соціальні конфігурації простору людських стосунків у межах державних формацій.
\end{zznagolos}

\ii{03_01_2022.stz.news.ua.hvylya.1.civilizacia_voli.pic.1}

Люди часто забувають, що навколо них Світ Людей. І часто-густо починають шукати
відповіді на виклики сучасності не з самих себе, як головної причини, а з
наслідків від власного усвідомлення причин проблеми.

Сучасний світ знаходиться у глибокій кризі. Вона охоплює увесь простір людських
стосунків між Людиною та Державою. Механізм колективного виживання людей –
Держава, явно та зосереджено заважає розвиватись Людині.

Ми виросли з «коротких штанців» державно-національних проектів, які задають
соціальні конфігурації простору людських стосунків у межах державних формацій.
У своїй більшості, ми вже не сприймаємо порядок співжиття, що нав’язує
державний механізм насильства-примусу. Не сприймаємо його як цінність та
критичну необхідність. Маючи реальну можливість виживати «поза державою», ми
стали потребувати від державних утворень виконання меншої кількості соціальних
функцій.

На сьогодні, актуальними лишились Міждержавне представництво, Мілітарна
функція, Функція забезпечення правосуддя-справедливості (забезпечення роботи
Суду). Всі решта функцій більш дієво-успішно та вельми ефективно виконує так
зване «Громадянське суспільство» (низка Інститутів та організацій, не
пов’язаних із державними органами).

Саме конфігурація та функціональність структури державного утворення, сьогодні
є головним питанням у створенні умов для гідного, заможного та спокійного життя
Людини. Із цими задачами не можуть впоратись інструменти соціального
структурування, які застосовувались до цього часу. Сьогодні ідеології, партії
та релігійні інструменти працюють на вельми обмежену аудиторію.

Окрім того, ці інструменти жодним чином не захищають Людину від природніх лих
та загроз життю. Стихійні лиха та пандемії показують неспроможність сучасних
державних утворень бути ефективними захисниками. Держави постійно намагають
вирішити проблеми одних людей за рахунок інших. Цей підхід стає неприйнятним,
не етичним та викликає відразу у великої кількості людей сучасних. Держави
стали вельми некомфортними для життя людей, які не входять у державні
механізми.

Заглиблюючись у питання історії та практики виникнення людських структур для
ефективного виживання, нескладно помітити що найбільш стійкі та витривалі
людські утворення виживали через існування Спільноти. Специфічна людська
формація, що бере на себе відповідальність за долю підконтрольних територій та
людського простору і яка виробляє ефективну модель облаштування щоденного
життя. Так звана Економічна модель стосунків, що реалізовувалась Спільнотою
забезпечувала контрольований нею людський простір вигодою та сенсом перебування
у ньому. Як правило, ця модель стосунків базується на етичних уявленнях
Спільноти та культурних особливостях організації колективного життя. Вигода тут
повинна поєднуватись із етичним комплексом світоглядних уявлень Людини про
добро та зло, правильність та прийнятність.

Правила в такому людському просторі є наслідком Світоглядної моделі буття
Спільноти. Правила спрямовуються на підтримання порядку стосунків в рамках
обраної Спільнотою Економічної моделі. Механізмом забезпечення порядку, або
регулятором стосунків виступає соціальний механізм Держава. Отже Держава є
наслідком економічного бачення Спільноти. І саме Спільнота породжує Державу для
забезпечення порядку. І джерелом права та порядку виступає також Спільнота, а
не Держава. Такий порядок структурування людського простору прийнято називати –
Цивілізація.

У сучасному світі колективних стосунків ми повсякчасно спостерігаємо інший
порядок утворення соціального простору врегульованого життя. Часто-густо, саме
Держава (низка чиновників) намагається створювати суспільний простір і
відфільтровує людей на приналежність до цього простору. Із плином часу та
розвитком обставин, настає момент аби переглянути порядок створення
організованих просторів життя людей. Момент використовувати Цивілізаційну
логіку для впорядкування просторів «Людей розумних».

Україна, державне утворення, яке в сучасному світі є найбільш показовим
прикладом протистояння Людини та Держави. Сама правова основа державного
утворення (Конституція) визначає лише два типи юридичного життя в межах країни.
«Влада» та «Народ». Людський простір поділений виключно на ці два стани. При
цьому соціальне явище «Влада» детально та предметно описано текстом
Конституції, а «Народ» функціонально встановлюється у якості живильного поля,
джерелом матеріального благополуччя для «Влади».

Соціальна група «Влада» на власний розсуд та етику розпоряджається всіма
матеріальними цінностями та благами, які створює «Народ». Виходячи з юридичної
логіки така соціальна конструкція приречена на протистояння соціальних груп, бо
все що не є «Владою», є «Народом» і навпаки.

Соціальна група «Влада» привласнила собі інструмент регуляції стосунків –
механізм держави та почала виробляти сенси та способи організації життя для
усіх людей в контрольованому просторі. Ба більше, узурпувала саме право
визначати параметри людськості в нелюдський, машинний-механістичний спосіб. В
якому приналежність окремої людини до Людей та «Народу» неможливе, якщо немає
«довідки» (спеціального документу визнання) від «Влади».

Тридцять років існування цього геополітичного державного утворення показали
справедливість описаної логіки та призвели до визрівання соціальної ситуації,
яка визначається стало вираженою кризою стосунків між Державою та Людиною. Цей
конфлікт може бути вирішений, або перемогою однієї з сторін, або консенсусом та
переформатуванням структури соціального простору. Тобто перезаснування
державного утворення «Україна» неминуче та є наслідком природніх чинників, що
утворились внаслідок невдалого моделювання простору людських стосунків.

Якщо наслідувати саме Цивілізаційний підхід до структурування українського
людського простору, то першим питанням слід з’ясувати чи існує в реальності
«Українська Спільнота», яка є носієм саме Української ідентичності та культури.
Бо саме Ідентичність та Культура можуть стати основою для створення Економічної
моделі в рамках прийнятних для Спільноти норм поведінки та правил співжиття. Чи
існує «Українство» (унікальний комплекс світоглядних засад) об’єктивно і чи
достатньо у цього феномену-явища світоглядних засад для проявлення складної
структури впорядкування величезного географічного простору під назвою
Українські Землі.

Будь-яке дослідження питання культурних засад існування Українства виявляє
неймовірно виражену Свободу практично в усьому, що стосується розуміння Людини.
Також дослідження виявляють чітку зв’язаність та навіть ототожнення себе із
своєю землею, із своїм краєм, який сприймається як генератор стану Добробуту.
Розглядаючи ґенезу українського слова Воля, неважко побачити, що цим словом
позначається три явища, які в інших мовах та культурах позначаються окремими
словами. Воля-Земля-Простір, Воля-Концентрація Духу-Людина, Воля-Свобода. Це
один стан свідомості, при якому ці три поняття повинні означати один
нерозривний стан буття.

Цілком логічно що, ця особливість і є головним ідентифікуючим маркером що
вирізняє носіїв Українства. Сакральне зображення феномену Волі у вигляді
тризуба дає можливість розглядати висунуту теорію українства, як культурного
феномену, що зародився та сформувався у часи до проявлення відомих нам
державних утворень на теренах Земель Українських.

Три цінності Всесвіт, Людина та Свобода, які знаходяться у просторі принципу
Свободи, утворюють ідеологічну конструкцію, яка цілком придатна до проявлення у
вигляді регульованого законодавчими актами простору співжиття людей. При цьому
процес проявлення порядку життя за принципом Свободи Волі, жодним чином не
виділяє етноси або нації. Цей принцип придатний для будь-якої людини, яка його
дотримується. Це набір факторів носить характер культурного феномена та цілком
придатний для практичного застосування принципу Свободи Волі у створенні
регулярних, структурованих формацій.

Отже, якщо у сучасному людському просторі людей, які свідомо живуть в
Українських Землях існує автентичний культурний код, який дозволяє безпомилково
визначати його носіїв, то маємо усі підстави стверджувати про існування
Спільноти.

Разом із тим, сучасний стан речей в українському людському просторі, не
дозволяє говорити про домінуючу роль носіїв феномену Волі в утворені
формалізованого простору співжиття. Українська Спільнота (носії феномену Волі)
на сучасному етапі почали тільки-тільки усвідомлювати власну унікальність та
спроможність до створення власного порядку співжиття.

Але, об’єктивне існування культурного коду ідентичності дає всі можливості
відтворення формалізованого державного утворення на його підставі.
Універсальність та придатність феномену Волі до структурування простору
стосунків неодмінно стане в пригоді коли існуюча сьогодні система стосунків
«Держави Україна» прийде у повний і логічний занепад.

Проявлення соціального гравця на ім’я Український Народ справа найближчого
часу. І, сьогодні вже не існує сил, які можуть не допустити перезаснування
українського геополітичного проекту на підставах Українськості. Живий процес
відтворення культурного коду Українства у формалізовану державну формацію
відбудеться обов’язково через проявлення живої етичної суті носіїв феномену
Волі.

Людино-центрична суть феномену Волі не лишає варіацій щодо трактування мети
існування такої Спільноти. Головним аспектом такого простору буде Людина, її
права та свободи, а також Економіка, спрямована на максимальну реалізацію
людськості. Наслідком існування такої формалізованої конструкції буде поява у
формалізований світ стосунків певних вимог до розвитку внутрішнього стану
Людини. Така державна формація буде сприяти проявленню Людей особливої якості в
параметрах людськості-людяності.

Світ людей, нарешті має змогу «повернутись обличчям» до первісної причини свого
виникнення. Нарешті приділити достатньо уваги щодо створення умов для гідного
існування та розвитку людських істот.
