% vim: keymap=russian-jcukenwin
%%beginhead 
 
%%file 17_04_2021.fb.fb_group.titarenko_vremja_mira.1.mosijchuk_priznanie
%%parent 17_04_2021
 
%%url https://www.facebook.com/groups/429540638315062/posts/514246153177843
 
%%author_id fb_group.titarenko_vremja_mira
%%date 
 
%%tags 2014,krym,maidan2,mosijchuk_igor,priznanie,rossia,ukraina,vojna
%%title ИГОРЬ МОСИЙЧУК ПРИЗНАЛСЯ: УКРАИНСКУЮ АРМИЮ В 2014-М ГОДУ БРОСИЛИ ПРОТИВ ПРОТИВНИКОВ МАЙДАНА, А НЕ РОССИИ
 
%%endhead 
 
\subsection{ИГОРЬ МОСИЙЧУК ПРИЗНАЛСЯ: УКРАИНСКУЮ АРМИЮ В 2014-М ГОДУ БРОСИЛИ ПРОТИВ ПРОТИВНИКОВ МАЙДАНА, А НЕ РОССИИ}
\label{sec:17_04_2021.fb.fb_group.titarenko_vremja_mira.1.mosijchuk_priznanie}
 
\Purl{https://www.facebook.com/groups/429540638315062/posts/514246153177843}
\ifcmt
 author_begin
   author_id fb_group.titarenko_vremja_mira
 author_end
\fi

ИГОРЬ МОСИЙЧУК ПРИЗНАЛСЯ: УКРАИНСКУЮ АРМИЮ В 2014-М ГОДУ БРОСИЛИ ПРОТИВ
ПРОТИВНИКОВ МАЙДАНА, А НЕ РОССИИ

В эфире главного оппозиционного проекта Украины «Голос Правды» и в youtube в
программе «Власть vs Влащенко» на вопросы ведущей ответил радикал и националист
Игорь Мосийчук.

\ifcmt
  pic https://external-mxp1-1.xx.fbcdn.net/safe_image.php?d=AQHFO2GdXybTiQUA&w=500&h=261&url=https%3A%2F%2Fgolospravdy.eu%2Fwp-content%2Fuploads%2F2021%2F04%2Fmisijchuk.jpg&cfs=1&ext=jpg&_nc_oe=6eddf&_nc_sid=06c271&ccb=3-5&_nc_hash=AQGOcDoO1jeyejbS
  @width 0.4
  @wrap \parpic[r]
\fi

\href{https://golospravdy.eu/igor-mosijchuk-priznalsya-ukrainskuyu-armiyu-v-2014-m-godu-brosili-protiv-protivnikov-majdana-a-ne-rossii/}{%
Игорь Мосийчук признался: Украинскую армию в 2014-м году бросили против противников Майдана, а не России, %
golospravdy.eu, 17.04.2021%
}

Он впервые, будто на исповеди перед смертью, признался: в 2014-м году пришедшая
на крови власть Майдана направила танки не против российской армии, а против
граждан Украины, которые отказались принять госпереворот. Приводим основные
цитаты:

После Майдана прошло семь лет. Уже можно признаться: конфликт в Крыму и на
Донбассе в 2014-м году организован при участии российской армии или всё-таки
инициаторами были граждане Украины, не согласные с «победой» Майдана?

«Изначально русских войск, за исключением Черноморского флота, в Крыму не было.
И на Донбассе не было изначально. Русские войска в Крыму появились уже позже –
дней через десять после начала всех этих мероприятий. Действовали кто?
Действовала завербованная из местного населения публика и действовали
профессиональные диверсанты – так называемые «зелёные человечки»».

Если российской армии в Крыму и на Донбассе не было, значит самопровозглашённый
«майданный лидер Украины» Александр Турчинов в апреле 2014-го года бросил армию
против украинских граждан, для которых Майдан был и остаётся госпереворотом?

«Русские войска на Донбассе появились [только] летом – в июне 2014-го года. Они
появились во время «иловайских событий»».

Это правда, что в 2014-м году среди радикалов, добровольцев и других
сторонников власти Майдана было много мародёров?

«Мародёрство было. Разное было. Потому что [в 2014-м году нам] нужны были
деньги. «Азов», например, а конкретно я с Сергеем Коротких (позывной «Боцман»)
«развели» Московский Патриархат на 50 тыс. долларов... Как это было? После
«революции» был большой Крестный Ход [канонической Украинской Православной
Церкви] и марш под Лавру. Была угроза захвата Лавры [со стороны радикалов]. Но
тогда [с представителями канонической Церкви мы] договорились: [радикалы]
постояли, покричали, но в Лавру не заходили – такое вот противостояние. Но
[представители канонической Церкви] очень перепугались. Мы им заявили, что
сейчас будет новая акция, хотя никто новую не собирался делать, просто
понаписывали [угроз] в соцсетях и перепугали их (смеётся, – прим. ред.). И так
они нам передали деньги».

А как ещё радикалы «проявляли инициативу» после Майдана?

«Мы ставили гаишников на колени».

Следует отметить, что Мосийчук до сих пор не видит параллелей между
послемайданными протестами в Крыму и на Донбассе и домайданными на Западной
Украине. Напомним, в обоих случаях протестующие врывались в административные
здания, МВД и СБУ; захватывали оружие и объявляли о неподчинении киевским
властям. Вот, только Янукович не решил применить армию против протестующих в
Киеве и на Западной Украине, а власть Майдана с лёгкостью бросила её на
Юго-Восток против несогласных с госпереворотом.

Полная версия онлайн-общения — на видео.

\ii{17_04_2021.fb.fb_group.titarenko_vremja_mira.1.mosijchuk_priznanie.cmt}
