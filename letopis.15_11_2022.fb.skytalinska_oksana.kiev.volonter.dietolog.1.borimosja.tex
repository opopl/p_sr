% vim: keymap=russian-jcukenwin
%%beginhead 
 
%%file 15_11_2022.fb.skytalinska_oksana.kiev.volonter.dietolog.1.borimosja
%%parent 15_11_2022
 
%%url https://www.facebook.com/O.Skytalinska/posts/pfbid02CdXq4dUf6RprxjQ2xH8xwLjmUz5ws77jcbNVYCRjshwAGutswjPJ4ExvhYPa5gUql
 
%%author_id skytalinska_oksana.kiev.volonter.dietolog
%%date 
 
%%tags volonter
%%title Борімося -- поборемо!
 
%%endhead 
 
\subsection{Борімося -- поборемо!}
\label{sec:15_11_2022.fb.skytalinska_oksana.kiev.volonter.dietolog.1.borimosja}
 
\Purl{https://www.facebook.com/O.Skytalinska/posts/pfbid02CdXq4dUf6RprxjQ2xH8xwLjmUz5ws77jcbNVYCRjshwAGutswjPJ4ExvhYPa5gUql}
\ifcmt
 author_begin
   author_id skytalinska_oksana.kiev.volonter.dietolog
 author_end
\fi

Борімося -- поборемо! 

UPD (оновлення): за годину після публікації цього посту поступило рівно...нуль
гривень.  ..

Після цього -- 3 333 грн. І то -- знаєте від кого? Від бійця, який зараз воює у
Бахмуті! Вдумайтесь. Поки що все.

--------------

Коли я читаю про війну та допомогу фронту, я знаю, що це означає. Це колосальна
робота, без перебільшень.

Я про волонтерів, які все роблять самі.

Уявіть собі мою звичайну зайнятість, коли є багато різнопланових проектів,
щоденні консультації, інтерв'ю ЗМІ, які треба проговорити на камеру або
наговорити у мікрофон, або написати текстово. Це все без якихось вихідних і
особливих днів на відпочинок.

До цього з лютого додалось ще волонтерство.

Так, я добровільно на це пішла, це мій свідомий вибір, мій захист батьківщини.

І нагадаю, що я -- волонтер "нижчої ланки" і до того ж -- "повного циклу". Коли
я читаю про фонди, же є великі офіси, купа працівників, аналітиків, озброєної
охорони, позашляховики, то знайте -- у мене нічого з цього немає.

Ніхто мене не охороняє, немає позашляховика і офісу в центрі міста, немає
аналітиків-військових, немає піар-служби.

Є дві людини, до яких пишуть виснажені хлопці з Бахмута, Ізюма, Миколаєва та
інших, справді бойових місць.

Вони кидають нам списки потреб, а ми, всі "в милі", шукаємо то все в інтернеті,
де є дешевше, обдзвонюємо, торгуємось, паралельно я збираю кошти, які
більш-менш активно приходять у перший день написання посту і вже на третій день
майже не поступають.

Розподіляються кошти так: хоч потрохи, але по всім критичним напрямкам. І все
-- максимально оперативно, без бюрократичної тягомотіни, лише фотозвіт. 

У мене немає "фінансового резерву", який є у великих фондах, як я сьогодні
вранці прочитала на УП, я вишкрібаю останні гроші і нерідко ще й докладаю свої
власні. І при цьому все роблю сама -- вишукую в інтернеті, купую, пишу пости в
соцмережах, веду переписку із бійцями (це дуже багато часу), читаю про
кількість 200тих та 300тих. 

Ще мені пишуть поранені бійці, я ще тримаю контакт із лікарями і підтримую
морально хлопців.

\ii{15_11_2022.fb.skytalinska_oksana.kiev.volonter.dietolog.1.borimosja.pic.1}

Інколи я така "викручена" емоційно і фізично, що бажаю лише одного -- щоб
швидше цей день закінчився. Для мене щодня -- День Бабака: одне і те ж. 

Друзі! Ті, хто допомагає і підтримує -- я знаю Вас майже поіменно, уклін Вам!

Війна -- це важко і виснажливо.

Але це наша країна і війна не триватиме вічно.

Ми згадуватимемо про ці непрості часи.

Але зараз, ми не маємо права здаватися. 

Допомагати фронту -- важка-важка робота, навіть для тих, хто допомагає
фінансово, бо треба від чогось відмовлятися.

Два дні тому до мене заїхали хлопці, які захищали Ізюм, приезли багато ворожих
трофеїв, включно з консервами. Будемо якось реалізовувати через аукціон, щоб
допомагати їм.Я віддала їм плитоноски, павербанки, підсумки, аптечки, багато
кави, шоколаду, какао, батончиків, сухого молока (дякую \href{https://www.facebook.com/profile.php?id=100000946332209}{Валентина Бенедь},
\href{https://www.facebook.com/maria.mali.7140497}{Maria Malynovskaya}, 
\href{https://www.facebook.com/oksana.tsebrivska}{Oksana Tsebrivska} )

Зараз потрібні:

Дрони два.

Рації (мінімум 4, дві вже купили).

Павербанків 20.

Комплектів для приготування їжі.

Ноутбуки та планшети -- хоча б по 2.

Не рахуючи теплого одягу (термобілизни, шапок, рукавиць).

Тактичні рукавиці.

Хоч якусь машину на ходу. 

Якщо Ви допомагаєте великим відомим фондам купувати байрактари, не забувайте
про маленьких мурахо-бджіл, які роблять щоденну важку роботу, без вихідних, без
тихих сімейних вечорів за келихом вина, без ретрітів та "перезавантажень", а
реагують миттєво на потреби і також рятуєть тисячі життів.

"Слава Україні" \dshM означає присвячувати своє життя в темні часи своїй
Батьківщині. Все інше потім надолужимо: і конференції, і келихи з вином за
перемогу.
@igg{fbicon.flag.ukraina}

PAYPAL: o.skytalinska@gmail.com

4149 5001 4943 7181 Райфайзен
5375 4114 1959 3632 Монобанк

\#україна \#українапонадусе \#дієтолог \#дієтологСкиталінська

\ii{15_11_2022.fb.skytalinska_oksana.kiev.volonter.dietolog.1.borimosja.orig}
\ii{15_11_2022.fb.skytalinska_oksana.kiev.volonter.dietolog.1.borimosja.cmtx}
