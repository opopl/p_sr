% vim: keymap=russian-jcukenwin
%%beginhead 
 
%%file topics.vojna.my.11.filmy.1.mind_games.tetjana_rudenko
%%parent topics.vojna.my.11.filmy.1.mind_games
 
%%url 
 
%%author_id 
%%date 
 
%%tags 
%%title 
 
%%endhead 

Тот вариант, когда начало было многообещающее, чем финал...

В самом начале нас знакомят с военным психологом, молодой девушкой, готовящейся
к свадьбе. И все вроде бы гладко и не предвещает беды, но... Ее похищает
неизвестный, и в круг подозреваемых, в первую очередь, попадают два близких ей
человека, жених и подруга. Но телефонный разговор этих двоих по поводу
исчезновения главной героини сужает этот круг вдвое, причем, как окажется, не в
правильном направлении. С этой точки зрения, сценарий не плох, так как фильм с
самого начала интригующий.

С момента появления в трейлере Брюса, начинается самое интересное. Они не
знакомы были до этого, кто он? Родственник одного из ее покончивших с собой
пациентов? Нет, более похож на психа. Тогда почему он выбрал ее? Зачем позволил
ей пользоваться рацией? Он же псих, но не дурак.

В том, что на "том конце провода" все было зараннее спланировано и шло по
сценарию маньяка, не было сомнения. Зачем?
