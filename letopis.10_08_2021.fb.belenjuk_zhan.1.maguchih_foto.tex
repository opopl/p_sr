% vim: keymap=russian-jcukenwin
%%beginhead 
 
%%file 10_08_2021.fb.belenjuk_zhan.1.maguchih_foto
%%parent 10_08_2021
 
%%url https://www.facebook.com/zhanbeleniuk/posts/2988562744763039
 
%%author Беленюк, Жан
%%author_id belenjuk_zhan
%%author_url 
 
%%tags __aug_2021.maguchih.foto.olimpiada.lasickene,belenjuk_jean,foto,maguchih_jaroslava,olimpiada.tokio,ukraina
%%title Ярослава дуже світла, чемна та спокійна дівчина! Вона реальний герой цієї Олімпіади!
 
%%endhead 
 
\subsection{Ярослава дуже світла, чемна та спокійна дівчина! Вона реальний герой цієї Олімпіади!}
\label{sec:10_08_2021.fb.belenjuk_zhan.1.maguchih_foto}
 
\Purl{https://www.facebook.com/zhanbeleniuk/posts/2988562744763039}
\ifcmt
 author_begin
   author_id belenjuk_zhan
 author_end
\fi

Трапилася на очі наша спільна світлина з Ярославою Магучіх, на нагородженні
кращих спортсменів місяця кілька років тому! 

Ярослава дуже світла, чемна та спокійна дівчина! Вона реальний герой цієї
Олімпіади! У важкій боротьбі виборола таку цінну нагороду для нашої збірної! І
це лише в 19 років!! До прикладу, Я особисто в свої 19 років навіть мріяти не
міг про такий успіх на Олімпіаді! Це однозначно велика звитяга патріота своєї
країни!! З чим її однозначно вітаю!

\ifcmt
  pic https://scontent-cdg2-1.xx.fbcdn.net/v/t1.6435-9/234370159_2988448531441127_7213686018454669463_n.jpg?_nc_cat=108&ccb=1-5&_nc_sid=8bfeb9&_nc_ohc=cWkdCD1UUR8AX9qJM_J&_nc_ht=scontent-cdg2-1.xx&oh=2b74159c3524f30601bbbb9eb58bf1d0&oe=613AC567
  width 0.4
\fi

Спортсмени - особливі люди! Інколи, вони пізнають реалії життя трохи пізніше!
Постійне перебування на зборах, всі проблеми і питання вирішує тренер, від тебе
чекають лише високий результат, а що там коїться за вікном… в тебе нема ні
часу, ні сил сконцентрувати увагу і розібратися! Тож, Я особисто розумію
дівчину, яка в стані шаленої ейфорії обіймається з усіма навколо! Без жодної
думки про те, як подібне буде сприйматися на Батьківщині. Це вже з досвідом,
коли і емоції краще контролюєш і навчаєшся трохи заглядати наперед і попередньо
задаєш питання собі: «а як на це подивляться наші воїни на фронті, що захищаюсь
країну від агресора?! Чи буде їм приємно, адже вони теж наші ГЕРОЇ і (знаю
особисто) дуже вболівають і слідкують за нашими олімпійцями!!»- тоді починаєш
пильнувати все, що коїться навколо тебе і все, що ти говориш, розуміючи, що в
даний момент за тобою спостерігає увесь світ!

Але цькувати молоду дівчину за це - однозначно не правильно! Можливо проблема в
тому, що їй ніхто не пояснив наслідки подібних речей.. Тоді це питання до
представників нашої делегації.. знаю, що з представниками команди ОКР вели
роз‘яснювальну роботу і заздалегідь попереджали як вести себе в тих чи інших
ситуаціях, та відповідати на деякі питання! 

Мені особисто дуже хотілося б, щоб спорт був поза політикою, але, він насправді
дуже щільно інтегрований в політику! Особливо Олімпійскі ігри - це напевне
найполітизованіші змагання сьогодення! 

Деякі країни готові вкладати шалені гроші як в підготовку своїх атлетів, та
створення усієї необхідної матеріально технічної бази, так і не гребують навіть
підкупом суддів і спортсменів, покривання вживання допінгу і т.д. І все заради
перемог і високого місця у підсумковому рейтингу! Бо це один із суттєвих
важелів пропаганди деяких режимів! До того ж, це не почалося нещодавно, ще ігри
1936го року у нацистський Німеччині використовувалися Гітлером як показник
домінування арійської раси!

Тож, впевнений, що кожний український спортсмен винесе свій урок з цієї
ситуації! І буде більш обачніший в майбутньому! Адже завжди, в момент коли ти
знаходишся в піднесеному стані і не очікуєш ніяких неприємностей, хтось
пильнує, щоб нанести неочікуваний удар!✊🏾
