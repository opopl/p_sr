% vim: keymap=russian-jcukenwin
%%beginhead 
 
%%file 30_01_2022.fb.nazarov_maksim.zhurnalist.1.o_patriotizme
%%parent 30_01_2022
 
%%url https://www.facebook.com/maks.nazaroff/posts/4877884448946141
 
%%author_id nazarov_maksim.zhurnalist
%%date 
 
%%tags patriotizm,strana,ukraina
%%title О патриотизме
 
%%endhead 
 
\subsection{О патриотизме}
\label{sec:30_01_2022.fb.nazarov_maksim.zhurnalist.1.o_patriotizme}
 
\Purl{https://www.facebook.com/maks.nazaroff/posts/4877884448946141}
\ifcmt
 author_begin
   author_id nazarov_maksim.zhurnalist
 author_end
\fi

для любителей демократии, которые затравили какую-то неосторожную, вероятно
глупенькую (не знаю лично) девочку-редактора канала Рада за нелюбовь к
украинскому языку и флагу

в США гражданин может сжечь флаг государства прямо на лужайке Белого дома (и
сжигал), после чего будет оправдан в рамках святой для американцем конституции,
которая гласит о свободе слова и мысли (и был оправдан) 

другое дело, а может ли позволить себе редактор государственного политического
канала такую демократию в Украине? очевидно, нет

и её история не о смелости  

осознав это девушка извинилась (но бесполезно, ситуация уже определила ее как
проф неликвид) 

я, например, люблю свою страну, украинский язык и все государственные символы
(от гимна каждый раз мурашки по спине, хоть первую строку надо бы переписать) 

но это не значит, что я буду лепить без каждого повода флажки на аватарки,
надевать вышиванку и горланить гимн под каждым нелепым политическим событием

почему не буду? да ведь последние восемь лет псевдопатриоты поглумились над
государственными символами куда эффектнее, чем кто-либо публично заявлявший о
нелюбви к ним, вывели патриотизм в категорию цены бутылки водки и просто всех
хорошенько збли

будьте искренними перед собой, но и башкой тоже думайте 

особенно в такое время
