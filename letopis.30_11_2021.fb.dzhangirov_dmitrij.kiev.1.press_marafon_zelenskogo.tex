% vim: keymap=russian-jcukenwin
%%beginhead 
 
%%file 30_11_2021.fb.dzhangirov_dmitrij.kiev.1.press_marafon_zelenskogo
%%parent 30_11_2021
 
%%url https://www.facebook.com/permalink.php?story_fbid=1319758658469541&id=100013062576677
 
%%author_id dzhangirov_dmitrij.kiev
%%date 
 
%%tags politika,press_konferencia,ukraina,zelenskii_vladimir
%%title Пресс-марафон Зеленского
 
%%endhead 
 
\subsection{Пресс-марафон Зеленского}
\label{sec:30_11_2021.fb.dzhangirov_dmitrij.kiev.1.press_marafon_zelenskogo}
 
\Purl{https://www.facebook.com/permalink.php?story_fbid=1319758658469541&id=100013062576677}
\ifcmt
 author_begin
   author_id dzhangirov_dmitrij.kiev
 author_end
\fi

Бывают моменты у политиков, идущих к катастрофе, когда они пытаются проявить
силу, а на самом деле это выглядит откровенной слабостью. Именно так произошло
с уже ставшим знаменитым «пресс-марафоном» Гаранта – эта акция, несмотря на
предъявление серьезных обвинений Ринату АХМЕТОВУ, пошла ЗЕЛЕНСКОМУ \& Co в
жесткий «минус». Если отбросить проплаченных ботов, то амплитуда комментариев –
от едких насмешек до жесткого негатива и откровенных оскорблений.

Кстати, в тусовке «блогеров по вызову» самоиронично шутят, что если раньше
никто из порядочных экспертов не хотел поддерживать ЗЕЕЛЕНСКОГО, то сегодня не
желают даже непорядочные. Даже за деньги. 

А ведь по замыслу Офиса Президента этот «пресс-марафон» должен был показать
решительность Гаранта и ничтожность предъявляемых ему обвинений (коррупция в
окружении и «Вагнергейт») как раз накануне обращения ЗЕЛЕНСКОГО к Верховной
Раде 1 декабря. Теперь же сразу много «черных лебедей» свидетельствуют и том,
что завтра в Раде произойдет «избиение младенцев».

При этом, ермаки\&подоляки главного «черного лебедя» выпустили своими руками,
пригласив на встречу с первым лицом государства и — согласно Конституции –
главнокомандующим журналистов с пониженной электоральной ответственностью, типа
Юрия БУТУСОВА. Которые дерзят и по-хамски пререкаются, а президент предстает в
кадре испуганным и психованным человечком, который что-то мямлит в ответ на
острые вопросы, срывается на истерику, оправдывается в стиле «сам дурак» и
просто оскорбляет журналистов.

Продемонстрированная президентом слабость создала прецедент, как легко и просто
вывести его из себя, когда он забывает роль ГОЛОБОРОДКО и предстает «просто
ЗЕЛЕНСКИМ», чьи человеческие качества находятся существенно ниже «общественно
приемлемых».   

Стало понятно, что, задав или просто выкрикнув с места во время речи президента
неудобный вопрос («день рождения ЕРМАКА», «офшоры КОЛОМОЙСКОГО», «кто в ОП слил
операцию по вагнеровцам» или без затей – о невыполненных предвыборных
обещаниях) станет для ЗЕЛЕНСКОГО триггером для механизма саморазоблачения своей
крайне непривлекательной сущности. 

И в итоге есть риск, что никто даже слушать не будет обращение президента к
Верховной Раде, которое со старта превратится в хайп и скандал. Кстати, на
самом деле, никто и не собирается слушать обращение Зеленского – все готовятся
нападать.

Причем эти вполне конкретные «все», которые готовы перейти от обороны к
наступлению, стали очередным рукотворным «черным лебедем» выпущенным с Банковой
как следствие «войны против всех» с помощью неправовых механизмов в виде СНБО.

И действительно, в той или иной степени фронды с президентской вертикалью
находятся такие разнородные силы, как мелкие предприниматели, которых давят
налогами (SaveФОП), бизнесмены средней руки, которых записали в
«контрабандисты» и «воры в законе», и крупный бизнес в лице Рината АХМЕТОВА и
других олигархов, и часть бывших «слуг», переходящих под знамена Дмитрия
РАЗУМКОВА, и влиятельные политические игроки вроде Арсена АВАКОВА, ободряемого
англо-саксами, и большинство СМИ. 

Пока что с президентом остались только «приватовцы» во главе с Игорем
КОЛОМОЙСКИМ, но история о том, как они «слили» ЯНУКОВИЧА, подсказывает, что они
пока просто не придумали, как выгодно реализовать этот «токсичный политический
актив».

А «черные лебеди» продолжают лететь даже сегодня – 30 ноября, за день до
выступления ЗЕ в Раде.

Так, 30 ноября Конституционный суд отказался приводить к присяге судей
Александру ПЕТРИШИНА и Оксану ГРИЩУК, которых 26 ноября президент Владимир
ЗЕЛЕНСКИЙ назначил судьями КСУ.

По этому поводу Конституционный суд принял специальное постановление об
отсрочке присяги этими лицами. 

КСУ исходил из того, что на момент издания ЗЕЛЕНСКИЙ указа о назначении судьями
ПЕТРИШИНОЙ и ГРИЩУК, все шесть должностей по квоте главы государства были
заняты, и вакансий, на которые Президент мог бы их назначить, не было.

По сути, это демонстрация позиции КСУ по решению Зеленского об отстранении
ТУПИЦКОГО и КАСМИНИНА (оба это решение оспаривают в Конституционном суде). То
есть, КСУ это отстранение не признаёт.

Кроме того, сегодня же КМИС выдал рейтинг партий, согласно которому «Слуги»
отстали от «Евросолидарности» Петра ПОРОШЕНКО, и лишь в пределах погрешности
опережают «Батькивщину» Юлии ТИМОШЕНКО, только что сделавшей новый боевой
тюнинг...

В общем, «стая политических товарищей», увидев, как «Акела промахнулся» на
пресс-конференции, готовится сделать Зеленскому «веселую жизнь». 

А совсем недавно был обнародован опрос, согласно которому ЗЕЛЕНСКИЙ в
гипотетическом втором туре проигрывает РАЗУМКОВУ.

Так что ночные сны на первый день зимы у преЗЕдента вряд ли напоминают добрую
сказку про маленького эксцентричного пианиста, который стал лидером
государства, которое не пожалела для него недобрая фея ...

\ii{30_11_2021.fb.dzhangirov_dmitrij.kiev.1.press_marafon_zelenskogo.cmt}
