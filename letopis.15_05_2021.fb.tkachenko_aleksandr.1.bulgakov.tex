% vim: keymap=russian-jcukenwin
%%beginhead 
 
%%file 15_05_2021.fb.tkachenko_aleksandr.1.bulgakov
%%parent 15_05_2021
 
%%url https://www.facebook.com/oleksandr.tkachenko.ua/posts/4110535479014023
 
%%author 
%%author_id 
%%author_url 
 
%%tags 
%%title 
 
%%endhead 
\subsection{Сьогодні 130 років видатному киянину – Михайлу Булгакову}
\Purl{https://www.facebook.com/oleksandr.tkachenko.ua/posts/4110535479014023}

Сьогодні 130 років видатному киянину – Михайлу Булгакову. Письменнику, який
називав Київ найкращим  Містом (обов’язково з великої літери) та для якого
Володимирська гірка була Єрусалимом.

\ifcmt
  pic https://scontent-frt3-2.xx.fbcdn.net/v/t1.6435-9/186361459_4110531069014464_338182255574298544_n.jpg?_nc_cat=101&ccb=1-3&_nc_sid=730e14&_nc_ohc=kFbzGjeAMG4AX8wMAav&_nc_ht=scontent-frt3-2.xx&oh=e33c05cb3be40f612dad8cbdca5fc06a&oe=60C5C9D8
\fi

І в Білій Гвардії, і в Майстрі та Маргариті ви знайдете всім відомі київські
вулиці та адреси. У тому числі Перша київська гімназія – нині жовтий корпус
Університету, де я теж навчався.

Можна по-різному ставитися до його сприйняття до боротьби українців за свою
незалежність, як і до його несприйняття радянської влади. 

Але геніальність його творів – перевірена часом. Саме тому в день його
народження кияни несуть в знак любові до його пам’ятнику та будинку на
Андріївському узвозі, 13 – жовті маргаритки.


