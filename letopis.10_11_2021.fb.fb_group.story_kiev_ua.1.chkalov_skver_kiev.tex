% vim: keymap=russian-jcukenwin
%%beginhead 
 
%%file 10_11_2021.fb.fb_group.story_kiev_ua.1.chkalov_skver_kiev
%%parent 10_11_2021
 
%%url https://www.facebook.com/groups/story.kiev.ua/posts/1794391384090988
 
%%author_id fb_group.story_kiev_ua,gorbov_vadim.kiev
%%date 
 
%%tags chkalov_valerij.sssr.letchik,gorod,kiev,kiev.skver.chkalova,pereimenovanija,skver
%%title Славу и известность  Валерия Чкалова в Советском Союзе  можно сравнить разве что с Юрием Гагариным
 
%%endhead 
 
\subsection{Славу и известность  Валерия Чкалова в Советском Союзе  можно сравнить разве что с Юрием Гагариным}
\label{sec:10_11_2021.fb.fb_group.story_kiev_ua.1.chkalov_skver_kiev}
 
\Purl{https://www.facebook.com/groups/story.kiev.ua/posts/1794391384090988}
\ifcmt
 author_begin
   author_id fb_group.story_kiev_ua,gorbov_vadim.kiev
 author_end
\fi

Бурная дискуссия в  нашей группе  о скверике Валерия Чкалова и его
переименовании в сквер Ступки заставила меня достать с книжной полки запыленную
книгу «Чкалов» из серии ЖЗЛ, написанную его другом и членом экипажа, знаменитом
героем-лётчиком Байдуковым. Искал я в этой книге эпизоды пребывания  Валерия
Чкалова в Киеве, где он наверняка бывал, но так и не нашёл. 

Славу и известность  Валерия Чкалова в Советском Союзе  можно сравнить разве
что с Юрием Гагариным, да и судьбы у них схожие. 

\ifcmt
  ig https://scontent-lga3-1.xx.fbcdn.net/v/t1.6435-9/254890323_4411150889013206_8395148685652864554_n.jpg?_nc_cat=109&ccb=1-5&_nc_sid=825194&_nc_ohc=cFFnHjrqb5UAX8icmwL&_nc_ht=scontent-lga3-1.xx&oh=d9839d3bbccd8a6c5e8d993bc2068d2d&oe=61B00386
  @width 0.4
  %@wrap \parpic[r]
  @wrap \InsertBoxR{0}
\fi

В честь Валерия Чкалова и его экипажа президент США тоже организовывал
торжественные приемы и банкеты после исторического перелёта самолёта АНТ-25 из
СССР в США через Северный полюс в 1937 году. 

Во время дискуссии мне встретился комментарий о том, что улица Валерия Чкалова
в Киеве носила его имя не случайно, и что Валерий Чкалов на этом месте выступал
на митинге и торжествах по случаю автопробега. Надо будет поискать информацию
в киевских газетах того периода. 

В завершении хочу напомнить, что улица Гончара, бывшая Чкалова - абсолютный
рекордсмен среди улиц нашего Города по переименованиям. В ней отражение всей
нашей история. 

Судите сами: 

\begin{itemize}
  \item - Маловладимирская с 1850 года 
  \item - Столыпинская с 1911
  \item - Григория Гершуни с 1919
  \item - Ладо Кецховели с 1937 
  \item - Чкалова с 1938 
  \item - Антоновича 1941-1943
  \item - Чкалова 
  \item - Олеся Гончара с 1996.
\end{itemize}

\begin{cmtfront}
\uzr{Ирина Козина}

во всех наших переименованиях последних лет суть не в конкретных именах, а в
делении истории на "нашу, хорошую" и "чужую, плохую и враждебную". век назад
это уже было. идем шаг в шаг за теми, кого стремимся "забыть". почему-то никто
не желает понимать, что переменить историю невозможно - она уже состоялась
	
\end{cmtfront}

\ii{10_11_2021.fb.fb_group.story_kiev_ua.1.chkalov_skver_kiev.cmt}
