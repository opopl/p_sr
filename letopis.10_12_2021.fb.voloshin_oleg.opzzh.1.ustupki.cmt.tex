% vim: keymap=russian-jcukenwin
%%beginhead 
 
%%file 10_12_2021.fb.voloshin_oleg.opzzh.1.ustupki.cmt
%%parent 10_12_2021.fb.voloshin_oleg.opzzh.1.ustupki
 
%%url 
 
%%author_id 
%%date 
 
%%tags 
%%title 
 
%%endhead 
\subsubsection{Коментарі}

\begin{itemize} % {
\iusr{Igor Lavrynenko}
В Москве ситуация аналогичная

\href{https://sevastopol.su/news/nabrosyatsya-kak-shakaly-strelkov-predrek-rossii-voynu-po-vsemu-perimetru}{%
Набросятся, как шакалы – Стрелков предрёк России войну по всему периметру, sevastopol.su, 13.02.2021%
}

\begin{itemize} % {
\iusr{Oleg Voloshin}
\textbf{Igor Lavrynenko} если б его слушали в 2014, все б уже тогда решили для себя. К счастью для нас, не слушали

\iusr{Igor Lavrynenko}
\textbf{Oleg Voloshin} православные вместе с Малофеевым его слушают

\iusr{Igor Lavrynenko}
\textbf{Oleg Voloshin} к несчастью для нас его не слушали. Если бы слушали было бы лучше - не один малайский Боинг, а дюжина...
\end{itemize} % }

\iusr{Aleksandr Nikin}
Креативный класс? Это кто?

\begin{itemize} % {
\iusr{Oleg Voloshin}
\textbf{Aleksandr Nikin} это калька из российских политических дискуссий. Не знаю, как иначе назвать эту смесь офисного планктона, блогеров, журналистов, пиарщиков, сотрудников офисов западных компаний. Они все считают себя очень креативными )

\iusr{Aleksandr Nikin}
ну точно не представителями креативніх индустрий.. Я б їх назвав диванними або кухонними патріотами
\end{itemize} % }

\iusr{Eugene Nayshtetik}
То есть, если будут более компромисными, воздух улучшится?  @igg{fbicon.smile} 

\begin{itemize} % {
\iusr{Oleg Voloshin}
\textbf{Eugene Nayshtetik} 

Нет. Но могли б свою бескомпромиссность проявлять в борьбе за улучшение среды
обитания, а не во имя абстрактный идейных установок. «Это застопорит наше
движение в евроатлантические структуры!» - ну что это за детский лепет. Куда мы
вообще движемся и, главное, зачем. Знаешь, как в песне Шнура, «а лучше б мусор
выносил». В России тоже таких крикунов хоть отбавляй. Но когда ради
качественной смены характера отношений с КНР Кремлю нужно было отдать полтора
острова на Амуре, их никто не спрашивал. А то каждый пермяк резко бы стал
нуждаться во всем Тарабаровом острове.

\iusr{Eugene Nayshtetik}
\textbf{Oleg Voloshin} 

ну, у нас такая модель общественных отношений, когда не спрашивают, сколько
островов отдавать, попросту невозможна. Какая-никакая, но демократия. Прям
история о Кемской волости )))

\iusr{Oleg Voloshin}
\textbf{Eugene Nayshtetik} 

ага. Точно. Разница только в том, что такого рода вопросы нигде на майдане не
решают. Иначе бы все со всеми бесконечно воевали. Представляю, если б испанцев
спрашивали про Гибралтар каждый раз при обсуждении любых вопросов
взаимоотношений с Британией. В центре Белграда до сих пор здание Генштаба не
восстановлено после попадания американской ракеты, но это не мешало вчера
Вучичу делать комплименты американской демократии на саммите. Нужно уметь
перевернуть страницу. Особенно в отношениях с тем, кто сильнее.

\iusr{Eugene Nayshtetik}
\textbf{Oleg Voloshin} 

ну, при чем здесь Майдан? Есть конституционная процедура о внесении изменений в
Конституцию. На сколько я помню, по этому разделу речь идет о референдуме. В
чем проблема?

\end{itemize} % }

\iusr{Sergei Cherny}
Отлично, как всегда!  @igg{fbicon.thumb.up.yellow} 

\iusr{Евгений Минченко}
Алло, шеф, эт я, Лёлик))

\iusr{Олексій Романов}
\href{https://www.youtube.com/watch?v=q-PkF8VM0NU}{%
На это я пойтить не могу..., youtube, 17.05.2016%
}

\iusr{Андрей Лукашов}
Уступки по Донбассу уже произошли, когда подписали Минские соглашения.
Как можно не пойти, если уже согласились идти?
А если отказались идти на согласованные уступки, значит отказались от Минских соглашений.
\end{itemize} % }
