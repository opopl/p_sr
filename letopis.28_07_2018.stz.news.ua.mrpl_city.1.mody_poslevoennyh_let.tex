% vim: keymap=russian-jcukenwin
%%beginhead 
 
%%file 28_07_2018.stz.news.ua.mrpl_city.1.mody_poslevoennyh_let
%%parent 28_07_2018
 
%%url https://mrpl.city/blogs/view/mody-poslevoennyh-let
 
%%author_id burov_sergij.mariupol,news.ua.mrpl_city
%%date 
 
%%tags 
%%title Моды послевоенных лет
 
%%endhead 
 
\subsection{Моды послевоенных лет}
\label{sec:28_07_2018.stz.news.ua.mrpl_city.1.mody_poslevoennyh_let}
 
\Purl{https://mrpl.city/blogs/view/mody-poslevoennyh-let}
\ifcmt
 author_begin
   author_id burov_sergij.mariupol,news.ua.mrpl_city
 author_end
\fi

Врезалась в память цифра из статистического отчета, опубликованного в годы
перестройки и гласности. В победном 1945 году легкая промышленность Советского
Союза произвела 4 (четыре) погонных сантиметра хлопчатобумажной ткани на душу
населения страны, да и то, наверное, большая часть сотканного материала пошла
на одежду для армии. Поэтому не удивительно, что полки и прилавки
немногочисленных тогда промтоварных магазинов были пусты. Если что-то и
появлялось в них, то это что-то немедленно распределялось по карточкам.
Понятно, почему так сложилось: все силы и средства государства были брошены на
восстановление разрушенного войной народного хозяйства, прежде всего
металлургии и машиностроения, а возрождение и постройка новых ткацких и швейных
фабрик были отложены до лучших времен.

Как же выглядело абсолютное большинство мариупольцев в первые послевоенные
годы? Сказать бедно – ничего не сказать. Современная бабушка, просящая подаяние
подле хлебной лавки, в своем наряде оказалась бы модницей рядом с женщинами
описываемых здесь лет. Можно было, например, встретить учительницу, спешащую в
школу в мужской спецовке, банковскую служащую в платьице, скроенном из
полинявшего баракана – ткани, употреблявшейся для обивки мебели, или работницу
какого-нибудь заводишки в блузке из грубо перекрашенной, так называемой
солдатской бязи.

\textbf{Читайте также:} 

\href{https://mrpl.city/blogs/view/mariupolskie-vypuskniki-obyazatelno-dolzhny-zapomnit-svoj-prazdnik-okonchaniya-shkoly}{Мариупольские выпускники обязательно должны запомнить свой праздник окончания школы, Нина Павлюк, mrpl.city, 27.04.2018}

Еще \enquote{живописнее} были одеты ученики младших классов: косоворотки, сметанные из
лоскутов сатина, сохранившихся каким-то чудом с довоенных времен, пальтишки,
сооруженные из Бог весть откуда добытого немецкого болотного цвета мундира,
мамины кофты с подвернутыми рукавами и полами до колен, в холодное время
перевязанные крест-накрест шерстяными платками. Это у мальчишек. Не лучше было
облачение девочек: видавшие виды платьица, кофточки из пряжи свитеров,
недоеденных молью, юбочки, длина которых увеличивалась любым подвернувшимся под
руку куском ткани, иногда совершенно несовместимым ни по расцветке, ни по
рисунку, ни по фактуре. А что было делать? Дети росли, а до появления моды на
мини-юбки было еще далеко. Иногда можно было встретить какую-нибудь девчушку в
ладном платьице в шотландскую клетку – подарок американских благотворительных
обществ. Старшеклассникам было чуть лучше, они донашивали вещи не вернувшихся с
войны отцов, отошедших в мир иной дедушек и бабушек да удовлетворялись одеждой,
купленной по случаю на толчке, - \enquote{супермаркете} той эпохи под открытым небом.
Это могла быть и матросская форменка, и офицерский китель, и старомодное пальто
с воротничком, обшитым полоской черной бархотки по моде начала ХХ века,
купленное у вдовы дореволюционного врача или инженера, и толстый свитер из
обмундирования водолаза. 

Демобилизованные воины донашивали военную форму, лишенную знаков отличия и
боевых наград, тщательно подштопанную, вычищенную нашатырным спиртом или
авиационным бензином, если это был френч и галифе, и застиранную до белизны,
если это была гимнастерка и солдатские шаровары. В большинстве заводских цехов,
а тем более на мелких предприятиях не было раздевалок и бань, привычных в наши
дни. Так что слесарям и токарям, литейщикам и жестянщикам, плотникам и
каменщикам приходилось щеголять в рабочей робе от дома на работу и с работы
домой. Вот где было бы поле деятельности для Шерлока Холмса: определять по
расположению на одежде и форме масляных пятен, по цвету следов строительного
раствора, по штанинам, прожженным искрами расплавленного металла, кто есть кто
по профессии.

А обувь? Чиненные-перечиненные туфли, ботинки и сапоги, грубые рабочие ботинки,
тапочки, верх которых выкраивался из брезента, а подошва – из обрезков
прорезиненной транспортерной ленты. Но это еще ничего. Хуже было «вербованным»
- юношам и девушкам, привезенным из сел и деревень на восстановление
предприятий. Им выдавали колодки – заменители обуви, сработанные из деревянного
подобия ступни, к которому прибивался гвоздями верх брезентового ботинка. В
летнюю пору были не такой уж редкостью и вовсе босоногие люди, а дети - обычное
дело.

\textbf{Читайте также:} 

\href{https://mrpl.city/news/view/mariupoltsam-na-zametku-vremya-razdevat-detej}{%
Мариупольцам на заметку: в жару не кутайте детей, Анастасія Селітріннікова, mrpl.city, 21.06.2018}


Зимняя обувь – особая песня. Ее главным представителем были бурки, как сказано
в определении одного кроссворда - \enquote{мягкая стеганая зимняя обувь}. Есть сильное
подозрение, что идея их конструкции пришла в наши края из Китая. Во всяком
случае, на фотографиях в старинном журнале изображены китайские повстанцы,
обутые в известную для мариупольцев старшего поколения \enquote{мягкую стеганую обувь}.
На производство бурок шли совсем уж выношенные зимние пальто, шинели любых
армий и родов войск, \enquote{москвички} - полупальто, вышедшие ныне из моды, не
гнушались портные и демисезонной одеждой, если удавалось прикупить ее по
сходной цене. Все это \enquote{добро} распарывалось, наскоро очищалось от пыли и грязи
и из драпа ли, сукна ли или другой грубой ткани выкраивались две половинки
будущей обувки, на них укладывалась вата, сверху прикрывалось подкладкой из
сатина, ситца, саржи (отнюдь не новых), все простегивалось на швейной машинке.
Затем половинки сшивались, вся эта \enquote{конструкция} выворачивалась \enquote{налицо} -
бурок был готов.

Бурки шили мужские, детские и дамские. Мужские и детские делали до колен,
дамские – сантиметров на пятнадцать выше лодыжки. Верх дамских бурок украшался
меховой опушкой. Конечно, носить бурки вне помещения просто так было нельзя.
Счастливчики одевали поверх них галоши фабричные, все остальные носили так
называемые чуни - изделия, склеенные из автомобильных покрышек. Справедливости
ради надо отметить, что столь нехитрая, на первый взгляд, обувь обладала
прекрасными гигиеническими свойствами. Те, кто ее носил, не знали, что такое
мозоли, натоптыши и другие неприятности стоп.

Зимняя обувь – особая песня. Ее главным представителем были бурки, как сказано
в определении одного кроссворда - \enquote{мягкая стеганая зимняя обувь}. Есть
сильное подозрение, что идея их конструкции пришла в наши края из Китая. Во
всяком случае, на фотографиях в старинном журнале изображены китайские
повстанцы, обутые в известную для мариупольцев старшего поколения
\enquote{мягкую стеганую обувь}.  На производство бурок шли совсем уж
выношенные зимние пальто, шинели любых армий и родов войск, \enquote{москвички}
- полупальто, вышедшие ныне из моды, не гнушались портные и демисезонной
одеждой, если удавалось прикупить ее по сходной цене. Все это \enquote{добро}
распарывалось, наскоро очищалось от пыли и грязи и из драпа ли, сукна ли или
другой грубой ткани выкраивались две половинки будущей обувки, на них
укладывалась вата, сверху прикрывалось подкладкой из сатина, ситца, саржи
(отнюдь не новых), все простегивалось на швейной машинке.  Затем половинки
сшивались, вся эта \enquote{конструкция} выворачивалась \enquote{налицо} -
бурок был готов.

Бурки шили мужские, детские и дамские. Мужские и детские делали до колен,
дамские – сантиметров на пятнадцать выше лодыжки. Верх дамских бурок украшался
меховой опушкой. Конечно, носить бурки вне помещения просто так было нельзя.
Счастливчики одевали поверх них галоши фабричные, все остальные носили так
называемые чуни - изделия, склеенные из автомобильных покрышек. Справедливости
ради надо отметить, что столь нехитрая, на первый взгляд, обувь обладала
прекрасными гигиеническими свойствами. Те, кто ее носил, не знали, что такое
мозоли, натоптыши и другие неприятности стоп.

Каковы были пристрастия в сфере головных уборов? Мечтой всех мальчишек было
добыть бескозырку военного моряка или \enquote{летчицкий} шлем. Предметом
зависти были \enquote{носители} буденовок, суконных головных уборов в виде
древнерусских шлемов с большой звездой, пилоток, а еще больше - офицерских
фуражек с лакированными козырьками. У юношей вожделенной фуражкой была
мичманка. Если эти предметы не удавалось достать, приходилось довольствоваться
кепками довоенными цивильного покроя с огромными козырьками да тюбетейками.
Именно в это время стали входить в моду кепки восьмиклинки, у которых длина
козырька не должна была превышать ширины двух плотно сложенных пальцев. А
зимой? Шапки-ушанки, изредка кубанки – круглые изделия шапочного искусства,
донышки которых украшались накрест пришитыми двойными полоскам из сутажа.

Что касается мариупольчанок, то ассортимент их головных уборов, конечно же, не
был так разнообразен, как сейчас. Главным из них, безусловно, был берет, чаще
всего белый. Отчаянные модницы где-то добывали шляпы со средней ширины полями и
высокими слегка конусообразными тульями, украшенными спереди пряжками или
подобиями кокард. В то достославное время на улицах можно было еще встретить
пожилых дам в платьях строгого покроя из льняного волокна и в белоснежных
панамах. Это были выпускницы мариупольской Мариинской женской гимназии, ставшие
учительницами после окончания дополнительного, педагогического класса. Но всего
чаще женщины первых послевоенных лет убирали свои головы летом косынками, зимой
шалями и собственноручно связанными платками.

Люди всеми силами стремились продлить жизнь одежде и обуви. Заплаты на них
вовсе не были редкостью. Находились изобретательные мастерицы, умевшие узором
из лоскутков или вышивкой закрыть прореху на платье или блузке; или того более:
из двух износившихся платьев сшить новое, так называемое комбинированное.
Существовала технологическая операция, носившая название \enquote{перелицовки}. К
примеру, потертый и залоснившийся от долгой носки пиджак распарывали, затем
сшивали вновь, но так, что изнанка, лучше сохранившаяся, становилась верхом, а
прежний верх превращался, соответственно, в изнанку обновленного швейного
изделия. Латки на локтях и штанишках большинства школяров были непременными
атрибутами их одеяний. Никого это не только не смущало, наоборот –
свидетельствовало о рачительности и аккуратности мальчишки. Обувь также
бесконечно ремонтировали. Просили сапожников поставить подметки из материала
попрочнее, скажем, из кожемита – заменителя натуральной кожи. Чтобы защитить
каблуки от износа, на них прибивали стальные подковки.

Прошли годы. Сейчас вряд ли можно уговорить даже воспитанного в послушании
первоклассника, а тем более первоклассницу надеть на себя что-нибудь даже
слегка заштопанное. Ушли в небытие такие работы сапожников, как перетяжка
обуви, в редчайших случаях портные занимаются перелицовкой. Чтобы наш молодой
современник понял, что представляли собой бурки, приходиться пускаться в
пространные объяснения. Износившуюся обувь нынче просто выбрасывают, поскольку
зачастую ее конструкция такова, что отремонтировать просто невозможно. Да и
стоимость ремонтов стала сопоставимой с ценой новой пары обуви. Зачем морочить
себе голову реставрацией старья, когда к услугам покупателей множество
магазинов одежды и обуви на любой вкус и карман: от дорогих бутиков до лавок с
товарами \enquote{секонд-хенд}.
