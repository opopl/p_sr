% vim: keymap=russian-jcukenwin
%%beginhead 
 
%%file 15_09_2021.fb.fidanjan_elena.1.vremja_kogda.cmt.lina_ivasjuk_nostalgia_sssr
%%parent 15_09_2021.fb.fidanjan_elena.1.vremja_kogda.cmt
 
%%url 
 
%%author_id 
%%date 
 
%%tags 
%%title 
 
%%endhead 
\paragraph{Ліна Івасюк - Ностальгія за СССР - риса розумовообмеженої людини}

\begin{itemize} % {
\iusr{Ліна Івасюк}
Ностальгія за СССР - риса розумовообмеженої людини. Як вона в КМДА займає таку посаду?

\iusr{Алла Шевчук}
Психіатричну експертизу при вступі на посаду проходили?

\begin{itemize} % {
\iusr{Vladimir Karasovskiy}
\textbf{Алла Шевчук} вы не пройдёте))

\iusr{Алла Шевчук}
\textbf{Vladimir Karasovskiy} а ви психіатр?

\iusr{Milan Zlatić}
\textbf{Alla Shevchuk} він довбойоб

\iusr{Vladimir Karasovskiy}
\textbf{Milan Zlatić} не злись, Сонечко))

\iusr{Vladimir Karasovskiy}
\textbf{Алла Шевчук} а вы к врачу хотите?)

\iusr{Vladimir Karasovskiy}
\textbf{Milan Zlatić} фу, какой ты злючка((

\ifcmt
  ig https://scontent-frt3-1.xx.fbcdn.net/v/t39.30808-6/242008041_914888136045048_5110080347085387807_n.jpg?_nc_cat=108&_nc_rgb565=1&ccb=1-5&_nc_sid=dbeb18&_nc_ohc=Dc8Ua_ZuLnEAX8af9_f&_nc_ht=scontent-frt3-1.xx&oh=4a146a8d96dc9d16398caae55c56d498&oe=61476957
  @width 0.3
\fi

\iusr{Milan Zlatić}
\textbf{Vladimir Karasovskiy} людською мовою розмовляй, я ледь розумію твоє тунгуське кавкання

\iusr{Vladimir Karasovskiy}
\textbf{Milan Zlatić} ну чего ты, заинька?))

\end{itemize} % }

\iusr{Инна Вакова}
 @igg{fbicon.face.womiting} вата(((

\iusr{Nan London}
І зарплатня 90, 120. І взуття по 45-100 рублів.
І черги на все і за всим. І будь таким як усі. Одяг і взуття пристойні тільки індпошив. Білизна взагалі "сюррр". @igg{fbicon.face.tears.of.joy} 

\iusr{Vladimir Karasovskiy}
\textbf{Nan London} вот что автоматы с газировкой наделали)))

\iusr{Sergiy Parhomenko}
Ага, а ще колись були часи, коли люди вірили, що Земля - плеската (

\begin{itemize} % {
\iusr{Vladimir Karasovskiy}
\textbf{Sergiy Parhomenko} сейчас некоторые верят, что Евромайдана не зря стояв)

\iusr{Andrian Kryk}
\textbf{Vladimir Karasovskiy} о, ватка набежала
Давай расскажи нам, за что стоял антимайдан.
А мы вместе поржем над тобой)

\iusr{Vladimir Karasovskiy}
\textbf{Andrian Kryk} ой не смешите)))
\end{itemize} % }

\iusr{Oleksandr Gusev}

А що там по розстрілах тисяч українців?

\iusr{Ганна Віняр}

Платня в кухаря й продавця 60 крб. Бо в нього була можливість вкрасти або дістати продукти.

\iusr{Юрко Бобренків}
Ганьба!

\iusr{Игорь Синельник}
А когда появилась туалетная бумага в ссср, микроволновки, иностранная качественная техника?)))

\iusr{Сергій Вожжов}
І якраз по темі:

\ifcmt
  ig https://scontent-frt3-1.xx.fbcdn.net/v/t1.6435-9/242211696_4626600097362981_3837956743738918836_n.jpg?_nc_cat=102&_nc_rgb565=1&ccb=1-5&_nc_sid=dbeb18&_nc_ohc=O4u6MOLExy4AX-DEu7f&_nc_ht=scontent-frt3-1.xx&oh=ed1e62592d845e62cf252804514b4b70&oe=6167C69D
  @width 0.3
\fi

\iusr{Петро Григорович Коновалов}
А за часів Сталіна її би за це розстріляли.

\iusr{Дмитро Єгоров}
\textbf{Олена Фіданян} їдь в Донецьк, там все так як написала

\iusr{Віталій Максимкін}
Звідки вас стільки совкодрочерів тупорилих? У вас не тільки логіка відсутня ,а й мізки!

\iusr{Віталій Постриган}
любительку сталіна, лєніна і концтаборів - у відставку!

\iusr{Stanislav Sarioglo}
Чергова дурня.. цікаво якого року народження автор цього посту  @igg{fbicon.thinking.face} 

\iusr{Олекса Кібкало}

Зошит за 3 коп, бо його робили з лісу, який безкоштовно валили у Сибіру
політв'язні, вчені та всілякі інші "вороги народу". Українців серед них було
вкрай багато. Так що туга за отим совковим щастям з газировкою - це туга по
злочинному диктаторському режиму. Вова Кремлівський (х\#*ло) - окупант - теж за
ним сумує


\iusr{Виталий Бреус}

О, налетіли фейсбучні експерти, провокацією запахло. Ви краще підніміть питання
яке Клічко мав право підписувати "постанову" про підвищення тарифів на
транспорт, чи спитав він думку киян? Чи Труханов з його замами в Одесі те ж
саме зробили з 15 вересня. Або чому всі обленерго належать рудому. Державної
зради в їх діях ніхто часом не бачить? Ні? А якщо придивитись? Так, у совку за
розкрадання розстрілювали, а у нас кожен другий чиновник живе на відкатах і ся
почуває дуже добре, і при Янику, і при Пороху, і зараз при Зеленському нічого
не змінилось!

Треба проводити дератизацію в Україні, і робити це так щоб народу жилось добре
тут і зараз, тоді не треба буде згадувати часи коли було краще.


\iusr{Olga Gladkevych}
Це жах...

\iusr{Віталій Чернявський}
підкажіть будь ласка, а скільки доставка піци коштувала?

\iusr{Iryna Nemyrovych}

Та за шо нам така освітня державна службовиця:( принаймні майте сміливість
вибачитися та визнати помилку. Судячи з вашого профілю, ваша родина достатньо
притомна і ви зробили помилку. Сильно хочу у це вірити


\iusr{Igor Kotsar}
Ще забули згадати десятки мільйонів розстріляних.

\iusr{Milan Zlatić}

Хотів шось написати, але, бачу, все написано до мене.
У відставку, просто нахуй на вулицю.

\iusr{Рустам Левін}
Вата?

\iusr{Danylo Golota}
Яке прикре самогубство

\iusr{Oleksandr Bohdanovych}

\ifcmt
  ig https://scontent-frt3-1.xx.fbcdn.net/v/t39.30808-6/241963508_3003441749929189_8815675340875001141_n.jpg?_nc_cat=108&ccb=1-5&_nc_sid=dbeb18&_nc_ohc=rLJ_axidzE4AX9eE1Cx&_nc_ht=scontent-frt3-1.xx&oh=3b7dc9d91a9dcb875d56ba795392b70b&oe=61478829
  @width 0.3
\fi

\iusr{Oleksandra Kudrytska}
І при цьому була епідемії туберкульозу, боткіна і відповідна тривалість життя. Попив водички , передай іншому.

\iusr{Yaroslav Vedmid}
Пздц

\iusr{Юлія Звєкова}
Це що за диковина у мене в стрічці?

\iusr{Малецька-Колісник Юлія}
\textbf{Юлія Звєкова} Я теж шокована...

\iusr{Людмила Поліщук}
Цій жінці 90 років?

\iusr{Елена Бася}
\textbf{Людмила Поліщук} А чого ви думаeте що всi дев яносторiчнi дурепи?

\iusr{Людмила Поліщук}
\textbf{Елена Бася}
Пробачте. Подумала, що старечий маразм. Трохи некоректно вийшло. Але може люди зрозуміють.

\iusr{Dmytro Motornyi}

\ifcmt
  ig https://scontent-frx5-1.xx.fbcdn.net/v/t1.6435-9/242203081_2026806157473149_8587172388803265240_n.jpg?_nc_cat=111&_nc_rgb565=1&ccb=1-5&_nc_sid=dbeb18&_nc_ohc=2yo7Mc2BG7AAX8KUORk&_nc_ht=scontent-frx5-1.xx&oh=d4eb7384ebd0b8d41de00073f9f6455c&oe=61666B34
  @width 0.3
\fi

\iusr{Artem Osypian}
О, боже.. де твій квиток у Ростов?

\iusr{Tymur Savchuk}
Боже, яке мракобісся і дичина...  @igg{fbicon.man.facepalming} 

\iusr{Плетенчук Дмитро}
Боже, ці всраті совки виховують наших дітей. Агов, жіночко, вам скіки рочків?
Бо мені 40 і я не упоротий совок. Вам 60?

\begin{itemize} % {
\iusr{Елена Бася}
\textbf{Плетенчук Дмитро} Може не треба ейджизму? Мiй дiдусь до 94 рокiв дожив i був проти совка

\iusr{Плетенчук Дмитро}
\textbf{Елена Бася} ваш дідусь виключення. Вам пощастило мати такого родича.

\iusr{Елена Бася}
\textbf{Плетенчук Дмитро} так, безперечно пощастило
\end{itemize} % }

\iusr{Катерина Кучер}

\ifcmt
  ig https://scontent-frx5-1.xx.fbcdn.net/v/t39.30808-6/242121543_10225629511151914_7861106904207661965_n.jpg?_nc_cat=105&_nc_rgb565=1&ccb=1-5&_nc_sid=dbeb18&_nc_ohc=S8OepRe0BH0AX-KUv_l&_nc_ht=scontent-frx5-1.xx&oh=450bd7bb4ba9e82594557fa5934ed42d&oe=6147C5A8
  @width 0.3
\fi

\begin{itemize} % {
\iusr{Ольга Панченко}
\textbf{Kateryna Kucher} там мав бути портрет Конфуция @igg{fbicon.face.rolling.eyes}  Або хоча б Стетхема  @igg{fbicon.face.unamused} 
\end{itemize} % }

\iusr{Максим Балабко}
Я звісно перепрошую, але на фото видно 3 апарати без стаканів і лише 1 стакан в руках школяра. Питання. Де інші 2 стакани, які не знищили вандали?

\iusr{Viktoria Franko}
... а ще катували/вбивали всіх інакомислячих
Про відсутность продовольчих товарів, черги навіть смішно згадувати на на фоні вище згаданого

\iusr{Андрій Боднарчук}
Оооооооо... "Директор у Департамент освіти і науки Києва" - совкодрочить. Це кайф. Ах, какую страну патєрялі!

\iusr{Микола Ільїнов}
Це дійсно написав директор департаменту освіти, чи то прикол такий?

\iusr{Леда Иванова}
Страшно представити про що ви мрієте , як резутльтат сучасної освіти!
ГУЛАГ, залізна завіса, партійні збори з приводу адюльтеру?)))

\iusr{CVetlana Ki}
Більченко \#2?

\iusr{Олексій Шумак}
Коли ви, совки, нарешті здохните всі?

\iusr{Kirill Dobrolyubov}

Вот же набросились на человека, как коршуны на дятла! Вы что, не знаете уровня
нашего образования и науки? Или действительно не знаете? Ну ок, теперь
знаете...

\iusr{Настя Йонка}

\ifcmt
  ig https://scontent-frt3-1.xx.fbcdn.net/v/t39.30808-6/241816449_4280690695341097_776380215398140036_n.jpg?_nc_cat=108&_nc_rgb565=1&ccb=1-5&_nc_sid=dbeb18&_nc_ohc=7Q2BrRhriXIAX8Bss2H&_nc_ht=scontent-frt3-1.xx&oh=e666d76b7ad9582e5e5e5c1b675c97cc&oe=6147937F
  @width 0.3
\fi

\iusr{Nada Slobodan}
І я була молодою, навіть незайманою.
Боже, і оцевотво нашим дітям щось може викладати та рекомендувати ....
Я, стара бабця трьох онуків, розумію що савОк то є те саме, що теперішня Північна Корея, а людина, набагато молодша за мене, дрочить на савОк.
Коли ви вже всі відзеленієте, матір Божа, святі при хаті ....

\iusr{Ірина Левунець}
Що це? Що це за треш!?

\begin{itemize} % {
\iusr{Елена Константинова}
\textbf{Ірина Левунець} Зайдіть на сторінку Андрій Верлань, схоже, там відповідь знайдете в одному з останніх відео
\end{itemize} % }

\iusr{Ігор Кондратюк}

Ну і що? Багато чого не було в часи, коли комуністи безкарно злочинно керували 1/6 сушею світу

\iusr{Дмитро Гоменюк}
Жалкуємо,що це тільки спогади!

\iusr{Alla Dmytrivna}
разом з тим напишіть: скільки пенсії отримувала ваша бабуся, де вона працювала, де жила, в яких і за чим чергах стояла. На чому готувала їсти, і де все те бралося по тих 3 - 5 коп, і чи ходила вона чистити кукурудзу на тік цілу осінь, та збирати буряки на полі, та травити мишей щоб отримати бодай той мізер доплати в колгоспі до її пенсії.

\iusr{Евгений Васильев}
Я можу купити вам квиток в "ЛДНР", там майже так, як ви пишете

\iusr{Ром Самураєнко}
Барак сраний був ваш срср!

\iusr{Taras Semenyuk}
Якщо так скучили за совком, то можливо і зп вам 60-80 грн платити треба?

\begin{itemize} % {
\iusr{Alla Dmytrivna}
\textbf{Taras Semenyuk}
Ні, краще пенсію в 24 гривень.
\end{itemize} % }

\iusr{Юрій Кравчук}
На Колимі взагалі тоді все бесплатно було в гулагах та таборах на 15 років))))

\iusr{Ирина Бородань}
В коментарях вище все й так описали.
Класно, что в нас адекватних, небайдужих і освічених людей все таки більше
Про всяк випадок тегну \textbf{Віталій Кличко}
\textbf{Департамент освіти і науки Києва}
\textbf{Валентин Мондриевский}.
Бо такі дописи - це дніще від освітянина і потрібна відповідна реакція

\iusr{Nazar Kirichenko}
Як же про "плАмбир" забули ??? Жах ...

\iusr{Ліля Секелик}
Люди, ваші коментарі - прекрасні.  @igg{fbicon.face.smiling.eyes.smiling} 

\iusr{Sergij Pletsky}

дуже гарно пам'ятаю як в рідне місто нарешті завезли в серпні ті самі хвалені
щоденики за 14 копійок... черга від дверей універмагу тягнулась по вулиці
декілька десятків метрів, ми з матір'ю простояли скількись там годин і так і не
дочекалися омріяного щоденичка - вони тупо закінчились.


\iusr{Дмитро Гоменюк}

Думаю, що авторка цього репосту не закликає до повернення у радянські часи.
Вона просто показала, що було позитивного у ті часи. Доречі авторка є патріотом
України і прогресивною освітянською людиною. Саме за часів її керівництва
освіта м. Києва набула найбільшого розвитку!

\begin{itemize} % {
\iusr{Oleksandra Oceandrop}
\textbf{Дмитро Гоменюк} на мій досвід, особливо дошкільна та інклюзивна. Epic fail

\iusr{Tamara Sorochan}
\textbf{Дмитро Гоменюк} і в чому позитив? Якби, наприклад, тема зайнятості дітей у позаурочний час, організований дитячий відпочинок або щось подібне, що дійсно було і було позитивно, то всі би зрозуміли. Таку білєбєрду про газіровку і бублики постять у одноватниках у великій кількості. Думаю, і це звідти. Дописи шановної Олена Фіданян читають вчителі, керівники освіти столиці. Це гідний рівень? Тепер я не здивована, що управлінці освіти допустили жахливу радянську форму в одному із закладів на 1 вересня. У них, мабуть, синдром ностальгії.

\iusr{Вячеслав Вегера}
\textbf{Дмитро Гоменюк} , йклмн, пошуки такого позитиву серед гівна - це також звичка радянських начальників: в кожній промові обов'язково потрібно було окремо про недоліки і позитив..)))))

\iusr{Татьяна Рябенко}
\textbf{Дмитро Гоменюк} і сонце, сонце світить в місті Києві яскравіше, "саме за часів її керівництва", ви не помічали?
\end{itemize} % }

\iusr{Александр Иванов}
Фєєрічно

\iusr{Наталья Искра}
А зарплаты были какие?

\iusr{Юлия Махмуди}
Дура блять, столько людей погибло от ковида

\iusr{Татьяна Прудникова}
Рогалік за 6 коп. Без серветок, кульків, немитими руками ) ,то було дитинство

\iusr{Світлана Петрович}

Той досягає розквіту, хто не ганьбить минуле, а вміє аналізувати і не знищувати
все вщент досягнуте. Адже сьогодні екологія в жахливому стані, здоров'я
молодого покоління ми втратили, не маємо впевненості в безпеці... Ось що
важливо взяти до уваги і виправляти. А не все валити в одну купу


\iusr{Pavlo Shatohin}
\obeycr
Е-е. Ну, нехай буде зошит. Середня ЗП в концтаборі 169 руб (1980), середня ЗП в Україні 14345 грн (2021). Отже:
Концтабір: 0.03/169 = 0.0177\%
Україна: 1.00/14344 = 0.00697\%
В Україні в два з половиною рази дешевше. Питання: я правильно пишу слово "невігластво"?
\restorecr

\iusr{Виталий Шумиленко}
Якщо це не стьоб, тоді питаннячко:
- Чому у "великій безвандальній" країні "щасливих" людей на 4 автомати газводи лише 1 склянка?

\iusr{Дмитро Андрощук}
Ах да... колбаса по 2. 20 и сысыру боялся весь мир!)))Ещё железный занавес и мы пабидили всех!

\iusr{Тетяна Вовк}
це про стакани на цепку, щоб не вкрали

\ifcmt
  ig https://scontent-frx5-1.xx.fbcdn.net/v/t1.6435-9/241633058_10209841178983042_1701458232330756712_n.jpg?_nc_cat=105&ccb=1-5&_nc_sid=dbeb18&_nc_ohc=L1LR6fSpnGQAX_ZailZ&_nc_ht=scontent-frx5-1.xx&oh=b674dd0ce923f0aa124552c5379d311b&oe=6168B8B8
  @width 0.3
\fi

\iusr{Світлана Бордун}
\url{https://www.facebook.com/groups/sovokseries}

\iusr{Олександр Кутовий}
А щоб той бублик був п'ять копійок, максимальна ЗП ланкової в колгоспі була 70 руб., а пенсія 20 руб.
А щоб колгоспники не втікали, паспорти їм почали видавати з 1975 року.
Коли вже ці совкодрочери нажеруться тієї ковбаси по 2.20 і нап'ються газ води з сіропом по 3 коп.?

\iusr{Малецька-Колісник Юлія}
Фу....

\iusr{Юлія Підмогильна}
Ага, з одного стакана.
І через одного виразка шлунку.
Бо ложки-виделки-посуд у совку фігово мили в столовках.
Ех, завод-колхоз, моя любов...

\iusr{Yarema Bachynsky}
Навіщо цю маячню писати?

\iusr{Елена Константинова}
\textbf{Yarema Bachynsky} зайдіть на сторінку Андрій Верлань, одне з останніх відео роз'яснить

\iusr{Євген Сагуйченко}
Йобнута…

Мабуть затужила за часами, коли жопу витирали газетами, бо не було
елементарного туалетного папіру.

За часами, коли ті щоденник по 14 коп, що ця непритомна совкова тужителька
пише, завозилися виключно по святам, і за ними стояла черга.

А воду пили з крана, бо просто не мали можливості отримати питну якісну воду, і
травилися хлором та іншими небезпечними сполуками.

Ну а, що до факту написання цього пости, не звичайною людиною, а цілим
начальником по освіті такого рангу, то я вважаю що вона має бути звільнена з
займаної посади, за пропаганду комуністичного режиму і за тупість.

\iusr{Микола Ставицький}

\ifcmt
  ig https://scontent-frx5-1.xx.fbcdn.net/v/t1.6435-9/242113541_986790482192553_7463551831459193658_n.jpg?_nc_cat=111&_nc_rgb565=1&ccb=1-5&_nc_sid=dbeb18&_nc_ohc=cmRedwMvY7EAX8WHegH&_nc_oc=AQk457ZJFiQEdIAERmy5ofjYn4YhvTc2v1HMtanfsC1G89An-z45zwZTuezmd9Apx5A&_nc_ht=scontent-frx5-1.xx&oh=7a3d245607272b0867d9f84dc3e2cbcb&oe=6169C38E
  @width 0.3
\fi

\begin{itemize} % {
\iusr{Тетяна Вовк}
\textbf{Микола Ставицький} , в 80-х на ВДНХ в Москве "выбрасывали" импорные
товары дефицитные (сама привезла из командировки 2 финских зимних комбинизона
племяннице, в очереди не стряла , был пропуск служебный, до открытия выставки
была у дверей магазина), но очереди стояли в те магазины громадные пока
какой-то начальник в 1986 году не сказал: "Выставка достижений нашего народного
хозяйства, а очереди за импортным. Убрать импорт с выставки". И убрали ((
\end{itemize} % }

\iusr{Oleksandr Denysenko}
Кожен побачив в пості щось своє... І відреагував на своє розуміння посту...

\iusr{Оксана Джус}
 @igg{fbicon.man.facepalming}  без слів... головне ж ковбаса

\iusr{Oksana Chernova}
Це ж треба так всратися!

\iusr{Oleksandr Gusev}

Роботи було багато.

\ifcmt
  ig https://scontent-frx5-2.xx.fbcdn.net/v/t39.30808-6/242091297_10220679282068372_2086549825075048398_n.jpg?_nc_cat=109&_nc_rgb565=1&ccb=1-5&_nc_sid=dbeb18&_nc_ohc=JomSli1u41kAX_vRU16&_nc_ht=scontent-frx5-2.xx&oh=d0f2dc0fd9b911100fa889b37d6ecc0b&oe=6147C149
  @width 0.3
\fi

\iusr{Анатолий Анатолиев}
Хвора людина

\iusr{Сашко Клочан}

\obeycr
Пані, у вас на аватарці є "рабів до раю не пускають". Сподіваюсь ви в курсі, що совок , це країна рабів? Вас не пустили б.)
Ще на одній аві ви дякуєте захисникам... Ну ви ж в курсі, що саме ці люди кістьми лягли, щоб такі як ви не повернули совок?
Що за каша у вас в голові?)
За шестидесятників, розстріляне відродження, геноциди щось казати вам? Мабуть зайве.... Таким як ви сироп по три копійки вимив пам'ять....
Як мені хочеться жити з такими як ви в різних країнах, як би ви тільки знали.
\restorecr

\iusr{Oleg Slavynskyy}
вам на пенсію вже пора

\iusr{Olga Saulenko}
Не можу зрозуміти, як ОЦЕ тут з'явилося! Боюся, що зламали акаунт

\iusr{Vlodko Gray}
Який совковий бред

\iusr{Borys Lebeda}
А про те що за торгівлю джинсами або валютою можна було сісти у в'язницю не хочете згадати?

\iusr{Анатолий Анатолиев}
І ці хворі на голову займаються освітою наших дітей, а по факту скільки років цій настальгуючій за совком особі

\begin{itemize} % {
\iusr{Микола Швирид}
\textbf{Українець Українець} вона не просто займається, вона КЕРУЄ столичною освітою.

\iusr{Анатолий Анатолиев}
\textbf{Микола Швирид} отож
\end{itemize} % }

\iusr{Sergii Ruzhytskyi}

В нас і досі купа совка, який Ви оплакуєте. Самі ж, мабуть, і досі копирсаєтесь
в непотрібній документації, репортажі про шкільні туалети без дверей ледь не
щодня... В кримінальних зведеннях - схеми на закупівлях ноутбуків для шкіл...

\iusr{Андрей Зиновьев}
Когда запрещали украинский язык, когда депортировали, сажали в тюрьмы, когда морили голодом

\iusr{Yurchenko Sergiy}
Цей "постік" та вибрані гостро-влучні коментарі потрапили в нашу авторську рубрику \textbf{\#АНТИсССр}  @igg{fbicon.no.mobile.phones} 

\iusr{Sergii Ruzhytskyi}
При нагоді, повпливайте, щоб з продажу вилучили щоденники із зображенням сталіна

\iusr{Lena Ferbyak}

\href{http://argumentua.com/stati/epidemii-v-sssr-chuma-kholera-meningit-angina-gripp}{%
Эпидемии в СССР: чума, холера, менингит, ангина, грипп ..., argumentua.com, 19.03.2020%
}

просто залишу це тут. Все інше вже згадали в коментарях. Мені лише цікаво, яким
чином дітей навчатимуть вмінню шукати та аналізувати інформацію, якщо з
критичним мисленням з самого верху трабли

\iusr{Alexandr Podlasov}

\ifcmt
  ig https://scontent-frt3-1.xx.fbcdn.net/v/t1.6435-9/242041629_4277065205681167_4008174806702889564_n.jpg?_nc_cat=106&_nc_rgb565=1&ccb=1-5&_nc_sid=dbeb18&_nc_ohc=2FTjeWyjY50AX_heGpw&_nc_ht=scontent-frt3-1.xx&oh=89d4c2913f36de256eabb99a2c8c1b61&oe=6167E107
  @width 0.3
\fi

\iusr{Kosteckiy Artem}
От совкодро...

\iusr{Михайло Думяк}
Яка дрімуча та безпросвітна у нас освіта!
Ви, нащадок не розстріляного виродження!

\iusr{Вячеслав Чеш}
Ви там як, норм почуваєтеся? Ніде не тисне, совєтські стаканчікі самиє совєтськіє, оце все норм?

\iusr{Світлана Дубина}
вам зламали акаунт? (((

\iusr{Oksana Danylova}
\textbf{Світлана Дубина} от у мене теж таке враження, дякую за здоровий глузд

\iusr{Anatoly Traxel}

не було ковіду - був туберкульоз. забула тубдеспансери в кожному селі? якраз
через антисанітарію і що всі лизали з одного стакану. Хочеш в ссср - дуй в
придністров'я або "днр-лнр" там вже повернулись до тих часів такі як ти

\iusr{Анна Верещага}
Пані, складайте повноваження, на вас чекає Північна Корея.
Задовбали дрочити на труп, некрофіли махані.

\iusr{Максим Повненький}
пропоную обмін квартири в Макіївці (ОРДЛО), на Вашу.

\iusr{Olga Gorodenko}
Не знаю, чи було вже

\ifcmt
  ig https://scontent-frt3-1.xx.fbcdn.net/v/t1.6435-9/242121255_1958449140983860_9195268521068974393_n.jpg?_nc_cat=104&_nc_rgb565=1&ccb=1-5&_nc_sid=dbeb18&_nc_ohc=qUS-n_MlC5wAX9W0FsS&_nc_ht=scontent-frt3-1.xx&oh=7c28589a87f3bd1d7eba7c3af16d40c8&oe=6167C55C
  @width 0.3
\fi

\begin{itemize} % {
\iusr{Yurchenko Sergiy}
\textbf{Olga Gorodenko} що саме?

\iusr{Olga Gorodenko}
\textbf{Yurchenko Sergiy} "правильный текст"))

\iusr{Olha Tsvyntarna}
\textbf{Olga Gorodenko} у пані Олени роздвоєння особистості?))

\iusr{Yurchenko Sergiy}
\textbf{Olga Gorodenko} вже побачив "фотошоп чи дописку" чиюсь, бо не розгледів одразу  @igg{fbicon.no.mobile.phones} 
\end{itemize} % }

% -------------------------------------
\ii{fbauth.shvyryd_nikolaj.kiev.ukraina.partia.nova_ekonomika}
% -------------------------------------
 
Гнати з посади цього табачника в спідниці.

\iusr{Olha Tsvyntarna}
Ага, набив шлунок ковбасою по 2.20, заїв пломбіром, став у 20-літню чергу на квартиру, розказав дітям про подвіґ піанєра-ґєроя валоді дубініна (всі ж цей треш пам'ятають?) і про те, як погано всі живуть в капіталістічєскіх странах, дізнався в тьоті Люс... Ещё

\iusr{Matviy Getman}
І в ті часи ще розстрілювали людей в концтаборах..... Зато дешева газіровка.... І газ! В камерах!

\iusr{Юрий Евенко}
Я прошу прощения: а вы (с маленькой) в каком веке живете не пробовали для интереса уточнить?
И да: есть "прекрасные" места, где так же восхищаются "совком". Вас там примут с обьятьями.
Или наручниками.

\iusr{Sergii Ruzhytskyi}
Я не вірю, що притомна людина, та ще й така що опікується освітою - могла таке написати. Тому й не здивуюсь, якщо скоро повідомлять про те що сторінку хакнули))

\iusr{Olha Tsvyntarna}
\textbf{Sergii Ruzhytskyi} на жаль

\iusr{Oleksandr Solomakha}
ага, одна склянка на чотири автомати з водою))) Певно піонер зпіонерив)))

\iusr{Наталия Дегтяренко}
Пані пришелепкувата дурепа, бажаю Вам щоб Ваші діти і онуки жили в СРСР!

\begin{itemize} % {
\iusr{Світлана Скрипник-Дмитричук}
\textbf{Natalia Degtyarenko}
Люто плюсую!
І бажано в часи Андропова!

\iusr{Ганка Клив}
\textbf{Natalia Degtyarenko} але бажано не на території України. (Вибачте за доповнення)

\iusr{Наталия Дегтяренко}
\textbf{Ganka Kliv} то навіть і не обговорюється! Пані має зібрати речі - гайда вперде! До світлого майбутнього, наприклад у Північну Корею!
\end{itemize} % }

\iusr{Olha Tsvyntarna}
Я пішла в освіту працювати, щоб була хоч якась перевага совкам, які й досі, на 31-му році незалежності, продовжують промивати мізки молодому поколінню.

\iusr{Halyna Rykova}
Не боялися пити воду з вуличних автоматів, бо ідіоти, а ще замовчувалась статистика по туберкульозу. Він, звичайно, приємніший за ковід

\iusr{Anatoly Traxel}

я пам'ятаю стояв в черзі за хлібом в магазин аж на вулицю отаким хлопчаком біля
отаких автоматів...але ніхто ту воду не пив бо стояли за хлібом. Після школи по
пів години мав відстояти, це була моя робота щоб на вечерю був хліб в хаті.
Їсти не було що особо, крім бараболі і хліба і паганих консервів. В автоматах
одна склянка на три автомати і та вічно надщерблена. два автомати неробочі, їх
ремонтували раз в рік на "пєрвомай" а ця худоба тупо бреше. Якщо з носка дати
по пездаку - вона добре все згадає. І як босі ходили, заношували старе взуття
до дір і як голодні були і як вчителі били дітей та знущалися як хотіли і
особливо про рівноправ'я партійної верхушки і пролєтаріїв. вже не говорячи про
гулаги

\iusr{Mykola Voskalo}
І ШО, НАПИЛАСЬ ГАЗІРОВКИ?

\begin{itemize} % {
\iusr{Віталій Кодря}
\textbf{Микола Воськало} дотепер пукає
\end{itemize} % }

\iusr{Vadym Yermakov}
Невдалий жарт!

\iusr{Nikolay Markin}
Єб@нутим немає спокою!
Я правильно написав? Без помилок?
Якщо ви навчалися у той "професорки", тоді зрозумілі і ваши шмарклі по
сересеру... Якщо ви вже жили при ньому...

% -------------------------------------
\ii{fbauth.radionov_vladimir.kiev.ukraina.harkov.trener.sport.kozackij_dvobij}
% -------------------------------------
 
Молодец!!! Поддерживаю!!!

Украина была космической державой, с населением более 50 млн. С пятой
экономикой в мире. А про социальную защиту, систему образования, уровень
бесплатной медицина на уровне того времени я вообще молчу. Об этом нужно
говорить нашим детям, чтоб не забывали о достижениях нашей Родины в прошлом и,
чтоб хотя бы немножко пытались приблизиться к этому уровню. Елена, огромное Вам
спасибо.

\begin{itemize} % {
\iusr{Sergiy Minyuk}
\textbf{Vladimir Radionov} Ага, помню этот уровень, стоматологию без анестезии, железные зубы вместо нормальных коронок, очереди за всем, в том числе и за туалетной бумагой, беспаспортных колхозников, пашущих как рабы за трудодни, без пенсии вообще, а когда им таки дали паспорта и пенсию в 5-7 рублей, за которую можно было купить полтора кило вареной колбасы. Об этом нужно говорить нашим детям, чтоб не забывали о лжи и бесчеловечности совка и совкодрочеров.

\iusr{Владимир Радионов}
\textbf{Sergiy Minyuk} У меня свои, совсем другие и полностью положительные
воспоминания. В момент развала СССР мне было 19 лет. И я прекрасно помню и
осознаю те времена. Добра и позитива тогда было намного больше. И не надо меня
лечить, я лишь высказываю своё мнение, на которое имею право. Вы оставайтесь
при своём. Всего Вам наилучшего.

\iusr{Irena Boychuk}
\textbf{Vladimir Radionov} Проїзд 3 коп, зарплата 100 крб. Масло 3.40/кг
(періодично дефіцит, переплата 100\%). Одяг сіро-буро-малиновий мішкуватий, або
шити самому. Рахуйте, думайте.

\iusr{Irena Boychuk}
\textbf{Vladimir Radionov} Ви, пардон, якого року народження? Ви не жили і не виживали в той час. Батьки працювали і вирішували всі проблеми. А ви "пороху не нюхали", якщо вам було 19 у 1992 році.

\iusr{Владимир Радионов}
\textbf{Irena Boychuk} рахуйте

\iusr{Владимир Радионов}
\textbf{Irena Boychuk} Кто рассказал Вам такой бред???

\iusr{Олександр Парфірійович}
\textbf{Владимир Радионов} в Днр совкодрочеров поважають. Може пора вже туди

\iusr{Владимир Радионов}
\textbf{Irena Boychuk} Мои родители работали простыми инженерами на заводе. Зарплата их была приблизительно 350 руб. на одного. Когда им было по 25 они получили от завода 3-х комнатную квартиру. Подчёркиваю, получили, а не купили. Ежегодный отдых на море по профсоюзным путевкам. И т.д. Не пудрите мозги людям.

\iusr{Владимир Радионов}
\textbf{Олександр Парфірійович} Едь, тебя никто не держит.

\iusr{Oleg Sydor}
Хер стояв то й ностальгія )

\iusr{Владимир Радионов}
\textbf{Oleg Sydor} Завидно @igg{fbicon.laugh.rolling.floor}{repeat=3} 
 · 3 ч.
\iusr{Владимир Радионов}
\textbf{Oleg Sydor} Сейчас гейпарадов много, если у тебя не стоит, можешь и по другому удовольствие получить @igg{fbicon.laugh.rolling.floor}{repeat=4} 
Здобулы!!!
 · 3 ч.
\iusr{Anatoly Traxel}
\textbf{Vladimir Radionov} Ви прекрасна ілюстрація рівня розвитку інжєнєров в ссср і їх творєній.

\iusr{Владимир Радионов}
\textbf{Anatoly Traxel} Спасибо.
 · 3 ч.
\iusr{Oleg Sydor}
Їбанутий
 · 3 ч.
\iusr{Владимир Радионов}
\textbf{Oleg Sydor} Я вижу.

\iusr{Александр Мочалов}
\textbf{Владимир Радионов} ???? про добро и позитив 1991 году))))
 · 2 ч.
\iusr{Олександр Кравченко}
Тоді просто в пана радіонава хуй стояв, не те що сійчас )))

\iusr{Irena Boychuk}
\textbf{Vladimir Radionov} Я тоді жила, вчилася і працювала. А що саме ви назвали словом "бред"?

\iusr{Irena Boychuk}
\textbf{Vladimir Radionov} 350 це дуууже хороша зарплата.Були в партії? Я теж працювала інженером. Зарплата була 115 крб в 1982 і 150-160 крб у 1992, коли пішла в декрет. Не могла навіть стати на чергу на житло, бо 1-кімнатна хрущівка 28 кв м на нас з мамою вважалася мало не розкішшю. Норма, щоб стати на чергу - менше 5.5 м на людину.

\iusr{Halyna Rykova}
Мои родители работали простьіми инженерами и получали по 120 рублей на каждого. Вьі просто врете. Только чиновники имели подобньіе зарплатьі и партийньіе функционерьі

\iusr{Halyna Rykova}
Очереди на жилье бьіли непоборньіми и коррумпированньіми. Даже не представляю, кто вообще получал квартирьі, если не бьіл в партии или в руководящем составе предприятий

\iusr{Владимир Радионов}
\textbf{Irena Boychuk} Я тоже в 1990 учился и работал... И получал около 1000 руб. Но это продлилось не долго, до 1992, потом дефолт, инфляция и 1000 превратилась в копейки. Вы и сами всё прекрасно помните.

\iusr{Владимир Радионов}

\textbf{Irena Boychuk} Да, естественно мои родители, как руководители были в
партии, Вы знаете, что по другому тогда не было, такие были времена, и мой Отец
до сих пор работает на том же предприятии начальником БТК, как и раньше. Видно
разные предприятия, разные города, разные должности. Что я могу сказать? У меня
только позитивные воспоминания о тех временах. Свою историю нужно знать и
уважать. Это наша история.


\iusr{Владимир Радионов}

"Итак, история принадлежит тому, кто охраняет и почитает прошлое, кто с
верностью и любовью обращает свой взор туда, откуда он появился, где он стал
тем, что он есть; этим благоговейным отношением он как бы погашает долг
благодарности за самый факт своего существования." Ф. Ницше

\index{Ницше, Фридрих}

\iusr{Irena Boychuk}
\textbf{Vladimir Radionov} Саме тому я не забуваю про минуле, і не замінюю його на пропагандистські міфи і стереотипи. До речі, є й плюси з мого минулого - агітпроп і фальш бачу за кілометр, та вмію читати між рядками. Це у нас з радянського досвіду. Адже не всі тоді були зазомбовані.

\iusr{Irena Boychuk}
\textbf{Vladimir Radionov} 

ІТР тоді в партію не сильно приймали. Треба було дуже вислужитися. Багато для
кого це було як мрія))) Приймали без питань "рабочий класс". Ну і квартир не
роздавали всім. У нас люди стояли по 40 років в черзі. Передавали її дітям, а
ті онукам.

Я не стояла на черзі, бо в нас з мамою була аж одна кімната на двох в хрущівці.
Більше "нє палагалось". Але за України я заробила і купила житло. За "білу"
зарплату.

\iusr{Маріанна Малина}
\textbf{Vladimir Radionov}, який безсоромний брехун! Це в яких таких інженерів була по 350 рублів зарплата? 120 була середня зарплатня в простого інженера! І 25 років квартири ніхто не отримував! Добре, якщо сімейний гуртожиток в такому віці давали! Брешете і не червонієте!!! Сором!

\iusr{Halyna Rykova}
\textbf{Владимир Родионов}, так вьі определитесь, ваши родители бьіли "простьіми инженерами" или "руководители в партии"

\end{itemize} % }

\iusr{Олесь Клименко}
Тобто совок запам‘ятався отакою відкритою хуйньою? А де ж рекордні виплавки чавуну і прочія? Це мене тішить)))
 · 4 ч.
\iusr{Vit Onopriyenko}

ЧТО БЫЛО ПРИ СТАЛИНЕ. ЧТО МЫ ПОТЕРЯЛИ.

\obeycr
1. ПЕНОПЛАСТ ДЕЛАЛИ ИЗ МОЛОЧНОЙ ПЕНЫ. МОЖНО БЫЛО ДЕТЕЙ КОРМИТЬ.
2. СИЛА ГРАВИТАЦИИ БЫЛА СЛАБЕЕ ПРОЦЕНТОВ НА 80. ЛЮДИ НА ДОМ ЗАПРЫГИВАЛИ С РАЗБЕГУ.
3. ЧЕЛОВЕК ЖИЛ В СРЕДНЕМ 150-190 ЛЕТ. БОЛЕЗНЕЙ НЕ СУЩЕСТВОВАЛО, КРОМЕ ТРУДОВЫХ МОЗОЛЕЙ.
4. ЕСЛИ НА УЛИЦЕ СПОТКНЕШЬСЯ И УПАДЕШЬ — ЛЮДИ ПОДБЕГАЛИ, ДЕНЬГИ В КАРМАН ЗАСОВЫВАЛИ, В ГУБЫ ЦЕЛОВАЛИ, ПРЕДЛАГАЛИ ВЫПИТЬ, ПОРОДНИТЬСЯ.
5. ЗАЙЦЫ И КУРОПАТКИ СРАЗУ НА СКОВОРОДУ ЗАЛЕТАЛИ.
ХЛЕБ ПОКУПАЕШЬ — ТЕБЕ ЕЩЕ ДОПЛАЧИВАЮТ.
К РЕКЕ СТРАШНО ПОДОЙТИ БЫЛО: НАЛИМЫ В КОТЕЛОК ПРЫГАЛИ.
6. ДЕД РАССКАЗЫВАЛ: ЛЮДИ НОЧЬЮ ПРОСЫПАЛИСЬ ОТ СЧАСТЛИВОГО ДОБРОГО СМЕХА. УТРОМ ВСЕ ОБЛИВАЛИСЬ ЛЕДЯНОЙ ВОДОЙ ИЗ ВЕДРА.
7. СРОК БЕРЕМЕННОСТИ СОСТАВЛЯЛ 4,5 МЕСЯЦА. ДЕТИ РОЖДАЛИСЬ ПО 12-15 КИЛОГРАММ С БЕЛОКУРЫМИ ВОЛОСАМИ, ЯСНЫМИ ГОЛУБЫМИ ГЛАЗАМИ И ВОЛЕВЫМИ УМНЫМИ ЛИЦАМИ — СРАЗУ НА ПРОИЗВОДСТВО ПРОСИЛИСЬ.
8. ВОДА В ВОЛГЕ БЫЛА СЛАДКАЯ, КАК ПАТОКА. А ЕНИСЕЙ СОСТОЯЛ ИЗ ТЕМНОГО ПИВА.
9. ЗИМОЙ БЫЛО МИНУС ТРИСТА, ВСЕ РУМЯНЫЕ ХОДИЛИ.
10. ЯГОДЫ РОСЛИ НА ОПУШКЕ С КОТА РАЗМЕРОМ. КОТЫ БЫЛИ С СОБАКУ, СОБАКА С КОРОВУ, А КОРОВА КАК ЦЕХ, А В ЦЕХУ МУЖИКИ В ШАХМАТЫ ИГРАЛИ ПО МЕТОДИКЕ БОТВИННИКА — КОНЕМ МАТ СТАВИЛИ С ПЕРВОГО ХОДА!
\restorecr

\iusr{Anatoly Traxel}
\textbf{Vit Onopriyenko} як ми це витримали...

\iusr{Толик Гавриш}
і так, звиняйте, вробилась головний освітянин столиці...

\iusr{Андрій Чумаченко}

Пані Олено, давайте зробимо "роботу над помилками" (була так форма виправлення
помилок в школі). По-перше: цей пост не екологічний, а політичний. Поясню чому.
Ви - публічна особа, державний службовець, керівник достатньо високого рангу.
Отже - зобов'язані чітко формулювати преамбулу такого посту. По-друге: Ви не
журналіст. Тому Вас, на жаль, не супроводжує прессекретар (тільки не
Нікіфоров)))Якщо б він був неподалік, то 2-3 реченнями у Вашому пості виправив
би помилки - убрав би "скрєпи", які стали (на жаль) головною руйнівною силою
російської пропаганди. Нібито "ми, за порєбріком, збираємо до купи новий Союз,
а ви біжете на Захід". Чи знаєте, що цей "ковбаснийзадвап'ятьдесят" "скрєп" вже
дончанам стоїть поперек горлянки?! Упевнений - знаєте. Навіть розумію, що
ностальжи за молодістю теж зіграло з Вами свій "жарт"...Але ж слід просто
відверто зізнатися собі: щось не так я наваяла...Знаєте, закордонні колеги
Вашого рівня добре знаються на політичній толерантності. Не тому, що вони такі
продвинуті, а тому, що постійно навчаються. Саме так. У них формула "вік живи -
вік навчайся" має обов'язкове наповнення: курси підвищення кваліфікації.)))
Чому б і Вам не долучитися до такого, (до речі теж старого й доброго)
формату?)))


\iusr{Mykhaylo Kukharchuk}
Ностальгія за совком головного мозку.

\iusr{Аня Світла}
От ти кончена?

\iusr{Anna Kompanets}
Так, а що Ви хотіли цим постом сказати? Все було, тільки сексу не було?

\iusr{Кира Иванова}
Той випадок, коли потрібна реакція \textbf{\#рукаобличчя}

\iusr{Evgen Vorobyov}

Не знаю чи цей навіжений потік маячні авторка написала сама чи тупо
перекопіпастила у когось, але це не применшує рівня його неадеквату, особливо
враховуючи, що вона займає посаду в органі місцевого самоврядування.

Так от, якби авторка вміла трошки аналізувати дані, а не лише продукувати і
поширювати дезінформацію у вільний від роботи (чи замість неї) час, вона могла
б з"ясувати, що показник розвитку людського капіталу (Human Development Index)
в Україні станом на "совєтський" 1990 рік був 0,7, а сьогодні становить 0,8
(тобто суттєво поліпшився).

Ну і нарешті про "дрібниці", пов"язані з охороною здоров"я: авторка чомусь не
згадує, наприклад, рівень комфорту і якості стоматологічних процедур у совку,
або ж продукти жіночої гігієни, які були доступні тоді жінкам. Ну так, зате
були грьобані автомати зі склянками, через які можна було прекрасно поширювати
інфекції.

\iusr{Dmytro Zgonnik}

Фиданян Елена Григорьевна — руководит столичной сферой образования с апреля
2014 года. Часто попадала в скандалы, связанные с коррупцией,
злоупотреблениями, махинациями с тендерами, стимулировании школьных поборов.
СМИ сообщали, что в 2015 году Фиданян выписала доверенность на представление
интересов Департамента образования некому гражданину Жукову. Журналисты
обнаружили, что данная персона не имела к образовательной сфере никакого
отношения, зато была замечена в темных делах. За несколько лет до этого Жуков
проходил в судебном деле, как рейдер. Любопытно, что он получил документ для
предоставления интересов Департамента на том же участке, на котором ранее
пытался «отжать» недвижимость. В прессе гадали, чиновница либо допустила
служебную халатность, либо сознательно дала шанс рейдеру захватить участок
земли. Также журналисты обратили внимание на то, что Фиданян без стеснения
меняет политические предпочтения. На парламентской кампании 2012 года Фиданян
была в команде регионала Виталия Журавского, близкого соратника экс-мэра
Леонида Черновецкого. Елена ездила по округу, по которому он баллотировался и
активно агитировала за него. При этом, Фиданян оставалась госслужащей, а им
принимать участие в предвыборной гонке запрещено законом. Через несколько
месяцев после выборов Елена резко сменила политический вектор, став помощницей
соратницы Юлии Тимошенко и народной депутатки Лилии Гриневич. Деятельность
Фиданян на посту директора Департамента образования и науки вызывала вопросы у
простых жителей столицы и представителей власти. СМИ сообщали, что лишь за один
2014 год ведомство, руководимое Фиданян провалило значительную часть
региональных программ. В том же году она отметилась решением отправить киевских
школьников на каникулы во время холодов. Якобы это должно было сэкономить
бюджетные средства. В конце концов, эта идея закончилась провалом, а краснеть и
оправдываться пришлось киевскому мэру.

\url{https://my.ua/persons/olena-fidanian}

\iusr{Dmytro Zgonnik}

В начале 2016 года в Киевсовете была зарегистрирована петиция с требованием
уволить Фиданян из Департамента. Создатели петиции отмечали, что чиновница не
соответствует занимаемой должности, не имеет профессиональных навыков, а ее
работа негативно влияет на репутацию госслужащего. Фиданян обвиняли в том, что
она содействует школьным поборам, осуществляет госзакупки по завышенным ценам,
ведёт рейдерскую деятельность в отношении киевских учебных заведений. С
требованием отставки чиновницы к стенам здания правительства даже выходили
студенты. Петиция набрала более десяти тысяч подписей, но большая часть из них
просто пропала. Позднее выяснилось, что они были заблокированы потому что имели
признаки сфальсифицированных. Поэтому до рассмотрения ее столичными депутатами
дело так и не дошло. В том же 2016 году столичную общественность возмутило
решение Департамента образования ликвидировать единственную в городе вечернюю
школу для незрячих и слабовидящих людей. Аргументация о том, что только здесь
можно было получить среднее образование людям разных возрастов с проблемами
зрения чиновников не интересовало. Для них необходимо было оптимизировать
бюджет. К киевскому градоначальнику неоднократно обращались нардепы с просьбами
отстранить Фиданян от должности.

\url{https://my.ua/persons/olena-fidanian}

\iusr{Dmytro Zgonnik}

По информации парламентариев, руководительница Департамента проводила
договорные тендеры, вместо электронных учебников закупала устаревшие нетбуки.
Елену обвинили даже в том, что она саботировала вопросы
национально-патриотического воспитания юных киевлян. Злоупотребления при
проведении тендерных закупок заинтересовали и правоохранителей, которые
возбудили несколько уголовных производств. Сотрудники прокуратуры обнаружили,
что в период с 2016 по 2019 год в тендерах на общую сумму в 20,3 млн гривен
побеждали фирмы с признаками фиктивности. Журналисты киевских интернет-изданий
сообщали, что сфера образования за два года (2018-2019) по каким-то
невыясненным причинам не освоила полтора миллиарда гривен. В тоже время
родители школьников жаловались на постоянные поборы в учебных заведениях.
Столичные депутаты заметили любопытную опечатку в сметах Департамента. Так, в
списке из 71 учреждений образования, где запланированы капремонты, 16 объектов
дублируются. Департамент образования стал одним из «героев» журналистского
расследования программы «Гроші» о коррупции в киевских школах и о других
нарушениях закона. Среди них было взяточничество при подборе персонала.
Представители прессы отметили, что директора многих школ не справляются со
своими обязанностями. Но это не мешает им годами сохранять за собой руководящие
посты. По информации СМИ, все попытки снять Фиданян с поста главы столичного
образования заканчивались ничем. Все они разбились о мощную политическую
«крышу» чиновницы. В совместной собственности чиновнице принадлежит квартира
(86 кв.м.) в Киеве. Муж Фиданян владеет земельным участком (1250 кв.м.) в селе
Процив Киевской области. Своей машины у Елены нет. Ее супруг ездит на
Volkswagen Pointer 2006 года выпуска. За 2020 год зарплата директора
Департамента образования составила 610 тысяч гривен. Олег Фиданян пополнил
семейный бюджет на 500,7 тысячи гривен. Елена Фиданян, согласно декларации, не
имеет ни банковских счетов, ни наличных средств. У ее мужа на счету в банке
хранится 336 тысяч гривен.

\url{https://my.ua/persons/olena-fidanian}

% -------------------------------------
\ii{fbauth.shevchuk_vladislav.kiev.ukraina.dobrobat.oun.aktivist}
% -------------------------------------
 
\textbf{Олена Фіданян}, 

а ще в ці «чудові часи» вбивали лише за те, що українець.

Дуже огидно, що жертви репресивної радянської машини не зрівняються для
директорки департаменту освіти і науки Києва з склянкою чаю за 2 копійки, і
періодом «коли спина не боліла».

Коли ж ви вже наїстеся  @igg{fbicon.man.facepalming} 

\iusr{Irena Boychuk}
Пані, а ви особисто в "ті часи" жили? Щось я сумніваюся. Дивлячись на ваше любування
 · 13 ч.
\iusr{Veta Black}

\ifcmt
  ig https://scontent-frx5-2.xx.fbcdn.net/v/t39.1997-6/s168x128/851587_233289413496156_795063318_n.png?_nc_cat=1&ccb=1-5&_nc_sid=ac3552&_nc_ohc=-U285EvZod0AX_TCP1h&_nc_ht=scontent-frx5-2.xx&oh=8a5d196e18c9ec2047b2c23011e4e298&oe=61484696
  @width 0.2
\fi

\iusr{Natalia Panasyuk}
Жесть

\iusr{Олександр Парфірійович}
фіданян - совкодрочерка, за рахунок киян. НауКівіця

\iusr{Kyrylo Zaplotynskyi}

А ще були часи, коли цей тоталітарний режим знищив більшу частину еліти і
інтелігенції, влаштував Голодомор, розстріляв купу народу. І навіть коли
розстріли перестали практикувати - інакомислячих кидали за грати, країна була
окупована і ізольована. Але подібні вам совкодрочери згадують чомусь тільки про
пломбір.


\iusr{Sambora Svetlana}

Я щаслива , що я і мої діти живуть саме в цей час.

Я пам'ятаю роки радянської влади, коли були голодні на ковбасу, а банани то
була розкіш. Коли я ходила в туфлях в 6 класі, моя мама змогла мені їх купити
на виріс , коли ще я була в 1. Ці коричневі жахливі туфлі я пам'ятаю по цей
день. І що там ще було прекрасного???? No comments, вибачте, не втрималась

\iusr{Олександр Єфімов}
Коли ви , нарешті , закінчитесь ? Просто цікаво , де та матка , що плодить таких ?!


\iusr{Марина Довбня}
\textbf{Олександр Єфімов} в мавзолеї лежить...

\iusr{Алексей Шелиховский}
пост специфический, но комментарии говорят о многом @igg{fbicon.face.smiling.black} 

\iusr{Екатерина Вигилёва}
От цікаво, якщо все було так добре, так дешево і так дружньо, а ще й освіта
найкраща, то чому ж він розпався?!! Можливо тому, що все вищесказане - брехня і
партійна пропаганда?  @igg{fbicon.face.smiling} 
@igg{fbicon.face.tears.of.joy}{repeat=3} 

\iusr{Eva Teo}
Час, коли перший вчитель учив, що москва - столиця України

\iusr{Екатерина Вигилёва}
Пані \textbf{Інна Совсун} , будь ласка, зверніть увагу, хто працює в освіті. Та ще й не на найнижчих посадах.
Заздалегідь дякую.

\begin{itemize} % {
\iusr{Nick Lyakhovic}
\textbf{Екатерина Вигилёва} о, лингвонаци включились, травою «врага народа» затеяли. Правнучки комиссаров не переведись ещё.
\end{itemize} % }

\iusr{Евгений Пустоход}

То це мабуть через таких чиновників щороку люди мають через Громадський бюджет
вибивати туалети, фарбування стін, ремонт столової, заміну вікон і т.д. для
шкіл, садочків?

Те, що Зобов'язане виділятися з регулярного бюджету - то "3 копійки", на думку
чиновників, які мають вирішувати ці питання.

Ви застрягли в совку глибоко, і тягнете туди ж місто.

\iusr{Андрей Воробьев}
Ого яка маячня.

\iusr{Ольга Мовчан}
Ага, мешканцям ГУЛАГУ й інших таборів про це розкажіть.

\begin{itemize} % {
\iusr{Nick Lyakhovic}
\textbf{Ольга Мовчан} 

Вы ещё польские издевательства над украинцами вспомните.
Выросли Вы и Ваши родители в принципиально другом государстве, получив
образование и соц Блага, которых не было у 80\% планеты. Благодарность предкам

\iusr{Ольга Мовчан}
\textbf{Nick Lyakhovic} розкажіть про це Степанові Хмарі, який зустрів незалежність на Лук'янівці, розкажіть пасторам протестанських церков і греко-католицьким священикам, які на свободі жили менше, ніж за гратами, розкажіть усім, кого закатували в тюрмах. Це продовжувалося до кінця 80х. Для мене саме в цьому і є прінципіально друге государство.
 · 12 ч.
\iusr{Олександр Молодий}
\textbf{Nick Lyakhovic} , ще одне чмо совкодрочерне. так вали нахер в московію. Там якраз будують те, за чим отакі недоумки ностальгують..
\end{itemize} % }

\iusr{Владимир Гелич}
Щиро бажаю вам повернутися у ті часи.

\begin{itemize} % {
\iusr{Леся Гафич}
\textbf{Володимир Геліч} приєднуюсь до побажання
\end{itemize} % }

% -------------------------------------
\ii{fbauth.manjko_vitalij.kiev.ukraina.kpi.founder.nova_kraina}
% -------------------------------------

От були часи! Моя вчителька (206 школа, Київ) вимагала, аби ми їй приносили
суху ковбасу (1984 рік). Бо такий продукт не продавався у вільному доступі. І
коли отримала відмову, то мої оцінки різко знизилися. А ще наші родичі з Нової
Одеси приїжджали регулярно в Київ, аби купувати вершкове масло, вироблене на
новоодеському сирзаводі. У них ніколи не було в продажі масла. Пам'ятаємо
епідемії холери, гепатиту тощо. Страшні лікарні пам'ятають всі, хто не був
елітою. Ми не знали слова "булінг", але ми знали шкільні бойкоти та звіряче
насилля. До речі, про низькі ціни. Родина моєї дружини – вчителі. Все життя у
жебрацтві. Курочка лише на великі свята. Новий рік та 8 березня. Решта днів на
кашах. І таки так, вартість низьких – закатовані незгодні з режимом, насилля,
поламані долі, суцільний концтабір.

\iusr{Ірина Фют}
Не все так добре було! Не сумуйте!

\iusr{Ірина Фют}
А ще сексу не було, а діти десь брались

\iusr{Владимир Радионов}

"Итак, история принадлежит тому, кто охраняет и почитает прошлое, кто с
верностью и любовью обращает свой взор туда, откуда он появился, где он стал
тем, что он есть; этим благоговейным отношением он как бы погашает долг
благодарности за самый факт своего существования."

Ф. Ницше

\iusr{Nick Lyakhovic}

Поддерживаю Вашу точку зрения. Сегодняшние нацики- большевики точно также
разрушают цивилизацию, запрещают языки, уничтожают культуру. 100 лет прошло, за
это время СССР успел измениться из урода в более цивилизованное и социальное
гсюосударство с кучей изъянов, но и с массой достижений и стать, в итоге,
лучше. А Украина упала в смутные времена лингвонацистов и сексотов.
Дискриминационным русскоязычных достигла апогея вместе с воровитостью властей.
Фашистских Уолла грантов превращают в героев. Страна больна, увы

\iusr{Юрій Михалевич}
Не зрозумів? \textbf{Віталій Кличко} \textbf{Vitali Klichko} - у Вас в департаменті освіти хомо совьєтікуси працюють? Ностальжі за країною, яка знищила десятки мільйонів українців? Женіть поганою мітлою!

\iusr{Pavel Karakay}
А ще були Гулаги , десіденти , вороги революції , з/п в 120 руб, колгоспники з трудоднями та без паспортів , ще могли розстріляти за продаж ждинців , тотальна цензура ... та ще самий брехливий Гімн держави , від першого до останнього слова (((((
А з іншої сторони майже все до 2,2 , ну щоб щастя було ((((

\iusr{Mishka Oss}
Совком запахло та нафталіном. Може пора їти з посади та звикати до землі?

\iusr{Таня Кучеряшка}
Хочется написать - и...?

\iusr{Ірина Дідковська}
Часи, коли мій дідусь сам написав докторську, а захистити не дали, бо не був комунякою.
Совко...ри, йдіть лісом!

\iusr{Гордон Шамвей}
Ну і кончені ж у нас чиновники від освіти.... З смердючим совком в голові.
\textbf{Olena Fidanian} ви єбанута ?

\iusr{Майя Руденко}

\obeycr
Час, коли моїй бабусі заборонили ходити до школи, адже вона була з родини розкуркуленого пана-шляхтича...
Час, коли до Сибіру тебе відправляли лише за те, що ти українець.
Час, коли виморили голодом цілу націю.
Час, коли влаштували Букримський плацдарм, а головнокомандувач радянського війська казав: "Бабы ёще нарожают".
Час, коли релігійність була вироком.
Час, коли національне визнавалося злочином.
Гарні часи пропагуєте.
P.S. В 1985-му в м. Ніжин санстанція заборонила використовувати такі автомати - рівно, як і стакани для розливу соків у магазинах: в місті розпочалася епідемія сіфілісу.
\restorecr

\iusr{Лариса Вакульницька}

ви не боялись пити з вуличних автоматів? а що, в СРСР не діяли закони природи,
не розмножувались бактерії і віруси?

Вас треба звільнити не лише тому що ви ностальгуєте за тоталітарним устроєм, а
й тому що ви просто неймовірно неосвічена. Не може керувати департаментом
освіти людина настільки тупа.

\iusr{Леся Гафич}
Чи ви з глузду з'їхали, чи що це за маячня?

\iusr{Артем Донець}
Киев, у вас образование осоветилось. Меняйте срочно.

\iusr{Andrii Podanenko}
І сиділа б на зоні за цей текст українською

\iusr{Valerii Liubarskyi}
Нафталін

\iusr{Andrii Podanenko}
\textbf{Святослав Літинський} тут файний спільний друг у нас є. Прямо зомбі

\iusr{Галина Недашківська}
Що за чортівня у Департаменті освіти?

\ifcmt
  ig https://scontent-frt3-1.xx.fbcdn.net/v/t1.6435-9/s851x315/241972318_3026644610941512_695168478384166823_n.jpg?_nc_cat=104&_nc_rgb565=1&ccb=1-5&_nc_sid=dbeb18&_nc_ohc=NP3YADEtCDcAX8cp6Lx&_nc_oc=AQn94iUDBLu6O3Hw5TjPloTMLilpCvsQRfbeV9dYAnLFAgdR6CAz12mv7h-EIe_7CcI&_nc_ht=scontent-frt3-1.xx&oh=87bfe3fa70df0406a2ef8b6ed2491777&oe=61685935
  @width 0.3
\fi

\iusr{Аліна Федченко Сабат}
За совком скучили, товаріщ дірекор дєпартамента??

\ifcmt
  ig https://scontent-frx5-2.xx.fbcdn.net/v/t39.30808-6/s851x315/242032577_4609535562392718_4893014423715975780_n.jpg?_nc_cat=109&_nc_rgb565=1&ccb=1-5&_nc_sid=dbeb18&_nc_ohc=RW8MRj2LgCgAX_1M3Eb&_nc_ht=scontent-frx5-2.xx&oh=6dfc6f95a480de0e91627678bf36dc0a&oe=6146CCAB
  @width 0.2
\fi

\iusr{Эдуард Бражнік}
Дефицит тотальный был в конце 80-× и начале 90-×. Всё по талонах, полки пустые.
Но зато газировка с сиропом по 3 коп.  @igg{fbicon.face.sad.but.relieved} 

\iusr{Анна Котик}
Не застала цього всього

\iusr{Богдан Жишко}

Про ГУЛАГ, про 12 міністерств ГУЛАГУ ще розкажіть, щоб люди знали, як працювала
радянська економіка й чому все так дешево!

Сподіваюся, що скоро ви знайдете нову роботу.


\iusr{Галина Недашківська}

Мого прадіда розстріляв Ваш совок у Биківні і не за 2.20, а так, даром. З
бабусиної родини, розкулаченої, не всі з Київщини до Сибіру доїхали. Ця
"путівка" була також на по 2.20, ба більше, безкоштовна. Пропонуєте
поностальгувати за совком???  @igg{fbicon.anger}{repeat=3} 

\iusr{Sosna Sosna}
1 копейка при зарплате в 100 рублей, то же самое что 1 грн при зарплате в 10000
грн. Можно ли сейчас купить тетрадь за 3 грн? Да! Пирожок или булку в
супермаркете за 5 грн купить вполне реально, а вот почему его нельзя купить за
эту цену в школьной столовой - вопрос к главе департамента образования КМДА!

\iusr{Alex Kozlovsky}
Коли ви вже нажретесь цих бубликів з маком за 5 копійок ? Радянські освітні реконструктори...

% -------------------------------------
\ii{fbauth.kravchenko_aleksandr.kiev.ukraina}
% -------------------------------------
 
Are you ohueli v KMDA?

\iusr{Анна Агейченко}
А сколько этой курице лет? Кто знает?
И эти люди отвечают за образование детей((

\iusr{Анна Агейченко}
Оу... Может это она так с фейерверками уволиться решила?))

% -------------------------------------
\ii{fbauth.molodyj_aleksandr.lvov.ukraina.direktor.park.kultury}
% -------------------------------------
 
Тупа безмозгла ідіотка...

\iusr{Володимир Гарматюк}

Совкодрочери, напевно, не знають про такі речі? Хоча, дикректор департаменту
освіти і науки мала би знати! 

\href{https://www.radiosvoboda.org/a/28137445.html}{%
Радянські концтабори для дітей під час Голодомору 1932-1933 років, Есміра Інзик, radiosvoboda.org, 24.11.2016%
}

\iusr{Наташа Комар}
 @igg{fbicon.man.facepalming} 
 · 13 ч.
\iusr{Anatoliy Lustyk}
Тільки от в автоматах на фото склянок нема

\iusr{Oleksii Fesenko}

Бублик 5 коп. У магазині зараз 5 грн. На середню заробітну плату можна придбати
2400 бубликов. В СРСР 2400 бубликов коштували 120р, що було трохи вище за
середню  @igg{fbicon.smile} 

Щоденник 14 коп. Такий самий виробництва Харків зараз коштує 14 грн. Але такий
щоденник ваша дитина тихесенько загубить, або тактично відмовиться

\iusr{Евгений Жилкин}

Такие времена, что за пост в западных соцсетях уже б сидели перед кгбшником в
обморочном состоянии, после чего либо лагеря либо работать уборщицей в 100км от
Киева за 90 рублей (40 кг вареной колбасы, немножко бубличков, зато березовым
соком хоть залейся)

\iusr{Andrii Podanenko}
\textbf{Iryna Solovey} \textbf{Тетяна Терен} - тут спільний друг \textbf{Aksinya Kurina} любов'ю до цього посту реагує...

\iusr{Роман Сініцин}

Боже, хто керує цим містом((

\iusr{Victoria Zhulkovska}
І шкрябали б ви цю маячню на шпалерах вдома, бо телефонів також не було

\iusr{Andriy Pilschak}
 @igg{fbicon.man.facepalming}{repeat=3} 
\textbf{Телебачення Торонто}
\textbf{Майкл Щур}


\end{itemize} % }
