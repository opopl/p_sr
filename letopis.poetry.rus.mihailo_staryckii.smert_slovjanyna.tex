% vim: keymap=russian-jcukenwin
%%beginhead 
 
%%file poetry.rus.mihailo_staryckii.smert_slovjanyna
%%parent poetry.rus.mihailo_staryckii
 
%%url https://www.myslenedrevo.com.ua/uk/Lit/S/Starycky/Poetry/SmertSlovjanyna.html
%%author 
%%tags 
%%title 
 
%%endhead 

\subsubsection{Смерть слов’янина}
\Purl{https://www.myslenedrevo.com.ua/uk/Lit/S/Starycky/Poetry/SmertSlovjanyna.html}

Лежав він у крові, пронизаний ножем…
В останній боротьбі його стискались руки,
Хапаючись з безпам’ятної муки
За груди, спечені вогнем.
В очах іще немов життя було,
Але вже смерть вкривала його чоло…
А тут вогонь аж гоготів навколо
Та злизував пожежею село.
Крізь чад і дим, немов у пеклі тім,
Мигтіли де-не-де озвірені катюги;
Мішався крик і регіт од наруги,
Що коїлась невольникам святим…
А він лежав, безсилий помогти
Своїм братам, тут гинувшим за волю,
І в голові його, змордованій од болю,
Мутнилися думок, гадок крихти;
Похмурі дні мигтіли в очіх сном, –
Гірке життя під канчуком ворожим,
Заплямлене покором тим негожим,
Зневажене невольницьким ярмом…
Єдина мить просвітку уночі, –
Кохання запальне, садок, кущі крушини,
І очі лагідні голубоньки-дівчини,
І хвилі кіс на пишному плечі…
І скрута знов – невільна праця, нуд,
Наруга над святим того невіри-пана,
Ображена звірюкою кохана,
І знявшийся на бузувіра люд…
Аж ось учувсь йому знайомий глас –
Він долинув крізь полум’я бурхливе…
Востаннє зір борця зайнявся з гніву
І з стогоном в несилі тяжкій згас.
1877 
