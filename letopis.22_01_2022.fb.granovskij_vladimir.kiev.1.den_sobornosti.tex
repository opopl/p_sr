% vim: keymap=russian-jcukenwin
%%beginhead 
 
%%file 22_01_2022.fb.granovskij_vladimir.kiev.1.den_sobornosti
%%parent 22_01_2022
 
%%url https://www.facebook.com/vladimir.granovski/posts/4872671989459905
 
%%author_id granovskij_vladimir.kiev
%%date 
 
%%tags den.ukr.sobornosti,prazdnik,sobornist,ukraina
%%title День Соборности, чей это праздник?
 
%%endhead 
 
\subsection{День Соборности, чей это праздник?}
\label{sec:22_01_2022.fb.granovskij_vladimir.kiev.1.den_sobornosti}
 
\Purl{https://www.facebook.com/vladimir.granovski/posts/4872671989459905}
\ifcmt
 author_begin
   author_id granovskij_vladimir.kiev
 author_end
\fi

День Соборности, чей это праздник? 

Как хорошая идея превращается в свою противоположность и становится
профанацией. 

\ii{22_01_2022.fb.granovskij_vladimir.kiev.1.den_sobornosti.pic.1}

До тех пор, пока четверть населения называют «ватниками»; пока три четверти
населения родились в стране, название которой запрещено произносить; пока под
видом патриотов по стране шествуют факельные колоны с нацистскими лозунгами и
символикой; пока День Победы называют днём «победобесия» (тех кто нас защитил
ценой собственной жизни от рабства называют не героями, а \enquote{бесами}); пока мы
собственных граждан, родных для многих людей на Донбасе называем террористами;
пока мы обезвоживаем «свой» Крым; пока есть запрет в СМИ на использование
русского языка и и.д. и т.п. праздновать день Соборности преждевременно. Но
день этот нужен, чтобы о Соборности подумать всерьёз, а не устраивать
показанные мероприятия  с пафосом и словоблудием. Мало радости пока для
праздника, но много причин для скорби.

P.S. В Стоунхендж синонимом Соборности считается слово Мир. И начинается любая
Соборность с мира в собственной Душе.
