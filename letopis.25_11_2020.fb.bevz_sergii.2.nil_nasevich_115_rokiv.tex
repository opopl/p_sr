% vim: keymap=russian-jcukenwin
%%beginhead 
 
%%file 25_11_2020.fb.bevz_sergii.2.nil_nasevich_115_rokiv
%%parent 25_11_2020
 
%%url https://www.facebook.com/spollloh/posts/212513390325346
 
%%author Бевз, Сергій
%%author_id bevz_sergii
%%author_url 
 
%%tags 
%%title Ніл Хасевич, 115 років від дня народження
 
%%endhead 
 
\subsection{Ніл Хасевич, 115 років від дня народження}
\label{sec:25_11_2020.fb.bevz_sergii.2.nil_nasevich_115_rokiv}
\Purl{https://www.facebook.com/spollloh/posts/212513390325346}
\ifcmt
	begin_author
   author_id bevz_sergii
	end_author
\fi
\index[names.rus]{Хасевич, Ніл!Український художник, борець за волю України, головний графік УПА}

Сьогодні минає 115 років від дня народження видатного українського художника,
борця за волю України, головного графіка УПА Ніла Хасевича.

Майбутній автор знаменитих повстанських дереворитів прийшов у світ 25 листопада
1905 в селі Дюксин на Рівненщині в родині диякона. 

У 13-річному віці юнак потрапив під потяг. Після трагічного випадку він
самотужки виготовив дерев’яний протез для скаліченої ноги і твердо вирішив
продовжувати навчання в художній майстерні. 

\ifcmt
pic https://scontent.fiev6-1.fna.fbcdn.net/v/t1.0-9/127457327_212513333658685_3974472370144729915_o.jpg?_nc_cat=106&ccb=2&_nc_sid=730e14&_nc_ohc=jHu2kySTWtsAX87pDnS&_nc_ht=scontent.fiev6-1.fna&oh=f0f0eb6fc14616d5f912ff74072fd603&oe=5FE3E03B
caption Ніл Хасевич (1905-1952)
\fi

1926 Ніл Хасевич вступає до Варшавської академії прикладних мистецтв. Молодий
художник швидко здобуває славу віртуозного митця. Уже в юному віці він отримує
численні міжнародні нагороди, зокрема, премію «Ватикан» (1931), диплом
Варшавської академії (1932).

Після завершення навчання Ніл Хасевич вступає до лав ОУН і особисто знайомиться
зі Степаном Бандерою та іншими лідерами визвольного руху. Талановитий художник
принципово відмовився від громадянства держави-окупанта. Це закривало йому шлях
до успішної кар’єри в міжвоєнній Польщі. Але принципи для нього стояли вище за
матеріальні блага.

У вирі Другої світової війни Ніл Хасевич долучається до лав УПА. Його основною
зброєю в боротьбі за державність стали віртуозні малюнки. Він плідно працював у
крайовій референтурі пропаганди ОУН на Волині, формуючи візуальний образ
визвольного руху. Блискучі роботи графіка використовувались у численних
підпільних виданнях. Саме Ніл Хасевич розробив дизайн повстанських грошей
(бофонів) та повстанських нагород.

Ілюстрації Ніла Хасевича надихали українців на боротьбу, спонукаючи чинити опір
окупантам. Він відмовився виїхати на Захід, до кінця продовжуючи боротьбу. 4
березня 1952 р. Ніл Хасевич застрелився в криївці під час нерівного бою з
окупантами. Він відійшов у вічність нескореним воїном.

Пам’ятаємо, пишаємось!
