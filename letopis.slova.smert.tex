% vim: keymap=russian-jcukenwin
%%beginhead 
 
%%file slova.smert
%%parent slova
 
%%url 
 
%%author 
%%author_id 
%%author_url 
 
%%tags 
%%title 
 
%%endhead 
\chapter{Смерть}
\label{sec:slova.smert}

%%%cit
%%%cit_head
%%%cit_pic
%%%cit_text
\enquote{В субботу, 19 июня, около 4 часов утра в Теребовлянский отдел полиции
поступило сообщение от местной жительницы о том, что в доме своего брата она
обнаружила \emph{мертвой} его 28-летнюю жену. На место происшествия выезжала
следственно-оперативная группа, руководство Главного управления Национальной
полиции в области}, - говорится в сообщении.  На теле женщины имелись признаки
насильственной \emph{смерти}. Обследовав территорию хозяйства, полицейские обнаружили
на приусадебном участке \emph{мертвым} 39-летнего мужа \emph{погибшей}. По предварительным
данным, он совершил самоубийство, взорвав себя взрывным устройством.
%%%cit_comment
%%%cit_title
  \citTitle{В тернопольской области мужчина убил жену утюгом и взорвал себя бомбой}, Полина Пронина, strana.ua, 22.06.2021
%%%endcit

%%%cit
%%%cit_head
%%%cit_pic
%%%cit_text
Еще раз. Вся наша жизнь состоит из череды бессмысленных и роковых \emph{смертей}.
Вчера я узнал, что мой знакомый с гайсинского рынка \emph{умер} от короны. Молодой
пацан. И на этой же неделе дальний родственник в Закарпатье \emph{умер} от рака. А
пару лет назад мамина подруга \emph{утонула} на море. Так бывает. Не все доживают до
пенсии.  И военные тоже в особой группе риска. Они \emph{гибнут} на учениях, \emph{гибнут} по
синьке, \emph{гибнут} при неосторожном обращении с оружием. И, естественно, \emph{гибнут} на
войне
%%%cit_comment
%%%cit_title
\citTitle{Наши ребята гибнут там, где у власти нет ни стратегической, ни тактической цели}, 
Игорь Лесев, strana.ua, 08.07.2021
%%%endcit

%%%cit
%%%cit_head
%%%cit_pic
%%%cit_text
Муки, страждання, сльози, \emph{смерть}, стихії, лиха, голод, багато дітей, багато
роботи, мало землі, ще менше щастя і ніяких радощів, а я живу, радію, я
щасливий, як ніхто на світі.  Але ж брат приїхав недаремно. Не для щастя він
приїхав. Сувора й холодна сила за ним, та він не бачить її і не знає. Тільки я
повинен відкрити йому очі, хоч для цього й треба мені прокричатися крізь тисячу
років.  Як це неймовірно тяжко!
%%%cit_comment
%%%cit_title
\citTitle{Тисячолітній Миколай}, Павло Загребельний 
%%%endcit
