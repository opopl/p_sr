% vim: keymap=russian-jcukenwin
%%beginhead 
 
%%file 2016.uinp.22_sichnja_den_sobornosti_ukrainy.6.posylannja
%%parent 2016.uinp.22_sichnja_den_sobornosti_ukrainy
 
%%url 
 
%%author_id 
%%date 
 
%%tags 
%%title 
 
%%endhead 

\subsubsection{Корисні Інтернет-посилання}

1. Гай-Нижник П. ІV Універсал Української Центральної Ради // Особистий сайт:
\url{http://www.hai-nyzhnyk.in.ua/doc/video_iv_universal..php}

2. Жежера В. Як приймався ІV Універсал, який проголосив незалежність України у
1918 році // Україна. Історія великого народу: 

\url{http://www.litopys.com.ua/encyclopedia/ukra-na-p-d-chas-revolyuts-1905-1907-rr/yak-pryymavsya-iv-universal-yakyy-progolosyv-nezalezhnist%60-ukrayiny-u-1918-rotsi/}

3. Зінченко О. Незалежність №1: Коли Грушевський насправді її оголосив, чому
Винниченко сумнівався, а Єфремов був проти // Історична правда:
\url{http://www.istpravda.com.ua/articles/2015/01/26/146960/}

4. Ісаюк О.  Забута незалежність. За один рік до Соборності // Історична
правда: \url{http://www.istpravda.com.ua/columns/2012/01/22/69636/}

5. Чоп Г. П'ять спроб України оголосити незалежність: від Центральної Ради до
ОУН // IPress.ua: \url{http://ipress.ua/articles/pyat_sprob_ukrainy_progolosyty_nezalezhnist_vid_tsentralnoi_rady_do_oun_26149.html}

6. Верстюк В. День Соборності України: історія виникнення традиції й свята //
Європейська Україна: \url{http://eukraina.com/publ/human_development/den_sobornosti_ukrajini_istorija_viniknennja_tradiciji_j_svjata/6-1-0-185}

7. Гай-Нижник П. Акт злуки УНР та ЗУНР: втілення і крах ідеалу Соборної України
// Особистий сайт: \url{http://www.hai-nyzhnyk.in.ua/doc/172doc.php}

8. Галущак М. \enquote{Потрібний живий ланцюг Львів-Луганськ}: інтерв'ю із істориком
Олегом Павлишиним // Історична правда: \url{http://www.istpravda.com.ua/articles/2013/01/22/109449/}

9. \enquote{Ланцюг Єднання} у січні 1990 року. Родинні фото // Історична правда:
\url{http://www.istpravda.com.ua/artefacts/2013/01/23/109614/#12}

10. Сеньків М. Акт злуки УНР та ЗУНР – знакова подія української історії // СНУ
імені Л.Українки. Історичні студії. – 2013:
\url{http://esnuir.eenu.edu.ua/bitstream/123456789/6620/1/Senkiv.pdf}

11. Скорич Л. Акт злуки 22 січня: передумови і наслідки // Львівська
політехніка. Історичні науки. – 2008. С. 97–101:
\url{http://vlp.com.ua/files/15_8.pdf}

12. Соборна Україна: від ідеї до сьогодення // Електронна бібліотека НЮУ імені
Я.Мудрого: \url{http://library.nlu.edu.ua/index.php?option=com_k2&view=item&id=304:soborna-ukraina-vid-idei-do-sohodennia&Itemid=236}

13. Тимченко Р. Акт злуки 22 січня 1919 р. та проблеми його реалізації
(січень–листопад 1919 р.) // Український історичний збірник – 2009. – Вип. 12.
– С. 183–193:
\url{http://dspace.nbuv.gov.ua/bitstream/handle/123456789/10692/23-Tymchenko.pdf?sequence=1}

14. Файзулін Я., Скальський В. Свято Злуки: унікальні фото від Інституту
національної пам'яті // Історична правда:
\url{http://www.istpravda.com.ua/artefacts/ 2011/01/22/17352/#19}

15. 1919: Петлюра приймає військовий парад. Фото і кінохроні\hyp{}кальні матеріали //
Історична правда: \url{http://www.istpravda.com.ua/videos/2010/11/12/4193/}
