% vim: keymap=russian-jcukenwin
%%beginhead 
 
%%file 01_11_2022.stz.news.ua.donbas24.2.okupanty_mrpl_majzhe_znyschyly_mural_milana.txt
%%parent 01_11_2022.stz.news.ua.donbas24.2.okupanty_mrpl_majzhe_znyschyly_mural_milana
 
%%url 
 
%%author_id 
%%date 
 
%%tags 
%%title 
 
%%endhead 

Яна Іванова (Маріуполь)
01_11_2022.jana_ivanova.donbas24.okupanty_mrpl_majzhe_znyschyly_mural_milana
Маріуполь,Україна,Мариуполь,Украина,Mariupol,Ukraine,Mural,Мурал,Mariupol.Mural,Маріуполь.Мурал,Мариуполь.Мурал,Маріуполь.Мурал.Мілана,date.01_11_2022

Окупанти в Маріуполі майже знищили мурал «Мілана»: навіщо вони це роблять
(ФОТО)

Мурал «Мілана» був символом трагедії 2015 року та надії на мир

Росіяни практично повністю закрили будівельними матеріалами мурал «Мілана» в
Маріуполі, на якому зображена трирічна дівчинка, яка постраждала від ворожого
обстрілу в січні 2015 року і в якої загинула мати, передає Донбас24.

Про це повідомляє радник мера Маріуполя Петро Андрющенко.

Мурал розташований в центрі міста і мав особливе значення для маріупольців, які
шанували пам'ять загиблих та опікувалися долею дитини. За словами Петра
Андрющенка, біля цього будинку досі не прибрані залишки від мін і місто в
окупації залишається небезпечним для людей. Окупанти закривають зображення
будматеріалами під приводом «відновлення теплового контуру».

Читайте також: Поранене та ув'язнене місто у роботах маріупольського художника

«Мурал "Мілана" знищено майже повністю. Залишились лічені дні до моменту, коли
він зникне. Окупанти розуміють, навіщо вони це роблять. Викорінити пам'ять та
будь-яку згадку про власні злочини. Натомість обіцяють містянам намалювати
чергове "патріотичне" російське лайно», — зазначає Петро Андрющенко.

Автором зображення на стіні 15-поверхівки є український художник Олександр
Корбан — мурал був створений у місті за ініціативою Фонду Ріната Ахметова
восени 2018 року. Героїнею муралу стала дівчинка Мілана Абдурашитова, якій на
момент терористичного акту у 2015 році було три роки — росіяни вранці 24 січня
відкрили вогонь з «Градів» та «Ураганів» по мирному мікрорайону «Східний». У
той день загинула 31 людина, серед яких була і мама Мілани — жінка накрила
собою дитину і врятувала доньці життя.

Згодом корегувальника вогню по «Східному» Валерія Кірсанова судили у Маріуполі
18 червня 2019 року, але по «Закону Савченко» він вийшов на свободу.

Читайте також: «Маріуполь монументальний»: у Львові надрукують книгу про
втрачене мистецтво (ФОТО)

«Мілана, маленька тендітна дівчинка, вразила мене своєю силою, хоробрістю,
любов’ю до життя, незважаючи на її трагічну життєву історію. Через її образ
хочеться передати всю ніжність, доброту та беззахисність дитини, чия
променистість та життєлюбність здатні надихнути кожного. Цим малюнком хочеться
нагадати, що бойові дії на Донбасі продовжуються. Вони дуже близько, поруч із
нами», — зазначав свого часу художник Олександр Корбан.

Маріупольці опікувалися дитиною — вона пройшла психологічну реабілітацію,
вийшла зі стану шоку та навчилася жити далі; пережила декілька операцій та
щороку проходить курс відновлення по програмі Фонду Ріната Ахметова
«Реабілітація поранених дітей», оскільки під час обстрілу втратила ніжку. Також
Мілана Абдурашитова стала героїнею фотокниги «Донбасс и Мирные».

За словами Петра Андрющенка, мурал «Мілана» в Маріуполі більше неможливо буде
відновити навіть фізичним демонтажем.

Нагадаємо, що загалом росія знищила більше 50\% будинків у Маріуполі — збитки
перевищують 14 млрд доларів. Також окупанти демонтували пам’ятник Жертвам
голодомору і політичних репресій радянського тоталітарного режиму.

Ще більше новин та найактуальніша інформація про Донецьку та Луганську області
в нашому телеграм-каналі Донбас24.

ФОТО: архів Донбас24, Олександр Корбан, Петро Андрющенко
