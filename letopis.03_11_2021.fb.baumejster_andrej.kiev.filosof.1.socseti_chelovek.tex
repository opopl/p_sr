% vim: keymap=russian-jcukenwin
%%beginhead 
 
%%file 03_11_2021.fb.baumejster_andrej.kiev.filosof.1.socseti_chelovek
%%parent 03_11_2021
 
%%url https://www.facebook.com/andriibaumeister/posts/4401249826663209
 
%%author_id baumejster_andrej.kiev.filosof
%%date 
 
%%tags baumejster_andrej.filosof.kiev,chelovek,internet,obschenie,obschestvo,psihologia,socset
%%title Социальные сети и Человек
 
%%endhead 
 
\subsection{Социальные сети и Человек}
\label{sec:03_11_2021.fb.baumejster_andrej.kiev.filosof.1.socseti_chelovek}
 
\Purl{https://www.facebook.com/andriibaumeister/posts/4401249826663209}
\ifcmt
 author_begin
   author_id baumejster_andrej.kiev.filosof
 author_end
\fi

Нет более приятного и даже сладостного (чтобы не сказать сладострастного)
занятия на фб и в других социальных сетях, чем замечать и констатировать
отсталость, неразвитость и звериную дикость некоторых пород людей.

Как это бодрит! Как укрепляет ряды! Как повышает самооценку индивида и группы!
Иногда кажется, что социальные сети для того и созданы. Они поймали в сети весь
мир. Они проглотили реальность и даже не поперхнулись.

Для чего социальные сети и "группы общения" "людей доброй воли" и всего
"прогрессивного человечества"? 

- Чтобы скреплять племенную, стадную общность и родство.

- Чтобы метить территории.

- Чтобы гнать, разрывать на части и загрызать насмерть "чужаков", "странных",
"других", "зомбированных", дисидентов, недоумков, инакомыслящих, тупых,
"выродков", "дебилов" .

Ведь они же отвратительные, они вызывают омерзение, не правда ли? Их инаковость
ощутима на запах, на нюх, она ощущается инстинктивно. От них воротит, "тошнит",
их "нельзя больше терпеть". 

Как устарели книжки про интеллект, про эмоциональный интеллект, про эмпатию.
Надо писать книги об инстинктах, о нюховом интеллекте, об интеллектуальных
запахах и ощущениях.

Здравствуй, прекрасный новый век! Где всё становится понятным на нюх, на
притирку (шерсть к шерсти), на клык.

Нет, мне очень понятен этот ток сладострастия, бегущий по могучим спинам, когда
правоверные травят очередную жертву. От такого удовольствия трудно удержаться.

Это восстание древних инстинктов. Долгожданный возврат к истокам человечества.
Ребята, а может тот американец все-таки был прав? История кончается? И
начинается великая эпоха до-истории или пред-пост-истории? Эта ошибка природы,
человек, таки исправляется? Главный первородный грех биологического мира
(отпадение человека от животного) благополучно и победно искупается. Природа
залечивает свои раны. 

Кто же искупители? Дым каких всесожжений и жертв возносится на алтарь
биологического мира? Может подскажете, а то мне плохо видно. Поднимите мне
веки)))!!

