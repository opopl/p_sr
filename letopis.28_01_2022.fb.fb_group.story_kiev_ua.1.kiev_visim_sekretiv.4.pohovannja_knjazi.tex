% vim: keymap=russian-jcukenwin
%%beginhead 
 
%%file 28_01_2022.fb.fb_group.story_kiev_ua.1.kiev_visim_sekretiv.4.pohovannja_knjazi
%%parent 28_01_2022.fb.fb_group.story_kiev_ua.1.kiev_visim_sekretiv
 
%%url 
 
%%author_id 
%%date 
 
%%tags 
%%title 
 
%%endhead 
\subsubsection{4. ДЕВ'ЯТЬ ПАМ'ЯТНИХ КНЯЖИХ ПОХОВАНЬ В ЦЕНТРІ КИЄВА, про які майже ніхто не знає}

Чи не дивина? Подивіться нижче моє відео про це місце і ви також здивуєтесь.

Багато років воджу туди своїх друзів і гостей, які приїжджають в Київ. І майже
ніхто з них ніколи не чув про це місце.

Як так може бути, що славетні князі Руси зовсім не внесені у туристичні
путівники, не встановлені скрізь відповідні вказівники, не музеєфіковано
частину колишнього монастиря, де були поховання? Загадка для мене. 

\ii{28_01_2022.fb.fb_group.story_kiev_ua.1.kiev_visim_sekretiv.scr.2}

Тим паче, що місце це одне із найкрасивіших і найбільш затишних в столиці. 

Розгадка можливо в тому, що деякі з цих князів уславились як захисники Руси в
тч. від навал Ростово-Володимиро-Суздальских воєвак, а тому не вкладалися в
героїчний пантеон часів імперії/совєтів, умисно були забуті.

\ii{28_01_2022.fb.fb_group.story_kiev_ua.1.kiev_visim_sekretiv.scr.3}

Від Мстислава Великого, найстаршого сина Володимира Мономаха та його першої
дружини Ґіти (дочки англійського короля Гарольда II Ґодвінсона). Який був
засновником кам'яної церкви святого Федора (1129), в якій його першим поховали
в 1133 році. Тут знайшли спокій ще вісім кзязів Руси-України: 

Ізяслав Мстиславич (на пам’ятнику написано, що спочив 1154 року), Ростислав
Мстиславич (1168 р.), Ярополк Ізяславич (1169 р.), Мстислав Мстиславич (1173
р.), Володимир Мстиславич (1173 р.), Мстислав Давидович (1187 р.), Ізяслав
Ярославич (1195 р.), Гліб Юрієвич (1198 р.). 

\ii{28_01_2022.fb.fb_group.story_kiev_ua.1.kiev_visim_sekretiv.scr.4}

Як бачимо, Федорівський монастир, збудований сином Мономаха князем Мстиславом
Володимировичем, являв собою родинний монастир Мстиславичів (ще однієї
героїчної родини нашої історії, яку намагались з неї викреслити). Деякі з
представників цієї родини визнані святими. Але належного пам’ятника в Києві
жодному з них досі немає.

Знайти це місце можна кількома способами: зайшовши від пам’ятника княгині
Ольги, через арку і таємним парк готелю \enquote{InterContinental}, або через
одну з арок на вулиці Володимирській (біля буд. 9 і 5), або через арку буд.6 по
вул.В.Житомирській. 

Зараз тут ми можемо бачити згаданий пам’ятник з хрестом, невеличку сучасну
дерев’яну церкву св.Георгія, яка відома тим, що містить всередині петриківський
розпис, а також і старовинні фундаменти Федорівського монастиря (все це на
відео нижче). 

Місце наповнене історією. Але відвідуючи його треба завчасно підготуватись,
почитати.

Наприклад, хоча б про те, що Мстиславичі запам'ятались в київській історії як
захисники істинної Руси. 

В тч. від нападів Юрія Довгорукого, якого зовсім юним (як найменшого в родині)
відправили в далекі провінційні землі Залісся, які залежали від Руси, але нею
не були, а саме в Ростово-Суздальську землю. І там, на землях угро-фінського
населення, під боярським наставництвом, він виріс зовсім відірваним від
традицій Києва. Взяв собі за дружину доньку половецького хана Аєпи та фактично
складав з половцями ворожу опозицію до Руси. 

А його син Андрій Боголюбський взагалі вже сприймав Русь як ворожу територію,
сам був цілком і повністю типовим азіатом по крові та вихованню. Фактично, став
половцем з іншими домішками, якого один із найавторитетніших московських
істориків назвав першим \enquote{московитом (великоросом)}: 

«З Андрієм Боголюбським великорос вперше вийшов на історичну арену». 

І цей перший великорос, який започаткує те, що ми пізніше знатимемо під
Московією і що буде природним ворогом Руси протягом століть (та і досі),
увійшов в історію як той, хто першим розграбував і попалив Київ. Сталося це в
1169 році.

Як пише про це безпрецедентне варварство наш києво-руський літопис: \enquote{Узятий же
був Київ місяця березня у дванадцятий [день], у середу другої неділі посту. І
грабували вони два дні увесь город — Подолля, і Гору, і монастирі, і Софію, і
Десятинну Богородицю. І не було помилування анікому і нізвідки: церкви горіли,
християн убивали, а других в’язали, жінок вели в полон, силоміць розлучаючи із
мужами їхніми, діти ридали, дивлячись на матерів своїх. І взяли вони майна
безліч, і церкви оголили од ікон, і книг, і риз, і дзвони познімали… Запалений
був навіть монастир Печерський святої Богородиці поганими, але бог молитвами
святої богородиці оберіг його од такої біди. І був у Києві серед усіх людей
стогін, і туга, і скорбота невтишима, і сльози безперестаннії}. 

Як бачимо, нападників літописець русами не вважає, називає їх не просто
ситуаційними ворогами (як нам це подавали в совєцьких школах), а
\enquote{поганими} - термін який застосовувався до язичників, половців,
найбільших ворогів Руси. 

Але вже 1170 року Мстислав Ізяславич почав контрнаступ і повернув собі Київ
(невдовзі помер). Після чого Андрій Боголюбський зібрав небачене доти військо з
мокшанських боліт і половецьких степів та в 1173 році вирушив в похід з метою
остаточно зруйнувати столицю Руси. В грудні 1173 року під Вишгородом це військо
було дощенту розгромлене спільним силами під командуванням Мстислава
Ростиславича та луцького князя Ярослава Ізяславича, який після перемоги став
великим київським князем. 

От такі наші герої про яких в Києві майже нічого немає. А \enquote{першого
великороса} невдовзі після того знайшли вбитим на своїх болотах. Московити
історично не пробачають своїм вождям таких поразок. Тим паче від українців  @igg{fbicon.smile}   

Так було майже 850 років тому, так буде і зараз.
