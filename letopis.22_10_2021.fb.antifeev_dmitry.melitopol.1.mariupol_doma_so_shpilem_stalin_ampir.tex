%%beginhead 
 
%%file 22_10_2021.fb.antifeev_dmitry.melitopol.1.mariupol_doma_so_shpilem_stalin_ampir
%%parent 22_10_2021
 
%%url https://www.facebook.com/Dmitry.Antifeev/posts/pfbid0SruRBsAp9b53Juoz189VKR3eWALv4dBA3aXktTVeKJZDtASYRmLqVeCEihycEzNKl
 
%%author_id antifeev_dmitry.melitopol
%%date 22_10_2021
 
%%tags 
%%title Мариуполь - дома со шпилем (сталинский ампир)
 
%%endhead 

\subsection{Мариуполь - дома со шпилем (сталинский ампир)}
\label{sec:22_10_2021.fb.antifeev_dmitry.melitopol.1.mariupol_doma_so_shpilem_stalin_ampir}

\Purl{https://www.facebook.com/Dmitry.Antifeev/posts/pfbid0SruRBsAp9b53Juoz189VKR3eWALv4dBA3aXktTVeKJZDtASYRmLqVeCEihycEzNKl}
\ifcmt
 author_begin
   author_id antifeev_dmitry.melitopol
 author_end
\fi

Мариуполь - дома со шпилем 🏣 (сталинский ампир)

Одной из основных достопримечательностей города Мариуполя стали два жилых дома
со шпилем – восточный и западный. Дома размещаются у Театральной площади на
пересечении улицы Куинджи (быв Артёма) и проспекта Мира (быв Ленина). Восточный
и западный дома со шпилем были возведены в 1953 г. на месте бывшего здания
горисполкома, которое было разрушено в период войны, по проекту известного
киевского архитектора Л. Яновицкого (харьковский институт \enquote{Горстройпроект}).
Два дома разделены между собой проезжей частью и тротуарами улицы Куинджи. В
2000 г. западный дом со шпилем выкрасили в белый цвет, а восточный так и
остался естественного кирпичного цвета.

Оба здания были выполнены в традициях классицизма середины 20 столетия:
массивный рустованный цоколь, лепнина на стенах, пилоны и колонны ионического
ордера, арочные проемы в эркерах, угловые части домов украшают шпили и фигурные
парапеты. Благодаря шпилям эти дома являются архитектурным акцентом пересечения
улицы Куинджи и проспекта Мира. В центре – семиэтажная часть, к которой
прилегают четырех и пяти этажные крылья. Фактически, две высотки достаточно
долгое время доминировали в ландшафте центральной части Мариуполя, пока рядом с
ними не установили купол храма Покрова Божией Матери, который стал самым
высоким в Донецкой области.

Все видео тут 👉 \url{https://youtube.com/c/DimaAntifeev} 🎞

\#Україна \#Украина \#Донецькаобласть \#Донецкаяобласть \#Маріуполь \#Мариуполь
\#Архітектура \#Архитектура \#СталинскийАмпир \#ХранительИстории \#DimaAntifeev
\#Антифеев \#dronephotography \#dronestagram \#mavicair2
