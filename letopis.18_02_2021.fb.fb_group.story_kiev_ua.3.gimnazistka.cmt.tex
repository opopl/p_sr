% vim: keymap=russian-jcukenwin
%%beginhead 
 
%%file 18_02_2021.fb.fb_group.story_kiev_ua.3.gimnazistka.cmt
%%parent 18_02_2021.fb.fb_group.story_kiev_ua.3.gimnazistka
 
%%url 
 
%%author_id 
%%date 
 
%%tags 
%%title 
 
%%endhead 
\zzSecCmt

\begin{itemize} % {
\iusr{Петр Кузьменко}

Трогательно и романтично. Спасибо! Ближе всего то, что мне подолянину, пришлось
побывать и скорбить на месте гибели Поэта на Моргородке Владивостока в зданиях
флотского экипажа, где была та самая его последняя точка...  @igg{fbicon.heart.sparkling} 

\iusr{Пани Елена}
Прекрасно написано!

\iusr{Оксана Денисова}
\textbf{Пани Елена} Спасибо большое!!!

\iusr{Igor Rostov}
С какого года в университет начали принимать девушек учиться на юридическом факультете??

\begin{itemize} % {
\iusr{Пани Елена}
\textbf{Igor Rostov} 

Надя Хазина училась в гимназии Жекулиной, особенностью которой было обучение
девочек по программе мужских гимназий.

В 1878г. в Киеве открыли Высшие женские курсы.

А революция 1905 года открыла для женщин двери почти всех университетов
Российской Империи.

\end{itemize} % }

\iusr{Alla Zakon}
Интересно и очень красиво написано. Спасибо

\iusr{Olena Klymenko}
Спасибо большое!!!! Очень интересные факты из жизни Поэта.

\iusr{Natalya Tarasenko}
Оксана, спасибо большое за настоящую КИЕВСКУЮ ИСТОРИЮ.  @igg{fbicon.heart.sparkling}  @igg{fbicon.bouquet} 

\iusr{Irina Locteva}
Удивительная история

\iusr{Олена Медведева- Прицкер}

Спасибо, Оксана ! Наверное я знала, но забыла. Надя Мандельштам, жена поэта,
киевлянка, с такой трудной судьбой и почетной миссией в сохранении поэзии.

\iusr{Сергій Супрун}
Очень интересно, спасибо

\iusr{Ирина Халед Эльсаед}
Спасибо. Хорошая история! @igg{fbicon.bouquet}  @igg{fbicon.thumb.up.yellow} 

\iusr{Владимир Новицкий}
Огромное спасибо, за напоминание об этих прекрасных и талантливых людях.

\iusr{Елена Константинова}
Как здорово об известных фактах еще раз услышать так романтично. Спасибо большое.

\iusr{Виктория Логинова}
Спасибо, Оксана! Очень интересно. Я этого не знала. Какой же все-таки удивительный наш Город.

\iusr{Sasha Ty}
Чудово!
Яке цікаве поряд
...ты только чуточку прищурь глаза...
А ХЛАМ живий, довелось насолоджуватись, але не відав що він існував раніш
 @igg{fbicon.smile} 

\begin{itemize} % {
\iusr{Оксана Денисова}
\textbf{Sasha Ty} кафе ХЛАМ тогда было на улице Николаевской ( сейчас Городецкого) в подвале гостиницы « Континенталь».
\end{itemize} % }

\iusr{Алла Шевченко}
Спасибо, удивительные наши киевляне

\iusr{Vira Parkhomenko}
Удивительная история о жизни Поэта и его жены! Как она пронесла любовь к нему через всю жизнь!

\iusr{Vira Parkhomenko}
Спасибо за это повествование!

\iusr{Людмила Панова}
Огромное спасибо!

\iusr{Sofiya Kazakova}
Дякую за пост

\iusr{Людмила Запорожець}
Спасибо!

\iusr{Олег Олабін}

Добре, що знаємо правду, але ще більш гірко за те, що скільки талановитих,
молодих, розумних знищив совок. Скільки поломаних надій, скалічених долей.


\iusr{людмила петрухина}
Благодарна за память о киевлянах, столько интересных людей, как многого о них не было известно.

\iusr{Svitlana Poleva}
Обожаю Ваши истории! Спасибо большое

\iusr{Оксана Денисова}
\textbf{Svitlana Poleva} Спасибо Вам!

\iusr{Irina Kaminsky}

Где—то в в конце 60—х я ехала в троллейбусе от Крещатика вверх по б. Шевченко.
Вдруг слышу женский голос : «Нам пора, Первая гимназия». После Булгакова и
Паустовского я знала, где это. Но услыхать об этом просто в житейском
контексте! Поднимается пара: высокие, стройные красивые, в чёрном. Очень
пожилые, но на зависть. Вот какие были коренные киевляне!

\begin{itemize} % {
\iusr{Оксана Денисова}
\textbf{Irina Kaminsky} Какая чудесная история ! И какая Киевская! Спасибо большое Вам за неё !

\iusr{Olga Ryabova}
\textbf{Оксана Денисова} Наша бабушка Соня училась в одном классе с сестрой Михаила Булгакова и видела царя Николая ll, когда он нанес визит в Киевскую женскую гимназию

\iusr{Оксана Денисова}
\textbf{Olga Ryabova} С ума сойти! Даже не верится, что все это так рядом от нас по времени ❤ ️ 

\iusr{Olga Ryabova}
\textbf{Оксана Денисова} Родилась в 1894, умерла на 101-м году жизни

\ifcmt
  ig https://scontent-frt3-1.xx.fbcdn.net/v/t1.6435-9/151794209_3781376131908217_1039968659152289846_n.jpg?_nc_cat=106&ccb=1-5&_nc_sid=dbeb18&_nc_ohc=jfSf1QD3xh8AX9sYfvz&_nc_ht=scontent-frt3-1.xx&oh=00_AT_O26neS6MdrieO-hcemUK8CzQvj1H_VcP_7Ho2PkdJPQ&oe=62073A1A
  @width 0.2
\fi

\end{itemize} % }

\iusr{Наталья Крутенко}
\textbf{Надежда Хазина} 

закончила мою родную СШ 138, на тогдашней Львовской улице, 27... Тогда -
гимназию Жекулиной.

\begin{itemize} % {
\iusr{Оксана Денисова}
\textbf{Наталья Крутенко} 

это действительно так, когда-то это была гимназия Жекулиной, которая учила
девочек по программе мужских гимназий!

\iusr{Наталья Агранат}
\textbf{Наталья Крутенко} это так я училась в этой школе и слышала что раньше была гимназия

\iusr{Арт Юрковская}

Я живу почти напротив и часто посещаю ее - там избирательный участок. Теперь
буду знать.
\end{itemize} % }

\iusr{Wiktoria Arto}
Спасибо! Просто невероятная история и очень печальная. @igg{fbicon.heart.red}

\iusr{Валентина Луконина}

Как тепло и интересно написали Вы о Наде Хазиной! Вечная память!

Пишите, Оксана, у Вас это чудесно получается!

\iusr{Оксана Денисова}
\textbf{Валентина Луконина} Спасибо Вам большое!

\iusr{Анна Носаль-Цепелева}

БЛАГОДАРЮ, КАКАЯ ТРОГАТЕЛЬНО - ПЕЧАЛЬНАЯ ИСТОРИЯ ЖИЗНИ И ЛЮБВИ ЭТИХ ТАЛАНТЛИВЫХ
МОЛОДЫХ ЛЮДЕЙ! @igg{fbicon.heart.sparkling} 

\iusr{Евгения Ерёменко}
Спасибо большое!

\iusr{Елена Солдатова}
Очень интересная информация. Впрочем, как всегда!

\iusr{Irena Visochan}
Спасибо! Какой прекрасный пост! Любовь и преданность!

\iusr{Иван Петренко}
Оксана! Трогательно и интересно. Спасибо!

\iusr{Людмила Краснюк}
Трогательная история... Замечательно красиво изложена! Спасибо большое!
@igg{fbicon.heart.red}

\iusr{Виктория Косцевич}
Какая трагедия...

\iusr{Vasili Vlasenko}

Оксана, Ваши знания о Киеве и его знаменитых личностях вызывает огромное
уважение. Спасибо за интереснейший рассказ.

\iusr{Оксана Денисова}
\textbf{Vasili Vlasenko} Спасибо Вам огромное!

\iusr{Renata Miroshnichenko}
Оксана, выражаю Вам свое восхищение! История и стиль изложения - прекрасны!

\iusr{Оксана Денисова}
\textbf{Renata Miroshnichenko} Спасибо Вам огромное, мне очень приятно !

\iusr{Lara Ilich}

на улице Марии Занковецкой в 2018 году установили мемориальную доску Осипу и
Надежде.. В этом доме жили родители Нади Хазиной.

\ifcmt
  ig https://scontent-frt3-1.xx.fbcdn.net/v/t1.6435-9/152130405_3798404416869729_6986114551023002998_n.jpg?_nc_cat=104&ccb=1-5&_nc_sid=dbeb18&_nc_ohc=zPJMZpxrbHUAX_oVgTo&_nc_ht=scontent-frt3-1.xx&oh=00_AT-dn1DDJL_pj3L7blcSrJwImjadW-clBz_o8SoMcbg_gQ&oe=620AA8EF
  @width 0.2
\fi

\iusr{Оксана Денисова}
\textbf{Lara Ilich} Да, я там вожу экскурсии и тоже о них рассказываю!

\iusr{Alina Gaiduchenko}
Изумительный рассказ. Пишите во благо всех, кто читает

\iusr{Оксана Денисова}
\textbf{Alina Gaiduchenko} Спасибо большое! Пишу @igg{fbicon.grin} 

\iusr{Марианна Смакова}
Спасибо, дорогая Оксана.

\iusr{Valya Jrchenko}

Город Киев имеет такую замечательно плодородную, историческую и культурную
почву, что позволила произростить целую плеяду знаменитых на весь мир людей
искусства, науки, медицины и военного дела. Центр, обрамлённый бриллиантами
знаменитостей на все времена. Киев стоит такой награды.


\iusr{Любов Франгу}
Проникливо і романтично. Щиро дякую Вам!

\ifcmt
  ig https://scontent-frt3-1.xx.fbcdn.net/v/t1.6435-9/152189750_442690400480803_5703444711765887122_n.jpg?_nc_cat=106&ccb=1-5&_nc_sid=dbeb18&_nc_ohc=JpgOTEZm5M8AX9H-OyJ&_nc_ht=scontent-frt3-1.xx&oh=00_AT9Tp-zxynsut2j-uEecMScMhH--O7ce_lkktI5Uw7NqIA&oe=62083D72
  @width 0.2
\fi

\iusr{Елена Сидоренко}
Спасибо большое Оксана! Я была недавно на Вашей экскурсии и слышала эту
историю, очень интересно! @igg{fbicon.heart.beating} 

\begin{itemize} % {
\iusr{Оксана Денисова}
\textbf{Елена Сидоренко} Спасибо Вам! Приходите ко мне на другие Экскурсии!

\iusr{Елена Сидоренко}
\textbf{Оксана Денисова} обязательно!@igg{fbicon.heart.red}
\end{itemize} % }

\iusr{Лариса Ярмолюк}
Спасибо за такой интересный рассказ!

\iusr{Наталія Кохан}
Дякую. Чудова історія. Взнала нові подробиці про життя чудових людей.

\iusr{Таня Сидорова}
Дякую. Дуже зворушливо та цікаво.

\iusr{Ольга Филенко}
Может вы что-то напутали? На ул. Софиевской по четной стороне последний номер
идет 20. Доска есть на доме №3 Б.

\begin{itemize} % {
\iusr{Оксана Шляхова}
\textbf{Ольга Филенко} номер 3 б

\iusr{Лариса Коваленко}
\textbf{Ольга Филенко} адреса 3б

\iusr{Оксана Денисова}
\textbf{Ольга Филенко} Доска внизу, последний дом перед Майданом 3 б, может Вы просто прочитали этот номер как 36?

\iusr{Ольга Филенко}
\textbf{Оксана Денисова} У вас в тексте именно 36
\end{itemize} % }

\iusr{Людмила Чуприна}
Спасибо Киеву! Стольких людей и судеб пригрел и окрылил!

\iusr{Aniouta Chamonix}
Спасибо огромное !!

\iusr{Таня Гур}
Чудесный рассказ! @igg{fbicon.hands.applause.yellow} 

\iusr{Мария Дроботько}
Спасибо большое.

\iusr{Светлана. Чистякова}
Светлая память прекрасной женщине Наде Хазиной и её любимому поэту Мальденштаму, горькая судьба, светлая память.

\iusr{Татьяна Грицай}
\textbf{Оксана Денисова}, заворожили уже который раз. Спасибо огромное.
Потеплеет и сразу попрошусь к Вам на экскурсию, чтобы ещё и послушать.

\iusr{Оксана Денисова}
\textbf{Татьяна Грицай} Татьяна , спасибо большое! Приходите на экскурсии, буду рада Вас видеть!

\iusr{Valeriya Belaya}
Спасибо большое!

\iusr{Марина Тоцкая}
Благодарю

\iusr{Неоніла Ганюк}
Дякую

\iusr{Natalya Olyanishyna}
Как по улицам Киева - Вия...

\iusr{Наталия Нежура}
Спасибо!

\iusr{Iryna Fokina}
Оксана, спасибо большое!

\iusr{Тамара Чайко}
Благодарю @igg{fbicon.heart.growing}  @igg{fbicon.heart.blue}
@igg{fbicon.heart.purple}  @igg{fbicon.heart.beating} 

\iusr{Ирина Прозоровская}
Спасибо. Очень интересно!!!!

\iusr{Александр Андриевский}
Сказочная история...

\iusr{Ирина Иванченко}

Всю свою жизнь положила на его алтарь... Жена декабриста - ассоциация... Спасибо
ей, что нам дала возможность наслаждаться его стихами... Спасибо вам, Оксана, за
то, что дали возможность вспомнить о них ещё раз...

\iusr{Оксана Денисова}
\textbf{Ирина Иванченко} Спасибо Вам!

\iusr{Dina Kozlova}
Спасибо !
Очень светлая грустная история ...

\iusr{Мария Бутковская}
Спасибо за интересную историю из жизни замечательных людей нашего города !

\iusr{Ірина Литвиненко}
Спасибо!

\iusr{Людмила Нечай}
Интересно

\iusr{Людмила Ковальская}
Дякую!!

\iusr{Ірина Гриценко}
Дякую, дуже цікаво. Просто зачитуюся.

\iusr{Светлана Макарова}
Спасибо, спасибо, спасибо!

\iusr{Vadik Nazarenko}
Спасибо, очень интересно.

\iusr{Любовь Плотникова}

\iusr{Ольга Дзюбан}

У вісімдесятих роках управління \enquote{Київрембуд} робило реконструкцію цього
будинку. Мінялись дерев'яні перекриття на залізобетонні, робилось
перепланування приміщень. Керував реконструкцією Бунчук Василь Петрович.
Мабуть, його вже давно немає...


\iusr{Сергей Хромешкин}
Благодарю за экскурс в историю... шикарно...

\iusr{Ольга Малевская}
Читайте прекрасные воспоминания Н. Мандельштам!

\begin{itemize} % {
\iusr{Lilia Grigoorova}
\textbf{Ольга Малевская} 

Совсем недавно перечитывала... Какие нечеловеческие испытания выпали на их долю.
Перед стоической верностью Надежды Мандельштам склоняю голову !


\iusr{Ольга Малевская}
Согласна полностью!

\end{itemize} % }

\iusr{Светлана Прокопенко}
Благодарю!

\iusr{Любава Васильченко}
Интересно: где чейчас потомки известных киевлян?

\iusr{Жанна Строкина}

Унеси меня в ночь, где течет Енисей, где сосна до звёзды достает. Потому что не
волк я по крови своей и меня только равный убьет. О. Мальдештам

\iusr{Татьяна Оксаненко}
Невероятно! Спасибо, Вам, Оксаночка. Удивительная преданность любви, памяти о
человеке, его творчеству!

\iusr{Раиса Карчевская}
Спасибо большое за очень интересный пост

\iusr{Ирина Данькова}
Очень интересно !!

\iusr{Лариса Кусенко}
Огромное спасибо, помнить о таких людях, так повествов ать, спасибо Вам!

\iusr{Татьяна Коршак}
Благодарю.

\iusr{Тамара Метленко}
С благодарностью

\iusr{Olga Svidan}
Дякую, серце відтає.

\iusr{Alla Vorobjova}
Дякую, історія проймає.

\iusr{Роксолана Рудяка}

Благодарю Вас, Оксана! Это такое удовольствие - читать подобные посты с
чудесным экскурсом в истории тех улиц, на которых вырос!!

\iusr{Оксана Денисова}
\textbf{Роксолана Рудяка} Спасибо большое!

\iusr{Катерина Сотник}

Однажды я был свидетелем встречи Маяковского с Мандельштамом. Они не любили
друг друга. Во всяком случае, считалось, что они полярные противоположности,
начисто исключающие друг друга из литературы. Может быть, в последний раз перед
этим они встретились еще до революции, в десятые годы, в Петербурге, в
«Бродячей собаке», где Маяковский начал читать свои стихи, а Мандельштам
подошел к нему и сказал: «Маяковский, перестаньте читать стихи, вы не румынский
оркестр». Маяковский так растерялся, что не нашелся, что ответить, а с ним это
бывало чрезвычайно редко. И вот они снова встретились.

В непосредственной близости от памятника Пушкину, тогда еще стоявшего на
Тверском бульваре, в доме, которого уже давным-давно не существует, имелся
довольно хороший гастрономический магазин в дореволюционном стиле.

Однажды в этом магазине, собираясь в гости к знакомым, Маяковский покупал вино,
закуски и сласти. Надо было знать манеру Маяковского покупать! Можно было
подумать, что он совсем не знает дробей, а только самую начальную арифметику,
да и то всего лишь два действия — сложение и умножение.

Приказчик в кожаных лакированных нарукавниках — как до революции у Чичкина — с
почтительным смятением грузил в большой лубяной короб все то, что диктовал
Маяковский, изредка останавливаясь, чтобы посоветоваться со мной.

— Так-с. Ну, чего еще возьмем, Катаич? Напрягите все свое воображение. Копченой
колбасы? Правильно. Заверните, почтеннейший, еще два кило копченой
«Московской». Затем: шесть бутылок «Абрау-Дюрсо», кило икры, две коробки
шоколадного набора, восемь плиток «Золотого ярлыка», два кило осетрового
балыка, четыре или даже лучше пять батонов, швейцарского сыра одним большим
куском, затем сардинок...

Именно в этот момент в магазин вошел Осип Мандельштам — маленький, но в очень
большой шубе с чужого плеча, до пят, — и с ним его жена Надюша с хозяйственной
сумкой. Они быстро купили бутылку «Кабернэ» и четыреста граммов сочной ветчины
высшего сорта.

Маяковский и Мандельштам одновременно увидели друг друга и молча поздоровались.
Некоторое время они смотрели друг на друга: Маяковский ядовито сверху вниз, а
Мандельштам заносчиво снизу вверх, — и я понимал, что Маяковскому хочется
как-нибудь получше сострить, а Мандельштаму в ответ отбрить Маяковского так,
чтобы он своих не узнал.

Я изучал задранное лицо Мандельштама и понял, что его явное сходство с
верблюдиком все же не дает настоящего представления о его характере и
художественно является слишком элементарным. Лучше всего изобразил себя сам
Мандельштам:

«Куда как страшно нам с тобой, товарищ большеротый мой! Ох, как крошится наш
табак, щелкунчик, дружок, дурак! А мог бы жизнь просвистать скворцом, заесть
ореховым пирогом... Да, видно, нельзя никак...»

Он сам был в этот миг деревянным щелкунчиком с большим закрытым ртом, готовым
раскрыться как бы на шарнирах и раздавить Маяковского, как орех.

Сухо обменявшись рукопожатием, они молчаливо разошлись. Маяковский довольно
долго еще смотрел вслед гордо удалявшемуся Мандельштаму, но вдруг, метнув в мою
сторону как-то особенно сверкнувший взгляд, протянул руку, как на эстраде, и
голосом, полным восхищения, даже гордости, произнес на весь магазин из
Мандельштама:

— «Россия, Лета, Лорелея».

А затем повернулся ко мне, как бы желая сказать: «А? Каковы стихи? Гениально!»

Это была концовка мандельштамовского «Декабриста»:

«Все перепуталось, и некому сказать, что, постепенно холодея, все перепуталось,
и сладко повторять: Россия, Лета, Лорелея». <...>

(Валентин Катаев. Трава забвения, 196З)


\iusr{Оксана Денисова}
\textbf{Катерина Сотник} Класс! Когда-то читала, но давно, а ведь дивно написано!

\iusr{Катерина Сотник}

«...этой ночью мне долго и сладко снился Осип Мандельштам, бегущий в дожде по
Тверскому бульвару при свете лампионов, мимо мокрого чугунного Пушкина со
шляпой за спиной, вслед за экипажем, в котором я и Олеша увозили Надюшу.

Надюша – это жена Мандельштама. Надежда Яковлевна. Мы увозили её на Маросейку –
угол Покровского бульвара, в пивную, где на первом этаже, примерно под
кинематографом «Волшебные грёзы, выступали цыгане. У нас это называлось:
«Поедем экутэ ле богемьен» («слушать цыган»).

Мы держали Надюшу с обеих сторон за руки, чтобы она не выскочила сдуру из
экипажа, а она, смеясь, вырывалась, кудахтала и кричала в ночь:

— Ося, меня умыкают!

Мандельштам бежал за экипажем, детским, капризным голосом шепелявя несколько в
нос:

— Надюса, Надюса... Подождите! Возьмите и меня. Я тоже хочу экутэ.

Но он нас так и не догнал, а мы, вместо чтобы ехать на Маросейку – угол
Покровского бульвара, почему-то поехали в грузинский ресторан, который тогда
помещался не там, где сейчас находится «Арагви», и даже не там, где до «Арагви»
помещалась «Алазань», а – вообразите себе! – в том доме на бывшей Большой
Дмитровке, а теперь Пушкинской, где сейчас находится служебный ход Центрального
детского театра, что может показаться совершенно невероятным, но тем не менее
это исторический факт, и содержало эту шашлычную частное лицо, так как был
разгар нэпа.

Но это, в сущности, не важно, а важно то, что шёл дождь и мы таки втащили
Надюшу за руки на второй этаж в отдельный кабинет – удивительно скучную и плохо
освещенную комнату, никак не обставленную и похожую скорее на приемную в
собачьей лечебнице. Сюда нам принесли бутылку «телиани», а как только его
принесли, тотчас появился мокрый и возбужденный Мандельштам, прибежавший по
нашему следу, и он сейчас же начал с завыванием, шепеляво и очень внушительно –
«как Батюшков Дельвигу»! – читать новые стихи, нечто вроде:

Я буду метаться по табору улицы темной,

За веткой черемухи в черной рессорной карете,

За капором снега, за вечным за мельничным шумом.

И так далее – можно проверить и восстановить по книжке Мандельштама, если её
удастся достать, – мне именно так приснилось: «если ее удастся достать», а
Мандельштам моего старого сновидения тем временем сел пить «телиани», вспомнил
гористую страну и, шепеляво завывая, стал вкрадчиво и вместе с тем высокомерно,
даже сатанински гордо декламировать о некоей ковровой столице над шумящей
горной речкой и о некоем духанчике, где «вино и милый плов».

И духанщик там румяный

Подаёт гостям стаканы

И служить тебе готов...

Хорошо в подвале пить, —

Там в прохладе, там в покое

Пейте вдоволь, пейте двое,

Одному не надо пить.

Его мольбы не имели никакого практического смысла, так как мы пили вчетвером и
само собой подразумевалось, что одному не надо пить. Одному надо было только
платить! Затем пошли очаровательные трюизмы:

Человек бывает старым,

А барашек молодым,

И под месяцем поджарым

С розоватым винным паром

Полетит шашлычный дым.

Собственно говоря, всё это мне вовсе не снилось, а было на самом деле, но так
мучительно давно, что теперь предстало передо мной в форме давнего, время от
времени повторяющегося сновидения...»

(В.Катаев. Святой колодец).

\begin{itemize} % {
\iusr{Оксана Денисова}
\textbf{Катерина Сотник} Спасибо!!!

\iusr{Катерина Сотник}
\textbf{Оксана Денисова}
Не за что ☺.
\end{itemize} % }

\end{itemize} % }
