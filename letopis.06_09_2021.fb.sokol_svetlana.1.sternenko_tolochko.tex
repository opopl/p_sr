% vim: keymap=russian-jcukenwin
%%beginhead 
 
%%file 06_09_2021.fb.sokol_svetlana.1.sternenko_tolochko
%%parent 06_09_2021
 
%%url https://www.facebook.com/permalink.php?story_fbid=173701854841195&id=100066041441718
 
%%author_id sokol_svetlana
%%date 
 
%%tags radikalizm,rusmir,sternenko_sergei,tolochko_pp,ukraina
%%title Радикал Стерненко атаковал историка Петра Толочко
 
%%endhead 
 
\subsection{Радикал Стерненко атаковал историка Петра Толочко}
\label{sec:06_09_2021.fb.sokol_svetlana.1.sternenko_tolochko}
 
\Purl{https://www.facebook.com/permalink.php?story_fbid=173701854841195&id=100066041441718}
\ifcmt
 author_begin
   author_id sokol_svetlana
 author_end
\fi

Ведь эти все Стерненки в силу своей ущербности  и  мерзости ни перед чем не
останавливаются..... безмозглая биомасса страшна.... Толочко надо спасать....!

***

Радикал Стерненко атаковал историка Петра Толочко. 

Правосек требует изгнать его из академии наук и лишить званий из-за «любви к
русскому миру»

Националисты во главе с убийцей-радикалом Сергеем Стерненко начали травлю
академика Петра Толочко, требуя лишить его званий и регалий за поддержку
«русского мира». Триггером истерики патриотов стало интервью академика, которое
он дал Елене Бондаренко накануне юбилея Независимости.

\ifcmt
  pic https://scontent-frt3-1.xx.fbcdn.net/v/t1.6435-9/241439759_173699704841410_5414088738854899104_n.jpg?_nc_cat=107&ccb=1-5&_nc_sid=730e14&_nc_ohc=_wMYSzhkuSgAX_gDcXV&_nc_ht=scontent-frt3-1.xx&oh=69407c858da7bc94de1f196dee62bfc4&oe=615D1CD4
  @width 0.8
\fi

Пётр Толочко рассказал, что Конституция Филиппа Орлика вообще не является
Конституцией, в ней нет упоминания об Украине и украинцах. Документ написан на
русском, а оригинал хранится в Москве. «По сути, это договор гетьмана в экзиле
со своими старшинами, и рассчитан этот документ на то, что если удастся
вернуться в Малороссию и восстановить свою власть с помощью шведского короля
Карла, то мы будем пользоваться положениями этого документа», — пояснил
историк. Также Толочко заявил, что гордится, когда его называют малороссом.

«Мы, малороссы, создавали большую империю — Российскую империю. И я себя сам
считаю малороссом, ничего обидного или унизительно, как пытаются доказать
„вятровичи‟, в этом нет. Мы все происходим от одного корневища Древней Руси,
Потом мы потеряли своё название, Русь не потеряла, а мы — да. Малая Русь — это
не от размеров страны, это от её сердцевины, как Иль де Франс, как малая
Польша. Гордиться надо, что в днепровской губернии эта русская жизнь
зарождалась, а не стыдиться или обижаться», — объяснил он.

По его словам, название «Украина-Русь» придумал Грушевский, кстати,
завербованный английской разведкой. Он поставил Украину впереди, вроде она была
первой и главной. И теперь это название активно эксплуатируют творцы новой
украинской истории, препираясь с Россией на тему, кто главнее.

«Наше отречение от исторического прошлого, это всё равно, что отказ от своих
родителей. Вот есть где-то за океаном богатая тётя, я к ней пойду, а родителей
вычеркну из памяти. Отрекаясь от своих пращуров, от их деяний, от памяти про
них мы теряем фундамент. Пример: ПЦУ уже говорит, мол, зачем идеализировать
Владимира и Крещение? Дескать, и до него были христиане. Но именно князь
Владимир, крестив Русь, изменил весь строй огромного государства, он ввёл Русь
в византийское содружество, и Русь, благодаря этому, получила необычайное
ускорение. А теперь оказывается, что и это нам не нужно...», — сетует историк.
Он подчёркивает, что не раз отмечал в своих статьях: и оранжевая революция, и
революция гидности сдетонируют распад Украины. «Я оказался пророком: и Крыма
нет и Донбасса, считайте, нет. Государство, которое строится без фундамента —
поскольку мы от всего отказываемся... от корней, от прошлого, от истории — оно
нежизнеспособно», — сказал в интервью программе «Ежу понятно» Пётр Толочко.

Стоит ли удивляться, что эти исторические констатации вызвали истерику в стане
национал-патриотов, которые лепят из Украины «Антироссию»?

Бывший правосек Стерненко, ныне находящийся под «крылышком» американской
лоббистки Ульяны Супрун, начал активную травлю Петра Толочко, добиваясь его
изгнания из Национальной академии наук Украины (НАНУ).

«Пока одних украинцев из учебных заведений выпускают, другие чувствуют себя
хорошо и, получая зарплату из наших налогов, отрицают существование украинцев и
Украины, катаются по Москве и обливают грязью нашу страну», — возмутился
обвиняемый в убийствах и пытках активист. Он напомнил, что благодаря «активной
позиции» патриотов накануне нового учебного года из одесских и киевских ВУЗов
изгнали таких «коллаборантов», как Денис Сытник и Евгения Бильченко: они
выступали против мовы, поддерживали оккупантов. А теперь пришла пора приняться
и за более внушительную фигуру — академика Петра Толочко.

Стерненко предъявил Толочко обвинения, что академик считает Украину частью
русского мира, а русских и украинцев — одним народом. Кроме того, исторические
труды украинского академика хвалит и читает сам депутат Госдумы Константин
Затулин. Ну и вишенка на торте: «Помогает Толочко и Виктору Медведчуку. Он
выступает на его каналах и на русском языке рассказывает, что на Украине травят
русскоязычных... кому-то там запрещают говорить на родном языке», — глумится
Стерненко.

Надо сказать, что у него нашлось много сторонников, которые благодарят убийцу
за важную работу на благо государства и заявляют, что «сумасшедшему деду не
место в научных кругах Украины». А некоторые заявляют, что все археологические
ценности с раскопок «Толочко вывез в Эрмитаж и музеи Москвы, за что получал от
России деньги и звания».

В общем, никто с фактами, которые приводит историк Толочко, спорить не
собирается. Ведь фактаж необходимо изучать, а у стерненкоподобных нехватка ума
и образования. Но им достаточно просто плеваться ядом в того, кто рушит их
хуторское представление об окружающем мире. 

И пока плевки получаются смачными...

Ольга Талова
ИА «Антифашист»
06.09.2021
