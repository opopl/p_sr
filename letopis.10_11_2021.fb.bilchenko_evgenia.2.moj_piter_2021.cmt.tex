% vim: keymap=russian-jcukenwin
%%beginhead 
 
%%file 10_11_2021.fb.bilchenko_evgenia.2.moj_piter_2021.cmt
%%parent 10_11_2021.fb.bilchenko_evgenia.2.moj_piter_2021
 
%%url 
 
%%author_id 
%%date 
 
%%tags 
%%title 
 
%%endhead 
\subsubsection{Коментарі}
\label{sec:10_11_2021.fb.bilchenko_evgenia.2.moj_piter_2021.cmt}

\begin{itemize} % {
\iusr{Анна Сергеева}

Захожу почитать. Любопытно. "БЖ была готова бежать по минным полям республик,
как ей советовали мирные трактористы Донбасса, дабы на Васильевский остров
доставить себя грузом 200.". Евгения, а что, просто сесть в автобус или поезд и
выехать в Россию нельзя было? Вы хотите сказать, что из Украины, как из СССР,
не разрешен выезд? И надо проходить собеседование в горкоме, обкоме,
парт. ячейках, чтобы дали добро? Я понимаю, поэт - натура творческая и с
фантазией. Но так врать-то зачем? Просто вы понимаете, что приехав в любимую
Россию, вы столкнетесь с банальными бытовыми трудностями. Никто вам не оплатит
жилье и не станет содержать, и в МГУ тоже вряд ли на должность профессора
возьмут. Поэтому и сидите в ненавидимой Украине, где масса "фаш-тов и
бандер-цев". Р. s. Куда интересней читать ваши сравнения себя с Христом.

\begin{itemize} % {
\iusr{Евгения Бильченко}
\textbf{Анна Сергеева} 

Рассказываю как человек, который много знает, ибо прошел на себе. Врагам народа
вроде меня из Украины не просто разрешен выезд, а патриотическая общественность
просто делает рыжему биографию, как говорила драгоценная Анна Андреевна, био,
куда входит всё: судилища, общественная травля, лишение средств к
существованию, уличные нападения, физические и с камерой, слежка у дома,
занесенного в "Миротворец"... Я всё подробно в свое время расскажу, как Украина
обращается с иным мнением. Чтобы дали добро на ПМЖ в РФ, да, надо проходить
собеседование в СБУ, идти туда лично, но в моих планах не было ни ПМЖ, ни СБУ,
ибо в последнем меня просто допрашивали по статье 258. Интересные допросы,
кстати, я их описала в Amnistia International. После того, как через Oxana
Chelysheva на них надавили ООН, меня хотя бы прекратили спрашивать о Захаре. Но
это лирика. Вернёмся к праксису. Выехать можно, въехать в другое государство в
условиях карантина - очень трудно. Согласно пункту 5 законодательства РФ въезд
запрещен всем, кроме: родственников 1 линии, на работу по определенным видам
работы (спорт, кино, органы власти), по медсправке на лечение, если она прошла
контроль на госуслугах РФ, то есть предоставляется реальный диагноз. Мой случай
- последний. Что там ещё у вас? Нет, на работу в МГУ меня не возьмут
профессором, а на Украине меня не берут преподавателем ни в один вуз, только
уборщицей, как сторонницу русского мира, о чем я оповещала общественность,
опубликовав список вакансий, пошарьте ниже. Но мне от России ничего не надо, у
нас с ней по любви секс, а не за плату. Что там ещё? Вроде все. Странно, натура
с фантазией я, а творческие истерики с полным незнанием миграционного
законодательства закатывает обыватель, вроде адекватный человек. Прямо мне аж
неловко от собственного прагматизма)

\iusr{Евгения Бильченко}
\textbf{Анна Сергеева} вы ещё заходите "пачитать", вас всегда не хватает, у вас комменты прикольные!

\iusr{Максим Чикало}
\textbf{Евгения Бильченко} 

на Позняках в Киеве до недавно жила такая себе Пикта Светлана, так же как и вы,
тронутая умом гражданка, с полным как и у вас, отсутствием понимания причинно -
следственных связей. Так вот она таки после некоторых уговоров отправилась на
родину дидов и живёт себе где-то там среди так близких сердцу, крепостных.

Так что, мадам, дерзайте, всё в ваших руках. Сами освободитесь от "фошиздов" и
освободите "фошиздов" от себя!!!!

\end{itemize} % }

\iusr{Сергей Никонов}

Для меня душевное устремление - Логос. Верю в МИР, дружбу порядочных людей и
прекращение войны. Практическое приложение Логоса - рынок. Это может быть и по
жизни, и по науке, включая экономику. Идеалы важны, но без учета реальностей
они ничтожны или опасны. Сегодня нужно действовать в условиях рыночной или
смешанной экономики. Даже если соглашаться с идеей креативной экономики, то у
нее рыночный принцип действия. Приведу пример. Хотите заказать стихотворение
или для Вас это единственный выход? Тогда ищите поэтов с подходящим стилем и
устраивающей ценой. Что касательно выезда. Напишите реально, как нелегально в
Россию попасть. Ведь болеть-то не хочется, а бюрократы не пустят по легальным
причинам. Ну ж это и политически правильно. "Несознательные" с точки зрения
украинского государственного патриотизма личности покидают страну. Многие люди
хотят, да не у всех родня. Пока мне нравится Россия за самобытность в этом
глобальном и глобализируемом по западным стандартам мире. Остается собой и
стоит на своем. Как и Китай.

\begin{itemize} % {
\iusr{Евгения Бильченко}
\textbf{Сергей Никонов} 

допустим, я тоже не против рынка, если Логос в приоритете, чистый коммунизм не
сработал. Я и сама стала частью рынка через фриланс, я не традиционалист в
лаптях. И мне нравится, что в России брендирование не глокалистское. Тут
национальный бренд не является сервисным приложением глобального, хотя
имплицитный либерализм пытается через бюрократический аппарат глокализм
толкнуть как лайт-версию "Яблока" через new lefts.

\end{itemize} % }

\iusr{Максим Чикало}

Поток душевно-больного человека или человека, живущего в параллельной реальности...

\begin{itemize} % {
\iusr{Евгения Бильченко}
\textbf{Максим Чикало} дякую, це саме те, що треба. Бажаю справжньому українцю трохи більше, аніж 31 підписувач, бо ж, очевидно, що ваша реальність є актуальнішою, і гарну подругу, аби, зрештою, була вже нормальна світлина, а не ці інфернальні селфачі.

\iusr{Максим Чикало}
\textbf{Евгения Бильченко} мадам, абсолютно искренне, обратитесь к врачу....

\iusr{Евгения Бильченко}
\textbf{Максим Чикало} к онкологу, с удовольствием. Выслать карту? К психологу с удовольствием. Тоже принимаю донаты. К психиатру не могу, в самом разгаре травли, когда ваши побратимы придумывали карательную психиатрию для Бродского, я решила вас опередить и сама пошла клянчить транквилизатор на Фрунзе. Представляете, не дали. Поразились мужеству и стойкости нервов, посоветовали бифрен и ещё какие-то пустышки.

\iusr{Максим Чикало}
\textbf{Евгения Бильченко} 

украинский врач вам не поможет, чтобы вас реально вылечили, вам нужно страну
проживания поменять на российскую федерацию и уже там воспользоваться услугами
местных специалистов. Это на 100\% решит все ваши фобии, душевные и психические
расстройства! Это единственный и верный выход!

Не благодарите!

\iusr{Евгения Бильченко}
\textbf{Максим Чикало} не благодарю, потому что не могу подарить вам этого счастья.
\end{itemize} % }

\iusr{Виталька Ульяненко}
СУПЕР @igg{fbicon.heart.red}

очень хорошо написали, Евгения!
Грех не поделится контентом!
Репост.

\iusr{Виталька Ульяненко}
Тут видео из Ютуба, супер.

\iusr{Тим Печеніг}

Євгенія Віталіївна! Можливо можу висловити вам те що з вами не погоджуюсь, але
те що ви молодець що не витираєте відрізні від вашої думки коменти, а навпаки
досить змістовно на них відповідаєте. Не зрозумів з вашого тексту ви вже наразі
в РФ чи згодом переїзджаєте?

\end{itemize} % }
