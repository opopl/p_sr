% vim: keymap=russian-jcukenwin
%%beginhead 
 
%%file 25_10_2019.news.pravda_com_ua.SPARTA.doroga_k_SPARTE
%%parent 25_10_2019.news.pravda_com_ua.SPARTA
 
%%url 
%%author 
%%tags 
%%title 
 
%%endhead 

\subsubsection{Дорога к С.П.А.Р.Т.Е.}
\label{sec:25_10_2019.news.pravda_com_ua.SPARTA.doroga_k_SPARTE}

Учхоз с древнегреческим названием "С.П.А.Р.Т.А." находится в селе Караван в 25
километрах от Харькова. Автобус доезжает до небольшого городка Люботин. До
самой деревни дважды в день в не самое удобное время курсирует маршрутка.

"C.П.А.Р.Т.А.нцы" подхватывают нас на полпути на стареньком Volkswagen Golf. На
приборной панели машины корректором выведено слово "Тезис", ниже висит длинный
список под названием "Дефектная ведомость". За рулем сидит женщина средних лет
в соломенной шляпе, рядом на пассажирском сидении --- знакомая нам Ольга Широкая,
которая предварительно одобрила наш приезд.

– Томка я, --- радушно протягивает руку женщина в соломенной шляпе и тут же с
энтузиазмом начинает расспрашивать нас о том о сем.

\ifcmt
img_begin 
	url https://img.pravda.com/images/doc/a/e/aec4750-1sparta.jpg
	caption Томка --- Тамара Костюк занимает в С.П.А.Р.Т.Е. должность старшего гвардейца.
Все фото Константина Буновского.
	width 0.7
img_end
\fi
  
За окном простирается типичный деревенский пейзаж --- неприметные домики с
поросшими бурьяном участками и покосившимися заборами. Через несколько минут
тряски на грунтовой дороге доезжаем до пункта назначения.

– Как говорится, милости просим, чем богаты, тем и рады! --- улыбается Тамара и
приглашает нас в беседку "Спартак".

