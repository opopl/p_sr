% vim: keymap=russian-jcukenwin
%%beginhead 
 
%%file 13_12_2020.news.ru.ukraina_ru.kevorkjan_konstantin.1.obezbolivanie_cherez_nenavist
%%parent 13_12_2020
 
%%url https://ukraina.ru/opinion/20201213/1029953893.html
 
%%author Кеворкян, Константин
%%author_id kevorkjan_konstantin
%%author_url 
 
%%tags 
%%title Обезболивание через ненависть. Альтернативный разум на страже завоеваний Майдана
 
%%endhead 
 
\subsection{Обезболивание через ненависть. Альтернативный разум на страже завоеваний Майдана}
\label{sec:13_12_2020.news.ru.ukraina_ru.kevorkjan_konstantin.1.obezbolivanie_cherez_nenavist}
\Purl{https://ukraina.ru/opinion/20201213/1029953893.html}
\ifcmt
	author_begin
   author_id kevorkjan_konstantin
	author_end
\fi

\ifcmt
pic https://cdn1.img.ukraina.ru/images/102933/55/1029335584.jpg
\fi

\textbf{Недавно уважаемый коллега Дмитрий Скворцов выставил на своей странице в ФБ
множество фотографий современного Киева - и я ужаснулся. В любом большом
городе, если поискать, можно найти заброшенные и запущенные места. Но то, что
показал Дмитрий - нередко самый центр когда-то наичистейшего города, любимого с
детства Киева - ныне замызган и загажен}

Но ведь киевляне снова переизбрали действующего мэра, господина \textbf{Кличко} —
значит, в большинстве своём, их это устраивает. Его отрыв от «хозяйственника»
\textbf{Александра Попова} в любом случае огромен. Более того, Виталий Кличко с
подобными муниципальными «достижениями» претендует на пост президента всея
Украины! И не без шансов.

Люди на Украине быстро привыкают к новому «стандарту некачества»: худшей среде
обитания, дешевой еде, фейковым новостям. Например, многие до сих пор свято
уверены, что в «стране-агрессоре» живётся значительно хуже нежели им, и эта
уверенность, возможно, реально помогает им выжить.

Вспоминается душевное потрясение российского блогера \textbf{Бека
Ибрагимбекова} после разговора с некоторыми жителями Незалежной: «Только что
закончил общение по скайпу со старыми знакомыми моей семьи, которые живут под
Киевом. Внимание, объясняю в чем дело — они думают… что в Крым мост не
построили и нет аэропорта.  Они меня уверяли, что это компьютерный монтаж! ОНИ
НЕ ВЕРЯТ!!:))))))» [орфография автора]

В комментариях читатели приводят и другие выразительные примеры альтернативной
реальности, прочно укоренившейся в украинском бессознательном: «Ну, вот у нас
куча подобных придурков твёрдо убеждена, что в России "мрут как мухи от ихней
прививки, влияющей на ДНК". Это цитата была»; «Угу, запорожские родственники
считают, что напал Путин и бомбит Донецк» и т.д.

Разумеется, это заострённый пример, но альтернативная реальность действительно
играет роль некоего наркотика, которое многим гражданам Украины помогает
справиться с ежедневным стрессом, подменяя окружающую жизнь искусственными
«перемогами», например, грандиозной Битвой за Борщ.

«Мы не просто заявили о том, что борщ — украинский. Мы оказали сопротивление
России и ее пропаганде, и нас поддержал мир», — сообщает тамошний телекулинар
\textbf{Евгений Клопотенко}; «Почему Россия пытается присвоить себе украинский борщ? Это
принципиальный вопрос. Чей борщ, того и народ», — делает глубокомысленный вывод
украинский журналист \textbf{Вахтанг Кипиани}; а министр культуры \textbf{Александр Ткаченко}
одолевает супостата с помощью ЮНЕСКО, куда подаёт заявку о борще как
нематериальном наследии Украины.

Изрядно обедневшие за последние годы «пэрэможци» горды и счастливы — кроме тех,
кто задыхается от нехватки кислорода в украинских больницах и скончался от
разгула эпидемий, убит в криминальных разборках или на фронте, умирает от
тарифомора и вопиющей нищеты.

Зато Украина как бы участвует в освоении космоса. «Смотрю на "плюсах" [канал
"1+1"] сюжет о том, как Украина (!) участвует (!) в гонке за освоение (!) Луны
(!), — щедро расставляет восклицательные знаки в своём ТГ-канале вполне
промайданный политолог \textbf{Юрий Романенко}. — Все мы понимаем, что государство,
которое постоянно клянчит пару ярдов у МВФ, чтобы закрыть постоянно растущую
дырку в бюджете, ни в какой гонке за Луну априори участвовать не может. Но это
такая форма национального мазохизма — все время возбуждать себя дурацкими
фантазиями, чтобы поднять свою значимость в своих глазах, не замечая хохот
всего мира над этими дурными потугами».

Реальность скучна, реальность грязна, реальность тосклива. Ну — повышают себе
члены цензурного Национального совета по телевидению и радиовещанию зарплаты в
2,5 раза до 70-80 тысяч гривен (плюс надбавки, премии и доплаты на
оздоровление), но они же с «вражеской пропагандой» воюют, им же надо.

Или вот представитель уполномоченного Верховной Рады по правам человека в
Луганской и Донецкой областях \textbf{Павел Лисянский} в Святогорске избил пожилого
охранника, который попросил украинского «правозащитника» покинуть ресторан
после закрытия заведения. Или вице-премьер Украины \textbf{Олег Уруский} демонстративно
нарушил карантинные меры в Турции; или министр инфраструктуры \textbf{Владислав Криклий}
пытался с просроченными тестами на COVID прорваться на встречу с
премьер-министром Казахстана… Зато — «Борщ наш»!

Как говорит киевский шоумен \textbf{Орест Лютый}, надо «вести просветительскую работу на
этом этапе национального формирования… Вон от Москвы! Во всех вариациях это
должно звучать в данном случае. Москва — это азиатская, скажем так, варварская
эстетика, византийщина, ложь, постоянная манипуляция».

В развитие патриотической темы, неравнодушные патриоты сейчас активно постят в
социальных сетях сводную таблицу «43 агрессии России за сто лет». Российский
журналист, обозреватель газеты «Коммерсант» \textbf{Максим Юсин} с изумлением
комментирует очередной пример альтернативной украинской реальности: «Посмотрите
— там есть поистине гениальные вещи. Например, агрессия [России] против Сирии
в… 1967 году. Или против "государства" Дагестан — аж дважды, в 1994 и 1999.
Камбоджа, Бангладеш, Северная Осетия (!), Ливан в 1982 году (они нас точно не
перепутали с Израилем?). В общем, чтение для гурманов».

И чем хуже жизнь, тем большие дозы обезболивающего необходимы — может хотя бы
за «поребриком» жизнь хуже!? И почему там зарплаты выше, а тарифы меньше — мы
же за европейские пенсии стояли на майдане.

Зато мы прогрессивные, и привязанный к воздушным шарикам государственный прапор
запустили на Крым — пусть летит в сторону не существующего нового
Симферопольского аэропорта, бутафорской трассы «Таврида» и Крымского моста из
папье-маше…

У каждого своя реальность, просто у некоторых — галлюцинативная.

