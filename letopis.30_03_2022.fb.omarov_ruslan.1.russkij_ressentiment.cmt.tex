% vim: keymap=russian-jcukenwin
%%beginhead 
 
%%file 30_03_2022.fb.omarov_ruslan.1.russkij_ressentiment.cmt
%%parent 30_03_2022.fb.omarov_ruslan.1.russkij_ressentiment
 
%%url 
 
%%author_id 
%%date 
 
%%tags 
%%title 
 
%%endhead 
\zzSecCmt

\begin{itemize} % {

\iusr{Андрей Маргулев Иностранный Агент}

\href{https://activatica.org/content/d5f30299-9434-404b-b405-8d8dcfaae368/istoriya-psevdogosudarstva-rossijskogo}{%
История псевдогосударства российского, Андрей Маргулев, activatica.org, 30.03.2022%
}

\raggedcolumns
\begin{multicols}{2} % {
\setlength{\parindent}{0pt}

Я начну с актуального. Альфой и омегой российской государственности являлось и
является принципиальное непризнание любых ошибок на пути ее формообразующей
«исторической» или «божественной» миссии: торжества над другими народами.
Именно по этой причине пока существует режим, выбравший своей целью уничтожение
украинской государственности, любые «мирные процессы» с его участием будут
фикциями. А украинской государственности в российской традиционной парадигме
быть не может и не должно.

Летом 2014 года я написал работу «Рождение нации»
(\url{http://margulev.narod.ru/nationborn.html}), где в качестве исходного
пункта было взято консенсуальное признание большинством россиян некоего
«пункта»:

\ifcmt
  ig https://storage.googleapis.com/activatica/uploads/c6aface6-9623-4471-8578-d656eaffdded
\fi

«Пункт этот не нов, а имеет более чем давнюю историю, так что Путину не
пришлось в этой части ничего придумывать. Звучит он, примерно, так: Украина –
это исторически некая \enquote{окраинная} область Великой России, которая
достаточно искусственным путем получила статус государства. Поскольку это
\enquote{окраина} – русский язык там оказался \enquote{испорчен}, а понятие
\enquote{украинская нация} – националистический миф, выдуманный уязвленным
сознанием части \enquote{окраинной} элиты. И, раз нет \enquote{нации} –
государство является искусственным институтом, который можно безболезненно
\enquote{упразднить}».

Заканчивалась эта работа так:

«То, что происходило в ноябре 2013 – феврале 2014 на Майдане и вообще в Киеве,
внешне напоминало все, что угодно – массовые беспорядки, восстание, революцию –
но только не то, чем являлось в своем глубинном значении: рождением Нации.

В эти считанные месяцы через граждан Украины, непосредственно вовлеченных в
противостояние, проявились все те вызревшие в ее исторической судьбе качества,
которые не имеют ничего общего с качествами \enquote{старшего брата}, прошедшего
многовековую отрицательную византийско-татарскую селекцию. /.../

Самоценность свободы выбора, сохраненная в языковой культуре, истории, и,
главное – в генетической памяти украинцев со времен столь трагического и столь
героического Войска Запорожского, – вот что преобразило Украину в эти месяцы,
вот чего не могли учесть кремлевские политтехнологи.

Ну, а все, что произошло и еще произойдет далее – это просто закрепление \enquote{точки
невозврата} пробудившейся украинской нации к тому летаргическому состоянию, в
который она пребывала, ожидая своего исторического часа...»

Ныне Украина продемонстрировала свою полную состоятельность всему миру. Крах
теории «псевдогосударственности» Украины становится очевиден даже прозревающему
российскому населению. А вот с государственностью российской дело как раз
обстоит совсем иначе.

Собственно, ей я уже посвятил две работы: «Прощание с Химерой» (2020) и
«История с географией» (2021), – целями которых было выявление химерической
природы этой государственности. Химера – существо противоестественное и потому
обреченное на мучительное существование больного организма с неизбежной его
гибелью по катастрофическому сценарию. Противоестественность российской
государственности состоит, как обосновывалось в этих работах, в ее культурной
двухуровневости, когда традиционная «избранническая» культура базового уровня
постоянно конфликтует с модерновой, заимствованной у Запада культурой верхнего
уровня. И вот только что мы стали свидетелями некоего явления, неизбежного
ввиду неизбывности такого конфликта.

Дело в том, что одно из проявлений химеризма в области государственного
управления России состоит в том, что истинными целями этого управления являются
не декларируемые государственные интересы, а интересы самих управителей. В
«нормальных» государствах, формировавших модерновый слой российской культуры,
это явление называется «коррупцией» и подавляется. В России же это явление
становится камуфлируемым, но в реальности системным принципом управления. Путин
– это воплощение химерической российской государственности, стал вершиной всей
этой системы, которую вполне можно назвать «государственно-частным
партнерством» в той полностью извращенной в России форме, когда прибыль идет
частным лицам, а убытки – государству.

Тот грандиозный военный провал, который явила Россия в своей агрессии против
Украины – это «момент истины», который счастливый строитель этой системы как-то
не учел. Дело в том, что вся эта система управления была полностью заточена под
обеспечение финансовых потоков «наверх» – туда, где узкий круг лиц формировал
тайные фонды для тайных экспансий, а попутно, обеспечивал роскошным
существованием причастных к этому формированию лиц. Экспансия с помощью тайных
финансовых операций – подкупа элит и найма агентов – должна была по замыслу
творцов этой системы обеспечивать не только их благоденствие, но и
экономическую, в том числе и военную состоятельность государства.

Что же получилось в результате? Действительной целью любого, в том числе и
оборонного заказа – той целью, которая только и оценивалась вертикалью
заказчиков – стало ответвление «наверх» денежных потоков с каждого уровня.
Заказы перешли в разряд полувиртуальных сущностей, достаточных в форме тех или
иных моделей и «опытных образцов» и безудержной лживой пропаганды. И в тот
момент, когда химерическое псевдогосударство, опьяненное наркотиком собственной
пропаганды до степени потери связи с реальностью, решило обрушиться всей
накопленной, казалось бы, мощью на своего заведомо более слабого соседа, вся
эта полувиртуальная «мощь» рассыпалась в прах – ибо была им изначально.

\end{multicols} % }

\end{itemize} % }
