% vim: keymap=russian-jcukenwin
%%beginhead 
 
%%file 22_12_2021.fb.fb_group.story_kiev_ua.2.novogodnie_igrushki.cmt
%%parent 22_12_2021.fb.fb_group.story_kiev_ua.2.novogodnie_igrushki
 
%%url 
 
%%author_id 
%%date 
 
%%tags 
%%title 
 
%%endhead 
\zzSecCmt

\begin{itemize} % {
\iusr{Валентина Валентина}

\ifcmt
  ig https://scontent-frx5-2.xx.fbcdn.net/v/t39.1997-6/p370x247/47501240_208073730068277_8669225005453672448_n.png?_nc_cat=1&ccb=1-5&_nc_sid=0572db&_nc_ohc=z06OTD6vq34AX8O59t0&tn=lCYVFeHcTIAFcAzi&_nc_ht=scontent-frx5-2.xx&oh=00_AT_QDCZBHNNECLfftXhbr1JPAh4qoj_9U9Uf8gbSrjZRyg&oe=61CB2625
  @width 0.2
\fi

\iusr{Наталия Озерова}
Пахнет детством! Пирожками и мандаринами!

\iusr{Тамара Ар}

Действительно!, Новые игрушки это модерн, креатив и холодная красота Снежной
королевы! Старые игрушки это теплота, доброта, сказка, волшебство!


\iusr{Катя Бекренева}
Елочные бусы еще были)

\begin{itemize} % {
\iusr{Светлана Манилова}
\textbf{Катя}, у нас есть. @igg{fbicon.smile} 

\iusr{Катя Бекренева}
\textbf{Светлана Манилова} У меня дома, в родительской квартире остались, что со всем этим сделала невестка (жена брата), я не знаю. Я еле выцарапывала кое-какие вещи из своей квартиры, и это в 18 лет) Это отдельная грустная история.

\iusr{Светлана Манилова}
\textbf{Катя}, это бусы еще со времен детства моей свекрови. Ей уже 82. @igg{fbicon.smile} 

\iusr{Катя Бекренева}
\textbf{Светлана Манилова} У нас были ещё дореволюционные украшения.

\iusr{Ирине Вильчинская}
\textbf{Светлана Манилова} у меня уже не сохранились. Они крупные были,стеклянные, дутые. Я, в память о них, купила меленькие стеклянные и набрасываю на ветки...

\iusr{Ольга Писанко}
\textbf{Катя Бекренева} и у нас есть !
\end{itemize} % }

\iusr{Юлия Тертица}
На Подоле сейчас выставка старых игрушек

\begin{itemize} % {
\iusr{Irina Boiko}
\textbf{Юлия Тертица} где, очень интересно

\iusr{Юлия Тертица}
\textbf{Irina Boiko} в Центр української культури і мистецтва, вул. Хорива, 19В

\iusr{Irina Boiko}
Спасибо

\end{itemize} % }

\iusr{Наталья Шевченко}
І в мене є такі іграшки!

\iusr{Светлана Манилова}

Моя первая елочная игрушка- \enquote{колокольчик}, будучи уже старенькой и
облупившейся, всегда занимала почетное место на моей елке. Недавно она совсем
рассыпалась, и без нее стало немножко грустно. Когда ее выбрасывала (тяжело
писать это слово...), аккуратно завернув в салфетку, я будто прощалась со своим
старым другом... \enquote{Мне б туда, где ёлка в вате... Где едва за тридцать папе...
Мама шьёт на праздник платье... скоро Новый год... Где намеренья не лживы и
пока ещё... ВСЕ ЖИВЫ... И чисты души порывы... И она поёт...} @igg{fbicon.heart.red}

\begin{itemize} % {
\iusr{Ирине Вильчинская}
\textbf{Светлана Манилова} 

да, Каждый Новый год - это как встреча с прошлым. Перебираешь игрушки, мишуру,
вспоминаешь, КАК это было ТОГДА, когда вся семья еще была полной, с родителями,
бабушками... Как с мамой наряжали сосенку (большую, под потолок!), как
оборачивали в фольгу орехи с нашего ореха-гиганта, как развешивали конфеты и
мандаринки (самые вкусные потом потихоньку, после праздника, исчезали, а бумажки
еще висели @igg{fbicon.face.wink.tongue} ), как мама добавляла последний штрих - на кончики сосновых лап
цепляла кусочки ваты (СНЕГ!). И как, просыпаясь по утрам, уже после праздника,
увидев нарядную елочку, ты ощущал ПРАЗДНИК! А в последнее время ТОЙ жизни стали
покупать смереки. Теперь не покупаю деревца - только ветки. И игрушки,
соответственно, изменили свой \enquote{формат}. Исчезла огромная малиновая шишка,
которая была впору сосне, но уже совсем не по размеру изящным смерековым
веткам, исчезли большущие часы... Но шары все еще лежат в коробках и каждый раз
я их с любовью перебираю... Они ведь ОТТУДА...

\begin{itemize} % {
\iusr{Светлана Манилова}
\textbf{Ирине}, до слез! Очень созвучно! Спасибо!

\iusr{Ирине Вильчинская}
\textbf{Светлана Манилова} 

А еще я помню ТЕ гирлянды. Самодельные с лампочками двух цветов - зеленого и
малинового. Их сделал мой дядя, мамин брат. Он работал на \enquote{Коммунисте},
радиотехник. Зеленые были выкрашены зеленкой, а малиновые даже не знаю ,чем.
Каждая была \enquote{одета} в пластиковую \enquote{рубашечку} молочного
цвета.... Но гирлянда была старенькая и каждый год выяснялось, что она не
работает уже. И тогда начинался целый ритуал. Я бежала к дяде с просьбой
починить... Он, бурчал (предпраздничная суета, хлопоты - да и своя гирлянда
была в той же кондиции!) - доставал волшебный аппарат (я с благоговением
произносила - \enquote{тестер}!) и начинал \enquote{прозванивать} цепь.
\enquote{Виновница} аварии была вычислена и заменена, наша елка снова
загоралась яркими огоньками, подмигивавшими сквозь блестящую мишуру уютно и
тепло.

\iusr{Светлана Манилова}
\textbf{Ирине}, как все знакомо!

\iusr{Наталья Иванова}
\textbf{Ирине Вильчинская}
Как чудесно Вы написали про ТУ, ДОРОГУЮ СЕРДЦУ, ЖИЗНЬ!!! Все ещё живы и праздник настоящий!!!
 @igg{fbicon.face.blowing.kiss}  @igg{fbicon.evergreen.tree}  @igg{fbicon.snowman.without.snow} 

\iusr{Natasha Levitskaya}
\textbf{Ирине Вильчинская}
Спасибо, Ирина! Прекрасные воспоминания, трогательные!

\end{itemize} % }

\iusr{Наталья Иванова}
\textbf{Светлана Манилова}

Готова подписаться под каждым Вашим словом, особенно это стихотворение - до
слез. А с игрушками у меня то же самое происходит. Когда бьются старые, любимые
игрушки, то не могу выбросить, заворачиваю в мягкую бумагу и оставляю в
коробке. @igg{fbicon.heart.red}

\end{itemize} % }

\iusr{Петр Кузьменко}

\ifcmt
  ig https://scontent-frx5-1.xx.fbcdn.net/v/t39.30808-6/269600985_4851628501556062_4650071641437513348_n.jpg?_nc_cat=111&ccb=1-5&_nc_sid=dbeb18&_nc_ohc=FoYPRsGnvYQAX-yl8TB&_nc_ht=scontent-frx5-1.xx&oh=00_AT8WYDTz1j2FBQlMkJXOJhe1NWRDmYaW-AZxGm9VVtjmZw&oe=61CB3BFD
  @width 0.4
\fi

\iusr{Тома Храповицкая}

Как здорово! Всё вспоминается, таинство украшения именно сосны не ёлки, где
какие шары и сосульки, котики в домиках, мальчик с дудочкой и много других
игрушек ёлочных, сохранившихся из Детства... Они помнят, и каждый год ждут
своего выхода! Каждый год, кто-то из Них бьётся и ощущение как потерял Друга,
оттуда из Нашего Детства...


\iusr{Ольга Писанко}

У нас с братом была игра- мы друг другу загадывали отыскать где какая игрушка
висит...

\begin{itemize} % {
\iusr{Ольга Фролова}
\textbf{Ольга Писанко} Я сейчас так играю со своей внучкой))))))
\end{itemize} % }

\iusr{Катя Бекренева}

А мне вспомнилась елка с 81го на 82ой) Сыну было 8 месяцев, муж купил красивую
сосну, сказал не трогать, а то она плохо укреплена. Приходил он поздно,
вечерний КПИ заканчивал, а мне было невтерпёж, я попробовала вроде нормально,
устойчиво. Я нарядила. Среди ночи мы с мужем проснулись от грохота, спросонья
не можем сообразить в чем дело, потом дошло - елка грохнулась. Самое хорошее,
что ребенок не проснулся), а мы с мужем пылесосили палас среди ночи. На
следующий день сгоняла в Минский универмаг, игрушки покупала)


\iusr{Светлана Александренко}
Замечательные стихи и прекрасный, теплый пост!

\ifcmt
  ig https://scontent-frx5-2.xx.fbcdn.net/v/t39.1997-6/p370x247/47501240_208073730068277_8669225005453672448_n.png?_nc_cat=1&ccb=1-5&_nc_sid=0572db&_nc_ohc=z06OTD6vq34AX8O59t0&tn=lCYVFeHcTIAFcAzi&_nc_ht=scontent-frx5-2.xx&oh=00_AT_QDCZBHNNECLfftXhbr1JPAh4qoj_9U9Uf8gbSrjZRyg&oe=61CB2625
  @width 0.2
\fi

\iusr{Irina Boiko}
Обожаю старые игрушки

\iusr{Лидия Скрипченко}

\ifcmt
  ig https://i2.paste.pics/2470c3b4ac6da09b474c352bef97f5ee.png
  @width 0.2
\fi

\iusr{Катя Бекренева}

Самое интересное, что по игрушкам можно проследить историю) Были снежинки, были
красноармейчики, кукуруза, ракета, велосипедики из бус, колокольчики и они
звенели)


\iusr{Анна Шустерман}

У меня целый ящик старых игрушек, но уже года три я его даже не достаю, ставим
малюсенькую ёлочку и несколько красивых игрушек. А раньше всегда ставили ёлку
до потолка, именно ёлку, папа был эстет и сосна для него была просто деревом, и
запах сосны его тоже не устраивал, и я тоже так привыкла. Украшением всегда
руководил папа, он стоял на стремянке и говорил мне какого размера шарик
подать, потом спускался вниз, оглядывал ёлку со всех сторон и опять рассказывал
мне куда что вешать, чтобы было красиво, и даже когда мы уже вместе не жили, я
старалась украсить именно так, потому что придёт папа и обязательно оценит. А
сейчас уже некому оценить, скучаю... @igg{fbicon.snowflake}

\begin{itemize} % {
\iusr{Irena Visochan}
\textbf{Анна Шустерман} Как я тебя понимаю, дорогая! @igg{fbicon.face.downcast.sweat} 

\iusr{Валентина Валентина}
\textbf{Анна Шустерман} Скучай... Игрушки для детей береги!

\begin{itemize} % {
\iusr{Анна Шустерман}
\textbf{Валентина Валентина} , им не надо, у них свои.

\iusr{Валентина Валентина}
Никогда ничего не знаешь...

\iusr{Natasha Levitskaya}
\textbf{Анна Шустерман}
Вы правы, у них свои... А для нас, это память, ассоциации с счастливыми моментами.
\end{itemize} % }

\iusr{Olena Ivanenko}
\textbf{Анна Шустерман} 

Анечка, у меня тоже самое... не для кого.... 2 коробки старых игрушек, Дед
Мороз - почти мой ровестник, а последние пару лет в ходу несколько ярких
шариков ... так, для надо...

\end{itemize} % }

\iusr{Alla Kenya}
До сих пор берегу, которые не разбились

\iusr{Татьяна Сирота}

Как хорошо, как тепло Вы, Наташа, написали @igg{fbicon.hands.applause.yellow}{repeat=3} 
Мы с мужем уже забыли, когда покупали ёлочные игрушки... Ах, да!

В этом году он купил серебряный шар. Очень ему захотелось. А остальные игрушки
на нашей ёлке все старые. Самые \enquote{молодые} из них, куплены примерно 30-35 лет
назад. А остальным лет по 60 - 65. Можно сказать, наши ровесники...

\begin{itemize} % {
\iusr{Віта Ілейко}
\textbf{Татьяна Сирота} каждый год покупаю шар с символом года)

\iusr{Татьяна Сирота}
\textbf{Віта Ілейко} Внуки приедут к нам встречать Новый год. И мы 31
декабря (так у нас повелось) будем вместе с ними наряжать ёлку...

\ifcmt
  ig https://scontent-mxp1-1.xx.fbcdn.net/v/t39.30808-6/269751089_612093013433699_6468509187437250334_n.jpg?_nc_cat=109&ccb=1-5&_nc_sid=dbeb18&_nc_ohc=UhS66x3jygkAX-vLvdW&_nc_ht=scontent-mxp1-1.xx&oh=00_AT-Au8BkZJwd613aNBBUEB6FQKJcHxUIN7yJG-P2Jlj9jQ&oe=61CB2D74
  @width 0.4
\fi

\iusr{Віта Ілейко}
\textbf{Татьяна Сирота} замeчатeльно!


\end{itemize} % }

\iusr{Надежда Давиденко}

\ifcmt
  ig https://scontent-mxp1-1.xx.fbcdn.net/v/t39.30808-6/269796767_2243304232475927_7967713871646773551_n.jpg?_nc_cat=103&ccb=1-5&_nc_sid=dbeb18&_nc_ohc=TUCq-H58EvAAX_SlkX7&_nc_ht=scontent-mxp1-1.xx&oh=00_AT89nPCricYFuAQtEyWa-w7htkwbWD8fen9aqmDjpz7HsA&oe=61CA696E
  @width 0.3

	ig https://scontent-mxp1-1.xx.fbcdn.net/v/t39.30808-6/269776539_2243304405809243_6588784479479867673_n.jpg?_nc_cat=106&ccb=1-5&_nc_sid=dbeb18&_nc_ohc=TKh5gsZSy0gAX8bBJpf&_nc_ht=scontent-mxp1-1.xx&oh=00_AT9o3QGeA539kkhLahPR66Nklc_R4U0mfEc5YwXRw8kHfA&oe=61CAAA72
  @width 0.3
\fi

\iusr{Галина Гурьева}
Ооо, мой любимый гномик!

\iusr{Надежда Давиденко}

\ifcmt
  ig https://scontent-mxp1-1.xx.fbcdn.net/v/t39.30808-6/269779536_2243304509142566_5702475603494327266_n.jpg?_nc_cat=106&ccb=1-5&_nc_sid=dbeb18&_nc_ohc=GPBwXYH-yX0AX_EGi6W&_nc_ht=scontent-mxp1-1.xx&oh=00_AT8iCXZDWOF-OWeo1iMkHL2We3XOmxz1D45_RvMA6fOSQg&oe=61CB3AE1
  @width 0.3
\fi

\iusr{Ivanna Prykhodko}

\ifcmt
  ig https://scontent-mxp1-1.xx.fbcdn.net/v/t39.30808-6/269760477_4649813245104906_6418323236984885877_n.jpg?_nc_cat=103&ccb=1-5&_nc_sid=dbeb18&_nc_ohc=rV2Sw1bNkncAX-mTsJD&_nc_ht=scontent-mxp1-1.xx&oh=00_AT8eLtEn5BDRRNf9J68qP3hotGXLtbchgXiW8EMHACkRcA&oe=61CB0EA7
  @width 0.4
\fi

\begin{itemize} % {
\iusr{Анна Шустерман}
\textbf{Ivanna Prykhodko} , о, и у меня до сих пор есть шарик в ромбике, сосулька с заснеженным кончиком и шар с тремя вмятинками, а внутри вмятинок очень красиво.)

\iusr{Ivanna Prykhodko}
\textbf{Анна Шустерман} А белый зайчик, ещё до военный, из папье моше.

\iusr{Анна Шустерман}
\textbf{Ivanna Prykhodko} , у меня довоенных нет, самые старые 50-го года, когда родители поженились.

\iusr{Анна Шустерман}
\textbf{Ivanna Prykhodko} , ещё, при более тщательном рассмотрении, огурчик обнаружила, который у нас тоже сохранился.)
\end{itemize} % }

\iusr{Іллона Зейкан}
щемить...

\iusr{Наташа Хлизова}
У меня ещё немножко раритетов тоже сохранилось.

\iusr{Елена Сидоренко}

Скоро и я достану заветный чемоданчик с игрушками.. он путешествовал с нами по
всей необъятной стране, и хрупкие игрушки сохранились @igg{fbicon.heart.beating} 


\iusr{Владимир Дубровский}

Не только игрушки, а новогодние вечера в школах: помню в первом классе
массовик- затейник придумал: он называет животное женского рода, а мы в ответ
мужского- то есть:

\obeycr
Он.Корова!
Мы- бык
Он. Курица
Мы - петух
Коза.
Козел.
И наконец:
Он. Оса
Весь зал: Осел !
\restorecr


\iusr{Larisa Lappo}

Такие приятные воспоминания. Многое помнится, что-то еще сохранилось...
Спасибо!


\iusr{Elena Marijchuk}

В этом году в Тулузе ВСЕ украшения для елки ИЗ КИТАЯ и ТОЧНЫЕ реплики советских
украшений 50-х годав!!!! Аж мороз по коже...

\begin{itemize} % {
\iusr{Надежда Владимир Федько}
\textbf{Elena Marijchuk} В дитинстві, згадую, у нас було багато китайських іграшок...

\iusr{Valentina Syssoena}
\textbf{Elena Marijchuk} Я бы купила все .
\end{itemize} % }

\iusr{Надежда Владимир Федько}

Вдень дістав ящики з іграшками, щоб завтра вибрати, які будуть на ялинці.

(За давньою сімейною традицією ялинку ставлю 25 грудня).

Прочитав Вашу розповідь... Вибрав з ящика трошки...

\ifcmt
  ig https://scontent-mxp1-1.xx.fbcdn.net/v/t39.30808-6/269874097_4833176536741766_1189214163472860500_n.jpg?_nc_cat=102&ccb=1-5&_nc_sid=dbeb18&_nc_ohc=TFzS8nP5RhkAX_LKvFq&_nc_ht=scontent-mxp1-1.xx&oh=00_AT8iI5yK-XiwNxUo3X5LTWmq1SNUpCv0ZdRmQORhoEH6cQ&oe=61CAE452
  @width 0.4
\fi

\begin{itemize} % {
\iusr{Natasha Levitskaya}
\textbf{Надежда Владимир Федько} Чудесно!

\iusr{Анна Шустерман}
\textbf{Надежда Владимир Федько} , три фонарика и две шишки ещё живы.)

\iusr{Надежда Владимир Федько}
Як згадаю... скільки іграшок розбилося і були викинуті...

\iusr{Марія Голуб}
\textbf{Nadegda Volodymyr Fedko} чи є тут із найпершої вашої спільної із Надійкою ялинки?

\iusr{Надежда Владимир Федько}
\textbf{Марія Голуб} Так. Завтра викладу повну колекцію іграшок.
\end{itemize} % }

\iusr{Victoria Novikov}
 @igg{fbicon.hands.applause.yellow}{repeat=3} 

\iusr{Tanya Podkolzina}


\ifcmt
  tab_begin cols=3,no_fig,center

     pic https://scontent-mxp1-1.xx.fbcdn.net/v/t39.30808-6/269797716_1535673623465632_8263662377970563906_n.jpg?_nc_cat=111&ccb=1-5&_nc_sid=dbeb18&_nc_ohc=a13DAkl9_MEAX9uXMDh&_nc_ht=scontent-mxp1-1.xx&oh=00_AT8QoqxQQmiqAq1hHmrMFhcrDh1zqrwOPY4153yT18QaTw&oe=61CAB839

		 pic https://scontent-mxp1-1.xx.fbcdn.net/v/t39.30808-6/269761868_1535674033465591_2459310119187321370_n.jpg?_nc_cat=111&ccb=1-5&_nc_sid=dbeb18&_nc_ohc=GpzLUtLGgBwAX8Pn1kQ&_nc_ht=scontent-mxp1-1.xx&oh=00_AT9QiBuceVNE-DFl2yzWIpcbOJODvZ17iUzsRWsXG2bHXg&oe=61CA3F72

		 pic https://scontent-mxp1-1.xx.fbcdn.net/v/t39.30808-6/269691687_1535674760132185_8831186776661655253_n.jpg?_nc_cat=100&ccb=1-5&_nc_sid=dbeb18&_nc_ohc=L9v2bbHXtUQAX9UE49e&_nc_ht=scontent-mxp1-1.xx&oh=00_AT8yqRPvi3y1D3wZ72TYsJ6ZJ4df8F8_9SlmcrQiVlZSGQ&oe=61CA7CA2

  tab_end
\fi

\iusr{Alena Sidleckaya}
А у меня в этом году сосна с вот такими НАСТОЯЩИМИ игрушками из детства )

\iusr{Надежда Владимир Федько}
Із спогадів про дитинство...
Був спеціальний папір, який світився у темряві... З нього вирізали сніжинки і вішали на ялинку.

\iusr{Татьяна Сыч}
Да, помню все эти игрушки.

\iusr{Татьяна Оржеховская}
Прекрасные воспоминания детства

\iusr{Жанна Кулишова}
У меня есть игрушки, такие как на фото .. @igg{fbicon.heart.suit}

\iusr{Svetlana Loguinova}
У меня тоже некоторые из таких игрушек остались! Воспоминания о моем детстве. Становится так тепло и уютно при виде таких игрушек!

\iusr{Ирина Татарчук}
Дед мороз 1965 года выпуска, ещё моей мамы. Очень хорошо сохранился.

\ifcmt
  ig https://scontent-mxp1-1.xx.fbcdn.net/v/t39.30808-6/268954287_2672665353028999_9133721113323509904_n.jpg?_nc_cat=105&ccb=1-5&_nc_sid=dbeb18&_nc_ohc=p13TqpTf9ucAX974Sp9&_nc_ht=scontent-mxp1-1.xx&oh=00_AT_QmP83bnVv_k-9uHAPE0qjbCbhmLNAywNvdBrENkAI1A&oe=61CBA59C
  @width 0.3
\fi

\iusr{Ирина Татарчук}
Немного игрушек тоже сохранилось.

\ifcmt
  ig https://scontent-mxp1-1.xx.fbcdn.net/v/t39.30808-6/269791612_2672665823028952_2549595110143402182_n.jpg?_nc_cat=104&ccb=1-5&_nc_sid=dbeb18&_nc_ohc=2ptb9dXB-HAAX8sDbZo&_nc_ht=scontent-mxp1-1.xx&oh=00_AT-We8VQNPiBFljwWAOr_fZiqq6bqkGGUyUnZhLHIVw9ww&oe=61CBB553
  @width 0.3
\fi

\iusr{Александра Цымбал}

Полюбовалась я на Ваши игрушечки и придумала новую традицию на старый лад:
отправлять родственникам и друзьям посылочки с открытками и ёлочной игрушкой)

Спасибо!

\begin{itemize} % {
\iusr{Natasha Levitskaya}
\textbf{Александра Цымбал}
Чудесная традиция!

\iusr{Александра Цымбал}
\textbf{Natasha Levitskaya} благодарю)
\end{itemize} % }

\iusr{Татьяна Шиверская}
Надо и свои достать

\iusr{Валентина Козачук}
у меня еще пару игрушек старых осталось

\ifcmt
  ig https://scontent-mxp1-1.xx.fbcdn.net/v/t39.30808-6/269807837_3240509809504472_7792522920331941001_n.jpg?_nc_cat=106&ccb=1-5&_nc_sid=dbeb18&_nc_ohc=8AyxF8SyruEAX9JAyKX&_nc_ht=scontent-mxp1-1.xx&oh=00_AT9EjxyHl8UJcPNUs59eaTJk8_8eVeL96_VdHzKn4-sJ_w&oe=61CA6739
  @width 0.4
\fi

\iusr{Анна Шустерман}
\textbf{Валентина Козачук} , у меня, один в один, такая же.)

\iusr{Ирина Архипович}
Волшебно!! Обожаю те игрушки!!  @igg{fbicon.hand.ok}  @igg{fbicon.thumb.up.yellow}  @igg{fbicon.heart.eyes} 

\iusr{Валентина Козачук}
любимый зайка

\ifcmt
  ig https://scontent-mxp1-1.xx.fbcdn.net/v/t39.30808-6/269824819_3240510532837733_4024591852173035454_n.jpg?_nc_cat=106&ccb=1-5&_nc_sid=dbeb18&_nc_ohc=Tsi1yDI1CaUAX_3qsj0&_nc_ht=scontent-mxp1-1.xx&oh=00_AT-1_0icRmbP5BgXffEAYX_ION0R5i0AtHwGEsiQuxouFA&oe=61CC0152
  @width 0.4
\fi

\iusr{Ирина Татарчук}
\textbf{Валентина Козачук} У нас такой был, так мне нравился, к сожалению разбился (((

\iusr{Наталья Джиганюк}

Спасибо большое Вам. Да игрушки моё детство. Огромная ёлка, от пола до потолка,
и игрушки, конфеты, мамины пироги с вишнями.

\begin{itemize} % {
\iusr{Natasha Levitskaya}
\textbf{Наталья Джиганюк}
И Вам спасибо! Всё-таки, это очень теплые воспоминания!

\iusr{Наталья Джиганюк}
\textbf{Natasha Levitskaya} да, спасибо.
\end{itemize} % }

\iusr{Светлана Дубински}

\ifcmt
  ig https://i2.paste.pics/e2572d6eb6d7acfe7f04ccf5e282cd9d.png
  @width 0.2
\fi

\iusr{Татьяна Иванова}

Сейчас появились мастера, которые делают игрушки из бумаги и ваты, как в моем
детстве. Очень красивые, но они ручной работы и дорогие. Я все-таки не
удержалась и купила одну!

\iusr{Пелагея Шкраба}

красота

\iusr{Татьяна Иванова}

\ifcmt
  ig https://scontent-mxp1-1.xx.fbcdn.net/v/t39.30808-6/269798630_10221908427519865_4451312857687116288_n.jpg?_nc_cat=102&ccb=1-5&_nc_sid=dbeb18&_nc_ohc=rKpmkGxmn5gAX9dB_7X&_nc_ht=scontent-mxp1-1.xx&oh=00_AT_NQlKWoEva_GGlwUrRyJv37B4HCPzQ33SNn0Div05Cog&oe=61CAC5B8
  @width 0.3
\fi

\iusr{Маргарита Субач}
Наше не забывемое детство

\iusr{Юлия Тертица}
Наш Дед Мороз, по легенде с 1946 года

\ifcmt
  ig https://scontent-mxp1-1.xx.fbcdn.net/v/t39.30808-6/269833177_10216635235752334_2299762495787819107_n.jpg?_nc_cat=109&ccb=1-5&_nc_sid=dbeb18&_nc_ohc=k2oUAyyWLoUAX-bim6j&_nc_ht=scontent-mxp1-1.xx&oh=00_AT9aO9iWrusPhT93bb8Bp_a2Fh3ac70hzXmy2saolJ3dEw&oe=61CA3DD3
  @width 0.3
\fi

\iusr{Стремоухова Наталья}

Мне, маленькой девченке, было очень интересно рассматривать новогодние игрушки
на елке, потому что там были зайчики, медведи, волки, домики, сосульки и много
чего интересного. А сейчас висит шар, а рядом прикололи цветок, разве детям это
интересно?

\begin{itemize} % {
\iusr{Natasha Levitskaya}
\textbf{Стремоухова Наталья}
Другие времена!
\end{itemize} % }

\iusr{Константин Ткачев}
Есть некоторые

\iusr{Наталья Никифорова}
Іграшки
Із
Нашого
Дитинства

\iusr{Наталья Никифорова}
На
Ялинці
Птиці
Снігурі
Синиці

\iusr{Наталия Ковалева}
Как я завидую таким большим елкам!!!

\iusr{Лариса Безрукова}

\ifcmt
  ig https://i2.paste.pics/c50fc1d38f34cad93bb0b2fe432d67d5.png
  @width 0.2
\fi



\end{itemize} % }
