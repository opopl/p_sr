%%beginhead 
 
%%file 15_04_2021.fb.arximisto.1.pervaja_gimnazia_dar_spisok_vypusknic_1903_1916_godov
%%parent 15_04_2021
 
%%url https://www.facebook.com/arximisto/posts/pfbid0radEgaV6LeqXWLkv7EAfrTVfvaAjfEK9JrAsYuA6SyXH7foi4t6fgJbojd1Ci6VQl
 
%%author_id arximisto,fedorovskij_jaroslav.mariupol
%%date 15_04_2021
 
%%tags 
%%title Первая гимназия Мариуполя получила в дар уникальный список выпускниц 1903-1916 годов
 
%%endhead 

\subsection{Первая гимназия Мариуполя получила в дар уникальный список выпускниц 1903-1916 годов}
\label{sec:15_04_2021.fb.arximisto.1.pervaja_gimnazia_dar_spisok_vypusknic_1903_1916_godov}
 
\Purl{https://www.facebook.com/arximisto/posts/pfbid0radEgaV6LeqXWLkv7EAfrTVfvaAjfEK9JrAsYuA6SyXH7foi4t6fgJbojd1Ci6VQl}
\ifcmt
 author_begin
   author_id arximisto,fedorovskij_jaroslav.mariupol
 author_end
\fi

Первая гимназия Мариуполя получила в дар уникальный список выпускниц 1903-1916 годов

\#новости\_архи\_города

Мариинская женская гимназия Мариуполя, ныне гимназия № 1, получила в дар копию
архивного списка выпускниц 1903-1916 годов от Ярослава Федоровского, уроженца
Мариуполя.

Список уже доступен в музее и Интернете для всех, кто интересуется историей
гимназии и своих семей, по словам Анны Дежец, руководителя музея школы.

\enquote{Это честь для меня – вернуть из забвения сотни имен выпускниц Мариинской
гимназии, как заявил Ярослав Федоровский. Сейчас я живу в Киеве, работаю в ИТ
сфере, но я всегда интересовался своей родословной. 

Я был приятно удивлен, что моя дальняя родственница в 1916-17 гг. училась в
Мариинской женской гимназии, но не успела ее закончить из-за большевистской
революции. Тогда-то я и задался целью – отыскать данные в архивах обо всех
выпускницах гимназии. Это оказалось непростой задачей, но – кто ищет, тот
всегда найдет. В списке – 1137 имен мариупольчанок!}

\enquote{Подарок Ярослава – очень символичный, считает Анна Дежец. В этом году
наша школа будет отмечать свое 145-летие! Как и Александровская мужская
гимназия, она была открыта в 1875-76 году Феоктистом Хартахаем, легендарным
просветителем и историком приазовских греков.  Выпускницы Мариинки стали
учительской элитой как Мариуполя, так и всего Приазовья, а их внуки и правнуки
по-прежнему живут среди нас.  Уважительное отношение к истории – отличительная
черта нашей школы.  Еще в 1977 году в ее стенах был открыт музей. Мы всегда
рады любым посетителям – как выпускникам школы и их потомкам, так и всем, кто
интересуется историей Мариуполя}.

По мнению Андрея Марусова, директора ГО \enquote{Архи-Город}, \enquote{145-тая
годовщина школы делает чрезвычайно актуальным вопрос о предоставлении зданию
Мариинки статуса памятки архитектуры и истории. Здание Александровской мужской
гимназии получило его еще при Советском Союзе. 

Между тем, по неизвестным причинам в 2019 году киевские разработчики
историко-архитектурного опорного плана Мариуполя не включили Мариинку даже в
перечень потенциальных памяток...}

Могут ли горожане самостоятельно инициировать предоставление статуса памятки
Мариинке? Как это сделать и где найти средства?

Приглашаем послушать ответы инициативной группы на эти вопросы в передаче МТВ
\enquote{Ранок Маріуполя} в 9 утра 16 апреля, пятница...

\#перша145річниця

=======================

Цифровая копия списка выпускниц Мариинской женской гимназии Мариуполя в
1903-1916 годах доступна здесь\footnote{\url{https://cutt.ly/BvpjT28}}
\footnote{Internet Archive: \url{https://archive.org/details/mariinka_graduates_1903_1916_final}}
\footnote{В оригинале списков выпускниц за 1914 и 1916 гг. отсутствуют некоторые страницы}

Более детальную информацию по теме будут рады предоставить Ярослав Федоровский\footnote{098 208 7660, 
\url{https://www.facebook.com/YaroslavFedorovskyi}}, Анна Дежец (098
822 9455, \url{https://cutt.ly/0vpx5D2}) и Андрей Марусов (096 463 6988)

%\ii{15_04_2021.fb.arximisto.1.pervaja_gimnazia_dar_spisok_vypusknic_1903_1916_godov.cmt}
