% vim: keymap=russian-jcukenwin
%%beginhead 
 
%%file 06_10_2021.fb.ukrainskoju_movoju_bud_laska.1.razgovor_mova_kiev.cmt
%%parent 06_10_2021.fb.ukrainskoju_movoju_bud_laska.1.razgovor_mova_kiev
 
%%url 
 
%%author_id 
%%date 
 
%%tags 
%%title 
 
%%endhead 
\subsubsection{Коментарі}

\begin{itemize} % {
\iusr{Євген Філінський}

Та московинська є легшою щодо кількості синонімів , а тому ледачі її більше
люблять і всіляко прудчаються проти української ..........................Нашим
грубшим словником лупити по лобі кожному воїну - захиснику

\iusr{Lilia Bezbah}

Я б сказала так: хто хоче ще 50 років російської агресії, хто хоче відправляти
дітей на кордон з расійськім сусідом, то продовжуйте насаджувати тут руський
мір. А хто не хоче, щоб Путя кожен рік "намагався захищати права
руськоговорящіх", той говорить сам і вчить дітей і онуків української.

\begin{itemize} % {
\iusr{Євген Філінський}
\textbf{Lilia Bezbah} Сумніваюсь пані що на війну підуть оті зросійщені діти з викревленим поняттям братерства
\end{itemize} % }

\iusr{Богдана Самарик}
українська мова ще для тих, хто з дитинства знає, що таке московія. І знає це з розповідей рідних, і передаються вони з покоління в покоління.

\iusr{Марія Затайдух}
Знайома ситуація @igg{fbicon.face.downcast.sweat} 

\iusr{Диана Задорожна}

В мене в школі постійно розмовляли на російській мові а для мене це було не
дуже тому що наша солов'їна мова набагато краще та цікавіше говорити на
український.

Ще пам'ятаю ходила в українську школу так там був урок з руським язиком я тоді
поміняла школу там де його немає

\begin{itemize} % {
\iusr{Надія Синявська}
\textbf{Диана Задорожна} але власне ім'я - російською. ,,по справах, Господи.....''

\iusr{Диана Задорожна}
\textbf{Надія Синявська} в мене Українське ім'я
\end{itemize} % }

\iusr{Татьяна Оксютенко}
ПІДТРИМУЮ АВТОРА ПОВНІСТЮ!!! @igg{fbicon.thumb.up.yellow}{repeat=3} 

\begin{itemize} % {
\iusr{Українською Мовою Будь Ласка}
Дякую

\iusr{Надія Синявська}
\textbf{Татьяна Оксютенко} чому ж тоді не ,,Тетяна'', а русская Татьяна?
\end{itemize} % }

\iusr{Вадим Шмирьов}
Співчуваю... Лишається сподіватись, що то пройде... Мабуть у хлопця з'явилась російськомовна пасія...

\begin{itemize} % {
\iusr{Українською Мовою Будь Ласка}
Хто його зна... Філософія «какая разніца» навряд чи посприяє цьому.
\end{itemize} % }

\iusr{Наталія Щира}
Відчуваю Ваші емоції...

\iusr{Олена Матвійчук}

Так набридли розчарування, так втомилась боротись кругом і зі всіма. Особливо
вдома, коли треба просити дітей виправляти онучку. Та найбільший ворог російський
ютюб. Сама подарувала планшет, що поробиш Мабуть, ця боротьба вічна.

\begin{itemize} % {
\iusr{Українською Мовою Будь Ласка}

Щоб не розчаровуватись, не потрібно очаровуватись) Ніщо не вічне. У нас умови
набагато кращі ніж у наших батьків.

«Лупаймо сю скалу, нехай ні жар, ні холод не спине вас»


\iusr{Ірина Фаріон}
\textbf{Максим Лободзінський} дякую

\iusr{Українською Мовою Будь Ласка}
\textbf{Ірина Фаріон} це Вам дякую, бо саме Ви свого часу відкрили мені очі і показали «що» робиться у нашому домі.
\end{itemize} % }

\iusr{Oksana Klokova}

Діти кажуть, що краще вчити англійську, польську чи іншу іноземну, бо жити в
Україні не збираються... Не раз таке чула. Сумно.

\begin{itemize} % {
\iusr{Bogdanna Kozel}
\textbf{Oksana Klokova} то покручі,яких виховала мати.
Вони думають що мова ,А не праця- тяжка,на межі зусиль - дає багатство.
Коли приїдуть в Англію то будуть безликі ,бо свого не знатимуть ,та і чужого не вчитимуть ,якщо те чуже не приноситиме грошей.
Бо праця,це гроші ,А мова ,то душа!

\iusr{Ірина Фаріон}
\textbf{Oksana Klokova} то скажіть їм, аби вже звідси пиляли і все заберіть у них.
\end{itemize} % }

\iusr{Ярослав Дубень}

Ви всі так чудово все пишете, а чому ж на виборах ви не обрали українця
призидентом?

\begin{itemize} % {
\iusr{Українською Мовою Будь Ласка}
\textbf{Ярослав Дубень} ну, моя совість у цьому питанні чиста.
\end{itemize} % }

\iusr{Ольга Кожевникова}
Живуча бацила меншовартості.

\iusr{Інна Суверток}
Дістала вже ця "какая\_разніца"

\iusr{Надежда Товста}
Мова має значення!

\begin{itemize} % {
\iusr{Natalia Burkat}
\textbf{Надежда Товста} А чому "Надежда ", а не Надія?
\end{itemize} % }

\iusr{Тетяна Медвідь-Семеняка}
Так це наше сьогодення, але революцію робить свідома меншість. Тримаймося.

\iusr{Мария Шевченко Шостачук}
А давайте розповсюдимо пост ,всі хто підтримує Максима,гадаю він не проти....

\ifcmt
  ig https://scontent-mia3-1.xx.fbcdn.net/v/t39.30808-6/244295650_991662038342667_2722083650502433230_n.jpg?_nc_cat=106&_nc_rgb565=1&ccb=1-5&_nc_sid=dbeb18&_nc_ohc=2fEALxtMQW4AX-GlF3P&_nc_ht=scontent-mia3-1.xx&oh=b98ff8107ad2c1321e18fcd3f185f838&oe=6162FDE8
  @width 0.4
\fi

\iusr{Віктор Католик}
Добре сказано.

\iusr{Українською Мовою Будь Ласка}

\href{https://www.radiosvoboda.org/a/dity-i-ukrayinomovne-seredovyshche/31480901.html}{%
Не «русифікувати»: як у Києві знайти для дитини повністю україномовне середовище?, radiosvoboda.org, 28.09.2021%
}

\begin{itemize} % {
\iusr{Kate Rina}
\textbf{Максим Лободзінський} 

Вам повага за вашу позицію і подяка за цей пост. Якби ж ця молодь читала такі
пости.. Вони ж збиваються в зграї, як ті ворони і каркають чужою мовою. Огидно..

\iusr{Українською Мовою Будь Ласка}
\textbf{Kate Rina} дякую за щирі слова.

\iusr{Ярко Гулій}

Дуже болюче і актуальне питання. Мені особисто було, в совкові часи, нелегко
перебороти цей узькомірський прес. А тепер, я бачу, що все залишилось майже, як
було. Маю в родині аналогічну проблему і дуже через це переживаю. Нам головне
не здаватись і робити наполегливо і розмірено свою справу. Тримаймося! За нами
правда!

Все буде Україна!

\end{itemize} % }

\iusr{Диана Задорожна}

Якщо більше кількість людей говорила на державній мові то і другі брали приклад
а так немає майже в кого брати приклад.

Ось наприклад у вас взяла приклад розмовляти на державній мові

\begin{itemize} % {
\iusr{Українською Мовою Будь Ласка}
\textbf{Диана Задорожна} дякую тобі! Дуже тішусь з твого вибору. Ти молодець!

\iusr{Диана Задорожна}
\textbf{Максим Лободзінський} спасибі

\iusr{Kate Rina}
\textbf{Диана Задорожна} ви молодець. Але мова починається з імені. Ви Діана, чи Диана??

\iusr{Диана Задорожна}
\textbf{Kate Rina} по документах Діана

\iusr{Українською Мовою Будь Ласка}
\textbf{Диана Задорожна} це пані говорить про те, як підписана у фейсбуці. Підписана російською, а не українською.

\iusr{Kate Rina}
\textbf{Диана Задорожна} то чому називаєте себе Диана???
\end{itemize} % }

\iusr{Andriy Izvitskyi}

Як я це розумію... Теж маю купу знайомих, які все життя прожили в селі, а
кілька років в Києві - і вже кацап. Це слабкі душею люди, їм, по суті, всеодно,
якою мовою говорити і хто над ними цар. Так, саме цар чи пан, бо вільно вони
жити не вміють, треба дороговказ і нагайку. І таких ще дууже багато в Україні,
навіть, я б сказав, більшість. Тому еволюція буде довгою...

\begin{itemize} % {
\iusr{Kate Rina}
\textbf{Andriy Izvitskyi} якщо держава не візьме свою нагайку.. і не зажене це узькоязике стадо в україномовний хлів. Без мови, без роботи. Без мови- без громадянства. Тільки так!

\iusr{Andriy Izvitskyi}
\textbf{Kate Rina} поки що інший сценарій. Без прививки - без роботи. І говори хоч китайською.

\iusr{Валентина Дмитракович}
\textbf{Andriy Izvitskyi} Прививки добровільні.
\end{itemize} % }

\iusr{Олег Іщенко}
А мені в кайф розмовляти українською з москворотими.

\iusr{Рая Протазанова}
До сліз... дуже сподіваюсь на критичну більшість "білих ворон"!

\begin{itemize} % {
\iusr{Українською Мовою Будь Ласка}
\textbf{Рая Протазанова} ми не повинні бути білими воронами у власній державі. Це катастрофа, коли титульна нація дискримінована.

\iusr{Рая Протазанова}
\textbf{Максим Лободзінський} ви звернули увагу на те, що білі ворони взяті в лапки; я та сама "біла ворона" в своєму близькому оточенні...сподіваюсь, що в недалекому майбутньому, білими воронами будуть ті, хто не шанує рідної мови!
\end{itemize} % }

\iusr{Олена Пилипчук}

Повністю підтримую слова автора @igg{fbicon.hands.shake} . На жаль, навколишнє середовище, оточуючі
дуже впливають на світогляд, знаю такі приклади зі свого оточення. Тут важливо
мати принципову позицію і ніяке оточення не буде мати впливу. Україна та мова
нероздільні, вони в серці свідомих українців, а для "каких разніц" треба
жорсткий вплив держави для того, щоб вони згадали, що українці!


\iusr{Лідія Михайлівна}

Не просто ,,кози скачуть", а мама казала ,,усі кози скачуть"". Таке вдале
прислів'я, що доступне для розуміння і дитині. Диву даєшся, що ГІДНІСТЬ має,
скоріше, ота коза, ніж людина. Дітям меншим пояснити легше цю мовну тему. А з
підлітками -- підходити з різними аргументами неодноразово. Бо не тільки мову,
а й все інше чому вчили з дитинства переосмислюють по своєму, часто засмучені,
роздратовані повчаннями дорослих. Спостерігаю, що такі розмови делікатно
підводити на прикладі когось збоку. А ще, коли ,,піде" на розмову про життя,
про виховання. Бачу, що з дівчатками простіше, не впираються так, як хлопці.
Дай, Боже, сили і мудрості і старшим, і молодим, і дітям та розуміння захищати
своє УКРАЇНСЬКЕ !

\iusr{Галина Цема}
Я, нажаль, за своє життя так і не навчилась плавати. А зараз я неначе пли ву в
морі під теплими промінчиками сонця. Ме ні надає наснаги, в мене виросли крила,
я більш щаслива від розуміння, що можна не соромлячись говорити рідною мовою.
Всіляко її досконалювати і відчувати себе дійсно українкою !

\iusr{Людмила Щербакова}
100\% - во підтримую автора. Дійсно українська мова тільки для сильних духом. Отож, єднаймося!

\iusr{Valentina Kubik}
Пиздити російськомовних - наруку х уйлу. Тоді ігноруватиінах уйпосилати.

\iusr{Олег Щербан}

Просто на українськомовних чиниться тиск. Важко бути успішним без використання
російської і на даль мало хто про цю проблему говорить. Українська дуже часто в
кращому випадку на показ.

\begin{itemize} % {
\iusr{Українською Мовою Будь Ласка}
\textbf{Олег Щербан} слушно. Погоджуюсь
\end{itemize} % }

\iusr{Олег Щербан}

Я взагалі часто дивлюсь інстаграми футболістів з заходу України. І дуже сумно,
що більшість їх навіть ведуть російською. Бути українськомовним це як бути в
гетто.Тр й же Ярмоленко як соромився розмовляти українською? В Динамо все па
руски. Дітям нав'язують мислення українська мова для села і нікому не потрібна.
Згадайте ту ж історію Назарія Русина, який сказав, що змушений говорити на
ламаній російській..

\iusr{Леся Пристанська}

Нещодавно приїхали до нас в гості діти наших друзів з Києва. Як гарно вони
розмовляють українською, ну просто як соловейки! Як я тішилась! І тут почула,
що між собою .... москалять. Шкода, бо їм таки " какая разніца"!

\iusr{Шарупа Олександр}
+


\end{itemize} % }
