% vim: keymap=russian-jcukenwin
%%beginhead 
 
%%file 02_10_2021.fb.nazarenko_oleg.1.muzej_istorii_ukrainy_rusi
%%parent 02_10_2021
 
%%url https://www.facebook.com/permalink.php?story_fbid=1677925549066546&id=100005473502267
 
%%author_id nazarenko_oleg
%%date 
 
%%tags istoria,muzei,rusj.ukraina,ukraina,ukraina_rusj
%%title Я давно хочу построить Музей истории Украины- Руси!
 
%%endhead 
 
\subsection{Я давно хочу построить Музей истории Украины-Руси!}
\label{sec:02_10_2021.fb.nazarenko_oleg.1.muzej_istorii_ukrainy_rusi}
 
\Purl{https://www.facebook.com/permalink.php?story_fbid=1677925549066546&id=100005473502267}
\ifcmt
 author_begin
   author_id nazarenko_oleg
 author_end
\fi

\ifcmt
  pic https://scontent-frx5-1.xx.fbcdn.net/v/t1.6435-9/243801279_1677919059067195_3170519423358746143_n.jpg?_nc_cat=105&_nc_rgb565=1&ccb=1-5&_nc_sid=730e14&_nc_ohc=XZzml_swMtsAX8eieU-&_nc_ht=scontent-frx5-1.xx&oh=b7f3911a8ddd7d59c2a17baea631fcd5&oe=61821C20
  @width 0.8
\fi

Я давно хочу построить Музей истории Украины-Руси! Но пока часть моей
коллекции, которая касается Кавказа, бесплатно можно посмотреть в грузинском
ресторане «Шато Марани» в Хотяновке(Киевское море). Скоро там открывается
винный погреб, в котором помимо вина Вы можете увидеть не только бронзовые
изделия и оружие Коба́нской культуры (Koban culture) - археологической культуры
на Кавказе позднего бронзового и раннего железного веков(подобные предметы есть
в Эрмитаже в РФ и Сен-Жермен-АН-Ле в Франции), но и древнегреческие амфоры для
вина(помните, как в Колхиду греки за Руном ходили), карамультуки - древние
ружья и знаменитые кубачинские сабли и много всего! Скоро открытие погреба!
