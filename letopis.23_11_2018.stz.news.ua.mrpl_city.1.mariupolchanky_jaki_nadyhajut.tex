% vim: keymap=russian-jcukenwin
%%beginhead 
 
%%file 23_11_2018.stz.news.ua.mrpl_city.1.mariupolchanky_jaki_nadyhajut
%%parent 23_11_2018
 
%%url https://mrpl.city/blogs/view/mariupolchanki-yaki-nadihayut
 
%%author_id demidko_olga.mariupol,news.ua.mrpl_city
%%date 
 
%%tags 
%%title Маріупольчанки, які надихають...
 
%%endhead 
 
\subsection{Маріупольчанки, які надихають...}
\label{sec:23_11_2018.stz.news.ua.mrpl_city.1.mariupolchanky_jaki_nadyhajut}
 
\Purl{https://mrpl.city/blogs/view/mariupolchanki-yaki-nadihayut}
\ifcmt
 author_begin
   author_id demidko_olga.mariupol,news.ua.mrpl_city
 author_end
\fi

У листопаді все частіше відчувається слабкість, все більше хочеться відпочивати
та менше працювати. Сірі, холодні будні швидко набридають, не вистачає тепла,
чогось солодкого, гарного і яскравого. З огляду на це пропоную вам, дорогі
читачі, поринути разом зі мною у світ витонченої жіночої краси та безмежного
таланту, який, сподіваюся, трохи підніме всім настрій і зробить день більш
яскравим.

% Євгенія Іванівна Вєтрова
\ii{23_11_2018.stz.news.ua.mrpl_city.1.mariupolchanky_jaki_nadyhajut.pic.1_2}

Першою красунею, про яку піде мова, буде заслужена артистка УРСР, яка працювала
на початку 1960-х років, \textbf{Євгенія Іванівна Вєтрова}, темпераментна і
різнопланова актриса, яка в театрі була відома ще і як розумний педагог.
Працювала Євгенія Вєтрова з молоддю багато, наполегливо, вечорами, в будь-яку
вільну від вистав годину. Багато разів бачили вогні рамп актрису у таких
виставах, як \enquote{Іркутська історія}, \enquote{Барабанщиця}, \enquote{Сила
кохання}, \enquote{Шлях до щастя}, \enquote{Павутина}, \enquote{Куховарка},
\enquote{Четверо під одним дахом}, \enquote{Грози проходять}.

Всі її ролі злободенні, різнопланові, знаходили відгук у серці глядача,
хвилювали його своєю правдою та емоційною наповненістю.

\textbf{Читайте також:} \emph{Благодійна діяльність театральних працівників Маріуполя}%
\footnote{Благодійна діяльність театральних працівників Маріуполя, Ольга Демідко, mrpl.city, 13.11.2018, \par
\url{https://mrpl.city/blogs/view/blagodijna-diyalnist-teatralnih-pratsivnikiv-mariupolya}\par
Internet Archive: \url{https://archive.org/details/13_11_2018.olga_demidko.mrpl_city.blagodijna_dijalnist_teatr_pracivnyk_mrpl}
}

У 1964 р. до Маріуполя приїхала яскрава, творча індивідуальність з великим
театральним досвідом народна артистка УРСР \textbf{Роза Іванівна Под'якова}, яка
працювали на маріупольській сцені порівняно недовго – всього 5 років.

Розі Под'яковій завжди хотілося грати сучасниць, створювати яскраві образи,
сильні, вольові характери. Для неї кожна нова робота була накопиченням досвіду,
кристалізацією виграшних штрихів і інтонацій. \emph{\enquote{Це була дуже яскрава артистка,
неймовірно цікава. Напрочуд красива жінка, розумна, талановита}}, – згадує С.
Отченашенко. Кореспондент \enquote{Приазовского рабочего} С. Гольдберг підкреслив, що
особливо важливою для актриси стала роль Комісара у п'єсі В. Вишневського. Роза
виходила на сцену зібрана, одухотворена, жила цією роллю, цим образом,
зливаючись з ним.

% Роза Іванівна Под'якова
\ii{23_11_2018.stz.news.ua.mrpl_city.1.mariupolchanky_jaki_nadyhajut.pic.3_4}

\textbf{Читайте також:} \emph{Новый год в Мариуполе встретят в Театральном сквере - с шоу-программой и вкусной ярмаркой}%
\footnote{Новый год в Мариуполе встретят в Театральном сквере - с шоу-программой и вкусной ярмаркой, Яна Іванова, mrpl.city, 23.11.2018, \url{https://mrpl.city/news/view/novyj-god-v-mariupole-vstretyat-v-teatralnom-skveres-shou-programmoj-i-vkusnoj-yarmarkoj}}

Незабутній внесок у театральне мистецтво Маріуполя, безумовно, зробила \textbf{Наталя
Микитівна Юргенс}. Іноді здавалося, що акторському дару актриси підвладне все:
від іскрометного канкану в \enquote{Донні Люції} до трагічної скам'янілості в фіналі
\enquote{Васси Желєзнової}. Для неї не існувало рамок амплуа, не було ролей великих і
маленьких. У \enquote{Живи та пам'ятай} В. Распутіна вона зіграла майже безсловесну
Катерину з такою наповненістю і віддачею, ніби її роль і є головною.

% Наталя Микитівна Юргенс
\ii{23_11_2018.stz.news.ua.mrpl_city.1.mariupolchanky_jaki_nadyhajut.pic.5_6}

Життєлюбність і оптимізм, ідейна переконаність і пристрасне відстоювання своєї
позиції створювали в кожній виставі, де грала Н. Юргенс, особливе біополе, що
впливало не тільки на глядачів, а й на її партнерів. Працювати з нею було
нелегко, бо Наталя Микитівна вимагала від кожного, хто поруч, максимальної
самовіддачі, високого професіоналізму, безмежної відданості справі. Недоліки,
невдачі в роботі театру ставали її особистим болем. Зате і будь-якому успіху Н.
Юргенс щиро раділа, як власному. Вона з особливою дбайливістю і увагою
ставилася до роботи молодих акторів, і як член ради наставників чимало сил і
часу віддавала своїм підопічним. Вона була переконана: наставництво – процес
взаємозбагачення. А Наталя Юргенс завжди була готова не тільки вчити, а й
вчитися.

\textbf{Читайте також:} \emph{Солдат Швейк и денщик Шельменко едут в Мариуполь}%
\footnote{Солдат Швейк и денщик Шельменко едут в Мариуполь, Олена Онєгіна, mrpl.city, 22.11.2018, \url{https://mrpl.city/news/view/soldat-shvejk-i-denshhik-shelmenko-edut-v-mariupol}}

Однією з найпопулярніших у Маріуполі актрис завжди залишалася і залишається
\textbf{Світлана Іванівна Отченашенко}. Створені нею образи ставали предметом
глядацького відгуку й захоплення. На відміну від Полтави, в Маріуполі молоду
Світлану Отченашенко відразу ж ввели в поточний репертуар театру. Вистава за
п'єсою Ю. Едліса \enquote{Крапля в морі} (1966 р.) стала першою роботою актриси.
Олівцем від руки в програмку вписано: \enquote{Віра-Отченашен\hyp{}ко}. Після ролей Валі
Анощенко (\enquote{Російські люди} К. Симонова), Ніси (\enquote{Дурочка} Лопе де Вега) і Тані
Свєтлової (\enquote{Гліб Космачов} М. Шатрова) в колективі закріпився статус С.
Отченашенко як однієї з провідних актрис театру. І як підтвердження тому –
звання заслуженої артистки України, отримане в 1972 р., у 27 років! З кожною
новою роллю зростала майстерність актриси. 

% Світлана Іванівна Отченашенко
\ii{23_11_2018.stz.news.ua.mrpl_city.1.mariupolchanky_jaki_nadyhajut.pic.7_8}

Особливо запам'яталася С. Отченашенко в ролі Віктоші з \enquote{Казок старого
Арбату} О. Арбузова – юна істота, яка змогла перевернути розмірене життя двох
літніх чоловіків, ролі яких виконували Борис Сабуров і Микола Земцов. На сцені
панували такі чисті стосунки, випромінювалося таке тепло, що відтавали та
очищалися серця глядачів.  Віктошу-Отченашенко вітав автор п'єси Олексій
Арбузов, її органічністю захоплювався Зіновій Гердт.

Для Світлани Отченашенко дуже важливою, творчо значущою була робота над
підготовкою програм і проведенням вечорів поезії Анни Ахматової та Марини
Цвєтаєвої. Такі вечори відбулися не тільки в Маріуполі, але й у Донецьку,
Мінську. \enquote{\emph{І мені дуже радісно}, – зазначила актриса, – \emph{що проведені поетичні
вечори знайшли відгук в серцях людей}}.

\textbf{Читайте також:} \emph{Мариупольский сквер, Сергей Буров, mrpl.city, 14.10.2018}%
\footnote{Мариупольский сквер, Сергей Буров, mrpl.city, 14.10.2018, \url{https://mrpl.city/blogs/view/mariupolskij-skver}\par%
Internet Archive: \url{https://archive.org/details/14_10_2018.sergij_burov.mrpl_city.mariupolskij_skver}
}

\ii{23_11_2018.stz.news.ua.mrpl_city.1.mariupolchanky_jaki_nadyhajut.pic.9}

З 1974 року на маріупольській сцені працювала \textbf{Людмила Олімпіївна Руснак}. За час
роботи вона зіграла десятки ролей. У кожному характері, від безпритульного до
королеви, Людмила Руснак шукала і знаходила внутрішню сутність, найтонші нюанси
людської душі, що і робило її героїнь правдивими та щирими, близькими та
зрозумілими. І як результат – увага режисерів, критики, вдячні оплески, любов і
захоплення публіки.

\textbf{Отже, протягом другої половини XX ст. на маріупольській сцені працювали
талановиті, унікальні, витончені та непересічні актриси, чия енергетика, краса
і самобутність надихали та змушували глядачів приходити до театру знову і
знову...}

\emph{Всі використані світлини зберігаються в Поточному архіві Донецького
академічного обласного драматичного театру (м. Ма\hyp{}ріуполь) та особистому архіві
народної артистки України С. І. Отченашенко.}

\clearpage
