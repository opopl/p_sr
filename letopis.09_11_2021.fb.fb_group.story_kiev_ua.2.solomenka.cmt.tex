% vim: keymap=russian-jcukenwin
%%beginhead 
 
%%file 09_11_2021.fb.fb_group.story_kiev_ua.2.solomenka.cmt
%%parent 09_11_2021.fb.fb_group.story_kiev_ua.2.solomenka
 
%%url 
 
%%author_id 
%%date 
 
%%tags 
%%title 
 
%%endhead 
\zzSecCmt

\begin{itemize} % {
\iusr{Вячеслав Настецкий}
в исполнении Геннадия Банникова:

\ifcmt
  ig https://i2.paste.pics/7e3daf5efd798ebbea117ac7d10ff634.png
  @width 0.3
\fi

\iusr{Виктория Рыженко}
\textbf{Вячеслав Настецкий} Как поделиться именно этим видео?

\iusr{Вячеслав Настецкий}

Караваевские бани (Георгий Майоренко, киевский бард и художник). Сэм- Семён
Рубчинский, руководитель театра авторской песни. Автор - Леонид Духовный
(киевский бард).


\ifcmt
  tab_begin cols=3,no_fig,center
		 pic https://i2.paste.pics/e6e67f3e6502dcfa175f573af075784a.png
		 pic https://i2.paste.pics/487d7a43c70e9071e1aba3ff3270e0f2.png
		 pic https://i2.paste.pics/87eebfb9f6bca6e246ccb6465c592d0a.png
  tab_end
\fi

\iusr{Татьяна Субина}
\textbf{Вячеслав Настецкий} Спасибо

\iusr{Игорь Моржецкий}
Прекрасный район! Прекрасная песня!
Спасибо, Жора!
Основательно познакомился с этим районом, правда, поздно. Но надолго - встретил там свою будущую жену. Теперь они со мной навсегда - и район и супруга  @igg{fbicon.smile} 

\iusr{Георгий Майоренко}
\textbf{Игорь Моржецкий} А что я? Песня \textbf{Leonid Dukhovny} благодарность ему!!!

\iusr{Лаванда Степная}
Замечательная песня..Прекрасный район. @igg{fbicon.collision}  @igg{fbicon.heart.sparkling} 

\iusr{Зиновий Народецкий}
!!!!!

\iusr{Оксана Бидная}

\ifcmt
  ig https://i2.paste.pics/12b260537dde4750bc4d38689ce01600.png
  @width 0.2
\fi

\iusr{Яна Каган}

Вы написали о людях, которых я помню с детства и Лёню Духовного и Сэма
Рубчинского. Мой дядя с ними очень дружил. Духовный, кажется в Америке, а Сэм, вроде
в Киеве?


\iusr{Вячеслав Настецкий}
\textbf{Яна Каган} да

\iusr{Наталия Кугаевская}
До 1981 работала в этой школе, красивая улица Волгоградская и школа замечательная была и дети супер

\begin{itemize} % {
\iusr{Вячеслав Настецкий}
\textbf{Наталия Кугаевская} Вы преподавали историю, Наталия Петровна?

\iusr{Наталия Кугаевская}
\textbf{Вячеслав Настецкий} да

\iusr{Вячеслав Настецкий}
\textbf{Наталия Кугаевская} прочтите, пожалуйста, моё сообщение в мессенджере
\end{itemize} % }

\iusr{Наталия Озерова}
Молодцы! Супер! А был Зализнычный!

\iusr{Любов Осадчая}
Дякую дуже!!! Народилась, виросла і живу на Соломенке!!!

\iusr{Ludmila Krywicka}
Ай, спасибо, это мой район, все это помню!!!!!

\iusr{Кретов Андрей}
Спасибо!

\iusr{Лидия Щепкина}

Классная песня Если бы отцифровать. @igg{fbicon.hands.applause.yellow}
@igg{fbicon.face.smiling.eyes.smiling}  @igg{fbicon.hearts.revolving} 

\iusr{Сергей Богославский}
\textbf{Лидия Щепкина} Так уже.  @igg{fbicon.laugh.rolling.floor}  Не с магнитофона же Вы ее слушаете.

\iusr{Надежда Левченко}
Наша жизнь в Киеве началась именно с Соломенского района, как и моя профессиональная деятельность участковым педиатром.

\iusr{Valentyna Boyko}

Вчилася до 1973 року в сусідній школі по Волгоградській, 187-й. Потім в іншу
школу перейшла, хоча жила все ще в тому районі.

Саме на 30 травня 2020 року планували зустріч університетського курсу, також
40-річчя випуску. Через локдаун перенесли на невизначений час, хоча підготовку
почали чи не за півроку до того. Тому що багато хто живе за кордоном, чи не в
Києві, а в травні ще не були відновлені авіаперельоти, та й з поїздами і
автобусами були проблеми...


\iusr{Светлана Митюненко}
Огромное спасибо Очень душевно

\iusr{Tetiana Lytvynova}
Спасибо!
N187, 1975-1985

\iusr{Татьяна Сирота}

@igg{fbicon.heart.red}{repeat=3}
Здорово!

\end{itemize} % }
