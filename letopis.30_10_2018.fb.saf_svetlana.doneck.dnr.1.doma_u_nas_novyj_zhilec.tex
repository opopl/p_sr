% vim: keymap=russian-jcukenwin
%%beginhead 
 
%%file 30_10_2018.fb.saf_svetlana.doneck.dnr.1.doma_u_nas_novyj_zhilec
%%parent 30_10_2018
 
%%url https://www.facebook.com/leksiya58/posts/2060515500925839
 
%%author_id saf_svetlana.doneck.dnr
%%date 
 
%%tags dnr,donbass,doneck,ptica,semja,zhizn
%%title А дома... Дома у нас новый жилец)
 
%%endhead 
 
\subsection{А дома... Дома у нас новый жилец)}
\label{sec:30_10_2018.fb.saf_svetlana.doneck.dnr.1.doma_u_nas_novyj_zhilec}
 
\Purl{https://www.facebook.com/leksiya58/posts/2060515500925839}
\ifcmt
 author_begin
   author_id saf_svetlana.doneck.dnr
 author_end
\fi

Малая скачала себе игру с загадками. Для детей. То и дело приходится гуглить
мне, ну не догадываюсь об ответах на некоторые!)

С вечера бабахало. Что, где - не знаю. С утра тоже неспокойно шумело, пару раз
громко и тряско. 

И так, поинтенсивнее, уже с неделю или больше. 

Но спроси кого через полчаса, так и забудут уже, привыкли: "Где? Когда? Да
вроде не слышал. Или нет, стой, было, да". Привыкли настолько, что детвора во
дворе разве что примолкнет на мгновение от споткнувшегося о дома эха, да и
дальше себе бегают, играя, и их родители, остановят разговоры, прислушиваясь,
ругнутся, может быть, про себя, да снова о привычном и мирном, вроде и не было
ничего.

\ii{30_10_2018.fb.saf_svetlana.doneck.dnr.1.doma_u_nas_novyj_zhilec.pic.1}

А осень, будто спохватившись на пороге, забыв что-то, вернулась чуть не бабьим
летом: тепло, солнечно, безветрено. Оно и правильно, каникулы же!

А дома... Дома у нас новый жилец) 

Было такое, что на день рождения малой не только подарками, но и деньгами на
них дарили. И так как-то выходило, что дареное, по мелочи - но как вода сквозь
пальцы у неё. А ведь была задумка! Но на неё уже не осталось. Не хватает. И
решили таки выйти, чтобы помочь владелице потратить остаток с толком, с
радостью, но на что-то нужное.

И зашли в зоомагазин. Потому что он был по пути, не более того. Кроликов
глянули, мышек/крысок всяких, щеночков погладили, на черепашках застряли... На
двух огромных в разных клетках попугаев посмотрели. Такой если клюнет, то мало
не покажется!

А потом...

Я увидела его сразу! Среди всех ярких, шумных, скачущих.

Никакой на их фоне.

И вроде он сам это понимает, и потому отстранённо, нахохлившись, и даже как-то
свысока на весь этот шум-гам-тарарам взирал. Нет, не обижаясь на соседей или на
себя, не с презрением к ним, нет, а - в своих мыслях где-то витал. 

И - всё! Меня заклинило. Позвала своих, посмотреть. 

И он стал рассматривать нас. 

Вот не думали даже! Ещё за улетевшим Яшей тоска не унялась, и о замене и слова
никем сказано не было.

А вот - раз! И прикипела. Я. 

Говорю: "Как хотите. Но он - наш". И малая, тут же, с восторгом, объявила, что
остаток подарочной суммы она тратит здесь! Ещё и на мелочёвку новому жильцу
хватило, и для Синтии малая подарочков для погрызть набрала.

Знакомьтесь, Арчи! Арчибальд, то есть)

Он побледнее, чем на фото. Просто светло-серый. Чуть с голубизной брюшко,
бледно-синие родинки на щёчках, бело-серые крылышки, и - бородка! Бородка - это
нечто!)

Важный такой. И - с характером.

Пока обживается, снимает стресс от переезда)  

Пришлось и Кики сфотать, очень уж требовала и её вниманием не обойти, трясла
пальчиками клетку)

А почему "тридцать белых коней на двух красных холмах разбегутся, сшибутся и
снова затихнут впотьмах" - это ЗУБЫ, я так и не поняла) Тогда уж тридцать два!
Или двадцать восемь)

\ii{30_10_2018.fb.saf_svetlana.doneck.dnr.1.doma_u_nas_novyj_zhilec.cmt}
