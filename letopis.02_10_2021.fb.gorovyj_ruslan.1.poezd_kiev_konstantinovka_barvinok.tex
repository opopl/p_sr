% vim: keymap=russian-jcukenwin
%%beginhead 
 
%%file 02_10_2021.fb.gorovyj_ruslan.1.poezd_kiev_konstantinovka_barvinok
%%parent 02_10_2021
 
%%url https://www.facebook.com/gorovyi.ruslan/posts/6432505906760340
 
%%author_id gorovyj_ruslan
%%date 
 
%%tags jazyk,kiev,konstantinovka,mova,poezd,ukraina
%%title Потяг Київ - Костянтинівка
 
%%endhead 
 
\subsection{Потяг Київ - Костянтинівка}
\label{sec:02_10_2021.fb.gorovyj_ruslan.1.poezd_kiev_konstantinovka_barvinok}
 
\Purl{https://www.facebook.com/gorovyi.ruslan/posts/6432505906760340}
\ifcmt
 author_begin
   author_id gorovyj_ruslan
 author_end
\fi

\obeycr
Потяг Київ - Костянтинівка.
- Прасипаємся, пасажири! - голосом  левітана розвиває тишу вагона провідниця, - слєдующая станция Барвєнкаває. Гатовімся.
- Яка станція? - перепитую.
- Барвєнкаває.
- Може Барвінкове?
- Да нєт, вроді Барвєнкаває.
- Та ні, - кажу. Точно Барвінкове. Журнал Барвінок в дитинстві виписували? Ото на честь села назвали.
- Шутітє?
- Чого б я жартував? І пісня є ще. «Несе Галя воду».
- Тоже пра сєло?
- Про Барвінок. А за ней Іванко як барвінок в’ється…пам’ятаєте.
- Запуталі ви мєня с самага утра. Может ета па рускі так, Барвєнкаває.
- Російською барвінок знаєте як буде? Могильник.
- А, господі.
- Кажу ж все в них через сраку.
- Ладно. Пайду я будіть пасажирав.
- Я думаю ми їх вже всіх побудили.
\restorecr

\ii{02_10_2021.fb.gorovyj_ruslan.1.poezd_kiev_konstantinovka_barvinok.cmt}
