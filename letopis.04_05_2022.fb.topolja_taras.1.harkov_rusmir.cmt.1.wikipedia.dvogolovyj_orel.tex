% vim: keymap=russian-jcukenwin
%%beginhead 
 
%%file 04_05_2022.fb.topolja_taras.1.harkov_rusmir.cmt.1.wikipedia.dvogolovyj_orel
%%parent 04_05_2022.fb.topolja_taras.1.harkov_rusmir.cmt
 
%%url 
 
%%author_id 
%%date 
 
%%tags 
%%title 
 
%%endhead 

\raggedcolumns
\begin{multicols}{2} %{
\setlength{\parindent}{0pt}

Двоголовий орел. Матеріал з Вікіпедії — вільної енциклопедії.

\textbf{Двоголо́вий оре́л} — фантастичний птах, гербова фігура в геральдиці. Зображається
у вигляді орла, який має дві голови. Використовується в декоративному мистецтві
та геральдиці. Двоголовий орел походить зі Стародавнього Близького Сходу.
Найдавніші знахідки зображення орла датуються III—II тисячоліттям до н. е.
Пізніше використання символу відродилося у високому середньовіччі, особливо він
використовувався сельджуками, артукідами, зангідами, огузами та монголами. З
XIII століття двоголовий орел стає все поширенішим у Європі, зокрема,
використовувався у Візантійській імперії, Священній Римській імперії, та
Московському князівстві (з 1547 року Московське царство). Нині двоголовий орел
є гербом Албанії, Сербії, Чорногорії та Росії.

\ifcmt
  ig https://upload.wikimedia.org/wikipedia/commons/thumb/2/28/Imperial_Coat_of_Arms_of_the_Empire_of_Austria.svg/486px-Imperial_Coat_of_Arms_of_the_Empire_of_Austria.svg.png
	@caption Двоголовий орел Австрійської імперії
\fi

\textbf{Двоголовий орел у давні часи}. Найдавніші знахідки археологів із
зображенням двоголового орла датуються III—II тисячоліттям до н. е.
Фантастичний птах зображувався на амулетах, печатках та знаряддях праці в
Бактрії-Маргіані. Унікальна знахідка срібної сокири із зображенням
антропоморфічного демона з двома орлиними головами нині зберігається в
Нью-Йорку в музеї мистецтва Метрополітен[1]. Також відомі знахідки бактрійських
амулетів із зображенням двоголового орла, який тримає у своїх кігтях зміїв.
Пізніше подібні зображення перейняли хетти[2]. В ассирійців зображення
антропоморфічного демона вперше з'являється на печатках царя Еріба-Адада I (бл.
1380—1354 до н. е.) знайдених у місті Ашшур[3]. У давні часи символ двоголового
орла використовувався хеттами та шумерами. Неподалік від турецького поселення
Богазкей[en], де знаходилася столиця Хеттського царства Хаттуса, було знайдене
зображення двоголового орла вибитого на скелі. Хетти зображували орла на
циліндричних печатках, стінах, глеках та на інших виробах та будівлях. Іноді
двоголового орла зображували зі здобиччю в кігтях (зазвичай із зайцем).
Найпізніша знахідка символу датується XIII ст. до нашої ери. Зображення такого
орла збереглося на «воротах Сфінкса» в Аладжа-Хююк. Пізніше його зображення
було знайдено при розкопках у місті Лагаш. У Мідії, на сході Хеттської держави,
було знайдено зображення двоголового орла датованого VI ст. до н. е. Зображення
двоголового орла траплялось і в стародавньому Єгипті на ассирійських
пам'ятниках датованими VII—VI ст. до н.е[4]. У 1906 році археологічною групою
на чолі з М. О. Макаренком[5] на територіях сіл Попівки та Басівки (Роменський
район, Сумська область, Україна) було знайдене велике скіфське городище VI—III
ст. до н. е. У Попівці було знайдено бронзовий амулет який нагадував зображення
двоголового птаха. Елліс Міннз у своїй книзі «Scythians and Greeks: a survey of
ancient history and archaeology on the north coast of the Euxine from the
Danube to the Caucasus» залишив зображення цих амулетів, хоча точне
місцезнаходження оригіналів невідоме[6][7]. У 70-х роках XX ст. на
Красномаяцькому могильнику (мис Красний Маяк, Сухумі, Абхазія) були знайдені
бронзові дзвоники, на яких зображували птаха, що нагадував двоголового орла[8].
Дослідники віднесли ці предмети до VII—V ст. до н. е. до сармато-аланської
культури. Подібні предмети також знаходили за межами комплексу та на інших
городищах Абхазії

\textbf{Двоголовий орел у міфології та християнських віруваннях}. 

Образ двоголового орла використовувався в релігійних віруваннях. Шумерологи
вважають, що двоголовий орел у шумерській міфології був символом бога грози та
злив Нінурти. На кінцівці його палиці розміщувалися дві голови орла, або лева.
Згідно з шумерськими віруваннями, завдяки палиці Нінурт перемагав ворогів і
приборкував непокірних. Шумери також вважали, що орел посланець богів, що
створював зв'язок поміж людьми та богами. Хетти також використовували цей
символ у релігії. Атрибутами бога грози Тешуба були подвійна сокира та
двоголовий орел на ній[11]. Пізніше подібну атрибутику запозичили греки на
Криті й зображували з палицею Зевса. У стародавній Греції бог сонця Геліос
подорожував небом на колісниці, яка була запряжена 4 конями. Зустрічаються
рідкісні зображення, де колісниця запряжена двома двоголовими орлами[12].

В II—I ст. до н. е. двоголовий орел у індоєвропейського племені хуритів
вважався охоронцем Світового дерева[4]. В індуїзмі двоголового орла називали
Гандаберунда («ганда» (гінді गण्ड) — сильний, «берунда» (гінді भेरुण्ड) —
двоголовий). У літописному тексті «Вішну-пурана» (IV—V ст. н. е.)
розповідається про перевтілення бога Вішну в птаха Гандаберунда, коли йому
потрібна була чародійна сила[13].

У тюркській міфології двоголовий орел мав ім'я Оксоко та складався з двох голів
птахів Турула та Конрула[en] (також інші назви Семрук і Меркрут) і вважався
священним птахом бога сонця Буркута[en] (в середні віки Тенгрі)[14]. Найдавніші
знахідки зображення Оксоки датуються 3800 роками до н.е[15]. У міфології тюрків
II—XIII ст. Оксока став найбільш оспіваним птахом — сином бога Тенгрі.
Найчастіше птах описувався на верхівці дерева Байтерек, що з'єднувало Небо із
Землею. Орел оглядав своїми головами все навколо і якщо наближалися вороги, чи
злі духи, він злітав із дерева і вступав з ними в бій[16]. На початку XI ст.
образ Оксоки (Семрука) почав зображуватися на гербі імперії Сельджуків[17]. У
1993 році в Казахстані карбувалася обігова мідно-нікелева монета номіналом у 10
тенге із зображенням священного птаха, як згадка про давні вірування тюркських
народів[18]. У давніх віруваннях юкагірів, евенків, хантийців, мансійців,
ненців та селькупів двоголового птаха називали Ексекюєм (в деяких народів
Уксусі). В оповідках місцевих мисливців збереглася легенда в якій
переповідається про фантастичного двоголового птаха, який коли пролітав над
землею закривав крилами небо. На жаль, незначні відомості про цього птаха
описані в небагатьох джерелах, зокрема в науково-фантастичному романі В.
Обручева «Земля Санникова»[19]. 

У XXVIII—XXII ст. до н. е. у стародавньому Єгипті на печатках зображували
корону фараонів урею, пізніше почали зображувати бога Сонця Ра та бога Світла
Гора (або у вигляді синкретичного божества Ра-Горахті) з соколиними (або
яструбиними) головами повернутими в різні боки. Ці боги підтримували крилате
сонце[en] та тримали в руках кадуцеї[20]. Подібні символи використовувалися
також у країнах Близького Сходу — Месопотамії та Анатолії. В імперії
Ахеменідів, а згодом у Персії, подібна символіка стала зображуватися в
зороастризмі. У центрі фаравахару (перс. فروهر) зображувався фраваші[en] —
подоба янгола-охоронця[21]. Починаючи з III ст. до н. е. крилатий сонячний диск
використовували на своїх печатках Юдейські царі[22]. Прообраз крилатого сонця
також використовували у своїй символіці алхіміки[23]. У 1911 році крилате сонце
Юдеї використовував у своїй книзі «Святе повчання» Чарльз Расселл[24].
Єгипетське крилате сонце в різних варіантах також почало широко
використовуватися в інших християнських віруваннях — масони[25], теософи,
розенкрейцери, церква Єднання[en], тощо[26]. У 1717 році в Лондоні виникла
Перша Велика Ложа створена масонами (пізніше виник союз із 4 лож під назвою
Об'єднана Велика Ложа Англії). У 1871 році на обкладинці своєї книги «Мораль та
догма давнього та прийнятого шотландського уставу» Альберт Пайк використовував
зображення масонського двоголового орла з короною та з мечем у лапах[27].
Символом для емблеми масонів послугувало єгипетське зображення крилатого сонця
із зображенням на нім Ра-Горахті запозичене ранніми християнами, яке також
зображувалося і в єгипетському стилі на обкладинці «Шотландського уставу»[en]
(Прийнятий у травні 1801 року в м. Чарлстон (Південна Кароліна, США)[28][29].
Над головами орла в сяйнистому трикутнику зображувалася цифра 33 — найвищий
ступінь (градус) масонства[de]. Пізніше в трикутнику зображувалося всевидяче
око, яке також було запозичене з єгипетських вірувань (Див. також Око
Гора)[30]. Символом Автономної Чернечої Держави Святої Гори Афон (грец.
Αυτόνομη Μοναστική Πολιτεία Αγίου Όρους) у Греції було обрано знамено
Візантійської імперії династії останніх Палеологів, на якому орел є і нині
офіційною емблемою монастирської землі й зображується також на спеціальних
дозволах в'їзду — діамонітіріонах (грец. Διαμονητήριο)[31]. Також двоголовий
орел у часи середньовіччя зображувався на православних храмах Костантинополя,
Албанії, Кіпру та Еллади[32]. 

\textbf{Середньовічна Азія}.

Двоголовий орел використовувався як емблема сельджуцьких[33] та тюркських
племен під час інтронізації Тогрула в 1058 році в Мосулі (Північний Ірак) та
проголошення його «султаном Сходу і Заходу». За часів правління атабека Мосула
та Халеба (Сирія) Імада ад-Діна Зенгі (1127—1146) карбувалися дирхами із
зображенням двоголового орла[34]. Згодом двоголовий орел, як символ,
використовувався Румськими султанами в XIII столітті та огузами[35]. До 1090-х
років турки-сельджуки завоювали більшість візантійських міст у Малій Азії та
вийшли до проток Дарданелли та Босфору. У 1077 році було засновано Конійський
султанат, який використовували прапор на якому був зображений сельджуцький орел
з луком на грудях[36]. На арабських монетах зангидів та артукідів у XII—XIII
ст. карбувалися монети із зображенням орла. Двоголовий орел був поширеним
символом у Персії як символ перемоги, також і в Золотій Орді під час розширення
своїх володінь. Відомі дирхами артукіда Насір ал-Діна Махмуда[en] (1219—1234)
із зображенням двоголового орла, які карбувалися у 1195—1210 роках. У 1200—1221
роках на аверсі зображувалася 6-кутна зірка[37]. У 90-х роках XIII ст.
зображення двоголового орла почали карбуватися на мідних пулах Улус Джучі на
монетних дворах Солхату та Кирк-Ору[38]. Збереглися велика кількість монет за
часів Узбек-хана (1313—1341) та Джанібека (1342—1357) із зображенням
фантастичного птаха[39]. Зображення двоголового орла також трапляється на
монетах і в бейликів в Анатолії та мамелюків у Єгипті, датовані кінцем XIII ст.
На території Каїрської цитаделі була знайдена посріблена латунна куля,
виготовлена в 70-х роках XIII ст. на замовлення мамелюцького еміра Бадр ал-дін
Байсарі († 1278) на якій зображено двоголового орла[40]. Нині зберігається в
Британському музеї в Лондоні[41][42]. Наприкінці XIII ст. за часів правління
Івана II Комніна (1280—1297) почали карбуватися монети із зображенням
двоголового орла в Трапезундській імперії. Пізніше карбувалися монети імперії
із зіркою Давида та із зображенням двоголового орла на реверсі. Цей тип монет
помітно наслідував дирхамам артукіда Насір ал-Діна Махмуда[43]. У середньовіччі
орел з'являється на штандартах турків-сельджуків, які, окрім як у бойових діях
використовували штандарт і в прикрашенні підставки під Коран[44]. 

\end{multicols} % }
