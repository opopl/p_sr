% vim: keymap=russian-jcukenwin
%%beginhead 
 
%%file 09_12_2018.fb.fb_group.story_kiev_ua.1.petr_nesterov
%%parent 09_12_2018
 
%%url https://www.facebook.com/groups/story.kiev.ua/posts/953272448202890
 
%%author_id fb_group.story_kiev_ua,liderman_feliks.kiev
%%date 
 
%%tags kiev,nesterov_petr
%%title ЛЮДИ ПРОСЛАВИВШИЕ КИЕВ. ПЁТР НЕСТЕРОВ
 
%%endhead 
 
\subsection{ЛЮДИ ПРОСЛАВИВШИЕ КИЕВ. ПЁТР НЕСТЕРОВ}
\label{sec:09_12_2018.fb.fb_group.story_kiev_ua.1.petr_nesterov}
 
\Purl{https://www.facebook.com/groups/story.kiev.ua/posts/953272448202890}
\ifcmt
 author_begin
   author_id fb_group.story_kiev_ua,liderman_feliks.kiev
 author_end
\fi

ЛЮДИ ПРОСЛАВИВШИЕ КИЕВ. ПЁТР НЕСТЕРОВ

Этот красавец мужчина знаменитый летчик Пётр Николаевич Нестеров, автор
знаменитой «петли Нестерова» исполнение которой состоялось в нашем родном Киеве
в 1913 году. 

Легендарный летчик родился в Нижнем Новгороде15 (27) февраля 1887 года в семье
офицера-воспитателя кадетского корпуса Николая Федоровича Нестерова. После
неожиданной смерти отца материальное положение семьи сильно ухудшилось. Его
мать, Маргарита Викторовна, не имея средств на оплату жилья,  переехала вместе
с четырьмя детьми во Вдовий дом. В 1897 году Нестеров поступил в Нижегородский
кадетский корпус, который закончил  в 1904 году. 

\ii{09_12_2018.fb.fb_group.story_kiev_ua.1.petr_nesterov.pic.1}

После окончания корпуса Петра Николаевича в числе шести лучших выпускников
направили в Михайловское артиллерийское училище. 

Увлечение авиацией началось у Петра Николаевича с 1910 года. В 1912 году
Нестеров сдал экзамены на звания пилота-авиатора и военного лётчика, и уже в
сентябре 1912 года 25-летний поручик Пётр Нестеров совершил первый
самостоятельный полёт, а в 1913 году — окончил курс авиационного отдела
Офицерской Воздухоплавательной школы в Гатчине. 

В мае 1913 года его назначили в авиационный отряд, формировавшийся в Киеве.
Вскоре Пётр Николаевич стал командиром отряда. Идея «мёртвой петли» зародилась
у Нестерова ещё до 1912 года, но в этом году, будучи учеником гатчинского
авиационного отряда, он уже теоретически доказал возможность её выполнения. 

Для доказательства своей идеи, согласно которой «в воздухе для самолёта всюду
опора», 27 августа 1913 года в Киеве над Сырецким полем  Нестеров впервые в
мире выполнил на самолёте «Ньюпор—4» с двигателем «Гном» в 70 лошадиных сил
замкнутую петлю в вертикальной плоскости. Этим манёвром Нестеров положил начало
высшему пилотажу.

Спустя шесть дней эту сложную авиационную фигуру повторил француз Адольф Пегу.
Именно это событие получило широкую огласку и в иностранной, и в российской
прессе. 11 мая этого же года Нестеров совершил большой перелет из Киева в
Гатчину, покрыв 1.200 верст в 8 часов. Этим  своим  перелетом Нестеров побил
все русские рекорды на продолжительность полета, на скорость, на скорость с
пассажиром и так далее. 

Нестеров также разрабатывал вопросы взаимодействия авиации с наземными войсками
и ведения воздушного боя, освоил ночные полёты. В августе 1913 года он
возглавил групповой перелёт (в составе трёх машин) по маршруту Киев — Остёр —
Козелец — Нежин — Киев с посадками на полевых аэродромах. Во время перелёта
впервые в истории авиации проводилась маршрутная киносъёмка. Кроме того  в
первой половине 1914 года Пётр Николаевич осуществил два перелёта: Киев —
Одесса за 3 часа 10 минут и Киев — Гатчина за 9 часов 35 минут. Для того
времени это было большим достижением.

10 февраля 1914 года Киевское Общество воздухоплавания отметило Нестерова за
его заслуги, присудив ему золотую медаль Общества. Позже Киевское городское
руководство от лица города вручило отважному пилоту-новатору памятный золотой
жетон, с которым Пётр Николаевич никогда не расставался.

Осуществив за время Первой мировой войны войны 28 вылетов, 8 сентября (26
августа по старому стилю) 1914 года около городка Жолква Пётр Николаевич
Нестеров совершил свой последний подвиг — протаранил самолёт противника и погиб
в получившем серьезную аварию самолёте.

В 1958 году именем Петра Нестерова названа улица в Киеве на Шулявке вблизи
бывшего Сырецкого ипподрома, где он впервые выполнил петлю Нестерова.

В публикации использованы материалы сети

\ii{09_12_2018.fb.fb_group.story_kiev_ua.1.petr_nesterov.cmt}
