% vim: keymap=russian-jcukenwin
%%beginhead 
 
%%file 17_05_2021.fb.ukrainec_ostap.1.mova
%%parent 17_05_2021
 
%%url https://www.facebook.com/permalink.php?story_fbid=4142841109109685&id=100001515094776
 
%%author 
%%author_id 
%%author_url 
 
%%tags 
%%title 
 
%%endhead 

\subsection{Хочу повернутися до Михайла Дідуха, похованого в Русові неподалік Стефаника}
\Purl{https://www.facebook.com/permalink.php?story_fbid=4142841109109685&id=100001515094776}

Хочу повернутися до Михайла Дідуха, похованого в Русові неподалік Стефаника.

Його ім'я за тогочасними правилами писалось як Мїхайло Дїдух. В цих двох словах
криються десятиліття правописних срачів задовго до того, як ми почали робити це
в інтернеті.

Словом, через домінування москвофільства, зокрема в церковних колах, довший час
для запису використовувався етимологічний принцип, опертий на історичну
практику написання певного слова. З лекалами такої практики було поганенько -
церковнослов'янська і російська. В кращому випадку люди буквально писали не
тією мовою, якою розмовляли, в гіршому виходив лютий галицько-російський
суржик. "Язичіє", взорована на "старослов'янщину" літературна мова, насправді
не була монолітним правописом, а існувала десь на цьому спектрі. Вправність у
язичіє прямо залежала від освічености автора і навіть в альманахах одного
видавництва, де, по ідеї, були редактори, можна знайти тексти з дуже різним
балансом української/російської. 

Попри це, існували також сили умовного добра, їх втілювали фонетисти. Вони були
силами умовного добра просто через те, що спосіб "пишу як чую" не працює на сто
відсотків. Хоч фонетичний рух по суті був низовий, і перш ніж оформитись у щось
осмислене був просто письмом людей, котрі знали азбуку, але не граматику. Але
вже в часи Франка інституційний срач був такий масштабний, що пан Каменяр
навіть окремий есей написав про те, як його дістали ці правописні війни, в яких
ніхто крім нього не має рації (що характерно, десь так і було). Ох, якби він
знав про правопис-2019...

Так от. Коли фонетисти прийшли до влади, прийняли кулішівку як адекватний
підхід до української мови, а Желехівський на її основі створив фонетичний
правопис, годящий для галицького стандарту, стало видно одну цікаву штуку. Не
те щоб це було щось нове. Просто цей момент ніхто (крім, частково, Франка)
просто не брав до уваги в дискусіях про підходи. 

Жоден підхід насправді не працює. 

Етимологічний підхід не працює з самоочевидних причин: якщо ти звичайна людина,
котрій доводиться вивчати практично іншу мову просто для того, щоб навчитися
писати, звісно що про єдиний стандарт можна забути. Оскільки мова письма
настільки сильно відрізняється від того, що коїться в узусі, вона буде дуже
швидко змішуватися з дуже різними артефактами, аж поки стане ненормована. В
оптимальному випадку - розділиться на кілька окремих неформалізованих говірок
чи піджинів. Це сталося з латиною, це сталося з англійською, це сталося з
іспанською і це - до певної міри - сталося з російською. 

І от фонетичний підхід, покликаний працювати на природну пластичність мови, не
працює з тих само причин. Проблема дуже схожа на ту, що виникає з абсолютно
вільним ринком: все працює тільки якщо всі учасники діють абсолютно
раціонально. Оскільки всі учасники не діють абсолютно раціонально, оскільки в
кожного різний слух і внутрішнє поняття фонетики, письмо, засноване на чистому
фонетичному принципі, даватиме на виході ще більшу кашу, хоч і буде доступнішим
для вивчення (не для читання). Строго кажучи, ні елітарний етимологічний, ні
егалітарний фонетичний принципи насправді не можуть дати стабільне письмо.

Желехівський, як до нього Куліш, були надзвичайно освіченими філологами. Вони
розуміли мову значно краще за сільського каменяра, котрий перебивав ім'я з
метрики на хрест. Я вже писав, що Мїхайло вимовлялося не через "Ми" і не через
"Мі". У багатьох галицьких діалектах, особливо вздовж Дністра, там вимовляється
звук давніший ['е], котрий пом'якшує попередню приголосну. Правопис
Желехівського враховує цей фонетичний нюанс і, оскільки [і], що розвинулось із
етимологічного [е] та на місці ѣ ми пишемо через ї, давнішої фонеми це теж
стосується. Чому такі складнощі? Тому що з плином часу в українській мові е
перестав пом'якшувати попередню приголосну, тому багато де перейшов ув [і],
єдиний звук переднього ряду, який пом'якшення зберігав. Звук, позначений ятем,
українською теж наблизився до [і], натомість у російській дав е, бо там воно
йотоване. А над Дністром цей звук просто нікуди не перейшов і зберігся. І
правопис, створений у цих фонетичних реаліях, ці реалії відобразив. До речі,
сто років тому я написав би "місьці", зберігаючи праслов'янський єрь. І це
більше відповідає фонетичному принципу, оскільки там є не просто етимологічний
єрь, а й повноцінна м'якість. Але ні, ми всі пишемо це слово з твердою с, хоч
пам'ятаємо, що позиційно (як і етимологічно), там є м'якість.

Правопис Желехівського, як і будь-який інший правопис, проблему етимології не
вирішував, бо вона невирішувана. Етимологію доводиться враховувати, бо вона
прописана в мові. Зрештою, за різницею в мовах і діалектах ми й дізнаємось
етимологію. Каменяреві з Русова непотрібно знати етимологію голосних переднього
ряду в своєму діалекті, аби розуміти, чому в імені Мїхайло він пише не ту
літеру, яку читає. Ну і плюс ми працюємо в межах заданого алфавіту. Якщо не
рахувати спеціальної появи літери ґ після повторної появи цього звука, ми не
дуже пристосували свою мову до потреб. У нас є звук [ў], але немає літери на
його позначення. Непорозуміння і обурення павзами і авдиторіями - саме через
це. У нас немає окремих літер на позначення звуків [d͡ʑ]і [d͡z], хоча звуки є.
При цьому є диграф щ, хоча в українській мові він переважно читається як [ʃt͡ʃ].
В російській, скажімо, це один довгий звук: [ɕː]. Іноді логіки в тому, як певні
явища опиняються в мові, немає. Насправді "історично склалось" це дуже часто
спрощення для "не чіпайте, етимологічний підхід". 

Просто там, де етимологісти вимагали вивчати окремий правопис суто для письма,
фонетисти спростили написання до типових форм (внутрішні закони мови носіїв не
муляють, на те вони внутрішні закони), але вимушено залишили етимологію в
написанні запозичень. Що породжує срачі сьогодні, але на той час було цілком
логічно, бо ці запозичення потрапляли зазвичай через книжну мову, котра
народними версіями не цікавилась. Сьогодні це дає нам деякі дуже цікаві
розбіжності між діалектними назвами деяких предметів, особливо реманенту і
техніки.

Письмо - це не мова. Якщо ми вчимось писати - ми маємо до діла з етимологією,
обов'язково. Від нашого розуміння історії мови залежить лише розуміння
мотивації, що лежить в основі процесів (не всіх, обов'язкове збереження
подвоєнь у запозиченнях не піддається логічному розумінню). Оце та річ, яку
переважно називають "інтуїтивна грамотність" чи, не так точно, "мовне чуття".
Для володіння мовою ця річ абсолютно необов'язкова, вона просто корисна. Ну і
допомагає не так нервувати від змін.
