% vim: keymap=russian-jcukenwin
%%beginhead 
 
%%file 09_11_2020.fb.fb_group.ukrainska_mova_dlja_vsih.1.jazyk_moskovity_mova.cmt
%%parent 09_11_2020.fb.fb_group.ukrainska_mova_dlja_vsih.1.jazyk_moskovity_mova
 
%%url 
 
%%author_id 
%%date 
 
%%tags 
%%title 
 
%%endhead 
\subsubsection{Коментарі}

\begin{itemize} % {
\iusr{Володимир Савченко}
Постити — поширювати.
Справа в тім — річ у тім.

\iusr{Iryna Mykolayivna}
Це той горе-малювака, що і історію писав для Московії!

\begin{itemize} % {
\iusr{Вѧчеславъ Баско}
\textbf{Iryna Mykolayivna} Історію Московії писав американець?

\begin{itemize} % {
\iusr{Iryna Mykolayivna}
\textbf{В'ячеслав Баско} історію Московії і Руси писали німці на замовлення Петра!

\iusr{Максим Равлюк}
\textbf{В'ячеслав Баско} я Вам щиро співчуваю  @igg{fbicon.frown}  Темні люди, що поробиш...

\iusr{Вѧчеславъ Баско}
\textbf{Максим Равлюк} Вони напевно думають що розвиток російської мови відбувалося ось так:

\ifcmt
  ig https://scontent-lga3-2.xx.fbcdn.net/v/t1.6435-9/124855278_1742433319260938_3163407162000655011_n.jpg?_nc_cat=106&ccb=1-5&_nc_sid=dbeb18&_nc_ohc=Iusn-mhKgtEAX-M5o7T&_nc_ht=scontent-lga3-2.xx&oh=04b358d0bcc991f37452f8778d630e4a&oe=6182EDA6
  @width 0.4
\fi

\end{itemize} % }

\end{itemize} % }

\iusr{Borys Lebeda}

Якщо українці хочуть залишитися як нація, треба шанувати науку, а не видавати казки за науку.

Дивіться тут: 

\href{https://youtu.be/VWWyj1xpudo}{%
Звідки береться мова? Частина 1: генетика, нейрологія і рекурсія | Квантова філологія, %
Твоя Підпільна Гуманітарка, youtube, 20.09.2020%
}

Автора прошу не називати мій коментар позицією меншовартості. Українську мову
ми шануємо тому що вона наша, а не тому що вона найкрутіша в світі.

\begin{itemize} % {
\iusr{Evhen Sicheslav}
\textbf{Borys Lebeda} якби ж ви справді вважали, що вона "наша"! а не якийсь покруч з польської, татарської та московської. Вибачте, але спочатку Ви обстоюєте думку, що українська мова "переважно складається" з іншомовних запозичень, а потім кажете, що вона - "наша".

\begin{itemize} % {
\iusr{Borys Lebeda}
\textbf{Євген Салік} Саме так і є. Якось всі нації живуть зі своїми мовами, які здебільшого складаються з запозичень. Українцям також варто навчитися любити свою мову за те, що вона своя, а не тому що вона автентичніша за інші ...
\end{itemize} % }

\end{itemize} % }

\iusr{Andrii Dmukhailo}

Всім відомо, що створення московитського/російського язику мало на меті лише
одне - Зазіхнути на історію Русі.

\iusr{Надія Овійчук}

\ifcmt
  ig https://scontent-lga3-2.xx.fbcdn.net/v/t39.1997-6/s168x128/851575_582402885104024_929794036_n.png?_nc_cat=106&ccb=1-5&_nc_sid=ac3552&_nc_ohc=ZChyF-SRo0sAX_NcG8x&_nc_ht=scontent-lga3-2.xx&oh=4582221d9dec43bbdd847b414c05e4fa&oe=6163388D
  @width 0.2
\fi

\iusr{Serhii Nik}
Це вже не в перше )) чорти постійно шось подсовивают.
З початку думал ,ну то мені кажется , з граматікой у мене кепско , але 'дуб рослина хвойна' це зовсім ))

\begin{itemize} % {
\iusr{Валерій Швець}
\textbf{Serhii Nik} У цього чорта є прізвище Борис Лебеда. Правда він раніше був Лободою, є такий зловредний буря'н, але переїхавши в місто назвався "по городському", як казав герой з "За двома зайцями" - "по модньому". А я скажу: "на мокшанський манер"

\begin{itemize} % {
\iusr{Borys Lebeda}
\textbf{Валерій Швець} моє прізвище було таким принаймі 150 років, таке прізвище є і серед чехів.
\end{itemize} % }

\end{itemize} % }

\iusr{Василь Барвінок}
"в корні невірне" - господи, якою це мовою? Це як патріот "до мозга костєй"?

\begin{itemize} % {
\iusr{Evhen Sicheslav}
\textbf{Василь Барвінок} імєнна!  @igg{fbicon.face.grinning.squinting} 
\end{itemize} % }

\iusr{Валерій Швець}
А мені це дерево нагадує старий ясен геть побитий омелою. Так і треба
відноситись до цього покруча

\begin{itemize} % {
\iusr{Borys Lebeda}
\textbf{Валерій Швець} Це наукова праця. Щирі українці таке ненавидять.

\begin{itemize} % {
\iusr{Evhen Sicheslav}
\textbf{Borys Lebeda} 

насправді, це чергова АНТИНАУЧНА ретрансляція цілої низки міфів, чималою мірою
запущених в світ московитами. Вони на це планомірно працювали і працюють. Саме
тому зараз Ви навіть в закордонних публікаціях легко можете прочитати про
скіфське золото, як про культурне надбання Росії, про три братських народи, що
постали з одного кореня, та про українців, як про різновид росіян і цілу купу
инших.

І так, Ви цілком праві -УКРАЇНЦІ УЕ НЕНАВИДЯТЬ.

І чого б це, еге ж?

\iusr{Borys Lebeda}
\textbf{Євген Салік} Ваші твердження більш антинаукові чим навіть "нова хронолонія" від росіян. Чим більше Ваша маячня поширюються тим менше людей бажають бути українцями.

\iusr{Evhen Sicheslav}
\textbf{Borys Lebeda} в чому саме, але конкретно? в чому їх антинауковість?

\iusr{Borys Lebeda}
\textbf{Євген Салік} наука, це коли спочатку іде 1. припущення, потім 2. критерії, потім 3. дослідження, потім 4. висновки. У Вас спочатку йде висновок, потім Ви шукає дослідження, яке його б підтвердило.

\iusr{Валерій Швець}
\textbf{Borys Lebeda} 

Я з кремлівськими холуями не дискутую, але вам зроблю виняток:

1) Якщо українська мова споріднена з білоруською на 84\%, з польською на 70\%,
словацькою на 68\% а з російською тільки на 62\%, то чому в цьому "дереві" вони в
такому порядку не розташовані?

2) Щодо розгалуження. Місце розгалуження має відповідати ступеню спорідненості.
Тобто російська мова має мати ближче розгалуження з болгарською (у них
спорідненість 73\%), а потім вже з українською.

Тобто, якщо простими словами: Йде розгалуження української та російської, вище:
російської та болгарської, ще вище: української та словацької, а ще вище
української з польською, і найвище з білоруською. Надіюсь, я доступно вам
роз'яснив?

І ще деякі роздуми, це звичайно, якщо ви хочете знати правду:

\url{http://interklasa.pl/portal/dokumenty/r_mowa/strony_ukr02/mowa/00_mowa06.htm}

\iusr{Borys Lebeda}
\textbf{Валерій Швець} я дочитав тільки до слів де Ви мене називаєте "кремлівським холуєм".
Те що Ви пишете не відповідає дійсності та образливо. Наступного разу спробуйте чемніше ставиться до україномовних громадян.

\iusr{Валерій Швець}
\textbf{Borys Lebeda} Можливо я погарячкував з кремлівським холуєм, але тоді ви, вибачте, "корисний ідіот" якого використовують підсвідомо.

\iusr{Borys Lebeda}
\textbf{Валерій Швець} Чим я можу бути кориснішим кремлю ніж українець, що називає співгромадян ідіотами та холуями?
Ви не думали припинити хамити іншим і завдати цим шкоду ворогам?

\end{itemize} % }

\end{itemize} % }

\iusr{Marta Roma}
Ці московити перекручують історію, тому їм ніхто не вірить і потрібно їх відокремити від усього світу !!!

\iusr{Анна Винокурова}

Цю та інші гарні малюночки за формою з рашистською пропанандою за суттю, хтось
гарненько профінансував, розробив, і поширює.  \textbf{\#рашизм}

\begin{itemize} % {
\iusr{Borys Lebeda}
\textbf{Анна Винокурова} якщо всі науки називати рашизмом, то для України в цьому світі місця не буде. Краще поширювати антинаукову маячню серед росіян
\end{itemize} % }

\iusr{Вѧчеславъ Баско}

Ви чи вважаєте що російська і українська взагалі ніколи не мали спільного
предка? Хоча б на рівні праслов'янської (предка всіх слов'янських мов), або
більш давнього - прабалтослов'янської (предка слов'янських та балтійських
мов)... або може навіть на рівні праіндоєвропейської мови (предка
індоєвропейських мов) - теж не було нічого спільного? Не, напевно предка між
ними не було навіть на рівні ностратичної мови... Та не, російську мову
привезли на Землю інопланетяни, в той час як інші розвинулися природним шляхом.
На вашому все було так?:

\ifcmt
  ig https://scontent-lga3-2.xx.fbcdn.net/v/t1.6435-9/123968344_1742412739262996_4482623702446042538_n.jpg?_nc_cat=105&ccb=1-5&_nc_sid=dbeb18&_nc_ohc=aMhtJLMgCogAX8vIt2x&_nc_ht=scontent-lga3-2.xx&oh=6d66b25b1867c542d0c5cfb3155b2502&oe=61861527
  @width 0.4
\fi

\begin{itemize} % {
\iusr{Валерій Швець}
\textbf{В'ячеслав Баско} 

А яка нам справа до російської, нехай про неї болить голова у московитів. Чому
ви не говорите про мови, які нам ближчі, наприклад білоруська, польська,
словацька. Що до російської, то, якщо ви трохи цікавились історією, то мали б
знати, що українську мову придумали не поляки, чи австрійці, а вона існувала на
нашій території тисячі років. Фрески з написами у Софії Київській це доводять.
Не даром кажуть, коли київські князі ріднилися з монархами Європи, вели
торгівлю, воювали, розвивали просвітництво, на території Московії жили
напівдикі угро-фінські племена. Ці племена звичайно мали свої споріднені мови,
але, як ви вже догадуєтеся, до слов'янських мов вони мали далеку спорідненість.
Володимир вводить християнство, привозить з собою болгарських просвітителів,
які розробляють церковну мову для богослужінь, яка звичайно була на основі
староболгарської. Так, як українська мова вже була сформована, то наші предки
формально віднеслися до церковної мови, тобто, вона не дуже вплинула на
розмовну українську мову, до тогож вони не дуже відрізнялися, звичайно відносно
угро-фінських мов. А церковні місіонери, які навертали московські племена в
християнство, звичайно це робили на церковній, болгарській мові, яка була
зовсім не зрозуміла місцевим і їм залишалося, або її вивчити, або загинути.
Тому так і пішло, що розмовною мовою на Московщині стала болгарська, вона
змішалась з місцевими мовами, потім царі привнесли туди романо-германських
слів, і десь в 17 на початку 18 століть російська мова відносно завершила своє
формування.

\begin{itemize} % {
\iusr{Вѧчеславъ Баско}
\textbf{Валерій Швець} я говорю про різні мови, а не тільки про російську, я цікавлюся лінгвістикою.
> "які нам ближчі, наприклад білоруська, польська, словацька."
- В української мови процент загальної лексики:
\begin{itemize}
  \item ▪️ 84\% з білоруською
  \item ▪️ 70\% з польською
  \item ▪️ 68\% з словацькою
  \item ▪️ 68\% з сербською
  \item ▪️ 62\% з чеською
  \item ▪️ 62\% з російською
\end{itemize}
Різниця всього пару відсотків. Крім білоруської.
> "знати, що українську мову придумали не..."
- Звісно.
> "існувала на нашій території тисячі років."
- Українській мові не більш 1000 років.

\iusr{Borys Lebeda}
\textbf{В'ячеслав Баско} класифікацію мов роблять все одно не за лексикою, а за граматикою та синтаксисом.

\end{itemize} % }

\end{itemize} % }

\iusr{Варвара Шпигун}

Стандартна класифікація індоєвропейських мов була стилізована під дерево, у
якого розмір гілок пропорційний кількості носіїв мови. Що тут можна ще
побачити? Але ж ні. Аттіла, борщ та "Мрія" - багата у вас уява, взірець
наукового мислення.

\iusr{Evhen Sicheslav}

українська як і будь-яка (нештучна) мова світу проходила різні етапи, і 1000
років тому, звісно ж, звучала дещо інакше, так само як і та ж староанглійська,
наприклад.

Наш ворог дуже постарався, аби нам не залишилось майже жодних письмових згадок
(згадаймо, як вивезли весь архів Почаївської лаври - його так і не знайдено
досі, як знищувались усі книги та рукописи, написані "малоросійськім нарєчієм",
як "випадково" вщент вигоріла секція україністики у столичних бібліотечних
архівах, тоді безповоротньо були втрачені тисячі рукописів та стародруків,
згадаймо, що Пересопницьке Євангеліє свого часу уціліло ВИПАДКОВО, якби не цей
випадок ми б не МАЛИ ЖОДНОГО ПРИМІРНИКА). Той хто хоч трохи цікавився цими
речами - знає, наскільки зухвало та знущально звучать закиди на зразок - а
докажитє! покажитє нам пісьмєнниє істочнікі !

Але ще більше дратує хор "українських" підспівувачів, які дружно приєднуються
до подібних знущаннь, роблячи вигляд, що їм нічого невідомо про методичне,
планомірне й наполегливе знищення будь яких слідів нашої мовної культури
протягом щонайменше 3-х столітть...   @igg{fbicon.face.symbols.mouth} 

\iusr{Dmitro Wolkof}
moskal's'ka штучна мову

\begin{itemize} % {
\iusr{Вѧчеславъ Баско}
\textbf{Dmitro Wolkof} на підставі чого? Є роботи західних лінгвістів які доводять це, що російська мова штучна?

\begin{itemize} % {
\iusr{Dmitro Wolkof}
\textbf{В'ячеслав Баско} Так! Бажання б було їх знайти, а не захищати капійську.

\iusr{Вѧчеславъ Баско}
\textbf{Dmitro Wolkof} ну о, бажання є, то давайте, руки в ноги і вперед, шукайте ці наукові роботи що доказують це. А то "... - не мешки ворочать".
\end{itemize} % }

\end{itemize} % }

\iusr{Виктор Кравцов}
 @igg{fbicon.eyes}  @igg{fbicon.megaphone}  @igg{fbicon.thumb.up.yellow} 

\end{itemize} % }
