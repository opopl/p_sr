% vim: keymap=russian-jcukenwin
%%beginhead 
 
%%file 03_12_2021.fb.bystrjakov_vladimir.1.strana_inficirovana.cmt
%%parent 03_12_2021.fb.bystrjakov_vladimir.1.strana_inficirovana
 
%%url 
 
%%author_id 
%%date 
 
%%tags 
%%title 
 
%%endhead 
\subsubsection{Коментарі}

\begin{itemize} % {
\iusr{Андрей Матвиишин}
Ну почему же за границу не выпускали? Выпускали. Туристами. При очень жестком контроле.

\begin{itemize} % {
\iusr{Владимир Быстряков}
\textbf{Андрей Матвиишин} 

до 70-х выпускали только в соц. страны (ГДР, Чехословакия, Болгария, Польша....)
Никаких кап. стран ! Поверьте.... я на своей шкуре испытал такое "выпускали"...
Когда мне пришло приглашение от мэра одного из Итальянских городов на концерты,
меня даже не поставили в известность.... из Минкульта ответили, что я занят на
внутренних гастролях...

\end{itemize} % }

\iusr{Anna Bankovskaya}

"Но это наша Родина!" Я об анекдоте... Мне иногда кажется, что я, и все мои
любимые, талантливые, как вы, родились не там. Как будто спутали вокруг, вдоль
и поперёк - в кокон... И эта бабочка будет долго рождаться.

\iusr{Владимир Быстряков}
\textbf{Anna Bankovskaya} права 100\%....

\iusr{Ирина Цветкова}

Наши враги добились своего. Все против всех. Такую страну можно взять голыми
руками. А ведь когда-то были единой нацией, гордились нашими
достижениями: Победой, армией, космосом, Олимпиадой и нашими спортсменами,
наукой, культурой, кинематографом... Куда всё это ушло? Всех разобщили, нет
ничего такого, что объединяло бы людей. И кто-то хорошо работает на
разъединение.

\begin{itemize} % {
\iusr{Владимир Быстряков}
\textbf{Ирина Цветкова} По кальке раздела Югославии

\iusr{Елена Задорина}
\textbf{Владимир Быстряков} а кто ее теперь помнит, Югославию? стёрта с карты....
\end{itemize} % }

\iusr{Natalia Mieshkova}
Ложка дегтя портит бочку меда))))

\begin{itemize} % {
\iusr{Владимир Быстряков}
\textbf{Natalia Mieshkova} Ложка дегтя портит бочку меда.... Зеленскому

\iusr{Natalia Mieshkova}
\textbf{Владимир Быстряков} нам тоже достается..
\end{itemize} % }

\iusr{Виктор Злочевский}

Вы видимо не слушали выступление Зеленского. Он много всего объяснил. Мы - за
последнее время достигли небывалых высот. Впервые за годы незалежности
минимальная зарплата достигла \$500. Впервые врачам зарплата увеличена в два
раза! Вот-вот увеличат зарплату учителям. Наша армия - самая мощная в мире.

Дети будут получать на счет деньги, процент за добычу полезных ископаемых!

Медицина обеспечена всем необходимым - лекарствами и техникой. Полностью
укомплектован парк автомобилей скорой помощи! И еще много-много другого. Так
что рывок уже произошел. Мы просто его пока не заметили. Нужно видимо лучше
искать!

\begin{itemize} % {
\iusr{Владимир Быстряков}
\textbf{Виктор Злочевский} за Зеленским - наше Будущее!!!! Ураааааа.....(раскатисто)

\iusr{Евгения Зубкова}
\textbf{Виктор Злочевский} это похоже на изречение : " пилите гири они золотые... "

\iusr{Леонид Сорокин}
\textbf{Владимир Быстряков} Васюковщина расцветает.

\iusr{Лариса Солодкова}
\textbf{Виктор Злочевский} У нас Президент - романтик

\iusr{Vera Vera}
\textbf{Виктор Злочевский} бурные продолжительные аплодисменты  @igg{fbicon.hands.applause.yellow}{repeat=5} 

\iusr{Тамара Васильева}
\textbf{Леонид Сорокин} Васюки отдыхают .....

\iusr{Татьяна Шанель}

\ifcmt
  ig https://scontent-frt3-1.xx.fbcdn.net/v/t39.1997-6/s480x480/21386458_2016938778539112_7350638039485382656_n.png?_nc_cat=106&ccb=1-5&_nc_sid=0572db&_nc_ohc=3K1Fzu2Hm1MAX-QVJhT&_nc_ht=scontent-frt3-1.xx&oh=6923ba00f350745a55d3d5a69b8b99aa&oe=61B08238
  @width 0.2
\fi

\iusr{Татьяна Павлович}
А чё у Зеленского по арифметике двойка была? Считает он как-то по своему. А он
же юрист, право изучал! @igg{fbicon.laugh.rolling.floor}{repeat=3} 

\iusr{Виктор Злочевский}
\textbf{Владимир Быстряков} "Это даже хорошо, что пока нам плохо!"  @igg{fbicon.beaming.face.smiling.eyes}  (Из мульфильма).

\iusr{Владимир Михайленко}
\textbf{Виктор Злочевский}

\ifcmt
  ig https://scontent-frt3-1.xx.fbcdn.net/v/t39.30808-6/263422571_300822051942149_5435755712024641892_n.jpg?_nc_cat=106&ccb=1-5&_nc_sid=dbeb18&_nc_ohc=79Oy13Wozb0AX-p8lOE&_nc_oc=AQk37TBcABpqyVGj1fy_O3bRd4s4rzFYNsJtDbf7TYzEGG7-j0y7eo3_XmNofPXLs8M&_nc_ht=scontent-frt3-1.xx&oh=9c4ea31ffaf05e1d80707e6eabbd2c67&oe=61AFDED7
  @width 0.4
\fi

\iusr{Владимир Быстряков}
\textbf{Татьяна Павлович} изучал "Основы КВН"...

\iusr{Бурвиков Сергей}
\textbf{Виктор Злочевский} увеличена у кого чьи учителя чья армия научитесь читать между строк

\iusr{Татьяна Павлович}
\textbf{Владимир Быстряков} И игру на рояле необычным инструментом.

\iusr{Elena Sibiriakova}
\textbf{Виктор Злочевский}

\href{https://www.youtube.com/watch?v=Oj1cIGuWGi8}{%
Александр Градский - Экспедиция, youtube, 01.11.2019%
}

\iusr{Нина Юрова}
Ох , и славно на бумаге!Да забыли про овраги, а по ним ходить...

\iusr{Lyubov Volkova}
\textbf{Лариса Солодкова} Вiн просто видае бажане за дiйсне i при тому мае чудову здатн1сть не червонiти. Малюе повiтрянi замки, на зразок передвиборних.

\iusr{Елена Задорина}
\textbf{Виктор Злочевский} если успеют врачи ухватить эту двойную зарплату... мне кажется, УЕЗЖАЮТ быстрей, чем их успеет догнать двойная з. пл....


\end{itemize} % }

\iusr{Юрий Ватник}

Да, Германия тридцатых годов прошлого века рванула в лидеры. Во всем. В первую
очередь в личнном ощущении себя как гражданина полноценного государства.
Хороший пример.

Вот потому, я и не желаю успеха стране, гражданином которой являюсь.

Не приведи господь. Лучше быть жертвой....

Хотя, да, тоде не хочется.

\begin{itemize} % {
\iusr{Владимир Быстряков}
\textbf{Юрий Ватник} вы путаете Германию 30-х с современным лидером Европы. Читайте внимательнее...
\end{itemize} % }

\iusr{Татьяна Корнилова}
Свобода слова, доведенная до абсурда)

\iusr{Ирина Виноградова}

Те народы, у которых рабство как явление вызывает стойкое чувство брезгливости
и неприятия на генетическом уровне, вовремя мобилизуются и сплачиваются.
Забывают все дрязги и обиды перед лицом опасности или катастрофы. Те же народы,
для которых колониальный режим приемлем и допустим, в итоге обречены на то, что
их просто поглотит колонизатор. Вот и все. Осталось каждому решить, какое
чувство у него конкретно вызывает чувство рабства. Я для себя давно решила, что
лучше смерть, чем рабство.


\iusr{Наталия Осипова}
ой, как же хочется вернуться в тот махровый)) из нынешнего феодального!

\iusr{Виктор Злочевский}
Да, уж!

\iusr{Олег Германович Левин}

\url{https://www.facebook.com/permalink.php?story_fbid=3021312568196276&id=100009526865231}

\iusr{Сергей Быков}
Если дров не наломаем - замёрзнем ведь!

\iusr{Ольга Никульская}
А я за развал не голосовала, и этим горда!




\end{itemize} % }
