% vim: keymap=russian-jcukenwin
%%beginhead 
 
%%file 06_02_2022.yz.figuristochka.1.ukraincy_valieva
%%parent 06_02_2022
 
%%url https://zen.yandex.ru/media/suvorova_znaet/ukraincy-protiv-valievoi-v-seti-obsujdaiut-nesportivnoe-povedenie-sbornoi-na-komnadnom-turnire-oi-61ffb85a07fcbd5138fe5456
 
%%author_id yz.figuristochka
%%date 
 
%%tags figurnoje_katanie,olimpiada,olimpiada.pekin.2022,ravnodushie,rossia,sport,ukraina,valieva_kamila.rossia.figuristka
%%title "Украинцы против Валиевой". В сети обсуждают неспортивное поведение сборной на комнадном турнире ОИ
 
%%endhead 
 
\subsection{\enquote{Украинцы против Валиевой}. В сети обсуждают неспортивное поведение сборной на комнадном турнире ОИ}
\label{sec:06_02_2022.yz.figuristochka.1.ukraincy_valieva}
 
\Purl{https://zen.yandex.ru/media/suvorova_znaet/ukraincy-protiv-valievoi-v-seti-obsujdaiut-nesportivnoe-povedenie-sbornoi-na-komnadnom-turnire-oi-61ffb85a07fcbd5138fe5456}
\ifcmt
 author_begin
   author_id yz.figuristochka
 author_end
\fi

После победного выступления Камиле Валиевой аплодировали стоя все соперники.
Кроме сборной Украины. Сидевшие плечо к плечу с другими фигуристами, украинские
атлеты продемонстрировали равнодушие.

Ладно бы это был рядовой прокат, но российская одиночница выдала историческое
выступление. Оно уже войдет в золотой фонд Олимпийских игр, — уверены
болельщики. 

\subsubsection{Исторический прокат}

Чистый тройной аксель, идеальное исполнение всех остальных элементов и
завораживающее катание. Короткая программа Камилы Валиевой под музыку Кирилла
Рихтера In memoriam не оставляет равнодушным никого. Ни болельщиков, ни других
спортменов. Как этой 15-летней девочке удается творить волшебство на льду, не
может объяснить ни один эксперт. Выступления Камилы — это гармоничное сочетание
сложнейших технических элементов и неподражаемого артистизма. «Катается
шепотом», — как говорит Татьяна Анатольевна Тарасова. А так как на трибунах
олимпийского катка в Пекине почти нет зрителей, то этот шепот просто оглушает.
Даже у телевизоров зрители боятся пошевелиться, да что там пошевелиться —
боятся дышать, пока катается Камила.

Ее программа о девочке, которая ловит бабочку, а в конце отпускает. Образ
перекликается с заложенным в музыку смыслом. Композитор Кирилл Рихтер написал
In memoriam как произведение-память об ушедших близких. Для Камилы Валиевой —
особая тема. Фигуристка со слезами на глазах в одном своем интервью рассказала,
что потеряла бабушку. И катается в память о ней. 

\subsubsection{В чем секрет успеха?}

И, конечно, мы, зрители, чувствуем эту энеретику. Считываем историю,
рассказанную на льду так легко и так глубоко одновременно. И задаемся вопросом,
а в чем секрет успеха Валиевой. Только ли в технике и артистизме? Пожалуй, нет.
Спортсменка одарена талантом, который тренерам — Этери Тутберидзе, Сергею
Дудакову и Даниилу Глейхенгаузу удалось раскрыть в этой программе. Именно в
этой. В произвольной мы увидим совсем другую Камилу — более жесткую, волевую,
страстную танцовщицу болеро. Но это в проризвольной. Ее можно не любить, но не
восхищаться при чистом исполнении — нельзя. 

\subsubsection{Реакция соперников}

А вот In memoriam заставляет плакать даже мужчин, о чем они не стесняются
говорить и оставляют щемящие сердце комментарии. «Камила — ты наше золото». Вот
только все это решили не замечать фигуристы сборной Украины. Честны ли они были
перед собой в тот момент? Полагаю, что нет. Потому что даже главные конкуренты
российской команды — сборные США, Японии, Китая — встали, чтобы
поприветствовать после проката нашу девочку. Ведь спорт — это уважение к
соперникам, которые сильнее, которые показывают, что нет предела совершенству.
В эти мгновения не важно, кто победил, это моменты истины, так сказать. 

Но украинцы, как по указке, кто сидел в телефоне, а кто-то вяло хлопал. Ни один
не встал. Сделали ли это они потому что их «попросили»? Неизвестно. И мы
заметили, что они это сделали. Вот только это неспортивное поведение не
сказалось на результате Камилы, а сказалось на отношении к атлетам украинской
сборной.

А вы что думаете по поводу этой ситуации?
