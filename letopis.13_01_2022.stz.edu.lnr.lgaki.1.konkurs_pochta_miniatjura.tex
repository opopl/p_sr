% vim: keymap=russian-jcukenwin
%%beginhead 
 
%%file 13_01_2022.stz.edu.lnr.lgaki.1.konkurs_pochta_miniatjura
%%parent 13_01_2022
 
%%url https://lgaki.info/novosti/bolee-sotni-zayavok-poluchili-organizatory-konkursa-pochtovyh-miniatyur-posvyashhennogo-20-letiyu-akademii-matusovskogo
 
%%author_id lgaki
%%date 
 
%%tags donbass,lnr,lgaki,konkurs,pochta,hudozhnik,isskustvo,studenty,obrazovanie,lugansk
%%title Более сотни заявок получили организаторы конкурса почтовых миниатюр, посвященного 20-летию Академии Матусовского
 
%%endhead 
\subsection{Более сотни заявок получили организаторы конкурса почтовых миниатюр, посвященного 20-летию Академии Матусовского}
\label{sec:13_01_2022.stz.edu.lnr.lgaki.1.konkurs_pochta_miniatjura}

\Purl{https://lgaki.info/novosti/bolee-sotni-zayavok-poluchili-organizatory-konkursa-pochtovyh-miniatyur-posvyashhennogo-20-letiyu-akademii-matusovskogo}
\ifcmt
 author_begin
   author_id lgaki
 author_end
\fi

Декан факультета изобразительного и декоративно-прикладного искусства главного
творческого вуза Донбасса Наталья Феденко уточнила: пока все заявки –
исключительно от студентов факультета. Но ректор Академии Валерий Филиппов
подчеркивает: принять участие в конкурсе могут студенты всех направлений
подготовки, не только художники. Главное – интересная, остроумная,
выразительная идея, а доработать ее специалисты помогут!

\ii{13_01_2022.stz.edu.lnr.lgaki.1.konkurs_pochta_miniatjura.pic.1}

Напомним, что 2022-й – год 20-летия Академии Матусовского. И конкурс почтовых
миниатюр – одна из списка творческих акций, которыми вуз планирует отметить эту
дату.

Свои идеи о том, как могут выглядеть посвященные 20-летию вуза почтовая марка,
почтовый конверт, штамп, студентов ЛГАКИ просят направлять на электронную почту
\url{lg3_2012@mail.ru} до конца января. Итоги подведут и победителей назовут 14
февраля.

На фото – одно из присланных на конкурс работ. Её автор — Евгения Бондарь.

