% vim: keymap=russian-jcukenwin
%%beginhead 
 
%%file 16_09_2022.stz.news.ua.donbas24.1.top_5_knyg_donechyny_varto_prochytaty_ukraincjam
%%parent 16_09_2022
 
%%url https://donbas24.news/news/top-5-knig-avtoriv-doneccini-yaki-varto-procitati-vsim-ukrayincyam
 
%%author_id demidko_olga.mariupol,news.ua.donbas24
%%date 
 
%%tags 
%%title Топ-5 книг авторів Донеччини, які варто прочитати всім українцям
 
%%endhead 
 
\subsection{Топ-5 книг авторів Донеччини, які варто прочитати всім українцям}
\label{sec:16_09_2022.stz.news.ua.donbas24.1.top_5_knyg_donechyny_varto_prochytaty_ukraincjam}
 
\Purl{https://donbas24.news/news/top-5-knig-avtoriv-doneccini-yaki-varto-procitati-vsim-ukrayincyam}
\ifcmt
 author_begin
   author_id demidko_olga.mariupol,news.ua.donbas24
 author_end
\fi

%\ii{16_09_2022.stz.news.ua.donbas24.1.top_5_knyg_donechyny_varto_prochytaty_ukraincjam.txt}
\ii{16_09_2022.stz.news.ua.donbas24.1.top_5_knyg_donechyny_varto_prochytaty_ukraincjam.pic.front}

\begin{center}
  \em\color{blue}\bfseries\Large
  Є книги, завдяки яким можна поринути в далеке минуле свого рідного краю 
\end{center}

Є невеличкий список книг маріупольських та донецьких авторів, які точно стануть
цікавими мешканцям Донбасу. Адже вони розповідають про історію, культуру та
традиції міст Донеччини, містять унікальну інформацію та підійдуть на будь-який
настрій та смак. А в деяких з них піднімаються ті питання, \emph{які є актуальними} і
в наші дні. Загалом ці книги варто прочитати всім українцям.

\subsubsection{1. \enquote{Як Україна втрачала Донбас}}

\ii{16_09_2022.stz.news.ua.donbas24.1.top_5_knyg_donechyny_varto_prochytaty_ukraincjam.pic.1}

\ii{insert.read_also.demidko.donbas24.donbaski_legendy}

Щоби зрозуміти сучасність правильно, потрібно провести\par\noindent справжнє історичне
дослідження. Це й вирішили зробити журналісти Денис Казанський та Марина
Воротинцева у своїй праці \textbf{\enquote{Як Україна втрачала Донбас}}. Серед книг про сучасну
Україну це дослідження є надзвичайно важливим. \href{https://www.facebook.com/den.kazansky}{\emph{Денис Казанський}}%
\footnote{\url{https://www.facebook.com/den.kazansky}}
 — журналіст, до 2014 року жив у Донецьку, у 2020 році брав участь у Тристоронній контактній
групі з врегулювання ситуації на Донбасі в ролі представника від Донеччини. У
зв'язку з війною на Донбасі він був змушений покинути Донецьк. \emph{\textbf{Марина
Воротинцева}} також має чималий досвід у журналістиці та до початку війни
очолювала й редагувала видання \enquote{Восточный вариант} у Луганську. Автори докладно
розповідають новітню історію двох областей — Донецької та Луганської. Чому все
склалося саме так і сьогодні на цих землях проживають люди, орієнтовані скоріше
на росію, ніж на свою батьківщину? Як відбувалося становлення потужних
донбаських кланів, які постраждали від ворожнечі, яку самі ж і розпалили? Ця
книжка необхідна для правильного розуміння історії сьогодення, яка оголює
непросту правду та змушує замислитися. Сьогодні ця книга набула особливої
актуальності та стане цікавою для всіх, хто цікавиться історією своєї країни та
хоче більш детально розібратися в подіях, які й наразі продовжують викликати
багато дискусій.

\subsubsection{2. \enquote{Шлях додому}}

\ii{16_09_2022.stz.news.ua.donbas24.1.top_5_knyg_donechyny_varto_prochytaty_ukraincjam.pic.2}

\textbf{\enquote{Шлях додому}} — п'єса \emph{\textbf{Ігоря Тура}}, за якою у 2020 році на сцені Донецького
академічного обласного драматичного театру (м. Маріуполь) була поставлена
вистава, над якою працювали режисер-постановник \href{https://www.facebook.com/profile.php?id=100013458379308}{Андрій Луценко}%
\footnote{\url{https://www.facebook.com/profile.php?id=100013458379308}}
(наразі боронить
Україну в лавах ЗСУ) та художниця \href{https://www.facebook.com/alena.murashkina.3}{Олена Мурашкіна}.%
\footnote{\url{https://www.facebook.com/alena.murashkina.3}} Співавтором сценарію став
заступник начальника поліції Донецької області \href{https://www.facebook.com/people/Артем-Кисько/100008528200769}{Артем Кисько}.%
\footnote{\url{https://www.facebook.com/people/Артем-Кисько/100008528200769}}
Він переселенець та учасник бойових дій. Його спогади та історії його колег стали основою постановки.

\begin{leftbar}
\emph{\enquote{Ця п'єса про любов, ненависть і зраду. Головна ідея твору у тому, що ми — усі
різні, але Батьківщина у нас одна}}, — розповів автор п'єси Ігор Тур.
\end{leftbar}

\begin{leftbar}
\emph{\enquote{Всі рани загоюються. Але є рани такі, що не повинні гоїтися ніколи. Вони
болять і нагадують всім громадянам про те, що цілісність країни не відновлено,
йде війна. Єдиний шлях повернення додому — це єднання}}, — наголосив Артем
Кисько.
\end{leftbar}

Дія п'єси відбувається напередодні Нового року в квартирі звичайної
пенсіонерки, де в силу різних обставин збирається досить строката компанія. У
кожного з персонажів — свій біль, своя історія і своя правда. Хтось виживає як
може, хтось навіть отримує сумнівні дивіденди від нестабільної ситуації, а
хтось намагається надавати посильний опір обставинам. Протистояння гідності й
безпринципності, вірності обов'язку і кар'єризму, істинного патріотизму і
пристосуванства набуває особливої гостроти, коли за свої переконання та
принципи людина може поплатитися здоров'ям і життям. Ментальна деокупація як
запорука повернення тимчасово окупованих територій, — ось на чому був зроблений
головний наголос у п'єсі \enquote{Шлях додому}. Сьогодні ця історія знайде відлуння у
серці кожного українця. Наразі Ігор Тур закінчив ще одну п'єсу \textbf{\enquote{Квитки
дійсні!}}, події якої відбуваються в будівлі драмтеатру Маріуполя напередодні
його знищення російським літаком.

\subsubsection{3. \enquote{Історія одного дня}}

\ii{16_09_2022.stz.news.ua.donbas24.1.top_5_knyg_donechyny_varto_prochytaty_ukraincjam.pic.3}

Всім, хто віддає перевагу книгам, сповненим гумору, рекомендую оповідання, що
стало для мене справжньою знахідкою — \par\noindent\textbf{\enquote{Історія одного дня}} \emph{\textbf{Петра Валерійовича
Каменського}}. За один день головний герой твору \href{https://donbas24.news/news/yak-meskanci-priazovya-v-minulomu-karali-necesnix-cinovnikiv-xabarnikiv-yaki-zlovzivali-vladoyu}{Григорій Ілляшенко}%
\footnote{Як мешканці Приазов'я в минулому карали нечесних чиновників-хабарників, Ольга Демідко, donbas24.news, 25.07.2022, \par%
\url{https://donbas24.news/news/yak-meskanci-priazovya-v-minulomu-karali-necesnix-cinovnikiv-xabarnikiv-yaki-zlovzivali-vladoyu}%
}
вирішив провчити колишнього члена грецького суду Маріуполя Логафетова за зловживання
довіреною йому владою. Виявивши неабиякі акторські здібності, він представився
уповноваженим імператора і розпорядився 
\begin{quote}
\em\enquote{за грабежі та вбивство і взагалі за
всі зловживання позбавити (Логафетова) всіх прав стану зі засланням в алтайські
заводи у вічні працівники, а маєток його продати з публічного торгу і
задовольнити всіх боржників...}. 
\end{quote}
Маріупольська влада, налякана \enquote{начальницькою}
особою, завзято виконала всі вказівки, заарештувала Логафетова і навіть
поголила йому півголови. Але, на жаль, благородна справа Г. Ілляшенка була
розкрита… Цей короткий переказ — ніщо в порівнянні з самою книгою, яка дозволяє
краще зрозуміти побут, відносини між різними народами, особливості життя в
Маріуполі наприкінці XIX ст. І якщо вірити автору, ця неймовірна історія не
була вигадкою.

\subsubsection{4. \enquote{Одного разу в СРСР}}

\ii{16_09_2022.stz.news.ua.donbas24.1.top_5_knyg_donechyny_varto_prochytaty_ukraincjam.pic.4}

Всім, хто дійсно любить Маріуполь і його історію, варто звернути увагу на книгу
\textbf{\enquote{Однажды в СССР}} письменника, фотографа та військового \href{https://www.litmir.me/a/?id=15380}{\emph{Андрія Марченка}},
\footnote{\url{https://www.litmir.me/a/?id=15380}}
яка, можливо у майбутньому буде екранізованою. Книга переносить читачів у
брежнєвські часи. Життя міста у моря в 1970-ті не може не зачарувати. В центрі
сюжету доля молодого героя, який, образившись на тодішні реалії і панування
радянської номенклатури, разом з армійським товаришем готує пограбування, що
сколихне приморське місто. В книзі дуже точно передана атмосфера того часу.
Сторінки роману дозволяють перенестися у минуле і побачити вулиці Маріуполя 50
років тому. Динамічний сюжет, екшн та переживання за героїв захоплять юних
читачів з перших сторінок книги. До речі, у автора є й багато інших книг,
події яких відбуваються в Маріуполі та Приазов'ї.

\subsubsection{5. \enquote{Огонь и агнец}}

\ii{16_09_2022.stz.news.ua.donbas24.1.top_5_knyg_donechyny_varto_prochytaty_ukraincjam.pic.5}

А любителям досліджень раджу ознайомитися з твором маріупольського поета,
прозаїка та есеїста \textbf{\emph{Анатолія Ніколіна}} \textbf{\enquote{Огонь и агнец}}.
Це есе-дослідження про поета і прозаїка, що працював у Франції — \emph{Бокова
Миколу}. Автор активно переписувався з Боковим до його смерті, а коли дізнався,
що батько Бокова ще й проживав у Маріуполі, вирішив з'ясувати всі деталі
сімейної історії письменника. Стиль Ніколіна відрізняється образністю,
афористичністю, використанням свіжих метафор і дуже багатою мовою, тому викличе
справжнє естетичне задоволення у всіх, хто ознайомиться з його роботою.

\ii{insert.read_also.demidko.donbas24.uinp_mify_druga_svitova_vijna}

Ще більше новин та найактуальніша інформація про Донецьку та Луганську області
в нашому телеграм-каналі Донбас24.

ФОТО: з відкритих джерел

\ii{insert.author.demidko_olga}
