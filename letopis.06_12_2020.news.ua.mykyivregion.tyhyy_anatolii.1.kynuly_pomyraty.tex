% vim: keymap=russian-jcukenwin
%%beginhead 
 
%%file 06_12_2020.news.ua.mykyivregion.tyhyy_anatolii.1.kynuly_pomyraty
%%parent 06_12_2020
 
%%url https://mykyivregion.com.ua/analytics/medicnii-skandal-na-kiyivshhini-yak-likari-svidkoyi-pokinuli-pomirati-xvorogo-z-insultom
 
%%author Тихий, Анатолій
%%author_id tyhyy_anatolii
%%author_url 
 
%%tags 
%%title Кинули помирати: як на Київщині лікарі "швидкої" не схотіли рятувати хворого з інсультом
 
%%endhead 
 
\subsection{Кинули помирати: як на Київщині лікарі \enquote{швидкої} не схотіли рятувати хворого з інсультом}
\label{sec:06_12_2020.news.ua.mykyivregion.tyhyy_anatolii.1.kynuly_pomyraty}
\Purl{https://mykyivregion.com.ua/analytics/medicnii-skandal-na-kiyivshhini-yak-likari-svidkoyi-pokinuli-pomirati-xvorogo-z-insultom}
\ifcmt
	author_begin
   author_id tyhyy_anatolii
	author_end
\fi

\index[rus]{Киев!Скорая, безответственность, 08.12.2020}

\begin{leftbar}
	\bfseries
Ситуація з коронавірусом набирає обертів, але пацієнтів зі звичайними хворобами
менше не стало. При цьому чимало лікарень перепрофільовані \enquote{ковідними}, тому
інших пацієнтів вони відмовляються приймати. До редакції порталу \enquote{Моя Київщина}
звернулися люди, які постраждали чи то від халатності лікарів, чи то від
безнадійної медичної системи. А може і від того, й іншого...
\end{leftbar}

\ifcmt
pic https://mykyivregion.com.ua/imagecache/large/img/78d25bf44938fdf5df77043337ba2f361607374675.jpg
cpx Фото з мережі Інтернет
\fi

Ситуація розвивалася наступним чином: літньому чоловіку з Боярки, який зараз
перебуває на пенсії, одного дня після обіду стало зле. Спочатку звертатися до
лікарів не став, вирішив спробувати приборкати свій організм самостійно. Проте
це не вдалося, і згодом родина звернулася до лікарів швидкої.

Бригада, яка приїхала на виклик, довго не могла встановити діагноз: запідозрили
інсульт, але для подальшої діагностики та лікування потерпілому запропонували
їхати в Білу Церкву. Звісно, всі розуміють ситуацію з коронавірусом та те, що
частина лікарень не приймає пацієнтів, але їхати з Боярки в Білу Церкву дуже
далеко і довго, тому почали шукати альтернативний варіант. Після телефонної
розмови з лікарем у Василькові та підтвердження майбутнього огляду, \enquote{швидка} зі
скандалом погодилася їхати саме туди, що значно ближче і швидше.

У Василькові ближче до 12 ночі пацієнта оглянули, дійсно запідозрили інсульт,
але диференціювати геморагічний чи ішемічний (від цього залежить подальше
лікування) там не змогли. Як пояснив лікар, у його медзакладі немає
відповідного обладнання, зокрема магнітно-резонансного томографа (МРТ), тому
рекомендував зробити це дослідження. Єдиний оптимальний варіант – Київська
обласна клінічна лікарня (КОКЛ), яка повинна цілодобово приймати пацієнтів та
надавати спеціалізовану допомогу. Протягом декількох годин вирішувалося питання
про госпіталізацію хворого до КОКЛ, але єдине що запропонували їм по телефону –
приїхати вранці й зробити необхідне обстеження.

Саме в цей час, коли вирішувалося питання щодо проведення МРТ головного мозку,
бригада \enquote{швидкої} не дочекалася госпіталізації чоловіка і просто поїхала (!).
Фактично хворий залишився посеред ночі на порозі лікарні, де йому не могли
надати належну допомогу. І тут постає питання: хто відповідальний за життя
людини, яка потребує допомоги? Чи не старший лікар підстанції "швидкої" має
домовлятися з приймальними відділеннями щодо проведення обстежень та
госпіталізацій? Що робити пацієнтам, які опиняються на волосині від смерті?
Добре коли є близькі, які можуть взяти на себе відповідальність за життя
хворого, але що робити, коли немає нікого?

Далі ситуація розгорталася наступним чином: син поклав хворого батька на заднє
сидіння власного авто й поїхав до Києва, паралельно шукаючи допомоги. Знайомі з
медичної сфери підказали, що вони мають повне право бути прийнятими обласною
лікарнею, тому вирушили саме туди. При цьому рекомендували вимагати на
проведенні обстеження, адже співробітники КОКЛ часто необґрунтовано відмовляють
хворим, особливо вночі. Ближче до третьої ночі хворому все-таки зробили МРТ,
після чого розпочали лікування.

Через те, що був втрачений дорогоцінний час, а хворому не була вчасно надана
допомога, його чекає тривала реабілітація після перенесеного інсульту. Лікарі у
відділенні запевнили, що якби розпочалося лікування вчасно, то прогноз був би
набагато кращим. Хто ж винен у цій ситуації будуть з’ясовувати правоохоронці,
адже родина вирішила добиватися справедливості, аби інші не потрапили в таку
ситуацію.

На жаль, повна безвідповідальність та байдужість наших медиків - повсюдна. А
отже, ми маємо проблему системного характеру, яку треба вирішувати теж не
епізодично, а комплексно. Чи здатна нинішня влада - на загальноукраїнському та
обласному рівні? З огляду на свіжі рішення Київської облради - зокрема,
призначення головою постійної комісії з питань охорони здоров'я депутата від
\enquote{Слуги народу} Марії Кисельової - помічниці нардепа Олександра Дубінського, яка
разом з ним працювала на \enquote{плюсах} і ніколи не мала до медицини жодного
стосунку, викликає чимало запитань. Адже фактично влада замість того, щоб зараз
- у цей вкрай складний період - давати \enquote{зелене світло} професіоналам, продовжує
сунути на відповідальні посади абсолютних профанів. Це знищує галузь, яка і так
перебуває в коматозному стані... 
