% vim: keymap=russian-jcukenwin
%%beginhead 
 
%%file 26_01_2022.stz.news.lnr.lug_info.1.turnir_sportivnaja_borjba
%%parent 26_01_2022
 
%%url https://lug-info.com/news/turnir-respublikanskoj-federacii-sportivnoj-bor-by-sobral-130-sportsmenov-iz-lnr-i-dnr
 
%%author_id news.lnr.lug_info
%%date 
 
%%tags borjba.sportivnaja,donbass,lnr,lugansk,sport,turnir
%%title Турнир Республиканской федерации спортивной борьбы собрал 130 спортсменов из ЛНР и ДНР
 
%%endhead 
 
\subsection{Турнир Республиканской федерации спортивной борьбы собрал 130 спортсменов из ЛНР и ДНР}
\label{sec:26_01_2022.stz.news.lnr.lug_info.1.turnir_sportivnaja_borjba}
 
\Purl{https://lug-info.com/news/turnir-respublikanskoj-federacii-sportivnoj-bor-by-sobral-130-sportsmenov-iz-lnr-i-dnr}
\ifcmt
 author_begin
   author_id news.lnr.lug_info
 author_end
\fi

Открытый турнир Республиканской федерации спортивной борьбы среди младших
юношей \enquote{Новогодние старты}, который прошел в столице ЛНР, собрал 130
спортсменов из Луганской и Донецкой Народных Республик. Об этом сообщила
пресс-служба Министерства культуры, спорта и молодежи ЛНР со ссылкой на
главного судью соревнований Олега Плехова.

\ii{26_01_2022.stz.news.lnr.lug_info.1.turnir_sportivnaja_borjba.pic.1}

\enquote{Турнир прошел в специализированном зале греко-римской борьбы Луганского
высшего училища физической культуры. В соревнованиях приняли участие 130
спортсменов из Свердловска, Краснодона, Молодогвардейска, Красного Луча,
Луганска, Кировска, Стаханова, Донецка и Тореза. Также подали заявки команды из
российских Новочеркасска и Нальчика, но, к сожалению, из-за эпидемической
ситуации они не смогли к нам попасть}, – рассказал Плехов.

\ii{26_01_2022.stz.news.lnr.lug_info.1.turnir_sportivnaja_borjba.pic.2}

Состязания прошли в двух возрастных категориях: среди участников 2010-2012
годов рождения, а также 2013 года рождения и младше.

\enquote{Победители были определены в весовых категориях от 24 до 52+ кг для борцов
2010-2012 годов рождения и от 22 до 48 кг для спортсменов 2013 года рождения и
младше}, – уточнили в спортивной организации.

Победители и призеры соревнований получили грамоты, медали и подарки.

Президент Республиканской федерации спортивной борьбы ЛНР Алексей Воронюк
отметил, что организация тесно сотрудничает с Федерацией спортивной борьбы
Рязанской области,

\enquote{В ближайших планах у спортсменов – турниры в Тамбове и Рязани, на
которые они отправятся при поддержке Министерства культуры, спорта и молодежи
ЛНР и Республиканского физкультурно-спортивного общества \enquote{Динамо} }, –
рассказал он. 
