% vim: keymap=russian-jcukenwin
%%beginhead 
 
%%file 30_04_2020.fb.fb_group.story_kiev_ua.1.mozaika_kiev
%%parent 30_04_2020
 
%%url https://www.facebook.com/groups/story.kiev.ua/posts/1338756699654461/
 
%%author_id fb_group.story_kiev_ua,petrova_irina.kiev
%%date 
 
%%tags gorod,isskustvo,kiev,kultura,mozaika
%%title Хто знає де ці мозаїки розташовані?
 
%%endhead 
 
\subsection{Хто знає де ці мозаїки розташовані?}
\label{sec:30_04_2020.fb.fb_group.story_kiev_ua.1.mozaika_kiev}
 
\Purl{https://www.facebook.com/groups/story.kiev.ua/posts/1338756699654461/}
\ifcmt
 author_begin
   author_id fb_group.story_kiev_ua,petrova_irina.kiev
 author_end
\fi

Хто знає де ці мозаїки розташовані?

\begin{tabular}{cc}

\ii{30_04_2020.fb.fb_group.story_kiev_ua.1.mozaika_kiev.pic.1}
&
\ii{30_04_2020.fb.fb_group.story_kiev_ua.1.mozaika_kiev.pic.2}
\\

\end{tabular}

І ще один сумний епізод. На Байковому цвинтарі є стіна біля крематорію. Не всі
знають, що це велетенська могила, де поховано витвір мистецтва майстрів Ади
Рибачук та Володимира Мельниченка.  Після створення  Стіну пам'яті залили
бетоном за розпорядженням радянських чиновників. На щастя, робили скульптори
дуже якісно, а заливали, як і багато чого, халтурно. Зараз  Володимир
Мельниченко намагається відкрити для нащадків це панно. 

Доповнення, розлогіші описання вітаються. Ця тема оголошується автором "free of
politics". Прохання до адміну та модераторів викидати політичні інсінуації. Ми
ПРОСТО згадуємо красу нашого Міста! Дякую за розуміння.

\ii{30_04_2020.fb.fb_group.story_kiev_ua.1.mozaika_kiev.pic.stena_bajkovoje}

\ii{30_04_2020.fb.fb_group.story_kiev_ua.1.mozaika_kiev.cmt}
