% vim: keymap=russian-jcukenwin
%%beginhead 
 
%%file 06_04_2021.fb.mironenko_petr.1.amosov
%%parent 06_04_2021
 
%%url https://www.facebook.com/petro.myronenko/posts/3100369573524016
 
%%author 
%%author_id 
%%author_url 
 
%%tags 
%%title 
 
%%endhead 

\subsection{Не сподівайтеся на медицину - Амосов}
\label{sec:06_04_2021.fb.mironenko_petr.1.amosov}
\Purl{https://www.facebook.com/petro.myronenko/posts/3100369573524016}

\ifcmt
  pic https://scontent-bos3-1.xx.fbcdn.net/v/t1.6435-9/169331605_3100368976857409_9032428863058032683_n.jpg?_nc_cat=111&ccb=1-3&_nc_sid=730e14&_nc_ohc=9PnUrQiLwmEAX_Eaqgd&_nc_ht=scontent-bos3-1.xx&oh=3c0a50adbc78c7402072c8ce40579ccb&oe=6094EB01
\fi


ДО УВАГИ.

Сьогодні це важливо знати кожному громадянину України.

Академік Микола Амосов:

- Не сподівайтеся на медицину. Вона непогано "лікує" багато хвороб, але не може зробити людину здоровою. Поки вона навіть не може навіть навчити людину, як стати здоровою.
- Більше того: бійтеся потрапити в полон до лікарів! Часом вони схильні перебільшувати слабкість людини і могутність своєї науки, створюють у людей уявні хвороби і видають векселя, які не можуть оплатити.
- Щоб бути здоровим, потрібні власні зусилля, постійні і значні. Замінити їх не можна нічим. Людина створена настільки досконалою, що ПОВЕРНУТИ ЗДОРОВ'Я МОЖНА З БУДЬ-ЯКОЇ ТОЧКИ ЙОГО ЗАНЕПАДУ. Тільки необхідні зусилля зростають у міру віку і поглиблення хвороб.
- Для здоров'я однаково необхідні чотири умови: фізичні навантаження, обмеження в харчуванні, загартування, час і вміння відпочивати. І ще - щасливе життя! На жаль, без перших умов воно здоров'я не забезпечує. Але якщо немає щастя в житті, то де знайти стимули для зусиль, щоб напружуватися і голодувати? На жаль!
- Природа милостива: достатньо 20-30 хвилин фізкультури в день, але такий, щоб захекатися, спітніти і щоб пульс почастішав вдвічі. Якщо цей час подвоїти, то буде взагалі чудово.
-Потрібно обмежувати себе в їжі. Вибирайте будь-який спосіб: жити постійно впроголодь і бути худим весь час або їсти досхочу, а потім, кілька разів на рік, голодувати повністю.
- Вміти розслаблятися - наука, але до неї потрібен ще й характер. Якби він був!
- Кажуть, що здоров'я - щастя вже саме по собі. Це невірно: до здоров'я так легко звикнути і перестати помічати. 
-Ще одне - спеціально для молодих. До мене нерідко приходять налякані юнаки і дівчата, які вже тримаються за свою ліву половину грудей, скаржаться на серцебиття і розгортають сувої електрокардіограм. На щастя, у більшості з них ніякої хвороби, крім детренованості. Подивишся такого на рентгені: серце у нього - маленьке - стиснулося від бездіяльності. Йому б бігати або хоча б танцювати щодня, а він сидить сиднем і цигарку не випускає з рота. І спить вже з таблетками. 
Отож, не забувайте: рух, обмеження в їжі, загартовування і час для сну з повноцінним відпочинком - дійсно необхідні. Можете мені повірити.
