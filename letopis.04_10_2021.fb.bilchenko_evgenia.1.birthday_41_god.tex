% vim: keymap=russian-jcukenwin
%%beginhead 
 
%%file 04_10_2021.fb.bilchenko_evgenia.1.birthday_41_god
%%parent 04_10_2021
 
%%url https://www.facebook.com/yevzhik/posts/4309778169057260
 
%%author_id bilchenko_evgenia
%%date 
 
%%tags bilchenko_evgenia,birthday
%%title БЖ. Не жертва: моим читателям и союзникам
 
%%endhead 
 
\subsection{БЖ. Не жертва: моим читателям и союзникам}
\label{sec:04_10_2021.fb.bilchenko_evgenia.1.birthday_41_god}
 
\Purl{https://www.facebook.com/yevzhik/posts/4309778169057260}
\ifcmt
 author_begin
   author_id bilchenko_evgenia
 author_end
\fi

БЖ. Не жертва: моим читателям и союзникам

Поздравляйте меня здесь, пожалуйста: я стесняюсь не увидеть кого-то, в
комментариях все проще. Начался мой 41 год. Во всех смыслах. Никто не обещал,
что будет легко. Очень хочется сказать вам что-то такое итоговое, важное, но
без пафоса, я не выношу пафоса. Весь день для вас придумывала эти слова. Один
мальчик мне сказал, что я сама выбрала этот путь, потому что у меня была
возможность прогнуться. И теперь довольно неудачно строю из себя образ жертвы.
Аспиранту спасибо за этот последний триггер-урок: иногда нас учат наши ученики.
Точнее, - всегда, иначе учитель исчезает.

\ifcmt
  ig https://scontent-frt3-1.xx.fbcdn.net/v/t39.30808-6/244462652_4309778329057244_1636300530541452271_n.jpg?_nc_cat=106&_nc_rgb565=1&ccb=1-5&_nc_sid=8bfeb9&_nc_ohc=SHZUn5HH_JwAX_WFhY9&_nc_ht=scontent-frt3-1.xx&oh=60cd0feaff689f4e70be554c693fc17d&oe=61601D9D
  @width 0.4
  %@wrap \parpic[r]
  @wrap \InsertBoxR{0}
\fi

Благодаря мальчику я поняла фразу одной итальянской женщины, дочки
расстрелянного в Ардеатинских пещерах участника Сопротивления: "Не смейте
называть меня ребенком "невинной жертвы"! Я ребенок Победителя". 

Да, я не жертва. Мальчик прав. Жертва - это объект. Жертву не спрашивают, когда
и куда ведут. Она не имеет выбора. Победитель - боец, субъект, и всегда имеет
выбор. Крайний наш выбор - это жизнь или смерть. Христа я не считаю
объектом-жертвой. Это окончательная Победа Субъекта над миром объектностей. 

Во мне действительно есть страшная, темная, грешная сторона войны - воевать из
самолюбия, из отчаяния, из тупого упрямства. Я её подавляю. Церковь особое
место в этом занимает. Есть глупая, слабая, деморализирующая, чисто
постмодерная сторона полупацифизма - саму себя считать жертвой, сейчас же культ
идентичностей жертв. Часто я впадаю в эту воображаемую идентичность. Этого
никак нельзя русскому поэту.

Но во мне есть и третья сторона: я готова воевать по голосу совести вне всякого
самолюбия и выгоды, и это не чей-то, а именно мой выбор, не жертва, а
самостоятельное решение. Я это четко знаю и, как говорила Зоя в кино: "Я
выдержу". Это - не сравнение себя с Космодемьянской, меня реально вставило
кино. Я просто уже знаю. То есть, если я не умру, то я выдержу всё. А если
умру, - это и есть: "Выдержу". Это одно и то же. Меня это поразило. Самое
реальное, что есть, - невидимое. Кто-то называет его Богом, кто-то - Логосом,
кто-то - архетипом, кто-то цивилизацией, кто-то - поэзией. Но я точно теперь
знаю, что Символическое и есть Реальное. За это стоит умирать, ещё больше -
воевать, если на него нападают, чтобы уничтожить, и больше всего - жить и
любить.

Теперь о прозе. Если вас возникнет желание меня поздравить, вот номер карты,
поверьте мне сейчас ничего не лишнее, даже 10 гривен:

5168 7456 0304 8882 

Бильченко Евгения приват банк

Или международным переводом на:

BILCHENKO IEVGENIIA

Сразу говорю: я не стесняюсь просить, я не впадаю в гордыню всемогущества, я
работаю почти все свободное от лечения время. Но и полезной вам быть хочу.
Потому я была бы очень рада в ответ на ваши донаты подарить вам свои стихи,
книги и статьи. Авторство перестало иметь смысл. Во мне всё потеряло смысл,
кроме Логоса, и того, что "я не могу иначе". Потому обязательно пишите мне в
личку, не стесняйтесь, я могу не сразу увидеть, но я мониторю всё. Я же -
копирайтер: он обязан следить. Ну, и приходите завтра "на меня посмотреть" на
маленький итоговый сольник "Новое и лучшее" (встреча несколькими постами ниже).

... Если вы не придёте и не пришлёте ничего, я всё равно буду верить вам,
любить вас, говорить о вас на языке своих странных рифм и писать вам о том
духовном мире, который сделал нас одной цивилизацией - сильной и доброй, -
среди других, таких разных цивилизаций в огромном Божьем космосе. Обещаю!
Странным образом, но этот запретный в моей стране русский мир, с его взлетами и
падениями, рождениями и смертями, грехами и покаяниями, жутким и прекрасным
образом отражаю я. Это не специально и не самовозвеличивание, это зеркало.
Просто он и есть такой, а не как по телеку. Пост писался ради последних слов. Я
очень хочу, чтобы вы их прочувствовали.

\ii{04_10_2021.fb.bilchenko_evgenia.1.birthday_41_god.cmt}
