% vim: keymap=russian-jcukenwin
%%beginhead 
 
%%file 04_01_2019.stz.news.ua.lb.1.arhitekturnyj_atlas_dorevoljucijnogo_mariupolja.3.osobnjak_giacintova
%%parent 04_01_2019.stz.news.ua.lb.1.arhitekturnyj_atlas_dorevoljucijnogo_mariupolja
 
%%url 
 
%%author_id 
%%date 
 
%%tags 
%%title 
 
%%endhead 

\subsubsection{Особняк Гиацинтова}

Ниже, по той же улице сохранился ещё один дореволюционный особняк в стиле
модерн. В нём проживал Василий Иванович Гиацинтов – основатель и владелец
мариупольского реального училища. Интерьер дома соответствовал высокому статусу
хозяев - наборные паркеты, лепные украшения, резные дубовые двери, окна с
медными ставнями. Ограждал территорию кованый решетчатый забор, декорированный
головами змей, а украшением заднего двора был большой сад. После революции
имущество Гиацинтова национализировали и передали народу, что привело к
постепенному разграблению особняка. В сентябре 1943 года этот дом, вместе со
многими другими, также попал под прицел немецких факельщиков.

\ii{04_01_2019.stz.news.ua.lb.1.arhitekturnyj_atlas_dorevoljucijnogo_mariupolja.3.osobnjak_giacintova.pic.1}

После восстановления и до 90-х годов министерство здравоохранения использовало
помещения уникального особняка в качестве аптечного склада. Но в конце ХХ
столетия историческому зданию придумали более достойное назначение. По идее
общественников особняк Гиацинтова должен был стать художественным музеем имени
уроженца Мариуполя Архипа Куинджи. Процесс затянулся, долго искали мецената, да
и объём работ оказался внушительным. Но всё же 29 октября 2010 года состоялось
торжественное открытие нового музея в старом особняке. Восстановить
первоначальный внутренний интерьер и территорию заднего двора, конечно, не
удалось. Но оригинальный парадный фасад этого дома запоминается любому
прохожему, пусть даже случайно завернувшему на Георгиевскую улицу.

\ii{04_01_2019.stz.news.ua.lb.1.arhitekturnyj_atlas_dorevoljucijnogo_mariupolja.3.osobnjak_giacintova.pic.2}
\ii{04_01_2019.stz.news.ua.lb.1.arhitekturnyj_atlas_dorevoljucijnogo_mariupolja.3.osobnjak_giacintova.pic.3}
