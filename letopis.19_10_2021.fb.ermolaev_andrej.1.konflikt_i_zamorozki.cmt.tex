% vim: keymap=russian-jcukenwin
%%beginhead 
 
%%file 19_10_2021.fb.ermolaev_andrej.1.konflikt_i_zamorozki.cmt
%%parent 19_10_2021.fb.ermolaev_andrej.1.konflikt_i_zamorozki
 
%%url 
 
%%author_id 
%%date 
 
%%tags 
%%title 
 
%%endhead 
\subsubsection{Коментарі}

\begin{itemize} % {
\iusr{Сергій Злобін}

Любая война, рано или поздно заканчивается миром (доказано историей). Вопрос во
времени: продолжают гибнуть мирные граждане и военные, разрушаются дома и
производственные предприятия, ломаются человеческие судьбы, трудоспособное
население покидает территорию конфликта и т.д. Чем раньше закончиться
противостояние - тем быстрее появиться возможность восстановиться экономике. Но
погибших на этой войне уже не вернуть...

\iusr{Анатолий Чубинский}

Чем «мельче» политик у руля, тем более ему хочется быстрого и победоносного
решения. Видимо, ооочень мелкий... И тут, вероятно, найден источник постоянной
выгоды. Мудрая народная мудрость: «Кому война, а кому мать родная»- и не
поспоришь...

\iusr{Виктор Мироненко}

В целом, все так, дорогой АВ. Но для этого нужно, признать, что конфликт, а
точнее война, пусть гибридная и короткая, была и ее фактические результаты


\iusr{Павел Битман}
\textbf{Андрій Єрмолаєв}. Вы очень правы .

\iusr{Дмитрий Коломийченко}

И да и нет.

Нет:

1. Потому, что он является частью более масштабного конфликта (Россия-Запад)

2. Потому, что стабилизирует квазиустойчивую гос. систему Украины после 14
года. То есть конфликт стал стабилизирующим фактором политики.

\end{itemize} % }
