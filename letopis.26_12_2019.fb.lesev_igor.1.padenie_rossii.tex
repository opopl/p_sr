% vim: keymap=russian-jcukenwin
%%beginhead 
 
%%file 26_12_2019.fb.lesev_igor.1.padenie_rossii
%%parent 26_12_2019
 
%%url https://www.facebook.com/permalink.php?story_fbid=2890539604310457&id=100000633379839
 
%%author_id lesev_igor
%%date 
 
%%tags rossia,ukraina,ukraina_russia_relations
%%title Падение России
 
%%endhead 
 
\subsection{Падение России}
\label{sec:26_12_2019.fb.lesev_igor.1.padenie_rossii}
 
\Purl{https://www.facebook.com/permalink.php?story_fbid=2890539604310457&id=100000633379839}
\ifcmt
 author_begin
   author_id lesev_igor
 author_end
\fi

Падение России

Есть только две вещи, которые нас – проживающих в Украине от Львова до Луганска
– объединяет без всяких условностей. Это считать деньги в чужом кармане и
смотреть, а что там с Россией.

Россия – это наша общая формула помешательства. Она для нас давно уже не в
рамках рациональности, скорее, это вопрос веры. В Луганске веруют в величие
России. Во Львове в ее непременный развал. А где-то между этими верованиями
идут наши альбигойские войны, потому что кто-то в этих верованиях непременный
еретик и его нужно покарать.

\ifcmt
  ig https://scontent-frt3-1.xx.fbcdn.net/v/t1.6435-9/80896880_2890539370977147_1507457676068519936_n.jpg?_nc_cat=104&ccb=1-5&_nc_sid=730e14&_nc_ohc=i2iQQfixkKYAX8ZhFvJ&_nc_ht=scontent-frt3-1.xx&oh=5c667e746610cd14fbe2a1f115b32dc0&oe=61AAA9C7
  @width 0.4
  %@wrap \parpic[r]
  @wrap \InsertBoxR{0}
\fi

Но если смотреть на Россию, скажем так, атеистическим взглядом, то что-то с
этим соседом не так. Он постоянно ломается и всегда хромой. Нет, бывают черные
полосы. Случаются национальные трагедии. И феерические обломы. Но когда весь
распорядок жизни страны состоит из бесконечной черной полосы, то может вся
страна движется не поперек, а вдоль?

Вот навскидку без гугла. Утонула подлодка. Сгорел авианосец. Разбился первый
серийный истребитель типа пятого поколения. Обосрались с «Северным потоком-2».
Вдруг на шестом году санкций выяснили, что у самих нет судна-трубоукладчика. В
очередной раз и снова для местных очень неожиданно отстранили от мирового
спорта. Поставили раком и заставили заплатить три ярда зелени Украине, а ведь
сколько до этого визжали, что «никогда этого не будет».

А какую публичную порку на днях дал Лукашенко? Вот прямо на «Эхо Москвы». Нет
проекту объединения, а если сунетесь, в НАТО молчать не будут. Да и наша дружба
может закончиться выемкой «Дружбы». И ведь через неделю Лука будет как ни в чем
не бывало снова болтать с Путиным. Вот согласитесь, удивительной живучести
прохвост.

Но в целом все эти просеры имеют системность. Ну ничего же подобного и в таком
количестве не происходит в других крупных странах мира – Штатах, Китае, Индии.
Ну не валится там постоянно все ТАК и не обламываются глобально все начинания
тоже ТАК.

Ладно, туда или не совсем туда идет Россия, не нам решать. Но вот если ты
идешь, а с места не двигаешься, зато все кругом ломается, то, наверное, что-то
в этом движении не совсем так.

Вот мы все ржали с нашего Вакарчука на ру-корпоративах. Типа, квасной патриот
на подтанцовках под балалайки и черную икорку, но зато с тризубом на пузе.
Вакарчук – это, действительно, наш местный диагноз, который смущает. Но я вот
что-то не помню, чтобы в ру-СМИ как-то особо смущались, а что водочные магнаты
из России делают в Майами?

Или схемы Коболева/Витренко в «Нефтегазе». Мы задаемся вопросом, а кто эти
фирмы-прокладки, которые дают нам наценку на ру-газ и затем оседают в неведомых
офшорах?

Но я как-то не замечаю на ру-сайтах, чтобы кто-то особо рьяно интересовался, а
кто же из частных лиц владеет акциями «Газпрома»? А кого вообще в России
смущает, что четверть акций «Газпрома» приходится на держателей американских
депозитарных расписок?

Мы говорим о нашей украинской компрадорской элите. А чем ру-элита принципиально
в этом плане отличается от нашей? Дочь Пескова в Париже. Дочь Лаврова в
Лондоне. У всяких грызловых, астаховых, якуниных и легиона депутатов Думы – вся
детвора пристроена на Западе. Вот тут мы ватничек рвем на груди (в наших
реалиях идет замена на вышиванку), а их «цветы жизни» пээмжуются западнее
Калининграда.

В этом плане Россия еще более шизофреничная страна, чем наша. Присутствие детей
наших чиновников на Западе, по крайней мере, совпадает с провозглашенным
западным курсом. А там вроде как «холодная война», но только для быдло-плебеев
из Челябинска и Вологды. А вот у внутримкадовской элитки война заканчивается в
транзитной зоне Шереметьево.

Вот эта внутренняя шизофрения России, которая очень наглядно обозначена на их
гербе, делает непривлекательной страну для всех ее соседей. Без исключения. К
России тянутся только придуманные страны-изгои, вроде Абхазии или ДНР, или
режимно-бесперспективные сообщества, типа асадовской Сирии. Остальные по всему
периметру гигантской границы плюются, начиная от фьордов Норвегии и заканчивая
самой по себе чмошной КНДР. Но даже северокорейские погранцы периодически
захватывают российские рыболовные суда.

Россия отталкивает соседей. Пугает их. Потому что Россия не умеет дружить. Ей
нужно непременно затащить соседа в постельку с последующим требованием секса на
регулярной и безоплатной основе. Поэтому соседи шарахаются. И даже когда вроде
как дружат в рамках Таможенных союзов и ОДКБ, никогда не остаются с ночевкой и
всегда напряжены. Казахи, белорусы, да все.

Но ведь и ТАКАЯ Россия появилась не вдруг, и не из неоткуда. У такой России
есть отцы-основатели. Они сейчас пучат глазки и удивляются, хотя именно они в
90-е и породили эту Россию. Я о Штатах и Европе.

Россия уже в 90-е была униженной и карикатурной страной с радующими любого
русофоба куцыми границами. Вроде бы, возьмите и пригрейте. Ну ладно, не грейте,
но просто типа на равных говорите. Ну хотя бы просто без хамства. Но нет, надо
же было непременно пинать.

Ведь именно коллективный Запад русских в 90-е напугал до усрачки. Сделали из
сербов единственно виноватых во всех балканских войнах, сократив государство до
пригородов Белграда. А ведь русские смотрели не на Сербию, а смотрели в
зеркало. Расширили НАТО так, что дальше только Монголия. Еще и унизительно для
России прибалтов туда взяли. Топили за чеченцев во время войны.

А ладно бы, построили хоть где-то витрину анти-России. Да вот хотя бы из
Украины. Вот показали бы «новое ФРГ». Но вот то, что прямо сейчас делают с
Украиной – внешнее управление, тотальная деиндустриализация, долговая петля и
колониально-хищнический вывод ресурсов – пугает и так перепуганных русских еще
больше.

Вот что Запад предлагает России, чтобы она была непутинской даже после
физического отхода Путина? Ну вот что?

И вот нам здесь со всем этим жить. Между мародерами и шизофрениками. Еще и с
внутренним окружением мародеров и шизоидов, каждое утро начинающих молебен во
славу/проклятие России.

\ii{26_12_2019.fb.lesev_igor.1.padenie_rossii.cmt}
