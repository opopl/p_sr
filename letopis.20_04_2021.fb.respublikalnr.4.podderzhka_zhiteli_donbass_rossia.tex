% vim: keymap=russian-jcukenwin
%%beginhead 
 
%%file 20_04_2021.fb.respublikalnr.4.podderzhka_zhiteli_donbass_rossia
%%parent 20_04_2021
 
%%url https://www.facebook.com/groups/respublikalnr/permalink/799573617345098/
 
%%author 
%%author_id 
%%author_url 
 
%%tags 
%%title 
 
%%endhead 
\subsection{В поддержку жителей Донбасса! Писатели России вышли на улицы в поддержку Донбасса}
\Purl{https://www.facebook.com/groups/respublikalnr/permalink/799573617345098/}

Жители Донбасса — это часть разделенного русского народа, и когда чуть ли не
каждый день в народных республиках гибнут мирные люди, нельзя оставаться в
стороне. Об этом ФАН заявили члены Союза писателей России — участники серии
пикетов в поддержку Донбасса.

Акция, приуроченная к седьмой годовщине начала военной агрессии Киева против
жителей Новороссии, прошла 17 апреля в Москве и Московской области. Напомним,
что приказ о начале так называемой «антитеррористической операции» (АТО) был
опубликован на сайте президента Украины 14 апреля 2014 года. В результате
продолжающихся по сей день боевых действий десятки тысяч человек погибли, более
миллиона стали беженцами.

Появление Донецкой и Луганской народных республик стало ответом на украинскую
агрессию, ведь изначально протестовавшие в городах восточной Украины требовали
федерализации, а не независимости.

\subsubsection{Стоп геноцид}

Участники акции — поэты, прозаики, общественные деятели — разместили в своих
социальных сетях информацию об этой печальной дате, а потом, уже в частном
порядке, провели одиночные пикеты. При этом в полном соответствии с
требованиями московских властей — все пикетчики использовали маски.

Как стало известно Федеральному агентству новостей, активисты встали с
одиночными пикетами у памятника выдающемуся русскому поэту Сергею Есенину на
Тверском бульваре, рядом с Литературным институтом имени Горького.

Участники акции обратили внимание на то, что именно сейчас в Донбассе
наблюдается резкое обострение боевых действий. Военизированные формирования,
подконтрольные Киеву, обстреливают территории народных республик из тяжелых
вооружений, что приводит к жертвам как среди военных, так и среди мирного
населения. Активисты убеждены в том, что происходящее уже восьмой год является
ничем иным, как геноцидом русского населения региона.

Один из плакатов гласил: «#Стоп геноцид. 7 лет войне против Русского Донбасса.
Маски не спасают от снарядов». На другом, напоминающем о недавней трагической
гибели маленького жителя ДНР Владика Дмитриева, было написано «Равнодушие
страшнее снарядов».

По предварительным данным следствия, пятилетний ребенок погиб в результате
взрыва неустановленного устройства, которое было сброшено с украинского
беспилотника в жилой сектор поселка Александровского. Незадолго до этого
снайпер одного из военизированных формирований Украины застрелил на линии
соприкосновения пенсионера, вышедшего во двор покормить кур.

Некоторые участники пикета прочли у памятника Есенину свои предельно
пронзительные стихи, посвященные войне в Донбассе. Затем активисты направились
к посольству Украины в Леонтьевском переулке, где также провели серию одиночных
пикетов.

Дипведомство выглядело весьма неприветливо, ощетинившись металлическими
заграждениями. Гораздо более дружелюбным оказались сотрудники полиции,
дежурившие у входа на территорию посольства. Они не только не стали
препятствовать проведению пикетов, но и дали ряд ценных рекомендаций по их
проведению. В который раз московская полиция продемонстрировала свою здравую,
адекватную позицию.

«Полиция с народом!» — отметил один из пикетчиков.

«Миллионы людей в Донбассе ждут от нас поддержки»


\ifcmt
  pic https://scontent-dfw5-2.xx.fbcdn.net/v/t1.6435-9/175557560_124554126391362_5896582839098744878_n.jpg?_nc_cat=106&ccb=1-3&_nc_sid=b9115d&_nc_ohc=Sazp5Ba86n0AX8QvZdw&_nc_ht=scontent-dfw5-2.xx&oh=916a233e62d9cf45b1ade8dc82b434ac&oe=60A329B7

	pic https://scontent-dfw5-1.xx.fbcdn.net/v/t1.6435-9/176183949_124554146391360_2677939806126906024_n.jpg?_nc_cat=111&ccb=1-3&_nc_sid=b9115d&_nc_ohc=ezxHj3CUX2wAX-4qUOB&_nc_ht=scontent-dfw5-1.xx&oh=88666c3530d32b55424da9cd03cd52ac&oe=60A4B883

	pic https://scontent-dfw5-2.xx.fbcdn.net/v/t1.6435-9/175964275_124554133058028_7639807031976782781_n.jpg?_nc_cat=104&ccb=1-3&_nc_sid=b9115d&_nc_ohc=QotH8BOUN2QAX-PZwcA&_nc_ht=scontent-dfw5-2.xx&oh=9246d7abe929dd3e8192c48b89a51915&oe=60A6B99A

	pic https://scontent-dfw5-1.xx.fbcdn.net/v/t1.6435-9/176221227_124554153058026_514320623084712666_n.jpg?_nc_cat=111&ccb=1-3&_nc_sid=b9115d&_nc_ohc=XErcPaVYh4oAX-xxHgX&_nc_ht=scontent-dfw5-1.xx&oh=511ff7d4a03cfc272d052ddd1e741d61&oe=60A52447

	pic https://scontent-dfw5-2.xx.fbcdn.net/v/t1.6435-9/176183949_124554149724693_8435815666879457856_n.jpg?_nc_cat=106&ccb=1-3&_nc_sid=b9115d&_nc_ohc=WR1pp-_4FtEAX9gfISm&_nc_ht=scontent-dfw5-2.xx&oh=8b557a6cbc2136848471dd7eb61ca6c3&oe=60A3F77A

	pic https://scontent-dfw5-2.xx.fbcdn.net/v/t1.6435-9/175557560_124554126391362_5896582839098744878_n.jpg?_nc_cat=106&ccb=1-3&_nc_sid=b9115d&_nc_ohc=Sazp5Ba86n0AX8QvZdw&_nc_ht=scontent-dfw5-2.xx&oh=916a233e62d9cf45b1ade8dc82b434ac&oe=60A329B7

	pic https://scontent-dfw5-1.xx.fbcdn.net/v/t1.6435-9/176183949_124554146391360_2677939806126906024_n.jpg?_nc_cat=111&ccb=1-3&_nc_sid=b9115d&_nc_ohc=ezxHj3CUX2wAX-4qUOB&_nc_ht=scontent-dfw5-1.xx&oh=88666c3530d32b55424da9cd03cd52ac&oe=60A4B883

	pic https://scontent-dfw5-2.xx.fbcdn.net/v/t1.6435-9/175964275_124554133058028_7639807031976782781_n.jpg?_nc_cat=104&ccb=1-3&_nc_sid=b9115d&_nc_ohc=QotH8BOUN2QAX-PZwcA&_nc_ht=scontent-dfw5-2.xx&oh=9246d7abe929dd3e8192c48b89a51915&oe=60A6B99A
\fi


Литературный клуб «Словороссия» проводит подобные акции в Москве уже в третий
раз, рассказал в беседе с ФАН писатель, публицист, сопредседатель клуба,
секретарь правления Союза писателей России Алексей Полубота.

«За прошедшие 7 лет для многих граждан России война в Донбассе стала чем-то
отвлеченным, далеким, — обратил внимание писатель. — Между тем, люди, в том
числе мирные жители ДНР и ЛНР, продолжат гибнуть. А ведь это не просто наши
соотечественники. Жители Донбасса — это часть разделенного русского народа. И
быть равнодушными к их судьбе — значит быть глухим и равнодушным к судьбе
своего народа. Не случайно на наших плакатах написано «Равнодушие страшней
снарядов».

Полубота уверен, что дело не только в погибших и изувеченных на этой войне, которая ведется против всего Русского мира.

«Дело и в том, что миллионы людей в Донбассе ждут от нас поддержки и не в
последнюю очередь моральной. Каждый раз мы надеемся, что выходим на подобную
акцию в последний раз, что в Донбассе наконец наступит мир. Надеемся и сейчас»,
— заявил собеседник ФАН.

Член Союза писателей РФ и СП ДНР, общественный деятель Наталья Макеева обратила
внимание на то, что ситуация в Донбассе требует самого пристального внимания
России.

«Выход из кризиса в Новороссии зависит от России. От политической воли нашего
руководства. Каким будет этот выход — вопрос открытый. Вариантов достаточно
много. Очевидно одно — после всего, что произошло, ДНР и ЛНР больше не могут
быть в составе Украины. Это, как если бы жертва с трудом вырвалась из рук
маньяка, а ее поймали и отдали обратно ему в руки», — считает Макеева.

По ее мнению, Киев многократно демонстрировал свою недоговороспособность и на
сегодняшний день возможности переговорного процесса можно считать исчерпанными.

«Если бы руководство Украины хотело мира, оно бы пошло весной 2014 года на
федерализацию, ведь это — совершенно нормальная, принятая во всем мире
практика. Тогда не было бы ни войны с многотысячными жертвами, ни удара по
экономике, ни многого другого. Но, похоже, в настоящем центре принятия решений,
то есть явно не в Киеве, не посчитали такой сценарий для себя возможным», —
констатировала член Союза писателей.

Кандидат филологических наук, гражданка ДНР и России Ольга Блюмина убеждена:
после всего, что творил Киев эти семь лет, русский Донбасс уже не будет в
составе Украины.

«Восьмой год надежд. Восьмой год разочарований. Ожидания. Веры. Приход нового
президента Украины, чуть позже смена хозяина Белого дома, ознаменовались
активизацией военных действий против народа Донбасса. Каждая из двух последних
недель помечена смертью мирных донбасских русских. Один из них — пятилетний
ребенок. Русским, живущим в России, не все равно, и мы дали это понять у
украинского посольства. Русский Донбасс не будет частью уничтожающего его
государства», — уверена она.

Член Союза писателей России Людмила Семенова рассказала в беседе с ФАН, что
заставило ее прийти к украинскому диппредставительству.

«У меня есть твердая позиция в отношении родного мне донбасского народа, —
пояснила Семенова. — Считаю, что этот народ имеет право на самоопределение.
Людей лишают жизней, домов только потому, что они верны своим убеждениям и
хотят жить по справедливым законам. Как многодетную мать, меня особенно волнует
судьба детей Донбасса. Почему они должны жить под страхом смерти? Мне кажется,
уже пора открыть глаза на происходящее и вступиться за людей, которые в полной
мере претерпели свою долю страданий».

Также в акции принял участие член Союза писателей России, сопредседатель клуба «Словороссия» Григорий Шувалов.

Отметим, что серия одиночных пикетов также прошла в подмосковном Реутове, у
памятника выдающемуся конструктору Владимиру Челомею.

«В последние дни в Донбассе чуть ли не каждый день гибнут мирные люди, в том
числе дети. Когда убивают людей, нельзя оставаться в стороне. Нужно жестко
пресекать нарушение режима прекращения огня со стороны Украины, и неважно
какими методами — экономическими, политическими или силовыми. Украинская власть
должна знать, что ее действия не останутся безнаказанными», — заявил Алексей
Полубота.

\subsubsection{Дипломат-шпион}

Примечательно, что в тот же день, 17 апреля, украинские дипломаты оказались в
центре скандала. Федеральная служба безопасности задержала украинского консула
в Санкт-Петербурге Александра Соснюка в момент передачи ему гражданином РФ
информации закрытого характера из баз данных правоохранительных органов и ФСБ
России. Позднее ЦОС ФСБ распространил сообщение, что украинский дипломат искал
базы данных на граждан РФ на питерском рынке «Юнона».

«Интересовался жилищными прописками, по машинам, регистрационные номера. Еще
данными ФСБ и базой данных уголовных дел», — уточнили в ведомстве.

В связи с этим инцидентом в МИД РФ был вызван временный поверенный в делах
Украины Василий Покотило.  «С российской стороны было указано на недопустимость
подобного рода деятельности, несовместимой со статусом консульского сотрудника
и наносящей ущерб интересам безопасности Российской Федерации», — заявили в МИД
России.

Также Покотило было объявлено о нежелательности его пребывания на территории
России. Ему было рекомендовано покинуть пределы РФ в течение 72 часов, начиная
с понедельника, 19 апреля.

Наталья Макеева
