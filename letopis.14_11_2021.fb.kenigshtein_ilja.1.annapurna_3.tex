% vim: keymap=russian-jcukenwin
%%beginhead 
 
%%file 14_11_2021.fb.kenigshtein_ilja.1.annapurna_3
%%parent 14_11_2021
 
%%url https://www.facebook.com/kenigshtein/posts/4451508774902202
 
%%author_id kenigshtein_ilja
%%date 
 
%%tags alpinizm,annapurna.3.gora,balabanov_nikita.ukr.alpinist,gimalai,gory,nepal,ukraina
%%title Аннапурна III - Михаил Фомин, Никита Балабанов и Вячеслав Полежайко
 
%%endhead 
 
\subsection{Аннапурна III - Михаил Фомин, Никита Балабанов и Вячеслав Полежайко}
\label{sec:14_11_2021.fb.kenigshtein_ilja.1.annapurna_3}
 
\Purl{https://www.facebook.com/kenigshtein/posts/4451508774902202}
\ifcmt
 author_begin
   author_id kenigshtein_ilja
 author_end
\fi

Я хочу поделиться одной историей, которая происходит прямо сейчас, на наших
глазах. Эта история прекрасна, и она станет ещё прекрасней, если получит
огласку. 

Ранее я писал про своё восхождение на Манаслу (8,163) в конце сентября.
Несколькими днями позже, после нашего штурма, на вершину Манаслу вышли
фрирайдеры Анна Тайбор (Польша), Федерико Секки и Марко Майори (Испания). Они
достигли вершины без кислорода, и успешно спустились вниз на лыжах. Анна Тайбор
стала первой в мире женщиной, которой удалось спуститься с вершины Манаслу на
лыжах. Сейчас Анна - национальный герой Польши.

\ifcmt
  ig https://scontent-frt3-2.xx.fbcdn.net/v/t39.30808-6/257279365_4451460168240396_3720964391554369230_n.jpg?_nc_cat=103&ccb=1-5&_nc_sid=8bfeb9&_nc_ohc=wc86PkmtjKcAX8G2mZl&_nc_ht=scontent-frt3-2.xx&oh=3d82f80dd72324ef5435eb6d6db9c210&oe=619C5485
  @width 0.4
  %@wrap \parpic[r]
  @wrap \InsertBoxR{0}
\fi

Пару дней назад в Киев вернулись три украинских альпиниста - Михаил Фомин,
Никита Балабанов и Вячеслав Полежайко, которые только что поднялись на
Аннапурну III (7,555) по юго-восточному хребту. Юго-восточный хребет Аннапурны
III до этого дня считался абсолютно неприступным, представляя из себя кошмарную
стену высотой почти 3000 метров, подняться по которой до вершины пика не
удавалось никому. Ник Колтон и Тим Лич из Великобритании сделали первую попытку
в 1981 г, и развернулись на высоте 6400 м. После этого десятки сильнейших
альпинистов мира безуспешно пытались пройти этот маршрут. Но юго-восточный
хребет в течение 40 лет оставался неприступным, представляя одну из последних
«великих проблем Гималаев».

И вот эта проблема разрешилась благодаря украинским альпинистам. Сейчас про это
событие пишут многие западные профильные и непрофильные СМИ. Про Украину на
этот раз вспоминают в контексте невероятной победы мастерства и силы духа, а не
в контексте поддельных сертификатов.

Чтобы просто осознать, через что они прошли: 

Штурм длился 18 дней. 100 метров в день - столько удавалось пройти по маршруту
- настолько сложным был рельеф склона. На 12 день у них закончился запас еды,
после чего, в оставшиеся дни, ежедневный рацион составлял 1,5 батончика в
сутки. В предпоследний день у альпинистов вообще закончилась вся еда. 

Спуск изначально планировался в сторону Мананги, но они поняли, что при ветре
80-100 км/ч сил пройти 3,5 км по гребню на высоте 7300+ не хватит. Поэтому
спускались в сторону южного базового лагеря Аннапурны. Спуск длился три дня,
спускались почти наугад. На высоте 5,400 всё самое сложное осталось позади.
Вертолёт смог забрать ребят на высоте 5,000 метров. За экспедицию, в сумме на
троих, они потеряли 40 кг веса. 

Всё это я узнал благодаря рассказу Никиты и информации, которую прочёл в
профессиональных СМИ.

Ещё раз их имена: Михаил Фомин, Никита Балабанов и Вячеслав Полежайко. За всю
историю мирового альпинизма, именно украинские альпинисты в ноябре 2021 смогли
подняться на вершину Аннапурны III по юго-восточному хребту. Я этим событием
очень горжусь. 

Но также я хотел бы, чтобы этим гордились все, кому небезразлично, что
происходит с Украиной. В том числе правительство, в лице например Министерства
молоди та спорта. Государству нужно реагировать на такие события и делать это
вовремя. Можно же поздравить героев, а не делать вид, будто ничего особенного
не происходит. 

В Украине почти отсутствует экосистема поддержки спортсменов - в целом, и
высокогорных альпинистов в частности. Очевидно, что государство не может и не
должно финансировать все виды спорта, в том числе и высокогорный альпинизм. Мы
уже давно не живем в СССР. Но государство однозначно может и должно отмечать
своих героев за их выдающиеся заслуги, чествовать и награждать тех, кто вопреки
невозможному, поднимается на неприступные вершины и разворачивает на них флаг
Украины. Государство могло бы популизировать альпинизм, формируя тем самым
экосистему финансовой и моральной поддержки спортсменов - фонды, действующие
федерации, частные благотворительные организации и т.д. как это делают все
цивилизованные страны. 

Как это делает, например, Польша.

Потому что иначе подобное безразличие выглядит как концепция «ебитесь сами как
хотите», где практически любое министерство играет роль баржи с арбузами, а
каждый министр с первого дня своей каденции - сбитый лётчик. Государство
отдельно, люди отдельно. Зачем безразличием и бездействием уничтожать всё, чем
можно и нужно гордиться - с одной стороны, и судорожно искать повод для новых
смыслов единства общества - с другой? Это путь в никуда.

Но история сама по себе прекрасна. И таких историй в Украине - очень много,
нужно только увидеть. А для того чтобы увидеть, нужно просто захотеть это
сделать. 

На фото со мной - Nikita  Balabanov, человек невероятной силы духа.
