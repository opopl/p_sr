% vim: keymap=russian-jcukenwin
%%beginhead 
 
%%file 05_08_2020.news.strana_ua.beirut
%%parent 05_08_2020
 
%%endhead 
\subsection{Взрыв в Бейруте. Власти Ливана объявили разрушенный город зоной бедствия, в стране могут ввести режим ЧП}
\url{https://strana.ua/news/282531-vzryv-v-bejrute-vlasti-livana-objavili-razrushennyj-horod-zonoj-bedstvija-v-strane-mohut-vvesti-rezhim-chp.html}
  
\vspace{0.5cm}
{\ifDEBUG\small\LaTeX~section: \verb|05_08_2020.news.strana_ua.beirut| project: \verb|letopis| rootid: \verb|p_saintrussia|}
\vspace{0.5cm}

\index{Катастрофы!Взрыв в Бейруте, Ливан, 4 августа 2020}
\index{СМИ!Украина!strana.ua}

Высший совет обороны Ливана (ВСО) официально объявил столицу страны Бейрут, где
вчера, 4 августа, прогремел чудовищной мощности взрыв (погибло 78 человек -
Ред.) зоной бедствия и порекомендовал правительству объявить в столице
двухнедельное чрезвычайное положение.

"Военное ведомство выделит часть своего личного состава для выполнения
специальных задач и поручит Следственному комитету разобраться в причинах этой
катастрофы", - заявили в ВСО  местным СМИ, как передает "Интерфакс".

В заявлении ВСО говорится, что президент Ливана Мишель Аун принял решение
выделить 100 млрд ливанских фунтов из бюджета на 2020 год в виде чрезвычайных
ассигнований, а правительству рекомендовано объявить в Бейруте чрезвычайное
положение. Аун поручил выделить кредиты для больниц и выплатить компенсации
семьям погибших.

Ранее "Страна" сообщала, что груз селитры, которая уничтожила половину Бейрута
везли из Грузии в Мозамбик.
  
