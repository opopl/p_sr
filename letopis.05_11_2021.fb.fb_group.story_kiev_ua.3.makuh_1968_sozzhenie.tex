% vim: keymap=russian-jcukenwin
%%beginhead 
 
%%file 05_11_2021.fb.fb_group.story_kiev_ua.3.makuh_1968_sozzhenie
%%parent 05_11_2021
 
%%url https://www.facebook.com/groups/story.kiev.ua/posts/1790822807781179
 
%%author_id fb_group.story_kiev_ua,poluektov_igor
%%date 
 
%%tags 1968,gorod,kiev,politika,sssr,suicid,ukraina
%%title Василь Макух - самоспалення - 1968
 
%%endhead 
 
\subsection{Василь Макух - самоспалення - 1968}
\label{sec:05_11_2021.fb.fb_group.story_kiev_ua.3.makuh_1968_sozzhenie}
 
\Purl{https://www.facebook.com/groups/story.kiev.ua/posts/1790822807781179}
\ifcmt
 author_begin
   author_id fb_group.story_kiev_ua,poluektov_igor
 author_end
\fi

В цей день, п’ятого листопада 1968 року, у переддень головного більшовицького
свята, коли центр Києва був переповнений міліцією та одягненими в цивільне
КДБ-ми, звичний порядок було раптово порушено.

З Бессарабської площі, під’їзду будинку №27 на Хрещатику, вибіг живий смолоскип
- Василь Макух, повністю охоплений вогнем...

\ifcmt
  ig https://scontent-frt3-1.xx.fbcdn.net/v/t1.6435-9/253515812_4483679525072294_823171429230827593_n.jpg?_nc_cat=108&ccb=1-5&_nc_sid=825194&_nc_ohc=5OI1AZKDb-cAX90h8sx&_nc_ht=scontent-frt3-1.xx&oh=2cb01c54546d951c9739d65b15ba4d69&oe=61ACB447
  @width 0.4
  %@wrap \parpic[r]
  @wrap \InsertBoxR{0}
\fi

втамовуючи жахливий біль, швидко він біг центральною вулицею Києва у бік площі
Жовтневої революції та вигукував: 

«Геть колонізаторів!»,«Хай живе вільна Україна!»

Біг, допоки напівживим не впав на бруківку…

Від несумісних з життям опіків, не доживши кількох днів до свого 41-го дня
народження, помер. Але його вогонь палає й досі. 

Все своє життя, кожен його день він боровся за незалежну Україну, зазнав
поневірянь, а коли спільно з друзями задумав організувати гучний протест в
столиці, то побратими були на випередження заарештовані. Тож вирішив діяти сам.

Потужний імпульс до боротьби внесли події серпня 1968 року, коли військо так
званого «мирного» совка окупувало Чехословаччину та принесло борцям за її
незалежність першу «Небесну сотню».

Свій вибір Василь Макух зробив без вагань: у жовтні листами розлучився з
рідними і друзями, зібрав торбину, в якій, крім яблук, була ще наповнена
бензином пляшка і рушив на здійснення вчинку.

.

Василь запам’ятався надзвичайною добротою, чутливістю, переживав всім серцем не
лише за долю України, але і кожного українця. Всі, хто його знав, відмічали ці
людські якості.

Мав двох маленьких дітей, яких надзвичайно любив та про яких казав:

«Оля – це моя душа, а Володя – моє серце».

Кажуть, що останніми його словами були: «...мій син знатиме, що я не
мовчав...».

Справжньому герою, людині, які віддала найдорожче Україні, було не однаково,
він розумів «какая разница»(с).

.

Його рішучий протест 1968 року проти колонізації та русифікації України, проти
радянського вторгнення в Чехословаччину, став першим у новітній історії Європи
актом самопожертви проти національного гноблення.

Відразу за ним той самий шлях мученицького протесту обрали: Ян Палах
(Чехословаччина, січень 1969 року), Ілля Ріпс (Латвія, квітень 1969 року),
Ромас Каланта та його послідовники (Литва, травень 1972 року), кримський
татарин Муса Мамут (1978 рік). Подвиг Василя Макуха повторив Олекса Гірник. 21
січня 1978 року, у 60-ту річницю проголошення УНР, він спалив себе на Чернечій
горі в Каневі на могилі Тараса Шевченка в ім’я захисту української мови. Його
подвиг відзначений званням Героя України.

.

І якщо кожен з цих героїв увінчаний пам'яттю і шаною своїх народів, то
московська влада зробила все можливе, щоб про Василя Макуху в Україні забули
раз і назавжди, щоб його ім’я було стерте з історії, щоб ніхто і ніколи не
згадав про нього.

Вдову Лідію кілька місяців поспіль викликали на допити до КДБ. Але вона взагалі
нічого не знала, бо чоловік умисно все тримав в таємниці і від неї (знаючи, що
це збереже їй життя).

Жінку позбавили роботи, прирікши разом з дітьми на животіння. На Львівщині
сестру Параску наступного дня після його смерті викликали до райвідділу КДБ на
допит. Під час катувань, згадував її чоловік, «їй відбили нутрощі, вона кілька
років харкала кров’ю і згасла, мов свічка, ще зовсім молодою...».

За наказом НКВД драматичну подію – самоспалення Василя Макуха офіційно повністю
замовчали, засекретили всі докази і згадки. Листівку пам’яті, яку видали
самвидавом по його смерті, негайно було вилучено, а всіх її авторів кинуто в
табори.

Та з подачі українських дисидентів про неї відразу повідомила радіостанція
«Свобода»:

«Громадянин України, Василь Макух, протестуючи проти комуністичного
тоталітаризму, поневолення українського народу і агресії СРСР проти
Чехословаччини, здійснив у Києві акт самоспалення. Перед мужнім вчинком
українця схиляє голови світова спільнота».

Чеський дослідник, Петр Блажек з Інституту вивчення тоталітарних режимів
вказує, що мотив його вчинку був важливішим за власне життя.

"Сьогодні багатьом важливо зрозуміти: подібний спосіб протестування нам важко
сприйняти, виглядає шокуючим... Але ми не знаходимо у жодного з них психічних
розладів, як це намагалися довести комуністи. Навпаки, вони мали дуже тонке
почуття справедливості та честі, мали пояснення цим крайнім і радикальним
вчинкам", - пояснює історик Блажек.

І якщо довгі роки Василь Макух і його подвиг лишався забутим і засекреченим, то
Ян Палах – 20-річний студент філософського факультету Карлового університету у
Празі, який 16 січня 1969 року спалив себе на Вацлавській площі, став у себе на
батьківщині національним героєм. Його посмертно нагороджено орденом Томаша
Масарика І ступеня. Похорон загиблого борця переріс у політичну маніфестацію. У
демократичній Чехії на його честь споруджено пам’ятник. А на місці
самоспалення, поблизу Національного музею, ніколи не в’януть квіти. Чеський
астроном Любеш Когоутек назвав відкритий ним астероїд на честь свого
співвітчизника «834 Палах». У світі його ім’я стало символом боротьби проти
тоталітарного режиму, зі спадщиною якого й досі не покінчено в Україні.

Лише нещодавно, в 2017 році, справедливість було відновлено, на фасаді будинку
№27 на Хрещатику також встановлено меморіальну дошку пам’яті українському герою
– Василю Макуху.

\ii{05_11_2021.fb.fb_group.story_kiev_ua.3.makuh_1968_sozzhenie.cmt}
