% vim: keymap=russian-jcukenwin
%%beginhead 
 
%%file slova.obschestvo
%%parent slova
 
%%url 
 
%%author 
%%author_id 
%%author_url 
 
%%tags 
%%title 
 
%%endhead 
\chapter{Общество}
\label{sec:slova.obschestvo}

%%%cit
%%%cit_head
%%%cit_pic
%%%cit_text
А Данилов – это зеркало Зеленского. Как и Потураев. Как Третьякова, как
Ткаченко, как Резников и как остальное стадо «слуг народа» системно и
последовательно расчеловечивающие наше \emph{общество}.  Ну и чтобы для финального
контраста. «Настане день, коли можна буде пройтись недільним Донецьком,
Луганськом і Ялтою та побачити, як набережною гуляють родини з дітками,
фотографуються молодята, у парку дідусь вчить онучку кататись на велосипеді, а
поруч вуличний скрипаль неймовірно виконує мелодію Миколи Скорика. І все це
відбувається під синьо-жовтими прапорами». Это из поносика Блогера к 9 мая,
если кто не идентифицировал. Ну а пока кримці Бог вам послал водички. Ха-ха, вы
же ее путиноиды хотели? Ну и, конечно же, мы любим вас, возвращайтесь
%%%cit_comment
%%%cit_title
\citTitle{Бог Данилова - это бог расчеловечивания}, 
Игорь Лесев, strana.ua, 20.06.2021
%%%endcit

%%%cit
%%%cit_head
%%%cit_pic
%%%cit_text
Нынешнее незавидное состояние \emph{российского общества} является логичным
результатом развития по пути превосходства материального развития над духовным,
особенно бурного в последние десятилетия. Предшествующий период отказа от
любых этических ценностей ради обогащения не мог привести к иному результату,
кроме наблюдающегося огромного имущественного расслоения, разделения \emph{общества}
на страты по признаку материального благосостояния и образу жизни, выработки
специфической культуры и этики
%%%cit_comment
%%%cit_title
\citTitle{Революция Духа – единственный путь спасения России}, 
Юрий Барбашов, voskhodinfo.su, 30.06.2021
%%%endcit

%%%cit
%%%cit_head
%%%cit_pic
%%%cit_text
Но \emph{общество} не захотело услышать, не прониклось, не зажглось
перспективными идеями и по итогу не поддержало их адептов. В результате на
президентских выборах официальным лидером страны стал лицемерный жадный
олигарх, а затем его сменил слабый самовлюбленный имитатор.  А вот в пору
обретения независимости одна только прокламация Народного Руха привела к
самопроизвольному взрывному образованию \emph{общественных} ячеек по
практической поддержке новой идеологии Украины
%%%cit_comment
%%%cit_title
\citTitle{В Украине была только одна настоящая революция - в 1991-м}, 
Александр Кочетков, strana.ua, 16.07.2021
%%%endcit

%%%cit
%%%cit_head
%%%cit_pic
%%%cit_text
Дело не только в том, что они кормят армию придворных экспертов, жирующих за
счет иностранных грантов и бюджетных дотаций. \emph{Тоталитарное общество}
обречено существовать в режиме тотальной несвободы. Оно всегда стремится к
полному искоренению инакомыслия, отвечая репрессиями на последовательную
критику в адрес власти – чтобы защитить ее в условиях обострения социального
кризиса.  А диссидентов неизбежно обвиняют в работе на внешнего врага, как это
было в пятидесятые годы в Америке, во времена маккартистской истерии, когда
патриотическая общественность травила «красных агентов», разыскивая их среди
публицистов и кинозвезд Голливуда
%%%cit_comment
%%%cit_title
\citTitle{Цензоры считают пророссийской пропагандой любую критику украинских реалий / Лента соцсетей / Страна}, 
Андрей Манчук, strana.news, 27.10.2021
%%%endcit

%%%cit
%%%cit_head
%%%cit_pic
%%%cit_text
І мова тут не про якість чи \emph{суспільне} значення, а про характер,
проблеми, з якими вони стикаються (наприклад, відчуття постійної загроженості,
боротьба за виживання, непозбувна конкуренція з "великим" сусідом) та завдання,
які стоять перед ними.  У цій класифікації немає нічого принизливого, але вона,
немов правильно поставлений діагноз, дає розуміння багатьох внутрішніх процесів
нашого письменства.  До середини ХХ століття польську літературу також заносили
до цієї категорії, але після появи Мілоша, Герберта, В. Ґомбровича, Шимборської
чи Капусцінського все змінилося, і літературознавцям доводиться вигадувати інші
"парадигми".  Хто знає, може, колись таке станеться і з нами, але до того це
треба добре попойти
%%%cit_comment
%%%cit_title
\citTitle{Письменник Володимир Діброва: Шевченко залишиться одним із наріжних
каменів українства. Але його спадщина – не музейний експонат, а джерело
енергії}, Ольга Полюхович, life.pravda.com.ua, 27.10.2021
%%%endcit

%%%cit
%%%cit_head
%%%cit_pic
%%%cit_text
Пишут, что в Харькове люди собирают деньги на юридическую помощь семье убитого,
которые понимают, что в \emph{сословном обществе}, государство качественную защиту им
предоставлять не намерено. Собственно, никто из прагматиков и не рассчитывает,
в данном случае, на какое то там "законное возмездие" этой личинке человека,
убившей себе подобного.  В \emph{феодальном обществе} нет силы права, однако очень
хорошо работает право силы.)) До каких пор мажоры будут безнаказанны?
%%%cit_comment
%%%cit_title
\citTitle{В украинском феодальном обществе существует только право силы / Лента соцсетей / Страна}, 
Богдан Савруцкий, strana.news, 28.10.2021
%%%endcit

%%%cit
%%%cit_head
%%%cit_pic
%%%cit_text
— Я — Хари Сэлдон, — сказал незнакомец за секунду до того, как фотографии этого
человека, много раз виденные в различных газетах и журналах, вспыхнули в памяти
Гаала Дорника.  Психоистория... Гаал Дорник, используя математические
концепции, доказал, что психоистория является тем ответвлением математики,
которое касается реакции человеческих \emph{обществ} на стабильные социальные и
экономические стимулы... ...Из всех этих выводов следовало, что, используя точные
математические данные, можно определенным образом воздействовать на
человеческие \emph{общества}. Необходимые сведения о степени такого влияния могут быть
определены Первой Теоремой Сэлдона, которая... Кроме того, человеческое \emph{общество}
не должно знать что-либо о психоисторическом анализе, чтобы реакция данного
общества не направлялась этим знанием и не отражалась на его эволюции...  В
основе всей психоистории лежит разработка функций Сэлдона, которые выражают
отношения и определяют зависимость между личностями и социальными
экономическими силами, такими, как...
%%%cit_comment
%%%cit_title
\citTitle{Основание}, Айзек Азимов
%%%endcit

%%%cit
%%%cit_head
%%%cit_pic
%%%cit_text
Единственное, что сегодня с системным успехом получается у "фабрики грез",
работающей под куполом украинского управленческого цирка - это плодить
когнитивные диссонансы.  Обратите внимание, что коммуникация с \emph{обществом} была
системно провалена всеми президентами Украины. Не только потому, что
позиционирование в украинских политических реалиях не предполагает позиции. Но
и потому, что коммуницировать-то особо и не с кем - \emph{общества}, как реального,
взрослого, ценностно ориентированого и самоосознанного субьекта в Украине нет.
Есть отдельные талантливые, умные, реально мыслящие бизнесмены, врачи,
философы... А \emph{общества} реально мыслящего нет.  Украинцы живут и "мыслят"
аффектами, руководствуются и мотивируются ими же.  Осознанно или неосознанно
власть, которая плоть от плоти народной, управляет украинцами по лебоновским
законам толпы - через аффекты и массовый психоз
%%%cit_comment
%%%cit_title
\citTitle{По каждой проблеме украинский народ впадает в истерику и поляризуется / Лента соцсетей / Страна}, 
Владислав Михеев, strana.news, 02.11.2021
%%%endcit
