%%beginhead 
 
%%file 04_03_2023.fb.kipcharskij_viktor.mariupol.1.r_k_tomu__bulo_take_
%%parent 04_03_2023
 
%%url https://www.facebook.com/permalink.php?story_fbid=pfbid02HF4q3o27CJTApHLZBZwxuKhKqV8U73Sag1dcRppizpsAWQRTj4jjZ9snPb4DpBiwl&id=100006830107904
 
%%author_id kipcharskij_viktor.mariupol
%%date 04_03_2023
 
%%tags mariupol,mariupol.war,dnevnik,04.03.2022
%%title Рік тому  було таке:  День 9 - 4.03.22. П'ятниця
 
%%endhead 

\subsection{Рік тому  було таке:  День 9 - 4.03.22. П'ятниця}
\label{sec:04_03_2023.fb.kipcharskij_viktor.mariupol.1.r_k_tomu__bulo_take_}

\Purl{https://www.facebook.com/permalink.php?story_fbid=pfbid02HF4q3o27CJTApHLZBZwxuKhKqV8U73Sag1dcRppizpsAWQRTj4jjZ9snPb4DpBiwl&id=100006830107904}
\ifcmt
 author_begin
   author_id kipcharskij_viktor.mariupol
 author_end
\fi

Рік тому  було таке: 

День 9 - 4.03.22. П'ятниця. 

Гупає рідко і тихо - ніби здалеку. Чи то орки вгомонилися, чи то... Ліворуч за
Нептуном (плавбасейн) всю ніч ніби палала заря - до першої дня там підіймається
чорний дим.

Пішли по ліки: аптеки на Хресті (перехрестя Металургів та Блажевича) не
працюють; всі великі магазини закриті: в АТБ та Гастрономі N1 за скляними
дверима входи перекрито палетами. Ані ліків, ані телефонних карток ми не
купили.

Працювала Енергія та овочеві павільйони - зайняли чергу: я в Енергію, Оля в
овочевий. В Енергію черга підійшла раніше, тож купили що було: сир (творог),
сметану, соки Садочок. Потім повернулися в чергу по овочі, постояли ще досить
довго (люди брали всього багато): купили картоплю, бурак, яблука і мандарини.
Добре, що є гроші - картки можна сховати... Продавці казали, що на Оптовому
ринку овочі ще є. Можна було би поїхати машиною і взяти більше і дешевше, але
де ж потім взяти бензин? Заправки вже кілька днів не працюють. 

Люди в черзі казали, що орки взяли  Гнутово, Павлополь, Чермалик, Талаковку,
Сартану - села по лівий бік Кальміусу. Мабуть, слідом за людьми з тих сел та
передмістя повтікали пси: їх дуже багато і вони геть загидили тротуари. А от
котів не чути - ніби й не березень...

Сміття, звісно, не вивозять... Вітер розносять папір та пакети,   розвішує по
деревах.

Добре навантажилися: Оля пішла вперед, я шкультигав за нею. Сергій пішов нас
шукати, бо нас довго не було, а телефони мовчать. 

Настя варить рибний суп з щуч'ячих голів, що були у морозильнику.

14:30. Чорний дим з боку інтернату для глухонімих. Зібрали речі йти до підвалу.
Кавалок чорного диму над госпіталем прийняли за вертоліт - зчинився переполох:
а раптом почне стріляти? Акумулятор розрядився до 12,3 вольта - один адаптер
перестав працювати, інший ще витягує.

В квартирі відносно тепло. Оскільки майже весь день були на вулиці - стояли в
чергах, втомилися (особливо моя нога), тож поїли і ліг відпочивати. Усі
збираються до світла у нашій кімнаті: Оля грає з онуками у карти, шашки,
слухають новини...

Навздогін: на вулиці було трохи морозу; стояли майже цілий день без руху - і не
застудилися. Не пам'ятаю, але здається, що були у масках від ковіду. 

Добре, що маємо запас своїх "традиційних" ліків. Я прийняв рішення розтягнути
їх: зменшити дозування майже удвічі: дещо примати раз на добу, ламати таблетки
навпіл... Без еутіроксу я протримаюсь тижнів два-три, може місяць.

%\ii{04_03_2023.fb.kipcharskij_viktor.mariupol.1.r_k_tomu__bulo_take_.cmt}
