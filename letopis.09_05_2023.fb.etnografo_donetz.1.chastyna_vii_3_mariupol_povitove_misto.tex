%%beginhead 
 
%%file 09_05_2023.fb.etnografo_donetz.1.chastyna_vii_3_mariupol_povitove_misto
%%parent 09_05_2023
 
%%url https://www.facebook.com/etnografo.donetz/posts/pfbid0Ufx1n3Lcp57m8bcUYSWWbCtg5f4DEgmi2JZXMq1vnbepJ16ppL2ZYkQa7bFe33cfl
 
%%author_id etnografo_donetz
%%date 09_05_2023
 
%%tags 
%%title Частина VII/3 - Маріуполь повітове місто
 
%%endhead 

\subsection{Частина VII/3 - Маріуполь повітове місто}
\label{sec:09_05_2023.fb.etnografo_donetz.1.chastyna_vii_3_mariupol_povitove_misto}

\Purl{https://www.facebook.com/etnografo.donetz/posts/pfbid0Ufx1n3Lcp57m8bcUYSWWbCtg5f4DEgmi2JZXMq1vnbepJ16ppL2ZYkQa7bFe33cfl}
\ifcmt
 author_begin
   author_id etnografo_donetz
 author_end
\fi

Частина VII/3

\#КольороваДонеччина 

\#Маріуполь

Маріуполь повітове місто. Торгові площі, ринки, держустанови, навчальні
заклади, релігійні споруди, парки і т.д

Ця частина заключна про Маріуполь

Перша частина Маріуполь порт 

Друга Маріуполь металургійний 

Їх можна знайти за хежтегом наведеним вище.

Якщо Вам відомі ще кольорові листівки Маріуполя, які ви тут не побачили
додавайте їх в коментах

Зараз часто люди за допомоги штучного інтелекту розфарбовують ч/б світлини та
листівки. Не так, багато, але до 1917 створювали і кольорові листівки, їх
розфарбовували ті хто їх і друкував. Цікаво чи розфарбував би їх ШІ так само,
чи ні.
