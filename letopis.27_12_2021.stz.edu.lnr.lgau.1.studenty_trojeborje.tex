% vim: keymap=russian-jcukenwin
%%beginhead 
 
%%file 27_12_2021.stz.edu.lnr.lgau.1.studenty_trojeborje
%%parent 27_12_2021
 
%%url http://lnau.su/novosti/sportsmeny-lgau-zanyali-prizovye-mesta-v-otkrytom-turnire-po-troeboryu-klassicheskomu
 
%%author_id 
%%date 
 
%%tags 
%%title Спортсмены ЛГАУ заняли призовые места в открытом турнире по троеборью классическому
 
%%endhead 
\subsection{Спортсмены ЛГАУ заняли призовые места в открытом турнире по троеборью классическому}
\label{sec:27_12_2021.stz.edu.lnr.lgau.1.studenty_trojeborje}

\Purl{http://lnau.su/novosti/sportsmeny-lgau-zanyali-prizovye-mesta-v-otkrytom-turnire-po-troeboryu-klassicheskomu}

\ii{27_12_2021.stz.edu.lnr.lgau.1.studenty_trojeborje.pic.1}

Спортсмены кафедры физического воспитания Луганского государственного аграрного
университета (ЛГАУ) с 24 по 26 декабря приняли участие в Открытом турнире
Луганской Народной Республике (ЛНР) среди юниоров и юниорок, юношей и девушек
по троеборью классическому, где заняли две золотых, две серебряных и три
бронзовых медали.

\ii{27_12_2021.stz.edu.lnr.lgau.1.studenty_trojeborje.pic.2}

Турнир проводился в соответствии с календарным планом республиканских и
международных физкультурных мероприятия и спортивных мероприятий ЛНР на 2021
год, утвержденным приказом Министерства культуры, спорта и молодежи ЛНР.

Организаторами является Общественная организация «Федерация пауэрлифтинга
Луганщины» и Министерство культуры, спорта и молодежи ЛНР при содействии
Министерства внутренних дел (МВД) ЛНР и Государственного учреждения ЛНР
«Республиканский центр физического здоровья населения «Спорт для всех».

От команды ЛГАУ приняли участие семь спортсменов. В первый день выступили:

\begin{itemize} % {
\item Дмитрий Книжников, учащийся 10 класса, в будущем студент ЛГАУ. Он
выступал в весовой категории до 53 кг, заняв третье место и показал такие
результаты: приседания 70 кг, жим лежа 52.5 кг, становая тяга 100;

\item Данил Деревянкин, студент факультета экономики и управления АПК. Он
выступал в весовой категории до 83 кг и занял третье место с результатом:
приседания 110 кг, жим лежа 100 кг, становая тяга 130 кг;

\item Андрей Запорожцев, студент факультета ветеринарной медицины. Он выступал
в весовой категории до 93 кг и занял третье место с результатом: приседание 115
кг, жим лежа 107.5 кг, становая тяга 175 кг.
\end{itemize} % }

В первый день соревнования наши спортсмены забрали три бронзовые медали. Во
второй день команду аграрного вуза представили четыре спортсмена:

\begin{itemize} % {
		\item Данил Зачиняев, студент факультета землеустройства и кадастров. Он выступал
		в весовой категории до 93 кг и занял первое место, впервые выполнив
		норматив мастера спорта по классическому пауэрлифтингу с результатом:
		приседания 210 кг, жим лежа 175 кг, становая тяга 285. Также он установил
		рекорд ЛНР;

		\item Иван Савченко, студент инженерного факультета. Он выступал в весовой
		категории до 93 кг и занял второе место с результатом: приседания 165 кг,
		жим лежа 105 кг, становая тяга 195 кг;

		\item Никита Мынка, студент факультета пищевых технологий. Он выступал в весовой
		категории до 120 кг и занял первое место с результатом: приседания 230 кг,
		жим лежа 95 кг, становая тяга 230 кг. Также он установил рекорд в сумме
		троеборья ЛНР;

		\item Денис Дерман, студент факультета экономики и управления АПК. Он выступал в
		весовой категории до 120 кг и занял второе место с таким результатом:
		приседания 120 кг, жим лежа 115 кг, становая тяга 160 кг. Также он
		установил рекорд ЛНР по жиму лежа с результатом 115 кг.
\end{itemize} % }

Отметим, что наши спортсмены установили три рекорда Луганской Народной
Республики по классическому пауэрлифтингу, что и позволило в очередной раз
подтвердить статус сильнейшей команды среди юниоров ЛНР.

\ii{27_12_2021.stz.edu.lnr.lgau.1.studenty_trojeborje.pic.3}

\begin{zznagolos}
«Это наши первые старты по пауэрлифтингу. Команда университета вновь
доказала, что ЛГАУ одна из сильнейших команд в ЛНР. Я очень доволен ребятами и
их результатами. Для нас в эти два месяца было много всего хорошего и
результативного. В новом году будут новые цели и новые победы. Спасибо всем за
поддержку, мы будем и дальше радовать наш аграрный университет и Республику
победами. Я счастлив, что у меня такая золотая и главное воспитанная команда по
пауэрлифтингу и тяжелой атлетике», − рассказал тренер команды, чемпион Европы и
мира по пауэрлифтингу и силовому двоеборью Андрей Костин. 	
\end{zznagolos}

\ii{27_12_2021.stz.edu.lnr.lgau.1.studenty_trojeborje.pic.4}

Тренер выразил благодарность кафедре физического воспитания ЛГАУ, которая
всегда поддерживает и переживает за своих подопечных и, конечно же, аграрному
университету, за то, что открывает перед всеми студентами новые возможности.

Поздравляем тренера команды и участников с достойным результатом!

Пресс-центр университета,

фото участников мероприятия
