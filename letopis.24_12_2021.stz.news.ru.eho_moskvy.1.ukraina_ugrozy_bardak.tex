% vim: keymap=russian-jcukenwin
%%beginhead 
 
%%file 24_12_2021.stz.news.ru.eho_moskvy.1.ukraina_ugrozy_bardak
%%parent 24_12_2021
 
%%url https://echo.msk.ru/blog/amountain/2956960-echo
 
%%author_id gornyj_aleksandr.blogger.krym
%%date 
 
%%tags ukraina,rossia,infvojna
%%title Украина. С этим бардаком и угрозами надо жёстко завязывать
 
%%endhead 
\subsection{Украина. С этим бардаком и угрозами надо жёстко завязывать}
\label{sec:24_12_2021.stz.news.ru.eho_moskvy.1.ukraina_ugrozy_bardak}

\Purl{https://echo.msk.ru/blog/amountain/2956960-echo}
\ifcmt
 author_begin
   author_id gornyj_aleksandr.blogger.krym
 author_end
\fi

Нет братского народа, есть соседнее государство и не более. Я устал от этих
сказок про братский народ и с этим надо завязывать. Было достаточно времени
определиться, кто и с кем, так что не давите на жалость. Наши пути разошлись и
можно сколь угодно манипулировать фантомными болями прошлого, но мы не просто
разошлись, но и стали геополитическими противниками со своими интересами,
которые не совпадают. 

Это не значит, что надо воевать друг с другом, но с этим надо завязывать. 

Огоньку в топку добавил и турецкий МИД, который обозначил свою позицию и с кем
он будет в случае конфликта. Нет друга Эрдогана, а есть Турция, у которой есть
свои интересы и не более. Пока мы покупаем помидоры, ездим туда отдыхать, даём
им строительные контракты – нас будут терпеть и улыбаться. История с войной в
Нагорном Карабахе очень чётко показала, как с нами будет дружить Эрдоган. На
его фоне Западная толерантная Европа просто безобидный младенец. 

Очевидно, что нас готовят к возможному военному конфликту, который очень всем
нужен, но его не будет. Киев понимает, что если сунется с военной провокацией
на Донбасс или в Крым, то получит хорошего леща и мощь российской армии в
догонку. Нужна ли Киеву провокация? Всё зависит от расклада во внутренней её
политике, точнее рейтинга и популярности Зеленского. 

Путин довольно чётко обозначил красные линии, переход через которые приведёт к
непоправимому. Так что надо быть просто безумцем, чтобы их перейти.

Сколько лет я пишу, что между Россией и Украиной нужен жёсткий визовый режим, и
полный запрет иметь гражданам России вторые украинские паспорта, тем более в
Крыму. Тут засело полно страдающих по Украине на всех уровнях, так что нам и
диверсантов киевских тут не надо. Они прилюдно молятся на портреты Путина в
своих кабинетах, влезают в ЕР, а вечерами достав украинские паспорта и флаги
прижимают их к к груди, и поют украинские гимны. 

Доколе это терпеть?! Спрашивают крымчане, которые голосовали за Россию? В Крыму
много русских и именно их обошла Россия, раздавая льготы и блага тем, кто
выступал категорически против возвращения в Россию. Разве это справедливо? Надо
запретить владеть землёй всем тем, кто имеет украинский паспорт.

Какой есть выход? Он очень простой. Усиление военной группировки на границах с
Украиной, подвоз туда самого современного и мощного вооружения, чтобы в Киеве
не было никаких иллюзий, и желаний на провокации. Необходим Указ Президента,
что в случае малейшей провокации отвечать жёстко, в сотни раз жёстче, чем
провокация.

Украина должна забыть про Донбасс и Крым, которые она потеряла в результате
киевского госпереворота и решения людей вернуться домой в Россию. Все попытки
раскачивать ситуацию и накликать на Россию санкции должны расцениваться, как
военные действия против России и иметь жёсткий ответ в любой форме, в том числе
и военной. 

Что можно ещё добавить? Нечего. Впереди Новый Год и хочется, чтобы точки
жёстко были расставлены ещё в этом.  
