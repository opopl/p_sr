% vim: keymap=russian-jcukenwin
%%beginhead 
 
%%file 28_06_2021.fb.medvedeva_olesja.1.ludi
%%parent 28_06_2021
 
%%url https://www.facebook.com/olesia.medvedieva/posts/1640063029518146
 
%%author Медведева, Олеся
%%author_id medvedeva_olesja
%%author_url 
 
%%tags chelovek,medvedeva_olesja,mnenie,toshnota,ukraina
%%title Тошнит от извиняющихся, за свою позицию, за свои взгляды и мысли, людей
 
%%endhead 
 
\subsection{Тошнит от извиняющихся, за свою позицию, за свои взгляды и мысли, людей}
\label{sec:28_06_2021.fb.medvedeva_olesja.1.ludi}
\Purl{https://www.facebook.com/olesia.medvedieva/posts/1640063029518146}
\ifcmt
 author_begin
   author_id medvedeva_olesja
 author_end
\fi

Тошнит от извиняющихся, за свою позицию, за свои взгляды и мысли, людей. Когда
на них бузят в комментариях, заступаться уже даже не хочется. Их прогиб все
равно никто не оценит. Одни будут праздновать победу, другие - фыркнут за
слабость и самоцензуру. 

Кто-то может сказать, ну это он(а) чтоб работу не потерять, проблем не
создавать. Типа извинился и рассосалось. Не. Если взгляды не совпадают с
работодателем, то рано или поздно работу придётся сменить.

Часто именно в этом эмоциональном порыве,  в сторис или публикации и есть та
самая позиция, которую человек имеет и о которой он молчит, боясь не попасть в
мейнстрим и чьей-то получить публичное порицание.

Не критикуют только тех, кто ничего из себя не представляет и тех, кто ничего
путного в своей жизни не сделал. Боитесь критики? Молчите, не коммуницируйте.
Высказали позицию? Умейте отстаивать! Какой бы она не была.

Так кстати и делает агрессивное меньшинство. Кто-то прогибается под них, а
кто-то нет. Сегодня одни у власти, а завтра все поменяется.

Зачем тогда жить, быть, вообще что-либо делать, если за это потом какие-то
мудаки будут требовать от тебя извинений. 

«Тут могло бы быть видео извиняющегося, за использование русского языка
Зеленского, пост бывшего маркетолога Сильпо или сторис любой девочки в
Инстаграм, которой не нравится запрет на въезд артистов».

Такие дела

\begin{itemize}
\iusr{Александр Рыхлицкий}
100\% Согласен! Забирають себе на стену!

\iusr{Вячеслав Беленький}
100%
пост с лебезением \enquote{сильпошника} одним махом отбил желание его информационно защищать.

\iusr{Olga Golub}
Моя позиция - это мое я. Как мне отказаться от моего я?

\iusr{Буян Мария}
Леся, спасибо! Что Вы не боитесь! Глядя на Вас, тоже не так страшно.

\iusr{Alexander Olegovich}
Все очень верно. Человек имеет право на позицию или не имеет права именоваться человеком.

\iusr{Андрей Костеров}
А вообще многие стоят за свои убеждения? Я знаю таких, но их единицы.

\iusr{Александр Никифоров}
Давно пролетели мы дно,
коль главный звыняйло куйло
\end{itemize}
