% vim: keymap=russian-jcukenwin
%%beginhead 
 
%%file 24_06_2021.fb.nicoj_larisa.1.varash_festival_mova_jazyk
%%parent 24_06_2021
 
%%url https://www.facebook.com/nitsoi.larysa/posts/938801726686200
 
%%author Ницой, Лариса
%%author_id nicoj_larisa
%%author_url 
 
%%tags festival,fotograf,jazyk,mova,ukraina,ukrainizacia
%%title Вараш. Літературний фестиваль на площі
 
%%endhead 
 
\subsection{Вараш. Літературний фестиваль на площі}
\label{sec:24_06_2021.fb.nicoj_larisa.1.varash_festival_mova_jazyk}
\Purl{https://www.facebook.com/nitsoi.larysa/posts/938801726686200}
\ifcmt
 author_begin
   author_id nicoj_larisa
 author_end
\fi

\obeycr
\noindent - Вітаю. Мені сказали, що ви відома людина, і щоб я вас сфотографувала.
Вараш. Літературний фестиваль на площі. До мене підійшла жінка з фотоапаратом. Розмовляє гарною українською, але з російським акцентом. Видно, що мовою її спілкування є російська. 
- Ну, не знаю, може й відома, - сміюся я. - Якщо треба сфотографувати, фотографуйте. 
- Скажіть, а чим ви відома? - питає після кількох знімків. 
- Навіть не знаю, що Вам сказати. Це довга розмова. Мабуть легше прогуглити.
Мене звати Лариса Ніцой. Єдине що, коли ви прогуглите, то одразу вискочить
інформація про мої роги і мітлу, на якій я літаю. То, якщо вас справді
цікавить, подивіться спочатку \enquote{Люту українізацію} Антіна Мухарського з Ларисою
Ніцой і Сергія Іванова на каналі \enquote{Ісландія} з Ларисою Ніцой. А тоді відкривайте
те, що пишуть про роги і мітлу. 
Дивлюся, людина кліпає на всі мої слова про \enquote{Ісландію} і Іванова, про Антіна Мухарського і \enquote{Люту українізацію}... 
- Ви не знаєте, що це, так? - питаю в неї. 
- Ні, ніколи не чула, - дивується та. 
Наша розмова відбувається біля намету, в якому йде презентація Остапа Дроздова. Киваю на нього:
- А Дроздова знаєте?
- Нууууу, - каже фотографка, ніби вона зовсім з іншої планети. Але ж так і є. ЇЇ російськомовний акцент говорить сам за себе, вона з іншого світу.
Кажу їй.
- Дивіться. В Україні паралельно існує два культурних світи. 
- Серйозно? - дивується фотографиня.
- Так. У нас існує два паралельних культурних світи, які між собою іноді
дотикаються, а іноді й ні. Очевидно, що я з Дроздовим належимо до одного
культурного світу. А от ви, скоріш за все, належите до іншого культурного
світу, тому ви про нас нічого й не чули. 
\smallskip
Фотографиня замислюється. І я теж у цей час думаю про одних українців,
вихованих на Лєрмонтових і Толстих, Висоцьких чи Газманових - і про інших
українців, які виховані на Коцюбинських, Франках, Жаданах чи Середах. Ми живемо
в одній країні, але ніби в різних вимірах. І це так і є. 
\smallskip
Мова. Пояснення всьому. Мова формує мислення. Смаки. Свідомість. Світогляд.
Якою мовою розмовляєш, до таких письменників, співаків, фільмів і тяжієш. Вони
формують твою свідомість, світогляд, а відтак спільні смаки, спільне захоплення
чимось, спільне бачення майбутнього. 
\smallskip
Мовний поділ наших громадян формує поділ на два культурних світи. Як у нашому
випадку з цією жінкою. Наші різні мови - формують різні культурні середовища.
\smallskip
Саме тому мовній політиці інші держави (крім нашої країни) відводять одне з
ключових місць. Щоб народ, сформувавши спільний світогляд, ніби одна команда,
разом ішов до спільного майбутнього. 
\smallskip
\ifcmt
  ig https://scontent-lga3-2.xx.fbcdn.net/v/t1.6435-9/204333890_938801700019536_3518919734710036486_n.jpg?_nc_cat=104&ccb=1-3&_nc_sid=8bfeb9&_nc_ohc=XLpwNQc-XnEAX_mq-JG&_nc_ht=scontent-lga3-2.xx&oh=259aaa788890682711d413749e95944e&oe=60D9D301
\fi
\smallskip
Наша держава не дбала і не дбає про мову, а значить не дбала про формування спільного світогляду. В результаті наше суспільство - амбівалентне. Це коли його розриває навпіл від того, що в нас протилежносвітоглядні громадяни. А протилежні сівтогляди приводять до того, що ми не одна команда. Більше того, ми не маємо спільного бачення майбутнього нашої країни. Про це різне бачення нашого майбутнього нашими громадянами говорять усі соціології. Тому ми тупцяємося на місці і не можемо піти вперед, як інші нації. І лише економіка нас не врятує. Навіть якщо на возі лежать мішки золота, але при цьому у воза впряглися лебідь і щука, віз не зрушить з місця. 
Не знаю, чи запам'ятала та жінка мої поради, чи дивилася наш ютуб, дуже сподіваюся, що це буде для неї відкриттям. 
\smallskip
Бо я далі мусила піти в Гончаренко-центр на зустріч з прекрасними варашцями.
Вони прийшли гуртом з площі. Ми не розходилися майже три години. Говорили,
говорили. Ми були однодумцями з нашого спільного українського світу.  
\smallskip
А потім на вулиці мене перестрічали варашці-перехожі. 
\smallskip
- Пані Ларисо! Це ви? - і усмішка на все лице.
- Я.
- Ой, мій чоловік не повірить, що я Вас бачила! 
- То давайте сфотографуємося!
- Справді, давайте! - і непідробна радість моя і перехожих.
\smallskip
Пишу не тому, що хвалюся, от, мовляв, мене впізнали на вулиці, яка я \enquote{зірка}.
Не в тому щастя, а в спільнодумстві, і в спільнобаченні, і спільнорозумінні, і
в спільнорадості.  Яке щастя розуміти, що їхала в місто \enquote{пані Ларисо, в нас
стільки російськомовних, ну ви ж розумієте, Кузнєцовськ, все таке}, а зустріти
суцільних однодумців, чемних, щирих, радісних зі спільним баченням нашого
майбутнього.  
\smallskip
Надихаюче.
\smallskip
Дякую, Варашику!
\restorecr
\verb|#Вараш|
