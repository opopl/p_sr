% vim: keymap=russian-jcukenwin
%%beginhead 
 
%%file 12_04_2021.youtube.zharkih_ekaterina.istoria_post_jazyk
%%parent topics.mova_jazyk
 
%%url 
 
%%author 
%%author_id 
%%author_url 
 
%%tags 
%%title 
 
%%endhead 
\subsection{Киевлянка написала пост в защиту русского языка. Начался Ад! История от Екатерины Жарких}
\label{sec:12_04_2021.youtube.zharkih_ekaterina.istoria_post_jazyk}

\url{https://www.youtube.com/watch?v=Q4OnROizp4Y}

1,527 views Premiered on 12 Apr 2021

Екатерина Жарких

5.35K subscribers

Однажды журналистка Екатерина Жарких, которая родилась и выросла в Киеве,
написала в Facebook пост о том, как на своём опыте сталкивается с
дискриминацией родного русского языка. 

От квот на ТВ и отсутствия выбора на книжных полках и в кинотеатрах, до
оскорбительной таблички в ресторане в центре Киева, где русский язык
маркируется как язык некультурного быдла. Написала и... получила неожиданно
бурную реакцию. Вплоть до теории, в которой ей отводится особая роль в новом
"плане Путина-Медведчука" по "языковой провокации против Украины", сродни
"распятому мальчику" ))

О потоках ненависти и попытках запугать её, о реальных мотивах, побудивших
написать этот пост и манипуляциях в СМИ по поводу сути поста, а также о
языковой политике Украины при Зеленском — говорим с невероятно искренней и
смелой Екатериной Жарких.

\ii{12_04_2021.youtube.zharkih_ekaterina.istoria_post_jazyk.cmt}


