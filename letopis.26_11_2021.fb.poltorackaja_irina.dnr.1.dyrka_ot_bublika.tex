% vim: keymap=russian-jcukenwin
%%beginhead 
 
%%file 26_11_2021.fb.poltorackaja_irina.dnr.1.dyrka_ot_bublika
%%parent 26_11_2021
 
%%url https://www.facebook.com/permalink.php?story_fbid=1109353039812575&id=100022136689689
 
%%author_id poltorackaja_irina.dnr
%%date 
 
%%tags deti,donbass,vojna,volonter
%%title ПОМИНАЛЬНЫЙ НАБОР или ИСТОРИЯ ДЫРКИ от БУБЛИКА
 
%%endhead 
 
\subsection{ПОМИНАЛЬНЫЙ НАБОР или ИСТОРИЯ ДЫРКИ от БУБЛИКА}
\label{sec:26_11_2021.fb.poltorackaja_irina.dnr.1.dyrka_ot_bublika}
 
\Purl{https://www.facebook.com/permalink.php?story_fbid=1109353039812575&id=100022136689689}
\ifcmt
 author_begin
   author_id poltorackaja_irina.dnr
 author_end
\fi

ПОМИНАЛЬНЫЙ НАБОР или ИСТОРИЯ ДЫРКИ от БУБЛИКА.

Много лет я занималась подготовкой и выдачей новогодних подарков, вместе с вами
, мои друзья-благодетели, и вместе с Анастасия Михаиловскаия мы не раз
подготавливали новогодние подарки. Но, когда мне сказали, что дети уже какают
конфетами и подарки начинают получать задолго до Нг и прекращают в феврале, мой
пыл поостыл, а напрасно. Получают подарки от волонтеров-одиночек, от
волонтерских групп, от благотворительных фондов и народных депутатов, и т.д.
Получают все, кроме детей наших защитников. Привозила, и не раз, и деткам в
подразделения, не во все ( все охватить невозможно), детям устраивали праздник,
собирались семьями в каждой роте или бате, а вот встречая 21-й год, я оплошала,
решила, что деткам военнослужащих, будут выделены подарки. Выделили! Парни
смеются, вспоминая: "В пакете были две печенюшки, три конфетки и...бублик" -
словно с поминок принесли. В итоге, мы сами скинулись и сделали нашим детям
нормальные подарки."

\ifcmt
  ig https://scontent-frx5-1.xx.fbcdn.net/v/t39.30808-6/260024701_1109352966479249_2648315919994747440_n.jpg?_nc_cat=105&ccb=1-5&_nc_sid=730e14&_nc_ohc=LIbz8fXTDQ4AX8qT6Me&_nc_ht=scontent-frx5-1.xx&oh=09ce47e78e0b82a0cdf46410d3d1f664&oe=61A56D1E
  @width 0.4
  %@wrap \parpic[r]
  @wrap \InsertBoxR{0}
\fi

Вот как-то так. Представьте: вы работаете, получаете з/п и кроме основных
выплат, должны сдать по 2000р., оторвав их из семейного бюджета на нужны
взвода, роты, полка: тот же строительный материал и т.д. А если должность
выше рядового, то и все пять! Как вам такой расклад на каждый м-ц?! А тут ещё
и новогодние подарки! Понятное дело, каждый родитель делает подарок своим
чадам, но плюс к этому, ещё и с работы приносит, а тут - купил подарок и тот,
который с работы - сам оплатил. ну не абсурд ли?! Не понесешь же детям бублик
в качестве новогоднего подарка, объясняя тем, что, мол, папа в окопе, ему
некогда бегать по магазинам?! 

Так вот решила я в этом году возобновить традицию именно для детей
военнослужащих. Беру на себя три роты. Три роты- 120 детей, да-да, именно такая
цифра получилась (62+ 34+24). Если сделать хорошие подарки, исходя из суммы
200р. на подарок, то получается, что нам нужно собрать 24 000р. Гложут меня,
конечно, смутные сомнения, т.к. даже после публикаций Юлия Андриенко, наш сбор
на систему видеонаблюдения и прочее замер, кроме тех поступлений, о которых
писала ранее. Но это же ДЕТИ! Давайте все вместе станем добрыми Дедушками
Морозами и Снегурочками и подарим детям самых сильных и самых мужественных
парней, маленькую новогоднюю сладкую радость. Прошу писать в сообщениях к
переводу " детям".

Сбер 4276 5204 3552 1659 Николай Юрьевич М.

ЦРБ 1604 2387 5752 9569 до 02/22 Ирина Полторацкая

пэй пал, вестерн - в личку.

\ii{26_11_2021.fb.poltorackaja_irina.dnr.1.dyrka_ot_bublika.cmt}
