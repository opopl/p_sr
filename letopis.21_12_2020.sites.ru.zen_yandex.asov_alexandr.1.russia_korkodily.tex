% vim: keymap=russian-jcukenwin
%%beginhead 
 
%%file 21_12_2020.sites.ru.zen_yandex.asov_alexandr.1.russia_korkodily
%%parent 21_12_2020
 
%%url https://zen.yandex.ru/media/id/5eeaf725a3dca453cfdd4b58/krokodily-lvy-slony-i-nosorogi-drevnei-rusi-5fb7c1dc572b862575132dce
 
%%author 
%%author_id asov_alexandr
%%author_url 
 
%%tags 
%%title Крокодилы, львы, слоны и носороги... Древней Руси
 
%%endhead 
 
\subsection{Крокодилы, львы, слоны и носороги... Древней Руси}
\label{sec:21_12_2020.sites.ru.zen_yandex.asov_alexandr.1.russia_korkodily}
\Purl{https://zen.yandex.ru/media/id/5eeaf725a3dca453cfdd4b58/krokodily-lvy-slony-i-nosorogi-drevnei-rusi-5fb7c1dc572b862575132dce}
\ifcmt
	author_begin
   author_id asov_alexandr
	author_end
\fi

\begin{leftbar}
  \begingroup
    \em\Large\bfseries\color{blue}
"Коркодил — зверь водный, егда имать человека ясти, плачет и рыдает, а ясти не
перестает"...

Азбуковник XVI в.
  \endgroup
\end{leftbar}


\noindent\fcolorbox{red}{gray}{%
\begin{minipage}{\textwidth}
	\begingroup
	\em\Large\bfseries\color{white}
	Есть в наших летописях, свидетельствах путешественников по Древней Руси и
	окрестностям, описания будто бы исторических событий, к которым современные
	историки не знают как относиться. Что это? Выдумки, байки, или в самом деле
	нечто важное? Ну... как говорил премудрый Козьма Прутков: "Некоторые вещи нам
	не понятны, не потому что наши понятия слабы, но потому, что сии вещи не входят
	в круг наших понятий"...
	
	Так давайте же попробуем расширить сей "круг", дабы разобраться в таком
	неожиданном вопросе: были или нет, крокодилы и другие экзотические животные, да
	и тропические растения, в старину, в наших северных широтах, где они казалось
	бы никак не могут хорошо себя чувствовать вне теплиц и зоопарков... 
	\endgroup
\end{minipage}
}

Итак, откроем Псковскую летопись, и прочтём:

\begin{leftbar}
  \begingroup
    \em\Large\bfseries\color{blue}
«В лета 7090 (1582 от Р.Х.) Поставиша город земляной в Новгороде. Того же
лета изыдоша коркодили лютии звери из реки и путь затвориша; людей много
поядоша. И ужасошася людие и молиша Бога по всей земли. И паки спряташася,
а иних избиша. Того же году преставился царевич Иван, в слободе...».
(Полное собрание русских летописей, том 30, Псковская летопись 1582 года).
  \endgroup
\end{leftbar}

Вот дела! Это как понимать? Не Новгородчина, а какая-то "Неукротимая планета"
фантаста Гарри Гаррисона... Новгородцы даже возводят валы, как будто для того,
чтобы оградиться от неких лютых тварей, "коркодилов", кои многих "поядоша", и
только потом стрельцы их едва-едва "избиша"...

И это — в летописи, вовсе не предполагающей каких-либо измышлений и мечт... И
для летописца сие событие, хоть и заметное, и поставлено оно в ряду войн и
бедствий, типа недавнего нашествия Стефана Батория, да смерти царевича Ивана,
но в принципе — вполне рядовое... Ну, было и было... И как сему не верить?

И это вовсе не единичное свидетельство! О северных "коркодилах", или змеях с
ногами, говорят и путешественники — Сигизмунд Герберштейн, побывавший на Руси в
1532 году, а также Джером Горсей в 1589 (цитаты из их сочинений см. далее).

Да и "Сказание о Словене и Русе" из Хронографа 1679 года поминает "окоянного
волхва и чародея" Волхвова, который обращался в "лютого зверя коркодела",
который "залегаше в той реце Волхове водный путь" и "овых пожираше, овых
извергая потопляше"... И это не считая многочисленных фольклорных свидетельств
о неких ящерах, коим будто бы поклонялись и приносили жертвы древние русичи...

\ifcmt
  pic https://avatars.mds.yandex.net/get-zen_doc/1593343/pub_5fb7c1dc572b862575132dce_5fb7d2abccd7953aaac198c8/scale_1200
	caption Охота на крокодилов в устье Нила. Гравюра Яна Колларта 1586 г. Но что-то не похоже, что действие происходит в Египте: приманивают "египтяне" явно европейского облика и в крестьянских одеждах, крокодилов — на свиней (а арабы свиней не разводят, Коран запрещает); на берегу — скалы, горы и как будто ёлки... Это не низменное устье Нила, скорее Волга или Волхов...
\fi

Вообще-то, наши старые летописи просто пестрят такого рода "криптозоологией".
Встречались в наших широтах, согласно древним источникам, и далеко не только
сказочным, не одни лишь крокодилы, но и носороги, слоны...

Да-да! Россия — родина слонов! Без шуток... Почти. Жили тут и крокодилы с
носорогами!.. И даже, судя по всему, обезьяны... ("А-ах!.. И зелёный
попугай!..")

И это помимо разного рода сказочных угрищ-драконов, русалок да леших,
навий-упырей, подземной чуди белоглазой, дивов, и прочей нечисти — совсем уж
мифической... Но это пока в сторону.

\subsubsection{Палеоклиматология и крокодилы Древней Руси}

Итак, разберёмся вначале с крокодилами! Животина эта, конечно, экзотическая, но
вполне себе реальная... Только жить она как будто не может среди медведей, да
зайцев с ёжиками...

Животина эта хладнокровная, зима с морозами для неё смертельна. Крокодилы — те
вообще обитатели тропиков... Впрочем, их родня аллигаторы — те могут жить и
несколько севернее. И, кстати, они-то как раз в отличие от одиночек-крокодилов,
и нападают стаями, как это и описано в Псковской летописи.

\ifcmt
  pic https://avatars.mds.yandex.net/get-zen_doc/3915824/pub_5fb7c1dc572b862575132dce_5fb7d9fc572b8625753ea0fb/scale_1200
	caption Древнерусская "адописная" икона. Нападение адских "коркодилов"
\fi

Заметим, самое высокоширотное место обитания аллигаторов в современной Америке
— это штат Северная Каролина, и там климат такой, как у нас в Краснодарском
крае. То есть бывают и зимы со снегом, и лёд покрывает реки с озерами, но
ненадолго, на недельку-другую. Собственно, и аллигаторы в это время
вмораживаются в лёд и впадают в спячку, только нос оставляют надо льдом, чтобы
чуть-чуть дышать. Но более пары недель в сём состоянии анабиоза они протянуть
не могут... Да и их тогда легко изловить, сцапать за нос, — хищникам, как,
кстати, и людям... Предполагают, что кое-кто из них роет пещеры с подачей
воздуха, будто медведи...

Да и наши крокодиловые фермеры, что под Анапой и на Азове, подтверждают — могут
аллигаторы перезимовать и у нас, на юге России. Как-то даже один из аллигаторов
сбежал и перезимовал в Кубани, и его едва отловили потом... 


\ifcmt
  pic https://avatars.mds.yandex.net/get-zen_doc/1918125/pub_5fb7c1dc572b862575132dce_5fb7dbe593538c30396e13ac/scale_1200
	caption Аллигатор, вмороженный в лёд в Северной Каролине
\fi

Однако, Новгородчина — это вовсе не субтропики, не юг, где ещё кое-как
аллигаторы выживают на воле, и то не всякую зиму. Но было ли так всегда? Что о
том говорят палеоклиматологи?.. А говорят они следующее.

До 14—17 веков на Европейской равнине около двух тысячелетий наблюдался т.н.
"климатический оптимум", сменившийся затем Малым Ледниковым периодом, в коем мы
живём поныне в ватниках и валенках. Обусловлено сие было изменением течения
Гольфстрима. А почему он так периодически виляет, есть масса интереснейших
теорий, кои мы здесь затрагивать не будем (ну, просто недосуг)..

И кстати, те же палеоклиматологи нередко в подтверждение своих теорий как раз и
ссылаются на летописи и хроники. К примеру, неожиданное похолодание и неурожаи,
вызвавшие голод в Московском царстве в начале 17-го века, по их мнению и
послужило истинной причиной Смуты... А до того климат был гораздо более тёплым
и, судя по всему, даже у нас, в Средней полосе, по сути субтропическим.

Известно, что тогда виноградарство и виноделие процветало в Северной Европе, в
том числе и на Руси. К примеру, в "Слове о погибели Земли Русской" говорится:
"О светло светлая и украсно украшена Земля Русская! И многими красотами
удивлена еси... (и далее: реками... горами.... холмами... ) и (внимание!)
винограды обительны..."

То есть у нас, на севере, монастыри в 12 веке, разводили виноград и производили
монастырское вино. Не случайно же в фольклоре у нас "колядки" сменяются
"виноградьями", а былинные богатыри из бочек то и дело зачёрпывают зелено вино,
поят им голь перекатную, и не похоже, чтоб оно было импортируемым из тёплых
стран (тогда это дорогое удовольствие было б доступно лишь князю да боярам — и
маленькими кубками, а не ковшами)...

Заметим также, страной винограда — Виндланд викинги, большие любители вина,
назвали тогда Север Канады. Они же назвали и Гренландию — ведь это имя в
переводе значит "Зелёная земля", хотя сейчас она покрыта ледником.

То есть до 17-го, и тем паче до 14-го века, когда температура, в целом, была
где-то на 6 градусов выше современной, климат на Руси вполне подходил как для
тропических, так и субтропических животных, да и растений. И их мы в самом деле
встречаем в древних источниках предостаточно.

Кроме винограда, если судить по ведическому имени, у нас в старину разводили и
батат (ведь старорусское его имя "ямс" распространено по всему свету, от
Америки до Африки, см. об этом здесь\Furl{https://zen.yandex.ru/media/id/5eeaf725a3dca453cfdd4b58/sekrety-dolgoletiia-russkih-volhvov-molodilnye-iabloki-5f8b1a604ab7c3765a56a598?integration=site_desktop&place=layout}). Также на Северном Кавказе, то есть в
древней Русколани и Пятигорье, разводили арбузы и гранаты, гранатовые сады там
вырубались гуннами в 4-м веке (согласно "Ярилиной книге").  

\subsubsection{\enquote{Экзоты} Древней Руси}

Однако вернёмся к "экзотическим животным" Древней Руси и вообще Севера Европы.
Какие кроме крокодилов-ящеров, там встречались? Обратимся к источникам. Итак...

\paragraph{1) Древнерусские крокодилы.}

\ifcmt
  pic https://avatars.mds.yandex.net/get-zen_doc/3822405/pub_5fb7c1dc572b862575132dce_5fb7de389d2ffe38ee0d3a5f/scale_1200
	caption «Змий велик», похожий на крокодила. Миниатюра «Лицевого летописного свода», XVI в
	width 0.5
\fi

Вначале опять обратим внимание на уже упомянутых крокодилов ("коркодилов"),
которые всегда производили большое впечатление. Вот свидетельство агента
Английской торговой компании Джерома Горсея за 1589 год (тогда он ехал по своим
делам из Польши в Россию): 

\begin{leftbar}
  \begingroup
    \em\large\bfseries\color{blue}
«Я выехал из Варшавы вечером, переехал через реку, где на берегу лежал ядовитый
мертвый крокодил, которому мои люди разорвали брюхо копьями. При этом
распространилось такое зловоние, что я был им отравлен и пролежал больной в
ближайшей деревне, где встретил такое сочувствие и христианскую помощь, что
чудесно поправился...»
  \endgroup
\end{leftbar}

\paragraph{2) Саламандры.}

А вот есть свидетельство Сигизмунда Герберштейна... Его обычно также относят к
свидетельствам о северных крокодилах, но на мой взгляд здесь речь идёт скорее
об исполинских саламандрах... Уж если бы Герберштейн, образованнейший человек
своего времени, увидел бы крокодила, он бы так его и назвал...

\begin{leftbar}
  \begingroup
    \em\large\bfseries\color{blue}
				«Эта область (Жемайтия) изобилует рощами и лесами, в которых иногда
				можно видеть привидений… Там и поныне очень много идолопоклонников,
				которые кормят в своих домах (вроде пенатов) неких змей на четырех
				(коротких) лапках, напоминающих ящериц, с чёрным жирным телом, не более
				трех пядей в длину; называются они giwoites.

		В установленные дни они производят в домах очищающие обряды и, когда змеи
				выползают к поставленной пище, всем семейством со страхом поклоняются
				им до тех пор, пока те, насытившись, не вернутся на свое место. Если с
				ними случается какое-либо несчастье, они приписывают это тому, что
				плохо кормили и принимали домашнее божество (змею)».
  \endgroup
\end{leftbar}

Исполинская саламандра — это земноводное до полутора-двух метров в длину, чему
приведённое выше описание вполне соответствует... Имеет прописку саламандра
ныне только в Китае и Японии... Но это официально.


\ifcmt
  pic https://avatars.mds.yandex.net/get-zen_doc/3768331/pub_5fb7c1dc572b862575132dce_5fb7e98accd7953aaae8820e/scale_1200
	width 0.4
	caption "Держу" саламандру — пакость какая... (Фотошоп, разумеется).
\fi

Подобно тритонам и лягушкам, саламандра вполне может пережить в спячке и самую
лютую северную зиму, закопавшись в донный ил.

Так что саламандры могли бы жить и у нас... И их скелеты (в отличие от
крокодильих), как раз и находили в Европе, так что они точно жили у нас
недавно... Да только их истребили разные"лыцари", добивавшиеся благосклонности
своих "прынцесс" и "королевичн"...

Но, возможно, где-то в глухих углах они и доселе живут. Не исключено, что и у
нас на Волге (родом я с Нижегородчины). По крайней мере от своей бабушки Любы я
слышал былички о "саламандёрах", местных драконах (и, кстати, сделал их героями
своей детской книжки "Приключения Ярослава Буслаева").

\paragraph{3) Львы под Черниговом.}

И в русском фольклоре, и на стенах древнерусских храмов лев встречается тоже.
Обычно считается, что именно об охоте на льва вспоминал князь Владимир Мономах
(1112-1125) в своём "Поучении". А принимают его за льва, поскольку львиные
охоты тогда почитались престижной королевской забавой в Европе и на Кавказе.

Вот цитата из "Поучения" Мономаха, где тот, рассказывая об охоте, описывает
помимо прочего, как некий "лютый зверь" вскочил к нему на бёдра и вместе с
конём опрокинул.

\begin{leftbar}
  \begingroup
    \em\large\bfseries\color{blue}
"(...) А вот что я в Чернигове делал: коней диких своими руками связал я в
пущах десять и двадцать, живых коней, помимо того, что, разъезжая по
равнине, ловил своими руками тех же коней диких. Два тура метали меня
рогами вместе с конем, олень меня один бодал, а из двух лосей один ногами
топтал, другой рогами бодал; вепрь у меня на бедре меч оторвал, медведь мне
у колена потник укусил, лютый зверь вскочил ко мне на бедра и коня со мною
опрокинул. И Бог сохранил меня невредимым. И с коня много падал, голову
себе дважды разбивал и руки и ноги свои повреждал — в юности своей
повреждал, не дорожа жизнью своею, не щадя головы своей"
  \endgroup
\end{leftbar}

\ifcmt
  pic https://avatars.mds.yandex.net/get-zen_doc/1895332/pub_5fb7c1dc572b862575132dce_5fb7f65393538c3039990932/scale_1200
	caption Лев. Миниатюра на русской шкатулке. 15 в.
	width 0.4
\fi

Вспомним, львы некогда заселяли всю Евразию — от Туманного Альбиона до Колымы,
переселялись также и в Северную Америку (речь о пещерном и американском львах).
Эти ископаемые животные вымерли совсем недавно — около 10 000 лет назад.

А в Южной Европе, так называемый европейский лев обитал и в исторические
времена. Он был известен в Греции и на Балканах (вспомним немейского льва), на
Карпатах, на Кавказе, в Малой Азии, на Аппенинах и на Пиренеях. Ну, а на Руси,
возможно, одного из последних львов истребил на охоте как раз Владимир
Мономах... И было это в саванне... под Черниговом.

\paragraph{4) Волжский Носорог.}

Да-да, есть свидетельство и о древнерусском (или булгарском) носороге,
обитавшем на Волге в 9-10-м веках.

Как известно, в 921–922 халиф аль-Муктадир отправил посольство в Волжскую
Булгарию. Секретарём сего посольства и был прославившийся затем своими путевыми
заметками Ибн Фадлан. И среди прочего он рассказал нам и об охоте булгар на
носорогов...

\begin{leftbar}
  \begingroup
    \em\large\bfseries\color{blue}
				Недалеко... широкая степь, о которой передают, что в ней есть животное
				меньшее, чем верблюд, по величине, но выше быка. Голова его, это голова
				барашка, а хвост его — хвост быка, тело его — тело мула, копыта его
				подобны копытам быка. У него посередине головы один рог толстый
				круглый; по мере того, как он возвышается (приближается к кончику) он
				становится всё тоньше, пока не сделается подобным наконечнику копья.
				И... я видел у (буларского) царя три больших миски, похожих на
				йеменские (раковины) "джаз'", о которых(мисках) он мне сообщил, что они
				сделаны из основания рога этого животного. И сообщают некоторые
				(кое-кто) из жителей (этой) страны, что это (животное) носорог.
  \endgroup
\end{leftbar}

\ifcmt
  pic https://avatars.mds.yandex.net/get-zen_doc/3892121/pub_5fb7c1dc572b862575132dce_5fb802f5ccd7953aaa100627/scale_1200
	caption Носорог. Гравюра А. Дюрера. 
\fi

Носорога у нас, кстати, мало знали и часто путали с мифическим единорогом, и
ставили в ряд с "водными конями", пасущимися на дне озёр и рек..

\ifcmt
  pic https://avatars.mds.yandex.net/get-zen_doc/3822405/pub_5fb7c1dc572b862575132dce_5fb885449d2ffe38eead02a3/scale_1200
	caption Единорожец и водный конь. Из Бестриария.
	width 0.5
\fi

\paragraph{5) Обезьяны в русских лесах.}

Судя по всему, упоминаются в наших древних источниках и обезьяны. Кем кроме как
обезьяною мог быть Лесной Див, который "кличетъ връху древа", предупреждая
половцев? Иногда же сей зверь спускается с деревьев "уже връжеся Дивъ на
землю...". Это согласно "Слову о полку Игореве".

А в "Велесовой книге" поминаются лесовики (лесичи) под Киевом (самостийные
бандерлоги?). 

\begin{leftbar}
  \begingroup
    \em\large\bfseries\color{blue}
		"И около него мы поселились в лесах дубовых. И там лесовики на ветвях
				колыхаться начали, а бороды у них хмелем утыканы, и волосы в травах. И
				листами зелёными и водорослями, которые в низинах и болотах собирались,
				они истыкивали бороды..." (Велесова книга. Род: 1:2:10)
  \endgroup
\end{leftbar}

\ifcmt
  pic https://avatars.mds.yandex.net/get-zen_doc/3930378/pub_5fb7c1dc572b862575132dce_5fb806e593538c3039b24313/scale_1200
	caption Макаки. Старая литография.
	width 0.5
\fi

Полагаю, обезъян в старину считали отдельным лесным народом, лешими, у коих был
и собственный "царь", и, судя по сербскому эпосу, его имя было Святибор...

И вообще-то, сии звери были известны также по всей Южной Европе. Вот к примеру,
"чудо лесное, поймано весною"... в Гишпании, судя по русским лубкам.


\ifcmt
  pic https://avatars.mds.yandex.net/get-zen_doc/2783222/pub_5fb7c1dc572b862575132dce_5fb8922793538c3039162a11/scale_1200
	caption "Чудо лесное" и "чудо морское". Русский лубок.
	width 0.5
\fi

А позднее, когда они стали совсем уж сказочными персонажами, их, похоже, путали
и с лесными оборотнями — колдунами-коркодилами, коих изображали как бородатых
орангутангов...


\ifcmt
  pic https://avatars.mds.yandex.net/get-zen_doc/3808423/pub_5fb7c1dc572b862575132dce_5fc49be94c127965dba81526/scale_1200
	caption Русский лубок. Баба-яга едет на свинье с коркодилом сражаться.... На мой взгляд, сей бородатый коркодил более похож на лешего-орангутанга....
\fi

\paragraph{6) Россия — родина слонов!}

И, разумеется, среди всех сих "экзотов" не мог затеряться и слон... Не мышь
всё-таки! Кстати, слон — это сугубо русское, или даже древнерусское, имя
(отсюда "слоняться", "прислоняться", "заслон"). И выходит, давненько были
известны слоны на Руси...

Упомянут он и в "Ярилиной книге", однако уже как царское животное, наряду с
такими "чудесами", как летающий Единорог и говорящая птица Гамаюн. 

\begin{leftbar}
  \begingroup
    \em\large\bfseries\color{blue}
"И в это время, как бы на поклон царю, выступал слон, — животное, не виданное в
полночных странах... Возвратясь же в свои края, волхвы рассказывали о нём, и не
верили им, будто есть такой зверь. Ведь слон — велик как гора, а ноги его
толсты, как брёвна, и носят они тело его... И со лба его, будто полоз горный,
свисает хобот. И им слон то машет направо и налево, то поднимает его вверх, то
обвивает им себя, а когда выпускает воздух чрез ноздри, то трубит и мотает
головой. И тем подобно бегемоту, чудищу древних лет, приводит всех в трепет..."
("Ярилина книга", Ягорий 2:9)
  \endgroup
\end{leftbar}

То есть для русских в IV-м веке ( да и потом в XII веке, когда слона поместили
на известном барельефе в Юрьев-Польском) он уже был "экзотом". 

\ifcmt
  pic https://avatars.mds.yandex.net/get-zen_doc/1589334/pub_5fb7c1dc572b862575132dce_5fb809369d2ffe38ee51504c/scale_1200
	caption Барельеф слона на Георгиевском соборе в Юрьев-Польском. XII в.
	width 0.5
\fi

И, кстати, ноги слону на барельефе Георгиевского собора в Юрьев-Польском
приделали от грифона, то есть изображали его явно не с натуры. Но похоже!

Выходит, слоны в лесах Владимирщины, да и юга Руси, меж берёз и сосен не
бродили, и травку там не щипали. Их, скорее всего, только поставляли в царские
зверинцы... И на том спасибо!

Однако слонов древние русы всё же знали, ведь когда-то и сами русы-арии пришли
с юга, из Индостана, с Пенжа-Пенджаба (как утверждают дощечки "Велесовой
книги").

\paragraph{7) Аах!... И зелёный попугай!}

Ага! И попугаи тоже у нас водились! Причём явно на воле. Ведь попугаи
изображались и на росписях древнерусских коробьев, наряду со сказочными
полканами-кентаврами...

И для того, чтобы встретиться с попугаем, вовсе не нужно было "долго-долго...
ехать, топать и бежать" по тропинке, да по дорожке, как поётся в детской
песенке...

На попугаев, летающих на воле, охотились и у нас, в Великом Устюге. И об этом
хорошо помнили ещё в XVII веке...


\ifcmt
  pic https://avatars.mds.yandex.net/get-zen_doc/3531468/pub_5fb7c1dc572b862575132dce_5fc4965a63d5740415d31767/scale_1200
	caption Роспись коробьи. Полкан и попугай. Великий Устюг. XVII в.
\fi

Вот такая она — древнерусская криптозоология!

\subsubsection{PostScriptum}

Подведём итог. Ряд свидетельств подтверждают, что многие ныне экзотические
животные на Руси встречались и на воле.

К таким я отношу львов и саламандр, хотя бы потому, что их скелеты европейским
палео-зоологам известны. Носорог на Волгу тоже вполне мог забрести, например,
из Афганистана, где он и сейчас встречается. Хотя это мог быть и реликтовый
шерстистый носорог, известный по древним доисторическим останкам, находимым на
Волге.

Обезьяны-дивы? Почему бы и нет, но вопрос по-прежнему дискуссионный, ибо
останков их пока не найдено. А в древнерусской литературе можно найти много
чего, от пардусов-гепардов, до водяных с русалками.

Что касается слонов и крокодилов, есть теория, что они просто поставлялись из
Персии боярам или в царские и княжеские зверинцы. Крокодилы-де могли и сбежать
по пути... Правда, это не объясняет повсеместное почитание "ящеров" и очень
реалистичные их изображения в древнерусском искусстве.

Впрочем, сие может быть наследием гораздо более древних времён... 

