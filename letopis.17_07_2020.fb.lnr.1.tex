% vim: keymap=russian-jcukenwin
%%beginhead 
 
%%file 17_07_2020.fb.lnr.1
%%parent 17_07_2020
 
%%endhead 
\subsection{По-русски в Донбассе говорят только собаки и агрессор, --- Леонид Емец}
\label{sec:17_07_2020.fb.lnr.1}
\url{https://www.facebook.com/groups/LNRGUMO/permalink/2852158871562335/}
  
\vspace{0.5cm}
{\ifDEBUG\small\LaTeX~section: \verb|17_07_2020.fb.lnr.1| project: \verb|letopis| rootid: \verb|p_saintrussia|\fi}
\vspace{0.5cm}

На русском языке разговаривает агрессор в Донбассе. Об этом заявил экс-депутат
украинского парламента от фракции “Народный фронт” Леонид Емец, сообщает
корреспондент издания "Донбасс сегодня".

Он отметил, что внесение законопроекта в Верховную раду об отсрочке перевода
школьного образования исключительно на украинский язык является открытым
преступлением.

По версии Емца, представители Владимира Зеленского в украинском парламенте
решили защищать русский язык, пока на фронте с Россией продолжают гибнуть
украинские солдаты.

«Сегодня для меня это не язык Лермонтова, Булгакова и Высоцкого. Это язык
животных, способных расстрелять медика на поле боя. Язык агрессора, который
вторгся на нашу землю убивать наших украинских людей», --- заявил Емец.

Экс-депутат подчеркнул, что новый законопроект теперь сплотит всех украинцев,
которым не всё равно и “они встанут на защиту Украины”.

Законопроект, представленный украинскими депутатами на рассмотрение всего
парламента предлагает перенести тотальную украинизацию учебников и школьной
программы на 2023 год, поскольку в данный момент у страны нет денег на быстрое
перевооружение преподавателей новыми учебными планами и литературой на
украинском языке.

В настоящий момент страна может перейти на новый стандарт школьного образования
с запретом русского языка уже 1 сентября 2020 года. 
