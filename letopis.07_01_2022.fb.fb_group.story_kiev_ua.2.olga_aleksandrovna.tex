% vim: keymap=russian-jcukenwin
%%beginhead 
 
%%file 07_01_2022.fb.fb_group.story_kiev_ua.2.olga_aleksandrovna
%%parent 07_01_2022
 
%%url https://www.facebook.com/groups/story.kiev.ua/posts/1835394263324033
 
%%author_id fb_group.story_kiev_ua,kirkevich_viktor.kiev
%%date 
 
%%tags istoria,kiev
%%title Рождественский рассказ о замечательной женщине
 
%%endhead 
 
\subsection{Рождественский рассказ о замечательной женщине}
\label{sec:07_01_2022.fb.fb_group.story_kiev_ua.2.olga_aleksandrovna}
 
\Purl{https://www.facebook.com/groups/story.kiev.ua/posts/1835394263324033}
\ifcmt
 author_begin
   author_id fb_group.story_kiev_ua,kirkevich_viktor.kiev
 author_end
\fi

Рождественский рассказ о замечательной женщине. На репродуцированной картине
изображена Рождественская ночь. Картина была написана Ольгой Александровной в
1930 году в Дании. Она светится!

Многие годы, собирая открытки, обратил внимание на почтовые карточки с
сюжетами, преимущественно на военные темы. На хорошо выполненном рисунке в
нижнем углу были инициалы: «О.А.» Это были рисунки Ольги Александровны
Романовой - родной сестры Николая II. Мало кто знает, что у нее с Киевом
связано многое и в личной жизни, и в деятельности на благо раненным и больным
воинам.

\ii{07_01_2022.fb.fb_group.story_kiev_ua.2.olga_aleksandrovna.pic.1}

Отца она потеряла, когда ей было всего 12 лет. Но её мать помогла развить
способности и найти свое место в жизни, как ни странно, довольно трудной. Ольга
Александровна никогда не была «принцессой на горошине». В юные годы была
некрасивой, в отличие от своей сестры Ксении. Но Ольга обладала твердостью
характера и целеустремленностью. От отца ей досталась та особенность, которая в
народе называется нелюдимостью, то, что никак не подходит великой княгине,
нежелание быть на сборищах, а больше времени проводить дома, в кругу семьи,
играя на скрипке и рисуя. Предпочитая животных придворным, была ловкой и
подвижной, разрабатывая гибкость в спорте, а не на балах. 

Но пришло время, когда «сорвиголове» и императорской сестре пора выходить
замуж. Её первый муж Петр - сын принца Александра Петровича Ольденбургского и
герцогини Евгении Максимилиановны Лехтенбергской. Это был очень симпатичный и
мягкий человек, взявший от своего отца, генерала, командира Гвардейского
корпуса, лучшие качества, присущие отечественному офицерскому корпусу старой
формации. Но свою брачную ночь жених провел за игорным столом с
приятелями-офицерами, к которым испытывал не только платоническо-дружеские
чувства. «За пятнадцать лет нашего брака принц Ольденбургский и я ни разу не
состояли в супружеских отношениях!» - признавалась Ольга полвека спустя. Муж
быстро проиграл миллион рублей своей супруги, и оставил её в покое. 

Она по-прежнему проводила свое время в прогулках
с собаками, рисовала и играла на скрипке. Так было до апреля 1903 г., пока брат
Михаил не взял ее на парад, где ей на глаза попался молодой гвардеец-кирасир.
Тут младшую сестру государя охватила страстная любовь на всю жизнь. 22-летняя
княгиня заставила своего брата представить её избраннику, после чего у них
возникло всепоглощающее взаимное чувство. Она, проявив истинно твердый
характер, незамедлительно нашла своего вялого и безвольного принца в домашней
библиотеке, объявила, что любит другого и хочет развода. Принц Пётр, который
при всех своих специфических наклонностях ничуть не удивился, не запаниковал, а
предложил ей семь лет на размышление. 

\ii{07_01_2022.fb.fb_group.story_kiev_ua.2.olga_aleksandrovna.pic.2}

Далее всё произошло, как в модном
телесериале, ибо вскоре принц назначил молодого капитана кирасиров Николая
Куликовского своим личным адъютантом с повелением поселиться в его доме на
Сергиевской улице в Санкт-Петербурге. Таким образом, молодые влюбленные могли
жить под одной крышей, а принц старался не мешать их сердечным делам, взяв с
них обещание, держать всё в тайне. Это был поразительный menage a trois.
Интересно, как он действовал на практике? В конце 1950-х годов
восьмидесятилетняя и крайне набожная Ольга Александровна утверждала - между ней
и Николаем, который был на год её старше, ничего такого не было до самого 1916
г., когда они смогли обвенчаться. Зная силу характера и волю великой княгини,
можем надеяться, что и в данном случае память ей не изменила.

Свою незаурядность Ольга Александровна проявила и на войне. Когда Куликовского
отправили на фронт, она пришла к мужу, и заявила об направлении на войну
сестрой милосердия, и то, что не вернется к нему. Так великая княгиня вместе с
Ахтырским полком, шефом которого являлась, уехала на фронт. Многие из
представительниц Дома Романовых надели форму сестер милосердия, и, лишь
некоторые, только для позирования для фотографов. Большинство проявили в этой
роли активную деятельность, особенно, сестра императора. Её самоотверженная
служба не прекращалась даже под обстрелом, что очень тревожило брата и мать.

Княгиня Ольга была награждена Георгиевской медалью, которой не придала никакого
значения, уверенная - ничего героического не совершала. Лишь настойчивые
уговоры её однополчан, что таким образом награжден весь Ахтырский полк,
заставили её надеть медаль. 

Всю жизнь она следовала девизу: «Быть, а не
казаться!», и в первые месяцы войны находилась в лазарете Гусарского Ахтырского
Е.И.В.Великой Княгини Ольги Александровны полка. Часть сражалась неподалеку от
Ровно. Она работала рядовой медсестрой. С утра до вечера проводила в госпитале,
ухаживала за ранеными, перевязывала их, писала письма неграмотным, читала
письма ослепшим, не преминув одобрить всех - хорошим ласковым словом. Она не
чуралась самой неблагодарной, грязной работы, и солдаты не верили, что эта
обаятельная, стройная, молодая женщина - родная сестра императора.

Госпиталь, где работала Ольга Александровна, по случаю отступления из Галиции,
был переведен в Киев. Отношение к императорской семье ухудшилось. Великая
княгиня вспоминала: «Вскоре я заметила, что многие доктора и сестры избегают
смотреть на меня. Среди солдат ослабла дисциплина, все принялись горячо
обсуждать вопросы политики. Дальше – больше. Однажды мне едва не размножили
голову. Как-то вечером мы работали с одной сестрой в аптечном отделении. Не
знаю, что заставила меня повернуть голову, но я вдруг увидела, что эта женщина
со сверкающими глазами, и с искаженным ртом, замахнулась на меня огромной
банкой с вазелином. Я вскрикнула, она уронила банку и выбежала на улицу.
Медсестру отправили в монастырь». Скорее всего, местом ссылки
горе-покусительницы на княгиню, был Покровский монастырь. Спустя несколько дней
к Ольге в киевский госпиталь приехал её Куликовский, её «Кукушкин», как она
любила его называть. Когда его отпуск закончился, Ольга Александровна
отправилась в Петроград уладить свою личную жизнь. Она и не догадывалось, что
это её последнее посещение родного города. Всей прислуге дома на Сергеевской
уплатила годовое жалование, так как муж все деньги продолжал проигрывать в
карты.

Княгиня Ольга вернулась в Киев, в котором близость фронта сказывалась на
настроении населения. Малейшему событию, казалось, будничной жизни, придавали
политический оттенок. Большую радость принесло Ольге Александровне сообщение от
матери, что она продает свой Аничкин дворец и переезжает в Киев. Теперь, по
приезду Марии Федоровны, дочь каждый день смогла обедать у матери во дворце,
радуясь этой незначительной передышке от изнурительной службы в госпитале. 

Сама
Ольга Александровна писала Марии Николаевне, дочери Николая II: «Совсем времени
нет выходить на воздух, вчера 8 часов перевязывала, а третьего дня 10 1/2 часов
работали и только наскоро проглатывали свою еду в неурочные часы. Я люблю,
когда много работы». Это подтверждает и её мать, в письме от 16.07.1916 царю из
Киева: «Ольга так занята, что я её даже мало вижу, у нее 400 раненых солдат и
около 28 офицеров». Мария Федоровна в каждом послании сообщала о его сестре.
В.к. Александра Михайловича его служебные обязанности привели на Украину. Он
жил в собственном поезде, стоявшем на путях недалеко от киевского вокзала. В
его вагоне имелась ванна, и сестра императора время от времени пользовалась ею.
Нехватка топлива в Киеве привела к тому, что горячей воды не было даже в
госпитале.

Николай ІІ не упускал возможности приехать в Киев и навестить мать и сестру.
Сам государь оставил в дневнике следующую запись: «28 октября. Пятница. В 10
1/2 прибыли в Киев. Дорогая Мама встретила на станции. С Алексеем поехал в
Софийский собор, а затем во дворец. Посидел с Мама. Завтракали втроем. Произвел
юнкеров 5 Киевской школы в прапорщики во дворе. Поехали с Мама к Ольге в ее
лазарет. Она поправляется после горловой болезни. После чая у Мама вернулись в
поезд. В 8 обедал с Мама и оставался с ней до 11 1/2 ч.».

Как и во всех дневниковых записях «Помазанника Божьего», присутствует только
конкретная фиксация событий, даже без малейшей попытки дать оценку или
характеристику события. Этим она не особенно отличается от записей юного
цесаревича. Поэтому, ни слова не было сказано о главной причине данного приезда
в Киев. Дело в том, что его сестра Ольга, великая княгиня, совершила
экстраординарный поступок, добилась развода, хотя её супруг и не пытался
склеивать свой неудавшийся брак. Существование бывших супругов вдали друг от
друга было довольно продолжительным. Поэтому сестра императора имела намерение
вступить в брак, но с обычным дворянином, не из высшей знати, что само по себе
грозило грандиозным скандалом. 

Члены императорской семьи, а в первую очередь Александра Федоровна, были против
развода, и тем более, морганатического брака.  Она недолюбливала Ольгу и
стремилась лишний раз упрекнуть венценосного супруга за его родню и писала
Николаю II в Ставку: «Я все понимаю и не упрекаю ее за стремление, прежде всего
к свободе, а затем к счастью, но она вынуждает тебя идти против законов семьи,
- когда это касается самых близких, это еще больнее.  Она - дочь и сестра!
Перед всей страной, в такое время, когда династия переживает такие тяжелые
испытания и борется против революционных течений, - это грустно. Общество
нравственно распадается, и наша семья показывает пример...  Может это нехорошо,
но я надеюсь, что Петя не даст развода...». Но Петр Олденбургский развод дал, а
Николай II, не послушав на этот раз супругу, согласился на брак сестры. Мария
Федоровна одобрила решение сына. 

По прибытию из Киева в Могилев, царь писал супруге: «...Ольгу мы видели два
раза, она уже встала вчера и выглядит хорошо, хотя и худа, - такое спокойное
хорошее выражение лица. Она письменно просила разрешение повенчаться в субботу
5 ноября. Она, конечно, спросила об этом и Мама, и я принял ее сторону, сказав,
что, по моему мнению, надо покончить с этим делом. Раз оно должно случиться,
пусть случиться теперь! Она хочет взять отпуск на две недели и затем вернуться
к своей работе. Мама намерена еще пожить в Киеве, который она очень любит».

Венчалась в.к. Ольга Александровна с ротмистром лейб-гвардии Кирасирского полка
Н.А. Куликовским 4 ноября 1916 г. в Николаевской церкви в Киеве. Определить
именно в какой церкви, пока что не удается. Только на Подоле их три. Была еще
Николаевская церковь в Предмостно-Никольской слободке, там венчались Николай
Гумилев с Анной Ахматовой. Есть данные, что происходило это в церкви Святого
Василия. Скорее всего, рангу венчающихся более подходил Никольский военный
собор, но они отказались от него. Зачем им лишняя помпезность и огласка? Это
обряд бросал тень на весь царствующий дом, поэтому его совершали в тайне. 

Ольга Александровна вспоминала: «На церемонию пришли Мама и Сандро.
Присутствовали два или три офицера Гусарского Ахтырского полка и немногие мои
подруги из числа сестер милосердия. Потом персонал лазарета устроил в нашу
честь обед. Тем же вечером я вернулась на дежурство в палату. Но я была
действительно счастлива. У меня сразу прибавилось сил. Стоя в церкви рядом с
моим любимым «Кукушкиным», я решил смело глядеть в лицо будущему, каким бы оно
не оказалось. Я была благодарна Всевышнему за то, что Он даровал мне такое
счастье». 

Побольше
сведений находим в тех же воспоминаниях в.к. Александра Михайловича: «Что
касается Великой княгини Ольги Александровны, то самые заклятые враги династии
не могли сказать ничего, кроме самого хорошего, о ее бескорыстной работе по
уходу за ранеными. Женщины с такими качествами представляют собою редкое
явление. Всегда одетая как простая сестра милосердия, и разделяя с другой
медсестрой скромную комнату, она начинала свой рабочий день в 7 часов утра, и
часто не ложилась всю ночь, когда надо было перевязать вновь прибывших
раненных. Иногда солдаты отказывались верить, что сестра, которая терпеливо и
нежно за ними ухаживала, была родной сестрой государя и дочерью императора
Александра III. Ее личная жизнь сложилась несчастливо. Она была первым браком
замужем за принцем Петром Александровичем Ольденбургским, человеком с ней
совершенно различным по характеру. Великая княгиня любила искренно и глубоко
одного офицера-кирасира по фамилии Куликовский. Мы все надеялись, что государь
разрешит ей развестись с мужем и вступить в новый брак. И я был очень рад,
когда однажды ясным зимним утром в 1916 году мы сопровождали Ольгу
Александровну и ротмистра Куликовского в маленькую церковь в пригороде Киева.
Это была исключительно скромная, почти тайная для всех свадьба: невеста, жених,
вдовствующая императрица, я, две сестры из Красного Креста и четыре офицера
Ахтырского гусарского полка, шефом которого состояла Великая княгиня. Служил
старенький батюшка. Его слабый голос шел, казалось, не из церкви, а раздавался
откуда-то издалека. Все мы были очень довольны».

Дальнейшая судьба новобрачных прослеживается весьма скудно. Ольга постоянно
была возле матери и продолжала работу в своем госпитале. Но через некоторое
время, ожидая ребенка, она уже не могла выполнять столь тяжелую и грязную
работу. Тем более, что однажды, в день возвращения матери из Могилева, она,
беременная, при выходе из автомобиля, оступилась и неудачно упала. После этого
переживая за ожидаемого ребенка, больше времени проводила с матерью.

Несмотря на уговоры своих родственников, Мария Федоровна не хотела покидать
Киев, к которому она очень привыкла, к тому же она хотела быть поближе к сыну.
Ее зять, Александр Михайлович, вспоминал впоследствии: «Я просил великую
княгиню Ольгу Александровну постараться убедить вдовствующую императрицу
переехать в Крым. Сначала я встретил решительный отпор: она не хотела уезжать
от Ники еще дальше. Если это варварское правительство не позволит Ники
переехать в Киев, заявила она, после того как нам удалось ей разъяснить
настоящее положение государя, то почему же она не могла сопровождать его в
сибирскую ссылку? Его жена Алекс слишком молода, чтобы нести бремя страданий
одной. Она чувствовала, что Ники очень нуждается в поддержке матери». Вскоре
начались враждебные выпады местных властей против императорской семьи. Им не
позволяли посещать госпиталь, так как в добром отношении к ним раненных и
персонала видели монархическую пропаганду. Был издан приказ - всем членам
бывшей императорской семьи покинуть губернский город. Это так возмутило Марию
Федоровну, что «пришлось почти, что нести императрицу на вокзал. Она боролась
до последней минуты, желая оставаться и, заявляя, что предпочитает, чтобы ее
арестовали и бросили в тюрьму». Так в конце марта 1917 г. мать-императрица со
своей дочерью Ольгой, мужем дочери Ксении, – в.к. Александром Михайловичем,
отправились в Крым.

Этот последний путь группы Романовых из Киева достоин описания. В городе они
подвергались большой опасности, но и уехать было невозможно, если бы не
инициатива и самоотверженные усилия Александра Михайловича. Ему удалось найти
поезд, стоявший на заброшенном полустанке за пределами города и суметь привлечь
на свою сторону небольшой отряд, оставшихся верными императору саперов,
строящих мост через Днепр. Они согласились сопровождать поезд в течение всего
полного опасностей пути в Крым.

Семейство покинуло Киев ночью. Чтобы не вызывать подозрений, каждый добирался
до поезда самостоятельно. Когда вдовствующая императрица, Александр Михайлович,
Ольга с мужем разместились, то, даже после отправления, долго ехали молча,
каждый погруженный в свои думы. С ними следовали несколько придворных Марии
Федоровны. Служанка Ольги, верная Мимка, по собственной инициативе отправилась
в Петроград, чтобы забрать хотя бы часть драгоценностей, оставшихся в доме её
хозяйки, но их вернуть не удалось. 

Великая княгиня вспоминала: «Ночь была
холодная. На мне не было ничего, кроме формы сестры милосердия. Чтобы не
привлечь к себе внимания, уходя из лазарета, пальто я одевать, не стала. Муж
накинул мне на плечи свою шинель. В руках у меня был маленький саквояж. Помню,
я посмотрела на него, на мою мятую юбку и поняла, что это все, что у меня
осталось». Романовы, так и не поняли, как добрались до Севастополя. Четыре дня
на каждой станции их поезд атаковали толпы беженцев, которых, как могли,
сдерживали верные саперы. По прибытию их ожидали несколько автомобилей из
авиационной школы, находящейся в ведении Александра Михайловича. Романовы
направились в Ай-Тодор - его имение. Старшая дочь Ксения, со своими тремя
сыновьями приехали немного позже из Петербурга. С ними прибыла её дочь Ирина со
своим щеголем-супругом Феликсом Юсуповым. В имении Ай-Тодор, кроме Марии
Федоровны, разместились её дочери. С Ксенией был Александр Михайлович и их
шесть сыновей. Ольга - со своим мужем, ушедшим в отставку полковником. У них
родился сын Тихон, который был зачат в Киеве (!). Всё Августейшее семейство в
Крыму пережило обыски, грабежи, различные угрозы, а также постоянные перемены
власти. Сила духа, в первую очередь присущая Марии Федоровны и её дочери Ольги,
поддерживало всех.

Вскоре в.к. Ольга Александровна осталась единственной представительницей дома
Романовых в России. Вместе с мужем и сыном она проживала на Кубани. Здесь 23
марта 1919 г. родился их второй сын Гурий. Популярность великая княгиня вполне
заслужила за свою личную скромность и необычайную трудоспособность на благо
страны. Она, с присущей ей деликатностью, отвергала все попытки провозглашения
её императрицей, которые вынашивали монархические круги в белых частях Юга
России.

Когда Красная армия приблизилась к станице, великая княгиня с семьей и
небольшой свитой, перебрались в Ростов, затем в Константинополь, а в 1920 г.
приехала к матери в Данию. Там она со своей семьей жила до 1948 г. в Баллерупе
близ Копенгагена. Её дом вскоре стал общественным местом русской колонии в
Дании. Она по-прежнему много времени отдавала рисованию. 

Ее дети, получив образование в Русской гимназии в Париже, военную службу
проходили в королевской гвардии. Жизнь в Датском королевстве была бы и далее
спокойной, если бы не нота Советского правительства, посланная этой маленькой
стране после войны. В нем обвиняли Ольгу Александровну в пособничестве
«предателям родины». Не исключено, что существовал приказ Сталина об её
ликвидации. Дело в том, что она помогала всем землякам, попавшим в беду, вне
зависимости от их политических симпатий. Их хотели выдать по требованию СССР,
где, в лучшем случае, их ожидала Сибирь.  Великая княгиня помогала им скрыться,
пряча некоторых даже у себя дома, но затем опасаясь за судьбу своих близких,
вынуждена была переехать с семьей в Канаду, где благополучно прожила до своей
кончины в ноябре 1960 г. Похоронена она на русском кладбище «Норс-Йорк» в
Торонто и на два года пережила своего супруга. Они часто вспоминали свое
пребывание и венчание в Киеве, городе с которым их так много связывало.

Из моей книги \enquote{История Киева. Люди и судьбы}
