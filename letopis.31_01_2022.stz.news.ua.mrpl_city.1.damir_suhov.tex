% vim: keymap=russian-jcukenwin
%%beginhead 
 
%%file 31_01_2022.stz.news.ua.mrpl_city.1.damir_suhov
%%parent 31_01_2022
 
%%url https://mrpl.city/blogs/view/damir-suhov-spodivayusya-shho-same-mariupol-stane-moim-spravzhnim-domom
 
%%author_id demidko_olga.mariupol,news.ua.mrpl_city
%%date 
 
%%tags 
%%title Дамір Сухов: "Сподіваюся, що саме Маріуполь стане моїм справжнім домом..."
 
%%endhead 
 
\subsection{Дамір Сухов: \enquote{Сподіваюся, що саме Маріуполь стане моїм справжнім домом...}}
\label{sec:31_01_2022.stz.news.ua.mrpl_city.1.damir_suhov}
 
\Purl{https://mrpl.city/blogs/view/damir-suhov-spodivayusya-shho-same-mariupol-stane-moim-spravzhnim-domom}
\ifcmt
 author_begin
   author_id demidko_olga.mariupol,news.ua.mrpl_city
 author_end
\fi

\ii{31_01_2022.stz.news.ua.mrpl_city.1.damir_suhov.pic.1}

Нещодавно у \emph{Донецькому академічному обласному драматичному театрі (м.
Маріуполь)} почав працювати яскравий і непересічний актор, який за плечима має
понад 100 ролей в серіалах та кіно, \emph{\textbf{Дамір Сухов}}. Я вирішила зустрітися з
артистом і розпитати його, чому він вирішив переїхати саме до Маріуполя. В
процесі інтерв'ю виявилося, що планів у Даміра дуже багато, адже він хоче не
тільки працювати в театрі, але й переїхати до Маріуполя та відкрити у місті
першу кіношколу. Думаю, маріупольцям буде цікаво ближче познайомитися з
людиною, яка вирішила стати нашим земляком.

\ii{31_01_2022.stz.news.ua.mrpl_city.1.damir_suhov.pic.2}

Актором Дамір мріяв стати з дитинства. При цьому хлопець хотів саме зніматися в
кіно. Народився Дамір в Санкт-Петербурзі. На жаль, рано втратив матір. Хлопчика
виростила бабуся, до якої він і переїхав ще маленьким у Бердянськ. Батько,
попри військову службу і велике завантаження, брав активну участь у вихованні
сина. До речі, справжнє прізвище актора – Коломийченко. Сухов – це не тільки
творчий псевдонім, а й дівоче прізвище мами, так артист зберігає пам'ять про
неї. Актор має татарське походження. всі родичі – Сухови, живуть у Мордовії, у
Саранську.

\ii{31_01_2022.stz.news.ua.mrpl_city.1.damir_suhov.pic.3}

З дитинства Дамір займається саме тим, що йому особисто до вподоби. Якщо йому
щось подобається, він може віддаватися цій справі цілком і повністю. Незважаючи
на велику кількість двійок у школі, його часто відправляли на олімпіади з
історії та літератури, адже він має феноменальну пам'ять і з легкістю
запам'ятовує великий масив інформації. Коли з'явилися відеопрокати, з ранку до
ночі дивився касети. Йому подобалося абсолютно все – від драми до комедії.
Батько-військовий приїхав до Даміра, коли йому було 14–15 років. Хлопець одразу
повідомив, що хоче стати саме актором. Спочатку батько не був у захваті від
цієї ідеї. Проте зараз дивиться із задоволенням кіно з його участю та пишається
сином.

\ii{31_01_2022.stz.news.ua.mrpl_city.1.damir_suhov.pic.4}

Завдяки яскравій зовнішності, Даміру вдалося вступити до Дніпропетровського
театрального училища, де він зустрів викладачку, яку й досі вважає своїм
головним вчителем. Це \emph{\textbf{Неллі Михайлівна Пінська}}, яка його не тільки підтримала,
а й посприяла формуванню Даміра як актора. А далі завдяки щасливому випадку
почалися перші зйомки в кіно. Він зайшов до кастинг-директорки \emph{\textbf{Альони Багорат}},
заговорив... А вона йому: \emph{\enquote{Знову хохол!.. Зараз прийде ще один божевільний}}. І
зайшов високий хлопець із дуже дивною поведінкою. Він повідомив, що його звуть
Сергій, і він буде режисером. Саме він запропонував Даміру піти на студію
\enquote{Ринку медіа}, де його дружина могла влаштувати його – бідолашного артиста – у
масовку, щоб хоч 200 рублів заробити. Він пішов, його сфотографували і через
три тижні зателефонували. На кіностудії дали пачку сценарію, вийшов режисер
\emph{\textbf{Борис Горлов}} і сказав: \emph{\enquote{Сподіваюся, ти мене не підведеш}}. Вже наступного дня
він знімався – грав старшого лейтенанта міліції та працював із актором \emph{\textbf{Євгеном
Сидихіним.}} То був фільм \enquote{Казка про щастя}. З цього фільму і розпочалася
кінематографічна \enquote{казка про щастя} нашого героя. А \emph{\textbf{Сергій Полуянов}}, який
відправив тоді Даміра до своєї дружини, невдовзі став другим режисером
\enquote{Адмірала} і покликав його зніматися.

\ii{31_01_2022.stz.news.ua.mrpl_city.1.damir_suhov.pic.5}

Проте перш ніж потрапити на великий екран актору довелося змінити багато
професій. Він працював барменом, офіціантом, охоронцем. Навіть, навчаючись у
Санкт-Петербурзі в Школі російської драми, він встигав вночі працювати
продавцем шаверми, треба ж було за щось жити.

\ii{31_01_2022.stz.news.ua.mrpl_city.1.damir_suhov.pic.6}

Цікаво, що здебільшого в кіно актор отримує ролі негативних персонажів, проте в
реальному житті Дамір досить проста, легка і позитивна людина. Всі його знайомі
дуже здивовані його перевтіленнями, бо на відміну від своїх героїв він ніколи
не кричить і не скандалить, ненавидить конфлікти.

Водночас у артиста є й власні табу на деякі ролі. Він точно розуміє, що ніколи
не зможе зіграти чоловіків, які ображають дітей чи жінок і вже відмовлявся від
фільмів, де потрібно було грати сепаратиста.

Завдяки чудовій формі та спортивній підготовці Дамір може впоратися з різними
завданнями режисерів. Так він обходиться без допомоги каскадера, хоча в деяких
проєктах доводиться грати не зовсім безпечні для життя та  здоров'я сцени.

\ii{31_01_2022.stz.news.ua.mrpl_city.1.damir_suhov.pic.7}

У чоловіка є сім'я: дружина \emph{\textbf{Вікторія}} і п'ятирічний син \emph{\textbf{Мирослав}}. Поки синові не
виповнилося півтора роки, він повністю присвячував себе сім'ї. Це дуже  важливо
для Даміра, адже він нарешті, знайшов те, про що завжди мріяв. Потім настав час
повернутися до професії. Друзі з Дніпра дали контакти київських агентів, він
зателефонував \emph{\textbf{Вікторії Ткаченко}} – вислав резюме та поїхав до столиці. Йому
запропонували взяти участь у перших пробах, і він їх успішно завалив. Але
Вікторія вірила в нього. Насправді, запорука гарної акторської кар'єри – це
добрий агент. І якщо студентам у цьому плані легше (агенти самі шукають таланти
у театральних вишах), то акторам більш старшій віковій категорії важко
переконати когось працювати з ними. У 2017 році Дамір пройшов проби у серіал
\enquote{Вперше прощається}. Грав слідчого, закоханого в героїню \emph{\textbf{Ольги Арнтгольц}}.
Наразі актор має великий список ролей в серіалах та кіно. Він зіграв більше,
ніж у 100 серіалах та фільмах, серед яких: \enquote{Штамп у паспорті} (2018), \enquote{Я
подарую тобі світанок} (2018), \enquote{Як довго я на тебе чекала} (2019), \enquote{Не жіноча
праця} (2019), \enquote{Готель \enquote{Купідон}} (2019), \enquote{Слуга народу-3. Вибiр} (2019),
\enquote{Таємниці} (2019), \enquote{Час іти, час повертатись} (2020), \enquote{Розтин покаже} (2020),
\enquote{Два серця} (2020), \enquote{Мій милий знайда} (2020), \enquote{Мій чоловік, моя жінка} (2020),
\enquote{Сага} (2020), \enquote{Кейс}, (2021), \enquote{Провінціал} (2021) \enquote{Ламані лінії} (2022) та
інші. Цікаво, що два серіали – \enquote{Правила лабіринту} та \enquote{Щасливчик Пашка} – він
зняв як другий режисер. Деякими своїми роботами Дамір дуже пишається. Зокрема,
для нього цінним і важливим досвідом стала участь в зйомках серіалу
\enquote{Відступники} (2022 рік, режисер \emph{\textbf{Валерій Ібрагімов}}). Актор вважає, що вийшло
дійсно якісне кіно.

\ii{31_01_2022.stz.news.ua.mrpl_city.1.damir_suhov.pic.8}

На думку Даміра, він ще не повною мірою реалізував свій потенціал, як актор
театру. У 2001–2003 роках. працював у Дніпропетровському російському театрі.
Зараз у Донецькому академічному обласному драматичному театрі (м. Маріуполь)
артист вже працює над новою роллю у виставі \enquote{Театральна рулетка} (режисер
\emph{\textbf{Сергій Мусієнко}}), прем'єра якої запланована на березень. Актор у захваті і від
роботи режисера, і від талановитих колег. Вважає, що такий легкий комедійний
спектакль – про закулісне життя артистів обов'язково сподобається
маріупольським глядачам.

Через велику кількість ролей у артиста дуже напружений графік. Даміру хотілося
б більше часу проводити з сім'єю. Водночас він зізнається, що для нього
постійні зйомки – це вже норма. Сміючись, поділився, що було б непогано
постійно працювати над однією роллю в якомусь багатосерійному серіалі, де він
би був повністю зайнятий років 30 і, щоб зйомки відбувалися у Маріуполі поруч
із сім'єю. Все ж їздити постійно до столиці і пропускати дорослішання сина –
дуже важко. Актор зауважив, що йому не вистачає стабільності і постійності. До
Маріуполя хоче переїхати з родиною, адже вважає, що місто досить швидко
розвивається і тут є багато перспектив. Його вразило, наскільки Маріуполь
змінився за останні декілька років. Дружина Даміра теж налаштована скоріше
переїхати і хоче, щоб син пішов у перший клас саме в Маріуполі.  Також наш
герой помітив, що серед маріупольців багато усміхнених і позитивних людей. У
такому молодому і потужному місті артист хоче працювати та жити. Дамір дуже
сподівається, що саме Маріуполь стане справжнім домом для нього та його сім'ї.

\ii{31_01_2022.stz.news.ua.mrpl_city.1.damir_suhov.pic.9}

У Маріуполі Дамір Сухов не тільки збирається придбати помешкання, але й в
партнерстві з актором театру і кіно \emph{\textbf{Глібом Михайличенком}} відкрити першу
кіношколу в Донецькій області. З Глібом Дамір познайомився під час спільних
зйомок, він був приємно вражений талантом хлопця і дуже швидко знайшов з ним
спільну мову. Незважаючи на те, що Гліб мешкає в Києві, він готовий їздити до
Маріуполя і працювати разом з Даміром в новій кіношколі, де читатимуть лекції
та проводитимуть майстер-класи відомі вітчизняні та зарубіжні актори і
режисери. Кожен з учнів кіношколи зможе знайти для себе, що він шукає – від
натхнення для створення персональних проєктів до потрібних скіллів, необхідних
для роботи в кіно вже наступного дня після випуску. Ініціативу Даміра
підтримує Департамент культурно-громадського розвитку Маріуполя. Вже відомо,
що кіношкола знаходитиметься у приміщенні Камерної філармонії. Відкриття
планується в березні 2022 року.

\ii{31_01_2022.stz.news.ua.mrpl_city.1.damir_suhov.pic.10}

Впевнена, що Маріуполю дуже пощастило з Даміром і завдяки його появі культурне
життя міста стане ще більш насиченим і яскравим.

\begingroup
\em
\textbf{Улюблена книга:} 

\begin{quote}
\enquote{Три товариші} та \enquote{На Західному фронті без змін} Еріха Марії Ремарка; дебютний роман американського автора Чака Поланіка \enquote{Бійцівський клуб}.
\end{quote}
 
\textbf{Улюблений фільм:} 

\begin{quote}
\enquote{Роккі} (1976 рік), \enquote{Ідеальний світ} (1993 рік) та \enquote{Форрест Ґамп} (1994 рік).
\end{quote}

\textbf{Побажання маріупольцям:} 

\begin{quote}
\enquote{Бажаю Маріуполю та його мешканцям продовжувати так само потужно розвиватися і приємно вражати своїми змінами. І головне - всім нам бажаю миру та спокою!}.
\end{quote}
\endgroup
