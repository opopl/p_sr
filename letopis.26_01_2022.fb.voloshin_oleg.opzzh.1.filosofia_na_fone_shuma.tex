% vim: keymap=russian-jcukenwin
%%beginhead 
 
%%file 26_01_2022.fb.voloshin_oleg.opzzh.1.filosofia_na_fone_shuma
%%parent 26_01_2022
 
%%url https://www.facebook.com/oleg.voloshin.7165/posts/5145915618774791
 
%%author_id voloshin_oleg.opzzh
%%date 
 
%%tags rossia,russofobia,ukraina,antirossia
%%title Немного философии на фоне окружающего шума
 
%%endhead 
 
\subsection{Немного философии на фоне окружающего шума}
\label{sec:26_01_2022.fb.voloshin_oleg.opzzh.1.filosofia_na_fone_shuma}
 
\Purl{https://www.facebook.com/oleg.voloshin.7165/posts/5145915618774791}
\ifcmt
 author_begin
   author_id voloshin_oleg.opzzh
 author_end
\fi

Немного философии на фоне окружающего шума. 

В течение всех трёх десятилетий после распада СССР националистические силы и их
западные покровители имели почти неограниченные возможности формировать
антироссийские и даже антирусские настроения в Украине. На эти цели тратились
огромные силы и деньги. Не случайно русский язык из высшего образования исчез
ещё в 90-е. 

В какие-то периоды этими возможностями пользовались более, в какие-то менее
активно, но полностью соответствующая идеологическая работа не прекращалась
никогда. А в последние восемь лет она лежит в основе государственной
гуманитарной политики на всех уровнях. 

Одновременно противодействие России такому курсу никогда не было сколько-нибудь
эффективным. А отдельные практические шаги России (одно присоединение Крыма
чего только стоит) прямо противоречили целям популяризации идеи сотрудничества
с Россией и нахождения с ней совпадающих интересов. 

Но несмотря на все эти объективные препятствия условно «пророссийские»
настроения, признаки принадлежности к общему культурно-идеологическому полю и
за три десятилетия оказались не изжиты естественным образом. Адептом
соответствующих целях в итоге приходится прибегать к методам принуждения:
квотам, запретам, санкциям, закрытию СМИ, откровенным репрессиям и тд.
Привлекать к этим усилиям не только внутренние резервы, но и карательную мощь
западных стран. И все ради к удивлению актуальной задачи: недопущения «реванша»
пророссийских сил, расширения сферы использования русского языка, увеличения
общественной поддержки иного курса. За сокращение товарооборота с Россией уже,
кажется, и бороться перестали. 

Все это заставляет предположить, что «глубинный народ» в Украине (скорее
интуитивно, чем сознательно) последовательно отвергает навязываемую
искусственную модель «анти-России». Откажись завтра Запад активно деньгами и
давлением поддерживать соответствующий курс, и некий баланс все равно будет
восстановлен естественным образом. За счёт так прославляемых Западом сил
рыночной экономики (делающих русскоязычный продукт более конкурентоспособным) и
плюралистической демократии (гарантирующих весомое представительство
соответствующих сил в парламенте). 

В этом смысле Москве совершенно не нужно военное вторжение. Однако необходимы
гарантии Запада относительно купирования наиболее одиозных форм зачистки всего
русского здесь. Плюс реинтеграция Донбасса на более менее адекватных условиях.
Этого-то русофобы и боятся. Время и история играют не на их стороне. Потому в
ближайшее время может быть всякое.
