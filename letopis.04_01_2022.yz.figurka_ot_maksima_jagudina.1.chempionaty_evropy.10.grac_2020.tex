% vim: keymap=russian-jcukenwin
%%beginhead 
 
%%file 04_01_2022.yz.figurka_ot_maksima_jagudina.1.chempionaty_evropy.10.grac_2020
%%parent 04_01_2022.yz.figurka_ot_maksima_jagudina.1.chempionaty_evropy
 
%%url 
 
%%author_id 
%%date 
 
%%tags 
%%title 
 
%%endhead 
\subsubsection{Чемпионат Европы 2020 (Грац)}

Вот мы и подошли к последнему на момент написания этой статьи Чемпионату
Европы, который успели провести два года назад перед \enquote{объявлением войны} с
коронавирусом. Это был уже турнир нового уровня с элементами ультра-си в
исполнении невероятного трио Хрустального ТЩК Трусова-Щербакова-Косторная,
которые в сезоне 19/20 выиграли совместными усилиями все турниры в которых
принимали участие, словно соперниц вокруг не существовало. 

\ii{04_01_2022.yz.figurka_ot_maksima_jagudina.1.chempionaty_evropy.10.grac_2020.pic.1}

15-летняя Александра Трусова к своим многочисленным юниорским победам по ходу
сезона не смогла добавить победы уже на взрослом уровне. На чемпионате Европы
она снова, как и ранее в финале Гран-при и на Чемпионате России, оказалась
только третьей после своих подруг по группе.

\ii{04_01_2022.yz.figurka_ot_maksima_jagudina.1.chempionaty_evropy.10.grac_2020.pic.2}

Ровесница Саши Анна Щербакова приехала в Грац после триумфа на Чемпионате
России, где она впервые в своей карьере исполнила на выезд три четверных
прыжка. В \enquote{австрийском шатре} прокат произвольной получился чуть менее
успешным, к тому же и судьи возможно были к Анне слишком строги с
недокрутом-невидимкой на четверном флипе. К серебру финала Гран-при Анна
добавила серебро Чемпионата Европы.

Старшая в этом трио 16-летняя Алена Косторная стала лучшей фигуристкой сезона
19/20 и по наградам, и по рейтингу. На Чемпионате Европы ее прокаты не были
безупречными, как на Чемпионате России, но и Анна Щербакова тоже ошибалась - в
результате к своей победе в финале Гран-при Алена прибавила титул действующей
чемпионки Европы, которой является вот уже 2 года.

Десять лет истории пролетели незаметно, российское женское фигурное катание
прошло путь от \enquote{выжженной земли} после Бутырской и Слуцкой до полной доминации
в Мире, от зрительской безучастности до самого медийного вида спорта в котором
вдруг стали разбираться буквально все.

От истории движемся к настоящему времени. Впереди Чемпионат Европы 2022,
который будет предварять моя традиционная серия Превью с раскладом сил в каждой
дисциплине, а пока смотрим на расписание. Соревновательная часть турнира
привычно растянута сразу на 4 дня, у мужчин и женщин на этот раз будет один
день перерыва между короткой и произвольной программами. В каждой дисциплине у
России максимальное представительство и хорошие шансы на призовые места.

\textbf{Расписание, время московское}

\ii{04_01_2022.yz.figurka_ot_maksima_jagudina.1.chempionaty_evropy.10.grac_2020.pic.3}

Всего доброго и до новых встреч на моем канале !
