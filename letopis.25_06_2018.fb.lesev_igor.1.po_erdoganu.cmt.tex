% vim: keymap=russian-jcukenwin
%%beginhead 
 
%%file 25_06_2018.fb.lesev_igor.1.po_erdoganu.cmt
%%parent 25_06_2018.fb.lesev_igor.1.po_erdoganu
 
%%url 
 
%%author_id 
%%date 
 
%%tags 
%%title 
 
%%endhead 
\subsubsection{Коментарі}
\label{sec:25_06_2018.fb.lesev_igor.1.po_erdoganu.cmt}

\begin{itemize} % {
\iusr{Михаил Подоляк}

отличный экспресс/комментарий... и да, Эрдоган таки попытается устроить в
стране «супер/демократию»), но все это неизбежно приведёт к повторным досрочным
выборам уже через 2-3 года)... Потому что он НЕ знает классического
византийского принципа «разделяй и властвуй» (проще - всех обманывай), а
предпочитает гопнический принцип «ша вломлю»... в том числе и на внешних
рынках, где врать нужно вообще самозабвенно)))...

\iusr{Игорь Лесев}

все так, только у него есть стержень, харизма и удача. А приспосабливается на
внешних рынках он настолько нагло, что в Тель-Авиве могли бы, как говорит наш
гарант, "открывать тетрадь и записывать"

\iusr{Михаил Романов}
Ну про божество это перебор. Эрдогана поддерживают далеко не все. Порядка 30\% - жестко против его и пути, который он предлагает

\iusr{Игорь Лесев}
ну так об этом в тексте и говорится)) ориентир модели политического устройства у Эрдогана - это Азербайджан, Казахстан и страны Средней Азии. Но именно ТАК построить не получится, хотя он к этому со времен Ататюрка максимально УЖЕ приблизился

\iusr{Ярослав Козачок}

Шмкарно! Я бы еще добавил один аспект. Он заключается в том, что "на днях"
истекает 100-летний срок унизительного для Турции мира по итогам 1 мировой. Они
до сих пор ограничены во многом - морю, недрам, территории... И для того, чтобы
вернуть свое, нужен Эрдоган. Другой навряд ли справится)


\iusr{Игорь Лесев}
Спасибо, Ярослав Викторович. Эпитет "шикарно" от вас дорогого стоит)

\iusr{Andrei Mazur}
Интересно. Полез в первоисточники искать инфу за упомянутый в комментариях договор )))

\iusr{Дмитрий Марунич}
мы чугуний производили в свое время. Он оказался не настолько востребован

\iusr{Дмитрий Марунич}
по Турции. В экономике ситуация сейчас печаль. Девал и на нет сходит рост ВВП

\begin{itemize} % {
\iusr{Игорь Лесев}
знаете, в экономике Китая тоже можно найти печаль - спадает темпы роста ВВП, перегружен внутренний рынок и намечается торговая война с главным покупателем продукции - Штатами... нету в мире экономик, где все идеально. Но если рядом поставить какую-то Т... Ещё

\iusr{Дмитрий Марунич}
\textbf{Игорь Лесев} я пишу со слов резидентов ТР.

\iusr{Mike Schachmann}
этот чмурик давеча расмешил тут всех. обратился до его паствы, тут. в европе. они его сильно любят. чем дальше от турции, тем больше любят. так он их призвал куплять эту лиру, бо она рухнула, и шоб как то стабилизировать курс, он их попросил обменять евро на эти фантики.
\end{itemize} % }

\iusr{Andrei Mazur}

Интересный пост. Спасибо. Зацепили аналогии Восток Украины - курды и адепты
майдана - Эгейский регион.  Правда, позволю себе уточнить. Пять наших западных
областей по своей экономической роли в стране и рядом не валялись с тем весом,
что имеет Эгейский регион в Турции. Но это так - соображения по ходу чтения

\begin{itemize} % {
\iusr{Игорь Лесев}
ну так ВВП одного Стамбула больше, чем всей Украины... куле уж тут сравнивать

\iusr{Andrei Mazur}
\textbf{Игорь Лесев} не знал сей факт.))) Показательно.

\iusr{Игорь Лесев}
ВВП китайской провинции Гуандун больше, чем ВВП всей России)) много интересностей в нашем мире))

\iusr{Andrei Mazur}
\textbf{Игорь Лесев} О! Не зря на вас подписался))).

\iusr{Игорь Лесев}
ну, спасибо) будем вместе теперь изучать ВВП стран мира, регионов и отдельных городов))

\iusr{Andrei Mazur}
\textbf{Игорь Лесев} Лучше за Турцию  @igg{fbicon.smile}  давно интересуюсь страной
\end{itemize} % }

\iusr{Анатолій Яворський}
Краще бути халіфом на годинку, чим - ... Правильну версію підкаже час. Або Муамар...

\iusr{Галина Акимова}
Все правильно! Хороший анализ !

\iusr{Игорь Лесев}
спасибо)

\iusr{Михаил Мищишин}

Существенный момент подмечен и сравнение с Украиной, на мой взгляд - в тему и точное.

\end{itemize} % }
