% vim: keymap=russian-jcukenwin
%%beginhead 
 
%%file 01_08_2021.fb.fb_group.story_kiev_ua.2.kvitka_osnovjanenko_pismo.cmt
%%parent 01_08_2021.fb.fb_group.story_kiev_ua.2.kvitka_osnovjanenko_pismo
 
%%url 
 
%%author 
%%author_id 
%%author_url 
 
%%tags 
%%title 
 
%%endhead 
\subsubsection{Коментарі}

\begin{itemize}
%%%fbauth
%%%fbauth_name
\iusr{Volodymyr Nekrasov}
%%%fbauth_url
%%%fbauth_place
%%%fbauth_id
%%%fbauth_front
%%%fbauth_desc
%%%fbauth_www
%%%fbauth_pic
%%%fbauth_pic portrait
%%%fbauth_pic background
%%%fbauth_pic other
%%%fbauth_tags
%%%fbauth_pubs
%%%endfbauth
 

Згідно правил, не можу дати посилання. Але кому цікаво як справді мають чиатись
Ять та інші зниклі букви раджу переглянути на ютюбі канал Ідея Олександрівна -
Ять, Ікавізм та Редуковані. Юна філологиня дуже просто все пояснює. Доречі, наш
рідний Город таки Київ \Smiley[1.0][yellow]



%%%fbauth
%%%fbauth_name
\iusr{Алла Марьянчик}
%%%fbauth_url
%%%fbauth_place
%%%fbauth_id
%%%fbauth_front
%%%fbauth_desc
%%%fbauth_www
%%%fbauth_pic
%%%fbauth_pic portrait
%%%fbauth_pic background
%%%fbauth_pic other
%%%fbauth_tags
%%%fbauth_pubs
%%%endfbauth
 
Папа, наверное нас помнит. Аркадий , Алла. Привет большой! Ещё раз извините

%%%fbauth
%%%fbauth_name
\iusr{Алла Марьянчик}
%%%fbauth_url
%%%fbauth_place
%%%fbauth_id
%%%fbauth_front
%%%fbauth_desc
%%%fbauth_www
%%%fbauth_pic
%%%fbauth_pic portrait
%%%fbauth_pic background
%%%fbauth_pic other
%%%fbauth_tags
%%%fbauth_pubs
%%%endfbauth
 
Извините за вопрос: Ваш отец- Валерий?

\begin{itemize}
%%%fbauth
%%%fbauth_name
\iusr{Алла Марьянчик}
%%%fbauth_url
%%%fbauth_place
%%%fbauth_id
%%%fbauth_front
%%%fbauth_desc
%%%fbauth_www
%%%fbauth_pic
%%%fbauth_pic portrait
%%%fbauth_pic background
%%%fbauth_pic other
%%%fbauth_tags
%%%fbauth_pubs
%%%endfbauth
 

Когда-то, очень давно, 60-65 годы мы общались в одной компании Вам знакомы ,
наверное, фамилии Микулич, Яхнин. Я знаю, что родителей уже нет. Светлая память
им!

\begin{itemize}
%%%fbauth
%%%fbauth_name
\iusr{Maksim Pestun}
%%%fbauth_url
%%%fbauth_place
%%%fbauth_id
%%%fbauth_front
%%%fbauth_desc
%%%fbauth_www
%%%fbauth_pic
%%%fbauth_pic portrait
%%%fbauth_pic background
%%%fbauth_pic other
%%%fbauth_tags
%%%fbauth_pubs
%%%endfbauth
 
\textbf{Алла Марьянчик} да, конечно хорошо знал их! Папа жив


%%%fbauth
%%%fbauth_name
\iusr{Алла Марьянчик}
%%%fbauth_url
%%%fbauth_place
%%%fbauth_id
%%%fbauth_front
%%%fbauth_desc
%%%fbauth_www
%%%fbauth_pic
%%%fbauth_pic portrait
%%%fbauth_pic background
%%%fbauth_pic other
%%%fbauth_tags
%%%fbauth_pubs
%%%endfbauth
 
Извините, ради Бога. Неверная информация.
\end{itemize}

%%%fbauth
%%%fbauth_name
\iusr{Maksim Pestun}
%%%fbauth_url
%%%fbauth_place
%%%fbauth_id
%%%fbauth_front
%%%fbauth_desc
%%%fbauth_www
%%%fbauth_pic
%%%fbauth_pic portrait
%%%fbauth_pic background
%%%fbauth_pic other
%%%fbauth_tags
%%%fbauth_pubs
%%%endfbauth
 
? Вы его знаете?


%%%fbauth
%%%fbauth_name
\iusr{Maksim Pestun}
%%%fbauth_url
%%%fbauth_place
%%%fbauth_id
%%%fbauth_front
%%%fbauth_desc
%%%fbauth_www
%%%fbauth_pic
%%%fbauth_pic portrait
%%%fbauth_pic background
%%%fbauth_pic other
%%%fbauth_tags
%%%fbauth_pubs
%%%endfbauth
 
\textbf{Алла Марьянчик} да

\begin{itemize}
%%%fbauth
%%%fbauth_name
\iusr{Алла Марьянчик}
%%%fbauth_url
%%%fbauth_place
%%%fbauth_id
%%%fbauth_front
%%%fbauth_desc
%%%fbauth_www
%%%fbauth_pic
%%%fbauth_pic portrait
%%%fbauth_pic background
%%%fbauth_pic other
%%%fbauth_tags
%%%fbauth_pubs
%%%endfbauth
 
\textbf{Maksim Pestun} Вы похожи. Мы - старые очень друзья Ваших родителей. Хорошо их помню


%%%fbauth
%%%fbauth_name
\iusr{Алла Марьянчик}
%%%fbauth_url
%%%fbauth_place
%%%fbauth_id
%%%fbauth_front
%%%fbauth_desc
%%%fbauth_www
%%%fbauth_pic
%%%fbauth_pic portrait
%%%fbauth_pic background
%%%fbauth_pic other
%%%fbauth_tags
%%%fbauth_pubs
%%%endfbauth
 
\textbf{Maksim Pestun} если это не совпадение, то да. Мама Ваша- Света ?


%%%fbauth
%%%fbauth_name
\iusr{Maksim Pestun}
%%%fbauth_url
%%%fbauth_place
%%%fbauth_id
%%%fbauth_front
%%%fbauth_desc
%%%fbauth_www
%%%fbauth_pic
%%%fbauth_pic portrait
%%%fbauth_pic background
%%%fbauth_pic other
%%%fbauth_tags
%%%fbauth_pubs
%%%endfbauth
 
\textbf{Алла Марьянчик} да, Света
\end{itemize}

\end{itemize}

%%%fbauth
%%%fbauth_name
\iusr{Max Gopencko}
%%%fbauth_url
%%%fbauth_place
%%%fbauth_id
%%%fbauth_front
%%%fbauth_desc
%%%fbauth_www
%%%fbauth_pic
%%%fbauth_pic portrait
%%%fbauth_pic background
%%%fbauth_pic other
%%%fbauth_tags
%%%fbauth_pubs
%%%endfbauth
 
Багато цікавого з коментів дізнався.


%%%fbauth
%%%fbauth_name
\iusr{Volodymyr Nekrasov}
%%%fbauth_url
%%%fbauth_place
%%%fbauth_id
%%%fbauth_front
%%%fbauth_desc
%%%fbauth_www
%%%fbauth_pic
%%%fbauth_pic portrait
%%%fbauth_pic background
%%%fbauth_pic other
%%%fbauth_tags
%%%fbauth_pubs
%%%endfbauth
 

Дякую! Люблю такі теми) До 1928 року не було офіційного українського правопису.
Грушевський і Нечуй-Левицький багато багато списів поламали поки звели докупи
різні погляди українців щодо норм української мови. Ось для прикладу Захар
Беркут 1883. Подивітся наскільки відмінна літературна мова тогочасного Львова і
Харкова.

\ifcmt
  ig https://scontent-cdg2-1.xx.fbcdn.net/v/t1.6435-9/227959809_10219047375006549_7820047395817386662_n.jpg?_nc_cat=108&ccb=1-4&_nc_sid=dbeb18&_nc_ohc=_AmAZOG8of0AX-jbGoR&_nc_ht=scontent-cdg2-1.xx&oh=91367d913fa62974bd25584a2e6c0e41&oe=613420CB
  width 0.4
\fi

\begin{itemize}
%%%fbauth
%%%fbauth_name
\iusr{Леся Падалка}
%%%fbauth_url
%%%fbauth_place
%%%fbauth_id
%%%fbauth_front
%%%fbauth_desc
%%%fbauth_www
%%%fbauth_pic
%%%fbauth_pic portrait
%%%fbauth_pic background
%%%fbauth_pic other
%%%fbauth_tags
%%%fbauth_pubs
%%%endfbauth
 
Як це, словник Грінченка був, а правопису не було)

\begin{itemize}
%%%fbauth
%%%fbauth_name
\iusr{Volodymyr Nekrasov}
%%%fbauth_url
%%%fbauth_place
%%%fbauth_id
%%%fbauth_front
%%%fbauth_desc
%%%fbauth_www
%%%fbauth_pic
%%%fbauth_pic portrait
%%%fbauth_pic background
%%%fbauth_pic other
%%%fbauth_tags
%%%fbauth_pubs
%%%endfbauth
 
\textbf{Леся Падалка} так, але словник Грінченка вже був пізніше за згаданих тут видань.


%%%fbauth
%%%fbauth_name
\iusr{Леся Падалка}
%%%fbauth_url
%%%fbauth_place
%%%fbauth_id
%%%fbauth_front
%%%fbauth_desc
%%%fbauth_www
%%%fbauth_pic
%%%fbauth_pic portrait
%%%fbauth_pic background
%%%fbauth_pic other
%%%fbauth_tags
%%%fbauth_pubs
%%%endfbauth
 
\textbf{Volodymyr Nekrasov} коментар про "до 1928 не було")
\end{itemize}

%%%fbauth
%%%fbauth_name
\iusr{Леся Падалка}
%%%fbauth_url
%%%fbauth_place
%%%fbauth_id
%%%fbauth_front
%%%fbauth_desc
%%%fbauth_www
%%%fbauth_pic
%%%fbauth_pic portrait
%%%fbauth_pic background
%%%fbauth_pic other
%%%fbauth_tags
%%%fbauth_pubs
%%%endfbauth
 
\textbf{Volodymyr Nekrasov} можливо це якось пов'язано з тим, що не було офіційної української держави?))

\begin{itemize}
%%%fbauth
%%%fbauth_name
\iusr{Volodymyr Nekrasov}
%%%fbauth_url
%%%fbauth_place
%%%fbauth_id
%%%fbauth_front
%%%fbauth_desc
%%%fbauth_www
%%%fbauth_pic
%%%fbauth_pic portrait
%%%fbauth_pic background
%%%fbauth_pic other
%%%fbauth_tags
%%%fbauth_pubs
%%%endfbauth
 
\textbf{Леся Падалка} саме так. Зрештою всі мови через це пройшли. Яскравий приклад - німецька мова, але не хохдойч.


%%%fbauth
%%%fbauth_name
\iusr{Юрий Стебельский}
%%%fbauth_url
%%%fbauth_place
%%%fbauth_id
%%%fbauth_front
%%%fbauth_desc
%%%fbauth_www
%%%fbauth_pic
%%%fbauth_pic portrait
%%%fbauth_pic background
%%%fbauth_pic other
%%%fbauth_tags
%%%fbauth_pubs
%%%endfbauth
 
\textbf{Леся Падалка} у описываемого на этой странице Франком селянского народа и не могло быть своей державы.
У горожан своё огромное государство на Руси было.
\end{itemize}

\end{itemize}

%%%fbauth
%%%fbauth_name
\iusr{Леся Падалка}
%%%fbauth_url
%%%fbauth_place
%%%fbauth_id
%%%fbauth_front
%%%fbauth_desc
%%%fbauth_www
%%%fbauth_pic
%%%fbauth_pic portrait
%%%fbauth_pic background
%%%fbauth_pic other
%%%fbauth_tags
%%%fbauth_pubs
%%%endfbauth
 

Також, якщо вам цікаво, 1833 рік - рік видання альманаху з цією "суплікою" - це
рік ще початкового періоду активного формування української абетки в
наближеному до сучасного вигляді (а мова і абетка також змінюються постійно,
свідки чого ми й зараз). Лише 35 років тому вийшов друком перший україномовний
літературний твір (Енеїда, і теж в російському альманасі), заснований на
реальному, народному мовленні. І література а з нею і літературна мова
знаходилися в стадії формування, підбору графічних засобів (літер) під
реальність мовлення. Про все це можна почитати, і не лише в книжках, навіть
інтернеті)

\begin{itemize}
%%%fbauth
%%%fbauth_name
\iusr{Max Gopencko}
%%%fbauth_url
%%%fbauth_place
%%%fbauth_id
%%%fbauth_front
%%%fbauth_desc
%%%fbauth_www
%%%fbauth_pic
%%%fbauth_pic portrait
%%%fbauth_pic background
%%%fbauth_pic other
%%%fbauth_tags
%%%fbauth_pubs
%%%endfbauth
 
\textbf{Леся Падалка} пам'ятаю, як з розпадом совка, вчителька повідомила нас,
що в абетці з'явилася літера "ґ"... але ж її майже не використовували, поки
Росія не змусила згадати знову)
\end{itemize}

%%%fbauth
%%%fbauth_name
\iusr{Max Gopencko}
%%%fbauth_url
%%%fbauth_place
%%%fbauth_id
%%%fbauth_front
%%%fbauth_desc
%%%fbauth_www
%%%fbauth_pic
%%%fbauth_pic portrait
%%%fbauth_pic background
%%%fbauth_pic other
%%%fbauth_tags
%%%fbauth_pubs
%%%endfbauth
 
Дуже цікаво, є літера "ы", як і в білоруській мові.... також слово "москалі" 😃


%%%fbauth
%%%fbauth_name
\iusr{Olga Shol}
%%%fbauth_url
%%%fbauth_place
%%%fbauth_id
%%%fbauth_front
%%%fbauth_desc
%%%fbauth_www
%%%fbauth_pic
%%%fbauth_pic portrait
%%%fbauth_pic background
%%%fbauth_pic other
%%%fbauth_tags
%%%fbauth_pubs
%%%endfbauth
 
А ещё очень интересно бывает Григория Сковороду в оригинале читать


%%%fbauth
%%%fbauth_name
\iusr{Tovarish Inka}
%%%fbauth_url
%%%fbauth_place
%%%fbauth_id
%%%fbauth_front
%%%fbauth_desc
%%%fbauth_www
%%%fbauth_pic
%%%fbauth_pic portrait
%%%fbauth_pic background
%%%fbauth_pic other
%%%fbauth_tags
%%%fbauth_pubs
%%%endfbauth
 

Ось таке щастя випало (хахаха) на мою долю! Дозвольте звернути вашу увагу на
особливу "вертлявість" висловів у бажанні авторів цих тестів донести до нас
неохідність такого іспиту. Особисто мені так смішно з усього цього цирку, аж
плакати хочеться. А ще, соромно, що доводиться приймати участь у цьому фарсі.\par

\url{https://test.ua/dm}

\begin{itemize}
%%%fbauth
%%%fbauth_name
\iusr{Tovarish Inka}
%%%fbauth_url
%%%fbauth_place
%%%fbauth_id
%%%fbauth_front
%%%fbauth_desc
%%%fbauth_www
%%%fbauth_pic
%%%fbauth_pic portrait
%%%fbauth_pic background
%%%fbauth_pic other
%%%fbauth_tags
%%%fbauth_pubs
%%%endfbauth
 
А пункт 4 "Говоріння" взагалі позорище!
\end{itemize}

%%%fbauth
%%%fbauth_name
\iusr{Olena Skrypka}
%%%fbauth_url
%%%fbauth_place
%%%fbauth_id
%%%fbauth_front
%%%fbauth_desc
%%%fbauth_www
%%%fbauth_pic
%%%fbauth_pic portrait
%%%fbauth_pic background
%%%fbauth_pic other
%%%fbauth_tags
%%%fbauth_pubs
%%%endfbauth
 

Тому що і алфавіт, і правопис український примусово змінювали "науковці" за
вказівкою москви у 20 - 30- ті рр. щоби наблизити їх до російського.

\begin{itemize}
%%%fbauth
%%%fbauth_name
\iusr{Юрий Стебельский}
%%%fbauth_url
%%%fbauth_place
%%%fbauth_id
%%%fbauth_front
%%%fbauth_desc
%%%fbauth_www
%%%fbauth_pic
%%%fbauth_pic portrait
%%%fbauth_pic background
%%%fbauth_pic other
%%%fbauth_tags
%%%fbauth_pubs
%%%endfbauth
 
\textbf{Olena Skrypka} у кожному сэли своя говирка. И только советская власть
помогла возникнуть, стандартизироваться и утвердиться единой для всех сёл
огромной Украины мове.

\begin{itemize}
%%%fbauth
%%%fbauth_name
\iusr{Olena Skrypka}
%%%fbauth_url
%%%fbauth_place
%%%fbauth_id
%%%fbauth_front
%%%fbauth_desc
%%%fbauth_www
%%%fbauth_pic
%%%fbauth_pic portrait
%%%fbauth_pic background
%%%fbauth_pic other
%%%fbauth_tags
%%%fbauth_pubs
%%%endfbauth
 
\textbf{Yuriy Stebelskiy} Ошибаетесь)

\end{itemize}

\end{itemize}

%%%fbauth
%%%fbauth_name
\iusr{Елена Морозова}
%%%fbauth_url
%%%fbauth_place
%%%fbauth_id
%%%fbauth_front
%%%fbauth_desc
%%%fbauth_www
%%%fbauth_pic
%%%fbauth_pic portrait
%%%fbauth_pic background
%%%fbauth_pic other
%%%fbauth_tags
%%%fbauth_pubs
%%%endfbauth
 

Нечуй-Левицький типография Чоколова 1905г (третье издание). Язык такой же.
Пыталась выяснить - внятного ответа не получила. Как версия - украинский для
русскоязычных. (книга из семейной библиотеки, такой же экз есть в Вернардсого)

\ifcmt
  ig https://scontent-cdg2-1.xx.fbcdn.net/v/t39.30808-6/228957872_4192208717515118_3236244246188935693_n.jpg?_nc_cat=111&ccb=1-4&_nc_sid=dbeb18&_nc_ohc=nc2mwZsSZOEAX8Nz3Y9&_nc_ht=scontent-cdg2-1.xx&oh=d1cd8c2cc988d082e532d425433f79f3&oe=61127F24
  width 0.4
\fi

\begin{itemize}
%%%fbauth
%%%fbauth_name
\iusr{Катерина Сотник}
%%%fbauth_url
%%%fbauth_place
%%%fbauth_id
%%%fbauth_front
%%%fbauth_desc
%%%fbauth_www
%%%fbauth_pic
%%%fbauth_pic portrait
%%%fbauth_pic background
%%%fbauth_pic other
%%%fbauth_tags
%%%fbauth_pubs
%%%endfbauth
 

Це надруковано не фонетично "живою" «кулішевкою», а штучно наближеною до
російського правопису т.зв. штучно сформованою "ярижкою" – згідно Валуєвського
указу (ненадовго відміненого у 1906-14 рр.), заборонявся друк книг з відмінною
граматикою, правописом тощо від російської мови. У тому числі – і літери "ї",
"є".

\end{itemize}

%%%fbauth
%%%fbauth_name
\iusr{Катерина Сотник}
%%%fbauth_url
%%%fbauth_place
%%%fbauth_id
%%%fbauth_front
%%%fbauth_desc
%%%fbauth_www
%%%fbauth_pic
%%%fbauth_pic portrait
%%%fbauth_pic background
%%%fbauth_pic other
%%%fbauth_tags
%%%fbauth_pubs
%%%endfbauth
 

– Мовна і мовно-діалектична ситуація також засвідчує це. Всеохоплююче
утвердження літературної української мови (київо-полтавсько-чернігівського
діалекту) співпадає у часі із досягненням територіальної єдності (фактично,
40-і роки ХХ ст.) і постає в основному як офіційно нав’язуване явище.
Складність у мовному єднанні українського етносу полягає також і в тому, що у
ХХ столітті він змушений був бути об’єктом мовно-граматичних експериментів з
боку тих чи інших наукових чи політичних угруповань, які водночас виступали
виконавцями різноманітних політичних замовлень (від Скрипника і Шумського до
Сталіна і Суслова включно). Ця «експериментна» тенденція у мовній сфері
«благополучно» відродилася у 90-ті роки ХХ ст., перейшовши у століття ХХІ.

%%%fbauth
%%%fbauth_name
\iusr{Катерина Сотник}
%%%fbauth_url
%%%fbauth_place
%%%fbauth_id
%%%fbauth_front
%%%fbauth_desc
%%%fbauth_www
%%%fbauth_pic
%%%fbauth_pic portrait
%%%fbauth_pic background
%%%fbauth_pic other
%%%fbauth_tags
%%%fbauth_pubs
%%%endfbauth
 

– Україна як єдине територіально-державне ціле утворилася фактично у 30–40-х
роках ХХ століття. До того часу, на протязі багатьох сотень та десятків років
її регіони перебували у складі різних державних об’єднань, сфері впливу
досить-таки протилежних культурно-духовних орієнтацій. Українське населення цих
регіонів мимоволі поступово переймало стиль мислення, соціалізації та
світосприйняття домінуючих народів і титульних націй.

-Центром формування та консолідації українського етносу ще в київо-руський час
стало Середнє Подніпров´я – Київщина, Переяславщина і Чернігово-Сіверщина. Воно
поступово набувало провідного значення у суспільно-політичному та
культурно-духовному житті. Саме тут утворилося етнічне ядро українства, до
якого на протязі усієї історії тяжіло населення інших українських земель.  , з
ХІІІ століття етнічний, соціальний, державотворчий та культурно-духовний
розвиток українства відбувався в умовах територіальної розчленованості.

– Етнічні процеси у Подніпров’ї, існування на його території козацтва,
Запорізької Січі сприяли формуванню етнічної самосвідомості, утвердження тут
назв «українці», «український народ». Назва національного ядра ставала
загальноетнічним іменем народу і етнічної території. На протязі історичного
розвитку українського етносу у ситуації політико-культурного розділу,
іноземного панування, не могла скластися тісна економічна, політична та
культурна спільність етносу. За таких умов окремі регіонально-етнічні групи
українців мали відмінності у мові, побуті, культурі, світосприйнятті. В
українському етносі сформувалися етнічно-локальні, етнолігвістичні,
етнокультурні, етносоціальні, етнорелігійні спільності – субетноси:

– З ХV–ХVII ст. українці відомі в Європі як окремий народ з власною етнічною
територією, мовою, побутом, з поділом населення на ряд етнічно-локальних груп –
волинян, полян, батюків, буковинців, верховинців, галичан, карпаторусів,
ополян, підгірців, пінчуків, поліщуків, польовиків, райків, тавричан,
угрорусів, червонорусів, чорноморців. У цей період в офіційному та
повсякденному вживанні Московської Русі до населення Подніпров’я та
Наддніпрянщини застосовується термін «черкаси», рівнозначний етноніму
«українці».

-Етнічно-локальний район північного Лівобережжя – Сіверщина
(Чернігово-Сіверська земля) заселений в основному севрюками або сіверянами –
нащадками східнослов’янського племені сіверян.

-Слобідська Україна або Слобожанщина (східні землі) входила до складу
російської держави, а її населення склалося із автохтонних росіян та
українських переселенців з Речі Посполитої, білорусів, волохів, сербів.

-Волинь (північно-західні землі) була споконвічно заселена автохтонним
східнослов’янським населенням, яке завжди називало себе волинянами.

Західна частина української етнічної території – Бойківщина, Лемківщина,
Гуцульщина, Закарпаття, Буковина, Прикарпаття – заселена етнокультурними та
етнолінгвістичними групами (бойки, лемки, гуцули, поляни, угроруси,
червоноруси, тощо) і завжди мала суттєві етно-локальні особливості
лінгвістичного, побутового, політичного характеру.

Таким чином, незважаючи на формування «національного ядра» і розвиток етнічної
самосвідомості, об’єктивні труднощі ставали значною перешкодою у всіх
різновидах комунікації між частинами розділеного етносу.

%%%fbauth
%%%fbauth_name
\iusr{Арт Юрковская}
%%%fbauth_url
%%%fbauth_place
%%%fbauth_id
%%%fbauth_front
%%%fbauth_desc
%%%fbauth_www
%%%fbauth_pic
%%%fbauth_pic portrait
%%%fbauth_pic background
%%%fbauth_pic other
%%%fbauth_tags
%%%fbauth_pubs
%%%endfbauth
 

Тот язык, который мы учили в школе - унифицированный язык в 1923 г. Институтом
мовознавства, который и был для этого создан. Зачем? Для того чтобы наконец
преподавать в его школах. Не все знают, а многие предпочитают не знать, что 95\%
населения не имело возможности получать образование и были безграмотны. Дети
работали, а не учились.

Для всеобщего обучения на родном,украинском языке - надо было сначала построить
в каждом селе школу, сотни тысяч школ. Для общего языка для всей Украины с ее
местными диалектами надо было разработать единый понятный язык.Написать его
грамматику(которой было несколько вариантов), создать единый алфавит, составить
и напечатать словари, потом написать учебник и напечатать его тиражом сотни
тысяч ( представьте какие были это деньги в стране, которая после гражданской
войны), открыть факультеты для подготовки учителей украинского
языка,переводчиков. Выделить деньги на печать книг, в том числе научных.
Выделить деньги на создание украинских издательств, на гонорары писателяи и
поэтам.Подготовить тысячи редакторов - и сначала их выучить унифицированному
языку ( вместо диалектов). Построить тысячи школ и подготовить десятки тысяч
учителей украинского языка. 

Это была дорогая и беспрецендентная по сложности работа. При царе такого ничего
не было от слова вообще. Поэтому - тот язык ,который мы называем "украинский
литературный" является синтезированным новоделом и мало кто толком его знает.
Почему? Потому что городская интеллигенция как говорила на русском ,так и
продолжила, а сельские люди говорят на своем местном украинском языке, который
обозвали "суржиком".А тот язык ,который был при царе у украинской интеллигенции
- тоже получается экзотическим. Вот такое положение... Недавно ехала с одной
сельской женщиной ,бригадиром на ферме. Какой язык! Чудо! Невозможно
наслушаться! Живой, образный ,вкусный! Звучит естественно, музыкально. Какие
обороты! Как все слажено - и мысль и образ! Давно такого не слышала. И да - это
"суржик". Но на самом деле это и есть древний народный украинский язык Киевской
области, который я слышала в селе у бабушки под Киевом. Еще замечу ,что язык на
котором говорили в 17-18 веке очень отличен от современного - читаем
оцифрованные документы или например письма Мазепы к Мотре Кочубей. Он
отличается полонизмами до такой степени, что невозможно читать. 300 лет
оккупации Руси Польшей сказались на языке очень сильно. Половина документов
вообще написана на латинице с такими польщизмами и германизмами, что это как
иностранный язык выглядит для современного украинца. Читать невозможно от слова
совсем. То есть и русский и украинский сложились во второй половине 19 века.

\begin{itemize}
%%%fbauth
%%%fbauth_name
\iusr{Катерина Сотник}
%%%fbauth_url
%%%fbauth_place
%%%fbauth_id
%%%fbauth_front
%%%fbauth_desc
%%%fbauth_www
%%%fbauth_pic
%%%fbauth_pic portrait
%%%fbauth_pic background
%%%fbauth_pic other
%%%fbauth_tags
%%%fbauth_pubs
%%%endfbauth
 

Складність у мовному єднанні українського етносу полягає також і в тому, що у
ХХ столітті він змушений був бути об’єктом мовно-граматичних експериментів з
боку тих чи інших наукових чи політичних угруповань, які водночас виступали
виконавцями різноманітних політичних замовлень (від Скрипника і Шумського до
Сталіна і Суслова включно). Ця «експериментна» тенденція у мовній сфері
«благополучно» відродилася у 90-і роки ХХ ст., перейшовши у століття ХХІ.

Дане становище не сприяє перетворенню мови на визначальний націотворчий фактор.
Мова може бути значним допоміжним засобом націотворення на фоні
політико-економічної єдності. Мова є дійсно однією із головних ознак етнічної
спільності. Історія мови тісно пов’язана з історією її носія. Нові риси мови
можуть розповсюджуватися з однієї території на іншу у результаті високого
соціально-політичного престижу носіїв мови (говору, діалекту), зумовленого
суспільними й культурними чинниками. У мову етносу чи його частини, особливо
через пограничне двомовне населення можуть проникати елементи іншої спорідненої
чи й неспорідненої мови.

Тільки спільні внутрішні економіко-політичні та суспільно-громадянські
тенденції розвитку у поєднанні з активними зв’язками між населенням
забезпечують мовному явищу перспективу поширення. Мова, носії якої розселені на
значній території, до того ж, розподіленої політичними кордонами, підпадає під
чужорідний культурний вплив, розгалужується на менші одиниці – наріччя,
діалекти, говори, говірки зі специфічними особливостями у звукових,
граматичних, словникових системах.

Літературно-писемна мова є наддіалектичним феноменом, є хронологічно пізнішою
за мову повсякденного спілкування і, водночас, є знаряддям зв’язків, що охоплює
усю етнічну територію чи більшість її і є у підсумку засобом консолідації
народу, творення єдиної духовної культури.

\begin{itemize}
%%%fbauth
%%%fbauth_name
\iusr{Катерина Сотник}
%%%fbauth_url
%%%fbauth_place
%%%fbauth_id
%%%fbauth_front
%%%fbauth_desc
%%%fbauth_www
%%%fbauth_pic
%%%fbauth_pic portrait
%%%fbauth_pic background
%%%fbauth_pic other
%%%fbauth_tags
%%%fbauth_pubs
%%%endfbauth
 

У процесі комунікації спочатку в усному мовленні, а потім і в мові етносу на
окремій території виробляються спільні тенденції, які, там де зв’язки (преш за
все, – економіко-господарські) між населенням є тісними та інтенсивними,
починають поширюватися. Однак у ситуації економічного і політичного
розмежування етнічної території усі різновиди зв’язків між етносом ускладнені і
перетворення мови на націотворчий фактор є проблематичним.

\end{itemize}

%%%fbauth
%%%fbauth_name
\iusr{Арт Юрковская}
%%%fbauth_url
%%%fbauth_place
%%%fbauth_id
%%%fbauth_front
%%%fbauth_desc
%%%fbauth_www
%%%fbauth_pic
%%%fbauth_pic portrait
%%%fbauth_pic background
%%%fbauth_pic other
%%%fbauth_tags
%%%fbauth_pubs
%%%endfbauth
 

Да, но работа над созданием "всеукраинского" языка началась по инициативе
советской власти в 1923г. Такую громадную работу провести! И это после двух войн
и немецкой оккупации! И это в то время, когда любой товар за границей покупался
только за зерно!  · 


%%%fbauth
%%%fbauth_name
\iusr{Арт Юрковская}
%%%fbauth_url
%%%fbauth_place
%%%fbauth_id
%%%fbauth_front
%%%fbauth_desc
%%%fbauth_www
%%%fbauth_pic
%%%fbauth_pic portrait
%%%fbauth_pic background
%%%fbauth_pic other
%%%fbauth_tags
%%%fbauth_pubs
%%%endfbauth
 

Игор Потабачний! Этот новый синтезированный и унифицированный "литературный "
украинский язык никто не употреблял. Село говорит на диалектах, а городская
интеллигенция на русском. Да - в школах изучали, но речь идет об его массовом
употреблении в быту.

\begin{itemize}
%%%fbauth
%%%fbauth_name
\iusr{Александр Моргацкий}
%%%fbauth_url
%%%fbauth_place
%%%fbauth_id
%%%fbauth_front
%%%fbauth_desc
%%%fbauth_www
%%%fbauth_pic
%%%fbauth_pic portrait
%%%fbauth_pic background
%%%fbauth_pic other
%%%fbauth_tags
%%%fbauth_pubs
%%%endfbauth
 
\textbf{Арт Юрковская} Институт мовознавства, был создан в 30 году, позже , на
основе языковедческих комиссий. Эти комиссии активно насыщали мову галицизмами
(они же полонизмы ) и активно же упраздняли слова общие с русским .
\end{itemize}

%%%fbauth
%%%fbauth_name
\iusr{Ihor Potabachniy}
%%%fbauth_url
%%%fbauth_place
%%%fbauth_id
%%%fbauth_front
%%%fbauth_desc
%%%fbauth_www
%%%fbauth_pic
%%%fbauth_pic portrait
%%%fbauth_pic background
%%%fbauth_pic other
%%%fbauth_tags
%%%fbauth_pubs
%%%endfbauth
 
\textbf{Арт Юрковская} "мало кто толком знает" - те саме можна сказати і про російську мову))


%%%fbauth
%%%fbauth_name
\iusr{Maksim Pestun}
%%%fbauth_url
%%%fbauth_place
%%%fbauth_id
%%%fbauth_front
%%%fbauth_desc
%%%fbauth_www
%%%fbauth_pic
%%%fbauth_pic portrait
%%%fbauth_pic background
%%%fbauth_pic other
%%%fbauth_tags
%%%fbauth_pubs
%%%endfbauth
 
\textbf{Арт Юрковская} в каждой области свой

\end{itemize}

%%%fbauth
%%%fbauth_name
\iusr{Irina Korol}
%%%fbauth_url
%%%fbauth_place
%%%fbauth_id
%%%fbauth_front
%%%fbauth_desc
%%%fbauth_www
%%%fbauth_pic
%%%fbauth_pic portrait
%%%fbauth_pic background
%%%fbauth_pic other
%%%fbauth_tags
%%%fbauth_pubs
%%%endfbauth
 

Моя прабабушка 1895 г.р. разговаривала на русском, заметно отличавшемся от
современного русского языка. Умерла в 1991г.

\begin{itemize}
%%%fbauth
%%%fbauth_name
\iusr{Дмитрий Гольдман}
%%%fbauth_url
%%%fbauth_place
%%%fbauth_id
%%%fbauth_front
%%%fbauth_desc
%%%fbauth_www
%%%fbauth_pic
%%%fbauth_pic portrait
%%%fbauth_pic background
%%%fbauth_pic other
%%%fbauth_tags
%%%fbauth_pubs
%%%endfbauth
 
\textbf{Irina Korol} Я тоже помню свою прабабушку, 1892г, и она разговаривала
на вполне понятном ребенку, стандартном русском... ну правда иногда вставляла
словечки на идише 😆

\begin{itemize}
%%%fbauth
%%%fbauth_name
\iusr{Irina Korol}
%%%fbauth_url
%%%fbauth_place
%%%fbauth_id
%%%fbauth_front
%%%fbauth_desc
%%%fbauth_www
%%%fbauth_pic
%%%fbauth_pic portrait
%%%fbauth_pic background
%%%fbauth_pic other
%%%fbauth_tags
%%%fbauth_pubs
%%%endfbauth
 
\textbf{Дмитрий Гольдман} я по-вашему, написала, что моя прабабушка говорила на
НЕПОНЯТНОМ языке? Давайте видеть именно то, что написано.  · 


%%%fbauth
%%%fbauth_name
\iusr{Дмитрий Гольдман}
%%%fbauth_url
%%%fbauth_place
%%%fbauth_id
%%%fbauth_front
%%%fbauth_desc
%%%fbauth_www
%%%fbauth_pic
%%%fbauth_pic portrait
%%%fbauth_pic background
%%%fbauth_pic other
%%%fbauth_tags
%%%fbauth_pubs
%%%endfbauth
 
\textbf{Irina Korol} я вас понял. Вы написали что русский отличался, я написал
- что нет. Остальное - шутка. Жаль что шутку вы не поняли. Удачи.
\end{itemize}

\end{itemize}

%%%fbauth
%%%fbauth_name
\iusr{Vadim Vadim}
%%%fbauth_url
%%%fbauth_place
%%%fbauth_id
%%%fbauth_front
%%%fbauth_desc
%%%fbauth_www
%%%fbauth_pic
%%%fbauth_pic portrait
%%%fbauth_pic background
%%%fbauth_pic other
%%%fbauth_tags
%%%fbauth_pubs
%%%endfbauth
 
Пиду вдягну ботынкы.

% -------------------------------------
\ii{fbauth.sotnik_katerina.kiev_istorii}
% -------------------------------------
 
\textbf{Maksim Pestun}

Кое-что об исторических перипетиях, притеснениях и преференциях "родственных" и "неродственных" мов и языков...

– Часто, в украинско-русском (особенно – российско-русском) пространстве
социальных сетей, СМИ, научных исследованиях, публикациях натыкаешься на вечные
"споры самоутверждения" – чьи предки древнее, чей язык первичнее, чья культура
глубиннее, чьи песнесотрясания мелодичнее, а кто в целом – более велик, и
проч...

* * * * * * * *
\begin{itemize}
\item – «Общѣрусскій праязыкъ распался на отдѣльныя нарѣчія еще въ эпоху доисторическую, въ конце VIII или начале IX вѣка;

\item - образованіе русскаго государства, объединеніе имъ подъ одной державой всѣхъ
русскихъ племенъ не могло привести къ образованію одного общаго народнаго
языка;

\item - оно не успѣло создать и общаго книжнаго языка, ибо книжнымъ языкомъ той эпохи,
языкомъ общимъ и для Кіева, и для Ростова, и для Новгорода и для Галича былъ
вывезенный вмѣстѣ съ книгами изъ Болгаріи церковно-славянскій языкъ.»
\end{itemize}

[А.А. Шахматов «Курс истории русского языка» (1908-1911)]. *

××××××××××××

– В своё время ещё (в Российской империи) русский лингвист, академик Алексей Шахматов отмечал:

"Язык Приднепровья и Красной Руси XI в. – вполне рельефная, вероятно
обозначенная, ярко-индивидуальная единица, и в ней слишком легко и выразительно
можно распознать прямого предка современного малороссийского языка".

* * *

– Академик Агафангел Крымский отмечал:

«В украинском X в. появляется очень мягкий звук «і» из старого дифтонга «ие».
Тогда как на севере этот дифтонг с XI в. начал превращаться на «е», в Киеве XI
в. – на «і» ...

\begin{itemize}
  \item - Это главный признак украинского произношения того времени.
  \item – Этот признак мы находим и в киевских граффити Софийского собора, Золотых ворот и др.: «в морі», «на Желяні», «апріля» и др.
  \item √ В именовании, напр.: «Господи, помози рабу своєму Петрові», «А прізвище (ПРЪЗЪВИЩЬ) моє Саєтат».
  \item – √ : В граффити княжеского Киева XI-XIII вв. фиксируем чисто украинские грамматические признаки:
  \item - звательный падеж существительных: "владыкО, СтефанЕ, головО" и т.д.;
  \item - окончание «у» в родительном падеже единственного числа мужского рода: "спорУ ("з того спорУ");
  \item - форму глаголов без «т»: "печЕ";
  \item - глаголы прошедшего времени, оканчивающихся на «в»: "писаВ, ходиВ, молиВся" и т.д.;
  \item - глаголы с окончанием «ти»: "включиТИ", "писаТИ", "скончаТИ" и т.д.;
  \item - мягкое «ц» в конце слов: "чернеЦЬ", "(помилуй) ВалерЦЯ", "поможи АрхипЦЮ" ...
  \item - прилагательные теряют на конце «я»: "многопечальнА", "благодатнІшА" и т.д.
\end{itemize}

* * *

– Ось напис XI ст., який читаємо в книзі історика Сергія Висоцького (котрий є
ГОЛОВНИМ дослідником і дешифрувальником УСІХ ДАВНЬОРУСЬКИХ графіті Києва,
вимушено виданій в брежнєвські часи російською):

\begin{itemize}
  \item - у виданому російському перекладі: «Мать, не желая ребёнка, бежала прочь...».
  \item - в оригіналі (фотооригінал засвідчує): «Мати, не хотячи дитичя, біжя гет...»
\end{itemize}

* * *
– В граффити « Господи, помози рабу своєму Луці, владичину дяку...» и подобных –
идёт переход «к» в «ц» в дательном падеже единственного числа (Лука – ЛуЦІ).

И изменение согласной «к» перед суффиксом «ин» на «ч» ("владика" – владиЧИн) –
в полном соответствии с современным украинским правописанием.

*******

– Ну, а об "неотличимости" украинского языка от языка северо-восточных
"кузенов" – это "былины" ещё из XVI-XVII вв., хорошо характеризирующая и
некоторых нынешних россиян, упорно постулирующих “неразрывное сродство“ этих
языков:

– Например, 1626 г. Киевский митрополит Иосиф Краковский составил акафист
Святой Варваре. Московская Церковь захотела закупить часть тиража, но при
условии его перевода на владимиро-суздальско-тверской вариант (тогдашний
русский). Заодно синод приказал собрать со всех церквей Московского царства
книги украинской печати и не использовать.

Где девать украинские книги – решилось почти сразу: в 1627 г. указом царя
Михаила приказано было книги украинской печати собрать и сжечь.

И уже в 1693 г. "увенчался" письмом Московского патриарха о запрете печатания
любых книг на украинском языке.

P.S. Подробнее см. у меня на странице, иначе главадмин мне сделает "фэ" за
размещение ссылок.

%%%fbauth
%%%fbauth_name
\iusr{Катерина Сотник}
%%%fbauth_url
%%%fbauth_place
%%%fbauth_id
%%%fbauth_front
%%%fbauth_desc
%%%fbauth_www
%%%fbauth_pic
%%%fbauth_pic portrait
%%%fbauth_pic background
%%%fbauth_pic other
%%%fbauth_tags
%%%fbauth_pubs
%%%endfbauth
 
\textbf{Maksim Pestun}

Фактично, унормування різноманітних різновидів української літературної мови
розпочалося в останній третині 19 ст. П. Кулішем.

Гігантський внесок (у вигляді "живої" наддніпрянської говірки) було у цей
період здійснено І.Нечуй-Левицьким (у більшості перевидань можна спостерегти,
як він шліфував слова, оберти тощо, намагаючись позбавитися від "штучності",
наближаючи до повсякденної вимови та їхнього використання).

Хоча вони взаємно й критикували один одного.

Фактично, літературна мова, на основі наддніпрянських говірок була сформована
ще у кінці 18 ст. Котляревським.

Квітка-Основ'яненко, будучи мешканцем Харківщини і з патріотичних причин,
прагнучи зробити внесок у внесення укр.мови у широкий вжиток творів та
публіцистики, чомусь відійшов від "котлярівщини", формуючи якусь свою
"кабінетну" квазіукрмову.

Це можна спостерегти, читаючи його україномовні (нечитабельні) та
російськомовні (дуже легко читабельні, візміть, хоча б того ж "Пана
Халявського").

Аналогічна ситуація була у 1920-х рр., коли під впливом НТШ, Скрипник та
Шумський створили взагалі відірваний від живої мови правопис, з глобальним
введенням у нього наддністрянських, зокрема галиційських субетносних говірок,
діалектизмів та правописних форм.

\begin{itemize}
%%%fbauth
%%%fbauth_name
\iusr{Volodymyr Nekrasov}
%%%fbauth_url
%%%fbauth_place
%%%fbauth_id
%%%fbauth_front
%%%fbauth_desc
%%%fbauth_www
%%%fbauth_pic
%%%fbauth_pic portrait
%%%fbauth_pic background
%%%fbauth_pic other
%%%fbauth_tags
%%%fbauth_pubs
%%%endfbauth
 
\textbf{Катерина Сотник}

\ifcmt
  ig https://scontent-cdg2-1.xx.fbcdn.net/v/t1.6435-9/230197883_10219048073304006_1133266871565465035_n.jpg?_nc_cat=111&ccb=1-4&_nc_sid=dbeb18&_nc_ohc=lpTSffVHL3kAX8xpFjI&_nc_ht=scontent-cdg2-1.xx&oh=e1a1296162905bbca1a2ec96ea396901&oe=6134052A
  width 0.3
\fi

\begin{itemize}
%%%fbauth
%%%fbauth_name
\iusr{Катерина Сотник}
%%%fbauth_url
%%%fbauth_place
%%%fbauth_id
%%%fbauth_front
%%%fbauth_desc
%%%fbauth_www
%%%fbauth_pic
%%%fbauth_pic portrait
%%%fbauth_pic background
%%%fbauth_pic other
%%%fbauth_tags
%%%fbauth_pubs
%%%endfbauth
 
\textbf{Volodymyr Nekrasov}
Щиро дякую, спасибі !
\end{itemize}

%%%fbauth
%%%fbauth_name
\iusr{Volodymyr Nekrasov}
%%%fbauth_url
%%%fbauth_place
%%%fbauth_id
%%%fbauth_front
%%%fbauth_desc
%%%fbauth_www
%%%fbauth_pic
%%%fbauth_pic portrait
%%%fbauth_pic background
%%%fbauth_pic other
%%%fbauth_tags
%%%fbauth_pubs
%%%endfbauth
 
\textbf{Катерина Сотник} натрапив на цікаву дискусію між читачем і редактором записана у Віснику НТШ, Львів 1903р.

\ifcmt
  ig https://scontent-cdg2-1.xx.fbcdn.net/v/t1.6435-9/228024794_10219048070103926_4842252995529398364_n.jpg?_nc_cat=111&ccb=1-4&_nc_sid=dbeb18&_nc_ohc=cJEVJCAsU98AX8Gshat&_nc_ht=scontent-cdg2-1.xx&oh=75628b7400ad49aa32948cf675bb37e9&oe=61320A72
  width 0.3
\fi

\end{itemize}

%%%fbauth
%%%fbauth_name
\iusr{Natali Kazakova}
%%%fbauth_url
%%%fbauth_place
%%%fbauth_id
%%%fbauth_front
%%%fbauth_desc
%%%fbauth_www
%%%fbauth_pic
%%%fbauth_pic portrait
%%%fbauth_pic background
%%%fbauth_pic other
%%%fbauth_tags
%%%fbauth_pubs
%%%endfbauth
 
як важко читати українську мову, записану російською (


%%%fbauth
%%%fbauth_name
\iusr{Катерина Сотник}
%%%fbauth_url
%%%fbauth_place
%%%fbauth_id
%%%fbauth_front
%%%fbauth_desc
%%%fbauth_www
%%%fbauth_pic
%%%fbauth_pic portrait
%%%fbauth_pic background
%%%fbauth_pic other
%%%fbauth_tags
%%%fbauth_pubs
%%%endfbauth
 
\textbf{Maksim Pestun}

Можливо, цей видавець Йогансен був пращуром величного творця зі спільноти т.зв.
«Розстріляного Відродження» ?

Майк (справжнє ім’я – Михайло) Йогансен (1896-1937) — україн­ський поет,
прозаїк, перекладач. Іноді його називають найменш дослідженим письменником
своєї епохи.

Він полишив широке поле для інтерпретацій —літературознавці охоче й сміливо
записують його то до формалістів, то до конструктивістів, то до авангардистів,
то до футуристів.

Свої перші вірші Йогансен писав російською та німецькою мовами, потім остаточно
перейшов на укра­їнську. І в поезії, і в прозі він зберігав творчу
індивідуальність, часто вдавався до експериментів — поєднував прозу й поезію в
одному творі, до містифікацій. За сімнадцять років творчої діяльності видав
вісім книг віршів, десять прози, чотири книжки для дітей та дві з
літературознавства.

1937 року був заарештований та страчений.

\begin{itemize}
%%%fbauth
%%%fbauth_name
\iusr{Катерина Сотник}
%%%fbauth_url
%%%fbauth_place
%%%fbauth_id
%%%fbauth_front
%%%fbauth_desc
%%%fbauth_www
%%%fbauth_pic
%%%fbauth_pic portrait
%%%fbauth_pic background
%%%fbauth_pic other
%%%fbauth_tags
%%%fbauth_pubs
%%%endfbauth
 
\textbf{Maksim Pestun}
Благодарю.


%%%fbauth
%%%fbauth_name
\iusr{Maksim Pestun}
%%%fbauth_url
%%%fbauth_place
%%%fbauth_id
%%%fbauth_front
%%%fbauth_desc
%%%fbauth_www
%%%fbauth_pic
%%%fbauth_pic portrait
%%%fbauth_pic background
%%%fbauth_pic other
%%%fbauth_tags
%%%fbauth_pubs
%%%endfbauth
 
Это другой Иогансен. Они были знакомы по Питеру. У того были нотные магазины. Семья погибла в 33-37 гг.

\ifcmt
  ig https://scontent-cdg2-1.xx.fbcdn.net/v/t39.30808-6/228177594_2978770649059763_1759544674336033208_n.jpg?_nc_cat=108&ccb=1-4&_nc_sid=dbeb18&_nc_ohc=q-XyYVDuOHYAX8jgi6E&_nc_oc=AQn4w9WnnKXzxwIv9dB09dMV-AcLM8ThmZNKYlNVcW8uso6nY4IOZkHbCsJHYQvdaQE&_nc_ht=scontent-cdg2-1.xx&oh=e23e265bf1fa15d038764c6274475e3a&oe=61114417
  width 0.4
\fi

%%%fbauth
%%%fbauth_name
\iusr{Катерина Сотник}
%%%fbauth_url
%%%fbauth_place
%%%fbauth_id
%%%fbauth_front
%%%fbauth_desc
%%%fbauth_www
%%%fbauth_pic
%%%fbauth_pic portrait
%%%fbauth_pic background
%%%fbauth_pic other
%%%fbauth_tags
%%%fbauth_pubs
%%%endfbauth
 
\textbf{Maksim Pestun}
Добрэ.


%%%fbauth
%%%fbauth_name
\iusr{Maksim Pestun}
%%%fbauth_url
%%%fbauth_place
%%%fbauth_id
%%%fbauth_front
%%%fbauth_desc
%%%fbauth_www
%%%fbauth_pic
%%%fbauth_pic portrait
%%%fbauth_pic background
%%%fbauth_pic other
%%%fbauth_tags
%%%fbauth_pubs
%%%endfbauth
 
Я спрошу у товарища!

\end{itemize}

%%%fbauth
%%%fbauth_name
\iusr{наталья колесник}
%%%fbauth_url
%%%fbauth_place
%%%fbauth_id
%%%fbauth_front
%%%fbauth_desc
%%%fbauth_www
%%%fbauth_pic
%%%fbauth_pic portrait
%%%fbauth_pic background
%%%fbauth_pic other
%%%fbauth_tags
%%%fbauth_pubs
%%%endfbauth
 
Вот вам и суржик🤔 любезный друг

\begin{itemize}
%%%fbauth
%%%fbauth_name
\iusr{Volodymyr Nekrasov}
%%%fbauth_url
%%%fbauth_place
%%%fbauth_id
%%%fbauth_front
%%%fbauth_desc
%%%fbauth_www
%%%fbauth_pic
%%%fbauth_pic portrait
%%%fbauth_pic background
%%%fbauth_pic other
%%%fbauth_tags
%%%fbauth_pubs
%%%endfbauth
 
\textbf{наталья колесник} є чіткі визначення щодо суржика, діалекта, говірки,
жаргона, сленга. Знайомство з різними діалектами було навіть в радянські часи,
в програмі вивчення укр.мови у середній школі.


%%%fbauth
%%%fbauth_name
\iusr{Maksim Pestun}
%%%fbauth_url
%%%fbauth_place
%%%fbauth_id
%%%fbauth_front
%%%fbauth_desc
%%%fbauth_www
%%%fbauth_pic
%%%fbauth_pic portrait
%%%fbauth_pic background
%%%fbauth_pic other
%%%fbauth_tags
%%%fbauth_pubs
%%%endfbauth
 
не уверен, что классик украинской литературы писал на суржике...

\begin{itemize}
%%%fbauth
%%%fbauth_name
\iusr{Maksim Pestun}
%%%fbauth_url
%%%fbauth_place
%%%fbauth_id
%%%fbauth_front
%%%fbauth_desc
%%%fbauth_www
%%%fbauth_pic
%%%fbauth_pic portrait
%%%fbauth_pic background
%%%fbauth_pic other
%%%fbauth_tags
%%%fbauth_pubs
%%%endfbauth
 
хотя по сути похоже!
\end{itemize}
\end{itemize}

%%%fbauth
%%%fbauth_name
\iusr{Лара Чернухина}
%%%fbauth_url
%%%fbauth_place
%%%fbauth_id
%%%fbauth_front
%%%fbauth_desc
%%%fbauth_www
%%%fbauth_pic
%%%fbauth_pic portrait
%%%fbauth_pic background
%%%fbauth_pic other
%%%fbauth_tags
%%%fbauth_pubs
%%%endfbauth
 
Це дійсно цікаво🤔 змішана абетка...

\begin{itemize}
%%%fbauth
%%%fbauth_name
\iusr{Харченко Олег}
%%%fbauth_url
%%%fbauth_place
%%%fbauth_id
%%%fbauth_front
%%%fbauth_desc
%%%fbauth_www
%%%fbauth_pic
%%%fbauth_pic portrait
%%%fbauth_pic background
%%%fbauth_pic other
%%%fbauth_tags
%%%fbauth_pubs
%%%endfbauth
 
\textbf{Лара Чернухина} це "ярижка". Нею писали до того, як Куліш розробив
український алфавіт. А потім Емський указ "кулишівку" заборонив, та знову
друеували "ярижкою".
\end{itemize}

%%%fbauth
%%%fbauth_name
\iusr{Людмила Мозговая}
%%%fbauth_url
%%%fbauth_place
%%%fbauth_id
%%%fbauth_front
%%%fbauth_desc
%%%fbauth_www
%%%fbauth_pic
%%%fbauth_pic portrait
%%%fbauth_pic background
%%%fbauth_pic other
%%%fbauth_tags
%%%fbauth_pubs
%%%endfbauth
 
Интересный украинский был...

\begin{itemize}
%%%fbauth
%%%fbauth_name
\iusr{Харченко Олег}
%%%fbauth_url
%%%fbauth_place
%%%fbauth_id
%%%fbauth_front
%%%fbauth_desc
%%%fbauth_www
%%%fbauth_pic
%%%fbauth_pic portrait
%%%fbauth_pic background
%%%fbauth_pic other
%%%fbauth_tags
%%%fbauth_pubs
%%%endfbauth
 
\textbf{Людмила Мозговая} просто Квитка-Основ'яненко писал в те времена, когда
вынужден был пользоваться российским алфавитом. На звучание мовы это не
отобразилось. Просто, прочитайте текст вслух.
\end{itemize}

%%%fbauth
%%%fbauth_name
\iusr{Maryna Chemerys}
%%%fbauth_url
%%%fbauth_place
%%%fbauth_id
%%%fbauth_front
%%%fbauth_desc
%%%fbauth_www
%%%fbauth_pic
%%%fbauth_pic portrait
%%%fbauth_pic background
%%%fbauth_pic other
%%%fbauth_tags
%%%fbauth_pubs
%%%endfbauth
 

Дореволюційні видання "Кобзаря" теж мали подібний вигляд

\ifcmt
  ig https://scontent-cdg2-1.xx.fbcdn.net/v/t39.30808-6/230187903_4208617919204206_5935384992975787112_n.jpg?_nc_cat=107&ccb=1-4&_nc_sid=dbeb18&_nc_ohc=2FhnqxBKCHgAX9rD4UO&_nc_ht=scontent-cdg2-1.xx&oh=a71e9323997e4d1ff6075d0aab64f63e&oe=61117D02
  width 0.4
\fi

\begin{itemize}
%%%fbauth
%%%fbauth_name
\iusr{Харченко Олег}
%%%fbauth_url
%%%fbauth_place
%%%fbauth_id
%%%fbauth_front
%%%fbauth_desc
%%%fbauth_www
%%%fbauth_pic
%%%fbauth_pic portrait
%%%fbauth_pic background
%%%fbauth_pic other
%%%fbauth_tags
%%%fbauth_pubs
%%%endfbauth
 
\textbf{Maryna Chemerys} пункт 2 Емського указу (1876), дозволяв друкувати
українською обов'язково "...без отступления от общерусской орфографии ".
\end{itemize}

%%%fbauth
%%%fbauth_name
\iusr{Харченко Олег}
%%%fbauth_url
%%%fbauth_place
%%%fbauth_id
%%%fbauth_front
%%%fbauth_desc
%%%fbauth_www
%%%fbauth_pic
%%%fbauth_pic portrait
%%%fbauth_pic background
%%%fbauth_pic other
%%%fbauth_tags
%%%fbauth_pubs
%%%endfbauth
 
Я погуглив для вас
\url{https://uk.m.wikipedia.org/wiki/Ярижка}

\begin{itemize}
%%%fbauth
%%%fbauth_name
\iusr{Volodymyr Nekrasov}
%%%fbauth_url
%%%fbauth_place
%%%fbauth_id
%%%fbauth_front
%%%fbauth_desc
%%%fbauth_www
%%%fbauth_pic
%%%fbauth_pic portrait
%%%fbauth_pic background
%%%fbauth_pic other
%%%fbauth_tags
%%%fbauth_pubs
%%%endfbauth
 
\textbf{Харченко Олег} ще цікавою є латинська абетка - Лучуківка
\end{itemize}

%%%fbauth
%%%fbauth_name
\iusr{Natalya Eydelman}
%%%fbauth_url
%%%fbauth_place
%%%fbauth_id
%%%fbauth_front
%%%fbauth_desc
%%%fbauth_www
%%%fbauth_pic
%%%fbauth_pic portrait
%%%fbauth_pic background
%%%fbauth_pic other
%%%fbauth_tags
%%%fbauth_pubs
%%%endfbauth
 
по-нiмецьки с "i" , а всё остальное с "и"


%%%fbauth
%%%fbauth_name
\iusr{Daria Dmytrieva}
%%%fbauth_url
%%%fbauth_place
%%%fbauth_id
%%%fbauth_front
%%%fbauth_desc
%%%fbauth_www
%%%fbauth_pic
%%%fbauth_pic portrait
%%%fbauth_pic background
%%%fbauth_pic other
%%%fbauth_tags
%%%fbauth_pubs
%%%endfbauth
 
У 1833 році жодних норм правопису не існувало. Кожен писав, як вважав собі правильним.

\end{itemize}

