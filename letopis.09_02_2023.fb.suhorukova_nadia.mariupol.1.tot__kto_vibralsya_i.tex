%%beginhead 
 
%%file 09_02_2023.fb.suhorukova_nadia.mariupol.1.tot__kto_vibralsya_i
%%parent 09_02_2023
 
%%url https://www.facebook.com/permalink.php?story_fbid=pfbid0CGcStoqAdfuRVmfXGwuPJFbAVafcGF8A91YWgo4rxy58YF5cgRwvPTqTDGHqyRRJl&id=100087641497337
 
%%author_id suhorukova_nadia.mariupol
%%date 09_02_2023
 
%%tags mariupol,mariupol.war
%%title Тот, кто выбрался из Мариуполя - грезит этим городом, тоскует по нему,  томится без него
 
%%endhead 

\subsection{Тот, кто выбрался из Мариуполя - грезит этим городом, тоскует по нему,  томится без него}
\label{sec:09_02_2023.fb.suhorukova_nadia.mariupol.1.tot__kto_vibralsya_i}

\Purl{https://www.facebook.com/permalink.php?story_fbid=pfbid0CGcStoqAdfuRVmfXGwuPJFbAVafcGF8A91YWgo4rxy58YF5cgRwvPTqTDGHqyRRJl&id=100087641497337}
\ifcmt
 author_begin
   author_id suhorukova_nadia.mariupol
 author_end
\fi

Тот, кто выбрался из Мариуполя -  грезит этим городом, тоскует по нему,
томится без него.  

Пишет посты, выкладывает фото.  Ковыряет раны палкой. 

Это очень больно. 

Больнее, чем сидеть под бомбами. 

Тогда мы знали, что нужно выжить. 

Любой ценой.

Или умереть, но сразу. 

Не мучаться от ран, не истекать кровью, не выть  под завалами. 

Мы молились о спасении или счастливой смерти.

Или -  или.

Чтобы прямое попадание. 

Или многотонная бомба. 

Только  насквозь – от девятого до первого этажей. 

Без мучений. 

На них  нет ни сил, ни отваги. 

Когда кричишь от боли, тебя боятся и перестают любить.

Не знают, как помочь. 

Хотят, чтобы ты замолчал. 

Мы -  другие. Наша планета - мертвый город у Азовского моря.

В Литве, я  обрадовалась, когда услышала  от волонтера украинский язык.

Спросила, откуда она.

Женщина  ответила: "Я из Киева. Я как и вы". 

Меня передернуло. 

"Нет, не как и я. Я из Мариуполя". 

Она меня поняла: "Давно?"

Ответить я не смогла. Мне не хватило воздуха. 

Другая женщина, литовка, спросила: "Было страшно?"

Я  кивнула.

А потом давилась слезами и молчала. 

Как я могу объяснить   ей весь ужас? 

Тоску по уничтоженному городу.

По  мертвым высоткам, по погибшим  близким. 

Передать словами холод подвала, в котором со мной были мои маленькие
племянники. 

Дать почувствовать дикий страх, вгрызающийся в душу,  и сжимающий ледяной рукой
сердце и горло. 

Как это можно рассказать за  пять минут? 

Как объяснить, что мы, мариупольцы, другие?

У нас  навсегда  украли наше   счастье.

Я жду стадию принятия. Один человек сказал, что она  будет, и я приму все, что
у меня отняли.

Я смирюсь и не буду пробивать головой стену. 

Не буду сходит с ума,  когда ненависть зашкаливает и хочется отомстить. 

Вернуться в город и отомстить тем, кто убивал нас. 

Тем, кто не давал ни одного шанса. 

Кто давил на гашетку в самолете, перезаряжал орудия,  наезжал танком на наши
живые души.

Кто-то из нас научился смеяться, радоваться мелочам, кормить бездомных котиков,
высаживать цветы в чужой стране, находить новых знакомых и не отвечать на
вопросы о боли и страхе. 

Если мы улыбаемся, это не значит, что мы стали прежними. 

Мы теперь другие. 

Наша боль разрослась до космического масштаба, но мы ее прячем от чужих глаз. 

Наш страх  не только о  Мариуполе. 

Мы боимся  за Харьков, Бахмут,  Запорожье,  Северодонецк, Николаев, Одессу и
другие украинские города. 

Мы знаем, как это, когда на твоих глазах умирает родной город, где до сих пор
есть люди. 

Наши люди.

Я написала это больше полугода назад. 

Этот текст погиб вместе с моим аккаунтом, но остался в книге. 

А ещё у людей, которые смогли его сохранить. 

Большинство из них мариупольцы. 

Наш город болит внутри нас ещё сильнее.  

Он отдаляется и уходит в безнадежье. 

А мы  по-прежнему хотим вернуться домой. 

В наш украинский Мариуполь.
