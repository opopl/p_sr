% vim: keymap=russian-jcukenwin
%%beginhead 
 
%%file 11_11_2018.fb.lesev_igor.1.po_vyboram_v_ldnr
%%parent 11_11_2018
 
%%url https://www.facebook.com/permalink.php?story_fbid=2168387423192349&id=100000633379839
 
%%author_id lesev_igor
%%date 
 
%%tags dnr,donbass,lnr,ukraina,vojna,vybory
%%title По выборам в ЛДНР
 
%%endhead 
 
\subsection{По выборам в ЛДНР}
\label{sec:11_11_2018.fb.lesev_igor.1.po_vyboram_v_ldnr}
 
\Purl{https://www.facebook.com/permalink.php?story_fbid=2168387423192349&id=100000633379839}
\ifcmt
 author_begin
   author_id lesev_igor
 author_end
\fi

По выборам в ЛДНР.

Формально, бессмысленное и бестолковое действо, учитывая, что они не будут
никем признаны, включая Кремль. Ну разве что Южной Осетией, которая тоже особо
никем не признана... кроме России.

\ifcmt
  ig https://scontent-frx5-1.xx.fbcdn.net/v/t1.6435-9/45808221_2168386986525726_5311227343589605376_n.jpg?_nc_cat=111&ccb=1-5&_nc_sid=730e14&_nc_ohc=5TQHQX-QuaEAX-0yOiU&_nc_ht=scontent-frx5-1.xx&oh=59d0c3c9ee75edfde4d0d3d70ac73a15&oe=61BB25BA
  @width 0.4
  %@wrap \parpic[r]
  @wrap \InsertBoxR{0}
\fi

Но эти выборы имеют определенную практическую функцию. Во-первых, они
легализуют для местного населения военно-полевой режим, существующий уже пятый
год в ЛДНР. Тот же Пушилин становится теперь не назначенцем от хрен пойми кого,
а хоть как-то делегированным парнягой от части Донецкой области.

Во-вторых, сами выборы в системе западных ценностей – это священная корова.
Другого симпатичного инструмента делегирования властных полномочий на Западе
пока что не придумали. А потому Запад по определению не может выступать против
выборов. Теперь там будут говорить, что они прошли «не так», и их нужно
переиначить. Но в любом случае, это уже будет дискурс вокруг некой субъектности
ЛДНР. Особенно, когда на Западе нет единой согласованной позиции по роли России
в этом процессе – оккупировала она часть Донбасса или же «только» поддерживает
сепаратистские образования. А между этими позициями очень гигантская разница.

В-третьих, Россия продавливает то, что и прописано в Минских соглашениях –
прямые переговоров между Киевом и самопровозглашенными республиками. Теперь в
ЛДНР появляется свой «избирательный цикл» (это уже вторые выборы) и
легализуется мытьем и катаньем субъектность местной верхушки.

Позиция Киева здесь существенно ослабевает, хотя бы потому что она всегда тупо
неизменна и никак не трансформируется в зависимости от внешних факторов. В
Киеве опять скажут, что выборы прошли «под дулами автоматов» и вообще, это
пародия на выборы. Последний тезис во многом верен. К выборам в ЛДНР были
допущены только согласованные в Москве ребята, о чем говорит хотя бы
запредельно маленький список кандидатов. Но даже очень специфические выборы в
Чечне с неизменным 90+ процентами за Рамзана не ставят под малейшее сомнение в
РФ и мире легализацию Кадырова.

\ii{11_11_2018.fb.lesev_igor.1.po_vyboram_v_ldnr.cmtfront.1.igor_maximov}

А Пушилин и Пасечник – это теперь не только «шестерки Кремля», но и чуть-чуть
легализованные шестерки, избранные по околодемократической процедуре. Да, можно
оспаривать механизм самой процедуры их избрания. Но в Москве будут предлагать
все это оспаривать Киеву уже непосредственно с Донецком и Луганском.

В-четвертых, Москва легализирует в ЛДНР очень удобных для себя персонажей. Они
пластилиновые. Надо входить «на приемлемых условиях в Украину» - будут входить.
Принимать миротворческую миссию ООН – будут принимать. Строить суверенные
республики – будут строить. Возвращаться к проекту Новороссия – будут
возвращаться.

Киев же вообще не выстраивает никакой своей гибко политики относительно ЛДНР,
хотя формально и юридически претендует на эту территорию. По сути, все
договоренности отданы на откуп старшему брату из Вашингтона. А в Штатах:

а) у самих нет комплексного видения решения «украинского вопроса», кроме
совершенно нефункционального лозунга о «территориальном единстве Украины»,
который в данном случае как раз и блокирует возвращение именно Донбасса Киеву и 

б) вся украинская политика Вашингтона рассматривается через призму
ослабления/вытеснения России с украинского пространства. Для наших
националистов это сказочно благая идея-фикс, но по факту это способствует
только тому, что Россия с Донбасса – который ей нахер не упал – не только не
уходит, но еще и интегрирует его в свое политико-экономическое поле.

Ну а Киев за все эти годы так и не создал в ЛДНР своей проукраинской партии.
Вообще, хотя бы какой-то дискуссионной площадки для местных. Наоборот, все
коммуникации отрубаются как у сумасшедшего хирурга-маньяка.

В итоге, мы получаем еще один шажок, отделяющий две половинки некогда единых
областей от Украины. И самое паршивое во всей этой затянувшейся кровавой
истории, что нам друг на друга все больше становится плевать. А значит все
меньше остается мотивов понять друг друга, а потом прилагать чудовищные усилия,
чтобы скрестить бобра с тумбочкой. Нет так нет. Катимся в разные стороны. Вам
плохо и нам плохо. Вот и все, что нас сейчас объединяет.

\ii{11_11_2018.fb.lesev_igor.1.po_vyboram_v_ldnr.cmt}
