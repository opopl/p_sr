% vim: keymap=russian-jcukenwin
%%beginhead 
 
%%file 14_12_2021.stz.news.ua.radiosvoboda.1.peredovaja
%%parent 14_12_2021
 
%%url https://www.radiosvoboda.org/a/reportazh-iz-peredovoyi-na-donbasi/31608451.html
 
%%author_id chappl_ejmos
%%date 
 
%%tags 
%%title «Моя мама не знає, що я тут»: Репортаж із передової на Донбасі
 
%%endhead 
\subsection{«Моя мама не знає, що я тут»: Репортаж із передової на Донбасі}
\label{sec:14_12_2021.stz.news.ua.radiosvoboda.1.peredovaja}

\Purl{https://www.radiosvoboda.org/a/reportazh-iz-peredovoyi-na-donbasi/31608451.html}

\ifcmt
 author_begin
   author_id chappl_ejmos
 author_end
\fi

\ii{14_12_2021.stz.news.ua.radiosvoboda.1.peredovaja.pic.1}

Історії про затяжну гібридну війну, що триває на сході України.

«Жодного кроку праворуч чи ліворуч з цієї стежки», – тихим голосом каже нам
військова ЗСУ Ірина. Зараз 18:30, але навкруги вже темно. Перед нами поля, де
можуть бути міни, тому ходити тут можна тільки безпечними вузькими стежками.

Двоє солдатів, що вели нас, визначали засніжену дорогу до лінії фронту
інстинктивно, мов лісові звірі: кілька хвилин прямо вниз до лісосмуги, потім
продертися через чагарники, перш ніж повернути востаннє, і перед нами –
відкрите поле. Підійшовши до окопів, вони зупинилися, щоб по рації сповістити
про наше наближення.

\ii{14_12_2021.stz.news.ua.radiosvoboda.1.peredovaja.pic.2}

Я працював з досвідченим посередником з проєкту Donbas Frontliner, щоб
дізнатися, що думають солдати про небезпеку вторгнення. Перебуваючи далеко від
думки американських медіа чи уряду, здається, молоді українці, які зіткнулися
віч-на-віч з озброєними російськими військовими, найкраще мають оцінювати
реальність такої загрози. Переважна більшість відповідає на це запитання –
«Ні». Солдати не вірять у вторгнення. Мій український колега, який в той час
працював на передовій, підтвердив, що ця думка дуже поширена. «Всі солдати
кажуть мені, що вторгнення неможливе. Але це солдати, – додає він, – деякі з
них не були у великому місті протягом семи місяців».

Коли ми прибули в село до місць дислокації українських бійців, то поговорили з
кількома людьми, виконавши тим самим нашу основну репортеську роботу. Але через
короткий світловий день, мені було мало, що показати. Тому ми вирішили
дочекатися похмурого грудневого світанку, щоб зробити ще фото. А потім якомога
швидше вибиратися звідси, скоріш за все, тією ж лісовою стежкою.

План був чудовий, але абсолютно наївний. Не будучи військовим, на війні ви
втрачаєте будь-яку свободу дій. Ви, як дитина, яку потрібно передати
відповідальним дорослим, мало контролюєте те, що станеться й коли саме. Це був
для мене незабутній урок, який я засвоїв вже наступного ранку.

\ii{14_12_2021.stz.news.ua.radiosvoboda.1.peredovaja.pic.3}

Тієї ночі ми знову наділи бронежилети і шоломи та вирушили у темряву неподалік
від форпосту бойовиків, відчуваючи, як під ногами поскрипує підморожена дорога.
Раптом в кількох метрах почулися постріли з автоматичної зброї – здається,
вороги ледь не помітили солдатів.

\ii{14_12_2021.stz.news.ua.radiosvoboda.1.peredovaja.pic.4}

На заставі двоє військових готувались відправитися в окопи на чотиригодинну
бойову зміну. Одна попросила не фотографувати її, оскільки її мама вважає, що
вона працює в офісі далеко від лінії фронту. А фото її єдиної дочки, яка прямує
в темряву з автоматом Калашникова в руці, може бути надто важким для неї, щоб
це витримати.

Коли ми вийшли на вулицю, на село опустився туман. Це жахлива погода для
фронтових частин. У будинку, де ми мали спати, старший солдат похмуро дивився
на екран, на якому зазвичай було видно поле бою. Але зараз він світився
невиразним білим кольором. У таку туманну погоду обидві сторони використовують
«диверсійні» команди, що безшумно підкрадаються до поля бою, тероризуючи
солдатів в окопах, які напружено слухають всю ніч.

Наступного ранку моторошний світ темряви та шепоту, до якого ми прибули,
виявився гарним український селом, мов з листівки. В окопах, що починалися
прямо за сільською хатою, ми фотографували солдатів і різну зброю, ще
радянських часів, включно з кулеметом ПК «покемон». Далі нам було час іти.

Лише коли ми вийшли з села на дорогу, стало зрозуміло, що фактично ми не
полишаємо лінію фронту, а рухаємося безпосередньо через лінію зіткнення на іншу
передову позицію. Ймовірно небезпека була приблизно такою ж, як і раніше, але
відчуття непідконтрольності своїх рішень тривожило й змушувало серце завмирати.

У більшості місць на дорозі туман або будинки робили нас непомітними з позицій
бойовиків. В інших місцях солдати перебігали відкриту місцевість, намагаючись
уникнути снайперського вогню.

Коли ми дісталися наступної позиції, то дізналися, що хтось приїде за нами
машиною. Напередодні одна з солдатів проскочила. Хоча вона й почувалася досить
комфортно у цьому районі, але починала нервувати в машині. Цивільним
автомобілям у цьому районі їздити досить безпечно, але бойовики іноді
обстрілюють українські військові машини кольору хакі протитанковими ракетами.

