% vim: keymap=russian-jcukenwin
%%beginhead 
 
%%file 21_11_2021.fb.fb_group.literatura_isskustvo_istoria_fotografia.1.chelovek_risunok_new_year.pic.5
%%parent 21_11_2021.fb.fb_group.literatura_isskustvo_istoria_fotografia.1.chelovek_risunok_new_year
 
%%url 
 
%%author_id 
%%date 
 
%%tags 
%%title 
 
%%endhead 

\ifcmt
  ig https://scontent-lga3-1.xx.fbcdn.net/v/t39.30808-6/257942981_457091132701856_7533055327694705859_n.jpg?_nc_cat=1&ccb=1-5&_nc_sid=b9115d&_nc_ohc=i2hmVDg0H5cAX-2fJxM&_nc_ht=scontent-lga3-1.xx&oh=c0faaea01393828e8363bf6df22d49ee&oe=619FB7E5
  @width 0.4
\fi

\iusr{Светлана Мелешко}
Очень добрые были открытки, рвдостные и улыбчивые. @igg{fbicon.hands.applause.yellow}{repeat=3} @igg{fbicon.heart.red}

\iusr{Оксана Ворочук}

Людина дарувала радість і новорічний святковий настрій усій країні ! Всі вітали
один одного, поштова скринька була переповнена яскравими листівками... Шкода ,
що життя перемололо талановитого художника... Це була ціла епоха.. Дякую за
цікаву інформацію !

\iusr{Натали Бирюк}
И покупали очень много, ведь надо всех поздравить с таким праздником

\iusr{Ольга Полозюк}

Спогад дитинства, в нас в школі завжди до нового року кожен клас малював
плакат, я пам'ятаю саме цей малюк я намалювала в сьомому класі, а потім напевно
через років десять після школи прийшла на новорічне свято і цей малюнок і ще
багато інших таких старих малюнків прикрашали новорічний зал,, дуууже теплі
спогади,,,. Велике дякую автору за такі чудові малюнки!!!


\iusr{Татьяна Кузнецова}

спасибо автору художемкц.. такая открытка от родителей из Казахстана висит 30
лет у меня вГрузии..

\ifcmt
  ig https://scontent-lga3-1.xx.fbcdn.net/v/t39.30808-6/259507896_3186342471587772_4244228341855468976_n.jpg?_nc_cat=100&ccb=1-5&_nc_sid=dbeb18&_nc_ohc=rG3BNlgf4-sAX8u6EHw&_nc_ht=scontent-lga3-1.xx&oh=dc1ca0524b3c335dfeaf3b1dd31dcfdc&oe=61A0F402
  @width 0.3
\fi

\iusr{Людмила Петренко}
Да, это радость и моего детства!

\iusr{Людмила Петренко}
Некоторые до сих пор храню...

\iusr{Светлана Фатеева}

Покупали заранее в киосках Союзпечати много открыток и конвертов, садились за
неделю до праздника подписывали и потом пачку бросали в синии ящик.

\iusr{Світлана Криворучко}
І така у мене була откритка

\iusr{Татьяна Мачула}
Столько радости и света..

\iusr{Elena Makarova}
Надо поискать, у меня и календарики вроде есть в таком стиле ) как раз 1968, 70 годы)

\iusr{Елена Дыба}

Смотрю на открытки и вспоминаю случай из детства, мы жили в общежитии, и почта
вся раскладывать на подоконнике, к праздникам было море открыток. Как ребёнку
пройти мимо этих удивительных рисунков ,мы их тайно забирали, а раньше в
открытке писали целые письма... у меня была приличная пачка.... Как-то мама
увидела мою "Коллекцию", ох и было мне...... нет меня не били, ругали ,говорили
,что люди ждут эти открытки, плачат, переживают, а их нет нет..... Вообщем, я всю
пачку выложила на подоконник, и люди получили свои открытки с большим
опозданием... я этот случай очень хорошо запомнила, с тех пор чужое не
беру.... Урок... а открытки необыкновенные ....!!! яВ них очень много тепла ....


\iusr{Оксана Пяст}
У нас, восмидесятников это было на ряду с марками коллекционирование открытки и конечно же в основном это был Зарубин. У меня кстати до сих пор они есть

\iusr{Marina Kormiltseva}
Художник-волшебник: пересматриваю иногда и переношусь в саое детство, на минутку забывая проблемы нынешнего дня.

\iusr{Элеонора Страх}
Всегда просила маму покупать мне его открытки
