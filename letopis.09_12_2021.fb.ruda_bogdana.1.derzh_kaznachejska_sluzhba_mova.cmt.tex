% vim: keymap=russian-jcukenwin
%%beginhead 
 
%%file 09_12_2021.fb.ruda_bogdana.1.derzh_kaznachejska_sluzhba_mova.cmt
%%parent 09_12_2021.fb.ruda_bogdana.1.derzh_kaznachejska_sluzhba_mova
 
%%url 
 
%%author_id 
%%date 
 
%%tags 
%%title 
 
%%endhead 
\subsubsection{Коментарі}

\begin{itemize} % {
\iusr{Вася Заровний}

В Українській мові це колись було нормально. В багатьох письменників і поетів є
така форма. Правда сучасні правила написання кажуть, що держави з чіткими
кордонами і статусом мають писатися з \enquote{в}, а з \enquote{на} - території з умовними
межами. Тут складна справа... В писанні про свою землю у нас часто вживалось
\enquote{на} (на Русі, на Україні). До того ж сама назва Україна/Вкраїна має в собі
\enquote{в}, тобто назва означає всередині своєї землі. Можливо тому раніше вживали
\enquote{на}, бо вроді як \enquote{в} двічі якось не звучить.

Тепер же дійсно оце \enquote{на} ріже слух і аж око сіпається і мимоволі стискаються
пальці щоб задушити @igg{fbicon.face.angry} 

\begin{itemize} % {
\iusr{Богдана Руда}
\textbf{Вася Заровний} 

історія історією. Так було колись, справді. Проте нині це як образа вже. Та й
стаття ж не про історію України, де б то могло бути доречним, а просто стаття
із цікавими фактами. І написана вона ось, у 2015 році, а не сотні і навіть
десятки років тому @igg{fbicon.face.smirking} 


\iusr{Roman Teslenko}
\textbf{Vasyl Zarovnyi} 

За часів Русі мову українською ніхто не називав. І я вважаю правильним називати
мову руською. Україна - це назва територіальної одиниця Русі, як зараз
\enquote{область} чи \enquote{район}, вживати котру в якості етноніму не бачу доцільним.
Нововведення харківської інтелігенції, що почала називати народ українцями
через свої малоросійські комплекси та сепаратистські настрої, накоїло надто
багато путанини. А радянська влада цю назву закріпила. І тепер уже виглядає
так, наче Росія прямий спадкоємець Русі, а українці і не русь зовсім. Окремі
персонажі навіть русофобами вважаються. Іронія долі.

\iusr{Богдана Руда}
\textbf{Roman Teslenko} ну то я ж про ситуацію вже в теперішніх реаліях і у тому всьому що наплутали/наміняли тощо

\iusr{Roman Teslenko}
\textbf{Богдана Руда} 

та чого воно Вас так заділо? Ви що: вчителька української мови? Навіть
вчителька української мови в нас казала \enquote{да} і \enquote{а іменно}. А слухаючи як
більшість пересічних смертних в Україні розмовляє, як Ви живете, взагалі, без
депресії?

\iusr{Богдана Руда}
\textbf{Roman Teslenko} складно @igg{fbicon.face.grinning.squinting} 
Виправляючи «на Україні» в україномовних та «укрАинский» у російськомовних)

\iusr{Roman Teslenko}
\textbf{Богдана Руда} то перестаньте так серйозно запарюватися. Всіх людей не перевчите. Любіть свою сім'ю та радійте життю.


\iusr{Богдана Руда}
\textbf{Roman Teslenko} дякую)
\end{itemize} % }

\end{itemize} % }
