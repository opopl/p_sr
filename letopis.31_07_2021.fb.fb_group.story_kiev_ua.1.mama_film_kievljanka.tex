% vim: keymap=russian-jcukenwin
%%beginhead 
 
%%file 31_07_2021.fb.fb_group.story_kiev_ua.1.mama_film_kievljanka
%%parent 31_07_2021
 
%%url https://www.facebook.com/groups/story.kiev.ua/posts/1719241484939312/
 
%%author Киевские Истории
%%author_id fb_group.story_kiev_ua
%%author_url 
 
%%tags chelovek,film,foto,kiev,kievljane,ukraina
%%title История Киевлянки - моя Мама - Фильм
 
%%endhead 
 
\subsection{История Киевлянки - моя Мама - Фильм}
\label{sec:31_07_2021.fb.fb_group.story_kiev_ua.1.mama_film_kievljanka}
 
\Purl{https://www.facebook.com/groups/story.kiev.ua/posts/1719241484939312/}
\ifcmt
 author_begin
   author_id fb_group.story_kiev_ua
 author_end
\fi

Хочу рассказать вам сегодня историю киевлянки - историю, изложенную в виде
любительского фильма созданного в Израиле. 

Приобщившиеся, смогут увидеть в нем Киев конца 60-х, начала 70-х годов и героев
моих публикаций («2 портсигара», «Дом патриарха», «Так прошла жизнь»,
«Октябрьский дворец» и др.): моего папу - Зигмунда, снимавшего все своей
кинокамерой «Кварц-9»; моих бабушек - Берту старшую, тыкающую дули и Еву,
кормящую братика Женю супом; улыбающегося дедушку Иосифа возле своего дома на
Владимирской, 49 (монстра Гопкало тогда еще не было); меня на раскладушке под
столом в комнатке в 10кв.м. и многих других.

\ifcmt
  tab_begin cols=2

     pic https://scontent-cdt1-1.xx.fbcdn.net/v/t1.6435-9/228144427_4154179704688799_307400805074722508_n.jpg?_nc_cat=105&ccb=1-3&_nc_sid=b9115d&_nc_ohc=M7xYu6Wl5JMAX9m2AF2&_nc_ht=scontent-cdt1-1.xx&oh=6ffcc6c867e8ecf60904d38cf35df1fa&oe=612D3448

     pic https://scontent-cdt1-1.xx.fbcdn.net/v/t1.6435-9/228533568_4154179998022103_3858040811162819867_n.jpg?_nc_cat=106&ccb=1-3&_nc_sid=b9115d&_nc_ohc=0BkBp8wSaUkAX-w64b6&_nc_ht=scontent-cdt1-1.xx&oh=ee25885c5b24f112845134a1e4fe458e&oe=612DFCD8
     
  tab_end
\fi

Но главная звезда фильма моя мама – Берта младшая и фильм этот был подарен ей
на юбилей.

Это мама сцементировала нашу семью, создала в доме уют и тепло, выходила, после
лагерных болячек, папу, воспитала и вырастила сыновей, успевая всегда всех
обшить, обвязать, вылечить, обнять и утешить. Мама столько вложила в своих
сыновей что стала для нас с братом незаменимым другом, с которым просто хочется
каждый день разговаривать. Что мы, хвала Скайпу, ежедневно и делаем. 

Да и для киевоведения, моя мама уникальный человек - единственный, который
помнит, какие полоски и разводы были на красном носу гранитного Ленина, по
монументу которому она ползала почти целый год, пока этого идола не увезли со
склада моего деда устанавливать на Бессарабку в декабре 1946г.  

\ifcmt
  tab_begin cols=2
    width 0.2

     pic https://scontent-cdt1-1.xx.fbcdn.net/v/t1.6435-9/228752196_4154179854688784_313725891898174159_n.jpg?_nc_cat=101&ccb=1-3&_nc_sid=b9115d&_nc_ohc=iYeeSSDdd3cAX9-DpIg&_nc_ht=scontent-cdt1-1.xx&oh=47626759aec4b240a61d1948bc0daaf3&oe=612F1E95

     pic https://scontent-cdt1-1.xx.fbcdn.net/v/t1.6435-9/227424449_4154179208022182_528568199182424794_n.jpg?_nc_cat=105&ccb=1-3&_nc_sid=b9115d&_nc_ohc=1bq6U1LOCbYAX9YoGqg&_nc_ht=scontent-cdt1-1.xx&oh=3f1e9699aaad691706fe79aa07bfce9b&oe=612CCA3A
    width 0.23

  tab_end
\fi

Мама уже 30 лет живет далеко от Киева, в другом славном городе, у границ
которого остановился Всемирный потоп и который в пять раз древнее Киева. Вопрос
к Знатокам - что это за уникальный город, который, к тому же расположен на
землях, отданных Всевышним для пропитания Адаму и Еве после их изгнания из рая?

Живет мама далеко, но активно читает наши киевские группы и сердце ее здесь в
любимом Городе ее юности. Как видим, Киевлянка — это диагноз.

Первого августа у мамы День рождения.

Поздравляю, люблю и преклоняюсь.
