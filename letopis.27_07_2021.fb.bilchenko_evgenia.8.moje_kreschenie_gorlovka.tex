% vim: keymap=russian-jcukenwin
%%beginhead 
 
%%file 27_07_2021.fb.bilchenko_evgenia.8.moje_kreschenie_gorlovka
%%parent 27_07_2021
 
%%url https://www.facebook.com/yevzhik/posts/4100472899987789
 
%%author Бильченко, Евгения
%%author_id bilchenko_evgenia
%%author_url 
 
%%tags bilchenko_evgenia,donbass,gorlovka,kreschenie,ukraina,vojna
%%title БЖ. Моё Крещение
 
%%endhead 
 
\subsection{БЖ. Моё Крещение}
\label{sec:27_07_2021.fb.bilchenko_evgenia.8.moje_kreschenie_gorlovka}
 
\Purl{https://www.facebook.com/yevzhik/posts/4100472899987789}
\ifcmt
 author_begin
   author_id bilchenko_evgenia
 author_end
\fi

БЖ. Моё Крещение.

Сегодня все говорят о Крещении Руси. Я не могу не сказать тоже. Дорогие братья
и сёстры, вы очень прекрасно шли, вы - очень красивые, очень. Я видела море в
вас. Простите меня, что я напоминаю в этот светлый день о самом темном, что
может быть в мире. Видимо, христианство так устроено, что рядом со Светом оно
ставит страдание. Я не могу сегодня листать ленту. Я вижу всё время страшные
кадры, и это безумно больно. Мне даже сейчас больно это писать, так больно, что
дышать невозможно, но я хочу написать об этом, я хочу еще раз вспомнить 27 июля
времен, когда я не понимала и не видела. И мне очень обидно, что мои
соотечественники здесь массово не видят до сих пор.

\ifcmt
  pic https://scontent-lga3-2.xx.fbcdn.net/v/t1.6435-9/225283390_4100473849987694_4661281943077622440_n.jpg?_nc_cat=105&ccb=1-3&_nc_sid=730e14&_nc_ohc=fVWpYkJxvI0AX8lDTfu&_nc_ht=scontent-lga3-2.xx&oh=7ba75a30fb30fd7f783f060add4d032e&oe=61275CD5
  width 0.4
\fi

ГОРЛОВКА. Этим именем зовется моё боевое крещение. Горловка стала моим
Крещением. Я помню этот день. Точнее, вечер. Вечер был невыносим. Вечер зимы
2016 года. Ранней, колкой и такой, что в горле горлицей билось: "С меня
хватит". Я ничего еще толком не знала, но это "хватит" не давало дышать. Я
стояла у бульвара Шевченко напротив дома 55. Я это помню четко. Я набрала
одного поэта, второго, третьего... Везде - бред и не то, не то, не то. Это
стало настолько чужим, что вызывало почти физическую тошноту. Самым ужасным
было то, что у меня не было слов для описания своего состояния. Вообще не было
слов.

И я позвонила Саше. Не важно, как её фамилия. Это известная женщина, довольно
проукраинская, она больше со мной не общается, уже очень давно, но она добрая.
Я бы хотела, чтобы она знала, что я благодарна ей за тот разговор, так же, как
за все подарки. Это важно. Так вот, Саша поняла. Я не знаю, что она поняла (в
голове вращается какой-то роковой приговор: "Она поняла, что БЖ - другая, не
наша, что ли"), но она вывела меня на горловчан. Это был первый удар под дых.
Точнее... Я не могу это описать. Лучше и хуже этого только бездна, которая
цветаевская - без дна. Это Реальное. Слишком Реальное. Голое. Ничем не
прикрытое. И от этого священное. Горловка. 

Именно тогда я взяла первое интервью у учительницы Тани. Уже тогда вокруг меня
взорвались местные люди: негодуя, витийствуя, бормоча фальшивые слова. А у меня
была только Таня. У меня была Таня месяцами в голове. Даже сама Таня не
понимает, что она со мной сделала. Получается, что она меня родила заново. 

Потом появились ребята: Макс, Егор, стихи Егора, стихи Макса, их нервное родное
(мое) образование, их язык, которого я не слышала со времен смерти бабушки,
родной язык, как будто долго-долго ходил и вдруг - пришел. Нет, это не
погружение в идеологию. Фу, как можно об этом думать! Это было кромешным
пунктирным выпадением из всего - в себя. И уже потом появилась студентка Алина,
вплоть до 2021 года, до нашей с ней Одессы, которую я показывала с лучше
стороны, чтобы... Чтобы - что? У меня сильно чесались руки показать ужасное, но
я видела, как ей у Дома профсоюзов. 

ГОРЛОВКА. В 2019 году (два года прошло с тех пор) я начала и бросила большую
поэму "Украина в бреду" - жестким твардовским слогом. Показать её нет и не было
возможности. Мне пришлось её подарить одному хорошему человеку, чтобы он издал
это как своё. Не важно, кому и где, но в 2021 году это случилось. Я так
захотела. И от всей поэмы остался только осколок, который я никому не отдам.
Да, мне хватило дерзости зачитать это в Украинском доме. Сбежался весь дом. Из
соседних лотков выставки люди вытягивали длинные страусиные шеи и не смели мне
ничего сказать. Крещение жизнью и смертью для меня ждало своего часа...
ГОРЛОВКА.

Из ненаписанной поэмы (отрывок)

\begin{multicols}{2}
\obeycr
Горловка, здравствуй, Горловка, дело - швах.
Все они верят. Я же - трещу по швам.
Все они - прахом к праху, слоном к слону -
Верят в оборонительную войну.
\smallskip
Верят, что выручают вас от чужих.
В их золотых сердечках гниют чижи.
Даже, когда они перестанут петь,
Мамка подарит пряник, а папка - плеть.
Мамка - американская, папка - свой.
\smallskip
Завтра они поднимут щемящий вой.
Завтра я выйду, словом убив уют.
Завтра меня гвоздями к столбу прибьют.
\smallskip
Горловка, здравствуй, Танечка, здравствуй, мать. 
Как же мне сладко - сервером обнимать.
Как же мне горько - рвать себя на куски,
Чтобы раздать взбесившимся от тоски.
\smallskip
Горловка, здравствуй, горло, привет, Егор.
Долог наш поэтический разговор.
Музы не стоят пушек, а пушки - муз.
Братка Максимка, наш уязвим союз.
\smallskip
Вот он - весь в язвах, ягодах, янтарях:
Я не могу еще раз его терять.
Я не могу ещё раз тебя предать.
Мост за спиной пылающий - благодать.
Запад, который я ради вас сожгла.
\smallskip
Нет, я не жду прощенья из-за угла.
Я не люблю смешных показушных сцен.
Ветер боится собственных перемен.
Горловка, здравствуй, горлица, - два крыла.
\smallskip
Шла - пирожок нашла, но полно стекла
Вместо начинки: скушай да истеки
Кровью одной-единой для всех реки...
\smallskip
2019 г.
\restorecr
\end{multicols}

\ii{27_07_2021.fb.bilchenko_evgenia.8.moje_kreschenie_gorlovka.cmt}

