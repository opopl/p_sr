% vim: keymap=russian-jcukenwin
%%beginhead 
 
%%file 10_10_2021.fb.fb_group.story_kiev_ua.1.gruppa
%%parent 10_10_2021
 
%%url https://www.facebook.com/groups/story.kiev.ua/posts/1771609139702546
 
%%author_id fb_group.story_kiev_ua,koval_oleg
%%date 
 
%%tags facebook,kiev
%%title Сегодня, спустя почти 4 года, мы приняли в ряды нашего сообщества 100 000 участника!
 
%%endhead 
 
\subsection{Сегодня, спустя почти 4 года, мы приняли в ряды нашего сообщества 100 000 участника!}
\label{sec:10_10_2021.fb.fb_group.story_kiev_ua.1.gruppa}
 
\Purl{https://www.facebook.com/groups/story.kiev.ua/posts/1771609139702546}
\ifcmt
 author_begin
   author_id fb_group.story_kiev_ua,koval_oleg
 author_end
\fi

Без малого четыре года назад в преддверии Рождества 2017 года мне в голову
пришла мысль о создании клуба киевлян, в котором каждый желающий сможет
поделиться своей маленькой историей, связанной с Киевом. А как известно, все
самые таинственные и загадочные истории начинаются как раз в канун Рождества.
Название напросилось само собой, пальцы сами отпечатали два простых слова –
«Киевские истории». Сегодня я хочу вспомнить и рассказать вам историю создания
сообщества КИ. 

\ifcmt
  pic https://scontent-frt3-1.xx.fbcdn.net/v/t1.6435-9/245045122_4470130789714708_3724912292663364976_n.jpg?_nc_cat=102&ccb=1-5&_nc_sid=825194&_nc_ohc=181tpFvW99wAX-vGfoe&_nc_ht=scontent-frt3-1.xx&oh=0914d18c621e90353523501eddfdf685&oe=618A4090
  @width 0.8
  %@wrap \parpic[r]
\fi

К 24 декабря 2017 года в моей записной книжке накопилось с десяток маленьких
зарисовок-очерков о Киеве моего детства. Такие себе ламповые истории, которые я
разместил в только что основанную группу. Так вышло, что при создании групп по
умолчанию приглашаются 45 человек или около того (сейчас не вспомню), и как по
волшебству из списка моих друзей в группе появились первые подписчики. Эти люди
и стали первыми участниками нашего сообщества. Группа быстро росла и обрастала
новыми киевскими историями. В ней воцарилась душевная доброта и гармония. Помню
первых модераторов: Татьяну Петровскую, Феликса Лидермана, Евгению Бочковскую и
душу нашего клуба киевлян Светлану Манилову. Мы стали друзьями, а уже весной
организовалась первая встреча в Ботаническом саду, где Евгения Бочковская со
своим супругом Виктором провели экскурсию по местам своего детства и юности. 

Сегодня, спустя почти 4 года, мы приняли в ряды нашего сообщества 100 000
участника! Группа выросла, пройдя нелегкий путь становления. За время
существования сообщества «Киевские истории» мы провели множество замечательных
встреч и познавательных мероприятий. Актив группы стал инициатором нескольких
проектов по спасению культурного наследия нашего города. Впереди еще много
планов, и я искренне рад и благодарен за все это замечательным людям -
киевлянам, активным участникам нашей группы. 

Быть добру.  @igg{fbicon.face.smiling.eyes.smiling} 

Автор картины: участник группы "Киевские истории" Константин Нагорный.

\ii{10_10_2021.fb.fb_group.story_kiev_ua.1.gruppa.cmt}
