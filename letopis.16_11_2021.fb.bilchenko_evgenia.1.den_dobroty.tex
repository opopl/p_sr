% vim: keymap=russian-jcukenwin
%%beginhead 
 
%%file 16_11_2021.fb.bilchenko_evgenia.1.den_dobroty
%%parent 16_11_2021
 
%%url https://www.facebook.com/yevzhik/posts/4446880885346987
 
%%author_id bilchenko_evgenia
%%date 
 
%%tags bilchenko_evgenia,dobrota
%%title БЖ. День доброты: каким он должен быть
 
%%endhead 
 
\subsection{БЖ. День доброты: каким он должен быть}
\label{sec:16_11_2021.fb.bilchenko_evgenia.1.den_dobroty}
 
\Purl{https://www.facebook.com/yevzhik/posts/4446880885346987}
\ifcmt
 author_begin
   author_id bilchenko_evgenia
 author_end
\fi

БЖ. День доброты: каким он должен быть

Был такой день 13 ноября. Он связан с обществом слепых. Люди, не обладающие
знанием духа, - слепы. В Питере было когда-то общество для слепых - дом
милосердия и доброты. Истинная доброта - не мимишная.

\ifcmt
  ig https://scontent-frt3-1.xx.fbcdn.net/v/t39.30808-6/257723017_4446882175346858_2448108399768246406_n.jpg?_nc_cat=106&ccb=1-5&_nc_sid=8bfeb9&_nc_ohc=L6GlUVBGER0AX9x8LhD&_nc_ht=scontent-frt3-1.xx&oh=164be146eb48fd465d3b2517c81b7f48&oe=619B6B12
  @width 0.4
  %@wrap \parpic[r]
  @wrap \InsertBoxR{0}
\fi

Я хочу сказать спасибо Герман Владимиров Герману Николаевичу Владимирову за
документальный фильм "Дети Донбасса" ( ЛенТВ24 ) : открытого показа пока не будет,
потому кадров нет, - всё делаю, как решают творцы этого доброго, гуманного и
очень честного кино. Никакой художественный фильм (ни "Солнцепек", ни даже
"Донбасс.Окраина", который я люблю) так меня не впечатляли. 

Я плакала весь получасовой показ. Особенно потрясла выдержанная строго и
ненавязчиво в ткани фильма смысловая нить между хронотопами: блокада Ленинграда
- блокада мирного населения До, старики - дети, прошлое - настоящее, зло
похороненное - зло, восставшее в виде зомби. В кино нет ничего
пропагандистского, ничего пережатого, ничего пафосного и наносного. 

Герман Николаевич долгое время фигурировал в моих текстах как "Гарри", и, когда
я смотрела фильм, один из посвященных моему коллеге и другу текстов, так и
всплывал в памяти, вы его все помните, текст прошел весь путь через фестивали и
клипы, без биографических ссылок:

\begin{center}
\begin{multicols}{2}
\large\color{blue}\em
\obeycr
Гляди-ка, спирт меня не торкнул.
Сваргань мне чай, как дёготь, чёрный.
Я застрелю их под Егорку,
Я застрелюсь под Башлачёва.
Прошли суму, тюрьму, окопы,
Догматы, маты, перемены:
Нас победили остолопы,
Нас победили бизнесмены.
Детей войны. Героев Трои.
Живую плоть глобальных топок.
«Мы наш, мы новый банк построим» —
Сказал нам выспренный ушлёпок.
Осталось что? Полынь и ругань.
СашБаш и Летов. С лёгким паром.
Когда стреляли мы в друг друга,
Стрелялись мы с тобой на пару.
Двойные суициды шествий
Пешком: от бытности до быта.
Молчит во гробе Чернышевский...
І я не знаю, що робити.
\restorecr
\end{multicols}
\end{center}

Герман, спасибо. Вот таким должен быть День Доброты. Мои книги и многое иное
для дел добрых при мне. Спасибо за возможность передать привет моей студентке
из Горловки Алина Голоперова   и ее бабушке  через журналистов. Остальное в
личке. Фото - символизирует истину, красоту и главное - категорический
моральный императив Канта о долженствования, о мужестве быть вопреки.
Концентрат фильма о добре.

 \#рецензииотБЖ
