% vim: keymap=russian-jcukenwin
%%beginhead 
 
%%file 22_02_2022.fb.spivak_dmitrij.2.bajden_i_putin
%%parent 22_02_2022
 
%%url https://www.facebook.com/SpivakDmytriy/posts/5238784689473880
 
%%author_id spivak_dmitrij
%%date 
 
%%tags biden_joe,geopolitika,putin_vladimir,rossia,ukraina,usa
%%title БАЙДЕН и ПУТИН...
 
%%endhead 
 
\subsection{БАЙДЕН и ПУТИН...}
\label{sec:22_02_2022.fb.spivak_dmitrij.2.bajden_i_putin}
 
\Purl{https://www.facebook.com/SpivakDmytriy/posts/5238784689473880}
\ifcmt
 author_begin
   author_id spivak_dmitrij
 author_end
\fi

БАЙДЕН и ПУТИН... 

Они очень опытные политики и очень хорошо друг друга знают. 

Они знают сильные и слабые стороны обоих государств. Владеют информацией о
вызовах и проблемах друг друга. 

Их разведывательные и другие специальные и аналитические подразделения имеют
точные данные о возможных комбинациях. 

Россия решает свои задачи. Нахраписто и нагло, но очень системно и в рамках
сегодняшних возможностей. Они упрочили своё положение в Сирии, Иране,
Северо-Западе Африки, откуда Франция черпает ресурсы и сблизились с Китаем,
который является главным противником США. Кроме того, у них появилось
гиперзвуковое оружие в виде ракет, которые они могут разместить в
непосредственной близости от Америки, в Венесуэле или на Кубе. Плюс к этому -
энергетическая зависимость Европы от поставок российского газа. 

Это их козыри. Они сегодня ловят момент, чтобы поделить мир и сферы влияния.

Америка решает свои задачи, и прежде всего - восстановление утраченного былого
величия после провальной концовки в Афганистане, неудавшегося блицкрига в Сирии
и сближения Китая с Россией, и достройки СП-2. Их мощь заключается в том, что
они могут накладывать аресты на активы и вводить санкции, которые бьют по
экономикам отдельных отраслей и целых стран. Доллар - мировая валюта. Торговые
отношения пока зависимы от санкционной политики США. Плюс военная мощь. Военный
Бюджет впечатляет своими космическими цифрами. И конечно, из партнёры в Европе
- Британия, Польша и страны Балтии. 

Что касается немцев и французов, то тут не все так гладко и однозначно. Старая
Европа хочет выстраивать свою политику сдерживания и продвигать свои  интересы. 

В общем, две первые ядерные державы схлестнулись. И территория Украины, к
сожалению для всех нас, стала буферной зоной и разменной монетой в этих
разборках. 

Путин вчера пошёл Ва-банк, признав ЛДНР и понимая, что сегодня пойдёт парад
санкций. Ставки подняты до предела. 

Байден сегодня огласил достаточно серьёзные антироссийские санкции, которые
задевают финансовый спектр России, но и отрицательно скажутся на самих Штатах. 

Окно возможностей никто не закрывает, и мосты никто не сжигает. Торговля нефтью
продолжается, и SWIFT не задействован....

Они скоро встретятся, чтобы попытаться снизить накал и хоть о чем то
договориться. Большая война никому не нужна. Большая война уничтожит экономику
обоих государств, и тогда Китай автоматически может стать мировым лидером. 

Уверен, что они как то договоятся. В том числе, по Украине. Жаль, что без
нашего участия. Мы так и не стали субъектом в мировой политике. 

Но будем надеятся, что их договорённости снизят накал страстей и позволят нам
восстановить мчащуюся под откос экономику... 

И конечно, принесёт мир!!!! 

Верю, что настоящей войны на нашей территории удасться избежать. Главное,
избежать провокации....
