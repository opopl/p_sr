% vim: keymap=russian-jcukenwin
%%beginhead 
 
%%file 25_12_2020.news.ru.inosmi.1.argentina_russia_sputnik_v
%%parent 25_12_2020
 
%%url https://inosmi.ru/social/20201225/248834277.html
 
%%author 
%%author_id 
%%author_url 
 
%%tags 
%%title Аргентинцы о доставке в страну вакцины «Спутник V»: спасибо, щедрая Россия!
 
%%endhead 
 
\subsection{Аргентинцы о доставке в страну вакцины «Спутник V»: спасибо, щедрая Россия!}
\label{sec:25_12_2020.news.ru.inosmi.1.argentina_russia_sputnik_v}
\Purl{https://inosmi.ru/social/20201225/248834277.html}

\begin{leftbar}
  \begingroup
    \em\Large\bfseries\color{blue}
Информация о том, что первые партии российского препарата «Спутник V»
доставлены в Аргентину с воодушевлением встречена как местными СМИ, так и
жителями страны. «Исторический момент! Первые дозы облегчения и надежды. С
праздником, аргентинцы! Спасибо щедрому русскому народу! Сегодня День святого
Путина!», — так комментируют читатели начало процесса вакцинации.
  \endgroup
\end{leftbar}


\ifcmt
  pic https://cdn1.img.inosmi.ru/images/24882/91/248829106.jpg
\fi

Самолет Аргентинских авиалиний с первой партией вакцины «Спутник V» на борту
приземлился в аэропорту города Эсейса в 10:25 по местному времени. В страну
доставлено 300 тысяч доз вакцины. Их распределят по провинциям страны и в
течение 24 часов доставят в пункты назначения. В начале января в страну
прибудет следующая партия вакцины — 4,7 миллиона доз, а в начале февраля — еще
15 миллионов.

Уже на следующей неделе в Аргентине начнется самая крупная в истории страны
кампания вакцинации, в проведении которой примут участие 116 тысяч медицинских
работников, 7 тысяч 749 учреждений здравоохранения и более 10 тысяч
добровольцев.

Президент Аргентины Альберто Фернандес поблагодарил Владимира Путина за
приверженность, проявленную по отношению к стране. «Открывается путь надежды,
но пандемия еще не закончилась. Нужно продолжать заботиться о себе», — заявил
президент.

Аргентинские издания широко освещают это событие:

Pájina 12 (Аргентина): Лучшие фото приземления самолета с русской вакциной  

Послание Альберто Фернандеса после прибытия русской вакцины: «Открывается путь
надежды, но пандемия еще не закончилась» (Clarín, Аргентина) 

Pájina 12 (Аргентина): Как будет проходить вакцинация «Спутником V» в Аргентине

Вакцинация в Аргентине: что произойдет после прибытия первой партии «Спутника
V» (Pájina 12, Аргентина)

***

\subsubsection{Комментарии читателей}

Flacoyordi

Как хорошо, да? Это исторический момент.

Hugonov 

Браво! Первые дозы облегчения и надежды для аргентинского народа.

Dianaibarra

Национальная гордость!

\verb|Jaque_mate|

Отличная новость для людей из групп риска и безупречная работа правительства. И
все это несмотря на то, что средства дезинформации, глупые политики и обычные
идиоты постоянно вставляли палки в колеса. С праздником, Аргентина, с
праздником, аргентинцы!!!

Naumkliksberg

Хорошо, что у нас теперь есть вакцина, но в то же время надо, чтобы во всех
больницах в наличии была плазма из сыворотки крови лошадей. Уже доказано, что
это лучший способ лечения тяжелых случаев, она на 50\% уменьшает смертность от
ковида и освобождает койки.

Graciela

Я слышала, что в Чили привезли 10 тысяч доз вакцины от Pfizer при населении в
20 миллионов… Как хорошо, что в таких серьезных условиях у нас есть наше
правительство. В конце концов, за него я и голосовала.

Japoflores 

Спасибо Альберто [Фернандесу] и Кристине [Киршнер] за то, что они о нас
заботятся.

Wilfredo

Я уверен, этими первыми вакцинами привьют тех, кто младше 60, ведь именно они
сейчас чаще заражаются. Для нас, тех, кому за 60, это будет облегчением,
косвенным, но тем не менее. Особенно для тех, кто не может исключить контакты с
молодыми людьми, ведь именно они сегодня основной источник заражения.

Oblicua (в ответ Wilfredo)

Кажется, все немного иначе: в первую очередь, их распределят среди медработников. Молодые будут последними, я думаю.

Wilfredo (в ответ Oblicua)

С 300 тысячами вакцин можно хорошо продвинуться. В Чили с огромной оглаской
получили 10 тысяч. А по радио говорят, что чуть ли не каждому достанется,
ха-ха-ха.

Clinicasanlazarus

Спасибо щедрому русскому народу. Надеюсь, с выплатами мы ничего не напутаем.

Luigi1960 (в ответ Clinicasanlazarus)

Этот щедрый русский народ спас нас еще и от нацизма, выиграв Вторую мировую войну в Европе и потеряв 25 миллионов человек павшими.

Amalita (в ответ Clinicasanlazarus)

Они остались без рабочих рук, как ты говоришь, погибло 25 миллионов мужчин.
Женщинам пришлось выполнять всю работу. Я слышала, чтобы дискредитировать СССР,
фотографировали, как женщины в мужской одежде мели улицы Москвы или управляли
трамваями. Какая мерзость.

\verb|Sergio­_­­­Roman| 

Я так счастлив! Это огромное достижение для человечества!

Hhe

Сегодня День святого Путина!!!!
