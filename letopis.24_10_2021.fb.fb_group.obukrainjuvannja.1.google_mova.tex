% vim: keymap=russian-jcukenwin
%%beginhead 
 
%%file 24_10_2021.fb.fb_group.obukrainjuvannja.1.google_mova
%%parent 24_10_2021
 
%%url https://www.facebook.com/groups/2145849382411277/posts/2958376287825245
 
%%author_id fb_group.obukrainjuvannja,kozachenko_bazhan
%%date 
 
%%tags google,internet,jazyk,mova,ukraina
%%title Google - запити українською
 
%%endhead 
 
\subsection{Google - запити українською}
\label{sec:24_10_2021.fb.fb_group.obukrainjuvannja.1.google_mova}
 
\Purl{https://www.facebook.com/groups/2145849382411277/posts/2958376287825245}
\ifcmt
 author_begin
   author_id fb_group.obukrainjuvannja,kozachenko_bazhan
 author_end
\fi

Останніми днями Мережею ширять чутки, же ніби додавання "-и" до запиту в Ґуґлі
(Google) дозволяє прибрати з подачі всі розповіді московською мовою. Люде
сповіщають, же в когось воно працює, в іншних НЕ працює. Хай там як, але
подають те як перемогу і здобуток.

\ifcmt
  pic https://scontent-frt3-1.xx.fbcdn.net/v/t39.30808-6/247303537_631272991216271_1162676476045628414_n.jpg?_nc_cat=108&ccb=1-5&_nc_sid=825194&_nc_ohc=5px4TbpW9EcAX_GQLiK&_nc_ht=scontent-frt3-1.xx&oh=50b49365a1e4df706638441a1a48278e&oe=617C18DA
  @width 0.7
\fi

А я хочу запитати, як свого часу і про мовні обмеження для поширювальників
проміжного вмісту: чому саме задля здобутку українського в Україні люде мають
щось вчиняти, додавати, просити, нагадувати, перемикати мову мережців ? Чому в
Україні вказують кількість українських творів, а НЕ чужинських ? Чому б навпаки
додавання "-и" ні дозволяло б бачити москвамовне, а без нього, за промовчання,
люде бачитимуть лише українське ?

То "тут є господар українець" чи той вислів лише для гасла використовують ?

Я сприймаю таке ставлення як кістку кинуту псу: "Відчипіть ся, ось вам зробили
можливість." Нас кожного разу ніби примушують попросити дозволу на українське.

\ii{24_10_2021.fb.fb_group.obukrainjuvannja.1.google_mova.cmt}
