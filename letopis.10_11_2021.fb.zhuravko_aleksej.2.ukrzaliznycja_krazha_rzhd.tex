% vim: keymap=russian-jcukenwin
%%beginhead 
 
%%file 10_11_2021.fb.zhuravko_aleksej.2.ukrzaliznycja_krazha_rzhd
%%parent 10_11_2021
 
%%url https://www.facebook.com/permalink.php?story_fbid=1596136654064634&id=100010049438147
 
%%author_id zhuravko_aleksej
%%date 
 
%%tags krazha,rossia,ukrzaliznycia
%%title Укразализныця украла у Российских железных дорог двадцать локомотивов
 
%%endhead 
 
\subsection{Укразализныця украла у Российских железных дорог двадцать локомотивов}
\label{sec:10_11_2021.fb.zhuravko_aleksej.2.ukrzaliznycja_krazha_rzhd}
 
\Purl{https://www.facebook.com/permalink.php?story_fbid=1596136654064634&id=100010049438147}
\ifcmt
 author_begin
   author_id zhuravko_aleksej
 author_end
\fi

Алексей Журавко 

Дорогие друзья!

Не могут на Украине прожить ни одного дня, чтобы не опозориться на весь мир.

Укразализныця украла у Российских железных дорог двадцать локомотивов. 

Именно этим объясняется нежелание РЖД попускать на территорию Украины даже
поезда с углём из Казахстана. 

\ifcmt
  ig https://scontent-frt3-1.xx.fbcdn.net/v/t39.30808-6/254590544_1596136597397973_232555881327524451_n.jpg?_nc_cat=107&ccb=1-5&_nc_sid=8bfeb9&_nc_ohc=6-75PO80EhoAX8xf6bz&_nc_ht=scontent-frt3-1.xx&oh=4cfe0335eb38373fb4478d94c179fef4&oe=6191C04E
  @width 0.4
  %@wrap \parpic[r]
  @wrap \InsertBoxR{0}
\fi

И  дело вовсе не в обстрелах Донбасса и турецких «Байрактарах».

«Ещё в октябре 2020 года председатель правления украинской государственной
железнодорожной компании "Укрзалізниця" Владимир Жмак заявлял, что компания
разрушена полностью, 95\% локомотивов непригодны к работе.

К ноябрю 2021 года проблему украинские власти начали решать в привычном ключе.
И холдинг „Российские железные дороги‟ потерял больше 20 магистральных
локомотивов, их присвоило ПАО „Украинская железная дорога‟. Это, не считая
слухов о замене новых тележек российских грузовых вагонов на украинские
50-летней давности.

Но, налаженная, в лучших традициях украинской государственности, система
экспроприации дала сбой. РЖД не изъявила желания стать донором, и
транспортировка грузов в украинском направлении начала резко сокращаться. В том
числе и угля.

Поскольку украинские власти не утруждают себя взглядом на карту (по их
представлениям, Украина находится в центре земли, а вокруг располагаются мелкие
региональные державы вроде США, Германии, Великобритании, Китая и России),
решено заказать уголь из Казахстана. И тут пришло понимание, что без РЖД эту
задачу не решить никак. А там есть серьёзные вопросы, например, по поводу двух
десятков украденных локомотивов, фотографий которых на украинских просторах
полно в сети.

Ситуация патовая. По словам депутата Верховной рады Ивана Крулько, сейчас на
Украине есть 88 энергоблоков ТЭС, из них на сегодняшний день 37 функционирует,
а 51 не работает, они закрыты из-за недостатка топлива. Более того, самая
мощная в Киевской области Трипольская ТЭС остановила работу из-за снижения до
минимума поставок угля. На очереди веерные отключения.

Получается, что история с прекращением поставок того же угля — это не про
блокаду, что вообще России несвойственно (в отличие от). Это печальный рассказ
о том, как престарелый вор украл у добродушного соседа сумку, в которой тот
носил вору то бесплатную, то дешёвую еду из жалости. И сосед откровенно
оторопел от такой наглости».

Показательно, что вышеупомянутый Жмак действительно заявлял, что подвижной
состав «Укрзализныци» разваливается на ходу, а износ инфраструктуры компании в
скором времени достигнет отметки в 100\%.

«Износ 958 электровозов и 227 тепловозов составляет 96\%; 41 138 грузовых
вагонов — 89\%; 2 040 пассажирских вагонов — 88\%; 34 тысяч устройств ж/д
автоматики и связи — 68\%; ж/д пути изношены на 34\%, а контактная сеть (9 319
км) — на 71\%», — заявил в своём отчёте Владимир Жмак.

Как видим, новые эффективные европейские менеджеры действительно взялись за
решение проблем в меру своей специализации. А именно — стали воровать активы у
соседей.

Недаром же говорят: не плюй в колодец, еще пить из него придется...

С уважением,

Алексей Журавко
