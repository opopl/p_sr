% vim: keymap=russian-jcukenwin
%%beginhead 
 
%%file 04_01_2019.stz.news.ua.lb.1.arhitekturnyj_atlas_dorevoljucijnogo_mariupolja.5.dom_vracha_gampera
%%parent 04_01_2019.stz.news.ua.lb.1.arhitekturnyj_atlas_dorevoljucijnogo_mariupolja
 
%%url 
 
%%author_id 
%%date 
 
%%tags 
%%title 
 
%%endhead 

\subsubsection{Дом врача Гампера}

Этот необычный дом с элементами неоготики, долгое время был местом, где жил и
работал главный врач мариупольской земской больницы – Сергей Фёдорович Гампер.
\enquote{В народе} здание называется просто - \enquote{дом Гампера}, несмотря
на то, что владел недвижимостью купец Шнейдерович. По информации краеведа
Сергея Бурова, дом построен во второй половине позапрошлого столетия, и за свою
историю несколько раз перестраивался. Когда-то, на знаменитой башне красовались
цифры \enquote{1897}, сообщая год её постройки. Периметр ограждал внушительный
забор, а во дворе располагался сад с небольшим водоёмом, в котором плавали
птицы. Жители прогуливались по усыпанной морским песком аллее. Таким был
\enquote{дом Гампера} до революции.

\ii{04_01_2019.stz.news.ua.lb.1.arhitekturnyj_atlas_dorevoljucijnogo_mariupolja.5.dom_vracha_gampera.pic.1}

После 1917 года здание перешло в коммунальную собственность, огромные комнаты
разделили и в доме поселили несколько семей. В таком виде предстаёт перед нами
\enquote{дом Гампера} и в ХХІ веке. Он разделен на несколько хозяев, а двор
беспорядочно застроен подсобными постройками и завален хламом. Хотя,
квартиранты \enquote{дома Гампера} и не дают ему превратиться в руины, чтоб не
остаться самим на улице, - от былого великолепия мало что осталось. Но пока ещё
не всё потеряно. Имея заботливого и состоятельного хозяина, \enquote{дом
Гампера} вполне может стать визитной карточкой Мариуполя.

\ii{04_01_2019.stz.news.ua.lb.1.arhitekturnyj_atlas_dorevoljucijnogo_mariupolja.5.dom_vracha_gampera.pic.2}
\ii{04_01_2019.stz.news.ua.lb.1.arhitekturnyj_atlas_dorevoljucijnogo_mariupolja.5.dom_vracha_gampera.pic.3}
\ii{04_01_2019.stz.news.ua.lb.1.arhitekturnyj_atlas_dorevoljucijnogo_mariupolja.5.dom_vracha_gampera.pic.4}
\ii{04_01_2019.stz.news.ua.lb.1.arhitekturnyj_atlas_dorevoljucijnogo_mariupolja.5.dom_vracha_gampera.pic.5}
\ii{04_01_2019.stz.news.ua.lb.1.arhitekturnyj_atlas_dorevoljucijnogo_mariupolja.5.dom_vracha_gampera.pic.6}
