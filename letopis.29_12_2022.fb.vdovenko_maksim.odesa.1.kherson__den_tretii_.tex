%%beginhead 
 
%%file 29_12_2022.fb.vdovenko_maksim.odesa.1.kherson__den_tretii_
%%parent 29_12_2022
 
%%url https://www.facebook.com/snarkfog/posts/pfbid0Fifoxs7m25t93tPgGF3BiNkkwP3bLFYxZzidQchsiX5YTrt2Hw9MqN979yLjCuZ1l
 
%%author_id vdovenko_maksim.odesa
%%date 29_12_2022
 
%%tags herson
%%title Херсон. День третий. Чёрный день
 
%%endhead 

\subsection{Херсон. День третий. Чёрный день}
\label{sec:29_12_2022.fb.vdovenko_maksim.odesa.1.kherson__den_tretii_}

\Purl{https://www.facebook.com/snarkfog/posts/pfbid0Fifoxs7m25t93tPgGF3BiNkkwP3bLFYxZzidQchsiX5YTrt2Hw9MqN979yLjCuZ1l}
\ifcmt
 author_begin
   author_id vdovenko_maksim.odesa
 author_end
\fi

Херсон. День третий.

Чёрный день.

Специально пишу на языке, который нужно учить только для того, чтобы
допрашивать пленных. Чтобы могли почитать те, для кого \enquote{не всё так однозначно}.

Едем в микрорайон Остров с очередной гуманитаркой. Остров - самое опасное место
в Херсоне. Он ближе всего к линии фронта, которая проходит по Днепру. Остров
постоянно обстреливают. Там нет света и отопления и восстановить их невозможно,
пока рашистов не отбросят подальше от города. Очень быстро выгружаемся. Ко мне
подходит бабушка, спрашивает, когда же дадут свет.

- Никогда. Вам надо эвакуироваться. Пока идут обстрелы, невозможно восстановить
свет.

- Куда же я эвакуируюсь?

- Эвакуационные поезда в Хмельницкую область, там вас разместят, накормят,
будут выплачивать пособие.

- Мне 70 лет, я умру в своей квартире.

- Штош.

Эвакуация добровольная, никого насильно забирать мы не имеем права. Садимся
обратно в машину с водителем Сережей и быстро, очень быстро улетаем с Острова.
По дороге виден дым из развалин здания после очередного прилёта.

Пьем кофе, общаемся с другими волонтёрами. Здесь помогает много людей из разных
городов. Где-то неподалеку слышен постоянный гул канонады. Уже начинаем
потихоньку различать, где работает наша артиллерия, где вражеская. Местные
различают прекрасно.

Новая задача: вновь отвезти гуманитарку в пункт выдачи, но уже в другое место.

Приезжаем, заходим в здание (это важно), осматриваемся, и тут совсем близко
раздаются один за другим мощные раскатистые удары.

Саша кричит:

- Воздух! Ложись!

Только падаем на пол, как снаряд или град прилетает прямо в нас и взрывается на
улице в 7 метрах. От смерти нас отделяет только стена. Уши закладывает грохот,
окна вылетают. Всё заволакивает пылью. Первые секунды не понимаешь, что
происходит.

- Все живы??

- Там мої хлопці!

- Где?

- Там!

Они должны были быть на улице.

Бежим к выходу. Под ногами хрустят битые стекла, всё в пыли. Хлопцы живы, один
легко ранен осколками окна. Пара царапин, ничего серьёзного.

Опять слышим звуки, похожие на выходы артиллерии. Не можем понять, это орки
продолжают обстрел или это работает наша контрбатарейка. Бежим обратно в
здание. Всем страшно.

Садимся в самом защищенном месте, без окон, за двумя стенами. Убежища или даже
подвала здесь нет.

- Могут быть еще прилёты, не выходите на улицу! Не подходите к окнам!

В Телеге уже первые новости о местах прилётов, попадании возле ЦУМа, рынка,
больницы, фото погибших гражданских. Позже узнаем, что погибло 7 человек и
ранено 58, из которых 18 - тяжело. Русские мрази сознательно обстреляли центр
города. Пытаемся связаться с остальными группами волонтёров нашей команды. Все
живы-здоровы. Ник, волонтёр из Британии, находился возле ЦУМа и уехал за 3
минуты до прилёта. Раньше он ненавидел исключительно путина. Теперь он
ненавидит русских вообще ("это не лично путин нажал на кнопку, убившую людей в
Херсоне"). Добро пожаловать в наш клуб, Ник.

В штабе нам говорят, что оставаться в городе опасно, ожидают новых прилётов.
Приказывают эвакуироваться. Уезжаем в Николаев. Там напиваемся.
