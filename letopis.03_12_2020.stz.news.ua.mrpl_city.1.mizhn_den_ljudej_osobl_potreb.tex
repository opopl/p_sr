% vim: keymap=russian-jcukenwin
%%beginhead 
 
%%file 03_12_2020.stz.news.ua.mrpl_city.1.mizhn_den_ljudej_osobl_potreb
%%parent 03_12_2020
 
%%url https://mrpl.city/blogs/view/mizhnarodnij-den-lyudej-z-osoblivimi-potrebami
 
%%author_id demidko_olga.mariupol,news.ua.mrpl_city
%%date 
 
%%tags 
%%title Міжнародний день людей з особливими потребами
 
%%endhead 
 
\subsection{Міжнародний день людей з особливими потребами}
\label{sec:03_12_2020.stz.news.ua.mrpl_city.1.mizhn_den_ljudej_osobl_potreb}
 
\Purl{https://mrpl.city/blogs/view/mizhnarodnij-den-lyudej-z-osoblivimi-potrebami}
\ifcmt
 author_begin
   author_id demidko_olga.mariupol,news.ua.mrpl_city
 author_end
\fi

Сьогодні, \emph{3 грудня}, у всьому світі відзначають \textbf{Міжнародний день людей з
обмеженими можливостями}. Цей день було засновано в 1992 році Генеральною
Асамблеєю ООН. Традиційні заходи Дня людей з обмеженими можливостями
присвячуються висвітленню й вирішенню їхніх нагальних проблем.

Від того, як суспільство ставиться до тих, кому з тих чи інших причин важко або
неможливо самостійно подбати про себе залежить рівень цивілізованості цього
суспільства. Люди з особливими потребами не вимагають до себе жалю, тому що, як
ніхто давно усвідомили що це шлях внікуди. Вони пристосовуються як можуть зі
всіх своїх сил. Так сталося, адже на їхньому місці могла би опинитися будь-яка
інша людина. І для виживання таких людей не потрібно щось особливе, лише
елементарне – можливість жити, \emph{жити серед людей, бути в суспільстві і відчувати
себе звичайними людьми}, як і всім нам. І все ж, на мою думку, люди з обмеженими
можливостями вражають жагою до життя і силою духу – на їхньому прикладі
потрібно вчитися.

Вражаючим прикладом мужності є історія австралійського мотиватора та оратора
\emph{\textbf{Ніколаса Джеймса \enquote{Ніка} Вуйчича}}.

\ii{03_12_2020.stz.news.ua.mrpl_city.1.mizhn_den_ljudej_osobl_potreb.pic.1}

Ніколас народився з синдромом тетраамелії – рідкісним спадковим захворюванням,
що призводить до відсутності чотирьох кінцівок. Частково була одна стопа з
двома зрощеними пальцями, що дозволило хлопчикові після хірургічного поділу
пальців навчитися ходити, плавати, кататися на скейті, грати на комп'ютері і
писати. З 1999 року Ніколас почав виступати в церквах, тюрмах, школах і дитячих
притулках і незабаром відкрив некомерційну організацію \enquote{Життя без кінцівок},
почавши благодійну діяльність і допомагаючи інвалідам по всьому світу. Він
об'їздив понад 55 країн, виступаючи в школах, університетах та інших
організаціях.

\ii{03_12_2020.stz.news.ua.mrpl_city.1.mizhn_den_ljudej_osobl_potreb.pic.2}

Сьогодні, 3 грудня, ввечері у маріупольському драматичному театрі \href{https://mrpl.city/news/view/mariupoltsam-pokazhut-nesokrushimuyu-zhazhdu-zhizni-fridy-kalo}{відбудеться
прем'єра моновистави \enquote{Фріда}}%
\footnote{Мариупольцам покажут несокрушимую жажду жизни Фриды Кало, Богдан Коваленко, mrpl.city, 30.11.2020, \par%
\url{https://mrpl.city/news/view/mariupoltsam-pokazhut-nesokrushimuyu-zhazhdu-zhizni-fridy-kalo}
}
режисерки-по\hyp{}становниці \emph{\textbf{Людмили Колосович}}. У 6
років маленька Кало перенесла поліомієліт. Хвороба висушила її праву ногу, тому
жінка все життя ховала її під брюками або довгими національними сукнями. У 18
років дівчина потрапила в страшну автокатастрофу. Автобус, в якому вона їхала,
врізався у трамвай. Залізний прут струмоотримувача проткнув тіло дівчини,
пошкодив матку, зламавши тазостегнову кістку. А хребет був зламаний у 3 місцях,
ребра переламані. Незважаючи і всупереч всьому \emph{\textbf{Фріда Кало}} вижила. Але все в неї
всередині так і залишилося покаліченим. Тридцять дві операції, роки в гіпсі і
на інвалідному візку та нескінченний біль. Попри фізичний біль, найулюбленіша
художниця всього мексиканського народу занадто любила життя, щоб зневіритися чи
здатися.

В Маріуполі прикладом для наслідування залишається \href{https://mrpl.city/blogs/elena-molodanova}{\emph{\textbf{Олена Молоданова}}},
\footnote{\url{https://mrpl.city/blogs/elena-molodanova}}
\emph{співзасновниця громадської організації \enquote{Українські журналісти з інвалідністю}},
\emph{спортсменка, активний борець за права людей з інвалідністю}. Її енергія,
оптимізм, вміння радіти життю, досягати перемог і поставлених цілей вражали і
надихали багатьох містян.

\ii{03_12_2020.stz.news.ua.mrpl_city.1.mizhn_den_ljudej_osobl_potreb.pic.3}

Наразі у Маріуполі завдяки ентузіазму та небайдужості \href{https://mrpl.city/blogs/view/elena-berkova-za-vneshnej-hrupkostyu-skryvaetsya-silnyj-chelovek}{керівниці Дитячої
громадської організації \emph{\enquote{Маріупольська школа класичної хореографії} \textbf{Олені
Берковій}}}%
\footnote{Елена Беркова: за внешней хрупкостью скрывается сильный человек, ДК Молодежный, mrpl.city, 26.02.2019, \par%
\url{https://mrpl.city/blogs/view/elena-berkova-za-vneshnej-hrupkostyu-skryvaetsya-silnyj-chelovek}%
}
існує інклюзивний хореографічний клас, де діти займаються на
інвалідних візках. Наприкінці грудня, якщо не завадять, карантинні заходи,
колектив покаже дитячу балетну інклюзивну виставу  \enquote{Лускунчик}. Коли бачиш
щасливі очі дітей, що пересуваються на інвалідних візках, розумієш як їм
подобається виступати на сцені, серце переповнюється почуттям глибокої
вдячності за таку можливість. У ПК \enquote{Молодіжний}, як і в інших громадських
місцях Маріуполя, не вистачає пандусів. І піднімати дітей потрібно батькам та
викладачам. Але ця сцена дарує неймовірні емоції та незабутні враження. У цьому
колективі діти з особливими потребами можуть розкрити свій творчий потенціал і
відчути себе потрібними. Це дуже важливо для таких діточок та їхніх батьків.

Проблема людей з обмеженими можливостями є для мене дуже близькою та
зрозумілою, адже моя мама не ходить вже 14 років. На мою думку, українське
суспільство не зовсім готове сприймати людей із особливими потребами. Вони
щодня стикаються з непереборними перешкодами як у пошуках роботи, так і в
пересуванні по місту. В Україні й досі для людей-інвалідів не пристосований,
зокрема, ані громадський транспорт, ані підземні й наземні переходи, ані
магазини тощо. Я вважаю, що кожному з нас необхідно бути більш толерантними.
Тільки-но ми перестанемо цуратися людей з обмеженими можливостями і позбудемося
патології в суспільстві, яка не дозволяє нам спокійно сприймати всіх без
винятку, незважаючи на індивідуальні відмінності, саме тоді ситуація зминиться.
Люди з особливими потребами перестануть сидіти в чотирьох стінах, а ми з вами
ще на один маленький крок станемо ближчими до держави, в якій ми всі так хочемо
жити.

Люди з особливими потребами – такі ж як і ми. Їх багато, ми їх так мало бачимо
серед нас тому, що ми не вміємо бути серед них.

\textbf{Читайте також:} \emph{Международный день инвалидов: как помогают людям с инвалидностью в Мариуполе?}%
\footnote{Международный день инвалидов: как помогают людям с инвалидностью в Мариуполе?, Богдан Коваленко, mrpl.city, 03.12.2020, \par%
\url{https://mrpl.city/news/view/mezhdunarodnyj-den-invalidov-kak-pomogayut-lyudyam-s-invalidnostyu-v-mariupole}
}
