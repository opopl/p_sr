% vim: keymap=russian-jcukenwin
%%beginhead 
 
%%file topics.vojna
%%parent topics
 
%%url 
 
%%author_id 
%%date 
 
%%tags 
%%title 
 
%%endhead 

\clearpage
\chapter{Українська Вітчизняна Війна}

%\ii{topics.vojna.titlepage}

%\ii{23_02_2022.tg.tkachev_jurij.2.sprashivajut_storona_ukrainy}
%\ii{23_02_2022.tg.volja_vladimir.1.budjte_oberezhni}

\ii{topics.vojna.pics}
\ii{topics.vojna.linky}

\ii{topics.vojna.mariupol}
\ii{topics.vojna.harkov}
\ii{topics.vojna.irpen}

\ii{topics.vojna.mirnoe_naselenie}

\ii{topics.vojna.todo}

%\ii{topics.vojna.zagalne}
\ii{topics.vojna.ogljady}

% 24_02_2022 		- 		02_03_2022
\ii{topics.vojna.week.1}

% 03_02_2022 		- 		09_03_2022
\ii{topics.vojna.week.2}

% 10_03_2022 		- 		16_03_2022
\ii{topics.vojna.week.3}

% 17_03_2022 		- 		23_03_2022
\ii{topics.vojna.week.4}

% 24_03_2022 		- 		30_03_2022
\ii{topics.vojna.week.5}




% vojna.09_03_2022.1.pdf
% https://mega.nz/file/08E1QARZ#lptMoRBsXDBuR_CLQ98Wr44fUBqcVqFDVfKB6FTzgw0

%Доброго ранку, Руслане! Слава Україні! Сподіваємось, з Вами все добре. Посилаємо Вам версію
%літопису про війну, версія актуальна на сьогоднішній ранок. Там більш ніж 100 сторінок.
%Без сумніву, нам знадобиться Ваша допомога в оформленні, і в наповненні, і в ідеях, тому що
%все це виглядає ще дуже сиро, а наші можливості обмежені. Посилання на pdf-файл ми 
%зараз кинемо...
