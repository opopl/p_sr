% vim: keymap=russian-jcukenwin
%%beginhead 
 
%%file slova.istoria
%%parent slova
 
%%url 
 
%%author 
%%author_id 
%%author_url 
 
%%tags 
%%title 
 
%%endhead 
\chapter{История}
\label{sec:slova.istoria}

В українській \emph{історії} цей міф про спартанців також проявився. Відомою
стала \emph{історія} з козацьких воєн, розказана Миколою Костомаровим, який
своєю чергою переповів німецького хроніста Йоахіма Пасторія та польського
шляхтича Станіслава Освенцима. Під час битви під Берестечком сталася цікава
подія, яку хроністи описали приблизно так: «Одна ватага козацька, зложена з
двох або трьох сот козаків, зробивши засіку на однім острівці, ставила мужній
опір так відважно, що Потоцький велів їм обіцяти життя, коли піддадуться. Але
вони не схотіли того прийняти. На знак повного відречення викинули з чересів
своїх гроші у воду, і так сильно почали відбиватися від наших, що кінець кінцем
піхота мусила на них гуртом наступати. А хоч розірвала (їх укріпленнє) і
розігнала з острову — але вони уступилися на болота, не хотячи піддатись, так
що там їх кожного доводилось добивати. А оден з них, добившися до човна, на
очах короля і всього війська дав докази мужності не хлопської! Кілька годин
відбивався він з того човна косою, зовсім не зважаючи на стрільбу — котра чи то
припадком його не трапляла — чи такий твердий був, що кулі його не брали. Аж
оден Мазур з Ціхановецького повіту, розібравшися на голо і по шию бредучи,
насамперед косою вдарив, а потім оден солдат пробив посередині пікою або списом
і добив його таким чином — на велику потіху і вдоволеннє короля, що довго
дивився на сю траґедію»,
\citTitle{Як метафори та міфи керують нашою уявою}, Богдан Шумилович, zaxid.net, 09.06.2021

Люди люблять виразні \emph{історії}, особливо якщо вони драматичні. Такою є \emph{історія}
про 300 спартанців, які захистили батьківщину від більшої кількісно армії
персів. Проте якщо ви дивилися фільм Зака Снайдера, то могло виникнути
відчуття, що перед вами радше ідеологічний соцреалізм, історія про боротьбу
американців зі злим Іраном, а не сюжет про давнє минуле.  У сонячний день
серпня 480 року до н. е. грецьке місто-держава Спарта святкувало одне з
найважливіших фестивалів року, тижневе свято «Карнея», яке було даниною поваги
Аполлону Карнейосу, богу зграй та стад домашньої худоби. На час празнику було
заборонено брати участь у будь-якій війні, щоб бог не зіпсував джерела їжі на
наступний рік. Зазвичай це не було проблемою, але тут з'явився перський цар
Ксеркс, що зібрався помститися за свого батька, який 10 років тому безуспішно
намагався захопити Грецію,
\citTitle{Як метафори та міфи керують нашою уявою}, Богдан Шумилович, zaxid.net, 09.06.2021

Люди схильні вірити у все, що заманеться – брехливим каналам Медведчука чи
«чесним» медіа Порошенка, незрозумілим сайтам типу «Страна», які поширюють
фейки та міфи. Чому, запитаєте ви? Дуже просто: реальність не має смислу,
\emph{історія} – це теж набір беззмістовних сюжетів. Смислу потоку \emph{історії} надають
\emph{історики} чи письменники, а в нашому світі – кінематографісти чи навіть блогери.
Хто створить більш переконливу картинку реальності, той і перемагає в битві за
уяву. І ця картинка не мусить бути правдивою. Окрім того, люди схильні
сприймати світ через стереотипні історії, і якщо ми прийняли якийсь міф чи
метафору з раннього віку, то її дуже складно змінити. Іншими словами, коли я
бачу подію чи якийсь феномен у реальності, я його співвідношу з уже наявним у
моїй голові образом. Себто коли польський шляхтич бачив, як б'ється на смерть
ватага козаків, то він побачив \emph{античну історію}, яку засвоїв ще дитиною від
своїх вчителів. Ми розуміємо реальність лише тоді, коли вона має для нас смисл!,
\citTitle{Як метафори та міфи керують нашою уявою}, Богдан Шумилович, zaxid.net, 09.06.2021

Новое время родило когорту историков-самоучек, которые умеют выжать сенсацию
нового знания буквально из одного документа. Даже целые книги пишут. Вот, к
примеру, наткнулся на очередной тг-канал с секретами тайных архивов кейджиби. А
там выложили уникальное свидетельство подлых замыслов Империи Зла - мишени
Осоавиахима. Враг на них носит каску Броди (Mk III), следовательно \enquote{в СССР
учились стрелять по британцам}. Но камон, друзья, такие же каски носили в
армии США. Значит в СССР готовились стрелять по американцам. А форма на парне
похожа на немецкую, так что СССР готовился первым напасть на Германию, а
британцев косить уже потом. И самое главное: в конце 1930-х годов в РККА было
решено создать стальной шлем для частей ПВО и гражданской обороны. За образец
взяли каску Броди и с 1938 года начали делать ее на Ленинградском металлическом
заводе.  Следовательно, в СССР готовились стрелять по своим! Страшная правда
про заградотряды ПВО раскрыта,
\citTitle{Новое время родило когорту историков-самоучек}, Дмитрий Заборин, strana.ua, 10.06.2021


Всім відомо, що \emph{історія} не знає умовного способу, що минуле взагалі краще не
чіпати окрім як з виховною чи пізнавальною метою. Всі знають і все одно
регулярно граються в \emph{альтернативну історію}. Бо це, зрештою, не так про минуле,
як про сьогодення. Яким би воно могло бути, \enquote{якби} – так, це закляття чарує хай
там що, надто в суспільствах, що ніяк не виборсаються з перехідного періоду.
Ось типова мрія про минуле: Галицько-Волинське князівство поступово стає одним
з наймогутніших східноєвропейських гравців, укладає кілька вигідних
регіональних пактів, звільняє українські землі від Орди і перетворюється на
Королівство Русь. Місцеве самоврядування та народні вольності посилюються, так
що в середині ХІХ століття Русь стає конституційною монархією і входить у
новітню добу цивілізованою високорозвиненою демократією,
\citTitle{Якби | Газета «День»}, Дмитро Десятерик, day.kyiv.ua, 09.06.2021


%%%cit
%%%cit_pic
%%%cit_text
Думаю, пора уже американскому ВПК платить России \enquote{откат}, ведь без
разрекламированной \enquote{российской угрозы} вооружение США перестали бы покупать и
миллионы рабочих мест в США пришлось бы сокращать))) Для Украины данная \emph{история}
показательна тем, что налаженное производство самолетов марки \enquote{Антонов}, это 80
000 рабочих мест НА УКРАИНЕ, а не в США или ЕС.  Об этом прекрасно знали и в ЕС
и в США, по этому \enquote{наши партнеры} руками продажных шкур во власти \enquote{Антонов}
практически уничтожили. Зато у нас теперь благодаря \enquote{нашим партнерам}, рабочие
места массово создают в зоне боевых действий на Донбассе, в виде \enquote{зарплат}
солдатам контрактникам за убийства украинцев
%%%cit_comment
%%%cit_title
\citTitle{Forbes призывает Европу срочно покупать американские противолодочные самолеты}, 
Дмитрий Василец, strana.ua, 14.06.2021
%%%endcit

%%%cit
%%%cit_pic
\ifcmt
  pic https://gdb.rferl.org/957CC1E1-23E0-4D73-B1CB-B9DDAA77B5EE_w1597_n_r1_st.jpg
	caption Остромирове Євангеліє – одна з найвидатніших пам’яток письменства періоду України-Русі. Переписане у 1056–1057 роках. Пам'ятка української мови
\fi
%%%cit_text
Перша міфологема називає тисячолітній рукопис «культурным достоянием России»
відповідно до старої імперської схеми: Київська Русь – колиска трьох братніх
народів, старший брат – російський народ, а відтак уся спадщина тисячолітньої
Русі – «достояние России».  На відміну від ідеологічних схем, \emph{історичні факти}
та наукові висновки у цій темі свідчать про інше. А саме: переписувач
Остромирового Євангелія – дяк Григорій – був киянином, замовник – Остромир –
походив з київського боярського роду і переписували та оздоблювали славетний
манускрипт у Києві, в 11-му столітті – тобто за великокнязівської доби
української історії. Цей період в \emph{офіційній історії} нашої країни сьогодні
називають Україна-Русь або Київська держава. Таку назву запровадив видатний
\emph{історик} Михайло Грушевський
%%%cit_comment
%%%cit_title
\citTitle{Чому написане у Києві Остромирове Євангеліє – «достояние России»?},
Ірина Костенко; Ірина Халупа, radiosvoboda.org, 13.06.2021
%%%endcit

%%%cit
%%%cit_pic
%%%cit_text
У Росії тих років, років молодості Путіна, було прийнято ігнорувати цілі пласти
\emph{української історії} або оголошувати їх частиною \emph{історії} російського народу. Так
і виходило, що на території Радянського Союзу єдиним \enquote{повноцінним} державним
утворенням була саме Росія. А інші народи, якщо і створювали держави, так потім
повинні були шукати в Росії захисту від ворога. І вже тому будь-які спроби
відділення від Російської імперії, а потім від Радянського Союзу, сприймалися
носіями таких поглядів як прояв справжньої невдячності
%%%cit_comment
%%%cit_title
\citTitle{Путін знову береться за історію України}, 
Віталій Портников, gazeta.ua, 14.06.2021
%%%endcit

%%%cit
%%%cit_head
%%%cit_pic
%%%cit_text
Таким образом, мы видим, что \emph{история} католической экспансии не закончена. Она
не может быть закончена, пока не будет достигнута главная её цель – отрыва
части русских от русского мира на всех уровнях, раздробления и ослабления
России 
%%%cit_comment
%%%cit_title
\citTitle{Католическая экспансия на русские земли: история и современность}, 
Светлана Пикта, teleskop.media, 14.06.2021
%%%endcit

%%%cit
%%%cit_head
%%%cit_pic
%%%cit_text
Страна, которая не имеет \emph{истории} — не имеет будущего. Страна, которая живет
прошлым, никогда не увидит ни настоящего, ни будущего. Страна, у которой нет
\emph{исторических корней}, будет оккупантом, будет поглощать и мимикрировать,
стараясь, стать похожей на народ, нацию, страну.  У Российской Федерации
\emph{история} начинается с 12 июня 1990 года. У СССР с 30 декабря 1922 года. \emph{История}
Российской Империи начинается с 2 ноября (22 октября) 1721 г. Далее \emph{история}
Московии, московско-хазарского улуса и московских болот. И, что самое главное,
это отдельная \emph{история}, отдельных государств, а не \enquote{\emph{история} России}, как пишут
российские \enquote{\emph{историки}}
%%%cit_comment
%%%cit_title
\citTitle{Страна без исторических корней всегда будет оккупантом}, 
Олена Степова, news.obozrevatel.com, 17.06.2021
%%%endcit

%%%cit
%%%cit_head
%%%cit_pic
%%%cit_text
Современная Россия отчаянно старается написать себе \emph{тысячилетнюю историю},
опираясь то на правопреемничество СССР, то на правопреемничество монархии, то и
во все, смешивая в кучу коммунизм и монархизм, не зря же, потомок царя Николая
лобызал дуло пушки на крейсере \enquote{Аврора} с криками \enquote{не забудем-повторим}. В
общем, лучшее в мире советское образование, дает о себе знать. СССР создавал
симулякры, особенно \emph{исторические}. Результат этого \emph{исторического коктейля}
страшен. В головах россиян смешано все, коммунизм, капитализм, монархизм, трусы
и крестик
%%%cit_comment
%%%cit_title
\citTitle{Страна без исторических корней всегда будет оккупантом}, 
Олена Степова, news.obozrevatel.com, 17.06.2021
%%%endcit

%%%cit
%%%cit_head
%%%cit_pic
%%%cit_text
Всего на территории Польши и Украины в течение 30 лет после войны планировалось
заселение восьми миллионов немцев. А местного населения в европейской части
СССР немцы собирались оставить около 15-20 миллионов человек.  Насколько такие
планы могли быть реализованы, даже в случае захвата всей территории до Урала -
вопрос отдельный.  Но даже их наличие говорит многое о том, что уравнивать
нацистский и советский \enquote{тоталитарные режимы} - это явная
\emph{историческая} манипуляция. Немцы вели речь об уничтожении украинцев,
русских и белорусов как народов. И жители Украины того времени в массе своей
это прекрасно понимали даже без прочтения \enquote{плана Ост} - большая их
часть воевала в рядах Красной армии
%%%cit_comment
%%%cit_title
  \citTitle{22 июня - 80 лет нападения на СССР. Что немцы готовили для украинцев}, Максим Минин, strana.ua, 22.06.2021
%%%endcit

%%%cit
%%%cit_head
%%%cit_pic
%%%cit_text
У нас есть возможность, но нет желания повторять \emph{историю}. Потому что мы помним
чем может обернуться это повторение. И никому не желаем повторения \emph{исторических}
событий связанных с приходом в гости незваных гостей. Это не значит, что мы
боимся, это значит, что не хотим причинять зло своим соседям.  Мнение простого
обывателя, желаю всем добра
%%%cit_comment
%%%cit_title
\citTitle{И чего они так боятся, мы же добрые и пушистые}, 
Мнение Простого Обывателя., zen.yandex.ru, 28.05.2021
%%%endcit


%%%cit
%%%cit_head
%%%cit_pic
%%%cit_text
Духовное развитие есть там, где вы признаете общее благо, где вы признаете
трансцендентное, где вы признаете такие вещи, как всечеловечество, Бог, не
отрекаетесь от Бога во имя крови врага, а признаете, что есть Бог независимо от
того, какие у тебя отношения с врагом. Это очень важная вещь, потому что в этом
возможно духовное развитие. Если страна прекращает духовное развитие своей
элиты, она оказывается на \emph{мусорнике истории}.  Украина сегодня не просто
попрошайка, она на \emph{мусорнике истории}, потому что она не способна оформить общее
благо через иные концепты, нежели национал-патриотический. Мыслили, мыслили и
скатились в то, что было придумано более двухсот лет назад. Вот все, что мы
можем. Это же чушь. А что же такое достоинство?
%%%cit_comment
%%%cit_title
\citTitle{Сергей Дацюк: Украина сегодня - не просто попрошайка, она на мусорнике истории}, 
Сергей Дацюк; Людмила Немыря, hvylya.net, 28.06.2021
%%%endcit

%%%cit
%%%cit_head
%%%cit_pic
\ifcmt
  pic https://img.strana.ua/img/article/3405/kommunisticheskoj-partii-kitaja-96_main.jpeg
  width 0.4
	caption В этом году Коммунистическая партия Китая будет праздновать свой 100-летний юбилей
\fi
%%%cit_text
\emph{История} – это политика, обращенная в прошлое. И нет, пожалуй, более
показательного подтверждения данного высказывания, чем история Коммунистической
партии Китая – причем как в самой КНР, так и за рубежом.  Столетие незаметного
события 1921 года – первый съезд КПК – на днях будет праздноваться
полуторамиллиардной страной как ее главный праздник. 1 июля по этому поводу в
КНР пройдут массовые торжества.  Но если \enquote{политизация} собственной
\emph{истории} каждым государством объяснима и логична, то \enquote{качели} в
зарубежной оценке событий и личностей современной \emph{истории} Китая не могут не
удивлять.  И первый съезд Компартии является точкой отсчета для всех версий
этой \emph{истории}
%%%cit_comment
%%%cit_title
\citTitle{Коммунистической партии Китая сто лет - Как появилась КПК}, 
Дмитрий Коротков, strana.ua, 28.06.2021
%%%endcit

%%%cit
%%%cit_head
%%%cit_pic
%%%cit_text
Уроки \emph{історії} завжди болючі. Адже закон небес простий: хто їх не засвоює, той
вивчатиме їх знову і знову.  Назвіть мені націю, яка б не мала своїх травм, що
мігрували з попередніх століть, від покоління до покоління, що викликають
комплекси, мало не фізичні муки умів, які, не засвоївши попередніх уроків,
намагаються осягнути причини, чому вони не вміють жити інакше.  Мало кому
вдається проробити таку колосальну роботу так успішно і швидко в масштабах
нації, як, скажімо, це вдалося Німеччині
%%%cit_comment
%%%cit_title
\citTitle{Українці не розуміють одне одного не через мову, а через небажання слухати, чути і сприймати}, 
Юлія Мендель, www.pravda.com.ua, 07.07.2021
%%%endcit

%%%cit
%%%cit_head
%%%cit_pic
%%%cit_text
И вот произошел обстрел и Богдана не стало. Меня колбасит три дня, хотя вроде
бы и не совсем близкий мне человек. Но я уже три дня не могу его отпустить. Как
будто убита моя частичка детства. А еще я видел, как моя мама – она сейчас у
меня в Киеве – боялась полдня позвонить тете Оле. Что можно сказать подруге,
которая прямо сейчас хочет лечь в один гроб к своему сыну?
И знаете, что самое мерзкое в \emph{истории}, в которой обрывается жизнь молодого
пацана? Она абсолютно бессмысленна. Не говорите, что он умер за свою страну или
за свою семью. Вам это непременно озвучит элитка, чьи дети учатся в Лондоне, и
у которой недвижка в Испании и Хорватии. Богдан погиб там, где у власти нет ни
стратегической, ни тактической цели. Его смерть ничего не приближает и ничего
не разрешает.  Ладно, этой элитке не приемлемо выполнение Минских соглашений.
Строя ТАКУЮ Украину их в принципе невозможно выполнить, вопрос закрыт
%%%cit_comment
%%%cit_title
\citTitle{Наши ребята гибнут там, где у власти нет ни стратегической, ни тактической цели}, 
Игорь Лесев, strana.ua, 08.07.2021
%%%endcit

%%%cit
%%%cit_head
%%%cit_pic
\ifcmt
  pic https://img.strana.ua/img/article/3432/statja-vladimira-putina-69_main.jpeg
  width 0.4
	caption Фото: РИА Новости 
\fi
%%%cit_text
Сразу подчеркну, что стену, возникшую в последние годы между Россией и
Украиной, между частями, по сути, одного \emph{исторического} и духовного
пространства, воспринимаю как большую общую беду, как трагедию. Это, прежде
всего, последствия наших собственных ошибок, допущенных в разные периоды. Но и
результат целенаправленной работы тех сил, которые всегда стремились к подрыву
нашего единства. Формула, которая применяется, известна испокон веков: разделяй
и властвуй. Ничего нового. Отсюда и попытки сыграть на национальном вопросе,
посеять рознь между людьми. А как сверхзадача – разделить, а затем и стравить
между собой части единого народа.  Чтобы лучше понять настоящее и заглянуть в
будущее, мы должны обратиться к \emph{истории}. Конечно, в рамках статьи
невозможно охватить все события, произошедшие более чем за тысячу лет. Но
остановлюсь на тех ключевых, поворотных моментах, о которых нам – и в России, и
на Украине – важно помнить
%%%cit_comment
%%%cit_title
\citTitle{Статья Путина \enquote{Об историческом единстве русских и украинцев}. Полный текст}, 
Игорь Рец, strana.ua, 12.07.2021
%%%endcit

%%%cit
%%%cit_head
%%%cit_pic
%%%cit_text
Национальный \emph{исторический миф} нужно строить на героях и праведниках, а не
обидах и поражениях.  В эти дни в Польше проходят памятные мероприятия
приуроченные к 78-й годовщине Волынской трагедии - убийства более 50 тыс.
поляков и нескольких тысяч украинцев.  А украино-польские отношения ожидает
ежегодное плановое обострение, когда одна сторона будет настаивать на \enquote{геноциде
и ответственности за него}, а вторая \enquote{на гражданской войне и старых обидах} -
обе на своих жертвах.  Ведь обе стороны пытаются политически
инструментализировать эту трагедию, сделать её часть \emph{исторического мифа},
комплекса \emph{исторической жертвы} с претензией на политические дивиденды в будущем.
Конфликт.  Но есть другой подход
%%%cit_comment
%%%cit_title
\citTitle{Национальный исторический миф нужно строить на героях и праведниках}, Руслан Бортник, strana.ua, 11.07.2021
%%%endcit

%%%cit
%%%cit_head
%%%cit_pic
\ifcmt
  pic https://gdb.rferl.org/66D1B9A8-875E-49FC-AA01-7D1C159D6E05_w650_r1_s.jpg
  width 0.4
	caption Олексій Гарань, політолог, історик
\fi
%%%cit_text
\emph{Історик} Олексій Гарань в розмові з Радіо Свобода уточнює, що насправді Путін
повертається до імперських тез більше, ніж до радянських, бо за Радянського
Союзу українці та росіяни були не одним народом, а двома, хоча і братніми.
«Те, що зараз проголошує Путін, це суперечить навіть радянській точці зору, це
суперечить тому, що казав Ленін, і це повернення до «самодєржавста». Тобто
«один народ, одна віра, і один самодержець», – пояснює професор університету
«Києво-Могилянська академія», науковий директор фонду «Демократичні ініціативи»
імені Ілька Кучеріва Олексій Гарань
%%%cit_comment
%%%cit_title
\citTitle{Спецоперація під назвою «стаття Путіна»}, 
Марія Щур; Сашко Шевченко, www.radiosvoboda.org, 13.07.2021
%%%endcit

%%%cit
%%%cit_head
%%%cit_pic
%%%cit_text
«У самій Росії люди сидять в тюрмі за іншу інтерпретацію \emph{історії}, за те, що
вони можуть порівняти режим Гітлера і Сталіна», – пояснює Любов Цибульська.
«Маніпуляція з \emph{історією} в Росії створює певну ідентичність та певне ставлення
до інших народів. Якщо їм вдасться виховати нове покоління з таким ставленням,
то можна з легкістю гарантувати, що воно буде агресивно ставитися до сусідів,
буде готове воювати за імперські погляди», – вважає Любов Цибульська.
Тому, на її думку, вивченню своєї справжньої \emph{історії}, популяризації її через
художні твори, кіно, телесеріали, має бути для українців важливою формою
самозахисту
%%%cit_comment
%%%cit_title
\citTitle{Спецоперація під назвою «стаття Путіна»}, 
Марія Щур; Сашко Шевченко, www.radiosvoboda.org, 13.07.2021
%%%endcit

%%%cit
%%%cit_head
%%%cit_pic
%%%cit_text
С точки зрения хуторянского мышления все очевидно: раз страна называется
Украина, то и язык должен быть только украинским, и население должно быть
исключительно украинцами, и \emph{история} должна писаться исключительно в русле
украинской государственности. Но современным идеологам не приходит в голову,
что подобная идеология имеет и оборотную сторону медали
%%%cit_comment
%%%cit_title
\citTitle{О будущем украинского и русского народов}, 
Виктор Медведчук, strana.ua, 15.07.2021
%%%endcit


%%%cit
%%%cit_head
%%%cit_pic
%%%cit_text
\emph{Історична пам'ять} у столиці України обернено пропорційна поверховості
висоток.  Чим більше новобудов, чим вище хмарочоси, тим смутніші спогади про
те, яким був Київ у минулому.  Що було на місці ЖК "Ярославів Град" і як
виглядав Сінний ринок? Яким був прибутковий будинок Уткіна, знесений три місяці
тому? Що таке садиба Мурашка, і в чому її цінність?  Що ви знаєте про будинок
Лошевича, найстарішу будівлю на вулиці Саксаганського, яку розвалили восени
2020-го? Чому будинок "Квіти України", по якому пройшлися ковшем екскаватора, –
не потворна радянська "коробка", а навпаки?  У більшості киян відповідей на ці
незліченні запитання немає. А меншість поки тільки вчиться їх усе голосніше та
виразніше ставити.  Юліан Чаплінський, урбаніст, колишній головний архітектор
Львова, розглядає знищення \emph{історичної спадщини} у столиці і в цілому в
Україні як частину глибинної проблеми
%%%cit_comment
%%%cit_title
\citTitle{Урбаніст Чаплінський: Київ – це така \enquote{згвалтована мама}. Місто б'ють, а воно все одно всіх любить}, 
Євген Руденко, www.pravda.com.ua, 23.07.2021
%%%endcit

%%%cit
%%%cit_head
%%%cit_pic
%%%cit_text
Главное в этом году - резкое, практически беспрецедентное в \emph{истории}
нашего народа обнищание людей. Советскую экономику новые власти остановили, а
рыночную еще не создали.  Нерв 1992 года - это первые попытки руководства
страны отступить от рамки, которую задал 1991 год. Когда Украина возникла как
плод компромисса очень разных групп (в том числе представителей партаппарата,
директоров предприятий), большинство из которых, так же как и в целом население
(что показало голосование за Кравчука, а не за Черновола) не были настроены
националистически и не хотели полностью рвать с Россией.  По факту же в 1992
году последовало очень много тревожных звоночков: первые конфликты вокруг
Черноморского флота и церковный раскол с попыткой вывести УПЦ из-под власти
Московского патриархата. Плюс массированное навязывание националистической
повестки в гуманитарной сфере.  Причем последним занимались зачастую те, кто
еще недавно преподавал \emph{историю КПСС}, научный коммунизм, а то и вовсе
были партаппаратчиками. В течение буквально одного-двух месяцев они
перекрасились и превратились в украинских националистов.  В целом
националистический крен и рост напряженности в отношениях с Россией стали
серьезным фактором внутренней дестабилизации, что привело к серьезным
последствия уже в следующем 1993 году.  "Страна" вспоминает главные события
1992 года и анализирует их
%%%cit_comment
%%%cit_title
\citTitle{Хмурое утро независимости. Чем запомнился украинцам шоковый 1992 год}, 
Игорь Гужва; Олеся Медведева; Максим Минин, strana.ua, 27.07.2021
%%%endcit

%%%cit
%%%cit_head
%%%cit_pic
%%%cit_text
Путін своєю абсолютно безграмотною \emph{«історичною»} статтею про єдиний трьохголовий
народ відкрив скриньку Пандори.
Зеленський відповів Путіну спочатку коротко: «ми не один народ», і була надія,
що він цим обмежиться, цілком переконливо заявивши, що писати \emph{історичні}
трактати — справа істориків, а доля політиків — створювати кращі умови  для
життя своїх сучасників. Але Президент України не втримався й учора, виступаючи
з нагоди річниці хрещення Київської Русі, дав Путіну  цілком розгорнуту
відповідь із яскраво вираженими особистими інтонаціями:
«Київська Русь — це мати нашої \emph{історії}. 24 області України та півострів Крим —
її рідні діти. І вони за правом — її спадкоємці. А двоюрідним племінникам і
дуже далеким родичам не треба зазіхати на її спадок. І намагатися довести свою
причетність до \emph{історії} тисячі років і тисячі подій, перебуваючи від місць, де
вони відбувалися, за тисячі кілометрів». Кінець цитати
%%%cit_comment
%%%cit_title
\citTitle{Який народ хрестив князь Володимир?}, 
Ігор Яковенко, day.kyiv.ua, 29.07.2021
%%%endcit

%%%cit
%%%cit_head
%%%cit_pic
%%%cit_text
ФІАСКО ЕТНІЧНОЇ УКРАЇНІЗАЦІЇ. "Група Рейтинг" надто поважна соціологічна
служба, а Oleksiy Antypovych надто авторитетний соціолог-практик, аби
запідозрити їх у розповсюдженні замовної провокації. Схоже, соціологія лише
зафіксувала повне фіаско "розбудови нації" за лекалами Донцова, яку українська
влада практикує останні 7 років...  Здавалося б, ніхто у гуманітарній сфері так
наполегливо не втілював гасло "Геть від Москви!" як українська влада за часів
Порошенка (та й Зеленського також). Топонімічна революція, пам'ятникопад,
наполегливі намагання стерти з \emph{історичної пам'яті} усі об'єктивні досягнення
часів перебування України в складі російської імперії та СРСР, нав'язування
антиісторичного наративу України як колонії, глорифікація Степана Бандери,
примусова українізація сфери обслуговування, схоже, дали абсолютно зворотній
ефект.  Україна виявилась надто складною та розмаїтою, аби втиснути її в
прокрустове ложе етнічної нації: одна мова, одна \emph{історія}, одна релігія
%%%cit_comment
%%%cit_title
\citTitle{Фіаско етнічної українізації}, 
Генадій Друзенко, analytics.hvylya.net, 28.07.2021
%%%endcit

%%%cit
%%%cit_head
%%%cit_pic
%%%cit_text
Как известно, \emph{история} не только является наукой о прошлом, но также
выполняет важнейшую функцию социальной памяти. Поэтому отношение к своему
прошлому является одной из главных характеристик состояния общества.  В этой
связи современное представление «Украины» о своём прошлом свидетельствует о
тяжёлом состоянии, в котором находится современное «украинское общество»,
живущее \emph{историческими} галлюцинациями и фобиями, страдающее от комплекса
\emph{исторической неполноценности}.  Очевидная неудовлетворённость «украинцев»
своим прошлым приводит их к стремлению раздвинуть и обогатить скудный,
малонаселённый \emph{исторический} ландшафт путём «\emph{исторической
апроприации}» чужих событий, явлений, героев
%%%cit_comment
%%%cit_title
\citTitle{Синдром киевской исторической неполноценности}, 
Олег Сапожников, regnum.ru, 21.07.2021
%%%endcit

