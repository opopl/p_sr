% vim: keymap=russian-jcukenwin
%%beginhead 
 
%%file 30_05_2021.fb.zharkih_denis.1.kto_my
%%parent 30_05_2021
 
%%url https://www.facebook.com/permalink.php?story_fbid=2998498060363591&id=100006102787780
 
%%author_id zharkih_denis
%%date 
 
%%tags future,ideologia,medvedchuk_viktor,strana,ukraina
%%title Кто мы?
 
%%endhead 
 
\subsection{Кто мы?}
\label{sec:30_05_2021.fb.zharkih_denis.1.kto_my}
 
\Purl{https://www.facebook.com/permalink.php?story_fbid=2998498060363591&id=100006102787780}
\ifcmt
 author_begin
   author_id zharkih_denis
 author_end
\fi

Кто мы? 

Суд над Виктором Медведчуком в любом случае решит судьбу Украины. Вопрос суда
не юридический, а чисто политический: что Украина выберет - законность или
политический вектор. 

\ifcmt
  pic https://scontent-frt3-1.xx.fbcdn.net/v/t1.6435-9/192178322_2998480670365330_3502454640574524436_n.jpg?_nc_cat=102&ccb=1-5&_nc_sid=730e14&_nc_ohc=a0y1Yb0Jw8QAX92DePL&_nc_ht=scontent-frt3-1.xx&oh=7c3ffc87077b624d38fb135a9d0d3490&oe=619C2D71
  @width 0.6
\fi

Подобный суд стал вообще возможен благодаря тому, что украинские политики после
второго Майдана начали закреплять некоторые политические взгляды в законы. То
есть, постепенно, не сразу, некоторые политические взгляды в Украине становятся
незаконными. Этот процесс еще не завершен, но осуждение Виктора Медведчука
поставит в этом вопросе жирную точку. А именно то, что гражданин Украины не
имеет право оспаривать политический курс страны, это определяет, кто угодно, но
не он. А если он сдуру попытается на это курс повлиять, то будет осужден. 

Все, кто потирает руки, и думает, что на место идеолога оппозиции придет другой
кандидат, ошибаются. Места оппозиции в Украине после осуждения Медведчука не
будет. Любого оппозиционера, который не согласен с курсом страны, будет ждать
тюрьма. 

Но ведь ничего нового. И в Российской Империи, и в СССР оппозиции не было, ее
оприходовали в тюрьму. Имеем дело с традициями, но вот вопрос, к чему они
приведут. СССР трудно назвать демократическим государством, но это было
социальное государство, хоть и ущемлявшее права богатых, но решавшее проблемы
бедных. Украина совершенно не социальное государство, потому, кроме нищеты, нас
ждет и несвобода. Украина продолжает двигаться в идеологии Российской империи -
"Армия. Мова. Вера". Растет класс новых феодалов, "понадусеры", которые имеют
право доить всех остальных. Есть свои великороссы, есть черносотенцы, царь
батюшка, значит, будут и бомбисты, эсэры и революционные матросы. Ну, это если
по традициям пойдем. 

Возвращаясь к Медведчуку нужно подчеркнуть один его поступок - он не убежал, а
сам пришел в прокуратуру. Агенты других стран так не поступают, так поступают
граждане своей страны. Виктор Медведчук, рискуя свободой, а, возможно, жизнью,
не только себя, но и своих близких, поставил перед страной вопрос: "Кто мы?". И
это вопрос ждет ответа. Что для украинца ближе - законность или политическая
конъюнктура? Сразу забегу вперед - если политическая конъюнктура, то кто нам
мешал колебаться с линей партии в СССР? А если мы купили свободу и дорого за
нее заплатили, то почему сейчас должны отдать ее Зеленскому? Он заслужил такой
подарок? 

Суд над Медведчуком в любом случае войдет в историю Украины. Тут не может быть
то-то среднее арифметическое - либо оправдан, либо осужден. И либо в стране все
диктуют прозападные политики, которые сдают украинские интересы по первому
требованию, а украинских граждан упекают за решетку вне всяких законов, либо в
этой стране есть закон, который защитит любого гражданина. Ведь Медведчук
действовал, как оппозиционер, и ничего незаконного не делал. 

Ну и последнее, суд над Медведчуком решает - мир или война. Политические мотивы
осуждения Медведчука просты - вел переговоры с врагом. Если осудят, то у нас
война, а не агрессия, потому, что говорить с агрессором и даже с террористами
законы не запрещают. Но тогда вопросы к Зеленскому, а он себя как вел на войне?
И тогда его тоже можно судить на том же основании, да, собственно, и Порошенко.
А вот оправдательный оправдательный приговор, тоже приговор Зеленскому, что ж
он человека мира обвинил. Короче, Зеленскому, что в лоб, что по лбу, это
процесс приговор. Зачем он его затеял? Выигрыша тут ему, даже теоретически, не
светит. 

А Медведчук выиграет в любом случае, вопрос, какую цену, он за это заплатит, но
он кинул в игру все, что не сделал пока ни один украинский оппозиционер.
Выиграет ли украинский народ, вот в чем главный вопрос? А тут тоже понятно, как
ответит на вопрос "Кто мы?", так и будет жить. И никак иначе.

\ii{30_05_2021.fb.zharkih_denis.1.kto_my.cmt}
