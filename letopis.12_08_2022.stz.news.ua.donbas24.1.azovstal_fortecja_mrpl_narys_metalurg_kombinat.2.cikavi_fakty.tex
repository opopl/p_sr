% vim: keymap=russian-jcukenwin
%%beginhead 
 
%%file 12_08_2022.stz.news.ua.donbas24.1.azovstal_fortecja_mrpl_narys_metalurg_kombinat.2.cikavi_fakty
%%parent 12_08_2022.stz.news.ua.donbas24.1.azovstal_fortecja_mrpl_narys_metalurg_kombinat
 
%%url 
 
%%author_id 
%%date 
 
%%tags 
%%title 
 
%%endhead 

\subsubsection{Цікаві факти про \enquote{Азовсталь}}

При побудові заводу \enquote{Азовсталь} у 30-тих роках минулого сто\hyp{}ліття було виявлено
маріупольський могильник, датований 3 тисячоліттям до нашої ери. Згодом були
проведені розкопки та винайдені унікальні артефакти та поховання епохи
енеоліту.

% 7 - Йде зведення заводу «Азовсталь», 1931 рік.
\ii{12_08_2022.stz.news.ua.donbas24.1.azovstal_fortecja_mrpl_narys_metalurg_kombinat.pic.7}

Історія кожного цеху комбінату \enquote{Азовсталь} заслуговує на окремий літопис. Адже
усі заводські підрозділи за часи свого існування мають визначні дати, а головне
— пов'язані з долею людей, які \enquote{підіймали завод}. Однією з таких визначних
постатей був Володимир Лепорський. З 1938 року він працював на \enquote{Азовсталі}
спочатку начальником зміни в мартенівському цеху, з 1955 року — головним
інженером, а з 1956 року до 1981 — директором заводу \enquote{Азовсталь}. Саме
Володимир Лепорський визначив міць комбінату, традиції, характер і перспективи
на десятиліття вперед. А ще — ідею створити стислий і впізнаваний логотип
\enquote{Азовсталі}, в колективі підхопили цю думку і розробили знак \enquote{АС}, який
асоціюється з комбінатом.

% 9 - Логотип «Азовсталі».
\ii{12_08_2022.stz.news.ua.donbas24.1.azovstal_fortecja_mrpl_narys_metalurg_kombinat.pic.9.logotip}

% 10 - "Азовсталь", 1960−1963 роки.
\ii{12_08_2022.stz.news.ua.donbas24.1.azovstal_fortecja_mrpl_narys_metalurg_kombinat.pic.10}

% 8 - Володимир Лепорський.
\ii{12_08_2022.stz.news.ua.donbas24.1.azovstal_fortecja_mrpl_narys_metalurg_kombinat.pic.8.leporskij}

Окрім того, були на заводі і виробництва, які не дуже притаманні металургії.
Мова йде про цех виготовлення кришталю. Свого часу азовстальський кришталь був
дуже популярним! Келихи, чарки, вази — ще в середині 2000-х не було кращого
сувеніра з Маріуполя. А у 80-х роках кришталеві келихи були вищим показником
добробуту в домі.

% 11 - Набір азовстальського кришталя.
\ii{12_08_2022.stz.news.ua.donbas24.1.azovstal_fortecja_mrpl_narys_metalurg_kombinat.pic.11.kryshtal_nabor}

% 12 - Цитата, написана у квітні 2022 року.
\ii{12_08_2022.stz.news.ua.donbas24.1.azovstal_fortecja_mrpl_narys_metalurg_kombinat.pic.12}

Кіноісторія Маріуполя також щільно пов’язана з комбінатом \enquote{Азовсталь}. Вперше
кіношники прибули до міста у 1949 році, коли в мартенівському цеху заводу
\enquote{Азовсталь} розпочалися зйомки епізодів для кіноепопеї \enquote{Падіння Берліна}. Також
на \enquote{Азовсталі} знімали художні фільми \enquote{Вогонь}, \enquote{Біле коло}, \enquote{Найспекотніший
місяць}, \enquote{Велика розмова}. Можна побачити \enquote{Азовсталь} і у сценах фільму
\enquote{Маленька Віра}, який часто ототожнюють з Маріуполем.

Але будемо відвертими: чи велика кількість людей впізнала \enquote{Азовсталь} у
перелічених кінокартинах? Ні. Натомість фільм\par\noindent українського захисника Дмитра
Козацького з позивним Орест, який він зняв за день до виходу з заводу у травні
2022 року, облетів весь світ.

Цей фільм-прощання показує, на що перетворився один з азовстальських цехів
після обстрілів рашистів. Орест разом з іншими українськими воїнами провів на
\enquote{Азовсталі} 84 дні.

\href{https://archive.org/details/video.21_05_2022.babylon13.fortecja_mrpl_ostannij_den_azovstal}{%
Відео: Фортеця Маріуполь. Останній день на Азовсталі, Орест (Дмитро Козацький), \#BABYLON'13, 21.05.2022}%
\footnote{\url{https://archive.org/details/video.21_05_2022.babylon13.fortecja_mrpl_ostannij_den_azovstal}} %
\footnote{\url{https://www.youtube.com/watch?v=azldlmJUolc}}

\ifcmt
  ig https://i2.paste.pics/PT175.png?trs=1142e84a8812893e619f828af22a1d084584f26ffb97dd2bb11c85495ee994c5
  @wrap center
  @width 0.8
\fi

\ifcmt
  ig https://i2.paste.pics/PT17O.png?trs=1142e84a8812893e619f828af22a1d084584f26ffb97dd2bb11c85495ee994c5
  @wrap center
  @width 0.8
\fi

\ifcmt
  ig https://i2.paste.pics/PT185.png?trs=1142e84a8812893e619f828af22a1d084584f26ffb97dd2bb11c85495ee994c5
  @wrap center
  @width 0.8
\fi

