% vim: keymap=russian-jcukenwin
%%beginhead 
 
%%file 12_05_2021.fb.mironenko_petr.2.medvedchuk
%%parent 12_05_2021
 
%%url https://www.facebook.com/petro.myronenko/posts/3128643280696645
 
%%author 
%%author_id 
%%author_url 
 
%%tags 
%%title 
 
%%endhead 
\subsection{ЧИТАЮ НОВИНИ: Медведчука не затримали}
\label{sec:12_05_2021.fb.mironenko_petr.2.medvedchuk}
\Purl{https://www.facebook.com/petro.myronenko/posts/3128643280696645}

ЧИТАЮ НОВИНИ:
Медведчука не затримали;
Медведчуку підписали підозру;
Генеральний і голова СБУ коментують підозру Медведчуку;
У будинку Медведчука обшуки;
Весь пар у свисток, а паровоз не рухається вперед.
Згадав кінець лютого 2005 року.
Державний секретар України Олександр Зінченко у моїй присутності заявив: "Медведчук буде сидіти першим".
Минуло 16 років.
Мильні опери щодо санкцій і підозр, як і політична доцільність, руйнують державні інститути влади і перетворюють Конституцію у папірець, якому все менше довіри у суспільстві.
Команда Офісу Президента України ще не зрозуміла, що теорія і практика державного будівництва не є сценарієм, що можна переписати у випадку негативних наслідків.
А Генеральному прокурору краще зайнятися причинами загибелі наших військових у мирний час з 2014 року. В АТО і після АТО можна знайти безліч випадків, що підтвердять злочинні дії офіцерів і генералів щодо підготовки військ до виконання бойового завдання, наслідком чого гинули військові. 
Але це все попереду. Суспільство буде знати правду про всі події на фронтах, де генерали і офіцери займались бізнесом, а солдати в цей час гинули.
Медведчук сидіти не буде тому, що поруч прийдеться посадити ще й тих, кого добре знає Медведчук.
З повагою П.В. Мироненко
