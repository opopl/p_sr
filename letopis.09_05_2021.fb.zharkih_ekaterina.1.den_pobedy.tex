% vim: keymap=russian-jcukenwin
%%beginhead 
 
%%file 09_05_2021.fb.zharkih_ekaterina.1.den_pobedy
%%parent 09_05_2021
 
%%url https://www.facebook.com/%D0%95%D0%BA%D0%B0%D1%82%D0%B5%D1%80%D0%B8%D0%BD%D0%B0-%D0%96%D0%B0%D1%80%D0%BA%D0%B8%D1%85-112669904207013
 
%%author 
%%author_id 
%%author_url 
 
%%tags 
%%title 
 
%%endhead 

\subsection{9 Мая. Нам навязывают Комплекс жертвы вместо гордости. Зачем?}
\label{sec:09_05_2021.fb.zharkih_ekaterina.1.den_pobedy}
\Purl{https://www.facebook.com/%D0%95%D0%BA%D0%B0%D1%82%D0%B5%D1%80%D0%B8%D0%BD%D0%B0-%D0%96%D0%B0%D1%80%D0%BA%D0%B8%D1%85-112669904207013}

9 мая. День Победы.

Для многих украинцев всё ещё #День_Победы. Но из него пытаются сделать «день примирения», да и вообще принизить этот праздник, и вместо гордости внушить нам комплекс жертвы или даже вины. В Украине уже не один год пытаются это сделать. Чтобы люди стыдились своих предков, которые отдали жизни за мирное будущее своих детей.
Скажу, что думаю по этому поводу. 
Эту дату привязывают к коммунистическому прошлому, а «коммунизм это плохо» просто по умолчанию. Значит и праздник неправильный, который лучше бы вычеркнуть из памяти. Но постойте, нам что, теперь нужно предать забвению огромный период своей истории? Полный, кстати, славных страниц. Забыть и ограбить самих себя? Зачистить память и отупеть? А лучше вообще придумать новую историю, где мы бы с вами пили баварское? 
Или вот утверждения, что Победа была 8 мая, а не 9. Это же бред и наивная манипуляция! Как будто берут и забывают о разнице в часовых поясах, что в СССР тогда уже было 9 мая. Людей пытаются запутать и отвлечь от главного — от смысла этой даты. 
Также День Победы пытаются представить, как чужой, не наш праздник, а праздник «российский». Ну, а раз мы с Россией мягко говоря не дружим, то и праздновать вместе с ними нечего. Но и в Израиле этот день считается праздником, а также и в других странах – мы разве с ними тоже враги? Это победа всего мира над Злом. 
Разве ваши родственники не погибали за эти земли, или как? Тогда был Советский Союз, и его народ собрал и отвоевал сегодняшние Украинские территории, на которых проживали наши деды-прадеды. Они полили её своей кровью. Эта война коснулась почти каждой семьи.
И как после этого 9 мая может быть праздником только для России? Это великая дата для всех, кто чтит память героев, которые проявили мужество и невероятную силу противостоять самому грозному врагу, которому сдавались многие европейские государства.
 Супергерои, которые шли к победе несмотря на страх смерти, и умирали с Верой, что ценой своей жизни подарят будущее своим детям, то есть нам.
 И мне кажется, в этот день просто по-человечески хочется уважить память этих людей и вспомнить своих родственников, которые принимали участие в этих страшных событиях. Вспомнить их истории, а почти все из них – это истории про доблесть.
 И эти истории десятилетия спустя вдохновляют на борьбу со злом, даже там, где грозит смерть тебе и твоим товарищам, но вы знаете за что вы боритесь. Поэтому и хотят уничтожить этот праздник. Чтобы мы с вами чувствовали себя просто безвольными жертвами обстоятельств, которые не могут ни на что повлиять.
 Чтобы нам вечно было стыдно за себя, за свой род. Потому что таким человеком проще манипулировать, влиять на него. Понятно, что все это идёт не от тех людей, которые являются потомками победителей, ведь какой смысл унижать себя? Скорее это инициатива тех, кто проиграл, правда? И теперь ходят, зигуя, по городам Украины? Опираясь на безволие и малодушие тех, кто предал и продал победу своих предков за чувство своей безопасности, за то, чтобы быть «в тренде пропаганды». 
Друзья, дети и внуки победителей! Не дайте себя запутать, не смейте стыдиться своего рода. Мы - потомки героев, и мы тоже стать героями, сделав максимум, чтобы наши дети жили в лучшей стране, в лучшем мире, чем тот, в котором мы живем сегодня. Храните память о подвигах, передавайте её детям, в этом нет ничего постыдного. А примиряться с теми, кто выбрал сторону коричневой чумы, кто стрелял в своих же и в прыжке переобувался — нет в этом ничего доблестного. Как по мне стоит чтить людей, которые не предавали, которые совершали подвиги во имя своей семьи и Родины.
Это мое виденье 9 мая, но если вам интересно, как киевляне относятся к этой дате и будут ли вообще праздновать, смотрите специальный опрос на моем канале. Ссылка в первом комментарии.
Поздравляю вас С Днём Победы. Вечная слава и вечная память Победителям.
