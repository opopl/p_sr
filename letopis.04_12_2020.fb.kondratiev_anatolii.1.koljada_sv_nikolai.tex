% vim: keymap=russian-jcukenwin
%%beginhead 
 
%%file 04_12_2020.fb.kondratiev_anatolii.1.koljada_sv_nikolai
%%parent 04_12_2020
 
%%url https://www.facebook.com/permalink.php?story_fbid=10159245730492718&id=657602717
 
%%author 
%%author_id kondratiev_anatolii
%%author_url 
 
%%tags 
%%title 
 
%%endhead 

\subsection{Святий Миколай=Ханука, або як у наших дітей, одурюючи, забирають Коляду!}
\Purl{https://www.facebook.com/permalink.php?story_fbid=10159245730492718&id=657602717}

\ifcmt
  author_begin
   author_id kondratiev_anatolii
  author_end
\fi

\index[rus]{Традиции!Коляда}
\index[rus]{Церковь!Святые!Николай Чудотворец}

Дорогі мої українці, невже ви не бачите очевидний факт вашого одурення адептами
юдо-християнізму? 

Історичний факт існування особи так званого святого Миколая ніким не доведено,
існують тільки історичні казки, а точніше фальшування. На юдейську Хануку
(починається з 25 кеслева) юдейські батьки дарують своїм дітям подарунки
(Ханука гелд) тільки за те, що вони себе гарно поводять. Тобто, подарунки
даються не за знання чи працю, а за покірну поведінку. Натомість українським
дітям на Коляду (починалася з 21 грудня) дітям дарують подарунки за знання
пісень, колядок, вміння співати та танцювати. А щоб відбити цю нашу традицію,
котра спонукала маленьких українців мати певні уміння, а не покірність, було
видумано свято, назву якого пов"язали з Днем, коли закінчувалося Коло Кумирів -
Ни Кола. Адже річне Коло Кумирів в українців закінчувалося 18 грудня святом
Лади, а з 19 грудня вже не було Кола, тобто цей день називали Ни Кола. Таким
чином нас одурили, а ми, як ті вівці, прийняли це одурення. А вже 1 січня, на
календарний Новий рік, українці святкують День Господаря Пана і по Україні
розливається пісня, Добрий вечір, Добрий, Пане Господарю...

Сьогодні адепти юдо-християнізму зі своєю чорною метою - позбавлення українських
дітей пам'яти про Коляду, відкривають будинки мітичного святого Миколая. А чи
знайдеться у нас хоча б одна особа, котра відкриє у столиці чи іншому місті
будинок Коляди???
