% vim: keymap=russian-jcukenwin
%%beginhead 
 
%%file 14_09_2021.fb.jegorushkina_katerina.1.mova_osobyste
%%parent 14_09_2021
 
%%url https://www.facebook.com/kateryna.yegorushkina/posts/5136748906351925
 
%%author_id jegorushkina_katerina
%%date 
 
%%tags cennosti,chelovek,identichnost',jazyk,mova,slovo,ukraina
%%title Про мову. Дуже особисте
 
%%endhead 
 
\subsection{Про мову. Дуже особисте}
\label{sec:14_09_2021.fb.jegorushkina_katerina.1.mova_osobyste}
 
\Purl{https://www.facebook.com/kateryna.yegorushkina/posts/5136748906351925}
\ifcmt
 author_begin
   author_id jegorushkina_katerina
 author_end
\fi

Про мову. Дуже особисте.

Я виросла у Броварах, у російськомовній родині, навчалася у російській школі та
російському класі. Моє ніжне ім’я було — Катюша.

Уже багато років поспіль мене питають, чи я не зі Львова. Хоча я не маю ані
львівського акценту, ані характерних слів. Людям досі буває незвично чути добру
українську на наших теренах.

Колись я прийняла рішення. І крок за кроком йшла до нього.

Що ж на це вплинуло? Ключові спогади.

Пам‘ятаю свою вчительку української, Галину Василівну. Вона називала мене
незвично — Катрусею. Спілкувалася українською не лише під час уроків, а й на
перервах. З усіма, незалежно від мови, якою до неї зверталися. 

Для мене це був урок гідності.

Пам‘ятаю молоду і красиву вчительку географії родом із Західної України. Класна
керівничка попросила батьків дозволити їй вести уроки українською. Батьки
дозволили, і її  мова мене зачарувала.

Для мене це був урок цілісності.

Пам‘ятаю дивовижну пісню, що линула пізно ввечері над хутором Григорівка, де я
часто літувала в дитинстві.

Я вийшла надвір подивитися на зорі, і та жива пісня, співана тут і зараз,
живими голосами, запустила мені сироти попід шкіру.

Вона стала уроком пам‘яті і краси.

Я розуміла, що запас моєї сили волі обмежений, тож спрямувала його на створення
для себе україномовного середовища. Долучилася до місцевого
літературно-мистецького об‘єднання «Криниця», мала чудових учителів на цьому
шляху. 

Вирішила вступати до Могилянки. 

Я ставала все більш україномовною, але не з усіма. Мені було капець як важко і
незвично. Усі близькі друзі на той момент були російськомовними.

Українська стала для мене цінністю, яку я дуже поважала в інших, але вона не
була моєю.

Моєю вона стала, коли її спробували відібрати. За часів регіоналів влада почала
рухатися у напрямку русифікації. Студентами ми вийшли на мітинг, і це
перемкнуло щось у мені назавжди, лишивши в постійному лексиконі лише одне слово
російською, яке прохоплюється в мене за особливих обставин (цензурне, до речі). 

Я поставила собі питання: що для мене важливіше — цінність чи звичка? Цінності
утворюють ядро моєї особистості, а звички — хоча й сильні, проте більш
поверхневі. Я не можу дозволити собі поступитися цінністю заради звички. Це все
одно, що мати з-поміж цінностей любов і жити з нелюбим чоловіком. Мати
демократичні цінності і працювати на авторитарного керівника, заспокоюючи себе
«плюшками». Таке нехтування власними цінностями може деформувати особистість, і
шлях повернення до себе буде непростим.

Українська стала тією ниткою, яка з’єднала мене з моїми розкуркуленими предками
з Херсонщини, з-поміж яких не всі вижили на шляху до Сибіру.

Вона з’єднала мене з пращурами з Кубані, з-поміж яких не всі вижили після
Голодомору, а їхнім нащадкам до прізвища Мудрик додали -ов.

Вона з’єднала мене з усіма пращурами, які принаймні раз у житті співали
українських пісень на своїх хуторах.

Як казав мій викладач Володимир Пилипович Моренець, українська — це мова серця,
російська — мова горла. Я перевірила, промовляючи різні слова. І це виявилося
правдою.

Я прислухалася  до власного голосу і він став глибоким, грудним. Я з легкістю
перенесла ці відчуття у свою поезію (попри те, що багато років писала
російською, мала публікації і перші відзнаки). Я зазвучала інакше і невдовзі
була приємно вражена цим голосом.

Ні, я не володію мовою. Я не можу нею володіти, як і не можу підкорити гірську
вершину. Вони більші за мене. Я можу лише увійти в потік мови, насолоджуючись
словами й звуками, досліджуючи їх і часом створюючи свої.

Я не можу нікого заохотити говорити українською. Це лише мій досвід і я поважаю
шлях кожної людини. Та, можливо, стану для когось ключовим спогадом, уроком
гідності, цілісності, пам‘яті чи краси.

І наостанок дрібка моїх улюблених слів.

Повітря, осоння, гарбуз, чорногуз, лелека і лилик.

Світанок, розмай, деревій, немовля.

Розкішний, смарагдовий, бурштиновий і суголосний.

Розпашілий, розпружений і неозорий.

Тішитися, пестити, кохати.

Тиша, чарунок.

Мрія.

А які слова любите ви?

(с) К. Єгорушкіна, 14 вересня 2021 року
