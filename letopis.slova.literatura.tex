% vim: keymap=russian-jcukenwin
%%beginhead 
 
%%file slova.literatura
%%parent slova
 
%%url 
 
%%author 
%%author_id 
%%author_url 
 
%%tags 
%%title 
 
%%endhead 
\chapter{Литература}

%%%cit
%%%cit_head
%%%cit_pic
%%%cit_text
Еще раз подчеркнем, феерией называлась не \emph{литературная} основа, а само шоу,
действо. Пьесы, которые ставились таким образом, драматурги не обозначали как
феерию. Потому что играть их можно было по-разному.  Пока не пришел Метерлинк.
В 1905 году бельгийский символист Морис Метерлинк пишет свою самую знаменитую
пьесу. Она называется \enquote{Синяя птица}. И впервые он сразу ставит на обложке ее
жанр - пьеса-феерия.  Так термин перекочевал из театральной среды в
\emph{литературную}. Теперь читатель, даже не видя самой постановки, должен был
представлять то действие, которое разворачивается в пьесе, именно как красочную
и волшебную феерию
%%%cit_comment
%%%cit_title
\citTitle{В двадцатом веке было написано только три феерии. Одна бельгийская, две наши}, 
ЛИТИНТЕРЕС, zen.yandex.ru, 03.06.2021
%%%endcit

