% vim: keymap=russian-jcukenwin
%%beginhead 
 
%%file 05_04_2022.fb.muzej_dytachoj_tvorchosti.ukr.1.risunok_karta_ukrainy
%%parent 05_04_2022
 
%%url https://www.facebook.com/kidscreativitymuseum/posts/517878356374267
 
%%author_id muzej_dytachoj_tvorchosti.ukr
%%date 
 
%%tags 
%%title Малюнок - карта України
 
%%endhead 
 
\subsection{Малюнок - карта України}
\label{sec:05_04_2022.fb.muzej_dytachoj_tvorchosti.ukr.1.risunok_karta_ukrainy}
 
\Purl{https://www.facebook.com/kidscreativitymuseum/posts/517878356374267}
\ifcmt
 author_begin
   author_id muzej_dytachoj_tvorchosti.ukr
 author_end
\fi

Мене звати Аня Загородня. Родом із Києва. Одного дня 24.02.2022 мене розбудили
батьки, сказали, що почалась війна. Росія напала на нас. Коли  мій Київ почали
бомбити, нам зателефонували знайомі з Шаргороду й запропонували пожити в їх
квартирі, за що їм дуже вдячна. Ми дуже швидко зібрали речі, які встигли та
виїхали до Шаргороду. Безмежно пишаюся нашими українцями, що у такий скрутний
час, кожен ладен допомогти чим може.

\ifcmt
  ig https://scontent-mxp1-1.xx.fbcdn.net/v/t39.30808-6/277757021_517878259707610_801032978547524663_n.jpg?_nc_cat=108&ccb=1-5&_nc_sid=0debeb&_nc_ohc=TcD-0YhZ6AEAX_IVn9L&_nc_ht=scontent-mxp1-1.xx&oh=00_AT-hNrVVdyHGtzYKPy35dnHF8LXIuXIJh8L3ycgub4Wqog&oe=62653E83
  @wrap center
  @width 0.8
\fi

Щодо ідеї з картою України, я довго думала, що можу зробити цікаве та творче
для підтримки нашої армії. І в якийсь із сонячних днів, подивившись на мого
8-річного братика, котрий також хоче, але не знає як допомогти нашим
захисникам, я вирішила, що можна залучити діток зробити карту України разом.
Мені залишалося скупити весь папір а2 та скріпити його в цілий ватман) фарби,
фломастери, синьо-жовті стрічки... І ось майже все готово. З моєю сім’єю та
рідними намалювали карту, зшили синьо-жовтий прапор, а з маленьких стрічок
зробили бантики для кожного учасника цього заходу. І ми це зробили! Намалювали!
Кожна дитина залишила свою долоньку на плакаті і написала побажання для нашої
рідної України. Я вірю в те, що думки  матеріалізуються, а побажання тим
більше. Долучаємо дітей до творення нашого майбутнього!

А цей  плакат поїхав  до наших бійців на фронт. Наш патріотизм та згуртованість
одна зі зброй для ворога. А кожен з нас робить все можливе для об'єднання
нашого національного  духу та створення світлого та мирного майбутнього! Дякую
всім хто долучився та приклав свою долоньку задля нашої перемоги!"
@igg{fbicon.hands.pray} @igg{fbicon.heart.red} @igg{fbicon.flame}
