%%beginhead 
 
%%file 19_06_2023.fb.knuba.1.knuba_help_mariupol_univ
%%parent 19_06_2023
 
%%url https://www.facebook.com/knubaofficial/posts/pfbid02u3YXhWs618oxBDggpF4cYFyqpjuQP9vtjRPEK4wf9XBeqxAwppN4fRoiF534Ygal
 
%%author_id knuba
%%date 19_06_2023
 
%%tags 
%%title КНУБА ДОПОМАГАЄ СПІЛЬНОТІ МАРІУПОЛЬСЬКОГО УНІВЕРСИТЕТУ
 
%%endhead 

\subsection{КНУБА ДОПОМАГАЄ СПІЛЬНОТІ МАРІУПОЛЬСЬКОГО УНІВЕРСИТЕТУ}
\label{sec:19_06_2023.fb.knuba.1.knuba_help_mariupol_univ}

\Purl{https://www.facebook.com/knubaofficial/posts/pfbid02u3YXhWs618oxBDggpF4cYFyqpjuQP9vtjRPEK4wf9XBeqxAwppN4fRoiF534Ygal}
\ifcmt
 author_begin
   author_id knuba
 author_end
\fi

КНУБА ДОПОМАГАЄ СПІЛЬНОТІ МАРІУПОЛЬСЬКОГО УНІВЕРСИТЕТУ

Маріупольський державний університет - офіційна сторінка вже понад рік працює в
Києві. Цьому передували окупація рідного міста та втрата матеріально-технічної
бази: ворожими ракетами були пошкоджені усі навчальні корпуси.

У важких умовах на допомогу освітянам з Маріуполя прийшли колеги з Київського
національного університету будівництва і архітектури. Ректор КНУБА Петро
Куліков гостинно прийняв академічну спільноту.

Співробітники, викладачі та студенти Маріупольського університету отримали у
розпорядження  четвертий  навчальний  корпус  київського ЗВО. На його базі вони
змогли відновити роботу та організувати гуманітарний штаб для допомоги
маріупольцям. Нині у будівлі МДУ тривають активні ремонтні роботи, з якими
університету допомагає відновлюватися Європейський Союз.

Окрім корпусу, представники Маріупольського університету отримали змогу
проживати у гуртожитку КНУБА, що розташований поряд із місцем роботи та
навчання. Співробітники університету можуть вільно відвідувати наукову
бібліотеку та спортивний комплекс КНУБА, а також  використовувати актову залу
для власних потреб.

\enquote{Я щиро вдячний Петру Кулікову – доброму другу Маріупольського університету.
У важкий час він виказав підтримку колективу незламного вишу своєю готовністю
надати усе необхідне для нашої злагодженої роботи. Щодня ми докладаємо зусиль
для втілення у життя амбітних планів і надійна підтримка від однодумців
допомагає це робити ефективніше. Разом із КНУБА ми реалізуємо багато спільних
проєктів на благо України},

- ректор МДУ Микола Трофименко.

Сьогодні університети роблять кроки до кращого майбутнього.  У межах програми
імені Фулбрайта МДУ і КНУБА розробили магістерську освітньо-професійну програму
\enquote{Менеджмент міжнародних інфраструктурних проєктів}, на яку абітурієнти зможуть
вступити вже цього року. Вона сприятиме відбудові нескореної країни після
Перемоги.🇺🇦

Інформаційний відділ

\url{https://www.knuba.edu.ua/knuba-dopomagaye-spilnoti-mariupolskogo-universytetu}
