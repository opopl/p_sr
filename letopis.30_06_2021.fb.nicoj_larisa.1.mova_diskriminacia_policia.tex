% vim: keymap=russian-jcukenwin
%%beginhead 
 
%%file 30_06_2021.fb.nicoj_larisa.1.mova_diskriminacia_policia
%%parent 30_06_2021
 
%%url https://www.facebook.com/nitsoi.larysa/posts/942386886327684
 
%%author Ницой, Лариса
%%author_id nicoj_larisa
%%author_url 
 
%%tags absurd,diskriminacia,kurjez,marazm,mova,policia,ukraina,ukrainizacia
%%title Мовна дискримінація і національна поліція, яка підтримує порушників закону
 
%%endhead 
 
\subsection{Мовна дискримінація і національна поліція, яка підтримує порушників закону}
\label{sec:30_06_2021.fb.nicoj_larisa.1.mova_diskriminacia_policia}
\Purl{https://www.facebook.com/nitsoi.larysa/posts/942386886327684}
\ifcmt
 author_begin
   author_id nicoj_larisa
 author_end
\fi

Мовна дискримінація і національна поліція, яка підтримує порушників закону.
Десь місяць-два тому з'явилася біля метро ця точка. Тримають її люди неукраїнської зовнішності. Торгує нечесана дівка і мужики з чорними бородами. Наші всі торгують мовчки, ці ж зазивають на всю вулицю:
- Падхаді, свєжія клубніка, свєжая чєрєшня! 
Проходиш повз, як гаркне, як гаркне: 
- Жєнщіна, бєрьом клубніку!
- Українською, будь ласка. Будете говорити українською, тоді й куплю. 
Сьогодні те саме. Іду, нікого не чіпаю.
- Женщіна, берьом чєрєшню, сладкая, свєжая.
- Господи, коли ж ви вже заговорите українською! - і йду собі далі.
- А ета апять ета бальная. Глєваха па тєбє плачєт! - на всю вулицю.
- Це ще що за хамство! - розвертаюся. - Що ви собі дозволяєте? 
- Я жіву в пригранічнам гарадкє (І це не заважає їй знати про Глеваху), я так всєгда разгаварівала. Я в свабоднай странє, как хачю так і гаварю!
- У нашій вільній країні є закони. І за цими законами ви зобов'язані з покупцями розмовляти українською.
- Срать я хатєла на всє ваші укрАінскіє закони!
Ну тут я вже не стерпіла, телефоную в поліцію.
- Давай, давай, звані! - верещить на всю вулицю дівка. - У тєбя паспарт єсть? Я на тєбя тєлєгу накатаю, как ти мєня бидлам абзивала. Ти мєня уніжала. Здєсь всє свідєтєлі! Сумашедшая! Бальніца па тєбє плачєт.
Вереск тієї дівахи на всю вулицю стояв такий, що поліція, яка взяла трубку, запитала мене, чи мене б'ють.
 - Ні, ще не б'ють, - сказала я, - але до цього недалеко. Думаю, що скоро битимуть. Тому і викликаю поліцію... 
Пройшла година. Стою. Жду. Набрала поліцію, чи їде наряд.
- Їдуть, їдуть, - чую в трубці. 
Пройшло півтори години. Я змерзла. Продавці повдягали куртки. Я ж відчуваю, як від холоду починає зводити спину, яку мені не можна студити. Згадую слова лікаря, що моя спина - це психосоматика (все від нервів). Зітхаю. Діваха вже не кричить, лиш жаліється перехожим, що он, стоїть самашєдшая, каторая прінуждаєт єйо гаваріть на українскам. Перехожі озираються на мене і співчутливо радять дівасі не звертати на самашедшую вніманія. Я згадую, що треба було зняти ті верески на відео. В стресі про все забуваєш. Витягаю телефон, знімаю хоч щось.
Час від часу до дівахи навідуються бородані, контролювати торгівлю. Бородані вислуховують її смішну версію про самашедшу і разом з нею ржуть. 
Стемніло. Я стояла, тремтіла і чекала поліцію, яка приїде і мене захистить. Не приїхали...
Після отаких випадків хочеться піти і повіситися. Але, я віруюча. Я ніколи не накладу на себе руки. Усі рідні і друзі попереджені, якщо що, і це виглядатиме як самогубство, то це не я. Тому стою, терплю. Жду. Жду безрезультатно.  
До яких пір буде продовжуватися це приниження українців в Україні? До яких пір об нас витиратимуть ноги? До яких пір Потураєви, Бужанські і Ткаченки подаватимуть законопроекти про відтермінування штрафів? 
До яких пір мене кончатимуть на моїй землі? Скільки ще терпіти цих стресів і психологічного знищення? Де і в кого шукати прихистку?
Пройшло дві з половиною години. Поліція не приїхала. Беру пакуночки, чвалаю додому. Таке враження, що увесь світ мені улююкає в спину.
Звертаюся до всіх через фейсбук, прошу небайдужих адвокатів обізватися до мене і допомогти мені подати на поліцію позов в суд за неналежне виконання цією службою своїх обов'язків. 
Як гадаєте, допоможуть?
