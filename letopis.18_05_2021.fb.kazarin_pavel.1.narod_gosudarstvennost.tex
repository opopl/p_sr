% vim: keymap=russian-jcukenwin
%%beginhead 
 
%%file 18_05_2021.fb.kazarin_pavel.1.narod_gosudarstvennost
%%parent 18_05_2021
 
%%url https://www.facebook.com/kazarin.pavel/posts/3876431092405187
 
%%author 
%%author_id 
%%author_url 
 
%%tags 
%%title 
 
%%endhead 

\subsection{Легко быть большим народом с собственной государственностью}
\label{sec:18_05_2021.fb.kazarin_pavel.1.narod_gosudarstvennost}
\Purl{https://www.facebook.com/kazarin.pavel/posts/3876431092405187}

Легко быть большим народом с собственной государственностью. 

Когда за плечами столица, история, собственный коллективный миф. И совсем иное
– быть меньшинством, которое вынуждено жить по чужим правилам. Крымские татары
знают об этом как никто. 

Их меньше трехсот тысяч – примерно один спальный район Киева. Порядка 15% от
населения полуострова. В Крыму просоветское большинство любило рассказывать
небылицы о крымских татарах. В ходу были байки о лагерях подготовки боевиков в
горах. О схронах с оружием. Все ждали погромов и межэтнических столкновений.

Впрочем, все это было неправдой. Иначе после аннексии полуострова ФСБ уже бы
показала нам и «лагеря» и «схроны». А довоенные столкновения на полуострове
случались не столько по национальному признаку (крымские татары против славян),
сколько между коренным народом и властью. 

Между тем, предпосылок для конфликтов было хоть отбавляй. Крымские татары
вернулись из депортации в конце 80-х - начале 90-х. В тот самый момент, когда
советские сбережения обесценились. Когда государство занималось чем угодно, но
только не обустройством жизни. Реституции не случилось, землю под строительство
выделяли неохотно, работы было мало. Но условное ружье, висевшее на стене с
1944 года, в нарушение всех театральных законов так и не выстрелило. Вместо
этого крымские татары воссоздавали самих себя – заново и в новых условиях. 

Они были отчетливо антисоветскими – на фоне просоветского крымского
большинства. Советские лозунги и знамена вызывали у них естественное
отторжение. А как еще можно реагировать на флаги режима, который оболгал твой
народ и отправил в изгнание на пятьдесят лет? 

Эта рельефность выделяла крымских татар из остального населения полуострова. Не
случайно коренной народ полуострова всякий раз поддерживал на выборах те
украинские политические силы, которые не спекулировали на ностальгии по СССР. 

При этом они оставались вещью в себе. Потому что украинец из Крыма мог
переехать в Херсон, русские – в Краснодарский край, а у крымских татар не было
другой родины вне Крыма. Именно поэтому после аннексии они оказались в
ситуации, когда неприятие новой реальности сочетается с невозможностью сбежать
от нее. 

До войны крымских татар часто упрекали в нежелании интегрироваться в крымское
пространство. В том, что они цепляются за собственную обособленность, в том,
что у них есть свой этнический парламент – курултай, и правительство – меджлис.
Но, в определенном смысле, все это было закономерно. Потому что помимо
идеологических различий крымским татарам попросту не было куда интегрироваться.
Вокруг царил олигархический дарвинизм.

Майдан родил для Украины шанс на новые правила – но крымские татары не успели
им воспользоваться. Российское вторжение отобрало у них революцию, подарив
взамен контрреволюцию. И теперь они оказались в сложных условиях.

Для Москвы крымские татары стали костью в горле – сразу по двум причинам. С
одной стороны, они портят миф о тотальной поддержке триколоров в Крыму. С
другой - они нарушают российскую традицию, когда власть является единственным
субъектом в окружении разрозненных обывателей. 

Крымских татар отличает то, что у них есть опыт выживания в условиях
враждебного или равнодушного государства. У них есть горизонтальная
сплоченность и опыт взаимовыручки. И это вызывает у российских властей
особенное раздражение. 

Потому что современная Россия – это образец того, как выглядит этатизм. Власть
стремится лишить субъектности любые неофициальные центры влияния.  Пытается
добиться того, чтобы любые ростки жизни рождались лишь в государственнном
инкубаторе. Современная Россия – это образец того, как средство становится
целью. Как государство перестает существовать во имя собственных граждан, а
начинает жить само для себя и во имя себя. Запрещает любой активизм и
самостоятельность. 

Кремль воспринимает собственных граждан лишь как винтики в системе. Отсутствие
индивидуальности становится главным критерием качества. Государство становится
сверхценностью, а значение личности определяется ее способностью находиться на
службе у вертикали. 

Россия объявила вне закона свидетелей Иеговы – за их отказ служить в армии.
Ограничивает деятельность публичных лекториев и запрещает митинги. И точно так
же она пытается лишить субъектности крымских татар. Пытается их раздробить и
атомизировать. 

Потому что субъектность этого народа, равно как и его опыт жизни вне
государства, служит для вертикали угрозой. Кремль не умеет делиться
полномочиями. Он ценит умение петь хором и радоваться по расписанию. Отсутствие
привычки думать считает достоинством. 

Тем, кто готов присягнуть новым флагам, в Крыму демонстративно протянут руку и
откроют чиновничьи кабинеты. Тем, кто откажется – грозят обыски и уголовные
дела. 

В Москве помнят о том, что крымские татары сумели пережить Советский Союз. Тот
самый, что пытался вычеркнуть коренной народ Крыма из истории. И потому
российские власти пытаются лишить крымских татар той внутригрупповой
сплоченности, которая в свое время помогла им это сделать.
