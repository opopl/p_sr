% vim: keymap=russian-jcukenwin
%%beginhead 
 
%%file 02_11_2020.fb.roman_barashev.1.dr_mihail_pljackovskii_85
%%parent 02_11_2020
 
%%url https://www.facebook.com/roman.barashev/posts/2152431934889274
%%author 
%%tags 
%%title 
 
%%endhead 

\subsection{Михаил Пляцковский - 85 лет - день рождения!}
\label{sec:02_11_2020.fb.roman_barashev.1.dr_mihail_pljackovskii_85}
\Purl{https://www.facebook.com/roman.barashev/posts/2152431934889274}
\Pauthor{Барашев, Роман}
\index[names.rus]{Пляцковский!Михаил}

\obeycr
"От улыбки хмурый день светлей,
От улыбки в небе радуга проснётся,
Поделись улыбкою своей,
И она к тебе не раз ещё вернётся!"
\restorecr

\ifcmt
pic https://scontent.fiev22-2.fna.fbcdn.net/v/t1.0-9/123269344_2152429991556135_7350817965836907358_n.jpg?_nc_cat=102&ccb=2&_nc_sid=730e14&_nc_ohc=-GgRX4hBtygAX-Cb7tr&_nc_ht=scontent.fiev22-2.fna&oh=de75ffb53a7584bfa786bf7122318ca5&oe=5FD3EC5C

pic https://scontent.fiev22-1.fna.fbcdn.net/v/t1.0-9/123164620_2152433118222489_8240109477832635650_n.jpg?_nc_cat=107&ccb=2&_nc_sid=730e14&_nc_ohc=g8Cf_YEXp-wAX-Pkmmf&_nc_ht=scontent.fiev22-1.fna&oh=c943e0794aeb0f7d6d377430712ab397&oe=5FD1D297

pic https://scontent.fiev22-2.fna.fbcdn.net/v/t1.0-9/123505089_2152430078222793_6084674917149155119_o.jpg?_nc_cat=104&ccb=2&_nc_sid=730e14&_nc_ohc=COJOpKILHLYAX_L2vd5&_nc_ht=scontent.fiev22-2.fna&oh=df7e911be7402ff80af1169d83cdda62&oe=5FD25406
\fi

Сегодня -- 85 лет со дня рождения советского поэта-песенника  Михаила
ПЛЯЦКОВСКОГО. Родился в Енакиево Донецкой области. После окончания средней
школы работал в многотиражке «За металл» Енакиевского металлургического завода.
И вёл литературную страничку в газете «Енакиевский рабочий».

Конечно же, мы вспомним этого замечательного поэта на следующем занятии с
детьми в студии "ЖУРНАЛИСТИКА ДЛЯ ШКОЛЬНИКОВ". И его крылатые строки, летящие
через времена-поколения.

"Мы поедем, мы помчимся на оленях утром ранним", "И так прекрасно возвращаться
под крышу дома своего", "Моё детство -- красный конь", "Друг в беде не бросит,
лишнего не спросит", "Не повторяется такое никогда"...

Как только выдастся свободная минутка, спою и сам чего-нибудь Пляцковского на
музыку... Заказывайте :) А пока предлагаю послушать, как -- ах как! -- исполнил
песню Михаила Спартаковича из безоблачного далёка-далека задушевный человек
Олег Ларюшин, который мне о поэте и напомнил. 

А также рекомендую обратить внимание на авторский проект Олега Владимировича
«История одной песни» (ссылка -- в первом комментарии).
