% vim: keymap=russian-jcukenwin
%%beginhead 
 
%%file 01_05_2022.fb.dykyj_evgen.1.inaksha_vijna
%%parent 01_05_2022
 
%%url https://www.facebook.com/evgen.dykyj/posts/10160040371118808
 
%%author_id dykyj_evgen
%%date 
 
%%tags 
%%title ІНАКША ВІЙНА
 
%%endhead 
 
\subsection{ІНАКША ВІЙНА}
\label{sec:01_05_2022.fb.dykyj_evgen.1.inaksha_vijna}
 
\Purl{https://www.facebook.com/evgen.dykyj/posts/10160040371118808}
\ifcmt
 author_begin
   author_id dykyj_evgen
 author_end
\fi

ІНАКША ВІЙНА

(от наскільки довго не мав часу на писанину, рівно такий довгий лонгрідіще і
тримайте  @igg{fbicon.smile}  )

Ну що, мої котики та зайчики, ось і спливли вже два місяці й тиждень від
початку великого вторгнення. І хоч останні 67 діб злилися для нас у один
довгий-предовгий день, слід потроху звикати до нової шкали обчислення часу
війни – не у днях, навіть не у тижнях, а у місяцях.

Перший етап війни закінчився місяць тому, коли останні орки виповзли з наших
північних областей. Для РФ закінчився невдалий бліцкриг, для нас так само
завершилася «епоха героїчної партизанщини».

Що далі? Для обох сторін – тривала та безкомпромісна війна на виснаження.
Виснаження ресурсів, зброї, техніки, та в першу чергу – фізичних та моральних
сил армії та суспільства.

Щоб там не мудрували окремі «голуби миру» та «генії переговорного процесу»,
сама суть цієї війни така, що компроміс неможливий. В підсумку цієї війни на
карті світу залишиться або незалежна демократична Україна, або кремлівський
режим. Нашим ворогам цілком очевидна ціна питання, тож вони наразі кинуть всі
наявні ресурси для того, щоб в ході другого етапу війни отримати те, що ми не
дали їм на першому. Вони намагатимуться наступати всюди, де лише зможуть –
тепер вже не «ліхім нальотом», а так би мовити класично.

Насправді поки ми обговорювали в першу чергу битву за столицю, на Донбасі всі
ці два місяці тривала зовсім інша війна аніж на Поліссі та Слобожанщині. Всі ці
понаддва місяці там щодня наші позиції відпрацьовувала ворожа авіація, потім
вступала арта, а поки арта змушувала захисників не висовуватись з окопів, до
наших позицій підходила їхня піхота, тож одразу після останнього розриву
снаряду починався близький контактний бій. 

Це – фронт у стилі Другої світової, і саме на неї, а не на Чечню чи Афган, нам
слід орієнтуватись, щоб зрозуміти характер нашої війни у наступні декілька
місяців.

В цій війні більше не буде стрімких переміщень військ, калейдоскопу окупованих
– звільнених – знову окупованих – ще раз звільнених міст, чудернацької мозаїки
ЗСУ, ТрО, різноманітної «спецури» та ще більш різноманітних партизанських
загонів. Війна стає регулярною, «правильною», основна роль у війні остаточно
переходить до ЗСУ, всі решта лише «на підхваті» інколи за потребою.

А країна чітко поділяється на окуповані терени, фронт, прифронтову смугу та
тил. Тил годуватиме фронт та даватиме місце для відпочинку тим хто його тримає,
а фронт спершу триматиме удар ворожого наступу, а потім поволі переходитиме у
наступ за звільнення окупованих територій.

Різницю між фронтом та тилом трохи змазуватимуть постійні ракетні удари по всій
території країни (не варто втішатись ілюзіями – запас ракет у орків ще чималий,
хоч і зовсім не нескінченний). Але ж камон, мої котики та зайчики, час від часу
реагувати на повітряні тривоги (чи як автор цього опусу давно вже не реагувати
 @igg{fbicon.smile}  ) – далеко не те саме що жити та працювати в зоні досяжності арти, і тим
більше не те що приймати бій. Тож тил таки буде тилом, а звуки сирен не дадуть
зовсім вже відірватись від реальності та вирішити що війна стосується лише тих,
хто наразі на «передку» (як то було останні вісім років, між АТО та великим
вторгненням).

Ті, хто зараз на передку, не потребують моїх порад, тож текст для ближнього та
дальнього тилів, а також для тих під окупацією, хто ще має можливість читати
наші ресурси.

Почну із останніх. Деокупація неминуча, але вона буде нешвидкою. Ворог мав два
місяці для підготовки оборони захоплених впродовж першого місяця теренів, і він
ці два місяці не згаяв. На жаль, тепер нам доведеться відбивати кожне окуповане
місто у важких наступальних боях, це забере і час, і чимало життів. 

Тож якщо хтось не виїздить з-під окупації у сподіванні що скоро Південь
повторить історію визволеної півночі – так робити не варто. Якщо є хоч маленька
шпарина крізь яку можна виїхати на наш бік фронту – варто робити це чим швидше,
бо ці шпарини з кожним днем затикають все щільніше, і скоро їх може зовсім не
стати.

Решта тексту для нас, тиловиків. Нова часова шкала вимагає переосмислення
кожним свого місця та ролі у нашій спільній війні, і перебудови свого життя під
реалії цього її етапу.

В першу чергу нам слід прийняти реалії іншої шкали часу. Режим «одного дуже
довгого дня» та напруги всіх сил «на ривок» був неминучим на початку війни, але
зараз він вже призводить до надмірної втоми, що на довгій дистанції неприйнятно
і закінчується виснаженням. Як не дивно, слід почати розписувати свій час,
виділяти окремо працю та відпочинок, налагоджувати хай і тимчасовий, але все ж
облаштований побут та навіть випрацьовувати нові щоденні звички.

Цю війну слід не пережити чи тим більше перечекати, а прожити. З катастрофи,
екстремального стану або пригоди вона має перетворитись у нашому сприйнятті на
один з періодів життя кожного з нас, як були ними школа, університет, строкова
служба тощо. Кожний з нас має спокійно, виважено та водночас вперто прожити
кожний свою власну війну, і тільки так у нас всіх вистачить сил дожити до
Перемоги.

А перш ніж налагоджувати своє нове довгограюче військове життя, варто сісти та
спокійно подумати про те, де кожний з нас може бути найбільш корисним, і в кого
на що вистачить сил та ресурсів.

Столиця та загалом Північ на цьому етапі війни явно не потребують такої
кількості тероборони, те  саме стосується центральних областей країни, не
кажучи вже про Захід (який і раніше її стільки не потребував).

Тож кожний із нас, хто в перші дні війни знайшов своє місце у лавах ТрО, має
добре подумати, чи варто далі лишатись нібито і бійцем, але за сотні км від
фронту на нікому не потрібних блок-постах, чи ж варто визначитись із одним із
трьох шляхів – до лав ЗСУ на фронт (або ж до тих підрозділів ТрО, які будуть
реорганізовані у де-факто звичайні піхотні частини та також відправлені на Схід
та Південь), до скорочених в рази частин ТрО, які фактично стануть охороною
промислових та адміністративних об’єктів, чи ж назад до мирного життя та
професії.

Останнє рішення жодним чином не менш гідне за перші два, особливо якщо мова про
людей які створюють додану вартість, робочі місця, сплачують податки – фронт
потребуватиме безліч ресурсів, і чим активніше працюватимуть всі можливі
бізнеси, тим більше цих ресурсів буде доступними. Мужики, харош відгодовуватись
на волонтерських бутербродах на блоках, відпочили та й досить!  @igg{fbicon.smile}  Прикольна
пригода закінчилась, пора таки знову пахати та заробляти!  @igg{fbicon.smile} 

Те саме із волонтерством – кожному час оцінити свої сили та ресурси. Хто має
можливість і далі утримувати себе самого, плюс ще й виділяти час на фултаймову
волонтерську допомогу фронту або слабкішим за себе цивільним – прекрасно, на
жаль і фронт, і безліч наших співгромадян у тилу ще дуже довго потребуватимуть
допомоги. Але можливо частині з нас варто в першу чергу обміркувати можливість
повернення до довоєнної праці, або ж пошуку нової, і лише якусь частку часу
присвячувати волонтерці, чітко розподіливши час на роботу, волонтерство та на
відпочинок (нагадую – без останнього нас надовго не вистачить, а ми заходимо у
довге протистояння).

Тим, хто у перші дні записався до військкоматів та досі так і не отримав
виклику, також варто замислитись над можливими шляхами. Для початку слід
визначитись, наскільки сильним є бажання саме воювати. Якщо цілком вистачить
почуття виконаного обов’язку від того що став на облік, але якщо так і не
призвуть то спати погано не станеш – варто просто ще раз заглянути до
військкомату та перевірити, чи ж дійсно ти наявний у списках, чи у бардаку
перших днів війни всі ті наспіх зібрані списки давно вже загублені, і в
останньому випадку поновити «місце в черзі». Оскільки ми заходимо в довгу
війну, швидше за все черга рано чи пізно прийде, і загроза «пропустити все
цікаве» насправді не така вже і велика. А тим часом сконцентруватись на праці,
та за можливості на волонтерстві у вільний від праці час.

Якщо ж сумління все ж вимагає безпосередньої участі у бойових діях (свідомо не
пишу «у війні», бо у війні беруть участь не лише бійці на передовій, але і весь
активний тил, задіяний на забезпеченні фронту, прийомі евакуйованих тощо) –
здаю «лайфхак»: кроткий шлях на фронт починається не з військкомату, а
безпосередньо з бойового підрозділу. Командирам частин потрібні мотивовані
бійці, тож слід знайти якусь із частин, де ще не всю штатку встигли заповнити,
а якраз іде комплектація, пройти співбесіду з командиром – і вже за його
направленням бігти до військкомату, причому не за місцем прописки, а за місцем
комплектації частини.

І в момент вибору шляху, описаного абзацем вище, так само слід мати на увазі –
це не на кілька тижнів героїчного чину, а на місяці (якщо не довше) військового
життя без можливості перервати це в будь-який час за бажанням.

Ну і зрештою всім нам варто звикати до того, що новини стануть набагато менш
цікавими, і дуже часто зводитимуться до «на фронті без змін». 

Це «без змін» означатиме зовсім не те, що ніхто нічого не робить. Навпаки, день
«без змін» - це день, коли десятки тисяч бійців тримають оборону та пробують
контрнаступати, але сили сторін умовно рівні, і ніхто не просунувся на значимі
відстані.

Кожний такий день «без змін» коштуватиме сотень життів і нам, і ворогам,
вимагатиме напруження всіх сил та ресурсів – і так довго-довго, допоки у однієї
із сторін не сформується вирішальна перевага на тій чи іншій ділянці
довжелезного фронту. 

Така перевага буватиме спершу не тільки у нас, тож слід спокійно сприймати
тимчасові відступи із якихось районів, і так само без ейфорії сприймати
просування вперед. Довгий час все це ще буде хитким та непевним, і загалом
виглядатиме як «крок вперед  - крок назад».

Така вже ця фаза війни – не схожа ні на що із 21 століття, натомість дуже схожа
на зменшену копію Другої світової. Рано чи пізно вона і закінчиться так, як
Друга світова – повним крахом режиму, який її розпочав. Але до цього ще слід
дожити, довоювати та допрацювати.

Переконаний, мої котики та зайчики, що ми на це здатні. Здатні не гірше, аніж
були наші дідусі та бабусі, які прожили (от саме так, не пере-, а про-) Другу
світову. 

Мільйони наших персональних історій колись зіллються у один розділ у
майбутньому підручнику історії, розділ про нашу Визвольну Війну. І лише ми
знатимемо, якою різною вона була для кожного з нас, які доводилось робити
вибори, які різноманітні проблеми кожний вирішував на загальному тлі великої
історії. 

Наші онуки цього не розумітимуть, як досі ми зовсім не повно і не зовсім вірно
розуміли історії дідусів та бабусь. І так, мої котики та зайчики, ми воюємо
зокрема за те, щоб їм це ніколи не стало зрозумілим.

Слава Україні.
