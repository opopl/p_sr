%%beginhead 
 
%%file 20_01_2023.fb.fb_group.mariupol.pre_war.1.mo__mar_upolsk__robo
%%parent 20_01_2023
 
%%url https://www.facebook.com/groups/1233789547361300/posts/1393017204771866
 
%%author_id fb_group.mariupol.pre_war,jarmolenko_igor.mariupol
%%date 20_01_2023
 
%%tags mariupol,hudozhnik,isskustvo
%%title Мої маріупольські роботи. Частина 3
 
%%endhead 

\subsection{Мої маріупольські роботи. Частина 3}
\label{sec:20_01_2023.fb.fb_group.mariupol.pre_war.1.mo__mar_upolsk__robo}
 
\Purl{https://www.facebook.com/groups/1233789547361300/posts/1393017204771866}
\ifcmt
 author_begin
   author_id fb_group.mariupol.pre_war,jarmolenko_igor.mariupol
 author_end
\fi

Мої маріупольські роботи. Частина 3. Всі місця вгадаєте? )
