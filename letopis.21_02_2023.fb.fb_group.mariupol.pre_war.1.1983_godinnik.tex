%%beginhead 
 
%%file 21_02_2023.fb.fb_group.mariupol.pre_war.1.1983_godinnik
%%parent 21_02_2023
 
%%url https://www.facebook.com/groups/1233789547361300/posts/1414480812625505
 
%%author_id fb_group.mariupol.pre_war,kipcharskij_viktor.mariupol
%%date 21_02_2023
 
%%tags mariupol,mariupol.pre_war,1983,gorod
%%title 1983 Годинник
 
%%endhead 

\subsection{1983 Годинник}
\label{sec:21_02_2023.fb.fb_group.mariupol.pre_war.1.1983_godinnik}
 
\Purl{https://www.facebook.com/groups/1233789547361300/posts/1414480812625505}
\ifcmt
 author_begin
   author_id fb_group.mariupol.pre_war,kipcharskij_viktor.mariupol
 author_end
\fi

1983 Годинник

Є в Маріуполі кілька місць, назви яких звучать дивно для \enquote{іногородніх}:
Старий Аеродром, Мухіно, Чонка, Стрілка, Площа, Поворот, Хрест, Перехід...

Перехід знаходиться у самісенькому сучасному центрі міста (середмісті), а от
у 1950-ті роки то була околиця міста (передмістя). Таку назву носить
перехрестя проспектів Миру та Металургів тому, що там єдиний (поки що?) в
Маріуполі підземний перехід. А що ж було до підземного переходу? А було
звичайне перехрестя з світлофором, до того ж світлофор був не першим в
місті: раніше світлофор поставили на перехресті проспекту Металургів та
вулиці Фонтанної (дивіться спогад про циганку).

Підземний перехід будували відкритим способом: екскаваторами викопали
котлован, в якому змонтували арматурні каркаси, закрили опалубкою з
дерев’яних щитів і залили бетоном. Коли бетон набрав міцності, усе перекрили
бетонними плитами і засипали землею. Виходи з переходу почали прикрашати
гранітними плитами, але раптом роботи зупинили, а вже покладені плитки
почали відбивати. Кажуть, що та плитка сподобалась комусь з керівництва
області і тому на Переході поклали якусь жовту.

За проспектом Миру по правому боці колись стояв стовп, на якому висів
круглий електричний годинник із стрілками – стрілка хвилин щохвилини
стрибала, а годинник при цьому цокав.   Стовп з годинником прибирали на час
спорудження Переходу і повернули на місце по закінченні робіт. Потім
з’явився цифровий годинник на будинку з майстернею Арнаутова. За кілка років
той годинник вийшов з ладу, його визнали не ремонтонепридатним і замінили
(несправний годинник подарували міському ліцею і ... о диво! його таки
відремонтували і він знову показував час над входом до ліцею!).

Давайте зробимо коротеньку екскурсію проспектом Металургів. Почнемо з вулиці
Варганова (Соборна, яка не стала вулицею Савчука). В напрямку вулиці Казанцева
на Варганова була зупинка автобусу (здається, 61-го), до аеропорту (не
аеродрому – в Іллічівськом районі, не аеропорту на проспекті Будівельників, де
потім була музична школа, а того, що дорогою на Мангуш-Першотравневе). За
цегляним парканом був \enquote{туалет типу сортир} (майже у центрі міста!), де по
вечорах поціновувачі \enquote{зеленого змія} влаштовували \enquote{паті} – взимку, коли був наш
вихід на ДНД (добровольна народна дружина, або \enquote{дуракам нечего делать}), ми
мусили перевіряти, аби там ніхто не заснув і не замерз.

 Ліворуч був магазин спорттоварів \enquote{Білий парус} куди можна було зайти
подивитись на новесенькі ще не розпаковані мотоцикли \enquote{Ява} та \enquote{Чезета} та
акваланги – мрію кожного пацана... Вище \enquote{Парусу} були каси Аерофлоту. До речі,
долетіти до Київа коштувало лише на кілька рублів дорожче, ніж доїхати потягом,
але в літаку ще чимось пригощали, а в потягу треба було їсти за свої. Шкода, що
я не записував свої мандрівки по відрядженнях – цікаво було б дізнатися,
скільки кілометрів я \enquote{намотав}...

Навпроти кас Аерофлоту було \enquote{широко відоме у вузьких колах} кафе, здається
\enquote{Каштан} – там збиралися \enquote{круті} того часу (чи то качки-культуристи, чи то
каратисти). На розі будинку (тоді ще без годинника) майже перед перехрестям
була часова майстерня, куди я пару разів приносив збирати свій (точніше –
батьків наручний годинник) після \enquote{вивчення} його механізму... Правіше від
перехрестя, в тому ж будинку були два магазини – гастроном та кулінарія, які на
жаргоні називали \enquote{близнюками} або просто \enquote{яйцями} (однояйцеві близнюки). Після
роботи як не в одному, то в іншому можна було спробувати купити додому щось
їстівне (не дефіцит на кшталт зеленого горошку, чи майонезу, а, наприклад,
кальмарів чи морської капусти). Здається, в кулінарії, я спостерігав таке:
старенька бабця в черзі за оселедцями виголосила промову: \enquote{От, нам добре – нам
партія до свята оселедці дає! А в тій Америці, хто ж про негрів подбає?}.

За касами Аерофлоту у дев'ятиповерхівці на розі був магазин з квітами, а за ним
(вздовж проспекту Миру) – залізничні каси, ЦУМ, гастроном Донецьк, магазин
Дитячий Світ...

Але йдемо далі проспектом Металургів: за переходом пам'ятника Куїнджі тоді ще
не було – там стояв величезний символ єдності робочих та селян Серп та молот!

Навпроти цього символу був магазин \enquote{Жемчуг} (\enquote{Перлина}), де можна було придбати
інший символ єдності: шлюбні кільця! Оскільки золоті вироби були таким самим
дефіцитом як і інші ознаки заможності: кришталь, килими, ..., джинси,
нарешті!), їх продавали за довідкою з РАГСу (реєстрація актів громадянського
стану), яку можна було отримати неподалік – у Центральному РАГСі.  Навпроти
РАГСу височила будівля Гіпромезу. За вулицею Республіки (Університетська) була
"перша бурса" – ПТУ №1; коли ПТУ увійшло до складу ПДТУ і стало 11-м корпусом,
її поєднали переходами з Бібліотечним та П'ятим корпусами. До речі, переходи
проектував наш хороший знайомий Віктор Лебедев... Тож, за бурсою стояв
Бібліотечний корпус (бібліотека була на першому поверсі – читацький зал поруч з
кінозалом - та у цоколі (книгосховище), разом з лабораторією ТМ). Бібліотечний
корпус поєднаний з Першим (Старим) корпусом двоповерховим переходом. На
третьому поверсі колись була військова кафедра (у 1977-му році вона переїхала
на Лівий берег). Молоді аспіранти, отримавши промислову лазерну установку,
ввечері зі зварювального корпусу направляли червону цятку перед будкою вахтера
кафедри і спостерігали за його намаганнями дістати її (тільки на кажіть, що не
ганяло кота \enquote{лазерною} указкою!)!

Навпроти бібліотечного корпусу на першому поверсі п’ятиповерхівки була
Прокуратура (якось, запізнюючись, я \enquote{тормознув} \enquote{приватника-бомбилу}.
Під’їхавши до бібліотечного корпусу, я попросив зупинитись і, достаючи з кишені
грошу, витягнув аспірантський квіток: червону книжечку з гербом СРСР і написом
дрібними літерами \enquote{Жданівський металургійний інститут} і, нижче, великими:
\enquote{Аспірантура}. Водій чомусь від грошей відмовився! (Вже потім я збагнув:
навпроти – прокуратура, а на \enquote{ксиві} – \enquote{аспірантура}! За вулицею Георгієвською,
навпроти Старого корпусу стояла \enquote{сталінка}, на лоджії якої час від часу
репетирував трубач оркестру Драматичного Театру... Під час лекцій з фізики у
аудиторії 314 ми мали вибір що саме слухати: лекцію чи трубу...

Далі, за перехрестям з вулицею Апатова (Італійською), магазин \enquote{Біла акація},
куди на великій перерві бігали за булочками та бубликами (та й по пиво, чого
вже там, хоча у пляшках воно коштувало майже удвічі дорожче, ніж на розлив!).
Нижче \enquote{Акації} була аптека (не пам'ятаю, чи був вже тоді виставочний зал імені
Куїнджі?), а навпроти аптеки – Палац Культури Коксохімічного заводу, куди я в
дитинстві якийсь час ходив вчитись грати на трубі.

Нижче Коксохіму, як і зараз, стояв приватний сектор, а ліворуч – за вулицею
Пушкіна – ще одна \enquote{сталінка}, на кінці якої (вулиця Семенішина) був
\enquote{Вітамін-бар}, який студенти називали \enquote{П'ятий корпус ЖдМІ} (після того, як
з'явився справжній П'ятий корпус, Вітамін-бар став \enquote{шостим}. Десь на межі
1980-х Вітамін-бар став відділом РАГСу, а \enquote{шостим корпусом} \enquote{призначили} пивний
бар \enquote{Кавказ} на проспекті Мира (Хліб дю Солей). 

За вулицею Семенішина починався Міський Сад, де був Літній театр (не плутати з
кінотеатрами!) – це був справжній театр, у якому свого часу виступали відомі
артисти, а Палацу творчості, який будували як Палац Піонерів, ще не було...

Ну а далі проспект збігав вниз. Ліворуч, після сходів з Міськраду стоїть
приватний будинок, в якому жив і вирізав свої дерев'яні картини та скульптури
художник Коротков (у 2010-ті його роботи перенесли у аудиторію 217 ПДТУ.
Нарешті, проспект закінчувався на площі, де колись було трамвайне коло, а за
залізничними коліями плескалося море...

Але повернемося до старого годинника на стовпі (і мого наручного годинника).

У 1983-му році я жив у приватному будинку (у тещі на млинцях - у приймах) на
Горькому селищі (або імені Горького), поблизу П'ятих воріт заводу Ілліча та
зупиночного пункту 1256. Якось надвечір, перед брудною або важкою роботою я
зняв наручний годинник і повісив його на бильце металевого ліжка (пам'ятаєте
металеві ліжка з блискучими кульками?). Вранці я прокинувся, взяв годинник,
подивився час і ... очманів: було вже двадцять хвилин по сьомій (час
орієнтовно)!!! 40 хвилин до початку робочого дня! Кожного ранку кожен
співробітник (окрім викладачів) мав записати своє прізвище у журналі обліку
робочого часу! Рівно о 8:00 (з останнім сигналом відліку часу по радіо),
завідувач лабораторії підводив риску червоною пастою: усі, чиї прізвища були
під червоною рискою – були порушниками трудової дисципліни! 

Але от що \enquote{цікаво}: робочий день закінчувався о 16:12 і ніхто не відмічав, о
котрій саме ти пішов з роботи, аби не раніше. З роботі я зазвичай їздив
електричкою о 17:20 (була ще й о 18:20 і ще одно після 20:00). До вокзалу пішки
(з горки) можна було дійти в середньому за 15 хвилин, тож я виходив з
лабораторії о 17:00, \enquote{переробляючи} майже \enquote{академічну годину} – 45 хвилин. Але
це нікого не цікавило. Головне – вчасно прийти на роботу!

Зазвичай я доїжджав до Залізничного Вокзалу електричкою (зазвичай, бо
електрички теж іноді запізнювалися і тоді в дію вступав план \enquote{Б} - бігом!), а
потім тролейбусом підіймався на гору. Поки я наспіх, не вмикаючи світла,
збирався, за вікнами загуркотів по рейках состав. Вискочивши з двору, я побачив
тільки його червоні задні ліхтарі, які віддалялися... Тобто, електричка, хоч і
відстала від свого графіку, але все ж поїхала без мене.

7:25. План \enquote{Б}: бігти навколо заводу Ілліча повз поліклініку, повз Перші ворота
і до Левченка – де зупинялися тролейбуси 12 та 13, трамвай 1, та була кінцева
зупинка автобусу 8-го маршруту. 

7:33. Я на Шевченка. Забігаю у \enquote{гармошку} - Ікарус 8-го маршруту. 

7:37. Автобус все ще стоїть на кінцевій. В салоні – тільки я (водій у
кабіні). Може щойно відійшов попередній автобус, бо зазвичай в такий час
вільних сидячих місць не було (автобус йшов через усе місто аж до магазину
Азов'є і люди після нічної зміни воліли подрімати сидячи). Я питаю:

- А чого не їдемо? 

Водій:

\obeycr
- Як хочеш, то поїхали.
- Я на роботу запізнююсь!!!
- А о котрій твоя робота? 
- О восьмій!
- Встигнемо...
\restorecr

Автобус неспішно рушає з місця...

Зупинка Другі Ворота: ніхто не заходить. Більше того: на вулицях майже нема ані людей, ані машин.

7:40. Треті ворота. На зупинці нікого. Водій дивиться на мене у дзеркало і
не зупиняється. Більше того: він додає \enquote{газу}. Автобус мчить без зупинок.

7:53. Центральний Ринок. Я вискакую з автобусу і біжу що є сил: навіть на
Варганова (Соборна) встигаю раніше за \enquote{мій} автобус. Ніхто не заважає мені
бігти: на вулицях майже нікого.

7:56. Перехід, якого тоді ще не було. Годинник на стовпі показує той самий час.

8:00. Я влітаю у Перший корпус, мало не збивши з ніг вахтера, що стояв у
дверях, стрімголов \enquote{взлітаю} на третій поверх, смикаю ручку 316-ї... Що
таке? Чому двері закриті на ключ? Відкриваю. Темно. Вмикаю світло.
Порожньо... Чому? Чому на вулицях майже не було ані машин, ані людей? Чому
перед крильцем не було курців? Чому порожньо в лабораторії? Телефоную у
кабінет завлаба... Не бере... На кафедру... Теж ніхто не відповідає...

\obeycr
Щось сталося?
Вмикаю радіо...
Тиша...
Може надзвичайний стан?
Що має робити військовозобов'язаний?
З'явитись у військомат з військовим білетом. 
Білет, звісно ж вдома.
\restorecr

Хвилин 40 я намагався збагнути цю ситуацію, випалив кілька сигарет... Йти до
вахтерів і показувати свою необізнаність я не хотів. Добре, якщо всі
чоловіки у військоматі, то де ж секретарка кафедри? Де лаборантки?

Нарешті у радіо щось зашурхотіло...

Потім заграв гімн...

\enquote{Передаємо сигнали точного часу. Початок шостого сигналу відповідає шостій годині...}

!!!

ШОСТА!!!

Не 8:47, як показує мій годинник!!!!

Пояснення одне: надвечір мій годинник зупинився десь о 19:40. Вранці, коли я
взяв його у руки і, ймовірно, струсив, він пішов знову, показуючи 7:40 в той
час, коли насправді було десь 5:30. 

Те, що годинник на стовпі і годинник на руці показали майже однаковий час – збіг.

А от пережитий мною стрес - реальність

Питання: що думав водій, який о п'ятій віз пасажира, який боявся запізнитися на роботу о восьмій?

Фото:

1. Старий перехід
2. Сучасний перехід (за деревами)
3. Гастроном Донецьк та ЦУМ
4, 5 РАГС тоді і зараз
6. Бібліотека
7. Бібліотечний корпус та будинок з трубачем (ліворуч)
8. Літній театр у Міському Саду.
