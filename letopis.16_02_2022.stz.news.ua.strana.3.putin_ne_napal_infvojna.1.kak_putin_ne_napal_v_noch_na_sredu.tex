% vim: keymap=russian-jcukenwin
%%beginhead 
 
%%file 16_02_2022.stz.news.ua.strana.3.putin_ne_napal_infvojna.1.kak_putin_ne_napal_v_noch_na_sredu
%%parent 16_02_2022.stz.news.ua.strana.3.putin_ne_napal_infvojna
 
%%url 
 
%%author_id 
%%date 
 
%%tags 
%%title 
 
%%endhead 

\subsubsection{Как Путин не напал в ночь на среду}

О том, что 16-го февраля Россия может пойти в атаку, на Западе сообщили еще на
прошлой неделе.

Это было явно частью скоординированной кампании. Сначала медиа назвали \enquote{время
Ч}, а потом власти США начали срочную эвакуацию своих граждан и дипломатов. 

Далее начался эффект домино: другие страны, в том числе очень далекие от
Украины, вроде Ирака, Багам или Японии, начали спешно сокращать свое
присутствие.

Логика их рассуждений, видимо, была такой: мы не знаем, кто на кого нападет. Но
если американцы убегают, значит, затевается что-то нехорошее. На что, кстати, и
был скорее всего расчет Соединенных Штатов. 

Это вызвало глобальный шквал паники вокруг Украины. Под угрозой оказались
авиационные и морские перевозки. И на этом фоне британские СМИ вчера поддали
еще больше жару: заявили, что Россия нападет ровно в три часа ночи на 16-е
февраля. 

Учитывая контекст - повальное бегство из Украины зарубежных гостей во главе с
американцами - информация о \enquote{нападении в три часа} ночи была воспринята
максимально серьезно. Причем даже теми, кто не верил в планы России на кого-то
нападать (эта часть украинцев ожидала неких провокаций для вовлечения РФ в
войну). 

Но ждали абсолютно все, кто следит за новостями. И вот, ничего не произошло, а
ночь прошла абсолютно спокойно. 

