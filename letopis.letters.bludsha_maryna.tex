% vim: keymap=russian-jcukenwin
%%beginhead 
 
%%file letters.bludsha_maryna
%%parent letters
 
%%url 
 
%%author_id 
%%date 
 
%%tags 
%%title 
 
%%endhead 

Доброго дня, пані Марино! Пишу Вам з приводу Вашого путівника Путівник по
Маріуполю. Pictures inside us. Сьогодні натрапив на посилання в одному пості.
Як я розумію, він був знищений через те, що Тільда - роскомпанія, і це звісно
правильно, що не треба підтримувати роскомпанії під час війни (я побачив
повідомлення на адресі сайту, що цей сайт був знищений). Але, на жаль, я не
зміг знайти ніде копію сайту, і це дуже неймовірно сумно, оскільки наскільки я
бачу, це була просто неймовірно красива робота по Маріуполю. Щодо мене - мене
звати Іван, я киянин, програміст, живу і працюю в Києві. У мене є проєкт
Літопису Війни, де я займаюсь систематизацією та архивацією різних публікацій,
і взагалі кажучи, не тільки про війну. Останнім часом я активно займаюсь
Маріуполем... Як все це виглядає, можна подивитись ось тут
https://t.me/kyiv_fortress_1 або ж тут
https://archive.org/details/@kyiv_chronicler (викладено лише насправді мала
частина архівованих матеріалів). Якщо у Вас або Ваших друзів/колег залишились
матеріали сайту, я думаю, було б добре перенести сайт на якусь іншу платформу
або взагалі переписати заново, із тими самими матеріалами, бо задумка дійсно
чудово. І це особливо важливо, оскільки Маріуполь, на мій погляд, потребує не
тільки чисто військового звільнення, а також культурно-духовного відродження.
Так, фізично він був знищений, але в духовному просторі він залишився, у
вигляді спогадів, фото, книжок, постів на фб. Але... якщо не займатись
систематично збереженням спадщини Маріуполя, то існує, на мій погляд, велика
загроза в тому, що Маріуполь буде забутий в культурно-духовній площині також, і
це буде означати фактично духовну смерть Маріуполя. І зокрема, як одна ознак
цього процесу... я майже не бачу книжок про Маріуполь в Києві. Оббігав купу
книгарень - книжок про Маріуполь або немає зовсім, або одна-дві назва
(невеличкі книжки Сухорукової та Шишацького - про трагічні події), а про
красивий, гарний, мирний Маріуполь - зовсім немає. Тому я і займаюсь зараз
Маріуполем, записую, систематизую. Маріуполь - неймовірно гарне і красиве
місто, справжня Перлина біля Моря! - яке неодмінно знову воскресне і засяє
новими барвами! Але щоб цього досягти, треба ще дуже багато роботи попереду...
А щодо сайту, буду дуже радий, якщо первісні матеріали збереглися, і якщо є
можливість відтворити сайт на новій платформі на новій адресі, або ж хоча б
заархівувати та опублікувати матеріали (не в формі сайту, а в формі книжки - я
це так роблю ). З повагою, Іван.

Доброго дня! Звінок чи відео - для мене це той самий персональний контакт, про який я
казав, тільки трохи дальніший по інтенсивності, звісно. Крім того, наскільки я
розумію по тому, що я встиг вже прочитати в Ваших постах, Ви скоріше
екстраверт, любите спілкуватись багато, знайомитись, ведете публічну діяльність, 
а я навпаки - інтроверт. Програміст, що ж поробиш. Писати і читати
- я люблю, а от спілкуватись вживу - значно менше. І якщо Ви колись дивились
фільм Ігри Розуму (2001), то певні психологічні риси головного героя того фільму є і у мене (я навіть
написав невелику оповідь про той фільм і свої враження, якщо цікаво, можу потім
вислати). А щодо довго писати... (1) я бачу, Ви якраз пишете багато чого кожен
день, добре все пишете, - дуже змістовно, послідовно і цікаво! - так що я
впевнений, що це не буде проблеми для Вас сформулювати всі Ваші побажання щодо
сайту у зрозумілій і чіткій формі (2) Крім того, розробка сайту - робота над
документами, реалізація або обговорення якихось технічних моментів  - це доволі
розумово насичена робота, і тут якраз треба писати, якщо треба, писати стільки,
скільки треба, чітко формулювати, що треба робити і у який спосіб. (3) До речі,
не знаю чи граєте Ви у шахи. Я цю гру дуже люблю і відносно непогано граю. В
шахах є різні стилі гри (бліц - швидка гра, дається 5 хвилин на гру ), звичайна
партія (30 хвилин на партію, скажімо), і є... гра по переписці. Зараз не знаю,
як з цим, тому що вже зараз у нас Інтернет Епоха, і я активно за шаховим світом не слідкую,
просто іноді заходжу пограти онлайн з випадковими суперниками. А раніше люди залюбки грали
по переписці.  Починали партію, слали один одному листи, партія могла... місяць
тривати ))) так от, повертаючись до питання співпраці. Якщо Ви дійсно хочете
знову все відновити, і також співпрацювати у роботі над цим, Вам доведеться так
чи інакше витрачати час на те, щоб писати... і можливо доволі багато писати...
Тепер а щодо власне моїх пропозицій, що можна було б зробити. (1) необхідно
зробити електронний варіант-книжку, яка буде все це містити, що було на сайті,
і яку будь-хто міг би скачати і зберегти собі. Як Ви вже бачили, все це можна
зробити і доволі швидко. Я на цьому вже набив руку, і це не проблема зробити
для мене книжку (з покажчиком, заголовком, внутрішніми лінками, - все як
треба) з даних матеріалів за відносно малий час (декілька днів, якщо тільки цим
і займатись) (2) ну і крім варіанту-книжки, необхідно, звісно відновити сайт
сам по собі, і звісно, я думаю, він буде ще кращий, ніж був, тому що туди можна
буде додати багато всякого, чого не було в старій версії (3) а щодо технічної
реалізації самого сайту, є дві опції - або (3.1) підшукати платформу (як-от
вордпресс або щось інше) або (3.2) написати сайт з нуля. Із цим усім я зможу
Вам допомогти. В термінах часу і витрачання сил - мушу однак зауважити, що
зараз мій основний пріорітет - це архівація, систематизація та поширення того,
що пишеться і що потрібно зберегти, тому я би почав з варіанту книжки. А сайт
можна зробити набагато крутішим, але це питання часу, оскільки це не так швидко
робиться, як компіляція ось таких файлів-книжок, які ви вже бачили
