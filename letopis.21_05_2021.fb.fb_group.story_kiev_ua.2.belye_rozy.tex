% vim: keymap=russian-jcukenwin
%%beginhead 
 
%%file 21_05_2021.fb.fb_group.story_kiev_ua.2.belye_rozy
%%parent 21_05_2021
 
%%url https://www.facebook.com/groups/story.kiev.ua/permalink/1666864816843646
 
%%author Киевские Истории
%%author_id fb_group.story_kiev_ua
%%author_url 
 
%%tags 
%%title «Белые розы и одно венское кресло»
 
%%endhead 
 
\subsection{«Белые розы и одно венское кресло»}
\label{sec:21_05_2021.fb.fb_group.story_kiev_ua.2.belye_rozy}
\Purl{https://www.facebook.com/groups/story.kiev.ua/permalink/1666864816843646}
\ifcmt
 author_begin
   author_id fb_group.story_kiev_ua
 author_end
\fi

\textbf{Svetlana Kievljanka}

Полгода я писала,писала,писала,все глубже и глубже затягивала вас в прошлое...Но всему,даже самому интересному,когда-то приходит конец.
Я помню некоторые комментарии,которые трогали до слез «Пожалуйста,не бросайте нас,мы привыкли....».И это было так искренне...
Осенью прошлого года я очень осторожно поинтересовалась , а что,если я познакомлю вас с мемуарами мужа...Отклик был сумасшедшими.Я глазам своим не верила.Что?! Так бывает?! Оказалось,бывает...Передать словами то,что мы с мужем пережили за эти полгода,я не могу.Восторг - самое скромное ощущение...И мы очень благодарны вам за все это.
Но....я хочу открыть вам маленькую тайну.Наверное,сейчас я буду точно в таком же состоянии,как тогда,осенью...И вот,почему.
Перед тем,как заставить мужа рассказать мне ВСЕ,я пыталась писать «для себя».Пробовала себя в рассказах.И написала их...(не поверите,не помню!) то ли шесть,то ли восемь)))Но практически все они связаны с нашим городом! С Киевом.Более того,многие ситуации,которые я включила в сюжет,реальны...
Судить о качестве я не могу.Я не профессиональный литератор.Но мне хотелось бы показать вам кое-что.Как вы к этому отнесётесь? Обещаю любую критику принять адекватно)))Мне интересна сама идея.Вам это нужно? Или...я слишком много на себя беру,учитывая то,что уже было опубликовано в группе от имени мужа? 
Сейчас я дам небольшой отрывок из одного из рассказов.Самое начало...Если это будет «провал», я замолкну ,обещаю))) Ну? С Богом?
«Белые розы и одно венское кресло».


\ifcmt
  pic https://scontent-bos3-1.xx.fbcdn.net/v/t1.6435-9/p843x403/188476768_2928705024069807_1134717124157587793_n.jpg?_nc_cat=108&ccb=1-3&_nc_sid=825194&_nc_ohc=1pRSz7erlMUAX_xn5XQ&_nc_ht=scontent-bos3-1.xx&tp=6&oh=e6f86d040c73f8abc97167630938cb4e&oe=60D1AEB3
\fi


Сегодня она нарушила сразу три правила,которые были в доме незыблемыми с незапамятных времён ! Она прогуляла работу,сославшись на недомогание,она вынесла на балкон старинное венское кресло и она закурила...Нет! В принципе,она не курила! Это так,баловство,но под рюмочку коньячку,извлеченного из старого дубового буфета - самое оно! Да и сигареты-то не ее,Ленка как-то забыла.А сегодня эта тоненькая сигаретка стала символом протеста! 
Да,да.Неля сегодня протестовала! Вот как проснулась,сразу решила , а что если.....И понеслось...
Первым делом она извлекла из шифоньера кусок изумительной сирийской ткани .Это был потрясающей красоты шёлк,купленный как-то по случаю в командировке.Он сиял на Солнце шоколадно-золотыми бликами,напоминая о поездке,о фантастическом аромате восточного кофе на оживленном рынке,об экзотических фруктах и громкой , непонятной речи. По сути,этот отрез ей всучил навязчивый продавец.Она просто не знала , как от него отвязаться, а потому единственным вариантом было купить этот кусок ткани.И вот совсем недавно она наткнулась на него, перебирая вещи.Идея возникла моментально - она , наконец , перетянет то старинное кресло,в котором так любил сидеть ее отец.До отца - дед.Изредка в нем мог лежать их старенький пудель Гоша.Тогда папа или дедушка пересаживались на диван.Только этим троим разрешалось прикасаться к семейной реликвии - креслу с высокой гнутой спинкой.Несколько раз мама предлагала обновить обшивку,но мужчины не соглашались,ссылаясь на антикварную ценность вещи.Но шло время,сначала не стало Нелиного дедушки,а в прошлом году тихо один за другим ушли мама и отец...Мама просто уснула и не проснулась,а отец стал все чаще закрываться в кабинете и разговаривать с ней.Однажды вечером дочь застала его сидящим в этом самом кресле с широко открытыми глазами,которые уже не видели ничего.В ладони папа крепко сжимал два обручальных кольца.Он ушёл следом за мамой на сороковой день после ее смерти.Неля осталась одна.Ну,как одна...Не совсем.Но об этом позже.
А пока она глотнула коньячку, сделала затяжку,резко закашлялась,поморщившись,затушила сигарету и , вытянув ноги,начала вспоминать сегодняшний день.
Вообще-то какие-то странные предчувствия преследовали ее уже несколько дней.Не тревога,нет.Просто ощущение , что должно произойти что-то значимое , что-то , что перевернёт ее жизнь.Что именно , она не понимала . А потому решила брать судьбу в свои руки и действовать. Решение перетянуть кресло - первое , что пришло ей в голову.Тем более , она наткнулась на тот отрез . «Может, это знак какой?» - подумала Неля.Ее сосед Витёк - потрясающий мастер-краснодеревщик.Руки золотые,характер тоже.Вот только беда - любил выпить,а,точнее,запить.Это в прошлом веке народ ремонтировал мебель,сейчас легче купить новую,чем восстанавливать рухлядь.И все чаще сосед оставался без работы, перебиваясь случайными заработками.Так,по случаю, мог табуретку кому-то отремонтировать, да и то только своим , не хотел терять «лицо», все-таки не подмастерье - мастер.... «Только бы он был трезвым» ,-думала Неля , летя по ступенькам вниз! Впервые в жизни она солгала, позвонив на работу и сообщив о внезапном недомогании.День был свободен! 
Уже через час Витёк колдовал над креслом.Аккуратно,чтобы не поцарапать,снимал старую,потертую ткань,плавными,неторопливыми движениями вынимал старые гвоздики.И улыбался....Было видно - работа доставляла ему истинное наслаждение! К вечеру кресло стояло,как новое.Нет,он,конечно,настаивал на лаке и прочих изысках,но хозяйке кресло нужно было прям сейчас! Некогда было ждать! Потому что с него должны были начаться перемены! Какие? Посмотрим...
Вдвоём они перетащили громадное кресло на балкон , выходящий во двор , который принято называть колодцем ,Витёк ещё раз окинул взглядом свою работу,и,зажав в кулаке приличную сумму , рванул отмечать удачный день.
А Неля...Она уселась в глубокое кресло,поставила на маленький ломберный столик хрустальный графинчик с ароматным напитком и принялась вспоминать...
Их квартира находилась в районе , который сегодня принято называть элитным - центр Киева , в трёх шагах все правительственные здания ,Нацбанк . Интеллигентная публика , старики с тросточками и дамы в шляпках с вуалью - то, что Неля  помнила с самого раннего детства.А ещё лифт.Не такой , как сейчас , а громадный , чугунный , где двери открывали сами пассажиры.Он ехал , грохотал  и скрипел жутко , Неля боялась его , как какого-то чудища из страшных сказок , потому взбегала на свой пятый этаж пешком.Да и квартиры располагались не так , как строят сейчас , а в шахматном порядке - все на полу-пролетах.Практически на всех дверях висели красивые бронзовые таблички с именами хозяина «Адвокат Бернштейн И.Я», «Доктор хирургии Вильнович А.Д.» , «Инженер Покатило В.В». На их двери тоже была такая табличка - «Яков Львович Розенберг.Стоматолог».Это был ее дедушка.Семья жила в этом доме с незапамятных времён.С трудом пережив годы революции и всяческие терроры,Яков Львович и его жена остались живы.У революционеров тоже болели зубы.Их пытались тронуть,но высокое начальство сказало «штыл» , семью оставили в покое.
Так они и жили в квартире с трехметровыми потолками,со встроенным под огромным подоконником холодильником,с ванной на чугунных ножках в форме львиных лап.По огромному холлу мотался на трехколёсном велосипеде их сын Аркаша,как позже и Неля, а на большущем квадратном балконе,увитом «девичьим» виноградом,бабушка сидела над большущим медным тазом и перебирала вишню,вытаскивая косточки каким-то чудным приспособлением.Да,да,все новое - давно забытое старое! 
Окна и балкон выходили в двор-колодец.А потому все значимые и не очень события были известны всем,кто жил по соседству - акустика во дворе была отменная.По воскресеньям во двор приходил старьёвщик и кричал зычным голосом «Стааарые вещи! Покупаю стааарые вещи». А в субботу весь двор заполнялся звуком «вжжжик,вжжжик» - это приходил точильщик.Иногда раздавался голос лудильщика «Кастрюли,чайники! Лудим,зудим»! 
Раз в неделю Неля с дедушкой ходили в прачечную,по дороге заправляя сифоны с газировкой.Для неё всегда брали сифон с малиновым сиропом.Неподалеку располагалась пожарная часть,а потому весь двор был в курсе,если где-то случался пожар...
Так и прошло ее детство в кругу добрых и вежливых людей.Она выросла,по семейной традиции поступила в мединститут и стала потомственным стоматологом.
В семье все было строго и красиво - крахмальное белье и белоснежные скатерти,воскресные обеды с ароматными пирогами и празднование 1-го мая.Почему-то именно этот праздник запомнился ей больше всего.Ей всегда казалось,что с именно с этой датой связана какая-то семейная тайна.Другого объяснения такой любви к первомаю она найти не могла.День,как день.И картошку сажать негде,как другие делают - дачей семья не обзавелась,а в селе родни не было...
Но вернёмся на несколько лет назад.
Неля была девчонкой-сорванцом,а потому все чердаки-подвалы и бомбоубежища района были обследованы ею досконально.Именно там,в одном из бомбоубежищ совсем юная леди из приличной семьи и познакомилась с Женькой.Он жил в соседнем подъезде в квартире с такой же бронзовой табличкой на дверях.Вместе они проводили кучу времени,а в своё время отправились в один класс школы по соседству.Интерес у друг другу они потеряли классе в пятом - Женьку отправили в музыкальную школу,а Нелли - на бальные танцы.Каникулы они тоже проводили поразнь, отправляемые родителями в разные пионерлагеря.
И вдруг , переступив порог 10-го класса,они встретились глазами.Та самая искра проскочила и разгорелся костёр.Они начали понимать,как их тянет друг к другу.Все чаще они задерживались в парке, не спеша домой.Сбегали с уроков,боясь быть разоблаченными, чтобы поесть мороженое в мельхиоровых стаканчиках.Иногда,пропуская занятия в своих кружках и школах,отправлялись в кино.Влюбленность - вот,что это было...
Женька ненавидел свою скрипку до ужаса,ему больше нравилась живопись.А потому в портфеле всегда лежал альбом и цветные карандаши.Потом появилась пастель.Родителям пришлось сдаться и Женьке купили мольберт и краски.Начинал он с акварели,потом перешёл на масло.Но в художественную школу отдавать подростка никто не спешил , да и поздно уже было.Но работы пацана-самоучки поражали - такая игра красок,такая фантазия,такая непохожесть ни на кого.Талант.Это был настоящий талант.
——————
Свидание было назначено на 18:00.Женьки не было.Неля перебирала в пальцах носовой платочек и поглядывала на часы.Он впервые не пришёл.
«Что-то случилось,определённо,что-то случилось...» ,- думала она.Прошел час,ещё полчаса...А ещё через пятнадцать минут она стояла перед его дверью и слышала громкие женские голоса там,внутри.Они перемежались слезами и криком.Неля долго не решалась нажать на звонок. «Наверное,деда не стало.....» 
Но дед сидел в кухне,курил одну за другой папиросы «Казбек» и молчал.Ему,потомственному инженеру,парторгу завода,коммунисту со стажем было больно осознавать все то,что происходит в этот момент в его квартире.Кричали двое - его жена Лиза и дочь Катя.Они периодически расходились в разные комнаты,крик прекращался,но ненадолго.Первой взрывалась старшая и все начиналось сначала...Звонок в дверь в столь негостевое время привёл в чувство обеих.Вытерев слезы,к двери поплелась мама Жени.На пороге стояла Неля.
- Теть Кать,что случилось? Дедушка?
- Да какой дедушка, причём тут дедушка.Ай ,- махнула рукой женщина,- сейчас я Жеку позову...
Неля стояла на пороге не в силах понять,что же произошло.
- Пойдём.Прости,не смог выбраться.Я тебе все объясню...
Женька был чернее тучи.Накинув на плечи куртку,схватив фотоаппарат,он взял девушку за руку , они вышли в прохладу майского вечера.
Они сидели на склонах Днепра.Говорил Женька.Неля понимала,что...ничего не понимает.Или не хочет понимать.Или ее мозг отказывается понимать....Они уезжают.Они уезжают в США.Они уезжают в США навсегда.Родители получили разрешение на выезд.Очнулась она тогда,когда почувствовала,как Жека трясёт ее за плечи.
- Ну что ты молчишь? Скажи хоть что-нибудь! Пожалуйста,не молчи.Ты же видишь,мне ещё тяжелее! 
А что сказать? Что она может ему сказать прямо сейчас? Все.Это конец.Конец их любви,конец романтике,конец всему! Да все! Конец жизни! Все,ради чего она жила - в пропасть.....Они уезжают навсегда...
- Так! Прекрати! Я заберу тебя! Я клянусь! Заберу,как только стану на ноги! Мы будем вместе! Ты меня слышишь?! Нет конца, это начало.Разве можно быть счастливыми здесь? Жизнь - она там! Свобода - она там! Ты понимаешь? Это клетка! И даже не золотая,а ржавая...А там.Там мы будем счастливы!!! Нель! Любимая.Ты просто дождись меня,умоляю.Несколько лет.Всего несколько лет.Представь,что меня в армию забрали.Во флот.На три года! Представила?
Что он говорит......Господи.Какая Америка? Родители,дедушка,институт на носу.Да и школу , для начала ,  закончить надо.Все....Это все...Это конец.
Неля медленно поднялась и двинулась в сторону дома.Ее кавалер забегал вперёд,щёлкал фотоаппаратом,снимал ее сзади,сбоку.На память.Она не видела ни-че-го...Пустота и боль.Вот,что занимало ее душу.Точнее,опустошало.
Дома она не плакала.Просто сидела,поджав ноги,на балконе и молчала.Мама несколько раз заглядывала,напоминая,что пора ложиться,потому что послезавтра первый выпускной экзамен , но Неля так и просидела до первых лучей Солнца.Какой сон,когда жизнь под откос...
Вот и позади экзамены.С Женькой все эти дни она старалась не говорить - да,нет,не знаю.Боль не отпускала...И разочарование.На выпускном все веселились.Мальчишки упились за углом школы,девчонки пили шампанское прямо в школе.Сегодня все стали взрослыми.Вдруг и сразу.Учителя кивали головами и грозили пальчиком,но тщетно.50 их бывших воспитанников шли во взрослую жизнь.Под утро все двинулись на обзорную площадку к Днепру.Традицию встречать рассвет первого взрослого дня придумал кто-то очень давно,но ее не нарушали.
Только двоих не было - Женьки и Нели.Уйдя с середины выпускного,они пошли на своё привычное место на склонах Днепра и смотрели на тихую реку и просыпающийся город.Молчать больше было нельзя.Надо было поговорить.Жека говорил долго и сбивчиво,как будто решил выговориться за весь месяц.Неля отвечала тихо и спокойно.Он рисовал ей перспективы,а она отгоняла их молча, зная , что все это лишь слова.Он уедет,она останется.И так будет всегда.Так и никак иначе.
И тут где-то близко грохнуло «Белые розы! Белые розы! Беззащитны шипыыыы!» .....
————————-
Курить больше не хотелось.
- Фу,гадость какая,- вслух произнесла Неля,налив ещё граммульку коньяку.
И вечер такой тихий....Звезды.И вдруг !
«Белые розы! Белые розы!...»
Это ещё что такое?! С ума , что ли кто-то сошёл?! Лет 15 эту дрянь не слышала! Ностальгия кого-то замучила?! Ненавижу! Ненавижу эту песню....
————————-
Провожали Женькину семью всем двором.Любили их.Стол заказали в ресторане неподалёку.Нелиных тоже позвали.Не было только Женькиного деда .Не простил он дочь-предательницу.Родина воспитала, образование дала,профессию,а они....Нет.Не простил...
————————
Ладно.Ложусь.День чудесный. Но...Не то.Не то случилось.Что-то другое должно быть.Но что??? Завтра,что ли? Посмотрим.
