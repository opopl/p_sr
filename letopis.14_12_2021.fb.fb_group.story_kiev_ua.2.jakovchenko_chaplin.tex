% vim: keymap=russian-jcukenwin
%%beginhead 
 
%%file 14_12_2021.fb.fb_group.story_kiev_ua.2.jakovchenko_chaplin
%%parent 14_12_2021
 
%%url https://www.facebook.com/groups/story.kiev.ua/posts/1818295128367280
 
%%author_id fb_group.story_kiev_ua,opolskaja_nina
%%date 
 
%%tags chaplin_charli,jakovchenko_nikolaj.akter,kiev,kino,kultura,teatr
%%title "Украинский Чарли Чаплин" - Николай Федорович Яковченко
 
%%endhead 
 
\subsection{\enquote{Украинский Чарли Чаплин} - Николай Федорович Яковченко}
\label{sec:14_12_2021.fb.fb_group.story_kiev_ua.2.jakovchenko_chaplin}
 
\Purl{https://www.facebook.com/groups/story.kiev.ua/posts/1818295128367280}
\ifcmt
 author_begin
   author_id fb_group.story_kiev_ua,opolskaja_nina
 author_end
\fi

Его называют \enquote{украинский Чарли Чаплин}. Николай Федорович Яковченко
входит в число лучших комедийных актеров страны. Он обладал редким уникальным
талантом, образы, созданные им, отличались большой жизненной правдой... Это
была не просто игра, а жизнь на сцене. Актер настолько чувствовал своих
персонажей, знал их характеры и этим покорял публику. Николай шутил-Моя
кормилица-это моя рожа.  Он просто завораживал своим обаянием. Доверчивое
открытое лицо и глаза с каким-то взглядом просто в самую душу, искренние и
непосредственные.. 

Родился Николай Яковченко в городе Прилуки Полтавской губернии. Специального
актерского образования не имел, в 1918 году дебютировал в Прилуках на
любительской сцене.  Потом играл в разных театрах страны. 

\begin{multicols}{2} % {
\setlength{\parindent}{0pt}

\ii{14_12_2021.fb.fb_group.story_kiev_ua.2.jakovchenko_chaplin.pic.1}
\ii{14_12_2021.fb.fb_group.story_kiev_ua.2.jakovchenko_chaplin.pic.1.cmt}

\ii{14_12_2021.fb.fb_group.story_kiev_ua.2.jakovchenko_chaplin.pic.2}
\ii{14_12_2021.fb.fb_group.story_kiev_ua.2.jakovchenko_chaplin.pic.2.cmt}

\ii{14_12_2021.fb.fb_group.story_kiev_ua.2.jakovchenko_chaplin.pic.3}

\ii{14_12_2021.fb.fb_group.story_kiev_ua.2.jakovchenko_chaplin.pic.4}
\ii{14_12_2021.fb.fb_group.story_kiev_ua.2.jakovchenko_chaplin.pic.4.cmt}
\end{multicols} % }

С 1928 года - в Киевском театре им. Франко. В 1931 году переехал в Харьковский
театр революции, в 1934 году возвратился в театр Франко на должность
\enquote{актер высшей категории}. Гнат Юра шутил о возвращении актера как о
\enquote{возвращении блудного сына}. 

С сентября 1939 года по март 1940 года участвует в Советско-финской войне. По
возвращению восемь месяцев работает в Киевском театре музкомедии. Накануне
Сталинградской битвы  в годы ВОВ актеры выступали в штабе фронта и госпиталях.
Николай Яковченко  был награжден боевыми наградами-медали \enquote{За оборону
Сталинграда}, и \enquote{За доблестный труд в ВОВ 1941-1945гг}. 

В 1943 году присвоено звание Заслуженного артиста УССР. Главные театральные
роли-Долгоносик, Бублик (\enquote{В степях Украины}, \enquote{Платон Кречет}).
Затем игра в спектаклях \enquote{Мартын Боруля}, \enquote{На западе бой},
\enquote{Богдан Хмельницкий}, \enquote{Макар Диброва}, \enquote{За двумя
зайцами}.

Николай Федорович снялся в фильмах \enquote{Максим Перепелиця}, \enquote{За
двумя зайцами}, \enquote{Вечера на хуторе близ Диканьки}. Незабываемые яркие
роли! Звание народного артиста получает в 1970 году, за четыре года до ухода,
ну, хоть при жизни. Коллеги любили и уважали актера, зрители восторгались его
игрой, даром от Всевышнего так непринужденно и искренне исполнять роли. 

По жизни Николай Федорович был добрым и человечным. Его байки в театре и вне
его пользовались огромным успехом и вызывали добрую улыбку, а то и хохот. В
личной жизни были трудности, ушла из жизни жена, актриса театра, остался с
дочками. Потом не стало старшей дочери.... Можно понять, как было трудно... 

А вот и скверик возле театра им. Франко.. Знаменитый киевский актер сидит на
лавочке, а у ног его - преданный и любимый пёс Фанфан.. Собака стала для
пожилого человека другом и собеседником, он привязался к ней всей душой.
Киевляне полюбили памятник... Спина собаки и колено бронзового Яковченко уже
вытерты до блеска.. 

Памятник стал для жителей Города близким и родным. От него веет какой-то
умиротворенностью и спокойствием. А создан был памятник в 2000году в честь
100-летия Николая Федоровича Яковченко. На площади перед родным театром, где
долгие годы жизни играл актер свои любимые комические роли... Единственной
драматической ролью актера является главная роль в фильме \enquote{Дед левого крайнего}
(1974 год, реж. Леонид Осыка). Покоится на Байковом кладбище в Киеве.... Помним
и гордимся..
