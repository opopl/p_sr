% vim: keymap=russian-jcukenwin
%%beginhead 
 
%%file 13_08_2022.stz.news.ua.donbas24.2.azovstal_v_maljunkah_hudozhnykiv_z_usjogo_svity.txt
%%parent 13_08_2022.stz.news.ua.donbas24.2.azovstal_v_maljunkah_hudozhnykiv_z_usjogo_svity
 
%%url 
 
%%author_id 
%%date 
 
%%tags 
%%title 
 
%%endhead 

Алевтина Швецова (Марiуполь)
13_08_2022.alevtina_shvecova.donbas24.azovstal_v_maljunkah_hudozhnykiv_z_usjogo_svity
Маріуполь,Україна,Мариуполь,Украина,Mariupol,Ukraine,Азовсталь,Azovstal,Малюнок,Рисунок,Drawing,Artist,Художник,date.13_08_2022

«Азовсталь» в малюнках художників і ілюстраторів з усього світу (ФОТО)

Протистояння українських захисників і захисниць на території металургійного
гіганта — комбінату «Азовсталь» — вже увійшло в світову історію. Потужним
відображенням долі оборонців «Азовсталі» стали арти, малюнки, заклики врятувати
людей з обложеного рашистами комбінату, і ці зображення не залишали байдужими
користувачів соціальних мереж і людей з різних міст і країн, які виходили з
плакатами на мирні акції, присвячені порятунку захисників «Азовсталі».

До 89-річчя комбінату «Азовсталь» Донбас24 підготував добірку ілюстрацій і
малюнків, які були створені художниками і художницями з усього світу. На них
зображені мужні та сміливі захисники України; цивільні мешканці та мешканки
Маріуполя, які ховалися в азовстальських бункерах від потужних російських
обстрілів, що ні на мить не вщухали; дружини та діти оборонців, які чекають
вдома на своїх рідних, а також металургійний гігант, який через
російсько-українську війну перетворився на фортецю Маріуполя.

Читайте також: «Азовсталь» — фортеця Маріуполя: нарис про металургійний
комбінат (ВІДЕО).

Ще більше новин та найактуальніша інформація про Донецьку та Луганську області
в нашому телеграм-каналі Донбас24.

ФОТО: з відкритих джерел, арт обкладинки Dmytro Pozhirauskas.
