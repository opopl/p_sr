% vim: keymap=russian-jcukenwin
%%beginhead 
 
%%file 22_01_2022.stz.news.lnr.lug_info.1.koncert_muzykant_iz_moskvy_filarmonia
%%parent 22_01_2022
 
%%url https://lug-info.com/news/pianist-iz-moskvy-vystupil-v-luganske-s-simfoniceskim-orkestrom-filarmonii
 
%%author_id news.lnr.lug_info
%%date 
 
%%tags filarmonia,kultura,lnr,lugansk,moskva,muzyka,pianist,rossia
%%title Пианист из Москвы выступил в Луганске с симфоническим оркестром филармонии
 
%%endhead 
 
\subsection{Пианист из Москвы выступил в Луганске с симфоническим оркестром филармонии}
\label{sec:22_01_2022.stz.news.lnr.lug_info.1.koncert_muzykant_iz_moskvy_filarmonia}
 
\Purl{https://lug-info.com/news/pianist-iz-moskvy-vystupil-v-luganske-s-simfoniceskim-orkestrom-filarmonii}
\ifcmt
 author_begin
   author_id news.lnr.lug_info
 author_end
\fi

Московский пианист Константин Хачикян принял участие в программе \enquote{Чайковский и
Григ} академического симфонического оркестра Луганской академической
филармонии. Об этом сообщили в отделе информационной политики и рекламы
учреждения культуры.

\ii{22_01_2022.stz.news.lnr.lug_info.1.koncert_muzykant_iz_moskvy_filarmonia.pic.1}

\enquote{В первом отделении программы коллектив под управлением заслуженного деятеля
искусств ЛНР Александра Щурова представил Симфонию № 1 Петра Чайковского.
Лауреат международных конкурсов Константин Хачикян из Москвы во втором
отделении исполнил Концерт для фортепиано с оркестром Эдварда Грига. После
бурных оваций на бис прозвучала \enquote{Баркарола} Франца Шуберта и Ференца Листа}, –
говорится в сообщении.

\ii{22_01_2022.stz.news.lnr.lug_info.1.koncert_muzykant_iz_moskvy_filarmonia.pic.2}

Музыкант из РФ отметил, что Концерт для фортепиано с оркестром Грига он
исполнил в Луганске впервые.

\enquote{Сегодня у меня состоялась премьера. С симфоническим оркестром мне работалось
легко, приятно. Александр Щуров – прекрасный человек, чуткий и талантливый.
Очень порадовал теплый прием слушателей. Луганская публика прекрасно
разбирается в тонкостях музыкального искусства}, – сказал Хачикян.

Щуров рассказал, что \enquote{работа с солистами из Российской Федерации – это всегда
большая радость и значимое событие как для музыкантов, слушателей}, так и для
него лично.

\enquote{Высокий профессионализм нашего гостя помог нам с легкостью преодолеть все
трудности и подарить зрителям замечательный концерт}, – отметил дирижер.

В финале программы начальник управления культуры и туризма Министерства
культуры, спорта и молодежи ЛНР Сергей Рожков вручил пианисту из Москвы
благодарственную грамоту и цветы.

Впервые Хачикян выступил для жителей столицы Республики в программе
симфонического оркестра Луганской академической филармонии в ноябре 2019 года.

Константин Хачикян – дипломант и обладатель специального приза от фонда \enquote{Новые
имена} на фестивале имени Нейгауза (Москва, 2013), лауреат Х конкурса имени
Шопена (Эстония, 2014, третья премия), Международного фестиваля классической
музыки и конкурса юных пианистов Astana Piano Passion (Астана, 2014, вторая
премия), победитель конкурса пианистов имени Юдиной (Санкт-Петербург, 2015) и
Московского международного конкурса юных пианистов имени Шопена (первая премия
и специальный приз за лучшее исполнение мазурок, 2016). В 2017 году Константин
Хачикян стал победителем конкурса Moscow piano open и обладателем второй премии
и приза зрительских симпатий на Международном конкурсе имени Шопена в Канберре
(Австралия).
