% vim: keymap=russian-jcukenwin
%%beginhead 
 
%%file 27_06_2022.stz.news.ua.donbas24.1.bilosarajskij_majak_symvol_nadii_bezpeky.txt
%%parent 27_06_2022.stz.news.ua.donbas24.1.bilosarajskij_majak_symvol_nadii_bezpeky
 
%%url 
 
%%author_id 
%%date 
 
%%tags 
%%title 
 
%%endhead 

Ольга Демідко (Маріуполь)
Маріуполь,Україна,Мариуполь,Украина,Mariupol,Ukraine,Білосарайський маяк,Білосарайська коса,date.27_06_2022
27_06_2022.olga_demidko.donbas24.bilosarajskij_majak_symvol_nadii_bezpeky

Білосарайський маяк: символ надії і безпеки (ФОТО)

Історія найстарішого маяка Північного Приазов'я

Кажуть, що маяки вже відходять у минуле, адже зараз знаходять шлях завдяки
супутникам, суперсучасному обладнанню. Але найстаріший маяк Північного
Приазов'я, який бачив різні епохи, війни і революції, на справі довів, що він є
надійнішим за будь-яку надсучасну техніку. Місцева мешканка Ірина Єрьоменко
наголосила, що «цей маяк витримав і останні воєнні події та наразі біля своїх
стін збирає жителів коси, які згадують все, що їм довелося пережити за останні
декілька місяців». Виконуючи свою життєво необхідну функцію, він вже давно став
історично важливим символом, який зберігає пам'ять поколінь, допомагає в
подоланні небезпеки та невтомно освітлює кораблям шлях до берега, дає надію на
швидкий відпочинок і такий потрібний прихисток...

Історія будівництва маяка 

Вже 187 років вогонь маяка вказує кораблям заповітний шлях до берега. Його
історія має свої особливості і варта нашої уваги. Свого часу білосарайському
маяку присвячували розвідки і статті краєзнавці, журналісти та просто небайдужі
маріупольці. Серед них: Р. Саєнко, Л. Кудріна, О. Панков, А.Цяцько.

Будівництво Білосарайського маяка було продиктовано потужним розвитком
зовнішньоекономічних зв’язків. Розширення морської торгівлі Маріуполя,
Таганрога і Ростова-на-Дону на початку XIX століття змусило владу потурбуватися
про безпеку плавання кораблів у Таганрозькій затоці. Перший маяк збудували в
1811 році в районі нинішнього другого спуску до с. Мелекіне, біля основи
Білосарайської коси. Дерев’яний Білосарайський маяк був побудований
таганрозькими купцями на власні кошти. Висота маяка становила 18,5 метри.
Завдяки значній висоті вогню над рівнем моря (53,9 метри) його було видно вночі
на відстані 20 миль (1 миля = 1,852 км). Маяк перебував у віданні Таганрозького
градоначальства, утримувався за рахунок коштів місцевого купецтва. У 1829 році
його передали дирекції Чорноморських маяків. Коли дерев’яна споруда стала
ветхою, було прийнято рішення про будівництво кам’яного маяка безпосередньо на
Білосарайській косі. Після закінчення будівництва дерев’яний маяк був зламаний.
У 1835 році на початку вузької частини Білосарайської коси був споруджений
білокам'яний маяк.

Архітектура

Будували маяк майже сім місяців — з 16 квітня по 6 листопада. Він складався з
восьмигранної вежі і прибудованих до неї житлових приміщень, в яких
розміщувалися доглядач і матроси, що обслуговували і охороняли маяк.
Будівництво обійшлося казні в 50 000 рублів — вельми солідну на ті часи суму.
Для міцності використовувався керченський штучний камінь жовтого кольору.
Керував будівництвом поручик Дрозжін. Такі кам’яні вежі здатні витримувати
натиск штормових вітрів і хвиль будь-якої сили. Це найкращий зразок маякової
архітектури кінця XIX століття. Не може не вразити, що обладнання для маяка
було виготовлене в Парижі. Подвір'я маяка захищав суцільний високий кам’яний
паркан. У дворі по обидва боки воріт знаходилася кухня, де харчувався персонал
маяка, і магазин (склад). При маяку був колодязь із прісною водою. Усередині
маячна вежа мала чотири яруси, які були з'єднані сходами, що вели до «ліхтаря»
— скляної надбудови. Тут знаходилося 14 ламп з рефлекторами. Щовечора доглядач
маяка піднімався сходами нагору, щоб запалити лампи, а вранці — погасити їх.
Найголовнішим обов’язком службовців маяка було забезпечення справного
освітлення.

Незважаючи на те, що вежа маяка мала висоту всього близько 22 метрів, вона
добре проглядалася з моря.

Відкриття і подальша робота маяка

Білосарайський маяк вперше запалив свій вогонь 25 серпня 1836 року (за старим
стилем). Вогонь був постійний, білий, помітний майже на відстані 10 миль. У
1878 році побудували новий кам'яний житловий флігель для двох маякових
служителів. У 1890 році для попередження нещасних випадків під час туманів на
території маяка було побудовано окрему кам'яну будівлю, в якій встановили
паровий канадський свисток для подачі сигналів. Наприкінці XIX століття
обладнання маяка оновили.

Як свідчить «Лоція Чорного та Азовського морів», видана в 1903 році, на
Білосарайському маяку був встановлений апарат з калильною лампою, який працював
на гасі. Так на маяку з'явилася власна електростанція.

Крім того, у кам'яній будівлі встановили паровий канадський свисток, який під
час туману видавав протягом кожної хвилини три послідовних коротких звуки
тривалістю в три секунди, розділених двома проміжками мовчання в п'ять секунд.
За кожним третім звуковим сигналом слідував період мовчання в сорок одну
секунду. Звук свистка було чути при попутному вітрі на відстані трьох з
половиною миль. Така ж сигналізація проводилася і взимку, якщо лід в морі
починав ламатися і відходити від берегів, — для попередження рибалок про
загрозливу їм небезпеку. Поруч з маяком працювала рятувальна станція,
забезпечена санними човнами та іншим рятувальним обладнанням.

На південному краю Білосарайської коси встановили знак чорного кольору, що
складається з щогли з вісьмома підпорами і двома горизонтальними рейками. Його
висота — 60 футів. Служив він для позначення краю коси. Протягом двадцятого
століття обладнання маяка неодноразово оновлювалося. Проте скільки б у
подальшому не відбувалося технічних удосконалень, зовнішній вигляд об'єкту так
і залишився незмінним до наших днів. Білосарайський маяк привертав увагу
багатьох відомих людей. Так, його відвідували у різний час Белла Ахмадуліна,
Олег Янковський, Алла Пугачова.

Цікаво, що всередині маяка збережені старовинні круті сходи, двері з клеймом
французькою мовою, що свідчить про те, що вони були виготовлені фірмою Ейфеля у
Парижі в 1882 році. Коли у 2011 році Білосарайський маяк відвідали французькі
кінематографісти з відомої в Європі кінокомпанії «Nofilm», вони були просто
вражені. Іноземцям було приємно дізнатися, що обладнання, встановлене на маяку,
має французьке походження і було виготовлено наприкінці XIX століття в Парижі.
Про це свідчить і табличка, встановлена на маяку. Гості запропонували
керівництву зареєструвати його як надбання ЮНЕСКО, адже він дійсно є надбанням
культурної спадщини людства. Як це не дивно, сьогодні маяк продовжує справно
нести свою службу. Він ніколи не давав збою, адже за весь період діяльності не
зафіксовано жодного випадку позаштатної ситуації, завжди вчасно подавав і
продовжує подавати світловий сигнал.

ФОТО: bezviz.co.ua та mistomariupol.com.ua
