% vim: keymap=russian-jcukenwin
%%beginhead 
 
%%file 23_11_2020.news.ua.strana.1.covid_suicides_ukr.obstanovka_v_bolnicah
%%parent 23_11_2020.news.ua.strana.1.covid_suicides_ukr
 
%%url 
 
%%author 
%%author_id 
%%author_url 
 
%%tags 
%%title 
 
%%endhead 

\subsubsection{Обстановка в больницах}
\label{sec:23_11_2020.news.ua.strana.1.covid_suicides_ukr.obstanovka_v_bolnicah}

Причины, которые побуждают ковид-больных совершать самоубийств,
неизвестны. Их может быть несколько: тяжелое течение болезни, страх перед
опасным вирусом, нестабильное психическое состояние и другое. 

Впрочем, даже людям с крепкими нервами сложно пребывать в украинских
инфекционках. 

Отделения сейчас заполнены плотно. В соцсетях публикуют снимки, на которых
койки размещают даже в коридорах. В большинстве больниц капитальный ремонт
делался очень давно или никогда. В некоторых нет душевых кабин. Туалеты
находятся в ужасном состоянии. Местами даже отсутствует свет - для того,
чтобы подсветить капельницу, включают фонарик на мобильном телефоне
(подробнее о том, что представляют собой ковидные больницы - в нашем
материале\Furl{https://strana.ua/articles/analysis/298581-pochemu-kovidnye-bolnitsy-stali-adom-i-dlja-patsientov-i-dlja-vrachej.html}). 

Украинка, которая лечилась в одной из поликник Киева, рассказала \enquote{Стране}, как
за неделю лечения в стационаре она едва не заболела депрессией.

\enquote{Помимо того, что нет никаких условий, давит сама больничная обстановка.
Все мои соседи по палате были больны серьезно. Один плачет, второй стонет.
Свежий воздух не подается, дышать нечем - некоторые теряют сознание во
время попыток встать с кровати. Редкие разговоры между пациентами сводятся
к тому, что \enquote{кислорода на всех не хватит}, \enquote{наши легкие поражены}, 
\enquote{мы все умрем}}, - рассказывает киевлянка, которая излечилась от коронавируса. 

Женщина считает: если бы пробыла в инфекционном отделении еще неделю,
оттуда сразу уехала бы в психбольницу. \enquote{Самое сложное, когда кто-то из
отделения умирает. Новость об этом распространяется сразу. В этот момент
все, кто старался бодриться, падают духом}, - говорит киевлянка. 

По ее словам, депрессию нагоняет в том числе и безденежье пациентов: \enquote{Все
понимают, что чем сложнее и дольше будет заболевание, тем больше денег
придется отдать семье. Так прямо по телефону родственникам и говорят, что
не хотят быть обузой. Но кто-то более сильный духом, а кто-то ломается}. 

