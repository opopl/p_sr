% vim: keymap=russian-jcukenwin
%%beginhead 
 
%%file 27_01_2022.stz.news.ua.den.1.kraina_komu_nichogo_vtrachaty
%%parent 27_01_2022
 
%%url https://day.kyiv.ua/uk/blog/polityka/v-krayini-zalyshatsya-ti-komu-nema-shcho-vtrachaty
 
%%author_id news.ua.den,slivko_jevgenija
%%date 
 
%%tags obschestvo,strana,ukraina
%%title В країні залишаться ті, кому нема що втрачати
 
%%endhead 
 
\subsection{В країні залишаться ті, кому нема що втрачати}
\label{sec:27_01_2022.stz.news.ua.den.1.kraina_komu_nichogo_vtrachaty}
 
\Purl{https://day.kyiv.ua/uk/blog/polityka/v-krayini-zalyshatsya-ti-komu-nema-shcho-vtrachaty}
\ifcmt
 author_begin
   author_id news.ua.den,slivko_jevgenija
 author_end
\fi

Бажання політиків розігнати суцільну паніку навколо Російсько-Української війни
цілком передбачуване: хтось очікує нових провладних дивідендів, комусь зручно
«топити» конкурентів, а владі, за будь-яких часів, було простіше маніпулювати
суспільством, яке перебуває у стані перманентного страху.

Закордонні журналісти, які висвітлюють події в Україні, часто привозять додому
«картину маслом» про події в «трохи дивній», корумпованій державі зі збройним
конфліктом на сході, пишучи про країну вічної ностальгії за «совком», але зі
стрімким бажанням до змін.

Станом на сьогодні, майже всі міжнародні ЗМІ та політики озвучили перспективу
повномасштабного нападу на Україну з боку РФ: фейсбук активно готує тривожний
ранець і пророкує київське життя в окопах; влада каже, що приводів для
хвилювань немає, а громадяни знову скуповують гречку і готують до найгіршого
навіть своїх дітей, розповідаючи про те, як треба себе поводити під час
військових зіткнень.

Дуже актуальним у розрізі цих подій є навіть не суцільна паніка українських
політиків і не дозвіл на евакуацію іноземних дипломатів з боку міжнародних
партнерів. Набагато цікавішим питанням є евакуація наших українських політиків,
точніше, їхня самоліквідація.  Справа в тім, що багатьом з них дійсно є що
втрачати, бо у деяких тут кілотонни готівки і всякого дорого барахла: ікон,
золота, килимів, шуб і автівок тощо. Всі ці надбання, як ви розумієте, набагато
більше турбують подібних екземплярів аніж питання ТРО чи спроможності ЗСУ дати
відсіч ворогу.

Очевидно, що найбільш продумані вже точно вивезли це все у більш безпечніші
європейські місця. Адже накручені реальністю наступу абсолютно всі: від тих,
хто впевнений у тому, що над Києвом літатимуть снаряди і тих, хто не хоче
втрачати те, що зароблено «непосильним трудом».

В принципі, про європейське поневіряння українців вже давно ходять не легенди,
а цілком очевидні факти: громадяни обирають для місця навчання найближчу
Польщу, а більш «продвинуті» відправляють своїх нащадків  у затишну Австрію.

А тут, в Україні, де вічна рознарядка і умовні колгоспники, вони лише утримують
бізнес: в жодній Австрії стільки не заробиш, як тут, на простодушних
співвітчизниках.

І от якраз перспектива втрати цього бізнесу штовхає молодняк, що зріс на
батьківських постулатах про «режим», панікувати і всерйоз обмірковувати план
втечі з «окупованої» України. Адже прийдуть «нові» і відберуть же. Відіжмуть,
навіть не глянувши на елітне прізвище.

Врешті, цей масовий маразм чимось навіть корисний, адже в державі залишаться
тільки ті, яким або немає чого втрачати, або ті, які відчувають  внутрішній
обов'язок за свою країну. Очевидно, що останніх набагато менше, але без них нам
ніяк.

Євгенія СЛІВКО, політичний радник

