% vim: keymap=russian-jcukenwin
%%beginhead 
 
%%file 17_12_2021.fb.fb_group.story_kiev_ua.1.kiev_oktjabrskij_dvorec.cmt
%%parent 17_12_2021.fb.fb_group.story_kiev_ua.1.kiev_oktjabrskij_dvorec
 
%%url 
 
%%author_id 
%%date 
 
%%tags 
%%title 
 
%%endhead 
\zzSecCmt


\begin{itemize} % {
\iusr{Margarita Kaminsky}

Thank you. I did not know the history of this palace.

\iusr{Victoria Novikov}

Спасибо, очень интересно. Побывала в нем очень много раз на концертах. Когда
Ленком привёз спектакль « Юнона и Авось» ( чуть ли не 40 лет назад!), то я
смотрела его, по-моему, именно в Октябрьском.

\iusr{Gennady Henry Sergienko}

Дякую за публікацію. Завжди приємно прочитати про перше місце роботи, хоча в
історії палацу є і сумні сторінки...

\iusr{Tatyana Mazerati}
spasibo za interesniy ocherk

\iusr{Chyzhyshyn Oksana}
Дякую, пізнавально.

\iusr{Юрий Стебельский}

Вот и стали прозывать петербуржца Дмитрия Бегичева Дмытром.  Скоро уже и Михаил
Афанасьевич Булгаков станет на селянский манер Мыхайлом Опанасовычэм, а там
глядишь и Джо Байден будет Йосыпом.

\begin{itemize} % {
\iusr{Roman Sytnyk}
\textbf{Yuriy Stebelskiy} А не соромно носити \enquote{селянське} ім'я Юрій (Юрко) замість Георгій (Джордж)? В мене два прадіди-селяни були Юріями.

\iusr{Вера Милашич}
\textbf{Юрий Стебельский} Торговый бренд Михаил Воронин так им и остался, как запатентован, так и есть, только биг борды написаны укр. буквами. Как хотите так и произносите вам никто не мешает. Не создавайте бурю в стакане воды!

\iusr{Darya Chertkova}
\textbf{Yuriy Stebelskiy} так Юрко!
\end{itemize} % }

\iusr{Татьяна Вакуленко}

Дворец моего детства и юности! Ансамбль бального танца \enquote{Ритм} под руководством
Валерия Николаевича Корзинина, Зайцевская \enquote{Горлица} и \enquote{Ветерок}, которым руководил
Дмитрий Семенович Подвысоцкий.

Огромная благодарность замечательным педагогам, которые ввели в мир танца тысячи детей!

А на елках мы выступали с самого детства)

\iusr{Нина Кубанова}
\textbf{Татьяна Вакуленко} СПАРТАК.-спортивная гимнастика. Ну и ...елки

\iusr{Галина Шабалтас}

А я - пела в хоре «Щедрик» при Октябоьском дворце! Мы обожали наш дворец, с его
мраморными лестницами и колоннами, старинными креслами, множеством укромных
местечек, которые мы исследовали в перерывах между репетициями. Однажды нас
послали в костюмерную, которая находилась где-то с тыльной стороны дворца ,мы
спускалиськ куда-то, потом долго поднимались, очутившись в мрачном коридоре, с
рядом дверей, расположенных на небольших расстояниях друг от друга,... «Как в
тюрьме»,- сказал кто-то, стало вдруг страшно, аж мороз по коже... Энергетика
здесь была совсем другая... мы быстро взяли костюмы и убежали оттуда. Спустя много
лет, я узнала о замученных в этом здании людях, годы прошли, а стены все
помнят!

\begin{itemize} % {
\iusr{Татьяна Ховрич}
\textbf{Галина Шабалтас}, 

а я танцевала, сначала, в детском ансамбле \enquote{Ветерок}, затем - в ансамбле
народного танца \enquote{Горлица}. Золотые были времена!

\iusr{Галина Шабалтас}

\textbf{Татьяна Ховрич} 

Лет в пять я думала, что мое призвание - балет! Мама отвела меня в правое крыло
Октябрьского дворца, где находилась студия балета. Ее руководитель, делая всем
замечания во время первой репетиции, сказала, что у меня «дурацкие коленки»! Я
разобиделась, мир лишился «великой» балерины! А уж через много лет я стала петь
в хоре... и не жалею, с ним связано столько чудесных воспоминаний! Так что Левое
крыло дворца - победило!

\end{itemize} % }

\iusr{Ирина Художница}
Чудова історія

\iusr{Larysa Karbysheva}

А мне как то неприятно подчёркивание того, что бомбила советская авиация.... Под
немцами оставаться надо было что ли?

\end{itemize} % }
