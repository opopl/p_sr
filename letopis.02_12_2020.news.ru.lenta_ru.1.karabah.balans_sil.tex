% vim: keymap=russian-jcukenwin
%%beginhead 
 
%%file 02_12_2020.news.ru.lenta_ru.1.karabah.balans_sil
%%parent 02_12_2020.news.ru.lenta_ru.1.karabah
 
%%url 
 
%%author 
%%author_id 
%%author_url 
 
%%tags 
%%title 
 
%%endhead 

\subsubsection{Изменение баланса сил}

Пока стороны конфликта заняты срочными делами, но рано или поздно им придется
вернуться к ключевому вопросу региона — статусу Нагорного Карабаха. Для этого
нужно определиться, в каком формате будут идти дальнейшие переговоры между
Арменией и Азербайджаном, считают эксперты.

Старый формат Минской группы, который действовал со времен подписания первого
протокола о прекращении огня в Карабахе в 1994 году в Бишкеке, более нельзя
считать эффективным. За все 30 лет существования формата протокол так и остался
его единственным достижением. Переговорная группа никак себя не проявила во
время нынешней, уже третьей карабахской войны, подчеркнул Притчин. Более того —
на бесполезность переговоров прямо указали уже в Баку и до начала боев, и
после.

\begin{leftbar}
	\bfseries
Нынешняя эскалация конфликта является одной из причин неэффективности
деятельности Минской группы и минского процесса в целом\par
\em
		Хикмет Гаджиев\par 
		советник президента Азербайджана\par
\end{leftbar}

После без преувеличения оглушительного успеха на поле боя Азербайджан уже не
собирается разговаривать ни с Арменией, ни тем более с руководством Арцаха.
Максимум, с кем готовы наладить диалог представители официального Баку — это
руководители армянской общины Нагорного Карабаха, которые будут представлять
интересы населения. А вот с кем Азербайджан хочет разговаривать и кого требует
включить в переговорный формат, так это Турцию, которая с первых часов боевых
действий активно поддерживала своего закавказского союзника.

\ifcmt
pic https://icdn.lenta.ru/images/2020/12/01/19/20201201191209064/pic_0b6cccd014ddddce778dda4f6cb0a53b.jpg
caption Колонна грузовиков российских миротворцев Фото: Максим Блинов / РИА Новости
\fi

Армения всегда выступала и выступает против любого участия Турции в карабахском
конфликте — непризнание Анкарой геноцида армян начала XX века делает
невозможными полноценные дипломатические контакты между этими странами. Но
Ереван сейчас не может диктовать условия, его позиции ослаблены как поражением,
так и разгоревшимся вслед за ним внутриполитическим кризисом.

\begin{leftbar}
	\bfseries
{\Huge\color{orange}\centering 2660}
погибших
составили военные потери армянской стороны осенью 2020 года. Азербайджан до сих пор не раскрыл свои потери
\end{leftbar}

Премьер Пашинян сообщил о подписании проигрышного для Армении соглашения без
лишнего шума — через соцсети. Он назвал его содержание болезненным для себя и
для всего народа и призвал к сплочению. Народ действительно сплотился, только
не вокруг национального лидера, а против него. В ночь на 10 ноября протестующие
против сдачи территорий штурмовали парламент, избили спикера и разгромили
резиденцию премьера. Беспорядки продолжались и продолжаются до сих пор,
половина правительства отправлена в отставку, но людей это не успокоило.

\ifcmt
pic https://icdn.lenta.ru/images/2020/12/01/19/20201201190750881/pic_08447f953418f5e9b38d1456ac149314.jpg
caption Погромщики в здании правительства Армении Фото: Илья Питалев / РИА Новости
\fi

Армяне воспринимают трехсторонние соглашения ни много ни мало как предательство
родины. Многие также оказались возмущены тем, что в ходе войны власти скрывали
истинное — и весьма плачевное — положение дел, рапортуя о том, что ситуация на
фронте под контролем. Народное недовольство подогревают и видные политические
деятели, например, сам президент Армении, экс-глава Совбеза. Пашинян считает,
что чист перед своим народом. Он утверждает, что на подписании заявления о
прекращении огня настаивали военные, и альтернативы принятому решению не было,
иначе бы Азербайджан захватил весь Карабах. Ему вторит и руководство Арцаха.


\begin{leftbar}
	\bfseries
Генштаб мне докладывал, что с ресурсами у нас проблемы. И политическое
руководство Арцаха разделяло эту позицию. Подписанным документом мы получили
то, что не смогли бы удержать в случае продолжения. Степанакерт остался, по
сути, беззащитным\par
\em
Никол Пашинян\par
премьер-министр Армении 
\end{leftbar}

Пока ситуация в Армении остается неопределенной. Однако Малашенко полагает, что
отставка премьера была бы ошибкой. Пашинян является «оптимальным вариантом» для
армянского общества сегодня, говорит эксперт: по его мнению, глава
правительства адекватно вел себя в сложившейся ситуации и пошел на такой
поступок, как подписание заявления о перемирии, будучи психологически готовым к
тому, что его станут оплевывать. «Бегать по улицам в армянской ситуации —
делать только хуже», — резюмировал политолог.

Однако и вопрос включения Турции в переговоры по Карабаху как полноправного
участника остается спорным. Армения утверждала, что турецкие военные напрямую
участвуют в войне и поставляют на фронт боевиков с Ближнего Востока. Этот факт
позже подтвердили Россия, Франция и США. По данным СМИ, турецкая группировка
войск из 600 человек действительно находилась в Азербайджане с летних
совместных учений и сыграла ключевую роль в координации воздушной операции в
Карабахе.

\ifcmt
pic https://icdn.lenta.ru/images/2020/12/01/18/20201201180416027/pic_ef090984e1b7952fa8e6835e0d1247af.jpg
caption Фото: Валерий Мельников / РИА Новости
\fi

Теперь Алиев уже не стесняется поддержки своего коллеги Реджепа Тайипа
Эрдогана. Он охотно рассказывает, что турецкий парламент официально разрешил
отправить военных в Азербайджан для участия в миротворческой миссии. В Кремле
его постоянно одергивают: миротворческой миссией занимаются только российские
военные, турецкие к этой работе не допущены.

\begin{leftbar}
	\bfseries
{\Huge\centering\color{orange} 1960}\par
российских миротворцев будут находиться в Нагорном Карабахе в течение пяти лет
\end{leftbar}

Однако представители Анкары вместе с московскими коллегами действительно
принимают участие в работе центра по контролю за соблюдением перемирия. По
словам Станислава Притчина, турецкие военные действительно уже присутствуют в
регионе, но ни их количество, ни место дислокации непонятно. Пока же у Анкары
достаточно ограниченное поле для маневра, поскольку она не является участником
соглашения о перемирии. А кроме того, хотя подходы Баку и Анкары к карабахской
проблеме схожи, их долгосрочные цели различаются, считает эксперт: для Турции
важно поддерживать реноме глобального игрока на Кавказе, для Азербайджана —
демонстрировать «суверенность и самодостаточность».

По мнению Притчина, в данный момент первоочередными вопросами все же являются
проблемы гуманитарной сферы: в регионе необходимо восстановить экономические,
правовые и межкультурные контакты. А также обе стороны должны в короткие сроки
выстроить более-менее устойчивую систему взаимодействия армян и азербайджанцев.
Поскольку тесное соседство победителей и проигравших слишком взрывоопасно, и,
если не снизить напряженность на местах, до решения геополитических вопросов
дело может так и не дойти еще много лет.
