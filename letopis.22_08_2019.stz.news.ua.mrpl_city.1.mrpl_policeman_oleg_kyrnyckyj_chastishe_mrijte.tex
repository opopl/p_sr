% vim: keymap=russian-jcukenwin
%%beginhead 
 
%%file 22_08_2019.stz.news.ua.mrpl_city.1.mrpl_policeman_oleg_kyrnyckyj_chastishe_mrijte
%%parent 22_08_2019
 
%%url https://mrpl.city/blogs/view/mariupolskij-politsejskij-genij-oleg-kirnitskij-chastishe-mrijte
 
%%author_id demidko_olga.mariupol,news.ua.mrpl_city
%%date 
 
%%tags 
%%title Маріупольський поліцейський Олег Кирницький: "Частіше мрійте!"
 
%%endhead 
 
\subsection{Маріупольський поліцейський Олег Кирницький: \enquote{Частіше мрійте!}}
\label{sec:22_08_2019.stz.news.ua.mrpl_city.1.mrpl_policeman_oleg_kyrnyckyj_chastishe_mrijte}
 
\Purl{https://mrpl.city/blogs/view/mariupolskij-politsejskij-genij-oleg-kirnitskij-chastishe-mrijte}
\ifcmt
 author_begin
   author_id demidko_olga.mariupol,news.ua.mrpl_city
 author_end
\fi

\ii{22_08_2019.stz.news.ua.mrpl_city.1.mrpl_policeman_oleg_kyrnyckyj_chastishe_mrijte.pic.1}

Продовжуючи розглядати життєвий і творчий шлях непересічних маріупольців,
пропоную познайомитися ближче з молодим чоловіком, який вражає своїм
інтелектом, глибоким розумінням глобальних проблем та водночас легкістю і
простотою в спілкуванні. Колеги вважають його генієм, батьки – опорою, а він
зовсім незухвалий, скромний, готовий і надалі займатися самовдосконаленням.
Мова піде про інспектора відділу комунікації Головного управління національної
поліції – \textbf{Олега Володимировича Кирницького}.

\ii{22_08_2019.stz.news.ua.mrpl_city.1.mrpl_policeman_oleg_kyrnyckyj_chastishe_mrijte.pic.2}

Народився Олег у Харцизьку Донецької області у лютому 1991 року в родині
вчителів. Мати – вчителька молодших класів, батько – вчитель історії. З
дитинства хлопець розумів - наскільки непроста робота у батьків, але,
незважаючи на це, хотів стати викладачем і працювати в технікумі або вузі.
Батьки завжди розуміли сина, його вибір та підтримували у всьому. У 2013 році
закінчив історичний факультет Донецького національного університету імені
Василя Стуса з відзнакою. Отримав спеціальність \enquote{Політологія}. Обрав саме цю
спеціальність, адже завжди любив займатися аналі\hyp{}тикою, прораховувати можливі
варіанти та аспекти подій. Після уні\hyp{}верситету вирішив продовжити навчання в
аспірантурі. Пройшовши конкурс, вступив на денне відділення на філософію. Так
хлопець став молодшим співробітником кафедри філософії. Але події 2014 року не
дали можливості продовжувати наукову діяльність у рідному Донецьку. Олег
пам'ятає, коли 26 травня 2016 року ворожі війська взяли під контроль аеропорт,
пам'ятає понівечені тіла та відчай, з яким було впоратися важко. В університеті
не знали, що робити, люди переживали за особисту долю, не хотіли втрачати
житло. Складно уявити, що довелося пережити молодому аспіранту, життя якого
раптом розділилося на \enquote{до} і \enquote{після}...

\ii{insert.read_also.demidko.minjajlo}

Олегу з батьками довелося виїхати до Святогірська, де колись мешкала його
бабуся. Там він самотужки добудував будинок. Звісно, що під час війни було
важко. Нашого героя рятувала його здібність працювати своїми руками. Загалом,
Олег – перфекціоніст, в усьому прагне досягти досконалості, інакше не може. Ця
його риса інколи приносить чимало труднощів, бо зробити абияк вважає
неприпустимим. Вміє працювати з деревом, каменем, професійним будівельним
інструментом. Може і будинок збудувати, і дах зробити, адже будь-які роботи
йому під силу.

\ii{22_08_2019.stz.news.ua.mrpl_city.1.mrpl_policeman_oleg_kyrnyckyj_chastishe_mrijte.pic.3}

Прийнявши для себе рішення вступити на службу у правоохоронних органах
Донеччини, Олег розпочав підготовку у маріупольському центрі первинної
підготовки \enquote{Академія поліції}. Вже влітку 2017 року колишній аспірант
почав працювати в якості правоохоронця на території Слов'янського відділу
поліції.  Водночас, ще перебуваючи у лавах слухачів курсів первинної підготовки
поліцейських, 27 червня 2017 року у Запорізькому національному уні\hyp{}верситеті
Олег захистив дисертацію на тему \enquote{Віртуальний вимір соціального
конструювання}.

\ii{22_08_2019.stz.news.ua.mrpl_city.1.mrpl_policeman_oleg_kyrnyckyj_chastishe_mrijte.pic.4}

У своїй роботі науковець осмислював, яким чином та за посередництвом чого в
розумінні людини оформлюються і закладаються шаблонні рішення, забобони,
переконання, упередження, на основі яких проводиться підміна соціальної
реальності в інформаційних війнах та конфліктах. Йому вдалося довести, що у ХХІ
столітті непомітні на перший погляд матеріальні повсякденні предмети (від
смартфонів до ранкової чашки кави) все більшим чином визначають подобу нашої
соціальної реальності, конкуруючи вже із державними структурами. Захист
відбувся успішно і Олег, перебуваючи на службі, отримав науковий ступінь
кандидата філософських наук, доктора філософії.

\textbf{Читайте також:} \emph{Яке воно, курсантське життя: День відкритих дверей у маріупольській \enquote{Академії поліції}}%
\footnote{Яке воно, курсантське життя: День відкритих дверей у маріупольській \enquote{Академії поліції}, Донецький юридичний інститут, %
mrpl.city, 23.12.2018, \par%
\url{https://mrpl.city/blogs/view/yake-vono-kursantske-zhittya-den-vidkritih-dverej-u-mariupolskij-akademii-politsii}
}

Вирішивши працювати у поліції, він пройшов 7 етапів складного відбору. 20
березня 2017 року розпочав проходити курси в Маріуполі. Отримав фах –
\enquote{Дільничний офіцер поліції}. Спочатку працював дільничним офіцером поліції
Слов'янська. Основним напрямом роботи молодого співробітника була робота у
складі слідчо-оперативної групи, у складі якої він ніс службу на той момент в
зоні проведення АТО. Працюючи у Слов'янському відділі поліції, паралельно із
службовою діяльністю, правоохоронець навчався на заочному відділенні Донецького
юридичного інституту, отримуючи вже третю необхідну для повсякденної роботи
наукову спеціальність \enquote{магістр права}. З травня 2019 року працює інспектором
відділу комунікації Головного управління національної поліції. Керівництво ГУНП
вирішило перевести чоловіка, дізнавшись, що він має вчений ступінь. Роботою
Олег задоволений, адже працює за спеціальністю й, головне, має можливість
втілювати ідеї на практиці. Зараз на посаді чоловік разом із усією командою
відділу комунікації займається формуванням нових комунікаційних стратегій для
підвищення рівня довіри населення до поліції. 

\ii{22_08_2019.stz.news.ua.mrpl_city.1.mrpl_policeman_oleg_kyrnyckyj_chastishe_mrijte.pic.5}

Маріуполь дуже подобається, по духу \enquote{авантюризму} нагадує довоєнний Донецьк.
Дуже багато прогресивної молоді. На думку чоловіка, у місті наявні саме ті
інтелектуальні ресурси та промислові можливості, які обов'язково допоможуть
рухатися вперед. Любить пляж \enquote{Піщанку}, де грає у волейбол та Приморський парк,
в якому часто бігає.

Надихає Олега філософія, адже ця форма пізнання світу дає можливість бути
універсальним. Після роботи часто перечитує філософські праці, які надають йому
творчу наснагу та певну свіжість ідей. Деякі праці філософів, на його думку,
випереджають час (наприклад, \enquote{Буття і час} Мартіна Гайдеггера), але це не
провина автора, а тимчасова недосконалість суспільства, яке не готове зрозуміти
прогресивних авторів.

\ii{22_08_2019.stz.news.ua.mrpl_city.1.mrpl_policeman_oleg_kyrnyckyj_chastishe_mrijte.pic.6}

У майбутньому хотів би продовжити займатися, окрім робочих моментів, й науковою
діяльністю та написати ще одну фундаментальну працю, яка б продовжила та
розширила глибину дисертаційних ідей. Головне для науковця, щоб його роботи
було цікаво читати і вони приносили суспільну користь. Для нього стало б
особистою трагедією, якщо б праця, написана його рукою, була нецікавою.

\ii{22_08_2019.stz.news.ua.mrpl_city.1.mrpl_policeman_oleg_kyrnyckyj_chastishe_mrijte.pic.7}

Наразі Олег Володимирович працює над проектом, присвяченим налагодженню
ефективної взаємодії між суспільством і поліцією. Думаю, що незабаром проект
буде представлений громадськості.

\textbf{Читайте також:} \emph{Захисник України Руслан Пустовойт - герой з непростою долею}%
\footnote{Захисник України Руслан Пустовойт - герой з непростою долею, Вячеслав Аброськин, mrpl.city, 31.05.2019, \par%
\url{https://mrpl.city/blogs/view/zahisnik-ukraini-ruslan-pustovojtgeroj-z-neprostoyu-doleyu}
}

\ii{22_08_2019.stz.news.ua.mrpl_city.1.mrpl_policeman_oleg_kyrnyckyj_chastishe_mrijte.pic.8}

\textbf{Улюблена книга:} \enquote{Презентація себе у повсякденному житті} Ервінга Ґофмана, \enquote{Буття і час} Мартіна Гайдеггера, \enquote{Правила промови}, \enquote{Наглядати і карати}, \enquote{Історія сексуальності} Мішеля Фуко.

\textbf{Улюблений фільм:} \enquote{Одного разу на Дикому Заході} (1968 рік), \enquote{Апокаліпсис сьогодні} (2001 рік), \enquote{На межі} (1997 рік).

\textbf{Хобі:} Знає досконало вогнепальну зброю від XIX століття до сьогодення. Займався дзюдо. Знається в годинникових механізмах. Може зібрати годинник сам. Цікавиться локальними військовими конфліктами та тактикою дій спецпідрозділів.

\textbf{Порада маріупольцям:} 

\begin{quote}
\em\enquote{Частіше мрійте! Будь-яка думка завжди знаходить вихід у світ матеріального, тому мріяти слід частіше}.
\end{quote}
