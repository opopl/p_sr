% vim: keymap=russian-jcukenwin
%%beginhead 
 
%%file 24_12_2014.fb.bilchenko_evgenia.1.imidzh_nauka
%%parent 24_12_2014
 
%%url https://www.facebook.com/yevzhik/posts/764650216903424
 
%%author Бильченко, Евгения
%%author_id bilchenko_evgenia
%%author_url 
 
%%tags bilchenko_evgenia,nauka
%%title БЖ. "НАЗАД ДО ЛІТЕРАТУРНОГО ШИНКА, КРЕЗІ ДОКТОР!"
 
%%endhead 
 
\subsection{БЖ. \enquote{НАЗАД ДО ЛІТЕРАТУРНОГО ШИНКА, КРЕЗІ ДОКТОР!}}
\label{sec:24_12_2014.fb.bilchenko_evgenia.1.imidzh_nauka}
\Purl{https://www.facebook.com/yevzhik/posts/764650216903424}
\ifcmt
 author_begin
   author_id bilchenko_evgenia
 author_end
\fi

БЖ. "НАЗАД ДО ЛІТЕРАТУРНОГО ШИНКА, КРЕЗІ ДОКТОР!", або 
До питання про мій академічний імідж )))
ІНСТИТУТ ФІЛОСОФІЇ імені Г.С. Сковороди НАНУ:
"Наша колега Евгения Бильченко зі своєю статтею "РЕЛІГІЙНА СЕМАНТИКА ПОВЕДІНКИ СУЧАСНОЇ БОГЕМИ: МЕТОДОЛОГІЧНІ КОНТЕКСТИ ДІАЛОГІЧНОЇ КУЛЬТУРОЛОГІЇ ТА КУЛЬТУРОЛОГІЇ РЕЛІГІЇ"
Стаття присвячена культурологічному аналізу імпліцитних сакральних смислів у символічній поведінці представників богемних кіл ХХ-ХХІ ст. (авангарду та постмодерну). Автор вводить нову методологію релігійної інтерпретації світських поведінкових жестів як текстів – синтез діалогічної культурології та культурології релігії. Ключові позиції розвідки: типологія (з опорою на З. Баумана) культурних суб’єктів на «паломника» (класичний тип) і «бродягу» (некласичний тип) − та філософське осмислення феномену єдності трагічного та іронічного (с опорою на М.О. Бердяєва та У. Еко) у поведінці «бродяги».
Не слушайте нашего смеха, 
слушайте ту боль, которая за ним. 
Не верьте никому из нас, верьте тому, что за нами
(Олександр Блок)
Статтю читайте в текстовому документі: \url{http://vk.com/id91823257?w=wall91823257_16767%2Fall} 
І якщо не важко - репост! )))

\url{https://vk.com/institutumphilosophiaekioviensis}

\ifcmt
  pic https://scontent-lga3-1.xx.fbcdn.net/v/t1.18169-9/10606143_766526736755730_4373523810047671755_n.jpg?_nc_cat=102&ccb=1-3&_nc_sid=9267fe&_nc_ohc=_cPisjmkpskAX804P_O&_nc_ht=scontent-lga3-1.xx&oh=649d7471d34f6303f0ba864c0cd3db43&oe=60E7C770
  width 0.4
\fi

\emph{Евгения Бильченко}
Ссылка на странную дискуссию с Харьковской Академией культуры:\par
\url{https://www.facebook.com/yevzhik/posts/764645840237195?comment_id=764755380226241&notif_t=like}
