% vim: keymap=russian-jcukenwin
%%beginhead 
 
%%file 15_04_2021.fb.zagrebelnyj_igor.1.papa_svoboda_osuzhdenie_chelovek
%%parent 15_04_2021
 
%%url https://www.facebook.com/permalink.php?story_fbid=2984283811896252&id=100009439885823
 
%%author 
%%author_id 
%%author_url 
 
%%tags 
%%title 
 
%%endhead 

\subsection{Новизна думок - Лев XIII - Церква - Свобода, Рівність, Братерство}
\label{sec:15_04_2021.fb.zagrebelnyj_igor.1.papa_svoboda_osuzhdenie_chelovek}
\Purl{https://www.facebook.com/permalink.php?story_fbid=2984283811896252&id=100009439885823}

Формулювання з однієї з енциклік Лева XIII: \enquote{...коли фатальна новизна думок
захопила багатьох}. Причому йшлося про засудження реалій не сучасного Левові
часу, а далекого XVI ст.; \enquote{починаючи з тієї епохи} - пояснює Папа - народ
\enquote{прагнув свободи, більшої за справедливу}. 

І таких засуджень \enquote{новизни думок} чи, наприклад, \enquote{пристрасті до
новизни} у документах соціального вчення Церкви XIX ст. дуже багато.

Зрештою, відомий \enquote{Syllabus} чітко засуджував думку, що 
\enquote{Римський Першосвященник може і має примиритися з прогресом, лібералізмом і сучасною цивілізацією}.

Ех, не вміли у XIX ст. папських документів писати. Де тут про \enquote{свободу,
рівність, братерство}? Де про \enquote{відкрите суспільство}? Де заклики до
еліт, які дотримуються радикально ворожих християнству поглядів, створити
\enquote{механізми глобального управління}?
