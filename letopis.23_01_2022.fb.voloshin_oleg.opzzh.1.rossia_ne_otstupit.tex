% vim: keymap=russian-jcukenwin
%%beginhead 
 
%%file 23_01_2022.fb.voloshin_oleg.opzzh.1.rossia_ne_otstupit
%%parent 23_01_2022
 
%%url https://www.facebook.com/oleg.voloshin.7165/posts/5133096063390080
 
%%author_id voloshin_oleg.opzzh
%%date 
 
%%tags rossia,ukraina
%%title Россия не отступит, пока ее голос вновь не будет звучать при обсуждении судеб Украины
 
%%endhead 
 
\subsection{Россия не отступит, пока ее голос вновь не будет звучать при обсуждении судеб Украины}
\label{sec:23_01_2022.fb.voloshin_oleg.opzzh.1.rossia_ne_otstupit}
 
\Purl{https://www.facebook.com/oleg.voloshin.7165/posts/5133096063390080}
\ifcmt
 author_begin
   author_id voloshin_oleg.opzzh
 author_end
\fi

Более всего в наблюдаемых околополитических дискуссиях меня меня поражает
самообман на предмет того, что все стихнет и общие условия и правила останутся
прежними. От этого столь многие политические игроки живут устаревшими целями и
действуют неадекватными методами. 

Причём ведущие мировые СМИ, публикации западных и российских экспертов,
элементарная логика и смелость воспринять реальность должны были бы давно
убедить, что статус-кво недостижим. Кстати, неплохо бы напомнить, что полностью
крылатое выражение статус кво звучит как status quo ante bellum - положение
бывшее до войны. А война в самом разгаре, пусть пока и не в горячей форме.

Россия не отступит, пока ее голос вновь не будет звучать при обсуждении судеб
Украины. Вопрос только в том, каким методом этот результат будет достигнут. 

Потому совершенно смешны рассуждения «об удобном для Зеленского кандидате во
втором туре». Даже смешнее, чем аналогичные рассуждения в 2013-м о роли
Тягнибока для Януковича. Только тогда за право голос решительно боролся Запад.

\ii{23_01_2022.fb.voloshin_oleg.opzzh.1.rossia_ne_otstupit.cmt}
