% vim: keymap=russian-jcukenwin
%%beginhead 
 
%%file 03_12_2021.fb.tolkachev_aleksej.1.299_rokiv_skovoroda
%%parent 03_12_2021
 
%%url https://www.facebook.com/oleksiy.tolkachov/posts/4878728702146803
 
%%author_id tolkachev_aleksej
%%date 
 
%%tags birthday,obschestvo,skovoroda_grigorii,ukraina
%%title 299 років - Григорій Савич Сковорода
 
%%endhead 
 
\subsection{299 років - Григорій Савич Сковорода}
\label{sec:03_12_2021.fb.tolkachev_aleksej.1.299_rokiv_skovoroda}
 
\Purl{https://www.facebook.com/oleksiy.tolkachov/posts/4878728702146803}
\ifcmt
 author_begin
   author_id tolkachev_aleksej
 author_end
\fi

Сьогодні велике свято - 299 років тому народився український геній - Григорій
Савич Сковорода! Чим він може бути нам цікавий?

Ділюся своїми думками у свіжому випуску: 

\href{https://youtu.be/3BK6xlQucXw}{%
300-летие Григория Сковороды — как будем праздновать?, %
youtube, 23.11.2021%
}

\begin{multicols}{2}
Приближается годовщина - 300 лет со дня рождения гениального Григория
Сковороды. И мы создаём общественный оргкомитет празднования великой даты
\enquote{Сковорода 300} Украинцам может показаться, что эта тема скучная и устаревшая.
Но дело в том, что этот человек родился не в то время! Складывается ощущение,
что он опередил время на несколько сотен лет. Григорий Сковорода нёс те идеи,
которыми мы дорожим сейчас. Он воспринимал жизнь через свою особую призму -
философию сердца. Он стал первым украинским хипстером. Он был первым веганом.
На его философском наследии построен Евросоюз. На идеях Григория Сковороды
будет построена цивилизация будущего. Какие уникальные идеи были у Григория
Сковороды? Чем он отличался от своих  современников? Ответы – в новом блоге
Алексея Толкачёва

Алексей Толкачёв в социальных сетях:
\url{https://www.facebook.com/oleksiy.tolkachov}

Книги Алексея Толкачёва \url{https://ridero.ru/author/tolkachyov_aleksei_fyp7l}

\end{multicols}

Його ідеї настільки випередили свій час, що лише зараз можуть бути осмислені на
новому цивілізаційному рівні! Схований потенціал філософського доробку
Сковороди - практично безмежний!

\ifcmt
  ig https://scontent-frt3-1.xx.fbcdn.net/v/t39.30808-6/263756169_4878670368819303_4956801075511376596_n.jpg?_nc_cat=104&ccb=1-5&_nc_sid=730e14&_nc_ohc=DrXSBQGfIOcAX8qhndH&_nc_ht=scontent-frt3-1.xx&oh=cbbdea96e217db7231c9e4136557cc97&oe=61B10129
  @width 0.4
  %@wrap \parpic[r]
  @wrap \InsertBoxR{0}
\fi

Великий ювілей наступного року - 300-ліття від народження Сковороди - це
прекрасна нагода для популяризації Савича на лише на загальнонаціональному, але
й міжнародному рівні!

Саме з цією метою ми з поважним товариством створили Комітет \enquote{Сковорода-300} -
\enquote{Всеукраїнський громадський комітет з відзначення 300-ї річниці Григорія
Сковороди}.

До складу Комітету \enquote{Сковорода 300} увійшли:

\begin{itemize}
  \item – \href{https://www.facebook.com/ian.valietov}{Ян Валетов} (м. Дніпро), письменник;
  \item – \href{https://www.facebook.com/profile.php?id=100020629927113}{Семён Глузман}, психіатр, колишній політв’язень радянських таборів;
  \item – \href{https://www.facebook.com/profile.php?id=100001035721714}{Oleksandra Gumenna}, в.о. Президента Національний університет \enquote{Києво-Могилянська академія};
  \item – \href{https://www.facebook.com/lyubko.deresh}{Любко Дереш}, письменник;
  \item – \href{https://www.facebook.com/oles.doniy}{Олександр Доній} (голова Оргкомітету), ідеолог;
  \item – \href{https://www.facebook.com/doroshenko.kostyantyn}{Doroshenko Kostiantyn}, куратор, культуролог;
  \item – \href{https://www.facebook.com/gennadiy.druzenko}{Gennadiy Druzenko}, правник, очільник Центр конституційного моделювання / Center for Constitutional Design, співзасновник та голова Наглядової Ради Перший добровольчий мобільний шпиталь ім. Миколи Пирогова; 
  \item – \href{https://www.facebook.com/taras.kompanichenko}{Taras Kompanichenko}, музика, керівник гурту - 
  \href{https://www.facebook.com/Хорея-Козацька-Chorea-Kozacky-173908926023379/}{Хорея Козацька - Chorea Kozacky}; 
  \item – \href{https://www.facebook.com/piddubniy.igor}{Игорь Поддубный} (м. Харків), журналіст, режисер-документаліст, голова ГО «Спілка сприяння пошуку та повернення історичних та культурних цінностей України»; 
  \item – \href{https://www.facebook.com/oleksiy.tolkachov}{Олексій Толкачов}, візіонер-футуролог, письменник, просвітник, громадський діяч.
\end{itemize}

Приєднуйтесь! 

Якщо у вас є ідеї щодо відзначення ювілею великого Сковороди у наступному році,
то пропонуйте, діліться думками! Ми охоче об'єднаємо з вами зусилля!

З Днем народження, Савич!
\ii{03_12_2021.fb.tolkachev_aleksej.1.299_rokiv_skovoroda.cmt}
