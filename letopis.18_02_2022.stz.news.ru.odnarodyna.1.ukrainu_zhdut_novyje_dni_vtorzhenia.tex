% vim: keymap=russian-jcukenwin
%%beginhead 
 
%%file 18_02_2022.stz.news.ru.odnarodyna.1.ukrainu_zhdut_novyje_dni_vtorzhenia
%%parent 18_02_2022
 
%%url https://odnarodyna.org/article/ukrainu-zhdut-novye-dni-vtorzheniya
 
%%author_id katerinchuk_roman
%%date 
 
%%tags napadenie,rossia,ugroza,ukraina
%%title Украину ждут новые «дни вторжения»
 
%%endhead 
 
\subsection{Украину ждут новые «дни вторжения»}
\label{sec:18_02_2022.stz.news.ru.odnarodyna.1.ukrainu_zhdut_novyje_dni_vtorzhenia}
 
\Purl{https://odnarodyna.org/article/ukrainu-zhdut-novye-dni-vtorzheniya}
\ifcmt
 author_begin
   author_id katerinchuk_roman
 author_end
\fi

Если кто-то подумал, что после так и не случившегося «вторжения» российской
армии на Украину истерика вокруг этой страны пойдёт на спад, то он глубоко
ошибается.

Наутро после «дня вторжения», который указом президента Зеленского превратили в
«день единства», коллективный Запад с плохо скрываемым азартом продолжил
обещать всё те же мероприятия, только чуть позже.

Американский журнал Politico объявил новую дату «вторжения» российской армии на
Украину – 20 февраля. Оказывается, дата 16 февраля называлась «экспертами»
специально, чтобы отвлечь внимание.

\begin{zznagolos} 
«Важно, что высокопоставленные помощники Байдена никогда
публично не заявляли, что 16 февраля будет определённым днём вторжения, ведь
нападение может произойти в любое время и, возможно, даже до завершения
Олимпиады. Время после 20 февраля всегда было более важным. Есть ещё одна
причина, по которой 20 февраля имеет большое значение: в этот день завершается
Мюнхенская конференция по безопасности»,	
\end{zznagolos}

– уверяет Politico.

Да ещё и военные учения на территории Белоруссии завершаются 20 февраля, так
что явно 20 февраля выглядит лучше, чем 16-е, отмечает издание.

\ii{18_02_2022.stz.news.ru.odnarodyna.1.ukrainu_zhdut_novyje_dni_vtorzhenia.pic.1}

Нет, это не шутка. Но гражданам Украины не совсем понятно другое – что пан
Зеленский запланирует на «день вторжения-2», ведь празднование «дня единения»
дважды в неделю будет выглядеть немного странно.

Британия со своей стороны тоже поспешила «успокоить» – Украине предстоит ещё
несколько месяцев прожить в таком состоянии, ожидая наступления каждый день или
два раза в неделю, что тоже многовато. С таким заявлением выступила министр
иностранных дел Великобритании Лиз Трасс, а накануне она жизнерадостно
объявила, что российские войска займут Киев «очень быстро» и всё это «весьма
вероятно».

Кстати, «успокаивала» Лиз Трасс прямо в Киеве, куда она прибыла с рабочим
визитом.

\ii{18_02_2022.stz.news.ru.odnarodyna.1.ukrainu_zhdut_novyje_dni_vtorzhenia.pic.2}

Не хотят останавливаться и в Польше – глава Бюро национальной безопасности
Павел Солох объявил: сегодня вероятность российского нападения на Украину
намного выше, чем вчера. Правда теперь угроза существует в самых разных
вариантах.

\begin{zznagolos}
«Сообщения последних нескольких часов вызывают тревогу, многое говорит о том,
что вероятность нападения России на Украину сегодня выше, чем вчера. Всё
происходящее сегодня у границы с Украиной – это не деэскалация, не отвод войск
от границы, это самая настоящая ротация»,	
\end{zznagolos}

– заявил Солох, ссылаясь на неких «специалистов по современным войнам», которые
не сомневаются – все действия России являются маневром, предшествующим
наступлению. И то, что военная техника российской армии возвращается в места
постоянной дислокации – это тоже подготовка к наступлению, «эксперты» не могут
ошибаться.

\ii{18_02_2022.stz.news.ru.odnarodyna.1.ukrainu_zhdut_novyje_dni_vtorzhenia.pic.3}

Если бы в Киеве объявили конкурс на самый лучший прогноз по теме «вторжения»,
то первый приз, несомненно, достался бы Пентагону – там сидят настоящие мастера
своего дела.

Глава Пентагона Ллойд Остин трагически сообщил, что «в Соединённых Штатах
хорошо видят, как российские войска медленно приближаются к украинской границе,
пополняя по дороге запасы крови». Получился готовый сценарий фильма ужасов для
Голливуда. Как именно российские войска пополняют запасы крови и где они её
берут, осталось тайной, но звучит впечатляюще.

\begin{zznagolos}
«Я не так давно сам был солдатом и знаю, что такого рода действия вы не будете
предпринимать без причины... Поэтому мы вместе с союзниками будет сохранять
бдительность»,	
\end{zznagolos}

– успокоил глава Минобороны США, выступая на пресс-конференции в штаб-квартире
НАТО в  Брюсселе.

\ii{18_02_2022.stz.news.ru.odnarodyna.1.ukrainu_zhdut_novyje_dni_vtorzhenia.pic.4}

И так во всех без исключения западных СМИ – какую газету не откроешь, на какой
сайт не зайдёшь, везде только Kremlin, Putin и Russian troops on Ukraine's
borders (Кремль, Путин и российские войска на украинской границе). Других
новостей нет.

Между тем курс гривны, стабилизировавшись в «день вторжения», снова пошёл вниз,
экономика Украины агонизирует, осталось сделать контрольный выстрел. Или два. А
чтобы исправить ситуацию, понадобятся десятилетия. Такой вот подарочек от США и
Британии.

Глава пропрезидентской парламентской фракции «Слуга народа» Давид Арахамия уже
открыто обвинил ряд западных СМИ в публикации фейков.

\begin{zznagolos}
«Пройдёт время, и мы сделаем ретроспективный анализ, каким образом большие и
очень известные СМИ начали распространять информацию хуже, чем Скабеева с
Соловьёвым. Откровенные фейки в CNN, Bloomberg, WSJ. Мы должны это изучить,
потому что это элементы гибридной войны. Эта истерия стоит сейчас стране
два-три млрд долларов ежемесячно. Мы считаем ежедневно потери экономики. Когда
кто-то решает переместить посольство во Львов, они должны понимать, что такая
новость будет стоить для экономики Украины несколько сотен миллионов долларов»,	
\end{zznagolos}

– сказал Арахамия. Видимо, в критической ситуации у «слуг» проснулся голос
разума. И понятно, почему – крах экономики будет означать, что и «команду Зе»
снесёт с политического Олимпа.

Но других главных и стратегических «партнёров» у Украины нет. Те, что есть,
похоже, начали нагнетать ещё сильнее, чем до «дня вторжения-1». А начавшееся
обострение на Донбассе только подогревает истерику.
