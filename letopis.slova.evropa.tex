% vim: keymap=russian-jcukenwin
%%beginhead 
 
%%file slova.evropa
%%parent slova
 
%%url 
 
%%author 
%%author_id 
%%author_url 
 
%%tags 
%%title 
 
%%endhead 
\chapter{Европа}
\label{sec:slova.evropa}

\ifcmt
  pic https://scontent-lga3-2.xx.fbcdn.net/v/t1.6435-9/202998360_317230440106414_620997140432873039_n.jpg?_nc_cat=102&ccb=1-3&_nc_sid=dbeb18&_nc_ohc=dNsSg3PS8-0AX9Kki4u&_nc_ht=scontent-lga3-2.xx&oh=43e85291d63405249bde9ba42a2dfaa3&oe=60D68E99
\fi

%%%cit
%%%cit_pic
%%%cit_text
Первый русский митрополит Илларион восхищённо говорил: «Церковь дивна и славна
всем окружным странам, яко же иной не найдется во всех землях от востока до
запада».  Итальянский путешественник Павел Аллепский писал, что «ум
человеческий не в силах её обнять».  Епископ киевский Иосиф Верещинский в XVI
веке отзывался о Софии Киевской так: «Весьма многие согласны в том, что в целой
\emph{Европе} нет храмов, которые по драгоценности и изяществу украшений стояли бы
выше константинопольского и киевского»
%%%cit_comment
%%%cit_title
\citTitle{Древнерусское зодчество. Собор Святой Софии в Киеве}, 
Искусство С Ириной Дружининой, zen.yandex.ru, 15.05.2021
%%%endcit

Перешлите кто-нибудь главе \emph{Европарламента}. Пусть человек порадуется, что
его призыв услышали в Донецке, который в самом \emph{Европарламенте} не слышен.
И пусть Давид Сассоли или еще какой-нибудь Борэль на пару с баронессой Эштон
еще несколько подобных акций придумают.  Можно продумать флешмоб, в котором
рассказывается о том, что раз в сто лет \emph{европейские} страны собираются в
кучу, чтобы получить люлей от России; это просто вариант, я не настаиваю... Но
все-таки скоро почти сто лет, как \emph{европейцы} без живительных российских
люлей живут.  Грустно небось?, 
\textbf{Донецк – Европа!}, Сергей Лебедев (Лохматый), voskhodinfo.su, 02.06.2021

А вы как думали: зря, что ли, пан президент так этих самых
«активистов-праворадикалов-патриотов» слушается и во всем им угодить старается?
Он же скоро будет не только за клоуна всей державы, но и за шамана – будет
камлать и постоянно мантры озвучивать: «Примите в НАТО! Примите в \emph{ЕС}!
Дайте денег! Дайте оружия! Мы – щит для \emph{Европы!} \emph{Европа}, защити
нас! Джо, позвони мне, позвони!» В общем, мантр у пана украинского президента
много – успевай только озвучивать!,
\citTitle{Форма не главное – главное содержание! И в спорте тоже...}, Мысли Бабы Яги, zen.yandex.ru, 07.06.2021


%%%cit
%%%cit_head
%%%cit_pic
%%%cit_text
Инфантильный набор претензий. Настоящий повелитель нелепости.  Как Зеленский
размазал канцлера.  Посмотрел поносик Блогера к будущему канцлеру Германии.
Персонаж, у которого в стране послица США обозначена в местном рейтинге первой
по влиятельности среди Ж-пола, решил поучительно подрючить первую экономику
\emph{Европы}. Жаль ВВП не измеряется в галактическом гоноре и свином
невежестве. Зеленский может в этом показать международный мастер-класс
%%%cit_comment
%%%cit_title
\citTitle{Пацан с района задал четкие вопросы будущему канцлеру / Лента соцсетей / Страна}, 
Игорь Лесев, strana.ua, 29.06.2021
%%%endcit

%%%cit
%%%cit_head
%%%cit_pic
%%%cit_text
Будущее независимой Украины лежит не в создании проблем для России с одной
стороны, а \emph{Европы} с другой. Будущее Украины лежит как раз в противоположном – в
решении этих проблем. Украина должна быть интегратором России в \emph{Европу} и \emph{Европы}
в Россию. Сегодняшние россияне, которые смотрят на результаты интеграции
Украины в \emph{ЕС}, туда не хотят. Для них \emph{европейский} вектор с помощью украинской
политики стал антирекламой.  Для \emph{европейцев} украинцы тоже антиреклама России.
Выходит, что для мира и спокойствия в \emph{Европе} нужно ликвидировать Украину? И
есть недавний пример – Югославия, которую растащили на отдельные государства и
территории
%%%cit_comment
%%%cit_title
\citTitle{О будущем украинского и русского народов}, 
Виктор Медведчук, strana.ua, 15.07.2021
%%%endcit

%%%cit
%%%cit_head
%%%cit_pic
%%%cit_text
В \emph{Европе} решено зеленеть до посинения.  В странах \emph{ЕС} с сегодняшнего дня
запрещено употреблять слово "пьяный", теперь следует говорить "альтернативно
трезвый". Международное движение ЖПВ (жизни пьяных важны) своим лидером избрало
Ефремова.  Мигранты с БВ - это немного запоздалый ответ на крестовые походы
%%%cit_comment
%%%cit_title
\citTitle{Почти правдивые новости - 13}, Мак Сим, zen.yandex.ru, 25.10.2021
%%%endcit

%%%cit
%%%cit_head
%%%cit_pic
%%%cit_text
А внизу ухмыляющийся народ. Расколотый на несколько частей. Кто в \emph{Европы} лыжи
вострит, кто в Россию. Остающиеся поп-корном и дровами запасаются. Глядя в
заиндевевшее окно на готовых полыхнуть очередным «майданом» националистов,
подстрекаемых речами Фарион. Если раньше билетик за просмотр комика с
«Кварталом» карман не тянул, то теперь дороговато спектакль обходится. Так и до
занавеса недалеко. Поскольку, всё едут и едут нескончаемые иноземные ревизоры...
Настоящие
%%%cit_comment
%%%cit_title
\citTitle{Такие смешные: украинская политика... через призму творчества Гоголя}, 
Исторические напёрстки, zen.yandex.ru, 28.10.2021
%%%endcit

%%%cit
%%%cit_head
%%%cit_pic
\ifcmt
  pic https://avatars.mds.yandex.net/get-zen_doc/3994559/pub_6180d6d880f2655b868c60cf_6180e71183df717f00dafdae/scale_1200
  @width 0.4
\fi
%%%cit_text
Есть такая старая колониальная методика в «обучении туземцев», сменившая прямое
кровавое принуждение: если не можешь опровергнуть целое, начинаешь исключать
(искажать) частности. Это тоже Россия пережила, похмельна по сей день
рецидивами. Когда всякие «вятровичи» подвергают ревизии подвиг советского
народа в Отечественной войне, к примеру.  Не в силах отменить сам факт Великой
Победы, начинают упражняться в малых частностях. Немного времени и денег... и
Зоя Космодемьянская становится клинической мазохисткой. Александр Матросов
пьяный в стельку спотыкается перед амбразурой, генерала Карбышева поливают
горячей водичкой при плюсовой температуре. Вермахт не побеждён талантом
сталинских военачальников и умением солдата Красной Армии, а «завален трупами».
\emph{Европа} не освобождена, а оккупирована. Перед этим — зверски изнасилована
%%%cit_comment
%%%cit_title
\citTitle{Как Украина была колонией СССР... Не наоборот ли?}, 
Исторические напёрстки, zen.yandex.ru, 02.11.2021
%%%endcit

%%%cit
%%%cit_head
%%%cit_pic
%%%cit_text
Переконаний, що перемога буде за нами. Але якою ціною? Наведу приклади. У
прифронтовій зоні вздовж Луганської області зруйновано 13 бібліотек, книжок
немає. Поруч потужні телетрансляції з Луганська, де сидять окупанти. Що зробили
держава, аби закрити цю ділянку? Нічого. Лише волонтери встановили малопотужні
пристрої, які глушать ті сигнали, що йдуть з Луганська. Це ж стосується
прифронтової зони і з боку самого Донецька.  Ні на Луганщині, ні на Донеччині
ви не зможете купити українську газету, придбати українську книжку. Проте на
окупованих територіях видається близько 200 газет і журналів. У самому
Луганську виходить 3 журнали для дошкільнят. І скрізь заборонена українська
мова, українська книга.  А що робить держава, що робить Верховна Рада, Кабінет
Міністрів України, президент для того, щоб цього не було? На жаль, ворог
знаходиться не лише на кордоні, але й у нас у серці, у нашій столиці. І генерує
те, що подає нам ворог.  Перед нами стоять надзвичайно складні завдання.
Пригадайте Тараса Шевченка: \enquote{Нема на світі України, Немає другого Дніпра}.
Правда завжди перемагає. Нам потрібно зробити все, щоб не тільки вистояти в
цьому світі, а й стати потужною \emph{європейською} державою
%%%cit_comment
%%%cit_title
\citTitle{Як виграти інформаційну війну? – Слово Просвіти}, ,slovoprosvity.org, 01.11.2021
%%%endcit
