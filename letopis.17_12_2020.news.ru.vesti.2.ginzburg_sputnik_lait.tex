% vim: keymap=russian-jcukenwin
%%beginhead 
 
%%file 17_12_2020.news.ru.vesti.2.ginzburg_sputnik_lait
%%parent 17_12_2020
 
%%url https://www.vesti.ru/article/2500215
 
%%author 
%%author_id 
%%author_url 
 
%%tags vaccine,covid
%%title Гинцбург: "Спутник V лайт" поможет предотвратить тяжелые случаи ковида
 
%%endhead 
 
\subsection{Гинцбург: \enquote{Спутник V лайт} поможет предотвратить тяжелые случаи ковида}
\label{sec:17_12_2020.news.ru.vesti.2.ginzburg_sputnik_lait}
\Purl{https://www.vesti.ru/article/2500215}

\ifcmt
pic https://cdn-st1.rtr-vesti.ru/vh/pictures/xw/308/119/5.jpg
\fi

Директор Центра имени Гамалеи Александр Гинцбург рассказал в интервью
телеканалу "Россия 24" о лайт-вакцине, о которой в ходе Большой
пресс-конференции упомянул президент Владимир Путин. Глава государства отметил,
что ею можно будет привить сразу несколько десятков миллионов человек.

\textbf{- Александр Леонидович, на Большой пресс-конференции президента большой пласт
вопросов был так или иначе связан с коронавирусом. Много вопросов по поводу
вакцины. Александр Леонидович, я хотел вас поблагодарить лично за вакцину
"Спутник V". Я сделал первую прививку еще в сентябре, потом вторую через три
недели. Действительно работает, проверил, антитела есть, все замечательно. Но
вот сегодня Владимир Путин говорил о лайт-вакцине. Вы наверняка знаете об этом
что-то. Я, если честно, первый раз об этом слышу. Лайт-вакцина, которая создает
иммунитет, но не так надолго, как вакцина "Спутник V". Расскажите поподробней,
пожалуйста.}

- Мне кажется, что Владимир Владимирович все очень подробно пояснил по поводу
необходимости наличия такого препарата. Единственное, что я могу добавить, это
связано с тем, что никакая промышленность ни Российской Федерации, ни Западной
Европы, ни Америки, ни всего земного шара одновременно в условиях пандемии не
может быстро произвести необходимое количество вакцинного препарата.

Тот вакцинный препарат, о котором говорил президент, позволяет предложить некий
компромисс между продолжительностью действия вакцинного препарата "Спутник V",
который обладает очень хорошей, по всей видимости, продолжительностью действия,
и необходимостью быстро сбить пик заболеваемости. Даже не столько
заболеваемости, сколько смертности от данного заболевания, использовав только
один компонент этой вакцины. Понимая при этом, что данный компонент будет
действовать далеко не на 100 процентов, а для разных категорий вакцинируемых
немного по-разному. Там будет и 85 процентов для одной категории, для других
категорий этот процент немного будет отличаться в одну или другую сторону. Но,
самое главное, такой подход позволит охватить большое количество населения и
предотвратить тяжелые случаи заболевания со смертельным исходом.

В то же самое время, я повторяю, продолжительность действия будет небольшая. Но
это определенный компромисс, связанный с тем, что ни одна промышленности ни
одной страны не может одновременно произвести все необходимое количество
вакцин.

Стратегия вакцинирования наших граждан в нашей стране, насколько я понимаю,
будет продолжена та, которая выбрана исходно и которая направлена на
использование "Спутника V" – двухкомпонентной векторной вакцины, которая
создает хороший прочный иммунитет и, я могу еще добавить, по всей видимости, на
длительный период.

\textbf{- Очень на это рассчитываем. Александр Леонидович, если я правильно понял,
лайт-вакцина – это первый компонент от вакцины \enquote{Спутник V}?}

- Совершенно верно. И Российский фонд прямых инвестиций в настоящее время на
зарубежных площадках, которые требуют одновременно введения в категорию тех
стран, которые будут обладать собственным вакцинным препаратом, и также хотят
как те страны, которые сейчас самостоятельно эту вакцину разрабатывают,
защитить своих граждан. И, соответственно, использование такого подхода будет
определенным компромиссом для того, чтобы максимальное количество жителей Земли
охватить вакцинными препаратами, в том числе и созданными на основе той
разработки, которая исходно была в институте Гамалеи.

\textbf{- Александр Леонидович, расскажите, пожалуйста, о взаимодействии центра имени
Гамалеи и компании AstraZeneca.}

- Да, здесь существенные подвижки есть навстречу друг другу. Исходно наши
западные коллеги придерживались стратегии разработки вакцинного препарата по
классическим стандартам, которые заключаются в тех же стандартах, которые
используются для создания вакцин против гриппа. Грипп, как мы знаем, сезонное
заболевание, которое продолжается как правило 3-4 месяца и поэтому вакцины,
которые созданы против гриппа, однократно используются и дают защиту на 3-4
месяца.

Но когда мы занимались разработкой вакцинного препарата против возбудителя
COVID-19, мы исходили из того, что данное заболевание не будет протекать по
эпидемиологическим законам, по которым протекает инфекция, обусловленная
вирусом гриппа. И мы, с моей точки зрения, были правы, так как мы сейчас все
видим, что это заболевание протекает все 12 месяцев в году. Поэтому иммунитет
надо создавать длительный, напряженный и рассчитывать, что продолжительность
иммунитета должна быть несколько лет. Поэтому мы сразу создали два компонента
этой вакцины, двухкомпонентную вакцину. Второй компонент усиливает действие
первой. Не только усиливает, но и позволяет в результате повторной вакцинации с
большой вероятностью надеяться на образование "клеток памяти", которые
длительное время будут нас защищать. По всей видимости, мы угадали и именно по
такому механизму работает та вакцина, которая сейчас называется "Спутник V".

У наших английских коллег и у AstraZeneca имеется только один компонент, нет
эффективного бустирующего компонента. Если второй раз, как наши коллеги
продемонстрировали из Оксфорда из AstraZeneca, использовать первый компонент,
то усиление происходит, но очень в незначительной степени от использования
второго компонента, который соответствует по своей структуре первому. Поэтому
между Российским фондом прямых инвестиций и AstraZeneca была достигнута
договоренность о возможности проведения клинических испытаний, в которых бы
использовался в качестве первого компонента тот, который был создан в Оксфорде,
а в качестве второго бустирующего компонента – тот компонент, который входит в
состав вакцины "Спутник V".

Насколько я понимаю, в ближайшее время данные клинические испытания начнутся, и
тогда в руках у наших зарубежных коллег будет очень эффективный вакцинный
препарат, который, можно считать, будет создан совместными усилиями и, я
надеюсь, мы все порадуемся о том, что на тех мощностях, которыми располагает
AstraZeneca, производиться вакцина, созданная усилиями ученых как Англии, так и
Российской Федерации.

\textbf{- Это как раз очень хороший пример того, о чем сегодня тоже и президент России
говорил – давайте жить дружно. Это пример международного сотрудничества,
которое приводит к эффективным результатам.}

- К реальным эффективным результатам.
