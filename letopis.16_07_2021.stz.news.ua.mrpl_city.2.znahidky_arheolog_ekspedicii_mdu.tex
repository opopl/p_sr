% vim: keymap=russian-jcukenwin
%%beginhead 
 
%%file 16_07_2021.stz.news.ua.mrpl_city.2.znahidky_arheolog_ekspedicii_mdu
%%parent 16_07_2021
 
%%url https://mrpl.city/blogs/view/unikalni-znahidki-arheologichnoi-ekspeditsii-mdu
 
%%author_id demidko_olga.mariupol,news.ua.mrpl_city
%%date 
 
%%tags 
%%title Унікальні знахідки Археологічної експедиції МДУ
 
%%endhead 
 
\subsection{Унікальні знахідки Археологічної експедиції МДУ}
\label{sec:16_07_2021.stz.news.ua.mrpl_city.2.znahidky_arheolog_ekspedicii_mdu}
 
\Purl{https://mrpl.city/blogs/view/unikalni-znahidki-arheologichnoi-ekspeditsii-mdu}
\ifcmt
 author_begin
   author_id demidko_olga.mariupol,news.ua.mrpl_city
 author_end
\fi

\ii{16_07_2021.stz.news.ua.mrpl_city.2.znahidky_arheolog_ekspedicii_mdu.pic.front}

Щороку археологічна експедиція Маріупольського державного університету
продовжує багаторічну і плідну роботу, спрямовану на виявлення, вивчення й
охорону археологічних пам'яток Приазов'я. На сьогодні вона є єдиною в Донецькій
області, що має офіційне право проводити розкопки. Пам'ятаю, у 2009 році я теж
проходила археологічну практику, тоді ми досліджували Амвросіївську стоянку.
Наразі з впевненістю можу сказати, що кожен предмет, здобутий під час
археологічних досліджень є унікальним та неповторним, а пам'ять про розкопки і
життя в польових умовах залишається на все життя.

Цього літа учасники археологічної експедиції Маріупольського державного
університету розбили наметовий табір поблизу села Комишувате (село в
Мангушській селищній громаді Маріупольського району Донецької області України).
У цій місцевості почав дослідження ще в 1989 році засновник археологічної
експедиції МДУ – Володимир Кульбака, який був вчителем для сучасних археологів
Приазов'я – В. Забавіна та С. Небрата. Торік к.і.н., доцент кафедри історичних
дисциплін та керівник Археологічної експедиції МДУ В'ячеслав Забавін зіставив
карти і з'ясував, що з 5 курганів, які значилися в цьому курганному могильнику
було досліджено всього 3. Цікаво, що на базі цього могильника минуле нашого
краю вивчає вже третє покоління археологів.

\ii{insert.read_also.demidko.kulbaka}
\ii{16_07_2021.stz.news.ua.mrpl_city.2.znahidky_arheolog_ekspedicii_mdu.pic.1}

В одному з двох курганів, який виявився елітарним вдалося знайти унікальні
поховання, що мають форму кам'яних гробниць і відносяться до епохи пізньої
бронзи зрубної культури (16–14 ст. до н.е.). За попередніми висновками
археологів у першому похованні кургану знаходився чоловік. З огляду на
супровідний інвентар швидше за все – служитель культу, або жрець, шаман.
Головне, що перед його обличчям були розташовані досить  рідкісні для нашого
регіону залишки дерев'яної чаші, окутої бронзовою накладкою. В'ячеслав Забавін
розповів, що з керамічного посуду зазвичай вживалися звичайні напої, а
дерев'яний – призначався саме для вживання священних напоїв – \emph{сома-хаома}. Ще
Володимир Кульбака довів, що сома-хаома була сумішшю коров'ячого молока і соку
маку, тобто галюциногенним напоєм, за допомогою якого шаман досягав камлання –
входив у екстаз для спілкування із духами. Знайдена накладка на чаші також
досить рідкісна для Приазов'я. Із 1 500 поховань у регіоні, подібні накладки
були знайдені тільки у сімох. Накладка з цього поховання має унікальний розмір
– 3 сантиметри завширшки та 15 сантиметрів у довжину. Вже ця унікальність
підкреслює елітарність досліджуваного курганного комплексу.

У другому, ймовірно, був похований родич – онук чи правнук, який теж мав
високий статус у суспільстві. Як розповів заступник керівника археологічної
експедиції \textbf{Сергій Небрат}, при похованому було знайдено не один горщик, як
трапляється  в переважній більшості випадків, а два. Тільки один був вище
рівнем. Також була виявлена і м'ясна їжа, що також підвищує статус цієї
померлої людини.

\textbf{Читайте також:} \emph{Мариупольские археологи раскопали захоронение древнего жреца}%
\footnote{Мариупольские археологи раскопали захоронение древнего жреца, Богдан Коваленко, mrpl.city, 13.07.2021, \par%
\url{https://mrpl.city/news/view/mariupolskie-arheologi-raskopali-zahoronenie-drevnego-zhretsa}
}
\ii{16_07_2021.stz.news.ua.mrpl_city.2.znahidky_arheolog_ekspedicii_mdu.pic.2}

Ректор Маріупольського державного університету, професор \textbf{Микола Трофименко}
всіляко підтримує археологічну експедицію і сподівається, що знахідки стануть
цікавими для туристів і науковців як з України, так і з-за кордону. Ці розкопки
відвідали навіть друзі міста  з Батумі, які допомагають розробляти стратегію
розвитку туризму Маріуполя. Ректор налаштований створити дуже яскраві і
унікальні локації, завдяки яким цікаво буде знайомитися із знахідками  як
студентам та викладачам, так  і тим людям, які будуть відідрати наше місто.

Наразі у наших археологів є необхідне оснащення, цілий наметовий табір, рації і
квадрокоптер. Цьогоріч між Маріупольським державним університетом та Де Пе
\enquote{Ілліч – Агро Донбас} агрохолдингом HarvEast було підписано договір щодо
спільного збереження пам'ятників архітектури на території Приазов'я. Завдяки
цій співпраці у археологів з'явився навіть свій душ. Дуже корисною стала і
техніка агрохолдингу.

\ii{16_07_2021.stz.news.ua.mrpl_city.2.znahidky_arheolog_ekspedicii_mdu.pic.3}

Під проливними дощами, і спекотним сонцем викладачі,\par\noindent студенти-першокурсники,
та волонтери з Києва, Донецької і Запорізької областей живуть в польових умовах
вже третій тиждень. До речі, цьогоріч експедиція пережила великі зливи досить
стійко. І дівчата, і хлопці проявили неабияку терплячість, і пройшли справжню
перевірку на витривалість під час несприятливих погодних умов. Водночас
археологи вміють і культурно відпочивати. Учасник експедиції \textbf{Марк Бородай}
цьогоріч створив для всіх цікаву і насичену культурну програму, присвячену
святу Івана Купала. Студенти мали можливість поринути у загадковий світ
східного слов'янства. Марк розробив три частини цього заходу: це був магічний
обряд, коли дівчата плели разом з хлопцями косу, потім сплітали віночок з
польових квітів. Все це завершувалося дуже цікавим факельним видовищем, яким і
закінчився цей магічний обряд. А на посвяті в археологи, що є найголовнішою
культурною подією в кожному сезоні, студенти навіть шукали скарб і повинні були
впоратися з великою кількістю завдань та загадок.

\ii{16_07_2021.stz.news.ua.mrpl_city.2.znahidky_arheolog_ekspedicii_mdu.pic.4}

Протягом останніх 5 днів своєї роботи археологи планують дослідити ще один
курган з цього могильника. Всі знахідки зберігатимуться в музеї історії та
археології МДУ.

\textbf{Читайте також:} \emph{В Мариуполе на кургане и древнем поселении установили охранные знаки}%
\footnote{В Мариуполе на кургане и древнем поселении установили охранные знаки, Богдан Коваленко, mrpl.city, 24.03.2021, \par%
\url{https://mrpl.city/news/view/v-mariupole-na-kurgane-i-drevnem-poselenii-ustanovili-ohrannye-znaki}
}

\emph{Всі світлини з архіву Археологічної експедиції МДУ}
