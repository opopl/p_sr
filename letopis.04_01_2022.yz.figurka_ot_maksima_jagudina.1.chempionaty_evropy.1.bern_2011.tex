% vim: keymap=russian-jcukenwin
%%beginhead 
 
%%file 04_01_2022.yz.figurka_ot_maksima_jagudina.1.chempionaty_evropy.1.bern_2011
%%parent 04_01_2022.yz.figurka_ot_maksima_jagudina.1.chempionaty_evropy
 
%%url 
 
%%author_id 
%%date 
 
%%tags 
%%title 
 
%%endhead 

\subsubsection{Чемпионат Европы 2011 (Берн)}
\label{sec:04_01_2022.yz.figurka_ot_maksima_jagudina.1.chempionaty_evropy.1.bern_2011}

В то время турнир проходил с предварительным полуфиналом, где фигуристки с
наиболее низким рейтингом показывали свои произвольные программы и лучшие из
них в итоге выходили в финальный раунд, где уже непосредственно шла борьба за
медали.

\ii{04_01_2022.yz.figurka_ot_maksima_jagudina.1.chempionaty_evropy.pic.2}

Победу на домашнем турнире одержала 26-летняя Сара Майер для которой это был
уже 10й Чемпионат Европы и как оказалось последний, после этого сезона Сара
закончила свою спортивную карьеру. Уровень прокатов лидеров тогда был в
принципе сопоставим с нынешним средним уровнем европейских фигуристок, в
частности Майер победила с 5 тройными прыжками в произвольной программе без
каскадов 3+3. Безусловный фаворит трехкратная чемпионка Европы Каролина Костнер
можно сказать сама отдала победу фигуристке из Швейцарии, совершив большое
количество ошибок.

Выступление россиянок стало самым успешным, начиная с 2006 года, что в
частности позволило завоевать третью квоту на следующий год. Для 18-летней
\textbf{Ксении Макаровой} 4 место стало в итоге самым большим достижением в карьере на
международных стартах, а вот \textbf{Алену Леонову} в ее 20 лет главный триумф ждал в
следующем сезоне. Пока же буквально через 3 месяца на московском Чемпионате
Мира Алена наберет на 30! баллов больше, чем в Берне и остановится в шаге от
бронзовой медали.

