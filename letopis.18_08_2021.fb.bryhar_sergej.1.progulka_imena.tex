% vim: keymap=russian-jcukenwin
%%beginhead 
 
%%file 18_08_2021.fb.bryhar_sergej.1.progulka_imena
%%parent 18_08_2021
 
%%url https://www.facebook.com/serhiibryhar/posts/1767010233498921
 
%%author Бригар, Сергей
%%author_id bryhar_sergej
%%author_url 
 
%%tags identichnost',kakaja_raznica
%%title Алєкс, Джулі, Дженні - і все це я почув протягом однієї прогулянки
 
%%endhead 
 
\subsection{Алєкс, Джулі, Дженні - і все це я почув протягом однієї прогулянки}
\label{sec:18_08_2021.fb.bryhar_sergej.1.progulka_imena}
 
\Purl{https://www.facebook.com/serhiibryhar/posts/1767010233498921}
\ifcmt
 author_begin
   author_id bryhar_sergej
 author_end
\fi

Алєкс, Джулі, Дженні - і все це я почув протягом однієї прогулянки. Батьки постійно називають своїх малюків саме так. 

Ні, я не маю наміру когось засуджувати. То, зрештою, особистий вибір. Але, як
на мене, звучить трохи дивно. От щось тут не так.

Мені, якщо вже чесно, теж подобається португальський варіант імені старшого
сина. Його б там називали Назарено. Як на мене, гарно. Але я ж не називаю його
так завжди. Те ж саме, власне, і з молодшим, якого б у західній Європі називали
Еустакіо. Це теж непогано. Але не для постійного використання. Головне -
українські варіанти!

От чуєш усі ці Ендрю, Віллі, Пітер, і відчувається глобальний комплекс
меншовартості, який буквально накрив людей з головою.  (Але цей допис не про
імена як такі! Це просто дуже помітна ознака загального явища, про яке мені
нині думається)

Насправді я вже давно спостерігаю за деякими людьми, які намагаються вибудувати
щось на зразок нової ідентичності. "Русскоязичниє європєйци"?... Напевно, якось
так.

\ifcmt
  pic https://scontent-cdg2-1.xx.fbcdn.net/v/t39.30808-6/236900180_1767010156832262_7159113027760006055_n.jpg?_nc_cat=100&_nc_rgb565=1&ccb=1-5&_nc_sid=8bfeb9&_nc_ohc=VSl23B8WRmEAX85TB7H&_nc_ht=scontent-cdg2-1.xx&oh=7ca342d89ad98f50ffeae27ea660d2e5&oe=61242457
  width 0.4
\fi

Здається, вони абсолютно не бажають бути українцями ("Зачєм? Ето же нє
прєстіжно"). Росіянами, наскільки я розумію, - теж. Хоча, якщо дивитися
комплексно, то майже завжди з'ясовується, що російського в них більше... Це
така собі  космополітичність на московській ліберальній заквасці (а на якому
питанні то все закінчується, ми з вами знаємо), спрямована кудись туди, де
краща якість життя. Але якщо вірити їм на слово, то, повторюся: "бути росіянами
вони не хочуть".

От нещодавно розмовляв з батьком двох малих дітей, який переконаний - принаймні
на словах, - що необхідно в усьому орієнтуватися на захід. Такий собі
"заходофіл". Багато розмовляли про Європу, США, Канаду, функціональність різних
іноземних мов. Я говорю:

"Так, англійська мова, безперечно, важлива, це нові горизонти, свобода, але
обов'язкова і українська, адже будь-яка європеїзація апріорі неможлива, якщо не
постане справді українська Україна - за змістом, без вкорінення у національну
основу ми нічого не збудуємо, і ніколи не станемо повноцінною частиною західної
цивілізації". 

\obeycr
- Ну так укрАінскій в школє учат. Знать оні єго будут.
- Але цього мало. Мову потрібно не просто десь чути і використовувати лише за коайньої потреби, наприклад для шкільної оцінки. Вона має стати невід'ємною частиною повсякдення.
- Ну вот тут я с тобой нє согласєн.
- Чому ж?
- Твоі взгляди устарєлі. Чувак, ми в Одєссє. Тут ета мова - романтіка какая-то. Ти - романтік. А я - практік. Я понімаю, что нужно для нормальной жизні. А всьо остальноє - лишняя суєта.
- А якщо землю, на якій ти живеш, і яку не розумієш як територію з національною основою, в тебе відбере інша держава? Що тоді? Пристосування до іншого порядку?
- Ти про Россію? Нє, в Россію нє хочу. Я за то, чтоб імєть срєдства, і в случає чєго обустраіваться в нормальних широтах.
- Тобто, поки в Україні умовно нормально, ти тут, а якщо заштормить - втеча?
- Совєршенно вєрно...
\restorecr

Він, до речі, теж називає своїх  малих на західний манер. 

Під час розмови з такими персонажами не полишає відчуття безнадійності. На
конструктив розраховувати складно. Це люди без національної основи, люди, що
застрягли між Росією і Заходом, та вперто "не бачать" Україну, ставлячись до
неї лише як до бляклого економічного придатку, сірого простору відносного
спокою... Вони дуже обурюються через те, що в нас тут гірше, ніж "там", гірше,
ніж мало би бути, що влада "полюбляє обдирати громадян"...

Я завжди хочу сказати таким (та власне, й кажу) приблизно таке: 

"Ви ж самі, представники величезної секти "какаяразніца", тягнете до влади тих,
хто натхненно обдирає громадян, по суті, просто цементуєте олігархічний лад, ви
самі навіть не маєте бажання розбудовувати державу, ігноруючи ключові фактори
її формування та зміцнення. Саме ви виховуєте своїх Алєксів, Джулі та Дженні
(так, звісно, питання далеко не лише в зовнішніх атрибутах, ті ж імена можуть
бути які завгодно - тут маємо справу зі складним комплексом факторів, я знаю) в
атмосфері, яку я харатеризую як "показове ігнорування України". Всі негативи,
які ми сьогодні спостерігаємо - це і ваша особиста провина. Так, і моя - теж.
Безумовно. Ми всі могли би бути кращими: стараннішими, впевненішими,
наполегливішими, свідомішими. Я міг би бути конструктивнішим, активнішим,
продуктивнішим! Але різниця між мною і такими як я, та вами, полягає в тому, що
вам, здається, - хоча, я дуже хотів би помилятися - все одно, а нам - ні...".

\ii{18_08_2021.fb.bryhar_sergej.1.progulka_imena.cmt}
