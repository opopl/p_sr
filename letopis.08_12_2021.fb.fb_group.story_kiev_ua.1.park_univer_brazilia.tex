% vim: keymap=russian-jcukenwin
%%beginhead 
 
%%file 08_12_2021.fb.fb_group.story_kiev_ua.1.park_univer_brazilia
%%parent 08_12_2021
 
%%url https://www.facebook.com/groups/story.kiev.ua/posts/1814527125410747
 
%%author_id fb_group.story_kiev_ua,denisova_oksana.kiev.ukraina.gid
%%date 
 
%%tags brazilia,gorod,istoria,kiev,park,universytet_shevchenka
%%title История парка напротив Университета - император Бразилии Педру II
 
%%endhead 
 
\subsection{История парка напротив Университета - император Бразилии Педру II}
\label{sec:08_12_2021.fb.fb_group.story_kiev_ua.1.park_univer_brazilia}
 
\Purl{https://www.facebook.com/groups/story.kiev.ua/posts/1814527125410747}
\ifcmt
 author_begin
   author_id fb_group.story_kiev_ua,denisova_oksana.kiev.ukraina.gid
 author_end
\fi

\begin{multicols}{2} % {

Мы с вами привыкли, что напротив Университета прекрасный парк, но ведь так было
не всегда, у этого парка очень занимательная история. И появился парк здесь
благодаря императору Бразилии  Педру II. Вы думаете я шучу, нет. 

\setlength{\parindent}{0pt}

\ii{08_12_2021.fb.fb_group.story_kiev_ua.1.park_univer_brazilia.pic.1}
\ii{08_12_2021.fb.fb_group.story_kiev_ua.1.park_univer_brazilia.pic.1.cmt}

\ii{08_12_2021.fb.fb_group.story_kiev_ua.1.park_univer_brazilia.pic.2}
\ii{08_12_2021.fb.fb_group.story_kiev_ua.1.park_univer_brazilia.pic.2.cmt}
\end{multicols} % }

Здание Университета было построено в 1843 году и власти Киева планировали
нарезать напротив здания участки под застройку. Но вмешались пожарные : мол,
территорию перед учебным корпусом застраивать нельзя. В случае пожара сюда
будут эвакуироваться студенты, преподаватели. И власти потеряли к этой
территории всякий интерес. И в течение 33 лет перед университетом был пустырь,
на  котором бродили бродячие псы, пугая студентов и гимназистов находящихся
рядом гимназий. 

А в 1876 году в Киев приехал император далекой Бразилии Педру II. Он был
человеком образованным, очень любил путешествовать и в  феврале 1876 года
вместе с супругой Терезой-Кристиной отправился в кругосветное путешествие -
США, Скандинавия, Европа, Ближний Восток… Не осталась вне его внимания и
Российская империя. В середине августа высокий гость прибыл в Петербург. Однако
императора Александра II там не застал – летом и осенью монарх обычно оставлял
столицу и жил в своем крымском имении Ливадия. Три недели бразилец знакомился с
«Петра твореньем» и окрестностями, а затем отправился поездом в Крым. Однако по
дороге заехал в Киев, о котором он слышал как о «втором Иерусалиме». 

И вот 7 сентября 1876 года императорская чета прибыла в Киев, поселили их в
«Гранд-отеле» на Крещатике. Несколько дней они осматривали все киевские
достопримечательности, побывали на двух спектаклях в театре, вечером был
фейерверк в парке «Шато-де Флер». Приятной неожиданностью для заморского гостя
оказалось присуждение ему звания почетного члена Императорского университета
святого Владимира. Соответствующее решение принял Совет университета в день
прибытия бразильца в Киев и его любезно пригласили для вручения диплома.
Осмотрев аудитории, химическую лабораторию и библиотеку, гость глянул в окно и
удивился: «Такой чудесный университет и такой нелепый пустырь перед ним…». 

На следующий день Педру II выехал в Ливадию и  наверняка позабыл вчерашнюю
реплику насчет пустыря. Но киевский генерал-губернатор Александр
Дондуков-Корсаков перепугался, вдруг Педру II расскажет свои впечатления от
университета и пустыря перед ним в Ливадии Александру II. Он вызвал городского
садовника Карла Христиани и велел немедленно представить проект парка перед
университетом. Уже наутро садовник привез чертежи и на пустыре тут же начались
работы. Начали очень активно, а потом как-то расслабились и  часто
останавливались. И Университетский парк открылся лишь через 10 лет – в 1887
году. За это время в Киеве сменились еще два генерал-губернатора. А уважаемый
Педру II, я думаю, так и не узнал, какую роль он сыграл в появлении всеми нами
любимого парка.

Вот написала и сразу, почему-то, захотелось пересмотреть фильм «Здравствуйте, я
ваша тетя».
