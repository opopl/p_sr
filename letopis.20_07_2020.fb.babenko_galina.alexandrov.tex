% vim: keymap=russian-jcukenwin
%%beginhead 
 
%%file 20_07_2020.fb.babenko_galina.alexandrov
%%parent 20_07_2020
 
%%endhead 
\subsection{Трижды Герой Социалистического труда из армии Врангеля}
\label{sec:20_07_2020.fb.babenko_galina.alexandrov}
\url{https://www.facebook.com/permalink.php?story_fbid=2955010927958870&id=100003499757878}
  
\vspace{0.5cm}
{\small\LaTeX~section: \verb|20_07_2020.fb.babenko_galina.alexandrov| project: \verb|letopis| rootid: \verb|p_saintrussia|}
\vspace{0.5cm}

\index{Люди!Александров (физик, академик)}

Александров говорил, что никогда ему не было так страшно, как лежать за
пулеметом перед лавой Красной конницы и не сметь открывать огня до команды

Видать, огонь вел неплохо, если были боевые награды и если красноармейская
шашка до его шеи не дотянулась...

Академик Анатолий Петрович Александров, возглавлявший Академию Наук с 1975 по
1986-й годы, был выдающимся физиком, одним из основателей советской ядерной
энергетики. Разработчик защиты военных кораблей от магнитных мин, один из
создателей атомных ледоколов, академик Александров был трижды Героем
Социалистического Труда, кавалером 10 орденов Ленина, лауреатом четырех
Сталинских, одной Ленинской и одной Государственной премий, а также немыслимого
числа других наград.

Куда меньше известно, что этот гениальный советский ученый в юности сражался
против Красной Армии в войсках Врангеля, и даже имел награды. Для посвященных
это не было секретом еще с 70-х, но в Интернете информация, что молодой юнкер
1903 г.р. Александров был пулеметчиком с боевыми наградами у Врангеля, воевал
отлично, был ранен, появилось буквально несколько лет тому назад.

Он был третьим ребёнком в семье надворного советника и мирового судьи Петра
Павловича и Эллы Эдуардовны Александровых. Надворный советник --- чин дворянский.
Пётр Павлович исполнял должность мирового судьи в маленьком городке Тараща,
удалённом от губернского Киева на 120 вёрст. Корни по отцовской линии --- в
Саратове, дед занимался торговлей зерном по происхождению русский. Мать по
происхождению наполовину шведка.

В 1919 году, в разгар Гражданской войны, Анатолий Петрович окончил Киевское
реальное училище. Вероятно, это был последний выпуск. Аттестат давал право на
поступление в университет на физико-математический или медицинский факультет.

Красная армия овладела Киевом 5 февраля 1919 года. Александров с приятелем в
это время был на даче в Млынке.

Далее семейные хроники доносят: «На обратном пути в Киев на железнодорожной
станции Фастов Толя встретил знакомого офицера, соседа по киевской квартире.
Офицер сказал молодым людям, что в город ехать нельзя, и если они истинные
патриоты, то они должны защищать свою родину и встать в ряды Белой гвардии.
Мальчики ушли с ним на фронт»

16-летний Толя Александров вступил в Белую армию добровольцем. Когда белые
потерпели поражение в Крыму, будущий академик предпочел остаться в России, и
оказался в плену у красных, где ему грозил расстрел.

Доподлинно неизвестно как он его избежал, но версия, ходившая еще в 70-е в
Курчатовском институте гласила следующее: после падения врангелевского Крыма он
сидел в подвале в ожидании расстрела, когда его вызвала на допрос какая-то
комиссарша и сказала: "Убегай, дурачок, если сумеешь." Толя Александров избежал
непоправимого, вернулся домой, продолжил обучение и стал выдающимся советским
ученым. Он сумел и стал кем стал.

Естественно, своей белогвардейской биографии академик Александров не
афишировал, хотя иногда проговаривался об этом в кругу домашних. Также
неизвестно доподлинно, знали ли о прошлом великого ученого советские
спецслужбы. Возможно, кто-то наверху предпочел закрыть глаза на «грехи
молодости» президента Академии Наук.

Повезло. Повезло и ему, повезло и России в лице СССР.  Это показывает, что
красные или белые - это не столь важно. Важно, кто из них на данном этапе может
построить сильное государство, справедливое общество, воспитать достойных
людей. А размежевания - это на потеху западному миру.

Антон Деникин. Вроде вражина еще тот, но упрекнуть его в аморальности
невозможно. Он искренне верил в то, что отстаивал с оружием в руках, он всё
равно за Россию. Деникин воевал за Россию - потому в Великую Отечественную НЕ
сотрудничал с немцами, хотя с новой российской властью не замирился.

Обратный пример Краснов, который воевал за вольную казакию и в Великую
Отечественную увидел шанс снова вернуться к заветным мечтам... За что и понес
справедливое наказание.

Ещё из зигзагов истории. Один из сталинских маршалов белогвардейский офицер. Не
царский, а именно белогвардейский. Им был Говоров.

https://chervonec-001.livejournal.com/3348583.html
  
