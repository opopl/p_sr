% vim: keymap=russian-jcukenwin
%%beginhead 
 
%%file 10_11_2021.fb.fb_group.story_kiev_ua.2.konstantin_kievskij_muchenik.cmt
%%parent 10_11_2021.fb.fb_group.story_kiev_ua.2.konstantin_kievskij_muchenik
 
%%url 
 
%%author_id 
%%date 
 
%%tags 
%%title 
 
%%endhead 
\subsubsection{Коментарі}

\begin{itemize} % {
\iusr{Ольга Гураль}

Любители совка сейчас начнуть писать, что зато колбаса была по 2.20 руб.за кило
и мороженное вкусное. Хотя, пардон, это уже в 60-70-х! А в 1937 и жрать-то
особо нечего было! Да и денег не хватало. По словам моей бабули, жившей в
Киеве, селедочка с картошкой - радость, а мясо с базара - по праздникам! А вот
черный "воронок" (газик военный) из НКВД по ночам ездил с обысками и арестами
постоянно. И все жители затаив дыхание прислушивались - в какой дом и какую
квартиру будут ломиться!

\begin{itemize} % {
\iusr{Вадим Горбов}
\textbf{Ольга Гураль} а кто четыре миллиона доносов написал?

\begin{itemize} % {
\iusr{Anna Rymarenko}
\textbf{Вадим Горбов} Вы уверены, что так и было?

\iusr{Сергій Батурин}
\textbf{Anna Rymarenko} 

А як було? При мені Володимир Яворівський в спілці письменників розповідав:
коли уперше в кінці 80-х відкрили архіви КДБ, з'ясувалося, що письменник Іван
Ле написав на колег понад 90 (!!!) доносів.

\iusr{Вадим Горбов}
\textbf{Anna Rymarenko} я их лично часто видел в архивах КГБ-СБУ

\iusr{Вадим Горбов}
\textbf{Людмила Гаценко} перше і останнє попередження, тут вам не політ клуб Грановського-Кононенко та 8\%
\end{itemize} % }

\iusr{Anna Rymarenko}
\textbf{Ольга Гураль} !!!

\iusr{Клибановский Евгений}
\textbf{Ольга Гураль} 

Если вы проведите сравнения с нынешней эпохой, то увидите тонкую нить и связь с
сегодняшним, методы уничтожения людей более гуманный, и их еще не кто не
отменил! С эпохи так ненавистного вами СОВКА минуло уже 30 лет!! И количество
жителей Украины и без репрессий расстрелов и тд сократили с 52 миллионов до
почти 30 это ли ни есть ГЕНОЦИД! Как думаете коммунисты пришли к власти
потому-что в стране все было отлично и простые рабочие и крестьяне просто от
нечего делать захотели смещение власти ЦАРЯ!? Очень просто судить об истории
однобоко, быть за левых или за правых! Сделайте так что-бы за вас были и левые
и правые! Вот это и будет Рай на земле!

\begin{itemize} % {
\iusr{Светлана Маркевич}
\textbf{Клибановский Евгений} Спасибо, достойный ответ!

\iusr{Клибановский Евгений}
\textbf{Светлана Маркевич} и вам спасибо за поддержку, в стране советов я прожил только до 10 лет. Не застав эпоху репрессий, плохо учась в школе, в 40 пишу с ошибками за что не раз пытались унизить, но я понимаю сердцем и мозгом, то что некоторые нехотят даже вилеть;(

\iusr{Max Gopencko}
\textbf{Клибановский Евгений} простите, но что-то вы сильно заврались, какие 30 миллионов? Население Украины - 45. А сократилось оно не из-за массовых репрессий, а из-за оттока людей в более успешные страны, что было их свободным выбором. Плюс Россия оттяпала кусок территории. А последствия совка мы ещё будем разгребать не одно десятилетие... дай Бог, что бы разгребли.

\iusr{Ольга Гураль}
\textbf{Клибановский Евгений} Нет, это не геноцид! Людией не убивают. А когда появилась возможнтсть выезжать за границу, то едут туда жить и путешествовать, и по др. причинам. Молодежь во всем мире так колесит и никто не говорит о геноциде
\iusr{Людмила Гаценко}
\textbf{Эдуард Эдуардович Ахинеев}

А не берете за розрахунок відтік людей в інші країни, причому ще й раніше, ще
за ссср, особливо, коли відкрилася "залізна завіса"? По всьому світі люди
завжди мігрували. Крім мешканців авторитарних країн. А спала завіса - і
баааагатьом захотілося переїхати в хороше життя, особливо тим, хто начебто
повертався на землю предків (правда, незрозуміло, чому вони їхали ще й до
Німеччини, не за покликів пращурів, а до тих, хто їх колись знищував...). То
чого ж від хорошого шукати кращого? Зараз же стільки ностальгійних спогадів про
рай ссср?

Завжди і зараз з Польщі люди їдуть до Німеччини, з Німеччини - до
Нідерландів... Перелік руху може тривати й тривати. В тих країнах також
відбувається відтік населення. Народжуваність також знижується. Скрізь, крім
азіатських країн (хоча й там відбуваєт ься коригування).

Просто треба бути незааг'а нжованим реалістом. "Хто живе з поверненою назад
головою, не бачить майбутнього".

\iusr{Люба Микицька}
\textbf{Olga Gural} єто любимое слово у любителей совка.
\end{itemize} % }

\iusr{Клибановский Евгений}

Дот той поры пока нас делят на белых и красных, римлян и галлов, сторонников
фашизма или Ленинизма, за мову или все равно на коком языке главное то что
человек, христианин или мусульманин привитых или не привитых, мы сами будем
уничтожать себе подобных!

\begin{itemize} % {
\iusr{Max Gopencko}
\textbf{Клибановский Евгений} а что будет до тех пор, пока люди будут врать, преувеличивать и на полном серьёзе сравнивать репрессии и иммиграцию. Вот вы с какой целью соврали про 30 миллионов?

\iusr{Клибановский Евгений}
\textbf{Max Gopencko} а сколько нас миллионов?! Перепись населения в Украине была последний раз в каком году?!

\iusr{Max Gopencko}
\textbf{Клибановский Евгений} 

встречный вопрос: а вы на какую перепись опирались, когда про 30 миллионов соврали? Как считали?

Я то озвучил нынешние данные, которые можно подсчитать методом исключения
умерших и уехавших с момента последней переписи. А вы просто соврали. Ради
чего?

\iusr{Клибановский Евгений}

Последняя официальная перепись населения Украины прошла в 2001 году. По её
итогам население страны составляло 48 240 902 постоянных жителя и 48 457 102
человека наличного населения.

\iusr{Max Gopencko}
\textbf{Клибановский Евгений} 

вы не обращаетесь ко мне, дабы я не заметил вашего ответа?

Ещё раз: отнять с последней переписи умерших и уехавших - будет 45 миллионов.
Это называется МАТЕМАТИКА. А вот вы про 30 миллионов банально соврали. Я и
спрашиваю - с какой целью?

\end{itemize} % }

\iusr{Людмила Гаценко}
\textbf{Вадим Горбов}
Заувага незрозуміла й тенденційна. Власних думок в коментарях не приймаєта? Цікавий погляд. І жодного стосунку до різних клубів.

\iusr{Вадим Горбов}

я все сприймаю, але до чого тут виборці і ваші власні електоральні симпатіі. Я
доречі голосував в першому турі за кандидата в президети Юрія Володимировича
Тимошенко із Коломиї

\begin{itemize} % {
\iusr{Людмила Гаценко}
\textbf{Вадим Горбов}

Певно, що я з розумінням і спокійно сприймаю все. І зауваження. Навіть не
йшлося про мої електоральні погляди. Просто констатація. Я поважаю усі погляди,
навіть абсолютно протилежні моїм. І в усіх аспектах життя. Кожен має право на
свою (якщо така є) думку.

Удачі й успіхів.

\iusr{Вадим Горбов}
навзаєм
\end{itemize} % }

\end{itemize} % }

\iusr{Наталья Полякова}
Вечная память невинно замученных!

\iusr{Наталья Полякова}
Понизили в должности! и они потом, как заслуженные пенсионеры, выращивали помидоры и ловили карасей на дачах. Господь справедлив! И он за все спросит каждого!

\iusr{Тетяна Юріївна Химерик}
Светлая память ему и всем погибшим в том числе и моим предкам в эти жестокие часы.

\iusr{Людмила Скомаровська}
Светлая и уважительная память. Амінь*****

\iusr{Tatyana Krivtsova}
Царство небесное погибшим на войне Добра и зла!@igg{fbicon.heart.suit}

\iusr{Вероника Руснак}
Я часто раньше бывала на его могилке

\begin{itemize} % {
\iusr{Ирина Кучер}
\textbf{Вероника Руснак} а какой это участок ? Если идти по центральной аллее ?

\begin{itemize} % {
\iusr{Вероника Руснак}
\textbf{Ирина Кучер} Я не помню участок, но это почти в конце, справа. И немного вглубь пройти.

\iusr{Вероника Руснак}
\textbf{Ирина Кучер} где-то за 38

\iusr{Ирина Кучер}
\textbf{Вероника Руснак} спасибо. У меня похоронены близкие мне люди на Лукьяновке . Обязательно схожу и найду захоронение

\iusr{Вероника Руснак}
\textbf{Ирина Кучер} У меня тоже там дед, папа и прабабушка. Я частенько захожу в церковь и на такие могилки

\iusr{Ирина Кучер}
\textbf{Вероника Руснак} и у меня дед , папа , родной дядя и два брата моего деда , погибшие в войну .
\end{itemize} % }

\end{itemize} % }

\iusr{Микола Веселий}

Нет ничего хуже быть песчинкой, бесправной, которую могут по доносу в любой
момент выслать в сибирские лагеря. Никакая колбаса по 2.20 это не компенсирует

\begin{itemize} % {
\iusr{Валерий Винарский}
\textbf{Микола Веселий} 

Вы путаете колбасу с лагеррями. При колбасе по 2.20 никаких лагерей НЕ БЫЛО Вы
ни при каких лагерях не жили. Прабабушки может жили но не вы. Чего так
переживаете? А лагеря вам сейчас нужны чтобы обелять национеалистичеое безумие
сегодняшней страны.

\begin{itemize} % {
\iusr{Микола Веселий}
\textbf{Валерий Винарский} 

Кто уцелел и вышел из лагерей, тому колбаса по 2.20 была небольшая радость.
Лагеря мне не нужны ни сейчас, ни потом. А помнить о них нужно, чтобы такое не
повторилось. И давали и Вам высказать свою точку зрение здесь публично, а не на
кухне. Никого я не обеляю, наоборот критикую, когда надо власть, в частности в
этой группе, которая о Киеве - по существу за отжатие киевской земли. И против
переименования сквера им. Чкалова - это самое свежее, лучше почитайте

\iusr{Галина Петрова}
\textbf{Валерий Винарский} ??? Стуса когда осудили?

\iusr{Валерий Винарский}

Сейчас памятью времени при котором никто из нас не жил пользуются те кто хочет
заслонить прежними давними временами МЕРЗОСТЬ творящуюся сегодня, как например
вялотекущий Голодомор, который умееньшил за 30 лет население Украины вдвое.

\end{itemize} % }

\end{itemize} % }

\iusr{Олександр Корнієнко}
Вічна й Світла ПАМ'ЯТЬ невинно закатованим і загиблим! @igg{fbicon.cry} 

\iusr{Елена Елена}
Сначала били самых родовитых.
Потом стреляли самых работящих.
Потом ряды бессмысленно убитых, росли из самых немолчащих...
Игорь Кохановский

\iusr{Maksim Pestun}

Поражаюсь некоторым нашим активным участникам, (не хочу называть фамилии)
которые уверены, что октябрьский переворот позволил многим проявить себя и был
самым светлым явлением в 20 веке. Не так важно, что они повторяют это вслед за
Путиным. Важно, что единственным положительным моментом в ряду беспримерных
трагедий, которые последовали за этим событием, весь мир узнал истинную цену
"освобожденному народу", который "проявил себя" во всей красе. К сожалению,
многие, особенно на Западе, стали забывать к чему подобное может привести. А
может и не знали вовсе, так как не имели "самолучшего в мире советского
образования"...

\begin{itemize} % {
\iusr{Валерий Винарский}
\textbf{Maksim Pestun} 

Советское образование было действительно самолучшим, если оно поззволило Вам
стать директором УТ. И вообще что Вы против своей личной жизни и советской
судьбы имеете - Чем Вас лично ущемил советский Союз? У меня например есть что
рассказать про советские УТЫСКЫ и КГБ. Так он мне хоть дал защиту от
уничтожения моей нации мою квартиру и канализацию. А что дала мне вильна й
незалежна? А-а-а-а - пенсию - 3000 и тарифы - 2500.

\begin{itemize} % {
\iusr{Maksim Pestun}
\textbf{Валерий Винарский} 

если бы ссср меня чем-то обделил лично, то мои рассуждения о нем были бы
предвзятыми и необъективными. Моя семья относительно мало пострадала от
советской власти, но я хорошо знаю, как пострадали миллионы других. Именно это
не дает мне спокойно относится к столь любимому Вами монстру. Я остро чувствую
чужую боль и ненавижу всю серость, ложь, ограниченность и подлость того строя,
управляемого людьми с низкими моральными качествами. Так была устроена система.
По другому было не пробиться на самый верх. У Вас же явный Стокгольмский
синдром

\iusr{Валерий Винарский}
\textbf{Maksim Pestun} 

Ненавидьте остро несправедливость но еще раз повторяю не смешивайте сталинский
и хрущовский режимы. Ведь при Хрущове Подгорном Шелесте Брежневе Коротченко
Кальченко прочим Ко и Щербицком Украина цвела и пахла - приросла огромными
чужими территориями которые завоевала не она а весь СССР и даже российским
Крымом. Будучи отстроенной за 5 послевоенных лет всем Советским Союзом, тем же
Союзом была спасена от чернобыльской катастрофы. Рязанским Псковским и
Ленинградским полкам пел самолично под стенами саркофага. Ваша ненависть к
прошлому работает на сегодняшний украинеоненавистнический режим в котором якобы
патриоты потому что с чубами и в вышиванках грабят ее как никаким коммунякам не
снилось. Потому что в 1991 году коммуняки сориентировались и перекрасились и
получили гораздо больше свободы воровать

\iusr{Даниэль Шапшович}
\textbf{Валерий Винарский}, 

Советская власть духовно убила наш народ и уничтожила еврейскую культуру. И
никогда Советская власть не ставила себе целью спасти евреев от физического
уничтожения. Это был побочный эффект от победы над нацистами. Похожим образом
охотник, стреляющий в лису, не ставит себе целью защитить зайца, но один из
результатов удачного выстрела в лису - спасение зайца.

\end{itemize} % }

\iusr{Евгений Александров}
\textbf{Maksim Pestun} Люди просто чтят часть своей истории со всеми ее плюсами
и минусами, а не вычленяют из нее одну грязь и ужас.

\begin{itemize} % {
\iusr{Maksim Pestun}
\textbf{Евгений Александров} если бы новая власть в ссср, а позже в России осудила бы преступления своих попередныкив, то тогда можно было рассматривать разные периоды. А перекрасившиеся большевики мало чем отличавшиеся по идеологии от предыдущих взяли на себя всю ответственность за прошлое. Так что, вычленять хорошое - себя обманывать!
\end{itemize} % }

\iusr{Вадим Горбов}
\textbf{Maksim Pestun} 

зависит от происхождения. Тут у нас в группе есть и потомки дворян и
недорезанных большевиками помещиков и буржуев. Я как правнук купца второй
гильдии, революции не одобряю и как внук профессора медицины, пролетариат мягко
говоря не люблю.

\begin{itemize} % {
\iusr{Светлана Манилова}
Вадим, а что делать тем, кто не относился ни к одним и ни к другим из Вами перечисленных? Мне, например. @igg{fbicon.smile} 

\iusr{Вадим Горбов}
\textbf{Светлана Манилова} 

так я тоже по другой линии родословной из бедного штетла как и вы.
кустарь-одиночка. И только после 1917 года нашего брата начали принимать в
университеты и разрешили селиться в Киеве и культурных центрах. И это тоже
бесспорный исторический факт.)

\end{itemize} % }


\end{itemize} % }

\iusr{Ксения Тычинская}

Спасибо за рассказ об этом священномученике, ничего не знала о нем. Жаль, что
даже в среде верующих людей Украины историю и подвиг новомучеников мало
вспоминают. К сожалению, такое возможно не только при советской власти, но при
любом режиме, сеющем распрю в народе. Дай Бог, чтобы больше не повторялось...

\begin{itemize} % {
\iusr{Max Gopencko}
\textbf{Ксения Тычинская} 

дело не только в распре, но и в методах режима. Одно дело, спровоцировать людей
на внутреннее противостояние, а совсем другое - убивать и ломать всех неугодных
режиму.

\begin{itemize} % {
\iusr{Клибановский Евгений}
\textbf{Max Gopencko} Тоже самое делают и сейчас! Только более гуманно!

\iusr{Max Gopencko}
\textbf{Клибановский Евгений} ага, прям более гуманные "тройки" миллионами расстреливают людей, предварительно заставляя копать их себе могилы ака в Сандармохе.

\iusr{Олег Сошевский}
\textbf{Max Gopencko} ,да тройки и сейчас сидят в вечернюю пору в ресторанах и решают схемы дальнейшего развития собственного бизнеса путем грабежа украинского народа.
Люди стали чаще умирать, и не только из-за эпидемии, но и от голода и болезней.

\iusr{Max Gopencko}
\textbf{Олег Сошевский} вы на полном серьёзе сравнили то, что происходит сейчас, с массовыми расстрелами и геноцидом-Голодомором миллионов людей? Если да, то у меня для вас плохие новости.

\iusr{Олег Сошевский}
\textbf{Max Gopencko} ,видно для вас это тоже новость, что геноцид бывает разный.

\iusr{Max Gopencko}
\textbf{Олег Сошевский} видимо это для вас новость, что при нынешнем "геноциде" есть возможность выжить, а не когда тебя к стенке ставят или собранный трудом урожай отбирают.
\end{itemize} % }

\iusr{Людмила Волжанка}
\textbf{Ксения Тычинская} все повторяется

\iusr{Клибановский Евгений}
\textbf{Людмила Волжанка} увы 8(

\iusr{Ксения Тычинская}

Друзья, может хватит спорить, при каком режиме хуже? Ну какое-то распыление
ресурса, который можно потратить на что-то более полезное (и приятное).  @igg{fbicon.wink} 
\end{itemize} % }

\iusr{Валентина Мойсеєнко}
Переименовать одну из улиц !!!!!!

\iusr{Любовь Лукасевич}
На жаль такі дописи не читають ті що хочуть железной руки сталіна.

\begin{itemize} % {
\iusr{Люба Микицька}
\textbf{Любовь Лукасевич} не тільки читають а й коментують. Тут їх більшість.(((
\end{itemize} % }

\end{itemize} % }
