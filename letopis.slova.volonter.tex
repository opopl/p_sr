% vim: keymap=russian-jcukenwin
%%beginhead 
 
%%file slova.volonter
%%parent slova
 
%%url 
 
%%author_id 
%%date 
 
%%tags 
%%title 
 
%%endhead 
\chapter{Волонтер}
\label{sec:slova.volonter}

%%%cit
%%%cit_head
%%%cit_pic

\ifcmt
  tab_begin cols=3

     pic https://regnum.ru/uploads/pictures/news/2021/10/29/regnum_picture_16354937487650433_normal.jpg

     pic https://regnum.ru/uploads/pictures/news/2021/10/29/regnum_picture_16354937648279337_normal.jpg

		 pic https://regnum.ru/uploads/pictures/news/2021/10/29/regnum_picture_16354938438476921_normal.jpg

  tab_end
\fi
%%%cit_text
После получения заявки \emph{волонтеры} Александр Калашник и Анна Анцупова отправились
закупать необходимое. По правилам, в заказе не может быть алкогольной и
табачной продукции. Но любые хозяйственные вещи первой необходимости \emph{волонтёры}
закупят. В этот раз бабушка вместе с продуктами заказала миски, щетки, порошки.
Напрямую из магазина \emph{волонтеры} корректируют заказ по телефону, а в конце
сверяют весь список и сумму заказа
%%%cit_comment
%%%cit_title
\citTitle{Волонтёры Москвы вышли на помощь — фоторепортаж}, 
Ярослав Чингаев, regnum.ru, 29.10.2021
%%%endcit

%%%cit
%%%cit_head
%%%cit_pic
%%%cit_text
З 2015 року я працюю з \emph{волонтерами}. Мій чоловік конт­рактний військовий. Я дуже
добре знаю цю тему. Мала цього літа нагоду проїхатися разом з волонтерами
Донеччиною. Об’їхали практично всю доступну зону навколо Донецька. Відвідували
військові частини, цивільних. У Покровську були у релігійному містечку, де
протестанти опікуються біженцями з зони бойових дій. А також в інтернаті в
Залізному, де опікуються дітьми. Мовна і демографічна ситуація просто
катастрофічна.  Якщо говорити про інформаційну війну, то ми не воюємо, а
програємо на всіх фронтах. У людей немає українського телебачення, радіо,
української преси. Позбавлені його і військові частини. Тому що в Луганську і
Донецьку стоять потужні глушники. Транслюється лише все російськомовне
%%%cit_comment
%%%cit_title
\citTitle{Як виграти інформаційну війну? – Слово Просвіти}, , slovoprosvity.org, 01.11.2021
%%%endcit

%%%cit
%%%cit_head
%%%cit_pic
\ifcmt
  tab_begin cols=3
     pic https://storage.lug-info.com/cache/d/a/6efdef44-e347-44e1-9ed3-865786128efb.jpg/w700h474%7Cwm
     pic https://storage.lug-info.com/cache/d/6/13a2ecc1-e47c-4df3-83e9-f9ecaa7ec399.jpg/w1000h616%7Cwm
		 pic https://storage.lug-info.com/cache/4/e/d9678e05-e6c4-4f8a-96cd-0b81bd90da52.jpg/w1000h616%7Cwm
  tab_end
\fi
%%%cit_text
Торжественное мероприятие, посвященное Дню добровольца (\emph{волонтера}), состоялось
в Луганском академическом русском драматическом театре имени Павла Луспекаева.
Об этом с места события передает корреспондент ЛИЦ.
За участие в акции \enquote{Рука помощи} общественного движения (ОД) \enquote{Мир Луганщине},
развитие волонтерского движения и значительный личный вклад в патриотическое
воспитание подрастающего поколения благодарностями главы ЛНР отмечены Екатерина
Гунченко, Светлана Мовчан, Никита Нечуваев, Юлия Шинкарева и Ирина Шкуренко.
Еще 40 активистов получили грамоты и благодарности Правительства, Министерства
культуры, спорта и молодежи ЛНР и ОД \enquote{Мир Луганщине}.
\enquote{Что значит быть \emph{волонтером}? Это значит быть человеком с открытым сердцем,
бескорыстным, неравнодушным, тем человеком, который первым подаст руку помощи
тому, кто больше всего в ней нуждается. Вы делаете наш мир лучше, добрее}, -
сказала заместитель председателя Правительства ЛНР Анна Тодорова
%%%cit_comment
%%%cit_title
\citTitle{Луганский Информационный Центр – Почти полсотни активистов получили награды от властей ЛНР ко Дню волонтера}, 
, lug-info.com, 06.12.2021
%%%cit_url
\href{https://lug-info.com/news/pochti-polsotni-aktivistov-poluchili-nagrady-ot-vlastej-lnr-ko-dnyu-volontera}{link}
%%%endcit
