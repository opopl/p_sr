% vim: keymap=russian-jcukenwin
%%beginhead 
 
%%file 05_09_2021.fb.druzenko_gennadiy.1.revolution_on_human_scale
%%parent 05_09_2021
 
%%url https://www.facebook.com/gennadiy.druzenko/posts/10158498879213412
 
%%author_id druzenko_gennadiy
%%date 
 
%%tags chelovechnost,chelovek,future,obschestvo,revolucia,strana,ukraina
%%title REVOLUTION ON HUMAN SCALE
 
%%endhead 
 
\subsection{REVOLUTION ON HUMAN SCALE}
\label{sec:05_09_2021.fb.druzenko_gennadiy.1.revolution_on_human_scale}
 
\Purl{https://www.facebook.com/gennadiy.druzenko/posts/10158498879213412}
\ifcmt
 author_begin
   author_id druzenko_gennadiy
 author_end
\fi

REVOLUTION ON HUMAN SCALE.

Наразі ми в Центр конституційного моделювання / Center for Constitutional
Design завершуємо український переклад програмної статті Брюса Акермана, назву
якої ми переклали як «Революція з людським обличчям». 

Бо революції - це не завжди про ріки крові, але завжди про зміну системи
координат. Під цим кутом зору, революція в Україні була лише одна, причому саме
така, як її описує єльський професор: мирний перехід від брежнєвського
соціалізму до олігархічного капіталізму. 

\ii{05_09_2021.fb.druzenko_gennadiy.1.revolution_on_human_scale.pic}

В результаті цієї революції ми кардинально розширили простір свободи і майже
настільки ж кардинально втратили в соціальній справедливості. Відкрили для себе
глобальний світ і майже розучились мислити глобальними викликами та амбіціями.
Розгризли пуповиння, яке єднало нас з імперією, яку творили наші предки, але в
череві якої нам у ХХ столітті явно стало затісно. 

Після того, як ми пустились імперського берега, ми окрім соціальної, пережили
не менш радикальну технологічну революцію: найближчим конфідентом, другом і
порадником наших дітей став смартфон. Але технологічна революція не змінила
суті української системи: в ній забагато вольниці і замало правил та
справедливості. Бо як правильно сформулював Пікетті, за дикого капіталізму
(вільного ринку) багаті стають ще багатшими, а бідні - ще біднішими. 

Ми потребуємо на другу революцію з людським обличчям. Яка збалансує свободу
відповідальністю, можливості - справедливістю, а приватний успіх - спільним
добром. Яка перетворить український правовий нігілізм у переконання, що за
правилами жити вигідно. І врешті-решт дасть нам відчуття суб’єктності у не
надто дружньому і солідарному світі. 

Я щиро переконаний, що революції народжуються з розмов. Це саме той випадок,
коли спочатку було слово. Коли малі спільноти народжують та вербалізують нові
сенси та наративи. А потім ці наративи раптом заливають масову свідомість як
гураган Іда штат Нью Джерсі…

Тому щиро дякую Ігор Шевченко за його ініціативу RoofParty і цікаве товариство,
що збирається в нього на даху.

Далеко не з усіх розмов народжуються революційні зміни. Але вони точно не
народжуються з життя в телевізорі чи Фейсбуці. 

До того ж RoofParty має одну безсумнівну перевагу перед ОПУ: погляд з Ігоревого
даху набагато стратегічніший аніж з Банкової чи Грушевського. Перспектива
набагато ширша 🙂.

\ii{05_09_2021.fb.druzenko_gennadiy.1.revolution_on_human_scale.cmt}
