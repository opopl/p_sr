% vim: keymap=russian-jcukenwin
%%beginhead 
 
%%file 04_11_2021.fb.fb_group.story_kiev_ua.1.semja_kiev_okupacia
%%parent 04_11_2021
 
%%url https://www.facebook.com/groups/story.kiev.ua/posts/1790248867838573
 
%%author_id fb_group.story_kiev_ua,bubnov_jurij
%%date 
 
%%tags gorod,istoria,kiev.okupacia.1941,nacizm,okupacia,pamjat,semja,ukraina,vospominanie
%%title Моя семья. Жизнь и приключения во время немецкой оккупации
 
%%endhead 
 
\subsection{Моя семья. Жизнь и приключения во время немецкой оккупации}
\label{sec:04_11_2021.fb.fb_group.story_kiev_ua.1.semja_kiev_okupacia}
 
\Purl{https://www.facebook.com/groups/story.kiev.ua/posts/1790248867838573}
\ifcmt
 author_begin
   author_id fb_group.story_kiev_ua,bubnov_jurij
 author_end
\fi

Информация только для администратора группы. *Более двух лет назад, я сделал
несколько публикаций о жизни моей семьи в оккупированном Киеве, основанных на
их собственных воспоминаниях. Теперь же, хотелось бы повторить две из них, в
основном, по двум причинам. Первая - в это время, 78 лет назад, 7 ноября 1943
года Киев был освобождён от немецких оккупантов, и вторая немаловажная причина
- за прошедшие два года количество участников группы значительно увеличилось и,
возможно, новым читателям это покажется интересным. Вопрос: не нарушаю ли этим
я правила группы? Далее - текст публикации*.

Ранее, я уже публиковал воспоминания моих родных о первых часах и днях
оккупации Киева немецкими войсками. Эти истории я назвал "Хлебопекарня" и
"Подписка". Что происходило далее, с конца 1941 по август 1943 года, как они
жили, а точнее выживали в это страшное время - ничего в моей памяти не
сохранилось, кроме каких-то, не связанных между собой, эпизодов. Очень сожалею.
Но всё-таки хорошо помню их рассказы о том, как в конце лета 1943 года, они и
другие жители домов на Евбазе, попавших по приказу немецких властей в запретную
зону, вынуждены были покинуть свои жилища и отправиться в никуда, не зная, что
их ждёт впереди. Об этом я и хочу рассказать. 

Моя семья. Жизнь и приключения во время немецкой оккупации. 

Небольшое, но как мне кажется, необходимое предисловие. 

Комментируя воспоминания моих родных и близких о жизни в оккупированном Киеве,
некоторые авторы иногда задают вопросы, как им кажется, с патриотических
позиций (часть цитирую дословно): "А почему это ваши родители не эвакуировались
вместе со всеми? Они ждали прихода фашистов?" "Почему ваш отец не ушёл на фронт
сражаться против захватчиков, а остался на оккупированной территории?" "Все,
оставшиеся в оккупации так или иначе помогали врагам, и таким образом, являются
предателями" "Вам не стыдно за своих родителей?" и т.п. Отвечать на подобные
комментарии считаю бессмысленной тратой времени. Не объяснять же им, что
подавляющее большинство населения, в данном случае Киева, осталось на
оккупированной территории не по своей воле. Для тех, кого это действительно
интересует, открыты архивы и собрано масса свидетельств участников этих
событий. Но у них на этот счёт своя, ничем непробиваемая позиция: раз остался –
предатель. Хотя я уверен, что каждый может иметь и свободно высказывать свою
точку зрения на какие-либо события, мне становится грустно, когда люди страдают
извращенными представлениями о нашем прошлом. Что же касается непосредственно
моих родных, ответ можно найти в моей публикации "Хлебопекарня".

\ifcmt
  tab_begin cols=3

     pic https://scontent-frx5-2.xx.fbcdn.net/v/t1.6435-9/253482008_1049707202488060_4326260091571865177_n.jpg?_nc_cat=109&ccb=1-5&_nc_sid=b9115d&_nc_ohc=RrWTyhuhw8QAX8du4Dx&_nc_ht=scontent-frx5-2.xx&oh=2fe99a200fbb894e4db56411d511cd94&oe=61A9007E

     pic https://scontent-frx5-1.xx.fbcdn.net/v/t1.6435-9/252007851_1049707362488044_9057746978164008781_n.jpg?_nc_cat=111&ccb=1-5&_nc_sid=b9115d&_nc_ohc=InvrTkdGyqwAX8ItnjM&_nc_ht=scontent-frx5-1.xx&oh=1df506d31d9806bafddcdb1eac6f601e&oe=61ACB10C

		 pic https://scontent-frt3-1.xx.fbcdn.net/v/t1.6435-9/252142832_1049707452488035_3805340209224081979_n.jpg?_nc_cat=104&ccb=1-5&_nc_sid=b9115d&_nc_ohc=XbrL5Kob8wkAX92iZ3G&_nc_ht=scontent-frt3-1.xx&oh=343f3423c0a802021b8828a0d10e6174&oe=61AB341D

  tab_end
\fi

После окончания войны, те кто оставались на временно оккупированной территории
ещё долго находились под пристальным вниманием соответствующих органов. Многие
помнят, как вплоть до 80-х при поступлении на работу даже простым рабочим или
уборщицей в какую-то зачуханную контору, при заполнении анкеты должны были
ответить на вопрос: "Находились ли Вы и Ваши близкие родственники на временно
оккупированной территории?" или "Были ли Вы и Ваши близкие родственники в плену
или интернированы в период Великой Отечественной войны?". 

\ifcmt
  tab_begin cols=3

     pic https://scontent-frt3-1.xx.fbcdn.net/v/t1.6435-9/252682298_1049707735821340_3940490509802199056_n.jpg?_nc_cat=108&ccb=1-5&_nc_sid=b9115d&_nc_ohc=Dj2aLa0RQ6UAX-8G2Uv&_nc_ht=scontent-frt3-1.xx&oh=56e7252211452b895700287cb4290018&oe=61AA5FE5

     pic https://scontent-frt3-1.xx.fbcdn.net/v/t1.6435-9/252018299_1049707825821331_7723014161120459105_n.jpg?_nc_cat=107&ccb=1-5&_nc_sid=b9115d&_nc_ohc=bJPkmh69bycAX9Kz49x&tn=lCYVFeHcTIAFcAzi&_nc_ht=scontent-frt3-1.xx&oh=a24e9bd63a0b8c80c8d5b239955b0d8f&oe=61AAFE89

		 pic https://scontent-frx5-2.xx.fbcdn.net/v/t1.6435-9/252279342_1049707949154652_1343918452851876903_n.jpg?_nc_cat=109&ccb=1-5&_nc_sid=b9115d&_nc_ohc=uzLDqrVPc8QAX8rMAD7&tn=lCYVFeHcTIAFcAzi&_nc_ht=scontent-frx5-2.xx&oh=856d5eae7f1cd3d7d5a074c06bff9624&oe=61AA1A97

  tab_end
\fi

Для справки. Данные взяты из архивных документов, на которых ещё до недавнего
времени стоял гриф "Секретно". До начала войны население Киева составляло
приблизительно 846 тысяч человек. Около 200 тысяч киевлян были мобилизованы в
армию и ушли на фронт – подавляющее их большинство погибло. Еще 325–350 тысяч
человек эвакуировались вместе с предприятиями и организациями столицы. Около
100 тысяч человек во время оккупации было вывезено в Германию на принудительные
работы.

Кроме того, десятки тысяч людей погибли от голода и холода, попали в число
убитых во время чисток по национальному признаку или были расстреляны как
заложники. Часть населения бежала в близлежащие села, спасаясь от голодной
смерти. В итоге, из оставшихся в городе 400 тысяч человек в ноябре 1943 года
освободителей встречало лишь 180 тысяч киевлян, находившихся в оккупации. Не
много ли для предателей?

Итак, начну с небольших отрывков из опубликованных воспоминаний Валентина
Бубнова - моего сводного брата, так как в дальнейшем это поможет разобраться в
некоторых ситуациях. 

Часть 1. Валентин Бубнов, 1935 года рождения.

«До войны я жил с родителями в одной из комнат коммунальной четырехкомнатной
квартиры двухэтажного флигеля по улице Менжинского (Дмитриевской), 16. Моей
маме, Лие Авраамовне, минул 31 год. Родом она была из небольшого городка
Новоград-Волынский, или самого большого местечка в мире, как посмеиваясь,
говорил отец - Григорий Лукич — русский, уроженец Одессы. Отец работал
механиком в хлебопекарне, расположенной здесь же, в нашем дворе, а мама –
кассиром в аптеке поблизости.

Наш двор был самым большим на улице Дмитриевской в районе Евбаза и состоял из
двух неравных частей. В основной части двора, которую мы называли большим,
находилась хлебопекарня. Вторая часть представляла собой узкий невзрачный
дворик, заканчивающийся большим подземным погребом. В трех двухэтажных флигелях
нашего двора было множество коммунальных квартир и, как правило, в проживающих
там семьях было по одному-два ребенка. Самые приятные воспоминания - это
хождение друг к другу в гости на дни рождения вместе с родителями. Одними из
наших соседей по квартире была еврейская семья - пожилая женщина, все ее звали
Куцька, и ее сын Пиня. Однажды, Пиня спас меня от верной смерти, вытянув из
моей глотки, застрявший там заводной ключик от машинки, который я случайно
проглотил.

Летом 1941 года, после начала войны, все еврейское население нашего двора
начало постепенно эвакуироваться. Оставались, в основном, семьи смешанных
браков. К ним относилась и наша семья. Родители и сестры моей мамы, жившие в
Новоград – Волынске, уговаривали ее эвакуироваться, но мама решила остаться.
Такое же решение приняли и многие другие «смешанные» семьи в нашем дворе. Они
убеждали друг друга, что слухи о зверствах немцев по отношению к евреям очень
сомнительны – не могут же цивилизованные люди убивать женщин и детей.

Однажды, это было сразу же после оккупации Киева немецкими войсками, раздался
громкий стук в двери нашей квартиры. Мама открыла дверь, и в комнату вошли два
вооруженных немецких солдата. Один остался возле дверей, а второй начал что-то
маме объяснять. Мама предложила им выпить чаю, они не отказались. Между ними и
мамой велась беседа на немецком языке, которым мама хорошо владела. Солдаты
вели себя корректно, никаких угрожающих действий не проявляли. Вскоре они ушли.
А через несколько дней был Бабин яр и мамы не стало. На мои вопросы, где мама,
отец отвечал: - Мама уехала к дедушке в Новоград – Волынский, и вернется не
скоро.

С тех пор, по непонятной мне причине, отец запретил выходить во двор, и я был
вынужден целый день сидеть в квартире. Вскоре, мы с отцом переехали жить из
нашего двухэтажного флигеля в большой пятиэтажный дом в том же дворе. Квартира
находилась на четвертом этаже и проживала там семья Голубояр - Нина
Константиновна - моя новая мама, ее мама – бабушка Настя и бабушка Маня, сестра
бабушки Насти, с мужем, Павлом Яковлевичем. Дома меня быстро крестили, повесили
на шею крестик, а бабушка Настя научила читать первые строки «Отче наш». В то
время я не понимал, что в случае, если меня обнаружат немцы, вся моя новая
семья будет расстреляна.

Однажды, случилось то, чего больше всего опасались в моей новой семье. Наш
сосед по двору А. (Примечание Ю. Бубнова. Валентин называет фамилию полностью.
Я же, сокращаю её умышленно) донес жившему в нашем доме «фольксдойче» Шнайдеру,
что этажом выше живет коммунист (отца исключили из партии в 1937 году) и
«жи...ок», то есть я. Шнайдер, которого часто, по словам соседей, посещали
немцы в черных мундирах (эсэсовцы), ответил, что этот вопрос не в его
компетенции и передал разговор моему отцу, посоветовав быть осторожнее.

А. продолжал свою работу доносчика. Кроме этого, он вербовал молодежь нашего
двора на работу в Германию и его дочь стала первой, кто туда уехал. 

После освобождения Киева, А. устроился на работу завскладом, но проворовался и
получил десять лет тюрьмы. Отсидев всего три года, вернулся домой, прожил
несколько месяцев и умер. В 1945 году вернулась его дочь, отправленная по
инициативе своего отца в Германию. Приехала не одна, с ней была маленькая
дочурка. Сын А. в том же году вернулся из немецкого плена, спился и вскоре
умер».

Благодаря рассказам моей мамы, бабушек и воспоминаниям моего сводного брата, в
памяти сохранились некоторые эпизоды жизни моей семьи во время оккупации, в
частности, периода сентябрь – ноябрь 1943 года, которые для более удобного
восприятия событий, я попытаюсь объединить в один рассказ.

Не помню, чтобы об рассказывали мужчины, в частности, мой отец и дед, он же мой
крёстный. Вспоминали только женщины. Да собственно, и женщины вели разговоры на
эту тему негромко и «на кухне». Иногда, у нас в гостях собирались друзья и
знакомые, вместе с которыми мои родители пережили ужасы оккупации. Больше всех
мне запомнились наши соседи по дому и двору: Жолковские – тётя Аня с мужем и
сыном Славиком и Атамановские – дядя Сева и тётя Надя с сыном Юрой. После
немного выпитого (следует сказать, в нашей семье спиртные напитки, мягко
говоря, не слишком уважали, но для гостей делали исключения), взрослые часто
начинали застольные разговоры "А помнишь?" и, перебивая друг друга, но всё-таки
в пол голоса (срабатывала привычка быть осторожными) начинались воспоминания о
"тех временах". Мне бы слушать и запоминать эти разговоры, так нет же – голова
была занята другим.

Часть 2. Запретная зона

В сентябре 1943 года среди жителей Киева начали распространяться слухи о
наступлении Красной армии и поражениях немцев. Говорили, что бои идут уже в
районе Броваров. Источником этих слухов, как всегда, был наш базар, наш Евбаз.
Как бы подтверждая это, по приказу военного коменданта оккупационных властей, в
Киеве ряд районов был объявлен запретной зоной. Нам не повезло - наш дом на
Воровского, 51 и весь двор оказался внутри этой зоны, граница которой в районе
Евбаза проходила по улицам Дмитриевской и Саксаганского. На основании этого
приказа, все жители, чьи дома расположены в этих пределах, к вечеру 26 сентября
должны ее покинуть. За пребывание в запретной зоне гражданских лиц без особого
разрешения – расстрел. На выполнение приказа было дано не более двух-трех дней.

Нас было семеро – отец, мама, бабушка Настя, ее сестра Мария с мужем, Павлом
Яковлевичем - моим дедушкой, он же мой крестный (Ковган П.Я. - главное
действующее лицо моей публикации «Подписка»), мой сводный брат Валя и я,
возрастом неполных одиннадцати месяцев.

Начали лихорадочно собираться, даже не представляя себе, куда мы должны ехать.
Первая мысль была – что взять с собой из вещей - уже сейчас холодно, а на носу
зима. Таким образом, вопрос о легкой одежде решился сам собой. А нужно ли брать
с собой кастрюли, сковородки и другие кухонные принадлежности? А на чем спать,
чем укрываться? Решали целый день. О запасе продуктов речь не шла – их просто
не было. Наконец, кое-как собрались, упаковали чемоданы и узлы с вещами, что
могли – одели на себя. Встал вопрос: а как мы все это понесем? Отец тут же
бросился на базар, который находился буквально метрах в двадцати от нашего
двора – стоило только перейти Дмитриевскую. Вернулся часа через два, довольный
– ему удалось выменять или купить (уже никто не помнит) небольшую двуколку –
тележку на двух колесах.

И тут же возникла большая проблема. Как быть с Валей? Более двух лет родные
прятали его, на улицу и во двор он не выходил, никто из соседей его не видел. О
нем знали только самые близкие друзья моих родителей, в том числе крестный отец
и крестная мама Вали. Все хорошо, но как быть с внешностью? А внешность у него,
говоря откровенно, не славянская. Нашли выход. Погода стояла уже достаточно
прохладной, и голову брата закутали так, что видны были только глаза. В общем,
на этом этапе все обошлось. 

Примечание моё. Как-то я пробовал подсчитать, сколько же человек знали о том,
что мои родные прячут еврейского мальчика. Оказалось – человек пятнадцать,
вместе с моими родственниками и соседями. Немало. Почти всех знаю по именам и
фамилиям, некоторых помню только визуально. И всем им грозила смертная казнь за
укрывательство одного еврейского мальчика. Вскоре, после освобождения Киева,
мама усыновила Валю. Так у меня появился сводный брат. А в 1995 году, моей
маме, Нине Константиновне Бубновой (Голубояр), посмертно было присвоено звание
«Праведник Бабьего Яра». Подробнее см. «Бабий яр. Спасители и спасенные». — К.,
2005, стр. 49; «Бабий яр. Книга памяти». — К., 2005, стр. 233-234. Хотя, должен
сказать, там есть значительные ошибки (опечатки?) в датах.

Отец впрягся в двуколку, две бабушки и Валя толкали ее сзади, мой крестный нес
чемоданы, а меня на руках несла мама. Выехав со двора на Дмитриевскую, тут же
влились в громадный поток таких же как мы, переселенцев. Картина была ужасная –
многие везли детей, по двое, а то и по трое, в одной коляске, а больных и
стариков, которые сами не могли ходить – несли на закорках. Провожающих не было
– уходили все. Слухи ходили самые разные, но все сводились к одному - идем на
вокзал, где нас ждут товарные вагоны для эвакуации. Правда, никто не знал куда
нас повезут. В общей толпе мы медленно шли по Дмитриевской, пересекли Галицкую
площадь и вышли на ул. Саксаганского, затем повернули направо на ул.Коминтерна,
откуда уже был виден вокзал. Люди с узлами, тележками, колясками – все это
понемногу двигалось, потом стояло, потом опять двигалось. Слышался сильный гул
толпы, и все это было похоже на демонстрацию нищих. Начиная от Коминтерной,
дорога была оцеплена немецкими солдатами с двух сторон, и выйти оттуда уже было
невозможно. Вдруг, пронесся слух, что всех отправляют в Германию. До вокзала
оставалось каких-то метров пятьсот. Нас охватила паника. Отчаявшись, отец
бросился к одному из солдат оцепления и стал жестами просить, чтобы тот нас
выпустил. Неизвестно, что подействовало на солдата – то ли вид нашей «команды»,
то ли орущий младенец на руках матери, то ли еще что-то, но оглянувшись по
сторонам (а расстояние между солдатами было метров пятнадцать), он скомандовал:
«Weg!». И мы, как рысаки, вырвались из общего потока сначала на улицу
Жилянскую, а затем, проходными дворами и короткими перебежками, оказались опять
на Евбазе. И тут мама вспомнила, что ее хорошая знакомая живет на улице
Володарского, 3. Прибежали туда. Слава Богу, этот дом, больше похожий на барак,
находился за запретной зоной. Несмотря на то, что все помещения были
переполнены, мамина знакомая помогла нам разместиться. В доме одну половину
занимали мужчины, другую – женщины. Ночь прошла спокойно.

На следующее утро, знакомый хозяйки полицай предупредил её, что будет очередная
облава – ищут мужчин, уклоняющихся от работы. Узнав об этом, отец и его товарищ
Сева Атамановский (я помню его как "дядю Севу"), который жил в нашем дворе,
сделали себе в ноги подкожную инъекцию скипидаром. То же самое проделали и
другие мужчины. Через два часа ноги у всех посинели и распухли, поднялась
температура, запах в помещении стоял жуткий. На следующий день пришли немцы.
Начали сбрасывать одеяла со всех лежащих «больных» и, увидев эти синие,
«гниющие» ноги, быстро ушли. Через два-три дня, опухоль значительно
уменьшилась, синева исчезла, температура упала. У кого-то нашелся самогон, и
жители барака отпраздновали свое «освобождение».

Часть 3. Бегство

Через несколько дней, опасаясь следующих облав, отец принял решение бежать из
города. Шли долго, почти без отдыха, пока не добрались до села Борщаговка. В
селе пробыли чуть больше недели, после чего вынуждены были уйти - там тоже
начались облавы. Какое-то время жили еще в двух маленьких селах, названия
которых никто не вспомнил. Покинув последнее место ночевки, поздно вечером,
когда было уже совсем темно, добрались до села Белогородка, которое находилось
в километрах двадцати – двадцати пяти от Киева. Было очень холодно и хотелось
есть. Постучались в несколько хат - в одних хозяева не открывали, в других
говорили, что мест уже нет, все занято беженцами. И только на противоположной
стороне села, в одном из последних домов, нам разрешили остаться. Внутри было
темно, свет падал только с улицы – светила луна. На топчанах и прямо на полу,
вповалку лежали люди, кто-то храпел, кто-то громко разговаривал, кто-то курил.
Запах стоял ужасающий, но нам, после всего пережитого, показалось, что мы в раю
– было тепло и очень хотелось спать. Увидев, что у мамы на руках маленький
ребенок, хозяйка провела ее в какой-то закуток, где оказалось немного
свободного места. Мама легла прямо на пол, постелив под меня свое пальто, а с
другой стороны примостился брат. Отец и все остальные были где-то рядом. Меня
раздирало от кашля уже две недели. Иногда заходился до такой степени, что все
думали, я задохнусь. Не помогало ни обтирание самогоном, ни отпаивание теплой
водой, а молока, кроме грудного, не было. И здесь, в этой хате, я надрывался от
кашля. Какой-то мужчина, из глубины комнаты закричал маме: «Заткни своєму
виродкові пельку, або я придушу його». Я продолжал кашлять, и через минуту,
наступая прямо на лежащих, он подошел к маме и нагнулся над ней. У моего отца
был очень тяжелый кулак. Мужчина упал. Тут же, к нему подбежали двое и
выволокли из хаты. Когда все немного успокоились, к маме подошла хозяйка, с
кружкой теплого молока и говорит: «Дай це йому випити, бо інакше він помре».
Случилось чудо – ни утром, ни в следующие дни, я уже не кашлял. Может там было
не только молоко.

Примечание моё. Как-то летом, а это была середина 50-х, и я уже был учеником
пятого или шестого класса, когда играл с друзьями во дворе, с балкона меня
позвала мама. Прихожу домой. В комнате, за накрытым столом, сидит вся семья и
какая-то незнакомая женщина. Мама говорит: «Юра, подойди сюда, познакомься. Это
тётя….. (называет её имя, которое я, к сожалению, не запомнил) из Белогородки.
Если бы не она, ты мог бы умереть». Женщина заплакала, погладила меня по голове
и сказала: «Ти був такий худюсенький, так бухикав, думала, що помреш». Вскоре я
об этом забыл, и только спустя много лет, услышав от родных, что же произошло в
Белогородке, вспомнил эту женщину. Как они нашли друг друга? Для меня до сих
пор загадка.

Хата, в которой мы жили, находилась почти на окраине села, на небольшой
возвышенности, откуда был виден вдалеке слева лес, где находились немцы, прямо
– большое колхозное поле, а за ним – шоссе.

А дня через два произошло следующее. Днем, со стороны Киева по направлению к
лесу полетели огненные стрелы, и сразу же раздался жуткий вой, от которого все
пришли в ужас (так рассказывала мама). Огненные стрелы падали на лес, где
находились немцы, и очень быстро лес превратился в большой пылающий костер.
Местные жители и многие беженцы упали на колени и начали креститься. Одна из
беженок кричала: «Это господь Бог карает грешников геенной огненной». Конечно,
никто из нас не знал, что это были «Катюши». Внезапно вой прекратился и
наступила тишина. Через какое-то время послышался звук приближающейся машины, и
кто-то из местных, внизу с дороги, закричал: «Сюди, сюди! Наші! Наші!». Все,
как по команде, бросились на крик. Спустившись в низину, увидели броневик, а
возле него трое вооруженных солдат в шапках-ушанках, полушубках и валенках.
Когда подбежали поближе, вдруг кто-то крикнул: «Так це ж переодягнені німці!».
Все бросились бежать. Оказывается, на плечах солдат были погоны – никто же не
знал, что в Красной армии вместо кубиков и ромбиков были введены погоны. Когда
один из солдат, выругавшись матом, закричал, что они свои, все остановились, а
затем бросились их обнимать. Поговорив о чем-то с местными, солдаты сели в
броневик и уехали. А ночью, прямо на поле перед селом начался бой – были слышны
выстрелы, взрывы и крики. Женщины с детьми спрятались в подвале дома, а
мужчины, укрывшись, кто где смог, наблюдали за боем. К утру выстрелы
прекратились, и мы опять услышали звук приближающихся машин. Радостные,
выскочили на улицу, и здесь же остановились – это были немцы. Выйдя из машин,
они начали обыскивать каждую хату, подвалы и дворы. У всех мужчин проверили
документы – видимо, искали красноармейцев. К счастью, все обошлось. Часа через
два, солдаты сели в машины и уехали по направлению к лесу.

Ближе к вечеру, опять начали стрелять «Катюши», и на поле снова были слышны
выстрелы и взрывы, а рано утром в Белогородку вошли наши части. Узнав от
красноармейцев, что Киев освобожден, отец приказал нашей «команде» немедленно
собираться в дорогу и возвращаться домой. Нам предстояло идти в сторону шоссе
через большое поле, на котором ночью произошел бой. 

(Продолжение следует)

Фотографии.

1. 1938 год. Мой отец – Бубнов Григорий Лукич. Фотография из семейного архива.

2. 1941 год. Валя Бубнов. Фотография из семейного архива.

3. 1942 год. Венгерские солдаты на Галицкой площади. На заднем плане – ул.
Менжинского (Дмитриевская). Высокие дома - мой дом – Дмитриевская, 16
(Воровского, 51) и Дмитриевская, 10. Фотография из открытых источников.

4. 1943 год. Обычный день на Евбазе. Фотография из открытых источников.

5. 1942 год. ул. Менжинского (Дмитриевская). Чуть в глубине снимка, правее -
ул. Чкалова. Несмотря на строгие запреты немецких властей, торговля шла и за
пределами непосредственно самого базара. Фотография из открытых источников.

6. 1942 год. Продавец грампластинок на Евбазе. Фотография из открытых
источников.

\ii{04_11_2021.fb.fb_group.story_kiev_ua.1.semja_kiev_okupacia.cmt}
