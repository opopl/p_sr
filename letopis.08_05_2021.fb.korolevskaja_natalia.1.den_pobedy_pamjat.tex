% vim: keymap=russian-jcukenwin
%%beginhead 
 
%%file 08_05_2021.fb.korolevskaja_natalia.1.den_pobedy_pamjat
%%parent 08_05_2021
 
%%url https://www.facebook.com/Nataliya.Korolevska/posts/4205663026161342
 
%%author 
%%author_id 
%%author_url 
 
%%tags 
%%title 
 
%%endhead 

\subsection{Оба моих деда воевали в Великой Отечественной Войне}
\label{sec:08_05_2021.fb.korolevskaja_natalia.1.den_pobedy_pamjat}
\Purl{https://www.facebook.com/Nataliya.Korolevska/posts/4205663026161342}

Оба моих деда воевали в Великой Отечественной Войне. 

Шамрай Пётр Игнатьевич - дедушка по маминой линии был призван на фронт из
Чкаловского лётного училища, куда после окончания школы поступил «покорять
небеса» в 1938 году. Летчиком прошёл всю войну до победного конца. 

1 марта 1952 года во время испытания нового самолёта, в звании капитана, дедушка разбился...

Награждён медалью «За победу над Германией в Великой Отечественной войне
1941–1945 гг.» и медалью «За боевые заслуги».  Королевский Василий Васильевич -
дедушка по папиной линии. 

Был призван на фронт во время обучения в Московской промышленной академии,
которая с началом Войны была эвакуирована в Челябинск. Оттуда, командиром
танкового экипажа прибыл на фронт. Прошел всю Великую Отечественную войну
танкистом. Участвовал в Сталинградской битве, получил осколочное ранение
головы, но не выбыл из строя и, в составе Челябинской танковой дивизии дошёл до
самого Берлина, где на Рейхстаге оставил свою роспись "Красный Луч.
Королевский". 

После войны восстанавливал родной Красный Луч. Прошел трудовой путь от
начальника цеха до главного инженера Краснолучского машиностроительного завода. 

Я очень горжусь подвигом моих дедов и миллионов тех, кто не жалея себя сражался
за Родину и мирное будущее для нашей страны. Для меня и моих детей 9 Мая всегда
будет Днём Победы в Великой Отечественной Войне, самым главным праздником в
году! Мы всегда будем чтить память героев, которые всем нам подарили жизнь!
