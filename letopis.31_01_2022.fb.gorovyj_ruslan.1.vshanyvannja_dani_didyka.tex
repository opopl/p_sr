% vim: keymap=russian-jcukenwin
%%beginhead 
 
%%file 31_01_2022.fb.gorovyj_ruslan.1.vshanyvannja_dani_didyka
%%parent 31_01_2022
 
%%url https://www.facebook.com/gorovyi.ruslan/posts/7174430285901228
 
%%author_id gorovyj_ruslan
%%date 
 
%%tags 
%%title Вшанування Дані Дідіка
 
%%endhead 
 
\subsection{Вшанування Дані Дідіка}
\label{sec:31_01_2022.fb.gorovyj_ruslan.1.vshanyvannja_dani_didyka}
 
\Purl{https://www.facebook.com/gorovyi.ruslan/posts/7174430285901228}
\ifcmt
 author_begin
   author_id gorovyj_ruslan
 author_end
\fi

Для мене лютий - найважчий місяць року. Найпекучиший, найщемкіший. Місяць
втрат.

Щоразу відчуваю одночасно неймовірну тугу і сором. Тугу за втраченими рідними
людьми, і сором за те, що досі не закрито один зайстрашніших гештальтів. Мова
про вшанування Дані Дідіка.

\ii{31_01_2022.fb.gorovyj_ruslan.1.vshanyvannja_dani_didyka.pic.1}

Минає ще один рік з загибелі. Ще один рік брехні та маневрування між
політичними краплинами керівництва 11 школи Харкова. 

Десь рік тому я мав серйозну розмову з однією абсолютно патріотично
налаштованою людиною і абсолютно проукраїнськими цінностями щодо директорки
школи.

- Як ти можеш бути з нею в друзях? Навіть у фб? - запитав я.

- Я певен, що сталася якась помилка, вона точно за назву школи.

Потім був текст, який ви бачите на фото.

Чи досі людина дружить з патріоткою, яка стояла біля гроба учня і обіцяла
назвати школу я не знаю, бо більше не спілкувався. Сподіваюся за рік людина
знайшла натхнення щось для себе з‘ясувати... а для мене, повторюся, лютий наразі
це місяць болю і сорому.

\ii{31_01_2022.fb.gorovyj_ruslan.1.vshanyvannja_dani_didyka.cmt}
