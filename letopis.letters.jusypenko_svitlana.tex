% vim: keymap=russian-jcukenwin
%%beginhead 
 
%%file letters.jusypenko_svitlana
%%parent letters
 
%%url 
 
%%author_id 
%%date 
 
%%tags 
%%title 
 
%%endhead 

А оцей кроль - це просто щось неймовірне, як на мене. Це просто Чудо - розумі'

Доброго дня! Дякую! Тут... треба трохи розписати, що до чого... Справа в тому,
що... я бачу, що у нас тут в українському інформаційному просторі існує певні
(не дуже добрі тенденції, насправді - я на це дивлюсь зі свого боку як
програміст, я живу в Києві) такі штуки... я би назвав це Розрив Реальності...
це важко пояснити тут у коментарі. Але суть така. Зараз йде війна, і йде війна
не тільки на фізичному полі бою, але також війна мисленнєва, смислова. І щодо
українського інформаційного простору. Ось наприклад, оцей Кролик. Ви наприклад,
як я зрозумів, не знали, що він - оцей Кролик (та інші кролики та писанки,
всього їх значить 60 штук), свого часу приїхав в Маріуполь. Маріупольці ж, зі
свого боку, пам'ятають про цю виставку, бо це, як на мене, одна з найяскравіших
подій із усіх тих, що відбувались на Театральній Площі перед Драматичним
Театром, і ці спогади є для них дуже-дуже цінними. Для маріупольців, що змогли
вирватись з того пекла в березні-квітні 2022, взагалі, будь-які фото або навіть
такі прості речі, як квиток на тролейбус або трамвай з довоєнного Маріуполя, є
неймовірно цінними... Розумієте... їх рідні будинки, оселі знищили, вбили їх
друзів, рідних, батьків і т.д. І вони в більшості своїй дотепер мучаються всім
цим... Мені щойно недавно було дуже боляче бачити, знаєте, що, вираз на обличчі
однієї маріупольської поетеси, яка дуже багато зробила в Маріуполі для
української культури, опублікувала одну прекрасну дитячу книжку (і вона є у
мене в друзях тут). На аватарці фейсбучній вона вся усміхнена, але в пості
недавньому - де вона читала вірші в Києві - обличчя в неї дуже-дуже засмучене і
навіть я би сказав перекошене від внутрішнього болю... Дивишся на це і
розумієш, щось там було дійсно неймовірно страшне, що навіть після року людину
так перекошує... Так от. Щодо розриву сприйняття реальності в інформаційному
просторі, це проблема насправді усіх нас, і мене теж, не хочу нікого
образити... Ви от не знали - що вони побували в Маріуполі. Я до певного моменту
взагалі нічого не знав про зайців оцих , у свій час, про Добропарк, виставки в
Маріуполі та Києві раніше взагалі дізнався лише може місяць тому... (в 2018
році я взагалі нічого не знав, що в Києві було така чудо-виставка, хоча я
киянин, і туди прийшло тисячі людей дивитись на отих кролів) Маріупольці ж
навпаки - не знають, звідки взагалі оті кролі та писанки звалились на них в
2019, хто їх зробив, скільки взагалі неймовірно багато зусиль було потрібно для
всього цього і т.д. І також... музейні працівники теж, вони особливо напевне не
знають передісторію цих кролів. Ну, кролі, писанки... колись вони там 5 років
тому були виставлені, ну, поставимо їх поки тут. Гарні кролі, звичайно, але є
також виставка скла або археології, і чим ті кролі краще за виставку про
практичну археологію Київської Русі?... Так приблизно я це бачу. Розумієте,
кожен - включаючи мене - в області своєї реальності бачить те, що його оточує в
його реальності або що йому прийшлось пережити і т.д. і не знає багато чого
іншого, що має безпосередній насправді зв'язок між собою... І поки не почнеш
ретельно копати... то так і не взнаєш повну картину... 

... ну і... як на мене, знаєте, оцей Кролик, із зеленим листячком, який сумно
стоїть собі десь на останньому поверсі музею, от знаючи тепер його
передісторію... ну, це - просто Чудо із Чудес якесь. Тому що я знаю, яка
жахлива трагедія сталась в Маріуполі. Але цей Кролик такий самий, що на тому
фото, де він стоїть в центрі Театральної Площі в 2019, і всі на нього
дивляться, дітлахи підходять, гладять, люди дивуються - подивіться, яка
неймовірна краса!! - і цей же кролик був переді мною на відстані 10-20
сантиметрів, коли я його фотографував на тому поверсі... Розумієте... Це дуже
потужний Символ. Символ незнищенності Культури попри усі жахіття війни, Символ
того, що ми переможемо. Символ того, що Маріуполь обов'язково відродиться. І
також, це Жива Пам'ять, це Часточка Живого Міста Маріуполь... Міста... де кожен
день щось було цікаве, щось відбувалось, фестивалі, виставки... дуже-дуже
всього багато... Ось він ще позавчора стояв в кімнаті художниці на якихось
газетках... а до того був взагалі просто білим куском пластику (чи з чого там
він був зроблений)... вона його оглядає, ось, ух, закінчила, класс! потім...
він опиняється на Софійській Площі... потім, раптово виринає вже в Маріуполі в
2019... час біжить собі, невпинно біжить собі далі... вибори президента,
ковід... потім... війна, Маріуполь, Київ, Буча, Харків, захоплення ЗАЕС і інше
і інше і інше... безліч страшних подій, і також радісних! - час біжить... а він
десь там стоїть... і ось! ось він знову з'явився, цілий-цілісінький... вже рік
і більше йде страшна війна, світ взагалі стоїть на межі ядерної катастрофи...
але оцей тендітний Зайчик, начебто просто розмальований кусок пластику,
розмальований фарбами, які напевне можна купити в будь-якому магазині в
Києві... він стоїть тут, прямо переді мною... як привіт із живого справжнього
українського Маріуполя... і я майже плачу, знаєте, коли фотографую... не кожен
день у мене були такі враження, як отого квітневого дня в Музеї Історії Міста
Києва...

якую! так, звичайно можна... Так, я читав деякі пости в тій групі, як вони
створювались, обговорення, все це неймовірно цікаво!! От шкода, що я все це
пропустив... взагалі, знаєте, пропустив повністю, нічогісінько не знав про
Софійську Площу тоді... І так шкода, що пропустив тоді... Ну а в цьому році...
коли я вже все це трохи почитав, я побіжав на Площу, думаю, може там будуть оті
писанки... Але я побачив тільки сумний стенд про тих, кого чекають додому (є в
мене теж в альбомах). Писанки виставили у дворі Михайлівського, як я дізнався
пізніш... але біля самої Софії їх не було, як не було і кролів... Добре... Я
думаю, що краще, якщо Ви це зробити, якщо Ви згодні з тим, що я написав. Я
людина стороння, мене ніхто не знає... взагалі не артистична людина, а
тєхнарь-програміст. А Ви, наскільки я розумію, досить таки активну участь
приймали в усьому цьому... І ще... (1) Я тут в процесі дослідження всього цього
використав деякі фото та зробив деякі скріни (ну тобто щодо історії створення
того Кролика), і дещо виклав у себе в альбомах, і також вставив в деякі свої
намальовані картинки... Сподіваюсь, художники не образяться на таке
використання фото їх творчості та таке згадування їх імен... Потім (2) я на
сторінці викладаю наприклад усілякі Загадки. Вони може здаються на перший
погляд, наче я божевільний, але... ну, короче, це окрема тема, ніскільки я не
божевільний і насправді кожен день заробляю гроші якраз своїм розумом як
програміст... Але... Ну, короче, ось так. А коментар, звичайно, можна
опублікувати, я буду тільки радий від цього. І знаєте, от художники оці. От
коли я пішов в музей та коли я був в Михайлівському, я старався ретельно все
обфоткати... і інколи треба було залазити під саме оте яйце, щоби знайти ім'я
автора. Художники - дуже скромні люди... Створили такі неймовірні чудеса, з
такою любов'ю і натхненністю творили... але так інколи важко знайти, хто це
зробив... І їх вся ця творчість - це просто щось абсолютно неймовірне, як на
мене, і я думаю, більше людей повинно дізнатись про все це, пам'ять про
дивовижну виставку біля Софії не повинна просто лежати десь там в архівах
групи, так само як і оце Зайченя заслуговує на більше, ніж просто стояти десь
на задвірках музею, особливо в наш страшні час, коли людям так потрібно
знаходити сили, душевні і духовні...
