%%beginhead 
 
%%file 22_12_2020.fb.arximisto.1.istoriko_arhitekturnyj_plan_povestka
%%parent 22_12_2020
 
%%url https://www.facebook.com/arximisto/posts/pfbid07CYWgxVsnKXYkKivnXoNomkLGj7bJ4yHi8KLFtNbxGCUCaLgieG8JfYRUQWNkWL3l
 
%%author_id arximisto
%%date 22_12_2020
 
%%tags 
%%title Историко-архитектурный план Мариуполя – в повестке дня сессии горсовета
 
%%endhead 

\subsection{Историко-архитектурный план Мариуполя – в повестке дня сессии горсовета}
\label{sec:22_12_2020.fb.arximisto.1.istoriko_arhitekturnyj_plan_povestka}

\Purl{https://www.facebook.com/arximisto/posts/pfbid07CYWgxVsnKXYkKivnXoNomkLGj7bJ4yHi8KLFtNbxGCUCaLgieG8JfYRUQWNkWL3l}
\ifcmt
 author_begin
   author_id arximisto
 author_end
\fi

Историко-архитектурный план Мариуполя – в повестке дня сессии горсовета

\#новости\_архи\_города

Принятие историко-архитектурного опорного плана Мариуполя внесено в повестку
дня второй сессии городского совета 8-го созыва, которая должна состояться
завтра, 23 декабря. Об этом свидетельствует повестка дня сессии, утвержденная
распоряжением городского головы № 440р от 10.12.2020.

\ii{22_12_2020.fb.arximisto.1.istoriko_arhitekturnyj_plan_povestka.pic.1}

Историко-архитектурный опорный план – это ключевой градостроительный документ в
сфере охраны недвижимых объектов культурного наследия. Он должен содержать
описание исторического ареала города, памяток \textbackslash объектов наследия, их границ,
режимов использования и охранных зон (зоны охраны археологического культурного
слоя, зоны регулирования застройки и т.п.), согласно законам \enquote{Об охране
культурного наследия}, \enquote{О регулировании градостроительной деятельности} и т.п.

С 1 января 2019 года новое строительство, реконструкция или ремонт в
исторических ареалах городов Украины должны соответствовать требованиям этого
плана.

Прежде всего, его обязаны принять около 400 населенных пунктов Украины, которые
входят в т.н. Список исторических населенных мест Украины, утвержденный
Кабинетом Министром еще в 2001 году.

На территории Донецкой области, подконтрольной правительству Украины, в Список
входят Мариуполь, Краматорск, Славянск, Святогорск и Бахмут.

План для Мариуполя был разработан киевским ПОГ \enquote{Институт культурного
наследия} по заказу мэрии в 2018 году (стоимость составила 349 900 грн.). В
феврале 2019 года план был обнародован на сайте мэрии для общественного
обсуждения.

27 ноября 2019 года Научно-методический совет по вопросам охраны культурного
наследия при Министерстве культуры Украины рассмотрел проект плана, решил, что
он требует \enquote{значительной доработки}, и отправил свои рекомендации в
управление культуры и туризма Донецкой облгосадминистрации. 

23 декабря 2019 года управление культуры и туризмы ДонОГА утвердило только зоны
охраны памяток местного значения, а весь проект был отправлен на доработку.

Как сообщили \enquote{Архи-Городу} в главном управлении градостроительства и
архитектуры Мариупольского горсовета, проект был откорректирован в соответствии
с упомянутыми рекомендациями и был готов для голосования еще в начале осени
этого года.

В плане определены исторический ареал Мариуполя и его охранные зоны, охранные
зоны памяток культурного наследия, а также археологических культурных слоев. 

Авторы плана рекомендуют предоставить статус памяток архитектуры местного
значения 74 историческим зданиям, а также четко определить статус памяток
археологии. 

По мнению директора ОО \enquote{Архи-Город} Андрея Марусова, 
\begin{quote}
\em\enquote{
план необходимо принимать
ввиду его исключительной важности для охраны архитектурного и археологического
наследия Мариуполя.

Вместе с тем, нужно четко понимать, что план обеспечивает защиту от уничтожения
только девяти зданиям. Из них шесть – это памятки архитектуры местного
значения. Речь идет о Башне, здании Александровской гимназии (индустриальный
техникум), двух домах со шпилями, драмтеатре и отеле \enquote{Континенталь}.

Остальные три – это памятки истории. Это здание низшего механико-технического
училища 1901 года (ПТУ № 3; в нем учился Герой Советского Союза А. Губенко), дом
по адресу ул. Радина, 2 (в нем действовали подпольщики в 1941-43 гг.) и дом по
адресу ул. Георгиевская, 37 (в нем жил писатель Серафимович).

Взятие под защиту остальных 70 зданий в историческом ареале города зависит,
прежде всего, от деятельности Департамента культурно-общественного развития
Мариупольского горсовета. Ибо это его прямая задача – \enquote{обеспечение реализации
государственной политики в сфере охраны культурного наследия}, разработка и
подача предложений о внесении исторических зданий в Реестр недвижимых памяток
Украины...}
\end{quote}

Проект \enquote{Историко-архитектурный опорный план г. Мариуполя Донецкой области по
определению границ и режимов использования зон охраны памятников и исторических
ареалов} см. \url{https://mariupolrada.gov.ua/ru/page/proekti-detalnogo-planu}
