% vim: keymap=russian-jcukenwin
%%beginhead 
 
%%file 06_12_2021.stz.news.ua.mrpl_city.1.30_richnyj_jubilej_mdu
%%parent 06_12_2021
 
%%url https://mrpl.city/blogs/view/30-richnij-yuvilej-mdu
 
%%author_id demidko_olga.mariupol,news.ua.mrpl_city
%%date 
 
%%tags 
%%title 30-річний ювілей МДУ
 
%%endhead 
 
\subsection{30-річний ювілей МДУ}
\label{sec:06_12_2021.stz.news.ua.mrpl_city.1.30_richnyj_jubilej_mdu}
 
\Purl{https://mrpl.city/blogs/view/30-richnij-yuvilej-mdu}
\ifcmt
 author_begin
   author_id demidko_olga.mariupol,news.ua.mrpl_city
 author_end
\fi

\subsubsection{Частина I: історія та головні досягнення}

\begin{quote}
\em Маріупольський державний університет формує серце і розум, об'єднує традиції і
сучасність, відкриває багато нових та перспективних можливостей і виховує та
випускає справжніх професіоналів своєї справи. Сьогодні МДУ виповнюється лише
30 років, але довгий і яскравий шлях, пройдений ним, надихає. Він став визнаним
суспільством та міжнародною спільнотою унікальним закладом вищої освіти і
попереду на нього чекають чергові перемоги, злети, здобутки та нові досягнення.
\end{quote}

\ii{06_12_2021.stz.news.ua.mrpl_city.1.30_richnyj_jubilej_mdu.pic.1}

У вересні 1991 року з ініціативи Маріупольського міського
на\hyp{}ціонально-культурного товариства греків, Донецького державного університету
та за підтримки Маріупольської міської ради було засновано Маріупольський
гуманітарний коледж при Донецькому державному університеті. На п'яти
спеціальностях навчалося 102 студенти. Навчальний процес забезпечували 12
викладачів, з яких лише троє мали науковий ступінь.

\ii{06_12_2021.stz.news.ua.mrpl_city.1.30_richnyj_jubilej_mdu.pic.2}

На думку ректора МДУ, \emph{\textbf{Миколи Валерійовича Трофименка}},

\begin{quote}
\em\enquote{історичними подіями та
історичним рішенням стало перейменувати Маріупольський державний гуманітарний
університет в 2010 році у Маріупольський державний, адже саме тоді розпочався
шлях класичного університету}. 
\end{quote}

Сьогодні університет має розгалужений перелік освітніх програм та
спеціальностей. Працює 6 факультетів (факультет філології та масових
комунікацій, історичний, економіко-правовий, факультет іноземних мов, факультет
грецької філології та психолого-педагогічний факультет). Що стосується
контингенту, Микола Валерійович наголосив, що наразі в МДУ навчається більше 4
тисяч студентів та в планах ставати ще більшими, ще масштабнішими, щоб
Маріуполь перетворювати в по-справжньому студентське місто.

\ii{06_12_2021.stz.news.ua.mrpl_city.1.30_richnyj_jubilej_mdu.pic.3}
\ii{06_12_2021.stz.news.ua.mrpl_city.1.30_richnyj_jubilej_mdu.pic.4}

Радник ректора МДУ, почесний генеральний консул Республіки Кіпр у Маріуполі
\emph{\textbf{Костянтин Васильович Балабанов}} згадує, що починалось все практично з нуля, без
кадрової та навчальної баз, тим більше без міжнародних зв'язків. 

\begin{quote}
\em\enquote{І мало хто,
тоді вірив, що з гуманітарного коледжу, в якому навчалося 100 студентів, і ми
займали 10 кімнат у цьому колишньому будинку політпросвіти вийде сучасний
авторитетний університет. Ми працювали практично цілодобово}. 
\end{quote}

Костянтин Васильович підкреслив, що він дуже радий, що з самого початку роботи ректором
був оточений талановитими, сумлінними, працьовитими людьми і в багатьох
випадках вони є \textbf{авторами успіху Маріупольського державного університету}. 

\begin{quote}
\em\enquote{Я хочу сказати, що мені життя подарувало зустрічі з дуже цікавими людьми. Це
сильні, талановиті, авторитетні керівники, відомі дипломати, видатні вчені та
чудові студенти і я сьогодні хочу висловити всім їм сердечну подяку за дружбу,
за допомогу у розвитку Маріупольського державного університету}.
\end{quote}

\ii{06_12_2021.stz.news.ua.mrpl_city.1.30_richnyj_jubilej_mdu.pic.5}
\ii{06_12_2021.stz.news.ua.mrpl_city.1.30_richnyj_jubilej_mdu.pic.6}
\ii{06_12_2021.stz.news.ua.mrpl_city.1.30_richnyj_jubilej_mdu.pic.7}
\ii{06_12_2021.stz.news.ua.mrpl_city.1.30_richnyj_jubilej_mdu.pic.8}

Перший проректор Маріупольського державного університету \emph{\textbf{Олена Валеріївна
Булатова}} вважає, що створення у 1991 році нового закладу освіти стало
доленосним рішенням. Про інженерно-технічні спеціальності Маріуполя було відомо
у всьому Радянському Союзі і тут виникає принципово новий напрям розвитку
освіти і науки, пов’язаний з гуманітарним. Проте, як підтвердив час, за ці 30
років, починаючи з підготовки істориків, філологів, сьогодні Маріупольський
державний університет – це реальний класичний університет, в якому гармонійно
поєднуються і розвиваються гуманітарні, суспільні, природничі та математичні
науки.

\ii{06_12_2021.stz.news.ua.mrpl_city.1.30_richnyj_jubilej_mdu.pic.9}
\ii{06_12_2021.stz.news.ua.mrpl_city.1.30_richnyj_jubilej_mdu.pic.10}

Важливими чинниками успішного розвитку університету є його кадровий потенціал
та якість освітніх послуг, які він надає. Проректор з науково-педагогічної
роботи \emph{\textbf{Тетяна Василівна Марена}} розповіла, що у складі науково-педагогічних
працівників МДУ сьогодні працює близько 50 докторів наук, професорів, понад 120
кандидатів наук, доцентів. Якщо враховувати ще і реорганізацію Донецького
державного університету управління, який вливається в структуру Маріупольського
державного університету, ці показники ще більше зростуть,тобто кількість
докторів наук збільшиться до 70, а кандидатів наук – до 170. Кадровий потенціал
університету дійсно вражає. Поступово вибудовується внутрішня система
забезпечення якості вищої освіти і ця система не є статичною, вона постійно
розвивається та вдосконалюється.

\ii{06_12_2021.stz.news.ua.mrpl_city.1.30_richnyj_jubilej_mdu.pic.11}
\ii{06_12_2021.stz.news.ua.mrpl_city.1.30_richnyj_jubilej_mdu.pic.12}

Міжнародне співробітництво було і залишається одним із пріоритетів розвитку
МДУ, є невід'ємною складовою життя університету та розвивається в рамках
єдиного процесу інтеграції вищої школи України у світову систему вищої освіти.
Завдяки енергійній діяльності \emph{\textbf{Костянтина Васильовича Балабанова}} та \emph{\textbf{Миколи
Валерійовича Трофименка}} було підписано понад 120 договорів про співпрацю,
прийняли понад 700 делегацій найвищого рівня. Костянтин Васильович зауважив,
що:  

\begin{quote}
\em\enquote{Ми один із небагатьох вишів в Україні, який відвідали 3 президенти
європейських країн. Це Президенти Грецької Республіки Константінос
Стефанопулос, Каролос Папуліас та Президент Республіки Кіпр Деметріс
Христофіас. Завдяки цим візитам про Маріуполь, Федерацію грецьких товариств
України, а також про наш університет тоді вперше дізналися у всьому світі.
Повинен сказати, що лише останнім часом ми прийняли міністрів закордонних справ
Литви, Швейцарії, Греції, послів США, Японії, більшість послів країн
Європейського Союзу, які дали високу оцінку діяльності нашого університету}, 
\end{quote}
– наголосив Костянтин Васильович.

\ii{06_12_2021.stz.news.ua.mrpl_city.1.30_richnyj_jubilej_mdu.pic.13}

Проректор МДУ з науково-педагогічної роботи (міжнародні\par\noindent зв'язки) \emph{\textbf{Олена
Георгіївна Павленко}} підкреслила, що \emph{Маріупольський державний університет
створив свою модель міжнародної співпраці}. І така міжнародна співпраця зараз
відбувається у всіх напрямах міжнародної діяльності, включаючи участь
науково-педа\hyp{}гогічних працівників, здобувачів освіти в проєктах Еразмус+,
міжнародних програмах, грантових програмах та програмах академічної
мобільності.

\ii{06_12_2021.stz.news.ua.mrpl_city.1.30_richnyj_jubilej_mdu.pic.14}
\ii{06_12_2021.stz.news.ua.mrpl_city.1.30_richnyj_jubilej_mdu.pic.15}

Університет має потужний науково-педагогічний колектив.\par\noindent Вчені Маріупольського
державного університету – це не тільки справжні фахівці за своїми науковими
напрямами, це і визнані авторитетні дослідники в Україні та за кордоном. \emph{\textbf{Олена
Валеріївна Булатова}} зазначила, що \emph{сьогодні науковці університету входять до
багатьох структур загальнодержавного рівня від діяльності яких залежать
стратегічні напрями розвитку вищої школи в Україні}. Зокрема, ректор МДУ Микола
Трофименко входить до складу Ради Фонду при Президентові України з підтримки
освіти, науки та спорту. Водночас вчені Маріупольського державного уні\hyp{}верситету
беруть активну участь у розробці багатьох важливих стратегічних документів, які
стосуються розвитку Маріуполя і загалом Донеччини. Йдеться про стратегію
розвитку Маріуполя 2030 та про програму розвитку малого і середнього бізнесу та
туризму. Багато викладачів МДУ входить до наукової Ради Міністерства освіти і
науки. Вони є розробниками державних стандартів саме з вищої школи, тобто від
рішень експертної думки науковців Маріупольського державного університету
залежить зокрема і розвиток вищої школи в Україні. А це в свою чергу посилює
авторитет  та значущість Маріуполя на українських теренах.

\ii{06_12_2021.stz.news.ua.mrpl_city.1.30_richnyj_jubilej_mdu.pic.16}
\ii{06_12_2021.stz.news.ua.mrpl_city.1.30_richnyj_jubilej_mdu.pic.17}

Цьогоріч у МДУ розпочав роботу психолого-педагогічний факультет, що, за словами
Миколи Валерійовича Трофименка, було стратегічною метою та правильним рішенням,
адже є запит від маріупольської міської територіальної громади, від Донецької
області загалом на педагогів різного спрямування. Цей факультет утворився із
чотирьох кафедр, які спеціалізуються на підготовці вихователів дошкільних
закладів, вчителів і керівників шкіл, фахівців із фізичної культури та
практичних психологів. Завдяки актуальності та затребуваності спеціальностей
ППФ став одним із найчисельніших – контингент налічує 688 студентів.

Декан психолого-педагогічного факультету \emph{\textbf{Леніна Вікторівна Задорожна-Княгницька}}
наголосила, що 
\begin{quote}
\em\enquote{три місяці роботи цього факультету переконали, що в нас дуже
талановиті студенти, дуже відповідальні викладачі і нас чекає велике майбутнє
за умови, що ми будемо гарно і плідно працювати}.
\end{quote}

Керівництво університету пишається здобутками ювіляра та готове реалізовувати
нові важливі плани і завдання. Радник ректора МДУ Костянтин Балабанов
підкреслив, що університет став колискою гуманітарної освіти у Маріуполі та
важливим центром якісної підготовки фахівців у Донецькій області, одним із
провідних центрів співпраці України із зарубіжними країнами в галузі освіти,
науки та культури. На думку \emph{\textbf{Костянтина Васильовича Балабанова}}, \emph{формула успіху
МДУ дуже проста – це велика працелюбність та любов до своєї справи всіх
співробітників університету}. Дійсно, саме самовіддана та титанічна праця всіх
викладачів і студентів допомагають Маріупольському державному університету
досягати великих успіхів.

Перший проректор МДУ Олена Булатова розповіла, що за 30 років розвитку
Маріупольського державного університету вдалося створити дуже багато цікавих
наукових шкіл, які розвивають важливі наукові напрями і проблематику, що
сьогодні актуальна для розвитку української науки. Зокрема, це наукова школа з
історії та історіографії історії країни, яку очолює професор \emph{\textbf{В. Романцов}}, з
теорії міждержавного та міжрегіонального співробітництва (професор \emph{\textbf{К.
Балабанов}}), з теорії дошкільної освіти (професор \textbf{\emph{К. Щербакова}}), інформаційної
політики (професор \emph{\textbf{Г. Почепцов}}). Тобто наразі існують такі нові напрями, які 30
років тому Маріуполь зовсім не знав. А сам Маріупольський державний університет
наразі є дуже важливим центром трансформації громади.

\ii{06_12_2021.stz.news.ua.mrpl_city.1.30_richnyj_jubilej_mdu.pic.18}

МДУ готує фахівців за рівнями вищої освіти \enquote{Бакалавр}, \enquote{Магістр} та \enquote{Доктор
філософії} за 12 галузями знань: освіта та педагогіка; культура і мистецтво;
гуманітарні науки; соціальні та поведінкові науки; журналістика; управління та
адміністрування; право; екологія; інформаційні технології; сфера
обслуговування; публічне управління та адміністрування; міжнародні відносини. В
аспірантурі МДУ здійснюється підготовка здобувачів вищої освіти ступеня доктора
філософії за спеціальностями: історія та археологія; економіка; політологія;
право; публічне управління та адміністрування; міжнародні економічні відносини.
А в докторантурі готують докторів наук за спеціальностями: міжнародні
економічні відносини, історія та археологія.

В університеті також функціонують спеціалізовані вчені ради з економічних,
політичних та історичних наук.

\ii{06_12_2021.stz.news.ua.mrpl_city.1.30_richnyj_jubilej_mdu.pic.19}

Для студентів Маріупольський державний університет вже давно став другою
домівкою, справжньою родиною, яка надихає на нові звершення. Зокрема, голова
студентської ради МДУ, студентка 2 курсу спеціальності \enquote{Право} \emph{\textbf{Дар'я Дзигар}}
поділилася, що університет подарував їй купу можливостей, багато ідей та дуже
цікаві будні. Завдяки рідному університету вона може реалізовувати власний
потенціал. Для голови студентської ради факультету філології та масових
комунікацій, студента 2 курсу спеціальності \enquote{Журналістика} \emph{\textbf{Владислава
Міцинського}} МДУ – це гарні викладачі, чудові творчі одногрупники та купа
позитивних емоцій.

\ii{06_12_2021.stz.news.ua.mrpl_city.1.30_richnyj_jubilej_mdu.pic.20}

Завдяки Маріупольському державному університету здійснилася найголовніша мрія у
студентки 3 курсу спеціальності \enquote{Журналістика} \emph{\textbf{Ксенії Місюревіч}}. Вона почала
працювати на Маріупольському телебаченні і вже рік успішно веде ранкове шоу на
найкращому каналі міста.

А студентка 2 курсу спеціальності \enquote{Культурологія} МДУ \emph{\textbf{Катерина Лютікова}}
зазначила, що МДУ – це той затишний  простір, який не тільки  допомагає
реалізувати амбітні плани, це ще й друга родина та \emph{справжнє серце Донецької
області}.

