% vim: keymap=russian-jcukenwin
%%beginhead 
 
%%file 21_02_2022.fb.fb_group.story_kiev_ua.1.den_ekskursovoda
%%parent 21_02_2022
 
%%url https://www.facebook.com/groups/story.kiev.ua/posts/1866341810229278
 
%%author_id fb_group.story_kiev_ua,kirkevich_viktor.kiev
%%date 
 
%%tags ekskursia,gorod,kiev,kievljane
%%title Сегодня День экскурсовода
 
%%endhead 
 
\subsection{Сегодня День экскурсовода}
\label{sec:21_02_2022.fb.fb_group.story_kiev_ua.1.den_ekskursovoda}
 
\Purl{https://www.facebook.com/groups/story.kiev.ua/posts/1866341810229278}
\ifcmt
 author_begin
   author_id fb_group.story_kiev_ua,kirkevich_viktor.kiev
 author_end
\fi

Сегодня День экскурсовода. Поздравляю всех, кто проводил экскурсии, а также
всех, кто на них побывал. Вспомню прошлое! Особенно КБП и Э.

Эти загадочные буквы расшифровываются весьма просто - Киевское бюро путешествий
и экскурсий, - любимая организация для многих киевлян. Оно осталось только в
памяти и у некоторых записью в трудовой книжке, как и у меня. Бюро нет. Многие
ветераны давно... увы! в миру ином. А уцелевшие единицы на заслуженном отдыхе. Но
могу заверить, что трудно назвать организацию в Киеве, сыгравшей большую роль в
эстетическом, духовном воспитании широких слоев населения. Без преувеличения
могу сказать в своей просветительской, гуманистической деятельности, КБП и Э
обогнала все музеи, концертные залы, и тем более, творческие союзы.

Люди старшего поколения помнят, а как это можно забыть, 10\% профсоюзные
путевки, купленные в Киевском Бюро. По ним в скольких городах посчастливилось
побывать. А экскурсии по городу, и по окрестностям Киева! Как хотелось
познавать прекрасный мир прошлого, столь важного для воспитания духовности. В
те годы впервые стали посещать памятники культуры (так в прошлом назывались
церкви). А сколько их сберегли, благодаря подогретому экскурсоводами интересу!?

Так что, давайте вспомним первые потуги этой организации, созданной 1 апреля
1966 г. Какой этот год был для нашей страны, мы подробно узнаем из передачи
«Намедни» Леонида Парфенова.

Нельзя забыть первопроходцев в столь важном воспитательном процессе. Их было
пять штатных экскурсоводов, и как положено в дальнейшем КБП и Э, среди них было
80\% женщин. Тамара Гладких, Лидия Носова, Ирина Плямм, Жанна Топчий и мужчина –
Владимир Нереков. Бюро было создано на базе «Киевской турбазы» где в штате не
предусматривались экскурсоводы, и, вначале, размещались в одной комнате по
адресу Энгельса (Лютеранская),6. Для меня это весьма знаменательно! Я до 1954
года проживал вместе с родителями, в 2-х комнатной квартире, общей площадью с
кухней – 16 кв.м. Она находилась как раз над комнатой бюро (вход у нас был со
двора).

\raggedcolumns
\begin{multicols}{2} % {
\setlength{\parindent}{0pt}

\ii{21_02_2022.fb.fb_group.story_kiev_ua.1.den_ekskursovoda.pic.1}
\ii{21_02_2022.fb.fb_group.story_kiev_ua.1.den_ekskursovoda.pic.1.cmt}

\ii{21_02_2022.fb.fb_group.story_kiev_ua.1.den_ekskursovoda.pic.2}
\ii{21_02_2022.fb.fb_group.story_kiev_ua.1.den_ekskursovoda.pic.2.cmt}

\ii{21_02_2022.fb.fb_group.story_kiev_ua.1.den_ekskursovoda.pic.3}

\end{multicols} % }

Тематика была простой – только обзорная экскурсия. Их за небольшую оплату
проводили - по три в день. Транспорта почти не было, поэтому все приезжающие
шли пешком с вокзала до площади Богдана Хмельницкого, причем в действующий
Владимирский собор заходить было категорически запрещено. Рассказывали обо
всем, что было достойно внимания по пути. Как понимаете, состав, как
слушателей, так и повествующих - был молодежным. Кто мог выдержать такую
прогулку? В зрелом возрасте был И. М. Скуленко, но и он преодолевал этот маршрут.
Нужно отметить, что Иван Михайлович был не только старейшим экскурсоводом, но и
учителем и наставником для молодых коллег. Их образование обязательно было
высшим, но гуманитариев принимали на работу с ІІІ курса вуза.

Через пару месяцев из общества «Знания», которое также проводило экскурсии,
соответственно с вокзала, пришли Валентина Лунева и Ива Марьянова. Позже стала
работать Елена Костенко. С их помощью тематику предлагаемых экскурсий несколько
разнообразили. Какой разговор, безусловно, политического направления, а другого
не могло быть в канун 50-летия Октябрьской революции. Так появилась экскурсия
«По местам революционной славы», что привело к подготовке подобных маршрутов к
соответствующим датам. Первой загородной экскурсией стал Канев на теплоходе, а
с появлением экскурсионных автобусов – Новые Петровцы и Умань. Из-за
непопулярности храмов, внимание на Чернигов обратили лишь в конце 1970-х годов,
тогда же появился и «Древний Киев». В первые годы существования Бюро стали
проводить поездки на строящуюся Киевскую ГЭС, где туристы в пыли и грязи
знакомились с ускоренным производственным процессом. Автобусы были, так
называемые «газоны» - с ручкой-рычагом для открытия двери, которые представляли
опасность для сидящего впереди нашего коллеги. На Аскольдовой могиле произошел
трагический случай – выпав из автобуса, погибла девушка, которая зарабатывала
на свадьбу. Это не остановило сотрудников бюро, которых становилось все больше
и больше. Часто экскурсии проводились в открытых грузовых машинах с надписью
«люди». Направление поворота транспорта определялось стуком рукой по кабине.
Один удар – налево, два удара – направо, а три – прямо.

С годами КБП и Э стало самым лучшим в СССР. Этому способствовало особенное
положение города в нашей стране. В Москву советский народ ездил за товаром, а в
Ленинграде, чтобы отметиться в своей культурности. В этих городах впоследствии
не раз приходилось убеждаться в слабой подготовке местных экскурсоводов,
которые не особенно завлекали граждан и не работали с местными жителями.

В Киеве, приходилось стараться, так как приезжающие исправно посещали
экскурсии, не зависимо от их содержания. Особенно это отмечалось в 1980-е годы,
когда туристы из среднеазиатских республик не пропускали ни одной автобусной
экскурсии. Такое предельное внимание и постоянное присутствие привело к тому,
что приходилось разрабатывать темы так, чтобы было интересно всем. Вскоре,
начиная с конца 1970-х, на экскурсии стали завлекать киевлян, что стало
массовым явлением. В то время стали популярными экскурсии, проводимые Элеонорой
Рахлиной, Анатолием Халепой, Татьяной Золотаревой…

Расцвет КБП и Э выпадает на середину 1980-х годов, когда во главе его стали
В.Г.Павлюк и Н.В.Грицык. Они с помощью значительных здоровых сил среди
сотрудников начали обновление Бюро. К этому времени подготовка киевских
экскурсоводов достигла высокого профессионального уровня, благодаря
созидательной работы архитектурно-исторической и краеведческой секций.
Экскурсии, подготовленные членами этих профессиональных объединений, отличались
высокой интеллектуальностью, скрупулезностью подбора материала и стройностью
методической разработки. Появились таланты и в других секциях, например,
Александр Ершов прельщал многочисленных слушательниц на любой из проводимой им
теме. Других они не хотели слушать, объявляя бойкот любому приехавшему вместо
него. Надежда Верещагина стал разрабатывать тематику по истории христианства.
Экскурсионной работой были охвачены все школы, средние и высшие учебные
заведения, санатории и дома отдыха. Населению предлагали маршруты самой
разнообразной тематики. Были возобновлены курсы экскурсоводов, которые стали
впоследствии Институтом. Обслуживая приезжих и отправляя сограждан в другие
города и курорты, КБП и Э выполняла годовой план на 30 млн. руб., то есть
оборот большого завода. В то время численный состав достигал 700 человек, при
1500 внештатных экскурсоводов и групповодов.

В 70-е годы показухи и очковтирательством, называемые «застоем», создавать
новые темы было достаточно сложно. Обзорная экскурсия постоянно обновлялась, то
есть из нее изымались занимательные исторические материалы, более всего
интересовавшие людей, а появлялись пустые фразы руководителей страны с
множеством ничего не значащих цифр. При этом впаривались данные о свершениях,
достижениях, однодневные материалы партии и правительства, материалы партийных
съездов и комиссий. Многочисленные проверяющие от различных организаций любили
незаметно сесть в автобус, и, не особенно владея темой, более пеклись о наличии
тех или иных пропагандистских фраз. В этом и было особенное отличие современной
экскурсии в Украине, любой западной страны от работы советского экскурсовода в
застойные времена. Тогда экскурсия была частью воспитательного процесса, тогда
как по логике она должна быть образовательной и развлекательной. То есть, как
сказал один юморист 1920-х годов: «В СССР от агитпропа работы польза двойная –
она лишает сна Чемберлена и вылечивает от бессонницы советских граждан».

Так происходило до распада СССР. Не думайте, что сыграли какие-то политические
силы. Ерунда! Просто месткомы разных организаций от мала до велика, не стали
покупать на свои деньги путевки «По Ленинским местам» в Ульяновск, а
распределяли их более эффективно. Короче. Распались массовые профсоюзы,
перестали работать туристические бюро в различных городах, а как следствие
этого неуклонного процесса перестал существовать СССР.

Ветеран туризма с 1970 г., экскурсовод І категории В.Г. Киркевич
