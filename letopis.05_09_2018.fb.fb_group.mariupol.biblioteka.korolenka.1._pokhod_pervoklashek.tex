%%beginhead 
 
%%file 05_09_2018.fb.fb_group.mariupol.biblioteka.korolenka.1._pokhod_pervoklashek
%%parent 05_09_2018
 
%%url https://www.facebook.com/groups/1476321979131170/posts/1842112935885404
 
%%author_id fb_group.mariupol.biblioteka.korolenka,kibkalo_natalia.mariupol.biblioteka.korolenko
%%date 05_09_2018
 
%%tags mariupol,dity,isskustvo,hudozhnik,shkola
%%title Поход первоклашек в  Мобильную галерею
 
%%endhead 

\subsection{Поход первоклашек в  Мобильную галерею}
\label{sec:05_09_2018.fb.fb_group.mariupol.biblioteka.korolenka.1._pokhod_pervoklashek}
 
\Purl{https://www.facebook.com/groups/1476321979131170/posts/1842112935885404}
\ifcmt
 author_begin
   author_id fb_group.mariupol.biblioteka.korolenka,kibkalo_natalia.mariupol.biblioteka.korolenko
 author_end
\fi

После уроков 5 сентября первоклашки ОШ № 3 вместе со своей учительницей
Маргаритой Вячеславовной Поповой пришли в Мобильную галерею «Мир увлечений»
Центральной городской библиотеки им. В.Г. Короленко. Сейчас в галерее открыта
выставка «Воспитание добром», посвященная выдающемуся украинскому педагогу
Василию Сухомлинскому. В экспозиции представлены творческие работы учащихся
художественного отделения Школы искусств (преп. Любовь Васильевна Макаренко).
Это иллюстрации к  детским рассказам и сказкам Сухомлинского.

Малыши внимательно посмотрели рисунки, также внимательно послушали рассказ
Любови Васильевны о своей талантливой и уже добившейся больших успехов ученице
Анастасии Журавлевой. Последнее достижение 12-летней художницы - первое место в
номинации «Самый творческий ребенок» во Всеукраинском конкурсе
«Чудо-ребенок-2018». Пришедшая специально для встречи с первоклашками Настя
Журавлева познакомила их с некоторыми своими картинами, выполненными в разных
техниках живописи,  рассказала о своих трудах и победах. 

А затем малыши с увлечением  рисовали-выдували феерический  разноцветный
фонтан, льющийся из хобота веселого слоненка. Конечно же, им все помогали: и
Настя, и обе учительницы, и родители, и библиотекари. Поход первоклашек в
Мобильную галерею завершился дружеской фотосессией.
