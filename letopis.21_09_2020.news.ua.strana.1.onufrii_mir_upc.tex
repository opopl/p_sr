% vim: keymap=russian-jcukenwin
%%beginhead 
 
%%file 21_09_2020.news.ua.strana.1.onufrii_mir_upc
%%parent 21_09_2020
 
%%url https://strana.ua/news/290878-v-svjato-uspenskoj-svjatohorskoj-lavre-donetskoj-oblasti-prokhodit-forum-ljudi-mira.html
 
%%author 
%%author_id 
%%author_url 
 
%%tags 
%%title Предстоятель УПЦ Онуфрий назвал взаимное прощение единственным путем к мирной жизни
 
%%endhead 
 
\subsection{Предстоятель УПЦ Онуфрий назвал взаимное прощение единственным путем к мирной жизни}
\label{sec:21_09_2020.news.ua.strana.1.onufrii_mir_upc}
\Purl{https://strana.ua/news/290878-v-svjato-uspenskoj-svjatohorskoj-lavre-donetskoj-oblasti-prokhodit-forum-ljudi-mira.html}

18:19, 21 сентября 2020

\ifcmt
pic https://strana.ua/img/article/2908/78_main-v1600695588.jpeg
caption Блаженнейший митрополит выступил с обращением по миротворческому процессу на Украине. Фото: facebook.com/Church.ua
\fi

\index[names.rus]{Онуфрий!Митрополит Киевский!Форум \enquote{Люди мира}, 21.09.2020}

В понедельник, 21 сентября, в Свято-Успенской Святогорской лавре в
Донецкой области прошел форум \enquote{Люди мира}. В нем участвовали жители
Донбасса и других регионов Украины, откликнувшиеся на обращение Украинской
Православной Церкви (УПЦ).\Furl{https://strana.ua/news/287370-nachal-rabotu-orhkomitet-foruma-ljudi-mira-initsiirovannoho-upts.html}

Идея Форума состоит в том, чтобы включить в переговорный процесс по Донбассу
обычных людей, непосредственно пострадавших от войны на Востоке Украины.

Сегодня во время форума выступил с обращением по миротворческому процессу
в Украине блаженнейший митрополит Онуфрий, предстоятель УПЦ.

\enquote{Мир - это Божий дар....Война - это кровь, страдания, слезы. Как правило,
в войнах, с их большими геополитическими, политическими, экономическими и
другими интересами, сильные мира сего не видят простых людей, не чувствуют
их боли, страданий и переживаний. Именно поэтому сегодня на Форуме мы
прислушиваемся к свидетельству простых людей, живущих в прифронтовой зоне
и особенно страдают от войны. Цель сегодняшнего нашего общения заключается
в том, чтобы с одной стороны показать простых людей и их страдания, а с
другой - засвидетельствовать сильное желание мира} - обратился
предстоятель УПЦ.

Он добавил, что первым шагом на пути к миру \enquote{должно быть установление
человеческого общения между обеими сторонами конфликта}.

По его словам, финалом миротворческого процесса должно быть взаимное
прощение. \enquote{Не из мести и обид, а из прощение открывается путь к
примирению. С принятия обеими сторонами конфликта общего будущего, того
будущего, которое они сами построят вместе в мире начнется новая страница
в истории нашей общей многострадальной страны}, - добавил  митрополит
Онуфрий.

\ifcmt
tab_begin cols=2
	caption В Свято-Успенской Святогорской лавре Донецкой области проходит форум Люди мира. Скриншот: Facebook/ УПЦ
	pic https://strana.ua/img/forall/u/11/55/QIP_Shot_-_Screen_1238.jpg
	pic https://strana.ua/img/forall/u/11/55/QIP_Shot_-_Screen_1239.jpg
\fi

Напомним, \enquote{Страна} анонсировала сегодняшний форум. К тому же мы
публиковали разговор с участниками \enquote{Люди мира} о войне, мире и смысле
мероприятия, о котором подробнее почитать можно
здесь.\Furl{https://strana.ua/articles/istorii/290420-v-svjatohorske-21-sentjabrja-projdjot-forum-ljudi-mira.html}
