% vim: keymap=russian-jcukenwin
%%beginhead 
 
%%file 11_02_2023.fb.fb_group.mariupol.pre_war.10.zimovii_mar_upol_201.cmt
%%parent 11_02_2023.fb.fb_group.mariupol.pre_war.10.zimovii_mar_upol_201
 
%%url 
 
%%author_id 
%%date 
 
%%tags 
%%title 
 
%%endhead 

\qqSecCmt

\begin{itemize} % {
\iusr{Kostas Petkaris}

боже.. я был там. в январе 2019... как будто вчера..

\ifcmt
  igc https://scontent-ams4-1.xx.fbcdn.net/v/t39.1997-6/851586_126361977548599_392107290_n.png?stp=dst-png_s168x128&_nc_cat=1&ccb=1-7&_nc_sid=ac3552&_nc_ohc=pO0A5uQwpFIAX_gbj8S&_nc_ht=scontent-ams4-1.xx&oh=00_AfBCs2rX7ySYNI3mGfQA1ERo0dTyrehTA1RYjkbhOyGziw&oe=63FA2313
	@width 0.1
\fi

\iusr{Sergey Drovorub}

Зимова казка. Для Маріуполя вона майже завжди була короткою.

Зате якщо морозило, та ще й з вітром, то ховайся хто може. Або налідь на
тротуарах і шляхах, теж іноді додавала клопіт.

\end{itemize} % }
