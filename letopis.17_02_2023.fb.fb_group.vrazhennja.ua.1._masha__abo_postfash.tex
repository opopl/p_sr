%%beginhead 
 
%%file 17_02_2023.fb.fb_group.vrazhennja.ua.1._masha__abo_postfash
%%parent 17_02_2023
 
%%url https://www.facebook.com/groups/vrajenniya.ua/posts/1827340367624394
 
%%author_id fb_group.vrazhennja.ua,livinska_tetjana
%%date 17_02_2023
 
%%tags kniga,literatura
%%title "Маша, або Постфашизм" - Ярослав Мельник
 
%%endhead 

\subsection{"Маша, або Постфашизм" - Ярослав Мельник}
\label{sec:17_02_2023.fb.fb_group.vrazhennja.ua.1._masha__abo_postfash}
 
\Purl{https://www.facebook.com/groups/vrajenniya.ua/posts/1827340367624394}
\ifcmt
 author_begin
   author_id fb_group.vrazhennja.ua,livinska_tetjana
 author_end
\fi

\#читаюфантастику \#читаюксд 

📘\enquote{Маша, або Постфашизм} 

🖊️Ярослав Мельник

288 с.

🖨️Видавництво Старого Лева, 2016

З перших рядків ця книга вибила твердий ґрунт в мене з-під ніг. Давненько мене
так не обурювали, не викликали огиди найвищого рівня вигадане суспільство з
його світоглядними засадами буття в цілому та погляди й переконання окремого
пересічного громадянина зокрема. Оповідь, до речі, ведеться від імені головного
героя. Й добру третину книжки мені хотілося прокричати йому в лице: \enquote{Ти
сліпий?!} Отже, автор вміло розбурхав емоції й таким чином підсадив на гачок -
зацікавив твором, тому поки не дочитала, не заспокоїлася. 

3899 рік. В планетарній державі під назвою Рейх люди живуть сито й спокійно.
Репортер Дмитро аж двічі на тиждень подає матеріали в газету, хоча він міг би й
зовсім не працювати, \enquote{а просто спати і їсти}, але хвилюється, що \enquote{тоді не мав
би соціального статусу, а це загрожує сприйняттю себе, шкідливо для психіки.
Багато хто божеволів, зачинившись у своєму маєтку, відгородившись від світу.} 

Дмитро з дружиною та сином опікуються, як і багато їхніх співгромадян, власною
фермою. Мають будинок, який опалюють дровами, хлів і загороди, де тримають
тварин. Оскільки індустріальна цивілізація зазнала краху (в далекому
минулому!), то люди різні роботи виконують без допомоги техніки: орють землю
без тракторів, перуть без пральних машин і т.п. Але, є одне \enquote{але}. Всю важку (й
не дуже) фізичну роботу виконують стори своїми руками - ой, одруківка - лапами;
а ходять вони не ногами - а копитами. Потрібно бути уважним, адже \enquote{спеціальні
служби Рейха стежать за правильним ужитком слів, охороняють споконвічні поняття
цивілізації}. 

Маша - одна з багатьох стор, які живуть в хліві в Дмитра. Саме вона своєю
емоційною поведінкою зрушила ту глибу впевненості чоловіка в тому, що стори -
тварини. Молода й приваблива, а якщо розчесана й у сукні - то виглядає, як
справжня жінка. Ніякої різниці. Сумніви - перший крок від сліпоти до прозріння.
Звісно, Дмитро пройде цей шлях повністю. Пройде з Машою, яка йому довірилась. 

Саме відповідальність за іншу живу душу, за життя беззахисної людини змінить
докорінно його ставлення до власного життя - з того, хто пливе в човні за
течією, він перетвориться на того, хто сам спрямовує свій човен. Воля головного
героя загартується в боротьбі з природними стихіями. Він відчує внутрішню
свободу і вже не захоче її втрачати. З постфашиста він стане людиною. 

Як гадаєте, чи дозволить неогуманістичне постфашистське суспільство жити вільно
такій людині, якою став Дмитро? 

А чи самі однодумці головного героя допустять існування такої держави?

Боротьба триває, як і саме життя. 

\url{https://bookclub.ua/}
