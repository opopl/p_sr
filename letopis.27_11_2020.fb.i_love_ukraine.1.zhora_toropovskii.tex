% vim: keymap=russian-jcukenwin
%%beginhead 
 
%%file 27_11_2020.fb.i_love_ukraine.1.zhora_toropovskii
%%parent 27_11_2020
 
%%url https://www.facebook.com/i.love.ukrainee/posts/673884039983748
 
%%author 
%%author_id 
%%author_url 
 
%%tags ato,ukraine
%%title 
 
%%endhead 

\subsection{Жора Тороповський - пам'ятаємо}
\Purl{https://www.facebook.com/i.love.ukrainee/posts/673884039983748}

\index[names.rus]{Тороповський, Жора!Смерть, 17.09.2014}

За його голову сєпари призначили винагороду.

Жора Тороповський народився у Борисполі.

Змалечку був патріотом. Мав чорний пояс із тхеквондо. 

У 8-му класі з мамою переїхав до Києва. Вчився в НАУ.

Був активним учасником Революції Гідності. Потім приховав свій вік та
неповнолітнім вирушив добровольцем на Схід. Йому було лише 17 років.

Потрапив у 40 батальйон тероборони «Кривбас». Був у розвідувально-диверсійному підрозділі.

Брав участь в багатьох боях. Врятував побратимів під Савур-Могилою. Не ховався
за спинами в Іловайську. Був наймолодшим і найсміливішим.

17 вересня 2014 Жору вбили.

Він їхав потягом з Києва на Донбас через Дніпро, у військовій формі. На нього
напали, тіло знайшли на колії.

Він був єдиним сином у матері, загинув у 18 років.

Сєпари на «Трибуналі» обіцяли винагороду за його голову. Вбивць не знайдено, справу закрили.

Жору посмертно нагородили орденом «За мужність» ІІІ ступеня та низкою інших нагород.

В його честь перейменовано вулиці у Києві та Борисполі.

Пам‘ятаймо!

\ifcmt
pic https://scontent-waw1-1.xx.fbcdn.net/v/t1.0-9/127990866_673882893317196_1596106595635129306_n.jpg?_nc_cat=102&ccb=2&_nc_sid=8bfeb9&_nc_ohc=K2AL_CQCBawAX-mVWFX&_nc_ht=scontent-waw1-1.xx&oh=57899dc9a9e7a5df64ca25c96426b02f&oe=5FF248FA
caption Жора Тороповський
\fi
