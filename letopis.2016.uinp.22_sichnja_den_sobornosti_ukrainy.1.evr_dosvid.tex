% vim: keymap=russian-jcukenwin
%%beginhead 
 
%%file 2016.uinp.22_sichnja_den_sobornosti_ukrainy.1.evr_dosvid
%%parent 2016.uinp.22_sichnja_den_sobornosti_ukrainy
 
%%url 
 
%%author_id 
%%date 
 
%%tags 
%%title 
 
%%endhead 

\subsubsection{Європейський досвід}

Традиція святкувати кілька Днів незалежності є сталою в Європі.

В Литві святами є День Держави 6 липня (в пам'ять 1253 р.), а також Дні
відновлення державності та незалежності 16 лютого (1918) і 11 березня (1990). В
Латвії святковими є День проголошення республіки 18 листопада (1918) та День
декларації незалежності 4 травня (1990). В Естонії є День незалежності 24
лютого (1918) та день відновлення незалежності 21 серпня (1991). В Чехії
святкують День державності 28 вересня (935), День незалежності Чехословаччини
28 жовтня (1918) і День відновлення незалежності 1 січня (1993).

Ці країни, як і Україна, здобули або відновили свою незалежність внаслідок
Першої світової війни, а потім втратили її у міжвоєнний час або після Другої
світової війни, і знову відновили з падінням комуністичного режиму.

\textbf{Тому 22 січня важливо відзначати і як День Соборності, і як День Першої
Незалежності.} Святкування цього дня саме в такому форматі підкреслює тяглість
української державної традиції у ХХ ст.

