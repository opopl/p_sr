% vim: keymap=russian-jcukenwin
%%beginhead 
 
%%file 11_05_2022.fb.denysenko_larysa.1.nasilnik_osvoboditel
%%parent 11_05_2022
 
%%url https://www.facebook.com/larysa.denysenko/posts/10223504742141243
 
%%author_id denysenko_larysa
%%date 
 
%%tags 
%%title Про природу насильницького асвабадітєльства
 
%%endhead 
 
\subsection{Про природу насильницького асвабадітєльства}
\label{sec:11_05_2022.fb.denysenko_larysa.1.nasilnik_osvoboditel}
 
\Purl{https://www.facebook.com/larysa.denysenko/posts/10223504742141243}
\ifcmt
 author_begin
   author_id denysenko_larysa
 author_end
\fi

Сьогодні я послухала один перехват СБУ, там дружина окупанта, котра живе в
прикордонному місті, куди насильно вивозять українських дітей, спочатку
негодує, що діти відмовилися малювати ось цю колорадську гидоту до 9 травня і
говорили, що це не їхнє свято і символіка, а потім каже, що треба було з ними
зробити: колоти наркоту, вирізати зірки на статевих органах і вбивати. Сам
окупант у цій розмові явно почувається не надто комфортно. 

Але все це про природу насильницького асвабадітєльства.

Про ось цей синдром героя-звільнювача, котрого треба вшановувати і молитися на
нього за те, що тебе звільнили. 

І молитися за героїню-матушку, котра допустила тебе, мале нацистське звіря, на
свою територію. 

Бути вдячним та молитися за те, що тебе вивезли без твоєї згоди і часто без
батьків на чужу територію. Бо ж визволили, бо це ж територія безпеки! То треба
складати оди, складати ручки в молитві і не висовуватися. 

Вони не думають, не розуміють, не хочуть знати, що нищать ідентичність дитини,
змушують її жити за чужими правилами, змушують відмовлятися від своєї культури,
саме змушують, бо вона сюди не прибула усвідомлено і вільно. 

Та сама насильницька природа звільнювача, це коли в окупованому місті жінку
ставлять перед вибором, насильство над тобою буде тихим, тому що я – герой, а
не сволота, тому що я забезпечу тобі та твоїй нацистській родині захист, бо ти,
бач, мені симпатична, і давай жити так, щоб ти це приймала, цінувала, а я тебе
буду любити. Саме любити, яке ж тут насильство? Любов та привілейоване
ставлення! 

Бо вже ж наче люблю, бо багато для тебе роблю, інші правила ніж для решти, то
чому ти це не цінуєш, чому ти викабенюєшся, чому ти доводиш, сука, мене до
того, що я вимушений вдаватися до насильства?! 

Я ж цього не хотів, я приніс тобі добро, а що робиш ти? Заперечуєш, не
приймаєш, така невдячність суча. То навчати треба тебе, навчати розуму,
вдячності та повазі. Навчати через насильство і упокорення, якщо ти не приймаєш
любові та звільнення. Довела і отримала.

%\ii{11_05_2022.fb.denysenko_larysa.1.nasilnik_osvoboditel.eng}

\ii{11_05_2022.fb.denysenko_larysa.1.nasilnik_osvoboditel.cmt}
\ii{11_05_2022.fb.denysenko_larysa.1.nasilnik_osvoboditel.cmtx}
