% vim: keymap=russian-jcukenwin
%%beginhead 
 
%%file 22_04_2022.fb.ivleva_viktoria.moskva.ng.1.golosa_iz_metro.cmt
%%parent 22_04_2022.fb.ivleva_viktoria.moskva.ng.1.golosa_iz_metro
 
%%url 
 
%%author_id 
%%date 
 
%%tags 
%%title 
 
%%endhead 
\zzSecCmt

\begin{itemize} % {

\iusr{Алексей Малобродский}
Спасибо, Вика, за свидетельства и за всё, что Вы делаете

\iusr{Victoria Ivleva}
\textbf{Алексей Малобродский} сочтёмся ужо при встрече

\iusr{Денис Ющенко}
Нужны матрасы надувные, помните такие были, на них может теплей на бетоне

\iusr{Victoria Ivleva}
\textbf{Денис Ющенко} кто-то приносит...

\iusr{Sergey Forduy}
Спасибо, можно с вашего разрешения перепощу?

\begin{itemize} % {
\iusr{Victoria Ivleva}
\textbf{Сергей Фордуй} обязательно!

\iusr{Sergey Forduy}
Спасибо! Обнимаю! Берегите себя!

\iusr{Evgenya Moiseeva}
\textbf{Виктория Ивлева} спасибо!
\end{itemize} % }

\iusr{Nataliia Rieznikova}

Вика, огромное спасибо за репортаж. Моя мама с Салтовки, с Бучмы. Приехала ко
мне в Краков 7 марта. Согласилась, после того, как попала под обстрел. Это
возле родника. Ехала трое с половиной суток.Ей 72 года. Сама весит 50 кг, везла
с собой двух своих кошечек, они вместе 13 кг весят. Еле живую ее встретила в
Пшемышле. С сегодняшнего дня, туда проезд закрыт. В ее подъезде остался один
сосед. Я, уже не преследую цель донести до орков страдания украинцев. Я,
преследую цель, о неотвратимости наказаказания. Мы не забудем, и не простим.

\begin{itemize} % {
\iusr{Нина Черадионова}
А почему проезд в Пшемышль закрыли?

\iusr{Valentyna Shulova}
\textbf{Нина Черадионова} очевидно, закрыли проезд в обстреливаемый район Харькова.

\iusr{Victoria Ivleva}
\textbf{Наталия Резникова} маму обнимите покрепче пожалуйста

\iusr{Нина Черадионова}
\textbf{Валентина Шулёва}, 

Пшемышль - это город в Польше и едут туда со стороны Львова. Так что, совершенно
не \enquote{очевидно}(с), как вы предполагаете. Я свой вопрос задала Наталии
Резниковой.

\iusr{Avrora Kreyser}
\textbf{Nataliia Rieznikova} а я моих старичков с улицы Бучмы встретила в Варшаве 11 марта. Нет больше у них своего дома. Не забудем, не простим.

\iusr{Ірина Гаєва}
\textbf{Нина Черадионова} Місто то не закрили. Відмінили потяги евакуаційні.
\end{itemize} % }

\iusr{Женя Дисс}
Вика!

\begin{itemize} % {
\iusr{Victoria Ivleva}
\textbf{Женя Дисс} чо?

\iusr{Женя Дисс}
\textbf{Victoria Ivleva} ти крута.

\iusr{Victoria Ivleva}
\textbf{Женя Дисс} боками разве что

\iusr{Maria Korotayeva}
\textbf{Виктория Ивлева} не только)))

\iusr{Maria Korotayeva}
\textbf{Женя Дисс} и она таки крута!

\iusr{Victoria Ivleva}
\textbf{Мария Коротаева} тише это наша тайна

\iusr{Женя Дисс}
\textbf{Victoria Ivleva} та ладно, знаю я ваши тайньі)
\end{itemize} % }

\iusr{Ольга Слободская}
Господи, ты в Харькове?! береги себя, пожалуйста

\begin{itemize} % {
\iusr{Victoria Ivleva}
\textbf{ольга слободская} да это вы все себя берегите!

\iusr{Ольга Слободская}
\textbf{Виктория Ивлева} обнимаю!

\iusr{Зенина Елена Борисовна}
\textbf{Victoria Ivleva}
Вика, смените смеющуюся аватарку. Война же.

\iusr{Дмитрий Баевский}
\textbf{Ольга Слободская} а что? В Харькове много опасных мест. Но так вообще и во Львове опасно может быть.
Зато у нас свободно

\iusr{Svetlana Vereshchagina}
\textbf{Виктория Ивлева} посмотрите пожалуйста личку

\iusr{Svetlana Lyutkina}
\textbf{Зенина Елена Борисовна} и где логика? Хорошо, когда человек смеётся.

\iusr{Maret Antonenko}
\textbf{Зенина Елена Борисовна} Это вы из Москвы Виктории в Харькове обьясняете про войну и рекомендации даёте? А надо ли? Вдруг Виктория без вас уже знает про войну? Не думали об этом?
\end{itemize} % }

\iusr{Alexander Parkhomenko}
Дякую, Віка! Обіймаю!

\iusr{лариса дмитриева}
Это будет летопись великого противостояния между новым и старым миром.Спасибо.

\iusr{Елена Россина-Черкасова}
Вика, Вы невероятной силы духа человек! @igg{fbicon.heart.red}

\iusr{Iryna Viskovatova}

Вика, Вы на Салтовке, это же невероятно как-то, с ума сойти. Спасибо за
репортаж практически из дома, как я хочу туда вернуться.

\iusr{Тамара Богомолова}
Спасиб, дорогая Вика! Будьте, пожалуйста осторожны!@igg{fbicon.heart.red}

\iusr{Татьяна Дендюк}

Спасибо. Смутил только 1 момент \enquote{Кормят здесь хорошо. Спасибо Фельдману,
узбекам, гриль-бару}. Все эти люди, кто готовит, привозит еду заслуживают
отдельного репортажа. Хотя бы потому, что эти незаметные герои нашего времени
подвергают себя гораздо большему риску.

\iusr{Olga Popova}

Я уже почти месяц работаю с детьми на станциях метро, которые в Харькове
работают как убежища. Если Вам интересно, как там все происходит, велкам.

Это реально больно.
Это реально непросто.

\iusr{Irina Soboleva}
Вика, спасибо Вам  @igg{fbicon.heart.sparkling}  мой любимый город. Я продержалась 57 дней. Передохну чуть и вернусь. @igg{fbicon.hands.raising} 

\iusr{Людмила Юрьевна Пантелеева}
О Боже!!! Как же тяжело! Молю, чтобы всё ЭТО быстрее закончилось!  @igg{fbicon.heart.yellow}  @igg{fbicon.heart.blue} 

\iusr{Ольга Кашайкина}
Сил все это выдержать. Верим ЗСУ и в победу!  @igg{fbicon.heart.blue}  @igg{fbicon.heart.yellow} 

\iusr{Alla Bossart}
Вика, это потрясающе. Сердце рвется. А они не верят!!!

\iusr{Рыжая Ната}
\textbf{Алла Боссарт} Прихожу на работу, а они какую- то хрень патриотическую в стихах и с музыкой слушают,лица серьезные,воодушевленные. Жуть. Готовы в бой.

\iusr{Alina Koval}

Вот такие реалии нашей жизни в Украине теперь, мы жили себе спокойно, никого не
трогали, ни на кого не нападали, ни с кем не воевали и вот...

\begin{itemize} % {
\iusr{Наталия Лесконог}
\textbf{Алина Коваль} мы очень хотели напасть, но не успели, это вам на россии почти всякий скажет

\iusr{Elena Richter}
\textbf{Наталия Лесконог}, не надо их цитировать, слишком много чести. Пусть они варятся в своём котле с ядом.

\iusr{Ірина Білосвіт}
\textbf{Алина Коваль} 

мы с вами слишко жили, слишком! Вольно, так, как мы хотели... а в том краю так
нельзя....Только идет он вслед за кораблем, медленней чем нам хочется, но идет... И
все у нас буде наша Украина!

\iusr{Віолета Левіна}
\textbf{Алина Коваль} Гитлер так тоже нападал, страны жили себе, никого не трогали. Обыкновенный фашизм
\end{itemize} % }

\iusr{Светлана Валова}

Наши многострадальные украинцы при самых неблагоприятных обстоятельствах всегда
остаются людьми. Как у них чисто, как все организовано, спокойно. Если бы это
были Россияне - обязательно был бы бардак. Вика, как всегда,восхищаюсь тобой и
тем, что ты делаешь. Спасибо. Ты такая классная.

\iusr{Gleb Nikolaev}

Виктория, Вика, спасибо! Хотел было лайкнуть \enquote{мы вместе}, но вовремя себя
одёрнул. Мой диван - не бетонный пол в метро (((

\iusr{Olena Titarenko}
Мой родной, любимый Харьков....

\iusr{Elena Efimova}
Спасибо, Вика.

\iusr{Alisa Dubytskaya}

Это станция была нашим домом, мы там с 24 три дня жили...

Удивительно, что метро может стать домом, когда у тебя есть свой дом.

Спасибо за репортаж из дома  @igg{fbicon.hands.pray}

\iusr{Екатерина Хмелева}
Спасибо за свидетельства. Будьте целы. Берегите себя  @igg{fbicon.hands.pray} 🏻 ☘

\iusr{Elena Makarova}
Дорогая Вика, спасибо Вам.

\iusr{Rosa Gimatdinova}
Спасибо за этот репортаж!

\iusr{Natasha Kondakova}
Вика, спасибо вам...

\iusr{Руслан Баймурадов}

Как это понятно и больно... Вспоминаю мою Чечню в войну. Враг - тот же, методы
- те же, страдания ни в чём не повинных людей - то же... Но в таких страданиях
рождается очень твердый характер и сила духа. Непобедимого духа. Дай Боже вам,
украинцы, выстоять и победить!

\begin{itemize} % {
\iusr{Ilona Gasanova}
\textbf{Руслан Баймурадов} главное чтоб результат был не тем же, что у Чечни ...

\iusr{Руслан Баймурадов}
\textbf{Ilona Gasanova} , надеюсь, у Украины хватит сил.

\iusr{Ірина Гаєва}
\textbf{Руслан Баймурадов} ми встоїмо й усе відбудуємо.
\end{itemize} % }

\iusr{Tetiana Horbatiuk}
\textbf{Спасибо Вика} зато, что не даёте забыть все это ,,,

\iusr{Тамара Сидак}
Нет слов, только боль, везде, в сердце, в каждой клеточке..

\iusr{Вита Пономарева}
Родной, любимый Харьков. Больно. Благодарю за роботу. Молюсь @igg{fbicon.hands.pray} 

\iusr{Наталья Урина}
Дорога Вікторіє, дякую, що ви з нами! Люблю вас! @igg{fbicon.heart.blue}  @igg{fbicon.heart.yellow} 

\iusr{Olena Sychova}
Віка! Велике дякую Вам!  @igg{fbicon.heart.blue}  @igg{fbicon.heart.yellow} 

\iusr{Helen Bresniak}
СпасиБо!

\iusr{Ирина Хмелевская}
Спасибо

\iusr{Ирина Филон}
Дякую

\iusr{Svetlana Koroleva}
Благодарю!!!!

\iusr{Марина Чередник}
Рідне місто. Біль.

\iusr{Viola Strode}
 @igg{fbicon.flag.ukraina} @igg{fbicon.hands.pray} @igg{fbicon.heart.red}🇱🇻

\iusr{Svetlana Gumenuk}

Мой племянник доброволец не профи военный. В харьковском метро, когда были
бомбежки в начале войны, он познакомился с девушкой. Они решили пожениться.

Уже расписались официально (по законам военного времени это можно сделать
быстро). Даже видео было на одном из больших телеграмм каналов.

\iusr{Irina Kara}
Какой ужас, какой ужас.. за что?

\iusr{Lora Soroka}

Вика, низкий Вам поклон. Вы- необыкновенная. Харьков - мой родной, бесконечно
любимый город. Я жила на Салтовке и работала там в школе. Ученики мои жили на
Салтовке.... Как же больно

\iusr{Tatjana Chimes}

Вика, спасибо большое за ваши репортажи. Сколько беды... Всем сердцем с
Украиной и украинцами.

\iusr{Galina Laurinavičienė}
Спасибо за эти свидетелства

\iusr{Anna Salnikova}

Знакомая девочка с Салтовки - выехала в первый же день к бабушке в Полтавкую
область. Я у нее спросила - как там твой дом? Она заплакала - его больше
нет.....

\iusr{Ira Kagran}
Господи, помоги уже наконец

\iusr{Kira Volynskaya}
Спасибо большое. Я тоже утащила на перепись. Какие люди!

\iusr{Olga Myakenko}
Спасибо за честность. Спасибо, что вы с нами!

\iusr{Irina Besedina}
\textbf{Dina Brodetska}.. Динуль твой Харьков.. Читаю и плачу...  @igg{fbicon.cry} 

\iusr{Misha Nodelman}
Спасибо за летопись, Вика

\iusr{YeLena Zhukovskaya}
Моє фото.
КИЇВ. Метро. Перший день війни...

\ifcmt
  ig https://scontent-mxp1-1.xx.fbcdn.net/v/t39.30808-6/278563204_3192969474308745_2019450309366076708_n.jpg?_nc_cat=101&ccb=1-5&_nc_sid=dbeb18&_nc_ohc=ccREoXmEUNcAX9ivKW7&_nc_ht=scontent-mxp1-1.xx&oh=00_AT9ikDSyuHWbOrhIvu6OjmzziQ4j__SCBI-LDhUXn5544g&oe=6269A219
  @width 0.3
\fi

\iusr{Sevda Kurbanova}

У моей приятельницы пожилые родители в Харькове тоже с первых дней в метро
находятся безвылазно. Отцу 85, матери 84.

\iusr{Григорий Михнов-Вайтенко}
Метро 2022

\iusr{Ekaterina Mamontova}
\textbf{Григорий Михнов-Вайтенко} такая же ассоциация сразу возникла, с романами Глуховского.  @igg{fbicon.cry} 

\iusr{Irina Eryomenko}
Вика,  @igg{fbicon.heart.yellow}  @igg{fbicon.heart.blue} .
Потому что слов нет  @igg{fbicon.frown} 

\iusr{Elena Levchenko}
Боль.Боль.

\iusr{Дина Кедрова}
Храни Вас Бог.

\iusr{Эвелина Меленевская}
В голове не укладывается, как такое возможно. Сил вам.

\iusr{Юлия Нескоромная}
\textbf{Антон Вуйма}
Чтобы ты знал -как украинцы от ваших рашистов «асвааабаадителей» прячутся

\iusr{Alex Melnik}
Виктория, сил Вам и удачи!)
Украина с Викторией обязательно Победит!
Молимся за это:
И да \enquote{сгинут наши вороженьки!}
\url{https://youtu.be/7WeYI6enRKo}

\iusr{Marina Yurevna Markina}

Виктория, спасибо Вам огромное! Вы смелый и честный человек!

\iusr{Yurii Tashkinov}
Спасибо, что вы там! Сил вам и вашим героям

\iusr{Жанна Евдокимова}

Вика, бесконечно люблю и уважаю Вас. Господи, как достучаться здесь в России до
глухих и слепых?? Как можно хотеть смерти людей? Это ужас.

\iusr{Yury Mihaylidi}
Сил вам все это пережить и одолеть зло!

\iusr{Anastasia Nilskaya}
Боль страшная

\iusr{Зоя Саватеева}
Удивили МУЖИКИ ... им нет работы в войну?

\iusr{Ratushnyak Oleksandr}
\textbf{\#RussiaIsEvil}

\iusr{Наталія Наталія}

Я только в прошлом сентябре спланировала себе поездку в Харьков и много там
чего посмотрела... И на выставке была, и в театре, и пешком по городу, и парк
Горького, и студенты, и Дворец бракосочетаний с многонациональными
брачующимися... Понравилось.. И теперь вот они расстреливают это тепло и
красоту, людей и животных, здания и дороги.. Ужас, смириться невозможно..

\iusr{Inga Vasilieva}
Господи какой стыд... какая боль

\iusr{Maria Hotina}
Когда же кончится этот кошмар??! Запредельное зло

\iusr{Нина Черадионова}

Какой ужас! Спасибо вам, Виктория Ивлева (\textbf{Victoria Ivleva}) за эти короткие
интервью. Они короткие и, в то же время, очень ёмкие. Очень надеюсь, что ваши
читатели из России поймут весь ужас происходящего. И Вы берегите себя, ведь
бомбят Харьков, как скаженные.

\iusr{Katja Dronova}
Господи...

\iusr{Olexander Starish}

Надеюсь, идеологическая пуповина о то ли одном народе, то ли о народах-братьях
будет навсегда разорвана именно этой Российско-русской - между
Московщиной/Россией и Киевской Русью/Украиной - войной.

\iusr{Irina Matskevich}
Под землёй....

\iusr{Elena Schlosser}
Спасибо за Вашу летопись!@igg{fbicon.flag.ukraina}

\iusr{Оксана Пічкур}

Это ещё хорошо, что есть куда спрятаться. В Мариуполе нет метро. Не возможно
передать это чувство беззащитности и бессилия перед падающими с неба бомбами и
прилетающими снарядами. Иногда было просто невыносимо это терпеть и казалось,
что уже все, конец. Но из нашей квартиры было видно позицию наших военных.
Когда я видела как они спокойно, занимаются своими делами, приезжают, уезжают,
то становилось спокойнее. Я им даже завидовала. Казалось, уж лучше что-то
делать, чем сидеть в ванной и ждать смерти.

\iusr{Виктория Носарева}
У Глуховского \enquote{Метро 2033}. Он ошибся всего на две 1-цы-Метро 2022. Пусть будут прокляты рашисты!

\iusr{Yulia Kuznetsova}
Спасибо, Виктория

\iusr{Kira Lander}
Мой Харьков, я скоро вернусь @igg{fbicon.heart.red}

\iusr{Saida Sosedko}
 @igg{fbicon.heart.broken} 

\iusr{Elena Prihodko}
Город моей институтской юности. Любимый.
Так больно ...

\iusr{Galina Bondareva}

Невозможно. Башкой качаю, ногти , сорри, грызу. Ужас забирает и ненависть,
такая, что иногда еле справляешься. Горе, горе, горе...

\iusr{Марина Рогачева}
Плачу

\iusr{Karina Gyulushanyan}
господи какой ужас, боже мой. вот за что(..

\iusr{Alla Arkhangelskaya}
Господи. Плачу, плачу, плачу

\iusr{Екатерина Харитонова}
Спасибо вам, Вика

\iusr{Konstantin Levay}
Как же больно читать... Берегите себя!

\iusr{Iryna Sagina}
Дорогие мои харьковчане!!! Нет слов...

\iusr{Olga Grishina}

Вот этим волонтерам мы тут как раз и искали кастрюлю на 50 литров. К счастью,
одну им удалось купить, а другую им подарили

\end{itemize} % }
