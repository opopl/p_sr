% vim: keymap=russian-jcukenwin
%%beginhead 
 
%%file 15_01_2022.fb.fb_group.story_kiev_ua.2.ulica_grafskaja.cmt
%%parent 15_01_2022.fb.fb_group.story_kiev_ua.2.ulica_grafskaja
 
%%url 
 
%%author_id 
%%date 
 
%%tags 
%%title 
 
%%endhead 
\zzSecCmt

\begin{itemize} % {
\iusr{Натали Булах}
Спасибо! Такие теплые воспоминания о родном ГОРОДЕ!

\iusr{Владимир Мареев}

Красивая улица.

\iusr{Валентина Мещерякова}

Спасибо за Ваш рассказ. Прочитала с удовольствием. Мне очень нравится ул.
Лютеранская, её дома и особая атмосфера. Я тоже коренная киевлянка с ул.Ленина
( теперь Б. Хмельницкого).

\begin{itemize} % {
\iusr{Александр Гринь}
\textbf{Валентина Мещерякова} 

Было давно, но часть детства прошла на ул. Ленина, 51... Случается бывать там не
часто, но всегда с удовольствием вспоминаю ту необыкновенную атмосферу конца
50х - начала 60х!!! @igg{fbicon.thumb.up.yellow} 

\iusr{Елена Еременко}
\textbf{Александр Гринь} Я тоже жила на Ленина, 51, в 1970 мы получили квартиру и переехали. Через проходняк я ходила в 50-ю школу, что была по переулку Белинского

\iusr{Александр Гринь}
\textbf{Елена Еременко} В 50й школе учился с 1по5 класс (1959-64гг), а после переехали на Русановку!

\iusr{Елена Еременко}
\textbf{Александр Гринь} Я училась с 1958 года и до 8-го класса, она была восмелеткой.
Я жила в кв 33, а Вы в какой?!

\iusr{Валентина Мещерякова}
\textbf{Александр Гринь} А я тоже училась в школе 50 с 1966 по 1973. Потом нас всех перевели в 135 школу на ул. Коцюбинского. Я жила тогда в доме 65 по ул. Ленина.

\iusr{Александр Гринь}
\textbf{Елена Еременко} 

N уже и не помню, но парадное прямо в первом дворе, этаж 4, балкон, коммуналка,
10 соседей (и все жили мирно несмотря на условия), рядом сейчас отель \enquote{Opera}
кажется...


\iusr{Елена Еременко}
\textbf{Александр Гринь} у меня тоже первый двор, парадное прямо против входа во двор

\iusr{Елена Еременко}
\textbf{Александр Гринь} Да рядом в доме отель, садик \enquote{Дом молитки} снесли, на его месте - парковка отеля
\end{itemize} % }

\iusr{Оксана Денисова}

Очень люблю эту улицу, вожу по ней экскурсии. Вам огромное спасибо за Ваши
воспоминания, они бесценны, так интересно читать @igg{fbicon.heart.red}

\iusr{Елена Муравьева}

Василий Тарновский в 1876 г. снял квартиру на углу улиц Лютеранской и
Левашевской в доме Александра Гудимы-Левковича. На Лютеранской улице жила и
будущая жена его сына, тоже Василия Тарновского. Графиня Мария О'Рурк Тарновская
стала одной из самых известных женщин-убийц.

\begin{itemize} % {
\iusr{Maksim Pestun}
\textbf{Елена Муравьева} известная история! На днях был в этом доме. Там об этом ничего не знают...

\iusr{Maksim Pestun}

А еще на втором этаже жила моя одноклассница. Ее дедушка был ректором института
ГВФ. Бывал у нее в гостях.

\iusr{Елена Муравьева}

я живу рядом гвф. почти год разбираюсь с историей марии тарновской. половина
того что о ней написано - полная чушь. но то, что она жила на лютеранской -
истина безусловная

\iusr{Maksim Pestun}
\textbf{Елена Муравьева} 

думаю, как и в любой истории, выдумка перемешана с правдой, но то, что ее семья
жила в этом доме не оспаривается.

\iusr{Anastasia Raint}
\textbf{Елена Муравьева} 

скажите пожалуйста, можно ли где-то прочесть о ней!?!?... Я имею ввиду, более-
менее , правду ! @igg{fbicon.face.upside.down} 

\iusr{Елена Муравьева}

наверное, нет. все что я нашла \enquote{калька} с какой-то дряной халтурки, в которой
перепутано праведное и грешное. хотя она, действительно, преступница. а вот
неуголовная сторона ее жизни искажена до невозможности.


\end{itemize} % }

\iusr{Светлана Александренко}
Блестяще! Ваш юмор и интеллект неотразимы! Спасибо за пост.

\iusr{Inna Kontra}

Spasibo za Vas rasskaz. Ocheny lublu Kijev, prijatnie vospominanija, ja zila na
ulice Bankovaja(Ordzenikidze)i vsegda vspominau nase detstvo, sanki, skola.
Spasibo.

\iusr{Вікторія Святненко}
Чудесные воспоминания... спасибо!

\iusr{Галина Гурьева}
Очень тепло, уютно, по~домашнему)Самое то в промозглый зимний вечер. Спасибо!

\iusr{Vladislav Priemsky}
Дякую за майстерно написані та душевні спогади!

\iusr{Анна Сидоренко}
Спасибо.

\iusr{Юрий Галецкий}
Я тоже начинал учиться в 147 им Радищева. А потом.... Суп с котом

\iusr{Анастасия Турчина}
Прекрасные воспоминания @igg{fbicon.dizzy} 

\iusr{Liudmila Kabanova}

Прекрасные воспоминания. Спасибо большое. Тем более, что с детства знакома с
ближе прилегающими улицами: Липская, Шолковичная, Круглоуниверситетская.


\iusr{Efim Resser}

Очень интересный рассказ! Я тоже учился в 147 школе с 1 класса вместе с сыном
директора школы Трофимом Ивановичем Уриловым Юрой. К сожалению он умер
несколько лет назад! А кто ещё был в вашем классе? Мне очень интересно!

\begin{itemize} % {
\iusr{Maksim Pestun}
\textbf{Efim Resser} С Сыном Урилова я сидел за одной партой. Хорошо помню его отца.

\iusr{Efim Resser}
\textbf{Maksim Pestun} если вы сидели с Юрой за одной партой, то где сидел я и все наши одноклассники?

\iusr{Maksim Pestun}
Видимо, это другой сын был... @igg{fbicon.smile} 

\iusr{Maksim Pestun}
Школу между собой называли \enquote{Уриловка} @igg{fbicon.smile}  Т. И. Урилов
был фронтовиком без руки. Его очень уважали

\begin{itemize} % {
\iusr{Maksim Pestun}
Саша Чайка, Леня Хайкельсон, Олег Гузерчук, Ира Жукова, Аня Швыдкая, Вассер....

\iusr{Александра Тарнавская}
\textbf{Maksim Pestun} 

Тоже усилась в 147 школе, помню нашего Трофима Ивановича Урилова! При мне школу
переименовали (соединили) с 132-й, которая была на Дарвина

\iusr{Efim Resser}
\textbf{Maksim Pestun} 

тогда это не тот класс где учился Юра Урилов. Мы учились с 1959 до 1969 года и
в 8 классе к нам присоединили 83 школу, как раз о которой вы рассказывали! В
этой школе ещё до войны работал Трофим Иванович. Там училась ещё моя мама и
выпуск был в 1937 году!

\iusr{Efim Resser}
\textbf{Maksim Pestun} 

тогда это не тот класс где учился Юра Урилов. Мы учились с 1959 до 1969 года и
в 8 классе к нам присоединили 83 школу, как раз о которой вы рассказывали! В
этой школе ещё до войны работал Трофим Иванович. Там училась ещё моя мама и
выпуск был в 1937 году!

\iusr{Maksim Pestun}
\textbf{Александра Тарнавская} я как раз в этом году (когда произошло слияние) ушел в другую школу.

\iusr{Ксения Погорлецкая}
\textbf{Maksim Pestun} 

Саша Чайка жил в доме на Крутом Спуске; дом угловой, квартира на первом этаже ?
Родителей его не помните?

\iusr{Maksim Pestun}
\textbf{Ксения Погорлецкая} 

помню его отца! Часто бывал у них в гостях в огромной коммуналке. Потрясающий
дом! Пару лет назад с ним виделся

\iusr{Александр Навольнев}
\textbf{Maksim Pestun} Я её заканчивал, но она в 1970 г. поменяла номер и стала \#132.

\iusr{Наталья Кузнецова}
\textbf{Maksim Pestun} 

простите за нескромность, но у Трофима Ивановича была не фронтовая травма. Это
так, к седению. ЮРА Урилов умер когда ему только исполнилось 60. Скоропостижно.
Все воспоминания настолько близки и все сразу перед глазами, как вчера...

\end{itemize} % }

\iusr{Лариса Сенецкая}
\textbf{Efim Resser}

Фима, я тоже твоя одноклассница, если помнишь! Часто вспоминаю наших учителей:
Марию Абрамовну Степанскую, Семёна Александровича Зарицкого...

\begin{itemize} % {
\iusr{Лариса Сенецкая}

\ifcmt
  ig https://scontent-frt3-1.xx.fbcdn.net/v/t39.30808-6/272028313_4850460445073920_7127285786249525119_n.jpg?_nc_cat=102&ccb=1-5&_nc_sid=dbeb18&_nc_ohc=WlMYvUPWddIAX-XA7YH&_nc_ht=scontent-frt3-1.xx&oh=00_AT_kYC-jDn1oJMO8jnArhVKqrIZ2CHTdIkVE0PclI32AsA&oe=61EB3CEB
  @width 0.2
\fi

\iusr{Efim Resser}
\textbf{Лариса Сенецкая} 

привет! Помню тебя хорошо, но мне кажется ты ушла раньше! Мы встречались все
вместе последний раз в 2019 на 60 летие окончания школы. Было человек 18. Галя
Свяженина, оба Гусева, Саша Олофинский, Вася Муковоз, Корсакова, Фоменко,
Лисицын, Янкин, Ячменев, Терехова, Казаченко, Таня Медведева и ...

\iusr{Лариса Сенецкая}

Фима, я в десятом классе брала академотпуск по состоянию здоровья и поэтому
закончила школу на год позже. Сто лет не видела никого из наших. Помню всех!
Нашего дома на Лютеранской 31 уже давным-давно нет... Живу на Минском массиве.

\iusr{Сергей Малютин}
\textbf{Лариса Сенецкая} 

Это и мои учителя в старших классах. Начало моей учебы в 147 школе - 1953 год.
Тогда она была мужской. А в следующем году добавили девочек.

\iusr{Лариса Сенецкая}
\textbf{Сергей Малютин}

А я с первого класса в 1959 году и до пятого класса училась в 77 школе, а в 147
перешла в пятом классе.

\end{itemize} % }

\end{itemize} % }

\iusr{Ірина Бондарєва}

Очень красиво, талантливо и интересно!

Спасибо! Мне посчастливилось бывать пару раз в гостях у скульптора Ковалева,
автора памятника Лысенко возле оперного... кстати, не Ваш дедушка? Правда,
кажется, он был одинок... Он жил в доме, который есть на Ваших фото... очень
интересная квартира... две большие смежные комнаты

\iusr{Maksim Pestun}
\textbf{Ірина Бондарєва} 

это не мой дедушка, но я бывал у него в розовом доме в мастерской

\iusr{Lessya Kotovskaya}

Написано с чувством.

Спасибо за рассказ.

Вспомнилось своё детство... кстати... очень похожее и с такими же похожими
воспомина6иями про кораблики, перегонки, санки и т.д., правда, достаточно
далеко от Киева;)

\iusr{Наталия Жминко Сычевска}

Редко. Кто так нежно, именно нежно, вспоминает о своём детстве на любимой
улице ! Моя улица выше Левашовская - К.Либкнехта - Шелковичеая. Были еще
названия, но я их не застала.

Все, что Вы написали, пережили и мы

А институт сахара уже давно не здесь. Это здание хотел кто-то забрать из власти
Януковича. Вот с тех пор и стоит пустой.  Спасибо Вам за ваши воспоминания.

\begin{itemize} % {
\iusr{Alexey Paschenko}
\textbf{Наталия Жминко Сычевска} 

Это здание Германия просила отдать под немецкий центр (гимназия, университет
Гёте). Ведь оно и построено было на деньги немецкой общины как гимназия Было
письмо, подписанное Меркель.

Но....

\begin{itemize} % {
\iusr{Наталия Жминко Сычевска}
\textbf{Alexey Paschenko} 

даже в нем пару лет был КТИПП- пищевой институт. А для школы очень хорошее
здание ! Вы же видели , что творится возле школы номер 86 !

\iusr{Alexey Paschenko}
\textbf{Наталия Жминко Сычевска} 

к сожалению, власть имущие не хотят чтобы в здании развивался образовательный
центр. Возможно здание уже продано и тут будет очередной жилой, офисный или
административный комплекс

\end{itemize} % }

\iusr{Олена Андурова}
\textbf{Наталия Жминко Сычевска} 

До Левашовской была аптекарская. Это тоже моя родная улица, жила в доме 7а во
втором парадном. Как попадаю туда, сжимается сердце. Сейчас живу с мужем на
Прорезной, но так хочется обратно...

\end{itemize} % }

\iusr{Наталия Калинина}

Ваш дом я хорошо знаю! В нем жила моя школьная пожружка. А мой дом на другой
стороне перекрестка - угол Лютеранской и Кругло-Университетской))) Спасибо за
теплые вопоминания!


\iusr{Світлана Проценко}
Спасибо Киевлянин

\iusr{Lara Ilich}

как я люблю такие личные воспоминания... Может быть они не так документально
точны, но зато овеяны таким теплом, которое впечатляет сильнее любых
документов. На санках по Лютеранской! А мы уже позже, гораздо позже, ещё в 70-е
съезжали на санках по Либкнехта -Шелковичной. А дочка моя ходила в тот садик на
горке над площадью Франко. Весной 1987 года, когда растаял снег, пришли мерять
уровень радиации на территории садика. На следующий день я спрашиваю, каков
результат? Они мне с горечью ответили: \enquote{а вы что, не видите, что мы
слой земли снимаем}...

\begin{itemize} % {
\iusr{Виктор Бухтияров}
\textbf{Лара Іліч} хороший был детский садик. С тех пор я и не ем молочные рисовую и гречневую каши.  @igg{fbicon.smile} 

\iusr{Lara Ilich}
\textbf{Виктор Бухтияров} а манную, состоящую из одной пенки?

\iusr{Олена Андурова}
\textbf{Lara Ilich} 

Из-за радиации в свое время закрыли площадку для прогулок в садике на
пер. Шевченко, перенесли ее куда-то в яры за дом. Неужели там фон был лучше? Садик
был рядом с домом, а я водила сына в сад на Горку. Далековато, но хоть в парке
гуляли, а не за домом.


\iusr{Lara Ilich}
\textbf{Олена Андурова} фон везде очень изменчивый. у нас перед домом было чисто, а за домом - очень грязно. достаточно ветру занести частицу - и всё...
\end{itemize} % }

\iusr{Наталья Корниенко}
Всем советую весной пройтись по Лютеранской с фотоаппаратом, удивительно и познавательно!

\iusr{Лариса Артемчук}
Школа на Энгельса была 83.
А директор был Попов.
А насчёт 147 школы, Юра Урилов учился в параллельном классе.
С Гусевыми, Бобовским, Муковоз, Шурепов.
Так?
Значит мы были знакомы

\begin{itemize} % {
\iusr{Maksim Pestun}
я пришел после 83 ей школы только во второй класс. Учился до 6 го

\begin{itemize} % {
\iusr{Лариса Артемчук}
\textbf{Maksim Pestun}
Понятно. А мы пришли из 83 школы
По-моему в 8 класс

\iusr{Maksim Pestun}
я подозреваю, что у Урилова был не один сын.

\iusr{Maksim Pestun}
сейчас мне кажется, что его звали не Юра...
\end{itemize} % }

\iusr{Efim Resser}
\textbf{Лариса Артемчук} 

конечно мы были знакомы. Все эти выше перечисленные ребята учились со мной в
одном классе, а вы с Ирой Пименовой, Таней Коротковой, Талой( не помню фамилию)
которая потом вышла замуж как раз за Юру Урилова! Так что будем обмениваться
воспоминаниями! @igg{fbicon.hands.shake}  @igg{fbicon.hibiscus} 

\begin{itemize} % {
\iusr{Maksim Pestun}
\textbf{Efim Resser} мы учились в разные годы... И у Юры, видимо, был брат

\iusr{Лариса Артемчук}
\textbf{Maksim Pestun}
У Юры была сестра старшая и брат Вова
По-моему

\iusr{Лариса Артемчук}
\textbf{Efim Resser}
Тала Кузнецова. Мы с ней общаемся.
А Юра лет 5 назад умер

\iusr{Maksim Pestun}
\textbf{Лариса Артемчук} я вроде вспомнил. Моего одноклассника звали Вова

\iusr{Лариса Артемчук}
\textbf{Maksim Pestun}
Значит все мы помним
\end{itemize} % }

\iusr{Лариса Сенецкая}
\textbf{Лариса Артемчук}
Да, в нашем классе учились Светик и Саша Гусевы, Светик хорошо пел в школьном
хоре, помню Васю Муковоза, Лёню Самофалова...

\iusr{Виктория Лазния}
\textbf{Лариса Артемчук} 

Я училась до 7го класса в 51-ой школе, а с 7-го по 11-ый - 83-ей. И Попов
преподавал у меня математику в старших классах, кстати, очень \enquote{вразумительно}
все объяснял.

\iusr{Лариса Артемчук}
\textbf{Виктория Лазния}
Я его помню всегда в чёрном костюме. Но у нас он ничего не читал. Я про Попова

\iusr{Ольга Шадрина Черная}
\textbf{Лариса Артемчук} Попов Василий Тихонович. Математику преподавал

\end{itemize} % }

\iusr{Svitlana Chernova}
Спасибо Вам за интересные воспоминания @igg{fbicon.face.smiling.eyes.smiling} 

\iusr{Олена Медведева- Прицкер}

Прекрасные воспоминания Максима! Вы написали, что в один из домов на Печерске
сносили радиоприемники по приказу немецкой власти. Хотелось бы узнать
достоверно, да спросить уже не у кого. В пятидесятые годы у нас был старенький
радиоприёмник «Рекорд», который родители сдавали куда-то на хранение после
начала войны. Папа ушёл на фронт в июне 1941, мама собрала и увезла в эвакуацию
семью, состоящую из детей, бабушек, деда, беременной сестры. Куда киевляне
сдавали радиоприёмники до оккупации Киева немцами?

\begin{itemize} % {
\iusr{Maksim Pestun}
\textbf{Олена Медведева- Прицкер} 

спросить уже не у кого... думаю, таких мест было несколько. Одно из них на
Крещатике, которое подорвали одним из первых. Мне рассказывал дедушка про Кирху

\begin{itemize} % {
\iusr{Ксения Погорлецкая}
\textbf{Maksim Pestun} Мама с бабушкой всю оккупацию прожили в Киеве на Лютеранской (дом на углу Кругло-Университетской там где был хлебный а сейчас антикварный) и никогда не одна из них о сдаче радиоприемников в Кирхе не упоминали.

\iusr{Maksim Pestun}
\textbf{Ксения Погорлецкая} а в этом доме жила моя хорошая знакомая Зина Чайка. У нее еще другие фамилии были. Комната на последнем этаже.

\iusr{Maksim Pestun}
Моя одноклассница тоже жила в том же парадном

\iusr{Ксения Погорлецкая}
\textbf{Maksim Pestun} Парадное в котором она жила выходило на Кругло-Университетскую? Я жила на последнем этаже но парадное выходило на Лютеранскую. Все 4 квартиры обоих парадных последнего - 4 этажа - выходили на один т.н. "черный" балкон. Вспомнить Зину не м... Ещё

\iusr{Maksim Pestun}
\textbf{Ксения Погорлецкая} С Зиной только что списывался. Возможно, Вы помните ее маму. Мама жива. Их парадное выходит на Круглоуниверситетскую. Часто там бывал. Она потом купила квартиру в 25 номере по Лютеранской.

\iusr{Maksim Pestun}
Зине 56 позавчера исполнилось

\iusr{Maksim Pestun}
Над их квартирой помнится был пожар в конце 90тых

\iusr{Ксения Погорлецкая}
\textbf{Maksim Pestun} 

Зина моложе меня на 5 лет и странно что не могу ее вспомнить. Она жила в
квартире дверь кухни или окно выходило на \enquote{черный} балкон? Возможно, помню
маму. О пожаре не знала так как поменяла комнаты - к огромному сожалению! - и
переехала в 92 году.


\iusr{Ксения Погорлецкая}
\textbf{Maksim Pestun} Значит горел чердак т.к. квартира была на последнем этаже.

\iusr{Ксения Погорлецкая}
\textbf{Maksim Pestun} Сашу Чайку с которым Вы учились помню хорошо, он был старше и дружили наши мамы.
\end{itemize} % }

\end{itemize} % }

\iusr{Инна Валентиновна}

Как в песне \enquote{зато у нас было детство,} настоящее, активное. Спасибо за такое
душевное повествование. Легко читается, без лишнего нагромождения фактов из
Википедии. Пишите, нас мало, но мы как говорится в \enquote{тельняшках}.


\iusr{Людмила Шокина}
Спасибо! Очень интересный рассказ-воспоминание. Познавательно на все времена.
Ваша история - это история Киева! БРАВО!

\iusr{Luda Veprik}
А в каком году вы переехали на лютеранскую? Мы там жили.мой сын также катался с горки до самой арки

\begin{itemize} % {
\iusr{Maksim Pestun}
\textbf{Luda Veprik} Примерно в 1962 году

\begin{itemize} % {
\iusr{Luda Veprik}
\textbf{Maksim Pestun} а мы в 1986

\iusr{Luda Veprik}
Мы жили в доме номер 3

\iusr{Maksim Pestun}
\textbf{Luda Veprik} у меня там жили знакомые. Таня Чистякова и Сережа Давыдов

\iusr{Luda Veprik}
А сейчас. когда были. даже во дворик нельзя зайти

\iusr{Luda Veprik}
\textbf{Maksim Pestun} к сожалению. не знаю

\iusr{Ксения Погорлецкая}
\textbf{Luda Veprik} В 86 катания на Лютеранской уже не было, катались с горки возле Дома с химерами и на Шелковичной.

\iusr{Luda Veprik}
\textbf{Ксения Погорлецкая} конечно катались до 94 года. а дальше не знаю. наши дети катались прям на протяжении дома номер3

\iusr{Ксения Погорлецкая}
\textbf{Luda Veprik} Это же крошечный кусочек улицы. От третьего дома до арки два шага. Раньше катание было с верха - почти от здания кирхи.

\iusr{Лина Дериенко}
\textbf{Maksim Pestun} Сережа мой одноклассник... Как мир тесен.
\end{itemize} % }

\iusr{Luda Veprik}
И катались возле дома с химерами

\iusr{Luda Veprik}
\textbf{Ксения Погорлецкая} дети были маленькие. им хватало этого отрезка

\iusr{Ксения Погорлецкая}
\textbf{Luda Veprik} Возле Дома с химерами было очень активное катание до позднего вечера. Все освещено, снег сверкает..

\iusr{Luda Veprik}
\textbf{Ксения Погорлецкая} да. я это помню

\iusr{Татьяна Захарова}
\textbf{Luda Veprik} мальчишки всегда старались подрезать, крутая горка была, нужно было притормозить перед ступеньками
\end{itemize} % }

\iusr{Олена Шинкаренко}

Читая такие теплые воспоминания детства ловишь себя на мысли, что чем-то они
похожи на мои. И снежные зимы, и весенние ручейки, и пронзительная боль от
срубленного тополя под окном... Спасибо автору за экскурс в историю Киева. Очень
люблю эту часть Печерска. Прогулки по Липкам всегда доставляют огромное
удовольствие! А уж какое счастье жить в \enquote{истории}.


\iusr{Zagoroui Valentina}
Спасибо за прекрасный рассказ. В моей жизни очень много связано с этими улицами.

\iusr{Татьяна Зубко Маркина}
Спасибо за прекрасные воспоминания

\iusr{Светлана Алексеееко}
Спасибо!

\iusr{Нина Алексеева}
Благодарю вас

\iusr{Раиса Карчевская}
Спасибо за прекрасные воспоминания детства

\iusr{Олена Іванівна}

Спасибо, вспомнила своё детство. Жила в Пассаже, училась в 117 школе. На
Лютеранской жили многие одноклассники, друзья из других классов. Брат ходил в
детсад, который Вы вспоминали). Давно было...


\iusr{Татьяна Артеменко}
Спасибо!

\iusr{Лена Рытова}

 @igg{fbicon.heart.with.ribbon} А что Вы можете рассказать о доме под номером 13, Лютеранская? Живу там более
8 лет, моя мечта сбылась @igg{fbicon.face.eyes.star} хотелось бы, как можно, побольше узнать о этом
доме....

\begin{itemize} % {
\iusr{Maksim Pestun}
\textbf{Лена Рытова} к сожалению, с этим домом у меня ничего не связано. В 15 жила одноклассница Ира Жукова в огромной квартире на втором этаже. Дочка архитектора.
\end{itemize} % }

\iusr{Александра Коломейко}
Так мы учились в одной школе 83!

\iusr{Александра Коломейко}

Нас перевели в 10 классе в 94 школу. Замечательный директор 83 школы Василий
Федорович Попов. А учителя - наша любимая классная Татьяна Иосифовна Линдерман,
Нина Михайловна, завуч, Раиса Аркадьевна, Мария Михайловна, Нина Михайловна

\begin{itemize} % {
\iusr{Виктория Лазния}
\textbf{Александра Коломейко} 

Татьяна Иосифовна Линдерман и у меня была классной руководительницей. Кстати,
на редкость преподавала пение, а вообще-то знакомила нас с основами истории
музыки, что было редкостью. К сожалению, имени отчества её не помню, но она
дружила с Татьяной Иосифовна.

А до этого нашей классной была очень любимая нами Карина Евсеевна Байтальская,
которая просто ушла из школы, но мы с ней долго общались потом.

\iusr{Виктория Лазния}
Прекрасно преподавал украинскую литературу Зинаида Павловна Бобышкина, дала очень многое. Интеллигентнейший человек. Любила её.

\iusr{Maksim Pestun}
\textbf{Александра Коломейко} вот это память у вас! Завидую!

\iusr{Maksim Pestun}
\textbf{Александра Коломейко} теперь понял. Я проучился в 83 только год, а Вы почти все 10 @igg{fbicon.smile} 

\iusr{Ольга Шадрина Черная}
\textbf{Александра Коломейко} По моему Попов Василий Тихонович

\iusr{Ira Yakunenko}
\textbf{Александра Коломейко} не Мария Михайловна, а Мария Алексеевна, а директор, конечно, Тихонович.
\end{itemize} % }

\iusr{Людмила Мозговая}

И я вчера была на Лютеранской... с экскурсоводом. Хотя выросла на Липках,
ходила в 51 школу и по всем тем улицам много раз гуляла. Обожаю тот район.


\iusr{Елена Реутова}

А в садик на Лютеранской мы, похоже, ходили в один и тот же.

\begin{itemize} % {
\iusr{Татьяна Захарова}
\textbf{Елена Реутова} а какой номер?

\iusr{Елена Реутова}
\textbf{Татьяна Захарова} 107 детский сад
\end{itemize} % }

\iusr{Natasha Levitskaya}

Какие чудесные, теплые воспоминания детства! Прекрасная Киевская история! Спасибо, Максим!

\iusr{Oksana Sha}

Сразу вспомнила, у меня подруга жила в этом доме, где детский сад был во дворе.
«Прекрасные воспоминания детства»

\iusr{Сергей Шевченко}

Окунулся в детство... Бабушка жила на Энгельса, в 21-м номере, как раз напротив
церкви... Любимые на все времена места... Спасибо за пост!

\begin{itemize} % {
\iusr{Лариса Артемчук}
\textbf{Сергей Шевченко}

А напротив дома вашей бабушки, в угловом доме около церкви жил мой одноклассник
Вова Кондратюк. И была у него старшая сестра Галя-красавица.

Она пела в ансамбле «Мрия»

\end{itemize} % }

\iusr{Мария Дидык}

Дуже ДЯКУЮ за ЧУДОВУ РОЗПОВІДЬ Здоровя Вам та радості радості на Вашому
життєвому шляху Дуже цікаво і змістовно написано Я давно люблю цю вулицю Часто
там буваю Ми з Мамою ходили інколи в Лютеранську Церкву коли її реставрували В
школу ім Лесі Українки і сьогодні везуть дітей з різних кінців Києва Ще раз
ДЯКУЮ Божої допомоги та Божого Благословіння Вам Амінь

\iusr{Ira Yakunenko}

Я жила в четвёртом номере. Зимние катания на санках от 83 школы до арки, снег в
фонарях, ручьи весной. Училась в 83 школе с 1-го по 9класс. Учителя, которых
помню всю жизнь, с благодарностью и любовью. Очень люблю бродить по этому
району, где знакома каждая подворотня, каждый уголок этого благословенного
места. Спасибо за эти воспоминания.

\iusr{Maksim Pestun}

Сегодня был в доме, где жил Паустовский, Тарновский и графиня О'Рурк

\ifcmt
  ig https://scontent-frx5-1.xx.fbcdn.net/v/t39.30808-6/271935292_3100664613537032_7091700215556156987_n.jpg?_nc_cat=111&ccb=1-5&_nc_sid=dbeb18&_nc_ohc=ORBZyncN6qQAX8IOTqY&_nc_ht=scontent-frx5-1.xx&oh=00_AT_kxXUQ_7zOEnfowJaEZpGb4Jd9KEQYAXERXlN_8JHPnQ&oe=61E9BEB4
  @width 0.3
\fi

\iusr{Виктор Киркевич}
Я жил на Лютеранской,6.

\iusr{Maksim Pestun}
красивый дом Рахиль Майкапар. Его еще называли фантазией кондитера

\iusr{Светлана Енина}
Побывала вместе с Ваии в детстве с ощущениями своего детства, стало тепло и грустно

\iusr{Sergyi Grabar}

Дякую, пане Максиме! Гарна розповідь про Київ нашого дитинства. Відразу згадуєш
і себе за подібними розвагами (маленьки кораблики з кори). Тільки адреса була
інша, але київська.

\iusr{Вікторія Юдіна}
Невероятный рассказ. Всё так дорого сердцу.

\iusr{Оксана Комар}

Дякую за цікаві факти, @igg{fbicon.face.smiling.eyes.smiling} я теж жила на Енгельса( Лютеранській) 27/29 з 1979-1999
і 117 школу закінчила. Але що це колишня вулиця Графська не знала...


\end{itemize} % }
