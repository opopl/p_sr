% vim: keymap=russian-jcukenwin
%%beginhead 
 
%%file letters.mariupol.katja_karnauh
%%parent letters
 
%%url 
 
%%author_id 
%%date 
 
%%tags 
%%title 
 
%%endhead 

Добрый день, Катерина! Христос Воскрес! Спасибо за Ваши посты о Мариуполе!
Особенно с писанками, очень красиво! Хотел бы тут немного написать о Мариуполе, попутно вспомнив Киев...
Возможно, Вас заинтересует то, чем я сейчас занимаюсь в связи с Мариуполем и его культурой, историей и т.д.
(Вы написали в своем недавнем посте, что Киев для Вас - это Сердце,
а Мариуполь - это Душа). Итак...

Меня зовут Иван, я киевлянин, программист и живу и работаю в Киеве. Много
сейчас читаю про Мариуполь, у меня есть также несколько книжек про Мариуполь.
К сожалению, до войны не успел побывать в Мариуполе, и вообще до войны вообще
как то не особо задумывался о  Мариуполе...  Но... я принимаю близко к сердцу,
все, что произошло с Мариуполем и с мариупольцами. Страшная, невообразимая
трагедия...  которая забрала близких, родных, детей, родителей, раскидала
мариупольцев, по всему миру... Тем не менее, я уверен, лучшие времена для
Мариуполя еще впереди, и отчаиваться точно не надо... Потому что Город - это не
только постройки, улицы, парки, сами по себе... Город - это также Духовная
Сущность, в том числе - это книги, память, фото, истории, люди...  И я как раз
из того Города, который как раз много-много раз разрушали, сжигали, разоряли,
здесь творились ужасные трагедии и бедствия, но Киев каждый раз восставал,
залечивал свои раны, и становился только краше! И я уверен, что Мариуполь тоже
обязательно восстанет из пепла, как птица-Феникс, и снова засияет, да,
прекрасная Жемчужина-Город на берегу Азовского Моря!

... Да, как я сказал выше, я программист. В свободное время от работы... я
также занимаюсь систематизацией разных публикаций о войне, у меня есть
отдельный проект Летописи Войны, который я веду - пока что сам - довольно уже
давно, собственно говоря, я сохраняю от забвения посты на фейсбуке, потом
раскидываю их по авторам, темам. Технология, которую я использую, называется
LaTeX (не знаю, имеете ли вы отношение к физике или математике, но это та
технология, которая повсеместно используется учеными в научном мире для записи
и публикации своих статей и исследований ). Так вот.  Я сохраняю в печатном
виде, то, что я читаю, считаю важным для сохранения...  Про Мариуполь я в
последнее время тоже сохраняю, и довольно много. И про довоенный мирный
Мариуполь, и для ужасы войны. У меня есть телеграм канал
https://t.me/kyiv_fortress_1, там уже довольно много выложено, так же как и на
этом фейсбук - аккаунте, также я постепенно выкладываю собранные  материалы - с
уважением к авторам - я всегда указываю, кто и когда что создал - на
https://archive.org - это интернет сайт, посвященный сохранению всего в
Интернете, ссылка на меня там https://archive.org/details/@kyiv_chronicler.
Зачем все это делается... знаете... Мариуполь физически то убили... но
Мариуполь не умер. Мариуполь остался в Духовном Пространстве, и это очень очень
важно... Но если не записывать... если этим не заниматься... есть риск, что
Мариуполь умрет духовно... и это будет уже навсегда. А духовная смерть - это...
еще страшнее, чем смерть физическая... Поэтому я этим и занимаюсь, хотя я и
киевлянин, вообще то говоря. Хочется, знаете, тоже приехать в наш Украинский
Возрожденный Мариуполь, как гость, в свое время ) 

... Совсем недавно я составил несколько сборников по Мариуполю, все они
доступны для просматривания и скачивания в виде pdf-файлов, надеюсь Вам будет
интересно (1) по группе Мариуполь довоенный, 72 страницы, выложена здесь у меня
на страничке в альбомах, ссылки там указаны на загрузки файлов (2) в основном
по культурной жизни Мариуполя, но есть также и посты с прерасными
фотографическими работами, вот тут -
https://acrobat.adobe.com/id/urn:aaid:sc:EU:4ecce413-b27c-4216-9cd7-1fce3a3bb09c
и также тут
https://mega.nz/file/V7JUga6T#dK6wXslj5ZI4rNaTcUSG7R1M93QH3oa54O5c5AK159Q как
оно все выглядит, смотрите скрины ниже (файлы pdf большие !!!, нужно подождать,
пока загрузятся полностью для просмотра и скачивания - там есть оглавление и
возможность передвигаться по ссылкам внутри, также индексация по авторам) ну и
сейчас приготовил небольшой сборничек, так сказать, в честь Пасхи, самые разные
фоточки, и Мариуполя, и Украины (выложил у себя на страничке).

С уважением, Иван.
