% vim: keymap=russian-jcukenwin
%%beginhead 
 
%%file poetry.unsorted
%%parent poetry
 
%%url 
 
%%author 
%%author_id 
%%author_url 
 
%%tags 
%%title 
 
%%endhead 

%\ii{15_10_2021.fb.kazakov_igor.1.kotjata_bajden_rasskaz}

%https://strana.ua/opinions/328704-ukrainskoe-obshchestvo-hluboko-bolno-i-nuzhdaetsja-vo-vrachevanii.html
Этот поезд в огне,
И нам не на что больше жать.
Этот поезд в огне,
И нам некуда больше бежать.

Эта земля была нашей,
Пока мы не увязли в борьбе,
Она умрет, если будет ничьей.
Пора вернуть эту землю себе.


Людей неинтересных в мире нет.
Их судьбы — как истории планет.
У каждой все особое, свое,
и нет планет, похожих на нее.
А если кто-то незаметно жил
и с этой незаметностью дружил,
он интересен был среди людей
самой неинтересностью своей.
У каждого — свой тайный личный мир.
Есть в мире этом самый лучший миг.
Есть в мире этом самый страшный час,
но это все неведомо для нас.
И если умирает человек,
с ним умирает первый его снег,
и первый поцелуй, и первый бой...
Все это забирает он с собой.
Да, остаются книги и мосты,
машины и художников холсты,
да, многому остаться суждено,
но что-то ведь уходит все равно!
Таков закон безжалостной игры.
Не люди умирают, а миры.
Людей мы помним, грешных и земных.
А что мы знали, в сущности, о них?
Что знаем мы про братьев, про друзей,
что знаем о единственной своей?
И про отца родного своего
мы, зная все, не знаем ничего.
Уходят люди... Их не возвратить.
Их тайные миры не возродить.
И каждый раз мне хочется опять
от этой невозвратности кричать.

%https://www.youtube.com/watch?v=Y_J4ik6ES5Y
Почему останусь, или Письмо пакующим чужой чемодан

В последние годы у нас появилось немало людей, которые в ответ на любое
несогласие с ними, рекомендуют покинуть страну. Мол, если вам не нравится
здесь, уезжайте. Хотя, как правило, именно тех, кто рекомендует чемодан и
вокзал, и стоило бы отправить куда-нибудь для отдыха. В том смысле, чтоб
Украина от них отдохнула.

Мой путь по жизни неудобен – 
Так в мире решено ином.
Вручён мне дар писать о злобном,
Нелепом, пошлом и смешном.

Писать о происках лукавых,
Смотреть, чтоб враг не искусил…
И часто это не по нраву
Носителям бесовских сил.

И мне шипят лжепатриоты,
Те, что паскудят отчий край:
– Нам не нужна твоя работа!
Давай-ка, лучше уезжай.

Ну что ж, у нас такие беды,
Так тяжко сердцу и уму,
Глядеть невмочь. Но не уеду.
Скажу, панове, почему.

Не стану говорить, что землю,
Где жить дано мне, я люблю.
Любовь ваш разум не приемлет,
Кто любит, тот «слуга Кремлю».

Не вспомню то, что край наш милый
Отец прошёл в военный час,
И что здесь матери могила – 
Такие речи не для вас.

Но по другим ещё причинам
Я не уеду никогда.
Кому оставлю Украину?
Вам? Не смешите, господа.

Ведь вы страну, под пенье гимна,
Всю раскрадёте до основ,
И пустите гостеприимно
В её поля – её врагов.

Да и, живя вдали, негоже
Учить народ свой свысока.
Так стал бы я на вас похожим,
Как «патриот издалека».

С полей Европы слать приветы,
На них батрачить – не по мне.
А вот, и главный мой секрет вам – 
Зачем останусь я в стране.

Вдруг все отправятся в дорогу
Туда, где сытно и тепло?
Кто станет здесь молиться Богу?
За дом, за город, за село.

Ради кого нам явит милость
Наш Бог Христос? По чьим слезам?
Нет, не уеду. Огорчились?
Сочувствую не сильно вам.

Когда же, поздно или рано,
На Суд придём в конце пути,
Я из родной земли восстану,
Чтобы сказать: «Господь, прости!»

2021, Киев
ЯнТаксюр


Наш с вами мир построен на Любви, 
И к Жизни нужно относиться без печали, 
И пусть не всё пока ещё сбылось, 
Лишь бы фальшиво ноты не звучали, 
Пусть заиграет для души оркестр, 
Давая право верить и любить. 
Из всех подарков на земле и всех торжеств, 
Подарок самый главный - просто жить.


