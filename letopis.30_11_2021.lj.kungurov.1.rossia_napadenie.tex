% vim: keymap=russian-jcukenwin
%%beginhead 
 
%%file 30_11_2021.lj.kungurov.1.rossia_napadenie
%%parent 30_11_2021
 
%%url https://kungurov.livejournal.com/300785.html
 
%%author_id lj.kungurov
%%date 
 
%%tags napadenie,rossia,ukraina,ugroza
%%title Война с Украиной: ЗА/ПРОТИВ
 
%%endhead 
\subsection{Война с Украиной: ЗА/ПРОТИВ}
\label{sec:30_11_2021.lj.kungurov.1.rossia_napadenie}

\Purl{https://kungurov.livejournal.com/300785.html}
\ifcmt
 author_begin
   author_id lj.kungurov
 author_end
\fi

Говорят, у аквариумных рыбок события удерживаются в памяти 2 секунды. Не
измерял, не знаю. Но вот у людей с памятью действительно неважно. Есть память
базовая: человек помнит, как его зовут, адрес, где живет, буквы алфавита
врезаются в мозг на всю жизнь, приобретенные профессиональные навыки достаточно
устойчивы. Даже совершенно бесполезную информацию, как например, тексты
попсовых песен люди помнят назубок десятилетиями.

\ii{30_11_2021.lj.kungurov.1.rossia_napadenie.pic.1}

Однако память об окружающем мире, память, отвечающая за социальную адаптацию,
обладает, как бы это сказать – пластичностью. Проще говоря, человек в данном
случае старается помнить не реальные факты об окружающем мире, а то, что для
него более комфортно в настоящий момент. Комфортно же иметь представления «как
у всех», быть с большинством. Именно это заставляет мозг, во-первых,
воспринимать информацию фрагментарно, игнорируя ту, что не вписывается в
шаблон; во-вторых, человек должен быстро забывать ту информацию, которая
утрачивает «актуальность». Не важно, насколько она правдива, главное, что
сейчас принято считать «правдой» нечто иное.

Из этого следует, что психика среднего обывателя ориентирована на то, чтобы
воспринимать социальную реальность с чистого листа каждый день. Не важно, что
было вчера, тем более – полгода назад. Реальность – это то, о чем сказали по
зомбоящику с утра, а эмоционально воспринимать это следует так, как велят
соцсети, ибо это сегодня индикатор, показывающий, «как все думают». Только
таким образом индивид может пребывать в гармонии с миром.

Кто-то удивится: как можно сохранять душевный комфорт, некритически принимая
очень некомфортную информацию, например от коронабесов: мол, мы все умрем,
спасения от вируса нет, надо прятаться в бункере и т. д. На самом деле все
легко объяснимо: для овер 90\% людей крайне некомфортна ситуация, в которой им
приходится нести ответственность хотя бы за себя. Если же он верует, что мудрое
начальство заботится о нем, ведь им там сверху виднее, если он видит, что стадо
тревожно мычит и мечется из стороны в сторону под кнутом пастуха, то гораздо
комфортнее примкнуть к стаду и мычать в унисон. Потому что в противном случае
придется признать, что вокруг живут совершенно ебнутые зомби, а «добрые
пастухи» – маньяки-фашисты. Каково будет жить с таким осознанием реальности?
Не-е-е, уж лучше быть вместе с испуганным стадом, чем противопоставлять себя
ему. Лучше – не думать, не пытаться анализировать реальность, не помнить
ничего, что заставляет сомневаться в спускаемой сверху картинке.

В результате общество приобретает весьма полезный коллективный социальный навык
– некритически воспринимать любую мейнстримную информацию. То есть отношение к
ней может кардинально отличаться не в зависимости от убедительности аргументов
(их же надо анализировать!), а в зависимости от источника, втюхивающего готовые
выводы.

Вот я вкратце объяснил механизм популярности теории заговора о том, что «Путин
вот-вот нападет на Украину». Стадо охотно верит в это просто потому, что
привыкло в это верить. Если говорить об украинском стаде, то это еще и приятно,
ибо повышает самооценку: мол, вот какие мы невъе…енные, что исчадие зла люто
нас ненавидит, боится и в бессильной злобе лязгает зубами у наших границ,
хочет, но не решается напасть, потому что мы сильны, как никогда. Слава нам!

Украинским властям грех не использовать тягу стада к самообману и самогипнозу.
Раздутая внешняя угроза – проверенный веками способ отвлечь массовку от
внутренних проблем и своей роли в их создании. А разве европейским властям не
нужно то же самое – сместить фокус внимания обывателя от своего феерического
обсера в борьбе с ковидлой в сторону внешней угрозы? Заокеанскому старперу Джо
явно так же не помешает повод для насупливания бровей и зачитывания по бумажке
решительных заявлений. Америка же типа смотрящий за миром во всем мире. Ну как
бы… Самое смешное, что и для Пыньки  милитари-истерика – подарок судьбы. Он же
известный торговец страхом. Тут же страх даже создавать не надо, он уже есть,
надувай щеки, делай безумные глаза, изображай припадочного психа с ядерной
дубинкой, а потом диктуй условия для деэскалации.

Насколько военный конфликт в формате большой войны с открытым участием в ней
Орды реален? Для того, чтобы доказать его абсолютную невозможность, мне хватит
двух абзацев, а от читателя – совершенный минимум аналитических способностей.
Ведь проанализировать предстоит всего один факт, а именно: численность
сосредоточенных на границе с Украиной путинских зольдатиков – аж целых 92
тысячи, как хором докладывают ЦРУ и ГУР. Не мало ли против 204 тысяч украинских
вояк (всего численность ВСУ – 255 тысяч, включая гражданский персонал)? Для
сравнения: в феврале 1994 г. для решительного штурма Грозного Москва
сосредоточила группировку федеральных сил в 75 тысяч штыков, которой
противостояло порядка 15-17 тысяч чеченских… даже не солдат, а полупартизан, не
обладающих ни единым командованием, ни авиацией, почти не имеющих тяжелого
вооружения.

США в 2003 г. для атаки против Ирака, имеющего армию хоть и больше (370 тыс.
чел.), но куда слабее, чем нынешняя украинская, сосредоточили группировку более
чем в 300 тысяч солдат коалиции. При этом стоит иметь в виду, что янки
традиционно делают ставку не на численность живой силы, а на подавляющее
техническое превосходство. Военнослужащие в ходе военных действий менялись на
основе регулярной ротации (то есть реально задействовано сил было куда больше,
чем в моменте). Плюс к этому на территории Ирака силы вторжения опирались на
свою обширную «пятую колонну», прежде всего, шиитов и курдов. Ближневосточный
ТВД практически идеален для войны с туземцами – погода почти всегда летная,
местность – пустынно-равнинная, что делало задачу дистанционного выбивания
саддамовских сил такой же простой, как на полигоне.

Не стоит забывать и о том, что Ирак перед началом войны находился в полной
международной изоляции. Для Кремля же ситуация обратная: именно РФ находится в
изоляции, у Киева же полно друзей. Друзей ненадежных, не будущих за Украину
воевать, и вообще, воспринимающих ее, скорее, в качестве инструмента, нежели
партнера. Но это лучше, чем ничего и однозначно дает плюс в карму.

Однако главное даже не в этом. Весь этот озабоченный пердеж про вторжение,
якобы назначенное на январь-февраль (пиздец, менее удачного времени года сложно
придумать!) обходит стороной такой вопрос: а Путину это вторжение ЗАЧЕМ? Чтоб
захватить Украину или ее часть, пробить сухопутный коридор в Крым, расширить
границы Луганды до Дуная, добиться сертификации СП-2? Какая конечная, а не
промежуточная цель всей этой гипотетической войны? Вот эти все горы трупов,
рухнувшая вписту экономика, санкции, введенные по самые гланды, статус мирового
изгоя – ради чего? Должен же быть какой-то гипотетический профит.

Ничего сколь-нибудь разумного, кроме решения проблемы выборов-2024 не
просматривается. Но разве нужна война, чтобы выиграть выборы? Для этого, как
показывает практика, достаточно памфиловского блокчейна. А с несколькими
тысячами несогласных разберется ОМОН. Если вдруг опять кремлевский карлик
возжелает предстать в амплуа собирателя земель русских, так без всякой войны
можно будет присоединить минское генерал-губернаторство. Аборигены на первых
порах даже будут рады аншлюсу, поскольку меньшее зло освободит их от большего.

И вообще, поскольку у обывателя отсутствует память, я готов напомнить: в
среднем на Незалэжной полтора раза в год случается истерика по поводу скорого
неминуемого вторжения путинских орд, в масс-медиа вбрасываются карта с синими
стрелками, нацеленными на Одессу и Львов, назначается дата вторжения и… ничего
не происходит. Смешно, но начиная с марта 2014-го (а вверху я поместил
картинку, датированную этим временем) украинцы столько раз впустую вопили о
кацапской интервенции, что сами перестали в нее верить. Поэтому летом в ходе
активных боевых действий на Донбассе «мудрые» киевские атаманы решили наступать
вдоль границы, в результате чего наступающая группировка ВСУ была ударами с
территории РФ рассечена на части и ликвидирована в нескольких котлах, самым
крупным из которых оказался мариинский.

Так что, братцы-украинцы, знайте: если власть начинает пугать вас скорым
наступлением путин-ваффе, то это туфта на все 100\%. Значит под шумок будут вас
свои же немножко грабить. А чтоб вы не сильно расстраивались, в марте объявят,
что Путтин в очередной раз зассал, клоун спас Украину, всем надо гордиться
собой и бурно радоваться.
