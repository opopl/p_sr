% vim: keymap=russian-jcukenwin
%%beginhead 
 
%%file 26_12_2021.fb.fb_group.story_kiev_ua.5.staraja_shuljavka.cmt
%%parent 26_12_2021.fb.fb_group.story_kiev_ua.5.staraja_shuljavka
 
%%url 
 
%%author_id 
%%date 
 
%%tags 
%%title 
 
%%endhead 
\zzSecCmt

\begin{itemize} % {
\iusr{Natasha Levitskaya}

Спасибо, Георгий! Как всегда, очень интересно о ваших предках! Все ваши
публикации в группе, это такая далёкая история вашей семьи, документы, фото и
главное, ваши поиски и знание этой истории.

Просто респект, как сейчас говорят и уважение вам! Спасибо! @igg{fbicon.hands.applause.yellow} 

\begin{itemize} % {
\iusr{Георгий Майоренко}
\textbf{Natasha Levitskaya} Благодарю. Некоторые фото обнаружились совсем недавно. Лежали в закромах и мама вдруг их нашла! А в них - эпоха!
\end{itemize} % }

\iusr{Людмила Ива}

Огромное спасибо, Георгий! Очень интересные воспоминания! Легко и приятно
читать! @igg{fbicon.rose} 

\begin{itemize} % {
\iusr{Георгий Майоренко}
\textbf{Людмила Ива} Спасибо, с наступающим! Удачи и добра!

\iusr{Людмила Ива}
\textbf{Георгий Майоренко} взаимно! Здоровья и благополучия Вам!)
\end{itemize} % }

\iusr{Надежда Кулиш}

\ifcmt
  ig https://i2.paste.pics/ebe47a0be53cf220cc6bd46c44e9c4c8.png
  @width 0.2
\fi

\begin{itemize} % {
\iusr{Георгий Майоренко}
\textbf{Надежда Кулиш}

\ifcmt
  ig https://scontent-mxp2-1.xx.fbcdn.net/v/t39.1997-6/s168x128/17527812_1652591011433966_4391041969200037888_n.png?_nc_cat=1&ccb=1-5&_nc_sid=ac3552&_nc_ohc=5c8vh81yn3oAX90m1uL&_nc_ht=scontent-mxp2-1.xx&oh=00_AT846U1f9Qe7omF6o7C2gmjQPg1WJPzH_rpqKDPUws5xOA&oe=61DA0558
  @width 0.1
\fi

\begin{itemize} % {
\iusr{Надежда Кулиш}
\textbf{Георгий Майоренко} Училась в школе на Шулявке

\iusr{Георгий Майоренко}
\textbf{Надежда Кулиш} Это в какой? Отец мой учился в той, что рядом с кинотеатром Довженко, а я - до 5 класса в 41.

\iusr{Надежда Кулиш}
\textbf{Георгий Майоренко} В 74, на Смоленской ул. возле КПИ

\iusr{Lyubov Pakholchenko}
\textbf{Надежда Кулиш} Моя мама закончила 74 школу перед самой войной. У меня сохранились ее табеля за все классы. После освобождения Киева в этой школе был госпиталь, потом опять школа.
\end{itemize} % }

\end{itemize} % }

\iusr{Olena Kharkevich}
Очень интересно! Спасибо!

\iusr{Георгий Майоренко}
\textbf{Olena Kharkevich} Спасибо за добрые слова! Успехов и позитива!

\iusr{Iryna Naidonova}
Благодарю!

\iusr{Георгий Майоренко}
\textbf{Iryna Naidonova} Спасибо! Удачи и достатка в новом году!

\iusr{Виктор Задворнов}

В течение 1990-2005-х \enquote{на базе} международного рекламного агентства \enquote{Посредник}
выходила районная газетв \enquote{Шулявка} с еженедельной ТВ-программой. Всегда с
удовольствием редактор Евгений Хорунжий публиковал на страницах газеты
исторические очерки о правобережном Киеве. А уж о Соломенке, Шулявке, Галаганах
- особенно.

\begin{itemize} % {
\iusr{Георгий Майоренко}
\textbf{Виктор Задворнов} Помню эту газету! Это ещё во времена \enquote{Авизо}, \enquote{Экспресс-объявы}. Доинтернетная эпоха!

\iusr{Виктор Задворнов}
\textbf{Георгий Майоренко} газета \enquote{Шулявка} поддерживалась отчасти администрацией Жовтневого района. Очень по душе пришлась из-за дешевизны пенсионерам и жителям Правобережья.

\iusr{Георгий Майоренко}
\textbf{Виктор Задворнов}

\ifcmt
  ig https://i2.paste.pics/f6789ca517f4a844329db36ce7368ef1.png
  @width 0.2
\fi

\iusr{Виктор Задворнов}
\textbf{Георгий Майоренко} И Вам спасибо за воспоминания.

\end{itemize} % }

\iusr{Svitlana Plieshch}

Великолепная история, познавательная, интересная и глубокая. Столько истории в
коротком рассказе. Училась в КПИ, район знакомый, многого не знала. Огромная
благодарность


\iusr{Георгий Майоренко}
\textbf{Svitlana Plieshch} Благодарю за добрые слова. Удачи и добра!

\iusr{Раиса Карчевская}
Большое спасибо, Георгий!
Как всегда очень интересно

\begin{itemize} % {
\iusr{Георгий Майоренко}
\textbf{Раиса Карчевская} Благодарю! Удачи и добра в новом году!

\iusr{Раиса Карчевская}
\textbf{Георгий Майоренко} Спасибо большое - взаимно
\end{itemize} % }

\iusr{Георгий Майоренко}
Благодарю! С наступающим! Удачи в новом году!

\iusr{Раиса Карчевская}
Взаимно. Жду новых публикаций

\begin{itemize} % {
\iusr{Георгий Майоренко}
\textbf{Раиса Карчевская} Спасибо огромное за добрые слова! Новые публикации в новом году!

\iusr{Раиса Карчевская}
\textbf{Георгий Майоренко}

\ifcmt
  ig https://i2.paste.pics/555ed7238379c69093d9f56c9c201872.png
  @width 0.2
\fi

\end{itemize} % }

\iusr{Светлана Мищенко}
Очень интересно, ярко. описывает район, где жили ваши родные и выросли вы.
Эти места мне тоже близки, на территории КПИ вырос мой отец, окончил КПИ.
Дальнейших вам успехов в творчестве

\begin{itemize} % {
\iusr{Георгий Майоренко}
\textbf{Светлана Мищенко} 

Ну так от Шулявки до Львовской (ваши родные места) было рукой подать! Недавно
обнаружил, что бабушку мою крестили в Златоустовской церкви на Евбазе! Нашел
запись в метрической книге. Киев тогда был не такой огромный, как сейчас.

\iusr{Ирина Панченко}
\textbf{Георгий Майоренко} 

Моя бабушка училась в техникуме (или не знаю как училище тогда называлось) на
Львовской площади, так рассказывала, что ходила пешком туда и обратно. (с Шулявки)

\iusr{Георгий Майоренко}
\textbf{Ирина Панченко} 

Приветствую, Ирина! А ваши предки, если я не ошибаюсь, жили до революции на
Борщаговской 27? Янковский Людвиг Францевич?

\end{itemize} % }

\iusr{Татьяна Соловьева}

Помните, у Ильфа и Петрова \enquote{Одноэтажная Америка}? Ваша \enquote{Одноэтажная Шулявка} -
не менее увлекательна!

\begin{itemize} % {
\iusr{Георгий Майоренко}
\textbf{Татьяна Соловьева} 

Ильф и Петров это - супер! У них и фельетоны шикарные. Когда-то попалось их
подписное издание, читал с упоением! Чего стоила сатирическая стенгазета из
банно-прачечного комбината \enquote{Мочалкой по черепу!}. А отец мой так
зачитал \enquote{12 стульев} и \enquote{Золотого теленка}, что книга
превратилась в ветошь! Все родственники хохотали.


\iusr{Галина Компанченко}
\textbf{Георгий Майоренко} Обажала это произведение и я.. А вот \enquote{Однооэтажная Америка}, прочлась мною единажды..

\iusr{Георгий Майоренко}
\textbf{Галина Компанченко} Надо будет перечитать. Классику можно читать бесконечно.
\end{itemize} % }

\iusr{Константин Тюрменко}
Жил на Борщаговской 159. Теперь этого дома нет. Его мой отец построил

\iusr{Caty Si}

Спасибо большое за такие интересные исторические факты моей родной Шулявки!
Часть из вашего рассказа уже мне знакома от моего дедушки, который всю свою
жизнь (примерно с конца 50х) проработал на заводе Петровского и получил от него
квартиру в своих же \enquote{заводских} домах. С середины 70х и по сей день мы с
родителями являемся коренными жителями Шулявки @igg{fbicon.face.smiling.eyes.smiling}.

\begin{itemize} % {
\iusr{Георгий Майоренко}
\textbf{Caty Si} Благодарю за добрые слова! Удачи и здоровья в новом году!

\iusr{Caty Si}
\textbf{Георгий Майоренко} Вам здоровья и благополучия в Новом году и, конечно же, вдохновения на новые интереснейшие истории! @igg{fbicon.face.smiling.eyes.smiling} 
\end{itemize} % }

\iusr{Арт Юрковская}

Очень интересно! И про революцию интересно. А то сейчас получается, что ее
сделала высадка с Марса. Меня всегда интересовал вопрос куда исчезли потомки
революционеров, чекистов, петлюровцев и немецких полицаев? В основном потомки
Прони Прокоповны, князей и раскулаченных делятся воспоминаниями. Первые
проклинают большевиков, которые лишили их поместий и балов, а вторые за то, что
увели единственную корову (вариант - лошадь) в колхоз. А где же все остальные?

\begin{itemize} % {
\iusr{Георгий Майоренко}
\textbf{Арт Юрковская} 

Увы, сейчас не приветствуется объективный взгляд на исторические события. Как в анекдоте.

Прапорщик объявляет солдатам :

-Сейчас все пойдут грузить люмимний!

Один из строя:

- Не люмимний, а алюминий?

-Сказано - люмимний, значит люмимний!

А самый умный пойдет грузить г...но.

\iusr{Ірина Кравець}
\textbf{Арт Юрковская} 

в одном из Ваших откровений Вы сообщили, что предок воевал у Петлюры. Так что, Вы
же пишите о трофейных вещах, как о нечто драгоценном, а по сути о мародёрах. Не
скажите... Разные воспоминания у киевлян. И почему не пишут? Не дано - это
первое Унаследовать надо было. 2) - потомки непишущих ещё и в школе плохо учились.


\iusr{Эдуард Джафаров}
\textbf{Арт Юрковская} , да и после революции если кто и сидел, то только по политической статье. За пошлую уголовщину деды нибожежмой!!)
\end{itemize} % }


\iusr{Людмила Козак}
СПАСИБО!

\iusr{Георгий Майоренко}
\textbf{Людмила Козак}

\ifcmt
  ig https://scontent-mxp2-1.xx.fbcdn.net/v/t39.1997-6/s168x128/17527812_1652591011433966_4391041969200037888_n.png?_nc_cat=1&ccb=1-5&_nc_sid=ac3552&_nc_ohc=5c8vh81yn3oAX90m1uL&_nc_ht=scontent-mxp2-1.xx&oh=00_AT846U1f9Qe7omF6o7C2gmjQPg1WJPzH_rpqKDPUws5xOA&oe=61DA0558
  @width 0.1
\fi


\iusr{Нина Гордийчук}
Спасибо за интересный рассказ.

\iusr{Георгий Майоренко}
\textbf{Нина Гордийчук} Благодарю за добрые слова! С наступающим!

\iusr{Сергій Савелій}

Цікаво. Випало і мені доторкнутися, щоправда, до сучасної Шулявки. У середині
80-х жив у малосімейці на Довженка, відразу за будинком з тилу заходив гуляти в
Пушкінський парк з донькою, а старший син ходив до школи на Ванди Василевської
( нині Богдана Гаврилишина), відразу за кінотеатром Олександра Довженка (
нині,на жаль, колишнього)...

\begin{itemize} % {
\iusr{Георгий Майоренко}
\textbf{Сергій Савелій} Рiднi мiсця! Успiхiв в новому роцi!

\iusr{Сергій Савелій}
\textbf{Георгий Майоренко} Навзаєм!

\iusr{Юрий Панчук}
\textbf{Сергій Савелій} 102а школа на вулиці Шулявській, яка перетинає Богдана Гаврилишина, я теж там навчався ))

\iusr{Сергій Савелій}
\textbf{Yury Panchuk} Запам'ятав класну керівничку сина, така вредненька була, в літах, чи Ганна Лук'янівна, чи Галина Лук'янівна...
\end{itemize} % }

\iusr{Владимир Дубровский}
Я уверен, что в 7 или 8 веке жил человек по прозвищу Шелява, что посадил лес.

\iusr{Георгий Майоренко}
\textbf{Владимир Дубровский} Хорошая версия про Шеляву! А он упоминается в каких-то летописях?

\iusr{Oksana Perederiy}
Мой дедушка - Чепурковский Антон Степанович, был настоятелем Марие-Магдалиновской церкви.

\begin{itemize} % {
\iusr{Георгий Майоренко}
\textbf{Oksana Perederiy} Как интересно! А дедушкин архив не сохранился?

\iusr{Maria Minyaeva}
\textbf{Oksana Perederiy} с моей мамой училась Ира Чепурковская в 74 школе

\iusr{Oksana Perederiy}
\textbf{Maria Minyaeva} Ира Чепурковская - моя сестра. Ей 85 лет. А фамилия мамы?

\iusr{Maria Minyaeva}
\textbf{Oksana Perederiy} Галина Поперенко, ей тоже 85  @igg{fbicon.smile}  Передавайте Ирине привет от моей мамы!

\iusr{Maria Minyaeva}
\textbf{Oksana Perederiy} Если хотите, можем списаться в личке подробнее
\end{itemize} % }

\iusr{Lola Shpilskaya}

Шулявский двор. Фото довоенное, приблизительно 1936 года или чуть позже.
Девочка в шапочке на руках (возможно родственница) это моя мама

\ifcmt
  ig https://scontent-mxp2-1.xx.fbcdn.net/v/t39.30808-6/270016191_4291968530909806_8647595297430959391_n.jpg?_nc_cat=103&ccb=1-5&_nc_sid=dbeb18&_nc_ohc=8k83yG2HL7MAX_D6iRl&_nc_ht=scontent-mxp2-1.xx&oh=00_AT8W4HfF_PjWiMfPiDIMWfrOrbED4lRBETlsx3XRy0tJ7Q&oe=61DA8B46
  @width 0.3
\fi

\begin{itemize} % {
\iusr{Георгий Майоренко}
\textbf{Lola Shpilskaya} Как здорово, что фото сохранилось! Такие довоенные фотографии на вес золота!

\iusr{Lola Shpilskaya}
\textbf{Георгий Майоренко} 

у меня есть фотографии прабабушки с детьми, прадедушки с одним из сыновей. А
все собираюсь рассказать о семье, но что-то не складывается. А Вы очень красиво
рассказали о Шулявке


\iusr{Георгий Майоренко}
\textbf{Lola Shpilskaya} 

Как здорово, что фотографии сохранились! Если опубликуете, будет замечательно.

\iusr{Tanja Macsimenco}

Шулявка - мой родной район. Рядом со 142 школой, где я училась, сохранился, как мы
называли, \enquote{поповский дом}. Саму церковь Марии\hyp Магдалины разрушили и на ее месте
построили райком. Вокруг него сохранился поповский сад с грушами, орехами,
кустами смородины. В детстве мы туда лазили через забор. А в поповском доме
сделали Дом пионеров с разными кружками. Я ходила в кукольный кружок. Мы ставили
спектакль \enquote{Волк и семеро козлят}, сами учились делать куклы. Там еще был
фотокружок, кружок кройки и шитья, был струнный оркестр. Прекрасно помню
Борщаговка. На ней и прилегающих переулках жили мои одноклассники. Моя тетя жила
на улице Светлой в частном доме. На Шулявку замечательные парки: Пушкинский (как
много там было развлечений. Летний кинотеатр, летняя эстрада, качели, карусели,
шахматный клуб, бильярдная, танцплощадка) Политехнический парк (какие там
проходили выпускные вечера Политехнического института!), Зоопарк. А кинотеатр
им. Довженка. Там показывали самые новые фильмы - наши и зарубежные. И первые
станции метро тоже начинали отсюда. Политех, Большевик. Да забыла сказать - в мой
родной Серый дом напротив Метро. Из окна смотрела как его строили. Много можно
вспоминать.

\iusr{Maria Minyaeva}
\textbf{Tanja Macsimenco} Моя мама до сих пор называет \enquote{серый дом}

\end{itemize} % }

\iusr{Юлия Шеремет}
Дуже цікаво написано, і з повагою і любов'ю!

\iusr{Георгий Майоренко}
\textbf{Юлия Шеремет} Спасибо, Юлия! С наступающим!

\iusr{Владимир Новицкий}
Спасибо, Вы очень интересно излагает события, любим Вас читать!!

\iusr{Георгий Майоренко}
\textbf{Владимир Новицкий} Благодарю, уважаемый Владимир. Ваше внимание особенно дорого! Здоровья и благополучия в новом году!

\iusr{Светлана Здор}

Боже мой, как интересно. Тем более, так сложилось, что мы живем с некоторых
пор (10лет) на Шулявке, возле Индустриального. И так приятно прочитать живую
историю этого района Киева. Восхищаюсь такими талантливым и преданным людьми.
Спасибо.

\iusr{Георгий Майоренко}
\textbf{Светлана Здор} 

Благодарю. К огромному сожалению, историей Шулявки никто вплотную не занимался.
И там много белых пятен и загадок.


\iusr{Margarita Kaminsky}

До 1988 года работала на Шулявке в 102 школе (за кинотеатром Довженко).

\begin{itemize} % {
\iusr{Георгий Майоренко}
\textbf{Margarita Kaminsky} В этой школе отец учился. А я - в 41.

\iusr{Юрий Панчук}
\textbf{Margarita Kaminsky} Я Вас помню, русский язык и литература, а еще вы работали библиотекарем. Всего вам наилучшего, здоровья и благополучия ))

\iusr{Margarita Kaminsky}
\textbf{Yury Panchuk}
Спасибо, Юра! Всё правильно. И Вам всех благ!
\end{itemize} % }

\iusr{Роза Миколаївна Кириченко}

А мой дедушка Соловей Никита Ефимович до революции тоже служил в этой церкви.
Возможно они служили одноаременно

\begin{itemize} % {
\iusr{Георгий Майоренко}
\textbf{Роза Миколаївна Кириченко} 

О, это известная фамилия на Шулявке! Как начнёшь листать справочники и
метрические книги, часто встречается. Многих детей крестил \enquote{на районе}!

\iusr{Роза Миколаївна Кириченко}
\textbf{Георгий Майоренко} Спасибо. Но я пока ничего не знаю о нем.

\iusr{Георгий Майоренко}
\textbf{Роза Миколаївна Кириченко} Это поправимо.

\iusr{Роза Миколаївна Кириченко}
\textbf{Георгий Майоренко} надеюсь
\end{itemize} % }

\iusr{Роза Миколаївна Кириченко}
Спасибо за очень интересный рассказ.

\begin{itemize} % {
\iusr{Георгий Майоренко}
\textbf{Роза Миколаївна Кириченко} Удачи и благополучия в новом году! Здорово, что ваши жили рядом с нашими в давние времена! И были прихожанинами одной церкви.

\iusr{Роза Миколаївна Кириченко}
\textbf{Георгий Майоренко} возможно и знали друг друга. И Вам всего наилучшего в новом году.

\iusr{Георгий Майоренко}
\textbf{Роза Миколаївна Кириченко} Где-то виделись, так это точно!
\end{itemize} % }

\iusr{Татьяна Косолапова}

Спасибо... воспоминания о Шулявке. Я тоже в 60-десятых годах жила там, в доме
специалистов...

\begin{itemize} % {
\iusr{Георгий Майоренко}
\textbf{Татьяна Косолапова} 

Колоритный был район. Я эту одноэтажную Шулявку помню смутно.
Через пару лет после моего рождения ее снесли и построили новые
дома. А отец мой помнил, когда в речке Лыбеди рыбу ловили, и на
плотах по ней плавали.

\end{itemize} % }

\iusr{Мария Маслова}

Спасибо, очень интересно и всесторонне!

Моя тетя много лет прожила жила на Выборгской, это недалеко от описываемых Вами
мест. И Татьяну Гурьеву помню по порталу «Что хочет автор» и его отделению
«Пишущая Украина».

\iusr{Георгий Майоренко}
\textbf{Мария Маслова} Благодарю! Бывал на Выборгской!

\iusr{Lyubov Pakholchenko}

Мои родственники по линии бабушки - Жаворонковы, жили на ул. Шулявской.
Песчаная, Борщаговская, Полевая, Фабричная - все названия улиц помню с
детства. По рассказам бабушки в Кадетскую рощу ходили на майовку с самоваром. У
бабушки было много подруг на Шулявке. Пушкинский парк, зоопарк, больница
Калинина, парк Политеха - Шулявка родина моя.

\begin{itemize} % {
\iusr{Георгий Майоренко}
\textbf{Lyubov Pakholchenko} Замечательно! Наши люди! Здоровья и удачи в новом году!

\iusr{Lyubov Pakholchenko}
\textbf{Георгий Майоренко} Спасибо! Взаимно поздравляю Вас и всех ваших и наших!!!!!
\end{itemize} % }

\iusr{Людмила Шова}
Очень интересно.
Мои предки тоже с Шулявки.
Бабушка Серафима жила на Борщаговской 100 и работала в КПИ.
А Мама заканчивала 74 школу и потом училась в КПИ.

\begin{itemize} % {
\iusr{Георгий Майоренко}
\textbf{Людмила Шова} Получается, наши с вашими соседями были? А как девичья фамилия бабушки Серафимы?

\begin{itemize} % {
\iusr{Людмила Шова}
\textbf{Георгий Майоренко} Добрый день, Георгий.
Да, интересно получается.
Бабушку звали Нарабан Серафима Леонтьевна.

\iusr{Георгий Майоренко}
Благодарю. Наверняка она должна фигурировать в дореволюционных справочниках \enquote{Весь Киевъ}

\iusr{Людмила Шова}
\textbf{Георгий Майоренко}
Георгий, а где это справочник, в инете он есть?

\iusr{Людмила Шова}
Я ещё не занималась поисками, мне интересно, если Вы в курсе, то подскажите, как и где искать

\iusr{Георгий Майоренко}
Я вам в личку сброшу ссылку. В группе нельзя ссылки публиковать.

\iusr{Maria Minyaeva}
\textbf{Людмила Шова} а это не ваша семья - потомки Пушкина?

\iusr{Людмила Шова}
\textbf{Maria Minyaeva} Доброе утро Мария.
А почему Вы спросили о потомках Пушкина, по фамилии Нарабан есть какие то сведения?
Мне становиться всё интереснее.
Прийдеться углубляться в историю семьи.

\iusr{Maria Minyaeva}
\textbf{Людмила Шова} 

с фамилией нет, но где-то в тех домах жили какие-то две женщины, про которых в
те дни говорили, что они то ли потомки, то ли родственницы Пушкина. Но мамина
одноклассница, которая жила с ними в одном дворе, ничего уже не помнит,
почему-то..

\end{itemize} % }

\iusr{Maria Minyaeva}
\textbf{Людмила Шова} а мама Ваша какого года? моя 36-го тоже в 74 училась и потом в КПИ и жили где-то рядом

\begin{itemize} % {
\iusr{Людмила Шова}
\textbf{Maria Minyaeva} Доброе утро Мария.
Да, мама Зоя Янко тоже родилась в 1936 году. К сожалению, в этом году Мама ушла в иной мир.
Есть фотографии и наша память.

\iusr{Maria Minyaeva}
\textbf{Людмила Шова} моя мама была ее одноклассницей... Соболезную...
Только что смотрели выпускную школьную фотографию, на ней все такие молодые еще...

\iusr{Людмила Шова}
\textbf{Maria Minyaeva}
А Ваша мама помнит мою маму?
Как ее имя и фамилия

\iusr{Maria Minyaeva}
\textbf{Людмила Шова} конечно помнит, вот только что мне ее на фото показывала. Галина Поперенко и тут же в комментах еще одна их одноклассница нашлась

\iusr{Людмила Шова}
\textbf{Maria Minyaeva} супер, как всё рядом и тонко

\iusr{Людмила Шова}
\textbf{Maria Minyaeva} А ваши родные тоже на Шулявуе жили

\iusr{Maria Minyaeva}
\textbf{Людмила Шова} да, на Полевой, через несколько домов от ваших, но я родилась как раз через две недели после того, как их дом снесли и не успела там пожить...

\iusr{Maria Minyaeva}
\textbf{Людмила Шова} просто тема такая получилась интересная, куда привлеклись все причастные  @igg{fbicon.smile} 
\end{itemize} % }

\end{itemize} % }

\iusr{Людмила Лялькова}

\ifcmt
  ig https://i2.paste.pics/25ae9f0f8eb84152ad27cea22c57a9e4.png
  @width 0.2
\fi

\iusr{Георгий Майоренко}
\textbf{Людмила Лялькова} Благодарю! Удачи в новом году!

\iusr{Татьяна Иванова}
Спасибо за рассказ!

\iusr{Елена Ясь-Лебединская}
Родина моя! Серый дом, 142 школа...

\begin{itemize} % {
\iusr{Lyubov Pakholchenko}

В сером доме был галантерейный магазин, там продавали птичек из настоящих
перьев, красоты необыкновенной. Как же я хотела такую! Но, послевоенные годы и
мне не купили. Мы с бабушкой шли через парк политеха на Полевую и я устроила ей
такой концерт, что какой-то студент пытался взять меня на руки и успокоить. Ох
и получил он, извините, по мордам. Почему-то помню все, как будто это было
вчера.

\iusr{Вера Никитина}
\textbf{Елена Ясь-Лебединская} 

Я тоже училась в 142 школе с 1958 по 1960 гг. Родилась и до 8 лет жила на ул.
Борщаговской на пересечении с переулком Бондаренко. Помню тополя вдоль улицы.

\iusr{Георгий Майоренко}
\textbf{Елена Ясь-Лебединская} 

Серый дом смотрелся аристократично! Да и сейчас гляжу на него с восхищением!
\end{itemize} % }

\iusr{Leonid Dukhovny}

Шулявка для меня тоже родной район. Я здесь жил долгих 10 счастливых лет, Здесь
родился мой старший сын. Здесь и случилась со мной беда. А песню эту Саша
Розенбаум написал в свой первый приезд в Киев. Я ему тогда открывал наш город
(мы гуляли по Крещатику). А вечером, за рюмкой чая, он спел один куплет и
припев и сказал: \enquote{Это вот тебе}... Правда, остальные куплеты он дописал
только лет через 5, когда уже гремел во всех магнитофонах и посвящения были уже
другими.... Конечно же у меня есть песня о Шулявке. Вот фрагмент

\obeycr
....Здесь любовь я повстречал,
Здесь я жизнью рисковал,
Здесь фортуна спела мне отбой.
Но, куда б не занесло,
Всё отдам, чтоб повезло
Мне по-новой встретиться с тобой.
Пр. Эх, мать - Шулявка, дорогая,
Много о тебе я знаю.
Знаю даже ветренность твою.
Но, оставив сладки речи,
Вспоминаю наши встречи,
Оттого, наверное, пою!
\restorecr

\begin{itemize} % {
\iusr{Георгий Майоренко}
\textbf{Leonid Dukhovny} 

Спасибо огромное, дорогой Леонид, за воспоминания! Вы - не только классик
авторской песни, но и кладезь киевских историй. Книгу воспоминаний не
собираетесь опубликовать? Это было бы - супер!!!

\begin{itemize} % {
\iusr{Leonid Dukhovny}
\textbf{Георгий Майоренко} 

Дорогой земляк мой, эта книга перед вами. Я на страницах ФБ, отвечая на
вопросы, написал больше по обьёму, наверное, чем \enquote{Сага о Форсайтах}....Впрочем
книга \enquote{Житие не святого, но Духовного или НедоСказанная аксиома} уже давно
написана. А вот сверстать её никак не удаётся. Боюсь, наврное...

\end{itemize} % }

\iusr{Георгий Майоренко}
\textbf{Leonid Dukhovny} А ваша песня о Шулявке есть где-то в записях концертов?

\iusr{Leonid Dukhovny}

Она есть в записи моего прощального концерта 1992г. Она есть на моем СД.
Георгий, все эти аудио гуляют по интернету. Вы можете ещё обратиться к проф
Стефану Машкевичу, который скурпулёзно исследовал некоторые мои песни.
Простите, но, как водитсяч, я сам - \enquote{безлошадный}!

\begin{itemize} % {
\iusr{Георгий Майоренко}
\textbf{Leonid Dukhovny} Спасибо, обращусь.

\iusr{Leonid Dukhovny}
\textbf{Георгий Майоренко} А я тоже хочу сказать вам спасибо за прекрасный очерк о нашей с вами мгогострадальной Шулявке
\end{itemize} % }


\end{itemize} % }

\iusr{Nataliya Orel}

Чтож, и я свои 5 копеек о Шулявке добавлю...

Шулявка, деревня, вошла в состав Киева, жителям официально сменили название
\enquote{казеные крестьяне} на благозвучное \enquote{мещане}, а за утраченые
областью, пардон, губернией, земли компенсировали киевскими землями в
Пуще-Водице.... ну, а что былодалее мы прочли выше. ( читала както об этом
документы в областном архиве )

\begin{itemize} % {
\iusr{Юрий Панчук}
\textbf{Nataliya Orel} 

Слово \enquote{казенные} напоминает о Казенных дачах, которые находились между парком
КПИ и улицей Гарматная. Некоторые из них достояли аж до конца 70-х годов. И все
улицы этого квартала, параллельные проспекту носили названия Дачная 1,2,3,4.

Сам проспект уловно назывался Дачная 1 на этом участке.

\iusr{Nataliya Orel}
\textbf{Yury Panchuk} не совсем. Казеными крестьянами или казеным имуществом называлось все, что принадлежит государству.
\end{itemize} % }

\iusr{Татьяна Зубко Маркина}

\ifcmt
  ig@ name=scr.hands.applause
  @width 0.2
\fi

\iusr{Георгий Майоренко}
\textbf{Татьяна Зубко Маркина}

\ifcmt
  ig@ name=fbpic.hand.flowers
	@width 0.1
\fi

\iusr{Нина Бондаренко}

Дякую за цікаву розповідь. Велика частина мого життя теж пов'язана з Шулявкою.
Мій тато більше п'ятидесяти років працював на з-ді \enquote{Більшовик}. Я закінчила КПІ
і теж працювала на заводі.

\ifcmt
  ig@ name=scr.hands.applause
  @width 0.2
\fi

\begin{itemize} % {
\iusr{Георгий Майоренко}
\textbf{Нина Бондаренко} Благодарю! Удачи в новом году!

\iusr{Нина Бондаренко}
\textbf{Георгий Майоренко} И Вам здоровья мира и благополучия  @igg{fbicon.face.smiling.eyes.smiling} 
\end{itemize} % }

\iusr{Maria Minyaeva}
Хм, а Сказины это должно быть наши родственники... Мама уже спит утром расспрошу

\begin{itemize} % {
\iusr{Георгий Майоренко}
\textbf{Maria Minyaeva} А ваши Сказины с Шулявки? Если да, то вполне возможно.

\iusr{Maria Minyaeva}

мы не смогли понять. Наши жили в районе 80-х номеров по Борщаговской, после
войны там было два брата и две сестры Сказины примерно 1900 годов рождения.
Моей маме они двоюродные дяди и тети. Но непонятно могут ли они быть связаны с
тем, о ком пишете Вы, если у того было 5 дочерей и не было сына.

\iusr{Георгий Майоренко}
\textbf{Maria Minyaeva} Я посмотрю по своим материалам. Вполне возможно, что дети домовладельцев.
\end{itemize} % }

\iusr{Надежда Грабовская}
С наступающим Новым годом!

\ifcmt
  ig https://i2.paste.pics/2d9310f7ff4f9b56d23c1d16989ed371.png
  @width 0.2
	@name scr.congrats.salut
\fi

\begin{itemize} % {
\iusr{Георгий Майоренко}
\textbf{Надежда Грабовская} Благодарю, Надежда! Быть добру!

\iusr{Надежда Грабовская}
\textbf{Надежда Грабовская} Спасибо, хочется верить в лучшее!
\end{itemize} % }

\iusr{Татьяна Гурьева}

Ну а мы были соседями в далёком прошлом. А сейчас нашли друг друга! И материал
замечательный получился, ведь Шулявка не просто район. И многим я обязана Вашим
замечательным архивным находкам. Вообщем, рукописи не горят, родственные связи
не ржавеют.

\begin{itemize} % {
\iusr{Георгий Майоренко}
\textbf{Татьяна Гурьева} С наступающим! Пусть год будет плодотворным и интересным!!!

\iusr{Татьяна Гурьева}
\textbf{Георгий Майоренко}  @igg{fbicon.face.neutral}  @igg{fbicon.heart.green}  @igg{fbicon.hands.pray} 
\end{itemize} % }

\iusr{Надежда Грабовская}

\ifcmt
  ig https://i2.paste.pics/c8b073f2f3640b634d8d477d6179d533.png
  @width 0.2
\fi

\iusr{Maria Minyaeva}

Ну вот, благодаря Вашему посту нашлись две старые подруги (одноклассницы),
которые не виделись лет 30. Сейчас 1.5 часа разговаривали. Спасибо!

\begin{itemize} % {
\iusr{Георгий Майоренко}
\textbf{Maria Minyaeva} Да, это здорово! Будем считать это предновогодними чудесами!
\end{itemize} % }

\iusr{Julia Panchul}
cпасибо за фото. я жила на территории КПИ с 1946 и мне знаком тот район

\iusr{Георгий Майоренко}
\textbf{Julia Panchul} Благодарю! Удачи и добра в новом году!

\iusr{Yefim Zakon}

Задолго до войны жил немного на Борщаговской, а в 70-80-х на б. Лепсе, 7б,
напротив КИИГА... Есть, что вспомнить. Спасибо, Георгий и с наступающим!

\iusr{Георгий Майоренко}
\textbf{Yefim Zakon} Удачи и добра в новом году!



\end{itemize} % }
