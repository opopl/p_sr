% vim: keymap=russian-jcukenwin
%%beginhead 
 
%%file 02_02_2017.news.ua.rap_ua.1.yarmak
%%url https://rap.ua/nastoyashchaya-populyarnost-eto-kogda-lyudi-tebya-lyubyat-intervyu-s-yarmakom-chast-1/
%%parent 02_02_2017
 
%%endhead 

\subsection{«Настоящая популярность --- это когда люди тебя любят»: интервью с Ярмаком. Часть 1}
\label{sec:02_02_2017.news.ua.rap_ua.1.yarmak}

\url{https://rap.ua/nastoyashchaya-populyarnost-eto-kogda-lyudi-tebya-lyubyat-intervyu-s-yarmakom-chast-1/}

Ярмак --- самый популярный рэпер нашей страны. Разговор с ним так затянулся, что
нам пришлось разделить интервью на две части. В нем много его размышлений и
откровений, инсайдов, мыслей про распад Стольного  Града. Едва переступив порог
студии «Стольного», что на Ярославовм Валу в самом сердце столицы, я включил
диктофон.  

Да, я постараюсь. Всё примерно в 90 bpm. Конечно. Хорошо. Братишка, не могу уже
говорить. Ага, обнял, давай. Фэйм звонил, --- объясняет Ярмак.

Как раз на выходных писал, звал на студию в гости.

А он всех приглашает, он молодец на самом деле. Фэйм делает такую работу,
которую не делает никто, ему --- огромный респект! Я счастлив, что с ним
познакомился. В 2011-м году, в самом начале творческого пути, я выступал в Ялте
у Басты на разогреве. Тогда Баста только создал Газгольдер, туда всех набрали,
и вот у меня на глазах такая картина развернулась: сидит Фэйм на подоконнике,
подходит к нему Баста. И такой ему: привет, Фэйм, как дела, и начинает звать
его на Газгольдер. А Фэйм в ответ: да нет, спасибо, мне это не интересно всё. А
я стою просто офигевший --- на тот момент в моём сознании Газгольдер был номер
один! А Фэйм им --- да нет, фонарь. Я тогда подумал: ничего себе, какой штрих
самодостаточный!

Мы сейчас на сайте пишем колонки о проблемах украинского рэпа. Какие основные
проблемы ты бы мог назвать?

То, что делает весь наш украинский рэп --- этого недостаточно, чтобы собирать
залы и зарабатывать. К этой точке прийти пиздец как непросто --- многие думают,
что выпустят клип, он наберет миллионы и понесутся концерты. Нет! Нужно
выпустить десятки клипов, которые наберут миллионы --- и только тогда пойдёт
какая-то движуха.

Я считаю, что человек начинает собирать залы, когда количество просмотров
перевалило как минимум за 10 миллионов. И то --- залы не будут большими. Пример –
MiyaGi & Эндшпиль. Они когда преодолели приблизительно эту отметку --- начали
собирать по 800 человек, по 1000. У нас очень мало ребят, которые достигли
такого. Вот вы недавно делали рейтинг, 40 самых просматриваемых клипов прошлого
года, и кто-то в комментариях написал, что, мол, очень печально, что только 10
групп перевалило отметку в миллион просмотров. Вот какая у нас в Украине
ужасная ситуация с продуктом, подходом к нему, никто не знает правил.

Правил чего?

Правил индустрии.



    Слушай
    Смотри
    Читай
    Будь в сердце событий
    TOP20UA

    Отправить материал
    Размещение рекламы

Новости
02 Фев, 2017
«Настоящая популярность --- это когда люди тебя любят»: интервью с Ярмаком. Часть 1

Ярмак --- самый популярный рэпер нашей страны. Разговор с ним так затянулся, что нам пришлось разделить интервью на две части. В нем много его размышлений и откровений, инсайдов, мыслей про распад Стольного  Града. Едва переступив порог студии «Стольного», что на Ярославовм Валу в самом сердце столицы, я включил диктофон.  

Да, я постараюсь. Всё примерно в 90 bpm. Конечно. Хорошо. Братишка, не могу уже говорить. Ага, обнял, давай. Фэйм звонил, --- объясняет Ярмак.

Как раз на выходных писал, звал на студию в гости.

А он всех приглашает, он молодец на самом деле. Фэйм делает такую работу, которую не делает никто, ему --- огромный респект! Я счастлив, что с ним познакомился. В 2011-м году, в самом начале творческого пути, я выступал в Ялте у Басты на разогреве. Тогда Баста только создал Газгольдер, туда всех набрали, и вот у меня на глазах такая картина развернулась: сидит Фэйм на подоконнике, подходит к нему Баста. И такой ему: привет, Фэйм, как дела, и начинает звать его на Газгольдер. А Фэйм в ответ: да нет, спасибо, мне это не интересно всё. А я стою просто офигевший --- на тот момент в моём сознании Газгольдер был номер один! А Фэйм им --- да нет, фонарь. Я тогда подумал: ничего себе, какой штрих самодостаточный!

Мы сейчас на сайте пишем колонки о проблемах украинского рэпа. Какие основные проблемы ты бы мог назвать?

То, что делает весь наш украинский рэп --- этого недостаточно, чтобы собирать залы и зарабатывать. К этой точке прийти пиздец как непросто --- многие думают, что выпустят клип, он наберет миллионы и понесутся концерты. Нет! Нужно выпустить десятки клипов, которые наберут миллионы --- и только тогда пойдёт какая-то движуха.

Я считаю, что человек начинает собирать залы, когда количество просмотров перевалило как минимум за 10 миллионов. И то --- залы не будут большими. Пример --- MiyaGi & Эндшпиль. Они когда преодолели приблизительно эту отметку --- начали собирать по 800 человек, по 1000. У нас очень мало ребят, которые достигли такого. Вот вы недавно делали рейтинг, 40 самых просматриваемых клипов прошлого года, и кто-то в комментариях написал, что, мол, очень печально, что только 10 групп перевалило отметку в миллион просмотров. Вот какая у нас в Украине ужасная ситуация с продуктом, подходом к нему, никто не знает правил.

Правил чего?

Правил индустрии.

Кстати, ты же недавно хотел провести совместно с Фэймом лекцию о хип-хопе, где
хотел рассказать о специфике индустрии и инструментах продвижения. Эти правила
ты и имеешь в виду?

Да, я хочу это сделать. Эти правила --- тот минимум, который должен знать
начинающий артист. Например, вышеупомянутые 10 миллионов просмотров. Миллионов
5 нужно, чтобы о тебе узнали.

А если брать не YouTube, а какие-то другие показатели?

Общество стало слишком ленивым и воспринимает всё только визуально. В наше
время «В жизни так бывает» так бы не стрельнула. Аудиокомпозиции сейчас –
пережиток прошлого, они стреляют только у уже известных артистов. Нельзя просто
выложить песню и стрельнуть на весь мир --- я это точно знаю, потому что мы
анализировали всё это. Миллионы хитов умирают каждый день, а тонны говна
всплывают на поверхность, потому что у него правильный маркетинг и промо.
Поэтому без клипа в наше время нереально пробиться.

Вообще самое первое правило --- психологическое, то, с чего всё начинается. Самый
главный вопрос: зачем ты это делаешь? Мною в 9 классе, например, двигало
желание нравиться девочкам, потому что они любили красавчиков, а я таким не
был. Я начал играть в КВН, у меня получилось, я сразу стал звездой в школе.
Однако важно ощущать, любишь ли ты то, что делаешь, и если любви там нет, а
есть только удовлетворение комплексов, то с этим нужно срочно завязывать. У
меня была настоящая любовь, мне нравилось дарить улыбки.

Все хотят всю жизнь только получать --- в нашей стране особенно. Ни хуя не
сделать, но получать. Минимально что-то сделают --- а хотят такого фидбека, чтоб
им золотые слитки падали с неба!

Важно понимать --- есть ли тебе что сказать этому миру. Вот у меня всегда было
ощущение, что мне есть что сказать, я еще в детстве во дворе рассказывал что-то
друзьям, хотел их изменить, в университете так же происходило. И КВН был
инструментом, в котором у меня это не совсем получилось. А когда я начал
заниматься музыкой, я понял, что это мое, это огромная сила. Пускай меня
услышат 10 000, но из них человек 100 что-то изменит в себе или на планете.
Когда я осознал всю эту силу, это случилось уже после того, как я стал
популярным --- где-то во времена Майдана, так совпало, что у меня тогда произошёл
перелом, я понял, как музыка влияет на людей. И когда я понял, что мне дали
автомат, а я из него стреляю, как дебил, направо-налево --- я начал задумываться,
куда нужно стрелять и нужно ли стрелять вообще.

Так что первая вещь --- нужно понять, любит ли человек своё дело, вторая --- есть
ли ему что сказать. Потом уже идут реальности индустрии. Песней сейчас
стрельнуть не реально, пришла эра информации.

Что для тебя значит «стрельнуть»?

Вот недавний пример --- Грибы.

А если классную песню возьмут на радио --- будет считаться, что она стрельнула?

Нет (смеётся). У нас вся эта история не работает --- или очень много надо денег
проплатить на радио.

Ты считаешь, что традиционный, эстрадный путь, не работает?

Работает, но не на ту аудиторию, наверное, которую ты хочешь. Радио навязывает,
а хочется, чтобы человек сам загрузил твою песню в свой плейлист, заморочился,
нашел шнур, удалил пару ненужных видео ради этой песни, вот тогда это любовь
(улыбается). Либо настоящее признание --- через Интернет. Либо «эффект звезды» –
через радио и ТВ.

Но можно же совместить, чем ты и занимаешься.

Это гармония, это то, к чему я иду. Я раньше говорил --- только через интернет,
никаких каналов и радиостанций, а потом понял, что это неправильный путь и
начал в себе что-то менять. Понял, что если твой клип показывают и по
телевизору, то это лишним не будет, это показатель того, что мы отвоевываем
земли Мордора (смеётся).

Ты перестал материться в песнях с тех пор, как твои клипы взяли в телевизор?

Я перестал материться, когда понял, что у меня автомат в руках. В новом альбоме
только один мат --- но он там на своём месте.

На меня никогда не влияли ни форматы, ничего такого. Моя задача --- делать то,
что выходит из меня. Если оно подходит под формат радио или телека --- ОК. Не
подходит --- ну и ладно.  В первую очередь мы делаем то, что хотим.

Как, кстати, ты к Грибам относишься?

Ну, тут с разных сторон можно посмотреть. Но больше рад их появлению, чем нет.

Как музыка тебе это нравится?

Нет. Это не моё направление, оно деструктивное, тема наркотиков мне не близка.

Но там не так много наркотиков.

Там нормально наркотиков --- «велосипед, два колеса, одно тебе, другое мне». Мне
тяжело им дать какую-то оценку, потому что я знаю, что там парни --- рэперы,
Чатти и Симптом, но то, что происходит в их песнях, это зачастую не рэп, там
иногда очень слабые рифмы.

Мне кажется, они специально это делают.

Вот! И, как мне сказал Влади из «Касты» --- те, кто их критикует, воспринимают их
как рэперов, а если эти песни услышит обыкновенный обыватель, то для него это
будет круто.

ОК, чем они тебе нравятся?

Своей стильностью! Они очень стильные, за это им уважуха. Нравится, как звучит
Симптом. И, если откровенно, когда у меня что-то получалось и Чатти писал
какие-то неприятные вещи --- то у меня такого нет, я всё это перерос.

А что он писал?

Ну, была ситуация --- у меня концерт в Стереоплазе, и наш Михалыч пишет --- мол,
долгожданный концерт Сани, приходите, а Чатти в ответ --- ни хуя не долгожданный,
что-то такое. Вообще я очень рад, что Грибы появились, их всё-таки приписывают
к хип-хопу, и они серьёзную планку задали --- собрали 6000 в Киеве, по Украине
хорошо собрали.

А что ты думаешь по поводу их тура по России?

Я это не хочу комментировать. Я пытался разобраться в деталях, почему вся эта
каша происходит между Украиной и Россией, и кто там что мутит.

Проблема в том, что половина людей, которые у власти --- они всё-таки
подконтрольны России, разным криминальным группировкам. Много расследований
проводят журналисты, разоблачают всех, рассказывают, кто под кем, кто кого
контролирует. И, конечно, эта война не закончится, когда одна часть Украины –
это русская криминальная группировка, а вторая более прозападная. Настоящая
война давно закончилась --- она была тогда, когда пошли батальоны, когда на фронт
пошли мои друзья, которые рассказывали, что они по 7 раз высоту захватывали.
Они берут высоту --- им приказывают отступить. Опять берут --- снова приказывают
отступить. А сейчас это уже не война, это шахматы --- политики выбивают себе
условия получше, я в этом участие не принимаю. Я считаю, что главный враг в
Киеве и пока тут ничего не поменяется, там и дальше будут гибнуть ни в чем не
виновные парни. Я много раз там бывал и видел, как все это для многих
превратилось в бизнес. Я не обобщаю, достойных парней там хватает, но из тех,
кто действительно мог что-то изменить в стране, многих положили в котлах. Мне
часто говорят «А чего ж ты не пошел в АТО?».  У власти только этого и ждут,
чтобы всех, кто реально борется, где-то присыпали песочком на востоке. Так вот
ХУЙ ВАМ!

Тем не менее, ты с концертами в Россию не ездишь.

Не езжу. Потому что у нас в стране нереальные проблемы, в первую очередь
психологические. Россия --- она огромная, дайте мне разобраться в своей стране,
вот и всё, мне некомфортно здесь жить.

Но ты же не едешь разбираться в стране --- ты едешь показать свою музыку.

Нет. У меня как-то так сложилось, что в моих песнях звучат определённые
призывы,  и песни эти повлияли на многих людей. Я совершенно не стесняюсь того,
что меня слушает много подростков --- они готовы воспринимать, готовы что-то в
себе менять, готовы впитывать. Они ищут себе авторитетов на улице --- и у меня
так же было, я не слушал, что мне говорили дома, я слушал, что говорили в
песнях, ещё где-то.

Какие песни оказали на тебя наибольшее влияние?

О, у меня большой разброс в моих вкусах --- очень мне, например, нравился «Бумбокс». Баста, «Каста» --- я когда в 9 классе учился, наверное, год слушал песню «Вокруг шум», это для меня супер-хит был. Я слушал всё --- мог и «Руки вверх», мог и «Nickelback».  Мне никто не рассказывал никогда о музыке --- такая яма была в индустрии! Появлялось много интересных музыкантов, а музыкальных редакторов, крутых передач о музыке не было. Всё закончилось на журнале X3M и «Території А», которые были фиг знает когда. Я никогда не забуду 2003 год, когда приехали Bomfunk MC’s --- это была моя любимая группа, я записал на кассету их выступление, и сотни раз потом пересмотрел его. Вот закончились «Таврійські ігри» --- и настала эра фейковых звезд, и я в эту эру доформировывался.

Что для тебя фейковая звезда?

Это звезда, которой купили популярность. Я никогда такого не хотел --- когда ты на радио и на каналах, твой инстаграм накручен, а тебя не любят. Любовь нужно заслужить и это очень тяжёлый труд. Я считаю, что настоящая популярность --- это когда люди тебя любят. Мне не нужна такая популярность, чтоб меня каждая собака знала, но не уважала. Потому я постоянно работаю над собой, над всем абсолютно --- даже на танцы скоро пойду.

Многие не понимают, зачем ты вообще собрал Стольный? Ты был один, зарабатывал, всё было хорошо.

Я тогда ещё ничего не зарабатывал. Я с ними познакомился, когда у меня первый клип вышел. Они для меня были на тот момент мастера. Я не умел читать рэп --- у меня были другие навыки, но читать я не умел. А они на тот момент довольно интересно читали, мне хотелось у них поучиться, я их очень уважал за то, что у них есть такие таланты.

Я в своей жизни уже убедился, что гуртом легше и батька бить --- потому что когда вы вместе что-то делаете --- вы все друг у друга чему-то учитесь. Если ничего не делать --- ничего не будет. А если делать хотя бы что-то --- оно всё равно каждому что-то даст. Если б мы тогда не собрались --- у ребят, может быть, все сложилось гораздо хуже.

Как проходило ваше развитие?

Изначально мы начинали делать все с Михалычем, Марти, Фиром и АрХангелом, потом Никита пошел своей дорогой. Где-то через годик я подтянул Тофа --- он учился во Львове, заканчивал четвертый курс, и он мне очень понравился, я до сих пор его фанат. Вот как Баста --- фанат Гуфа, так я --- фанат Тофа. Мы висели у него на хате --- поругались с организаторами, потому что они отказались нас во второй раз кормить --- да, такие вот у нас гастроли в Украине, даже на самом высоком уровне такое бывает! Так что мы послали организаторов, и Тоф пригласил нас к себе домой. У него очень прикольная квартира, с выходом на крышу --- это был невероятный вечер! Мы пели друг другу песни, показывали демки --- и я Тофу на листике написал, чтоб он собирал вещи и переезжал к нам в Киев. Он очень счастлив был, в итоге набухался тогда.

Мы первый год жили вместе --- в одной кровати даже спали. Было крутое время --- мы постоянно находились вместе, многому научились друг у друга, тогда родилась и «22», и многие другие песни.

Был период, когда мы впервые собрали Зелёный театр, полторы тысячи человек --- для украинского хип-хопа это был рекорд, до этого только Гига когда-то тысячу собирал, мы взяли новую планку. Мы тогда ещё были малолетками, которым хотелось что-то доказать, мы так горды были собой, что мы такую движуху навели --- но концерт и правда был очень крутой, прям тру-тру-тру.

Два года назад мы в «Форсаже» выступали --- то ли Стольным, то ли я соло, и ко мне подошёл Гига. Говорит --- мол, Сань, я смотрю, вы молодцы, развиваетесь, я сейчас ищу лейбл, хотел бы с вами встретиться… Разговор получился нелепым --- я начал рассказывать, что у нас своя идеология, что-то в таком духе --- и он обиделся. Мне неудобно стало --- я по жизни вообще очень дружелюбный человек. Мне было неловко и я с этим год жил до момента, когда мы решили на Платформе сделать концерт --- нам тогда нужен был МС, и я предложил позвать Гигу. Мне интересно было с ним пообщаться, потому что я до этого сложил о нём не самое лучшее впечатление --- жил слухами; тот одно сказал, тот другое. А когда он сел с нами за стол --- я понял, что это просто брат мой.

«Хочешь, чтоб у нас была хорошая страна --- начни с себя». ОК, но многие не понимают, что ты имеешь в виду, а уступить место в транспорте и не сорить на улице для них недостаточно. Что делать?

Я, к сожалению, пока не выработал чёткую и понятную систему, как говорить об этом. Я обладаю определёнными знаниями, которыми я пытаюсь поделиться в интервью, встречах со студентами, и т.д. У меня есть товарищ, который выработал просто офигеннейшую сиситему, в которой всё это рассказывается.

Начинается всё со времени --- он на доске пишет, сколько мы живем, сколько спим, сколько на что тратим. Потом спрашивает --- сколько времени в день мы сидим в соцсетях? Например, три часа --- и получается, что мы несколько лет жизни кладём на это. Так благодаря элементарной математике становится понятно, насколько мы непродуктивно расходуем своё время. Я его знания пытаюсь перенять --- у него есть система, а я могу эти знания передавать.

Через песни?

И через песни тоже. Я, например, практически уверен в том, что мы перерождаемся. Вот у меня на ЕР «Миссия Орион» есть песня, называется «Пилигрим».Там есть такие строчки: «Сам себе выбирай дан и петиции/ Сам себе выбирай, где и кем родиться». Я уверен, что дети выбирают себе родителей, я почти что убеждён.

Как это происходит?

Я не знаю. Когда я буду знать, наверное, меня повесят на кресте (смеётся). Я
уверен, что я выбрал себе самую обычную семью, выбрал родиться в Борисполе, а
не где-то в Кировоградской области --- потому что вот, чувак, Киев рядышком,
нужно немного поработать --- и всё будет. Я убеждён, что выбрал себе такой
сценарий в жизни.

А если бы ты выбрал не быть Ярмаком из Борисполя?

Не знаю, скажу честно --- когда меня спрашивают, что я хотел бы изменить --- нет,
ничего.

Стандартный ответ.

Я не скажу, что у меня что-то плохо. Да, может, я, как Цукерберг, в 21 год
миллионером не стал, но в мои 25 у меня дела идут неплохо.

Есть какие-то детали, которые я, возможно, хотел бы изменить. Но всё равно всё
в итоге приходит к пути, по которому я иду. Потому сейчас я уверен в своих
силах. Думаю, что то, что люди от меня слышали --- это от силы 30 процентов от
того, что будет в моей творческой карьере.

Слава пришла ко мне в 20 лет --- я был ни фига  не сформированным, ни фига не
понимал ни в чем --- ни в музыке, ни в психологии, но было огромное желание
что-то делать. И это желание в итоге привело меня сюда.

Я не жалуюсь ни на что --- если посмотреть на русский рэп– среди моих сверстников
почти никого нет с таким уровнем популярности. Есть Мияги и Эндшпиль, которые
сейчас появились, но я сомневаюсь, что они долго продержатся, потому что мне
рассказывали, какое количество травки они употребляют.

Снупу это не мешает.

Мне всегда апеллируют Куртом Кобейном, например, другими персонажами.
Статистика, к сожалению такова: 0,000001 процента, все остальные заканчивают
очень плохо.

Выходит, это как раз Снуп и есть.

Да, и это плохая статистика! Я за хотя бы 80 на 20! Это такое --- повезёт –не
повезёт, как звёзды сложатся.

Ты против лигалайза, декриминализации?

Я просто ещё не был в странах, где полный лигалайз, я был лишь в Голландии.
Голландцы говорят, что курят там только туристы, местные --- ну, те, кто в
кофешопах работает. А местные --- мне кажется, их уже задолбали эти все
весельчаки под травкой.

Ты слышал, что в Грузии декриминализовали?

Да, слышал. Наверное, в Грузии не самые глупые люди --- я был там и мне очень
понравилось, но пока не разберусь с опытом других стран --- не хочу
высказываться. Например, я был в Арабских Эмиратах и мне там тоже очень
понравилось, хотя там запрещены все наркотики. Это пример того, как из говна
сделали пулю.

При бешеных деньгах.

Я тебе скажу --- в нашей стране не меньше бабок, чем в Эмиратах. Например, у меня
есть друг, у которого папа --- соляной магнат. Он говорит, что у нас соли
столько, что можно весь мир обеспечить --- и спрос есть. У нас столько залежей
разных пород и всего остального!

Слушай, но соль всё-таки дешевле нефти.

Да, дешевле, но в некоторых странах её совсем нет, и она тупо экспортируется.
Когда я приехал из Дубая и написал, что там люди молодцы, сделали сказку , один
парень прокомментировал: если бы была у нас нефть, мы бы тоже построили сказку!
А другой сразу снизу пишет: если бы была у нас нефть --- мы бы все пробухали! Вот
это истина! Проблема не в ресурсах --- в России вот и алмазы, и нефть, и газ, и
другое. И что, они живут как Эмираты? Нет! Поэтому я в Россию не лезу со своими
мыслями, песнями, --- там хватает проблем, там должны быть люди, которые должны
пытаться изменить ситуацию.

Но если не брать во внимание твою просветительскую миссию --- это же деньги, в
конце концов. Большие.

Очень большие! У меня там сейчас были бы очень большие деньги, там платили
всегда в долларах, там огромнейший тур можно сделать --- как поехал, так на 3
месяца сразу. Но я никогда ничего не измерял только деньгами. Если говорить про
тру --- наверное, в этой стране нет большего тру, чем я. Я никогда не продавался
за какие-то гроши, мне важна миссия и идея, и если деньги идут параллельно им –
ОК. Вот мне предлагают огромные бабки за рекламу водки, и я за эти деньги могу
начать снимать крутые клипы, купить себе три тачки, квартиры и всё остальное –
но я помню, что у меня есть автомат, которым я не хочу убивать людей.

А ты не думал взять деньги и отдать их на благие дела?

Да я и так отдаю. Что это за благотворительность --- тут 10 000 убил, тут двум
помог? У меня именно из Украины в ВК 450 000 подписчиков --- из них выпило,
например, 50 000, а 20 000 жестоко набухалось. Из этих 20 000 у 5 очень плохо
сложится жизнь, такая статистика. Зато я, например, двоим помогу.

А почему именно двоим, а не сотне человек?

Потому что чтоб действительно помочь, нужны не деньги, нужны знания.

Знаю, что ты помогаешь детским домам.

Я им в первую очередь даю знания --- приезжаю, общаюсь с ними, спрашиваю об их
проблемах, пытаюсь что-то подсказать, подаю пример.

Какие у них сейчас проблемы?

У них нехватка внимания и любви. Я им привожу Xbox, а он им не очень интересен,
для них главный подарок --- что я приехал и уделил им внимание. Мы постоянно их
возим куда-то --- в аквапарк, на шоу Мамахохотала, например. Я ж говорю –
деньгами не поможешь. Ну, вот есть у меня, например, миллион гривен --- и чтоб
помочь сотне людей, нужно раздать по 10 000. К сожалению, они их не спасут –
они, например, купят себе по айфону и это ничем им не поможет. Другое дело –
если я расскажу им формулу счастья или закон силы мысли.

В чем формула счастья? Ты счастлив?

Я счастлив, и это получилось само собой. Первое --- это любимый человек. Когда у
человека есть взаимная любовь, происходит перезапуск молекул в организме, ты
никогда не будешь болеть --- короче, это физика и химия, не углубляемся, просто
поверьте на слово. Второе --- это любимое дело. Меня постоянно окружают люди,
которые занимаются не тем, что любят. Например, сериал нам монтировал Петя,
который все заработанные деньги тратил на примочки для гитар --- он мечтал
сделать свою группу. Приходил к нам с Фэймом чувак миллионер, владелец
спортклубов, говорит, всю жизнь мечтал петь --- а потом то деньги надо было
зарабатывать, то тяжелые 90-е…

Судя по тому, что он миллионер, всё же что-то да получилось.

Но он не до конца счастлив. Деньги --- не то, что делает его счастливым. Вроде
всё есть --- отдыхает на Бали, тачки, дома, но что-то не так.

Третье --- это правильное окружение, потому что оно строит вашу судьбу. Когда я
жил в Борисполе, всё шло очень медленно. Там живут обычные люди, которые не
достигли никаких высот, мыслят очень шаблонно и дельный совет вряд ли смогут
дать.

А в Киеве необычные?

В Киев съезжаются все-таки лучшие со всей Украины, тут больше шансов найти
какую-то нишу. Вот мы ж с тобой нашли какую-то свою нишу --- рэп, мы как-то все
коннектимся друг с другом. А там люди вообще не знали, что такое рэп, для них
рэп --- это Серёга.

Когда я начал заниматься рэпом, надо мной смеялись. Я всю жизнь был уважаемым,
ровным пацаном, который разруливал любые движухи: от школьных до серьёзных
разборок, а не таких, о которых я слышу в песнях про падики. Я знаю людей –
настоящих бандитов, сейчас в моём окружении есть серьёзные люди.

Я видел у тебя в инстаграме, как ты играешь тему из «Бригады» на клавишах --- для
тебя вся эта бандитская романтика близка?

Да. Я не знаю, почему так сложилось --- может, опыт прошлой жизни, но в детстве у
меня любимый сериал был «Бригада». В новом альбоме я пытаюсь всё это из себя
выжить --- просто я понимаю, что это далеко не идеальный мир, и как всё это плохо
заканчивается --- совсем не так, как в фильмах.

Я жил в Борисполе --- там все мурчат, там какая-то такая мода была, после 90-х
всё это осталось --- драки, авторитеты, и я так сформировался, к сожалению. Я
подчеркиваю --- к сожалению! Потому что я знаю, что ничего крутого в этом нет. Я
когда рассказываю людям про жизнь грузчиков из аэропорта, которые выносили
наворованное золото в ладонях после смены или местного авторитета, который на
велике банк ограбил, люди за голову берутся, это все реальная жизнь, но все эти
люди плохо заканчивают. Какие-то жизненные вещи нужно пройти --- уличные драки,
например. Я считаю, что человек должен попробовать всё в этой жизни --- тогда он
поймёт истину.

«Всё» --- это и героин, например, тоже?

Ну, возможно, героин нет. Я в своей жизни пробовал траву. ЛСД или кокаин --- нет.
Когда я учился на втором курсе --- бухал достаточно плотно, и проблемы из-за
этого были, я всё это прошёл, знаю, чем это заканчивается.

Так вот, правильное окружение.

Я сейчас создаю окружение, которое мне действительно родное. Это люди, которые
не пьют, не курят, пытаются как-то изменить нашу планету, улучшить её. Мое
окружение становится сейчас максимально похожим на меня. Мы даже с моей
командой расписали, к чему мы идем в ближайшие пять лет.

Наша глобальная миссия --- максимально изменить мир к лучшему.

Также мы в Киеве хотим построить Дом развития. Там будет современная
музыкальная школа, секция единоборств, танцевальный класс, будут разные
образовательные кружки.

Напоминает Мастерскую, которую строит Дорн.

Я не знаю, что именно Ваня сейчас делает --- но знаю, что он созидательный
парень, и я с уверенностью могу сказать, что он в сто раз больше тру, чем 99
процентов рэперов.

А что для тебя тру?

Для меня тру --- это когда ты в жизни такой же, как в песнях. Тру может быть
совершенно разным --- у Тимати тоже свой тру, он не снюхал весь кокаин Москвы, а
построил бизнес, хотя я знаю кучу мажоров, которые загубили возможности, данные
им родителями.  У Скриптонита --- своё тру, он парень из Павлограда, который
употребляет наркотики. У нас в Украине почти нет тру --- не могу кого-то назвать,
просто у нас всё плохо с хип-хопом в стране. Я со многими би-боями и хип-хоп
деятелями общался, с которыми у нас сейчас наладился коннект, например, Scream
и Vag, мы раньше как-то и не пересекались. И мы пришли к выводу, что идём-то мы
всё равно все в одном направлении, цель у нас одна --- что же нам делить? Славу?
Тогда это просто эгоизм --- во мне его нет. Я вообще не боюсь любого
соперничества, меня, скорее, заводит, когда кто-то круче меня, значит, нужно
больше работать над собой, не быть королём среди шутов, побеждать лучших.

Раз ты не боишься соперничества --- ты бы вышел на баттл?

Меня интересует только Оксимирон. Меня не интересует хайп, мне просто есть что
ему сказать. 
