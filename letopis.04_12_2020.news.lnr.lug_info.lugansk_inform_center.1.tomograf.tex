% vim: keymap=russian-jcukenwin
%%beginhead 
 
%%file 04_12_2020.news.lnr.lug_info.lugansk_inform_center.1.tomograf
%%parent 04_12_2020
 
%%url http://lug-info.com/news/one/minzdrav-lnr-priobrel-kompyuternyi-tomograf-dlya-kliniki-meditsiny-katastrof-foto-62507
 
%%author ЛуганскИнформЦентр
%%author_id lugansk_inform_center
%%author_url 
 
%%tags lnr,medicina
%%title Минздрав ЛНР приобрел компьютерный томограф для клиники медицины катастроф (ФОТО)
 
%%endhead 
 
\subsection{Минздрав ЛНР приобрел компьютерный томограф для клиники медицины катастроф (ФОТО)}
\label{sec:04_12_2020.news.lnr.lug_info.lugansk_inform_center.1.tomograf}
\Purl{http://lug-info.com/news/one/minzdrav-lnr-priobrel-kompyuternyi-tomograf-dlya-kliniki-meditsiny-katastrof-foto-62507}
\ifcmt
	author_begin
   author_id lugansk_inform_center
	author_end
\fi

\index[names.rus]{Сулименко, Марина!Фото, томограф, ЛНР, 04.12.2020}

\ifcmt
tab_begin cols=4
	caption Минздрав ЛНР приобрел компьютерный томограф для клиники медицины катастроф (ФОТО), Фото: Марина Сулименко / ЛИЦ

	pic http://img.lug-info.com/cache/1/c/(842)_2.jpg/1000wm.jpg
	pic http://img.lug-info.com/cache/f/5/(960)_1.jpg/1000wm.jpg
	pic http://img.lug-info.com/cache/8/8/(297)_12.jpg/1000wm.jpg
	pic http://img.lug-info.com/cache/6/0/(324)_11.jpg/1000wm.jpg
	pic http://img.lug-info.com/cache/a/1/(409)_10.jpg/1000wm.jpg
	pic http://img.lug-info.com/cache/3/4/(433)_9.jpg/1000wm.jpg
	pic http://img.lug-info.com/cache/8/2/(476)_8.jpg/1000wm.jpg
	pic http://img.lug-info.com/cache/f/5/(572)_7.jpg/1000wm.jpg
	pic http://img.lug-info.com/cache/1/3/(582)_6.jpg/1000wm.jpg
	pic http://img.lug-info.com/cache/0/b/(652)_5.jpg/1000wm.jpg
	pic http://img.lug-info.com/cache/d/a/(742)_4.jpg/1000wm.jpg
	pic http://img.lug-info.com/cache/7/4/(798)_3.jpg/1000wm.jpg
tab_end
\fi

\index[names.rus]{Пархомчук, Демьян!ЛНР!и.о. директора ЦЭМПМК}

Министерство здравоохранения ЛНР приобрело компьютерный томограф для клиники
медицины катастроф, являющейся филиалом Луганского республиканского центра
экстренной медицинской помощи и медицины катастроф (ЦЭМПМК). Об этом сообщил
исполняющий обязанности директора ЦЭМПМК Демьян Пархомчук.

\enquote{В этом филиале было впервые (в ЛНР) организовано отделение экстренной
неотложной медицинской помощи. Министерством здравоохранения приобретен
компьютерный томограф, который будет находиться в этом отделении, и в ближайшее
время он будет установлен}, - проинформировал Пархомчук, уточнив, что по
программе\Furl{http://lug-info.com/news/one/lnr-do-2023-goda-sozdast-i-realizuet-programmu-razvitiya-zdravookhraneniya-pasechnik-35104} развития здравоохранения Республики клиника получит еще несколько
единиц современного медицинского оборудования.

Исполняющий обязанности директора ЦЭМПМК рассказал, что Республика создавала
отделение экстренной медпомощи на основе опыта Российской Федерации. Здесь есть
свой операционный блок и реанимационная. В сутки клиника может принять до 100
пациентов и оказать им неотложную медицинскую помощь.

\enquote{Это отделение краткосрочного лечения больных. Оно позволяет нам проводить
сортировку поступающих в лечебное учреждение пациентов. Выполняются
сортировочные мероприятия, больным оказывается помощь, неотложные мероприятия.
После оказания помощи пациент может оказаться на койке краткосрочного
пребывания либо на койке динамического наблюдения, (где может пробыть) до трех
суток}, - пояснил Пархомчук, добавив, что пациент, получивший помощь в клинике,
в дальнейшем либо выписывается на амбулаторное лечение, либо переводится в
профильное лечебное учреждение.

\enquote{Создание клиники сокращает время нахождения пациента и сокращает, самое
главное, время до оказания специализированной квалифицированной медицинской
помощи, приближая медицинские технологии и медицинские услуги к пациенту}, -
уточнил он.

Заведующая отделением экстренной неотложной медицинской помощи центра
Александра Симонович отметила, что клиника в первую очередь принимает больных
из Луганска.

\enquote{Когда пациент сюда поступает, он сортируется. Если его состояние среднее или
тяжелое, то он определяется как носилочный, и тогда он попадает в
противошоковую палату. Такой пациент в первую очередь будет осмотрен врачом
неотложных состояний}, - рассказала она.

Заведующая отметила, что в отделении есть все необходимое для оперативного
обследования пациента и постановки точного диагноза, что позволяет быстрее
начать лечение. 

В августе 2019 года Правительство ЛНР приняло распоряжение\Furl{https://sovminlnr.ru/akty-soveta-ministrov/rasporyazheniya/19590-o-reorganizacii-gosudarstvennogo-uchrezhdeniya-luganskoy-narodnoy-respubliki-specializirovannaya-zheleznodorozhnaya-bolnica-putem-ego-prisoedineniya-k-gosudarstvennomu-uchrezhdeniyu.html} о реорганизации
железнодорожной больницы в Луганске путем присоединения учреждения к ЦЭМПМК. В
результате реорганизации в марте текущего года была создана клиника медицины
катастроф как филиал центра экстренной медицинской помощи.

Напомним, 12 мая 2018 года глава ЛНР Леонид Пасечник обратился к представителям
законодательной и исполнительной власти Республики, представив пятилетнюю
программу развития ЛНР \enquote{Наш выбор}.\Furl{http://lug-info.com/news/one/io-glavy-lnr-predstavil-pyatiletnyuyu-programmu-razvitiya-respubliki-nash-vybor-foto-35133} В частности, он отметил, что власти ЛНР до
2023 года планируют разработать и реализовать республиканскую программу
развития здравоохранения, включающую создание условий для обеспечения средним и
младшим медперсоналом всех медицинских учреждений, сохранение и улучшение
высококвалифицированной медицинской помощи; строительство системы доступной
медицинской помощи в отдаленных населенных пунктах; повышение уровня
обеспечения лекарственными средствами; развитие материально-технической базы
медицинских учреждений.

{\bfseries 
ЛуганскИнформЦентр — 04 декабря — Луганск
}

