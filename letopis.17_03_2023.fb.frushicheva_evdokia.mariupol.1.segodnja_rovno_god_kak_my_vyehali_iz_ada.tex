%%beginhead 
 
%%file 17_03_2023.fb.frushicheva_evdokia.mariupol.1.segodnja_rovno_god_kak_my_vyehali_iz_ada
%%parent 17_03_2023
 
%%url https://www.facebook.com/evdokia.frushicheva/posts/pfbid031sNLTfpvLqJXShgkpd4GTsCfrza6bdj453JkNRg7nB9v789SmD8HFSBpyUt3YAzGl
 
%%author_id frushicheva_evdokia.mariupol
%%date 17_03_2023
 
%%tags mariupol.war,mariupol,text.story
%%title Год... Сегодня ровно год, как мы выехали из ада
 
%%endhead 

\subsection{Год... Сегодня ровно год, как мы выехали из ада}
\label{sec:17_03_2023.fb.frushicheva_evdokia.mariupol.1.segodnja_rovno_god_kak_my_vyehali_iz_ada}

\Purl{https://www.facebook.com/evdokia.frushicheva/posts/pfbid031sNLTfpvLqJXShgkpd4GTsCfrza6bdj453JkNRg7nB9v789SmD8HFSBpyUt3YAzGl}
\ifcmt
 author_begin
   author_id frushicheva_evdokia.mariupol
 author_end
\fi

Год... Сегодня ровно год, как мы выехали из ада. Частично стала забывать
детали.

Война пришла в наш двор не сразу.

День 6-й Ночью было шумно, загорелся дом в соседнем дворе, пожарных нет, он
горит...

День 7-й Открыли магазин Мерто и люди стали выносить все, что можно. Первый
прилет по Метро. В соседнем дворе горит уже два дома. Огонь перекинулся через
переход. 

День 9-й Открыли Порт-Сити. Первый серьезный прилет в дом. Не стало газа -
последнее благо цивилизации. В соседнем дворе два дома ещё горят.

День 12-й самый тихий день, когда мы решили, что все уже закончилось. Только
дома в соседнем дворе ещё тлеют.

День 13-й - поздравление с международным женским днём. "Фейерверк", а ещё
какой-то мужчина подарил нам одну розу на всех. Мы ее поставили в баночку с
водой в подъезде. Жаль, что телефон разрядился, хотелось сделать фото. Роза
стояла на фоне стопки дров и разбитых окон.

День 14-й Пришел знакомый с 6-го этажа за тёплыми одеялами. Попросила ключи на
всякий случай, вдруг пожар... Соседский мальчик сделал мне бесценный и самый
лучший подарок в моей жизни. Он подзарядил мой телефон и провел меня в центр,
чтобы связаться с близкими. Я звонила и кричала в трубку, что МЫ ЖИВЫ!!! 

Возвращение домой. Мы шли, потом бежали, потом падали на землю... Мы
возвращались в ад. Я не раз ловила себя на мысли, что если бы не дети, я не
вернулась. Я жила бы на улице, но только бы не возвращаться в этот ад... 

В квартиру знакомого с 6 этажа влетел снаряд, вынесло ударной волной двери,
ключи теперь не нужны...

День 15-й. Я разрешила ходить в обуви в квартире, все равно уже нечем было мыть
полы и не было смысла подметать, осколки летели в окна постоянно. Стены дома
тряслись от прилетов, в панике мы не успели собраться и бежали с дома. Три
импровизированных бомбоубежища отказались нас принять, мест нет, пошли в
Драмтеатр. Там было очень холодно и голодно. Ночь на картонках недалеко от
входных дверей.

День 15-й Нас приняли родственники. Я всегда чувствовала себя в безопасности с
ними. Тогда казалось, что теперь все будет хорошо и мы все преодолеем.

День 16-й Рано утром попытка вернуться домой, чтобы взять оставшиеся продукты.
Дойти не смогла всего лишь метров 200, попала под обстрел, ползти и выжить... 

Оказывается люди с села привозят на рынок молочку, купили творог и масло.
Творог прокисший, но собачка ест с удовольствием. Ночью не выдержала нервная
система, орали друг на друга, потом разрыдались и до утра рыдали в обнимку.

День 17-й горит второй подъезд, идти некуда. Тушим пожар. Тягаем ведра воды со
сточной ямы, мужчины словили пожарную машину, немного помогли пожарники, но все
ещё горит. Мужчина с частного сектора несёт нам воду из своих запасов...
Сказал, что ещё потом себе наносит, главное потушить пожар. Люди возле колодца
не пустили нас, им нужна вода тоже. В три часа ночи разошлись, пожар потушен.

День 18-й Нас угостил сосед вкуснейшим супом, сваренным на утке. Он был одним
из главных тушителей пожара. Он охотник и есть запасы мяса. Вкуснейший суп,
желудок просил ещё, но стыдно было просить добавки. 

День 20-й Страшно выходить, снаряды падают рядом, летят осколки, но нужно
готовить еду. Котелок кипит, а я кручусь вокруг дерева, прислушиваясь куда
прилетит. Главное донести в квартиру и не пролить ни капли... Ноги и руки не
слушаются, идёшь на негнущихся ногах. Главное донести еду... На ужин у нас не
запаренная гречка, как обычно, у нас утка по пекински. Съели по кусочку,
вкусно... Нас угостили мандаринами.

День 21-й Утром на улице никого нет. Никто не рискует выйти и разжечь костер.
Собаку выгулять не получается, пытаемся выгулять в подъезде, но чемний хлопчик
не понимает наших просьб. Его трясет и он терпит, учили ходить в туалет только
на улице... В квартире ходим в обуви, это настораживает...

В 14.00 начинает гореть дом, тушить нечем, нужно выезжать. Страшно, но умрем в
любом случае, а так хоть попытаемся... Гараж целый, значит и машина тоже, а
значит нужно пытаться выехать.

На выезде с города стоит много российской техники, сбавляем ход, но едем... 

Первая ночь в тепле у людей с соседнего села. Слышен гул самолётов, но нас
успокаивают, это не сюда, они летят в город.

День 22-й Бердянск. Ездят машины, работают рынки и магазины. Ощущение
сюрреализма... Работает общественный транспорт, есть мобильная связь и
интернет. Можно купить еду и наесться. Развыв шаблонов.

День 25-й 16 блокпостов, 3 часа в очереди перед Запорожьем, трасса на Днепр,
съемная квартира и душ!!! Наконец-то мы все приняли душ. Мы выехали из ада и мы
выжили!

%\ii{17_03_2023.fb.frushicheva_evdokia.mariupol.1.segodnja_rovno_god_kak_my_vyehali_iz_ada.cmt}
