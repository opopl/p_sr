% vim: keymap=russian-jcukenwin
%%beginhead 
 
%%file 09_01_2022.fb.gorovyj_ruslan.1.kiev_saksaganskogo.cmt
%%parent 09_01_2022.fb.gorovyj_ruslan.1.kiev_saksaganskogo
 
%%url 
 
%%author_id 
%%date 
 
%%tags 
%%title 
 
%%endhead 
\zzSecCmt

\begin{itemize} % {
\iusr{Александр Иванов}
Як у тому анекдоті: \enquote{Людина, яка може і не хоче, називається падлюкою.}

\iusr{Павло Редзель}
Хочеться вирвати язика особі за \enquote{пасіба}, \enquote{почьом},
\enquote{сімчясов}... це є гірше, аніж російська мова - бо то є клятий суржик.

\begin{itemize} % {
\iusr{Alex Grudiy}
\textbf{Павло Редзель} я б суржик вивчив тіки за то, шо ним розмовля Подерв'янського ський

\iusr{Юлія Васильєва}
\textbf{Павло Редзель} це жива мова, вона дає шанс українській мові вижити. І це прекрасно. Звісно чиста українська мова найкраща, але перехідний період без суржика годі й уявити

\iusr{Ruslan Gorovyi}
\textbf{Павло Редзель} не треьа нікому нічого виривати. Це тупік. Суржик теж потрібен

\iusr{Тетяна Бурлак}
\textbf{Павло Редзель} краще суржик, ніж язік.
\end{itemize} % }

\iusr{Ірина Коваленко}

Я також ходю в магазин, де є єдиний продавець, який говоре українською. Ах, як
же він говоре!!! Ну пісня!!!!!

І колись питали, чого з черги іду виключно до нього, я й сказала, шо єдина
притомна людина з людською мовою. Зависли. Тепер жду, коли просвітліє)


\iusr{Сергій Білоног}

У Києві так можна зробити, а в Харкові, Одесі, Дніпрі навряд.

Там майже все російською

\begin{itemize} % {
\iusr{Ирина Чернова}
\textbf{Сергій Білоног} 

Вчора була у книжковому магазині, на касі дівчина розмовляла солов'їною дуже
гарно, десь сім із десяти людин, спілкувались мовою. Було дуже приємно. Якось
так.


\iusr{Ruslan Gorovyi}
\textbf{Сергій Білоног} в себе на районі завжди можна спробувати знайти пару місць де тебе підтримають
\end{itemize} % }

\iusr{Yaryna Kalina}

Варус \url{https://www.facebook.com/varuszap} з якогось дива запустив
російськомовного робота. \enquote{Приходите к нам исчо}.

\begin{itemize} % {
\iusr{Тетяна Чеховська-Косцова}
\textbf{Yaryna Kalina}
Ви маєте на увазі у месенджері?
В мене українською

\ifcmt
  ig https://scontent-frx5-1.xx.fbcdn.net/v/t39.30808-6/271621898_4795204877228458_3018546956736843835_n.jpg?_nc_cat=105&ccb=1-5&_nc_sid=dbeb18&_nc_ohc=uuiNP2_ctWAAX8czwFl&_nc_ht=scontent-frx5-1.xx&oh=00_AT8iPOQfwm30tSRY3jtidKyLL47FqXD--fpFWlIFShkKyQ&oe=61E05298
  @width 0.2
\fi

\iusr{Yaryna Kalina}
\textbf{Тетяна Чеховська-Косцова} ні, одна з кас самообслуговування сьогодні заговорила російською

\iusr{Тетяна Чеховська-Косцова}
\textbf{Yaryna Kalina}
Ого. Дивина((

\iusr{Тетяна Бурлак}
\textbf{Yaryna Kalina} там є вибір мови. То хтось на триколор натиснув

\iusr{Yaryna Kalina}
\textbf{Тетяна Бурлак} не бачила, подивлюсь. Одна каса з трьох, обслуговувалось декілька покупців. Чого ж тоді персонал, не переключив на українську?

\iusr{Тетяна Бурлак}
\textbf{Yaryna Kalina} їм то не треба. Я таку функцію у новусі бачила. Теж зі мною паздаровалась така каса, я відразу знайшла, те там перемикач.

\end{itemize} % }

\iusr{Oksana Avramova}

У мене з початком року закінчилися пільги для рузькощелепних. Просто кажу їм,
що не розумію їх, переходять на рідненьку

\iusr{В'ячеслав Маноха}
і руханка до того ж

\iusr{Лара Руснак}
Із задоволенням роблю те саме  @igg{fbicon.face.happy.two.hands} 

\iusr{Роман Дашко}
+++

\iusr{Svitlana Shabarova}
У нас на районі чомусь тільки в аптеках говорять українською

\iusr{Ruslan Gorovyi}
\textbf{Svitlana Shabarova} вже непогано

\iusr{Лариса Гуріна}

Ми з сестрою так ходили по кирилівському базарчику кілька років тому.

- А у вас є отакий самий кухлик ( магнітик, кораблик) тільки з написом
українською?

- Нєт...

- Шкода... хотіла взяти штук 30 (20, 50) друзям на сувеніри.

- Так бєрітє, какая разніца...

- Ну що ви, різниця величезна... «Кірілавка» - це просто непристойно, мене ніхто
не зрозуміє.

\end{itemize} % }
