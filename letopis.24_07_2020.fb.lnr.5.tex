% vim: keymap=russian-jcukenwin
%%beginhead 
 
%%file 24_07_2020.fb.lnr.5
%%parent 24_07_2020
 
%%endhead 

\subsection{Украинцам придется жевать технический картофель со всех уголков мира}
\url{https://www.facebook.com/groups/LNRGUMO/permalink/2872267849551437/}

\index{Авторы! Игорь Петров}

Украинские домохозяйства все меньше выращивают картошку, а фермеры не могут
нарастить объемы ее производства, поэтому в стране картофеля уже не хватает.

Украина начала закупать за рубежом картошку технических сортов, и ее импорт
будет расти.

Об этом на пресс-конференции в Киеве заявил председатель Ассоциации поставщиков
торговых сетей Алексей Дорошенко, передает корреспондент «ПолитНавигатора».

«Кто для нас выращивает овощи? В основном --- фермеры, картошку --- в основном
домашние хозяйства. Они с каждым годом все меньше могут высаживать и собирать.
По луку мы в этом году, возможно, снова превратимся из импортера в экспортера,
а по картошке позитивных прогнозов пока нет.

Промышленных производителей картошки всего 5\%, 95\% производителей --- огороды.
Люди выращивали картошку вне зависимости от ее себестоимости. Сейчас они не
могут конкурировать с фермерами, все больше ее выращивается промышленно и при
отсутствии кредитов, я не знаю сколько лет понадобится, чтобы фермеры
выращивали всю картошку, в которой есть необходимость.

Поэтому для нас реальностью будет картошка технических сортов из всех стран
мира», --- резюмировал Алексей Дорошенко.

Игорь Петров
  
