% vim: keymap=russian-jcukenwin
%%beginhead 
 
%%file 27_03_2018.stz.news.ua.mrpl_city.1.najstarishyj_kinoteatr
%%parent 27_03_2018
 
%%url https://mrpl.city/blogs/view/najstarishij-kinoteatr-mariupolya
 
%%author_id demidko_olga.mariupol,news.ua.mrpl_city
%%date 
 
%%tags 
%%title Найстаріший кінотеатр Маріуполя
 
%%endhead 
 
\subsection{Найстаріший кінотеатр Маріуполя}
\label{sec:27_03_2018.stz.news.ua.mrpl_city.1.najstarishyj_kinoteatr}
 
\Purl{https://mrpl.city/blogs/view/najstarishij-kinoteatr-mariupolya}
\ifcmt
 author_begin
   author_id demidko_olga.mariupol,news.ua.mrpl_city
 author_end
\fi

\ii{27_03_2018.stz.news.ua.mrpl_city.1.najstarishyj_kinoteatr.pic.1}

Як не дивно, найстаріший кінотеатр Маріуполя \enquote{Победа} пережив події
Громадянської та Другої світової війни, першим отримав найсучасніше
обладнання, але сьогодні ледве зводить кінці з кінцями. Які таємниці зберігає
історія кінотеатру і чи можливе його збереження, спробуємо розібратися...

\ii{27_03_2018.stz.news.ua.mrpl_city.1.najstarishyj_kinoteatr.pic.2}

Точна дата, коли вперше відкрив двері глядачам кінотеатр \enquote{Победа}, загублена,
але світлини та листи доводять, що кінотеатр працював ще в довоєнний час. За
версією маріупольських істориків, кінотеатр було відкрито в період 1908–1910
років місцевим підприємцем Адабашевим, він називався \enquote{Илюзион}.

У цілому на початку ХХ століття кінематограф в Маріуполі користувався великою
популярністю. Драми, комедії та трагедії, кінохроніка демонструвалися в театрі
І. І. Уварова, цирку братів Яковенків, в концертній залі \enquote{Континенталь}
і в нових електротеатрах \enquote{ХХ век}, \enquote{Иллюзион},  \enquote{Чары},
\enquote{Одеон}, \enquote{Гигант}, \enquote{Колизей}, \enquote{Ампир},
\enquote{Мишель}. У заводських селищах працював кінотеатр \enquote{Вечерний
отдых}. Однак після громадянської війни залишився тільки один –
\enquote{Победа}, який тоді був перейменований в \enquote{Родину}. Під час
Другої світової війни кінотеатр продовжував працювати та називався
\enquote{Soldatenkino}. У 1943 році в \enquote{Soldatenkino} сталася велика
пожежа. Будівлю відновлювали, в тому числі й німецькі військовополонені, після
того як місто звільнили радянські війська. Після відновлення кінотеатр відкрив
свої двері для глядачів у 1946 році під назвою \enquote{Победа}. З того моменту
по сьогоднішній день кінотеатр радує містян кінопрем'єрами.

\ii{27_03_2018.stz.news.ua.mrpl_city.1.najstarishyj_kinoteatr.pic.3}

Першим директором кінотеатру стала Дар'я Привалова. Дотепер її з любов'ю і
повагою згадують у кінотеатрі. 37 років свого життя (з 1946 по 1983 рік) ця
людина присвятила себе кінотеатру. Саме під її керівництвом кінотеатр став
найкращим у місті й заклав традиції, багато з яких дотримуються й донині.

У 70-80-х роках в СРСР доходи від валового збору кіномережі займали в бюджеті
країни третє місце, поступаючись тільки продажу лікеро-горілчаним товарам і
тютюну. Кінотеатр \enquote{Победа}, за словами колишнього директора Галини Іллівни
Бургу, не відставав: збори в ті часи становили мільйони рублів при копійчаних
цінах на квиток.

\ii{27_03_2018.stz.news.ua.mrpl_city.1.najstarishyj_kinoteatr.pic.4}

З приходом в 1988 році Галини Бургу в кінотеатрі почалися перебудови, які в
першу чергу відбилися на зовнішньому вигляді. Будівля кінотеатру з 60-х рр.
була облицьована плиткою і, по суті, нічим не відрізнялося від інших
кінотеатрів радянської епохи. Більшість вікон були закладені цеглою, в
кабінетах було темно і незатишно. У ході ремонту було багато зроблено для
повернення кінотеатру колишнього вигляду, переобладнані кінозали. Реконструкція
велася з перемінним успіхом. Нарешті, в 1992 р. ремонт був закінчений і
оновлений кінотеатр знову відкрив свої двері перед глядачами. Були встановлені
великий екран, сучасні крісла. Стіни облицьовані звукопоглинальними панелями,
без яких якісне сприйняття звуку було б неможливим. Також було піднято підлогу
кінозалу, що, на жаль, призвело до скорочення місць на 200 крісел.

\ii{27_03_2018.stz.news.ua.mrpl_city.1.najstarishyj_kinoteatr.pic.5}

У квітні 2004 р. було відкрито новий, сучасний кінозал, який відповідав кращим
світовим стандартам.

Дуже часто в кінотеатрі проходили фестивалі, огляди, зустрічі з акторами, а
також творчими групами. Варто відзначити, що традицію проведення фестивалів
кінотеатр зберіг і донині. Проводяться фестивалі дитячого кіно, тематичні
фестивалі французького, австрійського, німецького, італійського, польського
кіно, фестивалі короткометражного кіно.

Традиційним стало проведення нічних показів кращих прем'єр\hyp{}них фільмів. Влітку
2010 року в кінотеатрі \enquote{Победа} було встановлено найсучасніше обладнання для
перегляду фільмів у форматі 3D. Встановлене обладнання 2010 року,
укомплектовано одним з найдорожчих комплектацій 3D обладнання в Україні для
показу 3D Dolby Digital кіно. Це перший кінотеатр Маріуполя, який показав
глядачам фільм в 3D real форматі на найбільшому екрані в 16 метрів по ширині, в
найбільшому залі на 295 місць, де яскравість і чіткість об'ємної картинки
забезпечує проектор CHRESTIE CP2230 в комплекті з найпотужнішою лампою OSRAM на
6000 Вт. На сьогодні 3D кіно в кінотеатрі \enquote{Победа} демонструється в двох залах.

%\ii{27_03_2018.stz.news.ua.mrpl_city.1.najstarishyj_kinoteatr.pic.6}
%\ii{27_03_2018.stz.news.ua.mrpl_city.1.najstarishyj_kinoteatr.pic.7}

\ifcmt
  tab_begin cols=2,no_fig,center,separate,no_numbering

  pic https://mrpl.city/uploads/posts/redactor/fzcjupzsanibmdu5.jpg
  pic https://mrpl.city/uploads/posts/redactor/tfnkf3kt7fnkbuce.jpg

  tab_end
\fi

Сьогодні \href{https://mrpl.city/news/view/v-mariupole-prodayut-pobedu}{%
найстаріший кінотеатр Маріуполя виставлений на продаж}\footnote{%
В Мариуполе продают \enquote{Победу}, Анастасія Селітріннікова, mrpl.city, 31.01.2018, %
\url{https://mrpl.city/news/view/v-mariupole-prodayut-pobedu}%
}. Причина –
нестача фінансування для утримання будівлі. Не\hyp{}зважаючи на труднощі, кінотеатр
продовжує організовувати безкоштовні покази для дітей з інтернатів або
військовослужбовців. Як і раніше, діють знижки для пенсіонерів, студентів і
школярів. Кінотеатр знаходиться в приватній власності, тому виправити ситуацію
або якось вплинути немає можливості. У кінотеатрі до цього часу зберігається
один з останніх представників плівкових кінопроекторів XX ст., багато світлин,
що висвітлюють діяльність кінотеатру: афіші, світлини виступів акторів, серед
них фото таких яскравих акторів, як Євген Леонов і Леонід Биков.

\ii{27_03_2018.stz.news.ua.mrpl_city.1.najstarishyj_kinoteatr.pic.8}
\ii{27_03_2018.stz.news.ua.mrpl_city.1.najstarishyj_kinoteatr.pic.9}

Варто зазначити, що наразі в Маріуполі діє всього три кінотеатри: \enquote{Мультиплекс}
в ТРЦ \enquote{ПортCity}, \enquote{Савона} на проспекті Будівельників і \enquote{Победа}, який, маючи
найдовшу історію, потрапив в зону ризику. Залишається лише сподіватися, що
кінотеатр, який вистояв декілька воєн, зможе впоратися і з кризовою економічною
ситуацією.

\ifcmt
  ig https://i2.paste.pics/PL9MI.png?trs=1142e84a8812893e619f828af22a1d084584f26ffb97dd2bb11c85495ee994c5
  @wrap center
  @width 0.9
\fi
