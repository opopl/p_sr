% vim: keymap=russian-jcukenwin
%%beginhead 
 
%%file 21_02_2023.stz.news.ua.donbas24.1.filarmonia_neskorenyh_mrpl_renesans_koncert_kyiv.txt
%%parent 21_02_2023.stz.news.ua.donbas24.1.filarmonia_neskorenyh_mrpl_renesans_koncert_kyiv
 
%%url 
 
%%author_id 
%%date 
 
%%tags 
%%title 
 
%%endhead 

Ольга Демідко (Маріуполь)
donbas24.news
«Філармонія нескорених» — Маріупольський «Ренесанс» виступив з концертом у Києві
21_02_2023.olga_demidko.donbas24.filarmonia_neskorenyh_mrpl_renesans_koncert_kyiv
Маріуполь,Україна,Мариуполь,Украина,Mariupol,Ukraine,Київ,Киев,Kyiv,Kiev,Concert,Music,Концерт,Музика,Музыка,date.21_02_2023

«Філармонія нескорених» — Маріупольський «Ренесанс» виступив з концертом у Києві (ВІДЕО)

20 лютого у День вшанування пам’яті Героїв Небесної Сотні у Національній
філармонії України відбувся концерт Камерного оркестру «РЕНЕСАНС»
Маріупольської камерної філармонії на чолі з Василем Крячком — Заслуженим
діячем мистецтв України, талановитим диригентом та директором Маріупольської
філармонії. Концерт став частиною авторського проєкту Національної філармонії
України «Філармонія нескорених», який спрямований на підтримку музикантів з
окупованих територій.

Читайте також: Міцніші за залізо: як ЗСУ героїчно боронять Донеччину в умовах
зими (ФОТО)

Після довгої перерви Маріупольський камерний оркестр «Ренесанс» вперше виступив
з оновленим складом. Директорка Департаменту культурно-громадського розвитку
Маріупольської міської ради Діана Трима наголосила, що це дуже важливий крок,
який потрібен для збереження Маріупольської філармонії в Україні.

«У січні було оголошено набір до складу Маріупольської камерної філармонії, вже
20 лютого на сцені Національної філармонії України можна було побачити всіх
нових учасників, які пройшли конкурс. Звісно, серед учасників є і маріупольці.
Але, на жаль, не всі змогли долучитися. Насправді, це єдина філармонія в
Україні, яка фінансувалася містом, вона досить молода, але вже має свою
історію. Для нас дуже важливо, щоб Маріуполь, попри всі складнощі та низку
проблем, продовжував «звучати», — розповіла Діана Трима.

Відомо, що директор Маріупольської філармонії Василь Крячок у березні 2022 року
перетворив приміщення філармонії на прихисток. Він щодня ходив під обстрілами,
адже знав, що там чекають люди, яким потрібна допомога. Будь-яку гуманітарку
від окупантів Василь Михайлович відмовлявся отримувати. Диригент бачив,
наскільки зруйноване місто, його намагалися підштовхнути до співпраці, але він
не зрадив своїх цінностей. 20 лютого у Києві після окупації Маріуполя відбувся
перший концерт Камерного оркестру «РЕНЕСАНС» Маріупольської камерної
філармонії.

«Я дуже сподіваюся, що і в Києві ми знайдемо свого слухача. Дякуємо кожному,
хто прийшов нас підтримати у такий непростий час», — підкреслив Василь Крячок.

Читайте також: Постійні обстріли Часового Яру — як змінилося місто за рік війни
(ФОТО)

До концерту також долучився чудовий музикант, блискучий скрипаль, Заслужений
артист України Назарій Пилатюк, який вразив своєю винятковою майстерністю.
Перший концерт провели в День пам’яті Героїв Небесної Сотні. Загиблих вшанували
хвилиною мовчання. Після цього пролунав саундтрек до одного з найбільш
драматичнх фільмів про голокост — «Список Шиндлера». У репертуарі «Ренесансу» є
і пісні гурту «Океан Ельзи» у класичній варіації. Загалом концертна програма
складалася з відомих творів Дж. Перголезі, Т. Віталі, Б. Фроляк, А. Вівальді,
Є. Петриченка.

Читайте також: Незламний Покровськ: як змінилося місто за рік війни (ФОТО)

Глядачі у залі не стримували своїх емоцій від концерту та аплодували за кожен
виступ.  «Браво музикантам, браво маестро, те що ви робите, дуже важлива
справа. У вашому мистецтві продовжує жити Маріуполь», — наголосив маріуполець
Максим Миргородов.

«Для маріупольців — це справжнє свято. Концерт вийшов незабутній. Я дуже вірю,
що вже незабаром ми зможемо почути цей талановитий оркестр і в нашому
звільненому Маріуполі» , — зазначила Катерина Осипенко.

Читайте також: Для маріупольців до Дня закоханих провели концерт у Києві
(ВІДЕО)

Наразі Маріупольський камерний оркестр «Ренесанс» виступатиме у Києві, але вже
найближчим часом представить свою програму і в Європі.

«Музика — це дуже гарний бартер для Європи, адже немає мовного бар'єру. Саме
тому зараз у розробці окремі проєкти, які дозволять представити Маріуполь і за
кордоном», — поділилася планами директорка Департаменту культурно-громадського
розвитку Маріупольської міської ради Діана Трима.

Раніше Донбас24 розповідав, про Міжнародний день рідної мови.

Ще більше новин та найактуальніша інформація про Донецьку та Луганську області
в нашому телеграм-каналі Донбас24.

Фото: з відкритих джерел
