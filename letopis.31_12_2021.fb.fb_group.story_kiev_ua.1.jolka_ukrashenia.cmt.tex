% vim: keymap=russian-jcukenwin
%%beginhead 
 
%%file 31_12_2021.fb.fb_group.story_kiev_ua.1.jolka_ukrashenia.cmt
%%parent 31_12_2021.fb.fb_group.story_kiev_ua.1.jolka_ukrashenia
 
%%url 
 
%%author_id 
%%date 
 
%%tags 
%%title 
 
%%endhead 
\zzSecCmt

\begin{itemize} % {
\iusr{Книжный магазин \enquote{КИ}}
Приобрести книгу можно в нашем магазине, перейдя по ссылке

\href{https://story.kiev.ua/product/ded-moroz-i-snegurochka-ukrainskie-arteli-i-fabriki-igrushek-1939-1969}{%
«Дед Мороз и Снегурочка. Украинские артели и фабрики игрушек 1939-1969», story.kiev.ua%
}

\iusr{Книжный магазин \enquote{КИ}}

\href{https://story.kiev.ua/product/did-moroz-i-sniguronka}{%
Дід Мороз і Снігуронька, story.kiev.ua%
}

\iusr{Ирина Татарчук}
1965 год Наш Дед Мороз. (изготовлен из ваты).

\ifcmt
  ig https://scontent-frx5-1.xx.fbcdn.net/v/t39.30808-6/271130831_2678041192491415_8370820137427806325_n.jpg?_nc_cat=110&ccb=1-5&_nc_sid=dbeb18&_nc_ohc=5Hhk-LhlYUgAX9M2XrV&_nc_ht=scontent-frx5-1.xx&oh=00_AT8Vs4zQAaSvjXLDoRSNV1EqJ3XZEUXQOAfK6rlxjt0y8w&oe=61D681F7
  @width 0.3
\fi

\begin{itemize} % {
\iusr{Виктория Чекалкина}
\textbf{Ирина Татарчук} это «селянин» был с конфетой

\iusr{Ирина Татарчук}
\textbf{Виктория Чекалкина} Ну этого я не знаю ... Это ещё моих родителей !
\end{itemize} % }

\iusr{Лариса Волошина}
Точно не помню, скорее всего дедушка 1965 года. Потерял посох и пояс, но жив.

\ifcmt
  ig https://scontent-frt3-1.xx.fbcdn.net/v/t39.30808-6/270833637_2653027008326431_6686401651507316465_n.jpg?_nc_cat=104&ccb=1-5&_nc_sid=dbeb18&_nc_ohc=f8By95Ld50AAX-CM8tF&_nc_ht=scontent-frt3-1.xx&oh=00_AT_4Sf0JGRQPSSx_kvaUT0LxZ_-_X--tWDXG_ufnDUeqvA&oe=61D6BC8B
  @width 0.3
\fi

\begin{itemize} % {
\iusr{Арт Юрковская}
Какое красивое лицо у него!

\iusr{Лариса Волошина}
Да, правда, лицо у него очень выразительно.

\iusr{Виктория Чекалкина}
\textbf{Лариса Волошина} чудесный большой дедушка Киевской Фабрики игрушек «Победа». Настоящий раритет!
\end{itemize} % }

\iusr{Marina Kovalenko}
Наш не дожил. Я хотела посмотреть что у него под шубой)

\iusr{Nadia Matura}

Ёлочка из крашеных куриных перьев есть в киевском музее игрушки. Только боюсь
такую ёлочку нельзя ставить в квартире с котами, они ее растерзают

\iusr{Виктория Чекалкина}
\textbf{Nadia Matura} да, елочка там есть.

\iusr{Ольга Балаба}

\ifcmt
  ig https://i2.paste.pics/03f95270f411571c0e71d5edaba8af2f.png
	@name scr.jolka
  @width 0.2
\fi

\iusr{Арт Юрковская}

Наверное этих, артельных, кукол осталось небольшое количество! Это моя страсть и
сама начала их делать. Никакая штамповка не заменит их. Спасибо, что вы их
собираете, храните и написали книгу!


\iusr{Виктория Чекалкина}
\textbf{Арт Юрковская} благодарю Вас! Уже три книги - две на русском и одна на украинском языках

\iusr{Надежда Кулиш}

\ifcmt
  ig https://i2.paste.pics/8580c4124c43dbf359a1fc62a7905541.png
  @width 0.2
\fi

\iusr{Елена Чичик}

\ifcmt
  ig https://scontent-frx5-2.xx.fbcdn.net/v/t39.1997-6/s168x128/64338789_1420704821405917_4657022271569788928_n.png?_nc_cat=1&ccb=1-5&_nc_sid=ac3552&_nc_ohc=GFQkEBLVWU4AX98jzC_&_nc_ht=scontent-frx5-2.xx&oh=00_AT_Deewi4LRli-H0u0EtyDADl6FLfo1x7Ag_5K662mN9pQ&oe=61D6C712
  @width 0.1
\fi

\iusr{Виктория Чекалкина}
Вот она елочка из гусиных перьев

\ifcmt
  ig https://scontent-frt3-1.xx.fbcdn.net/v/t39.30808-6/270764817_5303221653045464_4961644549325387376_n.jpg?_nc_cat=108&ccb=1-5&_nc_sid=dbeb18&_nc_ohc=cTHylKPa4YIAX8t_xmx&_nc_oc=AQmWXMI1avZTpRhIzM2qr0KKcNWrd1dQv-NuL0gMyRMM3MwlYNC8-vsW1MB9NIPo-5c&_nc_ht=scontent-frt3-1.xx&oh=00_AT9N07BIVr36i7wJs5BC1Qh89xF44WKlMsKbVLelo5YmaQ&oe=61D65800
  @width 0.3
\fi


\iusr{Наталія Гуменюк-Гуральська}
Незнаю какого года он) может кто подскажет

\ifcmt
  ig https://scontent-frt3-1.xx.fbcdn.net/v/t39.30808-6/270651968_6943187619086094_2226336075963616667_n.jpg?_nc_cat=104&ccb=1-5&_nc_sid=dbeb18&_nc_ohc=4SzDNwZzLKYAX-0wkO5&_nc_ht=scontent-frt3-1.xx&oh=00_AT8NbbbOBfSkWVoWg6DJTC3CrU6ycOmHwFerYiMK46oLAw&oe=61D5D8C4
  @width 0.3
\fi

\begin{itemize} % {
\iusr{Виктория Чекалкина}
\textbf{Наталія Гуменюк-Гуральська} 

таких дедов по образцу автора Юркевич выпускало как минимум 5 украинских
артелей и много артелей рсфср в 1950-х годах. Этикетки обычно нет, она
крепилась к ногам и мешала устойчиво стоять под Елкой.

\iusr{Наталія Гуменюк-Гуральська}
\textbf{Виктория Чекалкина} спасибо вам
\end{itemize} % }

\iusr{Наталія Гуменюк-Гуральська}

\ifcmt
  tab_begin cols=2,no_fig
		 @resizebox 0
		 @minipage 0.6

     pic https://scontent-frx5-1.xx.fbcdn.net/v/t39.30808-6/270591470_6943194735752049_6265056940011815062_n.jpg?_nc_cat=100&ccb=1-5&_nc_sid=dbeb18&_nc_ohc=al2QuJiJqb4AX-a3SgR&_nc_ht=scontent-frx5-1.xx&oh=00_AT-l6hB5k-NjtHb3ImFk3_TbQJGpWUhn1Rad2K9hfAp9_Q&oe=61D6CBBE

		 pic https://scontent-frx5-1.xx.fbcdn.net/v/t39.30808-6/270759764_6943195309085325_5847411430725314541_n.jpg?_nc_cat=111&ccb=1-5&_nc_sid=dbeb18&_nc_ohc=V9BmF4Qs_JUAX8JDhy7&_nc_ht=scontent-frx5-1.xx&oh=00_AT-Z_Nt2vXMPDoRxiWdQ9d31XYsY5riltmlTGFScaVVRCg&oe=61D6ABC0

  tab_end
\fi

\iusr{Татьяна Ткаченко}

\ifcmt
  tab_begin cols=2,no_fig
		 @resizebox 0.6

     pic https://scontent-frx5-1.xx.fbcdn.net/v/t39.30808-6/270743586_440100971112678_105022999208083215_n.jpg?_nc_cat=110&ccb=1-5&_nc_sid=dbeb18&_nc_ohc=r1ERmie2QeYAX-AJRby&_nc_ht=scontent-frx5-1.xx&oh=00_AT98wsOUkG5gZtNiPSb8P0Od9qL-BNgiG5u4jHpVtgUrCQ&oe=61D73005

		 pic https://scontent-frt3-2.xx.fbcdn.net/v/t39.30808-6/270757835_440101017779340_1547070412822393325_n.jpg?_nc_cat=103&ccb=1-5&_nc_sid=dbeb18&_nc_ohc=_UEexUU6tscAX8d591l&_nc_ht=scontent-frt3-2.xx&oh=00_AT8hhkr603GWRVHVJLiqpHPJKGbjQOrmBSAMN0S79EejTw&oe=61D5D31C

  tab_end
\fi

\iusr{Светлана Мишина}

Дедов Морозов чем-то посыпали для блеска (может битое стекло?). Оставалось на
руках и нельзя было глаза тереть, пока не помоешь. Воспоминание детства... а так
они очень красивые были)

\begin{itemize} % {
\iusr{Олена Андурова}
\textbf{Светлана Мишина} Скорее всего, присыпали молотой слюдой.

\iusr{инна голуб}
\textbf{Светлана Мишина} посыпали бертолетовой солью... мне мама в садик так украшала наряд Снегурочки. Уже больше полвека я храню флакон из- под маминых духов, наполненный этой красивой блестящей сыпучей смесью слегка розового цвета...

\iusr{Виктория Чекалкина}
\textbf{Светлана Мишина} «елочный снег» делала артель Лабораторсиекло и в количестве 800 кг отправляла на фабрику Победа. Да, нужно было защищать глаза и вдыхать было вредно)
\end{itemize} % }

\iusr{Петр Кузьменко}

Наш Дед Мороз. Куплен в Киеве в 50 - 60е годы бабушкой и дедушкой супруги.
Всегда с нами.

\ifcmt
  ig https://scontent-frx5-1.xx.fbcdn.net/v/t39.30808-6/270756375_4885759491476296_6458550482744771875_n.jpg?_nc_cat=105&ccb=1-5&_nc_sid=dbeb18&_nc_ohc=mpUw7j3JGfgAX91nDkk&_nc_ht=scontent-frx5-1.xx&oh=00_AT-OJnMI33YEaGDTIRNFEWW66LtoXajRjevW1fyS3J_5kg&oe=61D583BE
  @width 0.2
\fi

\iusr{Helen Ferrum}

\ifcmt
  ig https://scontent-frx5-2.xx.fbcdn.net/v/t39.1997-6/s168x128/106128797_1001217123667325_8827439800803529943_n.png?_nc_cat=1&ccb=1-5&_nc_sid=ac3552&_nc_ohc=RwSC67rPn78AX_EgDKy&_nc_ht=scontent-frx5-2.xx&oh=00_AT_uMPVmUjywkNgBnXTVTZLhw_JOcKdUK8UOjY37HGbcTw&oe=61D727A6
  @width 0.1
\fi

\iusr{Инна Яковлева}
Снегурочка начала 60-х.

\ifcmt
  tab_begin cols=3,no_fig
		 @minipage 0.6

     pic https://scontent-frt3-1.xx.fbcdn.net/v/t39.30808-6/270917177_1889854657863481_1237056374314647732_n.jpg?_nc_cat=104&ccb=1-5&_nc_sid=dbeb18&_nc_ohc=GzNrf1lHLRQAX96jjzL&_nc_ht=scontent-frt3-1.xx&oh=00_AT-MWqOA9TUEUWcInKk1UzAr1BOlO5SsPdGcRRcokG06dg&oe=61D6D166

		 pic https://scontent-frt3-2.xx.fbcdn.net/v/t39.30808-6/270816356_1889854774530136_270058245691230940_n.jpg?_nc_cat=101&ccb=1-5&_nc_sid=dbeb18&_nc_ohc=VHfCETZSoPAAX9Op1Ta&_nc_ht=scontent-frt3-2.xx&oh=00_AT9_NOfAkwUsG5XsCpXfpnf6EMnfFtXNoNoyzr2czOtevQ&oe=61D59387

  tab_end
\fi

\iusr{Оксана Пасичко}
1968. Домашний, уютный, чудесный старый Дедушка Мороз! @igg{fbicon.heart.suit}

\ifcmt
  ig https://scontent-frx5-1.xx.fbcdn.net/v/t39.30808-6/270756281_3114315698790411_4179477328965980662_n.jpg?_nc_cat=111&ccb=1-5&_nc_sid=dbeb18&_nc_ohc=rQU0chx_RwEAX9V_oLD&_nc_ht=scontent-frx5-1.xx&oh=00_AT-BtQdTlwUxIH-zfHa0SAGpowiP8ojaEe4VWkB8XkZp_Q&oe=61D68A8F
  @width 0.2
\fi

\iusr{Елена Гнеденко}

Могу только позавидовать всем, у кого сохранились старые ёлочные украшения. У
меня было много И грушек послевоенных лет. Каждый год я пополняла их запасы. И
так было до семидесятых годов. прошлого столетия,пока в очередной раз, муж с
дочерью, наряжая ёлку, не уронили весь ящик на пол. Остались только считанные,
небьющиеся. Теперь с удовольствием слежу за рубрикой о ёлочных игрушках. А что
ещё остаётся делать!!!

\begin{itemize} % {
\iusr{Светлана Мишина}
\textbf{Елена Гнеденко} все, что осталось) Даже на ёлку не вешаю, что бы последние не разбить.

\ifcmt
  ig https://scontent-frt3-1.xx.fbcdn.net/v/t39.30808-6/270131290_1196467987427384_2161401817046944664_n.jpg?_nc_cat=107&ccb=1-5&_nc_sid=dbeb18&_nc_ohc=eq4dOui2AwgAX_Kn16_&_nc_ht=scontent-frt3-1.xx&oh=00_AT-TyA-K8Jz9mTUWcCGSGp-S-bvnIZeYvDrwh_LAGVYHBA&oe=61D5C644
  @width 0.4
\fi

\iusr{Елена Гнеденко}
\textbf{Светлана Мишина} 

Припоминаю: были и у меня такие. На прищепках удобно, но нужно было выбрать
потолще веточку, чтобы игрушка не. упала. Спасибо за напоминание. С
наступающим Новым годом.

\iusr{Галина Решетюк}
\textbf{Елена Гнеденко} А я в 80-х грохнула коробку, когда доставала с антресоли.

\iusr{Алена Жураховская}
\textbf{Галина Решетюк} дома у мамы есть

\end{itemize} % }

\iusr{Оксана Пасичко}

Автору спасибо! Но автор ошибается: фабрика \enquote{Победа} производила новогодние
игрушки из ваты и папье-маше в 1968-м году тоже. Этикетка моего чудесного Деда
Мороза:

\ifcmt
  ig https://scontent-frx5-2.xx.fbcdn.net/v/t39.30808-6/271024421_3114321062123208_7807369801990949897_n.jpg?_nc_cat=109&ccb=1-5&_nc_sid=dbeb18&_nc_ohc=PLE2h08a9iYAX-90vSJ&_nc_ht=scontent-frx5-2.xx&oh=00_AT8gnJM909JS_2jLVTd-dGmSi3v2ILEDHyfLNHC9KwOOYg&oe=61D613BF
  @width 0.3
\fi

\begin{itemize} % {
\iusr{Виктория Чекалкина}
\textbf{Оксана Пасичко} 

основываясь на архивных документах - выпуск дедов морозов и сненурочек из Ваты
с лицами Папье маше ручной работы перекатился с 3 квартала 1967 года. Далее был
пластик. Об этом говорится и в газетах того времени. Очень интересно, что у Вас
есть такая фигура, датированная позднее. Можете поделиться как она выглядит?


\iusr{Оксана Пасичко}
\textbf{Виктория Чекалкина} Я поделилась! Фото! В комментарии выше!  @igg{fbicon.face.happy.two.hands} 
\end{itemize} % }

\iusr{Margarita Kaminsky}
Все наши ёлочные игрушки остались в Киеве.

\begin{itemize} % {
\iusr{Victoria Novikov}
\textbf{Margarita Kaminsky} 

Как мы могли их оставить?! Чем мы думали? Но 30 лет назад мне казалось, что
это ерунда и старьё. Очень жалею, что оставили ёлочные игрушки и выбросили
открытки. Но мы ехали с двумя чемоданами на человека и, конечно, они бы не
поместились. Память осталась. Это тоже немало.

\iusr{Margarita Kaminsky}
\textbf{Victoria Novikov}
И мы ехали с двумя чемоданами на человека.
С меня на таможне сняли и взвесили золотые серёжки без камней. Они весили полтора грамма.

\iusr{Victoria Novikov}
\textbf{Margarita Kaminsky} 

А у меня появились первые золотые серёжки именно благодаря отъезду. У мамы
расшатались и выпали многие зубы( к счастью мне здесь давно сделали такую
пластмасску, чтоб надевать ночью на зубы. Для нЭрвных, которые скрежещут
зубами  @igg{fbicon.laugh.rolling.floor} ). И перед отъездом она удалила оставшиеся 6 зубов, чтоб не иметь с
ними проблему в Америке, и сделала челюсти. А на 2 зубах у неё были золотые
коронки. И вот мне сделали сережки. И у меня их точно также взвешивали на
таможне.  @igg{fbicon.face.tears.of.joy}  @igg{fbicon.laugh.rolling.floor} 

\end{itemize} % }

\iusr{Таня Сидорова}

З прийдешнімі святами. Нажаль ні дід мороз, ні снігуронька не збереглися. А ваша
чудова колекція іграшок, в культурному центрі, повернула в дитинство. Дякую, хай
щастить.


\iusr{Виктория Чекалкина}
\textbf{Таня Сидорова} Дякую Вам!

\iusr{Юрий Мундиров}

\ifcmt
  ig https://scontent-frx5-1.xx.fbcdn.net/v/t39.30808-6/270828780_3007628256218946_8875982637400156183_n.jpg?_nc_cat=100&ccb=1-5&_nc_sid=dbeb18&_nc_ohc=dtaM1iLQGLUAX_W94t3&_nc_ht=scontent-frx5-1.xx&oh=00_AT9-5aLOOZYMRu10Z9WqFuZejyWqX5SsexxD-gd586X2Kw&oe=61D6D55D
  @width 0.3
\fi

\begin{itemize} % {
\iusr{Наташа Шведова}
\textbf{Юрий Мундиров} У меня были такие снежинки!
\end{itemize} % }

\iusr{Юрий Мундиров}

\ifcmt
  ig https://scontent-frt3-1.xx.fbcdn.net/v/t39.30808-6/270816354_3007631312885307_2761451819463828190_n.jpg?_nc_cat=106&ccb=1-5&_nc_sid=dbeb18&_nc_ohc=u-8l5lv107UAX8xSAvC&_nc_ht=scontent-frt3-1.xx&oh=00_AT81NiQCDWVtuWQ9HdWdOy2FD1m6tafZpPxO2W85f-4Cew&oe=61D6D61A
  @width 0.2
\fi

\iusr{Оксана Венгерова}


\ifcmt
  tab_begin cols=4,no_fig

     pic https://i2.paste.pics/dea2ee92bb0fc476cbb3547eb5970aba.png
		 pic https://i2.paste.pics/07a728d941226d6b6f0a015ff5cc2a43.png
		 pic https://i2.paste.pics/e63335c8c9971d98a9607aecea4a9096.png
		 pic https://i2.paste.pics/eff5fb502c3b13db3d440b29279b7437.png

  tab_end
\fi

\iusr{Лариса Мысник}

\ifcmt
  ig https://scontent-frx5-1.xx.fbcdn.net/v/t39.30808-6/270779269_1555990061440040_940482661638181236_n.jpg?_nc_cat=111&ccb=1-5&_nc_sid=dbeb18&_nc_ohc=mZLz1AZVDMAAX9--9lE&_nc_ht=scontent-frx5-1.xx&oh=00_AT9W8OWg5TYFABsvujnnNtmwmDbYCJOGKLuRBCQOn68GTw&oe=61D5FB5D
  @width 0.2
\fi

\iusr{Светлана Лещенко}

\ifcmt
  ig https://scontent-frt3-1.xx.fbcdn.net/v/t39.30808-6/270373257_4745187665556998_2724228903586923099_n.jpg?_nc_cat=106&ccb=1-5&_nc_sid=dbeb18&_nc_ohc=C781JXfL-5YAX8zLmoi&_nc_ht=scontent-frt3-1.xx&oh=00_AT_VCx76x3Bqp7h6u040Znp4VgN-_T--umROV5HbEhuIUg&oe=61D59A05
  @width 0.2
\fi

\iusr{Елена Рачина}
У меня в детстве были такие дед мороз и снегурочка они мне достались от бабушки.

\iusr{Муся Владимировна}

\ifcmt
  ig https://scontent-frt3-1.xx.fbcdn.net/v/t39.30808-6/271186045_409382077540325_3339666790044124595_n.jpg?_nc_cat=102&ccb=1-5&_nc_sid=dbeb18&_nc_ohc=ruYK8z6K55QAX_Gqr1F&_nc_ht=scontent-frt3-1.xx&oh=00_AT8TsmD3g3YC-0cT8I7FYm8hepLvMFrJBDL526R6VmKYFQ&oe=61D60755
  @width 0.3
\fi

\iusr{Defektolog Logoped}
Я сдала в музей деда мороза, ватные игрушки, часть игрушек 60-70г.

\iusr{Defektolog Logoped}

\ifcmt
  ig https://scontent-frt3-1.xx.fbcdn.net/v/t39.30808-6/271145148_980295376237125_8495899841859398390_n.jpg?_nc_cat=108&ccb=1-5&_nc_sid=dbeb18&_nc_ohc=yEB_6v3gi1MAX_VNoca&_nc_ht=scontent-frt3-1.xx&oh=00_AT-3d1KwEc7ZpN45g9Z53Nw0_JByBa993bSp2FpWXPIEsg&oe=61D6829B
  @width 0.3
\fi

\iusr{Галина Гурьева}
Остался ещё такой грибочек.

\iusr{Defektolog Logoped}

\ifcmt
  ig https://scontent-frt3-1.xx.fbcdn.net/v/t39.30808-6/270257297_980295496237113_2541697246995502806_n.jpg?_nc_cat=107&ccb=1-5&_nc_sid=dbeb18&_nc_ohc=cSdP3L1jml0AX9_WCXl&_nc_ht=scontent-frt3-1.xx&oh=00_AT9u6qUjqbPOZXK6qJkfyCxXacndPnszAUngJ8hPrh9qIQ&oe=61D686CC
  @width 0.2
\fi

\iusr{Larisa Peleshuk}
И у нас такой дедушка был

\ifcmt
  ig https://scontent-frx5-2.xx.fbcdn.net/v/t39.1997-6/s168x128/64338789_1420704821405917_4657022271569788928_n.png?_nc_cat=1&ccb=1-5&_nc_sid=ac3552&_nc_ohc=GFQkEBLVWU4AX98jzC_&_nc_ht=scontent-frx5-2.xx&oh=00_AT_Deewi4LRli-H0u0EtyDADl6FLfo1x7Ag_5K662mN9pQ&oe=61D6C712
  @width 0.1
\fi

\iusr{Tamara Kom}
у меня много игрушек елочнім мои ровесники - 70 лет

\iusr{Галина Гурьева}

Волшебный домик моего детства! Мне казалось, что ночью там зажигается свет. Он
совсем старенький, но и в этом году висит на елке.

\ifcmt
  ig https://scontent-frt3-2.xx.fbcdn.net/v/t39.30808-6/271192140_447380853522451_8590539770146930848_n.jpg?_nc_cat=103&ccb=1-5&_nc_sid=dbeb18&_nc_ohc=y3ykTbuaVbAAX9SySEm&_nc_ht=scontent-frt3-2.xx&oh=00_AT9XLPjJZ0NF1pHkivc-m_qPNgbZGRrVACTuupZLOFYKBw&oe=61D5ECA9
  @width 0.2
\fi

\iusr{Kaurkovska Veronika}
Интересно, в чем тонкость технологии производства ватных игрушек с лицами из папье-маше.?

\iusr{Olga Kynal}


\ifcmt
  tab_begin cols=3,no_fig,center

     pic https://scontent-frx5-1.xx.fbcdn.net/v/t39.30808-6/271003953_4993214707389602_4682033624845744350_n.jpg?_nc_cat=100&ccb=1-5&_nc_sid=dbeb18&_nc_ohc=BQqPadrCW1QAX9HRQ0v&_nc_ht=scontent-frx5-1.xx&oh=00_AT-5z4yMvuG0XLjxU4bUV0HUy76pWgrcvMF5oCPXIC-5bQ&oe=61D60822

		 pic https://scontent-frx5-1.xx.fbcdn.net/v/t39.30808-6/270926228_4993215344056205_1180752877521042869_n.jpg?_nc_cat=110&ccb=1-5&_nc_sid=dbeb18&_nc_ohc=MBXpXZkfb_IAX8oCL8N&_nc_ht=scontent-frx5-1.xx&oh=00_AT_hlUW-QFy3fXjZ9FoJKeAxhfyOs2oPMeUy_yNGCUAR0g&oe=61D715B5

		 pic https://scontent-frx5-1.xx.fbcdn.net/v/t39.30808-6/270937987_4993215964056143_6962080463053794670_n.jpg?_nc_cat=111&ccb=1-5&_nc_sid=dbeb18&_nc_ohc=Uhimj5A_2DgAX_8Blwr&_nc_ht=scontent-frx5-1.xx&oh=00_AT815o7hm_yMuHJBfZG8bSsvtJ3k3cLANXF5roApX0Ub0w&oe=61D5D455

  tab_end
\fi

\iusr{Olga Kynal}


\ifcmt
  tab_begin cols=3,no_fig,center
		 pic https://scontent-frt3-1.xx.fbcdn.net/v/t39.30808-6/270293675_4993216594056080_4890233135189377540_n.jpg?_nc_cat=108&ccb=1-5&_nc_sid=dbeb18&_nc_ohc=yEs9zZKFdjEAX9yyoxe&_nc_ht=scontent-frt3-1.xx&oh=00_AT-_stdpDWOzDRRHH-EbSwFuk_cOm0guQSFNnvkF4fEuEQ&oe=61D7085D

		 pic https://scontent-frt3-2.xx.fbcdn.net/v/t39.30808-6/271006457_4993217444055995_3604716061157487682_n.jpg?_nc_cat=103&ccb=1-5&_nc_sid=dbeb18&_nc_ohc=DaYuLCS6SvsAX9D_srV&_nc_ht=scontent-frt3-2.xx&oh=00_AT_iOotZA4rQ2bg0g4wY8qPWPBK4Ox4JPyi2g6nDXrBX7g&oe=61D6FDF6

		 pic https://scontent-frt3-1.xx.fbcdn.net/v/t39.30808-6/270738375_4993218207389252_5351205096058554378_n.jpg?_nc_cat=107&ccb=1-5&_nc_sid=dbeb18&_nc_ohc=4TNmshIDGD8AX97Y6oD&_nc_ht=scontent-frt3-1.xx&oh=00_AT_GGgfJ0frIyAXS8jnSmali0HQSPdBn2X6H-Scm3DALAg&oe=61D7520E


  tab_end
\fi

\iusr{Olga Kynal}
У мене збереглись скляні іграшки й паперові, яким більше 65 років... тільки хатинці небагато - 57...

\iusr{Julia Panchul}

В 40-х годах моя мать со мной сами делали ёлочные игрушки и деда Мороза из ваты
и сжатой бумаги, битых стекляных ёлочных игрушек, других материалов в доме.


\iusr{Lubov LV}

В этом году муж купил совсем маленькую сосну и я решила ее украсить винтажными
игрушками. Их осталось совсем мало и многие годы мы ними не пользовались. И Дед
Мороз ещё из детства))))

\ifcmt
  ig https://scontent-frt3-1.xx.fbcdn.net/v/t39.30808-6/271131627_1593786450963578_2164055511130495967_n.jpg?_nc_cat=104&ccb=1-5&_nc_sid=dbeb18&_nc_ohc=8oF_fgFgpxAAX8asEoz&_nc_ht=scontent-frt3-1.xx&oh=00_AT_jFF8FWRU4lg9GXSPdUaSMm3rSikqN8qEvJAW7Hq6nBA&oe=61D5A0F7
  @width 0.2
\fi

\begin{itemize} % {
\iusr{Наталия Малиненко}
\textbf{Lubov LV} супер елочка  @igg{fbicon.santa.claus}  Дед мороз очень коасивый

\iusr{Юрий Белоусов}
\textbf{Lubov LV} - у нас такой же дома, один в один!
\end{itemize} % }

\iusr{Nataliia Shevchenko}
Залишилося лише це яблуко, але скільки чудових спогадів....

\ifcmt
  ig https://scontent-frt3-1.xx.fbcdn.net/v/t39.30808-6/271055941_3167772810216306_6727718494978026457_n.jpg?_nc_cat=107&ccb=1-5&_nc_sid=dbeb18&_nc_ohc=ajfXEC9rPnUAX95yz-Y&_nc_ht=scontent-frt3-1.xx&oh=00_AT-pfuo47mLQNN-A6g85e8_m1w8qRdy31r7iAY5iU8H4kw&oe=61D5FB43
  @width 0.2
\fi

\iusr{Алёна Новиковп}
Я знаю людей, у которых был именно такой Дед мороз!

\iusr{Алёна Новиковп}

А у меня Дед мороз, после Кремлевской ёлки 1967 года, и до сих пор есть. И не хочу
упрёков, это Москва, Россия....... Это детство моего брата, это моё детство, а не те
твари которые забрали эти впечатления у моих внуков.... Да Я живу в
Украине, больше скажу, Я Киевлянка от рождения....


\iusr{Галина Гордеева}
Такие знакомые игрушки из далекого детства.

\iusr{Юрий Петров}
55г

\ifcmt
  ig https://scontent-frt3-1.xx.fbcdn.net/v/t39.30808-6/270077193_3037752666492761_6686616074772563665_n.jpg?_nc_cat=104&ccb=1-5&_nc_sid=dbeb18&_nc_ohc=nXA5mu-U3lwAX-t17lp&_nc_ht=scontent-frt3-1.xx&oh=00_AT-attqItKLRBdRkyCrK29Bq2cosmF5NT5-detDQ2YDE8w&oe=61D702E9
  @width 0.3
\fi


\iusr{Наталья Чубук}

\ifcmt
  ig https://scontent-frx5-1.xx.fbcdn.net/v/t39.30808-6/270181985_1922354681280805_9153140067142474835_n.jpg?_nc_cat=111&ccb=1-5&_nc_sid=dbeb18&_nc_ohc=hXKdk277wbEAX-MZdZC&_nc_ht=scontent-frx5-1.xx&oh=00_AT8DUtPtGD3bcv2jMEd-SI3x4QLUd0aLYzxZ5wosMeQ4HQ&oe=61D6C055
  @width 0.2
\fi

\iusr{Александра Коломейко}

\ifcmt
  ig https://scontent-frt3-1.xx.fbcdn.net/v/t39.30808-6/270302674_10216483495638671_8289526562431823300_n.jpg?_nc_cat=106&ccb=1-5&_nc_sid=dbeb18&_nc_ohc=9_OWzmsRYT8AX_Z0ZC9&_nc_ht=scontent-frt3-1.xx&oh=00_AT_HJR-yspfYeQw_DYqv2SVYhbbW5RLOVFO5tbtofnHI-g&oe=61D62D2C
  @width 0.2
\fi

\iusr{Светлана Шварц}

И моему Дедушке Морозу @igg{fbicon.santa.claus}  28лет @igg{fbicon.face.happy.two.hands} И ёлочка @igg{fbicon.christmas.tree}  тоже маленькая @igg{fbicon.face.smiling.eyes.smiling} И теперь моя внучка
пыталась заглянуть в мешок Деда Мороза, как когда то, много лет назад, мои
дети @igg{fbicon.face.smiling.eyes.smiling} 

\ifcmt
  ig https://scontent-frx5-1.xx.fbcdn.net/v/t39.30808-6/270103298_441782544400849_6721346075384039822_n.jpg?_nc_cat=111&ccb=1-5&_nc_sid=dbeb18&_nc_ohc=jU-7oDRFIygAX8WMLoT&_nc_ht=scontent-frx5-1.xx&oh=00_AT_ersoICHt6elVzfv-gI8hSlRyItb1ug2DGyJIhQYQpMQ&oe=61D73D31
  @width 0.2
\fi

\iusr{Светлана Радченко}

\ifcmt
  ig https://scontent-frx5-2.xx.fbcdn.net/v/t39.30808-6/271190662_1509496519424345_8881086898860656128_n.jpg?_nc_cat=109&ccb=1-5&_nc_sid=dbeb18&_nc_ohc=JZAYLjKQsTkAX-2VRzn&_nc_ht=scontent-frx5-2.xx&oh=00_AT-Uqa6Jkgaz-4VTN2HFKWIhNY2qgPzjqbcK3kBzt7QaKw&oe=61D58FFB
  @width 0.2
\fi


\iusr{Любовь Подлесная}

И у меня был такой Дед Мороз и снежинки, но к сожалению не сохранились.
Хотелось новые игрушки более современные иметь. Да, елочные игрушки меня
завораживали, особенно домик под ёлкой с лампочкой внутри и когда включали елку
в этом маленьком домике тоже зажигался свет это было волшебно.

\iusr{Елена Ковтун}

\ifcmt
  ig https://scontent-frt3-2.xx.fbcdn.net/v/t39.30808-6/271130123_4643957089013323_3962489790591148804_n.jpg?_nc_cat=101&ccb=1-5&_nc_sid=dbeb18&_nc_ohc=Q-ssbtRr01cAX91OzUw&_nc_ht=scontent-frt3-2.xx&oh=00_AT9Btf1VVDWhlKXNNLO6T7QuAI1SB-Q6mnGJmwf4NnQxqQ&oe=61D6418A
  @width 0.2
\fi


\end{itemize} % }
