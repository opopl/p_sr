% vim: keymap=russian-jcukenwin
%%beginhead 
 
%%file 13_11_2020.fb.evgeniy_maslov.1.dostoevskii_kak_umiral
%%parent 13_11_2020
 
%%url https://www.facebook.com/maslovevgeniy14/posts/1346952588979779
%%author 
%%tags 
%%title 
 
%%endhead 

\subsection{Как умирал Достоевский}
\label{sec:13_11_2020.fb.evgeniy_maslov.1.dostoevskii_kak_umiral}
\Purl{https://www.facebook.com/maslovevgeniy14/posts/1346952588979779}
\Pauthor{Ткачев, Андрей (Протоиерей)}
\Pauthor{Маслов, Евгений}

\index[writers.rus]{Достоевский, Федор Михайлович!Смерть}

\ifcmt

pic https://scontent.fiev22-1.fna.fbcdn.net/v/t1.0-9/125239366_1346952568979781_6522832848700085460_o.jpg?_nc_cat=107&ccb=2&_nc_sid=730e14&_nc_ohc=jUFiWS_52doAX_u4y7m&_nc_ht=scontent.fiev22-1.fna&oh=ff73e5a11c4cc3417a9199e8b2984405&oe=5FD29393
caption Достоевский на смертном одре

\fi

"В цикле «Смерть замечательных людей» поговорим о смерти Федора Михайловича
Достоевского. В 1881 году обрел покой этот человек. Интересно в его смерти
то, что он был один из тех немногих, кому перед смертью читали Евангелие. По
разным источникам - разные тексты. В одних книжках из серии ЖЗЛ я видел, что
супруга Анна Григорьевна читала ему «Притчу о блудном сыне». А в других
источниках - Евангелие от Марка, крещение Иисуса Христа.

Но так или иначе, он умер под звуки Евангелия, сказал жене очень важные слова:
«Анечка, я тебе даже в мыслях не изменял». К нему пришли дети под
благословение, он благословил их и отдал Богу душу, этот чахоточный бывший
каторжник, высохший, измученный, о котором критики говорили, что он в посконной
рубахе стоит на коленях перед Богом за все человечество.

Потом начались интересные вещи. Его хотели похоронить в лавре, но Митрополит
Санкт-Петербуржский и наместник лавры сказал: «С чего это какого-то писателишку
возле монахов хоронить, нечего, понимаешь!»

Но благо, Победоносцев Константин Петрович, всесильный обер-прокурор, стукнул
тощим кулачком по столу и привел всех в жуткий трепет. И Достоевского
похоронили не только в лавре, а еще и за деньги лавры. Все взяли на себя.
Потому что Победоносцев понимал, кого хороним. А митрополит тогдашний не
понимал.

Сегодня в пантеоне знаменитых людей XIX века, по дороге к лавре, с правой
стороны дорожка протоптана только к нему. Там и Баратынский лежит, и дедушка
Крылов, и Жуковский, и Чайковский, и Бородин, и Мусоргский. Однако дорожка
всегда к Достоевскому протоптана. И засыпанная цветами могила, написан эпиграф
из Евангелия от Иоанна к роману «Братья Карамазовы»: «... зерно, падши в землю,
не умрет, то останется одно».

В общем, Достоевский умер. Его положили в Святодуховскую церковь
Александро-Невской лавры, ныне не действующую, занятую Культурным Центром. И
там он лежал. К нему пришли псалтырь почитать. Лежит покойник, в головах стоит
аналой, на аналое лежит псалтырь, приходили люди и читали по кафизму. Потом
пришли другие - «я тоже хочу помолиться, и я, и я», и читают по кафизму, потом
по псалму, потом по три строчки, потом по одной строчке, потом по одному слову.

Слово прочел и в сторону, прочел и в сторону. Потому что желающих прочесть
псалтырь у гроба Достоевского оказалось несколько тысяч. Люди вдруг
почувствовали, кого же они потеряли. Приходили молиться разночинцы, студенты,
окончившие и не окончившие курс. Приходили дамы, приходили священники,
приходили офицеры, приходили интеллигенты.

Впервые робко крестились те, кто уже забыл креститься давно. Потому что
безбожие прошло по России широкой волной, многие вообще перестали ходить в
церковь. И вдруг они стали снова молиться над гробом покойника, который всю
жизнь только о том и говорил: «Веруйте, кайтесь, Христос есть, и Бог есть, и
бессмертие души есть, все есть, кайтесь, веруйте».

Когда его выносили из храма, чтобы обнести вокруг церкви под пение тропарей за
упокоенных «Волной морскою», то обнести не получилось, потому что был битком
набит людьми весь церковный двор. Нужно было гроб передавать буквально по
головам. То есть гроб шел, а люди не двигались. Не могли. Так много было
народу. В окрестных домах были открыты настежь окна, в окнах стояли люди с
зажженными свечами.

И когда его несли к месту упокоения, пели «Святый Боже, Святый Крепкий, Святый
Бессмертный, помилуй нас». Весь город пел, казалось, весь Петербург сошелся.

Таких похорон Россия не знала, кроме разве что двух. Подобного рода масштабные
похороны были у генерала Скобелева, Белого Генерала. Того самого, который бил
турков в Болгарии, и в Туркистане воевал. И потом похороны Иоанна
Кронштадского, которые тоже собрали к себе всю Россию, это было нечто
грандиозное.

Похороны праведника - это подтверждение того, что Бог есть. Похороны праведника
- это праздник. Это событие, которое  рождает в людях не страх и трепет, а
радость и умиление, и странные слезы благодарности и любви.

Вот так умер Федор Михайлович, и жизнью и смертью подтвердил весь путь своей
жизни, он вел и ведет нас, по сути тащит за руки сквозь петербуржские
подворотни, сквозь прокуренные кухни, где спорят о жизни, ведет нас к Иисусу
Христу. И смерть его была еще одним доказательством, что он не ошибся. Все, что
он делал и говорил, было правильно."

