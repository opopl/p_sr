% vim: keymap=russian-jcukenwin
%%beginhead 
 
%%file 29_12_2021.vk.pyzhova_evgenia.doneck.1.skripka_vika
%%parent 29_12_2021
 
%%url https://vk.com/wall231665559_27999
 
%%author_id pyzhova_evgenia.doneck
%%date 
 
%%tags skripka,muzyka,donbass,deti,doneck,dnr
%%title Дети тянутся даже к мельчайшим частичкам прекрасного - только покажи!
 
%%endhead 
\subsection{Дети тянутся даже к мельчайшим частичкам прекрасного - только покажи!}
\label{sec:29_12_2021.vk.pyzhova_evgenia.doneck.1.skripka_vika}

\Purl{https://vk.com/wall231665559_27999}
\ifcmt
 author_begin
   author_id pyzhova_evgenia.doneck
 author_end
\fi

\ii{29_12_2021.vk.pyzhova_evgenia.doneck.1.skripka_vika.pic.1}

Вика переживала и боялась выступать перед одноклассниками больше, чем перед
полным залом людей. Почему-то она была уверена, что друзья, которые слушают
Моргенштерна (надеюсь, правильно его имя написала) и прочих современных звёзд,
не оценят и не поймут скрипку. Она постоянно говорила мне о том, что друзья
будут смеяться.

Но все прошло успешно. Друзья слушали с восторгом, вызывали на бис, и даже
после праздника выбегали в коридор и делились впечатлениями, как Вика классно
играла. А Вика была абсолютно счастлива, потому что её поняли. Теперь
одноклассники тоже хотят играть на скрипке :) И это лучшая награда для
творческого человека!

Когда-то Вика увидела Олю на фестивале. И влюбилась. Она болела скрипкой 4
года, и, наконец, упросила меня отдать ее в Школу искусств №8. Поначалу мне
казалось, что месяц или два попробует, не получится - бросит. Но время идёт, и
Вика все ближе к своей мечте - быть похожей на ту самую волшебную Олю, которая
так впечатлила ее на фестивале

Первые попытки выучить очередную мелодию всегда заканчиваются неудачей и
слезами на тему \enquote{У меня никогда не получится}. В такие моменты мне очень жаль
мою принцессу, но все, что я могу для нее сделать - это твердить, что она
обязательно справится, и нужно просто много раз пробовать ещё, ещё и ещё

Образ скрипача - такой лёгкий и воздушный, как будто они вышли из сказки, и не
принадлежат этой, человеческой реальности. Но за этим образом - взрослый и
серьезный многочасовой труд над каждой фразой, собранной из нот

Каждый ребенок, который играет на скрипке, регулярно совершает настоящий подвиг
в борьбе с собственными ленью и неверием в себя. А нам остаётся уважать их и
поддерживать на этом пути

Возвращаясь к теме современных детей, которые слушают Моргенштерна. Не все
потеряно, как видите. Дети тянутся даже к мельчайшим частичкам прекрасного -
только покажи!

\ii{29_12_2021.vk.pyzhova_evgenia.doneck.1.skripka_vika.cmt}
