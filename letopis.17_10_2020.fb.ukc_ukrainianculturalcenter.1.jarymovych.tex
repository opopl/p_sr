% vim: keymap=russian-jcukenwin
%%beginhead 
 
%%file 17_10_2020.fb.ukc_ukrainianculturalcenter.1.jarymovych
%%parent 17_10_2020
 
%%url https://www.facebook.com/UKC.UkrainianCulturalCenter/posts/2695184380747154
 
%%author 
%%author_id 
%%author_url 
 
%%tags 
%%title 
 
%%endhead 

\subsection{Яримович Михайло Іванович - NASA}
\label{sec:17_10_2020.fb.ukc_ukrainianculturalcenter.1.jarymovych}
\Purl{https://www.facebook.com/UKC.UkrainianCulturalCenter/posts/2695184380747154}

\verb|#Українці_в_світі|


\ifcmt
  pic https://scontent-mad1-1.xx.fbcdn.net/v/t1.6435-9/121833344_2695184290747163_4681758816316061924_n.jpg?_nc_cat=107&ccb=1-3&_nc_sid=8bfeb9&_nc_ohc=XxCftIwBHVMAX-6Hh1Y&_nc_ht=scontent-mad1-1.xx&oh=db132fa2384b9b58c25be02062a30efa&oe=609681E4

	pic https://scontent-mad1-1.xx.fbcdn.net/v/t1.6435-9/122056471_2695184327413826_5305030512261498243_n.jpg?_nc_cat=100&ccb=1-3&_nc_sid=8bfeb9&_nc_ohc=zB9zl3r6WSAAX-3841P&_nc_ht=scontent-mad1-1.xx&oh=a6c6d6a1b5ae3dd498a2e77adafebd3c&oe=6095ECB6
\fi

13.10.1933р. - в українському Підляшші народився Яримович Михайло Іванович,
один із керівників NASA (США)  — інженер винахідник в галузі літально-космічних
дослідів у США, дійсний член НТШ й УВАН, почесний член Товариства українських
інженерів у Америці.

Він брав участь у конструюванні космічного корабля для польоту на місяць,
відомого під назвою «Аполлон» (основний системотехнік цього проекту), з 1962
працював у Крайовій адміністрації аеронавтики і космічних дослідів. 1970—1973 —
директор дорадчої групи для аерокосмічних дослідів при НАТО в Парижі; з 1973
головний науковий консультант при штабі Військово-повітряних сил США у
Вашингтоні. З 1975 асистент-адміністратор при федеральному агентстві в питаннях
освоєння джерел енергії. 

Як автор праць з аеродинаміки і космічної технології та їх практичного
застосування, 1973 відзначений найвищою нагородою за особливі заслуги на
цивільній службі при військово-повітряних силах США («US Air Force Exceptional
Civilian Service Award»).

Після проголошення незалежності України 1991 М. Яримович плідно співпрацює зі
своєю історичною батьківщиною. За його участю реалізується міжнародний проект
«Морський старт» (Sea Launch), де використовується українська ракета
«Зеніт-3SL».

М. Яримович — іноземний член НАН України (технічна механіка, США, обраний 25
листопада 1992), почесний член Наукового Товариства імені Шевченка та
Української Вільної Академії Наук.

Він — лауреат міжнародної відзнаки НАТО «Медалі Кармана» (2001). Активний
учасник громадського українського життя США: зокрема, керував студентською
секцією Товариства Українських Інженерів Америки. Від 1982 — президент
Американського інституту астронавтики і аеронавтики. Від 1984 — член
Міжнародної Академії астронавтики, яка об'єднує понад 1000 найвидатніших
науковців всіх ділянок космонавтики світу. Від 1996 — М. Яримович — президент
цієї Академії.

\verb|#УКЦ #Знай_своїх|
