% vim: keymap=russian-jcukenwin
%%beginhead 
 
%%file 26_12_2021.fb.fb_group.story_kiev_ua.5.staraja_shuljavka.pic.7.cmt
%%parent 26_12_2021.fb.fb_group.story_kiev_ua.5.staraja_shuljavka
 
%%url 
 
%%author_id 
%%date 
 
%%tags 
%%title 
 
%%endhead 

\iusr{Нина Опольская}
Душевная обстановка передается через года  @igg{fbicon.hearts.revolving} 

\iusr{Нина Опольская}
Великолепное фото!

\iusr{Владислав Бережной}

Открытое окно, пусть даже только внутренняя рама, наводит на мысль, что это
может быть какое-то другое время года?

\iusr{Георгий Майоренко}
\textbf{Владислав Бережной} Пару дней назад и я окно открывал.

\iusr{Александр Квитницкий}
\textbf{Владислав Бережной} 

Ты не представляешь как тогда топили... Батареи обжигали.

Мы тоже жили с открытым окном на Красноармейской.

\iusr{Владислав Бережной}
\textbf{Александр Квитницкий} - ну, очевидцам верю  @igg{fbicon.wink} 

\iusr{Квітницька Ганна}
\textbf{Владислав Бережной} 

я купила на батареи такие железные отсекатели( еще и побегать за ними
пришлось), чтоб наша тетя, у которой писменный стол, стоял у батареи, не дай
бог не обожглась)))

Перла их на себе аж из \enquote{Киянки}, если кто помнит)))

\iusr{Анна Сидоренко}
Жарко очевидно было, потому и окно открыто...

\iusr{Георгий Майоренко}
\textbf{Анна Сидоренко} Проветренное помещение - залог здоровья!

\iusr{Анна Сидоренко}
Ну, а я о чём?

\iusr{Віта Ілейко}
интeрeсно, гдe ваша мама на фото, папу узнала?

\iusr{Георгий Майоренко}
\textbf{Віта Ілейко} Это ещё до знакомства с ней. Рядом с отцом его двоюродная сестра.

\iusr{Нина Опольская}
Все красивые и благородные

\iusr{Евгения Бочковская}
Открытые лица...
Искренние лица !
