% vim: keymap=russian-jcukenwin
%%beginhead 
 
%%file 27_03_2020.stz.news.ua.mrpl_city.1.teatr_zhyttja_mrpl_den_teatru
%%parent 27_03_2020
 
%%url https://mrpl.city/blogs/view/teatralne-zhittya-mariupolya-z-nagodi-mizhnarodnogo-dnya-teatru
 
%%author_id demidko_olga.mariupol,news.ua.mrpl_city
%%date 
 
%%tags 
%%title Театральне життя Маріуполя. З нагоди Міжнародного дня театру
 
%%endhead 
 
\subsection{Театральне життя Маріуполя. З нагоди Міжнародного дня театру}
\label{sec:27_03_2020.stz.news.ua.mrpl_city.1.teatr_zhyttja_mrpl_den_teatru}
 
\Purl{https://mrpl.city/blogs/view/teatralne-zhittya-mariupolya-z-nagodi-mizhnarodnogo-dnya-teatru}
\ifcmt
 author_begin
   author_id demidko_olga.mariupol,news.ua.mrpl_city
 author_end
\fi

\ii{27_03_2020.stz.news.ua.mrpl_city.1.teatr_zhyttja_mrpl_den_teatru.pic.1}

На всіх екскурсіях чи своїх заняттях я постійно повторюю, що маріупольці мають
свою виняткову і неповторну театральну культуру. І це дійсно так! Професійні та
аматорські колективи нашого міста завдяки унікальному почерку, власному стилю,
незабутній манері гри не втомлюються вражати вибагливого глядача. Сьогодні, 27
березня, у всьому світі відзначається \textbf{Міжнародний день театру}. З нагоди цієї
події організовуються різні національні та міжнародні театральні заходи. Однак
через карантинні заходи театральний фестиваль в Маріуполі, на жаль, скасований.
Всесвітній день театру не є державним святом і не є вихідним днем. Проте це не
заважає вважати його професійним святом всіх театральних діячів і робітників
театрів. З нагоди свята пропоную згадати, які театри діяли в Маріуполі раніше і
які продовжують працювати сьогодні.

Попри політичну та економічну ситуацію театр в Маріуполі завжди користувався
неабияким попитом. Підтвердженням цьому є народження перших \enquote{храмів музи
Мельпомени} в нашому місті, коли підприємливі грецькі купці відкривали театри у
власних амбрах. Народженням професійного театру ми повинні завдячувати
справжньому ентузіасту, закоханому в свою справу \textbf{Василю Леонтійовичу
Шаповалову}. Різноманітний репертуар професійних театрів, постійна наявність у
регіоні гастролюючих труп, розвиток аматорського руху та активна благодійна
діяльність акторів посприяли формуванню власної самобутньої і водночас
унікальної культури міста. До професійних маріупольських театрів, які діяли в
різний час, входили: Зимовий театр, Державний грецький театр, Державний
драматичний театр ім. 13-річчя жовтня Маріупольський музично-драматичний театр
ім. Т. Шевченка, Державний театр російської драми у м. Маріуполь, Обласний
драматичний російський театр (м. Маріуполь). На маріупольській сцені можна було
побачити виступи російських, українських, польських, іта\hyp{}лійських та навіть
алжирських труп. Завдяки великому попиту столицю Приазов'я відвідали відомі і
видатні актори та театральні колективи, серед них: П. Орленьов, Н.
Єрмоленко-Южина, Д. Южин, В. Мейєрхольд, М. Кропивницький, І. Карпенко-Карий,
П. Саксаганський, М. Садовська-Барілотті та інші. Водночас творчість
аматорських драматичних колективів Маріуполя у 50-ті роки XX ст. була важливою
складовою театрального життя Північного Приазов'я та компенсувала відсутність
постійних професійних труп. Особливою майстерністю та унікальним репертуаром
відрізнялися від інших Народний театр ПК \enquote{Азовсталь}, український драматичний
гурток при ПК \enquote{Азовсталь}, театр-клуб \enquote{Діалог}, театр сатири Палацу культури
\enquote{Іскра} та інші.

Впродовж 1960-х років у місті постав і розпочав роботу \textbf{Обласний драматичний
російський театр м. Маріуполя}, що посприяло подальшому піднесенню і підйому
творчої активності митців та пожвавленню театрального життя. Глядачів з різних
міст захоплювали різноманітний репертуар маріупольського театру, яскраві
акторські особистості та висока сценічна культура, на чому неодноразово
наголошувалось у рецензіях та відгуках. Відкриття малої сцени в маріупольському
театрі дало колективу додатковий простір для творчості, практичні можливості
пошуку і втілення нових виразних форм діалогу та зближення з глядачем.
Майстерність і талант таких непересічних майстрів, як О. Утеганов, Б. Сабуров,
В. Бугайов, Н. Юрген, Н. Білецька, С. Отченашенко, А. Сорокко, В. Ахрамєєв, М.
Земцов, М. Алютова, М. Ковальчук, на яких рівнялася здібна акторська молодь,
були міцним творчим ядром однодумців, сприяли зростанню професіоналізму та
гуртуванню театральної трупи, були міцним творчим ядром однодумців.

\ii{27_03_2020.stz.news.ua.mrpl_city.1.teatr_zhyttja_mrpl_den_teatru.pic.2}

Сьогодні в Маріуполі діє декілька професійних театрів і низка народних та
самодіяльних. \textbf{Донецький академічний обласний драматичний театр (м. Маріуполь)}
на чолі з мудрим керівником \textbf{Володимиром Володимировичем Кожевніковим} сповідує
такі усталені репертуарні принципи, як послідовне освоєння класичної спадщини,
пошук сучасних п'єс і готовність до експериментів. Вражає, що на сцені можна
побачити виступи як українською, так і російською мовою. Лірична \textbf{Анжеліка
Добрунова}, сміливий \textbf{Анатолій Левченко}, неординарна \textbf{Євдокія Тіхонова}, вишуканий
\textbf{Арутюн Кіракосян} вражають своїми режисерськими задумами і не втомлюються
дивувати щорічними прем'єрами. А міцний акторський ансамбль забезпечує виставам
успіх.

Вагомий внесок в театральне життя міста робить і \textbf{Маріуполь\hyp{}ський театр ляльок} на
чолі з \textbf{Іриною Анатоліївною Руденко}, яка, маючи педагогічну і режисерську
освіту, всіма силами підтримує життя унікального жанру театрального мистецтва.

Ще один професійний театр міста \enquote{Terra Incognita: свій театр для своїх} за
досить короткий час зміг не тільки завоювати визнання глядачів, але й отримати
високу оцінку від театральних критиків. Завдяки енергійній діяльності керівника
театру Левченка Анатолія Миколайовича за 4 місяці театр підготував 4 прем'єри.
Є членом асоціації \enquote{Український Незалежний театр}. На двох прем'єрах,
поставлених за творами українських сучасних драматургів, були присутні автори.
Також в театрі проходив всеукраїнський проєкт \enquote{Весна}. А вистава \enquote{Голос тихої
безодні} була прийнята до участі у всеукраїнському фестивалі \enquote{ГРА} разом з
провідними театрами країни.

Бажання знайти оригінальні засоби сучасного мистецького втілення відчуваються в
роботі \textbf{Народного театру \enquote{Театроманія}}, засновником та керівником якого є
\textbf{Тельбізов Антон}. Сучасну авангардну режисура і драматургію можна знайти в
новому \textbf{театрі-авторської п'єси \enquote{Концепція}}. Справжнім відкриттям для
маріупольців став нещодавно створений театр \textbf{\enquote{Драмком}} на чолі з \textbf{Наталею
Гончаровою}, який вже на першій прем'єрі зібрав цілу залу маріупольців і
викликав величезний інтерес у глядачів.

\textbf{Читайте также:} \emph{Мариупольский драмтеатр публикует видеозаписи спектаклей. Где их посмотреть бесплатно?}%
\footnote{Мариупольский драмтеатр публикует видеозаписи спектаклей. Где их посмотреть бесплатно?, Богдан Коваленко, %
mrpl.city, 26.03.2020, \par%
\url{https://mrpl.city/news/view/mariupolskij-dramteatr-publikuet-videozapisi-spektaklej-gde-ih-posmotret-besplatno}
}

Те, що в Маріуполі співіснує стільки театральних колективів не може не
дивувати. Головне, що всі вони мають підтримку глядачів і готові й надалі
вражати своїми виставами та досягненнями. Закликаю всіх містян не забути
привітати улюблених митців і щиро подякувати за їхній талант і самовіддану
працю. Зі свого боку сподіваюся на плідну співпрацю з усіма театральними
діячами міста, адже карантин все ж закінчиться і попереду на нас чекає безліч
незабутніх творчих зустрічей і майстер-класів. Наостанок всім причетним до
цього чудового свята хочу побажати побільше натхнення і задоволення від роботи.
Нехай оплески не змовкають, нехай захоплення глядачів будуть вам великої
нагородою. Успіху вам у праці і бурхливих овацій.

З Днем театру!

\textbf{Смотрите также:} \emph{Как безопасно прогуляться по Мариуполю во время карантина: все виртуальные экскурсии в формате 360˚ от MRPL.CITY}%
\footnote{Как безопасно прогуляться по Мариуполю во время карантина: все виртуальные экскурсии в формате 360˚ от MRPL.CITY, Богдан Коваленко, mrpl.city, 22.03.2020, \par%
\url{https://mrpl.city/news/view/kak-bezopasno-progulyatsya-po-mariupolyu-vo-vremya-karantina-vse-virtualnye-e-kskursii-v-formate-360-ot-mrplcity}
 }
