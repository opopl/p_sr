% vim: keymap=russian-jcukenwin
%%beginhead 
 
%%file 20_01_2022.fb.fb_group.story_kiev_ua.1.detstkie_zabavy
%%parent 20_01_2022
 
%%url https://www.facebook.com/groups/story.kiev.ua/posts/1844066102456849
 
%%author_id fb_group.story_kiev_ua,ljashenko_vadim
%%date 
 
%%tags deti,foto,kiev,vov
%%title Детские забавы
 
%%endhead 
 
\subsection{Детские забавы}
\label{sec:20_01_2022.fb.fb_group.story_kiev_ua.1.detstkie_zabavy}
 
\Purl{https://www.facebook.com/groups/story.kiev.ua/posts/1844066102456849}
\ifcmt
 author_begin
   author_id fb_group.story_kiev_ua,ljashenko_vadim
 author_end
\fi

Детские забавы. 

Это фото, сделанное немецким офицером в 1942 году, я обнаружил на аукционе.
Окрестности Киева, дорога вымощена брусчаткой, крайний справа мальчик носит
кепку - это, обычный тип одежды того времени. Оба мальчика держат в руках
какие-то обручи и палочки. Что это? Обратимся к воспоминаниям наших старожилов.

\ii{20_01_2022.fb.fb_group.story_kiev_ua.1.detstkie_zabavy.pic.1}

Фрагмент из книги Виталия Баканова \enquote{Детство 50х-60х}: "Где оно, уже забытое
состояние детства с беззаботным настроением и «величайшими проблемами». Кто
сильнее? Кто выше прыгает? Кто быстрее бегает?

Вспоминается еще одна игра, ну очень подвижная! Вы когда-нибудь гоняли до
изнеможения обруч? Жаль... Находили старый обод от велосипедного колеса. В
крайнем случае -  само колесо.  Еще один метод достать колесо – это снять с
бочки у овощного магазина! Было три способа гонять колесо - палочкой,
кочерёжкой и прутиком. Последнее не путали с первым и держали за высший
пилотаж. Палочкой - легкой и круглой (сантиметров 60-80) - колесо гонят перед
собой, попеременно постукивая по его сторонам. Колесо часто уходит в сторону,
часто падает - это может быстро вначале могло наскучить нетерпеливому и
честолюбивому ребенку. Однако постепенно осваивается эта наука, и ребенок
получает величайшее удовольствие".

Фрагмент из книги Вадима Ляшенко \enquote{От Евбаза до Шулявки}: \enquote{В первой половине XX
столетия среди советской детворы было популярно развлечение – катание обруча.
Металлический обруч, обычно от бочки, гоняли перед собой специальной
«кочергой». К «кочерге» крепился спичечный коробок, в котором лежали «запчасти»
для обруча – шурупы, гвоздики, шайбы. Среди обручей особенно популярной
считалась «звездочка» – диск из нержавеющей стали. «Звездочка» представляла
собой диск сцепления из коробки передач разбитой техники. Из-за маленьких
зубчиков, аккуратно выточенных по краям, «звездочка» громко звенела во время
движения}.
