% vim: keymap=russian-jcukenwin
%%beginhead 
 
%%file 17_04_2021.fb.smolij_andrij.1.konsul_ukraina_rossia_ukrcentr
%%parent 17_04_2021
 
%%url https://www.facebook.com/groups/promovugroup/permalink/955134478393747/
 
%%author 
%%author_id 
%%author_url 
 
%%tags 
%%title 
 
%%endhead 
\subsection{Все що має бути - це рішуче викорінення російського з України}
\label{sec:17_04_2021.fb.smolij_andrij.1.konsul_ukraina_rossia_ukrcentr}
\Purl{https://www.facebook.com/groups/promovugroup/permalink/955134478393747/}

\ifcmt
  pic https://scontent-frt3-1.xx.fbcdn.net/v/t1.6435-9/175069815_3940994332605006_6055407832389035898_n.jpg?_nc_cat=108&ccb=1-3&_nc_sid=8bfeb9&_nc_ohc=vLW48WDQrVMAX9rl29G&_nc_ht=scontent-frt3-1.xx&oh=7a13320b2991581b2933267725948049&oe=60A0F3ED
\fi

Росія затримує та висилає українського консула з Санкт-Петербурга. Звинувачує в
«шпіонажі» та «зборі даних». Очікувано. Так само як очікувано, що в самій
Москві вже давно немає українського культурного центру. 

До речі, на відміну від України, де досі працює російський бізнес, діє п‘ята
колона Кремля у вигляді політичних партій, а медійні «рупори» Путіна (крім
каналів Медведчука) досі не закриті.

Яка ж реакція влади? 

Поки ніякої. Все як завжди.

Яка має бути реакція і що Україна має зробити в ситуації коли агресія лише посилюється.

Перше. Закриття кордону, який досі як решето. Громадяни РФ досі в‘їжджаю і виїжджають. 

Друге. Закриття московського бізнесу в Україні. Націоналізація кремлівського майна. Його тут дуже багато.

Третє. Повна відмова від російського медіа продукту. Заборона російського в
Україні в публічному просторі. Як можуть рекламуватись всякі «моргенштерни» та
інша кремлівська погань? Про які концерти може йти мова? 

Ах так, забув. 

Серіал «Папік 2» виробнитцва Кварталу 95 Зеленського став одним з
найрейтинговіших на федеральному телебаченні РФ. Чи це не торгівля на крові? 

Все що ми зараз спостергаємо - це напівкроки, напівдії, напівслова. Все що має
бути - це рішуче викорінення російського з України. Лише так переможемо і не
дамо Путіну «захищати російське» в Україні.
