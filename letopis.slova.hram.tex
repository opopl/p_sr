% vim: keymap=russian-jcukenwin
%%beginhead 
 
%%file slova.hram
%%parent slova
 
%%url 
 
%%author 
%%author_id 
%%author_url 
 
%%tags 
%%title 
 
%%endhead 
\chapter{Храм}
\label{sec:slova.hram}

%%%cit
%%%cit_head
%%%cit_pic
%%%cit_text
Почему закрывались глаза на захват уже действующих православным \emph{храмов}, а
строительству новых, чему я была непосредственным свидетелем, активно мешают на
всех уровнях? Особенно ярко это проявляется в сёлах и небольших городках, там,
где легче безнаказанно давить православных. Так, в городе Буча, что под Киевом,
местные власти очень быстро «нашли» землю под строительство \emph{храмов} киевского
патриархата в тех местах, где им указали архитекторы, а \emph{каноническому храму}
нашли место только на свалке, прекрасно зная, что там ставить \emph{храм} невозможно.
Вроде бы и закон соблюдён, и цель достигнута. Красивое новое здание у
«филаретовцев», а православные вынуждены служить ... в заброшенной столовой
завода стеклотары. Вы скажете: частный случай. Ничего подобного. Любой юрист,
который специализируется на земельном кодексе, предоставит вам столько подобных
случаев за свою практику, что вы задохнётесь от возмущения
%%%cit_comment
%%%cit_title
\citTitle{Католическая экспансия на русские земли: история и современность}, 
Светлана Пикта, teleskop.media, 14.06.2021
%%%endcit

