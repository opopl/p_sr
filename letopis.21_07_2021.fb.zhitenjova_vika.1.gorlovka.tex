% vim: keymap=russian-jcukenwin
%%beginhead 
 
%%file 21_07_2021.fb.zhitenjova_vika.1.gorlovka
%%parent 21_07_2021
 
%%url https://www.facebook.com/zhitenevaviktoriya/posts/4146666538742801
 
%%author Житенева, Вика
%%author_id zhitenjova_vika
%%author_url 
 
%%tags donbass,gorlovka,ukraina,vojna
%%title Сьогодні в горлівчан річниця. Річниця початку кінця
 
%%endhead 
 
\subsection{Сьогодні в горлівчан річниця. Річниця початку кінця}
\label{sec:21_07_2021.fb.zhitenjova_vika.1.gorlovka}
 
\Purl{https://www.facebook.com/zhitenevaviktoriya/posts/4146666538742801}
\ifcmt
 author_begin
   author_id zhitenjova_vika
 author_end
\fi

\index{Горловка! Война! Обстрел, 21.07.2014}

Сьогодні в горлівчан річниця. Річниця початку кінця. До 21 липня, дивлячись по
телевізору новини про Слов'янськ і Краматорськ, ми сподівалися( ну, багато хто
з нас) що у нас все обійдеться. Не обійшлося. І 21 липня поділило наше життя на
до і після. Близько 4-ї ранку 5-й квартал прокинувся від звуків розриву
снарядів. Я витягнула з під ліжка переляканого кота, узяла паспорт і в халаті
та домашніх тапках побігла сама не знаючи куди. Бігли всі. Вже на вулиці хтось
сказав що треба бігти в дитячий садок у нашому дворі - там є підвал. 

\ifcmt
  pic https://external-cdt1-1.xx.fbcdn.net/safe_image.php?d=AQG0SVvNraSmf5Be&w=500&h=261&url=https%3A%2F%2Fi.ytimg.com%2Fvi%2FZGUxlHjHSpA%2Fmaxresdefault.jpg&cfs=1&ext=jpg&ccb=3-5&_nc_hash=AQGCjWb2ciN6JyDx
  width 0.8
\fi

\href{https://www.youtube.com/watch?v=ZGUxlHjHSpA}{%
Дзержинск-Горловка ~ 21.07.14 - Обстрелы, OBSOLETE\_EN1GMA HARDY, youtube, 21.07.2014%
}

Я пам'ятаю натовп людей в цьому підвалі. Пам'ятаю як хтось сказав що Правий
сектор знаходиться в Новолуганці і це вони обстріляли нас. Пам'ятаю як я
зраділа, тому що вірила що український снаряд ніяк  не може в мене потрапити,
бо це якось несправедливо - я ж за Україну.... Потім хтось кудись подзвонив і
сказав що можна йти по хатах - це "наші" хлопчики знеструмили трансформатор,
поки буде тихо, але краще швидше бігти з міста. Навіщо треба було виводити з
ладу трансформатор мені незрозуміло й досі. Як і для чого треба було
висаджувати в повітря міст за пару тижнів до цього. Міст сполучав два райони,
5-й квартал і Бесарабку. Яке стратегічне значення мав цей міст, я думаю, не
знають навіть ті хто його висаджував в повітря. Напевно їм було просто цікаво.
А місцеві дурні раділи: ворог не пройде.

Загалом, після цього раннього обстрілу в місті почався хаос. Подруга з дочкою
вирішила поїхати у Святогірск, а я не захотіла : наївно вважала що це усе скоро
закінчиться, а я пересиджу у тітки - все ж не так страшно в компанії. Я сиділа
на лавочці біля під'їзду в очікуванні машини і спостерігала за істерією, що
відбувалася навкруги. Всі покидали місто із швидкістю звуку.  Ну, як до цього
дружно ходили на референдум, так тепер дружно тікали з міста. 

Потім було набагато страшніше, звичайно. Потім ми навчилися відрізняти Гради
від мінометів.

Потім ми почали розуміти де саме стріляють і  варто стрибати в погреб чи можна
ще поспати півгодини, бо в тому ж самому погребі ми сиділи півночі і коли тобі
на голову падають банки з консервацією, то тобі не до сну.

Потім дні змішалися з ночами і було все одно  хто там загинув з " мирних"
жителів, аби тільки не мої...

Я не пам'ятаю дату коли снаряд влучив в мою квартиру. А 21 липня буду пам'ятати
до самої смерті.

\ii{21_07_2021.fb.zhitenjova_vika.1.gorlovka.cmt}
