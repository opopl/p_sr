% vim: keymap=russian-jcukenwin
%%beginhead 
 
%%file 14_12_2021.stz.news.ua.mrpl_city.1.mdu_30_richnyj_jubilej_plany
%%parent 14_12_2021
 
%%url https://mrpl.city/blogs/view/30-richnij-yuvilej-mdu-1
 
%%author_id demidko_olga.mariupol,news.ua.mrpl_city
%%date 
 
%%tags 
%%title 30-річний ювілей МДУ - подальші плани ювіляра
 
%%endhead 
 
\subsection{30-річний ювілей МДУ - подальші плани ювіляра}
\label{sec:14_12_2021.stz.news.ua.mrpl_city.1.mdu_30_richnyj_jubilej_plany}
 
\Purl{https://mrpl.city/blogs/view/30-richnij-yuvilej-mdu-1}
\ifcmt
 author_begin
   author_id demidko_olga.mariupol,news.ua.mrpl_city
 author_end
\fi

7 грудня Маріупольський державний університет відсвяткував своє 30-ти річчя.
Час дізнатися про подальші плани ювіляра на наступні роки його діяльності.

\ii{14_12_2021.stz.news.ua.mrpl_city.1.mdu_30_richnyj_jubilej_plany.pic.1}

Всі свої ректорські мрії \emph{\textbf{Микола Валерійович Трофименко}} заклав у стратегію
розвитку університету. Наразі відбувається трансформація системи управління
МДУ, вона стає більш мобільною і здатною відповідати на всі виклики сучасності,
тому що заклад вищої освіти сьогодні працює як велика корпорація. Ректор
Маріупольського державного університету наголосив, що керівництво повинно дуже
пильно слідкувати за запитом ринку праці, стейкхолдирів, роботодавців.
Університет має працювати над комунікацією з державними, місцевими та
регіональними органами влади для того, щоб маневрувати і розвиватись, адже
тільки розвиток, для якого закладається підґрунтя сьогодні і розвиток у
майбутньому – дозволить зберегти університет та зробити його по-справжньому
великим регіональним класичним закладом вищої освіти. Микола Валерійович
поділився, що керівництво закладу дивиться на різні галузі міста, які ще не
охоплені вищою освітою не тільки в Маріуполі, але й загалом у всій Донецькій
області. Зокрема, це аграрний, медичний мистецький напрями та напрям розвитку
міського господарства. Ректор МДУ зауважив, що дуже відчутна підтримка Голови
Донецької обласної державної адміністрації (керівника обласної
військово-цивільної адміністрації) \emph{\textbf{Павла Олександровича Кириленка}}, який
підтримав всі ініціативи щодо розширення структури університету завдяки
приєднанню коледжів. Першим коледжем, який висловив бажання приєднатися до
структури МДУ став електромеханічний. Разом з висококваліфікованими
спеціалістами коледжу, ректор впевнений, вдасться розвивати і створювати в
майбутньому факультет міського господарства Маріупольського державного
університету для того, щоб задовольнити  попит на висококласних фахівців
комунальних підприємств Маріуполя.

\ii{14_12_2021.stz.news.ua.mrpl_city.1.mdu_30_richnyj_jubilej_plany.pic.2}

Одне з головних завдань ректора є розвиток інфраструктури університету.
Наступного року розпочнеться серйозний проєкт щодо розвитку студентського
містечка МДУ, буде створений кампус по-справжньому європейський та сучасний.
Водночас дуже багато робиться, щоб підтримувати сучасну матеріально-технічну
базу університету. Відбуваються інвестиції завдяки коштам Маріупольської
міської ради, Донецької обласної державної адміністрації, Міністерства освіти і
науки України, Державного фонду регіонального розвитку, програми \enquote{Велике
будівництво} Президента України. Цього року МДУ приєднався до спеціальної
програми для покращення вищої освіти в галузі кібербезпеки від проєкту USAID.
На економіко-правовому факультеті були відкриті Лабораторія інформаційної
кібернетичної безпеки та Лабораторія системного аналізу. На факультеті
філології і масових комунікацій успішно працює телестудія МДУ. Вона дозволяє
готувати журналістів. Гордістю Маріупольського державного університету є Музей
історії та археології, а також археологічна експедиція. Щорічні знахідки
експедиції поповнюють фонди Маріупольського краєзнавчого музею. Наразі тривають
роботи щодо створення Музею науки – єдиного такого музею в Донецькій області,
одного з трьох регіональних музеїв, які створюються сьогодні в країні. Цей
музей стане фішкою університету і одним з туристичних магнітів Маріуполя. Також
плануються капітальні ремонти в корпусах університету і ще більше їхнє
оновлення, щоб студентам було затишно і комфортно навчатися. До речі, зараз
завершується ремонт фасаду гуртожитку МДУ, який незабаром отримає ще кращий
вигляд, адже стане сучаснішим, теплішим та затишнішим.

\ii{14_12_2021.stz.news.ua.mrpl_city.1.mdu_30_richnyj_jubilej_plany.pic.3}

Другим своїм завданням ректор ставить розвиток кадрового потенціалу
університету. Кожна освітня програма має відповідати своїм акредитаційним
програмам незалежно від того, коли акредитація відбуватиметься. Було проведено
дуже серйозний моніторинг, аудит, завдяки якому були виявлені всі плюси і
мінуси в освітніх програмах МДУ. Маріупольський державний університет у 2021
році пройшов 18 акредитацій з експертами з Національної агенції з питань
забезпечення і якості вищої освіти. Всі акредитації мали досить позитивні
результати. Продовжують розвиток і наукові школи та спеціалізовані вчені ради
університету.

\ii{14_12_2021.stz.news.ua.mrpl_city.1.mdu_30_richnyj_jubilej_plany.pic.4}

Наприкінці жовтня 2021 року в МДУ розпочалися безкоштовні курси з англійської
мови, завдяки яким з'являться англомовні освітні програми та збільшиться
кількість публікацій англійською мовою.

\ii{14_12_2021.stz.news.ua.mrpl_city.1.mdu_30_richnyj_jubilej_plany.pic.5}

Третім завданням є збільшення контингенту студентів. Багато зусиль приділяється
профорієнтаційній роботі. На думку Миколи Валерійовича, МДУ повинен бути
сучасним та дуже потужним, щоб кожен студент міг пишатися тим, що він є
студентом саме Маріупольського державного університету. Сьогодні навчатися у
стінах ювіляра – не менш престижно та круто, ніж в Києві, Львові, Харкові чи
Одесі. І той попит на освітні програми МДУ, який сьогодні вже є, показує, що
університет рухається в правильному напрямі.

\ii{14_12_2021.stz.news.ua.mrpl_city.1.mdu_30_richnyj_jubilej_plany.pic.6}

Своє 30-ти річчя МДУ відсвяткував у Драматичному театрі. На урочистих зборах
відбулося вручення відзнак, Почесних грамот і подяк різного рівня. За вагомий
внесок у науковий, освітній, культурний розвиток регіону, високий
професіоналізм, сумлінну працю та з нагоди 30-річчя створення Маріупольського
державного університету колектив закладу був нагороджений почесними грамотами
від Маріупольської міської ради та Маріупольської районної державної
адміністрації. Радник ректора МДУ, почесний генеральний консул республіки Кіпр
у Маріуполі \emph{\textbf{Костянтин Васильович Балабанов}} отримав медаль \enquote{Володимир Мономах}
Національної академії педагогічних наук України. Він  зазначив, що для нього це
велика честь отримати найвищу нагороду Національної академії педагогічних наук
України. \emph{\enquote{Це висока оцінка моєї скромної праці і це насамперед оцінка успішної
роботи нашого дружного колективу}}.

\ii{14_12_2021.stz.news.ua.mrpl_city.1.mdu_30_richnyj_jubilej_plany.pic.7}

За сумлінну та результативну працю понад 30 працівників уні\hyp{}верситету отримали
нагороди Комітету Верховної Ради України, Національної академії педагогічних
наук України, Донецької обласної державної адміністрації, Маріупольської
міської ради. Зокрема, за вагомий особистий внесок у розвиток вищої освіти
України орденом княгині Ольги III ступеня була нагороджена перший проректор
Маріупольського державного університету \emph{О. В. Булатова}, нагрудний знак
\enquote{Відмінник освіти} отримала декан історичного факультету В. Ф. Лисак, почесною
відзнакою \enquote{За заслуги перед м. Маріуполем} нагородили \emph{Ю. І. Чентукова}, професор
кафедри раціонального природокористування та охорони навколишнього середовища
\emph{Г. О. Черніченко} отримав Нагрудний знак Обласної державної адміністрації \enquote{Знак
Пошани}. Почесні грамоти і подяки Національної академії педагогічних наук
України вручили \emph{Т. В. Марені, Б. М. Свірькому, Ю. С. Сабадаш, Ю. О. Демидовій,
Г. М. Масловій}.  Почесні грамоти та подяки Обласної державної адміністрації
отримали \emph{М. В. Трофименко, А. В. Осіпцов, О. В. Ємененко, Н. В. Балабанова,
Т. М. Грачова, Т. В. Шабельник}. Почесними грамотами Маріупольської міської ради і
подяками міського голови нагородили \emph{А. М. Дощік, О. Г. Павленко, С. В. Безчотнікову,
Л. В. Задорожну-Княгницьку, М. В. Булика} та ін.

\ii{14_12_2021.stz.news.ua.mrpl_city.1.mdu_30_richnyj_jubilej_plany.pic.8}

На святковому зібранні колектив Маріупольського університету привітали численні
почесні гості, серед яких народні депутати України \emph{Дмитро Лубінець} і \emph{Сергій
Тарута}, секретар Маріупольської міської ради \emph{Ксенія Сухова}, заступник міського
голови Маріуполя \emph{Олександр Кочурін}, голова Маріупольської районної
держадміністрації Наталя Букрєєва, голова Маріупольської районної ради \emph{Степан
Махсма}, голова Федерації грецьких товариств України \emph{Олександра
Проценко-Пічаджи}. Привітання освітянам та студентам від від імені Президента
України та керівника Донецької обласної державної адміністрації \emph{Павла Кириленка}
зачитала \emph{\textbf{Юлія Костюніна}}. Міністр освіти і науки України \emph{\textbf{Сергій Шкарлет}} записав
відеозвернення до колективу університету, в якому наголосив, що 
\begin{quote}
\em\enquote{Маріупольський
державний університет завжди перебуває в авангарді реформування освіти та науки
України. Роками ви відточували свою майстерність, переймаючи кращий досвід і
впроваджуючи новітні практики. Зичу примножувати власні досягнення та
збагачувати освітні та наукові здобутки країни, ніколи не втрачати віри у
власні сили та розум, бажати більшого, робити усе можливе, досягати
найкращого!}.
\end{quote}

Долучилися до привітань ювіляра високоповажні друзі та партнери ювіляра.
Маріупольський міський голова \emph{\textbf{Вадим Бойченко}} наголосив, що МДУ, попри свій
молодий вік, є найамбітнішим університетом міста.

\ii{14_12_2021.stz.news.ua.mrpl_city.1.mdu_30_richnyj_jubilej_plany.pic.9}

За словами члена Наглядової ради МДУ \emph{\textbf{Бориса Колеснікова}}, якщо для
військовослужбовців патріотизм –  це захист батьківщини, то для працівників
університету патріотизм – це їхні студенти. І чим більше успіху і здобутків в
ім'я процвітання України досягнуть випускники МДУ, тим більше університет
пишатиметься своїми випускниками. Секретар Міської ради \emph{\textbf{Ксенія Сухова}}
зазначила, що історія становлення сучасного Маріуполя нерозривно пов'язана із
Маріупольським державним університетом. Вона подякувала всьому педагогічному
складу за високий професіоналізм та побажала університету нових перемог та
старанних студентів, а самим студентам – знайти своє місце у професійній сфері.
Народний депутат України \textbf{\textbf{Сергій Магера}} зауважив, що 30 років у Маріуполі існує
та працює унікальна установа яка формує майбутнє України. Він побажав МДУ
процвітання, зростання та ще більшого розвитку.

\ii{14_12_2021.stz.news.ua.mrpl_city.1.mdu_30_richnyj_jubilej_plany.pic.10}

Президент \enquote{Маріупольського телебачення} \emph{\textbf{Микола Осиченко}} наголосив, що МДУ для
Маріупольського телебачення – це джерело професіоналів, молодих, амбітних,
наполегливих, перспективних людей.

\begin{quote}
\em\enquote{Ми намагаємось максимально забезпечити їм
комфортні умови роботи, щоб вони змогли всі свої теоретичні навички, отримані в
університеті, навчитися реалізовувати на практиці тут. Я бажаю, щоб ще 130, 330
років МДУ розвився, діджиталізувався та розширював міжнародні зв'язки. Я
впевнений, що ми як Маріупольське ТБ допомагатимемо на практиці відточувати всі
навики, знання, які дають студентам у цьому чудовому виші}.
\end{quote}

Директор-художній керівник Донецького академічного обласного драматичного
театру (м. Маріуполь) \emph{\textbf{Володимир Кожевніков}} підкреслив, що МДУ – це дуже
потужний заклад вищої освіти, з яким театр готовий співпрацювати. Він висловив
слова щирої вдячності за багаторічну діяльність університету і наголосив, що
співробітники МДУ для театру є дуже важливими глядачами, тому всіх запросив
частіше приходити на вистави.

Керівник Маріупольського регіонального центру ПУМБ \emph{\textbf{Андрій
Висоцький}} зауважив, що з 1991 року університет формував світогляд
маріупольської молоді, яка потім вливалася до лав клієнтів та співробітників
банку.

\begin{quote}
\em\enquote{Найскладніше, що є у
навчанні – це поєднати непоєднуване – це широта знань та вузька спеціалізацію.
Ось ці два складні компоненти я бажаю поєднати у своїх методиках, у своїй
програмі навчання}.
\end{quote}

\ii{14_12_2021.stz.news.ua.mrpl_city.1.mdu_30_richnyj_jubilej_plany.pic.11}

Заступник голови Наглядової ради МДУ, почесний професор Маріупольського
державного університету \emph{\textbf{Нікола Франко Баллоні}} розповів, що знайомий з
університетом дуже багато років. І, на його думку, це один із найкращих
університетів України. 

\begin{quote}
\em\enquote{Дуже сильні студенти – зацікавлені та здібні. Я бажаю,
щоб університет продовжив свій розвиток, удосконалився і був відомим не лише в
Україні, бо, на мою думку, МДУ – це бутоньєрка на полі України. Величезних
успіхів університету та Маріуполю}.
\end{quote}

А директор з міжнародних зв'язків та відносин з інвесторами групи \enquote{СКМ} \emph{\textbf{Джок
Міндоза Вілсон}} вважає, що Маріуполь це ключова точка Донбасу. А про майбутнє
міста розповідає саме МДУ, який динамічно розвивається та має найкращих
викладачів, тому це майбутнє Маріуполя дуже позитивне.

Керівництво університету, його викладачі та студенти також привітали дорогого
ювіляра. Зокрема, радник ректора Маріупольського державного університету,
почесний генеральний консул республіки Кіпр у Маріуполі \emph{\textbf{Костянтин Балабанов}}
побажав рідному університету подальшого успішного розвитку, цікавих проєктів та
нових міжнародних програм.

\ii{14_12_2021.stz.news.ua.mrpl_city.1.mdu_30_richnyj_jubilej_plany.pic.12}

Декан психолого-педагогічного факультету МДУ {\em\bfseries Леніна\par\noindent Задорожна-Княгницька} у
тридцятиріччя університету побажала легких перемог, злетів, і, щоб ці злети
були тільки черговою сходинкою на шляху до наступних злетів.

Студентка 3 курсу спеціальності \enquote{Журналістика} МДУ \emph{\textbf{Ксенія Місюревіч}} побажала,
щоб у Маріупольському державному університеті були тільки амбітні та
зацікавлені студенти, завдяки яким університет буде процвітати та набувати
нових масштабів.

\ii{14_12_2021.stz.news.ua.mrpl_city.1.mdu_30_richnyj_jubilej_plany.pic.13}

Перший проректор Маріупольського державного університету \emph{\textbf{Олена Булатова}}
підкреслила, що: 

\begin{quote}
\em\enquote{Маріупольський університет – це насамперед люди, люди
досвідчені, життєвий шлях яких дозволяє сьогодні ділитися цим практичним
досвідом зі студентами не тільки в опануванні майбутніх професій, але й в
частині того, як бути чемною людиною. Для мене університет – це молоді люди,
амбітні, креативні, інноваційні, які вже сьогодні змінюють цей недосконалий
світ на краще. Маріупольський державний університет – це велика родина. Віват,
Маріупольський університет. Віват, родина МДУ!}.
\end{quote}

Зі свого боку ректор Маріупольського державного університету \emph{\textbf{Микола Трофименко}}
побажав мудрості, здоров'я та розвитку. 

\begin{quote}
\em\enquote{Нашою спільною роботою ми показуємо,
що можемо досягати ще більших вершин і я думаю, що не може бути якихось
обмежень для нашого колективу, тому що він дуже амбіційний, прогресивний,
сучасний! І ми обов'язково будемо досягати ще більших висот, будемо підкорювати
всі вершини і робитимемо Маріупольський державний університет справжньою
окрасою Маріуполя, Донецької області, по-справжньому регіональним класичним
сучасним університетом}.
\end{quote}

\ii{14_12_2021.stz.news.ua.mrpl_city.1.mdu_30_richnyj_jubilej_plany.pic.14}
