% vim: keymap=russian-jcukenwin
%%beginhead 
 
%%file 06_11_2021.fb.bryhar_sergej.1.varlamov_film_pro_ukrainu.cmt
%%parent 06_11_2021.fb.bryhar_sergej.1.varlamov_film_pro_ukrainu
 
%%url 
 
%%author_id 
%%date 
 
%%tags 
%%title 
 
%%endhead 
\subsubsection{Коментарі}

\begin{itemize} % {
\iusr{Олег Оникій}
Яким чином? Навколо лише росмова і це дуже прикро!!!

\begin{itemize} % {
\iusr{Serhii Bryhar}
\textbf{Олег Оникій} 

Навіть не знаю, що відповісти. Поки якимось чином утримую цей фронт. У мене
вдома крім української регулярно чути польську, білоруську, чеську...
Російської - обмаль. Наразі малі розмовляють лише українською.

\iusr{Олег Оникій}
\textbf{Serhii Bryhar} 

так, вдома я намагаюся максимально розмовляти з дітьми рідною, дружина вже
майже на 50відсотків перейшла. Але, за межами родини все російськомовне!!!
Навіть коли в магазині я купив в магазині своїй старшій, яка гуляла з подругою,
печива, на касі після спілкування з касиром українською(касир також
обслуговувала українською, в наших атб відсотків на 90 все ж таки виконується
закон!) подруга доньки запитала в неї - "твой папа что, укрАинец?". І з плином
часу я все більше відчуваю негативне ставлення старшої доньки до української! А
їй лише 12! Сам в розпачі!!!

\iusr{Serhii Bryhar}
\textbf{Олег Оникій} 

Знаєте, ще років 10 тому я був переконаний, що ніколи не залишу своє місто.
Розраховував боротися за Україну тут і зараз. Не хочу переоцінювати свій
скромний внесок у загальну справу, але думаю, що щось корисне таки зробив.
Тепер от боротьба стала більш індивідуальною справою.. І тим не менш, зараз я
розумію, що не мешкатиму тут усе життя. Є якісь межі... Вони відчутні. Я б
сказав, що відчуття битви з вітряками зараз особливо сильне...

\iusr{Олег Оникій}
\textbf{Serhii Bryhar} 

таке відчуття, що якась невидима сила намагається виштовхнути нас з нашої
рідної землі, з нашого міста, з рідної країни..

Благо, наразі я перемикнувся і маю надію реалізувати об'єднуючий фактор наших
містян проти свавілля місцевої влади. А на це потрібен час. І по закінченні
цього часу, в залежності від отриманих результатів, можливо і я відчую цю
боротьбу з вітряками.. але так хочеться сподіватися що на той час вже все буде
інакше..

\iusr{Serhii Bryhar}
\textbf{Олег Оникій} 

Добре, коли є на що сподіватися, у що вірити. Найкраще - це рух до мети і
відчуття власної ефективності. В мене ось тут, на місці, усього цього вже
немає.

\iusr{Lyudmyla Shumey}
\textbf{Олег Оникій} 

Я вважаю, дитині дванадцяти років спокійно вже можна відповідати: так
українець, а ти ні? Нам тільки здається, що діти пропускають мимо вух, що
кажуть дорослі. Я впевнена, ваша відповідь закарбувалася б цій дівчинці на все
її життя.

\iusr{Олег Оникій}
\textbf{Lyudmyla Shumey} 

так це я почув краєм вуха, як кажуть. Звичайно, я провів розмову з донькою,
розпитав про що вони говорили, що і як треба в таких випадках відповідати.

Таких моментів, коли за моєю спиною крутили пальцями біля скроні було безліч
(зараз люди потроху, але звикають), були часи, коли, навіть на моє "доброго
ранку" запитували один в одного "єто что, прікол какойто?". Працювати в
великому колективі де відсотків 10 людей відверто проросійські, 70 - какая
разніца, а 20 більш менш можна якось активізувати, хоча б на відстоювання своїх
інтересів, дуже непросто..

Дуже непроста і болюча тема, як я розумію Сергія! Чи є закиди нашим емігрантам,
наприклад, в тій же Канаді, які віддалено підтримують, продовжують, плекають,
популяризують, ..., люблять Україну, але не повертаються, справедливими? Це те
ще питання..

\end{itemize} % }

\iusr{Yurchenko Sergiy}

Це той "блохер" типу "урбаніст" ??

Колись випадково 1 відео про якусь країну бачив з ним.

Достатньо було почути, що якийсь "руССкий" приперся в гості (!) та "вчив жити"
(звісно, він їм не в очі казав, а сам собі на селфі-камеру) місцевих жителів і
про "рівень розвитку" тієї країни.

І це той "роССотурісто", яке з "розвинутої" Масковії припхався  @igg{fbicon.man.facepalming} 

\begin{itemize} % {
\iusr{Serhii Bryhar}
\textbf{Сергій Юрченко} 

Це, до речі, махровий совок. Ліберал типу і все таке... Совок галімий. Свого
часу бачив теж кілька відео. Сам би навіть не додумався. Але знайомі підкинули
(на, посмотрі, как московія подихаєт)... то було відео про помираючу Воркуту. І
чувак, бідний, на кожному кроці: "вот какіє класниє в союзі дєлалі вивєскі, влт
какіє класниє билі свєтофлри, вот какая архітектура, вот как било
прєкрасно-то... Нє, ну це жах. Якщо це "ліберал", то я ієромонах...)).

\iusr{Yurchenko Sergiy}
\textbf{Serhii Bryhar} 

майже всі росіяни (окрім патріотів України з нашої країни, які росіяни етнічно
і громадяни нашої держави) та 100\% "русские" - вороги для нас українців і
нашої України.

Тому розбиратися в "сортах" ворогів - це нерозумно.

Ворогів України треба знищувати.

Єдині хто це робить ефективно, наші ЗСУ.

Але не "правопохоронні органи" України.

Шана і слава ЗСУ!
\end{itemize} % }

\iusr{Radiy Radutny}

Якщо підказки - це help та документація, то російськими версіями краще не
користуватися. Переклад часто невдалий і купа помилок.

\begin{itemize} % {
\iusr{Serhii Bryhar}
\textbf{Radiy Radutny} Ми з дружиною пройшли кілька курсів з налаштування сайту на вордпресі. Так, російською.. Десь це допомогло, а щось гарно зайшло і англійською.
\end{itemize} % }

\iusr{Людмила Демянчук}

Люблю той вислів "совок галімий" , чітко і ясно про чиюсь життєву позицію , а
гасло " Геть від Москви" - це головний орієнтир боротьби за незалежність
України !

\iusr{Українською Будь Ласка}
Напишіть, будь ласка, про які художні фільми йшла мова?

\begin{itemize} % {
\iusr{Serhii Bryhar}
\textbf{Oleksa Mormor} 

Так це ж явище змінне. Якщо говорити лише про старе і вже майже класичне, то є
лише, здається три переклади фільмів Бергмана, фактично не перекладені фільми
"Доги 95", є, здається, один переклад робіт Антоніоні ("Фотозбільшення",
здається - і все). Далі, можна згадати роботи хоч і суперечливого, але
геніального Пазоліні, Вісконті, можна згадати братів Каурісмякі, чиї роботи
здебільшого проходять повз... До речі, Вайда, Кеслевський, Зануссі теж - або
польською, або російською. Та взагалі, все не згадаєш. Багато... Я згоден, що з
масовим і популярним ситуація більш-менш виправляється, але ж це не все..

\iusr{Українською Будь Ласка}
\textbf{Serhii Bryhar} Дякую!

І певно, усі є в перекладі англійською. Тому чудово, що знайшли привід її
підтягнути. Я пару років через силу з онлайн-перекладачем читав і дивився нею
все, що цікавило, і це дуже допомогло.

Деякі свідомі батьки привчають дітей до англійської, бо якість і обсяг
інформації величезні порівняно з московською - потім менша спокуса й
відсутність потреби дивитися за поребрик. Країни Балтії зі схожих причин
ставлять на навчання англійської.

\end{itemize} % }

\iusr{Олександр Бродецький}

А що у тому фільмі?.. В яких аспектах на Україну наїжджає? Бо чи подужаю шукати
і, поготів, дивитися, - навряд чи...

\begin{itemize} % {
\iusr{Serhii Bryhar}
\textbf{Олександр Бродецький} 

Виходить, "гріхів" у мене таки більше, ніж говорю). Ну бо так, не втримався і
подивився... А там все: і "прісоєдінєніє" Крима по желанію граждан", і
"екстрємістскіє сіли прі власті", і "гражанскоє протівостояніє (гражданская
война) на востокє", і "упразднєніє офіціально статуса русского язика с
ущємлєнієм прав русскоязичних граждан (котрих "фактічєскі большинство")", і
"Украіна до совєцкого союза нікогда нє била в одніх граніцах"... Так відразу
все й не зафіксуєш. Фактично, оцей от "ліберал", пройшовся по ключових тезах
московського пропагандонства, вважаючи їх істинами.

\end{itemize} % }

\iusr{Sergiu Devdyk}

Ой, як погоджуюсь з вами Сергію. Ба більше я не розумію, чому нас має цікавити
як ''запалає'' на московитах від того, що в нас відбувається. Та взагалі, шо
скажє світова спільнота. Ми маєм відштовхуватись від того, що буде і шо є добре
для нашої країни, для нашого народу.

\iusr{Галина Слободянюк}
Так і робимо @igg{fbicon.biceps.flexed} 

\iusr{Тетяна Лук'янова}

Та всі вони, ті русоблохери козли - що Варламов, що варандей  @igg{fbicon.thumb.down.yellow}  , що інші.
Варандею ще мало по писку надавали у Меджибожі, нехай не бреше, що просто так
причепилися.

\iusr{Ntina Ntoubrova}

В Україні є російськомовний контент місцевого виробництва. Можна починати
заміщати людям те, що вони звикли чути в РФ - тим що зроблено у нас. А вже
потім давати аналоги українські, зокрема якщо вони якісніші.

Наприклад в школі прибрали українську літературу - а ввели "всесвітню
літературу". Де є "література мов меншин України. І чомусь в підрозділі
"російська мова" - там стоять РОСІЙСЬКІ поети. А мали би стояти твори
українських ЗРОСІЙЩЕНИХ ЄВРЕЇВ (бо вони в нас тут багато що написали в 20му
ст.). У першу чергу. Чи твори Куркова. Твори одеських чи маріупольських чи
київських письменників. Старшокласницям буде ЦІКАВО почитати "ведьмінські"
цикли МІСЬКОГО роману російськомовної киянки-феміністки Лади Лузіної. Замість
читати Толстого чи Чехова чи упасібоже Дастаєвского.

Тобто якби тут патріоти вчасно підсуєтилися - то дискурс "приберемо із
літератури все нудне і застаріле" - міг би відбитися саме на осучасненні
відбору саме РОСІЙСЬКОМОВНОЇ літератури. Щоб позбавити її совковофільства та
руссоімперства.

А відгородити не можна. Діти принесуть з вулиці, зі школи, із сімейного
корпоратива чи дитячого свята. Бо злісніші русифікатори дітей на сьогодні - це
1) тамади: 2) аніматори. ((

\begin{itemize} % {
\iusr{Тетяна Лук'янова}
\textbf{Ntina Ntoubrova} , перепрошую, а де це забрали українську літературу? Новв програма?
\end{itemize} % }

\iusr{Олексій Пономаренко}
Повністю погоджуюсь з Вами.
З приводу малечі то маючи дворічну дитину вкрай мало популярних мультфільмів для малечі з українським перекладом.
Наприклад Лунтік українською взагалі не можу знайти в you tube.
Little angels вкрай мало серій,
Мімімішки також московитською!!!
А що поробиш коли дитина саме це дивиться!!?

\iusr{Oleksandr Kirkov}
з четверга долучаюся до цього процесу (розвитку україномовного простору).

\end{itemize} % }
