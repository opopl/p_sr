% vim: keymap=russian-jcukenwin
%%beginhead 
 
%%file 28_07_2021.fb.nicoj_larisa.1.grin_grey_mova
%%parent 28_07_2021
 
%%url https://www.facebook.com/nitsoi.larysa/posts/959748354591537
 
%%author Ницой, Лариса
%%author_id nicoj_larisa
%%author_url 
 
%%tags grin_grey.gruppa.ukr,jazyk,mova,muzyka,ukraina,ukrainizacia
%%title Мовний скандал і Грін Грей
 
%%endhead 
 
\subsection{Мовний скандал і Грін Грей}
\label{sec:28_07_2021.fb.nicoj_larisa.1.grin_grey_mova}
 
\Purl{https://www.facebook.com/nitsoi.larysa/posts/959748354591537}
\ifcmt
 author_begin
   author_id nicoj_larisa
 author_end
\fi

Мовний скандал і Грін Грей. 

Передмова.

Кілька років тому, коли я почала брати активну участь в роботі над Законами про
квоти на мови, про освіту, про мову – я з родиною з власної ініціативи поїхала
в Латвію, Литву, Естонію, щоб вивчити, як мовне питання працює в цих країнах.
Нашим «гідом» був депутат Латвійського сейму. Він всюди нас возив своєю
автівкою і пояснював будь-які нюанси. Мені було цікаве все: від закону на
папері до «як це працює на вулиці серед людей чи в магазині/установі».

Перше, що мене вразило. Їдучи в його авто і роздивляючись краєвиди за вікном,
краєм вуха я вловила увімкнуту радіохвилю. Співала Лайма Вайкулє «Ах вєрнісаж,
ах вєрнісаж, какой портрєт, какой пєйзаж…». Хто народився в ссср, той знає цей
хіт і заспіває його серед ночі крізь сон. 

Я прислухалася. І яке ж було моє здивування. Лайма Вайкулє співала свій «Вернісаж» латиською мовою! 

Я висловила своє здивування. Депутат мені пояснив, що радянська зірка естради
всі пісні в Латвії співає латиською, бо таке законодавство. Я запитала, чи були
скандали з боку Лайми Вайкулє. Мені депутат сказав, що не чув про скандали. І
чому б то їй скандалити, коли вона патріот своєї країни і свідома громадянка. 

Слід зазначити, що й їхнє радіо стало об’єктом мого дослідження. Переклацала
багато їхніх радіохвиль, Лайму чула ще кілька разів з іншими піснями – і знову
латиською. Як і інших співаків - латиською і англійською.  

З того часу пройшло 5 років. З Україною повним ходом воює Московія. Воює в
першу чергу за ПАНУВАННЯ в Україні СВОЄЇ московської мови. В Україні з нагоди
відновлення 30-ліття Незалежності України від Московії запланували концерт з
українськими «хітами», які звучали за ці 30 років. Московськомовний хіт Грін
Грея завернули. Грін Грей починає скандал про ущімлєнія.

Грін Грею! Я вас не знаю, ви мене теж, але мова не про це. 30 років тому (чи
скільки там) ваша пісня, яку ви написали і співали московською мовою, була
«хітом». З того часу багато води витекло. Світ змінився. Змінилася ситуація в
країні. Змінилося життя. Змінилися закони… Якщо ви такий патріот України (як ви
стверджуєте на ТБ) – що вам заважає заспівати вашу пісню УКРАЇНСЬКОЮ МОВОЮ?
«Під дощем, під дощем почекаємо й підем…» Це був би гарний крок на зустріч
своїй країні. Пісня заграла б новими відтінками. Це привернуло б до вас увагу.

Що заважає перекласти пісню українською? 

Відсутність таланту? Зневага до країни? Зневага до Законів цієї країни?
Упоротість? Що? Що заважає патріоту України співати українською мовою?

Чому Лайма Вайкулє змогла, а ви – ні? 

Українці. Не піддавайтеся на ці провокації. Анітрохи не вагаючись, тримайте
міцно оборону.

\ii{28_07_2021.fb.nicoj_larisa.1.grin_grey_mova.cmt}
