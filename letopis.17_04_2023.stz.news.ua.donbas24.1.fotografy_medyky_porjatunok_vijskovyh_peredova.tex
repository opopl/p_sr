% vim: keymap=russian-jcukenwin
%%beginhead 
 
%%file 17_04_2023.stz.news.ua.donbas24.1.fotografy_medyky_porjatunok_vijskovyh_peredova
%%parent 17_04_2023
 
%%url https://donbas24.news/news/fotografi-pokazali-yak-mediki-rizikuyut-zittyam-zaradi-poryatunku-viiskovix-na-peredovii-foto
 
%%author_id news.ua.donbas24,veremeeva_tetjana.zhurnalist.donbas24.sumy
%%date 
 
%%tags 
%%title Фотографи показали, як медики ризикують життям заради порятунку військових на передовій
 
%%endhead 
 
\subsection{Фотографи показали, як медики ризикують життям заради порятунку військових на передовій}
\label{sec:17_04_2023.stz.news.ua.donbas24.1.fotografy_medyky_porjatunok_vijskovyh_peredova}
 
\Purl{https://donbas24.news/news/fotografi-pokazali-yak-mediki-rizikuyut-zittyam-zaradi-poryatunku-viiskovix-na-peredovii-foto}
\ifcmt
 author_begin
   author_id news.ua.donbas24,veremeeva_tetjana.zhurnalist.donbas24.sumy
 author_end
\fi

\ii{17_04_2023.stz.news.ua.donbas24.1.fotografy_medyky_porjatunok_vijskovyh_peredova.pic.front}
\begin{center}
  \em\bfseries\Large
Костянтин та Влада Ліберови разом із військовими лікарями були на передовій
\end{center}

Повномасштабна війна в Україні триває другий рік. Поки християни східного
обряду в усьому світі \href{https://donbas24.news/news/velikden-golovni-tradiciyi-i-prikmeti-svyata}{\emph{святкують Великдень}},%
\footnote{Великдень — головні традиції і заборони свята, Ольга Демідко, donbas24.news, 15.04.2023, \par\url{https://donbas24.news/news/velikden-golovni-tradiciyi-i-prikmeti-svyata}}
росіяни \href{https://donbas24.news/news/rosiyani-u-velikodnyu-nic-zruinuvali-cerkvu-na-zaporizzi-ta-obstrilyali-doneccinu-detali-foto}{обстрілюють територію нашої країни}.%
\footnote{Росіяни у Великодню ніч зруйнували церкву на Запоріжжі та обстріляли Донеччину — деталі, Наталія Сорокіна, donbas24.news, 16.04.2023, \par\url{https://donbas24.news/news/rosiyani-u-velikodnyu-nic-zruinuvali-cerkvu-na-zaporizzi-ta-obstrilyali-doneccinu-detali-foto}}

Захисники України кожного дня, за будь-яких умов ризикують власним життям.

Фотографи Костянтин та Влада Ліберови показали, як лікарі медичної служби Ульф
працюють на передовій та рятують життя військових.
