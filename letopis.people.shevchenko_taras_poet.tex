% vim: keymap=russian-jcukenwin
%%beginhead 
 
%%file people.shevchenko_taras_poet
%%parent people
 
%%url 
 
%%author_id 
%%date 
 
%%tags 
%%title 
 
%%endhead 

%https://life.pravda.com.ua/culture/2021/10/27/246302/

%%%cit
%%%cit_head
%%%cit_pic
%%%cit_text
Диявольську силу Півночі, брутальну, облудну, цинічну, забріхану, протягом
вісьмох століть незмінну в усіх своїх огидних барвах хамелеона, ненавидів він
всім серцем своїм, всею душею своєю і всім помишленієм своїм. Ненавидів, як
ненавидить людина вільна того, хто плює їй в душу; хто топче ногами її гідність
людську; хто трупами народів встелював свій шлях історичний. Ця ненависть
полумям бухає з кожного рядка, написаного ним. Та не тільки чужій деспотії
належала його ненависть.  Він твердо тямив, що «коли б не похилилися раби, то
не стояло б над Невою отих осквернених палат "деспотів". Коли б не похилилися
раби... Цих рабів, слуг чужинця бачив він подостатком на Україні. Була це
численна порода рідних по крови земляків, які "помагали москалеві
господарювати" та з матері останню свитину здирати. Не тільки Петрові-катові й
"Петровим собакам", не тільки Катерині "голодній вовчиці", слав він прокляття,
але й Галаганам і Кочубеям, сучасникам-"шашелям". До них звертався: «погибнеш,
згинеш, Україно, не стане й знаку на землі! Сама розіпнешся у злобі, сини твої
тебе убють!». Цих синів-виродків проклинав \emph{Шевченко}. Він бачив брата у кожнім
землякові, та не тоді, коли цей земляк ставав Каїном. Не коли "рідні" Каїни
продавали як "лакеї в золотій оздобі" чужого пана. Не тоді, коли пишалися
московською "кокардою на лобі", витертім з усякого почуття сорому й чести
%%%cit_comment
%%%cit_title
\citTitle{Заповіт Шевченка}, , pravyysektor.info, 10.03.2018
%%%endcit

%%%cit
%%%cit_head
%%%cit_pic
%%%cit_text
Награди вас Господи и близких вам бесконечной радостию и невозмутимым счастием
и спокойствием! Я ожил, я воскрес! и остальные дни праздника я провожу как бы в
родном семействе, между вас и Николая Осиповича, мои милые, мои добрые, мои
великодушные друзья! Я так обрадован, так осчастливлен вашим ласковым приветом,
что забываю гнетущее меня девятилетнее испытание. Да, уже девять лет как
казнюся я за грешное увлечение моей бестолковой молодости. Преступление мое
велико, я это сознаю в душе. Но и наказание безгранично, и я не могу понять,
что это значит? Конфирмован я с выслугою. Служу как истинный солдат.  Один мой
недостаток, что не могу делать ружьем так, [как] бравый ефрейтор, но мне уже 50
лет.  Мне запрещено писать стихи, я знаю за что, и переношу наказание
безропотно. Но за что мне запрещено рисовать? Свидетельствуюсь сердцеведцем
Богом – не знаю.  Да и судьи мои столько же знают, а наказание страшное! Вся
жизнь была посвящена божественному искусству. И что же? Не говорю уже о
материальной пытке, о нужде, охлаждающей сердце. Но какова пытка нравственная?
О, не приведи Господи никому на свете испытать ее, хотя с великим трудом я,
однакож, отказал себе в самом необходимом. Довольствуюсь тем, что царь дает
солдату. Но как отказаться от мысли, чувства, от этой неугасимой любви к
прекрасному искусству. О, спасите меня, или еще один год – и я погиб. Какое мое
будущее? Что у меня на горизонте?  Слава Богу, если богадельня. А может быть…
О, да не возмутится сердце ваше, мне снится иногда бедный ученик Мартоса и
первый учитель покойника Витали.  Малодушное, недостойное пророчество! но вода
на камень падает и камень пробивает
%%%cit_comment
%%%cit_title
\citTitle{22.04.1856 р. До А. І. Толстої}, Тарас Шевченко
%%%endcit

%%%cit
%%%cit_head
%%%cit_pic
%%%cit_text
В 15 лет Тарас оказался в числе дворовых Павла Васильевича Энгельгардта, и тот
обратил внимание на склонность юноши к рисованию. Энгельгардт решил сделать из
\emph{Шевченко} домашнего живописца и отдал его в обучение.  Учась, \emph{Шевченко}
познакомился со многими знаменитыми художниками, в частности с Карлом Брюлловым
и Алексеем Венециановым. Те решили помочь \emph{Шевченко} выкупиться из крепостной
неволи и начать жизнь свободного художника. Но его владелец решил извлечь из
этого максимальную выгоду, особенно увидев, какие знаменитые люди занялись
судьбой его крепостного. Общественное порицание Энгельгардту было как с гуся
вода, тем более, что в его окружении его мотивы понимали очень хорошо, а
желание академиков воспринимали как блаж
%%%cit_comment
%%%cit_title
\citTitle{Трагическая судьба Тараса Шевченко}, История России, zen.yandex.ru, 05.05.2019
%%%endcit

%%%cit
%%%cit_head
%%%cit_pic
%%%cit_text
\emph{Тарас Григорьевич Шевченко} (укр. Тарас Григорович Шевченко) — усатая легенда
Украины, тролль, лжец/правдоруб и просто добрый человек, а по совместительству
военнослужащий Российской армии. Ни разу в жизни не назвал себя «украинцем».
Также почитаем в Казахстане (настолько почитаем, что порт названный в честь
него в постсоветском Казахстане переименован из порт Шевченко, в порт Курык),
где отбывал ссылку. По основной специальности художник, но в истории и
срачевойнах известен прежде всего как поэт и писатель. Вылезает из земли к тем,
кто поест про́клятого сала.
%%%cit_comment
%%%cit_title
\citTitle{Тарас Шевченко}, lurkmore.to
%%%endcit

%%%cit
%%%cit_head
%%%cit_pic
%%%cit_text
Миф третий: образ «Великого Кобзаря». Портреты \emph{Тараса Шевченко}, на которых он
изображен одетым в «жупан и каракулевую шапку», уже на протяжении более чем ста
лет украшают классные кабинеты в украинских школах, а также используются
украинофилами на митингах и манифестациях. В сознании «украинского народа»
образ \emph{Тараса Григорьевича} неразрывно связан с образом «простого мужика из села,
в вышиванке, жупане и с обильной растительностью на лице». Эдакая «неотесанная
деревенщина, которая боролась и страдала за лишения украинского народа».
Однако, и на самом деле, \emph{Тарас Шевченко} был ещё тем модником, почитателем
светских вечеринок и «приличного общества». Двадцать семь лет из своей жизни
\emph{Шевченко} провел на территории современной Российской Федерации, где он сотни,
если не тысячи раз был гостем на различных банкетах, баллах и прочих
великосветских приемах
%%%cit_comment
%%%cit_title
\citTitle{Три мифа о Тарасе Шевченко, взрывающие мозги украинцам}, ss69100.livejournal.com, 18.07.2020
%%%endcit18.07.2020, 

%%%cit
%%%cit_head
%%%cit_pic
%%%cit_text
\emph{Тарас Шевченко} — главный поэт Украины, национальная гордость страны. Его
литературное творчество многообразно, помимо поэзии, опирающейся на украинские
фольклорные мотивы, \emph{Шевченко} много писал и на русском языке — это стихи,
около 20 повестей, переписка.  Для Украины \emph{Шевченко} имеет такое же
значение, как для России Александр Пушкин.  Его именем названы театры, площади,
улицы, учебные заведения. На Украине насчитывается 1256 памятников
\emph{Шевченко}
%%%cit_comment
%%%cit_title
\citTitle{Человек дня: Тарас Шевченко}, polit.ru, 09.03.2021
%%%endcit

%%%cit
%%%cit_head
%%%cit_pic
\ifcmt
  pic https://avatars.mds.yandex.net/get-zen_doc/2993437/pub_5ed4ad88125fcf03b498e28c_5ed4c29f095c4a0a976f9269/scale_1200
  @width 0.4
\fi
%%%cit_text
В 1858 году \emph{Шевченко} возвращается в Петербург. Он прошёлся по Академии
художеств. Его душа наполнилась хорошими, добрыми воспоминаниями. Семья вице
президента академии Фёдора Толстого встретила его радушно и тепло. Они долго
добивались освобождения \emph{Шеченко} от "солдатчины".  Академия художеств станет для
Шевченко родным домом, буквально; он поселился в одной из комнат. Начинает
заниматься гравюрой. Его радушно приглашают всюду и везде он читает свои стихи,
поёт песни. Его принимают так, что он часто смущался, будучи человеком простым
и сердечным.  В Петербургском пассаже ему устроили настоящую овацию. Тарас был
так тронут, что слёзы сами катились из его глаз
%%%cit_comment
%%%cit_title
\citTitle{Объективно о Тарасе Шевченко}, История России, zen.yandex.ru, 01.06.2020
%%%endcit

%%%cit
%%%cit_head
%%%cit_pic
%%%cit_text
205 лет назад, 9 марта 1814 года, родился знаменитый малороссийский художник и
поэт \emph{Тарас Шевченко}. Он стал знаковой фигурой в среде украинской интеллигенции,
его образ стал знаменем агрессивного украинского национал-шовинизма. Хотя сам
\emph{Шевченко} никогда не разделял русских и малороссов (южную часть русского
суперэтноса).  \emph{Тарас} родился в Киевской губернии, в семье крепостного
крестьянина. Рано осиротел и познал трудности жизни бедняка и беспризорника.
Прислуживал у дьячка-учителя, у которого выучился грамоте, затем у
дьячков-живописцев (богомазов), у которых научился первым навыкам рисования.
Был пастухом. Затем в 16 лет стал служить в семье дворянина Энгельгардта. \emph{Тарас}
показал способности в рисовании, поэтому помещик решил его обучить, чтобы
сделать домашним художником
%%%cit_comment
%%%cit_title
\citTitle{Шевченко без украинизма}, , topwar.ru, 09.03.2019
%%%endcit

%%%cit
%%%cit_head
%%%cit_pic
%%%cit_text
Пошли дальше...

Если честно, я б на месте наших «ура-патриотов» вообще исключил бы поэму
«Катерина» из школьной программы. Потому как считаю что поступки, описанные в
произведении, позорят украинцев.

И это не проступок Катерины или того самого военного офицера (москаля), а
именно вот эта всенародная травля оступившегося человека, когда человек
человеку – волк.

\obeycr
«… А соседки злые речи
с матерью заводят:
«Мол, не зря в твой дом ночами
москали приходят...»
\restorecr

Для меня, воспитанного советской литературой, с пропагандируемой ею дружеской
поддержкой, соседской взаимовыручкой (вспомним хотя бы «Тимур его команда»), и
прочими идеалами социалистического общества, читать про то, что соседи
затравили семью до такого состояния, что мать выгоняет единственную дочь с
внуком из дому - как минимум, дико.

И это благородные, гостеприимные украинцы? Желающие объединиться для борьбы с
панами за свободу и независимость. Да тут в одном селе, люди друг другу готовы
глаза повыцарапывать. Какое может быть объединение?

\obeycr
«... а соседи, а соседки
уши прожужжали.
Пересуды да насмешки
злобою повиты...»
\restorecr

Позор таким соседям. И это ж не какие то пришлые люди, нет, это те самые
односельчане, которые не одно десятилетие жили рядом с семьей Катерины бок о
бок и, возможно, по сельской традиции, вместе отмечали праздники. Но стоило
Катерине оступиться, так они тут же толкают ее в спину и «заклевывают» до
смерти.

«...Вот что делают на свете
людям сами ж люди!
Того вяжут, того режут,
тот сам себя губит..»

Ну и как относиться к таким человеческим взаимоотношениям?  Гордиться такими
предками? Ставить их в пример детям? Да мы тут, по большому счету имеем
классический пример группового буллинга (травли), того самого от которого
предостерегают современные школьные методички
%%%cit_comment
%%%cit_title
\citTitle{Тарас Шевченко. Поэма «КАТЕРИНА». Мнение обывателя}, pikabu.ru, 27.04.2021
%%%endcit
