% vim: keymap=russian-jcukenwin
%%beginhead 
 
%%file 18_02_2021.fb.fb_group.story_kiev_ua.1.podgotovka_ekzamen
%%parent 18_02_2021
 
%%url https://www.facebook.com/groups/story.kiev.ua/posts/1600690063461122/
 
%%author_id fb_group.story_kiev_ua,babarika_gennadij
%%date 
 
%%tags ekzamen,kiev,obrazovanie,studenty,ucheba
%%title Подготовка к экзамену
 
%%endhead 
 
\subsection{Подготовка к экзамену}
\label{sec:18_02_2021.fb.fb_group.story_kiev_ua.1.podgotovka_ekzamen}
 
\Purl{https://www.facebook.com/groups/story.kiev.ua/posts/1600690063461122/}
\ifcmt
 author_begin
   author_id fb_group.story_kiev_ua,babarika_gennadij
 author_end
\fi

Подготовка к экзамену

О подготовке к экзаменам в студенческие годы историй не счесть.  Случались
разные - занятные, и комичные. Так повелось, что в последний день перед
экзаменом  мы частенько готовились втроём.   Традиция, зародившаяся  на первом
курсе в институтской читалке за конспектированием первоисточников,  и
сохранялась на протяжении пяти лет. Так можно восполнить пробелы в темах и
конспектах, если таковые имелись. Да, что в этом такого необычного, если бы не
один случай. Но без  предыстории не обойтись.

В нашей группе АА-5 о количестве 30-ти человек  была одна девушка! Для
автомобильного факультета КАДИ явная аномалия, но приятная. Вообще-то, на наш
поток из 175 человек,  было, аж, четыре аномалии!  И, в нашей группе оказалась
самая красивая. Каким образом (не смотря на немаленький конкурс), а главное
зачем, девушки поступили,  по всем меркам и понятиям на чисто мужской
факультет, выяснилось  в процессе учёбы.  

А курьёз произошёл на третьем курсе на весенней сессии. Ольга,
пользовавшаяся вниманием всех 29 особей мужеского пола нашей группы, при
этом, особо не выражала своей,  напросилась к нам в компанию.

Естественно, ни кто и не думал отказывать, хотя и предполагали, что  до
подготовки к экзамену по политэкономии,  в такой компании явно дело не
дойдёт. Так оно и вышло. Для начала в конспекты по первоисточникам,
некоторое время пытались  дописывать пропущенное. Рьяные преподаватели
общеполитических дисциплин практиковали проверку конспектов
первоисточников при сдаче экзаменов, но это было скорее их особенность,
чем правило. 

Нудная работа в компании трёх парней явно не доставляла
удовольствия Ольге. Поэтому  подготовка быстро трансформировались в
питие кофе под сигаретку с неоднократным повторением, разговоры обо всём,
но не о предмете,  и легко отхватили полдня времени.  И всё бы на этом и
закончилось, если бы Ольга, глянув на часы, засобиралась уходить. Но,
перед этим бросила, мальчики мне надо подкраситься. Да, на здоровье.
Достав из девичьей сумки походно-дежурный арсенал красоты, начала
подкрашивать ресницы. 

И тут её как током ударило, а давай тебе накрашу ресницы. Без всякой скромности
замечу, у меня было чего красить. Ребята услышали, и хихикнули, а что слабо.
Куда  деваться, давай. Под обсуждение, смех и подначивание процесс  прошёл на
ура. Сразу заявила, тебе пойдут светло-голубые тени. 

Как по мне, так было всё равно. Ресницы накрашены, глаза подведены и наложены
тени на веки. И без того выразительные глаза, стали весьма ...даже затрудняюсь
подобрать слова, какие. Да, и конечно, легкая светло-бордовая губная помада,
завершила процесс моего омакияжевания. 

Осмотр в зеркале себя, конечно,  ошеломил, но, выглядел достойно, без перебора.
А кидать в меня камнями, не надо. В половой ориентации  отклонений не
наблюдалось, а вот подурачиться... да, кто откажется в студенческие годы?! 

Далее не спеша Ольга проделала свои  необходимые макияжные штрихи, на дорожку
выпила  ещё индийского растворимого кофе и выкурила сигаретку «Киев», и совсем
неожиданно, скомандовала – проведи меня до троллейбуса. Без всякого опасения за
свой вид,  пошёл провожать. И ребята хороши, ни кто не подсказал смыть макияж. 

Ладно, если бы только провожанием до троллейбуса и закончилось,  а то меня
чёрт дёрнул проехать несколько остановок. Пассажирка сзади попросила
закомпостировать талон, и в момент, когда обернулся, отметил слегка
ошарашенный взгляд женщины средних лет передающей талончик,  с тем же
выражением  забрала закомпостированный, отводя взгляд в сторону. Меня  это,
совсем не насторожило.  Сидевшая на боковом одиночном сидении  Ольга поехала
дальше до метро, махнув рукой с лёгкой улыбкой, а я вышел.   

Троллейбусная поездка среди дня не в часы пик оказалась  разминкой. На обратной
дороге решил зайти в гастроном.  В процессе выбивания чека на товар,
восседавшую в монументальной кассе пожилую  грузную кассиршу ввёл в
замешательство.   Два раза, переспрашивала, била по клавишам кассового аппарата
и периодически кидала на меня взгляд поверх плюсовых линз в очках.  Мелькнуло в
голове, наверное, уже устала, ждёт прихода сменщицы. Заветный чек в руке,  у
прилавка впереди всего пару человек,  подошла моя очередь.  

Молодая продавщица, беря чек и не отрывая от меня взгляд, резко выкрикнула по
имени сотрудницу из соседнего отдела. При этом успела переспросить, что мне
отпускать. Без задней мысли, машинально прикрыл глаза, и кивнул головой в знак
подтверждения. Открыв глаза, увидел подбегающую сбоку к ней сотрудницу с
вопросом – чего, мол, звала?! Та, глядя в мою сторону, головой ей слегка
кивнула в мою сторону и тихо сквозь зубы сказала - смотри.  Мне бы и вспомнить,
каков я.  Но, без всякого подвоха и дальше верчу головой в стороны, ожидая
отпуска взвешиваемого куска колбасы. 

Подошедшая продавщица   уставилась   гипнотическим взглядом, как бы ни веря
своим глазам.  До окончания отпуска товара  не отходила от коллеги. Про себя
отметил, во как,  ждёт, пока меня обслужит, надо же. От советских торговых
работников подобное терпение не всегда можно было ожидать.  Покупку в пакет и к
выходу из гастронома, не обратив внимания на прыснувших со смеху в след
продавщиц.  Мимолётные взгляды встречавшихся прохожих вообще меня не
настораживали. Хорошо, не далеко пришлось идти.

Глаза «открылись»  в тот момент, когда в прихожей вешал ключи возле зеркала
и машинально взглянул...   Тут и пришло осознание причины дивных взглядов,
замешательств и удивлению  от вида молодого парня в макияже в середине 70-х
годов в СССР. Отмечу, удивительно, ни кто не проронил, в мою сторону ни слова!
Наверное,  думали, парень забыл грим смыть или что-то в этом роде. На следующий
день после сдачи экзамена, за распитием традиционного кофе на Крещатике, друзья
дико потешались, слушая рассказ о вчерашнем.
