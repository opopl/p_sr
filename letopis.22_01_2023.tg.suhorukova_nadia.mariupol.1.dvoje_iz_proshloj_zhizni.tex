% vim: keymap=russian-jcukenwin
%%beginhead 
 
%%file 22_01_2023.tg.suhorukova_nadia.mariupol.1.dvoje_iz_proshloj_zhizni
%%parent 22_01_2023
 
%%url https://t.me/maripol_hope/1214
 
%%author_id suhorukova_nadia.mariupol
%%date 
 
%%tags mariupol
%%title Двое из прошлой жизни
 
%%endhead 
 
\subsection{Двое из прошлой жизни}
\label{sec:22_01_2023.tg.suhorukova_nadia.mariupol.1.dvoje_iz_proshloj_zhizni}
 
\Purl{https://t.me/maripol_hope/1214}
\ifcmt
 author_begin
   author_id suhorukova_nadia.mariupol
 author_end
\fi

Двое из прошлой жизни. 

Они воевали в 2014-м. 

Оба пошли добровольно. 

У них одинаковые имена, но характеры, как небо и земля. 

Один смешной и несуразный. 

С ним постоянно что-то случалось. 

Мог  выдумать и   приврать, чтобы казаться круче.  

Если он болел,  у него была  высочайшая температура. 

Он звонил и говорил : "У меня 41 градус. Мне нужно отлежаться. А завтра я
приду". 

Если хромал, то это старое ранение,  вражеская пуля в ноге давала о себе знать. 

"Если ничего не сделать,  ногу могут отрезать", -  смотрел на меня печальными
глазами и вздыхал. 

Его дальние  родственники периодически  находились в коме,  и он их спасал. 

"Ты знаешь, какие лекарства им нужны? Таких нигде нет. Я их за огромные деньги
достал. Бегал,  целую неделю искал"

Короче,  в одном человеке  жили  сразу три мушкетёра и братья Гримм.  

Второй -  спокойный,  сдержанный,  как будто слегка  сонный. 

Слишком принципиальный.

 Упертый, но без эмоций. 

Иногда бесил со страшной силой. 

Своей принципиальностью. 

Никогда нельзя было угадать  о чем он думает. 

В нем было что-то немецкое.  Четкое и точное. 

Ему  дали позывной - немецкое имя. 

Его практически невозможно было вывести из себя. 

Оба пришли к нам после передовой. 

Мы  знали, что подружились они  на войне. 

Наш Мариуполь,  тогда, в 2014,   был под угрозой захвата орков  и его отстояли. 

Эти двое  тоже спасали город.

Когда спасли , вернулись домой   и устроились работать на местный  канал.  

Первого я гоняла нещадно, дважды увольняла и  часто ругала. 

Он возвращался и начинал заново. 


Мне сейчас очень стыдно. 

Я смотрела видео и кричала:

 "Что ты наснимал,  гад? Откуда у тебя руки растут? Ты -  издеваешься надо
 мной?"

Он шел и переснимал. 

Потом некоторое время работал, как часы. 

Его хватало на месяц и  все начиналось сначала. 

Второй работал без сбоев. 

Долго не мог научиться. 

Мы хихикали над ним. Никто не хотел с ним монтировать. 

Слишком медленно и печально он делал сюжеты.  

А  потом  -   стал одним из лучших. 

Трудился, как метроном. Без усталости и нытья. И практически идеально. 

По ночам  писал книгу о войне. 

О том, как пошел в 14 году в добробат. 

Однажды дал мне почитать рукопись. 

Мне понравилось. 

Думала, он бесчувственный, а он оказался талантливым. 

Помню,  в его рукописи был эпизод про парня, у которого мама порвала  паспорт,
чтобы он не пошел воевать. 

Но он все равно сбежал   без паспорта.  

Хотел защищать Украину. 

И погиб в первом же бою. 

Писатель,  из моего офиса,  сделал вывод: 

"Значит мама была права, что не пускала его"

В этой фразе  было что-то детское и беззащитное. 

 В конце февраля 22 года  - эти двое, как и многие мариупольцы,  оказались в
 блокаде. 

Я ничего о них не знала. И, если честно, даже не стремилась узнать.  

На тот момент у меня были свои проблемы, и я почти забыла об этом дуэте. 

Только через несколько месяцев, после того как я выбралась из блокады, узнала,
что первый - смешной и несуразный, тоже смог  выйти  из города и пошел воевать
за Украину. 

Он некоторое время звонил моему племяннику и  рассказывал необыкновенные
истории о своих приключениях. 

Мы смеялись: "Болтун - находка для шпиона"

А потом его телефон перестал отвечать . 

Заглох. "Абонент не може прийняти ваш дзвінок"

Уже несколько месяцев не может. 

Кажется, он  был в Донецкой области на передовой. 

Мы боимся, что он погиб. 

А его никто не будет искать. 

О нем и в мирное время мало кто беспокоился. 

Он слишком отчаянный,  безбашенный и одинокий. 

Второй тоже - исчез. 

Для нас всех. 

Он остался в Мариуполе и перешёл на сторону оккупантов. 

Никто сначала в это не поверил. 

Подумали - враньё. 

В 2014 он воевал за Украину. 

Писал о войне. 

Отказывался нарушать свои принципы. 

И вдруг стал коллаборантом. 

А потом это  подтвердилось.  

Он сам и подтвердил. 

Мариупольские друзья с одинаковыми именами.   

Из прошлой жизни. 

Разные - как небо и земля.
