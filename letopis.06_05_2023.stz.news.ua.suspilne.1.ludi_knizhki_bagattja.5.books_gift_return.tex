% vim: keymap=russian-jcukenwin
%%beginhead 
 
%%file 06_05_2023.stz.news.ua.suspilne.1.ludi_knizhki_bagattja.5.books_gift_return
%%parent 06_05_2023.stz.news.ua.suspilne.1.ludi_knizhki_bagattja
 
%%url 
 
%%author_id 
%%date 
 
%%tags 
%%title 
 
%%endhead 

\subsubsection{Книжки і дарують, і повертають}

Департамент культури вийшов на зв'язок. Я почала збирати якісь дані про своїх
співробітників, їх у мене 96. Слава богу всі живі, були поранення, але всі
вижили, всі зараз знаходяться більш менш в безпеці. Мало хто виїхав в Україну,
хтось виїхав до Росії, хтось виїхав за кордон. Але велика кількість залишилась
в Маріуполі і на мій превеликий жаль, працює там на окупанта — відкривають
бібліотеки. Про когось я знала, так скажемо, хто чекав, але тихенько сидів,
хтось відкрився з такого боку, що знайти й не знала, й чекати, не чекала.

\ii{06_05_2023.stz.news.ua.suspilne.1.ludi_knizhki_bagattja.5.books_gift_return.pic.9}

Донька знайшла роботу у Дніпрі, вона вчителька української мови й літератури.
Відкрились центри \enquote{Я Маріуполь}, почали функціонувати центри \enquote{Я Маріуполь-
культура}. Тоді ми Приїхали до Дніпра з донькою у липні. Директорка
департаменту каже, що потрібно збирати книжки, адже нічого немає, а треба ж
буде з чимось повертатись.

Чесно кажучи, я не дуже вірила в цю затію. Але написали в соцмережах про акцію:
маріупольська бібліотека потребує літератури, надсилайте книги тільки
українською мовою. Там все спалюється, там дуже багато постраждалих книжок. В
ці часи бомбардувань їх розтягнули мешканці, бо людям треба було палити чимось
свої костри. І там вже люди на себе не були схожі: усі меблі з бібліотек і
техніку розграбували одразу, а книжки — все пішло на розпал багаття. Що там
залишилось, я навіть не знаю. Співробітники, з якими ще на початку
спілкувалась, сказали, що повибирали книжки, що більш менш вціліли, і принесли
до бібліотек, які там почали працювати.

\ii{06_05_2023.stz.news.ua.suspilne.1.ludi_knizhki_bagattja.5.books_gift_return.pic.10}

\begin{qqquote}
Але зараз знаю, що всю українську літературу, всі українські книжки списують,
куди вони їх дівають, я не знаю. В мене немає ні фото, ні відео, куди вони їх
дівають, але я розумію, що коли ми повернемось, їх там не буде.	
\end{qqquote}

Коли ми почали акцію у вересні, я думала, що 100-200 книжок надішлють. Але коли
я почала отримувати посилки, я зрозуміла, що мені кудись треба їх дівати, бо у
квартирі, я їх не вміщу. Тому я звернулась до колег в Дніпровській центральній
бібліотеці, вони надали невелике приміщення, куди я складаю ці книжки.

Один читач прийшов у бібліотеку ім. Короленка й взяв книжку 16 лютого 2022
року. І коли виїхав, він забрав її з собою. І коли я збирала книжки, він
надіслав мені її сюди, коли я. Повернув книжку до бібліотеки.

Надсилають звідусіль, з усіх куточків країни просто люди, книговидавці,
благодійні фонди, бібліотеки. Були з Коломиї, Олександрії, Дніпро 500
екземплярів книжок передали. У Тернополі оголошували такі акції про допомогу
бібліотеці Маріуполя.

Отак ми збираємо книжки, працюємо дистанційно, ведемо соціальні мережі, збираю
електронні версії книжок наших маріупольських поетів і письменників. У нас дуже
потужний сайт, де знаходиться електронна бібліотека.

\begin{qqquote}
Всі чекаємо на перемогу й на повернення. Хоча додому вже казати не можна, бо
дому немає, але Маріуполь — це наш дім. І ми чекаємо на повернення до
Маріуполя.	
\end{qqquote}
