% vim: keymap=russian-jcukenwin
%%beginhead 
 
%%file 08_12_2020.news.ru.mail_ru.2.putin_vaccine_nagradil
%%parent 08_12_2020
 
%%url https://news.mail.ru/politics/44462621/
 
%%author 
%%author_id 
%%author_url 
 
%%tags 
%%title Путин наградил врачей за вклад в борьбу с пандемией коронавируса
 
%%endhead 
 
\subsection{Путин наградил врачей за вклад в борьбу с пандемией коронавируса}
\label{sec:08_12_2020.news.ru.mail_ru.2.putin_vaccine_nagradil}
\Purl{https://news.mail.ru/politics/44462621/}

Президент России Владимир Путин подписал указ о награждении медицинских
работников за самоотверженность в борьбе с коронавирусом. Соответствующий
документ опубликован на официальном портале правовой информации в понедельник,
7 декабря.

\ifcmt
pic https://retina.news.mail.ru/pic/4f/dc/image44462621_78a5af588d2cdd1ee23a6ff2a01af301.jpg
\fi

Так, главный врач московской клинической больницы № 1 «Группы компаний «Медси»
Татьяна Шаповаленко была награждена орденом Дружбы. Еще четыре сотрудника этой
больницы, а также заместитель главврача по хирургии Клиники научной медицины
награждены медалью ордена «За заслуги перед Отечеством» II степени.

Орден Пирогова получили 14 медицинских работников, а сотрудница Ставропольского
научно-исследовательского противочумного института Ольга Семенова была
удостоена медали Луки Крымского.

Награды за большой вклад в организацию работы по предупреждению и
предотвращению распространения заболевания получили 20 человек.

Также за большой вклад в организацию работы по выпуску препарата для лечения
COVID-19 орденом Почета был награжден председатель совета директоров
курганского ОАО «Синтез» Дмитрий Зубов, а президент ОАО «Синтез» Рустем Муратов
и вице-президент по развитию бизнеса и исследованиям АО «Алиум» Алексей Чупин
были удостоены ордена Дружбы.

24 ноября сообщалось, что Владимир Путин наградил 98-летнюю Зинаиду Корневу,
собравшую более 4,5 млн рублей для помощи медработникам в разгар пандемии.
