% vim: keymap=russian-jcukenwin
%%beginhead 
 
%%file 17_12_2021.stz.news.lnr.lug_info.1.podarki_zaschitniki_novyj_god
%%parent 17_12_2021
 
%%url https://lug-info.com/news/shkol-niki-alchevska-peredali-zashitnikam-respubliki-novogodnie-podarki
 
%%author_id 
%%date 
 
%%tags 
%%title Школьники Алчевска передали защитникам Республики новогодние подарки
 
%%endhead 
\subsection{Школьники Алчевска передали защитникам Республики новогодние подарки}
\label{sec:17_12_2021.stz.news.lnr.lug_info.1.podarki_zaschitniki_novyj_god}

\Purl{https://lug-info.com/news/shkol-niki-alchevska-peredali-zashitnikam-respubliki-novogodnie-podarki}

Школьники Алчевска в рамках инициированной общественным движением (ОД) \enquote{Мир
Луганщине} акции \enquote{С Новым годом, солдат!} передали военнослужащим Республики
подарки. Об этом сообщила пресс-служба ОД.

\enquote{Учащиеся общеобразовательных учреждений города Алчевска, представители
Алчевского казачьего кадетского корпуса приняли участие в акции \enquote{С Новым годом,
солдат!} и передали активистам проекта \enquote{Волонтер} общественного движения \enquote{Мир
Луганщине} открытки и рисунки для защитников Республики. Воспитанники
учебно-воспитательного комплекса \enquote{Надежда} собрали для защитников Республики
посылку с предметами первой необходимости}, - говорится в сообщении.

Работник исполкома Алчевского территориального отделения \enquote{Мира Луганщине} Анна
Корсун отметила, что учащиеся города уже не первый год принимают участие в
подобной предновогодней акции, \enquote{проявляя творческий подход и энтузиазм в
поздравлении тех, кто сейчас стоит на защите рубежей Родины}.

\ii{17_12_2021.stz.news.lnr.lug_info.1.podarki_zaschitniki_novyj_god.pic.1}

По мнению активистов, проведение подобных акций с детства учит ребят делать
добрые дела и благодарить тех людей, которые защищают Отечество.

\enquote{Хочется пожелать солдатам скорейшего завершения войны и мирного неба над
головой, чтобы каждый Новый год они отмечали в кругу своей семьи}, – пожелала
координатор проекта \enquote{Молодая гвардия} в Алчевске Юлия Шинкарева.

Ранее проект \enquote{Волонтер} ОД \enquote{Мир Луганщине} запустил акцию \enquote{С Новым годом,
солдат!} по сбору новогодних открыток и подарков для военнослужащих ЛНР.

ОД \enquote{Мир Луганщине} – крупнейшее республиканское движение, в рядах которого
насчитывается около 111 тыс. человек. Движение было создано перед выборами
депутатов Народного Совета ЛНР, которые состоялись 2 ноября 2014 года. На
выборах депутатов Народного Совета второго созыва, которые прошли 11 ноября
2018 года, движение заручилось поддержкой 74,12\% избирателей, получив 37
мандатов в пятидесятиместном парламенте ЛНР. Председателем движения на
состоявшемся в Луганске 17 февраля 2018 года третьем общереспубликанском съезде
\enquote{Мира Луганщине} единогласно был избран в то время исполняющий обязанности
главы ЛНР Леонид Пасечник.
