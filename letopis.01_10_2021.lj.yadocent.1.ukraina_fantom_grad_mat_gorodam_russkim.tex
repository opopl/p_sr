% vim: keymap=russian-jcukenwin
%%beginhead 
 
%%file 01_10_2021.lj.yadocent.1.ukraina_fantom_grad_mat_gorodam_russkim
%%parent 01_10_2021
 
%%url https://yadocent.livejournal.com/1379582.html
 
%%author_id 
%%date 
 
%%tags 
%%title Украина – это фантом. Какой град еси «мать городам русским»? 
 
%%endhead 
\subsection{Украина – это фантом. Какой град еси «мать городам русским»?}
\label{sec:01_10_2021.lj.yadocent.1.ukraina_fantom_grad_mat_gorodam_russkim}

\Purl{https://yadocent.livejournal.com/1379582.html}

\ifcmt
  author_begin
    author_id lj.yadocent
  author_end
\fi

Каждому русскому человеку знаком перифраз из древнерусской литературы, гласящий
что: «Киев – это мать городов Русских». Современная украинская идеология
перекручивает данный исторический факт таким образом, что вместо русского
национального единства на выходе получается целая антирусская доктрина.

\ifcmt
  tab_begin cols=3
     pic https://ic.pics.livejournal.com/kerchenit/89147395/26113/26113_800.png

     pic https://ic.pics.livejournal.com/kerchenit/89147395/26453/26453_original.jpg
     @caption Император Византии Роман Лакапин

     pic https://ic.pics.livejournal.com/kerchenit/89147395/26694/26694_800.jpg
     @caption Крещение Руси Великим князем Киевским Владимиром Святославовичем
  tab_end
\fi

Украинская идеология вбивает в головы современных жителей Украины следующие
идиомы, связанные с Киевом: если Киев – это мать городов русских, значит, Киев
– это колыбель русской цивилизации; если Киев находится на территории Украины,
значит, Украина – это наследница русского государства; если Киев – это колыбель
русского государства и Украина – это наследница русского государства, значит
Украина и колыбель, и наследница, и вообще – сама суть древнего русского
государства.

Таким образом, просто манипулируя историческими фактами о том, что якобы Киев –
это мать городам русским, и кто обладает Киевом – тот обладает Русью,
современные украинские идеологи превозносят Украину в глазах её населения и
приравнивают Россию к побочному продукту жизнедеятельности древнерусских
князей. То есть у нас есть Киев и мы великие, а у вас нет «матери городов
русских» – значит, вы ничтожные и никчёмные. А теперь обо всём по порядку.

Изначально не было никакой Киевской Руси, была Русь – на Руси была столица, в
столице правил князь, который был старшим в роду. Спустя столетия междоусобных
войн, разорения Руси монгольскими ордами и литовскими войсками для
разграничения исторических периодов существования государства историкам
потребовалась удобная терминология. Так период, когда Новгород был столицей
Руси, получил название – Новгородская Русь, когда столицей был Киев – Киевская
Русь, Владимир – Владимирская Русь и т.д.

Вместе с тем, как бы современные украинские идеологи не пытались доказать
древность и самобытность украинского языка и украинской культуры, ни одна
археологическая находка, сделанная в Киеве, не доказывает существование ни
украинского языка, ни национальной самоидентичности. Напротив, Киев доказывает
нам совершенно обратное.

В торговом договоре, заключённом в 944 году, между князем Игорем и византийском
императором Романом Лакапином нет ни одного слова об «украх», «украинцах» или
«рутенах». Зато в данном договоре присутствует «Русь», «русин», «страна
русская», «великий князь русский».

Точно также, когда в 988 году, при внуке Игоря святом князе Владимире
крестилось киевское население, это событие вошло в древние письменные памятники
под названием «Крещение Руси», а не «крещение Укрии», «Украины» или «Рутении».

Когда при мудром сыне его, князе Ярославе I, создавался в Киеве первый
законодательный кодекс, название своё он получил не какое-либо иное, а «Русская
правда», в которой также нет ни одного слова об «украх», «украинцах» или
«Украине».

\ifcmt
  tab_begin cols=2

     pic https://ic.pics.livejournal.com/kerchenit/89147395/27064/27064_original.jpg
     @caption Великий князь Киевский Ярослав Владимирович.

     pic https://ic.pics.livejournal.com/kerchenit/89147395/27283/27283_original.jpg
     @caption Анна Русская — дочь Великого князя Киевского Ярослава Владимировича.

  tab_end
\fi

Дочь Ярослава I, вышедшая замуж за короля Франции Генриха I, известная
французской истории под именем «Анн дэ Руси», то есть Анна русская. Но нынешние
украинские идеологи и мифотворцы попросту не хотят замечать те исторические
факты, которые совершенно не укладываются в концепцию современной Украины.

"Украинская пропаганда старается уверить, будто «север» этнически и идейно был
враждебен «югу». Главный их довод – походы двух Суздальских князей в XII веке
на Киев. Но борьба князей, их личных выгод и честолюбий была на Руси явлением
повсеместным. Нет того направления, в котором бы не наступали княжеские
дружины, добывая стол своему князю. Само деление князей на северных и южных до
последних лет XII века является искусственным. Основание местных княжеских
линий – это явление более позднее. В X-XII веках тот же князь княжил то на
севере, то на юге. У правящего на севере князя бывали всегда братья или
сыновья, сидевшие на южных престолах. Например:

1) у Великого князя Киевского Святослава Храброго, – один сын, Владимир Святой,
княжил на севере в Новгороде;

2) Великий князь Владимир Святой, из 12 своих сыновей, – семерых посадил
княжить в северных городах, а именно: в Новгород, Полоцк, Ростов, Смоленск,
Псков, Ростов же, Муром и Суздаль;

3) из пяти сыновей Ярослава Мудрого – трое княжили на севере в Новгороде и
Смоленске;

4) сам Ярослав Мудрый, олицетворявший собой величие Руси киевского периода, –
прожил на севере 28 лет.

Из книги князя А. М. Волконского «Имя Руси в домонгольскую пору», издательское
общество «Единство», Прага, 1929 год.

Разобравшись с тем историческим фактом, что древний град Киев не имеет никакого
отношения к племенам ни «укров», ни «украинцев» необходимо детально
остановиться на его культурном и политическом значении.

Бесспорно, Киев был столицей русского государства: в Киеве находился
великокняжеский трон, княжеская казна, канцелярия и все сопутствующие
государственные органы. В Киеве также находился митрополит и прочее высочайшее
духовенство церкви. Но является ли Киев колыбелью русской цивилизации? Является
ли он первой столицей и городом, без которого бы не было ни Руси, ни
современной России? Нет, и нет – категорическое.

«Мать городов русских» – это перифраз из древнерусской литературы, используемый
для обозначения Киева. Оригинальный фрагмент из «Повести временных лет», 882
года, звучит следующим образом: «и сел Олег, княжа, в Киеве, и сказал Олег: "Да
будет это мать городам русским"»...

То есть Олег пришёл в Киев с войском откуда-то из другого места, разместился в
Киеве и сказал, что теперь не то, другое место, а Киев будет «Матерью городам
русским».

Другими словами, фраза, приписываемая Олегу, представляет собой семантическую
кальку с греческого слова метрополия, то есть «мать», «город» и означает не что
иное, как провозглашение Киева столицей Руси. Но, если Олег всего лишь перенёс
столицу из другого места в Киев, откуда тогда есть, пошла, началась Русь?
Откуда тогда, есть, пошла земля русская?

...«и пришёл Рюрик с Русью княжить»... Именно с этого момента современная
историческая наука ведёт отсчёт существования классического древнерусского
государства. А пришёл Рюрик со своим войском, называемым в летописях Русью
княжить в город Ладога, которая и стала первой матерью городам русским.

\ifcmt
  tab_begin cols=3
     pic https://ic.pics.livejournal.com/kerchenit/89147395/27826/27826_800.png
     pic https://ic.pics.livejournal.com/kerchenit/89147395/28152/28152_800.png
     pic https://ic.pics.livejournal.com/kerchenit/89147395/28371/28371_800.png
  tab_end
\fi

С принятием в конце X века Русью христианства, Киев становится резиденцией
русского митрополита. Именно с X века совпадение политического и церковного
центров в сочетании с длительным периодом единовластия киевских князей привело
к формированию на Руси устойчивого института столицы, что было нетипичным для
большинства европейских стран того времени. После того, как в 1240 году Русь
была разграблена и выжжена монголо-татарскими ордами на передний план русского
государства выходит славный град Владимир, основанный в 1108 году Владимиром
Мономахом. Ещё с 1157 года, когда сюда из Суздали перенёс свою резиденцию князь
Андрей Юрьевич Боголюбский, Владимир выходит на политический Олимп русского
государства, однако отечественная история признаёт Владимир «матерью городам
русским» лишь с 1243 года. Сюда же, во \sbfu{Владимир}, как в новую столицу Руси, была
перенесена резиденция русского митрополита из Киева.

\ifcmt
  tab_begin cols=4
     pic https://ic.pics.livejournal.com/kerchenit/89147395/28418/28418_800.png
     pic https://ic.pics.livejournal.com/kerchenit/89147395/28792/28792_800.png
     pic https://ic.pics.livejournal.com/kerchenit/89147395/29125/29125_800.png
     pic https://ic.pics.livejournal.com/kerchenit/89147395/29209/29209_800.png
  tab_end
\fi

Спустя полтора столетия «матерью городам русским», а значит, и столицей
русского государства, становится \sbfu{Москва}, куда также переносится
резиденция русского митрополита. В 1712 году, по прошествии 323 лет, столица
России была перенесена Петром I из Москвы в \sbfu{Санкт-Петербург}. В
город, специально основанный молодым русским царём в качестве новой столицы. На
целых 206 лет город Петра становится центром и светской, и научной и
религиозной жизни. Лишь свержение русского самодержавия определило то, что в
1918 году «мать городам русским» была снова перенесена из Петрограда в
\sbfu{Москву}, где столица русского государства и резиденция русского
митрополита находятся и по сей день.

Таким образом, в разное время в русском государстве существовало как минимум
пять официальных столиц, среди которых Киев не был ни первой «матерью городам
русским», ни даже второй. Более того, Киев не был даже первой столицей
Украинской Советской Социалистической Республики, государства, чью территорию,
экономические богатства и население в 1991 году унаследовала Украина XXI века.

Первой столицей советской Украины стал \sbfu{Харьков}. Красивый большой
город, с широкими проспектами, зданием «Госпрома», где и заседало советское
правительство Украинской Советской Социалистической Республики, до переноса
столицы в Киев. Более того, по воспоминаниям очевидцев Киев 20-30-х XX века был
намного меньше, беднее и не шёл ни в какое сравнение с Харьковом, истинной и
первой столицей Украинской Советской Социалистической Республики.

\ifcmt
  tab_begin cols=2
     pic https://ic.pics.livejournal.com/kerchenit/89147395/29591/29591_800.jpg
     @caption Вид на Харьковский Госпром и площадь Конституции, конец 1920-х годов.

     pic https://ic.pics.livejournal.com/kerchenit/89147395/29897/29897_800.jpg
     @caption Памятник Богдану Хмельницкому в Киеве: «Богдану Хмельницкому. Единая неделимая Россия».
  tab_end
\fi

Ещё одна идиома, которую внедряют в головы жителей современной Украины, гласит
о том, что Киев якобы был столицей русского государства гораздо дольше по
времени, чем любой другой город России был опорой самодержавия. Этот аргумент
сошедших с ума украинских историков также является ложью:

\begin{itemize} % {
\item \sbfu{Ладога}, а ныне город \sbfu{Старая Ладога} – был «Матерью городам русским» 2 года;

\item \sbfu{Новгород}, а ныне \sbfu{Великий Новгород} – 18 лет;

\item \sbfu{Киев} – 361 год;

\item \sbfu{Владимир} или \sbfu{Владимир на Клязьме} – был «матерью городам русским» 146 лет;

\item \sbfu{Москва} – 322 года (с 1389 по 1712 год) и 100 лет (с 1918 по 2018 год), а
всего – 422 года.

\item \sbfu{Петроград} или \sbfu{Санкт–Петербург} – неизменно оставался столицей на протяжении
206 лет.
\end{itemize} % }

Как видно, даже арифметика у современных украинских политтехнологов и историков
странная. Киев был «матерью городам русским» 361 год, а Москва – 422 года, но
при этом Киев, якобы самая долго существовавшая столица Руси.

И так во всём. Куда ни глянь, везде украинцы – это самая древняя и великая
нация, везде и всегда существовал украинский язык, а все киевские князья –
поголовно истинные украинцы и патриоты Украины.

И неважно, что ни князья, ни население древнего русского государства знать не
знали о существовании какой-то древней Украины. Неважно, что в древности не
существовало и не дошло до нас ни одного упоминания об украинском языке.
Неважно, что даже в XIX веке большинство населения современной Украины,
созданной благодаря стараниям Владимира Ильича Ленина, называло себя русскими,
русским народом или русским племенем.

Это всё неважно, ведь Украина – это фантом. Фантом, созданный на теле России,
взращённый за счёт России, паразитирующий на теле России, но при этом
ненавидящий своё русское прошлое и отрекающийся от своей русской души.

