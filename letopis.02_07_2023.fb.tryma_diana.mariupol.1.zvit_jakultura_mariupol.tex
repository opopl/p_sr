%%beginhead 
 
%%file 02_07_2023.fb.tryma_diana.mariupol.1.zvit_jakultura_mariupol
%%parent 02_07_2023
 
%%url https://www.facebook.com/100002192860709/posts/pfbid0iQAPWzpZgAojVdPFuTzvoCd5gKrrMTfWDMtiHBdJXAMwcPDsC2Ey8D723UuZ2rQil
 
%%author_id tryma_diana.mariupol
%%date 02_07_2023
 
%%tags 
%%title І як обіцяла, наступним постом - звіт
 
%%endhead 

\subsection{І як обіцяла, наступним постом - звіт}
\label{sec:02_07_2023.fb.tryma_diana.mariupol.1.zvit_jakultura_mariupol}

\Purl{https://www.facebook.com/100002192860709/posts/pfbid0iQAPWzpZgAojVdPFuTzvoCd5gKrrMTfWDMtiHBdJXAMwcPDsC2Ey8D723UuZ2rQil}
\ifcmt
 author_begin
   author_id tryma_diana.mariupol
 author_end
\fi

І як обіцяла, наступним постом - звіт.

Якщо робити висновок за рік, то що до результату роботи з червня 2022 року
галузі Культура та проєкту ЯКультура Маріуполь вийде така картина: 

\emph{\color{blue}\bfseries Центральна міська публічна бібліотека}

Оголошено акцію по збору книжок для відновлення книгозбірень після деокупації
міста. (акція по збору триває). На 01.07.2023 року отримано більш ніж 2 000
примірників різножанрової літератури. Дарувальники – благодійні фонди та
організації, книговидання, автори, бібліотеки Україні та Європи, навчальні
заклади та ВНЗ, не байдужі українці з різних куточків країни та ін. Триває збір
оцифрованих версій творів краєзнавчого характеру для поповнення електронної
бібліотеки, що розміщена на сайті (за рік зібрано біля 20). За рік отримано
біля 150 посилок. Компанія \enquote{Нова пошта} надає безкоштовну доставку книжок для
відновлення книгозбірень. Участь та проведення  різноманітних заходів
(телеміст, флешмоби, літературні квести, літературні години, онлайн-марафон,
зустрічі з письменниками, дискусійні клуби, панельна дискусія тощо.) Всього –
35 власних заходів (у т.ч. 16 онлайн.) Участь у фахових та суміжних заходах,
проектна діяльність. Ведеться робота в соціальних мережах та підтримується
робота веб – сайту установи.

\emph{\color{blue}\bfseries Маріупольський краєзнавчий музей:}

Збір інформації про стан музейних будівель, майно та фонди, моніторинг
інформації, щодо музейних предметів. Розпочато відновлення даних фондової
колекції. Відновлено каталог художніх  та графічних робіт музею. Створено
меморіальні виставки до Дня річниці трагедії Драматичного театру,  памяток
історії та архітектури \enquote{Вписані в історію - зруйновані війною}, до річниці
окупації Маріуполя \enquote{Маріполь \#86 НаЖиво}. Проведено виставку \enquote{Зруйнована, але
не знищена. Грецька культурна спадщина з фондів Маріупольського краєзнавчого
музею}.  Створено каталог  Всеукраїнського культурно - мистецького проекту
\enquote{Маріуполь душа України}. 

Відновлено цифрові копії колекції поштових листівок кінця ХІХ – поч. ХХ ст.
Участь в онлайн-тренінг, воркшопах, семінарах. Розпочато  збір
експонатів; проєкт \enquote{Маріуполь - невтрачена пам'ять}, зі збору усних свідчень та
спогадів маріупольців.

Відновлено публікації музею на сторінках соціальних мереж: Фейсбук, Телеграм,
Інстаграм.

\emph{\color{blue}\bfseries Маріупольська художня школа ім. А. І. Куїнджі}

Єдина з мистецьких шкіл, що продовжила свою роботу після окупації міста. На
території України залишилося чотири викладача і разом з учнями у кількості 40
людей. Робота  проводилась у змішаній формі, дистанційній та очній. Очна форма
доступна була у чотирьох містах Львів, Дніпро, Одеса, Кропивницький, в містах,
де знаходились наші викладачі при центрах підтримки \enquote{Я Маріуполь}. У Львові у
палаці Культури ім Гната Хоткевича відкритий очний клас. Підписано меморандум
про співпрацю з Львівським органим залом. Проведено дві виставки \enquote{Нащадки
спадщини Куїнджі}, \enquote{Діти Війна} в галереї органного залу. Залучені від донорів,
волонтерів художні матеріали дозволили займатися арт терапією не тільки з
дітьми, а також дорослими маріупольцями, проведено безліч майстер класів, по
розпису кераміки \enquote{Трипільські візерунки}, \enquote{Петриківський розпис}, \enquote{Візерунки
писанки}. Східний варіант вручив відзнаку \enquote{Незламні зі сходу} Маріупольська
художня школа ім. А. І. Куїнджі.

\emph{\color{blue}\bfseries Маріупольська камерна філармонія} відновила роботу в форматі камерного оркестру
\enquote{Ренесанс} з лютого 2023 року. Проведено 10 концертних виступів, концерти в
Кривому Розі, Львові. Започатковано спільні проєкти та напрацювання на
наступний концертний сезон. \enquote{Міський палац культури \enquote{Український дім}} став
базою для проекту ЯКультура Маріуполь, соціальна реабілітація через культуру

\emph{\color{blue}\bfseries Театральний колектив  \enquote{Театр авторської п'єси \enquote{Conception}}} займається
поширенням маріупольської культури та підтримкою внутрішньо переміщених осіб,
має значні напрацювання  в культурі під час війни та промоції  міста Марії.
Проведено велику кількість відкритих показів та виїздів містами України.
Створенні відеопривітань та відеоверсій віршів. Театральна школа для
маріупольських дітей. Хореографія для групи соціальної реабілітації. 

ЯКультура Маріуполь при центрах Відвідування театрів, філармоній, музеїв,
бібліотек, екскурсії, мітингів, флешмоби, акції, фотовиставки, театральні
вистави, спрямовані на поширення  інформації про злочини Росії, клуби людей
золотого віку, пізнавальна-розважальні програми для дітей та дорослих,
знайомство з творчістю відомих особистостей. Робота і підтримки ЗСУ – в'язання
маскувальних сіток,  виготовлення подарунків та оберегів для воїнів, відвідання
поранених воїнів у шпиталі. м. Вінниця - з вересня  2022 р.  374 заходів, які
відвідали 4734 осіб.  м. Дніпро - з червня 2022р - 518 заходів, 7450 осіб. м.
Київ - з червня 2022 року -  546 заходів, 8520 осіб. м. Івано-Франківськ -  298
заходів, які відвідали 2956 осіб. м. Кривий Ріг - з серпня 2022 р. 524 заходів,
які відвідали 5240 осіб. м. Кропивницький  з вересня 2022 р. проведено 421
захід, які відвідали 3980 осіб. м. Львів з липня 2022 р. 308 заходів, які
відвідали 3889 осіб. м. Одеса з липня 2022р. 449 заходів, які відвідали 5150
осіб. м. Рівне з  січня 2023 року було проведено 388 заходів  обсягом 3535
людей. , в м. Тернопіль з листопада  471 захід, які відвідали 4115 осіб.  м.
Черкаси з вересня 2022 р. по червень 2023 р. проведено 163 захід, які відвідали
2154 осіб. м. Чернівці з  жовтня 2022 р.  проведено 391 захід, які відвідали
3795 осіб.

Далі буде....
