% vim: keymap=russian-jcukenwin
%%beginhead 
 
%%file 15_02_2022.stz.news.ua.strana.1.vse_v_shrony
%%parent 15_02_2022
 
%%url https://strana.news/news/376734-vo-lvove-opublikovali-kartu-s-nehotovymi-k-vtorzheniju-bomboubezhishchami.html
 
%%author_id harchenko_aleksandra
%%date 
 
%%tags bomboubezhische,lvov,ukraina
%%title "Все в схроны!" Во Львове опубликовали карту с непригодными для спасения бомбоубежищами
 
%%endhead 
 
\subsection{\enquote{Все в схроны!} Во Львове опубликовали карту с непригодными для спасения бомбоубежищами}
\label{sec:15_02_2022.stz.news.ua.strana.1.vse_v_shrony}
 
\Purl{https://strana.news/news/376734-vo-lvove-opublikovali-kartu-s-nehotovymi-k-vtorzheniju-bomboubezhishchami.html}
\ifcmt
 author_begin
   author_id harchenko_aleksandra
 author_end
\fi

Во Львове, который считается самым \enquote{безопасным} городом на случай \enquote{вторжения
России}, тем не менее вспомнили о бомбоубежищах.

Львовская мэрия опубликовала карту из шести тысяч \enquote{бункеров}, которую львовяне
раскритиковали. И оказалось, что большинство из них находятся в подвалах домов
и непригодны для укрытия: там нет ни вентиляции, ни света, ни тепла, часть
затоплены и захламлены. Другие власти давно продали, и там находятся бары,
магазины и квартиры.

В сети многие шутят: \enquote{В схроны! Все в схроны} (по-местному, \enquote{крыивки}, где
прятались от Красной армии бандеровцы во время Второй Мировой войны). 

\subsubsection{\enquote{Авиаудар сделает укрытие братской могилой}}

Наши источники в мэрии говорят: еще с советских времен никто не занимается
бомбоубежищами, многие незаконно ушли с молотка, а карту сделали для отвода
глаз.

\ii{15_02_2022.stz.news.ua.strana.1.vse_v_shrony.pic.1}

\enquote{Во Львове многие дома валятся, весь центр надо чинить, подвалы и
чердаки - это вообще ж.па. Гнилые трубы, проваленные фундаменты - это
повсеместно. Где жители почистили-починили, там нормально. А в основном все
убитое. Так что бомбежка превратит Львов в кучу обломков, под которыми
похоронит жителей, а \enquote{бомбоубежища} станут братскими могилами. Для
бункера нужно очень крепкое специальное перекрытие. Но для полтонной бомбы и
это не помеха. В подвалах можно прятаться при артобстрелах, которые вряд ли
угрожают Львову}, - говорит наш источник в местном управлении по чрезвычайным
ситуациям.  

Формально районные администрации обязали жэки и ОСМД подписать акты
обследования подвалов, и если нужно - почистить их и приготовить для
\enquote{укрытия}.  Иными словами, повесили подготовку \enquote{бомбоубежища}
на жителей.

\enquote{Жители сами вынуждены приводить в порядок подвалы - это если власти их не
продали. Максимум, что сделала мэрия - повесила таблички \enquote{Укрытие}. Одно из
бомбоубежищ, которое находится на балансе приватизированного завода, - это
свалка разбитых ртутных ламп, и даже недолго посидев в таком укрытии, можно
отравиться  парами ртути. Другой участок над большим укрытием, например,
продали под застройку жилого дома фирме свекрови экс-нардепа партии
\enquote{Самопомощь} Ирины Сусловой (а в мэрии говорят, что якобы будут там строить
школу)}, - говорит источник. Кстати, по закону нельзя продавать бомбоубежища,
которые относятся к стратегической инфраструктуре, нельзя продавать.

Но наши источники в мэрии говорят, что власти поставили на поток их продажу,
списывая их как непригодные для использования (то есть специально доводят их до
состоянии мусорника).

Оказалось, что, например, в жилмассиве Малоголосковском, где живет больше 10
тысяч человек, нет ни одного бомбоубежища.

А пригодных укрытий во Львове - раз, два и обчелся.

\enquote{Высота подвала для укрытия должна быть не менее 170 сантиметров, двери должны
быть железные, помещение очищено, проходы - свободны. Должны быть туалеты и
канализация, запасной выход}, – говорит Олег Дубачик, начальник отдела
управления по вопросам чрезвычайных ситуаций.

\subsubsection{\enquote{Света нет годами - проводка сгнила}}

Львовяне в соцсетях возмущаются, что власть не готовит бомбоубежища накануне
объявленного \enquote{вторжения Путина}, которое в России отрицают, а Запад
накручивает. 

\enquote{По карте в моем районе на Чупрынки, 96 есть укрытие на 118 человек. А это уже
давно заброшенный старый дом с выбитыми стеклами. Когда-то он был жилой, до
1939 года вообще особняком был. Потом там был офис, но уже лет 10 он закрыт.
Когда-то там жили мои друзья, я заходила и видела, как там внутри. Так в нем и
20 человек не поместится, не то что 118}, - пишет под постом в группе \enquote{Горячей
линии города} Наталья Розбиская.

\ii{15_02_2022.stz.news.ua.strana.1.vse_v_shrony.pic.2}

\enquote{Самый центр, Валовая, 16 - подвал не годится для укрытия, там все завалено
хламом, потому что раньше там жили бездомные. А по карте считается укрытие на
более чем 300 человек!} - вторит Галина Балучинская.

\enquote{Кто обследовал подвал в доме 37 по ул. Стороженко и решил, что там может быть
укрытие для 114 человек? Подвал в этом доме в критическом состоянии, там
проваливается земля}, - негодует Елена Заят-Сорокивская.

\enquote{Подвалы? Есть подвалы в центре, в которые лучше туда не заходить. Так и
поступают жэковцы. В некоторых подвалах годами освещения нет, потому что
проводка сгнила, а жэку все равно}, - возмущается Зеня Грушевская.

\enquote{Большинство бомбоубежищ находятся в полуподвальных помещениях панельных домов,
в цокольных этажах, со стеклянными окнами (при любом взрыве здание развалится и
всех похоронит заживо). А на балансе числятся у предприятий, которых уже годами
не существует. Например, хранилище по улице Скорини, 38, балансодержатель -
ЛКП, которое ликвидировано давно}, - добавляет Владимир Кириллов.

\enquote{А кто перед публикацией карты проверял эти места на пригодность? В мой подвал
ни одна комиссия не приходила, а если там будут укрываться люди, то это будет
массовое захоронение, а не укрытие. Огромные трещины в стенах, подмокающий
фундамент, крыша без опоры. Велком}, - делится София Ткаченко. 

\enquote{Представляю, целый подъезд, или даже несколько подъездов, многоэтажки в
подвале. Жесть. Сколько того подвала? А еще с тревожными чемоданами}, - пишет
Леонид Широков.

\enquote{Разъяснения провели, наклейку бомбоубежища приклеили, премии выписали,
средства потратили. Не, ну а шо - иначе не умеют}, - иронизирует Николай
Гиренко.
