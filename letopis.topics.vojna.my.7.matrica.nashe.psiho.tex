% vim: keymap=russian-jcukenwin
%%beginhead 
 
%%file topics.vojna.my.7.matrica.nashe.psiho
%%parent topics.vojna.my.7.matrica.nashe
 
%%url 
 
%%author_id 
%%date 
 
%%tags 
%%title 
 
%%endhead 

\paragraph{17:21:55 17-10-22 Валентина Александровна}

Konstantin Kuzmin

Путинский “техасский стрелок”
Вы наверно знаете про Ошибку техасского стрелка?
Жил в Техасе стрелок, который всегда попадал в яблочко. Стена сарая где он жил была вся дырявая от выстрелов и изрисована мишенями. И все пули попадали точно в десятку. Но суть была в том, что техасский стрелок не умел стрелять и палил наугад. А уже потом вокруг дырок в стене рисовал мишени.
Ошибка техасского стрелка — это частный случай когнитивного искажения, которое связано со склонностью людей подтверждать свою точку зрения. Оно вызвано предрасположенностью человека воспринимать только ту информацию, которая поддерживает и подтверждает его убеждения. Все остальные факты, которые ставят его точку зрения под сомнение или опровергают её, он игнорирует или оспаривает.
Вы наверное тоже поражались упертости и изворотливости путинистов. Как они умеют закрыть глаза на очевидные факты и вылепить конфетку из любого говна, что наложит Власть.
Поведение этих "патриотов" напоминает того техасского стрелка. У них нет какой-то внятной логичной концепции, описывающей происходящее в Мире. У них нет критериев что хорошо и что плохо. Они подстраивают концепции под события, когда те уже произошли. Рисуют мишени вокруг дырок в стене, трактуют любые события так чтобы они всегда доказывали их правоту.
Они выбрали примитивную и простую идею слепого патриотизма, где надо просто верить, что наша страна самая лучшая и, соответственно, мы тоже. Путин великий правитель, который не ошибается, а Россия всегда права, чтобы она ни делала. Это для них сверхценная идея на которой держится их картина мира. И они оправдают все что угодно, нарисуют мишень в нужном месте, лишь бы доказать истинность своей веры.
С такими людьми бесполезно вести диспуты, приводить факты, использовать логику. Они будут отбиваться от этого с помощью тактики техасского стрелка.
Люди никогда не откажутся от своей веры, пока она представляет для них жизненную ценность.
