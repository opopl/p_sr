% vim: keymap=russian-jcukenwin
%%beginhead 
 
%%file letters.mariupol.demidko_olga.10
%%parent letters.mariupol.demidko_olga
 
%%url 
 
%%author_id 
%%date 
 
%%tags 
%%title 
 
%%endhead 

... в Інтернет Архіві вже 86 Ваших записів ) можете самі перевірити. Я спочатку
записую в Архів, а потім вже роблю зшивку ну а потім отримую один великий файл,
з якого я друкую книжку. Щодо зібрання Ваших творів, напевне, ще тиждень-два і
буде готово, бо дійсно дуже-дуже багато всього, і я все роблю дуже-дуже
ретельно, тому досить повільно. Але книжки на декілька томів точно буде! і от... я думаю.. нащо я
це роблю. А дійсно. Нащо. Кому воно потрібно. Наскільки я зрозумів, Вам це не
дуже потрібно. Надіслати їх Вам очевидно неможливо, показати - теж... Начебто Ви десь недалеко в фізичному просторі... в одному місті...
десь там ходите, щось робите, займаєтесь своїми справами...
але ментально Ви нескінченно далеко... ну, так вже вийшло і це мені так здається вже нічим не змінити... Мені чи потрібно - в принципі ну таке... грошей я на цьому не заробляю, лише
час витрачаю, хоча вже ж таки попутно розробляю цікаві програмні трюки. Маріупольцям - вони іншим зайняті. Киянам - киянам Маріуполь
старий ... якось їм не до того... Тому... в принципі це нікому не потрібно,
такий от висновок. Але тоді нащо сидіти днями і ночами, переписуючи Ваші пости, якщо це нікому не потрібно?
А от нащо. Просто знаєте дуже-дуже цікаво. Отримую велике естетичне задоволення від Вашого стилю )
дуже вже щиро все, вишукано, просто дуже класно ) просто еталон української літературної мови, от чесно ) 
ну і крім того, дуже цікаво, що ж там було в тому Маріуполі раніше )) і чим більше я заглиблююсь, тим все цікавіше стає )
от. Роздрукую собі потім книжки, буду друзям показувати і можливо на ніч
собі читати, якщо буду мати час на це ) оце от знаєте хтось колись... взяв написав дуже класні речі ...
а вони висять в повітрі і жалко буде якщо пропадуть... ось я їх і підібрав ) вже не пропадуть )
