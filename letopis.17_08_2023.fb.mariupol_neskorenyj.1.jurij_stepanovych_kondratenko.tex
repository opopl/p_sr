%%beginhead 
 
%%file 17_08_2023.fb.mariupol_neskorenyj.1.jurij_stepanovych_kondratenko
%%parent 17_08_2023
 
%%url https://www.facebook.com/100066312837201/posts/pfbid0SaHWR9on46sYJGzyhSZTpwZW5fGgSnsTc7VLnaWJRsRGNUzbmGCUsnSUuJoATeqAl
 
%%author_id mariupol_neskorenyj
%%date 17_08_2023
 
%%tags 
%%title Юрій Степанович Кондратенко
 
%%endhead 

\subsection{Юрій Степанович Кондратенко}
\label{sec:17_08_2023.fb.mariupol_neskorenyj.1.jurij_stepanovych_kondratenko}

\Purl{https://www.facebook.com/100066312837201/posts/pfbid0SaHWR9on46sYJGzyhSZTpwZW5fGgSnsTc7VLnaWJRsRGNUzbmGCUsnSUuJoATeqAl}
\ifcmt
 author_begin
   author_id mariupol_neskorenyj
 author_end
\fi

📍Ще один учасник проєкту  \enquote{Маріуполь нескорений}  маріупольський
художник Юрій Степанович Кондратенко. 

Талановитий графік, майстер плакату Юрій Степанович Кондратенко все творче
життя відображає значущі події минулого і сьогодення.  Для лаконічності плакату
він завжди знаходить цікаве вирішення ідеї.

Майстер працює і у живопису і у акварелі.🎨 Його найулюбленіші жанри – пейзаж і
натюрморт. В творах Юрія Степановича відчувається вдумливий підхід к відбору
форми та кольору. Його роботи вражають цікавими світло-тіньовими переходами і
вдалими композиційними рішеннями.

🖼 Остання персональна виставка Юрія Степановича була присвячена його 80-річчу
і проходила у маріупольському художньому музеї ім. А. І. Куїнджі у 2019 році.
Дуже запам'яталась відвідувачам тієї виставки його картина \enquote{Церква  Різдва
Пресвятої Богородиці}.  В ній вся любов митця до історії Маріуполя. Адже у цій
церкві хрестили маленького Архипа Куїнджі. Працюючи над створенням цього твору
, Юрій Степанович вивчав  архіви маріупольського краєзнавчого музею, вдивлявся
в старі світлини нашого міста. Тому такою достовірною була ця картина.  На
жаль, доля цього твору на цей час невідома.

😪Війна вигнала  Юрія Степановича з рідного міста, зараз він мешкає у
Чернігівській області. Але майстер не припиняє творчої роботи. З його картинами
можна буде познайомитися на виставці \enquote{Маріуполь нескорений}, відкриття якої
відбудеться 21 серпня  поточного року у Київській Галереї мистецтв \enquote{Лавра}.

Департамент культурно-громадського розвитку Маріупольської міської ради\par
Diana Tryma\par
Галерея мистецтв \enquote{Лавра}\par
Костянтин Чернявський\par
Група \enquote{Національна Спілка Художників України}\par
Маріупольська міська рада\par
Місто Марії\par
КУ \enquote{Маріупольський краєзнавчий музей}\par
\#Маріупольнескорений \#виставка \#культурнадеокупація \#Маріуполь \#Київ \#художники\par
