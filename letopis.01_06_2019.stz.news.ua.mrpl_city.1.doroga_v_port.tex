% vim: keymap=russian-jcukenwin
%%beginhead 
 
%%file 01_06_2019.stz.news.ua.mrpl_city.1.doroga_v_port
%%parent 01_06_2019
 
%%url https://mrpl.city/blogs/view/doroga-v-port
 
%%author_id burov_sergij.mariupol,news.ua.mrpl_city
%%date 
 
%%tags 
%%title Дорога в порт
 
%%endhead 
 
\subsection{Дорога в порт}
\label{sec:01_06_2019.stz.news.ua.mrpl_city.1.doroga_v_port}
 
\Purl{https://mrpl.city/blogs/view/doroga-v-port}
\ifcmt
 author_begin
   author_id burov_sergij.mariupol,news.ua.mrpl_city
 author_end
\fi

% 1 - Ресторан "Приморский"
\ii{01_06_2019.stz.news.ua.mrpl_city.1.doroga_v_port.pic.1}

На старых планах Мариуполя эта дорога именовалась \textbf{Нижним портовским шоссе.}
Начиналось шоссе там, где современный проспект Металлургов круто ниспадает к
гостинице \enquote{Турист}, а вторая его оконечность находилась близ теперешней
троллейбусной остановки \enquote{Морвокзал}. Появилось оно, вероятнее всего,
одновременно с началом строительства порта у Зинцевой балки. Трудно сказать,
что было раньше - Нижнее портовское шоссе или параллельный ему отрезок
Екатерининской железной дороги от станции Мариуполь до станции Мариуполь-порт.
В мариупольских анналах сохранилась запись: \emph{\enquote{6 марта 1911 года владелец
портовского магазина В. И. Рыбальченко открыл автомобильный маршрут \enquote{Город –
порт} с остановками у часовни, у собора, у своего магазина}}. Часть этого
маршрута пролегала как раз по шоссе, упомянутого выше.

\textbf{Читайте также:} 

\href{https://archive.org/details/16_06_2018.sergij_burov.mrpl_city.primorskij_bulvar}{%
Приморский бульвар, Сергей Буров,	mrpl.city, 16.06.2018}

% 2 - Санаторий со львами
\ii{01_06_2019.stz.news.ua.mrpl_city.1.doroga_v_port.pic.2}

В середине 30-х годов прошлого века был проложен одноколейный трамвайный путь
по улице Котовского, и дальше - вдоль моря к порту. Это был 4-й маршрут. По
нему летом до войны ходили экзотические трамваи с открытым верхом. Разминовка
встречных трамваев происходила на разъездах. Их было несколько, один из них -
\enquote{третий} - находился неподалеку от бывшей железнодорожной школы. Вспоминается
картина. Лето, жара, трамвай останавливается у 3-го разъезда в ожидании
встречного, движущегося из порта. Пассажиры высыпают на улицу. Кто-то
отправляется в рядом стоящий магазин, чтобы разжиться бутылкой пива или ситро,
кто-то идет к торговцу семечками, тщедушному лилипуту со сморщенным старушечьим
личиком, иные покуривают папиросы и махорочные самокрутки - у кого что есть.
Наконец появляется ожидаемый вагон. Вагоновожатые обменивались жезлами,
пассажиры занимали места в вагонах, движение продолжается.

% 3 - Санаторий ЦК Угольщиков
\ii{01_06_2019.stz.news.ua.mrpl_city.1.doroga_v_port.pic.3}

Трамвайный маршрут № 4 прекратил свое существование после того, как 17 марта
1971 года вместо трамваев в порт по Приморскому бульвару, - т.е. по бывшему
Нижнему портовскому шоссе, - стали ходить троллейбусы маршрута № 4 \enquote{Драмтеатр -
Азовское морское пароходство}. Трамвайные пути были частично демонтированы.
Слово \enquote{частично} употреблено справедливости ради - поскольку на улице
Котовского проржавевшие рельсы можно было увидеть в окружении брусчатки через
много лет после пуска троллейбусов 4-го маршрута.

\enquote{Нижнее портовское шоссе} - \enquote{Приморский бульвар}, между ними были и другие
названия приметного городского объекта Мариуполя. В середине 30-х годов, когда
вдоль шоссе на пригорках было построено не\hyp{}сколько ведомственных санаториев и
домов отдыха, появилось новое название - улица Санаторная. В 1949 году в
преддверии 70-летия со дня рождения генералиссимуса, вождя всего прогрессивного
человечества то ли по указанию \enquote{сверху}, то ли по собственной инициативе,
решено было дать улице Санаторной новое имя - проспект И. В. Сталина. Дать-то
дали, а когда посмотрели на булыжную мостовую всю в колдобинах и ямах,
несколько десятилетий не знавшую ремонта, на покосившиеся разномастные заборы,
заросли бурьяна вдоль тротуаров, редкие фонарные столбы с разбитыми лампами, то
пришли в ужас. О том, как была разрешена эта щепетильная проблема, рассказала в
свое время ныне покойная доцент, кандидат технических наук Лена Назаровна
Кудрявцева, в описываемый период ее отец Назар Львович Кудрявцев занимал
высокий пост в нашем городе. Она поведала:

\begin{quote}
\em
- Назар Львович часто ездил в Москву \enquote{выбивать} деньги для города. Посредником
в этом деле было Постоянное представительство УССР при Совмине СССР. В этом
учреждении он попросил денег на хотя бы асфальтирование проспекта Сталина. Там
ему сказали, что такую проблему может решить только Берия. И дали телефон этого
всемогущественного человека. Папу терзали сомнения: неизвестно, чем мог
закончиться этот разговор - выделением денег на реконструкцию проезжей части
проспекта или ссылкой в места не столь отдаленные за плохое состояние его, тем
более, носящее имя Великого Вождя. Папа все же решился позвонить. Трубку снял
сам Берия. Назар Львович сказал, что он - председатель Ждановского исполкома, и
что в городе есть проспект имени товарища Сталина, но из-за отсутствия средств
невозможно его заасфальтировать. Берия произнес лишь одно слово: \enquote{Хорошо}, и
тотчас положил трубку. Папу охватило волнение. Что означало это \enquote{хорошо}?
Оставалось только ждать. Трудно передать какие эмоции пришлось пережить папе.
Но через сутки он узнал о выделении нашему городу определенной суммы для
асфальтирования проспекта.
\end{quote}

Кстати, очень долго это была единственная дорога с асфальтовым покрытием в
исторической застройке города. Было большим удовольствием промчаться на
велосипеде по гладкой поверхности проспекта имени Сталина от начала до его
конца, вдыхая свежий морской воздух...

Иосиф Виссарионович отошел в мир иной, а его место занял, как известно, Никита
Хрущев. Повременив немного, он осудил культ личности, и слетели с постаментов и
огромные фигуры вождя в долгополой шинели, и его бюсты с маршальскими
регалиями, а с ними и таблички в Мариуполе с надписью \enquote{Пр. им. Сталина}. Их
место заняли надписи \enquote{Санаторный проспект}...

18 января 1966 года первым секретарем Ждановского горкома КПУ был избран
Владимир Михайлович Цыбулько. Он рьяно взялся за благоустройство города. В
перечень намеченных им работ попал и Санаторный проспект. Там к тому времени
асфальт, положенный стараниями Кудрявцева, износился, сквозь дыры на нем
выглядывала старая брусчатка. Был выполнен проект реконструкции проспекта, вся
его длина была распределена между промышленными предприятиями и строительными
организациями. И... все началось с раскачки. Дела шли вяло. Новому секретарю
горкома это не понравилось. Ходила в народе то ли быль, то ли легенда, как в
один осенний дождливый день Владимир Михайлович вызвал на совещание
руководителей, причастных к реконструкции. Они прибыли в горком на служебных
\enquote{Волгах}. Чтобы сильно не замочить туфли, быстренько проскочили по мокрому
тротуару в здание. А навстречу к ним спускался по лестнице прихрамывая
Цыбулько. Он был в резиновых сапогах... На этот раз совещание прошло на ходу под
проливным дождем, от места, где сейчас находится гостиница \enquote{Турист}, до
судоремонтного завода. Реконструкция после этого еще как оживилась. Когда она
завершилась, бывшие шоссе, затем улица, потом проспект превратились в
\textbf{Приморский бульвар.}

\textbf{Читайте также:} 

\href{https://mrpl.city/news/view/rusalochka-pennaya-vecherinka-i-morskoj-fud-kort-mariupol-gotovitsya-k-gobyfest-video}{%
Русалочка, пенная вечеринка и \enquote{морской} фуд-корт: Мариуполь готовится к \enquote{GobyFest}, Анастасія Селітріннікова, mrpl.city, 30.05.2019}
