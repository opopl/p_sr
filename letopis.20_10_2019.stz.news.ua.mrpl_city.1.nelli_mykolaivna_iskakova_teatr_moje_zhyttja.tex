% vim: keymap=russian-jcukenwin
%%beginhead 
 
%%file 20_10_2019.stz.news.ua.mrpl_city.1.nelli_mykolaivna_iskakova_teatr_moje_zhyttja
%%parent 20_10_2019
 
%%url https://mrpl.city/blogs/view/nelli-mikolaivna-iskakova-teatr-i-e-moe-zhittya
 
%%author_id demidko_olga.mariupol,news.ua.mrpl_city
%%date 
 
%%tags 
%%title Неллі Миколаївна Іскакова: "Театр і є моє життя"
 
%%endhead 
 
\subsection{Неллі Миколаївна Іскакова: \enquote{Театр і є моє життя}}
\label{sec:20_10_2019.stz.news.ua.mrpl_city.1.nelli_mykolaivna_iskakova_teatr_moje_zhyttja}
 
\Purl{https://mrpl.city/blogs/view/nelli-mikolaivna-iskakova-teatr-i-e-moe-zhittya}
\ifcmt
 author_begin
   author_id demidko_olga.mariupol,news.ua.mrpl_city
 author_end
\fi

\ii{20_10_2019.stz.news.ua.mrpl_city.1.nelli_mykolaivna_iskakova_teatr_moje_zhyttja.pic.1}

24 жовтня святкуватиме ювілей людина, яка вже \emph{50 років} віддано служить
маріупольському театру. Вона завжди чуйна, відповідальна і дуже надійна.
Робочий кабінет \textbf{Неллі Миколаївни Іскакової} постійно заповнений людьми: одні
звертаються за рішенням виробничих проблем, інші – за порадою і допомогою в
особистих справах. І вона вирішує, радить, допомагає...

Народилася наша героїня в центрі Росії – в Тулі. Батько був учасником
партизанського руху в Брянських лісах, має низку нагород – орденів і медалей.
Мама була домогосподаркою – мудрою і турботливою. Прожила довге життя (98
років). Неллі з дитинства любила театр. Особливо подобалося слухати передачі
\enquote{Театр біля мікрофона}.

\ii{20_10_2019.stz.news.ua.mrpl_city.1.nelli_mykolaivna_iskakova_teatr_moje_zhyttja.pic.2}

Під час навчання дівчинка брала участь у шкільній самодіяльності, де в журі
була народна артистка Алла Демідова, вона помітила у Неллі талант. Пізніше
Неллі Миколаївна ще зустріне цю видатну артистку в інституті, де вона буде
вести гурток і намагатиметься переконати дівчину пов'язати своє життя з
театром.

\ii{20_10_2019.stz.news.ua.mrpl_city.1.nelli_mykolaivna_iskakova_teatr_moje_zhyttja.pic.3}

Втім, спочатку Неллі закінчила Тульський педагогічний інститут ім. Льва
Миколайовича Толстого (філологічний факультет). До речі, ім'я видатного
письменника червоним рядком проходило крізь життя нашої героїні. Після
університету вона працювала в школі імені Л. Толстого. Її батьки жили в Ясній
Поляні, де Неллі Миколаївна народила свого сина Андрія. Цікаво, що своїх дітей
– \textbf{Наталю та Андрія} – Неллі Миколаївна разом з чоловіком назвали на честь
головних героїв роману-епопеї Л. М. Толстого \enquote{Війна і мир}.

Під час навчання в університеті студентка відвідувала курси радіодиктора,
найбільше їй подобалося читати новини. Також вона закінчила курси медичних
сестер.

До Маріуполя приїхала у 1968 році вже з чоловіком – \textbf{Іскаковим Рудольфом}. Він
був актором, вдвох вони і вирішили спробувати свої сили у міському театрі. Їхня
сім'я стала чудовим прикладом творчого союзу.

\ii{insert.read_also.demidko.otchenashenko}
\ii{20_10_2019.stz.news.ua.mrpl_city.1.nelli_mykolaivna_iskakova_teatr_moje_zhyttja.pic.4}

Неллі Миколаївна була прийнята в театр на посаду помічника головного режисера з
літературної частини в 1969 році. Маючи вищу гуманітарну, але не спеціальну
освіту, їй довелося цю професію швидко освоювати в процесі роботи. І головним
\enquote{двигуном} стали небайдужість, любов до театру, прагнення професійно
вдосконалюватися. Вона освоювала гори спеціальної літератури, отримувала
консультації у завлітів інших театрів, вчилася у Всесоюзному інституті
підвищення кваліфікації в Москві і стала професіоналом. Неллі Миколаївна
забезпечувала театр новими п'єсами: шукала нові твори вітчизняних і зарубіжних
драматургів, пропонувала для постановки найбільш вдалі. 
