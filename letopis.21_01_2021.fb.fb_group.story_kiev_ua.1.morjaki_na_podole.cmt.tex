% vim: keymap=russian-jcukenwin
%%beginhead 
 
%%file 21_01_2021.fb.fb_group.story_kiev_ua.1.morjaki_na_podole.cmt
%%parent 21_01_2021.fb.fb_group.story_kiev_ua.1.morjaki_na_podole
 
%%url 
 
%%author_id 
%%date 
 
%%tags 
%%title 
 
%%endhead 
\zzSecCmt

\begin{itemize} % {
\iusr{Vitali Andrievski}

Молодец Петр. Я подолянин, но к Морполиту никакого отношения не имел. Зато
много лет работал на Днепре на круизных теплоходах, в основном на ВУЧЕТИЧЕ. Да
тогда Днепр был очень оживленный. Начальником Главречфлота был Н. А. Славов,
благодаря которому и Днепр и многое на Подоле оживилось. Жаль, что после него
все это пришло в запустение, пассажирского флота практически не существует

\begin{itemize} % {
\iusr{Віта Дімінська}
\textbf{Vitali Andrievski} Мій батько працював в відділі кадрів при А. Н. Славові. Ще був теплохід 25 з'їзд

\iusr{Vitali Andrievski}
\textbf{Віта Дімінська} 

був не тiльки 25 зiзд, на якому до речi, менi теж довелося працювати. Загалом
було 12 круiзних теплоходiв. На жаль зараз немае жодного з них


\iusr{Віта Дімінська}

Як казав мій батько, було три флагмани Главрічфлота :25, ВУЧЕТИЧ... третього не
пам'ятаю! На вихідних ходили на Канів, там і зустрічались.


\iusr{Vitali Andrievski}
\textbf{Віта Дімінська} 

було загалом 12 теплоходiв 301 i 302 проектiв. Першим був у 1976 роцi \enquote{Евгений
Вучетич} а потiм \enquote{25 сьезд}, \enquote{Ленин}, \enquote{советская Россия}, \enquote{добролюбов}. Это
были 301 проекта а потом пошли 302 проекта \enquote{ Глушков}, \enquote{Ватутин}, \enquote{Рыбалко},
\enquote{Ватченко}, \enquote{кошевой}, \enquote{лавриненков} и последний был \enquote{Т. Шевченко}. Загалом 12
теплоходiв. Жодного на жаль зараз на Днiпрi не залишилось

\end{itemize} % }

\iusr{Евгений Мартынович}
Спасибо за содержательный рассказ!

\iusr{Елена Свец Баркан}

Я в свое время работала в училище на Андреевской и была в совете молодых
педагогов Подольского района, отвечала за культмассовую работу. Организовывала
вечера отдыха, как они тогда назывались. Проходили они чаще всего в кафе
\enquote{Ярославна} на Константиновской. Молодые педагоги обычно были женского пола и,
для того, чтобы вечера проходили весело и приятно, приходилось прибегать к
помощи комсоргов КВВМПУ. Ходила к ним на проходную, договаривалась о количестве
ребят. Не знаю, может кто-то таким образом свою судьбу устроил?)

\begin{itemize} % {
\iusr{Петр Кузьменко}
\textbf{Елена Свец Баркан} не сомневаюсь! @igg{fbicon.wink} 

\iusr{Анатолий Кашпур}
Так вы, Елена, свахой работали?  @igg{fbicon.beaming.face.smiling.eyes} 

\iusr{Елена Свец Баркан}
\textbf{Анатолий Кашпур} скорее организатором мероприятий (массовиком-затейником) @igg{fbicon.grin} 

\iusr{Нелли Кузьменко}
\textbf{Елена Свец Баркан} Не поверишь, я на танцы ходила только тогда, когда Петюня на вахта стоял дежурный по клубу.

\iusr{Нелли Кузьменко}
\textbf{Елена Свец Баркан} И когда наши ребята, с Петькиного класса, на дискотеки играли, группа \enquote{Наследники}

\iusr{Роман Воловенко}
\textbf{Елена Свец Баркан} 

тогда модно было проводить случки между муж и жен уч заведениями зарождались
новые семьи нас авиакурсантов возили на случку в пердильное училище в Кр Рогу.

\end{itemize} % }

\iusr{Ольга Беляева}

А я с подружками на дискотеку в КВВМПу ходила... Потом выпускной, ребята в
форме с аксельбантами и кортиком-красавцы!!! А еще розы шикарные у главного
входа!!!! Как вчера...


\iusr{Марина Бортницкая}

Спасибо, чудесно написали повествование, Я жила во 2 номере Андреевского
спуска, мои окна выходили на Военно-морское училище, часто смотрела, с братом на
подготовки к парадам, жаль, всё ушло, и якоря убрали, только остались пару кустов
роз,,,

\begin{itemize} % {
\iusr{Петр Кузьменко}
\textbf{Марина Бортницкая} это мой родной дом. О нём есть мой отдельный пост в группе.

\iusr{Марина Бортницкая}
\textbf{Петр Кузьменко} 

спасибо поищу, с удовольствием прочитаю, на Андреевском спуске 2 жили с 1974 года
до отселения 1990 года, училась в 20 школе, мой брат в 100 школе, каждый раз, когда
бываю на Андреевском находит грусть, и в моем доме сейчас типа модное заведение
реберня Пьяная вишня,

\iusr{Петр Кузьменко}
\textbf{Марина Бортницкая} 

просто нажмите на мою аватарку в группе. Увидите все мои публикации.
Практически в каждом моём посте есть упоминание о моём родном доме.

\iusr{Марина Бортницкая}
\textbf{Петр Кузьменко} спасибо, мне интересно

\iusr{Марина Бортницкая}
\textbf{Петр Кузьменко} Спасибо большое, нашла, очень интересно
\end{itemize} % }

\iusr{Оксана Тернавская}
Спасибо большое Вам за рассказ от семьи военного моряка)))

\iusr{Олександр Громов}
Я со второго выпуска, 1972 год. Когда мы поступили было только два здания.

\iusr{Олег Гринюк}

Я жил в этом доме Волошская - Григория Сковороды, ранее
Набережно-Никольская 12/4, т. е. до 1977 года, когда наш дом и был передан КВВМПУ, и
поступал в 1975 экзамены сдавали в Лютеже.

\begin{itemize} % {
\iusr{Павел Пауль}
\textbf{Олег Гринюк} дом известный. Жили там замечательные люди.

\iusr{Олег Гринюк}
\textbf{Павел Пауль} 

дом Водников, капитаны Теплоходов, мой отец закончил в 1955 КИевский речной
техникум, потом ЛИВТ БЫЛ Начальником ОТК КССРЗ им. Сталина, потом просто
судостроительный завод., а в доме действительно жили Прекрасные люди, научные
работники, профессора, механики СТешек-барж, их капитаны, двор был очень дружный.


\iusr{Павел Пауль}
\textbf{Олег Гринюк} 

брат моего деда на Ленкузне мастером котельного цеха был. Марчук Евгений
Григорьевич. Содеди его Русаковы. Начальником Гимс был Виктор. Часто бывал там
в детстве.


\iusr{Олег Гринюк}
\textbf{Павел Пауль} 

Русаковы жили в нашем подъезде, Вова был гребцом бойдарочником чемпионом мы
дружили, потом они жили там где Ген. Прокуратура на Подоле, Раиса Соя гребчиха то
же чемпионка и ее семья мои соседи по коммуналке, да были люди в наше время....


\iusr{Павел Пауль}
\textbf{Олег Гринюк} я старше, а то могли бы и встречаться. Батя мой одно время тоже там жил. Он механиком на теплоходах работал.

\iusr{Олег Гринюк}
\textbf{Павел Пауль} значит моего отца точно знал.

\iusr{Павел Пауль}
\textbf{Олег Гринюк} однозначно. Директором завода помню Матяш был, деканом института Платонов А. М.

\iusr{Олег Гринюк}
\textbf{Павел Пауль} это друзья моего отца, я их хорошо помню, пацаном был,

\iusr{Павел Пауль}
\textbf{Олег Гринюк} на охоте пацаном с ними рос. Начальником пароходства Бобровникова ещё был. Марчук Вильям Павлович-отец мой.

\iusr{Олег Гринюк}
\textbf{Павел Пауль} Бобровникова если не ошибаюсь Анатолием звали.

\iusr{Павел Пауль}
\textbf{Олег Гринюк} Бобровникова Николай Андреевич.
\end{itemize} % }

\iusr{Ольга Кирьянцева}

Спасибо большое Вам за рассказ. Так замечательно написано! С огромной любовью и
гордостью за свое училище. Я жила некоторое время у родственников на Подоле,
все пути мои проходили через Красную площадь. И мне всегда хотелось хоть одним
глазком подглядеть, где и как учатся эти молодые, красивые, подтянутые
ребята-курсанты). Это ж столько девчачьих сердечек замирало и таяло при виде
тех красавцев!)) Знаю по себе)). А если удавалось попасть на их построение на
площади, то это был настоящий празник для души)). Спасибо! Вы были (и
остаётесь, я уверена,) - классными!

И с праздником Вас и всех Ваших друзей - выпускников училища.

\iusr{Елена Лясковская}

Да, попасть к вам на дискотеку было нелегко ! Но мне с подружкой однажды
повезло в далёком 1985 году @igg{fbicon.thumb.up.yellow}  @igg{fbicon.woman.dancing} 

\iusr{Ольга Писанко}
А мы были пару раз на дискотеке с подругой в году 79!

\iusr{Недоля Алексей}

А у нас со второго курса 12 роты, 1 группы Рижского Мореходного Училища в
Киевский Морполит перевёлся в 1987 году Литовченко Нормунд. Удивительная
история: я коренной Киевлянин уехал в Ригу, он, Латыш в Киев.

Хотелось бы узнать его судьбу. Может кто учился с ним вместе?


\iusr{Юлия Лепашева}

А у нас в доме батюшка живет, Сейчас батюшка, а в конце 80 ых бегал в форме
Морполита.:):):) Наверное полит. работники и духовники миссию выполняют
одинаковую.:):):)

\begin{itemize} % {
\iusr{Петр Кузьменко}
\textbf{Юлия Лепашева} об этом я писал в комментариях к посту о библиях в нашей группе. Еще в 1980-м олимпийским году я собирался поступать в Духовную семинарию.  @igg{fbicon.hands.shake} 
\end{itemize} % }

\iusr{Ілона Луценко}

Мои дедушка с бабушкой жили на Спасской улице и всегда гуляли в парке моряков.
Дедушка очень хотел что я его Внучка вышла замуж за военного моряка. Его мечта
сбылась через много лет.

\iusr{Alik Gurevich}

Пётр! Можно поправить несколько неточностей в Вашем замечательном
повествовании... С 1-го июня 1967 года началось формирование училища,
торжественное открытие состоялось 1-го октября того же года! Знаю точно, так
как в январе 1967 года был направлен в школу ОСНАЗ ВМФ и до середины июня
проходил обучение и службу там, в здании на Красной площади. А после, нас
перевели оканчивать обучение на Рыбальский .... Так как начался ремонт зданий и
образование училища.

\begin{itemize} % {
\iusr{Петр Кузьменко}
\textbf{Alik Gurevich} 

спасибо за уточнение, которое уверен с удивлением воспримут наши выпускники.

\iusr{Владимир Картавенко}
\textbf{Петр Кузьменко} 

вначале был командующий Днепровской военной флотилией матрос Полупанов, потом
младший морполит и две школы на Рыбальском. Решением Щербицкого В.В. подписаны
постановления о создании КВВМПУ и 316 УО ОСНАЗ в/ч 20884 ЧФ.

\end{itemize} % }

\iusr{Владимир Решетило}
Спасибо за воспоминания. Немного грустно, что так быстро всё проходит. 1974-1978.

\iusr{Valentina Belkova}
Эти исторические факты нужно передавать из поколения в поколение. Спасибо вам!

\iusr{Alla Zgurzhnitsky}
Я помню, моя одноклассница просила меня идти с ней мимо училища. Ей нравились курсанты.

\iusr{Valentina Belkova}

Мой муж рассказывал нашему сыну и про Реч. вокзал, Ленинскую кузню и про
Могилянку, но к сожалению ему было не интересно. Это только наше поколение
сейчас вспоминает и интересуется.

\iusr{татьяна неверовская}
\textbf{Valentina Belkova} но все равно рассказывать надо, постепенно всему свое время..... !

\iusr{Ирина Шушпанова}

А я выпускница Киевского судостроительного техникума в 1984 г получила в
награду мужа - замполита (выпуск КВВМПу 1985 г)


\iusr{Петр Кузьменко}
\textbf{Ирина Шушпанова} моя жена выпускница КССТ 1985 года. Мы знаем выпускников КВВМПУ 1985 года. С некоторыми давно и плотно дружим.

\iusr{Оксана Гриценко}
Учились мои родственники - Строй Николай !!! где-то в 70-71 годах!

\iusr{Татьяна Смирнова}

У КВВМПУ викладав мій батько, був начальником лабораторії мій брат. А зараз я
викладаю у НаУКМА. Так життя цілої родини виявилось пов'язаним з цими
історичними будівлями.


\iusr{Ольга Ненашева}
Мой папа окончил это училище в 1956 году, а мама университет в этом же году и поехали в Севастополь

\iusr{Анна Мерикова}

В этом училище работала в медсанчасти - стоматолог Людмила Серафимовна - врач от
Бога и с золотыми ручками. Она была хорошей знакомой моей бабушки.


\iusr{Татьяна Лещинер}

Я гостила у бабушки в Киеве, на Подоле, на Спасской, 16. В 1968 году. Так
курсанты ходили мимо дома в баню. Каждую неделю. Один раз пели \enquote{Легендарный
Севастополь}, на сл. неделе \enquote{Ах какая Черноморочка, как такую не любить}. Мы
смотрели в окно.

\iusr{Евгений Бабенко}
Помню ППР ( парт. полит. работу ) - посидели, попи₴дели, разошлись...
Столько людей дурью маялись. А если бы во благо?

\iusr{Андрей Пожарский}
Отлично!

\iusr{Наталья Сенниченко}

Курсанты Высшего Военно - Морского политического училища каждый вечер ходили
строем и с песнями мимо окон дома, где я жила. Я тогда была школьница, но до
сих пор помню слова этих песен ( \enquote{Черноморочка}, \enquote{Легендарный Севастополь} и
т.д.). Потом, когда я заканчивала школу, наш дом ( улица Волошская 10) забрали
для нужд этого училища, а жильцов расселили в разные районы города. Нам
досталась квартира на Левом берегу в панельной девятиэтажке и мой папа почти
двухметрового роста в.первое время постоянно бился головой о форточку с
непривычки ( на Подоле потолки были более 4-х метровой высоты и почти метровой
ширины подоконники). Радовало одно - близость Днепра, мы привыкли жить \enquote{у
воды}. В доме на Волошской жили мои дедушка и бабушка, родилась моя мама, там
же они жили во время оккупации и там же во дворе погибла в 1942-м году моя
бабушка. Сейчас в нём располагаются несколько факультетов Киево - Могилянской
Академии.

\iusr{John Smilerainman}

Одно из сильных впечатлений детства это приход двух настоящих моряков к нам в
детский садик. Форма, какие-то рассказы, а в конце подарили по золотому якорю.

Уверен, что это были курсанты Морполита  @igg{fbicon.smile} 

\iusr{Alexandr KatzNacht}
Это называлось вечерняя прогулка - обязательный ритуал в вооруженных силах.

\iusr{Сергей Удовик}
Я в КВВМПУ переподготовку проходил

\iusr{Valentina Kharenko}

Моя мама Анна Владимировна Харенко много лет преподавала немецкий язык в
училище. К курсантам относилась по-матерински. Заведовала кафедрой иностранных
языков Нинель Михайловна Сазонова.

\begin{itemize} % {
\iusr{Sergey A. Mozgovoy}
\textbf{Valentina Kharenko} у Вас замечательная мама! Анна Владимировна - прекрасный человек и педагог!  @igg{fbicon.hearts.revolving}  @igg{fbicon.rose} 

\iusr{Valentina Kharenko}
\textbf{Sergey A. Mozgovoy} 

Спасибо. Мне кажется, я Вас видела в училище. А мамочка, к сожалению, ушла от
нас в 2016 году. До последних дней декламировала Гете, Гейне на немецком языке,
прекрасно общалась на английском языке, сохранила чистое сознание и тягу ко
всему новому. А в те годы она жила своей работой, уходила с прекрасным
настроением на работу, рассказывала о своих курсантах, очень их любила, писала
методички, бежала на консультации. Я по-прежнему думаю, чтобы сказала мама по
этому поводу, как бы она порадовалась за нас.


\iusr{Мария Константиновская}
\textbf{Valentina Kharenko} а химию преподавала моя подруга Балаба Тамара Александровна...
\end{itemize} % }

\iusr{Татьяна Шиверская}
Боже мой, так все помнить! как я хотела в юности выйти замуж за моряка. Но увы!
Преклоняюсь перед автором

\iusr{Игорь Луговский}
Девчонки все крутились там у центрального входа, ждали дискотеку,!  А, мы, трижды
играли с морполитом в футбол и два раза выигровали Ещё там препадовали боевое
самбо, вход где якорь. со стороны кр. площади И конечно знал троих ребят с района
что служили в оркестре Морпалита Юра мореман стал мужем моей одноклассницы
Марины. Морпрлит это часть истории и украшение района

\iusr{Анна Мерикова}
А еще, когда я была студенткой педучилища мы бегали на дискотеку в морполит....

\iusr{Виктория Салтанова}

Петр, спасибо Вам за рассказы. Наш учитель математики Семен Исаакович, до школы
преподавал в морполите. Мечтала выйти замуж за моряка... Вышла за летчика)))

\begin{itemize} % {
\iusr{Петр Кузьменко}
\textbf{Виктория Масич} 

Семён Исакович сам заканчивал это же училище, когда оно ещё было средним
политическим. Об этом училище я упоминал в моём посте. Когда мы учились,
выпускники того училища, среди которых были и Семён Исакович, наш школьный
преподаватель математики, и тогдашние наши преподаватели КВВМПУ, собирались на
30 - летие своего выпуска. Нам они казались седыми ветеранами. А теперь, 40 -
летие нашего выпуска не за горами. В 2026...

\end{itemize} % }

\iusr{Sergey A. Mozgovoy}
Два вопроса: 1. Где сейчас Знамя училища? 2. Кто и куда дел музей училища?

\begin{itemize} % {
\iusr{Петр Кузьменко}
\textbf{Sergey A. Mozgovoy} 

знамя училища так же как и экспонаты музея хранятся в Украинском Музее Великой
отечественной войны в Киеве. Сейчас он называется музеем второй мировой войны.
По запросу комитета ветеранов КВВМПУ знамя выдают на встречи выпускников.
Правда в сопровождении сотрудника музея.


\iusr{Sergey A. Mozgovoy}
\textbf{Петр Кузьменко} Спасибо!
\end{itemize} % }

\iusr{Наталия Педос}
Очень интересное повествование. Спасибо!

\iusr{Petr Ivanov}

мой дядя... уже давно ушедший ..был в 70годах начальником политотдела училища
..кап 1..... и после школы у меня был только один путь ..стать курсантом.... в
1973г. после окончания школы.. меня повезли в Лютеж.. сдавать экзамены.... и тут
пошло.. что то не так ..и я слава Богу завалил математику... и был послан домой
..на Волошскую... я абсолютно не расстроился... Но родители наехали на дядю
Васю... который был в отпуске и не проконтролировал племяша.... был серьезный
разбор ..и меня тут же по согласованию с начальником училища тут же взяли на
=работу=...слесарем в автопарк училища.... и еще такой же =работник= стал
маляром.... мы приняли присягу и стали заниматься с преподавателями в порядке
так сказать исключения..... Внимательно вникнув в то что меня ждет в дальнейшем
я быстренько убежал совсем в другие совсем не худшие войска... и не пожалел об
этом решении ни разу..... единственное что осталось у меня в военном
билете.... профессия... аквалангист... т. к я закончил курсы в училище.... А товарищ
мой =маляр= поступил в 1974г.. и благополучно оттрубил в Североморске на
под. лодке... ушел на гражданку капитаном первого ранга.... с кучей
проф. болячек... и с вопросом... как жить дальше.... Короче... других уж нет.. а те
далече..... Иногда встречались ..пили по 100гр.... и я ни чуть не жалею... что так
сложилось..... насколько я знаю.. больше гражданских
=специалистов.. слесарей.. маляров=..в КВВМПУ не брали!!!

\iusr{ZabaVa Da}

Петр, спасибо, приятные воспоминания из детства тоже с морским оттенком-в детсад
приходили красивенные морячки, как раз напротив огромные ворота со
звездой.. напротив Фроловского @igg{fbicon.face.rolling.eyes} 

\begin{itemize} % {
\iusr{Петр Кузьменко}
\textbf{Рита Хомик} 

это мой 98 садик. Я писал о нём в своих постах в группе. Кстати монастырь
Флоровский, а улица неподалёку Фроловская.

\begin{itemize} % {
\iusr{Валерий Панчук}
\textbf{Петр Кузьменко} на руси флоров не было поэтому привычней фрол поэтому фроловская да и мы всегда говорили фроловский монастирь

\iusr{Петр Кузьменко}
\textbf{Валерий Панчук} и мы говорили. Но правильно Флоровский.
\end{itemize} % }

\iusr{Валерий Панчук}
Конечно в честь греческого святого флора

\iusr{Валерий Панчук}
\textbf{Рита Хомик} там вч вв сейчас нацгвардии

\begin{itemize} % {
\iusr{Петр Кузьменко}
\textbf{Валерий Панчук} ВВ там было всегда. Новороссийский полк ВВ.

\iusr{Валерий Панчук}
\textbf{Петр Кузьменко} а почему новороссийский проясните

\iusr{Петр Кузьменко}
\textbf{Валерий Панчук} 

мы школьниками были там на экскурсии. Название получено потому, что
подразделение НКВД, на основе которого полк был сформирован проявило героизм в
Великой отечественной войне при боях за Новороссийск.

\end{itemize} % }

\end{itemize} % }

\iusr{Наталия Беляева}

Пойти на танцы в \enquote{морполит}? Замечательная идея! Надо отметить, что играл там
вокально-инструментальный ансамбль, состоящий из курсантов этого же училища, и
играли ребята хорошо!  @igg{fbicon.hand.waving} 

\begin{itemize} % {
\iusr{Петр Кузьменко}
\textbf{Наталия Беляева} 

это были мои однокурсники и друзья. Ансамбль назывался \enquote{Наследники}.
Благодарю, что помните!@igg{fbicon.heart.red}

\iusr{Ирина Нищимная}

Я тоже бегала с подружкой на дискотеки в морполит, живо все так написали,
спасибо, помню как мы ходили смотреть выпускной в Морполите, как жаль, что
утрачено такое заведение,

\end{itemize} % }

\iusr{Андрей Могильный}

Небольшая ремарка: Киевской академии водного транспорта тоже уже нету) она
вошла в дуит (Государственный университет инфраструктуры и технологий )
спасибо. Интересный рассказ!


\iusr{Ирина Архипович}
Спасибо за интереснейший экскурс!!!  @igg{fbicon.hands.applause.yellow}  @igg{fbicon.face.happy.two.hands} 

\iusr{Ольга Рожненко}
Петр, спасибо за рассказ, пишите ще!

\iusr{Сергей Сергеев}

Как не помнить морполит - прм и прд караула на гарнизонной гауптвахте, участие
в соревнованиях КВО, встреча с патрулем от морполита на конечной 62 автобуса,
которые всегда заканчивались кроссом (был такой фетиш у патрулей - поймать не
менее 5 курсантов с нарушением формы одежды) по моему родному Подолу и т.д...
))))


\iusr{Арман Кеосаян}

Помню, когда был выпуск, курсанты срывали и бросали на Ильинской золотистые
якоря, а дети бегали и подбирали. Особенно вечерние песни запомнились -
\enquote{Маруся, кап, кап, кап...} Лет через десять встрелись выпускники Морполита и
оказалось, что это была песня как раз их роты. Встретились как родные: подоляне
и бывшие курсанты. Хотя уже все было в прошлом: и Морполит, и старый Подол


\iusr{Павел Пауль}
Флаг на трубе Тэц 1 - это тоже курсанты...

\iusr{Анна Василенко}

Спасибо за рассказ. Я студентка педучилища бегали с подружкой на дискотеки
Морполита. Было очень круто))) весело. И ребята с которыми мы встречались
бегали к нам в своих комбезах.)))))

\begin{itemize} % {
\iusr{Naty Lukina}
\textbf{Анна Василенко} 

мы там тоже пропадали, компания была внушительная, 3 года дружили до их
выпуска, кто женился таки и со всеми поддерживаем до сих пор отношения, хоть
многие поразъехались...

\iusr{Анна Василенко}
\textbf{Naty Lukina}, да знакомо))) я тоже чуть не вышла)))
\end{itemize} % }

\iusr{Yurii Kadochnikov}

Ще була така класна рота забезпечення учбового процесу. Ото хлопцям повезло
служити моряками та ще й в Києві ))

\end{itemize} % }
