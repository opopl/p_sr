% vim: keymap=russian-jcukenwin
%%beginhead 
 
%%file 04_07_2023.stz.news.ua.mrpl.0629.1.misto_nadii_proekt_sergij_rodionov
%%parent 04_07_2023
 
%%url https://www.0629.com.ua/news/3623121/misto-nadii-koncepcia-vidnovlenna-mariupola-vid-komandi-sergia-rodionova-foto
 
%%author_id news.ua.mrpl.0629
%%date 
 
%%tags 
%%title Місто надії. Концепція відновлення Маріуполя від команди Сергія Родіонова, - ФОТО
 
%%endhead 
 
\subsection{Місто надії. Концепція відновлення Маріуполя від команди Сергія Родіонова, - ФОТО}
\label{sec:04_07_2023.stz.news.ua.mrpl.0629.1.misto_nadii_proekt_sergij_rodionov}
 
\Purl{https://www.0629.com.ua/news/3623121/misto-nadii-koncepcia-vidnovlenna-mariupola-vid-komandi-sergia-rodionova-foto}
\ifcmt
 author_begin
   author_id news.ua.mrpl.0629
 author_end
\fi

%\ifcmt
  %ig https://i2.paste.pics/41e32a272257cf91472ba3c38b2e266e.png
  %@wrap center
  %@width 0.9
%\fi

\ifcmt
  ig https://s.0629.com.ua/img/section/newsInternalIcon/is/2/1cover_64a3e8a79d672.jpg
  @caption 0629 продовжує публікувати проєкти майбутнього розвитку українського Маріуполя після деокупації, які нещодавно були представлені на Львівському урбан форумі. Сьогодні представляємо концепцію міста від команди Сергія Родіонова.
  @wrap center
  @width 0.9
\fi

Проєкт Ре: Маріуполь розпочався у квітні 2022 року з візуального маніфесту
Сергія Родіонова, українського дизайнера родом з Маріуполя, автору
бренд-стратегії міста та співзасновника ініціативи про майбутнє України \enquote{Мрія}.
Маніфест був представлений у вигляді книжки, сайту та виставки у Independent
School for the City (Роттердам, Нідерланди), де на той час навчався Сергій. 

Після 6 місяців роботи проєкт трансформувався у проєкт відбудови міста,
створений командою українських архітекторів та міжнародних експертів. Окрім
Сергія до команди \enquote{Ре: Маріуполь} увійшли Яна Бучацька, архітекторка та
співзасновниця студії Sets, Анна Камишан, архітекторка, художниця,
спізасновниця \enquote{Мрії}, архітектори Павло Заботін та Віктор Копейкін. Проєкт
розроблений на запрошення платформи маруіпольського муніцпалітету
Mariupol.Reborn. Частина матеріалів створених для проєкту \enquote{Ре: Маріуполь} були
створені завдяки підтримці Агентством США з міжнародного розвитку (USAID).

\begin{leftbar}
\begingroup
\bfseries
"Маріуполь отримав шанс на повне перезавантаження та оновлення через
жахливі руйнування війни. Пам'ятаючи про ціну, покладену за його оборону, а у
майбутньому і деокупацію, ми закладаємо цінність людського життя в основу
нового міста. 

Ми дослідили актуальні підходи до трансформації сучасних сталих міст у
відповідь на виклики майбутнього та поєднали цей світовий досвід із унікальною
культурою та історією Маріуполя.

Проєкт \enquote{Ре: Маріуполь} — водночас мрія, план і бачення повоєнної відбудови
міста, якими ми прагнемо поділитися зі світом. Це нагадування про місто і
людей, яких намагалися залякати і розбити вщент. Про місто і людей, які
зустрінуться знову. Але вже іншими ніж були раніше", — описує команда.
\endgroup
\end{leftbar}

Попри те, що війна ще не закінчена, команда вважає, що відбудова українських
міст має бути системною та добре спланованою заздалегідь, щоб після деокупації
рухатись швидко. До того ж ми не можна дозволити дати її зробити твердою рукою
згори вниз, як це було колись, без врахування думки людей. Залучення громади,
ефективна партисипація потребують часу і тому краще почати якомога раніше.

\enquote{До початку роботи ми ставили питання, як ефективно трансформувати місто,
зробивши його стійким до викликів майбутнього і водночас комфортним для життя
та повним можливостей для самореалізації. А головне — як повернути в місто
енергію, як залучити до нього найкращих людей?} — зазначає ініціатор проєкту
Сергій Родіонов.

Бачення Маріуполя від Ре: Маріуполь починається з безпеки: \enquote{Так, це фортеця,
східний форпост Європи, міжнародна військова база. Паралельно з цим в місті
з'являється Кампус трансформації міста, мозковий центр реконструкції, де
найкращі світові та українські спеціалісти різних сфер експериментують та
змінюють середовище. Місто активно відбудовується з використанням місцевих
екологічних та перероблених матеріалів з увагою до навколишнього середовища.
Запускаються процеси очищення землі та води, що отруювалися протягом двох
століть. Нові технології в екологічному будівництві та фіторемедіації стануть
базою для створення наукових та дослідницьких кластерів. Наука та освіта
замість важкої промисловості мають стати новою ідентичністю міста. А культура
та туризм доповнять економічний портрет міста}.

Міський простір також трансформують під нову ідентичність.

\begin{leftbar}
\begingroup
\bfseries
\enquote{Ми бачимо Маріуполь в майбутньому значно компактнішим, містом не лише
мікрорайонів, а й із активним насиченим центром, розгорнутим до води –
Азовського моря та річок, із 15-хвилинною пішою доступністю будь-яких
повсякденних сервісів та функцій міста}.
\endgroup
\end{leftbar}

\ii{04_07_2023.stz.news.ua.mrpl.0629.1.misto_nadii_proekt_sergij_rodionov.pic.1}
\ii{04_07_2023.stz.news.ua.mrpl.0629.1.misto_nadii_proekt_sergij_rodionov.pic.2}

Аби привернути увагу широкої аудиторії до проєкту, а маріупольцям дати надію у
повернення додому, команда вирішили проілюструвати кілька власних рішень та
мрій у вигляді шести магнітів, що притягуватимуть містян і туристів:

\textbf{1. Центральний вокзал Маріуполя} — великий транзитний термінал, який сполучає
поїзди, автобуси, місцевий громадський транспорт (трамваї, електроавтобуси,
пароми) та автомобілі. Станція розташована в географічному центрі міста в гирлі
річки Кальміус між історичним центром та новим постіндустріальним районом
Маріуполя на місці доменних печей \enquote{Азовсталі}.

\ii{04_07_2023.stz.news.ua.mrpl.0629.1.misto_nadii_proekt_sergij_rodionov.pic.3.vokzal}
\ii{04_07_2023.stz.news.ua.mrpl.0629.1.misto_nadii_proekt_sergij_rodionov.pic.4}

\textbf{2. Музей Маріуполя та Базарна площа.} Багато маріупольців хочуть бачити початок
відліку історії міста від заснування козацької фортеці, яка була саме на цьому
місці. Нам подобається ідея повернення справжнього центру міста до його
оригінальної локації – ринкової площі біля фортеці. Відновлена площа в
майбутньому має стати центром міського життя, місцем проведення концертів,
торгівлі на вихідних та щорічного ярмарку. Фортецю не треба відтворювати, але
новий сучасний музей міста своєю формою може окреслювати периметр колишнього
козацького форпосту.

\ii{04_07_2023.stz.news.ua.mrpl.0629.1.misto_nadii_proekt_sergij_rodionov.pic.5}
\ii{04_07_2023.stz.news.ua.mrpl.0629.1.misto_nadii_proekt_sergij_rodionov.pic.6}

\textbf{3. Центр пам'яті \enquote{Драмтеатр}}. Атакований у березні 2022 року драматичний театр
завжди був місцем зустрічі містян. Після трагедії він став всесвітньо відомим
місцем, пам'яттю про сотні невинно загиблих. У новому Маріуполі він має
зберегти свою меморіальну функцію. Ми пропонуємо на місці руїн створити центр,
що поєднає пам'ятку і новий простір, що доводить — життя перемагає. Фасадна
сторона зміститься на бік колишнього скверу, і отримає продовження у вигляді
амфітеатру, де можуть гуляти містяни і грати діти. Тут саме життя і стає
театром.

\ii{04_07_2023.stz.news.ua.mrpl.0629.1.misto_nadii_proekt_sergij_rodionov.pic.7}
\ii{04_07_2023.stz.news.ua.mrpl.0629.1.misto_nadii_proekt_sergij_rodionov.pic.8}

\textbf{4. Центральний пляж та Кампус \enquote{Мозаїка}.} Нові напрямки морського відпочинку
доповнять улюблені вітрильництво та греблю та ялах. На місці старого
залізничного вокзалу з'явиться Кампус \enquote{Мозаїка}, що стане мозковим центром
відродження Маріуполя. Старі рейки обростуть парком. На території центрального
пляжу з'явиться перший технологічний центр дослідження Азовського моря.

\ii{04_07_2023.stz.news.ua.mrpl.0629.1.misto_nadii_proekt_sergij_rodionov.pic.9}
\ii{04_07_2023.stz.news.ua.mrpl.0629.1.misto_nadii_proekt_sergij_rodionov.pic.10}

\textbf{5. Центр Сталі.} Постіндустріальні території у майбутньому стають
центрами культури, однак можуть зберігати своє ДНК і розповідати відвідувачам
про індустріалізацію, металургію та метал як матеріал — частини історії, без
яких неможливо уявити маріупольське минуле.

\ii{04_07_2023.stz.news.ua.mrpl.0629.1.misto_nadii_proekt_sergij_rodionov.pic.11}
\ii{04_07_2023.stz.news.ua.mrpl.0629.1.misto_nadii_proekt_sergij_rodionov.pic.12}

\textbf{6. Центр пам'яті \enquote{Азовсталь}.} Ще одне переосмислення промислових
територій Маріуполя. Місце, що стало укриттям для кількох тисяч мешканців і
захисників міста навесні 2022, перетвориться на величезний меморіальний
парк-заповідник.  Ми залишимо територію, обмежену каналом, недоторканою,
надавши природі можливість поглинути і \enquote{вилікувати} її. Тут
проводитимуться екскурсії, а окремі простори можуть використовуватися для
мистецьких резиденцій, де переосмислюватимуть теми героїв Маріуполя.

\ii{04_07_2023.stz.news.ua.mrpl.0629.1.misto_nadii_proekt_sergij_rodionov.pic.13}
\ii{04_07_2023.stz.news.ua.mrpl.0629.1.misto_nadii_proekt_sergij_rodionov.pic.14}

Онлайн ознайомитися із пропозицією команди Ре: Маріуполь можна на \href{https://re-mariupol.com}{сайті проєкту} 
та в \href{https://www.instagram.com/re_mariupol}{інстаграмі}.

\textbf{Читайте також по темі:}

\begin{itemize} % {
  \item \href{https://www.0629.com.ua/news/3620811/novij-centr-mista-obliccam-do-mora-memorial-azovstal-e-cotiri-versii-vidnovlenogo-mariupola-foto}{Новий центр міста. Обличчям до моря. Меморіал \enquote{Азовсталь}. Є чотири версії відновленого Маріуполя, - ФОТО}

  \item \href{https://www.0629.com.ua/news/3622392/bezpeka-dim-robota-versia-vidnovlenogo-mariupola-vid-urbanisticnoi-koalicii-ro3kvit-foto}{\enquote{Безпека. Дім. Робота}. Версія відновленого Маріуполя від урбаністичної коаліції Ro3kvit, - ФОТО}
\end{itemize} % }

\ifcmt
  ig https://i2.paste.pics/51d9e21b8273651bb05593af4fffa6a4.png
  @wrap center
  @width 0.9
\fi
