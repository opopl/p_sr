% vim: keymap=russian-jcukenwin
%%beginhead 
 
%%file 12_12_2021.fb.fb_group.story_kiev_ua.2.zhizn_svjato_krivuca_chast_pervaja
%%parent 12_12_2021
 
%%url https://www.facebook.com/groups/story.kiev.ua/posts/1817140365149423
 
%%author_id fb_group.story_kiev_ua,fedjko_vladimir.kiev
%%date 
 
%%tags chelovek,isskustvo,kiev,kultura,zhizn
%%title Життя – це свято. Розмова з Наталі Кривуцею (Частина I)
 
%%endhead 
 
\subsection{Життя – це свято. Розмова з Наталі Кривуцею (Частина I)}
\label{sec:12_12_2021.fb.fb_group.story_kiev_ua.2.zhizn_svjato_krivuca_chast_pervaja}
 
\Purl{https://www.facebook.com/groups/story.kiev.ua/posts/1817140365149423}
\ifcmt
 author_begin
   author_id fb_group.story_kiev_ua,fedjko_vladimir.kiev
 author_end
\fi

Життя – це свято: твори мистецтва в інтер'єрі сучасної квартири
Розмова з Наталі Кривуцею (Частина I)

***

Життя – це свято!

І коли первісний мисливець це зрозумів, то почав зображати на стінах свого
житла – печери – людей і тварин, сцени полювання ... Кам'яний скребок був
першим інструментом художника, – полотном служили граніт і базальт. Потім в хід
пішли найрізноманітніші матеріали: дерево, глина, кістка, шерсть, камінь, метал
і скло. З них виготовлялися дивовижні речі, які стали прабатьками зразків
народного прикладного мистецтва і живопису. Йшов час, але інтерес до живопису;
карбованим, литим, кованим, гравірованим і філігранним виробам з металу,
кераміки, дерева та інших матеріалів не згасав. Кожен намагався оформити своє
житло якомога вишуканіше ... Здійснення основної потреби мистецтва в житті
людини, поступово перенеслося і в громадські приміщення.

\ii{12_12_2021.fb.fb_group.story_kiev_ua.2.zhizn_svjato_krivuca_chast_pervaja.pic.1}

Створення інтер'єру – це прекрасний спосіб висловити свою індивідуальність.
Неповторність і ексклюзивність житлу надають смаки й уподобання його
господарів. Це можуть бути різні сувеніри і оригінальні предмети, зібрані під
час подорожей, колекції різних видів кераміки, картини або сімейні реліквії. В
даний час практично жоден інтер'єр не обходиться без використання різних
предметів мистецтва.

Але що ми знаємо про силу впливу творів мистецтва? 

Чи вміємо ми правильно їх вибирати і розміщувати в своїй квартирі? 

Які критерії відбору художніх творів для інтер'єру або колекції? 

Яку роль в житті людей грали твори мистецтва в епоху Наполеона і у
вікторіанську епоху Англії; в дореволюційній Росії і в СРСР? 

Яку роль відіграють твори мистецтва в приватних володіннях сучасної Японії, США
і пострадянської України? 

\ii{12_12_2021.fb.fb_group.story_kiev_ua.2.zhizn_svjato_krivuca_chast_pervaja.pic.2}

Як правильно поєднати художні твори з інтер'єром квартири?

Про це та про багато іншого моя розмова з Наталі Кривуцею, відомим
мистецтвознавцем і громадським діячем, доктором філософських наук, полковником
миротворчих сил, киянкою.

- Володимире, мені дуже приємно спілкуватися з Вами, але я змушена відразу
внести ясність щодо заданих Вами питань. Тому що для того щоб відповісти на
них, в принципі, потрібно прослухати курс історії мистецтв як мінімум п'яти
років, який слухають студенти художніх вузів. Чому? Ви задали питання про
Англію і про Францію; Росія – дореволюційний період і СРСР; сучасна Японія,
США, Україна ... 

Однак, за п'ять років люди, навіть добре навчаючись, іноді
цього не знають і не розуміють. Адже мистецтвознавство – не лише знання, але ще
і чуття. 

\ii{12_12_2021.fb.fb_group.story_kiev_ua.2.zhizn_svjato_krivuca_chast_pervaja.pic.3}

Тому розповісти Вам про все на декількох сторінках блогу практично
неможливо. Хоча, Ви знаєте, вивчити історію мистецтв, як виявилося, на протязі
тільки студентських років теж неможливо. Все змінюється, все видозмінюється і,
наприклад, та історія мистецтв, яку ми вчили ще по Грабарю або за підручниками
інших педагогів, зараз вже змінилася. Нові відкриття здійснилися, помінялися
поняття, змінилася соціологічна, емоційна і політична налаштованість
суспільства. І істотно змінилися погляди людей на квартиру, на інтер'єр, на
поняття «що таке колекція» і «що таке колекціонування».

Колекціонування дореволюційній Росії або, як Ви питаєте, вікторіанської епохи –
це зовсім інше. Тоді ніхто нічого не колекціонував. Купували для того, щоб
внести твір мистецтва в свій інтер'єр. Чому? Зрозумійте, розшарування
суспільства було величезним. І два, три, а то один відсоток з усього населення
міг собі дозволити це купити, замовити. І замовити – було не просто замовити
картину. Це повинен був бути обов'язково парадний портрет членів сім'ї,
обов'язково своїх предків, які що-небудь зробили і чогось досягли ... 

\ii{12_12_2021.fb.fb_group.story_kiev_ua.2.zhizn_svjato_krivuca_chast_pervaja.pic.4}

Потім
галерея поповнювалася спадкоємцями. Обов'язково повинні бути пейзажі, які
говорять про кругозір людини, про його подорожі. Особливо були в моді
італійські пейзажі, французькі пейзажі, неодмінно повинні бути присутніми
натюрморти. 

І кожна країна мала своїх художників, які були яскравими
представниками того чи іншого жанру мистецтва. Тому що таких як «малі
голландці» не можна було знайти в культурі живопису Франції та Італії.

Наприклад, італійські пейзажі – Каналетто і так далі, російські пейзажі, і
пейзажі північних художників. Абсолютно іншими були можливості показу, і
завдання показу, парадних портретів або портретів в колі сім'ї. Змінювалися
смаки, змінювалися пріоритети, змінювалися завдання. Тому розповісти в одному
інтерв'ю про те, що таке історія культури багатьох народів, яка з'єднувалася не
тільки у них на території, але і перепліталася, містками перекидалася з однієї
країни в іншу, з Іспанії до Італії, з Італії в Німеччину, або з Німеччини в
Англію – це шалено складно. Це непосильне завдання для нашої розмови. Хоча ми
можемо взяти якусь конкретику і про неї поговорити. Але в принципі я зрозуміла,
що основне Ваше питання про те, що таке сучасний будинок і сучасна культура
обстановки в цьому будинку.

Якщо ми будемо рівнятися на те, що було колись, мені це нагадує багато квартир,
які, наприклад, я бачу на Оболоні. У мене є подруги, у мене є друзі, знайомі,
до яких я приїжджаю в гості, заходжу в брудний під'їзд дев'ятиповерхового або
ще радянського періоду будинку, який тепер переобладнаний. Під'їзд залишився
тим же, і я піднімаюся на якийсь сьомий-восьмий поверх, заходжу і бачу: меблі в
позолоті, позолота на стінках і мені смішно. Тому що, коли в такому будинку
радянського періоду намагаються зобразити палац XVIII століття у себе в
інтер'єрі, при цьому виходячи за двері, яка теж позолочена, і виходять в
брудний під'їзд, куди виходять четверо дверей всі по-різному оббиті, – смішно
говорити про те, що ця людина може зрозуміти, чи сприйняти і використовувати
нашу з Вами розмову. 

У колишніх прибуткових будинках або тих будинках, які належали багатим людям в
Росії, наприклад, або у нас в Україні, зроблені музеї ... Зараз наслідуючи їм
роблять нові будови, в яких теж ти заходиш і бачиш, то чи готичні якісь
елементи, або і готичні, і рококо, і бароко і класику ... і ти усвідомлюєш, що
людина в стилях нічого не розуміє. Але він знає тільки одне, що якщо він
заробив гроші і має можливість купити таку квартиру, то поради йому не
потрібні. Він все знає! Бо якщо він знає як заробити ці гроші, то вже тим
більше він знає, як йому потрібно обставити квартиру. І це найбільше нещастя
нашого часу. Адже приходячи до лікаря, ми ніколи не говоримо, що ми знаємо, як
нас лікувати. Ми все-таки просимо, щоб нас полікували і не вказуємо лікарю, як
нам вирвати зуб чи як лікувати або розглядати нашу гінекологію. А коли людина
приходить до митця, або купує меблі, або створює свій інтер'єр, він знаючи, що
він все знає, допускає такі помилки, за які потім соромно самому собі, що ти
знав, поважав цю людину, і як тобі гірко доводиться розчаровуватися,
потрапляючи до нього в будинок.

Саме дрібниці і аксесуари будинку, кажуть і про внутрішню культуру, і про
вихованість, і про освіченість його господаря. Тому дрібниць ніколи не існує в
оповіданні про себе. Кожна наша річ розповідає про свого господаря. Адже речі
до нас потрапляють не просто тому, що ми їх купуємо. Дещо нам дарують. Дещо ми
набуваємо в подорожах на пам'ять. І, може бути, воно не підходить до нашого
інтер'єру, але не можна викинути, тому що для нас це пам'ять.

Є дуже гарні дизайнерські будинки, дуже красиві дизайнерські квартири, де немає
нічого зайвого, де господарі постаралися почути тих консультантів, які були у
них під час розробки інтер'єру їхнього будинку. І вони їх почули, і вони
зробили так, як це радили фахівці. І вийшло чудово. Але ... Людина живе, і він
все одно приносить щось в свій будинок. Тому що ми не безрукі, чи не безокі, що
не безголові. Ми все одно десь буваємо, десь купили книжку, десь купили
картинку, десь купили скульптуру, десь купили якийсь сувенір, і нам потрібно
місце для того, щоб все це зайняло гідне місце в нашому будинку. І виходить, що
в дизайнерську, вивірену до міліметра, – де і що повинно стояти, – квартиру
потрапляють речі, які вже там зайві. Вони вже не відповідають інтер'єру. І ми
забуваємо, що господар – жива людина. Зі своїми пріоритетами і своїми
бажаннями. Що, до речі, майже ніколи або дуже рідко враховують дизайнери, які
працюють над інтер'єром. Тому, кажучи про те, що було раніше і що є зараз, ми
повинні усвідомлювати – це абсолютно несумісні поняття і категорії. Що
прийнятно для японця, то незрозуміло європейцеві. Що зрозуміло і прийнятно для
європейця, і йому потрібно – це його потреби, то абсолютно неприйнятно для
китайця, малайці або корейця.

Давати будь-які поради – це плювати проти вітру! Зрештою, будь-яка людина все
одно вибере свій шлях в тому, як він бачить себе в тому чи іншому місці. Це
стосується офісу, це стосується будинку, це стосується вибору, вибачте мене,
дружини, чоловіка та виховання наших дітей. Це все ланки одного ланцюга. Ніколи
жодна картина в будинку не з'являється випадково. Так, іноді вона з'являється
випадково, коли її дарують. Це найгірше, що може бути. Тому що хороша картина з
хорошою енергетикою, потрапляючи в наш будинок, виховує смак і підвищує
культуру нашу, наших дітей, наших онуків і так далі ... Ми можемо викинути все.
Ми можемо поміняти машину, поміняти меблі. Але, як правило, твори мистецтва не
йдуть з дому. Вони живуть довго і дуже важливо, щоб це був висококласний твір,
який буде вносити позитивний елемент в нашу долю. Якщо ж ця робота з поганою
енергетикою, вона буде псувати наш настрій, наш смак, вона буде опускати нашу
культуру на нижчу ступінь розуміння, і це ж буде відбуватися з нашими дітьми і
з нашими онуками. Вибір автора, вибір теми і внесення художнього твору в ваш
будинок – велике, важке і дуже відповідальне завдання. І добре, якщо ви
володієте смаком і знаннями, які потрібні для цього, тому що мистецтвознавство,
дизайн – це така ж наука, як і всі інші. Є абетка – А, Б, В, Г, Д ..., не
вивчивши якої ви не зможете читати книгу культури, і вже тим більше складати
оди в її честь. Виходячи з цього, дуже важливо, коли ми говоримо про інтер'єр,
забути про тих дизайнерів, які говорять, що картинка повинна бути під колір
стіни або шпалер, а оформлення повинно бути в тон або в одному струмені з
меблями. Це все неправильно. Картина це окреме життя вашого будинку. Це окреме
життя певного періоду вашого життя. Це окреме ваше захоплення. Вони можуть
змінюватися. У нашому житті може змінюватися все. Змінюються чоловіки – іноді,
змінюється робота – часто. Іноді змінюються і твори мистецтва, які нам близькі.
Спочатку ми любимо класику, потім ми раптом починаємо любити модернізм. Звідки
це відбувається? Тільки від наших знань. Чим більше ми знаємо, тим більшого ми
хочемо. Чим більшого ми хочемо, тим більше нам здається, що ми можемо це
привнести в свій будинок, забуваючи про те, що бажання і можливість придбати –
ще не означають, що це потреба вашого будинку.

Будинок – це живий організм. Він живе своїми завданнями і в кожній культурі, і
в кожну епоху ці завдання були свої. Чому так змінювалося, наприклад, після
епохи розвиненого і прекрасного періоду Давньої Греції – занепад Греції;
піднесення прекрасної культури стародавнього Риму – занепад стародавнього Риму;
Середньовіччя, застій Середньовіччя, який не був застоєм в культурі, як такої,
це було в політичній і соціальній сфері. А в культурі це дало чудову готику,
чудові зразки культури і твори Дюрера та інших художників, великих майстрів.

Потім настає велике Відродження і прекрасні часи роботи таких майстрів як
Леонардо да Вінчі, Рафаель, Мікеланджело. У кожній країні були свої епохи і
коливання можливостей розвитку суспільства. І розвиток суспільства несло в
культуру ту чи іншу струмінь розвитку. І розвиток знаходив свої форми. Ці форми
стосувалися, перш за все, соціальної налаштованості, економічних можливостей і
завдань, які стояли перед розвитком міста як такого, його структури. Тобто
розвиток починався з розвитку міста та з потреби що-небудь побудувати. 

Далі починався розвиток архітектурних стилів, які потім за собою вабили всі
інші стилі, які повинні були прийти на допомогу оформлення внутрішнього
оздоблення архітектури. Так змінювалися стилі після Відродження .... 

Потім почалися ті завдання, які стояли вже на рубежі ХХ століття, перед тими
творцями, які починали творити нові міста і робити нову соціальну структуру.
Так само як змінювалася вся історія людства, синусоїдою в різних країнах і в
різні періоди, так само змінювалася королева мистецтва архітектура, яка
диктувала можливості тим живописцям, скульпторам, а вже потім дизайнерам, які
приходили для того, щоб наповнити цю архітектуру тією красою, яка відповідає
часу і стилю.

Тому кожна епоха відрізнялася своєю модою на будівлі, своєю модою на їх
оздоблення, своєю модою на культуру оформлення інтер'єру, на культуру подачі
меблевого дизайну і, природно, мода переходила на костюм, на макіяж, на людей
як таких, і це все відповідало тому, що було в даний момент в будинку. Тому
брати окремо культуру міського одягу і інтер'єру нерозумно, тому що коли у тебе
інтер'єр зроблений в позолоті, з якимись елементами класики, рококо, і заходить
господиня в джинсах і в кедах в цей інтер'єр, це виглядає навіть не по-дурному,
навіть не смішно. Це виглядає безглуздо. Немає нічого дурнішого в нашому житті,
ніж ця невідповідність і безглуздість. Це те, що говорить про нашу неможливість
жити нормальним життям в тому середовищі, в якому ми зараз існуємо.

***

Далі буде

***

Наталі люб’язно дозволила мені сфотографувати інтер'єри квартири та твори
мистецтва з її колекції.

***

\ii{12_12_2021.fb.fb_group.story_kiev_ua.2.zhizn_svjato_krivuca_chast_pervaja.cmt}
