%%beginhead 
 
%%file 13_09_2021.fb.loskutova_natalia.mariupol.1.robocha_narada
%%parent 13_09_2021
 
%%url https://www.facebook.com/1427894275/posts/pfbid02Hv5iZCQ6MWKxYDY3jDi7U3DkwHuNcF829DGL3NW6n8cJAjDwhXGB9Cr6VkZCqyvyl
 
%%author_id loskutova_natalia.mariupol
%%date 13_09_2021
 
%%tags mariupol,mariupol.pre_war,language.french
%%title Перекладацька діяльність представників МДУ з французькими партнерами міста
 
%%endhead 

\subsection{Перекладацька діяльність представників МДУ з французькими партнерами міста}
\label{sec:13_09_2021.fb.loskutova_natalia.mariupol.1.robocha_narada}

\Purl{https://www.facebook.com/1427894275/posts/pfbid02Hv5iZCQ6MWKxYDY3jDi7U3DkwHuNcF829DGL3NW6n8cJAjDwhXGB9Cr6VkZCqyvyl}
\ifcmt
 author_begin
   author_id loskutova_natalia.mariupol
 author_end
\fi

Перекладацька діяльність представників МДУ з французькими партнерами міста

13 вересня 2021 у залі засідань Маріупольської міської ради відбулася робоча
нарада віцепрем'єр-міністра України – Міністра з питань реінтеграції тимчасово
окупованих територій України Олексія Резнікова з представниками Маріупольської
міської ради, народними депутатами України, представниками Донецької обласної
державної адміністрації, представниками КП \enquote{Компанія \enquote{Вода
Донбасу}} та КП \enquote{Маріупольводоканал}. 

\ii{13_09_2021.fb.loskutova_natalia.mariupol.1.robocha_narada.pic.1}

Окрім цього у нараді брали участь
французькі партнери, які реалізують проєкт будівництва заводу з очищення та
постачання якісної питної води, а саме президент компанії BETEN ENGENIERIE Жан
Рош, директор проєкту від компанії Stereau Арно Тьєс та керівник проєкту від
компанії EGIS Eau Хью Маклаган. 

Сторони обговорювали техніко-економічне
обґрунтування проєкту, можливість використання додаткового фінансування,
доцільність залучення додаткової команди фахівців для проведення більш
ґрунтовних досліджень. Враховуючи те, що велику кількість питань було
поставлено французьким партнерам і вони мали надавати детальні пояснення,
значну роль у цій зустрічі відігравали перекладачі, зокрема представниця МДУ,
викладачка французької мови Наталія Лоскутова. Для перекладачки це далеко не
перший досвід роботи з поважними гостями, які завжди залишаються задоволеними
якістю перекладу. Проте кожна зустріч має свої особливості, і сьогодні
перекладачка вперше здійснювала синхронний переклад з української мови на
французьку і навпаки. Беручи до уваги те, що сторони дійшли згоди у всіх
важливих питаннях, робота пані Наталії сприяла порозумінню всіх сторін. А
оскільки реалізація проєкту буде тривати ще майже три роки, у перекладачки буде
ще багато можливостей відшліфувати свою майстерність.

\ii{13_09_2021.fb.loskutova_natalia.mariupol.1.robocha_narada.pic.2}

%\ii{13_09_2021.fb.loskutova_natalia.mariupol.1.robocha_narada.cmt}
