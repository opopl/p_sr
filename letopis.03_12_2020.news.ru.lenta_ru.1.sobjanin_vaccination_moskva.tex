% vim: keymap=russian-jcukenwin
%%beginhead 
 
%%file 03_12_2020.news.ru.lenta_ru.1.sobjanin_vaccination_moskva
%%parent 03_12_2020
 
%%url https://lenta.ru/news/2020/12/03/privivka/
 
%%author 
%%author_id 
%%author_url 
 
%%tags 
%%title Собянин объявил о старте массовой вакцинации от коронавируса в Москве
 
%%endhead 
 
\subsection{Собянин объявил о старте массовой вакцинации от коронавируса в Москве}
\label{sec:03_12_2020.news.ru.lenta_ru.1.sobjanin_vaccination_moskva}
\Purl{https://lenta.ru/news/2020/12/03/privivka/}

\ifcmt
pic https://icdn.lenta.ru/images/2020/12/03/13/20201203132418234/pic_19b5bd68da8ac9cb2d278680ba988922.jpg
caption Фото: Сергей Киселев / РИА Новости
\fi

\index[rus]{Коронавирус!Россия!Вакцинация}
\index[names.rus]{Собянин, Сергей!Мэр Москвы}

Мэр Москвы Сергей Собянин в своем блоге объявил о старте массовой вакцинации от
коронавируса в столице.

Начиная с 4 декабря для работников образования, здравоохранения, а также
городских социальных служб будет доступна электронная запись на прививку от
коронавируса. Работа прививочных пунктов начнется с 5 декабря. По словам
Собянина, прививочные пункты оборудованы медицинскими холодильниками, персонал
прошел специальное обучение.

Как рассказал градоначальник, по мере поступления больших партий препарата
вакцинация будет доступна и для других москвичей.

2 декабря президент России Владимир Путин поручил на следующей неделе начать в
России масштабную вакцинацию от коронавируса. Первыми вакцинацию должны пройти
врачи и учителя. По словам Путина, в ближайшие дни объем произведенной в стране
вакцины достигнет двух миллионов доз, что позволяет начать «если не массовую,
то масштабную вакцинацию» препаратом «Спутник V». Первая в мире
антикоронавирусная вакцина, разработанная в центре имени Гамалеи, была
зарегистрирована в августе.
