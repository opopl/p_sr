% vim: keymap=russian-jcukenwin
%%beginhead 
 
%%file 03_09_2021.fb.kotenko_tatiana.1.brajchevskij_mihail_muzej
%%parent 03_09_2021
 
%%url https://www.facebook.com/kotenko.moracha/posts/4170217749762803
 
%%author_id kotenko_tatiana
%%date 
 
%%tags brajchevskij_mihail.ukraina.filosof.istorik.1924_2001,istoria,kultura,muzei,ukraina
%%title Треба створити музей на честь українського історика, мислителя і громадського діяча Михайла Брайчевського
 
%%endhead 
 
\subsection{Треба створити музей на честь українського історика, мислителя і громадського діяча Михайла Брайчевського}
\label{sec:03_09_2021.fb.kotenko_tatiana.1.brajchevskij_mihail_muzej}
 
\Purl{https://www.facebook.com/kotenko.moracha/posts/4170217749762803}
\ifcmt
 author_begin
   author_id kotenko_tatiana
 author_end
\fi

\begin{verbatim}
	#УС   
\end{verbatim}

Треба створити музей на честь українського історика, мислителя і громадського
діяча Михайла Брайчевського 

«Моя кохана Україно,
Рожевих мрій блакитна даль!
У нас з тобою путь єдина,
Єдина радість і печаль…»

Михайло Брайчевський «Сповідь»

У цьому році виповнюється 20 років від дня смерті філософа і поета української
історії Михайла Брайчевського - народився він 6 вересня 1924 року, помер 23
жовтня 2001 року.

\ifcmt
  ig https://scontent-frt3-1.xx.fbcdn.net/v/t1.6435-9/241133184_4170238539760724_8501218121513246184_n.jpg?_nc_cat=107&_nc_rgb565=1&ccb=1-5&_nc_sid=730e14&_nc_ohc=3CvKPvGKK2kAX91WHeo&_nc_ht=scontent-frt3-1.xx&oh=e4f3b7ff250368236827265c0a58058c&oe=615F34CB
  @width 0.4
  %@wrap \parpic[r]
  @wrap \InsertBoxR{2}
\fi

Ще за життя він став знаковою фігурою своєї доби. Це була людина-енциклопедист.
За час своєї інтенсивної наукової діяльності, яка тривала понад півстоліття, М.
Брайчевським написано понад 600 праць, в яких автор подає власне тлумачення
багатьох історичних подій. Глибокий та системний аналіз різних видів джерел
(писемних, археологічних, лінгвістичних, етнографічних тощо) дозволив створити
власну концепцію давньої та середньовічної історії України.

Він активно досліджував питання пізньосередньовічної історії, розробляв
проблему використання кібернетики в археології. Чимало його праць присвячено
різноманітним явищам культури, літератури, мистецтва. Багато уваги вчений
надавав проблемі збереження історичної та культурної спадщини України. 

Велику цінність становить поетична спадщина Михайла Брайчевського.

Брайчевський також захоплювався малярством. Його портретна галерея державних та
культурних діячів Київської Русі, виконана в оригінальній манері, експонувалась
не лише в Україні, а й у Великобританії.

Михайла Брайчевського дуже турбувало становище української мови в Україні. У
2000 році він звернувся до Голови НТШ в Україні, академіка НАН України Олега
Романіва із листом, у якому зазначав:

«Парадокс, але в незалежній Україні українська мова не лише не отримала
розвитку, а навіть стала витіснятися з ужитку мовою сусідньої держави. Все
більш брутальне зросійщення засобів масової інформації, російськомовне
викладання у переважній більшості шкіл і вузів півдня й сходу України, книжкова
інтервенція з Росії – все це може призвести до справжнього геноциду української
мови в Україні, до втрати незалежності… »  

Свої конкретні пропозиції для виправлення становища у мовному питанні Михайло
Брайчевський виклав у відкритому листі Президенту України, опублікованому в
газетах «Столиця» від 8.04.2000 та «Українське Слово» від 27.04.- 10.05.2000.

Головними серед його рекомендацій є такі:

«Українська мова є обов’язковою для всіх державних інституцій. Незнання її
автоматично робить неможливим посідання будь-яких адміністративних посад у
державних установах.

Виключається можливість виступів українських парламентарів в стінах парламенту
недержавною мовою. Під час офіційних візитів за межі України державні посадові
особи мусять робити виступити та давати інтерв’ю засобам масової інформації
державною мовою України.

Для забезпечення можливості опанування українською мовою тим, хто нею не
володіє, створити в усіх містах України курси прискореного вивчення української
мови – з гнучким графіком занять. Оплата праці викладачів, оренди приміщень має
здійснюватися не за державний рахунок, а коштом слухачів курсів.

Викладання в школах ведеться українською мовою. Як виняток, у місцях
компактного проживання національних меншин допускається створення шкіл з
викладанням місцевою мовою, але з обов’язковим предметом української мови та
літератури.

Викладання в школах російської мови як окремого предмету відмінити; для осіб,
які бажають її вивчати, передбачити факультативні заняття у позаурочний час (як
для другої іноземної мови).

Викладання у вищих та середніх спеціальних навчальних ведеться виключно
українською мовою.»

Інтенсивна наукова діяльність не заважала Михайлові Брайчевському завжди
перебувати у вирі суспільних подій, які відбувалися в Україні. Активну участь
брав Михайло Брайчевський свого часу в діяльності знаменитого Клубу творчої
молоді. Він приходив туди після роботи і розповідав слухачам про історичні
пам’ятки й споруди Києва, про історію їх виникнення. Розповіді доповнювались
екскурсіями.

На початку 1960 року було створено Клуб творчої молоді «Сучасник», який очолив
Лесь Танюк. На засідання клубу приходили В’ячеслав Чорновіл, Василь Симоненко,
Ліна Костенко, Іван Дзюба, Іван Світличний, Євген Сверстюк, Алла Горська,
Віктор Зарецький, Опанас Заливаха та багато інших. З лекціями на історичну тему
виступали Михайло Брайчевський, Олена Апанович, Олена Компан, Григорій Логвин,
Олесь Силин. Згуртовував молодь у своєму домашньому музеї видатний художник,
скульптор, збирач українського народного мистецтва св. п. Іван Гончар.

Було б добре створити музей імені Михайла Брайчевського, присвячений йому і
членам Клубу творчої молоді. Влаштувати його можна у приміщенні колишньої школи
№83, розташованої на Лютеранській вулиці. У цій школі, як розповідав Юрій
Кухарчук, свого часу учився Михайло Брайчевський, а зараз школа стоїть пусткою.
Я, коли проходжу повз школу, мрію, щоб надали їй друге життя. Там зручні для
музею приміщення.

Заслужений працівник культури України

Ольга Веремійчик

\ii{03_09_2021.fb.kotenko_tatiana.1.brajchevskij_mihail_muzej.cmt}
