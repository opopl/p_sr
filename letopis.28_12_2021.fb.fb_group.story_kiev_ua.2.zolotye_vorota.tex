% vim: keymap=russian-jcukenwin
%%beginhead 
 
%%file 28_12_2021.fb.fb_group.story_kiev_ua.2.zolotye_vorota
%%parent 28_12_2021
 
%%url https://www.facebook.com/groups/story.kiev.ua/posts/1828262760703850
 
%%author_id fb_group.story_kiev_ua,fedjko_vladimir.kiev
%%date 
 
%%tags gorod,istoria,kiev,zolotyje_vorota.kiev
%%title Справжня історія Золотих воріт...
 
%%endhead 
 
\subsection{Справжня історія Золотих воріт...}
\label{sec:28_12_2021.fb.fb_group.story_kiev_ua.2.zolotye_vorota}
 
\Purl{https://www.facebook.com/groups/story.kiev.ua/posts/1828262760703850}
\ifcmt
 author_begin
   author_id fb_group.story_kiev_ua,fedjko_vladimir.kiev
 author_end
\fi

Справжня історія Золотих воріт...

***

Ми звикли до вигляду руїн, які називаються «Золоті ворота»... 

Але чи справді ці руїни відносяться до Золотих воріт, з яких починався Київ у
часи свого розквіту і простягався далі на північ і схід до величезної
Кожем’яцької ущелини, внизу якої лежить інша частина древнього Києва – Поділ?

\ii{28_12_2021.fb.fb_group.story_kiev_ua.2.zolotye_vorota.pic.1}

Можна категорично стверджувати – НІ!

Передбачаю обурливі коментарі... Та хто ти такий, щоб «категорично
стверджувати»?!

Так! Я не археолог, не архітектор і не знавець давнього Києва. Але я – людина
допитлива і уважний читач)

\ii{28_12_2021.fb.fb_group.story_kiev_ua.2.zolotye_vorota.pic.2}

Декілька років тому назад був у мене цікавий проект, надиханий дружбою з отцем
Борисом Табачеком, настоятелем Свято-Володимирського патріаршого собору столиці
– створити електронну серію книг-репринтів «Киев и его святыни». 

\ii{28_12_2021.fb.fb_group.story_kiev_ua.2.zolotye_vorota.pic.3}

За основу я взяв 5 книг – путівників по Києву, виданих у 1864 – 1917 роках.
Першу частину проекту я реалізував – перевів путівники в електронну форму
(*pdf). 

\ii{28_12_2021.fb.fb_group.story_kiev_ua.2.zolotye_vorota.pic.4}

Щодо запланованої другої частини – поєднати старі путівники з сучасними
фотографіями православних святинь Києва, про які йде мова у путівниках, то на
жаль, у зв’язку із завантаженістю військовими і волонтерськими справами, а
потім і з важкою хворобою дружини, реалізувати не вдалося. 

\ii{28_12_2021.fb.fb_group.story_kiev_ua.2.zolotye_vorota.pic.5}


Вважаю, що членству нашого чудового клубу буде цікаво прочитати справжню
історію Золотих воріт!

P.S.

У зв’язку із складністю перекладу тексту книги українською мовою, я вирішив
просто навести уривок про історію Золотих воріт в оригіналі путівника.

\ii{28_12_2021.fb.fb_group.story_kiev_ua.2.zolotye_vorota.pic.6}
\ii{28_12_2021.fb.fb_group.story_kiev_ua.2.zolotye_vorota.pic.7}

\ii{28_12_2021.fb.fb_group.story_kiev_ua.2.zolotye_vorota.cmt}
