% vim: keymap=russian-jcukenwin
%%beginhead 
 
%%file 17_02_2022.stz.news.ua.fraza.1.kak_zapadnyje_partnery_opuskajut_ukrainu.3.finansy_marshi
%%parent 17_02_2022.stz.news.ua.fraza.1.kak_zapadnyje_partnery_opuskajut_ukrainu
 
%%url 
 
%%author_id 
%%date 
 
%%tags 
%%title 
 
%%endhead 

\subsubsection{Финансы поют... военные марши}
\label{sec:17_02_2022.stz.news.ua.fraza.1.kak_zapadnyje_partnery_opuskajut_ukrainu.3.finansy_marshi}

Курс гривны к доллару во время кризиса неуклонно рос и реально грозил
достигнуть психологической отметки в 30 гривен за доллар. Тот факт, что это
было связано, прежде всего, с эскалацией военных маневров России и с истерией,
раздутой вокруг этого Западом, подтверждается тем, что как только в СМИ
появились сообщения об окончании российских маневров и отводе российских войск
от границ Украины, гривна немедленно стала укрепляться. Хотя до стабилизации
еще далеко, и неизвестно, чем это окончится в конечном итоге.

Даже новость о возможном признании Москвой ЛДНР на курс гривны особого влияния
не оказала. Хотя должна была. Очевидно, потому что эта новость была не в фокусе
медиа, и особого шума вокруг нее не было.

Более того, даже после новостей об отводе российских войск продолжился
панический вывод из Киева во Львов (или вовсе в родные пенаты) дипломатов
целого ряда стран. Но даже это не помешало укреплению гривны. Это еще одно
доказательство того, что именно раздутая в западных СМИ паника на тему «Путин
нападет» стала главной причиной падения гривны в частности и огромного
финансово-экономического ущерба в целом.

Потери Украины и ее финансово-банковской системы от раздутой паники
иллюстрируются следующими цифрами.

Для удержания курса гривны Национальный банк Украины только 14 февраля вынужден
был продать на межбанковском валютном рынке 377 млн долларов. А с начала 2022
года для стабилизации курса Нацбанк продал 1,93 млрд долларов из валютных
резервов.

По данным Национального банка Украины, в январе 2022 года физлица забрали из
банков свои денежные накопления в размере 23 млрд грн. Таким образолм украинцы
начали выводить деньги из банковского сектора на случай непредвиденных событий,
преимущественно связанных с угрозой войны.

Все это время на мировом финансовом рынке наблюдался массовый сброс украинских
еврооблигаций. Резко взлетели ставки по украинским бондам с погашением в
сентябре 2022 года. Если в пятницу 11 февраля они находились на уровне 15,83\%,
то уже в понедельник, 14 февраля, утром поднялись до 28,54\% — это новый
максимум.

И дело не только в том, что держатели избавлялись от украинских долговых бумаг.
Кстати, если бы во власти находились настоящие умные патриоты, хотя бы наиболее
критичную часть этих облигаций, хорошо поторговавшись, можно было бы на волне
паники дешево выкупить, чтобы минимизировать расходы на обслуживание долга. Но
наши «патриоты» умеют только размахивать прапором и петь гимн, а в промежутках
воровать у родного отечества.

Но хуже всего то, что теперь Украина полностью отрезана от международных рынков
заимствования. Нам либо денег в долг не дадут, либо будут давать под
космические проценты, обслуживать которые просто нереально. То есть дырки в
бюджете, в том числе и для расчетов по предыдущим долгам, затыкать будет нечем.
За это нашим англо-саксонским «друзьям», раздувшим истерию, следует сказать
отдельное огромное «спасибо».

Впрочем, и это, думается, делалось с явным расчетом. Теперь Украина попадет в
еще большую зависимость от Запада, прежде всего от Штатов. Поскольку сможет
получить финансирование либо от Вашингтона, либо от МВФ, то есть от того же
Вашингтона, причем под кабальные условия, в частности, под повышение
коммунальных тарифов и т.п.

Как я написал выше, Киеву Штаты предложили 1 миллиард под гарантии США. Только
здесь довольно своеобразная схема, которая совершенно не гарантирует даже
низкий процент. Под эти гарантии Минфин выпускает долговые обязательства,
уплата которых гарантируется правительством США. Если Украина не сможет
расплатиться, нас опять нагнут, чтобы платить какие-нибудь ВВП-варранты, как
это уже было. Не говоря уже о том, что на этой сделке наживутся банки, которые
получат хорошие комиссионные. Но главное — поскольку украинские долги, как
сказано выше, резко обесценились, никто не гарантирует низкий процент и скорее
всего, доходность будет высокой, то есть Украина получит дополнительную
кредитную удавку.

