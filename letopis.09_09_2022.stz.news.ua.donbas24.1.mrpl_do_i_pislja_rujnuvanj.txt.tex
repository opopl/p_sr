% vim: keymap=russian-jcukenwin
%%beginhead 
 
%%file 09_09_2022.stz.news.ua.donbas24.1.mrpl_do_i_pislja_rujnuvanj.txt
%%parent 09_09_2022.stz.news.ua.donbas24.1.mrpl_do_i_pislja_rujnuvanj
 
%%url 
 
%%author_id 
%%date 
 
%%tags 
%%title 
 
%%endhead 

Алевтина Швецова (Марiуполь)
Маріуполь,Україна,Мариуполь,Украина,Mariupol,Ukraine,date.09_09_2022
09_09_2022.alevtina_shvecova.donbas24.mrpl_do_i_pislja_rujnuvanj

Маріуполь до та після руйнувань: як зараз виглядають історичні будівлі міста Марії 

Квітучий і сучасний Маріуполь до повномасштабного вторгнення росії вражав
місцевих мешканців і гостей міста Марії своєю архітектурою. Будівлі в центрі
Маріуполя, які зводилися більше 100 років тому, вже давно заслужили звання
візитівок міста.

Після 24 лютого 2022 року ворожа авіація та артилерія руйнувала все навколо:
вщент знищений історичний центр міста разом з його унікальною архітектурою. На
сторінці Маріуполь — туристичне місто у Facebook з'явився допис з унікальними
світлинами, які демонструють стан будинків на зараз.

Це не просто фотографії, а наглядні порівняння: що сталося з містом за останні
пів року. Небайдужа маріупольчанка знайшла на смітнику листівки, які минулого
року виготовила команда культурно-туристичного центру «Вежа». На них зображені
вуличні маріупольські пейзажі та архітектурні пам'ятки. Дівчина побувала в цих
локаціях і зафіксувала, як сьогодні виглядають ці місця. Коментарі зайві.

Нагадаємо, раніше Донбас24 писав, що путін намагається відкотити назад зернову угоду та вдається до шантажу.

Ще більше новин та найактуальніша інформація про Донецьку та Луганську області в нашому телеграм-каналі Донбас24.

ФОТО: Маріуполь — туристичне місто
