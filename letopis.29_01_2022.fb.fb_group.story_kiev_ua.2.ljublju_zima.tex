% vim: keymap=russian-jcukenwin
%%beginhead 
 
%%file 29_01_2022.fb.fb_group.story_kiev_ua.2.ljublju_zima
%%parent 29_01_2022
 
%%url https://www.facebook.com/groups/story.kiev.ua/posts/1850426641820795
 
%%author_id fb_group.story_kiev_ua,kyzylova_myroslava.kiev
%%date 
 
%%tags kiev,kievljane,pismo,zima
%%title Люблю зиму! Час, де менше метушні, кольорів, роботи!
 
%%endhead 
 
\subsection{Люблю зиму! Час, де менше метушні, кольорів, роботи!}
\label{sec:29_01_2022.fb.fb_group.story_kiev_ua.2.ljublju_zima}
 
\Purl{https://www.facebook.com/groups/story.kiev.ua/posts/1850426641820795}
\ifcmt
 author_begin
   author_id fb_group.story_kiev_ua,kyzylova_myroslava.kiev
 author_end
\fi

Люблю зиму! Час, де менше метушні, кольорів, роботи!

Як дихається легко! Прохолода повітря сама вливається в тебе, освіжає,
прояснює. Лаконічність пейзажу - трохи чорного та біле-біле. Легка інтрига під
покривалом снігу. Земля спочиває... Рівновага спокою заповнює і тебе.

Люблю зиму. Час, в якому ти більше належиш собі. Коли можна дозволити собі
спонтанний відпочинок чи побавити себе дивними дрібницями) 

\raggedcolumns
\begin{multicols}{3} % {
\setlength{\parindent}{0pt}

\ii{29_01_2022.fb.fb_group.story_kiev_ua.2.ljublju_zima.pic.1}
\ii{29_01_2022.fb.fb_group.story_kiev_ua.2.ljublju_zima.pic.1.cmt}

\end{multicols} % }

Знайшлася пачка старих листів з мого дитинства. Читати цікаво і смішно.
Дитяча безпосередність, відвертість, наївність. Накрило спогадами. Інший вік
та інша епоха.

Про що писали? Про те, як важко дочекатися канікул, і «Чи ти приїдеш до нас?»
Про оцінки в табелі) Про страшні секрети, які розповім лише при зустрічі (хоча
більшість із них зводилася до того, як ВІН на неї подивився і, о чудо – передав
записку)) Про те, як до свята прикрашали зал у школі, але при цьому більше
подробиць, як дуріли (в сенсі бешкетували). І, звичайно, про великі плани на
майбутнє, – «Хочу стати слідчим, але бабуня каже, що не жіноча робота»))

...І виринають у пам’яті обличчя, епізоди, запахи і звуки, давно забуті. Листівки
неначе живі. До 8 Березня – пахнуть натяком на весну та життя, що буде
попереду.

Як свідки епохи, що минула, примітивні зображення до радянських свят. Ми,
діти, теж вітали одне одного з 7 листопада і бажали успіхів у навчанні.
Знайшла дивні вітання незнайомих мені співробітників (з роботи батьків) та їх
побажання бути справжнім ленінцем. ...

Час мого дитинства – тотальна дружба. У класі, у дворі нашої п’яти поверхівки
на Відрадному, цілою вулицею у бабусі. І в листуванні. Мода на листування прямо
культивувалася. Діставали адреси незнайомих однолітків з інших міст чи країн
(особливо з ГДР). Або просто адресували невідомому учню якоїсь школи, такого-то
класу, такого-то за списком у класному журналі.

Та була ще одна причина пристрасті до листування – колекціонування.

У 70-ті прямо якась пошесть. Здається цим хворіла більшість. Збирали все:
фантики цукерок, марки, листівки, переводки, картинки від конвертів, наклейки
із сірникових коробок, пляшок, монети, фото акторів і т. п. Діти і дорослі, мов
діти, були одержимі тим захопленням. Збирали навіть наші вчителі) Радість
володіти, вивчати інфо, демонструвати і, звичайно, обмінюватися і якнайвища
радість – дарувати. В обмінах чи даруванні у нас дітей шліфувався характер,
етика – жаль-не жаль, дарма-не дарма, міцніло відчуття дружби. Частіше таких
хобі було не одне. Якщо натрапляв на інший скарб, що не відносився до твого
напрямку, все одно брав його – подарувати друзям. Пригадую, як у 5-му класі я
влаштувала лотерею на великій перерві. Розігрували фото акторів кіно. Журнал
«Новини кіноекрану» ми з Антошкою (моєю найліпшою подругою) старанно порізали
на кадри і фото, поскручували папірці-номерки. Скільки ж у всіх було щирої
дитячої радості, просто так, задарма – маленьке несподіване свято)

Серед листів є листівки-дарунки, просто так чи як найдорожчий скарб, що віддала
подружка із колекції тата. Їм більше 100 років.

А ось кумедні застереження з листів: «...якщо я буду висилать тобі марки, які в
тебе будуть, то ти їх висилай обратно, бо я теж висилаю другим дівчатам і
сильно утрачатимусь.» Перші бізнес відносини)) Та це не про жадібність. Часи
були інші, інші статки, та й звідки у нас дітей ті гроші. Листівка – 2 чи 3
коп. Це проїзд на трамваї, склянка солодкої газировки, булочка, аж цілих 3
коробки сірників. Тому булочка часто проходила мимо))

...Якось з книжного ринку на Петрівці зазирнула на блошиний ринок, що поряд. Тут
і досі люди у віці шукають скарби до своїх колекцій, як сплески емоційної
пам’яті.

...Зима, люблю тебе! Дякую за чистоту білого світу в снігу! За віхолу
сніжинок-спогадів і броунівський рух, що заніс мене в дитинство мого життя.
