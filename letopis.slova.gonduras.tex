% vim: keymap=russian-jcukenwin
%%beginhead 
 
%%file slova.gonduras
%%parent slova
 
%%url 
 
%%author 
%%author_id 
%%author_url 
 
%%tags 
%%title 
 
%%endhead 
\chapter{Гондурас}
\label{sec:slova.gonduras}

%%%cit
%%%cit_head
%%%cit_pic
%%%cit_text
О вредности социальных шаблонов, в которые нас загоняют манипуляторы.  Покидая
\emph{Гондурас}, я в который раз задумался о шаблонах, в которые нас загоняют
манипуляторы. Помните - «не ту страну назвали \emph{Гондурасом}!» На самом деле, страну
населяют добрые, отзывчивые люди. Есть и жестокие банды и негодяи.. Но где их
нет?! Есть хорошие дороги (не везде). Есть древняя история и перспективное
будущее. (Если хотите - расскажу об удивительных перспективных проектах).  Но
хочу вам рассказать о ммстных чиновниках, которые определяют уровень
доброжелательности/не доброжелательности страны (как, впрочем, и уровень
коррупции)
%%%cit_comment
%%%cit_title
\citTitle{Не все так просто и однозначно с этим Гондурасом... / Лента соцсетей / Страна}, Константин Стогний, strana.ua, 24.06.2021
%%%endcit
