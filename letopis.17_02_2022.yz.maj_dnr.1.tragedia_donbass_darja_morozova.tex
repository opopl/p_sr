% vim: keymap=russian-jcukenwin
%%beginhead 
 
%%file 17_02_2022.yz.maj_dnr.1.tragedia_donbass_darja_morozova
%%parent 17_02_2022
 
%%url https://zen.yandex.ru/media/id/5f8f226b1fe36c1d9e02a36b/my-nikomu-ne-pozvolim-obescenivat-tragediiu-donbassa--daria-morozova-620e2d144290e17e68c5a634
 
%%author_id yz.maj_dnr
%%date 
 
%%tags dnr,donbass,tragedia,vojna
%%title Мы никому не позволим обесценивать трагедию Донбасса – Дарья Морозова
 
%%endhead 
 
\subsection{Мы никому не позволим обесценивать трагедию Донбасса – Дарья Морозова}
\label{sec:17_02_2022.yz.maj_dnr.1.tragedia_donbass_darja_morozova}
 
\Purl{https://zen.yandex.ru/media/id/5f8f226b1fe36c1d9e02a36b/my-nikomu-ne-pozvolim-obescenivat-tragediiu-donbassa--daria-morozova-620e2d144290e17e68c5a634}
\ifcmt
 author_begin
   author_id yz.maj_dnr
 author_end
\fi

\ii{17_02_2022.yz.maj_dnr.1.tragedia_donbass_darja_morozova.pic.1}

Я ничего не буду комментировать или добавлять к заявлению Уполномоченного по
правам человека ДНР, Полномочного представителя Донецкой Народной Республики на
Минских переговорах Дарьи Морозовой. Я просто полностью процитирую Дарью
Васильевну:

\begin{zzquote}
"Следственный комитет России возбудил уголовное дело об обнаружении массовых
безымянных захоронений жертв украинской агрессии в Донбассе. По данным
Комитета, в период с августа по октябрь 2021 года из пяти стихийных массовых
захоронений на территории ДНР и ЛНР были эксгумированы останки не менее 295
гражданских лиц, убитых обстрелами ВСУ в 2014 году.

Уголовное дело возбуждено по части 1 статьи 356 Уголовного кодекса РФ, которая
четко определяет характер преступлений ВСУ в нашем регионе – жестокое обращение
с гражданским населением, применение в вооруженном конфликте средств и методов,
запрещенных международным договором.

Речь идет об умышленном уничтожении мирных жителей. О таком масштабе
вооруженной агрессии, при которой люди были лишены возможности даже хоронить
убитых в соответствии с канонами. Фактически данное заявление Комитета стало
очередным подтверждением слов Президента Российской Федерации Владимира Путина,
который назвал события в Донбассе геноцидом.

Напомню, что с 2014 года только на территории ДНР жертвами обстрелов ВСУ стали
свыше 5000 человек. В числе погибших – 91 ребенок. Еще 8000 человек ранены. И,
к сожалению, мы понимаем, что в реальности количество погибших значительно
больше. Что, в свою очередь, и подтверждают факты обнаружения мест массовых и
стихийных захоронений. С начала конфликта в ДНР выявлены порядка 130 таких
локаций. И это лишь около 40\% от общего количества.

Я не стану давать оценку моральным качествам официального спикера Белого дома,
которая позволила себе назвать озвученные СК данные дезинформацией. Но
приглашаю всех, кто отрицает геноцид в Донбассе, присоединиться к поисковым
мероприятиям в Республиках. Увидеть изувеченные останки женщин и детей.
Посмотреть в глаза родственникам граждан, пропавших без вести в ходе конфликта.
А главное – объясниться с людьми, годами живущими в горе безвестности.

Мы восемь лет наблюдаем, как международное сообщество пытается игнорировать
трагедию Донбасса. И все это время цивилизованными, правовыми способами
добиваемся адекватной реакции на происходящее в регионе. Но спекулировать на
нашем горе и обесценивать его не позволим никому.

Мы фиксируем каждый факт, доказывающий, что действия киевского режима в
Донбассе имеют признаки преступлений против человечества. И уверяю, не
остановимся, пока виновные не будут привлечены к ответственности. Напомню, что
военные преступления и преступления геноцида не имеют срока давности. Суд
неизбежно свершится. И ответить придется не только инициаторам и исполнителям
смертоносных приказов, но и пособникам расцвета фашизма в современной Украине".	
\end{zzquote}

\url{https://ombudsman-dnr.ru/?p=18882}
