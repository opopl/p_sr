%%beginhead 
 
%%file 27_11_2022.fb.mariupol.turystychne_misto.1.ekskursijni_proguljanky_240_paraleli
%%parent 27_11_2022
 
%%url https://www.facebook.com/mistoMarii/posts/pfbid02S2ybCcFR4kDVWHvLsD5KxuyKoh25hCyv95urkXr7smaUhXsEft9sonrp9xbCMwCkl
 
%%author_id mariupol.turystychne_misto
%%date 27_11_2022
 
%%tags 
%%title Екскурсійні прогулянки - проєкт "2:40. Паралелі"
 
%%endhead 

\subsection{Екскурсійні прогулянки - проєкт \enquote{2:40. Паралелі}}
\label{sec:27_11_2022.fb.mariupol.turystychne_misto.1.ekskursijni_proguljanky_240_paraleli}

\Purl{https://www.facebook.com/mistoMarii/posts/pfbid02S2ybCcFR4kDVWHvLsD5KxuyKoh25hCyv95urkXr7smaUhXsEft9sonrp9xbCMwCkl}
\ifcmt
 author_begin
   author_id mariupol.turystychne_misto
 author_end
\fi

%ia_tags: Маріуполь,Україна,Мариуполь,Украина,Mariupol,Ukraine,Ольга Демідко,Алевтина Швецова,Марія Кутнякова,Excursion,Екскурсія,Экскурсия,Kiev,Kyiv,Киев,Київ,Ужгород,Вільнюс,date.27_11_2022

З Маріуполем у серці 💙 Екскурсійні прогулянки в межах проєкту \enquote{2:40. Паралелі}
вийшли за межі України

Цими вихідними маріупольці, які наразі перебувають у Києві, Ужгороді та
Вільнюсі, разом з місцевими жителями цих міст відвідали надзвичайно цікаві та
пізнавальні екскурсії від ініціативних та закоханих у місто Марії гідів.

🗺 У столиці тематичну прогулянку історично важливою вулицею проводила
кандидатка історичних наук, доцентка МДУ, ексурсовод, журналістка
Маріупольського телебачення і Донбас24 \href{\urlDemidkoIA}{Ольга Демідко}:

\begin{quote}
\enquote{Я вкотре  приємно
вражена нашими людьми. Маріупольці, які були вимушені виїхати з рідного міста
та перебувають наразі в Києві, попри дощ і холод, прийшли на екскурсію, щоб
поринути в історію своєї країни та згадати архітектурні магніти рідного
міста.🏰😊 Сьогодні  я побачила щирих, відкритих та сильних духом Людей, серед
яких були і мої студентки, 🤗які, попри всі випробування зберегли внутрішнє
Світло і вірять всім серцем в деокупацію Маріуполя та якомого швидшу Перемогу
нашої країни! 🙏🇺🇦 }
\end{quote}

Амбасадорка Маріуполя, журналістка, громадська та культурна діячка Алевтина
Швецова проводила паралелі в історії Ужгорода та Маріуполя: 

\begin{quote}
\enquote{Дивовижно, але для
мене паралелі вишукалися в одну пряму – минуле, сьогодення, майбутнє. все
завдяки зустрічі Sergey Vaganov і Ирина Горбасёва. Наче пішохідний місточок в
старому Ужгороді – до мети не дійдеш, якщо застрягнеш над водою. тому попри
пережите жахіття, ми пам'ятаємо, шануємо минуле й йдемо далі...}
\end{quote}

🇱🇹 А громадська та культурна діячка, журналістка Марія Кутнякова у Вільнюсі
розповідала про головні цікавинки міста янголів та приазовської культурної
столиці: 

\begin{quote}
\enquote{Знаходила схожі історії, будівлі, пам'ятники... ми стоїмо біля
пам'ятника водоносу, тут я згадувала Вежу та міні-Нільсена. Ех, хочеться
додому, пройтися не знищеним старим центром. Дуже хочеться. Пам'ятаю бабуся
розказувала мені, як поверталася у 1944 році до деокупованого Маріуполя: \enquote{Маша,
ми їхали головним проспектом, а на ньому всі будівлі потрощені. Ти навіть
неуявляєш як це}. Тепер уявляю. Тоді місто відбудували, і цього разу теж.
Головне деокупувати}.
\end{quote}

Дякуємо нашім гідам та екскурсантам і вже чекаємо на нові цікаві прогулянки
іншими містами.
