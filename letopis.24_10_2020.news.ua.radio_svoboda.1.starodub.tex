% vim: keymap=russian-jcukenwin
%%beginhead 
 
%%file 24_10_2020.news.ua.radio_svoboda.1.starodub
%%parent 24_10_2020
%%url https://www.radiosvoboda.org/a/30903758.html
 
%%endhead 

\subsection{Українська Стародубщина, яка нині у складі Росії. Край козацького літописання}
24 жовтня 2020, 23:56, Петро Кралюк
\url{https://www.radiosvoboda.org/a/30903758.html}

Тестамент (духовний заповіт) 1731 року на користь дружини і дочки військового
товариша Петра Романовського, онука Романа Ракушки-Романовського, якого
вважають автором «Літопису Самовидця». (Копія 1730-х років, засвідчена
військовим канцеляристом Григорієм Покорським) Фото:cdiak.archives.gov.ua
(Official website)

Дивитись коментарі (Рубрика «Точка зору»)

Попри те, що до Росії відійшло чимало українських етнічних територій, одна з
них має особливе значення для нас. Це – Стародубщина, яка нині є частиною
Брянської області Росії та займає в її складі 12 районів із 27. Це –
Гордіївський, Злинківський, Климівський, Клинцівський, Красногорський,
Мглинський, Новозибківський, Погарський, Почіпський, Стародубський, Суразький
та Унецький. Їхня загальна площа становить 14 762 квадратних кілометри (понад
40 відсотків від загальної території області).

Історія цієї української землі надзвичайно цікава. Її детально описав
український історик, який мешкає в Москві, Ігор Роздобудько в своєму
монографічному дослідженні «Стародубщина. Нарис українського життя краю».
Предки цього дослідника походять зі Стародубщини.

\ifcmt
pic https://gdb.rferl.org/A26CB8D3-63D6-4974-9B74-7E877CDA63D6_w1597_n_r0_st.png
\fi

Стародубщина на мапі сучасної Східної Європи

\ifcmt
pic https://gdb.rferl.org/4E5B5D5A-955B-4CEE-84AB-A99005F9070C_w1023_r0_s.jpg
pic https://gdb.rferl.org/C2AD597C-33CA-4C3C-9D03-7B6EAC2B067D_w1023_r0_s.png
pic https://gdb.rferl.org/0DB3E481-910D-4819-8228-09F965A0960B_w1023_r0_s.jpg
\fi

Супутниковий знімок усієї Брянської області

\ifcmt
img_begin 
  tags 
  url https://gdb.rferl.org/F5BB500D-1DCA-41F1-9791-F82F54C16324_w1597_r0_s.jpg
  caption Околиця міста Стародуба (частина міста на фото ліворуч). Травень 2019 року
  width 0.7
img_end
\fi

\ifcmt
img_begin 
  url https://gdb.rferl.org/AE6C8B4D-A15C-4731-8456-8AD06471A506_w1597_r0_s.jpg
  caption Околиця міста Стародуба (частина міста на фото ліворуч). Червень 2019 року 
  width 0.7
img_end
\fi

Не будемо звертатися до давньої історії краю. Стародубщина була в складі Русі,
існувала і, як удільне князівство, підпорядковувалася золотоординським ханам,
входила до Великого князівства Литовського й Московії. У 1618 році за умовами
Деулінського перемир’я з Москвою ця земля стала частиною Речі Посполитої.
Стародуб у 1620 році отримав Магдебурзьке право (це якраз 400 років тому – чом
би Україні не відзначити такий ювілей?). Під час повстання під проводом Богдана
Хмельницького українські козаки зайняли Стародубщину. Із 1663 року тут існував
Стародубський козачий полк.

\ifcmt
img_begin 
  tags 
  url https://gdb.rferl.org/4B8F9B92-25A3-4B55-A3ED-399357383814_w650_r0_s.jpg
  caption Козацький полковий собор Різдва Христового у Стародубі. Головний храм
  Стародубського козацького полку. Побудований у 1677 році на місці знищеної
  пожежею дерев’яної церкви 1617 року
  width 0.7
img_end
\fi

Потрапивши після Переяславської ради до складу Московії, а потім Російської
імперії, Стародубщина залишалася одним із значних осередків українського життя.
Тут народилися або були пов’язані з нею чимало діячів української культури,
українського громадського й політичного життя. Наприклад, родинні корені
гетьмана Павла Скоропадського зі Стародубщини, а сам він навчався в
Стародубській гімназії.

\ifcmt
img_begin 
  url https://gdb.rferl.org/0B3951A2-F731-4D97-9E6D-4F78D0EA0E92_w650_r0_s.jpg
  caption Павло Скоропадський, гетьман Української Держави (29 квітня – 14 грудня 1918)
  width 0.7
img_end
\fi

«Стародубське походження» має Леся Українка – у цьому краї, в місті Мглин,
народився її батько Петро Косач.

\ifcmt
img_begin 
  url https://gdb.rferl.org/6CB50A56-1654-4D37-BA89-A842A263BA86_w650_r0_s.jpg
  caption Батько Лесі Українки Петро Косач (1 січня 1842, Мглин, Чернігівська
  губернія, Стародубщина (нині Брянська область Росії) – 15 (28) квітня 1909,
  Київ) – український юрист, громадський діяч, освітянин, дійсний статський
  радник, меценат, член «Старої громади»
  width 0.7
img_end
\fi

\ifcmt
img_begin 
  url https://gdb.rferl.org/1B9E502E-765E-4331-B6DE-DACDE4014A21_w650_r0_s.jpg
  caption Леся Українка з матір’ю Оленою Пчілкою. Ялта, 1898 рік
  width 0.7
img_end
\fi

Зі стародубського козацького роду були й предки відомого дослідника та
мандрівника Миколи Миклухо-Маклая, про що він сам говорив.

\ifcmt
img_begin 
  url https://gdb.rferl.org/EBD39637-F39B-4FD3-A807-C1D07E9BB35E_w650_r0_s.jpg
  caption Микола Миклухо-Маклай (1846–1888) – учений-природознавець,
  антрополог, етнограф, мандрівник і суспільний діяч. Нащадок запорозького
  козацького сотника Степана Макухи, який походив із Стародубського повіту на
  Чернігівщині і був відомим інженером-залізничником, захоплювався творчістю
  Тараса Шевченка, клопотався про звільнення опального поета із заслання
  width 0.7
img_end
\fi

Можна назвати ще низку інших українських культурних діячів, пов’язаних із цим
краєм.

У 1918 році Стародубщина перебувала в складі українських державних утворень. За
умовами Брестського миру 1918 року вона ввійшла до Української Народної
Республіки.

\ifcmt
img_begin 
  url https://gdb.rferl.org/9BEA9312-ED9D-41BF-87E8-9EE9AACA3ADD_w1597_r0_s.jpg
  caption «Мапа Української Народної Республіки», видана в Харкові в 1918 році
  width 0.7
img_end
\fi

А згодом Стародубщина була у складі Української Держави гетьмана Павла Скоропадського.

\ifcmt
img_begin 
  url https://gdb.rferl.org/332B2C30-537D-4F76-8D0A-D3CD27658870_w650_r0_s.jpg
  caption Стародубщина на «Загальній карті України» М. Дячишина, 1918 рік
  width 0.7
img_end
\fi

\ifcmt
img_begin 
  url https://gdb.rferl.org/CD38A2F9-9EA4-4654-9B31-750D13CAC7DF_w1597_r0_s.jpg
  caption Карта 1918 року. Повна назва мапи: «Загальна карта України. Зладив М.
  Дячишин. Заходом i накладом час. „Свобода”, орґану Українського Нар. Союза в
  Америцї». Масштаб 1:2580000. Формат мапи 85x52 см. (Щоб відкрити мапу в
  більшому форматі, натисніть на зображення. Відкриється у новому вікні)
  width 0.7
img_end
\fi

Однак наприкінці 1918 року цей край зайняли більшовики, де розпочали терор
проти українців. За згодою маріонеткового українського
«робітничого-селянського» уряду Стародубщина перейшла до Росії. Відповідно до
рішення російського більшовицького уряду від 25 травня 1919 року цей край
увійшов до складу Гомельської губернії, яка, своєю чергою, стала частиною
Російської Соціалістичної Федеративної Радянської Республіки.

Коли ж більшу частину цієї губернії в 1926 році віддали Білорусі, Стародубщина
дісталася російській Брянській губернії. Радянській Україні віддали лише
невелику частину краю – Семенівську волость. Таке адміністративне становище
Стародубщини стало одним із головних чинників її русифікації.

\ifcmt
img_begin 
  url https://gdb.rferl.org/36b2a396-54ab-4d1a-9d73-8ce5fd7acb8a_w650_r0_s.jpg
  caption Українські жінки міста Стародуба, 1900-і роки
  width 0.7
img_end
\fi

\ifcmt
img_begin 
  url https://gdb.rferl.org/0CDA1960-A6E1-4F15-A801-29FD1DC2D36E_w650_r0_s.jpg
  caption Сечасне фото. Дитячий етнографічний колектив «Веселий екіпаж» (місто Стародуб)
  width 0.7
img_end
\fi

\subsubsection{Козацьке літописання на Стародубщині}

Саме на Стародубщині зародилося й розвинулося українське козацьке літописання,
яке відіграло чималу роль у становленні новочасної української національної
свідомості. Один із перших літописів, в яких знайшла відображення ідеологія
козацької старшини, був Літопис Самовидця.

Про автора цього твору можемо сказати наступне. Аналіз тексту дає підстави
стверджувати, що він під час повстання Богдана Хмельницького опинився на боці
козаків, був наближений до гетьманського уряду, став свідком різноманітних
подій другої половини XVII століття, пов’язаних із діяльністю козацьких
гетьманів.

url https://gdb.rferl.org/0EE47F4A-7509-4250-B181-A7A273A9B0E0_w1597_n_r0_st.jpg
caption 

Герб міста Стародуба, який був створений на основі герба 1620 року і герба
Стародубського полку. Стародуб – історичний центр історико-етнографічної землі
Стародубщина, центр одного з 10 адміністративних полків Гетьманщини і єдиний –
поза межами сучасної України.

У 1676–1702 роках Самовидець опиняється на Лівобережжі й починає описувати
події, які відбувалися в цьому регіоні. Схоже, тоді він мешкав у Стародубі,
оскільки часто говорить про події в цьому місті, в тому числі про такі, що мали
суто місцевий характер. Можна припустити: саме в Стародубі й була написана
основна частина Літопису Самовидця.

В українській історіографії набула поширення думка, що автором зазначеного
твору був виходець із дрібношляхетської родини Роман Ракушка-Романовський. Таку
версію запропонували Микола Петровський та Михайло Грушевський. Потім вона
знайшла помітну підтримку.

Про самого Ракушку-Романовського відомо, що він належав до освічених людей,
займав високі становища в гетьманському уряді, а в 1676 році став духовною
особою і оселився в Стародубі, де, ймовірно, завершив Літопис Самовидця. Тут
він отримав парафію, завів господарство і прожив аж до смерті в 1703 році.
Показово, що в Літописі Самовидця є звернення до церковних питань, особливо в
другій частині.

Літопис обіймає опис подій від часів повстання під проводом Богдана
Хмельницького до 1702 року. Його справедливо вважають одним із найкращих творів
українського козацького літописання, який став своєрідним бестселером для
пізніших козацьких літописців, українських істориків та письменників. Він,
поширюваний у чисельних списках, був важливим джерелом для літописів Григорія
Грабянки й Самійла Величка, «Короткого опису Малоросії», на основі якого були
укладені компілятивні праці з історії України Якова Лизогуба, Василя Рубана й
Олександра Рігельмана.

\ifcmt
img_begin 
  url https://gdb.rferl.org/E5279CD7-420E-4E77-BEBF-1567F6DF466C_w650_r0_s.jpg
  caption «Літопис Самовидця». Видання 1878 року
  width 0.7
img_end
\fi

У 1846 році Літопис Самовидця опублікували і він набув ще більшої популярності.
Широко використовували його українські історики ХІХ-ХХ століть – Михайло
Костомаров, Орест Левицький, Дмитро Яворницький, Дмитро Багалій, Іван
Крип’якевич та інші. Велику увагу цій пам’ятці приділив Пантелеймон Куліш – її
він використовував не лише в історичних працях, а й художньо-літературних
творах. Зверталися до Літопису Самовидця Тарас Шевченко, Іван Нечуй-Левицький,
Михайло Старицький, Зінаїда Тулуб, Олександр Довженко, Василь Стус та інші.

\ifcmt
img_begin 
        url https://gdb.rferl.org/116B459E-4EA8-44D2-80B3-8992E9D4E9A2_w1597_n_r0_st.jpg
        caption Тестамент (духовний заповіт) 1731 року на користь дружини і
        дочки військового товариша Петра Романовського, онука Романа
        Ракушки-Романовського, якого вважають автором «Літопису Самовидця».
        (Копія 1730-х років, засвідчена військовим канцеляристом Григорієм
        Покорським)
        width 0.7
img_end
\fi

Відомою поетичною пам’яткою, яка була створена на основі козацького
літописання, стала поема Семена Дівовича (Дзівовича) «Розмова Великоросії із
Малоросією 1762 року». Її автор походив із давнього козацького роду, який осів
на Стародубщині. Отримав добру освіту – навчався спочатку в Києво-Могилянській
академії, а потім у Петербурзі. Працював перекладачем у Генеральній військовій
канцелярії у Глухові та архіваріусом Малоросійського генерального архіву за
часів гетьманування Кирила Розумовського.

\ifcmt
img_begin 
  url https://gdb.rferl.org/38B7A2E5-6D36-4A15-9895-919128C036A1_w650_r0_s.jpg
  caption Кирило Розумовський (1728–1803) – український військовий, політичний та державний діяч, гетьман України. Представник козацького роду Розумовських.
  width 0.7
img_end
\fi

Згаданий твір поданий як поетичний діалог між Великоросією й Малоросією, власне
Україною. Переважно Великоросія ставить питання. Вона не дуже обізнана з
історією Малоросії. Малоросія, навпаки, багато говорить, намагається
проінформувати свою співбесідницю.

\begin{minipage}
	Дівович акцентує увагу на героїчних сторінках минулого Малоросії, котра постає
	як автономна земля.
\end{minipage}

Семен Дівович переважно акцентує увагу на героїчних сторінках минулого
Малоросії, котра постає як автономна земля, що сама собі вибирає покровителя. І
хоча Мала Росія піддалася російському цареві, але це підданство трактується не
як вияв безсилля. Мовляв, Мала Росія вчинила таке, бо російський цар є
християнським монархом. У поемі говориться про самостійність Малої Росії, про
те, що вона може обходитися без інших, зокрема й без Великоросії. Із твору
випливає, що Велика й Мала Росія – дві автономні частини однієї держави, якою
править християнський монарх.

\subsubsection{Ідеї українського автономізму на Стародубщині}

Поема «Розмова Великоросії із Малоросією 1762 року» була одним із перших
творів, де утверджувалася ідея українського автономізму, що набув поширення
серед нащадків козацької старшини вкінці XVIII – на початку XIX століть.
«Біблією» цього автономізму став твір «Історія русів».

Перша згадка про нього припадає на 1828 рік, коли був знайдений рукопис твору в
Стародубському повіті Чернігівської губернії. Спочатку «Історія русів»
поширювалася в рукописних копіях. А в 1846 році її опублікували в Москві
тогочасною російською мовою. У першодруці був зазначений автор твору – покійний
тоді єпископ Георгій Кониський. Зроблено це було для того, щоб надати
авторитетності твору. Проте дуже швидко з’явилися сумніви щодо авторства цієї
особи.

\ifcmt
img_begin 
	url https://gdb.rferl.org/B5A4F8D6-5CA6-46C8-A395-27D1C7666D9B_w650_r0_s.jpg
	caption «Історія Русів» в українському перекладі Івана Драча
img_end
\fi

До сьогоднішнього дня питання авторства «Історії русів» не є вирішеним.
Називалися різні імена її ймовірних творців. На основі аналізу твору можемо
констатувати наступне: його автор жив у другій половині ХVІІІ – на початку ХІХ
століть; мав непогану освіту; служив у російській армії, воював з турками,
добре знав південь України; жив на Стародубщині. Принаймні в творі часто
говориться про події в цьому краї. І не даремно саме тут були знайдені перші
його списки! Принаймні можемо не сумніватися, що остаточна редакція «Історії
русів» з’явилася на Стародубщині.

Цей твір, спираючись на традиції козацького літописання, подає історію України
як окремої землі зі своїми політичними традиціями, відмінними від російських.

Цей твір, спираючись на традиції козацького літописання, подаває історію
України як окремої землі зі своїми політичними традиціями, відмінними від
російських. Автор «Історії русів», попри лояльність до російської імперської
влади, все ж негативно оцінює Московію. Він дає тонку характеристику
менталітету правлячої еліти цієї держави. Вказує на її жадібність до
владолюбства та пиху. Говорить про непостійне правління царське і знищення
самих царів. Ця непостійність, на думку автора, обумовлюється відсутністю
сталої релігії та добрих звичаїв. Все це в кінцевому рахунку веде до того, що
війни з Московією є неминучі й безконечні для всіх народів. Зрештою, в творі
можна зустріти чимало інших антиросійських моментів. Принаймні в ньому
росіяни-московіти нерідко постають ворогами русів-українців.

В російській літературі «Історія русів» трактувалася як «небезпечний» витвір
українського націоналізму.

«Історія русів» справила великий вплив на Тараса Шевченка. Деякі твори Кобзаря
є своєрідною переінтерпретацією її сюжетів. Під впливом «Історії русів»
перебували також Микола Костомаров і Пантелеймон Куліш. Використовували її й
інші українські автори ХІХ століття, зверталися до неї Микола Гоголь і навіть
Олександр Пушкін, окремі російські письменники та історики.

За часів радянської влади про «Історію русів» фактично «забули». Зате в 1956
років вона вийшла в Нью-Йорку в перекладі українською мовою. Також український
переклад Івана Драча цієї книги з’явився в 1991 році. Твір українські
інтелектуали розглядали як важливу пам’ятку, що сприяла розвитку української
національної свідомості. Зате в російській літературі «Історія русів»
трактувалася як «небезпечний» витвір українського націоналізму.

Чи випадково російські більшовики, які продовжили імперську політику царської
Росії, забрали в українців Стародубщину? В історії нічого просто так не буває.
Принаймні цим кроком вони відрізали від України землі, що відіграли важливу
роль у формуванні української національної свідомості.

Окремі пам'ятки архітектури на Стародубщині:

\ifcmt
img_begin 
  url https://gdb.rferl.org/6B1E5976-7062-43D8-B99B-28B376843B4D_w1023_r0_s.jpg
  caption Стародубщина. Воскресенський собор (1771) у місті Почеп, збудований
  гетьманом України Кирилом Розумовським, архітектор Антоніо Рінальді
img_end
\fi

\ifcmt
img_begin 
url https://gdb.rferl.org/DC2A9C2F-0A4E-437C-9317-4996CAFEF7B9_w1023_r0_s.jpg
caption Стародубщина. Церква Трійці Живоначальної в селі Гриневі Погарського
району. Була збудована в 1802 році замість старої дерев’яною на гроші
українського шляхтича, генерал-поручика, графа Іллі Безбородька
img_end
\fi

\ifcmt
img_begin 
url https://gdb.rferl.org/6F6E7B35-CB3D-49E6-B991-2430A5A70C5E_w1023_r0_s.jpg
caption Стародубщина. Церква Трійці Живоначальної в Погарі. Була перебудована в
1783 році. Попередній храм датують 1717 роком, що був збудований полковником
Захаром Іскрою, українським військовим і політичним діячем, який походив із
козацько-старшинського роду
img_end
\fi

\ifcmt
img_begin 
url https://gdb.rferl.org/A2357FEE-2AE3-49D4-B5BE-38FE98273C58_w1023_r0_s.jpg
caption Церква Зачаття Анни в селищі Погарі. Цегляна церква з елементами українського бароко побудована на межі XVIII-XIX століть
img_end
\fi

\ifcmt
img_begin 
url https://gdb.rferl.org/17981EFD-0326-4A4A-B189-D7C52D82A1F2_w1023_r0_s.jpg
caption Стародубщина. Руїни особняка українського аристократа козацького стану, державного діяча Російської імперії Петра Завадовського (1738–1812) у Ляличах Суразького району
img_end
\fi

\ifcmt
img_begin 
url https://gdb.rferl.org/DEF38339-AEA7-4DED-B375-0C0FBB197464_w1023_r0_s.jpg
caption Стародубщина. Руїни особняка українського аристократа козацького стану, державного діяча Російської імперії Петра Завадовського (1738–1812) у Ляличах Суразького району
img_end
\fi

\ifcmt
img_begin 
url https://gdb.rferl.org/96353028-6C97-445A-A834-CCFA58BF8E54_w1023_r0_s.jpg
caption Стародубщина. Церква Святої Катерини в Ляличах Суразького району. Побудована в 1793–1797 роках
img_end
\fi

\ifcmt
img_begin 
url https://gdb.rferl.org/71A67F15-292F-4960-A6B6-494FC53228DA_w1023_r0_s.jpg
caption Стародубщина. Почепський район, село Красний Ріг. Церква Успіння Пресвятої Богородиці 1777 року, побудована на замовлення графа Олексія Розумовського (1748–1822), сина гетьмана України Кирила Розумовського (1728–1803)
img_end
\fi

\ifcmt
img_begin 
url https://gdb.rferl.org/55DE4C3D-3E57-436F-B70C-5B704E67AD5A_w1023_r0_s.jpg
caption Спасо-Преображенська церква у Новозибкові. Старообрядницький храм, збудований за українськими зразками. Храм споруджували протягом трьох років, з 1911-го по 1914 рік
img_end
\fi

\ifcmt
img_begin 
	url https://gdb.rferl.org/5D88BF4C-3175-4098-B6BC-96681441111B_w1023_r0_s.jpg
	caption Стародубщина. Руїни собору Успіння Пресвятої Богородиці в Успенському Кам'янському монастирі. Селище Забрама Климівського району. Цегляний собор був зведений в 1776–1779 в стилі українського бароко замість колишнього дерев'яного. (Фото: Николай Смолянкин, № 3200548001 (Creative Commons))
img_end
\fi

\ifcmt
img_begin 
	url https://gdb.rferl.org/0EDBBFE2-0888-41AF-BAD1-7FBC081DD4AE_w1023_r0_s.jpg
	caption Стародубщина. Руїни собору Успіння Пресвятої Богородиці в Успенському Кам'янському монастирі. Селище Забрама Климівського району. Цегляний собор був зведений в 1776–1779 в стилі українського бароко замість колишнього дерев'яного
img_end
\fi

Петро Кралюк – голова Вченої ради Національного університету «Острозька
академія», професор, заслужений діяч науки і техніки України

Думки, висловлені в рубриці «Точка зору», передають погляди самих авторів і не
конче відображають позицію Радіо Свобода

