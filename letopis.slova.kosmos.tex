% vim: keymap=russian-jcukenwin
%%beginhead 
 
%%file slova.kosmos
%%parent slova
 
%%url 
 
%%author 
%%author_id 
%%author_url 
 
%%tags 
%%title 
 
%%endhead 
\chapter{Космос}
\label{sec:slova.kosmos}

%%%cit
%%%cit_head
%%%cit_pic
%%%cit_text
А у нас раздают \emph{космические} советы \emph{космической} глупости.  Дополню Александр
Рябоконь немного цифрой, чтобы окончательно стало понятно, что философия
господина Макогона мягко говоря - ни о чем.  Как в известном анекдоте: о доме
надо думать.  Как сохранить свою конкурентноспособность и обеспечить
потребителей внутри страны, а не решать вопросы \emph{космического масштаба}, раздавая
советы такого же уровня глупости
%%%cit_comment
%%%cit_title
\citTitle{В первую очередь надо думать о том, как обеспечить население внутри страны / Лента соцсетей / Страна}, 
Валентин Землянский, strana.ua, 03.07.2021
%%%endcit

%%%cit
%%%cit_head
%%%cit_pic
\ifcmt
  pic https://day.kyiv.ua/sites/default/files/21-07-52/21_korolovserhii.jpg
  width 0.4
\fi
%%%cit_text
Звертаючись до біографії Сергія Корольова, неможливо оминути його \emph{космічні}
звитяги, обов’язковою умовою котрих завжди є слово «перший»: перший штучний
супутник Землі, запущений на орбіту, перший знімок зворотної сторони Місяця
(преса всього світу називала його «Знімком століття»), перший \emph{космічний політ}
Білки і Стрілки, перший політ Юрія Гагаріна у \emph{космічний простір}, перший в
історії \emph{космонавтики} вихід у відкритий \emph{космос}, здійснений Олексієм Леоновим.
Відомо, що, окрім Гагаріна, претендентів на першу подорож людини в \emph{космос} було
кілька, серед них — українець Павло Попович і чуваш Андріян Ніколаєв. Однак
всі, включно з самими \emph{космонавтами}, розуміли неможливість їхньої участі в
Першому польоті, адже Першими завжди мають бути росіяни. «Ви, хлопці,
уособлюватимете дружбу народів — українець і чуваш», — пояснював Корольов,
якого вже встигли занести до списку російських учених. Бо Корольов — Головний
конструктор, він Перший, він достойний бути росіянином.
Але помандруймо до витоків. Де ж починався автор перших \emph{космічних} здобутків
«братства народів» та найбільш засекречений радянський учений?
%%%cit_comment
%%%cit_title
\citTitle{Про двох хлопців з Житомира: Короленка і Корольова}, Олександра Кльосова, day.kyiv.ua, 30.07.2021
%%%endcit

%%%cit
%%%cit_head
%%%cit_pic
\ifcmt
  pic https://voskhodinfo.su/uploads/posts/2021-10/1634797254_033.jpg
  @width 0.4
\fi
%%%cit_text
Вы слышали о последнем гражданине СССР, первом дважды Герое: Союза и России
(звезда Героя Советского Союза № 11595 и звезда Героя Российской Федерации №
1), о \emph{космонавте} Крикалёве Сергее Константиновиче, которого… забыли в
\emph{космосе}?  Его отправляли на станцию "МИР" с ещё советского Байконура на
5 месяцев, но спустя 4 месяца Союза не стало, а у молодой России не было
денежных средств для возврата Крикалёва на Землю. Пока в Москве шел переворот,
станция "МИР" продолжала вращаться вокруг Земли, а командировка последнего
советского \emph{космонавта} "затянулась" на 311 дней
%%%cit_comment
%%%cit_title
\citTitle{Настоящий герой!}, , voskhodinfo.su, 21.10.2021
%%%endcit

%%%cit
%%%cit_head
%%%cit_pic
\ifcmt
  pic https://avatars.mds.yandex.net/get-zen_doc/1590748/pub_6163c5c94449432c00826645_6163f6664449432c00becf18/scale_1200
  @width 0.4
\fi
%%%cit_text
Камила Валиева была невероятна в произвольной программе и выдала \emph{космический}
прокат. Даже одно падение с тройного акселя никак не сказалось на общем
впечатлении от проката. Фигуристка легко и играючи исполнила три четверных –
сальхов и два тулупа, один из них во второй половине программы. При этом Камила
не забывала презентовать программу и исполнять сложные транзишены и связки.
Работа рук, корпуса, пластика и растяжка спортсменки по-настоящему
завораживают. Вращения Камилы – это просто отдельный вид искусства
%%%cit_comment
%%%cit_title
\citTitle{Её не победить. Камила Валиева даже с падением забрала мировые рекорды Трусовой и Косторной}, 
Любимое фигурное катание, zen.yandex.ru, 11.10.2021 
%%%endcit

%%%cit
%%%cit_head
%%%cit_pic

\ifcmt
  tab_begin cols=2
     pic https://bigkyiv.com.ua/wp-content/uploads/2021/10/dsc09030-1-1536x1025-1.jpg
     pic https://bigkyiv.com.ua/wp-content/uploads/2021/10/dsc08965-2-scaled.jpg
		 pic https://bigkyiv.com.ua/wp-content/uploads/2021/10/dsc08998-2-scaled.jpg
  tab_end
\fi
%%%cit_text
У Національному  історико-етнографічному заповіднику \enquote{Переяслав}, саме
у селі В’юнище,  що знаходиться поблизу  Переяслава,  є старовинна дерев’яна
церква святої Параскеви П’ятниці, яка була побудована у 1891 році. Споруда
вважається пам’яткою архітектури.  За радянських часів, а саме 27 березня 1979
року, у церкві відкрили Музей світопізнання та мирного освоєння \emph{космосу}.
З 2003 року назву змінили на Музей \emph{космосу}. Ідея створення такого музею
належала історику і культурологу Михайлу Сікорському – сину видатного
українського авіаконструктора Івана Сікорського. Допомагали наповнювати
колекцію музею академік Олександр Ішлінський та доктор технічних наук Сергій
Малашенко.  Саме завдяки їхній наполегливості вдалося зібрати доволі рідкісні
предмети, що пов’язані з освоєнням \emph{космосу} 
%%%cit_comment
%%%cit_title
\citTitle{Свята Параскева і Всесвіт: біля Києва є унікальний музей космосу у старовинній церкві (ФОТО)}, 
Любава Лисичкіна, bigkyiv.com.ua, 26.10.2021
%%%endcit

%%%cit
%%%cit_head
%%%cit_pic
%%%cit_text
1985 год. УССР была одним из немногих мест в мире, где строили авианесущие
крейсера (примерялись к авианосцам) , самолёты (до трёх сотен в год) и
\emph{космические} ракето-носители. Что из этого осталось сегодня? Ноль. К моменту
«независимости» газификация домовладений и городов «колонии» Украины колеблется
возле отметки 90\%. В «метрополии» России — не выше 40\%.  Это потому, что там
проходили экспортные газопроводы? Ну да... Хотя некоторые союзные партийные
функционеры возражали, но решение о таком маршруте было принято именно в пользу
УССР, а не в пользу РСФСР. Почему так вышло? А потому, что наверху у
колонизаторов-русских толком не было патриотов. Сплошь интернационалисты с
широко распахнутой душой
%%%cit_comment
%%%cit_title
\citTitle{Как Украина была колонией СССР... Не наоборот ли?}, 
Исторические напёрстки, zen.yandex.ru, 02.11.2021
%%%endcit

%%%cit
%%%cit_head
%%%cit_pic
\ifcmt
  tab_begin cols=3
     pic https://yaro.dp.ua/images/berdnyk.jpg
     pic https://avatars.mds.yandex.net/i?id=489ba7d5378c1e3f26d0ef7c681b659c-4552607-images-thumbs&n=13
		 pic https://proprikol.ru/wp-content/uploads/2020/06/krasivye-kartinki-zvezd-53.jpg
  tab_end
\fi
%%%cit_text
О небесний океане, чому одвіку мовчиш? Яка невідома \emph{космічна} буря пробудила
тебе? Чому ти почав гнівно котити буруни буття в метагалактичних просторах,
кидати піну життя у примхливий, калейдоскопічний політ? Чи не краще було б тобі
спати у вселенському спокої, колихаючи в неосяжному лоні міріади зерен
ненародженого життя? А так — полонив сам себе пасткою часу, древньою павутиною
Хроноса, і борсаєшся в ній, паче барвистий метелик-ефемера. А все-таки ні, не
метелик! Відчуваю в собі страхітливу силу, яка дорівнює потугою силам зірок і
галактик. Знаю: мушу розгадати дивовижну таємницю. Приховане знання тривожить,
бентежить, розриває. Я ніби льох, начинений динамітом. Жадаю іскри підпалу і
страхаюсь її. Де ж, коли з’явиться благословенний обрій розкриття?
%%%cit_comment
%%%cit_title
\citTitle{Зоряний Корсар}, Олесь Бердник
%%%endcit
