% vim: keymap=russian-jcukenwin
%%beginhead 
 
%%file 03_12_2020.news.dw.satanovskii_sergei.1.sputnik_v_dw_journalist
%%parent 03_12_2020
 
%%url https://www.dw.com/ru/vakcina-sputnik-v-kak-mne-delali-privivku-ot-koronavirusa/a-55809583
 
%%author Сатановский, Сергей
%%author_id satanovskii_sergei
%%author_url 
 
%%tags covid,vaccine,sputnik_v
%%title Вакцина "Спутник V": как корреспонденту DW делали прививку от коронавируса
 
%%endhead 
 
\subsection{Вакцина "Спутник V": как корреспонденту DW делали прививку от коронавируса}
\label{sec:03_12_2020.news.dw.satanovskii_sergei.1.sputnik_v_dw_journalist}
\Purl{https://www.dw.com/ru/vakcina-sputnik-v-kak-mne-delali-privivku-ot-koronavirusa/a-55809583}
\ifcmt
	author_begin
   author_id satanovskii_sergei
	author_end
\fi

Незадолго до начала в России массовой вакцинации корреспондент DW принял
участие в тестировании российского препарата "Спутник V". Рассказ от первого
лица.

\index[rus]{Коронавирус!Вакцина!Спутник V}

\ifcmt
	pic https://static.dw.com/image/55808926_303.jpg
	caption Корреспондент DW Сергей Сатановский после вакцинации
	width 0.5
	fig_env wrapfigure
\fi

Масштабная вакцинация от коронавируса должна начаться в России еще до середины
декабря. Такое поручение президент Владимир Путин дал в среду, 2 декабря,
правительству страны. Первыми прививки будут делать врачам и учителям. Но
параллельно продолжаются и тесты зарегистрированной еще летом вакцины "Спутник
V". В испытаниях этого изобретения центра эпидемиологии им. Н.Ф. Гамалеи по
собственной инициативе принял участие и корреспондент DW.  Как записаться на
исследование

Еще в сентябре интернет стал пестрить контекстной рекламой "Станьте
добровольцем в исследовании вакцины от COVID-19". Информации о вакцине "Спутник
V" на тот момент было немного, да и количество вновь выявленных случаев не
казалось тревожным.

\ifcmt
	pic https://static.dw.com/image/54580672_403.jpg
	caption Вакцина "Спутник V"
	width 0.5
	fig_env wrapfigure
\fi

К концу ноября ситуация изменилась, число инфицированных\Furl{https://www.dw.com/ru/vrachi-o-vtoroj-volne-koronavirusa-v-rossii/a-55777448} пошло резко вверх, и я
решил рискнуть. Финальным аргументом в пользу вакцинации стал опыт друзей,
которые тяжело болели и были заблокированы в своих квартирах под надзором
приложения "Социальный мониторинг".

Поиск по запросу "сделать прививку от коронавируса" привел меня на сайт
московской мэрии mos.ru. Там нужно было заполнить анкету. Ее составителей
интересовало, болел ли я новым коронавирусом, общался ли с зараженными в
последние две недели и имею ли хронические заболевания. Медицинские чиновники
утверждали, что переболевшим COVID-19 не следует прививаться в этом сезоне.

Примерно через неделю представительница департамента здравоохранения московской
мэрии по телефону предложила мне выбрать поликлинику, где я пройду
предварительное обследование.

\subsubsection{Обследование в больнице доктора Мясникова}

На обследование я отправился в клиническую больницу им. Жадкевича в Можайском
районе. Уже после укола я узнал, что главврач этой клиники - доктор Мясников,
телеведущий и друг журналиста Владимира Соловьева. В начале пандемии он
оценивал шанс россиян заразиться COVID-19 как "ноль целых и ноль в периоде".
Стало тревожно, но отступать было поздно.

Перед обследованием терапевт спросила, какие лекарства я принимал в последнее
время, есть ли аллергии, какие хирургические вмешательства переносил. Пришлось
звонить родителям, чтобы точнее сформулировать название операции, которую мне
делали в пять лет. Во время беседы врач передала мне информационной листок (на
16 страниц) участника исследования и устно повторила то, что в нем изложено.

\ifcmt
	pic https://static.dw.com/image/55809129_403.jpg
	caption Клиническая больница им. Жадкевича
	width 0.5
	fig_env wrapfigure
\fi

Там, например, сообщается, что цель исследования - оценить эффективность,
иммуногенность и безопасность препарата Гам-КОВИД-Вак (торговая марка "Спутник
V"). В доклинических исследованиях лекарство изучено на животных - например, на
сирийских хомячках и морских свинках. На них "Спутник V" показал свою
безопасность и эффективность, а на добровольцах из числа людей - только
безопасность, говорится в документе.

\subsubsection{Страхование жизни на 2 млн рублей}

Всего участниками исследования станут 40 тысяч человек: 30 тысяч получат
вакцину, а 10 тысяч - плацебо. Через 21 день после первой прививки нужно снова
явиться в больницу - введут второй компонент вакцины. Кроме того, полагается
вести дневник самонаблюдения в приложении Check Covid-19 - больше от меня
ничего не требовалось. Но есть еще и дополнительное исследование, поучаствовать
в котором мне предложила терапевт.

Оказалось, что основное исследование не предполагает анализа крови на
иммуногенность - то есть способность вакцины формировать иммунный ответ в виде
антител, способных нейтрализовать вирус. Такие анализы проводят в
дополнительном исследовании, за участие в котором государство платит 8,5 тысяч
рублей. От меня потребуется только сдать не больше 95 мл крови, поэтому я
согласился.

\ifcmt
  pic https://static.dw.com/image/55809159_403.jpg
  caption В коридоре больницы им. Жадкевича
  width 0.5
  fig_env wrapfigure
\fi


"Хотя участие в исследовании и безопасно, на протяжении всего исследование для
вас будет действовать специальная страховка", - заявила терапевт. В полисе ОМС
участника исследования вакцины  стоят следующие расценки. В случае моей смерти
выгодоприобретатели (в моем случае это родители) получат 2 млн рублей.
Инвалидность в зависимости от группы оценена в 500 тысяч - 1,5 млн рублей.
Ухудшение здоровья без инвалидности обойдется страховщику не более чем в 300
тысяч рублей.

После беседы я подписал согласие на участие в исследованиях и отправился на
тестирование. Меня проверили на ВИЧ, сифилис, гепатиты С и В, коронавирус (ПЦР
и антитела) и наличие в моче наркотических веществ. Ничего не нашли, к
вакцинации допустили.

\subsubsection{Вакцина хранится при -28$^\circ$С}

Проведенные анализы действительны в течение одной недели, поэтому ровно через 7
дней я пришел на вакцинацию. На дверях поликлиники - наклейка "Заходите за
вакциной против COVID-19". Внутри больницы к блоку, где проходит вакцинация,
ведут указатели. В блоке - стойка регистрации и четыре кабинета: вакцинация,
терапевт, анализы, комната отдыха.

В десять утра участников исследования в этом блоке было двое: я и женщина лет
тридцати. Я отметился у терапевта, сдал кровь (для исследования на антитела мне
нужно сдавать ее несколько раз) и зашел в кабинет вакцинации. Маленькая комната
с парой стульев, медицинским столиком и двумя холодильниками российской марки
Pozis. Стоимость таких холодильников может достигать примерно 150 тысяч рублей
за штуку.

\ifcmt
  pic https://static.dw.com/image/55809056_403.jpg
  caption После прививки полагается полчаса провести в комнате отдыха 
  width 0.5
  fig_env wrapfigure
\fi

Врач сделал безболезненный укол в левое плечо, после чего я провел полчаса в
комнате отдыха. Врач сказал, что у некоторых пациентов после укола падает
давление, но у меня оно было в норме. Мне выдали сертификат участника
исследования и отпустили домой. Напоследок сказали, что может подскочить
температура. Если это произойдет, то можно сбить ее парацетамолом. Еще
попросили в течение ближайших трех месяцев не планировать ребенка - влияние
вакцины на сперму пока не изучено.

\subsubsection{Озноб и высокая температура}

Уже после вакцинации я стал читать, какая реакция была у других участников
исследования - журналистов и членов группы "Результаты вакцинации" в Telegram
(их там больше 1800). У людей к вечеру после введения вакцины (или плацебо) и
на следующий день поднималась температура, болели мышцы. Уходя с работы около
семи вечера, я уже чувствовал головную боль, легкий озноб и головокружение.
Впрочем, не исключал, что это было самовнушением после десятков прочитанных
рассказов.

Ближе к ночи сомнений в том, что организм реагирует на прививку не осталось -
температура подскочила до 38,6. Боль в мышцах от недавних занятий йогой
обострилась. Все это было неприятно, но радовало, что, вероятнее всего, мне
попалась вакцина, а не плацебо. Я выпил парацетамол, записал симптомы в
электронный дневник и лег спать. На следующий день температура держалась на
отметке 37, а через день побочные эффекты ушли.

Через три недели мне введут второй компонент вакцины, а примерно на 42-й день
после первого укола у меня должны появиться антитела. Но вопрос о том, будут ли
эти антитела эффективны в нейтрализации вируса, пока остается открытым.
