% vim: keymap=russian-jcukenwin
%%beginhead 
 
%%file 28_11_2020.fb.bieloglazova_antonina.1.napadenija
%%parent 28_11_2020
 
%%url https://www.facebook.com/antonina.bieloglazova/posts/4277982855563280
 
%%author Белоглазова, Антонина
%%author_id bieloglazova_antonina
%%author_url 
 
%%tags 
%%title Быть украинским журналистом - это значит знать, что если на тебя однажды нападут, твоё государство тебя не защитит
 
%%endhead 
 
\subsection{Быть украинским журналистом - это значит знать, что если на тебя однажды нападут, твоё государство тебя не защитит}
\label{sec:28_11_2020.fb.bieloglazova_antonina.1.napadenija}
\Purl{https://www.facebook.com/antonina.bieloglazova/posts/4277982855563280}
\ifcmt
	author_begin
   author_id bieloglazova_antonina
	author_end
\fi

Быть украинским журналистом - это значит знать, что если на тебя однажды
нападут, твоё государство тебя не защитит. И осознавая это, ты продолжаешь
работать журналистом. 

Никого не привлекут, а если привлекут, то отпустят очень скоро, дело закроют,
виновные не понесут должное наказание. 

Быть украинским журналистом - это когда государство не на твоей стороне.

Реакция стоящих рядом людей, которые спокойно наблюдали за нападением на
журналистку телеканала  NewsOne Виолетту Товкес, многое говорит о нашем
обществе.

Ранее уже были нападения на журналистов телеканала NewsOne: Bogdan Aminov,
Bilera Darina, Anastasiia Pshenychna и теперь на Виолетта Товкес.

Обратите внимание, что почти все эти нападения происходят в открытую, в прямом
эфире. Нападающие не делают это «до» прямого эфира или «после». Они это делают
как раз для того, чтоб именно попасть в объектив камеры.

В случае с Анастасией Пшеничной, нападающие даже сами записали момент нападения
на свою камеру и потом выложили это в открытый доступ.

Таким образом нам пытаются сказать, что они не боятся получить срок и отсидеть,
потому что их никто не накажет. Они чувствуют себя спокойно. А все эти
нападения - это очередная попытка устрашения журналистов.

