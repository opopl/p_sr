% vim: keymap=russian-jcukenwin
%%beginhead 
 
%%file slova.rusalka
%%parent slova
 
%%url 
 
%%author 
%%author_id 
%%author_url 
 
%%tags 
%%title 
 
%%endhead 
\chapter{Русалка}

%%%cit
%%%cit_pic
\ifcmt
  pic https://oreltimes.ru/wp-content/uploads/2019/12/vij5ws-ei2e.jpg
	caption Сказочная карта России
\fi
%%%cit_text
\begin{itemize}
  \item \emph{Русалка} - Псков.
  \item Садко - Гостцы, Новгородская область.
  \item Салика - Йошкар-Ола, Марий Эл.
  \item Снегурочка - Кострома.
  \item Снеговик - Архангельск.
  \item Снежинка - Сургут, Ханты-Мансийский автономный округ (Югра).
  \item Соловей-разбойник - Береза, Курская область.
  \item Тол Бабай - Шаркан, Удмуртия.
  \item Топтыгин - Пошехонье, Ярославская область.
  \item Улып-великан - Чебоксары, Чувашия.
  \item Урал-Мороз - Арамиль, Свердловская область.
  \item Хозяйка Севера Лоухи - Лоухи, Карелия
\end{itemize}
%%%cit_title
\citTitle{Вы знали, что существует сказочная карта России?}, 
Душевный Шагомер, zen.yandex.ru, 17.03.2021
%%%endcit

