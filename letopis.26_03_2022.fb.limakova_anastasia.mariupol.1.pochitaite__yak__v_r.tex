%%beginhead 
 
%%file 26_03_2022.fb.limakova_anastasia.mariupol.1.pochitaite__yak__v_r
%%parent 26_03_2022
 
%%url https://www.facebook.com/anastasiia.limakova/posts/pfbid021zKda5X8why9BqUhrDkxrC1322WWroy8iLpKL8hnBU69nfSayNkY4jVMTdobuaoRl
 
%%author_id limakova_anastasia.mariupol
%%date 26_03_2022
 
%%tags mariupol,mariupol.war,poezia
%%title Почитайте, які вірши писали маріупольці під обстрілами у подвалах
 
%%endhead 

\subsection{Почитайте, які вірши писали маріупольці під обстрілами у подвалах}
\label{sec:26_03_2022.fb.limakova_anastasia.mariupol.1.pochitaite__yak__v_r}

\Purl{https://www.facebook.com/anastasiia.limakova/posts/pfbid021zKda5X8why9BqUhrDkxrC1322WWroy8iLpKL8hnBU69nfSayNkY4jVMTdobuaoRl}
\ifcmt
 author_begin
   author_id limakova_anastasia.mariupol
 author_end
\fi

Почитайте, які вірши писали маріупольці під обстрілами у подвалах. З якою вірою та вдячністю до захисників. До сліз: 

\raggedcolumns
\begin{multicols}{2} % {
\setlength{\parindent}{0pt}

Сидели не все по подвалам

Не мог кто спуститься туда – 

И нет там живих под навалом

Бетона, железа, стекла.

Дома разбивали ракетьІ

И миньІ и бомбьІ и Град

И вечное зарево где-то – 

Восход? Или может закат?

Все вместе смешалось: боль, слезьІ

И ненависть крепла к врагу

Когда бездьІханное тело

ТьІ видишь на каждом шагу

ТяжельІе длинньІе ночи Сменяли 

тяжелие дни

Молилась я Господу: Отче!

ВоенньІм и нам помоги!

И стойкостью наших любимьІх –

Защитников нашей земли

Их мужеством, смелостью, силой 

Дивилась я в страшньІе дни.

И стал Мариуполь для русских-

Для орков – рашистов из тьмьІ

Той костью, что в горле застряла

И яростью их сатаньІ

Мой город красивий и крепкий

Он братской могилою стал

Собою закрьІл он народ УкраиньІ

И Духом СвободьІ не пал.

И в это столетье, поверьте,

С тоской оглянувшись назад –

Единственний в мире , на свете

Теперь есть родной «Азов» Град
\end{multicols} % }
