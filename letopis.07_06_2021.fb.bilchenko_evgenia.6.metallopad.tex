% vim: keymap=russian-jcukenwin
%%beginhead 
 
%%file 07_06_2021.fb.bilchenko_evgenia.6.metallopad
%%parent 07_06_2021
 
%%url https://www.facebook.com/yevzhik/posts/3960693147299099
 
%%author Бильченко, Евгения
%%author_id bilchenko_evgenia
%%author_url 
 
%%tags 
%%title БЖ. Металлопад
 
%%endhead 
 
\subsection{БЖ. Металлопад}
\label{sec:07_06_2021.fb.bilchenko_evgenia.6.metallopad}
\Purl{https://www.facebook.com/yevzhik/posts/3960693147299099}
\ifcmt
 author_begin
   author_id bilchenko_evgenia
 author_end
\fi

БЖ. Металлопад

Почему эта жизнь, свиваясь вокруг кольцом,
Играет со мною в доброго-злого, но полицейского?
Как будто бандит на улице с благородным лицом
Предлагает: "Жизнь или кошелек?" - Академии и Лицея
Не поминая, но выбор мой - меж Платоном и Аристотелем,
И то не всегда, потому что "всегда" - это текст Толстого.
Меч из света - предельно твёрд - и прорубит сквозь сто теней
Солнечное окно без печали да без истомы.
Но тени - из воздуха. И они надо мной сгущаются.
Что им - алмазный удар? Ведь они - молекулы. Дух - предмет.
Вещность Бога открыта тем, кто хоть раз пережил несчастье
И счастье быть понятым и прочтённым без ответа или в ответ.
Я стала собственной книгой: apriori не издаваемой.
За есенинским горизонтом спит рязанская Галилея.
А над степью Донбасса и Крыма звёзды любви дарами на
Небе развешены, как картины: бесплатная галерея.
И я, запрокинув щегла Мандельштама, гляжу, раздирая, темень:
Вот золотая  падает - металлическою кукушкой
На бошку, оттикивая моё, уходящее в вечность время.
И то не всегда, потому что "всегда" - не Толстой, а за гранью - Пушкин.
7 июня 2021 г.

\ifcmt
  pic https://scontent-cdg2-1.xx.fbcdn.net/v/t1.6435-9/196780008_3960693110632436_5291875713057515455_n.jpg?_nc_cat=108&ccb=1-3&_nc_sid=8bfeb9&_nc_ohc=EBY3kh1cvOMAX-NmOyt&_nc_ht=scontent-cdg2-1.xx&oh=76d8b0b822f75e4450f5921a8339d2df&oe=60E29921
\fi

