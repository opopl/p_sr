% vim: keymap=russian-jcukenwin
%%beginhead 
 
%%file 08_09_2021.fb.bryhar_sergej.1.mova_jazyk_surzhik.cmt
%%parent 08_09_2021.fb.bryhar_sergej.1.mova_jazyk_surzhik
 
%%url 
 
%%author_id 
%%date 
 
%%tags 
%%title 
 
%%endhead 
\subsubsection{Коментарі}

\begin{itemize}

% -------------------------------------
\ii{fbauth.bryn_vladimir.lvov.ukraina}
% -------------------------------------
 
Повага за цю боротьбу. Де "моя хата скраю - першим в ворога стріляю."

\ifcmt
  ig https://scontent-frx5-2.xx.fbcdn.net/v/t39.1997-6/s480x480/12385791_539003332919315_2129379260_n.png?_nc_cat=1&ccb=1-5&_nc_sid=0572db&_nc_ohc=QS3HA4n3sTEAX-oO9lU&_nc_ht=scontent-frx5-2.xx&oh=545f7c328cece8f50c0078ba30a6a435&oe=613FFFD4
  @width 0.2
\fi

%%%fbauth
%%%fbauth_name
\iusr{Лиза Ребар}
%%%fbauth_url
%%%fbauth_place
%%%fbauth_id
%%%fbauth_front
%%%fbauth_desc
%%%fbauth_www
%%%fbauth_pic
%%%fbauth_pic portrait
%%%fbauth_pic background
%%%fbauth_pic other
%%%fbauth_tags
%%%fbauth_pubs
%%%endfbauth
 

Звісно, можна повчитися на українській філології та отримати другу вищу освіту.
Навіть для себе, але нема часу, бо вдома треба здобувати другу середню освіту.
Вас теж це чекає. З міжнародним днем грамотності!

\begin{itemize}
%%%fbauth
%%%fbauth_name
\iusr{Serhii Bryhar}
%%%fbauth_url
%%%fbauth_place
%%%fbauth_id
%%%fbauth_front
%%%fbauth_desc
%%%fbauth_www
%%%fbauth_pic
%%%fbauth_pic portrait
%%%fbauth_pic background
%%%fbauth_pic other
%%%fbauth_tags
%%%fbauth_pubs
%%%endfbauth
 
\textbf{Ліза Ребар} Дякую! Навзаєм \Smiley[1.0][yellow]! Гарна дата для гарного допису))).
\end{itemize}

%%%fbauth
%%%fbauth_name
\iusr{Prymara Makeeva}
%%%fbauth_url
%%%fbauth_place
%%%fbauth_id
%%%fbauth_front
%%%fbauth_desc
%%%fbauth_www
%%%fbauth_pic
%%%fbauth_pic portrait
%%%fbauth_pic background
%%%fbauth_pic other
%%%fbauth_tags
%%%fbauth_pubs
%%%endfbauth
 
узкоязичні мовознавці... на яких філфаках вчили мову? проффєсура, курвочкаряба...

%%%fbauth
%%%fbauth_name
\iusr{Сергій Лащенко}
%%%fbauth_url
%%%fbauth_place
%%%fbauth_id
%%%fbauth_front
%%%fbauth_desc
%%%fbauth_www
%%%fbauth_pic
%%%fbauth_pic portrait
%%%fbauth_pic background
%%%fbauth_pic other
%%%fbauth_tags
%%%fbauth_pubs
%%%endfbauth
 
Остання фраза варта популяризації: будь яка українська краща, ніж мова окупанта. Невже будуть заперечення?

%%%fbauth
%%%fbauth_name
\iusr{Наталія Квасницька}
%%%fbauth_url
%%%fbauth_place
%%%fbauth_id
%%%fbauth_front
%%%fbauth_desc
%%%fbauth_www
%%%fbauth_pic
%%%fbauth_pic portrait
%%%fbauth_pic background
%%%fbauth_pic other
%%%fbauth_tags
%%%fbauth_pubs
%%%endfbauth
 
@igg{fbicon.thumb.up.yellow}{repeat=3}

%%%fbauth
%%%fbauth_name
\iusr{Oleksandr Chumak}
%%%fbauth_url
%%%fbauth_place
%%%fbauth_id
%%%fbauth_front
%%%fbauth_desc
%%%fbauth_www
%%%fbauth_pic
%%%fbauth_pic portrait
%%%fbauth_pic background
%%%fbauth_pic other
%%%fbauth_tags
%%%fbauth_pubs
%%%endfbauth
 

Я упоротим завжди кажу, що російська це висока і вишукана мова, тому російською
треба або говорити тільки літературною, або не говорити нею взагалі.

%%%fbauth
%%%fbauth_name
\iusr{Ntina Ntoubrova}
%%%fbauth_url
%%%fbauth_place
%%%fbauth_id
%%%fbauth_front
%%%fbauth_desc
%%%fbauth_www
%%%fbauth_pic
%%%fbauth_pic portrait
%%%fbauth_pic background
%%%fbauth_pic other
%%%fbauth_tags
%%%fbauth_pubs
%%%endfbauth
 

Справа в тому, що певна кількість наших співвітчизників дуже зрусифікована. І в
них дійсно побутує якась дивна ілюзія, що україномовні люди "придурюються",
"вони неправі". Такі є люди, зрусифіковані в 2-3му поколінні, зокрема ті, які в
маленькому віці були перевезені до міста або народилися у подружжі таких
недавно переїхавших, що вже втратили контакти із сільською ріднею.

І ще цікавий момент з моїх спостережень... Якщо україномовні співвітчизники
завжди сприймали та до війни 2014го року поблажливо з російськомовними
спілкувалися та відчували їх рідними "бо в нас один паспорт! бо в нас єдина
батьківщина! бо ми з тобою змалечку знайомі та дружимо!". То зрусифіковані
відчувають себе в україномовному середовищі як справжні імігранти, що потрапили
до іншої країни і їм досить незручно і неприємно в цьому знаходитися. А нюанс
міститься у тому, що тема іміграції в нашій совєцькій літературі, що
російської, що української, ніколи в пострадянських школах не розбиралася з
позитивної та проактивної точки зору, тільки через осуд або співчуття. По ТБ та
в ЗМІ про неї не говорять інакше як через симпатію та підтримку людини, що не
може знайти своє місце в новому для себе оточенні. Дуже нечасно наші
співвітчизники передивляючись фільми на тему страждань імігрантів із невмінням
влитися в нове суспільство - відчуватимуть бажання обуритися та заперечити: "ну
навіщо ти приїхав/ла тоді? або вертай назад - або змінюйся, бо ти тепер вже
маєш тут жити! чому хтось має під тебе підлаштовуватися??"..

Тому зросійщені вже ментально і через встановлений таким культурний світогляд
"заточені" на те, щоб відчайдушно відстоювати свою ідентичність від зазіхань. І
саме через такі "внутрішньо-імігрантські" відчуття в 2014му році кримчани у
першу чергу відчули любов до окупанта.

Якби була в нас представлена масово література про те, як імігрант стає "своїм"
в новому середовищі (а вона є!!) - може, зрусифікованим було б легше влитися до
загальноукраїнського моря. Хоча б на рівні відчуття себе "одним із нас",
замість протистояння "я і вони".

\begin{itemize}
%%%fbauth
%%%fbauth_name
\iusr{Roman Romanow}
%%%fbauth_url
%%%fbauth_place
%%%fbauth_id
%%%fbauth_front
%%%fbauth_desc
%%%fbauth_www
%%%fbauth_pic
%%%fbauth_pic portrait
%%%fbauth_pic background
%%%fbauth_pic other
%%%fbauth_tags
%%%fbauth_pubs
%%%endfbauth
 
\textbf{Ntina Ntoubrova} ну це один із пазлів. але важливий да!
\end{itemize}

%%%fbauth
%%%fbauth_name
\iusr{Світлана Троцюк}
%%%fbauth_url
%%%fbauth_place
%%%fbauth_id
%%%fbauth_front
%%%fbauth_desc
%%%fbauth_www
%%%fbauth_pic
%%%fbauth_pic portrait
%%%fbauth_pic background
%%%fbauth_pic other
%%%fbauth_tags
%%%fbauth_pubs
%%%endfbauth

\ifcmt
  ig https://scontent-frt3-1.xx.fbcdn.net/v/t39.1997-6/s180x540/104088995_259375198670368_6998486030201420399_n.png?_nc_cat=106&ccb=1-5&_nc_sid=ac3552&_nc_ohc=ore_2vKtSgcAX-Fl9J7&tn=lCYVFeHcTIAFcAzi&_nc_ht=scontent-frt3-1.xx&oh=b97fd08e6abd82b0cb92919ab474bee0&oe=613EA8FA
  @width 0.3
\fi

% -------------------------------------
\ii{fbauth.bilenjkij_nazarij.kiev.ukraina}
% -------------------------------------

Цікава позиція) Але, знаючи скільки ще цієї бидло-маси і скільки нових стають
дорослими, то я веду більш жорстку політику з ними, відразу на..й, якщо клепки
трохи є саме зрозуміє, якщо немає, то не варто і вступати в дискусію! @igg{fbicon.wink} 

\begin{itemize}
%%%fbauth
%%%fbauth_name
\iusr{Yaroslav Datsko}
%%%fbauth_url
%%%fbauth_place
%%%fbauth_id
%%%fbauth_front
%%%fbauth_desc
%%%fbauth_www
%%%fbauth_pic
%%%fbauth_pic portrait
%%%fbauth_pic background
%%%fbauth_pic other
%%%fbauth_tags
%%%fbauth_pubs
%%%endfbauth
 
\textbf{Назарій Біленький} Справа в тому що посилаючи на х.. чи роблячи сварку ви тратите енергію і псуєте собі настрій. Отак з 10-к за день, день за днем, і вони вас "зроблять". А якщо вдається вивести з рівноваги опонента і ні на грам не зіпсувати собі настрій, а навпаки ... то це ж навіть краще. У мене так не виходить, або виходить зрідка. Але якщо ви досягли "ступеню дзен" то це тільки супер.

%%%fbauth
%%%fbauth_name
\iusr{Назарій Біленький}
%%%fbauth_url
%%%fbauth_place
%%%fbauth_id
%%%fbauth_front
%%%fbauth_desc
%%%fbauth_www
%%%fbauth_pic
%%%fbauth_pic portrait
%%%fbauth_pic background
%%%fbauth_pic other
%%%fbauth_tags
%%%fbauth_pubs
%%%endfbauth
 
\textbf{Yaroslav Datsko} помиляєтесь, мені зручна ця система і я нею користуюсь вже більше 10 років. Я втрачаю час і енергію коли намагаюся щось довести, а так просто рубаю раз і досить
\end{itemize}

%%%fbauth
%%%fbauth_name
\iusr{Ірина Шостинська}
%%%fbauth_url
%%%fbauth_place
%%%fbauth_id
%%%fbauth_front
%%%fbauth_desc
%%%fbauth_www
%%%fbauth_pic
%%%fbauth_pic portrait
%%%fbauth_pic background
%%%fbauth_pic other
%%%fbauth_tags
%%%fbauth_pubs
%%%endfbauth
 
Я б не вживала кацапізма "зливати".

% -------------------------------------
\ii{fbauth.kochan_orest.ukraina}
% -------------------------------------

Остання фраза варта того, щоб її надрукувати на плакатах і постерах та
розвішували на видних місцях по всій країні. Для себе я розробив таку шкалу
цінності стилю мовлення (у порядку спадання): діалект, бо людина попри всі
"наїзди" на українські діалекти не встидається своєї мови і говорить так, як
звикла вдома; літературна українська мова (людина витратила зусилля на її
вивчення, але на жаль, це не свідчить, що людина нею користується достатньо
часто в усіх випадках); мішанка з мов з переважанням української; мішанка мов з
переважанням інших мов; погана російська; літературна російська (тут йде мова
про ідейного ворога - людина замість вчити українську вчила російську).

% -------------------------------------
\ii{fbauth.valevskaja_viktoria.rakovnik.chehia.herson}
% -------------------------------------
 
З малюком будуть на жаль проблеми. Ця агресивна москворота дійсність навколо
мало кого не перемелює. Але наснаги вам і стійкості.

\begin{itemize}
%%%fbauth
%%%fbauth_name
\iusr{Serhii Bryhar}
%%%fbauth_url
%%%fbauth_place
%%%fbauth_id
%%%fbauth_front
%%%fbauth_desc
%%%fbauth_www
%%%fbauth_pic
%%%fbauth_pic portrait
%%%fbauth_pic background
%%%fbauth_pic other
%%%fbauth_tags
%%%fbauth_pubs
%%%endfbauth
 
\textbf{Вікторія Валевська} Ми ж не прив'язані саме до цього середовища)...

%%%fbauth
%%%fbauth_name
\iusr{Oleg Mykhailov}
%%%fbauth_url
%%%fbauth_place
%%%fbauth_id
%%%fbauth_front
%%%fbauth_desc
%%%fbauth_www
%%%fbauth_pic
%%%fbauth_pic portrait
%%%fbauth_pic background
%%%fbauth_pic other
%%%fbauth_tags
%%%fbauth_pubs
%%%endfbauth
 
\textbf{Вікторія Валевська} від впливу не сховаєш. але малюк виросте і зрозуміє, яка його рідна мова. я виріс у Дніпрі, де навколо був російський суржик. правильну російську я чув тільки від моєї вчительки російської мови. Але закладену в мене в дитинстві українську я зберіг. і зараз в побуті і вдома, на роботі спілкуюсь українською.

%%%fbauth
%%%fbauth_name
\iusr{Вікторія Лопата}
%%%fbauth_url
%%%fbauth_place
%%%fbauth_id
%%%fbauth_front
%%%fbauth_desc
%%%fbauth_www
%%%fbauth_pic
%%%fbauth_pic portrait
%%%fbauth_pic background
%%%fbauth_pic other
%%%fbauth_tags
%%%fbauth_pubs
%%%endfbauth
 
\textbf{Oleg Mykhailov} теж відновлюю наявний досвід і бачу на скільки цей досвід слабший у моєї дитини, бо в неї вже інше покоління дідів і бабів, що... вижили в містах🤔і це лишило на них значні відбитки🤦
\end{itemize}

%%%fbauth
%%%fbauth_name
\iusr{Людмила Глоба}
%%%fbauth_url
%%%fbauth_place
%%%fbauth_id
%%%fbauth_front
%%%fbauth_desc
%%%fbauth_www
%%%fbauth_pic
%%%fbauth_pic portrait
%%%fbauth_pic background
%%%fbauth_pic other
%%%fbauth_tags
%%%fbauth_pubs
%%%endfbauth
 
Я прочитала в другому реченні - "зберися і вбий". І подумала, правильно зробите!

%%%fbauth
%%%fbauth_name
\iusr{Олена Тарасова}
%%%fbauth_url
%%%fbauth_place
%%%fbauth_id
%%%fbauth_front
%%%fbauth_desc
%%%fbauth_www
%%%fbauth_pic
%%%fbauth_pic portrait
%%%fbauth_pic background
%%%fbauth_pic other
%%%fbauth_tags
%%%fbauth_pubs
%%%endfbauth
 
Совок, він, наче та зарраззза, ніяк не відпускає

%%%fbauth
%%%fbauth_name
\iusr{Leonid Bedratyuk}
%%%fbauth_url
%%%fbauth_place
%%%fbauth_id
%%%fbauth_front
%%%fbauth_desc
%%%fbauth_www
%%%fbauth_pic
%%%fbauth_pic portrait
%%%fbauth_pic background
%%%fbauth_pic other
%%%fbauth_tags
%%%fbauth_pubs
%%%endfbauth
 

Ще один висновок - вони ставлять подібні питання зовсім не для того щоб
отримати відповіді, відповіді їх не цікавлять зовсім. Вони так тестують нашу
здатність до опору і чи ми розуміємо насправді що відбувається. Тому справді,
на такі питання не треба відповідати а перехоплювати ініціативу

\begin{itemize}
%%%fbauth
%%%fbauth_name
\iusr{Yaroslav Datsko}
%%%fbauth_url
%%%fbauth_place
%%%fbauth_id
%%%fbauth_front
%%%fbauth_desc
%%%fbauth_www
%%%fbauth_pic
%%%fbauth_pic portrait
%%%fbauth_pic background
%%%fbauth_pic other
%%%fbauth_tags
%%%fbauth_pubs
%%%endfbauth
 
\textbf{Leonid Bedratyuk} Я би сказав не відповідати на їх питання взагалі, а починати про своє, або питання на питання, повертати їх питання дзеркально відобразивши на рос. мову.
\end{itemize}

%%%fbauth
%%%fbauth_name
\iusr{Svitlychnyi Olexandr}
%%%fbauth_url
%%%fbauth_place
%%%fbauth_id
%%%fbauth_front
%%%fbauth_desc
%%%fbauth_www
%%%fbauth_pic
%%%fbauth_pic portrait
%%%fbauth_pic background
%%%fbauth_pic other
%%%fbauth_tags
%%%fbauth_pubs
%%%endfbauth
 

я в таких випадках, коли мені починають розповідати про "нечистьій укрАинский",
починаю їм влаштовувати маленький тест на "чистоту російської". Власне, саме з
укрАинского можна далі і не тестувати) Це рівень знань російської за 2 клас)
Кілька разів тестував за допомогою продуктів. Дуже кумедно чути, як носії "я
всегда говорил по-русски" кажуть "буряк", "цибуля", а в Одесі ще "синие"...)))


%%%fbauth
%%%fbauth_name
\iusr{Oksana Ostapenko}
%%%fbauth_url
%%%fbauth_place
%%%fbauth_id
%%%fbauth_front
%%%fbauth_desc
%%%fbauth_www
%%%fbauth_pic
%%%fbauth_pic portrait
%%%fbauth_pic background
%%%fbauth_pic other
%%%fbauth_tags
%%%fbauth_pubs
%%%endfbauth
 

Я вперше була в Одесі влітку 2013. Найбільше мене вразила велика кількість
україномовних. Особливо серед персоналу кафе і ресторанів. Не очікувала. Тому,
коли приїхала туди знову у 2015, ніяк не могла зрозуміти, де всі вони поділись.
Що сталось? У моїй голові була впевненість, що після 2014 україномовних стане в
рази більше. Але реальність була іншою, на жаль. І досі вона не змінилась

\begin{itemize}
%%%fbauth
%%%fbauth_name
\iusr{Сергій Лащенко}
%%%fbauth_url
%%%fbauth_place
%%%fbauth_id
%%%fbauth_front
%%%fbauth_desc
%%%fbauth_www
%%%fbauth_pic
%%%fbauth_pic portrait
%%%fbauth_pic background
%%%fbauth_pic other
%%%fbauth_tags
%%%fbauth_pubs
%%%endfbauth
 
\textbf{Oksana Ostapenko} Може, вам просто здалося? У 2013- му не могло бути багато...

%%%fbauth
%%%fbauth_name
\iusr{Yaroslav Datsko}
%%%fbauth_url
%%%fbauth_place
%%%fbauth_id
%%%fbauth_front
%%%fbauth_desc
%%%fbauth_www
%%%fbauth_pic
%%%fbauth_pic portrait
%%%fbauth_pic background
%%%fbauth_pic other
%%%fbauth_tags
%%%fbauth_pubs
%%%endfbauth
 
\textbf{Oksana Ostapenko} Мабуть мода, кон'юнктура

%%%fbauth
%%%fbauth_name
\iusr{Oksana Ostapenko}
%%%fbauth_url
%%%fbauth_place
%%%fbauth_id
%%%fbauth_front
%%%fbauth_desc
%%%fbauth_www
%%%fbauth_pic
%%%fbauth_pic portrait
%%%fbauth_pic background
%%%fbauth_pic other
%%%fbauth_tags
%%%fbauth_pubs
%%%endfbauth
 
\textbf{Сергій Лащенко} майже всі до мене відповідали тоді українською.

%%%fbauth
%%%fbauth_name
\iusr{Сергій Лащенко}
%%%fbauth_url
%%%fbauth_place
%%%fbauth_id
%%%fbauth_front
%%%fbauth_desc
%%%fbauth_www
%%%fbauth_pic
%%%fbauth_pic portrait
%%%fbauth_pic background
%%%fbauth_pic other
%%%fbauth_tags
%%%fbauth_pubs
%%%endfbauth
 
\textbf{Oksana Ostapenko} Можливо, ви симпатична, впевнена в собі, і це впливає... А я от був на Печерських пагорбах (Київ), то української майже не чув. В районі університету мову чути від студентів, але враховуючи те, що є багато іноземців та місцевих зросійщених, то й тут мови не більше 30 відсотків. Не так вже й суттєво Київ відрізняється від Одеси чи Харкова )))
\end{itemize}

%%%fbauth
%%%fbauth_name
\iusr{Регіна Лялюшко-Лозицька}
%%%fbauth_url
%%%fbauth_place
%%%fbauth_id
%%%fbauth_front
%%%fbauth_desc
%%%fbauth_www
%%%fbauth_pic
%%%fbauth_pic portrait
%%%fbauth_pic background
%%%fbauth_pic other
%%%fbauth_tags
%%%fbauth_pubs
%%%endfbauth
 
Дякую. Аргументація класна😊

%%%fbauth
%%%fbauth_name
\iusr{Lacea Victoria}
%%%fbauth_url
%%%fbauth_place
%%%fbauth_id
%%%fbauth_front
%%%fbauth_desc
%%%fbauth_www
%%%fbauth_pic
%%%fbauth_pic portrait
%%%fbauth_pic background
%%%fbauth_pic other
%%%fbauth_tags
%%%fbauth_pubs
%%%endfbauth
 

\ifcmt
  ig https://scontent-frx5-2.xx.fbcdn.net/v/t39.1997-6/s168x128/198679738_1581651558693954_1103589833244933008_n.png?_nc_cat=1&ccb=1-5&_nc_sid=ac3552&_nc_ohc=63fh8eOLNEYAX9H1JNK&tn=lCYVFeHcTIAFcAzi&_nc_ht=scontent-frx5-2.xx&oh=78cd72dbfdccebc82a9eb0b3739cf34a&oe=613EA4F7
  @width 0.2
\fi

%%%fbauth
%%%fbauth_name
\iusr{Юрій Олександренко}
%%%fbauth_url
%%%fbauth_place
%%%fbauth_id
%%%fbauth_front
%%%fbauth_desc
%%%fbauth_www
%%%fbauth_pic
%%%fbauth_pic portrait
%%%fbauth_pic background
%%%fbauth_pic other
%%%fbauth_tags
%%%fbauth_pubs
%%%endfbauth
 
а хіба у вас суржик? по тому, як ви пишете, то і розмовляти маєте близько до літературної.

\begin{itemize}
%%%fbauth
%%%fbauth_name
\iusr{Yakovenko Anatoliy}
%%%fbauth_url
%%%fbauth_place
%%%fbauth_id
%%%fbauth_front
%%%fbauth_desc
%%%fbauth_www
%%%fbauth_pic
%%%fbauth_pic portrait
%%%fbauth_pic background
%%%fbauth_pic other
%%%fbauth_tags
%%%fbauth_pubs
%%%endfbauth
 
\textbf{Юрій Олександренко} ...пишете.

%%%fbauth
%%%fbauth_name
\iusr{Serhii Bryhar}
%%%fbauth_url
%%%fbauth_place
%%%fbauth_id
%%%fbauth_front
%%%fbauth_desc
%%%fbauth_www
%%%fbauth_pic
%%%fbauth_pic portrait
%%%fbauth_pic background
%%%fbauth_pic other
%%%fbauth_tags
%%%fbauth_pubs
%%%endfbauth
 
\textbf{Юрій Олександренко} Та наче стежу за собою. Але "їм" не важливо! Якщо таке питання взагалі було сформульовано, то людина й не намагалася прислухатися... Оці слова - намагання вибити із зони комфорту, примусити підлаштуватися.

%%%fbauth
%%%fbauth_name
\iusr{Юрій Олександренко}
%%%fbauth_url
%%%fbauth_place
%%%fbauth_id
%%%fbauth_front
%%%fbauth_desc
%%%fbauth_www
%%%fbauth_pic
%%%fbauth_pic portrait
%%%fbauth_pic background
%%%fbauth_pic other
%%%fbauth_tags
%%%fbauth_pubs
%%%endfbauth
 
\textbf{Анатолій Яковенко} дякую!
\end{itemize}

%%%fbauth
%%%fbauth_name
\iusr{Liudmyla Chayka}
%%%fbauth_url
%%%fbauth_place
%%%fbauth_id
%%%fbauth_front
%%%fbauth_desc
%%%fbauth_www
%%%fbauth_pic
%%%fbauth_pic portrait
%%%fbauth_pic background
%%%fbauth_pic other
%%%fbauth_tags
%%%fbauth_pubs
%%%endfbauth
 
Дякую вам

%%%fbauth
%%%fbauth_name
\iusr{Lacea Victoria}
%%%fbauth_url
%%%fbauth_place
%%%fbauth_id
%%%fbauth_front
%%%fbauth_desc
%%%fbauth_www
%%%fbauth_pic
%%%fbauth_pic portrait
%%%fbauth_pic background
%%%fbauth_pic other
%%%fbauth_tags
%%%fbauth_pubs
%%%endfbauth
 
"Чистої" росiйської в Українi 0...ZERO... з Українською все набагато краще. Це факт. Але дивної росiйської забагато...

%%%fbauth
%%%fbauth_name
\iusr{Олена Єрьоменко}
%%%fbauth_url
%%%fbauth_place
%%%fbauth_id
%%%fbauth_front
%%%fbauth_desc
%%%fbauth_www
%%%fbauth_pic
%%%fbauth_pic portrait
%%%fbauth_pic background
%%%fbauth_pic other
%%%fbauth_tags
%%%fbauth_pubs
%%%endfbauth
 
Так

%%%fbauth
%%%fbauth_name
\iusr{Iryna Molko}
%%%fbauth_url
%%%fbauth_place
%%%fbauth_id
%%%fbauth_front
%%%fbauth_desc
%%%fbauth_www
%%%fbauth_pic
%%%fbauth_pic portrait
%%%fbauth_pic background
%%%fbauth_pic other
%%%fbauth_tags
%%%fbauth_pubs
%%%endfbauth
 
Ключове слово: МИ НЕ СЛАБКІ. Повільно опонента розігрівати, а злити той простий потрафе. як казала моя бабця

% -------------------------------------
\ii{fbauth.brodeckij_aleksandr.chernovcy.ukraina.filosofia}
% -------------------------------------

А з іншого боку, вона визнає цінність чистої української. Бо є ж і такі
малороси, які кажуть: "Вот суржик - это и есть укрАинский, он очень близок к
русскому. А "чистый украинский" - это под влиянием польского". Ось ця друга
позиція набагато шкідливіша, ніж визнання цінності чистої української.

\begin{itemize}
%%%fbauth
%%%fbauth_name
\iusr{Alla Krasowska}
%%%fbauth_url
%%%fbauth_place
%%%fbauth_id
%%%fbauth_front
%%%fbauth_desc
%%%fbauth_www
%%%fbauth_pic
%%%fbauth_pic portrait
%%%fbauth_pic background
%%%fbauth_pic other
%%%fbauth_tags
%%%fbauth_pubs
%%%endfbauth
 
\textbf{Олександр Бродецький}
Та нічого вона не визнає,
просто провокує сперечання.

%%%fbauth
%%%fbauth_name
\iusr{Олена Олійник}
%%%fbauth_url
%%%fbauth_place
%%%fbauth_id
%%%fbauth_front
%%%fbauth_desc
%%%fbauth_www
%%%fbauth_pic
%%%fbauth_pic portrait
%%%fbauth_pic background
%%%fbauth_pic other
%%%fbauth_tags
%%%fbauth_pubs
%%%endfbauth
 

То є давня й сто разів обіграна маніпуляція, кінцеву мету якої позначає та
мантра, що в пана Сергія наприкінці (краще як завгодно криво парускі). Сама ця
ґаспажа може й не знати, що її використовують як знаряддя, бо з її точки зору
все виглядає цілком логічно, навіть природно. Це все ж не дурні люди придумали.

\end{itemize}

%%%fbauth
%%%fbauth_name
\iusr{Андрій Костюченко}
%%%fbauth_url
%%%fbauth_place
%%%fbauth_id
%%%fbauth_front
%%%fbauth_desc
%%%fbauth_www
%%%fbauth_pic
%%%fbauth_pic portrait
%%%fbauth_pic background
%%%fbauth_pic other
%%%fbauth_tags
%%%fbauth_pubs
%%%endfbauth
 

Тільки пів години тому обговорював з донецьким колегою причини, які б спонукали
його перейти на українську в повсякденному спілкуванні. Його варіант - тільки
якщо його оточення буде суцільно україномовним. Я знову почув про роздратування
коли він або оточуючі говорять суржиком. Тоді я його запитав чи він дратувався
коли почав вчити англійську і чи стало це приводом, що припинити її вчити далі.
Щось знітився@igg{fbicon.wink}


%%%fbauth
%%%fbauth_name
\iusr{Березенська Неля}
%%%fbauth_url
%%%fbauth_place
%%%fbauth_id
%%%fbauth_front
%%%fbauth_desc
%%%fbauth_www
%%%fbauth_pic
%%%fbauth_pic portrait
%%%fbauth_pic background
%%%fbauth_pic other
%%%fbauth_tags
%%%fbauth_pubs
%%%endfbauth
 

Такі закиди і насмішки про суржик чула колись від колег, але моя відповідь була
проста - у кожній країні проблеми з культурою мови. У Лондоні це суржик
"кокні", який має дуже віддалене відношення до літературної англійської мови,
або у США літературною мовою розмовляють викладачі, політики, проповідники,
тощо - всі інші на суржику.

% -------------------------------------
\ii{fbauth.chupryna_evgenia.kiev.ukraina.literatura.songwriter}
% -------------------------------------

Я відповідаю, що я насправді російський філолог, і прошу говорити зі мною
українською, бо нема сил слухати цей покруч, який видають за мову Пушкіна і
Тургенєва. А ви знаєте, кажу я, що москвини, коли чують таку російську, як ото
у вас, кажуть: ви гаварітє, как із жопьі? А ви знаєте, кажу я, що вони
російськомовних українців називають "дамбас"?

\begin{itemize}
%%%fbauth
%%%fbauth_name
\iusr{Валентина Семеняк}
%%%fbauth_url
%%%fbauth_place
%%%fbauth_id
%%%fbauth_front
%%%fbauth_desc
%%%fbauth_www
%%%fbauth_pic
%%%fbauth_pic portrait
%%%fbauth_pic background
%%%fbauth_pic other
%%%fbauth_tags
%%%fbauth_pubs
%%%endfbauth
 
\textbf{Євгенія Чуприна} дякую! Беру на озброєння)))

%%%fbauth
%%%fbauth_name
\iusr{Svitlana Chub-Krywuzka}
%%%fbauth_url
%%%fbauth_place
%%%fbauth_id
%%%fbauth_front
%%%fbauth_desc
%%%fbauth_www
%%%fbauth_pic
%%%fbauth_pic portrait
%%%fbauth_pic background
%%%fbauth_pic other
%%%fbauth_tags
%%%fbauth_pubs
%%%endfbauth
 
\textbf{Євгенія Чуприна} бравіссімо.

%%%fbauth
%%%fbauth_name
\iusr{Nadiia Obertan}
%%%fbauth_url
%%%fbauth_place
%%%fbauth_id
%%%fbauth_front
%%%fbauth_desc
%%%fbauth_www
%%%fbauth_pic
%%%fbauth_pic portrait
%%%fbauth_pic background
%%%fbauth_pic other
%%%fbauth_tags
%%%fbauth_pubs
%%%endfbauth
 
Я також беру на озброєння!
\end{itemize}

%%%fbauth
%%%fbauth_name
\iusr{Галина Мовлянова}
%%%fbauth_url
%%%fbauth_place
%%%fbauth_id
%%%fbauth_front
%%%fbauth_desc
%%%fbauth_www
%%%fbauth_pic
%%%fbauth_pic portrait
%%%fbauth_pic background
%%%fbauth_pic other
%%%fbauth_tags
%%%fbauth_pubs
%%%endfbauth
 
\ifcmt
  ig https://scontent-frx5-2.xx.fbcdn.net/v/t39.1997-6/p480x480/91521538_1030933857302751_5093925307199520768_n.png?_nc_cat=1&ccb=1-5&_nc_sid=0572db&_nc_ohc=eraWzT5ZbxsAX_r_kp9&_nc_ht=scontent-frx5-2.xx&oh=97d30b8e177b35d9fb695cdad904bdb8&oe=613F2677
  @width 0.2
\fi

%%%fbauth
%%%fbauth_name
\iusr{Анна Бутова}
%%%fbauth_url
%%%fbauth_place
%%%fbauth_id
%%%fbauth_front
%%%fbauth_desc
%%%fbauth_www
%%%fbauth_pic
%%%fbauth_pic portrait
%%%fbauth_pic background
%%%fbauth_pic other
%%%fbauth_tags
%%%fbauth_pubs
%%%endfbauth
 
Літературна українська і як розмовляє народ завжди відрізняється. Суржик у побуті багатьох україномовних . Діалекти різняться. Це нормально .

%%%fbauth
%%%fbauth_name
\iusr{Atmeshvar Anatoly}
%%%fbauth_url
%%%fbauth_place
%%%fbauth_id
%%%fbauth_front
%%%fbauth_desc
%%%fbauth_www
%%%fbauth_pic
%%%fbauth_pic portrait
%%%fbauth_pic background
%%%fbauth_pic other
%%%fbauth_tags
%%%fbauth_pubs
%%%endfbauth
 
Треба самим нападати: мені не подобається гидотний московський суржик.



%%%fbauth
%%%fbauth_name
\iusr{Анатолій Грива}
%%%fbauth_url
%%%fbauth_place
%%%fbauth_id
%%%fbauth_front
%%%fbauth_desc
%%%fbauth_www
%%%fbauth_pic
%%%fbauth_pic portrait
%%%fbauth_pic background
%%%fbauth_pic other
%%%fbauth_tags
%%%fbauth_pubs
%%%endfbauth
 
Дякую за добрий допис

%%%fbauth
%%%fbauth_name
\iusr{Ярослав Шевчук}
%%%fbauth_url
%%%fbauth_place
%%%fbauth_id
%%%fbauth_front
%%%fbauth_desc
%%%fbauth_www
%%%fbauth_pic
%%%fbauth_pic portrait
%%%fbauth_pic background
%%%fbauth_pic other
%%%fbauth_tags
%%%fbauth_pubs
%%%endfbauth
 
з цікавістю і прихильно слідкую за твоїми міркуваннями, був в Одесі в різних місцях і ситуаціях, знаю, вислів НЕ В МОЇХ ПРАВИЛАХ є маскалізмом

%%%fbauth
%%%fbauth_name
\iusr{Olga Tumenko}
%%%fbauth_url
%%%fbauth_place
%%%fbauth_id
%%%fbauth_front
%%%fbauth_desc
%%%fbauth_www
%%%fbauth_pic
%%%fbauth_pic portrait
%%%fbauth_pic background
%%%fbauth_pic other
%%%fbauth_tags
%%%fbauth_pubs
%%%endfbauth
 

Навіть якщо ми і розмовляємо суржиком, то все рівно наша, українська мова. Так, ще
не зовсім літературна, але ми вчимося після того, що в нас скільки років все було
зросійщене. Але краще так, мовою, чим їх язиком.

%%%fbauth
%%%fbauth_name
\iusr{Олександр Патякало}
%%%fbauth_url
%%%fbauth_place
%%%fbauth_id
%%%fbauth_front
%%%fbauth_desc
%%%fbauth_www
%%%fbauth_pic
%%%fbauth_pic portrait
%%%fbauth_pic background
%%%fbauth_pic other
%%%fbauth_tags
%%%fbauth_pubs
%%%endfbauth
 
Питання до скрєпоносних: Чи є в країнах Європи окремо сільські мови.

\begin{itemize}
%%%fbauth
%%%fbauth_name
\iusr{Ірина Дмитрів}
%%%fbauth_url
%%%fbauth_place
%%%fbauth_id
%%%fbauth_front
%%%fbauth_desc
%%%fbauth_www
%%%fbauth_pic
%%%fbauth_pic portrait
%%%fbauth_pic background
%%%fbauth_pic other
%%%fbauth_tags
%%%fbauth_pubs
%%%endfbauth
 
\textbf{Олександр Патякало} влучно!)
\end{itemize}

%%%fbauth
%%%fbauth_name
\iusr{Vitaliy Chemerys}
%%%fbauth_url
%%%fbauth_place
%%%fbauth_id
%%%fbauth_front
%%%fbauth_desc
%%%fbauth_www
%%%fbauth_pic
%%%fbauth_pic portrait
%%%fbauth_pic background
%%%fbauth_pic other
%%%fbauth_tags
%%%fbauth_pubs
%%%endfbauth
 

Шлях до гарної української лежить завжди через погану українську. Треба лише
почати говорити!

Що стосується суржика(не плутати з діалектом), то я однозначно проти нього, бо
вважаю це мовним каліцтвом. Мову треба вчити і вдосконалювати, а не калічити!


%%%fbauth
%%%fbauth_name
\iusr{Пасічник Тетяна}
%%%fbauth_url
%%%fbauth_place
%%%fbauth_id
%%%fbauth_front
%%%fbauth_desc
%%%fbauth_www
%%%fbauth_pic
%%%fbauth_pic portrait
%%%fbauth_pic background
%%%fbauth_pic other
%%%fbauth_tags
%%%fbauth_pubs
%%%endfbauth
 

Сподобалось!!! Респект!!! Не буду матюкатися при конфліктних суперечках про
мову. Ви дали мені гарний і переконливий приклад. Щиро дякую за науку!!!


%%%fbauth
%%%fbauth_name
\iusr{Nadija Kima}
%%%fbauth_url
%%%fbauth_place
%%%fbauth_id
%%%fbauth_front
%%%fbauth_desc
%%%fbauth_www
%%%fbauth_pic
%%%fbauth_pic portrait
%%%fbauth_pic background
%%%fbauth_pic other
%%%fbauth_tags
%%%fbauth_pubs
%%%endfbauth
 

Це ви в Одесі "воюєте", а я, коли приїжджаю на Волинь, то повірте, стан речей
не кращий, а можливо, і гірший. Бо люди там свято впевнені, що розмовляють
українською, хоча через слово, їхня мова кишить русизмами. Внучка в дитсадку
вже виховательок виправляє))

\end{itemize}

