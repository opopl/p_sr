% vim: keymap=russian-jcukenwin
%%beginhead 
 
%%file 23_12_2021.fb.savenkova_faina.2.naumov_ivan_predislovie
%%parent 23_12_2021
 
%%url https://www.facebook.com/permalink.php?story_fbid=439540997666838&id=100048328254371
 
%%author_id savenkova_faina
%%date 
 
%%tags donbass,kniga,literatura,lnr,rossia
%%title Предисловие Ивана Наумова - "Мир, которого нет"
 
%%endhead 
 
\subsection{Предисловие Ивана Наумова - \enquote{Мир, которого нет}}
\label{sec:23_12_2021.fb.savenkova_faina.2.naumov_ivan_predislovie}
 
\Purl{https://www.facebook.com/permalink.php?story_fbid=439540997666838&id=100048328254371}
\ifcmt
 author_begin
   author_id savenkova_faina
 author_end
\fi

Ну и наконец  предисловие писателя, сценариста и поэта Ивана Наумова Огромное
спасибо.

Для меня это очень важно и почетно.

Трудно себе представить более необычный авторский дуэт: девочка-драматург,
недавно сделавшая первые шаги в бурное море Фантастики, и многоопытный
писатель-воин, прошедший огонь и воду не только в вымышленных мирах, но и в
нашей непростой действительности. 

\ii{23_12_2021.fb.savenkova_faina.2.naumov_ivan_predislovie.pic.1}

Их знакомство на гостеприимной и героической земле Донбасса вряд ли было
случайностью – где-то на небесных осях вовремя повернулись нужные шестерёнки, и
звёзды выстроились в правильную комбинацию.  «Мир, которого нет» – второй
совместный роман творческого дуэта, приключенческая история на стыке
современной сказки и городского фэнтэзи, фантасмагории и философской притчи.

Фаина Савенкова вложила в текст непосредственность, волшебность, чуть наивное,
но искреннее восприятие несовершенства «мира взрослых», скованного догмами,
выдуманными надобностями и социальными статусами. 

Александр Конторович привнёс в историю ироничное и точное понимание, как любой
стройный проект идёт вкривь и вкось из-за отсутствия «идеальных исполнителей»,
к каким странностям и нелепостям приводят попытки создания безупречно
функционирующих структур. Люди – не винтики. И даже винтики – иногда люди.

В наше быстрое время принято описывать новое через отсылки к старому и
известному. Что ж, в моём восприятии первыми референсами стали обе
Кэрролловских «Алисы», «Королевство кривых зеркал», «Путешествия Гулливера».

Позже вспомнился роман одного из столпов киберпанка Руди Рюкера «Белый свет». А
развязка «Мира, которого нет» неожиданно дала горьковатое и совсем недетское
послевкусие, зацепила за живое, поставила вопросы, на которые не так-то просто
подобрать удовлетворительный ответ. Здесь уже без аналогий и спойлеров – так
интереснее!

Иван Наумов

Писатель, сценарист, поэт
