% vim: keymap=russian-jcukenwin
%%beginhead 
 
%%file 30_06_2021.fb.krjukova_svetlana.1.supermarket_chelovek
%%parent 30_06_2021
 
%%url https://www.facebook.com/kryukova/posts/10159505589888064
 
%%author Крюкова, Светлана
%%author_id krjukova_svetlana
%%author_url 
 
%%tags chelovek,gorod,kiev,krjukova_svetlana,ukraina,zhizn
%%title Вечерние супермаркеты за час до закрытия - шкаф человечьих слабостей и привычек
 
%%endhead 
 
\subsection{Вечерние супермаркеты за час до закрытия - шкаф человечьих слабостей и привычек}
\label{sec:30_06_2021.fb.krjukova_svetlana.1.supermarket_chelovek}
\Purl{https://www.facebook.com/kryukova/posts/10159505589888064}
\ifcmt
 author_begin
   author_id krjukova_svetlana
 author_end
\fi

Вечерние супермаркеты за час до закрытия - шкаф человечьих слабостей и
привычек. Хотите понять жизнь человека, изучите его вечерний гастро-шопинг. 

Вот взять, к примеру, меня или мою бабушку, которая всю нашу неприязнь к этой
процедуре - поход по супермаркетам - хорошо изложила двумя фразами:
«Просаживать столько времени в очередь за кормёжкой?! Что может быть
утомительней?». 

Я в супермаркетах люблю потеряться, чтобы писать там посты, где-то у аквариума
с карасями, или часами глазеть на людей. Они так увлечены процессом, как я ими.

Итак, 22.15. Кассирши под конец дня уморились как вяленые помидоры, на автомате
щёлкая товары прибором, похожим на лазерный эпилятор. 

Изо всех закоулков веет меркантильным духом и уходящим днём. Неизменный
охранник в чёрных очках и самых начищенных туфлях, преисполнен сознанием своей
значимости. Чудным образом он уравновешивает в себе одном секьюрити из фильма
«телохранитель» и недовольного дядю Ваню, который всю жизнь хотел стать
президентом, но работает охранником в гастрономе, который нынче по-модному
называется супермаркет. Он пристально, словно сканер, следит за залом. 

Лицо парней, озарённое предвкушением шумной вечеринки посреди рабочей недели
заметно контрастирует с равнодушным лицом продавщицы на раздаче. Для них это «А
насыпьте нам килограмм Оливье!» - центральное украшение холостяцкого стола. Для
неё - 157 заказ, выданный за вечер. Ей плевать, что они будут есть на ночи, как
и с кем напиваться. Она молча раздает заказы и поглядывает на часы. 

Два коренных киевлянина, из тех, что заходят по вечерам в магазин в целях
социализации, жалуются на цены на яйца и на столичных застройщиков, которые
вместо домов понастроили на печерских холмах ящики из-под мыла и испортили
Киев. 

Дама с собачкой, в коротких шортах, умело зажав пушистого питомца подмышкой,
одной рукой укладывает в тележку две бутылки белого, второй записывается по
телефону на детокс-обвёртывание уже на завтра. 

- Танечка, я сегодня планирую напиться. А к пятницы надо как огурчик. Завтра у
тебя, - звонко анонсирует дама с собачкой, в очередной раз шокируя  меня этой
манерой продуманных современных женщин все события в жизни - от незащищенного
секса до похода к гинекологу на следующий же день - планировать
заблаговременно. 

Уставший от дорогих и капризных ремонтов работяга, в пропахшей трудом и потом
футболке, тащит на кассу холодную бутылку пива и вкусную колбасу - ему хватит.
Супермаркеты не поломали законы неравенства, но сделали их незаметными, как
сделалось это с дорожной пылью или городскими пробками. 

Самые нормальные - дети. Они выбирают товары по этикеткам, по ярким,
мультяшные, улыбающимся. . В любое время суток у них неизменны - аппетит, вкусы
и настроение. Они, глютено-глюкозные наркоманы, как нормальные люди, всегда
хотят мороженого, шоколад, сосиски и сладкую воду. Короче, все тут по классике
- детям мороженое, женщинам - шампанское!

\ii{30_06_2021.fb.krjukova_svetlana.1.supermarket_chelovek.cmt}
