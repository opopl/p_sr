% vim: keymap=russian-jcukenwin
%%beginhead 
 
%%file 21_02_2023.stz.news.ua.donbas24.2.mizhn_den_ridnoj_movy.txt
%%parent 21_02_2023.stz.news.ua.donbas24.2.mizhn_den_ridnoj_movy
 
%%url 
 
%%author_id 
%%date 
 
%%tags 
%%title 
 
%%endhead 

21.02.2023
Ольга Демідко (Маріуполь)
Україна,УкраїнськаМова,РіднаМова,Свято,date.21_02_2023
21_02_2023.olga_demidko.donbas24.mizhn_den_ridnoj_movy

Міжнародний день рідної мови

21 лютого в Україні святкують Міжнародний день рідної мови

В Україні вже 23 роки поспіль відзначають Міжнародний день рідної мови. Це
свято бере початок від трагічних подій 1952 року в Бангладеш. А офіційно дата
була встановлена у 1999 році, коли ЮНЕСКО прийняла рішення підтримати мовне
різноманіття у світі. З огляду на щорічне зменшення мов у світі, яке пов’язано
зі зникненням носіїв мови та втратою культури певного народу, це свято набуває
все більшої важливості та актуальності.

Читайте також: Як швидко перейти на українську мову: корисні поради

Історія і традиції дати

Міжнародний день рідної мови започаткували у листопаді 1999 року на ХХХ сесії
Генеральної конференції ЮНЕСКО з метою захисту мовної й культурної
багатоманітності. Головним гаслом свята є: «Мови — це важливо!». Дату обрали не
випадково. Саме 21 лютого 1952 року в Бангладеш відбувся збройний конфлікт
через мову. Тоді загинуло 4 студенти. Рідна мова — це носій знань і цінностей,
які передаються від одного покоління іншому в якості нематеріального
культурного спадку. З 1971 року, після проголошення незалежності Бангладеш, цей
день відзначають у країні як день мучеників, які загинули за рідну мову. За
пропозицією цієї країни ЮНЕСКО проголосило 21 лютого Міжнародним днем рідної
мови. На жаль, близько 43% мов сьогодні перебувають на межі зникнення. А 40%
людей не мають змоги отримати освіту рідною мовою. Загалом у Європі під
загрозою перебувають 30 мов, 13 із яких — на межі зникнення.

Цього дня в Україні проводяться різноманітні освітні заходи, присвячені рідній
мові в школах, університетах та бібліотеках. Також цього дня на виставках та
майстер-класах популяризують народне мистецтво.

Читайте також: Мовні норми вступають в силу — яких правил треба дотримуватися

Уповноважений із захисту державної мови Тарас Кремінь привітав всіх українців
із цим чудовим святом і наголосив, що «за українську мову, національну
ідентичність та право бути господарями на своїй землі гинуть найкращі сини та
доньки країни».

«За українську мову, національну ідентичність та право бути господарями на
своїй землі гинуть кращі сини та доньки країни».

Цікаві факти про українську мову

Сучасна українська мова за останніми підрахунками має близько 256 тисяч слів.

Особливістю української мови є наявність великої кількості
зменшувально-пестливих форм. Навіть слово «вороги» може звучати як
«вороженьки». А назви всіх дитинчат тварин є іменниками середнього роду:
теля, котеня, жабеня.

Українську мову забороняли близько 134 разів.

Найдовше слово має аж тридцять літер. Воно означає хімікат для боротьби зі шкідниками: «дихлордифенілтрихлорметилметан».

Найближчою до української мови за лексичним запасом є білоруська — 84%
спільної лексики. Далі йдуть польська і сербська (70% і 68% відповідно),
потім — російська (62%). Якщо порівнювати фонетику й граматику, то
українська має від 22 до 29 спільних рис з білоруською, чеською, словацькою
й польською мовами, а з російською — тільки 11.

Читайте також: Слово 2022 року: який вираз став головним в рік війни

6. Українська мова, на відміну від інших слов’янських мов, зберегла кличний
відмінок. Тому в нас є цілих сім відмінків.

7. Цікаво, що у 448 році візантійський історик Пріск Панійський під час
перебування в таборі гунського володаря Аттіли на території сучасної України,
записав слова «мед» і «страва». Це була перша згадка українських слів. Риси
української мови наявні ще в писемних пам’ятках XIV — XV століття.

8. У різні історичні періоди українську мову називали по-різному: про́ста,
руська, русинська, козацька тощо. Історично найуживанішою назвою української до
середини XIX століття була назва «руська мова».

9. Найбільша кількість слів в нашій мові розпочинається на літеру «п», а
найменша — на літеру «ф».

10. Наша мова дуже багата на синоніми. Зокрема, слово «горизонт» має 12
синонімів: обрій, небозвід, небосхил, крайнебо, круговид, кругозір, кругогляд,
виднокруг, видноколо, виднокрай, небокрай, овид.

Текст підготовлено за наступними джерелами: suspilne.media, np.pl.ua

Нагадаємо, раніше Донбас24 розповідав про безкоштовні курси української мови.

Ще більше новин та найактуальніша інформація про Донецьку та Луганську області
в нашому телеграм-каналі Донбас24.

Фото: з відкритих джерел
