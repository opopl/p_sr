% vim: keymap=russian-jcukenwin
%%beginhead 
 
%%file 15_10_2020.news.ua.radio_svoboda.1_ponomariv
%%parent 15_10_2020
%%url https://www.radiosvoboda.org/a/30893319.html
 
%%endhead 

\subsection{Олександр Пономарів. Вічна Покрова}

\url{https://www.radiosvoboda.org/a/30893319.html}

(Рубрика «Точка зору»)

Валерій Ясиновський

Сиротіємо...

Ідуть у засвіти Титани, опускаючи на наші рамена уселенський тягар.

Залюднюють світлими душами святі небесні катедри, щоб уже звідти зоріти на нас,
учнів своїх, зором Тих, хто знає все. Пильним усевидющим поглядом не впокоєної
вічності, а доскіпливої Перестороги і вимогливого Сподівання.

\subsubsection{Олександр Данилович Пономарів}

Дивовижно сплів у цій постаті Творець непідробну м'яку інтелігентність з
істинно бійцівською, сталевою непохитністю. З могуттю Жерця Храму, якому він
офірував своє Земне Причетництво.

Святого Храму Української Мови. Того предвічного Слова, що незрушно стоїть на
сторожі коло нас і на чатах України. 

Для мене він був хрестоматійним Шістдесятником. Не в університеті, тоді він іще
тільки-но починав своє викладацтво, делікатно, у тіні блискучої Алли Петрівни
Коваль.

Ми, випускники 1981-го, здається, були першими, кого взяв під крило Олександр
Данилович, тоді ще не увінчаний славою професор, а лише старший викладач. Вів
семінарські заняття з практичної стилістики.

На перший погляд – викапаний тип героя раннього Тичини. «Срібнотонний». Але то
була тільки видимість. Бо згодом і ми відчули, як в оцій
інтелігентно-пастельній оболонці нуртує міцна воля, яка там потужна опірність
тогочасним партійно-ідеологічним «святощам».

Він миттєво парирував агресивний російськомовний кіч, спроби відмазатися від
занять, бо «я с русского города і мнє тяжєло по-укрАінскі», а тим паче будь-які
наміри – вільні чи невільні – упослідити українську мову.

Міг так убрити самовпевненого неука, що від сірки його слів це неуцтво
світилося на вузьколобих роками.

 Акція на підтримку української мови в Києві (архівне фото)

Якось почув, що ми з одногрупником Олегом Біликом мешкаємо в одній
гуртожитській кімнаті зі студентами-греками. «Товариство, можна я до вас
навідаюся? Поспілкуюся з носіями мови. Я ж завершую перекладати українською
Яніса Ріцоса».

І якогось вечора прийшов. Пили грецький чебрецевий чай з українським бубликами
і грецьким-таки медом, а він допізна про щось жваво жебонів новогрецькою з
нашим Хрістосом і Дімітрасом, які аж пашіли од задоволення поговорити своєю
рідною мовою з українцем, та ще й про таку знакомиту постать.

Наші друзі-греки обожнювали Ріцоса: практично кожен із їхнього студентського
земляцтва мав опасистий – з виду наче Біблія – том його віршів і поем. Вони
часто і виспівували їх як псалми: Ріцос був для них «орфеєм комунізму»,
«елладським Маяковським»...

За місяць чи два ми вже тримали в руках свіжі-пахнющі томики українськомовних
поезій Яніса Ріцоса – подарунок Олександра Даниловича. 

Він уповні постав для мене Шістдсятником пізніше, наприкінці 1980-х, коли я й
сам збагатився знаннями не з тогочасних підручників та хрестоматій. Коли доля
нас звела у знаменитому Хорі «Гомін» Леопольда Ященка. Він не співав, але часто
бував на концертах. Чомусь особливо запам'ятав його у Гідропарку, на
святкуванні Івана Купала: сидів у добірному товаристві В'ячеслава Чорновола,
Івана Гончара, Сергія Плачинди, Анатолія Мокренка... І саме присутність його,
мого викладача, розковувала, знімала з мене, тодішнього соліста, звичний трем
вічного хвилювання перед глядачем. 

Востаннє ми поговорили біля Володимирського собору. На панахиді-прощанні з
Леопольдом Ященком. Вони були давніми і добрими приятелями.

Випадали ще стрічі на Дні факультету, але у тій вічній шамотні вичерпувалися
кількома вітальними приязними словами.

«Яка житейська насолода буває вільна від печалі?»

Він пішов на Покрову, під нетлінний омофор Заступниці. Але залишив нам Храм.
Розкішний, величний, несмертний.

Вічнопам'ятний Воїне, у зброю Світла зодягнений, слава Тобі!

Дякую, Вчителю...

І доземно схиляю голову. 

Валерій Ясиновський – журналіст, завідувач НВМ «Грінченко-інформ» Інституту
журналістики Університету Грінченка

Думки, висловлені в рубриці «Точка зору», передають погляди самих авторів і не
конче відображають позицію Радіо Свобода.
