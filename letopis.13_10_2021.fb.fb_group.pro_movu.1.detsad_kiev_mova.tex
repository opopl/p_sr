% vim: keymap=russian-jcukenwin
%%beginhead 
 
%%file 13_10_2021.fb.fb_group.pro_movu.1.detsad_kiev_mova
%%parent 13_10_2021
 
%%url https://www.facebook.com/groups/promovugroup/posts/1066363740604153
 
%%author_id fb_group.pro_movu,marusik_taras
%%date 
 
%%tags deti,detsad,jazyk,mova,ukraina,ukrainizacia
%%title Украинский Язык - Киев - Детский Сад
 
%%endhead 
 
\subsection{Украинский Язык - Киев - Детский Сад}
\label{sec:13_10_2021.fb.fb_group.pro_movu.1.detsad_kiev_mova}
 
\Purl{https://www.facebook.com/groups/promovugroup/posts/1066363740604153}
\ifcmt
 author_begin
   author_id fb_group.pro_movu,marusik_taras
 author_end
\fi

Правдивий аналіз ситуації в столиці від Ірини Штогрін. Мені теж дуже б
хотілося, щоб Київ звучав, як Тернопіль чи Франківськ. Коли опиняєшся у цих
двох містах, особливо в Тернополі, місто звучить українською, практично всюди.

\ifcmt
  pic https://external-frt3-2.xx.fbcdn.net/safe_image.php?d=AQGWVgMQ1-ZUTIlZ&w=500&h=261&url=https%3A%2F%2Fgdb.rferl.org%2F30D746B5-0144-4F51-AE14-580A907B88DC_w1200_r1.jpg&cfs=1&sx=0&sy=16&sw=1200&sh=626&ext=jpg&_nc_oe=6edab&_nc_sid=632586&ccb=3-5&_nc_hash=AQFGfjHzmla7S_UN
  @width 0.4
\fi

\url{https://www.radiosvoboda.org/a/dity-i-ukrayinomovne-seredovyshche/31480901.html}{%
Не «русифікувати»: як у Києві знайти для дитини повністю україномовне середовище?, radiosvoboda.org, %
28.09.2021%
}

«Чи справді важко, наприклад у Києві, уникнути так званої «ранньої спонтанної
русифікації» дітей. Чи є це проблемою? Про це у трьох розповідях мам».

«Знайти в Києві повністю україномовне освітньо-виховне середовище виявилося
проблемою. Хоч бери і переїжджай у Львів чи Франківськ. І зі школою те саме:
українська лише під час уроків, і то не завжди. І російська у спілкуванні на
перервах. От так мій син і вивчив російську».

«Через кілька місяців безрезультатних пошуків, ми вже засумнівалися у тому, «чи
існує взагалі можливість не занурювати дитину, принаймні перші роки життя, у
російськомовне середовище у столиці України?»... досі україномовна дитина
відчуває себе трохи іноземцем у Києві...».

«Запитання в одній із так званих «мамських груп» із проханням порадити школу,
де вчителі та діти розмовляють українською, провисівши на сторінці лише трохи
більше години, спричинило такий сплеск негативу, що модератори були вимушені
видалили його. «Езжайте во Львов», «зачем заострять», «в чем проблема знать
много языков», «какая разница» «важно принимать». 

«...хоч ми живемо в столиці України – Києві, але нам досі часто доводиться
пояснювати, що наша дитина не розуміє російської. Це напружує. Як напружує і
те, що доводиться уточнювати, що ми не із Західної України». 

«Перший рік ми витрачали 40 хвилин на дорогу в один бік і досі ми щодня
добираємось до садочка не пішки. Такою виявилася для нас ціна можливості не
«русифікувати» дитину в Україні з перших років життя».

\begin{cmtfront}
\large
\uzr{Микола Микола}

Агресор напав не на конкретний адрес, вулицю, будинок чи місто! Агресор напав
на Україну!  Війна не Сході нашої Держави!!! Війна в Україні!!! А скільки ще
ВОРОГІВ в тилу, в кожному місті, селі, на кожній вулиці!!!  Боже допоможи
справитись з цією навалою!  Облагорозуми їх...

\uzr{Lars Strashevsky}

Ніколи НЕ ПЕРЕХОДЬ НА ВОРОЖУ МОВУ, ТИ ПОКАЗУЄШ СВОЮ РАБСЬКУ ПОКІРНІСТЬ, ТА
СОРОМ У СЕБЕ ВДОМА НА РІДНІЙ ЗЕМЛІ, і своїм вчинком показує рускоязичніку що
тут він пан а ти раб.

\uzr{Oksana Klokova}

Неможливо уникнути. Нас замало, щоб змінити ситуацію. Я працюю у художній
школі, і не вдається умовити дітей розмовляти українською хоча б на уроках.
Хтось просто не вміє, хтось свідомо відмовляється. Є, звісно, і такі, які зі
мною спілкуються українською, але між собою - майже завжди дитина "ламається",
бо хоче бути своєю... Сумно це...

\uzr{Galina Bogarada}

У нас теж така проблема в Києві, але боремось - обидві внучки розмовляють
тільки українською, а російська для них - мова ворога

\end{cmtfront}

\ii{13_10_2021.fb.fb_group.pro_movu.1.detsad_kiev_mova.cmt}
