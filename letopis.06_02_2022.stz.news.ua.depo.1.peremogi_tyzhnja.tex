% vim: keymap=russian-jcukenwin
%%beginhead 
 
%%file 06_02_2022.stz.news.ua.depo.1.peremogi_tyzhnja
%%parent 06_02_2022
 
%%url https://www.depo.ua/ukr/war/peremogi-tizhnya-v-oos-zabaneni-prorosiyski-internet-propagandisti-i-sudovi-viroki-posibnikam-boyovikiv-l-dnr-202202061420860
 
%%author_id 
%%date 
 
%%tags ato,donbass,oos,ukraina,vojna
%%title Перемоги тижня в ООС: Забанені проросійські пропагандисти і судові вироки посібникам бойовиків
 
%%endhead 
 
\subsection{Перемоги тижня в ООС: Забанені проросійські пропагандисти і судові вироки посібникам бойовиків}
\label{sec:06_02_2022.stz.news.ua.depo.1.peremogi_tyzhnja}
 
\Purl{https://www.depo.ua/ukr/war/peremogi-tizhnya-v-oos-zabaneni-prorosiyski-internet-propagandisti-i-sudovi-viroki-posibnikam-boyovikiv-l-dnr-202202061420860}

\begin{zznagolos}
Війна проти російських окупантів триває. Цього тижня Кремль недорахувався на
Донбасі 10 своїх бойовиків, трьох затримали українські спецслужби і ще двоє
отримали за злочини проти України 8 і 9 років в'язниці
\end{zznagolos}

У 2022 році Depo.ua продовжує відслідковувати перемоги нашої армії. Цього тижня
Збройні сили України разом зі спецслужбами та дипломатами можуть записати собі
в актив наступні звитяги:

\subsubsection{Втрати живої сили бойовиків \enquote{Л-ДНР}}

Щодня на прицільні обстріли російських бойовиків українські захисники надавали
адекватну відповідь.

Упродовж тижня на окупованому Донбасі втрати російських найманців склали,
щонайменше, 10 осіб. З них 5 безповоротні (ліквідовані, вантаж 200) і 5
санітарні (поранені, вантаж 300).

Зокрема, втрати російсько-окупаційних військ 31 січня склали три особи. Двоє
безповоротні (9 омсп 1 АК та 2 омсбр 2 АК) і один санітарний (1 омсбр 1 АК).

1 лютого безповоротно вибув зі строю терорист 9 омсп 1 АК.

2 лютого поранили трьох окупантів (один з 2 омсбр 2 АК і двоє з 9 омсп 1 АК).

Втрати російських бойовиків 3 лютого склали одну особу. Більше ніколи не
турбуватиме українських захисників терорист 1 омсбр 1 АК.

4 лютого одного бойовика ліквідували  (9 омсп 1 АК) і один отримав поранення (2
омсбр 2 АК).

\subsubsection{Підтримка України у світі}

2 лютого уряд Швеції ухвалив виділення близько 5,4 мільйона доларів Україні для
зміцнення стійкості на тлі агресії Росії.

Зокрема, Україна отримає кошти протягом наступних трьох років у межах спільного
з Великою Британією, Канадою, Швейцарією та США фонду.

Фонд зміцнить стійкість до російської агресії на Сході та Півдні України,
сприяючи поліпшенню надання державних послуг, розширенню діалогу та
інклюзивному економічному розвитку.

3 лютого Україна вперше взяла участь в міністерській зустрічі Бухарестської
дев'ятки НАТО – політичного та безпекового формату взаємодії держав-членів
східного флангу Альянсу.

Міністр закордонних справ України Дмитро Кулеба взяв участь в міністерській
зустрічі Бухарестської дев’ятки НАТО на запрошення свого румунського колеги
Богдана Ауреску.

Кулеба детально розповів про концентрацію російських військ поблизу державного
кордону та на тимчасово окупованих територіях України. Міністри погодилися, що
агресія Росії загрожує безпеці не лише України, але й інших країн Центральної
Європи і євроатлантичному простору загалом.

Низка членів Бухарестської дев’ятки НАТО поінформувала, що на запит нашої
держави готує рішення, які допоможуть посилити оборонні спроможності,
кіберзахист та економічну стабільність України.

4 лютого Сенат Польщі одноголосно ухвалив резолюцію про підтримку України. За
проголосували всі 92 сенатори.

Сенат підтримав політичну та фінансову допомогу Україні, постачання зброї та
участь солдатів з країн-членів НАТО у підготовці українських військових.

\enquote{Сьогодні, коли Україна стоїть перед смертельною загрозою – загрозою чергового
вторгнення, – особливо важливо, щоб влада Республіки Польща надійно
підтримувала нашого сусіда. Без незалежної України немає безпечної Польщі.
Вільна Україна – одна з гарантій безпеки Польщі}, – йдеться у резолюції.

\subsubsection{Блокування пропагандистів Росії в інтернеті}

4 лютого YouTube заблокував канали \enquote{Перший незалежний} та UkrLive, які
пов'язані з депутатами від проросійської партії \enquote{Опозиційної платформи
– За життя}.

Як зазначає платформа, їх заблокували через \enquote{порушення правил спільноти
YouTube}.

Обидва канали було створено пропагандистами заблокованих рік тому телеканалів
ZIK, \enquote{112 Україна} та NewsOne.

Також 4 лютого відеохостинг YouTube заблокував канали \enquote{міністерства
інформації} та \enquote{народної міліції} \enquote{ДНР}.

Так само YouTube заблокував канали \enquote{ЛНР}: видалив пропагандистський
канал ЛІЦ і \enquote{народна міліція} \enquote{ЛНР} і заблокував канал
\enquote{ДТРК ЛНР}.

Того ж дня незаконне військове угрупування "Народна міліція" "ЛНР" заявила про
блокування своєї сторінки в соцмережі Facebook.  Озброєння для України від
західних партнерів

2 лютого Чехія та Україна підписали угоду про передачу зброї ЗСУ.

Нам передадуть артилерійські боєприпаси та медичну допомогу.

Також 2 лютого влада Польщі ухвалила рішення про передачу Україні оборонного
озброєння та боєприпасів, серед яких офіційно називаються ПЗРК \enquote{Перун}
(Piorun).

ПЗРК \enquote{Перун} призначений для ураження повітряних цілей, що низько
летять – як на зустрічних, так і на наздоганяючих курсах.

Читайте також: Його \enquote{батько} збивав росіян: Що за ПЗРК \enquote{Перун}
дає Україні Польща

1 лютого до України прибув шостий літак з військовою допомогою від США.

У черговій партії військової допомоги – 84 тонни боєприпасів.

3 лютого в Україні приземлився сьомий літак з військовою допомогою від США.

У цій партії Україна отримала 85 тонн бойових боєприпасів для гранатометів.

Читайте також: \enquote{Стінгери}, \enquote{Джавеліни}, вогнемети і
гранатомети: Яку зброю вже дають Україні західні партнери

\subsubsection{Викриття та затримання бойовиків \enquote{Л-ДНР} і їхніх
посіпак}

2 лютого в Донецькій області встановили місцезнаходження і затримали учасницю
угруповання \enquote{ДНР} \enquote{Пятнашка}.

За даними слідства, затриманою є мешканка Мирнограду на Донеччині, яка у 2014
році добровільно приєдналась до незаконних збройних формувань ОРДЛО. Там вона
отримала автоматичну зброю та боєприпаси.

Після \enquote{курсу} підготовки у лавах бойовиків, ватажки залучали її до
виконання завдань проти підрозділів Збройних сил України. Згодом зловмисниця
покинула угруповання і тривалий час переховувалась від правосуддя.
