% vim: keymap=russian-jcukenwin
%%beginhead 
 
%%file 14_06_2022.fb.lonska_tetjana.1.ostoron
%%parent 14_06_2022
 
%%url https://www.facebook.com/tatiana.lonskaja/posts/pfbid02tnZoLJjtHUnbp7gVVY3rHeuBvagkpp7Jz6BaNUCS4i6drZQQeB9jsi7DFphtxRHUl
 
%%author_id lonska_tetjana
%%date 
 
%%tags 
%%title Я, насправді, не розумію, як зараз можна бути осторонь
 
%%endhead 
 
\subsection{Я, насправді, не розумію, як зараз можна бути осторонь}
\label{sec:14_06_2022.fb.lonska_tetjana.1.ostoron}
 
\Purl{https://www.facebook.com/tatiana.lonskaja/posts/pfbid02tnZoLJjtHUnbp7gVVY3rHeuBvagkpp7Jz6BaNUCS4i6drZQQeB9jsi7DFphtxRHUl}
\ifcmt
 author_begin
   author_id lonska_tetjana
 author_end
\fi

Пишу українською, бо цей допис виключно для українців, і я маю надію, що більше ніхто не прочитає.

…Я йду по вулиці, поволі проходжу повз літню терасу якогось кафе. Усі
столики зайняті, з прочинених дверей закладу голосно лунає весела музика.
Різкий запах смаженого м'яса, що парує на тарілках в руках  офіціантки,
чомусь викликає відразу. А у вухах невгамовно дзвенить фальшивий регіт
двох напівголих дівуль, які намагаються сподобатись чоловікам - добряче на
підпитку, що сидять за сусіднім столиком.  

Я відвожу погляд, і він натикається на чорний обрис, що наближається з
іншої сторони вулиці. Це молода мама з двома маленькими дітками, яка
нещодавно з'явилася у нашому ЖК. Переплутати неможливо – вона завжди в
чорному, бо носить жалобу за загиблим чоловіком. В лютому він встиг
вивезти сім'ю з Маріуполя, а через дві доби загинув на фронті.  З її
мамою, що залишалася в підвалі свого будинку, зв'язок перервався іще в
березні, тож шанс, що вони коли-небудь зустрінуться, прямує до нуля. Її
сукня немов прокреслена чорною тушшю. І сама вона – ніби чорна витинанка
на яскравому тлі рожевого куща шипшини – з чорною порожнечею усередині.
Котить візочок з немовлям, міцно тримає за руку старшого трирічного сина.
Я знаю, що через пару секунд ми з нею зустрінемось в одній точці, а тоді
вона прямуватиме до літньої площадки кафе. І мені відчайдушно хочеться
поросити її: «Будь ласка, не йдіть туди»... І, порівнявшись із нею, я
опускаю очі – мені соромно перед цією удовою, якій лише двадцять два...

Я довго не могла дати назву своєму емоційному протесту проти таких гучних
веселощів, проти фоточок з екскурсій і розваг. Ну, дійсно, що такого?
Життя продовжується. Дехто виїхав в першу добу війни за кордон, хтось живе
в містах, де відносно спокійно. Для них війна – це пару вибухів лютневого
ранку і жахлива картинка в екрані телевізора чи смартфона. Так, є багато
українців, яким не пощастило, і вони сповна відчули на собі усі жахіття
війни. Але ж перші не винні у біді других. Невже їм стане легше від
колективного страждання? Та справа не в стражданнях. Це відчуття поваги до
горя тих, хто поруч.

...Пам'ять повертає мене на п'ять років назад, коли в спекотному червні я
сиділа за столиком такого ж кафе. Моя мама лежала у паліативному
відділенні, а я, зі своєї жагучої потреби, проводила біля неї дні, що
плавно перетікали в тижні.. потім в місяці... Одного разу до мене приїхала
подруга, щоб підтримати мене, і ми тихенько сиділи з нею за чашкою
м'ятного чаю у малесенькому кафе. Таких кафе там було  багато, адже
родичів, які бажали бути зі своїми помираючими близькими до останнього,
вистачало, і всім необхідно було щось їсти – хоча б один раз на добу. І от
поряд за столиком з'явилась гучна компанія - в одного із них був день
народження. Я думаю, що іменинник працював десь поруч, і в обідню перерву
друзі з тортом і пляшкою шампанського вирішили його привітати. В мене не
викликало це жодних претензій – кафе не на території лікарні, тож кожний
має право до нього завітати. Єдине, що було важко – ввічливо відмовитися
від запрошення приєднатися до святкування. Ювіляр, вочевидь, був людиною з
щедрою душею і не розумів, чому ми відмовляємось бодай від торту? Поки
йшли приготування, мені довелося пошепки сказати йому, що в мене помирає
мама. І настрій в мене зараз зовсім не святковий, тож ми просто продовжимо
пити свій чай. Потім в мене задзвонив телефон, я відійшла до вікна й
відволіклася на розмову. А коли повернулася, столик поряд з нами був
порожнім... Чоловік попросив мою подругу передати мені його співчуття, а сам
забрав своїх гостей до іншого кафе...

Через роки я розмірковую про почуття того дня. Чи було б зрадою, відносно
моєї мами, якби я з'їла той шматок торта? Та ні... Вона була під професійним
наглядом 24 години на добу і, бідна, переважно і не знала, що я весь час
поряд із нею. Чи могло чиєсь свято збільшити моє горе? Також, ні... Та втім,
річ була не в мені. Річ була в тому чоловікові, який висловив повагу до
моїх почуттів, моєї біди. Бачивши мене вперше в житті, він просто вважав
для себе неможливим радіти поряд з людиною, якій зараз дуже погано й
боляче...

А сьогодні боляче мільйонам людей. І я звертаюся до тої частини українців,
яка не розуміє цього і дистанціюється від їх почуттів. В мирні часи, коли
ставалася якась серйозна аварія, в країні оголошували жалобу, відміняли
гучні розваги та свята. Так от, у нас з вами тепер така ж жалоба, що,
нажаль, розтягнулася більш ніж на сотню днів... Та головне – до її
закінчення ще дуже далеко...

Колись, коли всі ми шукали щастя і сенс життя, коучі-психологи радили нам:
«яке б нещастя не сталося, шукай в ньому нові можливості для себе». Я
розумію, що дехто скористався цією порадою не зовсім етично. Чому б в
перші доби війни не виїхати з Західної України за кордон, не отримати
статус «біженця», а в цей час не здати за космічну ціну свою квартиру тим,
хто, дійсно, потребує цього - втік разом з дітьми від постійного
бомбардування? Чому б не скористатися можливістю продати гуманітарку або
військове знаряддя? Чому б безкоштовно не порозважатися, не подивитися
світ?

Хочеться правильного розуміння. Я ні в якому разі не закликаю весь час
плакати, «посипати голову попелом», не їсти смачного, не пити кави, не
розважати дітей, не купувати нових речей, не посміхатися, не зустрічатися
з друзями, не гуляти. Просто є певна межа людяності й пристойності.  Якщо
б у вашої подруги, яка, приміром, живе в Парижі, помер чоловік, або, не
дай Боже, дитина, і ви б приїхали до неї, щоб підтримати. Чи скористалися
б ви в цю поїздку нагодою, щоб помилуватися Парижем, зайнятися шопінгом,
відвідати музеї, Діснейленд та Мулен Руж? А щоб відчули ви, на місці цієї
подруги, якщо б хтось так вчинив з вами?

Можливо, хтось, хто продовжує жити, начебто нічого не сталося, просто не
замислюється, що його безневинна, на його погляд, дія може когось боляче
поранити. Тоді просто погляньте навколо – скільки людей зараз страждає.
Якщо ви голосно включаєте веселу музику в своєму авто, просто подумайте,
що хтось, проходячи повз, саме в цю хвилину може отримати звістку, що його
будинок в Чернігові зруйновано. Якщо ви постите фоточки із європейських
парків розваг, згадайте бучанських діточок, на очах яких спочатку
згвалтували, а потім вбили їх мам.  Якщо ви хизуєтеся своїми новими
прикрасами, згадайте дівчат в простих білих сукнях, які танцювали
прощальний шкільний вальс на тлі обгорілих руїн своєї школи в Харкові.
Якщо у вашу хвору (бо інакшого слова не підберу) уяву прийшла думка
запустити вночі феєрверк, подумайте, у скількох дітей і жінок, які дивом
вирвалися з Миколаєва, зараз розпочнеться істерика. Якщо ви, голосно
сміючись, їсте на вулиці шашлик під чарку холодної горілки, подумайте, що
повз ваш столик може проходити мати, чий єдиний син зараз воює на Сході,
або ж молода удова, чоловік якої назавжди залишився в Маріуполі...

Я не знаю, чи стане їм гірше й болючіше від чиєїсь необережної байдужості.
Втім, річ не в них. Річ в вас, і в вашій неповазі до людей, що поряд з
вами.

Ця клята війна розділила життя на «до» і «після» - хочемо ми цього, чи ні.
А ще вона стала справжнім лакмусовим папірцем героїзму, волі, чуйності,
людяності. В більшості випадків я пишаюся своєю нацією й тим, що я є
українкою. І мені, дійсно, боляче, що навкруги стільки горя й біди,
руйнувань і смерті. Але деколи мені стає соромно. І я, насправді, не
розумію, як зараз можна бути осторонь. Адже ті, хто вже постраждав – не є
винними в чомусь. Адже ті, кого ще не торкнулося – не є обраними. Проте,
якщо це не ваша війна і не ваш біль, то, мабуть, це вже і не ваша країна...

\ii{14_06_2022.fb.lonska_tetjana.1.ostoron.orig}
\ii{14_06_2022.fb.lonska_tetjana.1.ostoron.cmtx}
