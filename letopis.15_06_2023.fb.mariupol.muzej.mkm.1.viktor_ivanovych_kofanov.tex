%%beginhead 
 
%%file 15_06_2023.fb.mariupol.muzej.mkm.1.viktor_ivanovych_kofanov
%%parent 15_06_2023
 
%%url https://www.facebook.com/100093184796939/posts/pfbid02BjMBERiaUed2ZBnPhiABP7CdafvJLURZD5fx5ZmgEwb1mgrNxWjoJ4tcoGMp4GErl
 
%%author_id mariupol.muzej.mkm
%%date 15_06_2023
 
%%tags 
%%title Пам’яті художника - Віктор Іванович Кофанов
 
%%endhead 

\subsection{Пам'яті художника - Віктор Іванович Кофанов}
\label{sec:15_06_2023.fb.mariupol.muzej.mkm.1.viktor_ivanovych_kofanov}

\Purl{https://www.facebook.com/100093184796939/posts/pfbid02BjMBERiaUed2ZBnPhiABP7CdafvJLURZD5fx5ZmgEwb1mgrNxWjoJ4tcoGMp4GErl}
\ifcmt
 author_begin
   author_id mariupol.muzej.mkm
 author_end
\fi

Пам'яті художника

Два роки тому пішов з життя талановитий художник, світла, мудра та дуже щедра
людина – Віктор Іванович Кофанов.😪 Наприкінці життя у 2018, 2019 роках він
встиг провести у художньому музеї ім. А. І. Куїнджі свої персональні виставки
живопису та графіки і це стало знаковою  культурною подією в житті міста.✨️🎉🖼

Відвідувачі музею любили його творчість за щирість почуттів, високу
професійність, особливу естетику його творів. 🎨🥰

Віктор Іванович Кофанов був корінним маріупольцем, народився у нашому місті у
1941 році. Дуже непростими були його дитячі роки, але бажання малювати
прокинулось в ньому досить рано. 👏💗

З 70-х років XX ст. починається творча діяльність митця. Йому вдається
опанувати складні графічні техніки – гравюру на картоні, ліногравюру, офорт.
Він був майже єдиним у Маріуполі  майстром чорно-білого і кольорового офорту.

❗️👏І у цей же час його графічні роботи експонуються на виставках, їх
відзначають відомі мистецтвознавці. До речі, одна з графічних робіт В.І.
Кофанова \enquote{Вулиця Куїнджі} (гравюра на картоні)  всі роки існування художнього
музею експонувалась в одному з його залів. 🖼😊

З 90-х років  XX ст.  Кофанов, залишаючись уже відомим у країні графіком,
починає опановувати і живопис. Він міг побачити красу у досить непримітних
речах – стежинці у степу, колодязі на подвір'ї, старому будинку. Але все це
було відображенням краси душі митця, тому так вражало глядача. Сам майстер про
себе казав: \enquote{Живу і пишу у згоді з природою і собою}.❗️✨️✅️

Віктор Іванович Кофанов був людиною, яка щиро, без усяких компромісів, любила
свою землю, свою Батьківщину і це також  знайшло відображення у його
творчості.🏞

Художник пішов з життя у червні 2021 року, за півроку до початку цієї страшної
війни. Велика частина його спадщини – 98 робіт зберігалась у колекції музею. 

Зараз дві картини Віктора Івановича Кофанова експонуються на виставці
\enquote{Нескорений Маріуполь}, яка мандрує світом з серпня 2022 року, розповідаючи
мовою образотворчого мистецтва про Маріуполь та його трагедію. ❗️І це дуже
символічно.

\#mkmmariupol \#mariupol \#маріупольськиймузей \#маріуполь

\#відродимомаріупольськиймузей \#працюєморазом \#художник
