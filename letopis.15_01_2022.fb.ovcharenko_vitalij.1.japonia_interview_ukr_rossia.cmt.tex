% vim: keymap=russian-jcukenwin
%%beginhead 
 
%%file 15_01_2022.fb.ovcharenko_vitalij.1.japonia_interview_ukr_rossia.cmt
%%parent 15_01_2022.fb.ovcharenko_vitalij.1.japonia_interview_ukr_rossia
 
%%url 
 
%%author_id 
%%date 
 
%%tags 
%%title 
 
%%endhead 
\zzSecCmt

\begin{itemize} % {
\iusr{Іван Синєпалов}
Ну ти ж сказав, що ми готові за свою честь, не вмирати, а вбивати?)

\iusr{Vitalii Ovcharenko}
\textbf{Іван Синєпалов} я сказав що Росія і головне самі росіяни дорого заплатять за свою чергову атаку.

\iusr{Roman Eridan}
То ти вже медійна особа, слухай )

\iusr{Vitalii Ovcharenko}
\textbf{Roman Eridan} та куди там.

\iusr{Mykola Mykolaiovych Dolynskyi}
We are building the perfect weapon to deliver Emperor's will

\ifcmt
  ig https://i2.paste.pics/6dc7a26764a59be87145bbdd20c9c4b9.png
  @width 0.2
\fi

\begin{itemize} % {
\iusr{Vitalii Ovcharenko}
\textbf{Mykola Mykolaiovych Dolynskyi} найкраща зброя це самурайський дух!

\iusr{Mykola Mykolaiovych Dolynskyi}
\textbf{Vitalii Ovcharenko} це мав бути мій другий коментар, але підступний Овчаренко-сан зіграв на випередження

\iusr{Vitalii Ovcharenko}
\textbf{Mykola Mykolaiovych Dolynskyi} це мудрість столітньої сакури...

\iusr{Mykola Mykolaiovych Dolynskyi}
\textbf{Vitalii Ovcharenko} може ти вже бачив, але принаймні лайка твого там немає
\url{https://www.facebook.com/2170688466513405/posts/3044942879087955/}
\end{itemize} % }

\iusr{Євгенія Чуприна}
Ми воюємо саме за те, щоби нас не вбивали

\iusr{Yasa Sirko}

Норм світосприйняття. В мене таке саме. Усе життя його реалізовую ))) Чесно
мені погуй на євротолерантні цінності з їхньою моделлю демократії, гендером -
фемінізмом та лгбт )))) Я за Українську Україну та наші традиції та цінності.

\begin{itemize} % {
\iusr{Дмитрий Карней}
\textbf{Yasa Sirko} зазвичай толерантність европейців найбільш хвилює всяких латентних представників меньшин. Може пан щось в собі приховує?  @igg{fbicon.wink}  Може до псіхоаналітика?

\iusr{Vitalii Ovcharenko}
\textbf{Yasa Sirko} дивно всі ж інші за гендер воюють!

\iusr{Yasa Sirko}
\textbf{Дмитрий Карней} написано що хвилює ? Написано - погуй . Зовсім протилежне
чи не так ))) Так що не збуджуйся а краще вивчай методичку )))

\iusr{Дмитрий Карней}
\textbf{Yasa Sirko} якщо людині щось "погуй" то вона про те не згадує і не пише. Але вас вочевидь то бентежить і у вас то болить. Тому не ігноруйте мою пораду.

\iusr{Yasa Sirko}
\textbf{Дмитрий Карней} 

якщо мені \enquote{погуй} то я й пишу \enquote{погуй} якщо мене щось \enquote{бентежить} чи
\enquote{болить} я так й пишу \enquote{бентежить} чи \enquote{болить} ))) Якщо у пана його \enquote{Гугл}
підсвідомості до вирваних з контексту слів видає таку \enquote{гуйню} то пану дійсно
потрібно до спеціаліста. Хоча можливо для пана це є норма )))

\iusr{Дмитрий Карней}
\textbf{Yasa Sirko} ага. Хто всрався? Невістка... (с)
\end{itemize} % }

\iusr{Віктор Дихановський}

Японці не стали воювати за свою честь і гідність у 1945 році. І в 1993 році не
забрали свої північні території, коли в Московії був безлад.

\begin{itemize} % {
\iusr{Vitalii Ovcharenko}
\textbf{Віктор Дихановський} у 1993 році уе було б розв’язування війни. А у 45 ще одна ядерна бомба.

\iusr{Родомир Степной}
\textbf{Віктор Дихановський} Ага) у самогубці стояли черги добровольців, але японці не хотіли воювати)))
\end{itemize} % }

\iusr{Valentyn Krasnoperov}
Треба було гандама попросити аби воно парочку прислали нам на фронт

\begin{itemize} % {
\iusr{Vitalii Ovcharenko}
\textbf{Valentyn Krasnoperov} кого?)

\iusr{Valentyn Krasnoperov}
\textbf{Vitalii Ovcharenko} то японський робот в якого залізаєш і мочиш ворогів. З аніме

\iusr{Vitalii Ovcharenko}
\textbf{Valentyn Krasnoperov} ми його на російські позиції і камікадзе!
\end{itemize} % }

\iusr{Родомир Степной}

Коли вже були скинуті ядерні бомби та імператор записав звернення до народу,
знайшлися заколотники, які увірвалися до його палацу. Вони хотіли зірвати
заплановане звернення і продовжувати воювати до кінця!

\begin{itemize} % {
\iusr{Vitalii Ovcharenko}
\textbf{Родомир Степной} справжній японський дух!

\iusr{Родомир Степной}
\textbf{Vitalii Ovcharenko} небезпечні люди)
\end{itemize} % }

\iusr{Vitaliy Kolomiets}
Писав якось про честь дещо в юридичному аспекті. Може тобі пригодиться

\href{https://lb.ua/blog/vitalii_kolomiets/441901_pro_chest_storozhu_simvoliv.html}{%
Про честь та сторожу символів, Віталій Коломієць, lb.ua, 11.11.2019%
}

\raggedcolumns
\begin{multicols}{2} % {
\setlength{\parindent}{0pt}

\ifcmt
  ig https://i.lb.ua/114/45/5db6c8149c9f3.jpeg
\fi

Останні кілька місяців були повідомлення про зняття українських прапорів у
Станиці Луганській, інших містах та навіть з військової техніки.

Якби старомодно це не виглядало, але війна йде саме за символи. Символи - це
знаки певних цінностей. Цінності для багатьох людей складають основу їх
ідентичності. Власне, віра у різні цінності робить біологічно ідентичних гомо
сапієнсів - різними.

Один знак може нести величезний пласт інформації, коли ми бачимо знак Mercedes,
Toyota чи Tesla ми розуміємо про що це. Це \enquote{щось} є результатом життя і праці
тисяч людей.

Людина, яка приймає певні цінності, які визначатимуть її цілі і вчинки, також
приймає символи цих цінностей. Хрест для християн це ж символ, який концентрує
тисячі і тисячі інструкцій життя.

Людина, представляючи себе, не каже, що вона 80-ти кілограмова водно-білкова
емульсія з когнітивними здібностями. Взагалі, відповідь на питання \enquote{хто ти} є
відповіддю \enquote{про цінності}: мама, лікар, правоохоронець, священник, інженер,
військовий, журналіст, українець.

Носій певних цінностей може їх проявляти через інші символи, наприклад, одяг: військовий, суддя, лікар, монах.

Здатність бути вірним своїм цінностям - є честю. Коли людина чесна в
сповідуванні своїх цінностей це проявляється у всіх її вчинках, тоді така
людина має честь. Честь офіцера, лікаря, священника.

Але коли декларуються одні цінності, а сповідуються інші - це відсутність честі
(сором, ганьба).

Той, хто має честь, з повагою носить символи тих цінностей, які сповідує.
Шеврон з українським прапором - це символ такої цінності як \enquote{свобода}. Власне,
наш прапор є символом двох важливих цінностей: \enquote{хліб} - результат власної праці
та \enquote{небо} - символ свободи. Їх розміщення - ще один символ того, що \enquote{Свобода
вище Хліба}. Це незрозумілі цінності для росіян. Їм чужа \enquote{воля} робити свою
справу в \enquote{щирій праці}, вони ніколи не \enquote{положать тіло за свободу}.

Тому ми чужі, тому й воюємо кожен з вірою в свої цінності.

Наше \enquote{слава} дуже близьке до поняття \enquote{честь}.

Ми дуже мало приділяємо уваги такій моральній категорії як \enquote{честь}, хоча
юридично вона не менш значуща як життя і здоров'я.

Багато, хто згадує про життя і здоров'я людини як найвищу соціальну цінність.
Хоча в ст. 3 Конституції на рівні з іншим згадана Честь: "людина, її життя і
здоров'я, честь і гідність, недоторканність і безпека визнаються в Україні
найвищою соціальною цінністю".

Інша стаття, яка зобов'язує дотримуватися законів і забороняє посягати на права
людини також забороняє посягати на честь: "кожен зобов'язаний неухильно
додержуватися Конституції України та законів України, не посягати на права і
свободи, честь і гідність інших людей" (ст.68).


 
Нація - концентрує свої цінності в своїх символах. Нерозуміння свого прапору, гімну, чи гербу є нерозумінням цінностей, нерозуміння цінностей - є дезорієнтацію, а дезорієнтований - це 60-80-100 кілограмова водно-білкова емульсія.

Шанування символів є обов'язком кожного громадянина України (ст. 65 Конституції). Шанувати символи - означає також сповідувати цінності, які закладені в ці символи.

Посягання на символи - це посягання на честь тих, хто сповідує цінності, зашифровані в ці символи.

Честь військового - це також бути добрим сторожем наших спільних символів. Не давайте знімати наші символи з нашої землі.

\end{multicols} % }




\end{itemize} % }
