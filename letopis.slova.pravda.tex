% vim: keymap=russian-jcukenwin
%%beginhead 
 
%%file slova.pravda
%%parent slova
 
%%url 
 
%%author 
%%author_id 
%%author_url 
 
%%tags 
%%title 
 
%%endhead 
\chapter{Правда}

Який тільки народ не адаптував сюжет стародавньої казки про відомих українцям
братів \emph{Правду} і Кривду. Проте найкраще, здається, це вдалося зробити в
англосаксонському світі, де цю моралізаторську казку назвали «Засліплення
правди фальшивістю». І тут йдеться не тільки про маніпуляції та перебріхування.
Мова йде також про тих, хто облудою і фальшем забирає в інших здатність бачити,
чути і розуміти. Хоча у відомій казці є безліч незрозумілих для сучасної людини
поведінкових практик, а сюжет містить багато спрощень, основна мораль зрозуміла
– \emph{неправдою} щасливим не станеш. Але цього разу нам важливо зосередитися на
тому, що одному з братів за допомогою цинічних маніпуляцій вдалося виманити у
брата його коржа. Ще цікавішим є те, чому брат \emph{Правда} погодився на те, щоб його
скалічили до напівживого стану,
\textbf{Правда і Кривда}, Василь Расевич, zaxid.net, 04.06.2021

