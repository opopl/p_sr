% vim: keymap=russian-jcukenwin
%%beginhead 
 
%%file slova.pravilo
%%parent slova
 
%%url 
 
%%author 
%%author_id 
%%author_url 
 
%%tags 
%%title 
 
%%endhead 
\chapter{Правило}
\label{sec:slova.pravilo}

%%%cit
%%%cit_head
%%%cit_pic
%%%cit_text
Сергей Фурса написал длинный пост о том, что у украинцев не нужно ничего
спрашивать, потому что они ни в чем не разбираются. Это очень понятный и
рукопожатный пост – не спрашивать же, в самом деле, мнения людей, которые живут
в \emph{неправильных} регионах, разговаривают на \emph{неправильном} языке, голосуют за
\emph{неправильных} политиков, празднуют \emph{неправильные} праздники и имеют \emph{неправильных}
предков?  Но вслух об этом можно было и не говорить. Тем более что большинство
«реформ» в стране и так вполне официально предусматривают передачу прав и
полномочий международным структурам, которые «лучше знают как надо»
%%%cit_comment
%%%cit_title
\citTitle{В этой Украине живут какие-то неправильные украинцы}, Вячеслав Чечило, strana.ua, 07.07.2021
%%%endcit

%%%cit
%%%cit_head
%%%cit_pic
%%%cit_text
Вот я здесь пожила почти два месяца и поняла, что \emph{правильно} российская
пропаганда запугивает людей Украиной, \emph{правильно} им впаривает, что здесь
пьют кровь младенцев и всякую другую, \emph{правильно} им навешивает жуткой
нацистско-бандеровской лапши на их доверчивые уши, \emph{правильно} их доводит
до того состояния, что они реально боятся сюда ехать (я сейчас исключаю всякого
рода вирусную составляющую и тот естественный факт, что их в принципе сюда
никто особо и не приглашает).  Они боятся, потому что их научили бояться. И
ненавидеть. И \emph{правильно}. Потому что если россияне опять станут ездить в
Украину так же массово, как и раньше, они прямо носом почуют, насколько же
здесь свободнее
%%%cit_comment
%%%cit_title
\citTitle{Два месяца живу в Украине и теперь понимаю, почему россиянам навешивают лапшу}, 
Елена Рыковцева, news.obozrevatel.com, 05.07.2021
%%%endcit

%%%cit
%%%cit_head
%%%cit_pic
%%%cit_text
А почему бы и нет? На Украине работает \emph{правило}, что следующий президент
обязательно должен быть хуже предыдущего, а превзойти в этом Зеленского - это
очень сложная, почти неразрешимая задача. Лёша по степени своей
беспринципности, умению предать и обмануть переплюнет даже Вову, я в него
верю). Уровень... эээ... неумности, назовём это так, у Гончаренко тоже высокий.
Кто не видел, посмотрите его интервью, где он объясняет возникновение его
незадекларированных капиталов. Так, по его словам, 700 тыс. долларов Лёше
одолжил его тесть, такие деньги отец жены скопил, работая водителем в Газпроме
еще в советское время...
%%%cit_comment
%%%cit_title
\citTitle{Леша Гончаренко - будущий президент Украины?}, Мак Сим, zen.yandex.ru, 24.10.2021
%%%endcit

%%%cit
%%%cit_head
%%%cit_pic
%%%cit_text
Сегодня в Украине - дата "Ч" для невакцинированных педагогов.  Именно с 8 ноября
их, согласно приказу Минздрава, их должны отстранять от работы без сохранения
зарплаты.  Как предупредил накануне министр здравоохранения Виктор Ляшко,
\emph{правило} будет соблюдаться. На вопрос в телеэфире, значит ли, что
невакцинированные учителя с понедельника уже могут не выходить на работу,
министр ответил: "Да, согласно действующему законодательству, они будут
отстранены". В профсоюзе педагогов, который ранее просил власти не допустить
массовых отстранений учителей без сохранения зарплаты, сегодня собрали срочное
совещание по этому поводу - решают, что делать дальше. Ранее профсоюзы просили
Минздрав отсрочить новые требования на месяц, но там им на встречу не пошли.
Федерация профсоюзов Украины уже заявила о незаконности отстранений.  Но, как
показал наш опрос, далеко не во всех образовательных учреждениях тут же
кинулись отстранять учителей
%%%cit_comment
%%%cit_title
\citTitle{"За отстранение людей придется отвечать". Как с 8 ноября учителей начали лишать работы и зарплат}, 
Анастасия Товт; Людмила Ксенз; Александра Харченко, strana.news, 08.11.2021
%%%endcit
