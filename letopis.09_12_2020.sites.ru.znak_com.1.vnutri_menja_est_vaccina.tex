% vim: keymap=russian-jcukenwin
%%beginhead 
 
%%file 09_12_2020.sites.ru.znak_com.1.vnutri_menja_est_vaccina
%%parent 09_12_2020
 
%%url https://www.znak.com/2020-12-09/rabotaet_li_rossiyskaya_vakcina_ot_koronavirusa_sputnik_v_i_kak_prohodyat_ee_ispytaniya
 
%%author 
%%author_id 
%%author_url 
 
%%tags vaccine,sputnik_v,covid
%%title «Внутри меня есть вакцина»
 
%%endhead 
 
\subsection{«Внутри меня есть вакцина»}
\label{sec:09_12_2020.sites.ru.znak_com.1.vnutri_menja_est_vaccina}
\Purl{https://www.znak.com/2020-12-09/rabotaet_li_rossiyskaya_vakcina_ot_koronavirusa_sputnik_v_i_kak_prohodyat_ee_ispytaniya}

\begin{leftbar}
  \begingroup
    \em
Работает ли российская вакцина от коронавируса «Спутник V» и как проходят ее
        испытания
  \endgroup
\end{leftbar}

Как себя чувствует человек после российской вакцины против коронавируса
«Спутник V»? Что нужно предусмотреть, если вы решили поставить прививку? Почему
российские власти слабо продвигают «Спутник V», а регионы пока не получают
крупных партий вакцины, несмотря на объявленную с 5 декабря масштабную
вакцинацию? Своим мнением по этому поводу делится политолог Кирилл Шулика,
который стал добровольцем на третьем этапе испытаний.

\ifcmt
  %reload 1
  pic https://img.znak.com/967690.jpg
  width 0.5
\fi

«Науке я доверяю больше, чем людям, которые не считают нужным надевать маску»

В Москве испытания вакцины проходят в нескольких поликлиниках. Можно записаться
на сайте, но я слышал отзывы, что оставившим там свои данные не перезванивали.
Поэтому просто выбрал другой путь — пришел в одну из этих поликлиник с
паспортом. У меня даже не было полиса ОМС, там мне его сделали.

Конечно, прежде чем решиться на вакцинацию «Спутником V», я оценивал риски.
Главное, что я понимал — прививаться все равно придется, так как по сути все
человечество пришло к выводу — это единственный способ победить коронавирус.

С точки зрения запуска в гражданский оборот, наша вакцина идет вровень с
китайской Sinopharm, оксфордской AstraZeneca, а также разработками Pfizer и
Moderna. «Спутник V» сделан на единой аденовирусной платформе с китайской и
оксфордской вакцинами. Отличие есть у оксфордской — они используют аденовирус
не человека, а шимпанзе. Мои знакомые врачи отмечают: аденовирусная платформа
появилась не вчера и аденовирусы нам всем знакомы, они, например, могут
провоцировать простуду.

\begin{leftbar}
\begingroup
\em\large\bfseries
Годами ждать появления импортной вакцины не хотелось. Кроме того, давайте
        честно скажем: все существующие прививки сделаны на коленке, поэтому
        риск есть при использовании любой из них.
\endgroup
\end{leftbar}

При этом любая социальная активность сталкивает тебя с людьми, которые не
считают нужным держать дистанцию, носить маску или даже прикрывать рот, когда
кашляют на весь вагон метро. И это очень сильно повышает вероятность заразиться
коронавирусом.

С другой стороны, есть наука, российские ученые. Может быть, не самые передовые
в мире, но в то же время нет никаких сомнений, что они ставили перед собой
задачу спасти нас с вами от болезни. Да, у них могло не получиться, но науке я
доверяю больше, чем людям, которые не считают нужным надевать маску.

\subsubsection{«Если нет противопоказаний, хуже точно не будет»}

И все же сначала я планировал прививаться только после окончания третьего этапа
испытаний, чтобы посмотреть, что вообще получится при более массовом
использовании вакцины. Но осенняя волна заставила ускориться.

Дополнительным толчком послужила информация от различных представителей
медиатусовки, которые пошли добровольцами на испытание, неплохо себя
чувствовали и получили антитела. Это достаточно известные люди, и, если бы
что-то пошло не так, об этом узнала бы вся страна. Разработчик же приветствовал
появление таких добровольцев, будучи уверенным в препарате.

%\ifcmt
  %pic https://img.znak.com/967687.jpg
  %caption Вакцина хранится в специальных холодильниках при температуре –18 градусов, Фото - Cover Images / Keystone Press Agency / Global Look Press
  %width 0.5
  %fig_env wrapfigure
%\fi

Также я посоветовался с тремя врачами: академиком, который возглавлял
крупнейший медицинский центр в стране, двумя кандидатами наук и доктором
коронавирусного стационара.

Общее мнение докторов — если нет противопоказаний, хуже точно не будет,
безопасность вакцины доказана.

Еще одним аргументом стала фраза одного из врачей, что пока речь идет о 30 тыс.
доз вакцины, сделанных разработчиком для испытаний, их качество не вызывает
сомнений. А вот при масштабном производстве могут быть разного рода сбои,
особенно если все это будет делаться со спринтерской скоростью. Замечу, что, к
счастью, пока спринтерской скорости не видно из-за проблем не только с
производством, но и с логистикой.
