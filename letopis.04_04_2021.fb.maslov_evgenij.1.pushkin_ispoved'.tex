% vim: keymap=russian-jcukenwin
%%beginhead 
 
%%file 04_04_2021.fb.maslov_evgenij.1.pushkin_ispoved'
%%parent 04_04_2021
 
%%url https://www.facebook.com/maslovevgeniy14/posts/1451059598569077
 
%%author 
%%author_id 
%%author_url 
 
%%tags 
%%title 
 
%%endhead 

\subsection{А.С. Пушкин \enquote{Исповедь}}
\url{https://www.facebook.com/maslovevgeniy14/posts/1451059598569077}

Вечерня отошла давно,
Но в кельях тихо и темно.
Уже и сам игумен строгий
Свои молитвы прекратил
И кости ветхие склонил,
Перекрестясь, на одр убогий.
Кругом и сон и тишина,
Но церкви дверь отворена;
Трепещет . . . луч лампады,
И тускло озаряет он
И темну живопись икон,
И позлащенные оклады.
И раздается в тишине
То тяжкий вздох, то шепот важный,
И мрачно дремлет в вышине
Старинный свод, глухой и влажный.
Стоят за клиросом чернец
И грешник — неподвижны оба —
И шепот их, как глас из гроба,
И грешник бледен, как мертвец.
Монах
Несчастный — полно, перестань,
Ужасна исповедь злодея!
Заплачена тобою дань
Тому, кто, в злобе пламенея,
Лукаво грешника блюдет
И к вечной гибели ведет.
Смирись! опомнись! время, время,
Раскаянья . . . . покров
Я разрешу тебя — грехов
Сложи мучительное бремя.
А.С. Пушкин "Исповедь"
1823 г.


\ifcmt
  pic https://scontent-amt2-1.xx.fbcdn.net/v/t1.6435-9/168584938_1451059391902431_7065257123039373396_n.jpg?_nc_cat=101&ccb=1-3&_nc_sid=8bfeb9&_nc_ohc=DoeqFjGFDH4AX8s1WnQ&_nc_ht=scontent-amt2-1.xx&oh=a71bb09eadbca93fd1abeba58e121ecf&oe=6090FBAE

	pic https://scontent-amt2-1.xx.fbcdn.net/v/t1.6435-9/169052362_1451059475235756_2487110747950697268_n.jpg?_nc_cat=106&ccb=1-3&_nc_sid=8bfeb9&_nc_ohc=blF7cjBOngoAX-EcP60&_nc_ht=scontent-amt2-1.xx&oh=9243e2bd74b243527d386bc905ed607d&oe=608FA891

	pic https://scontent-amt2-1.xx.fbcdn.net/v/t1.6435-9/169391502_1451059575235746_6921675772721485630_n.jpg?_nc_cat=105&ccb=1-3&_nc_sid=8bfeb9&_nc_ohc=XwpcpgmqNqgAX-_tv6S&_nc_ht=scontent-amt2-1.xx&oh=1be8dc4dd5452ab06296369afbacd010&oe=608DE3C5
	
\fi

