% vim: keymap=russian-jcukenwin
%%beginhead 
 
%%file 04_12_2021.fb.tolkachev_aleksej.1.mria_pro_ukrainu.cmt
%%parent 04_12_2021.fb.tolkachev_aleksej.1.mria_pro_ukrainu
 
%%url 
 
%%author_id 
%%date 
 
%%tags 
%%title 
 
%%endhead 
\subsubsection{Коментарі}

\begin{itemize} % {
\iusr{Іван Киричевський}

Ну, можливо, мрія про Місто майбутнього в Орбіті ще зможе втілитись, якщо
зважити на те, що Енергоатом в тій місцевості уже запланував свої проекти).

\iusr{Олексій Толкачов}
\textbf{Іван Киричевський} Краще б вже без Енергоатому...  @igg{fbicon.smile} 

\iusr{Ігор Орел}
моя дочка ще писала тоді про омріяну Україну

\begin{itemize} % {
\iusr{Олексій Толкачов}
\textbf{Ігор Орел} Так, до нас на корнкурс письмових робіт?

\iusr{Ігор Орел}
до вас же...

\iusr{Олексій Толкачов}
Дякую, що нагадали про цей проект! Якраз у ці дні - 10 років вже як минуло!
\end{itemize} % }

\iusr{Tisha Lad}

Иметь мечту это подвиг, реализовывать мечту это геройство

Согласовать мечты умеющих мечтать это невозможно но самая необходимая задача в
такое удивительно катастрофическое время.

Вовлечь в мечту чтобы совместно из мечты играть это ВЕЛИКОЕ ЧУДО!

Интеллектуалы не нужны невеждам и ханжам с более выраженными мотивами сеничек (
чтоб не написать баранам  @igg{fbicon.face.smiling.eyes.smiling} ) прошлому

Интеллектуалы нужны друг другу, будущему, человечности

Я думаю если бы 13 лет назад когда 90-95\% загоревшихся вашей мечтой имели
понятную технологию согласования и долевого участия в реализации замысла то
чудо бы свершилось...

@igg{fbicon.face.smiling.eyes.smiling}  с радостью готова согласовать с вами
мечты и увидеть как с помощью социально архитектурной технологии цивилизации
3.0

Так сказать невозможного дракона, выйти в синергетическое взаимодействие..

У меня по ощущениям что в этот раз тот самый час  @igg{fbicon.face.smiling.eyes.smiling} 

Для всплеска волны Кайроса  @igg{fbicon.face.smiling.eyes.smiling} 

\iusr{Любов Мандзюк}

Чудові мрії, але, нажаль, названі і причини їх невтілення. То ж, бажаю скорішої
реалізації задуманого, бо і я з Вами у прекрасному Українському майбутньому
хочу пожити!

\iusr{Эн Меркар}

Мрії є у кожного. Кожен українець знає \enquote{як треба}. це можна назвати мріями, або
їх зародками. Ми маємо розробити таку систему відносин у суспільстві, де такі
\enquote{як треба} нас, українців, будут реалізовуватися. Адже мрії, часто виростають з
болю, з невдоволення існуючим порядком. Боротьба з болем - це онтологічна
частина людського буття.

\begin{itemize} % {
\iusr{Эн Меркар}

Двіжок має стати такою систмою відносин. Якщо потрібна допомога в реалізації,
буду радий допомогти. Допоки у нас в Україні не з'явиться Громадський
кікстартер, у нас не сформується ні економічний, ні соціальний капітал для
модернізації владної системи в державі. Нові еліти саме в цьому напрямку мають
рухатись. І швидко. Суто мої думки.

\iusr{Эн Меркар}
а ще вчених треба вербувати)
\end{itemize} % }

\iusr{Oleg Fedoroff}

Требуется исход? Или какая то иная х-ня? Ну нет у нас политической воли и
национального интереса! Предлагаю строить Африку. Деньги есть. Нужны реальные
проекты, предприятия, люди.

\iusr{Mike Kaufman-Portnikov}

А мрія має бути одна на всіх? Це, коли лихо... або класна @igg{fbicon.biceps.flexed} ідеологічна машина.
А, коли добре, то й мрії мабуть різні ... Чи помиляюсь?

\iusr{Igor Matveev}

Слова важко заходять людям. А слова складені в складні речення - ще гірше.
Спробуйте просувати свої ідеї через соціальні ролики з гарним візуалом. Відео,
де сидять і говорять люди теж не дійде до широкого загалу. Я можу допомогти з
реалізацією таких роликів.

\end{itemize} % }
