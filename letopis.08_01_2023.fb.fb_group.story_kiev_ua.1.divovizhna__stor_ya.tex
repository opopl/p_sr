%%beginhead 
 
%%file 08_01_2023.fb.fb_group.story_kiev_ua.1.divovizhna__stor_ya
%%parent 08_01_2023
 
%%url https://www.facebook.com/groups/story.kiev.ua/posts/2111264809070309
 
%%author_id fb_group.story_kiev_ua,purik_oksana
%%date 08_01_2023
 
%%tags 
%%title ДИВОВИЖНА ІСТОРІЯ
 
%%endhead 

\subsection{ДИВОВИЖНА ІСТОРІЯ}
\label{sec:08_01_2023.fb.fb_group.story_kiev_ua.1.divovizhna__stor_ya}
 
\Purl{https://www.facebook.com/groups/story.kiev.ua/posts/2111264809070309}
\ifcmt
 author_begin
   author_id fb_group.story_kiev_ua,purik_oksana
 author_end
\fi

ДИВОВИЖНА ІСТОРІЯ ♥️

Влітку я потрапила до квартири, де зупинився час... Речі 60-70-80-х років –
тобто не деякі речі, а всі! Історія доволі цікава)

Квартира належить відомій особистості. Чоловік виріс в цій квартирі з батьками,
старшою сестрою, яка померла в молодому віці та бабу – вона щотижня писала
листи своєму чоловіку, який загинув на фронті. Власник квартири викупив її
кілька років тому, після того, як її продали батьки ще в 90-х та виїхали на пмж
до Ізраїлю.

Усі речі не просто з минулих часів - вони такі самі, як були тоді. Купляли їх
де завгодно – на барахолках, в інтернеті. Меблі, посуд, іграшки, пральна
машина, пилосос... 

Своє життя в цій трикімнатній квартирі чоловік описав у книгах – саме вони
врятували мене в перші дні війни. Вибухи, безробіття, двоє лежачих хворих
вдома, страх, невизначенність – я слухала ці аудіокниги і навіть уявити не
могла, що вже за кілька місяців потраплю сюди!

Можливо по старим фото сім'ї ви здогадаєтесь кому належить ця квартира? 😉
