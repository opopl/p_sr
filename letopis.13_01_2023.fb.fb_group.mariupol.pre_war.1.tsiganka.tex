%%beginhead 
 
%%file 13_01_2023.fb.fb_group.mariupol.pre_war.1.tsiganka
%%parent 13_01_2023
 
%%url https://www.facebook.com/groups/1233789547361300/posts/1388462605227326
 
%%author_id fb_group.mariupol.pre_war,kipcharskij_viktor.mariupol
%%date 13_01_2023
 
%%tags mariupol
%%title Циганка
 
%%endhead 

\subsection{Циганка}
\label{sec:13_01_2023.fb.fb_group.mariupol.pre_war.1.tsiganka}
 
\Purl{https://www.facebook.com/groups/1233789547361300/posts/1388462605227326}
\ifcmt
 author_begin
   author_id fb_group.mariupol.pre_war,kipcharskij_viktor.mariupol
 author_end
\fi

Циганка

Якось чудового весняного ранку 1979-го року Студент їхав  тролейбусом 12-го
маршруту до інституту. 

Або, як зазвичай казали мешканці Іллічівського району - до Міста. Наприкінці
XIX століття іноземні промисловці знайшли дуже привабливою ідею вибудувати там,
де потоки вугілля із Донбасу будуть зустрічатися з потоками залізної руди із
Керченського півострова - в гирлі річки Кальміус – металургійний завод. Але
бургомістр Маріуполя, напевно, щось знав про \enquote{розу вітрів} і тому вперто стояв
на своєму: тільки за п'ять верст на північ від міста! І нічого товариство
Нікополь змінити не змогло: у 1886-му році обладнання куплених у Сіетлі
(Сполучені Штати) металургійного і трубного заводів розібрали, навантажили на
пароплави, привезли через моря-океани в Маріуполь, вивантажили і почали
встановлювати заводи серед степу і далеко від води. В пошуках роботи на
будівництво потягнулися селяни і навколо будівництва з'явилися халабуди,
землянки, курені в яких оселилися спочатку будівельники, а потім і робітники
заводу. Разом з тим Провіданс почав будувати для своїх робітників бараки з
\enquote{дикого каменю} - вапняку. В середині довгих одноповерхових споруд майже
впритул стояли дерев'яні нари, на яких після роботи спали робітники. З часом
люди одружувалися, заводили дітей і створювали родинні кубельця,
відгороджуючись занавісками з грубої тканини \enquote{особистий простір}. Пізніше над
бараками вибудували другий поверх з червоної цегли. Таке житло називали
\enquote{колонами} від \enquote{колонія} - поселення. Отак деякий час і існували окремо Місто і
Завод, розділені подекуди полями й садами, а то й просто степом по обидва боки
річки Кальчик. З часом Завод і Місто потягнулися одне до одного приватними
будиночками і, в врешті-решт, зустрілися і зрослися. У 1830-мц році італієць С.
Гальяно майже за містом збудував склади збіжжя, паровий млин та макаронну
фабрику, так от з часом вони опинилися практично в центрі міста...

Тролейбус жваво пробіг проспектом Металургів повз багатоповерхівки до Чонки
(П'ятий мікрорайон), скотився до річки, переїхав міст і від "Малої землі" почав
підійматися угору. Після хлібозаводу він почав рухатися повільніше і врешті
зупинився нижче тої самої макаронної фабрики. Студент, стоячи з лівого боку
салону, бачив у вікно приватні будиночки. Між ними і дорогою на яскравій
молодій яскраво-зеленій траві стояло кілька циганських возів з полотняними
халабудами - кибиток. Голоблі возів були задрані у гору і навіть трохи назад і
на них був розвішений якійсь одяг. Випряжені кони зі стриноженими передніми
ногами паслися між кибитками. 

З одної з халабуд визирнула молода циганка у сорочці, потерла руками очі,
позіхнула, потягнулася, а потім зняла з голоблі щось, що виявилося спідницею і
одягнула через голову. Потім вона стала на возі попереду халабуди і почала одна
за одною знімати з голоблі і одягати на себе спідниці. Кажуть, що коли помирає
циган, то його дружині залишається лише те, на ній - одяг, прикраси. Ось чому
вони носять на собі багато сережок, кілець, намист, спідниць...

Правда це чи ні, але спідниць було багато. Та навряд чи хто з глядачів їх
рахував: циганка стояла навпроти ранішнього сонці, проміні якого безсоромно
просвічували її тоненьку сорочку і вимальовували усі вигини молодого тіла над
спідницями...

Напевно, вона знала, що її розглядають з вікон машин і тролейбусів: хтось із
водіїв навіть посигналив, але вона не звернула на це уваги і продовжувала брати
спідниці, одягати через голову, щось застьобувати чи зав’язувати на поясі,
закидати назад голову, відкидаючи хвилю чорного волосся, піднімати руки, аби
його поправити його...

Нарешті машини зрушили і циганка з її міні спектаклем залишилися позаду, в
минулому…

Наступного ранку було похмуро і сіяв дрібненький дощик - може через це, а може
отримавши від чоловіка чи ще когось прочухана, циганка не вийшла \enquote{на
біс}.  Потім були вихідні, а по них кибитки зникли... І лише за півстоліття
з’явились у спогадах...

На фото будинок бургомістра.

Фото не моє.

Елена Маріупольська зізналася, що фото зробила вона.

П.С. Дякую Александру Ганичу за цікавий історичний матеріал і роз'яснення
\enquote{колона-колонія}.
