% vim: keymap=russian-jcukenwin
%%beginhead 
 
%%file 19_11_2020.news.ru.vz.prostakov_gleb.1.mjagko_vozvraschat_ukrainu
%%parent 19_11_2020
 
%%url https://vz.ru/opinions/2020/11/19/1071142.html
 
%%author Простаков, Глеб
%%author_id prostakov_gleb
%%author_url 
 
%%tags ukraina_russia_relations
%%title Пора начинать «мягко» возвращать Украину
 
%%endhead 
 
\subsection{Пора начинать «мягко» возвращать Украину}
\label{sec:19_11_2020.news.ru.vz.prostakov_gleb.1.mjagko_vozvraschat_ukrainu}
\Purl{https://vz.ru/opinions/2020/11/19/1071142.html}
\ifcmt
	author_begin
   author_id prostakov_gleb
	author_end
\fi

Недоумение, граничащее с раздражением, вызвала у меня на днях новость, в
которой говорилось о необходимости создания в России платформы по деоккупации
Украины от внешнего управления. Я повелся на заголовок новости, но прочитанным
остался разочарован. Оказалось, что такое предложение прозвучало из уст
депутата Госдумы, как реакция на планы Киева провести в следующем году саммит
«Крымской платформы», где должны обсуждаться вопросы возвращения полуострова
Украине. То есть не серьезная инициатива, наполненная проектами и действием, а
информационный треш в ответ на такой же информационный треш с украинской
стороны.

Меж тем вопрос деоккупации Украины можно и нужно ставить на популярный нынче
проектный принцип работы. Для этого в России есть необходимая инфраструктура в
лице Россотрудничества, которая, к сожалению, пока не демонстрирует видимой
эффективности. В ответ на упреки звучат стандартные оправдания: мол, там же
вражеское окружение, даже, простите, пернуть нельзя, не привлекая внимания СБУ.
Все, однако, не столь печально: СБУ не всесильна и не вездесуща, к тому же
очень многое можно делать, укладываясь даже в рамки законов украинских
джунглей.

Возьму на себя смелость предложить пять конкретных проектов, к реализации
которых можно и нужно приступать здесь и сейчас: 

\subsubsection{Информационный сайт «Украинцам о России»}

Чтобы быть эффективной, информационно-просветительская работа не должна носить
пропагандистский характер. На Украине существует огромный запрос на эмиграцию,
причем не только в Европу, но и в Россию. Ключевое здесь – поиск работы,
возможности для учебы и, конечно, обретения гражданства. За последние годы
президент Владимир Путин и Госдума сделали немало для упрощения процедур
обретения гражданства украинцами – как из непризнанных ДНР и ЛНР, так и для
остальной Украины. А миграционная стратегия России предусматривает решение
демографических проблем, в том числе за счет мигрантов, русских мигрантов.

Сейчас, однако, в Россию едут, как правило, те, кому уже нечего терять на
Украине – люди, ищущие спасения, будь то от бедности, преследования за
инакомыслие и т. д. Меж тем Россия должна позаботиться и том, чтобы спрос на
нее возник и у вполне успешных граждан Украины, которые не могут раскрыть свой
потенциал на нынешней Украине, но которые хотели бы это сделать в родном для
себя окружении. Тем более, что сейчас у украинцев есть возможность получать
российское гражданство без отказа от украинского. Это ведь в натуральной форме
воплощает в жизнь разговоры о братских народах: раз братский – значит, и Родина
одна, и гражданство единое.

Да, это экспансия. Законная, дружественная экспансия, которая касается не
территорий, но людей. Информацию нужно донести в максимально удобоваримой
форме. Сайт под условным названием «Украинцам о России» должен быть сугубо
утилитарным. Он должен быть сосредоточен на трех основных темах: 

– возможности эмиграции в Россию (упрощенные процедуры получения вида на
жительства и гражданства, программы переселения соотечественников и т. п.);

– возможности получения образования в России (программы вузов и квоты для
иностранных студентов, упрощенные процедуры получения гражданства для
выпускников);

– возможности трудоустройства в России (привлечение высококвалифицированных
специалистов, освоение Дальнего Востока и др.).

\ifcmt
  pic https://img.vz.ru/upimg/m10/m1071142.jpg
	fig_env wrapfigure
	width 0.3
\fi

Такой информационный ресурс должен стать плотным как свинец, быстрым как пуля.
Украинцы, у которых только промелькнула мысль о том, чтобы связать свою судьбу
с Россией, должны обнаружить на таком сайте всю необходимую информацию в
несколько кликов. 

\subsubsection{Межмуниципальная коммуникация}

Вместо того, чтобы продолжать пинг-понг с украинскими олигархами и их
партийными проектами, ожидая, что страна устанет от нищеты и настырной
украинизации, а власть упадет в руки неким пророссийским силам, стоит
спуститься на землю и установить контакт с простыми украинцами. Сделать это
можно как на уровне отдельных организаций и людей (об этом ниже), так и на
уровне регионов.

Думаю, вы удивитесь, что многие украинские города все еще имеют побратимов в
России. Для Харькова – это Москва, для Николаева и Львова (да-да, в том числе
для Львова) – это Санкт-Петербург. При этом представления о России, крупных
российских городах, сформированы у большинства украинцев десятилетия назад:
когда деревья были большими, а Советский Союз – единственной страной, в которой
все мы жили вместе. Будет честным сказать, что о современной России (не
российской политике из уст пропагандистских СМИ) среднестатистический украинец
знает почти так же мало, как среднестатистический американец об Украине,
которая то и дело всплывает в тамошних фильмах и сериалах, как насквозь
коррумпированная мафиозная страна. 

Не нужно общих слов. Расскажите на форумах, круглых столах и конференциях –
благо сейчас все можно делать онлайн и никакое СБУ не помеха – о том, что
действительно может волновать украинцев. Например, об опыте Москвы по борьбе с
пробками и точечной застройкой (сверх актуально для задыхающихся Киева, Одессы
и Днепра); о том, как город преображает подсветка; о том, как строить и
ремонтировать дороги.

Украинская политическая жизнь сейчас смещается в регионы. Там – главный центр
принятия решений. Региональные элиты все меньше зависят от Киева и все больше –
от голосов избирателей на местах. Россия может помочь сделать эти голоса
инструментом своего влияния на Украине. 

\subsubsection{Взаимодействие с лидерами мнений}

Могу уверенно предположить, что перечень организаций и людей на Украине, с
которыми взаимодействует Россотрудничество и другие профильные организации,
сродни мертвым душам, голосовавшим за Джо Байдена на последних выборах
президента США. Но в отличие от американских демократов, деньги и ресурсы
России ушли в песок, не дав видимого результата.

Необходимо выйти за узкие рамки взаимодействия только с «русскими общинами» –
таковые либо уже не существуют, либо их активность максимально
маргинализирована. Вместо этого – коммуникация с организациями и персоналиями,
в целом лояльными русской культурной повестке, но не представляющими собой
«русских общин» в классическом смысле.

Совместные проекты с такими организациями, как «Куликово поле» в Одессе, «Полк
победы» в Запорожье, Институтом правовой политики и социальной защиты им. Ирины
Бережной в Киеве и многими другими, вполне могут задать рамку той самой
общественной дипломатии, «мягкой силы». 

\subsubsection{Проекты в сфере поддержки русского языка}

Такие проекты и вовсе стержень и одна из главных задач Россотрудничества. С
сентября 2020 года в украинских школах полностью прекратилось преподавание на
русском языке, что, конечно же, акт культурного геноцида в отношении
русскоязычных украинцев. С ноября этого года украинский язык силой внедрен во
всю сферу обслуживания.

Поддержка, в том числе финансовая, негосударственных учреждений по изучению
русского языка, литературы, истории России и украинской истории в общем
контексте с российской – должна стать важным направлением работы. Онлайн-курсы
будут очень востребованы, в том числе потенциальными абитуриентами российских
вузов с Украины. 

\subsubsection{Проекты в сфере культуры}

Не секрет, что последние культурные проекты Россотрудничества на украинском
направлении представляли собой, скорее, отчетную формальность и не оказывали
видимого влияния на украинскую информационную повестку. Многие украинцы с
удовольствием (хотя, конечно, и не без опасения) едут в Россию, будь то
шахматный турнир или танцевальный конкурс. Каждый такой приезд украинцев в
Россию сопровождается громким медийным скандалом с разбором полетов на Украине.
Что ж, пусть цветут сто цветов, а таких скандалов будет больше.

Например, у Россотрудничества есть проект «Посольство мастерства», который
никак не соотносится с Украиной. А ведь приглашение и организация приезда
украинских исполнителей (в стране финансируется исключительно «патриотическая»
культура) из разных городов Украины на концерты, семинары и конкурсы – отличная
идея. Скажем, на тот же конкурс им. П.И. Чайковского. Российская школа
исполнительского мастерства в классической музыке – лучшая в мире. Приобщиться
к ней – подарок для любого музыканта, вне зависимости от его политических
предпочтений. Почему бы не организовать в России культурный диалог далеких от
политики и пропаганды людей, объединенных общим культурным кодом?

Чтобы выстроить инфраструктуру российского влияния на Украине фактически с
нуля, понадобится не один год. Это сложная кропотливая работа. Кому-то
покажется, что вернуть Украину в материнское лоно можно куда проще – поставив
на условно «украинского Додона». Но, как показали последние события в Молдавии,
эта ставка эфемерна и плодоносит очень недолго. Пора начинать настоящую работу. 

\begin{itemize}

\iusr{Mike Kripke} Москва 21 день назад  

Никаких украинцев признавать нельзя. Поляки интегрируют хохлов с помощью т.н.
"карты поляка". Тот кто берёт в руки карту поляка, тот в душе
отказывается от чубатого хохлизма. Никакой бабушки-польки у
него никогда не было, он соврал в анкете на получение карты.
Или сфальсифицировал документы. Но полякам это и не важно, они
это прекрасно понимают.

\iusr{Иван Медвед} Москва 21 день назад  

Сотрудничать с Украиной нужно без сомнений. Но исключительно там, где Россия
имеет сушественную выгоду. От просто "сотрудничества" с Россотрудничеством я
выгод не вижу как и предоставшение мест на обучение в вузах беспатно, помощь
при переселении итд

\iusr{Victor Borisov} Новосибирск 21 день назад  
а зачем? экономически Украина все дальше от нас.

\iusr{Алексей Жильцов} 22 дня назад  
Вернуть земли, а Дуркаину переселить в Галичину, мягко не выйдет. А в другом варианте никаких возвращений Дуркаин не надо - своих дураков и предателей хватает.

\iusr{Сергей К} Москва 23 дня назад  

Украину вообще не надо возвращать, потому что Украина - вообще не страна. Надо
забирать себе области заселенными русскими, а области заселенные бандеровцами -
отдать обратно в Польшу. И пусть бандеровцев усмиряет сафьяновый сапог
польского пана.

\iusr{Виктор Ставровский} Видное 23 дня назад  

Культурные мероприятия хоть с трудностями, но проходят. В остальных областях у
нас для этого нет надежного тыла. Любые препятствия, не говоря уж о санкциях
вызывают вой антинациональных кругов как в бизнесе, так и в науке и
образовании.

\iusr{Roman Kalytovskyi} 23 дня назад  

Фашисты, вы на столько глупы что за Украину я спокоен, вы ничего с ней не сделаете.

\iusr{Андрей Княжев} 21 день назад

Но ведь на Украине нацистский режим ведет войну с собственным народом. Долго ли
сможет полоумная продержаться. Рано или поздно украинцы выгонят своих нацистов
и воссоединятся с Россией.

\iusr{Mike Kripke} 21 день назад

Минус 15 миллионов за 30 лет населения Украины - вы сами всё сделаете. Надо просто не мешать,

\iusr{Сергей Сергеев} 23 дня назад  

Украину уже можно возвращать не мягко, а просто признав независимость ДНР и
ЛНР. Вряд-ли Байден будет церемониться, а значит надо торопиться

\iusr{Роман Суворов} Домодедово 24 дня назад  

Зачем нам все эти телодвижения? Украина, при любой власти — может быть только
«Анти-Россией», иначе смысла в более мелком русском государстве просто нет.

Так что и заниматься этой ерундой никакого практического смысла, кроме
кормления занимающихся этим чиновников я не вижу.  Жалко в 2014-м не до давили.
Надо было ещё тогда делить, отправляя западню в Польшу, как им мечтается, и
всего делов.  Но теперь уж что - будем ждать следующего удобного момента.

\iusr{Андрей Княжев} 24 дня назад  

Украина действительно больна. Больна нацизмом. И это неудивительно, поскольку
Украину создали Ленин и коммунисты. А коммунизм и нацизм - идеологии
родственные. Но жителей Украины нам все-равно придется спасать и лечить.
Почему? А вот не так давно прошло интервью Пyтина СМИ с диковатым названием
ТАСС.

В нем Пyтин еще раз, уже официально и во всеуслышание, заявил, что мы -
великороссы и украинцы - один народ. Браво! Пyтина и нас всех можно поздравить
с этим громадным шагом на пyти к выздоровлению не только Украины, но всей
остальной России.

Нy! Осталось совсем немного. Осталось сделать еще только один маленький шажок и
назвать, наконец, то загадочное явление или событие, которое нас разъединило. А
именно, что УССР - Украину и украинство создала коммунистическая власть, создал
Ленин. Малорусский диалект русского языка коммунисты объявили "украинским
языком". С 20-х годов и до самого своего конца коммунисты вели насильственную
украинизацию населения. Делалось это в целях обмана, пропаганды и разложения
буржуазных государств на Западе.

А поскольку коммунизма и советской пропаганды уже давно нет, теперь туда же, на
помойку истории должна уйти и полоумная вильная. Территория БУ должна
воссоединиться с Россией как ГДР воссоединилась с ФРГ.

Мешает этому только трусость, глупость, подлость и скудоумие нынешней
российской элиты, состоящей в основном из тех же бывших коммунистов -русофобов,
ненавидящих нашу страну.

\iusr{Крошка Цахес} Казань 24 дня назад  

Ycpaiна - последняя страна, которую надо "мягко возвращать".  Сборище ленивых
жло6ов и предателей, понимающих только силу.  Как только русские опять поставят
xoxлов вровень с собой, так те предадут русских, как это делали неоднократно и
будут лизать немецкие или польские сапоги.

\iusr{Игорь Николенко} 24 дня назад

Самая большая русская страна в Европе.

\iusr{Голос Разума} 24 дня назад

Игорь Николенко, Самая большая русская страна в Европе это Россия

\iusr{Александр Михайловский} 24 дня назад

Игорь Николенко, Украина - это замая большая недорусская недострана и не только
в Европе, но и в общем по Галактике.

\iusr{Игорь А} Мурманск 24 дня назад  

вот ежели граждане России заживут достаточно богато/безмятежно, то Россия для
всех соседей станет очень привлекательной. а пока у нас зарплата основной части
граждан находится на уровне МРОТ, то увы.. хоть песни пой, хоть караул кричи -
не привлекательно..

\iusr{Голос Разума} 25 дней назад  

Опять рассуждения гражданина Украины на тему "Россия должна". Успокойтесь вы
уже наконец. Ничего вам Россия не должна. Она должна заботиться о своих
гражданах, а не устраивать судьбу украинских голодранцев. Сами. Все сами. Как
смогли устроить себе это болото самостоятельно, так и выбирайтесь из него.
Хлебните незалежности полной ложкой. А когда вдоволь наедитесь и сами
разберетесь со своими проблемами, тогда можно и обсуждать что-то.

\iusr{Голос Разума} 24 дня назад

Игорь Николенко, Попробую привести доступную для вас аналогию. О вас ушла жена.
И не просто ушла, а продолжает устраивать скандалы, постоянно судиться с вами,
делает вам пакости при первой же возможности. При этом он нашла себе еще
несколько мужиков, с которыми гуляет и к вам возвращаться не желает, поскольку
она независимая женщина. Вы так и будете считать ее своей родной, вбухивать в
нее свои силы и средства в надежде, что она будет к вам благосклонна?

\iusr{Игорь Николенко} 24 дня назад

Голос Разума, некорректное сравнение. Тут, скорее, о детях говорить следует,
если вспоминать русские города. Или о братьях, если украинские сёла.

\iusr{Андрей Княжев} 25 дней назад  

Последние события на Украине разрушили один из главных мифов советской эпохи:
миф об украинской нации и украинском языке. В советское время и по сей день эта
тема замалчивалась, а тут, наконец, выяснилось, что и украинская нация и
украинский язык, есть ни что иное, как выдумка советской пропаганды.

Советская власть нанесла огромные раны русскому народу. Одним из самых тяжких
преступлений является создание большевиками в 1922 году искусственных
псевдогосударств; Украинской и Белорусской ССР, исходя из чисто политических,
пропагандистских целей. В 1945 году Сталин включил эти псевдогосударства в
состав ООН. Вон когда еще Украина и Белоруссия обрели полную «незалежность»!

Эти «государства» исправно голосовали на ассамблеях ООН, поддерживая политику
Советского Союза, и ничего, Запад не протестовал. Однако в 1991 году мина,
заложенная большевиками, сработала: Россия оказалась разорванной на три части.

Мы действительно стали самым большим разделенным народом в мире. Русские люди,
проживавшие на этой земле от начала Руси, оказались заложниками ситуации, став
гражданами искусственных государств. И сегодня мы видим стремление людей
вернуться к своему естественному состоянию, объединившись в национальное
государство – Россию.

\iusr{Сергей Шимановский} 20 дней назад

Александр Канис, разница в том, что русский язык обогатился заимствованиями, но
сам язык остался как и был русским. украинскую мову не хотят обогатить, её
хотят противопоставить. это уже иная задача, не развить народа в нацию, а
деградировать до уровня бандерлогов, постоянно агрессивных к братьям и
единоверцам. или как назвал их Айтматов - манкурты.

\iusr{Александр Канис} 20 дней назад

Сергей Шимановский, я не согласен с вами.

\iusr{Сергей Шимановский} 20 дней назад

Александр Канис, а я живу на границе с Украиной довольно долгий период и вижу
всё собственными глазами. со мной согласны местные украинцы. и мне этого
достаточно.

\iusr{Петр Кратов} 25 дней назад  

Неужели у автора есть рецепт "перевоспитания" бандеро-нацистов? По-моему,
нацизм-это клинический случай, когда болячка удаляется исключительно
хирургическим ᴨутем. Либо заканчивается летальным исходом. Зачем нам эта
больная нацизмом страна? Пусть переболеет сначала.

\iusr{Андрей Княжев} 25 дней назад

Но вся Россия еще недавно была больна коммунизмом. И сейчас еще не изжила эту
болезнь. Что нам теперь, от родины отказаться? Пусть окончательно выздоровеет
сначала?

\iusr{Алексей Афанасьев} Москва 25 дней назад  

Проблемы можно в итоге решить только включением русской части бывшей УССР в состав РФ по образцу Крыма. Или псевдонезависимостью по типу южной Осетии все остальное не работает. И нерусская часть бывшей УССР даром не нужна. Только в Европу.

\iusr{Алексей Афанасьев} Москва 25 дней назад  

ППроблемы на Украине для России могут быть решены только тремя факторами. А
точнее их сочетанием. Один фактор называется ополченцы. Но для его применения
сейчас надо чтобы Украина сделала что то, выходящее за рамки прекращения огня и
прочих перемирий. Второй фактор называется экономика. Украинские остатки
экономики и так движутся к коллапсу, и рано или поздно его достигнут.
Контрольными выстрелами должны быть прекращение транзита газа через Украину и
повышение стоимости энергоносителей до предпандемийного уровня. 55-60 за
баррель. Думаю этого будет достаточно. Третий фактор - надо привести в порядок
экономику ДЛНР. И подтянуть ее хотя бы уровня Ростова и Воронежа

\iusr{Андрей Княжев} 25 дней назад  

Все это разумно и действительно необходимо. При этом нужно напоминать
украинцам, что мы единый русский народ, а УССР-Украина была искусственно
создана на русской земле нелегитимной коммунистической властью. Теперь, когда
Совдепии и советской пропаганды уже давно нет, исчез и сам смысл существования
Украины.

Единственная проблема заключается в том, что у власти в самой России до сих пор
остаются те же самые бывшие, коммунисты-русофобы, ненавидящие и презирающие эту
страну и ее народ, выходцами из которого они в большинстве являются. Но у
коммуниста или совка, как у блатных нет национальности и нет отечества. И по
этим причинам, а также в силу своего скудоумия, трусости и подлости нынешние
российские власти не способны начать процесс объединения нашей страны.

\iusr{Андрей Княжев} 24 дня назад

Алексей Афанасьев, Ну и демагог ты Алеша. Как все коммунисты. Несешь чушь
какую-то. Ответь на вопрос: украинцы и белорусы, по-твоему, это русские или это
отдельные "братские" народы?

\iusr{Алексей Афанасьев} 23 дня назад

Белорусы и Малороссы русские, а галичане-бандеровцы нет 

\iusr{Юрий Россиянин} 25 дней назад  

Вот еще маниловщина. Ничего там не будет на этой Юкрейне. Изменения возможны
только по типу Донбасса. Победа Донбасса и будет крахом бандеровщины.

\iusr{Николай Кузнецов} Санкт-Петербург 25 дней назад  

Не-не-не-не-не...!!!! Не надо.Было,За всё спасибо собаки хохло...пые

\iusr{Иван Гродзовский} Москва 25 дней назад  

Мягко? С мягким членом на бабу не залазят!

\iusr{Дэвид Нездешний} 25 дней назад

Плохо было жулью, СССР не позволял украсть слишком много. И даже если кому-то
удавалось, то спокойно пользоваться наворованным всё равно было нельзя. За сам
факт обладания бессовестно огромными средствами правоохранители уже бы
размотали по полной программе.

\iusr{Сергей Симаков}25 дней назад  

А надо ли возвращать этот кусок дерьма?

\iusr{Андрей Княжев} 25 дней назад

А это наша русская земля. Твоя родина, сынок. И там где ты живешь, тоже кусок дерьма?


\end{itemize}
