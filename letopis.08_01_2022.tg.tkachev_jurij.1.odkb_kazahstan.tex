% vim: keymap=russian-jcukenwin
%%beginhead 
 
%%file 08_01_2022.tg.tkachev_jurij.1.odkb_kazahstan
%%parent 08_01_2022
 
%%url https://t.me/dadzibao/5038
 
%%author_id tkachev_jurij
%%date 
 
%%tags kazahstran,odkb,rossia
%%title И о войсках ОДКБ в Казахстане
 
%%endhead 
\subsection{И о войсках ОДКБ в Казахстане}
\label{sec:08_01_2022.tg.tkachev_jurij.1.odkb_kazahstan}

\Purl{https://t.me/dadzibao/5038}
\ifcmt
 author_begin
   author_id tkachev_jurij
 author_end
\fi

И о войсках ОДКБ в Казахстане.

Понимаете, это ведь первая проба пера ОДКБ, первая активная операция этой
организации как организации. Первый шанс показать товар лицом - но и первое
испытание.

И уже хватает сил, работающих на то, чтобы это испытание провалили. На то,
чтобы показать как \enquote{российские и белорусские вояки помогают диктатору подавить
протест казахов}. И это ляжет в основу антироссийской риторики в Казахстане на
долгие годы вперёд. 

Подчеркну: ляжет совершенно точно. Даже если ни один российский и белорусский
солдат не то что ни одного выстрела не сделает, а даже вообще патрон в
патронник не дошлёт и с территории аэропорта не выйдет. В умении создавать
события, которых не было, нашим врагам (хоть они у нас, к счастью, общие) опыта
не занимать, и этой работой специалисты уже озадачены.

И противопоставить этому можно лишь одно: активные контрмероприятия. И не
\enquote{разоблачения фейков} (это само собой), а создание конструктивной повестки:
войска ОДКБ пришли в Казахстан  защищать мирных жителей. Тем более, что это же
правда, не так ли? 

Но правде мало быть правдой в реальном мире. Ей надо ещё воплотиться в мире
виртуальном. А пока в мире виртуальном о действиях контингента ОДКБ информации
ничтожно мало, и в основном она состоит из вбросов противоположной стороны.

А такая информация должна быть! Даже не так: соответствующие профессионалы с
российской, белорусской и казахской стороны обязаны сделать так, чтобы она
была.

\enquote{Армянские военные сопровождают скорые помощи и машины пожарных}. Не
требуется сопровождение? Ну и ладно. Но пусть посопровождают.
\enquote{Белорусские военные открывают пункт раздачи воды и продуктов первой
необходимости}. И раздавайте хлеб и чай бабушкам под камеры дружественных СМИ.
Лучше пусть раздают улыбчивые молоденькие девушки в ладно сидящей форме (ну,
вы, надеюсь, поняли мою мысль).  \enquote{Военнослужащие Киргизии взяли под
охрану детский садик}. \enquote{Российские военные вернули электричество в
населённый пункт такой-то}. Как это - нет проблем с электричеством? В смысле,
вот за всё это время ни в одном населённом пункте Казахстана не было аварий на
электросетях? Была парочка? Вот и отлично. Вот пусть российские военные их
устранят. Что, местные ремонтники с ними сами справятся? Здорово. А вот в
данном конкретном случае пусть разок-другой устранят российские военные.

Только пусть ни в чью дурную голову не придёт самим поломать и самим же
починить, а то с вас станется. А потом вас спалят (а вас спалят, вы не
американцы, чтобы в такие сложные игрушки играться) и будут справедливо
позорить.

Вы же видели фото американских солдат, играющих с иракскими детьми? Вы правда
думаете, что эти фото сделаны случайно? 

А ведь перед вами куда более простая задача, ведь, в отличие от американских
военных, вы не расстреливаете этих самых детей и их родителей с беспилотников.

На дворе XXI век, и войны ведутся не только на полях сражений, но и в головах
удалённых от них на тысячи километров обывателей. И далеко не всегда победа на
поле боя окупает поражение в информационной кампании.
