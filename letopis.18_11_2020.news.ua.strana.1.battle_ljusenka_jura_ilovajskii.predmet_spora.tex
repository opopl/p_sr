% vim: keymap=russian-jcukenwin
%%beginhead 
 
%%file 18_11_2020.news.ua.strana.1.battle_ljusenka_jura_ilovajskii.predmet_spora
%%parent 18_11_2020.news.ua.strana.1.battle_ljusenka_jura_ilovajskii
 
%%url 
%%author 
%%tags 
%%title 
 
%%endhead 

\subsubsection{О чем поспорили Бутусов и Арестович}
\label{sec:18_11_2020.news.ua.strana.1.battle_ljusenka_jura_ilovajskii.predmet_spora}

Все началось с поста главреда \enquote{Цензора} от 11 ноября, в котором тот
раскритиковал Минобороны за закупку гаубиц \enquote{Дана}. Они, по мнению
Бутусова, устарели.

\ifcmt
pic https://strana.ua/img/forall/u/0/92/%D0%B4%D0%B0%D0%BD%D0%B0.png
\fi

На претензию внезапно ответил Арестович - хотя она его никак не касалась.
Но явно затрагивала парафию Зеленского, что и вынудило Алексея начать
гибридную войну в соцсетях. 

Он назвал Бутусова \enquote{фельдмаршалом} (то есть намекнул на некомпетентность
журналиста в военных вопросах) и упрекнул в том, что тот \enquote{сеет записную
зраду}. А также напомнил, что при Порошенко армия закупала вооружение
производства 70-х годов. 

\ifcmt
pic https://strana.ua/img/forall/u/0/92/%D0%BE%D1%82%D0%B2%D0%B5%D1%82_%D0%BD%D0%B0_%D0%B4%D0%B0%D0%BD%D1%83.png
\fi

На следующий день советник Украины в ТКГ вызвал Бутусова на баттл.

\enquote{Юра, тут народ хочет баттла.

Или как у нас иногда говорят, \enquote{дізнатися істину}.

По \enquote{Вагнеру}, по военной экспертизе, по штурму тюрем, по Иловайску, по
\enquote{Данам}, про то, кто, где служил и многому другому.

Придёшь?}

\ifcmt
pic https://strana.ua/img/forall/u/0/92/%D0%B2%D1%8B%D0%B7%D0%BE%D0%B2.png
\fi

Через три дня Бутусов ответил, что на баттл при посредниках не согласен,
но может записать с Арестовичем интервью. И предложил локацию - Офис
президента. Намекая этим, что военный эксперт транслирует месседжи из ОП. 

При этом главред "Цензора" обозначил круг вопросов, которые прозвучат:
так, он усомнился в том, что Арестович имеет звание майора и 33 раза ходил
за линию фронта, служа в разведке ВСУ (о чем Алексей когда-то признавался
сам). 

\enquote{Заявленное число \enquote{боевых выходов} Арестовича в \enquote{тыл врага} в 18-19-м году
в то время, когда активных действий не велось, а фронт был стабилен,
звучит просто удивительно, поскольку все эти многочисленные рейды в тыл
врага Арестович совершал мимоходом, в перерывах между частыми поездками в
Киеве и многочисленными интервью на телеканалах}, - написал журналист. 

Также Бутусов заявил, что спросит, как Арестович из \enquote{порохобота}
превратился в \enquote{зелебота}. И напомнил, что в прошлом Арестович - актер,
игравший травести \enquote{Люсеньку}. Тем самым намекая, что вся его военная
биография - фарс. 

\ifcmt
pic https://strana.ua/img/forall/u/0/92/%D0%B1%D1%83%D1%82%D1%83%D1%81%D0%BE%D0%B2(20).png
\fi

Арестович ответил в комментариях и назвал Бутусова \enquote{черным
пропагандистом}, а не журналистом. И от интервью с ним отказался,
настаивая на дебатах.

\ifcmt
pic https://strana.ua/img/forall/u/0/92/%D0%B0%D1%80%D0%B5%D1%81%D1%821.png
\fi

Сегодня же Арестович написал пост о том, что уже согласен на интервью с
Бутусовым, но теперь сам главред \enquote{Цензора} от него отказывается. Также
спикер ТКГ опубликовал фото наградных книжек от Минобороны, которые должны
подтверждать военное прошлое Арестовича. 

Далее Арестович снова назвал Бутусова \enquote{Юрой Иловайским} и \enquote{фельдмаршалом}
- намекая, что он, несмотря на свои амбиции военного эксперта, никаких
оснований считать себя таковым не имеет. И делая отсылку к истории с
участием Бутусова в планировании провальной для Украины иловайской
операции.

