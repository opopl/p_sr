% vim: keymap=russian-jcukenwin
%%beginhead 
 
%%file 09_02_2023.fb.fb_group.mariupol.pre_war.1.mariupol_ot_sergey_d.cmt
%%parent 09_02_2023.fb.fb_group.mariupol.pre_war.1.mariupol_ot_sergey_d
 
%%url 
 
%%author_id 
%%date 
 
%%tags 
%%title 
 
%%endhead 

\qqSecCmt

\iusr{Tanya Komisar}

Такі незвичні ракурси

\begin{itemize} % {
\iusr{Олена Сугак}
\textbf{Tanya Komisar} спасибо автору, который выложил непозволительное колличество фото... Пришлось разделить...

\iusr{Tanya Komisar}
\textbf{Олена Сугак} в нього прекрасні фото
\end{itemize} % }

\iusr{Елена Сартания}

Спасибо большое за воспоминания

\iusr{Елена Сартания}

\ifcmt
  igc https://i.paste.pics/79dc2c09d5ef43dcdf6d8daa08292037.png
	@width 0.1
\fi

\iusr{Клавдия Макарова}

И это мой Мариуполь. Красавец... был. Фото изумительные, спасибо. А фото
\enquote{Сиртаки} случайно у вас нет.

\begin{itemize} % {
\iusr{Олена Сугак}
\textbf{Клавдия Макарова} у него не было.

\iusr{Sergey Drovorub}
\textbf{Клавдия Макарова} поищу. Где-то есть фото ночью, но с угла возле \enquote{Смака} в 54 доме и врядли там что-то видно будет.

\begin{itemize} % {
\iusr{Клавдия Макарова}
\textbf{Sergey Drovorub} 

Спасибо большое. Мы там с подругами любили встречаться и что нибудь отмечать. И
конечно Новый год 2022 мы тоже там отмечали, фотографировались. Мечтала свой
день рождения там отметить 5 марта, но увы. Видела на видео там одни кирпичики.
В мыслях иду туда, встречаю друзей. Мы все сейчас живём воспоминаниями.

\iusr{Sergey Drovorub}
\textbf{Клавдия Макарова} снеговик - моя работа) Сиртаки в кадре, все как заказывали 🙂
Один раз там пробовал чир-чир, очень вкусно.

\ifcmt
  igc https://scontent-fra5-2.xx.fbcdn.net/v/t39.30808-6/330191754_580413433986020_1303678400451480404_n.jpg?_nc_cat=107&ccb=1-7&_nc_sid=dbeb18&_nc_ohc=DXqIcquzJ_UAX_KVFeR&_nc_ht=scontent-fra5-2.xx&oh=00_AfC_ELbyaqVB9pOoF0jDJdnT1IgvvCF7YwBdItAQcYL-0w&oe=6405E4B3
	@width 0.4
\fi

\iusr{Sergey Drovorub}
\textbf{Клавдия Макарова} Для меня, это еще и бывший молочный магазин, а рядом слева бывший магазин \enquote{Радуга}.

\iusr{Олена Сугак}
\textbf{Sergey Drovorub} 

помню в детстве там был гатроном.Слева отдел кондитерка. Где продавались на
палочке петушки,звездочки и зайчики! Мне покупали много разных! До следующей
поездки в город с левого. А в сиртаки один раз отмечали день радио с нашим
коллективом радиоцентра АМП.

\iusr{Sergey Drovorub}
\textbf{Олена Сугак} я из гастронома больше помню лимонад и булочки. А вот мороженое, наверное в молочном было. Не помню уже.

\iusr{Олена Сугак}
\textbf{Sergey Drovorub} вам сколько лет? В мое детство, почему-то петушки были только в этом магазине...

\iusr{Sergey Drovorub}
\textbf{Олена Сугак} 44

\iusr{Олена Сугак}
\textbf{Sergey Drovorub} о! 10 лет разницы... у меня Петушки на палочке мали значення...

\iusr{Sergey Drovorub}
\textbf{Олена Сугак} петушки помню смутно, наверное так и было, как Вы пишите, просто вкусы разные и запомнилось то, что больше нравилось. Семки на остановке, тоже тема 🙂

\iusr{Олена Сугак}
\textbf{Sergey Drovorub} 

оооо! На остановке киоск с плетёнками! Подходишь, заказываешь. При тебе
скручивают, жарят и обильно посыпают сахарной пудрой! Выдают горячую, пахнущую
плетёночку!!!! Мммм прямо запах ощутила.... вкусно!

\iusr{Sergey Drovorub}
\textbf{Олена Сугак} оооо! Да! Это была суть Мариуполя! По 13 копеек. Бомба еда и лакомство.

\iusr{Клавдия Макарова}
\textbf{Sergey Drovorub} Для меня тоже. Помню ходила туда за молоком и рядом было мороженое его продовали на вес. Вот только память и осталось нечего и внукам показать и рассказать они конечно, что то и запомнили, но не всё и фото все сгорели. Спасибо за фото. Я тоже любила фотографировать и много было особенно старых улиц.

\iusr{Sergey Drovorub}
\textbf{Клавдия Макарова} пожалуйста.

\end{itemize} % }

\end{itemize} % }

\iusr{Нина Обернова}

\ifcmt
  igc https://i.paste.pics/7ab89e7218cac8ed0def626ea0c62086.png
	@width 0.1
\fi
