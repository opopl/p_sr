% vim: keymap=russian-jcukenwin
%%beginhead 
 
%%file 08_12_2018.stz.news.ua.mrpl_city.1.svetilniki
%%parent 08_12_2018
 
%%url https://mrpl.city/blogs/view/svetilniki-kogda-v-mariupole-ne-bylo-e-lektroe-nergii
 
%%author_id burov_sergij.mariupol,news.ua.mrpl_city
%%date 
 
%%tags 
%%title Светильники: когда в Мариуполе не было электроэнергии
 
%%endhead 
 
\subsection{Светильники: когда в Мариуполе не было электроэнергии}
\label{sec:08_12_2018.stz.news.ua.mrpl_city.1.svetilniki}
 
\Purl{https://mrpl.city/blogs/view/svetilniki-kogda-v-mariupole-ne-bylo-e-lektroe-nergii}
\ifcmt
 author_begin
   author_id burov_sergij.mariupol,news.ua.mrpl_city
 author_end
\fi

\ii{08_12_2018.stz.news.ua.mrpl_city.1.svetilniki.pic.1.kaganec}

Сейчас, когда вечером вдруг отключается электричество, в доме начинается
паника. Как назло оказывается, что аккумулятор в фонарике сел, мобильник не
заряжен, а стеариновая свечка, припасенная в кухонном шкафу на такой случай,
оказалась заставленной банками с кофе и чаем, пузырьками и коробочками с
лекарствами. Наконец, обжигая догорающими спичками пальцы, извлекается
злополучная свеча, теперь она заполняет кухню тусклым светом. Поужинали. Чем
заниматься дальше? Экран телевизора темен, телефон с отъемной трубкой не
работает, а вязать, шить и читать при свече уже несколько поколений отвыкло.
Остается только сетовать: как обитатели здешних мест могли жить без
электричества? Действительно – как?

\textbf{Читайте также:} \href{https://mrpl.city/blogs/view/da-budet-svet}{Да будет свет!, Кирилл Папакица, mrpl.city, 11.11.2018}

\ii{08_12_2018.stz.news.ua.mrpl_city.1.svetilniki.pic.2_3}
\ii{08_12_2018.stz.news.ua.mrpl_city.1.svetilniki.pic.4_5}

Пока не удалось найти достоверных сведений о том, чем освещали свои жилища
запорожцы, несшие службу в укреплении близ реки Кальмиус. Как это делали
греки-переселенцы из Крыма, заселившие в 1780 году город, остается только
догадываться. Может быть каганцами – плошками с салом и фитилем (на фото)?

\ii{08_12_2018.stz.news.ua.mrpl_city.1.svetilniki.pic.6.ploshka}

Может самодельными сальными свечами? Но вот когда было налажено у нас
промышленное производство свечей, можно узнать из статьи преподавателя
Мариупольской Александровской мужской гимназии М. И. Кустовского, (\enquote{Мариуполь и
его окрестности}, 1892 г.) Вот что он пишет: \enquote{\em Свечное дело у нас началось в
1850 году с основанием завода г. Дикаревым сначала на Почтамтской, потом на
Торговой улице. В 1862 году основан свечной завод Файном на Почтамтской улице.
С 1870 года выделка сальных свечей прекратилась, их вытеснили стеариновые свечи
и керосин. Выделку восковых свечей Дикарев продолжает до сих пор. В 1887 году
основан завод восковых свечей Даниловым на Торговой, а в 1890 году – Медведевым
у большого фонтана}.

Керосиновые лампы разных конструкций и фасонов появились в России в 1861 и
1862 и вытеснили все другие источники света. Тогда же, или чуть позже,
появились они и в нашем городе. Известно, что в начале декабря 1898 года
открыта первая в Мариуполе частная электростанция, она принадлежала Елизавете
Томазо. Это, по понятиям конца XΙX века, чудо техники находилось на улице
Харлампиевской во дворе местного обывателя Попова. Двор этот примыкал к
гостинице \enquote{Континенталь} (теперь в этом здании располагается Дворец молодежи),
и электростанция, оснащенная керосиновым двигателем, как раз и предназначалась
для ее освещения.

\textbf{Читайте также:} 

\href{https://mrpl.city/news/view/fasad-istoricheskogo-zdaniya-v-mariupole-lishilsya-poloviny-podsvetki-fotofakt}{%
Фасад исторического здания в Мариуполе лишился половины подсветки, Денис Росін, mrpl.city, 29.11.2018}

26 февраля 1899 года Д. А. Хараджаев получил от местных властей разрешение на
установку электрогенератора, оснащенного керосиновым двигателем, для устройства
электрического освещения его дома на Александровской площади. Теперь на его
месте стоит дом со шпилем. Его адрес – ул. Артема, 48. 20 января 1905 года
городская дума разрешила прокладку проводов от электростанции, принадлежавшей
Елизавете Томазо, к частным домам на улицах Георгиевской и на Александровской
площади. 12 апреля 1908 года городская управа заключила договор с инженером Р.
Шильдгауэром об устройстве в нашем городе электрического освещения. Наконец, в
ноябре 1908 года на улице Малофонтанной была построена электростанция, давшая
ток насосам городского водопровода, а заодно и уличному освещению, а также в
дома обывателям.

Шли годы, все больше комнат мариупольских домов стало озаряться электрическими
лампочками. Это благо цивилизации иногда прерывалось в годы гражданской войны,
когда останавливалась электростанция из-за отсутствия горючего для двигателей.
Но топливо умудрялись привозить, и снова начинали стучать движки насосов,
подача электроэнергии и воды возобновлялась, снова загорались лампочки.

\textbf{Читайте также:} \href{https://mrpl.city/news/view/nakanune-zimy-mariupolchanka-invalid-ostalas-bez-sveta-i-gaza}{%
Накануне зимы мариупольчанка с инвалидностью осталась без света и газа, Анастасія Папуш, mrpl.city, 30.10.2018}

Потом было 22 июня 1941 года. Оккупация города гитлеровцами. Зная, что вот-вот
враг захватит город, работники электростанции попытались вывести ее
оборудование из строя. Но немцам удалось за короткое время все восстановить. А
перед тем, как покинуть город в сентябре 1943 года под ударами Красной Армии,
захватчики так изуродовали электростанцию, что она уже не подлежала
восстановлению. Дома и улицы погрузились во тьму. И тут история освещения
мариупольских домов как бы вернулась вспять.
