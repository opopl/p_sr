% vim: keymap=russian-jcukenwin
%%beginhead 
 
%%file slova.inostranec
%%parent slova
 
%%url 
 
%%author_id 
%%date 
 
%%tags 
%%title 
 
%%endhead 
\chapter{Иностранец}
\label{sec:slova.inostranec}

%%%cit
%%%cit_head
%%%cit_pic

\ifcmt
  tab_begin cols=4
     pic https://avatars.mds.yandex.net/get-zen_doc/3985561/pub_6198dae7bc2e532a0ac99182_6198dcb96e8e2140dcad0501/scale_1200
     pic https://avatars.mds.yandex.net/get-zen_doc/5290304/pub_6198dae7bc2e532a0ac99182_6198dcccd3594a24250c5cad/scale_1200
		 pic https://avatars.mds.yandex.net/i?id=53348b8ab92a8a906a09f5b3198f9e15-4197982-images-thumbs&n=13
		 pic https://cdn.pixabay.com/photo/2017/08/10/20/56/donkey-2627750_1280.jpg
  tab_end
\fi
%%%cit_text
Странности русского языка: «Девичник» – это женская вечеринка, а «бабник» -
любвеобильный мужчина.  Как объяснить \emph{иностранцу}, что "как можно лучше"
- это хорошо, а "как нельзя лучше" - это уже отлично?!  Как объяснить
\emph{иностранцу}, что «коза» и «козёл» - это одно и то же животное, но разного
пола, а «оса» и «осёл» – два совершенно разных?  Как объяснить
\emph{иностранцам}, что «жрать как свинья» - это очень много есть, а «нажраться
как свинья» - это как будто и не есть вовсе?
%%%cit_comment
%%%cit_title
\citTitle{Странности и причуды удивительного русского языка}, Досужник, %
, zen.yandex.ru, 20.11.2021
%%%endcit
