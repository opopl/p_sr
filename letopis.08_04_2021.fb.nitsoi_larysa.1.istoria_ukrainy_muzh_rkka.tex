% vim: keymap=russian-jcukenwin
%%beginhead 
 
%%file 08_04_2021.fb.nitsoi_larysa.1.istoria_ukrainy_muzh_rkka
%%parent 08_04_2021
 
%%url https://www.facebook.com/larysa.nitsoi/posts/4188059794559454
 
%%author 
%%author_id 
%%author_url 
 
%%tags 
%%title 
 
%%endhead 
\subsection{День народження - мій чоловік - Андрій Ніцой - історія України - РККА}
\Purl{https://www.facebook.com/larysa.nitsoi/posts/4188059794559454}

\ifcmt
  pic https://scontent-mxp1-2.xx.fbcdn.net/v/t1.6435-9/170441331_4188055111226589_3333147401919092642_n.jpg?_nc_cat=104&ccb=1-3&_nc_sid=730e14&_nc_ohc=CpYFjmro24AAX-u_r3S&_nc_ht=scontent-mxp1-2.xx&oh=99ce0b375d36aa64e4423328ab86670f&oe=60A092DF

	pic https://scontent-mxp1-2.xx.fbcdn.net/v/t1.6435-9/170492548_4188056037893163_1643046718564714183_n.jpg?_nc_cat=106&ccb=1-3&_nc_sid=730e14&_nc_ohc=8Bt_pMxFMGQAX-xxehp&_nc_ht=scontent-mxp1-2.xx&oh=ac961b7a68b8e333d9bf5f15b456ad59&oe=609ED1C9
\fi

Сьогодні День народження в мого чоловіка, Андрія Ніцоя. А ще в цей день - день
Ангела в Ларис. Ми з чоловіком уже багато років сперечаємося, хто з нас у кого
ангел. Він каже, що то я, Лариса, його ангел. А я кажу, що то він мій ангел.
Ангел-охоронець. Я завдячую йому здійсненням багатьох моїх мрій. 

Мені пощастило, що мій чоловік мені друг. А ще він дружить з моєю мамою, своєю
тещею. У нас зовсім не так, як показує 95 квартал чи дизельшоу стосунки
«теща-зять». Буває я злюся на них обох, що вони обоє заодно, роблять якусь
шкоду. Наприклад, їдять разом те, чого не можна. Мама приготує - а він радий
старатися. Влітає їм обом від мене. Але я не пам’ятаю такого, щоб він на людях
повстав проти мене, навпаки, я можу завжди розраховувати на його плече.

Мій чоловік дуже розумний. Він увесь час щось читає. А ще він постійно готує
якісь аналітичні висновки.  Він патріот. От і сьогодні, син виконує тестові
завдання з історії. Чи побачите ви в цих питаннях якийсь підступний зміст,
навіть якщо я вас попередила, що він там є?

- Якими подіями завершилася Громадянська війна в Україні? 

А).  Розгромом Червоною Армією військових частин Української Галицької Армії
та укладанням Ризького мирного договору.  

Б).  Розгромом Червоною Армією Армії
Української народної республіки та польської армії маршала Ю.Пілсудського 

В).  Розгромом Червоною Армією військ генерала П.Врангеля в Криму та основних сил
повстанських загонів Н.Махна.  

Г).  Розгромом Червоною Армією збройних сил
півдня Росії генерала А.Денікіна та військ Директорії УНР.

У мого коханого одразу купа зауважень. 

\begin{itemize}
\item - Знову ми перебуваємо в парадигмі Росії. 

\item - Чому ми й досі називаємо громадянською війною те, що насправді було
інтервенцію росії в Україну. Точно як зараз замість  назвати війну війною росії
проти нас, ми називаємо це терористичною операцією. 

\item - Чому в усіх 4 пунктах ідеться про перемогу Червоної Армії – і виховується
таким чином наше пораженство. Хто б не воював, перемогла Червона Армія.

\item - Чому про Червону Армію говориться так, ніби це наше, рідне, зрозуміле,
замість говорити армія росії, яка вторглася в Україну, а от  частини УГА,
польські, врангельські, пілсудські, Денікінські, Махна, УНР, наше зібране в
кучу разом з ненашим – ніби наше для нас теж чуже.
\end{itemize}

Щоб бачити ці підтексти, мало бути патріотом, треба добре знати історію. НАШУ.
УКРАЇНСЬКУ. Треба бути розумним, щоб уміти співставляти і миттєво бачити
продовження московської спецоперації проти майбутнього України  і виховання
українців-пораженців та вплив на дітей через освіту.

Я щаслива, що маю вдома живу ходячу енциклопецію, до якої можна звернутися з
питанням і отримати кваліфіковану відповідь. 

А ще мій чоловік – порядна людина. З почуттям обов’язку. Він міг зробити
карколомну кар’єру високого посадовця, чи політика. Але для цього йому треба
було б зрадити своїм принципам. Не раз життя підкидало йому кучеряві шанси, але
для цього йому треба було б стати пристосуванцем. Інші змогли. Він – ні.
Звісно, матеріально, ми втрачали від цього. 

Іноді я згадую йому. 

- От, відмовився отоді-то, від отого-то. Мали б сьогодні і те, і те. А ти не поступився. 

А він у відповідь:

- Це мені каже та, яка на шкоду собі, на шкоду своїм і родинним  справам іде на
телебачення і замість пристосовуватися, валить правду в очі, забуваючи про
власну вигоду? Та, яка ставить українське над особистим?»

Ми сміємося з ним. Обіймаємося. І йдемо гуляти. За ручки. Як в молодості. Я
дуже рада, що мій чоловік – мій однодумець. А ще… Зайшла я на сторінку до нього
взяти якесь фото з ним для цього допису, а там що не фото, то на ньому Я… 

Ледве вибрала самого. Перше фото зроблене колись раніше. Друга світлина
зроблена сьогодні вранці. Ці фото передають його стан. Він постійно в розумовій
роботі. Прокинулася – немає. Пішла шукати. Працює. Ця робота ніколи не буде на
шкоду селу, місту чи нашій країні в цілому.  

Побажайте, будь ласка, моєму панові Андрію здоров’я.
