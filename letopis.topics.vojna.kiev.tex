% vim: keymap=russian-jcukenwin
%%beginhead 
 
%%file topics.vojna.kiev
%%parent topics.vojna
 
%%url 
 
%%author_id 
%%date 
 
%%tags 
%%title 
 
%%endhead 

\section{Київ}
\label{sec:topics.vojna.kiev}

% https://www.facebook.com/groups/story.kiev.ua/posts/1927855000744625

% https://www.facebook.com/groups/story.kiev.ua/posts/1929964153867043
% https://www.facebook.com/groups/story.kiev.ua/posts/1909661372563988
% https://www.facebook.com/groups/story.kiev.ua/posts/1911053082424817
% https://www.facebook.com/groups/story.kiev.ua/posts/1917755155087943
% https://www.facebook.com/groups/story.kiev.ua/posts/1935412639988861/

% https://www.facebook.com/andriibaumeister/posts/5049542531833932
% https://www.facebook.com/groups/story.kiev.ua/posts/1936575893205869/
% https://www.facebook.com/groups/story.kiev.ua/posts/1936556659874459/

\begin{itemize} % {
\item \hyperlink{02_03_2022.fb.tovt_anastasia.1.mama_smotri_ja_streljaju}{%
Мама, смотри, я стреляю!, Анастасия Товт, facebook, 02.03.2022%
}

\item \hyperlink{04_03_2022.stz.news.ua.pravda.1.reportazh_z_kieva}{%
Репортаж з Києва. Війна очима журналіста, Дмитро Рясний, pravda.com.ua, 04.03.2022%
}

\item \hyperlink{10_03_2022.tg.krjukova_svetlana.1.zhizn_v_kieve}{%
Жизнь в Киеве продолжается, принимая свои новые формы, Светлана Крюкова, telegram, 10.03.2022
}

\item \hyperlink{11_03_2022.fb.fb_group.story_kiev_ua.2.misto_i_neljudy}{%
Місто і нелюди, Сергій Грабар, Київські Історії, facebook, 11.03.2022%
}

\item \hyperlink{07_04_2022.fb.fb_group.story_kiev_ua.1.misto_i_pamjatnyky_chastyna_1}{%
Місто і пам'ятники. Частина І, Сергій Грабар, Київські Історії, 07.04.2022%
}

\item \hyperlink{13_04_2022.fb.fb_group.story_kiev_ua.1.geroi_oborony_kieva}{%
ГЕРОЇ ОБОРОНИ КИЄВА, Max Poznyak, Київські Історії, facebook, 13.04.2022%
}

\item \hyperlink{16_04_2022.fb.fb_group.story_kiev_ua.1.vecher_tishina}{%
Вот снова вечер, тишина, Elena Sapunova, Київські Історії, facebook, 16.04.2022%
}

\item \hyperlink{24_04_2022.fb.fb_group.story_kiev_ua.1.kiev_v_marte_prodolzhenie}{%
Киев в марте. Продолжение, Александр Алексеенко, Київські Історії, facebook, 24.04.2022%
}

\end{itemize} % }
