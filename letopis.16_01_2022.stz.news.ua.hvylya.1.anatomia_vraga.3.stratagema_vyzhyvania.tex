% vim: keymap=russian-jcukenwin
%%beginhead 
 
%%file 16_01_2022.stz.news.ua.hvylya.1.anatomia_vraga.3.stratagema_vyzhyvania
%%parent 16_01_2022.stz.news.ua.hvylya.1.anatomia_vraga
 
%%url 
 
%%author_id 
%%date 
 
%%tags 
%%title 
 
%%endhead 

\subsubsection{Стратагема выживания}
\label{sec:16_01_2022.stz.news.ua.hvylya.1.anatomia_vraga.3.stratagema_vyzhyvania}

Находясь на периферии славянского мира, окруженное \enquote{дружественными}
княжествами, имея длительный контакт с кочевой степью, с Золотой Ордой, как с
территорией перманентного и неумолимого хаоса, маленькое Московское княжество
было вынуждено выработать свою особую стратагему выживания.

Стратагема заключалась в нескольких базовых пунктах:

1. Сакральность единоначальной и беспрекословной вертикали принятия легитимных
государственных решений в вопросах внутренней и внешней политической повестки.

2. Использование любых доступных мер для устранения \enquote{территории хаоса} -
\enquote{кочевой степи}, как опасного фронтира, угрожающего субъектности московских
элит и столице - Москве.

3. Увеличение своей территории путем экспансии, для увеличения \enquote{буферной зоны},
соприкасающейся с \enquote{территорией хаоса} и отдаляющего фронтир от столицы -
Москвы, а следовательно, от элиты.

4. Культивирование в народе мировоззрения культа агрессивной экспансии, как
эффективного механизма защиты от \enquote{территории хаоса} (культа войны) и создание
пантеона государственных побед.

5. Превращение клерикальных институтов (церкви) в де-факто государственные для
создания идеологемы жертвенности \enquote{защиты от территории хаоса} в православной
\enquote{крепости стабильности} и поддержания сакральности единоначальной
государственной вертикали.

Последовательно и системно используя эту стратагему, начиная уже с князей
Даниила, Василия Темного, Ивана Калиты, Дмитрия Донского, маневрируя в
сложнейших отношениях с увядающей Золотой Ордой, Москва дипломатией, интригами,
покупкой ярлыков на княжение, непрерывными войнами \enquote{за объединение}, до 1478
года подчинила своей власти большинство соседних удельных княжеств, а великого
московского князя Ивана III, после уничтожения суверенитета вечевой
Новгородской республики, начали именовать \enquote{собирателем земель русских} и
\enquote{государем всея руси}. По большому счету, все пункты московской стратегии
государственного функционирования, институционально, работают и по сей день.

