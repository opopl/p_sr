% vim: keymap=russian-jcukenwin
%%beginhead 
 
%%file 11_11_2020.news.ru.vedomosti_ru.1.vaccine_sputnik_v
%%parent 11_11_2020
 
%%url https://www.vedomosti.ru/society/news/2020/11/11/846541-razrabotchik-vaktsini-sputnik-otsenil-ee-effektivnost
%%author 
%%author_id 
%%tags 
%%title 
 
%%endhead 


\subsubsection{Разработчик вакцины «Спутник V» оценил ее эффективность в 92\%}
\Purl{https://www.vedomosti.ru/society/news/2020/11/11/846541-razrabotchik-vaktsini-sputnik-otsenil-ee-effektivnost}
\index[rus]{Коронавирус!Вакцина!Спутник V}

Эффективность российской вакцины от коронавируса «Спутник V», согласно
промежуточным данным третьей фазы пострегистрационного исследования с
участием 40 тыс. добровольцев, составила 92\%, сообщается в совместном
пресс-релизе Центра им. Гамалеи и Российского фонда прямых инвестиций
(РФПИ).

«В статистический анализ было включено 20 подтвержденных случаев
заболевания. По распределению 20 подтвержденных случаев (выявленных в
группе плацебо и в группе, получавшей вакцину) было определено, что
эффективность «Спутник V» составила 92\%», – говорится в сообщении. Оценка
эффективности была проведена среди более 16 000 добровольцев через 21 день
после получения первой дозы вакцины или плацебо.

Первыми вакцину в сентябре начали получать добровольцы среди персонала
«красных зон» российских госпиталей. Наблюдение за 10 000 вакцинированных
также подтвердило эффективность «Спутник V» на уровне более 90\%. В ходе
клинических исследований в России первую дозу вакцины получили 20 000
добровольцев, первую и вторую дозы – более 16 000. Непредвиденных
нежелательных явлений у участников исследования не выявлено. У части из
них наблюдалась боль в месте введения вакцины, повышение температуры тела,
слабость, утомляемость и головная боль, уточняется в сообщении.

Наблюдение за добровольцами продолжится в течение полугода, затем будет
подготовлен финальный отчет. Третья фаза клинических исследований также
проводится в Белоруссии, Венесуэле, ОАЭ и ряде других стран. В России
также проходит клиническое исследование вакцины с участием людей старшего
возраста. Директор Центра им. Гамалеи Александр Гинцбург отметил, что
публикация промежуточных результатов пострегистрационных испытаний делает
возможным начало массовой вакцинации россиян в течение ближайших недель.

Ранее по теме: Турция планирует производить российскую вакцину «Спутник V»

Ранее американская компания Pfizer сообщила, что ее совместная с немецкой
BioNTech вакцина от коронавируса BNT162b2 по результатам третьей фазы
клинических испытаний доказала свою эффективность на 90\%. В испытаниях приняли
участие 43 538 добровольцев.

Россия стала первой в мире страной, зарегистрировавшей собственную вакцину от
коронавируса «Спутник V» Центра им. Гамалеи. Она была зарегистрирована 11
августа, производство препарата началось 15 августа. В данный момент проходят
пострегистрационные испытания вакцины, в ходе которых планируется привить 40
000 добровольцев.


