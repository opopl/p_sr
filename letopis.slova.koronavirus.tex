% vim: keymap=russian-jcukenwin
%%beginhead 
 
%%file slova.koronavirus
%%parent slova
 
%%url 
 
%%author 
%%author_id 
%%author_url 
 
%%tags 
%%title 
 
%%endhead 
\chapter{Коронавирус}
\label{sec:slova.koronavirus}

Так что не так с \emph{коронавирусом}?  Давно не писал на тему, но есть
неприятные открытия. Итак. 1. Вы уже переболели? Я да.  У каждого протекает
болезнь по разному, но самое неприятное, о чем молчат, это реальные
последствия.  По данным британских неврологов, коронавирус, оказывается, может
привести к серьезным повреждениям мозга и центральной нервной системы и стать
причиной психоза, паралича или инсульта...  В журнале Brain опубликовали данные
о том, что коронавирус может стать причиной серьезных повреждений мозга даже у
пациентов с легкими симптомами COVID-19 или у тех, кто уже выздоровел.
\url{https://www.ucl.ac.uk/news/covid-neurology}\par
\url{https://cutt.ly/GnWzY0N}\par Но не хочу пугать, просто спрошу, а как Вы
спите?,
\citTitle{Ковид оказался гораздо коварнее и опаснее, чем предполагалось}, Анатолий Амелин, strana.ua, 08.06.2021

%%%cit
%%%cit_head
%%%cit_pic
%%%cit_text
За последние сутки в 85 регионах России выявлено 13 510 случаев заражения
\emph{коронавирусом}, это максимум с 15 февраля. По информации оперативного штаба,
наибольшее количество новых зараженных зафиксировали в Москве (6701). Это самый
высокий показатель суточной заболеваемости с 26 декабря 2020 года, когда было
выявлено более 7400 заболевших. Следом за столицей Московская область (889) и
Санкт-Петербург (867). В Нижегородской области, которая находится по этому
показателю на четвертом месте, выявлено 179 случаев. Наименьшее – в Магаданской
области (5), Ненецком автономном округе (3) и Чукотском автономном округе (1).
Всего в России выявлено более 5 млн 193 тыс. случаев и зафиксировано 126 073
смерти. За прошедшие сутки от \emph{коронавируса} умерли 399 человек, полностью
выздоровели 9986. В штабе советуют всем, кто почувствовал любые симптомы ОРВИ,
оставаться дома и вызывать врача
%%%cit_comment
%%%cit_title
\citTitle{Коронавирус приспособился к лекарствам}, Андрей Резчиков, vz.ru, 12.06.2021
%%%endcit

В экспертной среде указывают, что сразу несколько факторов способствуют
ухудшению ситуации с \emph{коронавирусом}. Завкафедрой вирусологии РМАНПО,
доктор медицинских наук Елена Малинникова прежде всего связывает рост
заболеваемости с ослаблением внимания людей к эпидемической обстановке – многие
перестали регулярно носить маски в общественных местах, не соблюдают социальную
дистанцию. «Летом население более активно. Люди больше общаются друг с другом и
чаще совершают поездки в разные регионы страны. После выпускных вечеров многие
дети и подростки активно контактировали между собой. Не секрет, что молодежь
чаще болеет бессимптомно и тестируется реже взрослых. Они могли стать активным
источником распространения инфекции», – считает Малинникова
%%%cit_comment
%%%cit_title
\citTitle{Коронавирус приспособился к лекарствам}, Андрей Резчиков, vz.ru, 12.06.2021
%%%endcit

%%%cit
%%%cit_head
%%%cit_pic
\ifcmt
  pic https://img.strana.ua/img/article/3419/vladimir-menshov-biohrafija-94_main.jpeg
	width 0.4
	caption Владимир Меньшов. Фото: РИА Новости 
\fi
%%%cit_text
О \emph{коронавирусе} как причине смерти пишет в своем официальном сообщении и
концерн \enquote{Мосфильм}. \enquote{Киноконцерн Мосфильм и семья
Владимира Меньшова с прискорбием сообщают, что сегодня Владимир Валентинович
ушел из жизни. Выдающийся советский и российский кинорежиссер, актер,
сценарист, продюсер, член правления \enquote{Мосфильма} скончался от
последствий \emph{коронавирусной инфекции}. Мы потеряли нашего дорогого друга,
мосфильмовца, отдававшего родной киностудии все свои творческие силы и яркий
талант, по-настоящему народного режиссера, замечательные фильмы которого знают
и любят миллионы людей}, - сообщили на киностудии \enquote{Мосфильм}
%%%cit_comment
%%%cit_title
\citTitle{Владимир Меньшов - биография, фильмы, отношения с Украиной}, 
Оксана Малахова; Анна Копытько, strana.ua, 05.07.2021
%%%endcit

%%%cit
%%%cit_head
%%%cit_pic
\ifcmt
  pic https://strana.news/img/forall/u/0/0/FCoJsiQXsAYXDyb.jpg
  @width 0.4
\fi
%%%cit_text
"Локдауны из-за \emph{коронавируса} вызвали перебои в поставках и подорожание
перевозок. В свою очередь логистические проблемы и увеличение затрат привели к
кризису предложения: продавцы жалуются на нехватку товаров и необходимость
поднимать цены для компенсации расходов. Товары есть, но они на складе. В
магазинах - часто урезанный выбор. Не хватает даже зубной пасты и
презервативов. Зато полно костюмов и украшений для Хэллоуина", - рассказал
"Стране" американский журналист Эд Гринберг
%%%cit_comment
%%%cit_title
\citTitle{Картинки вместо еды. Как в Британии и США возник дефицит товаров и почему там вспоминают СССР}, 
Александра Харченко, strana.news, 27.10.2021
%%%endcit
