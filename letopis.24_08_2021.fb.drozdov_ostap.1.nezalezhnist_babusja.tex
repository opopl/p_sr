% vim: keymap=russian-jcukenwin
%%beginhead 
 
%%file 24_08_2021.fb.drozdov_ostap.1.nezalezhnist_babusja
%%parent 24_08_2021
 
%%url https://www.facebook.com/ostap.drozdov/posts/4186985758063951
 
%%author Дроздов, Остап
%%author_id drozdov_ostap
%%author_url 
 
%%tags babushka,nezalezhnist,semja,ukraina
%%title Бабцина Незалежна Україна датується 28 травня 1947 року
 
%%endhead 
 
\subsection{Бабцина Незалежна Україна датується 28 травня 1947 року}
\label{sec:24_08_2021.fb.drozdov_ostap.1.nezalezhnist_babusja}
 
\Purl{https://www.facebook.com/ostap.drozdov/posts/4186985758063951}
\ifcmt
 author_begin
   author_id drozdov_ostap
 author_end
\fi

\begin{multicols}{2}

Прошу уважно подивитися на цю світлину.  

Вона датована 28 травня 1947 р. Зліва внизу ви побачите цю дату. 28 травня 1947
року арештували мою бабцю. Перед вами – вишивка, яку вона вишила в тюрмі. 

Бабці було 17 з половиною років. Очікуючи на депортацію в Сибір, вона потайки
вишила цю вишивку за кілька тижнів. Нитки брала з чого завгодно: з хусток, зі
светрів інших в’язнів, із підкладки своєї  куртки бронзового кольору, зі
спідниць. Цю вишивку вона ховала зіжмаканою в спідній білизні. Ця вишивка їхала
з нею в Красноярський край, була там 8 років і повернулася додому. 

Над наївно зображеною Богородицею – так само незграбно вишитий текст: «Мати
Божа Неустанної Помочи ратуй мене». 

Знизу вишивка підписана: «Спомин з тюрми».

Ця вишивка – родинний документ про те, що шлях моєї родини до Незалежної
України почався 28 травня 1947 року.

Я звертаюся до всіх тих, хто 24 серпня святкуватиме буцімто 30-річчя буцімто
народження України. Подивіться на цей спомин з тюрми, вишитий 17-річною
політув'язненою, і збагніть нарешті, що нашій країні – не 30. І не 40. І навіть
не 50. 

Наша країна не з’явилася з чистого аркуша 24 серпня 1991 року. Вона не постала
з нуля. Вона могла постати лише тому, що упродовж століть кривавого поневолення
було кому вишивати спомини з тюрми. А ще раніше – помирати у лавах повстанців.
А ще раніше – проголошувати УНР. А ще раніше – творити українське письменство.
А ще раніше, і пізніше, і завжди – зберігати українську ідентичність,
українську душу, українську  мрію всупереч усім заборонам. Їх було незчисленні
мільйони попередників - отих безіменних незалежних Українців. 

І тому Незалежній Україні – не 30. І тому треба святкувати не народження країни
- а її відновлення. І тому вона не молода держава, а давня, вимріяна й
окроплена кровію не одного покоління. 

Бабцина Незалежна Україна датується 28 травня 1947 року. У когось іншого це
інша дата. І кожна з таких предтеч справжньої незалежності має право попросити
своїх нерозумних нащадків: віддайте належне всій цій величній передісторії, без
якої апріорі не могло би з’явитися 24 серпня 1991 року. Відмовтесь від цієї
змалілої цифри 30. 

Ні. Нам не 30. Це занадто егоїстично й невдячно. Це вкрай несправедливо. Це
нечесно. Нам не 30. 

Зараз ця вишивка, оправлена в простеньку рамку, висить достойно й вельможно в
кімнаті 90-річної авторки. Живої, здорової, рухливої й незалежної. Висить
нарівні з іконами. 

Отакою є моя родинна історична арифметика: 

28 травня 1947 року = 24 серпня 1991 року.

\begin{minipage}{0.45\textwidth}
\ifcmt
  ig https://scontent-cdt1-1.xx.fbcdn.net/v/t39.30808-6/240396436_4186985691397291_8936517935569475806_n.jpg?_nc_cat=101&ccb=1-5&_nc_sid=730e14&_nc_ohc=l_hvqv6WO2AAX_pYuoF&_nc_ht=scontent-cdt1-1.xx&oh=aacc8dfc208df404cfb2d921548d5c6b&oe=6130EF41
\fi
\end{minipage}

\end{multicols}
