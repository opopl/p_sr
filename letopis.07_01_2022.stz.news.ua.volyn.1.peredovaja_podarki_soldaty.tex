% vim: keymap=russian-jcukenwin
%%beginhead 
 
%%file 07_01_2022.stz.news.ua.volyn.1.peredovaja_podarki_soldaty
%%parent 07_01_2022
 
%%url https://www.volyn.com.ua/news/202623-pyrohy-varenyky-i-mishky-dlia-voiniv-volontery-z-lutska-povezly-na-peredovu-hostyntsi
 
%%author_id 
%%date 
 
%%tags vojna,donbass,ukraina
%%title Пироги, вареники і мішки для воїнів: волонтери з Луцька повезли на передову гостинці
 
%%endhead 
\subsection{Пироги, вареники і мішки для воїнів: волонтери з Луцька повезли на передову гостинці}
\label{sec:07_01_2022.stz.news.ua.volyn.1.peredovaja_podarki_soldaty}

\Purl{https://www.volyn.com.ua/news/202623-pyrohy-varenyky-i-mishky-dlia-voiniv-volontery-z-lutska-povezly-na-peredovu-hostyntsi}

Півсотні коробок печива, 10 коробок домашніх пирогів і стільки ж відер
вареників. Волинські волонтери 6 січня повезли українським воїнам на схід
смаколики до Різдвяних свят, – розповів Суспільному керівник волонтерського
центру «Серце патріота» Сергій Балицький.

\begin{zznagolos}
Зі слів Сергія Балицького на сході волонтери відвідають три військові
підрозділи. Бійцям на Різдво обов'язково везуть кутю і теплі речі.	
\end{zznagolos}

«Є понад 1,5 тисячі мішків їм на укріплення. Туди хлопці пісок, вода не затікає
і все. Для них це будівельний матеріал», – каже волонтер. Додає: часто бійці
просять привезти запчастини до машин та інструменти. Зі слів Сергія Балицького
на сході волонтери відвідають три військові підрозділи. Бійцям на Різдво
обов'язково везуть кутю і теплі речі.

\ii{07_01_2022.stz.news.ua.volyn.1.peredovaja_podarki_soldaty.pic.1}

Речі зібрали з допомогою небайдужих. На Театральному майдані у скриньку для
збору грошей для потреб військових перехожі кладуть купюри. «Добре, що люди
допомагають, чим можуть. А ми вже підключаємо меценатів або самі купуємо за
свої кошти речі, які потрібні бійцям», – каже волонтерка Ольга Балицька.

Перед відправленням на схід волонтери зібралися на Театральному майдані Луцька.
Сюди небайдужі приносять речі, які хочуть передати солдатам.

Бус та причіп везуть у підрозділи Збройних Сил за 1,5 тисячі кілометрів, аби на
Різдво бійці були з подарунками, – каже волонтерка Ольга Балицька. Передачі для
солдатів мають забрати по дорозі в Полтаві та Харкові.

Як розповів керівник волонтерського центру «Серце патріота», за майже вісім
років війни це 261 поїздка волонтерів до зони бойових дій.

Читайте також: Порошенко в Авдіївці передав комплекс відеоспостереження
десантникам
