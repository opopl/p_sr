% vim: keymap=russian-jcukenwin
%%beginhead 
 
%%file 06_06_2023.stz.news.ua.donbas24.1.u_kievi_vidbulos_proschannja_iz_geroichnym_mariupolcem
%%parent 06_06_2023
 
%%url https://donbas24.news/news/u-kijevi-vidbulosya-proshhannya-z-geroyicnim-mariupolcem
 
%%author_id demidko_olga.mariupol,news.ua.donbas24
%%date 
 
%%tags 
%%title У Києві відбулося прощання з героїчним маріупольцем (ВІДЕО)
 
%%endhead 
 
\subsection{У Києві відбулося прощання з героїчним маріупольцем (ВІДЕО)}
\label{sec:06_06_2023.stz.news.ua.donbas24.1.u_kievi_vidbulos_proschannja_iz_geroichnym_mariupolcem}
 
\Purl{https://donbas24.news/news/u-kijevi-vidbulosya-proshhannya-z-geroyicnim-mariupolcem}
\ifcmt
 author_begin
   author_id demidko_olga.mariupol,news.ua.donbas24
 author_end
\fi

\begin{center}
\Large\em\bfseries\color{blue}
5 червня у столиці на Байковому кладовищі прощалися з Данилом Подибайлом
\end{center}

\ifcmt
  ig https://i2.paste.pics/2c968114c865e23a340bcab3db6d220b.png
  @wrap center
  @width 0.9
\fi

Україна щодня втрачає найкращих своїх синів. 5 червня українці попрощалися з
музикантом, поетом та письменником із Маріуполя — Данилом Подибайлом, якому
було всього 26 років. На Байковому кладовищі зібралося багато маріупольців, які
запам'ятали свого співмістянина, як щирого патріота та світлу Людину та прийшли
провести в останню путь Героя і віддати належну шану його подвигу!

\textbf{Читайте також:} \href{https://donbas24.news/news/legendam-zsu-prisvyatyat-marki-ukrposta-anonsuje-svyatkovii-vipusk}{%
Легендам ЗСУ присвятять марки: Укрпошта анонсує святковий випуск, Яна Іванова, donbas24.news, 13.10.2022}

\subsubsection{Сторінки з біографії}

Данило — син Марії Подибайло, кандидатки історичних наук, відомої волонтерки і
громадської діячки, яка створила та очолила громадський центр \enquote{Новий
Маріуполь}. Змалечку хлопець відвідував лекції мами і навіть сам їх проводив.
Був дуже здібним хлопчиком (читати почав у 2 рочки!). Для своїх двох сестричок
був не тільки старшим братом, але й наставником. Мав багато друзів, завжди був
дуже надійним і відповідальним. 

\begin{leftbar}
	\begingroup
		\bfseries
{\color{blue}\em\enquote{Яке щастя було тебе народити! Яке щастя було тебе мати своїм Сином!  Яке
щастя, що ми говорили про свою любов буденно! Дякую, що залишив мені
так багато світла. І тепла. Бо не задача прожити 26, 33, 75, 124,5
роки. Задача прожити. Щоб було про що. А тут ти майстер спорту!}}, —
наголосила мати Данила, Марія Подибайло. 
	\endgroup
\end{leftbar}

\begin{leftbar}
	\begingroup
		\bfseries
{\color{blue}\em\enquote{Так як він мене підтримував і розумів, мене не підтримував ніхто. Він завжди
був поруч. Навіть, коли був у Бахмуті, питав мене, як себе почуваю}}, —
поділилася молодша сестричка Христина. 
	\endgroup
\end{leftbar}

\ii{06_06_2023.stz.news.ua.donbas24.1.u_kievi_vidbulos_proschannja_iz_geroichnym_mariupolcem.pic.1.maria_i_danja}

\textbf{Читайте також:} 

\href{https://donbas24.news/news/u-kijevi-prezentuvali-muzicnii-videoklip-koxannya-stal}{%
У Києві презентували музичний відеокліп \enquote{Кохання сталь} (ВІДЕО), Ольга Демідко, donbas24.news, 15.05.2023}

Данило співав на багатьох патріотичних мітингах. Не маючи професійної музичної
освіти, за власним бажанням та завдяки підтримці мами йому вдалося опанувати
гру на гітарі. Часто писав власні пісні. Співати почав з 11 класу. Мав
поетичний хист та писав книжки. Закінчив Маріупольський державний університет,
за освітою — історик.

\begin{leftbar}
	\begingroup
		\bfseries
{\color{blue}\em\enquote{Ще у студентські роки проявляв виключні патріотизм і любов до рідної
країни. Хлопець писав пісні про АТО, організовував концерти та виїзди
музичних гуртів до військових частин}}, — підкреслили в університеті.
	\endgroup
\end{leftbar}


\ii{06_06_2023.stz.news.ua.donbas24.1.u_kievi_vidbulos_proschannja_iz_geroichnym_mariupolcem.pic.2.danja_gitara_scena}

\textbf{Читайте також:} \href{https://donbas24.news/news/rosiyani-vistrelili-sobi-v-nogu-do-cogo-prizvede-pidriv-grebli-na-kaxovskii-ges}{%
Росіяни \enquote{вистрелили собі в ногу} — до чого призведе підрив греблі на Каховській ГЕС, Еліна Прокопчук, donbas24.news, 06.06.2023}

Був одним з найкращих у Маріуполі майстрів з освітлення сцени та звукорежисером. Крім цього, вивчав IT. Після війни хлопець мріяв запустити власний стартап та продовжити письменницьку діяльність.

\subsubsection{Життя після повномасштабного вторгнення}

У березні 2022 року Данило зміг вирватись із блокади Маріуполя. Проте виїжджати
було складно, потрапив до полону, де пройшов всі допити, після чого його
пропустили. Коли опинився на підконтрольній Україні території, вступив до лав
тероборони. Стояв на захисті Києва та брав участь у звільненні Харкова.
Останнім місцем, де мужній хлопець тримав оборону, був Бахмут. Наруто — це
позивний Данила. Всі побратими Данила підкреслюють, що він і його друг з
позивним Раста були надзвичайно відважними. Вони допомагали вибивати тили та
резерви російських загарбників у Кліщіївці. Саме це дозволило просуватися
флангам. Загалом хлопці знайшли і знищили таку велику кількість ворожої техніки
і складів, що були представлені до державної нагороди — Ордену \enquote{За мужність}.
Та, на жаль, отримати не встигли, адже загинули 1 червня у бою під Бахмутом. 

\begin{leftbar}
	\begingroup
		\bfseries
{\color{blue}\em\enquote{Мої хлопчики, найкрутіші аеророзвідники, кращі з кращих пілотів загинули
1 червня під Бахмутом. Вони були найкращими друзями і одним екіпажем,
їх позивні Раста і Наруто знали всі... Підрозділ аеророзвідки \enquote{Ангели}
пам'ятатиме вас вічно! Помста ворогу прийде з неба і ми знаємо, що ви
будете з нами в цей час! Ви будете з нами завжди!}}, — зауважила Наталія
Шевчук. 
	\endgroup
\end{leftbar}

\textbf{Читайте також:} 

\href{https://donbas24.news/news/taktika-tisyaci-poriziv-zsu-kontratakuyut-u-doneckii-ta-zaporizkii-oblastyax}{%
Тактика \enquote{тисячі порізів} — ЗСУ контратакують у Донецькій та Запорізькій областях, Еліна Прокопчук, donbas24.news, 05.06.2023}

\ii{06_06_2023.stz.news.ua.donbas24.1.u_kievi_vidbulos_proschannja_iz_geroichnym_mariupolcem.pic.3.danja_zsu}
\ii{06_06_2023.stz.news.ua.donbas24.1.u_kievi_vidbulos_proschannja_iz_geroichnym_mariupolcem.pic.4.danja}

\subsubsection{Прощання з Данилом у столиці}

Данило був гідним сином і братом, люблячим чоловіком і татом, справжнім другом,
щирим українцем і маріупольцем, закоханим у своє місто. За що б хлопець не
брався, завжди доводив справу до високих результатів. Попрощатися з Данилом
прийшли його побратими, родичі, друзі та багато маріупольців, які були знайомі
з цим талановитим юнаком. Кожен хотів висловити подяку мамі, Марії Подибайло,
та провести відважного хлопця в останню путь. 

\begin{leftbar}
	\begingroup
		\bfseries
{\color{blue}\em\enquote{Я щаслива, що ти встиг знайти всі відповіді на ті глибинні питання, які
стільки років зводили тебе з розуму аж до роздертого серця і оголеноі
душі... Мій Наруто! Мій СанКай! Моє щастя, дароване мені за якусь дуже
хорошу поведінку, мабуть. Але коли це я встигла такою зразковою бути?
Може ти мені авансом? Подумаю над цим якось...}}, — поділилася Марія
Подибайло.
	\endgroup
\end{leftbar}

\ii{06_06_2023.stz.news.ua.donbas24.1.u_kievi_vidbulos_proschannja_iz_geroichnym_mariupolcem.pic.5.bajkove}

\textbf{Читайте також:} \href{https://donbas24.news/news/baxmut-staje-tirom-ukrayina-povertaje-kontrol-nad-visotami}{%
БЛОГ Бахмут стає тиром — Україна повертає контроль над висотами, Олександр Коваленко, donbas24.news, 02.06.2023}

\begin{leftbar}
	\begingroup
		\bfseries
{\color{blue}\em\enquote{Дякуємо за те, що був! За те, що є і завжди будеш з нами. В іншому
вимірі, в іншому просторі, в іншому житті... Але знаємо точно, що будеш!
Спочивай з миром, Данило! Ти заслуговуєш на відпочинок. А ми тут ще
трохи повоюємо}}, — зазначила Марина Авраменко.
	\endgroup
\end{leftbar}

\begin{leftbar}
	\begingroup
		\bfseries
{\color{blue}\em\enquote{Я знаю, відчуваю, впевнена — ти будеш поруч, своїми
книгами, віршами, піснями, це 10 000\%}}, — зазначила Ірина Бекетова.
	\endgroup
\end{leftbar}

Після поховання Марія Подибайло зібрала всіх присутніх на творчий вечір Данила.
Маріупольці дивилися його кліпи, слухали пісні. Його мама, бабуся, побратими,
цивільна дружина, сестри і друзі ділилися теплими спогадами про Даню. 

\begin{leftbar}
	\begingroup
		\bfseries
{\color{blue}\em\enquote{Дивимося всі ці відео, де Даня співає, виступає і завжди посміхається і
дуже хочеться йому аплодувати стоячи}}, — підкреслила Алевтина Швецова. 
	\endgroup
\end{leftbar}

До того ж, саме на цю дату, 5 червня, було заплановано прем'єру фільму про
підрозділ \enquote{Ангели}, де служили Наруто і Раста (його найближчий друг, який
загинув з Данилом в одному бою). І прем'єра відбулася.

\textbf{Читайте також:} 

\href{https://donbas24.news/news/koxannya-vsuperec-viini-voyin-zsu-pid-baxmutom-zrobiv-propoziciyu-svoyii-divcini-video}{%
Кохання всупереч війні: воїн ЗСУ під Бахмутом зробив пропозицію своїй дівчині (ВІДЕО), %
Наталія Сорокіна, donbas24.news, 29.05.2023%
}

\ifcmt
  ig https://i2.paste.pics/72ceedb1e5c5f0db7c9318d28a08e0c5.png
  @wrap center
  @width 0.9
\fi

\href{https://youtu.be/eQQ6MhdL8LQ}{%
БЕЗ ПРАВА НА ДУБЛЬ | Документальний фільм, Факти тижня, youtube, 05.06.2023%
}

\begin{leftbar}
	\begingroup
		\bfseries
{\color{blue}\em\enquote{Я дивилася цей фільм і пишалася тим, які неймовірні люди стали на захист
нашої України. А ще думала про те, що саме цей фільм потрібно
перекласти і демонструвати в світі, аби всім, хто цього ще не зрозумів,
пояснити, що воно таке — український патріотизм, з чого складається
українська душа і чому ми обов'язково переможемо}}, — поділилася
маріупольська письменниця Оксана Стоміна.
	\endgroup
\end{leftbar}

Мама і сестри Данила планують видати його книжки. Тож, вже найближчим часом
можна буде їх придбати.

Раніше Донбас24 розповідав, що воїн ЗСУ \href{https://donbas24.news/news/voyin-zsu-geroyicno-evakuyuvav-rodinu-malenkkogo-xlopcika-z-baxmuta-video}{героїчно евакуював родину маленького
хлопчика з Бахмута}.

Ще більше новин та найактуальніша інформація про Донецьку та Луганську області
в нашому \href{https://t.me/donbas24}{телеграм-каналі Донбас24}.

Фото: з архіву Донбас24

\ii{insert.author.demidko_olga}
