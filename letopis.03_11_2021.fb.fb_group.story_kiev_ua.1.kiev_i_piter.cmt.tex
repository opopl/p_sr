% vim: keymap=russian-jcukenwin
%%beginhead 
 
%%file 03_11_2021.fb.fb_group.story_kiev_ua.1.kiev_i_piter.cmt
%%parent 03_11_2021.fb.fb_group.story_kiev_ua.1.kiev_i_piter
 
%%url 
 
%%author_id 
%%date 
 
%%tags 
%%title 
 
%%endhead 
\subsubsection{Коментарі}

\begin{itemize} % {
\iusr{Катя Бекренева}

Мой муж ездил на Кировский в командировки, в то время его родной брат был в
Питере почти год на курсах при ЛПИ. Рассказывал как гуляли по городу, а потом
"брали Зимний ") Город очень красивый.


\iusr{Валентина Луконина}

Чудесно написано. А о парадных - вообще бальзам на раны! Спасибо, Пётр!

\iusr{Анжелика Ангел}
В Питере говорят "парадная".

\begin{itemize} % {
\iusr{Юрий Никитин}
\textbf{Анжелика Ангел} увы, не все. Особенно в спальных районах или в новых «человейниках» все реже и реже услышишь это слово.

\iusr{Анжелика Ангел}
\textbf{Юрий Никитин} просто в спальных районах нет "черных" лестниц)
\end{itemize} % }

\iusr{Ольга Сергиенко}
Люблю родной Киев, а Питер моё место силы. Мне его не хватает очень сейчас

\iusr{Наталья Емельянова}
Есть еще один город, где употребляют слово "парадное". Это Одесса.

\begin{itemize} % {
\iusr{Петр Кузьменко}
\textbf{Наталья Емельянова}, не зря она Южная Пальмира. И "сюдой", в моём рассказе услышанное в Ленинграде, оттуда.

\iusr{Larisa Olesyuk}
\textbf{Петр Кузьменко} вообще, если услышали тудой-сюдой, то не от коренного жителя Северной Пальмиры

\iusr{Петр Кузьменко}
\textbf{Larisa Olesyuk} , не спорю. Но, приятно! @igg{fbicon.wink} 

\iusr{Елена Калинина}
\textbf{Наталья Емельянова} разве что кто- нибудь понаехавший.

\iusr{Татьяна Досаева}
Слово парадное употребляли в Николаеве и в ..... Ташкенте

\iusr{Петр Кузьменко}
\textbf{Татьяна Досаева}, про Николаев не знаю. А про Ташкент узнаю у лучшего друга, коренного Ташкентца.
\end{itemize} % }

\iusr{Екатерина маковецкая}
спасибо)))) спасибо)))) моя бабушка ленинградка, а прожила всю жизнь на Подоле)))) и мое сердце на Андреевском спуске и на Невском)))

\iusr{Леся Сагайдачная}

Когда как говорю, могу подъезд, могу парадное. В Питере парадная. Но не суть) я
почему то никогда его не любила и схожести с Киевом не ощущала. В Москве - да,
в Питере нет. Не знаю почему.

\begin{itemize} % {
\iusr{Петр Кузьменко}
\textbf{Леся Сагайдачная} , а я не люблю Москву. Совсем. Не ощущаю себя там комфортно. Не город совсем. Какой-то базар-вокзал.

\iusr{Леся Сагайдачная}
\textbf{Петр Кузьменко} ну я там бывала когда ещё и Арбат был. И иная Москва. Но и Киев изменился. Просто в Москве половина семьи живёт. Хотя в Питере друзья, я действительно не могу это объяснить

\iusr{Оля Лемишко}
\textbf{Петр Кузьменко} Согласна с Вами- Москва холодная, спешащая, не уютная. А Питер какой-то светлый, добрый, приветливые лица. Возможно сейчас все по другому, ведь я и там , и там была ещё в 80.

\iusr{Леся Сагайдачная}
\textbf{Оля Лемишко} а вот у меня с точностью до наоборот. Питер холодный, а Москва широкая что-ли)) и весёлая. Но опять таки, давно всё было. Уже в 90-х иначе всё стало

\iusr{Оля Лемишко}
\textbf{Леся Сагайдачная} Возможно это ещё и зависит от того, с кем нам пришлось общаться ( родственников и близких друзей в это число не включаем), время года и даже погоду. Хорошей ночи Вам!

\iusr{Леся Сагайдачная}
\textbf{Оля Лемишко} а возможно)) да) хотя в Питере у меня очень хорошие друзья были из рок.клуба их. И Сайгон)) вам тоже
\end{itemize} % }

\iusr{Larisa Olesyuk}

Больше спасибо за воспоминание и такой доброжелательный пост. Очень тронута. В
юности для меня существовали три города: Киев, Ленинград и Москва. И я ездила
по этому Бермудскому треугольнику и наслаждалась тройной красотой.

\iusr{Мария Бойчук}

И так будет! И сейчас ездят те, кто знает, что их здесь любят и ждут. И не
только из Питера, но из Москвы и др. городов России. Постепенно народ
прозревает, сбрасывает с лица вуаль черной пропаганды, которая все еще пытается
пудрить мозги людей. Но здравый рассудок побеждает. Ничто не вечно под Луной.
Пройдет и это...

\begin{itemize} % {
\iusr{Мария Бойчук}

Почему? Я знаю многих россиян, которые уже открыли глаза и были просто удивлены
тем, как здесь у нас, в Киеве, все не так, как им рассказывают. Это от чистого
сердца. Я не понимаю, что Вам не понравилось?

\end{itemize} % }

\iusr{Iryna Morozovs'ka}

В 1998 была в круизе Москва-С.Петербург-Москва. В Питере поразили две вещи, не
считая достопримечательностей. Во-первых, к кому бы мы ни обращались, все не
просто показывали дорогу, а становились нашими гидами, причем совершенно
бесплатно и очень доброжелательно. А во-вторых, огромное количество пьяных
мужчин, женщин, молодежи и даже подростки и это было очень печально видеть.
Сейчас уже нет, мою эмпатию вырубили, когда "паребрик" пришел в мой Донецк и
видеть его в Киеве не горю желанием. Не может быть миролюбивого
демократического города в ортократическом государстве.

\iusr{Леся Сагайдачная}
\textbf{Iryna Morozovs'ka} Угу

\iusr{Алексей Добрынин}
черный пес петербург...

\iusr{Петр Кузьменко}
\textbf{Алексей Добрынин} ... морда на лапах...

\iusr{Jarek Yaroslav}

мы ездили с мамой на круизе по Скандинавии и у нас была остановка в Санкт Петербурге - город волшебный.
и я помню белые ночи это просто сумасшествие - энергия так прет хочется всю ночь гулять, творить, жить  @igg{fbicon.heart.growing} 

\iusr{Olena Elovskaya}

А мы тоже с классом ездили жили в школе на Электросиле. Зимой на Новый год.
Тогда в городе стоял жуткий мороз —40*,а в поезде, когда мы ехали было —10* (
трубу отопления прорвало). Замерзли, но было очень интересно. Кстати в
Ленобласти есть местность под названием Оболонь.

\iusr{Ольга Сергиенко}
Спасибо за этот пост. Это не модно сейчас. Но так не хватает.

\iusr{Константин Богуславский}

Питер и Киев и правда похожи. Для меня в Киеве есть огромный минус. Я когда
приезжаю в Киев и вижу это у меня даже портится настроение. Люди зачем-то
массово уродуют прекрасные старинные здания жлобскими самодельными балконами.

В Питере этого нет. Вообще. То есть совсем. В Питере нет старинных зданий, где
человек захотел и повесил себе балкон из белого говно-пластика. Речь про центр
и про старые здания, естественно. Ну может где-то и есть, но я как-то не
встречал.

\begin{itemize} % {
\iusr{Лариса Кушниренко}
\textbf{Konstantin Boguslavskiy} . 

Не представляю прекрасные здания Питера изуродованные пластиком застеклённых
царьбалконов. Еще не помню жутких графитти. Но, к сожалению, последний раз была
в 2013.

\begin{itemize} % {
\iusr{Ольга Кирьянцева}
\textbf{Лариса Кушниренко} 

таких жутких графитти и "самодельных" балконов нет, пожалуй, больше нигде. В
Новосибирске - точно. Если рисуют, то только в глубине дворов, на старых
заборах или эстакадах.

\iusr{Наталия Попко}
\textbf{Ольга Кирьянцева} 

В Новосибирске нет старых зданий(старый город -это особнячки), а здесь сносился
прилегающий дом и открывалась старая стена. Вид скажем не очень... Или старый
подъезд между домами, которому уже 150 лет... Здесь совсем другое...

\iusr{Ольга Кирьянцева}
\textbf{Наталия Попко} 

да, Новосибирску всего 130 лет. Но я о другом: люди здесь живут такие же, наши
современники, но почему-то они не уродуют центр города (да и окраины) нелепыми
балконами и уродскими рисунками.

\iusr{Наталия Попко}
\textbf{Ольга Кирьянцева} 

Ну, скажем ,застеклены балконы у Вас тоже по разному и сумасшествия с вагонкой у
вас нет. Это я согласна. Новосибирск очень похож на Киев, особенно район улицы
Челюскинцев-это как Соломинка у вокзала, где я выросла. Я была очень поражена.

\end{itemize} % }

\iusr{Лариса Кушниренко}

Для меня, города разные. Но, Петербург просто прекрасен. А то, во что
превращается центр Киева, вызывет у меня печаль. Особенно "радуют" застекленные
балконы с развешанными кондиционерами и антеннами, и графитти.

\end{itemize} % }

\iusr{Elena Kharchenko}

Моё детство до 7 лет проходило по переменно в этих двух городах. Так получилось .
Но Киев не сравнить по красоте и архитектурной гармонии ни с одним городом мира .
Ленинград холодный, абсолютно плоский.
Сейчас уже изменились оба не в лучшую сторону.

\iusr{Наталия Ковалева}
Спасибо за рассказ!

А я в 1987 году ездила с подругой почти на два месяца на курсы по
вычислительной машине "Роботрон"!

Побывали и на выставках и в музеях и в театрах и по пригородам поездили!

Жили на квартире, нам предоставили комнату, так еще и командировочные, потом
нам еще прислали по 120 рублей, я даже не представляю сколько бы это стоило.

В последние дни я еще поехала на электричке в Репино.

Как раз была красивая зима!!!

Ехали в Ленинград долго, сутки, а назад уже самолетом!

\iusr{Марианна Носаль}

Красивый город, а я к сожалению так и не съездила, собиралась в 2012, не вышло,
а потом как-то уже и страшно ездить для меня, не решилась.

\begin{itemize} % {
\iusr{Kit Gross}
\textbf{Марианна Носаль} 

я тоже боялся в Москву несколько лет назад лететь. Поехал. Никаких там проблем
не было. Нигде. Никаких косых взглядов, намеков, шуточек, хамства, злобы, хотя
во всех документах четко написано место рождения - Киев. В самом центре
гостинница Украина и на ней это написанно большими светящимися буквами, город
герой Киев на Красной Площади, вокзал Киевский и много чего другого
украинского. Повторяю, ни вражды, ни злости. Одно только отмечу, никто,
наверное на всякий случай, не поднимал тему политики, что правильно. Мы даже на
Украине друг с другом согласиться ни в чем не можем.

\iusr{Kira Dva}
\textbf{Марианна Носаль} 

абсолютно безопасно. Была в 2019 году. А когда перед отьездом, на набережной
Фонтанки, бросали монетки в Чижика- Пыжика, меня попросили перевести на
украинский знаменитое "Чижик-пыжик, где ты был?

\obeycr
На Фонтанке водку пил.
Выпил рюмку, выпил две
Закружилось в голове".
\restorecr

\end{itemize} % 

\iusr{Ludmila Debourdeau}

Благодарю вас за этот пост, такой тёплый и оптимистичный, очень понравилось.
Разделяю ваши чувства, хотя в Питере никогда не была. Не довелось. Но надеюсь
когда-нибудь для себя открыть этот прекрасный город Достоевского. Знаю его по
книгам, стихам, фильмам. Но вы правы, я думаю киевляне своей любовью схожи с
петербуржцами. Была у меня подруга-петербурженка, которая взахлёб с большой
любовью рассказывала о своём любимом городе, совсем как мы киевляне.

\iusr{Игорь Козиянчук}

Ау??? Какая похожесть??? Даже если сильно захотеть то не получится. Где в
Питере зелень? Парки? Скверы? Где горы, яры, буераки? Абсолютно
противоположность по растительности , по местности. Может быть по архитектуре?
Ровные выстроенные ряды в престольном городе. Центр культурв того времени -
эрмитаж, русский музей, кунскамера. Или может в Киеве есть хоть что-то похожее
на Кронштат??? Я уже не говорю про белые ночи и пронизывающий промозглый холод
рсенью и весной. Это два абсолютно разных города, мира, мировоззрения.

\begin{itemize} % {
\emph{Алексей Добрынин}
то притензии к автору - Киеву 1500 лет Питер был тогда в сперматозоиде)))

\begin{itemize} % {
\iusr{Петр Кузьменко}
\textbf{Алексей Добрынин} , вообще-то Киеву гораздо больше 1500. Но,даже по Вашей, партийно-притянутой версии, уже 1539. Тут не о возрасте городов, об их ауре и особенностях.

\iusr{Алексей Добрынин}
\textbf{Петр Кузьменко} согласен с вами, был не прав ...))
\end{itemize} % }

\iusr{Петр Кузьменко}
\textbf{Игорь Козиянчук} , 

сколько людей, столько и мнений. Об этом я и написал в начале. А Кронштадт так
правильно пишется. Доводилось там бывать. Но, это уже другая история.

\begin{itemize} % {
\iusr{Игорь Козиянчук}
\textbf{Петр Кузьменко} ??? Два семиотически одинаковых места в городах. Стрелка Васьльевского острова и Влалимирская горка с Андреевским спуском. Где похожесть? Даже если оооочень напиться
Это как говорят, что русские, которые приезжают как топ звезда рф Пореченков, на сафари поубивать украинцев, наши братья. ( Если вдруг щакрадется сомнение - спросите"братика" что он сделал, когда увидел видео как во время перемирия Пореченков стреляет в Украинцев на украине? И вам ответят........ Ничего. Разве что слова. Но не перестали ходить в театр, где он играет и не перестали смотреть фильмы с ним, не байкотировпли убийцу)

\iusr{Петр Кузьменко}
\textbf{Игорь Козиянчук} те кто пришёл на нашу землю со злом и оружием не русские и не ленинградцы или москвичи. Нечисть не имеет национальности и территориальной идентификации. Или Вы до сих пор ассоциируете немцев и Германию с нацизмом, а итальянцев и Италию с фашизмом?

\iusr{Aleksandr Obukhov}
\textbf{Игорь Козиянчук}
Ублюдков с обеих сторон предостаточно.
Но мы на их территроию не ходили, однозначно.

\iusr{Петр Кузьменко}
\textbf{Aleksandr Obukhov} с этим полностью согласен. @igg{fbicon.hands.shake} 

\iusr{Игорь Козиянчук}
\textbf{Петр Кузьменко} 

немцы и итальянцы сейчас не поддерживают фашистских убийц. В отличии от
питерцев и москвичей, которые ежедневно платят гонорары убийцу Пореченко. И да.
Европейцам стыдно за прошлое. А русские восхваляют иои тихо молчат, делая вид,
что они не причем. Сколько процентов из русских выступили против войны россии
против украины? ( Это когда военные рф сбивали самолеты с украинцами, в гуково
со своей территории растреливали украинских военных)??? Единицы. А остальные...
Именно с их молчаливого согласия....

\iusr{Петр Кузьменко}
\textbf{Игорь Козиянчук} 

Вы вспоминаете Пореченкова, мелкую экранную мразь, и забываете о великом
Макаревиче, великолепной Ахеджаковой и талантливого Невзорова. Да, выступить
против существующей системы трудно. Это настоящие герои, а героев всегда мало.
Однако, посмотрите, по прошествии времени (оптимистично надеюсь, что мы с Вами
доживём до этого) россияне будут относится к нынешнему режиму и его
преступлениям так же, как сейчас итальянцы относятся к фашизму и немцы к
нацизму.

\iusr{Игорь Козиянчук}
\textbf{Петр Кузьменко} 

Да. Герои есть. Но их ничтожное меньшенство по сравнению с великими
Смоктуновскими, Баталовіми, Михалковіми. Уві.

\iusr{Игорь Козиянчук}
\textbf{Петр Кузьменко} 

Если бы хотя ыі каждый 20!!!! выступил против войны, против завхвата крыма...
Вы же прекрасно это понимаете, что войны бы не было. Не было бы смертей. А
так.... Молчат или поддерживают. И только единицы, как с ветрояными мельницами
пытаютс остановить кровь. Во времена фашизма единицы тоже выступали против. А
большинство молчало. И это молчание привело к трагедии, геноциду тогда и
приводит сейчас..

\iusr{Игорь Козиянчук}

И на всякий случай напомню, что из каждых 100 рулей, которые вы потратите в
Писере сейчас, 1 вернется к вам в в виде пули. И убьт или вас, или вашего
соседа. Увы. Россия сделала такую реальность. А каждый концерт в Питере
Макаревича - это целый снаряд, который к нам прилетит. Увы. Таков механизм
хунты. Не важно ты за или против, но полюбому кормишь монстра

\iusr{Петр Кузьменко}
\textbf{Игорь Козиянчук} 

Вы перечислили выживших из ума стариков, когда-то бывших кумирами, и добавили
сюда жополиза (извините за грубость) Михалкова. Но у них это фамильное.
Вспомните его папочку. Брат стыдился фамилии, пока тоже не вошёл в
соответствующий возраст...

\iusr{Петр Кузьменко}
\textbf{Игорь Козиянчук} с этим не спорю. Но, мой пост вовсе не об этом!

\iusr{Петр Кузьменко}
\textbf{Игорь Козиянчук} согласен. Это психология людей плюс пропаганда и спецпропагпнда.
\end{itemize} % }

\end{itemize} % }

\iusr{Ольга Чекрыгина}

Оба города прекрасны. Но для меня они совсем разные. Попытаюсь объяснить
почему. Это ощущение появилось давно, как только попала в тогда ещё Ленинград.
Но сформулировать это отличие смогла только через много лет, попутешествовав по
Европе.

Петербург, так же как и Париж - это литературные произведения, писатели,
художники. Когда туда попадаешь - все знакомо. Но для меня все это выглядит,
как декорации.

А вот Киев, Иерусалим и Рим - это настоящее, живое. Идёшь и ощущаешь, что здесь
до тебя ходили Княгиня Ольга, Олег, Ярослав Мудрый. Здесь Владимир крестил
народ. А моя дочь крещена в той же церкви, что и Михаил Булгаков. Такое же
чувство "настоящести" я испытала в Иерусалиме у могилы Давида и в Риме, гуляя
по Форуму и глядя на этот город с холмов. И Питер, и Киев - неповторимы. Их
общая черта - люди, которые любят свой город и гордятся им, и могут
рассказывать о них без конца

\iusr{Юрій Марков}

Спасибо за интересный рассказ ...Многие позиции автора не просто совпадают, но
повторяют мои прямо точь в точь. После 7-го класса, нас школьников в Питер
возила наша учительница физкультуры - классный руководитель с ее напарницей
(тоже физрук)... Впечатления зашкаливали: белые ночи, Невский проспект,
Исаакиевский собор, фонтаны Петродворца, Летний сад... Позже, уже взрослым, я
много раз бывал там. Мариинский театр и Валерий Гергиев... В составе труппы
украинские фамилии: Шевченко, Огновенко, Михайленко... Театралы Питера и Киева
помнят известного дирижёра Константина Симеонова, который был главным дирижёром
оперных театров Мариинского (тогда Кировского) и Национальной оперы Киева
(тогда Киевский театр оперы и балета им. Т. Шевченка). Кстати, самое первое
место погребения Тараса Григорьевича Шевченка было на Смоленском кладбище в
Питере (там и сейчас имеется такой большой валун-памятник с соответствующей
надписью). Первые анестезиологи Украины - А. Трещинский, А. Маловичко ездили на
стажировку в клинику П. Куприянова. И потом завязались самые теплые отношения
между питерскими и киевскими анестезиологами...

\end{itemize} % }
