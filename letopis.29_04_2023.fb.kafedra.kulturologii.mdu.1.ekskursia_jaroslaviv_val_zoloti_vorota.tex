%%beginhead 
 
%%file 29_04_2023.fb.kafedra.kulturologii.mdu.1.ekskursia_jaroslaviv_val_zoloti_vorota
%%parent 29_04_2023
 
%%url https://www.facebook.com/100090026247557/posts/pfbid02tFz6yi5FM5C69kx6sbHp3At2P6uP7aXA33PN1EQH2jpQ8TLJwK9pXB87f1cR7L65l
 
%%author_id kafedra.kulturologii.mdu,demidko_olga.mariupol
%%date 29_04_2023
 
%%tags 
%%title Екскурсія - Ярославів Вал - Золоті Ворота
 
%%endhead 

\subsection{Екскурсія - Ярославів Вал - Золоті Ворота}
\label{sec:29_04_2023.fb.kafedra.kulturologii.mdu.1.ekskursia_jaroslaviv_val_zoloti_vorota}
 
\Purl{https://www.facebook.com/100090026247557/posts/pfbid02tFz6yi5FM5C69kx6sbHp3At2P6uP7aXA33PN1EQH2jpQ8TLJwK9pXB87f1cR7L65l}
\ifcmt
 author_begin
   author_id kafedra.kulturologii.mdu,demidko_olga.mariupol
 author_end
\fi

28 квітня в рамках діяльності Гуманітарного штабу МДУ кандидат історичних наук,
доцентка кафедри культурології \href{\urlDemidkoIA}{Ольга Олександрівна Демідко} провела екскурсію для маріупольців
вулицею Ярославів Вал, яка є однією з найстаріших в столиці, зберігає безліч
таємниць і багата на унікальні будівлі. 🏰💓 

Під час екскурсії учасники ознайомилися з історією найбільш старовинної
оборонної і сакральної пам'ятки архітектури Золоті ворота, легендарним будинком
Михайла Підгорського, домом барона М. Штейнгеля,  будинками, в яких жили
видатні лікарі України В. Образцов та М. Стражеско. ✨️🤩 Також екскурсанти
послухали розповідь про історію Караїмської кенаси, яка побудована у
дивовижному мавританському стилі. 🕍😊 До того ж, маріупольці побачили
дивовижного ворона Києва Крума, який вразив всіх своїм розміром. 💝🦅 Послухали
історію красивих будинків Родзянка та дізналися в яких будинках на вулиці
Ярославів Вал розташовувалася Українська академія наук, Українська державна
академія мистецтв, де жили Леся Українка та видатний авіаконструктор
українського походження Ігор Сікорський.☀️🤗 

Маріупольці залишилися задоволеними екскурсію та з нетерпінням чекають на нові заходи. 

🥰 Наступна екскурсія відбудеться 5 травня о 13 годині. 

❗️Реєстрація за телефоном: 097 196 19 66 

\#РодинаМДУ  \#Україна \#Маріуполь \#Київ  \#futurestartswithyou  

\#mdu\_cultural\_studies \#МДУ  \#нашівикладачі \#заходи
