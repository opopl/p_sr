% vim: keymap=russian-jcukenwin
%%beginhead 
 
%%file 06_11_2021.fb.bilchenko_evgenia.2.v_shest_chasov_vechera
%%parent 06_11_2021
 
%%url https://www.facebook.com/yevzhik/posts/4412703422098067
 
%%author_id bilchenko_evgenia
%%date 
 
%%tags bilchenko_evgenia,poezia
%%title БЖ. В шесть часов вечера после всего
 
%%endhead 
 
\subsection{БЖ. В шесть часов вечера после всего}
\label{sec:06_11_2021.fb.bilchenko_evgenia.2.v_shest_chasov_vechera}
 
\Purl{https://www.facebook.com/yevzhik/posts/4412703422098067}
\ifcmt
 author_begin
   author_id bilchenko_evgenia
 author_end
\fi

\ifcmt
  ig https://scontent-frt3-2.xx.fbcdn.net/v/t1.6435-9/252846836_4412703375431405_9171860503484424892_n.jpg?_nc_cat=103&ccb=1-5&_nc_sid=8bfeb9&_nc_ohc=S3XQ-j_Zv1MAX_ydzwa&_nc_ht=scontent-frt3-2.xx&oh=e45a321566a8db0c27bdf5d862b88640&oe=61AB11FC
  @width 0.4
  %@wrap \parpic[r]
  @wrap \InsertBoxR{0}
\fi

БЖ. В шесть часов вечера после всего
Зорге, привет. Мне кажется, это - сон.
Я ещё в сюре. Ты не серчай, того.
Чай и конфеты, новый домашний тон...
Речь без подтекстов - редкое торжество.
Зорге, привет. Как будто бы - "сорок пять"
Бабочкой золочёной стучит в висках.
Не победили мы, но и нас распять,
Так, чтобы насмерть, - хули: кишка тонка.
Зорге, привет. К вот этим вот на холо-
дильник магнитикам, - нет, я не шёл, - я грёб.
Я выносил их через такое зло,
Что на сегодня мне ни еда, ни гроб
Так не важны, как то, что старлей дополз
Там, где ни влёт не клеится, ни ползком.
Зорге, привет. Троллейбус - давно в депо...
Через проспект Вселенной бредём пешком.
6 ноября 2021 г.
