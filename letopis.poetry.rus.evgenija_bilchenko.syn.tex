% vim: keymap=russian-jcukenwin
%%beginhead 
 
%%file poetry.rus.evgenija_bilchenko.syn
%%parent poetry.rus.evgenija_bilchenko
 
%%url https://t.me/bilchenko_z/120
%%author 
%%tags 
%%title БЖ. Возвращение блудного сына
 
%%endhead 

\subsubsection{БЖ. Возвращение блудного сына}
\label{sec:poetry.rus.evgenija_bilchenko.syn}

\Purl{https://t.me/bilchenko_z/120}
\Pauthor{Быльченко, Евгения}

Какое же это счастье! Иметь над собой Отца.
Отец - огромное целое. Лик без личин лица.
Он - твой единый Дух. Он с тобой до конца.
И он всё прощает: и биполярки, и неврозы, и "гоп-цаца".

Он - постоянно рядом. Но не видишь его никак.
Он говорит: "Люблю", - даже если, слепой дурак,
Ты сам себя губишь и подыхаешь. В детстве случилось так,
Что папка юный не жил с семьёй, а дед говорил: "Кулак!"

"Покажи им кулак!", - говорил мне дед, погоны и три звезды.
Когда он водил меня в детский сад, нянька ждала беды.
Но дед был добр - и за справедливость. Боксёр, он давал под дых
Всем, кто обидел внучку его, а потом лишь вникал в суды.

Не стало деда - не стало внучки. Дождь моросит на лист
Последний в сквере... Меня убейте, но я - за патронализм.
Я - слишком свободен для сей свободы: демократия мне, - как глист:
Попал в кишечник и мерзко точит - праведных душ колит.

Колет меня лицемерие толерантности, точит и клонит в сон.
Не хочу идентичности, не хочу либеральной иронии and so on.
Когда васильковой кленовой гарью светится небосклон,
Я надеваю робу Отца, сшитую из икон

Искрящихся глаз, голубых одежд, окладов златых и риз.
Отец - молчаливо, предельно строг, плевал он на мой каприз.
Но когда я падаю вниз, - так безнадёжно вниз, -
Он выходит наружу и подаёт мне руку через карниз.

Посему я умру без страха. Не когда захочу, а в срок.
Отец наденет халат врача. Бога попросит: "Док!
Она не прошла ещё сто дорог, у нее - атрофия ног", -
И внемлет Бог моему Отцу, на то Он - и дедов Бог.

29 октября 2020 г.
