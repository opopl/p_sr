% vim: keymap=russian-jcukenwin
%%beginhead 
 
%%file 26_02_2022.fb.fb_group.story_kiev_ua.2.scho_u_vas_novogo
%%parent 26_02_2022
 
%%url https://www.facebook.com/groups/story.kiev.ua/posts/1869747373222055
 
%%author_id fb_group.story_kiev_ua,veksler_sergej.vostochnyj_ierusalim
%%date 
 
%%tags 
%%title Что у вас нового? Переводжу мій фейсбук на мою мову
 
%%endhead 
 
\subsection{Что у вас нового? Переводжу мій фейсбук на мою мову}
\label{sec:26_02_2022.fb.fb_group.story_kiev_ua.2.scho_u_vas_novogo}
 
\Purl{https://www.facebook.com/groups/story.kiev.ua/posts/1869747373222055}
\ifcmt
 author_begin
   author_id fb_group.story_kiev_ua,veksler_sergej.vostochnyj_ierusalim
 author_end
\fi

Что у вас нового?

Переводжу мій фейсбук на мою мову. Не можу більше бачити \enquote{что у вас нового?},
як це було написано раніше.

\enquote{Зачекайте. Налаштовиваємо українську мову}.

\ii{26_02_2022.fb.fb_group.story_kiev_ua.2.scho_u_vas_novogo.pic.1}

Нічого не змінилося. Мій Київ. Мій! І ніколи йому не стати Києвом Скабєєвої і
Жириновського, Києвом питерской подворотни.

Нічого не змінилося. Фашисти залишилися тими ж самими фашистами, не дивлячись
на те, якіми прапорами прикриваються, які слова і формулировки вони вживають
зараз для виправдовування своїх злочинів. 

\ii{26_02_2022.fb.fb_group.story_kiev_ua.2.scho_u_vas_novogo.pic.2}

«фашисты будущего будут называть себя антифашистами» - Черчілль знав, про кого
він говорить. Нажаль, його слова здійснюється.

Завдяки путіну знатиму краще Мову.

Відтепер фейсбук запитує:

«Що у вас на думці?» не \enquote{что у вас нового?}

Не фотография: світлина. Тепер так.

* \enquote{Анна Липей поширює ваш допис.}

Що у вас на думці? - тепер так питає мій фейсбук, українською мовою.

Щойно розмовляв з Києвом. З Галею.

Ми разом. Ми разом з першого класу.

Кожен день на зв'язку (останні роки навіть більше - завдяки можливостям, які
подарував час).

Я - українською мовою (нічого дивовижного тут немає), Галя російською.

Згадав про 46 ночних обстрілів, які додали у іврит російські слова \enquote{ракета} і
\enquote{скад}: саме тоді, узимку 1991-го, щоночі надходили \enquote{подарунки}, які нагадували
про військову могутність \enquote{батьківщини} (тоді вони були частиною сили Саддама
Хусейна).

Зараз той саме лютий, але рік 2022...

Сумно і прикро. 

Київ вистоїть. Українці, тримайтеся. Ми з вами (мова, яка пов'язана з скадами і
їх зброєю, тут недоречна).

Саддам чекає на путіна.

\textbf{\#тримаймося!}

Сьогодні. Президент української держави поруч з будинком з химерами архітектору
Владислава Лешека Городецького понад театром Соловцова, врятованим Лазарем
Ізраїльовичем Бродським (театр Івана Франка).

Мій Київ.

Мій Ізраїль сьогодні прокинувся таким. З золотом пшениці і з блакитним небом.
Країна моїх співдумців і співвітчизників тепер під українським прапором.

Друзі, які не думали ще вчора, що їх щось якось пов'язує з Україною, сьогодні
вони Україна. Ми разом.

Небайдужі до розгулу зла.

І сьогодні, як ніколи, відчуваю, що Київ і Єрусалим - це одне Місто. З великої
літери, як це писалося рукою Майстера, у Михайла Опанасовича Булгакова, і
тільки у двох випадках - Місто, коли йдеться про Київ і Єрусалим.
