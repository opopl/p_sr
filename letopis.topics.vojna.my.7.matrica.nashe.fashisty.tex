% vim: keymap=russian-jcukenwin
%%beginhead 
 
%%file topics.vojna.my.7.matrica.nashe.fashisty
%%parent topics.vojna.my.7.matrica.nashe
 
%%url 
 
%%author_id 
%%date 
 
%%tags 
%%title 
 
%%endhead 

\paragraph{21:18:25 20-08-22 Кира Кира}
%img Screenshot from 2022-08-20 21-16-43.png

21:00
Есть ли фашисты во власти в Украине?
Если коротко, нет.
Более 150 ученых из различных институтов, изучающих геноцид и нацизм, заявили, что риторика российских властей не подтверждается фактами.
Фашистская идеология в Украине напрямую запрещена законом, но даже правые и крайне правые партии, которые часто называются фашистскими российской пропагандой, не имеют сколь угодно значимой популярности.
В 2019 году украинские националистические партии и радикальные организации выдвинули единого кандидата на пост президента Украины. В итоге он набрал 1,6% голосов.
Победивший, на выборах с 74% голосов Владимир Зеленский имеет еврейские корни. Во время Второй мировой войны три родственника Зеленского стали жертвами Холокоста, а его родной дед прошел войну в составе Красной армии и был награжден боевыми орденами.
На последних парламентских выборах в Украине (в 2019 году) коалиция националистических и правых партий (в том числе признанная экстремистской и запрещенная в России организация "Правый сектор&quot;-) смогла набрать только 2,15% голосов, вообще НЕ ПРОЙДЯ В РАДУ . Праворадикальная партия "Свобода" получила в парламенте лишь ОДНО место по мажоритарному округу.
Бывший лидер "Правого сектора" Дмитрий Ярош лишился своего депутатского мандата.
Более того, поддержка населением радикальных партий Украины стабильно снижается, начиная с 2012 года. Тогда на выборах в Верховную раду партия "Свобода" набрала 10,44%. На выборах 2014 года радикальные партии в сумме получили 6,4%. В 2019 году - только 2,15%.
Поэтому утверждение наличия нацизма в Украине является всего лишь продуктом путинской пропаганды, рассчитанной на обработанных граждан, лишённых способности мыслить, желания искать разностороннюю информацию...
https://www.bbc.com/russian/features-60606430.amp.
