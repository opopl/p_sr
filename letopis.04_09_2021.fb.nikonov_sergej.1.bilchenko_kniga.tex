% vim: keymap=russian-jcukenwin
%%beginhead 
 
%%file 04_09_2021.fb.nikonov_sergej.1.bilchenko_kniga
%%parent 04_09_2021
 
%%url https://www.facebook.com/alexelsevier/posts/1574022042943105
 
%%author_id nikonov_sergej
%%date 
 
%%tags bilchenko_evgenia,kniga,kultura,literatura
%%title БЖ. Выход моей книги в "Алетейе" (СПб)
 
%%endhead 
 
\subsection{БЖ. Выход моей книги в \enquote{Алетейе} (СПб)}
\label{sec:04_09_2021.fb.nikonov_sergej.1.bilchenko_kniga}
 
\Purl{https://www.facebook.com/alexelsevier/posts/1574022042943105}
\ifcmt
 author_begin
   author_id nikonov_sergej
 author_end
\fi

Главная новость от Евгении Витальевны Бильченко на сегодня. Ее печатаю без выражения отношения. 

Евгения Бильченко.

БЖ. Выход моей книги в "Алетейе" (СПб).

\ii{04_09_2021.fb.nikonov_sergej.1.bilchenko_kniga.pic}

Могу сказать, братия, что такое Победа. Для меня это Победа. Вчера в
Санкт-Петербурге в знаменитом издательстве "Алетейя", которое, к тому же,
лучшее в сфере культурологии, вышла из печати моя книга "СЕНТИМЕНТАЛЬНОЕ
НАСИЛИЕ ЛИБЕРАЛИЗМА: ОТ ШОКА К КИТЧУ". Я в самых тайных мечтах не могла себе
представить, кто станет авторами моего предисловия. Чудесные научные статьи
написали: знаменитый философ, доктор философских наук, профессор, авторитет
культурологов Всея Руси, включая педуниверситет Драгоманова, Григорий Львович
Тульчинский, заслуженный деятель РФ, и писатель, публицист, учёный Александр
Куприянович Секацкий, лауреат премии Андрея Белого. В книге также присутствует
самое тонкое литературное предисловие Стефании Даниловой.

Сразу хочу сказать своим оппонентам, что меня не "подхватили': общий замысел
издать эту книгу, теперь дополненную, с главой "Алетейи" Игорем Савкиным, у нас
возник во время моей последней поездки в Питер в феврале 2020 года. Летом я
участвовала в окончательной корректуре. И вот - радость Победы. Книга уже
продается в Москве в "Фаланстере". Кому интересно: гонораров я не имею, все
права на русскую версию книгу переданы "Алетейе".

Есть и ещё новости: книгу уже переводят на английский и публикуют в Словакии
наши научные коллеги и единомышленники. На фото - наш первый покупатель,
аспирант Александра Секацкого. Огромное сердечное спасибо: моему другу Игорю
Савкину, уважаемым мной и миром учёным Григорию Львовичу, Александру
Куприяновичу, родному поэту Стефании Даниловой, "Алетейе", Питеру и всем, кто
сделал меня на миг счастливой. Это круто, братья. Правда. Спасибо за честь
причастности к большому культурологическом у делу. Ну, и работаем.
