% vim: keymap=russian-jcukenwin
%%beginhead 
 
%%file 27_11_2020.fb.volga_vasilii.1.kursovaja_po_miloserdiju
%%parent 27_11_2020
 
%%url https://www.facebook.com/Vasiliy.volga/posts/2763481943969220
 
%%author Волга, Василий Александрович
%%author_id volga_vasilii
%%author_url 
 
%%tags 
%%title КУРСОВАЯ ПО МИЛОСЕРДИЮ
 
%%endhead 
 
\subsection{Курсовая по Милосердию}
\label{sec:27_11_2020.fb.volga_vasilii.1.kursovaja_po_miloserdiju}
\Purl{https://www.facebook.com/Vasiliy.volga/posts/2763481943969220}
\ifcmt
	author_begin
   author_id volga_vasilii
	author_end
\fi

Я заканчиваю свой первый год обучения на Высших Богословских Курсах в Духовной
Академии Киево-Печерской Лавры. Жаль, конечно, что в наше сумасшедшее и
наполовину безбожное время, во время денег и насилия, очень немногие люди
способны почувствовать важность того, что мне сегодня кажется таким сверхважным
и таким сверх актуальным. 

Души у нас огрубели и осуетились.

Одна из дисциплин, по которой я пишу сегодня курсовую работу, называется
Нравственное Богословие. Тема моей курсовой - «Милосердие». И вот уже целый год
я сталкиваюсь с уникальным явлением, о котором мне хочется рассказать каждому
человеку вообще и каждого убедить почувствовать это так же, как чувствую
сегодня я сам. Ведь стоит только серьезно задуматься, что в самом центре Киева
есть Высшее Учебное заведение, где работают люди, которые, используя научный
метод, изучают Добро, Милосердие, Любовь, Сострадание, Благость; люди, которые
стандартизируют законы духовного мира, изучая опыт тех, кто смог победить зло в
себе, чтобы этот опыт стал достоянием других людей, чтобы каждый смог пойти по
пути Добра и Любви, а задумавшись над этим  можно серьезно изменить своё
отношение к образованию вообще.

Когда-то мне казалось, что важнее физики ничего нет. Что физика, изучая законы
материального мира, одна может сделать человечество сытым и довольным и люди не
будут ненавидеть друг друга и друг друга убивать. 

Потом уже, когда я стал изучать социологию и экономику, мне казалось, что важно
знать законы общественного развития, что важно (и главное --- возможно) правильно
настроить экономику и справедливо распределять меду людьми, то, что будет
давать физика, и что вот тут все станут довольны и счастливы, и каждый будет
рад каждому другому человеку, ибо все живут и трудятся друг ради друга. 

Но так мне казалось только в самом начале моих занятий общественными науками.
Потом науки эти меня разочаровали. Они научили меня тому, что люди, в их
социологических масштабах, руководимы исключительно стремлением к власти и
обогащению, что при построении любой экономической модели необходимо учитывать
именно эту тягу человека к власти и наживе. Один знаменитый немецкий философ
даже дал определение самой жизни, как воли к власти, а серьезные ученые,
которые учили меня социологии и экономике, такие как Пахомов Юрий Николаевич,
Масол Виталий Андреевич и другие, убеждали меня в том, что сама природа
человека не позволит человеку построить справедливое общество.

\ifcmt
pic https://scontent.fiev6-1.fna.fbcdn.net/v/t1.0-9/128173317_2763481920635889_4150323824730672389_n.jpg?_nc_cat=107&ccb=2&_nc_sid=730e14&_nc_ohc=r545i-gna6UAX8xv_Ui&_nc_oc=AQkQgwGciKo_0yj23OzeTIZaO7pDuxctW6v_a6kK0oK7WLZyMXn4wSrw5edYZBZLEGQ&_nc_ht=scontent.fiev6-1.fna&oh=23653406997ebf44dc3b39e8cb423de2&oe=5FEBE910
\fi

И вот сегодня я пишу курсовую по «Милосердию». 

Прожив полвека на земле я все сильнее начинаю осознавать, что слишком поздно я
начинал это осознавать)) Осознавать то, что нет более важной науки, чем наука о
добром сердце. Ибо без этого и физика не в прок, и экономика с социологией
бесполезны. Если люди не захотят учиться милосердию, то Эра Милосердия не
настанет никогда.
