% vim: keymap=russian-jcukenwin
%%beginhead 
 
%%file 05_02_2019.stz.news.ua.mrpl_city.1.fotomystectvo_mrpl_ekskurs_v_mynule
%%parent 05_02_2019
 
%%url https://mrpl.city/blogs/view/fotomistetstvo-mariupolya-ekskurs-u-minule
 
%%author_id demidko_olga.mariupol,news.ua.mrpl_city
%%date 
 
%%tags 
%%title Фотомистецтво Маріуполя: екскурс у минуле
 
%%endhead 
 
\subsection{Фотомистецтво Маріуполя: екскурс у минуле}
\label{sec:05_02_2019.stz.news.ua.mrpl_city.1.fotomystectvo_mrpl_ekskurs_v_mynule}
 
\Purl{https://mrpl.city/blogs/view/fotomistetstvo-mariupolya-ekskurs-u-minule}
\ifcmt
 author_begin
   author_id demidko_olga.mariupol,news.ua.mrpl_city
 author_end
\fi

\ii{05_02_2019.stz.news.ua.mrpl_city.1.fotomystectvo_mrpl_ekskurs_v_mynule.pic.1}

Після ознайомлення з \href{https://archive.org/details/19_01_2019.olga_demidko.mrpl_city.sosnovsky_take_krasyve_zhyttja}{%
життям талановитого і унікального фотохудожника Маріуполя Євгена Сосновського}
\footnote{Маріуполець Євген Сосновський: таке красиве життя, Ольга Демідко, mrpl.city, 19.01.2019, %
\url{https://mrpl.city/blogs/view/mariupolets-evgen-sosnovskij-take-krasive-zhittya}, %
Internet Archive: \url{https://archive.org/details/19_01_2019.olga_demidko.mrpl_city.sosnovsky_take_krasyve_zhyttja}%
}

хочеться знову і знову розглядати найбільш яскраві фото
нашого міста і його мешканців. Пропоную дізнатися, як розвивалося фотомистецтво
Маріуполя в минулі століття. До речі, ця тема є недостатньо вивченою і чекає на
свого дослідника. Побіжно діяльність фотографів Маріуполя розглядав краєзнавець
Аркадій Проценко та журналістка Таїсія Коваль.

Завдяки збереженим фото минулих століть можна визначити багато характерних
особливостей, які сьогодні не є визначальними. Так, у нижній частині знімка
золотим тисненням позначалися прізвища майстрів (І. Куюмджи, А. Стояновський,
М. Улахов, С. Полоцький, М. Краснопольський, А. Целінський, І. Шляпов, А.
Рубанчик та інші), а на зворотному боці – фірмові знаки фотографій.

\ii{05_02_2019.stz.news.ua.mrpl_city.1.fotomystectvo_mrpl_ekskurs_v_mynule.pic.2}

\textbf{Читайте також:} \emph{Фото с девочкой в греческом костюме из Сартаны победило во всеукраинском конкурсе}%
\footnote{Фото с девочкой в греческом костюме из Сартаны победило во всеукраинском конкурсе, Яна Іванова, mrpl.city, 04.02.2019, %
\url{https://mrpl.city/news/view/foto-s-devochkoj-v-grecheskom-kostyume-iz-sartany-pobedilo-vo-vseukrainskom-konkurse-foto}
}

Наприкінці XIX – на початку XX ст. фотографи працювали з громіздким
стаціонарним фотоапаратом, зі скляними бромно-срібними пластинками формату
13х18 сантиметрів. Перешкодою для гарних знімків ставали стовпи та дроти, що
часто погіршували композицію знімка. Тому при підготовці негативів до друку
майстри широко використовували ретуш, незважаючи на те, що, на їхню думку, вона
шкодила фотомистецтву. Проте у подальшому ретуш вважалася важливим компонентом
фотомайстерності.

\ii{05_02_2019.stz.news.ua.mrpl_city.1.fotomystectvo_mrpl_ekskurs_v_mynule.pic.3}

На тогочасних фото можна було побачити життя маріупольських вулиць, яскраві
архітектурні будівлі, собори. У кожного фотографа був свій стиль, власний
почерк, від якого залежала оплата роботи. У цьому матеріалі можна побачити
світлини різних фотографів. Так, завдяки збереженим фото І. Куюмджи можна
стверджувати, що знімки виконані вміло, професійним апаратом. Всі вони тоновані
в коричневі або сталеві тони, наклеєні на бристольський картон і оформлені
віньєтками, як це робили тоді всі без винятку професійні фотографи. Майстерня
Куюмджи мала найбільшу кількість замовників. Крім портретів, фотограф займався
і жанровою зйомкою. У 1989 році І. Куюмджи був нагороджений медаллю \enquote{За
старанність} і особистою подякою від Олександра III. Аркадію Проценку вдалося
встановити, за якою ціною продавалися фотографії І. Куюмджи: групові – по три
рублі, персональні – по два (на той час робочий отримував в місяць тридцять –
сорок рублів).

\textbf{Читайте також:} \emph{Мариупольцев приглашают на туристическую экскурсию по городу}%
\footnote{Мариупольцев приглашают на туристическую экскурсию по городу, Олена Онєгіна, mrpl.city, 04.02.2019, \url{https://mrpl.city/news/view/mariupoltsev-priglashayut-na-turisticheskuyu-e-kskursiyu-po-gorodu-foto}}

\ii{05_02_2019.stz.news.ua.mrpl_city.1.fotomystectvo_mrpl_ekskurs_v_mynule.pic.4}

Інший фотограф М. Улахов відрізнявся педантичністю у виборі ракурсу для зйомки.
До наміченого місця він приходив без камери: дивився, як в різний час доби
висвітлюється об'єкт зйомки. Зупинившись, нарешті, на одному з варіантів,
Улахов приходив вже з фотокамерою. І тут починалося найскладніше. Фотограф
знаходив потрібну йому точку, потім, без перебільшення, годинами дивився в
приготований до роботи апарат, терпляче чекав, коли з'явиться сонце, стихне
вітер або розсіється туман. Фотограф виконував кілька дублів, міняв експозицію,
а потім вибирав один негатив – той, який відповідав високим вимогам.

\ii{05_02_2019.stz.news.ua.mrpl_city.1.fotomystectvo_mrpl_ekskurs_v_mynule.pic.5_6}

Широкою популярністю користувалася фотомайстерня фотографа А. Стояновського,
яка знаходилася на вулиці Харлампієвській, 19, в будинку купця П. Попова. До
речі, збереглося фото П. Стояновського від 1907 р., на якому зображений П.
Попов, ймовірно, з дружиною. У фотографа були власні замовники. П. Стояновський
з великим захопленням проводив досліди з освітленням, шукав нові способи
обробки негативів і домігся успіху. У 1907 році його робота була удостоєна
золотої медалі.

У павільйонах фотографи знімали при денному світлі. Скляна стеля, закрита
рухомою шторою, скляна стіна на північ, рівне світло похмурого дня. У денному
павільйоні працювали в основному аж до 1915 року. У кожного фотографа була своя
вітрина, в якій він показував, як то кажуть, товар обличчям. Завдяки вітринам
створювалося враження й уявлення про роботу майстра, його стиль.

\textbf{Читайте також:} \emph{Раскрыт секрет яйца из Intagram, побившего мировой рекорд по лайкам}%
\footnote{Раскрыт секрет яйца из Intagram, побившего мировой рекорд по лайкам, mrpl.city, 04.02.2019, %
\url{https://mrpl.city/news/view/raskryt-sekret-yajtsa-iz-intagram-pobivshego-mirovoj-rekord-po-lajkam-foto}
}

Фотографи в Маріуполі були різні: одні догоджали смакам обивателів,
перетворюючи свою роботу у звичайне ремесло, інші від\hyp{}різнялися високим
професіоналізмом, що вплинуло на розвиток фотомистецтва у Маріуполі. Загалом
завдяки старим фото можна зазирнути в минуле, відновити зовнішній вигляд
Маріуполя, познайомитися з його мешканцями та пережити моменти, які вже ніколи
не повторяться.

Фото, наведені у статті, використані з наступних джерел: 

\begin{itemize}
\item посилання 1, \footnote{\url{https://etoretro.ru/city2629.htm}}
\item посилання 2, \footnote{\url{http://forum.gp.dn.ua/viewtopic.php?f=34&t=720&start=0}}
\item посилання 3. \footnote{\url{http://vse-grani.com/viewtopic.php?f=56&t=514&start=60}}
\end{itemize}

\ii{05_02_2019.stz.news.ua.mrpl_city.1.fotomystectvo_mrpl_ekskurs_v_mynule.pic.7}
% 8-9
\ii{05_02_2019.stz.news.ua.mrpl_city.1.fotomystectvo_mrpl_ekskurs_v_mynule.pic.8}
\ii{05_02_2019.stz.news.ua.mrpl_city.1.fotomystectvo_mrpl_ekskurs_v_mynule.pic.10}
