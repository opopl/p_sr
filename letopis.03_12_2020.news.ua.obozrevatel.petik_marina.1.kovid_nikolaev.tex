% vim: keymap=russian-jcukenwin
%%beginhead 
 
%%file 03_12_2020.news.ua.obozrevatel.petik_marina.1.kovid_nikolaev
%%parent 03_12_2020
 
%%url https://www.obozrevatel.com/society/lyudi-bukvalno-polzut-k-oknam-chto-proishodit-v-kovidnyih-bolnitsah-nikolaevschinyi.htm
 
%%author Петик, Марина
%%author_id petik_marina
%%author_url 
 
%%tags 
%%title "Больные кричат, у них истерика": врачи и пациенты откровенно рассказали о ситуации с COVID-19 на Николаевщине. Эксклюзив
 
%%endhead 
 
\subsection{\enquote{Больные кричат, у них истерика}: врачи и пациенты откровенно рассказали о ситуации с COVID-19 на Николаевщине. Эксклюзив}
\label{sec:03_12_2020.news.ua.obozrevatel.petik_marina.1.kovid_nikolaev}
\Purl{https://www.obozrevatel.com/society/lyudi-bukvalno-polzut-k-oknam-chto-proishodit-v-kovidnyih-bolnitsah-nikolaevschinyi.htm}
\ifcmt
	author_begin
   author_id petik_marina
	author_end
\fi

\index{Коронавирус!Украина!Николаев, истерика у больных, 03.12.2020}

\begin{itemize}
  \item В больнице Южноукраинска больными COVID-19 заняты все 64 койки
  \item В медучреждениях не хватает врачей и медсестер, их смены порой длятся по несколько дней
  \item Медики пытаются уволиться, чтобы не рисковать здоровьем
\end{itemize}

\ifcmt
pic https://i.obozrevatel.com/news/2020/12/2/ee3-2.jpg?size=972x462
\fi

\begin{leftbar}
	\bfseries\em
В Николаевской области в некоторых больницах уже заняты все места для больных
коронавирусом. Чтобы хоть как-то спасти положение, в регионе готовятся
закрывать детские и гинекологические отделения, и открывать на их базе места
для пациентов с COVID-19. Ситуация ухудшается с каждым днем.

На Николаевщине более 500 инфицированных на 100 тысяч населения, когда при
\enquote{норме} показатель равен 40. Волонтеры, власть и НСЗУ спешно закупают кислород
и кислородные концентраторы, которых катастрофически не хватает. А врачи и
пациенты рассказывают, что на самом деле происходит внутри больниц.
\end{leftbar}

\subsubsection{Больницы заполнены на 100\%}

В Николаевской области есть больницы, заполненные COVID-пациентами на 100\%.
Так, например, в больнице Южноукраинска заняты все 64 койки.

\enquote{Да, у нас есть план расширения, и, как только мест не будет хватать, мы сразу
же будем расширяться за счет гинекологии, детского отделения. Не хотелось бы,
конечно, ущемлять детей, потому что они болеют всегда, но что поделаешь}, –
говорит заместитель главного врача этого медучреждения Ирина Мазникова.

\ifcmt
pic https://i.obozrevatel.com/gallery/2020/12/2/istock-1203187628.jpg
\fi

По ее словам, в больнице хватает кислорода. \enquote{У нас установлены бочки со
сжиженным кислородом и мы каждый день привозим необходимое количество газа.
Тяжелых пациентов очень много, всем нужен кислород. Также мы планируем
установить еще 5-тонную бочку под жидкий кислород. Ждем 16 декабря поставки
оборудования. Еще 15 концентраторов нам обещало управление здравоохранения. И
если придется открывать новые места для больных, то все они будут обеспечены
кислородом}, --- говорит Мазникова.

Также больнице помогают волонтеры. Своими силами они тут ремонтируют один из
этажей, который пойдет для госпитализации больных коронавирусом.

Сейчас главная проблема в этой больнице --- нехватка медперсонала. Если раньше
выкручивались за счет частных семейных врачей, которые подрабатывали в
медучреждении, то теперь из-за наплыва больных они не могут тут работать.

Сложная ситуация и в больнице Баштанского района. Здесь из 25 коек, выделенных
под коронавирус, на сегодня заняты все.

\ifcmt
pic https://i.obozrevatel.com/gallery/2020/12/2/bashtanskaya-rajonnaya-bolnitsa.jpg
\fi

\enquote{У нас сегодня выявлено 22 новых случая, в понедельник --- 21. Число больных
очень быстро растет. Если все койки у нас заняты, то пациентов везут за 36 км в
соседний район. Там пока заполняемость 40\%. Никто без помощи не остается}, –
заверили в этой больнице.

Тут тоже утверждают, что сейчас ситуация с кислородом наладилась, из 25 коек 20
имеют кислородную точку, а еще 5 на концентраторах. Сам газ сюда поставляют из
Днепра.

\enquote{У нас также, как и везде, не хватает анестезиологов, нет инфекциониста, вместо
него работает терапевт. Многие медики болеют}, --- отметили в медучреждении.

Непростая ситуация и в самом Николаеве. В городской больнице №1 занято 96,7\%
коек, это 145 мест из 150. В Николаевском областном центре лечения инфекционных
болезней из 110 мест заняты 96 (87,3\%). В Николаевской больнице №5 --- 87,4\%, а
в больнице №3 --- 76,5\%.

\subsubsection{\enquote{Врачи и медсестры выгорают на работе}}

В целом ситуация на Николаевщине с обеспечением кислородом не очень радужная.
Всего лишь треть коек тут имеет доступ к такому важному газу. Об этом сообщили
в Николаевской областной администрации. Из 1557 коек кислородные точки имеют
только 502. Пока спасаются кислородными концентраторами, их всего 225 на
область. Но понятно, что для тяжелых пациентов они не подойдут, им необходим
кислород под высоким давлением.

\ifcmt
pic https://i.obozrevatel.com/gallery/2020/12/2/istock-12031876281.jpg
\fi

COVID-больницы области постоянно ищут специалистов: анестезиологов,
реаниматологов.

\enquote{Врачи и медсестры просто выгорают на работе. Смена длится 2-3 дня. Они спят
порой на полу. Очень много тяжелых пациентов, которые боятся умереть. Им нужен
не только кислород, с ними нужно сидеть, уговаривать не стаскивать маску, чтобы
не задохнуться. Рассказывать, что все с ними будет хорошо, вытирать слезы.

Что бы не рассказывали, но ситуация с кислородом более-менее в тех городах и
районах, где есть промышленные предприятия, которые взялись помогать больницам
с покупкой концентраторов и организацией кислородных точек. В остальных все,
как и везде. Нужно менять кислородные маски: кому-то стало легче, дай подышать
другому. Один лежит с сатурацией ниже 80 и смотрит, как его сосед дышит
кислородом. Люди буквально ползут к окнам, чтобы открыть и подышать. Это
невыносимо}, --- жалуется медсестра реанимационного отделения одной из районных
больниц.

Одна из бывших пациенток COVID-больницы Николаева рассказала, как она лежала
там. \enquote{Я ничего не могу сказать плохого о медперсонале. Они делают то, что
могут. Но их мало. Они не могут подойти ко всем, кому требуется. А больной не
понимает, он кричит, у него истерика, ему плохо, а кислорода на всех нет. Порой
дышали несколько человек по очереди по 10-15 минут от одной точки.

Если кто-то задерживал маску на минуту, то и до рукопашной доходило. Потому что
в палате на 8 человек только 3 точки. А потом кто-то умирал, его клали в мешке
в коридоре и перестилали постель, тут же клали другого пациента. Это очень
тяжело психологически выдержать. Еще вчера твоя соседка по телефону с дочкой
разговаривала, а утром ее уже нет. Не спасли. Умирают, когда два, когда три
человека за день. Я сама думала, что не выживу --- так плохо было, но врачи
спасли, спасибо им}, --- рассказывает женщина.

\subsubsection{Чиновники отказываются оплачивать смерть врача от COVID-19}

Что касается медиков, то в сложном положении тут оказались как больницы, так и
семейные врачи.

\enquote{Понимаете, пока не было этого наплыва больных с пневмониями, ОРЗ, гриппом, все
было еще более-менее. Сейчас мой рабочий день превратился в ад. Очереди под
дверями, половина из них не записана на прием, потому что уже нет мест. Телефон
не замолкает во время приема, люди звонят, у них температура, они задыхаются. В
результате получать больше я не стала, а работы прибавилось раза в три.
Пристроить пациента в больницу очень сложно, мест нет. Сделать ПЦР-тест за счет
государства невозможно, люди ждут неделю. У меня телефон звонит даже ночью}, –
жалуется врач Виктория из Николаева.

\ifcmt
pic https://i.obozrevatel.com/gallery/2020/12/2/kisen1.jpg
\fi

В Николаеве ранее произошел вопиющий случай. Чиновники отказали в выплате
денежной компенсации родным семейного врача, которая в октябре умерла от
COVID-19. Галина Колянчикова работала семейным доктором в ЦПСМП №3 Николаева. А
ее муж --- хирург.

Однако официальная комиссия, в которую входили представители городского
управления здравоохранения, городской администрации, Фонда социального
страхования, профсоюза, посчитали, что семейному врачу было 65 лет, она имела
сахарный диабет, поэтому не имела право работать с COVID-больными.

\ifcmt
pic https://i.obozrevatel.com/gallery/2020/12/2/gettyimages-1248492986.jpg
\fi

\enquote{После таких случаев работать не хочется. Мы каждый день рискуем
подхватить инфекцию. Пациенты нас не спрашивают, имеем мы право работать с
COVID-19 или нет. Мои знакомые врачи уже уволились, или ищут такую возможность.
Лучше посидеть пока дома, а когда пандемия стихнет, оформиться ФОПом и вести
свою практику}, --- говорит Виктория.
