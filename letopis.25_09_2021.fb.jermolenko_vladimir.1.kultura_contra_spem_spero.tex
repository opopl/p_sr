% vim: keymap=russian-jcukenwin
%%beginhead 
 
%%file 25_09_2021.fb.jermolenko_vladimir.1.kultura_contra_spem_spero
%%parent 25_09_2021
 
%%url https://www.facebook.com/volodymyr.yermolenko/posts/10158423020123358
 
%%author_id jermolenko_vladimir
%%date 
 
%%tags contra_spem_spero,future,kultura,ukraina
%%title Українська Культура - Contra Spem Spero
 
%%endhead 
 
\subsection{Українська Культура - Contra Spem Spero}
\label{sec:25_09_2021.fb.jermolenko_vladimir.1.kultura_contra_spem_spero}
 
\Purl{https://www.facebook.com/volodymyr.yermolenko/posts/10158423020123358}
\ifcmt
 author_begin
   author_id jermolenko_vladimir
 author_end
\fi

як же багато людей, які творили українську культуру, робили це без усілякої
певності, що вона має майбутнє. Їхнє геройство - передусім у тому, що вони
вигризали це майбутнє в реальності, яка всіма способами їм нашіптувала: це
кінець, фініш, смерть. "Завтра" у вас не буде. І вас теж не буде

значно легше творити, коли ти певен, що в тебе і твоєї спільноти є майбутнє.
Коли ти знаєш, що твоє творіння ти можеш передати у спадок. Значно складніше це
робити, коли ти НЕ знаєш, чи будуть ті, кому ти передаш це у спадок

ампутованість майбутнього - одна з наших великих ран, одна з причин, чому ми
такі слабкі у мисленні про великий час, з горизонтом хоча би в десятки років

але тим більше захоплення викликають наші класики, які творили contra spem
spero,  маючи надію на майбутнє, хоча факти навколо говорили про те, що
майбутнього у нас не буде, що надію треба залишити, що треба переступити поріг
Дантового пекла. Але вони не переступали. Не давали себе затягнути в цей сморід
вічної смерті

Бо пекло - це відсутність майбутнього. Це відсутність надії.

І ці наші класики вигризали це майбутнє в історії. Вгризалися мертвою хваткою в
цю жорстку і гірку плоть часу - і перемагали. Кожна ця битва за майбутнє була
битвою з пеклом. Кожна була битвою за сенс

Вигризати своє майбутнє у реальності, яка робить усе, щоб твоє майбутнє
ампутувати -  мабуть, це одна з рис української культури. Яка повторюється,
over and over again.

\ii{25_09_2021.fb.jermolenko_vladimir.1.kultura_contra_spem_spero.cmt}
