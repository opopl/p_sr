% vim: keymap=russian-jcukenwin
%%beginhead 
 
%%file body
%%parent 

%%endhead 

\ii{articles}

\ii{wikipedia}
\ii{anekdoty}
\ii{letters}

\ii{mariupol}
\ii{kiev}
\ii{harkiv}
\ii{albums}
\ii{daty}
\ii{trg}
\ii{sites}
\ii{posts}

\ii{archive}

%https://www.facebook.com/kyivhistorymuseum/posts/pfbid0nHRWnEMFpcgrLfu4yi92qkaL6KidBUZKjAd8eMFe7YHuAHDGaEsG5CSDCQ3VTmUnl

% расшифровки
\ii{transcripts}

% мое
\ii{moje}

% build
\ii{build}

% переводы
\ii{translations}

% литература
\ii{literatura}

\ii{list}

% города
\ii{goroda}

\ii{stihi}

\ii{citations}
\ii{dvk}
\ii{jumor}
\ii{nenavyst}
\ii{people}
\ii{poetry}
\ii{poslovicy}
\ii{proza}
\ii{saint_russia}
\ii{slovar}
\ii{slova}
\ii{songs}
\ii{topics}
\ii{writers}

\ii{stranaua}
\ii{todo}
\ii{test}

% https://www.facebook.com/marina.sidoniuk/posts/pfbid02jESojKV9QWTtw4pgqTJCnFCkaTU8hcF3rtp6gNxCKT82ifsZYUGERiVvqdbWRz3Bl
% https://www.facebook.com/MAlina.Mykhailova/posts/pfbid0rcSWyZYyHDbptqz2uVJAg1X1nGwUDwP49GVcwqhoj6o3NtXXSsFhwcyQP3ywyiBpl
% https://www.facebook.com/sergey.evsiukov/posts/pfbid02vWiv7W6wGTEGw7LuUBQbANm2MNehpr3xwTzeeaxoKWRAMCqFgWX8ZAaLux7KifRul
% https://www.facebook.com/lepska.kepska/posts/pfbid02vBwThsvoRsEBSUodxGaqd2D6Toa5yxxKMQCB9SkwQw39aYFi9Qm3bJirRwgbQeVXl
% https://www.facebook.com/serjvlk/posts/pfbid02FJyCzRoQHxZpmnbMq2DKsoB5modrg29UayX2cXmLVWQzMfHmP17j7ENyMNnNunhkl
% https://www.facebook.com/havryshko/posts/pfbid0NoxJjeAoajoG6kLKxhAwiPVjWWkRxR1aqb4KD7bTLwthJFXFYBMGEWXd3EE1s43Ul
% https://www.facebook.com/marjana.savka/posts/pfbid02FP1ftsHU17pBXneYntc8sWZP7YzPh2o5wFX1g7Yb4MtHwfWDyqu51vvQ4wbcHh5Jl
% https://www.facebook.com/victor.taran.5/posts/pfbid0Am1dxk4hDUXsdSRboXVoZu2riPzNuFK5vy5sAJovBsWtJaRZC4Ljcm1YzoxzqYLTl
% https://www.facebook.com/vitalii.chepynoga/posts/pfbid02Wg8EyuMfJjB819D8Y7wtYr3MXYn5cF25597dCsJ9vb3QDpuhDyqZam44XbfDNfTLl

% Привіт. Шо там, як ти там, Боже?
% https://www.facebook.com/LiudmylaHorova/posts/pfbid02jSTh1nhrj4sU1CLfz4JhT8Lqra44oS4t48SuFH6mNMvew9HhASFbCwEuZ5pA56pJl

% http://www.barabooka.com.ua/ilyustruyu-knizhki-dlya-ditej-svitlana-baluh

% https://www.facebook.com/dubinskaya999/posts/pfbid0V4KHVWDcwx7Fn8JG6ePxSHvukgRJLARr3ENQe5AmjFAYa6jeGcdYmXMtqJmKeTJol
% https://www.facebook.com/permalink.php?story_fbid=pfbid0LMDg2AZyEuwZcxccC1egBaHUA4UzVnGmHhBpgN7zajJTv7JzhaedWF7vTL3L9uYql&id=100009901380868
% https://www.facebook.com/vladchu/posts/pfbid02EFNpq4gv3tffD6SprTAM8DkB4ohD4ndyCeLczG1LWwYxCXrEHQVuv9Ynn4wen1Dfl

% https://www.facebook.com/groups/853117518526778/posts/926678901170639
% https://www.facebook.com/dmitro.raimov/posts/pfbid0Vsc4YxwLTEYKNDykfp6TjzU71uQt18Aorsi2ux54zf8X21gpjdbivUhdaw6ECoiql

% https://www.facebook.com/AnastasiyaPika/videos/2608610159262568/

%(зараннє вибачайте, нижче речі, за які мене можуть зараз закидати тапками...
%але... ) Ваша Книжка... Ви кажете, чи загине вона чи ні... Я розумію, Ви
%дуже-дуже хвилюєтесь, випускаючи своє творіння у великий світ... Книжка Ваша
%буде жити довго-довго, через покоління ваше Слово буде говорити нашим вже
%онукам і праонукам нащадкам, що і як було... А щодо ваших сумнівів, чи загине
%книга чи ні... Вбити книгу важко... вбити книгу - якщо вже так говорити...
%можуть такі речі - (1) байдужість людська (2) як не дивно, страх її читати. Ось
%(2) мене більш бентежить... оскільки мені дуже сумно бачити коментарі на кшталт
%- моє серце розірветься, якщо я почну читати, або ж.. я куплю, але не знаю, чи
%вистачить сил її читати. Люди добрі, вибачте... але... ми всі так хочемо
%Перемоги... всім серцем хочемо, всі про це кажуть. Але. Перемогу насправді
%здобувають сильні духом. Перемогу нам ніхто на блюдєчке с голубой каємочкой не
%принесе, її треба виборювати важкою працею, кожен день, кожень день... І
%зокрема через духовний розвиток, через переборювання самого себе... Так!
%Перемогу здобувають сильні духом люди, що здатні переборювати свої страхи,
%болі, переборювати в собі відчай, безнадію. Так, сильні духом здобувають
%перемогу! І пані Надія найкращий приклад, оскільки вона дійсно показала всім
%нам приклад, як переборювати свій біль і відчай! Вона пройшла через пекло, як і
%багато інших людей, але ж сил в неї вистачило книжку написати, так! І сил, як
%не дивно, вистачає на багато чого! А щодо страхів... тут лише питання - взяти і
%прочитати книжку, яку ви ж самі і замовили... Якщо вже замовили, то нащо оце
%бідкання, що я не знаю, чи зможу її читати? Якщо вже сказав А... то кажи і Б...
%Може тоді краще було її не замовляти, так, якщо не будете читати? Або ж що,
%книжка прийде і ви її на поличку поставите, чи як? Якщо вже немає сил у вас
%просто взяти і прочитати книжку,  навіть якщо там є важкі речі, то у мене
%просте питання, якщо ви боїтесь взяти в руки всього лиш - книжку - це книжка,
%не ракета і не бомба...   і прочитати, то... як ви збираєтесь перемагати... а
%ви взагалі Перемогу хочете чи ні? 

%Доброго дня! Дуже добре, що Ваша книжка виходить. Це велика подія, особливо з
%тої точки зору, що в Києві, якщо походиш по книгарням - то знайдеш одну-дві
%книги про Маріуполь - і усе... Тому... потрібні книжки про Маріуполь. Так, дуже
%потрібні! Багато-багато книжок. І не лише про жахіття, а про Мирне Місто
%Маріуполь. Про Майбутнє Маріуполя, про Віру в Маріуполь. Багато-багато книжок
%треба! Місту Маріуполю мало однієї книжки чи декількох - як би ці декілька
%книжок не були сильними і правдивими, Місту Маріуполю потрібні сотні, тисячі
%книжок... Так, сотні, тисячі різноманітних книжок... Щоби не провалитись в
%чорну безодню забуття...  І... Книжка називається Надія. Але є ще Віра та
%Любов. Так, Трійця - Надія, Віра, Любов. Тому... побажання 
%(і не тільки Вам, а взагалі всім маріупольцям - 
%бо хто ж краще за маріупольців зможе написати про Маріуполь?). Пишіть нові
%книжки... Не зупиняйтесь!  І... я вчора ходив по центру, Михайлівська площа. Як
%Ви думаєте, що я там побачив... Я недавно писав про фільм Місто 44. До речі...
%Якщо так подумати... 2022 = 2*22 = 44... Так от. Там я... несподівано! Варшаву
%побачив... Так, колись вщент зруйноване місто, а зараз мирне європейське місто
%стоїть поруч з Маріуполем на площі, обіймає його, шле Маріуполю привіт, і
%немовби каже - все буде добре, друже Маріуполь. Мене знищували так само, як і
%тебе. Але я відродилась. Я колись пережила жахіття такого ж масштабу і горя,
%але я відродилась з попелу, і там, де колись палало полум'я і горіли будинки,
%зараз весело граються діти, а замість зруйнованих мостів через Віслу - стоять
%нові, гарні мости... Так, все буде добре, друже Маріуполь, каже тобі Варшава!

%Доброго дня. Щодо російської мови... Я так скажу. Ви напевне знаєте, що зараз
%річниця Майдану, річниця розстрілу Небесної Сотні, річниця Революції
%Гідності... Україна вже назавжди змінилась завдяки Майдану... А я Вам нагадаю,
%з чого почався Майдан. А Майдан, власне кажучи, почався з короткого допису
%російською мовою (!) Мустафи Найєма у фейсбуці - 21.11.2013 - Встречаемся в
%22:30 под монументом Независимости. Одевайтесь тепло, берите зонтики, чай,
%кофе, хорошее настроение и друзей. Перепост всячески приветствуется! Ось такий
%лаконічний запис російською мовою. З нього все і почалось... і небайдужі, так -
%найкращі! люди все дуже добре зрозуміли, що треба робити, одяглись тепло, взяли
%парасольки, каву, хороший настрій і друзів, і все почалось! Як не крути. Ось
%так!

%Ви знаєте... в чомусь Ви праві... а в чомусь я посперечаюсь, щодо доцільності
%витрачання часу. Персонально щодо цією особи я вже все сказав, вона сама завела
%себе в глухий кут, так що дійсно немає сенсу далі з нею спілкуватись. Але...
%Розумієте. Так, таких людей багато... І... подібних людей ще більше в росії...
%мільйони Це поле бою, інформаційний фронт. Це дуже важливий фронт протистояння,
%і тут головна зброя - Слово, Розум. Треба розвитати ці навики, я так думаю, 

%Будь-ласка... Знаєте, це дуже обширна, велика тема насправді. Чи варто
%сперечатись чи ні. Тут ми згадали про Грецію. Маріуполь - грецьке місто...
%Поліс... Знаєте, колись я був у Херсонесі... Дуже-дуже особливе місце! Місце
%сходження цивілізацій та культур, місце - з якого християнство було прищеплено
%на Русь, на Київську Русь... Бо ж Володимир, перш ніж похрестити киян,
%хрестився сам в Херсонесі... Так от. Греція - колиска... колиска Культури,
%Науки, Літератури... У школі вивчають теорему Піфагора, п'ять постулатів
%Евкліда. Дивишся на небо або ідеш у Київський планетарій, бачиш... Як у відомій
%пісні, небо незміряне всипано зорями! Що то за божа краса! Зорі... Сузір'я...
%Велика Медведиця, Мала Медведиця... Давньогрецькі міфи... Все те, що ми маємо
%зараз - технології, наука - колись почало розвиватись у Греції... Греки
%дивились на небо... думали... а чому так, а не інакше? Чому існує щось, а не
%ніщо? Чому? Чому? І... як? Як зробити так, щоби було справедливе суспільство,
%що краще - аристократія чи демократія або ж тиранія? І так далі... і так далі..
%Я до чого. У греків була школа риторики, діалогу (Сократ) - вміння спокійно,
%обгрунтовано перемагати у суперечках, розплутувати щось. Так от. Тут проблема
%не конкретно в цій пані - вона, може, персонально дуже навіть непогана людина,
%а в тому, що часто люди собі вбивають в голову різні ментальні конструкції.
%Вбивають і вірять в це. Конструкція може бути хибна, але люди вперто вірять.
%Тут вже питання до того, як мозок працює на глибинному рівні, чому це так
%стається. І тут потрібні Розум і Слово, так, сильне, влучне Слово, щоби з цим
%боротись. Але звісно, важливо знати, коли і як писати, і коли варто
%зупинитись... А в більш глобальному плані - у нас зараз іде Війна з росією не
%тільки на військовому фронті, а також Війна Ментальна, Ідеологічна,
%Інформаційна, Когнітивна. Глобальна Когнітивна та Інформаційна Війна... І
%персонально в мене є певний досвід систематичного вивчення явища російських
%зомбі (у соцмережі однокласники)... Росіяни дійсно перетворились на стадо диких
%агресивних зомбі... Ох! Ризикую забити все тут своїми думками, вибачте! Хотів
%до речі тут спитати, чи знаєте ви Почепцова Георгія? Він спеціаліст саме по цій
%темі. Я деякий час тому купив його книжки, і він викладав в Маріуполі.
%Наскільки я знаю, він живий,  слава Богу, продовжує займатись тематикою
%інформаційних війн. Важлива тема, у мене тут є багато чим поділитись, але вже
%пізніше якось... Вибачте за стільки слів.

%Дякую! Тоді... ще трохи про Херсонес і Маріуполь... В Херсонесі, місті-державі,
%люди уявляли себе передовсім як Громадяни Херсонесу. Ось навіть клятва у них
%була! Не греки, а перш за все як Громадяни Херсонесу. Так само, як у Київській
%Русі були Кияни, Новгородці, Чернігівці, Галичани (жителі древнього Галича),
%інших міст... Кияни або ж чернігівці перш за все думали про себе як Киян або
%жителів Чернігова, а вже потім як свою приналежність до Русі... Київська Русь -
%скандинави називали її Гардаріка - Країна Міст, Країна Городів! І в цьому
%сенсі... ви писали... ви як маріупольці зараз у розпачі, так, Маріуполь
%зруйновано, вас розкидало по всьому світі... жахливий, трагічний рік! Але...
%історія справжнього, живого, Українського, Маріуполя не закінчилась в 2022
%році. Хоча... зараз на даний момент він залишився у вашій пам'яті, щасливий,
%радісний Український Маріуполь!  І... було питання у Вашому пості. Хто ж ви є?
%А я так думаю. Українці і Громадяни Маріуполя. Так, і Просто і Гордо, Знаєте.
%Незважаючи ні на що! Громадяни Міста Маріуполя, прекрасного найгарнішого Міста
%у Моря! Міста Трамваїв та парка Веселка, Міста Азовсталі та Героїв Полку Азов,
%Міста Щасливого та Трагічного... от колись Олександр Олесь вірша такого
%написав... З Журбою Радість обнялась... а далі ви самі вже продовжите, як
%забажаєте... А щодо Києва... я от киянин. Київ, як і будь яке місто, то живе не
%тільки у фізичному просторі, а у духовному... Півтора тисячі років історії,
%бездонний колодязь історії, з якого можна пити і пити... і який ніколи не
%висохне... і мені от про Київ напевне найбільш подобаються такі рядки із романа
%Павла Загребельного Смерть у Києві ... Київ був повен лагідного Сяйва. Воно
%лилося згори, із спокійного осіннього Неба, високо знесеного над Київськими
%Горами, знизу піднімалося йому навстріч Сяйво зелене, а між зеленим і
%високоголубим тихо плавало Золото Соборів, легко лягала поміж ними перша
%пожовклість листя, і неначе в душу вхлюпувалося оте м'яке світіння, і
%відчувалося, що, входячи в цей Город, мовби стаєш безсмертним... Розумієте...
%Місто, в якому стаєш безсмертним... Немає на Землі іншого Києва, немає більше
%ніде такого Міста! А яка фраза про Місто Маріуполь Вам найбільше до душі?

%о, це знову ти, Дурник! Те, що я написав вище  - це, зокрема, слова з творчості
%(1) Тараса Шевченка, нашого Кобзаря, у нього до речі, скоро День Народження!
%(2) Пісня нашого гурту Океану Ельзи (3) Вічна творчість Григорія Сковороди,
%нашого Філософа. Видно ти в школі замість уроків робив щось інше, не те, що
%треба зазвичай робити на уроках (а я нагадаю, на уроках треба вчитись, а не
%лізти до дівчат за сусідньою партою), то й не знаєш. А дарма не знаєш! Хіба ж
%не про таких як ти Тарас писав - а то залізете на небо, і ви не ви, і я - не я?
%Так що... те що ти просто дурник, мені вже зрозуміло досить добре ))) Просто
%Звичайнісінький Дурник Дурненко... так і буду тебе називати - Мессір Дурник.
%Дурник-Шкурник ))) так що йди  назад в школу вчитись, Дурник!

%https://www.facebook.com/events/719929419509214/?post_id=720187276150095&view=permalink

% https://www.topbestalternatives.com/calameo/
% http://www.moreofit.com/similar-to/www.calameo.com/Top_10_Sites_Like_Calameo/

% https://www.wikihow.com/Download-an-Image-from-Google-Maps
% https://techswift.org/2021/11/11/how-to-download-images-from-google-maps/

% Як правильно 🇺🇦: Актовий зал чи актова зала?
% https://www.facebook.com/tvoyamovafb/posts/pfbid0GgD7emrRtomhYABtmmmq916NUYMZzDfG63HtTymJzptA5u3zEefYmhqag3Wgt8ovl

% 13 декабря 1974 года был совершен самый дерзкий и знаменитый побег из СССР.
% https://www.facebook.com/groups/4173575829344829/posts/6129490433753349/

ох! не вибачайтесь, тут нема за що вибачатись... знаєте... я бачу Вас на
аватарці, Ви така вся усміхнена, просто неймовірне фото... а зараз, побачив
Ваше фото інше, про яке Ви кажете... Знаєте. Я вже казав, що я ніколи не був у
Маріуполі, ніяк із Маріуполем емоційно не пов'язаний був. А це означає, що все,
що пов'язано із Маріуполем, зокрема фото старі, пости, також пости про
блокадний Маріуполь, я відчуваю зовсім по-іншому. Більш того, війна мене
особливо не зачепила в тому сенсі, що... ну короче, як жив в Києві, так і живу.
І таких людей є насправді дуже-дуже багато, яких війна особливо не зачепила...
І все у мене начебто більш-менш добре. І от. Я вже прочитав дуже багато всього
про Маріуполь, і книжку Надії, і багато-постів-щоденників, короче купу всього.
І я скажу так, якщо чесно, після року війни я якось вже звик до всього цього
потоку новин, фото чогось зруйнованого, трагедій мене воно не чіпляє так
емоційно, як з початку війни. Тут це може звучати для когось образливо або
неприродньо, що я кажу, але так воно і є. Якщо брати наприклад Маріуполь, то ті
ж фото довоєнного Маріуполя я сприймаю зовсім по іншому, розумієте. Для Вас.
звичайно, розглядати кожну фотку дуже болісно, а для мене... Короче, я не вірю
в те, що Маріуполь зруйнований. От не вірю і все. От зараз мені скажуть, що я
божевільний, ну да, у мене не всі дома, якщо згадати діалог Аліси із Чеширським
Котом... Може взагалі тут же пожаліються на мій коментар і він зникне... Так,
звичайно, я знаю, скільки всього зруйновано, скільки горя і нещасть, я вже
передивився напевне сотні відео і постів, але емоційно, серцем, я не вірю в це.
От не вірю, що таке сталось. Це напевне звучить парадоксально і дивно, але це
так. А не вірю я тому що живу в Києві, а не в Маріуполі, і тому на першому
місці я все це сприймаю розумом, бо у мене немає того пережитого досвіду, коли
ти знаходишься в пеклі, і коли ти якимся чудом врятувався із того пекла...
втративши все, фактично... А у що ж я вірю, спитаєте Ви? Я вірю в те, що
Маріуполь так само живий, як на фото мирного часу. Оскільки я там ніколи не
був, я ці фото сприймаю майже так само, як наприклад фото інших місць, де я
ніколи не бував. І справа не в мені персонально, що я такий поганий або
неправильний, а справа в тому, що так взагалі працює людська психіка, бо кожен
живе в своїй окремішній реальності, і кожен сприймає через призму свого
пережитого досвіду і своєї власної реальності... І так. Я вірю в те, що зовсім
скоро сльози і сум в Ваших очах знову зміняться на посмішку і радість, я вірю в
те, що Ви знову опублікуєте свою прекрасну книжку про Маріка і Марічку, я вірю
в те, що все Повернеться на Круги Своя. Я вірю в те, що не тільки Маріуполь
буде звільнений військовим шляхом - молимось на ЗСУ!!! - я також вірю в те, що
у ДрамТеатрі у Новому Маріуполі знову будуть ставитись вистави, я вірю в те, що
в Маріуполі знову заграють фонтани, я вірю в те, що оживе парк Веселка, так, я
вірю в це! І я вірю в те, що усі повернуться додому, так, я в це вірю, бо якщо
оцей Кролик з Маріуполя-2019, якого я знайшов тут у Києві взагалі якимсь
неймовірним чином -  якимсь загадковим містичним чином знову винирнув в
Києві-2023, повернувшись таким чином додому, бо він спочатку був зроблений в
Києві... то значить і всі інші повернуться додому! Так, я в це вірю, пані
Оксано! Так, Ваші сльози скоро скінчаться, я в це вірю, так! Бо я також вірю в
те, що колись сказав Христос - і будете мати віру розміром із гірчичне зерно, і
скажете отій горі - йди сюди, і гора піде сюда! Так, це насправді дійсно
працює, перевірено на практиці, бо якщо я б вірив, що оті Кролики важливі, я б
ніколи б не знайшов отих Красенів на останньому поверсі. Так, і я вірю, що в
Маріуполі оцей Кролик знову буде стояти посеред площі, і усілякі дітлахи будуть
підходити, щоби його погладити... Так що все буде добре, пані Оксано!

%Цікаво знати!
%4 травня у 1852 році народилась Аліса Лідделл, прототип персонажа Аліси з книги
%Льюїса Керролла «Аліса в країні чудес», а також один із прототипів героїні в
%книзі «Аліса в Задзеркаллі».

%https://www.facebook.com/groups/1476321979131170/posts/6068960239867298/

% По своїх. Як "дружній вогонь" вбиває військових і чи є в цьому винні
% https://www.pravda.com.ua/articles/2023/05/4/7400547/

Навчальний посібник з лексикології французької мови «Слово та Поняття»

1427894275

https://www.0629.com.ua/news/3604357/sasko-protag-v-mariupoli-vidbuvsa-spravznij-urbicid-zniseno-vse-so-bulo-tkaninou-mista-i-fenomenalne-i-zvicajne

https://babel.ua/texts/48165-nauci-dosi-nevidomo-shcho-same-vidbuvayetsya-v-mozku-koli-lyudina-chhaye-neyrobiolog-oleksiy-boldiryev-vivchaye-vse-zhive-organizovuye-dni-nauki-ta-marshi-vchenih-babel-rozpovidaye-yogo-istoriyu

 так, Сікорський, киянин!!! Визначний киянин! І добре... щодо дискусії... тут
важливо мати якусь відправну точку... на що опиратись... справа то не в
Мєндєлєєві як такому, наскільки він там геніальний чи він московит або ні і
т.д, а в тому, що для мене відправними точками є (1) Київ, його історія,
архітектура, в тому вигляді, в якому вона існує в моєму світі... і т.д. Потім
(2) я сприймаю Людей як Людей, спочатку... а вже потім контекст національності
беру... Потім... взагалі... у мене є така собі Київська Теорія Відносності...
От у мене все йде через призму Києва... а чому спочатку Київ, а потім Україна.
А тому що... Україна - це поняття духовно-культурне, перш за все... тому що...
Україна існує вже тисячу років і більше, хоча держава у нас тільки останніх 30
років з гаком. Такий у мене світогляд. А Київ - як Серце України - це поняття
не тільки духовно-культурне, як Матір Городам Руським або ж як Серце України,
нервовий Вузол України, де сходяться разом маріупольці і одесити, львівяни і
донеччани... це також... Місто... яке існує об'єктивно, прямо зараз і тут.
Розумієте.. тому в цій теорії відносності, яку я сам собі придумав, Київ, в
якому я живу прямо зараз, і в якому я народився, виріс, ходив в школу... стоїть
на першому місці в цьому світогляді... І тому звісно досягнення Сікорського
стоять вище для мене як для киянина, ніж досягнення Менделєєва, але, як для
людини із науковим бекграундом та науковим типом мислення, я не можу спокійно
дивитись, як люди просто так беруть щось міняють в моєму Києві, не питаючи ніяк
моєї думки у прийнятний для мене та взагалі для всієї спільности киян...
спосіб. А якесь голосування в інеті щодо таких важливих та фундаментальних
речей - то взагалі повний треш (при всій повазі до постаті та досягнень Миколи
Пирогова, на якого оці придурки вирішили замінити Менделеєва).

ПРО ЩО ТИ МРІЄШ, КОЛИ ЖИВЕШ НА САМОТІ?

Данило Подибайло - Синьо-жовта балаклава [пісні про АТО]

Доброго дня, Олю! 

Знаєте... от я збираюсь викласти скоро відео, як я запалюю свічки вчора вночі
за Данила. Це можливо ще не зовсім ще зрозуміло що це за відео, так я поясню. Я
вчора вночі - по-перше, я чекав ракетної атаки, але мені то байдуже було. Я
збирав матеріали по Данилу. Далі, як першу публікацію я записав повідомлення
університету. Потім... І я думаю напевне це вам буде неймовірно важко дивитись.
Але я це зроблю. Ну, по-перше я не можу так робити, щоби спеціально під кого
підлаштовуватись, і я щось пишу або дію, виходячи перш за все із свого власного
світобачення. І я це зроблю, Тому що... я звичайно не знав Данила, але... от
знаєте. Я то і так збирався запалювати свічки, але...  от скоріше із почуття
такого собі обов'язку скоріше. Що от у мене все добре, а він загинув. Такий
молодий, талановитий. Я от зараз походив туди сюди, перевірив що там з моїми
сканами... виявляється, що така довжелезна черга в поліграфії що я махнув рукою
і сів на лавочку. Сижу собі такий втомлений... вітер віє, дітлахи граються там
десь на дитсадку. 

-----------------------------------

да... и вот что! слушайте!!! я конечно сорян за русский язык если что потому
что сейчас это не мега популярно идет война мерзкие твари на рос-тв вовсю какое
то гавно мерзкое все льют и льют хоть уши себе отрезай и выкидывай до того
тошно видеть тех тварей! как их вообще земля носит еще, и гнев божий не
разметал их нахер, как содом и гоморру... я вообще не понимаю! Вот реально, россия - это же содом и гоморра!!! 
не понимаю, почему эти твари все еще
живы, если на них реально кровь тысяч, да, сотен тысяч, кровь тех детей, что
погибли в день детей, да, на этих тварях также кровь Данилы, да! и на этих
тварях также смерть детского врача Оксаны из детской больницы Охматдета, ее убили
твари 10 октября, удар по центру... когда она просто как обычно на работу ехала деток лечить...
а Данила... бляха, такой молодой, пипец просто, это реально горе!!! я понимаю,
да, почему Вы сознание потеряли, и да, поверьте!  реально мне сердце временами
разрывает мысль, что Вы вот можете взять и просто вот так помереть!!! Пипец
просто, я понимаю, что парень отдал жизнь за Отчизну, он реально герой из
героев!!! Но нахрена кому еще нужна мертвая Оля, да, кому еще нужна мертвая
Оля!  скончавшая где то на улице внезапно потеряла сознание а сыночек ее
которого она держала за руку закричал от ужаса... мама, очнись!!! плачущий
мальчик...  над телом мертвой мамы...  

ее увезли посреди оживленной киевской
улицы...  и все...  а послезавтра похорон на Байковом, и сына стоит, плачет....
все мариупольцы стоят, плачут...  я тоже бы пришел сзади поплакать... реально
бы плакал как ребенок, вот отвечаю, хоть я давно уже такой себе дядька с
пузом!!  хоть лично Вас совершенно не знаю, то шо я там на 1 апреля стоял рядом
с Вами, где-то сбоку недалеко, ну это вообще не в счет, это же понятно!!!
бляха... я вот вообще не пойму!!! Кому это нужно!!!  А потом в фейсбуке бляха
некролог будет что Ольга Демидко умерла от разрыва сердца или...  я знаю, таких
случаев полно!!!  когда горе настолько велико, что оно реально тащит человека
на дно в могилу, и нет на свете силы, которая бы это
остановила!!! и человек либо угасает месяцами, постепенно становясь просто
тенью самого себя...  либо в какой то момент ломается как веточка напополам, и
все, нет Оли, нет!!!  Вот вы не думайте, кстати, чо мне там нужны ваши какие то
женские прелести, у меня вкусы здесь другие, и вообще пока война идет, да клал
я болт на женщин, война же!!! ужасный случай кстати был! вчера со соседкой
познакомился на этаже, оказалось милейшая особа, и я ей даже книжку Гарика
Корогодского дал почитать, полтора часа общались на философско-исторические темы, 
реально интересная особа... 

телефон дала, как никак соседи, шо поделать, но!!!
бляха, нахрена мне эти искушения сейчас!!!  я пошел к себе в квартиру и
плевался потом минут пять, потому что вот реально отвлекает, все эти мутки!!!
и я просто чисто по дружески говорю, как есть!!!  раз записались в друзья,
значит, дружба, да!!!  и я вот знаете, смотрю на его фотку, на фотку Данилы...
вот эта самая фотка, которая щас у вас в профиле...  и... глаза другие, чем у меня, серьгу
в ухе я не ношу, бородки у меня нет... вот посмотрел его видео, где он в студии
играет, поет песню...  я реально не очень на него похож и внешне и характером и
так далее... я вот на гитаре пытался учиться, но реально выучил только
простенькие мелодии... короче, это все не мое...  музыка... я полный чурбан в
этом...  и очень жаль, но шо поделать, не судьба...  но!!! вот руки!!! реально
руки!!! бляха, руки как у меня, ну бляха, вот руки!!!  мои руки!!! понимаете,
вот узнаю свои собственные руки в нем!!! это пипец просто!!!  ну сами
посмотрите видео, где я книжку под музыку переворачиваю и
скажете!!!

понимаете, вот реально как будто не его, а МЕНЯ, да, меня там убили под
Бахмутом, меня!!! вот как у Высоцкого, прямо то чувствую, хотя слова немного
другие здесь нужны!!!  вот не могу удержаться чтобы не вставить здесь!!!  Нам и
места в землянке хватало вполне, Нам и время текло для обоих.  Все теперь
одному. Только кажется мне, Это я не вернулся из боя.  Все теперь одному.
Только кажется мне, Это я не вернулся из боя.  Наши мертвые нас не оставят в
беде, Наши павшие - как часовые.  Отражается небо в лесу, как в воде, И деревья
стоят голубые.  Отражается небо в лесу, как в воде, И деревья стоят голубые... и
да... я понимаю, да, язык оккупантов, москворотый, узкая челюсть, все дела но
все таки!  Как ни крути говорю я на двух языках и одно на самом деле не
исключает другое!  Если на украинском то одно... короче говорю как патриот
Украины который всемерно за развитие украинского языка и культуры вот реально
так! Украинский язык это реально мега язык, сочный, певучий, воздушный, я когда
читаю Лину Костенко или Симоненко, вот реально балдею!!! и да, и он есть и
будет единственным
государственным, это вообще никаких вопросов!!! 

И я вот написал Вам чуть раньше
по русски... а потом вот сел, почесал голову и думаю... а нафига, вот щас Ольга
Александровна еще обидится, это же орки мерзкие у нее все забрали, выгнали,
отобрали ее город любимый, забрали ее мечту, ее море! да, я знаю, как много для
вас значит море!!! реально же Дети Моря вы все!!!  и эти мерзкие твари, которые
у вас все забрали, которые убили Витю Дедова, убили Сергея Давыдовыча Бурова,
убили Наталию Лоскутову, вашу мега-француженку! да, французский, если уж я
возьмусь изучать, буду по Лоскутовой изучать!!!  и реально, твари, не счесть
зла и мерзостей, которые они натворили всего лишь за год с гаком!!!  да! эти
твари на русском тоже ж говорят, реально же мерзкие сатанинские твари!!! но...
шо поделать, я кроме того что киевский программист и патриот Украины, я также
такой себе обычный киевский пацанчик на районе, понимаете, да! Вот как те
пацаны из группы Стольный Град, а их песню Вы уже видели, и вроде Вам она
понравилась, ну лайк же поставили верно! Надеюсь не из чисто такой вежливости и
обходительности, типа ну мудак, че прет со своими русскоязычным песнями,
ненавижу орков, ненавижу русню, ладно, поставлю лайк...  да, надеюсь, что не из
вежливости, а что да, что реально понравилось, да!  потому что это реально одна
из моих самых любимых песен про Киев!

И вот! Короче!! Я вас в Библиотеку записал, да! Вам это кажется наверное
смешным и прикольным, хахаха, вот чудачок смешной, мои диссертации позапихивал
в архив, и нахрена это ему нужно, непонятно... ну хай пишет...  но я записал,
да!!! а значит я в ответе за Вас, да, черт возьми, в ответе!!!!  Потому что я
сука Библиотекарь, как бы безумно и смешно это не звучало!!!  И нахрена мне
нужно, чтобы Вы умерли, нахрена! Какой тогда смысл, бляха, я же плакать
напролет ночами буду!!  и свеча за Данилу у меня горела прошлой ночью,
парафиновая! все провоняла всю квартиру капец!!!  а сегодня я все ноги промочил
пока за восковыми свечами бегал в ближайший храм, 400 гривен потратил, пачку
целую купил по 20 гривен штук! а потом зонтик в руку и бегом домой по лужам!!!
Так вот!!! За Данилу я буду ставить свечи каждую ночь, будут гореть, да,
сколько нужно, хоть бы всю жизнь, мне не жалко!!!  

НО... вы че тоже хотите, шоб я и за Вас свечи тоже ставил??? Вам это реально
нужно???  Мне это не нужно!!!  Вам это не нужно тоже.. так зачем умирать,
зачем, вот скажите, Оля, зачем???  Данила погиб как герой, да, герой!!! я и Вы
живем благодаря ему и таким же героям как он! да! и у него мои руки, да, именно
так!!! и я любому голову оторву если какая тварь про него что то плохо
скажет!!! НО, Зачем Вы себя так мучаете страшно, Оля, объясните, ну зачем???
Кому это нужно??? У вас больная мама, которая недавно попала в больницу!!! и
сынуля маленький!!! Вы о них подумали, да, каково им видеть Вас в таком
состоянии, да??? Мама ж ваша тоже может помереть, если Вы помрете, разве не
так, Вы об этом подумали??? А помрет ваша мама, так еще свечку нужно будет
ставить за маму!  Потом, Оксана Стомина, она реально близка к этому!! А когда я
вижу слезы Оксаны, мне жить вообще не хочется, да!! Хочется временами просто
пойти и утопиться в Днепре, когда я вижу, как Оксана плачет!!!  Вот она тоже
может помереть, еще плюс свечка! А потом и Женя Сосновский не выдержит, и туда
же!!! А потом эффект домино и все мариупольцы разом в могилу, общую,
братскую!!! И Наташа Дедова, и Надя Сухорукова, все все все!!! ДА ВЫ ЧО
ОБАЛДЕЛИ СОВСЕМ.  Эй, друзья, так свечей на всех не напасешься, и бабло я еще
не научился печатать, чтобы свечки фургонами завозить!!!  Да и дом можно
спалить походу, если вся комната будет в свечах!!! Бомжом на Вокзале меня
сделаете в итоге!!!

Слушайте, БУРОВ. Да!!! БУРОВ. СЕРГЕЙ ДАВЫДОВИЧ. МАРИУПОЛЬ И МАРИУПОЛЬЦЫ. МАРИУПОЛЬ. БЫЛОЕ. 
Черт возьми!!! Удивительная История!!!
Какое то чудо просто!!! Мне недавно один мой хороший дружбан - такой себе хлопец хоть куда козак!!! 
он меня периодически снабжает книжками
с приемок - куда какие то очередные феерические дебилы выкидывают ненужные им книжки - 
а как книжка может быть ненужной??? как книжка может быть ненужной??? темные времена нынче однако!!!
так вот. Приносит он стопочку книжек!!! И ШО!!! БУРОВА Я ВИЖУ. НУ ПИПЕЦ ВАЩЕ. КАК ЭТО ВОЗМОЖНО
Я НЕ ПОЙМУ. КАКОЕ ТО ПРОСТО ЧУДО ИЗ ЧУДЕС. Реально!!! Я так мечтал где-то достать Бурова!!! Но нифига,
нет его нигде!!! А тут бац, вижу!!! В самом низу стопочки!!! Буров, он самый!!! Я Вам говорил, что там книжки
на Просвещения 2 были сбоку снизу, Вы не забрали, а жаль!!! А это Буров!!! Одни чудеса со мной творятся!!! То я за Кроликами 
гоняюсь, то книжки нахожу по Куинджи в самом неожиданном месте, а это ваааааааааще, БУРОВ. БУРОВ. БУРОВ.
Ну не помирайте, Оля, давайте, я Вам Бурова буду читать лучше!!! Видосики делать буду!!! Фейнман с его веселыми приключениями
и философскими рассуждениями... это хорошо, конечно, и мне он близок.. но БУРОВ, ЭТО ЖЕ ВСЕ ВАШЕ РОДНОЕ.
Ну не помирайте, Оля, ну прошу Вас!!!!

І Вам теж дякую!!! І за ворона Крума, і за шрифти в монографії, і за багато
чого іншого!  Ви дійсно просто неймовірна людина, дуже багато зробили для
науки, Маріуполя і Києва!!  І синочок у вас дуже прикольний такий собі!!! Якщо
дівчатка його так обступають вже, значить все добре, такий малий, а вже знає як
із жінками треба поводитись!!! ))) Доброго ранку!! Вибачаюсь, я трохи
балабол... обіцяв, що всю ніч буду палити свічки і читати книжку!!...  але...
після певної от події...  дійсно, майже не спалив хату... а потім мене так
переклинило, що я звалився спати як мертвий і от щойно тілько прокинувся!!
спав як мертвий!!! короче...  сижу я сижу... там свічки горять за Даню та
Сергія Давидовича... у мене в квартирі короче дві кімнати, одна більша, інша
менша... сиджу я значить у більшій кімнаті - за ноутбуком, монітором,
клавіатурою - а у мене сигнал від ноутбука йде на великий екран, бо так
набагато зручніше, ніж за самим ноутбуком сідити - а в іншій кімнаті значить
стоїть ноутбук із портретом Дані та книжками Сергія Давидовича - ну і оті
пряники теж - ну, і свічки горять! і там теж ноутбук! у мене є декілька
ноутбуків - я ж програміст!!!... є короче більший потужніший ноутбук - я за ним
зараз сиджу - є також менший - ото він був на фото - і є також Еппл Макінтош -
він є у тому відео - де я Щедрика вставляю - на балконі сиджу - це
сирено-бабахова серія відео коли була сирена бабахи а мені начхати було от
сиджу на балконі і сиджу собі!!! - так... отой неймовірний Щедрик з Маріуполя
2021!!! - а я чогось вирішив передивитись, те, що я Вам ото зверху написав...
подивився... блін... от думаю, якось дуже тупо і смішно написав...  от
прочитав, і сам над собою почав сміятись!! ото я дійсно дурниць якихось

написав!!!  але Ви не витирайте, треба ж мати хвилинку для сміху!!! от якщо
буде Вам погано на душі... будете читати мої опуси, сміятись... хай буде )))
розумієте, якась там мертва Оля, якийсь там дядько з пузом, сука Бібліотекар,
бомж на Вокзалі, фургони свічок... щось мене якась просто тупо ржакаааааа
проперла... РЖУНІМАГУ бліііііііін... сижу и ржу над собою як Коняка якась!!!
ото думаю, що поважна серйозна вчена Ольга Олександрівна, кандидат історичних
наук або ж доктор філософії з історії, якщо по європейськи казати...  а ми ж в
Європу йдемо, правильно, так... наді мною втихаря сміється... вчора ще екзамени
приймала в університеті, напевне, там Вам було важко дивитись якщо студент
верзе повну нісенітницю щодо предмету екзамену (якщо у вас такі є - сподіваюсь
що ні - якось у мене враження було, що... у вас в універі студенти навпаки -
всі такі енергійні вмотивовані - була от чесно якась така от дуже драйвова
атмосфера на отій зустрічі...  що от дійсно відчуваєш, що і вам ректор - і ваші
викладачі - і ті студенти - що там зібрались... короче - гори збираються
перевернути!!!  рідко я таке бачив до того в інших місцях!) що сижу і просто
сміюсь сам над собою... щось у мене таке переклинило!!! але от знаєте... От
души написав (вот... от души, как говорят в Киеве...)... тому...  ну і
короче... а! тут згадую, треба свічки поміняти!!! Бо ж воскові свічки, а їх
треба міняти, бо досить швидко якось горять! Тому і купив їх багато!  Ну от!
Біжу в сусідню кімнату, а ТУУУУУУУУУТ ООООООООООООООООООООООООЙ короче СВІЧКИ
ДОГОРІЛИ ВЖЕ ДО КІНЦЯ І ПЛАСТИКОВА ПІДСТАВКА ТАК КАНЦЕЛЯРСЬКА ПЛАСТИКОВА
КОРОБОЧКА ГОРИТЬ ВОВСЮ!!!! ПІПЕЦ ПРОСТО!!! А ТАМ ЖЕ ПОРУЧ НОУТБУК КНИЖКИ
ААААААААААА ЗАРАЗ ВСЕ ЗАЙМЕТЬСЯ НА ПОВНУ ОТО ЛИХО ЛИШЕНЬКО БУДЕ!!!!  я біжу
швидко туди, хапаю оту підставку - а це підставка для ручок - короче з
канцеляріїї - там ще на дні було повно всяких скрєпочок - так от - ВОНА ГОРИТЬ
НА ПОВНУ!!! Я такий хапаю їх починаю ДУТИ НА НЕЇ ВОВСЮ ДУНУВ ДУНУВ РАЗ НЕ ГАСНЕ
ДУНУВ ДВА НЕ ГАСНЕ АЙЙЙЙЙЙЙЙЙЙЙЙЙЙЙЙЙЙ ДУНУВ ТРИ ПОГАСЛА УФФФФФФФФФФФФФ АЛЕ ДИМ
НАТОМІСТЬ НА ПОВНУ ЙДЕ І НАХАПАВСЯ ЩЕ ТРОХИ ОТОГО ПЛАСТИКУ СОБІ У РОТА!!!
сподіваюсь в легені отой горілий пластик не попав!!!  Дуже сподіваюсь!! от
зараз начебто зі мною все добре, сподіваюсь, що нічого в легені не попало, так!
А хто знає як воно буде далі!  Не знаю, так! Бо горілий пластик насправді дуже
шкідлива річ, це ж канцероген, так і до раку може дійти, не дай Боже!

ну і значить хапаю оту підставку, задув її! уффффффф! біжу в ванну кімнату ПІД
ВОДУ ПІД ВОДУ ЧОРТ ЗАБИРАЙ!!! ВСЕ ВОНЯЄ НАВКРУГИ НУ КАЦЕЦЦЦЦЦЦЦЦ!!! ДИМ ПО ВСІЙ
КІМНАТІ РОЗУМІЄТЕ!  Я такий... відкрив вікно, сів, почесав собі голову, ото
думаю і пригода!!! Таккккккк вирішив на сьогодні свічок досить!!! Але ноутбук
із портретом Дані не вимикав!!! Ноутбук із портретом Дані буде стояти вімкнений
ще довго, повірте мені, так! Мені не жалко, щоби отой чудовий портрет де Даня
грає на гітарі стояв у мене в кімнаті так! Ну і пішов собі спати! Але я думаю
знак хороший!!! Їм на Небесах сподобалось, так!!!  Бо значення цієї димовухи
таке - ДЯКУЮ ТОБІ ВАНЯ НУ ОТ ДІЙСНО ЗРОБИВ ДЛЯ НАС ПРИЄМНЕ ЩО АЖ АГОНЬ ОТ ВОНИ
АГОНЬ МЕНІ І НАДІСЛАЛИ У ВІДПОВІДЬ!!! Так що настрій у мене гарний, так! )))
значить я все зробив правильно, так!!! Ну добре! Йду собі по справам своїм, потрібно ж Книжками 
зайнятись, ну а Вам гарного настрою!!!

video.23_01_2019.dmitry_proskurin_and_danil_podybailo_ut_deus_single_2019.mp4

\ii{body.to_maria_podybajlo}

скинул на счет +40000 которая Алевтина выставила... вдруг Вы сомневались, что я
какое то виртуальное лицо... да, я был сегодня... всех видел... и вообще... мне
очень стыдно, чесно... прямо все внутри жжет, печет, поверьте... я Вас видел
сегодня... по-моему, это были Вы... по-моему, Вы, да... знаете, увидев Вас,
Ваше лицо... то есть насколько Вы не похожи были на ту, которая на фото... как
будто это не Вы, а Ваша Тень... и так больно было видеть Вас такой... и я понял
всю пропасть, которая теперь разделяет нас... простите, что пишу... надеюсь, та
рана, нанесенная мной как бесконечно острым ножом, все таки затянется... потому
что реально так стыдно, что готов землю есть и воду из Днепра пить, пока не
захлебнусь... и я очень надеюсь, что я изчезну из Вашей памяти навсегда...


