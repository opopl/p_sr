%%beginhead 
 
%%file 23_12_2022.fb.fb_group.mariupol.biblioteka.korolenka.1.novor_chn__yalinki_u
%%parent 23_12_2022
 
%%url https://www.facebook.com/groups/1476321979131170/posts/5675837962512863
 
%%author_id fb_group.mariupol.biblioteka.korolenka,lisogor_viktoria.mariupol
%%date 23_12_2022
 
%%tags mariupol,istoria,mariupol.istoria,novyj_god,prazdnik,jalynka,rizdvo
%%title Новорічні ялинки у Маріуполі: історія
 
%%endhead 

\subsection{Новорічні ялинки у Маріуполі: історія}
\label{sec:23_12_2022.fb.fb_group.mariupol.biblioteka.korolenka.1.novor_chn__yalinki_u}
 
\Purl{https://www.facebook.com/groups/1476321979131170/posts/5675837962512863}
\ifcmt
 author_begin
   author_id fb_group.mariupol.biblioteka.korolenka,lisogor_viktoria.mariupol
 author_end
\fi

\textbf{Сторінки історії}

\begingroup
\em\bfseries
Продовжуємо розповідати про те, як маріупольці готувалися до новорічних свят, у різні часи...

Новорічні ялинки у Маріуполі: історія
\endgroup

Новорічні ялинки для дітей у Маріуполі влаштовували ще у 19-му столітті. І не
лише сімейні, а й громадські. У місті були громадські та благодійні
організації, які цим займалися. Наприклад, Товариство піклування про дітей.

Палаців культури тоді не було, згадок про ялинки на майданах я теж ніде не
відшукала. Ялинки встановлювали у приміщеннях громадських зборів, у «дорослих»
адміністративних будинках, навіть у міській управі! Був і Дід Мороз -
професійний артист або актор-любитель, і як подарунки - в основному солодощі.

Щоправда, ялинки та ранки починалися вже в другій декаді грудня, задовго до
Нового року - перед «дитячим» днем Миколи Угодника. А прикрашена ялина
іменувалася не новорічною, а різдвяною - у багатьох сім'ях її і встановлювали
тільки до Різдва. На вершині дерева замість звичної нам радянської червоної
зірки сидів рум'яний янголятко. Громадські ранки для дітей у Маріуполі
влаштовували навіть під час громадянської війни.

\textbf{\emph{Півстоліття тому}}

Зустріч Нового, 1959 року на пожовклих сторінках міської преси - нотатки,
звіти, афіші, оголошення. У кожній школі, Палацах культури та дитячому
кінотеатрі «Перемога» - ялинки для дітей. Діду Морозу та Снігуроньці
допомагають школярі - учасники художньої самодіяльності. Вони проводять із
дітьми гри та вікторини, виступають із вокальними, хоровими, танцювальними
номерами, дек¬ламацією віршів. На кожен ранок, особливо у Палацах культури,
готується ціла концертна програма юних артистів.

Рівно півстоліття тому в новорічних виставах ДК «Азовсталь» було задіяно понад
сто (!) юних артистів «дитячого сектора самодіяльності». Драмгурток ДК
«Азовсталь» дав за час зимових канікул вісімнадцять (!) уявлень - порівняно з
навантаженням дорослих професійних акторів. Це були п'єси «Чудова сила»,
«Новорічна казка» Ю. Казакова та «На лінії фронту» А. Куліша. Зараз, через
півстоліття, боляче навіть писати ці слова: ДК «Азовсталь» - будівля, з якої
вигнали всі творчі колективи, - порожній уже рік.

\textbf{\emph{Перший у місті бегемот}}

Були й інші розваги для дітей на Новий рік, крім ялинок. У грудні 1958 року в
Жданов на «короткий час» приїхав «Зооцирк - звіринець». Розмістився на
території колгоспного ринку. «Вперше у місті демонструються: африканський
гіпопотам (бегемот), і навіть тварини та птахи різних частин світу... Годування
хижаків о 4 годині вечора». Отож, коли маленькі маріупольці вперше побачили
бегомота!

Розваги намагалися поєднувати з вихованням. На зимові канікули, що йшли відразу
за Новим роком, планувалися екскурсії для школярів: па завод імені Ілліча, в
краєзнавчий музей. Або - як вам тема, що пропонується для розмови з дітьми у
новорічної ялинки: «Завжди і в усьому будьте ленінцями»?

Сосни на міські дитячі ранки доставлялися в основному з території
Червоно-лиманського заповідника. Кожне велике підприємство теж влаштовувало
ранки у своїх ДК і мало власних постачальників.

Для старшокласників 31 грудня клуби та Палаци культури давали новорічні бали.
Велике гуляння в новому, щойно відкритому портівському клубі (ДК моряків). З
«правил поведінки» на новорічному балу, складених при зустрічі 1959-го:
«Сумуючий учасник вважається відсутнім. За жарт плати жартом... Категорично
забороняється бути наодинці». Незрозуміло: не можна бути одному чи не можна
бути з кимось наодинці?

\textbf{\emph{Іграшкова свиня для XXI з'їзду КПРС}}

У моїй сім'ї зберігаються іграшки повоєнних років: картонні рибки і звірятка,
склеєні з двох половинок і пофарбовані зверху блискучою фарбою. Матові, а не
сяючі, як зараз, кулі з товстого справжнього скла. Також овочі, фрукти та
сніговики із скрученої вати та знову-таки пофарбовані зверху.

За зовнішнім блиском ці іграшки явно поступаються сучасним, але за якістю і
довговічністю... Небачений, але твердості картон, незнайомий сучасним дітям
щільний кольоровий папір... Ліхтарик, зроблений з такого паперу, стоїть на
столі не гірше за пластмасовий. Картонну рибку не розрізають ножиці (каюся,
перевіряла). Моркву можна жувати — нічого їй не стане. Пам'ятаю, як у дитинстві
сусідський собака напав на мою улюблену іграшку - лимончик - і спробував його
розгризти! Безуспішно!

Ялинкові іграшки, за повідомленнями преси, що надійшли у продаж перед Новим,
1959-м роком: «блискучі різнокольорові кулі та зірочки, прапорці...».
Асортимент у магазинах небагатий: 19 грудня «Приазовський робітник» тільки
прогнозує завезення у продаж «новорічних кульків та коробок з кондитерськими
виробами».

Натомість заохочується рукоділля: у школах та гуртках діти майструють іграшки
самі. У подовженні ЗОШ №29 перед Новим, 1959 роком запрацювала ціла «фабрика
іграшок». Популярні свинки та сніговички, зроблені вручну з яєчних шкаралупок.

Новий рік у радянському повоєнному суспільстві — свято насамперед дитяче та
сімейне. Він займає набагато менше місця в календарі, ніж зараз. Навіть
пишеться з маленької літери - новий рік. У дорослому світі – інші дати,
ідеологічні. Звернемося знову-таки до зустрічі 1959 року - цілі лінії газет
присвячені майбутньому XXI з'їзду. Навіть дітей у школах приваблюють робити
подарунки для делегатів з'їзду КПРС.

\textbf{\emph{Білочка та людина-павук}}

Популярні дитячі новорічні костюми п'ятдесят років тому, судячи з розказаного
мені старшим поколінням і переглянутої преси, - «петрушки», «хлопушки»,
«мушкетери», «снігурочки», «сніжинки» та «ковзанярі». Ковзани тоді - дуже
популярний і дешевий вид спорту, ковзанки заливають у морозну погоду в парках
культури та відпочинку, в азовстальському, зокрема. Костюми ніде не продавалися
— у дитсадках їх майстрували вихователі, у школах — самі діти.

Пізніше - 70-ті роки. Для тих, хто старший - «шахову королеву», в клітьовій
пілотці і спідничці, і «мушкетеру». У мій дитсадковий час, наприкінці 80-х,
найпопулярнішим костюмом була знову-таки «сніжинка». У молодші шкільні роки –
«принцеса». По п'ять «принцес» припадало на один клас на ранку. А «мушкетер» –
загалом вічний «хіт» дитячих карнавалів.

Влітку я купила сучасну книгу «Карнавальні костюми для дітей». Викройки
традиційних "білочок" і "зайчиків" спокійно сусідять у ній з костюмами "східної
танцівниці", "королеви вогню", "індіанки" з дитиною за плечима, "шрека" і
"людини-павука". Найбільш «бескомілекспі» костюми в цій книзі: «відьма на мітлі
з рудою «грецькою» вибух на макаронній фабрикі», «вампір» у чорному плащі і
білій сорочці з моторошними іклами - вилитий аристократ Дракулав виконанні
Минського актора Бели Лугоші. І «безголовий» - із заляпаною «кров'ю» білою
манішкою замість голови (голову приховував білий ковпак).

М. Корольова

//Надо.-2008.-№52.-с.5.


