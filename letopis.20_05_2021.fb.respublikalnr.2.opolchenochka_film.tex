% vim: keymap=russian-jcukenwin
%%beginhead 
 
%%file 20_05_2021.fb.respublikalnr.2.opolchenochka_film
%%parent 20_05_2021
 
%%url https://www.facebook.com/groups/respublikalnr/permalink/815117799124013/
 
%%author 
%%author_id 
%%author_url 
 
%%tags 
%%title 
 
%%endhead 
\subsection{\enquote{Ополченочка} шагает по миру!}
\label{sec:20_05_2021.fb.respublikalnr.2.opolchenochka_film}
\Purl{https://www.facebook.com/groups/respublikalnr/permalink/815117799124013/}

Зимина, Олеся

"ОПОЛЧЕНОЧКА" ШАГАЕТ ПО МИРУ! Создатели "Ополченочки" подпишут договор с Россотрудничеством о показе фильма за рубежом

Киностудия "Лугафильм" продюсерского центра "Родина" планирует 20 мая подписать
договор с Федеральным агентством по делам СНГ, соотечественников, проживающих
за рубежом, и по международному гуманитарному сотрудничеству
(Россотрудничество) о передаче прав на некоммерческий показ фильма
"Ополченочка". Об этом ЛИЦ сообщил продюсерский центр "Родина".

\ifcmt
  pic https://scontent-iad3-1.xx.fbcdn.net/v/t1.6435-9/s720x720/189025122_135162395330535_2477684936768781439_n.jpg?_nc_cat=101&ccb=1-3&_nc_sid=825194&_nc_ohc=bdVjrPh4MR4AX9SlSos&_nc_ht=scontent-iad3-1.xx&tp=7&oh=41a3d1828da961d8f327c9e1db1120fb&oe=60C91B88
\fi

"Фильм о военнослужащих Донбасса "Ополченочка" шагает по миру! 20 мая 2021 года
состоится важнейшее культурное историческое событие в жизни Донбасса и Народных
Республик – торжественное подписание договора с Россотрудничеством о передаче
прав на некоммерческий показ фильма во всех представительствах Федерального
агентства за рубежом", - говорится в сообщении.

Авторы картины рассказали, что договор позволит перевести снятый луганскими
кинематографистами фильм на шесть языков и показать его в 80 странах мира.

Генеральный продюсер фильма, директор кинокомпании "Лугафильм" Роман Разум
отметил, что "через образы героев художественного фильма русские
соотечественники за рубежом смогут ознакомиться с реальной ситуацией в Народных
Республиках Донбасса".

Ранее он сообщал, что "Ополченочка" является сильным идеологическим оружием,
приближающим победу наравне с пушками", поэтому "Лугафильм" тщательно выбирает
время его выхода на большой экран. 

Фильм рассказывает о войне в Донбассе, судьбах его жителей, неожиданно для себя
оказавшихся в водовороте трагических событий.

Режиссер-постановщик – Алексей Козлов (Москва), оператор-постановщик, автор
идеи фильма – Роман Разум (Луганск), авторы сценария – Борис Табачник,
Владислав Плахута (Луганск), композитор – Ян Лещенко (Луганск). Продюсирование
по РФ – Юрий Мильчинский (Москва). В главных ролях: Мария Перн (Волгоград),
Наталья Колоскова (Смоленск), Наталья Стародубцева (Луганск), Юрий Миронцев
(Ростов-на-Дону), Виталий Лясников (Луганск), Вилен Бабичев (Полтава), Анатолий
Фалынский (Ростов-на-Дону). 

ЛуганскИнформЦентр — 19 мая — Луганск

\url{http://lug-info.com/news/one/sozdateli-opolchenochki-podpishut-dogovor-s-rossotrudnichestvom-o-pokaze-filma-za-rubezhom-67946}
