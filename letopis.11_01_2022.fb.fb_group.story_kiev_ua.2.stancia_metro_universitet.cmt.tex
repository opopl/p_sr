% vim: keymap=russian-jcukenwin
%%beginhead 
 
%%file 11_01_2022.fb.fb_group.story_kiev_ua.2.stancia_metro_universitet.cmt
%%parent 11_01_2022.fb.fb_group.story_kiev_ua.2.stancia_metro_universitet
 
%%url 
 
%%author_id 
%%date 
 
%%tags 
%%title 
 
%%endhead 
\zzSecCmt

\begin{itemize} % {
\iusr{Татьяна Вакуленко}
Квест! @igg{fbicon.hands.applause.yellow}{repeat=3} 

\iusr{Александр Мурашов}
Максим Горький ещё уцелел. Круто! Это покруче палеолитических раскопок на месте бывшего моря

\begin{itemize} % {
\iusr{Ольга Кирьянцева}
\textbf{Александр Мурашов} тоже об этом подумала)

\iusr{Татьяна Гордиенко}
\textbf{Александр Мурашов} , палеонтологических. Палеолит -это к археологам

\iusr{Александр Мурашов}
\textbf{Татьяна Гордиенко} зато красиво, поэтично и короче писать @igg{fbicon.heart.eyes} 

\iusr{Татьяна Гордиенко}
\textbf{Александр Мурашов} , это разные отрасли знаний - геология и археология.
Писать подобное - проявоять невежество. Впрочем, может вам это нравится

\iusr{Александр Мурашов}
\textbf{Татьяна Гордиенко} я специально так написал, так прикольнее) Вы же отреагировали=значит уже хорошо

\iusr{Татьяна Гордиенко}
\textbf{Александр Мурашов} , глупо

\iusr{Александр Мурашов}
\textbf{Татьяна Гордиенко} спсибо. ваше мнение важно и будет учтено
\end{itemize} % }

\iusr{Ольга Кирьянцева}

Интересно как. А я ещё в детском саду в свежем привезенном морском песке нашла
нечто, - круглое, сантиметров 7-8 в диаметре, состоящее из двух пластин,
соединенных перемычками и на верхней пластине как будто выдавленное изображение
наподобие цветка с пятью лепестками... До сих пор нигде не встречала ничего,
напоминающего ту мою находку. Очень интересно, что бы это могло быть...

\begin{itemize} % {
\iusr{Владимир Аксенов}
\textbf{Olga Kiryantseva} Это могло бы быть, что если бы эту находку сохранили, то сегодня были бы миллионершой и не в чем бы себе не отказывали.

\iusr{Ольга Кирьянцева}
\textbf{Владимир Аксенов} сама так думаю  @igg{fbicon.face.grinning.big.eyes} 
\end{itemize} % }

\iusr{Елена Сидоренко}
Интересно,спасибо. @igg{fbicon.heart.beating} 

\iusr{Тамара Игнатова}

\ifcmt
  ig https://scontent-frx5-2.xx.fbcdn.net/v/t39.1997-6/s168x128/93027172_222645632401274_7176243611145601024_n.png?_nc_cat=1&ccb=1-5&_nc_sid=ac3552&_nc_ohc=ReP062CEyJEAX-liui4&_nc_ht=scontent-frx5-2.xx&oh=00_AT-IOIlVoNgzXkah58GTmjkGWrzuCvab6koxemhq-HwHYA&oe=61E1D2F0
  @width 0.1
\fi

\iusr{Ирина Иванченко}
До Пешкова не добрались пешком \enquote{ревизионисты} ...Наверное, им всё - таки непросто достать его \enquote{На дне}...

\iusr{Vadim Basovskiy}

Чуть-чуть подправлю автора. Не мог он опустить пятачок на следующий день после
открытия метрополитена. Метро было открыто для регулярной работы 12-го ноября
1960-го года. То есть оплата в 5 копеек началась с денежной реформы 1961-го
года. До этого проезд стоил 50 копеек. Пассажир покупал в кассе бумажный
билетик и предъявлял его контролёру. Автоматы, принимавшие пятачки. появились
почти через год после открытия метро.

\begin{itemize} % {
\iusr{Александр Венге}
\textbf{Vadim Basovskiy} благодарю. Думаю, что Вы правы. Мал был и что-то путается. Будем считать историю про пятачок метафорой ))).

\iusr{Владимир Новицкий}
\textbf{Александр Венге} 

Александр, видимо у Вас, как и у меня ассоциация с орпусканием пятачка связана
с тем, что зайдя на станцию, как правильно заметил Вадим по бумажному билетику
- некоторые первые пасажиры на радостях попав впервые на экскалатор бросали
монетки, которые летели вниз и ударялись в стекляную будку за которой стояла
дежурная экскалатора. Таких развлекающихся быстро угомонила милиция. Я был
свидетелем этого первого спуска. Спасибо за интересный рассказ.


\iusr{Александр Венге}
\textbf{Vladimir Novitsky} у меня что-то с чем-то перемешалось в памяти. Мать все время рассказывала, как мы ходили на следующий день после открытия. Я только помню отчётливо праздничное настроение.

\iusr{Nina Steseva}
\textbf{Vadim Basovskiy} вы абсолютно правы, продавались талоны по 50 копееек 1947 года, а после реформы 1961 года - пятачок!
\end{itemize} % }

\iusr{Igor Plaschkin}
А откуда в Киев везли мрамор?


\end{itemize} % }
