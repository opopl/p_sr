% vim: keymap=russian-jcukenwin
%%beginhead 
 
%%file 25_01_2022.fb.fb_group.story_kiev_ua.1.vysockij_i_kiev.cmt
%%parent 25_01_2022.fb.fb_group.story_kiev_ua.1.vysockij_i_kiev
 
%%url 
 
%%author_id 
%%date 
 
%%tags 
%%title 
 
%%endhead 
\zzSecCmt

\begin{itemize} % {
\iusr{Елена Мельникова}
Вечно жив, вечно любим...

\iusr{Tatyana Smirnova}
Спасибо. Интересно, еженедельник знала много о бабушке

\iusr{Tatyana Smirnova}
Ничего не знала, телефон чудит

\iusr{Людмила Краснюк}

Какая удивительная киевская бабушка была у Владимира Высоцкого! @igg{fbicon.heart.red} Спасибо,
Оксана, что поделились с нами такой интересной историей ! @igg{fbicon.hands.pray}  @igg{fbicon.hearts.two} 

\iusr{Оксана Денисова}
\textbf{Людмила Краснюк} Спасибо Вам, рада, что интересно!

\iusr{Александр Лазаретник}
Шедевр. Второго Высоцкого нет. И никогда не будет.

\iusr{Оксана Денисова}
\textbf{Александр Лазаретник} Согласна!

\iusr{Евдокия Зуева}
Светлая память!

\iusr{Вика Нурибекова}

Говорили, что в архиваз студии Довженко были записи его концертов в том числе и
в НИИ, для студентов - для небольшой аудитории. Было около сотни бабин.

\begin{itemize} % {
\iusr{Gregory Kushnir}
\textbf{Вика Нурибекова}, 

я та моя сестра виросли на цих бобінах!.. В залі сміх не замовкав, десь навіть
до істерики.. Мама ще тоді казала, що хтось тайкома позаписував ці монологи та
пісні

\iusr{Вика Нурибекова}
\textbf{Gregory Kushnir} у нас были все бабины 94 штук, но пожар все уничтожил.

\iusr{Вика Нурибекова}
\textbf{Gregory Kushnir} 

а цитаты и сейчас актуальны : \enquote{настоящих буйных мало - вот и нету вожаков},
\enquote{зачем аборигены сьели Кука - молчит наука}, \enquote{поэты ходят пятками по лезвию
ножа и режут в кровь свои босые души}, \enquote{я не люблю фатального исхода, от жизни
никогда не устаю} и можно продолжать и продолжать...


\iusr{Gregory Kushnir}
\textbf{Вика Нурибекова}, 

усе це, та багато інших пісень Володимира Семеновича в нашій родині розібрані
на цитати)). Навіть у колі друзів дитинства це вже меми)), бо ці бабіни (у нас
із десяток) були не раз переписувані)

\end{itemize} % }

\iusr{Алена Милько}
Респект автору!

\iusr{Оксана Денисова}
\textbf{Алена Милько} Спасибо!

\iusr{Zyama Syu}
Таких, как Владимир Высоцкий больше нет! @igg{fbicon.heart.red}{repeat=3}

\iusr{Регина Бочковская}
Спасибо большое за рассказ о бабушке. Светлая память Владимиру и его замечательной бабушке.

\iusr{Оксана Денисова}
\textbf{Регина Бочковская} Спасибо Вам!

\iusr{Инна Смирнова}
Красиво как написали, спасибо!!!

\iusr{Оксана Денисова}
\textbf{Инна Смирнова} Спасибо Вам!

\iusr{Jacob Sukhoy}

Нужно было эмигрировать в Чикаго что бы попасть на выступление Высоцкого это
было в Январе 1979 года в этот день выпало немного снега хотя в эту зиму
1978/1979 годов была одна из самых больших снежных осадков за историю
города. Вот так город Чикаго встретил Высоцкого.  @igg{fbicon.hands.pray} 

\begin{itemize} % {
\iusr{Оксана Денисова}
\textbf{Jacob Sukhoy} Спасибо за Ваши воспоминания!

\iusr{Natalia Bastun}
\textbf{Jacob Sukhoy} interesting!!!!
\end{itemize} % }

\iusr{Валентина Белозуб}

Работая на Украинском телевидении (Крещатик 26), была свидетелем записи песен
Владимира Высоцкого в нашей студии! Незабываемые воспоминания, поскольку ни
капли звёздности в поведении Владимира не было! Работа, работа и уважение к
окружающим !!! Светлая память!!!

\begin{itemize} % {
\iusr{Оксана Денисова}
\textbf{Валентина Белозуб} Спасибо Вам большое, очень интересно!

\iusr{Валентина Белозуб}
\textbf{Оксана Денисова} Оксаночка, это был незабываемый концерт с единственным исполнителем Высоцким. Эмоции переполняли ...

\iusr{Оксана Денисова}
\textbf{Валентина Белозуб} Ох, завидую белой завистью @igg{fbicon.heart.red}

\iusr{Елена Шестопал}
\textbf{Валентина Белозуб} ,здорово!!!!

\iusr{Ольга Патлашенко}

А еще, несмотря на свой очень плотный график, в этот первый гастрольный приезд
в Киев в 71 году Вл. Сем. нашел возможность и время выступить перед студентами
и преподавателями КТИЛПа в нашем актовом зале. И мне, студентке 2 курса,
повезло побывать на этом незабываемом концерте! Концерт, который длился 2 часа,
прошел на одном дыхании.

И сегодня, в его день рождения, мы опять вспоминаем нашего кумира. Спасибо автору публикации

\iusr{Елена Дарова}
\textbf{Валентина Белозуб} А когда это было?

\end{itemize} % }

\iusr{Инна Валентиновна}

Я ходила на его концерт с мамой в Киеве. Столько милиции было вокруг. Такой
ажиотаж, столько народу! Спасибо за воспоминание. А Высоцкий действительно один
и неповторим. Киев имеет к нему непосредственное отношение по линии любимой
бабушки.

\begin{itemize} % {
\iusr{Оксана Денисова}
\textbf{Инна Валентиновна} Как Вам повезло, видеть и слышать Высоцкого вживую! Завидую по- хорошему!

\iusr{Инна Валентиновна}
\textbf{Оксана Денисова} согласна, это счастье видеть и слышать его.
\end{itemize} % }

\iusr{Natasha Levitskaya}

С Днём рождения, Владимир Семёнович! Помним!
Нам повезло - мы жили с вами в одно время!
Светлая и добрая память!@igg{fbicon.heart.red}

\iusr{Августина Тытянчук}
Светлая память!!!

\iusr{Igor Neronov}
Господи, а есть кто-то из «великих», в жилах которого нет иудейской крови, хотя
бы в третьем поколении?)))

\begin{itemize} % {
\iusr{Татьяна Демидова}
\textbf{Igor Neronov} нет, конечно. Сам Христос и Матерь божья были евреями. Народ избранный, умный, сильный духом своим

\iusr{Igor Neronov}
\textbf{Татьяна Демидова} Вы заглядывали в их паспорт? Или тетя Роза нашептала?))))

\iusr{Маргарита Марченко}
\textbf{Igor Neronov} конечно нет!

\begin{itemize} % {
\iusr{Igor Neronov}
\textbf{Маргарита Марченко} вот, так и раскрываются подробности пятой графы \enquote{Б} класса...))))

\iusr{Anna Vasilyuk}
\textbf{Igor Neronov} вирішили побути ложкою дьогтю в бочці з медом?!! @igg{fbicon.face.tears.of.joy} дякувати не треба.

\iusr{Igor Neronov}
\textbf{Anna Vasilyuk} 

Ну, не нравится мне Высоцкий, ни песни его, ни страсть к алкоголю и наркотикам,
так что, мне закатить глаза к небу и прохрипеть про \enquote{черный пистолет}?)))

\iusr{Татьяна Демидова}
\textbf{Igor Neronov} что вы знаете про его страсти? Только то, что вам написали. Хотя, Пушкин тоже не всем нравился....

\iusr{Маргарита Марченко}
\textbf{Igor Neronov} коНЭшно!
\end{itemize} % }

\end{itemize} % }

\iusr{Елена Муравьева}

История оккупации и бабушки Высоцкого очень не проста. Церковный брак не спас
бы ее от Бабьего Яра. К тому она осталась жить в том же районе, где жила до
войны и ее - еврейку и мать двух офицеров - могли опознать и выдать. Еще
факт... после взрывов на Крещатике ее дом был разрушен и она преспокойно заняла
комнату в центре города. Народ меж тем обживал склепы на кладбище. Жить было
негде. Прочитала недавно ччто ее супруг, работавший в оккупацию кем-то вроде
управдома, заставил несколько человек подписать свидетельство о том, что она не
еврейка.

\begin{itemize} % {
\iusr{Оксана Денисова}
\textbf{Елена Муравьева} 

Честно, я очень не люблю обвинять тех, кто не может себя защитить, да ещё и без
документальных свидетельств. Я считаю, что лучше просто принять факт того, что
она выжила во время войны, и слава Богу!

\begin{itemize} % {
\iusr{Rimma Turovskaya}
\textbf{Оксана Денисова} Я читала, что у нее были очень хорошие соседи, которые ее не выдали. И, кстати, разве ул. Франко пострадала от взрывов? Моя тетя жила на этой улице и я часто там бывала.

\iusr{Оксана Денисова}
\textbf{Rimma Turovskaya} Вы абсолютно правы, ее очень любили соседи, и она осталась жить в своём доме на улице Франко.

\iusr{Елена Муравьева}
\textbf{Оксана Денисова} понять подоплеку событий вовсе не значит обвинять. я, во всяком случае, не обвиняла эту женщину. тем паче,информации о ней мало и авторы оной не отличаются объективностью

\iusr{Nataliia Yakubenko}
\textbf{Оксана Денисова} ну я думаю, что это не обвинение, а свидетельство любви...
\end{itemize} % }

\iusr{Раиса Антизерская}
\textbf{Елена Муравьева} Слава Богу, что удалось выжить... Хорошо, что были люди, которые помогали...

\iusr{Палладій Юлия}
Спасибо !
Вечная память !!!!

\end{itemize} % }

\iusr{Ольга Галюк}
Дорога память о нем

\iusr{Анна Лебедева}

Просто вау! Спасибо, хотела спросить не известны ли вам адреса на Франка где
бабушка жила и на Крещатике где работала? Буду благодарна)

\begin{itemize} % {
\iusr{Оксана Денисова}
\textbf{Анна Лебедева} 

На Франко дом №20, он сохранился, квартира была на 2-м этаже. А на Крещатике
честно говоря, не знаю, где был салон, но Крещатик был весь разрушен, так что
дом все равно не сохранился.

\begin{itemize} % {
\iusr{Анна Лебедева}
\textbf{Оксана Денисова} ах, большое спасибо)

\iusr{Igor Neronov}
\textbf{Оксана Денисова} на ФранкО, в русском языке, мужской род, подобные фамилии не склоняются..)

\iusr{Иван Письменко}
\textbf{Оксана Денисова} Там даже квартиры в том состоянии уже однозначно нет. Все покупали и перепланировали.

\iusr{Оксана Денисова}
\textbf{Igor Neronov} исправила, спасибо!

\iusr{Оксана Денисова}
\textbf{Иван Письменко} Наверное, нет, но дом сохранился.

\iusr{Карине Киракосян}
\textbf{Оксана Денисова} я родилась на Франко 17, значит 20 через дорогу, пытаюсь вспомнить какой он @igg{fbicon.thinking.face} 

\iusr{Оксана Денисова}
\textbf{Карине Киракосян} Желтый, с лепными украшениями на фасаде.

\iusr{Карине Киракосян}
\textbf{Оксана Денисова} спасибо

\iusr{Кseniia Novikova}
\textbf{Карине Киракосян}

\ifcmt
  ig https://scontent-frx5-1.xx.fbcdn.net/v/t39.30808-6/272293081_10159815217713308_3033886365081010353_n.jpg?_nc_cat=110&ccb=1-5&_nc_sid=dbeb18&_nc_ohc=NalVLAXiOB4AX9rsg3I&_nc_ht=scontent-frx5-1.xx&oh=00_AT_saBVtOwMBeCe-S_EvpcEborVj7F4qOI9CQR9wSnVK8g&oe=61F792F9
  @width 0.3
\fi

\iusr{Сергей Евтушенко}
\textbf{Igor Neronov}, это так принципиально?

\iusr{Карине Киракосян}
\textbf{Кseniia Novikova} спасибо большое! Я уже поняла, там моя подруга живёт!

\iusr{Анна Земко}
\textbf{Карине Киракосян} 

На розі Франка і Чапаєва. Там, де місточок до під'їзду: коли повертаємо з
Чапаєва - можна по східцях пройти, а можна по тротуару піднятися круто вверх)
Я там ходила сотні разів, і через двір також. Моя бабушка мешкала на Чапаєва у
70-90-х роках)

\end{itemize} % }

\iusr{Иван Письменко}
\textbf{Оксана Денисова} Нужно посмотреть, интересно

\iusr{Светлана Манилова}
\textbf{Анна}, салон был возле нынешней Европейской площади.

\end{itemize} % }

\iusr{Тоня Олешко}

В этом же 1971, в какой то из дней, между спектаклями, Владимир Высоцкий дал
концерт в КОНФЕРЕНЦ-ЗАЛЕ АН Украины, на Владимирской. Сотрудники н.-и.
институтов математики, зоологии и ботаники заполнили зал, помню этот концерт,
как мы старались успеть цветы артисту вручить. Мне повезло!

\begin{itemize} % {
\iusr{Оксана Денисова}
\textbf{Тоня Олешко} Как Вам повезло, завидую!

\iusr{Тоня Олешко}
\textbf{Оксана Денисова} Да, воспоминания молодости...
\end{itemize} % }

\iusr{Lada Luzina}

А известно на каком кладбище бабушка?

\begin{itemize} % {
\iusr{Оксана Денисова}
\textbf{Lada Luzina} Насколько я знаю, она похоронена на Байковом кладбище, но могилу, честно говоря, я сама не видела.

\iusr{Светлана Манилова}
\textbf{Lada}, на Байковом.

\iusr{Сергей Константинович}
\textbf{Lada Luzina} хотите навестить
\end{itemize} % }

\iusr{Карине Киракосян}
И я родилась на улице И. Франко @igg{fbicon.heart.red}

\iusr{Татьяна Жалнина}
84 года могло бы исполнится

\iusr{Ольга Оксимец}
Обожаю Высоцкого! В моей машине всегда звучат его песни!


\iusr{Alla Kenya}
Любимейшому!Помним и любим @igg{fbicon.hands.pray} 

\iusr{Emilio Sacarias Estrada Bencomo}
@igg{fbicon.heart.red} @igg{fbicon.hands.pray} 

\iusr{Ирина Харченко}

\ifcmt
  ig https://scontent-frx5-1.xx.fbcdn.net/v/t39.1997-6/s180x540/245662798_290684512904195_58795289236017336_n.png?_nc_cat=111&ccb=1-5&_nc_sid=ac3552&_nc_ohc=YlRMubQOHqwAX8QAOzr&_nc_ht=scontent-frx5-1.xx&oh=00_AT_MYl017Orf0D42mQ7i5tuoHum_eVV-vK_hamIp13cmng&oe=61F67FFA
  @width 0.1
\fi

\iusr{Janna Soroka}
Я никогда не забуду спектакль в театре муз. Комедии и подпольный концерт в
клубе шампанских вин на Куреневке

\begin{itemize} % {
\iusr{Оксана Денисова}
\textbf{Janna Soroka} Как Вам повезло, что Вы это видели «вживую»!

\iusr{Tatiana Goncharova}
\textbf{Janna Soroka} а наша бабушка работала на заводе шампанских вин.

\iusr{Alisa Gerasimenko}
\textbf{Janna Soroka}

Я была на спектакле \enquote{Антимиры} в театре муз. комедии. Театр давал по два-и даже
три спектакля в день. Мне позвонили быстро сказав, что есть лишний билет, и
спектакль вот- вот начнётся. Я в ду`ше, но мне 21год. Это был мой рекорд по
бегу. Я бежала от кинотеатра Киев до Музкомедии, по дороге сломался каблук, но
я почти не опоздала. И этот спектакль со всеми актёрами я помню, как один из
немногих шедевров, которые пришлось увидеть за долгую жизнь.

Ещё я была на концерте, который артисты театра давали в ИПМ (Институт проблем
материаловедения). Это были Владимир Высоцкий, Вениамин Смехов и Хмельницкий.
Молодые, озорные, весёлые и очень хорошие. Конечно, ничего подобного я в жизни
не видела и не слышала. Помню, как сидела на полу и просто задыхалась от
восторга и счастья увиденного.

Странно, что никто из моих сотрудников не вспомнил.

Незабвенная история о любимых артистах.
\end{itemize} % }

\iusr{Jadviga Zemaitis}
Будучи ребенком, он частенько летом с родителями гостил у своих родственников в г. Гайсине Винницкой области...

\iusr{Оксана Денисова}
\textbf{Jadviga Zemaitis} Да, я читала об этом. Спасибо Вам!

\iusr{Світлана Бондаренко}

Спасибо за фото всеми любимого, особенно нашим поколением, Владимира Высоцкого.
Он был у нас в КТИЛП с концертом 70-71г., где я училась. Помню его с гитарой, в
чёрном гольфе... Была тёплой встреча @igg{fbicon.thumb.up.yellow}  @igg{fbicon.hand.ok}  @igg{fbicon.heart.sparkling} 

В нашей семье были пластинки, записи его песен. Бравооо...

А на стене память о нём @igg{fbicon.rose}{repeat=2} 

\ifcmt
  ig https://scontent-frt3-1.xx.fbcdn.net/v/t39.30808-6/272425558_2087102341452211_3757429442687306194_n.jpg?_nc_cat=108&ccb=1-5&_nc_sid=dbeb18&_nc_ohc=8j4WuDVxfUwAX_5duWG&_nc_oc=AQlvlFWwqoX-Yj0RrG0rVrIQvBAvGNEiPbeOzFTbvCsupp_B45A2T4-W61bD6feBMOs&_nc_ht=scontent-frt3-1.xx&oh=00_AT9JrpS9o3Nw2D7gA8uka9MxxRhXjxRbrPX5pLeooTMwAg&oe=61F80D9A
  @width 0.2
\fi

\iusr{Оксана Денисова}
\textbf{Світлана Бондаренко} Спасибо Вам за Ваши воспоминания, как Вам повезло, что видели его!

\iusr{Світлана Бондаренко}

\ifcmt
  ig https://scontent-frt3-1.xx.fbcdn.net/v/t39.30808-6/272685342_2087105004785278_3283427069043397854_n.jpg?_nc_cat=106&ccb=1-5&_nc_sid=dbeb18&_nc_ohc=i4o5IEINC-YAX_vZCdB&_nc_ht=scontent-frt3-1.xx&oh=00_AT_VTfotuAC2jvLIy6kcsGctLpIKVKRiwQeaB1UA9QXWag&oe=61F78849
  @width 0.2
\fi

\end{itemize} % }

