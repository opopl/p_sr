%%beginhead 
 
%%file 04_02_2022.fb.mrpl.capital.of.culture.1.mrpl2ua_kulturnyj_kod
%%parent 04_02_2022
 
%%url https://www.facebook.com/mrpl.capital.of.culture/posts/pfbid02o3rXatBJXxMLXmhK8oXstMbVhoucfArxtkeKkPRJy8pMFqyMUJ8zhXUJYPrFJUMkl
 
%%author_id mrpl.capital.of.culture
%%date 04_02_2022
 
%%tags 
%%title "Маріуполь Україні - Україна Маріуполю" - зшиваємо Україну через культурний код!
 
%%endhead 

\subsection{\enquote{Маріуполь Україні - Україна Маріуполю} - зшиваємо Україну через культурний код!}
\label{sec:04_02_2022.fb.mrpl.capital.of.culture.1.mrpl2ua_kulturnyj_kod}

\Purl{https://www.facebook.com/mrpl.capital.of.culture/posts/pfbid02o3rXatBJXxMLXmhK8oXstMbVhoucfArxtkeKkPRJy8pMFqyMUJ8zhXUJYPrFJUMkl}
\ifcmt
 author_begin
   author_id mrpl.capital.of.culture
 author_end
\fi

Новий культурно-пізнавальний проєкт \enquote{Маріуполь Україні - Україна
Маріуполю} - зшиваємо Україну через культурний код!

Донедавна більшість українців сприймали Маріуполь як промислове місто на краю
землі. У 2021 року Маріуполь змінив цей стереотип та показав всій країні свій
новий образ — творче місто біля теплого Азовського моря.

У 2022 році продовжуємо діалог мовою мистецтва від Сходу до Заходу, від Півдня
до Півночі та розкриваємо справжні українські цінності, якими пишаємось:
мистецтво, музика, люди, промисловість. Бачимо цю активність як стійку
можливість зшити Схід та Захід і довести що для України немає Сходу або Заходу,
а є на Сході або на Заході України. Ми єдині!

Цього року плануємо охопити 5 міст з різних куточків України та наповнити їх
культурно-мистецькими перлинами, показати приховані скарби, про які часто не
здогадуються мешканці.

10 арт-інсталяцій у подарунок отримають учасники проєкту: 5 об'єктів - символів
нашого міста біля моря відправляться до міст-партнерів, а 5 арт-символів
встановлять у Маріуполі як подарунок від від цих міст. Щоб завітавши у гості,
можна було відчути атмосферу рідного міста в самому віддаленому куточку країни
і навпаки - відчути частинку України в Маріуполі.

Навколо нових інсталяцій розвернуться  культурні події, які продемонструють
унікальність Маріуполя та інших міст України.

Слідкуйте за оновленнями!

Реалізація проєкту \enquote{Маріуполь Україні – Україна Маріуполю} стала можливою
завдяки щирій підтримці американського народу, наданої через Агентство США з
міжнародного розвитку (USAID) у рамках підтримки від проекту USAID
\enquote{Демократичне Врядування у Східній Україні}.

Маріупольська міська рада
