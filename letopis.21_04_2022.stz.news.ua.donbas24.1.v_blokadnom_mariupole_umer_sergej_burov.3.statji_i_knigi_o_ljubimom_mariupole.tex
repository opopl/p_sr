% vim: keymap=russian-jcukenwin
%%beginhead 
 
%%file 21_04_2022.stz.news.ua.donbas24.1.v_blokadnom_mariupole_umer_sergej_burov.3.statji_i_knigi_o_ljubimom_mariupole
%%parent 21_04_2022.stz.news.ua.donbas24.1.v_blokadnom_mariupole_umer_sergej_burov
 
%%url 
 
%%author_id 
%%date 
 
%%tags 
%%title 
 
%%endhead 

\subsubsection{Статьи и книги о любимом Мариуполе}

В каждом издании, где публиковался Сергей Буров, о нем отзывались как об очень
уважаемом авторе. Его очерки всегда читались с большим интересом и получали
хорошие отклики у читателей. Многие благодаря рассказам Бурова узнали город с
новой стороны, начинали замечать детали и поняли важность исторического
наследия. Сергей Давыдович был тем, благодаря кому многие заново полюбили
родной Мариуполь.

Его первая книга \enquote{Мариуполь. Былое}, созданная по материалам передач и
публикаций, вышла в свет в 2003 году. А в 2011 году в редакции \enquote{Приазовского
рабочего} издана следующая книга \enquote{Мариуполь и мариупольцы} — она стала
необходимой для библиотек города.

\begin{leftbar}
	\begingroup
		\bfseries
\qbem{Сергей Буров всегда был частью дружного коллектива редакции
\enquote{Приазовского рабочего}, его невероятные живые очерки мы публиковали с
удовольствием — в них оживал былой Мариуполь, он говорил голосами людей
былой эпохи и передавал их эмоции и настроения. Сергей Давидович не
просто создавал историю Мариуполя, он всем сердцем любил наш город и
популяризировал эту гордость и любовь в широкой читательской аудитории.
В типографии \enquote{Приазовского рабочего} вышли серии книг Сергея Бурова, и
эти книги всегда пользовались популярностью — мариупольцы и гости
нашего города с удовольствием погружались в живые истории прошлого,
становились их частью. Интеллигент и чуткий историк, очень скромный
человек и настоящий мариуполец — таким он запомнится мне навсегда}, —
говорит о постоянном авторе издания Елена Калайтан, главный редактор
газеты \enquote{Приазовский рабочий}.
	\endgroup
\end{leftbar}

О своем последнем разговоре с Сергеем Буровым вспоминает и заместитель главного
редактора газеты Анастасия Дмитракова. Она также подчеркнула, что он был в
первую очередь Другом для коллектива.

\begin{leftbar}
	\begingroup
		\bfseries
\qbem{Его любовь к родному городу была настолько велика, что малейшая
историческая неточность в чьих-либо опубликованных текстах — и он сразу
звонит и объясняет, почему автор не прав. Когда началась война и
пятничный номер газеты от 25 февраля не вышел, Сергей Давыдович
позвонил мне и спросил: \enquote{Настя, был Приазовский? Ждать?} Это был наш
последний разговор...} — делится она.
	\endgroup
\end{leftbar}

Основатель проекта по реставрации старины \enquote{Жизнь дверей имеет значение} Ярослав
Федоровский был одним из тех, кому Сергей Буров открыл Мариуполь заново.
Воспоминания Ярослава о своем знакомстве с краеведом такие же теплые, как и у
многих, кто знал лично этого человека.

Первую его книгу Ярослав прочитал в середине нулевых — на рассвете своего
интереса к изучению истории города.

\begin{leftbar}
	\begingroup
		\bfseries
\qbem{Буров был одним из первых, кто написал прекрасные книги и показал
обстановку, которая царила в конце 19-го и начале 20 веков в Мариуполе.
У меня были его книги, можно сказать, что он был моим учителем. С
Сергеем Буровым, этим замечательным человеком, я познакомился на одном
из краеведческих собраний в 2007-м или 2008 году. Потом я делился с ним
своими мыслями, задавал вопросы и получал ответы. Он сделал очень много
для города и был фигурой, которая помогла сотням, тысячам людей узнать
историю, о которой раньше даже и не задумывались. Из его книг я
познакомился с несколькими знаменитыми мариупольцами послевоенного
периода и открыл для себя город с другой стороны}, — сказал Ярослав
Федоровский.
	\endgroup
\end{leftbar}

Сергей Буров был также почетным гражданином Мариуполя, членом Национального
союза краеведов Украины и Национального союза кинематографистов Украины. 28
июня 2022 года ему бы исполнилось 85 лет...

