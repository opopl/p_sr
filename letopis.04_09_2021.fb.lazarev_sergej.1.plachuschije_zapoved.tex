% vim: keymap=russian-jcukenwin
%%beginhead 
 
%%file 04_09_2021.fb.lazarev_sergej.1.plachuschije_zapoved
%%parent 04_09_2021
 
%%url https://www.facebook.com/LazarevSergeyNikolaevich/posts/412008893616221
 
%%author_id lazarev_sergej
%%date 
 
%%tags biblia,chelovek,vera,zapoved
%%title "Плачущие" - это те, кто видят во всем Божественную волю
 
%%endhead 
 
\subsection{\enquote{Плачущие} - это те, кто видят во всем Божественную волю}
\label{sec:04_09_2021.fb.lazarev_sergej.1.plachuschije_zapoved}
 
\Purl{https://www.facebook.com/LazarevSergeyNikolaevich/posts/412008893616221}
\ifcmt
 author_begin
   author_id lazarev_sergej
 author_end
\fi

Вы когда-нибудь задумывались над второй заповедью блаженства "Блаженны
плачущие"?

Многие верующие восприняли это буквально и старались плакать по поводу и без. А
кто, на самом деле, плачет? Ребенок.

\ifcmt
  pic https://scontent-frt3-1.xx.fbcdn.net/v/t39.30808-6/218955263_381544693329308_6669405571185069813_n.jpg?_nc_cat=104&ccb=1-5&_nc_sid=730e14&_nc_ohc=nCcUXfXJCqwAX-MsD7H&_nc_ht=scontent-frt3-1.xx&oh=a39922b6914d0133facdcae5deb6d9bc&oe=613B3C43
  width 0.8
\fi

Потому что он маленький и не может дать сдачи. Когда большой обижает
маленького, - маленький плачет.

Тот, кто видит вокруг себя только людей,- будет ненавидеть и желать отомстить.

А тот, кто ощущает во всем Божественную волю, - не будет мстить, не станет
отвечать злом на зло.

Мстить Богу Бесполезно.

"Плачущие" - это те, кто видят во всем Божественную волю.

Они блаженны, то есть счастливы, потому что понимают,что все жизненные
трудности - это задачи, которые нужно решить, что все происходящее имеет высший
смысл и бояться, обижаться, сожалеть-бесполезно.

Нужно верить, бороться и постоянно пытаться реализовывать свои желания без
внутреннего озлобления и уныния.

Ведь тот, кто верит, то есть стремится и пытается без страха и злости, - тот
может и горы свернуть.

Тот, кто верит, - не станет бояться, не сложит руки, не начнет искать
виноватых, а будет, улучшая свой характер, приближаться к Богу.

А возможности для развития души, для ее изменения у нас огромные и доступны
каждому.

С.Н. Лазарев Опыт выживания, часть 7
