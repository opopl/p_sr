% vim: keymap=russian-jcukenwin
%%beginhead 
 
%%file 17_02_2022.tg.lesev_igor.1.chto_posleduet_za_priznaniem_lnr_dnr
%%parent 17_02_2022
 
%%url https://t.me/Lesev_Igor/314
 
%%author_id lesev_igor
%%date 
 
%%tags dnr,donbass,lnr,ukraina,vojna
%%title Что последует за возможным признанием ДНР и ЛНР
 
%%endhead 
 
\subsection{Что последует за возможным признанием ДНР и ЛНР}
\label{sec:17_02_2022.tg.lesev_igor.1.chto_posleduet_za_priznaniem_lnr_dnr}
 
\Purl{https://t.me/Lesev_Igor/314}
\ifcmt
 author_begin
   author_id lesev_igor
 author_end
\fi

Что последует за возможным признанием ДНР и ЛНР.

16 февраля на Украине случился новый большой праздник – День отражения
российской агрессии. Миллионы патриотов по призыву Зеленского вывесили
национальные флаги, надели вышиванки и в 10 утра спели гимн. И тут уже шансов у
Путина не было. Ровно в 3 часа ночи он дрогнул, вторжение было отменено, и
украинцы одержали легендарную перемогу, которую теперь потомки будут воспевать
веками.

И это ведь не выдержка из психиатрического диагноза пациента. Это абсолютно
официальная позиция украинских властей, где нет даже толики гиперболы. Чтобы
понимать и разделять в России подобные чувства, нужно быть немножко Лией
Ахеджаковой, т.е., патентованным русофобом.

Но между очередной грандиозной победой Украины, случилось событие не из мира
хоббитов и эльфов, а из мира реальной политики. Госдума проголосовала за
обращение к Путину признать независимость ДНР и ЛНР.

Российская политика достаточно закрытая и специфическая. Важные решения,
особенно стратегического направления во внешней политике, здесь не принимаются
эмоционально и с кондачка. Не нужно быть каким-то заслуженным специалистом по
ру-политике, чтобы понимать, что такое голосование Госдумы могло быть в чем-то
неожиданным для Владимира Путина. А это ключевой момент. Это означает, что
решение по будущему ЛДНР, да и в целом, будущему Украины полностью моделируется
Кремлем. С соответствующими опциями.

Причина, почему появилось подобное решение тоже вполне очевидно. Минские в их
политической части полностью саботируются украинской стороной, и в Кремле
максимально поднимают ставки. Правда, в России существует иллюзия, что Минские
соглашения возможно выполнить, была бы только политическая воля. Но это именно
иллюзия, и я коротко поясню почему.

Существуют две Украины – ленинская и бандеровская. Вернее, одна проходит
трансформация в другую. Современная Украина – это наследница УССР, а
субъектность последней создали большевики. Они начертили границы Украины,
проводили политику коренизации, при них же УССР стала членом-основателем ООН,
при Союзе там появился свой МИД, свои конституции и все прочие атрибуты
государственности, которые финализировались в 1991 году в сами знаете что.

Но ленинскую Украину основали два ключевых народа – русские и украинцы.
Соответственно, и республика была русских и украинцев. Были там, конечно же, и
десятки других народов, но мы говорим о базовых величинах. А базовые – это
русские и украинцы.

С 1991 года мультикультурный проект стал сворачиваться, и появилась концепция
бандеровской Украины. Украины для украиноязычных украинцев. Сначала появилась
конституция, где государственным языком был вписан только украинский. А в итоге
все докатилось до сегрегационного закона против русского языка и закона о
коренных народах. Это реалии современной Украины, в которой русским и русскому
нет места.

И вот Минские соглашения. Проблема их реализации для Киева ведь не в амнистии,
не в проведении выборов или контроле над границей. ЛДНР – это сохранившийся
осколок ленинской Украины, при том той ее части, где исторически доминирует
русское. А Минские предполагают этот осколок законсервировать в виде
автономного статуса с внутренним самоуправлением. И, естественно, со свободным
функционированием русского языка, что для бандеровской Украины в принципе
неприемлемо. Как можно будет пояснить, что в 100\%-ом русскоязычном Мариуполе
запрещены все русскоязычные школы, а в соседнем Донецке детям разрешено
образование на русском? Это несовместимые дефиниции, как наркомания и здоровый
образ жизни.

Именно в этом, а ни в каком бы то ни было еще плане, Минские не принимаются и
физически не могут быть приняты в бандеровской Украине. В Украине для
украиноязычных украинцев не может быть русскоязычных анклавов. Иначе вся
конструкция посыплется, а она для адептов этнической модели страны имеет
экзистенциальное значение.

Теперь мы имеем обращение Госдумы, призывающее к легализации
самопровозглашенных республик. Реакцию Путина мы уже знаем – он ответил, что
пока еще нужно исчерпать все возможности по реализации Минских. А выше я уже
сказал, почему этих возможностей нет. Поэтому, давайте прикинем, к чему может
привести легализация ЛДНР.

Первое, что, наверное, волнует власти и обывателя РФ – это новые санкции
Запада. Тут нужно исходить из двух значений – что есть и что может быть.

Есть то, что санкции против России со стороны Штатов/Запада – явление
постоянное. Когда-то была поправка Джексона-Вэника, ее поменяли на список
Магницкого, потом пошли санкции по Крыму и Донбассу. Вдруг случится
невероятное, и отменят последние санкции, то тут же придумают «список
Навального» или озаботятся судьбой чеченских геев. И здесь вопрос не в
«неправильном» политическом режиме России, а в том, что англосаксонский мир
исповедует лютую русофобию. Меняться что-то радикально может только под
воздействием каких-то невероятных тектонических сдвигов, как это уже было в
разгар Второй Мировой, но это как раз непредвиденные исключения.

Другое дело, а могут ли быть против России введены санкции «судного дня»? Ну
вроде отключения от SWIFT или эмбарго на продажу углеводородов? Учитывая общую
шизофрению западного политикума, однозначно на это ответить сложно. Не
запускают ведь тот же «Северный поток-2», хотя и давятся от цены на газ. Но
опять же, Россия по площади и ресурсам – это не Иран и не КНДР. По сути, это
отдельная геополитическая планета, которую в принципе невозможно изолировать в
многополярном мире. Зато Россию подобными санкциями можно подтолкнуть на прямой
военный союз с Китаем, чего очень опасаются в Вашингтоне.

Кроме того, здесь есть и другой момент. Вводя драконовские санкции против
России, но при этом отказываясь с ней открыто воевать – страна ядерная и
дураков нет – тут же разрушаются сдерживающие факторы против Москвы на том же
постсоветском пространстве. Условно говоря, если ты вводишь «смертельные
санкции» за признание ЛДНР, за которыми уже ничего страшнее нет, что тогда
России будет мешать присоединять то, что она посчитает нужным и своим? Санкции
ведь все равно страшнее не станут, а воевать никто за ту же Украину не
собирается, о чем публично на Западе 50 раз заявили.

Второй момент, которым в России апеллируют против признания ЛДНР – это утрата
«русского голоса» в Большой Украине. Так его в бандеровской Украине и так быть
не может. Только на прошлой неделе закрыли последний телеканал, который топил
за выполнение Минских. Любое политическое объединение или СМИ, выступающее хотя
бы за умеренный конструктивизм в отношениях с Россией, на Украине тут же
маргинализируется. И сломить этот процесс какими-то внешними договоренностями
уже не получится.

Третий момент, который уже по-настоящему беспокоит и Киев – это вопрос о
границах. ДНР и ЛНР претендуют на всю территорию Донецкой и Луганской областей.
Признавая субъектность республик, Москва, очевидно, будет за них еще и
расписываться в плане безопасности. Тут можно проводить аналогию с событиями в
2008 году в ЮО и Абхазии. Боевые действия тогда велись только в Южной Осетии,
но под шумок легитимизации проскочили еще и абхазы. Более того, им вернули еще
и Кодорское ущелье, которое с 1992 года было под контролем Грузии.

Это очень важный момент, потому что здесь уже стоит вопрос об «окончательном
решении» вопроса Донбасса. При этом решение, вполне возможно, вооруженным
путем.

По сути, Россия подобралась к своему Рубикону на донбасском направлении.
Минские в политической части Зеленским и Ко не будут, да и не могут быть
выполнены. Это объективно, потому что та модель Украины, которая сейчас
выстраивается, не допускает ничего русского в своем теле. А Минские соглашения
– это именно о таком допущении.

И есть уже репутационное решение, которое персонально на себя взял Путин.
Признавать или не признавать. А избиратель Путина – это не Лия Ахеджакова, не
Ксюша Собчак, и не Алексей Венедиктов. Это хмурые металлурги из Челябинска и не
менее хмурые водители маршруток из Кирова, у которых по Донбассу и Украине
давно уже сложилась вполне однозначная позиция.
