% vim: keymap=russian-jcukenwin
%%beginhead 
 
%%file 26_11_2020.fb.promovugroup.1.kollaborazionizm
%%parent 26_11_2020
 
%%url https://www.facebook.com/groups/promovugroup/permalink/868023347104861/
 
%%author 
%%author_id 
%%author_url 
 
%%tags 
%%title Чи потрібен на 7 році війни Закон про колабораціонізм?
 
%%endhead 
 
\subsection{Чи потрібен на 7 році війни Закон про колабораціонізм?}
\label{sec:26_11_2020.fb.promovugroup.1.kollaborazionizm}
\Purl{https://www.facebook.com/groups/promovugroup/permalink/868023347104861/}

Чи потрібен на 7 році війни Закон про колабораціонізм? Так, абсолютно потрібен.

Адже ми спостерігаємо, як реваншує п‘ята колона кремля, як повертаються ті, хто
в 2014 році пішов у політичне небуття, як прокремлівські медіа все більше
займають інформаційний простір України.

До, речі, петиція про необхідність такого закону набрала необхідну кількість
голосів на сайті президента.
(\url{https://petition.president.gov.ua/petition/105082})

Тепер справа за Верховною Радою та владою, яка вже півтора роки закриває очі, а
часто й відверто потакає поверненню антиукраїнського порядку денного.

Підтримуєте таке законодавство? 

Поширюйте, ставте «+» в коментарях.

\ifcmt
pic https://scontent.fiev6-1.fna.fbcdn.net/v/t1.0-9/127637604_3547482268622883_276819617347626576_o.jpg?_nc_cat=1&ccb=2&_nc_sid=8bfeb9&_nc_ohc=BGfAo82flOQAX88ZRpH&_nc_ht=scontent.fiev6-1.fna&oh=842dae7e36ddfcea1126ba78d2499bfd&oe=5FE4A5D3
fig_env wrapfigure
width 0.4
\fi
