% vim: keymap=russian-jcukenwin
%%beginhead 
 
%%file 16_03_2023.fb.karpenko_sergij.mariupol.1.segodnya_rovno_god_k.cmt
%%parent 16_03_2023.fb.karpenko_sergij.mariupol.1.segodnya_rovno_god_k
 
%%url 
 
%%author_id 
%%date 
 
%%tags 
%%title 
 
%%endhead 

\qqSecCmt

\iusr{Zakharchuk Yuriy}

Добре, що ви живі.

\iusr{Iuliia Isliamova}

Дочитала... плачу... у каждого своя страшная история...

\begin{itemize} % {
\iusr{Сергей Карпенко}
\textbf{Iuliia Isliamova} и она не закончена

\iusr{Iuliia Isliamova}
\textbf{Сергей Карпенко} да, к сожалению.
\end{itemize} % }

\iusr{Inna Zhuravska}

Я пам'ятаю як Ви вийшли на звязок в фб і я тоді дуже зраділа, що ви живі і що вибрались🙏

\iusr{Alexandr Korogod}

Да уж.....🤨Помню, как я мысленно с тобой разговаривал, когда не было связи,
чтобы валили быстрее. 🙄🤔. Сейчас вспоминая, нет ощущение времени . В какой
день, что?..Все восприятие шло единым потоком. Время будто замерло. Только
главное ощущение было всегда. Проскочим! 🙌

\iusr{Alexander Unger}

\enquote{на работу} (с)

промедли еще немного и стало бы труднее. Все переживали. А у нас их на въезде
остановили.

\iusr{Соня Гофман}

Ми пережили цей жах на Київщині. Коли орки зайшли в село, запитала сусіда,
полковника у відставці; Віктор Петрович, тепер ми в їхньому тилу? -Нет, мы - на
передовой. Стало сумно і страшно.
\url{https://www.facebook.com/100006197209390/videos/933903947746750}

\begin{itemize} % {
\iusr{Elen Len}
\textbf{Соня Гофман} И мы в Ирпене на передовой сидели до 18 марта, тогда один день был как неделя, и не выходя из состояния какого-то аффекта, просто на автомате....
\end{itemize} % }

\iusr{Julia Biobalast}

Добре, Сережа, що ви живи

\begin{itemize} % {
\iusr{Сергей Карпенко}
\textbf{Юлия Максимова} мабуть так
\end{itemize} % }

\iusr{Кирило Володимирович}

Как же здорово, что смог выехать! Потому с твоим характером в оккупации бкз шансов

\iusr{Oxana Kovalchuk}

А я до сих пор помню, почти наизусть нашу переписку в личке за день до войны

\enquote{ Не закончится хорошо. Они попрут, но огребут по полной. Напасть на Украину -
это, как мать ударить или единорога убить. Они будут прокляты до седьмого
колена, а мы будем жить!}......

Всё сбывается, док!

\begin{itemize} % {
\iusr{Сергей Карпенко}
\textbf{Oxana Kovalchuk} именно так

\iusr{Наталья Чалая}
\textbf{Oxana Kovalchuk} 

сейчас понимаю, что у многих историй есть нечто общее \enquote{Господь сохранил}... выехали в последние дни и часы...

\iusr{Сергей Карпенко}
\textbf{Наталья Чалая} просто мы не слышим тех, кто не успел, кому не так повезло(

\iusr{Oxana Kovalchuk}
\textbf{Сергей Карпенко} но тех,кого слышим - реально кто-то сохранил!
Моя история \enquote{год назад} ещё продолжается, но мистики просто \enquote{забагато}... И так у всех, кто выжил и смог.
\end{itemize} % }

\iusr{Tatyana Yemelianenko}

Можна зробити репост собі на сторінку?

\begin{itemize} % {
\iusr{Сергей Карпенко}
\textbf{Tatyana Yemelianenko} вэлькам
\end{itemize} % }

\iusr{Оля Суділовська}

Вы вернетесь. Мы все вернемся. Держись.

\begin{itemize} % {
\iusr{Сергей Карпенко}
\textbf{Оля Суділовська} вопрос только куда

\iusr{Оля Суділовська}
\textbf{Сергей Карпенко} в свободный Мариуполь. Поверь, вопрос \enquote{куда} решится.
\end{itemize} % }

\iusr{Елена Бронникова}

Плачу. Жена моего одноклассника выложила в ФБ их эвакуацию из Кириловки. Она
очень подробно описывала, частями. Не сразу решилась. Но об этом надо писать
каждому кто прошёл через это. Это нужно для настоящего и будущего.

\begin{itemize} % {
\iusr{Сергей Карпенко}
\textbf{Елена Бронникова} 

будущее могу только умозрительно выстроить в голове. Но толку с этих
построений, если у противника другая логика

\iusr{Елена Бронникова}
\textbf{Сергей Карпенко} 

Я не о том. Я о том, что бы не было в будущем поползновений рассказывать, что
всё не так однозначно. После года войны всё равно бытует мнение об
\enquote{освободителях} и \enquote{сами виноваты}.

Я про таких как я, живущих без тревог и взрывов, знающих о действительности из
СМИ и от знакомых. Я про то, что это должно быть в памяти, что бы не было
иллюзий. Я о том что моя боль несравнима с вашей, и сравнивать не нужно, но
надо знать.

Сумбурно, но вот так как-то.

\iusr{Сергей Карпенко}
\textbf{Елена Бронникова} ну у меня иллюзий точно нет, хотя знаю тех кто прошел этот же ад и сейчас несет прорашистскую пургу. Так что не все так однозначно))) ну для некоторых)

\iusr{Елена Бронникова}
\textbf{Сергей Карпенко} Эх, понимаю. Спасибо Вам.
\end{itemize} % }

\iusr{Hanna Luniachenko}

Мы с мамой тоже сегодня вспоминать будем этот день как день выхода из
Мариуполя. Как все утро бегали по паркингу, в котором жили две недели, и
напрашивались попутчиками в машину. Спасибо тем незнакомым людям, у которых
было место и которые вывезли нас.

\iusr{Alevtina Makmak}

🙌🙌🙌🙌🙌🙌🙌

\iusr{Сергей Чимшит}

Так а что курить-то? Бросил? 🙂

\begin{itemize} % {
\iusr{Сергей Карпенко}
\textbf{Сергей Чимшит} не, у меня был айкос и мешок стиков к нему 🤣

\iusr{Сергей Чимшит}
Судьба давала шанс...
\end{itemize} % }

\iusr{Sergii Marushchenko}

Пам'ятаю, дуже за тебе і твою сім'ю тоді переживав...

Добре, що вирвалися...

\iusr{Юлія Кальницька}

З днем народження!

\iusr{Alexander Matkovsky}

Вот именно. У него главная цель жизни - нагадить побольше. Ему это - в кайф(

\iusr{Ольга Истомина}

Я пам'ятаю, коли ви написали, що живі і виїхали, новини з Маріуполя, Харкова,
Ірпінь вбивали, це щастя знати, що хтось вибрався з пекла. Рсня повинна
заплатити за все, а проклятть на їхні голови достатньо.

\iusr{Iryna Dubianska}

Це нереально читати. Страшний, нелюдський злочин.

\iusr{Галина Солдатова}

И мы сегодня ровно год назад выбрались из Мариуполя.

\iusr{Герман Антонов}

Тримайтесь, Бро. Тримайтесь.

\iusr{Romanna Melnychenko}

Страшно навіть читати. Торік ми собі ще не уявляли, наскільки все страшно в
Маріуполі, в Бучі... та скрізь, куди прийшли асвабадітєлі.

\iusr{Alla Boyadjyan}

Какой ужас.

\iusr{Tetiana Tlustova}

Дякую за твої спогади. Рада, що ти вибрався з того пекла! У мене своя річниця
через 4 дня, але то вже інша історія...

\iusr{Виктор Пузанов}

Ты молодец, что остался жив и выжил своих

\begin{itemize} % {
\iusr{Сергей Карпенко}
\textbf{Виктор Пузанов} видимо есть какая то в этом цель

\iusr{Виктор Пузанов}

Я тоже думаю всё это время, что то, что наше село не взяли, это чтоб я ещё
что-то сделал. Живи, доктор, долгих лет тебе

\iusr{Александра Александровна}
\textbf{Сергей Карпенко} конечно есть) у нас же планы))) тамараааа)

\end{itemize} % }

\iusr{Оксана Петренко Кірічевська}

Дві родини з Маріуполя - ваша і ще одна, обидві вирвалися. У другої тяжкі
травми, але без втрат. Дуже тоді хвтлювалася за всіх, добре, що все зараз добре

\iusr{Юлія Козирєва}

Мы в тот день со штуки двигались (бежали) на Кирова, мимо драм театра, видели
то что сним сделали, думали на Кирова безопаснее, если это вообще можно было
назвать безопасностью. Как в замедленной съёмке, очень многие выехали, и это
чувство что все, реально конец...

\iusr{Ірина Олійник}

Тримайтеся, нехай надія вас не лишає. І розповідайте, розповідайте про цих
нелюдей!

\iusr{Horoshyan Denisidze}

Дякую, що ділишся спогадами.

\iusr{Надя Бучинская}

Я до сих пор не верю, что это все происходит. Держитесь. Девочкам привет!

\begin{itemize} % {
\iusr{Сергей Карпенко}
\textbf{Надя Бучинская} тебе отдельное спасибо от нас! 😘
\end{itemize} % }

\iusr{Юлия Свирская}

Читаю и рыдаю

Мы прошли такой же путь

Ад на нашей земле

\iusr{Олена Єрьоменко}

Це щастя, що вижили. Дякую, що поділилися

\iusr{Iryna Rassadina}

Я в Маріуполь заїду післязавтра рік тому.

Жирні сепароорки будуть віджимати мою автівку у Мангуші на НЛО о 7 ранку,
післязавтра.

Наразі я тільки приймаю рішення їхати туди, наодинці, з Дніпра (а до цього з
Харкова)

Завтра я ночую у Бердянську, виснажена 12 годинами у дорозі у Пекло...

\ifcmt
  igc https://scontent-fra5-1.xx.fbcdn.net/v/t39.30808-6/336683637_232574962588285_3988967217781558518_n.jpg?_nc_cat=111&ccb=1-7&_nc_sid=dbeb18&_nc_ohc=AD1Oi5wnOOgAX_MgoDM&_nc_ht=scontent-fra5-1.xx&oh=00_AfBqZd1q6D9X3jcsyXRsfDmD9JE91IYc7Gp3EGmiFwc4IQ&oe=643158E2
	@width 0.4
\fi

\iusr{Larysa Pinner}

Ох, Серёжка, как же я была рада увидеть первое твое сообщение, после того, как
вы выбрались из этого ада...

\iusr{Elena Olena}

Будьте здорові і щасливі!🙏🙏🙏❤️

\iusr{Александра Александровна}

Я потом напишу... сил нет(

\begin{itemize} % {
\iusr{Сергей Карпенко}
\textbf{Александра Александровна} у меня тоже нет. Не знаю зачем я это написал

\iusr{Александра Александровна}
\textbf{Сергей Карпенко} правильно, что написал... многие до сих пор даже не представляют(((
\end{itemize} % }

\iusr{Анна Бурашвили}

Хочу просто обійняти вас..

\iusr{Alla Kuchma}

Обіймаю.

\iusr{Павел Онищенко}

Ужас тільки що пережили Ви та багато людей. Читаю і просто мурашки по тілу.

\iusr{Lyudmyla Meteleshko}

Жуть.

\iusr{Oleg Wishnevski}

Я 25го вывез своих... было стремно

\iusr{Taniuchka Zayceva}

Хочеться сказать слова утешения, а их не существует(

\iusr{Ольга Бровенко}

Ми вибралися з міста 16 Марта, за півгодини до зриву Драмтеатру. Повернули
направо, стали у нескінчену чергу машин і побачили над містом чорний стовб
диму.... зв'язку ще не було, але згодом з'явився і ми узнали про Драм.

\iusr{Олена Хохлова}

Вижити в той час в Маріуполі - лотерея. Ми теж 15 марта побоялися виїхати, не
вірили у той коридор, а 16-го вже поїхали. В Запоріжжі були тільки 20-го,
стільки добиралися.

\iusr{Яна Володимирівна}

Ви не боялися фотографувати підбиту орковську техніку? Ми виїхали 22 березня.
Коли виїжджали, я повидаляла всі фото \enquote{воєнного Маріуполя}. Щоб на блокпостах в
орків не поставали питання що це, та навіщо. Та на всьому шляху до Запоріжжя
було багато підбитої їх техніки, ми раділи дуже), але не фотографували.

\begin{itemize} % {
\iusr{Сергей Карпенко}
\textbf{Яна Володимирівна} боялись. Это делалось на второй телефон, который лежал у кошек в переноске под зассаной пеленкой 😉
\end{itemize} % }

\iusr{Oksana Klimenko}

Мы ехали в 10.00 и приехали в Бердянск к утру практически, тьма этих всех
асвабадители хреновы и улыбки их .. сука твари убить хотелось каждого, а нужно
было делать вид типа рады .. но машины впереди и сзади досматривали даже
мужиков раздевали до носков это мы видели .. я своё ценное спрятала .. и думала
я же женщина, не тронут .. ага фиг как же ошибалась, позже увидела как они и
женщин и детей досматривали .. но в нашем случае сработало, что в машине много
женщин древних, ну так мы выглядели тогда это точно к своим +20 лет как
минимум!.. но пофиг главное выехали и живые ! День выезда наш один !

\iusr{Світлана Єлісєєва}

моя подруга теж 17-го выїхала. можливо, ти знаєш її, вона головний лікар
Медіфасту. була ( Аліна, Аліна Чера

\begin{itemize} % {
\iusr{Elena Chevela}
\textbf{Світлана Єлісєєва} рада за всіх що живі. Аліна багато допомагала моєї доньки, коли вона була вагітна.
\end{itemize} % }

\iusr{Татьяна Овчаренко}

Бережи вас Господь!!!🙏🙏🙏

\iusr{Mariya Shkodenko}

Безкінечне дякую, що ви живі!!!! І шо описали все це.

Я виросла в Маріуполі. Нашого дому нема. Людей нема.

Щаслива за кожну людину, що змогла вибратись з того пекла.

\begin{itemize} % {
\iusr{Сергей Карпенко}
\textbf{Маша Квитковская} нашого дому теж нема...

\iusr{Mariya Shkodenko}
\textbf{Сергей Карпенко} значить буде новий, ще краще.
\end{itemize} % }

\iusr{Галина Куззьменко Кузьменко}

Слава Богу вы живы.

\ifcmt
  igc https://scontent-fra3-1.xx.fbcdn.net/v/t39.1997-6/319997589_525052179684758_8860913523787471330_n.webp?stp=dst-webp_s180x540&_nc_cat=105&ccb=1-7&_nc_sid=ac3552&_nc_ohc=Qp8xsJ3AwVEAX_d4e-b&_nc_ht=scontent-fra3-1.xx&oh=00_AfBgiatqXjNhkCft-HdvOyqjIoJZec9Fa9lzbr-xRRtFUg&oe=64316B94
	@width 0.1
\fi

\iusr{Natalia Machardy}

Как хорошо что вы описываете это. Если кто-то будет описывать события - то
останется в памяти для других, а оно надо чтоб осталось

\iusr{Любовь Кушнерчук}

Серёжа, Слава Господу, вы выехали!!!

\iusr{Olena Ulanova}

Шедеврально, доктор!🤩🤗 Ви як вельмишановні шляхтичі перед обличчам смерті.
Сигари та бокалу з віски тіки не вистачає.

\iusr{Евгений Салко}

Привіт, Серьога! Розумію тебе, ми з Бучанського району Київської області на 9
день виїхали (жили з дітьми в підвалі), коли взірвали склади біля будинку, але,
дякуючи ЗСУ, до нас орки не дійшли... Бережіть себе! Напиши, як буде час, книгу.
Нехай твої спогади лишаться для дітей... Моя родичка так написала про звірства
совєєєтов на заході країни в 40-х! Нічого не змінилося з того часу, орки тоді і
зараз однакові, гніди!

\iusr{Марина Павлова}

Мы тоже выехали в этот день после обстрела нашего дома ...Ехали наугад, клятые
блокпосты орков, хотелось всё им высказать, но цель была выехать и бороться
дальше.

Грёбаный стыд.

\iusr{Павло Войновський}

Господи, який жах...

Ненавиджу тих нелюдей

\iusr{Elena Chevela}

Ми виїхали 16 березня. Двоє діб добиралися до Запоріжжя. Четверо дорослих та 10
місячна онука. Всі живі 🙏🙏🙏

\iusr{Елена Заркуа}

Пусть это все вернется путинистам стократно ! Око за око! По-другому никак!

\iusr{Marina Mikhno}

Ми теж виїхали, читаю і пригадую цей жах!

\iusr{Ірина Могиленко}

Слава всесвіту ви вибрались.... нема слів описати всю ненависть до рсні
