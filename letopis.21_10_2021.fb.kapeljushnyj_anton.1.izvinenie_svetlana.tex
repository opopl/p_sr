% vim: keymap=russian-jcukenwin
%%beginhead 
 
%%file 21_10_2021.fb.kapeljushnyj_anton.1.izvinenie_svetlana
%%parent 21_10_2021
 
%%url https://www.facebook.com/anton.kapelushny/posts/10220450373435039
 
%%author_id kapeljushnyj_anton
%%date 
 
%%tags jazyk,mova,programmirovanie,programmist,rabota,sobesedovanie,travlja,ukraina,ukrainizacia
%%title Вибачення від Світлани
 
%%endhead 
 
\subsection{Вибачення від Світлани}
\label{sec:21_10_2021.fb.kapeljushnyj_anton.1.izvinenie_svetlana}
 
\Purl{https://www.facebook.com/anton.kapelushny/posts/10220450373435039}
\ifcmt
 author_begin
   author_id kapeljushnyj_anton
 author_end
\fi


Після навали публікацій, відгуків на DOU і постійних звернень журналістів (до
речі, я не програміст, якщо що), я отримав у телеграм вибачення від Світлани,
як назвалась представницею компанії Affreact. Мені розповіли, як там "обурені
вчинком стажера-рекрутера", і що Єлизавету, з якою я і спілкувався, вже
звільнили. Формально, вибачення є. А по факту - знайшли поганого стажера, а
компнаія "ні сном, ні духом"?

\ifcmt
  pic https://scontent-frx5-1.xx.fbcdn.net/v/t1.6435-9/247392656_10220450347114381_2946341484109528016_n.jpg?_nc_cat=110&ccb=1-5&_nc_sid=730e14&_nc_ohc=cRk3UxO4O-wAX-4KBBn&_nc_ht=scontent-frx5-1.xx&oh=08d444a951488be9df6310c5d22ac758&oe=619B2C6A
  @width 0.8
\fi

Upd: та і загалом виникає питання: "а бил лі мальчік?" (с).

\begin{cmtfront}
\uzr{Anton Kapelushny}

"А был ли мальчик? — устойчивое выражение русского языка, означающее сомнение
говорящего в самом факте существования предмета обсуждения. Восходит к цитате
из романа Максима Горького «Жизнь Клима Самгина»."

\end{cmtfront}

\ii{21_10_2021.fb.kapeljushnyj_anton.1.izvinenie_svetlana.cmt}
