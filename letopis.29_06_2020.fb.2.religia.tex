% vim: keymap=russian-jcukenwin
%%beginhead 
 
%%file 29_06_2020.fb.2.religia
%%parent 29_06_2020
 
%%url https://www.facebook.com/permalink.php?story_fbid=2664697827118002&id=100007334310137
 
%%author 
%%author_id 
%%author_url 
 
%%tags 
%%title 
 
%%endhead 

\subsection{Християнство - це релігія для рабів (неєвреїв). Ось і вся біблія.}
\label{sec:29_06_2020.fb.2.religia}
\url{https://www.facebook.com/permalink.php?story_fbid=2664697827118002&id=100007334310137}

Християнство - це релігія для рабів (неєвреїв). Ось і вся біблія.

Головна суть християнства - виплекати побільше мерзенних рабів. Іудаїзм і
християнство формують два протилежних психотипу. Іудаїзм формує мислення
рабовласника. Християнство формує мислення раба. Ось і всі єврейські  ігри в
християнство. Саме з цієї причини в сучасних християнських країнах біля керма
управління стоять одні євреї. Куди не плюнь - скрізь і всюди єврейська лавочка.

Християнство вчить людину протилежного: «Ти - раб божий. Ти людина маленька,
грішний. Тобі треба покаятися, змирися, гордість - це гріх,  повнокровне життя
- це гріх. Думка про гріх - теж гріх. Ти не піклуйся про день завтрашній, так
говорив Ісус. Терпи, бо бог терпить і нам велів. Далі в морду - підстав іншу
щоку. Віднімають верхній одяг - віддай і сорочку. Гвалтують дружину -
запропонуй ще й дочкУ. Роздай свій маєток убогим, бери свій хрест і вперед. І
зійде на тебе благодать Його. А будеш чинити свавілля - загримиш в пекло «.
Християнство-злочин проти людства довжиною в 2000 років.  В ідеалі
євреї-рабовласники хотіли б, щоб раби-християни вели себе як їх вчить Христос:
«... любіть ворогів ваших, благословляйте тих, хто проклинає вас, добро робіть,
хто ненавидить вас, і моліться за тих, хто вас має» (Мф., 5: 43-44). Тобто,
ідеальний християнин - це убога і зацькована  істота, якій дають в морду, а він
підставляє іншу щоку. Йому плюєш в його пику, рушиш чоботом по його безмозкої
голові, а він ще просить. Треба, щоб чобіт був важкий, це дуже важливо. І у
відповідь він встає на коліна і добро кривдника, і ще молиться за нього. Ось це
справжній раб. Цей раб набагато вище якістю, ніж той раб, за яким треба
постійно наглядати і фізично його контролювати. Римські раби раз у раз
повставали, а треба б зробити так, щоб люди й не мріяли про свободу, забули про
неї. Щоб думали про «царстві небесному» і про свою душу. Про те, як би її
сильніше спотворити. Щоб взагалі забули такі поняття як ГОРДІСТЬ, ЧЕСТЬ,
СПРАВЕДЛИВІСТЬ, БЛАГОРОДСТВО, ВІДВАГА, МУЖНІСТЬ, СИЛА, ІНТЕЛЕКТ. Нічого цього
Ви в біблії не знайдете. Там повна відсутність всього волеутверждающего і
життєствердного.
(С)
