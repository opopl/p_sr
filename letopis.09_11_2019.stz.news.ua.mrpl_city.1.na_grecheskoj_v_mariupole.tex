% vim: keymap=russian-jcukenwin
%%beginhead 
 
%%file 09_11_2019.stz.news.ua.mrpl_city.1.na_grecheskoj_v_mariupole
%%parent 09_11_2019
 
%%url https://mrpl.city/blogs/view/na-grecheskoj-v-mariupole
 
%%author_id burov_sergij.mariupol,news.ua.mrpl_city
%%date 
 
%%tags 
%%title На Греческой в Мариуполе
 
%%endhead 
 
\subsection{На Греческой в Мариуполе}
\label{sec:09_11_2019.stz.news.ua.mrpl_city.1.na_grecheskoj_v_mariupole}
 
\Purl{https://mrpl.city/blogs/view/na-grecheskoj-v-mariupole}
\ifcmt
 author_begin
   author_id burov_sergij.mariupol,news.ua.mrpl_city
 author_end
\fi

Не раз и, наверное, не два, писалось и рассказывалось о бывшей
Марии-Магдалининской улице, то есть о четырех кварталах современной Греческой:
от улицы Семенишина до проспекта Мира. Но что-то снова необъяснимо влечет сюда.
Может потому, что со временем узнается нечто, что было ранее неизвестно...

\ii{09_11_2019.stz.news.ua.mrpl_city.1.na_grecheskoj_v_mariupole.pic.1}

На этот раз воображаемую прогулку начнем с дома, когда-то принадлежавшего \textbf{Илье
Петровичу Ножникову}. Здесь он принимал больных, отсюда совершал визиты в дома
мариупольцев, где были люди, нуждающиеся в медицинской помощи. В краеведческой
литературе встречается утверждение, будто доктор Илья Петрович Ножников лечил в
Крыму нашего великого земляка Архипа Ивановича Куинджи. Да, Архипа Ивановича
лечил доктор Ножников. Только не Илья Петрович, а Борис Петрович – врач, живший
в крымской Ялте, где он врачевал хворавших жителей и гостей курортного города.
В 1911 году Борис Петрович отошел в мир иной в возрасте шестидесяти лет...

\textbf{Читайте также:} \href{https://mrpl.city/blogs/view/vokrug-skvera-istoriya-tsentra-mariupolya}{%
Вокруг сквера: история центра Мариуполя, Сергей Буров, mrpl.city, 17.08.2019}

На углу Греческой и Георгиевской улиц, по адресу ул. Георгиевская, 41, стоит
дом со скошенным углом. Его давно покинули жильцы, стен его давно не касалась
кисть маляра, проемы окон и дверей наглухо заложены кирпичом. В этом доме 17
февраля 1915 года родилась \textbf{Ирина Александровна Белявская.} Ее отец - коллежский
асессор Александр Ивановича Белявский, мать - Мария Федоровна. Ирина
Александровна была видным историком-американистом, доктором исторических наук.
Она - автор семи монографий, посвященных истории, политике и экономике США, а
также множества статей, опубликованных в различных научных изданиях. В своих
исследованиях она использовала не только публикации и документы, имевшиеся в
Советском Союзе. Ей довелось работать с историческими документами и редкими
книгами в библиотеке Конгресса США. Ирина Александровна умерла в 2011 году на
96 году жизни, оставаясь до конца своих дней в здравом уме и ясной памяти.

\ii{09_11_2019.stz.news.ua.mrpl_city.1.na_grecheskoj_v_mariupole.pic.2}

Улица Греческая, дом 19, принадлежавший \textbf{Николаю Спиридоновичу Кертме}.
Но хозяин особняка в нем не жил, а сдавал в аренду консулу Великобритании в
Мариуполе Вальтону до тех пор, пока тот не построил себе особняк на
Георгиевской улице неподалеку от мужской гимназии. По данным, почерпнутым из
Адрес-календаря на 1910 год, нашему земляку Кертме принадлежали также дом на
Большой Садовой, три дома на Малой Садовой, три - на Марии-Магдалининской
улице. А еще по одному дому на Екатерининской и Таганрогской улицах и торговое
помещение на Базарной площади. Вся эта недвижимость сдавалась в аренду. Сам же
гласный Мариупольской городской думы господин Кертме жил в доме, современный
адрес которого - улица Греческая, 12. Помнится, в 60-70-е годы пошлого столетия
особняк \enquote{красовался} сахарной белизной своих стен, маскаронами в виде
ангелочков над окнами. В наше время он имеет довольно затрапезный вид.
Обшарпанные стены, кроме одной, обращенной на улицу Пушкина, да части фасада,
шелушащиеся кирпичи на арке ворот.

\textbf{Читайте также:} 

\href{https://mrpl.city/news/view/uslugi-s-ulybkoj-i-kofe-na-kryshe-krasoty-i-syurprizy-mariupolskogo-multitsentra-foto-360-plusvideo}{%
Услуги с улыбкой и кофе на крыше: красоты и сюрпризы мариупольского \enquote{Мультицентра}, Анастасія Селітріннікова, mrpl.city, 08.11.2019}

Дом по адресу Греческая, 10. Владельцем и двора, и дома до революции 1917 года
был торговец рыбой \textbf{Василий Иванович Челпанов}, старший брат известного ученого
психолога и логика, профессора Киевского и Московского университетов Георгия
Ивановича Челпанова. Здесь нужно попросить прощения у читателей. Автор этих
строк в некоторых своих публикациях, не проверив, писал, что здесь жил Г. И.
Челпанов, что не соответствует исторической правде. В послевоенные годы в этом
доме жила \textbf{Мария Федоровна Волошина}, выпускница Мариупольской Мариинской женской
гимназии и высших женских курсов Санкт-Петербурга, питомицы которых имели право
преподавания в старших классах гимназий. Мария Федоровна была учительницей
русского языка и литературы в частной гимназии Неонилы Саввичны Дарий. После
революции 1917 года гимназию национализировали, она стала называться школой №2.
Вместе с Марией Федоровной в доме жили ее дочь \textbf{Рена Ивановна Волошина},
инженер-электрик, высококвалифицированный проектировщик, а также внучка \textbf{Ольга
Викторовна Сергеева}, также инженер-электрик. Увы, все они в разное время
завершили свой земной путь. Новый хозяин сделал ремонт фасада дома. К парадному
входу пристроил тамбур, закрывший парадную дверь. Конечно, это искажает вид
старинного здания. Но что поделаешь?

Особняк по адресу Греческая, 13. Он примечательный тем, что в первые
послевоенные годы в нем размещалась городская библиотека имени В. Г. Короленко.
Чтобы стать ее читателем в те времена, нужно было сдать две-три книги. И
мариупольцы несли сюда чудом сохранившиеся в полусожженном Мариуполе томики
Пушкина и Тургенева, Шевченко и Гоголя, учебники и словари, а то и брошюрки,
издававшиеся для солдат и матросов в военную пору. Вот так силами народными
комплектовался книжный фонд. Там был и читальный зал. Посредине комнаты стоял
продолговатый стол, рядом диван, видавший виды, да еще несколько стульев. Когда
освободилось помещение на первом этаже пятиэтажки у сквера, туда перебазировали
Городскую библиотеку им. В. Г. Короленко. А в доме №13 открыли небольшой
книжный магазин.

\textbf{Читайте также:} 

\href{https://mrpl.city/news/view/krasnoe-serdtse-i-ogromnye-bukvy-v-mariupole-poyavilas-novaya-lokatsiya-dlya-selfi-fotofakt}{%
Красное сердце и огромные буквы: в Мариуполе появилась новая локация для селфи, mrpl.city, 07.11.2019}

\ii{09_11_2019.stz.news.ua.mrpl_city.1.na_grecheskoj_v_mariupole.pic.3}

Рядом с библиотекой в здании с затейливым фасадом (ул. Греческая, 11) был тогда
же развернут приемник-распределитель для детей, которых военное лихолетье
разлучило с родными и близкими: кого - на время, а кого - навсегда. Это были
беспризорники, их подбирали на вокзалах, кое-кто приходил в Мариуполь из
близлежащих сел и поселков. В приемнике их избавляли от насекомых, купали,
кормили, если нужно, лечили, а далее распределяли по детским домам. Между
прочим, в этом же здании подобное учреждение было организовано еще во время
Гражданской войны.

Старые дома – привет из прошлого. А некоторые из старых домов, как бы и без
истории.

История других чуть-чуть приоткрылась. Спасибо \textbf{Сергею Сергеевичу Данилову},
доценту, кандидату технических наук, патриоту Мариуполя. Он многое рассказал об
улице, которой трижды меняли название.
