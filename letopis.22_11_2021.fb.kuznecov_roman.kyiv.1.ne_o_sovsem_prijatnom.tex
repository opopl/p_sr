% vim: keymap=russian-jcukenwin
%%beginhead 
 
%%file 22_11_2021.fb.kuznecov_roman.kyiv.1.ne_o_sovsem_prijatnom
%%parent 22_11_2021
 
%%url https://www.facebook.com/roman.kuznetsov.ua/posts/4713105258748693
 
%%author_id kuznecov_roman.kyiv
%%date 
 
%%tags deti,kultura,muzyka,rossia,rusmir,ukraina
%%title Теперь не о совсем приятном
 
%%endhead 
 
\subsection{Теперь не о совсем приятном}
\label{sec:22_11_2021.fb.kuznecov_roman.kyiv.1.ne_o_sovsem_prijatnom}
 
\Purl{https://www.facebook.com/roman.kuznetsov.ua/posts/4713105258748693}
\ifcmt
 author_begin
   author_id kuznecov_roman.kyiv
 author_end
\fi

Теперь не о совсем приятном.

Листая новостную ленту Инстаграма довольно часто натыкаюсь на фото детворы
младшего и среднего школьного возраста в стиле гопников из лихих 90-х: а-ля
\enquote{Папин бродяга, мамин симпатяга}, \enquote{Брат за брата - такое за основу взято!} или
фотки в стиле \enquote{АУЕ} (\enquote{арестантский уклад един} - воровская молодежная
субкультура из россии).

\ii{22_11_2021.fb.kuznecov_roman.kyiv.1.ne_o_sovsem_prijatnom.pic.1}

Причем, чтоб вы понимали, фото не из оккупированного Донецка, Луганска или
восточно-украинского Харькова (там-то само собой понятно откуда это всё), а из
западных областей Украины! А это значит, что российская пропаганда в виде
воровского культа жизни и воровских порядков уже нацелена на наше подрастающее
поколение и делает свое черное дело. Они уже пытаются подражать этим
\enquote{супергероям} \enquote{Улиц разбитых фонарей}, \enquote{Бригады} или
\enquote{Брату-2}. А взрослые, даже часто из украино-патриотического сегмента,
часто забывают о многогранности российской пропаганды, которая как спрут: он
может и не ужалить своими 7-ю щупальцами, а вот восьмым - запросто. И часто,
слушая условное радио \enquote{Шансон}, Мишу Круга и просматривая российские
боевики, думайте про то, что хотя вы и презираете кремль, путина и русский мир,
но это ничего не значит, если Вы позволяете этому яду пусть даже и капельно
отравлять ваше украинское сознание теми порядками и культурой \enquote{русского
мира}, которые входят в голову в виде этого пророссийского контента. 

Если для взрослых - это еще полбеды, то дети - это таргет-аудитория идеологии
русского шовинизма для еще неокрепших умов маленьких украинцев; и пока их
мамы-папы заняты своими делами, дети как губка впитывают модели поведения
субкультуры  воровского мира, тексты песен русского рэпа типа тимати, который
очень любит учить как правильно любить \enquote{президента путина}, российскую попсу
или низкопробные рашистские фильмы в которых так, невзначай, сладко поют про
\enquote{русский Севастополь} и играют на берегу \enquote{Прощание славянки}; при этом детвора
со временем безоговорочно начнёт верить в то, что это \enquote{круто}, а \enquote{воровские
звезды и жизнь по понятиям есть удел настоящих мужиков, прошедших тюремную
школу жизни}. И это - только начало!

PS. на этом фото из Инсты мальчишке всего 11 лет. 

Делайте выводы, господа взрослые, что смотрят и слушают Ваши дети и внуки.
