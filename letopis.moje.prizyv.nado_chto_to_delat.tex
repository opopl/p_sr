% vim: keymap=russian-jcukenwin
%%beginhead 
 
%%file moje.prizyv.nado_chto_to_delat
%%parent moje.prizyv
 
%%url 
 
%%author_id 
%%date 
 
%%tags 
%%title 
 
%%endhead 

\subsubsection{Надо что-то делать}

Однако мы знаем, сколько сейчас несчастий происходит по всей территории
Украины. Ну что ж, надо что-то делать. Надо что-то делать, поскольку похоже все
движется к тому, что мы - украинцы и россияне, на самом деле будучи одной
крови, поскольку мы все славяне, просто перестреляем, перережем друг друга в
кровавом месиве, поскольку с обоих сторон идет мощная и непрерывная поддержка
ресурсами, оружием, людской поддержкой, информационной пропагандой. И конца и
края этому беспределу не видно... Вот, встретились делегации... пообщались,
попили водички, пофоткались и потом разъехались для консультаций... Дипломатия
- дело медленное, неторопливое. Ну а снаружи, в реальной жизни полный звиздец
какой-то творится. Москва пошла на Киев... Дочь пошла на Мать, на Мать Городам
Русским... Коса нашла на камень... Вова Вовыч пошел, яростно размахивая шашкой,
на Вову Саныча... Русский Медведь рассвирепел и пошел крушить все направо и
налево... И все дальше и дальше, все кровавей и кровавей... Русь пошла на Русь,
едрена мать. Россия пошла на Украину, несмотря на такую интересную
умиротворяющую статью Владимира Путина о едином народе и общих исторических
корнях. Ну что ж... 

\ifcmt
  tab_begin cols=2,no_fig,center

     %pic https://avatars.mds.yandex.net/i?id=3298ffaedd5083ad87bf3445e5acf60c-5579767-images-thumbs&n=13
		 pic https://i.mycdn.me/image?id=866088588409&t=50&plc=WEB&tkn=*dcR0YKEX7jvvnjwI2JCaRYQRpcw&fn=external_8
		 @caption Богдан Хмельницкий (играет Богдан Ступка), фильм \enquote{Огнем и Мечом} Ежи Гофмана, 1999 год

		 pic https://images.kinorium.com/movie/shot/449332/w1500_38244168.jpg
		 @caption Богдан Ступка в роли Тараса Бульбы и Владимир Вдовиченков в роли Остапа, фильм \enquote{Тарас Бульба} Владимира Бортко, 2008 год

  tab_end
\fi

(1) С нашей стороны идут сообщения о том, что мы мочим врага, что мы
обязательно победим, что наше дело правое, что оккупант будет изгнан, и что вы
все сгинете в аду. И вы можете сколько угодно над этим смеяться, и говорить,
что это все украинская пропаганда, но так оно на самом деле и есть, уж
поверьте.  Мы-то здесь живем, и мы все видим, что происходит, на самом обычном
человеческом уровне, на уровне похода в ближайший магазин, или же в гости к
соседу этажом ниже. И абсолютно простые люди, совсем недавно взявшие в руки
оружие, бьются до конца, и готовы отправлять вас в ад пачками, россияне.
Обычные троещинские пацаны обсуждают у себя на кухнях, как лучше сделать
коктейли Молотова, вместо того, чтобы сидеть в полутемном углу с любимой
девушкой. И может быть вы этого и не ожидали, может быть вы этого и не знаете,
но такова правда.  Спящий, могучий дух Богдана Хмельницкого, Тараса Бульбы и
всего Войска Запорожского, а также мужественных защитников Киева 1240 года,
бившихся до конца, и сложивших свои головы за любимый город, просыпается, да,
просыпается все сильнее и сильнее, - даже если у вас об этом и не пишут на
ваших сайтах, россияне.

(2) С вашей стороны пишут немного другое, понятное дело, не будете же вы
признавать, в самом деле, что именно ваша ракета попала в жилой дом на
Лобановского 6, это как-то не очень непатриотично для российского
телевидения... ну и... какова истинная картина, кто наступает, кто отступает -
очень трудно понять, на самом деле. У нас о войне пишут на русском и украинском
языках, у вас - на русском. И вот, знаете, кроме всем уже известного Русский
Корабль, Иди Нах@й, и относительного нового Русский Солдат, иди туда же, -
постеры в Протасовом Яру видели? - это там, где рядом крематорий, который также
называют Ракушкой, и Байковое кладбище, - прибавится же там работы! - появилось
также такое - Слава Нации!  П@зда Российской Федерации! Все на чистом русском
языке, который вы так ръяно бросились защищать, россияне. Да, россияне, как
говорил Гоголь, перед вами громада - русский язык! И знаете, мы любим, что бы
было просто, четко и лаконично.

