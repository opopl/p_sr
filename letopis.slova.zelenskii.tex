% vim: keymap=russian-jcukenwin
%%beginhead 
 
%%file slova.zelenskii
%%parent slova
 
%%url 
 
%%author 
%%author_id 
%%author_url 
 
%%tags 
%%title 
 
%%endhead 
\chapter{Зеленский}

Костянтине, а що Ви хочете від неосвіченого самодура, який не запнувшись ТРИЧІ
прочитав з папірця: \enquote{КонцерТ Оборонпром} (виступ при представлені Голови
Харківської ОДА у 2019)?
Він уяви не має що таке наукове середовище. Не вважаю \emph{О. С. Зеленського} світочем
науки, але навіть він міг йому міг пояснити ідиотизм задумки,
коментар, Lon Lon, \textbf{Щодо вчорашнього указу про \enquote{Президентський університет}}, 
Костянтин Матвієнко, pravda.com.ua, 01.06.2021

Зеленський і його оточення теоретично не спроможні на ретельну наполегливу
щоденну ефективну, а головне, СИСТЕМНУ роботу. Їм в усьому потрібен хайп – по
щучому \enquote{велению}, по їхньому \enquote{хотенію}, не злазячи з печі, в'їхати у щасливе
майбутнє. Мрійники з горщика, з правом на помилку))))
У даному випадку ці зелепухи у страшному сні не знають, яка величезна і
непроста робота необхідна для утворення пристойного вишу. Схоже на те, що
Зеленський повірив у те, що він \enquote{найвеличніший хтось-там сучасності}. Підписав
височайший указ, зайняв грошенят в Європах – і все, з печі можна не злазити до
наступного хайпу)))
коментар, \textbf{Щодо вчорашнього указу про \enquote{Президентський університет}}, 
Костянтин Матвієнко, pravda.com.ua, 01.06.2021

Президент \emph{В. Зеленський} отримав унікальну в українській історії можливість
провести швидкі та ефективні реформи. Натомість його політична сила і він сам
грузнуть у внутрішніх конфліктах, примітивному меркантилізмі та
некомпетентності команди, що змушує їх до намагань відтворювати модель взаємин
влади і суспільства за прикладом сумнозвісних попередників,
\textbf{Знову \enquote{Україна не Росія}, проте багатьом дуже хочеться стерти цю відмінність},
Костянтин Матвієнко, pravda.com.ua, 02.02.2021

В этом вопросе проявилась еще и некомпетентность \emph{Зеленского} и его советников,
которые при подготовке к интервью не подсказали ему, что экспортные поставки
немецкого оружия – это крайне чувствительная и тонкая тема. После второй
мировой войны Германия систематически отказывается посылать свои войска за
пределы страны даже на просьбы США. С экспортными поставками оружия – все еще
более точечно и чувствительно. В ежегодном отчете правительства Германии по
экспорту оружия есть специальный раздел под названием «Отказано в сделках». Там
можно найти очень четкий перечень оснований, которые являются причиной решения
об отказе в продаже оружия. Например, «Проблемы с соблюдением прав человека»,
«Внутреннее напряжение или вооруженные конфликты в пределах страны»,
«Возможность использования вооружения внутри страны или продажи его за ее
пределы по неприемлемым условиям». Попадаем мы под эти перечисленные критерии?
Так почему Офис и советники \emph{Зеленского} не удосужились изучить вопрос и
подготовить гаранта, чтобы через пару часов после опубликования интервью он не
получил публично по носу от МИД Германии?,
\textbf{Это странно, но Зеленский совершенно не умеет давать интервью}, Светлана Крюкова, 
strana.ua, 03.06.2021

Тогда почему Зеленский для себя решил, что Германия должна мощно заниматься
переоснащением нашей армии, а не своей?  Общий вывод – к интервью и
отечественным, и особенно крупным глобальным по охвату аудитории мировым СМИ
нужно очень серьезно готовится. Для них ответы в стиле КВНовских и квартальных
экспромтов категорически недопустимы. Если у нас - прокатит, то там - нет,
\textbf{Это странно, но Зеленский совершенно не умеет давать интервью}, Светлана Крюкова, 
strana.ua, 03.06.2021

Это странно, но \emph{Зеленский} совершенно не умеет давать интервью.  Он и
высказаться толком не умеет, и что говорить, не понимает. Прочла интервью
\emph{Зеленского} немецкому изданию. Это за него Мендель отвечала, как я поняла?
Изучила текст по форме и по сути. Пара наблюдений,
Светлана Крюкова, strana.ua, 03.06.2021

Мягкость к олигархам, \emph{Зеленский} как будильник, налог на Google..., Денис
Безлюдько, strana.ua, 03.06.2021

\emph{Зеленский} обнажил еще одну свою ипостась – работать будильником для высшего
звена госслужащих. По его словам, он каждый день просыпается рано в 5:30 или
6:00 и знакомится с новой информацией, после чего звонит министрам. Иногда
будит их даже раньше, чем сигналы телефонов: «Иногда я даю свои сигналы ранее
их iPhone». \emph{Зеленский} сказал, что они уже привыкли. Вообще отличная услуга. Я в
гостинницах обычно такие заказываю. Я бы на месте Портрета деньги бы брал за
побудку. Большие,
\textbf{Мягкость к олигархам, \emph{Зеленский} как будильник, налог на Google...}, Денис
Безлюдько, strana.ua, 03.06.2021

Я часто потом троллил госслужащих запросами о выписках из журнала ))) Ржал как
конь! Жаль отменили этот ИДИОТИЗМъ... А было весело. Но видимо \emph{Зелик}
вернет мне мое счастье ))) НацДержСлужба даже разъяснение официальное тогда
распространила!,
\textbf{В законе об олигархах всплыл старый идиотизм},
Михаил Чаплыга, strana.ua, 04.06.2021

Это крайне интересный поворот. Получается, что у \emph{Зеленского} по свистку
из США готовы лишать гражданства украинцев, которые не угодили американцам. А
также - наложить на них дополнительные санкции.  Судя по всему, в первую
очередь эта практика может ударить по тем бывшим регионалам, кто после Майдана
уехал в политическую эмиграцию. И сейчас критикует украинскую власть, 
\textbf{Врагам Америки пересмотрят гражданство Украины. Главные решения нового СНБО}, 
Максим Минин; Игорь Рец, strana.ua, 04.06.2021

\enquote{Нужен Песков на минималках}. Кого и как искал Офис президента на должность пресс-секретаря Зеленского,
Анастасия Товт, strana.ua, 04.06.2021

Вместо реальной борьбы с олигархатом обществу предлагается очередной перфоманс,
который \emph{Зеленский} устраивает для того, чтобы отвлечь внимание от того
позорного факта, что ни одного предвыборного обещания он не выполнил — «эпоха
бедности» не преодолена, никакого роста экономики и занятости не произошло от
слова совсем, тарифы на коммуналку рвутся вверх, а к зиме в тарифной вакханалии
может утонуть подавляющее большинство населения. И это уже не говоря о так
называемом \enquote{рынке земли} с 1 июля, который может свестись к
экспроприации земельных наделов мелких собственников под вывеской так
называемой консолидации. При этом, насколько можно понять, \emph{Зеленский}
хочет пойти на второй срок, а еще больше его второго срока хочет его окружение,
причем каденция уже подходит к экватору. В этих условиях \emph{Зе} и Ко в
привычной для себя манере пытаются раскрутить красочное шоу и переключить
внимание на «борьбу с олигархами», 
\textbf{Народу предложили пиар-клоунаду потешной Зе-деолигархизации},
Александр Карпец, strana.ua, 07.05.2021

