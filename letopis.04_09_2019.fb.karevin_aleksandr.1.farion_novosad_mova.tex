% vim: keymap=russian-jcukenwin
%%beginhead 
 
%%file 04_09_2019.fb.karevin_aleksandr.1.farion_novosad_mova
%%parent 04_09_2019
 
%%url https://www.facebook.com/permalink.php?story_fbid=2404753856430587&id=100006879888184
 
%%author_id karevin_aleksandr
%%date 
 
%%tags farion_irina,mova,novosad_anna.deputat.ukraina,ukraina,ukrainizacia
%%title Фарион - Новосад - разбор полетов
 
%%endhead 
 
\subsection{Фарион - Новосад - разбор полетов}
\label{sec:04_09_2019.fb.karevin_aleksandr.1.farion_novosad_mova}
 
\Purl{https://www.facebook.com/permalink.php?story_fbid=2404753856430587&id=100006879888184}
\ifcmt
 author_begin
   author_id karevin_aleksandr
 author_end
\fi

Этот пост предназначен для тех, кто день назад радостно поддержал (репостами,
комментами, "лайками") наезд Ирины Фарион на Анну Новосад. У меня для вас
хорошая новость! Новосад сдулась, признала свои "ошибки" и пообещала, что
впредь будет более внимательно готовить свои тексты. То есть - будет приближать
их к тому варианту украинского языка, который Ирина Фарион считает правильным.

\ifcmt
  tab_begin cols=2

     pic https://scontent-frx5-2.xx.fbcdn.net/v/t1.6435-9/69911935_2404718343100805_1695675671398842368_n.jpg?_nc_cat=109&ccb=1-5&_nc_sid=730e14&_nc_ohc=WCnEsEMGPdUAX__4p8l&_nc_ht=scontent-frx5-2.xx&oh=2a2dcc5946581f6ad52c7f65c8523728&oe=61593B4F

     pic https://scontent-frt3-1.xx.fbcdn.net/v/t1.6435-9/69871050_2404719836433989_3766559123135528960_n.jpg?_nc_cat=104&ccb=1-5&_nc_sid=730e14&_nc_ohc=6twfoECcTIAAX8NIVed&tn=lCYVFeHcTIAFcAzi&_nc_ht=scontent-frt3-1.xx&oh=459844a68ca34e74a74dca034bebd7a1&oe=6158DD59

  tab_end
\fi

Вы ведь этого хотели, правда? Вы ведь ради этого репостили и лайкали
Фарионовские правки? Ну так вы своего добились! Поздравляю!

А теперь небольшой "разбор полётов". Перепащивать тут скрин Фарионовских правок
не буду. Интернет кишит этими скринами, желающие легко найдут. Поэтому сразу
перейду к конкретике.


\ifcmt
  tab_begin cols=2

     pic https://scontent-frt3-1.xx.fbcdn.net/v/t1.6435-9/70252222_2404720319767274_2070673844447739904_n.jpg?_nc_cat=108&ccb=1-5&_nc_sid=730e14&_nc_ohc=Xq3XZ0d4LPQAX_lbJIn&tn=lCYVFeHcTIAFcAzi&_nc_ht=scontent-frt3-1.xx&oh=99370497fe5724efafd384fd17d83345&oe=61577C58

     pic https://scontent-frx5-1.xx.fbcdn.net/v/t1.6435-9/70263007_2404721459767160_8928071358554832896_n.jpg?_nc_cat=111&ccb=1-5&_nc_sid=730e14&_nc_ohc=9SqZY---MLoAX_n5EdP&_nc_ht=scontent-frx5-1.xx&oh=b5c5dc5b196fa725e3d272cfbcb0174b&oe=6158BE71

  tab_end
\fi


1. Фарион упрекнула Новосад за употребление формулировки "вдячна президенту".
По мнению Фарион, следует писать: "вдячна президентові". Между тем, в
украинском языке употребляются обе формулировки (см. скрин в этой публикации).
Почему Фарион не нравится первая форма и нравится вторая - понятно. Первая
формулировка аналогична принятой в русском языке. По-русски тоже сказали бы:
"благодарна президенту". А вот вторая формулировка на русскую не похожа, чем и
мила сердцу одиозной филологини. Вы эту формулировку считаете правильной по той
же причине?

2. Фарион считает ошибочным употреблённое Новосад "по діям та результатам".
Надо, дескать - "по діях та результатах". Между тем ещё "национально
сознательный" языковед Иван Огиенко признал, что у нас издавна употреблялась
именно форма слова, применённая Новосад (см. фото). Форма же, навязываемая
Фарион, появилась гораздо позднее, во времена польского ига. А в украинском
языке эта форма утвердилась уже в наше время, когда филология перестала быть
наукой и стала политикой. Разумеется, первоначальную форму Огиенко объявил
устаревшей, а Фарион - неправильной. Почему - понятно. Опять же, потому, что
первая форма аналогична русской. Ну и вы, вероятно, поддержали в данном вопросе
Фарион, руководствуясь тем же мотивом. Да?

3. Фарион не понравилось слово "продовжить". Причина та же: слишком уж это
слово похоже на русское "продолжит". Поэтому Фарион предложила заменить его на
"і надалі має". Вы с ней согласились. Ну, понятно почему. Между тем, слово
"продовжить" было употреблено Новосад правильно (см. 2-й скрин).

4. Фарион (и вам всем тоже) не понравилось слово "задачі". Это ж недопустимый
"русизм"! Между тем, слово "задача" (в единственном числе) есть в украинском
языке. Его можно найти в словарях. И не только в тех, что изданы в советское
время (коли нашу мову русифікували кляті комуняки!), а и, например, в "Новому
тлумачному словнику української мови" 2007 года издания (см. 2-е фото).

5. Фарион предъявила претензии к стилю Новосад. Ну, было бы, конечно,
замечательно, если бы все наши министры образования писали пушкинским (ой,
вибачте, шевченківським) стилем. Но, как мне кажется, это не обязательное
условие для занятия министерского поста.

Какие ещё "ошибки" Новосад мы не разобрали? Запятую один раз не поставила, а в
другой раз - точку? Да, это серьёзная причина для истерик в Интернете. Только
ведь тут очевидна невнимательность, а не неграмотность.

Впрочем, всё это уже практического значения не имеет.  Повторюсь: Новосад
сдулась и теперь (с вашей, между прочим, подачи) будет всё время оглядываться
на Фарион. Не думал я, что у этой филологини такое количество поклонников в
"ватной" среде (в том числе - среди моих ФБ-друзей). Ну, это ваш выбор. Только,
когда в следующий раз будете жаловаться на оголтелую украинизацию, вспомните,
что к этой украинизации и вы свои ручки приложили. Своей поддержкой Фарион
приложили. 

P.S. Ещё раз подчеркну - как я и писал в первом посте на эту тему, ни в коей
мере не защищаю пани министерку. Прекрасно помню, что она пришла в министерство
с Майдана, а ничего хорошего оттуда прийти не может. Но данный пост несколько о
другом.

\ii{04_09_2019.fb.karevin_aleksandr.1.farion_novosad_mova.cmt}
