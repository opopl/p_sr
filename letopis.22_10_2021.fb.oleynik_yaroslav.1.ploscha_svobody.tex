%%beginhead 
 
%%file 22_10_2021.fb.oleynik_yaroslav.1.ploscha_svobody
%%parent 22_10_2021
 
%%url https://www.facebook.com/100005261253500/posts/pfbid0BGkiPDMn21aJmu6cNzddmX9JLD2tD62A4R4jgW74k2AHJv4tBtQDMeCc39uVCEXAl
 
%%author_id oleynik_yaroslav
%%date 22_10_2021
 
%%tags 
%%title Площа Свободи
 
%%endhead 

\subsection{Площа Свободи}
\label{sec:22_10_2021.fb.oleynik_yaroslav.1.ploscha_svobody}

\Purl{https://www.facebook.com/100005261253500/posts/pfbid0BGkiPDMn21aJmu6cNzddmX9JLD2tD62A4R4jgW74k2AHJv4tBtQDMeCc39uVCEXAl}
\ifcmt
 author_begin
   author_id oleynik_yaroslav
 author_end
\fi

Хочу почати з того, що поняття  миру завжди має більше ніж, так би мовити, одну
сторону.

У Пабло Пікассо  - це найвідоміший символ миру \enquote{Голуб миру}, а в місті
з яким я ознайомлю далі - це головна площа, яка символізує свободу і мир.

\enquote{Площа Свободи} таку назву отримав комфортний простір, який розташований у
Маріуполі. 

Це місце, наповнене смислами і символами єдності і миру. Місце, яке розповідає
історію Домаха (назва міста в XVI).

А відомо, що проєкт освітлення площі Свободи отримав нагороду на міжнародному
конкурсі світлового дизайну Lighting Design Awards у Лос-Анджелесі?

Ну що полетіли...

•
•
•

\#маріупольтуристичний \#mariupol \#мариуполь 

\#маріуполь \#фото \#город \#0629

\#mariupol\_live \#ukraine \#аерофотосъемка 

\#aerophotography \#aerophoto

\#dronephotography \#videographer \#videography 

\#fly \#dronephotography \#instadrone

\#dronestagram \#dji \#djimavic2 \#droneglobe 

\#djiphotography \#ukrainetravel
