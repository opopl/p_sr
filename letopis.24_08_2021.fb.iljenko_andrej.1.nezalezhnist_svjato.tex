% vim: keymap=russian-jcukenwin
%%beginhead 
 
%%file 24_08_2021.fb.iljenko_andrej.1.nezalezhnist_svjato
%%parent 24_08_2021
 
%%url https://www.facebook.com/andriy.illenko/posts/6795191090521623
 
%%author Ильенко, Андрей
%%author_id iljenko_andrej
%%author_url 
 
%%tags nezalezhnist,prazdnik,ukraina
%%title День Незалежності став дійсно народним святом
 
%%endhead 
 
\subsection{День Незалежності став дійсно народним святом}
\label{sec:24_08_2021.fb.iljenko_andrej.1.nezalezhnist_svjato}
 
\Purl{https://www.facebook.com/andriy.illenko/posts/6795191090521623}
\ifcmt
 author_begin
   author_id iljenko_andrej
 author_end
\fi

День Незалежності став дійсно народним святом.

Це ставалося не одномоментно, на це пішли десятиліття. 

Пам’ятаю, як навіть на Майдані 2013-14 року виникали дискусії щодо гасла «Слава
Україні» — дехто навіть зі сцени закликав його не використовувати, бо «не на
часі».

Ще якихось років десять тому це свято по-справжньому було живим лише для
активної, патріотичної меншості. 

В усьому іншому це був більше офіціоз та формальні чиновницькі промови з
листочка. 

Сьогодні Київ святкував, як ніколи. Безліч радісних людей, які точно знають, що
таке Незалежність і чому її треба цінувати. І не лише Київ, а вся Україна.

Воістину, ніщо не зупинить ідеї, час якої настав.
