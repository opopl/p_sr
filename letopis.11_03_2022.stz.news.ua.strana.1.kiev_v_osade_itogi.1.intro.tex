% vim: keymap=russian-jcukenwin
%%beginhead 
 
%%file 11_03_2022.stz.news.ua.strana.1.kiev_v_osade_itogi.1.intro
%%parent 11_03_2022.stz.news.ua.strana.1.kiev_v_osade_itogi
 
%%url 
 
%%author_id 
%%date 
 
%%tags 
%%title 
 
%%endhead 

Столица сегодня была одной из главных тем фронтовых новостей. О том, что \enquote{Киев
в осаде}, написал советник главы Офиса президента Михаил Подоляк.

\ifcmt
  ig https://strana.news/img/article/3813/kiev-v-osade-38_main.jpeg
  @wrap center
  @width 0.8
\fi

Впрочем, это сильное преувеличение. Столица Украины не окружена российской
армией. Россияне медленно пытаются продвигаться на юг к западу от столицы, с
севера, а также по Левобережью.

Но везде это продвижение сопровождается боями с ВСУ. И пока до больной блокады
Киева еще далеко.

Одной из самых горячих точек в эти дни оказалась
\href{https://strana.news/news/381279-pod-brovarami-11-marta-idut-tjazhelye-boi.html}{окрестности
Броваров} (северо-восток от Киева). Там говорят о больших разрушениях в селах и
о гибели мирных жителей.  В той же местности создается масштабный укрепрайон
российских войск.

Параллельно с этим были опубликованы данные спутникового наблюдения, согласно
которым колонна российской армии, несколько дней назад замеченная в
\href{https://strana.news/news/380250-hostomel-snova-pod-kontrolem-vrazheskoj-armii.html}{районе
Гостомеля}, теперь расквартирована в близлежащих населенных пунктах, а
артиллерия выведена на боевые позиции.

Все это было оценено западными СМИ как подготовка к наступлению, хотя на самом
деле расквартирование войск в населенных пунктах скорее свидетельствует, как
минимум, о его отсрочке – судя по всему, войска противника просто спрятали от
ночных морозов.

Да и уже упомянутый факт интенсивных боев у восточной окраины Киева вкупе с
созданием укрепрайона свидетельствует в пользу плана окружения Киева, а него
его штурма.

Но, повторимся, пока Киев не окружен – южное направление и большая часть
восточного контролируется украинскими силами.
