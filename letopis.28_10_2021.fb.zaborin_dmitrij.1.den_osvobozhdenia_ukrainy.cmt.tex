% vim: keymap=russian-jcukenwin
%%beginhead 
 
%%file 28_10_2021.fb.zaborin_dmitrij.1.den_osvobozhdenia_ukrainy.cmt
%%parent 28_10_2021.fb.zaborin_dmitrij.1.den_osvobozhdenia_ukrainy
 
%%url 
 
%%author_id 
%%date 
 
%%tags 
%%title 
 
%%endhead 
\subsubsection{Коментарі}

\begin{itemize} % {
\iusr{Анатолий Викторович}
Дима, с Праздником! Мы помним!

\iusr{Martin Žour}
С праздником Дима!

\iusr{Anton Denikin}
Я таки недопонял. Так спиввитчызныки вместе с первыми оккупантами прогнали вторых?

\begin{itemize} % {
\iusr{Дмитрий Заборин}
не вникай )

\iusr{Anton Denikin}
\textbf{Дмитрий Заборин} Единственно правильная линия, ибо понять это невозможно.

\iusr{валентин генчев}
\textbf{Anton Denikin} не заморачивайтесь!
\end{itemize} % }

\iusr{Алёна Бестаева}
С праздником!

\iusr{Андрей Блинов}

Кусочек Закарпатья освобождали в ходе следующей операции, в ноябре 1944. Но
тогда формально это не была территория УССР по состоянию на 22.6.41

\iusr{Елена Штоквиш}
С праздником!

\iusr{Владимир Епифанов}

видимо этот великий бюджетный институт логической памяти , отвлекаемый от
научных трудов процессом пересчета зарплат и премий , просто забывает сообщить
о том , что все факельные шествия у нас проводятся в честь РККА .


\iusr{Владимир Епифанов}

знаю что пленные немцы работали над восстановлением разрушенных предприятий , а
как насчет европейськых итальянцив и румунив , особо тогда зверствовавших на
территории Украины ? они репарации платили или отрабатывали ?

\begin{itemize} % {
\iusr{Дмитрий Заборин}
\textbf{Владимир Епифанов} румыны союзники с 44 г., обошлись репарациями. Итальянцев было слишком мало.

\iusr{Владимир Епифанов}

Дмитрий , Вы уж наверняка слышали что именно итальянцы разрушили тогда
промышленный украинский Донбасс. 8-я итальянская армия (она же, ARMIR) прошла
всю Украину и дошла почти до самого Сталинрада - В момент полного формирования
в апреле 1942 года насчитывала 335.000 человек в своём составе. румынов было
даже больше .

\iusr{Анатолий Викторович}

\href{https://youtu.be/PaqfEcmpLYs}{%
4К/ Союзники Германии в кадрах кинохроники 1941 год/ вторая мировая война/Die Deutsche Wochenschau /, %
КамрадLife, youtube, 10.09.2021%
}

\iusr{Дмитрий Заборин}
\textbf{Владимир Епифанов} нафига им было его разрушать? Чтобы немцам работы прибавить? )) Все ж восстанавливали, Рейху угля давать

\iusr{Владислав Макаров}
\textbf{Владимир Епифанов} был целый фильм с Софи Дорен и Марчелло Мастрояни о "приключениях" итальянцев в после военном СССР. Рекомендую


\end{itemize} % }

\iusr{Евгений Статецкий}
Виват! С праздником!

\iusr{Александр Карпец}
Да уж... Бред сивой кобылы.
Но нам этот бред без интереса, разве только для поржать. Ибо у нас деды воевали, а потом восстанавливали страну и летели в космос.
А поэтому: НАСТОЯЩИМ героям слава!
Спасибо дедам, что они выстояли, победили и воссоздали!
И, к сожалению, позор нам, что мы не смогли все это сохранить и приумножить!..

\iusr{Владимир Михайловский}
В большей или меньшей степени в этой стране все так. С первого дня ее убогого существования.

\end{itemize} % }
