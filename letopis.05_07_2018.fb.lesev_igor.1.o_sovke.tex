% vim: keymap=russian-jcukenwin
%%beginhead 
 
%%file 05_07_2018.fb.lesev_igor.1.o_sovke
%%parent 05_07_2018
 
%%url https://www.facebook.com/permalink.php?story_fbid=1983669788330781&id=100000633379839
 
%%author_id lesev_igor
%%date 
 
%%tags obschestvo,psihologia,sovok,sssr
%%title О «совке»
 
%%endhead 
 
\subsection{О «совке»}
\label{sec:05_07_2018.fb.lesev_igor.1.o_sovke}
 
\Purl{https://www.facebook.com/permalink.php?story_fbid=1983669788330781&id=100000633379839}
\ifcmt
 author_begin
   author_id lesev_igor
 author_end
\fi

О «совке».

Слово знакомое нам с детства. А кому не знакомо, напомню, что это
пренебрежительное название СССР. Что-то вроде говно-страна.

Разбирать, почему к Союзу относились одни так, а другие совсем иначе, дело
абсолютно глупое. У каждого свои наклонности и предпочтения. Кто-то любит
зеленое, а кто-то ненавидит острое. Я сейчас о другом.

\ifcmt
  pic https://scontent-lhr8-1.xx.fbcdn.net/v/t1.6435-9/36683966_1983669434997483_2089884623277391872_n.jpg?_nc_cat=103&ccb=1-5&_nc_sid=730e14&_nc_ohc=wnTvC2wxVXEAX8FNBNR&_nc_ht=scontent-lhr8-1.xx&oh=1cc381bc006098527198c66df60cf60a&oe=61B82BB1
  @width 0.7
\fi

Оскорблять то, что давно закончилось и никогда не повторится – это закладывать
в самого себя и во все свои начинания базис неудачника. Все равно как говорить
о бывшей жене, что она непревзойденная сука. Как будто тебе эту «суку» впарили
вместе с мешком и котом.

Говорить «совок» - это называть своих папу и маму сраными неудачниками. А
бабушку и дедушку, тем более. А еще «совок» - это упрощение до уровня
черное/белое своей истории. А упрощение истории – значит ее полное непонимание.
И непонимание процессов, которые приводят к тем или иным последствиям. В том
числе, современным.

Еще «совок» - это пренебрежение к традициям. И пренебрежение к
соотечественникам, для которых традиции существуют. А пренебрежение к чужому –
это строительство песочного фундамента для своих собственных ценностей. Для
своих героев, своих кумиров и своих идентификационных предпочтений. Это значит,
что твои памятники будут разрушены точно также легко, как ты рушишь чужие. Но
самое обидное, что твои памятники будут разрушены не пришлыми оккупантами, а
твоими же детьми. Потому что потребители слова «совок» не могут в принципе
привить своим детям уважение к традициям и истории.

Наконец, обиходное «совок» разрушает пропагандистские возможности государства.
Согласитесь, как-то нелепо гордиться «самым большим в мире самолетом» и «первым
собранным в Киеве компьютером», и в соседнем предложении лепить слово «совок».
Для китайцев Мао – это не синоним «культурной революции», хунвэйбинов и
отстрела воробьев. Хотя в украинских учебниках по истории именно на этом делают
акцент. А для китайцев Мао – это Великий поход, а затем и проблемное, но
продвижение таких парняг, как Джоу Эньлай и Дэн Сяопин, которые и построили
контуры современного Китая. Но без Мао – дикого и кровавого – не было бы и
всего ЭТОГО.

Собственно, это и называется преемственностью. Преемственность – это как
каталог в библиотеке. Все на своей полочке. Все легко находится от «А» до «Я»,
а не от 9 утра и до обеда. Преемственность дает понимание. И преемственность –
враг хаосу. А хаос – это когда «гордишься», что Украина одна из основательниц
ООН, но при этом стесняешься назвать свою родину УССР.

Ну и, естественно, кто видит в своем прошлом «совок», никогда ничего сложнее
совка в своей жизни не построит.

P.S. Люди, которые выборочно идеализируют прошлое, в общем-то, такие же самые
идиоты, которые это же самое прошлое выборочно ненавидят.

\ii{05_07_2018.fb.lesev_igor.1.o_sovke.cmt}
