% vim: keymap=russian-jcukenwin
%%beginhead 
 
%%file 19_12_2021.fb.fb_group.story_kiev_ua.1.detstvo_shkola.cmt
%%parent 19_12_2021.fb.fb_group.story_kiev_ua.1.detstvo_shkola
 
%%url 
 
%%author_id 
%%date 
 
%%tags 
%%title 
 
%%endhead 
\zzSecCmt

\begin{itemize} % {
\iusr{Kovalenko Alexander}

Форма російських шкіл! В Україні були кітелі.

\begin{itemize} % {
\iusr{Надежда Владимир Федько}
\textbf{Kovalenko Alexander} \enquote{Единая советская школьная форма в нашей стране была введена в эпоху Сталина...}

\iusr{Надежда Владимир Федько}
\textbf{Kovalenko Alexander} Кітелі і брюки сірого кольору з напівшерстяної тканини були введені у 1962 році.

\ifcmt
  ig https://scontent-frx5-1.xx.fbcdn.net/v/t39.30808-6/268839306_4818154371577316_4381893464317755_n.jpg?_nc_cat=100&ccb=1-5&_nc_sid=dbeb18&_nc_ohc=kKPqdmdE0ycAX9XyBTW&_nc_ht=scontent-frx5-1.xx&oh=00_AT_9aq5Z4lyyfJHhh-xI1Gsq6X7npaWt-hPhcawkQyVwiQ&oe=61C652DE
  @width 0.4
\fi

\iusr{Ирина Бельская}

Дедушка вспомнил детство. Это прекрасно особенно с этими художественными
домыслами. А этот дедушка еще был и комсомольцем и членом партии иначе не до
жил бы до сегодняшних лет спокойно.

\begin{itemize} % {
\iusr{Надежда Владимир Федько}
\textbf{Ирина Бельская} 

Я був жовтенятком, піонером, комсомольцем... Пройшов \enquote{учебку} в Острі і службу
на Кубі. Працював у спецпідрозділі МВС; супроводжував літаки (після угону
Бразінскасами); маю спеціальну юридичну освіту; працював інспектором карного
розшуку...

Членом КПСС не був (прийняв принципове рішення ще під час служби на Кубі).

Брав участь у національно-визвольних змаганнях. У 1992 і в 1995 рр. очолював
українські делегації на переговорах в НАТО.

Був помічником-консультантом народного депутата Кушнірова. У 2006-му
балотувався у ВР України від партії \enquote{Нова Сила} Юрія Збітнєва. Старший офіцер.
Дотримуюся правих поглядів. І багато чого ще за своє життя вчинив \enquote{дєдушка}.

Дожив до сьогодення і спокійно продовжую жити далі)

Чого і Вам, пані Ірино, бажаю!

\iusr{Ирина Молибог}
\textbf{Надежда Владимир Федько} 

Я с Лукьяновки, всегда пешком на Крещатик. Горжусь, что ходила по улицам, там где
ходили Вы уважаемый автор Спасибо, родной человек.


\iusr{Надежда Владимир Федько}
\textbf{Ирина Молибог} Дякую Вам за добрі слова.
\end{itemize} % }

\iusr{Владимир Каледин}
\textbf{Kovalenko Alexander} 

Не надо этой \enquote{патриотческой} чуши! Форма была едина по всему СССР! А в
Ростове-на-Дону в 1963 она была такая и в Киеве с Харьковом! Не знаю, где вы там
носили \enquote{кители}!

\iusr{Надежда Владимир Федько}
\textbf{Владимир Каледин} 

Так. Уніформа зразка 1948-1962 рр. була єдиною для всього СРСР!

На фотографіях зустрічаються гімнастерки двох видів:

а) без кишень, на трьох ґудзиках із відкладним коміром - для школярів молодших
класів; 

б) армійського зразка, зі стоячим коміром на п'яти ґудзиках, із
кишенями з клапанами на ґудзиках - для школярів старших класів.

Учні молодших класів поверх коміра гімнастерки першого типу в урочистих
випадках носили білий комірець.

До гімнастерок покладався ремінь чорної шкіри з пряжкою жовтого металу.

\end{itemize} % }

\iusr{Evgen Zachepylenko}
КИЕВ.

\ifcmt
  ig https://scontent-frx5-2.xx.fbcdn.net/v/t39.30808-6/269604510_1053075048599928_8017809321685616508_n.jpg?_nc_cat=109&ccb=1-5&_nc_sid=dbeb18&_nc_ohc=OPxxZjVFeJoAX_70hN-&_nc_ht=scontent-frx5-2.xx&oh=00_AT-t3ify-SosMpdK4cXlITylSGJjnUgevEZUut5nlU8g-w&oe=61C6BB53
  @width 0.5
\fi

\iusr{Evgen Zachepylenko}
Киев женский класс

\ifcmt
  ig https://scontent-frt3-1.xx.fbcdn.net/v/t39.30808-6/268822245_1053077578599675_6266613558038583403_n.jpg?_nc_cat=107&ccb=1-5&_nc_sid=dbeb18&_nc_ohc=3LtBRGfjmlYAX9xjMoH&_nc_ht=scontent-frt3-1.xx&oh=00_AT9gTmqWpwnOE4dvuV24kldoxugr37vlxRTDS5GmNlzH-w&oe=61C64738
  @width 0.5
\fi

\begin{itemize} % {
\iusr{Надежда Владимир Федько}
\textbf{Evgen Zachepylenko} Загальношкільна форма була введена у 1948 році.

\iusr{Evgen Zachepylenko}
1958 десять лет прошло как ввели форму.

\ifcmt
  ig https://scontent-frt3-1.xx.fbcdn.net/v/t39.30808-6/267897889_1053215001919266_5226961007033343415_n.jpg?_nc_cat=106&ccb=1-5&_nc_sid=dbeb18&_nc_ohc=dP67ieXo6ugAX-q42NX&_nc_ht=scontent-frt3-1.xx&oh=00_AT9BGvPcUWwTeds1DdcNvs_btvBUCllBSEa9FXUCtBsAMQ&oe=61C67B41
  @width 0.4
\fi

\iusr{Ирина Молибог}

Сердце разрывается, сколько пережито этими девочками с раннего детства,
учительницы страдание в глазах. Не дай Боже войны.

\iusr{Надежда Владимир Федько}
\textbf{Ирина Молибог} Ми вже 7 років на війні. Правда, деякі заплющують на це очі!

\end{itemize} % }

\iusr{Gennady Henry Sergienko}
Чудова хвилююча розповідь. Дякую

\iusr{Владимир Каледин}
Носил я такую форму в первом классе! \enquote{Мышиного} цвета!

\begin{itemize} % {
\iusr{Надежда Владимир Федько}
\textbf{Владимир Каледин} 

Я написав \enquote{коричневого}, тому, що спочатку форма була темно-сірого, а потім,
після декількох прань, вона стала сіро-бурого кольору. Наступного року мені вже
купили форму більш якісну, сірого кольору.

\end{itemize} % }

\iusr{Александр Петренко}
Я пошел в школу в 1961 году именно в такой форме.

\iusr{Владимир Каледин}
\textbf{Александр Петренко} Я в 1963-м и тоже в такой же!

\iusr{Ольга Круковская}

С болью в сердце и нежностью к былому вспоминаю всю пройденную мной дорогу.
Подруги рядом и друзья! Забыть это никак нельзя. Я помню каждого, Их лица не
раз могли присниться.

\iusr{Анна Изряднова}
Идеология есть в любом государстве

\iusr{Надежда Владимир Федько}

Можна багато знати, можна перетворити свою голову на справжню енциклопедію
всіляких знань, але якщо ти не знаєш, ХТО ТИ, навіть якщо при цьому знаєш все
на світі — ти поверхневий!

Якщо ти не запитуєш себе постійно:

\obeycr
- «Хто я?»
- «Звідки я прийшов?»
- «Навіщо Господь мене створив?»
- «Навіщо я існую?»
- «У чому призначення мого життя?»
- «Якщо я вірю в Бога, то чого Він чекає від мене?»
- «Куди я іду?» — ти поверхневий!..
\restorecr

\iusr{Валентин Трусов}

Пані Надія, для Вас віршик - "Я сижу на вишенке, не могу накушаться, дядя ленин
говорит, надо мама слушаться.. @igg{fbicon.face.tears.of.joy} 

\begin{itemize} % {
\iusr{Надежда Владимир Федько}
\textbf{Валентин Трусов} Пані Надія вже перебуває в іншому вимірі. А за віршик дякую) Я його теж пам'ятаю.

\begin{itemize} % {
\iusr{Лариса Павловская}
\textbf{Nadegda Volodymyr Fedko} \enquote{Когда был Ленин маленький с кудрявой головой, он тоже бегал в валенках по горке снеговой}

\iusr{Валентин Трусов}
\textbf{Лариса Павловская} Интересно, где в дальнейшем делись его кудри? Уж не приложила ли к этому руку надежда константиновна?  @igg{fbicon.face.worried} 

\iusr{Валентин Трусов}
\textbf{Лариса Павловская} Дополню еще одним стишком - \enquote{Я маленькая девочка, я в школу не хожу, я ленина не видела, но я его люблю}...  @igg{fbicon.wink} 

\iusr{Лариса Павловская}
\textbf{Валентин Трусов} Автор в начале публикации этот стишок цитировал  @igg{fbicon.smile} 

\iusr{Лариса Павловская}
\textbf{Валентин Трусов} Сифилис, батенька  @igg{fbicon.smile} 

\iusr{Надежда Владимир Федько}

Несамовите вивчення марксизму, титанічна робота над ленінізмом та надмірне
споживання м'яса баранів, телят і кроликів довели Володимира Ілліча до втрати
кудрів!

\ifcmt
  ig https://scontent-frx5-2.xx.fbcdn.net/v/t39.30808-6/269654477_4823535344372552_4836450061655441671_n.jpg?_nc_cat=109&ccb=1-5&_nc_sid=dbeb18&_nc_ohc=C8bKkWu9ak4AX-An3Bb&_nc_ht=scontent-frx5-2.xx&oh=00_AT_GcKc0xK2S6aJLyJQJ5NxS3JfmMx_QpofJoqQLOpEBig&oe=61C72E55
  @width 0.4
\fi

\iusr{Валентин Трусов}
\textbf{Лариса Павловская} За удовольствия надо платить.. @igg{fbicon.face.nerd} 


\iusr{Надежда Владимир Федько}
\textbf{Лариса Павловская} До речі, в борделі \enquote{вождь мирового пролетариата} розплачувався партійними коштами)

\end{itemize} % }

\end{itemize} % }

\iusr{Вадим Сандино}

Еще бы вспомнить переход от фиолетовых чернил на шариковые ручки... вот это была
эпоха! Не то, что с кнопочных мобильников на смартфоны...

\begin{itemize} % {
\iusr{Надежда Владимир Федько}
\textbf{Вадим Сандино} 

Спочатку був перехід з обов'язкових піонерських пір'їв для ручок на інші
пір'я... Далі перехід на автоматичні ручки. Особливим шиком було мати китайську
ручку. А вже потім перехід на кулькові ручки. Але я і в технікумі, і в армії, і
після армії користувався китайськими авторучками.

\begin{itemize} % {
\iusr{Валентин Трусов}
\textbf{Надежда Владимир Федько} Залишились спогади, як ми вимивали чорнильниці і як нам забороняли в 1965 році користуватись кульковими ручками...

\iusr{Надежда Владимир Федько}
\textbf{Валентин Трусов} А ви мух у чорнильницю не макали, а потім пускали по зошиту \enquote{поганих хлопців}?

\iusr{Ольга Васильевна}
\textbf{Валентин Трусов} , да ладно! В 65м шариковые ручки у школьников?
\end{itemize} % }

\iusr{Надежда Владимир Федько}
\textbf{Вадим Сандино} Я і зараз досить часто пишу французькою авторучкою. В цьому є певний шарм)

\ifcmt
  ig https://scontent-frt3-2.xx.fbcdn.net/v/t39.30808-6/269249521_4819630971429656_8037085351578464092_n.jpg?_nc_cat=101&ccb=1-5&_nc_sid=dbeb18&_nc_ohc=Mm3XSPBUetwAX8bFyeH&_nc_ht=scontent-frt3-2.xx&oh=00_AT--agqRa6i1pZiZS6eT8Xo_PMjoF6Lp4h7g1GBR7whGHA&oe=61C7673B
  @width 0.4
\fi

\begin{itemize} % {
\iusr{Александр Петренко}
\textbf{Надежда Владимир Федько} Для особо торжественных случаев у меня есть \enquote{Parker}. А раньше тоже пользовался китайской с \enquote{золотым} пером. Авторучками я начал писать с 7 класса в 1967 году (разрешили).
\end{itemize} % }

\iusr{Любовь Емельянова}
\textbf{Вадим Сандино} 

Когда я пошла в 1й класс в 81 м году вроде бы мы были первыми первоклассниками
которым разрешили пользоваться шариковыми ручками. Вроде бы еще на 1 класс нас
старше еще учились писать чернильными. Но чернильные ручки были дома - иногда я
любила ими писать и заправлять ручку было интересно.

\end{itemize} % }

\iusr{Evgen Zachepylenko}
... не такой молодой и ... октябрь позабыли ...

\ifcmt
  ig https://scontent-frt3-1.xx.fbcdn.net/v/t39.30808-6/268822293_1053326268574806_6288525600996282347_n.jpg?_nc_cat=106&ccb=1-5&_nc_sid=dbeb18&_nc_ohc=b4oWm7YvrF0AX9lWH95&_nc_ht=scontent-frt3-1.xx&oh=00_AT_CEfPII-aLXdR7u7_0341UXbNAXFcikQLvBQn_Q_a6Fw&oe=61C6BB03
  @width 0.4
\fi

\begin{itemize} % {
\iusr{Валентин Трусов}
Бог есть, не дал этому вурдалаку лет жизни. Теперь следующий на очереди...

\iusr{Надежда Владимир Федько}
\textbf{Валентин Трусов} І не один...

\iusr{Надежда Владимир Федько}
\textbf{Evgen Zachepylenko} 

Далеко не всі були дебілоїдами в ті часи. Ми зростали в часи, коли був
засуджений \enquote{культ особи} Сталіна, коли Солженіцина висунули на Ленінську премію
за повість \enquote{Один день Івана Денисовича}. Ми знали, що Сталін був злочинцем! А
відповідно і вся комуністична партія.

Ми бачили такі чудові фільми, як \enquote{Звичайний фашизм} Ромма; \enquote{Попіл і діамант} із
Збігнєвим Цибульським у головній ролі; \enquote{Ніхто не хотів помирати} з Донатасом
Баніонісом.

Ті, хто хотів зробити кар'єру, були активістами в комсомолі, рвалися в
партію...

Нормальні люди отримували вищу освіту і ставилися до членства у комсомолі, як
до необхідної жертви Злу.

Ми слухали закордонні \enquote{радіоголоси} і порівнювали почуте з навколишньою
дійсністю.

Виступати проти Системи, основою якої був рабський стан психології більшості
радянського народу, було самогубством. Бунтарів чекали концтабори в Мордовії
або психіатричні лікарні.

В тих умовах нормальні люди вибирали стан \enquote{внутрішньої еміграції}.

Дуже влучно про це сказано у пісні \enquote{Скованные...}:

Одни слова для кухонь,
Другие для улиц!
\end{itemize} % }

\iusr{Надежда Владимир Федько}

Можу навести приклад критичного мислення із своїх спогадів, написаних для
особового фонду у Державному архіві Київської області:

"... Перші нестиковки радянської офіційної пропаганди з реальним навколишнім
життям я побачив вже під час навчання у першому класі. Як я вже казав, в класі
був створений гурт «жовтенят». За пару тижнів до цього класна керівниця почала
читати нам розповіді про Леніна. Однією з хрестоматійних історій була розповідь
про те, як Ленін сидячи в тюрмі і отримуючи щодня свіже молоко і хліб (!),
придумав спосіб таємно листуватися з товаришами на волі. Майбутній вождь
пролетаріату ліпив із свіжого хліба маленьку чорнильницю, наливав туди свіже
молоко і писав молоком на сторінках книг, які йому передавали для читання
родичі. На волі сторінки книг нагрівали над вогнем свічки і написане проступало
коричневими літерами.

В тому, що це була правда, сумнівів не було – прийшовши додому після уроків, я
сам спробував написати молоком декілька слів на шматочку газети, нагрів над
свічкою і побачив написане, яке проявилося. Правда, мої спроби зліпити
чорнильницю з хліба закінчилися невдачею. Чомусь хліб «не ліпився».

Дисонансом з цією ленінською історією було те, що я досить часто вистоював з
рідними чи сусідами від години до двох в черзі, щоб купити молока, і ще з
півгодини до години в черзі за хлібом, в той час як ув’язненому державному
злочинцю Леніну щодня подавали в камеру свіже молоко і хліб! Це викликало
думку, що Леніну не зовсім погано було сидіти в тюрмі!..."

\begin{itemize} % {
\iusr{Валентин Трусов}
І в еміграції йому непогано сиділось, на партійні кошти, полюблял кухоль - другий пива випить... @igg{fbicon.face.savoring.food} 
\end{itemize} % }

\iusr{Надежда Владимир Федько}

"... Ще одним дисонансом з реальним навколишнім світом були розповіді про
перебування Леніна у сибірському селі Шушенське (Шуша) Минусинского повіту
Єнісейської губернії на трирічному засланні. Дисонанс і заздрість викликали
харчування Леніна і Крупської!

У своїх спогадах Крупська так описує їх побут: «Дешевизна в этом Шушенском была
поразительная. Например, Владимир Ильич за свое «жалованье» – восьмирублевое
пособие – имел чистую комнату, кормежку, стирку и чинку белья – и то считалось,
что дорого платит. Правда, обед и ужин был простоват – одну неделю для
Владимира Ильича убивали барана, которым кормили его изо дня в день, пока всего
не съест; как съест – покупали на неделю мяса, работница во дворе в корыте, где
корм скоту заготовляли, рубила купленное мясо на котлеты для Владимира Ильича,
тоже на целую неделю. Но молока и шанег было вдоволь и для Владимира Ильича, и
для его собаки... В общем, ссылка прошла неплохо».

Як же не позаздрити! За тиждень з’їдати барана! (Це ж скільки ж баранів Ленін з
Крупською з’їли за три роки!?) Наступний тиждень щоденно їсти котлети з м’яса
(очевидно з телятини). Молока і шаньги стільки, що й собаку Ленін годував цією
їжею! Логічний висновок, доступний і дитині: дуже ситно жилося Леніну і
Крупській на засланні!

Наша ж сім’я м’ясо, яловичину або свинину, їла пару раз на два тижні! Яловичину
відварювали, потім м’ясо виймали, перекручували на м’ясорубці і наповнювали ним
млинці. А на бульйоні варили суп чи борщ. З свинини робили котлети, в пропорції
50:50 – свинина і черства булка, розмочена в молоці.

Сусіди теж не жирували, і їли м’ясне не частіше за нас. Оскільки їжа в
комуналці готувалася на спільній кухні, то приховати приготування м’ясної
страви було неможливо.

А за м’ясом теж потрібно було відстояти в довгій-довгій черзі!"

***

(Із моїх спогадів, написаних для особового фонду у Державному архіві Київської
област).

\begin{itemize} % {
\iusr{Лариса Павловская}
\textbf{Nadegda Volodymyr Fedko} А в концтаборах, створених ідейними послідовниками Леніна, нещасних політичних морили голодом й змушували каторжно працювати. Ось вам і \enquote{кривавий} царський лад!
\end{itemize} % }

\iusr{Надежда Владимир Федько}

Дозволю собі процитувавати ще один уривок із своїх спогадів, який показує, що
частина суспільства, до якої надлежав і я, мислили критично і не були в захваті
від радянської влади. Але жорстока реальність життя: або ти мовчиш, або ти
сидиш в концтаборі, заставляла імітувати лояльність до влади, в душі
залишаючись дисидентом:

"... Знову почалися читання розповідей про Леніна і про піонерську організацію,
яка носила його ім’я. Але читала нам вже не вчителька, а піонери з IV– VI
класів! Ключовою темою читань були піонери-герої! А культовою фігурою, піонером
№ 1 – Павлик Морозов!

У мене ж Павлик Морозов одразу викликав неприйняття через донос на близьких
родичів – батька й діда! Розмова з мамою і бабусею, підтвердили мої сумніви в
моральності вчинку Павлика. В той же час вони сказали, щоб у школі я ні в якому
разі не висловлювався негативно про Павлика Морозова! Моя неприязнь до Павлика
базувалася ще й на тому, що в шкільному середовищі доносительство
(ябедничество, стукачество) на товаришів вважалося недостойним вчинком! Вчителі
ж нам втовкмачували, що піонер повинен завжди говорити правду про себе, про
своїх товаришів, про батьків і родичів! Підтвердженням цієї ж тези служили
пропагандистські плакати, на яких же був зображений піонер-герой Павлик
Морозов.

Символом героїзму тих років був піонер-герой Павлик Морозов. Юний донощик,
зрадник власного батька, був зроблений національним героєм!

«Піонерська правда» писала: «Павлик не щадить нікого: попався отець – Павлик
видав його, попався дід – Павлик видав його. Павлика виростила і виховала
піонерська організація».

Під час тих читань я дізнався, що ідея залучення дітей до доносів при
сталінізмі мала потужну державну підтримку. Виховання донощиків стало важливим
напрямком ідеологічної діяльності. Донос подавався як нова якість радянських
людей: як їх відкритість і чесність, як критика, що сприяє поліпшенню життя, як
необхідний засіб для досягнення великої мети, в яку донощики щиро вірили! Тепер
до «святої ідеї» доносительства почали долучати і нас! Піонерська організація
хотіла виховати з нас не тільки «будівників комунізму», але й донощиків!

Безумовно, що в дев’ятирічному віці я не міг робити якісь глобальні світоглядні
чи філософські висновки, виробити якусь усталену світоглядну позицію... але в
пам’яті все це відкладалося і, як я впевнений, в подальшому додалося до
формування негативного ставлення до комуністичної ідеології, радянської влади і
керівників СРСР."


\iusr{Людмила Гаценко}

Шоста школа мала спеціалізацію \enquote{французька мова}, правда, не пам'ятаю,
з якого року. В 60-70-х роках мала цей нахил.

\begin{itemize} % {
\iusr{Мария Константиновская}
\textbf{Людмила Гаценко} так. Мій брат там навчався французської!

\iusr{Valentina Syssoena}
\textbf{Людмила Гаценко} я в 1974 г проходила педпрактику в этой школе по Франц яз

\iusr{Людмила Гаценко}
\textbf{Valentina Syssoena}
Bonjour.
\end{itemize} % }

\iusr{Мария Константиновская}
І шо?

\iusr{Надежда Владимир Федько}

Маю запитання до спільноти: Чи варто опублікувати спогади про дворові і шкільні
розваги того часу?

\begin{itemize} % {
\iusr{Людмила Гаценко}
\textbf{Nadegda Volodymyr Fedko}
Було б цікаво: одні згадали б, інші дізналися б, порівняли з іншими часами.

\iusr{Владимир Каледин}
\textbf{Надежда Владимир Федько} Так!

\iusr{Tamara Sorochan}
\textbf{Надежда Владимир Федько} приваблює велика кількість деталей, які зафіксувала дитяча свідомість і ставлення до подробиць. У цьому видно характер героя оповідань. Як на мене, цікаво. Це вже сторінки історії. До речі, по цих спогадах можна вивчати історію педагогіки в контексті епохи.

\begin{itemize} % {
\iusr{Надежда Владимир Федько}
\textbf{Tamara Sorochan} 

У мене мама була філолог, викладала російську мову і літературу в школі і в
педучилищі, була завучем в школі. Мене з дитинства привчила знати дві мови.

***

Пікантність певного періоду була в тому, що я вчився в старших класах у школі,
де мама була завучем. Але досить довго про це ніхто не знав, оскільки я
залишився на прізвищі батька - Федько, а мама після розлучення повернулася на
своє дівоче прізвище - Костенко.

Тільки після однієї моєї крутої бійки, коли мене викликали на педраду (це вже
було не із-за поцілунків) усі вчителі дізналися, що я син завуча. Мені після
цього прийшлося перейти в іншу школу, оскількии вчителі, які конфіліктували з
мамою, відігравалися на мені.

\end{itemize} % }

\iusr{Alisa Brekhova}
\textbf{Nadegda Volodymyr Fedko} варто. Дуже цікаво вас читати

\iusr{Ирина Молибог}
\textbf{Надежда Владимир Федько} Варто. Чэкаемо. З повагою.

\iusr{Надежда Владимир Федько}
\textbf{Ирина Молибог} Днями опублікую. Ця частина буде пригодницькою...

\end{itemize} % }

\iusr{Надежда Владимир Федько}

Я не претендую на лаври есеїста, новеліста чи письменника. Я військовий історик
і історик спецслужб.

У своїх спогадах я не хочу щось прикрасити, в тому числі і себе, чи комусь
сподобатися. Вони пишуться, як свідчення про життя, учасником і свідком якого я
був і являюся. Я багато чого бачив в житті, і доброго і поганого. Я пишу
спогади для того, щоб не було новітніх спекуляцій про \enquote{щасливе і заможне життя
в СССР}, навіюваних в кращому випадку ностальгією по молодості, а в гіршому -
інспірованих спецпропагандою однієї сусідньої держави.

Мені і смішно, і гірко читати, ностальгічні ридання по \enquote{бульйонних кубиках},
які \enquote{в СССР були кращими, ніж зараз} або по пиріжках з лівером...

Чомусь ці ностальгуючі не згадують про мільйони убитих у підвалах
ВЧК-ОГПУ-ГПУ-НКВС та в таборах ГУЛАГу; про мільйони померлих від Голодомору;
про 15 тисяч молодих хлопців, полеглих у злочинній війні в Афганістані... І
багато про що варто було б згадати ще...

***

Тут одна пані єхидно проїхалася щодо спогадів \enquote{дєдушки}... Очевидно її родичі
активно боролися з радянським режимом в лавах СС Дивізії \enquote{Галичина} чи в
підпіллі ОУН; а може активно виступили проти вводу радянських військ у
Чехословаччину; а може підняли голос протесту проти висилки Солженіцина та
Некрасова; а може...

Я з задоволенням прочитаю ці спогади і зніму кепі в знак пошани перед цими
борцями!

\iusr{Людмила Новікова}
\textbf{Nadegda Volodymyr Fedko} браво

\iusr{Naumova Svitlana}
Думаю, що бюста сталіна в 54 му не повинно було вже бути.

\iusr{Надежда Владимир Федько}
\textbf{Naumova Svitlana} В 1954-му ще були. Прибирання почалося десь у 1955-му... ХХ з'їзд партії відбувся у 1956-му.

\iusr{Надежда Владимир Федько}

"Ще один негатив відклався в дитячій пам’яті у зв’язку з любов’ю Леніна до полювання!

На читаннях класна керівниця закликала нас до самостійної роботи – читати і
переказувати розповіді про Леніна. Літератури про Леніна для дітей було багато.
Як мемуарної – спогадів Єлізарової, Крупської та старих більшовиків, так і
творів радянських письменників.

Один хлопець, з активістів (були вже такі), приніс в школу книжку спогадів
Крупської і на читаннях почав переказувати, як Ленін з товаришами одного разу
настріляли цілий човен зайців! Хлопець читає, вчителька схвально дивиться на
нього... І раптом одна дівчинка почала гучно плакати. Вчителька почала її
заспокоювати і виясняти, що стало причиною її плачу. Крізь сльози дівчинка
промовила, що їй дуже шкода зайців, і що дід Мазай зайців рятував, а Ленін з
товаришами поступив негарно! Вірш про «діда Мазая і зайців» усім нам був
відомий, бо це була дитяча класика російської літератури! Вірш був виданий
окремою, гарно ілюстрованою, книжкою, яку всі ми добре знали!

Після цього «самостійна робота» миттєво була згорнута!"

\ifcmt
  ig https://scontent-mxp1-1.xx.fbcdn.net/v/t39.30808-6/269593667_4820097734716313_3933412168194983592_n.jpg?_nc_cat=110&ccb=1-5&_nc_sid=dbeb18&_nc_ohc=XATkjYKBk8wAX_mHizU&_nc_ht=scontent-mxp1-1.xx&oh=00_AT918Qdo8LC6-V4D0g6J0nX1uEIGXTIcOMDO2wJNHz8X2Q&oe=61C791AE
  @width 0.4
\fi

\begin{itemize} % {
\iusr{Владимир Шипов}
\textbf{Надежда Владимир Федько} 

Очень характерный пример, иллюстрирующий истиную сущность тех, кто пришел к
власти в 1917 году, до большевицкого переворота спасали, после стреляли. И
наступило мрачное серо-черное с кровавым красным \enquote{кумачом} время на
долгие 70 лет... @igg{fbicon.thinking.face} 

\end{itemize} % }

\iusr{Татьяна Шиверская}
Читала и сама вспомнила свой первый класс, школу. Думала, что ничего не помню.
только целоваться стала намного позже.

\iusr{Borys Bazyl}
Форма у хлопчиків московська. В Україні була інша форма.

\begin{itemize} % {
\iusr{Владимир Каледин}
\textbf{Borys Bazyl} 

Какая на хрен \enquote{московська}!? От руссофобия уже ум за разум заходит!?
Вам сколько лет? На фото хрен разберёшь!


\iusr{Borys Bazyl}
\textbf{Владимир Каледин} 

Хлопче руцькомірний, ти може хрен і не бачиш, але в Україні була темно-синя
форма у стилі піджака, який ніколи не підперезувався. В Москві та в твоїй
улюбленій Московії форма була сіра у формі гмностьорки, що підперезувалася
паском з бляхою. На фото хлопчики в Москві на Бульварному кільці біля
пам`ятника Тимірязєву. Так що закрий фонтан. Україна і тоді була неРосією.

\iusr{Владимир Каледин}
\textbf{Borys Bazyl} 

Во-первых, руськомовний и коренной киевлянин, в отличие от вас \enquote{заезжего}! В
моем \enquote{улюбленом} Киеве форму темно синего цвета в виде брюк и пиджака под
белую рубашку в школах начали носить с 1967 года,а до этого по ВСЕМУ СССР, в
том числе и в Киеве была форма серого цвета с \enquote{гимнастеркой} под ремень с
бляхой и такая же серая фуражка! Это, если говорить по теме обсуждения! Я такой
же гражданин этой страны, как и вы и не важно на какой \enquote{мове} я с вами общаюсь!
Кстати, не важно где сделано фото, в Москве или Киеве, форма была ЕДИНАЯ!!!
Краска!

\iusr{Borys Bazyl}
\textbf{Владимир Каледин} в школу я пішов 1953 року, а закінчив у 1964. Так що не фантазуй хлопче. Може комусь і привозили форму з Москви, але всі ми носили українську форму і пишалися цим. На цьому ставлю крапку.
\end{itemize} % }

\iusr{Tamara Murga}
Людина не бувала в дитсадку, нічого про те не знає, але критикує. Як це по-совєцьки: на чітал, но осуждаю

\iusr{Надежда Владимир Федько}
\textbf{Tamara Murga} Я достатньо наслухався розповідей тих, хто ходив у дитячий садок)

\iusr{Dmytro Nohal}
всегда забавно читать комментарии.

\iusr{Михайло Наместник}

Я ваш одноліток і вчився неподалік в СШ 38, але жодних жовтенят у нас не було.
Десь у 3 класі приймали в піонери в старому музеї Леніна

\begin{itemize} % {
\iusr{Квітницька Ганна}
\textbf{Михайло Наместник} а в моей 161 школе, даже позже на...несколько лет))) были и звездочки и звенья(пионерия) и ячейки )))...
 @igg{fbicon.thinking.face} Продвинутая что ли школа была)))?
Для справки - в школу я пошла в 1972 году)

\iusr{Михайло Наместник}
\textbf{Квітницька Ганна} во, сравнила! Дык, в 72-м году, отмечая первый юбилей, я лежал под столом пьяный и на магнитофоне крутил ручку громче.

\iusr{Надежда Владимир Федько}
\textbf{Михайло Наместник} Нас теж приймали в піонери у старому музеї Леніна. Це була стандартна процедура.
\end{itemize} % }

\iusr{Аркадий Израилевский}
До 1957года обучение в школах было раздельное для мальчиков и девочек.

\begin{itemize} % {
\iusr{Надежда Владимир Федько}
\textbf{Аркадий Израилевский} До 1954-го.

\begin{itemize} % {
\iusr{Аркадий Израилевский}
\textbf{Надежда Владимир Федько} моя старшая сестра 1946г.р.пошла в школу в 1953 году. Первый и второй класс училась раздельно. В 1956 отменили раздельное обучение и классы стали совместные.

\iusr{Надежда Владимир Федько}
\textbf{Аркадий Израилевский} Про що ми сперечаємося? Я пішов у школу в 1954-му. Навчання хлопчиків і дівчаток з першого класу було спільним)

\iusr{Аркадий Израилевский}
\textbf{Надежда Владимир Федько} вас подводит память. Откройте Википедию, там написано, что отменили в 1956 году.

\iusr{Надежда Владимир Федько}
\textbf{Аркадий Израилевский} Невже я можу забути дівчинку, з якою сидів за одною партою?!
А Вікіпедія це сміттєве джерело!

\iusr{Аркадий Израилевский}
\textbf{Надежда Владимир Федько} не переживайте, в нашем возрасте проблемы с памятью приемлемы.

\iusr{Надежда Владимир Федько}
\textbf{Аркадий Израилевский} Постановою Ради Міністрів СРСР від 18 липня 1954 року повернули спільне навчання хлопчиків і дівчат.

***

Джерела, що заслуговують на довіру:

- Галина Гончарова. Мальчики налево, девочки направо. Период раздельного
обучения в СССР. Из книги «Время, вперед! Культурная политика в СССР». М.,
2013.

- Народное образование в СССР / Под ред. М. А. Прокофьева и других. М., 1967.

- Михаил Павловец. Что читали советские школьники. Arzamas, 21.03.2017.

- Тригубенко В. В., Нестеренко Л. С. Освіта і педагогічна думка в Києві за
тисячу років. К., 2002.

\iusr{Аркадий Израилевский}
\textbf{Надежда Владимир Федько} отправьте ссылку.

\iusr{Надежда Владимир Федько}
\textbf{Аркадий Израилевский} Заборонено правилами групи.
Погугліть і знайдете.
\end{itemize} % }

\end{itemize} % }

\iusr{Михайло Наместник}

Вы ошибаетесь, Аркадий - раздельное обучение отменили в июле 1954 года и в
сентябре я пошел в свой первый класс вместе с девочками.

\iusr{Надежда Владимир Федько}
\textbf{Михайло Наместник} 

Я пішов у школу у вересні 1954-го. Розсаджувала нас по партам класний керівник.
І принцип був: хлопчик - дівчинка за одною партою.

\iusr{Вадим Сандино}

Не спорьте: надо или не надо! Самое главное, что после этих споров к нам не
приходят со словами: мы к вам, профессор, и вот по какому делу!

\iusr{Юрий Блохин}

Рассказ интересный, если бы не определённые акценты, что делает рассказ в
некоторой степени пресным, у меня как-то прошло все более ярко и весело без
уделения внимания на некоторые мелочи, о которых я и не вспоминал бы.

\iusr{Надежда Владимир Федько}
\textbf{Юрий Блохин} 

Про школу написано декілька частин... Є про шкільні забави-розваги, є про
бійки, є про погані звички, є романтичне...

\iusr{Тамара Кипнис-Иванисова-}

Изумительные воспоминания. Все правда. Про девочек, мальчиков, октябрят и т.д.
Действительно, обучение начиналось с \enquote{ноля}. мало, кто умел читать. Но научить
читать и считать 42 ученика (столько было в моем 1 классе) - это подвиг, высший
пилотаж учителя. Много, чего можно добавить. Счастливое время. Читающие дети,
и, как правило, грамотные все, без исключения, независимо от того, какое
получили в дальнейшем образование. Кто-то возразит?

\begin{itemize} % {
\iusr{Михайло Наместник}
\textbf{Тамара Кипнис-Иванисова-} 

писать, тем более каллиграфическим почерком, не умел никто, но читали уже
многие. Во всяком случае \enquote{Приключения Робинзона Крузо} мы бегло читали еще в
первом классе.

\end{itemize} % }

\iusr{Надежда Владимир Федько}

Реальний пацан в інтер’єрі школи 1950-х...

***

- Чи любив я школу!?
- Дуже любив!
- Чи були у моєму класі такі, що не любили школу!?
- Таких не пам’ятаю!
- А навіщо я ходив у школу!?
- Школа давала три базові речі:

• Перші знання і, головне, навички систематично працювати з книгами і
інформацією;

• Налаштовувала на пошук свого місця «під сонцем» в плані вибору професії і
подальшого навчання: в ПТУ (професійно-технічному училищі), технікумі,
інституті або університеті;

• Спілкування з ровесниками і вміння вибудовувати стосунки в колективі,
завойовувати авторитет. Тобто, соціалізацію.

***

Якби мене протягом останніх шістдесяти п'яти років шістдесят п'ять разів
допитували на поліграфі («детектор брехні»), то мої відповіді були б незмінними
(читай вище!) і поліграф показав би, що я говорю чисту правду!

\iusr{Надежда Владимир Федько}

Люди народжуються нерівними, але рівноцінними.

Це – аксіома!

Люди народжуються нерівними, в тому числі і розумово, що проявляється вже в
перших класах школи. За п’ять років навчання я бачив з десяток живих
другорічників, які стартували разом зі мною в першому класі та в паралельному
першому класі. Хлопців і дівчаток, які не могли за рік осягнути шкільну
програму 2-го, 3-го, 4-го або 5-го класів, і їх залишали на повторний рік
навчання.

Ідеологічна доктрина в часи мого дитинства проголошувала «обов’язкову семирічну
освіту». Тому існували школи, які давали неповну середню освіту і після 7
класів видавали відповідний документ. Та існували школи, які давали повну
середню освіту і видавали атестат після закінчення 10 класу.

Теза про «обов’язкову семирічну освіту» досить часто вступала у протиріччя з
реаліями життя. Навчальний процес в школі налаштований на середній інтелект
школяра. При наявності в класі 30 – 35 учнів, вчитель просто фізично не може
приділити індивідуально кожному відповідну увагу (урок-то всього 45 хвилин).
Поступово розумова нерівність починає проявлятися. Виділяється група учнів,
яким навчання дається легко і які, як кажуть, «схоплюють все на льоту».
Виокремлюється група «зубрил», які тримаються на середньому рівні завдяки тому,
що зазубрюють навчальні матеріали до літери, до коми. І, нарешті, залишається
група учнів, які перебиваються з двійки на трійку, іноді піднімаючись до
четвірки. На якомусь етапі вони стають постійними двієчниками і залишаються на
другий рік навчання.

(У класі моєї доньки, як пішла в школу на початку 1980-х, був характерний
хлопчик, який залишився на другий рік у першому класі.)

Нам подобалося ходити в школу! Досить часто ми приходили раніше, щоб пограти з
однокласниками в ігри або просто поспілкуватися. Досить часто ми залишалися в
школі після уроків, щоб продовжити спілкування або пограти спільно в ігри.

Ми вивчали шкільні предмети і читали додаткову літературу. За власними
інтересами ми записувалися в технічні і творчі гуртки, які були в школі і в
Палацах піонерів у районах нашого проживання. Одні з нас були відмінниками і
закінчили навчання із золотими медалями; інші, в числі яких був і я, закінчили
навчання з кількома четвірками; деякі, дотягли на трійках до атестату про
неповну середню освіту. А дехто залишався на другий рік, знову залишався на
другий рік… і подальшу їх долю я не знаю.

Ми не були злостивими чи злобними. Незважаючи на різні конфлікти і бійки, ми
любили один одного. Ми билися на кулаках і портфелями. В той же час було табу
на бійку ногами чи ременями з бляхою. Я вже не кажу про застосування у бійці
ножика чи кастета.

Ми дражнили товстунів «жиропами» («жирна ж**а») не із злостивості, а тільки
тому, що вони приносили довідки про звільнення від уроків фізкультури, які їм
добували тупі ЯЖматері, що не розуміли яку величезну шкоду вони завдають
здоров’ю своїх улюблених чад. Посилаючись на довідки, жиропи відмовлялися
пересувати парти під час щотижневого прибирання класу. І цим прізвиськом ми
виражали свою зневагу до них. Одночасно ми ніколи не відмовляли жиропам у
допомозі по урокам і давали списувати.

У нас було багато енергії, яка потребувала виходу. Тому ми і бешкетували на
уроках! Але при цьому головним для нас завжди було навчання, отримання знань!

Хочу однак зазначити, що при всій потужній пропаганді ролі пролетаріату і
селянства в суспільстві, працювати за верстатом чи полоти буряки в полі мрій ні
в кого не було. Практично всі наші мрії були пов’язані з інтелігенцією,
технічною та творчою.

\iusr{Владимир Дубровский}

Не ходил в детский сад. И как же он прижился в коллективе? Я как раз ходил, и
не жалею, помню как пятеро мальчишек танцевали \enquote{матросский танец}, или
постановку \enquote{конница}- к стульям привязывали вырезанные из фанеры конские
головы, и те же пятеро сначала перебирали ногами под мелодию \enquote{Полюшко - поле},
а потом под мелодию \enquote{Тачанка} с пластмассовыми саблями бегали по залу. Я в этих
танцах не участвовал, но дома показывал.

\begin{itemize} % {
\iusr{Надежда Владимир Федько}
\textbf{Владимир Дубровский} 

У нас була величенька дворова банда. Приблизно осіб двадцять хлопчиків і
дівчат. Ми також дружили з такими самими бандами із сусідніх дворів. Ходили
гуляти на Володимирську гірку, на Андріївський узвіз, на подільські пагорби, в
Піонерський парк.

У нас було набагато більше свободи, ніж у сучасних дітей, над якими трясуться
батьки.

Поснідав і \enquote{бігати} (так тоді називалося гуляння). Додому приходиш
пообідати. А потім знову \enquote{бігати}. Влітку наші гуляння завершувалися
десь о 9-й вечора.

\iusr{Надежда Владимир Федько}

З нашого двору та із сусідніх більше десятка хлопчиків і дівчаток пішли в 1-й
клас разом зі мною. Там знову створилися шкільні банди)
\end{itemize} % }

\iusr{Elizaveta Gorbonos}

Я тоже пошла в школу в 54 на Артёма. это был первый год совместного обучения
мальчиков и девочек. А форма у мальчиков была тёмно-синей. И тоже, октябрята,
пионеры

Учительница первая была очень хорошей, ещё гимназию заканчивала и к нам
относилась очень хорошо. Делались специальные уроки внеклассного чтения:
\enquote{Белеет парус одинокий} и другие. Давно, очень давно это было.

\iusr{Надежда Владимир Федько}

Щодо шкільної форми... Я зараз упорядковую і оцифровую наш сімейний архів. Ось
фотографія з шкільного дитинства дружини... 4-й клас, 1962 рік. Усі дівчатка в
шкільній формі. А хлопчики хто в чому... На жаль запитати чому так було вже не
має можливості. Але це офіційна фотографія після закінчення 4-го класу.

\ifcmt
  ig https://scontent-mxp1-1.xx.fbcdn.net/v/t39.30808-6/269724888_4824546364271450_310983227666543983_n.jpg?_nc_cat=105&ccb=1-5&_nc_sid=dbeb18&_nc_ohc=BhW2Ukvay2gAX92kk5U&_nc_ht=scontent-mxp1-1.xx&oh=00_AT_UmbPzIN1eDV4cB_VQOoK9pyoge7ypzyw4UjinN4BkrA&oe=61C7A680
  @width 0.5
\fi

\begin{itemize} % {
\iusr{Любовь Емельянова}
\textbf{Надежда Владимир Федько} красивое фото. Скажіть будь ласка а як ви оцифровуєте фото? Які технічні засоби використовуєте - сканер чи фотоаппарат? Я теж робила спроби оцирувати - знімала на фото - не дуже виходить.

\begin{itemize} % {
\iusr{Надежда Владимир Федько}
\textbf{Любовь Емельянова} 

Я використовував два варіанти: сканер і смартфон. В результаті експериментів
прийшов до висновку, що смартфоном оцифровувати краще - градація відтінків
ширша. Зараз використовую смартфони \enquote{Samsung} A10 і \enquote{Redmi 7}.

Головне в репродукції - стабільність оригінала і камери, збереження геометрії
оригінала.

Для стабілізації оригінала використовую підставку. Для стабілізації смартфона
штатив (є спеціальні штативи для смартфонів). Але я використовую фотоштатив з
перехідником для смартфона.

\iusr{Надежда Владимир Федько}
\textbf{Любовь Емельянова} 

Ставите фотографію на підставку, а смартфон на штатив; кадруєте на екрані,
зберігаючи геометрію фотографії; торкаєтесь екрана, щоб зафіксувати різкість;
торкаєтесь вдруге до екрана, щоб сфотографувати. Я даю затримку автоспуском \enquote{5
сек}. Таким чином, позбавляюся можливих коливань смартфона. З кожного сюжету
роблю дубль.

Після закінчення фотографування переписую файли на комп'ютер і обробляю в
ACDSee або у Фотошопі (остаточно кадрую).

Успіхів)

\ifcmt
  ig https://scontent-mxp1-1.xx.fbcdn.net/v/t39.30808-6/269793158_4826261800766573_8910001320584898225_n.jpg?_nc_cat=107&ccb=1-5&_nc_sid=dbeb18&_nc_ohc=LI2PW9JEau4AX8RSTL5&_nc_ht=scontent-mxp1-1.xx&oh=00_AT8HFLUfNEL8GNKea9o2DRwupkJSKLWdS5bUoJubiLvGrQ&oe=61C7E67A
  @width 0.3
\fi

\iusr{Любовь Емельянова}
\textbf{Надежда Владимир Федько} 

дякую вам за відповідь! Все зрозуміло. А як ви фотографію фіксуєте? У мене
багато фото з коробки і вони завертаются кінці, тобто вони не пласкі. І сканер
їх би прижав. І ще пробувала склом з фоторамки прижимати - і фотогрофувати -
погано виходить - скло блікую або не дає глибини фото.

\iusr{Надежда Владимир Федько}
\textbf{Любовь Емельянова} 

Можна акуратно її вигнути в іншу строну. А можна прищепками за куточки прищепити до картонки.

***

Раніше були рамки для друку фотографій з лінійками, які притискали краї
фотопаперу і виходив білий кантик. Це ідеальний варіант для репродукції. Але
зараз знайти таку рамку майже неможливо. Хіба що на \enquote{блошиному ринку}.

\iusr{Любовь Емельянова}
\textbf{Надежда Владимир Федько} дякую за відповідь! Десь таку рамку я бачила вдома у мами. Папа займався фотографієй

\iusr{Надежда Владимир Федько}
\textbf{Любовь Емельянова} Пошукайте. Це ідеальний варіант. Лінійками можна притиснути фотографії різного розміру, а саму рамку поставити під кутом.

\end{itemize} % }

\end{itemize} % }

\iusr{Valentina Khmara}
А я свой советский садик вспоминаю с тёплом!

\iusr{Valentina Khmara}
Вся публікація пронизана негативом, а ви знаєте, що зараз відбувається в школах? Я своє дитинство згадую тільки з теплом.

\begin{itemize} % {
\iusr{Михайло Наместник}
\textbf{Valentina Khmara} можливо, і сучасні діти у вашому віці про школу будуть згадувати з теплом.

\iusr{Надежда Владимир Федько}
\textbf{Valentina Khmara} Зло в школи приносять діти з дому.
\end{itemize} % }

\iusr{Надежда Владимир Федько}

До дискусії про відміну роздільного навчання:
Київська середня школа № 117, 1 вересня 1955 року. Перший урок в одному з класів.
(Архів кінофотодоментів ім. Г.С. Пшеничного)

\ifcmt
  ig https://scontent-mxp1-1.xx.fbcdn.net/v/t39.30808-6/269706261_4826955680697185_3175808556151253347_n.jpg?_nc_cat=100&ccb=1-5&_nc_sid=dbeb18&_nc_ohc=WX8T_2LQ2ygAX9yoVUc&_nc_ht=scontent-mxp1-1.xx&oh=00_AT9n1961xVVMP0oXQL0ZYyBEIp4janSBO8Ntqgb2qKyrIg&oe=61C7E8A1
  @width 0.4
\fi

\iusr{Надежда Владимир Федько}

1 июля 1954 года вышло постановление Совета министров СССР №1342 о
восстановлении совместного обучения в Москве, Ленинграде и других городах
страны.

***

Учащиеся школы № 201 имени Героя Советского Союза Зои Космодемьянской после
занятий во дворе школы. Автор – В. Шаровский. Москва. 01 сентября 1954 г.
Главархив Москвы

\ifcmt
  ig https://scontent-mxp1-1.xx.fbcdn.net/v/t39.30808-6/269732373_4827018287357591_2042180790562555602_n.jpg?_nc_cat=111&ccb=1-5&_nc_sid=dbeb18&_nc_ohc=HtrZZuYXNV8AX8Gftzg&_nc_ht=scontent-mxp1-1.xx&oh=00_AT9UdeRks9J1vISgepfs7stPwA2NGyQP0jCUJmAC18EUaw&oe=61C813AC
  @width 0.4
\fi

\end{itemize} % }
