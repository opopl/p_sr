% vim: keymap=russian-jcukenwin
%%beginhead 
 
%%file 01_06_2023.stz.news.ua.donbas24.1.voin_zsu_evakuacia_rodyna_hlopchyk_bahmut
%%parent 01_06_2023
 
%%url https://donbas24.news/news/voyin-zsu-geroyicno-evakuyuvav-rodinu-malenkkogo-xlopcika-z-baxmuta-video
 
%%author_id demidko_olga.mariupol,news.ua.donbas24
%%date 
 
%%tags 
%%title Воїн ЗСУ героїчно евакуював родину маленького хлопчика з Бахмута (ВІДЕО)
 
%%endhead 
 
\subsection{Воїн ЗСУ героїчно евакуював родину маленького хлопчика з Бахмута (ВІДЕО)}
\label{sec:01_06_2023.stz.news.ua.donbas24.1.voin_zsu_evakuacia_rodyna_hlopchyk_bahmut}
 
\Purl{https://donbas24.news/news/voyin-zsu-geroyicno-evakuyuvav-rodinu-malenkkogo-xlopcika-z-baxmuta-video}
\ifcmt
 author_begin
   author_id demidko_olga.mariupol,news.ua.donbas24
 author_end
\fi

\ifcmt
  ig https://i2.paste.pics/d340dd4ce318e8bed7a9b70f3c7b9d1c.png
  @wrap center
  @width 0.9
\fi

\begin{qqquote}
\Large\bfseries
У День захисту дітей в Києві розповіли одну з дивовижних історій російсько-української війни
\end{qqquote}

У Міжнародний день захисту дітей, 1 червня, футбольний клуб \enquote{Маестро} провів
пресконференцію в приміщенні мультиспейсу Klitschko-Expoз з приводу презентації
власного благодійного спортивно-концертного туру \enquote{Ми з України!}. Крім цього,
на заході відбувся показ щемливого фільму, присвяченого щасливому порятунку
однієї сім'ї з Бахмута.

\textbf{Читайте також:} 

\href{https://donbas24.news/news/rosiyani-peretvorili-baxmut-na-ruyini-ta-popil-u-merezi-zyavilisya-svizi-kadri-z-mista-video}{Росіяни перетворили Бахмут на руїни та попіл: у мережі з'явилися свіжі кадри з міста (ВІДЕО)}

Головною подією на пресконференції став показ унікального фільму, присвяченого
сім'ї хлопчика з Бахмута, якого врятував гравець ФК \enquote{Маестро}, боєць ЗСУ Петро
\enquote{Стоун} Волощенко. У війську чоловік перебуває з перших днів війни. До
повномасштабного вторгнення киянин був досвідченим мандрівником - облетів 139
країн, 20 з яких - найнебезпечніші у світі. Спочатку чоловік подорожував
популярними для відпочинку країнами, однак згодом захопився небезпечними
маршрутами, які не порадить жоден турагент. 

\begin{leftbar}
  \begingroup
    \bfseries
      \enquote{Я шукав екстрим по всьому світі, а знайшов його в Україні...}, - наголосив Петро.
  \endgroup
\end{leftbar}

\ii{01_06_2023.stz.news.ua.donbas24.1.voin_zsu_evakuacia_rodyna_hlopchyk_bahmut.pic.1}

При виконанні бойових задач у Бахмуті військовий виявив декілька родин
цивільних, дім яких опинився на лінії фронту. У найбільш гарячій точці України
неочікувано для себе Петро потоваришував з 7-річним хлопчиком на ім'я Стасик.
Хлопчик вразив військового своєю мужністю та став дуже близьким Петру. Чоловіку
вдалося вмовити його батьків та організувати спецоперацію з евакуації родини у
безпечне місце. Через декілька днів дім, де жив Стасик, був зруйнований
російськими снарядами. Тому можна вважати, що завдяки своєчасній евакуації вся
сім'я була врятована. Важливо, що Петро Волощенко не залишив сім'ю Статика
напризволяще. Він особисто опікувався переїздом родини у безпечне місце, і коли
це відбулося, став хрещеним батьком Стасика. На пресконференції був і головний
герой цієї історії - Стасик та його батьки. Хлопчику вручили знаковий подарунок
- планшет.

\textbf{Читайте також:} \href{https://donbas24.news/news/zsu-prodovzuyut-prosuvatisya-pid-baxmutom-ta-nishhat-rosiyan-na-donbasi}{ЗСУ продовжують просуватися під Бахмутом та нищать росіян на Донбасі}

\ii{01_06_2023.stz.news.ua.donbas24.1.voin_zsu_evakuacia_rodyna_hlopchyk_bahmut.pic.2}

\begin{leftbar}
  \begingroup
    \bfseries
\enquote{Коли ми почали товаришувати, я пообіцяв йому, що заберу його. Я
переконав його батьків, що Стасику потрібне краще життя, навчання. З
божою поміччю вдалося їх вивезти}, - розповів Петро Волощенко.

\enquote{Наразі ми оговтуємося від багатомісячного життя під вибухами та звикаємо до тиші у Чернівцях}, - розповів батько Стасика. 
  \endgroup
\end{leftbar}

Ця унікальна історія вчергове нагадала скільки таких дітей як Стасик лишилися
без дома. І українські артисти знайшли спосіб як їм допомогти.

\textbf{Читайте також:} \href{https://donbas24.news/news/blagodiinii-projekt-ditinstvo-bez-viini}{Благодійний проєкт \enquote{Дитинство без війни}}

\ii{01_06_2023.stz.news.ua.donbas24.1.voin_zsu_evakuacia_rodyna_hlopchyk_bahmut.pic.3.video}

Зокрема, на пресконференції ведучий - шоумен та актор - Андрій Джеджула
повідомив, що ФК \enquote{Маестро} за підтримки благодійного фонду \enquote{Ми з України}
розпочинає благодійний концертно-спортивний тур \enquote{Ми з України!}. Артисти
завітають до багатьох міст, містечок і селищ України, аби влаштувати там свято
футболу і музики. За словами Джеджули, важливо це не відкладати на потім, коли
вже буде здобута перемога, потрібно допомагати прямо зараз. І це не проста
розважальна програма, адже головна мета туру - зібрати кошти у фонд підтримки
дітей та їх родин з прифронтової зони, що лишилися без даху над головою, речей,
іграшок та звичних речей. До зусиль ФК \enquote{Маестро} долучився відомий Благодійний
фонд \enquote{Ми з України}, який підтримує як героїчних воїнів України, так і всіх
цивільних, які постраждали внаслідок бойових дій.

\begin{leftbar}
  \begingroup
    \bfseries
\enquote{Всі ми живемо в Києві і бачимо, що коїться, що робить окупант, але це
відбувається не тільки в Києві, це відбувається щодня. Щодня гинуть
наші люди... В цих умовах особливої підтримки потребують наші діти, які
втрачають своє дитинство, свої домівки та свої сім'ї. Сьогодні всі діти
України потребують нашої допомоги}, - зауважив Андрій Джеджула.
  \endgroup
\end{leftbar}

\textbf{Читайте також:} 

\href{https://donbas24.news/news/v-ukrayini-zapuskayut-novu-programu-vidpocinku-ta-ozdorovlennya-dlya-ditei-xto-moze-skoristatisya}{В Україні запускають нову програму відпочинку та оздоровлення для дітей - хто може скористатися}

\ii{01_06_2023.stz.news.ua.donbas24.1.voin_zsu_evakuacia_rodyna_hlopchyk_bahmut.pic.4}

На пресконференції були присутні артисти ФК \enquote{Маестро}, що братимуть участь у
турі: заслужений артист України Петро Чорний, заслужений артист України Віталій
Борисюк та голова Благодійного фонду \enquote{Ми з України} Євген Дудко. Спеціальними
гостями на пресконференції були: голова Комітету з Національної премії України
імені Тараса Шевченка, міністр культури України (2014,2016−2019 р.) народний
артист України Євген Нищук та мер міста Бориспіль Володимир Борисенко. Кожен з
присутніх наголосив на важливості благодійного концертно-спортивного туру \enquote{Ми з
України!}. у підтримку постраждалих дітей України і закликали його підтримати.

\ii{01_06_2023.stz.news.ua.donbas24.1.voin_zsu_evakuacia_rodyna_hlopchyk_bahmut.pic.5.scr_video_presskonf}

Раніше Донбас24 розповідав, що
\href{https://donbas24.news/news/biloruski-soyuzniki-evakuyuvali-ukrayinskix-zaxisnikiv-z-pid-baxmuta-video}{білоруські
союзники} евакуювали українських захисників з-під Бахмута. Попри численні
обстріли захисникам вдалося забрати побратимів за 200 метрів від в'їзду в
місто-фортецю.

\emph{Ще більше новин та найактуальніша інформація про Донецьку та Луганську області
в нашому \href{https://t.me/donbas24}{телеграм-каналі Донбас24}.}

Фото: з архіву Донбас24
