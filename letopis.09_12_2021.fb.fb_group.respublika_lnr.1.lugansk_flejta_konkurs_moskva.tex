% vim: keymap=russian-jcukenwin
%%beginhead 
 
%%file 09_12_2021.fb.fb_group.respublika_lnr.1.lugansk_flejta_konkurs_moskva
%%parent 09_12_2021
 
%%url https://www.facebook.com/groups/respublikalnr/posts/933824413920017/
 
%%author_id fb_group.respublika_lnr,zimina_olesja
%%date 
 
%%tags devushka,donbass,flejta,konkurs,kultura,lnr,lugansk,muzyka,rossia
%%title Луганская флейтистка выступила на Международном конкурсе «Щелкунчик» в Москве
 
%%endhead 
 
\subsection{Луганская флейтистка выступила на Международном конкурсе «Щелкунчик» в Москве}
\label{sec:09_12_2021.fb.fb_group.respublika_lnr.1.lugansk_flejta_konkurs_moskva}
 
\Purl{https://www.facebook.com/groups/respublikalnr/posts/933824413920017/}
\ifcmt
 author_begin
   author_id fb_group.respublika_lnr,zimina_olesja
 author_end
\fi


В РАМКАХ ПРОГРАММЫ ИНТЕГРАЦИИ. Луганская флейтистка выступила на Международном
конкурсе «Щелкунчик» в Москве

Воспитанница Луганской музыкальной школы №2 Камилла Соболева стала дипломанткой
Международного телевизионного конкурса «Щелкунчик», который прошел в Москве с
29 ноября по 6 декабря. Об этом сообщили в Молодежном интеграционном центре
(МИЦ). 

\ii{09_12_2021.fb.fb_group.respublika_lnr.1.lugansk_flejta_konkurs_moskva.pic.1}

«С 29 ноября по 6 декабря в Москве состоялся XXII Международный телевизионный
конкурс «Щелкунчик», организатором которого выступил телеканал
«Россия-Культура», что предоставило огромный шанс конкурсантам выступить перед
многомиллионной аудиторией в прямом эфире. ЛНР на конкурсе представила
воспитанница Луганской музыкальной школы №2, флейтистка Камилла Соболева», –
говорится в сообщении.

В МИЦе рассказали, что в нынешнем году конкурс собрал 48 юных музыкантов,
которые выступили в номинациях «Струнные инструменты», «Фортепиано», «Духовые
инструменты» и «Ударные инструменты». 

\ii{09_12_2021.fb.fb_group.respublika_lnr.1.lugansk_flejta_konkurs_moskva.pic.2}

«Камилла Соболева получила высокие баллы в двух турах и звание дипломанта
конкурса. Жюри особенно отметило артистизм исполнения, умение «рассказать
историю» с помощью инструмента, а также высокое техническое мастерство юной
конкурсантки», – отметили в учреждении. 

Финал конкурса прошел на сцене Концертного зала имени Петра Чайковского в
сопровождении симфонического оркестра. 

«Я была рада поучаствовать в конкурсе такого масштаба, так как там собрались
ребята из России и зарубежных стран – Великобритании, Индии, Канады. Со многими
мне удалось познакомиться и чему-то научиться у них. В составе жюри были
довольно известные исполнители, это большая честь для меня – выступить перед
ними и выслушать мнение о моем исполнении. Все это дало мне больше стремления
для дальнейшей работы. Оказаться на такой сцене – мечта любого музыканта», –
поделилась впечатлениями участница конкурса. 

Поездка на конкурс был организована для участницы из ЛНР при поддержке
Министерства культуры, спорта и молодежи (МКСМ) в рамках интеграционных
мероприятий с РФ.
