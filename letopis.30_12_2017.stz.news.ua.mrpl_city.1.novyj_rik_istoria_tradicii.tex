% vim: keymap=russian-jcukenwin
%%beginhead 
 
%%file 30_12_2017.stz.news.ua.mrpl_city.1.novyj_rik_istoria_tradicii
%%parent 30_12_2017
 
%%url https://mrpl.city/blogs/view/novij-rik-istoriya-i-traditsii-svyatkuvannya
 
%%author_id demidko_olga.mariupol,news.ua.mrpl_city
%%date 
 
%%tags 
%%title Новий рік: історія і традиції святкування
 
%%endhead 
 
\subsection{Новий рік: історія і традиції святкування}
\label{sec:30_12_2017.stz.news.ua.mrpl_city.1.novyj_rik_istoria_tradicii}
 
\Purl{https://mrpl.city/blogs/view/novij-rik-istoriya-i-traditsii-svyatkuvannya}
\ifcmt
 author_begin
   author_id demidko_olga.mariupol,news.ua.mrpl_city
 author_end
\fi

Новий рік – найулюбленіше свято для мільйонів людей з різних країн. Це один з
небагатьох днів у році, коли люди майже з усього світу стежать за годинником,
п'ють шампанське і радіють новому року. Секрет такої популярності дуже простий:
новорічна ніч – той час, коли навіть дорослим дозволено вірити в дива. Цей
\enquote{дозвіл} виник у давні часи, адже Новий рік вважається одним з найперших свят
всього людства. Найбільш ранні документальні свідчення відносяться до третього
тисячоліття до нашої ери, адже точно відомо, що Новий рік відзначали навіть у
Месопотамії. Проте історики вважають, що свято має більш давню історію, а
значить, нашим новорічним традиціям щонайменше 5000 років.

Вражає той факт, що Новий рік, у тому вигляді, який є зараз, почали святкувати
ще в Стародавньому Єгипті. Протягом століть єгиптяни святкували вересневий
розлив річки Ніл, який знаменував собою початок нового посівного сезону і був
життєво важливою подією. Вже тоді було заведено влаштовувати нічні святкування
з танцями та музикою, дарувати один одному подарунки.

1 січня став першим днем Нового року при Юлії Цезарі: у нововведеному календарі
цей місяць був названий на честь бога Януса, одна голова якого дивиться в
минуле, а інша в майбутнє. Вважається, що саме тоді виник звичай прикрашати
будинки.

Однак у всьому світі Новий рік ще багато століть святкували або на початку
весни, або в кінці осені – відповідно до сільськогосподарських циклів. На Русі,
наприклад, до XV століття початок року святкували 1 березня.

У 1600 році свято перенесли на осінь, а ще через сто років, приблизно в той
самий час, що і по всій Європі, Петро I видав наказ про загальне святкування
Нового року 1 січня. Також він наказав влаштовувати в цей день феєрверки та
народні гуляння.

Існує легенда, що коли Англія переходила на січневе відзначення Нового року,
збунтувалися жінки королівства: вони вважали несправедливим намір уряду додати
кожній з них по кілька місяців віку. Чоловіки посміялися, але рішення не
змінили.

У всьому християнському світі Новий рік завжди вважався другорядним святом,
більш головною подією вважалося Різдво. Саме тому жителі більшості європейських
країн дарують один одному подарунки 25 грудня, та й сімейні обіди влаштовують у
Святвечір.

Так було і в Україні, але в епоху Радянського Союзу відзначати церковні події
було заборонено, і Новий рік швидко став найважливішим і найулюбленішим з усіх
офіційних свят.

Жителі ж більшості західних країн вже 2 січня виходять на роботу. А ті, хто
відзначає новий рік за місячним або суто національним календарем - китайці,
японці, іудеї, і зовсім в ці дні не відпочивають. 1 січня дорослі
відправляються в офіс, а діти – в школу.

Спробуймо перенестися в минуле і дізнатися, як маріупольці святкували Новий рік
раніше. Цікаво, що маріупольцям наприкінці XIX – початку XX ст. новорічну ніч
пропонували урочисто провести в громадських місцях. Наприклад, у 1912 р.
святкування Нового року з сім'єю можна було провести в залі Комерційних зборів.
Увійти дозволялося завдяки попередньому запису, оплата становила 1 р. 10
копійок. Кожен гість мав право пройти тільки з однією жінкою. Члени зборів і
студенти проходили безкоштовно. Схожі умови відвідування сімейного вечора були
в залі Маріупольських громадських зборів, розташованій у будинку Д. Хараджаєва.
Заможні маріупольці могли дозволити собі урочисто зустріти Новий рік у Великому
концертному залі готелю \enquote{Континенталь}, розташованому на вул. Харлампіївській.
Вхід на вечір був виключно за попереднім записом. Для публіки грав струнний
оркестр. На вечорі дамам безкоштовно роздавали бутоньєрки, а чоловікам –
сигари. Святкова вечеря на 1 персону коштувала 2 р. 50 коп. Для спостереження
за її приготуванням з Харкова спеціально був виписаний шеф-кухар Т. Ровенський.
Найбільшою популярністю серед маріупольців користувалися страви французької
кухні.

Якщо у нас була б можливість здійснити навколосвітню подорож і дізнатися, як
проходять святкування Нового року в різних країнах, ми б були дуже здивовані,
адже кожен народ має свої традиції й унікальні особливості святкування.
Розглянемо найбільш цікаві традиції. Так, французи запікають у новорічний пиріг
біб: той, кому він дістанеться, отримує титул \enquote{бобового короля} і право
роздавати вказівки протягом всієї святкової ночі. Дорослі намагаються вгадати
так, щоб біб дістався дитині.

У Болгарії заведено влаштовувати новорічні дитячі гуляння. Хлопці роблять
палички з кизилу, прикрашають їх червоною ниточкою, голівкою часнику, горіхами,
монетками та сухофруктами й бігають з ними по окрузі. Вони заходять в будинки
до сусідів і \enquote{стукають} паличками по спинах господарів: вважається, що таке
поплескування приносить людині удачу, здоров'я і добробут.

Головний герой новорічного карнавалу в Колумбії – це Старий рік. Він розгулює
на високих ходулях по вулицях і розповідає дітям смішні історії.

У Норвегії діти чекають подарунків від кози. Тому в ніч перед Новим роком вони
готують їй частування, залишаючи в своїх черевиках трохи сіна. На ранок замість
сухої трави вони знаходять подарунки.

На Кубі перед настанням Нового року діти наповнюють глечики, відра, тази та
миски водою, щоб опівночі разом з батьками вилити цю воду з вікон. Вважається,
що таким чином люди дякують за все найкраще, що приніс минулий рік.

Їжа – важлива частина новорічного святкування в Мексиці. Так, рівно опівночі
кожна дитина повинна отримати та з'їсти велику пряникову ляльку.

Новий рік – свято, яке об'єднує всіх людей, всю планету. І попри різні
традиції,  у новорічну ніч всі люди однаково щасливі! Сподіваюся, що кожен день
Нового 2018 року буде яскравим і барвистим, як іграшки на новорічній ялинці, а
майбутні місяці – не менше радісними, ніж зустріч Нового року! Всіх вітаю з
прийдешнім святом і бажаю здійснення найзаповітніших мрій!

\clearpage
