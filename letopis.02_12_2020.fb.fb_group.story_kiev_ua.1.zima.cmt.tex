% vim: keymap=russian-jcukenwin
%%beginhead 
 
%%file 02_12_2020.fb.fb_group.story_kiev_ua.1.zima.cmt
%%parent 02_12_2020.fb.fb_group.story_kiev_ua.1.zima
 
%%url 
 
%%author_id 
%%date 
 
%%tags 
%%title 
 
%%endhead 
\subsubsection{Коментарі}

\begin{itemize} % {
\iusr{Alex Guardians}
В наших реалиях о многоснежной зиме как-то даже думать не хочется...  @igg{fbicon.frown} 

\iusr{Василь Кульбеда}
МИКОЛА Федорович !!

\begin{itemize} % {
\iusr{Ирина Петрова}
\textbf{Василь Кульбеда} звісно, той дурнуватий Т9... якщо не вгледіш - так свиню і підсуне(((( звісно Микола Федорович!

\iusr{Igor Popell}
\textbf{Ирина Петрова} з Фнафанчиком? Чи не так звали його песика? Хоча Фанєчка може бути пестливим від Фанфан. ))

\iusr{Василь Кульбеда}
 @igg{fbicon.thumb.up.yellow} 

\iusr{Ирина Петрова}
\textbf{Igor Popell} 

так, песа звали Фанфан @igg{fbicon.heart.eyes}  але Микола Федорович його Фанєй кликав. Іде він , Фаня
поряд. Хтось питає : "0й, какая хорошенькая собачка! А как ее зовут?", Микола
Федорович так важно: "Це не собачка, це ПЕС! І звуть його Фанфан!!! Фаня,
пішли!" @igg{fbicon.heart.eyes}  На жаль, я не дуже багато пам'ятаю таких зустрічей, бо поняття того,
ХТО був тоді ось так, поряд, у скверику прийшли вже аж надто пізно....

\iusr{Светлана Дубински}

\ifcmt
  ig https://scontent-frx5-2.xx.fbcdn.net/v/t39.1997-6/s168x128/70056798_3066979713344706_4845512799754911744_n.png?_nc_cat=1&ccb=1-5&_nc_sid=ac3552&_nc_ohc=j1vFrRm4jwYAX8G_jNN&tn=lCYVFeHcTIAFcAzi&_nc_ht=scontent-frx5-2.xx&oh=3d97d6948d805d76eaecedd32194d48f&oe=619486EA
  @width 0.1
\fi

\end{itemize} % }

\iusr{Наталия Ковалева}
Жаль, что опять дождь!

\iusr{Ия Рудзицкая}

Добрый день, землячка! вас так долго не было, что я уже начала вас
разыскивать, спрашивала о вас у членов клуба, вижу, что все в порядке, может,
просто вы меня забанили, тогда сорри!

\begin{itemize} % {
\iusr{Ирина Петрова}
\textbf{Ия Рудзицкая} 

Дорогая Ия, как огорчительно, что доставила Вам хлопот. Конечно же нет! Сто
тысяч раз нет, как я могу Вас забанить , что Вы!!! @igg{fbicon.heart.eyes} ) Просто был перерыв из-за
некоторых причин, не суть! Я очень рада Вас видеть, здравия и благополучия Вам
с зимой вместе! Поклон сыну!

\iusr{Ия Рудзицкая}
\textbf{Ирина Петрова} спасибо за добрые слова!
\end{itemize} % }

\iusr{Людмила Гаврилюк}

\ifcmt
  ig https://scontent-frx5-2.xx.fbcdn.net/v/t39.1997-6/s168x128/70184899_3066974596678551_6608125351778320384_n.png?_nc_cat=1&ccb=1-5&_nc_sid=ac3552&_nc_ohc=ms3h1X8d5-QAX9uR_hR&tn=lCYVFeHcTIAFcAzi&_nc_ht=scontent-frx5-2.xx&oh=f9dad5d226201bb7fac38f43b9407643&oe=61959AB3
  @width 0.1
\fi

\iusr{Елена Захарова-Горянская}

Чудово. Картинка з мого життя. А ще зіпсоване пальто на ракеті, і верби біля
Франка, і Соловцова театр, як бабуся казала. І Мерінгівська, і сходинки в
молочному, і наша Сберкаса де вже в 86 мій син втік з каляски, коли шукали всі
перехожі, а знайшовся в магазині з поріжком на розі Лютеранської, бо чекав коли
я прийду.

\iusr{Ирина Петрова}

Так, мої бабуня та мама казали Мерінгівська) але, мені чомусь тоді, в
дитинстві, ввижався страшний мерін, один з тих, які неслись та топтали героїню
фільму "Вірні друзі". А "Заньковецької" - так дзвінко лунало, якось радісно)
та ще й Актриса!!!! То ж була МРІЯ!)))

\begin{itemize} % {
\iusr{Елена Захарова-Горянская}
\textbf{Ирина Петрова} 

у мене теж Мерінгівська незрозуміло - трохи страшне слово, але вулицю дуже
любили, а ще наш хлібний на розі, і аптека. чудові були аптекарки, навіть в
часи тотального дефіциту допомагали дитині з ліками.

\begin{itemize} % {
\iusr{Ирина Петрова}
\textbf{Елена Захарова-Горянская} да, хлебный со смешными щипцами на верёвке, деревянными полками и разделительным барьером) и аптека, которая казалась просто дворцом с деревянными резными стенами, рецептурный - слово-то какое замечательное))( кассирша с боевой раскраской ацтеков, колоритная до невозможности  @igg{fbicon.laugh.rolling.floor}  и пробники духов доя мамы за 24 копейки))) теперь в помещении магазин Dior, как раз вот только из театра шла мимо @igg{fbicon.face.happy.two.hands} 

\iusr{Maryna Chemerys}
\textbf{Ирина Петрова} Я вчилась у 117 школі, на перерві бігали до хлібного, щипці пам'ятаю!
\end{itemize} % }

\iusr{Люда Иванченко}
\textbf{Ирина Петрова} 

У меня дома родители тоже говорили: "Надо пойти на меринговскую". Там был
чудесный молочный магазин с круглыми творожками в шоколаде. Теперь таких нет.

\begin{itemize} % {
\iusr{Maryna Chemerys}
\textbf{Люда Иванченко} Спогад дитинства - круглі сирки в шоколаді!
\end{itemize} % }

\iusr{Ирина Петрова}
\textbf{Люда Иванченко} 

да, помню сырки в глазури, круглые, и бумажка так сложена, синим написано.
Сейчас глазированных сырков много, про вкусы судить не могу, честно, ни в
детстве, ни сейчас они как-то мимо)

А молочный - это материал для целого романа)))) как нибудь соберусь, накропать
зарисовку)))

\iusr{Alik Perlov}
\textbf{Ирина Петрова} и мои ходили в аптеку на Меринговскую ))))

\iusr{Maksim Pestun}
и мы

\iusr{Ирина Петрова}
\textbf{Maryna Chemerys} о! Конечно, все мы из одной стаи!

\end{itemize} % }

\iusr{Dennis Yurichev}
Фотка заснеженного памятника на лавочке хороша!

\begin{itemize} % {
\iusr{Ирина Петрова}
\textbf{Dennis Yu} да, так атмосферно получилось)
\end{itemize} % }

\iusr{Tatyana Pysana}
Спасибо, чудные фоточки

\ifcmt
  ig https://scontent-frt3-1.xx.fbcdn.net/v/t39.1997-6/p370x247/67892435_1316719561810070_4968934826808705024_n.png?_nc_cat=104&ccb=1-5&_nc_sid=0572db&_nc_ohc=8aKgk61Ub4AAX9RSAg1&_nc_ht=scontent-frt3-1.xx&oh=2df0ca0ec3f92b46eaf73db500fff447&oe=6194EF4C
  @width 0.3
\fi

\end{itemize} % }
