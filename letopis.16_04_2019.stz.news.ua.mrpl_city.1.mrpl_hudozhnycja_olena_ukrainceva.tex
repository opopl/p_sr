% vim: keymap=russian-jcukenwin
%%beginhead 
 
%%file 16_04_2019.stz.news.ua.mrpl_city.1.mrpl_hudozhnycja_olena_ukrainceva
%%parent 16_04_2019
 
%%url https://mrpl.city/blogs/view/mariupolska-hudozhnitsya-olena-ukraintseva-1
 
%%author_id demidko_olga.mariupol,news.ua.mrpl_city
%%date 
 
%%tags 
%%title Маріупольська художниця Олена Украінцева
 
%%endhead 
 
\subsection{Маріупольська художниця Олена Украінцева}
\label{sec:16_04_2019.stz.news.ua.mrpl_city.1.mrpl_hudozhnycja_olena_ukrainceva}
 
\Purl{https://mrpl.city/blogs/view/mariupolska-hudozhnitsya-olena-ukraintseva-1}
\ifcmt
 author_begin
   author_id demidko_olga.mariupol,news.ua.mrpl_city
 author_end
\fi

\ii{16_04_2019.stz.news.ua.mrpl_city.1.mrpl_hudozhnycja_olena_ukrainceva.pic.1}

Роботи \textbf{Олени Украінцевої}, талановитої маріупольської художниці, члена
Національної спілки художників України, відрізняються особливою самобутністю
створюваних образів. Завдяки власній енергійності та неабиякій завзятості вона
невтомно відкриває Маріуполь для українських та зарубіжних міст з кращих
сторін. Художниці вдалося реалізувати багато важливих і корисних проектів, які,
безумовно, посприяли примноженню і подальшому розвитку художнього мистецтва
Маріуполя.

Корінна маріупольчанка, народилася в сім'ї моряка. Оскільки тато дуже сумував
за родиною, він вирішив змінити професію і почав працювати у залізничному депо
– ремонтував вагони. Весь парк пасажирських вагонів станції Маріуполь пройшов
через його руки. Мама працювала кухарем. Цікаво, що Олена народилася в
пологовому будинку № 3, саме там вона й сама народила двох своїх синів.

\ii{16_04_2019.stz.news.ua.mrpl_city.1.mrpl_hudozhnycja_olena_ukrainceva.pic.2}

У дитинстві художниця любила малювати разом з сестрою. Раніше Олена думала, що
так було у всіх дітей – коли набридали ляльки і не було мультиків по
телебаченню, або коли друзі були зайняті, то на допомогу приходили альбом,
олівці чи фломастери, чи зошит і ручка, або хоча б один аркуш паперу і зламаний
олівець. Саме тоді перед дівчинкою самі собою з'являлися захоплюючі історії,
події і цікаві образи. Її зачаровували мальовані мультфільми – Мауглі,
Попелюшка, пригоди пінгвіна Лоло. Потім, коли вона побачила диснеївських
Красуню і чудовисько, Попелюшку, Аладдіна, то вирішила, що хоче стати
художником-мультиплікатором.

\textbf{Читайте також:} \emph{Мариуполец в команде с MRPL.CITY издал книгу об осуществлении мечты}%
\footnote{Мариуполец в команде с MRPL.CITY издал книгу об осуществлении мечты, Ганна Хіжнікова, mrpl.city, 11.04.2019, \par%
\url{https://mrpl.city/news/view/mariupolets-v-komande-s-mrplcity-izdal-knigu-ob-osushhestvlenii-mechty-foto}
}

Олена закінчила з відзнакою художню школу ім. А. І. Куїнджі, але коли прийшов
час вступати до \enquote{вишу}, на сімейній нараді вирішили, що потрібно отримати
\enquote{нормальну професію} і вона вступила до ПДТУ.  Закінчила інженерно-педагогічний
факультет, отримавши спеціальність педагога - економіста. Потім працювала за
фахом, вийшла заміж. У тридцять з хвостиком, залишившись без роботи, вона
зрозуміла, що може стати художником: або зараз, - або ніколи!  Вирішила
вступати до Харкова, адже коли вчилася в художній школі, про харківський вуз
ходили легенди.

Академія справила на Олену величезне враження – відразу і назавжди. Вона вперше
в житті пошкодувала про те, що це не сталося раніше, зрозуміла, що слід було
йти вчитися після школи на денне відділення. І, незважаючи на те, що на заочній
формі навчання не вийшло такого занурення у творче середовище, чотири роки на
найбільш \enquote{художній} спеціальності \enquote{Станковий живопис} не
пройшли даром. З другого курсу вона стала постійною учасницею Всеукраїнських
виставок і через рік після закінчення навчання набрала достатню їхню кількість,
щоб стати членом Національної спілки художників України.

\ii{16_04_2019.stz.news.ua.mrpl_city.1.mrpl_hudozhnycja_olena_ukrainceva.pic.3}

Сьогодні Олена Украінцева має достатній досвід участі у загальноміських
проектах, завдяки яким Маріуполь відкрили для себе багато українських та
зарубіжних міст. Найпершим і наймасштабнішим проектом стала пересувна виставка
живопису і графіки \textbf{\enquote{Маріуполь на межі}} (2015 рік), яка розповідає про реалії
життя в Маріуполі і його передмісті. Головною напарницею і колегою Олени стала
донецька (вже – маріупольська) художниця \textbf{Ганна Торкаєнко}. На експозиції
маріупольчанки привезли понад два десятки картин.

\ii{16_04_2019.stz.news.ua.mrpl_city.1.mrpl_hudozhnycja_olena_ukrainceva.pic.4}

На полотнах можна було побачити і мирні приазовські пейзажі, і портрети бійців,
і \enquote{натюрморти} з берцями і бронежилетами. З кожним новим відвіданим містом,
виставка змінювалася, організаторки знайомилися з новими людьми, і проект
розростався й доповнювався. Художниці співпрацювали з організаторами проекту \textbf{\enquote{У
нас єдина доля – ім'я їй Україна}}, зокрема, його керівником \textbf{Петром Грицюком}. На
відкритті виставок співали барди і місцеві ансамблі, читали вірші маріупольська
поетеса \textbf{Оксана Стоміна} та поети і письменники Сходу України і Криму, які
написали книгу \textbf{\enquote{По живому}}. До речі, Олена Українцева та Ганна Торкаєнко стали
одними з ілюстраторів цього видання і презентували його на відкриттях. У
складні для України роки художницям хотілось розповісти в інших містах про
Маріуполь, про те, що тут живуть гарні, відкриті, гідні люди, які люблять і
вболівають за Україну. Виставка проїхала 13 великих і малих українських міст та
завершилася експозицією у Верховній Раді України.

\ii{16_04_2019.stz.news.ua.mrpl_city.1.mrpl_hudozhnycja_olena_ukrainceva.pic.5}

Наступним став проект \textbf{\enquote{Україна сучасна і традиційна}} – фестиваль української
культури в Литві. Його сумісно з україно-литов\hyp{}ською організацією \textbf{\enquote{Шлях до мрії}}
та \textbf{Інгою Швендих} реалізовували з українського боку чотири мисткині – дві
художниці Олена Українцева та Ганна Торкаєнко, майстриня-мотанкарка \textbf{Яна
Шинкаренко} і фотограф-кореспондент-оператор \textbf{Аліна Понома\hyp{}рьова}. Вони привезли в
Литву сім виставок. Це було вісімнадцять валіз на чотирьох. Навіть складно
уявити як чотири тендітні жінки подужали стільки валіз, переїжджаючи з
Маріуполя на північ Литви через два кордони. Як зауважує Олена: \emph{\enquote{напевно,
вдруге такий подвиг зробити не наважимося...}}. У Литві крім виставок вони
проводили майстер-класи з традиційних ремесел, лекції-екскурсії експозиціями,
презентували книгу \enquote{По живому}, яка на той час була частково переведена
литовською.

\medskip
\ii{insert.read_also.demidko.stomina}
\medskip

\ii{16_04_2019.stz.news.ua.mrpl_city.1.mrpl_hudozhnycja_olena_ukrainceva.pic.6}

Ще одним великим проектом стала виставка \textbf{\enquote{ART WAR}}, яка пройшла в німецькому
місті Зіген за запрошенням професора університету Зігена Фолкера Вульфа. У
цьому проекті Олена вже була не організатором, а виступала учасником і, час від
часу, перекладачем. Організатором була Оксана Стоміна. Брали участь знову
чотири маріупольця, задіяні в створенні книги \enquote{По живому}. Для презентації її в
Німеччині книга була перекладена англійською.

Потім литовські партнери реалізовували освітній проект \textbf{\enquote{Крок у майбутнє}}, в
якому художниця взяла участь як педагог. Близько двох місяців вона викладала, а
інші учасники вивчали різні види традиційного прикладного мистецтва разом із
маріупольськими жінками. Після їхніх занять фахівець з Литви читала лекції про
те, як твори декоративного мистецтва можна продавати через інтернет.

Далі була Барселона і проект \textbf{\enquote{Дерево життя}}. Це була невелика пересувна художня
виставка про взаємодію сучасного популярного мистецтва та українських традицій
і культури. Брали участь три художниці з Приазов'я, серед них і наша героїня.

Крім того художниця періодично бере участь в різних загальноміських та
загальноукраїнських виставках і фестивалях.

\ii{16_04_2019.stz.news.ua.mrpl_city.1.mrpl_hudozhnycja_olena_ukrainceva.pic.7}

У Олени багато однодумців, адже на кожен проект знаходяться теми близькі різним
людям. Однак практично постійною напарницею залишається \textbf{Ганна Торкаєнко}. Наразі
вони розпочали роботу ще над одним проектом \textbf{\enquote{Пост міст}} – виставка живопису і
графіки, присвячена красі індустріального міста. Проект реалізуватиметься на
базі громадської організації \enquote{Ліга вільних художників \enquote{ARS ALTERA}}. Олена
виступає менеджером проекту. Завдяки виграному малому гранту буде отримано
фінансування від міської ради. Презентація проекту відбудеться вже влітку.

\ii{16_04_2019.stz.news.ua.mrpl_city.1.mrpl_hudozhnycja_olena_ukrainceva.pic.8}

Нашу героїню підтримують друзі і сім'я. Її сини пишаються тим, що мама у них
художник і що іноді її можна побачити по телебаченню, чи прочитати про неї у
газетах. Для Олени дуже важливо, що хлопці слідкують за її успіхами. Художниця
обожнює подорожувати, мріє побачити фіорди. Надихають нашу героїню чистий
аркуш, нові фарби і пензлі, цікава тема або завдання, ідея. Все інше – вже
пошук реалізації задуманого.

Маріуполь Олена любила завжди. Як люблять родича, чи близьку людину. Коли бачиш
її недоліки, і за це трохи соромно, коли думаєш, як її змінити і розумієш, що
починати потрібно з себе... У Маріуполі у неї є одне особливе місце, навіть
\enquote{улюблена точка}, звідки потрібно дивитися на це місце. Знаходиться вона біля
входу до студентської їдальні ПДТУ. Це кут університетського кварталу. 

\begin{quote}
\enquote{\emph{Раніше
там росло величезне розлоге дерево,} - розповідає Олена, - \em але зараз його
спиляли і навіть пня не залишилося. Але поруч є ще одне дерево і незабаром воно
наздожене попереднє. Є ще час, коли треба дивитися. Середина квітня. Десь
половина четвертого дня (15-30 опівдні). Ясна погода. Тоді сонце фарбує дерево
в такі яскраві, гарячі кольори, а будинок за деревом неймовірно живий і
справжній...}.
\end{quote}

З роботами маріупольської майстрині можна ознайомитися на сторінці в ФБ або
(частково) на стіні в Порт Сіті.

\textbf{Читайте також:} \emph{Гімназія у маріупольській Вежі}%
\footnote{Гімназія у маріупольській Вежі, VEZHA, mrpl.city, 15.04.2019, \par%
\url{https://mrpl.city/blogs/view/gimnaziya-u-mariupolskij-vezhi}
}

\ii{16_04_2019.stz.news.ua.mrpl_city.1.mrpl_hudozhnycja_olena_ukrainceva.pic.9}

\textbf{Улюблена книга Олени Украінцевої:} \enquote{Віднесені вітром} Мар\hyp{}ґарет Мітчелл, улюблений письменник – Сергій Лук'яненко.

\textbf{Улюблений фільм:} \enquote{Far, Far away} з Томом Крузом и Кім Бейсінгер.

\textbf{Порада маріупольцям:} 

\begin{quote}
\em\enquote{Порадити можу тільки одне – діяти. Самостійно або долучатися до рухів,
близьких по духу. Якщо поки нічого не знаходиться або зміни, які хочеться
здійснити, занадто глобальні, слід починати з малого. А саме: не кидати сміття
на вулицях, сортувати його по можливості вдома, ходити в театр, на концерти,
виставки, прибирати за домашніми вихованцями на вулиці, дотримуватися правил
дорожнього руху... Це банальне і нудні речі, але якщо їх буде робити більшість,
то місто зміниться}.
\end{quote}

\ii{16_04_2019.stz.news.ua.mrpl_city.1.mrpl_hudozhnycja_olena_ukrainceva.pic.10}
