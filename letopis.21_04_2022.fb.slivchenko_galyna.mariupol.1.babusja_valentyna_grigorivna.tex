%%beginhead 
 
%%file 21_04_2022.fb.slivchenko_galyna.mariupol.1.babusja_valentyna_grigorivna
%%parent 21_04_2022
 
%%url https://www.facebook.com/halina.slivchenko/posts/pfbid045oQhyv1He72jkSYMHAhQAtapnDetLh1brZm4ktCLus2JMgUKcFeVdHSwekRBzWVl
 
%%author_id slivchenko_galyna.mariupol
%%date 21_04_2022
 
%%tags mariupol,mariupol.war,death,mariupol.people
%%title Сьогодні ми дізналися, що наша бабуся Валентина Григорівна померла
 
%%endhead 

\subsection{Сьогодні ми дізналися, що наша бабуся Валентина Григорівна померла}
\label{sec:21_04_2022.fb.slivchenko_galyna.mariupol.1.babusja_valentyna_grigorivna}

\Purl{https://www.facebook.com/halina.slivchenko/posts/pfbid045oQhyv1He72jkSYMHAhQAtapnDetLh1brZm4ktCLus2JMgUKcFeVdHSwekRBzWVl}
\ifcmt
 author_begin
   author_id slivchenko_galyna.mariupol
 author_end
\fi

Сьогодні ми дізналися, що наша бабуся Валентина Григорівна померла. Ми знаємо,
що померла вона після 01.04.2022 р., що залишалася в своїй напівзгорілій
квартирі, з вибитими вікнами та дверима та вже декілька днів не вставала з
ліжка. Ми не знаємо про причини її смерті, про дату і час її смерті, де вона
похована. Є лише припущення, засновані на страшному сні, та думкою, що сусіди
не залишили би тіло у квартирі. 

\ii{21_04_2022.fb.slivchenko_galyna.mariupol.1.babusja_valentyna_grigorivna.pic.1}

Наша бабуся народилася 01.11.1934 р. у місті Харкові, дитиною пережила другу
світову війну у Харкові, будинок її родини знищили радянські війська, коли
полишали Харків, щоб якомога менше дісталося німцям. Її батька посадили на 10
років без права повернутися у місто за антирадянські погляди. Майже всі родичі
були розстріляні через те, що родина на той час вважалася заможньою. Війну вона
пережила разом із матір'ю, яка пізніше поїхала на Далекий Схід росії. Бабуся
залишалася декілька років в Харкові фактично безпритульною дитиною, отримала
важкі фізичні та психологічні травми. Лише на початку 50-х вона змогла приїхати
до своєї матері до Владивостока. Влаштувалася на роботу до бібліотеки льотної
військової частини, там познайомилася з моїм дідусем (він був військовим
льотчиком) та вийшла за нього заміж. Мій батько народився в селі Чернігівка
Приморського краю росії (Чернігівка фактично була заснована депортованими
радянською владою українцями). Коли Хрущов оголосив демобілізацію, вся родина
вирушила у Маріуполь, де на той час жили батьки дідуся. Батько діда був
головним інженером заводу Азовсталь. 

%\ii{21_04_2022.fb.slivchenko_galyna.mariupol.1.babusja_valentyna_grigorivna.pic.2}

Наша бабуся була незвичайною людиною. Я питаю себе, що в ній було головне? І на
думку приходить тільки одне: доброта, та доброта, яка вроджена і яку не можливо
підробити чи виховати в собі. Мій батько перейняв від неї цю вроджену доброту.
Вона не була щасливою людиною, на жаль. Останні півтори місяці свого життя вона
провела в пеклі, яке влаштувала росія,  в самотності і, скоріш за все, в не
повному розумінні, що відбувається.

Бабушка, я все помню, все наши походы на море, шелковицу, ночёвки у тебя дома,
мороженное в кафе "Солнышко", гоголь-моголь и самый вкусный цикорий в мире,
твои смешные песни, твой юмор, я все помню..я тебя люблю, покойся с миром,
родная моя.  Прости, что меня не было рядом😔

\ii{21_04_2022.fb.slivchenko_galyna.mariupol.1.babusja_valentyna_grigorivna.pic.3}
%\ii{21_04_2022.fb.slivchenko_galyna.mariupol.1.babusja_valentyna_grigorivna.cmt}
