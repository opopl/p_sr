% vim: keymap=russian-jcukenwin
%%beginhead 
 
%%file slova.princip
%%parent slova
 
%%url 
 
%%author 
%%author_id 
%%author_url 
 
%%tags 
%%title 
 
%%endhead 
\chapter{Принцип}
\label{sec:slova.princip}

%%%cit
%%%cit_pic
%%%cit_text
Ни один \emph{принцип} в истории человечества не унес столько миллионов
человеческих жизней, как \enquote{\emph{принцип} священной территориальной
целостности}. Когда какая-либо часть страны желает отделиться или получить
автономию в силу тех или иных причин, то на ее пути встает это \emph{жуткий
принцип}.  В угоду этому \emph{принципу} все государства принудительно толкали
и продолжают толкать в топку войны массы своих затюканных и малообеспеченных
граждан, которые не смогли откупиться от военных чиновников, потому что
сознательно, будучи в здравом уме, отдавать свою жизнь за этот \emph{принцип}
просто невозможно.  Ведь этот \emph{принцип} чисто государственнический, то
есть бюрократический. А разве может здравомыслящий человек добровольно отдавать
свое здоровье и жизнь за бюрократию и за ее интересы?
%%%cit_comment
%%%cit_title
\citTitle{Мы живем в мире, когда размер перестал иметь значение}, 
Андрей Головачев, strana.ua, 13.06.2021
%%%endcit

%%%cit
%%%cit_pic
%%%cit_text
\emph{Принцип} территориальной целостности или по другому \enquote{принцип никого никуда не
пущать} равносилен принципу запрета на разводы супругов. Представьте что в
стране действует запрет на развод. Глупо, да? А ведь когда-то такие
антигуманные \emph{принципы} действовали. И принуждали супругов мучаться друг с другом
до конца дней своих. Врагу не пожелаешь! Сейчас такой запрет воспринимается как
анахронизм.  Когда то (надеюсь!) таким же антигуманным анахронизмом будет
признан и \emph{принцип} священной территориальной целостности. Скоро логически
обосновать его смысл будет уже невозможно. И чем раньше это произойдет, тем
больше жизней на планете будет сохранено
%%%cit_comment
%%%cit_title
\citTitle{Мы живем в мире, когда размер перестал иметь значение}, 
Андрей Головачев, strana.ua, 13.06.2021
%%%endcit

%%%cit
%%%cit_head
%%%cit_pic
%%%cit_text
Революция Достоинства состояла в том, чтобы повысить свое достоинство.
Изобрести, придумать, распознать такие \emph{принципы}, которые выше жизни, и
чтобы эти \emph{принципы} были не свойность как внутри нации или свойность по
языку, а чтобы они имели трансцендентный уровень, к человечеству, к Богу, к
гармонии с природой. Вот на таком уровне осмыслить достоинство, в общем то так,
как его осмысляют американцы, которые поставили свободу выше всего. Французы,
которые поставили свободу, освобождение и попытку осмысления в некотором
нигилизме. Как немцы, которые положили некие идеи, мощные концепты превыше
всего. Как британцы, которые кладут идею аристократической стратегии и
существования вот этой монархии, которая хранит позицию общего блага. Каждая из
них как-то нашла способ представить или предъявить это общее благо. Но не мы
%%%cit_comment
%%%cit_title
\citTitle{Сергей Дацюк: Украина сегодня - не просто попрошайка, она на мусорнике истории}, 
Сергей Дацюк; Людмила Немыря, hvylya.net, 28.06.2021
%%%endcit

%%%cit
%%%cit_head
%%%cit_pic
%%%cit_text
Відомі політики з \emph{принципами}, які змінювали світ: Авраам Лінкольн,
Уінстон Черчилль, Махатма Ганді, Мартін Лютер Кінг, Вацлав Гавел, Лех Валенса,
Голда Мейер, Маргарет Тетчер, Шимон Перес та інші. Це такий же механізм
включення волі та її впливу на соціальну реальність.  Вирішуючи слідувати
\emph{принципу}, ми вмикаємо волю.  Такий же механізм працює на груповому
рівні, на рівні колективної свідомості чи підсвідомості
%%%cit_comment
%%%cit_title
\citTitle{Чи можна примусити збірну України розмовляти державною мовою?}, 
Євген Лапін, www.pravda.com.ua, 06.07.2021
%%%endcit

%%%cit
%%%cit_head
%%%cit_pic
%%%cit_text
Победить врага, вражду и войну суть вернуться к мыслительным усилиям и
преодолеть упрощение мышления более сложным мышлением.  Безопасность взаимного
сложного мира строиться на иных \emph{принципах}, нежели безопасность вражды.
Безопасность суть плохое представление, ибо не содержит представления о
независимом устройстве мира, безотносительно к знаемой опасности.  Опасность —
это не то, что известно, а то, что неизвестно. Поэтому система безопасности не
поможет избежать вражды и войны, она может помочь минимизировать известные
угрозы.  Вместо безопасности нужно представление о порядке. В этом смысле
порядок может осуществлять превенцию (упреждение), противодействие и адаптацию
к известным угрозам
%%%cit_comment
%%%cit_title
\citTitle{Враг}, Сергей Дацюк, analytics.hvylya.net, 18.11.2021
%%%endcit
