%%beginhead 
 
%%file 16_01_2023.fb.sosnovskij_evgenij.mariupol.1.feisbuk_znovu_pita__
%%parent 16_01_2023
 
%%url https://www.facebook.com/evgeny.sosnovsky/posts/pfbid02Kd4bMUhhcgbj8FpNipJMUxDtQThb46TyJLGyWX5P6a9c4J8ijQ5BYGFDiWzs7pLhl
 
%%author_id sosnovskij_evgenij.mariupol
%%date 16_01_2023
 
%%tags mariupol
%%title Фейсбук знову питає, що у мене на думці...
 
%%endhead 

\subsection{Фейсбук знову питає, що у мене на думці...}
\label{sec:16_01_2023.fb.sosnovskij_evgenij.mariupol.1.feisbuk_znovu_pita__}

\Purl{https://www.facebook.com/evgeny.sosnovsky/posts/pfbid02Kd4bMUhhcgbj8FpNipJMUxDtQThb46TyJLGyWX5P6a9c4J8ijQ5BYGFDiWzs7pLhl}
\ifcmt
 author_begin
   author_id sosnovskij_evgenij.mariupol
 author_end
\fi

Фейсбук знову питає, що у мене на думці... А на думці саме те, про що написав
Денис Казанский:

"Російський теракт у Дніпрі дозволяє краще усвідомити масштаби звірств
російської армії у Маріуполі.

Просто вдумайтеся – там щодня було по кілька таких зруйнованих багатоповерхівок
з людьми. За даними окупантів, повністю знищено 620 багатоквартирних будинків.
Це Дніпро помножений на 600.

У 600 разів більше загиблих сімей, дітей, свійських тварин. У 600 разів більше
горя та жаху.

Хоча ні. У Маріуполі було набагато, набагато гірше. Бо там усі ці завали ніхто
не розбирав. І шансів вижити не було ні в кого взагалі. Пам'ятаєте відео з
Дніпра, де в перші хвилини після влучення ракети люди кричать з-під руїн, а на
вулицях затори з карет швидких допомог? Так ось, у Дніпрі живих людей з під
руїн дістали. А у Маріуполі такі люди повільно вмирали. Не було швидких, не
було рятувальників, бо їх також розбомбили.

На прикладі Дніпра бачимо, що кожен обвалений під'їзд – це десятки загиблих.
Руйнування одного будинку завдало стільки жаху та горя. А в Маріуполі – УСЕ
місто таке. І все це методично, безжально, місяцями. І ПРОСТО ТАК. Ні для
чого."

На цих моїх фото з "Маріупольського щоденника" - тисячна частина того, що було
в Маріуполі. Те, що було навкруги мене, в 100-200 метрах від нашого будинку,
те, що залишилось від нашого будинку. Ми ніколи не дізнаємось, скільки людей
залишилось під тими завалами і було вивезено як сміття, скільки людей згоріло
заживо в тих будинках, залишивши по собі тільки попіл... І дай Боже, щоб більше
ніде не повторився Маріуполь.
