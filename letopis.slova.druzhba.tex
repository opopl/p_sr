% vim: keymap=russian-jcukenwin
%%beginhead 
 
%%file slova.druzhba
%%parent slova
 
%%url 
 
%%author 
%%author_id 
%%author_url 
 
%%tags 
%%title 
 
%%endhead 
\chapter{Дружба}
\label{sec:slova.druzhba}

%%%cit
%%%cit_head
%%%cit_pic
%%%cit_text
У проросійському болоті велика заруба. Чубляться Портнов/Лукаш і Шарій. Ще
вчора вони були найліпшими \emph{друзями}, а сьогодні вже один одного тваринками
називають. Що ж сталося? Там всього намішано. І президентські амбіції
Портнова, і \enquote{туалетна історія} Шарія, і бажання їхніх учорашних спільних \emph{друзів}
вислужитись перед Зеленським, десь і Медведчук є на радарах
%%%cit_comment
%%%cit_title
\citTitle{Шарія кинули всі. Втече до Москви?}, Олексій Братущак, blogs.pravda.com.ua, 20.06.2021
%%%endcit

%%%cit
%%%cit_head
%%%cit_pic
\ifcmt
  tab_begin cols=2
		width 0.4
     pic https://strana.ua/img/forall/u/0/36/8870e29b30de20a65403b30c30b4b37560dc7956d45f7795881761.jpg
		width 0.46

     pic https://strana.ua/img/forall/u/0/36/2021-07-01_09h35_48.png
  tab_end
\fi
%%%cit_text
Подробности нападения. 29 июня на матч Украина-Швеция, который завершился
выходом украинской сборной в четвертьфинал Евро-2020, пришли не только
болельщики команд-соперниц, но и российский фанат Денис Смоленский вместе со
своими \emph{друзьями} - украинцем и поляком. Трое \emph{товарищей} выбрали
яркие образы, чтобы показать приверженность к разным странам, но в то же время
единство.  Россиянин выбрал одежду с российской символикой, которая отличала
его на \enquote{сине-желтом} секторе - он был в красной футболке сборной
России, с флагом РФ и в шапке-ушанке.  И, как видно по снимкам, в начале матча
украинские болельщики спокойно реагировали на болельщика, укутанного в триколор
%%%cit_comment
%%%cit_title
\citTitle{На Евро 2020 избили российского фаната Дениса Смоленского, что известно об инциденте}, 
Анна Копытько, strana.ua, 01.07.2021
%%%endcit

%%%cit
%%%cit_head
%%%cit_pic
\ifcmt
  pic https://strana.ua/img/forall/u/0/36/2021-07-03_14h11_09.png
	width 0.4
\fi
%%%cit_text
Главное, чтобы хорошо играли. Первые, кого мы встретили были две
\emph{подружки}, которые говорят на разных языках. Одна - на русском, другая -
на украинском. Они ответили, что для них вообще не принципиально, на каком
языке будут говорить на пресс-конференции украинские футболисты. Главное -
чтобы было понятно.  \enquote{Все равно, кому как удобно. Кто-то говорит на
украинском, кто-то - на русском.  Может, кто-то не знает украинский... А
вообще, вроде бы классно играли. Я видела в Инстаграме все истории были только
о футболе}, - получили мы первый ответ
%%%cit_comment
%%%cit_title
\citTitle{Что говорят украинцы о пресс-конференциях футболистов на русском языке. Опрос Страны}, 
Антонина Белоглазова, strana.ua, 03.07.2021
%%%endcit

%%%cit
%%%cit_head
%%%cit_pic
\ifcmt
  pic https://strana.ua/img/forall/u/0/36/2021-07-03_14h15_31.png
	width 0.4
\fi
%%%cit_text
Двое \emph{товарищей} также были солидарны с девушками и предложили в первую очередь
разобраться с теми, кто критикует украинских футболистов за язык.  При этом
ребята в повседневной жизни общаются на украинском языке.  \enquote{Как им удобно,
пусть так и говорят. В принципе много людей на русском языке говорят}, -
ответили нам. И добавили, что критику за язык в основном считают хайпом
%%%cit_comment
%%%cit_title
\citTitle{Что говорят украинцы о пресс-конференциях футболистов на русском языке. Опрос Страны}, 
Антонина Белоглазова, strana.ua, 03.07.2021
%%%endcit

%%%cit
%%%cit_head
%%%cit_pic
%%%cit_text
\enquote{Такі рішення стратегічно важливі, адже на сьомому році війни з Росією у нас
був пам'ятник \emph{дружби} з Москвою. Звісно, зараз у росіян і ватників бомбитиме,
про що свідчить навіть ейджизм і сексизм у депутата від ОПЖЗ, який таким чином
пробував зірвати розгляд цього рішення. Але такі рішення правильні і цього
декілька років безуспішно добивалась громада району}, - розповіла LB.ua
ініціаторка рішення, депутатка від фракції \enquote{Голос} Аліна Михайлова
%%%cit_comment
%%%cit_title
\citTitle{\enquote{Київрада вирішила демонтувати пам'ятник дружби Києва та Москви}}, 
Олександр Рудоманов, lb.ua, 08.07.2021
%%%endcit

%%%cit
%%%cit_head
%%%cit_pic
%%%cit_text
Одним из важнейших и ошибочных шагов украинской власти была денонсация Договора
о \emph{дружбе}, сотрудничестве и партнерстве между РФ и Украиной. Таким
образом, страна осталась фактически без работающего инструмента влияния и
юридически оформленных отношений с крупнейшим соседом.  Денонсация этого
договора в 2018 году укладывается в логику нагнетания антироссийской истерии,
милитаризации государственной политики. Однако для экономики Украины, для
позиции страны на международной арене, для интересов миллионов граждан такое
решение имело катастрофические последствия. Такой шаг свел на нет все усилия,
направленные на поиск компромиссов и мирное урегулирование конфликта на
Донбассе
%%%cit_comment
%%%cit_title
\citTitle{О будущем украинского и русского народов}, 
Виктор Медведчук, strana.ua, 15.07.2021
%%%endcit

%%%cit
%%%cit_head
%%%cit_pic
%%%cit_text
Доброго времени суток, \emph{друзья}. Украинцы – это русские говорят все,
начиная от Путина и заканчивая простым русским человеком. Трубят эту формулу
всюду, и я решил сказать свое мнение, которое имеет определенный алгоритм и
тональность.  Нет, украинцы – это не русские, иначе бы они называли себя
русскими и назывались бы Россией, а не Украиной. Да, на Украине много русских,
как раз коренные жители этой территории, остальные пришлые, именно укры от
слова окраина. Территория Украины – это Малороссия для России и окраина для
польско литовского княжества, поэтому укры не русские, это холопы польско
литовских князьков.  Проведу небольшое сравнение, чтобы понять разницу.
Чемпионат по футболу, Украина играет с Австрией, спрашивают российского
болельщика
%%%cit_comment
%%%cit_title
\citTitle{За что я люблю Малороссию (Украину)}, Вестник, zen.yandex.ru, 14.07.2021
%%%endcit

%%%cit
%%%cit_head
%%%cit_pic
\ifcmt
  pic https://img.strana.ua/img/article/3478/foto-mahuchikh-i-13_main.jpeg
  width 0.4
	caption Мария Ласицкене и Ярослава Магучих. Кадр из видео Новости сегодня 
\fi
%%%cit_text
На Олимпиаде в Токио украинская легкоатлетка Ярослава Магучих, завоевавшая для
Украины бронзовую медаль по прыжкам в высоту, сделала \emph{дружеское фото} с
чемпионкой из России Марией Ласицкене (занявшей 1 место).
Девушки обняли \emph{друг друга} и развернули за печами флаги своих стран -
"сине-желтый" и флаг олимпийского комитета РФ, под которым выступает Россия.
После этого националисты обвинили украинскую спортсменку в "зраде", а в
Министерстве обороны вызвали "на ковер" (Магучих - лейтенант ВСУ). И это не
первый раз, когда украинские власти подключаются к травле спортсменов, которые
на Олимпиаде демонстрируют приятельские отношения с россиянами - несмотря на 7
лет пропаганды вражды
%%%cit_comment
%%%cit_title
\citTitle{\enquote{Поддерживают дискурс вражды}. Как власти кошмарят олимпийскую призерку Магучих за фото с россиянкой}, 
Анна Копытько, strana.ua, 09.08.2021
%%%endcit

%%%cit
%%%cit_head
%%%cit_pic
\ifcmt
  pic https://avatars.mds.yandex.net/get-zen_doc/5231691/pub_617a5f15aa83b66f09805d56_617a611c6a98e21cd2c5fa9a/scale_1200
  @width 0.4
\fi
%%%cit_text
Что острословы сегодня пишут? Даже не матерятся, не брызгают слюной. Не над
кем. Просто устало измываются, строго в рамках русской классики: «Проказник
Коломойчик, Посол, Козёл и хриповатый Вовчик затеяли сыграть квартет...» Точнее
не скажешь, особенно зная финал басни: «вы, \emph{друзья}, как ни садитесь...»
%%%cit_comment
%%%cit_title
\citTitle{Такие смешные: украинская политика... через призму творчества Гоголя}, 
Исторические напёрстки, zen.yandex.ru, 28.10.2021
%%%endcit

%%%cit
%%%cit_head
%%%cit_pic
%%%cit_text
Пусті уми ще й сьогодні можуть допитуватися: чому не став Гоголь українським
письменником, творцем прози української так само могутньої, як поезія Шевченка?
Ніхто ніколи не відповість на такі запитання, бо вони в суті своїй безглузді й
позачасові. Можна посилатися на виховання, яке панувало тоді в дворянських
садибах, на саме дворянство Гоголя (все ж таки 400 кріпаків і тисяча десятин
землі), на культурні традиції (Полтавщина вже перед Гоголем дала для російської
літератури І. Хемніцера, М. Хераскова, І. Богдановича, В. Капніста, В.
Нарєжного, М. Гнєди-ча), на тяжіння до російської культури і до великого
братнього народу, традицію, що йшла ще від Хмельницького і Переяславської ради,
і від Мартина Пушкаря, який життям заплатив за вірність Переяславській раді,
коли зрадник Виговський став снувати свої підлі Гадяцькі пакти (а Пушкар був
полтавський полковник!). Слід також взяти до уваги і надзвичайну, буквально
нелюдську чутливість Гоголя. Ще не відаючи про своє майбуття, він мовби
передчував ту доброзичливість і підтримку майбутніх своїх великих \emph{друзів} у
Петербурзі — Пушкіна, Жуковського, Плетньова, Дельвіга і, звичайно ж,
Бєлінського
%%%cit_comment
%%%cit_title
\citTitle{Три долі. Гоголь, Шевченко, Чехов}, Павло Загребельний
%%%endcit

%%%cit
%%%cit_head
%%%cit_pic
%%%cit_text
— Послушайте, \emph{друг} мой, — сказал Снивли. — Существует огромная разница между
одиноким ученым, идущим в Дикие Земли из-за интеллектуального любопытства, и
агентурой церкви, вторгающейся туда с огнем и мечом. У ученого, возможно, был
бы даже шанс вернуться живым. Конечно, он не был бы там в полной безопасности.
В Диких Землях водятся такие обитатели, от которых лучше держаться подальше, но
ученого терпели бы, так как он не нес бы с собой опасности для тамошних
жителей, не нес с собой войну. Если бы его и убили, то убили бы тихо. И никто
даже не знал бы, когда и как это произошло. Вы видите теперь разницу?
%%%cit_comment
%%%cit_title
\citTitle{Зачарованное паломничество}, Клиффорд Саймак
%%%endcit

%%%cit
%%%cit_head
%%%cit_pic
%%%cit_text
Повернення до початків і витоків завжди вимагає деякого такту. Я не можу знову
стати собою тодішнім, але в мене по закамарках відлежуються деякі речові
докази.  Вони й дозволяють повідновлювати певні затерті фрагменти: щó ми мали,
як тепер кажуть, на виході. Чи то пак на вході.  1987 року я вже писав
"Екзотичні птахи і рослини", тобто якийсь перший десяток віршів тієї майбутньої
збірки я вже бувало читав уголос \emph{друзям}. Рукопис попередньої, "Середмістя", що
її я вважав цілком готовою, застряг тим часом на кілька довгих літ у
видавничому плані "Радянського письменника". Щохвилини він міг із того плану
вилетіти. А міг і дочекатися своєї черги – років за три-чотири. Це, зрештою, з
ним і трапилося: він дочекався. "Радянський письменник" був мініатюрним
радянським союзом, і господарка його підлягала планам, а плани, як відомо,
переважно п'ятирічні
%%%cit_comment
%%%cit_title
\citTitle{Українську культуру довелося витягати з дна і перестворювати з уламків}, 
Юрій Андрухович, gazeta.ua, 29.10.2021
%%%endcit

%%%cit
%%%cit_head
%%%cit_pic
\ifcmt
  tab_begin cols=3

     pic https://avatars.mds.yandex.net/get-zen_doc/58826/pub_6183704f17cb4e669a6e14c1_61837b2fe51e69549cca6ae9/scale_1200
     pic https://avatars.mds.yandex.net/get-zen_doc/4361269/pub_6183704f17cb4e669a6e14c1_61837267b411ca01e4415431/scale_1200
		 pic https://avatars.mds.yandex.net/get-zen_doc/4457971/pub_6183704f17cb4e669a6e14c1_6183899317cb4e669a7a919f/scale_1200

  tab_end
\fi
%%%cit_text
Открываешь утром ленту \emph{друзей} и новостей. В Анадыре минус −23°. А у тебя за
окошком +8°, накрапывает дождь. И грустит, сложив руки за спиной, вечно молодой
Михаил Лермонтов, мужик в пиджаке, он же памятник.  В Пскове уже потягушеньки.
В Калининграде еще сон золотой. На Камчатке, которая для меня теперь - семья, с
которой мы летом были соседями на морском берегу в Севастополе, уже ужинают.
Мать гоняет веселых мальчишек, к которым отец безумно снисходителен и добр.
Они живут в Елизово. Вот здесь.  Но вот уже несколько лет приезжают в Любимовку
покупаться в теплом море, с \emph{друзьями} под виноградом посидеть.  Татьяна
из Кисловодска на рассвете поставила для нас светлый лик той, что спасала
страну и озаряет верой наши души. Напоминает, какой у нас сегодня праздник.
А вчера я видела на ее странице снимок малыша, что выпал из гнезда.  Сверху его
удерживает мать, снизу помогает отец. Это ли не образ Родины?
%%%cit_comment
%%%cit_title
\citTitle{Люблю свою страну}, Наталия Ефимова, zen.yandex.ru, 04.11.2021
%%%endcit
