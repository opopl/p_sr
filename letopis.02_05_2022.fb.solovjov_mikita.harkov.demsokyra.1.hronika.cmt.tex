% vim: keymap=russian-jcukenwin
%%beginhead 
 
%%file 02_05_2022.fb.solovjov_mikita.harkov.demsokyra.1.hronika.cmt
%%parent 02_05_2022.fb.solovjov_mikita.harkov.demsokyra.1.hronika
 
%%url 
 
%%author_id 
%%date 
 
%%tags 
%%title 
 
%%endhead 
\zzSecCmt

\begin{itemize} % {
\iusr{Тетяна Уразова}
Метро! Дайте мне метро- и я вернусь!!!

\iusr{Ольга Никулина}

Для Харькова критично вывести людей из метро. Многие сидят там 2 месяца, есть
квартира,есть куда уехать но страх не даёт им выйти на улицу. Есть и другие
истории...

\iusr{Inna Bermant}
Спасибо вам огромное за посты, Микита!

\iusr{Natalya Yutin}

Люди отказываются уезжать, потому что снимать жильё на Западной сейчас очень
дорого, а жить в спортзале не хотят?

\begin{itemize} % {
\iusr{Микита Соловйов}
\textbf{Natalya Yutin} 

Есть немало вариантов поселения. Бесплатного или за коммуналку. Не хоромы, но и
далеко не спортзал. Причем в многих странах.

\iusr{Natalya Yutin}
\textbf{Микита Соловйов} но знают ли об этом люди? Круг общения моей мамы явно не в курсе

\iusr{Eva Er}
\textbf{Natalya Yutin} 

в центре и на западе в небольших городках и сёлах можно жить за коммунальные. И
при этом получать 2000 грн в месяц как внутренние переселенцы. Несколько моих
знакомых так сделали, теперь спят по ночам и ходят в магазин, а не ждут
волонтёров. Просто не надо во Львове искать жильё, понятно что там сейчас
яблоку некуда упасть

\iusr{Микита Соловйов}
\textbf{Natalya Yutin} 

Прости, но если круг общения твоей мамы более менее совпадает с районом жизни в
Харькове, то там о массовой эвакуации речь и не идет. Соответственно, волонтеры
кругами с бубном не скачут предлагая варианты. Но выехать в крупные села или
маленькие городки и просто даже самостоятельно уже на месте снять за 2-3 тысячи
в месяц максимум приличное жилье вполне реально. А можно и за коммуналку, но
это уже или нужна помощь, или как повезет.

\iusr{Александр Смирнов}
\textbf{Микита Соловйов} 

очень сомнительное заявление и говорю как участник событий внутри этих многих
вариантов - их нет и про бесплатные речи не идёт. Средняя цена для семьи за
квартиру около 100-300\$ с потолком до 1000\$/месяц. Могу конкретно прувы по
областям собрать. Не все на западе готовы волонтерить, большинство хотят
заработать

\iusr{Микита Соловйов}
\textbf{Александр Смирнов} О каких населенных пунктах идет речь? Если о больших городах, то наверное вы правы. Потолка вообще нет, это понятно. Но если говорить о крупных селах, о пгт, то там вполне хватает вариантов в той категории, что я назвал.

\iusr{Александр Смирнов}
\textbf{Микита Соловйов} 

конкретно Ивано-Франковскую область, где я - без разницы хоть это
Ивано-Франковск, хоть Коломыя, хоть Калуш (я в Калуше), хоть Долина (коллеги
там) - при этом, стоит отметить, что огромное количество жилья брошено, как
только появилась возможность уехать беженцем - местные побросали все и унеслись
в Европу (на то есть причины). Далее о тех что знаю - Черновцы и под ними,
Винница и пригород, Ровно и пригород, Умань и около, Ужгород и около, ну Львов
думаю даже упоминать не стоило, но написал.

\iusr{Микита Соловйов}
\textbf{Александр Смирнов} 

Так опять \enquote{пригород}. Это именно то, о чем я говорю. Да, если искать в
окрестности крупных городов, облцентров, то естественно жилье дороже. Если
заехать вглубь, то цены падают очень сильно. Грубо говоря. Если хочешь найти
дешевое жилье: ищи его в месте до 5 тыс. населения в радиусе больше 30 км от
любого населенного пункта, о котором хоть раз в жизни слышал, от крупных в 50.

И напоминаю. Мы все еще говорим не о трудовой, экономической или еще какой-то
миграции, а об эвакуации. О переезде условно из Сев. Салтовки, Пятихаток,
Слатино или Р. Лозовой.

\iusr{Микита Соловйов}

Я постоянно говорю об одном и том же. Если на это нет ресурса (сил, денег,
круга приятелей, любого другого), то нельзя к эвакуации подходить с тем же
набором критериев, что к переезду. Нет, в большинстве случаев будет идти речь о
жилье в населенном пункте, куда до войны никто бы по своей воле за такие деньги
не переехал бы. Но ключевым в этом предложении является словосочетание \enquote{до
войны}. Эвакуация это не переезд в более комфортное место. Это переезд в место,
куда не падают снаряды. И где можно худо-бедно себя прокормить, если речь идет
о трудоспособных людях. Кстати в ХАрькове это является проблемой для очень
многих. Доля неработающих среди трудоспособных даже в спокойных районах по моим
оценкам 3/4 или больше.

\iusr{Александр Смирнов}
\textbf{Микита Соловйов} я понимаю, но в тех местах, что вы говорите нет предложений, если что разве родственники не живут. Тут пример те же Валки у меня

\iusr{Микита Соловйов}
\textbf{Александр Смирнов} 

Серьезно нет предложений? Вот было бы больше свободного времени, предложил бы
пари. С серьезной ставкой, естественно, я ленивый. Пари простое. Я приезжаю в
рендомный населенный пункт по этим критериям и за сутки нахожу там жилье в
указанной ценовой категории. Очень простым методом. Сначала обхожу просто
магазины и спрашиваю там. Если вдруг не помогло (что крайне маловероятно), то
просто иду по домам подряд.

Еще раз. Это опять вопрос комфорта. В этот раз психологического. Естественно
проще заранее сидя дома через интернет или телефон найти варианты, выбрать, все
взвесить и потом решать. И это безусловно хороший выбор в мирной жизни. Но
сейчас не мирное время. И я говорю об эвакуации не из Киева.

\iusr{Александр Смирнов}
\textbf{Микита Соловйов} 

не буду говорить за каждое село, но в большинстве случаев ситуация не такая как
вы описываете - никто в село так не приезжает, потому что ехать наобум тяжело.
По своему опыту - мы ехали в Калуш 5 дней, так как в начале марта были заторы
на дорогах и проблемы с бензином. Только в начальной точке - в Полтаве, у нас
была квартира, которую однако же мы получили за счёт местного родственника,
потом были Черкассы, и там ничего не нашлось, удалось поселиться в общежитии.
Встряв в дикие пробки и выбившись из графика попали в Хмельник, и снова ничего
- через горсовет поселили в детский сад. Наконец-то доехали до Калуша, но пока
ехали, наше жильё уже заняли другие. В основном люди боятся пускать к себе, а
слыша русский, это позиция усиливается. Проведя 2 месяца на Франковщине мне
порой уже тяжело писать на русском, окружение формирует переход

\iusr{Eva Er}
\textbf{Александр Смирнов} 

у меня есть пример вот такого выезда, человек, которого я знаю лично, 70+ лет.
Без машины и без особых навыков пользования ФБ. Выехала поездом, в какой-то
небольшой городок как Никита советует. Взяла с собой подругу, чтоб не грустно
было. Помогали волонтёры, детей рядом не было. Всё получилось, жильё нашлось.
Как искали я не знаю, но могу спросить

\iusr{Александр Смирнов}
\textbf{Eva Er} 

ну это отлично если я не прав, но опыт показывает, что это не правда или вы не
в курсе всех деталей - я всегда не верю, пока не доказано обратное, когда
рассказ начинается с формулировкой человек, которого я знаю мне рассказал. Я
рассказал собственный опыт и опыт моих коллег.

\iusr{Eva Er}
\textbf{Natalya Yutin} 

ну, круг общения Вашей мамы может и не знать, но у мамы же есть Вы. По опыту
вывоза своих родителей и ещё нескольких человек я могу сказать, что
координировали всё взрослые дети, искали информацию, планировали, связывались с
волонтёрами, если нужно было. Вывезли даже тех, кто боялся выходить из
квартиры.

\iusr{Eva Er}
\textbf{Александр Смирнов} 

вы невнимательно читали. Не человек, которого я знаю рассказал, а человек,
которого я знаю лично, ВЫЕХАЛ.

\iusr{Александр Смирнов}
\textbf{Eva Er} 

да, я прочитал. Но кстати не так как Никита предлагает, потому что помогли
волонтеры, но не в этом суть. Повторюсь, мой опыт показывает обратное, возможно
ваш пример подтверждает правило, а ваш знакомый 70+ исключение. Ну и кстати
говоря, я говорю про категорию 25-50 лет, к людям 70+ отношение другое надеюсь,
так же есть у нас дополнительные факторы - дети, животные, больше одного
человека

\iusr{Eva Er}
\textbf{Александр Смирнов} 

я могу уточнить, как именно помогали волонтёры, и рассказать. До вокзала
возможно помогли добраться. Из категории 25-50 у меня есть пример друга,
который вывез семью (двое детей) и ещё есть мама с сыном знакомые. Все сейчас
живут по ту сторону Днепра, оплачивая коммунальные.

\iusr{Eva Er}

Кстати, там где эта знакомая 70+ поселилась, она снимает часть дома, а в другой
части - семья с детьми, тоже кажется из Харькова

\iusr{Tetyana Sfandex}
\textbf{Александр Смирнов} 

знайома прийшла до сільради невеликого містечка, сказала що з Харкова і
потребує житло і її безоплатно розмістили на турбазі з її великими собаками

\iusr{Tetyana Sfandex}
\textbf{Александр Смирнов} 

якщо справді люди «слыша русский» не хочуть пускати, чому б не питати
українською?

\iusr{Marianna Markova}
\textbf{Александр Смирнов} 

у меня на руках сейчас прямой контакт на поселение в Италии женщин с детьми. От
Черновцов трансфер сказали что обеспечивают, соцпакет обеспечивают, языковые
курсы и бебиситтера тоже пообещали.

неделю назад был большой список жилья по Полтавской области -- бесплатного (да,
дачные домики, да удобства могут быть во дворе).

Вчера в Вайбер вот такое сбросили:

РІВНЕНЩИНА

Вараш, Рівненська область

Будівельників

На будь-який термін Сім'ї з дітьми (чоловік, дружина, діти) ДітиТварини (коти,
собаки) Жінки

Місць: 4

БЕЗКОШТОВНО ВІЗЬМУ ДО СЕБЕ НА КВАРТИРУ (ПРОЖИВАННЯ З ВЛАСНИКОМ, ДВОХКІМНАТНА, Є
ВІЛЬНА СПАЛЬНЯ. ВСІ УМОВИ) ЖІНКУ АБО СІМ'Ю (ДОРОСЛИХ НЕ БІЛЬШЕ 2 - Х) ТІЛЬКИ З
ДІТКАМИ. ЖИТЛО В РІВНЕНСЬКІЙ ОБЛ. М. ВАРАШ. ДІТКИ НЕ СТАРШЕ 14 РОКІВ. ТІЛЬКИ
ДЛЯ БІЖЕНЦІВ, ЯКІ ДІЙСНО ПОТРЕБУЮТЬ ЖИТЛО.

ТЕЛ: 0977196558.

И группу, где публикуют предложения по жилью.

Есть сайт Прихисток, и еще кроме него, где публикуют объявления.

Да, там много с \enquote{удобства во дворе}, но єто не 8 человек в одной
комнате в общежитии (что тоже как временній вариант я считаю нормально)

\iusr{Александр Смирнов}
\textbf{Tetyana Sfandex} 

ну подібна ситуація у нас була в Хмільнику, де купа санаторіїв, проте
відмовилися пустити на ніч, навіть за гроші. Запропонували купити заїзд на 10
днів. В міськраді поселили в дитячому садку і спорт залі школи

\iusr{Александр Смирнов}
\textbf{Tetyana Sfandex} 

дехто нажаль просто не вміє, проте це вважаю це виправданням. Особисто зараз
спілкуюсь українською і комфортно себе відчуваю

\iusr{Микита Соловйов}
\textbf{Александр Смирнов} 

Прочел вот весь диалог. Простите, но вообще не понял позицию. Вам рассказывают
о личном опыте кучи людей, у которых получилось. Вы же не приводите опыт тех,
кто вот просто так и сделал. А рассказываете о том, что вам кажется, так
тяжело, сложно и не получится. Вам кидает Мариана перечень вариантов, которые
вот есть прямо сейчас. Бери, созванивайся, езжай. Вас и это не убеждает. Причем
через Харьков-СОС так жилье нашли десятки, если уже не сотни людей. Но нет. Их
опыт не релевантен. А что тогда убедительно, просто любопытно?

\end{itemize} % }

\iusr{Olena Pavlenko}
А какие-то магазины работают, кроме продуктовых?

\begin{itemize} % {
\iusr{Roman Frolov}
\textbf{Olena Pavlenko} 

Comfy работает, стройматериальі, автосервис, на базаре открьітьі некоторьіе
лотки с одеждой, инструментом. Открьітьі отдельі хозтоваров в Классе и Росте а
также в Восторге.

\iusr{Микита Соловйов}
\textbf{Olena Pavlenko} Очень мало.

\iusr{Елена Вас}
\textbf{Olena Pavlenko} на Холодной горе кроме супермаркетов открылись обувные, одежды, Ева, Простор, Красный мак
\end{itemize} % }

\iusr{Вебстер Хантер}

Монтери КСП «Харківміськліфт» попри небезпеку ворожих обстрілів щодня працюють,
аби забезпечити мешканців міста електропостачанням.

Зокрема, ремонтними роботами щодня зайнятий 37-річний Антон Білозор - старший
майстер КСП «Харківміськліфт» з технічного обслуговування внутрішньобудинкових
електромереж.

На підприємстві чоловік працює з вересня 2018 року, але у житлово-комунальній
сфері вже понад 9 років. У мирний час Антон Білозор займався лише розподілом
заявок жителів та контролював їх виконання. Але під час війни кількість
електротехнічного персоналу, що обслуговувала район, зменшилась вп'ятеро, тому
для якнайшвидшого відновлення електропостачання у поки що Московському районі
старшому майстру доводиться постійно допомагати своїм підлеглим.

За словами Антона Білозора, бригади працюють у перервах між бомбардуванням,
особливо Салтівського житлового масиву, та роблять усе можливе, щоб забезпечити
харків'ян життєвонеобхідними комунальними послугами.

\#обличчя\_Харкова вопрос а чому вони об'яву про наймання працівників не подають?

\iusr{Виталий Буняев}

Прохожу мимо метро регулярно. Люди выходят оттуда, греются. Вряд ли видят себя
со стороны. Представляю, во что превратились станции.

.... я думаю, метро запустят тогда, когда освободят Циркуны и Тишки. И
количество обстрелов еще сократится. Сейчас, несмотря на освобождение Р.
Лозовой, Момотова и Кутузовки, одиночные прилеты могут бы в половине города с
северо-востока. Не говоря о модернизированных РСЗО.

\iusr{Svitlana Potselueva}

Вспомните постьі А.Сененко, когда люди отказьівались вьіезжать из Бучи или
Ирпеня. И невозможно бьло уговорить, когда каждая минута стоила жизни.

\iusr{Roman Frolov}

В Харьков сегодня вернулись стрижи, летают кругами как всегда.

Также опять дефицит топлива, как и по всей стране - опять ввели ограничения по
10л на заправках.

\iusr{Oleksandr Shevchenko}

Обов'язково дай дані по отказниках. Це важливо для планування проектів.

\iusr{Елена Борс}

Влада міста та області - злочинці

Створивши зручний піар-майданчик у метро, вони тримають у заручниках людей
міста, не даючи змогу безпечно пересуватися!

Метро повинно робити як транспорт! Частково, але повинно робити!

І пункти видачі гуманітарки - злочин. В той час, як навіть ті, хто нікуди не
рухався зі своїх хат, не отримують обов'язкової допомоги

Мер та голова ОДА - злочинці

\begin{itemize} % {
\iusr{Andrey Bogdanovich}
\textbf{Елена Борс} 

пункти видачі гуманітарки - скупчення людей, що є потенційною небезпекою
(мішенню для обстрілів), метро - піармайданчик Тєрєхова(той ще ватан) давно час
запустити, а не бомжатник там розводити, тим паче ресурси для розселення тих,
хто залишився без житла в місті є, гуртожитки студентські, школи та дитсадочки
теж можна використати для цих цілей. Квіточки висаджувати простіше ніж людей
розмістити, так картинка яскравіша, або на асфальтоукладачі на камеру в
\enquote{броніку} покататись!

\iusr{Ksenka Reznik}
\textbf{Елена Борс} і як ви собі уявляете Часткову роботу метрополітену?

\iusr{Елена Борс}
\textbf{Ксенька Резник}, 

Ви питаєте з нотою недовіри і навіть зверхнього кепкування

Але я відповім

У Києві метро працювало майже без зупинки, доречі

Одна колія звільняється. І по ній курсують потяги. Так, це лише човночним чином
- тільки туди, а потім у зворотному напрямку. Людей переселити на декілька
станцій можна. Ці станції зробити можна без можливості зупинок

Людей ПОТРІБНО вмовляти виїжджати! Організовувати евакуацію

Третій місяць найбезпечніший транспорт використовують злочинно - лише роблячи
піар-майданчик, загс, кружкі для дєтєй паінтєрєсам.

Так, там захищають кілька тисяч людей. При цьому мільйон людей лишають
можливості безпечного пересування містом.

Крапка

І Ви ще скажіть, що видача гуманітарної допомоги від влади зорганізована.

Це - прямі злочини.

\iusr{Елена Борс}
\textbf{Ксенька Резник}, 

подивилася ваш профіль. До лютого навіть натяку на війну. Штош, какгріцца.
Росіянського контенту було просто тьма. Захищайте і надалі Тєрєхова.

\iusr{Ksenka Reznik}
\textbf{Елена Борс} 

я захищаю людей які залишилися в метрополітені, тому що не можно порівнювати
рівень небезпеки у Києві та Харкові. А життя покаже хто з нас \enquote{какгріцца}

\iusr{Елена Борс}
Рівно піхуй
Дасвіданя

\iusr{Alla Yakhonina}
\textbf{Ксенька Резник} 

людей з метрополітену потрібно повертати в соціум!!! Я сама залишаюсь у Ха. За
час війни, кожен день гуляти виходила тільки вдень, переважно в 14:00-15:00 я
була вже дома. Сьогодні, перший раз гуляли біля дому до 21:00, я прийшла до
дому і зрозуміла, що мені некомфортно, лячно так би мовити, що я була \enquote{не в
домике}. А як люди, які в метро, їх потрібно повертати до нормального життя, бо
чим далі - тим тяжче!

І так, як би метро працювало - особисто я би мала роботу, і не стояла в
оверверібіг черзі за парой курячих ніг!

\end{itemize} % }

\iusr{Alexandr Shvachka}

Каждій день читаю о прилетах на ХТЗ и удивляюсь. Живу возле Зеленого Гая, езжу
в Центр по Московскому и не слішу близких прилетов в отличие от \enquote{віходов}
куда-то далеко.

\begin{itemize} % {
\iusr{Geniy Kovalski}
\textbf{Александр Швачка} та же фигня. Не слышу, в чатах местных молчок.

\iusr{Liudmyla Peletska}
\textbf{Geniy Kovalski} 

Может, это ложится в статистику ХТЗ? - почти каждый день прилёты по Немышле, по
видео и фото - склады за Южкабелем, + по Маселе почти через день?

\iusr{Geniy Kovalski}
Через день?!
Я глухой вероятно @igg{fbicon.thinking.face} 
\end{itemize} % }

\iusr{Олег Кононенко}

Дальнобойность МСТА-Б- 24км, Гиацинта - 28 км. Последний от Липцов до ХТЗ
достанет.

\begin{itemize} % {
\iusr{Liudmyla Peletska}
\textbf{Олег Кононенко} 

Последние массированные обстрелы по полосе от Масельского до Пролёта 15-19
апреля - уже офф подтверждено, что Ураганами и Смерчами с тер. Белг.обл.

\end{itemize} % }

\iusr{Ирина Апанович}
Спасибо вам большое, все ясно и понятно. Никакой воды)

\iusr{Andrey Bogdanovich}
Салтовка примерно с 21:00 без света.

\iusr{Elena Grechenko}

Но был сегодня прилёт на Отакара и по яру тоже... надеюсь, что последняя
информация о том, что наши отодвинули русских на 40 км правда и город будет
практически недоступен для обстрелов

\begin{itemize} % {
\iusr{Geniy Kovalski}
\textbf{Elena Grechenko} 40 на (!) восток.
Циркуны на месте.

\iusr{Elena Grechenko}
\textbf{Евген Ковальски} значит ждём...

\iusr{Боцман Юджин}
\textbf{Elena Grechenko} то есть ситуация усугубилась. раньше за проспект Науки РСЗО без парашютиков не залетало ни разу - только авиа...

\iusr{Viktoriia Levizka}
\textbf{Боцман Юджин} 

не несите х...ни. \enquote{раньше} и на Одесскую, конный, Ландау безо всяких \enquote{авиа}
залетало. я живу рядом с перекрестком Ландау+Олимпийская. он был обстрелян
ИМЕННО градом. да. один раз. но это будет подальше Саржиного яра с любого
направления. иксперд, блин! ситуация у него ухудшилась

\iusr{Микита Соловйов}
\textbf{Elena Grechenko} 

Вот просто интересно. А где вы взяли информацию, что отодвинули на 40 км? То
есть я вот такой не видел ни в одном серьезном источнике. Кроме того, есть
такой чит "открыть карту. Там прямо во всех нормальных нарисовано, что Циркуны
под орками. А это 2-3 км от города.

\iusr{Микита Соловйов}
\textbf{Женя Антонов} Ерунда. Залетало в любую точку города. И будет иногда залетать до конца войны. вопрос только в вероятностях.

\iusr{Elena Grechenko}
\textbf{Микита Соловйов} 

сейчас сложно понять, где серьёзные источники, где нет... одни цитируют
других, понять первоисточник сложно. Если брать за основу официальную сводку,
либо обращения Терехова, то нет, там такого не было. Но во всех тг каналах
вчера прошло про 40 км.

\ifcmt
  ig https://scontent-mxp1-1.xx.fbcdn.net/v/t39.30808-6/279701424_7623978290975996_3347761304553202402_n.jpg?_nc_cat=111&ccb=1-5&_nc_sid=dbeb18&_nc_ohc=q6KIcennZ0kAX_vRDoi&_nc_ht=scontent-mxp1-1.xx&oh=00_AT9IIxPJIwd9Z-Eb5x2FB1uXLcWdnFdYzMF5G0mFHcFbSg&oe=62775DB8
  @width 0.3
\fi

\end{itemize} % }

\iusr{Viktoriia Levizka}

МЕТРО!! МЕТРО!!! МЕТРО!!!!!!!!! Терехов совсем офонарел от отмывания бабла на
высаживании цветочков? ЧЕМ это городу сейчас поможет??? а метро, аж пищит!!!!
изменился он.... ага... ждите.... пиар и показуха наше все((((

\iusr{Vitaly Manachinsky}
Спасибо вам за ваши посты

\iusr{Oleg Belikov}

Вы хотите разумное объяснение отказов эвакуации, но думаю что это решения
эмоциональные, из-за шока. Забиться в угол и не высовываться даже если
предлагают помощь

\begin{itemize} % {
\iusr{Светлана Белоус}
\textbf{Oleg Belikov} 

ну какой в нашей бедной стране шок. уехать неизвестно куда, ни с чем и потерять
даже тот мизер, что человек имеет? по итогу жо... пу сохранит, а куда потом ее
приткнуть... наученые многолетним горьким опытом, что играть в азартные игры с
государством очень чревато, и сидят дома. 50/50, толи прилетит, толи нет.

\iusr{Oleg Belikov}
\textbf{Светлана Белоус} 

вот думаю, что это и есть эмоциональная реакция) если есть шанс сохранить, как
вы говорите, хотя бы ж..., здраво сделать все, чтобы ее сохранить, даже если
придется пожить пару месяцев где-то в общежитии, зато с питанием и без угрозы
для жизни. Есть защитный механизм психики рационализация, который генерирует
такие аргументы, но в отличии от аргументов реальных, с этими никак не
поспоришь, потому что основаны на страхе, а не реальной оценке положения вещей
- я говорю о тех, кто остаются прям на линии огня с ежедневным обстрелом

\end{itemize} % }

\iusr{Валентина Гринченко}

Пишут, что на любой войне 30\% жителей отказываются уезжать. И остаются до
последнего в своём доме. Для них это очень важно.

\begin{itemize} % {
\iusr{Ludmila Mavrody}
\textbf{Валентина Гринченко} 

да, тут трудно этих людей судить, военные психологи объяснаяют это разной
реакций на стрес при угрозе жизни. В 3х словах упрощенно это реакции: дерись,
беги или замри. Вот люди в состоянии ступора не способны реагировать на
аргументы/факты и предпочитают держаться своего родного дома, это в основном
пожилые люди и также страдающие \enquote{выученой беспомощьностью}

\end{itemize} % }

\iusr{Сергей Пузырьков}

На ХГ и Баварии кое-какой общественный транспорт уже ходит. Только скорее всего
не благодаря городским властям, а вопреки.

\begin{itemize} % {
\iusr{Светлана Белоус}
\textbf{Сергей Пузырьков} на Баварии еще не видела,но очень надеюсь.

\iusr{Marianna Markova}
\textbf{Сергей Пузырьков} Дуже гарні новини!

\end{itemize} % }

\iusr{Татьяна Потенихина}

Мне очень нравится все то, что Вы пишете, первое что я делаю утром читаю Вашу
заметку, а потом уже пью кофе, но Вы продолжаете непонимание что такое денег
нет совсем и как страшно на старости лет ехать в не известность с кучей болячек
и привычек на удачу

\begin{itemize} % {
\iusr{Alexander Veprik}
\textbf{Татьяна Потенихина} так від того що ти сидиш в підвалі без світла - болячок менше не стає і грошей не додається, якби...

\iusr{Микита Соловйов}
\textbf{Татьяна Потенихина} 

Не буду говорить, что понимаю. Не знаю, понимаю или нет. Но я знаю, что
волонтеры предлагают найденные условия расселения уж точно не хуже имеющихся на
тех же Пятихатках, 533-м, Сев Салтовке. Не говоря уже о метро. С заметно более
высоким уровнем комфорта. При необходимости опеки. Не говоря уже о
безопасности.

\end{itemize} % }

\iusr{Nataliya Panina}

Селяни без городів втрачають сенс життя, а ви їм рятуватися пропонуєте покинувши городи.

А з міських є ж латентні, які чекають, на жаль.

\begin{itemize} % {
\iusr{Marianna Markova}
\textbf{Наталя Паніна} багато пропозицій по селах, де городи садити можна
\end{itemize} % }

\iusr{Svitlana Kaplun}
Не лише Салтівка, ХТЗ і П'ятихатки:

\href{https://www.sq.com.ua/rus/novosti/03.05.2022/obstrel-spalnogo-raiona-v-xarkove-povrezdeny-devyat-domov-sgoreli-masiny}{%
Обстрел спального района в Харькове: повреждены девять домов, сгорели машины, sq.com.ua, 03.05.2022%
}


\ifcmt
  tab_begin cols=2,no_fig,center
     pic https://i2.paste.pics/dc240a98c548099721c6e00af77e66c0.png
     pic https://i2.paste.pics/3ad40f3d529f36fc2f551328ad7ac050.png
  tab_end
\fi

\iusr{Рассоха Ігор}

Абсолютно перевірено на лінії фронту на Донбасі протягом попередніх 8 років:
частина людей НЕ евакуюється практично за жодних умов. Були навіть такі, що
РОКАМИ сиділи у Пєсках.

\iusr{Ирина Федотова}

А, нічого вигадувати і не треба.

Треба запускати метро та весь інший транспорт, принаймні на ті райони, де
більш-менш тихо.

Інакше місто не оживе!

\iusr{Сашко Таушанян}

Я только за запуск общественного транспорта, у меня стоит бизнес. Я из этого не
могу добираться до места работы, также покупатели тоже не могут доехать, что бы
сделать покупки. Но не рановато ли? Обстрелы ещё продолжаются.

\begin{itemize} % {
\iusr{Микита Соловйов}

И будут продолжаться еще долго. Конец войны явно не завтра.

Нужно понять и сделать выбор. Я уже об этом писал раз десять. наверное. Или
основной целью является максимальная безопасность. Тогда нужно минимизировать
любую активность. Но тогда не очень понятно, зачем поощрять людей сидеть в
Харькове.

Или мы хотим с некоторым риском, но запускать экономику, обеспечить людей
работой и т.д. И тогда нужно идти на открытие общественного транспорта,
переселять куда-то людей из метро и проч.

\iusr{Сашко Таушанян}
\textbf{Микита Соловйов} я это тоже понимаю. Но может это сделать после того как откинуть с границ области.

\iusr{Микита Соловйов}
\textbf{Сашко Таушанян} 

А есть какие-то внятные прогнозы, когда это случится? Вот у меня нет. И у тех
экспертов, которым я доверяю, тоже. И все видят достаточно вероятные сценарии,
в которых это может занять многие месяцы. А это возвращает нас к тому вопросу о
приоритетах, который я задал. Потому что сидеть и ждать в надежде, что вот-вот
все наладится, контрпродуктивно. Хотя и понятно психологически.

\end{itemize} % }


\end{itemize} % }
