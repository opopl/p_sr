%HEADER <<<---1
% ==========================================
% 
% Filename: 
%		/home/op/wrk/p/p.GoossensLATEXWEB.djvu.tex
% Paper key:
%		GoossensLATEXWEB
% Original paper PDF file:
%		/doc/papers/ChemPhys/GoossensLATEXWEB.pdf
% Page range:
%		1-543
% Original DJVU file:
%		/home/op/doc/papers/djvu//ocr//ocr.GoossensLATEXWEB.djvu
% Invoked script:
%   vep
% Invoked script command-line options:
%   -g -i -s gr01 -t
% Created on:
% 	12:24:20 (Fri, 18-Oct-2013)
%
% ==========================================
%--->>>1
%==========1-100==========<<<---1
%==========1-10==========<<<---2
 
%%page page_1                                                  <<<---3
 
 
%%page page_2                                                  <<<---3
 
The ETEX Web 
Companion 
Integrating TEX, HTML, and XML 

 
%%page page_3                                                  <<<---3
 
Addison-Wesley Series on 
Tools and Techniques for Computer Typesetting 
This recently inaugurated series focuses on tools and techniques needed for computer typesetting and information processing with traditional and new media. Books in the series address the practical needs both of users and of system developers. Initial titles comprise handy 
references for ETEX users; forthcoming works will expand that core. Ultimately, the series 
will cover other typesetting and information processing systems, as well, especially insofar as 
those systems offer unique value to the scientific and technical community. The series goal is 
to enhance your ability to produce, maintain, manipulate, or reuse articles, papers, reports, 
proposals, books, and other documents with professional quality. 
Ideas for this series should be directed to the editor: mittelbach@awl . com. 
All other comments and questions should be sent to Addison-Wesley: awcse@aw1 . com. 
Series Editor 
Frank Mittelbach 
Manager ETEX3 Project, Germany 
Editorial Board 
Jacques André Tim Bray Chris Rowley 
Irisa/Inria-Rennes, France Yextuality Services, Canada Open University, UK 
Barbara Beeton Peter Flynn Richard Rubinstein 
Editor, TUGhoat, USA University College, Cork, Perot Systems, USA 
David Brailsford I7'el“"d Paul Stiff 
University of Nottingham, Leslie Lamport University of Reading, UK 
UK Creator of HTEX, USA 
Series Titles 
The ETEX Companion, by Michel Goossens, Frank Mittelbach, and Alexander Samarin 
The ETEX Graphics Companion, by Michel Goossens, Sebastian Rahtz, and Frank Mittelbach 
The HTEX I/Veh Companion, by Michel Goossens and Sebastian Rahtz 
Also from Addison-Wesley: 
ETEX: A Document Preparation System, Second Edition, by Leslie Lamport 

 
%%page page_4                                                  <<<---3
 
The ETEX Web 
Companion 
Integrating TEX, HTML, and XML 
Michel Goossens 
CERN, Geneva, Switzerland 
Sebastian Rahtz 
Elsevier Science Ltd., Oxford, United Kingdom 
with Eitan M. Gurari, Ross Moore, and 
Robert S. Sutor 
ADDISON-WESLEY 
An Imprint of Addison Wesley Longman, Inc. 
Reading, Massachusetts 0 Harlow, England 0 Menlo Park, California 
Berkeley, California 0 Don Nlills, Ontario 
Sydney 0 Bonn 0 Amsterdam 0 Tokyo 0 Mexico City 

 
%%page page_5                                                  <<<---3
 
Many of the designations used by manufacturers and sellers to distinguish their products 
are claimed as trademarks. VVhere those designations appear in this book and Addison 
Wesley Longman, Inc. was aware of a trademark claim, the designations have been printed 
in initial capital letters or all capitals. 
The authors and publisher have taken care in the preparation of this book, but make no 
expressed or implied warranty of any kind and assume no responsibility for errors or 
omissions. No liability is assumed for incidental or consequential damages in connection 
with or arising out of the use of the information or programs contained herein. 
The publisher offers discounts on this book when ordered in quantity for special sales. For 
more information, please contact: AVVL Direct Sales 
Addison Wesley Longman, Inc. 
One Jacob Way 
Reading, Massachusetts 01867 
(781) 944-3 700 
Visit A-W on the Web: http : //www . awl . com/cseng/ 
Library of Congress Camloging-in-Publication Data 
Goossens, Michel. 
The LaTeX Web Companion: integrating TeX, HTML, and XML/Michel 
Goossens, Sebastian Rahtz; with Eitan M. Gurari, Ross Moore, and 
Robert S. Sutor 
p. cm.-(Addison-Wesley series on tools and techniques for 
computer typesetting) 
Includes bibliographical references and index. 
ISBN 0-201-43311-7 
1. HTML (Document markup language) 2. XML (Document markup 
language) 3. LaTeX (Computer file) I. Rahtz, S. P. Q. II. Title. 
III. Series 
QA76.76.H94G66 1999 
005.7’2-dc21 
98-48199 
CIP 
Reproduced by Addison Wesley Longman, Inc. from camera-ready copy supplied by the 
authors. 
Copyright © 1999 by Addison Wesley Longman, Inc. 
All rights reserved. No part of this publication may be reproduced, stored in a retrieval 
system, or transmitted, in any form, or by any means, electronic, mechanical, 
photocopying, recording, or otherwise, without the prior consent of the publisher. Printed 
in the United States of America. Published simultaneously in Canada. 
ISBN 0-201-43311-7 
12 3456 7 8 9 l0-CRS-0302010099 
Firrt printing, May I 999 

 
%%page page_6                                                  <<<---3
 
Contents 
List of Figures xi 
List of Tables xv 
Preface xvii 
The Web, its documents, and ETEX 1 
1.1 The Web, a window on the Internet . . . . . . . . . . . . . . . . 3 
1.1.1 The Hypertext Transport Protocol . . . . . . . . . . . . . 4 
1.1.2 Universal Resource Locators and Identifiers . . . . . . . 5 
1.1.3 The Hypertext Markup Language . . . . . . . . . . . . . 6 
1.2 BTEX in the Web environment . . . . . . . . . . . . . . . . . . . 11 
1.2.1 Overview of document formats and strategies . . . . . . . 12 
1.2.2 Staying with DVI . . . . . . . . . . . . . . . . . . . . . . 14 
1.2.3 PDF for typographic quality . . . . . . . . . . . . . . . . 15 
1.2.4 Down-translation to HTML . . . . . . . . . . . . . . . . 16 
1.2.5 Java and browser plug-ins . . . . . . . . . . . . . . . . . . 20 
1.2.6 Other IéTEX-related approaches to the Web . . . . . . . 21 
1.3 Is there an optimal approach? . . . . . . . . . . . . . . . . . . . . 23 
1.4 Conclusion . . . . . . . . . . . . . . . . . . . . . . . . . . . . . . 24 
Portable Document Format 25 
2.1 VVhatis PDF? . . . . . . . . . . . . . . . . . . . . . . . . . . . . . 26 
2.2 Generating PDF from TEX . . . . . . . . . . . . . . . . . . . . . 27 
2.2.1 Creating and manipulating PDF . . . . . . . . . . . . . . 28 

 
%%page page_7                                                  <<<---3
 
Contents 
3 
2.2.2 Setting up fonts . . . . . . . . . . . . . . . . . . . . . . . 29 
2.2.3 Adding value to your PDF . . . . . . . . . . . . . . . . . 33 
2.3 Rich PDF with BTEX: The hyperref package . . . . . . . . . . . . 35 
2.3.1 Implicit behavior of hyperref . . . . . . . . . . . . . . . . 36 
2.3.2 Configuring hyperref . . . . . . . . . . . . . . . . . . . . 38 
2.3.3 Additional user macros for hyperlinks . . . . . . . . . . . 45 
2.3.4 Acrobat-specific commands . . . . . . . . . . . . . . . . . 47 
2.3 .5 Special support for other packages . . . . . . . . . . . . . 49 
2.3.6 Creating PDF and HTML forms . . . . . . . . . . . . . . 50 
2.3.7 Designing PDF documents for the screen . . . . . . . . . 59 
2.3.8 Catalog of package options . . . . . . . . . . . . . . . . . 62 
2.4 Generating PDF directly from TEX . . . . . . . . . . . . . . . . . 67 
2.4.1 Setting up pdfTEX . . . . . . . . . . . . . . . . . . . . . . 67 
2.4.2 New primitives . . . . . . . . . . . . . . . . . . . . . . . 74 
2.4.3 Graphics and color . . . . . . . . . . . . . . . . . . . . . 80 
The PHFXZHTML translator 83 
3.1 Introduction . . . . . . . . . . . . . . . . . . . . . . . . . . . . . 83 
3.1.1 A few words on history . . . . . . . . . . . . . . . . . . . 84 
3.1.2 Principles for Web document generation . . . . . . . . . 84 
3.2 Required software and customization . . . . . . . . . . . . . . . . 86 
3.2.1 Running BTEXZHTML on a BTEX document . . . . . . 87 
3.2.2 Installation . . . . . . . . . . . . . . . . . . . . . . . . . . 92 
3.2.3 Customizing the local installation . . . . . . . . . . . . . 98 
3.2.4 Extension mechanisms and BTEX packages . . . . . . . . 100 
3.3 Mathematics modes with BTEXZHTML . . . . . . . . . . . . . . 101 
3.3.1 An overview of I:°iTEX2HTML’s math modes . . . . . . . 102 
3.3.2 Advanced mathematics with the math extension . . . . . . 105 
3.3.3 Unicode fonts and named entities, in expert mode . . . . 108 
3.3.4 HTML 4.0 and style sheets . . . . . . . . . . . . . . . . . 110 
3.3.5 Large images and HTML 2.0 . . . . . . . . . . . . . . . 112 
3.3.6 Future use ofMathML . . . . . . . . . . . . . . . , . , . 114 
3.4 Support for different languages . . . . . . . . . . . . . . . . . . . 115 
3.4.1 Titles and keywords . . . . . . . . . . . . . . . . . . . . . 116 
3.4.2 Character-set encodings . . . . . . . . . . . . . . . . . . 118 
3.4.3 Multilingual documents using babel . . . . . . . . . . . 119 
3.4.4 Images using special fonts . . . . . . . . . . . . . . . . . . 120 
3.4.5 Converting transliterations using preprocessors . . . . . . 120 
3.5 Extending BTEX sources with hypertext commands using the 
html package . . . . . . . . . . . . . . . . . . . . . . . . . . . . . 124 
3.5.1 Hyperlinks to external documents . . . . . . . . . . . . . 126 
3.5.2 Enhancements appropriate for HTML . . . . . . . . . . 128 
3.5.3 Alternative text for hyperlinks . . . . . . . . . . . . . . . 132 
3.5.4 Conditional environments . . . . . . . . . . . . . . . . . 135 

 
%%page page_8                                                  <<<---3
 
Contents 
3.5 .5 Navigation and layout of HTML pages . . . . . . . . . . 137 
3.5.6 Example of linking various external documents . . . . . . 141 
3.5.7 Advanced features . . . . . . . . . . . . . . . . . . . . . . 145 
4 Translating I4\'1]3X to HTML using TEX4ht 155 
4.1 Using TEX4ht . . . . . . . . . . . . . . . . . . . . . . . . . . . . . 15 6 
4.1.1 Package options . . . . . . . . . . . . . . . . . . . . . . . 156 
4.1.2 Picture representation of special content . . . . . . . . . 15 9 
4.2 A complete example . . . . . . . . . . . . . . . . . . . . . . . . . 160 
4.3 Manual creation of hypertext elements . . . . . . . . . . . . . . . 164 
4.3.1 Raw hypertext code . . . . . . . . . . . . . . . . . . . . . 164 
4.3.2 Hypertext pages . . . . . . . . . . . . . . . . . . . . . . . 166 
4.3.3 Hypertext links . . . . . . . . . . . . . . . . . . . . . . . 167 
4.3.4 Cascading Style Sheets . . . . . . . . . . . . . . . . . . . 167 
4.4 How TEX4ht works . . . . . . . . . . . . . . . . . . . . . . . . . 169 
4.4.1 From BTEX to DVI . . . . . . . . . . . . . . . . . . . . . 169 
4.4.2 From DVI to HTML . . . . . . . . . . . . . . . . . . . . 169 
4.4.3 Other matters . . . . . . . . . . . . . . . . . . . . . . . . 170 
4.5 Extended customization of TEX4ht . . . . . . . . . . . . . . . . . 170 
4.5.1 Configuration files . . . . . . . . . . . . . . . . . . . . . 170 
4.5.2 Tables of contents . . . . . . . . . . . . . . . . . . . . . . 172 
4.5.3 Parts, chapters, sections, and so on . . . . . . . . . . . . . 175 
4.5.4 Defining sectioning commands . . . . . . . . . . . . . . . 177 
4.5.5 Lists . . . . . . . . . . . . . . . . . . . . . . . . . . . . . 178 
4.5.6 Environments . . . . . . . . . . . . . . . . . . . . . . . . 179 
4.5.7 Tables . . . . . . . . . . . . . . . . . . . . . . . . . . . . 180 
4.5.8 Small details . . . . . . . . . . . . . . . . . . . . . . . . . 182 
4.6 The inner workings of TEX4ht . . . . . . . . . . . . . . . . . . . 184 
4.6.1 The translation process . . . . . . . . . . . . . . . . . . . 184 
4.6.2 Running BTEX . . . . . . . . . . . . . . . . . . . . . . . 185 
4.6.3 Running the tex4ht program . . . . . . . . . . . . . . . 186 
4.6.4 A look at t4ht . . . . . . . . . . . . . . . . . . . . . . . . 187 
4.6.5 From DVI to GIF . . . . . . . . . . . . . . . . . . . . . . 188 
4.6.6 A taste of the lg file . . . . . . . . . . . . . . . . . . . . . 189 
4.6.7 The font control files . . . . . . . . . . . . . . . . . . . . 190 
4.6.8 The control file . . . . . . . . . . . . . . . . . . . . . . . 193 
5 Direct display of HIEX on the Web 195 
5 .1 IBM techexplorer Hypermedia Browser . . . . . . . . . . . . . . 196 
5.1.1 Basic formatting issues . . . . . . . . . . . . . . . . . . . 198 
5.1.2 Your browser and techexplorer . . . . . . . . . . . . . . . 200 
5.1.3 Adding hypertext links . . . . . . . . . . . . . . . . . . . 204 
5.1.4 Popping up windows and footnotes . . . . . . . . . . . . 208 
5.1.5 Using images, sound, and video . . . . . . . . . . . . . . 210 

 
%%page page_9                                                  <<<---3
 
Contents 
5.1.6 Defining and using pop-up menus . . . . . . . . . . . . . 211 
5.1.7 Using color in your documents . . . . . . . . . . . . . . . 215 
5.1.8 Building a document hierarchy . . . . . . . . . . . . . . . 218 
5.1.9 Running applications . . . . . . . . . . . . . . . . . . . . 220 
5.1.10 Alternating between two displayed expressions . . . . . . 220 
5.1.11 Printing from techexplorer . . . . . . . . . . . . . . . . . 221 
5.1.12 Searching in a document . . . . . . . . . . . . . . . . . . 222 
5.1.13 Optimizing your documents for techexplorer . . . . . . . 222 
5.1.14 Scripting techexplorer from Java and]avaScript . . . . . . 223 
5.2 WebEQ . . . . . . . . . . . . . . . . . . . . . . . . . . . . . . . . 224 
5.2.1 An introduction to WebTEX . . . . . . . . . . . . . . . . 225 
5.2.2 Adding interactivity . . . . . . . . . . . . . . . . . . . . . 229 
5.2.3 Using the APPLET tag with WebEQ . . . . . . . . . . . 230 
5.2.4 Preparing HTML pages via the WebEQ Wizard . . . . . 232 
5.3 Embedded content problems and future developments . . . . . . 234 
5.3.1 Expression size . . . . . . . . . . . . . . . . . . . . . . . 235 
5.3.2 Ambient style . . . . . . . . . . . . . . . . . . . . . . . . 236 
HTML, SGML, and XML: Three markup languages 239 
6.1 Will HTML lead to the downfall of the Web? . . . . . . . . . . . 239 
6.2 HTML 4: A richer and more coherent language . . . . . . . . . . 241 
6.2.1 HTML 4 goodies . . . . . . . . . . . . . . . . . . . . . . 242 
6.2.2 HTML 4, the end of the old road . . . . . . . . . . . . . 243 
6.3 VVhy SGML? . . . . . . . . . . . . . . . . . . . . . . . . . . . . . 243 
6.3.1 Different types of markup . . . . . . . . . . . . . . . . . 244 
6.3.2 Generalized logical markup . . . . . . . . . . . . . . . . . 245 
6.3.3 SGML to HTML and XML . . . . . . . . . . . . . . . . 247 
6.4 Extensible Markup Languages . . . . . . . . . . . . . . . . . . . . 248 
6.4.1 VVhat is XML? . . . . . . . . . . . . . . . . . . . . . . . . 249 
6.4.2 The components of XML . . . . . . . . . . . . . . . . . . 251 
6.4.3 Declaring document elements . . . . . . . . . . . . . . . 255 
6.5 The detailed structure of an XML document . . . . . . . . . . . . 256 
6.5.1 XML is truly international . . . . . . . . . . . . . . . . . 257 
6.5.2 XML document components . . . . . . . . . . . . . . . . 258 
6.5.3 The XML declaration . . . . . . . . . . . . \ . . . . . . . 258 
6.5.4 The document type declaration . . . . . . . . . . . . . . 259 
6.5.5 Document elements . . . . . . . . . . . . . . . . . . . . . 270 
6.6 XML parsers and tools . . . . . . . . . . . . . . . . . . . . . . . . 271 
6.6.1 Emacs and psgml . . . . . . . . . . . . . . . . . . . . . . 272 
6.6.2 The perlSGML programs . . . . . . . . . . . . . . . . . 275 
6.6.3 The DTDParse tool . . . . . . . . . . . . . . . . . . . . 277 
6.6.4 The Language Technology Group XML toolbox . . . . . 277 
6.6.5 Validating documents with XML parsers . . . . . . . . . 281 

%==========10==========<<<---2
 
%%page page_10                                                  <<<---3
 
Contents 
7 CSS, DSSSL, and XSL: Doing it with style 
7.1 Style sheet languages: A short history . . . . . . . . . . . . . . . . 
7.2 Programming or style sheets, which is better? . . . . . . . . . . . 
7.3 Formatting with Perl . . . . . . . . . . . . . . . . . . . . . . . . . 
7.3.1 Principles of operation . . . . . . . . . . . . . . . . . . . 
7.3.2 Generating a BTEX instance . . . . . . . . . . . . . . . . 
7.4 Cascading Style Sheets . . . . . . . . . . . . . . . . . . . . . . . . 
7.4.1 The basic structure of a CSS style sheet . . . . . . . . . . 
7.4.2 Associating style sheets with a document . . . . . . . . . 
7.4.3 A quick look at CSS properties . . . . . . . . . . . . . . . 
7.4.4 CSS style sheets for formatting XML documents . . . . . 
7.4.5 The invitation example revisited . . . . . . . . . . . . 
7.4.6 Generating HTML with another document instance . . . 
7.5 Document Style Semantics and Specification Language . . . . . . 
7.5.1 The components of DSSSL . . . . . . . . . . . . . . . . 
7.5.2 Creating style sheets with DSSSL . . . . . . . . . . . . . 
7.5.3 Introducingjade . . . . . . . . . . . . . . . . . . . . . . . 
7.5.4 The TEX back-end for Jade and the ]adeTEX macros . . . 
7.5.5 The Jade SGML transformation interface . . . . . . . . . 
7.5.6 Formatting real-life documents with DSSSL . . . . . . . 
7.6 Extensible Stylesheet Language . . . . . . . . . . . . . . . . . . . 
7.6.1 The general structure of an XSL style sheet . . . . . . . . 
7.6.2 Building the source tree . . . . . . . . . . . . . . . . . . . 
7.6.3 Template rules . . . . . . . . . . . . . . . . . . . . . . . . 
7.6.4 XSL processors . . . . . . . . . . . . . . . . . . . . . . . 
7.6.5 Patterns . . . . . . . . . . . . . . . . . . . . . . . . . . . 
7.6.6 Templates . . . . . . . . . . . . . . . . . . . . . . . . . . 
7.6.7 Formatting objects and their properties . . . . . . . . . . 
7.6.8 Proposed extensibility mechanism . . . . . . . . . . . . . 
7.6.9 Using XSL to generate HTML or HTEX . . . . . . . . . 
7.6.10 Using XSL to generate formatting objects . . . . . . . . . 
8 MathML, intelligent math markup 
8.1 Introduction to MathML . . . . . . . . . . . . . . . . . . . . . . 
8.1.1 MathML, Unicode, and XML entities . . . . . . . . . . . 
8.2 MathML software . . . . . . . . . . . . . . . . . . . . . . . . . . 
8.2.1 Equation editors . . . . . . . . . . . . . . . . . . . . . . . 
8.2.2 Web browser support for MathML . . . . . . . . . . . . 
8.2.3 Converting BTEX to MathML . . . . . . . . . . . . . . . 
8.2.4 Typesetting MathML . . . . . . . . . . . . . . . . . . . . 
A Example files 
A.1 An example BTEX file and its translation to XML . . . . . . . . . 
A.1.1 The BTEX source . . . . . . . . . . . . . . . . . . . . . . 
289 
289 
291 
292 
293 
294 
297 
298 
302 
303 
306 
309 
311 
312 
313 
315 
318 
325 
331 
335 
337 
338 
340 
341 
342 
344 
350 
353 
355 
357 
362 
367 
368 
371 
372 
373 
376 
379 
387 
391 
391 
391 

 
%%page page_11                                                  <<<---3
 
Contents 
A. 1 .2 BTEX converted to XML . . . . . . . . . . . . . . . . . . 393 
A. 1.3 Document Type Definition for XML version . . . . . . . 396 
A.2 Scripting examples for techexplorer . . . . . . . . . . . . . . . . . 399 
A.2.1 teched.html . . . . . . . . . . . . . . . . . . . . . . . . . 399 
A.2.2 teched.java . . . . . . . . . . . . . . . . . . . . . . . . . . 400 
Technical appendixes 403 
B.1 The HyperTEX standard . . . . . . . . . . . . . . . . . . . . . . . 403 
B.2 Configuring TEX4ht to produce XML . . . . . . . . . . . . . . . 404 
B.2.1 Starting from scratch . . . . . . . . . . . . . . . . . . . . 404 
B.2.2 Adding XML tags . . . . . . . . . . . . . . . . . . . . . . 407 
B.2.3 Getting deeper for extra configurations . . . . . . . . . . 410 
B.3 XML namespaces . . . . . . . . . . . . . . . . . . . . . . . . . . . 415 
B.4 Examples of important DTDs . . . . . . . . . . . . . . . . . . . . 417 
B.4.1 The DocBook DTD . . . . . . . . . . . . . . . . . . . . 417 
B.4.2 The AAP effort and ISO 12083 . . . . . . . . . . . . . . 419 
B.4.3 Text Encoding Initiative . . . . . . . . . . . . . . . . . . 420 
B.4.4 A DTD for BIBTEX . . . . . . . . . . . . . . . . . . . . . 421 
B.4.5 BTEX-like markup, from DTD to printed document . . . 433 
B.5 Transforming HTML into XML . . . . . . . . . . . . . . . . . . 450 
B.5.1 HTMLh1 XML . . . . . . . . . . . . . . . . . . . . . . . 452 
B.5.2 The Ertensible HyperText Markup Language . . . . . . 454 
B.6 Java event-based interface . . . . . . . . . . . . . . . . . . . . . . 459 
B.6.1 The SAX Java classes . . . . . . . . . . . . . . . . . . . . 459 
B.6.2 Running a SAX application . . . . . . . . . . . . . . . . . 461 
Intemationalization issues 465 
C.1 Codes for languages, countries, and scripts . . . . . . . . . . . . . 465 
C.2 The Unicode standard . . . . . . . . . . . . . . . . . . . . . . . . 475 
C.2.1 Character codes and glyphs . . . . . . . . . . . . . . . . . 477 
C.2.2 Unicode and ISO/IEC 10646-1 . . . . . . . . . . . . . . 477 
C.2.3 UTF-8 and UTF16 encodings . . . . . . . . . . . . . . 479 
C.3 Foreign languages in XML . . . . . . . . . . . . . . . . . . , . . 480 
C.3.1 Latin-based encodings . . . . . . . . . . . . . . . . . . . 480 
C.3.2 Handling non-Latin encodings with UTF-8 . . . . . . . . 483 
Glossary 489 
URL catalog 499 
Bibliography 5 1 1 
Index 5 1 5 

 
%%page page_12                                                  <<<---3
 
List of Figures 
1.1 Putting a mathematics handbook on the Web . . . . . . . . . . . . . 13 
1.2 Electronic documents and the Web . . . . . . . . . . . . . . . . . . . 14 
1.3 Standard IATEX output . . . . . . . . . . . . . . . . . . . . . . . . . . 15 
1.4 Hypertext DVI viewing with xdvi . . . . . . . . . . . . . . . . . . . . 16 
1.5 Hypertext DVI viewing with dviwindo . . . . . . . . . . . . . . . . . 16 
1.6 Hypertext DVI viewing with dviout . . . . . . . . . . . . . . . . . . 17 
1.7 Simple PDF display (using Acrobat) . . . . . . . . . . . . . . . . . . . 17 
1.8 PDF display designed for the screen . . . . . . . . . . . . . . . . . . . 18 
1.9 Conversion to HTML with math as pictures . . . . . . . . . . . . . . 19 
1.10 Conversion to HTML with math using Symbol fonts . . . . . . . . . 20 
1.1 1 Display using techexplorer browser plug-in . . . . . . . . . . . . . 21 
1.12 Conversion to HTML with embedded Math]VIL rendered with plug-in 22 
2.1 Distiller job options for graphics . . . . . . . . . . . . . . . . . . . . 29 
2.2 Distiller job options for fonts . . . . . . . . . . . . . . . . . . . . . . 29 
2.3 Display of PDF file using embedded fonts . . . . . . . . . . . . . . . 31 
2.4 Display of PDF file using Multiple Master substitution for fonts . . . 31 
2.5 Display of PDF file using bitmap fonts . . . . . . . . . . . . . . . . . 32 
2.6 Default appearance of test document in PDF enhanced with hyperref 37 
2.7 Test document displayed with xdvi . . . . . . . . . . . . . . . . . . . 37 
2.8 Test document showing use of colorlinks option . . . . . . . . . . 38 
2.9 Normal bibliography . . . . . . . . . . . . . . . . . . . . . . . . . . . 38 
2.10 The effect of the backref option . . . . . . . . . . . . . . . . . . . . 38 
2.11 Long link text split across lines . . . . . . . . . . . . . . . . . . . . . 40 

 
%%page page_13                                                  <<<---3
 
xii 
List of Figures 
2.12 
2.13 
2.14 
2.15 
2.16 
2.17 
2.18 
2.19 
2.20 
2.21 
2.22 
'\IJ'\IJ'uIJ'\IJ'\IJ'\IJ'sJJ'~)~)'\IJ'\IJ '\IJ'\IJ'\IJ'\IJ'\IJ'sA)'~)~)'\IJ'\IJ'.\IJ 
I\Jr--l--r--r--r--r--r--r--r-- r--\O%\]O\'J1.{>'vJl\)rO\Ooo\IO\U1-l>uoI\a>- O 
3.21 
PDF document displayed with no toolbar or interface and with multiple pages visible . . . . . . . . . . . . . . . . . . . . . . . . . . . . . 43 
PDF bookmarks open . . . . . . . . . . . . . . . . . . . . . . . . . . . 44 
Display of PDF document information . . . . . . . . . . . . . . . . . 44 
The effect of \ref vs. \autoref . . . . . . . . . . . . . . . . . . . . . 46 
Calculator written in PDF (by Hans Hagen) . . . . . . . . . . . . . . 51 
Simple Acrobat form . . . . . . . . . . . . . . . . . . . . . . . . . . . 5 3 
Simple form presented in HTML . . . . . . . . . . . . . . . . . . . . 56 
Screen-designed PDF file I . . . . . . . . . . . . . . . . . . . . . . . . 60 
Screen-designed PDF file II . . . . . . . . . . . . . . . . . . . . . . . 60 
Screen-designed PDF file III . . . . . . . . . . . . . . . . . . . . . . . 61 
Alternate screen-designed PDF file . . . . . . . . . . . . . . . . . . . 61 
Example BTEX source file . . . . . . . . . . . . . . . . . . . . . . . . 88 
Formatted PostScript output of example file . . . . . . . . . . . . . . 89 
HTML structure generated by BTEXZHTML . . . . . . . . . . . . . . 93 
Mathematics using IATEXZHTML with novice mode (default settings) 104 
Mathematics using professional mode as default with A/\/(S packages . 104 
Mathematics in an HTML page using professional mode . . . . . . . 104 
Mathematics in an HTML page using expert mode . . . . . . . . . . 104 
Mathematics using expert mode, requiring the special math extension 109 
Mathematics using expert mode and Unicode font characters . . . . . 109 
Mathematics using expert mode with large images, HTML 4.0, and 
style sheet effects . . . . . . . . . . . . . . . . . . . . . . . . . . . . . 113 
Mathematics using images of complete environments (HTML 2.0) . . 113 
Mathematics with images of complete environments . . . . . . . . . 115 
Mathematics using “thumbnails" hyperlinking to full-size images . . 115 
Example of preprocessing and transliteration with BTEXZHTML . . . 121 
Sample of Singhalese script produced with Indica . . . . . . . . . . 122 
Sample of Devanagari script for the Hindi language . . . . . . . . . . 125 
Sample of Sanskrit showing both traditional romanized forms . . . . 125 
Using \HTMLcode commands with BTEXZHTML . . . . . . . . . . . 138 
Using the \html info command with BTEXZHTML . . . . . . . . . . 140 
Linking external files (BTEX files) . . . . . . . . . . . . . . . . . . . . 142 
Linking external files (generated HTML files) . . . . . . . . . . . . . 144 
Example Makefile for composite documents . . . . . . . . . . . . . 146 
HTML structure generated by BTEX2 HTML . . . . . . . . . . . . . . 149 
Segmentation example (ETEX files) . . . . . . . . . . . . . . . . . . . 150 
A simple source file and the resulting HTML code . . . . . . . . . . . 156 
A source document and a display of its HTML files . . . . . . . . . . 157 
Display of an HTML file created by TEX4ht . . . . . . . . . . . . . . 161 
The standard BTEX Output of \reffig{4-3} . . . . . . . . . . . . . . . . 162 
The BTEX source file for the document of \reffig{4-3} . . . . . . . . . 163 

 
%%page page_14                                                  <<<---3
 
List of Figures 
4.6 
4.7 
4.8 
4.9 
4.10 
4.11 
4.12 
4.13 
4.14 
4.15 
4.16 
4.17 
4.18 
4.19 
4.20 
5.10 
5.11 
5.12 
5.13 
5.14 
5.15 
5.16 
5.17 
5.18 
5.19 
5.20 
6.1 
6.2 
6.3 
6.4 
6.5 
A bitmap with density of 144 dots per inch . . . . . . . . . . . . . . . 164 
A variant of \reffig{4-3} . . . . . . . . . . . . . . . . . . . . . . . . . . 165 
Input files: (a) try .tex and (b) try . cfg. Output files: (c) try.html 
and (d) try.css . . . . . . . . . . . . . . . . . . . . . . . . . . . . . 172 
Configuring the tables of contents . . . . . . . . . . . . . . . . . . . 174 
Configuring the section headings . . . . . . . . . . . . . . . . . . . . 175 
Configuring sections to make multiple files . . . . . . . . . . . . . . . 176 
Configuring navigation buttons . . . . . . . . . . . . . . . . . . . . . 178 
Defining a new sectioning command . . . . . . . . . . . . . . . . . . 178 
Configuring lists and environments . . . . . . . . . . . . . . . . . . . 179 
Adding new hooks . . . . . . . . . . . . . . . . . . . . . . . . . . . . 184 
The workflow and files of TEX4ht . . . . . . . . . . . . . . . . . . . . 185 
Runtime information in the log file from BTEX . . . . . . . . . . . . 186 
Messages fromrunning tex4ht . . . . . . . . . . . . . . . . . . . . . 188 
Messages from running t4ht . . . . . . . . . . . . . . . . . . . . . . 188 
Portions of a virtual hypertext font file cmr . htf . . . . . . . . . . . . 190 
Two examples of techexplorer displaying text and mathematics . . . 197 
Customizing the colors used to display your documents . . . . . . . . 201 
A techexplorer commutative diagram embedded within an HTML 
 . . . . . . . . . . . . . . . . . . . . . . . . . . . . . . . . . . . . 203 
Two techexplorer expressions embedded within an HTML page . . 204 
Two frames, each containing a techexplorer window . . . . . . . . 206 
A pop-up footnote in techexplorer . . . . . . . . . . . . . . . . . . 209 
Setting permissions within techexplorer . . . . . . . . . . . . . . . 212 
Inline video in the Professional Edition of techexplorer . . . . . . . 212 
A gradient box in a section heading . . . . . . . . . . . . . . . . . . . 216 
Simplified structure for a book . . . . . . . . . . . . . . . . . . . . . 218 
The document tree . . . . . . . . . . . . . . . . . . . . . . . . . . . . 219 
Searching for text in a document . . . . . . . . . . . . . . . . . . . . 222 
A simple ETEX editor built using techexplorer . . . . . . . . . . . 224 
Simple example of WebEQ . . . . . . . . . . . . . . . . . . . . . . . 225 
Simple example of WebEQ (\reffig{5-14}): Wizard input and output . 226 
Linebreaking by WebEQ . . . . . . . . . . . . . . . . . . . . . . . . 232 
An output HTML page generated by the WebEQ Wizard . . . . . . . 234 
Reasonable sizes for a techexplorer and WebEQ expression 235 
Decreasing the width of the display rectangle . . . . . . . . . . . . . 236 
Some style matching problems . . . . . . . . . . . . . . . . . . . . . 23 7 
Two instances of an article class . . . . . . . . . . . . . . . . . . . . . 246 
Emacs in psgml mode . . . . . . . . . . . . . . . . . . . . . . . . . . 273 
Emacs in dtd mode . . . . . . . . . . . . . . . . . . . . . . . . . . . . 273 
A visual editor for XML V 274 
The Amaya visual editor . . . . . . . . . . . . . . . . . . . . . . . . . 276 

 
%%page page_15                                                  <<<---3
 
7.1 
7.2 
7.3 
7.4 
7.5 
7.6 
7.7 
7.8 
7.9 
7.10 
7.11 
8.1 
8.2 
8.3 
8.4 
8.5 
B.1 
B.2 
C.1 
C.2 
C.3 
C.4 
C.5 
C.6 
List of Figures 
XML file formatted with PHEX using the sgmlspl procedure . . . . . 297 
XML file formatted with HTML using the sgmlspl procedure . . . . 310 
The DSSSL process . . . . . . . . . . . . . . . . . . . . . . . . . . . . 313 
Simple DSSSL style with RTF . . . . . . . . . . . . . . . . . . . . . . 322 
Simple DSSSL style with TEX . . . . . . . . . . . . . . . . . . . . . . 322 
Word97 view of RTF output . . . . . . . . . . . . . . . . . . . . . . . 324 
PostScript view of TEX output . . . . . . . . . . . . . . . . . . . . . . 324 
PostScript view of TEX output (alternate DSSSL formatting) . . . . . 326 
Mathematics generated with SGML, DSSSL, and TEX . . . . . . . . . 331 
XML to HTML transformation with DSSSL . . . . . . . . . . . . . . 336 
PDF generated from flow objects with fop . . . . . . . . . . . . . . . 365 
MathType equation editor . . . . . . . . . . . . . . . . . . . . . . . . 374 
Example equations typeset by TEX . . . . . . . . . . . . . . . . . . . 374 
Amaya Web browser showing MathML . . . . . . . . . . . . . . . . 377 
WebEQ “wizard" in action translating HTEX to MathML . . . . . . . 381 
MathML sample processed by the Jade DSSSL engine and TEX . . . 388 
PostScript rendering of our IAYIEX-based XML document . . . . . . . 451 
XHTML tag sets and profiles . . . . . . . . . . . . . . . . . . . . . . 453 
Character layout of Unicode (ISO/IEC 10646 BMP) . . . . . . . . . 476 
Entire 4-octet coding space of ISO/IEC 10646-1 . . . . . . . . . . . 478 
Structure of Group 00 of ISO/IEC 10646-1 . . . . . . . . . . . . . . 478 
A French invitation (I£>'IEX version) . . . . . . . . . . . . . . . . . . . 484 
A UTF-8 encoded XML file with Russian, Greek, and math . . . . . . 485 
HTML rendering of UTF-8 file with Netscape . . . . . . . . . . . . . 488 

 
%%page page_16                                                  <<<---3
 
List of Tables 
1.1 Basic HTML tags . . . . . . . . . . . . . . . . . . . . . . . . . . . . . 7 
1.2 Comparison of HTML and BTEX . . . . . . . . . . . . . . . . . . . . 11 
2.1 Possible values for PDF link View specifications . . . . . . . . . . . . 42 
2.2 Hyperref \autoref names . . . . . . . . . . . . . . . . . . . . . . . . 46 
2.3 Acrobat menu option link names . . . . . . . . . . . . . . . . . . . . 48 
2.4 hyperrezf options to specify drivers . . . . . . . . . . . . . . . . . . . 62 
2.5 Configuration options . . . . . . . . . . . . . . . . . . . . . . . . . . 62 
2.6 Extension options . . . . . . . . . . . . . . . . . . . . . . . . . . . . 63 
2.7 PDF-specific display options . . . . . . . . . . . . . . . . . . . . . . . 63 
2.8 PDF information options . . . . . . . . . . . . . . . . . . . . . . . . . 65 
2.9 Form environment options . . . . . . . . . . . . . . . . . . . . . . . . 65 
2.10 Forms options . . . . . . . . . . . . . . . . . . . . . . . . . . . . . . 65 
2.1 1 Acrobat page layout display options . . . . . . . . . . . . . . . . . . . 66 
2.12 Acrobat page transition options . . . . . . . . . . . . . . . . . . . . . 67 
2.13 PDF link actions . . . . . . . . . . . . . . . . . . . . . . . . . . . . . 79 
C.1 Language codes and names (ISO 639) . . . . . . . . . . . . . . . . . . 466 
C.2 Country codes and names (ISO 3166) . . . . . . . . . . . . . . . . . . 471 
C3 Script codes and names (ISO 15924) . . . . . . . . . . . . . . . . . . 473 
C.4 The ISO/IEC 8859 standards and the language families covered . . . 475 

 
%%page page_17                                                  <<<---3
 
Preface 
The aim of this book is to provide help for authors, primarily scientists, who want to 
invest in the Web or other hypertext presentation systems but are not living in the 
world of Microsoft Word or Quarl<Xpress. They have an investment in markup systems such as BTEX and have special needs in fields like mathematics, non-European 
languages, and algorithmic graphics. The book will tell them how to 
0 make full use of the Adobe Acrobat format from BTEX; 
a convert their legacy documents to HTML or XML; 
0 make use of their math in Web applications; 
0 use ETEX as a tool in preparing Web pages; 
0 read and write simple XML/ SGML; 
o produce high-quality printed pages from their Web-hosted XML or HTML 
s using TEX or PDF. 
ETEX as a document repository for the Internet 
The World VV1de Web has invaded all areas of society, and science is no exception 
to this rule. This should come as no surprise since the Web paradigm was born at 
CERN, one of the largest scientific laboratories in the world. 
The present ubiquitous Web interface is the result of basic research that took 
place in the first years of the 1990s at CERN, Before then use of the Internet 
had been mostly an affair of specialists. It needed the genius and insight of Tim 
Berners-Lee and collaborators to create a tool that allowed physicists participating 

 
%%page page_18                                                  <<<---3
 
xviii 
Preface 
in CERN’s high-energy physics program but located all over the world to exchange 
data and information via the Internet in an intuitive and “user-friendly" way. Their 
work led directly to the development of the HTML language, the HTTP protocol, and the URL addressing scheme-the three basic pillars on which the Web 
is built. From the very beginning, the group took the farsighted decision to share 
their work freely with the Internet community. Then, thanks also to the appearance of the graphic interface of the Mosaic browser, the Web paradigm was received 
enthusiastically by developers and users alike. The growth of the number of Web 
sites and users became exponential, culminating in the “Woodstock of the Web" at 
CERN in May 1994. CERN, a scientific laboratory dedicated to basic research, did 
not have the resources to coordinate Web development further, and hence these responsibilities were transferred to the international World Wide Web Consortium 
[=->W3C], which at present consists of three main components: the Laboratory for 
Computer Science at MIT [=->MIT], USA; INRIA [=->INRIA], France; and Keio 
University [=-> KEIO], Japan. The Consortium is supported by DARPA [=->DARPA] 
and the European Commission [=-> BC]. 
One lesson to be learned from the history of the advent of the Web is that 
basic research, in completely unexpected ways, can lead to very important and wideranging spin-offs for society. 
Although most people do not realize it, SGML (in the form of the ubiquitous 
lingua franca of the Web, HTML) is today without doubt the leading markup language for electronic documents. Similarly BTEX has been used for over a decade for 
marking up scientific documents. Even today there is no viable alternative to print 
texts containing a lot of mathematics without using ETEX. Therefore it seems reasonable to look for ways to find a (possibly) automatic procedure to translate IATEX 
documents in a form that is exploitable on the Web. Conversely, documents marked 
up in XML and HTML should be able to benefit from the high typographic qualities 
of the TEX processor. 
Therefore in this book we explain how ETEX can be used as the central component of an electronic document strategy for the Web. We show how you can 
reuse your existing ETEX documents on the Web by translating them into HTML, 
and how, by using some BTEX extension packages, you can more fully exploit the 
hypertext capabilities of HTML. Today HTML and Web browsers cannot deal very 
well with nontextual document components, such as pictures (which are translated 
into bitmap images) or mathematics. We also address the translation of BTEX into 
PDF and the possibilities of interpreting BTEX commands directly by extensions of 
a browser. 
We also introduce you to the secrets of XML, the extensible markup language, 
which uses a subset of SGML and which is set to replace HTML as it allows for 
application-dependent extensions. In particular, we look at MathML-the mathematical markup language-its syntax and how it can be generated, and what it can 
be used for. 

 
%%page page_19                                                  <<<---3
 
Preface 
Going in the other direction, we discuss various strategies to transform Web 
source documents marked up in XML or HTML into I5TEX or PDF for optimal 
printing, in particular using DSSSL and XSL style sheets. 
Many tools for transforming TEX-based source files into HTML have been 
developed over the years. The programs described in this book are a representive 
sample chosen mainly because we were familiar with them and have used them 
ourselves. The absence of a description of other tools in this book in no way implies 
that we consider them to be less useful or of inferior quality. 
Logical structure of the book 
We suggest that all readers look at Chapter 1 before going any further, because this 
chapter introduces how we think-that the Web is not a threat to BTEX, but an 
opportunity and why you should or should not continue to Write in I5TEX. We also 
present a short introduction to the Web from the point ofview of the ETEX user. 
Chapter 2 treats the subject of how to marry hyperdocuments with page fidelity 
using the Portable Document Format (PDF). 
The conversion of I-5TEX documents into HTML is tackled in Chapters 3 and 4. 
In Chapter 3 we discuss ETEXZHT ML, which uses Perl to interpret ETEX source 
documents and to generate HTML code. Extension packages can be easily added 
in the form of Perl routines, while various extensions to the ETEX language make 
ETEXZHTML a real high-performance tool to generate hypertext documents. 
We take a different approach in Chapter 4, where TEX4ht uses a redefinition of 
I-5TEX’s TEX macros to generate HTML or XML, possibly using also the MathML 
application for expressing the mathematics. 
Recently we have seen the development of browsers (with plug-ins) that are 
able to interpret mathematical markup directly. Chapter 5 looks at implementations 
that can direcly interpret large subsets of native I5TEX code without prior translation into HTML, in particular techexplorer, a plug-in for Netscape and Internet 
Explorer developed by IBM, and WebEQ, a]ava applet for rendering math. 
Chapter 6 looks at the broader picture and gives a gentle introduction to 
SGML (Standard Generalized Markup Language); it explains how XML (Extensible Markup Language), a simpler and more “Internet and user-friendly" variant of 
SGML will become an important element in any future document strategy for the 
Internet. It is anticipated that XML, combined with object databases and other current object-oriented technologies, will revolutionize our document management at 
all levels. Tools for authoring and interpreting XML will be described, and we will 
spend some time building a I5TEX-like XML markup language. 
TEX was originally developed by Don Knuth to print his math books in accordance with the highest standards of the typographic art. Therefore it should come 
as no surprise that TEX has been proposed as a typesetting engine for Web mate
%==========20==========<<<---2
 
%%page page_20                                                  <<<---3
 
Preface 
rial. Tools to translate XML sources into various output formats are described in 
Chapter 7. The use of Cascading Style Sheets (CSS), Document Style Semantics 
and Specification Language (DSSSL), and Extensible Stylesheet Language (XSL) 
for controlling the translation process will be detailed. 
Chapter 8 tackles the “hot" issue of how to take maximal advantage of I-5TEX’s 
optimal mathematical notation to translate ETEX markup into XML and MathML 
(Mathematical Markup language), a companion to XML to present and work with 
math on the Web (see Foster (1999) for an overview of various ways to handle math 
on the Web). 
The book ends with appendixes that contain technical information to complement the chapters in the book. We provide an introduction to Web namespaces, 
discuss internationalization issues, and review a few important XML DTDs. 
History and authorship 
VVhen T/ye EYEX Graphics Companion was in its early stages, Sebastian Rahtz and 
Michel Goossens intended to include coverage of the Portable Document Format, 
SGML, and the Web in that book. It became apparent, however, that the hypertext 
and SGML material would require a whole book of their own, so as soon as the 
Graphics Companion was completed, work started on this I/Veli Companion. Even more 
than is the case with most TEX work, the packages and programs related to the Web 
and TEX were changing very rapidly; it was decided, therefore, to ask the authors 
of three of the most important packages to work with Rahtz and Goossens, to make 
sure that the chapters would be up-to-date and accurate. 
The chapter on ETEXZHTML is primarily the work of Moore; that on TEX4ht 
the work of Gurari; and that on IBM techexplorer and WebEQ that of Sutor; 
Goossens and Rahtz shared the remaining chapters between them. Gurari, Moore, 
and Sutor also contributed significantly to the rest of the book by commenting on 
material, contributing sections, and discussing the issues involved. 
It is, perhaps, a tribute to the Internet that the five authors never met in person 
as a group during the entire writing and editing process. The nearest they came was 
a pleasant dinner in Saint-Malo at the 1998 EuroTEX meeting, where all but Eitan 
Gurari were present. 
Using, and finding, all those packages and programs 
Unless explicitly mentioned otherwise, all packages and programs described in this 
book are freely available in public software archives; some are in the public domain, 
while others are protected by copyright. Some programs are available only in source 
form or work only on certain computer platforms, and you should be prepared 
for a certain amount of “getting your hands dirty" in some cases. We also cannot 

 
%%page page_21                                                  <<<---3
 
Preface 
guarantee that later versions of packages or programs will give results identical 
to those in our book. Many of them are under active development, and new or 
changed versions appear several times a year; we completed this book in the winter 
of 1998-1999, and tested the examples with versions current at that time. 
As regular users of the World Wide Web will know, keeping track of URLs is a 
tricky, error-prone process as sites continually disappear or change their structure. 
In this book, therefore, we do not give formal URLs in the text, but rather give 
pointers (typeset like “[=->W3C]") to a catalog of URLs (starting on page 499). This 
catalog will be kept up to date and will be available in the directory info/lwc on 
the CTAN nodes, where the source code of most of our examples will also be made 
available. We have tried to clear up some of the fog of acronyms by providing a 
glossary of terms (starting on page 489). 
Colophon 
This book was prepared using ETEX. The main text font is Adobe Janson, the sans 
serif font is Y&Y’s European Modern Sans, the math is set in Y&Y MathTime Plus, 
and the literal typewriter text is set in Y&Y’s European Modern typewriter. 
The BTEX style was refined and generalized by Frank Mittelbach from that 
developed by him and Sebastian Rahtz for The EYEK Graphics Companion, which, in 
turn, was derived from the style by Frank Mittelbach and Michel Goossens for The 
[HEX Companion. 
Acknowledginents 
We are grateful to Nelson Beebe (University of Utah), Tim Bray (Textuality), Mimi 
Burbank (Florida State University), David Carlisle (NAG), Hans Hagen (Pragrna), 
Han The Thanh (Masaryk University, Brno), T. V. Raman (Adobe Systems), D. P. 
Story (University of Akron), Michael Downes (American Mathematical Society), 
Peter Flynn (University College, Cork), Chris Maden (O’Reilly), Thomas Merz 
(Munich), and Chris Rowley (Open University) for advice, encouragement, and 
comments on draft chapters. 
Sebastian Rahtz would like to take this opportunity to thank Tanmoy Bhattacharya, David Carlisle, Patrick Daly, Yannis Haralambous, and many others, for 
their help with the hyperref package, and Berthold Horn (Y &Y) for sponsoring 
part of the development. 
Eitan M. Gurari is very thankful to Gertjan Klein and Sebastian Rahtz for their 
contribution to the development of TEX4ht. Gertjan’s help came at early stages of 
the project, offering important code and advice for making TEX4ht a portable tool 
and providing numerous detailed comments and suggestions for configuring the 
output. Sebastian got involved in the project at later stages, providing an enormous 

 
%%page page_22                                                  <<<---3
 
xxii 
Preface 
amount of feedback, setting up challenging objectives, collaborating in the development of interesting configuration files, aggressively promoting the system, and 
heavily editing my contribution to this book. Aside and beyond the professional 
aspects, Gertjan and Sebastian were great Net associates! 
Robert Sutor wants to express express his gratitude to Bill Pulleyblank, Marshall Schor, and Dickjenks of the IBM Research Division for their support during 
the time techexplorer was developed. 
Ross Moore would like to acknowledge first Nikos Drakos, for his foresight in 
designing a translator such as ETEXZHTML and establishing its basic design principles. There is insufficient space here to list all those who have made significant contributions; we thank them all. Among them we especially wish to acknowledge Marcus Hennecke and Herb Swan, who were the most significant contributors when 
Nikos could no longer be involved. We also wish to acknowledge Jens Lippman, 
Scott Nelson, and Marek Rouchal who continue to supply the support necessary to 
develop, maintain, and distribute the latest revisions of the BTEXZHTML program. 
Second Ross wants to thank Michel Goossens, Mimi]ett,]erold Marsden, Robert 
Miner, and Kristoffer Rose for supporting visits to various places around the world, 
where ideas for extensions to BTEXZHTML were discussed and/or developed; some 
of these visits have directly affected the contents of this book. 
On the publishing side, Frank Mittelbach (series editor) did an excellent job 
of trying to keep us on the straight and narrow path, and Peter Gordon (Addison 
Wesley Longrnan, Inc.) provided all the encouragement, support, jokes, and help 
any authors could want. VVhen it came to production,]ohn Fuller, Helen Goldstein, 
and Maureen Willard were very patient with our idiosyncrasies, steered us safely to 
completion, and edited our dubious prose into real English. 
Feedback 
We would like to ask you, dear reader, for your collaboration. We kindly invite 
you to send your comments, suggestions, or remarks to any of the authors. We will 
be glad to correct any mistakes or oversights in a future edition and are open to 
suggestions for improvements or the inclusion of important developments we may 
have overlooked. We will maintain a list of errata in a file called webcomp. err in 
the ETEX distribution, and this will contain current addresses for the authors. 
Many of the Web applications that we describe in this book continue to evolve 
rapidly. The source code of the examples in the directory info/lwc of the 
CTAN nodes will be kept up-to-date to guarantee that the code will work with 
future versions of W3C specifications. This applies in particular to the XSL files 
of Sections 7.6, B.4.5, and C.3 since the syntax of XSL is not yet finalized. 

 
%%page page_23                                                  <<<---3
 
CHAPTER 1 
The Web, its documents, 
and ETEX 
In this chapter we will look at how the World Wide Web was born and the main 
components that make it into a genuine global cultural and language-independent 
communication tool. This includes a short introduction to HTML, the markup 
language used in most Web documents at present. Most of the tools discussed in 
this book generate HTML or, more generally, XML markup, so that an elementary 
knowledge of its syntax will come in handy later. 
Then we will take a bird’s-eye view of the various approaches that have been 
developed to deal with content-rich scientific documents focusing on the role of 
ETEX as an input or output format in this environment. This will lead us to the 
conclusion that the development of the Web and its new view of an electronic document should be considered an enrichment of the toolbox available to scientists to 
communicate results and data. 
The Internet, and in particular the World Wide Web, reached a peak of public 
visibility when President Bill Clinton mentioned them in his State of the Union 
address before the United States Congress on January 27, 1998 [‘-> SOTU98]: 
We should enable all the world’s people to explore the far reaches of cyberspace. Think of this-the first time I made a State of the Union speech 
to you, only a handful of physicists used the World Wide Web. Literally, 

 
%%page page_24                                                  <<<---3
 
The Web, its documents, and ILVIEX 
just a handful of people. Now, in schools, in libraries, homes and businesses, millions and millions of Americans surf the Net every day. . . . But 
we also must make sure that we protect the exploding global commercial 
potential of the Internet. . . . 
For one thing, I ask Congress to step up support for building the next 
generation Internet. It’s getting kind of clogged, you know. And the next 
generation Internet will operate at speeds up to a thousand flmes faster 
than today. 
And onjanuary 19, 1999, President Clinton was proud to announce [hr SOTU99]: 
. . . We are well on our way to our goal of connecting every classroom and 
library to the Internet. 
Nowadays not only does the VVhite House take the Web seriously (the above quote 
was taken from one of the thousands of government documents available on its 
Web site), but every company, school, organization, public or commercial utility, 
and, before long, every individual will want a presence on the Web. It has become 
common practice in scientific (and not-so-scientific) publications to use URL addresses to refer to supplementary information, and in many cases scientific work is 
now available first on a Web site long before it is published in “paper" form in a 
recognized journal. 
In the scientific world we have seen an extremely swift evolution from a kind of 
hesitation to embrace the new Web technology to the enthusiasm of using preprint 
databases to speed up the dissemination of information. The main problem that 
remains to be solved is qualiq, both in content and form. The quality of the content can be guaranteed by adopting the established peer review system, building 
upon the expertise of many of the existing publishing houses that find a new role 
as “information verification agents." The quality of presentation is a problem that 
is not yet fully solved. There have been many public debates about whether current computer screens can provide the necessary detail to represent faithfully the 
visual multidimensional information inherent in a mathematical or chemical formula. Several attempts have already been made to come up with ways to overcome 
the coarseness of the computer screen (at best a few pixels per millimeter), keeping 
the flexibility of interactive hypertext searching possibilities. 
Consider also the situation in parts of the world, such as Russia, Southeast Asia, 
and Africa, that are facing severe financial constraints, and where it is often out of 
the question even to consider printing multiple copies of a highly technical document. Electronic dissemination via the Web is the only way, then, to publish. Thus 
the Web is not only an additional medium for the traditional publishing establishment, but a necessity for the larger part of the world to participate in sharing the 
information and benefit from the wealth and progress it creates. 
VVhat form will information on the Web take in future? The I/Vorld VVz'de I/Vela 
Consortium (VV 3 C) was formed to encourage software companies to work together 

 
%%page page_25                                                  <<<---3
 
1.1 The Web, a window on the Internet 
to come up with solutions that guarantee interoperability and statelessness of electronic documents and data on the Internet. 
VVhen we look at our own system (TEX), we see that the Web is not a threat to 
ETEX but an additional opportunity. We expect thatscientists will continue to use 
the tools they feel most confortable with for writing their documents (be it BTEX 
or other applications), but they will have new opportunities for making the results 
available. 
We hope to convince you in this book that, thanks to the tools we describe 
in the following chapters, you do not have to choose between TEX’s typographic 
quality and the global connectivity of the Web: You can have both. 
1.1 The Web, a window on the Internet 
The popularity of the Internet expanded greatly across national and subject barriers 
during the 1990s, and this is in great part due to the advent of the World Wide Web 
which was developed at the European Laboratory for Particle Physics (CERN) 
[=-> CERN] in the early 1990s. 
Before looking at the Web itself, let us say a few words about the Internet. 
The need for reliable countrywide communication channels for the United States 
military complex led to the development of a packet-switching network. Arpanet 
allows messages to reach their destination via different routes, thus guaranteeing 
delivery under adverse conditions. The work on Arpanet was the basis of Transmission Control Protocol/Internet Protocol (TCP/IP), a method that is now used 
almost universally on networks to divide messages into separate datagrams, each 
identified by a unique sequence number. The messages reach their destination via 
a variety of routes and are reassembled there to deliver the original message text 
(Tanenbaum (1996)). Since Arpanet, and subsequently the Internet, which replaced 
Arpanet around 1988, interconnect all kinds of computers and physical networks, 
TCP/IP has become the lingua franca of communication protocols. 
In the 1980s, electronic mail (e-maz'l) was the most popular form of communication between users on a network, with File Transfer Protocol (ftp) and terminal 
emulation (telnet) serving as direct-connection applications when needed. However, 
a stateless, easy-to-use information distribution system, where the connection does 
not have to be kept alive, was long overdue. 
By 1989 TCP/IP had also become well established internally at CERN, with the 
Internet replacing most of the proprietary homegrown communication protocols. 
It was in that year that Tim Berners-Lee put forward his first ideas of what would 
become the World Wide Web. He could build upon the expertise he had gained 
in the area of “distributed computing," in particular working on remote procedure 
calls. Tim also benefited from the “Mac" and “NeXT" cultures, two computer systems that offered a rich and user-friendly interface to (distributed) programming. 
These systems were very popular with many physicicts in those days (the Mac still 

 
%%page page_26                                                  <<<---3
 
The Web, its documents, and ILVIFX 
is). By mid-1990 Tim and a colleague (Robert Cailliau)1 had fiiialized the three 
basic software protocols of the World Wide Web:2 1. Hypertext Transport Protocol (HTTP), the method that allows various WWW 
servers to communicate (see Section 1.1.1); ' 
2. Universal Resource LOI.‘tlt07’ (URL), a universal addressing scheme to locate all 
information on the Internet (see Section 1.1.2); 
3. Hypertext Markup Language (HTML), the language for marking up the information (see Section 1.1.3). 
By the end of 1990 a demonstration version of a Web client on a NeXT computer existed. It was soon to be followed by a linemode browser that could be easily 
interfaced to the many information formats available at CERN. Thus the open 
architecture (via the URL) allowed us to reuse the thousands of existing documentation pages and to integrate them neatly with the new VVWW paradigm. 
Although the Web was presented at a seminar in June 1990 and introduced to 
the whole of the High Energy Physics community in the CERN Computer Newsletter 204, by December 1991 not more than ten Web servers existed in the world 
and the physicists regarded the W as just one more of those computer gadgets 
that diverted them from doing physics. VVhat was needed was a killer application that 
would really show the advantages of the Web as an interface to the global Internet. 
Although a few browsers offering a graphical user interface were available at the 
beginning of 1993, it was Marc Andreessen’s Mosaic that got the ball rolling. For 
the first time users could “click away" and jump between documents comprising 
the information web in a simple and “visual" way. 
In May 1994 the first World Wide Web Conference took place at CERN; it 
later became known as the “Woodstock of the Web." Indeed, 1994 can be considered a real turning point, with CERN handing over the development of the W 
to the World Wide Web Consortium [=->W3C] and the very influential Netscape 
Company being created. Since then Microsoft, and by 1998 all the other giants of 
the software industry, have become interested in the advantages of the global Web. 
Nowadays, there is hardly any company or government agency in the world that, 
alongside a fax number and an e-mail address, does not proudly display the URL of 
its home page on business cards and letterheads. 
1.1.1 The Hypertext Transport Protocol 
HTTP is the language which WWW servers use to talk to one another; technically 
speaking, it is an application-level protocol for distributed, hypermedia information systems. The first version (confusingly known as HTTP/0.9) was developed in 
lRobert was a specialist in Hypercard, the Macintosh hypertext program that used a scripting language called hypertalk. 
2For more details on the early history of the Web see [‘-->WEBHIST] and [‘--> RAGHIST]. 

 
%%page page_27                                                  <<<---3
 
1.1 The Web, a window on the Internet 
1990 by Tim Berners-Lee and collaborators in the framework of the World Wide 
Web initiative at CERN. It was a simple protocol for raw data transfer across the 
Internet. In 1996 HTTP/ 1.0 introduced messages using a Multipurpose Internet 
Mail Extensions (MIME) [¢->RFC2045] format. This allowed some semantic interpretation of the content thanks to elementary support of metadata. However, some 
problems remained, and this led to the definition of HTTP/1.1 in January 1997 
[n->HTTPRFC]. It introduced “search," “front-end update," and “annotation." It 
also allowed for use of Uniform Resource Identifiers (URIs). Moreover, HTTP can 
now also be used as a generic protocol for communication between user agents and 
proxies/ gateways to other Internet systems, thus allowing basic hypermedia access 
to resources available from diverse applications. 
Since the World Wide Web is still growing extremely fast and further evolution depends very much on the efficiency of HTTP, some questions have been 
raised about the monolithic structure of the HTTP specification. Therefore, the 
Internet Engineering Task Force (IETF) [c->IETF] is at present preparing a New 
Generation specification for the HTTP protocol. It will take a modular approach, 
working outward from an efficient and small core, supplemented by message transport, general-purpose remote method invocation, and document processing layers. 
This should guarantee global scalability, network efficiency, and transport flexibility 
(see [¢->HTTPNG] for more information). 
1.1.2 Universal Resource Locators and Identifiers 
A URL is the “postal address" of a document on the Web. The syntax and semantics of the Uniform Resource Locator (URL) for locating and accessing resources 
via the Internet were first developed in 1990 at CERN and later were formalized 
[<-> RFC1738]. 
It was soon realized that the original URL syntax was a little too rigid to deal 
with moving resources, such as a change of the Internet address of the server where 
a document resides. Hence the idea of a unifying syntax for the expression of names 
and addresses of objects on the Internet was introduced [¢->RFC1630]. The Web 
must be able to deal with objects accessed using an extensible number of protocols, 
present and future, with the access instructions of the protocol encoded into the 
form of the address string of the object. 
A Universal Resource Identifier (URI) [¢->RFC2396] is a member of this universal set of names in registered namespaces and addresses referring to registered 
protocols. A URL is a special case of a URI, which maps onto an access algorithm 
using network protocols. A Uniform Resource Name (URN) is intended to serve 
as a persistent, location-independent, resource identifier that can map other namespaces that share the properties of URNs into URN space [¢->RFC2141]. 
The general structure of a URI is 
scheme : //hostname [:port] /path/ft’ lename [#f'r-agment] 

 
%%page page_28                                                  <<<---3
 
The Web, its documents, and EYIEX 
Using the HTTP protocol you can have, for instance, 
http://info.internet.isi.edu:80/in-notes/rfc/files/rfc1630.txt 
http://www.oasis-open.org/cover/topics.html#entities 
while examples on a local file system (Microsoft Windows and UNIX) are 
file:D:\lark\lark.html 
file:/afs/cern.ch/user/g/goossens/hagel.html 
You can also specify connection-type protocols, such as ftp or telnet, as follows: 
ftp://ftp.jclark.com/pub/xml/xt.zip 
telnet://mycomputer.cern.ch 
1.1.3 The Hypertext Markup Language 
HTML is a simple markup language for creating documents that will appear on the 
Web. It did not have a formal description (an SGML “Document Type Definition"; 
see Section 6.3.3) until 1995 [¢->RFC1866]. It is a subset of this HTML Version 2 
that we will review (another quick introduction can be found at [¢->RAGHTML]). 
Nowadays, HTML Version 4 (see Section 6.2) offers a much wider functionality, 
although the basic principles remain the same. Furthermore, HTML is being recast 
in a form compatible with XML (see Appendix B.5 and [¢-> HTMLINXML]). \reftab{1-1} 
gives an overview of some basic HTML tags. Following we will review a few of them 
in more detail. This should allow you to understand the structure of an HTML 
source document if you ever see one. Most browsers will let you see the HTML 
source of the display with something like a “View source" menu item. 
1.1.3.1 The minimal HTML document 
A simple example of HTML is the following: 
<TITLE>The simplest HTML example</TITLE> 
<H1>A level one heading</H1> 
<P>Welcome to the world of HTML! 
<P>Let’s have a second paragraph. 
HTML uses tags primarily to tell the Web browser how to display the text. The 
example given uses 
0 the <TITLE> tag (which has a corresponding </TITLE> end tag) to specify the 
title of the document, 
0 the <H1> header tag (with corresponding end </H1>), and 
o the <P> start-of-paragraph tag. 

 
%%page page_29                                                  <<<---3
 
1.1 The Web, a window on the Internet 
\reftab{1-1}: Basic HTML tags 
General structure of an HTML document 
<html> 
<head> 
<title> . . .</title> document title 
<meta>. . .</meta> generic meta information 
</head> 
<body> . . . </body> 
</html> 
Elements used inside the body 
Text elements 
<p> start of new paragraph 
<pre> . . . </pre> preformatted text (may include embedded tags--not all tags allowed) 
Headers 
<h1> . . .</h1> first-level header <h2> . . .</h2> second-level header 
<h3> . . .</h3> third-level header <h4>. . .</h4> fourth-level header 
<h5> . . .</h5> fifth-level header <h6>. . .</h6> sixth-level header 
Logical markup 
<em> . . .</em> emphasis <code>. . .</code> computer code 
<samp> . . . </samp> sample output <kbd> . . . </kbd> text entered at keyboard 
<var> . . .</var> variable instance <cite> . . .</cite> citation 
Physical font styles 
<b>. . .</b> bold text <i>. . .</i> italic text 
<u>. . .</u> underlined text <tt>. . .</tt> typewriter text 
Lists 
<dl> begin definition list <ul> begin unordered list <ol> begin ordered list 
<dt> term <dd> data <1i> items in list <1 i> items in list 
</dl> end definition list </ul> end unordered list </ol> end ordered list 
Hyperlinks and zmclyors 
<a name=``anchor_name''> . . .</a> define an anchor 
<a href="#anchor_name"> . . .</a> link to anchor in same file 
<a href="URI#anchor_name">. . . </a> link to a target location in another file 
Images 
<img src=``URI''> . . .</img> include a graphic image 
HTML3 tags consist of a left angle bracket (<), followed by the element type 
name and zero or more attributes, and closed by a right angle bracket (>). Tags 
usually occur in pairs, for example, <H1> and </H1>, where the ending tag looks 
just like the starting tag except for the slash (/). In the example, <H1> signals the 
start of a top-level heading, </H1> its end. 
3The syntax of markup based on the SGML standard will be discussed in detail in Chapter 6. 

%==========30==========<<<---2
 
%%page page_30                                                  <<<---3
 
The Web, its documents, and HIFX 
Many elements do not need to be used in begin/end pairs (such as the <P> 
start-of-paragraph tag in the example) since the start of the next tag often closes the 
previous one. Nevertheless, it is good practice always to provide close tags for those 
that are open (take into account: the correct nesting, as explained in Chapter 6). 
Note also that HTML tags (and attributes) are case insensitive, that is, <title> 
is equivalent to <TITLE>, or even <TiTlE>. 
1.1.3.2 Titles and headings 
Every HTML document should have a title that is generally displayed separately 
from the document and is primarily used for document identification (it is shown 
in the window title of most graphical browsers). Choose a meaningful title, since it 
will generally show up in Internet searches. The title should go at the beginning of 
the document and be enclosed between <title> . . . </title> tags. 
HTML has six levels of headings, usually displayed using larger and/or bolder 
fonts than in normal body text. The first heading in each document should be 
tagged <H1>. The syntax of the heading tag is 
<Hy>Tc$t of heading</Hy> 
where y is a number between 1 and 6 specifying the level of the heading. 
1.1.3.3 Lists 
HTML supports unnumbered, numbered, and description lists. No paragraph separator is required for list items. The tags for the items in the list implicitly terminate 
the previous list item. 
Unnumbered lists ~ 
Unnumbered or unordered lists start with an opening list <UL> tag. You then enter the <LI> tag followed by the item text (the closing tag can be omitted) and 
terminate the list with a closing list </UL> tag. 
HTML source with two items Text generation 
<UL> o apples 
<LI>apples 
<LI>bananas ‘ bananas 
</UL> 
Browsers can display an unnumbered or unordered list in various ways, for 
example by using bullets, filled circles, or dashes to show the items. As explained in 
Chapter 7, style sheets should be used to control the rendering of HTML elements. 

 
%%page page_31                                                  <<<---3
 
1.1 The Web, a window on the Internet 
Numbered lists 
A numbered or ordered list is quite similar in structure to an unordered list except 
that the <OL> tag is used instead of the <UL> tag. Items are tagged using the same 
<LI> tag. On the output medium the items will be numbered. 
HTML source with three items Text generation 
<OL> 1. oranges 
<LI>oranges 
Qppeaches 2. peaches 
<LI>grapes 3_ a CS 
</DL> Sr P 
Once again, the actual symbols used for ordering the items (Arabic or Roman 
numerals, alphabetical letters, and so on) can be controlled by a style sheet. 
Description lists 
A description list consists of a set of description terms (tagged <DT>) and description 
dam (tagged <DD>). 
HTML source with three items Text generation 
<DL> 
<DT>URI _ _ 
<DD>Universal Resource Identifier URI Umversal Resource Identlfier 
<DT>URL URL Universal Resource Locator 
<DD>Universal Resource Locator 
<DT>URN</DT> URN Universal Resource Name 
<DD>Universal Resource Name</DD> 
</DL> 
The <DT> and <DD> entries can contain multiple paragraphs (separated by paragraph tags), lists, or other descriptive information. Note how we used an end tag 
for the last entry. 
1.1.3.4 Hypertext links 
The chief power of HTML comes from its ability to link regions of text (or images) to an “anchor" (location) in the same or an external document. These regions 
are typically highlighted by the browser to indicate that they are hypertext links. 
HTML’s single hypertext-related directive is the <A> element. 
Anchors can be used to move to a particular section in a document. Suppose 
you want to define a link from “documentA" to a particular section in “documentB." 
First, you need to set up a named zmclyor in “documentB" as follows: 
<A NAME=``myname''>target te$t</A>... 

 
%%page page_32                                                  <<<---3
 
10 
The Web, its documents, and Ié'ltX 
Then, to link to the target text region, as defined earlier, create a link in “documentA." Include the name of the target document, as well as the name of the target 
region itself, as follows: 
My link to the target is <A HREF="documentB.html#myname">he'r*e</A>. 
VVhen you click on the word “here" in “documentA," the browser will load 
“documentB" and position the top of the window at the start of the paragraph 
“target text." 
1.1.3.5 Images 
Browsers can display bitmap (usually GIF, PNG, or JPEG) images inside documents. You can include such an image by using an <IMG> tag (or more generally, in 
HTML4, an <0BJECT> tag; see Section 6.2.1). For instance, the following two ways 
of including the GIF image “mypict . gif" are equivalent: 
<IMG SRC="mypict.gif"> 
and 
<0BJECT DATA="mypict.gif" TYPE="image/gif"> 
Such images are inline components so that you might have to place them on the 
screen by specifying alignment details or by putting them in a separate paragraph. 
In general, by using a URI as the SRC or DATA field, you can include an image 
residing anywhere on the Internet. You can also include the URI of the image inside 
an anchor so that it is up to the user to get the image or not (saving bandwidth if 
the image is large) by specifying something similar to the following: 
My picture is <A HREF="http://host/path/mypict.gif">here</A>. 
1.1.3 .6 Special characters 
Some ASCII characters are treated specially by HTML and, therefore, cannot be 
used directly in the text. These include <, >, and KL. To produce such characters 
as part of normal text, you have to use a so-called entity reference4 that consists 
of an ampersand as the start symbol, followed by the entity name, followed by a 
semicolon as a terminator. For the above three characters this would be Salt ; , &gt ; , 
and Stamp ;, respectively. 
Several entity sets exist. An example is the Latin 1 (ISO885 9-1) character set, 
where in addition to many others, the following entities are defined: 
&ouml; generates a lowercase o with an umlaut: 6, 
ecntilde; generates a lowercase n with a tilde: f1, 
\ 
&Egrave; generates an uppercase E with a grave mark: E. 
4See Section 6.5.4.3 to know more about the SGML entity reference mechanism. 

 
%%page page_33                                                  <<<---3
 
1.2 HIEX in the Web environment 
11 
\reftab{1-2}: Comparison of HTML and ETEX 
Description HTML E‘I}3X equivalent 
Document sectioning commands 
level I <H1> te.'z:t</I-11> \chapter{text} 
level 2 <1-12> te.'ct</I-12> \section{text} 
level 3 <1-13> te.'z:t</I-13> \subsection{text} 
level 4 <H4> t e.'z:t</l‘I4> \subsubsect ion{text} 
level 5 <H5> te.'ct</I-15> \paragraph{text} 
level 6 <H6> te.'z:t</I-16> \subparagraph{text} 
new paragraph <P> \par 
Highlighting text 
emphasis <EM> te.'z:t</EM> \emph{text} 
hold font <B> temt</B> \t extbf {text} (\mathbf {text}) 
teletypefont <TT> te.'z:t</TT> \texttt {text} (\mathtt{text}) 
Lists 
ordered list <DL>. . .</OL> \begin{enumerate}. . .\end{enumerate} 
unordered list <UL>. . .</UL> \begin{itemize} . . . \end{itemize} 
list item <LI>te.'z:t \item te.'z:t 
description list <DL> . . . </DL> \begin{description} . . . \end{des cript ion} 
description term <DT> term \item [term] 
description data <DD> te.'z:t te.'z:t 
Special characters 
accents (e. g., e) Kceacute; \’ e 
umlauts (e.g., it) &uuml ; \"u 
new line » <BR> \newline 
1.2 ETEX in the Web environment 
In the previous sections we described how the revolution of generalized access and 
of distribution of electronic documents on the Internet, in particular via the World 
Wide Web, has come about, and what the technical principles that underpin its 
success are. 
But how does that new environment fit into the picture of producing and distributing scientific documents of the highest possible typographic quality-an area 
where ETEX is still largely unchallenged? 
First let us note the similarity of approach between HTML and ETEX in areas 
of structural representation of the information. In \reftab{1-2} we present a comparison of markup in the two systems. At a basic structural level it is not too difficult to 
translate between ETEX and HTML, thus allowing convenient reuse of our investment in ETEX documents for display on the Web. 
Present-day browsers do not always represent complex information (tables, 
mathematics, pictures) in a precise and acceptable way. As we explain later, in such 
a case the information can be represented in various ways, perhaps as inline bitmap 

 
%%page page_34                                                  <<<---3
 
12 
The Web, its documents, and HIEX 
images or as external files, using the PDF or PostScript formats. As an appetizer of 
what is possible with quite simple tools, consider the Digital Library of Mathematical 
Functions Project by the National Institute of Standards and Technology (NIST). 
They are planning to provide the complete contents of the Handbook of Mathematical Functions (Abramowitz and Stegun (1972)) in electronic form [c->NIST"HMF], 
with interactive VRML graphs,5 links to software to calculate functions, and so on. 
\reffig{1-1} shows examples from the chapter about Airy functions. 
1.2.1 Overview of document formats and strategies 
\reffig{1-2} shows various formats in which electronic documents for the Web can 
be prepared and ways to transform between them. In the upper left-hand corner we 
represent a L‘:TEX document. A standard and well-known procedure is to compile 
the I°:TEX file with TEX, generating a DVI file, and from there to obtain a PostScript 
file with dvips. The latter file can be printed, if needed. Alternatively, you can 
generate a PDF file, directly from ETEX with pdfTEX or indirectly by translating 
the PostScript file into PDF (for example, with Adobe’s Acrobat Distiller) or by 
translating the DVI file into PDF (e.g., with dvipdfm). The PDF file can still be 
printed, if needed, but it is equally convenient to make it available on the Internet, 
since it can contain hyperlinks and is editable. 
As mentioned at the beginning of this section, we can also exploit the similarities of the generic markup schemes of IBTIEX and HTML (or more generally, as 
we explain in Chapter 6, via an instance of an XML DTD) to translate ETEX into 
HTML (or an XML instance). This is labeled “L2H" (“LZX") in \reffig{1-2}. We 
can also translate between XML and HTML with an “XZH" program. Note that for 
rendering such files on the Web, we can make use of a style sheet language (such as 
CSS, DSSSL, or XSL; see Chapter 7). 
We can also go in the other direction and translate XML (HTML) files into 
ETEX and, therefore, profit from the powerful typographic machinery of TEX to 
obtain printable documents of good quality (software labeled “XZL" in \reffig{1-2}). 
We expect that XML will be used more and more as a storage format for electronic 
documents in the medium-term future and that ETEX will become an interesting 
back-end for printing purposes. 
In the following sections we give a short overview of various representations 
of the same ETEX document, starting from a sample document shown as standard 
IATEX output in the “normal" \reffig{1-3}. The text, which is taken from a physics 
document, contains some simple inline and displayed math formulae that will allow 
us to show the main characteristics of the proposed solutions. The source of the 
document is referenced in Appendix A. 1. 
5 The Virtual Reality Modeling Language is one of the ideas that was born in informal discussion during 
the “Woodstock of the Web" in May 1994 at CERN. It allows for creating an animation in a threedimensional virtual world, featuring full interactivity at the viewer level. See [<-->WEB3D] and ISO/IEC 
standard 14772:1997ISO/IEC:14772-1 (1998). 

 
%%page page_35                                                  <<<---3
 
1.2 Ié'ltX in the Web environment 13 
C 11 Aigyunctions . W4 .1. 
oz. VER 
IL 1 Nowtion ll.J(Il) Complex VIflflbk 
2 mdwfinhle l1.gIInII.3: lAi(:)i. ::z‘+I'y. 
: complex vu-inble. 
c complex plane (excluding infinily), 
1 minim (ezwludinnginflnily). 
Aiy hm:I1um:'I‘he namion 
(11 1.1) Mu), mm. 
is dun iolefliegg (191l?).'l.mer,ihe say. of: wax clmueii 
Annflur notation in that afFock 119(6)-. 
(1141) 1/(t)= WWI). V(a:) = s/?Ai<x)Tlbh l1.I:AdditionI1ViluI.lizflion.I of M. 
Scorer‘: nunmnnmsee §1L1_2) The not-Iion 
11" GM mm l mini law) any) |.n'(,)i mam) tum) 
Tsmac magi-n ims Lmggs’ mac 
'%l‘-¥VLMl4.kfl@.V9&\_’m|..i&@ 
is (me to Sggrer 1950 . 
A..omernmimnm.:ai1muk' (19 9 : 
C A. Aigz functions: 
Physics applications »,c. w. 
Cllrk 
Q. manna nienhgnjg 
A.1 Particle motion in a region of constant force 
(linear potential) 
m schmdinger emotion for me wuvefimclion ms, :1 an describe: Ipmicl: 
ormm M nizjemnaconnmxfmce I‘ is 
_ - 5 
A.1.l fi,'f:v°9[x.:> - r -x em; = m;,1=(x.r), 
when 5 - 1.054E7306(6-!) X 1fl““JI is dividsd by 21. 
We shame I" In am Ii: nzgalive direction arm: Cunesinn cnnniinne :, ..e 
r _ 4:; «hi: i. e..uav.iem In defining . polentinl fimciinrn 
(Am) V(:) = F: 
Eq. mix) a. mlv-dby loplmtion ofvu-inhln. w» wrih 
(A13) m. r) = e*~'**-'"""/'u'«.<z). 
when the comm-nu 2,;..;, in mrpectiveiy the emgy -nd the awn 
cotlpotmm om. pn1i::|e's mmneiinui in the plane pcrpmdiculx w Ir, mu 
m n :51: am. (if 
‘ EH 
(b) 
\reffig{1-1}: Putting a mathematics handbook on the Web 
(a) the notation used for the Airy functions; (b) an example of a physics application in quantum mechanics where Airy functions are used; (c) a static visual 3D representation of an Airy 
function with a complex argument; (d) a VRML visualization (using a plug-in) initiated by 
activating the link indicated by the arrow on the HTML page shown in (c). 

 
%%page page_36                                                  <<<---3
 
14 
The Web, its documents, and EVIEX 
L2X 
HTML 
Multimedia 
Audio-video 
\reffig{1-2}: Electronic documents and the Web 
1.2.2 Staying with DVI 
The TEX community has been involved with hypertext for some time (see Carr 
et al. (1991)), and the open architecture of the \special command makes it relatively easy to embed, for instance, hypertext linking commands in DVI files. We can 
then extend viewers to use these link commands and to use them as simple hypertext browsers. The hyperref package (see \refsec{2_3_Rich_PDF_with_LaTeX_The_hyperref_package}) has drivers for the \special 
commands supported by various drivers (See Appendix B.1). Figures 1.4, 1.5, and 
1.6 show three different DVI viewers (xdvi [<->XDVI] under Linux and dviwindo 
[<->YANDY] and dviout [¢->DVIOUT] under Windows, respectively) displaying our 
file and highlighting the bibliographical citations in the first paragraph as links. 
The applications also support the loading of a Web browser when a URL link is 

 
%%page page_37                                                  <<<---3
 
1.2 I1}'I];X in the Web environment 
15 
3 Vavilov theory 
Vavilov[5] derived a more accurate straggling distribution by introducing the 
kinematic limit on the maximum transferable energy in a single collision, rather 
than using Emax = 00. Now we can write[2]: 
f (e, 5.9) = 14) (A~a2) 
5 
where 
45 (,\n,fl2) = 45 (s) Wds c 2 0 
<Z>(8) = exp [N(1 + 3%)] EXP W (s)], 
w (8) = 811m + (8 + BZN) l1n(8/N) + E1(8/'i)l - N8"/‘, 
and 
E1 = [DC t'1e"dt (the exponential integral) 
Av : Kl[€g€’A//_fl2] 
The Vavilov parameters are simply related to the Landau parameter by 
)\L = /\1,/Ii - lnn. It can be shown that as N -> O, the distribution of the 
variable AL approaches that of Landau. For N 3 0.01 the two distributions 
are already practically identical. Contrary to what many textbooks report, the 
Vavilov distribution does not approximate the Landau distribution for small Ii, 
but rather the distribution of AL defined above tends to the distribution of the 
true A from the Landau density function. Thus the routine GVAVIV samples 
the variable AL rather than A. For Ii 3 10 the Vavilov distribution tends to a 
Gaussian distribution (see next section). 
\reffig{1-3}: Standard ETEX output 
encountered, but it cannot be integrated within a Web browser. We will mention 
one DVI viewer that does integrate with browsers in Section 1.2.6. 
1.2.3 PDF for typographic quality 
Another simple approach, and one that guarantees typographic quality to the same 
level as that of a DVI file, is to generate a PDF file with dvipdfm or pdfTEX or with 
Acrobat Distiller starting from a PostScript file. In this case with the help of a PDF 
browser, such as Acrobat (see \reffig{1-7}), we can easily navigate through the document and exploit hypertext information that might be present in the ETEX source 
(IATEX cross-references and bibliographic citations can be automatically turned into 
hyperlinks). Web browsers can be configured to load Acrobat as a helper application 
when they meet PDF files, and they will then display your document in a browser 
window. Acrobat can also pass URL links to the browser to resolve, giving a seamless integration. This simple approach is explained in Chapter 2, where we also 
discuss page designs optimized for reading on the screen (\reffig{1-8}). 

 
%%page page_38                                                  <<<---3
 
16 
The Web, its documents, and ETEX 
3 Vavilov theory 
V'avil0\>[.fl tlerivml 21 niore aA‘,t:umt.e straggling distribution by int,rodu(:ing tho 
" kinematic limit on the maximum tralrsferalrle energy in 21 single collision. rather 
than using Em, : oc. Now we can writ,c[2]: 
r<e,o'~> = gas, (x..,w=') 
wliew 
_ I ctioo 
op ()..,.K.,_(i’) = 2_,/ rb(s)e"‘d.e rtz 0 
"7 rrwirao 
¢(s) : <=xI>[~(1+ 5%)] vxvlw(s)]‘ 
1/: (5) = slnrc + (S + B1:-c)[ln(s/K) + E1(S/K)] - IWFX/W, 
and 
«ac 
E;(z) = / F‘n“dt (the exponential integral) 
)t = K [figg -7’ -19] 
The Vmlilnv parametms‘ are simply related to the Landau parmiwter hy 
A: = A, /K - inn. It can be shown that as x A U, the distribution of the 
\reffig{1-4}: Hypertext DVI viewing with xdvi 
3 Vavilov theory 
Vavilov[E|] derived a more accurate straggling distribution by introducing t 
kinematic limit on the maximum transferable energy in a single collision, rath 
than using Emu = 00. Now we can write[[]]: 3 
mas) = gm («\...w.l32) 
where 
¢ (z\Ic,[32) = %];fi:°¢(s)eA‘ds 02 0 
4» (6) = exp [W(1+ 1327)] exp W (6)] . 
41(5) = sin K+ (5 + 1325) [ln(s/K) + E'1(s/5)] - Ice"/'°, 
and 
L , /°°«- 3% a: Na §~ == 
\reffig{1-5}: Hypertext DVI viewing with dviwindo 
1.2.4 Down-translation to HTML 
We might also decide to translate our ETEX source document directly into HTML 
with tools such as ETEXZHTML (see Chapter 3) or TEX4-ht (see Chapter 4). As 
the current commonly used browsers have no built-in way to display mathematics 
properly, the translation programs transform most nonstandard characters (Greek, 
mathematical, and so on) into bitmap pictures. \reffig{1-9} is a typical example of this 
approach. The information is correctly displayed, but we must be careful to tune the 
visual representation (font height, style, and so on) of the special characters so that 

 
%%page page_39                                                  <<<---3
 
1.2 I1}'I];X in the Web environment 
17 
on. Now we can write[lZl]: 
where 
¢.(x.,»<,B’) = /;:¢(s)e*‘d: czo 
¢(s) = =*P[‘<(1+l3’1')] ex1>[w(s)l. 
ur(s) = slni<+(,s‘+ BZK)[ln(s/K)+E1(s/i<)]-Ke'“'/'‘, 
‘ and 
5(2) = /:t-‘e-‘cit (theexponential integral) 
_ 3;E_ _ 2 
"" ‘ 2: if B] 
distribution for small K, but rather the distribution of M, defined above tends to the 
distribution of the true 7t. from the Landau density function. Thus the routine GVAV IV 
’ samples the variable 7&1 rather than 7\,.. For. I: _> 10 the Vavilov distribution tends to a 
‘lwu \|4)N’)(n'\.4|\.|4| 
H avilov try 
Vavilov[I] derived a more acclirate straggling distribution by introdlicing 
kinematic limit on the maximum transferable energy in a. single collision. ra 
than using Em“ = oc. Now we can write-.[I: 
./(€163) = (AU:"31.62) 
_ 1 «+1130 
(#2.. (>...,rc,fl2) = 2% Wm q‘)(s)e’\"ds c 2 0 
M8) = exp [M1 +.K3’*r)} expi *(s)], 
sln K + (5 + fizre) [ln(s/1:) + E;(s/rc)] -- Ice"/", 
I Exuh..r{;j};Wt}.o[};}$i;5"o 
3 Vavilov theory 
Vavilov[5] derived a more accurate straggling distribution by introducing the kinematic 
limit on the maximum transferable energy in a single collision, rather than usingijm,‘ = 
f(E'78S) = %¢v(R’vrK1B2) 
The Vavilov parameters are simply related to the Landau parameter by M, = 1/K 1nK. It can be shown that as K -> 0, the distribution of the variable 7&1, approaches that of 
Landau. For K 5 0.01 the two distributions are already practically identical. Contmry to 
what many textbooks report, the Vavilov distribution does not approximate the Landau 
\reffig{1-7}: Simple PDF display (using Acrobat) 

%==========40==========<<<---2
 
%%page page_40                                                  <<<---3
 
18 
The Web, its documents, and ETEX 
Aclobol Exchange - [n.pdf] 
L I 
3 Vavilov theory 
Vavi1ov[§] derived a rnore accurate stragglirtg distribution by introducing the kinematic limit on the 
maxirmimtransfemble energy in a single collision, rather than using Em,“ = oo. Now we can 
w'1'iLa[2]: 
1 
f(€:6‘s) = (Avrxrfig) 
where 
Q9» (Au 5'32) = L/G_ém¢(5)c"’d.s o> 0 
’ ’ Zrri ‘Hm " 
¢(6) = EXP l*°(1 +1350] exp H» (5)1. 
ti (6) = 6 ‘M + (5 + 13%) [1n(5/N) + 31$/5)] - N~‘=""°. 
and 
X 
E1(2) = f f'1c"df (the exponential integral) 
Previous Next First Last Elzada Quit 
\reffig{1-8}: PDF display designed for the screen 
their bitmap images in running text and mathematical displays match. However, 
since the user of a particular browser can generally change the default font and size 
of the normal text, it is unfortunately impossible to ensure that everyone sees the 
“right" result. In addition, of course, equations cannot “reflow" when the window 
size changes since they are fixed-size pictures. 
The advantage of these systems is that with some fine tuning quite an acceptable quality can be achieved with standard browsers. Also, the cross-references, 
bibliographic citations, and hierarchical structure of the ETEX source file can be 
translated into HTML hypertext anchor functionality, allowing for optimal navigation and integration in the Web. Far-reaching customization is possible via command options and extension files. 
On the other hand, because most non-Latin symbols are translated into 
bitmaps, many thousands of GIF or PNG bitmap images may be created for a scientific document of more than a few pages (depending, of course, on the complexity 
of the mathematical content and on whether reuse of images is allowed). Because 
all of these small files have to be downloaded together with the HTML source of 
the page, the time needed to display a page in a browser can be rather long. These 
images also take a lot of diskspace, and often a modification in the ETEX source 
necessitates rebuilding the whole set of images. Finally, the installation of a léTEXto-HTML translator is not always straightforward. 

 
%%page page_41                                                  <<<---3
 
1.2 I»’}'l];X in the Web environment 
19 
3 .lVavn|nv n.}~u.tMm. Ilnlelm-.| I xpm..=.} 
[n_e2\1] [p_rs‘x] lizrgttiill [tail] [912] 
3 Vavilov theory 
Vavilov; derived a more accurate straggling distribution by introducing the kinematic limit on the 
maximum transferable energy in a single collision, rather than using Emax = 00. Now we can writeg: 
w»,~.r') = fijmots)-**ds an 
' eiyicc 
I 6 _ E‘ U V) cm = e==p[~(1+n'7)] expmsyx. 
“' `` ''`` _e'' ""' where M») = sIn~+ts+n’x)nn(s;x)+ms/x)1-~-"’“.ma 
11(1) : v/(t"¢“a’r (them nential integral) 
3v 
ll 
The Vavilov parameters are simply related to the Landau parameter by A, = A‘, /x - In 1:. It can 
be shown that as I: _=, 0, the distribution of the variable A‘ approaches that of Landau. For x 5 
0.01 the two distributions are already practically identical. Contrary to what many textbooks 
report, the Vavilov distribution does not approximate the Landau distribution for small 1:, but 
rather the distribution of )5, defined above tends to the distribution of the true A from the Landau 
density function. Thus the routine GVAVIV samples the variable 3, rather than by . For x 3 10 the 
Vavilov disl:ribution tends to a Gaussian distribution (see next section). 
\reffig{1-9}: Conversion to HTML with math as pictures 
An alternative, more lightweight approach is the one offered by Ian Hutchinson’s TtH ([=>TTH]; also available in a commercial version with support and more 
features [h>TEX2HTML]). It exploits the presence of the Symbol font in most 
browsers and translates a BTEX source into HTML, displaying “special" characters 
with built-in fonts. \reffig{1-10} shows our reference document using this technique. 
Note how the integral sign is composed by superimposing smaller line segments. 
The big advantage of the TtH program is that it is very time efficient and can 
thus be used to provide an “on-the-fly" generation of HTML documents. This procedure works well for documents that contain fairly simple math (that is, using 
characters limited to those present in the Symbol font) and has minimal maintenance costs, since only the ETEX source is needed. Documents also continue to 
display properly when the user changes font or window sizes. 
The drawback is that the quality is not as good as with the “picture" procedures. 
Above all, we cannot control the presence in the source document of characters that 
fall outside the font set available in the browser. Of course, this will no longer apply 
when Unicode support in browsers becomes universal and all the supplementary 
math characters discussed in Chapter 8 are included. Nevertheless, there remains 
the problem of the precise placement of these characters in math displays, which 
will be acceptable only when browsers are equipped with a typographic engine that 
includes a certain level of “math knowledge." 

 
%%page page_42                                                  <<<---3
 
20 
The Web, its documents, and Ié\’I]5;X 
fiunulnlmn ul I m-u1y In ulmv.)i_|IIrn| Mum I lnlr-In-.-| I xplmrr 
3 Vavilov theory 
Vavilov[§] derived a more accurate straggling distribution by introducing the kinematic limit on the 
maximum transferable energy in a single collision, rather than using Emax = co. Now we can write 
[2_]: 
1 2 
f(s,5s) = -¢v(kv,Ic,ti) 
5; 
where 
2 1 (Hm 
¢‘,(7~‘,."'-»l5 )= ¢(s)e)‘Sdsc20 
271:1 c-hw 
Ms) = expl-c(1+ti2v)] exp[w(s)1, 
w(s) = SlI1|C+(S+fi2|C)[lI1(S/|C)‘l'E1(S/|C)]-|Ce_s/K, 
and 
\reffig{1-10}: Conversion to HTML with math using Symbol fonts 
A good way to generate HTML from IATEX is to work with a modified TEX 
engine; this is what MicroPress’ commercial VTEX has [HTEXPIDER]. Since it 
writes HTML directly, all TEX macros can be handled easily, although, of course, 
the math is still rendered as bitmap pictures. 
1.2.5 Java and browser plug-ins 
The browser plug-in techexplorer understands a large subset of the IATEX language and displays a IATEX source directly inside a browser. The result is shown in 
\reffig{1-11}, and the software is discussed in Chapter 5. Although this technique 
is quite fast, a serious drawback is that the techexplorer plug-in must be downloaded and installed on all browsers you want to use. Moreover, a separate version 
is needed for each computer platform. 
Finally, you can choose to translate math content into the MathML language, 
but in this case you must have access to a browser plug-in or some other procedure that can parse and display MathML (see \reffig{1-12}). An example is WebEQ 
['->WEBEQ], based on Java applets, which can produce good quality display with 
the crucial advantage that it can adapt to resizing of the browser window size and 
fonts. The approach can, however, be rather slow. First you must wait for the initial 

 
%%page page_43                                                  <<<---3
 
1.2 I1.\'I]§X in the Web environment 
21 
u. N»-I:;: .1...» f'""[" 5 
Vavllov theory 
Vavilov[bib-VAVI] derived a more accurate straggling distribution by introducing the kinematic 
limit on the maximum transferable energy in a single collision, rather than using Em, = on. 
Now we can write[bib-SCH1]: 
f(e,6s) = %¢( 7ivc,[52) 
where 
¢.,(M.v9B2) = .‘-,.IZ;£;°§¢<s>e"*ds czo 
¢<s> = exp[v<<1+l327>]exv[w<s>]. 
Lp(S) = s1mc+(s+p2x)[1n(s/x)+E,(s/x)]-xe-W, 
and 
E1(z) = I:,t‘le"dt (the exponential integral) 
1‘, = Kl: eE:_,Y:_fi2:| 
The Vavilov parameters are simply related to the Landau parameter by M = IL, I Kln K. It can 
be shown that as K->0, the distribution of the variable M approaches that of Landau. For 
K S 0.01 the two distributions are already practically identical. Contrary to what many 
textbooks report, the Vavilov distribution does not approximate the Landau distribution for 
small it, but rather the distribution of M defined above tends to the distribution of the true '1 
from the Landau density function. Thus the routine GVAVIV samples the variable M rather than 
1,. For it 2 10 the Vavilov distribution tends to a Gaussian distribution (see next section). 
\reffig{1-11}: Display using techexplorer browser plug-in 
download of the Java applets for displaying the MathML code; that can last well 
over half a minute. Then it takes another few seconds to render all but the most 
trivial equations. So all in all you will have to wait quite some time before a page 
is ready for viewing. Alternatively, you can install the Java code on your own machine so that it is always available; even so, you have to wait while equations are 
processed. Another problem with WebEQ is that it does not strictly process BTEX, 
but rather a variant called WebTEX. 
1.2.6 Other ETEX-related approaches to the Web 
Although the previous sections describe the major approaches for using TEX documents on the Web, there are some variations that might be worth investigating. 
However, their usage is fairly limited at this time. Several implement the TEX 
\special standards of the HyperTEX project, explained in more detail in Appendix B.1. 
1. Otfried Cheong’s Hyperlatex [HHYPERLTX] is a converter from a LATEX-like 
language to HTML. It does not support math, but it has its own commands that 
extend IATEX to make it a rich language for composing advanced HTML. 

 
%%page page_44                                                  <<<---3
 
22 
The Web, its documents, and I1}'IEX 
3 Vavilov theory 
Vavilov[§] derived a more accurate straggling distribution by introducing the 
kinematic limit on the maximum transferable energy in a single collision, rather 
than using ,9 Now we can write[g]: 
may 
&infin 
f( &ep5i;,6s) = 1¢v( 7lvc,[32) Where 
r:-:'&1nfin: 
¢v( Run‘, 52 ) = 1 r:+i&intin: ‘MS khds ‘flgeqio 
¢(s) = exp[vc(1+[32'y):|exp|:qJ(s):], 
1lJ(s) = slmc+(s+[32vc)[ln(s/vc)+E1(s/K)]-ice""‘, 
and 
E1(z) = Jim:-m;£'le"d£ (the exponential integral) 
\reffig{1-12}: Conversion to HTML with embedded MathlVlL rendered with plug-in 
Garth Dickie’s idvi is a DVI viewer (supporting the HyperTEX \special 
commands) written in Java. It is used as a Web browser plug-in, rendering 
a normal DVI file within a browser window and fetching resources like fonts 
over the Web as needed. Although an elegant idea, unfortunately development 
of this program seems to have stopped, and its performance in practice is poor. 
Kasper Peeter’s nDVI [*->NDVI] is a browser plug-in that renders DVI versions 
of an HyperTEX document directly. 
Russell Quoung’s ltoh is a ETEX to HTML converter written in Perl but without any support for math. 
Although T. V Raman’s work on Aster ['->ASTER] is not strictly related to the 
Web, anyone interested in rendering math other than on paper should look at 
it. It is a system for rendering documents that contain math marked up in TEX, 
using a voice synthesizer. Designed primarily for those with impaired vision, it 
is nevertheless a very important demonstration of how math that is well marked 
up can be analyzed and rendered in nontraditional ways. 

 
%%page page_45                                                  <<<---3
 
1.3 Is there an optimal approach? 
1.3 Is there an optimal approach? 
In deciding which of the various tools just presented is most suited in a given situation, consider the following regarding your document: 
0 Is it purely text, and straightforward BTEX? Then translate it directly into 
HTML (for instance, with TtH), or start preparing to work in XML. 
9 Does it contain lots of low-level math with homegrown macros that allow you 
to set up your own customized notation? Then you would probably use TEX4ht 
or ETEXZHTML (in the latter case, be prepared to write some Perl scripts implementing your extensions) or if you prefer a commercial solution, use MicroPress’ VTEX. 
0 Does it contain lots of “normal" math? Envision translating that into MathML, 
and use one of the browser plug-ins. 
0 Does it use a lot of non-Latin characters? You would probably want to use a 
If}TEX-to-XML converter and translate the non-Latin characters into Unicode. 
0 Does it have a complex layout (tables, perhaps) or typography that is essential 
to the document reader? Use PDF. 
9 Is your document fairly self-contained? Do you have little interaction with 
other Web material? Consider an approach based on DVI. 
A real-life document probably falls into more than one of the above categories, 
so the choice of the optimal strategy might not be straightforward. To help you 
make your choice and decide which is the best approach in each case, study the material provided in the following chapters. It describes in detail the various programs 
at your disposal. 
Do not forget when considering Web publication, to allow for readers who 
cannot see, or who have sight impediments like color blindness. The work of the 
Web Accessibility Initiative ['->WAI] is to look at ways in which Web documents 
can be rendered differently, and this may affect how you present your work. 
A very considerable problem for the author creating new BTEX documents is 
how to mark up material to take advantage of common Web “goodies" like linking, 
color, forms, pop-ups, and Java applets. Each package invents new syntax for these 
extensions. BTEXZHTML, techexplorer, and the hyperref package all would 
require you to mark up your document differently, and, unfortunately, the ETEX 
world has failed until now to check this unhealthy situation. We strongly recommend that you adopt the following strategy when writing ETEX: 
o Wherever possible, use “native" ETEX syntax like \ref and \cite, which can 
be automatically translated. 
23 

 
%%page page_46                                                  <<<---3
 
24 
The Web, its documents, and I1}'IEX 
o The hyperref package covers a great deal of the necessary functionality and 
has back-end drivers for a wide variety of TEX-based hypertext systems. It 
makes sense to hedge your bets by using the syntax of this package and having access to a variety of systems. 
o If you want portable DVI output, follow the HyperTEX conventions (this is the 
default for the hyperref package), and ask the authors of your favorite DVI 
drivers to support them. 
9 If you need to use package-specific syntax, try to isolate it in ETEX macros (if 
the system supports them), so that you can easily find them should you switch 
to another system. 
Remember, ETEX and the Web are still settling down together, and it is probably best to avoid committing yourself too much to any one system at this time. 
1 .4 Conclusion 
Whereas the advent and the ready availability of the personal computer drastically 
reduced the production cost of electronic documents, the creation of the Web made 
distributing these documents worldwide a lot cheaper, easier, and faster. Taken together, these two developments have considerably changed the economic factors 
controlling the generation, maintenance, and dissemination of electronic documents. In addition, thanks to the development of the XML family of standards and 
the ubiquity of the platform-independent Java language, it is now possible to have 
a unified approach to the vast amount of information stored in databases and to 
handle their representation in various customizable forms. 
We are convinced that BTEX has an important role to play in this new and 
integrated worldwide cyberspace, especially in the area of mathematics, for text input and for rendering the output. In due course ETEX may be complemented by 
a semantically richer Mathl\/IL representation. However, If}TEX’s greatest impact 
will remain in the area of typesetting, with TEX becoming an important intermediate format for generating high-quality printable PDF output. Studying the various 
tools described in this book will ensure not only that you will be ready for the 
XML revolution, but also that you can positively contribute to the richness and increasing wealth of the scientific hyperculture by translating your ETEX documents 
into PDF, HTML, or XML using one of the proposed techniques and making them 
available on the Web. You do not have to choose between using HIEX or another 
markup technique on the Web; you can use whichever is more appropriate in a 
given situation and profit from the advantages of both. 

 
%%page page_47                                                  <<<---3
 
CHAPTER 2 
Portable Document is 
Format 
For some applications, the methods employed to disseminate information across 
the World VVide Web are unacceptable. This is because they leave the rendering 
of the “page" to the reader’s software, not to the author’s software. Even if pure 
HTML and Cascading Style Sheets are used, the author does not know where line 
breaks will occur, and, of course, there is no concept of “page breaks." Graphics 
are often presented as low-resolution bitmaps with unreliable colors; table layout 
may be radically different. The author cannot even be sure which font will be seen 
by the reader, or whether some unsuitable symbols might be used in mathematics, 
for example. Finally, and perhaps most important, the current generation of Web 
browsers is not very sophisticated at typesetting and page makeup; the result of 
hitting the Print icon from a browser does not produce a high-quality result. 
Who cares about these issues? On the one hand, lawyers may regard it vital that 
an electronic document is exactly the same as the traditional printed copy, down to 
the line breaks. On the other hand, it might simply be that an author has spent 
a lot of time making a beautiful page and wants it to be seen as such. In between 
are applications where HTML output is simply not very “nice," such as for very 
complex forms, tables, and mathematical material. 
This chapter, set up in three parts, describes a solution-Portable Document 
Format (PDF). In the first part we take a general look at what PDF is, and what 
are the issues in creating it using TEX. In the second part, we describe a special 
BTEX package (hyperref) that enables you to make enhanced PDF documents 
from BTEX source, using a variety of back-end drivers and a high-level interface 
to hypertext commands. ‘ 

 
%%page page_48                                                  <<<---3
 
26 
Portable Document Format 
The final part of the chapter describes, in some detail, a special version of TEX 
that generates PDF. Many readers, perhaps most, will not need to understand the 
new PDF primitives added to this version of TEX, since packages like hyperref 
provide a familiar ETEX interface. However, pdfTEX offers tremendous possibilities for producing advanced interactive electronic documents, and confident TEX 
programmers will want to understand what is going on under the hood. 
2.1 VVhat is PDF? 
PDF is a descendant of Adobe Systems’ PostScript language. Although Postscript 
has served as the preeminent typesetting page description language for nearly a 
decade, it suffers from age and complexity. Some crucial features (such as metadata, 
crudely implemented using comment conventions, and full prepress color) were 
added only in later revisions, and there are many small variations in implementation. More important, however, PostScript is a full-blown programming language. 
It is hard to write interpreters fast enough to use it for rapid screen display and hard 
to write displayers that can be sure each page of a text can really be shown in isolation. Display PostScript solved some of the problems, adding interaction features; 
and at the same time, Adobe’s Illustrator program developed a functional subset 
of PostScript for its internal representation. Based on this experience, it seems, 
Adobe developed a second-generation page description language, Portable Document Format; the differences between this language and PostScript are crucial: 
0 There is no built-in programming language (Adobe did add]avaScript support 
in version 3.5 of the Acrobat Forms plug-in, but this has more to do with viewer 
implementation than with the PDF language). 
0 The format guarantees page independence, clearly separating resources from 
 objects. 
0 Hypertext and security features have been added to the language, allowing sophisticated interfaces to be built. 
0 Font handling allows the font itself not to be included in the file, accompanied by just enough information for applications that display PDF, like Adobe’s 
Acrobat, to mimic the font appearance. Acrobat does this using the ingenious 
Multiple Master font technology. 
9 A great deal of effort was expended in compression features to keep the size of 
PDF files small. 
Most of the advantages of PostScript remain: PDF guarantees page fidelity, 
down to the smallest glyph or piece of whitespace, while being portable across different computer platforms. PDF is used increasingly in the professional printing 
world as a replacement for PostScript and is now in its third revision (version 1.2 of 

 
%%page page_49                                                  <<<---3
 
2.2 Generating PDF from TEX 
27 
the format). Since 1996, it has been possible to display PDF embedded in the major 
Web browsers, alongside HTML, using plug-in technology. 
It is important to make a clear distinction between Portable Document Format, 
and Acrobat. PDF is an open language whose specification is published (although 
Adobe controls it); Acrobat is a family of commercial programs from Adobe which 
produces, displays, and manipulates PDF. The main components are 
0 Reader, which is distributed freely, for simply viewing and printing PDF files; 
0 Exchange, which has all the functionality of Reader, but allows for changing 
the file; 
o Distiller, which produces PDF files from PostScript files; 
o PDFWriter, which provides printer drivers for VVindows and Macintosh, allowing any application to “print" to a PDF file directly; 
0 Catalog, which makes indexes of collections of PDF documents; and 
0 Capture, which produces PDF files by performing optical character recognition 
on bitInap-scanned pages. 
Although Acrobat Reader is free and available on many (but not all) platforms, 
there are other viewers available as well; the best-known free ones are Ghostscript 
[¢-> GSHOME], which can also produce PDF from PostScript, and Xpdf ['->XPDF]. 
Both PDF and Acrobat are generally very well documented.1 Besides the main 
PDF specification ['->PDFSPEC] and the Acrobat documentation, there are many 
commercial books. Readers of this book who are accustomed to technical material 
will find I/Veb Publishing with Acrobat/PDF (Merz, 1998) an invaluable resource for 
all aspects of PDF that will not be covered in this book. Topics that serious users of 
PDF on the Web will need to address are 
0 Optimizing PDF files to allow page-at-a-time downloading; 
o Embedding PDF in HTML pages; 
Javascript and VBScript programming in conjunction with PDF; 
0 Processing forms data in Web servers; and 
0 Dynamic creation of PDF. 
2.2 Generating PDF from TEX 
TEX users discovered PDF at an early stage, and problems relating to TEX and 
PDF creation are now well understood. There are three broad areas to cover: how 
1A notable exception is Forms, which remains an arcane area. 

%==========50==========<<<---2
 
%%page page_50                                                  <<<---3
 
28 
Portable Document Format 
to create PDF, how to ensure good-quality font rendering, and how to add extra 
information like hypertext links and navigation buttons. We will deal with these 
issues in the following sections. 
2.2.1 Creating and manipulating PDF 
PDF documents can be created in four ways: 
1. Convert existing PostScript files to PDF using a “distiller" program. The 
Adobe Acrobat Distiller is the most powerful and sophisticated of these, but 
Ghostscript also performs well, as does NikNak [¢->NIKNAK]. This approach means that you can create PDF from any application that can produce 
PostScript (that is, almost everything). ’ 
2 . Use Adobe’s PDFWriter printer driver for VVindows and Macintosh to produce 
PDF from any normal application like a word processor or spreadsheet. 
3. Use Adobe’s Acrobat Capture software to offer a workflow in which existing 
printed pages are scanned, put through an optical character recognition system, 
and the result saved as PDF. There is a clever feature by which words that 
cannot be recognized are preserved as bitmaps, resulting in a reliable-loo/eing 
PDF file that might be a mixture of real text and small bitmaps. 
4. Use an application that writes PDF directly. In the TEX world we have pdfTEX 
(see \refsec{2_4_Generating_PDF_directly_from_TeX} on page 67), MicroPress’ VTEX ['->MICROPRESS], and a DVI 
driver, Mark VVicks’s dvipdfm. 
The most common method by far is the first, since almost all text formatting software can write good PostScript these days. The second is not really recommended 
for serious work, since it gives no opportunity to add hypertext links automatically 
and gives no control over features like compression and sampling of art work. It 
is useful for quick and dirty work, however. The third method is rather specialized 
and is really suitable only for large-scale projects converting legacy documents with 
experienced staff controlling the quality. The last method is, naturally, the ideal one, 
but there are few examples of suitable applications. The reason is not hard to findAcrobat Distiller is an excellent piece of software with great flexibility, and there is 
not much incentive to develop new back-ends for applications. 
Users of Acrobat Distiller should carefully check how they set up the application. Two of the “Job Options" panelsz are particularly important: the one that 
sets compression and graphics sampling (\reffig{2-1}), and the one that determines 
which fonts are embedded (\reffig{2-2}). We will talk more about fonts in the next 
2We describe and illustrate the Windows version here, but the Macintosh version is very similar; 
UNIX users need to consult their documentation to see how configuration files need to be written, or 
how the command line options are used. 

 
%%page page_51                                                  <<<---3
 
2.2 Generating PDF from TEX 29 
A. ..,n..¢: Im II/Cal|isch5 clip!-Regular 
/CMB1 El 
\reffig{2-1}: Distiller job options for graphics \reffig{2-2}: Distiller job options for fonts 
section, but be aware that Acrobat Distiller is manipulating included bitmap figures. You need to be sure that it is doing what you expect. It defaults to a behavior 
that produces small files, and this may not necessarily be your priority. 
Manipulating PDF files after creation is beyond the scope of this chapter. Suffice it to say that there is a large and rich selection of plug-ins for Acrobat to 
perform all sorts of jobs, including prepress functions, security enhancement, and 
marking~up comments. Merz (1998), pages 53-55, has a useful summary, and there 
are extensive catalogs maintained by Adobe ['->ADOBE] and independent vendors 
(for instance, [G->PDFZONE]). 
2.2.2 Setting up fonts 
One of the most confusing issues in both PostScript and PDF is the handling of 
different types of fonts. A PDF-producing application can deal with a font in one of 
three ways: First it can take the entire font and embed it in the file; second it can 
make a subset font of just those characters used in the document and embed that 
subset; or third it can simply embed some summary details about the font (such 
as its name, its metrics, its encoding, its type-sans serif, symbol, for exampleand clues about its design) and rely on the display application to show something 
plausible. This last strategy is preferred for documents that are to be delivered on 
the Web, since it creates the smallest files. The display application can work again 
in several ways. It can try to find the named fonts on the local system; it can simply 
substitute system fonts as intelligently as possible; or it can use Multiple Master 
fonts to mimic the appearance of the original font. 

 
%%page page_52                                                  <<<---3
 
30 
Portable Document Format 
Unfortunately for TEX users, their systems have traditionally depended on the 
use of fixed resolution bitmap (that is, .pk) fonts, since TEX was established before scalable fonts were a usable reality. These are embedded in PostScript output 
as Type 3 fonts (see Goossens et al. (1997), Section 10.3 for a full description of 
PostScript’s font types). Acrobat Distiller cannot deal with these fonts intelligently 
because there are no font descriptors available. It leaves them embedded in the PDF 
file, and Acrobat renders them very poorly They would print reasonably well if the 
original resolution were high enough. 
The contrast between three types of font display can be seen in Figures 2.3, 
2.4, and 2.5. The first two figures were both set in Monotype Baskerville, but in 
\reffig{2-4}, the font was not embedded in the PDF file, so Acrobat constructed 
a Multiple Master instance to match Baskerville as best it could. \reffig{2-5} uses 
TEX’s Computer Modern, but because bitmap PK fonts were used, the result is 
almost unreadable. 
Avoiding the problem of bitmap fonts in TEX output is clearly vital. If you intend to produce good-quality PDF, you need to find Type 1 (or TrueType, although 
this format is less well supported by most DVI drivers) versions of all the fonts that 
you intend to use and then inform the driver that it should use them. How this is 
done depends on the DVI driver; Y&Y’s dviwindo and dvipsone drivers, for instance, support (except in extremis) only Type 1 scalable fonts and can access whatever is installed in Adobe Type Manager. For the widely used dvips driver (see 
Goossens et al. (1997), Section 11 for more details), it is necessary to make sure 
that the fonts are listed in the file psfonts .map or in a map file referenced by a configuration file. For instance, to ensure that Computer Modern is treated properly, 
the TEX Live ['->TEXLIVE] distribution has a dvips configuration file conf ig. cms 
that loads cms .map; that file contains the following lines: 
cmb1O CMB1O <cmb10.pfb 
cmbsy1O CMBSY1O <cmbsy10.pfb 
cmbx1O CMBXIO <cmbx10.pfb 
cmr1O CMR1O <cmr10.pfb 
cmr12 CMR12 <cmr12.pfb 
cmr17 CMR17 <cmr17.pfb 
logo1O logo1O <logo10.pfb 
logo8 logo8 <logo8.pfb 
Usage would be something like 
latex myfile 
dvips myfile -Pcms -o myfile.ps 
to prepare a PostScript file (myfi1e.ps) that can be fed through Acrobat Distiller 
to make a PDF file. 

 
%%page page_53                                                  <<<---3
 
2.2 Generating PDF from TEX 
31 
V aéifiihan ll g‘; L [.a.;;.l:‘1 
1 Introduction 
Due to the statistical nature of ionisation energy loss, large fluctuations can occur in 
the amount of energy deposited by a particle traversing an absorber element. Continuous processes such as multiple scattering and energy loss play a relevant role in the 
longitudinal and lateral development of electromagnetic and hadronic showers, and in 
the case of sampling calorimeters the measured resolution can be significantly affected 
by such fluctuations in their active layers. The description of ionisation fluctuations 
is characterised by the significance parameter 253, which is proportional to the ratio of 
mean energy loss to the maximum allowed energy transfer in a single collision with an 
atomic electron 
I Introduction 
Due to the statistical nature of ionisation energy loss, large fluctuations can occur in 
the amount of energy deposited by a particle traversing an absorber element. Continuous processes such as multiple scattering and energy loss play a relevant role in the 
longitudinal and lateral development of electromagnetic and hadronic showers, and in 
the case of sampling calorimeters the measured resolution can be significantly affected 
by such fluctuations in their active layers. The description of ionisation fluctuations 
is characterised by the significance parameter :5, which is proportional to the ratio of 
mean energ loss to the maximum allowed energy transfer in a single collision with an 
atomic electron 
K: = 6 
Enmx 
\reffig{2-4}: Display of PDF file using Multiple Master substitution for fonts 

 
%%page page_54                                                  <<<---3
 
32 Portable Document Format 
Aclobal Exchange 1 Introduction 
Duo in the statistical nature of ionisation 011(:Ig}’ loss, art-;<: fl1.1Ct-1.l«:'Li’i01l5i can 
occur in the ainoum oi energy.-' deposited by a particle L.i‘ai't-1*siu;_3; an atusorbor 
element. C'onLlnuou.~:~ procresses such as multiple scattering and ei1erg;:-' 1035 play 
at relevant role in the longiniidinai and lateral d0\-'(:lOp1'I1(:1'lt of rrlectroinagnctic 
and h€'J.tiI‘0I].i(‘. 5:liowoi‘s, and in tin? traso of saniplin,-,4‘ C‘€1..lO1‘lII1(!i’01‘$ the measured 
resolution can be sigiiifitantl;-' allotted by such l1u<'tuaL.ion.3 in L-i1Cll‘aCl,l‘i-’0lE1}'0l‘&. 
The descriptioii of ionisation fiiictiiatioiis is characterised by the sigiiifice-.nc.e 
para ncter :-.2. which is proporticzanztl no the ratio of in-van energy loss to tho. 
maximuin allowotl 4;-iiorgg,-' L.rai1:;-'ict' in a -single collision with an atomic cit-cL.ro11 
\reffig{2-5}: Display of PDF file using bitmap fonts 
Note that in the last lines of the map file, the font name is in lowercase (logo8, 
as opposed to CMRIO). This is significant and happens because the fonts come from 
different sources. Most components of Computer Modern were originally put into 
Type 1 format by Blue Sky Research in the 1980s, subsequently enhanced by Y&Y, 
and then made freely available in 1996 through an arrangement brokered by the 
American Mathematical Society; these fonts all have uppercase names. Other members of the family (added by Taco Hoekwater, for example) have lowercase names. 
Confusingly, there is another set of Computer Modern Type 1 fonts prepared by 
Basil Malyshev in the early 1990s (the first version was named Paradissa and a subsequent revision, BaKoMa). These fonts have lowercase names, and are found in 
some TEX distributions. If you are confused about which versions you have, you 
need to examine one of the pfb files and look at the copyright notice. 
Many commonly used public domain technical fonts have been converted to 
Type 1 format; among those available in TEX archives are 
c All of the Computer Modern family (including BTEX additions); 
o The American Mathematical Society fonts; 
o The St. Mary’s Road symbol fonts; 
o The RSFS script font; 

 
%%page page_55                                                  <<<---3
 
2.2 Generating PDF from TEX 
33 
o The TIPA phonetic fonts; and 
o The X]-pic fonts. 
Type 1 and TrueType versions of the “European Computer Modern" (e c) fonts are 
available from MicroPress [h>MICROPRESS]. However, there are also two alternatives 
1. The ae package provides virtual fonts that match the ec fonts as much as possible and draws on the original Computer Modern fonts. There are a few missing 
characters, like guillemets, but this package is fine for many users. 
2. The commercial European Modern font set by Y&Y [h>YANDY] is a set of 
high-quality, fully hinted fonts that can fully replace ec.3 
There is one final, but very important issue, to consider. If you use commercial fonts (for example, Adobe or Monotype fonts that you have purchased, Y&Y’s 
Lucida Bright, or European Modern), you cannot embed the entire font in a PDF 
file and then gaily make it available on the Internet. This would clearly break your 
licensing conditions because other people can extract the fonts from your file. You 
must, at a minimum, subset the fonts, and possibly (for example, in the case of small 
vendors like Y&Y to whom font piracy presents a serious threat) pay additional 
license fees. Y&Y also insists that you must change Acrobat Distiller’s subsetting 
mechanism. By default it does not subset the font if more than 35% of the characters are used. You should set this to 99% (see \reffig{2-2}) to ensure that Distiller 
always subsets unless every character is used.4 This has, besides, the desirable quality 
of making the document smaller. 
2.2.3 Adding value to your PDF 
Creating a PDF image of your normal printed page is one thing; making an electronic document that takes advantage of all the features of PDF is another. At a minimum, cross-references need to have PDF hypertext links added. However, many 
people also expect the possibility of automatic bookmarks (the optional PDF “table 
of contents" on the left side of the display), that URLs be active links, and the possibility of adding new arbitrary links. The features can be added in four ways (in 
ascending order of preference): 
1. By laboriously adding manual links in Acrobat Exchange. This option is error 
prone and has to be repeated each time the document changes. 
3This book uses the European Modern sans serif and typewriter fonts. 
4It is generally imporsilzle to use 100% of most text fonts, since the encoding vector does not usually 
let you access all glyphs contained in a font. 

 
%%page page_56                                                  <<<---3
 
34 
Portable Document Format 
2. By running an application that tries to guess linking from information in the 
file. It is not a Very reliable method. 
3. By having your application embed special PostScript code in the output that 
can be recognized by Acrobat Distiller and turned into links, for example. 
4. By having your application generate PDF code directly. This will correspond 
to the cross-reference information in the source. 
The third method is the most widely used and is well supported by Adobe. 
Acrobat Distiller recognizes a special PostScript command, pdfmark, and this is 
used as a hook to insert a vast amount of functionality into a PDF file. As an example, 
[ /Color [1 O 0] /H /I /Border [0 O O] /Subtype /Link 
/Action << /S /GOTO /D (figure.1) >> /Rect [100 254 125 266] 
/ANN pdfmark end 
creates a hypertext link at the rectangle defined by Rect to a point in the document 
named figure . 1, and 
[ /Count 0 /Action << /S /GoTo /D (section.2) >7 /Title 
(Introduction) /OUT pdfmark end 
creates a bookmark entry with the text Introduction pointing at the destination 
section. 2. It is not our intention in this book to describe the pdfmark commands, 
since the majority of users will not create them directly. However, Adobe has produced good documentation [h>PDFMARKD], Thomas Merz [%>PDFMARKP] has an 
excellent freely available primer (a chapter of Merz (1998)), and D. P. Story’s Web 
site [h>ACROTEX] has a detailed and well-presented tutorial on using pdfmark directly in TEX. 
One question that arises, however, is how these pdfmark commands get into 
the PostScript file. TEX users have it rather easy, as almost all DVI to PostScript 
drivers allow for the insertion of raw PostScript into the output stream.5 Using 
dvips, for instance, you can use \special{ps: : . . . .} to insert any PostScript 
code you like. It is, however, unlikely that you will write these commands in your 
BTEX document since it is easier to do one of the following: 
1. Use the HyperTEX \special (see Appendix B.1 on page 403) commands to insert higher-level commands, which a driver converts to the necessary pdfmark 
commands; 
2. Use driver-specific \special commands, such as those supported by dviwindo 
or VTEX; or 
5Merz (1998), Chapter 6, describes techniques for other applications like Microsoft Word and 
FrameMaker. 

 
%%page page_57                                                  <<<---3
 
2.3 Rich PDF with ETEX: The hyperref package 
35 
3. Better yet, operate at the level of generalized BTEX commands in a macro 
package, which translate to whatever mechanism is appropriate for your setup. 
The last approach is that taken by hyperref and is described in Section 2.3. In 
a similar way, users of programs like Microsoft Word, FrameMaker, and PageMaker trap their existing cross-referencing mechanisms and write pdfmark commands into the output PostScript file. The completely open and programmable 
nature of TEX makes our application particularly amenable to such an approach. 
Apart from the hyperref package, the following packages are “PDF-aware": 
0 Packages that produce \special commands according to the HyperTEX conventions, such as Michael Mehlich’s hyper, are PDF-aware. The resulting DVI 
file can be processed with the -2 option of dvips to make a rich PostScript file 
for Acrobat Distiller. 
o The ConTEXT macro package by Hans Hagen has very full support for PDF 
in its generalized hypertext features. 
0 Rich PDF from texinfo documents can be created with pdftexinfo.tex, 
which is a slight modification of the standard texinfo macros. This is part 
of the pdfTEX distribution and works only with that system. 
c A similiar modification of the webmac, called pdfwebmac . tex, allows production of hypertext PDF versions of programs written in WEB. This is also part 
of the pdfTEX distribution. 
Finally, we must not forget what in many ways is the best solution-an application that writes PDF directly. We will look at one such solution, pdfTEX, in 
\refsec{2_4_Generating_PDF_directly_from_TeX}, which provides access to all features of PDF. The pdfTEX program 
adds a number of primitives to the TEX language that can be used directly. In practice, however, most people will find it easier to continue with the familiar BTEX 
syntax supported by the hyperref package since this has a driver that maps all the 
commands to the new pdfTEX primitives. 
2.3 Rich PDF with ETEX: The hyperref package 
The hyperref package by Sebastian Rahtzé derives from, and builds on, the work 
of the HyperTEX project (see Appendix B.1 on page 403 and [H->HYPERTEX]). It 
extends the functionality of all the BTEX cross-referencing commands (including 
the table of contents, bibliographies, and so on) to produce \special commands 
that a driver can turn into hypertext links; it also provides new commands to allow 
the user to write ad hoc hypertext links, including those to external documents and 
URLs. 
6VV1th considerable help over the years from many contributors, notably David Carlisle. 

 
%%page page_58                                                  <<<---3
 
36 
Portable Document Format 
The package supports a variety of DVI drivers; they use either the HyperTEX \special commands or, if designed to produce only PDF, literal PostScript 
\special commands or pdfTEX-specific primitives. The commands are defined in 
configuration files for different drivers, selected by package options. The following 
drivers are supported: 
hypertex For DVI processors conforming to the HyperTEX guidelines (that is, 
xdvi, dvips with the -z option, OzTeX, and Textures); 
dvips Writes \special commands producing literal PostScript, tailored for 
dvips. 
dvipsone Writes \special commands producing literal PostScript, tailored for 
dvipsone. 
pdftex Writes commands for Han The Thanh’s TEX variant which produces PDF 
directly (see \refsec{2_4_Generating_PDF_directly_from_TeX}). 
dvipdfm Writes \special commands for Mark VV1cks’s DVI to PDF driver 
dvipdfm. 
dviwindo Writes \special commands which Y&Y’s VV1ndows previewer interprets as hypertext jumps within the previewer. 
vtex Writes \special commands which MicroPress’ HTML and PDF-producing 
TEX variants interpret as hypertext jumps. 
Output from dvips or dvipsone7 must be processed using Acrobat Distiller to 
obtain a PDF file. The result is generally prefererable to that produced using the 
hypertex driver and subsequent processing with the command dvips -z. The 
advantage of a DVI file written using the HyperTEX \special commands is that it 
can also be used with hypertext viewers like xdvi. 
2.3.1 Implicit behavior of hyperref 
The package can be used more or less with any normal ETEX document by requesting it in the document preamble. You must make sure it is the last of the 
loaded packages to give it a fighting chance of not being overwritten since its job is 
to redefine many ETEX commands. Hopefully you will find that all cross-references 
work correctly as hypertext, unless the implicit option is set to false, in which 
case only explicit hyperlink commands will be processed. Options control the appearance of links and give extra control over PDF output. 
\reffig{2-6} shows the result of processing our test file (see Appendix A.1), with 
hyperref defaults, to PDF; \reffig{2-7} shows the same file displayed using xdvi. 
7Both these drivers support partial font downloading; it is advisable to turn it off when preparing 
PostScript for Acrobat Distiller, since this has its own system of making font subsets. 

 
%%page page_59                                                  <<<---3
 
2.3 Rich PDF with IMEX: The hyperref package 
XL 
Am edditimnl regime is defined by the cormributimi of the collisions with 
law energy transfer which can he stirnued with the zelatim {K0, where 10 
is the mean iv:m.':s.a.t'|on pate:/Lta‘.a..l of the atom. Landeu theory assumes that the 
rumba ofthae eu1l'isi.|:ms is high and consequently, ‘it has a iastfiationé/Io >9 1. 
In GEMIT (see URL ;L_t__t_;p:j}'_:gg_ 
the limit of Landau theory has . ' 
models talcixxg ‘mic account the atomic 9.1-ueture ofthe ma.ten'.a.l are used. This 
is import-:.nt in thin layer-s and gaseous mete:-ials. shows the beluviour 
of£fI° as a function of the layer thdchnasfmmelecuon of 100 1:eV'.md1 GeV 
of ldnetia energy in Argon, Sibcon and Uranium. 
In the £ol.low1'.ng sect’:/:ms, the different thearia and models for the energy 
has fluctuatim are desaribed. Fixst, the Landau theory and its limitatiom 
an d.i.s<.-ussed, and then, the Vaxilov and Gaussian straggling functions and the 
methods in the thin layeis and gaseous materials are presented. 
I] Introduction 
[>D Landautheory 
D Vavilov theory 
D Gaussian Theory 
[> D Urbén model 
2 Landau theory 
Fin a particle ocfm.a.ss1'n.. traversing a thielmess or£m.a,ten'.a.16‘a:, the Landau proba.b‘11'1ty distribution may be written in terms of the universal La.nd.a.u function 
450.) a.s[lI]]: 
:<e.6«> = :;¢<A> 
\reffig{2-6}: Default appearance of test document in PDF enhanced with hyperref 
nummtzr at tncnae ntuustms IS mgn, mm mwmquénuy, 11. was a rmtrimratn ,;,'1r» :9 1. 
In GEAR!‘ URL htt : nwinfnuzarn. ch endow: aunt ta11.h1:n1 , 
the Limit (If Latvian theory has lmeti met at {/13 = 50. 13-91c)w this linxit sapmfial 
m-ri-dais taking into ammmt um atmnic structure of Llm mamrial are mad. This 
is inxportant in thin layers and gasemis tuaatmialx. [~’ig1rrs: _l__showa the lmhavimir 
of {/10 324 a ftmctimi of tin?! laytz-or Llxiclcmzss fur an electron of 1100 keV and 1 Get’ 
of kinetic mmrrgy in ,Argon, Ami I§ra.t£nnn., 
In the following saactimrs, film tfiffarom thmricss mu! nicnisfals {hr the amrgy 
lms fiucttratirnn am -Lkzszrrilneci. First, the Lamfau theory and its liuxitzitimns 
tiismesed, and mm, the Vmrilmr and Gmssian straggling ftmehimxs and the 
nmtlmch in the thin layers and gamns mamrixfls am pI"€=§34?.£liEt(?li. 
2 Landau theory 
Fora particle afnmss in, tr'avt,m;hxg a tllixtkxmss of naamrial 53%, the Lamiau prehahility tflznrilmtimi xnay be written in terms of tin», universal Lamiau ftmcttinu 
¢(A} as:[11: 
ram; = gatéx} 
\reffig{2-7}: Test document displayed with xdvi 

%==========60==========<<<---2
 
%%page page_60                                                  <<<---3
 
38 
Portable Document Format 
where c 2 4. From the equations (11%), (lb) and (1%) and from the conditions 
(2')) and the following limits can be derived: 
,2 E _2 
mm -_ 5  g mm : (25) 
713(b,rmx + 1) + 1: 1 r: (Em.-.x + 1) + 7131 
This (‘auditions gives 2| lowvr limit to number of the iunisnfinns n3 for whirh the 
fast sampling <','.Ln he (lune: 
71;; 2 c’ (26) 
As in the conditions (22), (23) and ('21) the value of r: is as minimum 4, one gets 
713 3 16. In order to speed the simulation, the maximum value is Iised for 11. 
\reffig{2-8}: Test document showing use of colorlinks option 
References 
11] L.L1dI1(l2l|l. On the Energy Loss of Fast I’artirl«~.s by Ionisution. ()rigiinilly publislied in J. Phys, 8.201. 1944. Rerpints-d in D.ter Hziar. Editor. 
L.[).Landzm, Callecml papers. page 417. Peigaiiioii Press, Oxford. 1965. 
i2] B.S<-horr. Programs for the Lamlau and the Vavilov distributions and tho 
corrcspmirliixg rniidoni n\inibers. Comp. Phys. Comm.. 7:216. 1974. 
S.M.Seltzer and M..].B4argc:r. Energy loss straggling of protons and iiiosuiis. 
In Slmlizts in Fe7Letmtum of Chmyltd P(1TtlL'l(:’.9 in Matter, NIiclv‘<Lr Scieiim 
Series 39. Nat. Academy of Sciences. Washington DC, 1964. 
14] R.T.'ilInmi. On cm» stzitislits of particle i<lentificat'inn using rurrmrtiorr Natl. 
Inst. Meth.. 1592189. 1979. 
[5] P.V.V:wilov. ionisation lnsses of high energy llezwy particles. Snviztt Physics 
References 
[ll L.Lan<lziu. On the Energy Loss of Fast Particles by ionisation. Originally publislicd in J, Phg/s., 8:201, 1944. Rerpinted in D.ter Haar, Editor, 
L.D.L4mdmi, Collected primers. page 417. Pergamon Press, Oxford. 1955. 
l2] B.S(-liorr. Programs for the Liunlrui and the Vnvilov distributions and the 
cones-pomling random numbers. Cvmp. Phys. Comm., 722m. 1974. L’. 
[til S.M.Seltznr and .\/1,.].Berg<-.r. Energy loss straggling of protons and Iiiesons. 
In Studies 171 Pemrtmtwn of Chmyed Pamrrlrts m Matter, Nuclear Sci. co 
S('ri(‘.S 39, Nat Aczuiumy of set, , \’l/asliingtoii DC. 1964. 1 
14] R.Talman. On the statistics of particle identilication using ionization. Nucl. 
Inst. Mrth.. 159159, 1979. 2 
[5] P.V.V;ivilov. Ioniszition losses of high energy heavy particles. Snviet Physics 
JETP. 5:749, 1957. JETP, 5:749, 1957. % 
\reffig{2-9}: Normal bibliography \reffig{2-10}: The effect of the backref option 
Two commonly used package options are 
0 colorlinks, which colors the text of links instead of putting boxes around 
them (see \reffig{2-8}, which uses gray scales instead of color); and 
o backref, which inserts extra “back" links into the bibliography for each entry. 
Figures 2.9 and 2.10 show what happens with this option set on and off. Note: 
The backref and pagebackref options can work properly only if there is a 
blank line after each \bibitem (as there is if it is created by BIBTEX). 
2.3 .2 Configuring hyperref 
All user-configurable aspects of hyperref are set using a single “key-value" scheme 
(using the keyval package with the key Hyp). The options can be set either in the 
optional argument to the \usepackage command or with the command: 
\hyperset up{/eeywzlue pzzim} 
Note that optional argument of the package command uses an experimental extension to IéTEX’s syntax. I°:TEX imposes some restrictions on the detailed content of 

 
%%page page_61                                                  <<<---3
 
2.3 Rich PDF with I13-'I}3X: The hyperref package 
39 
package options and so this method may not always work; in general, options that 
involve only letters, digits, and punctuation will be safe. 
In addition, when the package is loaded, a file hyperref . cfg is read if it can 
be found; this is a convenient place to set options on a sitewide basis. Thus the 
behavior of a particular file could be controlled by: 
c A sitewide hyperref . of g setting up the look of links, adding backreferencing, 
and setting a PDF display default: 
\hypersetup{backref, 
pdfpagemode=Fu11Screen, 
colorlinks=true} 
o A global option in the file that is passed down to hyperref: 
\documentclass[dvips]{article} 
\usepackage{hyperref} 
o File-specific options in the \usepackage commands that override the ones set 
in hyperref . cfg: 
\usepackage[pdftitle={A Perfect Day},colorlinks=fa1se]{hyperref} 
Details of all the package options are given in \refsec{2_3_8_Catalog_of_package_options} on page 62. For 
many options you do not need to give a value because they default to the value 
true if used. These are the ones classed as Boolean. The values true and false 
can always be specified, however. 
General options 
A back-end driver can be chosen using one of the options listed in \reftab{2-4} on 
 62. If no driver is specified, the package defaults to loading the hypertex 
driver. Most drivers provide the expected behavior without further ado; if you use 
dviwindo, however, you may need to redefine the following command: 
\wwwbrowser 
This command tells the dviwindo driver what program to launch; the default is 
c:\netscape\netscape.exe. Thus users of Internet Explorer might add something like the following to hyperref . cf g: 
\renewcommand{wwwbrowser}{C:\string\Program\space 
Fi1es\string\P1us!\string\Microsoft\space 
Internet\string\iexp1ore.exe} 
A number of general options, listed in \reftab{2-5} on page 62, apply to all drivers. 
The importance of the breaklinks option is demonstrated in \reffig{2-11}; this 

 
%%page page_62                                                  <<<---3
 
40 
Portable Document Format 
atom. Landau theory assumes that the number of 
these collisions is high, and consequently, it has a 
restriction E/Io >> 1. In GEANT (see URL http: // 
wwwinfo . cern.ch/asdoc/geant/geantall.h1:m1). 
the limit of Landau theory has been set at 5/10 : 
50. Below this limit special models taking into 
\reffig{2-11}: Long link text split across lines 
example was processed using pdfTEX, which has allowed the URL to break, and 
has made each part into separate links. The other drivers are unable to manage this 
trick, and the URL would have to be left protruding into the margin. 
\reftab{2-6} on page 63 lists options that also apply to all drivers but provide 
extended functionality. There are various options to specify the color of text in 
links. All color names must be defined before use, following the normal system 
of the standard BTEX color package. Users must also realize that the color of 
colored links is part of the text; if you color URLs green and then print the page 
from Acrobat, for example, the text will be printed in green (that is, a gray scale on 
a black-and-white printer). 
Using the xr package with hyperref A collection of interacting files can be 
created automatically, using the xr package. However, since either dvi or pdf versions of the results may be used, the hyperref package does not necessarily know 
which files to refer to. Consider the following file: 
\documentclass{article} 
\usepackage{xr} 
\usepackage{hyperref} 
\externaldocument{other} 
\begin{document} 
See section \ref{facts} in the other file 
\end{document} 
The label facts is defined in the file other . tex; when the current file is processed, 
it reads other . aux and makes all of its labels available for \ref in the current file 
(this is the job of the xr package). Because we have loaded hyperref, the command 
\ref {f acts} creates a hypertext link referring to the other file. But does it ask 
for other.dvi (which is what we want if we are using dviwindo, for example), or 
other . pdf (which Acrobat can open directly)? The hyperref package does its best 
to guess correctly, but on occasion you may wish to override it by specifying the file 
suffix with the extension option. 
Options specifically for making PDF files 
When the target is a PDF file, there are many options to configure the output; these 
are listed in \reftab{2-7} on page 63. 

 
%%page page_63                                                  <<<---3
 
2.3 Rich PDF with I13-'I}3X: The hyperref package 
41 
Setting link views Setting the view for links in Acrobat can be complicated; 
unlike many other hypertext systems, Acrobat associates a magnification or zoom 
value with every link. \reftab{2-1} on the following page shows the possibilities, that is, 
a set of keys with a variable number of parameters. Unfortunately it is often rather 
hard for a hyperref user to work out what values to set for these parameters; they 
have to be expressed in the PDF default coordinate space, which is not necessarily 
the same as TEX is thinking in. The good news is that pdfTEX tries to work out 
sensible values for you, supplying default parameters for the commonly used keys, 
XYZ and FitBH; the bad news is that drivers using the pdfmark system do not supply 
defaults. So, if you say 
\usepackage[dvips, pdfview=FitBH]{hyperref} 
you would normally get a catastrophic result, since FitBH must be followed by a 
number. To make life a little easier in practice, hyperref supplies a value of -32768 
as a parameter8 to the view command, if none is explicitly given; this value is ignored by the pdfTEX driver. 
The default is always XYZ followed by an appropriate value for the driver, that is, 
the magnification does not change when a link is followed. A typical change would 
be to set 
pdfview=FitBH 
so that links jump to a view that fills the window with something rational, the width 
of the text area on the current page. 
Coloring links The color of link borders in Acrobat can be specified only as three 
numbers in the range O..1, giving an RGB color. You cannot use the colors defined 
in TEX. These colors do not form part of the text and will not show when printed. 
Setting the window display The options relating to the window display are 
demonstrated in \reffig{2-12}; this example was created with 
\usepackage[ 
pdftoolbar=fa1se, 
pdfmenubar=fa1se, 
pdfwindowui=fa1se, 
pdffitwindow=true, 
pdfpagelayout=TwoCo1umnLeft 
]{hyperref} 
Because the toolbar and menu bar have been removed, any interaction for the user 
must be provided by the document itself (including basics like a Quit button); the 
8This meaningless value forces Acrobat Distiller to set a usually sensible default for some keys. 

 
%%page page_64                                                  <<<---3
 
Portable Document Format 
\reftab{2-1}: Possible values for PDF link View specifications 
Key Pammetmr Dexoription 
XYZ left top zoom Set a coordinate for the upper left corner 
of the page portion to put in the window 
and a zoom factor. If left, top, or zoom is 
null, the current value is used. Thus 
values of “null null null" specify the 
same top, left, and zoom as the current 
. A value of O is the same as null. 
Fit Fit the page to the window. 
Fit}! top Fit the width of the page to the window; 
top specifies the y-coordinate of the top 
edge of the window. 
Fitv left Fit the height of the page to the window; 
left specifies the x-coordinate of the left 
edge of the window. 
FitR left bottom right top Fit the rectangle specified by left bottom 
right top in the window. 
FitB Fit the page bounding box to the 
window. 
FitBH top Fit the width of the page bounding box 
to the window; top specifies the 
y-coordinate of the top edge of the 
window. 
FitBV left Fit the height of the page bounding box 
to the window, lefl specifies the 
x-coordinate of the left edge of the 
window. 
command \Acrobatmenu (see Section 2.3.4 on page 47) comes in very useful here. 
The menu and toolbar can, in fact, be restored manually using Ctrl-Shift-M and 
Ctrl-Shift-B key sequences, respectively. 
Acrobat bookmark commands 
The bookmark commands need further explanation. They are stored in a file called 
jolmdmenout, and you can postprocess this file to remove BTEX codes if needed. 
The bookmark text is not pmcexxed by BTEX, so any markup is passed through literally. In addition, bookmarks must be written in Adobe’s PDFDocEncoding. \reffig{2-13} shows the effect of the bookmarksopen option and also the limitations of 
PDFDocEncoding, since math cannot be displayed. To aid any editing you need to 
do, the out file is not rewritten by BTEX on the next pass if it is edited to contain 

 
%%page page_65                                                  <<<---3
 
2.3 Rich PDF with I1¥Ii;X: The hyperref package 
43 
..~..._ . ._....». 
........................... 
ma . ;.<.,.n 
. _-=[_-M.-. ,.. 
I40 - --H" Ubflt 
-. ....v6 PM "IWOPW ‘o 
\reffig{2-12}: PDF document displayed with no toolbar or interface and with multiple pages visible 
the line \let\WriteBookmarks\relax. The hyperref package does try its best to 
convert internal encoding for European accented characters to PDFDocEncoding. 
PDF document information fields The options listed in \reftab{2-8} on page 65 
allow you to put text in PDF’s information fields. \reffig{2-14} shows the display of 
document information in Acrobat; the document was created with the following 
command in the document preamble: 
\usepackage[ 
pdfauthor={Maria Physicist}, 
pdftitle={Simu1ation of Energy Loss Straggling}, 
pdfcreator={pdfTeX}, 
pdfsubject={Energy Loss}, 
pdfkeywords={physics,energy} 
]{hyperref} 

 
%%page page_66                                                  <<<---3
 
44 
Portable Document Format 
D Introduction 
VD Landautheonr 
D Restrictions 
D Vatrilovtheorgr 
D Gaussian Theory 
VD Urban model 
D Fast simulation for $n__3 geq1E$ 
D Special sampling for lower part ofthe spectrum 
\reffig{2-13}: PDF bookmarks open 
D Introduction I: 
V D Landau theory 
D Vavilov theory 
D Gaussian The 
Simulation Of Ener LOSS Stra Iin V H 
V D Urbgm modei v IGVNN ,Hm~,Xmm.W , AAA 
Eegy Loss 
\reffig{2-14}: Display of PDF document information 

 
%%page page_67                                                  <<<---3
 
2.3 Rich PDF with I5I]§X: The hyperref package 
45 
2.3.3 Additional user macros for hyperlinks 
If you need to make references to URLs or write explicit point-to-point links, the 
following set of user macros is provided. Note that it is possible to differentiate links 
and anchors by category, but the feature is not seriously exploited in hyperref . 
\hyperbaseurl{url} 
A base url, prepended to other specified URLs to make it easier to write portable 
documents, is established. 
\href {url} {text} 
The text is made into a hyperlink to the url; this must be a full URL (relative to 
the base URL, if that is defined). The special characters # and ~ do not need to be 
escaped in any way. For example, 
\hre_f {http : / /www . tug . org/ ~rahtz/ nonsense . htm1#f un}{Some fun} 
\hyperimage{image url} 
The image referenced by the image url is inserted. 
\hyperdef {category} {name} {text} 
A target area of the document (the text) is marked and given the name catego7y.name. 
\hyperre f {url} {category} {name} {text} 
The text is made into a link to url#catego7y.name. 
\hyperref [label] {text} 
The text is made into a link to a point established with a normal BTEX \label 
command with the symbolic name label (see the following description of \ref * for 
a use for this syntax.) 
\hyperl ink{name} {text} 
\hypert arget {name}{text} 
A simple internal link is created with \hypertarget, two parameters of an anchor 
name, and anchor text. The \hyperlink command has two arguments: the name of 
a hypertext object defined somewhere by \hypertarget, and the text used as the 
link on the page. 
In HTML parlance, the \hyperlink command inserts a notional # in front of 
each link, making it relative to the current document; \href expects a full URL. 

 
%%page page_68                                                  <<<---3
 
Portable Document Format 
1. The typical energy loss is small compared to the maximum energy loss in 
a single collision. This restriction is removed in the Vavilov theory (see 
St‘('ll()Il 13). 
!° 
The typical energy loss in the absorber should be large compared to the 
binding energy of the most tightly bound electron. For gaseous detectors, 
typical energy losses are a few keV which is comparable to the binding energies of the inner electrons. In such cases a more sophisticated approach 
which accounts for atomic energy levels[ri] is necessary to accurately simulate data distributions. In GEANT, B. paramelerised model by L. Urban is 
used (see section 5). 
\reffig{2-15}: The effect of \ref vs. \autoref 
\autoref {label} 
This is a replacement for the normal \ref command that puts a contextual tag in 
front of the reference. The difference is shown in \reffig{2-15}, where the first section link was made using \autoref{. . .} and the second using \ref{. . .}. The 
former has the word “section" as part of the link, whereas the latter has just the 
number. The behavior of the former is often more friendly for users than that of 
the latter. 
The tag is worked out from the context of the original \label command by 
hyperref using the macros listed in \reftab{2-2}. The macros can be redefined in documents using \renewcommand; note that some of these macros are already defined 
in the standard document classes. The mixture of lowercase and uppercase initial 
letters is deliberate and corresponds to the author’s practice. 
\reftab{2-2}: Hyperref \autoref names 
Macro Default 
\f igurename Figure 
\tablename Table 
\partname Part 
\appendixname Appendix 
\equationname Equation 
\Itemname item 
\chaptername chapter 
\sectionname section 
\subsectionname subsection 
\subsubsectionname subsubsection 
\paragr aphname paragraph 
\I-Ifootnotename footnote 
\AMSname Equation 
\the oremname Theorem 
Sometimes you might want to make a link text all by yourself and do not want 
\ref and \pageref to form links. For this purpose, there are two variant commands: 

 
%%page page_69                                                  <<<---3
 
2.3 Rich PDF with ETEX: The hyperref package 
47 
\ref *{lal7el} 
\pageref *{lal7el} 
A typical use would be to write 
\hyperref[other]{that nice section (\ref*{other}) we read before} 
where we want \ref *{other} to generate the right number but not to form a link. 
We will do this ourselves with \hyperref. 
2.3.4 Acrobat-specific commands 
If you want to access the menu options of Acrobat Reader or Acrobat Exchange, 
the following command is provided in the appropriate drivers: 
\Acrobatmenu{menuoption}{text} 
The text is used to create a button that activates the appropriate menuoption. \reftab{2-39} lists the menuoption names you can use. A comparison of this list with the 
menus in Acrobat will show what they do. Obviously some are appropriate only to 
Exchange. 
As an example, let us add a menu bar in the footer of our document, using the 
f ancyhdr package: 
\usepackage{fancyhdr} 
\usepackage[colorlinks]{hyperref} 
\pagestyle{fancy} 
\cfoot{\NavigationBar} 
\newcommand{\NavigationBar}{% 
\Acrobatmenu{PrevPage}{Previous}~ 
\Acrobatmenu{NextPage}{Next}~ 
\Acrobatmenu{FirstPage}{First}~ 
\Acrobatmenu{LastPage}{Last}~ 
\Acrobatmenu{GoBack}{Back}~ 
\Acrobatmenu{Quit}{Quit}Z 
} 
The effect is shown in the following picture: 
and 
on 
E,(z) = / t“e"dt (theexponentialinlegral) 
1 
Previous Nexl His! Las! Flat? nun 
9This table was laboriously derived by Thomas Merz, who was experimenting with Acrobat Exchange, and published in Merz (1998), since the names are not listed in Adobe documentation. 

%==========70==========<<<---2
 
%%page page_70                                                  <<<---3
 
48 Portable Document Format 
\reftab{2-3}: Acrobat menu option link names 
Acrobat Menu Available options for \Acrobatmenu 
File Open, Close, Scan, Save, SaveAs, Optimizer-.SaveAsOpt, Print, 
PageSetup, Quit 
portlmage, ImportNotes, AcroForm:ImportFDF 
ExportNotes, AcroForm:ExportFDF 
G‘€neIV'a1Iufo;Li0pen£Inf0, ontsI:nfo;: tVVeb1i,nk:Base, 
Ljwnaclnfg -- V VV -- _ V _ A : 
GeneralPrefs, NotePrefs, FullScreenPrefs, Weblink:Prefs, 
AcroSearch:Preferences (Windows) or, AcroSearch:Prefs (Mac), 
Cpt:Capture 
** C 
Export 
F1le-i>i 
F1le-> Preferences 
L Paste, Clear, ,S e1ectAfl, Ql;e:CFileV,V V 
‘V ut€s,eT0uchi V ' ii i L 
Vléérs,1TofichUf5$§how€i3:fiiti1reSu§fiécts, " 
‘ . ‘ indsuspceét, Properties 
AcroForm:Duplicate, AcroForm:TabOrder 
t;urePages, AcroI7§‘onn:ActionsL, CropPages, RotatePages, 
L Da¥eteAllT L V L L I i L 
View ActualSize, FitV1sible, FitVVidth, FitPage, ZoomTo, FullScreen, 
FirstPage, PrevPage, NextPage, LastPage, GoToPage, GoBack, 
GoForward, SinglePage, OneColumn, TwoColumns, 
ArticleThreads, PageOnly, ShowBookmarks, ShowThumbs 
£1 L V Zoon10ut:, SelectText, Se1eccGraphicsr,VNote,_ 
‘ LL ivie:Mos§efPlayer,i 
AcroSrch-.Assist, AcroSrch-.PrevDoc, AcroSrch:PrevHit, 
AcroSrch:N xtHit, Acr Srch:NextDoe 
id T061133!‘ > "C1131? 
, VI", ': 
change, HelpScan, 
HelpCapture, HelpPDFWriter, HelpDistiller, HelpSearch, 
HelpCatalog, HelpReader, Weblink:Home 
About 
1.:-nu;-«uJ._um 

 
%%page page_71                                                  <<<---3
 
2.3 Rich PDF with IMEX: The hyperref package 
49 
The text can, of course, be any arbitrary ETEX piece of typesetting. The following variation uses symbols from the ZapfDingbats font Goaded with the pifont 
package) for the same set of menu options: 
\usepackage{pifont} 
\usepackage{graphics} 
\newcommand{\NavigationBar}{{\Large 
\Acrobatmenu{PrevPage}{\ref1ectbox{\ding{227}}} 
\Acrobatmenu{NextPage}{\ding{227}} 
\Acrobatmenu{FirstPage}{\ref1ectbox{\ding{224}}} 
\Acrobatmenu{LastPage}{\ding{224}} 
\Acrobatmenu{GoBack}{\ref1ectbox{\ding{249}}} 
\Acrobatmenu{Quit}{\ding{54}}% 
}} 
with the effect as follows: 
and 
Instead of a Dingbat, we could also have used a picture, with code like 
\Acrobatmenu{Back}{\includegraphics{backpic}} 
2.3.5 Special support for other packages 
hyperref tries to cooperate with as many other package as possible, but this laudable aim is sometimes impractical. Causes of conflict are 
c Packages that manipulate the bibliographical mechanisms. Peter Williams’s 
harvard package is supported. However, the recommended package is Patrick 
Daly’s natbib package which has specific hyperref hooks to allow reliable interaction. This package covers a very wide variety of layouts and citation styles, 
all of which will work with hyperref. 
o Packages that typeset the contents of the \label and \ref macros, for example 
showkeys. The hyperref package redefines all of these commands, unless the 
implicit-.=false option is used; then these packages will not work properly. 
0 Packages that do anything serious with the index. 
The hyperref package is distributed with variants on two useful packages designed 
to work especially well with it. These are xr and minitoc, which support crossdocument links using I6TEX’s normal \label/\ref mechanism and per-chapter tables of contents, respectively. 

 
%%page page_72                                                  <<<---3
 
50 
Portable Document Format 
2.3.6 Creating PDF and HTML forms 
It is fast becoming commonplace (and even necessary) to convert paper forms to 
an electronic equivalent. Many Web pages now use HTML forms to collect data 
in very complicated ways, but it is not widely realized that PDF contains all the 
same functionality. In this section, we look at the support in hyperref for creating 
full-fledged PDF (and HTML) forms. 
Those interested in forms should keep three things in mind: 
1. Fill-in forms are not the only use for form objects in PDF. D. P. Story 
[HACROTEX] and Hans Hagen [9 CONTEXT] use them for building sophisticated interactive applications and advanced navigation. \reffig{2-16} shows Hans 
Hagen’s calculator, developed in TEX (his CONTEXt package) with embedded 
JavaScript and graphics drawn using METR POST and delivered as PDF. 
2. The current powerful forms are a relatively recent addition to PDF; to use 
them, you need Acrobat 3.01 or later and the Forms 3.5 add-ons. 
3. Few PDF-generating applications, apart from TEX, really support markupbased creation of PDF forms, and most documentation deals with creating them 
manually using Acrobat Exchange. It is also a late addition to the hyperref 
package, although the interface may need to change. Note: only the pdftex, 
dvips, and 1:ex4ht drivers support forms. 
The excellent book by Thomas Merz is required background reading for understanding how PDF handles forms (see Merz (1998), Chapters 7 and 10), and 
Story’s Web site [h>ACROTEX] has both well-designed examples and a detailed tutorial on using pdfmark (see \refsec{2_2_Generating_PDF_from_TeX} on page 27) to create forms. You should 
also keep in mind that much underlying functionality requires use of JavaScript, 
which you will need to learn. VV1th the Forms plug-in Adobe supplies a manual called Acrobat Form: ]avaScript Olject Specification; Netscape’s }avaScrz'pt Ref 
erence Manual is also helpful. PDF forms have huge potential, and hyperref only 
scratches the surface of what they can do. 
The form support in hyperref is designed to mimic that in HTML. To this 
end, it requires you to put all your form fields inside a Form environment; only one 
is allowed per file. 
\begin{Form} [parameters] 
fields 
\end{Form} 
The parameters are a set of key-value pairs, as listed in \reftab{2-9} on page 65. 
Four types of form fields are supported: 
1. Text fields, that allow free entry of text-, 

 
%%page page_73                                                  <<<---3
 
2.3 Rich PDF with HIFX: The hyperref package 
\reffig{2-16}: Calculator written in PDF (by Hans Hagen) 
2. Checkboxes, that allow a box to be selected or deselected; 
3. Choice fields, that allow the user to choose one of a range of possibilities; and 
4. Push buttons, that instigate some action. 
In addition, there are special Submit and Reset fields. 
The following six macros are used to prepare fields: 
\TextField [options] {label} 
\CheckBox [options] {label} 
\ChoiceMenu [options] {label }{cboices} 
\PushButton [options] {label} 
\Submit [options] {label} 
\Reset [options] {label} 
Note that at the top level there is no distinction drawn between 
0 simple choice menus, where all possibilities are listed; 
0 pop-up choice menu, where the default is shown, and the rest appear only when 
the field is selected; 
51 

 
%%page page_74                                                  <<<---3
 
52 
Portable Document Format 
0 combo menus, where a list of possibilities is given, but the user can type in a 
new value;1° and 
0 radio fields, where one of a list of possibilities can be checked. 
There is a large set of options (see \reftab{2-10} on page 65) that affect what the form 
fields do. 
Making a field involves supplying a textual label, possibly a list of choices, (optionally) a default, and an initial selection. The way a list of choices is presented 
depends on which options are used; the list is simply separated by commas. For 
each item, it is possible to specify a visible string separately from the value actually 
returned, if this choice is made, by supplying two strings separated by an = sign for 
the choice. The first part is what is shown; the second part is the value returned. 
Each of the three main field types (text, checkbox, and choice) consists of two 
parts-the label and the field itself. The position of the label in relation to the field 
is determined by three macros that you can redefine: 
\LayoutTextField{label}{field} 
\LayoutChoiceField{label}{field} 
\LayoutCheckboxField{label}{field} 
These macros default to #1 #2, that is, the label is set to the left of the field. A 
typical redefinition might be 
\renewcommand{\LayoutTextField} [2] {\makebox [2in] {#1}#2} 
which would set all labels within a fixed-width box, 2 inches wide. 
VVhat is actually created as the typeset area for the field is determined by 
\MakeRadi oField{width}{height} 
\MakeCheckField{width}{height} 
\MakeTextField{width}{height} 
\MakeChoiceField{width}{height} 
\MakeButtonField{text} 
These macros default to making the field a rectangle width wide and height high, 
with the field contents placed in the center. The width and height default to the 
size of the field contents but can be overridden with options (see \reftab{2-10} on 
 65). The exception is the macro for button fields, which defaults to #1; it is 
used for push buttons and for the special \Submit and \Reset macros. 
You might also need to redefine the following macros; these are used to work 
out sizes when no other information is available: 
l0This type does not appear to be allowed in HTML. 

 
%%page page_75                                                  <<<---3
 
2.3 Rich PDF with Ié'I}3X: The hyperref package 
53 
A. ...|...: I .«..r.,....,.- |n.-.n..m. mu 
F\:ll name: "b° 33 ' ‘"5 
Address: I 
rolls 
is! Mountains 
, 1 
Flvoriie part of your travels: 
Hm you sun got your: Sword D Mithril coat [] Ring! 
Do you want to: Do it all again D Pretend it never happened D Write A 
book about it 
who made the ms? 1:} 
Select funniest name, or add one 
\reffig{2-17}: Simple Acrobat form 
Macro Default 
\Def aultfleight of Submit 12pt 
\Def aultwidthof Submit 2cm 
\Defau1tHeightofReset 12pt 
\Def aultwidthof Reset 2cm 
\Def aultHeightof CheckBox 0.8\baselineskip 
\DefaultwidthofCheckBox 0.8\baselineskip 
\Def aultfleight of ChoiceMenu 0.8\baselineskip 
\Def aultwidthozf ChoiceMenu 0.8\baselineskip 
\Def aultHeightofText \baselineskip 
\Def aultwidthof Text 3cm 
Note that all colors must be expressed as RGB triples in the range 0..1 (for example, color=0 0 0. 5). In general, these options simply provide an interface to 
the relevant PDF code which is fully documented in Bienz et al. (1996). Familiarity with that document is vital if you plan to go beyond the defaults and simple 
variations. 
Now that we have all the tools described, what about a simple example? \reffig{2-17} shows a typical form, demonstrating almost all of the different field types. 
Let us look in detail at the code that produced this form. First, in the document 
preamble, we load hyperref and then start a Form environment with an appropriate URL (simply mail it). 
\documentclass{article} 
\usepackage [bookmarks=f alse] {hyperref } 

 
%%page page_76                                                  <<<---3
 
54 
Portable Document Format 
\setlength{\parindent}{Opt}\setlength{\parskip}{10pt} 
\begin{document} 
\begin{Form}[action=mailto:srahtz,method=post] 
The following entries show the different field types: 
L 
Text field. Here we supply a value for the width; by default it would be the size 
of the default value: 
\TextField[width=3in,name=xname,va1ue={Bi1bo Baggins}]{Fu11 name: } 
pun n,me._ ilbo Ba Ins 
Text field, multiline. The text color and box color are changed and a dashed 
border is drawn around the field: 
\TextField[multiline,width=1in,name=address,borderstyle=D, 
color=1 1 1,backgroundcolor=O O .5, 
value={Bag End, The Hill, Hobbiton}]{Address: } 
Address: 
Choice. By default, the height of the field would be enough to hold all the 
choices, but we limit it to showing three at a time: 
\ChoiceMenu[default=Home,menulength=3,width=2in,name=travel,default=Beorn] 
{Favorite part of your travels:} 
{Tro11s,Misty Mountains,Beorn,Mirkwood,E1ves,Laketown,% 
Smaug,The Battle} 
Trolls 
Mist Mountains 
It 
Favorite part ofyour travels: 
Checkboxes. Only one box is checked at startup: 
Have you still got your: 
\CheckBox[]{Sword} 
\CheckBox[name=coat]{Mithril coat} 
\CheckBox[name=ring,checked]{\textbf{Ring!}} 
Have you Ilill got your: Sword l_Mithr')l coal D Ring! 

 
%%page page_77                                                  <<<---3
 
2.3 Rich PDF with I5I]§X: The hyperref package 55 
5. Choice, radio style. Note here the supplying of different values shown from 
those returned by the checked box: 
\ChoiceMenu[radio,default=Again,name=next, 
borderwidth=3,bordercolor=O 1 0]{Do you want to:} 
{Do it all again=Again, 
Pretend it never happened=Forget, 
Write a book about it=Write} 
Pretend it never happened D Write A 
Do you want to: Do it all 
book about it D 
6. Text field, password style. VVhen text is entered, it is shown as asterisks: 
\TextField[password,name=made]{who made the ring? } 
Who man the fins? E'E[::l 
7. Choice, combo style. This field type allows us either to select one of the provided options or to type in a new one: 
\ChoiceMenu[combo,default=Bofur,name=whatdwarf] 
{Select funniest name, or add one} 
{Bofur,Thorin,Go11um,Smaug,Ganda1f} 
This shows the state when the field is not active: 
Select funnies! name, or Add one 
And this shows the list popping up: 
Eflflflflfl 
8. PushButton. This type activates some]avaScript code when the field is clicked: 
\PushButton[name=xxx,onclick={app.beep(0)}]{Make a horrid beep} 
9. Send 8: Clear fields. These submit the contents of the form, and clear all the 
fields, respectively: 
\Submit{Send} \Reset{C1ear} 
Finally, complete the Form environment and end the document. 
\end{Form} 
\end{document} 

 
%%page page_78                                                  <<<---3
 
56 
Portable Document Format 
D;3§9h!:\h2I2¢rrs'§!ss"°rm-W * r . 
Full name: 
Address: 
Beam 
Mirkwood , 
Favofite part of Your lxavels‘-...,t,_____._  _. 
Have you still got your: E Sword F3 Mithnl coat El Ring! 
Do you want to: ('5 Do it all again 0 Pretend it never happened C Write a book 
about it 
Who made the nng'.7 V 
i 
1 
Select fiunniest name, or add one G§,r]_d|3»H|[,_ E 
\reffig{2-18}: Simple form presented in HTML 
The same BTEX file can also be used to produce an almost identical HTML 
form (\reffig{2-18}) by specifying the tex4ht option when loading the hyperref 
package. Chapter 4 discusses how to run tex4ht to get the HTML file. 
Now that we have a form ready to interact with, how can we get at the data? 
Merz (1998), Chapter 10, deals well with this issue, and in this book we can only 
summarize the possibilities. There are essentially two ways to process the form data: 
1. If the PDF form is viewed inside a Web browser and a suitable URL is provided, 
clicking on the Send button will post the data to the URL. 
2 . There is an Acrobat menu option (File -> Export -> Form Data) that prompts 
for the name of an FDF file in which to save the data. 
The FDF file format is described in detail in Bienz et al. (1996), Appendix H; and 
Adobe makes available a toolkit for writing programs to process it. It is a simplified form of PDF, and the following example (from our simple form) shows the 
straightforward data structures: ‘ 

 
%%page page_79                                                  <<<---3
 
2.3 Rich PDF with I-KIFX: The hyperref package 
57 
ZFDF-1.2 
1 0 obj 
<< 
/FDF << /Fields [ 
<< /V (Bag End, The Hill, Hobbiton)/T (address)>> 
<< /V /Off /T (coat)>> 
<< /V /nextl /T (next)>> 
<< /V /Yes /T (ring)>> 
<< /V /Yes /T (Sword)>> 
<< /V (Mirkwood)/T (travel)>> 
<< /V_(Ki1i)/T (whatdwarf)>> 
<< /V (Bilbo Baggins)/T (xname)>> 
] 
>> 
endobj 
trailer 
<< /Root 1 O R >> 
'/.'/.EOF 
2.3.6.1 _ Validating form fields 
It is possible to write JavaScript code to perform sophisticated validation of the 
contents of form fields (Merz (1998), pages 141-144, shows how to check an ISBN 
code, for instance), but Acrobat provides a range of built-in]avaScript functions for 
simple checking. These can be accessed by giving the function name and arguments 
with the keystroke, format, validate, and calculate options. Warning: These 
functions are undocumented and unsupported by Adobe! It is possible that they 
may no longer be available in later versions of Acrobat. 
A summary of the available ]avaScript functions follows; you should enter the 
code exactly as it appears here, for example 
\TextField[name=ndwarves, 
validate={AFRange_Va1idate\string\(true, 3, true, 10\string\);}] 
{How many dwarves came along: } 
Note the distinction between Keystroke functions, which determine what the 
user is allowed to type, and format functions, which determine how it is displayed. 
Usually, both will be supplied for a given field. 
Ensure field content is entered and formatted as a percentage or a number 
AFPercent_Keystroke\string\(places, 0\string\); 
AFPercent_Format\string\(value, O\string\); 
AFNumber_Keystroke\string\(pLaces, O, 0, 0, "", true\string\); 
AFNumber_Format\string\(places, 0, 0, 0, "", true\string\); 
where places is the number of decimal places. 

%==========80==========<<<---2
 
%%page page_80                                                  <<<---3
 
58 
Portable Document Format 
Ensure field content is entered and formatted as a date 
AFDate_Keystroke\string\ ( type\string\) ; 
AFDate_Format\string\ ( type\string\) ; 
where type is a number from the following table: 
1/3 5 3-Jan-81 10 Jan 3, 1981 
1/3/81 6 03-Jan-81 11 January 3, 1981 
01/03/81 7 81-01-03 12 1/3/81Z:30pm 
01/81 8 Jan-81 13 1/3/81 14:30 
3 -Jan 9 January-81 
-l>wN>--O 
Ensure field content is entered and formatted as a time 
AFTime_Keystroke\string\( type\string\) ; 
AFTime_Format\string\ ( typ e\string\) ; 
where type is a number from the following table: 
0 14:30 
1 2.30 pm 
2 14:30:15 
3 Z:30:15pm 
Format entry in a specific way 
AFSpecial_Format\string\( type\string\) ; 
where type is a number from the following table: 
0 Zip Code 
1 Zip Code + 4 
2 Phone Number 
3 Social Security Number 
Validate numbers to a given range 
AFR.ange_Validate\string\(true, minimum, true, maa:imum\string\); 
Specify that this field is derived from others 
AFSimple_Ca1cu1ate\string\("f'unction``, ''List of field names"\string\); 
The possible Values for fimction are SUM, PRODUCT, AVERAGE, MINIMUM, and MAXIMUM; 
the list of fields must be separated by commas. 

 
%%page page_81                                                  <<<---3
 
2.3 Rich PDF with HIEX: The hyperref package 
59 
2.3.7 Designing PDF documents for the screen 
For many people, simply having a PDF version of a printed document, with active 
links, is useful enough. Others, however, have started to consider the use of PDF 
for documents whose only existence is on the computer screen. 
In Section 2.3.4 on page 47 we saw how we can access all the menu options of 
Acrobat from within BTEX; this allows us to build a complete user interface using 
all the power of TEX. 
Let us consider some of the ways in which we can design a document just for 
the screen. See Figures 2.19, 2.20 and 2.21. 
c We set up a landscape page design, which has the same aspect ratio as the computer screen; we choose 6 in X4 in; page margins are adjusted to suit. Remember 
that BTEX gives one-inch margins by default; we need to crop the page so that 
the margins are of minimal width. The ETEX oneside option must be used, 
as there is no longer any concept of odd or even pages. V/Vhen the text width is 
small, you should consider using the \raggedright setting; 
c We set the font family to one suited for screen reading; we choose Lucida 
Bright; 
0 Since there is no reason to fill pages, we set section headings to start a new 
P3569 
0 We use color; section headings are set in blue, hypertext links are colored, table 
rows are shaded, etc.; 
o The \ref commands are replaced by \autoref (or by \hyperref if a more 
customized link is needed); this makes the colored links have a more visible 
context; 
0 Should you wish to disable the Acrobat menu and toolbar, a navigation and 
function bar can be put at the bottom of each page; 
c To provide a visible pointer to the current page’s location within the article, a 
progress gauge is placed at the bottom of each page: this works by comparing 
the current page number with the number of the last page, and constructing a 
colored bar of appropriate length. 
We would also need to work out a strategy for marginal notes and footnotes, since 
these will look unnatural in a screen display. 
\reffig{2-22} shows another possible design; this one has a navigation panel on 
the right-hand side of every page, including a summary table of contents. C. V. Radhakrishnan’s pdf screen package (building on hyperref) implements this scheme. 
It has options to generate sidebar or footer menus and commands to specify logos 
and addresses to appear on every page. 

 
%%page page_82                                                  <<<---3
 
60 
Portable Document Format 
1 Introduction 
Due to the statistical nature of ionisation energy loss, large fluctuations can occur in the amount of 
energy deposited by a particle traversing an absorber element. Continuous processes such as 
multiple scattering and energy loss play a relevant role in the longitudinal and lateral development 
of electromagnetic and hadronic showers, and in the case of sampling calorimeters the measured 
resolution can be significantly affected by such fluctuations in their active layers. The description 
of ionisation fluctuations is characterised by the significance parameter is, which is proportional to 
the ratio of mean energy loss to the maximum allowed energy transfer in a single collision with an 
atomic electron é 
EIT|8.X 
Ema,‘ is the maximum transferable energy in a single collision with an atomic electron. 
5: 
Zmeflzvz 
1+ 27mg/ma. + (me/mt)" 
max = 
Previous Next First Last Back Quit 
\reffig{2-19}: Screen-designed PDF file I 
2.1 Restrictions 
The Landau formalism makes two restrictive assumptions : 
1. The typical energy loss is small compared to the maximum energy loss in a single collision. 
This restriction is removed in the Vavilov theory (see section 3). 
2. The typical energy loss in the absorber should be large compared to the binding energy of the 
most tightly bound electron. For gaseous detectors, typical energy losses are a few keV 
which is comparable to the binding energies of the inner electrons. In such cases a more 
sophisticated approach which accounts for atomic energy levels[4] is necessary to accurately 
simulate data distributions. In GEANT, a parameterised model by L. Urban is used (see 
section 5). 
In addition, the average value of the Landau distribution is infinite. Summing the Landau 
fluctuation obtained to the average energy from the dE/dz tables, we obtain a value which is larger 
than the one coming from the table. The probability to sample a large value is small, so it takes a 
large number of steps (extractions) for the average fluctuation to be significantly larger than zero. 
This introduces a dependence of the energy loss on the step size which can affect calculations. 
A solution to this has been to introduce a limit on the value of the variable sampled by the Landau 
distribution in order to keep the average fluctuation to 0. The value obtained from the GLANDO 
Previous Next First Last Back Quit 
\reffig{2-20}: Screen-designed PDF file II 

 
%%page page_83                                                  <<<---3
 
2.3 Rich PDF with I-KIFX: The hyperref package 
Emax is the GEANT cut for <5-production, or the maximum energy transfer minus mean ionisation 
energy, if it is smaller than this cut-off value. The following notation is used: 
The model has the parameters f,-, E,-, C and r (0 3 7' 3 1). The oscillator strengths f,- and the 
atomic level energies E,» should satisfy the constraints 
f1+f2 
f1lnE1+fglnEg 
1 (4) 
In I (5) 
The parameter C can be defined with the help of the mean energy loss dE/dz in the following 
way: The numbers of collisions (m, i = 1,2 for the excitation and 3 for the ionisation) follow the 
Poisson distribution with a mean number (ni). In a step Am the mean number of collisions is 
(m) = 2A1 (6) 
Previous Next First Last Back Quit 
\reffig{2-21}: Screen-designed PDF file III 
Bulletin of Mathematical Biology 
Conservation of an Ecosystem 
through Optimal Taxation 
S. V. KRISHNA. 
Department of Mathematics. 
Andhra University. 
vi akhapatnam 530 003. India 
B. KAYMAKCALAN 
Department of Mathematics. 
Middle East Technical University, 
Ankara, 06531 Turkey 
In this paper we study a harvesting problem in the presence of a predator and a 
tax. The objective is to maximize the monetary social benefit as well as prevent 
the predator from extinction. keeping the ecological balance, 
(0 1998 Society for Mathematical Biulogy 
\reffig{2} .22: Alternate screen-designed PDF file (courtesy of River Valley Technologies and Focal Image Ltd.) 

 
%%page page_84                                                  <<<---3
 
62 Portable Document Format 
2.3.8 Catalog of package options 
\reftab{2-4}: hyperref options to specify drivers 
Option Value Default Description 
draft boolean false All hypertext options are turned off. 
pdftex boolean L Sets 3up.11yp§rra£ for use with the pdfI‘EX program. 
dvipdfm boolean Sets up hyperref for use with the dvipdfm driver. 
nativepdf. boqleafi V Anaiiasvfori dvipg, ‘ 7 g 7 ~ _ V 
An alias for dvips. 
pdf mark boo 
S '' Semup hypperref for use with the dvips driver. 
dvips 
hypertex Sets up hyperref for use with HyperTEX-compliant drivers. 
dviwindo as ;: Sets up hyperjztef afar use the dviwindo _ i 
dvipsone Sets up hyperref for use with the dvipsone driver. 
vtex moleen tsiu ihypeprpref for use with fMicroPress" VTEX; the PDFand . 
s S - e T L S i : léerids are detected automatically. 
|atex2htm| boolean Redefines mpatibility with BTEX2 HTML. 
te.x4ht hooiiuaiu‘ " 
\reftab{2-5}: Configuration options 
Option Value Default Description 
brea klinks boolean false Allows link text to break across lines. Since this cannot be 
accommodated in PDF, it is set true only by default if the pdftex 
driver is used. This makes links on multiple lines into different 
PDF 1' ks to the same target. 
extension text Set the file extension (for example, dvi), which will be appended to 
file links that are created if you use the xr package. 
implicit boolean true g i 1“ internals are redefined toproduce ls‘ T . 
Iinktocpage boolean false Sets the table of contents hyperlinks to be on the page number not 
on the text of the entry. 
nesting . Vboolean V fizlre to be nested; no drivers currently ince 
_yyytherHT1vILnoriPDFaI1owsit. ' p _ . _ _, _ 
Determines whether every page is given an implicit anchor at the 
top left corner. If this option is turned off, \tableof contents will 
not contain hyperlinks 
anchor boolean 
plainpages boolean * y 
eight of links is usually calculated by’ 
the driver simply as the baseline of contained text; this option 
forces \special commands to reflect the real height of the link 
(which could contain a graphic). 
raiselinks boolean 

 
%%page page_85                                                  <<<---3
 
2.3 Rich PDF with E'I]§X: The hyperref package 63 
\reftab{2-6}: Extension options 
Option Value Default Description 
backref boolean false Adds backlink text to the end of each item in the bibliography as a 
li of tion n b s. 
hyperindex boolean false Makes the text of index entries into hyperlinks. This is fairly fragile, 
and a serious project would need to implement its own scheme. 
cltecolor color green Color for blbllgraphical citations 111 text. 
menucolor color red Color for Acrobat menu items. 
,  W ; . 
urlcolor color cyan Color for linked network URLs. 
\reftab{2-7}: PDF-specific display options 
Option Value Default Description 
a4pa per boolean true Paper size is set to 210 mm X 297 mm. 
3: 
boolean 
Iegalpa per 
bookmarks boolean false Write a set of Acrobat bookmarks, in a 
manner similar to the table of contents, 
requiring two passes of ET X 
bookmarksnumbered boolean false If Acrobat bookmarks are requested, 
include the section numbers. 
0 
Set the PDF View for each link. 

 
%%page page_86                                                  <<<---3
 
64 Portable Document Format 
PDF-specific display options (cont) 
. 1 V . 
pdfstartpage name Set the page number on which the’ PDF file 
.. K ogisyapeoneayygg . L _ K 
pdfstartview name Fit Set the initial page view. 
.pdfh‘ighfigl§t " /I T Determine how link buttons behave when 
e o _ selected / I is for‘111jve1'se (tlacoglefar It}; the 
iiii _ 111 . , 
and /P 
7 (outline), 
citebordercolor RGB color 0 1 0 The color of the box around citations. 
fiiebordercolor j H RGB, color 0 .5 .5 The color of the box around links to files. 
linkbordercolor RGB color 1 0 0 D The color of the box around simple links. 
menuboréercoior ; . s 7“ color *1 “ 00 . L The color of ythegboxam Acrobat-menu 
bordercolor RGB color 1 1 0 The color of the box around links to pages. 
uribordercolor o a RGB color 0 1 I The color of the box around links to URLs. 
pdfpagescrop n n n n Set the default PDF crop box for pages. 
This should be a set of four numbers, like a 
PostScript BoundingBox. 
0,70 1' D styleyof aroundlinks. Defaultsto" a D 
D T D ’ boxewiith lines of lpt thickness, but the D T‘ T 4 
, coloxlinks option resets it toproduce no 
a border. ii ‘ 
pdftoolbar boolean false Determine whether the viewer’s toolbar is 
visible when the document is opened. 
pdfmenubar boolean fizlse o Detennine Whether the viewers menu bar L 
o if D H o T is visible. if 
pdfwindowui boolean false Determine whether the user interface 
elements in the document’s window are 
visible. 
, bqolean L o a false ; Determine whether the shouldfl _ y 
i “ it - s * r T resize the window displaying the dowment 
to fit the size of the first displayed pagelof“ 
pdfcenterwindow boolean false Determine whether the viewer should 
position the window displaying the 
document in the center of the computer’s 
monitor. 
pdfvagéfayoute D name ‘ SiagiePzzge me. layouts tl1‘e-‘tl1e l V. y . 
T document isiopiened. The possibilities are 
V V . T listed in \reftab{2-11} on page 66. 
pdfnewwindow boolean false Determine whether links that open another 
PDF file should start a new window or 
replace the contents of the current window 
with the new file.“ 
pdffitwihdowr “Under Windows, the new window is placed in exactly the same place on the screen as the existing window. To make them 
both visible at the same time, you have to use the VVindow->Tile . . . menu item; then adjust the windows by hand with the 
mouse, or use Ctrl-Tab. 

 
%%page page_87                                                  <<<---3
 
2.3 Rich PDF with HIEX: The hyperref package 65 
PDF-specific display options (cont) 
\reftab{2-8}: PDF information options 
Option Value Default Desmption 
baseurl URL Sets the base URL of the PDF document. 
. . S . d 
pdfauthor text T Sets the document mformation Author field. 
pdfcreator text Sets the document mformatlon Creator field. 
B OCllI‘Il€l'1tlI1 OI'I‘IlatlOI1 €yWOI' S B 
\reftab{2-9}: Form environment options 
Option Value Default Desmptian 
action URL The URL that will receive the form data if a Submit button is included in 
the form. 
method name Values can be post or get; this is used only when generating HTML (see 
[&>HTML4], Section 17.3). 
\reftab{2-10}: Forms options 
Option Value Default Description 
accesskey key (As per HTML) 
backgroundcolor RGB color 1 I 1 Color of field box. 
bordersep V H : content an bor er. 
borderwidth number 1 Width of box border (in points) 
charsize dimen Iopt Font Slze of field text. 

 
%%page page_88                                                  <<<---3
 
Portable Document Format 
Forms options (cont) 
color RGB color 0 0 0 Color of text in box. 
combo boolean false VVhether choice list is combo style. 
default Default value for field 
disabled boolean false Whether field is disabled. 
form at Javascript code to format the entry. 
height dimen Height of field box. 
hidden boolean false VVhether field is hidden. 
keystroke _]avaScn'pt code to control the keystrokes on entry. 
maxien number 0 Number of characters allowed in text field (0 means 
unlimited). 
menulength number 4 Number of elements shown in list. 
muitiiine boolean false VVhether text box is multiline. 
name name Name of field (defaults to the label contents). 
onbiur _lavaScript code. 
oncha nge Javascript code. 
onciick Javascript code. 
ondblclick _]avaScn'pt code. 
onfocus Javascript code. 
onkeydown Javascript code. 
onkeypress Javascript code. 
onkeyup Javascript code. 
onmousedown _lavaScript code. 
on mousemove Javascript code. 
onmouseout Javascript code. 
onmouseover Javascript code. 
onmouseu p _lavaScript code. 
onselect _lavaScript code. 
password boolean false VVhether text field is password style. 
popdown boolean fizlse VVhether choice list is pop-down style. 
radio boolean false VVhether choice list is radio style. 
readoniy boolean false VVhether field is read only. 
tabkey (as per HTML) 
validate _lavaScript code to validate the entry. 
value Initial value for field-not the same as the default. 
width dimen Width of field box. 
\reftab{2-11}: Acrobat page layout display options 
Name Description 
Sing1ePage Display a single page. 
Dnecolumn Display pages in a column. 
TwoColumnLeft Display the pages in two columns, with odd-numbered pages on the left. 
Twocolumnfiight 
Display the pages in two columns, with odd-numbered pages on the right. 

 
%%page page_89                                                  <<<---3
 
2.4 Generating PDF directly from TEX 67 
\reftab{2-12}: Acrobat page transition options 
Name Keg/(5) Desmption 
Split /Dm, /M Two lines sweep across the screen to show the new page; the lines can be either 
horizontal or vertical and can move from the center out or from the edges in. 
X green, appear and synchronously 
``aLg'':e.* '2,;a¢t.1.i.H£=.S-are either 
E 
*3: 
1’. 
5 
a 
§ 
. ,swfeep in the same directioii an (reveals a 
. horizontal or vertical. T X t E 
Box /M A box sweeps from the center out or from the edges in. 
wipe E T . . single line sweeps across the screen fromene edge 1;o,th,e._0ther, revealing 
Dissolve The page image dissolves in a piecemeal fashion to reveal the new page. 
T Gitttteil Similar t_o,Dissoive', except sweeps across imag in a wide band 
" mtponegside of the screensto themher; '. X 
The Keg/(5) dictate how the effect appears, for example pdfpagetransition={Blinds /Dm /V} 
/Di (Direction) The direction of movement, in degrees (counterclockwise). Values are generally in 90° steps. 
/Dm (Dimension) If a choice between horizontal or vertical is allowed, value is / H (horizontal) or /V (vertical). 
/M (Motion) If an effect can be from the center out or from the edges in, value is / I (in) or /0 (out). 
2.4 Generating PDF directly from TEX 
The purpose of Han The Thanh’s pdfTEX project" was to create an extension of 
TEX that can create PDF directly from TEX source files and possibly actually enhance the result of TEX typesetting with the help of PDF. The pdfTEX program 
contains TEX as a subset: VVhen PDF output is not selected, pdfTEX produces normal DVI output, otherwise it produces PDF output that looks identical to the DVI 
output. The next stage of the project is to investigate alternative justification algorithms, possibly making use of multiple master fonts. We will not, however, discuss 
that aspect of the program in this book. 
The pdfTEX program is based on the original TEX sources and web2c and has 
been successfully compiled on UNIX, Macintosh, Amiga, VV1n32, and DOS systems. 
2.4.1 Setting up pdfTEX 
The pdfTEX program [*-> PDFTEXS] is distributed with many of the free TEX packages," including MikTeX and fpTEX for Windows 32 , teTEX for UNIX, CMacTeX 
12We are grateful to Han The Thanh for considerable help with this section. 
13At the time of writing, none of the commercial TEX vendors have adopted pdfTEX, although MicroPress (VTEX) has its own direct PDF-generating TEX engine. 

%==========90==========<<<---2
 
%%page page_90                                                  <<<---3
 
68 
Portable Document Format 
for Macintosh, and the general web2c system on which most of these packages are 
based. 
In addition to the normal TEX fonts and macros, a pdfTEX distribution consists 
of the following items: 
pdftex . pool pool file, needed for creating formats; 
ttf2afm an external program to generate AFM files from TrueType fonts, needed 
to create TEX font metric files; 
pdftex. cfg pdfTEX configuration file (see \refsec{2_4_1_1_The_pdfTEX_configuration_file}); and 
map PostScript and TrueType font maps (see \refsec{2_4_1_3_Map_files}). 
VVhen pdfTEX is running, some extra search paths beyond those normally requested by TEX itself are used: 
VFFONTS the path where pdfTEX looks for virtual fonts; 
T1FONTS the path where pdfTEX looks for Typel fonts; 
TTFONTS the path where pdfTEX looks for TrueType fonts; 
PKFONTS the path where pdfTEX looks for PK fonts; and 
TEXPSHEADERS the path where pdfTEX looks for its configuration file 
(pdftex.cfg), font mapping files (map), encoding files (enc), and graphics 
files (see \refsec{2_4_2_3_Graphics_inclusion}). 
2.4.1.1 The pdfI‘EX configuration file 
When pdfTEX starts, it reads a configuration file called pdftex. cf g, searched for 
in the TEXPSHEADERS path. Because web2c systems commonly specify a private tree 
for pdfTEX where configuration and map files are located, this allows individual 
users or projects to maintain customized versions of the configuration file. It also 
means that individual TEX input files need not set any pdfTEX-specific macros. 
The configuration file is used to set default values for the following parameters, 
all of which can be overridden in the TEX source file: 
output_format Integer parameter specifying the output. A value greater than 
zero means PDF output, otherwise DVI output. 
compress_level Integer parameter specifying the level of text compression (using 
zlib). Zero means no compression, 1 means fastest, 9 means best, 2.8 means 
something in between. 
decimal_digits Integer parameter specifying the precision of real numbers in PDF 
code. Valid values are in the range 0..5. A higher value means more precise 
output, but it may also mean a larger size and more time taken to display or 
print. In most cases the optimal value is 2. 

 
%%page page_91                                                  <<<---3
 
2.4 Generating PDF directly from TEX 
69 
ima e_resolution Inte er arameter s eci 'n the default resolution of bitma 
E E P P E P 
image files that contain no resolution information. 
_width, page_height Dimension parameters specifying the page width and 
 height of PDF output. If not specified, then page width is calculated by 
taking the width of the box being shipped out and adding 2 X (horigin + 
\hoff set). The page height is calculated in a similar way. 
horigin, vorigin Dimension parameters specifying the offset of the TEX output 
box from the top left corner of the “paper." 
map The name of the font mapping file (similar to those used by many DVI to 
PostScript drivers); more than one map file can be specified, using multiple map 
lines. If the name of the map file is prefixed with a +, its values are appended 
to the existing set, otherwise they replace it. If no map files are given, a default 
file psfonts .map is searched for. 
A typical pdftex. cfg file looks like the following. It sets up output for A4 paper 
size and the standard TEX offset of 1 inch, and loads two map files for fonts. 
output_format 1 
compress_level O 
decimal_digits 2 
_width 210mm 
_height 297mm 
horigin lin 
vorigin 1in 
map standard.map 
map +cm.map 
2.4.1.2 Setting up fonts 
The pdfTEX program normally works with Type 1 and TrueType fonts; a source 
must be available for all fonts used in the document, except for the 14 base fonts 
supplied by Acrobat (the Times, Helvetica, Courier, Symbol, and Dingbats families). It is possible to use METRFONT-generated fonts in pdfTEX. It is strongly 
recommended, however, not to do so if an equivalent is available in Type 1 or TrueType format, since the resulting Type 3 fonts render very poorly in current versions 
of Acrobat. Given the free availability of Type 1 versions of all the Computer Modern fonts and the ability to use standard PostScript fonts without further ado, this 
is not usually a problem. 
2.4.1.3 Map files 
The pdfTEX program reads map filex specified in the configuration file (see \refsec{2_4_1_1_The_pdfTEX_configuration_file}), in which reencoding and partial downloading for each font are specified. Every font needed must be listed, each on a separate line, apart from PK fonts. 

 
%%page page_92                                                  <<<---3
 
70 
Portable Document Format 
The syntax of each line is similiar to the dvips map files" and can contain up to 
six space-separated fields: texmzme, baxemzme, fimtjlagx, fiomfile, encoding, and xpecial. 
The only mandatory field is texmzme, and it must be the first field. The rest of the 
fields are optional, but if baxemzme is given, it must be the second field. Similarly, if 
fomflagx is given, it must be the third field (if baxemzme is present) or the second field 
(if baxemzme is left out). It is possible to vary the positions of fmtfile, emodingfile, and 
xpecial, but the first three fields must be given in fixed order. 
texname: the name TEX uses, that is, the name of the TFM file. This must be 
present. 
basename: the PostScript font name. If not given, then it will be taken from the 
font file. Specifying a name that does not match the name in the font file will 
cause pdfTEX to produce a warning, so it would be best not to use this field 
if the font resource is available (this is the most common case). This option is 
primarly intended for use of base fonts and for compatibility with dvips map 
files. 
fimtflags: flags specifying some characteristics of the font. The following description is taken (with some modification) from the PDF specification (Bienz et al. 
(1996)), Section 7.9.2 (Font descriptor flags). 
The value of the Flags key in a font descriptor is a 32-bit integer that contains a collection of Boolean attributes. These attributes are true if the corresponding bit is set to 
1 in the integer. The following specifies the meanings of the bits, with bit 1 being the 
least significant. Reserved bits must be set to zero. 
Bit position Semantics Sample 
1 Fixed-width font Sample Text 
2 Serif font Sample Text 
3 Symbolic font 
4 Script font »3.A.M.7’.C8 T8967’ 
5 Reserved 
6 Uses the Adobe Standard Roman 
Character Set 
7 Italic Sample Yext 
8-16 Reserved 
17 All-cap font SAMPLE TEXT 
18 Small-cap font SAMPLE TEXT 
19 Force bold at small text sizes 
2 0-3 2 Reserved 
All characters in a fixed-width font have the same width, while characters in a proportional font have different widths. Characters in a serif font have short strokes drawn 
“Most dvips map files can be shared with pdfTEX without problems. 

 
%%page page_93                                                  <<<---3
 
2.4 Generating PDF directly from TEX 
at an angle on the top and bottom of character stems, while sans serif fonts do not 
have such strokes. A symbolic font contains symbols rather than letters and numbers. 
Characters in a script font resemble cursive handwriting. An all-cap font, typically used 
for display purposes such as titles or headlines, contains no lowercase letters. It differs 
from a small-cap font in that characters in the latter, while also capital letters, have been 
sized and their proportions adjusted so that they have the same size and stroke weight 
as lowercase characters in the same typeface family. 
Bit 6 in the flags field indicates that the font’s character set is the Adobe Standard Roman Character Set, or a subset of that, and that it uses the standard names for those 
characters. 
Finally, bit 19 is used to determine whether bold characters are drawn with extra pixels 
even at very small text sizes. Typically when characters are drawn at small sizes on very 
low resolution devices such as display screens, features of bold characters may appear 
only one pixel wide. Because this is the minimum feature width on a pixel-based device, 
ordinary nonbold characters also appear with one-pixel-wide features and cannot be 
distinguished from bold characters. If bit 19 is set, features of bold characters may be 
thickened at small text sizes. 
If no font flags are given, pdfTEX treats the font as 3, a symbol font. If we do 
not know the correct value, it would be best not to provide one; specifying a 
wrong Value of the font flags may cause Acrobat some problems. 
fimgfile: the name of the font source file. This must be a Type 1 or TrueType font 
file. The font file name can be preceded by one or two special characters that 
say how the font file should be handled. 
o If it is preceded by a <, then the font file will be partially downloaded, meaning that just those glyphs (characters) used in the document are extracted 
and put into a new subset font, which is then embedded in the output. 
This is the most common use and is strongly recommended for any font, as it 
ensures portability and reduces the size of the PDF output. 
0 If the font file name is preceded by a double <<, the whole font file will 
be included-all glyphs of the font are embedded, including the ones that 
are not used in the document. Apart from increasing the size of the PDF 
file, this option may cause problems with TrueType fonts too, so it is not 
recommended. It is useful in cases where the font is somehow strange and 
cannot be subsetted properly by pdfTEX. 
o If no character precedes the font file name, the font file is read, but nothing is embedded. Only the font parameters are extracted to generate a font 
descriptor that is used by Acrobat to simulate the font if needed. This Option is useful when we do not want to embed the font (that is, to reduce 
the output size) but do wish to use the font metrics and let Acrobat generate an instance that looks close to the original font, provided that resource 
is not installed on the system where the PDF output is viewed or printed. 
To use this feature, the font flags must be specified and have bit 6 set on, 

 
%%page page_94                                                  <<<---3
 
72 
Portable Document Format 
meaning that only fonts with the Adobe Standard Roman Character Set 
can be simulated. The only exception is the Adobe Symbol font, which is 
not very useful. 
0 If the font file name is preceded by a !, the font is not read at all and is 
assumed to be available on the target system. This option can be used to 
create PDF files that do not contain any embedded fonts. The PDF output 
then works only on systems where the font is available. It is not very useful 
for document exchange, because the file is not portable. On the other hand, 
it is very useful when we wish to speed up the running of pdfTEX while 
testing a document. This feature requires Acrobat to have access to all the 
installed fonts, including those that are only in the TEX support tree. 
Note that the standard 14 fonts are never embedded, even if they are marked 
for download in map files. 
encoding: name of a file containing an encoding vector to be used for the font. 
The file name may be preceded by a <, but the effect is the same. The format 
of the encoding vector is identical to that used by dvips (see Goossens et al. 
(1997), Section 11.2.4). If no encoding is specified, the font’s built-in default 
encoding is used. It may be omitted if we are sure that the font resource has 
the correct built-in encoding. In general, this latter option is highly preferred 
and is required to subset TrueType fonts. 
special: special instruction for font transformation as for dvips. Only specifications of SlantFont and Ext endFont are read; other instructions are ignored. 
If pdfTEX cannot locate a font in a map file, it will look first for a source with 
the extension pgc, a PGC source (PDF Glyph Container).15 If no PGC source is 
available, pdfTEX will try to use PK fonts in the same way as normal DVI drivers. 
Lines containing nothing apart from texmzme indicate that a scalable Type 3 
font should be used. For font types as Type 1, TrueType, and scalable Type 3, all 
requests for the font at any size will be provided by just one font in the PDF output. 
Thus if a font, for example, csr10, is listed in a map file, it will be treated as scalable. 
The font csr1O will be downloaded only once for csr10, csr1O at 12pt, etc. 
It does not hurt much if a scalable Type 3 font is not listed in a map file, except 
that the font source will be downloaded multiple times for different sizes, meaning 
the PDF output is larger. On the other hand, if a font is listed in a map file as scalable 
Type 3 and its PGC source is not scalable or not available (in this case pdfTEX will 
use PK fonts instead), the PDF output will be valid. However, some fonts may look 
ugly because bitmaps will be scaled. 
15 This is a text file containing a PDF Type 3 font, usually created using METFI POST with some 
utilities by Hans Hagen. In general, PGC files can contain whatever is allowed in a PDF page description 
to support fonts. At present PGC fonts are not very useful, since vector Type 3 fonts are not displayed 
very well in Acrobat. They may be more useful when Type 3 font handling gets better. 

 
%%page page_95                                                  <<<---3
 
2.4 Generating PDF directly from TEX 73 
Some sample map file entries 
Use a built-in font with font-specific encoding, that is, neither a downloaded font 
nor an external encoding is given. SlantFont is specified in the same way as for 
dvips. 
psyr Symbol 
psyro Symbol ".167 SlantFont" 
Use a built-in font with an external encoding (8r. enc). The < preceding the 
encoding file name may be omitted. 
ptmri8r Times-Italic <8r.enc 
ptmro8r Times-Roman <8r.enc ".167 SlantFont" 
Use a partially downloaded font with an external encoding: 
putr8r Utopia-Regular <8r enc <putr8a pfb 
putro8r Utopia-Regular <8r.enc <putr8a.pfb ".167 SlantFont" 
Use the Type 1 font name taken from the downloaded font itself: 
logo8 <logo8.pfb 
Adjust the width but not the stroke thickness: 
logodlo logobflo <logobf10.pfb ".913 ExtendFont" 
Use entire downloaded font without reencoding: 
pgsr8r Gillsans <<pgsr8a.pfb 
Use partially downloaded font without reencoding: 
pgsr8r Gillsans <pgsr8a pfb 
Do not read the font at all--the font must be available on the target system: 
pgsr8r Gillsans !pgsr8a.pfb 
Use an entire downloaded font with reencoding: 
pgsr8r GillSans <<pgsr8a.pfb 8r.enc 
Use a partially downloaded font with reencoding: 
pgsr8r Gillsans <pgsr8a.pfb 8r.enc 

 
%%page page_96                                                  <<<---3
 
74 
Portable Document Format 
Do not include the font, but extract parameters from the font file and reencode:16 
pgsr8r GillSans 32 pgsr8a.pfb 8r . enc 
Use a TrueType font in the same way as a Type 1 font: 
verdana8r Verdana <verdana.ttf 8r.enc 
2.4.1.4 TrueType fonts 
As we have seen, pdfTEX can work with TrueType fonts, and adding the font names 
to map files is straightforward. The only extra task for TrueType fonts is to create 
TFM font metric files. There is a program, ttf2afm, in pdfTEX distributions that 
can be used to extract AFM font metrics from TrueType fonts. Usage is simple: 
ttf2afm ttf-fi Le [encoding] 
The name of the TrueType font file is ttf-file, and the optional encoding specifies an 
encoding file, which is the same as those used in map files for pdfTEX and dvips. 
If the encoding is not given, all the glyphs in the AFM output will be mapped to 
/ . notdef. The ttf2afm program writes the output AFM to standard output. From 
this we can make a TFM from the AFM file (Goossens et al. (1997, Section 10.5)). 
If we need to know which glyphs are available in the font, we can run ttf2afm 
without encoding to get all the glyph names. 
To use a new TrueType font (times .ttf), the minimal steps (assuming that 
test .map is included in pdftex . cfg) on a UNIX system might be 
ttf2afm times.ttf 8r.enc >times.afm 
afm2tfm times.afm -T 8r.enc 
echo "times TimesNewRomanPSMT <times.ttf <8r.enc" >>test.map 
The PostScript font name, TimesNewRomanPSMT, is reported by afm2tfm 
but is not strictly needed in the pdfTEX map file. 
Ext endFont and SlantFont also work for TrueType fonts. 
2.4.2 New primitives 
The pdfTEX program adds a set of new primitives to TEX; they are described in the 
following sections, and allow the user access to features of the PDF format. 
16This only works for fonts with Adobe Standard Encoding. The font flags say what this font is like, 
so Acrobat can generate a similar instance if the font resource is not available on the target system. 

 
%%page page_97                                                  <<<---3
 
2.4 Generating PDF directly from TEX 
75 
2.4.2.1 Document setup 
\pdfoutput=n 
This integer parameter specifies whether the output format should be DVI or PDF. 
A value greater than zero means PDF output, otherwise DVI output. This parameter cannot be specified afier shipping out the first page. In other words, it must 
be set before pdfTEX ships out the first page if we want PDF output. This is the 
only parameter that must be set to produce PDF output; all other parameters are 
optional. 
\pdf compresslevel=n 
This integer parameter specifies the level of text compression via zlib. Zero means 
no compression, 1 means fastest, 9 means best, 2..8 means something in between. 
A value out of this range will be adjusted to the nearest meaningful value. 
\pdfpagewidth=dimen 
\pdfpagehe i ght =dimen 
These dimension parameters specify the page width and page height of PDF output. 
If they are not given, the page dimensions will be calculated as described in Section 
2.4.1.1. 
\pdfpagesattr={tokens} 
This token list parameter specifies optional attributes for every page of the PDF 
output file. These attributes can be MediaBox (rectangle specifying the natural size 
of the page), CropBox (rectangle specifying the region of the page being displayed 
and printed), and Rotate (number of degrees the page should be rotated clockwise 
when it is displayed or printed-must be 0 or a multiple of 90). 
\pdfpageattr={tokens} 
This is similiar to \pdfpagesattr, but it takes priority over it. It can be used to 
overwrite any attributes given by \pdfpagesattr for individual pages. 
2.4.2.2 The document information and catalog 
\pdf inf o{infio keyx} 
This allows the user to add information to the document info section; if this is 
provided, it can be seen in Acrobat Reader with the menu option File-v> Document 
Info-x» General. The info keyx parameter is a set of data pairs (a key, and a value). 
The key names are preceded by a /, and the values are in parentheses; all keys 

 
%%page page_98                                                  <<<---3
 
76 
Portable Document Format 
are optional. The possible keys are /Author, /CreationDate (defaults to current 
date), /ModDate, / Creator (defaults to “TeX"), / Producer (defaults to “pdfTeX"), 
/Title, /Subject, and /Keywords. 
/CreationDate and /ModDate are expressed in the form D:YYYYMMDDhh1nmss, 
where YYYY is the year, MM is the month, DD is the day, hh is the hour, mm is 
the minutes, and ss is the seconds. 
Multiple uses of \pdf info are permitted; if a key is given more than once, the 
first appearance will take priority. An example of the use of \pdf info follows: 
\PDFinfo{ 
/Title (examp1e.pdf) 
/Creator (TeX) 
/Producer (pdfTeX) 
/Author (Tom and Jerry) 
/CreationDate (D:19980212201000) 
/ModDate (D:19980212201000) 
/Subject (Example) 
/Keywords (cat;mouse) 
} 
\pdfcatalog{catalog keyx} openaction {action} 
The document catalog is similar to the document info section, and the available 
keys are /URI, which provides the base URL of the document, and /PageMode, 
which determines how Acrobat displays the document on startup. The possibilities 
for the latter are: 
/UseNone Open document with neither outlines nor thumbnails visible. 
/Useoutlines Open document with outlines visible. 
/ Us eThumbs Open document with thumbnails visible. 
/Fullscreen 0 en document in full-Screen mode. In full-screen mode, 
P A 4 
there are no menu bar, wmdow controls, or any other wmdow 
present. 
The default is /UseNone. 
The action is the action to be taken when opening the document; it is specified in the same way as for internal links (see \refsec{2_4_2_6_Destinations_and_links}), for example 
goto page 3 {/Fit}. 
2 .4.2 .3 Graphics inclusion 
\pdf image width width height laeiglat depth dept/7 {filemzme} 
Insert an image, optionally changing the width, height, depth, or any combination 
of these attributes. The default values are zero for depth and the image’s natural 
size for height and width. If all of them are given, the image will be resized to fit 

 
%%page page_99                                                  <<<---3
 
2.4 Generating PDF directly from TEX 
77 
the specified values. If some of them (but not all) are given, the rest will be scaled 
proportionally to keep the aspect ratio the same as that of the natural size. If none 
of them is given, then the image will be set at its natural size. The dimension of 
the image can be accessed by putting the \pdf image command into a TEX box and 
checking the dimensions of that box. 
The image type is determined by the extension of the file name. Thus png 
means PNG format and pdf means it is a PDF file; otherwise, the image is treated 
as JPEG. 
\pdf imageresolut ion=resolution 
If the image is a bitmap file and contains resolution information, then that is used; 
otherwise, \pdfimagereso1ution can be used to specify it. The default is 72 dpi. 
\pdf f ontpref ix{prefix string} 
\pdf imagepref ix{prcfix string} 
\pdfformpref ix{prr.fix string} 
Sometimes there are problems including a PDF file as an illustration, because of 
conflicts in fonts, image names or PDF form objects. These three commands allow 
you to change the default prefixes for names. So if an included PDF file has a font 
resource named / F34, and you find it conflicts with an / F34 in the current file, you 
can use \pdffontpref ix to name it, for example, /FF34. Using these commands 
is recommended only for experts. 
2.4.2.4 XObject Forms 
\pdf form number 
writes out the TEX box number as an XObject Form to the PDF file. 
\pdflastform 
returns the object number of the last XObject Form written to the PDF file 
\pdfrefform \name 
puts in a reference to the XObject Form called \na.me. 
These macros support “object reuse" in pdfTEX. The content of the XObject 
Form object corresponds to the content of a TEX box, which can contain text, 
pictures, and references to other XObject Form objects. The XObject Form can 
be used by simply referring to its object number. This can be useful in a large 
document with a lot of similar elements, since it avoids the duplication of identical 
objects. A common example is a document style that places an identical graphic or 
text in the header of every page. 

%==========100==========<<<---1
%==========100==========<<<---2
 
%%page page_100                                                  <<<---3
 
78 
Portable Document Format 
2.4.2.5 Annotations 
\pdfannot width width height height depth dept/7 {text} 
attaches an annotation at the current point in the text. The annotation content will 
be raw PDF code, as specified in text. 
\pdf last annot 
returns the object number of last annotation created by \pdfannot. These two 
primitives allow the user to create any annotation that cannot be created by 
\pdf annotlink (see following). 
2.4.2.6 Destinations and links 
\pdfdest ( num {num} I name {name} ) appearance 
establishes a destination for links and bookmark outlines. The link must be identified by either a number or a symbolic refname and the way Acrobat is to display the 
 must be specified. Appearance must be one of the following: 
f it fit whole page in window 
fith fit whole width of page 
fitv fit whole height of page 
fitb fit whole Bounding Box page 
fitbh fit whole width of Bounding Box of page 
fitbv fit whole height of Bounding Box of page 
xyz keep current zoom factor 
xyz can optionally be followed by zoom factor to provide a fixed zoom-in. The factor 
is like TEX magnification; that is, 1000 is the “normal" page view. 
\pdfannotlink height {beig/at} dep th, {dept/7} attr {attr} action 
starts a hypertext link. If the optional dimensions are not specified, they will be 
calculated from the box containing the link. The attributex (explained in great detail 
in Section 6.6 of the PDF manual) determine the appearance of the link. Typically 
they are used to specify the color and thickness of any border around the link. 
Thus /C [O.9 O 0] /Border [0 0 2] specifies a color (in RGB) of bright red 
and a border thickness of 2 points. 
The action can do many things; some of the possibilities are listed in \reftab{2-13}. 
\pdf endl ink 
ends a link; all text between \pdfannotlink and \pdfendlink will be treated as 
part of this link. The pdfTEX program may break the result across lines (or pages), 

 
%%page page_101                                                  <<<---3
 
2.4 Generating PDF directly from TEX 
\reftab{2-13}: PDF link actions 
Action Efirect 
 n Jump to page n. 
gotoynum flamber V V’ 
goto name {refname} Jump to a point established as number or name 
wi 
goto file {filename} " t 
thread num {number} 
thread name {refiuzme} Jump to thread identified by 
user {spec} Perform user-specified action. Section 6.9 of the 
PDF manual explains the possibilities. A typical 
use of this is to specify a URL, for example, / S 
/URI /URI (http: //www . tug. org/) 
in which case it will make several links with the same content. 
2.4.2.7 Bookmarks 
\pdf outline action count {count} {text} 
creates an outline (or bookmark) entry. The first parameter specifies the action to 
be taken and is the same as that allowed for \pdfannotlink. The count specifies the 
number of direct subentries under this entry; it is zero if this entry has no subentries 
(in which case it may be omitted). If the number is negative, then all subenu‘ies will 
be closed, and the absolute value of this number specifies the number of subenu‘ies. 
The text is what will be shown in the outline window; note that this is limited to 
characters in the PDFDocEncoding vector. 
2.4.2.8 Article threads 
\pdfthread num {num} name {name} 
starts an article thread. The corresponding \pdfendthread must be in the box in 
the same depth as the box containing \pdf thread. All boxes in this depth level will 
be u‘eated as part of this thread. An identifier (number or refname) must be specified; 
threads with the same identifier will be joined together. 
\pdfendthread 
finishes the current thread. 

 
%%page page_102                                                  <<<---3
 
80 
Portable Document Format 
\pdf threadhof f set =dimen 
\pdf threadvof f set =dimen 
specify thread margins. 
2.4.2.9 Miscellaneous 
\pdf lit eral {paf text} 
Like \special in normal TEX, this command inserts raw PDF code into the output. It allows support of color and text Uansformation and is used in the standard 
graphics package’s pdft ex driver. 
\pdfna.mes{data} 
puts dam in the names dictionary in the catalog. 
\pdf obj stream {text} 
is similar to \pdf literal, but the text is inserted as contents of an object. If the 
optional keyword stream is given, the contents will be inserted as a stream. 
\pdflastobj 
returns the object number of the last object created by \pdfob j. These primitives 
provide a mechanism allowing insertion of a user-defined object in PDF output. 
\pdftexversion 
returns the version of pdfTEX multiplied by 100; for example, for version 0 . 13b it 
returns 13. 
\pdftexrevision 
returns the revision of pdfTEX; for example, for version 0 . 13b it returns b. 
2.4.3 Graphics and color 
The pdfTEX program supports inclusion of pictures in PNG, JPEG, and PDF 
format. The most common technique-the inclusion of Encapsulated PostScript 
figures-is replaced by PDF inclusion. EPS files can be converted to PDF by 
Ghostscript, Acrobat Distiller, or other PostScript-to-PDF convertors. The bounding box of a PDF file is taken from the CropBox if available, otherwise from 
MediaBox. To get the right MediaBox, it is necessary to u‘ansform the EPS file 
before conversion so that the start point is at the (0,0) coordinate and the page size 

 
%%page page_103                                                  <<<---3
 
2.4 Generating PDF directly from TEX 
81 
is set exactly corresponding to the BoundingBox. A Perl script [<->EPSTOPDF] for 
this purpose has been written by Sebastian Rahtz. 
Other alternatives for graphics in pdfTEX are 
0 BTEX picture mode Since this is implemented simply in terms of font characters, it works in exactly the same way as usual. 
0 )Q-pic If the PostScript back-end is not requested, )§(-pic uses its own Type 1 
fonts and needs no special attention. 
0 tpic The tpic \special commands (used in some macro packages) can be 
redefined to produce literal PDF, using macros by Hans Hagen. 
o METRPOST Although the output of METRPOST is PostScript, it is in a 
highly simplified form, and a METF\POST-to-PDF conversion (written by Hans 
Hagen and Tanmoy Bhattacharya) is implemented as a set of macros that read 
METRPOST output and support all of its features. The type mps is supported 
by the ETEX graphics package for this purpose. 
The last two macro files are part of the ConTEXT macro package (supp-pdf . tex 
and supp-mis . tex), but they also work with ETEX and are available separately. 
The inclusion of raw PostScript commands-the technique utilized by the 
pstricks package (Goossens et al. (1997), Chapter 4)-cannot be supported." 
Although PDF is a direct descendant of PostScript, it lacks any programming language commands and cannot deal with arbiu‘ary PostScript. 
The standard BTEX graphics and color packages have pdftex options, 
which allow use of normal color, text rotation, and graphics inclusion commands. 
The implementation of graphics inclusion makes sure that however often a graphic 
is used (even if it is used at different scales or transformed in different ways), it is 
embedded only once. 
A number of samples of pdfTEX output can be found on the TUG Web server 
[<->PDF1“EXEX]. 
17This technique can be used with Micr0Press’ VTEX, which has a built-in PostScript interpreter. 

 
%%page page_104                                                  <<<---3
 
CHAPTER 3 
The ETEX2 HTML 
translator 
In this chapter we take a closer look at the HIEXZHTML n'anslator. It uses Larry 
Wall’s Perl language, together with other publicly available tools, to interpret HIEX 
source code and n'anslate it into hyperdocuments for viewing on the Web. 
After a short historic overview and a reminder of the basic principles of generating documents for the Web, we turn our attention to the various components of 
the system-highlighting installation, customization, and extension mechanisms. 
Support for mathematics in HTML is almost absent, hence we discuss in detail 
three basic math modes of BTEXZHTML and its extensions to optimize the display of mathematics on the Web. Then we go on to see how HIEX2 HTML handles 
non-English source documents, dealing with the u‘anslation of title and keywords, 
various encodings, special fonts, preprocessors, and so on. Last, but not least, we 
describe how you can use BTEXZHTML as a real hypertext production tool for 
HIEX sources by using extensions defined in the html package. 
3 .1 Introduction 
BTEX, based on the TEX typesetting system, has more extensive capabilities than 
most word-processing software. Its principal documentation (Lamport (1994)) 
presents as a “Document Preparation System." Similarly, HIEXZHTML is 
much more than a Stave As HTML option for HIEX; it is better described as a “Web 
Document Preparation System." 

 
%%page page_105                                                  <<<---3
 
84 
The IMZHTML translator 
ETEX2 HTML is free software, distributed under the GNU Public Licence. For 
the examples in this chapter we used version 99.1. It runs under Linux, OS/2, Windows NT, Windows 95, and DOS, as well as most flavors of UNIX, for which it was 
initially developed. 
By default the HTML pages produced conform to the HTML 3.2 specification. 
Thus the pages are readable in browsers that implement this or the more recent 
HTML 4.0 specification. If necessary, one can instead choose to produce pages that 
conform to HTML 2.0. This restriction uses images for the more complicated ETEX 
environments, such as tables and alignments within mathematics. Alternatively one 
can choose to generate extra CLASS and ID attributes according to the HTML 4.0 
specification. They allow style sheet information to be associated with particular 
environment types, or even with specific instances of environments, paragraphs, 
and snippets of text. Furthermore, a large number of European languages and dialects are supported, with LANG attributes being generated, for HTML 4.0, when 
several languages are used in a single document. 
BTEXZHTML is written in Larry Wall’s Perl language (VVall et al. (1996)), so 
that in principle BTEXZHTML runs on every system on which Perl has been installed, although some other software tools are needed before BTEXZHTML can 
initiate a Uanslation. Like BTEX, Perl and these other software applications are 
publicly available. Most of them probably came with the operating system when 
you bought your machine, were installed by the system manager on clustered PCs, 
or are readily available on CD-ROM for almost all computer platforms. Missing 
components can be easily obtained via the Web, as explained in Section 3.2. 
3.1.1 A few words on history 
Initial development of Ié‘TEX2HTML was done by Nikos Drakos. In 1995 Drakos 
released his program to the Internet community, via the I1}TEX2HTML mailing 
list [‘->L2HLIST]. Toward the end of 1996 a repository for use by developers was 
established at the University of Darmstadt in Germany. Since then many individuals have contributed code, provided optimizations for various platforms, worked 
on portability, and improved installation procedures. Many of these developments 
have been in direct response to requests received via the mailing list, to support 
particular ETEX commands and packages. 
Documentation, support for ETEX packages, strategies for image generation 
and extensions for the latest HTML specifications, and overall coordination have 
been handled mainly by Ross Moore.1 
3.1.2 Principles for Web document generation 
From the earliest days of its development, the World Wide Web was recognized 
as a means whereby highly technical, structured information could become more 
1A full list of contributors can be found in the ETEX2 HTML Users Guide. 

 
%%page page_106                                                  <<<---3
 
3.1 Introduction 
easily accessible. TEX and BTFX are used primarily with technical fields, where the 
data itself has far more importance than the way in which it is presented. Thus it is 
a natural setting for a u‘anslation tool aimed at presenting structured data in a way 
that is both readable and exploits the structure for easy navigation. 
The first versions of BTEXZHTML were developed by Nikos Drakos to 
take advantage of the newly emerging World Wide Web that could be used for 
computer-based learning and education. The information to be presented was always of foremost importance. Drakos also recognized the need for easy navigation 
through the presented information in ways that mesh with its logical structure. He 
listed several principles that he considered necessary for any serious software to 
generate Web documents [‘-> DRAKOSWWW]: 
1. automatic creation of structure-based hypertext webs; 
2. flexibility in specifying the desired node granularity; 
3. automatic generation of navigation aids; and 
4 
. inclusion of highly formatted information such as figures, tables, mathematical 
equations, diagrams, chemical formulae, and exotic languages. 
Based on these principles, the earliest versions of BTEXZHTML were quite 
effective, even with just the limited capabilities of the early versions of HTML 
['-> HTMLZSPEC]. With extensions introduced by later HTML specifications, these 
principles have been retained within the design of the BTEXZHTML u‘anslator; indeed, they have been firmly embraced and extended. Now all the environments that 
are commonly used for paper documents using BTEX are converted automatically 
into HTML structures and markup, giving the highest-quality results that can be 
reasonably achieved with a designated version of HTML. 
Principle 1 is, of course, the main goal of a conversion tool like BTEXZHTML. 
Principle 2 is implemented using configuration variables and command-line options that are described in detail in the BTEXZHTML Users Guide (Drakos and 
Moore (1998)). Examples are the $MAX_SPLIT_DEPTI-I and $MAX_LINK_DEPTH variables and -link and -split options (see Sections 3.2.3.2 and 3.2.3.3). 
A clear instance of principle 3 (and, of course, principle 1) is the example in 
Section 3.2.1 where we show how a BTEX source file is automatically translated by 
I§'I]3‘,X2HT ML into a hyperlinked web of HTML files, with navigation aids to move 
around easily. 
The present chapter contains many examples that illustrate how I£}TEX2HTML 
copes with principle 4, mainly for presenting mathematics, which is indeed highly 
structured. Many of the techniques applicable to mathematics apply in other situations as well. We will see that IIYTEXZHTML is more than just a translation tool. 
Using packages, configuration variables, and command-line options, we will see 
how the information in the same ETEX source can be presented as a collection of 
HTML pages in different ways. A specific translation can be tailored to suit considerations such as the version of HTML, capabilities of specific browsers, download 
times for the intended audience, and a document’s relation to other Web pages. 
85 

 
%%page page_107                                                  <<<---3
 
86 
The ETEXZHTML translator 
Documents can be constructed using “conditional code," allowing the best results 
for both the paper version, typeset with ETEX, and the HTML version, prepared 
using BTEXZHTML. These capabilities justify the designation of ETEXZHTML as 
a genuine “Web Document Preparation System." 
Another important application of principle 4 is presenting languages using exotic alphabets and scripts. Transliteration schemes are frequently used to express 
the languages used on the Indian subcontinent in a way that can be transmitted 
by computers. There are various pieces of preprocessing software that convert text 
coded via transliteration schemes into source suitable for typesetting using TEX and 
BTEX, exploiting special fonts. Several of these are supported in a fully automatic 
way by ETEX2 HTML. 
Not so much envisioned by Drakos at the beginning, but an important aspect of 
IQTEXZHTML today, is using it as a tool for creating hypertext documents. Therefore, ETEXZHTML comes with a macro package called html . sty, which contains 
markup extensions in the form of new HTML-related commands and environments 
(see Section 3.5). 
3.2 Required software and customization 
The latest ETEXZHTML software can be obtained from several distribution sites; 
the main ones are in California [<->L2HSA], Germany [<->L2HSC], and the CTAN 
archives [‘->L2HCTAN]. Development versions can be found at the developers’ 
site [<->CVSREPOS]. Distributed versions are named 1atex2htm1-yy_vv.tar.gz, 
where yy corresponds to the year, and vv to the version. For instance, in spring 
1999 the file 1atex2htm1-99_1 . tar . gz was the latest distributed version. 
The following list provides other pieces of software required for a standard 
BTEXZHTML installation. These are not provided as part of the distribution and 
must be obtained separately. 
Perl Version 5 (5.003 or later on some platforms) is required. BTEXZHTML also 
needs the database management module. See www.per|.com [‘->PERL] for all 
things Perl, including how to update to the latest version. 
HIEX When HTML is not sufficient for an adequate representation of some portion of the information, an image is created. Typesetting that piece of source 
using ETEX is the first step in the process of producing this image. 
dvips Converts DVI output from ETEX into a set of PostScript files. 
Ghostscript Interprets the PostScript files, rendering them as images using a 
high-quality bitmapped format. At the least version 4.02 is required to allow 
ETEXZHTML to produce good-quality “anti-aliased" images. For the latest 
version, consult the Ghostscript Home Page [<-> GSHOME]. 

 
%%page page_108                                                  <<<---3
 
3.2 Required software and customization 
netpbm Suite of graphics utilities. Each performs a task in the processing of a 
graphic image, such as cropping, background transparency, rotation, color reduction, interlacing, and conversion to a specific output format. Get the archive 
from the X Windows FTP site [*->NETPBM]. 
Other utilities For special purposes, such as processing transliterations of Indic 
scripts, ETEX2 HTML has special support available for preprocessors. 
The scripted nature of BTEXZHTML means that it has great flexibility. It is a 
basic principle in its continuing development to make use of software that already 
exists for specialized tasks. The tools just listed have the added advantage of being 
available at no cost. Other tools could be used, for instance, for graphics processing. 
It is straighforward to write Perl code to make a new piece of software that 
performs a vital task better, faster, or simply in a different way. Incorporating other 
software to work with BTEXZHTML amounts to the following tasks: 
1. Recognize which parts of the document source constitute information that 
needs to be processed using the new software. 
2. Extract this information and cast it into a form suitable for input to the specialized software. 
3. Prepare the correct command line to launch the software application for processing the requisite information. 
4. Include its output into the HTML pages being constructed, either as tagged text 
or as hyperlinks to external files such as graphic images. Some postprocessing 
may be required. 
The Perl programming language is well suited to performing these tasks. Its 
sophisticated pattern-matching abilities are ideal for recognizing particular forms 
of information contained within a larger document. 
3.2.1 Running ETEXZHTML on a ETEX document 
To process a document written for ETEX using ETEXZHTML, it is usually 
sufficient to issue a command such as the following. The extension . tex is optional. 
latex2html <fiZename>[.tex] 
Multiple files can be processed with a single command, using command-line options to specify where the resulting HTML pages are to be located and how they 
should be hyperlinked: 
latex2html <options> filel file2 file3 
87 

 
%%page page_109                                                  <<<---3
 
88 
The IMEXZHTML translator 
Z5oaoNo«w.AwN... 
Z 3 I A w N 
5 
E 
N 
o 
N 
N 
N 
w 
an 
nu»; 
o«vzA 
N 
\.I 
wNN 
oxoao 
ww 
N... 
~.. 
~.. 
wwwwww 
\oao\:o«v.A 
3 3 8 
A 
w 
i 
A 
v. 
3 
A 
\.I 
A 
an 
A 
o 
U: 
o 
mm 
N... 
u. 
~.. 
v. 
A 
kn 
v. 
Knvuknkn 
xoaoxxox 
ox 
o 
E 
\documentc1aaa{article} 
\uaepackage{graphicx} 
\uaepackage{francais} 
\uaepackage{makeidx} 
\newcommand{\Lcs}[1]{\texttt{\eymbo1{’134}#1}} 
\makeindex 
\title{Exemp1e d’un article en fran\c{c}ais} 
\author{Miche1 Gooesena} 
\begin{document} 
\maketitle 
\tab1eofcontents 
\1iatoffigures 
\1istoftab1ea 
\section{Une figure EPS} 
\index{aection} 
Cette section montre comment inclure une figure PostScript\cite{bib-PS} 
dans un document \LaTeX. La figure~\ref{Fpafig} 
est ins\’er\’ee dans le texte \‘a 1’aide de la commande 
\verb!\includegraphica{colorcir.epe}!. 
\index{figure} 
\index{PoatScript} 
\begin{figure} 
\begin{center} 
\begin{tabu1ar}{c@{\qquad}c} 
\includegraphics[width=3cm]{colorcir} & 
\includegraphics[width=3cm]{tac2dim} 
\end{tabu1ar} 
\end{center} 
\caption{Deux images EPS} 
\label{Fpafig} 
\end{figure} 
\section{Exemp1e d’un tableau} 
Le tab1eau~\ref{tab:exa} \‘a la page \pageref{tab:exa} 
montre 1’uti1isation de 1’environnement \texttt{tab1e}. 
\begin{tab1e} 
\begin{center} 
\begin{tabu1ar}{cccccc} 
\Lca{primo} & \primo & \Lca{aecundo} & \eecundo & \Lca{tertio} & \tertio \\ 
\Lcs{quatro} & \quatro& 1\Lca{ier} & 1\ier & 1\Lca{iere} & 1\iere \\ 
\Lcs{fprimo)}&\fprimo)& \Lcs{No} 10 & \No 10 & \Lca{no} 15 & \no 15 \\ 
\Lcs{og} a \Lc5{fg}&\og a \fg&3\Lca{ieme}&3\ieme & 10\Lcs{iemes}& 10\iemea 
\end{tabu1ar} 
\end{center} 
\caption{Que1quee commandes de 1’option \texttt{francais} de \texttt{babe1}} 
\label{tab:exa} 
\index{tab1eau} 
\end{tab1e} 
\begin{thebib1iography}{99} 
\index{r\’ef\’erences} 
\bibitem{bib-PS} 
Adobe Inc. 
\emph{PostScript, manuel de r\’ef\’erence (2\ieme \’edition)} 
Inter\’Editions (France), 1992 
\end{thebib1iography} 
\printindex 
\index{index} 
\end{document} 
\reffig{3-1}: Example l§TEX source file to be translated by BTEX2 HTML 

%==========110==========<<<---2
 
%%page page_110                                                  <<<---3
 
3.2 Required software and customization 
89 
Ex<~111pl<* 111111 111111 lv «~11 fr;111g.11s 
-1-1» 1 12.1.1111...-.1... 
...1.~ .11. 1 .. .1 .1... 1......" .1. ......1 
111111111 Gm»...-1..~ ‘ 
111111.11 ms: 2 Exemple d’un tableau 
1.. ...1.1.»... 1 ..1.. M. 2 1.1.1....» 1 ..:.1....u..1. .1.- 1‘1~1.».1.11.111~11.»11~ um. 
Table des xnatiéres 
1 11... 115...: 1:12: 1 
Références 
11111.10»... 1... r...1s...,.1 ..........1 .1. ../1-1...». ¢m.1..m1 1......L.1.»....., 
2 Exemyle an... 1.1.»... 2 “mm W2 
Table des figures 
. 1)<>11v.EI‘€ 1 
Liste des tableaux 
1 u11~1.....~ . ...1......1.1.s 41.1 |‘u1vr1«11rinncz-1):1k-baht! 2 
1 Une figure 1a1>s 
11.. 1 1.1.1.1 l111r11:r~El’S' 
\reffig{3-2}: Formatted PostScript output of example file shown in \reffig{3-1} 
To show how simple it is to run BTEXZHTML let us look at an example with 
a file 12hexa. tex, written in French. \reffig{3-1} shows the BTEX source file. On 
lines 2-4 we include three packages: graphicx, f rancais of babel, and makeidx. 
We also define a new command, \Lcs, that prints a backslash followed by its argument (line 5). Then, after printing the title with \maketitle, we desire a table of 
contents and lists of figures and tables (lines 10-13). The body of the text consists 
of two sections (lines 14-31 and 33-50). It is followed by a bibliography (lines 5258) and the index, which is generated from the various \index commands by the 
makeindex program and is included with the \printindex command (line 59). In 
the first section we have two graphics side by side (lines 25-26), while the second 
section shows several commands of babe1’s francais package in tabular form. 
If we run ETEX on this source file, l2hexa.tex, then run makeindex on 
12hexa. idx, and finally run BTEX a second time to resolve cross-references, and 
include the index file 12hexa. ind, we obtain as a result the three pages shown as 
PostScript files in \reffig{3-2}. 
To generate HTML files that correspond to this ETEX source file, it is enough 
to type: 
latex2html 12hexa 
To see what is happening, I/}TEX2HTML prints the following information on 
the user’s console. We are running with the standard settings of the system and 
will show in later sections how you can control one or more characteristics of the 
generated output files. 

 
%%page page_111                                                  <<<---3
 
90 
The IIYIEXZHTML translator 
8 e m u m m A w N _ 
This is LaTeX2HTML Version 99.1 release (March 30, 1999) 
by Nikos Drakos, CBLU, University of Leeds. 
Revised and extended by: 
Marcus Hennecke, Ross Moore, Herb Swan and others 
...producing markup for HTML version 3.2 
Loading /afs/.cern.ch/asis/src/Tex/archive/latex2html/versions/html3_2.pl 
ttt processing declarations *** 
OPENING /afs/cern.ch/project/tex/lwc/babel/l2hexa.tex 
Loading /afs/.cern.ch/asis/src/Tex/archive/latex2html/styles/texdefs.perl... 
Loading /afs/.cern.ch/asis/src/Tex/archive/latex2html/styles/article.perl 
Loading /afs/.cern.ch/asis/src/Tex/archive/latex2html/styles/graphicx.perl 
Loading /afs/.cern.ch/asis/src/Tex/archive/latex2html/styles/francais.perl 
Loading /afs/.cern.ch/asis/src/Tex/archive/latex2html/styles/makeidx.perl . . . . . . .. 
Reading ... 
'/.,++ . . . . . . . . . . . .. 
@@@@@@@@@@@@@@@@@@@@@@@ 
Reading /afs/cern.ch/project/tex/lwc/babel/l2hexa.aux ... 
Processing macros ...++ . . . . . . . . . . . . . . . . . . . . . . . . . . . 
Reading /afs/cern.ch/project/tex/lwc/babel/l2hexa.lof ... 
Processing macros ...++.... 
Reading /afs/cern.ch/project/tex/lwc/babel/12hexa.lot ... 
Processing macros ...++ . . . . .. 
Translating ... 
0/8:top of l2hexa: for l2hexa.html 
*** translating preamble *** 
*** preamble done *** 
£31/8:tableofcontents:.."Table des mati\‘{e}res" for node1.html 
£32/Bzlistoffigures:..``Liste des figures'' for node2.html 
3/8:listoftables:..``Liste des tableaux'' for node3.html 
4/8:section:...``Une figure EPS'' for node4.html 
5/8:section:.."Exemple d’un tableau" for node5.html 
7/8:textohtm1index:...``Index'' for node7.html 
8/8:sectionstar:.."\‘{A} propos de ce document..." for node8.html 
Writing image file ... 
This is Tex, Version 3.14159 (Web2C 7.2) 
(./images.tex 
LaTeX2e (1997/12/01> patch level 2 
Babel <v3.6h> and hyphenation patterns for english, french, german, dumylang, 

 
%%page page_112                                                  <<<---3
 
3.2 Required software and customization 
57 nohyphenation, loaded. 
as 
59 me»: processing 2 images -Hr-Ir 
70 
71 Generating postscript images using dvips . . . 
72 
73 images will be generated in 12h45774/ 
74 
75 This is dvipe(k) 5.78 Copyright 1998 Radical Eye Software (wwu.radica1eye.com) 
75 ’ Tex output 1998.09.07:1521’ -> 12h45774/image 
77 (-> 12h45774/imageool) <texc.pro><specia1.pro><color.pro>[1<colorcir.eps>] 
78 (-> 12h45774/image002) <texc.pro><apecia1.pro><color.pro>[2<tac2dim.eps>] 
79 Writing img1.gif 
80 Writing img2.gif 
31 
82 Doing section links . . . . . 
83 Doing table of contents . . . . . . . . . . 
84 Doing the index . . . . 
By default BTEX2 HTML will generate HTML files according to the HTML 3.2 
specification. As much is said on line 6; the relevant Perl file is loaded on line 8. 
At that point the input file is read (line 12), and parsing 12hexa.tex the necessary Perl style files are input (lines 14-18). Note in particular that for each of 
the packages which we use in the source file, we have a corresponding Perl script 
(lines 16-18). BTEXZHTML then reads l£§'IEX’s auxiliary files (lines 2 3, 25, and 
27), and goes on to the translation step at line 29. By default, the main HTML 
file will have the same name as the input source, but with the extension .htm1. 
In our case we get 12hexa.htm1. Now BTEXZHTML will generate files for each 
of the subdivisions in the BTEX source (the precise level is under user control). 
Thus (lines 37-59) we obtain eight HTML files, nodel .html to node8.htm1. Because the EPS files (lines 25-26 in \reffig{3-1}) cannot be handled directly by HTML 
browsers, ETEXZHTML translates these images into GIF (or PNG) bitmaps. Therefore, BTEXZHTML copies all the necessary BTEX code to file images.tex (line 
61), which is first compiled by BTEX (lines 63-67) and then cut into one~page EPS 
files by dvips (lines 75-78). These are handled by Ghostscript and the netpbm 
utilities (as explained in Section 3.2) and turned into GIF (PNG) images, with the 
names img1.gif, and img2 . gif (lines 79-80). Finally, BTEXZHTML updates all 
the hyperlinks between the various document components (lines 82-84). 
The result of the translation is put in the subdirectory 12hexa, whose file 
content follows. We first see the HTML and GIF files described earlier. The file 
12hexa. css is a CSS style sheet that provides some rudimentary hooks for customizing the way the various HTML tags will be displayed (see Section 3.3.4). We 
also see the file images .tex, which contains the ETEX code used to generate the 
images. (These images files can come in very handy if problems occur during image generation. In particular images . log will contain I£}'[EX’s error messages.) The 
last line shows auxiliary Perl files that relate information in the BTEX source files 
with that in the generated HTML files: 
l2hexa . css l2hexa . html 
nodel .html node2 .html node3 . html node4 . html 
91 

 
%%page page_113                                                  <<<---3
 
92 
The IMFX2 HTML translator 
node5 . html node6 . html node7 . html node8 . html 
img1.gif img2 . gif 
images . tex images . aux images . idx images . log 
images .pl internals . pl labels . pl 
The way the various files are interrelated is shown in \reffig{3-3}. The main file, 
12hexa.htm1, is at the center. From it all other files can be reached via hyperlinks: 
0 links to the table of contents (Tables des matiéres); 
links to the list of figures (Liste desfigares); 
links to the list of tables (Lzkte ale: tableaux); 
links to the first section (Unefigare EPS); 
links to the second section (Example d’zm tableau); 
links to the bibliography (Reférences); 
links to the index (Index); and 
links to information about the way the document was created (fl propos de ce 
document. . . ). 
®@@@9@® 
We also see how hyperlinks in the list of figures (9) point to the figure in question 
(in 9). The same holds for the table (in 6) that can be reached from the list of tables 
(9). More generally, the index (0) has hyperlinks to various points in the different 
HTML files. 
It must be emphasized that we had to do nothing special to have our source 
document divided into chunks of an optimal size for browsing on the Internet. All 
BTEX cross-references are translated automatically into hyperlinks; complex components (pictures and graphics; mathematics, which will be treated in detail in Section 3.3; and some environments) are transformed into bitmaps. Even translations 
of components for non-English documents (French in our case) are taken care of. 
Before going any further, we have to spend some time looking at how to install 
IATEXZHTML on your system and how to get the best using the various ways of 
customizing the output. 
3 .2.2 Installation 
Installation essentially consists of verifying that all the required packages are available. Exact paths to the executables are recorded, so that at runtime there is no need 
for time-consuming searches. 
Two Perl scripts, install-test and configure-pstoimg, are supplied as part 
of the BTEX2 HTML distribution to perform most of the tasks required for installation. Some manual editing is required in the file 1atex2htm1 . config before these 
can be run to get a successful installation. 

 
%%page page_114                                                  <<<---3
 
93 
3.2 Required software and customization 
fim PEME mo uuusow Nmbfl 2% Eofi AEFENNMHE >2 wouwuoaum Eabsbm 1:2 FE "Wm Bnwfl 

 
%%page page_115                                                  <<<---3
 
The EQEXZHTML translator 
Following are the steps required for installation of the ETEXZHTML software. 
The archive has a name like latex2ht1nl-99 . 1 .tar . gz to indicate clearly which 
version it contains. 
1. Decide where on the local file system the new software should be located. 
The most commonly chosen paths are /usr/local/latex2html/, /usr/local/bin/ 
latex2html/, and /usr/local/share/latex2html, although others are possible. The 
archive unpacks into a directory called latex2html, creating it if necessary. The main 
latex2html script, together with the configuration file latex2html.config and the 
installation scripts, are all copied into the top latex2html directory. 
2. Near the top of the latex2ht1n1 . conf ig script, specify the value of the variable 
$LATEx2HTMLDIR to be the complete path to the directory where these scripts are 
located, as chosen earlier in step 1. A complete path to the Perl program should 
also be given as the value of the $PERL variable on non-UNIX systems such as 
OS/2, VV1ndows 95 and Windows NT, and DOS. 
There are several variables whose values should be customized for the local site; for 
example, specifying exact paths for the commands to run ETEX and dvips is important 
on some platforms. These can be adjusted later (see Section 3.2.3 for customization 
options). Only $LATEX2HTMLDIR is needed for installation. 
3. Edit the first line of install-test to ensure that it contains a correct path to 
the command that is used to run Perl on the local system. Make similar edits, 
if necessary, to the first lines in 1atex2ht1nl, configure-pstoimg, makemap, 
texexpand, pstoimg, and pstoimg_nopipes. 
The initial path is # ! /usr/local/bin/perl. It is mandatory to change this to the location where the Perl executable lives.2 
4. Run the script install-test. 
After veri ‘n that the scri t is executable, t the commands ./install-test or 
g ‘ ‘ P 13’ 
perl install-test if simply Calling install-test is not sufficient. 
5. The install-test script writes the value of $LATEX2HTMLDIR into appropriate places in the latex2ht1nl and pstoimg scripts and reports whether the 
necessary software can be found. 
For any missing pieces, it could be that the software is missing, is not the correct version, or is not on the default search paths. Explicit paths can be specified in the file 
latex2html . config. If that is sufficient to fix the problem, then rerun install-test. 
6. The script pstoimg-conf ‘1g is run automatically to record complete paths to 
the utilities required for processing images; alternatively, it can be run sepa2If the $PERL variable was set in step 2, then it is not necessary to edit texexpand, pstoimg, 
and psto:‘ung_nopipes, if these are to be run only from BTEXZHTML. However, pstoimg and 
pstoimg_nopipes are useful utility scripts in their own right, and should be configured correctly to 
allow this. 

 
%%page page_116                                                  <<<---3
 
3.2 Required software and customization 
95 
rately. Specify -gif for GIF or -png for PNG on the command line, or simply 
type g or p when prompted for an image format. 
Complete paths to the various utilities are written into a file called local .pm. This file 
is read by pstoimg and pstoimg_nopipes when generating images. It can be edited if 
necessary, for example, to cope with Subsequent upgrades or if pstoimg-config found 
a wrong version. 
7. Ensure that the ETEX packages html . sty and htmllist . sty, contained in the 
$LATEX2HTMLDIR/texinputs/ subdirectory, can be found by ETEX on your 
System. 
For example, copy them to a subdirectory named .. ./texmf/tex/latex/html/ or 
. . . /texinputs/latex/html/ within the TEX hierarchy. On most TEX installations 
you also will have to update TEX’s filename database by running a program similar to 
Web2C’s mktexlsr. Alternatively, set the TEXINPUTS environment variable to search 
within $LATEX2HTMLDIR/texinputs/. 
8. Copy the navigation icons (the GIF, PNG, or DOS versions, as selected in 
step 6) to a place where they will be accessible “on the Web." Set the 
$ICONSERVER variable within 1atex2ht1nl . config to point to the URL for this 
location. Get the icons from the relevant subdirectory of $LATEX2HTMLDIR. 
This step is really one of customization, so it can be delayed until later. However, the 
best location is usually an images / subdirectory at the top level of the local Web site. 
Special administrative, or “root," privileges may be required to place the icons here. For 
use on DOS installations there is a special set of icons having short filenames. 
9. Support for special packages (optional). 
If you are interested in Indic-TEX/HTML support for traditional Indic 
scripts, then you should find files with the extensions .per1 and .sty in the 
$LATEX2HTMLDIR/ IndicTeX-HTML/ subdirectory. Copy them to a place where 
they can be found by Perl and BTEX, respectively. Do the same with the files in 
the $LATEX2HTMLDIR/XyMTeXHTML directory, if you want support for chemical structure diagrams based on XYMTEX. 
Copy, move, or link the .perl files to the $LATEX2HTMLDIR/styles/ subdirectory and 
copy, move, or link the .sty files to the TEX hierarchy. Alternatively, adjust variables to 
search the original location. 
Although the above procedure may seem rather daunting, it is actually not so 
difficult when one understands the purpose of each step and the options available 
to cope with variations on different platforms. Indeed, it is quite easy to update to 
a new version of BTEXZHTML entirely by hand editing. This is because the local 
configuration files, latex2ht1nl . config and local .pm, rarely need to be changed. 
Usually just steps 3 and 4 are sufficient to upgrade an existing installation to a later 
version, assuming that the previous latex2html . config has been saved for reuse. 

 
%%page page_117                                                  <<<---3
 
96 
The HIEXZHTML translator 
Let us see how the installation goes in practice. Suppose we have obtained the 
distribution from a repository and have copied the BTEXZHTML files into a directory tree; we then move to the top of that directory tree. The directory structure 
should be something like the following: 
IndicTeX-HTML 
XyMTeX-HTML 
cweb2html 
docs/changebar 
example 
foilhtml 
icon-dos 
icons.gif 
icons.png 
makeseg 
styles 
tests 
texinputs 
versions 
/hthtml /psfiles 
In the following sections we will mention the contents of most of these directories 
in more detail. For the moment, however, let us carry on with the installation of 
PHEXZHTML and execute the Perl script install-test. This will write the following information to the screen (we have represented hitting just “carriage return" 
by <CR>): 
5~Ou)\lO\knJi\AN-----.-._.,-_....._-._-... 
3\Oao\loxknJzuN-c~oao\lo~nnJsuN 
This is install-test for LaTeX2HTML V99.1 
Main script installation was... 0 1 ...successful. 
Testing availability of external programs... 
Perl version 5.004 is UK. 
texexpand was found. 
Setting up texexpand script...1 ...succeeded 
Checking for availability of DBM or NDBM (Unix DataBase Management)... 
DBM was found. 
Checking if globbing works... 
DVIPS version 5.78 is OK. 
pstoimg was found. 
Setting up pstoimg script...0 1 ...succeeded 
Setting up configure-pstoimg script...1 ...succeeded 
Looking for latex... 
latex was found. 
Styles directory was found. 
globbing is ok. 
Main set-up done. 
You may complete this set-up by configuring pstoimg now...proceed? <CR> 
This is configure-pstoimg V96.2 by Marek Rouchal 
Welcome to the Configuration of pstoimg! 
You will be guided in few steps through the setup of pstoimg, the part of 
latex2html that produces bitmap images from the LaTeX source. 
Type ‘configure-pstoimg -h’ for a brief usage information and a list of 
user-definable options. 

 
%%page page_118                                                  <<<---3
 
3.2 Required software and customization 
97 
32 Hit return to proceed to the next configuration step. <CR> 
34 Pstoimg can support both GIF and PNG format. 
35 Please note that there are certain legal limitations on the use of the GIF 
36 image format. 
38 If you go on, pstoimg will be configured for GIF or PNG format. 
39 You may reconfigure pstoimg at any time by saying configure-pstoimg. 
40 which format do you want to have supported? 
41 Answer g (GIF) or p (PNG). g 
43 Configuring for GIF format. 
44 Changing $IMAGE_TYPE in latex2html.config. ..succeeded 
46 Hit return to proceed to the next configuration step. <CR> 
48 Ghostscript Conf iguration 
49 ========================= 
51 Ghostscript is "/usr/local/bin/gs", Version 5.10 
52 Pstoimg will use the ppmraw device. 
53 Ghostscript library path is /usr/local/share/ghostscript/5.10 
55 Hit return to proceed to the next configuration step. <CR> 
57 Netpbm/Pbmplus Conf iguration 
53 ============================ 
60 ppmtogif is /usr/local/bin/X11/pplntogif 
61 ppmtogif understands -transparent. Good! 
62 ppmtogif understands -interlace. Good! 
63 pnmcrop is /usr/local/bin/X11/pnmcrop 
(>4 pnmflip is /usr/local/bin/X11/pnmflip 
65 ppmquant is /usr/local/bin/X11/ppmquant 
66 pnmfile is /usr/local/bin/X11/pnmfile 
67 pnmcat is /usr/local/bin/X11/pnmcat 
68 pbmmake is /usr/local/bin/X11/pbmmake 
70 Hit return to proceed to the next configuration step. <CR> 
72 Transparent/Interlaced Image Configuration 
75 Using netpbm to make transparent GI!-‘s. 
76 Using netpbm to make interlaced GI!-‘s. 
78 Hit return to proceed to the next configuration step. <CR> 
so Setup pstoimg 
3; ============= 
83 Updating local configuration file. . . 
85 well done! 
86 Pstoimg is now hopefully configured to run on your system. 
87 Type ‘pstoimg -h’ for a brief usage information. 
88 Please specify the desired image format in the file latex2html.config. 
The first part of the script (lines 1-20) locates all the tools needed by BTEXZHTML 
and verifies whether the installed versions are adequate. Next bitmap generation 
is initiated via an implicit call to the configure-pstoimg script (line 32). The 
main choice to be made is whether you want GIF or PNG images in your HTML 

 
%%page page_119                                                  <<<---3
 
98 
The Ié‘l]§X2HTML translator 
s (we chose GIF by answering g in line 41). Then we proceed with the configuration of Ghostscript (lines 48:53), the netpbm utilities (lines 57-68), and the 
generation of interlaced and transparent images (lines 72-76). Finally the program 
pstoimg is set up (lines 80-88). All in all, very little effort was required to initiate 
the BTEXZHTML system on our computer platform. 
If, in the future, changes that affect the locations of utilities not supplied as 
part of the ETEXZHTML distribution occur, you can simply rerun install-test 
or configure-pstoimg to record the new locations. 
3.2.3 Customizing the local installation 
There are dozens of configuration variables that affect aspects of the way the HTML 
s are created by BTEXZHTML. To preserve settings that are most appropriate 
for a particular site, for a particular author or group of authors, or for particular 
kinds of documents, BTEXZHTML provides a hierarchy of customization mechanisms. 
These mechanisms consist of reading files containing Perl code to set variables 
and define or redefine subroutines that will be used in the subsequent processing. In 
the following list, these files are named in the order in which they are processed by 
the main latex2html script. The most recently read file determines the eventual 
value of a variable. Thus Perl code in later-read files overrides variable settings from 
files loaded earlier. 
3.2.3.1 The file latex2ht1nl . config 
Site-specific customizations can be entered in the file latex2ht1nl . config, whose 
definitions apply to all users at a site. This file is located using the value of 
$LATEx2HTMLDIR which is entered in the latex2ht1nl script during the installation 
procedure described in the previous subsection. 
It is important to set SBICONSERVER equal to a directory “visible on the Web" so that the 
readers of the HTML pages can access the icons; see the installation step 8 earlier. The 
language used for titles of automatically generated HTML pages is a good candidate for 
customization here. VVhere users are subject to quota restrictions, it is also a good idea to set 
$TMP to a directory that is “world writable." This way the most diskspace-intensive part of 
the processing need not take a user over quota, thereby causing the job to fail. 
The Perl script pst0img_n0pipes is provided for platforms that do not support the use 
of UNIX “pipes" as a means of passing the output from one command to the input of another. 
This applies in particular to DOS. Set the $PSTDIMG variable to use pstoimg_nopipes. Such 
systems may not support “forking" either, for these also set $N0_FOR.K=1;. 
At some sites it is advisable to set $LATEX to include a full path to the command to invoke 
ETEX. Similarly, setting $DVIPS may be required or desired to establish a “virtual printer"3 
3Best-quality images are obtained using PostScript fonts. File sizes are kept to a minimum when 
Ghostscript is configured to find the .pfa or .pfb files themselves, using its Fontmap listing or by 
setting the GS_FONTPATI-I environment variable. This also requires setting $DVIPS=``dvips -Pgs'' ; and 
creating a configuration file conf ig . gs that refers to a file psfonts . gs that lists the fonts to be excluded 
from the output generated by dvips. 

%==========120==========<<<---2
 
%%page page_120                                                  <<<---3
 
3.2 Required software and customization 
99 
for Ghostscript. If PostScript fonts are not being used, there are variables that affect the 
automatic generation of font bitmaps using METFIFONT. Read the comments present in 
1atex2htm1. config for more information. 
3.2.3.2 The file .latex2htInl-init 
User-specific preferences are handled by the file $H0ME/.latex2ht1n1-init. It 
normally resides in the user’s home directory (hence the use of the variable $HOME). 
This file is commonly used to set a signature using the $ADDRESS variable or to use 
an alternative set of navigation icons via SBICONSERVER. The variable $TEXINPUTS can be 
set to allow customized input sources to be found. Similarly the $LATEX2HTMLSTYLES 
and $LATEX2HTMLVERSIONS variables can be adjusted to search directories containing customized Perl scripts for implementations of extra packages or extensions or as replacements for those distributed with ETEXZHT ML. Setting $INFO=""; suppresses the information page, which would otherwise be generated automatically. The scale factor variables 
$FIGURE_SCALE_FACTOR, $MATH_SCALE_FACTOR, and $DISP_SCALE_FACTOR, applied when 
creating images, may here be changed from the values set in 1atex2htm1.con;fig. 
You can also deposit a file . lat ex2html-init in the current directory. It allows 
you to set variables that will apply to all jobs run from that directory. 
Apart from overriding other customizations, typically this file is used to specify the “granularity" for splitting the output into HTML pages according to sectioning levels, via 
$MAX_LINK_DEPTH. $MAX_SPLIT_DEPTH affects creation of navigation links between pages, 
while $TOC_DEPTI-I determines which section headings are hyperlinked from the Table of 
Contents page. The level of HTML to be produced and the type of math translation to be 
performed can also be specified here. 
3.2.3.3 Command-line switches 
Many of the variables already Ihentioned, plus many others, can be set for a single 
job via Command-line switches. We will have the occasion to introduce a lot of them 
in examples later in this chapter. 
For example, -split 4 has the same effect as setting $MAX_SPLIT_DEPTH=4; for splitting 
the source at the level of \sect ion commands but not at \subsection commands. Similarly, 
-link 2 has the same effect as setting $MAX_LINK_DEPTH=2; for hyperlinks to headings 
deeper by two levels, on the same and other HTML pages. 
More configuration files can be loaded via -init_fi1e <imlt-f'£le>. Commandline options are processed in order of occurrence, including reading completely any 
<'im'.t-f'i.le>s as they occur. See the ETEXZHTML Users Guide for details on the available options, explanations of their uses, and configuration variable equivalents. 
3.2.3.4 Perl packages and the file <jobna1ne>.perl 
When a BTEX package using a \usepackage{<package>} command is requested, 
a Perl implementation is loaded from a file <package> .perl, if one can be 
found in the working directory or in any of the directories specified in the 
$LATEX2HTMLSTYLES variable. Furthermore, a file < j obna.me> .perl will be loaded 
if it exists for the particular job. This extension mechanism is explained next. 

 
%%page page_121                                                  <<<---3
 
100 
The HIEXZHTML translator 
3.2.4 Extension mechanisms and [HEX packages 
The output from BTEX2 HTML can be controlled by three separate mechanisms for 
loading files containing Perl code. Analogous to those for ETEX are the “documentclass" and “package-loading" mechanisms. The former uses the main argument to 
the \documentclass command and any class options, to load particular files; the 
latter uses the \usepackage command and any options. 
The extension mechanism handles translation issues that need not be of concern within the ETEX source but that are significant for the HTML output. For 
example, extensions control which version of HTML to produce, what font encoding to use, and whether mathematics should be rendered as images or parsed into 
small pieces. 
3.2.4.1 Document classes and class options 
When the document source contains a line \documentclass {<cZ.ass>} then a file 
<cZ.a.ss> .perl will be loaded, if found in the current directory or in any of the 
directories specified in the $LATEX2HTMLSTYLES variable. 
Note that a \documentclass or \documentstyle command is not strictly necessary, for 
ETEXZHTML will accept any text file as input to convert into HTML. Any macros will be 
substituted, if possible, assuming the file is ETEX source. 
VV1th class options, as in \documentclass [< optic-n>] {<cZ.a.ss>}, each class 
can cause a file <opt«lo-n> .perl to be loaded, if a file with this name can be found 
within the searched directories. Furthermore, a subroutine do_< cZ.a.ss>_<opt 7:0-n> 
is executed, if it has been defined in the loaded packages or initialization files. 
For example, in article .perl there is a definition for sub do_article_1eqno which sets 
a flag to indicate that equation numbers be placed at the left-hand side of displayed mathematics. 
For documents prepared originally for ETEX 2.09, BTEXZHTML treats the 
\documentstyle command as if it were \documentclass. Thus support for the 
style and for any named packages is loaded via the mechanism explained earlier. 
The standard document classes in ETEX provide code that controls the typesetting of 
the source material on the output page. However, these procedures have little relevance 
when creating an HTML document; the only thing currently implemented in the files 
article.per1, rep0rt.per1, and book.per1 is the structure for section numbering, for 
when section numbers are required. With a.msart.per1 and a.msbook.per1, extra commands are defined for use on the title page; support for AMS mathematics packages is loaded 
automatically. 
3.2.4.2 Packages and package options 
For BTEX instances that contain a command \usepackage {<package>}, a file 
<pa.cka.ge>.perl will be loaded, if it resides in the current directory or in one 
specified in the variable $LATEX2HTMLSTYLES. 

 
%%page page_122                                                  <<<---3
 
3.3 Mathematics modes with Ié‘1]§X2HTML 
101 
Most standard ETEX packages are implemented in this Way, as are many of the ETEX supported packages. The $LATEX2HTMLDIR/styles/ subdirectory, which is the default value 
for $LATE‘.X2H'I'MLSTYLE‘.S, contains the corresponding .per1 files. Alternatively, consult the 
ETEXZHTML Users Guide for a listing. 
For package options, as in \usepackage [<opt72on>] {<pa.cka.ge>}, a Perl subroutine named do_<package>_<opt71on> is executed after the package support has 
been loaded. If no such subroutine has been defined, a warning message is printed 
immediately and again at the end of the job. 
Of course, this subroutine may cause further .per1 files to be loaded. For example, 
do_babe1_:Erench causes loading of french.per1 to use French language keywords and 
titles. Similarly, do_inputenc_1atin2 loads support for the ISO-8859-2 encoding via a file 
$LATEX2HTMLDIR/versions/latin2.p1. 
3.2.4.3 Extensions and HTML versions 
The command-line switch -htIn1_version takes an argument that specifies extra 
Perl files to be loaded for the current job. As its name suggests, this switch specifies the version of HTML that is to be produced. Further filenames appended in 
a comma-separated list indicate extra files to load from directories listed in the 
$LATEX2HTMLVERSIONS variable. 
For example, -html_version 4.0,math,u.nicode,fra.me causes ETEXZHTML to read the 
Perl files htm14_0.p1, math.p1, and unicode.p1. By default these files are located in 
the subdirectory $LATEX2HTMLDIR/versions/, but this location can be changed with the 
$LATEX2HTMLVERSIDNS variable. The main versions available are: frame, htm13__2, htm14_0, 
lang, 1atin1 to 1atin6, math, and unicode. 
3.3 Mathematics modes with BTEXZHTML 
E-TEXZHTML provides a rich set of methods for displaying mathematics within 
Web pages. In our examples we use a file sa1npleMath . tex, listed here: 
\documentclass[a4paper,twoside]{article} 
\usepackage{htm1} 
Z\usepackage{amsmath} 
\renewcommand{\d}{\partia1}\providecommand{\bm}[1]{\mathbf{#1}} 
\providecommand{\Range}{\mathca1{R}}\providecommand{\Ker}{\mathca1{N}} 
\provideCOmmand{\Qhat}{\Vec{\mathbf{Q}}} 
\newcommandf\StAndrews}{\url{http://wuw-groups.dcs.st-gnd.ac.uk/~history}}Z 
\newcommand{\Pythagorians}{\htm1addnorma1link 
{Pythagorians}{\StAndrews/Mathematicians/Pythagoras.htm1}} 
\newcommand{\Fermat}{\htm1addnorma1link 
{Fermat, c.1637}{\StAndrews/HistTopics/Fermat’s_1ast_theorem.htm1}} 
\newcommand{\Wi1es}{\htm1addnorma1link 
{Wi1es, 1995}{http://wuw.pbs.org:80/wgbh/nova/proof}} 
\begin{document} 
\htm1head[center]{section}{Math examples} 
»-.-..-H...»-.-.-.ao\:o«v-Jsw~»-o~cao~:o«u-.au~..
 
%%page page_123                                                  <<<---3
 
102 
The EVIEXZHTML translator 
19 \begin{eqnarray} 
20 \phi(\lambda) & = & \frac{1} {2 \pi i}\int"{c+i\infty}_{C-i\infty} 
21 \exp \left( u \ln u + \lambda u \right ) du \hspace{1cm}\mbox{for } c \geq O \\ 
22 \1ambda & = & \frac{\epsilon -\bar{\epsilon} }{\xi} 
2} - \gamma’ - \beta"2 - \ln \frac{\xi} {E_{\rm max}} \\ 
24 \gamma & = & O.577215\dots \mathrm{\hspace{5mm}(Euler’s\ constant)} \\ 
25 \gamma’ & = & O.422784\dots = 1 - \gamma \\ 
26 \epsilon , \bar{\epsilon} & = & \mbox{actual/average energy loss} 
27 \end{eqnarray} 
R 
29 Since~\ref{eqn:stress-sr} or~\ref{gdef} should hold for arbitrary $\delta\bm{c}$Z 
30 -vectors, it is clear that $\Ker(A) = \Range(B)$ and that when $y=B(x)$ one has...\\ 
31 ...the \Pythagorians{} knew infinitely many solutions in integers to $a“2+b“2=c“2$. 
32 That no non-trivial integer solutions exist for $a“n+b"n=c“n$ with integers $n>2$ has long 
33 been suspected (\Fermat). Only during the current decade has this been proved (\wiles). 
54 
35 \begin{eqnarray}\label{eqn:stress-sr} 
36 V \bm{\pi}“{sr} & = & \left< \sum_i M_i \bm{V}_i \bm{V}_i 
37 + \sum_i \sum_{j>i} \bm{R}_{ij} \bm{F}_{ij}\right> \\ \nonumber 
38 E: = E: \left< \sum_i M_i \bm{V}_i \bm{V}_i 
39 + \sum_{i}\sum_{j>i}\sum_\alpha\sum_\beta \bm{r}_{i\alpha j\beta}\bm{f}_{i\alpha j\beta} 
40 - \sum_i \sum_\alpha \bm{p}_{i\alpha} \bm{f}_{i\a1pha} \right> 
41 \end{eqnarray} 
M 
45 \end{document} ZZZ requires \usepackage{amsmath} to continue 
44 
45 \begin{subequations}\label{bgdefs} 
46 \begin{align} B_{ij}"\a1pha & = 
47 \left(B_{ij}“\alpha\right)_O + \left(B_{ij}“\alpha\right)_a \label{bdef} \\ 
48 \left(B_{ij}“\alpha\right)_O & ‘ \frac{1}{2}\left(\frac{\d N_i"\alpha}{\d X_j} 
49 + \frac{\d N_j“\alpha} {\d X_i} \right) \label{b0def} \\ 
so \left(B_{ij}"\alpha\right)_a & = H_{ij}"{\a1pha \beta} a“\beta \label{budef} \\ 
51 H_{ij}“{\a1pha \beta} & = 
52 \frac{1}{2}\left( \frac{\d N_k“\alpha}{\d X_i} \frac{\d N_k‘\beta}{\d X_j} 
53 + \frac{\d N_k“\beta}{\d X_i} \frac{\d N_k"\alpha}{\d X_j} \right) \label{gdef} 
54 \end{align} 
55 \end{subequations} 
56 \end{d0cument} 
We also use a second file samp1eAMS.tex, identical to samp1eMath.tex but with 
the \end{document} (line 43) commented out and with package amsmath (deleting 
the ‘Z, on line 3) included. 
3.3.1 An overview of I*}'I]';X2HTML’s math modes 
Basically, BTEXZHTML has three modes for mathematics: the “novice" mode, the 
“professional" mode, and the “expert" mode. Extra features are available in some 
modes. We will look at each mode in turn. 
3.3.1.1 Novice mode 
Novice mode, which is the usual default, is also known as “simple math." Expressions using just ordinary alphabetic, numeric, and arithmetic characters are presented using the browser’s text font, italicized where appropriate. Superscripts and 
subscripts are used with HTML 3.2 and later. Any other symbol or macro causes 
an image to be made of the whole inline expression or logical part of an aligned 

 
%%page page_124                                                  <<<---3
 
3.3 Mathematics modes with ETEXZHTML 
103 
environment. Exceptions are styling macros such as \1nathrm, \Inathbf, \1nathtt, 
\bo1dmath, and \bIn. Text mode macros such as \textrIn, \textit, \texttt, and 
\rm, \bf, \it, \tt are also allowed for compatibility with older documents. This 
mode is suitable only when quite simple mathematical expressions occur within a 
document having mainly nonmathematical content. \reffig{3-4} shows an example 
of novice mode using the command: 
1atex2htm1 samp1eMath.tex 
In fact, the file sampleMath.tex contains too much mathematics to be suitable for novice mode. Notice how in \reffig{3-4} some inline equations are set using 
the browser’s text font, but others require images. Aligning images requires centering on the baseline whenever there is depth, as in a descender or the bottom of a 
parenthesis. The alignment works with some browsers, but not with others. In any 
case the linespacing comes out inconsistent or too wide; centering within a display 
also has defects. Compare the different sizes and placements of the “=" signs. Such 
problems can be overcome using HTML 4.0 and a CSS style sheet, as seen later in 
\reffig{3-10}. 
3 .3 .1 .2 Professional mode 
Professional mode is adequate for documents with a lot of mathematics in the form 
of displays or inline expressions, or when mathematical symbols are used frequently. 
It is the default mode when one of the AMS packages-amsmath, amsthm, amsopn, 
or amstex-is loaded and with the amsart and amsbook document classes. To maintain consistent style between inline and displayed expressions, images are made of 
all inline mathematics, and of each equation, formula, or aligned unit within displays. Where possible, the browser’s text font is used for the argument of \mbox 
and \text macros. \reffig{3-5} shows an example of professional mode using the 
command: 
1atex2htm1 samp1eAMS.tex 
In \reffig{3-5} the overall effect is more balanced compared to that in \reffig{3-4}. 
Although linespacing for the inline math is consistent, it is a little too wide because 
on each line there is an image with depth. Even the single “=" sign in the alignments is an image that occurs seven times. The browser’s text font is used for the 
\mbox in the last cell of equation (5). It fails to align correctly with images on the 
same line, as alignment between cells in a <TABLE> row is tied to the middle4 of 
the text, not to its baseline. For equation (2) the words are contained in an image, 
otherwise the \hspace command could not be handled. 
Using the -no_ma1;h option, the consistency in style afforded by professional 
mode is obtained without loading AM8 packages. This means “no simple math" 
4since ``TOP'' and “BOTTOM" would clearly be incorrect. 

 
%%page page_125                                                  <<<---3
 
 
%%page page_126                                                  <<<---3
 
3.3 Mathematics modes with EVIEXZHTML 
105 
parsing, favoring the use of images. Setting $N0_SIMPLE_MATH=1; has the same 
effect. An example can be seen in \reffig{3-6}, for which we used the command: 
1atex2htm1 -no_math -white -no_transparent samp1eMath.tex 
Images are shown on an opaque white background, (-no_1;ransparent and -white 
options). 
Summary 
Professional mode offers a consistent style for all mathematical expressions, in particular, using HTML’s <TABLE> tags for alignments and equation numbering. A 
drawback is that all mathematics requires the generation of an image, resulting in a 
potentially large number of bitmap files. This is probably not an issue when download time for images is not critical. 
3.3.1.3 Expert mode 
In expert mode the structure of mathematical expressions is broken into smaller 
pieces. It uses the browser’s text font wherever possible and generates images of 
pieces such as fractions and symbols. The many resulting small images often require 
less time to download than for a smaller number of larger images. Indeed, small 
images can often be reused in many HTML pages so that the overall size of the 
Web document can be significantly reduced. VVith very large manuscripts, the total 
diskspace required for the whole document becomes appreciably less than when 
using professional mode. 
Examples of expert mode are Figures 3.7-3.10. In the case of \reffig{3-7} we 
typed the command: 
1atex2htm1 -no_math -htm1_version 3.2,math -white \ 
-no_transparent samp1eMath.tex 
3.3.2 Advanced mathematics with the math extension 
To use expert mathematics mode and for detailed parsing of mathematical expressions, one should load the math extension module of Perl commands. An example 
is \reffig{3-8}, where we used the following command-line options: 
1atex2htm1 -no_math -htm1_version 3.2,math samp1eAMS.tex 
and where the version of HTML must be specified also. The browser’s text font is 
used wherever possible, resorting to images only for special symbols and structures 
that require vertical alignment, such as fractions, derivatives, integrals, summations, 
and large brackets. More details are shown in \reffig{3-7}, which uses an opaque 

 
%%page page_127                                                  <<<---3
 
106 
The Ié\'I}3X2HTML translator 
background. This clearly shows which math parts are rendered as images and how 
alignment is achieved by equalizing depth and height. 
The visual effect is generally quite good, but it does not have the complete 
consistency of the larger images obtained with professional mode. However, the 
user now has control over the size and face of the font used for the mathematics. 
Images do not rescale, but the size at which they are created can be altered. The 
variables affecting the size of images are the following (they are applicable in any 
mode): 
$MATH_SCALE_FACTOR magnification factor for creating images of mathematics 
and inline text. 
Recommended value: 1.4 (corresponds to 14 pt font size in the browser). 
$MATH_DISPLAY_FACTOR extra magnification used with images of displayed mathematics; it multiplies $MATH_SCALE_FACTOR. 
Recommended value: 1.2 (corresponds to 17 pt font size in the browser). 
$FIGURE_SCALE_FACTOR magnification applied when creating images of figures 
and whole environments. Recommended value: 1.6. 
Font characters used in displayed mathematics have a <BIG> tag applied. The 
$MATH_DISPLAY_FACTOR ensures images are scaled similarly. There is still a slight 
misalignment of adjacent cells with text and images, but it is less intrusive. Failure 
to place superscripts directly above subscripts is an irritation. The extra gap after 
italicized letters, produced in some browsers, is more annoying, especially when 
preceding a subscript. 
A major benefit of expert mode is the reduced diskspace required to store the 
smaller images, because much less area is rendered using bitmaps. This also means 
less data to be transferred, so access times are reduced. However, this is partially 
offset by the need for more separate connections to obtain the larger number of 
files for each HTML page. 
For just a single page containing a fair amount of mathematics, there is not a 
lot to gain, but when there are many pages, the same images of symbols are reused 
over and over. The first few pages may be slow to load; subsequent pages are usually 
much faster, since most of the required images are already available locally within 
the browser’s image cache. Indeed, image reuse can be so effective that fewer images 
may result from using expert mode. In a real-world example using professional 
mode, the translation of a paper in mathematical logic into 36 HTML pages totaled 
1.3 Mbytes with 392 images. In expert mode it required just 800 kbytes with only 
223 images. 
There are other advantages of expert mode that are designed to ensure that the 
information contained in the original BTEX document source is faithfully conveyed 
to the receiver. This is true even when there is trouble obtaining all the images 
required to build the intended picture in the browser window. 

 
%%page page_128                                                  <<<---3
 
3.3 Mathematics modes with HIEXZHTML 
Images of small pieces of BTEX source have the complete code contained in 
the ALT attribute of the <IMG> tag. For example, the summations in \reffig{3-8} are 
marked up in the HTML as follows: 
1 (TD ALIGN=``LEFT'' NUWRAP><IMG 
Z WIDTH=``233'' HEIGHT=``63'' ALIGN=``MIDDLE'' BURDER=``0'' 
3 sacs" img19 .gif " 
4 ALT="$\disp1aystyle \1eft&1t; \sum_i M_i \mathbf{V}_i \mathbf{V}_i 
5 + \sum_i \sum_{j&gt;i} \mathbf{R}_{ij} \mathbf{F}_{ij}\right&gt;$"></TD> 
This example also shows the use of entities 8511;; (line 4) and Kcgt; (line 5) to 
protect < and > from their special meanings in the HTML language. Most images 
are smaller and shorter, as in the next example: 
1 - <IMG 
2 WIDTH=``21'' HEIG!-IT=``35'' ALIGN=``MIDDLE'' BUR.DER=``0'' 
3 SRC=“img10.gif" 
4 ALT="$\disp1aystyle \beta"{2}_{}$"> 
Note on line 4 the braces and the empty subscript inserted automatically within 
\be1;a“{2}_{}. This ensures good BTEX style and consistent placement of the superscript across similar images. Hence, complete information can often be read 
using a text-only browser or with a browser mode where images are loaded only 
upon request. 
Building on this idea, when a mathematics environment is parsed into font 
characters and small images, the entire source is included as a comment within the 
HTML file. The next example shows an extract from the HTML source of \reffig{3-8}. These comments can be searched for occurrences of particular pieces of 
mathematics. Portions can be easily extracted for reuse within other If}'I]3X documents. There is no need for numbered entities within comments. 
(DIV ALIGN=``CENTER''> 
<!-- MATH: \begin{eqnarray} 
\phi(\1ambda) & = & \frac{1} {2 \pi i}\int“{c+i\infty}_{c-i\infty} 
\exp \1eft( u\1n u +\1ambda u \right) du \hspace{1cm}\mbox{for } c\geq O\\ 
\1ambda & = & \frac{\epsi1on -\bar{\epsi1on} }{\xi} 
- \gamma’ - \beta"2 - \1n \frac{\xi} {E_{\rm max}} \\ 
\gamma & = & O.577215\dots \mathrm{\hspace{5mm}(Eu1er’s\ constant)} \\ 
\gamma’ & = & O.422784\dots = 1 - \gamma \\ 
\epsi1on , \bar{\epsi1on} & = & \mbox{actua1/average energy loss} 
\end{eqnarray} --> 
(TABLE ALIGN=``CENTER'' CELLPADDING=``O'' WIDTH=``1OOZ''> 
(TR VALIGN=``MIDDLE''><TD NUWRAP ALIGN=``RIGHT''><IMG 
WIDTH=``35'' HEIGHT=``31'' ALIGN=``MIDDLE'' BURDER=``0'' 
SRC="img1 .g'1f" 
ALT="$\displaystyle \phi(\lambda)$"></TD> 
<TD ALIGN=``CENTER'' NUWRAP><IMG 
Z 3 I 3 Z S E E v m u o m ¢ w ~ H 
At the expense perhaps of longer download times, these other advantages can 
be obtained along with the images of professional style, by using the following set 
of command-line options: 
1atex2htm1 -no_math_paising -htm1_version 3.2,math filename 
107 

 
%%page page_129                                                  <<<---3
 
108 
The HIEXZHTML translator 
Summary 
Expert mode with the math extension guarantees consistency for all the mathematics. It also includes the ETEX source as comments into the HTML files, making text 
searches in the math possible. This strategy results in larger images, thus requiring 
greater download times. It is, however, an ideal solution if the aim is to retain a 
maximum amount of information, combined with images of good quality. 
3.3.3 Unicode fonts and named entities, in expert mode 
Further reduction in the number of images is obtained by requiring the browser to 
use special fonts for Greek letters and for some mathematical symbols. The Unicode (or ISO-106461) standard (see Appendix C.2) defines codes for such characters. These are allowable within HTML pages satisfying the 3.2 or 4.0 specifications. 
Currently browsers support only few, if any, of these characters. \reffig{3-9} shows 
the result when BTEXZHTML loads the unicode extension, as follows: 
1atex2htm1 -no_math -htm1_version 3.2,math,unicode samp1eAMS.tex 
Almost all of the first aligned display in \reffig{3-9} is built using font characters 
available to the browser. Small subscripted letters at and ,8 are quite acceptable. The 
variant epsilon comes with the font used here; BTEX source code could be modified 
to match, using \bar\varepsilon. HTML character references make the html files 
difficult to read, for instance, #946 for ,8 and #947 for y (line 5). 
(TD ALIGN=``LEFT'' NUWRAP><IMG 
WIDTH=``40'' HEIGHT=``47'' ALIGN=``MIDDLE'' BURDER=``O'' 
SRC="img5.gif" 
ALT="$\disp1aystyle {\frac{\epsi1o11 -\bar{\epsi1on} }{\xi}}$"> 
- &#947;’ - &#946;<SUP>2</SUP> - ln<IMG 
WIDTH=``44'' HEIGHT=``51'' ALIGN=``MIDDLE'' BURDER=``O'' 
SRC="img6.gif" 
ALT="$\disp1aystyle {\frac{\xi}{E_{\rm max}}}$"></TD> 
(TD WIDTH=1O ALIGN=``RIGHT''> 
(2)</TD></TR> 
E~oao\Ia~v-4:-‘-om.Images are still required for fractions and other constructions requiring vertical alignment. The most common perhaps are the variable-sized operators, such 
as integrals and summations. Superscripts and subscripts on these operators are 
included within the images; otherwise correct positioning cannot be obtained. 
Using the -entities command-line option, numbered character references 
are replaced by named entity references, such as &beta; and &gamma;. Although 
these are part of the entity set valid for HTML 4.0, only the most recent browsers 
actually support them. 

%==========130==========<<<---2
 
%%page page_130                                                  <<<---3
 
 
%%page page_131                                                  <<<---3
 
110 
The I‘}'Ij§X2HTML translator 
Summary 
Expert mode using the Unicode encoding requires substantially fewer images than 
the techniques presented earlier, because font characters can be used for many math 
symbols. However, present-day browsers have limited or no support for Unicode 
and do not have available adequate fonts for Greek and math symbols. This approach will certainly gain in importance in the medium-term future and will result 
in very fast download times. For the reasons just mentioned, its practical impact is 
nevertheless bound to remain quite limited for some time to come. 
3.3.4 HTML 4.0 and style sheets 
VVhen producing code for the HTML 4.0 specification, there is greater scope for addressing some of the visual defects mentioned earlier. All three mathematics modes 
are available, as well as their variants. Following are a few examples of use: 
latex2html -html_version 4.0 sampleMath.tex 
latex2html -no_math -html_version 4.0 sampleMath.tex 
latex2html -no_math -html_version 4.0,math sampleMath.tex 
latex2html -no_math -html_version 4.0,math,unicode sampleMath.tex 
latex2html -no_math_parsing -html_version 4.0,math sampleMath.tex 
Some simple style sheet effects are shown in \reffig{3-10}, for which we used 
the following command: 
latex2html -no_math_parsing -html_version 4.0,math \ 
-style sampleAMS.css sampleAMS.tex 
The style sheet effects shown include the following (the line numbers correspond 
to the CSS style sheet below): 
o a colored title with a frame of a different color (line 1); 
o boldened and colored equation numbering (line 2); 
0 colored background for the subequations environment (line 3); and 
o fixed line heights for paragraphs with inline images of mathematics (line 4). 
VVhen running on the file samp1eAMS . tex, ETEXZHTML automatically generates 
a “generic" CSS style sheet samp1eAMS . css, together with the HTML pages. This 
style sheet can then be edited to include declarations that implement the desired 
effects, for example: 
1 H2 { border : 2pt solid brown ; color : maroon ; padding : 5pt } 
2 TD.eqno {color : green ; fontweight : 700 } /* equation-number cells */ 
3 TABLE.subequations { background-color : #EOEOE0 ; } 
4 P { line-height : 18pt ; } 

 
%%page page_132                                                  <<<---3
 
3.3 Mathematics modes with IIHEXZHTML 
111 
Of course, these effects are visible only when the HTML pages are viewed using 
a browser that can interpret the style sheet language. \reffig{3-10} was obtained with 
a browser that understands the CSS language. 
Today, only CSS is supported as style sheet language. In the future, as other 
style sheet languages become commonly available, support to exploit their features 
will be added to ETEXZHT ML. For more information on the CSS language, see 
Section 7.4. 
Several techniques for defining CSS style sheet entries are available to be used 
with HTML pages constructed by ETEXZHTML. 
1. You can edit, using any text editor, the automatically generated style sheet after 
running ETEXZHTML. For each environment in the ETEX source, an empty 
“stub" will have been created. This stub can then be augmented to include the 
desired properties and values. 
2. You can link to a previously prepared style sheet, using the -style style-file 
command-line option. Alternatively, you can set the $STYLESHEET variable to 
have style-file as its string value in the form of a complete URL to the style sheet 
file’s location or a relative URL from where the HTML pages will reside. 
3. You can include style sheet information in the ETEX source itself. This can be 
done for environments using an optional argument to the \begin command 
(the html package should be loaded to use this feature). 
It is probably not a good idea to mix style information and source code too 
strongly, although it is sometimes useful to exert microcontrol on parts of a piece 
of text or a formula. Therefore let us consider in more detail the third alternative, and see how we can control the style by using the optional argument on the 
\begin commands of an environment. Following we describe three ways of using 
this feature: 
\begin [style-z'nfi;] 
With this first method an attribute ID=``umLd'' is included on the <DIV>, <SPAN>, 
or <TABLE> start tag generated for the contents of the environment. The unique 
identifier umld associates the element with the definition style-info in the style sheet. 
\begin [class] 
This second method associates a CLASS=`` class '' attribute with the generated HTML 
tag. Style information associated with class will be inherited. A stub is written into 
the style sheet to add the necessary information later. 

 
%%page page_133                                                  <<<---3
 
112 
The I*}'Ij§X2HTML translator 
\begin [class I style-z'nfi1] 
This third method is a combination of the previous two methods, with “I " separating the different types. 
The contents of style-inf?) can be the exact code for the style sheet entry. Alternatively, it can be a set of key='ua.lue pairs, using either “=" or “: " as a separator. 
The “ ; " delimits pairs, while “ , " can be used to separate multiple values for a single 
key. Names for class use alphanumeric characters. VVhen there is no I , the presence 
of =, :, or ; indicates that the argument contains style-infi1, and not class names. 
The following ETEX code can be used to create entries for the style effects 
visible in \reffig{3-10} (see Section 3.5.2 for a description of the \htmlsetstyle 
command). The following lines 1-3 are equivalent to lines 1, 2, and 4 in the CSS 
style sheet on page 111, while line 3 in the style sheet is replaced with a specification 
directly on the start of the subequat ions environments (line 5). 
\htmlsetstyle[H2]{}{border=2pt,solid,brown;color=maroon;padding:5pt} 
\htmlsetstyle[TD]{eqno}{color : green ; fontweight : 700 } 
\htmlsetstyle[P]{}{ line-height : 18pt } 
\begin[background-color=\#EOEOE0]{subequations} 
\n<:Av..:.wN.-. 
\end{subequations} 
Summary 
HTML4 and CSS let us apply style effects using any number of style sheets. Although many of the older generations of browsers have limited or no support for 
style sheets, this situation is improving rapidly. Thus a combination of HTML 4 and 
CSS will prove ideal in the medium term to offer full control on the presentation of 
information. 
3.3.5 Large images and HTML 2.0 
Another approach to presenting mathematics within Web pages is to use larger 
images, that is, images of whole environments. Visual appearance could well be 
the most important consideration with the amount of data and its speed of transfer 
irrelevant. This may be the case, for example, when browsing a CD-ROM or using a 
single machine or local area network. VVhen compatibility is required for browsers 
capable of interpreting just HTML 2.0 markup then this is the only viable approach 
because <TABLE> tags are not supported. 
Large images can be very attractive, as shown in \reffig{3-11} which was generated with the command: 
latex2html -no_math -html_version 2.0 sampleAMS.tex 

 
%%page page_134                                                  <<<---3
 
 
%%page page_135                                                  <<<---3
 
114 
The Ié'IEX2HTML translator 
\-o~v.Aw~>-However, since equation numbering is included as part of the image, it is 
impossible to reuse images. Hyperlinks associated with specific equations or subequations must target the image as a whole. VV"1th HTML 2.0, alignment is limited 
to just paragraphs and images. 
Large images can also be used with later versions of HTML, via the command 
\htm1image (line 9) and the environment makeimage (lines 2-6), both of which 
are defined in the html package. VV"1th Inakeimage explicit alignment (flushright 
environment on lines 1-7) must be imposed to get equation numbers occurring 
flush to the edge of the window. 
\begin{flushright} 8 \begin{flushright} 
\begin{makeimage} 9 \begin{eqnarray} 
\begin{eqnarray} 1o \htmlimage{} 
... ll ... 
\end{eqnarray} 12 \end{eqnaIray} 
\end{makeimage} 13 \end{flushright} 
\end{flushright} 
The command \htm1image requires an argument that may be empty. See \reffig{3-12} for an example. It uses source from a file samp1eMathImages.tex, modified from samp1eAMS . tex, indicated earlier. 
The full range of possible arguments to \htmlimage is listed in Section 3.5. 
Among them one helps alleviate some of the difficulties presented by large images 
by including only a smaller “thumbnail" directly on the HTML page: 
\htmlimage{thumbnail=.4} 
This acts as a button, hyperlinked to a full-size version of the image. 
\reffig{3-13} shows the result from a file sampleMathThu.mb.tex, which is just 
sampleAMS.tex with \htmlimage commands in the displayed environments. This 
technique has been used quite effectively with some electronic journals. 
Summary 
The “lowest common denominator" approach of using HTML 2.0 to guarantee 
math displays of good, high-quality TEX typesetting by using images has as a drawback all of the problems associated with large images. They can increase the transfer 
time of the document substantially, they occupy lots of diskspace, and they offer no 
possibility of image reuse. Nevertheless, this mode is still a valuable alternative with 
older browsers and when transfer times are not too much of an issue. 
3.3.6 Future use of MathML 
The parsed mathematics of expert mode goes some way toward producing valid 
MathML markup. Some work with ETEXZHTML has already been done to drive 

 
%%page page_136                                                  <<<---3
 
3.4 Support for different liguages 115 
Math examples 
1 n+ian K 
MA) = exp{'u11‘m + in} (11: for c 2 0 (1) 
::-im 
A = ‘f--y'-a'-zn--L (2) 
7 = 0.577215 . .. [Euler's constant) (3) 
7' = 0.422784 . a . = 1 ~ 7 (4) 
s,E = actual/average energy 105 (5) 
Since (Q) or (:1) should now for arbitrary an wactors, it is clear fnat MA) = 71(8) and 
thatvhsn y = B[z) one 
\reffig{3-12}: Mathematics with images of complete environments 
3 ...the £y_’thaggr_i_a_13§ knew infinitely many solutions in integers to 1:3 + 52 = :2. That no 
non-trivial integer solutions exist for a“ + b“ = c“ with integers 1: > 2 has long been 
simpectad (E_e_:mm,_g_._1g3Z), Only during the curl-am decade has this been proved (3§mgs,__1_9Q§). 
-- - {;I9nV|t;:_;I¢!5:; 
- {"-an-v+S*Z"-ea-I4:-ES-«Ll 
T‘ u :71‘. 1‘ I . 3’ 
ntyamamdt (.1 
W--£E‘a'E"~.'.'+! M 
ma. -4'-'_ ' «>1 
V’ ‘A 
\reffig{3-13}: Mathematics using “thumbnails" hyperlinking to full-size images 
the WebEQ wizard, producing Java applets and files of MathlVIL markup. For 
these, <APPLET> tags are produced for HTML 3.2 and <OBJECT> tags for HTML 4.0. 
However, at present this procedure works only with a subset of IATEX mathematics, so it is not described here further. In the future this work will be expanded to 
include all aspects of ETEX math covered by MathlVIL. 
3.4 Support for different languages 
Through the use of special fonts and macro packages, TEX allows typesetting in 
arbitrary languages. Much of this software can also be used with IATEXZHTML. For 
many languages an appropriate interface already exists, as part of the ETEXZHTML 
distribution. For others it is not too difficult to provide an interface by adapting an 
already existing one. 

 
%%page page_137                                                  <<<---3
 
1 16 The ILVIEXZHTML translator 
There are several ways in which ETEX2 HTML can provide support for creating 
HTML pages, using languages other than English. 
Titles and keywords. For many languages all that is needed is a translation of 
words and titles that IATEXZHTML automatically places on the HTML pages. 
Examples are titles like Chapter, Abstract, Bibliography, Index, and navigation 
commands like, Next, Up, and Previous. The title “About this document..." for 
the Information page should also be translated as well as the actual information 
contained on that page. 
Character-set encoding. Some languages use characters that are not part of the 
Latin 1 (ISO-8859-l) font encoding. Many Western European and Scandinavian languages are covered by Latin 1, while the Latin 2 (ISO-8859-2) encoding covers most Eastern European languages based on the Latin alphabet. 
The Latin 3 (ISO-8859-3), Latin 4 (ISO-8859-4), Latin 5 (ISO-8859-9) and 
Latin 6 (ISO-8859-l0) encodings cover some Mediterranean, Middle Eastern, 
and Northern European languages. IATEXZHTML provides the means to process correctly documents written using these encodings. Furthermore, there 
is limited support for the Unicode encoding and associated entity names for 
letters and accented characters. 
Images (using special fonts). Explicit font switching macros can be recognized 
by IATEXZHTML. An image can then be made of the appropriate portion of 
text, provided it is clearly delimited using braces. This is quite effective with 
single words and short phrases, but for larger chunks of text, separate images 
are made for each paragraph. 
Preprocessing and images. VV"1th more complex languages, such as those used 
in India and Southeast Asian countries, various transliteration and transcription schemes allow a codified representation using the characters available on 
a standard Latin keyboard. Preprocessing software to convert these encodings 
into TEX commands to access fonts that display the traditional forms already 
exists. IATEXZHTML is ideally suited for scripting such preprocessors to perform the required conversions before generating images for the traditional text. 
Appropriate interfaces to several preprocessors of Indic languages are included 
with ETEXZHTML under the name of Indic-TEX/HTML; the preprocessing 
software itself must be obtained separately. 
The support available for each of these strategies is discussed in more detail in 
the following subsections. 
3.4.1 Titles and keywords 
The ETEX2 HTML distribution includes Perl package interfaces for most of the languages and dialects covered by Johannes Braams’s Babel system. These are named 

 
%%page page_138                                                  <<<---3
 
3 .4 Support for different languages 
117 
english.perl, germa.n.perl, usorbia.n.perl, and so on. They correspond to the 
language definition (.ldf) files of the Babel distribution. However, for complete 
language support with IATEXZHT ML, more keywords than those present within 
the ldf files are required. 
Typically, such a “language interface" file defines a Perl subroutine for setting the titles and keywords, for example germa.n_t it les. Moreover, two variables, 
$default_la.nguage and $TITLES_LANGUAGE, are set to the language-identifier 
string, for example, german. Furthermore, there may be special macros or character 
sequences defined for use with the Babel package under IATEX that differ from the 
standard English/American use of IATEX. For example, there are sequences such 
as ``a, ''e, ..., "u for typesetting umlaut accents, and other sequences for dealing 
with quotation marks and hyphenation effects. The perl file contains appropriate 
code to allow these character sequences to be translated correctly for the HTML 
s. VVhen the language uses characters that are not available in the Latin 1 character set, an appropriate file to support the required encoding will be loaded (see 
Section 3.4.2). 
For some languages, in particular Dutch, Finnish, French, German, Spanish, 
Swedish, and Turkish, all strings needed by E'I];X2HTML have been translated, 
while for other languages and local variants, such as Afrikaans, Bahasa, Brazilian, 
Breton, Catalan, Croatian, Czech, Danish, Esperanto, the titles and keywords are 
taken from Babel’s ldf files. Additions are sought to fill some gaps; improvements 
and corrections are most welcome. 
A “language interface" file can be loaded in various ways so that translated 
words and phrases should override the English defaults. 
o Copy the contents into the latex2html . conf ig file and replace the English 
strings that are defined there. This sets the new language as a default for all 
documents processed with that installation of IATEXZHTML. 
o Create a copy of the latex2html.config file, and replace the defaults, as 
shown earlier. Now set the L2HCONFIG environment variable to load this alternative configuration file. This allows alternative language configurations to 
be used with the same BTEXZHTML installation. One needs only to adjust the 
value of the L2HCONFIG environment variable. 
o VV"1thin a .1atex2html-init file include a Perl command, such as 
&do_require_package ( " $LATEX2HTMLSTYLES/german .perl ") ; 
o Put I5TEX-like code within the document itself to use one of the packageloading mechanisms discussed in Section 3.2.4. Any of the following IATFX 
commands should work: 
\usepackage{germa.n} \usepackage [german] {babel} 
\docu.ment style [german] {style} \docu.ment class [german] {clam} 

 
%%page page_139                                                  <<<---3
 
118 
The ETEXZHTML translator 
where, in the last example, the language name should be a valid option to 
the <cLa.ss> document class. In fact, this is required only for ETEX to process the document correctly or if images need to be generated using IATEX. 
IATEXZHTML loads the appropriate perl file and attempts the translation to 
HTML, irrespective of the document being valid for IMEX or not. 
Multilingual documents can be produced using the babel package. Language 
segments can be presented with distinctive styles, as explained in Section 3.4.3. 
3.4.2 Character-set encodings 
Eight-bit language encodings, even for European languages using the Latin alphabet, use different glyphs in their upper range (codepoints 128-255). Upper-range 
characters present in the BTEX source are passed through to the HTML pages unchanged,5 when they are part of ordinary text. Thus for the browser to display the 
correct characters, it is necessary to know which character encoding is used. The 
encoding is specified as an attribute of the <ME‘.TA> tag inside the HEAD element part 
of the HTML document, for example: 
<HEAD> 
<META HTTP-EQUIV=``Content-Type'' CONTENT="text/html" 
CHARSET="'1so-8859-1"> 
</HEAD> 
Unless otherwise specified, IATEXZHTML uses the Latin 1 (ISO-8859-1) encoding since this is expected with HTML 3.2. Other encodings Latin 2 to Latin 5 
(see Table C.4), and Unicode (UTF-8) are supported. With Unicode, named entities (for instance, for accented characters) are converted to numerical character 
entity references, even with HTML 4.0, since many named entities are not among 
those listed at the W3C site [9 HTMLENTS]. This includes many characters from 
the various Latin 2' encodings just mentioned. 
For each supported encoding there exists a file in the $LATE‘.X2HTMLVERSIONS 
subdirectory, with a name such as latinl .pl, . . ., latin5.pl, unicode .pl. There 
are various ways to load these files and change the encoding. If several files are 
loaded, the last determines the encoding for the HTML pages, except whenever 
unicode .pl is one of them, in which case the UTF-8 encoding is used. 
Encodings can be loaded in various ways: 
o Load support for a particular character set, for example: 
latex2html -html_version 3.2,math,latin2 mydocument.tex 
5Exccpt for occurrences of case-changing macros such as \MakeUppercase, \MakeLowercase, 
\uppercase and \lowercase. 

%==========140==========<<<---2
 
%%page page_140                                                  <<<---3
 
3.4 Support for different languages 
119 
latex2html -html_veIsion 4.0,latin5,latin2,unicode mydocument.tex 
VVhen more than one encoding is specified, all the corresponding files are 
loaded, but HTML pages will use only the last one (for instance, unicode on 
the second line). This can be overridden by a \selectla.nguage command in 
the babel package in the document preamble; see Section 3.4.3. 
o Load the inputenc package within the preamble of the document, specifying 
the appropriate encoding, for instance, \usepackage [lat in2] {inputenc}. 
0 Load a “language interface" file, as explained in Section 3.4.1, which implicitly 
loads support for a particular character set. 
3.4.3 Multilingual documents using babel 
As mentioned earlier, special support for particular languages can be obtained by 
loading the babel package with options for the desired languages. VV"1thin a document the \selectla.nguage command chooses which language to use for a given 
portion of the text. The use of LANG tags described later assumes that HTML 4.0 is 
being used; otherwise no such attributes are put into the HTML pages. 
A \se1ect1anguage{language} command in the document preamble defines 
the language to be used for titles, keywords, and the navigation panels. VVhen there 
is more than one \select language command, the last is used. In practice, the ISO 
639 (ISO:639, 1988) identifier (see Table C.1 on page 466) for the language is used 
as value for the LANG attribute of the <BODY> tag. Furthermore, the document encoding is adjusted to suit this language. For example, <BDDY LANG=``tr''> results 
from \se1ectla.nguage{turkish}, and, furthermore, the document encoding becomes Latin 5 (ISO-8859-9). A different dialect of a language can result in a customized date format on the title page; for example, \selectla.nguage{america.n} 
and \selectlanguage{english} differ in this respect. 
For uses of \selectla.nguage {language} within the body of the document, 
the encoding is not affected. Instead, when the language is different from that of 
the document as a whole, all paragraphs within the specified portion inherit the 
appropriate LANG attribute. For example, \selectla.nguage{austria.n} will start 
subsequent paragraphs with <P LANG=``de-AT''>, until the next \select language 
or the current TEX grouping closes. Similarly <TABLE> tags inherit a language attribute. 
\htm11a.nguage style{germa.n} 
The command \htmlla.nguagestyle that is defined in the html package associates a class name with subsequent paragraphs, for instance, <P LANG=``de'' 
CLASS=``de''>. Thus a specific style can be associated with a given language, in this 
case German (de). 

 
%%page page_141                                                  <<<---3
 
120 
The HIFXZHTML translator 
3.4.4 Images using special fonts 
Both I6TEX’s \newfont and TEX’s \font commandsé receive special support in 
ETEXZHTML. These are used to define a macro for using a special-purpose font 
with a small portion of text. For example, font-selecting macros can be defined as 
follows, perhaps within a separate style file or document class file. 
\font\wncyr=wncyr10 '/. Cyrillic Roman font 
\font\wncyi=wncyi10 7. Cyrillic Italiced font 
\newfont{\SHa}{sinha10} 7. Haralambous’ Sinhala Font A 
\newfont{\SHb}{sinhb10} '/. Haralambous’ Sinhala Font B 
\newfont{\SHc}{sinhc10} '/. Haralambous’ Sinhala Font C 
Such macros are normally used declaratively within braced groupings, as follows: 
({\wncyi Russko-Singal\char126ski\char26\Slovar\char126\/}) , 
{\wncyr Rus\-ski .. . 
VVhen processed by IATEXZHTML, a single image is made of an entire grouping. This provides a way to include several languages within the same HTML 
document without the need to worry about character-set encodings or whether 
a browser is capable of displaying particular fonts. However, the problems with 
images are similar to those encountered with inline mathematics. As well as using 
Cyrillic fonts, the above lines of TEX source show code generated from a transliteration scheme for Sinhala. Such techniques are discussed next. 
3.4.5 Converting transliterations using preprocessors 
Languages using alphabets not based on Latin characters present a separate set of 
problems for preparation of compuscripts, for example, TEX or ETEX source documents. The usual solution is to use a “transliteration scheme"7 whereby a collection 
of Latin characters can represent a single character or syllable within the language 
being represented. Documents are prepared using character strings constructed according to the transliteration scheme. Using some other piece of software, this is 
then preprocessed to create TEX code for accessing the appropriate character, or 
set of characters, from a specially designed font. 
The example in \reffig{3-14} is from a document prepared this way for the 
Singhalese language script from Southern India and Sri Lanka. It first shows what 
the author actually typed, with #S and #N delimiting the Sinhala portion (lines 13). Then it shows the TEX code after preprocessing (lines 5-10) and finally the 
resulting output as seen on the Web. 
6The \newfont and \font commands are not the preferred way to define font commands in 
ETEX2g, which uses the New Font Selection Scheme. Nevertheless, they still work, appear frequently 
in older manuscripts, and are convenient for single fonts at specific sizes. 
7Here there is no need to make a distinction between “transliteration" and “transcription." The 
single term “transliteration scheme" is meant to encompass both of these related concepts. 

 
%%page page_142                                                  <<<---3
 
3.4 Support for different languages 
\bibitem{be1ko} {\wncyr Be1\char126koviq, A.A.} #Srusiyaanu"si.mha1a 
"sabdako.saya#N ({\uncyi Russko-Singa1\char126ski\char26\ 
S1ovar\char126\/}), {\uncyr Rus\-ski\char26\ \char23zyk}, 1983. 
\bibitem{be1ko} {\uncyr Be1\char126koviq, A.A.} 
{\SHb\char29a\char8}{\SHb\-\char69i}{\SHb\-\cha:r21a\charO}{\SHa\-\char213u} 
{\SHb\-\char53i}{\SHa\char1 1}{\sHb\-\char77a}{\sHb\-\char37a}{\sHb\ char53a} 
{\sHa\-\uhar237}{\sHb\-\ehar163a]-{\sHa\-\cha:5\char77a\cha:7}{\sHb\-\cha:61a} 
{\SHb\-\char21a} ({\wncyi Russko-Singa1\char126ski\char26\S1ovar\char126\/}), 
{\uncyr Rus\-ski\char26\ \char23zyk}, 1983. 
Eooux-a\m.aL»~._. 
‘ Bibliography 
: 1 Benaaoaaa, A.A. olaeaaga-mg G‘a53Q¢3m3§¢5( 
Pycc!eo-C‘hmaa.am:ut£ C/noaapb). Pycc:-mi Haws. 1983. 
2 C1ough,Rev.B.,£3'tD(3 9-58 qmcsogca (&‘md\*»?m¢~.E3:g211v.?tJ§I£.712r:9ioJz2ry), 
Wesleyan Mission Press, Kollupifiya, 1892, facsimile edition by Asian Educational 
Services, New Delhi, 1982. 
\reffig{3-14}: Example of preprocessing and transliteration with BTEXZHTML 
The preprocessor used with this example was lndica, which comes with 
Sinhala-TEX [G> SINTEX], developed by Yannis Haralambous. Indica handles 
twelve different lndic languages or dialects by interpreting four different transliteration or transcription schemes. It can create output for use with TEX and BTEX 
using a set of three special Sinhala fonts, or it can translate directly into Unicode 
numeric character codes. 
As well as handling the TEX source resulting from preprocessing, such as in 
\reffig{3-14}, BTEXZHTML can also translate the original compuscript to HTML 
without the initial preprocessing step. Of course, the image-generation mechanism 
must be used, but now the appropriately marked segments (such as those delimited by #S and #N) are collected into a file named images .pre. It is this file that is 
preprocessed, using Indica, to create the usual images .tex file that will be processed by ETEX, dvips, Ghostscript, and so on to create images for the HTML 
s. \reffig{3-15} presents part of the translation of a sample file that accompanies 
Sinhala-TEX [QSINTEX]. 
VVhen translated from the original (not preprocessed) source, each transliterated paragraph creates a separate image in order to keep image size at an acceptable 
level. This strategy is generally better than having an image for each letter or syllable, which can result in hundreds of images on a single page, even though they 
could be reused on different pages. The width of such images is determined by the 
$PAPERSIZE variable. In \reffig{3-15} a value of $PAPERSIZE = ’b5’ ; was used. 
Notice that, as with large images of mathematics, the original transliterated 
source is included as an HTML comment, preceding the image for each paragraph. 
VVhen it is not too long, it is also included as the ALT attribute for the image, per121 

 
%%page page_143                                                  <<<---3
 
122 
The Ié'l'EX2HTML translator 
f ‘$916911; gun ®@@t9’¢D E336; gem £c,::s$@2s3'.?" @255 evzsficicscaasfi 25J(f,<$;@ 
253636-92:5 d.c-mm caedomiflgoa ac.-. gggmaéozsfi cggem. 3 
1 “E38 :§2nDa8®ca gees" 
“c-3E-D oqcazfi-DEDES E-Dams. cfimcamaa (3353 {Queen gqea" 
‘*&‘JD (@155 09:36:?" 
“ozzmmaq @3610?" 
\reffig{3-15}: Sample of Singhalese (or Sinhala) script produced using the Indica 
preprocessor to interpret transliterated source 
haps with awkward characters being replaced by TEX equivalents. The HTML code 
for the second and third paragraphs of \reffig{3-15} follows. Notice in particular how 
the transliterated source is given, first as a comment (lines 1-3 and 9-11) and then 
with the ALT attribute (lines 7 and 15). 
1 <!-- INDICA S 
2 “ov janava~riye i"ndha1a~" 
3 --> 
4 <P><IMG 
5 WIDTH?-``186'' HEIGHT=``22'' ALIGN=``BOTTOM'' BORDER=``O'' 
6 SRC="img2.gif" 
7 ALT="\1q\1q ov janava~riye i"ndha1a~""></P> 
8 
9 
<!-- INDICA S 
10 “me~ dhesa"mbar ma~se. ethakota labana ma~se i"ndha1a~" 
11 --> 
12 <P><IMG 
13 wIDTH=``430'' HEIGHT=``22'' ALIGN=``BOTTOM'' BORDER=``O'' 
14 SRC="img3.gif" 
15 ALT="\1q\1q me~ dhesa‘mbar ma~se. ethakota labana ma~se 1"ndha1a~""></P> 
VVhichever translation method is used, Sinhala-TEX [=>SINTEX] and Indica 
must be available on the local system so that they can be scripted for use. Interface files in the subdirectory $LATEX2HTMLDIR/Indic-HTML/ accompany the 
HTEXZHTML distribution. These are ETEX packages for the various lndic languages, together with corresponding Perl implementations. Source documents 
need to load an appropriate package with options corresponding to the preprocessor and transliteration scheme. For example, either of the following lines is appropriate for a document with Singhalese in the sammmla transcription: 
\usepackage Eindica, samanala] {sinhlese} 
\usepackage Esinhala, samanala] {indica} 
Installation of Indic-TEX/HTML consists of making the language interface files 
available to ETEX and ETEXZHTML (file names are limited to a maximum 
of eight characters in the prefix so that some names look a little strange). 

 
%%page page_144                                                  <<<---3
 
3.4 Support for different languages 
The files should be copied, moved, or linked into the TEX hierarchy and the 
$LATEX2HTMLDIR/styles/ subdirectory, or their locations should be recorded in 
configuration variables, as discussed in Sections 3.2.3 and 3.2.4. 
Similarly, the file indica.per1 must be available to BTEXZHTML. Near 
the top of this file some Perl variables are assigned values. These may need 
to be adjusted for the local installation, or new values can be assigned in the 
latex2html.config or .1atex2htm1-init files. Most important is $INDIGA, 
which should hold the full path to the preprocessor, if specifying the command 
indica by itself is insufficient at runtime. 
3.4.5.1 Supported languages and preprocessors 
Other preprocessors for lndic languages are supported in a similar fashion, under 
the title of Indic-TEX/HTML; see TUGIndia Journal [=>TUGINDIA]. Since this 
article appeared, support has been added for Avinash Chopde’s itrans [9 ITRANS] 
preprocessor, so that more languages and transliteration schemes can be translated 
using IATEXZHTML. lndic-TEX/HTML now supports the following preprocessors 
and fonts: 
Indica supports the Bengali, Gujarati, Gurmukhi, Hindi, Kannada, Malayalam, 
Oriya, Sanskrit, Sinhala, Tamil, Telugu, and Tibetan languages, using Velthuis, 
CSX (ISO-646 extended), sammmla [HSAMANALA], and a standardized BTEX 
transliteration. Special fonts sinha, sinhb, and sinhc were designed by Yannis Haralambous. A BTEXZHTML translation [9 SINDOC] of the Sinhala-TEX 
[G> SINTEX] documentation is available. 
itrans supports Bengali, Devanagari, Gujarati, Hindi, Kannada, Marathi, Punjabi (Gurmukhi), Tamil, Telugu, and Romanized Sanskrit, using a variety of 
fonts-for example, It-.xBeng and BwTI Bengali, devnac and Xdvng, ItxGuj, 
kan, pun, tel, tamil, wntml, and CSUtopia. 
devnag from the University of Washington, translates Devanagari in the Velthuis 
transliteration for Hindi, Marathi, Nepali, and Sanskrit. It uses the dvng font. 
tamilize and tmilize translates to Tamil script using the wntml font, produced 
at the University of Washington. 
patc and mm preprocessors used with the Malayalam-TEX [HMALAYALAM] system by Jeroen Hellingman. They support Malayalam in either a traditional or 
reformed script, using the m fonts and a romanized form using accents and 
diacritical markings. Also there is support for various transcriptions of Tamil, 
using the wntml font, and of Devanagari, using devnag and the dvng font. 
For each of these preprocessors, interface files accompany ETEXZHTML as 
part of Indic-TEX/HTML. The preprocessors themselves, and whatever fonts they 
use, must be obtained separately and installed for use on the local platform. 
123 

 
%%page page_145                                                  <<<---3
 
124 
The ETEXZHTML translator 
For itrans [HITRANS] the file it-.rans.per1 currently handles all of the languages supported by itrans [HITRANS]. In the future it should become possible 
to use itrans as an option to specific language packages, as with Indica, and cause 
other language-specific customizations to be employed. Currently, the following 
commands are equivalent ways to load itrans [9 ITRANS] to translate a document 
in Sanskrit, with the Classical Sanskrit Extended (CSX) transliteration (the appropriate command for other languages and transliterations have similar forms). 
\usepackage [csx] {itrans} 
\usepackage [csx , sanskrit] {itrans} 
Near the beginning of the file itrans.per1 some Perl variables are set. These may 
need to be adjusted for the local installation-for instance, by assigning new values 
in the 1atex2htm1. config or .latex2htm1-init file. The variable SBITRANSPATH 
ensures that the preprocessor can find all the resources it needs. Another variable 
$ITRANS specifies the command to run itrans [9 ITRANS] on the local installation; 
it should hold the full path to the preprocessor if the command itrans by itself is 
insufficient at runtime. 
A IATEXZHTML translation [GHTRANS] of the itrans [GHTRANS] documentation is available from Avinash Chopde’s site. This describes the transliteration 
scheme used with each language. Figures 3.16 and 3.17 are extracts from these 
s. 
3.5 Extending ETEX sources with hypertext 
commands using the html package 
IATEXZHTML comes with a BTEX package file called html that defines macronames, commands, and environments. These allow hypertext linking and other 
special effects to be included in the generated HTML pages. Some commands have 
no effect when typesetting the compuscript with ETEX, while other commands 
use different information when creating a DVI file with ETEX or HTML files with 
BTEXZHTML. 
The commands defined in the html package can be grouped into several classes 
according to their functionality and purpose. Among these commands the variants 
that have an optional argument do not work with I§TEX2.09; their use requires 
I5TEX2g. 
o Commands that create hyperlinks to external Web pages or images, that is, 
hyperlinks to information that is not directly part of the document nor appears 
when BTEX processes the same source. 
0 Commands that present the same information as with BTEX but that can incorporate extra visual effects in the HTML version. 

 
%%page page_146                                                  <<<---3
 
3.5 Extending I-LVIFX sources with hypertext commands using the html package 125 
Nehru'sWords 
!rla1fi'Ivltn1wlgaf!*wwrfiI'fi‘k‘w13I‘irmrit'trwri‘=fi' 
wfixurarrnrgrrrn mwgwmrfitfitfifinwnfi 
«Hr fir Ira“ Flrl!rwN1’ri*itfhi1’4°:wnr?rur<a' 
shtawmirtuirrfr ifim'n:ih:i*=fi-aa%srfirau%@' 
aylififirzififimrngdkmfimrfitnw 
wmfifiwir 
it it 
jfiglgtxsmmm “twat: 
\reffig{3-16}: Sample of Devanagari script for the Hindi language, set with the 
devnag font and produced using the itrans [<-> ITRANS] preprocessor 
= New some sample text in romanized samh-it and then in sanskrit 
' kannagxyavndlukflmsumaptuksukaddamaj 
nudka:maphaIahem:i:h&rni&aunrI3a.a:tvakqm:ag:l||47|| 
ya , ,kuru£umndq¢isurigamIyakwddim#atnjaya| 
wb Jam bhflivdswnamwnyoga ucyate |] 48 || 
5 w 
wfimfififigglfiglnnruu 
fllwnficttfifi-:n=fl't[cuTu=zrtuTrrawirnw:I 
\reffig{3-17}: Sample of Sanskrit showing both traditional romanized forms, produced using the itrans [*->ITRANS] preprocessor 
o Commands that allow alternative text to be used with active hyperlinks for 
citations and cross-referencing. 
0 Some conditional environments that allow one to use different information 
with the DVI and HTML versions of a document. 
0 Commands that customize the visual layout of the HTML pages produced by 
BTEXZHTML and that have no counterpart in the DVI version. 
o Special macros that help with development and debugging, for instance, to 
control the amount of tracing information written to the screen while the document is being processed. 

 
%%page page_147                                                  <<<---3
 
126 
The I1}TEX2HTML translator 
o Macros that are designed to facilitate the use of the special Document Segmentation feature. This feature allows BTEXZHTML to treat a large source document in smaller parts, nevertheless, keeping the cross-reference information 
between the various parts intact as though the HTML structure were generated 
from a single-source document. 
In the following sections we discuss the commands belonging to each of these 
classes and give some examples of the most effective ways to use them. 
3.5.1 Hyperlinks to external documents 
Commands for hyperlinking to Web resources are \htm1addnorma1link and 
\htmladdnormallinkfoot. Also \htmladdimg is specifically for an inline hyperlink to an image that uses the <IMG> tag. This displays the image on the HTML 
, as if it were part of the document. 
\html addnormall ink [name] {link-text}{ URL} 
\htmladdnormall inkf oot [name] {link-text}{ URL} 
On the HTML page the link-text shows in the browser as an active hyperlink, usually 
in a different color or underlined. By clicking on the link-text, you will transfer to 
the target URL. In the BTEX version the link-text is typeset normally, giving no 
indication that anything about the particular piece of text is special. The optional 
name allows the place on the HTML page to be marked with a meaningful name. 
Generally it is poor style to have long URLs in sections of prose within a 
printed document; it simply looks bad, quite apart from practical problems concerning linebreaking and hyphenation. When such information must be provided 
explicitly, \htmladdnormallinkfoot is a good way to do it. This acts the same as 
\htmladdnormall ink on the HTML page but includes the URL as a footnote when 
typeset by ETEX. 
The \htm1addnormallink and \htm1addnormallinkfoot commands can be 
used directly in the body of the document source, although it is recommended not 
to do this. Instead they are best used within expansions of macros defined in the 
document preamble or within input files containing many such definitions. Generally the URL will refer to a location whose address is beyond the immediate control 
of the document’s author. If that address changes or ceases to become valid for some 
reason, the hyperlink needs to be changed. Collecting together all the URLs used 
within a document allows for easier maintenance, including such updating. 
For instance, the mathematics examples in Section 3.3 refer to these commands, as seen in lines 9-15, for the definitions, and lines 31-33, for their use. 
How this shows up in a browser can be seen in \reffig{3-4} and other figures in that 
section. 

 
%%page page_148                                                  <<<---3
 
3.5 Extending I1}'I]3;X sources with hypertext commands using the html package 
127 
\htmlurl{ URL} 
This special macro has the following property: If a macro \url is already defined 
from another package such as url, then \htmlurl binds directly to it. Similarly, 
if the other package is not yet loaded, but will be, then \htmlurl will adopt the 
same functionality and expansion as \url. Without another package, \htmlurl 
prints the URL in small typewriter style, with no provision for special characters or 
hyphenation, and \url is defined to be equivalent to \htmlurl. 
The main use for \htmlurl is within expansions of other macros, especially 
when the same URL is to be used in several contexts. For example, 
\newcommand{\rossURL}{\htmlurl{http://www.maths.mq.edu.au/~ross/}} 
\newcommand{\thisauthor}{\htm1addnorma1link{present author}{\rossURL}} 
\newcommand{\RossMoore}{\htmladdnorma1linkfoot{Ross Moore}{\rossURL}} 
gives the expected hyperlinks whenever \rossURL, \thisauthor, or \RossMoore 
are used in the document. Furthermore, the correct styles, hyphenation, and treatment of special characters are applied in the I/}TEX version. 
\htmladdimg [attributes] { URL} 
This command constructs an <IMG> tag that sources the specified URL. VVidth, 
height, and alignment information can be given as key=va.I.ue pairs in the optional attributes. Thus images can be displayed at sizes other than their natural 
size. You can use alignment to let text “flow" around an image-for instance, 
\htmladdimage [ali gn=1ef t] places the image at the left-hand side of the browser 
window, with text and paragraphs flowing down the right-hand side. 
The following example comes from the bibliography of the BTEXZHTML 
Users Guide. It shows the effect obtained in a browser. 
\newcommand{\AWcsengURL}{http://www.aw1.com/cseng} 
\newcommand{\AddWes}[1]{\htmladdnorma1link{Addison--Wesley}{#1}} 
\newcommand{\AWtheLGC}{\AWcsengURL/titles/0-201-85469-4} 
\newcommand{\LGCcover}{\AWtheLGC/coversm.gif} 
\htmladdimg[ALIGN=RIGHT]{\LGCcover} 
\bibitem{goossens:1atexGraphics} 
Michel Goossens, Sebastian Rahtz and Frank Mittelbach, 
\newb1ock \emph{The }\LaTeX\emph{ Graphics Companion}. 
\newb1ock ISBN 0-201-85469-4, Softcover 608 pages, 
\AddWes{\AwtheLGC/3', 1997. 
\bibitem{drakos:bask} 
\NikosDrakos, 
\newb1ock Text to Hypertext conversion with \latextohtm1. 

 
%%page page_149                                                  <<<---3
 
128 
Michel Goossens, Sebastian Rahtz and Frank Mitteltnach, 
fine L*'*TE.X t'=}v;zz.rir.i*:<* t'.‘1am}7@‘wiafi . 
ISBN 0-201-85469-4, Softcover 608 pages, Addison-Wesley; 1997. 
Nikos 
Text to Hypertext conversion with IJTEXZHTML. 
fiaekermffe, December 1993, Vol.3, No.2, pp 12-15. 
May 1994, CERN, Geneva, Switzerland. 
The HIFXZHTML translator 
rakos 
httnzffcbl.lnedsar:.uk!r1ikos!doc!www94.fwww94.html 
Note how both entries How to the left of the image. Macros used here follow the 
principle that information that may be reused, such as publisher’s data, should be 
defined in a single place, preferably in the document preamble. 
\htmlru1e [attributex] 
\htm1rule* [attributex] 
The \htmlrule command places a linebreak and a horizontal rule on an HTML 
; the *-version omits the linebreak. On a printed page, \hrule commands 
should be used sparingly as separators, since some extra whitespace is usually 
enough. However, browsers frequently put too much whitespace, so that breaking 
this space with a rule has greater visual appeal. Properties, such as width, thickness, 
and shadow, are included as a list called attributex; key='ua,lue pairs are recognized, 
but if the keys can be deduced, it is sufficient to provide values only, as in the following example: 
<BR CLEAR=``LEFT''> 
<HR WIDTH=“50%" SIZE=“5" NOSHADE> 
\htm1ru1e[50\% 5 noshade left] 
3.5.2 Enhancements appropriate for HTML 
\latextohtml 
This generates the ETEXZHTML logo, as in Example 3-5-1 in the previous section. 
Related logos such as TEX, BTEX, AMS, and )6] have special translations built into 
IMEXZHTML. More are available with the texnames package. 
\begin{makeimage} 
This environment forces an image to be made of its contents, and typeset to the 
width determined by the $PAPERSIZE variable. Since the default translation of an 
unusual environment is to create an image, its main use is to override the usual 
translation. For example, sometimes a table looks better when fully typeset by 
HTEX, or special formatting within paragraphed text may be desired. VVhen using 
the seminar package, the slide environments are often best handled as images. 

%==========150==========<<<---2
 
%%page page_150                                                  <<<---3
 
3.5 Extending I*}T]‘;X sources with hypertext commands using the html package 
129 
The environment’s contents are scanned for any \label commands, so that crossreferences can work as active hyperlinks. 
Normally a figure environment is transformed as a whole into a single image. 
However, when it contains one or more makeimage subenvironments, the contents 
of the figure are normally translated into HTML as a separate <DIV> division. 
Images are made exclusively of the contents of each makeimage subenvironment. 
In particular, you can use an empty makeimage environment to prevent a figure 
environment from being turned into an image. 
\htmlimage {gmpbic ejfictr} 
This command forces the creation of an image for an environment that would normally not be processed this way. It can also be used to include special graphic effects 
when an image is generated. 
Environments affected by the \htmlimage command include figure and 
table, displaymath, equation, eqnarray, math, inline mathematics, and the 
aligned environments of the amsmath and amstex packages, as well as slide from 
the seminar package, and any environment from another package having no special support in ETEXZHTML. The \htmlimage command with its (perhaps empty) 
argument may appear anywhere inside the environment to force an image to be 
made. ETEX ignores both the command and its argument. 
The graphic efleafr arguments that follow let you specify special effects to be 
used when creating the image. 
height=pixel3 set the HEIGHT attribute of the <IMG> tag to pzlxely. 
width=pz'xelr set the WIDTH attribute of the <IMG> tag to pixelr. 
align=option set the ALIGN attribute of the <IMG>. 
Recognized values are TOP, BOTTOM, MIDDLE, LEFT, RIGHT, CENTER. 
external create a hyperlink to an external image file, rather than placing the image inline. 
scale=rcale-faafor scale the image by the supplied rcale-factor. 
thumbnail=rmle-fartor combine both previous options, external and scale, and 
create a “thumbnail" version of the image, scaled by rcale-factor. This thumbnail 
is used inline as an active button hyperlink pointing to the full-size version. 
map=URL make the image into a server-side image-map; 
see the HTEXZHTML Users Guide for further details. 
usemap= URL make the image into a client-side image-map; 
see the IMEXZHTML Users Guide for further details. 
flip=option rotate or flip the image according to a recognized option: 
leftright, topbottom, rotate90 (or r90), rotate270 (or r270). 

 
%%page page_151                                                  <<<---3
 
130 
The ILVIFXZHTML translator 
transparent make the background of the image transparent when the default is 
to keep it opaque. 
notransparent or no_transparent keep the background of the image opaque 
when the default is to make it transparent. 
antialias request anti-aliasing of font characters within the image, irrespective 
of the default behavior for the image in question. 
noantialias or no_antialias do not use anti-aliased font characters, irrespective of the default behavior for the image in question. 
\begin [style-xbeet information] 
The addition of an optional argument to If}TEX’s \begin command is described 
with examples in Section 3.3.4. This is useful only when generating code for 
HTML 4.0 that supports Cascading Style Sheets (CSS). ETEX ignores the optional 
argument. 
The yzyle-xbeet information is written into the automatically generated . css style 
sheet file to be applied using CLASS and ID attributes. It can include names of extra 
CLASSes to which the subsequent environment should belong, as well as its preassigned default CLASS. Using an ID attribute, you can specify information to be 
applied exclusively to the subsequent environment. 
\htmlset style [tagx] {claxx}-{style-info} 
\htmladdtostyle [tagx] {elaxx}{xtyle-info} 
These commands are ignored in ETEX. Their purpose with ETEXZHTML is to 
write the style-info into the CSS style sheet. It will be applied to tags of type tagx 
and class Clan. The tagx may be a hierarchical list of tags, such as TABLE. TR . TD or a 
single tag, or they may even be empty. Similarly the class may be empty. Any string 
may be given as the style-info. Currently the only checking for validity is to convert 
“=" into “ : " and to replace any“, " with a space. 
\htm1languagestyle [claxx] {language} 
This command is to include the class claxx as an attribute for tags having language as 
the value of a LANG attribute. If clays is not specified, a default value equal to the ISO 
639 language code for language is used. Use a \htmlsetstyle command to apply 
particular style information. For example, the following code writes an entry into 
the CSS style sheet for setting German paragraphs with red text, using the babel 
package, in a document that was prepared mainly for a different language. 
\htm11anguagestyle{german} 
\htm1setstyle [P] {de}{color : red} 

 
%%page page_152                                                  <<<---3
 
3.5 Extending I:‘¥lEX sources with hypertext commands using the html package 131 
\htm1border{atm'butex} 
This command is similar to \htmlimage in that it affects the environment in which 
it occurs, requesting that a border be placed around the environment’s contents. 
It affects the minipage, tabbing, table, and makeimage environments, as well as 
those mentioned earlier for \htmlimage. It also affects theorem-like environments. 
The border is achieved by placing the contents inside a single <TD> cell of 
an HTML <TABLE> construct. The argument attributex can contain any attribute 
valid for HTML’s <TABLE> tag, such as a thickness for BORDER, alignment, width, 
CELLSPACING,andCELLPADDING 
\begin{htm1list} [marker] 
\htm1itemmark{marker} 
This environment is a replacement for the description list environment. It uses 
colored balls as bullets for the listed items. In ETEX it appears as an ordinary 
description list. This environment is defined in the htmllist package; it is usual 
to load this and the html package with a single command. Colors are specified with 
the marker argument, where marker can be RedBal1, OrangeBall, YellowBall, 
GreenBall, BlueBa11, PurpleBal1, PinkBall, or whiteBall. These correspond 
to icon images supplied with the navigation icons. 
\usepackage{htm1 , htmllist} 
Q Item 1: \begin{htmllist} [WhiteBall] 
This will have avhite ball. \item[Item 1:]This will have a white ball. 
Q’ Item 22 \item [Item 2:] 
This will also have avhite ball. This will also have a white ball. 
Q’ Item 32 \htmlitemmark{RedBall}'/. 
This villhavearecl ball. \item[Item 3:]This will have a red ball. 
\end{htmllist} 
\strikeout{text} 
This command applies <STRIKE> tags to the text, creating the effect of a line striking out the information as no longer applicable. The text is simply omitted from 
the ETEX typeset version. 
\item can deal sensibly with \strikeout{at least the \emph{Common} 
\LaTeX{} \emph{Commands} summarised at the back of\\}'/. 
Virtually all of the concepts and commands described in 
the \LaTeX{} \htmlcite{blue book}{1amp:latex}, where there is 
a meaningful interpretation appropriate to an \texttt{l-{TML} document. 
0 can deal sensibly with fi stunmarised-at-tlae-backefvirtually all of the concepts and commands described in the UTEX muilaggg, where there is a 
meaningful interpretation appropriate to an HTML document. Also many other LATEX 

 
%%page page_153                                                  <<<---3
 
132 
...HTML pages for which there is no direct I5TEX 
The I4‘I}3X2HTML translator 
3.5 .3 Altemative text for hyperlinks 
Page numbering has no meaning in an HTML document, since all navigation is 
done via active hyperlinks. Similarly numbering of sections, captions, and citations 
has reduced value because hyperlinks are used for cross-references as well. In addition to the If}TEX commands \ref and \cite, the commands described here offer 
alternative ways to handle cross-references and citations. 
\hyperref [keyword] . . . argumentx. . . 
This is a recommended replacement for the \ref and \pageref commands in 
ETEX. The optional keyword specifies which form is to be used when the document 
is processed by ETEX; the default is \ref. 
\hyperref [ref] {link-text}{pre-latex}{post-latex}{label} 
\hyperref {link-text} {pre-latex}{post-latex}{lalJel} 
\hyperref [page] {link-text} {pre-latex}{post-latex}{lalJel} 
\hyperref [pageref] {link-text}{pre-latex}{post-latex}{lalJel} 
\hyperref [no] {link-text}{latex-text}{lalJel} 
\hyperref [noref] {link-text}{latex-text}{lalJel} 
Each form in these groups of two yields identical results. The link-text is used only 
on the HTML pages, as text associated with a hyperlink that points to where the 
\label{lalJel} command occurred, in the same document, or in another document 
produced with ETEXZHTML. Only ETEX uses the pre-latex and post-latex parts. 
These are placed on either side of the reference text (or number) associated with 
the label. With the final pair of commands, ETEX makes no use of the label argument, although a hyperlink is still created for the HTML version. The example that 
follows clearly displays how the text of the various arguments is used differently by 
ETEX (shown at the left) and by ETEXZHTML (shown at the bottom). 
...\texttt{HTML} pages for which there 
is no direct \LaTeX{} counterpart. 
Most of these commands are discussed 
counterpart. Most of these commands are in detau in \hype1-1-ef{a 1ater section} 
discussed in detail in Section 4.8. 
{Section~}{}{misceffects}. 
The following commands implement effects on the HTML pages for which there is no direct 
L‘-TEX counterpart. Most of these commands are discussed in detail in a later section. 
For processing by BTEX there are macros corresponding to each option, 
namely, \hyperrefref, \hyperrefpageref, \hyperrefpage, \hyperrefno, and 
\hyperrefnoref. These commands process the arguments following the first optional one. The \hyperrefdef command handles the default case without any op
 
%%page page_154                                                  <<<---3
 
3.5 Extending WIEX sources with hypertext commands using the html package 133 
tion. This allows information that is presented with the help of a particular option 
to be used in a different way, simply by redefining a single command rather than 
the whole \hyperref structure. 
\hyperref [hyper] {URL}{catego7y}{mmze}{text} 
\hyperref [hyper] [label] {text} 
\hyperref [html] {link-text}{pre-latex}{post-lzztex}{lzzlJel} 
These additional variants were introduced to cope with the fact that Sebastian Rahtz’s hyperref package (see \refsec{2_3_Rich_PDF_with_LaTeX_The_hyperref_package}) also uses a command called 
\hyperref. The first form has no symbolic label, and its usage is similar to 
\htmladdnorma1link although it does allow for structure inside the hyperlink target. The second form uses an optional argument to hold the symbolic label, with 
the mandatory argument giving the hyperlink text. For both usages, by inserting 
the option [hyper] with each occurrence of \hyperref, ETEXZHTML will correctly process documents that load the hyperref package. On the other hand, a 
command starting with \hyperref [html] indicates that the given instance needs 
the version of \hyperref as implemented in the html package. Commands named 
\hyperrefhtml and \hyperrefhyper cater to these options for HTEX processing 
of the mandatory arguments. 
\htmlref{linle-text]-{llllyel} 
\htmlref [ext] {link-text} [prefix] {label} 
This is a shorter way to specify a hyperlink in the HTML version. It is equivalent to 
\hyperref [noref] with the latex-text and link-text being identical. For example, 
with \Verb|\htm1ref| 
\htm1ref{it’s easy to make links} 
With \htmlref it’s easy to make links. {f ig , examP1e}_ 
W1th \htm1ref it's e§ . 
The ext option signals that the reference is made to a location inside an external 
document. See the \externalref and \externallabels commands that follow 
for an explanation of how to use this mechanism and the optional prefix. For alternative ETEX processing, macros \htmlrefdef and \htm1ref ext can be redefined 
to cope with all but the first optional argument. 
\hypercite [keyword] . . . argumentr. . . 
This is a recommended replacement for the \cite and \nocite commands in 
BTEX. The optional keyword specifies the form to be used when the document is 
processed by ETEX; the default is \cite. 

 
%%page page_155                                                  <<<---3
 
134 
The I1}'IEX2HTML translator 
\hypercite [int] {link-text}{latex-text}{opt-latex}{label} 
\hypercite [cite] {link-text}{latex-text}{opt-latex}{label} 
\hypercite{linle-text}{latex-text}{opt-latex} {label} 
\hypercite [nocite] {link-text}{latex-text} [P*‘efix] {label} 
\hypercite [no] {link-text}{latex-text} [prefix] {label} 
\hypercite [ext] {link-text}{latex-text} [prefix] {label} 
VVhen typeset with BTEX, all commands of the first group act the same wayfirst placing the material in the latex-text argument, followed by issuing a command \cite [opt-latex] {label}. The second group of commands places a command 
\nocite{label} following the material in latex-text to force a bibliography entry, 
without any explicit reference marker on the printed page. 
The link-text argument corresponds to a hyperlink on the HTML page, which 
points to a location addressed by the symbolic label. The location could reside on 
a bibliography page attached to the Web document being built, or it could be at 
an external site. The use of symbolic labels available at external sites and the use of 
prefix are explained with the \externallabels command. 
For processing by BTEX there exists a macro corresponding to each option, 
namely, \hyperciteint, \hypercitecite, \hypercitenocite, \hyperciteno, 
and \hyperciteext. These commands process the arguments following the first 
optional one. The command \hypercitedef handles the default case without any 
option. This allows information presented with a particular option to be used in 
a different way, simply by redefining a single command rather than the whole 
\hypercite structure. 
\htmlcite{lin/e-text}{label} 
\htmlcite [ext] {lin/e-text} [prefix] {label} 
This is a shorter way to specify a hyperlinked citation on an HTML page. It is 
equivalent to \hypercite [nocite] with the latex-text and lin/e-text being identical; however, with the ext option, BTEX omits the bibliography entry. See the 
\externallabels command that follows for use of a prefix. For alternative BTEX 
processing, macros \htm1citedef and \htmlciteext can be redefined to cope 
with all but the first optional argument. 
References can also be made to a target identified by a symbolic label. Such 
labels are located in other Web documents created with ETEXZHTML and these 
documents can even reside at remote sites. In order to create hyperlinks pointing 
to specific locations in remote HTML pages the following command must be given: 
\externallabels [prefix] {URL}{loeal copy oflabels.p1file} 
The URL declares the remote location where the HTML pages are to be found. 
These pages become the target for hyperlinks in the local document. One needs 
first to obtain a copy of the file, usually called labels.pl, that contains infor
 
%%page page_156                                                  <<<---3
 
3.5 Extending ILYIEX sources with hypertext commands using the html package 135 
mation about the symbolic labels used when the remote pages were created. (See 
 92 where we mentioned the existence of that file when we discussed how 
ETEXZHTML operates.) To avoid name clashes between labels used in the various 
documents, it is wise to specify the optional argument prefix, which will be used as 
a prefix whenever a label from this site is referenced. This copy of labels .pl file 
can be located anywhere that is convenient on the local system. Its full path and 
filename must be given as the last argument of the \externallabels command. 
\externalref [prefix] {label} 
This is analogous to \ref for a label defined in a remote document. ‘VVhen typeset 
by ETEX, label will be ignored. In the HTML version, a hyperlink using a small 
icon as the visible marker is created. For a textual hyperlink, use instead \htmlref 
or \hyperref [noref] . See the \externallabels command, shown earlier, for the 
use of prefix. 
\externalcite [prefix] {label} 
This is the equivalent of \nocite for a bibliography entry residing in a remote 
document. VVhen typeset by BTEX, label will be ignored. In the HTML version, 
a hyperlink is created, using a small icon as the visible marker. For a textual hyperlink, use instead \htmlcite or \hypercite [ext]. See the \externallabels 
command, shown earlier, for an explanation of prefix. An example of how these 
commands are used is discussed in Section 3.5.6. 
3 .5 .4 Conditional environments 
Frequently there is material relevant only for the version of the document typeset 
by BTEX or, alternatively, only for the HTML version generated by ETEXZHTML. 
The expansion of a macro may need to be different in each version, perhaps for 
technical reasons. Special environments are provided to cope with such situations. 
\begin{htmlonly} 
The contents of the htmlonly environment are used only when the document is 
processed by ETEXZHTML and are completely ignored by BTEX. 
\begin{latexonly} 
The contents of the latexonly environment are used only when the document 
is processed by ETEX and are completely ignored by BTEXZHTML. Note that 
this environment has its contents contained within a TEX group and is within 
\begingroup. . . \endgroup. This environment cannot be used to set values for 
lengths or counters or to define macro expansions to be used only with ETEX. Use 
the construction described next when you need to do this. 

 
%%page page_157                                                  <<<---3
 
136 
The ILVIFXZHTML translator 
%begin{latexonly} 
%end{latexonly} 
These special comments have meaning only to ETEXZHTML. It discards them and 
the portion of document source that they surround. On the other hand, because 
BTEX simply sees the “Z," as a comment character, it ignores those lines completely. 
Contents are unaffected, so any \newcommand or \newenvironment definitions and 
adjustments to length or counter values will take effect normally 
\htm1{text for HTML only} 
\latex{textfor EYEX only} 
\latexhtml{latex-text}{/)tml-text} 
These are shorthand forms of the earlier environments, intended for use with small 
pieces of text. The command and its arguments are read, and perhaps processed, 
before being discarded. With large portions of input source, it is better to use the 
environments to be sure that unwanted side effects do not occur. However, there 
are situations where \html works and the htmlonly environment does not, due to 
the way TEX reads and tokenizes the input source. This is the case, for instance, 
with special characters like ‘‘8c’’ in tabular alignments. 
\begin{imagesonly} 
The contents of the imagesonly environment are written into the images .tex 
file for use with the BTEX code to be typeset when creating images for the HTML 
document. For example, you can set the background and foreground color for all 
generated images in this way. This feature is sometimes needed to force a correct 
macro expansion. 
\begin{comment} 
The contents of the comment environment are always ignored by both BTEX and 
ETEXZHTML. 
\begin{rawhtml} 
The rawhtml environment is used to insert raw HTML code directly into the document at the given location, with respect to text and paragraphing, where it occurs in the source. There is no checking of the included code to see whether it is 
well-formed or is valid in the context of the requested version of HTML. Such considerations are left entirely to the document author. See the following \HTMLcode 
command for a safer way to construct HTML code. 

 
%%page page_158                                                  <<<---3
 
3.5 Extending I§'I]§X sources with hypertext commands using the html package 137 
\HTMLcode [attributes] {tag} 
\HTMLcode [attributes] {tag}{tontent} 
This command constructs an HTML tag for the given tag with atu'ibutes that are 
determined by attributes, provided the tag is valid for the chosen version of HTML. 
The content argument provides the content when the tag being constructed requires 
an end tag. Invalid attributes for a tag are filtered from the list, but an invalid tag is 
rejected altogether. 
For the attributes, a list of k:ey='ua1.ue pairs can be given, separated by a space, a 
comma, or a newline character. Alternatively, it is sufficient to give just the desired 
values when it is possible to deduce which attributes are to accept those values. For 
example, 
<HR WIDTH="507." SIZE=``3'' 
NOSHADE ALIGN=``center''> \HTMLcode [50\"/. 3 noshade center] {HR} 
Macro expansion occurs for each of the arguments of \HTMLcode (tontent, tag, 
and attributes). This minimizes typing when using repetitive structures. \reffig{3-18} 
shows how this can be used in practice. 
End-of-line characters in a BTEX source are not ignored by ETEXZHTML, as 
they are by ETEX. Instead, they are retained in the resulting HTML files. Generally 
this does not affect the display constructed by browsers. However, it does affect the 
layout of the information in the . html files, should these be required to be read by 
a human. This explains why macro definitions are spread in the earlier example. 
The set of tags usable with the \HTMLcode command can be extended. This is 
done in Perl by defining for each new tag a “regular expression" that will match 
the names of its valid attributes. For each attribute there needs to be a regular 
expression to match the possible values that it will accept. If you are interested 
to see how this is achieved, you can examine the files versions/html3_2 .pl and 
versions/html4__O .pl that are part of the BTEXZHTML disuibution. 
3.5.5 Navigation and layout of HTML pages 
The commands described here are completely ignored by ETEX. 
\htmladdtonavigation{toa'e} 
This command provides an easy way to extend the navigation panel that is automatically produced by ETEXZHTML. As the visible marker for a hyperlink, the code 
argument typically uses other commands available for showing text or an image. 
Each use of \htmladdtonavigation appends exu‘a material, thereby allowing for 
quite extensive Customization. Such customization applies only within a single document. You can alter the navigation panel code provided with latex2htm1 . conf ig 
if you want to customize all documents to be processed by ETEXZHTML. 

 
%%page page_159                                                  <<<---3
 
138 
The PHEXZHTML translator 
\newcommand{\myalign}{center}\newcommand{\mylist}{UL} 
\newcommand{\myitem}[2]{\HTMLcode[disc]{LI}{\simpletest{#1}{#2}}} 
\newcommand{\simpletest}[2]{Z 
\HTMLcode{#1}{ a simple test of “#2" ,} using \HTMLcode{CODE}{<#1>} .} 
\newcommand{\tableopts}{10,border=5} 
\newcommand{\tablelist}[4][left]{\HTMLcode[#1]{DIV}{Z 
\HTMLcode[\tableopts]{TABLE}{\HTMLcode[bottom]{CAPTION}{#3}Z 
\HTMLcode{TR}{\HTMLcode{TD}{\HTMLcode{#2}{#4}}}Z 
}}\HTMLcode[all]{BR}} 
\tablelist[\myalign]{\mylist}{Z 
\textbf{A listing of the different text styles available in HTML 3.2}}{Z 
\myitem{B}{bo1d-face}\myitem{I}{italics}\myitem{TT}{teletype-text} 
\myitem{U}{underlining} 
\HTMLcode[circle]{LI}{\simpletest{STRIKE}{strikeout}} 
\myitem{EM}{emphasis style}\myitem{STRONG}{strong style} 
\myitem{CODE}{code style}\myitem{CITE}{citation style} 
\myitem{DFN}{definition style} 
\HTMLcode[square]{LI}{\simpletest{SAMP}{sample style}} 
\HTMLcode[square]{LI}{\simpletest{KBD}{keyboard style}} 
\myitem{VAR}{variab1e style}} 
E3323?-EEI3E:Seao\-a\v..»w~_ 
a simple test of “hold-face", using <B:> . 
.?.t.r.'*:.:;zfe rm! of ‘3r'ai&xIr.'r:¢',' using <I> . 
a simple test of ‘ ‘ teletype-text' ' , using <TT> . 
inple test of “underlini '_',_ using <U> . 
' “ ' ", using <STRIKE> . 
3 .<:r}:=:gzfe .=‘€.<'.=.‘ of ‘?.=..m;aE.m:r.'¢.¢.g7fe ',' using <EH> . 
a simple test of “strong style", using <STRONG> . 
a simple test of ‘ ‘ code style’ ' , using <CODE> . 
3 .<:r}:v:;2ie .=‘€.<'.=.‘ of ‘2:ri?1?.=5ri9.=:*.¢.=!}’f..¢ ',' using <C ITE:> . 
a simple test of “definition style", using <DFN> . 
a simple test of ‘ ‘ sample style‘ ' , using <SAMP> . 
a simple test of ‘ ‘keyboard style’ ' , using<KBD> _ 
.?.<:ri:qz2fe .=‘€.¢.t' of “s«v2r1£?£zf..=..¢.=.‘}’f..=. 'j using <VAR:> . 
O E] E] It II II II II C) II II II I 
A listing of the different text styles available in HTML 3.2 
\reffig{3-18}: Using \HTMLcode commands with ETEXZHTML 
\htmladdtonavigation{\htmladdnormallink 
{\htmladdimg[bottom]{httpz//bonk.ethz.ch/icon-files/find.gif}} 
{http://bonk.ethz.ch/asearch.html}} 
l_\|_e_:£t_l Hp] Previous| 
Next: Geologic Setting Up: Nature and cause of Previous: Nature and cause of 
In this example by Uli Wortmann, the author has provided an easy link to a search 
engine to facilitate finding keywords in Web pages from his document. 

%==========160==========<<<---2
 
%%page page_160                                                  <<<---3
 
3.5 Extending HIFX sources with hypertext commands using the html package 139 
\tab1eofchi1dlinks [option] 
\tableof childl inks* [option] 
Long HTML pages containing subsections, subsubsections, and deeper levels of 
sectioning can be produced. Depending on the degree of linking being used, such 
s usually start with a table of active hyperlinks to the (sub)sections contained 
therein. The above commands let you move this “mini-TOC" (table of contents) 
to where the \tab1eofchi1dlinks command occurs, or they let you omit it altogether. Allowable values for the option argument are 
off omit the mini-TOC on this HTML page. 
none omit the mini-TOC on this and all subsequent pages. 
on restore the mini-TOC for this HTML page. 
all restore the mini-TOC for this and all subsequent pages. 
The *-version omits a break tag (<BR>) that would otherwise be inserted before the 
mini-TOC. 
\html inf 0 [option] 
\html inf 0* [option] 
This command lets you alter the location of the technical “About this document" information. By default, it is positioned at the end on a separate page. Use 
of \htm1info defines an alternative position. The *-version omits the heading, 
thereby allowing the automatically generated information to be included as part of 
the document, with exn‘a material added afterward. Using the option off or none 
omits the information entirely. 
These options allow the technical information to be presented in a more atu'active and meaningful way. For an example of how this is used on a title page see 
\reffig{3-19} (and [c->MATHSYMP]). 
\bodyt ext {nttriliu tes} 
\htm1body [attributes] 
Both of these commands affect the contents of the <BODY> tag that is required on 
each HTML page. Most commonly, these are used to specify foreground and background colors. Colors for hyperlinks can be set this way as well. The command 
\bodytext is used to override the browser’s defaults, and all atu'ibutes to be altered must be specified together inside the attributes argument. ‘This will affect the 
current HTML page as well as all subsequent pages. VV1th \htmlbody, a single atu'ibute can be changed to have the specified value, retaining any other previous 
settings. These commands can be used several times on the same page, each use 
adds to (\htm1body) or overrides (\bodytext) the effects of previous occurrences. 

 
%%page page_161                                                  <<<---3
 
140 
The I*}'I]«3X2HTML translator 
\htm1rule 
\htm1head[center]{subsubsection}{About this document.} 
\htm1addimg[ALIGN=right]{../logos/aaslogo.gif} 
\begin{flushleft} 
This Web site has been constructed by \RossMoore, using the material 
supplied by the authors and the conference organisers. 
Most of this material will become available in book form, published by 
\0zAcadSci, during 1997. 
\end{flushleft} 
:5~ooo\lo\vu-é>V*N-‘ 
\htm1rule[all,width=350]\htmlinfo* 
I. 
\reffig{3-19}: Using the \html info command with ETEXZHTML 
The foreground color for the text and the background color can also be specified using the \color and \pagecolor commands, provided that the ETEX color 
package has been loaded. 
\htmlbase{ URL} 
This command allows the <BASE> tag to be set to a directory different from the one 
where the current document resides. 

 
%%page page_162                                                  <<<---3
 
3.5 Extending HIEX sources with hypertext commands using the html package 141 
3.5.6 Example of linking various extemal documents 
In this section we show how to handle a set of composite documents, taking advantage of the hypertext extensions described in Section 3.5. 
As a starting point, we take the Ié}TEX source document shown in \reffig{3-1} 
and divide it, for demonstration purposes, into four subdocuments. These subdocuments, as shown in \reffig{3-20}, include a “master" file (ex20 . tex) and three 
secondary files (ex21.tex, ex22.tex, and ex2bib.tex). We run all of these files 
through ETEX and then in a determined order through Ié}TEX2HTML. Indeed, as we 
use cross-references to refer to document elements in external documents (with the 
commands \externalref and \externallabels introduced in Section 3.5.3), we 
must first treat the secondary files ex21.tex, ex22 . ‘sex, and ex2bib.tex, before 
tackling the master file ex20 . tex. 
By default, Ié}TEX2HTML writes the files that it creates into a subdirectory 
with the same name as the original file. Therefore, we end up with files in four 
subdirectories, as shown following (directory names are underlined): 
ex20: ex20.css ex20.html labels.pl 
ex21: ex21.css ex21.html labels.pl 
img1.gif Timg1.gif 
images.idx images.tex images.log images.aux images.pl 
ggggz ex22.css ex22.html labels.pl internals.pl 
ex2bibt ex2bib.css ex2bib.html labels.pl internals.pl 
The files labels . pl in the various directories contain information associating the 
symbolic keys of the \label commands in the original ETEX source documents 
with the physical files. For instance, in lines 21-22 of the file ex20 .tex we reference the key “tab-exa" that is defined on line 27 of the file ex22 . tex. In fact, the 
labels .pl file in the ex22 directory contains the following information: 
# LaTeX2HTML 99.1 (March 30, 1999) 
# Associate labels original text with physical files. 
$key = q/tab-exa/; 
$external_labels{$key} = "$URL/" . qlex22.htm1l; 
$noresave{$key} = "$nosave"; 
$key = q/sec-tableau/; 
$external_labels{$key} = "$URL/" . q|ex22.html|; 
$noresave{$key} = "$nosave"; 
1; 
E~Ow\lo\\-n-Bu-nu»Here, indeed, we find, the key (line 3) and a definition of the file where it resides 
(line 4). 

 
%%page page_163                                                  <<<---3
 
*C%\lC7\\-I|:B\.aII\Jr142 
The HIEXZHTML translator 
Master file (ex20 .tex) 
\documentclass{article} 
\usepackage{htm1} 
%begin{1atexon1y} 
\usepackage[T1]{fontenc} 
%end{1atexon1y} 
\usepackage[dvips]{graphicx} 
\usepackage{francais} 
\begin{document} 
\begin{center}\Large 
Exemple d’un document composé\\ K 
\end{center} 
Z ~coc\:<7\Ln-BwN»-\htm1addnorma1link{Les Images}{../ex21/ex21.htm1} U 
\externallabels{../ex21}{../ex21/labels.p1} 15 
Reference a une figure externe~\externalref{Fpsfig}. 15 
\htm1addnorma1link{Les tab1eaux}{../ex22/ex22.htm1} H 
\externallabels{../ex22}{../ex22/labels.p1} 20 
Reference a un tableau externe~\externalref{tab-exa}.n 
L 
\htm1addnorma1link{La bib1iographie}% 25 
{../ex2bib/ex2bib.htm1} 24 
\end{document} M 
R 
N 
File containing the table (ex22 . tex) 
\documentclass{article} 
\usepackage{htm1} 
%begin{1atexon1y} 
\usepackage[T1]{fontenc} 
%end{1atexon1y} 
\usepackage[dvips]{graphicx} 
\usepackage{francais} 
\newcommand{\Lcs}[1]{\texttt{\symbo1{’134}#1}} 
\begin{document} 
\section{Exemp1e d’un tableau} 
\label{sec-tableau} 
Le \hyperref{tab1eau}{tab1eau }{}{tab-exa} montre 
1’uti1isation de 1’environnement \texttt{tab1e}. 
\begin{tab1e} 
\centering 
\begin{tabu1ar}{cccccc} M 
\Lcs{primo} & \primo & \Lcs{secundo} & \secundo & U 
S o m N m m s W N _ 
wt»... 
\Lcs{tertio} & \tertio \\u 
\Lcs{quatro} & \quatro& 1\Lcs{ier} & 1\ier & 
1\Lcs{iere} & 1\iere \\ 
\Lcs{fprimo)}&\fprimo)& \Lcs{No} 10 & \No 10 & 
\Lcs{no} 15 & \no 15 \\ 
\Lcs{og} a \Lcs{fg}&\og a \fg&3\Lcs{ieme}&3\ieme & 
10\Lcs{iemes}& 10\iemes 
\end{tabu1ar} 
\caption{Que1ques commandes de 1’option 
\texttt{french} de \texttt{babe1}}\label{tab-exa} 
\end{tab1e} 
\end{document} 
File containing images (ex21 . tex) 
\documentclass{article} 
\usepackage{htm1} 
%begin{1atexon1y} 
\usepackage[T1]{fontenc} 
Zend{1atexon1y} 
\usepackage[dvips]{graphicx} 
\usepackage{francais} 
\makeindex 
\begin{document} 
\section{Une figure EPS} 
\label{sec-figure} 
Cette section montre comment inclure une figure 
\externallabels{../ex2bib}{../ex2bib/labels.p1}% 
PostScript\externalref{bibPS} dans un document \LaTeX. 
La \hyperref{figure}{figure }{}{FpSfig} 
est insérée dans le texte a 1’aide de la commande 
\verb!\includegraphics{colorcir}!. 
\begin{figure} 
\htm1image{thumbnai1=O.4} 
\centering 
\begin{tabu1ar}{c@{\qquad}c} 
\includegraphics[width=3cm]{colorcir} & 
\includegraphics[width=3cm]{tac2dim} 
\end{tabu1ar} 
\caption{Deux images EPS}\label{Fpsfig} 
\end{figure} 
\end{document} 
File with the bibliography (ex2b ib . tex) 
\documentclass{article} 
\usepackage{htm1} 
%begin{1atexon1y} 
\usepackage[T1]{fontenc} 
Zend{1atexon1y} 
\usepackage[dvips]{graphicx} 
\usepackage{francais} 
\makeindex 
\begin{document} 
\begin{thebib1iography}{99} 
\bibitem{bib-PS}\label{§ibPS} 
Adobe Inc. 
\emph{PostScript, manuel de reference (2iéme édition)} 
InterEditions (France), 1992 
\end{thebib1iography} 
\end{document} 
\reffig{3-20}: Linking external files (BTEX files) 

 
%%page page_164                                                  <<<---3
 
3.5 Extending Ié'I]3X sources with hypertext commands using the html package 143 
The other files in the directories are HTML and CSS files, in addition to GIF 
images generated for material that Ié}TEX2HTML cannot gracefully translate into 
HTML. Here it is the case only for the image in ex21 . tex, where we generate the 
complete image of the table with the two EPS files (lines 21-24 in the ex21.tex, 
\reffig{3-20}). Additionally, on line 19 we indicate that we do not want to include 
the full image inside the HTML file, but rather a 40% reduced thumbnail image. 
Hence, in the directory ex21 we observe two GIF images: img1.gif, the complete 
image, and Timgl . gif, the corresponding thumbnail. 
Additionally, lines 3-5 of all BTEX source files in \reffig{3-20} show that we ignore the command \usepackage [T1] {f ontenc} during the Ié}TEX2HTML translation. This is because the font encoding is important for BTEX only, and all Latin 1 
characters, like the French diacritics present on several lines in the source documents, are handled correctly without further intervention by BTEX2 HTML. In fact, 
the more traditional macro-based representation of accented characters in ETEX 
sources (used in \reffig{3-1}) and the direct use of the Latin 1 character encoding 
yield the same result after translation into HTML. 
To guide ETEX2HT ML in translating these documents we also use a customization file, myinit . pl, containing customizations of Perl constants. 
# File myinit.pl 
$ADDRESS = "<em>Michel Goossens<BR>" . 
"Division IT<BR>" . 
"T&eacute;l. 767.5028<BR>" 
"$address_data[1]</em>"; 
$MAX_SPLIT_DEPTH = 0; # do not split document 
$N0_NAVIGATION = 1; # no navigation panel 
1; # Mandatory last line 
%\lO\\-II-B<oJIu>Lines 2--5 define the address information displayed at the end of each HTML page, 
line 6 declares that we want to keep the complete document in a single piece, and 
line 7 eliminates the navigation panels. 
Using this customization file we issue the following command to execute 
Ié}TEX2HTML on the files in \reffig{3-20}: 
latex2html -init_file myinit.pl -info ``External links test'' filenames 
The -init_file flag loads the customization file myinit .p1, while the -inf o flag 
changes the “About this document. . . (/l propos de ce document. . . )" information. Finally, f 71 lenames stands for the list of the four ETEX files-ex2bib . tex, ex21 .tex, 
ex22 . tex, and ex20 . tex-that are treated in the correct order. The result of running this command is shown in \reffig{3-21}. 

 
%%page page_165                                                  <<<---3
 
The HIFXZHTML translator 
144 
.$m3o.5 w 5:» 
@9503 ma 8N.m 2&8 muaufisoownam BM wan ucofiioom ofimomfioo “Eu 50¢ wofifino ubdonbm ea 1:>i.E och. JN.m oudmfl 
.0 325% ecu ~A@ wB8%E 98 ME: oocouumvu ofimmamomfin 05 mo Ohm» . nfinmxw mo 3 “E5 
ESQ vac ofi wax 08». Hwxw mo 3 2:3 “Eon fifim Ev xzafim .o-@ BU Eeuuxu Bwuumum a ma oEm=m>m owns: uamméwou 06 5:» and ecu E 
mannfidfi Q5 Boocsoo Hana v_:=.8m~E u ou mwnommuboo @ @3385: Saba v£r-.. .Q_u>_uoommou J8» . mmxm mo mm on: wax xop. «mam we mm 
on: no weamuw flunfl mfionofimuu J8» . omxw mo #N can 3 8:5 flaofinoow uomufi us» 5 mvcwfifioo Hwnwax mo unufismum 05 mm wummoomm 
woman omonnbw mo om: BEE >23. .Hw.H.__.m.n-.H@u.N®/ mwnwmafioo ufi firs wouusbmnoo P8 35 mooco._om8-mmouo PE @ ES 9 3828:: 
mBoEw HEP .08» . omxw we mm cum .5 .2 8:5 uouaom Nmbfl ofi E w:wEEoo M.fiHHHmE.HO._H.UU.m.__B.....Q/ 2% man: ucufisoov AEHE an 9 
wcufiom $33.5 8 waommuboo 9 «Eu .@ .6 £355: ufi mcrhiau wsoba BEL fiufifioumso Eon om? was noumfiuowfi ...§%S§§» 
8 mfi ii ash. .Eom mmouwwm Bu: u 0>wn25 van .m_u..=& coummvén on Ba P55 .R.pE1a oi aoumwuboumno ufi E woumusvuu m< 

 
%%page page_166                                                  <<<---3
 
3.5 Extending ILVIFX sources with hypertext commands using the html package 145 
3.5.7 Advanced features 
3.5.7.1 Using a Makefile 
With UNIX it is often useful to use a Makefile to simplify the commands that 
run a particular procedure, particularly when many command-line options or long 
filenames are required. The following Makefile provides a simple method for generating the example shown in \reffig{3-21}, just by typing make ex2. 
# Makefile for Example 2 
LTX = latex 
L2H2 = 1atex2htm1 -init_file myinit.p1 -info ``Externa1_links_test'' 
ex2 : ex21.aux ex22.aux 
$(L2H2) ex2bib ex21 ex22 ex20 
ex21.aux : 
$(LTX) ex21.tex 
ex22.aux : 
$(LTX) ex22.tex 
E5~ow~no\w-A-.-.Line 6 contains the call to ETEX2 HTML; the preceding line defines a “dependency" 
to ensure that segment sources have been processed by ETEX. The figure and table 
caption numbers are available in the .aux files. Each pair of lines (7-8 and 9-10) 
handles a call to BTEX for one segment. 
The code shown is a naive use of a Makefile, yet it is sufficient to indicate that 
much typing can be avoided. \reffig{3-22} is a Makef ile for use with another version 
of this example, and presented as a “segmented" document. In the next subsection 
we describe from the html package the macros that are designed specifically for this 
technique. The example itself is discussed in detail in Section 3.5.7.3. 
3.5.7.2 Using “segmented" documents 
The commands described in this section are for use mainly with the “Document 
Segmentation" strategy, although some can also be useful in other situations. Segmentation involves splitting the ETEX source into segments. Each segment is contained in a different file and is processed separately by I5TEX2HTML. 
Earlier versions of ETEX2 HTML could be excessively hungry for memory, particularly during the image-generation phase. Recent improvements have alleviated 
this need, but segmentation remains a useful strategy for keeping a Web document 
up-to-date. Segments can be updated individually or in groups without the need 
to reprocess the whole document completely. The ETEXZHTML Users Guide is 
maintained using this strategy; consult it for further details. 
\segment [align] {file}{.vectiom'ng}{tz'tle} 
Suppose that the value of xectioning were section; then BTEX would process the 
command as \section{tz'tle}, followed by \input file. First it dumps counter information into a file named file.ptr in the form of ETEX commands for setting 
counter values. This file is included as part of the job when ETEX2 HTML processes 

 
%%page page_167                                                  <<<---3
 
146 The I£‘l]§X2HTML translator 
# Makefile for Example 3 
1 
2 L2H = $(L2HNEW)/latex2html -link 4 
3 L2HTOP = $(L2HNEW)/latex2html -split 0 
4 LTX = latex 
5 EX = ex3 
6 HTML = ;html 
7 TX = .tex 
8 TEXES = $(EX)*$(TX) 
9 TOP = ex30 
10 BIB = $(EX)bib 
11 TEXTOP = $(TOP)$(TX) 
12 INT = internals.pl 
13 BIBREF = ’\#BIBLIO’ 
14 # Directories containing the segments 
15 EX30 = ../$(TOP)/$(TOP) 
16 EX31 = ../$(EX)1/$(EX)1 
17 EX32 = ../$(EX)2/$(EX)2 
1s EX3b = ../$(TOP)/$(TOP)$(HTML) 
19 # Titles for navigation to the segments 
20 EX30t = Exemp1e_d\’un_document_segmenté 
21 EX31t = Une_figure_EPS 
22 EX32t = Exemple_d\’un_tableau 
23 Ex3bt = Références 
25 COMMON = -info 0 -split 1 -link 4 -no_auto_link -biblio $(EX3b) -external_file $(TOP) \ 
26 -up_url $(EX30)$(HTML) -up_title $(EX30t) -index $(EX3b) -index_in_navigation 
27 update: $(TOP).ind 
22 make $(TOP)/$(INT) $(Ex)2/$(INT) $(Ex)1/$(INT); make $(TOP)/$(INT) 
29 fresh: 
rm $(EX)#/$(INT) $(TOP).aux; make $(TOP).ind; 
make $(TOP)$(HTML) $(EX)2$(HTML) $(EX)1$(HTML); make $(TOP)$(HTML) 
$(TOP)$(HTML) : 
$(L2HTOP) -down_url $(EX31)$(HTML) -down_title $(EX31t) -biblio $(BIBn£F) $(TOP) 
$(TOP)/$(INT) 2 $(TExTOP) $(BIB)$(TX) $(EX)1/$(INT) $(EX)2/$(INT) 
make $(TOP)$(HTML) 
$(EX)1$(HTML) : 
$(L2H) $(COMMON) ~t $(EX31t) -prev_url $(EX30)$(HTML) -prev_title $(EX30t) \ 
-down_url $(EX32)$(HTML) -down_title $(EX32t) $(Ex)1 
$(EX)1/$(INT) : $(EX>1$(Tx) 
make $(EX)1$(HTML) 
$(EX)2$(HTML) : 
$(L2H) $(COMMON) -t $(EX32t) -prev_url $(EX31)$(HTML) -prev_title $(EX31t) \ 
-down_url $(EX3b)$(BIBREF) -down_title $(EX3bt) $(EX)2 
$(EX)2/$(INT) : $(EX)2$(TX) 
make $(EX)2$(HTML) 
# handle LaTeX/dvips/makeindex etc. 
$(TOP).aux: $(TEXES) 
make dvi 
$(TOP).dvi: 
make dvi 
dvi: 
$(LTX) $(TExToP); $(LTX) $(TEXTOP); make $(TOP).ind 
$(TOP).ind: $(TDP).aux 
makeindex $(TOP).idx; $(LTX) $(TEXTOP); touch $(TOP).ind 
ps: 
make $(TOP).ps 
$(TOP).ps: $(TOP).dvi 
dvips $(TOP).dvi -0 
clean: 
1.. u.u.u.u.4.4>.>.>-A -A-A-A wwwwwwwwww 
9%‘€3£‘Go\3‘.£w~.-o~oeaxnoxmi».-N_$~oec\noxu.+w~-o 
rm #.ps #.dvi *.log #.ilg 
\reffig{3-22}: Example Makefile for documents in \reffig{3-23} 

 
%%page page_168                                                  <<<---3
 
3.5 Extending Ié'I]§X sources with hypertext commands using the html package 147 
the segment. The xectioning can be any of the recognized forms of the sectioning 
commands: part, chapter, section, subsection, paragraph, and so on. 
The segment file should not contain its own title, unless it is shielded inside 
a \begin{htmlonly}. . .\end{htmlonly} environment because it is already provided by the title argument. For IMEXZHTML, this title is stored using \htmlhead 
(see the following) in the .ptr file. This can be overridden using \htmlnohead, if 
desired. 
\htm1head [align] {.vectioning}{title} 
\htmlnohead 
The \htm1head command is normally read from a .ptr file (see \segment command earlier). In this form it becomes the first information to appear on the HTML 
, following the navigation aids. However, it may be desirable to suppress this 
and place other information, such as an image, earlier on the page. In this case the 
command \htmlnohead should be used, followed by \htmlhead at an appropriate 
place in the source. Indeed, an alternative title could be used, allowing different titles for the same HTML page when it is viewed as a stand-alone document or as a 
section within a larger document. 
Using the optional align, the title heading can be aligned to the center, the 
right, or the left (the default) at the top of the HTML page. Of course, BTEX 
ignores both \htmlnohead and \htmlhead and its arguments. 
\startdocument 
For a document segment, this command acts as an artificial marker indicating where 
the body of the segment is presumed to begin. Material occurring before this is 
treated as if it were in the preamble of a complete I5'1']§X document and will not 
appear on the HTML pages. Typically such material could be \newcommand definitions contained within \begin{htm1only}. . .\end{htmlonly}. 
\internal [type] {prefix} 
This command causes ETEXZHTML to load a Perl file containing information generated within a different segment of this (or another) document. For example, 
\interna1 [contents] {. ./segA/C} 
This sequence causes the file ../segA/Ccontents.pl to be read, resolving 
the path from the directory where the document is being processed. This 
file would contain information shown in the table of contents generated by 
\tableofcontents, if it were to appear within the segment being processed. Allowable values for type are section, contents, figure, table, index, images, and 
internal. 

 
%%page page_169                                                  <<<---3
 
148 
The ETEXZHTML translator 
\segmentcolor [model] {color} 
\segmentpage color [model] {color} 
These commands should occur only inside .ptr files. They pass color information, 
defined with I§TEX’s \color and \pagecolor commands, from one segment to the 
next or from the parent document to its segments. 
\endsegment [xettioning] 
This command is sometimes needed to ensure correct numbering of sections 
within a document that loads other segments. In ETEX it does nothing, but with 
IéTEX2HTML it corrects problems with the navigation panels and table of contents 
for a segmented document that may otherwise become confused. It should follow 
the \segment command to ensure that hyperlinks to subsequent pages do not get 
mixed in with that segment or with earlier segments. The xectioning should match 
the preceding segment. 
3.5.7.3 Segmentation example, with Makefile 
\reffig{3-23} shows the HTML pages that can be produced by organizing the ETEX 
document sources to make use of the segmentation strategy. Comparing \reffig{3-23} with \reffig{3-2}, we see that to enhance the usefulness of particular pages 
it is possible to organize the information so that it appears in a different order. In 
\reffig{3-23} the main HTML page is longer than the depth of the screen, so it has 
been split into two images. They are shown superimposed vertically at the left of 
the figure. Of course, when you use a browser, the vertical scroll bar allows all the 
information to be accessed. 
The source listings for the segments are given in \reffig{3-24}. These are similar 
to those in \reffig{3-20} for processing pieces of ETEX separately for HTML pages. 
Generally they are a little shorter because it is not necessary to process complete 
ETEX documents with ETEXZHTML. Look at the listing for the file ex30.tex. 
It contains \segment commands rather than \input or \include for each section. Thus when processed by ETEX, the files ex31.tex and ex32.tex are each 
input, starting a new section. The resulting .dvi file should give a result exactly 
as shown in \reffig{3-2}, except for some entries that are in the index. Prolific use 
of \htmlrule commands helps to segregate chunks of related information on the 
HTML page. 
Each of the files ex31.tex and ex32.tex has a \startdocument command 
in place of \begin{document} to indicate to l§TEX2HTML where the body begins. This is preceded by any required preamble material, mainly inside htmlonly 
conditional environments to avoid repetition when ETEX processes the document. 
There are \documentclass and \usepackage commands with ex31.tex because 
these will be needed when generating an image. With ex32.tex there is no need 

%==========170==========<<<---2
 
%%page page_170                                                  <<<---3
 
3.5 Extending Ié‘Ij§X sources with hypertext commands using the html package 149 
Premusl 0 , 
_ ' Mm up rmus[ max 
Example d’un document segmenté ‘ 
Ran Mom I Mldld Gunny: * Unlufl BPS 
``- u ''TEX.Ln 
enimi2f£dnukc1mnl1'nlded:htAmn-id: 
\inc1udograyh£z's!cI:1o .oplL 
O 
. fiL“%. V mu 
: Linnfiuuis I """E',',';'" 
. %‘.h“"""“*‘ ® 
 2:: 2 . 
O 3 
R»: Moon 
(991-D5v26 
R&£2uu:eifluxeL 
Mtéruaeeltnhlaam 
Mftrawu 
1 AdoheInc.PosrSa:jvt,nu.u¢Id¢rq;Ilmnw{2iimid:1’£an)!ntuE ( .. gm 
Bfllkum Q} Previous‘ mm} 
am H“ M“ as ‘ - Up: fluvhl€ mu 
En: fllnvihlun 
Le < wl'uxiInItundu'¢mh-onneuwatt 0. 
Eznmhimnhlm 
1uu_1:Qu.muaumwamue1-npmnsrmahaa 
bnbol 
5 
E‘ ‘wrim 1° ‘xncazndo 2° ‘work 5:: 3° 
A propos de ce document... : xqu-no u ikior 1* mm 1» 
, \!'prl.l0? \KG 10 N'1O \nDI5 '7 5 
lxnplnfmbanomlnnurfl 1" I I 
‘&ogI\fg «I» Minus 5' 1D\ienme: mil 
9'I’§x1Il’a9i.)u11uatwu |mnIMumm vwXmVmm9u2 hum! (Sqnnnh 
E;1::‘rlfl\tD1993, 1994,1995, 1995, mmmm comm Baud Leann; um. nu»-aw of mum 
comma 19w,1m.3.u_mm. Mmmmnapmmu. Mncqunfln unmmy, Bydnq. '"‘~9’~?‘ 
Th: ltnumnmm m. ‘ 
lmamml -nplit D -do\m_url . ./0:31/ox:-I1 .htn1 -do\m_t1t1o 
Una_1'ig'uro__3PB -b1b11c IHIBLIO 0130 
Thetrmshflmwuinfidmlbynan Muot~ean1993-09-z6 
mast Alum 
1993-09-Z5 
0 and 9 link the main document to the segments that are generated at the section level. 6) 
and ® are cross-reference links from the lists of figures and tables to the actual figures and 
tables. 9 links the thumbnail to the real figure. 
\reffig{3-23}: Segmented HTML structure generated by ETEXZHTML from the 
BTEX source of \reffig{3-24} 

 
%%page page_171                                                  <<<---3
 
c~u-ax..a.\..._ 
150 
The ETEXZHTML translator 
Master file (ex3O .tex) 
\documentclass{article} 
\usepackage[dvips]{graphicx} 
\usepackage{francais} 
\usepackage{htm1} 
\usepackage{makeidx} 
Zbegin{1atexon1y} 
\usepackage[T1]{fontenc} 
Zend{latexon1y} 
\internal{../ex31/} 
\interna1[sections]{../ex31/} 
X\interna1[contents]{../ex31/} 
\internal[figure]{../ex31/} 
\interna1[index]{../ex31/} 
\internal{../ex32/} 
\interna1[sections]{../ex32/} 
Z\internal[contents]{../ex32/} 
\internal[tab1e]{../ex32/} 
\internal[index]{../ex32/} 
\makeindex 
\title{Exemp1e d’un document segmenté} 
\author{Ross Moore \& Michel Goossens} 
\begin{document} 
\maketitle\htm1ru1e 
\tab1eofchi1dlinks\htm1ru1e 
Zbegin{1atexon1y} 
\tab1eofcontents 
%end{1atexon1y} 
\listoffigures\1istoftab1es\htm1ru1e 
\segment{ex31}{section}{Une figure BPS} 
\endsegment[section] 
\htm1{Référence A figure~\ref{Fpsfig}.} 
\segment{ex32}{section}{Exemp1e d’un tableau} 
\endsegment[section] 
\htm1{Référence A tab1eau~\ref{tab-exa}.} 
\htm1ru1e \input{ex3bib} 
\htm1ru1e \printindex \htm1ru1e 
\end{document} 
File with the bibliography (ex3bib . tex) 
\begin{thebib1iography}{99} 
\label{BIBLIO}\index{références} 
\bibitem{bib-PS} 
Adobe Inc. \emph{PostScript, manuel de reference 
(2ieme édition)} Interfiditions (France), 1992 
\end{thebib1iography} 
~oOc\Jo\\-II4>\o.aI\:>-File containing images (ex31 . tex) 
\begin{htm1on1y} 
\documentclass{article} 
\usepackage{makeidx,htm1} 
\usepackage[T1]{fontenc} 
\usepackage[dvips]{graphicx} 
\usepackage{francais} 
\input ex31.ptr 
\end{htm1on1y} 
\startdocument 
\index{section!figure} 
Cette section montre comment inclure une figure 
PostScript\cite{bib-PS} dans un document \LaTeX. 
La \hyperref{figure}{figure }{}{Fpsfig} 
est insérée dans le texte é 1’aide de la commande 
\verb!\includegraphics{colorcir}!. 
\index{figure} 
\index{PostScript} 
\begin{figure}[h] 
\htm1image{thumbnail=0.4} 
\centering 
\begin{tabu1ar}{c@{\qguad}c} 
\includegraphics[width=3cm]{colorcir} & 
\includegraphics[width=3cm]{tac2dim} 
\end{tabu1ar} 
\caption{Deux images EPS}\label{Fpsfig} 
\end{figure} 
File containing the table (ex32 . tex) 
\begin{htm1on1y} 
\usepackage{makeidx,htm1} 
\usepackage{francais} 
\input ex32.ptr 
\end{htm1on1y} 
\newcommand{\Lcs}[1]{\texttt{\symbo1{’134}#1}} 
\startdocument 
\index{section!tab1eau} 
Le \hyperref{tab1eau}{tab1eau }{}{tab-exa} 
montre 1’uti1isation de 1’environnement \texttt{tab1e}. 
\begin{tab1e}[h] 
\centering 
\begin{tabu1ar}{cccccc} 
\Lcs{primo} & \primo & \Lcs{secundo} & \secundo 
& \Lcs{tertio} & \tertio \\ 
\Lcs{quatro} & \quatro& 1\Lcs{ier} & 1\ier 
& 1\Lcs{iere} & 1\iere \\ 
\Lcs{fprimo)}&\fprimo)& \Lcs{No} 10 & \No 10 
& \Lcs{no} 15 & \no 15 \\ 
\Lcs{og} a \Lcs{fg}&\og a \fg&3\Lcs{ieme}&3\ieme 
& 10\Lcs{iemes}& 10\iemes 
\end{tabu1ar} 
\caption{Que1ques commandes de l’option \texttt{french} 
de \texttt{babe1}}\label{tab-exa} 
\index{tab1eau} 
\end{tab1e} 
\reffig{3-24}: Segmentation example (BTEX files) 

 
%%page page_172                                                  <<<---3
 
3.5 Extending I1}’I];X sources with hypertext commands using the html package 151 
for a \documentclass, but the makeidx, francais, and html packages are still 
needed. Both segments need to \input the corresponding ptr files so that section 
counters are set correctly. 
Notice how the \externallabels and \externalref commands that were 
used in \reffig{3-20} are not needed. Instead the navigation is handled using I5TEX’s 
usual \ref commands. This is possible due to the \internal commands within 
ex30.tex, which read the symbolic label information from the segments. Similarly one segment could resolve cross-references within another segment by including the appropriate \int ernal command. Information for the lists of Figures and 
Tables is read from the commands \internal [figure] and \internal [table], 
while information about the sections within each segment is obtained using the 
\internal [sect ion] commands. Similar information, perhaps with slight differences, for a table of contents page could be obtained using \internal [contents] . 
This is not needed in this example, since the automatically generated “mini" table 
of contents has all that is needed. It also explains why the \tab1eof cont ents command is being used only by BTEX, with the position of the mini table of contents 
being determined by \tab1eofchi1dlinks. 
Information for the index is collected through the \internal [index] commands. When the makeidx package is loaded, features such as hierarchical index listings become available. A simplified index is available when makeidx is not 
loaded. It is best to load makeidx in all segments; otherwise some index entries 
will show filenames of type html rather than section titles. For the use of \cite 
commands in ex31.tex, the target of the hyperlink is deduced by knowing on 
which html file the bibliography will appear. Such information is provided by the 
-bib1io command-line option, when that segment is processed. Navigation between the segments is provided via the buttons at the top of each HTML page, just 
as in a single, unsegmented document, independent of any cross-references and 
sectioning commands within the ETEX source. These hyperlinks are specified using command-line options. As a result the command line gets rather complex, and 
therefore it becomes essential to use a Makef ile. 
Look at the structure of the Makefile listing in \reffig{3-22}. Usually it defines 
variables to hold strings that are to be used often or that may reasonably be expected 
to change as a document is being developed. (This is similar to the use of macros in 
BTEX, only more so.) Lines 2-26 contain such definitions; their values are accessed 
via the $( . .) dereferencing notation. 
As an example, in lines 25-26 the variable COMMOM is built to contain commandline options, used when processing the segments. These command-line options will 
be used with both segments, some of which are constructed by dereferencing other 
variables. The actual command lines themselves are constructed in lines 36-38 and 
41-43. In those line groups, the command-line options specify the document title. 
The hyperlinks for the next/previous navigation buttons are also specified, as well 
as the actual file to be processed. Following is the resulting command line (lines 
1-6), and the start of the log messages for one of the segments (lines 8-27). The 
command line, which is very long, has been rearranged to make it readable. 

 
%%page page_173                                                  <<<---3
 
152 
The HIFXZHTML translator 
latex2html -link 4 -info 0 -split 1 -link 4 -no_auto_link -biblio ../ex30/ex30.html \ 
-external_file ex30 -up_url ../ex30/ex30.html -up_title Exemple_d’un_document_segmenté \ 
-index ../ex30/ex30.html -index_in_navigation \ 
-t Une_figure_EPS -prev_url ../ex30/ex30.html \ 
-prev_title Exemp1e_d’un_document_segmenté -down_url ../ex32/ex32.html \ 
-down_title Exemple_d’un_tableau ex31 
This is LaTeX2HTML Version 99.1 release (March 30, 1999) 
Drakos, CBLU, University of Leeds. 
:5~Om\lc\vI<b‘-NRevised and extended by: 
Marcus Hennecke, Ross Moore, Herb Swan and others 
...producing markup for HTML version 3.2 
Loading /usr/local/share/latex2html/versions/html3.2.pl 
*** processing declarations *** 
OPENING /services/www/texdev/LWC/MICHEL/ex31.tex 
Cannot create directory ex31/: File exists, reusing it. 
Reusing directory ex31/: 
Loading /usr/local/share/latex2html/styles/texdefs.perl... 
Loading /usr/local/share/1atex2htm1/styles/article.per1 
Loading /usr/local/share/latex2html/styles/makeidx.perl 
Loading /usr/local/share/latex2html/styles/html.perl 
~NN--N_.._._._.._._._.._. 
\.o\v..px...~_.o~om\uo\v-«n«...~ 
This Makefile makes it straighforward to maintain the example document. 
It suffices to type make ex31.html on the command line to execute the commands specified on lines 35-37. Alternatively, typing make ex31/ internals .pl 
would cause the command to be executed only if something has changed within 
the ETEX source file ex31.tex. Line 38 tests this “dependency," calling upon 
make ex31.html in line 39, only if necessary;8 otherwise there is no need to reconstruct ex31/ex31.html. Lines 43-44 handle the similar conditional update of 
ex32/ex32 .html. Updating the main page ex30/ex30.html is a little more complicated since it depends on changes in more than one ETEX source file and on 
having up-to-date information in the internals .pl files from the other segments. 
This dependency is stated in line 33. Unlike I{}TEX’s aux file, which is written after each call to ETEX on a source file, the internals .pl file is written only if the 
information it should contain has changed. Hence a segment can be updated for 
small changes without changing the date on its internals .pl. Such a date change 
would also cause the main segment to need updating. 
The ability to update conditionally just those HTML pages that need it is captured in lines 26-27, whereby all the dependencies are tested as a result of the command make update. This includes running ETEX to ensure that the dvi version is 
up-to-date, and the aux file has correct information. 
If something goes wrong or if a change that does not cause appropriate updates 
to occur is made, then reprocess everything using the rule on lines 28-30 by typing 
8It would be necessary only if a file named on the right-hand side of the “:" has a modification 
date/time later than the file named on the left-hand side. 

 
%%page page_174                                                  <<<---3
 
3.5 Extending HIEX sources with hypertext commands using the html package 15 3 
make refresh. Typically this is necessary after altering the definition of a macro 
expansion within a style file. 
Of course, a dependency on this style file could be added at an appropriate place 
within the Makefile. Lines 47-58 give rules for running BTEX via make dvi and 
dvips via make ps. The former does multiple runs of BTEX, then of makeindex, 
and finally of BTEX again. Unneeded log files are removed by make clean (lines 
59-60). 
Be aware that the make program can vary in minor details, on different UNIX 
platforms, and even with a different shell on the same platform. Note the need 
to “escape" special characters such as # and ’, using \# and \ ’ . On the particular 
platform and shell used by the author, it was necessary to avoid having spaces in 
strings that appear on the command line; hence the use of _ in the segment titles. 
3 . 5.7.4 Tracing and ‘debugging 
The commands in this section are intended primarily to aid BTEXZHTML developers and to extend the program with specialized user modules. 
\htmltracing{level} 
\htmltracenv{level} 
Both of these commands set the level of tracing to the specified level, which is an 
integer from 0 (no tracing) to 9; the default level is 1. The difference between 
\htmltracing and \htmltracenv depends on when the command is processed. 
The command \htmltracenv is processed in order with environments, whereas 
\htmltracing is treated as a command with a fixed static expansion. Therefore, 
processing can be delayed to follow environments at the same level of TEX grouping. Higher values of level produce more messages that concern more technical 
aspects of the translation process. Consult the BTEXZHTML Users Guide for the 
type of information displayed with each level. 
\html set {varial7le}{value} 
\html set [bar/J] {lee}/}{value} 
\html set env{varial7le}{value} 
\htmlsetenv [bay/J] {variable or key}{value} 
These \htm1 set and \htm1setenv commands allow Perl variables to be set or 
changed directly during processing by ETEXZHTML. The difference between the 
two forms has to do with when the command is actually processed. The command 
\htmlsetenv is processed in order with environments, whereas \html set is treated 
as a command with a static expansion. Therefore, processing can be delayed until 
after environments at the same level of TEX grouping. When the optional argument 
is used, the value is assigned to theleey for the hash specified by has/J. 

 
%%page page_175                                                  <<<---3
 
154 
The ETEXZHTML translator 
Summary 
This chapter has shown how BTEXZHTML can be used to translate ETEX source 
documents into a hyperlinked web of HTML files with source fragments that cannot 
be directly expressed in equivalent HTML constructs, such as mathematics, transformed into images. The default settings for the translation are often sufficient to 
obtain excellent results without user intervention. The user can, nevertheless, intervene directly in the translation process by setting command-line options or by 
editing one or more files specifically provided to facilitate user customization. 
We have studied in detail how to fine-tune mathematics rendering and have 
looked at the facilities that are available to handle various languages. We have given 
examples of an implementation to support languages used on the Indian subcontinent, but the techniques shown are more general. 
The final part has taught us how to profit from the rich set of extensions of the 
html package. This not only puts the full powers of HTML at our fingertips, but 
also allows us to build a real web of hyperlinked external documents. 
All in all, BTEXZHTML is a very flexible and complete tool. It can be used 
by a novice as a l7lack-l70x automatic translator to make BTEX source documents 
available on the Web in a simple and transparent way. However, in the hands of the 
devoted, BTEXZHTML becomes a full-fledged application that lets you create, edit, 
and manage your hypertext documents. BTEXZHTML thus provides a convenient 
bridge between the world of BTEX and the Internet. 

 
%%page page_176                                                  <<<---3
 
CHAPTER 4 
Translating ETEX to 
HTML using TEX4ht 
TEX4ht is a system that offers a way to create configurable hypertext versions of 
ETEX documents; it is an extension of TEX, in general, and of ETEX, in particular. 
In this chapter we concentrate on its default setup that produces HTML output. 
The system consists of two parts: style files to supplement I§TEX’s existing ones 
with HTML features and a postprocessor for extracting HTML files from the output of the TEX program. TEX4ht leaves the major task of processing the source 
documents into DVI code to TEX itself. Consequently, TEX4ht has access to the 
full power of TEX for handling fonts, macros, variables, category codes, and other 
important features that are quite commonly hidden from the users but are essential 
for style files. Most applications need to worry about only a few major features of 
the system. In the case of HTML output from ETEX sources, the user may hardly 
notice any intrusion from the system. 
Despite being a flexible authoring system rich in capabilities, TEX4ht keeps its 
features transparent for most applications. Typically, compiling and viewing documents that employ TEX4ht is as simple as doing the same without using the system. The first three sections of this chapter are geared toward such applicationsexplaining how the system can be employed by casual users who do not need special 
configurations. We list the most commonly used package options, show a complete 
example with its input and output, and briefly describe how the process works. 
The fourth section describes the new commands that are provided in the package for authors who wish to add explicit links, achieve HTML effects manually, 

 
%%page page_177                                                  <<<---3
 
15 6 Translating ETEX to HTML using TEX4ht 
or add information to the CSS style sheet. The fifth section explains how the system can be extensively configured to produce different effects from the same IATEX 
markup. 
The final part of the chapter goes into more detail of how the various parts 
of the TEX4ht process work and how they can be changed. This will be especially 
useful for system adminstrators installing TEX4ht for use by a group of authors. 
TEX4ht is available from either TEX archives or its own Web page 
[h>TEX4HT], where there are also many examples. Currently, TEX4ht is still a work 
in progress. A few of the features described here might change in time, but the 
changes are expected to be minor, as most features are completely stable. 
4. 1 Using TEX4ht 
To have a source ETEX file translated into HTML, many users will find it sufficient 
just to load the TEX4ht package tex4ht . sty (\reffig{4-1}). 
The translation is activated with a command of the form ht latex filename, 
or a similar system-dependent command. The output is stored in a set of files with 
the root file named f 7.‘ Lename . html. 
4. 1 .1 Package options 
The outcome of the translation of the ETEX source files into HTML can be varied 
using package options (for example, see \reffig{4-2}). 
There are a number of package options available. Some are of general interest; 
most are not at all essential. Some options are provided as shortcut mechanisms to 
offer convenient ways of activating useful features without needing to learn their 
details. 
<!DOCTYPE. HTML PUBLIC "-//w3c//DTD 
HTML 4.0 Transitional//EN"> 
<HTML> 
<HEAD> 
<!--try.html from try.tex 
(Tex4ht, 1998-05-27 00:01:00)-~> 
<TITLE>try.html</TITLE> 
<LINK REL=``stylesheet'' \documentclass{article} 
TYPE="text/css" HREF="try.css"> \usepackage{tex4ht} 
</HEAD><BODY> \begin{document} 
hello world hello world 
</BODY> </HTML> \end{document} 
\reffig{4-1}: A simple source file and the resulting HTML code 

 
%%page page_178                                                  <<<---3
 
4.1 Using TEX4ht 
A Perfect Day 
1 Introduction 
\documentclass{article} 
\usepackage 
\title{A Perfect Day} 
\author{L. Reed} 
RBEWHWS \begin{document} 
\maketitle 
\section{Introduction} 
See \cite{x}. 
\begin{thebib1iography}{} 
\bibitem{x} Nothing yet. 
\end{thebibliography} 
\end{document} 
\reffig{4-2}: A source document and a display of its HTML files 
For instance, the options 1, 2, 3, and 4 are simply shortcuts for a single 
\tableof contents and a few \CutAt and \TocAt* commands. Users looking for 
a more refined outcome than that offered by these options will need to invest a 
little effort into studying these commands (see \refsec{4_5_2_Tables_of_contents} on page 172 and \refsec{4_5_3_Parts_chapters_sections_and_so_on} on page 175) and using them directly. 
The first package option must be the name of a configuration file (see \refsec{4_5_1_Configuration_files} on page 170) or the option html. Otherwise, the first option is ignored. 
The order in which the nonleading options are listed is not significant. 
As its name suggests, the html option asks for HTML output. Without that 
option, the output should be a standard DVI file. The following are some of the 
other options that are available: 
1, 2, 3, or 4 These options request a breakup of the documents into hierarchies of 
hypertext pages. The hierarchies reflect the logical organization of the content, 
as specified by the sectioning commands \part, \chapter, \section, and so 
on. The Values of the option determine the desired depth of the hierarchy of 
hypertext pages. 
[html,2,sections+]{tex4ht} 

 
%%page page_179                                                  <<<---3
 
15 8 Translating I5"l]3X to HTML using TEX4ht 
sections+ The entries in tables of contents provide hypertext links to the sections to which they refer. This option asks for extra backward hypertext links 
from the titles of the sections to the tables of contents. 
next VVhen partitioning a document into hypertext pages along sections, navigation links to establish paths between the pages are introduced. In the default 
setting, some paths capture previous-next relationships that reflect the logical 
tree-structured organization of the sections, providing connections between 
immediate siblings. This option asks for alternative previous-next navigation 
links, reflecting the linear succession of all the sections in the document. 
pic~array, pic-displaylines, pic-eqnarray, pic-tabbing, pic-tabular 
These options request pictorial representations for the named environments 
instead of representations employing HTML tables. 
_13, “13, no_, no“ TEX4ht modifies the native definitions of the special characters _ and " in order to enable the inclusion of HTML tags for subscripts and 
superscripts in mathematical formulae. The first two options request alternative definitions (with category codes of 13 instead of 12 for the characters). The 
second pair of options disables that behavior. 
ref caption References to figure and table environments inserted with the \ref 
command are assigned hypertext links pointing to the entry points of these environments. This option makes hypertext links that point to the captions within 
these environments. 
3.2 This option associates a request for HTML version 3.2 with the html option, 
instead of the default association with the Transitional 4.0 version of HTML. 
The latter version is considerably closer in its nature to BTEX than 3.2 is. It 
differentiates issues of style from content and logical structures, and we recommend its use wherever possible. 
f onts+ The normal setting expects the browsers to use their own default fonts for 
the default font of the document. This option suggests a specific font for the 
browsers to use. 
no_st;yle This option turns off the loading of HTML features that are designed 
specifically for the specified style file. For instance, the option no_amsart asks 
TEX4ht not to load the code it holds for the amsart package and the option 
no_array makes a similar request for the code that TEX4ht has to offer especially for the package array. In such cases, the user might need to tailor private 
contributions for the packages. 
info This option requests clues about the features of the package to be written 
into the log file. 
htm This option is a variant of the html option in which filenames are given main 
names of, at most, eight characters, and the HTML filenames are given the 
extension of htm. 

%==========180==========<<<---2
 
%%page page_180                                                  <<<---3
 
4.1 Using TEX4ht 
159 
4.1.2 Picture representation of special content 
The ultimate challenge is to get all the translated objects expressed in terms of 
hypertext tags capturing both the structure and the semantics of the content. VVhen 
such goals cannot be met, alternative representations have to be used. 
4.1.2.1 Mathematics and special characters 
The inline math environments \ (formula\) and the display math environments 
\ [formula\] request pictorial representations for their content. 
On the other hand, plain TEX inline math environments $for7mtla$ and display math environments $$formula$$ produce a mixture of text output for simpler 
subformulae and pictures for the other parts. 
In addition to mathematics, some accented characters, some other special characters, and ETEX picture environments are automatically translated into pictures in 
the default setup. Other entities require explicit requests to be translated to bitmap 
pictures. 
4.1.2.2 Linking to, and making, pictures 
The bitmap pictures rendered by browsers can be specified in stand-alone files referenced from the HTML code. The command 
\Picture [alt] {filename attributes} 
creates such a reference to an existing file and specifies attributes and alternative 
text representation for the pictures. The [alt] component is optional; when it is 
not provided, a default text representation is provided by the system. If attributes 
are given, they must be separated from the filename by a space. 
For example, the command 
\Picture[TUG logo]{httpt//www.tug.org/logo.gif ID=``tuglogo''} 
references a GIF file containing the logo of the TEX Users Group (TUG). This 
command produces the HTML code 
<IMG SRC="http://www.tug.org/logo.gif" ALT=``TUG logo'' ID=``tuglogo''> 
The following is a variant with a pair of commands: 
\Picture+ [alt] {filename attributes} 
\EndPi cture 
This requests a bitmap picture to be created for what is placed between them. Here, 
however, a missing [alt] parameter is taken to be a request to produce an alternative 
content-related textual representation for the picture. 

 
%%page page_181                                                  <<<---3
 
160 Translating ETEX to HTML using TEX4ht 
\Picture+{}% 
\begin{tabular}{lcr} 
1 & 2 & 3\\ 
<IMG SRC="try0x.gif" xxx & xxx & xxx\\ 
ALT="1 2 3 1 & 2 & 3 
xxx xxx xxx \end{tabular}% 
1 2 3"> \EndPicture 
The plus (+) can be replaced by a star (*). In this case the content is typeset 
within a vertical box. The filename is optional within the command, as is the file 
extension. If either is omitted, a system-created entry is provided. 
The filename of the last bitmap picture created is recorded in 
\PictureFile 
The command 
\NextPi ctureFile{filemzme} 
provides a filename for the next bitmap picture, if no filename is provided in the 
\Picture command. The \NextPictureFile command can come in handy for 
reaching concealed \Picture commands. 
\NextPictureFile{mypic.gif} 
<IMG SRC="mypic.gif" ALT=``ab''> and \[\alpha‘\beta\] and 
<IMG SRC="mypic.gif" ALT="[Picture]"> \Picture{\PictureFile} use 
use the same bitmap file. the same bitmap file. 
The default setting assumes the extension gif for the names that TEX4ht assigns to the bitmap files. The extensions jpg and png can be requested as an alternative with, respectively, the package options j pg and png. On the other hand, the 
command 
\Conf i gurePi ctureFormat{extemi0n} 
can be used dynamically to change the setting. 
4.2 A complete example 
\reffig{4-3} shows a display of an HTML file by a browser that is capable of handling 
the features of the Transitional 4.0 version of HTML. The corresponding display of 
the DVI output is seen in \reffig{4-4}, and the source ETEX file is given in \reffig{4-5}. 
The HTML file includes bitmap pictures-for the symbols (1), A, and oo and 
for the array within the display math environment \ [. . .\] . The quality of the pictures depends on the monitor in use and the density of the bitmap pictures. The 

 
%%page page_182                                                  <<<---3
 
4.2 A complete example 161 
Simulation of Energy Loss Straggling 
Mafia Physicist 
August 6, 1998 
1 Landau theory 
The Landau probability distribution may be written in terms of tl1e universal Landau function ¢( 
A) as [1]. The oscillator strengths]? and the atomic level energies E}. should satisfy the 
constraints 
fl +f3=1 (1) 
]]lnE1+f2lnE2=ln] (2) 
The following values have been chosen: 
_ :3 ifz<2 _ 
f*‘l2/z M32 =’ f*"1‘f* 
E E; : 11332311’ Z} E1 : H 
r:|II.4The following values are obtained with c = 4: 
"'3 n'B,max "'3 n’B,max 
16 16 2000 29.63 
100 27.59 oo 32.00 
References 
[1] L.Landau. On the Energy Loss of Fast Particles by Ionisation. Originally published in J. 
Phys. , 8:201, 1944. Reprinted in D.ter Haar, Editor, L. D. Landau, Collected papers, page 
417. Pergamon Press, Oxford, 1965. 
\reffig{4-3}: Display of an HTML file created by TEX4ht 

 
%%page page_183                                                  <<<---3
 
162 Translating I5‘I]§X to HTIVIL using TEX4ht 
Simulation of Energy Loss Straggling 
Maria Physicist 
August 6, 1998 
1 Landau theory 
The Landau probability distribution may be written in terms of the universal 
Landau function ¢(}\) as The oscillator strengths and the atomic level 
energies E,- should satisfy the constraints 
f1+f2 = 1 (1) 
f1lnE1+f2lnE2 2 lnI (2) 
The following values have been chosen: 
f2={0 “Z32 :~ 11:1-/12 
2/Z if Z > 2 
1 
2 I 1 
E2 = 8V => E1 = 
r = 0.4 
The following values are obtained with c : 4: 
"3 77»B,maz "3 nB,maz 
16 16 2000 29.63 
100 27.59 00 32.00 
References 
[1] L.Landau. On the Energy Loss of Fast Particles by Ionisation. Originally published in J. Phys., 8:201, 1944. Reprinted in D.ter Haar, Editor, 
L.D.Landau, Collected papers, page 417. Pergamon Press, Oxford, 1965. 
\reffig{4-4}: The standard ETEX output of \reffig{4-3} 

 
%%page page_184                                                  <<<---3
 
4.2 A complete example 163 

\documentclass{article} 
\usepackage{tex4ht} 
\title{Simulation of Energy Loss Straggling} 
\author{Maria Physicist} 
\begin{document} 
\maketitle 
\section{Landau theory} 
\label{sec:phys332-1} 
The Landau probability distribution may be written in terms of the 
universal Landau function $\phi(\lambda)$ as \cite{bib-LAND}. The 
oscillator strengths $f_i$ and the atomic level energies $E_i$ should 
satisfy the constraints 
\begin{eqnarray} 
f_1 + f_2 & = & 1 \label{eq:fisum}\\ 
f_1 \1n E_1 + f_2 \1n E_2 & = & \1n I \label{eq:flnsum} 
\end{eqnarray} 
The following values have been chosen: 
\[ \begin{array}{1cl} 
f_2 = \left\{ \begin{array}{ll} 
0 & \mathrm{if}\, Z \leq 2 \\ 
2/Z & \mathrm{if}\, Z > 2 \\ 
\end{array} \right. &\Rightarrow & f_1 = 1 - f_2 \\ 
£_2 = 10 z*2 \mathrm{eV} &\Rightarrow &£_1 = \left(\frac{I}{E_{2}‘{f_2}} 
\right)‘{\frac{1}{f_1}} \\ 
r = 0.4 & & \\ 
\end{array} \] 
The following values are obtained with $c=4$: 
\begin{tabular}{l1crr} 
$n_3$ & $n_{B,max}$ & & $n_3$ & $n_{B,max}$\\ \hline 
16 & 16 & & 2000 & 29.63\\ 
100 & 27.59 & & $\infty$ & 32.00 
\end{tabular} 
\begin{thebib1iography}{10} 
\bibitem{bib-LAND} 
L.Landau. 
On the Energy Loss of Fast Particles by Ionisation. 
Originally published in \emph{J. Phys.}, 8:201, 1944. 
Reprinted in D.ter Haar, Editor, \emph{L.D.Landau, Collected 
papers}, page 417. Pergamon Press, Oxford, 1965. 
\end{thebibliography} 
\end{document} 
\reffig{4-5}: The BTEX source file for the document of \reffig{4-3} 

 
%%page page_185                                                  <<<---3
 
164 
Translating I-KIEX to HTML using 'IEX4ht 
0 fiZg2 
f2: 2/Z ifZ>2 =" f1=1'f‘°‘ 
f1 
E2 = 
7" = 0.4 
_ I 
=*E1-?‘ 
\reffig{4-6}: A bitmap with density of 144 dots per inch 

bitmaps in \reffig{4-4} have a density of 110 dots per inch, while \reffig{4-6} shows 
the corresponding display of the last picture with a density of 144 dots per inch. 
As contrast, \reffig{4-7} is a variant of \reffig{4-3}. It is obtained by including the 
following CSS code (see \refsec{4_3_4_css}) in the compilation of the source file: 
\Css{.maketitle{ border:so1id 5px; width: 100\% }} 
\Css{.sectionHead, .1ikesectionHead { 
text-a1ign:right; 
font-family: cursive; 
border-bottom:so1id 2px; }} 
\Css{.thebib1iography { font-size : 70\%; }} 
\Css{body { text-align: justify; }} 
4.3 Manual creation of hypertext elements 
A relatively small number of low-level commands provides the foundation on top 
of which the desired hypertext outcome can (usually) be tailored. These commands 
provide direct access to HTML tags, the means to produce hypertext pages, the 
support needed to establish hypertext links, and a method to request a specific appearance of the content. 
4.3.1 Raw hypertext code 
Many useful features can be achieved by brute force inclusion of small fragments of 
HTML code. The command 
\HCode{c0ntent} 
is handy on such occasions, because it holds the processing of its content just to 
macro expansion. This restricted mode of operation outputs the characters in raw 
 
%%page page_186                                                  <<<---3
 
4.3 Manual creation of hypertext elements 165 
Simulation of Energy Loss Straggling 
Maria Physicist 
August 7, 1998 
: :.e..ma...1I...«, 
3 The Landau probability distribution may be written in terms of the universal Landau fimction ¢( 
A) as [1]. The oscillator strengths j? and the atomic level energies E, should satisfy the I 
constraints * 
J? +f2=1 (1) 
! ;]lnE,+f2lnE,=lnI (2) 
The following values have been chosen: 
x2=[:,. 52:: = an-ra 
1I:,=1uz’ev => E1: 
2 
7‘:|J.‘lThe following values are obtained with c = 4: 
"'3 n’B,7mzx "'3 n'B,max 
IE iiiiii *1?‘ ___ W2iiio-om§§._§; 
100 27.59 oo 32.00 
lP4[mm 
[1] L.Landau. On the Energy Loss of Fast Particles by Ionisation. Originally published in.[ Phys. 8:201, 1944. ReprintedinD.ter 
Hear. Editor, L. D. Landau, Collected papers, page 417. Pergamon Press, Oxford, 1965. 
nu.~........i...»..;..m..»».«.......m.H.,...~...«.........,H.,.,.<.r.m.m...t...,.....,.......,. .. 
\reffig{4-7}: A variant of \reffig{4-3} 
format instead of replacing them with the corresponding symbols of the current 
font (see \refsec{4_6_7_The_font_control_files} for a description of how TEX4ht works with fonts). 
Make it 
Make it \HCode{<STRONG>}\texttt 
<STRONG>&1t ; STRONG&gt ;</STRONG> . { <STRONG>}\HCode{</STRONG>} . 

 
%%page page_187                                                  <<<---3
 
166 
Translating I-§'IEX to HTML using TEX4ht 
The command 
\Hnewline 
may be used to force line breaks within the parameters of the \HCode commands. 
The alternative 
\HChar{/2tml-c}mmcter-c0de} 
command, on the other hand, introduces raw characters at a rate of one character 
per instance of the command. For example, in the default setting the ETEX special 
tilde character ~ is translated to \HChar{160}. 
It should be noted that, unlike the use of \HCode, a parameter of the \verb 
command of ETEX is not necessarily translated to exactly the same format in the 
HTML file. All that is required is to present to the reader something that looks 
like the source content. The same holds for the text enclosed within a verbatim 
environment. 
4.3.2 
The documents are automatically partitioned into hypertext pages, based on their 
logical structure and the options you have chosen. For brute-force creation of hypertext pages, the pair of commands 
Hypertext pages 
\HP age {ent7y-zmc/Jor} 
\EndHPage{} 
is available. 
The page may incorporate a navigation button with a command of the form 
\ExitHPage {exit-anchor} 
for establishing a backward path to the parent of the page. 
V 
\HPage{Enter} 
and then \ExitHPage{exit} 
\EndHPage{} 
enter 
the hypertext page. 
If the parameter of \ExitHPage is empty, the navigation buttons employ the 
anchors supplied in the \HPage commands. 

 
%%page page_188                                                  <<<---3
 
4.3 Manual creation of hypertext elements 
167 
4.3.3 Hypertext links 

HTML uses the tag <A HREF=" target-fz' Ze#target-Z.oc" NAME=`` cur-Zoc'' 
parameters>anch,or</A> for representing hypertext links. Each tag of this kind 
assumes knowledge of where the target file is located and the desired entry point 
into the file. In addition, each of these tags provides a name for the current location 
and allows for other parameters. 
The following command offers a similar functionality, while automatically deriving the local target files when they are not explicitly given. 
\Link [target-file parameters] {target-Zoe}{cur-loc}ancbor\EndLink 
Specifically, when target-file is empty, TEX4ht assumes the target file belongs to the 
current document, and it takes it upon itself to find the file. A file containing a 
location named target-loe is searched for by the \Link command. 
The component [target-file parameters] is optional, if both target-file and parameters are empty. When parameters is not empty, it must be preceded by a space. 
An external link to 
\Link [http : //www . tug. org 
ID=`` ORG''] {}{}TUG\EndLink. 
An internal \Link 
{to}{}link\EndLink{} to 
\Link{}{to}here\EndLink. 
Within the parameters of the \Link command, the special characters ~, _, and ‘I. 
should be entered using the commands \s1;ring~, \string_, and \‘Z respectively. 
4.3.4 Cascading Style Sheets 
HTML is a language for identifying structures within hypertext documents; it resembles the way the \section instructions of IMEX are employed to identify logical structures within source documents. In both cases, the entities do not deal with 
how the structures are to be presented to the readers, delegating such details to 
specifications provided elsewhere through special purpose languages. 
Cascading Style Sheets (CSS) [‘->CSS2] is a language for specifying presentations for HTML entities. TEX4ht collects such requests within a special file named 
jobname. css and issues some requests of its own. 
It is beyond the scope of this chapter to describe the CSS language. However, it 
should be realized that the language is easy to learn and use by authors with a basic 
knowledge of HTML and a little familiarity with desktop publishing terminology. 

 
%%page page_189                                                  <<<---3
 
168 
Translating I1}'I]:_‘X to HTML using 'I]3X4ht 
The command 
\Css{CSS code} 
is provided for requesting presentation of document elements. 
Given source code of the form 
\section{Header} Text. 
\par 
More text 
TEX4ht will produce HTML output like this: 
<H2>1 Header</H2> <P CLASS=``noindent''> Text. 
<P CLASS=``indent''> More text 
and CSS code like this: 
P.noindent { text-indent: Oem } P.indent { text-indent: 1.5em } 
A request to color the title green can be made with a command of the form 
\Css{ H2 { color: green } }. 
A CSS file can be specifically requested with the environment 
\CssFi1e [list-ofifiles] 
content 
\EndCssFi1e 
CSS code can be imported from other files (list-of-file:), as well as explicitly given 
within the environment. 
The \EndPreamb1e command calls this environment to create a CSS file, if the 
user does not make an earlier request for such a file. 
The CSS file should include the comment / * css . sty */ on a separate line so 
that t4ht, the tex4ht postprocessor (see \refsec{4_6_4_A_look_at_t4ht}), can identify it as a place to 
put the content of the \Css instructions. Without such a line, that content will be 
ignored. The filename and the initial content of the file can be reconfigured with 
the \Configure{CssFile}{filename}{content} command. 
Inline CSS code can be created using the following environment: 
\Css content\EndCss 
Consider the following source and configuration files: 
Z try.tex Z try.cfg 
\documentclass{article} \Preamb1e{htm1} 
\usepackage[try]{tex4ht} \begin{document} 
\begin{document} 

%==========190==========<<<---2
 
%%page page_190                                                  <<<---3
 
4.4 How TEX4ht works 
169 

\Css \CssFi1e 
H2 { color : red; } H2 { color : blue; } 
\EndCss /* css.sty */ 
\Css{ H2 { color : green; }} \EndCssFi1e 
\end{document} \EndPreamb1e 
This will create the following HTML and CSS files: 
<!--- try.htm1 --> /* try.css */ 
<STYLE TYPE="text/css"> H2 { color : blue; } 
H2 { color : red; } H2 { color : green; } 
</STYLE> 
4.4 How TEX4ht works 
It takes three phases to translate a source document into hypertext (see \refsec{4_6_1_The_translation_process} 
on page 184): a compilation of the source by the TEX program into DVI code, a 
manipulation of the DVI code by the 1;ex4ht program, and a processing of looseend tasks required for completing the translation. 
4.4.1 From I’}’IEX to DVI 
IATEX requires two compilations of a source file by TEX to establish crossreferences, and TEX4ht might require a third compilation to get all the hypertext 
links in place. On rare occasions when a tabular environment is used, with many 
cells being merged, more compilations might be needed to let the system work out 
how the cells should look. 
When HTEX loads the TEX4ht package, it loads the file 1;ex4h1; . sty and looks 
at just a few lines there. Then it records a request for loading the file again at a later 
time, when it will scan the rest of the file. The second loading takes place when the 
\begin{document} code is encountered, at which time the requests made by the 
package options are also honored. 
Since TEX4ht enters into the picture only when \begin{document} is reached, 
some earlier user definitions might not get the full attention of TEX4ht, unless they 
are redefined in a configuration file. For instance, TEX4ht would have difficulties 
introducing HTML tags for the superscript of a macro \newcom1nand{\x}{a"{b}} 
which was defined before the start of the the document environment. 
4.4.2 From DVI to HTML 
DVI is a page description language that includes instructions for specifying what 
content should go at which location in a print-oriented medium. HTML, on the 
other hand, is a structure-oriented language with little regard to layout issues. 

 
%%page page_191                                                  <<<---3
 
170 
Translating I*}'I]§X to HTML using TEX4ht 

Consequently, in many respects, a translation from DVI to HTML is a backward process, having to reconstruct information that might have been lost in the 
translation from BTEX to DVI. This backward process may well fail, if the DVI 
code results from a source document that places things outside the normal stream; 
\hspace{-O . 6em} is a possible example. 
During translation from DVI to HTML, font calls are processed using virtual 
hypertext fonts (\refsec{4_6_7_The_font_control_files} on page 190). If these are missing or are inappropriate, new ones can be composed by the user without too much effort. 
4.4.3 Other matters 
The last phase of the translation turns its attention to the production of bitmap pictures from DVI code. To that end, TEX4ht relies on tools available for the current 
platform and that might also offer more than one route for producing the pictures. 
The creation of pictures is the bottleneck of the translation process and may 
take a long time to complete. Some shortcuts might be taken to speed up the process; for instance, if bitmaps from earlier compilations are already available, the 
system-dependent utilities may allow them to be reused. 
4.5 Extended customization of TEX4ht 
Most ETEX users should require very little background information, if any, in addition to that already covered in this chapter. But TEX4ht is a large system with many 
facets to explore. A taste of that world is provided in the following sections, which 
deal with some aspects of customizing and running the system. 
Most readers will probably be best served by quickly skimming over the following sections to get a general impression of the topics addressed. The details have 
little bearing on the How of content in these sections, and they can be ignored until 
they are needed for handling specific requirements. 
4.5.1 Configuration files 
TEX4ht introduces intermediate interfaces of its own, located between the interfaces used by IATEX within the source files, and those used by HTML within the 
output files. It does this by placing “hooks" in the IATEX style files; these are commands that the user can redefine to get the effect they want. The intermediate 
interfaces separate themselves from the style concerns of IATEX, on one hand, and 
offer structures similar to those of HTML, on the other. As a result, users can quite 
easily tailor a different outcome just by defining the different hooks to produce 
appropriate HTML code. 
The conventional wisdom of placing definitions together in separate customization files applies also to the configuration commands of TEX4ht. However, 

 
%%page page_192                                                  <<<---3
 
4.5 Extended customization of T]:_‘X4ht 
171 
an additional motivation for the configuration files stems from the need to direct 
the output to fit the structural requirements of HTML files. 
Specifically an HTML file consists of a header and a body, each part expecting 
a different type of content. The configuration files identify these parts and provide 
the content for the header. 
4.5.1.1 Implicit and explicit files 
If the first package option is not the name of a configuration file, a default configuration is used. It looks like this: 
\Preamble{optiom} 
\begin{documen1;} 
\EndPreamble 
However, if the first option does refer to a configuration file, then the configuration 
file must have the following structure: 
early definitiom 
\Preamble{optiom} 
definitiom 
\begin{do cument} 
imertiom into the header of the HTML file 
\EndPreamble 
\reffig{4-8} shows an example of source and configuration files, as well as the HTML 
and CSS files they produce. 
Upon reaching the \usepackage command, the file 1;ex4h1;.sty is partially 
loaded to scan a few definitions. Then the configuration file is read until the 
\Preamb1e command is encountered. The remainder of the style file is read and 
acted upon when the \begin{documen1;} command in the source document is 
reached. 
With the exception of a package option standing for the name of a configuration file, the distribution of the other options between the \usepackage and the 
\Preamble command is unimportant. Moreover, unlike the case for the first option 
of the \usepackage, no restriction is made on the type of the option that appears 
first in the \Preamble command. 
4.5.1.2 Embedded configuration 
The configuration file can also be embedded directly within the source document, 
instead of being indirectly incorporated through a \usepackage command. Such 
an approach might make the placement of the \begin{documen1;} instruction in 
the configuration file clearer. However, to preserve the authoring style promoted 
by IATEX, users are highly discouraged from employing this approach. 

 
%%page page_193                                                  <<<---3
 
172 
Translating I*}'I]§X to HTML using TEX4ht 
\documentclass{article} 
\usepackage[try,htm1]{tex4ht} 
\begin{document} 
\begin{itemize} 
\item First 
\item Second 
\end{itemize} 
\end{document} 
(3) 
\Preamb1e{} 
\Css{ UL{border : solid 1px;} } 
\begin{document} 
\HCode{<META NAME=``description'' 
CONTENT=``examp1e''>} 
\EndPreamb1e 
(b) 
<HTML><HEAD> 
<META NAME=``description'' 
CONTENT=``examp1e''> 
<LINK REL=``stylesheet'' 
TYPE="text/css" 
HREF="try.css"> 
</I-IEAD><BODY> 
<UL><LI>First 
<LI>Second 
</UL></BODY></HTML> 
(C) 
UL{border : solid 1px;} 
(d) 
\reffig{4-8}: Input files: (a) try . tex and (b) try . cfg. Output files: (c) try . html and 
(d) try . css 
The embedding can be achieved by replacing the instruction 
\usepackage [options] {1;ex4ht} with an \input{1;ex4h1; . sty} command, and 
substituting the \begin{documen1;} in the source document with the contents of 
the configuration file. In such a case, the options of the \usepackage command 
should migrate into the list of options of the \Preamb1e command. 
The following source document is the source of \reffig{4-8}(a), with the configuration file of \reffig{4-8}(b) embedded in it. 
\documentclass{article} 
\input{tex4ht.sty} 
\Preamb1e{htm1} 
\Css{UL { border : solid 1px; }} 
\begin{document} 
\HCode{<META NAME=``description'' CONTENT=``examp1e''>} 
\EndPreamb1e 
\begin{itemize} \item First \item Second \end{itemize} 
\end{document} 
4.5.2 Tables of contents 
The \1;ableofconten1;s command of IATEX is enriched with new features in 
TEX4ht. To allow for easy control over the kind of entries it includes, the way it 

 
%%page page_194                                                  <<<---3
 
4.5 Extended customization of 'Ij3X4ht 
173 
is presented, and the locations where it can be included, some of these features 
are indirectly activated by the package options 1, 2, 3, and 4 (see \refsec{4_1_1_Package_options} on 
 15 6). 
Remember that because one IATEX file will often generate many HTML files, 
each output may have its own table of contents. Hence, we use the term “tables of 
contents" instead of the usual “table of contents." 
4.5.2.1 Choice of entries 
The kinds of entries to be included in the tables of contents are determined 
by ETEX in the usual way. As an alternative, TEX4ht adds a variant command, 
\tab1eof contents [units] , in which the kind of entries need to be explicitly specified. The parameter units is a comma-separated list of names of sectioning commands (without backslashes). Starred versions (ending in *) are replaced by names 
with like prefixes, and appendixes are requested by the word appendix. Thus 
\tab1eofcontents[chapter,appendix,section,1ikesection] 
requests a table of contents with entries pointing to the logical units created by the 
\chapter, \section, and \section* commands. 
4.5.2.2 Local tables of contents 
The command 
\TocA1;{units} 
requests a table of contents at the start of each logical unit of the specified type. 
The parameter units is a comma-separated list similar to that offered for the new 
variant of the \1;ab1eofcontents command; the first name in the list specifies the 
unit that is to have a local table of contents. The other names specify the kind of 
entries to be included in the table. If they are preceded with a slash, they specify 
termination points for the tables of contents. Thus 
\TocAt[chapter,section,/likesection] 
requests a table of contents at the start of each chapter. The entries corresponding 
to the \section commands that follow should be included, but the list terminates 
upon reaching a \section* or the next \chapter. 
The tables requested by the command \TocA1;{units} appear immediately after 
the titles of the section units. The Variant 
\TocA1;*{units} 
produces similar tables that follow the preambles of the units instead of immediately 
following the titles. 

 
%%page page_195                                                  <<<---3
 
174 
Translating ETEX to HTML using TEX4ht 
\ConfigureToc{section} 
{\null}{~}{}{ } 
\ConfigureToc{likesection} 
{}{*~}{}{ } 
\tableofcontents[section,likesection] 
\section{A Long-expected Party} 
\section{The Shadow of the Past} 
\section*{Three is Company} 
\reffig{4-9}; Configuring the tables of contents 
4.5.2.3 Configuring the entries 
Each entry in the table of contents is derived from three fields: a mark, a title, and a 
 number. Typically the mark is the number of a section (or it is simply empty), 
and the page number is of little significance in this context. 
The following command can be used to determine how the contents entries 
for the logical units of type unit are to be created. The unit names follow the same 
conventions as those used in the parameter of the enhanced \tableofcontents 
command. 
\Conf i gureToc{unit}{mar/e}{title}{page}{end} 
Each contents entry will be composed of the mark parameter followed by the mark 
field, the title parameter followed by the title, and the page parameter followed 
by the page number. Finally comes the end parameter. If any parameter is empty, 
the corresponding field is omitted in the table of contents. The effect is shown in 
\reffig{4-9}. 
4.5.2.4 Configuring the tables 
Tables of contents have hooks before their entry points, after their exit points, after 
their last entries, at the start of each nonindented paragraph, and at the start of each 
indented paragraph. The following command configures the hooks; 
\Conf igure{tableof contents}{befiire}{end}{after}{n-par}{i-par} 
The hook after the last entry is processed within the environments of the table of 
contents. The hook after the exit point of the table of contents is processed within 
the environment around the tables. 
Local tables of contents can be further configured in a similar manner with the 

 
%%page page_196                                                  <<<---3
 
4.5 Extended customization of TEX4ht 
175 
\Configure{section} 
{\HCode{<HR>}} {} 
{Section \thesection: } {} 
\section{When I was one} 
I sucked my thumb. 
\section{when I was two} 
I buckled my shoe. 
\reffig{4-10}: Configuring the section headings 
commands: 
\Conf igure{To<:At }{l7efore}{zzfter} 
\Conf i gure{To<:At *}{I7efore}{;zfter} 
The new configurations must be supplied before the \TocAt and \To<:At* commands. 
4.5.3 Parts, chapters, sections, and so on 
Sectionin commands determine the underl 'n structures of documents and the 
u g I u u . y1 g . u ’ u y 
quite often guide the partitioning of hypertext documents into files. This subsection 
shows how such entities can be customized. 
4.5.3.1 Configuring the boundary points and titles 
The sectioning commands produce logical units characterized by their starting and 
ending points, as well as by their titles. The following command contributes content to such units; these are included at the start of the units, at the end of the units, 
before their titles, and after their titles, respectively. 
\Conf i gure{um't}{top}{l7ottom}{b%)re} {after} 
If the top and bottom parameters are both empty, that part of the configuration command is ignored and the old values remain in effect. The same applies to the parameters before and after. The effect is shown in \reffig{4-10}. 
The unit names follow the same conventions as those provided for the parameters of the enhanced \tableof contents command. 

 
%%page page_197                                                  <<<---3
 
176 
Translating I‘.‘:‘lt‘X to HTML using TEX4ht 
\CutAt{+sect ion} 
\section{When I was three} 
I found a key. 
\section{When I was four} 
I knocked on the door. 
\reffig{4-11}: Configuring sections to make multiple files 
4.5.3.2 Partitioning into files 
The package options 1, 2, 3, and 4 (see \refsec{4_1_1_Package_options} on page 156) implicitly activate 
the command 
\CutAt{um'ts} 
for partitioning the documents into files. The parameter units is a comma-separated 
list of unit names. The first name in the list specifies the logical units for which 
separate hypertext pages are requested. The hypertext pages extend until a unit 
whose name appears in the rest of the list is encountered. Thus 
\CutAt{chapter,likechapter,appendix,part,1ikepart} 
requests hypertext pages for the logical units defined by \chapter. Furthermore, 
the command says that starred chapters, appendixes, parts, and starred parts are 
logical units that should not be included within chapters. 
Typically the \CutAt{zmits} command is used along with a table of contents 
whose entries provide links to the hypertext pages that are requested by the command. A possible alternative is to employ the variant command \CutAt{+um'ts}, 
which, besides the hypertext pages, also creates links to the pages. The effect is 
shown in \reffig{4-11}. 
The links to the pages are enclosed between delimiters that are configurable by 
the command 

 
%%page page_198                                                  <<<---3
 
4.5 Extended customization of 'IEX4ht 
177 
\Conf igure{+CutAt}{zmit} {ldel}{rdel} 
Thus 
\Configure{+CutAt}{section}-[*~}{} 
requests left delimiters *~ and no right delimiters for the links to pages that result 
from the \section command. 
4.5.3.3 Setting boundary points 
The \CutAt{um'ts} and \CutAt{+um'ts} commands work out the end points of the 
sections that are placed in separate files. The end points of the other sections should 
be specified using the following command: 
\Conf i gure{em1mzit} {units} 
The endmzit stands for a unit name prefixed by an end, and the parameter units is a 
comma-separated list of unit names. 
4.5.3.4 Customizing the navigation buttons 
The hypertext pages of the sections include panels of navigation buttons both at 
the top and the bottom of each page. Each of the buttons is embedded between a 
left and a right delimiter. The buttons point to the next hypertext page, the front 
and the tail of the previous page, the front and the tail of the current page, and 
the parent page. The following command allows you to customize the buttons and 
their delimiters, and the effect is shown in \reffig{4-12}. 
\Conf igure{crosslinks}{ldel}{rdel}{next}{prev}{prev-tail}{fiont}{tail}{up} 
The next command deals with the panels themselves. It specifies content to be 
included before and after the front and tail panels, respectively: 
\Conf i gure { cro s s 1 inks+} {before-fi*ont} {afler-fitont} {before-tail} {afler-taz'l} 
4.5.4 Defining sectioning commands 
New sectioning commands can be introduced through instructions of the form 
\NewSe ct i onfimdname} {marker} 
for the TEX4ht package to be aware of their existence and to offer its standard 
services. The parameter marker specifies the markers to be submitted with the titles 
to the tables of contents. Typically such markers either are empty or consist of the 
sequence number of the current section unit. An example is shown in \reffig{4-13}. 

 
%%page page_199                                                  <<<---3
 
178 Translating ETEX to HTML using TEX4ht 
\Configure{crosslinks}{*~}{ } 
{$\Rightarrow$}{$\Leftarrow$} 
{}{}{}{$\Uparrow$} 
\section{when I was five} 
I caught a fish alive. 
\section{When I was six} 
I broke my sticks. 
\reffig{4}-12: Configuring navigation buttons 
\NewSection{\head}{\arabic{head}} 
\Configure{head} 
{\addtocounter{head}{1}}{} 
{\par\arabic{head}. }{} 
\ConfigureToc{head}{ } {: }{}{} 
\newcounter{head} 
\tableofcontents[head] 
\head{When I was seven} I went to heaven. 
\head{When I was eight} I met my fate. 
\reffig{4-13}: Defining a new sectioning command 
4.5.5 Lists 
The list and trivlist environments are basic structures of I§}TEX, on top of 
which quite a few environments are defined. Some of these are themselves variants 
of listing environments, for instance, the description, itemize, enumerate, and 
thebibliography environments. 
Other environments are display-oriented in nature, relying on empty-label, 
single-item lists just for their typesetting characteristics, for instance, the center, 
flushleft, flushright, quotation, quote, verbatim, and verse environments. 
The theoremlike environments, defined by the \newtheorem command, are also 
single-item lists, but their titles are offered as labels of \item commands. 
The following command provides content to be included before the lists, after 
the lists, before the labels of the items, and after the labels of the items. The effect 
is shown in \reffig{4-14}. 

%==========200==========<<<---1
%==========200==========<<<---2
 
%%page page_200                                                  <<<---3
 
4.5 Extended customization of 'I]3X4ht 179 
\ConfigureList{description} 
{}{\HCode{<HR>}} 
{\HCode{<HR>}} {: } 
\begin{description} 
\item[cat] milk. 
\item[rabbit] lettuce. 
\end{descript ion} 
\ConfigureEnv{em} 
{\HCode{<HR>}} 
{\HCode{<HR>}} {}{} 
\begin{em} 
An emphasized environment. 
\end{em} 
\ConfigureEnv {flushright} 
{\HCode{<DIV ALIGN=``RIGHT''>}} 
{\HCode{</DIV>}} 
{\Hcode{<HR>}} {\HCode{<HR>}} 
\begin{flushright} 
A paragraph of text within a 
flushed right environment. 
\end{flushright} 
\reffig{4-14}: Configuring lists and environments 
\Conf igureList {mzme}{pre-list} {post-list} {pre-label} {post-label} 
VVhen list environments are defined in terms of other list environments, the contribution of \Conf igureList applies only to the lists in the top layer. Since ETEX defines the description environment in terms of the list environment, and TEX4ht 
configures both of them with the \Conf igureList command, the configuration 
given to the list environment does not show within the description lists. 
4.5.6 Environments 
ETEX environments constructs are customizable by commands invoked at the entry 
and exit points: 
\ Conf i gureEnv{mzme} {before-emz} {zzfierenv} {before-list} {tzfier-1231} 

 
%%page page_201                                                  <<<---3
 
180 
Translating EQFX to HTML using TEX4ht 
The parameters before-environment and after-environment specify material to be 
placed before and after the named environment; if both parameters are empty, they 
are ignored. An example is show in \reffig{4-14}. 
Similarly, if at least one of the parameters before-list or zzfter-list is not empty, 
the environment is assumed to be realized in terms of a list-making environment. 
A call is then made to \Conf igureList{mzme}{bcflre-1ist}{zzfier-li5t}{}{} for configuring the underlying list-making commands. 
4.5.7 Tables 
TEX4ht goes a long way toward offering satisfactory representations for tables, but 
it does not provide a complete solution. Sometimes it fails, and special configurations or pictures might be called for. 
4.5.7.1 The array and tabular environments 
The array and tabular environments differ only in that the first is handled in 
math mode and the second is processed in normal mode. The following command 
customizes these environments, before the tables, after the table, before each row, 
after each row, before each entry, and after each entry. 
\Conf i gure{tab1e}{pre-tbl } {post-tbl } 
{pre-row}{post-row}{pre-entrj)/}{post-entry} 
To help configure the tables, the \HRow, \HCo1, and \ALIGN macros can be used. 
The first pair of macros produces the row and column numbers in which the commands appear; the third macro produces an encoding for the alignment information 
of the table, as we show here: 
\Configure{tabular} {}{} 
{\HRow: }{\HCode{<BR>}} 
{}{(\HCol)} 
\begin{tabular}-Cccc} 
A&B&C\\ D&E&F 
\end{tabular} 
For a centered column, \ALIGN gives a triplet made up of the digit 0, the column number, and the minus character -. For left-aligned, right-aligned, and paragraph columns, similar triplets are produced. The only difference is that the characters <, >, and p, respectively, are used instead of -. 
Tables with \mu1tico1umn entries need a few ETEX compilations to stabilize; 
TEX4ht slowly learns about the dimension of the spanning from information provided in earlier compilations. In configuring contributions to entries of tables, the 
\MULTISPAN macro may be tested to determine the number of columns spanned by 
the entries. 

 
%%page page_202                                                  <<<---3
 
181 
4.5 Extended customization of TEX4ht 
Consider the following source code: 
\Configure{tabular} {\HCode{<TABLE>}} {\HCode{</TABLE>}} 
{\HCode{<TR>}} {\HCode{</TR>}} 
{\HCode{<TD \ifnum \MULTISPAN>1 COLSPAN="\MULTISPAN"\fi>}} 
{\HCode{</TD>}} 
\begin{tabu1ar}{lr} \multicolumn{2}{c}{merge}\\ 
first & second \end{tabular} 
The output for this is the following: 
<TABLE><TR> <TD COLSPAN=``2''>merge</TD> </TR> 
<TR> <TD>first</TD> <TD>second</TD> </TR></TABLE> 
The package options pic-array and pic-tabular (\refsec{4_1_1_Package_options} on page 156) 
request a picture version of all the array and tabular tables, respectively. 
The \\ command is treated as a row separator. To avoid undesirable empty 
rows at the end of the tables, the \\ should not be inserted after the last row. On 
the other hand, the character ~ may be used to introduce invisible content for empty 
cells. This will allow for the possibility of empty and nonempty cells being treated 
differently by browsers. 
4.5.7.2 The eqnarray environment and the like 
The variants of the eqnarray environment are configurable by \Configure commands similar to those used for the array and tabular environments. Alternatively 
a picture version may be requested with the pic-eqnarray option. 
4.5.7.3 The tabbing environment 
The following command specifies contributions to be included before and after the 
rows, and before and after the entries of the tabbing environment. In addition, 
the command allows for a decimal number to specify a magnification factor for the 
widths of the entries. 
\Configure{tabbing} [mag] {pre-row}{post-row}{pre-ent1y}{post-entry} 
The component [mug] is optional when no change in magnification is desired. The 
contributions offered by the parameters pre-row, post-row, pre-entry, and post-entry 
are ignored when all of these parameters are empty. The command \TABBING may 
be used to set the widths of all the entries, where entries with no bound on their 
width have a zero for their specified widths. The trailing entries of the rows have 
this feature. 
Reconfiguring tables without compromising their properties is probably a task 
requiring more knowledge of raw TEX programming than most users possess. 

 
%%page page_203                                                  <<<---3
 
182 Translating HIEX to HTML using TEX4ht 
However, the amount of TEX code to be written is typically quite small in size. 
\newcount\c 
\def\Width#1//{\gdef\TABBING{#1}% 
\ifnum\c>O \HCode{ WIDTH="\the\c"}\fi} 
\ConfigureEnv{tabbing}{}{}{\Configure{HtmlPar}{}{}{}{}}{} 
\Configure{tabbing} 
{\HCode{<TABLE><TR>}} {\HCode{</TR></TABLE>}} 
{\HCode{<TD}\afterassignment\Width\c\TABBING//\HCode{>}} 
{\HCode{</TD>}} 
\begin{tabbing} 
LaTeX: \=tabbing\\ 
TeX: \>settabs 
\end{tabbing} 
The above fragment of ETEX source translates to the following HTML code. 
<TABLE><TR><TD WIDTH=``71''>LaTeX:</TD> 
<TD>tabbing</TD> </TR></TABLE> 
<TABLE><TR><TD WIDTH=``71''>TeX:</TD> 
<TD>settabs</TD> </TR></TABLE> 
A picture version may be requested with the package option pic-1; abbing. The 
variant pic-t abbing ’ applies only to those instances employing the \ ’ directive of 
the tabbing environment. That directive is not fully supported by TEX4ht. 
4.5 .8 Small details 
Most of the features described so far are tied to specific constructs of ETEX, and 
they are of little use elsewhere. The following features, of a more general-purpose 
nature, deal with basic issues. 
4.5.8.1 File names 
HTML files may result from requests made through the package options 1, 2, 3, 
and 4 and from \CutAt and \HPage commands. In such cases unless the users offer 
names of their own, the filenames are automatically created by the system. 
The \Fi1eName command can be used to find out the name of the current file. 
On the other hand, the command 
\NextFi1e{filemzme} 
may be used to suggest a name for the next HTML file. 
4.5.8.2 Conditional code 
The command 

 
%%page page_204                                                  <<<---3
 
4.5 Extended customization of 'I]3X4ht 183 
\ifHtm1 true-part\e1sefalse-part\fi 
enables us to choose content based on whether the html package option is used. 
4.5.8.3 Environments for scripts 
The \HCode command allows the user to write small fragments of raw code into 
the HTML file. The command 
\ScriptEnv{mzme}{prefix} {mflix} 
provides the means of defining environments for including larger fragments of raw 
code. 
\ScriptEnv{css} 
{\HCode{<STYLE TYPE="text/css"> 
<STYLE TYPE="text/css"> \Hnewline<!--}\Hnewline} 
<!-- {\HCode{-->\Hnewline</STYLE>}} 
UL { border : solid 1px; } \begin{css} 
H1 { color: green } UL { border : solid 1px; } 
--> ' H1 { color: green; } 
</STYLE> \end{css} 
4.5.8.4 Content for paragraph breaks 
The following command allows you to specify what material is to be inserted at 
the start of a paragraph and what is to be saved in \EndP at this point. There are 
separate parameters for when the first line of the paragraph is indented or not. 
\Conf igure{Htm1Par} {noindent-P} {indent-P} {noindent-save} {indent-save} 
The task of \EndP is typically to deliver code from the start of a paragraph to its 
end. 
\Configure{HtmlPar} 
{\HCode{<P><H2>}} 
{\EndP \Hcode{<P>}} 
{\HCOde{</H2>}} {} 
<P><H2>Head </H2><P> Body \1-loindent Head \Par Body 
There are extra commands to enable finer local control over the contributions 
at the start of paragraphs. The command \IgnorePar ignores the contribution of 
content at the start of the next paragraph, and the command \ShowPar provides 
content at the start of the next paragraph. Similarly the \NoIndent command says 
that the first line of the next paragraph should not be indented, and the \Indent 
command says the first line of the next paragraph should be indented. 

 
%%page page_205                                                  <<<---3
 
184 
Translating HIEX to HTML using 'IEX4ht 
\newcommand{\try}{\Pre Just trying.\Post} 
\NewConfigure{try}[2]{% 
\newcommand\Pre{#1}% 
\newcommand\Post{#2}} 
\Configure{try}{\HCode{<H2>}}{\HCode{</H2>}} 
\try 
\Configure{try}{}{} 
\try 
\reffig{4-15}: Adding new hooks 
4.5.8.5 Creating new hooks for TEX4ht 
The core of TEX4ht is programmed to deal with general situations created by the 
underlying machinery of TEX, and in general it does a good job there. However, 
the underlying features typically have very little to do with ytruttuml properties of 
commands defined in private and public style files. To capture such properties, the 
definitions must be extended to include boo/ex. To maximize the benefit of the hooks, 
they should be configurable. The command 
\NewConf igure{name} [digit] {axxignmentx} 
is designed for this purpose. 
This command introduces a hook configurable by a \Configure command. 
The digit specifies the number of configurable fields the \Configure command 
will need. These fields are accessible with the \NewConfigure command through 
the parameter names #1, #2, and so forth. An example is shown in \reffig{4-15}. 
TEX4ht provides hooks with initial values for the commands in the style files 
of ETEX, the plain file of TEX, style files of A_MS-l£§TEX and A/\48-TEX, and other 
commonly used packages. 
4.6 The inner workings of TEX4ht 
An insight into how the system operates can help with installation or with improving and extending its use. It can also explain the system’s capabilities and limitations. 
Although most of these issues are typically important for only a few, more advanced 
users, many readers might like to skim this section quickly just to get a general impression of the topics covered. 
4.6.1 The translation process 
The command line 
ht latex filename 

 
%%page page_206                                                  <<<---3
 
4.6 The inner workings of TEX4ht 185 
ITTh4L 
erL idv -~DVI-to-GIF-- GIF 
texTEX -DE tex4ht 
CSS-- - CSS 
tfm & htf 
fonts lg -- t4ht 
Int Filex 
tex4ht st 
* . htf Virtual hypertext fonts 
Output Filex 
jolmame. ot c 
\reffig{4-16}: The workflow and files of TEX4ht 
requests a translation of the source filemzme into HTML. The script ht calls the 
different utilities involved in the translation process (\reffig{4-16}); it consists of five 
steps, of which the first three involve running ETEX: 
latex f1; Lename 
latex f1) lename 
latex f 12 Lemma 
tex4ht f1‘, L ename 
t4ht f1‘. L ename 
4.6.2 Running IHEX 
The three compilations of the IATEX source by TEX are needed to ensure both 
proper references within hypertext links and proper arrangements of cells in tables 

 
%%page page_207                                                  <<<---3
 
186 
Translating IISIEX to HTML using 'I]3X4ht 
0O\IO\v\4>V"'\3"‘ 
(tex4ht.sty) (url.sty) 
(try.cfg (tex4ht.sty 
--- needs --- tex4ht try --(tex4ht.tmp) (try.xref) 
--- file try.css --(tex4hta.sty)) (try.aux)) [1] [2] 
(try.otc[3]) [4] 
1. 19 Writing try.idv[1] (try0x.gif) 
\reffig{4-17}: Runtime information in the log file from ETEX 
containing the \mu1tico1umn command. More compilations might be needed for 
sources in which the \mu1tico1umn command merges a large number of cells. 
The log file of the compilation will include information similar to that shown in 
\reffig{4-17} (without the line numbers). The third line in the example requests processing by the tex4ht program of the output file try.dvi to produce a try.htm1 
file and possibly other files as well. The fifth line tells us about the style file, named 
try. css, supplied for the HTML output. The eighth line mentions a bitmap file, 
named tryox. gif, that needs to be produced by t4ht from the first figure of the 
try. idv file (\refsec{4_6_4_A_look_at_t4ht} on the facing page). 
The first and second lines show the two times that the package file tex4ht . sty 
is read. The first time gets the \Preamb1e command and about half a dozen hooks 
with default configurations. The second time is activated by the \Preamb1e command, and it reads the portions of the style file selected by the options. The early 
set of hooks allows the customization of the headers before they are written during 
the second loading of the file. They include the hooks named HTML, HEAD, BODY, 
TITLE, TITLE+, and Htm1Par (use of these is discussed in Section B.2.1.4). 
During a compilation, TEX4ht stores the entries for the tables of contents in a 
toe file; these are to be used in the next run of the source document. The entries 
from the previous run are moved into an otc file so that they are available during 
the current run. The file try. otc shown in the seventh line of the example log file 
shows such a file in use. 
4.6.3 Running the tex4ht program 
Running a source file with ETEX outputs a standard DVI file, containing special instructions for the tex4ht package. The postprocessor program tex4ht1 uses these 
instructions to determine how the DVI code should be processed. They tell tex4ht 
where the output files should start and end, what names should be given to the files, 
1The tex4ht utility is programmed in C with system calls to a very few simple standard functions. 
Its hypertext fonts are system-independent files of plain text and, like fonts of TEX and its DVI output, 
are portable across all systems. This means that the tex4ht utility is easy to transport between different 
platforms, and the output independent of the platform on which the program is run. 

 
%%page page_208                                                  <<<---3
 
4.6 The inner workings of TEX4ht 
187 
the HTML decorations to be assigned to the symbols of the different fonts, where 
the code for the bitmap pictures resides, and so on. 
To perform the translation into HTML, the tex4ht utility needs to know where 
the font metrics of TEX reside (tfm files), where the private hypertext fonts of 
tex4ht are stored (see htf fonts, \refsec{4_6_7_The_font_control_files} on page 190), and other information 
that relates to the environment in which the utility is working. Some of this information might be included in the executable code of tex4ht during compilation. 
The rest of it is specified in a control (env) file (see \refsec{4_6_8_The_control_file} on page 193). 
An invocation of tex4ht, without any parameters, produces a usage message 
like the following: 
tex4ht in-file[.dvi] 
[-ttfm-font-dir] 
[-ihtf-font-dir] 
[-eenv-dir] 
[-dout-dir] 
[-gbitmap-file-ext] [-blg-divide-script] [-slg-gif-script] 
Usually the user will simply give the name of the dvi file. For instance, the 
command tex4ht try. dvi produces a main file named try. html. 
The command-line option -gb'£tma.p-f'£le-e:z:t determines the file extension 
of the bitmap files of picture symbols. The option -dout-d'£'r' specifies a directory 
for the output files (for instance, tex4ht try -d/tmp/ on a UNIX platform). The 
three options -een'u-dir, -ihtf-font-div‘, and -ttfm-font-d'1l7* specify directories to be searched for the control file (see \refsec{4_6_8_The_control_file} on page 193), the virtual 
hypertext fonts, and the TFM files. 
\reffig{4-18} shows an example of messages produced during a run of tex4ht. 
The second line shows the TEX metric cmr1O being found, and the third line shows 
the corresponding virtual hypertext font being loaded. The fifth line reports the 
loading of the file cmr .htf instead of cmbx1O .htf. The symbols of the bold font 
have the same HTML representations as those of the normal font. The two fonts, 
however, will end up with different presentations, using font information provided 
in the CSS file. 
The tenth line requests the execution of a script (see \refsec{4_6_6_A_taste_of_the_lg_file} on page 189). 
4.6.4 A look at t4ht 
When tex4ht is done with processing the DVI file of TEX, it leaves behind a scaleddown DVI file with the extension name idv. This consists of all the DVI code fragments that need to be translated into pictures, with each page holding exactly one 
picture. The first half of the file is for pictures that are requested in the source 
IIHEX file; the second half is for pictures requested in the fonts. 
The information on how the pictures should be named is recorded in a lg file; 
part of that information is also shown in the log file of TEX. For instance, in the 

 
%%page page_209                                                  <<<---3
 
188 
Translating I‘}'IEX to HTML using TEX4ht 
;...N..- 5cau\:O«v~4>wN>-‘ 
file try.html 
(/n/candy/tex/texmf/fonts/tfm/public/cm/cmr10.tfm) 
(/n/soda/tex4ht.dir/cmr.htf) 
(/n/candy/tex/texmf/fonts/tfm/public/cm/cmbx10.tfm) 
Loading ‘cmr.htf’ for ‘cmbx10.htf’ 
(/n/soda/tex4ht.dir/cmr.htf) 
(/n/candy/tex/texmf/fonts/tfm/jknappen/ec/ecsl1000.tfm) 
--- warning --- Couldn’t find font ‘ecsl1000.htf’ (char codes: 0--255) 
[1 file try.css ] 
Execute script ‘try.lg’ 
\reffig{4-18}: Messages from running tex4ht 
Entering try.lg 
dvips -mode ibmvga -D 110 -f try.idv -pp 1 > tmp.ps 
convert -crop 0x0 -density 110x110 -transparent #FFFFFF tmp.ps try0x.gif 
cmsy10-2a.gif already in 
\reffig{4-19}: Messages from running t4ht 
log of \reffig{4-17}, the eighth line has the statement try. idv[1] (tryox . gif) . It 
says that the picture from the first page of try. idv will be stored in a file named 
tryOx. gif. 
The lg file may also contain style information for the document’s css file, user 
requests for calls to system functions, and other types of entries. The lg file may, 
therefore, be regarded as a script to record the actions that must be taken after the 
tex4ht utility completes its job. 
The t4ht script has to execute the contents of lg file. This may be done by 
using a system-dependent program for interpreting the script, or the script itself 
may be executed, if it is expressed in terms of a scripting language recognized by 
the current platform. 
4.6.5 From DVI to GIF 
TEX4ht does not provide tools for converting DVI code into bitmap form. It relies 
on external tools being available for the task. 
The second and third lines of \reffig{4-19} show a two-step conversion of a DVI 
picture into a bitmap GIF file. The first step uses the dvips [h>DVIPS] driver to 
convert the picture into an intermediate file in PostScript. The second step calls 
the convert program (part of ImageMagick, [=>IMAGEMAGICK]) to complete the 
task. 
The fourth line in the example says that the file cmsy10-2a. gif already exists; 
therefore, it is not created again. 

%==========210==========<<<---2
 
%%page page_210                                                  <<<---3
 
4.6 The inner workings of TEX4ht 
The dimensions of the pictures depend on the size of the source documents 
and,'in the case of picture symbols, on the sizes of the fonts in use. However, the 
dimensions and quality of the pictures may also depend on the settings chosen for 
the external utilities in use and, of course, on the display used to view the HTML 
s. 
4.6.6 A taste of the lg file 
In essence, the lg file is just a wish list written by the tex4ht program. Some of 
the entries originate in the program itself. Other entries are requests made in the 
source document with tex4ht playing the intermediate role of passing the requests 
into the file. 
Requests from the source document can be made with a command of the form 
\Needs {request} 
They will end up in the lg file embedded within an envelope. The envelope itself 
is configurable with the \Configure{Needs}{content} command, where the latter 
command should use the token #1 to refer to the parameter of \Needs. 
Consider the following ETEX code: 
\Needs{chmod 644 *.html} 
\Configure{Needs}{#1} 
\Needs{Say hello} 
VV1th the configuration 
\Configure{Needs}{l. \the\inputlineno\space--- needs --- #1 ---} 
the source code contributes the following two lines to the lg file: 
1. 12 --- needs --- "chmod 644 *.html" --Say hello 
The default t4ht program distributed with TEX4ht will interpret the format 
of the first line 
1. integer --- needs --- ``content'' --as a system call to the UNIX command chmod 644 * .htm1. On the other hand, it 
ignores the second line because the utility was not programmed to recognize the 
pattern of that line. 
The lg file starts with all the contributions made in the source document, typically from \Needs commands in style files. It then lists the contributions originating 
in the tex4ht utility. The two types of contributions are separated by a distinguishing line, put there by tex4ht, in the lg file. 
189 

 
%%page page_211                                                  <<<---3
 
190 
Translating ETEX to HTML using TEX4ht 
cmr O 127 ’ffl’ " 15 
’G’ ’1’ Gamma O . . . . . . . . . . . . . . . . . . . . . . . .. 
= = =1= Delta 1 ’a’ " 97 
’Q’ ’1’ Theta 2 ’b’ " 98 
’/\\’ ’1’ Lambda 3 . . . . . . . . . . . . . . . . . . . . . . . .. 
’E’ ’1’ Xi 4 ’&#34;’ " 125 
’TT’ ’1’ P1 5 ’~’ " 126 
. . . . . . . . . . . . . . . . . . . . . . . .. ’\168\’ " 127 
’ffi’ " 14 cmr 0 127 
\reffig{4-20}: Portions of a virtual hypertext font file cmr . htf 
The default patterns for the contributions of the tex4ht utility are determined 
at the time the utility is compiled for the platform in hand. These patterns can be 
overridden in the runtime control file. 
The lg files generally consist of requests to create bitmap pictures, explicit contributions to the CSS file, and font information that implicitly asks for contributions 
to the CSS file. 
4.6.7 The font control files 
Text in normal ETEX output often has many symbols coming from different fonts. 
These symbols are put in the source file by a character or macro pointing to a 
symbol in an actual font. Thus, when using the Computer Modern Roman 10pt 
font, character a in the input means “set character number 98 from font cmr10" 
and \Gamma means “set character number 1." Because TEX4ht is not in charge of 
rendering the symbols, it supplies content for each symbol, instead of getting the 
glyph from the font file (as a DVI driver would usually do). This content is specified 
in TEX4ht’s virtual hypertext font files and is used by the Web browser to put a real 
character on the screen. 
4.6.7.1 Using the files 
For a given ETEX font, tex4ht assumes a virtual hypertext font, the main filename 
of which is a subset of the ETEX font name. If more than one such file is available, 
the one with the longest name is assumed. Accordingly, tex4ht searches in turn 
for the cmr10.htf, cmr1.htf, cmr.htf, cm.htf, and c.htf files when it needs a 
virtual hypertext font for a ETEX font named cmr10. 
Each virtual hypertext font file starts and ends with identical identification lines 
that specify the font name, the character code of the first symbol, and the character 
code of the last symbol (\reffig{4-20}). 
For each character, the file has a line consisting of three fields: a text string, a 
class number, and a (possibly empty) comment. The first and second fields must be 
delimited with a single character; any delimiters can be used, but within any one 
line they must all be the same. 

 
%%page page_212                                                  <<<---3
 
4.6 The inner workings of 'I]3X4ht 
191 
The class is a number between 0 and 255, where an empty class field is treated 
as 0. An entry with an even-numbered class contributes the content of the first field 
to the symbol. An entry with an odd-numbered class requests that a bitmap picture 
for the symbol be used, with the first field contributing an alternative content for 
character-based browsers. 
From input like “a $\Gamma$", TEX4ht will request a bitmap rendering for \Gamma in a file named cmr10-O. gif and produce the output 
a <IMG SRC="cmr10-O.gif" ALT=``G''> in the HTML file. The cmr1O in the filename indicates the ETEX font name; the 0 indicates the character number in the 
font. 
The first field in the entry may refer directly to characters in a font by placing the corresponding character code between backslashes. On the other hand, a 
backslash character \ must be represented by a pair of backslash characters \\. 
The symbols <, >, and & should be represented, for instance, by the strings 
&1t ; , &gt ; , and &a.mp; , respectively. 
4.6.7.2 Configuring the fonts 
The content retrieved for the symbols from the virtual hypertext font tables is written into the HTML files in a format that is governed by the following command. 
It provides a seven-component template for the symbols of the specified class. The 
delimiter must be a character that does not appear in the components. 
\Conf igureflltf } {class} {delimiter} 
{parameterI } {parameter2} {parameter-3} 
{parameter-4} {parameter5} {pammeter6} 
{parameter7} 
For example, the ETEX code \textsc{a} produces <SMALL>A</SMALL> in HTML 
3.2 mode and <SPAN CLASS=``sma11-caps''>A</SPAN> in HTML 4.0 Transitional 
mode. The htf font provides the content A of class 4 for the character ‘a’ in either 
mode. In the first mode, the markup is due to the default configuration for symbols 
of class 4 set by the command 
\Configure{htf}{4}{+}{<SMALL>}{}{}{}{}{}{</SMALL>} 
The only difference in the second mode is in the default configuration 
\Configure{htf}‘[4}{+}{<SPAN CLASS="}{}{}{}{}{small-caps">}{</SPAN>} 
For a symbol whose class is an even number, the first parameter is printed literally. The second parameter should comply with the C language conventions, and, 
if it is not empty, it is used to output the font name. The third and fourth parameters are used in a similar manner for writing the font size and its magnification, 
respectively. The remaining parameters are written literally, where either the fifth 

 
%%page page_213                                                  <<<---3
 
192 
Translating I1}'I]3‘X to HTML using 'I]3X4ht 
or the sixth parameter must be empty. The string contributed from the htf file is 
introduced just before the last parameter. 
Symbols of odd classes use the parameters in a similar manner to output the 
font name, the alternate string from the htf font, a second copy of the font name, 
the font size, the font magnification when it differs from 100%, and the character 
code. The configuration for class 0 is also used to provide extra markup to symbols 
of the other classes. 
The \NoFonts and \EndNoFonts commands suspend and resume, respectively, 
the contributions of the \Configure{htf} command. 
4.6.7.3 Adding style 
Contributions of htf fonts to the CSS file can be configured with the commands 
\Conf igure{htf -css}{cla.v5}{content} 
\Conf igure{htf -css}{fimtmzme}{attril2ute5} 
The command 
\Configure{htf-css}{4}{.small-caps {font-variant: small-caps; }} 
contributes . small-caps{font -variant: small-caps; } for symbols of class 4. 
On the other hand, the command 
\Configure{htf-css}{cmmi}{font-style: italic;} 
results in contributions like .cmmi-7{ font-size:70%; font-style: ita1ic;} 
and .cmmi-10{ font-style: italic;}. 
4.6.7.4 Font clues 
Existing virtual hypertext fonts may be redesigned by users to obtain alternative 
output; new ones may be produced to accommodate missing fonts. If no matching 
htf can be found, tex4ht will issue warning messages such as 
---warning --- Couldn't find font ‘fontname.htf’ 
(char codes: first--last) 
until the new fonts are provided. 
VV1th the package option ShowFont, source code like 
\font\x=fontname\ShowFont\x 
produces a picture showing the normal result for the different symbols in the given 
font. 

 
%%page page_214                                                  <<<---3
 
4.6 The inner workings of 'IEX4ht 
193 
4.6.8 The control file 
The task of the runtime control file is to allow the tex4ht utility to adjust itself 
to the platform on which it runs and to the needs of its users without having to 
be recompiled. The file is called either tex4ht . env or .tex4ht, and it might have 
more than one copy in a given installation. For instance (in order of priority), one 
copy may reside in a directory indicated with the -e option of the command line 
(see \refsec{4_6_3_Running_the_tex4ht_program} on page 186). A second file may reside in the working directory, 
and a third one may be in a directory whose location is hard coded within the 
program. 
The file itself is made up of entries identified by the first character in each line. 
The following are some of the possible options: 
t Identifies a directory to be searched for the font metric (tfm) files of TEX. 
i Identifies a directory to be searched for the virtual hypertext fonts (htf) of 
TEX4ht. 
a Different fonts of ETEX may consist of identical sets of symbols that vary just 
in size or style. Such fonts would translate to identical virtual hypertext fonts, 
so the “a" character introduces font aliases. 
g Identifies the extension name given to bitmap files. Currently such bitmap files 
are used only for picture symbols in virtual hypertext fonts. 
Consider the following control file (the line numbers are not part of the file): 
t/n/candy/tex/texmf/fonts/tfm/! 
i/n/soda/tex4ht .dir/ 
1/n/soda/tex4ht .dir/ht-fonts/! 
acmbx cmr 
acmsl cmr 
8-J'P8 
o«u..Aw~._ 
The first line points to a directory to search for font metric files that are not 
available in the current directory; the exclamation mark ! indicates that the search 
should extend to subdirectories of all depths. 
The second and third lines specify directories to locate htf fonts, where recursive searching into subdirectories is allowed within the directory listed in the third 
line. 
The fourth line states that requests for cmbx fonts should use the htf file 
cmr.htf. The fifth line is ignored because it starts with a blank character; this 
character is not within the options available for entries of the control file. 
The sixth line requests an extension name of jpg, instead of the default extension gif, for the bitmap files of the picture symbols. 

 
%%page page_215                                                  <<<---3
 
194 
Translating PXIEX to HTML using TEX4ht 
Summary 
This chapter has shown how TEX4ht can be used to translate ETEX documents 
into a HTML files, with a very extensive set of facilities to configure the results. 
The strength of this system is that it uses ETEX itself to read the file, permitting a 
far greater range of BTEX constructs (such as complex macros) to be handled than 
most other translators. 
Because much TEX4ht’s work is done by hooks in a ETEX style file, it is relatively easy to change it to generate different markup. In Appendix B.2 on page 404 
we look at how to make the system generate XML, and we give some concrete examples of a ETEX to XML translator, including MathML, in 8.2.3.2 on page 382. 

 
%%page page_216                                                  <<<---3
 
CHAPTER 5 
Direct display of ETEX 
on the Web 
In this chapter we discuss some applications that take ETEX input and render it 
directly within a Web browser; these are typically browser plug-ins or Java applets. 
The software packages are not based on the “real" TEX source code, so they do 
not use METRFONT fonts or DVI files. None of the applications currently renders 
all ETEX features, although you should check the product Web sites for news and 
updates. 
The most widely used browser plug-in for rendering ETEX input is IBM’s 
techexplorer Hypermedia Browser. This plug-in augments Netscape Navigator 
and Microsoft Internet Explorer and is available for several platforms. It renders 
a large amount of nonmath ETEX markup and is suitable for both showing math 
within an HTML document and for displaying full documents. Most of this chapter discusses how you can augment your ETEX documents for optimal interactive 
rendering within techexplorer. 
WebEQ is a widely usedjava applet for rendering math within a browser. Since 
it is a]ava applet, it works more or less automatically on several platforms. WebEQ 
offers rich functionality for displaying math, but it is limited in what it can do for 
text. You would use this applet most frequently within an HTML page. 
When using either of these programs remember that the resolution of a typical 
computer monitor is significantly less that that of a printed page. This means that 
relative sizes of elements such as base expressions and subscripts may be different 
from what you are used to seeing on a page. Also, the lower resolution may cause 
rule widths to vary, depending on where they appear on the screen. 

 
%%page page_217                                                  <<<---3
 
196 
Direct display of IIHEX on the Web 
There are several other practical problems for using any of these products for 
displaying math within HTML documents. In the final section of this chapter we 
examine these problems and discuss how future browsers and rendering applications will improve the current situation. We also describe the ways in which these 
applications might communicate with other programs to provide truly interactive 
scientific documents. 
5.1 IBM techexplorer Hypermedia Browser 
IBM’s techexplorer Hypermedia Browser‘ is a browser plug-in for Netscape 
Navigator and Microsoft Internet Explorer. IBM’s techexplorer directly renders 
a subset of TEX and IéTEX and can be used to display mathematical expressions 
within an HTML page or to display full documents within the browser window. 
\reffig{5} .1 displays two sections of our test document using techexplorer (see 
Section A.1 on page 391) directly. In the following sections, we will see some of the 
ways in which we can make more active and colorful documents. 
techexplorer attempts to be more than a renderer of scientific markup. The 
Introductory Edition (available for download from [=>TXPL]) displays documents but 
also adds features like hypertext links, GIF and JPEG images, user-defined menus, 
and hierarchical document navigation. The Professional Edition (which you have to 
purchase from IBM) builds on the features in the free Introductory Edition and adds 
support for printing, searching, inline video, ajava/]avaScript programming interface, and an add-in architecture for allowing techexplorer to communicate with 
other applications. Currently techexplorer is being enhanced by adding support 
for the Mathematical Markup Language (see Section 8.1 on page 368) 
techexplorer works well with browser frames and thus can be used to build 
sophisticated sites that combine HTML pages, techexplorer documents in windows, and Java applets that dynamically update the techexplorer documents and 
respond to user events within their windows. 
Because techexplorer is a plug-in, you need to get an appropriate version 
for the operating system that you are using. Versions are available for Microsoft 
Windows 9X and Windows NT, in addition to several UNIX flavors, including 
IBM AIX, Sun Solaris, SGI IRIX, and Linux. Check the Web site [=>TXPL] for 
updates and news about versions for other platforms. The following discussion of 
techexplorer is based on Version 2.0. 
The full documentation for techexplorer is shipped with the product and 
is also available from the Web site. We will not reproduce the reference material 
in the documentation here because techexplorer is continuing to evolve and, in 
particular, the supported subset of TEX and I£§TEX expands with each new release. 
The Compatibility section of the online documentation describes the features 
that techexplorer provides from TEX and BTEX. The Creating documents secltechexplorer Hypermedia Browser is a trademark of IBM Corporation. 

 
%%page page_218                                                  <<<---3
 
5.1 IBM techexplorer Hypermedia Browser 197 
Simulation of Energy Loss Straggling 
Maria Physicist 
Introduction 
Due to the statistical nature of ionisation energy loss, large fluctuations can occur in the 
amount of energy deposited by a particle traversing an absorber element Continuous 
processes such as multiple scattering and energy loss play a relevant role in the longitudinal 
and lateral development of electromagnetic and hadronic showers, and in the case of sampling 
calorimeters the measured resolution canbe significantly affected by such fluctuations in their 
active layers. The description of ionisation fluctuations is characterised by the significance 
parameter K, which is proportional to the ratio of mean energy loss to the maximum allowed 
energy transfer in a single collision with an atomic electron 
§ 
‘‘ = 7.3 
Em, is the maximum transferable energy in a single collision with an atomic electron. 
iwfizrz 
Emu: = 2» 
l+2ym,Im,+(m,Imx) 
where y = Elm," E is energy and m, the mass of the incident particle, [32 = 11/~/7 and m, 
is the electron mass F, comes from the Rutherford scattering cross section and is defined as: 
E _ 21:2: ‘ivzp& 
= 153.4‘2Tfip&x keV, 
F 
29.63 
20 I778 500 31.01 
50 24.24 1000 31.50 
100 27.59 00 32.00 
special sampling for lower part of the spectrum 
Ifthe step length is very small ( S 5 mm in gases, 5 2-3 pm in solids) the model gives 0 
energy loss for some events. To avoid this, the probability of 0 energy loss is computed 
HA5 = 0) = e-(<n,>+<n;>+<n3>) 
Ifthe probability is bigger than 0.01 a special sampling is done, taking into account the fact 
that in these cases the projectile interacts only with the outer electrons of the atom. An energy 
level E0 = 10 eV is chosento correspond to the outer electrons. The mean number of 
collisions can be calculated from 
i dl‘ 
<"> = WA" 
The number of collisions n is sampled from Poisson distribution. In the case of the thin layers, 
all the collisions are considered as ionisations and the energy loss is computed as 
71 
A5 = Z_é"D__ 
. 1 _ E5" 
[-1 ~jHrEmu+ED 
[1] L.I.andau. On the Euflgy Loss of Fast Particles by lonisalion. Orignaliy published in)’. Phys. , 8‘2Ul , 
1944. Reprinted in D. ta’ Hat, Editor, L D. Landau, Cbllecmdpqpers, page 417. Pergamon Press, 
Oxford, 1 9 6 5 . 
[2] Bischorr. Programs for the Lama» and the Vavilov distributions and the corresponding random 
numhas. Cbmp. Phys Clomm.,7:2lfi, 1974. 
S.M.Selmer and M.J.Berger. Energy loss stra ' ofprotons and mesons. In .ShJa'r'es In Penetration of 
_ -1 . in . W» . ' .....4 ,; .:_c....'..m .uh.i..'...¢ no N 
...'... an.. M. A 
A 
\reffig{5-1}: Two examples of techexplorer displaying text and mathematics 

 
%%page page_219                                                  <<<---3
 
198 
Direct display of ETEX on the Web 
tion of the online documentation describes the techexp1orer-specific features that 
you can use to create interactive scientific and technical documents. To facilitate 
your creating electronic documents that can be viewed on the screen as well as 
printed using BTEX, the developers have provided a style file that implements most 
of the new commands that are defined by techexplorer. See the online documentation to get this style file and to read about what it can do. 
In the following sections we describe the philosophy of techexplorer and 
discuss how you can use the techexplorer extensions to TEX and I6TEX to make 
online documents that are superior to simple electronic renditions of print documents. 
5.1.1 Basic formatting issues 
The techexplorer program emulatey TEX and I5TEX and does not use the real 
TEX program. Some basic TEX features such as category codes are not supported, 
and the major emphasis is on providing the standard macros and environments 
from IéTEX. Style files are not supported, but input files and simple plain TEX \def 
macros can be used. 
While techexplorer supports many commands from TEX and IéTEX, it sometimes accepts commands but then does nothing with them. Similarly it may support 
a subset of the functionality provided by a I6TEX environment. Commands or symbols that techexplorer does not understand at all are displayed in red within the 
text.Z The techexplorer documentation at the Web site contains an up-to-date 
listing of the supported symbols, commands, and environments. 
Release 2.0 of the Professional Edition of techexplorer supports the following commands: 
$ $$ -- --- \! \$ \& \> \, \/ \: \; \{ \} \[ \] \( \) \" \_ \\ \| “ _ 
\acute \addtocounter \Alph \alph \arabic \arccos \arcsin \arctan \arg 
\atop \author \begingroup \bf \bgroup \bibitem \big \big1 \bigm \bigr 
\Bigl \Big \Bigm \Bigr \bigg \biggl \biggm \biggr \Bigg \Bigg1 \Biggm 
\Biggr \bigskip \Bmatrix \bmatrix \bmod \bo1d \boxed \break \caption 
\cases \cdots \centering \centerline \cfrac \chapter \choose \cite \colon 
\color \colorbox \cos \cosh \cot \coth \csc \csch \date \ddot \ddots 
\ddotsb \ddotsc \ddotsi \ddotsm \def \deg \det \dfrac \dim \displaylines 
\displaystyle \dot \egroup \em \emph \endgroup \enskip \enspace 
\ensuredisplaymath \ensuremath \eqno \erf \errmessage \exp \fbox 
\fcolorbox \fnsymbol \footnotesize \frac \framebox \gcd \grave \H \hat 
\hbox \hfil \hfill \hfi111 \hline \hom \hphantom \hru1e \hsize \hskip 
\hspace \hspace* \hss \Huge \huge \idotsint \iff \ifmmode \iint \iiint 
\iiiint \imp1iedby \imp1ies \includegraphics \index \inf \it \kern 
\joinrel \ker \label \LARGE \Large \1arge \LaTeX \lbrace \lbrack \lcm 
\ldots \left \leftline \1eqno \1g \1im \liminf \limsup \1lap \ln \1og 
2Unlike HTML browsers that ignore markup they do not understand, techexplorer displays 
markup that is not understood in order to make it easier for authors to debug their documents. Thus 
techexplorer will always display all the content of a mathematical expression instead of mysteriously 
omitting part of it. 

%==========220==========<<<---2
 
%%page page_220                                                  <<<---3
 
5.1 IBM techexplorer Hypermedia Browser 199 
\lower \lowercase \lVert \lvert \makebox \maketitle \mathbb \mathbf 
\mathbin \mathca1 \mathchoice \mathc1ose \mathit \mathop \mathop 
\mathopen \mathord \mathre1 \mathsf \mathstrut \mathtt \matrix \max \mbox 
\medbreak \medspace \medskip \min \mit \negmedspace \negthinspace 
\newcommand \newcounter \newenvironment \newline \normalsize \not \notin 
\null \operatorname \over \overbrace \overbracket \overline 
\overrightarrow \overset \pagecolor \par \paragraph \parbox \part 
\phantom \pmatrix \mod \pod \Pr \prime \providecommand \qed \qedsymbol 
\qquad \quad \ragged1eft \quote \raise \raisebox \rbrace \rbrack 
\refstepcounter \re1ax \renewcommand \renewenvironment \right \rightline 
\r1ap \rm \root \Roman \roman \rule \rVert \rvert \sb \sc 
\scriptscriptsize \scriptscriptstyle \scriptsize \scriptstyle \shadowbox 
\sec \sech \section \setstyle \sf \sin \sinh \s1 \small \smallskip 
\smallmatrix \smash \sp \space \sqrt \stackrel \stepcounter \strut 
\subparagraph \subsection \subsubsection \sup \tan \tanh \TeX \text 
\textbf \textcolor \textit \textrm \textsf \textsl \textstyle \texttt 
\tfrac \thanks \thebib1iography \thechapter \theenumi \theenumii 
\theenumiii \theenumiv \theequation \thefigure \thefootnote 
\thempfootnote \thepage \theparagraph \thepart \thesection 
\thesubparagraph \thesubsection \thesubsubsection \thetab1e \thickspace 
\thinspace \tilde \tiny \title \today \tt \underbar \underbrace 
\underbracket \underline \underset \uppercase \va1ue \vbox \vdots \verb 
\Vmatrix \vmatrix \vphantom \vrule \vskip \vss \vtop \widehat \widetilde 
\2ag \2ig 
The following commands are accepted but ignored: 
\® \- \addcontentsline \addtocontents \addtolength \a11owbreak \and 
\bibliographystyle \boldmath \break \brokenpenalty \bye \cleardoublepage 
\clearpage \cline \clubpenalty \DeclareMathUperator \definecolor 
\displaywidowpenalty \documentclass \documentstyle \eject \end 
\floatingpenalty \font \fontencoding \fontfamily \fontseries \fontshape 
\fontsize \footnotemark \frenchspacing \fussy \goodbreak \hline 
\hyphenation \indent \interlinepenalty \1et \limits \linebreak \long 
\looseness \markboth \markright \multicolumn \newblock \newif \newlength 
\newpage \newtheorem \noalign \nobreak \nocite \nocorr \nofrenchspacing 
\noindent \nolimits \nolinebreak \nomargins \nonumber \nopagebreak 
\nopagenumbers \nu11 \pagebreak \pagenumbering \pagestyle 
\postdisplaypenalty \predisplaypena1ty \protect \putat \raggedbottom 
\relax \rgb \selectfont \setlength \settodepth \settoheight \settowidth 
\singlespace \sloppy \special \thispagestyle \typeout \unboldmath 
\usefont \usepackage \widowpenalty 
The following environments are at least partially supported: 
align align* abstract array Bmatrix bmatrix center description 
displaymath document enumerate eqnarray eqnarray* equation figure 
flushleft flushright gather gather* Huge huge itemize LARGE Large large 
math matrix minipage normalsize pmatrix quotation quote slide small 
smallmatrix tabbing table tabular tiny titlepage verbatim Vmatrix vmatrix 
For simplicity, when we discuss techexplorer markup support in the following exposition, we will refer to the “ETEX markup" rather than the longer, but 
more precise phrase, “TEX and BTEX markup subset." 

 
%%page page_221                                                  <<<---3
 
200 
Direct display of I§TEX on the Web 
The BTEX \newcommand and \newenviromnent commands can be used, but 
the optional argument is not currently permitted. Note that macro definitions 
are global to the document in which they are defined, unless the \gdef or the 
\g1oba1newcommand commands are used. In this latter case, the macro definitions 
are available to all documents currently in memory. To reiterate: techexplorer 
does not allow macro definitions to be local to the groups in which they are defined. They are either global within the defining document or within all active 
documents. Multiple active documents might occur if you use techexplorer to 
display several math expressions within an HTML document or if you use browser 
frames and have techexplorer documents in more than one frame. 
On the Microsoft VVindows 95 and VV1ndows NT platforms, techexplorer 
uses TrueType fonts. The Professional Edition of techexplorer provides a set 
of symbol fonts derived from the BlueSky and Y&Y PostScript renditions of the 
Computer Modern, BTEX, and AMS Symbol fonts. Similarly on UNIX platforms 
techexplorer uses PostScript versions of the symbol fonts, but it does not use 
METH FONT fonts. 
techexplorer does not process DVI files but rather reads and renders the document directly. This allows techexplorer to display BTEX markup that is dynamically generated by a]ava applet or another application. 
Page layout is performed with respect to the size of the techexplorer window, 
rather than by using style parameters in the document. For example, paragraphs 
are normally flowed to the width of the window. If you change the window size, 
techexplorer will try to reflow your document to fit the window.3 
Since all composition is done within your browser, techexplorer offers many 
ways of customizing the display environment. For example, you can set the standard 
fonts used and the foreground and background colors of the text. The default color 
for links is blue, but you can change it to another color if you prefer. \reffig{5-2} 
shows the standard property page for setting your color choices. You access the 
options property pages by clicking your mouse’s right button on an area of whitespace in a techexplorer document window and then choosing Options... from 
the menu displayed. 
5.1.2 Your browser and techexplorer 
Your Web browser can use techexplorer in one of two ways: 
0 to display full BTEX documents within the full client area of the browser, and 
0 to display BTEX markup in one or more windows within an HTML page. 
Version 2.0 of the Professional Edition of techexplorer supports the printing of 
full documents but not the markup in windows within an HTML page. 
3Note that in some cases the Web browser never tells techexplorer that the window has changed 
sizes, so the document is not recomposed. This happens most frequently when frames are involved. 

 
%%page page_222                                                  <<<---3
 
5.1 IBM techexplorer Hypermedia Browser 
201 
\reffig{5-2}: Customizing the colors used to display your documents 
5.1.2.1 Displaying full documents 
VVhen your documents contain nontrivial math expressions within sentences (as opposed to between paragraphs), techexplorer will produce a better looking document than an HTML page that has math in images, plug-in, or]ava applet windows. 
This is because techexplorer can better position and size the math expressions in 
relation to the surrounding text. Moreover, the formatting style and page background will be consistent across the text and the math. 
Since most Web browsers and servers are preconfigured to understand that files 
with the extension .tex also have MIME type application/x-tex, techexplorer 
will automatically be invoked for your BTEX documents if their URLs end in .tex. 
You can open your IATEX documents Via hypertext links in HTML or in other 
techexplorer documents or Via the usual browser methods for specifying URLs 
to open.4 
You can use techexplorer documents within browser frames in the same way 
that you include HTML documents. However, if you resize the browser window 
you will most likely have to reload the techexplorer documents to have them 
composed at the correct screen width (this is a shortcoming of the browser, not of 
techexplorer). 
4At the time this section was written, opening local techexplorer files from file lists within the 
browser worked more reliably in Netscape Navigator than in Microsoft Internet Explorer. You may 
need to type explicitly in the full local file name for Internet Explorer. 

 
%%page page_223                                                  <<<---3
 
202 
Direct display of IHFX on the Web 
5.1.2.2 Displaying math within HTML pages 
You can use the HTML EMBED element to include BTEX markup to be rendered by 
techexplorer within an HTML page. \reffig{5-3} shows a commutative diagram 
sandwiched between some HTML text. The HTML source for the page is 
<HTML> 
<BODY> 
This is an example of a commutative diagram 
from an algebraic geometry article. 
<CENTER> 
<EMBED SRC="excomm.tex" TYPE="application/x-tex" 
HEIGHT=110 wIDTH=400 NAME=“comm-diagram“> 
</CENTER> 
The text above and below the diagram are part of 
the HTML page. The diagram itself is rendered by 
<STRONG>techexplorer</STRONG> via an EMBED element. 
</BODY> 
</HTML> 
There are six important attributes of the EMBED element for techexplorer: 
SRC the URL of the document containing your BTEX markup. 
TYPE the MIME type of the data contained in your document. 
HEIGHT the height in pixels of the rectangle in which the markup should be rendered. 
WIDTH the width in pixels of the rectangle in which the markup should be rendered. 
ALIGN Use ALIGN=MIDDLE when you want the expression to float vertically so it 
looks better with respect to the baseline of the surrounding text. Simple expressions like 2: - 1 do not need this element, but more complex forms like fractions 
(g) and matrices ([ :I' i :|) will look better with this attribute setting. 
NAME a unique name that identifies the particular embedded techexplorer window. This is important if you use the techexplorer Scripting Interface to 
work with the techexplorer markup and the events that are generated within 
the window. You can omit the name if you are interested only in rendering the 
expression. 
Since the file extension of our example embedded BTEX document is .tex, the 
TYPE attribute is probably redundant. We recommend you use it anyway. 
If you choose a height or width that is too small, your techexplorer window will contain vertical or horizontal scrollbars, respectively. Because users can 

 
%%page page_224                                                  <<<---3
 
5.1 IBM techexplorer Hypermedia Browser 203 
§ This is an example of a commutative diagram from an algebraic geometry article. 
7=1(5P=c (Ll?) " 7l1(E.J7) 3 7lf“"‘(E) 
H lm lm 
n1(spec(L),;":) -) n1(U®L,i) ., nf*“*(U®L) 
E The text above and below the diagram are part ofthe HTMI. page. The diagram itself 
:' is rendered by techexplorer via an EMBED element. 
\reffig{5-3}: A techexplorer commutative diagram embedded within an HTML 
Page 
choose their own font sizes Via the fonts options property page, it is impossible 
to know ahead of time if your window size will be sufficient for everyone. Also, 
techexplorer includes a narrow margin around the rendered markup, so the 
window you need might be slightly larger than you expect. You might try using 
WIDTH="100%" to have your expression fill the full width of the HTML page. 
For browsers that are plug~in compatible with Netscape Navigator, you can 
include ETEX data in the EMBED element itself rather than point to an external file. Use the TEXDATA attribute with the markup as value along with 
TYPE="application/x-techexp1orer" to tell the browser that techexplorer 
should process the data. For example, 
<EMBED TYPE="application/x-techexp1orer" 
TEXD1-\.TA="\[\pmatrix{2&3&4&5\cr 6&7&8&9\cr -18:-28:-38:-4}\]" 
WIDTH=200 HEIGHT=90 ALIGN=MIDDLE> 
2 3 4 5 
(6 7 8 9) 
-1 -2 -3 -4 
at the location of the EMBED element. 
Here is another example (shown in \reffig{5-4}, using Microsoft Internet Explorer) that contains two math expressions. The first is within a sentence, and the 
other is in display mode. 
displays the 3 X 4 matrix 
<HTML> 
<HEAD> 

 
%%page page_225                                                  <<<---3
 
204 
Direct display of ETEX on the Web 
HTML with embedded lechexpliilei math - MI-smsnfl lnlemel Explmel ' 
E3D"3“t@E@EE'e*W¢ °si**9mE1 VVVVVVVVVVVVVVVVVVVV . 
The answer to Question I is J; and the answer to 
g 2 
Question 2 is _ 
4 
\reffig{5-4}: Two techexplorer expressions embedded within an HTML page 
<TITLE>HTML with embedded techexplorer math</TITLE> 
</HEAD> 
<BODY> 
<P ALIGN=CENTER> 
The answer to Question 1 is 
<EMBED TYPE="application/x-techexp1orer" 
TEXDATA="\pagecolor{white}$\frac{x}{y}$" 
ALIGN=MIDDLE 
WIDTH=33 HEIGHT=45> 
and the answer to<BR>Question 2 is 
<EMBED TYPE="application/x-techexplorer" 
TEXDATA="\pagecolor{white}$\pmatrix{2&3\cr 4&\frac{5}{6}}$." 
ALIGN=MIDDLE 
WIDTH=114 HEIGHT=78> 
</P> 
</BODY> 
</HTML> 
5.1.3 Adding hypertext links 
One of the primary advantages of electronic documents over paper ones is the 
possibility of having hypertext links. The proliferation of HTML Web sites and 
browsers has made hypertext a basic requirement for any application that displays 
documents interactively. 
When running under Microsoft VV1ndows 95, techexplorer has a maximum 
document length that corresponds roughly to 50 printed pages. For this reason, 
you will need to break longer documents into smaller ones and add hypertext links 
between them. These links can be placed directly in the text or in pop-up menus 
(see Section 5.1.6 on page 211). 

 
%%page page_226                                                  <<<---3
 
5.1 IBM techexplorer Hypermedia Browser 
205 
The general-purpose techexplorer command for creating a hypertext link is 
\docLink. 
\do CL ink [fra7neNa7ne] {url} [label] {expression} 
The two required arguments are url and expression. The expression is what you see 
on the screen, and url is the address of the document to which the browser will 
jump when the reader clicks on expression. The url is read in a special mode so that 
it can contain characters such as backslashes. 
Note that text does not have to be plain text: It can be anything that 
techexplorer can display, such as words, images, and mathematical expressions. 
As mentioned earlier, text will be displayed in blue, by default. If you want a specific 
hyperlink to be displayed in a given color, use \color within text. For example, 
This \docLink{Hyperlink.tex}{\color{black}hyperlink} 
might be indistinguishable from the surrounding text. 
If the surrounding text color was black, the word “hyperlink" in the example sentence would not stand out as something special. However, the mouse cursor will 
still change to a hand, and the status line will display the target for the link, that is, 
Hyperlink.tex. 
Unlike the default formatting in browsers, hyperlinks in techexplorer are not 
underlined; it does not make sense to underline links within mathematical expressions. Indeed, underlining in a math expression often has semantic significance. 
The url is passed to the Web browser, even if the document is another BTEX 
file for techexplorer to process. For such a IISIEX file, use the label optional argument to give a position in the document to which techexplorer should scroll 
when the document is displayed. If label cannot be found, the window is positioned 
at the top of the document. 
Examples 
9 Use the normal URL syntax for a Web site address. 
The \docLink{http://www.tug.org}{\TeX{} User’s Group} Web site 
contains much useful information about \LaTeX{}. 
o You can refer to local file names, but remember that they are not portable across 
operating systems, and they may not be the same on every user’s computer. 
The \docLink{c:\classes\m101\probset1.tex}{first problem set} 
contains instructions for how to submit your homework. 

 
%%page page_227                                                  <<<---3
 
206 
Direct display of I1}'I]§‘X on the Web 
First section E This is the first section. Click here to load the second 
second section in this frame. 
51 3 I‘(x) 
sec on I 2 #1 dx 
\reffig{5-5}: Two frames, each containing a techexplorer window 
o You can use relative addressing to link to files in the same directory as the current file. Here we jump to the label lecture2 in the reading. tex ETEX source 
file. This happens to be in the same directory as the document we are viewing. 
The reading assignment before the \docLink{reading.tex} 
[lecture2] {second lecture} will require about 30 hours. 
o Links don’t actually have to link to files. Here we have a mailto link so that 
the reader’s mail program is invoked when the user clicks on the link. 
If you would like, you can send mail to the techexplorer 
\doc:Link{mailto : techexpl@watson . ibm. com}{developers} . 
The ETEX documents that your browser displays via techexplorer are included in the general browser history and, in particular, can be accessed via “back" 
and “forward" navigation. While the browser saves information about the scroll 
position in the HTML documents it displays, techexplorer does not currently 
do this. Part of the problem is that techexplorer is usually unloaded completely 
and then reloaded between ETEX documents. Therefore it cannot save the current 
scroll state in memory. This may be remedied in a future release. 
If fi'ameName is specified and the current techexplorer document is included 
in an HTML frameset, the new document is displayed in the frame with the name 
fi'ameName. If you leave out fiwmeName, the new document is displayed in the current frame. 
The following example illustrates two frames, each containing a techexplorer 
window. The left frame “toc" contains a table of contents and the right frame 
“body" contains the sections. \reffig{5-5} is a screen shot of the table of contents 
and the first section. Following is the HTML frameset definition: 
<HTML> 
<HEAD> 

 
%%page page_228                                                  <<<---3
 
5.1 
IBM techexplorer Hypermedia Browser 207 
<TITLE> 
Frame test for techexplorer plug-in 
</TITLE> 
</HEAD> 
<FRAMESET CDLS="150,*"> 
<FRAME MARGINwIDTH=``4'' SRC=“toc.tex" NAME=``toc''> 
<FRAME MARGINwIDTH="4“ SRC="first.tex" NAME=``body''> 
</FRAMESET> 
</HTML> 
In the ETEX file for the table contents, the \docLinks indicate that the section file 
should be displayed in the “body" frame on the right. 
Z toc.tex 
\begin{itemize} 
\item \docLink[body]{first.tex}{First section} 
\item \docLink[body]{second.tex}{Second section} 
\end{itemize} 
The first section contains a link to the second. The frame name is not required 
because we want to open all sections in the same frame. 
Z first.tex 
This is the first section. Click here to load the 
\docLink{second.tex}{second section} in this frame. 
\[ 
\int_2“3 \, \frac{\Gamma(x)}{x - 1} \mathrm{d}x 
\] 
Similarly we don’t include the frame name in the following link to the first section. 
X second.tex 
This is the second section. Click here to load the 
\docLink{first.tex}{first section} in this frame. 
$$ 
\bmatrix{ 
1 & 0 & 0 \cr 
0 & 1 & 0 \cr 
0 & 0 & 1 
} 
$$ 

 
%%page page_229                                                  <<<---3
 
208 
Direct display of I1}'I]§X on the Web 
\labelLink{label} {text} 
When a hypertext link is to a target elsewhere in the same document, use 
\labelLink instead of \docLink. 
\docLink will work, but it’s overkill. You need to specify the document URL, 
and there might be a document maintenance problem if you decide to rename your 
files. 
The following example will create a local hyperlink to the last section of the 
current document: 
In the \labelLink{c5:final-section}{final section} we examine 
these problems and discuss how future browsers and rendering 
applications will improve upon the current situation. 
For both \docLink and \labelLink the text argument can contain other links. 
The rule for executing links is that the innermost link is the one that takes precedence. Consider the following: 
All \labelLink{label:red-phones}{red 
\labelLink{label:phones}{phones}} 
come with a 30 day money-back guarantee. 
When the mouse cursor is over “phones," the label zphones link will execute when 
the mouse button is clicked. When the cursor is over “red," the label : red-phones 
link will be the one that is executed. 
5.1.4 Popping up windows and footnotes 
When a document is displayed by techexplorer on the screen, it is not broken into pages. It is very inconvenient to scroll to the end of long documents to 
see footnotes. Furthermore, footnotes in math expressions within HTML pages 
don’t have access to the “end of the document." Therefore techexplorer supports 
\footnote by popping up a new window when you click on the footnote number 
in the text. \reffig{5-6} shows a footnote in a techexplorer document. 
\popupLink{window7ext}{caption}{text} 
A footnote is an example of a pop-up window link. You create pop-up window links 
by using \popupLink. 
The text is displayed in the document. When you click on text, a pop-up window containing windouflext is displayed. The window has caption displayed in its 
titlebar. When the mouse cursor passes over text, the caption is displayed on the 
browser status line. 

%==========230==========<<<---2
 
%%page page_230                                                  <<<---3
 
5.1 IBM techexplorer Hypermedia Browser 
209 
file:///eZ?C/techsiipl/documentation/E iiarnples/S utor82/geomirr. (ex 
Suppose that F is a geometrically irredLicible__lfl_l_i_sse §he_a__f: on a non-empty open 
subset U of P 3: with associated n - dimensi 1 ‘ ‘ ' 
either Letfbe a lisse sheaf on a open 
1p is E19-irreducible. Or subset UofP;i. fis said to be 
2. p is induced from a representation of c geometrically irreducible if P}-is 
_ _ an irreducible representation of 
3. there exists an integer d 2 2 dividing 71' S°°m( U ), or equivalently, if 
1 
d t 83 h ' I." -' d . . . . 
Pm M T w’ W ereflsa 1e ‘We u Ggem( F) acts irreducibly in its 
dimension n I d, and Li) is an irreducib; glven» remesentatlon 
dimension d which factors through a 
pair (-1-, Lu) is iiniqiie up to replacing it by (-183 x, Li) ®x '1 ) with X a character of 
finite order. 
\reffig{5-6}: A pop-up footnote in techexplorer 
Note that caption will be simplified so that it can be displayed using the single 
titlebar font. In particular, a math expression in caption will not be displayed in a 
two-dimensional format. In the following example, 
More information about this interesting expression is available 
\popupLink{Since $\alpha = 1$, the expression is simply 
$ (X-1) (x+1) $ . }{$\frac{x"2-\alpha}{\alpha}$}{here}. 
More information about this interesting expression 
i§_?"%ll§l?l,?? has-x ~ ~ 
[x“2-aipha 1 1 [alpha 1 
Since a = 1, the expression is simply 
(x-1)(x+ 1). 
the text displayed in the titlebar and status line is (x"2-alpha) / (alpha). 
The text in the pop-up window is fully formatted and can contain mathematics. 
Links within the window do not work yet, although they are displayed using the 
link color. If you include an image, sound, or video within the pop-up window text, 
make sure you use an absolute URL for the source location. 

 
%%page page_231                                                  <<<---3
 
210 
Direct display of I1}'I]§,X on the Web 
5.1.5 Using images, sound, and video 
techexplorer supports multimedia via images, sounds, and video. The Microsoft 
Windows versions have the most support, while the UNIX editions only support 
images. Since multimedia files can be large, there may be a delay before the image is 
rendered or before the audio or video file starts playing if the file has to be retrieved 
across the network. 
\includegraphics [lowerLeft] [lowerRig/Jt] {url} 
To include an image, use \includegraphics. This offers the most basic support 
from the ETEX graphics package. 
VVhile the lowerLeft and lowerRz'g/2t optional arguments are accepted, they are 
ignored. If the image addressed by url does not exist or has not yet been retrieved 
from the network, a “missing image" substitute is displayed in its place. 
You can use GIF and JPEG image types on any operating system platform supported by techexplorer. The file extension on url for GIF images should be . gif. 
For JPEG files, the images can have the file extensions .jpg, .jpeg, or .jpe. The 
Microsoft Windows versions also support 
VV1ndows bitmap files with file extensions . bmp and .dib; 
PCX images with file extension .pcx; 
o Targa images with file extension .tga; and 
TIFF images with file extensions .tif or .tif f. 
We recommend that you use GIF and JPEG images for maximum portability. Animated and transparent images are not supported on any platform yet; neither are 
Encapsulated PostScript files. When you print a techexplorer document, each 
image is scaled so that its size, relative to the text, remains the same. 
\backgroundimage{url} 
Use \backgroundimage to tile the techexplorer window background with animage. You can use any image that you might use with \includegraphics. When you 
print a techexplorer document, the background image is ignored. If you include 
more than one instance of \backgroundimage in your document, the last one encountered is the one that is used. If the image addressed by url does not exist or 
cannot be retrieved from the network, no background image is rendered. 
\backgroundsound{url} 
\includeaudio{url} 
The commands \backgroundsound and \includeaudio play audio files if your 
computer has the appropriate sound hardware. The sound file is played only once; 

 
%%page page_232                                                  <<<---3
 
5.1 IBM techexplorer Hypermedia Browser 
211 
perhaps a future version will allow looping. Only WAV files with file extension . wav 
are supported, and then only on the Microsoft VV1ndows platforms. This feature is 
supported only in the Professional Edition of techexplorer. 
By default, the sound file is played as soon as it can be loaded. \reffig{5-7} shows 
the standard property page for setting your techexplorer permissions. If you do 
not wish to have background sounds played when pages are loaded, uncheck the 
box labeled “Enable audio to autoplay when the document is opened". 
\audioLink{url} {text} 
To have audio played when the reader clicks the mouse button within an area on the 
screen, use an audio link. The sound file is retrieved from url, and text is displayed 
on the screen. Like \backgroundsound, only WAV files with file extension .wav 
on Microsoft Windows are supported. On other platforms, no sound is made when 
the link is executed. 
\videoLink{url}{text} 
Use a video link to play video in a pop-up window when the mouse button is clicked. 
The video file is retrieved from url, and text is displayed on the screen. Only AVI 
files with file extension . avi on Microsoft Windows are supported. When the video 
starts playing, it will continue to the end and then the pop-up window will close. 
A more interesting video option where inline video can be used is available to 
users of the Professional Edition of techexplorer. 
\inc ludevideo{autoPlay}{altemateYIext}{url} 
The video file is retrieved from url. If it cannot be found or has not yet arrived, the 
ulteintztei/ext is displayed on the screen (the alteinate'1ext is always rendered when 
the document is printed). If autoPlay is “t", the video will start playing as soon as it is 
loaded. You can unilaterally prevent videos from autoplaying by unchecking the box 
labeled “Enable video to autoplay when the document is opened" in the permissions 
option property page (see \reffig{5-7}). To pause, stop, or replay the video, click 
your right mouse button on the video and select from the menu. \reffig{5-8} shows 
an example of an inline video with the menu for controlling play. 
5.1.6 Defining and using pop-up menus 
A pop-up menu, also called a context menu, is a menu invoked by right clicking your 
mouse somewhere on the screen. techexplorer has a default pop-up menu that is 
used to access the customization options, including printing, searching, and information about the release of techexplorer you are using. You get the default menu 
by moving your mouse cursor to an empty area of the techexplorer window and 

 
%%page page_233                                                  <<<---3
 
212 Direct display of HIEX on the Web 
\reffig{5-7}: Setting permissions within \reffig{5-8}: Inline video in the Professional 
techexplorer Edition of techexplorer 
clicking the right mouse button. In the Professional Edition of techexplorer, this 
default menu looks like this: 
You cannot alter this menu, although the availability of the “topic" items will vary 
according to what you have defined in your document to aid the reader in moving 
through the document hierarchy (see Section 5.1.8 on page 218). 
You can define new menus that activate whenever you right click your mouse 
over a given area of text. VVhat are the entries in such menus? They are links, rules, 
and other menus. 
A link in a menu provides the menu item text and the action that occurs when 
that item is clicked. The most common types of links used in menu definitions are 
hypertext links and links that play sound or video. 

 
%%page page_234                                                  <<<---3
 
5.1 IBM techexplorer Hypermedia Browser 213 
Following is a simple three-item definition for a menu that plays songs: 
\newmenu{SinatraSongs}{ 
\audioLink{http://www.sinatrafan.com/songs/y©ung.wav} 
{You make me feel so young} 
\audioLink{http://www.sinatrafan.com/songs/pennies.wav} 
{Pennies from heaven} 
\audioLink{http://www.sinatrafan.com/songs/anything.wav} 
{Anything goes} 
} 
As you see, \newmenu is used to provide a named menu definition. In this case the 
name is Sz'mztmSong5, and the definition contains three \audioLinks. 
\newmenu{menuNzzme}{menuDdinition} 
\usemenu{menuName} {text} 
The \usemenu command associates the menu with some text. The same menu definition can be used for as many instances of \usemenu as you wish. 
For example, 
The songs of \usemenu{SinatraSongs}{Frank Sinatra} are 
loved by people of all ages. 
VVhen you right click your mouse on “Frank Sinatra," techexplorer displays 
li\HlH1l' In lr--l ii nwlnui 
If you click on any of the menu items, the corresponding song is played.5 
Use a rule in a menu definition to provide a separator line between entries. You 
can use \hrule, \vrule, or \rule since the rule is not actually drawn but is simply 
used as an indicator of the separator line position. 
Following is a definition that includes both songs and movies starring Frank 
Sinatra. 
\newmenu{SinatraSongsAndMovies}{ 
\audioLink{http://www.sinatrafan.com/songs/young.wav} 
{You Make me feel so young} 
\audioLink{http://www.sinatrafan.com/songs/pennies.wav} 
{Pennies from heaven} 
\audioLink{http://www.sinatrafan.com/songs/anything.wav} 
{Anything goes} 
\hrule X provides a separator line 
5That is, if the Web site and content existed! 

 
%%page page_235                                                  <<<---3
 
214 
Direct display of ETEX on the Web 
\videoLink{http://www.sinatrafan.com/movies/town.avi} 
{On the Town} 
\videoLink{http://www.sinatrafan.com/movies/manchurian.avi} 
{The Manchurian Candidate} 
\videoLink{http://www.sinatrafan.com/movies/eternity.avi} 
{From Here to Eternity} 
} 
This menu is used in the same way as the last one was used. 
The songs and movies of \usemenu{SinatraSongsAndMovies}{Frank Sinatra} 
are very popular. 
Now when you right click your mouse on “Frank Sinatra," the menu displayed is 
U H We I 
If you click on any of the menu items, the corresponding song or movie is played 
from our hypothetical Web site. 
If you put a \usemenu inside a \newmenu definition, you will get a submenu. 
Let’s rework the last menu so that songs and movies are separated into their own 
submenus. We’ll keep the initial definition for Sz'mztmSong5 as 
\newmenu{SinatraSongs}{ 
\audioLink{http://www.sinatrafan.com/songs/young.wav} 
{You make me feel so young} 
\audioLink{http://www.sinatrafan.com/songs/pennies.wav} 
{Pennies from heaven} 
\audioLink{http://www.sinatrafan.com/songs/anything.wav} 
{Anything goes} 
} 
and add a similar definition for SimztmMovie5: 
\newmenu{SinatraMovies}{ 
\videoLink{http://www.sinatrafan.com/movies/town.avi} 
{On the Town} 
\videoLink{http://www.sinatrafan.com/movies/manchurian.avi} 
{The Manchurian Candidate} 
\videoLink{http://www.sinatrafan.com/movies/eternity.avi} 
{From Here to Eternity} 
} 

 
%%page page_236                                                  <<<---3
 
5.1 IBM techexplorer Hypermedia Browser 
215 
Finally we’ll define SinatmSong5AndMovie52 with the two submenus, and we’ll include the separator: 
\newmenu{SinatraSongsAndMovies2}{ 
\usemenu{SinatraSongs}{Sinatra songs} 
\hrule Z provides a separator line 
\usemenu{SinatraMovies}{Sinatra movies} 
} 
The final, fully opened menu can look like one of the following: 
Almost any link except \altLink can be used within a menu definition. If you 
use a link command that takes a url argument but leave the argument empty (for 
example, \labelLink{}{Sect ion 5}), the menu item will be disabled and will be 
shown in gray. 
5.1.7 Using color in your documents 
techexplorer supports the \color, \textcolor, \colorbox, \fcolorbox, and 
\page color commands from the ETEX color package. The basic colors from the 
color package are provided, along with several other colors that are convenient for 
on-screen display. In the following list of supported colors, the new techexplorer 
color names are set in bold: aqua, black, blue, cyan, darkgray, fuchsia, gray, green, 
lightgray, lime, magenta, maroon, navy, olive, purple, red, silver, teal, white, and 
yellow. 
To change the color of rules, issue a \color command before the rule definit1on. 
\bgroup 
\color{red}% 
\hrule height Spt Z this will be red 
\egroup 
\hrule height Spt Z this will probably be black 
We enclose the color change and rule in a \bgroup/\egroup pair to preserve the 
default color setting. 
\backgroundcolor{color} 
The command \backgroundcolor_ can be used as a synonym for \pagecolor. Both 
commands set the background color behind the displayed text on the screen. If you 

 
%%page page_237                                                  <<<---3
 
2 16 Direct display of ETEX on the Web 
lllll‘()(lllCll0[1 
\reffig{5-9}: A gradient box in a section heading 
wish to match the standard “browser gray" in an HTML window, we suggest you 
use \pagecolor{silver}. A restriction with the current Professional Edition of 
techexplorer is that the background color is ignored when printing a document. 
Note that these background color commands work only for the window that 
displays your document. If you wish to change the background color for all 
techexplorer windows, 
1. right click your mouse on an area of whitespace in any techexplorer document window; 
choose Options... from the menu displayed; 
click on the Colors tab; 
click on the Standard background... button; and 
=~4>&~z~* 
select a color by first clicking on it and then clicking on the OK button. 
To test your choice, click the Apply button. If you don’t like what you see, go back 
to step 3 and repeat the process. Click the OK button to save your selection, or click 
Cancel to return to your previous background color. \reffig{5-2} shows the standard 
property page for setting your color choices. 
\rgb{red Value} {green I/alue} {blue I/dlue} 
The \definecolor command is not currently supported, but you can use \rgb to 
define new colors. The values for redl/alue, green Value, and bluel/alue are integers 
between 0 and 255, inclusive. For example,6 
\def \paleYellow{\rgb{255}{255}{128}} 
\pagecolor{\paleYellow} 
sets the background color of the techexplorer document window to a light yellow. 
The standard \<‘.olorbox and \fcolorbox commands allow you to place text 
in boxes with a given background color. techexplorer provides \gradientbox 
(see \reffig{5-9}) to let you create a box with the background color varying smoothly 
from one color to another. 
étechexplorer does not support \newcommand fully, so we use the TEX primitive \def. 

 
%%page page_238                                                  <<<---3
 
5.1 IBM techexplorer Hypermedia Browser 
217 
\gradientbox [U] {startColor}{endColor}{text} 
By default, the gradient begins with smrtColor on the left side of the box and finishes 
with em1Color on the right side. If the 1} optional argument is given,7 the gradient is 
drawn vertically from top to bottom.8 
\buttonbox [i] {text} 
\colorbuttonbox [i] {color}{text} 
To create boxes that look like standard Microsoft VVindows buttons with a gray 
background, use \buttonbox. If the 1' optional argument is given (again, literally), 
the button is drawn in an inverted state. Such a button looks as if it has been pressed. 
\buttonbox{This is a button box} 
\buttonbox [i] {This is an inverted button box} 
If you want a button that has a background color other than gray, use 
\colorbuttonbox. If the i optional argument is given, the button is drawn in an 
inverted state. 
\colorbuttonbox{green}{This is a green button box} 
\colorbuttonbox[i] {green}{This is a green inverted button box} 
Use color carefully in your documents because too much variation can be as 
confusing as overusing fonts from different families. As you should with any good 
BTEX document design, create macros or style files that encapsulate your design 
choices. You can then modify the macros or style files to achieve a global design 
change. Following is an example of a “Section" macro that uses a gradient box and 
a text color change:9 
\def\Section#1{\section{'Z 
\gradientbox{blue}{white}{\mbox{\color{white}#1}}}} 
\Section{Introduction} 
Note that for techexplorer you should not skip a space after the \color command. Regular ETEX may ignore the space, but techexplorer will not. 
7Literally specify the v, as in \gradientbox [V] {red}{white}. 
8The current UNIX versions of techexplorer do not draw gradients; so \gradientbox is the same 
as \colorbox with s'tartColor as the box color. 
9Remember that techexplorer does not support \newcommand fully. 

 
%%page page_239                                                  <<<---3
 
218 
Direct display of Ié'I}3X on the Web 
Table of contents 
toc.tex 
Preface Chapter 1 Index 
preface.tex 1.tex index.tex 
Section 1.1 Section 1.2 
1.1.tex 1.2.tex 
\reffig{5-10}: Simplified structure for a book 
5.1.8 Building a document hierarchy 
Most traditional documents are structured into tree-shaped hierarchies of chapters 
and sections. We often divide them into multiple files for convenience of editing, 
and the parts are included in some way into the main document when printing. 
Electronic versions of documents are also frequently broken into a number of files 
that reflect the hierarchy, but the files are usually viewed separately and are connected by hyperlinks. 
For example, consider the structure in \reffig{5-10}, where each node shows the 
title of the document part and the name of the file that contains the markup for the 
part. 
The table of contents is the root of this tree, and all top level book parts descend 
from this root. Some parts, like Chapter 1, have component sections themselves. 
techexplorer provides a standard way to connect the pieces of the hierarchy 
with hyperlinks. Use \aboveTopic to give the URL of the parent document of a 
given part.1° 
\aboveTopic{url} 
\nextTopic{url} 
\previ ousTopic{url} 
In the example in \reffig{5-10}, the file preface .tex should contain the command 
\aboveTopic{toc . tex}. 
Use \nextTopic to give the URL of the next sibling document on the same 
level as the given part. Thus the file preface.tex should contain the command 
\nextTopic{1 .tex}. Finally, use \previousTopic to give the URL of the previous 
sibling document on the same level as the given part. 
The file 1 . tex should contain the command \previousTopic{pref ace . tex}. 
10The parent document is the next document in the tree on the direct path to the root of the tree. The 
root of the tree has no parent. 

%==========240==========<<<---2
 
%%page page_240                                                  <<<---3
 
5.1 IBM techexplorer Hypermedia Browser 219 
Table of contents 
toc.tex 
Preface Chapter 1 Index 
preface.tex 1.tex index.tex 
\aboveTopic{toc.tex} \aboveTopic{toc.tex} \aboveTopic{toc.tex} 
\nextTopic{1.tex} \nextTopic{index.tex} \previousTopic{1.tex} 
\previousTopic{preface.tex} 
Secdon L1 Secdon L2 
1.1.tex 1.2.tex 
\aboveTopic{1.tex} \aboveTopic{1.tex} 
\nextTopic{1.2.tex} \previousTopic{1.1.tex} 
\reffig{5-11}: The document tree 
The root document part will not have an \aboveTopic command. The first 
section at a given level will not have a \previousTopic command, nor will the 
last section at the same level have a \nextTopic command. Though it is tempting 
always to define the next and previous topic in some way, it will probably be more 
confusing for your readers unless you provide some sort of document map. 
These hyperlinks are accessed by clicking the right button of your mouse in the 
techexplorer window. Unless you click over a user-defined pop-up menu (these 
are described in Section 5.1.6 on page 211), you will see a menu similar to this: 
In this menu, all three topics are available. Should one of the above, next, or previous topics not be defined in the document, the corresponding menu item will be 
“grayed out," that is, unavailable. 
\reffig{5-11} summarizes this discussion with an expanded diagram of the document tree at the beginning of the section. It includes the commands to link together 
the direct paths in the hierarchy. 

 
%%page page_241                                                  <<<---3
 
220 
Direct display of I1."l]§X on the Web 
5.1.9 Running applications 
\appLink{command}{text} 
You can use techexplorer as a browser-based interface for running commands on 
your computer. An application link executes a given command when you click on a 
given area of text on the screen. The command is what is executed when you click 
on the screen display of text. 
The following example will display a directory listing for the root directory on 
your C: drive under Microsoft VVindows: 
Click \appLink{dir C:\}{here} to see the root directory of 
your C drive. 
The command can refer only to programs that reside on your computer, including those that are part of the operating system. Thus application links are probably not portable across platforms nor possibly even across machines running the 
same operating system. However, they can be quite useful for applications where 
techexplorer is utilized as part of a larger electronic publishing solution. For example, interactive books that provide all the tools needed by each student might 
use some application links. 
There is an important security issue associated with application links or, indeed, 
with any Web software that can execute programs on your computer. You should 
know what the command does before you allow it to run, or at least you should 
trust the provider of the document that contains the application link. 
\reffig{5-7} on page 212 shows the standard property page for setting permissions within techexplorer. The default setting for application links is always to 
ask permission before they are executed. If you are a more trusting individual, you 
can select the option that allows application links always to run when invoked. Finally, if you want to prevent all application links from executing, choose “do not 
execute" from the choices. 
For most users, the default choice of always asking permission is probably best. 
However, if you do not know what the command will do, get more information 
before running it. 
5.1.10 Alternating between two displayed expressions 
\altLink{5econdText} {firstText} 
You can use an alternating link to toggle the display of an area of text between two 
choices. The fir5tText expression is initially displayed. When you click on that expression, techexplorer changes the display to secondfixt. Alternating links might 
seem to be a novelty, but they are useful in situations where you want the reader 

 
%%page page_242                                                  <<<---3
 
5.1 IBM techexplorer Hypermedia Browser 221 
to think about something before you give the “answer." Consider the following 
problem and its solution, showing the two alternate expressions: 
Problem: Solve for x in ax: + bx+ c = U. 
Solution; 2: = '.? 
Problem: Solve for x in axz + bx+ c = U. 
-bmfbz-4aa 
Solution: 2: = 20 
The basic expression was generated by 
\textbf{Problem:} Solve for $x$ in $ax“2+bx+c=0$. 
\textbf{Solution:} $x=\altLink{\frac{-b \pm\sqr‘t{b“2-4ac}}{2a}}{'?}$ 
b2-4ac 
If you click again on the 757-, the display will revert back to the form with 
the question mark. 
5.1.11 Printing firom techexplorer 
Only the Professional Edition of techexplorer on VV1ndows implements printing 
for whole documents that are displayed by the plug-in.11 The most reliable way 
of printing a techexplorer document in any browser is from the default context 
menu. You get the default menu by moving your mouse cursor to an empty area of 
the techexplorer window and clicking the right mouse button. In the Professional 
Edition of techexplorer, this default menu looks like the following: 
Fxgmw. 
Select Print... and then proceed through the menus to select and send your output 
to a printer. The number of pages of the document is not known until print composition time, so print either the entire document or only specific known pages. 
On some browsers (notably Netscape Navigator), you may also be able to use 
the Print... item from the File menu or the print button on the browser toolbar. 
“Priming is not yet supported for plug-in instances embedded within an HTML page. 

 
%%page page_243                                                  <<<---3
 
222 
Direct display of I‘¥IEX on the Web 
in Theorem 3.3 (3), we will suppose that p = P1? factors as I ® co where Iis 
representatiw. -c .i.'.....,..'..... 4 s- '7 .i.....;:............ .i...-....i. .. ::..;... ....-..'..... 
Find 
envalue ufBis inverse 0 a 0 . us - an are eat so as 125 an 1! 
21:eem( U) ) is finite and C is 323 times a matrix , C is ycfimes the id‘ 
H-1 is a pseudoreflection and it cannot be a’ 2 2 copies of anything. 
\reffig{5-12}: Searching for text in a document 
Version 2.0 of techexplorer was the first version to support any form of printing. In this release, the goal was to print the ETEX documents at a quality comparable to the HTML printed by the browser. In general, BTEX itself will do a better 
job of printing, if you can use it. See the online documentation to get more information about the style file that is provided to help you develop documents that can 
be processed by both techexplorer and BTEX. 
Future editions of techexplorer may give you more control over the page 
parameters such as margins. Also, the IBM developers are working on a PostScript 
generation to allow printing within future UNIX editions of the Professional Edition of techexplorer. 
5.1.12 Searching in a document 
The Professional Edition of techexplorer includes a searching facility to allow 
you to locate all occurrences of a string in a document. You start the search by 
clicking the right mouse button within the document and choosing Find... from the 
pop-up menu. Enter your search string and press the Find button. Text matching 
your string within your document will be selected. 
\reffig{5-12} shows an example of looking for the word “unipotent" in the document. By way of illustration, the results of a previous search are selected behind 
the dialog box. 
Note: You can use only the techexplorer find dialog via the document pop-up 
menu to locate text within a techexplorer document. The browser find dialog 
will look only within HTML pages and will not descend into embedded plug-in 
windows. Worse, some browsers will lock up if you invoke their find dialogs when 
a plug-in is in control of the whole browser window. 
5.1.13 Optimizing your documents for techexplorer 
This section is a cookbook for taking existing BTEX documents and making them 
usable within techexplorer. On the way, we’ll add some features to the electronic 

 
%%page page_244                                                  <<<---3
 
5.1 IBM techexplorer Hypermedia Browser 
223 
version of the documents that enhance the viewing experience. 
a Determine if your document is small enough to be processed by the 
techexplorer plug-in. If not, break it into smaller documents that can be hyperlinked together. 
a Even if your document is small, consider breaking it into hyperlinked components anyway. Readers like documents that arrive and render quickly. 
a Use \aboveTopic, \nextTopiC, and \previousTopic to build a document 
hierarchy (see Section 5.1.8 on page 218). 
o techexplorer does not yet support cross-references such as those created by 
\ref. These should be changed into explicit hyperlinks. 
a Add labels for all document sectioning commands. Use standard naming conventions for the labels so that you can use macros to simplify other markup. 
For example, use chapt-.er:2 or sectionzs .4. 
a Add hyperlinks where the text refers to phrases like “Chapter 2" or “Section 
3.4" (although you should remember that in hyperdocuments, phrases like 
“Chapter 2" make sense only for a static document). Use the labels you added 
earlier to create such hyperlinks. Use \<1ocLink for a hypertext jump outside 
a given document and \labelLink for jumps within the document. See Section 5.1.3 on page 204 for details. 
5.1.14 Scripting techexplorer from Java and Javascript 
The Professional Edition of techexplorer provides a programming interface for 
scripting via Java and JavaScript while operating within Netscape Navigator. This 
uses Netscape LiveConnect and, therefore, does not work within Microsoft Internet Explorer 4.0. 
The programming interface allows applets and scripts to register themselves as I 
listeners for mouse, key, and window focus events within techexplorer windows. 
The applets and scripts can also update the techexplorer documents within those 
windows. 
The details of the programming interface are beyond the scope of this book, 
but we offer the following annotated example (\reffig{5-13}) of a simple BTEX editor 
that is written in]ava and uses techexplorer to display the formatted markup. The 
editor consists of two windows within an HTML page. The upper window is owned 
by techexplorer and contains the rendered ETEX markup. The lower window 
contains the BTEX source as it is entered by the user. You can type when the cursor 
is over either window, although you may have to click in one window or the other 
for it to respond to your key strokes. Click the Clear input button to delete all the 
markup and the rendered display. \reffig{5-13} shows the editor with some sample 
input. 

 
%%page page_245                                                  <<<---3
 
224 
Direct display of ETFX on the Web 
T leched Sample LaTeX kdllol - elsnape V 
This is some text. Next we’ 11 see some mathematics: 
P bf?! x_1 
\J1"2+1 e-log5 
This is some text. Nextwsrll see some mathematics: 
1[P =1lefl[lbegin{array}{cc}1frac{a){b-1pi}&x-1u 
1sqrt{1Gamma’*2+1}& 9-11og{5}1enI:I{array}1right]1] 
\reffig{5-13}: A simple ETEX editor built using techexplorer 
The source code for the editor is listed in Appendix A.2. It consists of two files: 
The HTML source for the page is in teched.html (Section A.2.1 on page 399), 
and the Java applet source is in teched. java (Section A.2.2 on page 400). 
5.2 WebEQ 
WebEQ ['->WEBEQ] is a set of tools that includes a Java applet for displaying 
math." It supports the Mathematical Markup Language (MathML), which we will 
come back to in Chapter 8, and a collection of commands it calls WebTEX. 
The WebEQ package includes an editor that allows you to create a math expression and then save it as MathML source, a]PEG or PNG image, or a complete 
Java applet element that can be copied and pasted onto an HTML page. The editor 
can also open a file containing an existing MathML expression; it does not support 
saving your work in WebTEX form. 
The editor was created using the Java programming interface WebEQ. This 
interface allows you to add math rendering and image generation to your own applets and allows control of the WebEQ from JavaScript. You can dynamically build 
new math objects and manipulate the internal structure of existing objects. 
12WebEQ is a trademark of Geometry Technologies, Inc. 

 
%%page page_246                                                  <<<---3
 
5.2 WebEQ 
225 
test file 1Fliclosoft Intemel Explore: 
\reffig{5-14}: Simple example of WebEQ 
To ease the insertion of math objects into Web pages, the WebEQ VVizard 
is provided. This application takes a source file containing HTML markup and 
WebTEX or MathML markup and produces a new HTML file containing images or 
WebEQ applet tags for the mathematical expressions. \reffig{5-14} shows a slightly 
simplified section of our test file (refer to \reffig{5-1} on page 197) processed by 
the WebEQ VVizard to create images. \reffig{5-15} shows some WebTEX input to 
WebEQ’s VVizard and the HTML code that results. 
In the following discussion we will look in more detail at those features of 
WebEQ that pertain to WebTEX. We begin with a discussion of WebTEX and then 
show how to write the HTML APPLET tags to embed math in your document. We 
conclude by describing the WebEQ VVizard and look at an example of using the 
VVizard to insert images or APPLET tags automatically into your document. 
5.2.1 An introduction to WebTEX 
WebTEX is not quite TEX, and it is not quite ETEX. Usually a command is similar 
to something in TEX or ETEX, but it is important to check the documentation 
before writing new markup or converting existing markup to the WebTEX format. 

 
%%page page_247                                                  <<<---3
 
226 
Direct display of Ié’IEX on the Web 
Input to WebEQ Wizard: 
<HTML> 
<HEAD><TITLE>test fi1e</TITLE></HEAD> 
<BDDY> 
<H1>Vavi1ov theory</H1> 
\[ 
f \1eft ( \epsi1on, \de1ta s \right ) = \frac{1}{\xi} \phi_{v} 
\1eft ( \1ambda_{v}, \kappa, \beta‘{2} \right ) 
\] 
where 
\[ 
\phi_{v} \1eft ( \1ambda_{v}, \kappa, \beta*{2} \right ) = 
\frac{1}{2 \pi i} \int‘{c+i\infty}_{c-i\infty}\phi \1eft( s \right ) 
e*{\1ambda s} ds c \geq O 
\] 
</body> 
</html> 
Output from WebEQ VVizard: 
<HTML> 
<HEAD><TITLE>test fi1e</TITLE></HEAD> 
<body bgcolor=#COCOCO> 
<H1>Vavi1ov theory</H1> 
<P><CENTER> 
<app1et code="uebeq.Main" width=194 height=48 align=midd1e> 
<param name=eq va1ue="\disp1aystyle { 
f \1eft ( \epsi1on, \de1ta s \right ) = \frac{1}{\xi} \phi_{v} 
\1eft ( \1ambda_{v}, \kappa, \beta“{2} \right ) 
}n> 
<param name=color va1ue="#COCOCO"> 
<param name=parser va1ue=``webtex''> 
<img src="xxx1.png" a1t="xxx1.png" a1ign=absmidd1e> 
</app1et></CENTER><P> 
where 
<P><CENTER> 
<app1et code="webeq.Main" width=342 height=68 a1ign=midd1e> 
<param name=e1 va1ue="\disp1aystyle { 
\phi_{v} \1eft ( \1ambda_{v}, \kappa, \beta‘{2} \right ) = 
\frac{1}{2 \pi i} \int“{c+i\infty}_{c-i\infty}\phi \1eft( s \right ) 
e‘{\1ambda s} ds c \geq O 
}n> 
<param name=color va1ue="#COCOC0"> 
<param name=parser va1ue=``webtex''> 
<img src="xxx2.png" a1t="xxx2.png" a1ign=absmidd1e> 
</app1et></CENTER><P> 
</body> 
</html> 
\reffig{5-15}: Simple example of WebEQ (\reffig{5-14}): VVizard input and output 

 
%%page page_248                                                  <<<---3
 
5.2 WebEQ 
227 
We’ll begin by looking at features that are similar to those in TEX or ETEX. 
Group expressions using ‘{’ and ‘}’ and use _ and " to create subscripts and superscripts. 
Use \frac to create fractions and \binom to make binomial coefficients. The 
\sqrt and \root commands have their familiar syntax for creating radicals. The 
usual loglike operators such as \cos are all supported. You can create accents using \bar, \check, \dot, \ddot, \hat, \tilde, and \vec. Use \overbrace and 
\underbrace to draw stretchy braces above or below an expression, respectively. 
The \overset and \underset commands can position an expression above or below another expression, respectively. Simple text can be inserted into an expression 
by using \text. The argument can include embedded math (delimited by $. . . 3;), 
but not much else. 
The standard TEX or BTEX symbols are supported, but the AMS symbol set 
is not. Unlike techexplorer, WebEQ does not require extra fonts for its symbols." It contains the glyph images and format information within the applet for 
a fixed set of fonts and a fixed set of sizes. You can use the commands \mathrm, 
\mathit, \mathbf, \mathfr, \mathsf, \mathtt, \mathbb, and \mathcal to use 
Roman, italic, bold, fraktur, sans serif, typewriter, blackboard bold, and calligraphic 
font families, respectively. 
You can change the font size within an expression by using \textsize, 
\scriptsize, or \scriptscriptsize, in order of decreasing size. The formatting 
style can be explicitly changed via \displaystyle or \textstyle, but commands 
such as \scriptstyle, \scriptscriptstyle, \large, \small, and so on are not 
supported. 
VVhat about the real dzfiferemes between WebTEX and TEX/ETEX? Macros 
are created using the \define command, which has a syntax similar to the ETEX 
\newcomma.nd command. Macros are passed as parameters to the WebEQ applet, 
via either macros or macrofile parameter. (See page 231 for information about 
macros.) 
\f ont c o 1 or {color} {expression} 
The BTEX color commands are not supported, but you can use \fontcolor to set 
the color of a particular subexpression. The first parameter to \fontcolor is the 
#RRGGBB color specification, and the second is the expression to be displayed in that 
color. For example, 
\fontcolor{#cOcOcO}{x+y} 
displays x + y in a light gray color. 
“The Professional Edition of techexplorer does, however, provide symbol fonts in its distribution. 

 
%%page page_249                                                  <<<---3
 
228 
Direct display of Ié'IEX on the Web 
\multi s cr ipt {prescripts} {base} {scripts} 
\t ens or{bzzse} {scripts} 
WebTEX provides the \tensor command to allow you to specify tensors with all 
their subscripts and superscripts. The \multiscript command is a general purpose tool for placing subscripts and superscripts before and after the base expression. For example, 
\multiscript{_i‘j}{H}{_k‘l} 
formats as _ 
{Hi 
The usual delimiters are supported with \left and \right, but note that you 
do not use a period to indicate a matching but empty delimiter. You simply omit 
the delimiter. Watch this if you are converting expressions from TEX. 
WebTEX provides a powerful array-formatting facility that is also quite different from that familiar to ETEX users. The basic command is \array, & separates columns, and \\ separates rows. Column alignment is specified by \colalign 
within \arrayopts. Compare the BTEX expression 
x y+1z-1 
y - z x2 0 
given by the markup 
\begin{array}{lcr} 
x & y + 1 & z - 1 \\ 
y - z & x‘2 & 0 
\end{array} 
with the corresponding WebTEX markup 
\array{ 
\arrayopts{\colalign{left center right}} 
x &y+1&z-1\\ 
y - z & x‘2 & 0 
} 
You use the \rowalign command \arrayopts to adjust the entries vertically in 
each row. This allows you to push the entries in a given row down so that the each 
entry is bottom aligned. You specify top, bottom, axis, or baseline for each row. 
Additional array options allow you to draw a frame around the array with a 
choice of line styles, have entries span more than one row or column (thus generalizing \multicolumn), draw lines between rows or columns, adjust the padding 
between entries, and vertically adjust the whole array with respect to the baseline. 

%==========250==========<<<---2
 
%%page page_250                                                  <<<---3
 
5.2 WebEQ 
229 
This rich choice of options provides WebEQ with the internal facilities to implement the MathML table model. 
You can use \thinsp, \medsp, \thicksp, and \quad to add space within an 
expression. The \quad command is the same as in TEX, and the other three are 
the same as the familiar \thinspace, \medspace, and \thickspace. The \qquad 
command is not supported. 
\space{laeiglatl-{zleptb}{wizltl2} 
The \space command produces an empty area with specified height, depth, and 
width. This differs from its standard application within NFSS. 
\rule{beigbt}{dept/y}{widt/7} 
The \rule command produces a solid rectangle as in ETEX, but the arguments 
have a different interpretation. 
It is important to note that WebEQ ignores and does not show in WebTEX 
any command that it does not understand. In particular, if you make a typing mistake, part of your expression will not be displayed. You should always compare the 
displayed result with the markup to ensure that you got what you intended. 
5.2.2 Adding interactivity 
\href {url}{expression} 
WebTEX supports hypertext linking via the \href command. The expression is what 
you see on the screen (in blue), and url is the address of the document to which the 
browser will jump when the reader clicks text. 
\statusline{messczge}{expression} 
Use \statusline to change the message on the browser status line when the user’s 
mouse cursor passes over an expression. This is very useful to explain what parts of 
larger expressions mean. Note that the message is displayed on the status line using 
plain text, so don’t get too fancy with it. 
\f ghi ghl 1 gm: {color} {expression} 
\bghi ghl i ght {color} {expression} 
To highlight an expression further, use \fghighlight or \bghighlight. The 
\fghighlight command changes the foreground color, that is, the text color, when 
the mouse cursor passes over expression. Similarly \bghighl ight changes the background color when the mouse cursor is over expression. The color argument is of the 
form #RRGGBB, using the standard HTML RGB color definition format. 

 
%%page page_251                                                  <<<---3
 
230 
Direct display of Ié'IEX on the Web 
You can combine these features so that you can change colors and update the 
status line message at the same time. For example, 
\bghilight{#c0c0c0}{ 
\fghilight{#ff00O0}{ 
\statusline{This is y+1}{y + 1}}} 
changes the background color to gray, changes the expression font color to red, and 
displays This is y+1 on the status line when the mouse cursor lies over y + 1 on 
the screen. 
\toggle{expr1 }{expr2}{messzzgeI }{messzzge2} 
Just as techexplorer provides \altLink, WebEQ provides \toggle to alternate 
between two displayed expressions. The exprl is first displayed on the screen, and 
the status line displays message] when the mouse cursor passes over the expression. 
If you click on exprl , it will change to expr2, and the mouse-activated status line 
message will be messzzge2. 
5.2.3 Using the APPLET tag with WebEQ 
Use the HTML APPLET element to include WebTEX markup to be rendered by 
WebEQ within an HTML page. For example, 
<applet codebase=``classes'' 
code="webeq.Main" 
width=100 height=100 align=middle> 
<param name=eq value="\alpha“2-\frac{1}{\beta}"> 
<param name=color value=#FFFFFF> 
</applet> 
There are five important attributes for using the APPLET element for WebEQ: 
CODEBASE This is the directory containing the compiled Java bytecode for 
WebEQ. See the product documentation for the proper value, based on the 
way you intend to install and use the software. 
CODE This is always webeq.Main. 
HEIGHT This is the height in pixels of the rectangle in which the markup should 
be rendered. 
WIDTH This is the width in pixels of the rectangle in which the markup should be 
rendered. 
ALIGN Use ALIGN=MIDDLE when you want the expression to float vertically so it 
looks better with respect to the baseline of the surrounding text. 

 
%%page page_252                                                  <<<---3
 
5.2 WebEQ 
231 
The PARAM element is used within APPLET to give additional formatting and 
control information to WebEQ. The format for PARAM is 
<param name=TheName value=``TheValue''> 
where the possible values for TheName are allow_cut, color, controls, eq, 
linebreak, macrofile, macros, parser, size, and src. 
The values corresponding to the eq and src parameter names tell WebEQ how 
to get the source markup for the math expression. For eq, the value is the actual 
WebTEX markup. For src, the value is the URL of the document containing the 
markup, relative to the URL of the containing HTML page. 
Use parser to tell WebEQ whether the source is WebTEX or MathML. 
WebTEX is assumed if this parameter is not given. 
The values corresponding to the macros and macrof ile parameter names supply macros to apply to the WebEQ markup. Use a macros parameter to supply one 
or more explicit macro definitions. For example, 
<param name=macros 
value="\define{\a}{\alpha}\define{\b}[1]{\beta_#1}"> 
provides definitions for \a and \b, where the former takes no arguments and the 
latter takes one. If you wish to collect several macro definitions into a file, use 
macrofile to tell WebEQ the URL of the file. Like the value for src, this URL is 
relative to the URL of the containing HTML page. If you supply both macros and 
macrofile, the macros are pooled before the math expression is parsed. 
The size parameter value gives the initial font size in points for the math 
expression. WebEQ contains eight fonts at several fixed sizes. If the value you give 
is not one of those available, WebEQ will choose one in a nearby size. Related 
to size is the controls parameter. If this is true (the default), clicking the right 
mouse button when the cursor is over an applet will bring up a dialog box that 
allows you to change the font size. This will help you adjust the display of the math 
to the surrounding HTML text and will also allow you to correct for the differences 
in Java rendering across the various platforms. 
Another formatting parameter is color. It allows you to set the background 
color for the rectangle in which WebEQ displays the math expression. The value is 
of the form #RRGGBB, using the standard HTML RGB color definition format. Note 
that HTML color names like “red" are not allowed. Use this parameter to match 
the background of the math expression to the background color of the HTML page. 
If you want to change the color of an expression rendered by WebEQ, use the 
\fontcolor WebTEX command within the markup itself. 
For expressions that are too wide to fit in the available space, you can use the 
linebreak parameter with a value of true to tell WebEQ to try to break the expression into multiple lines. \reffig{5-16} shows how WebEQ can break the expression x9 + x8 + x7 + x6 + x5 + x4 + x3 + x2 + x + 1 into two lines. You will probably 

 
%%page page_253                                                  <<<---3
 
232 
Direct display of Ié'IEX on the Web 
Unbroken expression 
x9-|-x8-|-x7-|-xfi-|-x5-|-x4-|-x3-|-x2-|-x-|-1 
Line-broken expression 
x9+x8-|-x7+xfi-|-x5+ 
x4-|-x3-|-x2-|-x-|-1 
\reffig{5-16}: Linebreaking by WebEQ 
need to increase the HEIGHT attribute value to allow for the increased vertical space 
taken up by the expression. 
The final parameter is allow_cut. If the value of this is true, then subexpression selection is enabled, and it is possible to get generated MathML markup placed 
in a pop-up window for cut and paste. 
5.2.4 Preparing HTML pages via the WebEQ Wizard 
The WebEQ VVizard is an easy way to prepare HTML pages that have math in 
them. You start by creating an HTML source file that has WebTEX math markup 
contained in $ . . . $ for inline expressions and \ [. . . \] for displayed expressions." 
It is recommended that you use the file extension . src or . wiz for this source file 
to differentiate it from the final .html file that is generated. 
Next run the WebEQ VVizard to create the HTML file for use on the Web. 
The VVizard will convert your math expressions into either images or Java applet 
tags. The tags encapsulate the math markup in WebTEX, or in MathML from the 
source, or MathML generated from WebTEX source. The images can be in JPEG 
or PNG format, although only recent Web browsers support PNG, and you need 
Java 1.1 or later. The applet tags will include the estimated heights and widths for 
the rectangles in which the math expressions are displayed. 
The images may look more uniform on the screen across browser and operating system platforms and will print, although at a lower resolution than the 
surrounding text. The applet solution does not now allow printing, and there is a 
potential problem with the formatting rectangles: WebEQ truncates its display at 
the rectangle boundaries. The VVizard will allow a bit of padding to try to ensure 
that the estimated rectangle size is sufficient. You can increase this padding via an 
14MathML can also be entered using <math> . . . <\rnath>. 

 
%%page page_254                                                  <<<---3
 
5.2 WebEQ 233 
option (see Section 5.3.1 on page 235 for a discussion of rectangle size problems in 
both techexplorer and WebEQ). 
Other VVizard formatting options allow you to set the initial font size for expressions and background color. Further options determine whether linebreaking 
of long math expressions should occur. 
As an example of the WebEQ VVizard in action, we start with the following 
HTML source that has embedded mathematics: 
<HTML> 
<HAD> 
<TITLE>Example WebEQ Wizard Page</TITLE> 
</HAD> 
<BODY> 
<P> 
The answer to Question 1 is 
<!-- the following fraction will be shown inline --> 
$\frac{x}{y}$ 
and the answer to Question 2 is 
<!-- the following matrix will be shown in display mode --> 
\[\left( \array{2&3 \\ 4&\frac{5}{6}} \right).\] 
</P> 
</BODY> 
</HTML> 
If we choose to create Java applet tags that encapsulate the math input and use the 
default option settings, the following HTML source is generated: 
<HTML> 
<HEAD> 
<TITLE>Example WebEQ Wizard Page</TITLE> 
</HEAD> 
<body bgcolor=#COCOCO> 
<P> 
The answer to Question 1 is 
<!-- the following fraction will be shown inline --> 
<applet code="webeq.Main" width=12 height=32 align=middle> 
<param name=eq value="\frac{x}{y}"> 
<param name=color value="#COCOC0"> 
<param name=parser value=``webtex''> 
</applet> 
and the answer to Question 2 is 
<!-- the following matrix will be shown in display mode --> 
<P><CENTER> 
<applet code="webeq.Main" width=71 height=108 align=middle> 
<param name=eq 

 
%%page page_255                                                  <<<---3
 
234 Direct display of Ié'I];X on the Web 
L V a D:\Sulor\Bonks\Lale>cWebCompanion\e:<amp|e1.htrn| _ 
\reffig{5-17}: An output HTML page generated by the WebEQ VV1zard 
va1ue="\disp1aystyle {\1eft( \array{2&3 \\ 4&\frac{5}{6}} \right).}"> 
<pa:ram name=color va1ue="#COCOC0"> 
<param name=parser va1ue=``webtex''> 
</app1et></CENTER><P> 
</P> 
</BUDY> 
</HTML> 
The final result of WebEQ in action is displayed in \reffig{5-17}. Note the use of 
the a1ign=midd1e attribute setting so that the math objects float to roughly the 
correct position with respect to the text baseline. 
5.3 Embedded content problems and future 
developments 
As we mentioned in Section 5.1.2.1 on page 201, content displayed by plug-ins 
or Java applets embedded within an HTML page can have several rendering problems." These problems all concern the math not blending in seamlessly with the 
surrounding HTML text. In the following sections we discuss these problems and 
suggest possible ways they will be solved in future browsers. 
15 We will refer to this form of content as embedded content, as opposed to that which is shown in the 
full-client area of the browser. 

 
%%page page_256                                                  <<<---3
 
5.3 Embedded content problems and future developments 235 
Comparative display of\ sqrt.( 1 + y-z*2) 
with WID'I‘H=2ClCl and HEIGI-I'I‘=9Cl. 
] techexplurer 2.0 Example W\1V“_§bEQ_ g_.3"Exa_.rn le 
1/1+y-z2 4}/1+y-z2 
\reffig{5-18}: Reasonable sizes for a techexplorer and WebEQ expression 
5.3 .1 Expression size 
Embedded content is displayed within a fixed-size rectangle in an HTML page. 
This rectangle size is explicitly given by the HEIGHT and WIDTH attributes in the 
EMBED element (for plug-ins) or APPLET element (for]ava ap lets). 
Let’s use techexplorer and WebEQ to display ‘/ 1 + y - Z2 within an HTML 
. We’ll use relatively large font sizes for each, along with a HEIGHT value of 
90 pixels and a WIDTH value of 200 pixels. Following is the HTML markup for 
techexplorerz 
<EMBED TYPE="application/x-techexplorer" 
TEXDATA="\pagecolor{white}\(\sqrt{1 + y-z"2}\)" 
WIDTH=200 HEIGHT=90> 
This is the HTML markup for WebEQ: 
<APPLET CODEBASE=``c1asses'' CODE="webeq.Main" WIDTH=200 HEIGHT=90> 
<PARAM NAME=color VALUE=#FFFFFF> 
<PARAM NAME=size VALUE=18> 
<PARAM NAME=eq VALUE="\sqrt{1 + y-z"2}"> 
</APPLET> 
\reffig{5-18} shows these expressions rendered within a table by Microsoft Internet 
Explorer 4.0 on Windows NT. 
\reffig{5-19} shows what happens when we decrease the WIDTH value to 105. 
Now techexplorer has inserted a horizontal scroll bar to let you move left and 
right to view the whole expression. This is effective, but unattractive. The displayed 
math expression does not blend in cleanly with the surrounding elements of the 

 
%%page page_257                                                  <<<---3
 
236 
Direct display ofI1}'1'EX on the Web 
Comparative display of\sqrt( 1 + y-z*z) 
with IJIDTI-I=1CI5 and HE IGI-IT=11C|. 
ltecliexplnrer 2.0 Exampleg|WehEQ 2.2 Example 
X/1+y 1+y 
3% E 
\reffig{5-19}: The effect of decreasing the width of the display rectangle for 
techexplorer and WebEQ 
HTML page. WebEQ truncates the display of the expression on the right. Although 
this might look better, remember that the expression is ‘/ 1 + y - zz, not ./1 + y‘. 
The WebEQ VV1zard (see Section 5.2.4 on page 232) does provide a tool for 
estimating the correct rectangle size for an expression, so this somewhat dangerous 
truncation will occur only if you manually change the width of the rectangle or 
change the size of the font used. A larger font might cause truncation. For both 
techexplorer and WebEQ, a smaller font will cause excessive whitespace around 
the expression, again making the math stand out unattractively on the HTML page. 
The basic problem here is that the rectangle size is determined by the document author and not by the plug-in or applet. It is the rendering engine that knows 
the correct size for the rectangle. Future browsers must negotiate with the plug-in 
or applet to determine the optimal rectangle. 
Note that there may be real constraints on the size of the rectangle, so the 
math rendering software may need to be flexible in how it squeezes into the allowed 
space. WebEQ can line break math expressions, allowing a wide expression to fit 
into a too narrow rectangle. 
5.3 .2 Ambient style 
The final problem we consider concerns the formatting style of math expressions 
versus the surrounding HTML text. In order to have the math blend well with the 
text, the fonts should be the same (or at least work well together), and the font 
and background colors should be the same. Furthermore, if the HTML page uses a 
background image, the math expression rectangles should use the same image and 
be aligned correctly with the page background. \reffig{5-20} illustrates several of 
these problems. 

 
%%page page_258                                                  <<<---3
 
5.3 Embedded content problems and future developments 
237 
3 D \SuIm\Bonks\Lalm¢Webljompan|on\slyle hlml Mll'.‘l(}$Ull Inlenn 
I9’ . 
\reffig{5-20}: Some style matching problems 
The style information for the HTML page was given by the Cascading Style 
Sheet: 
P { 
font-family: sans-serif; 
} 
BODY {i 
background: silver; 
} 
Note that the math expression is rendered on a white background in a Roman-style 
font. 
Both techexplorer and WebEQ render their math opaquely on the screen: Everything in the window is overwritten, including the background. While it is possible to build plug-ins so that they write transparently over the screen, the fontand 
text-color problems remain. Furthermore, it is not sufficient to specify background 
color and font information in the markup for the mathematics, although this might 
look correct on the screen. This information should be obtained automatically by 
the plug-in or applet from the browser. Otherwise, every time you update the style 
for your HTML page, you need to go in and fix all the math expressions individually. 
Future browsers will share ambient style information with software that renders embedded content. As we move from HTML to XML documents and develop 
new programming interfaces for embedded, possibly nested, content renderers, the 
browser formatting facilities and information will become more widely accessible 
to other sofware used to display parts of the document. 

 
%%page page_259                                                  <<<---3
 
CHAPTER 6 
HTML, SGML, and XML: 
Three markup languages 
This chapter provides an insight into the relation between Standard Generalized 
Markup Language (SGML), the parent of all present-day nonpropriety markup languages, HyperText Markup Language (HTML), the lingua franca of the Web, and 
Extensible Markup Language (XML), a simplified version of SGML, that lies at the 
heart of a whole new family of applications optimized for use on the Web. 
We first explain why we think HTML cannot be the final answer for information interchange on the Web. Then we say a few words about SGML and its history 
to put the recent XML proposals into perspective, before we take a close look at 
XML. We describe its various components, including the Document Type Definition (DTD), and review a few of the existing tools to handle XML documents. 
6.1 Will HTML lead to the downfall of the Web? 
The reason HTML is so popular has much to do with its intrinsic simplicity (it is 
easy to learn), as well as its many nonstandard extensions that are offered by the 
various browser vendors to help users make their pages look profixxional and attractive. However, this Tower of Babel of incompatible extensions is a real threat to the 
integrity of the Web, since it kills the universal availability of the information. 
Most people love HTML because it is a clean little language that they can master in an afternoon. HTML is universal and runs in browsers everywhere. Moreover, 

%==========260==========<<<---2
 
%%page page_260                                                  <<<---3
 
240 
HTML, SGML, and XML: Three markup languages 
many tools come with an HTML back-end. However, in the real world one is often 
confronted with broken links and a lack of portable ways to format the information. 
Many of us have had to (mis)use tables, frames, Java, and other scripts to get the 
presentation we like, because of the lack of a real tool to craft universally displayable 
Web pages. 
It is probably worthwhile to look at the problem areas where we think HTML 
could be improved. 
o Invalid HTML Many commonly used utilities produce invalid HTML, or they 
introduce vendor-specific extensions. Most users do not validate their HTML 
source code, and browsers do not object to invalid HTML; most of the time 
they just skip the information that does not make syntactical sense. This makes 
it especially difficult to get consistent results between Web browsers and across 
computer platforms. 
o Broken link: Whenever a Web page is deleted or moves to a different host, all 
URL references to that page are invalidated. There has been talk about Uniform 
Rexource Name; (URN, see Section 1.1.2 and [h>URNIETF]) that would address 
s by name and provide a level of indirection to cope with mapping names 
on physical addresses (just like name servers for Internet addresses). 
a Fixed grammar The element and attribute set of HTML is fixed. HTML is said 
to have a fixed grammar, as described by a Document Dpe Definition (DTD), 
a formal specification that describes the syntax of an SGML application (see 
Sections 6.3.2 and 6.4.3). Thus one cannot adapt the language to cope with 
a specific set of new applications or extend its functionality to deal with new 
Web technology, except by extending the DTD. In the past, browser vendors 
have added their own extensions, resulting in Web pages optimized for a single 
browser, with other browsers unable to display the information fully. More 
recently with CSS, one can “extend" the visual presentation of HTML’s fixed 
tag set by using the class attribute and the span and div element types (see 
Section 7.4.1.4). 
0 Limited rapport fior metadata Only primitive support exists for metadatainformation describing the contents of the document, such as keywords, author, and data. The <meta> tag is a step in the right direction, but there is no 
standard way for putting it to work; user agents can just ignore it. Therefore 
it remains nontrivial for search engines to extract important key information 
about the source document. 
0 Alzxence of xtructural tag: Although HTML tags, such as <h [1-6] >, <div>, and 
<p>, could be used to structure the information, most Web applications and 
Web authors ignore this possibility, and use HTML tags merely for controlling the visual layout of the document. This unstructured approach makes it 
difficult to navigate through a tree (or network) of documents. 

 
%%page page_261                                                  <<<---3
 
6.2 HTML 4: A richer and more coherent language 
241 
o Data excbange difiiculties Because of its closed tag set aimed at presenting information on the Web, it is almost impossible to extract data according to 
tagged data fields. Moreover, only the Latin 1 character set, which does not 
even support Western European languages fully, is generally available. The use 
of HTML for other languages is based on extensions, preventing easy document interchange. Just think how you would view pages written in Russian or 
Japanese if your browser did not have Cyrillic or Kanji fonts or the right encoding. 
0 Absence of modern features As with any standard (even one coordinated by the 
Web Consortium, which responds relatively quickly to common practice), 
many modern ingredients are lacking. Among these are ways of refreshing information on the client side, exposing information present in dynamic entities, 
such as applets, and the unavailability of an object model. 
Recent work has tried to address one or more of these problems. One approach was to increase the functionality of HTML, and, therefore, HTML 4 was 
developed. To separate form and content better, the style sheet language, Cascading 
Style Sheets (CSS), was recommended. The Extensible Markup Language (XML) effort deals with application specificity and better data organization. Dynamic HTML 
(DHTML) goes some way toward adding a dynamic representation to Web pages. 
The standardized cross-language, cross-product version of DHTML is called the 
Document Object Model, DOM for short. DOM is bound to play an important role by 
allowing programs to access HTML (XML) elements as a structured collection of 
object data, each having a set of properties and methods. 
We look at some of these developments in this and later chapters. Bear in mind, 
however, that many new features are implemented only partially in the current 
generation of browsers. It will take some time to make even the more important 
browsers conform to the specifications and standards we will be describing. 
6.2 HTML 4: A richer and more coherent language 
On December 18, 1997, W3C issued HTML 4.0 as a W3C Recommendation, somewhat equivalent in status for the Web to an ISO or ANSI standard. The HTML 4 
specification is a document of over 360 pages and is available as an HTML 
PostScript or PDF file at [9->HTML4].1 Although HTML 4 addresses a few of the 
shortcomings listed in Section 6.1, it still offers only a fixed tag set and provides no 
generic method to tailor the markup language to a particular application. That is 
why HTML 4 will be the last of the existing generations of HTML standards; future 
work will be concentrated on building a more modular XML-based approach (see 
Section B.5). 
1It is likely that you will find a slightly revised version of the HTML 4 specification at that URL. At 
the time of writing the reference number of the document was REC-html40-19980424. 

 
%%page page_262                                                  <<<---3
 
242 
HTML, SGML, and XML: Three markup languages 
6.2.1 HTML 4 goodies 
Following are some of the more significant changes in HTML 4 with respect to the 
previous version 3.2 (released in January 1997): 
o A more complete model for tables (based on the CALS2 DTD). 
o A first step toward a clearer separation between content and form with the deprecation of element and attributes that control presentation (such as color and 
font size) and their replacement by Caxcading Style S/Jeetx (see the CSS Specification [L->CSS2] and Chapter 7). 
Any element can be identified by a (unique) ID attribute. It can be addressed as 
a destination anchor of a link, as shown in the following example: 
<H2 id=``mysect''>This is a uniquely identified section heading. 
<P id=``mypara''>This is my addressable paragraph. 
<P>As stated in a <A HREF=“#mypara“>paragraph</A> which 
is part of a <A HREF="#mysect">section</A>... 
Support for internationalization, by introducing language codes (see Table C.l 
on page 466) and making it possible to specify the writing direction, will make 
it easier to generate documents in almost any of the world’s languages. Making 
them universally readable is, of course, a huge software problem. 
Generic objects (images, applets, and other documents) can be embedded with 
the OBJECT element. If a given resource is not available, then another can be defined to run instead. The following HTML code first tries to execute a Python 
applet featuring electrons circulating in the LEP accelerator (lines 1-2). If that 
is impossible, showing an MPEG movie will be attempted instead (line 3), or 
else a static GIF image will be shown (line 4). Finally, if all that fails, a text string 
will be printed (line 5). 
<0BJECT title=``Electrons going round and round'' 
classid="http://www.cern.xxx/CirculatingElectrons.py"> 
<0BJECT data="CirculatingElectrons.mpeg" type="application/mpeg"> 
<0BJECT data="CirculatingElectrons.gif" type="image/gif"> 
Electrons circulating in the LEP tunnel. 
</0BJECT> 
</0BJECT> 
</0BJECT> 
2 CALS stands for ContinuousAcquz':ition and Lzfe-Cycle Support, a U.S. Department of Defense strategy 
for achieving effective creation, exchange, and use of digital data for weapons systems and equipment. 
CALS, adopted by at least half a dozen other countries’ military institutions, has been instrumental in 
promoting SGML as a markup language in general. More information is available starting from the CALS 
home page at [%>CALS]. 

 
%%page page_263                                                  <<<---3
 
6.3 Why SGML? 
243 
Note that the OBJECT element type replaces (and thus deprecates) elements 
such as APPLET and IMAGE. This allows a much cleaner and more generic approach to the handling of events, files, viewing pictures, and so on. 
0 Some more advanced features have also been introduced. They include media 
a'excriptorx, which allow the use of device-sensitive style sheets, and event attrilzatex, which, in conjunction with scripts, allow code to be executed when a 
given event occurs (for instance, when a document is loaded, or the mouse is 
clicked). Another innovation is the DIV and SPAN elements, which, when used 
with ID and CLASS attributes and style sheets, present authors with a generic 
mechanism for “extending" HTML by tailoring it to their needs and tastes. 
Currently no browsers fully support HTML 4, although Netscape version 4 
and MS Internet Explorer versions 4 and 5 go some way in the right direction. 
Moreover, to benefit fully from the possibilities of HTML 4, support for Cascading 
Style Sheets (CSS version 2) is a must, and browsers still need a lot of work here to 
become fully conforming. 
A description and comparison of the different versions of HTML were prepared 
by Ian Graham using a modified version of Earl Hood’s dtd2html and dtddiff 
programs (see Section 6.6.2). It can be found at [G->HOOD]. 
6.2.2 HTML 4, the end of the old road 
On June 22 , 1998, the W3C Consortium issued an activity statement that deals with 
the future of HTML [L->W3CFUTURE]. This issue had been debated by specialists 
during several workshops. It was generally felt that a completely new start should be 
taken by building a new generation of HTML, based on a genuine XML tag set and 
built in a modularized way. This would make the language more manageable and 
would provide a straightforward path to integrate HTML and already existing XML 
applications. See Section B.5 .2 for more details about XHTML, a reformulation of 
HTML as an XML application. 
6.3 Why SGML? 
Since the early 1980s we have witnessed an ever-quickening transition from book 
publishing, exclusively on paper, to various forms of electronic media. This evolution is merely a reflection of the fact that the computer and electronics have made 
inroads into almost every facet of human activity. In a world in which we have to 
deal with ever-increasing amounts of data, we depend more and more on the computer for preparing telephone directories, dictionaries, and law texts-to mention 
just a few examples. However, it is not just the volume of the data that is important, 
but also the ease with which it can be entered, maintained, viewed, exchanged, and 
distributed. 

 
%%page page_264                                                  <<<---3
 
244 
HTML, SGML, and XML: Three markup languages 
Once data has been stored in electronic form, one can derive multiple “products" (or “views") from a single source document. For instance, an address list can 
be turned into a directory on paper, put on CD-ROM, made available as a database 
to allow interactive or e-mail access on the Internet, or used to print a series of 
labels. Similarly a set of law texts or a series of articles on history marked up in a 
generic language can be published as a textbook containing complete law texts, or it 
can be used as the basis for a historic encyclopedia. Thanks to the generic markup 
strategy, it is straightforward to provide regular updates or to extract a subset of 
articles on a given subject. From the same electronic sources one can also offer a 
consultation service on the Internet, via gopher or W, or develop a hypertext 
system on CD-ROM. 
All of these applications suppose that the information is not saved in a format 
that is suited only for display or printing (for example, using a WYSIVVYG-oriented 
system). The hierarchical structure and logical relations between the various document components should be clearly marked. This approach, which forms the basis 
of the BTEX and SGML/XML-based languages, has the following strong points: 
a The quality of the source document is improved by making data entry easier, 
increasing readability, and allowing input to be validated more fully. 
0 Documents can be maintained more rationally, resulting in an improved life 
cycle. 
0 Publishing costs are reduced. 
0 Reuse of information is easier, thus adding value to documents (they can be 
printed, presented as hypertext, stored, and accessed in databases). 
6.3.1 Different types of markup 
Today every PC comes with a text processor, mostly of the WYSIWYG (what you 
see is what you get) type. Therefore many users consistently confuse information 
and document structure with presentation by associating formatting characteristics 
with various textual document components. 
This situation is similar to that prevailing in the 1970s with early formatting 
languages; specific codes were mixed with the (printable) text of the document in 
order to control typesetting at the micro level. For example, line and page breaks 
and explicit horizontal or vertical alignments or skips were specifically marked to 
compose the various pages. Most of the time these control characters were extremely application specific, and it was almost impossible to reuse sources marked 
up in one of these systems with any of the others. Nevertheless, this type of markup 
allows very precise control over the physical representation of a specific document 
and provides important advantages for fine-tuning the final layout for viewing and 
printing documents. An example of specific page markup is the following TEX 
fragment. It starts a new page and typesets a chapter title in a given hard-wired 

 
%%page page_265                                                  <<<---3
 
6.3 Why SGML? 
245 
way (large and boldface font, with the word “Chapter" and the number “2" handcoded). 
\vfil\eject 
\par\noindent 
{\large\bf Chapter 2: Title of Chapter} 
\par\vskip\baselineskip 
It should be clear that modifying a document that contains such explicit markup 
or trying to guess the “meaning" of the commands is extremely tedious. Thus a 
document planned for modification and targeted for multiple use must be marked 
up so that its logical structure and its physical representation are clearly separated. 
In logical or generic markup, the logical function of all document elementstitle, sections, paragraphs, figures, tables, bibliographic references, or mathematical equations-as well as the structural relations between these elements, must be 
clearly defined. ETEX is an enormous step forward in the right direction, as shown 
in the following code fragment, where a single line signals the beginning of a new 
chapter: 
\chapter{Title of Chapter} 
It is up to the “style" (or “class") specification to decide how a chapter gets started 
and how its title is typeset. At the same time, the chapter numbering and entering 
the chapter title into the table of contents, if desired, is under the global control 
of the style author. We merely have to specify the logical presence of the chapter 
together with its title. 
A similar approach can be seen in the following HTML fragment, although 
it must be emphasized that HTML is now primarily used as a presentation markup 
language. For instance, in the example that follows, the <H1> tag is usually used only 
to mark the presence of a “title heading of level 1" without implying the beginning 
of a new document section. 
<H1>Title of Chapter</H1> 
<P> 
Only recently has a higher level of abstraction for the relation between markup 
tags and visual presentation in HTML become possible via CSS style sheets. (This 
concept will be explained in Chapter 7; in particular see Section 7.4.) 
6.3.2 Generalized logical markup 
Several document instances can belong to a same document “class," since they have 
the same global logical structure. As an example let us consider two articles, A and 
B, with the explicit structure shown in \reffig{6-1}. 

 
%%page page_266                                                  <<<---3
 
246 
HTML, SGML, and XML: Three markup languages 
Article A Article B 
Title Title 
Section 1 Section 1 
Subsection 1.1 Subsection 1.1 
Subsection 1. Subsection 1.2 
Section 2 Subsection 1.3 
Section 3 Section 2 
Subsection 3.1 Subsection 2.1 
Subsection 3.2 Subsection 2.2 
Subsection 3.3 
Subsection 3.4 
Bibliography Bibliography 
\reffig{6-1}: Two instances of an article class 
It is evident from \reffig{6-1} that both articles are built according to the same 
logical pattern: a title, followed by one or more sections, each one subdivided into 
zero or more subsections, and a bibliography at the end. In IATEX we would say 
that the document instances belong to the same document class “article." 
To exploit the advantages of structured documents fully, the markup scheme 
must adopt a clear set of rules. VVhen using SGML, these rules are set down in 
the Document Type Definition (DTD). A DTD not only defines the allowed element types (the syntax), but it also describes the structural relations between the 
elements. A parser then checks whether marked up documents adhere to the DTD. 
This approach is somewhat similar to IATEX, where the syntax (the document 
markup) is defined in Lamport’s IHEX Manual; however, the underlying TEX implementation does not in most cases verify whether the elements are nested in an 
“allowed" way. For instance, you can use a subsubsection without explicitly defining a section or a subsection, just like in HTML. If the DTD is built correctly, this 
kind of ambiguity should not be possible with SGML. Moreover, in BTEX you can 
extend the language by defining new commands and environments; this also is not 
(practically) feasible in the case of SGML. 
Therefore for our example documents of type “article" the DTD should define 
elements for “title," “section," “subsection," and “bibliography." It should also express the fact that the title precedes sections that can contain subsections and that 
articles can have a bibliography at their end. The DTD assigns a name to each structural element, often an abbreviation that conveys the function of the element in 
question (for example, “sec" for a section (line 3), and “st it" for a section or subsection title (lines 3 and 5)). Using the DTD as defined, you can then start marking 
up the document source itself (article A or article B); use the “short" names defined 
for each document element. For instance, with “sec" one can form a start tag <sec> 
for marking the start of a section (line 3) and an end tag </sec> to mark its end 
(line 8); other document components operate in a similar fashion (see Section 6.4.2 
for more details). 

 
%%page page_267                                                  <<<---3
 
6.3 Why SGML? 
247 
<article> 
<tit>SGML and XML</tit> 
<sec><stit>Why SGML'?</stit> 
1 
2 
3 
4 <para> ... </para> 
s <ssec><stit>Different types of markup</stit> 
5 <para> . . . </pa1a,> 
7 </ssec> 
s </sec> 
9 </article> 
6.3.3 SGML to HTML and XML 
The idea that structured documents could be exchanged and manipulated if published in a standard open format dates back to early initiatives in the 1960s. In one 
endeavor a committee" of the Graphic Communications Association (GCA) created 
GenCode to develop generic typesetting codes for clients to send data to companies 
to be typeset using different printing devices. GenCode allowed them to maintain 
an integrated set of archives despite the records being set at multiple sites. 
In another effort, IBM developed the Generalized Markup Language (GML) for 
its big internal publishing problems, including managing documents of all kindsfrom manuals and press releases to legal contracts and project specifications. GML 
was designed to be used by batch processors to produce books, reports, and electronic editions from the same source file(s). 
GML provided a “simple" input format for typists, including elements of a tag 
syntax that we still recognize today. GML was optimized for speed of data entry. 
Moreover, since only a few types of documents existed, specific programs were developed to interpret the data tags for a particular document type---hardly a generic 
approach. Hence, with the advent of more types of documents, the Gencod and 
GML communities got together and started talking about standardization within 
the framework of ANSI, the American National Standards Institute. The ideas of 
DTD, markup, and syntax were formalized, and SGML was born. It was adopted as 
an ISO standard in 1986 (ISO:8879, 1986). 
SGML is a complex standard, and its use remained limited mainly to the spheres 
of large companies or organizations and a few research institutes. For instance, Anders Berglund, who was for many years responsible for text processing at CERN, 
the European Laboratory for Particle Physics (Geneva, Switzerland), introduced 
a prototype version of SGML (based on IBM’s GML) many years before the standard was even published. Thus it should come as no surprise that while working at 
CERN, Tim Berners-Lee, the inventor of the Web, was influenced by the look and 
feel of SGML while defining the syntax of HTML. His prime aim was not (at least 
originally) to comply to a formal DTD but to allow his early browser software to 
render the source material in a straightforward way. 
It took some years before HTML developers like Dan Connolly and Dave 
Raggett recognized the need to give HTML a firmer (and more formal) basis. The 
first HTML DTD was developed, and HTML was turned into a conforming SGML 

 
%%page page_268                                                  <<<---3
 
248 
HTML, SGML, and XML: Three markup languages 
application, making it possible to validate HTML documents formally against an 
HTML model represented by the DTD. 
HTML, in the form of HTML 4, the present recommendation as discussed 
briefly in Section 62, still offers only a limited set of elements to mark up documents. It should also be clear that HTML, even extended, will never be able to 
represent all documents that people want to keep on the Web or, more generally, 
in electronic form. 
On the other hand, because SGML has a rather large and complicated specification, it is not easy for an author to master all of its details or for a computer 
program to parse and manipulate complex SGML documents. Nevertheless, if we 
want to treat electronic documents in an optimal way, we must mark them up generically to indicate their logical structure, and we must choose, for that process, an 
agreed standard language to guarantee interchangeability. So what can we do with 
HTML being considered too limited and SGML too complex and without an agreed 
set of “standard" DTDs that are generally available? 
In his seminal paper, XZVIL, java, and tbe future of tbe VVeb ['->BOSAKXML], Jon 
Bosak points out three areas where HTML has severe shortcomings: extensibility 
(being able to define new elements and attributes for a document instance), the 
possibility of deep structure (allowing arbitrary nesting), and validation (checking of 
data before use). 
6.4 Extensible Markup Languages 
Even though HTML 4 is without doubt a step in the right direction if one wants 
to support the Web in a standard way, it is still too limited and too static to cope 
with all of the Web’s many application areas (databases, search engines, optimal 
presentation, professional printing, and data verification). 
Consequently, in the middle of 1996 the Web Consortium set up an SGML 
Working Group to tackle these problems. Under the chairmanship of Jon Bosak, 
the group developed the Extensible Markup Language (XML), which was finalized 
at the end of 1997 and issued as a W3C Recommendation on February 10, 1998 
(Extensible Markup Language (XZVIL) 1.0 ['->XMLSPEC] and errata ['->XMLERRATA]). 
XML is designed as a subset of SGML,3 so that any XML document is also a conformant SGML document. Existing SGML parsers and systems in general can be used 
with XML. Work to allow an augmented set of hypertext possibilities (see XZVIL 
Linking Language (XLink) ['->XLINKSPEC] and XML Pointer Language (XPointer) 
['->XPTSPEC]) and to define an XML style language XSL (see Extensible Stylesbeet 
Language XSL ['->XSL97]) is going on at present. The XSL language will be the 
subject of Section 7.6. 
3A few extensions to the original 1986 SGML ISO standard (ISO:8879 (1986)) were necessary. These 
were grouped in Annex K (l/Veb SGML Adaptations) and Annex L (Additional Requirements for XML), 
see [=>ISO8879TC2]. 

 
%%page page_269                                                  <<<---3
 
6.4 Extensible Markup Languages 
249 
A good source of information about XML is Peter F lynn’s “Frequently Asked 
Questions about the Extensible Markup Language" ['->XMLFAQ]. Other interesting 
Web sites are Elliotte Rusty Harold’s “Cafe con Leche XML News and Resources" 
[v->LECHE] and “SGML and XML News" ['->SGMLNEW], which is maintained by 
Robin Cover who also coordinates the “Extensible Markup Language (XML)" page 
['->XMLPAGE]. This very useful page collects a lot of information concerning XML. 
Pointers to a set of introductory articles on XML are available at ['->XMLINTRO]. 
These articles provide interesting reading if you are starting with XML. For those 
who are still fond of reading books, quite a few XML books have recently been published. Bradley (1998), Goldfarb and Prescod (1998), Harold (1998),]elliffe (1998), 
Leventhal et al. (1998), McGrath (1998), Megginson (1998), and St. Laurent (1997) 
are among the ones we have consulted. 
6.4.1 What is XML? 
The W3C recognized the fact that SGML’s scope is very broad and the language 
rather complex, both to learn and to implement. Therefore W3C decided to introduce a lightweight version and designed XML as a subset of SGML, doing away 
with its rarely used and more complex features. It is said that XML offers about 90% 
of SGML’s functionality at some 10% of its complexity, thus making sure that the 
ten commandments of XML (its design goals as specified by the W3C SGML Special 
Interest Group when they started their activities) can be fulfilled. 
6.4.1.1 XML’s ten commandments 
The XML Specification sets out the following goals for XML: 
1. XML shall be straightforwardly usable over the Internet. 
XML source documents must be viewable as quickly and easily as HTML documents 
(when XML-capable browsers and applications are available). 
2. XML shall support a wide variety of applications. 
XML is not only optimized for browsing but must allow for a whole realm of applications 
not limited just to the Web but more general. Application areas include convenient 
authoring, data presentation, content analysis and validation, and databases. 
3. XML shall be compatible with SGML. 
Since XML flli initio was defined as a “convenient" subset of SGML by people who were 
historically involved with the SGML effort, we can be confident that XML and SGML 
will interoperate without problems. 
4The number of books on XML that appeared in twelve months is far larger that the number of 
books published on SGML during the last twelve years-one more proof that XML is an important 
development. 

%==========270==========<<<---2
 
%%page page_270                                                  <<<---3
 
250 
HTML, SGML, and XML: Three markup languages 
10. 
It shall he easy to write programs that process XML documents. 
Given the expertise of the SGML community that knew which parts of SGML are difficult to implement and, therefore, could be eliminated, we can be sure that XML can be 
implemented in “a couple of weeks" by the average computer science student. In fact, 
as we will see later, more than ten XML parsers are freely available today. 
The number of optional features in XML is to he kept to an ahsolute minimum, ideally 
kept to zero. 
It goes without saying that optional features are never really optional: Whenever somebody starts using them, everybody has to implement the code to interpret those features. Therefore to minimize incompatibilities and confusion, no optional features are 
allowed in XML, and all XML parsers in the world should be able to interpret all XML 
documents. 
XML documents should he human-legihle and reasonahly clear. 
Experience has shown that it is always a big plus when you can read the contents of a 
document without having to put it through a dedicated program to decode it. It is so 
much more user-friendly to be able to display a document directly on screen or to make 
a small modification with your favorite editor. 
The XML design should he prepared quickly. 
In the past, browser vendors kept adding incompatible extensions to their programs. 
W3C was trying to follow these developments by putting out more and more complex 
HTML specifications. Nevertheless, as we saw earlier, they were unable to solve the 
real problems, so it was important to come up with a solution immediately. The XML 
Working Group did a marvelous job in minimal time. 
The design of XML shall he fiormal and concise. 
Point 4 wants XML to be easy to program. Here we state that the language itself should 
be easy to describe formally, thus allowing it to be analyzed by simple computer techniques and, at the same time, easy to master by the average document programmer. 
XML documents shall he easy to create. 
Intelligent editors or dedicated applications should not have too hard a time generating 
correct XML data. Also, XML should be tractable to write authoring systems. 
Yerseness in XML marleup if of minimal importance. 
To ease the implementation of all the previous tasks and to make XML more readable 
b arsers and humans , no minimization of marku is allowed. Conciseness alwa s 
Y p _ p Y 
comes second to clarity. 
6.4.1.2 XML opens a new window on the Web 
From what has been said earlier, it should be clear that XML’s goal is to go beyond 
being a “super-HTML," by becoming a genuine “lightweight SGML for the Internet." XML opens a completely new window on the Web: It allows designers and 

 
%%page page_271                                                  <<<---3
 
6.4 Extensible Markup Languages 
251 
programmers to present their data in a number of different ways by using embedding programming techniques within the standard syntax of XML. In principle they 
no longer need to combine various techniques and languages, such as HTML,]ava, 
scripting, Perl, CGI, ActiveX, and other tools and plug-ins. They all can be hidden 
via an XML abstraction level. 
As already mentioned, each element of a valid XML document must be declared 
in a DTD. This provides a formal definition for the XML language of the document 
class being considered. This also allows XML parsers to check the validity of document instances marked up according to that DTD, verifying, for instance, the correct nesting levels, whether all document components have been defined, and so on. 
Note, however, that strictly speaking the XML specification does not require that a 
DTD be present. For browsers, it could be too time-consuming for each document 
to download and parse a DTD and check the document against this DTD. Ideally 
XML applications should make sure at creation time that all documents adhere to 
a DTD, so that browsers can assume that they are correct. XML requires only that 
the document be well-fiormed (see Section 6.4.2.1). 
6.4.2 The components of XML 
XML is based on the concept that documents are composed of a series of entities 
(today it would probably be more fashionable to call them objetts). Each entity contains one or more elements, and each element can be characterized by zero or more 
attributes (properties) that describe the way in which each element is to be processed. The relationships between elements and the list of their possible attributes 
are specified in the DTD. 
The beauty of XML (SGML) is that, using this mechanism of defining a language with a DTD, each institute, group, company, organization, and so on, can 
define its own language for all the different kinds of documents they have to deal 
with. By being able to choose user-friendly markup tags adapted to a particular application domain or cultural environment, the use of these tags will be much easier 
to comprehend, and the markup error rate will be substantially lower than when a 
more generic markup scheme is used. Moreover, with the help of intelligent editors, 
which will hide the markup or guide the user by allowing only tags possible in the 
current context, it will be trivial to compose syntactically correct documents. 
Although SGML’s reference concrete syntax proposes certain characters to represent delimiters, and so on, these characters can nevertheless be chosen freely. 
Moreover, various parameters of an SGML instance can be defined at the document 
level. The character set (by default ASCII) can also be declared in the document 
instance. This complexity makes it difficult to write parsers, and, therefore, it was 
decided to allow only a fixed syntax for XML tags, entity references, and so on. 
Elements and their attributes, if any, are entered between matched pairs of angle brackets (< . . .>) with attribute values always between a pair of single or double 
quotes: 
<ename attr1=``val1'' attr2=’val2’ ...> 

 
%%page page_272                                                  <<<---3
 
252 
HTML, SGML, and XNIL: Three markup languages 
Entity references start with an ampersand and end with a semicolon: 
&eref; 
Element, attribute, and entity names are all m.ve~.vemitive; thus <HEAD>, <Head>, 
and <head> are tags corresponding to three different element types, while <head 
lev=``val'' Lev=``val2''> shows two different attributes-lev and Lev-for the 
head element type. It goes without saying that one should use the possibility to 
exploit such subtle differences with great care, so as not to confuse the user of 
your tag set. A simple rule, like “all lowercase," will certainly contribute to making 
the use of your DTD less error-prone. As for entities, such a constraint does not 
necessarily exist, since, for instance, &Aacute; and &aacute; may generate A and 
a, respectively (in the Latin 1 entity set). 
Comments are specified between < ! -- and -->. 
<!-- Inside a comment you can write <e>&</e> --> 
By construction XML is a subset of its parent-language SGML, and all software 
written to parse, validate, or otherwise handle SGML should also work with XML 
files.5 
An example of correct XML syntax is the following trivial document: 
<coolxml>XML is a cool idea!</coolxml> 
This XML document cannot, as such, be validated since no DTD is specified. It is, 
however, well-formed and complete. 
6.4.2.1 Valid and well-formed documents 
At the end of Section 6.4.1.1, we mention that XML documents should be wellfiormed but that they can also be valid. Let us go into a little more detail. 
To be well-fi2rmed, an XML document must follow a set of simple rules that 
enable the XML processor to parse the file correctly. 
First, there must exist a root element that encloses the entire document instance 
and does not appear as contents of any other element (docu on lines 1 and 6 of the 
example that follows). All tags must be balanced; that is, all elements must have both 
start and endtags present (<e11>. . .</el1>), unless one is dealing with an empty 
element that may use the notation <. . ./> (a slash preceding the closing bracket, 
such as <empel/> on line 4, although the more verbose syntax <empe1></empe1> is 
also possible). Tags must be properly nested (e12 starts and ends inside e11 in the 
example). All attribute values, which must necessarily be of type CDATA (character 
data), have to be enclosed inside quotes (as on line 2), and the < and & characters 
must be escaped as 8511:; and &amp;, respectively. A simple well-formed document 
follows: 
5 This is strictly true only for software that complies to the I/Veb SGML Adaptatiom[h>ISO8879TC2]. 

 
%%page page_273                                                  <<<---3
 
6.4 Extensible Markup Languages 
253 
1 <docu> 
. <el1 att=``val''>... 
<el2>...</el2> 
. . .<empel/> 
</el1> 
5 </docu> 
v:.g\~I~a 
In Section 6.6.5 we present documents that are not well-formed to various 
XML parsers and study how the generated error messages can help us correct mistakes. 
Valid XML documents are well-formed and must adhere to a DTD. A valid 
XML document must specify in the document declaration part at the beginning of 
the document instance which DTD has to be used and how the XML processor can 
access it on the local system or via the network. In the example that follows we 
show how the docu document type is declared by specifying an URL: 
1 <?xml version="1.0"?> 
2 <!DOCTYPE docu SYSTEM "http://www.bla.org/mydocu.dtd"> 
3 <docu> 
4 ... <el1>...<empel/>...</el1> 
5 </docu> 
A DTD (in full or in part) can also be included inside the document declaration 
part of the document. If everything is present inside the document instance itself, 
then we can declare the document “standalone." 
<?xml version="1.0" standalone=``yes''?> 
<!DOCTYPE coolxml [ 
<!ELEMENT coolxml (#PCDATA)> 
]> 
<coolxml>XML is a cool idea!</coolxml> 
u..sw~ 
Thus by adding a DTD, we have turned the well-formed trivial document introduced at the end of the previous section into a valid one. 
The distinction between well-formed and valid documents makes it easier to 
serve XML documents over the Internet without burdening the “consumer" application (such as a browser or a database query). The client application can assume that the document has been validated on the server-side. Thus we need to 
check only whether the document is well-formed, that is, whether the document 
was not corrupted during the transfer, for example. Of course, the client is free 
also to validate the document. One important point is that XML-conformant applications should report a fatal error and stop processing data for the client when 
presented with a document that is not well-formed, although they can continue 
finding errors. This is in contrast to the behavior of many HTML applications that 
today mostly ignore errors in HTML documents and proceed as though nothing 
were wrong, thus leading to possible problems later on. 

 
%%page page_274                                                  <<<---3
 
254 
HTML, SGML, and XML: Three markup languages 
6.4.2.2 A more complex document 
Let us now be a little more ambitious and define a language to compose texts for 
sending invitations to our friends. We could write something like the following: 
;:S\‘».'7.:$i3:5~oaa\Io~v->w~>-(invitation) 
<to>Anna, Bernard, Didier, Johanna</to> 
<date>Next Friday Evening at 8 pm</date> 
<where>The Web Cafe</where> 
<why>My first XML baby</why) 
(par) 
I would like to invite you all to celebrate 
the birth of Invitation, my first XML document child. 
</par) 
<par> 
Please do your best to come and join me next Friday 
evening. And, do not forget to bring your friends. 
</par> 
<par> 
I really look forward to see you soon! 
</par> 
<signature>Michel</signature> 
</invitation) 
This document is clearly marked up. All elements are delimited by start and 
end tags, they are properly nested, and there exists an outermost root element 
(invitation). Thus our document is truly well-formed. Nevertheless, there remains 
at least one shortcoming to this document; namely, its structure is hard to guess. 
We have merely indicated the semantic function of a few text strings, but it is not 
clear what the relation between the various document components is. 
To clarify the relation between the various document elements we subdivided 
our document into three parts: front, body, and back, corresponding to the introductory information, the message text itself, and the closing part, respectively. We 
also emphasize a few words in the text by bracketing them with <emph>. . . </emph> 
tags. Some comments were added as well. 
NNN-.-............_-...._.....~>--o~ooo\noxv-«nw~._oom\no\»..nwN»-<?xml version="1.0"?> 
<!DOCTYPE invitation SYSTEM "invitation.dtd"> 
<invitation> 
<!-- ++++ The header part of the document ++++ --> 
<front> 
<to>Anna, Bernard, Didier, Johanna</to> 
<date>Next Friday Evening at 8 pm</date) 
<where>The Web Cafe</where) 
<why>My first XML baby</why) 
</front> 
<!-- +++++ The main part of the document +++++ --> 
(body) 
(par) 
I would like to invite you all to celebrate 
the birth of <emph>Invitation</emph>, my 
first XML document child. 
</par> 
(par) 
Please do your best to come and join me next Friday 
evening. And, do not forget to bring your friends. 
</par> 
<par> 

 
%%page page_275                                                  <<<---3
 
6.4 Extensible Markup Languages ' 255 
23 I <emph>rea11y</emph> look forward to see you soon! 
24 </par> 
</body> 
<!-- +++ The closing part of the document ++++ --> 
<back> 
28 <signature>Miche1</signature> 
29 </back> 
30 </invitation) 
Nanak: 
\-mu. 
It is important to note that up to now we have said nothing about how this document should be rendered. The XML instance shown describes only the information and how its various structural elements are related. How an XML application 
handles this data is not specified. You can execute any action when encountering any 
of the tags in the document. You can render their content on an output medium, 
store it in a database, transform or combine it with other information, and so on. 
In Chapter 7, we study in detail how style languages (CSS, DSSSL, or XSL) allow us 
to transform XML information into a printable or viewable format. 
6.4.3 Declaring document elements 
In the earlier example we introduced a little language to mark up invitations in 
a convenient, clear, and easily processable way. If we want XML applications to 
validate documents written according to that scheme, we formally have to define 
our language. As explained earlier, this is done with the help of the Document 
Type Definition (DTD). The DTD formally defines the grammar of the language; 
in other words, it describes the structural relationship between the elements and 
their possible attributex. In the case of our invitation language, we could define 
the following DTD: 
1 <!-- invitation DTD --> 
2 <!ELEMENT invitation (front, body, back)> 
3 <!ELEMENT front (to, date, where, why’?)> 
4 <!ELEMEN'l' date (#PCDATA)> 
5 <!ELEMENT to (#PCDATA)> 
6 <!ELEMEN'l' where (#1-’CDA'l'A)> 
7 <'.ELEMENT why (#PCDA'l'A)> 
8 <!ELEMEN'l‘ body (par+)> 
9 <!ELEMEN'l‘ par (#1-’CDATA|emph)*> 
10 <!ELEMENT emph (#PCDATA)> 
ll <!ELEMEN'l' back (signature)> 
12 <!ELEMENT signature (#PCDATA)> 
Although DTD syntax is expressed in a special language that is not XML based,6 
it is quite straightforward and understandable (see Section 6.5.4 for a detailed discussion). For the moment let us describe in words the meaning of the various lines. 
Line 2 states that a document that uses the invitation element always has three 
6XML-Data [<->XMLDATA] defines a vocabulary for schemas that can be used for defining and 
documenting XML object classes and their relations. Based upon this work several proposals exist to 
describe DTD data using XML syntax: Document Content Description [=->DCD], Document Definition 
Markup Language [=>DDML], and Schema for Object-oriented XML [9 SOX]. 

 
%%page page_276                                                  <<<---3
 
256 
HTML, SGML, and XML: Three markup languages 
parts: front, followed by body, and terminated by back. Line 3 goes on to declare 
that the front element is a sequence ofa from, a to, a where, and, optionally, a why 
element (the fact that the why element is optional is signaled by the presence of the 
? sign). We can force more structure on the to element by requiring that all names 
be separate elements. In this case, line 3 becomes 
<!ELEMENT front (to+, date, where, why?)> 
and we would code the <to> information as 
<to>Anna</to><to>Bernard</to><to>Didier</to><to>Johanna</to> 
We could even require the to element to have explicit subelements for dealing with 
names (see Section B.4.4 that describes the BIBTEX DTD). Thus, replacing line 5 
in the original DTD by 
<!ELEMENT to (name+)> 
<!ELEMENT name (#PCDATA)> 
we would code the <to> information as 
<to><name>Anna</name><name>Bernard</name> 
<name>Didier</name><name>Johanna</name></to> 
It is up to the DTD developer to decide which approach is more appropriate for the 
application in question. 
Continuing our parsing of the DTD, we find line 8 tells us that the central body 
part of the invitation consists of one or more paragraphs (the sign + means one or 
more, while * means zero or more). According to line 9 each par element can itself 
contain parxed character data (#PCDATA) or emphasized text (flagged with <emph> 
tags). Finally (line 11) the back part has only a signature element. Each of the 
elements at the terminal nodes of the document structural tree (date on line 4, to 
on line 5, where on line 6, why on line 7, emph on line 10, and signature on line 12) 
can contain only #PCDATA. Such data is analysed (parsed) by the XML application 
and validated to see whether all references are known. 
6.5 The detailed structure of an XML document 
Now that we have a good idea of what an XML document looks like, it is time 
to discuss the structure of XML documents in more detail. We shall, therefore, 
consider the various components of a document instance and of the DTD. 
A detailed commented version of the XML recommendation is Tim Bray’s Annotated XML Specification [‘->AXML]. It provides a hypertext version of the XML 
Specification, complemented with historical and technical annotations, as well as 
examples and advice. It is a mmt for those who want to learn about XML. 

 
%%page page_277                                                  <<<---3
 
6.5 The detailed structure of an XML document 
257 
To be as precise as possible, we will make frequent references to the XML Specification Document that defines the XML language using 89 “productions."7 
6.5.1 XML is truly international 
From the very start, XML was designed to be international, and it can be used with 
all scripts in the world since it is based on the Unicode or ISO/IEC 10646 standards 
(see Section C.2). 
Indeed, XML allows you to use any Unicode character. It subdivides the Unicode set of characters into lettery (XMLPR [84] ), combining cbaractery (various kinds 
of diacritics that can be combined with letters, XMLPR [87] ), digits (Roman, Arabic, 
Bengali, and so on, XMLPR [88] ), and extendery (such as a middle dot, XMLPR [89]). 
XML names start with a letter, an underscore, or a colon and can continue with 
a letter, a digit, a dot, a hyphen, an underscore, a colon, a combining character, or 
an extender (XMLPR [4-5] ). Names cannot begin with the reserved character string 
“XML" (in upperor lowercase). 
The language in which parts of a document are written can be specified using 
a language code (XMLPR [33-38] ). A special attribute xml : lang can be declared for 
any document element and used in the document text. Suppose that we declare an 
attribute for an element type p as follows (see Section 6.5.4.2 for a description of 
the syntax of attribute declarations in a DTD): 
< !ATTLIST p Xmlzlang NMTUKEN #IMPLIED> 
Then we could write in our document instance something like 
<p xml:lang=``en-GB''>A favour, certainly!</p> 
<p xml:la.ng=“en-US">A favor, sure!</p> 
<p xml:lang="fr“>Une faveur, avec plaisir !</p> 
<p xm1:1a.ng=’de’>Eine Gunst, sicher!</p> 
Each parsed entity of an XML document can use a different character encoding (this would be a convenient way to include Russian or Chinese text snippets 
inside an otherwise English-language text). To allow for this possibility, specify the 
encoding using the encoding keyword on the text declaration of document entities 
(XMLPR [77] )--for instance, 
<'?xm1 encoding=’UTF-8"?> <-- one of the defaults --> 
<'?xml encoding=’ISO-8859-1’?> <-- Latin 1 (Western Europe) --> 
<'?xm1 encoding="ISO-2022-JP’"?> <-- Japanese encoding --> 
XML processors must recognize the UTF-16 (Unicode native 16-bit codes) and 
UTF-8 encodings (a trick to use 7-bit ASCII as is and all other characters as a multi7In the following, XML production rules are identified as XMLPR [xx] , where xx is the number in 
the XML Specification Document. 

 
%%page page_278                                                  <<<---3
 
258 
HTML, SGML, and XML: Three markup languages 
byte sequence occupying two to five bytes). All other encodings must be specified 
explicitly (XMLPR [80-81] ). 
6.5.2 XML document components 
XML documents must be well-formed. A well-formed XML document consists of a 
prolog, one or more elements, and possibly some trailing information consisting of 
procexxing instructions and comments (XMLPR [1]). 
A simple example of a valid document is the following: 
<!-- ======= Start of Prolog r 
<'?xml version="1.0" sta.ndalone=``yes'''?> <|.-- xml declaration I 
<!DOC'l'YPE coolxml [ <!-- DOCTYPE declaration I --> 
<!ELEMENT coolxml (#PCDA'l‘A)> <!-- ELEMENT declaration I --> 
I 
--> 
I 
1 
3 
4 
5 ]> <! -- end of DOCTYPE declaration --> 
6 <!-- ======= End of Prolog - --> 
7 
3 <!-- ======= Start of document elements =====================+ --> 
9 (coolxml) <!-- start tag (root element) I --> 
I0 XML is a cool idea! <!-- element content I --> 
11 </coolxm1> <!-- end tag (root element) I --> 
I2 <!-- ======= End of document elements =====================+ --> 
The prolog of the document can have an XML declaration (XMLPR [22-27]) and an 
XML document type declaration, followed by the actual informational contents of the 
document (using instances of various element types and entities). 
The following sections describe these various components in detail. 
6.5.3 The XML declaration 
The XML declaration, the very first thing that the XML parser encounters for a 
given document entity, declares such things as which version of XML you used to 
mark up your document (XMLPR [24-27]) and the encoding (XMLPR [80-81] , see 
earlier). It declares if the document is stand-alone (XMLPR [32] ), that is, if there are 
external declarations present. (Nothing is implied about external references, such as 
images, that can be present in a “stand-alone" document as long as they are declared 
internally.) The following is an example showing all three declarations: 
<?xm1 Version="1.0" encoding=’ISO-8859-1’ standa1one=``yes''?> 
<!DOCTYPE racine [<!ELEMENT racine (#PCDATA)>]> 
<racine>Sa1ut a vous de la racine du document !</racine> 
Treating these three lines with an XML processor we get no errors: 
java EventDemo racine.xml 
Start document: pubid=null, sysid=file2//localhost/home/racine.xml 
Resolving entity: name=[document], pubid=null, 
sysid=file://localhost/home/racine.xml 
Doctype declaration: name=racine, pubid=null, sysid=nu1l 
Start element: name=racine 

 
%%page page_279                                                  <<<---3
 
6.5 The detailed structure of an XML document 259 
Data: Salut a vous de la racine du document ! 
End element: name=racine 
End document : errors=0 
However, suppose we do not specify the encoding and present the following document to the parser: 
<?xml version="1.0" standalone=``yes''?> 
<!DOCTYPE racine [<!ELEMENT racine (#PCDATA)>]> 
<racine>Salut a vous de la racine du document !</racine> 
VVhe11 no encoding is explicitly specified, depending on the first byte of the file, 
UTF-8 or UTF-16 is chosen. In the present example, the XML processor will reject 
the document with an error message, as follows: 
java EVentDemo racineno-enc.xml 
Start document: pubid=null, 
sysid=file://localhost/home/racineno-enc.xml 
Resolving entity: name=[document], pubid=null, 
sysid=file://localhost/home/racineno-enc.xml 
FATAL ERROR: malformed UTF-8 sequence 
at file://localhost/home/racineno-enc.xml: line 1 
java.lang.Error: malformed UTF-8 sequence 
Since UTF-8 encoding was assumed, the accented letter “a" triggered an error condition. Its higher order bit is set, and only 7-bit ASCII characters are accepted as 
pass-through, one-byte sequences in the default encoding. Therefore for accented 
one-byte character input (French, German, Russian, and so on), an encoding mutt 
be specified or the document must be transliterated to use UTF-8 or UTF-16. 
6.5.4 The document type declaration 
The second part of the prolog is the XML document type declaration (XMLPR [28]). 
It starts with the literal string < EDOCTYPE and is followed by the type of the root 
element of the document. It contains further markup declaratiom, specified internally to the document (internal subxet) or in external entities (external mbset) or in 
both. The internal subset is read first so that you can declare elements, entities, or 
attributes that will be used when processing the external subset. 
In the previous examples the line 
<!DOCTYPE racine [<!ELEMENT racine (#PCDATA)>]> 
is a document type declaration, with root element “racine" and contains only an 
internal subset with a single-element declaration. 

%==========280==========<<<---2
 
%%page page_280                                                  <<<---3
 
260 
HTML, SGML, and XML: Three markup languages 
The internal and external subsets, taken together, provide the grammar for a 
given class of documents, known as the document type definition, or DTD. It can 
contain (XMLPR [29]) the following: 
Element declarationr start with < !ELEMENT and contain the type of the element 
and its content model. 
Attilmte declarationr start with < ! ATTLIST and for a given declared element contain one or more attrilmter together with their type and default values. 
Entity declaration; start with < !ENTITY and contain the name and definition of 
an entity. 
Notation declaration; start with < ! NOTATION and contain a name and an external 
or public identifier associated with a special notation that is used with entities 
or attributes. 
Procerring inrtmctionr are delimited by the character strings <? and ?> and contain nondocument data passed through to an application. 
Comments are delimited by the character strings < ! -- and --> and can be added 
for documentation and clarity. 
6.5.4.1 Element declarations 
Each element belonging to the logical structure of a document must be declared. 
This declaration (XMLPR [45-46]) specifies the type of the element as well its content model. The content specification can be any of the following four possibilities: 
1. 
The string EMPTY means that the element has no contents. 
<!ELEMENT linebreak EMPTY> 
The string ANY means any other element or character data, as long as the contents remain compatible with the rules of XML. 
< !ELEMENT container ANY> 
Child elementr means no character data is allowed (XMLPR [47-50] ). The contents are constrained to a (possibly nested) list of elements, specified between 
parentheses, that can be followed by the character ?, *, or +. Inside the parentheses, one can have a list of elements separated by , or I, or there can be 
further nesting. The meaning of these symbols is 
elements in document instance must be used in the order indicated; 
choice of one element in the list; 
one or more occurrences; 
zero or more occurrences; and 
zero or one occurrence. 
-\7*+--'
 
%%page page_281                                                  <<<---3
 
6.5 The detailed structure of an XNIL document 
261 
Referring to our “invitation" example in Section 6.4.3, we repeat part of its 
DTD here and discuss the meaning of its declarations. 
1 <!ELEMENT invitation (front, body, back)> 
2 <!ELEMENT front (to, date, where, why?)> 
3 <!ELEMENT body (par+)> 
Each invitation element (line 1) has exactly one front, body, and back element, and each element must be entered in that order in the document instance. The front element (line 2) must contain, in sequence, one to, date, 
where element, followed optionally by a why element. The body element (line 
3) has at least one par element. 
4. Mzbced content (XMLPR [51]) is character data (represented by the character 
string #PCDATA) interspersed with child elements. Moreover, #PCDATA must 
come first in the content model declaration, and you can constrain only the 
type of child elements, not the order or their number of occurrences, since 
only the “choice" operator is allowed. An example from our invitation DTD 
is the content model of the par element, which can contain character data interspersed with zero or more emph elements. 
< 1 ELEMENT par (#PCDATA I emph) *> 
6.5.4.2 Attribute declarations 
Attributes that can be specified on start tags or empty element tags associate namevalue pairs with an element instance. 
For validating, attributes must be explicitly declared in the DTD internal or 
external subsets. The DTD specifies which attributes can be used with which element types, what the type constraints are for these attributes, and what their default 
values are (XMLPR [52-53] ). For reasons of clarity and convenience, attribute declarations often immediately follow the declaration of the element they refer to. It 
is, however, possible to add attributes or to redefine attribute definitions by placing 
them in the internal subset. As with entities, the first occurrence is taken; further 
definitions are ignored. 
Attribute typey come in three categories (XMLPR [54] ): 
1. String type; (XMLPR [55]) can take any character string as data and are defined 
by the keyword CDATA. 
<!ATTLIST director name CDATA #REQUIRED> 
2. To/eenized typey (XMLPR [56]) introduce lexical and semantic constraints that 
fall into four basic groups: 
(a) ID is an XML name that should uniquely identify the given element type. 
(b) IDREF or IDREFS are references to elements defined with an ID attribute. 

 
%%page page_282                                                  <<<---3
 
262 
HTML, SGML, and XML: Three markup languages 
VVhen validating a document, the XML processor must check that IDREF 
values match the value of some ID attribute. 
(c) ENTITY and ENTITIES are references to entity names defined somewhere 
in the DTD. 
(d) NMTOKEN and NMTOKENS refer to a set of name tokens (XMLPR [7-8]) without further constraints; for example, they do not have to correspond to 
some attribute or entity declaration. 
To clarify this, let us consider the following attribute declarations: 
<!ELEMENT image EMPTY> 
<!ATTLIST image 
name ID #REQUIRED 
size ENTITY #REQUIRED 
bordercolor NMTOKEN ’red’ 
title CDATA #IMPLIED> 
For the element image we declare two required (name and size) and two optional (bordercolor and title) attributes. The name attribute is of type ID 
and will allow us to refer to images in the text of the document. For size, 
which is of the type ENTITY, we must use an entity definition to set its value. 
The attribute bordercolor, which is a name token, will, if no value is explicitly 
specified on an element instance, paint a “red" border. Finally, title allows us, 
if we wish, to associate any character data to the image. 
To complement these definitions, we can define another element type, imgref , 
to refer to images. 
<!ELEMENT imgref EMPTY> 
<!ATTLIST imgref name IDREF #REQUIRED> 
By thus defining the name attribute of the imgref element, we can use its value 
to refer to image instances elsewhere in the document source. 
Enumemted typex (XMLPR [57]) can take one of a list of values as specified in 
the declaration. There are two categories: 
(a) A notation type (XMLPR [58]) consists of the keyword NOTATION followed 
by one or more names of notations that must be declared in the DTD. 
<!ATTLIST image type NOTATION (giflepsltiff) #REQUIRED> 
Here we add an attribute type to our image element. It allows us to specify an image in three formats, namely, GIF, EPS, or TIFF. Note that each 
of these must have been declared with a NOTATION statement (see Section 6.5.4.4). 
(b) An enumeration (XMLPR [59]) is an explicit list of name tokens to be associated to the given attribute value. 

 
%%page page_283                                                  <<<---3
 
6.5 The detailed structure of an XML document 
263 
<!ATTLIST quotation type (inlineldisplay) ``inline''> 
This declares two possibilities-inline and display--for the type attribute of the quotation element. VVhen the attribute is not specified, 
then the value inline is used. 
To end this section on attributes we explain the format of the attribute default 
values, which are specified at the end of each entry (XMLPR [60]). There are four 
possibilites: 
#REQUIRED a value must be specified for the attribute whenever the element is 
used. 
#IMPLIED the attribute may, but need not, be assigned a value on instances of the 
element considered. 
’ defval ’ the default value for the attribute in question will be used if no value is 
given on the start tag of the element instance. 
#FIXED ’defval’ the attribute must alwzzyr have the default value def'uaZ specified in the DTD. 
Examples of all four cases are seen in the declaration for the attributes of the pict 
element following. We provided a default value for the title attribute, and fixed 
the bordercolor to take the value blue. 
<!ATTLIST pict name ID #REQUIRED 
size CDATA #IMPLIED 
title CDATA ’Default title’ 
bordercolor NMTUKEN #FIXED ’blue’> 
6.5.4.3 Entity declarations 
Foreign material (text fragments, special characters, images, external files) can be 
included in an XML source with the help of entity references. XML distinguishes 
two types (XMLPR [70-74] ): 
o General entities 
Declamtiom, which can occur only in the DTD, are of the form 
<!ENTITY GEName GEDef> 
General entity referencer (XMLPR [68]) can occur both in the DTD and the 
document instance and consist of an ampersand (85), followed by the name of 
the entity, followed by a semicolon (;). They are not expanded in the DTD. 
&GEName ; 

 
%%page page_284                                                  <<<---3
 
264 
HTML, SGML, and XZVIL: Three markup languages 
o Parameter entities 
Declaratiom, which can occur only in the DTD, are of the form 
<!ENTITY °/. PEName PEDef> 
Parameter entity referencex (XMLPR [69]) can occur only in the DTD part of 
the document. They consist of a percent sign (7.), followed by the name of the 
entity, followed by a semicolon ( ;). 
%PEName ; 
The definition part GEDef and PEDef can be an internal entity, whose definition 
is given in the DTD, and for which there is no separate associated physical storage 
object, or an external entity. 
Intemal entities 
An internal entity has its value specified inside the document declaration; it has no 
separate associated storage object. All internal entities are parsed. They are used 
for various purposes, details of which follow: 
o Definitions of abbreviated notations to ease repetitive entry of text strings (general entities), for example, 
<!ENTITY XML “Extensible Markup Language“> 
<!ENTITY MML ``Mathematical Markup Language''> 
o Definitions of abbreviations to input special characters, accents, or symbols 
(general character entities). As an illustration we give the set of five predefined 
general entities that all XML processors must recognizes They are expressed 
as numeric character referencey (XMLPR [66]) that represent characters in the 
Unicode character set in either decimal or hexadecimal representation. Character references are expanded immediately when recognized and are treated as 
character data. 
«ENTITY lt "&#60;"> «-- "<" --> 
«ENTITY gt “&#62;“> «-- ">" --> 
«ENTITY amp "&#38;"> «-- "&" --> 
«ENTITY apos "&#39;"> «-- ""' --> 
«ENTITY quot "&#34;"> «-- W --> 
Several standard character entity sets have been defined for such things as national characters (the ISO 885 9 series, Unicode, ISO 10646), graphical symbols, 
8These five entities are, strictly speaking, available only to well-formed documents that can parse 
without DTD. Valid documents must include a definition of these entities in their DTD if they need to 
reference them. 

 
%%page page_285                                                  <<<---3
 
6.5 The detailed structure of an XML document 
265 
mathematics, and so on. Following are a few more examples from the Unicode 
set, where we adopt hexadecimal notation. 
<!ENTITY cyrya "&#x044f;"> <!-- Cyrillic small letter ``ya'' --> 
<!ENTITY ggg "&#x22d9;"> <!-- Math ">>>" Very much greater than --> 
<!ENTITY U41-EOA "&#x4e0a;"> <!-- CJK ideograph (Chinese ``above'') --> 
Usually complete, predefined sets of such entities are made available to the 
document instance by including them as external entity references. At present, 
the W3C XML mathematics working group that defined MathML is discussing 
with the Unicode Consortium how to include a common set of generally used 
mathematics characters. 
o Definition of variables for use inside a DTD (parameter entities). This is very 
useful to modularize a DTD so that it becomes more easily maintainable. Two 
examples are shown: 
<!ENTITY 7. inline ``link I image I object I break I q''> 
<!ENTITY ‘A align "align (leftlcenterlrightljustify) #IMPLIED“> 
The first could be used to shorten content models by allowing “inline" elements to be included globally, while the second form is a preformatted entity 
that can be used to declare an “align" attribute in a consistent way with various 
element types. 
All internal entities must be declared in the internal or external DTD subsets. 
Entity references should follow their declaration in the source. 
An entity reference triggers, at the given point in the XML file, the substitution 
of the entity reference by its contents. Entity definitions can themselves refer to 
other internal and already defined entities. For instance, using the definition of the 
entity XML at the beginning of this section, we can declare another entity XMLS as 
follows: 
<!ENTITY XMLS "&XML; and other extensible languages"> 
External entities 
External entities are all those that are not internal. They are used to reference data 
external to the given document instance. Data included via such an entity reference 
can be either parsed or declared with the NDATA keyword, in which case the data 
remains unparsed (for instance, a picture or a binary file). 
External entity declarations come in three basic forms: 
1. The external identifier can be preceded by the keyword “SYSTEM," and followed 
by a system literal. This is also known as the system identzfier, which is used to 
retrieve the entity (XMLPR [75]). A system literal (XMLPR [11]) is an external 

 
%%page page_286                                                  <<<---3
 
266 
HTML, SGML, and XML: Three markup languages 
identifier in the form of a URI (Univemzl Rexource Identzfier; see Section 1.1.2) 
that is able to identify any resource on a given computer system. 
<!ENTITY % subdtd SYSTEM 
"http://www.mysys.org/XML/dtds/headings-xml.dtd"> 
<!ENTITY chapterl SYSTEM "chapter1.xm1"> 
The first declaration defines the parameter entity subdtd that corresponds to 
part of a DTD. It can be referenced from inside the internal or external DTD 
subsets. The second declaration defines a general entity chapterl pointing 
to an XML source file chapter1.xml in the same directory as the referring 
master file. Such declarations can be useful for subdividing large documents 
into smaller chunks for easier handling. These document fragments are then 
included in the master document via entity references. 
. The external identifier can be preceded by the keyword “PUBLIC." It must then 
also contain a public identifier literal-itself followed by a system literal in the 
form of a URI (XMLPR [75]). A public identifier (ISO/IEC:907(), 1991) is a 
name that is intended to be meaningful across systems and different user environments. Formally a public identifier is composed of several fields, separated 
by a double solidus, “//" (see, for instance, [=>FPISYNTAX] or the standard 
(ISO/IEC:9070, 1991)). In short, the first part is an owner identifier. The entries in the example that follows have a hyphen, -. This means that the identifiers were not formally registered, while the organization that created the file 
was the W3C (the Web Consortium). The second part of the public identifier 
(following the double solidus) is called the text identzfier. The first word indicates the public text clan‘ (such as DTD or ENTITIES) and is followed by the public 
text dexcription (such as HTML, Latinl). Then optionally after another double 
solidus, one finds the public text language and a code from ISO 639 (see Table C.1), which in our case is EN, for English. Finally this can be followed by a 
dixplay version, if needed. A server for helping you resolve public identifiers has 
been set up by Peter Flynn[=>FPISERVER]. 
Let us look at how these public identifiers are used with entities: 
<!ENTITY % html4-strict PUBLIC "'//W3C//DTD HTML 4.0//EN" 
"http://www.w3.org/TR/REC-htm140/strict.dtd"> 
<!ENTITY % IS01at1 PUBLIC "-//W3C//ENTITIES Latinl//EN//HTML" 
"d:\home\dtds\iso-lat1.ent"> 
In the first example we define a parameter entity html4-strict, known by 
the public identifier -/ /w3c/ /DTD HTML 4. 0/ / EN. Using this public identifier, 
the XML application will try to construct a URI to retrieve the file (for example, 
using the catalog file proposed by the OASIS consortium; see Section 6.6.5.1). 
If such a URI cannot be generated, the external entity reference will be resolved 
by using the explicit URI specified at the end. The second example defines 
IS0lat1 as a parameter entity to refer to the definition of Latin 1 characters. 

 
%%page page_287                                                  <<<---3
 
0.3 
1 IIC aetauea structure OI an AJVIL (IOCUIIICIIC 
267 
In this case, the URI specified at the end is an absolute path (on a Microsoft 
VVindows platform). 
For handling nonparsable data (such as EPS, GIF, or JPEG images; TEX source 
files; or binary files), we must specify the NDATA keyword followed by the name 
of a notation known to the XML system (XMLPR [76]). This allows the data to 
be passed to and handled by an application capable of interpreting the notation 
in question. 
<!ENTITY xmlfigl SYSTEM 
"http://www.myserver.edu/book-files/figures/xmlfigl.eps" NDATA EPS> 
Here we define an Encapsulated Postscript image that can be retrieved from 
a Web server. When the XML application parses the document and finds a 
reference to this general external entity, it must know how to handle such an 
EPS image. This is declared with a notation declaration (see Section 6.5.4.4). 
References to an unparsed entity can occur only in attribute values that were 
declared to be of types ENTITY or ENTITIES (see “Ti;/eenized typex" on page 261). 
Note that, as with attribute declarations, the first occurrence of an entity declaration takes precedence. This allows declarations to be made in the DTD’s internal 
subset, which is read before the external subset, thus overriding possible definitions 
for the same entity name in the external subset. To make life easier for nonvalidating parsers in the internal subset, parameter entity references cannot be used 
imide markup declarations. They can be used only at the top level where markup 
declarations themselves occur. As an example consider the following document: 
In 
<?xm1 versiUn="1.0"?> 
<!DOCTYPE doc E 
<!ENTITY % a ’<!ELEMENT a (#PCDATA)>’> 
Za; 
<!ENTITY % doc ’(a|b)’> 
<!ELEMENT doc (%doc;)*> 
<!ELEMENT b EMPTY> 
]> 
<doc><a>text</a><b/><a>more text</a></doc> 
The parameter entity reference on line 4 at the top level is all right, whereas the one 
on line 6 inside an element declaration is invalid in the internal subset. However, 
if we transfer lines 3-7 into the file penext.dtd and we make the XML source 
reference that file, then we find that this limitation applies only to the internal 
subset. Therefore the following document is a valid one: 
1 
Z 
3 
<?xml version="1.0"?> 
<!DOCTYPE doc SYSTEM "penext.dtd"> 
<doc><a>text</a><b/><a>more text</a></doc> 

 
%%page page_288                                                  <<<---3
 
268 
HTML, SGML, and XML: Three markup languages 
6.5.4.4 Notation declarations 
Notation declarations (XMLPR [82-83]) are related to how unparsed entities have 
to be handled. They associate a name with 
o the format of an unparsed entity; 
o the format of an element with a “notation" attribute; and 
o the application that is going to handle the data inside a processing instruction. 
A notation declaration, therefore, consists of a name for the notation, as discussed earlier, and an external or public identifier that allows the XML processor to 
locate a program to process data that is flagged to be in the given notation. 
1 <!NOTATION gif SYSTEM "c:\Program Files\Internet Exp1orer\Ie4.dl1"> 
2 <!NOTATION eps SYSTEM "/usr/local/bin/X11/gv"> 
On line 1 we declare, on Microsoft Windows, that GIF images should be handled 
by the Explorer 4 program; on line 2 we declare how to deal with EPS files on a 
UNIX system. 
6.5.4.5 Significant space 
In order to make documents" more readable and easier to maintain, it is often convenient to add blank lines or spaces. Most of the time this whitespace is not significant and is not intended for inclusion in the output instance of the document 
that is generated by the XML application. Sometimes, however, whitespace should 
be preserved as is in the output representation (e.g., for displaying computer code), 
and a special reserved attribute, xml : space, must be associated with the element 
type in question; for instance, 
1 <!ELEMENT computercode (#PCDATA)> 
2 <!ATTLIST computercode xml:space #FIXED ``preserve''> 
These lines instruct the XML processor to tell the XML application to preserve 
all whitespace (line breaks, tabs, and so on) for material inside a computercode 
element. XML processors never eliminate any space characters, since they must 
pass on to the application all material that is not markup. Therefore it is up to the 
application to honor requests about whitespace handling. Note that, unless told 
otherwise, most applications will fold multiple whitespace into a single space, thus 
ruining the intended layout as coded in the source material. 
6.5.4.6 Processing instructions 
Processing instructions (XMLPR [16-17]) allow documents to communicate information and instructions to specific applications; processing instructions are not part 
of the document’s character data. Their contents are passed on to the relevant application by the XML processor. 

 
%%page page_289                                                  <<<---3
 
6.5 The detailed structure of an XML document 269 
Processing instructions are enclosed inside the pair of characters <? and ?>. 
The opening <? is followed by a name identifying the target application (ml is a 
reserved name, see following); then comes the data for the application in question. 
A NOTATION declation (see Section 6.5.4.4) can be used to associate an application 
with such a name. An example of how to “talk to" a PRINT application follows: 
<?PRINT Clear Page'?> 
A special processing instruction is the line at the beginning of an XML document, for instance, 
<?xml version="1.0“ standalone="yes“?> 
Although this is called the XML declaration, it is really a processing instruction for 
the XML processor indicating version, encoding, and so on. 
6.5.4.7 CDATA sections 
CDATA sections (XMLPR [18-21] ) are used to escape text where markup should not 
be recognized. Such sections are allowed wherever character data can be used; they 
start with the literal string <! [CDATA[ and end with the literal string ]]>. Inside 
these delimiters any kind of markup can be used (except the closing string ]]>, of 
course). ETEX users will recognize similarities with the verbatim environment. In 
the next example the start and end tags, as well as the entity reference, are treated 
as character data and not as markup. 
<![CDATA[<p>An ampersand sign looks like &amp;.</p>]]> 
The XML processor will output the above line uninterpreted as 
<p>An ampersand sign looks like &amp;.</p> 
6.5.4.8 Conditional sections 
Inside the external DTD subset, conditional sections (XMLPR [61-65]) can be used 
to include or exclude parts based on the value of a keyword. Suppose that we have 
the following declarations in the external subset of our DTD: 
<!-- all of the following must be in the external subset --> 
<!ENTITY % complex ``IGNORE''> 
<!ENTITY % Simple ``INCLUDE''> 
<![ %comp1ex; [ 
<!ELEMENT table 
(caption?, (col*lcolgroup*), thead?, tfoot?, tbody+)> 
]]> 
<![ Ksimple; [ 
<!ELEMENT table (caption?, tbody)> 
]]> 
5~oan\na\v.4>~..-N... 

%==========290==========<<<---2
 
%%page page_290                                                  <<<---3
 
270 
HTML, SGML, and XML: Three markup languages 
The parameter entity complex (line 2) assigns the value IGNORE to the keyword 
controlling the conditional section; hence the first (complex) definition (lines 4-7) 
will be ignored, and the second (simpler) declaration (lines 8-10) for the table 
element (which is “included" via the simple entity on line 3) is retained. However, 
suppose that we put the following in the internal subset of our document instance: 
1 <!ENTITY ‘Z. complex ``INCLUDE''> 
2 <‘.ENTITY ‘A simple ``IGNORE''> 
Since the first occurring entity declarations take precedence, the values of the 
complex and simple parameter entities in the external DTD are overridden by 
the values in the internal subset. As a result (line 1) the keyword INCLUDE gets assigned to the conditional section that controls the complex definition of the table 
element type that will thus be used in the DTD. 
6.5 .5 Document elements 
Each XML document contains one or more elements (XMLPR [39] ), consisting of content (XMLPR[43]) enclosed in a stnrttng (XMLPR[40]) and an endtng 
(XMLPR [42] ), or of an empty-element tag (XMLPR [44]). Each element has a type, 
identified by a name (its “generic identifier"), and start and empty tags may have a 
set of attribute specifications. Each attribute specification (XMLPR [41]) has a name 
and a value. 
For an XML document to be valid, each element and entity must have been 
declared in the DTD, and their contents and attributes must correspond to the 
content model and attribute declaration. 
Let us look at part of our invitation document (Section 6.4.2.2) as an example: 
<body> 
<par> 
I would like to invite you all to celebrate 
the birth of <emph>Invitation</emph>, my 
first XML document child. 
</par> 
ooxlaxvl-B‘4-‘N’-' 
;}body> 
When we compare this with the invitation DTD (Section 6.4.3), we see that the 
elements are used in agreement with the relevant part of that DTD. 
1 <!ELEMENT body (par+)> 
2 <!ELEMENT par (#PCDATA|emph)*> 
3 <!ELEMENT emph (#PCDATA)> 

 
%%page page_291                                                  <<<---3
 
6.6 XML parsers and tools 
Let us end this section on document elements with an example of the use of 
empty elements. 
<!DOCTYPE emptyexample [ 
<!ELEMENT emptyexample (par*)> 
<par>The XML logo is seen in the image <imref name=``xm1-logo''/>.</par> 
<par>The image tag shows an alternative syntax for an empty element 
<image name=``xml-logo'' address="&logo-uri;"></image>.</par> 
</emptyexample> 
1 
2 
3 <!ELEMENT par (#PCDATA|imageIimref)*> 
4 <!ELEMENT image EMPTY> 
s <!ATTLIST image name ID #IMPLIED 
6 address CDATA #REQUIRED> 
7 <!ELEMENT imref EMPTY> 
8 <!ATTLIST imref name IDREF #REQUIRED> 
9 <!ENTITY logo-uri "http://www.ucc.ie/xml/xm1.gif"> 
10 ]> 
11 <emptyexample> 
n 
n 
E 
v. 
This valid XML document contains a par element type inside the root element 
emptyexample (line 2). The par element has a mixed content of parsed character 
data and image and imref elements (line 3) that are both declared empty (lines 4 
and 7). At the end of the DTD internal subset, we declare a general entity logo-uri 
(line 9), which we shall use to specify the location of the image. The content of the 
first par element (line 12) shows one of the two ways to represent an empty element in XML documents. The image reference tag <imref na.me=``xml-logo''/> 
in question has an attribute of type IDREF with a value of “xml-logo." For the document to be valid, there must exist another element with an attribute declared of 
type ID and a value of “xml-logo." We find such an element inside the second par 
element (on line 14). Here the identifier in question is defined inside the empty element image, which associates the image with a GIF file via a URI (and a reference 
to the entity logo-uri). Note the alternative syntax <image. . . > </image> for an 
empty element. 
6.6 XML parsers and tools 
Although the average user will not have to worry directly about XML parsers (just 
as Fortran, C, or C++ compilers for the programmer are only tools that compile 
computer programs into executable code). Nevertheless some information about 
the availability and usefulness of such parsers will help you to understand the use of 
these (and other) tools to generate, for example, HTML or PostScript output from 
your XML data. 
Several XML parsers are freely available on the Internet. Most of them are 
written in C++ or Java, while Python, Perl, and Javascript have also been used. 
271 

 
%%page page_292                                                  <<<---3
 
272 
HTML, SGML, and XML: Three markup languages 
New versions of browsers, such as Microsoft’s Internet Explorer Version 5, Mozilla 
(the successor of Netscape Version 4), and other tools, such as Microsoft’s Office 
suite, Adobe’s Acrobat and FrameMaker, and symbolic algebra packages, such as 
Mathematica and Maple, will all include “native" support for XML. 
In this section we limit ourselves to the editing and verification phases of 
XML documents; in the remaining chapters we will deal with ways of using the 
parsed data to manipulate, view, print, and exchange the information. Indeed, XML 
markup conveys very little about the semantic meaning of a given tag; in particular 
it says nothing about how a tag should be rendered. The basic principle of XML is 
that structure and semantics are completely decoupled, with style of rendering being addressed by XSL. This subject, together with other style-related developments, 
will be reviewed in Chapter 7. 
We start our overview with two examples of editors for XML data and DTDs. 
Then we say a few words about DTD-handling tools, before looking at some of the 
parsers. 
6.6.1 Emacs and psgml 
The emacs editor is one of the most often used editors on UNIX (and today is also 
popular on Microsoft Windows). It is highly customizable via Lisp code and can 
easily cope with the syntax of various languages. Lennart Staflin developed the basic 
SGML support in the form of an emacs “macro mode" psgml [HPSGML]. More 
recently, David Megginson developed an XML add-on to psgml [HPSGMLXML]. 
For editing DTDs Tony Graham contributed his tdtd macros [=>TDTD]. 
All these modes present you with menus and commands for inserting tags, thus 
helping you to enter only contextually valid tags and allowing you to edit attribute 
values in a separate window with information about types and defaults. They also 
identify structural errors. 
As an example, \reffig{6-2} shows the “front" part of our “invitation" document. 
From the “SGML" pull-down menu we chose the “List valid tags" entry. The curser 
is positioned just following the front element, as shown in the top left part of the 
figure. Parsing the DTD, psgml determines that only the to element is valid at that 
point and informs us in the bottom window. In fact, psgml is interfaced to James 
Clark’s nsgmls program, a full-blown SGML parser that will be described later. 
\reffig{6-3} is an example of how the tdtd mode is helpful in editing DTDs. 
Keywords, comments, entities, literal data, the content model, and delimiters are 
clearly identified since they are displayed in different colors. Moreover, at each 
point you can ask emacs for assistance. In particular, in the lower half of the example 
a list of available DTD commands can be seen. If you choose one, the program will 
walk you through a dialog where you can specify all relevant components for the 
element being dealt with. 
Most users probably prefer a more “visual" (WYSIWYG) approach to editing 
XML documents and creating or modifying DTDs. It is expected that by the end 

 
%%page page_293                                                  <<<---3
 
6.6 XML parsers and tools 273 
Buffers Files Tools Edit Search Hula SEHL Modify Have Markup Vieu DTD Help 
<!-- ++++ The header paihsu mrnr --> 
<front>| End E.laI-lent (c-c n 
<to>Anna, Bernard, Didi: Show Context (C-c C-c} 
<date>Next Friday Evenil Ilhat Elel-lent (c-c C-u) 
<Where>']-‘he Web cafe</Wh‘ List Valid Tags (C-c C-t} 
<WhY>t‘-‘Y first xii‘ babY< Shoufflide Ilarning Log (C-c C-1) 
</f1‘°1'11'-> Validate (C-c C-v) 
<!-- +++++ The main par Fug options , --> 
:bOd}>’> User Options ) 
Ipfrguld like to invite «Sm "“ ""“""* 
current element: front 
Element content: element 
Current element can not end here 
Valid start-tags 
‘ In current element: <to> 
: (Fundamental) --L1--Al 1 ---- -menu-bar SGML Reset Buffer 
\reffig{6-2}: Emacs in psgml mode acting on the invitation example 
ffers Files Tools Edit Search llule DSSSL SEIIL Iludifg have llarkup View DTD Help ‘ 
<!-- invitation DTD --> 
<!-- Mag 26th 1998 mg --> 
<!ELEMENT lflvltatlflh (Front, body, back) > 
<!ELEMENT it (to, date, where, whg“) > 
I!ELEMENT ‘H. (#PCDHTQ) > 
<!ELEMENT ." (#PCDHT9) > 
<lELEMENT (§PCDfiT9) > 
(RPCDRTQ) > 
(par ) > 
(fiPCDHTQ|emph)~ > 
(§PCDRTQ) > 
(signature) > 
u"e (§PCDHT9) > 
<!ELEMENT= 
<EELEMENT 
(!ELEMENT 
<!ELEMENT. 
. >Wrw -!5 HM -~ 
Click mou5e~2 on a Completion to select it. 
In this buPPer, tgpe RET to select the completion near point. 
Possible completions are: 
dtd-big~comment dtd~center-comment 
dtd-comment dtd~comment-setup 
dtd-declaration-setup dtd-declare~element 
dtd-design-comment~setup dtd-etags 
dtd-Pill-paragraph dtd-Pix~entities 
dtd-init-comment-setup dtd~insert-mdc 
dtd-join-comments dtd~mode 
dtd-recenter-comment dtd-recomment 
dtd-startup dtd-untabiPg-all 
``Li''Hl1 * * - - V V V - - -~'----' 
[u H 312", iwlfl L 1 , ' 
7 mu1et1‘_m:o 
\reffig{6-3}: Emacs in xml and dtd mode acting on the invitation example 

 
%%page page_294                                                  <<<---3
 
274 
~Visua| XML 
@ @tcI 
® date 
0 @where 
B #PCDATA 
[3 #PCDATA 
9 par 
D COMMENT 
ackw V 
signature 
[3 #PCDATA 
HTML, SGML, and XML: Three markup languages 
D Gr0upEnd 
E0 to 
G9 @dats 
3 where 
5 @why 
6 body 
\reffig{6-4}: A visual editor for XML 

 
%%page page_295                                                  <<<---3
 
6.6 XML parsers and tools 
of 1999 many visual XML editors will be available. As an early example, we show 
in \reffig{6-4} a prototype of such a WYSIWYG editor. It was developed by Pierre 
Morel and iillows you to edit tree-based views of XML documents (see the upper 
part of \reffig{6-4}) as well as DTDs (refer to the bottom part of the same figure). 
It is written in Java and offers the “look and feel" of the Java Foundation Classes 
(see [L->VISXML] for more information and for instructions on how to download 
the software). 
Another example is the W3C’s Amaya HTML testbed editor; its description is 
available at [L->AMAYA]. Strictly speaking it is not yet an XML editor, but since it 
allows you to experiment with CSS and MathN[L-you can edit complex mathematical expressions via a WYSIWYG interface-it is certainly representative of the 
editors we will see in the near future. \reffig{6-5} shows different views of the same 
document, a facility that eases entering and maintaining document sources. 
6.6.2 The perlSGML programs 
Earl Hood has developed perlSGML, a set of Perl programs and libraries for processing SGML DTDs and documents. The per1SGML distribution [L->PERLSGML] 
contains two Perl libraries (dtd.pl, a DTD parser, and sgml .pl, an SGML document parser), a set of Perl modules for handling SGML documents and managing 
entities, as well as a set of programs. They are briefly described here: 
dtd2html creates a set of HTML files to navigate and document an SGML DTD. 
dtddiff compares two DTDs and lists their differences. 
dtdtree outputs the hierarchical relations between the elements of a DTD as a 
tree structure. 
dtdview allows you to query a DTD interactively. 
stripsgml strips a file of its SGML markup and replaces entity references by standard ASCII characters. 
Let us have a closer look at the dtdtree program. 
1 dtdtree -dtd invitation.dtd 16 I I__why?) 
2 17 I I _ (#PCDATA) 
3 INVITATION 18 I 
4 19 I 
5 INVITATION 20 I __body, 
6 I_ (front, 21 I I_ (par+) 
7 I |_(to, 22 I I_ (#PCDATA I 
8 I I |_(#PCDATA) 23 I |__emph) * 
9 I I 24 I I _ (#PCDATA) 
10 I I__date, 25 I 
11 I I I _(#PCDATA) 26 I 
12 I I 27 I 
13 I I __where , 28 I __back) 
14 I I |_ (IIPCDATA) 29 |_ (signature) 
is I I 30 |_ (IIPCDATA) 
275 

 
%%page page_296                                                  <<<---3
 
 
%%page page_297                                                  <<<---3
 
6.6 XML parsers and tools 
As seen in the previous example, which is the tree representation generated by 
dtdtree for our invitation DTD, the dtdtree program provides a convenient 
visual representation of the hierarchical relations that exist between the various 
elements of a DTD. The command wextyped is shown on line 1. The structure of 
the various content models is clearly displayed. Take, for instance, the par element 
(line 21): It is at once evident that its parent is body (line 20) and its children are 
#PCDATA and emph elements (lines 22 and 23). 
6.6.3 The DTDParse tool 
Norman Walsh has written another tool for documenting DTDs with the help of 
HTML files. His tool is part of the DTDParse suite of programs, available from 
[h>DTDPARSE]. Once more Perl is the implementation language. Walsh takes a 
two-step approach: The DTD is first parsed and then entered into a database, which 
is easier to search for later access. Following we show the dtdparse utility generating the database for our invitation DTD: 
1 dtdparse -v 2 invitation.dtd 16 Expanding date 
2 Loading catalog: .7CATALOG 17 Expanding body 
3 Loading catalog: /usr/local/sgml/CATALOG 18 Finding parents. . . 
4 Loading catalog: ~/sgml/CATALOG 19 Finding parents of back 
5 Expanding entities. . . 20 Finding parents of signature 
6 Finding children. .. 21 Finding parents of par 
7 Expanding back 22 Finding parents of to 
8 Expanding signature 2: Finding parents of invitation 
9 Expanding par 24 Finding parents of why 
10 Expanding to 25 Finding parents of front 
11 Expanding invitation 26 Finding parents of emph 
12 Expanding why 27 Finding parents of where 
13 Expanding front 28 Finding parents of date 
14 Expanding emph 29 Finding parents of body 
15 Expanding where 30 Storing elements. . . 
Once the elements are stored in the database, we can transform that information into a set of HTML or UNIX man pages by using Norman Walsh’s dtd2htm1 
(this program is dzflerent from Earl Hood’s program, mentioned in Section 6.6.2) 
or dtd2man program, respectively. An interesting example of the use of this tool 
is the documentation of the DocBook DTD, which can be viewed at [=>DBVIEW] 
(this DTD, which is frequently used for marking up technical documentation, will 
be discussed briefly in Section B.4.1). 
6.6.4 The Language Technology Group XML toolbox 
The Language Technology Group of Edinburgh University has been developing 
SGML tools for many years. They recently released version 1 of their LTXML 
library. It consists of a developer’s toolkit, based on a C-based API for handling 
XML documents, as well as a set of stand-alone programs built using that API. The 
277 

 
%%page page_298                                                  <<<---3
 
278 
HTML, SGML, and XNIL: Three markup languages 
Language Technology Group has a lot of experience dealing with large corporacollections of text files consisting of several tens of millions of characters. To handle such large documents they had to develop extremely efficient batch-processing 
tools that are able to deal with the huge data streams. In the following we will look 
at a few of those tools. A more detailed description can be found at [=>LTXML]. 
6.6.4.1 XML transformations and regular expression searches 
The sgmltrans program translates XML files into some other format, such as 
HTML or ETEX. It is based on the paradigm that specific actions take place for 
given start or end tags, depending also on the context. The program allows you to 
print only some text onto the output stream. 
sggrep is similar to grep on UNIX. It allows searching a file for regular expressions, taking into account the tree structure of XML files. 
sggrep [-h] [-u base-urll I".-d doctype] [-v] [-n] [-r] 
[-m mark-query] [-a element-name] [-q query] 
[-s sub-query] [-t regexp] [--] ['i.np11.ts...] 
The more important arguments are 
-u base-url base URL used when resolving relative URLs. 
-n print no newline between output matches (the default is to print a newline). 
-q query pattern of the items to select, a path of terms separated by “/" where 
each term is an SGML element. 
-r attribute values in queries are regular expressions; 
-s sub-query selects subelements of query-selected item for regexp to match. 
-1: regexp regular expression to match against text directly contained in queryselected item (if no subquery) or in any subquery selected subelement of 
query-selected item. An empty string (“") matches anything, including 
empty elements; indeed, this is the only way to specify empty elements 
if they are needed. 
Next we look at a few examples using the invitation.xm1 file on page 254. 
The query string consists in specifying a “path" to reach the selected elements. 
sggrep -q "invitation/body/par" invitation.xml 
<?xml version=’1.0’ encoding=’UTF-8’?> 
<!DOCTYPE invitation SYSTEM "invitation.dtd"> 
(par) 
I would like to invite you all to celebrate 
the birth of <emph>Invitation</emph>, my 
first XML document child. 
</pa:> 
(par) 
Please do your best to come and join me next Friday 
'5~om\uo~u.-c>t..~... 

 
%%page page_299                                                  <<<---3
 
6.6 XML parsers and tools 
evening. And, do not forget to bring your friends. 
</par) 
(par) 
I <emph>rea11y</emph> look forward to see you soon! 
</par) 
sggrep -q ".*/emph“ invitation.xm1 
<?xm1 version=’1.0’ encoding=’UTF-8’?> 
<!DOCTYPE invitation SYSTEM "invitation.dtd"> 
<emph>Invitation</emph) 
<emph>rea11y</emph> 
sggrep -q "invitation/body/par[0]/emph" invitation.xm1 
<?xm1 version=’1.0’ encoding=’UTF-8’?> 
<!DOCTYPE invitation SYSTEM "invitation.dtd"> 
<emph>Invitation</emph) 
sggrep -q "././.[2]/." invitation.xm1 
<?xm1 version=’1.0’ encoding=’UTF-8’?> 
<!DOCTYPE invitation SYSTEM "invitation.dtd"> 
<emph>rea11y</emph> ‘ 
sggrep -q "./.[0]/.[2]" invitation.xml 
<?xm1 version=’1.0’ encoding=’UTF-8’?> 
<!DOCTYPE invitation SYSTEM "invitation.dtd"> 
<where>The Web Cafe</where> 
R-a\.~L-at-a\~\..aL-aI\)I\aI\1I\aI\aI\)I\1I\1I\2I\1».->->--y-4»-y--r--.-v-:>~v-Aw~...o~oao\Io«v-.Aw~.._o~o=:n\Io«m.z>w~vThe first command (line 1) queries all the par elements inside a body, inside an 
invitation element. We see the three paragraphs and their content (lines 4-15). 
The second command (line 17) uses a wildcard syntax for showing all emph elements 
(lines 20 and 21); while the third command (line 23) goes to the emph element in 
the first (counting starts from zero) par inside a body, inside an invitation element. We can also use a different kind of wildcard syntax, as in the fourth command 
(line 28), where we want to see any element inside a third element, inside any element, inside the top element; while the last command (line 33) shows us the second 
element inside the first element, inside the top element. 
A more complex tool is sgrpg, which allows XML queries and general transformations. You can select a set of XML elements and, if needed, transform them 
into a different format. This program allows quite complex (nested) queries. It is 
rather more general than sggrep. LTXML tools are very useful for simple ad hoc 
queries. However, general-purpose applications that have to locate information in 
XML files based on their hierarchical structure will probably use the XSL or DSSSL 
languages since they are better suited for complex searches. 
6.6.4.2 Other tools 
The LTXML system comes with a few more interesting little tools for handling 
XML files. We consider a couple of them: 
sgcount counts elements in an XML file. 
sgcount [-0 nb] [-12] ['i'n.p'u.ts...] 
-t counts onl to -level elements. 
Y P 
279 

%==========300==========<<<---1
%==========300==========<<<---2
 
%%page page_300                                                  <<<---3
 
280 HTML, SGML, and XML: Three markup languages 
-0 nb an integer defined as follows: 
0 default printout format; this is a table, showing for each document element, its type, its frequency, and its frequency accompanied by an attribute of type ID. 
1 shows only tag names and counts. 
2 shows only total number of tags. 
sgcount invitation.xm1 
invitation 1 
front 
to 
date 
where 
why 
body 
par 
emph 
back 
signature 
*Tota1* 
r-Ar-r-«M0Jr-«Hp-Ar-«r-Ar-A 
OOOOOOOOOOOO 
4 
sgcount emptyexamp1e.xm1 
emptyexample 1 0 
par 2 0 
imref 1 0 
image 1 1 
*Total* 5 1 
sgcount -o 1 emptyexamp1e.xm1 
emptyexample 1 
2 E E 8 3 5 E 3 G I E E Z S o m u 0 m A u N 24 par 2 
25 imref 1 
26 image 1 
27 *Total* 5 
n 
29 sgcount -o 2 emptyexamp1e.xm1 
3o 5 
n 
32 sggrep -q “invitation/body/par“ invitation.xm1 I sgcount -o 1 
33 par 3 
34 emph 2 
35 *Tota1« 5 
These examples show the number and type of elements in the invitation (lines 
1-13) and emptyexample (lines 15-20) XML files, as well as the effect of the -0 
option (lines 22-30). Of particular interest is the last command (line 32); here we 
have added the sggrep command to select the section of the invitation.xml file 
inside par elements, and we have counted the included elements (lines 33-3 5). 
The utility textonly strips all XML markup from a file. 
textonly [-h] [-u base-url] [-t tag] [-s c] [-x] filename 
The more relevant options are 
-u base-url base URL. 

 
%%page page_301                                                  <<<---3
 
6.6 XML parsers and tools 
-t tag outputs only text present inside the element tag. 
-s string outputs a separator between successive text units. Special values are 
’ ’ for a space, ’ \n’ for a newline (this can be useful if you want one 
word per line), and ’ ’ for the empty string. 
textonly -t par invitation.xml 
I would like to invite you all to celebrate 
the birth of Invitation, my first XML document 
child. 
Please do your best to come and join me next Friday 
evening. And, do not forget to bring your friends. 
I really look forward to see you soon! 
textonly -t signature invitation.xml 
Michel 
textonly -t front -s ’\n’ invitation.xml 
Anna, Bernard, Didier, Johanna 
Next Friday Evening at 8 pm 
The Web Cafe 
My first XML baby 
3 J I E E 2 5 o m N o m A W N _ 
We will use once more our invitation.xml example. The first textonly command (line 1) gets all data of par elements (lines 2-9); the second command (line 
10) gets the data inside the signature element (line 11); while the third command 
(line 12) gets all data in the front part of the document, adding a newline between 
the various text strings to separate them (lines 13-16). 
You can easily build many more tools using the LTXML API. For instance, the 
toolkit also contains an ESIS generator (see Section 6.6.5.2), which provides output 
similar to the parsers described in the next section. 
6.6.5 Validating documents with XML parsers 
It is often important and, in any case, useful to be able to check an XML file for 
being well-formed or valid. Sometimes we might want to analyze a document and 
decompose it into its constituent parts for further manipulation, possibly needing 
to get access to the DTD. In all such cases XML parsers come in handy. A complete 
and up-to-date list of XML tools, including parsers, can be found at [=>XMLPARS] 
and [=>XMLRES]. In the following sections we look at a few representative examples. As already mentioned, it is to be expected that the upcoming versions of the 
browsers and office tools that will be available mid-1999 will come with a built-in 
XML parser. 
6.6.5.1 Interoperability of XML documents 
An XML document must know where to find the data corresponding to external 
entity references. To make it possible to move documents between computers, a 
mechanism that would allow such transparent entity management had to be agreed 
upon by the SGML/XML community. 
281 

 
%%page page_302                                                  <<<---3
 
282 
HTML, SGML, and XML: Three markup languages 
Therefore the OASIS Consortium (Organization for the Advancement of 
Structured Information Standards, until 1997 known as SGML-Open) proposed 
a set of conventions to map external entity identifiers to filenames, URIs, or other 
storage objects. Moreover, they define a catalog file that contains public identifiers 
associated with each of the files to be interchanged. The details are at [¢->OASIS]. 
In order to get an idea of what a catalog file looks like, consider the following 
small file: 
OVERRIDE YES 
SGMLDECL "/afs/cern.ch/user/g/goossens/sgml/dtds/xml.decl" 
PUBLIC "ISO 8879-1986//ENTITIES Added Latin 1//EN`` ''iso-lat1.gml" 
PUBLIC "-//LWC//DTD Invitation//EN" 
"http://lwc.org/dtds/invitation.dtd" 
ENTITY ``mylogo'' "graphicsfiles/mylogo.eps" 
NOTATION ``gif'' "/usr/local/bin/xv" 
BASE "http: //wwinfo . cern. ch/~goossens/sgml/dtds/ " 
:5~ooc\Ic«v-AwN>When the OVERRIDE is YES (line 1), the PUBLIC, ENTITY, DOCTYPE, or NOTATION 
entries in the catalog will take precedence over an explicit system identifier that 
might have been specified for a given external identifier. SGMLDECL (line 3) specifies 
the location of the SGML declaration that defines the syntax of the SGML document. (For HTML or XML this declaration is fixed, but it should be specified for 
a general SGML parser like nsgmls; see Section 6.5.3.) Lines 5-7 associate a URI 
with a public identifier. 
A catalog file can also define entities and notations (lines 9 and 10). The BASE 
keyword allows system identifiers to be specified relative to the given absolute system path (line 11) 
Efforts are underway to cast catalogs into an XML syntax [=>XCATALOG]. 
6.6.5.2 The Element Structure Information Set 
Several parsers analyse XML documents and output a simple text representation 
of its Element Structure Information Set (ESIS) (see [h>ESIS] for a detailed description). ESIS was probably the first proposal for a standard output format for 
parsed SGML data. It is very convenient for transmitting data between a parser 
and an application, for example, one that will format or otherwise transform the 
input instance. Hence it comes as no surprise that many XML parsers provide an 
ESIS output stream making it easy to compare their behavior. Therefore we will 
show an example of ESIS output obtained by an XML processor. Note that most 
XML processors will also provide more modern interfaces in the form of eventbased Application Programmer Interfaces (APIs), such as SAX (see Section B.6), or 
tree-based APIs, such as DOM [h>DOMGEN]. 

 
%%page page_303                                                  <<<---3
 
6.6 XNIL parsers and tools 
283 
6.6.5.3 The nsgmls parser 
Many providers of SGML software have adapted their products to treat XML. As an 
example, let us begin with one of the longest standing and most general tools, the 
SGML suite of James Clark and, in particular, his nsgmls parser. This parser can 
be used to validate XML documents against a DTD; it generates ESIS output (see 
Section 6.6.5.2), which is also available with other parsers. 
nsgmls can be downloaded from James Clark’s home page at [9 SF]. The program is part of the SP suite of programs and is implemented in the C++ language. A 
binary executable version is available for various platforms, but you can also download the source to compile the program yourself in the rare case that the programs 
are not directly available as executables for your platform. The command-line options that follow are explained in detail in the documentation that comes with the 
program. (It can also be found at [hr SPDOC].) 
nsgmls [-vCegBdlprsu] [-b bctf] [-f e'r'ro'r_f'£Le] [-c cataLog_f'iLe] 
[-D dir] [-a L'£nk_type] [-A arch] [-E ma.'z;_e'r'ro'rs] [-i entity] 
[-w wa'rmlng_type] [-m cataLog_sys'id] [-0 o'utp'ut_opt'ion] 
[-t 'rast_f'ile] '13np'u.t_f7,'le(s) 
Usually only the -c switch to indicate the location of catalog file(s), -s (run xilently) 
to show only error messages, and -w to set the warning level, together with the 
name(s) of the XML input file(s) need concern the average user. By default ESIS 
output for the input document is generated (unless -s is specified). Before starting to parse the document, nsgmls looks for a “catalog" file that describes where 
commonly used SGML files, entity sets, and SGML declarations are located on the 
system (see the following section). 
6.6.5.4 Example of ESIS output 
As explained in Section 6.6.5 .2, SGML and XML parsers often provide ESIS output 
to represent the structure of the document. Following we show sample output generated by nsgmls. For clarity, we first explain the meaning of the most common 
elements of the ESIS output format. 
\\ a \\n a record end. 
\nnn character whose octal code is nnn. 
(gi start of element whose generic identifier is g'i,- attributes for this element 
are specified with A commands. 
) gi end of element whose generic identifier is gi. 
-data data. 
&na.me reference to external data entity name. 

 
%%page page_304                                                  <<<---3
 
284 
HTML, SGML, and XML: Three markup languages 
Aname key next element has an attribute name with keyword key (as described in 
Section 6.5.4.2). 
?p'i processing instruction with data pi. 
Nnname notation nname (preceded by p or s command to specify public or system 
identifier, respectively). 
ssysid system identifier sysid associated with some other commands (which it 
precedes). 
ppubid public pubid identifier associated with some other commands (which it 
precedes). 
C signals that the document was a conforming document (it will then be the 
last line in the output). 
We obtain the following ESIS output by running the file emptyexa.mple.xml 
(see page 271) through nsgmls. The meaning of the various keywords printed at 
the beginning of the lines should be clear from the earlier list. 
(emptyexample 
(par 
-The XML logo is shown in the image 
Aname TOKEN xm1-logo 
(imref 
)imref 
)par 
(par 
-The image shows an alternative syntax for an empty element \n\012 
Aname TOKEN xm1-logo 
Aaddress CDATA http://www.ucc.ie/xml/xm1.gif 
(image 
)image 
)par 
)emptyexamp1e 
C 
Z'£C7.’.Z5E:S»oan\uc~u..c=svSimilarly the ESIS output obtained after parsing the file invitation . xml (see 
 25 4) follows: 
?xm1 version="1.0" 
(invitation 
(front 
(to 
-Anna, Bernard, Didier, Johanna 
)to 
(date 
-Next Friday Evening at 8 pm 
)date 
(where 
-The Web Cafe 
)where 
(why 
-My first XML baby 
)why 
)front 
3::E§:S~o:x>~u:>.v.4>w~v
 
%%page page_305                                                  <<<---3
 
6.6 XML parsers and tools 
285 
(body 
(par 
'\n\012I would like to invite you all to celebrate\n\012the birth of 
(emph 
-Invitation 
)emph 
-, my\n\012first XML document child.\n\O12 
)par 
(par 
-\n\O12Please do your best to come and join me next Friday\n\012 
evening. And, do not forget to bring your friends.\n\012 
)par 
(par 
-\n\012I 
(emph 
-really 
)emph 
- look forward to see you soon!\n\012 
)par 
)body 
(back 
(signature 
-Michel 
)signature 
)back 
)invitation 
C 
6.6.5.5 Handling incorrect documents gracefully 
In order to show how parsers handle source files containing errors, let us consider 
the following (invalid) document-we will call it wrong. ml for convenience. 
G I S E E S o m u o m A W ~ _ 
<?xm1 version=“1.0"?> 
<!DOCTYPE wrong [ 
<!ELEMENT wrong (par*)> 
<!ELEMENT par (#PCDATAIemph)*> 
<!ELEMENT emph (#PCDATA)«> 
]> 
<wrong> 
<par>This part has wrong entity syntax &lt;par&gt.</par) 
<emph>Emph text outside scope.</emph> 
<par>Here comes another error <par>a second level 
paragraph</par>.</par> 
<par>A wrongly nested <emph>construct</par></emph>. 
Some more text outside valid scope. 
<par>Reserved characters "&“ “<`` ''>" “;" .</par) 
</wrong> 
We obtain the following error messages from running nsgmls in XML mode on the 
file above (we had to break some lines that were too long): 
nsgmls -s -wxml urong.xml 
wrong.xml:8:47:E: general entity "gt." not defined and no default entity 
wrong.xml:8:50:W: reference not terminated by refc delimiter 
urong.xml:9:5:E: document type does not allow element ``emph'' here; 
assuming missing ``par'' start-tag 
wrong.xml:10:4:E: document type does not allow element ``par'' here 
wrong.xm1:10:34:E: document type does not allow element ``par'' here 
wrong.xml:12:4:E: document type does not allow element ``par'' here 

 
%%page page_306                                                  <<<---3
 
286 
HTML, SGML, and XML: Three markup languages 
wrong.xm1:12:43:E: end tag for ``emph'' omitted, but OMITTAG NO was specified 
wrong.xm1:12:23: start tag was here 
wrong.xm1:12:50:E: end tag for element ``emph'' which is not open 
wrong.xm1:14:4:E: document type does not allow element ``par'' here 
wrong.xm1:14:26:W: character "&" is the first character of a delimiter 
but occurred as data 
wrong.xm1:14:30:W: character "<" is the first character of a delimiter 
but occurred as data 
wrong.xm1:15:7:E: end tag for ``par'' omitted, but OMITTAG NO was specified 
wrong.xm1:9:0: start tag was here 
It is important to realize that nsgmls continues parsing (and signaling errors) until 
the end of the file. Several of the parsers stop at the firstfiztal error and thus do 
not signal all problems present in the file.9 In our example the first message tells 
us about the absence of the entity reference end character (the semicolon ;) for gt 
on source line 8. On source line 9 the presence of the <emph>...</emph> tags out 
of context (outside of a par element) is signaled. The parser tries to be clever and 
inserts a <par> start tag on line 9, but this means that the <par> start tags on line 
10 are (rightly) detected as invalid (par elements cannot be nested according to the 
DTD). On line 12 the absence of the end tag </emph> is detected at the close of 
the par element (since no tags can be omitted in XML, the parser displays a No 
Omittag message). Then the close element tag for the emph element is found at the 
end of the line and declared invalid. Once more, on line 14 the <par> start tag is 
rejected, since the (implicitly inserted) <par> start tag on line 9 is still active. The 
out-of-context line 13 is not signaled, and the “reserved" characters “Sc" and “<" 
are flagged. Then there are some complaints about unbalanced par tags. One can 
conclude that most errors have been caught, but information out of context is not 
always detected. One should try to correct the errors starting at the top of the file 
and work down until all error messages (and warnings) have disappeared. To see a 
few other parsers at work, we present them with the same file wrong. xml and show 
the error message they generate. 
6.6.5.6 XML for Java 
XML for Java is a validating XML parser written in Java. It was developed by Kent 
Tamura and Hiroshi Maruyama of the Tokyo Research Laboratory, IBM Japan, and 
can be downloaded from [h>XML4J]. It has support for the DOM and namespaces 
and offers a prototype for XPointers [=>XPTSPEC]. 
VVhen we run the file “wrong.xml" through this parser, we get the messages 
that follow (we must set the classpath variable to inform the Java interpreter 
where the classes reside). Note that this application uses standard Java so that it 
9Although this strategy appears not to be very useful if one wants to locate all errors in a minimum of 
time, it is compatible with XML’s principle of “Draconian" error-handling. Indeed, in Section Terminolag of the XML Specification [<->XMLSPEC] under the heading fatal error it is stated that a conforming 
XML processor must “not continue normal processing" once it detects a fatal error (although it may 
continue processing the data, it does not have to). 

 
%%page page_307                                                  <<<---3
 
6.6 XML parsers and tools 
287 
can be run on any computer platform where the Java environment is installed (the 
output that follows was obtained on VV1ndows NT). 
set classpath=d:\xml4j\xml4j.jar;d:\jdk1.1.6\src 
java trlx wrong.xml 
wrong.xml: 8, 51: Reference must end with ’;’. 
wrong.xml: 8, 51: Undefined entity reference, "&gt.;". 
wrong.xml: 11, 22: Element "<par>`` is not valid because it does not follow the rule, ''(#PCDATAIemph)*". 
wrong.xml: 12, 42: "</emph)" expected. 
wrong.xml: 12, 43: Element name expected. 
wrong.xml: 14, 28: Reference must end with ’;’. 
wrong.xml: 14, 28: Invalid character, ’"’, in reference. 
wrong.xml: 14, 31: Element name expected. 
wrong.xml: 15, 7: "</par)" expected. 
wrong.xml: 15, 8: Element "<par>`` is not valid because it does not follow the rule, ''(#PCDATAIemph)*". 
Again, the out-of-context line 9 is not detected, and the second-level <par> start 
tag on line 10 is not flagged. On the other hand, at the end of line 11 an invalid 
content is signaled. The incorrectly nested par and emph elements on line 12 are 
correctly found, and, moreover, the “." at the end of line 12 is flagged as out of 
context (nothing is said about line 13). The reserved characters “KL" and “<" are 
detected, but the associated error messages are somewhat indirect. Similarly the 
complaints referring to line 15 are not particularly useful since they refer to an 
expected / par and an invalid par element. 
6.6.5.7 Event handling and the /Elfred parser 
/Elfred is a parser written by David Megginson. It is a nonvalidating Java parser, 
optimized for maximal speed and minimal size. It is portable and works with most 
Java implementations. An important point is that /Elfred supports Unicode to the 
fullest extent possible in Java. It correctly handles XML documents encoded using 
UTF-8, UTF-16, Unicode, ISO-10646, and ISO-8859-1, so /Elfred can handle all 
major languages. 
The parser and its documentation are available from the /Elfred home page at 
[=>AELFRED]. The class library comes with a few example classes that demonstrate 
how an application sees XML “parse events." It means that each time the parser detects an element, an attribute, an entity, character data, and so on, it does a callback 
to a class where the consumer application can then take the necessary action (e.g., 
write ESIS output or write some informative text or prepare output in an output 
language, such as TEX). The eventdemo . class, which comes with the /Elfred distribution, writes information about each “event" to the output stream. Following 
we show the generated output for the emptyexample .xml file: 
java -classpath "$CLASSPATH:java/aelfred" EventDemo emptyexample.xml 
Start document 
Resolving entity: [document], pubid=null, sysid=file:emptyexample.xml 
Starting external entity: file:emptyexample.xml 
Doctype declaration: emptyexample, pubid=null, sysid=null 
Start element: name=emptyexample 
Ignorable whitespace: "\n" 
\IO\vIJ>L-an)».
 
%%page page_308                                                  <<<---3
 
288 
HTML, SGML, and XML: Three markup languages 
Start element: name=par 
Character data: ``The XML logo is shown in the image '' 
Attribute: name=name, value=xm1-logo (specified) 
Start element: name=imref 
End element: imref 
Character data: "." 
End element: par 
Ignorable whitespace: "\n" 
Start element: name=par 
Character data: 
"The image shows an alternative syntax for an empty element \n" 
Attribute: name=name, va1ue=xm1-logo (specified) 
Attribute: name=address, va1ue=http://www-ucc.ie/xml/xml.gif (specified) 
Start element: name=image 
End element: image 
End element: par 
Ignorable whitespace: "\n" 
End element: emptyexample 
Ending external entity: fi1e:emptyexamp1e.xm1 
28 End document: errors=0 
---....-.-......-.-...-.vu.AuaI\a»-O~Om\Ic>\VI-BL-aI\2-O~O:D 
N 
o« 
N 
\I 
To make it easier to write code that can handle events coming from various parsers, 
David Megginson and Tim Bray (in collaboration with others) decided that it would 
be appropriate to define a standard event-based API for XML parsers that they called 
SAX (for Simple APIfior XML, see [h>SAX]). We will discuss this important topic in 
Section B.6. 
Summary 
In this chapter, we have looked at the two decades that preceded the advent of XML. 
We noted the first occurrence of specific markup, followed by the realization that 
a more generic approach had many benefits. This culminated after many years in 
the publication of the SGML standard in 1986. Then came the Web and the HTML 
earthquake. Finally it was realized that better formal foundations were needed if 
the Web was going to live up to its promise and provide a truly global environment 
for information exchange, storage, and handling: XML was born. 
We have discussed in some detail the various componenents of XML. We have 
shown how to construct XML applications optimized for a given task by defining 
a dedicated language whose syntax is described with a document type definition 
(DTD). Various tools to help us develop, analyze, and debug DTDs have been discussed. In the final section we showed how XML parsers can be used to analyze the 
structure of an XML document. In Chapter 7 we will use this information to display the contents of an XML document in various output formats using style sheet 
languages. 

 
%%page page_309                                                  <<<---3
 
CHAPTER 7 
css, DSSSL, and XSL: 
Doing it with style 
In this chapter we first give a short historical overview of the main style sheet languages. Then we explain how you can use programming tools to associate formatting commands with XML documents. The remaining sections discuss the CSS, 
DSSSL, and XSL style sheet languages in some detail and show how they associate 
style with XML elements. In particular, we will show how to generate HTML and 
ETEX output from XML source files. 
7.1 Style sheet languages: A short history 
As we explained in Section 6.3.1, when the first computer typesetting languages appeared, formatting commands and text were interspersed in the source documents; 
there was no direct way to express the structural relation between the various document components. Once the drawbacks of this situation were recognized, several 
authors developed higher-level markup systems, such as GML, IATEX, and Scribe. 
The structural commands of these systems became macros that were expressed in 
function of commands written in a lower-level typesetting language (Script, TEX, 
and so on). This approach tried to separate content and presentation and contributed to making document sharing and reuse a lot easier. 
Previously we discussed the markup languages SGML, XML, and HTML; we 
stressed that these languages, in principle, do not specify how the various document 

%==========310==========<<<---2
 
%%page page_310                                                  <<<---3
 
290 
CSS, DSSSL, and XSL: Doing it with style 
element types are to be represented visually (on paper or screen), aurally (e.g., for 
the visually impaired), and so on. 
This statement is, of course, not completely true for HTML, since it started 
life as a simple communication language for disseminating information only on 
the Web, so structure and presentation were fully mixed. We have to only think 
about such elements as FONT, B, or attributes like ALIGN, COLOR, or WIDTH that have 
nothing to do with structural relations between document elements per se. The flaw 
in the approach of mixing presentation and structuring elements was soon realized, 
and one of the first standards to be developed by the W3C was CSS1, which became 
a recommendation in December 1996 ([h>CSS1], Lie and Bos (1997)). Since then 
the language has been substantially extended, and in May 1998, CSS2 [h>CSS2] was 
issued as a recommendation, replacing CSSI. CSS is a declarative language using a 
specialized ad hoc notation that is not user-extensible. It is targeted at HTML, but 
it can also be used with XML, provided that the XML document has a reasonably 
simple and linear structure that can be displayed without extensive manipulation. 
On the other hand, the SGML community had been working on ways to associate presentation with SGML tags. An important landmark was the publication in 1990 of the Formatting Output Specification Instance (FOSI) specification 
[G->FOSI]. This American military standard, developed in the framework of the 
CALS initiative, proposes a way to render documents marked up in SGML. Each 
SGML document has to be delivered with a FOSI that also uses SGML syntax. A 
FOSI contains values for characteristics for every tag used in the SGML DTD, in 
particular for every context in which the tag has a unique formatting requirement. 
It also specifies characteristics for attributes that affect formatting. 
For many years within the framework of ISO, research had gone on to develop a 
standard to describe general transformations between SGML documents and their 
formatting. This work culminated in the publication of the Document Style Semantics and Specification Language (DSSSL) standard (ISO/IEC:10l79, 1996). 
However, the DSSSL Specification was considered too complex for use with Internet documents, and in 1996 a subset, DSSSL-online [G-> DSSSLONL], also known as 
XS, was proposed for use with XML. Yet, as discussed in Section 7.5.2, DSSSL’s (and 
hence XS’s) expression language is based on Scheme, a dialect of the programming 
language Lisp. The syntax of Scheme looks rather unfamiliar to most end users of 
the Internet; therefore in August 1997, several major players in the XML world submitted a proposal for an Extemible Style;/Jeet Language, XSL for short, which adopted 
an XML-based syntax [HXSL97]. It included most of the functionality of DSSSLonline and CSS1, but many desirable features for a style language for complex XML 
documents (as opposed to linear, one-pass HTML documents, the target of CSS), 
were still lacking. 
After a lot of discussion, particularly on various XML-related Internet discussion lists, such as the XSL [h>XSLMAIL] and DSSSL [c-> DSSSLLIST] mailing lists, an 
XSL Requirements Summary [=->XSLREQ] was published in May 1998. Soon there
 
%%page page_311                                                  <<<---3
 
7.2 Programming or style sheets, which is better? 
291 
after, the W3C XSL Working Group started work on defining a first core version 
for XSL, which addresses part of the main issues raised in the Requirements document. A first working draft was released in mid-August 1998 [%>XSLWD]. Some 
changes will surely still be made before the final specification is published in the 
second half of 1999. However, as we are convinced that the basic syntax of XSL 
will not change drastically, in Section 7.6 we will take a closer look at XSL, as it is 
defined in the current draft. 
If you want to remain informed about the latest developments in the area of 
W3C style sheet activities, in particular CSS and XSL, consult the W3C Style Web 
Page [~>W3CSTYLE]. 
It is evident that CSS and XSL are likely to coexist because they address somewhat different needs. CSS will remain the style sheet language for Web documents, 
especially for dynamic formatting of online documents for multiple media. On the 
other hand, XSL will allow you to handle complex documents, such as those using 
multiple columns, interleaved column sets with multiple text flows, footnote zones, 
synchronized marginalia, math formatting, and mixed vertical and horizontal writing directions. In other words XSL is the tool you need to work with genuinely 
internationalized automated print publishing. 
The formatting parts of XSL and CSS are expected to become quite similar. 
In particular, the XSL Working Group will ensure that all CSS-based properties 
and values in XSL have the same meaning as in CSS. Conversely, additional formatting functionality that is included in XSL will be exposed and described in such 
a way that it can be used from CSS. The important point is that all formatting 
functionality described in W3C recommendations should use the same underlying 
model and the same terminology. Therefore a W3C Working Group has been set 
up to define a common W3 C Formatting Made]. It will underpin the specifications 
of all W3C specifications that expose formatting functionality, including HTML, 
MathML, CSS, XSL, SMIL (Synchronized Multimedia Integration Language), and 
SVG (Scalable Vector Graphics). 
Detailed information about style sheets, especially CSS, as well as the old syntax 
of XSL, and some pages on DSSSL-online, can be found in Boumphrey (1998). 
7.2 Programming or style sheets, which is better? 
Users of SGML systems have, since the very beginning, been faced with the problem 
of printing or viewing their documents; many solutions have been developed over 
the years. A quite detailed overview of SGML filters for transforming SGML source 
documents into an output format can be found in Smith (1998), while Flynn (1998) 
provides you with a practical guide to the many (commercial and free) SGML and 
XML tools available. Most of these tools come with their own particular syntax for 
describing how SGML element types are to be rendered in the target language. The 

 
%%page page_312                                                  <<<---3
 
292 
CSS, DSSSL, and XSL: Doing it with style 
book reviews commercial systems that offer full-blown SGML environments, such 
as Baliye [~>BALISE], Omnimark [~>OMNIMARK], SGMLC [H SGMLC], as well as 
freely available tools based on Perl, awk, nsgmls, and so on. It should also be mentioned that several text processing and document handling tools, such as Adobe’s 
FmmeMaker [9-> FRM] and its PDF language and Microsoft’s Ofifice Suite, plan to use 
XML internally to save the structural information about the source document. This 
is supposed to make the exchange of electronic documents between applications of 
different vendors a lot easier. 
Most of the filters available to transform XML documents into a format to be 
viewed, printed, or otherwise made available on the Internet involve programming. 
Therefore we will start our overview by presenting a simple Perl-based system, 
where we ourselves will program what we want to output for each of the element 
types in the XML source document. In Section B.6 we will consider a more general 
approach based on Java. 
In our first programming examples we target ETEX and HTML directly, because it is easier for you to follow what we try to achieve. However, as we explain 
later, a more generic approach, based on the notion of fimmztting objecty, is almost 
always a better investment because starting from a single style sheet several output 
formats can be generated merely by exchanging back-ends. Thus the choices between a programming language or a style sheet approach and between using flow 
objects or targeting a formatting language directly are questions you should consider before embarking on a project. By showing you several possibilities, we hope 
we will be able to give you a feeling for what is best in a given situation. 
7.3 Formatting with Perl 
In Section 6.6.5 .2 we explained how XML parsers generate ESIS output to represent 
the structure of a document instance. David Megginson has developed SGMLSpm 
[=->SGMLSPM], an extensible Perl5 class library for processing the output from the 
sgznls and nsgmls parsers. The distribution comes with a simple sample application 
sgmlspl, that shows how to use the class library. 
The application sgmlspl can be used to convert SGML documents to other 
formats by providing a ypecification file, where you specify in detail how each element, external data entity, CDATA string, and so on should be handled. Two example 
SGML files-sgmlspm. sgml, describing the class library, and sgmlspl . sgml, describing the application sgmlspl itself-come with the distribution. These files are 
marked up according to the DocBook DTD. There are also two Perl speczfication 
filey, to1atex.pl and tohtml.pl, that provide code to transform the DocBook 
markup used in the two SGML files into BTEX and HTML markup, respectively. 
To show how the procedure operates we will use the sgmlspl application to 
prepare ETEX and HTML formats for the invitation example that was introduced 
in Section 6.4.2.2 and used on several occasions in Chapter 6. 

 
%%page page_313                                                  <<<---3
 
7.3 Formatting with Perl 
7.3.1 Principles of operation 
The application sgmlspl uses an event model for treating the ESIS representation 
of a document. That is, each time the XML parser writes an ESIS “event" corresponding to a certain configuration in the XML source file, sgml spl will “perform" 
an action, expressed as a set of Perl instructions defined in the xpeczfication file. In this 
file, which imports the SGMLSpm Perl5 class module, you can define Perl packages 
and routines, read files, and create variables. However, most of the time you can 
limit yourself to adding simple Perl code. Moreover, in the interest of maintainability and orthogonality, it is good practice to put all low-level formatting commands 
in language-specific output files (such as ETEX packages or CSS style files). 
The distribution comes with a skeleton file ske1.pl that will generate a specification file containing stubs for all elements in the source XML document. The 
Perl module works with any parser that generates ESIS output. For our example we 
have chosen James Clark’s nsgmls to run the following command sequence: 
nsgmls invitation.xm1 I perl sg;m1spl.p1 skel.pl > invitation.p1 
The first part of the command generates an ESIS output stream that is piped into 
the sgmlspl .pl Perl application. This itself is controlled by the file skel .pl. The 
output of this chain of programs is written in the output file invitation . pl. The 
exact contents of this customized skeleton file depend on the document instance 
because a new procedure reference is written to the file for each distinct “SGML" 
event in the source. Let us look at what we find inside that file. 
# SGMLSPL script produced automatically by the script sgmlspl.pl 
# 
# Document Type: invitation 
# Edited by: 
use SGMLS; # Use the SGMLS package. 
use SGMLS::0utput; # Use stack-based output. 
# 
# Document Handlers. 
# 
sgml(’start’, sub {}); 
sgml(’end’, sub {}); 
# 
# Element Handlers. 
# 
# Element: invitation 
sgml(’<invitation>’, “"); 
sgml(’</invitation>’, ““); 
# Element: front 
sgm1(’<front>’, ““); 
sgml(’</front>’, ""); 
N-~.o.\.-~._..--.--.-...._..--.-.--._ 
®\lD\\4I.5;\~N--O~O®\lD\\a.5s\uN»-csom\|o\\;..g\uN»[...J 
to 
o 
293 

 
%%page page_314                                                  <<<---3
 
294 
css, DSSSL, and XSL: Doing it with style 
The file contains a line calling the sgml procedure for each of the start tags (lines 22 
and 26) and end tags (lines 23 and 27) of the elements in the XML source document. 
At the end of the file default handlers are provided for other events that can occur 
when parsing an XML document, such as character data, entities, and processing 
instructions. 
7.3.2 Generating a ETEX instance 
Let us first concentrate on BTEX and generate a version of the skeleton file for 
translating the XML file into a ETEX file that can be printed. We want to put as little 
low-level ETEX code at the Perl level as possible, so we will merely store necessary 
information in a set of variables and let the ETEX package file invitation. sty 
deal with the formatting details. Following is the ETEX incarnation inv2lat .pl of 
the skeleton file invitation.pl: 
1 
2 # SGMLSPL script produced automatically by the script sgmlspl.pl 
3 # 
4 # Document Type: invitation --> customization for LaTeX 
5 # Edited by: mg (August 14:11 1998) 
6 W .. 
7 
8 use SGMLS; # Use the SGMLS package. 
9 use SGMLS: :0utput; # Use stack-based output. 
10 
11 # 
12 at Document Handlers. 
13 # 
14 sgml(’start’, sub (D; 
15 sgml(’end’, sub (D; 
16 
17 # 
18 # Element Handlers. 
19 # 
20 
21 # Element: invitation 
22 sgml(’<invitation>’, "\\documentclass[]'(article}\n" . 
23 "\\usepackage-(invitation}\n" . 
24 " \\begin{document}\n") ; 
25 sgml( ’</invitation>’ , “\\end{document}\n“) ; 
25 
27 # Element: front 
28 sgm1(’<front>’ , "\\begin{Front}\n"); 
29 sgm1( ’ </front> ’ , “ \\end{Front}\n“); 
31 # Element: to 
32 sgml(’<to>’, "\\To{''); 
33 sgm1(’</to>’, "}\n"); 
# Element: date 
36 sgml(’<date>’, "\\Date{"); 
sgm1(’</date> ’ , “}\n“); 
39 # Element: where 
40 sgm1(’<where>’ , "\\Where{"); 
41 sgml( ’</where>’ , "}\n“); 
43 it Element : why 

 
%%page page_315                                                  <<<---3
 
7.3 Formatting with Perl 
295 
44 sgml(’<why>’, “\\Why{"); 
45 sgml(’</why>’, “}\n"); 
47 # Element: body 
48 sgml(’<b0dy>’, “\\begin{Body}\n"); 
W sgm1(’</b0dy>’, "\\end{Body}\n"); 
51 # Element: par 
52 sgml(’<par>’, “\\par "); 
53 sgml(’</par>’, "\n"); 
55 # Element: emph 
56 sgml(’<emph>’, “\\emph{“); 
57 sgml(’</emph>’, "}“); 
59 # Element: back 
60 sgml(’<back>’, “\\begin{Back}\n"); 
61 sgml(’</back>’, "\\end{Back}\n“); 
63 # Element: signature 
64 sgml(’<signature>’, "\\Signature{“); 
65 sgml(’</signature>’, "}\n"); 
66 # 
67 # Default handlers 
68 # 
69 sgml(’start_element’,sub { die "Unknown element: " . $_[O]->name; }); 
70 sgml(’cdata’,sub { output $_[0]; }); 
71 sg.m1()re:’u u); 
72 sgml(’pi’,sub { die "Unknown processing instruction: “ . $_[O]; }); 
73 sgml(’entity’,sub { die “Unknown external entity: “ . $_[O]->name; }); 
74 sgml(’conforming’,"); 
75 
76 1; 
You can clearly see how the outer document element invitation has been made 
to correspond to the ETEX document initialization, where we load the package 
invitation (lines 22-24). The front, body, and back elements become Front, 
Body, and Back environments (lines 28-29, 48-49, and 60-61, respectively). Most 
other elements are transformed into high-level ETEX commands with the same 
name (lines 33-45, and 64-65). Only for the par and emph XML elements do we 
use the explicit basic ETEX equivalents \par (line 52) and \emph (lines 5 6-5 7). 
Now that we have prepared our Perl script we can run the command 
nsgmls invitation.xm1 I perl sgmlspl.pl inv2lat.pl > invitation.tex 
and obtain the following BTEX file: 
\documentclass[]{article} 
\usepackage{invitation} 
\begin{document} 
\begin{Front} 
\To{Anna, Bernard, Didier, Johanna} 
\Date{Next Friday Evening at 8 pm} 
\Where{Th¢ Web Cafe} 
\Why{My first XML baby} 
\end{Front} 
\begin{Body} 
\par I would like to invite you all to celebrate 
the birth of \emph{Invitation}, my 
5 E 3 o m N o m A w N _ 

 
%%page page_316                                                  <<<---3
 
296 
css, DSSSL, and XSL: Doing it with style 
first XML document child. 
\par Please do your best to come and join me next Friday 
evening. And, do not forget to bring your friends. 
\par I \emph{rea11y} look forward to see you soon! 
\end{Body} 
\begin{Back} 
\Signature{Miche1} 
\end{Back} 
\end{document} 
l\II\a»4----->-->-->-.-o~Om\lD\VIA\u 
Before formatting this file with ETEX, we must also look at the ETEX package 
file invitation. sty. In principle, thanks to the fact that we used high-level commands, we are free to format the above BTEX markup in many ways. In the instance 
that follows we choose one possible implementation, and it is perhaps interesting to 
clarify a few points about it. We use a tabular environment to typeset the front 
material (lines 10-16). Moreover, at the end (lines 25-29) we store the content of 
the components of the front material, as well as of the signature in global variables 
(\gdef),1 so that their value can be used inside the table (lines 12-15) and the boxed 
signature at the end (line 22). It is clear that it is straightforward to change the 
code shown. Without having to modzfir the XML source or Perl script upstream, you 
can command almost any presentation you want. 
2 invitation.sty 
2 Package to format invitation.xm1 
\setlength{\parskip}{1ex} 
\setlength{\parindent}{Opt} 
\pagestyle{empty}%Z Turn off page numbering 
\ReqnirePackage{array} 
\newenvironment{Front}Z 
{\begin{center}\huge \sffami1y Memorandum\end{center} 
\begin{flush1eft} 
\begin{tabular}{@{}>{\pfseries}P{.2\linewidth}@{}P{.8\linewidth}@{}}\hline 
} 
{To whom: & \@To \\ 
Occasion: & \@Why \\ 
Venue: & \@Where \\ 
When: & \@Date \\\hline 
\end{tabu1ar} 
\end{flush1eft} 
} 
\newenvironment{Body}{\vspace*{\parskip}}{\vspace*{\parskip}} 
\newenvironment{Back} 
{\begin{f1ush1eft}} 
{\hspace*{.5\linewidth}\fbox{\emph{\@Sig}} 
\end{f1ush1eft} 
} 
\newcommand{\To}[1]{\gdef\@To{#1}} 
\newcommand{\Date}[1]{\gdef\@Date{#1}} 
\newcommand{\Where}[1]{\gdef\@Where{#1}} 
\newcommand{\why} [1] {\gdef\<nwhy{#1}} 
\newcommand{\Signature}[1]{\gdef\@Sig{#1}} 
::S*O®\lO\\lI-5>\o4v\a.-----~.--._..--._-._.._.._.._. 
oo\uo~v.A~.a~.-o~ooo\no~v..;.w~ 
cu 
~o 
1Here the \gdef TEX commands define in a global way the command sequence given immediately 
following the value of its argument. For convenience, inside IEFTEX package and class files, the (9 character 
is used as a letter to define internal command names that are associated with user-callable commands 
(see Goossens et al. (1994), pages 15-16). 

 
%%page page_317                                                  <<<---3
 
7.4 Cascading Style Sheets 
297 
Memorandum 
To whom: Anna, Bernard, Didier, Johanna 
Occasion: My first XML baby 
Venue: The Web Cafe 
When: Next Friday Evening at 8 pm 
I would like to invite you all to celebrate the birth of Invitation, my first XML 
document child. 
Please do your best to come and join me next Friday evening. And, do not 
forget to bring your friends. 
I really look forward to see you soon! 
\reffig{7-1}: XML file formatted with ETEX using the sgmlspl procedure 
After compiling the ETEX file with TEX using the given package instance, we obtain 
the result shown in \reffig{7-1}. The frame is not part of the source but was added 
for clarity. 
7.4 Cascading Style Sheets 
The second version of the Cascading Style Sheets specification (CSS2, [~>CSS2]), 
published in May 1998, is a style sheet language that associates style (rendering 
information) with structured documents (primarily HTML, but also XML). It is 
important to realize that the CSS model allows both producers and consumers of a 
document to intervene in the process of controlling presentation. On the one hand, 
authors or publishers often prefer to define a set of style characteristics that give a 
distinctive and recognizable look to their publications. On the other hand, readers 
may have different expectations and be guided by their personal tastes, limitations 
in their software or hardware, or other physical constraints (color blindness, impaired sight, and so on). ‘ 
Therefore conflicts between different style sheets must be resolved. That is 
where the cascading in CSS comes into play. CSS has the possibility to assign an 
implicit or explicit priority to each style element. The style that has the highest 
priority wins. By default, the author’s style sheet overrides declarations in the user’s 
style sheet. However, the author and user can associate ! important keywords with 
certain rules, and for those the user specifications take precedence. 

 
%%page page_318                                                  <<<---3
 
298 
CSS, DSSSL, and XSL: Doing it with style 
In the following, we review those aspects of the CSS2 specification that we need 
to address the simple rendering of XML documents. For a complete treatment, the 
full specification or one of the many books on CSS should be consulted. 
7.4.1 The basic structure of a CSS style sheet 
A CSS style sheet consists of a list of statementy. These statements can be of two 
forms: at-ruler and rule yety. 
At-rules start with an @ sign followed immediately by an identifier (e.g., 
Qimport, Qcharset, @media) and terminate with a semicolon. CSS-aware software must ignore at-rules that it does not recognize. Especially interesting is the 
@import rule that allows style rules to be imported from other style sheets. All 
Qimport rules must precede all rule sets in the style sheet and must refer to the URI 
of the style sheet to be included. For instance, 
@import "mystyle.css"; 
will import the CSS style sheet mystyle . css. 
A rule set consists of a xelector followed by a declaration block. A declaration block 
starts with a left curly brace {, follows a set of semicolon-separated declarations, 
and terminates with a right curly brace }. A selector, whose syntax will be detailed 
following, is everything preceding the open curly brace of the declaration block. If 
the CSS application cannot parse a selector, it must ignore the complete rule set, 
that is, all declarations in the declaration block. 
selector {declaration 1; declaration 2, ...} 
A declaration consists of a property, followed by a colon ( : ), followed by a value. 
selector {property 1: value 1; property 2: value: 2} 
A property is an identifier. The syntax of value depends on the property in 
question. CSS applications must ignore declarations with an invalid property name 
or invalid value syntax. 
7.4.1.1 Selectors in more detail 
A selector is a pattern to select elements in the document tree to which a given style 
rule should be applied. If a certain element fulfills all the conditions of the pattern, 
then the selector is said to match the element. 
A selector is a chain of one or more yimple yelectory separated by combinators 
(whitespace, >, or +). 
Ayimple yelector is either an element-type selector or a zmiverxal xelector followed 
immediately by zero or more attribute yelectory, id selectors, or pxeadoclaxyey, in any 
order. 

 
%%page page_319                                                  <<<---3
 
7.4 Cascading Style Sheets 299 
Selections are made more specific by prepending a supplementary simple selector or combinator to an existing selector. The last simple selector can also have 
one pseudoelement appended. 
Several declarations for a same selector can be grouped, for instance, 
front {font-size: 12pt} 
front {font-style: bold} 
front {text-indent: Opt} 
is equivalent to 
front {font-size: 12pt; font-style: bold; text-indent: Opt} 
Similarly, identical declarations for multiple selectors can also be grouped. 
Thus the four statements: 
date {text-align: left} 
to {text-align: left} 
where {text-align: left} 
why {text-align: left} 
can be collapsed into the following single rule: 
date, to, where, why {text-align: left} 
The univemtl selector (*) matches the name of any single element in the document tree. A type selector matches the name of, at most, one element type in the 
document tree. 
Ancestor relationships can be conveniently expressed by enuneration. As an 
example, “A D" states that element type D can be any descendant of its ancestor A. 
front emph {font-style: italic} 
body emph {font-style: italic; color: blue} 
The first rules specify that emph elements with a front ancestor are represented in 
an italic font style, whereas emph elements with a body ancestor will be in an italic 
font and painted in blue. 
Direct parent-child relations are expressed with the > notation. For instance, 
body > par > emph {font-style: italic; color: red} 
specifies that emph elements that have a par parent and a body grandparent are in 
red italic type. 

%==========320==========<<<---2
 
%%page page_320                                                  <<<---3
 
300 
css, DSSSL, and XSL: Doing it with style 
Adjacent siblings can be selected with the + notation. For instance, the text 
associated with a why element that is immediately preceded by a where in the document tree will be painted in yellow by the following declaration: 
where + why {background-color: yellow} 
7.4.1.2 Handling attributes 
You can also associate a rule with an element for which an attribute has a certain 
value. The syntax allows several ways to select elements. The more important are 
shown in the following example: 
1 invitation[to] {font-size: 14pt} 
2 invitation[signature=``Peter''] {text-align: right} 
3 invitation[why~=``birthday''][date~=``Friday''] {background-color: green} 
Line 1 chooses all invitation elements with an attribute to, independent of 
the value specified for that attribute. Line 2 chooses invitation elements whose 
signature attribute is exactly Peter. Finally, line 3 chooses invitation elements 
with a why attribute containing the string “birthday" and a date attribute containing the string “Friday." 
To differentiate between default and explicitly specified attribute values, one 
can specify a general rule for an element and then specialize it by associating it with 
an attribute. 
1 invitation {text-align: left} 
2 invitation[why] {text-align: center} 
The general rule (line 1) specifies that invitation elements should be set left 
justified, whereas invitation elements with a why attribute specified should be 
center aligned (line 2). 
One can select specific elements in the document tree with an id selector and 
can associate specific styling information with them. In this case the DTD must 
declare id attributes for the elements in question. In particular, the HTML 4 DTD 
declares id attributes for all its elements. Hence, it is straightforward to associate 
a specific style with an element by using its id attribute. The syntax that is used in 
the style sheet is to precede the corresponding definition with a hash sign (#), for 
instance, 
1 *[color~=’myblue’] {color=``blue''} 
2 sect1#specialblue {color=``navyblue''} 
Line 1 defines the characteristics of all elements (we use the universal selector *) 
in the document with a color attribute equal to myblue. Line 2 selects among the 
sectl element types the one that has an id attribute equal to specialblue. 

 
%%page page_321                                                  <<<---3
 
7.4 Cascading Style Sheets 
301 
7.4.1.3 Pseudoclasses and pseudoelements 
The CSS2 Specification allows for a more fine-grained selection of information 
present in the document by using pxeudoelements (e.g., the first letter of a paragraph) and pseudotlassex (choosing elements by characteristics that cannot usually 
be deduced from the element’s name, its attribute, or its content). A few examples 
follow: 
1 body > parzfirst-child {text-indent: Opt} 
2 body * emph {colorz black} 
3 section > parzfirst-child emph {colorz red} 
Line 1 assigns a textindentation of zero points to a par element that is the first child 
of a body element. Line 2 says that all emph elements that have a body element as 
an ancestor should be shown in black. Finally, line 3 specifies that all emph elements 
inside a “first-child" par element inside a sect ion element should be in red. 
Some other pseudoclasses follow: 
: link i unvisited link; 
:visited visited link; 
: active element activated by the user; and 
:focus element in focus (accepting events). 
See Section 5.11 of the CSS2 Specification for more details. 
Similarly, pseudoelements allow one to single out the first line or the first letter 
of a document element or to generate text. See Section 5.12 of the CSS2 Specification. 
7.4.1.4 HTML’s span and div elements and class attribute 
Today, few browsers support the XML document model, so most documents must 
still be converted to HTML for viewing. As the HTML tag set is fixed, the HTML 
community came up with a convenient way to “extend" the language with the introduction of the span and div element types and the class attribute. We will use 
this combination in the CSS-based formatting examples in the following sections, 
so we want to explain its basis. 
CSS HTML I.'ltl.S‘.S‘€J‘2 allow you to create grouping schemes among styled HTML 
element types by specializing the style definition of a particular class with respect to 
the generic definition for the same element types that do not belong to that class. 
In CSS1 style sheets, a class name is preceded by a period (.), while CSS2 deprecates 
that format and replaces it with the attribute notation [class"=value] to identify 
class-specific commands. 
2This technique can also be used with XML by declaring a class attribute for each element in the 
DTD, as is the case in the HTML 4 DTD. 

 
%%page page_322                                                  <<<---3
 
302 
CSS, DSSSL, and XSL: Doing it with style 
For instance, in a style sheet you could specify the following: 
1 par {font-family: serif; font-size: 10pt} 
2 to {font-family: sans-serif; font-weight: hold} 
3 [class=``red''] {color: red} /* generic class specification */ 
4 to[class=``green''] {color: green} /* class specification to element */ 
5 .blue {color: blue} /* Deprecated syntax, HTML only, DO NOT USE! */ 
These specifications are used as follows (remember an XML DTD must declare a 
class element for each element for this to work): 
<par>Serif 10pt font.</par> 
<par class=``blue''>Serif 10pt but in blue.</par> 
<to class=``red''>Sans-serif, bold and in red.</to> 
<to class=``green''>Sans-serif, bold and in green.</to> 
.swN._. 
With HTML 4, the span element allows you to control the style of an inline 
text fragment, for instance, 
1 <p>Usual style with a <span class=``spec''>bit of text rendered using 
2 the “spec" style rule.</span> Back to the previous style.</p> 
On the other hand, the HTML 4 div element lets you apply a style to a whole 
block of text, which can include other HTML elements, as follows: 
1 <div class=``mydiv''> 
2 <p>The style “mydiv" controls this whole text block, 
3 including <q>this quote</q> and <cite>this citation</cite>.</p> 
4 <p>It even extends over several paragraphs, since here also 
5 the same “mydiv" style rules.</p> 
5 </div) 
In HTML the class, id, and style attributes can be used with the <span> and 
<div> tags. This allows for an implicit extension mechanism for HTML, by allowing you to define logical containers and apply a customized style to their contents. 
7.4.2 Associating style sheets with a document 
In HTML, <LINK> or <STYLE> mgr inside the document’s HEAD element are available to associate a style sheet with the document instance. Moreover, locally you 
can use STYLE attribute; to customize presentation for a single element. This latter 
practice is, however, discouraged, since it mixes form and content in a document. 
It is much better to use the id attribute to associate style information with one or 
more elements. 
On the document level, a W3C Note [¢->XMLSTYLE] by]ames Clark proposes 
to use an XML processing instruction with a target xml-stylesheet to provide 

 
%%page page_323                                                  <<<---3
 
7.4 Cascading Style Sheets 303 
the same functionality as the HTML <LINK> tag. For instance, the semantics of the 
following HTML 4 <LINK> tag: 
<LINK href="mystyle.css" re1=``stylesheet'' type="text/css"> 
would correspond to the following XML processing instruction: 
<?xml-stylesheet href="mystyle.css" type="text/css"?> 
Multiple processing instructions are allowed by using the alternate attribute 
on the processing instruction xml-stylesheet. The title attribute in the example is optional. 
1 <?xm1-stylesheet a1ternate=``yes'' title=``a1t1'' 
2 href="specia11.css" type="text/css"?> 
3 <?xm1-stylesheet a1ternate=``yes'' title=``a1t2'' 
4 href="specia12.css" type="text/css"?> 
5 <?xm1-stylesheet href="generic.css" type="text/css"?> 
7.4.3 A quick look at CSS properties 
CSS2 has over one hundred properties (see Appendix F of the CSS2 Specification 
for a tabular overview). This section takes a quick look at the more frequently used 
properties-those with which most ETEX users will be familiar and those which we 
will use in our examples. 
7.4.3.1 Setting background/foreground colors and images 
The color property controls the foreground color of an element, while the 
background-color property controls the background color or image of an element. 
Colors are specified by a predefined name or a triplet value in rgb (redgreenblue) color space, for example, the box element can have its text and background 
given in the following (equivalent) ways: 
box {color: red; background-color: yellow} 
box {color: #FFOOOO; background-color: #FFFFOO} 
box {color: rgb(255,0,0); background-color: rgb(255,255,0)} 
box {color: rgb(100%,0,0); background-color: rgb(100%,100%,O)} 
-bw~_ 
Line 1 uses one of the sixteen predefined color names in HTML and CSS. Lines 2-4 
express colors in function of the rgb color model, lines 2-3 as a number between 
zero (no such color component) and 25 5 (full quota of the given color component). 
In particular, line 2 uses a compact hexadecimal notation, while on line 3 a decimal 
representation is chosen. Line 4 expresses the same colors as percentages of the 
primary components. 

 
%%page page_324                                                  <<<---3
 
304 
CSS, DSSSL, and XSL: Doing it with style 
7.4.3.2 Fonts 
font-family Name of the typeface or font family. It can be a specific name, 
such as Baskerville or Helvetica, or one of the following generic 
names: serif, sans-serif, monospace, cursive, or fantasy. 
font-style Style of the typeface: normal, italic, or oblique. 
f ont-variant Variation of typeface: normal or small-caps. 
f ont-stret ch Amount of condensing or expanding. 
font-weight Weight (boldness) of the typeface. It can be a keyword (bold, 
lighter,...) or a number from the series 100, 200,..., 900 
(higher numbers are darker). 
font-size Size of the typeface. It can be specified in absolute (12pt, 
x-small) or relative units (larger, 120%, . . . ). 
Examples of definitions for different kinds of paragraphs follow. You can see 
that font weight, stretch, and size can all be expressed absolutely or relative to the 
value of the same characteristic of the parent element. 
1 P[class=``normal''] 
2 {font-family: serif} 
3 P[class=``italic''] 
4 {font-family: serif; font-style: italic} 
5 P[class=``bold''] 
5 {font-family: serif; font-weight: bold} 
7 P[class=``explight''] 
3 {font-family: serif; font-weight: lighter; font-stretch: expanded} 
9 P[class=``ttsc14''] 
1o {font-family: monospace; font-variant: small-caps; font-size: 14pt} 
11 P[class=``ssolarger''] 
12 {font-family: sans-serif; font-style: oblique; font-size: larger} 
There also exists a shorthand notation “font," which sets f ont-style, 
font-variant, font-weight, font-size, line-height, and font-family in one 
go, a usual practice in traditional typography. All properties that are not explicitly 
specified are set to their default values. Let us consider the following example: 
P {font: normal small-caps bigger/120% Helvetica} 
This sets the font-variant to “small-caps," the font-size one step “bigger" 
than that of the parent element, the line-height to “120%" of that font size (using the conventional typographer’s notation font-s'ize/ Line-height), and the 
f ont-f amily to “Helvetica." The keyword “normal" applies to the two remaining 
properties, font-style and f ont-weight. 

 
%%page page_325                                                  <<<---3
 
7.4 Cascading Style Sheets 305 
7.4.3.3 Text and visual formatting 
line -height Minimal height of each generated inline box. This is also used 
to specify the normal spacing between lines of text. 
text-indent Indentation of first line in a block of text. 
text-align Alignment of inline content of a block (left, right, center, 
or justify). 
text-decoration Decorations (underline, overline, blink, line-through). 
letter-spacing Spacing between text characters. 
word-spac ing Spacing between words. 
text-tra.nsform Capitalization control (capitalize, uppercase, lowercase, 
none). 
vertical-align Vertical alignment of inline boxes (baseline, top, middle, 
bottom, and so on). 
An example for a title heading follows: 
title {text-decoration: underline; text-transform: uppercase; 
letter-spacing: .1em; word-spacing: .5em} 
The text of the title element will be uppercase and underlined. The spacing between each letter will be z'mre.az.ved by .1 cm with respect to the normal spacing of 
the font, while .5 em will be added to the default space between words. 
7.4.3 .4 Boxes 
CSS’s formatting model is based on rectangular boxes that are generated from elements in the document tree. Each box has a content area and is surrounded optionally by margin, border, or padding areas. Box properties control the dimensions and 
characteristics of these rectangles and determine how the content of the element is 
formatted to fit into the provided space. 
The following properties: 
margin-top, margin-bottom, margin-left, margin-right, margin 
control the size of the top, bottom, left, right, and all margins. 
padding-top, padding-bottom, padding-left, padding-right, padding 
set padding on the top, bottom, left, right, and all sides of the element. 
border-top, border-bottom, border-left, border-right, border 
specify width, style, and color of the border on the top, bottom, left, right, or 
all sides of the element. 
There are more specific border-width, border-style, and border-color box 
properties to set each of these characteristics separately. 

 
%%page page_326                                                  <<<---3
 
306 
CSS, DSSSL, and XSL: Doing it with style 
The content width and height of a box are specified with the width and height 
properties (e.g., for scaling images). One can also make box material shift with 
respect of the current line using the float property, which can have as values left, 
right, or none, meaning no float. 
An example is the following itemize list element and its item child, for which 
we allow two variants: with and without a border. 
1 itemize {margin: lem lem lem lem; padding: .3em} 
2 item {marginz .5; padding: .2em Oem .2em .2em} 
3 item[class=border] 
4 {border-style: dotted; border-width: thin; border-color: red} 
Line 1 specifies that all margins of the itemize element are one quad (1 em) wide, 
with a padding equal to .3 em at all sides.3 Line 2 specifies a margin of half a quad 
for the item element inside the itemize element, with a padding equal to .2 cm at 
all sides except the right, where the text flows up to the margin. Finally, lines 3-4 
specify that for an item element of class border, on top of the characteristics of 
line 2, a border Consisting of a thin, red sequence of dots will be drawn. 
7.4.3.5 Displays 
The display property specifies how a certain block-level element has to be shown 
(displayed). This property can have many values, in particular, inline, block, 
run-in, and compact. A special value is none, which results in the element and 
its descendants generating no boxes in the output (formatting) tree and thus having no effect whatsoever on the layout. This does not imply that the material in 
question is invisible, since visibility itself is controlled by the visibility property, 
whose values are visible, hidden, and collapse. 
7.4.4 CSS style sheets for formatting XML documents 
With the information of the previous section we can now write a CSS style sheet for 
formatting an XML document. Once more, let us take our invitation document 
and transform it into HTML, the only language most present-day browsers display 
succesfully. 
We use the same procedure as in Section 7.3, this time with the following Perl 
script inv1html.pl: 
1 mm 
2 if SGMLSPL script produced automatically by the script sgm1sp1.p1 
3 it 
4 13 Document Type: inv1htm1.p1 (for HTML/CSS formatting) 
3Wl'1en only one value is specified, it applies to all four sides; with two values the top and bottom are 
set to the first value, the left and right to the second value; with three values, the top is set to the first, 
the left and right to the second, and the bottom to the third value; four values are assigned in the order 
top, right, bottom, left. 

 
%%page page_327                                                  <<<---3
 
7.4 Cascading Style Sheets 
5 # Edited by: mg (24 Aug 98) 
6 W W 
7 
8 use SGMLS; # Use the SGMLS package. 
9 use SGMLS::Uutput; # Use stack-based output. 
10 
11 # 
12 # Document Handlers. 
13 it 
14 sgml(’start’, "<HTML>\n<HEAD>\n" . 
U "<TITLE> Invitation (sgmlpl/CSS formatting) </TITLE>\n" . 
16 "(LINK href=\"invit.css\" rel=\"style-sheet\" type=\"text/css\">\n" . 
U "<!~- 24 August 1998 mg -->\n" . 
m "</HEAD>\n"); 
19 sgml(’end’, "</HTML>"); 
21 # 
22 # Element Handlers. 
23 # 
24 
25 sgml(’<invitation>’, "<BnDY>\n<H1>INVITATIDN</H1>\n"); 
26 sgml(’</invitation>’, "</BDDY>\n"); 
23 sgml(’<front>’, "<P><TABLE>\n<TBDDY>\n"); 
29 sgm1(’</front>’, "</TBODY>\n</TABLE>\n"); 
31 sgm1(’<to>’, "<TR><TD c1ass=\"front\">To: </TD>\n<TD>"); 
32 sgml(’</to>’, "</TD></TR>\n"); 
34 sgm1(’<date>’, "<TR><TD c1ass=\"front\">When: </TD>\n<TD>"); 
35 sgml(’</date>’, "</TD></TR>\n"); 
37 sgml(’<where>’, "<TR><TD c1ass=\"front\">Venue: </TD>\n<TD>"); 
38 sgml(’</where>’, "</TD></TR>\n"); 
% sgml(’<why>’, "<TR><TD class=\"front\">Dccasion: </TD>\n<TD>"); 
41 sgml(’</why>’, "</TD></TR>\n"); 
43 sgml(’<body>’, ""); 
44 5gm1()(/body)’, nu); 
46 sgm1(J(pa_r>)’ u(p>u): 
47 sgml(’</par>’, "</P>\n"); 
49 sgml(’<emph>', "<EM>"); 
so sgm1(’</emph>’, "</EM>"); 
52 sgml(’<back>’, ""); 
53 sgm1(’</back>’, ""); 
55 sgml(’<signature>’, "(P CLASS=\"signature\">"); 
56 sgml(’</signature>’, "</P>\n"); 
58 sgml(’start_element’,sub { die "Unknown element: " . $_[O]->name; }); 
59 sgml(’cdata’,sub { output $_[O]; }); 
61 1; 
On lines 14-18 we initialize the HTML document and define its header. On line 16 
in particular, we associate the style sheet invit . css with the HTML file. On line 
25 we start the body of the document with an H1 element, which will be typeset in a 
particular way. We decide to use a TABLE (line 28) for formatting the front material 
307 

 
%%page page_328                                                  <<<---3
 
308 
CSS, DSSSL, and XSL: Doing it with style 
and exploit the HTML/CSS class mechanism to control the formatting of the leftmost cells in the table (lines 31, 34, 37, and 40). The paragraphs (lines 46-47) and 
emphatic text (lines 49-50) elements are translated into their HTML equivalents. 
The signature is translated into a special signature class P element (line 55). The 
character data of the input file is transferred to the HTML file with the output 
statement on line 59. 
VVhen we run the command: 
nsgmls invitation.xm1 I perl sgm1sp1.p1 inv1htm1.pl > invcss.html 
we obtain the HTML file shown here: 
3 o m 4 m % ¢ W N _ 
----._...a._._a._......--.\lO\In.5\.4N»-O~OO0\lO\\n.;;\uN 
N 
no 
N 
0 
<HTML> 
<HEAD> 
<TITLE> Invitation (sgmlpl/CSS formatting) </TITLE> 
(LINK href="invit.css" rel=``style-sheet'' type="text/css"> 
<!-- 24 August 1998 mg --> 
</HEAD> 
<BUDY> 
<H1>INVITATIUN</H1> 
<p><TABLE> 
<TBUDY> 
<TR><TD class=``front''>To: </TD> 
<TD>Anna, Bernard, Didier, Johanna</TD></TR> 
<TR><TD class=``front''>When: </TD> 
<TD>Next Friday Evening at 8 pm</TD></TR> 
<TR><TD class=``front''>Venue: </TD> 
<TD>The Web Cafe</TD></TR> 
<TR><TD class=``front''>Uccasion: </TD> 
<TD>My first XML baby</TD></TR> 
</TBDDY> 
</TABLE> 
<P>I would like to invite you all to celebrate 
the birth of <EM>Invitation</EM>, my 
first XML document chi1d.</P> 
<P>Please do your best to come and join me next Friday 
evening. And, do not forget to bring your friends.</P> 
<P>I <EM>really</EM> look forward to see you soon!</P> 
(P CLASS=``signature''>Miche1</P> 
</BODY> 
</HTML> 
Finally, we have a look at the style sheet file invit . css, which serves as a link 
between the input XML source and the way the HTML output file will be displayed. 
1 
2 
3 
4 
5 
6 
7 
8 
9 
10 
1 
12 
13 
/* invit.css: CSS style-sheet for invitationl in HTML */ 
BODY {margin~top: 1em; /* global page parameters */ 
margin~bottom: 1em; 
margin~left: 1em; 
margin-right: 1em; 
font-family: serif; 
line-height: 1.1; 
color: black; 
} 
H1 {text-align: center; /* for global title */ 
font-size: x-large; 
} 
P {text-align: justify; /* paragraphs in body */ 

 
%%page page_329                                                  <<<---3
 
7.4 Cascading Style Sheets 
margin-top: 1em; 
} 
TD[class=``front''] { 
text~a1ign: left; 
font~weigl1t : bold ; 
/* table data in front matter */ 
‘-v-' 
{font-style: italic; /* emphasis in body */ 
} 
P[c1ass=``signature''l { /* signature */ 
text-align: right; 
font-weight: bold; 
margin-top: lem; 
---~._.._....._..-._ 
O\|II.h\o4tu»-O~€0D\lO\UI.A 
‘-v-' E 
Lines 2-8 set some global characteristics for the whole HTML document. We use 
the H1 (lines 10-11) element for making a large centered title. The treatment of 
paragraphs (lines 13-14) and emphatic text (line 20) is straightforward. Table data 
elements TD of class “front" (lines 16-18) have their text in bold and left-aligned, 
while paragraphs P of class “signature" have bold text, but right-aligned. However, current versions of both Netscape and Microsoft Internet Explorer do not 
yet implement the complete CSS2 Specification. Therefore for using the CSS style 
sheet shown earlier for viewing the HTML file with those browsers, we had to use 
old CSS1 syntax when specifying class-specific declarations-TD . front (on line 16) 
and P . signature (on line 22). 
The result of viewing the HTML file invcss .html with a browser that applies 
the CSS style sheet invit . css is shown in \reffig{7-2}. 
7.4.5 The invitation example revisited 
The invitation XML document and its DTD, first introduced in Section 6.4.2.2, 
contain only elements and do not use attributes. Let us introduce a variant of 
the invitation DTD, where most of the elements have been replaced by attributes. It is equivalent informationwise to the original version.4 The new DTD, 
invitat ion2 . dtd, follows: 
<!-- invitation2 DTD --> 
<!ELEMENT invitation (par+)> 
<.'ATTLIST invitation date cmm mzquiann 
signature CDATA #REQUIRED 
to cmn amsqumen 
where CDATA #REQUIRED 
why CDATA #IMPLIED > 
<!ELEMENT par (#PCDATA|emph)*> 
<!ELEMENT emph (#PCDATA)> 
~€0O\lO\\n.A\o4N.-We are now reduced to three elements: invitation (line 2), par (line 8), and emph 
(line 9). All supplementary information is specified as attributes of the invitation 
4See also Sections B.4.4.2 and B.4.4.3 in the Appendix where we develop two alternate DTDs for 
BIBTEX using a similar strategy. 
309 

%==========330==========<<<---2
 
%%page page_330                                                  <<<---3
 
310 
CSS, DSSSL, and XSL: Doing it with style 
\michel\lwcwork\invcss.html 
INVITATION 
To: Anna, Bernard, Didier, Johanna 
When: Next Friday Evening at 8 pm 
Venue: The Web Cafe 
Occasion: My first XML baby 
I would like to invite you all to celebrale the birth ot‘!m2itattcm, my first 
XML document child. 
Please do your best to come andjoin me next Friday evening. And, do not 
forget to bring your friends. 
I really look forward to see you soon! 
\reffig{7-2}: XML file formatted with HTML using the sgmlspl procedure 
element (lines 3-7). The XML document invitation2.xml itself is marked up 
relative to this DTD as follows: 
3C.2'C.E.``_''5~cao\uo~w-.c.-...~._ 
-_.---_. 
.-o~oan\l 
<?xml version="1.0"?> 
<!DOCTYPE invitation SYSTEM "invitation2.dtd"> 
<invitation to=“Anna, Bernard, Didier, Johanna" 
date=``Next Friday Evening at 8 pm'' 
where=``The web Cafe'' 
why=``My first XML baby'' 
signature=``Michel'' 
> 
<par> 
I would like to invite you all to celebrate 
the birth of <emph>Invitation</emph>, my 
first XML document child. 
</par> 
<par> 
Please do your best to come and join me next Friday 
evening. And, do not forget to bring your friends. 
</par> 
<par> 
I <emph>really</emph> look forward to see you soon! 
</par) 
</invitation) 

 
%%page page_331                                                  <<<---3
 
7.4 Cascading Style Sheets 
311 
One could argue that the structure of the document is a little less clear and that 
the attributes do not add anything, but it will serve as an example to show how 
attributes are handled. We emphasize once more that the information content is 
completely identical to that of the document on page 25 4. 
7.4.6 Generating HTML with another document instance 
To show that the principle of reusing style sheets also works for CSS, let us consider 
the alternate form invitation2 . ml of the invitation example as introduced in the 
previous section. Once more we use David Megginson’s SGMLSpm Perl module 
but with a different user script inv2htm1.p1, as follows: 
1 . 
2 it SGMLSPL script produced automatically by the script sgmlspl.pl 
3 # ' 
4 it Document Type: inv2html.pl (for HTML/CSS formatting) 
5 it Edited by: mg (25 Aug 1998) 
6 
7 
8 use SGMLS; it Use the SGMLS package. 
9 use SGMLS::0utput; it Use stack-based output. 
10 « 
11 it 
12 # Document Handlers. 
13 # 
14 sgml(’sta.rt’, sub { 
15 output "<1-ITML>\n<HEAD>\n"; 
16 output "<TITLE> Invitation (sgmlpl/CSS formatting) </TITLE>\n"; 
17 output "<LINK href=\"invit.css\" rel=\"style-sheet\" type=\“text/css\">\n"; 
18 output "<!-- 24 August 1998 mg -->\n"; 
19 output "</HEAD>\n“; 
20 D; 
21 sgml(’end’, "</1-ITML>"); 
22 
23 it 
24 # Element Handlers. 
25 # 
Z6 
27 it Element: invitation 
28 sgm1(’<invitat:'Lon>’, sub { 
29 my ($element,$%vent) = G!_; 
30 it First save the information for further use 
31 it Local variables 
32 my $date = $element->attribute(’date’)->value; 
33 my $to = $e1ement->attribute(’to’)->Value; 
34 my $where = $element->attribute(’where’)->value; 
35 my $why = $e1ement->attribute(’why’)->value; 
36 it Global variable (saved for end of document) 
37 $main: :GLsig = $element->attribute( ’signature’ )->va1ue; 
38 it Output the HTML commands needed for the front matter 
39 output "<BODY>\n<H1>INVITATION</H1>\n"; 
40 output "<P><TABLE>\n<TBODY>\n"; 
41 output "<TR><TD class=\"front\">To: </TD>\n<TD>$to</TD></TR>\n“; 
42 output “<TR><TD class=\“front\">When: </TD>\n<TD>$date</TD></TR>\n“; 
43 output "<TR><TD class=\"front\">Venue: </TD>\n<TD>$where</TD></TR>\n"; 
44 output "<TR><TD class=\"front\">0ccasion: </TD>\n<TD>$why</TD></TR>\n"; 
45 output "</TBODY>\n</TABLE>\n"; 
46 D; 
47 

 
%%page page_332                                                  <<<---3
 
312 
CSS, DSSSL, and XSL: Doing it with style 
48 sgm1(’</invitation)’ , sub{ it signature and end of document 
49 output "<P CLASS=\"signature\">$main: :GLsig</P>\n"; 
50 output "</BODY>\n"; 
51 3-) ; 
53 # Elements: par and emph 
54 sg-m1(’<pa.r>’, "<P>"); 
55 sgm1(’</pa.r>’ , "</P>\n"): 
57 sgm1(’<emph>’, "<EM>"); 
58 sgm1(’</emph>’, "</EM>"); 
so sgm1(’cdata’,sub { output $_ [O]; 3-); 
61 1; 
As with the script inv1html.pl shown in Section 7.4.4, we also need to handle attribute values here. As the attributes are specified on the <invitation> 
tag (lines 3-7 of the invitation2.xml source document on page 310), we 
have to extract these attributes when the invitation start element event occurs 
(lines 28-46). First, the element handle is extracted on line 29 into the variable 
$element. The value of the attributes is accessed using the value method of the 
SGMLS_Attribute class. The class is obtained by applying the attribute method 
of the SGMLS_E.lement class (lines 32-35 for the local variables-those preceded 
with the my specifier--and line 37 for the global variable $main: :GLsig, whose 
value we store until we encounter the end of the invitation element, where we 
use it to output the signature on line 49). Thus we generate essentially the same 
HTML code as in Section 7.4.4. In particular we reference (line 17) the same CSS 
style sheet (invit . css) as before. When we treat the XML source with the earlier 
Perl script, we obtain a file inv2css . html as follows: 
nsgmls invitation2.xm1 I perl sgm1sp1.p1 inv2htm1.p1 > inv2css.htm1 
Viewing this HTML file with a browser, we observe a layout identical to the one 
shown in \reffig{7-2}. This clearly shows that CSS styles are very convenient to 
manage the display characteristics of documents in a global way. 
7.5 Document Style Semantics and Specification 
Language 
Document Style Semantics and Specification Language (DSSSL) ISO/IEC:10l79 
(1996) is an international ISO/IEC standard to specify the formatting and transformation of SGML (and hence XML) documents. As we explained before, an SGML 
document should completely ignore the rendering and other processing aspects 
of its data. DSSSL’s two basic aims are to format SGML documents for paper or 
electronic media presentation and to transform SGML documents between markup 
schemes defined by different DTDs. DSSSL offers a standardized framework and 

 
%%page page_333                                                  <<<---3
 
7.5 Document Style Semantics and Specification Language 
313 
SGML 
Transformer 
document 
SGML DSSSL Output 
document formatter format 
\reffig{7-3}: The DSSSL process 
methods for associating processing information with SGML element instances or 
general classes of element types. 
This section contains an introduction to DSSSL that should allow you to 
develop a simple DSSSL style sheet for your XML document. More in-depth 
tutorial-like introductions to DSSSL have been written by Daniel M. German 
[QDSSSLTUTB] and Paul Prescod [QDSSSLTUTA]. A lot of DSSSL-related information is also available from James Clarl<’s DSSSL Web page [9 DSSSLCLARK]. 
7.5.1 The components of DSSSL 
DSSSL provides four distinct areas of standardization (clauses that follow refer to 
the DSSSL specification (ISO/IEC:l0179, 1996)): 
1. 
A language and processing model for transforming SGML documents into 
other SGML documents. Here we use a transformation language and a transformation specification, which is a list of associations (Clause 11). 
A language for specifying how to apply formatting characteristics to an SGML 
document. The formatting process is controlled by a style specification, which 
contains a list of construction rules. DSSSL standardizes only the form and 
semantics of the style language, not the formatting process itself (Clause 12). 
A Standard Document Query Language (SDQL) for identifying portions of 
an SGML document. SDQL allows for easy navigation through the hierarchical SGML structure and lets one identify pieces of a document for processing. 
DSSSL also defines a subset called the core query language (Clause 10). 
An expression language used in the previous languages to create and manipulate 
objects. DSSSL also defines a subset called the core exprexxion language. DSSSL’s 
expression language uses a side-effect-free subset of the Scheme Programming 
Language (Clause 8). 
\reffig{7-3} shows the complete. DSSSL process schematically. 

 
%%page page_334                                                  <<<---3
 
314 
CSS, DSSSL, and XSL: Doing it with style 
7.5.1.1 DSSSL’s style language 
In the following discussion we shall be mainly interested in DSSSL’s style language 
because it provides a standardized, powerful language for describing the formatting of SGML documents. In Section 7.5.3 we will look at Jade, a powerful and 
freely available DSSSL processor that implements the style language. It will be the 
workhorse for our examples. 
Central to the style language is the flow object tree. It is an abstract representation of how the source document and the formatting specifications are merged. 
At the nodes of this tree we find flow objectx that provide a standard framework for 
describing document layout with constructs, such as page sequences, paragraphs, 
tables, and artwork. Each flow object has a set of characteristics, such as page margins, paragraph indentation, a table border, and a picture’s height and width (these 
are similar to the CSS properties of Section 7.4.3). 
The result of formatting a flow object is a sequence of areax. An area is a rectangular box with a fixed width and height. It comes in two types: display and inline. 
Dixplay areax are not directly parts of lines; they have an inherent absolute orientation, and their positioning is specified by area containerx. These containers have 
their own coordinate system with a filling direction that defines their starting edge. 
They can have a fixed size or be allowed to grow. Various writing modes are possible 
inside such a container. 
Inline areax are parts of lines. An inline area has reference points on its edges so 
that subsequent inline areas can be positioned to form lines. Kerning can influence 
this positioning. 
Flow objects that are formatted to produce a sequence of inline areas are said 
to be inlined, while flow objects formatted to produce a sequence of display areas 
are said to be displayed. The fact that a flow object class instance can be only inlined, 
only displayed, or both inlined and displayed depends on the characteristics of the 
flow object or on the flow object attached to it. 
7.5.1.2 DSSSL’s other components 
The tramfbrmation language is a standard language for transforming the SGML 
markup, according to a given DTD, into a markup following another DTD. For 
instance, translating a DocBook document into “equivalent" TEI markup or, more 
probably, generating an HTML instance, because today most browsers support only 
that language. Formally the transformation language, especially in its expression 
part that is based on Scheme, has a lot in common with the style language. 
DSSSL specifications operate on trees of nodes to create new trees, such as 
the “flow object tree" that represents the formatted document instance or a new 
SGML document, obtained by a transformation. The nodes are organized into a 
specialized data structure, a “grove," a kind of tree of trees, or more technically, a 
possibly cyclic directed graph of nodes. Each node can have a set of “properties," 
that can have “atomic" values (such as strings, booleans, or integers), or be lists 

 
%%page page_335                                                  <<<---3
 
7.5 Document Style Semantics and Specification Language 
of nodes or references to nodes. This document model with groves as a parse tree 
data structure in the form of a graph of nodes with properties was developed for the 
specific needs of DSSSL and HyTime [h>HYTIME]. These two standards share the 
same fundamental abstract data view of SGML documents (for more information 
on groves, see [9 GROVES]). 
The Query Language is for selecting and returning document components in 
the form of nodes of the DSSSL/HyTime document model described earlier. Because we do not need this component of DSSSL we will not discuss it any further. 
7.5.2 Creating style sheets with DSSSL 
At first sight DSSSL’s syntax looks somewhat unusual, as it is based on Scheme, a 
member of the Lisp family of languages. However, as you will see in the examples 
that follow, once you “get used to the parentheses," DSSSL becomes quite straight~ 
forward. 
7.5.2.1 Construction rules 
As with other style languages, a DSSSL style sheet consists of a series of statements, 
called construction ruler, that “construct" a formatted document from an SGML 
source document. Let us first look at the line that follows. It instructs the DSSSL 
application that all par elements in the SGML source should produce a paragraph 
flow object in the output formatted flow object tree. 
(element par (make paragraph quadding: ’justify )) 
As we explained earlier, a flow object corresponds to a formatting object. The 
detailed description of DSSSL flow object classes, together with their characteristics, is given in Section 12.6 of ISO/IEC:l0179 (1996).5 For easy reference, we list 
all flow object classes available in DSSSL, together with their identification number 
in that section (for instance, entry 5 corresponds to Subsection 12.6.5 in the DSSSL 
Specification). Entries marked with 7 were not in the DSSSL-online subset, which 
was originally proposed as a style sheet language for XML. 
1 sequence 
display-group 
simple-page-sequence 
-sequencei 
column-set-sequencei 
paragraph 
\IO\'~J1-l>UJI\J 
paragraph-break 
5 An online PDF version of the DSSSL Specification for personal use is at [<->DSSSLPDF]. 
315 

 
%%page page_336                                                  <<<---3
 
316 
CSS, DSSSL, and XSL: Doing it with style 
8 line-f ield 
sideline 
10 anchorl 
11 character 
12 leader 
13 embedded-textl 
14 rule 
15 external-graphic 
16 included-container-areal 
17 score 
18 box 
19 side-by-sidel 
20 side-by-side-iteml 
21 glyph-annotationl 
22 alignment-point 
23 aligned-column 
24 multi-line-inline-notel 
25 emphasizing-markl 
26 flow object classes for mathematical formulael 
(1) math-sequence, (2) unmath, (3) subscript, (4) superscript, (5) script, 
(6) mark, (7) fence, (8) fraction, (9) radical, (10) math-operator, 
(11) grid, (12) grid-cell 
27 flow object classes for tables 
(1) table, (2) table-part, (3) table-column, (5) table-row, 
(6) table-cell, (7) table-border 
28 flow object classes for online display 
(1) scroll, (2) multi-mode, (3) link flow, (4) marginalia 
A list of the characteristics of all flow class objects, together with other useful 
tables, are available from Harvey Bingham’s DSSSL syntax summary Web page 
[&> DSSSLSUM]. 
For instance, if you want to find all the characteristics of the paragraph flow 
object that we used in the earlier example, then you should, according to the above 
list, consult Section 12.6.6. There you will find more than eight pages of characteristics. In particular, for quadding it states that it controls “the alignment of lines 
other than the last line in the paragraph . . . ." The relevant text also lists all the 
other values that quadding can take. 
Implementors can add their own flow objects._Iade (see Section 7.5.3) uses this 
possibility to generate SGML output; therefore you can use Jade to transform an 

 
%%page page_337                                                  <<<---3
 
7.5 Document Style Semantics and Specification Language 
XML document into HTML, for instance. Instead of writing Perl or Java code, as 
we have done in previous sections, you can, if you know a little about DSSSL, use 
Jade to transform XML documents into HTML, BTEX, and so on. 
For example, we can take the same par element type as earlier and instructJade 
to translate it into an HTML <P> tag, as follows (here we make use of Jade’s SGML 
extension): 
(element par (make element gi: ``P'')) ;creates an HTML paragraph 
7.5.2.2 A simple DSSSL specification 
DSSSL specifications are genuine SGML documents and are contructed according 
to a DTD. VV1th the Jade distribution comes a file style-sheet .dtd, which has its 
own public identifier. 
-//James Clark//DTD DSSSL Style Sheet//EN 
In fact, writing a DSSSL style sheet is not so difficult, and in most cases only a 
few flow objects have to be mastered. 
As an example, we will take our invitation example and generate simple 
HTML output (as we did with CSS in the previous section) but without writing 
the HTML code ourselves. Instead we use DSSSL’s flow objects and let Jade handle 
the translation. 
1 <!-- invitation.dsl --> 27 (make paragraph 
2 <!DUCTYPE style-sheet PUBLIC 28 (literal "Venue: ") 
3 "-//James Clark//DTD DSSSL Style Sheet//EN" 29 (process-chi1dren))) 
4 > 30 (element (front why) 
5 <style-sheet> 31 (make paragraph 
6 <style-specification) 32 (literal "Uccasion: ") 
7 <style-specification-body> 33 (process-children) )) 
8 (define tFontSize* 12pt) 34 (element (body par) 
9 (root 35 (make paragraph 
10 (make simple-page-sequence 36 quadding: ’justify 
II left-margin: 5mm 37 font-size: tFontSizet 
12 page-width: 100mm 38 space-before: 4-Fontsizet 
13 right-margin: 5mm 39 (process-children))) 
14 (make scroll 40 (element emph 
15 font-size: tFontSize* 41 (make sequence 
16 line-spacing: tFontSizet 42 font-posture: ’italic 
17 (process-children) ) )) 43 (process-children) )) 
18 (element (front date) 44 (element (back signature) 
19 (make paragraph 45 (make paragraph 
20 (literal "when: ") 46 quadding: ‘end 
21 (process-children))) 47 space-before: tFontSizet 
22 (element (front to) 43 (literal "From: ") 
23 (make paragraph 49 (process-children))) 
24 (literal "To: ") so </style-specification-body> 
25 (process-children))) 51 </sty}e-specification> 
26 (element (front where) 52 </style-sheet> 
Let us look at the DSSSL style sheet invitation.dsl above. Line 8 shows 
how we define a constant (*Fontsize*). By convention, we put stars at both ends 
317 

 
%%page page_338                                                  <<<---3
 
318 
css, DSSSL, and XSL: Doing it with style 
of the constants so that they are easily recognizable in the code. Then we compose the output representation by assembling flow objects. We define a root object 
and create a simple-page-sequence (line 10), setting some of its characteristics 
(lines 11-13). Inside the simple-page-sequence we create a scroll (line 14) that 
accepts flow objects for online display that do not have to be divided into pages. 
Apart from defining characteristics, inside a make expression and after the keyword 
argument list, we can specify a “content expression" that instructs the DSSSL processor what to put inside the current flow object. For instance, on line 17 we call 
the procedure pro cess-children, which processes the child nodes of the current 
element that is the root element (the complete document) in the present case. If 
no content expression is specified, DSSSL will include a (process-children) line 
automatically. However, if the current flow object cannot have children, it is called 
an “atomic flow object," for example, a character flow object. For clarity, though, 
it is good practice always to specify a content expression, if one can be present. In 
particular, you have to provide the content expression if you want to suppress some 
child nodes or otherwise selectively process children. Otherwise you have to introduce some flow objects that do not correspond directly to elements in the source 
document. 
The next part of the style sheet deals with the creation of flow objects for each 
of the possible elements in the source XML document. For instance, lines 18-21 
apply to date elements inside front elements (this is similar to the selector syntax 
front > element for CSS rules as explained in Section 7.4). On line 19 we create 
a paragraph flow object that builds line elements for inline areas. The paragraph 
element is always displayed. In the present case, it consists of the literal “when: " 
followed by whatever is the result of processing the child nodes of the current date 
element node. Similar paragraph flow objects are built for the other elements of 
the front matter. In lines 34-39 we encounter the code to deal with par elements 
inside a body element that is also turned into a paragraph flow object, but in this 
case we specify explicitly a few of its characteristics. In particular, we want to leave 
a space equal to the value of the variable *FontSize*, defined at the beginning of 
the file, in front of each new paragraph. Lines 40-43 declare how to handle emph 
elements, and here we introduce the sequence flow object that produces a concatenation of the areas produced by each of its children. In the present case we use it to 
set the font posture to italic. Finally for the back element and its signature (lines 
44-49), we build another paragraph flow object with the literal “From: " and the 
content of the element. 
This DSSSL style sheet and our XML source file can now be interpreted by a 
DSSSL processor to obtain formatted output. We will use Jade for our examples. 
7.5.3 Introducing Jade 
James Clark was the first to release a DSSSL processor. His Jade Games’ Awesome 
DSSSL Engine) is a fast C++ based processor that is available for VV1n32 and UNIX 

 
%%page page_339                                                  <<<---3
 
7.5 Document Style Semantics and Specification Language 
platforms [=>JADE].Jade implements a large subset of DSSSL’s style language, but 
it has (almost) no support for the transformation language. Jade can generate the 
following output formats: Microsoft’s RTF (Rich Text Format), TEX (with JadeTEX), 
XML, HTML (using a nonstandard DSSSL extension), a nonstandard “F OT" (Flow 
Object Tree in the form of flow objects as XML elements), and MIF (Adobe’s Ma/eer 
Interchange Format for FrameMaker). 
Binary distributions are available for some platforms, including Microsoft 
VV1ndows, which makes installation a simple matter of unpacking the distribution. 
On other platforms, Jade might have to be built from sources. 
Once Jade has been set up correctly, as described in the documentation, it is 
straightforward to run it. Jade uses an environment similar to that of nsgmls (see 
Section 6.6.5.1). In particular, you should set the SGML_CATALOG_FILES environment variable to inform Jade where the catalog file is located.Jade shares many of 
its command-line options with nsgmls. The others are explained on theJade home 
 [=>]ADE]. 
jade [-vCegG2] [-b encoding] [-f error_fiZe] [-c cataZog_sysid] 
[-D dir] [-a Zink_type] [-A arch] [-E mam_errors] 
[-i entity] [-w warning_type] [-d dsssZ__spec] [-V variable] 
[-t (fotlrtflhtmlltexlmif|sgml|xml)] [-0 output_fiZe] input_fiZe(s) 
For our purposes the more important new switches are -d, which allows you 
to specify the name of the DSSSL style sheet; -1;, which lets you define the type 
of output (the possibilities are listed above); -G, which initiates debugging and will 
show the evaluation stack when something goes wrong; and -0, which lets you 
specify the output file name. 
7.5.3.1 RunningJade 
Before we do some real work with Jade, let us mention an interesting feature that 
allows us to obtain the full textual content of a document. When an empty style 
sheet (empty . dsl, see below) is specified, the system executes a process-children 
instruction recursively for the root element down to the deepest level children. 
<!-- empty.ds1 --> 
<!DUCTYPE style-sheet PUBLIC "-//James Clark//DTD DSSSL Style Sheet//EN") 
<style-sheet> 
<style-specification> 
<style-specification-body> 
</style-specification-body> 
</style-specification> 
</style-sheet) 
co\lOx‘II-b\~w9>-The catalog file that is needed to associate the DSSSL public identifiers with 
system resources should contain the following: 
1 pmauc "-//James c1.-mg//urn DSSSL Flow Object Tree//EN`` ''fot.dtd" 
2 PUBLIC "-//James Clark//DTD DSSSL Style Sheet//EN`` ''style-sheet.dtd" 
3 PUBLIC "ISO/IEC 10179:1996//DTD DSSSL Architecture//EN`` ''dsss1.dtd" 
319 

%==========340==========<<<---2
 
%%page page_340                                                  <<<---3
 
320 
CSS, DSSSL, and XSL: Doing it with style 
VV1th these files and the XML document invitation. xml, we now run Jade. 
jade -t xml -d empty.dsl d:\jade\xml.dcl invitation.xml 
The XML declaration file xml .dcl must be loaded in front of invitation.xml 
because by defaultjade accepts only standard “SGML" syntax and will reject XMLspecific features, such as empty-element syntax. 
The output is shown here. We see that we get the character data content of all 
elements in the XML source file invitation . xml. 
Anna, Bernard, Didier, JohannaNext Friday Evening at 8 pmThe Web 
CafeMy first XML babyI would like to invite you all to celebrate the 
birth of Invitation, my first XML document child.Please do your best 
to come and join me next Friday evening. And, do not forget to bring 
your friends.I really look forward to see you soon!Michel 
Let us return to our DSSSL style sheet invitation. dsl that we introduced in 
Section 7.5.2.2 and run it through Jade with the XML source file invitation.xml 
to obtain various output formats. If we do not explicitly specify an output format 
(with the -t option),]ade produces an XML presentation of the flow object tree. 
jade -dinvitation.dsl d:\jade\xml.dcl invitation.xml 
The generated file invitation. fot follows. It displays clearly how the output 
is built according to the instructions for each element of the source tree described in 
the invitation .dsl style sheet. The order of the flow objects corresponds to the 
order with which the elements are constructed from parsing the source document 
invitation.xml (see page 254). 
<?xml version="1.0"?> <!-- invitation.fot --> 
<fot> 
<simp1e-page-sequence 1eft-ma.rgin="14.17pt" page-width="283.46pt" right-margin="14.17pt"> 
<scroll font-size=``12pt'' line-spacing=``12pt''> 
<paragraph> 
<a name=``0''/> 
<a name=``1''/> 
<a name=``2''/> 
<text>To: </text) 
<text>Anna, Bernard, Didier, Johanna</text> 
</paragraph) 
<paragraph> 
<a name=``3''/> 
<text>when: </text> 
<text>Next Friday Evening at 8 pm</text) 
</paragraph) 
<paragraph> 
<a name=``4''/> 
<text>Venue: </text) 
<text>The Web Cafe</text) 
</paragraph) 
<paragraph> 
<a name=``5''/> 
<text>Uccasion: </text) 
5~ow\:<7«v:-bwN-E E Z 3 G I S E 
N 
o 
N 
N 
N 
N 
9N 
-5 

 
%%page page_341                                                  <<<---3
 
7.5 Document Style Semantics and Specification Language 
321 
I-l}~.lI-I 
wavN! 
on 
N 
o 
9 W W W W o W W W W W 
a 0 W 2 W N W o O W W 3 W E 3 3 3 3 3 : 3 3 3 i & 3 2 3 3 g : 3 3 : 3 3 : 3 
<text>My first XML baby</text> 
</paragraph) 
<paragraph space-before=``12pt'' quadding=``justify'' font-si2e=``12pt''> 
<a name=“6"/> 
<a name=``7''/> 
<text> 
I would like to invite you all to celebrate 
the birth of </text) 
<sequence font-posture=``italic''> 
<a name=``8''/> 
<text>Invitation</text) 
</sequence> 
<text>, my 
first XML document child. 
</text) 
</paragraph) 
<paragraph space-before=``12pt'' quadding=``justify'' font-size=``12pt''> 
<a name=``9''/> 
<text> 
Please do your best to come and join me next Friday 
evening. And, do not forget to bring your friends. 
</text) 
</paragraph) 
<paragraph space-before=``12pt'' quadding=``justify'' font-size=``12pt''> 
<a name=``10''/> 
<text> 
I </text) 
<sequence font-posture=``italic''> 
<a name=``11''/> 
<text>really</text) 
</sequence> 
<text> look forward to see you soon! 
</text) 
</paragraph) 
<paragraph space-before=``12pt'' quadding=``end''> 
<a name=``12''/> 
<a name="13“/> 
<text>From: </text) 
<text>Michel</text> 
</paragraph) 
</scroll) 
</simple-page-sequence> 
</fot> 
VV1th the same input files we now ask Jade to generate an RTF file by entering 
the following command: 
jade -d1'nvitation.dsl -t rtf d:\jade\xml.dcl invitation.xml 
This generates a file invitation . rtf that we can View with Microsoft Word 
(\reffig{7-4}). Similarly by specifying -1: tex (rather than -1: rtf), Jade will transform our How object specifications into TEX code that can be interpreted by the 
TEX back-end ]adeTEX (see Section 7.5.4). In \reffig{7-5} we obtain an output file 
that is almost identical to \reffig{7-4}. 
7.5.3.2 Making tables in DSSSL 
Let us now get a little more ambitious and put the front material inside a DSSSL 
table construct to align the material better. 

 
%%page page_342                                                  <<<---3
 
CSS, DSSSL, and XSL: Doing it with style 
To: Anna, Bernard, Didier, Johanna 
When: Next Friday Evening at 8 pm 
Venue: The Web Cafe 
Occasion: My first XML baby 
lI'o: Anna, Bernard, Didier, Johanna 
When: Next Friday Evening at 8 pm 
Venue: The Web Cafe 
Occasion: My first XML baby 
I would like to invite you all to celebrate the birth of 
lwould like to invite you all to celebrate the birth of Invitation’ my first XML document ChildIrzvitation, my first XML document child. 
Please do your best to come and join me next Friday 
Please do your best to come andjoin me next Friday _ . . 
evenlng. And, do not forget to bring your fr1ends. 
evening. And, do not forget to bring your friends. 
I really look forward to see you soon! 
From: Michel 
I really look forward to see you soon! 
From: Michel 
\reffig{7-4}: Simple DSSSL style with RTF 
< !DOCTYPE style-sheet PUBLIC 
43 
\reffig{7-5}: Simple DSSSL style with TEX 
(make table-row 
1 
2 "-//James Clark//DTD DSSSL Style Sheet//EN" 44 (make table-cell 
3 > 45 (make paragraph quadding: ’start 
4 <style-sheet> 46 (literal "To:"))) 
5 <style-specification) 47 (make table-cell 
6 <style-specif ication-body> 48 (process-children) ) )) 
7 49 (element (front where) 
8 (define FontSize 12pt) so (make table-row 
9 51 (make table-cell 
10 (root 52 (make paragraph quadding: ’start 
11 (make simple-page-sequence 53 (literal "Venue:"))) 
12 left-ma.rgin: 2cm 54 (make table-cell 
13 page-width: 15cm 55 (process-children) ) )) 
14 right-margin: 2cm 56 (element (front why) 
15 (make scroll 57 (make table-row 
l6 font-size: Fontsize 51; (make table-cell 
17 (process-children)))) 59 (make paragraph quadding: ’start 
18 60 (literal "0ccasion:"))) 
19 (element front 61 (make table-cell 
20 (make sequence 62 (process-children)))) 
21 (make paragraph 63 
22 quadding: ’center 64 (element (body par) 
23 space-before: 20pt 65 (make paragraph 
24 font-weight: ’bold 66 quadding: ’justify 
25 font-size: 24pt 67 font-size: Fontsize 
26 (literal ``INVITATIUN'')) 68 space-before: Fontsize 
27 (make table 69 (process-children))) 
Z8 table-border: ttf ; no border 70 (element emph 
29 display-alignment: ’start 71 (make sequence 
30 space-before: 20pt 72 font-posture: ’ita1ic 
51 (make table-part 7s (process-children))) 
32 (make table-column width: 25mm) 74 
35 (make table-column width: 10cm) 75 (element (back signature) 
E4 (process-children))))) 76 (make paragraph 
35 (element (front date) 77 quadding: ’end 
36 (make table-row 78 space-before: FontSize 
S7 (make table-cell 79 (literal "From: ") 
38 (make paragraph quadding: ’start 80 (process-children)» 
39 (literal "When:"))) 81 
40 (make table-cell 2-12 </style-specification-body> 
41 (process-children) ) )) 33 </Style-specificat ion> 
s4 
(element (front to) 
</style-sheet‘) 

 
%%page page_343                                                  <<<---3
 
7.5 Document Style Semantics and Specification Language 
323 
Our approach is quite similar to the one we had with BTEX, but, of course, we 
have to specify the table in somewhat more detail. If we refer to the list of DSSSL 
flow object classes starting on page 315, we see that the table components (heading 
2 7) are described in Section 12.6.2 7 of the DSSSL specification, to which you should 
refer if you want to get more details about the way we construct the table. 
Lines 27-34 define a few general characteristics of the table (no border, column width, alignment, and distance from previous material) and instruct DSSSL to 
handle the children of the front element. These children are each handled in term: 
date (lines 35-41), to (lines 42-48), where (lines 49-55), and why (lines 56-62). In 
each case we construct the left-hand cell by specifying some literal text (lines 39, 
46, 53, and 60). Then in the right-hand cell, we deposit the contents of the front 
element’s child being considered. Lines 64-69 take care of starting a new paragraph 
by creating a paragraph flow object, while lines 70-73 translate XML’s emph element into an italic text sequence flow object. Finally lines 75-80 put the contents 
of the signature element right-justified (line 77), preceded by the literal From: 
(line 79). 
The main advantage of this approach is that invtab1.dsl, our DSSSL style 
sheet, can be used with Jade to obtain formatted output in various forms. In particular, we can generate RTF (for viewing or editing with Microsoft tools; \reffig{7-6} 
shows the RTF output Viewed with Microsoft Word97): 
jade -dinvtabl .dsl -t rtf d: \j ade\xml.dcl invitation.xml 
We can also generate TEX output. For this we use the flag -t tex, and postprocess the TEX file with ]adeTEX, discussed in more detail in Section 7.5.4. An 
output file is shown in \reffig{7-7}. 
jade -dinvtab1.dsl -t tem d:\jade\xml.dcl invitation.xml 
jadetex invitation 
It should be stressed that only the option specified with the -t flag was different 
between the two runs of Jade. 
7.5.3 .3 Handling attributes 
We can also write a style sheet for handling the invitation2 . xml source file. Here 
we need to get access to the value of the attributes of the elements to format the 
document. Such a style sheet invtab2 . dsl follows: 
<!DUCTYPE style-sheet PUBLIC "-//James Clark//DTD DSSSL Style Sheet//EN") 
<style-sheet> 
<style~specification> 
<style-specification-body) 
(define Fontsize 12pt) 
(root 
(make simp1e-page-sequence 
left-margin: 1cm 
-width: 10cm 
~c>ao\:oxvu4>x.a-V... 

 
%%page page_344                                                  <<<---3
 
324 
CSS, DSSSL, and XSL: Doing it with style 
MI: m» in Wuml 
INVITATION 
To: Anna, Bernard, Didier, Johanna 
When: Next Friday Evening at 8 pm 
Venue: The Web Cafe 
Occasion: My first XML baby 
Iwould like to invite you all to celebrate the birth of Irzvization, 
my first XML document child. 
Please do your best to come and join me next Friday evening. 
And, do not forget to bring your friends. 
I rmlly look forwardto see you soon! 
From: Michel 
INVITATION 
To: Anna, Bernard, Didier, Johanna 
When: Next Friday Evening at 8 pm 
Venue: The Web Cafe 
Occasion: My first XML baby 
I would like to invite you all to celebrate the birth of Invitation, 
my first XML document child. 
Please do your best to come and join me next Friday evening. 
And, do not forget to bring your friends. 
I really look forward to see you soon! 
\reffig{7-6}: Word97 View of RTF output 
10 
ll 
12 
13 
l4 
15 
16 
17 
IE 
19 
20 
21 
ZZ 
Z3 
Z4 
Z5 
Z6 
Z7 
Z8 
Z9 
30 
wwwwwww 
ac\.oxu..;>w~ 
-D -5-but 
§u.£.fi:‘3_o:o 
From: Michel 
right-margin: 1cm 
(make scroll 
font-size: Fontsize 
(process-children)))) 
(element invitation 
(make sequence 
(make paragraph 
quadding: ’center 
space-before: 20pt 
font-weight: ’bold 
font-size: 24pt 
(literal "INVITATIDN“)) 
(make table 
table-border: #f ; no border 
display-alignment: ’start 
space-before: 20pt 
(make table-part 
(make table-column width: 25mm) 
(make table-column width: 10cm) 
(make table-row 
(make table-cell 
(make paragraph quadding: ’start 
(literal "When:"))) 
(make table-cell 
(make paragraph quadding: ’start 
(literal (attribute-string “date“))))) 
(make table-row 
(make table-cell 
(make paragraph quadding: ’start 
(literal “To:“))) 
(make table-cell 
(make paragraph quadding: ’start 
(literal (attribute-string “to“))))) 
(make table-row 
(make table-cell 
(make paragraph quadding: ’start 
(literal "Venue:“))) 
\reffig{7-7}: PostScript View of TEX output 

 
%%page page_345                                                  <<<---3
 
7.5 Document Style Semantics and Specification Language 
325 
(make table-cell 
(make paragraph quadding: ’start 
(literal (attribute-string ``where''))))) 
(make table-row 
(make table-cell 
(make paragraph quadding: ’start 
(literal "Dccasion:"))) 
(make table-cell 
(make paragraph quadding: 'start 
(literal (attribute-string ``why'') ) ) ) ) )) 
(process-children) 
(make paragraph 
quadding: ’end 
space-before: Fontsize 
(literal "From: " 
(attribute-string "signature“))))) 
(element par 
(make paragraph 
quadding: ’justify 
font-size: Fontsize 
space~before: FontSize 
(process-children-trim))) 
(element emph 
(make sequence 
font-posture: ’italic 
(process-children-trim))) 
</style-specification-body> 
</style-specification> 
</style-sheet> 
\l\l\l\l\l\Io\0\D\D\D\ mam mmmmmmm tau: .5 
v~.s>«.4~.-o~oan\-o~u«$w~._.%,o¢n\.exuu.a«.4$.--o$mf. 
We use the same strategy to build the table here as we did in Section 7.5.3.2. 
However, instead of the process-children rule we extract the information with 
the procedure attribute-string, for which we specify the name of the attribute 
as a character string (see lines 35, 42, 49, 56, and 62). We also reduce the margins 
and page width somewhat (compare lines 8-10 in the two versions). 
We execute the same jade command as earlier, changing the DSSSL style sheet 
to invtab2.dsl and the XML input file to invitation2.xml. Then we run the 
obtained TEX file through jadet ex and finally get the EPS file shown in \reffig{7-8}, 
which should be compared to \reffig{7-7}. 
7.5.4 The TEX back-end for Jade and the JadeTEX macros 
As we mentioned earlier, Jade also has a TEX back-end. This offers several advantages: 
TEX is free, well-understood, and available for all machines. 
2. TEX is designed for rule-based batch typesetting. 
TEX is, generally speaking, good at page makeup and very good at paragraph 
makeup. 
4. TEX understands the full range of typesetting minutiae (hyphenation, fonts, 
math, and so on). 

 
%%page page_346                                                  <<<---3
 
326 
CSS, DSSSL, and XSL: Doing it with style 
INVITATION 
When: Next Friday Evening at 8 pm 
To: Anna, Bernard, Didier, Johanna 
Venue: The Web Cafe 
Occasion: My first XML baby 
I would like to invite you all to celebrate the 
birth of Invitation, my first XML document 
child. 
Please do your best to come and join me next 
Friday evening. And, do not forget to bring 
your friends. 
I really look forward to see you soon! 
From: Michel 
\reffig{7-8}: PostScript view of TEX output (alternate DSSSL formatting) 
We also have the perspective of using Unicode directly in TEX, with the Omega 
variant [‘->OMEGA]. This gives us a means to move onto the more complicated 
scripts, writing directions, and language conventions with which TEX itself has difficulties. 
Thus with Jade style sheets we have a chance to write device-independent specifications and to use TEX’s power to instantiate them. 
]ade’s TEX back-end was originally written by David Megginson and later modified by Sebastian Rahtz and Kathleen Marszalek. It has a very simple model: It 
emits a TEX command for the start and end of every flow object, defining any 
changed characteristics at the start of the command. This abstract TEX markup can 
then be fleshed out by writing definitions for each of the flow object commands, 
and this is what the ]adeTEX macro package provides. 
]adeTEX is implemented on top of the widely used ETEX macro package primarily because ETEX provides standardized font support (the New Font Selection System) that resembles that of DSSSL. The multilingual, color, graphics inclusion, hypertext, and tabular packages are also conveniently in place. This means 
that ]adeTEX provides a good shortcut to an implementation and allows us to see 
whether TEX can, in fact, meet the demands of DSSSL. For better performance, 
it would be possible to rewrite the font handling inside the Jade back-end and to 
optimize the handling of labels and references to minimize memory requirements 

 
%%page page_347                                                  <<<---3
 
7.5 Document Style Semantics and Specification Language 
327 
for cross-references. 
It is important for regular ETEX users to realize that this back-end cannot use 
I5TEX’s high-level constructs and their favorite style class files. Indeed, DSSSL expresses everything in terms of flow elements and has no notion of such familiar 
concepts as sections, lists, cross-references, or bibliographies. Thus TEX’s only responsibilities are page and line breaks; all the rest is specified by the DSSSL code. 
It would be possible to consider a system that translated the DSSSL style specification itself into a ETEX class file and to transform the document instance into 
a Ié}TEX file using high-level constructs, but this has not, to our knowledge, been 
attempted. 
7.5.4.1 Installation and usage 
After downloading the ]adeTEX macros from the CTAN archives [‘->]ADETEX], 
you should make them available to TEX. It is most convenient to build a new TEX 
format file. The sequence of commands you have to type would look like the following (if you are working with a modern TEX system based on Web2c 7.2 or later): 
tex jadetex.ins 
pdftex -ini "&pdflatex" -progname=pdfjadetex pdfjadetex.ini 
tex -ini "&hugelatex" -progname=hugetex jadetex.ini 
This produces two format files pdfjadetex.fmt and jadetex.fmt that can be 
moved to a directory where TEX can find them. The “huge" version of ETEX 
(hugelatex on the third line) needs to be set up, if it does not exist, since]adeTEX 
is extremely hungry for memory. Users of Web2c 7.2 (or a later version) can do this 
by adding the following lines to the file texmf . cnf: 
main_memory.hugetex = 1100000 
hash_extra.hugetex = 15000 
pool_size.hugetex = 500000 
string_vacancies.hugetex = 45000 
max_strings.hugetex = 55000 
pool_free.hugetex = 47500 
nest_size.hugetex = 500 
param_size.hugetex = 1500 
save_size.hugetex = 5000 
stack_size.hugetex = 1500 
and running 
tex -ini -fmt=hugelatex -progname=hugetex latex.ltx 
which generates huge1atex.fmt. Other TEX implementations have their own 
methods for generating format files with extra memory, and you should consult 
their documentation. 

 
%%page page_348                                                  <<<---3
 
328 
CSS, DSSSL, and XSL: Doing it with style 
Assuming everything was successful, you are then ready to generate a TEX file 
and run it through]adeTEX: 
jade -dinvtab1.dsl -t tex xml.dcl invitation.xml 
jadetex invitation Or pdfjadetex invitation 
dvips -E invitation -oinvitation 
The first line generates a TEX file containing commands reflecting the DSSSL flow 
object structure. These commands are interpreted by the ]adeTEX macros and 
pdfTEX or ETEX (line 2), generating directly a PDF or a DVI file, which can be 
transformed into an EPS file. This procedure was used to generate \reffig{7-7}. 
7.5.4.2 ]ade'l'EX, a closer look 
To clarify how]adeTEX handles the DSSSL information with the characteristics of 
flow objects, let us consider the following: 
(root (make simple-page-sequence 
center-footer: (page-number-sosofo) \SpS{'/. 
1 
Z 
3 font-family-name: body-font-family \def\fFamName{iso-serif} 
4 page~n-columns: 2 \def\PageNCo1umns{2} 
5 page-column-sep: 16pt \def\PageCo1umnSep{16\p@} 
6 header-margin; .5in \def\HeaderMargin{36\p@} 
7 footer-margin: .5in \def\FooterMargin{36\p@} 
8 left-margin: iin \def\LeftMargin{72\p@} 
9 right-margin: lin \def\RightMargin{72\p@} 
10 top-margin; iin \def\TopMargin{72\p@} 
H bottom-margin: iin \def\BottomMargin{72\p@} 
12 page-width: 21 1mm \def\Pageln'idth{598. 1 1\p<n} 
B page-height: 297mm)) \def\PageHeight{841.89\p@}} 
At the left-hand side we show a DSSSL specification that specifies parameters for 
a simple page sequence. At the right-hand side you see how ]ade’s TEX back-end 
translates this into an intermediate form that can be readily digested by the ]adeTEX 
macros ([‘->]ADETEXB] gives a detailed description of these parameters and the 
translation process). 
Look at the body of a document with the simple XML markup: 
some <it>go italic</it> others not... 
Look, too, at the following DSSSL declaration: 
1 (element it 
2 (make sequence 
3 font-posture: ’ita1ic 
4 (process-children-trim))) 
]ade’s TEX back-end will generate the following: 
1 some \Node{\def\E1ement{11}}% 
2 \Seq{\def\fPosture{ita1ic}}% 
3 go italic 

 
%%page page_349                                                  <<<---3
 
7.5 Document Style Semantics and Specification Language 
4 \endSeq{}\endNode{} others 
5 not . . .\endSeq{}\endNode{} 
Notice how the contents of the “italic" element type it has been processed as 
a DSSSL sequence that translates to TEX macros \Seq. . . \endSeq. The required 
changes in the font-posture characteristic are expressed with a TEX macro definition as a parameter to \Seq. Almost every object that comes out of Jade has an 
“Element" identifier (line 1) that can be used for cross-reference purposes. 
7.5.4.3 ]adeTEX and mathematics 
What about mathematics? This is TEX’s traditional strength and something that 
few typesetting systems handle well. Let us consider the following example; the 
XML markup and displayed result should be fairly clear. 
<fd><fr><nu>X</nu><de>Y></de></fr></fd> f 
Y 
We could Write a DSSSL specification like this: 
1 ; displayed equation 11 (make fraction 
2 (element fd 12 (procesrchildren-trim))) 
3 (make display~group 13 (element nu 
4 (make math-sequence 14 (make math-sequence 
5 math-display-mode: ’display 15 label: ’numerator 
6 min-leading; 2pt 16 (procesrchildren-trim))) 
7 font-posture: ’math 17 (element de 
8 (process-children-trim)))) 13 (make math‘5equenCe 
9 ; fraction 19 label: ’denominator 
10 (element fr 20 (procesrchildren-trim))) 
This uses the slightly difficult DSSSL concept of “ports" (the label characteristic) that allows the numerator and denominator to feed material to the relevant 
portions of the fraction flow object. ]ade’s TEX back-end would transform this into 
something like: 
1 \DisplayGroup{} 
2 \MathSeq{\def \MathDisplayMode{display} 
3 \def \MinLeading{2\p@} 
4 \def\Min.LeadingFactor{O} 
5 \def\fPosture{math}} 
6 \FractionSerial{} 
7 \insertFx-act ion.Bar{} 
8 \Fract ionflumeratori} 
9 \MathSeq{}X\endMathSeq{} 
10 \endFract ionNu.merat or{} 
1 \Fract ionbenominat ori} 
\MathSeq{}Y\endMathSeq{} 
12 
13 \endFract ionDenominator{} 
14 \endFract ionSerial{} 
15 \endMathSeq{} 
16 \endDisplayGroup{} 
We clearly see how the structure of the original formula has been conserved. 
Thanks to the TEX commands, which are called at each level, you could easily cus329 

%==========350==========<<<---2
 
%%page page_350                                                  <<<---3
 
330 
CSS, DSSSL, and XSL: Doing it with style 
tomize the presentation of the various math elements. The default implementation 
(simplified) of these macros is as follows: 
\def\FractionSeria1#1{#1\bgroup} 
\def\endFractionSeria1{\egroup} 
\def\FractionDenominator{} 
\def\endFractionDenominator{} 
\def\FractionNumerator{} 
\def\endFractionNumerator{\over } 
\def\insertFractionBar{} 
\lo\vw-b\o4N»-An interesting initiative is David Carlisle’s work on a DSSSL style sheet for 
mathematics expressed using the MathML DTD[‘->DSSSLMML]; we discuss this in 
more detail in Section 8.2.4. 
7.5.4.4 Is ]adeTEX usable in practice? 
It is not hard to process simple texts with Jade and to see more or less identical 
output from the RTF and the TEX back-ends (Figures 7.6 and 7.7). For conventional scientific publication, however, the simple page model that Jade implements 
is insufficient. The TEX back-end, therefore, implements a number of extensionsé 
which can be activated with the following DSSSL code: 
(declare-flow-object-class page-f1oat 
"UNREGISTERED::Sebastian Rahtz//Flow Object Class::page-float") 
(declare-flow-objact-class page-footnote 
"UNREGISTERED::Sebastian Rahtz//Flow Object C1ass::page-footnote") 
(declare-characteristic page-n-columns 
"UNREGISTERED::James Clark//Characteristic::page-n-co1umns" 1) 
(declare-characteristic page-column-sep 
"UNREGISTERED::James Clark//Characteristic::page-column-sep" 4pt) 
:n\noxv..x>m~._. 
These declarations allow you to specify a two-column page layout with specifications like the following: 
1 (make simple-page-sequence 
2 page-n-columns: 2 
3 page-column-sep: 16pt 
4 
and to create a new footnote flow object (page-footnote) or float new flow object (page-float). The result is demonstrated in \reffig{7-9}; it shows that a DSSSL 
specification, Jade, and ]adeTEX can produce plausible output for scientific texts. 
On the other hand, we found that mathematics support in RTF is too poor to provide acceptable output, so for math (at the moment) the only realistic way from 
XML to formatted output for print seems to be to go via TEX and ]adeTEX. 
The potential power of SGML/XML, DSSSL, and TEX working together really 
holds a lot of promise. Although there are some problems, the Jade DSSSL implementation already supports a huge amount of useful transformation and specification code, and TEX is close to being a DSSSL-capable formatter. 
6Support for multiple columns is also provided in the RTF back-end. 

 
%%page page_351                                                  <<<---3
 
7.5 Document Style Semantics and Specification Language 
331 
Sebastian Rahtz 
sementiam commendare sluduit. 
Test file for math, multicolumns, and footnotes 
Abstract: Altera C. Caesaris, qui illos publicatis bonis per municipia Italiae ui u'bueudos ac vinculis sempitemi 
acautem plures senatores ad C. Caesaris quam ad D. Silani sententiarn ' " viderent , M. Cicero ea, quae infra legitur_ oratione Silani 
‘ . existimabat. Cum 
1. Maths tests 
0. Simple fraction 
X 
Y 
1. display equation with radical 123 and fraction 
(27 + 3/) + \/ 123 
2. Display equation with supgr and subscripts 
11/1.. : (1 - eflg. 
3. Matrix with braces 
{a b ed 2 f }a:0b:2 
4. Line with I (after matrix) 
{a I: c d e f }\a:0b:2 
5. Line with I (before matrix) 
H 41 1. c d e f }a:0b:2 
6. Nested matrix with braces 
(X{a b c d e f }a:0b:2) 
7. Nested fraction X 
(z+y)+7 (ac+y)7-4a 
8 F 2 2 
. ence 
({ arm 17 c d e f }|:iXn5: 492mm : Ob : 2 
9. Boxing 
10. Some operators: summation, product, and integral. First. 
display math: 
b d f h 7272323? 
Z 1] f : sin (1 Z 
0 C 9 
iii1:':;= 
1 1. A radical with a radix 
V 123 
‘*3 Never leave home without rope, Sam could have told you. 
55 It always pays to be polite to trees that walk and talk 
6" Little can beat stewed rabbit in the heather. 
e 
. . _ b .1 f _ . 2222"" 
Now rnlrne math. 2“ [IE fe -.;‘ sin a Zunzfffi 
1.1. Second-level header 
2. Footnotes 
13. A footnote, number 63“ A footnote, number 65" A footnote, 
number 64“ 
3. Special character entities 
AElig .45 
And /\ 
Cap I'm 
Colon 
Cup v 
Dagger i 
Delta A 
ETH D 
Gamma F 
Gt >> 
La.rnbdaA 
Larr «Lt << 
OElig CE 
Omega 0 
Or 
Oslash 0 
\reffig{7-9}: Mathematics generated with SGML, DSSSL, and TEX 
7.5.5 The Jade SGML transformation interface 
Jade does not implement the DSSSL Transformation Language. However, it provides some simple, nonstandardized extensions to the DSSSL Style Language that 
allow it to be used for SGML transformations. 
These Jade extensions are available with the -t sgml and -1: xml options; the 
latter of which uses XML syntax for empty elements and processing instructions. 

 
%%page page_352                                                  <<<---3
 
332 css, DSSSL, and XSL: Doing it with style 
Following we give a list of the major SGML flow object classes that]ade defines 
to complement the standard ones of DSSSL. The extensions consist of a collection 
of flow object classes and their noninherited characteristics. They are used instead 
of the standard DSSSL-defined flow object classes. 
element A compound flow object that can have child flow objects so that both 
start and end tags are generated for element. 
empty-element An atomic flow object without child flow objects so that only a 
start is generated for empty-element. It is mostly for handling elements with a declared content of EMPTY. 
Both element and empty-element have two characteristics: 
gi String specifying the element’s generic identifier (default: the gi 
of the current node). 
attributes The element’s attributes as a list of lists, each consisting 
of exactly two strings-the first specifying the attribute name and 
the second the attribute value (default: empty list). 
processing-instruction An atomic flow object resulting in a processing instruction. Its characteristic is 
data String specifing the content of the processing instruction (default: empty string). 
document-type An atomic flow object generating a DOCTYPE declaration. Possible 
characteristics are 
name Required string giving the name of the document type that must 
be identical to the name of the document element type. 
system-id String specifying the system identifier of the document 
type (default: empty string). 
public-id String specifying the public identifier of the document 
type (default: empty string). 
entity A compound flow object storing its content in a separate entity. A possible characteristic is 
system-id System identifier of the entity. It must be a filename. 
This flow object emits no entity reference or declaration. 
entity-ref An atomic flow object creating an entity reference. It supports one 
characteristic: 
name The entity name. 
formatt ing-instruct ion An atomic flow object that inserts characters into the 
output without change. It has a single characteristic: 
data String to be inserted. The &, <, and > characters do not need to 
be escaped. 

 
%%page page_353                                                  <<<---3
 
7.5 Document Style Semantics and Specification Language 
333 
In any DSSSL specification that makes use of these flow object classes you must 
declare them with declare-flow-object-class as follows: 
(declare-flow-object-c1ass element 
"UNREGISTERED::James Clark//Flow Object Class::element") 
(declare-flow-object-class empty-element 
"UNREGISTERED::James Clark//Flow Object Class::empty-element") 
(declare-flow-object-class document-type 
"UNREGISTERED::James Clark//Flow Object Class::document-type") 
(declare-flow-object-class processing-instruction 
"UNREGISTERED::James Clark//Flow Object Class::processing-instruction") 
(declare-flow-object-class entity 
"UNREGISTERED::James Clark//Flow Object Class::entity") 
(declare-flow-object-class entity-ref 
"UNREGISTERED::James Clark//Flow Object Class::entity-ref") 
(declare-flow-object-class formatting-instruction 
“UNREGISTERED::James Clark//Flow Object Class::formatting-instruction") 
Following we show a style sheet that transforms the invitation.xml XML 
source file into an HTML file using]ade’s transformation interface: 
I E o m V m w A w N _ 
S 
w 
<!DOCTYPE style-sheet PUBLIC “-//James Clark//DTD DSSSL Style Sheet//EN"> 
<style-sheet> 
<style-specification) 
<style-specification-body> 
(declare-flow-object-class element 
“UNREGISTERED::James Clark//Flow Object Classzzelement“) 
(declare-flow-object-class empty-element 
“UNREGISTERED::James Clark//Flow Object C1ass::empty-element") 
(declare-flow-object-class document-type 
“UNREGISTERED::James Clark//Flow Object C1ass::document-type") 
(define Fontsize 12pt) 
(root 
(make simple-page-sequence 
left-margin: 25mm 
-width: 205mm 
right-margin: 25mm 
(make sequence 
font-size: Fontsize 
line-spacing: Fontsize 
(make document-type 
name: ``HTML'' 
public-id: "-//W3C//DTD HTML 3.2//EN") 
(make element gi: ``HEAD'' 
(make element gi: ``TITLE'' 
(literal "Invitation (XML to HTML transformation)")) 
(make empty-element gi: ``LINK'' 
attributes: (list (list "href“ "invit.css“) 
(list ``rel'' ``stylesheet'') 
(list ``type'' "text/css")))) 
(make element gi: ``BODY'' 
(make sequence 
(make element gi: ``H1'' 
(literal “INVITATION")) 
(process-children)))))) 
(element (front) 
(make element gi: ``TABLE'' 
attributes: (list (list ``border'' ``5'') 
(list ``frame'' “hsides") 
(list ``rules'' ``none'') 
(list ``width'' --100‘/.")) 
(process-children))) 

 
%%page page_354                                                  <<<---3
 
334 
css, DSSSL, and XSL: Doing it with style 
(element (front date) 
(make element gi: ``TR'' 
(make sequence 
(make element gi: ``TD'' 
attributes: (list (list ``class'' “front“)) 
(literal “When: “)) 
(make element gi: “TD" 
(process-children))))) 
(element (front to) 
(make element gi: ``TR'' 
(make sequence 
(make element gi: ``TD'' 
attributes: (list (list “class`` ''front“)) 
(literal “To: “)) 
(make element gi: ``TD'' 
(process-children))))) 
(element (front where) 
(make element gi: “TR" 
(make sequence 
(make element gi: “TD" 
attributes: (list (list “class" “front")) 
(literal "Venue: “)) 
(make element gi: ``TD'' 
(process-children))))) 
(element (front why) 
(make element gi: ``TR'' 
(make sequence 
(make element gi: ``TD'' 
attributes: (list (list ``class'' “front“)) 
(literal “Occasion: ")) 
(make element gi: “TD" 
(process-children))))) 
(element (body par) 
(make element gi: ``P'' 
(process-children))) 
(element emph 
(make element gi: ``EM'' 
(process-children))) 
(element (back signature) 
(make element gi: ``P'' 
attributes: (list (list “class" “signature")) 
(make sequence 
(literal “From: ") 
(process-children)))) 
</style-specification-body> 
</style-specification> 
</style-sheet> 
U\U\U\VI VIVIVIKIIVIVIVI-B-5-5-§ 
‘8°é?;'é"\°.?7'1?,°.§‘i°.$E§E'3$:n'3('3'.1'dR;'Z'B'$%Tn3l3‘.$$--c>~o§3a~v-ex-JN--c>~o<2o\Io~'3-3 
Now that we have looked at quite a few examples of DSSSL code, this DSSSL 
style sheet should be straightforward and easy to understand. Notice the “SGML" 
flow objects on lines 21 (document-type), 24, 25, 31, 33,...(element), and 27 
(empty-element); these special flow objects were declared on lines 5-10. Referring 
to the characteristics of these flow elements discussed earlier, we find on lines 22 
and 23 the name and public-id associated with the document type. For the elements (empty or not) we must specify the generic identifier (element type in XML 
terminology) gi, and we can specify attributes, if applicable. On lines 28-30 we 
see that attributes are specified as a list of lists, each consisting of a pair of strings: 
the first is the name of the attribute, and the second is its value. For instance, the 
LINK element specification on lines 27-30 results in line 4 shown in the HTML file. 

 
%%page page_355                                                  <<<---3
 
7.5 Document Style Semantics and Specification Language 
Similarly note the way we specify the attributes for the TABLE element type (lines 
39-42), which result in line 8 in the HTML file. 
The resulting HTML file invitation.html, following, is obtained typing the 
command: 
jade -dinvhtml.dsl -t xml -oinvitation.hmtl xml.dcl invitation.xml 
<!DOCTYPE HTML PUBLIC "-//W3C//DTD HTML 3.2//EN“) 
<HEAD> 
<TITLE>Invitation (XML to HTML transformation)</TITLE) 
<LINK href=“invit.css“ rel=``stylesheet'' type="text/css“> 
</HEAD) 
<BODY> 
<H1>INVITATION</H1) 
<TABLE border=``5'' frame=“hsides“ ru1es=“none" width=“100%“> 
<TR><TD class=“front">To: </TD> 
<TD>Anna, Bernard, Didier, Johanna</TD></TR) 
<TR><TD c1ass=“front">when: </TD) 
<TD>Next Friday Evening at 8 pm</TD></TR) 
<TR><TD c1ass=``front''>Venue: </TD> 
<TD>The Heb Cafe</TD></TR> 
<TR><TD class=“front“>0ccasion: </TD> 
<TD>My first XML baby</TD></TR> 
</TABLE) 
<P>I would like to invite you all to celebrate the 
birth of <EM>Invitation</EM>, my first XML document child.</P> 
<P>Please do your best to come and join me next Friday 
evening. And, do not forget to bring your friends.</P> 
<P>I <EM>really</EM> look forward to see you soonl</P> 
<P c1ass=“signature“>From: Miche1</P> 
</BODY> 
-N-.--..................._._._ 
-b\NIv--C>~Om\lo\Kn.hwr~4--o~Om\|o\\-n-bum... 
\reffig{7-10} shows the result of viewing the HTML file with an HTML browser 
that also uses our CSS style sheet invit . css to customize the formatting of the 
various HTML elements. Pay attention to the way the attributes of the TABLE elements, as specified in the XSL style sheet, generate horizontal lines over the full 
width of the screen above and below the table material. 
7.5.6 Formatting real-life documents with DSSSL 
Of course, the simple examples in this section give only a very limited idea of the 
possibilities of the DSSSL system. Therefore let us show you a more complex case 
before saying something about real-life solutions based on DSSSL. 
Suppose we are working in a multilingual environment, and, as with the babel 
package of ETEX, we would like certain strings to come out correctly in each of the 
languages we are interested in. Let us consider the following code fragment: 
(define (FIGNAME) 
(case (inherited-attribute-string “xml:lang") 
((``de'') ``Abbildung '') 
((``n1'') "Figuur ~-) 
(else "Figure “))) 
(define (TABNAME) 
(case (inherited-attribute-string “xml:lang“) 
((``de'') ``Tabelle '') 
a>\lO\vw-b\.4--335 

 
%%page page_356                                                  <<<---3
 
336 css, DSSSL, and XSL: Doing it with style 
Q:\michal\lwcwork\i_nv1.h1m 
INVITATION 
To: Anna, Bernard, Didier, Johanna 
When: Next Friday Evening at 8 pm 
Venue: The Web Cafe 
Occasion: My first XML baby 
I would like to invite you all to celebrate the birth of 
Imitation, my first )0/IL document child 
Please do your best to come and join me next Friday 
evening. And, do not forget to bring your fiiends. 
I really look forward to see you soon! 
From: hlichel 
\reffig{7-10}: XML to HTML transformation with DSSSL 
((``nl'') ``Tabel '') 
(else ``Table ''))) 
(element (caption) 
(make paragraph 
space-before: Gpt 
space-after: 10pt 
(make sequence 
font-weight: ’bold 
(literal (if (have-ancestor? ``figure'') (FIGNAME) (TABNAME))) 
(literal (format-number 
(if (have-ancestor? ``figure'') 
(element-number (ancestor ``figure'')) 
(element-number (ancestor ``table''))) 
ll1ll)) 
(literal ". ")) 
(process-children-trim))) 
--~___,_.._.._.__._._ 
.;;\~m-o~oon\los\-n.;.\.\~n\a.-o~o 
In Section 6.5.1 we introduced the attribute xmlzlang, which specifies the language of the content of a given element type. This attribute value is inherited by 
all children elements. So, if we have part of a document for which this attribute is 

 
%%page page_357                                                  <<<---3
 
7.6 Extensible Stylesheet Language 
337 
set to a given value, the case statements on lines 2 and 7 will activate one of the 
lines in the range 3-5 and another line in the range 8-10, depending on the result of 
the evaluation of the procedure “ (inherited-attribute-string "xm1:1ang")". 
For instance, with a value of ml 2 lang equal to “de," for German, FIGNAME would 
get the value “Abbildung" (line 3), and TABNAME would get the value “Tabelle" (line 
8). With these constant definitions we can construct the caption element using the 
appropriate character strings for the language considered. On line 17 the if statement verifies whether the ancestor of the caption element is figure, in which 
case the literal string FIGNAME will be placed into the paragraph flow object at that 
point, otherwise TABNAME. Similarly, lines 18-24 take care of putting the right figure or table number following the text string chosen earlier. In fact, to get the number, DSSSL counts the number of figure or table elements (the “if" test and its 
branches on lines 19-21). This number is then formatted using the f ormat-number 
procedure (line 18) and represented as a “decimal" number (hence the ``1'' on line 
23 specifies the format to be used for the number),7 followed by a dot (line 24). 
This example shows clearly the kind of manipulations that are possible with 
DSSSL using the many procedures and flow objects defined in the DSSSL standard. 
Another useful feature is the use of “modes," which allows certain elements to be 
processed (and output) more than once in different modes. For instance, heading 
titles or figure and table captions, can be output once when one is composing the 
main text, and a second time when contructing the table of contents and lists of 
tables and figures. 
If you want to find out how DSSSL is used in real-life applications dealing with 
large documents, you should take a look at Norman Walsh’s modular DocBook 
style sheets [G>DBDSSSL] for formatting SGML documents marked up using the 
DocBook DTD. Walsh provides two style sheets: a generic one for printing DocBook documents using RTF, TEX, or MIF, and another one for transforming them 
into HTML. Several hooks are provided to allow the user to customize the output. 
Apart from their practical usefulness, these complex style sheets are also a good 
place to learn about DSSSL. 
7.6 Extensible Stylesheet Language 
Work on the “Extensible Stylesheet Language" (XSL) is ongoing, and the syntax 
that we will present in this section is based on the specification available at the time 
of writing. It is thus possible that the final recommendation, which is expected by 
summer 1999, will differ in some details from what we describe here. Nevertheless, 
the basic principles of the XSL language, and the way it can be used with XML 
documents, will remain mostly unchanged. 
7Section 8 of the DSSSL Standard describes the expression language. It is here that you will find an 
explanation of the various procedures we have used in this and other examples. In particular, Section 
8.5.7.24 details the f ormat-number procedure. 

 
%%page page_358                                                  <<<---3
 
338 
CSS, DSSSL, and XSL: Doing it with style 
XSL is a language for expressing style sheets that describe rules for the presentation of XML elements. It proposes a syntax for describing two subprocesses: 
a tramfimnation of the source tree of the XML document into a result tree, and 
2. an interpretation of the result tree to produce fiormatting objects for output on 
various media, such as a computer screen, paper, or audio. 
An XSL style sheet consists of a set of template rules, with each rule having two 
parts: a pattern that is matched against elements in the source tree, and a template, 
which is instantiated to create part of the result tree. 
Source and result trees are separate objects, which can have quite different 
structures, since when constructing the result tree, information from the XML 
source can be reordered, or filtered or arbitrary branches can be added by the XSL 
application. XSL can be used for general XML transformations, for instance, to 
tranform XML into “well-formed" HTML. 
One of the main aims of XSL is to be able to use the result tree to format 
the XML information using the vocabulary defined in the XSL specification. The 
objects of this formatting vocabulary and their characteristics are very much influenced by the CSS and DSSSL languages, since one of XSL’s design aims is to provide 
a formatting functionality of at least CSS and DSSSL. 
7.6.1 The general structure of an XSL style sheet 
An XSL style sheet uses XML syntax. It contains one xsl : stylesheet document 
element, which contains zero or more xsl ztemplate elements specifying template 
rules. 
The following is an example of a simple XSL style sheet that constructs a result 
tree for a sequence of par elements containing emph elements using XSL’s formatting object vocabulary: 
<?xml version=’1.0’?> 
<xsl:stylesheet xmlns:xsl="http://www.w3.org/TR/WD-xsl" 
xmlns:fo="http://www.w3.org/TR/WD-xsl/F0" 
result-ns=``fo''> 
<xs1:temp1ate match="/"> 
<fo:basic-page-sequence font-fami1y=“He1vetica" font-size=“10pt" > 
<xs1:app1y-templates/> 
</fo:basic-page-sequence> 
</xsl:temp1ate> 
<xs1:template match=``par''> 
<fo:block indent-start-``10pt'' space-before=``12pt''> 
<xs1:apply-templates/> 
</fo:b1ock> 
</xs1:template> 
<xs1:template match=``emph''> 
<fo:inline-sequence font'style=``italic''> 
<xsl:app1y-templates/> 
</fo:inline-sequence> 
</xsl:template> 
</xsl:stylesheet> 
~._-._._._._._._..-._._ 
0~Om\lOs\-It-§vlI\as-0~Om~lOs‘~II-P'\4t\a%%page page_359                                                  <<<---3
 
7.6 Extensible Stylesheet Language 
Lines 2 and 3 define the xsl and fo (flow object) namespaces (see Section B.3 
for details). The string following xmlns: declares a shorthand for a namespace to 
allow the parser to interpret the elements in the document instance, a URI-like 
string indicating where those elements are defined. Thus line 2, which must always 
be present in the case of an XSL style sheet, specifies in which document the XSL 
syntax is defined, while line 3 defines the formatting object syntax. In fact, we target 
the XSL formatting objects namespace, whose elements are defined in a subsection 
of the XSL document, as can be seen from the URI. However, we could also use 
CSS formatting objects, and a proposal exists as a W3C note [G>XSLCSS]. In that 
case we would replace line 3 with the following namespace definition: 
xmlns:css="http://www.w3.org/TR/NOTE-XSL-and-CSS" 
Line 4 (result-ns) specifies the namespace for the result tree. In our example it 
is “fo," indicating that we express the output in terms of the formatting object 
vocabulary, defined on line 3. If we planned to use CSS flow objects instead, line 4 
would be replaced by: 
result-ns=``css'' 
Then we write three template rules. The first rule (lines 5-9) is for the root 
node, and it specifies that the document as a whole should be formatted as a “page 
sequence" formatting object, set in a 10 pt Helvetica typeface. The second rule 
dines 10-14) declares that each par element should result in a “block" formatting 
object, which is separated from the previous block by twelve points and whose first 
text line is indented by ten points. Finally the third rule (lines 15-19) declares that 
an emph element corresponds to a sequence formatting object, and that its contents 
are typeset in an italic typeface. 
A general XSL style sheet can contain zero or more instances of each of the 
nine elements (six are empty, lines 3-8, and three can have a content, lines 9-11) 
following. The ellipses ( . . .) indicate where additional content is possible. 
1 <?xml version="1.0"?> 
2 <xsl:stylesheet xmlns:xsl="http://www.w3.org/TR/WD-xsl"> 
3 <xsl:import href="..."/> 
4 <xsl:include href="..."/> 
5 <xsl:id attribute="..."/> 
5 <xsl:strip-space element="..."/> 
7 <xsl:preserve-space element="..."/> 
8 
9 
<xsl:macro name="..."> ... </xsl:macro> 
<xsl:attribute-set name="..."> ... </xslzattribute-set> 
10 <xsl:constant name="..." value="..."/> 
11 <xsl:template match="..."> ... </xsl:template> 
12 </xsl:stylesheet> 
339 

%==========360==========<<<---2
 
%%page page_360                                                  <<<---3
 
340 
css, DSSSL, and XSL: Doing it with style 
Line 2 declares the xsl namespace. On line 11 we recognize the template rule 
element xslztemplate, the workhorse of XSL style specifications. The order in 
which the nine possible children elements of the xsl : stylesheet element occur 
is not significant, except that xsl : import elements must always be specified first, 
at the beginning of the style sheet. In the following sections we explain the use of 
these elements in more detail as we need them. 
7.6.2 Building the source tree 
XSL operates on an XML document as a source tree. Documents with the same 
source tree will be processed identically by XSL. The XML tree data model allows 
for six kinds of nodes: 
Root node The root of the tree. It cannot occur anywhere else in the tree. It has 
a single child, namely the document element node. 
Element nodes Such a node exists for every element of the source document. Its 
children are the element nodes and characters of its content. All entity references are expanded, and character references are resolved. An order can be 
assigned to the nodes. In particular, the document order is identical to the order 
of the element start tags in the source document. It is possible to associate an 
identifier to an element by using a unique identifier that is an attribute declared 
as type ID in the DTD. Moreover, in the absence of a DTD you can specify in 
the XSL style sheet, using an <xs1 : id . . . /> tag, which attribute should be 
treated as the identifier of type ID 
Attribute nodes Each element node has an associated set of attribute nodes. Defaulted attributes are treated the same as specified attributes, while unspecified 
attributes declared as #IM1'-‘LIED in the DTD do not get an attribute node. 
Namespace Each element has an associated set of namespace nodes, one for each 
namespace prefix that is in scope for the element and one for the default namespace. 
Processing instruction nodes There exists such a node for each processing instruction in the source. The name of the processing instruction is its target. 
Comment nodes Every comment in the source generates a corresponding comment node. 
Before handing the tree to XSL for further handling, some whitespace is 
stripped, both from the source document and from the style sheet. The xml : space 
elements can be used to control this process, since they are retained in the source 
tree. Inside a style sheet, the xslztext element preserves space. Moreover, the 
xs1:strip-space and xsl zpreserve-space elements allow you also to control 
the way whitespace is stripped from source elements. 

 
%%page page_361                                                  <<<---3
 
7.6 Extensible Stylesheet Language 
7.6.3 Template rules 
The basic XSL building block is the template rule that describes how a given XML 
source element node is transformed into an XSL element node for further treatment. Templates are specified with the xsl ztemplate element, which has two basic 
parts: 
a a mat ch attribute, identifying the XML source node(s) to which the rule applies; 
and 
o the content of the xsl ztemplate element, which provides the template to generate the formatting object. An action or formatting, styling, and processing 
part details the transformation and styling of the resulting node. 
Let us reproduce here lines 15-19 of the first example in Section 7.6.1: 
1 <xs1:template match=``emph''> <!-- match pattern --> 
2 <fo:inline-sequence font-style=``ita1ic''> <!-- action template + --> 
3 <xsl : apply-templates/> < ! -' I --> 
4 </fo:sequence> <!-- action template + --> 
5 
</xs1:temp1ate> 
This template will match all elements of type empb (line 1). For each occurrence 
of an emph element in the source tree, a f o : inline-sequence formatting object 
(lines 2-4) should be added to the result tree. The xsl : apply-templates element 
(line 3) will recursively process the children of the source element in question. 
Only one template r11le can apply for each node in the source tree. To guarantee this, a detailed conflict-resolving mechanism exists, which is described in Section 7.6.5.6. 
VVhen no successful pattern matches exist for a r11le in the style sheet, a defimlt 
built-in template rule is implicitly applied for the root and all element nodes. It 
corresponds to the following definition: 
1 <xs1:template match="*|/"> 
2 <xsl:apply-templates/> 
3 </xsl:template> 
This default r11le stipulates that processing should continue with the child elements 
(xsl : apply-t emplates element on line 2). Although this template (line 1) matches 
the root and any other element, it has a smaller priority than any other template 
rule in a style sheet. This mechanism provides you with a convenient way to specify 
your own default rule by substituting your XSL commands for those on line 2. In 
this way your rule will override the default built-in behavior. 
Similarly there exists a b11ilt-in template r11le that copies text nodes through to 
the output tree. It has the form: 
1 <xs1:template match="text()"> 
2 <xs1:value-of se1ect="."/> 
3 </xsl:template> 
341 

 
%%page page_362                                                  <<<---3
 
342 
CSS, DSSSL, and XSL: Doing it with style 
This rule does not apply to processing instructions and comments. They have to be 
matched by an explicit rule, otherwise nothing is created. 
As an example, consider the following minimal “empty" (and trivial) style sheet, 
which declares only the xsl namespace. 
1 <xs1:stylesheet xm1ns:xs1="http: //www.w3 . org/TB./WD-xsl"> 
2 </xs1:stylesheet> 
This style sheet, which we call empty.xs1, will (implicitly) apply the built-in default template rule and process all elements of the document. This is similar to the 
DSSSL file empty. dsl that we introduced in Section 7.5.3.1. 
7.6.4 XSL processors 
Today, not many processors that can handle XSL style files exist. To run our examples we have chosen James Clark’s xt processor [QXTPROC], a Java implementation of the tree construction part of XSL. Yet, for the sake of completeness, at the 
end of this section we will look briefly at other possible choices. 
7.6.4.1 Introducing the x1: processor 
The xt program uses Clark’s xp XML parser [QXPPARS], which is also written in 
Java. It is thus sufficient, if you have Java installed on your machine, to download 
the xp and x1: zip archives from Clark’s Web site. They contain the Java archives 
sax . jar, xp. jar, and x1: . jar. Add them to your Java class path, and off you go. 
On Windows the commands could be something like (depending on which version 
of Java you have available) the following: 
1 set classpath=d:\xml\xt.jar;d:\xml\xp.jar;d:\xml\sax.jar;d:\jdk1.1.6\src; 
2 %JAVA_HOMEZ\bin\java com.jc1ark.xsl.sax.Driver X1 Z2 Z3 
where the JAVA_HOME environment variable (line 2) is the directory where yourjava 
installation lives. Similarly on UNIX (Bourne shell), you could write the following: 
DIB.=/afs/cern.ch/asis/src/archive/java 
cLAssPATH=$DIB/xc.jar:$D1R/xp.jar:$DIR/sax.jar:$cLAssPATH 
export CLASSPATH 
java com.jc1ark.xsl.sax.Dz-iver $1 $2 $3 
.pw~... 
The variable DIR (line 1) is the directory where you keep your Java archives, while 
line 4 assumes that the java program is in your search path for executables. 
If you save these lines in a command script xt .bat (Windows) or xt (UNIX), 
you can execute xt by typing: 
xt XML-source-file XSL-style-sheet Dutput-file 

 
%%page page_363                                                  <<<---3
 
7.6 Extensible Stylesheet Language 
343 
If the third argument is not specified, the output is written to the “standard output" 
(e.g., the computer screen). We can use the style sheet empty.xsl and the XML 
example invitation2 . ml of Section 7.4.5 with xt by entering the command: 
xt invitation2.xml empty.xsl 
We get the following output: 
I would like to invite you all to celebrate 
the birth of Invitation, my 
first XML document child. 
Please do your best to come and join me next Friday 
evening. And, do not forget to bring your friends. 
an\::.7\‘-Ix-b\»avu-I really look forward to see you soon! 
As expected, we see the contents of the text nodes for all elements of the document. 
The attribute values (of the invitation element) are not copied to the output. The 
copying-through feature of text nodes can assist you in developing an XSL style 
sheet step-by-step. Indeed, for a complex document this allows you to construct 
and fine tune the rules gradually, since the content of all nonspecified elements are 
copied to the output. It thus provides a convenient context for development and 
debugging. 
7.6.4.2 Other XSL applications 
We would like to mention two other XSL tools: the Koala XSL engine for Java 
[9 KOALAXSL] and f op [=->FOP], a program that converts XSL formatting objects 
to PDF. 
7.6.4.3 The Koala XSL engine 
The Koala XSL engine is a processor written in Java, using the SAX and DOM 
models. The package also contains xslSlideMaker, a postprocessor that handles 
slides prepared with XML and XSL and produces HTML output. 
The (UNIX) setup script koalaxsl takes the form: 
D=/afs/cern.ch/asis/src/archive/java/Xsl 
CLASSPATH=$D/domcore.jar:$D/sax.jar:$D/parser.zip:$D/xsl.jar:$CLASSPATH 
export CLASSPATH 
java fr.dyade.koala.xml.xsl.Main $* 
¢.\.a~... 
On line 2 we see the presence of four Java archives that are distributed with the 
package. To run the script, you can type: 
koalaxsl -r zsl-style-sheet zml-source~fiZe 

 
%%page page_364                                                  <<<---3
 
344 
CSS, DSSSL, and XSL: Doing it with style 
where koalaxsl is the name we gave to the script shown earlier. The result, which 
corresponds to the rules specified in the style sheet following the -r switch, is written to standard output. The xs1SlideMaker postprocessor is based on this parser. 
See the Koala Web page for more details [L-> KOALAXSL]. 
7.6.4.4 Formatting objects to PDF 
The f op system consists of a set of Java classes that, via SAX, reads an XML document representing formatting objects and turns it into PDF. The author James 
Tauber is working actively to let his tool cover most of the characteristics of XSL’s 
formatting objects. We discuss this tool in Section 7.6.10. 
7.6.5 Patterns 
A pattern is a string that selects a set of nodes (zero or more) in a source document. 
The selection is relative to the current node. The simplest pattern is the name of 
an element type; it selects all child elements of the current node with the given 
name. For example, the pattern par selects all the par(agraph) child elements of the 
current node. 
7.6.5.1 Element patterns 
A node is said to match a pattern if the node can be selected by the pattern. For 
instance, the pattern par matches any par element because if the current node was 
the parent of the par element, the par element would be one of the nodes selected 
by the pattern in question (even with par as document element, since the root is 
the parent of the document element). 
The union operator “ I " allows you to propose alternatives; for example, a I b I c 
matches a, b, and c elements individually. 
Parent-child relations are expressed with the help of the “path" notation using the “/" operator. As an example, the pattern par/emph first selects par child 
elements of the current node, and then for each such element it selects its emph 
children. 
The path operator / has a higher precedence than the union operator I , so that 
a/clb/c matches c elements that have either an a or a b element as parent. For 
reasons of readability you can leave whitespace around operators inside patterns 
so that the above pattern can also be expressed as a/ c I b/c, making its meaning 
somewhat clearer. 
More general ancestor-descendant relationships are expressed with the // operator that allows zero or more generations between the element at the left and 
right side of the operator. Hence body//emph selects all emph descendants of the 
body children of the current node. 
The “wildcard" character “*" can be used to represent any single element type, 
so that the pattern * selects all children of the current node, and */emph selects 

 
%%page page_365                                                  <<<---3
 
7.6 Extensible Stylesheet Language 
345 
all emph grandchildren of the current node. On the other hand, the pattern par/* 
matches any element with a par element as parent. 
The pattern “ . " selects the current node. This can be used, for instance, to 
represent the current node explicitly in ancestor-descendant relations, as in par/ / . , 
a pattern that selects all par ancestors of the current node, or . //emph that selects 
all its emph descendants. In a similar way, the pattern “ . . " selects the parent of the 
current node, so that . . / par selects par sibling elements of the current node. 
7.6.5.2 Patterns for other types 
The other node types are treated in a way similar to elements. 
In particular, attributes of an element are handled as child elements, but syntactically their name is prefixed with the “Q" character. For instance, @to selects 
the to attribute of the current element, whereas student/©name selects the name 
attribute of each student child element of the current node. The wildcard pattern 
©* selects all attributes of the current node. 
Similarly, the pattern comment () matches any comment node; text () matches 
any text node. Processing instructions use the pattern pi () where the argument 
allows you to specify the target, as in pi ( ``1atex'' ) , which matches any processing 
instruction with lat ex as target. 
7.6.5.3 Tests in patterns 
To qualify further the set of nodes selected by a pattern, you can use a test following 
the pattern by using square brackets as delimiters [J . Each node in the node set will 
be tested in turn, and the result node set will include only nodes that satisfy the test 
condition. Following we list what can be used inside tests: 
Patterns The test will be true if one or more nodes starting from the current node 
match the test pattern. For instance, par [emph] selects par children elements of the current node that have at least one emph child element, while 
student [©middle-name] selects student elements with a midd1e-name 
attribute. 
String You can compare a pattern or subpattern to a string. For instance, 
student [©midd1e-name=``Paul''] will match student child nodes of the 
current node that have an attribute middle-name equal to the string 
“Paul", while student [origin="BE“] selects student children with an 
origin child containing the text string “BE". 
Another example relevant to the file invitation. ml is 
<xsl:template match=’par [emph=``Invitation''] ’> 
This selects a par element with an emph child whose contents is the string 
“Invitation", that is, the first par in our XML source file. 

 
%%page page_366                                                  <<<---3
 
346 css, DSSSL, and XSL: Doing it with style 
Position You can impose positional constraints on a node relative to its siblings. 
In the following, the test at the left is successful if the node being tested 
fulfills the condition at the right: 
f irst-of -any () node is first element child. 
1ast-of-any () node is last element child. 
f irst-of -type () node is first element child of its element type. 
1ast-of-type () node is last element child of its element type. 
Booleans You can use the not() Boolean function or the and and or Boolean 
operators. An example is the pattern student [not (©midd1e-name)] that 
selects student child elements without midd1e-name attribute. Similarly 
the pattern p/emph [f irst-of-type () and 1ast-of -type ( )] matches 
an emph element when it is the only emph child of a p parent. 
The [] operator has a higher precedence than the union operator I, so that 
alb [@x] matches either a elements or b elements with an x attribute. For implementation reasons inside a [] , match pattern, /, / /, and E] are not allowed. 
7.6.5.4 Root and ancestor nodes 
VVhen a pattern starts with the path separator /, it represents the root node, which 
is interpreted in a special way. Thus a pattern that is just / matches the root node. 
The pattern / invitat ion matches the document element (if it corresponds to an 
invitation element), while /* always matches the document element, whatever it 
is called. More generally, the symbol // allows one to select descendants of the root 
node in a straightforward way. For instance, / / par selects any par descendants of 
the root node, meaning, in fact, that any par element will match that pattern. For 
patterns starting with / or // the current node is irrelevant. 
The first ancestor of the current node that matches a pattern can be selected 
using the syntax ancestor( . . .), where the argument contains a match pattern. 
selected. For instance, ancestor(article) / author selects all author children of 
the first ancestor of the current node that is an article. More generally, 
<xsl ztemplate match="footnote [ancestor(footnote)] "> 
matches footnote elements with a footnote ancestor, that is, we are dealing with 
nested footnotes elements. As this is forbidden, for instance, in ETEX, you can use 
such a pattern to warn about the presence of constructs that are otherwise allowed 
by the DTD (see Section B.4.5.1 for a discussion of this point). 
7.6.5.5 Selection by identifier 
To select individual nodes by identifier, the function id, whose argument can contain a blank-separated list of the identifier strings to be used in the match, is pro
 
%%page page_367                                                  <<<---3
 
7.6 Extensible Stylesheet Language 
vided. As an example, id( ’miche1g’) will select the element with ID michelg, 
while id( ’michelg sebastianr ’) would select elements with an ID equal to 
mi chelg or sebastianr. VVhen no element is found, the empty set is returned. The 
argument can also be a pattern rather than a literal string. In that case for each node 
selected by the pattern, the value of the node is treated as a whitespace-separated 
list of ID references. For example, if the current node is an element with an IDREF 
or IDREFS attribute (see page 261) named myref, then the pattern id(©myref ) will 
effectively dereference the myref attribute, get its value, and select the element(s) 
that it references. 
7.6.5.6 Resolving match conflicts 
VVhen several template rules apply to the same source tree element, the following 
strategy is applied: 
0 First, all matching template rules that are less important than the most important matching template rule or rules are eliminated from consideration. In particular, the default template (see Section 7.6.3) is less important than explicitly 
specified ones. Also rules and definitions in the importing style sheet are more 
important than those in imported style sheets, although rules and definitions 
in any given imported style sheet are considered more important than those 
present in previously imported style sheets. 
0 You can also specify an explicit priority attribute on rules (a positive or negative real number) to distinguish among Various matching rules. In this case the 
matching rule with the highest priority is selected. 
VVhen at the end still more than one matching template rule is selected, an 
error condition results. In that case the XSL processor signals the error, or it must 
use the matching template rule that occurs last in the style sheet. 
In practice such ambiguities should, of course, be minimized to ease the understanding, documentation, and maintainability of style sheets. 
7.6.5.7 A first complete example 
Let us consider the following XML document sectionexa.xm1, consisting of an 
article with a title, two authors, an abstract, two sections, each with a section title, 
and a few paragraphs: 
1 <article> 
2 <title>This is the article’s title</title) 
3 <author>Miche1 Goossens</author) 
4 <author>Sebastian Rahtz</author) 
5 <abstract>A <emph>short</emph> description of the contents</abstract> 
5 <section sectid=``S1''> 
7 <stitle>I-‘irst section title</stitle> 
8 <par ident=``first''>The first paragraph for this section.</par> 
9 <par ident=``norma1''>A normal paragraph with <emph>emphasised</emph> text.</par) 
10 <par>Here we have <emph>no</emph> attribute.</par) 
347 

 
%%page page_368                                                  <<<---3
 
348 
css, DSSSL, and XSL: Doing it with style 
(par ident="1ast“>This is the end of the section.</par) 
</section> 
(section sectid=``S2''> 
<stitle>Second section title</stitle> 
<par ident=``first''>The first paragraph for this section.</par) 
<par>Here we <emph>a1so</emph> have <emph>no</emph> attribute.</par) 
(par ident=``norma1'' id=``specia1''>A normal paragraph with 
<emph>emphasised</emph> text.</par> 
<par>Another attribute-less paragraph.</par) 
<par ident=``1ast''>This is the end of the section.</par> 
</section) 
</article) 
We want to address (match) the various elements of this document using 
the patterns introduced in this section. Therefore we construct a style sheet 
sectionexa.xs1. It contains an ad hoc set of templates, showing a few of the pattern rules in action. Note in particular how we must assign a priority value to 
some of the templates to ensure that only a single rule fires for each element in the 
XML source file. 
3 0 m Q o m A w ~ <?xm1 version=’1.0’?> 
<xs1:stylesheet xm1ns:xs1="http://www.w3.org/TR/ND-xsl" resu1t-ns=""> 
<xs1:temp1ate match="/"> 
<xs1:text>(*root*)</xsl:text> 
<xs1:apply-templates/> 
<xs1:text>(/*root*)</xsl:text> 
</xs1:temp1ate> 
<xs1:temp1ate match="*" priority=``-1''> 
<xs1:text>(*)</xs1:text> 
<xs1:app1y-temp1ates/> 
<xs1:text>(/*)</xs1:text> 
</xs1:temp1ate> 
<xs1:temp1ate match=``par''> 
<xs1:text>(T1)</xs1:text> 
<xs1:app1y-templates/> 
<xs1:text>(/T1)</xs1:text> 
</xs1:temp1ate> 
<xs1:temp1ate match="par[@ident]" priority=``1''> 
<xs1:text>(T2)</xs1:text> 
<xs1:apply-temp1ates/> 
<xs1:text>(/T2)</xsl:text> 
</xs1:temp1ate> 
<xs1:temp1ate match="par[@ident=’first’]" priority="2“> 
<xs1:text>(T3)</xsl:text> 
<xs1:app1y-temp1ates/> 
<xs1:text>(/T3)</xs1:text> 
</xs1:temp1ate> 
<xs1:temp1ate match="section[@sectid=’S2’] 
/par[@ident=’norma1’ and @id=’specia1’]"> 
<xs1:text>(T4)</xsl:text> 
<xs1:app1y-temp1ates/> 
<xs1:text>(/T4)</xs1:text> 
</xs1:temp1ate> 
<xs1:temp1ate match="section[1ast-of-type()]"> 
<xs1:text>(P1)</xs1:text> 
<xs1:app1y-temp1ates/> 
<xs1:text>(/P1)</xs1:text> 
</xs1:temp1ate> 
<xs1:temp1ate match="section[not(first-of-type())]/par[first-of-type()]"> 
<xs1:text>(P2)</xsl:text> 
<xs1:app1y-templates/> 
<xs1:text>(/P2)</xs1:text> 

 
%%page page_369                                                  <<<---3
 
7.6 Extensible Stylesheet Language 349 
S & S 3 S i S S S % 3 $ 3 K I K 3 3 3 S $ $ $ & $ 3 $ 
</xs1:temp1ate> 
<xs1:temp1ate match=``author''> 
(xsl:text>(A1)</xs1:text> 
<xs1:app1y-templates/> 
<xs1:text>(/A1)</xs1:text> 
</xs1:temp1ate> 
<xs1:temp1ate match="author[last-of-type()]" priority=``1''> 
(xsl:text>(A2)</xs1:text> 
<xs1:app1y-templates/> 
<xs1:text>(/A2)</xs1:text> 
</xs1:temp1ate> 
<xs1:temp1ate match="*[first-of-type() and last-of-type()]"> 
<xs1:text>(ND)</xe1:text> 
<xs1:app1y-templates/> 
<xs1:text>(/WD)</xs1:text> 
</xs1:temp1ate> 
<xs1:temp1ate match="emph[first-of-type() and 1ast-of-type()]" priority=``1''> 
<xs1:text>(E1)</xs1:text> 
<xs1:app1y-templates/> 
<xs1:text>(/E1)</xs1:text> 
</xs1:temp1ate> 
<xs1:temp1ate match="emphCnot (first-of-type() and last-of-type())]"> 
<xs1:text>(E2)</xsl:text> 
<xs1:app1y-templates/> 
<xs1=text>(/E2)</xs1:text> 
</xs1:temp1ate> 
</xs1:stylesheet> 
The XML source sectionexa. xml and the XSL style sheet sectionexa. xsl 
are processed with the X1; program by typing the following command: 
xt sectionexa.xm1 sectionexa.xs1 
This generates the following output: 
1 
2 
3 
4 
5 
6 
7 
8 
9 
0 
l 
12 
13 
I4 
15 
16 
17 
I8 
I9 
20 
2] 
22 
(*root*)(HD) 
(HD)This is the article’s title(/ND) 
(A1)Miche1 Goossens(/A1) 
(A2)Sebastian Rahtz(/A2) 
(HD)A (E1)short(/E1) description of the contents(/ND) 
(*) 
(ND)First section title(/ND) 
(T3)The first paragraph for this section.(/T3) 
(T2)A normal paragraph with (E1)emphasised(/E1) text.(/T2) 
(T1)Here we have (E1)no(/E1) attribute.(/T1) 
(T2)'1'his is the end of the section.(/T2) 
(/*) 
(P1) 
(WD)Second section title(/VD) 
(T3)The first paragraph for this section.(/T3) 
(T1)Here we (E2)a1so(/E2) have (E2)no(/E2) attribute.(/T1) 
(T2)A normal paragraph with 
(E1)emphasised(/E1) text.(/T2) 
(T1)Another attribute-less paragraph.(/T1) 
(T2)This is the end of the section.(/T2) 
(/P1) 
(/HD)(/*root*) 
Let us review these lines and see how they relate to the way the style sheet 
matches the lines in the XML source file. To clearly indicate which rule is applied, 
we added tags in front of and following each match pattern by using xslztext 

%==========370==========<<<---2
 
%%page page_370                                                  <<<---3
 
350 
CSS, DSSSL, and XSL: Doing it with style 
elements. This makes it easy to find out which rule matches various parts of the 
XML file. 
Lines 3-7 of the style sheet define the template for the root element. We observe, indeed, that it encloses the entire document, inclusive of the document element article. The default pattern for element nodes (lines 8-12 in the XSL file) 
applies to those nodes that are not matched by any other rules in the style sheet. 
This is the case only for the section element on line 6 of the XML file. Now we 
consider lines 13 to 33 in the XSL style sheet. The four match patterns use elements, children, or attribute qualifiers. All par elements (lines 8-11 and 15-20 in 
the XML source) match line 13 (T1), while the refinement of line 18 (T2) matches 
par elements that also have an ident attribute (lines 8, 9, 11, 15, 17, and 20 in 
the XML source). On line 23 we add one more constraint (T3) that matches par 
elements with their ident attribute equal to “first" (lines 8 and 15 in the XML 
source). The more complex pattern on lines 28-29 in the XSL style sheet (T4) 
matches a par element whose ident attribute has the value “normal," its id attribute has the value “special," and whose parent is a section element that has 
an attribute sectid equal to “S2" (only line 17 in the XML source matches all of 
these conditions). To disambiguate the par rules that would match more than one 
element, we had to add a priority attribute on pattern lines 18 and 2 3. 
The other patterns use positional qualifiers. An interesting case is the match 
pattern on line 54 (WD), where we combine a wildcard with the positional operators first-of-type and last-of-type. Because we combine them with the and 
Boolean operator, we, in fact, match elements that occur only once (since they are 
both first and last instances of their type) and that are not matched by a pattern with 
a higher priority (as on line 5 9, see next paragraph). Referring to the XML source, 
we see this is the case for the article element on line 1, the title element on line 
2, and the abstract element on line 5, as well as for the stitle elements on lines 
7 and 14, since they are unique inside their respective section elements. 
The remaining patterns in the XSL style sheet are straightforward and easy to 
understand. We merely comment on the match patterns on lines 59 (E1) and 64 
(E2). The first pattern (E1) looks for an emph element without siblings of the same 
type (to disambiguate with respect to the more general rule on line 54, we need the 
priority attribute) as on lines 9, 10, and 17-18 of the XML source, which contain 
only one emph element inside the par element. The second pattern (E2) looks for 
emph elements that have at least one sibling of the same type, as on line 16 of the 
XML source, which has two emph elements inside the par element. 
This example should have made it clear that XSL is a quite complete pattern 
language to select element nodes in an XML document tree. 
7.6.6 Templates 
Once a rule fires for a given element, the rule’s template is instantiated. A template 
can add literal result elements, character data (text), and instructions for creating 
fragments of the result tree (copying, sorting, numbering, or executing macros). 

 
%%page page_371                                                  <<<---3
 
7.6 
Extensible Stylesheet Language 
A template can output a set of formatting objects. Alternatively, as we shall 
show later, you can write HTML or even ETEX code directly. 
It is not our intention to describe them in any detail. We merely give a list of 
the instructions and then present an overview of the formatting objects in the next 
section. This should be enough to understand the examples later in this chapter. It 
should also be clear that some of the instruction names can still change in the final 
recommendation. 
Possible constructs that can occur inside templates as defined in the XSL style 
sheet DTD follow: 
xsl: 
xsl: 
xsl: 
xsl: 
xsl: 
xsl: 
xsl: 
xsl: 
apply-templates Processes child nodes, including text nodes. A select 
attribute lets you choose which nodes to process (see Section 7.6.5 for an 
overview of patterns that can be used). 
attribute Adds an attribute node specified by the name attribute to the containing result element node. The value of the created attribute is given by the 
content of the xsl : attribute element. 
attribute-sets Assigns a name to a set of attributes that can be referenced 
in an xsl zuse element. 
choose Allows selection of a node among several alternatives. It consists of a 
series of xsl zwhen elements and zero or one xslzotherwise element. Each 
xsl zwhen element has a test attribute specifying a select pattern. The content 
of the first (only!) xsl zwhen element whose test is true will be instantiated. 
If none of the xslzwhen elements selects a node, then the template of the 
xsl : otherwise element, if present, is instantiated, or else in the absence of 
an xsl : otherwise element, nothing is created. 
xsl : comment Creates a comment node in the result tree. The value of the comment is the content of the xsl : comment element. 
xslzconstant Creates a global string constant. The name attribute allows you 
to assign a name to identify the constant, while its value is specified with the 
value attribute. 
copy Copies the current node. The xsl : copy element will be replaced by a 
copy of the current node, including its namespace nodes, but the children and 
attribute nodes are not automatically copied. 
contents Used in macro processing to include the result tree fragment selected by the match pattern. 
element Creates an element with a computed name that is set equal to the 
value of the name attribute. Attributes and children of the created element are 
specified as the contents of the xsl : element element. 
f or-each VVhen the result document is known to have a regular structure, you can specify directly the template for selected elements by using the 
xslzfor-each element. It has a select attribute whose pattern selects element nodes for which the specified template will be instantiated. 
351 

 
%%page page_372                                                  <<<---3
 
352 
CSS, DSSSL, and XSL: Doing it with style 
xsl: 
xsl: 
xsl: 
xsl: 
xsl 
xsl 
xsl 
xsl: 
xsl 
xsl 
if Provides a simple if-then functionality. The attribute test specifies a select pattern. If one or more nodes are selected, then the content of the xsl : if 
element is instantiated, otherwise nothing is created. 
import Style sheets can be imported. Rules and definitions in the importing style sheet are considered more impammt than rules and definitions in any 
imported style sheets, while rules and definitions in an imported style sheet 
override, that is, are more important than those in previously imported style 
sheets. The href attribute tells the XSL processor where the style sheet resides by specifying its URI. The xsl : import elements must be grouped at the 
beginning of a style sheet. 
include Style sheets can be combined textually by including other style 
sheets. The href attribute tells the XSL processor where the style sheet resides by specifying its URI. If an included style sheet contains an xsl : import 
element, then the latter will be moved up to the beginning of the including 
style sheet to just after any existing xsl : import elements. 
invoke Used to invoke the processing of a macro defined with an xsl zmacro 
element inside a template. The name of the macro is specified with the help 
of the macro attribute. If the macro had arguments, then xsl : arg elements 
with name and value attributes can be specified if the defaults value set in the 
xsl zmacro element is not appropriate. 
zmacro Defines a macro containing multiple result fragments that can be referenced conveniently as a unit by using xsl : invoke. The macro is identified 
by assigning it a name with the name attribute. One or more macro arguments can be specified by including xsl :macro-arg elements with name and 
default attributes inside an xsl zmacro element. 
znumber Allows numbering in the source and result trees and provides control 
over the presentation format of the counters. See the XSL specification for 
details. 
zpi Creates a processing instruction node with a name specified by the name 
attribute and a value given by content of the xsl :pi element. 
sort Sorts nodes inside xsl : apply-templates and xsl : for-each elements. The xsl : sort has a select attribute that specifies a pattern evaluated 
for the node in question. The value of the first selected node is used as a sort 
key for that node. Further attributes are: 
order can be ascending or descending; 
lang specifies language of sort keys; 
data-type can be text or number; and 
case-order can be upper-f irst or lower-first. 
ztext Places text in the result tree. 
:value-of Computes generated text by extracting text from the source tree 
or by inserting a value of a text string with the help of the select attribute. 

 
%%page page_373                                                  <<<---3
 
7.6 Extensible Stylesheet Language 
7.6.7 Formatting objects and their properties 
Formatting objects are applied to the result tree node by being contained in the 
pattern part of the element. The general syntax is the following: 
<xs1:temp1ate match="pa.tte'rn"> 
<fo:fo'rma.tt'£ng-object (style p'rope'rty='“ua,L'u.e")*> 
[processing '£nst'r'u.ct1lons]* 
</fozformatting-object> 
</xsl:template> 
We have already seen an example of this syntax at the beginning of Section 7.6.1. 
The XSL specification tries, as far as possible, to define a formatting model that 
is compatible with both CSS and DSSSL in the names for objects, definitions, and 
property names. 
The draft XSL specification available in December 1998 defines only for1natting objects for basic word processor support. Following is a short overview of the 
available formatting objects. 
7.6.7.1 Layout formatting objects 
The following objects describe layouts for a series of pages (per chapter or for front, 
body, or back matter) 
basic-page-sequence Object describing the general layout of a (series of) 
“pages" (online, print, audio). It holds a set of si1nple-page-master or queue 
children. It is convenient as a means to define style rules that can be inherited 
by all “pages" of a document. 
simple-page-master Object to describe the general layout of a simple page by 
dividing it into five areas: header, body, footer, start-side, and end-side. 
In addition, one can specify a title that may be used, for instance, in the title 
bar of a Web browser. The characteristics of these areas and title can be defined 
with queue elements (see following). 
7.6.7.2 Content flow objects 
block Represents paragraphs, titles, headlines, captions, and so on. Blocks usually 
correspond to a rectangular area with a width equal to that of the containing area and a height determined by the material the block contains. A 
block can be separated from preceding and subsequent block-level objects. 
character Atomic unit to the formatter. This can be used to override one or 
more characters with specific glyph representations. 
display-graphic, inline-graphic Holds an image or vector graphic that is 
placed as a separate block or inline. 

 
%%page page_374                                                  <<<---3
 
354 
CSS, DSSSL, and XSL: Doing it with style 
disp1ay-link, inline-link Creates a (block level or inline) area that can be selected by the user to request traversal to another resource. A link contains 
link-end-locator flow objects. 
disp1ay-ru1e, inline-rule Draws a block-level or inline rule (line-segment). 
disp1ay-sequence, inline-sequence Groups content and allows the assignment of shared inherited properties. An example is choosing a local font 
change. They can be block level (like HTML’s DIV element) or inline (like 
HTML’s SPAN element). 
link-end-locator Target (destination) for a link. 
list-block A block-level object that acts as a container for the following flow 
objects: list-itezn, 1ist-itezn-label, and 1ist-item-body. 
1ist-itezn Groups the list-item-label and 1ist-itezn-body for each 
item in the list. 
list-item-label Contains the number or label of a list item. 
1ist-itezn-body Contains main content of a list item. Multiparagraph 
list items are formatted properly. If lists are nested, the second list 
must be contained as a child of a list-itezn-body flow object. 
-number Instructs formatter to construct and present a page number. 
queue Collects a sequence or tree of formatting objects to be presented in an 
area whose name corresponds to one of the queue names defined inside a 
simp1e-page-master element. 
Section 3 of the XSL Specification gives a detailed description of all of these 
formatting objects, enumerating all the possible attributes. 
7.6.7.3 Nlissing flow objects 
The current draft does not yet have support for the following: 
o multicolumn and more sophisticated page layouts; 
o layout-driven formatting, such as side-by-side material, floats, and extracted 
content (index, table of contents, endnotes, and so on), 
0 full internationalization (mixed scripts, locale-dependent formatting); 
o more formatter-generated text, such as auto-leaders, cross-references, citations, layout-derived numbering; 
o mathematics and tables; and 
o interactivity and multimedia. 

 
%%page page_375                                                  <<<---3
 
7.6 Extensible Stylesheet Language 
7.6.8 Proposed extensibility mechanism 
The current draft does not allow for scripting or for otherwise escaping to an (external) program or procedure to perform an action or (more or less) simple calculation 
that could influence the result tree. To gain some experience with such a functionality, which seems to be desired by many users of XSL, James Clark included in his 
xt program an experimental extensibility mechanism that is based on the idea of 
filtering fragments of the result tree through an object. 
7.6.8.1 A filtering mechanism 
Clark lets you use the element type xsl : invoke to specify a “filter." VV1th its 
class id attribute you can specify an external object (much like the HTML4 OBJECT 
tag). The content of the xsl : invoke element contains the result tree fragment to 
be filtered. The xsl : invoke element can start with one or more xsl : arg elements 
to present parameters to the object. The result of the xsl : invoke element is a “filtered result tree fragment." Thus the proposed extension mechanism affects only 
the result tree and does not act on the input source tree. The xsl : invoke elements 
can be nested, as seen in the second example that follows. 
In principle, the filters could use various interfaces, for example the DOM. 
However, in the current xt test implementation the Java object must implement 
an extension of the SAX Documentflandler interface (see Section B.6.1), as follows: 
package com.jc1ark.xs1.sax; 
1 
2 
3 public interface Filter extends org.xml.sax.DocumentHandler { 
4 void setDocumentHandler(org.xml.sax.DocumentHandler handler); 
5 void setParameter(String name, String value) throws SAXException; 
6 } 
Direct scripting is possible by allowing something as shown following, where 
the script would be defined by the contents of the value argument of the xsl : arg 
element (line 4). The script is later referenced by the select attribute of a 
xsl : for-each construct (line 5). 
1 <xsl:template match=``mytemplate''> 
2 <xsl:invoke c1assid="java:JSFi1ter"> 
3 <xsl:arg name=``script'' 
4 value=’ ... your Javascript code ... ’/> 
5 <xsl:for-each select=``value''> 
6 <xsl:apply-templates/> 
7 
8 
9 
10 
</xslzfor-each> 
</xsl:invoke> 
</xsl:template> 
7.6.8.2 Examples of using filters 
In order to get a feeling of what is possible in the current test implementation, we 
present two variants of demonstration programs that come with the xt distribution. 
The XSL style sheet writefile .xsl, shown here, writes a file whose name is 
defined by the filename attribute of the file tag (see line 4) in the XML source; 
355 

 
%%page page_376                                                  <<<---3
 
356 css, DSSSL, and XSL: Doing it with style 
the java call TextFi1e0utputFi1t er (line 3) is part of xt’s class library and actually 
instantiates the file. 
1 <xs1:stylesheet xmlns:xs1="http://www.w3.org/TE/wD-xs1" default-space=“strip"> 
2 <xs1:temp1ate match=``fi1e''> 
3 <xs1:invoke c1assid="java:com.jc1ark.xs1.sax.TextFi1e0utputFi1ter"> 
4 <xs1:arg name=``fi1e'' va1ue="{@fi1ename}"/> 
5 <xs1:app1y-templates/> 
6 </xsl invoke) 
7 </xs1:temp1ate> 
8 </xs1:stylesheet> 
Now we run the file writef i1es.xm1, shown here, together with the style sheet 
writefile . xsl through the xt XSL processor. 
<outputfi1es> 
<fi1e fi1ename="fi1ea.out">1O &1t; 20 
</fi1e> 
<fi1e fi1ename="fi1eb.out">43 &gt; 34 
An ampersand character: &amp;. 
</fi1e> 
</outputfi1es> 
\-ow-h-mm.This generates a file filea. out (see line 2 above), containing one line: 
10 < 20 
and a file fileb . out (lines 4-5 above), containing two lines: 
43 > 34 
An ampersand character: &. 
See how the f ilename attribute of the file element defines the name of the output 
file, and how the entities are translated into the characters they represent. 
A second example sums a set of numerical strings. We use the XSL style sheet 
makesum.xs1 as shown here: 
1 <xs1:stylesheet xmlns:xs1="http://www.w3.org/TE/wD-xs1" default-space=``strip''> 
2 <xs1:temp1ate match=``makesum''> 
3 <xs1:invoke c1assid="java:com.jc1ark.xs1.sax.TextFi1e0utputFi1ter"> 
4 <xs1:arg name=``fi1e'' va1ue="sum.out"/> 
5 <xs1:invoke c1assid="java:com.jc1ark.xs1.sax.Tota1Fi1ter"> 
6 <xs1:app1y-templates/> 
7 </xsl invoke) 
8 </xs1:invoke> 
9 </xs1:temp1ate> 
10 <xs1:temp1ate match="rea1“> 
ii <number><xs1:apply-templates/></number> 
12 </xs1:temp1ate> 
13 </xs1:stylesheet> 
The style sheet uses two templates. The first one (lines 2-9) matches a znakesum 
element in the XML input file. It defines the output file name sum.out with 
the help of the arg element of the first xslzinvoke element that uses the 

 
%%page page_377                                                  <<<---3
 
7.6 Extensible Stylesheet Language 
357 
class TextFile0utputFi1ter (lines 3-4, as in the previous example). A second 
xsl : invoke element (lines 5-7), nested inside the first, applies a class TotalFilter 
on the children of the znakesum element. This Java class is also part of the xt distribution and sums at set of character strings. The second template (lines 10-12) 
matches real elements and puts them between number tags in the result tree, where 
they will be summed by the Tota1Filter class. 
Suppose we present the following XML file znakesum.xznl together with the 
style sheet makesum . xsl to the X1: processor. 
<makesum> 
<rea1>3.0</rea1> 
<rea1>O.14159</rea1> 
<rea1>.OOOOO26536</rea1> 
</makesum> 
~«.4=wN._ 
Now the output file sum. out contains 3.1415926536, which is indeed the sum of 
the three numbers in the input file. 
7.6.8.3 Simple string evaluation 
Often we do not need the full complexity of the approach outlined earlier, since our 
need for extensibility can be limited to string-to-string manipulations. For instance, 
converting between units, formatting of counters, simple arithmetic, or stripping 
whitespace can be handled as a transformation on strings. For this, one could define 
a special case of xsl : invoke that would build a string starting from the text content 
of the result tree fragment being filtered, apply a script to that string, and replace 
the result tree fragment by a single text node containing the result of evaluating the 
script. As an example, suppose we want to add five to the string at the present result 
node fragment, we could then write something like 
<xs1:transform-string script="Number(this)+5"> 
<xs1:number format=``a''/> 
</xslztransform-string> 
where the xsl znumber element would generate a lowercase alphabetic representation of the number at the present result node after applying the script. 
We want to stress once more that the extension mechanism discussed in this 
section is a proposal and presents only one possible approach to the problem. The 
form finally adopted for the XSL language might end up being very different from 
what is described here. 
7.6.9 Using XSL to generate HTML or BTEX 
To show some of the possibilities of XSL in the area of formatting an XML document, once more we are going to use the two versions of the invitation XML 
files. First we will generate ETEX, in much the same way as we did with Perl, and 

 
%%page page_378                                                  <<<---3
 
358 CSS, DSSSL, and XSL: Doing it with style 
use the same class file. We have to define templates for all the XML elements that 
we want to treat, as shown in the following XSL style file invlat1.xs1: 
<?xm1 version=’1.0’?> 
<xs1:stylesheet xm1ns:xs1="http://www.w3.org/TE/WD-xsl" 
default-space=``strip'' 
resu1t-ns=""> 
<xs1:temp1ate match=``invitation''> 
<xs1:text>\documentclass[]{article} 
\usepackage{invitation} 
10 \begin{document} 
11 </xs1:text> 
12 <xs1:app1y-templates/> 
13 <xs1:text>\end{document} 
14 </xs1:text> 
15 </xs1:temp1ate> 
\O®\lQ‘-II-CIR»-‘lul17 <xs1:temp1ate match="invitation/front") 
18 <xs1:text>\begin{Front} 
19 \T0{</xs1:text> 
20 <xs1:va1ue-of se1ect=``to''/> 
21 <xs1:text>} 
22 \Date{</xs1:text> 
23 <xs1:va1ue-of se1ect=``date''/> 
24 <xs1:text>} 
25 \where{</xs1:text> 
25 <xs1:va1ue-of se1ect=``where''/> 
27 <xs1:text>} 
28 \Why{</xsl:text> 
29 <xs1:va1ue-of se1ect=``why''/> 
30 <xs1:text>} 
31 \end{Front} 
32 </xs1:text> 
33 </ks1:temp1ate> 
35 <xs1:temp1ate match="invitation/body"> 
36 <xsl:text>\begin{Body} 
37 </xs1:text> 
38 <xs1:app1y-temp1ates/> 
39 <xs1:text>\end{Body} 
40 </xs1:text> 
41 </xs1:temp1ate> 
43 <xs1:temp1ate match="invitation/body/par"> 
M <xs1:text>\par</xs1:text> 
45 <xs1:app1y-templates/> 
46 </xs1:temp1ate> 
48 <xs1:temp1ate match="invitation/body/par/emph"> 
49 <xs1:text>\emph{</xs1:text> 
50 <xs1:app1y-templates/> 
S1 <xs1:text>}</xs1:text> 
52 </xs1:temp1ate> 
54 <xs1:temp1ate match=“invitation/back"> 
SS <xs1:text>\begin{Back} 
so \Signature{</xs1:text> 
57 <xs1:va1ue-of se1ect=``signature''/> 
S3 <xsl:text>} 
so \end{Back} 
m </xs1:text> 
51 </xs1:temp1ate> 
62 </xs1:stylesheet> 

 
%%page page_379                                                  <<<---3
 
7.6 Extensible Stylesheet Language 
359 
After defining the usual namespace (xsl) on line 3, the next line (4) indicates 
that we want all “default" space to be collapsed. This means that only whitespace, 
which we enter ourselves, will be retained. This is useful because XML and XSL by 
default leave whitespace untouched, which can lead to strange effects with ETEX. 
The first template (lines 7-15) treats the invitation element. It initializes 
ETEX (lines 8-9), starts the document (line 10), and then lets XSL process all its 
child elements (line 12) before finalizing the document (line 13). The ETEX commands are entered literally in the output stream by the xsl ztext elements. Inside 
this element, space and newline characters are significant and are faithfully written 
to the output. 
The second template (lines 17-33) deals with the front element and its children. First we open the Front environment (line 18), write the \To command, and 
then tell XSL to go and fetch the content of the to element (line 20) with the help 
of the xsl : value command. Similarly we get the information for the \Date (lines 
22-24), \Where (lines 25-27), and \why (lines 28-30) commands. Lines 30-32 take 
care of closing the Front environment. 
We now arrive at the body element of the XML file, which is handled with lines 
35-41. The action consists of starting the Body environment (lines 36-37), telling 
XSL to handle the children of the XML body element (line 38), and closing the Body 
environment (lines 39-40). 
The par children of the body element are handled with the pattern on lines 
43-46; a ETEX \par command is issued for each such element (line 44) before 
continuing to treat children elements of par itself (line 45). 
The character content of the emph elements inside par, inside body, and inside 
invitation (line 48 is a good example of explicit ancestor match selection) gets 
transferred into the argument of the ETEX \eznph command (lines 49-51). 
Finally we arrive at the back element, which is handled on lines 54-61. Line 5 5 
starts a ETEX Back environment, while lines 5 6-5 8 take care of copying the content 
of the signature element into the argument of ETEX’s \Signature command. 
Line 59 ends the Back environment. 
We use the XML file invitation . xml defined on page 254 and the earlier XSL 
style sheet with the xt XSL processor, as follows: 
xt invitation.xml invlat1.xsl invlat1.tex 
The ETEX file inv1at1 .tex that we obtain follows. It is completely identical 
to the TEX file we showed on page 295. 
\documentclass[]{article} 
\usepackage{invitation} 
\begin{document} 
\begin{Front} 
\To{Anna, Bernard, Didier, Johanna} 
\Date{Next Friday Evening at 8 pm} 
\Where{The Web Cafe} 
\why{My first XML baby} 
\end{Front} 
\begin{Body} 
Eooo\.¢.«...u..N_. 

%==========380==========<<<---2
 
%%page page_380                                                  <<<---3
 
360 
css, DSSSL, and XSL: Doing it with style 
\par 
I would like to invite you all to celebrate 
the birth of \emph{Invitation}, my 
first XML document child. 
\par 
Please do your best to come and join me next Friday 
evening. And, do not forget to bring your friends. 
\par 
I \emph{rea11y} look forward to see you soon! 
\end{Body} 
\begin{Back} 
\Signature{Miche1} 
\end{Back} 
\end{document} 
Our next example will show how to generate an HTML file for the XML file 
invitation2 .xm1 described in Section 7.4.5. We take advantage of a feature of the 
XSL xt processor that allows you to specify which is the default namespace for the 
elements inside the XSL style sheet. Thus in the file invhtzn12 .xs1, on line 4, we 
specify that element types with unspecified namespace in the style sheet refer to 
the HTML4 specification. This makes it very convenient to mix XSL and HTML 
commands in the body of the templates. 
E:3~om\:o«vu4>wN>-<?xm1 version=’1.0’?> 
<xs1:stylesheet 
xmlns:xs1-"http://wwv.w3.org/TR/WD-xsl" 
xmlns-"http://www.w3.org/TR/REC-htm140" 
resu1t-ns=""> 
<xs1:temp1ate match="/"> 
<html> 
<head> 
<title> Invitation (XSL/CSS formatting) </title) 
<link href="invit.css" re1=``stylesheet'' type="text/css"/> 
<!-- 4 January 1998 mg --> 
</head> 
<body> 
<h1>INVITATION</h1> 
<tab1e> 
<tbody> 
<tr><td c1ass=``front''>To: </td> 
<td><xs1:va1ue-of se1ect=“0to“/></td></tr> 
<tr><td c1ass=``front''>When: </td> 
<td><xs1:va1ue-of select-``Qdate''/></td></tr> 
<tr><td c1ass=``front''>Venue: </td> 
<td><xs1:va1ue-of select-``owhere''/></td></tr> 
<tr><td c1ass=``front''>0ccasion: </td> 
<td><xs1:va1ue-of se1ect="@why"/></td></tr> 
</tbody> 
</tab1e> 
<xs1:app1y-templates/> 
<p class-``signature''><xs1:va1ue-of se1ect=``osignature''/></p> 
</body> 
</html> 
</xs1:temp1ate> 
<xs1:temp1ate match="invitation/par"> 
<p><xs1:app1y-temp1ates/></p> 
</xs1:temp1ate> 
<xs1:temp1ate match="invitation/par/emph"> 
<em><xs1:app1y-temp1ates/></em> 
</xs1:temp1ate> 
</xs1:stylesheet> 

 
%%page page_381                                                  <<<---3
 
7.6 Extensible Stylesheet Language 
361 
In this style sheet, lines 7-32 take care of the root element. In particular, the HTML 
file is initialized (lines 8-14) with the link to the CSS style sheet invit . css on line 
11. A level 1 heading is defined (line 15), and a table is started on line 16. This is to 
represent nicely the various attributes of the invitation element. For instance, on 
line 19 you can see how to get access to the attribute of an XML element by using 
XSL’s xsl zvalue-of element and the attribute Q specifier on its select attribute. 
In this case, we are after the value of the to attribute, which is put in the right-hand 
cell of the first row of the table that contains the string “To: " in the left-hand cell 
(line 18). The following pairs of lines 20 to 25 build rows for the other attributes of 
the invitation element. On lines 26-27 we end the table before instructing XSL 
first to process the children of invitation (line 28) and then to write an HTML 
p element of class signature with as content the value of the signature attribute 
of the <invitation> start tag (line 29). Lines 30-31 end the HTML document 
gracefully. 
The remaining tasks are to handle the par and emph elements. These are dealt 
with in lines 33-35 and 36-38, respectively. Compare the procedure to the one 
described in Section 7.4.6. 
The style sheet invhtznl2 .xsl and XML file invitation2.xml are presented 
to the xt processor. 
xt invitation2.xml invhtml2.xsl invhtml2.html 
We obtain the HTML file invhtznl2 . htznl, listed here. It is equivalent to the HTML 
file shown on page 308. 
<!DOCTYPE html PUBLIC "-//W3C//DTD HTML 4.0 Transitional//EN"> 
<html> 
<head> 
<title> Invitation (sgmlpl/CSS formatting) </title> 
<link href="invit.css" rel=“stylesheet" type="text/css"> 
</head> 
<body> 
<h1>INVITATION</h1> 
<table> 
<tbody> 
<tr><td c1ass=``front''>To: </td> 
<td>Anna, Bernard, Didier, Johanna</td></tr> 
<tr><td class=``front''>When: </td> 
<td>Next Friday Evening at 8 pm</td></tr> 
<tr><td c1ass=``front''>Venue: </td> 
<td>The web Cafe</td></tr> 
<tr><td c1ass="front“>0ccasion: </td> 
<td>My first XML baby</td></tr> 
</tbody> 
</table> 
<p>I would like to invite you all to celebrate 
the birth of <em>Invitation</em>, my 
first XML document chi1d.</p> 
<p>Please do your best to come and join me next Friday 
evening. And, do not forget to bring your friends.</p> 
<p>I <em>rea1ly</em> look forward to see you soon!</p> 
<p c1ass=``signature''>Miche1</p> 
</body> 
</htm1> 
hJI\JI\)I\JI\JI\Jl\lI\Jl\Jl\lb-|>-‘>-b-|>->-b-K>-r->coex-o«v->.w~>-o~ooo\uo«w-+.w--o~ooo\-o«v..uwN..
 
%%page page_382                                                  <<<---3
 
362 
CSS, DSSSL, and XSL: Doing it with style 
7.6.10 Using XSL to generate formatting objects 
As we have already explained several times, the best approach for setting up a style 
sheet for a given class of documents is to express it in terms of generic formatting 
objects. This way your style sheet can be translated into various output formats. 
We have shown that this works well in the case of DSSSL, where we used Jade as 
a processing engine (see Section 7.5.3). Similarly we can write an XSL style sheet 
using only XSL’s formatting objects and translate those into various output representations. However, at present, because the XSL standard is not yet finalized, 
no general-purpose tool, such as Jade, is available. Nevertheless, we can use James 
Tauber’s fop Java processor, which transforms (a subset of) XSL formatting objects 
into PDF. 
Let us first look at the XSL style sheet invfol . xsl that we have prepared for 
use with the invitation XML source file. 
1 <?xm1 version=’1.0’?> 
2 <xs1:stylesheet xmlns:xs1="http2//www.w3.org/TE/WD-xsl" 
3 xmlns:fo="http://www.w3.org/TE/WD-xsl/F0" 
4 resu1t-ns=``fo'' 
S default-space=“"> 
6 <xs1:constant name=``Fontsize'' va1ue=``12pt''/> 
7 <xs1:macro name=``1istitem''> 
8 <xs1:macro-arg name=``itemid''/> 
9 <xs1:macro-arg name=``itemtext''/> 
10 <fo:1ist-item id=“{arg(itemid)}"> 
11 <fo:1ist-item-label font-style=``ita1ic''> 
12 <xs1:va1ue-of se1ect=’arg(itemtext)’/><xs1:text>:</xs1:text> 
B </fozlist-item-label) 
14 <fo:1ist-item-body> 
15 <xs1:contents/> 
16 </fo:1ist-item-body> 
17 </fo:1ist-item> 
18 </xs1:macro> 
19 
20 <xs1:temp1ate match=’/’> 
21 <fo:basic-page-sequence font-fami1y=``serif'' font-size="{constant(Fontsize)}" 
22 margin-top=``15mm'' margin-bottom=``15mm'' 
23 margin-1eft=``15mm'' margin-right=``15mm'' 
24 page-width=``120mm'' id=``pageseq''> 
zs <xsl:app1y-templates/> 
26 </fo:basic-page-sequence> 
27 </xs1:temp1ate> 
29 <xs1:temp1ate match="invitation/front"> 
30 <fo:disp1ay-sequence> 
31 <fo:b1ock font-fami1y=``sans-serif'' font-size=``24pt'' 
H font-weight=``bo1d'' text-a1ign="center“> 
33 <xs1:text>INVITATION</xsl:text> 
34 </fo:b1ock> 
35 <fo:1ist-b1ock label-width=``2cm''> 
36 <xs1:invoke macro=``1istitem''> 
<xs1:arg name=``itemtext'' va1ue=``To''/> 
<xs1:arg name=``itemid'' va1ue=``1istto''/> 
<xs1:va1ue-of se1ect=``to''/> 
</xs1:invoke> 
<xs1:invoke macro=``1istitem''> 
<xs1:arg name=``itemtext'' va1ue=``when''/> 
<xs1:arg name=``itemid'' va1ue=``1istdate''/> 
<xs1:va1ue-of se1ect=``date''/> 
</xs1:invoke> 
w 
\$ $ $ 3 i 8 3 3 

 
%%page page_383                                                  <<<---3
 
7.6 Extensible Stylesheet Language 
363 
% <xs1:invoke macro=``1istitem''> 
47 <xs1:arg name=``itemtext'' va1ue=``Venue''/> 
48 <xs1:aIg name=``itemid'' va1ue=``1istwhere''/> 
49 <xs1:va1ue-of se1ect=``where''/> 
50 </xs1:invoke> 
51 <xs1:invoke macro=``1istitem''> 
Q <xs1:arg name=``itemtext'' va1ue=``0ccasion''/> 
53 <xs1:arg name=``itemid'' va1ue=``1istwhy''/> 
54 <xs1:va1ue-of se1ect=``why''/> 
55 </xs1:invoke> 
m </fo:1ist-b1ock> 
57 </fozdisplay-sequence> 
58 </xs1:temp1ate> 
60 <xs1:temp1ate match="invitation/body/par"> 
<fo:b1ock space-before="{constant(Fontsize)}"> 
<xs1:app1y-templates/> 
</fo:b1ock> 
</xs1:temp1ate> 
<xs1:temp1ate match="invitation/body/par/emph"> 
<fo:inline-sequence font-style=``ita1ic''> 
<xs1:app1y-templates/> 
</fozinline-sequence> 
</xs1:temp1ate> 
\I 000* 0 
S a 3 c m u 3 3 # $ E E 
<xs1:temp1ate match="invitation/back"> 
73 <fo:b1ock space-before="{constant(Fontsize)}" 
74 font-weight=``bo1d'' text-a1ign=``right''> 
75 <xs1:text>From: </xs1:text> 
76 <xs1:va1ue-of se1ect="signature“/> 
77 </fo:b1ock> 
78 </xs1:temp1ate> 
79 </xs1:stylesheet> 
In this style sheet we take an approach quite similar to what we did in the case of 
DSSSL (see Section 7.5.2.2). We declare a constant Fontsize (line 6) that controls 
the default size of the document font. To show how macros are handled in XSL, 
we define a macro named listitem on lines 7-18. The macro has two arguments: 
iteznid (line 8), which corresponds to an identifier of the list item (line 10), and 
itezntext (line 9), which is used for generating text for the item label (line 12). 
Notice how the syntax arg( . . . ) is used to access the actual value of the arguments. 
On line 15 the <xs1: contents/> element is a placeholder for the actual contents 
of the macro when it is invoked (see following). 
The page dimensions are defined (lines 21-26) inside the root template (lines 
20-2 7), where we also set the default size of the document font (line 21), using the 
constant Fontsize declared earlier. 
We then select the front part of the document (lines 29-5 8) and start by writing the text “INVITATION" in a sans serif 24 pt font centered on the output medium 
as a block object (lines 3 1-3 4). The rest of the front matter is displayed in a list (lines 
35-56). The list has a label width of 2 cm to write the fixed texts (line 35). For each 
of the four elements inside the front element, we invoke the list it em macro, defined previously. Let us review in detail one of the invocations. On lines 36-40 we 
deal with the to element. First we set the itezntext argument to the string “To" 
and the itemid argument to “1istto" identifier. The contents of the to element in 

 
%%page page_384                                                  <<<---3
 
364 
CSS, DSSSL, and XSL: Doing it with style 
the XML source document is retrieved with the <xsl:va1ue of se1ect="to“/> 
syntax. It is then consumed as body of the macro by the <xsl: contents/> element 
in the macro definition (line 15). All this gives the result shown on lines 12-15 
following. Lines 41-55 handle the date, where, and why elements in a similar way. 
We then go to the body part of the document, where for each paragraph (par 
element type) we generate a block object (lines 61-63), skipping a space equal to the 
font size (line 61). The eznph elements (lines 66-70) are transformed into a sequence 
using an italic font (lines 67-69). 
Finally, for the back matter (lines 72-78) we create a block object (lines 73-77), 
where, after setting the space before, we require a bold font and right-justified text 
(line 74). We output the string “From: " followed by the contents of the signature 
element type. 
This style sheet invfol . xsl and the XML source file invitation . xml can be 
compiled by]ames Clark’s xt tool, as follows: 
xt invitation.xml invfo1.xsl invfo1.fop 
The file invfol . f op contains the formatting objects. It has the following form (the 
correspondence between style sheet and generated code is easily identified): 
1 <fo:basic-page-sequence xm1ns:fo="http://www.w3.org/TE/WD-xsl/F0" 
2 font-fami1y=``serif'' font-size=``12pt'' margin-top=``15mm'' margin-bottom=``15mm'' 
3 margin-1eft=``15mm'' margin-right=``15mm'' page-width=``120mm'' id=``pageseq''> 
4 <fo:disp1ay-sequence> 
5 <fo:b1ock font-fami1y=``sans-serif'' font-size=``24pt'' 
6 font-weight=``bo1d'' text-a1ign=``center''> 
7 INVITATION 
8 </fo:b1ock> 
9 <fo:1ist-block label-width=``2cm''> 
<fo:1ist-item id=``1istto''> 
<fo:1ist-item~label font-style=``ita1ic''>To:</fozlist-item-label> 
<fo:1ist-item-body>Anna, Bernard, Didier, Johanna</fo:list-item-body) 
</fo:1ist-item> 
<fo:1ist-item id=``1istdate''> 
<fo:1ist-item-label font-style=``ita1ic''>when:</fozlist-item-label> 
<fo:1ist-item-body>Next Friday Evening at 8 pm</fo:1ist-item-body> 
</fo:1ist-item> 
<fo:list-item id=``listwhere''> 
<fo:1ist-item-label font-style=``italic''>Venue:</fozlist-item-label> 
<fo:1ist-item-body>The Web Cafe</fo:list-item-body> 
</fozlist-item> 
<fo:list-item id=``listwhy''> 
<fo:list-item-label font-style=``italic''>Dccasion2</fo:list-item-label> 
<fo:1ist-item-body>My first XML baby</fozlist-item-body> 
</fozlist-item) 
26 </fozlist-block) 
27 </fozdisplay-sequence) 
28 <fo:block space-before-optimum=``12pt''> 
29 I would like to invite you all to celebrate the birth of 
30 <fo:inline-sequence 
31 font-style=``ita1ic''>Invitation</fo2inline-sequence>, 
32 my first XML document child. 
33 </fo:block> 
34 <fo:block space-before-optimum=``12pt''> 
35 Please do your best to come and join me next Friday 
36 evening. And, do not forget to bring your friends. 
o 
~N-._..--.--......_.......»u4I\aI-©\o@\lQu\.J:vJI\J>-A 
N 
4: 
N 
V. 

 
%%page page_385                                                  <<<---3
 
7.6 Extensible Stylesheet Language 
38 
39 
41 
43 
INVITATION 
T 0: Anna, Bernard, Didier, Johanna 
When: Next Friday Evening at 8 pm 
Venue: The Web Cafe 
Occasion: My first XML baby 
I would like to invite you all to celebrate the birth of 
Invitation, my first XML document child. 
Please do your best to come and join me next Friday 
evening. And, do not forget to bring your friends. 
I really look forward to see you soon! 
From: Michel 
\reffig{7-11}: PDF generated from flow objects with fop 
</fo:block> 
<fo:b1ock space-before-optimum=``12pt''> 
I <fo:inline-sequence font-style=``ita1ic''>rea11y</fo:inline-sequence> 
look forward to see you soon! 
</fo:block> 
<fo:block space-before-optimum=``12pt'' font-weight=``bold'' text-align=``right''> 
From: Michel 
</fo:block> 
</fo : basic-page-sequence) 
This file can now be input into James Tauber’s fop program, as follows: 
java com.jtauber.fop.FDP invfo1.fob invfo1.pdf 
James Tauber’s FOP 0.5.0 
successfully read and parsed temp.:fob 
successfully wrote inv:fo1.pdf 
This program generates the PDF file invfo1.pdf shown in \reffig{7-11}. Although we used a completely different style sheet language and data model, we 
obtain a result that is quite similar to the one in \reffig{7-8}. This shows that XSL is 
certainly a promising technology. Once the specification is finalized, XSL will allow 
us to use XML-based syntax and tools for all stages of our document handling. 
365 

 
%%page page_386                                                  <<<---3
 
366 
CSS, DSSSL, and XSL: Doing it with style 
Summary 
It should be clear from the discussion in this chapter that currently there are various interesting approaches to handling XML-tagged files. Most of the methods presented rely on interpreting the output from an XML parser (directly or indirectly) 
and applying “action" rules to the various XML input elements. Interpreters, such 
as Perl, Java, or Python, are available to drive the process of generating BTEX or 
HTML output. However, we stress once again that it is important to program the 
translation at a high level of abstraction, via the use of a BTEX class or CSS style 
sheet that can be reused by various document instances, even those constructed 
according to a different DTD. 
We have looked in detail into three style sheet languages: CSS, DSSSL, and 
XSL. Caxcading Style Sheets are linked strongly with the Web and are thus particularly well-suited for separating form and content for screen display. More recently 
the second edition of the CSS specification has added capabilities in the area of 
multimedia and printing. 
Today the most complete formatting model is offered by the Document Style 
Semantic: and Specification Language. It is an ISO standard and can handle complex 
 layouts, tables, and mathematics. It is the only possible choice if one needs 
high-quality typography of nontrivial documents. 
The latest arrival on the style sheet market is the Extensible Style Language. 
It is supposed to offer, in due course, at least the functionality of both CSS and 
DSSSL, but with a syntax that is well integrated with the “X (extensible)" family 
of tools. Efforts are underway to ensure that the formatting models of all W3C 
activities, including CSS, XSL, and SVG, will converge. However, XSL is at present 
far from finalized, and it certainly is not easy to decide whether XSL will live up to 
its promises of offering a complete solution for handling all XML documents with 
style. 
In the few pages dedicated to each of these style sheet languages, we could only 
scratch the surface of each. Nevertheless, we have given you an idea of their basic 
functionality and general syntax. This should allow you to read and understand files 
written in any of these languages, modify them according to your needs, or even 
write your own. 
The information and examples presented in this chapter should give you 
enough insight to decide which of the various possibilities (CSS, DSSSL, and XSL) is 
most useful for solving your problem today. Moreover, thanks to the references in 
the text, you can choose to follow the evolution of these tools. It is to be expected 
that XSL and CSS will develop into f11ll-blown style sheet languages, the former 
as a viable replacement for DSSSL for complex offline documents, the latter for 
multimedia applications on the Web. 

 
%%page page_387                                                  <<<---3
 
CHAPTER 8 
MathML, intelligent 
math markup 
This chapter is directed at people who are currently using BTEX as their main 
markup format, but who also want to consider moving onto SGML-like systems. 
It is obvious that HTML is not rich enough to be considered a permanent storage 
medium, and from what we have seen in Chapters 6 and 7, it is clear that XML 
is the best way to go. However, many people are worried about how to deal with 
their legacy documents, particularly those containing math (often the reason they 
chose TEX in the first place). We will look at an answer to the math problemthe Mathematical Markup Language-and consider how to convert existing BTEX 
documents to XML. 
This chapter is divided into three parts. First we provide a summary of the 
new language; second we look at the first generation of software that supports the 
language; and third we go back to cover in more detail a practical I5TEX-to-XML 
(including MathML) translation system using TEX4ht. 
We must stress that the use of MathML is in its infancy, as is the development 
of reliable I£}'IFX-to-XML converters. In this chapter, we will be looking at early 
versions of software and suggesting some approaches to take, but you should not 
consider these as definitive. We will be obliged to be much vaguer and less precise 
about details, than elsewhere in this book. 

 
%%page page_388                                                  <<<---3
 
368 
MathNIL, intelligent math markup 
8.1 Introduction to MathML 
After being virtually ignored for some years, mathematics on the Web is making a 
comeback. Several small scale studies and experiments on how to deal with math in 
SGML and with browsers took place. Only recently, however, did the big players in 
the math business (computer algebra vendors like Maple, Mathcad, Mathematica; 
large scientific publishers like Elsevier and the American Mathematical Society; 
and software companies like Adobe, as well as the W3 C) get together to define a 
language to describe the structure and content of mathematical expressions. The 
first outcome of their efforts is a specification for MathML (Mathematical Markup 
Language). The f11ll details are available on the Web [‘--> MMLSPEC], as are extensive 
lists of MathML resources at [‘-->MMLRES] and [*->MMLGUID]. 
MathML is a markup language for math to be used with XML. Its design goals, 
as specified by the W3C Mathematics Working Group, state that MathML should 
0 encode mathematical material for teaching and scientific communication at all 
levels; 
0 encode both mathematical notation and its meaning; 
0 be well-suited to template and other math editing techniques; 
0 facilitate conversion to and from other math formats, both presentational and 
semantic. Output formats should include graphical displays, speech synthesizers, computer algebra systems’ input, other math layout languages such as TEX, 
plain text displays, and print media, including Braille. It is recognized that conversion to and from other notations may lose information in the process; 
0 allow the passing of information intended for specific renderers; 
0 support efficient browsing for lengthy expressions; and 
0 provide for extensibility (in as yet undefined ways). 
The original aim that MathML be easy to learn and to edit by band for basic math 
notation was effectively dropped when the working group decided to use XML 
rather than full SGML (the project started before XML). 
MathML contains two “views" of mathematics: prexentation markup and semantic markup. Presentation markup describes math notation, with expressions being 
built up using layout schemata, and specifying how to arrange subexpressions, such 
as fractions, and superand subscripts. Semantic markup describes mathematical 
objects and functions, where an expression tree is constructed with each node representing a particular schema, and branches representing its subexpressions. 
To illustrate the two approaches, let us look at the following simple formula 
that we will mark up according to the two schemes. 
x2-6x+9=O 

 
%%page page_389                                                  <<<---3
 
8.1 Introduction to MathML 369 
The ETEX code for this would be 
\[ x"{2} - 6x + 9 = o\] 
The first way of using MathML is to use presentation tags to describe the visual layout 
of a mathematical formula. 
<mrow> 
<mrow> 
<msup> 
<mi>x</mi) 
<mn>2</mn> 
</msup> 
<mo>-</mo) 
<mrow> 
<mn>6</mn> 
<mo>&Invisib1eTimes ; </mo) 
<mi>x</m'1> 
</mrow> 
<mo>+</mo) 
<mn>9</mn> 
</mrow> 
<mo>=</mo) 
<mn>0</mn> 
</mrow> 
;:iB:::;::5soan\Io~u-;;wN._. 
You see here two kinds of MathML tags. First there are those that contain 
data, such as <mi> for identifiers (lines 4 and 11), <mn> for numbers (lines 5, 9, 14, 
and 17), and <mo> for operators (lines 7, 10, 13, and 16). Second there are those 
that contain only other nested MathML elements, such as <msup> (lines 3-6) and 
<mrow> (lines 1-8, 2-15, and 8-12). The nested mrow elements denote terms-in 
this case the left-hand side of the equation functioning as an operand of the equal 
sign. By specifying the type of the data and marking terms, we greatly facilitate 
things like spacing for visual rendering, voice rendering, linebreaking, as well as 
automatic processing by external applications. Notice that, in contrast to ETEX, 
the <msup> element has two parts: the superscript itself and the material to which it 
is attached-<msup><mi>x</mi><mn>2</mn></msup>. Compare this with P}TEX’s 
x" {2}, where there is no markup before the x. 
The second way to use MathML is to express the xemzmtic content of the same 
expression as follows: 
1 <re1n> 
2 <eq/> 
3 <app1y> 
4 <p1us/> 
5 <app1y> 
6 <minus/> 
7 <app1y> 
3 <power/> 
9 <ci>x</c'1> 
10 <cn>2</cn> 
11 </apply) 
12 <app1y> 
13 <times/> 
14 <cn>6</cn> 

%==========390==========<<<---2
 
%%page page_390                                                  <<<---3
 
370 
MathML, intelligent math markup 
<ci>x</ci> 
</apply> 
</apply) 
<cn>9</cn> 
</apply) 
<cn>0</cn> 
</re1n> 
-..._-._-..._. 
>-ocanxxoxu. 
MathML content tags are typically contained within an <apply> tag that denotes a semantically meaningful expression (lines 3-19, 5-17, 7-11, and 12-16) and 
acts like a pair of parentheses. Prefix notation is used to build up the expression 
(the <plus/>, <times/>, and <minus/> tags). Token elements are used to indicate 
numbers (<cn>) and symbols (<ci>). 
Can MathML succeed in its aims? Some practicing mathematicians think that 
MathML presentation markup can be made to work, despite its extreme verbosity 
when compared to TEX. The biggest change in working methods and software that 
will be needed is to support a distinction between authoring convenience, archival 
markup, and rendering markup. BTEX users are accustomed to writing something 
like 
\newcommand{\Wotsit}{\ensuremath{E_{\beta}“{\infty}}} 
\Wotsit{} means whatever you like 
This combines the establishment of a higher level of abstract markup than standard BTEX, a typing shortcut, and the details of an actual rendition. In the XML 
world you would do this in two stages-first adding a new element <wotsit> to 
your Document Type Definition and second adding a rule to your style sheet that 
mapped the element onto 
<math> 
<msubsup> 
<mi>E</mi> 
<mi>&beta;</mi> 
<mo>&infty;</mo> 
</msubsup> 
</math> 
While this seems highly inconvenient compared to current BTEX, it has many advantages, such as better syntax checking, context-sensitive editing, searching, and 
the possibility of using different rendering engines. It could replace TEX markup 
for authoring in some systems. Naturally we can continue to use the TEX typesetting engine, Computer Modern fonts, and so on, for producing printed pages. An 
alternative scenario is that a “Next Generation" BTEX could adopt a syntax much 
closer to that of MathML and share many of the same advantages. 
Semantic math markup is another matter; for parts of math it can be established 
as a communication medium quite easily Some classify its coverage as everything 
up to university level, others more unkindly classify it as everything before the 20th 

 
%%page page_391                                                  <<<---3
 
8.1 Introduction to MathML 
371 
century! For research-level math, it is not clear whether the constraints of MathML 
will prove too much. When doing semantic math, we cannot adopt the system of 
ad hoc high-level “little languages" for a particular domain of math that would then 
expand to the fixed MathML semantic markup, since the point of content markup 
is to allow interchange between computer programs. Exchange of content must be 
at the highest level of meaning. 
8.1.1 MathML, Unicode, and XML entities 
MathML is an application of XML, so symbols must come entirely from the Unicode (see Appendix C.2) character set. There are two problems with this system: 
1. Unicode (version 2) does not contain the myriad special symbols mathematicians have developed over the years. 
2. When reading MathML (let alone writing it by hand), most authors will not 
(in the short term) be using fully Unicode-aware software or may be using 
the UTF-8 encoding (Appendix C.2.3). They may not, therefore, see the math 
symbols but would instead see some unintelligible encoding. 
The first problem should be solved by a project set up in 1997 by a group of scientific publishers and software suppliers (STIPUB). They made a comprehensive 
collection of all the characters used in modern scientific publishing and computer 
software. Those that were not present in the current Unicode were proposed, at 
the end of 1998, to the Unicode consortium for inclusion in the next revision of the 
Unicode character set. Until the proposed additions to Unicode are accepted, practical use of MathML has to include the Unicode private zone (see the following). 
Even when the characters are in Unicode, the problem of no immediately available 
fonts to cover the whole set remains. To remedy this, the same group of publishers 
and software vendors plans to commission a completely new set of math fonts and 
make them freely available. 
The second problem can be solved by the use of entity references in the DTD 
that is, using symbolic names for the characters instead of the Unicode character 
itself. Thus a qb is written and read as &phi ; . There are, however, some problems 
here too. First, entities declared in an XML must resolve eventually to characters 
in the Unicode scheme. If the character is not in the current version of Unicode, 
we must pick on an arbitrary position in Unicode’s “private zone," and all MathML 
software must agree on the convention. There is an obvious danger that some other 
application may use the same positions in the private zone, and be unable to coexist with MathML. Second, XML software does not have to read a DTD when 
it processes MathML documents, and so entities may have to be defined in the 
document rather than in the DT D (which is highly inconvenient) or the documents 
have to be preprocessed in some way to remove the entity references. 
The situation is confused by the fact that some MathML software (notably 
techexplorer and WebEQ) does not process documents using a normal XML 

 
%%page page_392                                                  <<<---3
 
372 
Mathl\/IL, intelligent math markup 
parser but rather recognizes a predefined set of entity names and renders them 
directly. 
The practical solution adopted in the MathML Recommendation is to publish a list of proposed entity names. This gives a corresponding Unicode position, 
if known; if not, a position in the private zone is proposed. The entity lists are 
extensions of the standard ISO 9573 lists and fall into the following groups: 
Mat}: symbols 
ISOAMSA Math Symbols: Arrows 
ISOAMSB Math Symbols: Binary Operators 
ISOAMSC Math Symbols: Delimiters 
ISOAMSN Math Symbols: Negated Relations 
ISOAMSO Math Symbols: Ordinary 
ISOAMSR Math Symbols: Relations 
MJVIALIAS MathML Aliases 
MMEXTRA MathML Additions 
General teclmiczzl symbols 
ISOTECH General Technical 
ISOPUB Publishing 
ISODIA Diacritical Marks 
ISONUM Numeric and Special Graphic 
ISOBOX Box and Line Drawing 
Scripts 
ISOGRK3 Greek Symbols 
ISOMSCR Math Script Font 
ISOMOPF Math Open Face Font 
ISOMFRK Math Fraktur Font 
ISOGRK1 Greek Letters 
ISOGRK2 Monotoniko Greek 
ISOGRK4 Alternative Greek Symbols 
ISOCYR1 Russian Cyrillic 
ISOCYR2 Non-Russian Cyrillic 
Solving the problem of the special math characters and developing widely available fonts to display them are a crucial task for the nascent MathML community, 
and rapid progress can be expected. 
8.2 MathNIL software 
Several vendors and organizations have announced plans to develop implementations of the MathML Recommendation. Currently several of the tools described 
in this book can interpret, render, or produce MathML markup from PHEX input. 
Similarly some level of support is finding its way into the popular browsers. 

 
%%page page_393                                                  <<<---3
 
8.2 MathML software 
We can divide MathML software into five categories: 
1. symbolic math manipulation packages; 
2. equation editors that will either export MathML or use it as their main storage 
format; 
3. Web browsers, and their plug-ins and helper programs; 
4. free-standing conversion programs from formats like BTEX; and 
5. typesetting systems that can process XML documents containing MathML. 
It is beyond the scope of this book to look in any detail at the first category; Mathematica [<->MATHEMATICA] and Maple [<->MAPLE] are planning to export and import MathML for exchange with other software. It is probable they will start to 
work with the xemzmtic part of the language. While this is clearly a very important 
use of MathML, implementations are not yet available. 
The remaining four categories of software are examined in the following sect1ons. 
8.2. 1 Equation editors 
Currently the best example of an interactive equation editor that works with 
MathML is the commercial MathType (version 4 onwards; see [<->MATHTYPE]), 
the full version of the equation editor used in Microsoft Word. In the short term, 
export to MathML is in presentation form, until experience is gained with interfaces 
for writing semantic math. 
MathType is designed to be used as an assistant to other VV1ndows or Macintosh programs (such as Word), so it does not have a native file type. Instead it 
puts a compact representation of the editable form of the math as a comment in 
one of its various output formats. You can also choose one of a variety of translators that determines what format will be placed on the VV1ndows or Mac clipboard 
when you do a “copy." These formats include MathML and various flavors of TEX 
(that is, plain TEX, I51'EX, and .A_MS-P}TEX); the translators that manage this are 
configurable and easily editable. For example, the rule file for P}TEX includes 
paren = "\left( #1 \right)"; // parentheses (both) 
brack = "\left[ #1 \right]"; // brackets (both) 
brace = "\left\{ #1 \right\}"; // braces (both) 
abrack = "\left\lang1e #1 \right\rang1e "; // angle brackets (both) 
bar = "\leftl #1 \right|"; // bars (both) 
dbar = "\left\| #1 \right\|"; // double bars (both) 
floor = "\left\lfloor #1 \right\rfloor "; // floor 
showing how MathType internal constructs (like brack) map quite neatly to TEX. 
373 

 
%%page page_394                                                  <<<---3
 
374 Math1\/IL, intelligent math markup 
Mamni 'D-\sréfiiéiiwcxté'xnii(nex\Iat}é»éichrIa £{2xm|\mi ino,cps"'[E'r55xn 
1 c+1'0o 
¢‘y‘(2’v7k7fl2) = 7 I _t ¢<s)e“ds 
nzcw 
¢<s) = exp [k<1+ 13%] exp wcsn f(e. as) = gxm, k, ,s2> 
- c+ioo 
E‘E xk5%=i- ¢mWm 
/1m=k ----;/-/32 ‘M W ’ 2m‘ Hm 
5 ¢<s) = exp [r<(1 +;82y)1 exp [w<s>1 
E -é 
X“ = _ / _ 2] 
[g V fl 
\reffig{8-1}: MathType equation editor \reffig{8-2}: Example equations typeset by TEX 
The editor screen is shown in action in \reffig{8-1}, where a few of the equations 
from the example text used in Chapter 1 have been entered. VVhen translation to 
AM8-ETEX is requested, the output looks like the following in TEX source. The 
result of typesetting this with TEX is shown in \reffig{8-2}. 
\[ 
\begin{gathered} 
\phi _{v} (\lambda _{v} ,k,\beta “{2} ) = 
\frac{1}{{2\pi i}}\int_{{c - i\infty }}“{{c + i\infty }} 
{\phi (s)e“{{\lambda 5}} ds} \hfill \\ 
\phi (s) = \text{exp }[\kappa(1 + \beta “{2} \gamma )] 
\text{ exp }[\psi (s)] \hfill \\ 
\lambda _{u} = k\left[ {\frac{{ \in - \bar \in }} 
{\xi } ‘ \gamma ’ - \beta “{2} } \right] \hfi1l \\ 
\end{gathered} 
\] 
The reason why the right and left brackets in the first line do not extend to cover 
the f32 is that simple “(" and “)" characters were used rather than the MathType 
“bracketed object" construct. 

 
%%page page_395                                                  <<<---3
 
8.2 MathML software 
VVhen we request a translation to MathML, we get the following (the third 
equation is omitted, to save space): 
<math displaystyle=’true’> 
<semantics> 
<mtable columnalign=’left’> 
<mtr> 
<mtd> 
<msub> 
<mi>&phi;</mi> 
<mi>v</mi> 
</msub> 
<mo stretchy=’false’>(</mo><msub> 
<mi>&lambda;</mi) 
<mi>v</mi) 
</msub> 
<mo>,</mo><mi>k</mi><mo>,</mo><msup> 
<mi>&beta;</mi> 
<mn>2</mn> 
</msup> 
<mo stretchy=’false’>)</mo><mo>=</mo><mfrac> 
<mn>1</mn> 
<mrow> 
<mn>2</mn><mi>&pi;</mi><mi>i</mi> 
</mrow> 
</mfrac> 
<msubsup> 
<mo>&int;</mo) 
<mrow> 
<mi>c</mi><mo>-</mo><mi>i</mi><mo>&infin;</mo> 
</mrow> 
<mrow> 
<mi>c</mi><mo>+</mo><mi>i</mi><mo>&infin;</mo> 
</mrow> 
</msubsup> 
<mrow> 
<mi>&phi;</mi><mo stretchy=’false’>(</mo> 
<mi>s</mi><mo stretchy=’false’>)</mo> 
<msup> 
<mi>e</mi> 
<mrow> 
<mi>&lambda;</mi><mi>s</mi> 
</mrow> 
</msup> 
<mi>d</mi><mi>s</mi> 
</mrow> 
</mtd> 
375 

 
%%page page_396                                                  <<<---3
 
376 MathML, intelligent math markup 
</mtr> 
<mtr> 
<mtd> 
<mi>&phi;</mi><mo stretchy=’false’>(</mo> 
<mi>s</mi><mo stretchy=’false’>)</mo> 
<mo>=</mo><mtext>exp </mtext> 
<mo stretchy=’false’>[</mo><mi>k</mi> 
<mo stretchy=’false’>(</mo><mn>1</mn><mo>+</mo><msup> 
<mi>&beta;</mi> 
<mn>2</mn> 
</msup> 
<mi>&gamma;</mi> 
<mo stretchy=’false’>)</mo> 
<mo stretchy=’false’>]</mo><mtext> exp </mtext> 
<mo stretchy=’false’>[</mo><mi>&psi;</mi> 
<mo stretchy=’false’>(</mo><mi>s</mi> 
<mo stretchy=’false’>)</mo><mo stretchy=’false’>]</mo> 
</mtd> 
</mtr> 
</mtable> 
</semantics> 
</math> 
It is interesting to compare the translation rules with those for ETEX, if we 
take the same constructs as before, the MathML rules look like the following: 
paren "<mrow><mo>(</mo>$+$n#1$-$n<mo>)</mo></mrow>$n"; 
brack = "<mrow><mo>[</mo> #1 <mo>]</mo></mrow>"; 
brace = "<mrow><mo>{</mo> #1 <mo>}</mo></mrow>"; 
abrack = "<mrow><mo>&langle;</mo> #1 <mo>&rangle;</mo></mrow>"; 
bar = "<mrow><mo>I</mo> #1 <mo>|</mo></mrow>"; 
floor = "<mrow><mo>&lfloor;</mo> #1 <mo>&rfloor;</mo></mrow>"; 
W3C’s experimental Web browser Amaya can also be used to create MathML. 
Although this is not intended to be a serious production tool, it shows that simple 
authoring, editing, and display are not too hard to set up. 
8.2.2 Web browser support for MathML 
The Amaya browser can be used to display MathML markup, simply by embedding <math> elements (and their children) in plain HTML. This is shown in \reffig{8-3} (and in \reffig{6-5} on page 276). The support for MathML is incomplete, 
not the least problem being the lack of suitable symbol fonts (see Section 8.1.1), 
but, nonetheless, it remains a useful technology preview. 
The techexplorer browser plug-in and the WebEQ Java helper (described 
in Chapter 5) also display MathML embedded in HTML. The markup required 

 
%%page page_397                                                  <<<---3
 
8.2 MathML software 377 
Q =4><s>a==iexp:£:1 i'fin:B":2¥}le¥éiE#u{s)1ri 
t-«- . first -L’ :-E-€[:i_“fi2 E . ~‘y_'.fl : :_ __ ,_ p 
at J m ago i _“MathMLftest»_ :;.- e L ; 
- L " L EHERAT0Rt2unt§m=£a'mnyaV1;3a 1;_ 
\reffig{8-3}: Amaya Web browser showing MathML 
is slightly different, as the former uses the <embed> tag and the latter uses the 
<applet> tag. Thus if we take a fragment of BTEX from our test file 
The Vavilov parameters are simply related to the Landau parameter by 
$\lambda_L = \lambda_v/\kappa - \ln\kappa $. It can be shown that as 
$\kappa \rightarrow 0$, the distribution of the variable $\lambda_L$ 
approaches that of Landau. 
and convert it to a mixture of HTML and MathML for rendering by techexplorer, 
the HTML looks like this: 
The Vavilov parameters are simply related to the Landau parameter by 
<embed src="htmlmathml71.mml" width=``215'' height=``46'' align="middle“>. 
It can be shown that as 
<embed src="htmlmathml72.mml" width=``89'' height=``37'' align=``middle''>, 
the distribution of the variable 

 
%%page page_398                                                  <<<---3
 
378 
MathML, intelligent math markup 
<embed src="htmlmathml73.mml" width=``50'' height=``42'' align=``middle''> 
approaches that of Landau. 
Each math fragment is contained in a separate file, the width and height of which 
when rendered have to be specified explicitly. We have already considered these 
issues in Section 5.3.1 on page 235. 
If we want to load the Java code of WebEQ to process the math, we would 
instead write: 
The Vavilov parameters are simply related to the Landau parameter by 
<applet code="webeq.Main" width=``215'' height=``46'' align=``middle''> 
<param name=``eq'' value=" 
<math><msub><mi>&lambda;</mi><mrow><mi>L</mi></mrow></msub> <mo>=</mo> 
<msub><mi>&lambda;</mi><mrow><mi>V</mi></mrow></msub> 
<mo>/</mo><mi>&kappa;</mi> 
<mo>-</mo> <mi>l</mi><mi>n</mi> <mi>&kappa;</mi> 
</math>"> 
<param name=``color'' value="#COCOC0"> 
<param name=``parser'' value=``mathml''> 
</applet>. 
It can be shown that as 
<applet code="webeq.Main" width=``89'' height=``37'' align=``middle''> 
<param name=``eq'' value=" 
<math><mi>&kappa;</mi><mo>&rarr;</mo> <mn>0</mn> 
</math>" > 
<param name=``color'' value="#COCOC0"> 
<param name=``parser'' value=``mathml''> 
</applet>, the distribution of the variable 
<applet code="webeq.Main" width=``50'' height=``42'' align=``middle''> 
<param name=``eq'' value=" 
<math><msub><mi>&lambda;</mi><mrow><mi>L</mi></mrow></msub> 
</math> "> 
<param name=``color'' value="#COCOC0"> 
<param name=``parser'' value="mathml“> 
</applet> 
approaches that of Landau. 
We embed the MathML code in the HTML inside an <app1et> element but again 
we have to supply the width and height. This is an annoyance for translation systems,1 and we must hope that future systems will incorporate size negotiation between browsers and special purpose helpers or plug-ins. 
It looks unlikely that the current generation of mainstream Web browsers 
(Netscape, Internet Explorer, and so on) will have direct support for MathML before the year 2000 at the earliest. Until then, we will have to work with plug-in 
applications. 
1These translations were created using experimental configuration files for TEX4-ht. 

 
%%page page_399                                                  <<<---3
 
8.2 MathNIL software 
379 
One possible way in which future browsers might support MathML is via transformation to a standardized vector graphics markup language (the equivalent of 
PostScript) for the Web. More details on developments of this language can be 
found at ['-> WSCGR]. 
8.2.3 Converting I5"lEX to MathML 
Conversions from ETEX are likely to be to presentational form in the shortand 
medium-term future, simply because ETEX and plain TEX are not designed to express math semantics. Even managing presentation markup is not straightforward. 
ETEX to XML converters in general can be based on a number of philosophies 
(see Rahtz (1995) for more discussion): 
1. Free-standing programs are written in a conventional language like C, Perl, or 
Java, which attempt to parse TEX markup. We have already seen examples of 
these in Chapter 1 (TtH) and Chapter 3 (ETEXZHTML). 
2. Systems based on TEX macros add to the DVI file information that a later 
program can extract. We have seen an extensive exposition of such an approach 
in Chapter 4 (TEX4ht). 
3. Systems are based on an extension of TEX, where the innards of the program 
are replaced so that it emits XML. 
The majority of TEX math is fairly easy to translate, since most constructs map 
to MathML presentation markup easily. Thus a fraction in ETEX, with a simple 
non-Latin character, translates from 
\frac{\xi}{E_{x}} 
t0: 
<mfrac> 
<mrow><mi>&xi;</mi></mrow> 
<mrow> 
<msub> 
<mi>E</mi> 
<mrow><mi>x</mi></mrow> 
</msub> 
</mrow> 
</mfrac> 
Unfortunately we have two problems to overcome, both of which are best 
solved by different methods. The first is the obvious one that ETEX authors writing 

%==========400==========<<<---1
%==========400==========<<<---2
 
%%page page_400                                                  <<<---3
 
380 
MathML, intelligent math markup 
non-trivial math very often make extensive use of macros to set up private markup 
schemes, something like 
\newcommand{\htheta}{\hat\theta} 
This is fairly easy for programs like TtH and ETEXZHTML to follow. However, 
consider a plain TEX delimited macro with conditionals in the expansion, such as 
\def\Fraction:#1/#2:{\ifmmode\frac{#1}{#2}\else #1 divided by #2\fi} 
Used with something like \Fraction : 36/72: this will almost certainly break most 
converters, unless they use the TEX parser. Of course, most ETEX users will not 
write this sort of macro; the ETEX \newcomma.nd does not support it, but the full 
power of TEX does get called up by advanced users. 
The second problem is that we have to deal with ETEX like the following: 
\[ (a+b)‘{2} \] 
This should translate to the MathML code 
<mrow> 
<msup> 
<mrow><mo>(</mo><mi>a</mi><mo>+</mo><mi>b</mi><mo>)</mo></mrow> 
<mrow><mn>2</mn></mrow> 
</msup> 
</mrow> 
As we already noted, superscripted expressions in MathML need to be marked explicitly. In BTEX, approaches based on simple redefinition of macros run into difficulties, since the macro programmer does not have straightforward access to the 
start of (a+b) . The programmer in C or Perl can backtrack quite easily and find the 
most plausible expression to include inside the MathML <msup>. VVhile there are 
not many constructs like this to deal with, they arise very often, and a translator has 
to have a reliable solution. One answer is for ETEX users to adopt an extended input syntax so that there is no ambiguity; thus the previous example could be marked 
up as 
\[ \Sup{(a+b)}{2} \] 
making translation to MathML trivial. This approach is used by WebEQ (see the 
following). Another answer is to rewrite the entire math-handling part of ETEX at 
a low level and identify the necessary subexpressions directly. There are currently 
no implementations of this idea. 
At the time of writing there are three preliminary implementations of BTEX 
to XML/MathML converters (described later), while techexplorer is likely to add 
the facility at some point. It should be clear from the following sections that we 

 
%%page page_401                                                  <<<---3
 
8.2 MathML software 
381 
~13‘ 
We Wizard 
$eta}*{hnfl:.r}$ 
maths: 
<msubsup> 
<mi>E< ,-’mi> 
<mi> 5.heta;<,-’mi> 
<mi>s.inft‘y:<,-’mi> 
< ,-’msubsup> 
!math> 
\reffig{8-4}: WebEQ “wizard" in action translating ETEX to MathML 
have the necessary technology to translate ETEX to MathML, but we can not yet 
recommend a reliable, tested setup that can process arbitrary ETEX documents. 
8.2.3.1 IIKIEX to MathML with WebEQ 
The first converter was written as part of the WebEQ software. This “wizard" is 
shown in action in \reffig{8-4}; it provides an interactive mode in which TEX code 
can be typed in one window, and the MathML equivalent is shown in another. 
Unfortunately, the translator does only TEX math (not a complete ETEX article) 
and cannot follow the more complex macros we described earlier. A system based on 
ETEXZHTML by Ross Moore ['-> LZHMML] does provide a translation for an entire 
ETFX document. It uses the existing HIEXZHTML system to translate normal text 
to HTML. It then passes bits it identifies as math to WebEQ for conversion to 
MathML, which are then embedded in the HTML. 

 
%%page page_402                                                  <<<---3
 
382 
MathML, intelligent math markup 
The disadvantage is that the WebEQ VV1zard does not fully support ETEX 
markup; it supports only a variant called WebTEX. This means that documents 
need extra preparation. There is more discussion of WebEQ in Section 5.2. 
8.2.3.2 Using TEX4ht to convert IIHEX to XNIL/MathML 
In Chapter 4 we looked at how TEX4ht can be extensively configured when writing HTML. Since HTML is simply an application of SGML, it should be obvious 
that TEX4ht can equally well be set up to generate an XML file. The tags that are 
generated for a particular ETEX command can be changed in configuration files to 
whatever you like. In Appendix B.2 we provide the full details of how to strip the 
TEX4ht conversion back to basics and then build up a new system based on simple 
XML tags. 
A full translation to MathNIL can, therefore, be configured using TEX4ht. This 
program solves the problem of “back-tracking" in superscripts and subscripts by 
using a two-stage process. Clues are embedded in the DVI file as TEX \special 
commands, and then the postprocessor analyzes the entire math expression to tiy 
and locate the start of the subscripted part, for example. 
Let us look at a complete example and generate a full XML file. The source file, 
as follows, is a subset of the example file given in Appendix A.1: 
\documentclass{article} 
\usepackage[x2ldemo]{tex4ht} 
\title{Simulation of Energy Loss Straggling} 
\author{Maria Physicist} 
\begin{document} 
\maketitle 
\section{Landau theory}\label{sec:phys332-1} 
The Landau probability distribution may be written in 
terms of the universal Landau function \cite{bib-LAND}. 
\subsection{Restrictions} 
The Landau formalism makes two restrictive assumptions: 
\begin{enumerate} 
\item The typical energy loss is small. 
\item The typical energy loss in the absorber should be 
large (see section \ref{urban}). 
\end{enumerate} 
\section{Urb\’an mode1}\label{urban} 
The following values are obtained: 
\begin{tabular}{llcrr} 
16 x 16 & x 2000 x 29.63\\ 
100 x 27.59 x x 100 x 32.00 
\end{tabu1ar} 
\begin{thebibliography}{10} 
\bibitem{bib-LAND} L.Landau. On the Energy Loss of Fast Particles by 
Ionisation. Originally published in \emph{J. Phys.}, 8:201, 1944. 
\end{thebibliography} 
\end{document} 
The configuration file is as follows: 
I \Configure{html}{xml} 
2 \Preamble{html,0.0,ref-,fonts} 

 
%%page page_403                                                  <<<---3
 
8.2 MathML software 
\Configure{HTML} {\IgnorePar\Tg<?xml version="1.0"?> 
\Tg<document>} 
{\Tg</document>} 
\Configure{section} 
{\EndP \IgnorePar\par \GetLabe1 \Tg<section \PutLabe1>} 
{\EndP \IgnorePar \Tg</section>} 
{\Tg<sti1:1e>}{\Tg</stitle>} 
\Configure{subsection} 
{\EndP \IgnorePar\par \GetLabe1 \Tg<subsection \PutLabe1>} 
{\EndP \IgnorePar \Tg</subsection>} 
{\Tg<stitle>}{\Tg</stitle>} 
\Configure{likesection} 
{\EndP \IgnorePar\par\GetLabe1 \Tg<section \PutLabe1 c1ass=``star''>} 
{\EndP \Tg</section>} 
{\Tg<s1:it1e>}{\Tg</stitle>} 
\ConfigureList{thebib1iography} 
{\EndP \GetLabe1 \Tg<bibliography \PutLabe1> 
\def\EndItem{\def\EndItem{\EndP \Tg</b1bitem>}}} 
{\EndItem \Tg</bib1iography>} 
{\EndItem \De1eteMark} 
{\Tg<bibitem id="\AnchorLabe1">\par} 
\ConfigureList{enumerate} 
{\EndP \GetLabe1 \Tg<1a1ist \PutLabe1 c1ass=“enumerate"> 
\def\EndItem{\def\EndItem{\EndP\Tg</item>}}} {\EndItem \Tg</1a1ist>} 
{\EndItem \De1eteMark} {\Tg<item>\par} 
\Configure{tabu1ar} 
{\Tg<tabu1ar preamb1e="\C1r">} {\Tg</tabu1ar>} 
{\Tg<row>}{\Tg</row>} {\Tg<ce11 \Hnewline>}{\Tg</ce11>} 
\Configure{maketitle} {H} {\Tg<title>}{\Tg</1:it1e>} 
\Configure{thanks author date and} 
{}{} {\Tg<author>}{\Tg</author>} {\Tg<date>}{\Tg</date>} {} {} 
\Configure{emph}{\Tg<emph>}{\Tg</emph>} 
\Config-ure{label}{id="#1"}{\Tg<pagelabel id="#1"/>} 
\Configure{pageref}{\Tg<pageref refid=“#1"/>} 
\Configure{ref}{\Tg<ref refid="#1"/>} 
\Configure{cite}{\Tg<cite refid="#1"/>} 
\begin{document} 
\EndPreamb1e 
\Configure{Htm1Par} {\EndP\Tg<P>} {\EndP\Tg<P>} {\Tg</P>} {\Tg</P>} 
oo'1‘-SICCESE-='5ooo\.o«u..s.. 
3 
N 
o 
N 
N 
N 
N 
94 
N 
.5 
-hgUJUJvJUJUJvJvJUJvJvJI\l|\:I\lI\lI\: 
... ~O®\lo\Vx.§vJI\l~-©*O®\lO\\I\ 
The first line of the configuration file requests an extension name xml, instead of 
html, for the output file. 
The package option fonts, in line 2 of the configuration file, requests hooks 
for the font commands of ETEX. On the other hand, the package option refasks 
the cross-reference commands of ETEX to keep their labels, instead of exchanging 
them for section and page numbers, and to use these labels within attributes of 
hypertext tags. In addition, this option adds a pair of commands 
\GetLabel 
\PutLabel 
for extracting the label of the next \label command and for inserting that label 
into the output file, respectively. 
The arguments id="#1" and \Tg{pagelabel id="#1"/} ofthe “label" hook 
in line 35 show how \PutLabel and \label, respectively, should set their labels. 
The \Clr command in line 29 produces the argument of the tabular environment that is preserved in the XML output. 
383 

 
%%page page_404                                                  <<<---3
 
384 MathML, intelligent math markup 
The \Conf '1gureList commands in lines 18 and 24 employ the following command at the end of their third argument: 
\De1eteMark 
It removes the native marks created at the start of the items by the corresponding 
list environments. 
\AnchorLabe1 
This command (used on line 23) then inserts these marks back into the output. 
When all this is run through TEX4ht, it produces the XML output as follows: 
<?xml version="1.0"?> 
<document> 
<title>Simulation of Energy Loss Straggling</title> 
<author>Maria Physicist</author) 
<date>November 9, 1998</date> 
<section id="sec:phys332-1"> 
<stitle>Landau theory</stitle> 
<p> The Landau probability distribution may be written in 
terms of the universal Landau function <cite refid="bib-LAND“/> . 
</p> 
<subsection> 
<stitle>Restrictions</stitle> 
<p>The Landau formalism makes two restrictive assumptions:</p> 
<lalist c1ass=``enumerate''> 
<item><p>The typical energy loss is sma11.</p></item) 
<item><p>The typical energy loss in the absorber should be 
large (see section <ref ref1d=``urban''/> )~</p></item) 
</lalist> 
</subsection) 
</section> 
<section id=``urban''> 
<stitle>Urb&aacute;n model</stitle> 
<p> The following values are obtained: 
</p><p> <tabular preamble=``llcrr''><row> 
<cell>16</cell><cell>16</cell><cel1></cell><cell>2000</cell> 
<cell>29.63</cell></row><row><cell>100</cell) 
<ce1l>27.59</cell><cell></cell><cell>100</cell><cell>32.00</cell) 
</row></tabular></p> 
</section> 
<section c1ass=``star''> 
<stitle>References</stitle> 
<bib1iography > 
<bibitem id=“bib-LAND“> 
<p>L.Landau. On the Energy Loss of Fast Particles by Ionisation. 
Originally published in <emph>J. Phys.</emph>, 8:201, 1944.</p> 
</bibitem> 
</bibliography) 
</section> 
</document) 
By its nature MathML deals with low-level structural properties that are not 
always handled by ET]-3X macros but rather are hardwired into TEX itself. This 
means that MathML is a natural candidate for exploiting the features of TEX4ht as 
described in the previous section and in Appendix B.2. 

 
%%page page_405                                                  <<<---3
 
8.2 MathML software 
385 
A few lines of simple ETEX is sufficiently structured to serve as an example for 
illustrating many of the features of a ETEX to MathML conversion: 
\begin-ieqnarray} 
\bar u &=& \int_{I}‘E g(x)dx \nonumber \\ 
E &=& \frac{I}{1-\mathrm{max}}\label-{xx} 
\end{eqnarray} 
VVhen typeset, this comes out as: 
E 
12 = /g(x)dx 
I I 
E = 1--- (1) 
-max 
Now we need to construct a new TEX4ht configuration file to deal with the math 
and to generate the right MathML: 
o~ooo\-o«v.4>w~... 
0 C E 5 E : 
~~~~~~-.....v..>wN._.o~ooo\u 
\Preamb1e{htm1,0.0,ref-,fonts,math} 
\Configure{$$}{\DviMath}{\EndDviMath}{} 
\Configure{eqnarray} 
{\GetLabe1 \Tg<eqnarray \FutLabe1>} 
{\GetLabe1 \Tg</eqnarray>} 
{\GetLabe1 \Tg<subeqn \PutLabe1>\Tg<math>} 
{\Tg</math>\Tg</subeqn>} 
{\ifnum \Co1=4 \Tg<mtext>\PauseMathC1ass \fi} 
{\ifnum \Co1=4 \EndPauseMathC1ass\Tg</mtext>\fi} 
\Configure{label}{id="#1" }{\Tg<pagelabel id="#1"/>} 
\Conf igure{SUBSUP} 
{\Send{BACK}{<msubsup>}\Tg<mrow>} 
{\Tg</mrow>\Tg<mrow>} 
{\Tg</mrow>\TG</msubsup>} 
\Configure{frac} 
{\Tg<mfrac>\Tg<mrow>} {\Tg</mrow>\HCode{<!--}} 
{\HCode{-->}\Tg<mrow>}{\Tg</mrow>\Tg</mfrac>} 
\Configure{mathrm}{\Tg<mi>\PauseMathC1ass} 
{\EndPauseMathC1ass\Tg</mi>} 
\Configure{accent}\=\bar{{}{}} 
{}{\Tg<mover accent=“true">#2\Tg<mo>\HCode{&0verBar;}\Tg</mo>\Tg</mover>} 
\Configure{MathC1ass}{0}{*}{<mi>}{</mi>}{} 
\Configure{MathC1ass}{1}{*}{<mo>}{</mo>}{} 
\Coufigure{MathC1ass}{2}{*}{<mo>}{</mo>}{} 
\Configure{MathClass}{3}{*}{<mo>}{</mo>}{} 
\Configure{MathC1ass}{4}{*}{<mrow><mo>}{</mo>}{} 
\Configure{MathC1ass}{5}{*}{<mo>}{</mo></mrow>}{} 
\Configure{MathC1ass}{6}{*}{<mo>}{</mo>}{} 
\Configure{MathC1ass}{7}{*}{<mn>}{</mn>}{0123456789} 
\begin{document} 
\EndPreamb1e 
With the exceptions of <eqnarray>, </eqnarray>, <subeqn>, and </subeqn>, 
all the tags in the configuration file are related to MathML. 
ETEX invokes the math environment $$fi)7mulzz$$ as part of the eqnarray environment. Line 2 in the configuration file has the task of setting the conditions for 
processing this material. 

 
%%page page_406                                                  <<<---3
 
386 MathML, intelligent math markup 
VVhenever TEX4ht processes a table, it holds in its “variables" \Row and \Col 
the row and column numbers of the current cell. \Co1 is used, in lines 8 and 9 of 
the configuration file, to allow for the equation numbers. 
<eqnarray> 
<subeqn><math> 
<mover accent=``true''> 
<mi>u</mi> 
<mo>&0verBar;</mo) 
</mover) 
<mo>=</mo> 
<msubsup> 
<mo>&int;</mo> 
<mrow><mi>I</mi></mrow> 
<mrow><mi>E</mi></mrow> 
</msubsup> 
<mi>g</mi><mrow><mo>(</mo><mi>x</mi><mo>)</mo></mrow> 
<mi>d</mi><mi>x</mi) 
<mtext></mtext> 
</math)</subeqn><subeqn><math> 
</math></subeqn><subeqn id=``xx''><math> 
<mi>E</mi) <mo>=</mo> 
<mfrac> 
<mrow><mi>I</mi></mrow> 
<mrow><mn>1</mn> <mo>-</mo) (mi) max </mi> </mrow> 
</mfrac> 
<mtext>(1)</mtext> 
</math></subeqn><subeqn><math> 
</math></subeqn> 
</eqnarray> 
8.2.3.3 I1}'IEX to MathML with Omega 
A robust approach is that taken by the third converter, using an extension of the 
TEX program called Omega by John Plaice and Yannis Haralambous ['-> OMEGA]. 
Omega’s main extensions to TEX are that it uses Unicode internally and a set of 
“translation processes" that are applied at input and output. To support MathML, 
Omega also outputs XML directly during the TEX run, under control of user 
macros. Thus the ETEX \frac command is redefined with 
\renewcommand{\frac} [2] {'Z. 
\SGMLstarttag-[mfrac}{#1]-{#2)-\SGMLendt ag-[mfrac]-} 
making use of some new Omega primitives:2 
\SGMLstarttag{tag name} 
\SGMLendtag{tag name} 
\SGMLentity{entity name} 
There is a new primitive to start generation of XML output instead of DVI. 
2 These names may change in time; there are also new primitives to specify attributes and to generate 
some special characters. The Omega documentation gives current details. 

 
%%page page_407                                                  <<<---3
 
8.2 MathML software 
387 
The considerable advantage of this approach is that the knowledge that TEX 
has about the inner workings of math mode can be harnessed; the program doex 
have some ideas about where, for example, superscripted expressions start, but does 
not make it available to macros. By changing the TEX engine itself, we can get at 
the details we need, and also be quite sure that there are no problems with user 
macros. The downside is that TEX does not in fact have quite as much knowledge 
as MathML demands (see Knuth (1986, p.129) for a discussion of what superscripts 
are attached to). It may be more sensible to simply have TEX place all the information it doex have in the DVI file in some form, and then use a scheme like TEX4ht’s 
for post-processing the DVI and writing XML. 
Like TEX4ht, Omega supports editable font encoding files that map characters 
in TEX math fonts to MathML entities and distinguish types of math operators. 
8.2.4 Typesetting MathML 
When we move away from the Web and the problems of how to generate MathML, 
and we want simply to print our new math documents, we are on more familiar 
territory. TEX is too heavyweight and monolithic to use in a dynamic environment 
like a browser, but as a batch formatting engine, it is excellent. We should easily be 
able to convert our XML markup to TEX and get the usual results. 
There are basically three approaches to using TEX as a formatting engine for 
XML, as we have already seen in Chapter 7: 
1. We can write a special-purpose program, using something like the SAX or DOM 
models, and convert <mfrac> elements directly to ETEX \frac, for example. 
2. We can use the XSL or DSSSL style languages to map MathML elements onto 
math formatting objects and then use TEX as a back-end for the style engine. 
3. We can write a TEX macro package to interpret and typeset XML directly. 
The first approach has the advantage of giving to the programmer good control 
over what comes out; there is no need to specify layout very precisely, since we 
can take advantage of existing ETEX packages (like AMS-BTEX) to do the hard 
work. The downside is that it requires a new program for each XML DTD, and 
the translation is specific to TEX; translation to another math formatter like that in 
FrameMaker would require an entirely separate XML to MIF converter. 
The second approach, going via the abstract Style sheet, is potentially far superior. It removes the dependence on a particular formatting engine and leaves the 
hard work of mapping math flow objects to TEX to be done just once, in a style 
sheet language implementation’s back-end. Unfortunately there are also some important problems: 
o The current version of XSL does not yet include any math formatting objects, 
and those in DSSSL have not been tested enough that we can sure they are up 
to the task. 

 
%%page page_408                                                  <<<---3
 
388 MathML, intelligent math markup 
f(E ,6s> = §<t>.<A_..k. 32> 
<I>.<A., k. /32) = 551 :i§:<t><s>e“ds 
<1><s> = eXp[k(1 + /323/)]eXp[(lI(s)] 
A, =k[€-;3 - 2/ ~32] 
\reffig{8-5}: MathML sample processed by the Jade DSSSL engine and TEX 
o If the results are not perfect, it is almost impossible to tweak the machinegenerated, low-level TEX code. 
0 We may get different results with different style engines, if they have errors in 
their back-end modules. 
0 All the work has to be done in the style sheet; we can no longer rely on the 
crutch of mature packages like AMS-ETEX. 
The fairly mature support for math in DSSSL is implemented in James Clark’s 
Jade and supported by the jadetex macros [G->]ADETEX]. David Carlisle has written a DSSSL style sheet that deals with practically all presentational and semantic 
MathML[‘-> DSSSLMML]. Most of the mapping is fairly easy to implement as that 
for fractions shows: 
(element mfrac 
(make fraction 
(let ((nl (children(current-node)))) 
(sosofo-append 
(make math-sequence 
label: ’numerator 
(process-node-list (node-list-first nl))) 
(make math-sequence 
label: ’denominator 
(process-node-list (node-list-rest nl))))))) 
The result of processing our earlier example through Jade and then jadetex is 
shown in \reffig{8-5}. There remain, however, some areas of MathML that are not 
dealt with properly (such as full support of stretch operators) due to limitations in 
DSSSL’s math flow objects. 
An interesting, intermediate, working method is TeXML, proposed and implemented by IBM’s Doug Lovell (see ['-->TEXML]). This is an XML representation of 
TEX syntax, and the typesetting of a document works in two stages: 
1. The transformation facilities of a language like XSL are used to convert the 
source XML document to a simpler XML representation using only TeXML 
elements. 

 
%%page page_409                                                  <<<---3
 
8.2 MathML software 
389 
2. The TeXML document is transformed to TEX using a simple program, and 
TEX is rtm on the result. 
The advantage of this approach is that any XML-to-XML transformation system 
(of which there are many) can be used without any need to worry about TEX backslashes and braces and special output. The user does, however, have access to more 
or less any TEX facility, like macros; and the translation to real TEX is straightforward. 
Summary 
After a brief explanation of why we think that MathML is an important development to exchange, store, and transform mathematics information on the Web, 
we discussed the problems of special characters and presented an overview of 
MathML-aware software packages. We looked at tools that generate MathMLfrom 
ETEX with some detailed examples of how TEX4ht can turn ETEX source files into 
XML and MathML. Finally we explained how ETEX can be used as a printing backend for MathML. 
It is evident that a long road remains to be traveled before the full expressive 
power of ETEX will be browsable on the Web in the form of MathML. Yet there 
is hope that in the medium-term future, users will be able to transform their ETEX 
source files into MathML for input into algebraic manipulation programs for viewing on the Web or for inclusion into one of the many document processing systems. 
At the same time it will be possible to profit from the typographic excellence of the 
TEX engine for rendering Web documents that contain a lot of math. The next millennium will probably see a perfect symbiosis between PHEX and XML/MathML, 
by allowing users to switch easily between the presentation most suited to their 
needs at any given moment. Thus you will not have to choose between MathML 
and ETEX for marking up your scientific texts or data; you will use the format you 
are most comfortable with and still profit from the benefit of both at no extra cost. 

%==========410==========<<<---2
 
%%page page_410                                                  <<<---3
 
Example files 
A.1 An example ETEX file and its translation to XML 
We have tried throughout this book to give as many examples as we can from the 
same document (a simple physics paper from CERN) so that readers can make sensible comparisons. In this section we give a partial listing of the ETEX source and a 
translation into XML prepared using TEX4ht (see Section 8.2 .3 .2 and Appendix B.2) 
with its corresponding DTD. 
The full version of the files can be found on CTAN (Comprehensive TEX 
Archive Network) in the directory info/lwc. 
A.1.l The ETEX source 
\documentclass{article} 
\usepackage{graphicx} 
\usepackage{url} 
\title{Simulation of Energy Loss Straggling} 
\author{Maria Physicist} 
\newcommand{\Emax}{\ensuremath{E_{\mathrm{max}}}} 
\newcommand{\GEANT}{\texttt{GEANT}} 
\begin{document} 
\maketitle 
~ooo\uo«v..s«.a~.\section{Introduction} 
13 Due to the statistical nature of ionisation energy loss, large 
14 fluctuations can occur in the amount of energy deposited by a particle 
traversing an absorber element. Continuous processes such as multiple 
scattering and energy loss play a relevant role in the longitudinal 
and lateral development of electromagnetic and hadronic 
18 showers, and in the case of sampling calorimeters the 
19 measured resolution can be significantly affected by such fluctuations 
u 9 G 

 
%%page page_411                                                  <<<---3
 
392 
Example files 
in their active layers. The description of ionisation fluctuations is 
characterised by the significance parameter $\kappa$, which is 
proportional to the ratio of mean energy loss to the maximum allowed 
energy transfer in a single collision with an atomic electron 
\[ 
\kappa =\frac{\xi}{\Emax} 
\] 
\Emax{} 
is the maximum transferable energy in a single collision with 
an atomic electron. 
\section{Vavilov theory} 
\label{vavref} 
Vavilov\cite{bib-VAVI} derived a more accurate straggling distribution 
by introducing the kinematic limit on the maximum transferable energy 
in a single collision, rather than using $ \Emax = \infty $. 
Now we can write\cite{bib-SCH1}: 
\begin{eqnarrayt} 
f \left ( \epsilon, \delta s \right ) & = & \frac{1}{\xi} \phi_{V} 
\left ( \lambda_{v}, \kappa, \beta*{2} \right ) 
\end{eqnarray#} 
where 
\begin{eqnarray#} 
\phi_{v} \left ( \lambda_{v}, \kappa, \beta‘{2} \right ) & = & 
\frac{1}{2 \pi i} \int‘{c+i\infty}_{c-i\infty}\phi \left( s \right ) 
e‘{\lambda s} da \hspace{2cm} c \geq 0 \\ 
\Phi \left ( s \right ) & = & 
\exp \left [ \kappa ( 1 + \beta*{2}\gamma ) \right ] 
~ \exp \left [ \psi \left ( s \right ) \right ], \\ 
\psi \left ( s \right ) & = & s \ln \kappa + ( s + \beta‘{2} \kappa ) 
\left [ \ln (s/\kappa) + E_{1} (s/\kappa) \right ] - \kappa e‘{-S/\kappa}, 
\end{eqnarray*} 
and 
\begin{eqnarray#} 
E_{1}(z) & = & \int‘{\infty}_{z} t‘{-1} e‘{~t} dt 
\mbox{\hspace{1cm} (the exponential integral)} \\ 
\lambda_v & = & \kappa \left [ \frac{\epsilon - \bar{\epsilon}}{\xi} 
- \gamma’ - \beta*2 \right] 
\end{eqnarray#} 
The Vavilov parameters are simply related to the Landau parameter by 
$\lambda_L = \lambda_v/\kaPPh - \ln\kappa $. It can be shown that as 
$\kappa \rightarrow 0$, the distribution of the variable $\lambda_L$ 
approaches that of Landau. For $\kappa \leq 0.01$ the two 
distributions are already practically identical. Contrary to what many 
textbooks report, the Vavilov distribution \emph{does not} approximate 
the Landau distribution for small $\kappa$, but rather the 
distribution of $\lambda_L$ defined above tends to the distribution of 
the true $\lambda$ from the Landau density function. Thus the routine 
\texttt{GVAVIV} samples the variable $\lambda_L$ rather than 
$\lambda_v$. For $\kappa \geq 10$ the Vavilov distribution tends to a 
Gaussian distribution (see next section). 
\begin{thebibliography}{10} 
\bibitem{bib-LAND} 
L.Landau. 
\newblock On the Energy Loss of Fast Particles by Ionisation. 
\newblock Originally published in \emph{J. Phys.}, 8:201, 1944. 
\newblock Reprinted in D.ter Haar, Editor, \emph{L.D.Landau, Collected 
papers}, page 417. Pergamon Press, Oxford, 1965. 

 
%%page page_412                                                  <<<---3
 
A.1 An example BIEX file and its translation to XML 
85 
86 
87 
88 
89 
90 
92 
93 
94 
95 
96 
97 
98 
100 
101 
102 
103 
105 
106 
I07 
108 
109 
A.1.2 
5 e w a Q m s W N n 
393 
\bibitem{bib-SCH1} 
B.Schorr. 
\newblock Programs for the Landau and the Vavilov distributions and the 
corresponding random numbers. 
\newblock \emph{Comp. Phys. Comm.}, 7:216, 1974. 
\bibitem{bib-SELT} 
S.M.Se1tzer and M.J.Berger. 
\newblock Energy loss straggling of protons and mesons. 
\newblock In \emph{Studies in Penetration of Charged Particles in 
Matter}, Nuclear Science Series~39, Nat. Academy of Sciences, 
Washington DC, 1964. 
\bibitem{bib-TALM} 
R.Ta1man. 
\newblock On the statistics of particle identification using ionization. 
\newblock \emph{Nucl. Inst. Meth.}, 1592189, 1979. 
\bibitem{bib-VAVI} 
P.V.Vavilov. 
\newblock Ionisation losses of high energy heavy particles. 
\newblock \emph{Soviet Physics JETP}, 5:749, 1957. 
\end{thebibliography} 
\end{document} 
I4\'I}§X converted to XNIL 
<?xml version="1.0"?> 
<!DDCTYPE document SYSTEM "latexexa.dtd" []> 
<document> 
<frontmatter> 
<title>Simulation of Energy Loss Straggling</title> 
<author>Maria Physicist</author> 
<date>January 14, 1999</date) 
</frontmatter> 
<bodymatter> 
<section id=``intro''> <stitle>Introduction</stitle> 
<par>Due to the statistical nature of ionisation energy loss, large 
fluctuations can occur in the amount of energy deposited by a particle 
traversing an absorber element. Continuous processes such as multiple 
scattering and energy loss play a relevant role in the longitudinal 
and lateral development of electromagnetic and hadronic showers, and 
in the case of sampling calorimeters the measured resolution can be 
significantly affected by such fluctuations in their active 
layers. The description of ionisation fluctuations is characterised by 
the significance parameter <inlinemath> 
<math><mi>&kappa;</mi></math></inlinemath>, which is proportional to 
the ratio of mean energy loss to the maximum allowed energy transfer 
in a single collision with an atomic electron 
<displaymath><math><mrow> 
<mi>&kappa;</mi><mo>=</mo) <mfrac> <mrow> 
<mi>&xi;</mi></mrow><mrow><msub><mi>E</mi><mrow><mi>max </mi> </mrow> 
</msub> </Mrow> </mfrac> </mrow></math></disp1aymath> 
<inlinemath><math><msub><mi>E</mi><mrow><mi>max </mi> </mrow> </msub> 
</math></inlinemath> is the maximum transferable energy in a single 
collision with an atomic electron. 
</section) 

 
%%page page_413                                                  <<<---3
 
394 
Example files 
<section id=``vavref''><stitle>Vavilov theory</stitle> 
<par>Vavilov<cite refid=``bib-VAVI''/> derived a more accurate 
straggling distribution by introducing the kinematic limit on the 
maximum transferable energy in a single collision, rather than using 
<inlinemath> zmath><msub><mi>E</mi><mrow><mi>max </mi> </mrow> </msub> 
<mo>=</mo><mi>&infin;</mi></math></inlinemath>. Now we can write<cite 
refid=``bib-SCH1''/>: <eqnarray><subeqn><math><mi>f</mi> <mfenced 
open=’(’ 
close=’)’><mi>&epsi;</mi><mo>,</mo><mi>&de1ta;</mi><mi>s</mi></mfenced> 
<mo>=</mo) 
<mfrac><mrow><mn>1</mn></mrow><mrow><mi>&xi;</mi></mrow></mfrac> 
<msub><mi>&phi;</mi)<mrow><mi>v</mi></mrow> 
</msub> <mfenced open=’(’ 
c1ose=’)’><msub><mi>&lambda:</mi><mrow><mi>v</mi></mrow> </msub> 
<mo>,</mo><mi>&kappa;</mi><mo>,</mo><msup><mi>&beta;</mi><mrow><mn>2</mn> 
</mrow> </msup> </mfenced> <mtext></mtext> </math></subeqn></eqnarray> 
where 
<eqnarray><subeqn><math><msub><mi>&phi;</mi><mrow><mi>v</mi></mrow> 
</msub> <mfenced open=’(’ 
close=’)’><msub><mi>&lambda;</mi)<mrow><mi>v</mi></mrow> </msub> 
<mo>,</mo><mi>&kappa;</mi><mo>‘</mo><msup><mi>&beta;</mi><mrow><mn>2</mn> 
</mrow> </msup> </mfenced> <mo>=</mo) 
<mfrac><mrow><mn>1</mn></mrow><mrow><mn>2</mn><mi>&pi;</mi><mi>i</mi></mrow> 
</mfrac><msubsup><mo>&int;</mo> 
<mrow><mi>c</mi><mo>+</mo><mi>i</mi><mi>&infin;</mi></mrow> 
<mrow><mi>c</mi><mo>-</mo><mi>i</mi><mi>&infin;</mi) 
</mrow></msubsup><mi>&phi;</mi><mfenced 
open=’(’ 
c1ose=’)’><mi>s</mi></mfenced><msup><mi>e</mi><mrow><mi>&1ambda;</mi><mi>s</mi) 
</mrow> </msup> <mi>d</mi><mi>s</mi><mspace 
width=’2cm’/><mi>c</mi><mo>&geq;</mo><mn>0</mn> <mtext></mtext> 
</math></subeqn><subeqn><math> </math></subeqn><subeqn 
><math><mi>&phi;</mi><mfenced open=’(’ c1ose=’)’><mi>s</mi></mfenced> 
<mo>=</mo) <mo>exp</mo><mfenced open=’[’ 
close=’]’><mi>&kappa;</mi><mrow><mo>(</mo><mn>1</mn><mo>+</mo) 
<msup><mi>&beta;</mi><mrow><mn>2</mn> 
</mrow> </msup> 
<mi>&gamma;</mi><mo>)</mo)</mrow></mfenced><mo>exp</mo><mfenced 
open=’[’ close=’]’><mi>&psi;</mi) <mfenced open=’(’ 
close=’)’><mi>s</mi)</mfenced></mfenced><mo>,</mo) <mtext></mtext> 
</math></subeqn><subeqn><math> </math></subeqn><subeqn 
><math><mi>&psi;</mi) <mfenced open=’(’ c1ose=’)’><mi>s</mi></mfenced> 
<mo>=</mo) <mi>s</mi><mo>1n</mo) 
<mi>&kappa;</mi)<mo>+</mo><mrow><mo>(</mo)<mi>s</mi><mo>+</mo><msup> 
<mi>&beta;</mi><mrow><mn>2</mn> 
</mrow> </msup> <mi>&kappa;</mi><mo>)</mo></mrow><mfenced open=’[' 
close=’]’><mo>ln</mo) 
<mrow><mo>(</mo)<mi>s</mi><mo>/</mo><mi>&kappa;</mi><mo>)</mo></mrow> 
<mo>+</mo><msub><mi>E</mi><mrow> 
<mn>1</mn> </mrow> </msub> 
<mrow><mo>(</mo><mi>s</mi)<mo>/</mo><mi>&kappa;</mi><mo>)</mo> 
</mrow></mfenced><mo>-</mo><mi>&kappa;</mi><msup><mi>e</mi><mrow> 
<mo>-</mo><mi>s</mi)<mo>/</mo><mi>&kappa;</mi> 
</mrow> </msup> <mo>,</mo> <mtext></mtext> </math></subeqn></eqnarray> 
and <eqnarray><subeqn><math><msub><mi>E</mi><mrow><mn>1</mn> </mrow> 
</msub> <mrow><mo>(</mo><mi>z</mi><mo>)</mo></mrow> 
<mo>=</mo><msubsup> <mo>&int;</mo> 
<mrow><mi>&infin;</mi></mrow><mrow><mi>z</mi></mrow></msubsup> 
<msup><mi>t</mi><mrow><mo>-</mo><mn>1</mn> 
</mrow> </msup> <msup><mi>e</mi><mrow><mo>-</mo><mi>t</mi) </m:ow> 
</msup> <mi>d</mi><mi>t</mi><mspace width=’1cm’/><mtext>(the 
exponential integral)</mtext> <mtext></mtext> </math></subeqn><subeqn 
><math> </math></subeqn><subeqn 
><math><msub><mi>&1ambda;</mi><mrow><mi>v</mi></mrow> </msub> 
<mo>=</mo) <mi>&kappa;</mi><mfenced open=’[’ 

 
%%page page_414                                                  <<<---3
 
A.1 An example I1}'I}3X file and its translation to XML 
395 
101 
102 
103 
105 
106 
107 
108 
109 
110 
111 
112 
113 
114 
115 
116 
117 
118 
119 
131 
close=’]’><mfrac><mrow><mi>&epsi;</mi><mo>-</mo><munderover 
accent=’true’><mi>&epsi;</mi><mrow></mrow><mo>&barwed;</mo></munderover> 
</mrow> <mrow><mi>&xi;</mi></mrow></mfrac> 
<mo>-</mo><mi>&gamma;</mi><mi>&prime;</mi) 
<mo>-</mo><msup><mi>&beta;</mi><mrow><mn>2</mn> </mrow> </msup> 
</mfenced> <mtext></mtext> </math></subeqn></eqnarray> 
</par> 
<par>The Vavilov parameters are simply related to the Landau parameter 
by <inlinemath><math><msub><mi>&lambda;</mi><mrow><mi>L</mi) </mrow> 
</msub> <mo>=</mo><msub><mi>&1ambda;</mi><mrow><mi>v</mi></mrow> 
</msub> <mo>/</mo><mi>&kappa;</mi><mo>-</mo><mo>1n</mo) 
<mi>&kappa;</mi></math></inlinemath>. It can be shown that as 
<inlinemath> <math> 
<mi>&kappa;</mi><mo>&rarr;</mo><mn>0</mn></math></inlinemath>, the 
distribution of the variable <inlinemath> <math> 
<msub><mi7&1ambda;</mi><mrow><mi>L</mi> </mrow> </msub> 
</math></inlinemath> approaches that of Landau. For <inlinemath> 
<math> 
<mi>&kappa;</mi><mo>&1eq;</mo><mn>0</mn><mo>.</mo><mn>0</mn><mn>1</mn> 
</math></inlinemath> 
the two distributions are already practically identical. Contrary to 
what many textbooks report. the Vavilov distribution <emph> does 
not</emph> approximate the Landau distribution for small 
<inlinemath><math><mi>&kappa;</mi></math></inlinemath>, but rather the 
distribution of <inlinemath> <math> 
<msub><mi>&lambda;</mi><mrow><mi>L</mi> </mrow> </msub> 
</math></inlinemath> defined above tends to the distribution of the 
true <inlinemath><math><mi>&1ambda;</mi></math></inlinemath> from the 
Landau density function. Thus the routine <texttt> GVAVIV</texttt> 
samples the variable <inlinemath> 
<math><msub><mi>&1ambda;</mi><mrow><mi>L</mi> </mrow> </msub> 
</math></inlinemath> rather than <inlinemath> <math> 
<msub><mi>&lambda;</mi><mrow><mi>v</mi></mrow> </msub> 
</math></inlinemath>. For <inlinemath> <math> 
<mi>&kappa;</mi><mo>&geq;</mo><mn>1</mn><mn>0</mn></math></inlinemath> 
the Vavilov distribution tends to a Gaussian distribution (see next 
section). </par> 
</section> 
</section> 
<section class=``star''><stitle>References</stitle> 
<bib1iography> 
<bibitem id=``bib-LAND''> 
<par>L.Landau. On the Energy Loss of Fast Particles by 
Ionisation. Originally published in <emph>J. Phys.</emph>, 8:201, 
1944. Reprinted in D.ter Haar, Editor, <emph>L.D.Landau, Collected 
papers</emph>, page 417. Pergamon Press, Oxford, 1965. </par> 
</bibitem> 
<bibitem id=``bib-SCH1''> 
<par>B.Schorr. Programs for the Landau and the Vavilov distributions 
and the corresponding random numbers. <emph>Comp. Phys. Comm.</emph>, 
7:216, 1974. </par> 
</bibitem> 
<bibitem id=``bib-SELT''> 
<par>S.M.Seltzer and M.J.Berger. Energy loss straggling of protons and 
mesons. In <emph>Studies in Penetration of Charged Particles in 
Matter</emph>, Nuclear Science Series 39, Nat. Academy of Sciences, 
Washington DC, 1964. </par> 
</bibitem> 
<bibitem id=``bib-TALM''> 
<par>E.Talman. On the statistics of particle identification using 
ionization. <emph>Nucl. Inst. Meth.</emph>, 159:189, 1979. </par> 
</bibitem> 
<bibitem id=``bib-VAVI''> 
<par>P.V.Vavilov. Ionisation losses of high energy heavy 

 
%%page page_415                                                  <<<---3
 
396 
Example files 
166 
167 
168 
170 
17] 
particles. <emph>Soviet Physics JETP</emph>, 5:749, 1957.</par> 
</bibitem> 
</bib1iography> 
</section) 
</bodymatter> 
</document) 
A.1.3 Document Type Definition for XML version 
Sooouamawwu 
«am»2 
Z 3 E 
5 
N... 
cc 
N 
N 
N 
N 
«.4 
N 
.5 
25 
<!-- latex.dtd: XML version of LaTex + MathML --> 
<!ENTITY Z fontchange ``emphltextitltextbfltextsfltextslltexttt'' > 
<!ENTITY Z misc “urllquadlhspacelvspaceIincludegraphics|footnote|tag|ent“> 
<!ENTITY Z xref "ref|cite|pageref"> 
<!ENTITY Z chunk "lalistlparltabularlfigureltable|align|bibliography"> 
<!ENTITY Z mathobj ``displaymathlinlinemathlequationleqnarray'' > 
<!ENTITY Z inline "#PCDATA|Xfontchange;lzchunk;lzmisc;|Zxref;|Zmathobj;“> 
<!ELEMENT document (frontmatter?,bodymatter)> 
<!ATTLIST document class CDATA ``article''> 
<!ELEMENT frontmatter (title,author,date?,abstract?,keywords?)> 
<!ELEMENT bodymatter ((par|section)#,appendix#)> 
<!-- front matter --> 
<!ELEMENT title (Zinline;)#> 
<!ELEMENT author (%inline;)#> 
<!ELEMENT date (#PCDATA)> 
<!-- structuring --> 
<!ELEMENT section (stitle,(%chunk;|subsection)t)> 
<!ATTLIST section 
class CDATA #IMPLIED 
id ID #IMPLIED> 
<!ELEMENT subsection (stitle,(Zchunk;lparagraph)‘)> 
<!ATTLIST subsection 
class CDATA #IMPLIED 
id ID #IMPLIED> 
<!ELEMENT paragraph (stitle,(Xchunk;|subparagraph)*)> 
<!ATTLIST paragraph 
class CDATA #IMPLIED 
id ID #IMPLIED> 
<!ELEMENT subparagraph (stitle,(ZChunk;)*)> 
<!ATTLIST subparagraph 
class CDATA #IMPLIED 
id ID #IMPLIED> 
<!ELEMENT stitle (Zinline;)#> 
<!-- font changes --> 
<!ELEMENT emph (Zinline;)#> 
<!ELEMENT textit (%inline;)*> 
<!ELEMENT textbf (%inline;)t> 
<!ELEMENT textsf (Zinline;)*> 
<!ELEMENT textsl (Zinline;)*> 
<!ELEMENT texttt (%inline;)*> 
<!-- lists --> 
<!ELEMENT lalist (item)#> 
<!ATTLIST lalist 
id ID #IMPLIED 
class (enumeratelitemizeldescription) #REQUIRED> 
<!ELEMENT item (Zinline;)*> 

 
%%page page_416                                                  <<<---3
 
A.1 An example HIFX file and its translation to XML 
110 
111 
112 
113 
114 
115 
116 
117 
118 
119 
<!-- bibliography --> 
<sELEmENT bibliography (bibitem)‘> 
<!ELEMENT bibitem (Zinline;)t) 
<!ATTLIST bibitem 
id ID #REQUIRED> 
<!-- floats --> 
<!ELEMENT table (Zchunk;lcaptionlincludegraphics)*> 
<!ELEMENT figure (xchunk;Icaptionlincludegraphics)#) 
<!ELEMENT caption (%inline;):> 
<!ATTLIST caption 
id ID #IMPLIED) 
<!ELEMENT includegraphics EMPTY) 
<!ATTLIST includegraphics 
width CDATA #IMPLIED 
height CDATA #IMPLIED 
scale CDATA #IMPLIED 
file CDATA #IMPLIED> 
<!-- tables --> 
<!ELEMENT tabular (hline|row)#> 
<!ATTLIST tabular 
preamble CDATA #REQUIRED) 
<!ELEMENT row (cell)#) 
<!ELEMENT hline EMPTY) 
<!ELEMENT cell (Xinline;)*> 
<!ELEMENT newline EMPTY) 
<!ATTLIST newline 
id ID #IMPLIED) 
<!-- low-level bits and pieces --> 
<!ELEMENT align (%inline;)t> 
<!ATTLIST align 
style CDATA #REDUIRED) 
<!ELEMENT url EMPTY) 
<!ATTLIST url 
name CDATA #REQUIRED> 
<!ELEMENT par (Zinline;)#> 
<!ELEMENT quad EMPTY) 
<!ELEMENT hspace EMPTY) 
<!ATTLIST hspace 
dim CDATA #REQUIRED) 
<!ELEMENT vspace EMPTY) 
<!ATTLIST vspace 
dim CDATA #REQUIRED) 
<!ELEMENT tag (#PCDATA)) 
<!ELEMENT ent EMPTY) 
<!ATTLIST ent 
value CDATA #REQUIRED 
name CDATA #REQUIRED) 
<!-- cross-refs --> 
<!ELEMENT cite EMPTY) 
<!ATTLIST cite 
refid IDREF #REQUIRED> 
<!ELEMENT ref EMPTY) 
<!ATTLIST ref 
refid IDBEF #REDUIRED> 
<!-- maths. must reduce to <math) elements for Mathfll --> 
<!ELEMENT equation (math)t> 
<!ATTLIST equation 
id ID #IMPLIED) 
<!ELEMENT displaymath (math)#) 
<!ELEMENT inlinemath (math)*) 
397 

 
%%page page_417                                                  <<<---3
 
398 
120 
121 
122 
1Z3 
124 
125 
126 
127 
128 
129 
130 
131 
132 
133 
134 
135 
136 
137 
138 
139 
140 
141 
142 
143 
145 
146 
147 
148 
149 
150 
151 
152 
153 
154 
155 
156 
157 
158 
159 
160 
161 
162 
163 
165 
166 
167 
168 
169 
170 
171 
172 
173 
174 
175 
176 
177 
178 
179 
180 
181 
182 
183 
184 
<!ELEMENT subeqn (math)*> 
<!ATTLIST subeqn 
id ID #IMPLIED> 
<!ELEMENT eqnarray (subeqn)#> 
<!ATTLIST eqnarray 
number (yeslno) ``yes'' 
id ID #IMPLIED> 
<!-- sub DTDs and entities --> 
<!--Added Math Symbols: 
<!ENTITY Z isoamsae.dtd 
<!--Added Math Symbols: 
<!ENTITY Z isoamsbe.dtd 
<!--Added Math Symbols: 
<!ENTITY Z isoamsce.dtd 
<!--Added Math Symbols: 
<!ENTITY Z isoamsne.dtd 
<!--Added Math Symbols: 
<!ENTITY % isoamsoe.dtd 
<!--Added Math Symbols: 
<!ENTITY Z isoamsre.dtd 
<!--General Technical--> 
<!ENTITY Z isoteche.dtd 
Arrows--> 
SYSTEM "isoamsae.dtd"> 
Binary Operators--> 
SYSTEM "isoamsbe.dtd"> 
Delimiters--> 
SYSTEM “isoamsce.dtd"> 
Negated Relations--> 
SYSTEM “isoamsne.dtd"> 
Ordinary--> 
SYSTEM "isoamsoe.dtd"> 
Relations-*> 
SYSTEM "isoamsre.dtd"> 
SYSTEM "isoteche.dtd"> 
<!--Numbers and Currency aymbols--> 
<!ENTITY Z isonume.dtd SYSTEM “isonume.dtd"> 
<!--MathML Aliases (From ISO PUB,DIA,NUM)--> 
<!ENTITY Z mmaliase.dtd 
<!--Greek Symbols--> 
<!ENTITY Z isogrk3e.dtd 
<!--Math Script Font--> 
<!ENTITY Z isomscre.dtd 
SYSTEM "mmaliase.dtd“> 
SYSTEM “isogrk3e.dtd"> 
SYSTEM "isomscre.dtd“> 
<!--Math Open Face Font--> 
<!ENTITY Z isomopfe.dtd 
<!--MathML Entities--> 
SYSTEM "isomopfe.dtd“> 
<!ENTITY Z mm1ent.dtd SYSTEM “mm1ent.dtd“> 
<!--Main MathML DTD --> 
<!ENTITY Z mathm1.dtd 
%mathm1.dtd; 
%isoamsae.dtd; 
%isoamsbe.dtd; 
Zisoamsce.dtd; 
Zisoamsne.dtd; 
Zisoamsoe.dtd; 
%isoamsre.dtd; 
%isoteche.dtd; 
%isonume.dtd; 
Zmmaliase.dtd; 
Xisogrk3e.dtd; 
Zisomscre.dtd; 
Zisomopfe.dtd; 
Zmm1ent.dtd; 
SYSTEM “mathm1.dtd“> 
Example files 

 
%%page page_418                                                  <<<---3
 
A.2 Scripting examples for techexplorer 
135 
186 
187 
188 
139 
190 
399 
<!ENTITY aacute "&#x00E1;“> 
<!ENTITY 0verBar "[0verBar]"> 
<!ENTITY negationslash "/"> 
<!-- end of 1atex.dtd --> 
A.2 Scripting examples for techexplorer 
A.2.1 teched.html 
5»ooo\-an-z;«.a~.. 
B: 
<HTML> 
<!-- teched.html --> 
<!-- (C) Copyright 1998 by Robert S. Sutor. All rights reserved. --> 
<HEAD> 
<META HTTP-EQUIV=``Content-Type'' 
CONTENT="text/html; charset=iso-8859-1"> 
<META NAME=``GENERATOR'' 
CONTENT="Mozi11a/4.01 [en] (Win95; I) [Netscape]"> 
<TITLE>teched Sample LaTeX Editor</TITLE) 
</HEAD> 
<!-- This is a very simple LaTeX editor built using the --> 
<!-- IBM techexplorer Hypermedia Browser and a Java applet. --> 
<BDDY> 
<CENTER> 
<!-- The upper window is controlled by techexplorer. We --> 
<!-- give the name ’teInput’ to this window. We are --> 
<!-- using a table to put a frame around the window. --> 
<TABLE BORDER=1> 
<TR> 
<TD> 
<EMBED TYPE="application/x-techexplorer" 
TEXDATA=“\(\)" 
NAME=``teInput'' wIDTH=600 HEIGHT=150> 
</TD> 
</TR> 
</TABLE) 
<!-- The lower window is handled by the ’teched’ Java --> 
<!-- applet. Like the techexplorer window, it is 600 --> 
<!-- pixels wide. --> 
<TABLE BORDER=1> 
<TR> 
<TD> 
<APPLET CODE="teched.class" 
NAME=“teched" ALIGN=CENTER 
WIDTH=600 HEIGHT=130 MAYSCRIPT></APPLET> 
</TD> 
</TR> 
</TABLE) 
</CENTER> 
</BODY) 
</HTML> 

 
%%page page_419                                                  <<<---3
 
400 
Example files 
A.2.2 teched.java 
// teched.java 
8 c m s Q m o m N N 
«.4 
pa 
.5 
15 
// (C) 
Copyright 1998 by Robert S. Sutor. All rights reserved. 
// We first import the classes we need from the standard Java 
// distribution. This will work with Java 1.0 or higher. 
import 
import 
import 
import 
// The 
java.awt.t; 
java.awt.event.#; 
java.lang.#; 
java.applet.Applet; 
following brings in the Netscape Liveconnect classes 
// that we will use. 
import 
netscape.javascript.JSObject; 
// These are the classes that we use that are exposed by 
// techexplorer. The first is the basic interface to the 
// plug-in. The others are the event and listener classes. 
import 
import 
import 
import 
public 
ibm.techexplorer.plugin.techexplorerPlugin; 
ibm.techexp1orer.awt.AWTEvent; 
ibm.techexplorer.awt.event.KeyListener; 
ibm.techexplorer.awt.event.KeyEvent; 
class teched 
extends java.app1et.App1et 
implements KeyListener 
{ 
// The JavaScript window object 
JSObject Window = null; 
// The JavaScript document object 
JSObject Document = null; 
// The techexplorer plug-in instance 
techexplorerPlugin tePlugin = null; 
// The editable text area for the markup source 
TextArea 
markupInputArea = null; 
// The ’Clear input’ button 
Button clearInputButton = null; 
// A utility buffer for holding the markup. 
StringBuffer 
markupString = new StringBuffer("“); 
public boolean action(Event evt, Object arg) { 
// We only handle the ’Clear input’ action. 
boolean result = false; 
if ( evt.target == clearInputButton ) { 
// Empty the markup edit area. 
markupInputArea.setText( "“ ); 
// Reinitialize the techexplorer document. This 
// needs to be a non-empty string to actually 
// updated the document, so we give it some 
// non-visible input. 
tePlugin.reloadFromTeXString( "\\(\\)" ); 
result = true; 

%==========420==========<<<---2
 
%%page page_420                                                  <<<---3
 
H0 
111 
112 
113 
114 
115 
116 
M7 
118 
120 
121 
123 
I29 
125 
126 
127 
Scripting examples for techexplorer 401 
return result; 
} 
public void init() { 
// Initialize the components we are displaying 
// with this Java applet. 
clearInputButton = new Button(``Clear input''); 
markupInputArea = new TextArea( 5, 80 ); 
this.setLayout( new FlowLayout() ); 
this.add( markupInputArea ); 
this.add( clearInputButton ); 
} 
public void keyPressed( ibm.techexplorer.awt.event.KeyEvent e ) { 
// We don’t do anything with this event given us by 
// techexplorer. But see ’keyTyped’. 
} 
public void keyTyped( ibm.techexplorer.awt.event.KeyEvent e ) { 
// This is a naive (but effective!) way of dealing with 
// keys coming to us from techexplorer. We grab the key 
// that was pressed and put it on the end of our markup. 
// Then we update the techexplorer window. 
if ( e.getSource() == tePlugin ) { 
markupInputArea.appendText( 
( new Character( e.getKeyChar() )).toString() ); 
// This replaces the document within the techexplorer 
// window with that gotten by parsing the string 
// passed to it. 
tePlugin.re1oadFromTeXString( markupInputArea.getText() ); 
} 
public void keyReleased( ibm.techexplorer.awt.event.KeyEvent e ) { 
// This is where we deal with key release events coming to 
// us from the techexplorer window. 
switch ( e.getKeyCode() ) { 
case KeyEvent.VK_DELETE: 
// When we see a ’delete’ key, we remove the last character 
// in the markup. 
if ( e.getSource() == tePlugin ) { 
markupstring = new StringBuffer( markupInputArea.getText() ); 
int length = markupString.length(); 
if ( length > 0 ) 
--length; 
markupString.setLength( length ); 
markupInputArea.setText( markupString.toString() ); 
tePlugin.reloadFromTeXString( markupInputArea.getText() ); 
} 
break; 
case KeyEvent.VK_ENTER: 
// When we see that the ’enter’ key has been pressed, we 
// insert a newline in the markup. This improves readability. 
if ( e.getSource() == tePlugin ) { 
markupInputArea.appendText( "\n" ); 
tePlugin.reloadFromTeXString( markupInputArea.getText() ); 

 
%%page page_421                                                  <<<---3
 
402 
Example files 
130 
131 
132 
13} 
134 
135 
136 
137 
138 
139 
150 
154 
155 
156 
I58 
159 
160 
161 
162 
164 
165 
166 
167 
168 
169 
170 
171 
172 
174 
178 
} 
break; 
default: 
break; 
} 
public boolean keyUp( Event evt, int key ) 
{ 
// This key is one from the markup input area. 
// When a key is released, update the techexplorer 
// document with the current markup. 
boolean result = false; 
if ( evt.target == markupInputArea ) { 
if ( evt.id == Event.KEY_RELEASE ) { 
int length = markupInputArea.getText().length(); 
if ( length > 0 ) 
tePlugin.reloadFromTeXString( 
markupInputArea.getText() ); 
else 
tePlugin.reloadFromTeXString( "\\(\\)" ); 
} 
result = true; 
} 
return result; 
} 
public void start() { 
// Initialize the Netscape JavaScript objects. 
Window (JSObject) JSObject.getWindow(this); 
Document = (JSObject) Window.getMember("document“); 
// Try to get the techexplorer plug-in object. 
tePlugin = (techexplorerPlugin) Document.getMember(``teInput''); 
if ( tePlugin == null ) 
// If we didn’t get it, print a debug message. 
System.out.println(“teched: start(): null teched"); 
else 
// Otherwise add the listener for techexplorer keys. 
tePlugin.addKeyListener( (KeyListener) this ); 
} 
public void stop() { 
if ( tePlugin == null ) 
// If we don’t have the techexplorer plug-in object, 
// print a debug message. 
System.out.println("teched: stop(): null teched"); 
else 
// Otherwise remove the listener for techexplorer keys. 
tePlugin.removeKeyListener( (KeyListener) this ); 

 
%%page page_422                                                  <<<---3
 
B 
Technical appendixes 
B.1 The HyperTEX standard 
An important standardization effort for TEX and hypertext is the HyperTEX project 
[u-> HYPERTEX]. It was set up by Tanmoy Bhattacharya and Mark Doyle for the EPrint archive [h>EPR]NT] at the Los Alamos National Laboratory in the United 
States under the leadership of Paul Ginsparg. The HyperTEX specification says that 
conformant viewers and translators must recognize the following set of \special 
constructs: 
href html:<a href 
name html:<a name 
end html:</a> 
image html:<img src = ``hr-ef_str7Lng''> 
base htm1:<base href = "hr-ef_st'r-7Lng"> 
" h'r'ef_s t'r'7Lng"> 
`` name_s tr-7Lng''> 
II 
The href, name, and end commands are used to do the basic hypertext operations of establishing links between sections of documents. The image command is 
intended (as with HTML browsers) to place an image of arbitrary graphical format 
on the page at the current location. The base command is used to communicate 
to the DVI viewer the full (URL) location of the current document so that files 
specified by relative URLs may be retrieved correctly. 
The href and name commands must be paired with an end command later 
in the TEX file-the TEX commands between the two ends of a pair form an ancbor in the document. In the case of an href command, the anchor is to be highlighted in the DVI viewer. When it is clicked on, it will cause the scene to shift 

 
%%page page_423                                                  <<<---3
 
404 
Technical appendixes 
to the destination specified by h'r-ef_st'r-wing. The anchor associated with a name 
command represents a possible location to which other hypertext links may refer, 
either as local references (of the form hre:E="#na.me_st7``1lng'' with the attribute 
na.me_st7"1lng identical to the one in the name command) or as part of a URL (of 
the form UBL#name_st'r-ring). Here h'r-ef_st'r-wing is a valid URL or local identifier, 
while na.me_st'r-wing could be any string at all: The only caveat is that " characters 
should be escaped with a backslash (\); if it looks like a URL name, it might cause 
problems. 
The HyperTEX \special commands are implemented by (at least) the following DVI drivers: xdvi under UNIX, dviout and dviwindo under VV1ndows, 
and OzTeX and Textures on the Macintosh. The dvips program also supports the 
commands, and can translate them into a form suitable for conversion to PDF. 
B.2 Configuring TEX4ht to produce XML 
The basic idea of BTEX is interfaces for identifying structural entities in documents 
and for associating presentations with such entities. SGML and its descendants aim 
at a generalization of such an idea in which the interfaces are provided in formats 
suitable for further processing. The ultimate objective of TEX4ht is to offer a tool 
that recognizes TEX-based interfaces in general, and of ETEX in particular, and 
that provides the means to translate these interfaces into arbitrary SGML-based 
representations. 
VV1th this in mind, as we saw in Chapter 4, TEX4ht delegates to the native 
TEX compiler the task of processing the source files, relieving itself from dealing 
with many painful details that have little to do with structural issues in documents. 
To identify the structure of a document, TEX4ht seeds the style files with hooks 
at strategic locations within the definitions of macros and then offers the means to 
configure the hooks to produce any sort of output you want. This means that you 
can create translations to different SGML-based representations. 
\refsec{4_5_Extended_customization_of_texht} on page 170 showed how the hooks can be configured. We now go 
on to describe a friendly interface for identifying and configuring hooks and then 
wander more deeply into low-level features associated with the TEX compiler. 
The reader is encouraged to get a live demonstration of the interface discussed 
in the next two sections, by actually running the different stages of the example. 
B.2.1 Starting from scratch 
The fast evolution of HTML between different versions in recent years and the 
emergence of the XML language highlight the importance of allowing for largescale modifications to the output of TEX4ht, to meeting the challenges that arise 
from new requirements. Let us look at how to strip TEX4ht back to basics and then 
build up a new converter. 

 
%%page page_424                                                  <<<---3
 
B.2 Configuring TEX4ht to produce XML 
405 
B.2.1.1 Loading empty configurations 
The package option 0.0 requests empty configurations for the hooks, implying an 
output consisting of text with no hypertext tags. For example, the following source 
document outputs just the text 1 Demo with A List without any tags: 
\documentclass{article} 
\usepackage[htm1,0.0]{tex4ht} 
\begin{document} 
\section{Demo} 
With 
\begin{description} 
\item[A] List 
\end{description} 
\end{document} 
We will keep this simple example in mind for the next few sections in order to 
demonstrate the outcome of the features discussed. 
B.2.1.2 Looking for hooks 
A major problem in reconfiguring the TEX4ht hooks is to find out where they 
are and what they can do. The option hooks gives us a good start by requesting 
pseudo/yypertext tag: for hooks that have no other configuration. A pseudo/Jypertext 
tag shows the name of a hook, the index of an argument within the hook, and, 
when it is not obvious, the number of arguments for the hook. 
Compiling the example source document with the html ,0.0 ,hooks package 
options will produce output that looks like this in a browser: 
In this display, the tag <HTML1:2> comes from the first argument of the “HTML" 
hook. That hook is configured with the command 
\Conf igure{HTML} {first-argument} {second-z1rgument}. 

 
%%page page_425                                                  <<<---3
 
406 
Technical appendixes 
The <EnV(description) 2> tag comes from the second argument of the 
description environmental hook, configured by the command 
\Conf igureEnv{descript ion} {first-argument} {secondargument} {tbirdwrgument} {fimrt/J-argument} 
The <List (description)4> tag comes from the fourth argument of the 
description list hook, configured by the command 
\ConfigureList{description} {first-argument} {secondargument} {third-argument} {fimrt/J-argument} 
B.2.1.3 Viewing the hooks 
VVhile current browsers are likely to display the output resulting from the hooks 
option, the file itself is missing a proper HTML code at the start and the end of 
the file. That code can be quite easily introduced by configuring the “HTML" hook. 
We can also give the tags special characteristics within the display to attract the 
attention of the reader to their presence. The appearance may be modified within 
the CSS file, as well as by changing the following default configuration: 
\Configure{hooks} 
{\HCode{<STRONG CLASS=``hooks''>&lt;}} {\HCode{&gt;</STRONG>}} {}{} 
By introducing a configuration file try. cf g as follows and loading it with the 
option list try,html,0.0,hooks, the HTML file will become legal, and the tags 
will shown in green. 
\Configure{HTML}{\StartHtml}{\EndHtm1} 
\newcommand\StartHtm1{\IgnorePar\HCode{<EDOCTYPE HTML PUBLIC 
"-//W3C//DTD HTML 4.0 Transitional//EN">\Hnewline 
<HTML><HEAD><LINK REL=``stylesheet'' 
TYPE=“text/css“ HREF=“\jobname.css“>\Hnewline 
<TITLE>\jobname</TITLE></HEAD>\Hnewline <BODY>} 
\Css{.hooks{color2green;}}} 
\newcommand\EndHtm1{\HCode{</BODY></HTML>}} 
\Preamble{} 
\begin{document} 
\EndPreamble 
B.2.1.4 Reconfiguring the hooks 
Starting from output consisting of pseudo/Jypertext tags, you can gradually configure 
the different hooks to get the output you want. Along the way, distracting tags of 
no interest can be removed by assigning invisible nonempty code to their corresponding parameters. 

 
%%page page_426                                                  <<<---3
 
B.2 Configuring TEX4ht to produce XML 407 
By configuring the hooks “BODY", “HEAD", “TITLE+", “TITLE", “TocAt*", 
“TocAt", and “toe" to add the \empty to their parameters, the example reduces 
to the following: 
1 DemO<secuon3:4> With 
<List(descr1pt1on)1> <List(description)2> A 
<List(description)S> List <List(description)4> <Env(description)2> 
<sectian2:4> 
VV"1th straightforward configurations supplied to the remaining hooks, you obtain a view like this: 
[DOC] [SECTION][NUM]1[/NUM][TITLE]Demo[/TITLE] With [DLIST] 
[MARK] A [/MARK]List E/DLIST] [/SECTION][/DOC] 
A configuration file which produces output like this is as follows: 
\Configure{HEAD}{\empty}{\empty} 
\Configure{HTML} 
{\IgnorePar\HCode{<HTML><HEAD> <TITLE></TITLE></HEAD><BODY>}[DOC]} 
{E/DOC]\HCode{</BODY></HTML>}} 
\Preamble{} 
\Configure{BODY}{\empty}{\empty} 
\Configure{TITLE}{\empty}{\empty} 
\Configure{TITLE+}{\empty} 
\Configure{TocAt}{\empty}{\empty} 
\Configure{TocAt*}{\empty}{\empty} 
\Configure{toc}{\empty} 
\ConfigureEnv{description} { [DLIST] }{ [/DLIST] }{}{} 
\ConfigureList{description}{}{}{[MARK]}{[/MARKJ} 
\Conf igure{section} { [SECTION] }{ [/SECTION] } 
{[NUM]\arabic{section}[/NUM][TITLE]}{[/TITLE]} 
\begin{document} 
\EndPreamble 
\Configure{Htm1Par}{}{\empty}{}{} 
B.2.1.5 Cleaning up 
Having completed configuring the hooks of interest, we can remove the code we 
introduced just for getting an improved view of the output, and we can take out the 
hooks option in the \usepackage command. 
B.2.2 Adding XML tags 
The XML language is good for exhibiting how TEX4ht can be reconfigured, so we 
will now look at adding some XML tags to our output. As long as the newly assigned 

 
%%page page_427                                                  <<<---3
 
408 
Technical appendixes 
configurations to the hooks do not contain HTML tags, the browsers should have 
no problem displaying the outcome of these assignments. 
TEX4ht provides the three commands: 
\Tg<name> 
\Tg</name> 
\Tg<name/> 
for producing start, end, and empty XML tags, respectively. The edit package option makes these new commands produce visible forms of the tags by introducing 
code for displaying the tags instead of actually creating them. 
For example, replacing the [name] with commands of the form \Tg<name> in 
the configuration file of our example, we get a view like this in HTML browsers: 
<DOC> <SECTION> <NUM> 1</NUM><TITLE>Demo</TITLE> 
With <DLIST> <MARK> A </MARK>List </DLIST> </SECTION> </DOC> 
using the following version of the configuration file: 
\Configure{HTML} {\Tg<DOC>}{\Tg</DOC>} 
\Preamble{} 
\ConfigureEnv{description} {\Tg<DLIST>}{\Tg</DLIST>}{}{} 
\ConfigureList{description}{}{}{\Tg<MARK>}{\Tg</MARK>} 
\Configure{section} {\Tg<SECTION>}{\Tg</SECTION>} 
{\Tg<NUM>\arabic{section}\Tg</NUM>\Tg<TITLE>}{\Tg</TITLE>} 
\begin{document} 
\EndPreamble 
The default look of the visible “tags" is set by the command 
\Configure{edit}{\HCode{<STRONG>&1t;}}{\HCode{&gt;</STRONG>}} 
Removing the edit option would make a proper XML document. 
B.2.2.1 Typesetting the abstract tags 
The job of the XML tags is to identify logical units within the document. During 
the editing phase in which the tags are introduced, it can be very useful to highlight 
the nature of the different tags and the structural relationships they maintain. 
To meet this objective, the \Tg tags can be enhanced with the commands 
\Conf i gure<name>{before} {after} 
\Conf igure</name>{l2efi;re}{after} 
\Conf i gure<name/ >{befi)re}{afier} 
for producing whatever sort of display you find useful. 

 
%%page page_428                                                  <<<---3
 
B.2 Configuring TEX4ht to produce XML 409 
The addition of commands like 
\Configure<NUM>{\HCode{<DIV ALIGN=``CENTER''>}}{} 
\C0nfigure</TITLE>{}{\HCode{</DIV>}} 
\Configure<DLIST>{\HCode{<BR>}}{} 
\Configure</DLIST>{}{\HCode{<BR>}} 
to the configuration file results in a layout like the following within the browser: 
<DOC> <SECTION> 
<NUM>1</NUM><TITLE>Demo</TITLE> 
With 
<DLIST> <MARK> A </MARK>List </DLIST> 
</SECTION></DOC> 
VVhen the first argument is a hyphen character -, the configuration commands 
produce just the second argument without showing the tag itself. In this case, the 
configuration commands define a virtual browser for the new tags. 
Consider the following variants of the configuration commands: 
\Configure<NUM>-{\HCode{<DIV ALIGN=``CENTER''>}} 
\Configure</NUM>-{ } 
\Configure<TITLE>-{} 
\Configure</TITLE>-{\HCode{</DIV>}} 
These commands will produce a browser display like this: 
B.2.2.2 Checking containment relationships 
The verif y option requests warning messages for unknown parent-child containment relationships among the tags defined by the \Tg command. Containment relationships can be specified by listing pairs of parent and child names with the string 
--> between the commands 

 
%%page page_429                                                  <<<---3
 
410 
Technical appendixes 
\Verify 
\EndVerify 
Different pairs within the listing should be separated by a comma. Thus if 
\Verify --> DOC, DOC --> SECTION \EndVerify 
is placed before the \begin{document} of the configuration file try . cf g, and the 
package option verify is used, you will get warning messages like these: 
4. --- warning --- SECTION --> NUM ? 
4. --- warning --- SECTION --> TITLE ? 
4. --- warning --- SECTION --> DLIST ? 
4. --- warning --- DLIST --> MARK ? 
All we have specified is that the SECTION tag is expected within a DOC tag. The 
system reports that it has found a MARK within a DLIST. 
The package option verify+ is an extension of the verif y option in which the 
containment relations found by the system are detailed in the log file. 
Similarly the hooks+ package option is an extension of the hooks option. Besides requesting a display of how parameters of hooks are used, this option also asks 
for a listing in the log file of how hooks are configured. This is useful when looking 
at example configurations which may provide useful guidelines on how to define 
new configurations. Typically such examples are spread around different files and 
might be buried within other unrelated code. 
A \Tg<argument> command is a special case of a \HCode{ <argument>} command that allows us, during an editing phase, to show the tags and to check that 
they satisfy proper containment relationships. A weaker \TG version of the command is offered for instances where you do not care about verifying containment 
relationships. On the other hand, a \tg variant of the \TG command which corresponds to an <argument>, instead of to \HCode{ <argument>}, is available. 
B.2.3 Getting deeper for extra configurations 
Our attention so far has been on structural features that apply mainly to large items 
such as chapters, lists, and tables. These entities are typically defined within style 
files; they do not really rely on the low-level features that are provided in the TEX 
typesetting engine. Now we need to look at some of the means available in TEX4ht 
for dealing with these low-level commands. 
B.2.3.l Math mode 
Mathematical formulae are rich in low-level TEX features. The default settings of 
TEX4ht provide graphic versions of mathematical formulae that are likely to have 
complex structures. The package option math requests an alternative setup in which 

%==========430==========<<<---2
 
%%page page_430                                                  <<<---3
 
B.2 Configuring TEX4ht to produce XML 411 
the formulae are seeded by hooks that can be configured to specify new output 
representations. 
The mathematical environments \(fi)rmula\), \[fi)rmula\], $fi)rmula$, and 
$$fi)7mula$$ can be configured with the following commands: 
\Conf igure{ 0 }{l7efi)re$at-start} {at-end$after} 
\Conf i gure{ [] } {before $$at-start} {at-end$$after} 
\Conf i gure{$} {l7efi)re}{after} {at-start} 
\Conf i gure{$$}{befi)re}{after} {at-smrt} 
The configuration of the latter pair of environments applies also to the math and 
displaymath environments, respectively. 
\Configure{()}{[equation]$}{$[/equation]} 
\ (a+b\) 
Some of the features we will look at require TEX4ht to be in a special mode of 
operation. The commands 
\DviMath 
\EndDviMath 
may be used to set this up. 
B.2.3.2 Math classes for symbols 
At the lowest level of math reside the primitive tokens representing numbers, variables, operators, parentheses, and so forth. Inspired by the classification of mathematical symbols within TEX, a similar (but independent) classification is applied to 
symbols in TEX4ht. 
Specifically the symbols might be assigned a class number between 0 and 9, 
where an initial classification puts large operators in class 1, binary operators in class 
2, relational operators in class 3, opening delimiters in class 4, closing delimiters in 
class 5, punctuation symbols in class 6, and the rest of the symbols in class 0. 
The following command may be used to associate output tags with the symbols 
of a particular class and to add symbols into the class: 
\Conf i gure{MathC1ass}{class-numl7er} {string}{l7efi)re}{afier} {symbols} 1 
If the string argument is not empty, it must be a single character not appearing in 
the arguments befbre and after. In this case, those arguments specify a content to be 
inserted before and after the symbols of the class. 

 
%%page page_431                                                  <<<---3
 
412 Technical appendixes 
The argument symbol: should be a sequence of symbols. VVhen it is not empty, 
the symbols specified in the argument are assigned to the specified class. 
\Configure{()}{\DviMath$}{$\EndDviMath} 
\Configure{MathC1ass}{0}{*} 
{[ordinary]}{[/ordinary]}{} 
\Configure{MathC1ass}{2}{*} 
{[operator]}{[/operator]}{} 
\Configure{MathC1ass}{7}{*} 
{ [digit] }{ [/digit] }{o123456789} 
\(a+1\) 
[ordinarylaf/ordinary]:[opcraT is: i 5 
‘ tor}+[ioperator} [digit}1[/digit] . r 
As when the arguments of \HCode are processed, the befbre and after are processed just for macro expansions and not for fonts, definitions, and computations. 
The commands 
\PauseMathClass 
\E‘.ndPauseMathC1ass 
can be used temporarily to suppress the contributions of the different classes. 
B.2.3.3 Math classes for delimiters and words 
Open and close delimiter symbols, which group tokens for the \Send{BACK} instructions (Section B.2.3.4), should be assigned their math classes with the command: 
\Conf igure{MathDe1imiters}{open}{close} 
VVhen the package option math is used, the default setting invokes the command 
\Configure{MathDelimiters}{(}{)} to assign the parentheses symbols “(" and 
“)" to the math classes 4 and 5 , respectively, and to inform TEX4ht about the pairing 
of these delimiters. 
TEX4ht provides the commands \mathord, \mathop, \mathbin, \mathrel, 
\mathopen, \mathclose, and \mathpunc for associating, respectively, classes 0 
through 6 to subformulae. The following instruction configures these commands, 
while indirectly referring to the commands through their class numbers: 
\Conf i gure{Formul aClas s} {clamnumber} {strin g} {before} {afier} 
If the argument string is not empty, it should be a character not appearing in the 
arguments befiore and after. If string is empty, we specify the configuration to be used 
for the characters of the specified clas.v-numl7er. 
\Config11re{O}{\DviMath$}{$\EndDviMath} 
\Configure{Formu1aC1ass}{4}{*}{}{[} 
\Configure{Formu1aC1ass}{5}{*}{]}{} 
\(\mathopen{open}a+1\mathc1ose{c1ose}\) 

 
%%page page_432                                                  <<<---3
 
B.2 Configuring 'I]3X4ht to produce XML 413 
B.2.3 .4 DVI-based groups 
The output of TEX is the low-level page-description language DVI, which relies 
heavily on a hierarchical grouping mechanism. That hierarchy reflects the structure specified explicitly and implicitly in the source document. You sometimes need 
access to this information when writing Uanslators. 
\Trace{GROUP} 
\E‘.ndTrace{GROUP} 
\Conf i gure{GROUP} {string} {open-I }{open-2}{close-I }{close-2} 
The first two commands delimit the region in which the groups are to be shown. 
The third command controls the manner in which the groups are shown. 
The arguments open-I and open-2 specify what is to be inserted at the start of 
groups, and, if the first of these arguments is not empty, the level of nesting and 
the group number are also included. The arguments close-1 and close-2 specify in a 
similar manner the content to be inserted at the end of groups. 
\Conf igure{GROUP}{*}{ [H] }{ UH] } 
\noindent \Trace{GR.0UP} 
\(A \stackrel{\string~}{=} B\) 
\EndTrace{GROUP} 
The next two commands enable us to submit content to the start and end points 
of the current group. The groups are identified by their level of nesting, relative to 
where the command appears. Level 0 refers to the group that immediately includes 
the \Send command. The higher levels refer to groups that follow the command. 
\Send{GROUP}{lez/el}{content} 
\Send{EndGROUP}{let/el}{conte72t} 
\Configure{ () }{\DviMath 
\Send{GR.0UP}{ 1}{ [1] } 
\Send{E'.ndGR.DUP}{1}{ [/1] } 
\Send{GR.0UP}{2}{ [2] } 
\Send{E‘.ndGR.0UP}{2}{ [/2] } 
$}{$\E‘.ndDviMath} 
\ (A \stackrel{\string~}{=} B\) 
The command 
\Send{BACK}{content} 
may be used to send content backwards over the most recent symbol, group, 
or region enclosed between delimiters that have been declared a pair with the 
\Configure{MathDe1imiters}{open}{close} command. 

 
%%page page_433                                                  <<<---3
 
414 
Technical appendixes 
B.2.3.5 Subscripts and superscripts 
There are several alternative ways to configure subscripts and superscripts. They 
can all make use of the \Send{BACK}{content} instructions, when the bases of the 
subscripts and superscripts also need to be configured. 
\Conf igure{SUB}{befiore} {zzfifer} 
\Conf igure {SUP }{befiore} {after} 
\Conf igure{SUBSUP}{befi)re}{bet'ween}{zzfier} 
\Conf igure{SUPSUB}{be79)re}{bet'ween}{zzfier} 
\Conf igure{SUB/SUP}{before-1}{bet'ween-1}{zzfier-1} 
{before-2}{bet'ween-2}{zzfiter-2} 
The first pair of commands are for cases where the subscripts and superscripts appear in isolation. 
The third command applies to adjacent subscripts and superscripts, but only 
if subscripts are always to appear before the superscript in the output, regardless 
of their order in the source Code. The fourth command is a variant of the third 
command, where subscripts always are to appear after the superscripts. The fifth 
command is another variant, requesting the output to preserve the order of the 
subscripts and superscripts in the source. 
The arguments before-1, bet'ween-1, and zzfiter-1 apply when the subscript precedes the superscript and the arguments before-2, bet'ween-2, and zzfter-2, when the 
superscript comes first. 
\Configure{ () }{\DviMath$}{$\EndDviMath} 
\Configure{SUBSUP} 
{\Send{BACK}{ [base] } [/base] [sb] } 
{ [/sb] [sp] } 
{ C/SP] } 
\ (a“b_c\) 
B.2.3.6 Accents 
TEX4ht deals with accents by looking up a table constructed by use of the following 
command: 
\Configure{accent}\tea;tcmd\math,cmd 
{{in-1}{out-1}{7:~n-2}{out-2}. . .{}{out-Za,st}} 
{do-found} {do-not-found} 
The pairs {in-1}{out-1}, {z'n-2}{out-2}, . . . set up series data records for the accent. 
The first field of the last pair of data fields must be empty. 
The table is assigned to a text-based accent command \te:ctcmd and a corresponding math-based accent command \math,cmd. 

 
%%page page_434                                                  <<<---3
 
B.3 XNIL namespaces 
415 
When an accent command is encountered in the source document, its table is 
searched for a record where the first field matches the argument of the command. 
If a match is found, the parameter do-found is activated. In this case, the second field 
of the pair is available as the macro parameter #1. 
If no match is found, the do-not-found is activated. In this case, the parameter 
#1 represents the accent command under its original meaning in BTEX, and the 
parameter #2 represents the argument of the command. 
\Configure{accent}\“\hat{a{a}{}{}} 
{ [hat] #1 [/hat] } 
{ [HAT] #2 [/HAT] } 
U’ {[vec]#2[/VeC]} 
$\vec a=(\hat a,\hat b)$ 
If the second argument is empty, the previous table for the accent commands is 
used. Similarly, if both do-found and do-not-found are empty, the previous fragments 
of code are used. 
B.3 XML namespaces 
An XML document may contain elements and attributes that refer to multiple software modules, each with its own vocabulary. Thus it is possible that the same names 
are used by one or more of these modules with different purposes; this then results 
in name collisions. Therefore a mechanism for name scoping, that is, indicating to 
which namespace a given element type or attribute refers, is highly desirable. 
The W3C tackled this problem in the “Namespaces in XML" recommendation 
[¢->XMLNS]. That document defines an XML namespace as a collection of names, 
identified by a URI, that are used in XML documents as element types and attribute 
names. 
Names from XML namespaces consist of a namespace prefix, which is mapped 
to a URI to select the namespace,1 followed by a single colon, and then a loczzlpzzrt. 
As URIs can contain characters that are not allowed in names, one uses a proxy 
that associates the namespace prefix with the given URI. Such an abbreviation is 
prefixed with the string xmlns. An example is the XSL style sheet that we discussed 
in Section 7.6.1, and where we have the following two declarations: 
1 <xsl:stylesheet xmlns:xs1="http://www.w3.org/TR./WD-xsl" 
2 xmlns:fo="http://www.w3.org/TR./WD-xsl/F0" 
1In fact this “URI" is, strictly speaking, only a string that uniquely identifies a namespace. It does 
not require corresponding to a real document on the Internet, although it is good practice to specify a 
genuine URI if it exists. 
\Configure{accent}\vec\vec{{}{}} 

 
%%page page_435                                                  <<<---3
 
416 
Technical appendixes 
We see the xmlns prefix, followed by the name of the namespace shorthand that 
will be used in the document to scope the different element types and attributes. 
For instance, in the same document we find constructs like: 
1 <xs1:temp1ate match=“par“> 
z <fo:b1ock indent-start=``10pt'' space-before=``12pt''> 
3 <xsl:apply-templates/> 
4 </fo:b1ock> 
5 </xsl:temp1ate> 
Lines 1 and 5 refer to the xsl namespace defined by the first of the two declaration 
of the xslzstylesheet element type above. Lines 2 and 4 refer to the formatting objects part of the XSL document, as defined by the second declaration of the 
xsl : stylesheet element (for f o namespace). The part of the name following the 
colon is called the local part of the name. It should have a meaning in the framework of the namespace defined by the URI. The prefix, defined in the document 
instance (e.g., the xsl and f o prefixes above), has no meaning outside of the document where it is declared, since it is merely a placeholder for a URI. It is thus 
important always to export the prefix declarations together with elements that use 
them. 
One should note that the prefix xml (any combination of upperand lowercase) 
is reserved for use by XML-related specifications. 
A special case is when one declares the default namespace by omitting the colon 
and namespace prefix. Thus element types without a namespace specifier refer to 
that default namespace. For instance, line 2 below declares the default namespace 
of the document to be HTML4, as defined in the HTML4 specification found at the 
given URI. 
1 <xs1:stylesheet xmlns:xs1="http://www.w3.org/TR/WD~xsl" 
2 xm1ns="http://www.w3.org/TR/REC-htm140"> 
In the style sheet instance one then can directly write: 
1 <xs1:temp1ate match="invitation/par/emph"> 
2 <em><xs1zprocess-children/></em) 
2 </xs1:temp1ate> 
where the <em> tags on line 2 are pure HTML. 
The default namespace, once declared, may be overridden. 
<’.7xn11 version=’ 1 .0’ ?> 
1 
2 . . . 
3 <particles> 
4 <!-- Default namespace is set to HTML --> 
5 <1-.ab1e xm1ns= ’http : //wuw .w3 . org/TR/R.EC~htm140 ’ > 
6 <tr><td>PartiC16</td><td>Mass</td><td>Detai1s</td></tr> 
7 <tr> 
5 <td><e1n>neutron</en1></td> 

 
%%page page_436                                                  <<<---3
 
P‘ 
4:Examples of important DTDs 41 7 
<td>939.56 MeV</kd> 
<td> 
<pdg xm1ns=“mypdgnamespace"> 
<!-- HTML namespace is no longer used --> 
<quarks>udd</quarks><1ifetime>886.7 s</1ifetime> 
<decay>p , e , anue</decay> 
</pdg> 
</td> 
</kr> 
</tab1e> 
</particles> 
N.-...._.»-.._..--...._._.... 
o~<>oa\:a~m.pwN._-o.o 
Starting with the table element (line 5), we use the HTML4 namespace as default until we get to the data in the third column (line 10). Here the default namespace for the pdg element and its children is no longer HTML, but it is set to 
mypdgnamespace (line 12). Once we leave the pdg element (line 15), the namespace will automatically become HTML again. 
Namespace scoping is not limited to element types; it can also be used for 
attributes and in DTD declarations. 
B.4 Examples of important DTDS 
In order to get a better idea of what DTDs for more complex documents look like, 
we will briefly look at a few representative examples in this section. We start by 
saying a few words about the DocBook, ISO-12083, TEI, and HTML DTDs. Then 
we discuss in much greater detail how to construct DTDs for BIBTEX and BTEX 
documents. 
B.4.1 The DocBook DTD 
The DocBook DTD was developed specifically for computer software documentation, such as user manuals and programming references. DocBook was originally maintained by the Davenport Group ['->DAVENPORT], a discussion forum 
sponsored by individuals representing large-scale producers and consumers of software documentation. Since July 1998, the maintenance has been handled by the 
DocBook Technical Committee under the umbrella of OASIS. Recently DocBook 
was adopted for the Linuxdoc-SGML initiative, which uses the SGML-Tools package ['-> SGMLTOOLS]. Previously, Linuxdoc-SGML was based on the qwertz DTD, 
mentioned in Section B.4.5. 
The DocBook DTD is available from ['->DOCBOOK]. It is a complete SGML 
DTD that has been used by many documentation providers to mark up their documents. More recently Norman Walsh has completely restructured the DTD to 
make it XML-compliant ['-> DBXML]. 
The DocBook DTD uses a “book" model for the documents. A book is composed of book elements such as prefaces, chapters, appendixes, and glossaries. Five 

 
%%page page_437                                                  <<<---3
 
418 Technical appendixes 
section levels are available, and these may contain paragraphs, lists, index entries, 
cross-references, and links. 
book 
meta information 
chapter 
sect1 
sect2 
sect1 
chapter 
sectl 
appendix 
sect1 
appendix 
sect1 
glossary 
The DTD leaves room for localizations. The user of the DTD is free to provide 
variant content models for appendixes, chapters, equations, indexes, and so on. 
<!ENTITY Z 1oca1.appendix.c1ass ""> 
<!ENTITY Z appendix.c1ass “appendix Z1oca1.appendix.c1ass;“> 
1 
2 
3 
4 <!ENTITY Z 1oca1.chapter.c1ass ""> 
5 <!ENTITY Z chapter.c1ass “chapter Z1oca1.chapter.c1ass;“> 
6 
7 
8 
9 
<!ENTITY Z 1oca1.index.c1ass ""> 
<!ENTITY Z index.c1ass “indexlsetindex Z1oca1.index.c1ass;“> 
10 <!ELEMENT book ((Zdiv.title.content;)?, bookinfo?, dedication?, 
n toc?, 1ot*, (glossary!bib1iographylpreface)*, 
12 (((Zchapter.c1ass;)+, reference*) I part+ 
U I reference+ ), 
14 (Zappendix.class;)*, (glossary!bibliography)*, 
U (Zindex.c1ass;)*, lott, toc?)> 
The book element type (lines 10-14) is quite a complex combination of different 
element types, but in a few cases (appendix, chapter, and index, see lines 1-8) the 
content model explicitly defines parameter entities that can be used for convenient 
customization by the user. Indeed, since the first definition of an entity takes precedence, all such parameter entities (starting with local.) can be defined before the 
complete DTD is read. In Sections B.4.4 and B.4.5 we show how this technique is 
applied in practice. 
Similarly in the case of the attributes for the book element type, the parameter entity local .book. attrib (defined on line 1 and referenced on line 8 below) 
provides a means to complement the attributes applied by the DTD. 
1 <!ENTITY Z 1oca1.book.attrib ""> 
2 <!ENTITY Z book.ro1e.attrib "Zro1e.attrib;"> 
3 
4 <!ATTLIST book Zlabel.attrib; 

 
%%page page_438                                                  <<<---3
 
B.4 Examples of important DTDs 
419 
Zstatus . attrib; 
'/.common. attrib; 
7.book.ro1e .attrib; 
'/.1oca1 .book. attrib; 
xoooxlaxuu 
In fact, Norman Walsh’s distribution contains a file dbgenent.ent, which has 
empty definitions for the user-definable entities discussed earlier. Thus it is sufficient to edit this file to declare your own supplementary general entities, notations, 
and parameter entities. 
B.4.2 The AAP effort and ISO 12083 
Since the publication of the SGML Standard in 1985 the American Association of 
Publishers (AAP) has been working on promoting SGML as an electronic standard 
for manuscript preparation. Over several years they developed a document, the 
AAP Standard that was later promoted by the Electronic Publishing Special Interest Group (EPSIG) and the AAP as the Electronic Manuscript Standard. This AAP 
Standard achieved two major goals. First, it established an agreed way to identify 
and tag parts of an electronic manuscript by proposing an SGML tagset, thus allowing computers to recognize these parts. Second, it provided a logical way to 
represent special characters, symbols, and tabular material, using only the ASCII 
character set. 
Based on the work mentioned earlier, the AAP and the European Physical Society (EPS) agreed upon a standard method for marking up scientific documents. 
This was later formalized as the International Standard ISO 12083, considered the 
successor to the AAP/EPSIG Standard, and four DTDs have been distributed by 
EPSIG as the “ISO" DTDs ['->EPSIG]. 
This DTD has a basic book structure consisting of chapters, sections, and subsections down to six levels. VVhat is interesting is that it includes a DTD for math, 
which is quite similar, although not identical, to the visual presentation model of 
MathML, as discussed in Section 8.1. For instance, the AAP math visual markup 
tags provide for the following element categories: 
character transformations <bo1d>, <ita1ic>, <sansser>, <typewrit>, 
<smallcap>, <roman>; 
fractions <fraction>, <num>, <den>; 
superiors, inferiors <sup>, <inf>; 
embellishments <top>, <midd1e>, <bottom>; 
fences, boxes, overlines, and underlines <mark>, <fence>, <post>, <box>, 
<overline>, <undrline>; 
roots <radical>, <radix>, <radicand>; 

 
%%page page_439                                                  <<<---3
 
420 
Technical appendixes 
arrays <array>, <arrayrow>, <arraycol>, <arraycel>; 
spacing <hspace>, <vspace>, <break>, <markref>; 
formulae <f ormul a>, <dformul a>, <df ormgrp>. 
Emphasis is on creating fences at the right places inside a formula. A simple 
markup example is the following: 
<formu1a> 
= Ezsum;<inf>n=1</inf><sup>10</sup> 
<fraction> 
<nu.m>1</nu.m> 
<den> 
<radical>3<radix>n</radica1> 
</den> 
</fraction> 
</formu1a> 
~oao\:a\u.-pww... 
B.4.3 Text Encoding Initiative 
The Text Encoding Initiative (TEI, see ['->TEIHOME]) emphasizes the interchange 
of textual information, although other forms of information, such as images and 
sound, are also considered. The basic aim is to make certain features of a text explicit in such a way as to aid the processing of that text by computer programs that 
run on a variety of computer platforms. The document is encoded by the introduction of markup in the sources. 
A set of guidelines for marking up documents was agreed. Those interested in 
these guidelines should see ['->TEIGU1DE]. Since the TEI is supposed to be used 
throughout the world across many disciplines, certain principles should be obeyed. 
They are briefly listed: 
o The common core of textual features should be easily shared. 
o Additional specialist features should be easy to add to (or remove from) a text. 
o Multiple parallel encodings of the same feature should be possible. 
o The richness of markup should be user-defined, with a very small minimal 
requirement. 
o Adequate documentation of the text and its encoding should be provided. 
A full set of guidelines has been prepared, but often only a manageable “starter 
set," TEI-Lite ['->TEILITE], is actually used in practice. It should, therefore, come as 
no surprise that work for adapting the TEI DTD to XML has started with TEI-Lite. 
Patrice Bonhomme has made an XML version of TEI-Lite, as well as a prerelease of 
the full TEI P3 DTD. He has also translated a few of the TEI reference documents 
into XML [v->TEIXML]. 

%==========440==========<<<---2
 
%%page page_440                                                  <<<---3
 
B.4 Examples of important DTDs 
421 
B.4.4 A DTD for BIBTEX 
As practice is the best way of assimilating ideas, it is an interesting problem to 
see how we can apply the XML techniques we have introduced in Chapter 6 to an 
area where most IIXTFX users have already some experience, that is HIEX itself, and 
BIBTEX databases. Therefore in this section we construct two Variants of a DTD for 
a BIBTEX database, explaining which are important points to consider. In the next 
section we build a DTD for a subset of BTEX. 
B.4.4.l Fine points for developing a DTD 
Consider the following important points when you decide to write a DTD: VVhat is 
the application area of its use, what kind of information do we want to model, is it 
important that humans can understand the markup, or are documents going to be 
marked up and maintained automatically? Similarly the DTD architecture should 
be easy to maintain. It should be relatively easy to find and correct bugs and to 
extend the DTD, even for somebody who was not involved in its original design 
and without relying on the help of the original author, who might no longer be 
available. 
DTD design is probably more an art than a science, and only a lot of practice 
will allow you to write good DTDs. Eve Maler and Jeanne El Andaloussi (Maler 
and Andaloussi (1996)) examine in detail a lot of the SGML issues involved in how 
to model data, implement DTDs, and maintain them. David Megginson’s recent 
book (Megginson (1998)) is more directly applicable to XML in that it discusses the 
differences between XML and SGML. He also dedicates a chapter to the various 
DTD we mentioned in Sections B.4.1 to B.4.3. 
One of the main decisions when building a model is what should be “content" 
and what should be “characteristics," in other words, which components of our 
document model should correspond to elements, and which ones to attributes. Various schools of thought exist, and many pages have been written about this subject; 
the conclusion is that there is almost never one unique answer (see ['-> ELEMATTR] 
for an overview). 
Oversimplifying, we can say that complex information, which contains markup, 
is almost always better enclosed inside elements, whereas attributes are convenient 
when there is a choice only between various keywords (e. g., a color if one of “red," 
“blue," or “green"), or when there is a set of characteristics of an element that can 
be specified simply (e.g., the height, the width, the name of a month, a country, or 
a language). 
Nevertheless, since XML has rather fewer constructs than SGML (see Norman 
Walsh’s discussion of how to convert SGML to XML DTDs ['->sGML2XML] and 
Megginson (1998)), we are more constrained in the choice of plausible document 
models than in the case of full-blown SGML. For instance, it is extremely tedious 
in XML to express in a content model that a set of element types can be entered in 
any order in a document instance. Similarly there is no way to specify inclusion or 

 
%%page page_441                                                  <<<---3
 
422 
Technical appendixes 
exclusion of certain element types with respect to model groups (see Eve Maler’s 
article on handling exceptions in SGML and XML ['->MALEREX]). Therefore constructing XML DTDs that provide maximum flexibility is no trivial task and often 
compromises and simplifications have to be made with respect to SGML. 
In the following two sections we present two quite different DTDs that model 
a BIBTEX database. The first is based almost exclusively on elements for containing 
the information, while the second tries to specify as much information as feasible 
using attributes. 
B.4.4.2 A first version of the BIBTEX DTD 
As a design decision, elements and attributes will carry the same names as the corresponding entries and fields described by Leslie Lamport in Appendix B of Lamport 
(1994). However, the link between entries in the database and citation keys in the 
document is made via an XML ID type attribute. We call this attribute id (line 
5), thus allowing each entry to be cross-referenced with the crossref attribute of 
type IDREF (line 6). Each entry gets these two attributes assigned via the parameter 
entity atype (defined on lines S-6 and referenced on lines 19, 24, and so on). 
Appendix B does not prescribe an order for specifying the fields (“required" or 
“optional") for the various entry types. However, since XML does not have an easy 
way to express the fact that certain entries must be present, but can be specified 
in any order (XML lacks SGML’s 8c operator inside model groups), we impose an 
order that we take to be the one in which the fields are enumerated by Lamport. 
Moreover, at the end of each entry we allow a set of optional fields with the help of 
the info parameter entity (line 14). 
Names are handled with names elements, which consist of one or more sets of 
first and last element pairs, separated by an empty and element (lines 99-102). 
Most of the simple fields are just declared as #PCDATA, while fields that can have 
a more complex content (mostly those that we cannot assign to attributes in the 
alternative version of the DTD, and which we will discuss in Section B.4.4.3) can 
also include emphatic elements (line 88). 
The complete text of the DTD biblioxmll .dtd follows (the line numbers in 
the previous paragraphs refer to those shown here): 
<!-- biblioxml1.dtd: XML DTD for BibTeX markup: version 1 --> 
<!-- Every biblio entry _must_ have an identifier and 
_can_ have a cross-reference to an existing entry --> 
<!ENTITY Z atype "id ID #REQUIRED 
crossref IDREF #IMPLIED"> 
<!-- Possible types of biblio entries --> 
<!ELEMENT biblio (#PCDATA I articlel bookl bookletl inbookl incollectionl 
inproceedingsl manuall mastersthesisl miscl 
phdthesisl proceedingsl techreportl unpublished)*> 
:3xcso\Ia\'4v-L-\~N--<!-- Optional annotation, note, ISBN, or key (the latter for sorting) --> 
<!ENTITY Z info "(annotelnotelISBNlkey)*"> 
v:.L.\44I\t 

 
%%page page_442                                                  <<<---3
 
B.4 Examples of important DTDS 
423 
<!-- An article from a journal or magazine --> 
<!ELEMENT article (author, title, journal, year, 
volume?, number?, pages?, month?, %info;)> 
<!ATTLIST article Zatype;> 
<!-- A book with an explicit publisher --> 
<!ELEMENT book ((author|editor), title, publisher, year, 
(volume|number)?, series?, address?, edition?, month?, %info;)> 
<!ATTLIST book Zatype;> 
<!-- A work that is printed or bound, but without a named --> 
<!-- publisher or sponsoring institution --> 
<!ELEMENT booklet (title , 
author, howpublished?, address?, month?, year?, Zinfo;)> 
<!ATTLIST booklet %atype;> 
<!-- A part of a book, usually untitled; --> 
<!-- (a chapter, a sectional unit, or just a range of pages) --> 
<!ELEMENT inbook ((authorleditor), title, (chapterlpageB)*, publisher, year, 
(volume|number)?, series?, type?, address?, edition?, month?, Zinfo;)> 
<!ATTLIST inbook %atype;> 
<!-- A part of a book with its own title --> 
<!ELEMENT incollection (author, title, booktitle, publisher, year, 
editor?, (volume|number)?, series?, type?, chapter?, 
s?, address?, edition?, month?, %info;)> 
<!ATTLIST incollection %atype;> 
<!-- An article in a conference proceedings --> 
<!ELEMENT inproceedings (author, title, booktitle, year, 
editor?, (volume|number)?, series?, pages?, address?, 
month?, organization?, publisher?, Zinfo;)> 
<!ATTLIST inproceedings %atype;> 
<!-- Technical documentation ~-> 
<!ELEMENT manual (title, 
author?, organization?, address?, edition?, month?, year?, Zinfo;)> 
<!ATTLIST manual %atype;> 
<!-- A master’s thesis --> 
<!ELEMENT mastersthesis (author, title, school, year, 
type?, address?, month?, %info;)> 
<!ATTLIST mastersthesis Zatype;> 
<!-- Miscelleneous: use this type if nothing else fits --> 
<!ELEMENT misc (author?, title?, howpublished?, month?, year?,%info;)> 
<!ATTLIST misc Zatype;> 
<!-- A Ph. D. thesis -~> 
<!ELEMENT phdthesis (author, title, school, year, 
type?, address?, month?, %info;)> 
<!ATTLIST phdthesis Xatype;> 
<!-- The porceedings of a conference --> 
<!ELEMENT proceedings (title , year , 
editor?, (volume|number)?, series?, address?, 
month?, organization?, publisher?, Zinfo;)> 
<!ATTLIST proceedings Zatype;> 
<!-- A report published by a school or other institution --> 
<!-- usually numbered within a series --> 
<!ELEMENT techreport (author, title, institution, year, 
type?, number?, address?, month?, Zinfo;)> 
<!ATTLIST techreport %atype;> 

 
%%page page_443                                                  <<<---3
 
424 Technical appendixes 
81 <!-- A document with author and title, but not formally published --> 
82 <!ELEMENT unpublished (author, title, note, 
83 month? , key'?)> 
84 <!ATTLIST unpublished '/.atype;> 
85 
86 <!-- For adding typographic emphasis to the information --> 
37 <!ELEMENT emph (#PCDATA)> 
88 <!ATTLIST emph style (textbflemphltextsfItextslltextttlquote) ``emph''> 
89 
90 <!ENTI'l'Y '/. inline "(#PCDATA|emph)=r"> 
91 <!-- The basic fields (Lamport (1994), pages 162-164) --> 
92 
93 <!-- Usually the address of publisher or institution --> 
94 <!ELEMENT address (#PCDATA|emph) =r> 
95 <!-- Annotation (not used by standard styles) --> 
96 <!ELEMENT annote (#PCDATA|emph)*> 
97 <!-- Author(s) in format described on pp. 157-158 --> 
98 <.'ELEMENT author (names)> 
99 <.|ELEMENT names ((first,last),(and,first,last)*)> 
100 <!ELEMENT last (#PCDATAlemph) =r> 
101 <!ELEMENT first (#PCDATA|emph)*> 
102 <!ELEMENT and EMP’I‘Y> 
103 <!-- Title of a book, used in incollection and inproceedings --> 
104 <!ELEMENT booktitle (#PCDATA|emph)*> 
105 <!-- Chapter (or sectional unit) number --> 
106 <!ELEMENT chapter (#PCDATA|emph)=o<> 
107 <!-- Edition of a book (e.g., ``third'') --> 
108 <!ELEMENT edition (#PCDATA|emph) *> 
109 <!-- Names of editor(s) --> 
110 <1ELEMENT editor (names)> 
111 <!-- Describe how something strange is published --> 
112 <!ELEMENT howpublished (#PCDATA|emph)*> 
113 <!-- Sponsoring institution of a technical report --> 
114 <|.E.LEMENT institution (#PCDATAlemph) 41> 
115 <!-- The ISBN number (non-standard, but useful --> 
116 <!ELEMENT ISBN (#PCDATA)> 
117 <!-- A journal’s name --> 
118 <!ELEMENT journal (#PCDATA|emph)*> 
119 <!-- Used for ordering the biblio entries --> 
120 <!ELEMENT key (#PCDATA)> 
121 <!-- Month of publication (or writing, if not published) --> 
122 <!ELEMENT month (#PCDATA)> 
123 <!-- Additional information to help the user --> 
124 <!ELEMENT note (#PCDATA|emph)=o<> 
125 <!-- Number of a journal, magazine, report, etc. --> 
126 <!ELEMENT number (#PCDATA)> 
127 <!-- Sponsor of conference or publisher of manual --> 
128 <!ELEMENT organization (#PCDATAIemph)*> 
129 <!-- page, or page range --> 
130 <!ELEMENT pages (#PCDATA)> 
131 <!-- Publisher’s name --> 
132 <.'ELEMENT publisher (#PCDATA|emph)=or> 
133 <!-- Name of the school where thesis was written --> 
134 <!ELEMENT school (#PCDATAlemph)*> 
135 <!-- Name of the series or set of books --> 
136 <!ELEMENT series (#PCDATA|emph)=r> 
137 <!-- A work’s title --> 
138 <!ELEMENT title (#PCDATA|emph)=o<> 
139 <!-- Type of a technical report, e.g., ``Research Note'' --> 
140 <!ELEMENT type (#PCDATA|emph)*> 
141 <!-- Volume number of a journal or multi-volume book --> 
142 <!ELEMENT volume (#PCDATA)> 
143 <!-- Year of publication (or writing, if not published) --> 
144 <11:1.121~11:1rr year (#PCDATA) > 

 
%%page page_444                                                  <<<---3
 
B.4 Examples of important DTDs 
Now we can use this DTD with an example document, and we choose the 
BIBTEX source shown in \reffig{13-4} of The BTEX Companion (Goossens et al. 
(1994)) as basis. Following you see how we mark up the document relative to our 
DTD. At the start of the file (lines 2-10) we define some entities that were handled by the bibnames . sty file or defined in the original file. Note how we use 
entities for the abbreviations of the names of the months (lines 12-23), which are 
predefined in BIBTEX. For reasons of convenience and typographic consistency, 
throughout the file we have used these entity references for recurring text strings. 
Of course, if we were to target a text processing system different from BTEX, we 
would have to define some of the entities differently (in particular lines 4-11, which 
are I1i‘IEX-specific). 
Citation references would use the values defined with the id attribute with 
each entry. Note how the cross~reference on line 85 also exploits XML’s ID/IDREF 
system. However, since the identifiers in an XML document instance share a global 
namespace, one should try to use a transparent and consistent naming scheme for 
document element identifiers. 
<!DOCTYPE biblio SYSTEM "bib1ioxm11.dtd" [ 
<!ENTITY AW ``Addison-Wes1ey''> 
<!ENTITY Awzadr ``Reading, Massachu5etts''> 
<!ENTITY emdash “---"> 
<!ENTITY endash ``--''> 
<!ENTITY ouml ’\"o’> 
<!ENTITY j-TUGboat ``TUGboat''> 
<!ENTITY LaTeX "\LaTeX{}"> 
<!ENTITY TeX "\TeX{}"> 
<!ENTITY WEB "\textsc{web}"> 
<!-- abreviations for the months --> 
<!ENTITY jan "Jan."> 
<!ENTITY feb "Feb."> 
<!ENTITY mar "Mar."> 
<!ENTITY apr "Apr.“> 
<!ENTITY may "May“> 
<!ENTITY jun "Jun."> 
<!ENTITY jul "Jul."> 
<!ENTITY aug "Aug."> 
<!ENTITY sep "Sep."> 
<!ENTITY oct "0ct."> 
<!ENTITY nov "Nov."> 
<!ENTITY dec "Dec."> 
24 ]> 
25 <bib1io> 
Z6 <manua1 id=``Dynatext''> 
-NN._...._....._._...._.-._ 
wN--o~oan\.a~v-4:-wN.-o~oan\.a~v-pww... 
27 <title>Dynatext, Electronic Book Indexer/Browser</title> 
Z8 <organization>E1ectronic Book Technology Inc.</organization> 
29 <address>Providence, Rhode Is1and</address> 
30 <year>1991</year> 
31 </manua1> 
32 <book id="Eijkhout:1991"> 
33 <author><names><first>Victor</first><1ast>Eijkhout</last></names></author> 
34 <title>&TeX; by Topic, a &TeX;nicians Reference</title> 
35 <pub1isher>&Aw;</pub1isher> 
36 <year>1991</year> 
37 <address>&AW:adr;</address> 
33 </book>, 
39 <techreport id="EVH:0ffice“> 
w <author><names><first>Eric</first><1a5t>van Hervijnen</1aat></names></author> 
425 

 
%%page page_445                                                  <<<---3
 
426 
Technical appendixes 
<title>Future Office Systems Requirements</title> 
<institution>CERN DD Internal Note</institution> 
<year>1988</year> 
<month>&nov;</month> 
</techreport> 
<article id="Fe1ici:1991"> 
<author><names><first>James</first><last>Felici</1ast></names></author> 
<title>PostScript versus TrueType</title> 
<journa1>Macworld</journal> 
<year>1991</year> 
<vo1ume>8</volume> 
<pages>195&endash;201</pages> 
<month>&sep;</month> 
</article> 
<techreport id="Knuth:WEB"> 
<author><names><first>Donald E.</first><last>Knuth</1ast></names></author> 
<title>The &WEB; System of Structured Documentation</title> 
<institution>Department of Computer Science, Stanford University</institution> 
<year>1983</year> 
<number>STAN-CS-83-980</number> 
<address>Stanford, CA 94305</address> 
<month>&sep;</month> 
</techreport> 
<phdthesis id="Liang:1983"> 
<author><names><first>Frank1in Mark</first><last>Liang</1ast></names></author> 
<title>Word Hy-phen-a-tion by Com-pu-ter</title> 
<school>Stanford University</schoo1> 
<year>1983</year> 
<address>Stanford, CA 94305</address> 
<month>&jun;</month> 
<note>Also available as Stanford University, Department of 
Computer Science Report No. STAN-CS-83-977</note> 
</phdthesis> 
<article id="Mitte1bach-Schoepf:1990"> 
<author><names><first>Frank</first><1ast>Mittelbach</1ast><and/> 
<first>Rainer</first><1ast>Sch&ouml;pf</1ast></names></author> 
<title>The New Font Selection temdash; User Interface to 
Standard &LaTeX;</title> 
<journal>&j-TUGboat;</journal> 
<year>1990</year> 
<vo1ume>11</volume> 
<number>2</number> 
<pages>297&endash;305</pages> 
</article> 
<incollection id="wood:color" crossref="Roth:postscript"> 
<author><names><first>Pat</first><1ast>wood</1ast></names></author> 
<title>PostScript Color Separation</title> 
<booktitle>Rea1 world PostScript</booktitle> 
<publisher>&Aw;</publisher> 
<year>1988</year> 
<pages>201&endash;225</pages> 
</inco11ection> 
<book id="Roth:postscript"> 
<editor><names><first>Stephen E.</first><last>Roth</last></names></editor> 
<title>Real World P0stScript</title> 
<publisher>&Aw;</publisher> 
<year>1988</year> 
<address>&Aw:adr;</address> 
<ISBN>0’§01-06663-7</ISBN> 
</book> 
<inproceedings id="Yannis:1991"> 
<author><names><first>Yannis</first><last>Haralambous</last></names></author> 
<title>&TeX; and those other languages</title> 
<booktitle>1991 Annual Meeting Proceedings, Part 2, &TeX; Users Group, 
Twelfth Annual Meeting, Dedham, Massachusetts, July 15--18, 1991 

 
%%page page_446                                                  <<<---3
 
B.4 Examples of important DTDs 
1% </booktitle> 
107 <year>1991</year> 
1011 <editor><names><first>Hope</first><last>1-Iami1ton</1ast></names></editor> 
1w <volume>12</volume> 
H0 <series>&j-TUGboat;</series> 
111 <pages>539&endash;548</pages> 
nz <address>Providence, Rhode Island</address> 
H3 <month>&dec;</month> 
H4 <organization>&TeX; Users Group</organization> 
us </inproceedings> 
us </bib1io> 
B.4.4.3 A second version of the BIBTEX DTD 
In developing an alternative version of a BIBTEX DTD we want to experiment with 
specifying as much information as possible with attributes. We already know that 
it would be very difficult to use attributes for complex fields, such as (book)title, 
author, and editor, which consist of names, note, and annote, which can contain 
free text notes. These are thus defined as elements (lines 179-194 at the end of the 
DTD), while all other fields are given as attributes following the definition of the 
various entries (lines 24-30, 34-43, and so on). 
In this version we provide the possibility of extending the DTD by introducing 
a parameter entity local . info (defined as the empty string on line 4) and putting a 
reference to it in the definition of the info parameter entity (line 7). The user of the 
DTD can thus define new elements by entering them in the Variable local . info, as 
shown in the following example. In this way, new fields can easily be made available. 
Similarly by defining the parameter entity local .biblio, the user can add new 
entry types‘ (we added a reference to local .biblio at the end of the definition 
of the content model for the root element biblio; see line 20). Line 6 declares 
local .keys, which provides a global way to add attributes to all entries. 
Below we show the alternative version of the BIBTEX DTD biblioxml2 .dtd: 
<!-- biblioxml2.dtd: XML DTD for BibTeX markup: version 2 --> 
Zloca1.keys;"> 
<!ENTITY Z month ’month (janlfeblmarlaprlmayljunljullauglseploctlnovldec) #IMPLIED’> 
<!-- Possible types of biblio entries --> 
<!ELEMENT biblio (#PCDATA I articlel bookl bookletl inbookl incollectionl 
inproceedingsl manuall mastersthesisl miscl 
phdthesisl proceedingsl techreportl unpublished 
%1oca1.bib1io;)*> 
1 
2 
3 <!-- Supplementary entry types, optional fields and attributes --> 
4 <!ENTITY Z 1oca1.info ""> 
5 <!ENTITY Z 1oca1.bib1io ""> 
6 <!ENTITY Z local.keys ""> 
7 <!ENTITY Z info "note Z1oca1.info;"> 
8 
9 <!-- All base elements must have ID and can have Cross-reference to --> 
10 <!-- an existing ID and include a key to sort. if needed --> 
11 <!ENTITY Z keys "id ID #REQUIRED 
n crossref IDREF #IMPLIED 
U key IDREF #IMPLIED 
14 
u 
M 
17 
n 
19 
N 
o 
N 
<!-- An article from a journal or magazine --> 
<!ELEMENT article (author,title,(Zinfo;)*)> 
N 
N 
N 
w 
427 

 
%%page page_447                                                  <<<---3
 
428 
Technical appendixes 
xxx: 
kallxl 
74 
<!ATTLIST article Zkeys; 
journal CDATA #REQUIRED 
year CDATA #REQUIRED 
Zmonth; 
number CDATA #IMPLIED 
s CDATA #IMPLIED 
volume CDATA #IMPLIED> 
<!-- A book with an explicit publisher --> 
<!ELEMENT book ((author|editor),title,(Zinfo:)*)> 
<!ATTLIST book Zkeys; 
publisher CDATA #REQUIRED 
year CDATA #F.E.QUIR.ED 
address CDATA #IMPLIED 
edit ion CDATA #IMPLIE.D 
Zmonth; 
number CDATA #IMPLIED 
series CDATA #IMPLIED 
volume CDATA #IMPLIE.D 
ISBN CDATA #IMPLIED> 
<!-- A work that is printed or bound, but without a named --> 
<!-- publisher or sponsoring institution --> 
<!ELEMENT booklet (title,author?,(Zinfo;)*)> 
<!ATTLIST booklet Zkeys; 
address CDATA #IMPLIED 
howpublished CDATA #IMPLIED 
Zmonth; 
year CDATA #IMPLIED 
ISBN CDATA #IMPLIED> 
<!-- A part of a book, usually untitled; --> 
<!-- (a chapter, a sectional unit, or just a range of pages) --> 
<!ELEMENT inbook ((author|editor),title,(%info;)*)> 
<!ATTLIST inbook Zkeys; 
s CDATA #REQUIRED 
publisher CDATA #REQUIRED 
year CDATA #REQUIRED 
address CDATA #IMPLIED 
chapter CDATA #IMPLIED 
edition CDATA #IMPLIED 
Zmonth; 
number CDATA #IMPLIED 
series CDATA #IMPLIED 
type CDATA #IMPLIED 
volume CDATA #IMPLIED 
ISBN CDATA #IMPLIED> 
<!-- A part of a book with its own title --> 
<!ELEMENT incollection (author,title,booktitle,editor'?,(Zinfo;)*)> 
<!ATTLIST incollection Zkeys; 
publisher CDATA #REQUIRED 
year CDATA #REQUIRED 
address CDATA #IMPLIED 
chapter CDATA #IMPLIED 
edition CDATA #IMPLIED 
Zmonth; 
number CDATA #IMPLIED 
s CDATA #IMPLIE.D 
series CDATA #IMPLIED 
type CDATA #IMPLIE.D 
volume CDATA #IMPLIED 
ISBN CDATA #IMPLIED> 
<!-- An article in a conference proceedings --> 

 
%%page page_448                                                  <<<---3
 
B.4 Examples of important DTDS 
110 
111 
112 
113 
114 
116 
136 
152 
153 
<!ELEMENT inproceedings (author,title,booktitle,editor?,(%info;)*)> 
<!ATTLIST inproc 
eedings Zkeys; 
year 
address 
Zmonth; 
number 
organization 
s 
publisher 
series 
volume 
ISBN 
<!-- Technical documentation --> 
<!ELEMENT manual (title,author?,(Zinfo;)*)> 
<!ATTLIST manual 
<!“ N master’s 
<!ELEMENT mastersthesis 
Zkeyss 
address CDATA 
edition CDATA 
organization CDATA 
Zmonth; 
year CDATA 
ISBN CDATA 
thesis --> 
<!ATTLIST mastersthesis Zkeys; 
school CDATA #REQUIRED 
year CDATA #RE.QUIRED 
address CDATA #IMPLIED 
Zmonth; 
CDATA #REQUIRED 
CDATA #IMLIED 
CDATA #IHLIED 
CDATA #IMPLIED 
CDATA #IMPLIED 
CDATA #IHLIED 
CDATA #IHLIED 
CDATA #IMPLIED 
CDATA #IMPLIED> 
#IMPLIED 
#IMPLIED 
#IMPLIED 
#IMPLIED 
#IMPLIED> 
(author,title,(%info;)*)> 
type CDATA #IMPLIED 
ISBN CDATA #IMPLIED> 
<!-- Miscelleneous: use this type if nothing else fits --> 
(((author,title)I(title,author))?,(Zinfo;)*)> 
<!ELEMENT misc 
<!ATTLIST misc 
Zkeys; 
howpublished CDATA 
Zmonth; 
year CDATA 
<!-- A Ph. D. thesis -~> 
sis (author,title,(%info;)*)> 
< 2 ELEMENT phdthe 
< : ATTLIST phdthe 
sis Zkeys; 
school CDATA 
year CDATA 
address CDATA 
Zmonth; 
type CDATA 
ISBN CDATA 
#IMPLIED 
#IMPLIED> 
mzqumsn 
maqumsn 
#IMPLIE.D 
#IMPLIED 
#IMPLIED> 
<!-- The proceedings of a conference --> 
<!ATTLIST procee 
dings Zkeys; 
year 
address 
Zmonth; 
number 
cmn mzqumsn 
cmn #IMPLIED 
CDATA #IMPLIED 
organization CDATA #IMPLIED 
publisher 
series 
volume 
ISBN 
CDATA #IMPLIED 
CDATA #IMPLIED 
CDATA #IMPLIED 
CDATA #IMPLIED> 
<!-- A report published by a school or other institution --> 
<!-- usually numbered within a series --> 
<!ELEMENT techreport (author,title,(Zinfo;)*)> 
429 

 
%%page page_449                                                  <<<---3
 
430 
Technical appendixes 
154 < ! ATTLIST techreport '/.keys; 
155 institution CDATA #REQUIR.ED 
156 year CDATA #REQUIR.ED 
15 7 address CDATA #IMPLIED 
158 '/.mor1th; 
159 number CDATA #IMPLIED 
160 type CDATA #IMPLIED 
161 ISBN CDATA #IMPLIED> 
162 
163 <!-- A document with author and title, but not formally published --> 
164 <!ELEMENT unpublished (author,title,(%info;)*)> 
165 <!ATTLIST unpublished '/.keys; 
166 address CDATA #IMPLIED 
167 '/Jnonth; 
168 number CDATA #IMPLIED 
169 type CDATA #IMPLIE.D> 
170 
171 <!-- For adding typographic emphasis to the information --> 
172 <!ELEMENT emph (#PCDATA)> 
173 <!ATTLIST emph style (bflemlitlsflsllttlqu) ``em''> 
n4 
175 <!-- The basic fields (pages 162-164) --> 
176 <!-- Only fields with names (author, editor) and titles are left as --> 
177 <!-- basic elements --> 
178 <!-- Author(s) in format described on pp. 157-158 --> 
179 <.|ELEMENT author (name,(and,name)*)> 
I80 <!ELEMENT name (first,1ast)> 
181 <!ELEMENT last (#1>CDATA)> 
182 <!ELEMENT first (#PCDATA)> 
183 <!ELEMENT and EMPTY> 
184 <!-- The names of the editor(s) --> 
185 <!ELEMENT editor (name,(and,name)=r)> 
186 <!-- The work's title --> 
187 <!ELEMENT title (#PCDATA|emph)*> 
188 <!-- Title of a book, used in incollection and inproceedings --> 
189 <!ELEMENT booktitle (#PCDATAIemph)*> 
190 <!-- Optional notes at end of entries --> 
191 <!-- Annotation (not used by standard styles) --> 
192 <!ELEMENT annote (#PCDATA|emph)*> 
193 <!-- Additional information to help the user --> 
194 <!ELEMENT note (#PCDATA|emph)*> 
We can use this new version of the BIBTEX DTD by referring to it in a 
little DTD mybib1io.dtd, which follows. Line 2 defines the parameter entity 
bib1io.dtd, which will allow us to include the DTD bib1ioxm12 .dtd on line 17. 
Lines 4-12 define the same general entities as lines 2-10 in the document instance 
of Section B.4.4.2. 
<!-- mybiblio.dtd (refers to "biblioxml2.dtd" --> 
<!ENTITY '/. biblio.dtd SYSTEM "bib1ioxm12.dtd"> 
< !ENTITY AW ``Addison-Wes1ey''> 
<!ENTITY AW:adr ``Reading, Massachusetts''> 
<!ENTITY emdash ``---''> 
<!ENTITY endash ``--''> 
<!ENTITY ou.m1 ’\"o’> 
< ! ENTITY j-TUGboat ``TUGboat '' > 
<!ENTITY LaTeX "\LaTeX{}"> 
<!ENTITY Tex "\TeX{}"> 
<!ENTITY WEB "\textsc{Heb}"> 
<!ENTITY '/. 1oca1.info "l url |a_nnote"> 
<!ELEMENT url (#PCDATA)> 
<!ENTITY '/. 1oca1.bib1io "| webdocument"> 
37.'.E.‘v'.'3'§1ZE<:o6\.o.u..>w..._ 

%==========450==========<<<---2
 
%%page page_450                                                  <<<---3
 
B.4 Examples of important DTDs 43 1 
I6 
17 '/.biblio .dtd; 
13 
19 
<!-- Document published on the Web --> 
20 <!ELEMENT webdocurnent (title,author‘?, (‘/.info;)=o<)> 
21 <!ATTLIST webdocument '/.keys; 
22 organi zation CDATA #IMPLIE.D 
23 Zmonth; 
24 year CDATA #IMPLIE.D> 
Because we cannot reference parameter entities before they are defined, we must be 
careful with the order in which we specify the various entries in the DTD. Therefore we now define the entity local. info (line 13), where we add a new field url 
plus include the annote field, which is defined in the DTD (line 192), but is not included in any of the content models. Line 14 defines the content of the url element 
as #PCDATA. Of course, we could have preferred to specify URLs as an attribute, in 
which case we would have to replace lines 13-14 with the following: 
<!ENTI'l'Y '/. 1ocal.:'rnfo "|a.nnote"> 
<!ENTI'l'Y '/. 1OCa1.key ``url CDATA IMPLIED''> 
However, in the example that follows, we will use the element definition. 
Finally, because we want to define a new entry type, webdocument, we include 
it in the local .biblio parameter entity (line 15). We include the orignal DTD 
via the entity reference biblio .dtd (line 17). The DTD will consume the values 
of all the “local" parameter entities, and the element definitions will be adapted 
accordingly. 
Only now (in lines 20-24) can we place the definition of the new entry type 
webdocument since it refers to Various parameter entities (info, keys, and month) 
that are defined in the DTD. For the same reason, we cannot define this element 
and its attributes in the internal subset of the document instance. The latter is read 
before the external subset, when these same parameters are not yet defined. 
With the mybiblio.dtd DTD we mark up the same BIBTEX database as in 
Section B.4.4.2. One recognizes how the information is now spread over elements 
and attributes. The fields that can be specified with attributes can be entered in 
any order, making data entry somewhat easier. Note how the names of the months 
have become name tokens of an attribute list (as defined on line 15 of the DTD 
biblioxml2 . dtd) and are no longer specified with general entity references (lines 
20, 29, and so on). As a novelty, at the end (lines 102-113) we add a supplementary entry that is not present in the original example. It is of type xmldocument as 
defined in the DTD mybiblio .dtd shown earlier in lines 20-24. 
<!DOC'l'YPE biblio SYSTEM "lnybiblio.dtd"> 
<title>Dynatext, Electronic Book Indexer/Browser</title> 
</manual> 
1 
2 <biblio> 
3 <rnanual id =``Dynatext'' 
4 organizat1'on="Electronic Book Technology Inc." 
5 address =``Providence, Rhode Island'' 
6 year =``1991''> 
7 
8 

 
%%page page_451                                                  <<<---3
 
432 
Technical appendixes 
<book 1d ="Eijkhout:1991" 
publisher-"&AW;" 
year =“1991" 
address =“&Aw:adr;"> 
<author> 
<name><firat>Victor</firat><1ast>Eijkhout</1a5t></name> 
</author> 
<title>&TeX; by Topic, a &TeX;nician5 Reference</title> 
</book> 
<techreport 1d ="EVH:0ffice" 
institution-``CERN DD Internal Note'' 
month -"nov“ 
year =``1988''> 
<author> 
<name><firat>Eric</first><1ast>van Herwijnen</1ast></name> 
</author> 
<title>Future Office Systems Requirements</title> 
</techreport> 
<article id ="Felici:1991" 
journal=``Macworld'' 
month =``sep'' 
year =``1991'' 
volume =``8'' 
s -"195&endash;201"> 
<author><name><first>James</first><last>Felici</last></name></author> 
<title>PostScript versus TrueType</title> 
</article> 
<techreport id ="Knuth:HEB" 
institution-``Department of Computer Science, Stanford University'' 
address -``Stanford, CA 94305'' 
month =``sep'' 
year -``1983'' 
number =``STAN-CS-83-980''> 
<author><name><first>Donald E.</first><last>Knuth</last></name></author> 
<title>The &WEB; System of Structured Documentation</title> 
</techreport> 
<phdthesis id ="Liang:1983" 
school-``Stanford University'' 
address=``Stanford, CA 94305'' 
month =``jun'' 
year =``1983''> 
<author><name><first>Franklin Mark</first><last>Liang</last></name></author> 
<title>Hord Hy-phen-a-tion by Com-pu-ter</title> 
<note>Also available as Stanford University, Department of 
Computer Science Report No. STAN-CS-83-977 
</note> 
</phdthesis> 
<article id-"Mittelbach-Schoepf:1990" 
journal="&j-TUGboat;" 
year=``1990'' 
volume=``11'' 
number-``2'' 
s="297&endash;305"> 
<author><name><first>Frank</first><last>Mittelbach</last></name> 
<and/> 
<name><first>Rainer</first><last>Sch&ouml;pf</last></name> 
</author> 
<title> 
The New Font Selection temdash; User Interface to Standard &LaTeX;" 
</title> 
</article> 
<incollection id -"Hood:color" 
crossref ="Roth:postscript" 
publisher="&AH;" 
year =``1988'' 

 
%%page page_452                                                  <<<---3
 
B.4 Examples of important DTDs 
74 pages ="201&endash;225"> 
<author><name><first>Pat</first><last>Hood</last></name></author> 
<title>PostScript Color Separation</title> 
<booktitle>Real World PostScript</booktitle> 
</incollection> 
<book id ="Roth:postscript" 
publisher="&AH;" 
year =``1988'' 
address ="&AH:adr;" 
ISBN =``0-201-06663-7''> 
<editor><name><first>Stephen E.</first><last>Roth</last></name></editor> 
<title>Real World PostScript</title> 
CCCC®®® ® ®®®%\l\l\l 
w--o»aoo\x§~.n§1.a~._ocoo\z3.’d 
</book> 
<inproceedings id ="Yannis:1991" 
series ="&j-TUGboat;" 
volume =``12'' 
s ="539&endash;548" 
organization="&TeX; Users Group" 
address =``Providence, Rhode Island'' 
month =``dec'' 
94 year =``1991''> 
95 <author><name><first>Yannis</first><last>Haralambous</last></name></author> 
96 <title>&TeX; and those other languages</title> 
97 <booktitle>1991 Annual Meeting Proceedings, Part 2, &TeX; Users Group, 
98 Twelfth Annual Meeting, Dedham, Massachusetts, July 15--18, 1991 
99 </booktitle> 
m0 <editor><name><first>Hope</first><last>Hamilton</last></name></editor> 
101 </inproceedings> 
1m <webdocument id=“xml-latest-news" 
103 organization=``0ASIS''> 
104 <title>SGMl and XML News</title> 
105 <author><name><first>Robin</first><last>Cover</last></name></author> 
mo <url>http://www.oasis-open.org/cover/sgmlnew.html</url> 
107 <note> 
108 Information about what is (relatively) new in the ‘SGML/XML web Page’ 
109 </note> 
H0 <annote>This page keeps you informed about the latest developments 
111 in the area of SGML, XML, and related areas. 
n2 </annote> 
H3 </webdocument> 
H4 </biblio> 
B.4.4.4 Summary for BIBTEX DTDs 
Based on the discussion in Sections B.4.4.2 and B.4.4.3, where we describe two 
versions of the BIBTEX DTD, it should be clear that the same data can be marked 
up in several equivalent ways without loss of information. This shows the flexibility 
of the XML approach, which also allows the designer of the DTD to provide hooks 
to extend and customize the data model and functionality of the original DTD. It, 
of course, depends very much on the application and, more important, the targeted 
user community to decide which form of specifying the information (that is, which 
kind of data schema or DTD) is more useful. 
B.4.5 I1}'I]3X-like markup, from DTD to printed document 
Several attempts have been made to develop an SGML application for E-TEX. The 
most successful is probably the qwertz Project by Tom Gordon of the German 

 
%%page page_453                                                  <<<---3
 
434 
Technical appendixes 
National Research Center for Computer Science [9 QWERTZ]. The latest version 
of the system converts HTML documents into BTEX with the help of a built-in 
Java SGML parser or James Clark’s nsgmls parser. The main advantage is having 
one source to generate both printed and Web documents, but, since the DTD is 
static, you cannot add your own macros, and math and pictures are treated specially. 
The main user of this system was, until recently, the Linux Documentation Project. 
However, this project now has adopted the SGMLtoo1s package, which is based on 
the DocBook DTD [-> SGMLTOOLS]. 
In this section we first build a mini-DTD for BTEX, which implements the 
structural elements of the I§}TEX language. This leaves aside a lot of the typographic 
elements, as well as the details of mathematics, which were discussed in Chapter 8 
in the framework of a more complete implementation. We then use the markup 
scheme defined by that DTD to write a multilingual sample document, that we will 
print with an XSL style sheet by translating the XML into BTEX. 
B.4.5.1 The I§1]5X-based DTD 
For our DTD model (minilatex . dtd), whose complete source text follows, we do 
not try to follow BTEX too closely, since in some cases BTEX confuses structural 
and presentation elements (so does HTML in most cases). For instance, sectioning 
elements become real containers, with a title, followed by paragraph-like material and floats (line 139) and lower-level sectioning elements (lines 142-155). Thus 
sectioning commands are structural only and indicate the subdivisions of the document, while their title must be specified separately. 
It is sometimes difficult to decide where to include a given BTEX element in 
the DTD. If}TEX has some elements that can occur almost anywhere, yet at the same 
time they have the constraint that they cannot be nested (for instance, footnotes and 
marginpars). We have chosen to ignore these elements and leave them under full 
user control. Such elements, if needed, can be inserted explicitly by extending the 
DTD, and it is left up to the user to exploit them wisely. Our sample file, presented 
in Section B.4.5 .3, will show you how you can achieve this extension mechanism. 
Copying good practice introduced in the HTML4 DTDs, we let most elements 
carry an id attribute of type ID so that they can be the target of cross-references 
(following), as well as style and class attributes (lines 3-5). The latter two come 
in handy when you want to visualize XML document elements as we explained in 
Chapter 7. We also provide an attribute xml : lang (line 7), which lets you specify 
the language in which a given part of the text is written. This is useful for spell 
checking, hyphenation algorithms, typographic rendering, and so on. 
The cross-reference system is fully based on XML’s identifier mechanism (the 
ID and IDREF token types for attributes, see Section 6.5.4.2). As explained earlier, 
we allow an id attribute on most elements, and, as seen in lines 43-50, the cite, 
ref, and ref elements all have an idref attribute. Taken together, we can 
reference almost any document element in a straightforward way. 
In writing the DTD, we have tried to keep the DTD somewhat modular by 

 
%%page page_454                                                  <<<---3
 
B.4 Examples of important DTDs 
435 
defining parameter entities to structure the information (fontchange, xref, and 
so on, on lines 10-27). To make it easy to extend the DTD, each of these parameter 
entities has a companion ending in .new (fontchange .new, xref .new, and so on). 
These are defined as empty strings, but the user can redefine them in the internal 
subset of the document instance or in another DTD. Thus new elements can be 
conveniently added at various points of the document hierarchy. 
Most of the entries in the source listing that follows should be simple to understand. The handling of mathematics (lines 162-172) refers to two further DTD 
fragments that are discussed separately in Section B.4.5 .2. 
<!ENTITY 
<!ENTITY 
<!ENTITY 
o w m u o m a u N _ 
<!ENTITY 
<!ENTITY 
<!ENTITY 
<!ENTITY 
<!ENTITY 
<!ENTITY 
<!ENTITY 
<!ENTITY 
<!ENTITY 
<!ENTITY 
<!ENTITY 
<!ENTITY 
<!ENTITY 
<!ENTITY 
<!ENTITY 
<!ENTITY 
<!ENTITY 
<!ENTITY 
<!-- The 
<!ENTITY 
<!ELEMENT 
<!ATTLIST 
<!ELEMENT 
<!ATTLIST 
<!ELEMENT 
<!ATTLIST 
<!-- minilatex.dtd: XML version of a subset of LaTeX --> 
<!-- Most elements have the following attributes 
-> 
"id ID #IMPLIED 
class CDATA #IMPLIED 
style CDATA #IMPLIED"> 
Z all 
<!- Most non-empty elements have this language attribute --> 
Z i18n 
Z basic 
"xmlzlang NMTUKEN #IMPLIED"> 
"Zall; Zi18n;"> 
<!-- Declaration of parameter entities for structuring the DTD --> 
Z fontchange.new "“> 
Z fontchange "emphltextbfltextscltextsfltextslltexttt Zfontchange.new;"> 
Z misc.new ""> 
Z misc "urllquadlhspacelvspacelincludegraphicsltaglent Zmisc.new;"> 
Z xref.new ""> 
Z xref "citelpagereflref Zxref.new;"> 
Z mathobj.new ""> 
Z mathobj ``displaymathlinlinemathlequationleqnarray'' > 
Z inline.new ""> 
Z inline "Zfontchange;lquoteltabular|Zmisc;|Zxref;|Zmathobj; Zinline.new;"> 
Z list.new ""> 
Z list "lalistldescriptionlenumeratelitemizelbibliography Zlist.new;"> 
Z preformat.new ""> 
Z preformat "verbatimlverse Zpreformat.new;"> 
Z likepara.new ""> 
Z likepara "Zlist;IquotationlalignlZpreformat; Zlikepara.new;"> 
Z floats.new ""> 
Z floats "figureltable Zfloats.new;"> 
top level document declarations 
<!ELEMENT document (frontmatter?,bodymatter,backmatter?)> 
<!ATTLIST document Zi18n; 
class CDATA ``article''> 
<!ELEMENT frontmatter (title,author?,date?,abstract?,keywords?)> 
<!ELEMENT bodymatter 
<!ELEMENT backmatter 
<!-- fontchanges --> 
<!ELEMENT 
<!ELEMENT 
<!ELEMENT 
<!ELEMENT 
<!ELEMENT 
<!ELEMENT 
<!-- cross-references --> 
((parlZlikepara;|Zfloats;[partIchapter!section)*,appendix*)> 
(indexlglossary)*> 
(#PCDATAlZinline;)x> 
(#PCDATAlZinline;)*> 
(#PCDATAlZinline;)*> 
(#PCDATAlZinline;)*> 
(#PCDATAlZinline;)*> 
(#PCDATA|Zinline;)*> 
emph 
textbf 
textsc 
textsf 
textsl 
texttt 
"refid IDREF #REQUIRED"> 
EMPTY> 
Zidref;> 
EMPTY> 
Zidref;> 
EMPTY> 
Zidref;> 
Z idref 
cite 
cite 
ref 
ref 
ref 
ref 

 
%%page page_455                                                  <<<---3
 
436 
Technical appendixes 
lll 
IIZ 
I13 
114 
115 
<!-- quoted material inline and displayed --> 
<!ELEMENT quote (#PCDATA|%inline;|par)*> 
<!ATTLIST quote Kbasic;> 
<!ELEMENT quotation (#PCDATA|Zinline;Ipar)*> 
<!ATTLIST quotation %basic;> 
<!-- lists --> 
<!ELEMENT description ((term*,item*)+)> 
<!ELEMENT enumerate ((term*,item*)+)> 
<!ATTLIST enumerate %basic;> 
<!ELEMENT itemize ((term*,item*)+)> 
<!ATTLIST itemize %basic;> 
<!ELEMENT lalist ((term*,item*)+)> 
<!ATTLIST lalist Zbasic;> 
<!ELEMENT term (#PCDATA|Zinline;)x> 
<!ATTLIST term %basic;> 
<!ELEMENT item (#PCDATA|%inline;|par|Zlikepara;)*> 
<!ATTLIST item Zbasic;> 
<!ELEMENT bibliography (bibitem)*> 
<!ATTLIST bibliography %basic;> 
<!ELEMENT bibitem (#PCDATAI%inline;IparIZlikepara;)*> 
<!ATTLIST bibitem %basic;> 
<!-- low-level bits and pieces --> 
<!ELEMENT url EMPTY> 
<!ATTLIST url name CDATA #REQUIRED> 
<!ELEMENT quad EMPTY> 
<!ELEMENT hspace EMPTY> 
<!ATTLIST hspace dim CDATA #REQUIRED> 
<!ELEMENT vspace EMPTY> 
<!ATTLIST vspace dim CDATA #REQUIRED> 
<!ELEMENT tag (#PCDATA)> 
<!ELEMENT ent EMPTY> 
<!ATTLIST ent value CDATA #REQUIRED 
name CDATA #REQUIRED> 
<!-- everything that can go into a paragraph --> 
<!ELEMENT par (#PCDATAlZinline;|Zlikepara;)*> 
<!ATTLIST par %basic;> 
<!-- ``floats'' and their contents --> 
<!ELEMENT figure (#PCDATA|parl%inline;l%likepara;Iincludegraphicslcaption)*> 
<!ATTLIST figure %basic;> 
<!ELEMENT table (#PCDATAlpar|%inline;lllikepara;|tabularlcaption)*> 
<!ATTLIST table %basic;> 
<!ELEMENT includegraphics EMPTY> 
<!ATTLIST includegraphics Zbasic; 
file CDATA #REQUIRED 
width CDATA #IMPLIED 
height CDATA #IMPLIED 
bb CDATA #IMPLIED 
scale CDATA ".5" 
angle CDATA #IMPLIED> 
<!ELEMENT caption (#PCDATAlparIZinline;)*> 
<!ATTLIST caption %basic;> 
<!-- tabular material --> 
<!ELEMENT tabular (hlinelrow)*> 
<!ATTLIST tabular Zbasicg 
preamble CDATA #REQUIRED 
width CDATA #IMPLIED 
border CDATA #IMPLIED> 
<!ELEMENT row (cell)x> 
<!ATTLIST row %basic;> 
<!ELEMENT hline EMPTY> 
<!ELEMENT cell (#PCDATA|Zinline;)*> 
<!ATTLIST cell Zbasic; 
rowspan CDATA ``1'' 
colspan CDATA ``1'' 
align (leftlcenterlright) ``center''> 

 
%%page page_456                                                  <<<---3
 
B.4 Examples of important DTDS 
437 
U6 
U7 
U8 
H9 
H1 
IR 
H3 
H4 
H5 
H6 
H7 
U9 
U0 
1n 
1B 
1% 
13 
B6 
IN 
B8 
U9 
U4 
U8 
U9 
1m 
1m 
1Q 
18 
1M 
1“ 
10 
lfl 
1m 
IN 
IN 
13 
U4 
U5 
1% 
1W 
1% 
<!-- verbatim material --> 
<!ELEMENT verbatim (#PCDATA)> 
<!ATTLIST verbatim Zbasic; 
xml:space (defaultlpreserve) ’preserve’> 
verse (#PCDATA|%inline;)*> 
verse Zbasic; 
xml:space (defaultipreserve) ’preserve’> 
<!-- things that go in the frontmatter --> 
newline EMPTY> 
title (#PCDATA|%inline;|thanks|newline)*> 
title Zbasic;> 
author (#PCDATA|%inline;|namelthankslinst)*> 
name (#PCDATAl%inline;)*> 
name %basic;> 
thanks (#PCDATAlZinline;|newline)*> 
thanks %basic;> 
<!ELEMENT 
<!ATTLIST 
<!ELEHENT 
<!ELEMENT 
<!ATTLIST 
<!ELEMENT 
<!ELEMENT 
<!ATTLIST 
<!ELEMENT 
<!ATTLIST 
<!ELEMENT 
<!ATTLIST 
<!ELEMENT 
<!ATTLIST 
<!ELEMENT 
<!ATTLIST 
inst (#PCDATAI%inline;lnewline)*> 
inst Zbasic;> 
date (#PCDATA)> 
date Zbasic;> 
abstract (par+)> 
abstract %basic;> 
<!-- structuring --> 
<!ENTITY Z sect 
"stitle, (parl%likepara;lZfloats;)* "> 
<!ELEMENT stitle (#PCDATAl%inline;)*> 
<!ATTLIST stitle Zbasic;> 
<!ELEMENT part (%sect;, (chapterlsectionlsubsectionlsubsubsection)#)> 
<!ATTLIST part %basic;> 
<!ELEHENT chapter (Zsect;, section*)> 
<!ATTLIST chapter %basic;> 
<!ELEMENT section (%sect;, subsection*)> 
<!ATTLIST section %basic;> 
<!ELEMENT subsection (%sect;, subsubsection~)> 
<!ATTLIST subsection %basic;> 
<!ELEMENT subsubsection (%sect;, paragraph*)> 
<!ATTLIST subsubsection Zbasic;> 
<!ELEMENT paragraph (Zsect;, subparagraph*)> 
<!ATTLIST paragraph %basic;> 
<!ELEMENT subparagraph (Zsect;)> 
<!ATTLIST subparagraph %basic;> 
<!ELEMENT appendix EMPTY> 
<!-- backmatter: index and glossary --> 
<!ELEMENT index (par|%likepara;)*> 
<!ATTLIST 
index %basic;> 
<!ELEMENT glossary (parl%likepara;)*> 
<!ATTLIST glossagry %basic;> 
<!-- LaTeX math constructs --> 
<!ENTITY Z LaTeXmath 
``INCLUDE''> 
<!ENTITY Z latexmath.dtd SYSTEM "latexmath.dtd"> 
<![ ZLaTeXmath; [ 
%latexmath.dtd; 
]]> 
<!ENTITY Z MathML 
``IGNORE''> 
<!ENTITY Z latexmml.dtd SYSTEM "latexmml.dtd"> 
<![ ZMathML; [ 
Zlatexmml.dtd; 
]]> 
<!-- Basic XML entities --> 
<!ENTITY lt "&#60;"> <!`` ''<" “> 
<!ENTITY gt "&#s2;"> <!-- ">" --> 
<!ENTITY amp "&#38;"> 
<!ENTITY apos "&#39;"> 
<!ENTITY quot "&#34;"> 
<!-_ nan --> 
<!__ nzn __> 
(5-- IN! --> 

 
%%page page_457                                                  <<<---3
 
438 
Technical appendixes 
B.4.5.2 BIEX math and our DTD 
As we mentioned in the previous section, we have not really addressed the subject 
of the mathematical markup in the main DTD file minilatex . dtd. 
Generally speaking, either we can use TEX markup for describing math (as 
we do in our sample document in the next section), or we can decide to use 
MathML, in which case we must include the full DTD for that language. In fact, 
as already mentioned, we have provided hooks for handling these two cases in 
our minilatex.dtd file (lines 163-172). Indeed, two entities, “LaTeXmath" and 
“MathML," control which of the two math models (BTEX or MathML markup) is to 
be used. By default we use BTEX markup (the value of LaTeXmath is INCLUDE (line 
163), whereas that of MathML is set to IGNORE (line 168)). 
Let us see what happens in each case. First we consider the simpler situation, 
where we decide to use BTEX-based markup (this is the default for our DTD). 
Then line 166 is active, and we include the file 1atexmath.dtd into the master 
DTD. This file follows: 
<!-- latexmath.dtd +++ inline and display math --> 
1 
2 < 9 ELEMENT inl inemath (#PCDATA) > 
3 < !A'1'TLIST inlinemath Za11;> 
4 < !ELEMENT displaymath (#PCDATA)> 
3 <!ATTLIST displaymath '/.al1;> 
6 < ! ELEMENT equation (#PCDATA) > 
7 < !A'1'TLIST equation '/.a11;> 
3 < ! ELEMENT eqnarray (#PCDATA) > 
9 < !ATTLIST eqnarray %al1;> 
10 < 9 ELEMENT align (#PCDATA) > 
11 < 9 ATTLIST align '/.a11 ; > 
This file latexmath.dtd contains the usual BTEX math containers, and we declare their contents as #PCDATA, so that we do not impose any constraints on their 
internal structure. Inside XML files the characters <, >, and & have to be escaped; 
therefore it is more practical to put the data inside the math elements inside a CDATA 
section, where one can use all BTEX characters without needing to use special precautions (in particular, see lines 5 1-5 3 of the sample XML file in Section B.4.5 .3). 
If, on the other hand, we decide to use MathML markup by setting the value of 
the entity MathML to INCLUDE (and that of LaTeXmath to IGNORE), we include the 
file latexmml . dtd, shown here: 
1 <!-- latexmml.dtd --> 
z <!ELEMENT equation (math)*> 
3 <!A'1'TLIST equation '/.all;> 
4 <!ELEMENT displaymath (math)*> 
5 <!ELEMENT inlinemath (math)*> 
6 <!ELEMENT subeqn (math)*> 
7 <!A'1'TLIST subeqn '/.all;> 
8 <!ELEMENT eqnarray (subeqn)*> 
9 <!ATTLIST eqnarray '/.a11; 
10 number (yeslno) ``yes''> 
ll <lELEMENT align ('/.inline;)#> 
12 <!ATTLIST align Zall; > 
14 <!-- ISO and MathML DTDs and entities --> 

 
%%page page_458                                                  <<<---3
 
B.4 Examples of important DTDs 
439 
\l\\l\ \l\\l\\l\ 3333 33 mmm 
3 o m 3 w N - 3 3 m N a m 3 W N 3 3 3 $ 3 3 m 3 W 3 2 3 
Since the MathML DTD uses the math element as global container, in the file 
latexmml . dtd we must declare the E-TEX containers as consisting of math elements 
(lines 2 and 4-6). The rest of the above DTD file contains a set of entity definitions 
(lines 14-40), as well as a reference to the MathML DTD itself (line 42). The files 
defined by these entities are all included at the end (lines 44-5 7). 
The mini1atex.dtd and latexmml . dtd DTD files can be used with the XML 
file derived from the E-TEX test file used throughout this book and shown in Ap<!--Added Math Symbols: 
<!ENTITY Z isoamsae.dtd 
<!--Added Math Symbols: 
<!ENTITY Z isoamsbe.dtd 
<!--Added Math Symbols: 
<!ENTITY Z isoamsce.dtd 
<!--Added Math Symbols: 
<!ENTITY Z isoamsne.dtd 
<!--Added Math Symbols: 
<!ENTITY Z isoamsoe.dtd 
<!--Added Math Symbols: 
<!ENTITY Z isoamsre.dtd 
<!--General Technical-<!ENTITY Z isoteche.dtd 
Arrows--> 
SYSTEM “isoamsae.dtd“> 
Binary Operators--> 
SYSTEM "isoamsbe.dtd"> 
Delimiters--> 
SYSTEM “isoamsce.dtd"> 
Negated Relations--> 
SYSTEM "isoamsne.dtd"> 
Ordinary--> 
SYSTEM "isoamsoe.dtd"> 
Relations--> 
SYSTEM "isoamsre.dtd“> 
> 
SYSTEM "isoteche.dtd"> 
<!--Numbers and Currency symbols--> 
<!ENTITY Z isonume.dtd SYSTEM "isonume.dtd“> 
<!--MathML Aliases (From ISO PUB,DIA,NUM)~~> 
<!ENTITY Z mmaliase.dtd 
<!--Greek Symbols--> 
<!ENTITY Z isogrk3e.dtd 
<!--Math Script Font--> 
<!ENTITY Z isomscre.dtd 
SYSTEM "mmaliase.dtd“> 
SYSTEM “isogrk3e.dtd“> 
SYSTEM "isomscre.dtd"> 
<!--Math Open Face Font--> 
<!ENTITY Z isomopfe.dtd 
<!--MathML Entities--> 
SYSTEM "isomopfe.dtd"> 
<!ENTITY Z mmlent.dtd SYSTEM "mmlent.dtd“> 
<!--Main MathML DTD --> 
<!ENTITY Z mathml.dtd 
Zmathml.dtd; 
Zisoamsae.dtd; 
%isoamsbe.dtd; 
Zisoamsce.dtd; 
Zisoamsne.dtd; 
Zisoamsoe.dtd; 
Zisoamsre.dtd; 
Zisoteche.dtd; 
%isonume.dtd; 
Xmmaliase.dtd; 
%isogrk3e.dtd; 
%isomscre.dtd; 
%isomopfe.dtd; 
Kmmlent.dtd; 
SYSTEM "mathml.dtd“> 
pendix A.1. Following is the start of that file: 
\!O\\II3\oJru-<?xml version="1.0"?> 
<!DOCTYPE document SYSTEM “minilatex.dtd“[ 
<!ENTITY Z MathML ``INCLUDE''> 
<!ENTITY % LaTeXmath ``IGNORE''> 
<!ENTITY Z inline.new 
<!ENTITY aacute "&#xOOE 
“lmath“> 
1;“> 
<!ENTITY 0verBar "[0verBar]“> 

 
%%page page_459                                                  <<<---3
 
440 
Technical appendixes 
8 <!ENTITY negationslash "/"> 
9 ]> 
10 <document> 
11 <front1natter> 
12 <title>Si111ulation of Energy Loss Straggling</title> 
13 <author>Maria Physicist</author> 
14 <date> January 14. 1999</date> 
15 </front111atter> 
16 <body'matter> . . . 
The beginning of the file defines the DTD to be used and sets the values of the entities MathML and LaTeXmath appropriately. We also define the math element needed 
as a container with MathML markup by introducing it in the inline .new entity 
parameter, so that it will be included in the inl ine elements in the minilat ex . dtd 
(line 19). The remaining lines (6-8) define entities that are referenced in the test 
document but are not defined in the already loaded entity reference sets. 
B.4.5 .3 A sample document 
Let us now show how we can use the minilatex DTD, which we introduced in 
Section B.4.5.1. In the present section we look at a sample XML file, marked up 
according to that DTD. The contents of the document are partly based on the text 
of Figures 9.2 and 9.3 in the Babel chapter of The ETEX Companion (Goossens et al. 
(1994)). 
On line 1 you notice the <?x1nl processing instruction that declares the use 
of the Latin 1 (ISO-8859-1) encoding in the document (used by the German text 
on lines 60-99 and 128-131 and the French text on lines 100-126 and 132-135). 
We also define (lines 3-8) a few entities that occur in the document text and are not 
defined in the external DTD subset. They are customized for translation into E-TEX. 
For translation into HTML or another format, these entities should be defined 
appropriately. The same holds for certain sections of the text, whose context we 
put inside CDATA sections, since we know we will be using If}'1];,X as back-end (for 
example lines 40 and 41, as well as the math segments on lines 51-53). In fact, we do 
not need to use a CDATA section to prepare output directly for E-TEX, since we can 
use XML entities to represent special characters, as shown on lines 42-43. It is only 
a matter of convenience as to which approach you want to take. Note, however, 
that we preceded the 81 character with a \ (lines 40, 42, and 43), since we know that 
ampersands must be escaped for use as a literal in E-TEX. 
At the end of the internal subset we define a footnote element that is to be 
included in the inline parameter entity (line 19 of the minilatex DTD), so that 
it will become available for use in the document instance that follows. Therefore 
we enter it into the inline.new parameter entity (line 9) and declare its contents 
as #PCDATA and par elements (line 10). However, par elements, according to line 
85 of the file minilatex .dtd, refer to the inline entity. Thus since our footnote 
element will also become part of inline, it is our responsibility to make sure that 
we do not inadvertently nest footnote elements (the DTD does not disallow this, 
but E-TEX forbids it; see Section 7.6.5.4 for a way to deal with this in XSL). 

%==========460==========<<<---2
 
%%page page_460                                                  <<<---3
 
B.4 Examples of important DTDs 
The two main parts of the document, its frontmatter (lines 13-19) and its 
bodymatter (lines 20-137), are clearly discernible. Similarly thanks to the sectioning tags (<section>, <subsection>, and so on) and their nesting, the hierarchical 
smlcture of the sectioning element tree can be easily reconstructed and checked for 
consistency. 
Various languages inside the same document are easily supported. The present 
document starts with a section in English (this is the default language for the document as declared on line 12 of the document instance), is followed by a section 
in German (starting on line 60 of the document instance), and then one in French 
(starting on line 100). In addition the two bibliographic bibitem elements have different language attributes (lines 128-131 are in German, while lines 132-135 are 
in French; the surrounding thebibliography element is in English--the default). 
The includegraphics command and its attributes are defined on lines 93-99 
in the minilatex DTD. It is used on line 74 of the example, where we want to 
include an Encapsulated PostScript image inside a figure element (lines 73-76). 
The rest of the markup should be quite straightforward, and we will not comment on the remaining parts of the sample file. 
Nevertheless, we want to draw your attention to an inportant point. As we 
mentioned at the beginning of this section, on lines 51-53 we directly use E-TEX’s 
math notation inside the inl inemath elements (this corresponds to the default setting in the minilat ex DTD). Although we use CDATA sections to escape the need 
to use entity references for certain characters, XML or XSL processors, which can 
output only valid XML files, will represent all <, >, and & symbols by their XML 
equivalents &lt;, &gt ;, and &amp;. Thus a postprocessor will have to transform 
these characters into their E-TEX equivalents, as shown at the end of the next sect1OI‘1. 
<?xml version="1.0" encoding-“ISO-8859-1"?> 
<!DOCTYPE document SYSTEM "minilatex.dtd"[ 
<!ENTITY LaTeX "\LaTeX{}"> 
<!ENTITY Tex “\TeX{}“> 
<!ENTITY dots “\dots"> 
<!ENTITY endash “--"> 
<!ENTITY emdash "---“> 
<!ENTITY nbsp "~"> 
<!ENTITY Z inline.new “|footnote"> 
<!ELEMENT footnote (#PCDATA|par)*> 
]> 
<document class=“article“ xm1:lang=“en“> 
<frontmatter> 
<title>The &LaTeX; DTD and multiple languages</title> 
<author><name>Michel Goossens</name> 
<thanks>Partly from an example in <emph>The &LaTeX; Companion</emph> 
</thanks></author> 
<date>August 4th, 1998</date> 
</frontmatter> 
<bodymatter> 
<section id=“sec-en“> 
<stitle>The basic princip1es</stitle> 
<pa.r> 
This is an example input file. We start in English to show the 
principle. You should especially pay attention that we have used 
N~NN-.--._._..-..-._._._.-._ 
xi-Juuatxa--O\Ow\lO\I-.c>\.aIv--o~ow\l0\v.n.£>\oJv~a._441 

 
%%page page_461                                                  <<<---3
 
442 
Technical appendixes 
slightly different notation for some of the common &LaTeX; constructs, 
such as the dashes, which come in three sizes: an intra-word dash, a 
medium dash for number ranges like 1&endash;2, and a punctuation 
dash&emdash;like this. Text can be emphasized as <emph>shown 
here</emph>. An ellipsis is made with &dots; Footnotes<footnote>This 
is a simple footnote.<par>It can also contain <texttt>par</texttt> 
elements.</par></footnote> are tricky constructs, since one must be 
careful not to nest them. 
</par> 
<subsection> 
<stitle>Dealing with special characters</stitle> 
<pa1'> 
XML has a different set of reserved characters than &LaTeX;, in 
particular, when you want to use any of the three characters 
<texttt><![CDATA[\&]]></texttt>, <texttt><![CDATA[<]]></texttt>, 
and <texttt><![CDATA[>]]></texttt>, you should enter them as 
<texttt>\&amp;amp;</texttt>, <texttt>\&amp;lt;</texttt>, and 
<texttt>\&amp;gt;</texttt>, respectively. 
</par> 
</subsection> 
<subsection id=“sec-math“> 
<scitle>&LaTeX; and mathematical formulae</stitle> 
<par> 
&LaTeX; and <emph>a fortiori</emph> &TeX; are very good at typesetting 
mathematical formulae, like 
<inlinemath><![CDATA[x3 y + z < 7]]></inlinemath> or 
<inlinemath><! [CDATA[a_{1} > x~{2n} + y"{2n} > x’]]></inlinemath> or 
<inlinemath><![CDATA[(A, B) = \sum_{i} a_{i} b_{i}]]></inlinemath>. 
Do not forget that for reasons of consistency, if you want to refer to 
a variable in one of the formulae, such as the symbol 
<inlinemath>x</inlinemath>, you must also use math mode in the text. 
</par> 
</subsection> 
</section> 
(section xml 1ang="de“> 
<stitle>Beispiel eines Textes in deutscher Sprache</stitle> 
<subsection> 
<stitle>Eine EPS Abbildung</stitle> 
<pa_r> 
Dieser Abschnitt zeigt, wie man eine Postscript-Abbildung 
<cite refid=“bib-PS“/> in ein Dokument einbinden kann. 
Abbildung&nbsp;<ref refid=``fig-psfig''/> wurde mit dem Befehl 
<verbatim> 
<![CDATA[\includegraphics[width=“3cm“]{fi1e=“colorcir.eps}]]> 
</verbatim> 
in den Text aufgenommen. 
</par> 
<figure id=“fig-psfig“> 
<includegraphics width="3cm“ fi1e=“colorcir.eps“/> 
<caption xml:lang=``de''>Ein EPS Bild</caption> 
</figure> 
</subsection> 
<subsection> 
<stitle>Beispiel einer Tabelle</stitle> 
<par>Die Tabel1e&nbsp;<ref refid=“tab-exag“/> auf Seite&nbsp;<pageref 
refid=``tab-exag''/> zeigt eine Tabelle. 
</par> 
<table id=“tab-exag“> 
(caption xm1:1ang="de“> 
Eingabe der deutschen Zusatzzeichen in &LaTeX;</caption> 
<tabular preamble=“ccccccc"> 
<row> 
<cell><texttt>&#34;a</texttt>&nbsp;a</cell> 
<cell><texttt>&#34;A</texttt>&nbsp;§</cell> 
<cel1><texttt>&#34;o</¢exttt>&nbsp;o</cell> 

 
%%page page_462                                                  <<<---3
 
B.4 Examples of important DTDS 
92 
93 
94 
95 
96 
97 
98 
99 
I00 
101 
102 
I03 
105 
106 
107 
108 
110 
111 
112 
113 
114 
I15 
116 
N7 
118 
119 
1Z0 
lZ| 
JZZ 
1Z3 
124 
IZS 
1Z6 
127 
128 
129 
I30 
131 
132 
133 
134 
135 
136 
137 
138 
443 
<ce11><texttt>&#34;D</texttt>&nbsp;5</ce11> 
<ce11><texttt>&#34;u</texttt>&nbsp;fi</ce11> 
<ce11><texttt>&#34;U</texttt>&nbsp;U</ce11> 
<ce11><texttt>&#34;s</texttt>&nbsp;&#223;</ce11> 
</row> 
</tabu1ar> 
</tab1e> 
</subsection> 
</section> 
<section xm1:1ang=“fr"> 
<stitle>Continuation du texte en franqais</stitle> 
<subsection id="sec-1ist“> 
<stitle>Traiter les 1istes</stitle> 
<par> 
Les listes sont utilisées fréquemment pour structurer ou mettre 
en évidence certains éléments d’un document (voir <cite refid=“bib-Liste“/>). 
</par> 
<itemize> 
<item>Ceci est le premier élément d’une liste non-ordonnée. Cheque élément 
de ce type de liste est précédé d’un signe distinctif, comme une 
puce, un tiret, etc. 
</item> 
<item>Ce second élément de la meme liste contient une 1iste de 
<emph>description</emph> imbriquée. 
<description> 
<term>XMl</term> 
<item>Meta langage pour définir des classes de documents</item> 
<term>xLL</term> 
<item>Langage pour définir des hyperliens entre différentes 
parties de documents xML</item> 
</description> 
Nous continuons notre texte a 1’intérieur de la premiere liste. 
</item> 
</itemize> 
</subsection> 
</section> 
<bib1iography> 
<bibitem id=“bib-PS" xm1:1ang=``de''> 
Adobe Inc. <emph>PostScript Handbuch (2. Auf1age)</emph> 
Addison-Wesley (Deutschland) GmbH, Bonn, 1991. 
</bibitem> 
<bibitem id="bib-Liste“ xm1:1ang=``fr''> 
Michel Goossens. Personnaliser 1es listes &LaTeX;. 
<emph>Cahiers GUTenberg</emph>, 17:32&endash;48, mai 1994. 
</bibitem> 
</bib1iography> 
</bodymatter> 
</document> 
B.4.5 .4 Preparing a printed version 
We can validate the sample file minilatexexa. xml, presented in the previous section, with any of the many available XML parsers against the DTD mini lat ex . dtd. 
As an example we can run it through the xml4 j parser, discussed in Section 6.6.5.6. 
xml4j.sh minilatexexa.xml 
minilatex.dtd: 18, 25: Warning: Entity name, "inline.new", already 
defined. This declaration will be ignored. 

 
%%page page_463                                                  <<<---3
 
444 
Technical appendixes 
This validating parser shows that our sample document is indeed valid relative to 
the minilatex DTD. Moreover, the parser warns us that the definition of the entity inline .new in the DTD is ignored. This is precisely what we want since, as 
explained in Section 6.5.4, the internal DTD subset is read first so that definitions 
in the internal subset are honored and override those in the external DTD (this is 
true only for entity and attribute definitions, not for elements, which thus cannot 
be redefined with respect to the external DTD). 
Another interesting application is to generate a printable version of our file. 
We can use an XSL style sheet minilatex .xsl, which we show below. Its purpose 
is to translate the XML instance minilatexexa.xml into a ETEX file. 
<?xm1 version=’1.0’?> 
<!-- mini1atex.xs1 --> 
<xs1:stylesheet xmlns:xs1="http://www.w3.org/TR/WD-xsl" 
xm1ns="http://uww.tug.org/latex“ 
default-space=“strip" 
result-ns=“"> 
<xs1:macro name=``label''> 
<xs1:if test="../@id"><xs1:text>\label{</xs1:text> 
<xs1:va1ue-of se1ect="../Qid"/><xs1:text>}</xs1:text></xsl:if> 
<xs1:text>} 
</xs1:text> 
</xs1:macro> 
<xs1:temp1ate match=“document"> 
<xs1:text>\documentclass[]{</xs1:text> 
<xs1:va1ue-of se1ect="@c1ass"/> 
<xs1:text>} 
\usepackage[dvips]{graphicx} 
\usepackage[T1]{fontenc} 
\begin{document} 
</xs1:text> 
<xs1:app1y-templates/> 
<xs1:text>\end{document} 
</xs1:text> 
</xs1:temp1ate> 
<!-- ================ Frontmatter element ======================= --> 
<xs1:temp1ate match=``frontmatter''> 
<xs1:app1y-templates/> 
<xs1:text> 
\maketitle 
</xs1:text> 
</xs1:temp1ate> 
<xs1:temp1ate match="frontmatter/title"> 
<xs1:text> 
\title{</xs1:text> 
<xs1:app1y-temp1ates/> 
<xs1:text>}</xs1:text> 
</xs1:temp1ate> 
<xs1:temp1ate match="frontmatter/author") 
<xs1:text> 
\author{</xs1:text> 
<xs1:app1y-temp1ates/> 
<xs1:text>} 
</xs1:text> 
</xs1:temp1ate> 
<xs1:temp1ate match="frontmatter/author/name"> 
<xs1:app1y-templates/> 
wwwwww-----_.....-_.._.__._.-._. 
vubwN--O0o0\Imvubwr~a>-O~Oa0\Io\\IIJr.wru»-O~Oac\Io\vu&wN-% Q fl & 3 i 3 S 3 8 S 3 3 3 

 
%%page page_464                                                  <<<---3
 
B.4 Examples of important DTDs 
111 
113 
114 
</xs1:temp1ate> 
<xs1:temp1ate match="trontmatter/author/thanks"> 
<xs1:text> 
\thanks{</xs1:text> 
<xs1:app1y-templates/> 
<xs1:text>}</xs1:text> 
</xs1:temp1ate> 
<xs1:temp1ate match="frontmatter/author/inst"> 
<xs1:text> 
\thanks{</ks1:text> 
<xs1:app1y-templates/> 
<xs1:text>}‘/xs1:text> 
</xs1:temp1ate> 
<xs1:temp1ate match="frontmatter/date"> 
<xs1:text> 
\date{</xs1:text> 
<xs1:app1y-templates/> 
<xs1:text>} 
</xs1:text> 
</xs1:temp1ate> 
<xs1:temp1ate match=“frontmatter/abstract") 
<xs1:text> 
\begin{abstract} 
</xs1:text> 
<xs1:app1y-templates/> 
<xs1:text> 
\end{abstract} 
</xs1:text> 
</xs1:temp1ate> 
<xs1:temp1ate match="frontmatter/keywords"> 
<xs1:text>\keyuords{</xsl:text> 
<xs1:app1y-temp1ates/> 
<xs1:text>}</xs1:text> 
</xs1:temp1ate> 
<!-- =2:-=2:-====--8--===-= Bodymatter element ===--==--=---==--== --> 
<xs1:temp1ate match-"bodymatterlpartlchapterlsectionlsubsectionl 
subsubsectionlparagraphlsubparagraph"> 
<xs1:app1y-templates/> 
</xs1:temp1ate> 
<1" :Il==3===II3="===‘ section hfiadillgs =I==3Il=3I-==II==IIl=II '-) 
<xs1:temp1ate match="part/stitle"> 
<xs1:text>\part{</xsl:text><xs1zapply-templates/> 
<xs1:invoke macro-``label''></xslzinvoke> 
</xs1:temp1ate> 
<xs1:temp1ate match="chapter/stitle"> 
<xs1:text>\chapter{</xsl:text><xs1:app1y-temp1ates/> 
<xs1:invoke macro-``label''></xslzinvoke> 
</xs1:temp1ate> 
<xs1:temp1ate match="section/stitle"> 
<xs1:text>\section{</xsl:text><xs1zapply-templates/> 
<xs1:invoke macro-``label''></xsl:invoke> 
</xs1:temp1ate> 
<xs1:temp1ate match-"subsection/stitle“> 
<xs1:text>\8ubsection{</xsl:text><xs1:app1y-temp1ates/> 
<xs1:invoke macro-``label''></xsl:invoke> 
</xs1:temp1ate> 
<xs1:temp1ate match="subsubsection/stitle"> 
<xs1:text>\subsubsection{</xsl:text><xs1:app1y-temp1ates/> 
<xs1:invoke macro=``label''></xsl:invoke> 
</xs1:temp1ate> 
<xs1:temp1ate match="paragraph/stitle"> 
<xs1:text>\paragraph{</xsl:text><xs1:app1y-temp1ates/> 
<xs1:invoke macro=``label''></xsl:invoke> 
</xs1:temp1ate> 
445 

 
%%page page_465                                                  <<<---3
 
446 Technical appendixes 
us <xs1:temp1ate match="subparagraph/stitle"> 
n6 <xs1:text>\subparagraph{</xsl:text><xs1:app1y-templates/> 
H7 <xs1:invoke macro=“label"></xs1:invoke> 
H8 </xs1:temp1ate> 
H9 <!-- - Font changes == 
no <xs1:temp1ate match=``emph''> 
n1 <xs1:text>\emph{</xs1:text> 
U2 <xs1:app1y-templates/> 
n3 <xs1:text>}</xs1:text> 
U4 </xs1:temp1ate> 
us <xs1:temp1ate match=``textbf''> 
n6 <xs1:text>\textbf{</xsl:text> 
U7 <xs1:app1y-templates/> 
us <xs1:text>}</xs1:text> 
no </xs1:temp1ate> 
no <xs1:temp1ate match=``textsc''> 
U1 <xs1:text>\textsc{</xsl:text> 
nz <xs1:app1y-temp1ates/> 
B3 <xs1:text>}</xs1:text> 
B4 </xs1:temp1ate> 
BS <xs1:temp1ate match=``textsf''> 
H6 <xs1:text>\textsf{</xsl:text> 
B7 <xs1:app1y-templates/> 
B8 <xs1:text>}</xs1:text> 
U9 </xs1:temp1ate> 
140 <xs1:temp1ate match=``texts1''> 
141 <xs1:text>\texts1{</xsl:text> 
142 <xs1:app1y-templates/> 
143 <xs1:text>}</xs1:text> 
H4 </xs1:temp1ate> 
145 <xs1:temp1ate match=``texttt''> 
146 <xs1:text>\texttt{</xsl :text> 
1w <xs1:app1y-templates/> 
148 <xs1:text>}</xs1:text> 
M9 </xs1:temp1ate> 
150 < ! " =================== Cross-references 
H1 <xs1:temp1ate match=``cite''> 
U2 <xs1:text>\cite{</xsl:text> 
153 <xs1:va1ue-of se1ect="@refid"/> 
154 <xs1-.text>}</xs1:text> 
us </xs1:temp1ate> 
U6 <xs1:temp1ate match=``pageref''> 
U7 <xs1:text>\pageref{</xsl:text> 
Us <xs1:va1ue-of se1ect="@refid"/> 
U9 <xs1:text>}</xs1:text> 
mo </xs1:temp1ate> 
ml <xs1:temp1ate match=``ref''> 
162 <xs1:text>\ref{</xs1:text> 
163 <xs1:va1ue-of se1ect="@refid"/> 
M4 <xs1:text>}</xs1:text> 
165 </xs1:temp1ate> 
M6 <!-- quotes, footnotes, verbatim 
M7 <xs1:temp1ate match ootnote"> 
168 <xs1:text>\footnote{</xsl:text> 
M9 <xs1:app1y-templates/> 
no <xs1:text>}</xs1:text> 
U1 </xs1:temp1ate> 
U2 <xs1:temp1ate match=“quote"> 
W3 <xs1:text> 
174 \begin{quote}</xs1:text> 
Us <xs1:app1y-temp1ates/> 
W6 <xs1:text>\end{quote}</xsl:text> 
W7 </XS1:temp1ate> 
W8 <xs1:temp1ate match=``quotation''> 
N9 <xs1:text> 

 
%%page page_466                                                  <<<---3
 
B.4 Examples of important DTDS 
I89 
211 
213 
214 
215 
216 
218 
219 
220 
221 
222 
223 
225 
226 
227 
228 
229 
230 
231 
232 
\begin{quotation}</xsl:text> 
<xs1:app1y-templates/> 
<xs1:text>\end{quotation}</xsl:text> 
</xs1:temp1ate> 
<xs1:temp1ate match=``verbatim''> 
<xs1:text> 
\begin{verbatim}</xsl:text> 
<xs1:app1y-templates/> 
<xs1:text>\end{verbatim}</xsl:text> 
</xs1:temp1ate> 
<!-- Lists --> 
<xs1:temp1ate match=``description''> 
<xs1:text> 
\begin{description} 
</xs1:text> 
<xs1:app1y-temp1ates/> 
<xs1:text> 
\end{description} 
</xs1:text> 
</xs1:temp1ate> 
<xs1:temp1ate match=“description/term"> 
<xs1:text> 
\item[</xs1:text> 
<xs1:app1y-templates/> 
<xs1:text>]</xs1:text> 
</xs1:temp1ate> 
<xs1:temp1ate match="description/item") 
<xs1:app1y-temp1ates/> 
</xs1:temp1ate> 
<xs1:temp1ate match=``enumerate''> 
<xs1:text> 
\begin{enumerate} 
</xs1:text> 
<xs1:app1y-temp1ates/> 
<xs1:text>\end{enumerate} 
</xs1:text> 
</xs1:temp1ate> 
<xs1:temp1ate match=``itemize''> 
<xs1:text> 
\begin{itemize} 
</xs1:text> 
<xs1:app1y-templates/> 
<xs1:text>\end{itemize} 
</xs1:text> 
</xs1:temp1ate> 
<xs1:temp1ate match="enumeratelitemize/item") 
<xs1:text> 
\item </xs1:text> 
<xs1:app1y-templates/> 
</xs1:temp1ate> 
<xs1:temp1ate match="bib1iography“> 
<xs1:text> 
\begin{thebib1iography}{99} 
</xs1:text> 
<xs1:app1y-templates/> 
<xs1:text> 
\end{thebib1iography} 
</xs1:text> 
</xs1:temp1ate> 
<xs1:temp1ate match="bibliography/bibitem"> 
<xs1:text>\bibitem{</xsl:text> 
<xs1:va1ue-of se1ect=“@id“/> 
<xs1:text>}</xs1:text> 
<xs1:app1y-temp1ates/> 
</xs1:temp1ate> 
447 

 
%%page page_467                                                  <<<---3
 
448 
Technical appendixes 
<!-- ======================= Mathematics =========:===I====I===I=== --) 
<xs1:temp1ate match=``inlinemath''> 
<xs1:text>$</xs1:text> 
<xs1:app1y-templates/> 
<xs1:text>$</xs1:text> 
</xs1:temp1ate> 
<xs1:temp1ate match=“disp1aymath"> 
<xs1:text> 
\begin{disp1aymath} 
</xs1:text> 
<xs1:app1y-temp1ates/> 
<xs1:text> 
\end{disp1aymath} 
</xs1:text> 
</xs1:temp1ate> 
<xs1:temp1ate match=“equation“> 
<xs1:text> 
\begin{equation} 
</xs1:text> 
<xs1:app1y-templates/> 
<xs1:text> 
\end{equation} 
</xs1:text> 
</xs1:temp1ate> 
<xs1:temp1ate match=``eqnarray''> 
<xs1:text> 
\begin{eqnarray} 
</xs1:text> 
<xs1:app1y-templates/> 
<xs1:text> 
\end{eqnarray} 
</xs1:text> 
</xs1:temp1ate> 
<!-- -========--===--===-=== A paragraph ==7-===--==-I-===--===--=2: --) 
<xs1:temp1ate match-``par''> 
<xs1:text> 
\par 
</xs1:text> 
<xs1:app1y-temp1ates/> 
</xs1:temp1ate> 
<!-- Tabular --> 
<xs1:temp1ate match-``tabu1ar''> 
<xs1:text> 
\begin{tabu1ar}{</xsl:text> 
<xs1:va1ue-of se1ect="@preamb1e“/><xs1:teXt>} 
</xs1:text> 
<xs1:app1y-temp1ates/> 
<xs1:text> 
\end{tabular} 
</xs1:text> 
</xs1:temp1ate> 
<xs1:temp1ate match="tabu1ar/row") 
<xs1:app1y-temp1ates/> 
<xs1:text>\\ 
</xs1:text> 
</xs1:temp1ate> 
<xs1:temp1ate match-"tabular/row/cell[not(1ast-of-type())]"> 
<xs1zapply-templates/><xs1:text>&amp;</xs1:text> 
</xs1:temp1ate> 
<xs1:temp1ate match="tabu1ar/row/cell[1ast-of-type()]"> 
<xs1:app1y-templates/> 
</xs1:temp1ate> 
<!-- I===l===Il======== ``floats'' and their contents ====IIx==I=== --) 
<xs1:temp1ate match=``figure''> 
<xs1:text> 

 
%%page page_468                                                  <<<---3
 
B.4 Examples of important DTDs 
449 
no \begin{figure}\centering 
311 </xs1:text> 
nz <xs1:app1y-temp1ates/> 
H3 <xs1:text>\end{figure} 
n4 </xs1:text> 
H5 </xs1:temp1ate> 
us <xs1:temp1ate match=``tab1e''> 
H7 <xs1:text> 
us \begin{tab1e}\centering 
no </xs1:text> 
no <xs1:app1y-templates/> 
H1 <xs1:text>\end{tab1e} 
nz </xs1:text> 
H3 </xs1:temp1ate> 
H4 <xs1:temp1ate match="figure/caption I table/caption") 
us <xs1:text>\caption{</xsl:text><xs1:app1y-temp1ates/> 
us <xs1:invoke macro=``label''></xsl:invoke> 
H7 </xs1:temp1ate> 
328 <!-- ==========-- ===== Includegraphics = 
H9 <xs1:temp1ate match=``includegraphics''> 
no <xs1:text> 
331 \includegraphics[</xs1:text> 
HZ <xs1:if test="@width"><xs1:text>width=</xs1:text> 
H3 <xs1:value-of se1ect="@width“/><xs1:text>, </xs1:text></xs1:if> 
H4 <xs1:if test="@height"><xs1:text>height=</xsl:text> 
us <xs1:va1ue-of se1ect="@height"/><xs1:text>, </xs1:text></xs1:if> 
D6 <xs1:if test=``Qbb''><xs1:text>bb="</xs1:text> 
H7 <xs1:va1ue-of se1ect="@bb"/><xs1:text>, </xs1:text></xs1:if> 
us <xsl:if test="@ang1e"><xs1:text>ang1e=</xs1:text> 
H9 <xs1:value-of se1ect="@ang1e"/><xs1:text>, </xs1:text></xs1:if> 
3M9 <xs1:if test="@sca1e"><xs1:text>sca1e=</xs1:text> 
341 <xs1:va1ue-of se1ect=“@sca1e"/><xs1:text></xs1:text></xs1:if> 
3% <xs1:text>]{</xs1:text><xs1:va1ue-of se1ect=“@fi1e"/><xs1:text>} 
343 </xs1:text> 
M4 </xs1:temp1ate> 
34¢ </xs1:stylesheet> 
Most of the above code should be more or less trivial to understand after reading Section 7.6, where we explained how XSL style sheets are written and how they 
can generate HTML or ETEX output. In fact, for each element in the document 
instance we have to provide a unique rule so that the XSL parser knows how to 
translate it into the target language. In our case, most XML elements map onetoone onto their ETEX equivalents (the template rules for the font-changing elements 
on lines 120-149 are a clear example). 
Therefore it should be enough to look in more detail at three areas, namely 
cross-references, tabular material, and the handling of optional attributes, such as 
those of the includegraphics element. 
As noted in Section B.4.5 .1, cross-referencing is based on XML’s id and idref 
attributes. Therefore we define a macro “label" (lines 8-13), which puts a ETEX 
\label command inside elements for which it makes sense in EHEX. For a given 
parent we look whether an id attribute is present (xsl : if construct on lines 9 and 
10). If it exists, we insert a \label command with the key extracted from the id 
attribute of the XML parent element (line 10). We end the macro by providing the 
closing brace for the enclosing element (lines 11-12). 

 
%%page page_469                                                  <<<---3
 
450 
Technical appendixes 
In the present version of the XSL file we reference this macro for all st itle 
elements of the sectioning commands (lines 93, 97, 101, 105, 109, 113, and 117) and 
the caption command (line 326). The bibitem command (lines 239-244) uses its 
own item identifier as key (line 241). 
References to these keys are made with the cite (line 151-155), pageref (line 
156-160), and ref (lines 161-165) elements. In each case we simply get the value 
of the ref id attribute and enter it as an argument in the relevant BTEX command. 
Tabular material (lines 286-293) specified with the tabular element is to be 
transformed into a format compatible with IéTEX’s tabular environment. We first 
get the preamble (line 289) and then handle the rows (lines 296-300) by ending 
each row with a \\ (line 298). Cells are a little trickier since we must be careful 
with the & column separators. Indeed, we must distinguish between nonterminal 
columns (lines 301-303), where we add a & separator following the contents (line 
302) and the last column of a row (which we choose with XSL’s last-of-type 
specifier; see lines 304-3 06), where we put only the contents (line 305). 
Finally at the end of the XSL file we treat the includegraphics command 
(lines 329-344). The only minor difficulty in this case is that we must be careful 
to verify which Optional attributes have been specified. Each of these attributes has 
its own xsl : if construct to test for its presence (lines 332-341). For each attribute 
that was specified, we get its value and enter it following the relevant selection string 
for ETEX (e.g., width= on line 332). At the end the mandatory filename (specified 
in the XML file with the file attribute) is output between curly braces (line 342). 
VVhen we run the sample file minilatexexa . xml together with the above XSL 
style sheet minilatex . xsl through the xt processor and then postprocess the result to obtain ETEX compatible output for XML reserved characters, we obtain a 
ETEX source file minilatexexa. tex.2 
xt mini1atexexa.xml ../minilatex.xsl I \ 
sed -e ’s/&gt;/>/g’ -e ’s/&lt;/</g’ -e ’s/\&amp;/\&/g’ \ 
> minilatexexa.tex 
After running the file minilatexexa.tex through the ETEX processor, we 
generated the PostScript file shown in Figure B.1. 
B.5 Transforming HTML into XML 
HTML’s popularity has grown exponentially in recent years because of the increased 
importance of distributing documents on the Web. HTML is used by ever more 
applications, for which new tags are constantly being introduced, so that HTML has 
grown into a compatibility nightmare. Cross-platform portability is increasingly 
2 Because xt generates UTF-8 output for non-ASCII characters, we had to transform the UTF-8 characters to Latin 1, which we did with the help of the Yudit program (see Section C.3.2). Future XSL 
programs will probably make it possible to choose the output encoding directly. 

%==========470==========<<<---2
 
%%page page_470                                                  <<<---3
 
B.5 Transforming HTML into XML 
451 
The IMEX DTD and multiple languages 
Michel Gnosse-ils ' 
August 4th, 1998 
1 The basic principles 
This is an example input file. We start in English to show the principle, You 
should especially pay attention that we have used slightly different notutlon for 
some of the common l¢’I}:_‘)( constructs. such as the datslles. whlrth come in three 
sizes: an intra-word dash. 3 medium dash for number ranges like 1-2. and n 
punctuation dnsh-lilte tllis. Text can be emphasized as .9*ht)1l/n hm An ellipsl.» 
is made with . Footnotes‘ are tricky constructs, since one must be mzeiul mil. 
to nest them 
1.1 Dealing with special characters 
XML lun a different set of rezwrvrxi characters llmu IA'I)3X, in pnrticulur, when 
you want to use any of the three characters 1, <, and >, you should enter them 
as lamp; . nu; , and lgt; . respectively, 
1.2 EIIEX and mathematical formulae 
I£(l):)( and a {mum "I}§)( are very good at typesetting mathelllathtnl iormulue, 
like 1 - 3y+ 2 < 7 or a. > 1?" +1/2" > z’ or (11.1?) = :‘u,b.. Do not lorget 
l-Tigure 1: Bill EPS Bild 
Tahle 1; Eiugatbe der dnlltschen Zusntzznichen in lmex 
``a ii ''A A ``c o ''D O ``u ii ''U U "S SS 
2.2 Beispiel einer Tabelle 
Die 'luholl--, 1 eul Selle 2 zeigt nine Tabelle 
3 Continuation du texte en frangais 
3.1 Traiter les listes 
L22: lisms sum. u£ill5駑< iréquenullent pour structurel’ nu nlettre en évidellct-, 
certaills élén-rents d'un docllment. (Volt [2]). 
. (‘eci est le prt-llliel‘ élémenr, dune liste nun-nrrlu.....se. Chnqur, élémeut dc, 
ce type de liste est précédé d‘uli signs distlnctil. comlne uue puce. un tuet, 
emf. 
. Ce secoxld élémcnl de la luélne liste onntieut une lists de descnptum nu 
hrlquée. 
XIVIL Meta lallgage pour défimr don‘ classes de (lonlllllenls 
XLL lnugnge pour définir lie.» hyperliens eutre difiérentes parties ue doc 
that for rensnns of consistency. if you want to refer to a variable In one of the 
formulae, such as the symbol 1'. you nulsl Also use math mode in thr text. "‘““'“"‘ 
Nous onmlnuons notrc tnxte 5 l'intél'ieul' de la premiere llstc. 
2 Beispiel eines Textes in deutscher Sprache 
2.1 Eine EPS Abbildung R``f''“’“°"5 
Dieser Ahschnitl zeigt, wie man elne . . . 3. Abbildung [1] in eill Dokulnenl lll M0179 Int Pvslsmvl Hrmlibuth (2 Aufluye) »\ddi=‘<>n'VVe51ey (Deulst‘heinbindeu kztnn Abbildung l wurde nut dem Belehl land) G``I''"- “Om 1991 
[2] Michel Goussens. Persulllialiser les lis-Les mpx. Ca/tiers (}U’['enbe7y, 
uncludegraphics[vi-ith=``3cm'']{r11e="colorcir.eps} [M2 48 ‘M “M 
in den Text aufgenunimen. 
‘Partly [mm an example in Tris IA1).~,\' Compamml 
‘This ls u simple footnote 
it can also rontum par element. 
Figure B. 1: PostScript rendering of our I£iTEX-based XML document 
difficult to guarantee, and this has now become the largest obstacle for HTML to 
keep up with the Internet’s rapid evolution. 
Therefore to find solutions for distributed document handling on an evergrowing set of different computer and communication environments, a successor 
to HTML is urgently needed. At a Future of HTML workshop held in May 1998, it 
was agreed that the way forward was to make a fresh start with a new generation of 
HTML, which would be defined as a suite of XML tag sets. 
This new HTML includes a core mg set, used to mark up headings, paragraphs, 
lists, hypertext links, images, and other basic documents idioms. On the other hand, 
the markup for forms, tables, multimedia, graphics, and so on is each defined by its 
own separate tag sets. These tag sets adopt XML syntax, and you will be able to use 
them in any combination. This will allow you to mark up a much wider range of 
documents than with a monolithic markup scheme. 
A clear separation of content and form is more strictly imposed, so that style 
sheets become increasingly important to associate such characteristics like color, 
font, and document layout with structural elements. Style sheets also help in de
 
%%page page_471                                                  <<<---3
 
452 
Technical appendixes 
scribing how markup can be adapted or transformed to optimize its rendering on 
different kinds of devices. 
In order to describe which HTML tags a given type of device supports and 
which style sheet features it implements, conformance profiles are used. The principle 
is that two different devices must render markup in the same way if they belong to 
the same conformance profile, and the document uses only features of that profile. 
A conformance profile not only specifies which HTML tags sets and which 
style sheet functionality are supported, but also contains information about semantic constraints, data format, scripting support, and so on. A device can send its 
profile to a Web server where transformational software can repurpose the markup 
in a simple and reliable fashion and return a document customized for the device 
in question. Document and device profiles will encode the information in Resource 
Description Framework (RDF, [9 RDF]), and in function of a set of DTDs. 
B.5.1 HTML in XML 
To formalize the points discussed in the previous section, W3C recently released 
a working draft [9>HTMLINXML] that reformulates HTML4.0 as an XML application and defines the corresponding namespaces. It introduces document profiles 
as a basis for interoperability between different sets of HTML in heterogeneous 
environments. 
B. 5 . 1 . 1 HTML modularization 
HTML has been split into modules, each corresponding to a well-defined set of 
HTML tags that can be mixed and matched at will. For instance, a “table module" 
would contain the elements and attributes necessary to support tables, and a “forms 
module" would contain elements and attributes needed for forms. 
This modularization is important to ease the support necessary to maintain and 
deliver content on an ever-increasing number of diverse platforms-not only computer screens, but also mobile devices (handheld computers, portable phones), television devices (digital televisions, TV-based Web browsers), and appliances (with a 
fixed function). They all have their own requirements and constraints. 
By dividing HTML into different modules, it becomes possible to specify for a 
given device which tag set it can deal with. These modules can be used as standard 
building blocks to support the device in question. 
B.5 .1.2 Document profiles 
The syntax and semantics of documents are specified by a document profile. Interoperability is guaranteed by enforcing conformance to a document profile that describes such things as which data formats (e.g., for images) can be used, levels of 
scripting, and style sheet support (see central bottom part of Figure B.2). The document profile is expressed in RDF. 

 
%%page page_472                                                  <<<---3
 
B.5 Transforming HTML into XNIL 453 
Tag sets 
°°‘e (DTD fragments) °h°“‘iS‘ry 
tables music 
DTD 
f°m‘S XML schema mad‘ 
graphics ‘ 
multimedia R D F 
XML 
Device - Device 
profiles profile profiles 
___________ ._ V _._._._._._._ 
1 
features ' Portable 
palmtop V : phone 
_i Document profile _« 
desktop - graphics formats TV 
3style sheet support I top Set 
I ' I 
Figure B.2: XHTML tag sets and profiles 
A DTD (or other schema system) specifies the syntax of documents that conform to a document profile in terms of which HTML modules are used, as well as 
which additional modules for other XML tag sets are available (chemical formulae, mathematics, musical notation, and vector graphics; see also the upper part of 
Figure B.2). 
A document profile that defines the minimal support expected of user agents 
consists of assertions written in RDF. It provides the basis for interoperability guarantees and can be used by servers to establish whether the server has a version of 
a document suitable for delivery to a user agent with a given device profile. This 
concept is explained in the next section. 

 
%%page page_473                                                  <<<---3
 
454 
Technical appendixes 
B.5.1.3 Device profiles 
The capabilities of browsers, as well as user preferences, are specified with the help 
of device profiles. This allows servers to select the appropriate variant of a document 
to deliver to a browser or other device, possibly by transforming the content, based 
on the match between the device profile of the browser and the document profile 
of the source document (see bottom part of Figure B.2). 
The descriptions available in document and device profiles should greatly simplify optimizing markup to match the needs of different devices by tuning the 
markup to the given class of devices in a simple and reliable fashion. 
B.5 .2 The Extensible HyperText Markup Language 
The Extensible HyperText Markup Language (XHTML; see [=>HTML1NXML]) is 
a reformulation of HTML 4.0 as an XML 1.0 application. XHTML 1.0 specifies 
three XML namespaces, corresponding to the three HTML 4.0 DTDs: Strict, Transitional, and Frameset. Each of these three namespaces is identified by its own URI. 
XHTML 1.0 provides the basis for a family of document types that will extend 
and subset XHTML. This will allow XHTML to support a wide range of new devices, applications, and platforms, as explained in Section B.5 .1. It is clear that not 
all of the XHTML elements will be required on all platforms. Therefore XHTML 
will be broken up into modules of small element sets that can be recombined to 
meet the needs of different communities. 
B.5.2.l Writing XHTML documents 
An XHTML document may be transmitted using one of three Internet Media 
Types. This allows document authors to create portable Internet content that can 
be served to generic XML applications (text/xml), to legacy HTML user agents 
(text/html), and to new XHTML applications (text/xhtml). 
An example of a simple XHTML document follows: 
<!DOCTYPE html PUBLIC “-//W3C//DTD XHTML 1.0 Strict//EN`` ''xhtmll-strict.dtd“> 
<ht:m1 xm1ns="ht:t:p://www.w3.org/Profiles/xhtml1-strict:.dt;d“> 
<head> 
<t:it:1e>An XHTML document;</tit:1e> 
</head> 
<body> 
<p>The <a href="htt:p://www.w3c.org/">w3C Home Page</a>,</p> 
</body> 
</html> 
<>oo\:o«m.:.w--Because XHTML is an XML application, certain practices that were perfectly 
legal in SGML-based HTML must be changed. A series of guidelines for delivering 
XHTML documents follow: 
0 XHTML document‘: mmt be well-fiormed XML. In practice this means that all elements must either have closing tags and that all the elements must nest cor
 
%%page page_474                                                  <<<---3
 
B.5 Transforming HTML into XNIL 
455 
-h\~N»vi-h\~Iunrectly. VVhen documents that are not well-formed are presented to an XML 
parser, a fatal error will occur, and the parser would usually stop processing at 
the first occurrence of such an error. This is rather unexpected to HTML users, 
since browsers usually ignore incorrect markup and continue processing. 
Use the tcmlns attribute to designate the document profile. The xmlns attribute 
on the html element must be used to designate the document profile. VVhen 
XHTML documents are delivered as text/html, the presence of the xmlns 
attribute implies that the contents of the html element are written in wellformed XML and must be processed according to the XML 1.0 specification. 
Tag and attributes names must be entered in lowercase. Names of element types and 
attributes are case-sensitive in XML (and hence XHTML). Since the XHTML 
DTDs define them in lowercase all tags and attributes should be entered in 
lowercase in the document instance. 
Closing tags are always required. All elements must have a closing tag. In particular, empty tags must end with />, for example <br /> (the space before the / 
is to allow the tag to be handled by current HTML browsers). 
Attribute minimization is disallowed. Attributes must be specified as name-value 
pairs, with the value between quotes. Thus a compact description list (dl) will 
have the form: 
<d1 compact=``compact''> 
<dt>France</dt><dd>Paris</dd> 
<dt>spain</dt><dd>Madrid</dd> 
</d1> 
Note that XHTML strips leading and trailing whitespace in attribute values, 
and maps sequences of one or more whitespace characters (including linebreaks) to a single interword space. 
Be careful with script and style elements. In XHTML the script and style 
elements are declared as having #PCDATA content, so that entities, such as &1t; 
and &a.mp;, will be expanded by the XML processor to < and 8:, respectively. 
If this is not desired, then you should wrap your script statements inside a 
CDATA marked section, that is, 
<script> 
<![CDATA[ 
. unescaped script content ... 
]]> 
</script> 
B.5.2.2 XHTML namespaces 
XML namespaces for three profiles are defined. 
http : //www . W3 . org/Prof iles/xhtml 1-strict .dtd 

 
%%page page_475                                                  <<<---3
 
456 
Technical appendixes 
For documents converted from HTML 4.0 strict and conforming to the 
XHTML xhtml 1-strict DTD. 
http://www.w3 . org/Profiles/xhtml1-transitional . dtd 
For documents converted from HTML 4.0 transitional (or loose), which includes a number of presentational elements and attributes. They must conform to the XHTML xhtm11-transitional DTD. 
http://www.w3.org/Profiles/xhtml1-frameset .dtd 
For documents converted from HTML 4.0 frameset and acting as frame sets. 
They must conform to the XHT ML xhtmll-frameset DTD. 
B.5.2.3 Converting existing content to XHT ML 
Dave Raggett’s HTML Tidy [=>HTMLTIDY] program allows you to tidy up existing 
HTML source files and transform them to XHTML. Tidy is able to correct a wide 
range of HTML problems. It will signal things that it cannot handle by itself, so 
that you can take the necessary action yourself. 
Tidy corrects the markup in a way that matches, where possible, the observed 
rendering in existing browsers, such as Netscape and Nlicrosoft Internet Explorer. 
Following is a list of the main problems that Tidy can detect and handle: 
0 detects and corrects missing or mismatched closing tags; 
0 corrects incorrect nesting; 
0 adds missing “/" in closing tags; 
0 completes lists by putting in “missing" tags; 
0 includes quotes around attribute values, when needed; 
0 reports unknown and proprietary attributes and element types (with respect to 
the HTML 4.0 DTDs); and 
o signals tags that lack a terminating “>". 
B.5.2.4 Running Tidy 
Tid uses terminal in ut stdin and screen ou ut stdout as default inY P tP 
put/output streams, if no filename is specified. Errors are written to stderr by 
default. The general form of the command sequence is as follows: 
tidy [[option.s] fiZen.ame]* 
Possible values for options are: 
-indent or -i indent element content; 
-omit or -o omit optional endtags; 
-wrap 72 wrap text at column 72 (default is 68); 
-upper or -u force tags to uppercase; 

 
%%page page_476                                                  <<<---3
 
B.5 Transforming HTML into XML 457 
-clean or -c replace font, nobr, and center tags by CSS; 
-raw do not output entities for chars 128 to 255; 
-ascii use ASCII for output, Latin 1 for input; 
-latinl use Latin 1 for both input and output; 
-utf 8 use UTF-8 for both input and output; 
-iso2022 use ISO2022 for both input and output; 
-numeric or -n output numeric rather than named entities; 
-modify or -m to modify original files; 
-errors or -e show only error messages; 
-f file write errors to fi Le; 
-xml use this when input is in XML; 
-asxml to convert HTML to XML; 
-slides to burst into slides on h1 elements; 
-he1p list command line options. 
For instance, the command line: 
tidy -f errs.txt -m myfile.html 
runs Tidy on the file index .htm1, updating it in place and writing error messages 
to the file errs.txt. Single-letter options apart from -f can be combined, for 
example, 
tidy -f errs.txt -imu myfile.html 
Let us take as an explicit example the HTML output we prepared on page 308. 
We reduce that file to a minimum number of tags, especially getting rid of all closing tags that are implied. The tidytest .html file we end up with is the following. 
<TITLE> Invitation (sgmlpl/CSS formatting) </TITLE> 
<LINK href="invit.css" rel=``style sheet'' type="text/css"> 
<H1>INVITATIflN</H1> 
<P><TABLE> 
<TR><TD class=``front''>To: 
<TD>Anna, Bernard, Didier, Johanna 
<TR><TD class=``front''>Hhen: 
<TD>Next Friday Evening at 8 pm 
<TR><TD class=``front''>Venue: 
<TD>The Web Cafe 
<TR><TD class=``front''>flccasion: 
<TD>My first XML baby 
</TABLE> 
<P>I would like to invite you all to celebrate 
the birth of <EM>Invitation</EM>, my 
first XML document child. 
<P>Please do your best to come and join me next Friday 
evening. And, do not forget to bring your friends. 
5 Z 3 E 3 5 E 3 8 o m u o m a w N _ 

 
%%page page_477                                                  <<<---3
 
458 
Technical appendixes 
19 <P>I <EM>rea11y</EM> look forward to see you soon! 
20 <P CLASS=``signature''>Miche1 
We run Tidy on the tidytest .html file with the following instruction: 
tidy -im tidytest.html 
Parsing "tidytest.html" 
line 4 column 3 - Warning: <TABLE> lacks ``summary'' attribute 
line 4 column 3 - Warning: trimming empty <P> 
We see that the <P> tag in line 4 of the HTML source file is redundant and that we 
should specify a summary attribute on the table tag. Let us look at the following 
generated XML output: 
<?xml version="1.0"?> 
<!DOCTYPE html PUBLIC "-//W3C//DTD HTML 4.0//EN"> 
<html> 
<head> 
<title> 
Invitation (sgmlpl/CSS formatting) 
</title> 
<link href="invit.css" re1=``style sheet'' type="text/css" /> 
</head> 
<body> 
<h1> 
INVITATION 
</h1> 
<tab1e> 
<tr> 
<td c1ass=``front''> 
To: 
</td> 
<td> 
Anna, Bernard, Didier, Johanna 
</td> 
</tr> 
<tr> 
24 <td c1ass=``front''> 
25 Hhen: 
Z6 </td> 
27 <td> 
23 Next Friday Evening at 8 pm 
29 </td> 
</tr> 
<tr> 
<td c1ass=``front''> 
Venue: 
</td> 
<td> 
The web Cafe 
</td> 
</tr> 
<tr> 
<td c1ass=``front''> 
Occasion: 
</td> 
<td> 
My first XML baby 
</td> 
</tr> 
NNN--._..--._.._.._._..--._.._. 
N-O~Ooa\lO«m-bWN'-C~Oa:\lO«m-§WN" 
N 
w 
3 $ £ 3 3 3 8 S 2 3 3 3 3 E 3 E 3 

 
%%page page_478                                                  <<<---3
 
B.6 Java event-based interface 
</tab1e> 
<p> 
I would like to invite you all to celebrate the birth of <em> 
Invitation</em>, my first XML document child. 
</p> 
<p> 
Please do your best to come and join me next Friday evening. 
And, do not forget to bring your friends. 
</p> 
(p) 
I <em>really</em> look forward to see you soon! 
</p> 
<p class=``signature''> 
Michel 
</p> 
</body> 
</html> 
% $ 3 % S 3 S § 3 $ 3 3 S 3 3 & 3 
You see how all tags have become lowercase, closing tags have been output written 
for all elements, and html, head, and body elements have been introduced. Tidy 
is thus a very convenient tool for transforming HTML files into correct XML files 
that conform to the XHTML DTD. 
B.6 Java event-based interface 
With so many XML parsers available, it is no light task for application writers to 
support all corresponding Application Programmer’s Interfaces (APIs). Therefore 
at the end of 1997, Peter Murray-Rust, the author of Jumbo, one of the first XML 
parsers written in Java [k>}UMBO], proposed on the XML-DEV XML developers list 
[<->XMLDEV] that parser writers should support a common Java event-based API. 
This idea was taken up by Tim Bray, one of the editors of the XML Specification and by David Megginson, the author of /Elfred (see Section 6.6.5.7) and the 
SGMLSpm system, described in Section 7.3. The design was a collaborative effort with a lot of contributors participating in the discussion on the XML-DEV list. 
It resulted in a single, standard, event-based API for XML parsers SAX (“A Simple 
API for XML," see [9 SAX]), which was released in May 1998. Implementations are 
currently available inJava and Python. 
In the next section, we will first briefly review the classes present in the SAXJava 
distribution. Then we will use a few of these classes to reimplement the translation 
of a variant of our invitation example to get it translated into ETEX. 
B.6.1 The SAX Java classes 
The SAX Java distribution of David Megginson [9 SAXJAVA] contains eleven core 
classes and interfaces together with three optional helper classes, although, as we 
shall see, a simple XML application needs only one or two of these. 
459 

 
%%page page_479                                                  <<<---3
 
460 Technical appendixes 
The SAX classes and interfaces can be subdivided into four groups: 
Interfaces for Parser Writers (org .xm1 . sax package) 
Parser Main interface to a SAX parser. It registers handlers for callbacks, 
sets the language environment for error reporting, and starts parsing the XML document. 
AttributeLi st Simple interface to iterate through an attribute list. 
Locator Simple interface to locate the current point in the XML source 
document. 
Interfaces for Application Writers (org . xml . sax package) 
Document-.Hand1er Interface normally used to implement an application. 
It provides access to basic document-related events, such as the 
start and end of elements. 
ErrorHand1er Interface for implementation of special error handling. 
DTDHand1er Interface that handles notations and unparsed (binary) entities (NOTATION and ENTITY declarations in the DTD). 
Em-.ityReso1ver Interface providing customized handling for resolving 
external entities. 
Standard SAX Classes (org.xm1.sax package) 
Inputsource Class containing information for a single input source. 
SAXException General SAX exception class. 
SAXParseException SAX exception class tied to a specific point in an 
XML document. 
HandlerBase Default implementations for DTDHand1er, ErrorHand1er, 
Document-.Hand1er, and EntityReso1ver. In our example we extend Handlerbase to implement a handler. 
Java-Specific Helper Classes (org . xml . sax .he lpers package) 
Classes not part of the core SAX distribution but provided as a convenience 
for]ava programmers. 
ParserFactory The static methods in this class allow the application to 
load SAX parsers dynamically at runtime, based on the class name. 
AttributeListImp1 Makes a persistent copy of an AttributeList or 
supplies a default implementation of AttributeLi st to the application. 

%==========480==========<<<---2
 
%%page page_480                                                  <<<---3
 
B.6 Java event-based interface 
461 
Locatorlmpl Makes a persistent snapshot of the value of a Locator at 
any point during a document parse. 
B.6.2 Running a SAX application 
To run a SAX application, you need a parser with a SAX interface, as well as the 
SAX libraries themselves. For our examples we use components written by David 
Megginson, namely his /Elfred parser and his_]ava implementation of SAX. 
To generate a ETEX or HTML format of an XML document, we need to write 
an event handler to receive information from the parser. As explained in Section B.6.1, a convenient interface is Documentflandler, which receives events for 
the start and end of elements, character data, and other basic XML stmctures. However, as we are not interested in implementing the entire interface, we can limit 
ourselves to the creation of a class that extends Hand1erBase, where we define only 
the methods we need. 
The list of methods available with the Hand1erBase class follows: 
characters(char E] . int, int) Receive notification of character data inside 
an element. 
endDocument ( ) Receive notification of the end of the document. 
endE1ement (String) Receive notification of the end of an element. 
error (SAXParseException) Receive notification of a recoverable parser error. 
f ata1Error (SAXParseException) Report a fatal XML parsing error. 
ignorab1eWhitespace(char[] , int, int) Receive notification of ignorable 
whitespace in element content. 
notationDec1 (String, String, String) Receive notification of a notation 
declaration. 
processinglnstruction(String, String) Receive notification of a processing instmction. 
reso1veEntity (String, String) Resolve an external entity. 
setDocumentLocator (Locator) Receive a Locator object for document events. 
startDocument () Receive notification of the beginning of the document. 
startE1ement (String, AttributeList) Receive notification of the start of an 
element. 
unparsedEntityDec1 (String, String, String, String) Receive notification of an unparsed entity declaration. 
warning( SAXParseExcept ion) Receive notification of a parser warning. 

 
%%page page_481                                                  <<<---3
 
462 Technical appendixes 
We also need the methods of the AttributeList interface: 
get-,Length() Return the number of attributes in this list. 
get-.Na.me (int) Return the name of an attribute in this list (by position). 
getType( int), getType(String) Return the type of an attribute in the list (by 
position or name). 
get-.Value( int), get-.Va1ue(String) Return the value of an attribute in the list 
(by position or name). 
Now we are ready to write ourjava class InvitationSAX . java, which extends 
the Hand1erBase class and uses the AttributeList interface. We will develop 
an application for formatting the XML document invitation2.xm1 described in 
Section 7.4.5, which is, information-wise, equivalent to the invitation.xm1 document that we handled with the Perl interface in Section 7.3. 
Following is the listing of InvitationSAX . java: 
import org.xml.sax.HandlerBase; 
import org.xml.sax.AttributeList; 
public class InvitationSAX extends HandlerBase { 
public void startElement (String Ename, AttributeList atts) 
{ if (Ename.equals(``invitation'')) 
{System.out.print("\\documentclass[]{article}\n" 
+ "\\usepackage{invitation}\n" 
+ "\\begin{document}\n" 
+ "\\begin{Front}\n"): 
for (int i = 0; i < atts.getLength(); i++) { 
String Aname = atts.getName(i); 
String type = atts.getType(i); 
String value = atts.getValue(i); 
if (Aname.equa1s(``date'')) 
System.out.print("\\Date{" + value + "}\n"); 
is else if (Aname.equals(``signature'')) 
19 System.out.print("\\Signature{" + value + "}\n"); 
20 else if (Aname.equals(``to'')) 
21 System.out.print("\\To{" + value + "}\n"); 
22 else if (Aname.equa1s(``where'')) 
H System.out.print("\\Hhere{" + value + "}\n"); 
24 else if (Aname.equals(``why'')) 
M System.out.print("\\Hhy{" + value + "}\n"); 
26 else System.out.print("INVALID ATTRIBUTE!!! " + value + "\n"); 
27 } // end attributes of invitation 
28 System.out.println("\\end{Front}"); 
29 System.out.println("\\begin{Body}"); 
} // end element invitation 
if (Ename.equals(``par'')) 
System.out.print("\\par "); 
if (Ename.equals(``emph'')) 
System.out.print("\\emph{"); 
} // End of startElement 
:3»ooa\no«m-tawny.3 3 G I S 5 
public void endElement (String Ename) 
{ if (Ename.equals(``invitation'')) 
System.out.print("\\end{Body}\n" 
+ "\\begin{Back}\n" 
+ "\\end{Back}\n" 
ggwwwwwwwwww 
»- ~Ooa\lO«\l-bxaara-0 

 
%%page page_482                                                  <<<---3
 
B.6 Java event-based interface 
42 + "\\end{document}\n"); 
43 if (Ename.equa1s(``emph'')) 
M System.out.print("}"); 
45 // if (Ename.equals(``par'')) ---> do nothing 
46 } // End of endE1ement 
47 
% public void characters(char ch[],int start,int length) 
49 { for (int i=start; i<start+length; i++) 
50 {System.out.print(ch[i]);} 
51 } // End of characters 
52 
53 
} // end of InvitationSAX 
We see some of the methods of the HandlerBase class at work: startE1ement 
(lines 6-3 5) for the beginning of elements, endE1ement (lines 37-47) for the end 
of elements, and characters (lines 49-53) for handling character data. In the first 
two methods we define an action that depends on the element being handled. In 
particular, at the start of the invitation element, we initialize the BTEX document 
(lines 9-12). We write to standard output by using the method System . out . print 
(lines 9, 18, 20, and so on), and its variant the System. out . println (lines 28 
and 29) method. We loop over the various attributes and use the methods of the 
AttributeList interface. The number of attributes is given by getLength (line 
13), which controls the number of iterations of the for loop (lines 13-2 7). Inside 
the loop getName (line 14) returns the name of the attribute, getType (line 15) 
its type, and getVa1ue (line 16) its value. Using the attribute name if statements 
(lines 17-2 6) take care of setting ETEX variables to their required value. 
Similarly, the beginning of a par element generates a \par command (lines 
31-32), and for the emph start tag a ETEX \emph command is started (lines 33-34). 
Regarding the end of the elements (lines 37-47), we do nothing for par (line 46 
is a]ava comment), and we close the curly brace for the emph end tag (lines 44-45). 
When we reach </invitation>, we end the ETEX document (lines 39-43). 
The character data of the XML source document is transferred from the character array ch, the first argument of the characters method (line 49), to the output 
stream by a loop (lines 5 1-5 2) whose range is given in the argument of the method. 
Our only remaining task is to write a main driver class MySAXApp. java that 
uses our handler InvitationSAX. This is shown following. 
// MySAXApp.java -- Main driver class 
// --> calls InvitationSAX which has customized code 
import org.xml.sax.Parser; 
import org.xml.sax.DocumentHandler; 
import org.xml.sax.helpers.ParserFactory; 
public class MySAXApp { 
static final String parserclass = "com.microstar.xml.SAXDriver"; 
public static void main (String args[]) 
throws Exception 
{ Parser parser = ParserFactory.makeParser(parserClass); 
Documentflandler handler = new InvitationSAX(); 
parser.setDocumentHand1er(handler); 
for (int i = 0; i < args.length; i++) {parser.parse(args[i]);} 
} 
} 
ov....GES8om\.amAw~_ 
463 

 
%%page page_483                                                  <<<---3
 
464 
Technical appendixes 
Our application first creates a Parser object by supplying a class name to the 
ParserFactory (defined on line 8 and used on line 11). Then our handler class 
InvitationSAX is instantiated (line 12) and registered with the parser (line 13). Finally all XML documents that are specified as arguments on the command line are 
parsed (the for loop on lines 14). The current implementation requires that the 
documents are specified as absolute URLs. 
We now are ready to compile our application and handler for the document 
invitation2 .xml and run it (the example is on Windows/NT). 
set c1asspath=. ;d:\ae1fred;d:\SAX;d:\jdk1.1.6\src 
d : \j avac Invitat ionSAX . java 
d:\javac MySAXApp.java 
d:\java MySAXApp filezdz/invitation2.xml > inv2.tex 
The first line declares where the Java class libraries are located; the next two 
lines compile our two Java classes with the Java compiler javac. The last line runs 
the application MySAXApp, which loads the other classes automatically on the file 
invitation2.xm1 (given as an absolute URL) and writes the resulting BTEX file 
inv2 . tex. That file is shown here: 
\documentclass[]{article} 
\usepackage{invitation} 
\begin{document} 
\begin{Front} 
\To{Anna, Bernard, Didier, Johanna} 
\Date{Next Friday Evening at 8 pm} 
\Hhere{The Web Cafe} 
\Hhy{My first XML baby} 
\Signature{Michel} 
\end{Front} 
\begin{Body} 
\par 
I would like to invite you all to celebrate 
the birth of \emph{Invitation}, my 
first XML document child. 
\par 
Please do your best to come and join me next Friday 
evening. And, do not forget to bring your friends. 
\par 
I \emph{rea11y} look forward to see you soon! 
\end{Body} 
\begin{Back} 
\end{Back} 
\end{document} 
~N-N_-_......___....-.pu.aN»-o»ooa\lo«v-bv-IN»-C~ooa\lO«VI-§'\~'NThe BTEX code is a little different from the one on page 295. We, nevertheless, 
can use (line 3) the same package invitation. sty as defined on page 296, since 
it is written in such a way that high-level commands guide the formatting process. 
Hence, different ETEX source files will produce identical results after typesetting 
(see \reffig{7-1}). This shows once more the advantage of writing high-level ETEX 
code on the application level and handling low-level typesetting or rendering commands in a style sheet. 

 
%%page page_484                                                  <<<---3
 
C 
Internationalization 
issues 
C.1 Codes for languages, countries, and scripts 
Table C.1 shows codes for the representation of names of languages as defined 
in ISO Standard 639. The first column is the current three-letter code (ISO:6392, 1998). The second column contains, where it is available, the older two-letter 
code (ISO:639, 1988). The third column contains the name of the language (or 
its variant). Although the two-letter code can only cope with the most common 
languages it is still used in many computer applications; in particular HTML and 
XML usually use only two-letter codes to identify languages. 
Table C.2 shows codes for the representation of names of countries as defined 
in ISO Standard 3166 (ISO:3166, 1997). The first column is the two-letter code 
that is used in HTML/XML for specifying the country; the second column provides 
a more mnemonic three-letter extension. The third column specifies the country 
name. 
Table C.3 shows codes for the representation of names of scripts as proposed 
in the draft for ISO Standard 15924 (ISO:15924, 1999). The first column is the 
two-letter code, and the second column provides a more mnemonic three-letter 
extension. The third column specifies the name of the script. 
Table C.4 gives an overview of the various ISO 8859 coding standards (see 
ISO/IEC:8839-1 (1998) and following). In particular, Latin 9 (ISO/IEC:8859-15), 
which was approved recently, is essentially equal to the Latin 1 layout for Western 
European languages. However, in Latin 9, there are a few less-used characters re
 
%%page page_485                                                  <<<---3
 
466 
Internationalization issues 
placed by 03, CE, and y (for French), §, §, i, and Z (for Finnish), and € (the Euro 
currency symbol). Of course, all these ISO/IEC 8859 tables should ideally be replaced by Unicode as soon as possible (see Section C.2). 
Table C.1: Language codes and names (ISO 639) 
aar aa Afar abk ab Abkhazian 
ace Achinese ach Acoli 
ada Adangme af a Afro-Asiatic (other) 
afh Afrihili afr af Afrikaans 
ajm Aljamia aka Akan 
akk Akkadian ale Aleut 
alg Algonquian languages amh am Amharic 
ang English, Old (ca. 450-1100) apa Apache languages 
ara ar Arabic arc Aramaic 
arn Araucanian arp Arapaho 
art Artificial (other) arw Arawak 
asm as Assamese ath Athapascan languages 
aus Australian languages ava Avaric 
ave Avestan awa Awadhi 
aym ay Aymara aze az Azerbaijani 
bad Banda bai Bamileke languages 
bak ba Bashkir bal Baluchi 
bam Bambara ban Balinese 
bas Basa bat Baltic (other) 
bej Beja bel be Belarussian 
bem Bemba ben bn Bengali 
ber Berber (other) bho Bhojpuri 
bih bh Bihari bik Bikol 
bin Bini bis bi Bislama 
bla Siksika bnt Bantu (other) 
bod bo Tibetan bra Braj 
bre br Breton btk Batak (Indonesia) 
bua Buriat bug Buginese 
bul bg Bulgarian cad Caddo 
cai Central American Indian (other) car Carib 
cat ca Catalan cau Caucasian (other) 
ceb Cebuano cel Celtic (other) 
ces cs Czech cha Chamorro 
chb Chibcha che Chechen 
chg Chagatai chk Chuukese 
chm Mari chn Chinook jargon 
cho Choctaw chp Chipewyan 
chr Cherokee chu Church Slavic 
chv Chuvash chy Cheyenne 
cmc Chamic languages cop Coptic 
cor kw Cornish cos co Corsican 

 
%%page page_486                                                  <<<---3
 
C.l Codes for languages, countries, and scripts 
467 
Language codes and names (cont) 
cpe Creoles and pidgins, cpf Creoles and pidgins, 
English-based (other) F rench-based (other) 
cpp Creoles and pidgins, cre Cree 
Portuguese-based (other) 
crp Creoles and pidgins (other) cus Cushitic (other) 
cym cy Welsh dak Dakota 
dan da Danish day Dayak 
del Delaware den Slave (Athapascan) 
deu de German dgr Dogrib 
din Dinka div Divehi 
doi Dogri dra Dravidian (other) 
dua Duala dum Dutch, Middle 
(ca. 1050-1350) 
dyu Dyula dzo dz Dzongkha 
ef i Efik egy Egyptian (Ancient) 
eka Ekajuk ell e1 Greek, Modern (post-145 3) 
elx Elamite eng en English 
enm English, Middle (1100-1500) epo eo Esperanto 
est et Estonian eth Ethiopic 
eus eu Basque ewe Ewe 
ewo Ewondo f an Fang 
fao f o F aroese f as fa Persian 
fat Fanti fij fj Fijian 
fin f i Finnish f iu F inno-Ugrian (other) 
fon Fon fra fr French 
frm F rench, Middle fro F rench, Old (ca. 842-1400) 
(ca. 1400-1600) 
fry fy Frisian ful Fulah 
fur F riulian gaa Ga 
gai ga Irish gay Gayo 
gba Gbaya gdh gd Gaelic (Scots) 
gem Germanic (other) gez Geez 
gil Gilbertese glg gl Gallegan 
gmh German, Middle High goh German, Old High 
(ca. 1050-1500) (ca. 750-1050) 
gon Gondi gor Gorontalo 
got Gothic grb Grebo 
grc Greek, Ancient (to 145 3) grn gn Guarani 
guj gu Gujarati gwi Gwich’in 
hai Haida hau ha Hausa 
haw Hawaiian heb he/iw Hebrew 
her Herero hil Hiligaynon 
him Himachali hin hi Hindi 
hit Hittite hmn Hmong 
hmo Hiri Motu hrv hr Croatian 
hun hu Hungarian hup Hupa 

 
%%page page_487                                                  <<<---3
 
468 
Internationalization issues 
Language codes and names (cont) 
hye 
ibo 
iku 
ilo 
inc 
ine 
ira 
is 1 
jaw 
jpr 
kaa 
kac 
kam 
kar 
kat 
kaw 
khm 
kik 
kir 
kok 
kon 
kos 
kro 
kua 
kur 
lad 
lam 
1at 
1ez 
lit 
1oz 
1ua 
lug 
lun 
lus 
mag 
mai 
ma1 
map 
mas 
mdr 
mga 
min 
113' 
iu 
jw 
ka 
ky 
ku 
1a 
1t 
m1 
Armenian 
Igbo 
Inuktitut 
Iloko 
Indic (other) 
Indo-European (other) 
Iranian (other) 
Icelandic 
Javanese 
Judeo-Persian 
Kara-Kalpak 
Kachin 
Kamba 
Karen 
Georgian 
Kawi 
Khasi 
Khmer 
Kikuyu 
Kirghiz 
Konkani 
Kongo 
Kosraean 
Kru 
Kuanyama 
Kurdish 
Ladino 
Lamba 
Latin 
Lezghian 
Lithuanian 
Lozi 
Luba-Lulua 
Ganda 
Lunda 
Lushai 
Magahi 
Maithili 
Malayalam 
Austronesian (other) 
Masai 
Mandar 
Irish, Middle (900-1200) 
Minangkabau 
Macedonian 
iba 
i j o 
ile 
ina 
ind 
ipk 
iro 
it a 
jpn 
jrb 
kab 
kal 
kan 
kas 
kau 
kaz 
khi 
kho 
kin 
kmb 
kom 
kor 
kpe 
kru 
kut 
lah 
lao 
1av 
1in 
101 
1tz 
1ub 
1ui 
luo 
mad 
mah 
mak 
man 
mar 
max 
men 
mic 
mi s 
ie 
ia 
id/in 
ik 
it 
ja 
k1 
ks 
kk 
IW 
ko 
10 
1V 
1n 
1b 
111!‘ 
Iban 
Ijo 
Interlingue 
Interlingua (International 
Auxiliary Language Association) 
Indonesian 
Inupiak 
Iroquoian languages 
Italian 
Japanese 
]udeo-Arabic 
Kabyle 
Kalaallisut 
Kannada 
Kashmiri 
Kanuri 
Kazakh 
Khoisan (other) 
Khotanese 
Kinyarwanda 
Kimbundu 
Komi 
Korean 
Kpelle 
Kurukh 
Kumyk 
Kutenai 
Lahnda 
Lao 
Latvian 
Lingala 
Mongo 
Letzeburgesch 
Luba-Katanga 
Luiseno 
Luo (Kenya and Tanzania) 
Madurese 
Marshall 
Makasar 
Mandingo 
Marathi 
Manx 
Mende 
Micmac 
Miscellaneous languages 
Mon-Khmer (other) 

 
%%page page_488                                                  <<<---3
 
C.l Codes for languages, countries, and scripts 469 
Language codes and names (comfl) 
mlg mg Malagasy mlt mt Maltese 
mni Manipuri mno Manobo languages 
moh Mohawk mol mo Moldavian 
mon mn Mongolian mos Mossi 
mri mi Maori msa ms Malay 
mul Multiple languages mun Munda languages 
mus Creek mwr Marwari 
mya my Burmese myn Mayan languages 
nah Aztec nai North American Indian (other) 
nau na Nauru nav Navajo 
nbl Ndebele, South nde Ndebele, North 
ndo Ndonga nep ne Nepali 
new Newari nia Nias 
nic Niger-Kordofanian (other) niu Niuean 
nld nl Dutch non Norse, Old 
nor no Norwegian nso Sohto, Northern 
nub Nubian languages nya Nyanja 
nym Nyamwezi nyn Nyankole 
nyo Nyoro nzi Nzima 
oci oc Occitan (post-1500) 03' i Ojibwa 
ori Oriya orm om Oromo 
osa Osage oss Ossetic 
ota Turkish, Ottoman (1500-1928) oto Otomian languages 
paa Papuan (other) pag Pangasinan 
pal Pahlavi pam Pampanga 
pan pa Panjabi pap Papiamento 
pau Palauan peo Persian, Old 
(ca. 600-400 B.C.) 
phi Philippine (other) phn Phoenician 
pli Pali pol pl Polish 
pon Pohnpeian por pt Portuguese 
pra Prakrit languages pro Provencal, Old (to 15 00) 
pus ps Pushto que qu Quechua 
raj Rajasthani rap Rapanui 
rar Rarotongan roa Romance (other) 
roh rm Rhaeto-Romance rom Romany 
ron ro Romanian run In Rundi 
rus ru Russian sad Sandawe 
sag sg Sango sai South American Indian (other) 
sal Salishan languages sam Samaritan Aramaic 
san sa Sanskrit sas Sasak 
sat Santali sco Scots 
sel Selkup sem Semitic (other) 
sga Irish, Old (to 900) shn Shan 
sid Sidamo sin si Sinhalese 
sio Siouan languages sit Sino-Tibetan (other) 

 
%%page page_489                                                  <<<---3
 
470 
Intemationalization issues 
Language codes and names (cont.) 
sla 
slv 
smo 
snd 
sog 
son 
spa 
srd 
srr 
ssw 
sun 
sux 
swe 
tah 
tam 
tel 
ter 
tgk 
tha 
tir 
tkl 
tmh 
ton 
tsi 
tso 
tum 
tut 
twi 
ukr 
und 
uzb 
ven 
vol 
wak 
war 
wen 
xho 
YaP 
yor 
zap 
zha 
znd 
zun 
s1 
sm 
sd 
es 
SS 
S11 
SV 
ta 
ta 
ts 
th 
tw 
112 
V0 
xh 
za 
Slavic (other) 
Slovenian 
Samoan 
Sindhi 
Sogdian 
Songhai 
Spanish 
Sardinian 
Serer 
Swati 
Sundanese 
Sumerian 
Swedish 
Tahitian 
Tamil 
Telugu 
Tereno 
Tajik 
Thai 
Tigrinya 
Tokelau 
Tamashek 
Tonga (Tonga Islands) 
Tsimshian 
Tsonga 
Tumbuka 
Altaic (other) 
Twi 
Ugaritic 
Ukrainian 
Undetermined 
Uzbek 
Venda 
Volapiik 
Wakashan languages 
Waray 
Sorbian languages 
Xhosa 
Yapese 
Yoruba 
Zapotec 
Zhuang 
Zande 
Zufii 
slk 
smi 
sna 
snk 
som 
sot 
sqi 
srp 
ssa 
suk 
sus 
swa 
syr 
tai 
tat 
tem 
tet 
tgl 
tig 
tiv 
tli 
tog 
tpi 
tsn 
tuk 
tur 
tvl 
tyv 
uig 
umb 
urd 
Vai 
vie 
vot 
wal 
was 
wol 
yao 
yid 
ypk 
zen 
zho 
zul 
sk 
se 
sn 
so 
st 
sq 
SI‘ 
SW 
‘tt 
tl 
ts 
tk 
us 
111' 
vi 
W0 
yi/ji 
zh 
zu 
Slovak 
Sami languages 
Shona 
Soninke 
Somali 
Sotho, Southern 
Albanian 
Serbian 
Nilo-Saharan (other) 
Sukuma 
Susu 
Swahili 
Syriac 
Tai (other) 
Tatar 
Timne 
Tetum 
Tagalog 
Tigre 
Tiv 
Tlingit 
Tonga (Nyasa) 
Tok Pisin 
Tswana 
Tiirkmen 
Turkish 
Tuvalu 
Tuvinian 
Uighur 
Umbundu 
Urdu 
Vai 
Vietnamese 
Votic 
Walamo 
Washo 
Wolof 
Yao 
Yiddish 
Yupik languages 
Zenaga 
Chinese 
Zulu 

%==========490==========<<<---2
 
%%page page_490                                                  <<<---3
 
C.1 Codes for languages, countries, and scripts 471 
Table C.2: Country codes and names (ISO 3166) 
AF AFG Afghanistan AL ALB Albania 
DZ DZA Algeria AS ASM American Samoa 
AD AND Andorra A0 AGO Angola 
AI AIA Anguilla AQ ATA Antarctica 
AG ATG Antigua and Barbuda AR. ARG Argentina 
AM ARM Armenia Aw ABw Aruba 
AU AUS Australia AT AUT Austria 
AZ AZE Azerbaijan BS Bi-IS Bahamas 
BH Bl-IR. Bahrain BD BGD Bangladesh 
BB BRB Barbados BY BLR. Belarus 
BE BEL Belgium BZ BLZ Belize 
BJ BEN Benin BM BMU Bermuda 
BT BTN Bhutan B0 BOL Bolivia 
BA BIH Bosnia and Herzegovina BW BWA Botswana 
BV BVT Bouvet Island BR. BRA Brazil 
I0 IOT British Indian Ocean Territory BN BRN Brunei Darussalam 
BG BGR Bulgaria BF BFA Burkina Faso 
BI BDI Burundi KH KHM Cambodia 
CM CMR Cameroon CA CAN Canada 
CV CPV Cape Verde KY CYM Cayman Islands 
CF CAF Central African Republic TD TCD Chad 
CL Cl-IL Chile CN Cl-IN China 
CX CXR. Christmas Island CC CCK Cocos (Keeling) Islands 
C0 COL Colombia KM COM Comoros 
CG COG Congo CD COD Congo, Democratic Republic 
CK COK Cook Islands CR. CRI Costa Rica 
CI CIV Céte d’Ivoire HR. HRV Croatia 
cu CUB Cuba CY CYP Cyprus 
CZ CZE Czech Republic DK DNK Denmark 
DJ DJI Djibouti DM DMA Dominica 
D0 DOM Dominican Republic TP TMP East Timor (provisional) 
EC ECU Ecuador EG EGY Egypt 
SV SLV El Salvador GQ GNQ Equatorial Guinea 
ER. ERI Eritrea EE EST Estonia 
ET ETI-I Ethiopia FK FLK Falkland Islands (Malvinas) 
F0 FRO Faroe Islands FJ FJI Fiji 
FI FIN Finland FR. FRA France 
FX FXX France, Metropolitan GF GUF French Guiana 
PF PYF French Polynesia TF ATF French Southern Territories 
GA GAB Gabon GM GMB Gambia 
GE GEO Georgia DE DEU Germany 
GI-I GHA Ghana GI GIB Gibraltar 
GR. GRC Greece GL GRL Greenland 
GD GRD Grenada GP GLP Guadaloupe 
GU GUM Guam GT GTM Guatemala 

 
%%page page_491                                                  <<<---3
 
472 
Internationalization issues 
Country codes and names (cont) 
GN 
GY 
HK 
IS 
ID 
IQ 
IL 
JM 
J 0 
KE 
KP 
KW 
LA 
LB 
LR 
LI 
LU 
MK 
MV 
MT 
MG 
MK 
MN 
MA 
NR 
NL 
NC 
NI 
NG 
NF 
N0 
PK 
PA 
PY 
PH 
PL 
PR 
RU 
SH 
GIN 
GUY 
HMD 
HKG 
ISL 
mu 
IRQ 
ISR 
JAM 
JOR 
KEN 
pnx 
KWT 
LAO 
LBN 
LBR 
LIE 
LUX 
MKD 
MWI 
MDV 
MLT 
MTQ 
MUS 
MEX 
MDA 
MNG 
MAR 
NRU 
NLD 
NCL 
NIC 
NGA 
NFK 
NOR 
PAK 
PAN 
PRY 
PHL 
POL 
PRI 
REU 
RUS 
SHN 
Guinea 
Guyana 
Heard Island and McDonald 
Islands 
Hong Kong 
Iceland 
Indonesia 
Iraq 
Israel 
Jamaica 
Jordan 
Kenya 
Korea, North 
Kuwait 
Laos 
Lebanon 
Liberia 
Liechtenstein 
Luxembourg 
Macedonia 
Malawi 
Maldives 
Malta 
Martinique 
Mauritius 
Mexico 
Moldova 
Mongolia 
Morocco 
Myanmar 
Nauru 
Netherlands 
New Caledonia 
Nicaragua 
Nigeria 
Norfolk Island 
Norway 
Pakistan 
Panama 
Paraguay 
Philippines 
Poland 
Puerto Rico 
Réunion 
Russian Federation 
Saint Helena 
GW 
HT 
HN 
HU 
IN 
IR 
IE 
IT 
JP 
KZ 
KI 
KR 
KG 
LV 
LS 
LY 
LT 
MD 
MG 
MY 
ML 
MI-I 
MR 
YT 
FM 
MC 
MS 
MZ 
NA 
NP 
AN 
NZ 
NE 
NU 
MP 
OM 
PW 
PG 
PE 
PN 
PT 
QA 
R0 
RW 
KN 
GNB 
HTI 
HND 
HUN 
IND 
IRN 
IRL 
ITA 
JPN 
KAZ 
KIR 
KOR 
KGZ 
LVA 
LSO 
LBY 
LTU 
MAC 
MDG 
MY S 
MLI 
MI-IL 
MRT 
MY T 
FSM 
MCO 
MSR 
MOZ 
NAM 
NPL 
ANT 
NZL 
NER 
NIU 
MNP 
OMN 
PLW 
PNG 
PER 
PCN 
PRT 
QAT 
ROM 
RWA 
KNA 
Guinea-Bissau 
Haiti 
Honduras 
Hungary 
India 
Iran 
Ireland 
Italy 
Japan 
Kazakhstan 
Kiribati 
Korea, South 
Kyrgyzstan 
Latvia 
Lesotho 
Libya 
Lithuania 
Macau 
Madagascar 
Malaysia 
Mali 
Marshall Islands 
Mauritania 
Mayotte 
Micronesia 
Monaco 
Montserrat 
Mozambique 
Namibia 
Nepal 
Netherlands Antilles 
New Zealand 
Niger 
Niue 
Northern Mariana Islands 
Oman 
Palau 
Papua New Guinea 
Peru 
Pitcairn 
Portugal 
Qatar 
Romania 
Rwanda 
Saint Kitts and Nevis 

 
%%page page_492                                                  <<<---3
 
C.1 Codes for languages, countries, and scripts 473 
Country codes and names (cont) 
LC LCA Saint Lucia PM SPM Saint Pierre and Miquelon 
VC VCT Saint Vincent and the Grenadines WS WSM Samoa 
SM SMR San Marino ST STP Sao Tome and Principe 
SA SAU Saudi Arabia SN SEN Senegal 
SC SYC Seychelles SL SLE Sierra Leone 
SG SGP Singapore SK SVK Slovakia 
SI SVN Slovenia SB SLB Solomon Islands 
S0 SOM Somalia ZA ZAF South Africa 
GS SGS South Georgia and the South ES ESP Spain 
Sandwich Islands 
LK LKA SriLanka SD SDN Sudan 
SR. SUR Suriname SJ SJM Svalbard andJan Mayen 
SZ SWZ Swaziland SE SWE Sweden 
CH CHE Switzerland SY SYR Syria 
TW TWN Taiwan TJ TJK Tajikistan 
TZ TZA Tanzania TH THA Thailand 
TG TGO Togo TK TKL Tokelau 
TO TON Tonga TT TTO Trinidad and Tobago 
TN TUN Tunisia TR. TUB. Turkey 
TM TKM Turkmenistan TC TCA Turks and Caicos Islands 
TV TUV Tuvalu UG UGA Uganda 
UA UKR. Ukraine AE ARE United Arab Emirates 
GB GBR. United Kingdom US USA United States 
UM UMI United States Minor Outlying UY URY Uruguay 
Islands 
UZ UZB Uzbekistan VU VUT Vanuatu 
VA VAT Vatican City State (Holy See) VE VEN Venezuela 
VN VNM Vietnam VG VGB Virgin Islands, British 
VI VIR Virgin Islands, U.S. WF WLF Wallis and Fortuna Islands 
EH ESH Western Sahara (provisional) YE YEM Yemen 
YU YUG Yugoslavia ZM ZMB Zambia 
ZW ZWE Zimbabwe 
Table C.3: Script codes and names (ISO 15924) 
Am Ama Aramaic® Ar Ara Arabic® 
Av Ave Avestan® Bn Ben Bengali@ 
Bh Bh.m Brahmi(Ashoka)@ Bi Bid Buhid@ 
Bo Bod Tibetan@ Bp Bpm Bopomofo© 
Br Brl Braille© Bt Btk Batak@ 
Bu Bug Buginese (Makassar) @ By Bys Blissymbols@ 
Ca Cam Cham@ Ch Chu Old Church Slavonic© 
Ci Cir Cirth© Cm Cmn Cypro Minoan© 
® Hieroglyphic and cuneiform, ® Right-to-left alphabetic, © Left-to-right alphabetic, 
(9 Brahmi-derived, © Syllabic, © Ideographic, ® Undeciphered. 

 
%%page page_493                                                  <<<---3
 
474 
Intemationalization issues 
Script codes and names (comfl) 
Co 
Cy 
Dv 
Eh 
E1 
Et 
Gm 
Gu 
He 
Hm 
Hr 
I-Iv 
Iv 
J 1 
Ka 
Kx 
Kh 
Kr 
Lf 
Lo 
LP 
Md 
M1 
My 
Nb 
Or 
Ph 
P1 
Pr 
Rn 
Sa 
S1 
SJ 
53' 
Ta 
Te 
Th 
Tw 
Vs 
Xf 
Xu 
Zx 
Zz 
Cop 
Cyr 
Dvn 
Egh 
E11 
Eth 
Gmu 
Gu j 
Heb 
Hrg 
Hvn 
Iv1 
J 1g 
Kam 
Kax 
Khn 
Krn 
Laf 
Lao 
Lpc 
Mda 
M1m 
Mya 
Nbb 
Dry 
Pah 
Pld 
Prm 
Rnr 
Sar 
S1b 
Syj 
Syr 
Tam 
Te 1 
Tha 
Twr 
Vsp 
Xf a 
Xug 
Zxx 
Zzz 
Coptic ® 
Cyrillic © 
Devanagari (N agari) @ 
Egyptian hieratic (D 
Greek (3 
Ethiopic © 
Gurmukhi @ 
Gujarati @ 
Hebrew ® 
Pahawh Hmong © 
Hiragana © 
Kok Turki nmic ® 
Indus Valley (7) 
Cherokee syllabary (9 
Georgian (Mxedruli) ® 
Georgian (Xucuri) © 
Hangul + Han 
Karenni (Kayah Li) @ 
Latin (Fraktur variant) 6) 
Lao @ 
Lepcha (Rong) @ 
Mandaean ® 
Malayalam @ 
Burmese @ 
Linear B (5) 
Oriya @ 
Pahlavi ® 
Pollard Phonetic 6) 
Old Permjc © 
Runic (Germanic) © 
South Arabian (2 
Canadian Aboriginal Syllabics ® 
Syriac Gacobite variant) ® 
Syriac (Estrangelo) ® 
Tamil @ 
Telugu @ 
Thai @ 
Tengwar ® 
Visible Speech (33 
Cuneiform, Old Persian (2 
Cuneiform, Ugaritic ® 
Code for unwritten languages 
Code for uncoded 
Cp 
Ds 
Ed 
E3 
E0 
G1 
Gt 
Ha 
H3 
H0 
Hu 
Hy 
J a 
Jw 
Kn 
Km 
Kk 
Ks 
1-8 
La 
Mh 
Me 
Mn 
Na 
08 
Os 
Ph 
Pq 
Ps 
Rr 
S i 
Sw 
Sn 
Ts 
Tb 
Tf 
Tn 
Va 
Xa 
Xk 
Yi 
2}’ 
Cpr 
Dsr 
Egd 
Esy 
Eos 
Glg 
Gth 
Han 
Hgl 
Hoo 
Hun 
I-lye 
J ap 
Jwi 
Kan 
Khm 
Kkn 
Kst 
Lag 
Lat 
May 
Mer 
Mon 
Naa 
Osm 
Phx 
Pqd 
Pst 
Rro 
S in 
SW 
Syn 
Tag 
Tbw 
Tfn 
Tna 
Vai 
Xas 
Xkn 
Yi i 
Zyy 
Cypriote syllabary © 
Deseret (Mormon) (3 
Egyptian demotic (D 
Egyptian hieroglyphs (D 
Etruscan and Oscan © 
Glagolitic (3 
Gothic ® 
Han ideographs © 
Hangul © 
Hanunéo @ 
Old Hungarian runic ® 
Armenian (3 
Han + Hiragana + Katakana 
Javanese @ 
Kannada @ 
Khmer @ 
Katakana ® 
Kharoshthi @ 
Latin (Gaelic variant) (3 
Latin (3 
Mayan hieroglyphs (D 
Meroitic ® 
Mongolian ® 
Linear A (53 
Ogham ® 
Osmanya © 
Phoenician ® 
Klingon pIQaD © 
Phaistos Disk (7) 
Rongo ® 
Sinhala @ 
Shavian (Shaw) (3 
Syriac (N estorian variant) ® 
Tagalog @ 
Tagbanwa @ 
Tifinagh ® 
Thaana ® 
Vai © 
Cuneiform, Sumero Akkadian (D 
Hiragana + Katakana (5) 
Y1 (53 
Code for undetermined 
(D Hieroglyphic and cuneiform, ® Right-to-left alphabetic, ® Left-to-right alphabetic, 
® Brahmi-derived, (ED Syllabic, © Ideographic, ® Undeciphered. 

 
%%page page_494                                                  <<<---3
 
C.2 The Unicode standard 
475 
Table C.4: The ISO/IEC 8859 standards and the language families covered 
8859-1 Western Europe, Latin America (Latin 1) 
8859-2 Eastern European languages (Latin 2) 
8859-3 Other Latin script languages (Latin 3) 
8859-4 North European languages (Latin 4) 
8859-5 Latin/Cyrillic 
8859-6 Latin/Arabic 
8859-7 Latin/Greek 
8859-8 Latin/Hebrew 
8859-9 Variant of Latin 1 for Turkey (Latin 5) 
8859-10 Lappish/Nordic/Eskimo languages (Latin 6) 
8859-11 Latin/Thai 
8859-12 (unassigned) 
8859-13 Baltic Rim (Latin 7) 
8859-14 Celtic (Latin 8) 
8859-15 Variant of Latin 1 for French, Finnish, and Euro symbol (Latin 9) 
C.2 The Unicode standard 
The Unicode standard (Unicode Consortium (1996)) is a universal character encoding standard used for representing multilingual texts on electronic media. It is 
supposed to make the international exchange of computer text files more straightforward. It not only contains most commonly used text characters of all major languages of the world but encodes various technical and mathematical symbols so 
that scientists and engineers might also gainfully adopt this standard. 
Unicode was developed mainly by industry and is supported by all major 
computer manufacturers. It is fully compatible with ISO/IEC standard 10646-1 
(ISO/IEC:10646-1:1993, 1993). Presently, working closely with academic and research groups, the Unicode Consorfium, which oversees the coordination of the 
Unicode standard [=>UNICODE], is busy filling the remaining slots of the approximately 65,000 positions with an agreed set of mathematical symbols and technical 
characters. 
Unicode is basically a 16-bit extension of the 7-bit ASCII standard and the 8-bit 
Latin 1 (see Table C.4) character code. However, even when 65 ,000 characters seem 
sufficient for encoding thousands of commonly used characters of all major world 
languages, more codepoints are needed for some applications. Therefore Unicode 
provides an extension mechanism, UTF-16, that maps onto further 16-bit planes of 
ISO/IEC 10646-1, allowing for the encoding of about 1,000,000 characters. This 
is more than sufficient to deal with all characters known to mankind, including all 
historic scripts of the world. 
Figure C.1 shows the layout of the Unicode plane: 25 6 rows numbered 00 to FF 
in hexadecimal of 25 6 characters. This row number, shown at the left-hand side of 
the figure, corresponds to the most significant byte of the 16-bit character code. By 

 
%%page page_495                                                  <<<---3
 
476 Internationalization issues 
(50 646RV Latin-l Supplement 
Latin Extended-A Latin Extended-B 
Latin Extended-B IPA Extensions Spacing Modifier Letter 
Combining Diacritical Marks Greek 
Cyrillic 
Armenian 
Arabic 
Devanagari Bengali 
Gunnukhi Gujarati 
Oriya Tamil 
Telugu Kannada 
Malayalam 
Thai Lao 
Hangul J amo 
Latin Extended Additional 
Greek Extended 
General Punctuation ® Currency Symbols 
Letterlike Symbols Number Forms Arrows 
Mathematical Operators 
Miscellaneous Technical 
Control Pictures OCR Enclosed Alphanumerics 
Box Drawing Geometric Shapes 
Miscellaneous Dingbats 
Dingbats 
CJK Symbols and Punctuation Hiragana Katakana 
Bopomofo Hangul Compatibility Jamo ' C] K Miscellaneous 
CJ K Miscellaneous Enclosed C] K Letters and Months 
C] K Compatibility 
Hangul 
Hangul Supplementary-A 
Hangul ‘‘ pplementary -B 
CJ K Unified Ideographs 
Private Use Area 
CJ K Compatibility ldeographs 
Alphabetic Presentation Forms 
Arabic Presentation Forms-A 
© ® @ Arabic Presentation Forms-B 
Halfwidth. Fullwidth Forms and Specials 
® Superscripts and Subscripts ® Combining Diacritical Marks for Symbols 
® Combining Half Marks ® CJ K Compatibility Forms © Small Form Variants 
Figure C.1: Character layout of Unicode (ISO/IEC 10646 BMP) 

 
%%page page_496                                                  <<<---3
 
C.2 The Unicode standard 
477 
construction Unicode is identical to the base plane of the 32-bit ISO/IEC 10646-1 
standard. The figure shows a few of the scripts covered by Unicode, namely Latin, 
Greek, Cyrillic, Armenian, Hebrew, Arabic, Devanagari, Bengali, Gurmukhi, Gujarati, Oriya, Tamil, Telugu, Kannada, Malayalam, Thai, Lao, Georgian, Tibetan, 
Japanese Kana, the complete set of modern Korean Hangul, and a unified set 
of Chinese/Japanese/Korean (CJK) ideographs. More scripts and characters, such 
as Ethiopic, Canadian Syllabics, Cherokee, Sinhala, Syriac, Burmese, Khmer, and 
Braille are in the process of being added. 
The central part (rows 20-27) of Figure C.1 shows that Unicode also includes 
punctuation marks, currency symbols, diacritics, mathematical symbols, technical 
symbols, arrows, dingbats, and so on. 
Rows 4E-9F contain ideographs for East-Asian languages. To avoid duplicate 
encoding, characters were unified within scripts across languages. This means that 
equivalent characters in form were mapped to a single code. In particular, Chinese/]apanese/Korean (CJK) consolidation is achieved by assigning a single code 
for each ideograph that is common to more than one of these languages. This 
set of characters is known under the name “Unified Han." This unification of the 
character codes does not mean, however, that the characters in question cannot be 
rendered by appropiate customized fonts to make them appear “natural" to native 
readers. 
C.2.l Character codes and glyphs 
It is important to realize that Unicode encodes only the code (“semantic meaning") 
of a character, not its rendering on an output medium (paper, screen, audio). For 
instance, “LATIN CAPITAL LETTER A" (codepoint U-0041), “GREEK CAPITAL 
LETTER ALPHA" (11-0391), and “CYRILLIC CAPITAL LETTER A" (U-0410) have 
a different semantic meaning, although they have the same visual representation, 
the glyph “A." 
Unicode defines only how characters are interpreted, not how glyphs are rendered as images. Unicode has nothing to say about size, weight, shape, writing 
direction, and so on of characters on the output screen. That is the responsibility 
of the glyph-rendering software in the viewer or printing engine. 
C.2.2 Unicode and ISO/IEC 10646-1 
Unicode is closely aligned with the international standard ISO/IEC 10646~1, also 
known as UCS (Universal Character Set). In fact, by construction, Unicode is identical to ISO/IEC 10646-1 and its amendments. 
The ISO/IEC 10646-1 standard, however, has a much greater application area, 
since it is basically a 32-bit code; thus it can encode over two billion characters. The 
canonical form of ISO/IEC 10646-1 uses a four-dimensional coding space consisting 
of 25 6 three-dimensional groups. Each group consists of 25 6 two-dimensional planes 

 
%%page page_497                                                  <<<---3
 
478 
Internationalization issues 
Group 00 /T 
Group 7F 
Each group contains 256 planes 
Group 01 / 
//Plane 00 of Group 7F 
Each plane has Plane 00 of Group 01 
256 x 256 
character 
cells 
Plane FF of Group 00 
Plane 00 of Group 00 
Figure C.2: Entire 4-octet coding space of ISO/IEC 10646-1 
Flowoctet 
Supplementary planes (223) 
L Cell-octet 
00 so FF 
A-zone 
4E 
l-zone 
AD 
0-zone 
E0 
Fl-zone °1 
FF 
Basic Multilingual Plane 
4 
Plane-octet 
l 
Private use 
planes (32) 
Figure C.3: Structure of Group 00 of ISO/IEC 10646-1 

 
%%page page_498                                                  <<<---3
 
C.2 The Unicode standard 
479 
with each plane containing 256 one-dimensional rows, each having 256 cells (see 
Figure C.2). Thus character codes consist of up to four octets, which, ordered from 
least to most significant, correspond to the cell (C-octet), row (R-octet), plane (Poctet), and group (G-octet) numbers. 
The first plane (Plane 00) of Group 00 is called the Basic Multilingual Plane 
(BMP). It is, by construction, identical to the Unicode layout. 
The subsequent 223 planes (01 to DF) of Group 00 (see Figure C.3), as well 
as planes 00 to FF in Groups 01 to 5 F are reserved for further standardization (for 
instance to cope with rarely used Han characters, old historic scripts). The last 32 
planes (E0 to FF) of Group 00, as well as all code positions of 32 groups (60 to 7F) 
are reserved for private use, and are not specified in ISO/IEC 10646-1. 
C.2.3 UTF-8 and UTF-16 encodings 
The ISO/IEC 10646-1 standard defines two transformation formats UTF-8 and 
UTF-16, which were also adopted by Unicode (Appendix A of Unicode Consortium (1996)). The UTF-16 transformation assumes 16-bit characters, just like Unicode itself, but it allows for a certain range of characters to be used as an extension 
mechanism to access an additional million characters using 16-bit character pairs. 
Perhaps more useful in the near future is the UTF-8 transformation format 
that transforms all Unicode characters into a variable length encoding of up to four 
bytes. The first 128 characters (the ASCII subset) have the same bit sequences in 
UTF-8 and in ASCII, making it easy to handle documents using only that subset. 
This means that a lot of existing software can be used with Unicode (and UTF-8) 
without a major software rewrite. 
UTF-8 can be described as a method to transform Unicode or ISO 10646-1 
streams into 8-bit streams, leaving ASCII as it is and expanding the other characters 
to up to six bytes. In fact, three bytes suffice for the 16-bit Unicode range; four 
bytes are enough for encoding about a million characters. The latter approach is 
similar to UTF-16 that reserves 2048 codepoints in Unicode (D800-DFFF) to index 
one million additional characters. These one million codepoints should be enough 
for all the rare Chinese ideograms and historical scripts that do not fit into the Base 
Multilingual Plane of ISO 10646. 
Following we show how the 31 bits of the ISO 10646 code are distributed over 
up to six UTF-8 bytes. It must be stressed that for Unicode, never more than three 
bytes are needed. 
ISO 10646 range covered 
UTF-8 representation 
Bits Hex Min Hex Max Byte Sequence in Binary 
7 00000000 0000007f Ovvvvvvv 
11 00000080 000007FF 110vvvvv 10vvvvvv 
16 00000800 0000FFFF 1110vvvv 10vvvvvv 10vvvvvv 
21 00010000 OOIFFFFF 11110vvv 10vvvvvv 10vvvvvv 10vvvvvv 
26 00200000 OSFFFFFF 111110vv 10vvvvvv 10vvvvvv 10vvvvvv 10vvvvvv 
31 04000000 7FFFFFFF 1111110v 10vvvvvv 10vvvvvv 10vvvvvv 10vvvvvv 10vvvvvv 

 
%%page page_499                                                  <<<---3
 
480 
Internationalization issues 
It is seen that any UTF-8 octet that starts with binary 0 is a sequence of one 
(pure ASCII). Any octet starting with 10 is a trailing octet of a multioctet sequence. 
Any other octet is the start of a multioctet UTF-8 sequence, with the number of 
binary 1 digits indicating the number of octets of the multioctet encoding sequence. 
This makes it efficient to find the start of a character starfing from an arbitrary 
location in an octet stream. 
To allow coding for a million characters with UTF-16, the range 0000D800 
to OOOODFFF is excluded from this conversion process. The following table shows 
in hexadecimal form the various ranges of the UTF-I6 and UTF-8 sequences corresponding to the complete UCS-4 encoding. A semicolon shows the separation 
between the basic information unit in each case. 
UCS-4 UTF-16 UTF-8 
0000 0001; 0001; 01; 
0000 007F; 007F; 7F; 
0000 0080; 0080; C2; 80; 
0000 07FF; 07FF; DF; BF; 
0000 0800; 0800; E0; A0; 80; 
0000 FFFF; FFFF; EF; BF; BF; 
0001 0000; D800; DC00; F0; 90; 80; 80; 
0010 FFFF; DBFF; DFFF; F4; 8F; BF; BF; 
001F FFFF; F7; BF; BF; BF; 
0020 0000; F8; 88; 80; 80; 80; 
03FF FFFF; FB; BF; BF; BF; BF; 
0400 0000; FC; 84; 80; 80; 80; 80; 
7FFF FFFF; FD; BF; BF; BF; BF; BF; 
C.3 Foreign languages in XML 
How do XML and XSL deal with non-ASCII sources? We explained in Section 6.5.1 
that XSL is based on Unicode. Thus it should not be Very difficult to deal with most 
common world languages. Therefore let us look at two examples that illustrate how 
to deal with such situations. 
C.3.1 Latin-based encodings 
In the first case we use an 8-bit ASCII extension, called Latin 1 (ISO-8859-1; see 
Table C.4), to represent a French text. The source file invitationf r . xml follows: 
<?xml version="1.0" encoding=``ISO-8859-1''?) 
<!DOCTYPE invitation SYSTEM "invitationfr.dtd"> 
<invitation> 
<!-- ++++ Partie entéte ++++ --> 
<entéte> 
<a>Anna, Bernard, Didier, Johanna</é> 
<date>Vendredi prochain 3 20 heures</date> 
<ou>Le Café du Web</ofl> 
<pourquoi>Mon premier bébé XML</pourquoi> 
</entéte> 
<!-- ++++ Partie corps ++++ --> 
:5»om\:ax«n-a>w~%==========500==========<<<---1
%==========500==========<<<---2
 
%%page page_500                                                  <<<---3
 
C.3 Foreign languages in XML 
481 
<corps> 
<par> 
J’ai 1e plaisir de vous inviter a la célébration 
de la naissance d’<emph>Invitation</emph>, mon 
premier enfant document XML. 
</par> 
<par> 
S’i1 vous plait, faites tout votre possible pour me rejoindre 
vendredi prochain. Et n’oub1iez pas d’emmener vos amis. 
</par> 
<par> 
Je me réjouis <emph>vraiment</emph> d’avance de votre présence. 
</par> 
</corps) 
<!-- ++++ Partie finale 
<fin> 
<5ignature>Miche1</signature) 
</fin) 
</invitation> 
++++ --> 
Accented characters were used in the text as well as for some of the element types, 
as seen in the DTD invitationfr . dtd, which follows: 
: 5 o m u 0 m + w w _ 
<?xm1 version=’1.0’ encoding=``ISO-8859-1''?> 
<!-- DTD invitation (version frangaise) --> 
<!-- 11 novembre 1998 mg --> 
<!ELEMENT invitation (entéte, corps, fin) > 
<!ELEMENT entéte (a, date, on, pourquoi?) > 
<!ELEMENT date (#PCDATA) > 
<!ELEMENT a (#PCDATA) > 
<!ELEMENT ofi (#PCDATA) > 
<!ELEMENT pourquoi (#PCDATA) > 
<!ELEMENT corps (par+) > 
<!ELEMENT par 
<!ELEMENT emph 
<!ELEMENT fin 
<!ELEMENT signature 
(#PCDATAlemph)* > 
(#PCDATA) > 
(signature) > 
(#PCDATA) > 
We can use the xt or koalaxsl tool for transforming this XML file into, for 
instance, HIEX. We chose to use the Koala tool (see Section 7.6.4.3) with the following XSL style sheet invlatlfr .xsl: 
E G E E S E S c m u o m a w N H 
S 
Q 
3 
<?xm1 version=’1.0’ encoding=``ISO-8859-1''?> 
<xs1:stylesheet xmlns:xs1="http://www.w3.org/TR/VD-xsl" 
default-space=``strip'' 
indent-resu1t=``no'' 
result-ns=""> 
<xs1:temp1ate match="/"> 
<xs1:text>\documentclass[francais]{article} 
\usepackage{invitation} 
\usepackage[T1]{fontenc} 
\begin{document} 
</xs1:text> 
<xs1:app1y-templates/> 
<x51:text>\end{document} 
</xs1:text> 
</xs1:temp1ate> 
<xs1:temp1ate match="invitation/entéte"> 
<xs1:text>\begin{Front} 

 
%%page page_501                                                  <<<---3
 
482 
Interrnafionalizafion issues 
\To{</xs1:text> 
<xs1:va1ue-of se1ect=``a''/> 
<xs1:text>} 
\Date{</xs1:text> 
<xs1:va1ue-of se1ect=``date''/> 
<xs1:text>} 
\Where{</xs1:text> 
<xs1:va1ue-of se1ect=``ou''/> 
<xs1:text>} 
\Why{</xs1:text> 
<xs1:va1ue-of se1ect=``pourquoi''/> 
<xs1:text>} 
\end{Front} 
</xs1:text> 
</xs1:temp1ate> 
<xs1:temp1ate match="invitation/corps"> 
<xs1:text>\begin{Body} 
</xs1:text> 
<xs1:app1y-templates/> 
<xs1:text>\end{Body} 
</xs1:text> 
</xs1:temp1ate> 
<xs1:temp1ate match="invitation/corps/par") 
<xs1:text>\par</xsl:text> 
<xs1:app1y-templates/> 
</xs1:temp1ate> 
<xs1:temp1ate match="invitation/corps/par/emph"> 
<xs1:text>\emph{</xsl:text> 
<xs1:app1y-templates/> 
<xs1:text>}</xs1:text> 
</xs1:temp1ate> 
<xs1:temp1ate match="invitation/fin") 
<xs1:text>\begin{Back} 
\Signature{</xs1:text> 
<xs1:va1ue-of se1ect=``signature''/> 
<xs1:text>} 
\end{Back} 
</xs1:text> 
</xs1:temp1ate> 
</xs1:stylesheet> 
On the first line we note, as in all the other XML files in this section, the presence of the encoding attribute with a Value equal to ISO-8859-1. We must specify 
the encoding, as explained in section 6.5.3, since only UTF-8 and Unicode are recognized by default. Furthermore (lines 8-11), we are specifying a few more BTEX 
commands than when we were dealing with English-only texts (see Section 7.3.2 on 
 295). In addition, we augmented the package file invitation. sty somewhat 
to deal with the French header. 
\-:y\v-4>w~>-Z invitation.sty 
Z Package to format invitation.xm1 
\setlength{\textwidth}{22pc} 
\setlength{\parskip}{1ex} 
\setlength{\parindent}{0pt} 
\pagestyle{empty}Z% Turn off page numbering 
\RequirePackage{array,ca1c} 

 
%%page page_502                                                  <<<---3
 
C.3 Foreign languages in XNIL 
\newcommand{\ToTit1e}{To whom} 
\newcommand{\WhyTit1e}{Occasion} 
\newcommand{\WhereTit1e}{Venue} 
\newcommand{\DateTit1e}{When} 
\newcommand{\SignatureTit1e}{From} 
\Dec1are0ption{francais}{% French text for fixed texts 
\renewcommand{\ToTit1e}{A} 
\renewcommand{\WhyTit1e}{A 1’occasion de} 
\renewcommand{\WhereTit1e}{0fl} 
\renewcomma.nd{\DateTit1e}{Quand} 
\renewcommand{\SignatureTit1e}{De la part de}} 
\newenvironment{Front}% 
{\begin{center} 
\Huge\sffami1y INVITATION 
\end{center} 
} 
{\begin{f1ush1eft} 
\ru1e{\linewidth}{1pt}\\[2mm] 
\begin{tabu1ar}{@{}>{\bfseries}11@{}} 
\TOTitle: & \@T0 
\WhyTit1e: & \@Why \\ 
\WhereTit1e: & \@Where \\ 
\DateTit1e: & \@Date 
\end{tabu1aI}\\[2mm] 
\ru1e{\linewidth}{1pt} 
\end{f1ush1eft} 
} 
\newenvironment{B0dy}{\vspace*{\parskip}}{\vspace*{\parskip}} 
\newenvironnent{Back} 
{\begin{f1ush1eft}} 
{\hspace*{.5\linewidth}\fbox{\SignatureTit1e: \emph{\0Sig}} 
\end{f1ush1eft} 
} 
\newcommand{\To}[1]{\gdef\0To{#1}} 
\newcommand{\Date}[1]{\gdef\@Date{#1}} 
\newcommand{\Where}[1]{\gdef\@Where{#1}} 
\newcommand{\Why}[1]{\gdef\@Why{#1}} 
\newcommand{\Signature}[1]{\gdef\@Sig{#1}} 
\Process0ptions 
l\I'\II\lI\~:I\lI\ll\Iv-->-vucn-->-:->->->--naxv->w--o~omua~u..»w~._o,¢m 
& 3 3 $ 3 3 3 E 3 S K 3 E S S E S 3 Q 3 
As can be seen, we parameterized the “fixed texts" that are used in the heading and 
the signature. We also check for the presence of the option francais to redefine 
the “fixed text" as needed. This more complex version should be compared to the 
previous one on page 296; the result is shown in Figure C.4. 
C.3.2 Handling non-Latin encodings with UTF-8 
Since XML can handle Unicode in a native way, we expect it should not be too 
involved to code languages that use non-Latin alphabets, such as Russian, Greek, 
Chinese, Japanese, Arabic, and so on. In this section we show how you can handle 
Russian, Greek, and a little math in a single file without problems; the same applies 
to more complex situations, however. 
We have used the Yudit [=>YUDIT] Unicode editor, which supports input in 
several languages and allows you to read and write in many encodings. It was originally written by Gaspar Sinai and features a relatively simple and intuitive graphical 
user interface. 
483 

 
%%page page_503                                                  <<<---3
 
484 
Internafionalizafion issues 
INVITATION 
« 
A: 
Oil: 
A l’occasion de: Mon premier bébé XML 
Quand: Vendredi prochain a 20 heures 
Anna, Bernard, Didier, Johanna 
Le Café du Web 
J ’ai le plaisir de Vous inviter a la célébration de la naissance 
d’ Invitation, mon premier enfant document XML. 
S’il Vous plait, faites tout votre possible pour me rejoindre 
vendredi prochain. Et n’oubliez pas d’emmener vos amis. 
Je me réjouis vmiment d’aVance de Votre présence. 
De la part de: Michel 
Figure C.4: A French invitation (ETEX version) 
Figure C.5 shows an XML-coded file. Its first part shows three ways you can 
input Russian text. We reproduce (part) of these lines here for closer scrutiny: 
ZISOcyr1; 
]> 
<mydoc> 
3 3 Z 3 o m u o m A w N H 
</par) 
<?xml version="1.0"?> 
<!DOCTYPE mydoc [ 
<IELEMENT mydoc (#PCDATA)> 
<!ENTITY Z ISOcyr1 SYSTEM "ISOcyr1.pen"> 
<par>The word Russian (asficfigfigazayaz) in Cyrillic: <br/> 
Using ISO Cyrillic set: 
&Rcy;&ucy;&scy;&scy;&kcy;&icy;&jcy; <br/> 
Using XML Unicode entities: 
&#x0420;&#x0443;&#x0441;&#x0441;&#x043a;&#x0438;&#x0439; 
Lines 4-5 define an external entity set ISOcyr1, which will be read on the local 
system in the file ISOcyr1 .pen. It contains definitions for Cyrillic letters, such as: 
<!ENTITY rcy 
<!ENTITY Rcy 
<IENTITY scy 
<!ENTITY Scy 
"&#x440;"> <!--smal1 er, Cyrillic --> 
"&#x420;"> <!--capital ER, Cyrillic --> 
"&#x441;“> <!--sma1l es, Cyrillic --> 
"&#x421;"> <!--capital ES, Cyrillic --> 

 
%%page page_504                                                  <<<---3
 
C.3 Foreign languages in XML 485 
<?xrnl version="l.0"?> 
3 <!DOCTYPE rnydocf 
. <!ELEM ENT mycloc (#PC DATA)> 
<!ENTITY % ISOI:yrl SYSTEM "ISOI:yr1.pen"> 
‘ %ISOI:yrl; 
J> 
<mydoc> 
=' <par>The word Russian (Pyccxnfi) in Cyrillic: <br/> 
Using ISO Cyrillic set.‘ 
‘ &Roy;&ucy;&scy;&scy;&kcy;&icy;&jcy; <br/> 
Using XML Unicode entities: 
&#x0420;&#X0443;&#x044l;&#x044l;&#x043a;&#x043B;&#x0439; 
} </par) 
3 <head>Russian-English correspondence </head) 
<eng>QqWw Ee RrTtYyUuI i Oo P p</eng> 
<pyI::>'I:u1> 233 Ee PpT'.rI;I17:VyI/IHOoHn</pyc> 
<eng>Aa S s Dd?-' E (33 HhJj KkL l</eng> 
5<pyc>Aacc.u,-;<z>c1: 1" rXxJj K K.Tll'I</pyc> 
‘<eng>Zz Xx CCVVB b Nn Mrn</eng> 
§<pyc>3 31:11:: hrs BBB6H HM»-1</pyc> 
<eng>YAYO YU EE ya yo yuee ch CH sh SH ts TS shch SHCH </eng> 
<pyc>HEfO23sé1~o:'a1x'-ImII.Ix.1L[u; I.I.I</pyc> 
: <heacl>Greek-English corresponclence</head) 
.<eng>QqWw Ee RrTtYyIiOo Pp</eng> 
<ell>Qq 5'2 OJEB PQTYYU I LOo l'I 3:</el7\> 
'_=<eng>Aa S s Dd F fGg HhJj KkI..l</eng> 
.5 <e7\l>AoL2oA5¢=:p 1"yHnJ)'KxA7\</e7\7\> 
'<eng>Z zXxC cVvB bNnM rn</eng> 
<e;\;\>z:z §XxVvB [mu M ;.i</e11> 
<heacl>Mathcharactersec/head.) 
; <par>Ancl here is one ofMaxwell's equations: 
T &#x220’T;&#x00B7;&#x0042;&#x003cl;&#x0030;</par) 
‘ </rnycloc> 
Figure C5: A UTF-8 encoded XML file with Russian, Greek, and math 
The above excerpt defines the small and capital Cyrillic letters “r" and “s" in function of their Unicode number. That is how we get Cyrillic letters by symbolic name 
on line 10 in our source file. Of course, as on line 12, we can also directly enter 
the Unicode character references ourselves (compare the entity references “&Rcy; " 
on line 10 and “&#x0420; " on line 12; decide which is more readable and maintainable). On the other hand, on line 8 we have entered Russian directly by using 
Yudit, which saved it as UTF-8, that is, two bytes for alphabetic non-ASCII characters. These byte-codes print in a funny way in the T1 encoding used in this book, 

 
%%page page_505                                                  <<<---3
 
486 
Intemationalizafion issues 
but if you look at the T1 character table (see, for instance, \reftab{9-1} on page 261 of 
Goossens et al. (1994)) you easily see that they represent the byte sequence 
DOAOD183D181D181DOBADOB8DOB9 
which, as explained in Section C.2.3, is the UTF-8 equivalent of the Unicode sequence on line 12. 
Going back to Figure C.5, we see several line pairs, where the first line shows 
which is the letter or letter combination to be entered on an English “qwerty" 
keyboard, and the second shows the result in Yudit. For instance, following the line 
“Russian-English correspondence," we selected “Russian" in the “Input" menu, 
and following the line “Greek-English correspondence," we selected “Greek" in 
the “Input" menu. Depending on the selected language “Input" menu, Yudit will 
show the correct character on screen and save it in an encoding you can select 
in the “Encoding" menu. So we had to switch input language English to Russian 
to English again, to input the lines in the Russian part of the text. Similarly we 
switched from English to Greek and back in the Greek part. You see how we put 
the text in the three languages inside elements with tag names that correspond to 
the language’s name in its native alphabet. Thus, as in Section C3. 1, where we used 
Latin 1 XML element names, XML can define element names that make sense to 
the native writers of documents in every part of the world (see also Section 6.5.1), a 
big step forward from HTML where the tag set was fixed and tag names had a real 
meaning only in English. 
The final part of the XML file is the following: 
<head>Math characters</head) 
<par>And here is one of Maxwell’s equations: 
&#x2207;&#x00B7;&#x0042;&#x003d;&#x0030;</par) 
</mydoc> 
.pwN._ 
Line 3 shows Unicode character references to write a small mathematics formula. 
Our next task is to transform this file onto something we can browse or print. 
The style sheet utf8.xsl that follows transforms the earlier XML file utf8.xm1 
into HTML. 
<?xm1 vers1on=’1.0’?> 
<h1>Hand1ing UTF-8 fi1es</h1> 
<xs1:app1y-templates/> 
</body> 
</html> 
1 
2 <xs1:stylesheet 
3 xmlns:xs1="http://www.w3.org/TR/WD-xsl" 
4 xm1ns="http1//www.w3.org/TR/REC-htm140" 
5 result-ns=""> 
6 <xs1:temp1ate match="/“> 
7 <htm1 xm1ns="http://www.w3.org/Profiles/xhtmll-transitional.dtd"> 
8 <head> 
9 <title>UTF8 fi1es</title) 
10 <meta http-equiv=``Content-Type'' content="text/html;charset=UTF-8" /> 
11 </head> 
12 <body> 
u 
14 
H 
16 

 
%%page page_506                                                  <<<---3
 
C.3 Foreign languages in XML 
</xsl:template> 
<xsl:temp1ate match=``br''> 
<br /> 
</xsl:template> 
<xs1:temp1ate match=``par''> 
<p><xslzapply-templates/></p> 
</xs1:temp1ate> 
<xsl:template match=``head''> 
<h2><xslzapply-templates/></h2> 
</xs1:temp1ate> 
<!-- eliminate English keyboard input --> 
<xsl:template match=``eng''> 
</xs1:temp1ate> 
<!-- transmit Russian keyboard input --> 
<xsl:template match="&#x0440;&#x0443;&#x0441;"> 
<p>&#x25c6;&#xOOaO;<xs1:app1y-templates/></p> 
</xsl:template> 
<!-- transmit Greek keyboard input --> 
<xsl:tem late match=``AI gfiifiz I i''> . _n n 
<p>l1§UR&§x00a0;<xslzapplly-temglfites/></p> <Xs1'template matchan > 
</xS1..temP1ate> <p>0&#x00a0;<xsl:apply-templates/> </p> 
</xs1:stylesheet> 
\.a».awww\.awww-NNNNN~N~_.._._ 
w\lO\V|->‘é"V*-O~oa<)\lo«-In-$\o2vx2»_.o.5m\. 
The most interesting part of the style sheet is its beginning. Line 4 defines the 
output namespace as HTML (see Section 7.6.9). This allows us to output straight 
HTML tags. Lines 6-17 enclose the whole file inside a correct HTML structure 
where we indicate that we deal with XHTML-based HTML-compatible XML code 
(line 7). Line 10 is very important because it tells the browser that the character set 
is UTF-8. The meaning of the remaining lines should be straightforward. In par~ 
ticular, lines 28-29 eliminate the English input lines from the output. Because XSL 
style sheets are genuine XML files, we can use the full set of alphabetic and ideo~ 
graphic Unicode characters; in particular we can refer to the non-Latin element 
names of our XML document (again they show up strangely on lines 35 and 36). In 
fact, we can mix native as well as character reference codes in the XSL file to refer 
to Unicode characters. On line 31 we use character references to match the Cyrillic 
string “pyc" and to put the Unicode character “25c6" (black diamond) at the beginning of each paragraph. For the Greek part (lines 34-3 5) we use the Unicode 
characters natively (we show the relevant part of the style sheet to the right of the 
lines in question). In both cases we add a nonbreaking space (&#x00a0 ;) between 
the diamond or bullet and the subsequent text. 
As a final comment, it is worthwhile to pay attention to the extra blank between 
the element name and the closing “/>" for empty elements (see lines 9 and 18 in 
the style sheet). This allows current HTML browsers that do not yet understand 
XML syntax to interpret these lines correctly (in fact they will ignore the final “/"~, 
see also Section B.5.2.1). 
We can run these two files through an XSL parser, such as xt, and output the 
HTML file utf8 .htm1: 
xt utf8.xm1 utf8.xs1 utf8.htm1 
tidy -utf8 -m utf8.html 
487 

 
%%page page_507                                                  <<<---3
 
488 Internationalization issues 
Figure C.6: HTML rendering of UTF-8 file with Netscape 
Finally we run the output of X1: through Tidy (see Section B.5.2.3) to translate everything into UTF-8 because xt translates some characters (such as Greek) 
into entity references, and we want a clean UTF-8 file. The result as viewed with 
Netscape is shown in Figure C.6. 

 
%%page page_508                                                  <<<---3
 
Glossary 
AINIL Astronomical Instrument Markup Language [9 AIML]. 
Instrument description language that encompasses instrument characteristics, control 
commands, data stream descriptions (including image and housekeeping data), message 
formats, communication mechanisms, and pipeline algorithm descriptions. 
Amaya W3C test-bed browser/authoring system [=>AMAYA]. 
Versatile and extensible tool provided by the W3C to demonstrate and test new developments in Web protocols and data formats. Among other things Amaya lets you edit 
complex mathematical expressions within HTML pages through a WYSI W YG interface. 
AML Astronomical Markup Language [<->AML]. 
Metadata exchange markup language for astronomy that supports a set of astronomical 
objects, such as article, table, image, person. 
Attribute see Section 6.5.4.2. 
Lets you specify named characteristics about an element type in an SGML/XML DTD. 
BIOML BIOpolymer Markup Language [h>BIOML]. 
Allows the specification of experimental information about molecular entities composed 
of biopolymers, such as proteins and genes. 
BMP Basic Multilingual Plane (see Section C.2). 
Subset of the 31-bit ISO/IECIO646-1 UCS encoding (plane 0, group 0). By construction 
its contents are identical to Unicode. 

 
%%page page_509                                                  <<<---3
 
490 
Glossary 
BSML Bioinformatic Sequence Markup Language [h>BSML]. 
Standard for the encoding and display of DNA, RNA and protein sequence information. 
CALS Continuous Acquisition and Life-Cycle Support [9 CALS]. 
United States Department of Defense initiative to improve weapon system acquisition 
and life-cycle support processes through accelerated creation and application of digital 
product data and technical information. 
CDATA Character Data (see Section 6.5.4.7). 
Information in an XML document that should not be parsed at all. This allows the 
use of the markup characters &, <, and > within the text, even though no elements or 
entities may appear in the section. CDATA declarations may appear in XML attributes 
(Section 6.5.4.2), and CDATA sections may appear in documents. 
CML Chemical Markup Language [‘-> CML]. 
Allows you to manage chemical information with XML. CML and associated tools provide a platformand convention-independent specification for information interchange 
in the molecular sciences. It comes with a browser Gumbo) for visualizing the data. 
CSS Cascading Style Sheets [9 STYLECSS] (see Section 7.4). 
A simple declarative language that allows authors and users to apply stylistic information 
(for fonts, spacing, color, and so on) to structured documents written in HTML or XML. 
Two levels of CSS style sheets, CSSI [=>CSS1] and CSS2 [h>CSS2], are available; work 
is continuing on CSS3. 
DCD Document Content Description [<-> DCD]. 
A structural schema facility for specifying rules covering the structure and content of 
XML documents. DCD is an RDF vocabulary and is intended to define document constraints in an XML syntax, including providing basic datatypes. 
DDML Document Definition Markup Language [9 DDML]. 
A simple schema language for XML to encode the logical content of a DTD as an XML 
document. This allows schema information to be explored and used with widely available XML tools. 
DOM Document Object Model [h>DOMGEN]. 
A platformand language-neutral interface that allows programs and scripts to access 
and update the content, structure, and style of documents dynamically. The document 
can be further processed and the results of that processing can be incorporated back into 
the presented page. A first level has been accepted as a recommendation [HDOML1], 
while work continues on a level 2 specification [9 DOML2] that will add interfaces for 
a CSS object model, an event model, and queries. 
DSSSL Document Style Semantics and Specification Language (see Section 7.5). 
International Standard ISO 10179 IS O/IEC:l0l 79 (1996) was adopted at the beginning 
of 1995. It presents a framework to express the concepts and actions necessary for transforming a structurally marked up document into its final physical form. Although this 

%==========510==========<<<---2
 
%%page page_510                                                  <<<---3
 
Glossary 
491 
standard is primarily targeted at document handling, it can also define other layouw, 
such as those needed for use with databases. More on DSSSL by_Iames Clark is available 
at [9 DSSSLCLARK]. 
DTD Document Type Definition (see Section 6.5.4). 
A set of rules describing which elements types are allowed in an XML document and 
what their content model is. A DTD also specifies the possible attributes of each element 
type and declares the entities referenced in the document, as well as the notations that 
can be used. 
EAD Encoded Archival Description [<-> EAD]. 
A nonproprietary encoding standard for machine-readable finding aids such as inventories, registers, indexes, and other documents created by archives, libraries, museums, 
and manuscript repositories to support the use of their holdings. 
GIF Graphics Interchange Format [Murray and vanRyper (1996)]. 
A format originally developed by Compuserve to facilitate image transfers between various platforms by storing multiple bitmap images in the same file. GIF files used the 
patented LZVV compression method. If licencing issues could be a problem, it is wise 
to envisage using the PNG format instead. 
HL7 Kona Proposal [‘-> KONA]. 
A method in which electronic healthcare records (EHR) can be created, exchanged, and 
processed using SGML/XML. 
HTML Hypertext Markup language (see Section 1.1.3). 
The most important SGML-based markup language used on the Web. 
HTTP Hypertext Transport Protocol (see Section 1.1.1). 
Method that allows WWW servers to communicate with each other. 
ICE Information and Content Exchange [9 ICE]. 
The ICE protocol defines the roles and responsibilities of syndicators and subscribers 
and defines the format and method of content exchange. It provides support for management and control of syndication relationships, so that ICE will be useful in automating content exchange and reuse. 
Java Object-oriented programming language developed by Sun [HJAVADOC]. 
Java was designed from the ground up to allow for secure execution of code across a 
network, even when the source code is untrusted and possibly malicious. Java offers 
cross-platforrn portability not only in source form, but also in compiled binary form. 
To make this possible, Java is compiled to an intermediate byte-code which is interpreted on the fly by the Java interpreter. Thus only the interpreter and a few native 
code libraries need to be ported on different platforms; Java programs themselves run 
unchanged everywhere. Since Java is quite a small language, with strong typing and no 
unsafe constructs, it is easy to read and write and, above all, relatively easy to debug. See 
[HJAVAFAQ] for a lot of useful information. 

 
%%page page_511                                                  <<<---3
 
492 
Glossary 
Javascript Scripting language [HJAVASC]. 
Netscape’s cross-platform, object-based scripting language for client and server applications. Javascript is not Java! It lets you create applications that run over the Internet. Client applications run in a browser, while server applications run on the server 
side. JavaScript allows you to create dynamic HTML pages that process user input and 
maintain persistent data using special objects, files, and relational databases. Javascript 
was first introduced by Netscape in their browser Netscape Navigator 2.0, but today a 
“standard" version exists under the name of Ecmascript [9 ECMASC]. 
JPEG Joint Photographic Experts Group [<->JPEG]. 
An ISO standard describing image compression mechanisms for either full-color or 
gray-scale images of natural, real-world scenes. It works well on photographs, naturalistic artwork, and similar material. For a good description see Murray and vanRyper 
(1996). 
JSML Java Speech Markup Language [QJSML]. 
A language used by applications to annotate text input to Java Speech API speech synthesizers. The JSML elements provide a speech synthesizer with detailed information 
on how to say the text. JSML includes elements that describe the structure of a document, provide pronunciations of words and phrases, and place markers in the text. 
JSML also provides prosodic elements that control phrasing, emphasis, pitch, speaking rate, and other important characteristics. Appropriate markup of text improves the 
quality and naturalness of the synthesized voice. JSML uses the Unicode character set, 
soJSML can be used to mark up text in most languages of the world. 
LinuxML Linux Markup Language [9 LINUXML]. 
Project devoted to changing the UNIX de facto standard for interprocess communication and storage from line-based ASCII records to XML. The idea is that UNIX commands will produce XML output. This would allow downstream programs, like sort 
or xterm, to understand the semantic content of the incoming data and hence do more 
useful things with it. 
MathNIL Mathematical Markup Language [h>W2CMATH] (see Section 8.1). 
MathML deals with the (re-)use of mathematical and scientific content on the Web and 
with other applications such as computer algebra systems, print typesetting, and voice 
synthesis. 
NIIME Multipurpose Internet Mail Extensions [9 RFC2045]. 
MIME is a set of specifications that support the structuring of the message body in terms 
of body parts. Body parts can be of various types, such as text, image, audio, or complete 
encapsulated messages. It also provides for the encoding of messages in character sets 
other than 7-bit ASCII. 
MPEG Motion Picture Experts Group [h>MPEG]. 
An ISO Standard that specifies how to encode data streams for compressing audio and 
video information. See ISO/IEC:l 1172 (1993); Murray and vanRyper (1996). 

 
%%page page_512                                                  <<<---3
 
Glossary 
OASIS Organization for the Advancement of Structured Information Standards 
[<-> OASIS]. 
A nonprofit, international consortium, composed of users and suppliers of products 
and services, and dedicated solely to product-independent document and data interchange. Founded in 1993 as SGML Open, OASIS has expanded to embrace the complete spectrum of structured information processing standards including XML, SGML, 
and HTML. 
OFE Open Financial Exchange [<-> OFX]. 
OFE standardizes the electronic exchange of financial data between financial institutions, business, and consumers via the Internet. Originally set up at the begim1ing of 
1997, it now supports a wide range of financial activities including consumer and small 
business banking; consumer and small business bill payment; bill presentment and investments, including stocks, bonds, and mutual funds. Other financial services, including financial planning and insurance, will be added in the future and will be incorporated 
into the specification. Markup is likely to be generated and interpreted by programs, 
with no human intervention. 
OSD Open Software Format Description. 
Software distribution and update via the network, including “push" updates of software 
and hands-free installation. Markup is likely to be generated by a software packaging 
program and used to install the software without ever being read by a human. 
Parser see Section 6.6. 
A program that converts a serial stream of markup (an XML or SGML file, for example) into an output structure accessible by another higher-level program. XML parsers 
may perform validation or check to see if a markup is well-formed as they process it. 
Section 6.6 presents a few XML parsers. 
PDF Portable Document Format (see Chapter 2). 
A descendant of Adobe Systems’ PostScript language, optimized for improving navigation and delivery on the Internet. In particular PDF introduces page independence, 
adds hypertext and security features, allows font subsitution, and incorporates performant compression features to minimize file size. 
Perl [=>PERL] 
A high-level programming language written initially by Larry Wall. It inherits a lot 
of features from the C programming language but also includes good ideas of many 
other UNIX tools. Perl’s process, file, and text manipulation facilities make it particularly 
well suited for tasks involving quick prototyping, system utilities, software tools, system 
management tasks, database access, graphical programming, networking, and WW 
programming. 
PGML Precision Graphics Markup Language [<-> PGML]. 
An XML instance implementing the PDF/PostScript imaging model. 
493 

 
%%page page_513                                                  <<<---3
 
494 Glossary 
PNG Portable Network Graphics [9 PNG]. 
An extensible file format for the lossless, portable, well-compressed storage of raster 
images. It is a patent-free replacement for GIF and can also replace many common 
uses of TIFF. Indexed-color, grayscale, and truecolor images are supported, as is an 
optional alpha channel. PNG works well in online viewing applications (like WWW) 
since it is fully streamable with a progressive display option. It is robust, providing both 
full file integrity checking and simple detection of common transmission errors. PNG 
can store gamma and chromaticity data for improved color matching on heterogeneous 
platforms. 
PostScript Page description language [9ADOBEPS]. 
A computer language that describes the appearance of a page, including elements such 
as text, graphics, and scanned images, to a printer or other output device. It was introduced by Adobe [9ADOBE] in 1985 . It has become the language of choice in high 
quality printing on a wide range of output devices, including black-and-white and color 
printers, imagesetters, platesetters, screen displays, and direct digital presses. 
RDF Resource Description Format [9 RDF]. 
Describes the contents of Web resources in order to enable automatic processing. May 
be used to describe the contents of a Web site, to provide additional information for 
search engines or intelligent agents, to declare property rights, and so on. 
SAX Standard API for event-based XML parsing [9 SAX]. 
Section B.6 shows how SAX can be applied to handle XML documents. 
SDML Signed Document Markup Language [9 SDML]. 
Language designed to allow the creation of digitally signed electronic documents. 
There is provision for tagging individual text items making up a document and grouping them into parts that can have business meaning and can be signed individually or 
together. Document parts can be added and deleted without invalidating previous signatures; signing, cosigning, endorsing, coendorsing, and witnessing operations can be 
performed on whole documents or on parts. 
SGML Standard Generalized Markup Language (see Chapter 6). 
An International Standard (ISO 8879:1986) that describes a generalized markup scheme 
for representing the logical structure of documents in a system-independent and 
platform-independent manner. 
SMIL Synchronized Multimedia Integration Language [9 SMIL]. 
An XML application that allows you to integrate a set of independent multimedia objects into a synchronized multimedia presentation. The functionality of SMIL includes 
describing the temporal behavior of a presentation, describing its layout on a screen, 
and associating hyperlinks with media objects. 
SOX Schema for Object-oriented XML [9 SOX]. 
A schema facility for defining the structure, content, and semantics of XML documents 
to enable XML validation and higher levels of automated content checking. It provides 

 
%%page page_514                                                  <<<---3
 
Glossary 
495 
basic intrinsic datatypes, an extensible datatyping mechanism, content model and attribute interface inheritance, a powerful namespace mechanism, and embedded documentation. 
Speech1VIL SpeechML Markup Language [9 SPEECHML]. 
A language for building network-based conversational applications that interact with 
the user through spoken input and output. SpeechML could be used to enable conversational access from a car, a telephone, a PDA, a desktop PC, and so on to information 
sources and applications anywhere on the Internet. 
SVG Scalable Vector Graphics [9 SVGSPEC]. 
A proposed open vector graphics format that works across platforms, for various output 
resolutions, in several kinds of color spaces, and on a range of available bandwidths. The 
aim of SVG is to make Web documents smaller, faster, more interactive, and displayable 
on a wider range of device resolutions from small mobile devices to office computer 
monitors to high resolution printers. 
TEI Text Encoding Initiative [BTEIHOME] (see Section B.4.3). 
A scholarly international project to promote the interchange and preparation of electronic texts. Structural features of a text are marked up in the source using a standardized 
scheme to ease exchange and processing by computer. 
TeXN[L [HTEXML] see Section 8.2.4. 
Allows you to typeset XML documents with the TEX formatter. 
TIFF Tag Image File Format [=>TIFF6]. 
Originally a method for storing black-and-white scanned images, although later treatment of color was added, so that TIFF has become the standard file format for most 
paint, imaging, and desktop publishing tools. 
UCS Universal Character Set 
ISO/IECIO646-1 international standard for the encoding of all writing systems and character encodings in the world. It uses a 31-bit encoding with Unicode as a pure 16-bit 
subset corresponding to the Basic Multilingual Plane (see Section C.2). 
Unicode see Section C.2 [<-> UNICODE]. 
A standard for international character encoding. Unicode supports characters that are 
two bytes wide rather than the 8-bit codes currently supported by most systems. This 
makes it possible to encode 65,536 characters rather than only 25 6 with one-byte encodings. 
URI Universal Resource Identifier (see Section 1.1.2). 
Universal addressing scheme to locate information on the Internet. 
URL Universal Resource Locator (see Section 1.1.2). 
A special form of a URI, originally used when the Web was invented. 

 
%%page page_515                                                  <<<---3
 
496 
Glossary 
Valid see Section 6.4.2.1. 
A valid document is a well-formed document that adheres to its DTD. 
VML Vector Markup Language [%>V1\/IL]. 
An XML application that defines a format for the encoding of vector information together with additional markup to describe how that information may be displayed and 
edited. 
VRML Virtual Reality Modeling Language [<->WEB3D] 
File format standard for 3D multimedia and shared virtual worlds on the Internet. 
VRML adds interaction, structured graphics, and extra dimensions (z and time) to 
HTML’s graphical interface. Applications areas of VRML include business manufacturing, scientific, and educational graphics, with more recently 3D shared virtual worlds 
and communities. 
WAP Wireless Application Protocol. 
An XML markup language to exchange information over narrow-band devices. 
WebDAV World VV1de Web Distributed Authoring and Versioning [9 WEBDAV] . 
Extensions to the HTTP protocol to provide better support for distributed authoring. 
Well-Formed see Section 6.4.2.1. 
A well-formed document may or may not have a DTD. Well-formed documents must 
begin with an XML declaration and contain properly nested and marked-up elements. 
WIDL Web Interface Definition Language [~>WIDL]. 
An XML application that defines a metalanguage that implements a service-based architecture over the document-based VVVVW resources. VVIDL allows interactions with 
Web servers to be defined as functional interfaces that can be accessed by remote systems over standard Web protocols and provides the structure necessary for generating 
client code in languages such as Java, C/C++, COBOL, and Visual Basic. 
W3C World VV1de Web Consortium [=>W3C]. 
This international industry consortium was founded in October 1994 to lead the World 
Wide Web to its full potential by developing common protocols that promote its evolution and ensure its interoperability. W3C is jointly hosted by the Massachusetts Institute 
of Technology Laboratory for Computer Science in the United States; the Institut National de Recherche en Informatique et en Automatique [INRIA] in Europe; and the 
Keio University Shonan Fujisawa Campus in Japan. Services provided by the Consortium include a repository of information about the World Wide Web for developers and 
users, reference code implementations to embody and promote standards, and various 
prototype and sample applications to demonstrate use of new technology. 
XCatalog Proposal to adopt XML syntax for catalog [L>XCATALOG]. 
Proposed specification based the OASIS document [G> ENTIMAN] for managing entities 
by mapping XML public identifiers to XML system identifiers using URIs. 

 
%%page page_516                                                  <<<---3
 
Glossary 
IH-ITML Extensible HyperText Markup Language [h>HTMLINXML] (see Section B.5.2). 
A specification that reformulates HTML 4.0 as an XML 1.0 application. XHTML specifies three document profiles as XML namespaces, each with its own URI. The semantics of the elements and their attributes are defined in the W3C Recommendation for 
HTML 4.0, and they will provide the foundation for future extensibility of XHT ML. 
Xlink XML Linking Language [<->XLINKSPEC]. 
Specifies constructs that may be inserted into XML resources to describe links between 
objects. A link (in the Xlink context) is an explicit relationship between two or more 
data objects or portions of data objects. XLink uses XML syntax to create structures 
that can describe the simple unidirectional hyperlinks of today’s HTML as well as more 
sophisticated multiended and typed links. 
XMI XML Metadata Interchange Format [=>XML]. 
Enables easy interchange of metadata between modeling tools based on the Object 
Management Group’s Unified Modeling Language OMG (UML) and between tools 
and metadata repositories using the OMG Meta Object Facility (MOF). This architecture allows tools to share metadata programmatically using CORBA interfaces specified 
in the MOF and UML standards or by using XML-based stream (or file) containing 
MOF and UML compliant modeling specifications (see [9 OMGTECH] for specifications of the OMG standards mentioned). 
XNIL Extensible Markup Language [HXMLSPEC]. 
A subset of SGML whose goal is to enable generic SGML to be served, received, and 
processed on the Web in the way that is possible with HTML. XML is designed for 
ease of implementation and for interoperability with both SGML and HTML. XML in 
general is discussed in Chapter 6, while the translation of BTEX (with and without math) 
to XML is dealt with in Chapter 8. 
XNIL-Data [%>XMLDATA] . 
XML vocabulary for schemas to define and document object classes. 
XML/EDI Electronic Data Interchange [9XMLEDI]. 
A standard framework to exchange different types of data--for instance, an invoice, a 
healthcare claim, project status-so that the information in a transaction, exchanged via 
an Application Program Interface (API), Web automation, database portal, catalog, or 
workfiow document or message can be searched, decoded, manipulated, and displayed 
consistently and correctly by first implementing EDI dictionaries and extending the 
vocabulary via online repositories to include our business language, rules, and objects. 
Xpointer XML Pointer Language [=>XPTSPEC]. 
A language that supports addressing into the internal structures of XML documents. 
In particular, it provides for specific reference to elements, character strings, and other 
arts of XML documents whether or not the bear an e licit ID attribute. 
P a Y XP 
497 

 
%%page page_517                                                  <<<---3
 
498 Glossary 
XSL Extensible Stylesheet Language [h>XSLWD] (see Section 7.6). 
A language for expressing style sheets. It consists of a transformation language and a 
formatting objects Vocabulary. 
XUL Extensible User Interface Language [h>XUL]. 
Describes the contents of windows and dialogs. XUL has constructs for typical dialog 
controls and for widgets such as toolbars, trees, progress bars, and menus. 

 
%%page page_518                                                  <<<---3
 
URL catalog 
ACROTEX; PDF-based math tutorials making extensive use of Acrobat forms. 
http : //www . math . uakron. edu/~dpstory/acrotex . html 
ADOBE: Adobe’s home page. 
http: //www. adobe. com/ 
ADOBEPS: PostScript information at Adobe. 
http : //www. adobe . com/prodindex/postscript/main. html 
AELFRED: David Megginson’s fElfred XML parser. 
http: //www . microstar . com/ae1fred.htm1 
AIML: Astronomical Instrument Markup Language. 
http : //pioneer . gsfc . nasa . gov/pub1ic/aim1/ 
AMAYA: Experimental browser/ authoring system. 
http : //www. w3 . org/Amaya/ 
AML: Astronomical Markup Language. 
http : //www . infm . ulst . ac . uk/'/.7Edami en/these/ 
ASTER: T. V. Raman’s system for spoken mathematics reading TEX source. 
http : //www . cs . cornell . edu/Info/People/raman/aster/demo . html 
AXML: The XML standard, annotated by Tim Bray. 
http://www.xm1. com/axml/axml .htm1 
BALIS E: SGML programming environment for structured documents. 
http://www.ba1ise. com 
BIOML: XML language for annotating biopolymer sequence information. 
http://wwmproteometrics. com/BIOML/ 

 
%%page page_519                                                  <<<---3
 
500 
URL catalog 
BOSAKXML: XML, java, and the Future of the W211 by Jon Bosak. 
http 2 //met alab . unc . edu/pub/ suninfo/standards/xml/why/xmlapps .html 
BSML: Bioinformatic Sequence Markup Language for graphic genomic displays. 
http : //visualgenomics . com/sbir/rf c . htm 
CALS: Continuous Acquisition and Life-Cycle Support. 
http : //navysgml . dt . navy .mil/cals .html 
CERN: The European Laboratory for Particle Physics. 
http : //www . cern. ch/Publi c/ 
CML: Chemical Markup Language resources. 
http 2 //xmlcml. org 
CONTEXT 2 The CONTEXt macro package by Hans Hagen. 
http : //www . ntg. nl/context/ 
CSS1: Cascading Style Sheets, version 1, W3C recommendation (December 1996). 
http : //www . w3 . org/TR/REC-CSS1 
CSS2: Cascading Style Sheets, version 2, W3C recommendation (May 1998). 
http : //www . w3 . org/TR./R.ECCSS2/Overview . html 
CVSREPOS: ETEXZHTML CVS repository. 
http : //saftsack . fs .uni-bayreuth . de/~latex2ht/ 
DARPA: Defense Advanced Research Projects Agency. 
http://www. arpa.mi1/ 
DAVENPORT: Davenport Group, developers of the Docbook DTD. 
http : //www . oasis-open. org/docbook/ 
DBDSSSL: The Modular DocBook Stylesheets. 
http : //nwalsh. com/docbook/dsssl/index .html 
DBVIEW: DocBook 3.0: User Element Index. 
http : //www . ora. com/homepages/dtdparse/docbook/3 . 0/elements . htm 
DBXML: The Docbook DTD translated XML. 
http : //nwalsh. com/docbook/xml/index . html 
DCD: Document Content Description for XML. 
http 2 //www . w3 . org/TR./NOTE-dcd 
DDML: Document Definition Markup Language Specification, Version 1.0. 
http 2 //www . w3 .org/TR./NOTEddml 
DOCBOOK: DocBook Homepage at OASIS. 
http : //www. oasis-open . org/docbook/ 
DOMGEN: OASIS page on the W3C Document Object Model (DOM). 
http : //www. oasisopen. org/cover/dom . html 
DOML1: Document Object Model (DOM) Level 1 Specification (Version 1), 
W3C Recommendation 1 October, 1998. 
http : //www . W3 . org/TR./R.ECDOMLevel1/ 

%==========520==========<<<---2
 
%%page page_520                                                  <<<---3
 
URL catalog 501 
DOML2: Document Object Model (DOM) Level 2 Specification (Working Draft). 
http : //www . w3 . org/TR./WD-DOM-Level-2/ 
DRAKOSW W W: From Text to Hypertext: A POSt-HOC Rationalisation of E‘TEX2HTML by 
Nikos Drakos. 
http 2 //www. cbl. leeds . ac . uk/nikos/doc/www94/www94 .html 
DSSSLCLARK: James Clark’s DSSSL information. 
http 2 //www . j Clark . com/dsssl/ 
DSSSLLIST: DSSSL Mailing List. 
http://www.mu1berrytech.com/dsssl/dssslist/index.html 
DSSSLMML: DSSSL style sheet for MathML. 
http : //www . nag. co .uk/proj ects/open.math/mm1-fi1es/ 
DSSSLONL: DSSSL Online. 
http: 
//meta1ab.unc . edu/pub/suninfo/staudards/dsssl/dssslo/do9608 16 . htm 
DSSSLPDF: DSSSL specification online in PDF. 
ftp : //ftp . ornl . gov/pub/sgml/WG8/DSSSL/dsssl96b .pdf 
DSSSLSUM: Harvey Bingham’s DSSSL syntax summary. 
http : //www . tiac . net/users/bingham/dssslsyn/index .htm 
DSSSLTUTA: Paul Prescod’s Introduction to DSSSL. 
http : //itrc .uwaterloo . ca : 80/ ~papresco/dsssl/tutorial .html 
DSSSLI U I B: Daniel M. German’s An Introduction to DSSSL. 
http 2 //csg . uwaterloo . ca/~dmg/dsss1/tutorial/tutorial . html 
DTDPARSE: Norman Walsh’s SGML DTD parser. 
http : //www . ora . com/homepages/dtdparse/ 
DVIOUT: Windows 9X/NT DVI viewer that supports HyperTEX \specials. 
http 2 //akagi .ms . u-tokyo . ac . jp/dvioute_help . html 
DVIPS: Tom Rokicki’s DVI to PostScript driver. 
http 2 //www.radicaleye . com/dvips .html 
EAD: Encoded Archival Description initiative. 
http 2 //www . loc . gov/ead/ead . html 
EC: European Union’s Web server. 
http 2 //europa . eu. Int/ 
ECMASC: Standard ECMA-262, ECZVIAScript Language Specification. 
http 2 //www . ecma . ch/sta.ud/ecma262 .htm 
ELEMATTR: SGML/XML: Using Elements and Attributes. 
http 2 //www . oasis-open . org/cover/elementsAndAttrs .html 
ENTIMAN: Entity Management. 
http 2 //www . oas is-open . org/html/a401 . htm 

 
%%page page_521                                                  <<<---3
 
502 
URL catalog 
EPRINT: Los Alamos e-Print archive of scientific papers preprints. 
http 2 //xxx . la.ul . gov/ 
EPSIG: Electronic Publishing Special Interest Group. 
http 2 //www . oasisopen . org/cover/epsig . html 
EPSTOPDF: Perl utility that makes page size equal to BoundingBox for EPS files. 
http 2//www . tug . org/applications/pdftex/epstopdf 
ESIS: ESIS-ISO 8879 Element Structure Information Set. 
http 2 //www . oasis-open . org/cover/WG8-n931a . html 
FOP: James Tauber’s Formatting Object to PDF Translator. 
http 2 //www . jtauber . com/fop/ 
FOSI: MIL-M-2 8001C standard for Formatting Output Specification Instance. 
http 2 //wwwcals . itsi . disa .mil/core/standards/28001C . PDF 
FPISERVER: Peter Flyrm’s server to resolve Formal Public Identifiers. 
http 2 //www . ucc . ie/cgi-bin/PUBLIC 
FPISYNTAX2 Formal Public Identifiers syntax. 
http 2 //www . oasis-open . org/cover/tauber-fpi .html 
FRM: Document authoring and publishing system (includes SGML support). 
http 2 //www . adobe . com/prodi11dex/fra.memaker/ 
GROVES: DSSSL Graph Representation of Property Values (groves). 
http 2 //www . oasis-open . org/cover/topics . htmlitgroves 
GSHOME: Ghostscript home page. 
http 2 //www . cs . wisc . edu/~ghost/ 
HOOD: HTML Document Type Definitions. 
http 2 //www . utoronto . ca/webdocs/HTMLdocs/HTML_Spec/html . html 
HTMLZSPEC2 HTML 2.0 Specification. 
http 2 //www .w3 .org/MarkUp/htmlspec/ 
HTML4: HTML 4.0 Specification. 
http 2 //www . w3 . org/TR./REC-html40/ 
HTMLENTS: Character entity references in HTML 4.0. 
http 2 //www . w3 . org/TR./REC-htm140/sgml/entities . html 
HTMLINXML: XHTML 1.0: Extensible HyperText Markup Language. 
http 2 //www . w3 . org/TR./WD-htmlin-xml/ 
HTMLTIDY2 Dave Raggett’s Tidy cleans up HTML pages and converts them to XHTML. 
http 2 //www . w3 . org/Peop1e/R.aggett/tidy/ 
HTTPNG: W3C and IETF HTTP-NG activity. 
http 2 //www.w3 . org/Protocols/HTTP-NG/Activity .html 
HTTPRFC: HTTP 1.1 specification. 
http 2 //www . w3 . org/Protocols/rfc2068/1‘fc2068 

 
%%page page_522                                                  <<<---3
 
URL catalog 503 
HYPERLTX: Otfried Cheong’s Hyperlatex BTEX to HTML translator. 
http : //www . cs .ust . hk/~otfried/Hyperlatex/ 
HYPERTEX: HyperTEX FAQ. 
http : //xxx . 1a.ul . gov/hypertex/ 
HYTIME: Hytime Standard. 
http : //www . ornl . gov/sgml/wg8/docs/n1920/html/111920 .htm1 
ICE: The Information and Content Exchange (ICE) Protocol. 
http 2 //www . w3 . org/TR./NOTE-ice 
IETF: The Internet Engineering Task Force. 
http 2 //www . ietf . org/ 
IMAGEMAGICK: John Christy’s image processing utilities. 
http : //www . wizards . dupont . com/ cristy/ 
INRIA: Institut national de recherche en informatique et en automatique. 
http://www. inria.fr/ 
ISO8879TC2: Web SGML adaptations. 
http : //www . ornl . gov/sgml/WG8/document/1955 . htm 
ITRANS: A package for printing text in Indian Language Scripts. 
http : //www . aczone . com/itra.us/ 
JADE: James Clark’s DSSSL implementation. 
http://www .jc1ark . com/j ade/ 
JADETEX: JadeTEX macro package for Jade TEX backend. 
ftp : //cta.u . tug . org/tex-archive/macros/j adetex/ 
JADETEXB: Te)G7OTBuilder: a Generic TEX backend for Jade. 
http : //www . j Clark . com/j ade/TeX . htm 
JAVADOC: Sun’s Java documentation page. 
http : //j ava . sun . com/docs/index . html 
JAVAFAQ: Cafe au LaitJava FAQs, News, and Resources. 
http : //metalab . unc . edu/j avafaq/ 
JAVASC: Netscape’s JavaScript Guide. 
http 2 //developer .netscape . com/docs/manuals/communicator/j sg'uide4/ 
JPEG: JPEG Frequently Asked Questions. 
http 2 //www . faqs . org/faqs/jpeg-faq/ 
JSML: Java Speech Markup Language Specification. 
http : //j ava . sun . com/products/java-media/speech/forDevelopers/JSML/ 
index . html 
JUMBO: JAVA-XML, The JUMBO browser. 
http 2 //ala . vsms . nottinghain . ac . uk/vsms/java/j umbo/ 
KEIO: Keio University. 
http://www.keio .ac .jp/ 

 
%%page page_523                                                  <<<---3
 
504 
URL catalog 
KOALAXSL: Koala XSL engine for Java. 
http 2 //www . inria . fr/koala/XML/xslProcessor/ 
KONA: HL7 Kona Proposal (health care). 
http 2 //www . mcis . duke . edu 2 80/standards/HL7/sigs/sgml/WhitePapers/KONA/ 
LZHCTAN2 ETEXZHTML sources on CTAN. 
ftp 2 //ctan . tug . org/tex-archive//support/lat ex2html/sources/ 
LZHDOC: BTEXZHTML online documentation. 
http 2 //www-dsed . 11n1 . gov/files/progra.ms/unix/1atex2htm1/ma.uua1/ 
LZHLIST: ETEXZHTML mailing list. 
mailto : 1atex2htm1©tug . org 
L2HM.ML2 Generating MathML markup using ETEXZHTML, WebEQ, and WebTEX. 
http ://www .geom.umn. edu/~ross/webtex/webtex/ 
LZHSA: BTEXZHTML source repository. 
http 2 //www-dsed . 11n1 . gov/f i1es/progra.ms/unix/ lat ex2htm1/sources/ 
LZHSC2 ETEXZHTML source repository. 
http 2 //saftsack . fs . uni-bayreuth . de/~1atex2ht/ 
LECHE: XML News and Resources. 
http 2 //metalab . unc . edu/xm1/ 
LINUXML2 UNIX in XML project. 
http 2 //www . ozemail . com . au/~birchb/1inuxm1/1inuxm1 . htm 
LTXML: LTXML XML Tools. 
http : //www. 1tg.ed. ac .uk/so;ftware/xm1/ 
MALAYALAM: Malayalam-TEX. 
ftp 2 //cta.u . tug. org/tex-archive/1a.uguages/ma1aya1a.m/ 
MALEREX2 SGML Exceptions and XML. 
http 2 //www . arbortext . com/Think_Tank/XML__Resources/SGML_Exceptions _ 
a.ud_XML/ sgm1_except ions_a.ud_xm1 . html 
MAPLE: Waterloo Maple home page. 
http : //www .map1esoft . com/ 
MATHEMATICA2 Wolfram Research Mathematica home page. 
http 2 //www . Wolfram . com/ 
MATHSYMP: National Symposium in Mathematics. 
http 2 //wwwmath . mpce .mq . edu . au/texdev/MathSymp/ 
MATHTYPE2 WYSIWYG equation editor outputting TEX or MathML. 
http 2 //www .mathtype . com 
MICROPRESS: Micropress home page. 
http 2 //www.micropressinc. com 
MIT: Massachusetts Institute of Technology. 
http 2 //web .mit . edu/ 

 
%%page page_524                                                  <<<---3
 
URL catalog 505 
MMLGUID: A comprehensive guide to MathML maintained by Pankaj Kamthan. 
http : //indy . cs . concordia . ca/math.ml/ 
MMLRES: MathML Resource List. 
http : //www . webeq. com/webeq/mathml/resources . html 
MMLSPEC: MathML specification. 
http : //www . W3 . org/TR/WD-math/ 
MPEG2 MPEG Frequently Asked Questions. 
http : //www .faqs . org/faqs/jpeg-faq/ 
NDVI: Netscape plugin for viewing DVI files (supports HyperTEX \specials). 
http : //norma . nikhef . nl/~t 16/ndvi_doc . html 
NETPBM: netpbm utilities. 
ftp : //ftp . x . org/contrib/utilities/11etpbm1mar1994 . p1 . tar . gz 
NIKNAK: Commercial PostScript to PDF convertor. 
http : //www. 5-d. com/niknak . htm 
NISTHMF: Digital Library of Mathematical Functions. 
http 2 //www . nist . gov/DigitalMathLib/ 
OASIS: Organization for the Advancement of Structured Information Standards. 
http : //www . oasis-open . org/ 
OFX: Open Financial Exchange. 
http://www. ofx.net 
OMEGA: TEX 16-bit implementation based on Unicode. 
http : //www . gutenberg. eu . org/omega/ 
OMGTECH: Object Management Group technical documents. 
http 2 
//www . omg . org/t e chproces s/meet ings/ s chedule/Te chno1ogy_Adopt ions . html 
OMNIMARK: SGML environment for managing and delivering personalized content on 
the Web. 
http : //www. omnimark . com 
PDFMARKD: Pdfinark documentation. 
http : //partners . adobe . com/supportservice/devrelat ions/technotes . html 
PDFMARKP: Pdfinark primer. 
http : //www . ifconne ction . de/~tm/ 
PDFSPEC: PDF specification. 
http ://www . adobe . com/ supportservice/devrelat ions / PDFS/ TN/ PDFSPEC . PDF 
PDFTEXEX: pdfIEX examples. 
http : //www .tug . org/app1ications/pdftex/ 
PDFTEXS: pdfTEX source. 
ftp : //ftp . cstug . cz/pub/tex/local/cstug/thanh/ 

 
%%page page_525                                                  <<<---3
 
506 
URL catalog 
PDFZONE: The PDF Zone. 
http : //www . pdfzone . com/ 
PERL: The Perl source home page. 
http : //www . perl . com/pace/pub/ 
PERLSGML: Perl programs and libraries for processing SGMLDTDs. 
http : //www . oac . uci . edu/ indiv/ehood/perlSGML . html 
PGML: Precision Graphics Markup Language. 
http://www.w3.org/TR/1998/NOTE-PGML 
PNG: The Portable Network Graphics home page. 
http : //www. cdrom . com/pub/png/ 
PSGML: Emacs Major Mode for editing SGML coded documents. 
http://www. lysator . liu. se/projects/about_psgml .html 
PSGMLXML: Patches to add XML to PSGML. 
http : //www . megginson . com/Software/psgmlxml19980218 . zip 
QWERTZ: The qwertz HTML to ETEX Converter. 
http : //nathan . gmd . de/proj ects/zeno/qwertz/qwertz .html 
RAGHIST: A history of HTML (Chapter 2 of Raggett et al. (1998)). 
http : //www . w3 . org/Peop1e/R.aggett/book4/ch02 .html 
RAGHTML: Raggett’s 10 minute Guide to HTML. 
http://www .w3 . org/MarkUp/Guide/ 
RDF: Resource Description Framework. 
http : //www.w3 . org/R.DF/ 
RFC1630: Universal Resource Identifiers in W W W. 
http : //info . internet . isi . edu : 80/in-notes/rf c/files/rfc1630 .txt 
RFC1738: Uniform Resource Locators (URL) Specification. 
http : //info . internet . isi . edu : 80/in-notes/rfc/files/rfc1738 . txt 
RFC1866: Hypertext Markup Language - 2.0. 
http : //info . internet . isi . edu:80/in-notes/rfc/files/rfc1866 .txt 
RFC2045 : Multipurpose Internet Mail Extensions (MIME) Part One: Format of Internet 
Message Bodies. 
http : //info . internet . isi .edu:80/in-notes/rf c/files/rf c2045 .txt 
RFC2141: Uniform Resource Name (URN) Syntax Specification. 
http : //info . internet . isi .edu : 80/in-notes/rf c/files/rf c2141 . txt 
RFC2396: Uniform Resource Identifiers (URI): Generic Syntax. 
http : //info . internet . isi . edu : 80/in-notes/rf c/files/rfc2396 . txt 
SAMANALA: Samanala Transliteration. 
http : //www-texdev . mp ce .mq. edu . au/12h/indic/Indi ca/sa.ma.nala/ 
SAX: Simple API for XML. 
http : //www . meggins on . com/SAX/ index . html 

 
%%page page_526                                                  <<<---3
 
URL catalog 507 
SAXJAVA: SAX Java distribution. 
http : //www . megginson. com/SAX/j avadoc/packages .html 
SDML: Signed Document Markup Language: W3C note. 
http : //www . w3 . org/TR./NOTESDML/ 
SC%4L2XIJL: SC%4LtO)flWL. 
http : //www . xml . com/xml/pub/98/07/dtd/index . html 
SGMLC: Programming language for processing SGML documents on MS Windows. 
http : //www . dircon. co .uk/sgml/ 
SGMLNEW: SGML and XML News. 
http : //www . oasisopen . org/cover/ sgmlnew . html 
SGMLSPM: SGMLSpm source archive. 
http : //www .megginson . com/Software/SGMLSpm1 . 03ii . tar . gz 
SGMLTOOLS: Home page of the SGMLtools project. 
http : //www . sgmltools . org/ 
SINDOC: Sinh-HTMLdocs. 
http : //www-texdev . mpce . mq . edu . au/l2h/ indic/Sinhala/lreport/ 
SINTEX: Sinhala-TEX. 
ftp : //cta.n . tug . org/texarchive/la.nguage/sinhala/ 
SMIL: Synchronized Multimedia Integration Language 1.0 Specification. 
http : //www .w3 .org/TR./R.ECsmil/ 
SOTU98: President Clinton’s 1998 State of the Union speech. 
http://www.whitehouse.gov/WI-I/SOTU98/address.htm1 
SOTU99: President Clinton’s 1999 State of the Union speech. 
http ://www .whitehouse . gov/WI-I/New/html/19990119-2656 .htm1 
SOX: Schema for Object-oriented XML. 
http://www.w3 . org/'I'R/NOTE-SOX/ 
SP: SP SGML parser. 
http://www.jclark. com/sp/ 
SPDOC: SP parser documentation. 
http://www.jc1ark.com/Sp/nsgmls.htm 
SPEECHML: SpeeclLML markup language. 
http : //www . alphaworks . ibm . com/formula/spee chml 
STYLECSS: W3C’s CSS stylesheet page. 
http : //Wm: . w3 . org/Style/CSS/ 
SVGSPEC: Scalable Vector Graphics Specification (Working Draft). 
http : //www . w3 . org/TR/WD-SVG/ 
TDTD: tdtd DTD editing Emacs macros. 
ftp : //ftp .mulberrytech. com/pub/tdtd/ 

 
%%page page_527                                                  <<<---3
 
508 
URL catalog 
TEIGUIDE: TEI Lite: An Introduction to Text Encoding for Interchange. 
http : //www. uic . edu/orgs/tei/intros/teiu5 .htm1 
TEIHOME: TEI Text Encoding Initiative home page. 
http : //www.uic . edu/orgs/tei/ 
TEILITE: TEI lite DTD. 
http : //www-tei . uic . edu/orgs/tei/p3/dtd/teilite . dtd 
TEIXML: XML version of TEI DTD. 
http : //www . loria . fr/ ~bonhomme/xml .html 
TEXZHTML: Commercial version of tth (adds additional features). 
http -. //www . tex2html . com 
TEX4HT: TEX4ht translator. 
http : //www . cis . ohio-state . edu/~g1.u:ari/TeX4ht/mn.html 
TEXLIVE: Ready-to~r1m CD-ROM distribution of TEX-related software (UNIX and 
Windows 9X/NT). 
http : //www . tug . org/tex1ive.html 
TEXNIL: System to typeset XML documents with TEX. 
http : //www . alphaworks . ibm . com/f ormu1a/texml/ 
TEXPIDER: MicroPress’ version of TEX that writes HTML directly. 
http : //www .mi cropress-inc . com/webb/wbstart .htm 
TIFF6: Specification of TIFF, version 6. 
http : //www . adobe . com/support service/devrelations/PDFS/TN/TIFF6 . pdf 
TTH: TEX to HTML translator. 
http : //hutchinson . belmont .ma . us/tth/ 
TUGINDIA: TUGIndia Journal. 
http : //ftp . gwdg . de/pub/da.nte/usergrps/tugindia/tugindial 1 . pdf 
TXPL: IBM techexplorer Hypermedia Browser. 
http : //www . software . ibm . com/enetwork/techexplorer/ 
UNICODE: Unicode Consortium home page. 
http://www.unicode . org 
URNIETF: IETF URN Working Group. 
http : //www . ietf . org/html . charters/urncharter . html 
VISXML: Visual XML. 
http : //www . pierlou . com/visxm1/ 
VML: Vector Markup Language. 
http : //www . w3 . org/TR./NOTE-VML 
WZCMATH: W3C’s Math home page. 
http : //www . W3 . org/Math/ 
W3C: World Wide Web Consortium home page. 
http://www.w3.org/ 

 
%%page page_528                                                  <<<---3
 
URL catalog 
509 
W3CFUTURE: W3C plans for markup after HTML. 
http://www.w3.org/Ma1'kUp/Activity . html 
WBCGR2 W3C Graphics. 
http: //www .w3. org/Graphics/Activity 
W3CSTYLE: W3C§ Style Page. 
http 2 //www . w3. org/Style 
WAI: Web Accessibility Initiative. 
http 2 //w3. org/WAI/ 
VVEB3D: The WEBBD (formerly VRML) home page. 
http 2 / / www . web3d . org/ home . html 
VVEBDAV: Web Distributed Authoring and Versioning (IETF WEBDAV Working Group). 
http 2 //www . ics .uci . edu/pub/ietf/webdav/ 
VVEBEQ: WebEQ Equation Editor. 
http 2 //www .webeq. com 
WEBI-HST: Little History of the World Wide Web. 
http : //www.w3. org/History .htm1 
WIDL: Web Interface Definition Language. 
http 2 //www . w3 . org/TR./NOTE~wid1 
XCATALOG: john Cowan’s XCatalog proposal. 
http 2 //www. ccil. org/~cowa.n/XML/XCata1og. html 
XDVI: Paul Vojta’s X Windows TEX previewer. 
http 2 //math .berke1ey . edu/ ~voj ta/xdvi .htm1 
XLINKSPEC: XML Linldng Language Specification. 
http 2 //www.w3 . org/TR/wD-xlink 
XML: XML Metadata Interchange. 
http : //www . oasis-open. org/cover/xmi .html 
XML4J2 XML for java. 
ht tp : / /www . alphaworks . ibm . c om/ f ormu1a/xm1/ 
XMLDATA: XML~Data, an XML vocabulary for schemas. 
http 2 //Www . w3 . org/TR/1998/NOTE-XMLdata/ 
XMLDEV2 XML developers’ list. 
http 2 //www. lists . ic . ac .uk/hypermai1/xm1-dev/ 
XMLEDI: XML/EDI: an E-business framework using XML. 
http : //www. geocities . com/wallstreet/Floor/5815/ 
XMLERRATA: XML 1.0 Specification Errata. 
http 1 //www . w3 . org/XML/xm119980210errata 
XMLFAQ: Peter Flynn’s XML FAQ. 
http://www.ucc .ie/xml/ 

 
%%page page_529                                                  <<<---3
 
510 
URL catalog 
XMLINTRO: XML introduction. 
http : //www . oasis-open . org/cover/xmlIntro . html 
XMLNS: Namespaces in XML. 
http : //www . w3. org/TR./WD-xml-na.mes/ 
XMLPAGE: XML page. 
http : //www . oasisopen. org/cover/xml.html 
XMLPARS: List of free XML software. 
http : //www. stud. ifi .uio .no/~larsga/linker/XMLtoo1s .htm1 
XMLRES: XML resources page. 
http : // capita . wustl . edu/XMLR.es/ 
XMLSPEC: The XML specification (Version 1). 
http : //Www . w3 . org/TR./REC-xml 
XMLSTYLE: XML stylesheet. 
http : //www . w3 . org/TR./xml-stylesheet 
XPDF: Xpdf, an Independent PDF viewer. 
http : //www. aimnet . com/~derek.n/xpdf/ 
XPPARS: James Clark’s xp XML parser. 
http://www.jc1ark.com/xml/xp/index . html 
XPTSPEC: XML Pointer Language Specification. 
http: //V-1ww.w3. org/TR./wD-xptr/ 
XSL97: XSL (original 1997 submission). 
http://V-1ww.w3.org/TR./NOTE-XSL.htm1 
XSLCSS: Using XSL and CSS together. 
http : //www . w3. org/TR/NOTE-XSL-a.ndCSS/ 
XSLMAIL: XSL Mailing List. 
http: //www .mu1berrytech. com/xsl/xsl-list 
XSLREQ: XSL Requirements Summary. 
http : //www . w3 . org/'I'R/WD-XSLReq/ 
XSLWD: XSL Working Draft (December 1998 version). 
http : //www . W3 . org/TR./1998/WD-xs1-19981216 
XTPROC: James Clark’s xt XSL transformation engine. 
http://www.jc1ark.com/xml/xt.html 
XUL: Extensible User Interface Language. 
http: //www . oasisopen. org/ cover/xul . html 
YANDY: Y&Y Inc. 
http 2 //www. ya.ndy. com/ 
YUDIT-. Yudit Unicode editor. 
http : //czyborra. com/yudit/ 

%==========530==========<<<---2
 
%%page page_530                                                  <<<---3
 
Bibliography 
Abramowitz, M. and Stegim, I. A. 1972. Handbook of Mathematical Functionx. 
New York: Dover Publications. 
Bienz, T., Cohn, R., and Meehan, J. R. 1996. Portable Document Format Reference 
Manual Version 1.2. San Jose, Calif; Adobe Systems Incorporated. Available 
online at [L>PDFSPEC]. 
Boumphrey, F. 1998. Profexxional Style Sheetsfior HTML and XML. Chicago: 
Wrox Press, Inc. 
Bradley, N. 1998. The XML Companion. Reading, Mass.: Addison Wesley 
Longman. 
Carr, L., Rahtz, S., and Hall, W. 1991. Experiments with TEX and hyperactivity. 
TUGboat, 12 (1), 13-20. 
Drakos, N. and Moore, R. 1998. The L“7EX2HTML Tranxlator, Uxer’: Guide and 
Manual. Accompanies the software, 1998. Online version available at 
[9 LZHDOC]. 
Flynn, P. 1998. Understanding SGML and XML 7501;. Norwell, Mass.: Kluwer 
Academic Publishers. 
Foster, K. R. 1999. Math on the Internet. IEEE Spectrum, 36 (4), 36-40. 
Goldfarb, C. F. and Prescod, P 1998. The XML Handbook. Englewood Cliffs, 
N.].: Prentice Hall. 
Goossens, M., Mittelbach, F., and Samarin, A. 1994. The L/HEX Companion. 
Reading, Mass.: Addison-Wesley. 

 
%%page page_531                                                  <<<---3
 
512 
Bibliography 
Goossens, M., Rahtz, S., and Mittelbach, F. 1997. The L“7EX Graphiv: 
Companion: Illuxtrating Documents with TEX and PoxtScript. Tools and Techniques 
for Computer Typesetting. Reading, Mass.: Addison-Wesley. 
Harold, E. 1998. XIVIL Extensible Markup Language. Foster City, Calif.: IDG 
Books Worldwide, Inc. 
ISO:15 924 1999. Code: for the Representation of Name: of Scriptx. International 
Organization for Standardization, Geneva, Switzerland. International Standard 
ISO 15924:1999. 
ISO:3166 1997. Codexfor the Reprexentation of Names‘ of Countriex and Their 
Suhdivixion:-Part 1: Country Codex. International Organization for 
Standardization, Geneva, Switzerland. International Standard ISO 3166-1:1997. 
ISO:639 1988. Code fior the Representation of Namex of Languagex. International 
Organization for Standardization, Geneva, Switzerland. International Standard 
ISO 639:1988. 
ISO:639-2 1998. Code for the Representation of Namex of Language:-Part 2: 
Alpha-3 Code. International Organization for Standardization, Geneva, 
Switzerland. International Standard ISO 639:1998. 
ISO:8879 1986. Infiormation Proce:xing-'Iext and Office Syxtemx-Standard 
Generalized Markup Language (SGML). First edition, 1986-10-15. International 
Organization for Standardization, Geneva, Switzerland. International Standard 
ISO 8879:1986. 
ISO/IEC:10179 1996. Information Tevhnology-Procexxing Languagex-Doeument 
Style Semantic: and Specifivation Language (DSSSL). First edition, 1996. 
International Organization for Standardization, Geneva, Switzerland. 
International Standard ISO/IEC 10179:1996. A PDF version is available online 
for personal use, see [=>DSSSLPDF]. 
ISO/IEC:10646-1:1993 1993. Infiormation Technology-Univerxal Multiple-Octet 
Coded Character Set (UCS)-Part 1: Arvhitecture and Basie Multilingual Plane, 
(with amendments). International Organization for Standardization, Geneva, 
Switzerland. International Standard ISO/IEC: 10646-1 : 1993. 
ISO/IEC:11 172 1993. Information Technology-Coding of Moving Picture; and 
Associated Audio for Digital Storage Media at up to ahout 1, 5 Mhit/:--Parts 1 to 4. 
International Organization for Standardization, Geneva, Switzerland. 
International Standard ISO/IEC 9070: 1 1 172. 
ISO/IEC:147 72-1 1998. Information Tevhnologjy-Computer Graphic: and Image 
Procexxing-The Virtual Reality Modeling Language-Part 1: Functional Specifivation 
and UTF-8 Encoding. International Organization for Standardization, Geneva, 
Switzerland. International Standard ISO/IEC 14772~1:1998. 

 
%%page page_532                                                  <<<---3
 
Bibliography 513 
ISO/IEC:8839-1 1998. Infiormation Technolog)/-8-Bit Single-Byte Coded Graphic 
Character Sets-Part 1: Latin Alphahet No. 1. International Organization for 
Standardization, Geneva, Switzerland. International Standard ISO/IEC 
8859-1:1998. 
ISO/IEC:9070 1991. Information Proces:ing-SGML Support Facilitiex Registration Protedurexfor Public Text Owner Identifierx. Second edition, 15 April 
1991. International Organization for Standardization, Geneva, Switzerland. 
International Standard ISO/IEC 9070:1991. 
Jelliffe, R. 1998. The Xll/IL and SGML Cookbook: Recipexfior Structured 
Information. Englewood Cliffs, N.].: Prentice Hall. 
Knuth, D. E. 1986. The TEXhoole, volume A of Computer; and '13/pexetting. 
Reading, Mass.: Addison-Wesley. 
Lamport, L. 1994. Za“7EX: A Document Preparation Syxtem: User’: Guide and 
Reference Manual. 2nd edition. Reading, Mass.: Addison-Wesley. 
Leventhal, M., Lewis, D., and Fuchs, M. 1998. Designing XIVIL Internet 
Applicationx. Englewood Cliffs, NJ; Prentice Hall. 
Lie, H. W. and Bos, B. 1997. Caxcading Style Sheets: Dexigningfior the Web. 
Reading, Mass.: Addison-Wesley. 
Maler, E. and Andaloussi,J. E. 1996. Developing SGML DTDII From Text to 
Model to Markup. Englewood Cliffs, NJ.: Prentice Hall. 
McGrath, S. 1998. XIVIL by Example: Building E-Commerce Applivations. 
Englewood Cliffs, N._].: Prentice Hall. 
Megginson, D. 1998. Structuring XIVIL Doeumentx. The Charles F. Goldfarb 
series on open information management. Englewood Cliffs, N.J.: Prentice Hall. 
Merz, T. 1998. Web Publishing with Acrohat/PDF. Berlin, Germany: 
Springer-Verlag. 
Murray, and vanRyper, W. 1996. Graphicx File Formats. 2nd edition. 
Sebastopol, Calif.: O’Reilly & Assoc. Inc. 
Raggett, D., Lam, J., Alexander, I., and Kmiec, M. 1998. Raggett on HTML 4. 
Reading, Mass.: Addison-Wesley. 
Rahtz, S. 1995. Another look at IATEX to SGML. TUGhoat, 16 (3). 
Smith, N. E. 1998. SGML/XML Filterx. Plano, Tex.: Wordware Publishing, Inc. 
St. Laurent, S. 1997. XML:A Primer. Portland, Ore.: MIS Press. 
Tanenbaum, A. S. 1996. Computer Networ/ex. 3rd edition. Englewood Cliffs, 
N._].: Prentice Hall. 

 
%%page page_533                                                  <<<---3
 
S 14 Bibliography 
Unicode Consortium 1996. The Unicode Standard, Verxion 2.0. Reading, Mass.: 
Addison-Wesley. 
Wall, L., Christiansen, T., and Schwartz, R. L. 1996. Programming Perl. 2nd 
edition. Sebastopol, Calif.: O’Reilly & Assoc. Inc. 

 
%%page page_534                                                  <<<---3
 
Index 
Throughout the index bold face page numbers are used to indicate pages with important information about the 
entry, for instance, the precise definition of a command or a detailed explanation; page numbers in normal type 
indicate a textual reference. 
AAP (American Association of 
Publishers), 419 
Acrobat program, 15, 25-81, 272 
Acrobat Capture program, 28 
Acrobat Distiller program, 12, 15, 
28-30, 33-36, 80 
Acrobat Exchange program, 33, 47, 
50 
Acrobat Reader program, 27, 47 
Adobe Type Manager program, 30 
ae package, 33 
lE|fred program, 287, 459, 461 
AFM (Adobe Font Metric), 68, 74 
afm2tfm program, 74 
Amaya program, 275, 376 
AMS (American Mathematical Society), 227 
amsart package, 158 
amsmath package, 103, 129 
amsopn package, 103 
amstex package, 103, 129 
amsthm package, 103 
And reessen, Marc, 4 
array package, 158 
awk program, 292 
babel package, 89, 118, 119, 130 
Berglund, Anders, 247 
Berners-Lee, Tim, xvii, 3, 5, 247 
Bhattacharya, Tanmoy, 81, 403 
Bingham, Harvey, 316 
Bonhom me, Patrice, 420 
Bosak, Jon, 248, 500 
Braa ms, Johannes, 116 
Bray, Tim, 256, 288, 459 
Cailliau, Robert, 4 
Carlisle, David, 35, 330, 388 
CGI (Common Gateway Interface), 
251 
Cheong, Otfried, 21, 503 
Chopde, Avinash, 123, 124 
Christy, John, 503 
Clark, James, 272, 283, 293, 302, 
313, 318, 342, 355, 364, 388, 
434, 503, 510 
CMacTeX program, 67 
color package, 40, 81, 140, 215 
\color, 215 
\colorbox, 215, 216 
\definecolor, 216 
\fcolorbox, 215, 216 
\pagecolor, 215 
\textcolor, 215 
Connolly, Dan, 247 
convert program, 188 
Cover, Robin, 249 
Cowan,]ohn, 509 
CSS (Cascading Style Sheets) language, 297-312 
Daly, Patrick, 49 
devnag program, 123 
Dickie, Garth, 22 
DOM (Document Object Model), 
241, 282, 286,343,355, 387 
Doyle, Mark, 403 
Dra kos, Nikos, 84, 85 
DSSSL (Document Style Semantics and Specification Language) language, xix, xx, 12, 
255,279,289-291,3l2-337, 
353, 362, 363, 366, 387, 388, 
491, 501, 502 

 
%%page page_535                                                  <<<---3
 
516 DTD (Document Type Definition) - hyperref Index 
DTD (Document Type Definition), grep program, 278 \htm1nohead, 147 
259-270 htmlonly environment, 135, 
dtd2htm| program, 243, 275, 277 Hagen’ Hans’ 35’ 50’ 72’ 81 147: 148 
dtdzman program, 277 Hara|ambous,Yannis, 121, 123, 386 \htm1ref, 133 
dtddiff program, 243, 275 Harold, Elliotte Rusty, 249 Qllfimlrule, 128, 148 
m1ru1e*, 128 
dtd parse program, 277 haward Package’ 49 \htm1set, 153 
dtdtree program, 275, 277 He||ingman,Jeroen, 123 \htm1setenv, 153 
dtdview program, 275 H°ekW3teV> T390: 32 \htm1setstyle, 112, 130 
dviout program, 14, 404 
dvipdfm program, 12, 15, 28,36, 62 
dvips program, 12, 30, 34-36, 62, 
70, 72-74, 86, 91, 94, 98, 
121,153,188,404 
-z option, 35, 36 
cms .map file, 30 
config. cms file, 30 
psfonts .map file, 30 
dvipsone program, 30, 36, 62 
dviwindo program, 14, 30, 34, 39, 
40, 62, 404 
El Anda|oussi,Jeanne, 421 
emacs program, 272 
ESIS (Element Structure Information Set), 281-284, 287, 292, 
293, 502 
fancyhdr package, 47 
Flynn, Peter, 248, 266 
fop program, 343, 344, 362, 365 
fpTEX program, 67 
FrameMaker program, 34, 35, 272, 
292, 319, 387 
francais package, 151 
German, Daniel M., 313, 501 
Ghostscript program, 27, 28, 80, 
86, 91, 98, 99,121, 502 
Ginsparg, Paul, 403 
GML (Generalized Markup Language) language, 247, 289 
Gordon, Tom, 433 
Graham, Ian, 243 
Graham, Tony, 272 
graphics package, 81, 210 
Hood, Earl, 243, 275, 277 
ht program, 185 
html package, 83, 111, 114, 119, 
124-1 5 3 
°/.begin{1atexon1y} notation, 
1 36 
'/.end{1atexon1y} notation, 
1 36 
\bodytext, 139 
comment environment, 136 
description environment, 131 
displaymath environment, 129 
\documentstyle, 100 
\endsegment, 148 
eqnarray environment, 129 
equation environment, 129 
\externalcite, 135 
\externallabels, 134, 135, 141 
\externalref, 135, 141, 144 
figure environment, 129 
\htm1, 136 
\htm1addimg, 126, 127 
\htm1addnor1na1link, 126, 133, 
144 
\htm1addnor1na1linkfoot, 126 
\htm1 addtonavi gat ion, 13 7 
\htm1addtostyle, 130 
\htm1base, 140 
\htm1body, 139 
\htm1border, 131 
\htm1cite, 134 
\HTMLcode, 137 
\htm1head, 147 
\htm1image,114,l29,131 
\htm1info, 139 
\htm1info*, 139 
\htm1itemmark, 131 
\htm11a.nguagestyle, 119, 130 
htmllist environment, 131 
\htm1tracenv, 153 
\htm1tracing, 153 
\htm1url, 127 
\hypercite, 133, 134 
\hyperref, 132, 133 
imagesonly environment, 136 
\internal, 147, 151 
\1atex, 136 
\1atexhtm1, 136 
latexonly environment, 135 
\1atextohtm1, 128 
makeimage environment, 114, 
128, 129, 131 
math environment, 129 
minipage environment, 131 
rawhtml environment, 136 
\segment, 145 
\segmentcolor, 148 
\segmentpagecolor, 148 
\se1ect1a.nguage, 119 
slide environment, 128, 129 
\startdocument, 147, 148 
\strikeout, 131 
tabbing environment, 131 
table environment, 129, 131 
\tab1eofchi1dlinks, 139, 151 
\tab1eofchi1dlinks*, 139 
htmllist package, 131 
HTTP (Hypertext Transfer Protocol), xviii, 4-6, 496, 502 
Hutchinson, Ian, 19 
hyper package, 35 
Hyperlatex program, 21 
hyperref package, 14, 23-26, 3567, 133 
a4paper option, 63 
a5paper option, 63 
accesskey option, 65 

 
%%page page_536                                                  <<<---3
 
Index 
hyperref (continued) 
517 
\Acrobatmenu, 41, 47 
action option, 65 
align option, 65 
\AMSname, 46 
a.nchorcolor option, 63 
\appendixname, 46 
\autoref, 46, 59 
b5paper option, 63 
backgroundcolor option, 65 
backref option, 38, 63 
baseurl option, 65 
\bibitem, 38 
bookmarks option, 63 
bookmarksnumbered option, 63 
bookmarksopen option, 42, 63 
bordercolor option, 65 
bordersep option, 65 
borderstyle option, 65 
borderwidth option, 65 
breaklinks option, 39, 62 
calculate option, 57, 65 
\chaptername, 46 
charsize option, 65 
\CheckBox, 51 
checked option, 65 
\ChoiceMenu, 51 
citebordercolor option, 64 
citecolor option, 63 
color option, 66 
colorlinks option, 36, 63, 64 
combo option, 66 
debug option, 62 
default option, 66 
\Defau1tI-IeightofCheckBox, 
53 
\Defau1tI-IeightofChoiceMenu, 
53 
\Defau1tI-IeightofReset, 53 
\Def aultl-Ieightof Submit, 53 
\Defau1tI-IeightofText, 53 
\Defau1tWidthofCheckBox, 53 
\Defau1tWidthofChoiceMenu, 
53 
\Defau1tWidthofReset, 53 
\Defau1tWidthofSubmit, 53 
\Defau1tWidthofText, 53 
disabled option, 66 
draft option, 62 
dvipdfm option, 62 
dvips option, 50, 62 
dvipsone option, 62 
dviwindo option, 62 
encoding option, 65 
\equationname, 46 
executivepaper option, 63 
extension option, 40, 62 
\figurename, 46 
f ilebordercolor option, 64 
filecolor option, 63 
Form environment, 50, 53, 55 
format option, 57, 66 
height option, 66 
\I-Ifootnotename, 46 
hidden option, 66 
\href, 45 
\hyperbaseurl, 45 
\hyperdef, 45 
\hyperimage, 45 
hyperindex option, 63 
\hyperlink, 45 
\hyperref, 45, 47, 59 
hyperref . cfg file, 38, 39 
\hypersetup, 38 
\hypertarget, 45 
hypertex option, 36, 62 
implicit option, 36, 49, 62 
\Itemname, 46 
keystroke option, 5 7, 66 
\label, 45, 46, 49 
1atex2htm1 option, 62 
\LayoutCheckboxField, 52 
\LayoutChoiceField, 52 
\LayoutTextField, 52 
legalpaper option, 63 
letterpaper option, 63 
linkbordercolor option, 64 
linkcolor option, 63 
linktocpage option, 62 
\MakeButtonField, 52 
\MakeCheckField, 52 
\MakeChoiceField, 52 
\Ma.keR.adioField, 52 
\MakeTextField, 52 
maxlen option, 66 
menubordercolor option, 64 
menucolor option, 63 
menulength option, 66 
method option, 65 
multiline option, 66 
name option, 66 
nativepdf option, 62 
nesting option, 62 
onblur option, 66 
oncha.nge option, 66 
onclick option, 66 
ondblclick option, 66 
oneside option, 59 
onfocus option, 66 
onkeydown option, 66 
onkeypress option, 66 
onkeyup option, 66 
onmousedown option, 66 
onmousemove option, 66 
onmouseout option, 66 
onmouseover option, 66 
onmouseup option, 66 
onselect option, 66 
anchor option, 62 
backref option, 38, 63 
bordercolor option, 64 
color option, 63 
\pa.geref, 46 
\pageref*, 47 
\pa.ragraphname, 46 
\partname, 46 
password option, 66 
pdf author option, 65 
pdfborder option, 64 
pdfcenterwindow option, 64 
pdf creator option, 65 
pdffitwindow option, 64 
pdfhighlight option, 64 
pdfkeywords option, 65 
pdfmark option, 41, 62 
pdfmenubar option, 64 
pdfnewwindow option, 64 
pdfpagelayout option, 64 
pdfpagemode option, 63 
pdfpagescrop option, 64 
pdfpagetransition option, 65 
pdfproducer option, 65 
pdfstartpage option, 64 
pdfstartview option, 64 
pdf subject option, 65 

 
%%page page_537                                                  <<<---3
 
518 
hyperref (continued) MurrayRust 
Index 
pdftex option, 50, 62, 80 
pdftitle option, 65 
pdftoolbar option, 64 
pdfview option, 63 
pdfwindowui option, 64 
plainpages option, 62 
popdown option, 66 
\PushButton, 51 
radio option, 66 
raiselinks option, 62 
readonly option, 66 
\ref, 46, 49, 59 
\ref*, 45, 47 
\Reset, 51, 52 
\sectionname, 46 
\Submit, 51, 52 
\subsectionname, 46 
\subsubsectionname, 46 
tabkey option, 66 
\tablename, 46 
tex4ht option, 50, 56, 62 
\TextField, 51 
\theoremname, 46 
urlbordercolor option, 64 
urlcolor option, 63 
\usepackage, 39 
validate option, 57, 66 
value option, 66 
vtex option, 62 
width option, 66 
\wwwbrowser, 39 
idvi program, 22 
IEC (International Electrotechnical 
Commission), 312, 465, 466, 
475, 477, 479, 489, 495 
IETF (Internet Engineering Task 
Force), 5 
Illustrator program, 26 
lmageMagick program, 188 
lndica program, 121 
inputenc package, 119 
Internet Explorer program, 39, 272 
itrans program, 123-125 
Jade program, 314, 316-321, 323, 
325, 326, 328-333, 362, 388, 
503 
jadetex package, 388 
Java language, 20-24, 115, 195, 
196, 200, 201, 223, 224, 230235, 292, 342-344, 362, 366, 
434, 504 
JavaScript language, 27, 50, 55, 57, 
65, 66,196, 223, 224 
JPEG (Joint Photographic Experts 
Group), 10, 77, 80, 196,210, 
224, 232, 267 
Jumbo program, 459 
Kamthan, Pankaj, 505 
keyval package, 38 
koalaxsl program, 481 
Lamport, Leslie, 422 
h\TEX2HTML program, 62, 83-154 
.latex2html-init file, 99 
$HOME/ .1atex2htm1-init file, 
99 
amsart .perl file, 100 
a.msbook.perl file, 100 
article .perl file, 100 
book.perl file, 100 
configure-pstoimg file, 94 
\documentclass, 100 
\docu.mentstyle, 100 
english.perl file, 117 
french.perl file, 101 
germa.n.perl file, 117 
html . sty file, 86, 95 
htmllist . sty file, 95 
images.pre file, 121 
images.tex file, 91,121,136 
indica.perl file, 123 
itra.ns.perl file, 124 
labels.pl file, 134, 135 
latex2html file, 94, 98 
latex2html.config file, 92, 
94,95,98,117, 137 
latinl .pl file, 118 
latin5.pl file, 118 
local . pm file, 95 
makemap file, 94 
math.pl file, 101 
pstoimg file, 94 
pstoimg_nopipes file, 94 
report .perl file, 100 
\selectla.nguage, 119 
texexpand file, 94 
unicode.pl file, 101,118 
\usepackage, 100, 101 
usorbia.n.perl file, 117 
Lisp language, 315 
Lovell, Doug, 388 
Itoh program, 22 
makeidx package, 151 
makeindex program, 153 
Maler, Eve, 421 
Malyshev, Basil, 32 
Maple program, 272 
Marszalek, Kathleen, 326 
Maruyama, Hiroshi, 286 
Mathematica program, 272 
MathML language, xviii-xx, 20, 21, 
23, 24, 115, 194,224, 225, 
229, 231, 232, 265, 275, 291, 
330, 367-389, 492, 504, 505 
MathType program, 373, 374 
Megginson, David, 272, 287, 288, 
292,311,326, 421, 459, 461 
Mehlich, Michael, 35 
Merz, Thomas, 34, 47, 50 
Microsoft Internet Explorer program, 195, 196 
Microsoft Word program, 34, 35, 
321 
MIF (Maker Interchange Format), 
387 
MikTeX program, 67 
MIME (Multimedia Internet Mail 
Extensions), 5, 201, 202 
minitoc package, 49 
mm program, 123 
Moore, Ross, 84, 381 
Morel, Pierre, 275 
Mosaic program, 4 
Mozilla program, 272 
Murray-Rust, Peter, 459 

 
%%page page_538                                                  <<<---3
 
Index 
natbib package, 49 
nDV| program, 22 
netpbm program, 87, 91, 98, 505 
Netscape Navigator program, 195, 
196 
NikNak program, 28 
nsgmls program, 272, 282-286, 
292,293,319,434 
OASIS (Organization for the Advancement of Structured Information Standards), 266, 
282, 417, 496, 500 
Office program, 2 72 
Omega package 
\SGMLendtag, 386 
\SGMLentity, 386 
\SGMLstarttag, 386 
Omega program, 326, 386-387 
OzTeX program, 36, 404 
PageMaker program, 35 
patc program, 123 
PDF (Portable Document Format) 
language, xvii-xix, 12, 15, 23, 
24, 25-81, 292, 315, 328, 
343, 344, 362, 365, 404, 493, 
499, 501, 505, 506, 510, 512 
pdfscreen package, 59 
pdfTEX program, 12, 15, 35, 39,41, 
62, 67-81, 328 
! notation, 72 
< notation, 71-73 
<< notation, 71 
8r. enc file, 73 
ExtendFont notation, 72, 74 
\pdfan.not, 78 
\pdfan.notlink, 78, 79 
\pdfcata1og, 76 
\pdfcompresslevel, 75 
\pdfdest, 78, 79 
\pdfendlink, 78 
\pdfendthread, 79 
\pdff ontpref ix, 77 
\pdffor1n, 77 
\pdffor1nprefix, 77 
\pdfimage, 76, 77 
natbib \pdf imagepref ix, 77 
\pdfimagereso1ution, 77 
\pdfinfo, 75, 76 
\pdf1astan.not, 78 
\pdf1astform, 77 
\pdf1astobj, 80 
\pdf1itera1, 80 
\pdfnames, 80 
\pdfobj, 80 
\pdf outline, 79 
\pdf output, 75 
\pdfpageattr, 75 
\pdfpa.geheight, 75 
\pdfpagesattr, 75 
\pdfpagewidth, 75 
\pdfrefform, 77 
pdftex.cfg file, 68, 69, 74 
pdftex.pool file, 68 
\pdftexrevision, 80 
\pdftexversion, 80 
\pdfthread, 79 
\pdfthreadhoffset, 80 
\pdfthreadvoffset, 80 
PKFDNTS environment variable, 
68 
psfonts .map file, 69 
Sla.ntFont notation, 72 -74 
supp-mis . tex file, 81 
supp-pdf . tex file, 81 
TIFONTS environment variable, 
68 
TEXPSHEADERS 
variable, 68 
TTFONTS environment variable, 
68 
V1’-‘FONTS environment variable, 
68 
PDFWriter program, 28 
Peeter, Kasper, 22 
per|SGML program, 275 
PGC (PDF Glyph Container), 72 
pifont package, 49 
P|aice,]ohn, 386 
Prescod, Paul, 313, 501 
psgml program, 272, 273 
environment 
pstoimg program, 98 
Tamura 
519 
pstricks package, 81 
Python language, 366 
Quoung, Russell, 22 
Radhakrishnan, C. V., 59 
Raggett, Dave, 247, 456, 502 
Rahtz, Sebastian, 35, 81,133,326 
Raman, T. V., 22, 499 
RDF (Resource Description Framework), 452, 453 
Rokicki, Tom, 501 
SAX (Simple API for XML), 282, 
288, 343, 344, 355, 387, 459461, 494, 507 
Scheme language, 315 
Scribe program, 289 
Script program, 289 
seminar package, 128, 129 
sgcount program, 279 
sggrep program, 278, 279, 280 
SGML-Too|s program, 417 
sgmls program, 292 
SGMLSpm program, 292,311,459 
SGMLtoo|s program, 434 
sgmltrans program, 278 
sgrpg program, 279 
showkeys package, 49 
Sinai, Gaspar, 483 
Sinhala-TEX program, 121-123 
SMIL (Synchronized Multimedia 
Integration Language) language, 291, 494 
SP program, 283, 507 
Staflin, Lennart, 272 
Story, D. P., 34, 50 
stripsgml program, 275 
SVG (Scaleable Vector Graphics) 
language, 291, 366 
t4ht program, 168, 185, 186, 188, 
189 
tamilize program, 123 
Tamura, Kent, 286 

 
%%page page_539                                                  <<<---3
 
520 
Tauber tex4ht 
Index 
Tauber, James, 344, 362, 365 
tdtd program, 272 
techexplorer package 
\aboveTopic,218,219,223 
\a1tLink,215,220,230 
\appLink, 220 
\audioLink,211,213 
\backgroundcolor,215 
\backgroundimage,210 
\backgroundsound,210,211 
\buttonbox,217 
\color,205 
\colorbuttonbox,217 
\def,198 
\docLink,205,208,223 
\gdef,200 
\g1oba1newcommand,200 
\gradientbox,216,217 
\includeaudio,210 
\includegraphics,210 
\includevideo,211 
\labelLink,208,223 
\newcommand,200 
\newenvironment,Z00 
\newmenu,213,214 
\nextTopic,218,219,223 
\popupLink,208 
\previousTopic,218,219,223 
\rgb,216 
\usemenu,213,214 
\videoLink,211 
techexplorer program, 23, 195-224 
TEI (Text Encoding Initiative), 314, 
417, 420, 508 
teTEX program, 67 
TEX4ht program, 155-194, 382386, 404-415 
tex4ht package, 155-194, 382386, 404-415 
V, 182 
\ (, 159, 411 
\), 159, 411 
.tex4ht file, 193 
\ [, 159, 411 
\\, 181 
“13 option, 158 
_13 option, 158 
\ textcmd, 414 
\] , 159, 411 
O . 0 option, 405 
3.2 option, 158 
1 option, 157,173,176,182 
2 option, 157,173,176,182 
3 option, 157, 173, 176,182 
4 option, 157,173,176,182 
\ALIGN,180 
\AnchorLabe1, 384 
array environment, 180, 181 
array option, 181 
center environment, 178 
\chapter, 173, 176 
\C1r, 383 
cmbx font, 193 
cmr.htf file, 193 
cmr1O font, 187, 190, 191 
\Col, 386 
\Configure, 168, 174, 175, 177, 
180,181,183,184,189,191, 
192, 405, 408, 411-414 
\ConfigureEnv, 179, 406 
\ConfigureList, 179, 180, 384, 
406 
\ConfigurePictureFor1nat, 
160 
\ConfigureToc, 174 
\Css, 168 
\CssFi1e, 168 
\CutAt, 157, 176, 177, 182 
\De1eteMark, 384 
description 
178, 406 
displaymath environment, 411 
document environment, 169 
\DviMath,411 
edit option, 408 
\empty, 407 
\EndCss, 168 
\EndCssFi1e, 168 
\EndDviMath, 411 
\End1-IPage, 166 
\EndLink, 167 
\EndNoFonts, 192 
\EndP, 183 
\EndPauseMathC1ass, 412 
\EndPicture, 159 
environment, 
\EndPreamb1e, 168, 171 
\EndTrace, 413 
\EndVerify, 410 
enumerate environment, 178 
eqnarray environment, 385 
eqnarray option, 181 
\ExitI-IPage, 166 
\Fi1eName, 182 
flushleft enviromnent, 178 
flushright environment, 178 
fonts option, 383 
fonts+ option, 158 
\Gamma, 190, 191 
\I-IChar, 166 
\I-ICode, 164, 166, 183,410, 412 
\HCo1, 180 
\Hnewline, 166 
hooks option, 405-407, 410 
hooks+ option, 410 
\HPage, 166, 182 
\HRow, 180 
\hspace, 170 
htm option, 158 
html option, 157, 158,183 
html , O . O , hooks option, 405 
\ifI-Itml, 183 
\IgnorePar, 183 
\Indent, 183 
info option, 158 
\item, 178 
itemize environment, 178 
jpg option, 160 
\label, 383 
\Link, 167 
list environment, 178 
math environment, 411 
math option, 410, 412 
\mathbin, 412 
\mathc1ose, 412 
\mathop, 412 
\mathopen, 412 
\mathord, 412 
\mathpunc, 412 
\mathre1, 412 
\mu1tico1umn, 180, 186 
\MULTISPAN,180 
\Needs, 189 
\NewConfigure, 184 

%==========540==========<<<---2
 
%%page page_540                                                  <<<---3
 
Index 
\NewSection, 177 
\newtheorem, 178 
next option, 158 
\NextFi1e, 182 
\NextPictureFi1e, 160 
no“ option, 158 
no_ option, 158 
no_style option, 158 
no_amsart option, 158 
no_a.rray option, 158 
\NoFonts, 192 
\NoIndent, 183 
\PauseMathC1ass, 412 
pic-array option, 158, 181 
pic-displaylines option, 158 
pic-eqnarray option, 158, 181 
pic-tabbing option, 158, 182 
pic-tabbing’ option, 182 
pic-tabular option, 158, 181 
\Picture, 159, 160 
\Picture+, 159 
\PictureFi1e, 160 
png option, 160 
\Preamb1e, 171, 172, 186 
\PutLabe1, 383 
quotation environment, 178 
quote environment, 178 
refoption, 383 
refcaption option, 158 
\Row, 386 
\ScriptE.nv, 183 
\section, 173, 177 
\section*, 173 
sections+ option, 158 
\Send, 412-414 
ShowFont option, 192 
\ShowPar, 183 
\TABBING, 181 
tabbing environment, 181 
\tab1eofcontents, 157, 172175 
tabular enviromnent, 
181, 383 
tabular option, 181 
tex4ht .env file, 193 
tex4ht.sty file, 156, 169, 171, 
185, 186 
\TG, 410 
180, 
teX4ht (continued) \Tg, 383, 408, 409, 410 
thebibliography 
ment, 178 
\TocAt, 173, 175 
\TocAt*, 157,173, 175 
\Trace, 413 
trivlist environment, 178 
try ,htm1 , 0.0 ,hooks option, 
406 
\usepacka.ge, 171, 172, 407 
\verb, 166 
verbatim environment, 
178 
\Verify, 410 
verify option, 409, 410 
verify+ option, 410 
verse enviromnent, 178 
tex4ht program, 56, 169, 185-190, 
192, 193 
c.htf file, 190 
cm.htf file, 190 
cmbx10.htf file, 187 
cmr.htf file, 187, 190 
cmr1.htf file, 190 
cmr10.htf file, 190 
TeXML language, 388, 389 
texnames package, 128 
environ166, 
textonly program, 280, 281 
Textures program, 36, 404 
Thanh, Han Thé, xxi, 36, 67 
Tidy program, 456, 488 
tmilize program, 123 
ttf2afm program, 68, 74 
TtH program, 19, 23, 379, 380 
TUG (TEX Users’ Group), 81, 159 
UCS (Universal Character Set), 
477,480,489 
Unicode 
ISO 10646 (Unicode) Character Encoding Standard, 108, 
110,116,118,121, 257, 264, 
265, 287,326, 371, 372, 386, 
466, 4754477, 479, 480, 482, 
483, 485-487, 489, 495 
WebTEX 
521 
URI (Universal Resource Identifier), 
5, 10, 266, 267, 271, 282, 
298, 339, 352, 415, 416, 454, 
495-497 
url package, 127 
URN (Universal Resource Name), 5, 
240 
VBScript language, 27 
Vojta, Paul, 509 
VRML (Virtual Reality Modeling 
Language) language, 12, 496 
VTEX program, 20, 23, 28, 34, 62, 
67, 81 
Wall, Larry, 33, 34 
VVaBh, PJonnan, 277, 337, 417, 
419,421,501 
web2c program, 67, 68 
WebTEX package 
\array,228 
\arrayopts,228 
\bar,227 
\bghigh1ight,229 
\binom,227 
\check,227 
\co1a1ign,228 
\cos,227 
\ddot,227 
\define,227 
\disp1aystyle,227 
\dot, 227 
\fghigh1ight, 229 
\fontcolor,227,231 
\frac,227 
\hat, 227 
\href,229 
\1arge,227 
\1eft,228 
\mathbb,227 
\mathbf,227 
\mathca1,227 
\mathfr,227 
\mathit,227 
\mathrm,227 
\mathsf,227 
\mathtt, 227 

 
%%page page_541                                                  <<<---3
 
522 
WebTEX - zlib 
Index 
\medsp,229 
\medspace,229 
\mu1tiscript,228 
\overbrace,227 
\overset,227 
\quad,229 
\right, 228 
\root,227 
\rowa1ign,228 
\ru1e,229 
\scriptscriptsize,227 
\scriptscriptstyle,227 
\scriptsize,227 
\scriptstyle,227 
\sma11,227 
\space,229 
\sqrt,227 
\statusline,229 
\tensor,228 
\text,227 
\textsize,227 
\textstyle,227 
\thicksp,229 
\thickspace,229 
\thinsp,229 
\thinspace,Z29 
\ti1de,227 
\togg1e, 230 
\underbrace, 227 
\underset, 227 
\vec, 227 
WebEQ program, 20, 21, 115, 195, 
224-234, 236, 237, 371, 376, 
378, 380-382 
Wicks, Mark, 28, 36 
Williams, Peter, 49 
Word97 program, 323 
World Wide Web Consortium 
(VV3C), 2 
Wortmann, Uli, 138 
xdvi program, 14, 36, 404 
XHTML (Extensible Hypertext 
Markup Language), 243, 
453-456, 459, 487, 497, 502 
XML (Extensible Markup Language) language, 248-288 
XML for Java program, 286 
Xp program, 342 
Xpdf program, 27, 510 
xr package, 40, 49, 62 
XSL (Extensible Style Language) 
language, 337-365 
XSL language 
* (wildcard) notation, 344 
. (ancestor) notation, 346 
. (parent node) notation, 
345 
. (current node) notation, 
345 
/ notation, 346 
// notation, 344, 346 
// (descendants) notation, 
346 
/ (path) notation, 344 
/ (root node) notation, 346 
Q (attributes) notation, 345 
[3 (tests) notation, 345, 346 
priority attribute notation, 
347 
l (union) notation, 344, 346 
macro notation, 363 
node selection notation, 344 
xs|S|ideMaker program, 343, 344 
xt program, 342, 343, 355-357, 
359-361, 364, 450, 481, 487, 
488 
Yudit program, 450, 433, 435, 486 
zlib program, 68, 75 

 
%%page page_542                                                  <<<---3
 
Other books from Addison Wesley Longman 
PostScript Language Reference Manual, Third Edition, Adobe Systems 
The PostScript language is widely recognized as the industry standard for page description. Incorporated into a broad range of printers, imagesetters, and computer 
displays, PostScript describes exactly how text, sampled images, and graphics will 
appear on a printed page or on a computer screen. 
The PostScript Language Reference-known as the Red Book-is the complete and authoritative reference manual for the PostScript language. Prepared by 
Adobe Systems Incorporated, the creators and stewards of the PostScript standard, 
it documents the syntax and semantics of the language, the Adobe imaging model, 
and the effects of the graphics operators. This Third Edition has been updated to 
include LanguageLevel 3 extensions, which unify a number of previous extensions 
and introduce many new features, such as high-fidelity color, support for masked 
images, and smoother shading capabilities. 
An accompanying CD-ROM contains the entire text of this book in Portable Document Format (PDF) ISBN 0-201-37922-8. 
I‘}'lEX: A Document Preparation System, Second Edition, Leslie Lamport. 
The definitive Ié'I]3X user's guide and reference manual, written by the system's 
creator, clearly documenting the latest 28 release. ISBN 0-201-52983-1 
The IHEX Companion, Michel Goossens, Frank Mittelbach, Alexander Samarin. 
A companion to the Lamport book and to any other introduction to Ié'IEX, this 
expertly compiled reference answers common user questions left open by the introductions, and describes add-on packages available to solve diverse problems. ISBN 
0-201-54199-8 
The I‘3‘I]§X Graphics Companion, Michel Goossens, Sebastian Rahtz, Frank Mittelbach. This handy reference describes techniques and tricks needed to illustrate 
BTEX documents, and answers common user questions about graphics and PostScript fonts. ISBN 0-201-85469-4 
BUGS in Writing: A Guide to Debugging Your Prose, Lyn Dupré. 
VVhat every scientific writer needs to know to write clearly, correctly, effectively. 
Numerous examples--often hilarious! Comprises 150 easily digestible segments. 
ISBN 0-201-3792 1-X 
Available w/Jere tec/mical books are sold. Or, order directly from Addison I/Vesley 
Longmzm:1-800-822-6339. http : //www . aw1.com/cseng 

 
%%page page_543                                                  <<<---3
 
