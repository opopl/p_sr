% vim: keymap=russian-jcukenwin
%%beginhead 
 
%%file 02_11_2020.stz.news.ua.mrpl_city.1.budivli_dramteatru_mrpl_60_rokiv
%%parent 02_11_2020
 
%%url https://mrpl.city/blogs/view/budivli-dramatichnogo-teatru-mariupolya-60-rokiv
 
%%author_id demidko_olga.mariupol,news.ua.mrpl_city
%%date 
 
%%tags 
%%title Будівлі драматичного театру Маріуполя 60 років
 
%%endhead 
 
\subsection{Будівлі драматичного театру Маріуполя 60 років}
\label{sec:02_11_2020.stz.news.ua.mrpl_city.1.budivli_dramteatru_mrpl_60_rokiv}
 
\Purl{https://mrpl.city/blogs/view/budivli-dramatichnogo-teatru-mariupolya-60-rokiv}
\ifcmt
 author_begin
   author_id demidko_olga.mariupol,news.ua.mrpl_city
 author_end
\fi

\ii{02_11_2020.stz.news.ua.mrpl_city.1.budivli_dramteatru_mrpl_60_rokiv.pic.1}

2 листопада 1960 р. засяяли вогні нового театру – відбулося його урочисте
відкриття. Для маріупольців ця подія стала справжнім святом і багато містян,
які бачили відкриття на власні очі, й досі згадують цей день з посмішкою.

З нагоди цієї важливої події я вирішила згадати \emph{10 найцікавіших фактів} про
будівлю Донецького академічного обласного драматичного театру (м. Маріуполь):

1. Однією з перепон для будівництва театрального приміщення стала директива,
згідно з якою будівлі театрів могли побудувати лише в обласних містах. Однак
міська влада Маріуполя (тоді Жданова) в усіх клопотаннях говорила про створення
обласного драматичного театру, а вже потім зазначала, що з різних об'єктивних
причин приміщення для нової установи культури доцільно звести у прилеглому
Маріуполі. Саме з тих часів на афішах театру з'явилася назва: \enquote{Донецький}, а в
дужках – \enquote{місто Маріуполь}.

\ii{02_11_2020.stz.news.ua.mrpl_city.1.budivli_dramteatru_mrpl_60_rokiv.pic.2}

2. Проєкт будівлі був розроблений архітекторами Київського ДІПроміста О.
Малишенком та О. Криловою за зразком Полтавського обласного театру. Також за
проєктом архітекторів був побудований і театр в Макііївці, тому вважається, що
у нашого театру є свої архітектурні родичі.

3. Будівництво театру йшло важко. З великими труднощами діставали необхідні
матеріали, були збої з фінансуванням. Велику роль у тому, що театр все ж таки
був побудований, відіграли колективи та керівники будівельних організацій і
великих підприємств міста: заводів імені Ілліча, \enquote{Азовсталі}, коксохімічного.

4. Загальна вартість будівництва склала – 14 мільйонів карбованців.

5. Будівництво тривало 4 роки (почалося навесні 1956 року). Будівля побудована
з кримського інкерманського сірого каменю.

\ii{02_11_2020.stz.news.ua.mrpl_city.1.budivli_dramteatru_mrpl_60_rokiv.pic.3}

6. Чудово обладнане приміщення для театру побудоване в стилі радянського
монументального класицизму з великою кількістю ліпних декоративних елементів.
Об'єм будівлі – 37 тис. м3. Центральна фасадна частина будівлі з портиком
коринфського ордера на квадратних колонах і віконними прорізами з напіварочним
завершенням злітає вгору кленоподібним фронтоном на консолях із гармонійною
скульптурною композицією, де чільне місце відводиться металургам і хліборобам
як основним працівникам Приазов'я, хвалебну оду яким виконують покровительки
мистецтва.

7. Велика глядацька зала театру розрахована на 800 місць (сьогодні 680).
Освітлення сцени забезпечили 5 софітових ферм, якими можна зарядити одночасно
42 декорації, тобто забезпечити резерв на дві вистави. Завдяки потужній
вентиляційній установці в театрі є можливість забезпечувати глядацьку залу
очищеним теплим повітрям в зимовий час і свіжим повітрям – у літній.

\ii{02_11_2020.stz.news.ua.mrpl_city.1.budivli_dramteatru_mrpl_60_rokiv.pic.4}
\ii{02_11_2020.stz.news.ua.mrpl_city.1.budivli_dramteatru_mrpl_60_rokiv.pic.5}

8. На виставу \enquote{Іркутська історія}, що представили на відкритті театру, отримали
запрошення всі будівельники, які брали участь у будівництві театрального
приміщення, тому глядацька зала була переповнена будівельниками.

9. У 1985 році була відкрита мала сцена театру на 60 місць.

10.  Приміщення драматичного театру є єдиним, що отримало статус пам'ятки
архітектури державного значення і одним з 6 пам'яток архітектури місцевого
значення.

\ii{02_11_2020.stz.news.ua.mrpl_city.1.budivli_dramteatru_mrpl_60_rokiv.pic.6}

Підтвердженням того факту, що відкриття будівлі Донецького академічного
обласного драматичного театру (м. Маріуполь) сталося саме 2 листопада 1960 року
сьогодні крім спогадів очевидців є і збережені програмки та запрошення, на яких
зазначена дата відкриття.

Напередодні відкриття театру я проводила опитування серед маріупольців і
питала, скільки на їхню думку років будівлі нашого театру. Відповіді були
різними: від 50 до 150 років. Але всіх містян об'єднували відповіді щодо
вагомої ролі та важливого значення театрального приміщення, яке для більшості
маріупольців є умовним центром міста та справжнім архітектурним магнітом, який
притягує до себе як корінних жителів, так і гостей та туристів міста.

