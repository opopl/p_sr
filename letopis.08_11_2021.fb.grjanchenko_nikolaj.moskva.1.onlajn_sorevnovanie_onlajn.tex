% vim: keymap=russian-jcukenwin
%%beginhead 
 
%%file 08_11_2021.fb.grjanchenko_nikolaj.moskva.1.onlajn_sorevnovanie_onlajn
%%parent 08_11_2021
 
%%url https://www.facebook.com/permalink.php?story_fbid=451017316640218&id=100051958604817
 
%%author_id grjanchenko_nikolaj.moskva
%%date 
 
%%tags internet,rossia,sambo,sorevnovanie,sport,volgograd
%%title Подготовка к онлайн соревнованиям идет полным ходом
 
%%endhead 
 
\subsection{Подготовка к онлайн соревнованиям идет полным ходом}
\label{sec:08_11_2021.fb.grjanchenko_nikolaj.moskva.1.onlajn_sorevnovanie_onlajn}
 
\Purl{https://www.facebook.com/permalink.php?story_fbid=451017316640218&id=100051958604817}
\ifcmt
 author_begin
   author_id grjanchenko_nikolaj.moskva
 author_end
\fi

Подготовка к онлайн соревнованиям идет полным ходом.

В офисе Волгоградской областной федерации самбо 3 ноября 2021 года прошло
заседание, на которое были приглашены ответственные  за проведение на местах
вторых демонстрационных онлайн соревнований Волгоградской области по самбо.

\ifcmt
  tab_begin cols=2

     pic https://scontent-mxp1-1.xx.fbcdn.net/v/t1.6435-9/254647805_451017046640245_1866319033539471790_n.jpg?_nc_cat=110&ccb=1-5&_nc_sid=730e14&_nc_ohc=A-HB6M0BMVIAX8_JWyW&_nc_ht=scontent-mxp1-1.xx&oh=21d6564fab06c92e3be114d9a960897b&oe=61AD4A3E

     pic https://scontent-mxp1-1.xx.fbcdn.net/v/t1.6435-9/254696129_451016893306927_3735096063389691391_n.jpg?_nc_cat=106&ccb=1-5&_nc_sid=730e14&_nc_ohc=tARrSADDZRwAX_rWhie&_nc_ht=scontent-mxp1-1.xx&oh=92f8c17ac411a1344cdc886741bc81b6&oe=61B0C2D6

  tab_end
\fi

Незадолго до этого с участниками соревнований были проведены пробные
подключения для того чтобы минимизировать все проблемы, которые потенциально
могут возникнуть при установлении видео связи и проведении спортивного
мероприятия.

\ifcmt
  tab_begin cols=3

     pic https://scontent-mxp1-1.xx.fbcdn.net/v/t1.6435-9/254823682_451016899973593_8327545047506810051_n.jpg?_nc_cat=102&ccb=1-5&_nc_sid=730e14&_nc_ohc=xQwLR4W0thIAX_eOmBI&_nc_ht=scontent-mxp1-1.xx&oh=53d6f0e6064efe7494ccd1d06ed1df55&oe=61AE594B

     pic https://scontent-mxp1-1.xx.fbcdn.net/v/t1.6435-9/254820973_451017049973578_1088749945913485706_n.jpg?_nc_cat=105&ccb=1-5&_nc_sid=730e14&_nc_ohc=kRVQNKZF6OcAX_UuKC_&_nc_ht=scontent-mxp1-1.xx&oh=c3c5b6dfd96f92f4ce54c5a3b87cb907&oe=61B08933

		 pic https://scontent-mxp1-1.xx.fbcdn.net/v/t1.6435-9/255061177_451017113306905_5974420614666585013_n.jpg?_nc_cat=110&ccb=1-5&_nc_sid=730e14&_nc_ohc=wrKQ8KrEZcoAX-S4ovJ&_nc_ht=scontent-mxp1-1.xx&oh=840a2179957e5aaac6d7d621e4a9e1c2&oe=61B07081

  tab_end
\fi

С присутствующими обсудили все нюансы, регламент мероприятия и технические
детали проведения данного турнира. Несмотря на то что при проведении самих
занятий в рамках проекта «самбо в школу» допускается использование обычной
школьной формы для занятий физической культурой,  на онлайн соревнованиях все
участвующие должны быть в специализированной форме для борьбой самбо. Все как
на настоящих соревнованиях, которые мы привыкли  проводить в очном формате.   

На мероприятие были пригашены учителя общеобразовательных школ и утвержден
состав судейской коллегии данного спортивного мероприятия.

В судейскую коллегию вошли следующие тренеры и спортсмены Волгоградской
области:

1 Калашников Виктор Алексеевич;

2 Халлыев Рустам Мухамметуратович;

3 Саакян Сурик Артурович;

4 Романовский Антон Олегович;

5 Тепловодский Богдан Дмитриевич;

6 Кулагина Анастасия Романовна. 

Свои предложения и коррективы внес Президент Волгоградской области по самбо,
мастер спорта по самбо, заслуженный юрист РФ, региональный представитель
межгосударственного союза городов-героев Грянченко Николай Васильевич,  а так
же  главный тренер Волгоградской области по самбо Филиппов Максим Владимирович,
который так же присутствовал на заседании. 

В борьбе за президентские награды примут участие следующие образовательные
организации региона:

Этап  1 МОУ «СШ № 140» Советского района г. Волгограда и МОУ «СШ № 128»
Дзержинского района г. Волгограда

Этап  2 МОУ «СШ № 85» Дзержинского района г. Волгограда и МОУ «Лицей № 6»
Ворошиловского района г. Волгограда

Этап 3 МОУ «СШ № 27» Тракторозаводского района г. Волгограда и  МОУ «СШ 72»
Краснооктябрьского района г. Волгограда

Этап 4 - 14-20 – 15-20 МОУ «СШ № 1» г. Калач-на-Дону

Готовить команды школ города и области будут следующие работающие в них
специалисты: 

1 Халлыев Мухаммедмурад Аннурдыевич (тренер) – школа № 140 Советский район

2 Филиппов Максим Владимирович (главный тренер Волгоградской области по самбо)
и Гульцова Людмила Афанасьевна (учитель физической культуры)    - школа № 85
Дзержинский район;

3 Леднёв Роман Владимирович (тренер)  – школа № 120 Дзержинский район; 

4 Ивлев Александр Николаевич (тренер)  – школа № 57 Краснооктябрьский район; 

5 Кувшинчиков Андрей Леонидович (учитель физической культуры) – школа 1 г.
Калач-на-Дону; 

6 Цыганок Сергей (учитель физической культуры) – Лицей № 6 Ворошиловский район; 

7 Репин Олег Анатольевич (тренер) - Школа № 27 Тракторозаводский район. 

Победитель соревнований будет определен по наибольшему количеству набранных
баллов, число которых будет зависеть от качества выполнения 11 приемов
программы онлайн соревнования. 

На торжественной церемонии, которая пройдет на следующий день после завершения
турнира, лучшим будут вручаться огромные кубки, щедро предоставленные фондом
Президентских грантов, так же будут поощрены спортсмены в 4 номинациях:

За лучшее выполнение  самостраховки;  

За лучшую технику выполнения приемов;  

За лучший бросок;  

За лучшее  ущемление  ахиллова сухожилия; 

Как видно из текста статьи, реализация проекта «Самбо – путь к победам», на
который в 2021 году был выделен президентский грант, идет полным ходом!
