% vim: keymap=russian-jcukenwin
%%beginhead 
 
%%file slova.evropa
%%parent slova
 
%%url 
 
%%author 
%%author_id 
%%author_url 
 
%%tags 
%%title 
 
%%endhead 
\chapter{Европа}
\label{sec:slova.evropa}

%%%cit
%%%cit_pic
%%%cit_text
Первый русский митрополит Илларион восхищённо говорил: «Церковь дивна и славна
всем окружным странам, яко же иной не найдется во всех землях от востока до
запада».  Итальянский путешественник Павел Аллепский писал, что «ум
человеческий не в силах её обнять».  Епископ киевский Иосиф Верещинский в XVI
веке отзывался о Софии Киевской так: «Весьма многие согласны в том, что в целой
\emph{Европе} нет храмов, которые по драгоценности и изяществу украшений стояли бы
выше константинопольского и киевского»
%%%cit_comment
%%%cit_title
\citTitle{Древнерусское зодчество. Собор Святой Софии в Киеве}, 
Искусство С Ириной Дружининой, zen.yandex.ru, 15.05.2021
%%%endcit

Перешлите кто-нибудь главе \emph{Европарламента}. Пусть человек порадуется, что
его призыв услышали в Донецке, который в самом \emph{Европарламенте} не слышен.
И пусть Давид Сассоли или еще какой-нибудь Борэль на пару с баронессой Эштон
еще несколько подобных акций придумают.  Можно продумать флешмоб, в котором
рассказывается о том, что раз в сто лет \emph{европейские} страны собираются в
кучу, чтобы получить люлей от России; это просто вариант, я не настаиваю... Но
все-таки скоро почти сто лет, как \emph{европейцы} без живительных российских
люлей живут.  Грустно небось?, 
\textbf{Донецк – Европа!}, Сергей Лебедев (Лохматый), voskhodinfo.su, 02.06.2021

А вы как думали: зря, что ли, пан президент так этих самых
«активистов-праворадикалов-патриотов» слушается и во всем им угодить старается?
Он же скоро будет не только за клоуна всей державы, но и за шамана – будет
камлать и постоянно мантры озвучивать: «Примите в НАТО! Примите в \emph{ЕС}!
Дайте денег! Дайте оружия! Мы – щит для \emph{Европы!} \emph{Европа}, защити
нас! Джо, позвони мне, позвони!» В общем, мантр у пана украинского президента
много – успевай только озвучивать!,
\citTitle{Форма не главное – главное содержание! И в спорте тоже...}, Мысли Бабы Яги, zen.yandex.ru, 07.06.2021

