% vim: keymap=russian-jcukenwin
%%beginhead 
 
%%file 06_12_2020.fb.palii_oleksandr.1.rus_oseledci
%%parent 06_12_2020
 
%%url https://www.facebook.com/oleksandr.palii/posts/1667632693408399
 
%%author Палій, Олександр
%%author_id palii_oleksandr
%%author_url 
 
%%tags 
%%title Русь - не лише з "оселедцями" на головах, але й у шароварах і шапках зі шликами
 
%%endhead 
 
\subsection{Русь - не лише з \enquote{оселедцями} на головах, але й у шароварах і шапках зі шликами}
\label{sec:06_12_2020.fb.palii_oleksandr.1.rus_oseledci}
\Purl{https://www.facebook.com/oleksandr.palii/posts/1667632693408399}
\ifcmt
	author_begin
   author_id palii_oleksandr
	author_end
\fi

\index[rus]{История!Русь}

Русь - не лише з \enquote{оселедцями} на головах, але й у шароварах і шапках зі шликами:

\begin{fancyquote}
Розповідь про країну русів і її міста. На схід від цієї країни - гори
печенігів; на південь від неї – ріка Рута (Рута, притока Росі, нині частіше ця
ріка називається Протока); на захід від неї - саклаби (слов'яни або склавени);
на північ - Ненаселені Землі Півночі. Країна русів - велика країна, і жителі її
несумирні, непокірливі, мають гордовитий вигляд, задиристі й войовничі. Вони
воюють із усіма невірними, що живуть навколо них, і виходять переможцями.
Володар їхній зветься Рус-Каган. Ця країна надзвичайно щедро наділена природою
в тому, що стосується всього необхідного (для життя). Частина з них
відрізняється люб'язністю обходження. Вони тримають у пошані знахарів (що
властиво для сарматів та багатьох індоіранських народів, аж до давніх і
сучасних персів). Щорічно вони виплачують десяту частину від свого видобутку й
торговельних доходів уряду... Зі 100 ліктів бавовняної тканини, більше або менш
того, шиють вони штани, які надягають, засукавши їх вище коліна. Вони носять
шапки з овечої вовни, з хвостами, що спадають позаду на їхні шиї. Мертвих вони
ховають із усім їхнім майном, одягом і прикрасами. (Також) вони поміщають їжу й
питво у могилу разом з мертвим. Куяба (Київ) – це місто русів, розташоване
ближче за все до земель ісламу. Це приємне місце й місце перебування володаря.
Воно виробляє хутра й коштовні мечі. 
\end{fancyquote}

(Перськомовна «Книга про межі світу від сходу до заходу»,  написана 982 р. про події ІХ ст.)

Реконструкція одягу русина за матеріалами розкопок біля Чернігова (10 століття) повністю підтверджує ці перськомовні дані.

\ifcmt
pic https://scontent.fiev6-1.fna.fbcdn.net/v/t1.0-9/130187045_1667631643408504_1062253342904546847_o.jpg?_nc_cat=101&ccb=2&_nc_sid=730e14&_nc_ohc=ocWGTDdw_rkAX-yfBCf&_nc_ht=scontent.fiev6-1.fna&oh=4c70b5ff26d71ccad82facff0074e150&oe=5FF610DE
caption Реконструкція одягу русина за матеріалами розкопок біля Чернігова (10 століття)
\fi
