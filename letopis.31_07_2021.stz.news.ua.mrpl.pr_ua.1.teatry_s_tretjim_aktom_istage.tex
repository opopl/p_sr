% vim: keymap=russian-jcukenwin
%%beginhead 
 
%%file 31_07_2021.stz.news.ua.mrpl.pr_ua.1.teatry_s_tretjim_aktom_istage
%%parent 31_07_2021
 
%%url http://pr.ua/ru/news/teatri-z-tretim-aktom-swho-pobachatq-mariupolqcsi-na-istage
 
%%author_id news.ua.mrpl.pr_ua,ivanova_jana.mariupol
%%date 
 
%%tags 
%%title Театры с третьим актом: что увидят мариупольцы на IStage?
 
%%endhead 
 
\subsection{Театры с третьим актом: что увидят мариупольцы на IStage?}
\label{sec:31_07_2021.stz.news.ua.mrpl.pr_ua.1.teatry_s_tretjim_aktom_istage}
 
\Purl{http://pr.ua/ru/news/teatri-z-tretim-aktom-swho-pobachatq-mariupolqcsi-na-istage}
\ifcmt
 author_begin
   author_id news.ua.mrpl.pr_ua,ivanova_jana.mariupol
 author_end
\fi

\begin{quote}
\Large
Театр-лаборатория и театр-стрим... Мариуполь станет средоточием сценического
искусства лучших коллективов страны. 
\end{quote}

Киев, Львов, Ивано-Франковск, Харьков и Днепр покажут современные спектакли и
затронут актуальные проблемы современности на фестивале IStage с 11 по 15
августа.

Уникальностью события станет так называемый третий акт — после действий на
сцене зритель вместе с создателями спектаклей сможет обсудить сюжет и роль
театра в решении общественных проблем. Новый фестиваль в Мариуполе покажут в
рамках проекта культурных инициатив \enquote{Діалог мовою мистецтва} при
поддержке Украинского культурного фонда.

Что такое театр-лаборатория и что именно увидят мариупольцы PR.UA рассказал
театральный куратор проекта \textbf{Андрей Палатный}.

\begin{quote}
\em
\enquote{Это будет фестиваль-лаборатория идей актуального театра. Лаборатория идей
означает, что мы выходим из концепции, что за каждым театром стоит определенная
школа, идея и философия. Собственно, фестиваль IStage и объединился в новую
платформу для диалога языком театрального искусства. И в рамках события
презентуем разные проекты из разных регионов Украины, которые отзеркалят
контекст современности. В августе Мариуполь объединят важные темы на сцене
языком театра. То есть с 11 по 15 августа ЦСИ \enquote{Отель Континенталь} станет
местом встречи и обмена идеями между разными театральными командами и
зрителями. Ключевыми вопросами станут: о чем мне говорит украинский театр, что
за идеи стоят за каждым из спектаклей, что нас объединяет? И мы приглашаем
посетителей окунуться в эти концепты}, — отметил Андрей Палатный.
\end{quote}

Кроме того IStage станет первым стриминговым театральным фестивалем, которые
объединит оффлайн-присутствующих зрителей трансляцией онлайн. По словам Андрея
Палатного, даже в период жесткого карантина подобных масштабных стриминговых
проектов не проводилось. Целью стрима станет доступность для зрителя, который
по разным причинам не смог попасть на событие.

\begin{quote}
\enquote{У нас есть пять дней, две сцены, восемь спектаклей, а также эксклюзивное
девятое событие. Мариуполь увидит семь театров из других городов. Интересных
спектаклей сегодня в Украине на самом деле много и все их хочется показать,
поэтому и появилась идея провести видеопоказы некоторых из них. Киев
представят три команды - Центр современного искусства \enquote{Дах}, \enquote{Дикий театр} и
Киевский академический театр драмы и комедии на левом берегу Днепра. Также
будет Харьковский альтернативний независимый театр \enquote{Нєфть}, Днепровский
молодежный театр \enquote{Віримо}, Івано-Фран\hyp{}ковский национальный драматический
театр, а Львов представит проект \enquote{Осень на Плутоне}. Также будет презентована
международная коллаборация Мариуполя с Лондоном. Последняя создана во время
карантина режиссером из Великобритании Джози Дейл-Джонс и народным театром
\enquote{Театромания} под руководством Антона Тельбизова в рамках проекта
международных театральных резиденций \enquote{Taking the Stage 2.0}. Проект назван \enquote{R
u there? We r here.../Вu там? Ми тут...} и будет показан в новой редакции и
более расширенной версии во второй раз в Украине}, — добавляет куратор
фестиваля IStage.
\end{quote}

Параллельно с этим будет действовать Марафон международных театральных
резиденций. На нем европейские режиссеры и творческие команды города проведут
разные театральные эксперименты, часть которых можно будет также увидеть на
IStage, а часть уже после фестиваля.

Премьерой также станет работа \enquote{Enter the Lab} режиссера, педагога и актрисы из
Швейцарии Мадлен Бонгард и четырех мариупольских актрис. Кроме того, будут
дискуссии, диалоги и кинопоказы. Всю программу IStage анонсируют только на
официальной странице \enquote{Маріуполь. Велика культурна столиця України}.

При этом, по словам Андрея Палатного, все события фестиваля для зрителя будут
бесплатны, единственное, что потребуется — предварительная регистрация онлайн,
поскольку количество мест ограничено. Мариупольцы при желании также смогут
регистрироваться на все спектакли.

Напомним, что также пройдет выставка художников второй резиденции \enquote{Маріуполь.
Місто з ідентичністю}.

\par\noindent\rule{\textwidth}{0.4pt}

Фото предоставлены Андреем Палатным
