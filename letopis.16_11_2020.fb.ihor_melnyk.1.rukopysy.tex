% vim: keymap=russian-jcukenwin
%%beginhead 
 
%%file 16_11_2020.fb.ihor_melnyk.1.rukopysy
%%parent 16_11_2020
 
%%url https://www.facebook.com/ihor.melnyk.svitpublishing/posts/2563749490584485
%%author 
%%author_id 
%%tags 
%%title 
 
%%endhead 

\subsection{Видання про унікальні рукописні шедеври України}
\label{sec:16_11_2020.fb.ihor_melnyk.1.rukopysy}
\Purl{https://www.facebook.com/ihor.melnyk.svitpublishing/posts/2563749490584485}
\Pauthor{Мельник, Ігор!Видавництво \enquote{Світ}}

\index[rus]{Книга!Рукописна!Видавництво \enquote{Світ}}

\ifcmt
pic https://scontent-waw1-1.xx.fbcdn.net/v/t1.0-9/125549240_2563748460584588_2757512922520356172_n.jpg?_nc_cat=100&ccb=2&_nc_sid=730e14&_nc_ohc=7GNVmY3bGd8AX82TMfd&_nc_ht=scontent-waw1-1.xx&oh=c0d7c13cfd155742340d36c8a4da8fc2&oe=5FDA7DF7
caption Українська рукописна книга - знахідка для поціновувачів раритетних видань
\fi

\enquote{Українська рукописна книга} - знахідка для поціновувачів раритетних
видань.

Її автор, світлої пам’яті доктор мистецтвознавства, професор Яким Запаско,
простежив понад восьмисотлітній шлях розвитку книжкового мистецтва в Україні.
Вчений-подвижник дослідив фонди найбільших книгозбірень України, Росії, побував
у монастирських бібліотеках Європи, вивчав матеріали у Північній Італії та у
Паризькій бібліотеці. Зібрав, ідентифікував, проаналізував величезну духовну
спадщину українського народу, переконливо довів, що багато національних скарбів
незаслужено забуті, пограбовано і привласнено представниками інших держав. Щоб
читач мав уявлення про те, що зроблено в описі рукописів, необхідно сказати про
докладно висвітлені теми, а саме: “Письмо”, “Оздоблення”, “Зміст”, “Мовні
особливості”, “Приписки”, “Історія рукопису” і “Література” (перелік подеколи
перевищує сто п’ятдесят назв). Автор зазначає: зовнішній вигляд і розміри
книги, формат, місце зберігання, шифр, кількість рядків на сторінці, кількість
аркушів, матеріал. Описано 128 манускриптів.

“Українська рукописна книга” набула широкого розголосу й популярності також
завдяки своєму високомистецькому вигляду. Автором макету і оформлення є
народний художник України, лауреат премії імені Тараса Шевченка Леонід
Андрієвський. На переконання Юрія Белічка, професора Української академії
мистецтв, заслуженого діяча мистецтв України, завдяки дизайнерському таланту
монографічна праця була виведена із рангу суто утилітарного,
науково-інформативного видання, призначеного для вузького кола фахівців, у ранг
мистецького… Відчуття урочистої святковості не полишає читача від першої до
останньої сторінки завдяки великим кольоровим ілюстраціям, які утворюють
своєрідну поліфонію образів, кольору, орнаменту, мальованих літер. Книга завжди
сприймається в просторі і часі й тому, здавалося б, психологічно важко
перегорнути усі її 478 сторінок. Та цього не трапляється завдяки мудро
продуманій архітектоніці макету, збагаченій, вишуканій ритміці чергування
ілюстрацій і знову таки завдяки наявності вільних від тексту полів. Не
випадково доктор мистецтвознавства Олександр Федорук сказав: “Якщо в науці є
місце для подвигу, то його в наші дні здійснив відомий львівський учений Яким
Прохорович Запаско”. Словом, це видання відповідає найвищим стандартам
художнього вирішення книги, завдяки чому гідно представляє доробок українського
народу в світі. Не випадково щорічне авторитетне журі фахівців книжки в Парижі
визнало «Українську рукописну книгу” (1995 р. ) найкращим виданням у нашій
державі а на Форумі видавців у Львові (1996р.) вона була удостоєне Гран-прі.
