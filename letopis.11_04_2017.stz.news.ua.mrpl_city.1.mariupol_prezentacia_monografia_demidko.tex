% vim: keymap=russian-jcukenwin
%%beginhead 
 
%%file 11_04_2017.stz.news.ua.mrpl_city.1.mariupol_prezentacia_monografia_demidko
%%parent 11_04_2017
 
%%url https://mrpl.city/news/view/v-mariupole-prezentuyut-illyustrirovannuyu-monografiyu-o-teatrah-goroda-foto
 
%%author_id ivanova_jana.mariupol,news.ua.mrpl_city
%%date 
 
%%tags 
%%title В Мариуполе презентуют иллюстрированную монографию о театрах города (ФОТО)
 
%%endhead 
 
\subsection{В Мариуполе презентуют иллюстрированную монографию о театрах города (ФОТО)}
\label{sec:11_04_2017.stz.news.ua.mrpl_city.1.mariupol_prezentacia_monografia_demidko}
 
\Purl{https://mrpl.city/news/view/v-mariupole-prezentuyut-illyustrirovannuyu-monografiyu-o-teatrah-goroda-foto}
\ifcmt
 author_begin
   author_id ivanova_jana.mariupol,news.ua.mrpl_city
 author_end
\fi

Об истории театральной культуры Мариуполя издана
\href{https://archive.org/details/book.2017.demidko_olga.monografia.illustrovana_istoria_teatr_kultury_mariupolja}{иллюстрированная
монография} историка \href{\urlDemidkoIA}{Ольги Демидко}, передает MRPL.CITY.

Труд охватывает период от становления театрального искусства в Мариуполе до
сегодняшнего дня. На 276 страницах книги читатель увидит более 200 фотографий
актеров, сцен спектаклей, афиш и программ.

Как рассказала MRPL.CITY автор монографии \href{\urlDemidkoIA}{Ольга Демидко}, театральное искусство
привлекало и завораживало ее с детства.

\enquote{Информацию для монографии я брала из разных источников. Был использован архив
драматического театра, личные архивы Светланы Отченашенко, Сергея Бурова,
Алексея Верещака, Народного театра ДК \enquote{Азовсталь} и Мариупольского
краеведческого музея. Также использованы театральные афиши, программы и
публикации в периодических изданиях}, - уточнила \href{\urlDemidkoIA}{Ольга Демидко}.

Книгу презентуют в Центральной библиотеке им. Короленко в четверг 13 апреля.
Монография вышла в рамках диссертации \enquote{Театр в культурно-мистецькому просторі
Північного Приазов'я (друга половина XIX – початок XXI ст.)} на средства
Британского совета, благодаря участии в программе \enquote{Активні Громадяни}.

Экземпляры монографии будут подарены краеведческому музею, драматическому
театру, самодеятельным театрам, упомянутым в книге, и библиотекам города.

Начало презентации в 15:00.
