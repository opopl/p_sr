% vim: keymap=russian-jcukenwin
%%beginhead 
 
%%file 13_02_2022.stz.news.ua.strana.1.vozduh_blokada.4.chto_pokazala_subbota
%%parent 13_02_2022.stz.news.ua.strana.1.vozduh_blokada
 
%%url 
 
%%author_id 
%%date 
 
%%tags 
%%title 
 
%%endhead 

\subsubsection{Что показала суббота?}
\label{sec:13_02_2022.stz.news.ua.strana.1.vozduh_blokada.4.chto_pokazala_subbota}

Итак, начатое в пятницу очень мощное нагнетание американцами темы \enquote{вторжения
России} в субботу продолжилось. При этом и Киев, и Москва вероятность скорой
войны опровергают.

Но их мнение уже никого не волнует, и Запад активно переводит панику на
\enquote{красный} уровень - когда что-то в ближайшее время точно должно произойти.
Причем может произойти что угодно - от военного конфликта или провокации до
дипломатического прорыва. 

Сценарии того, что может произойти, мы \href{https://strana.news/news/376303-ssha-zajavili-chto-rossija-napadet-15-fevralja-chto-proiskhodit.html}{вчера детально расписывали}. 

Пока ситуация не качнулась отчетливо ни к дипломатическому варианту, ни к
военному. Хотя уже появились идеи сделать Байдена \enquote{миротворцем} -
например, их озвучил близкий к Банковой и западным посольствам экс-нардеп
Сергей Лещенко. Он предложил президенту США в \enquote{день вторжения} (16
февраля) прибыть в Киев. 

Причем, это может быть вполне рабочая схема.

Ведь если Путин и не собирался нападать в реальности (а все говорит именно за
это), то Байден может прилететь 16 февраля в Киев, зафиксировать, что
\enquote{Путин испугался меня и не напал} и с триумфом, как человек
\enquote{остановивший вторжение}, улететь в Вашингтон.

Но это, если США планируют именно таким образом закрыть эту историю. А не
каким-то другим - например, через реальный военный конфликт Украины с Россией. 

Пока ни Киев, ни Москва не изъявляют желания такую войну начинать. Поэтому
главный нерв ближайших дней - не произойдет ли какой-то провокации, которая
должна будет вызвать войну. И кто именно эту провокацию устроит.
