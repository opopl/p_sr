% vim: keymap=russian-jcukenwin
%%beginhead 
 
%%file 05_03_2021.fb.molchanov_jurij.1.lvov_izbienie_mova.cmt.ftn.stranaua
%%parent 05_03_2021.fb.molchanov_jurij.1.lvov_izbienie_mova.cmt
 
%%url 
 
%%author_id 
%%date 
 
%%tags 
%%title 
 
%%endhead 
\footnote{%
Во Львове концерт уличных музыкантов на площади Рынок закончился их избиением
националистами.

На молодых людей, которые пели на русском языке, напала "языковая активистка",
но не добившись в словесной перепалке своего, ретировалась. После этого к
конфликту подключились молодые люди, которые избили музыкантов, а музыкальные
инструменты сломали.

Хронологию конфликта описала в своем посте журналист "Страны" и блогер Олеся
Медведева, прикрепив к посту видеозаписи из Instagram-сторис.

В них музыканты пишут, что их избили (предположительно - "Сокол", это
молодёжная организация партия "Свободы") и поломали инструменты.

Судя из видео, девушка, подписанная в соцсетях как Наталья Маркевич, пытается
словесно запретить молодым людям петь на улице песни на русском языке.

"Вы находитесь в общественном месте, вы нарушаете закон", - говорит девушка,
вероятно, считая, что закон как-то влияет на ограничение языков в общении на
улице. Музыкант на это пытается апеллировать к закону, но девушка снова и снова
повторяет "вы в общественном месте".

Наконец, собеседник говорит прямо: "Я русскоязычный? Я буду общаться на том
языке, на котором мне комфортно".

Позже девушка опубликовала фото разбитых музыкальных инструментов и сделала
обращение, в котором открыто призвала к насилию по отношению к "московским
с*кам"

"Может вы видели предыдущую сторис... Вы не представляете, но за прошлый день
это самая лучшая новость. Я пришла, открываю инстаграм, как бальзам на душу!
Это должно происходить не только во Львове, а в любом городе. С такими
московскими с@ками нужно делать подобное. Я не призываю к насилию, хотя нет,
призываю. Потому, что таких тварей нужно гнать, они не понимают
по-человечески", - говорит защитница языка на улицах.
}
