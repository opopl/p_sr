% vim: keymap=russian-jcukenwin
%%beginhead 
 
%%file 02_12_2021.fb.fb_group.story_kiev_ua.1.vodolazy_chernobyl
%%parent 02_12_2021
 
%%url https://www.facebook.com/groups/story.kiev.ua/posts/1810319312498195
 
%%author_id fb_group.story_kiev_ua,fedjko_vladimir.kiev
%%date 
 
%%tags avaria_na_chaes,chaes,chernobyl,vodolaz
%%title Водолази у вогні Чорнобильської аварії
 
%%endhead 
 
\subsection{Водолази у вогні Чорнобильської аварії}
\label{sec:02_12_2021.fb.fb_group.story_kiev_ua.1.vodolazy_chernobyl}
 
\Purl{https://www.facebook.com/groups/story.kiev.ua/posts/1810319312498195}
\ifcmt
 author_begin
   author_id fb_group.story_kiev_ua,fedjko_vladimir.kiev
 author_end
\fi

Водолази у вогні Чорнобильської аварії

З Віктором Лепнуховим я познайомився у 2007 році, коли обіймаючи посаду
заступника голови правління Видавничого Дому «Княгиня Ольга» випускав книжкову
серію «Спецслужбы в войнах ХХ века». Ми часто зустрічалися на різних заходах
нашої ветеранської організації – КГО «Товариство ветеранів розвідки
Військово-морського флоту». Неодноразово разом брали участь у виїзних
експедиціях. І тільки у 2011 році, напередодні 25-річниці аварії на
Чорнобильській АЕС, я дізнався, що Віктор був одним з активних ліквідаторів
наслідків цієї аварії. 

Після нашої «ювілейної» експедиції в «Зону», яка була здійснена по ініціативі
Віктора, я попросив його написати спогади про роботу водолазів на ЧАЕС у 1986
році. Віктор, дуже скромний чоловік, довго ухилявся від виконання мого
прохання. Я був настирний і, нарешті, мені все-таки вдалося розговорити його і
задокументувати дуже цінні спогади про той час. І отримати не тільки розповідь,
а й надзвичайно цінні його фотографії тих часів, які жодного разу не
публікувалися за 25 років.

Я вважаю необхідним поділитися цими спогадами і фотографіями Віктора Лепнухова
1986 року, а також з нашої поїздки в «Зону» 9 квітня 2011 року, з читачами
«Київських історій».

***

Віктор Лепнухов: У вогні Чорнобильського лиха

26 квітня 1986 року розрізало життя багатьох тисяч людей на «до» і все інше.
Так тоді і говорили: «це було до війни». У свої спогадах я хочу розповісти про
своїх друзів, з якими пройшов вогонь Чорнобильського лиха.

Коли вибухнув реактор Четвертого блоку, я дослужував строкову службу на
Північному флоті. В кінці травня, «зігравши ДМБ», прилетів до Києва. Оскільки
служив водолазом, влаштувався на роботу в «Київський експедиційний загін
підводно-технічних робіт ОСВОД УРСР». Пригадується О. Генрі – «Дороги, які ми
вибираємо». У Києві була достатня кількість водолазних організацій, але мене
занесло саме в ОСВОДовскій загін.

Петро Петрович Попов, водолазний фахівець підприємства, відразу попередив, що
треба буде їхати в Чорнобиль. Не треба було бути пророком – загін виконував
водолазні роботи на станції з 1978 року, з моменту запуску першого блоку.
Будь-яка електростанція має велике водне господарство. На ЧАЕС за воду
відповідав окремий підрозділ – гідроцех, до якого і були прикомандировані
водолази. Водолазні роботи досить специфічні, об'єкт на об'єкт не схожий,
скрізь є свої «хитрушки», тому було зрозуміло, що люди, які знайомі з об'єктом,
можуть працювати більш ефективно. Навесні в загін влаштувалося 4 «молодих»,
тільки зі служби. Попов попереджав кожного, хто залишився у загоні – прийняв
для себе рішення! Все по чесному! Діма Кіршенін, Валера Покрівець, Ігор
Синенко, мої побратими, які теж пройшли цю «війну».

Потрапив я на водолазну станцію, де старшиною був Серьога Марц, а другим
водолазом – Микола Кошмяков. Про кожного з цих хлопців, як і про інших моїх
товаришів, можна писати книги з серії ЖЗЛ. Сергій довгий час, до аварії,
керував водолазними роботами на ЧАЕС. А «Коша», так прозвали Колю, був взагалі
мало не корінним жителем Прип'яті – він був одружений на місцевій дівчині,
працівниці станції. Сам Коша був відомий в «широкому колу вузьких фахівців»
тим, що за часів своєї юності зіграв роль Котька Григоренко в телефільмі «Стара
фортеця».

Начальник загону – Бабенко Анатолій Трохимович, досить довго відбивався від
робіт в аварійній зоні, посилаючись на керівний документ – «Єдині правила..»,
що забороняють водолазам працювати в зоні радіоактивного забруднення. Але «як
мотузочці не витися...», був викликаний до Чорнобиля на засідання Урядової
комісії і на наступний день почалися нагальні збори і оформлення документів.

До групи першого кидка увійшли:

• Гасин Борис Григорович, головний інженер;

• Попов Петро Петрович, водолазний фахівець;

• Румянцев Юрій Володимирович, лікар-фізіолог;

• Марц Сергій Казимирович, водолаз 1 класу, старшина водолазної станції;

• Кошмяков Микола Володимирович, водолаз 2-го класу;

• Лепнухов Віктор Павлович, водолаз 3-го класу;

• Якубовський Анатолій Миколайович, водій.

Така ось команда. Вік від 20 до 50. Гасин – колишній командир БЧ-5 підводного
човна. Попов, в минулому, – водолаз-розвідник КТОФ. Наш штатний лікар –
Головань Володимир Іванович не зміг поїхати в це відрядження, знайшли Юрія з
Київського Морполіта. А решта проходили службу і набивали перші водолазні шишки
в ВМФ. Їхати – не їхати, питання не було. Теза одна: «Якщо не я, то хто ж». І
ніякої патетики, це було всередині кожного, вголос про це не говорили, треба
було захистити країну, тільки і всього.

Що говорити вдома – кожен вирішував самостійно (їхали в невідомість!). За
порадою «старших товаришів», я «поїхав на будівництво моста в м. Чернігів». Про
те, що робота в «Зоні» оплачується в п'ятикратному розмірі, а рік йде за три, –
ми дізналися тільки потрапивши на станцію.

Добралися до Страхолісся без пригод. «Білі пароплави», які прийшли на Прип'ять
з Волги, виглядали класно – оку було приємно, краса. Виділили нам каюти на
пароплаві «Узбекистан». Правда, в 3 класі, та і це добре, не до жиру.

Спорядження і обладнання вивантажили у вагончик, виділений нам. На Севрах такі
вагончики називають «балки». Чому я фокусую увагу на цьому вагончику, зараз
поясню. Ми також вивантажили туди гроші і документи (щоб не «забруднити» в
«Зоні»), але, крім цього, там був присутній і 40-літровий молочний бідон,
наповнений спиртом (пам'ятаєте такі бідони?). У зоні був «сухий закон», а
водолазам спирт по інструкції необхідний для дезінфекції. Таке ось багатство!
Яке ж було наше розчарування, коли повернувшись зі станції після першого дня
роботи і приготувавшись прийняти по парі крапель «радіопротектора», прописаного
нашим ескулапом, ми не виявили вагончик на місці. Велика робота – велика
плутанина, права рука не завжди знає, що робить ліва. Коли на вечірнє засідання
керівництва, з боєм увірвалася розгнівана бригада водолазів, настала німа
сцена, як у Гоголя. Оперативно, протягом 5 хвилин знайшлися сліди нашого
«втікача». Виявляється, поки ми були відсутні, наш «балок» перетягнули в інше
селище. Осідлавши КАВЗік, мчимо в погоню. Слава Богу – все на місці. Так що
вечеря вдалася!

* * *

Постійно переслідує спрага і дере в горлі. Організм перебудовується, а
електронні годинники виходять з ладу. Крім того, постійний присмак металу
(кислятина) в роті.

«Обезьяна встала очень рано, обезьяна съела три банана…». Не знаю, чому
радистам пароплавів сподобався І. Кармелюк, але під цю пісню ми прокидалися,
снідали і вирушали на роботу. Другий хіт – «Осінній лист» у виконанні М.
Караченцова, тут же перейменований нашими гострословами в «Дозиметрист, ты мне
среди зимы приснись». Начепивши на санпропускнику докторський ковпак і кілька
респіраторов- «пелюсток» вантажимося в автобуси. Проходить трохи часу шляху –
пересадка в «брудні» автобуси, завішені свинцевими листами. На них ми і
потрапляємо на станцію. Вивантажуємося у АПК (адміністративно-побутовий
корпус). До «війни» на АПК висіло гасло «Чорнобильська АЕС працює на комунізм».
Другу половину гасла прибрали, але залишився бюст Леніна перед входом (ЧАЕС
носила ім'я Леніна).

«Марчелло» (Марц) стає в «позу вождя» (права рука піднята у пориві, ліва
зачепила імпровізований жилет) і проголошує майже класичну фразу – «Нас
погубить радянська влада плюс електрифікація всієї країни». Віталі тоді і такі
настрої, але це був жарт, що розряджав обстановку, адже, по суті, ми потрапили
в самий центр драматичних подій.

* * *

При вході в будь-яку будівлю – ванна з водою і щітки – помий взуття, не розноси
«бруд» по приміщеннях. Вікна в АПК завішені свинцем.

Саркофаг ще не закрито, періодично йдуть простріли і викиди. Для роботи в
«брудній зоні» нашій бригаді виділяють автомашину ГАЗ-52 з будкою. На борту
напис «Аварійна водопроводу». Нескінченне поле діяльності для наших дотепників.
Видали і олівці-накопичувачі, і індикатори. Бувало таке, що ці накопичувачі
мужики закидали в бур'ян з ранку (там, де більше «світило») і забирали ввечері.
Цим грішили «партизани». Їх закликали через військкомат на перепідготовку і
раптом – «пожалтє голитися». Вони в зону не рвалися, їх ставили перед фактом.
Їм відводилася, як правило, сама чорна і брудна робота. І якщо добровольці
могли виїхати в будь-який момент, то «партизани» повинні були відпрацювати весь
термін, на який їх закликали. Із зони виводили тільки тих, хто нахапав 25 берів
(біологічний еквівалент рентгена). Ось і придумували всяке. І я не в праві їх
засуджувати.

* * *

Окремо хотів би сказати про війська хімзахисту. Вони були спрямовані в зону
аварії буквально в перші години. А на кого ще було сподіватися? Це їхня робота.
Вони займалися дозиметричною розвідкою, дезактивацією населених пунктів, стояли
в пунктах пропуску в зону. А після виведення цих частин, коли ситуація
«устаканилася» і необхідність в них відпала, частини ці були розформовані.
Велика кількість військовослужбовців, які брали участь в ліквідації аварії, і
до цього дня не мають ніяких пільг, покладених «ліквідаторам». Частини-то
немає, а «немає людини – немає проблеми», як говорив «батько народів».

* * *

Не можу не сказати ще ось про що. Хлопці берегли мене – «салажонка», як це було
колись на фронті. Коли доводилося працювати в Зоні, де «стріляло» серйозно,
вони знаходили причину відправити мене в більш чисте місце, якщо могли зробити
роботу самі.

* * *

Додали нашій групі і дозиметриста-розвідника з ДП-5. Хороший апарат для роботи
в зоні ядерного вибуху. А ще їм добре колоти горіхи. Але хоч такий в квітні був
за щастя. Поміряли рівень на схилах БНС (берегової насосної станції). Вона
закачує воду з Прип'яті в водне господарство ЧАЕС. «Світить» пристойно, а
працювати треба. Нашій групі поставлено завдання – провести обстеження і
з'ясувати стан ГТС (гідротехнічних споруд) станції.

Розкачуємо на схилі «сітку рабиця», накриваємо її зверху трьома шарами
листового свинцю. Місце для одягання водолаза готове.

З балкона БНС, там «чистіше», спускаємо шолом, там же знаходяться і транспортні
балони, з яких надходить повітря водолазу. Буквально бігом спускаємося вниз, в
хорошому темпі одягаємо водолаза, традиційний шльопок по шолому, водолаз – під
воду, а інші бігом наверх. Зайве випромінювання ні до чого!

***

«Марчелло» з-під води доповідає: «У прийомних камерах якась дурниця, схожа на
глину». Виносить наверх зразок. А хто його знає, що це таке. Поруч працюють
геологи, звертаємося з питанням до них. Геологи радіють, – це бентоніт, який
вони закачують під тиском в порожнечі тіла дамби і не можуть зрозуміти, куди
дівається цей розчин. Так, підкинули нам хлопці роботи – кілька водоприймальних
камер занесені цим матеріалом майже наполовину. Розчищати доведеться вручну.
Робота «каторжна», але куди дінешся? Щовечора, на санпропускнику, повертаючись
з роботи, отримуємо нові роби, оскільки наші, в яких відпрацювали один день,
«дзвенять» неймовірно

***

Їдемо після роботи, зупиняємося по «технічній» причині. Узбіччя залиті сумішшю
схожою на клей ПВА, щоб пил не розносився. Скрізь знаки – «узбіччя заражено,
вихід в ліс заборонений». Наш головний інженер, затятий грибник, знайшов в лісі
гриб з капелюшком більше півметра. У Попова витягується обличчя (хоча очі
сміються), – «Толик, тисни!». Водій різко тисне на педаль акселератора. Гасин
кілька секунд біжить за автобусом, потім розуміє марність своїх спроб. Автобус
гальмує, підбираємо нашого грибника, вже без гриба, і, по дорозі додому,
водолазний фахівець популярно пояснює «головному» чому саме для таких як він по
дорозі розставлені попереджувальні знаки.

* * *

За першу вахту встигли оглянути рибозагороджувач, підвідний канал, насосні
станції ВЗС. Станція готується до пуску першого блоку. З горем навпіл
виїжджаємо з «Зони» (миють автобус кілька разів), їдемо в Київ, на нашу базу.
Зазвичай базові робочі зустрічають всі водолазні станції, які повернулися з
відрядження. Допомагають розвантажитися, стандартні питання – як та що? А тут –
тиша, берегова база загону як вимерла. Ніде не душі! Ясна річ, адже ми в
«чорнобильських» робах, докторських ковпаках, в респіраторах. Вирішили
пожартувати. Дожартувалися… поховалися всі, бажаючих допомогти в розвантаженні
спорядження не спостерігається. Скидаємо «камуфляж», дістаємо з автобуса
залишки «шила» (спирт з ...); здається, нас впізнали, і починає підтягуватися
народ.

* * *

Оскільки режим роботи на станції був вахтовий, то у нас був час трохи
відпочити. Режим роботи 15х15. Але багато часу для відпочинку у нас немає,
проаналізувавши ситуацію, треба готуватися до наступної поїздки. Водолазний
фахівець, Попов Петро Петрович, світла голова і золоті руки, придумав два
фільтри для водолазної помпи. Тепер ми виграємо багато часу – нам не треба
виїжджати в «чисту» зону, щоб «забити» балони, ми можемо працювати прямо на
об'єкті. Починаємо збиратися в чергове відрядження. В нашій команді легка
ротація – наш штатний лікар-фізіолог, Головань Володимир Іванович, їде з нами.
Юра-лікар зник з мого поля зору. Довгий час не знав, як склалося його життя. І
ось недавно дізнався, що він став доктором медичних наук, займає високий пост.
Щиро радий за нього. Печевистий Саша, водолаз, студент, секретар комсомольської
організації загону теж в нашій групі.

* * *

Уже жовтень, вранці підморожує. Знайомі пароплави, обслуговуючий персонал,
пізнаючи нас, радо посміхається. Знайомі пісні по корабельній трансляції. При
в'їзді в «зону» краєм ока спостерігаю за Сашком, він сидить на сусідньому
сидінні. Скільки разів він проїжджав цією дорогою «до війни». Прип'ять був
одним з найбільш «молодих» міст в Радянському Союзі. Середній вік жителів – 26
років. І архітектори при будівництві попрацювали на славу. Саша дуже любив це
місто. Обличчя закрито респіратором, а в очах напруга і біль. Покинуті села і
міста. Величезні яблука, від яких ламаються гілки дерев. Свіжовипрана в квітні
білизна, яка сушиться у дворах і на балконах досі. Уже напівдикі собаки
шастають в пошуках їстівного. Перше знайомство з «Зоною» не може залишити
байдужим. Моторошне видовище, не дарма інтелектуали відразу провели паралель з
Тарковським і братами Стругацькими. Але це все відбувається з нами, а не на
екрані. Схоже, такі ж обличчя були і у нас місяць тому.

* * *

За час нашої відсутності виступила з концертами Алла Пугачова. Багато
співробітників станції показують автограф Алли Борисівни на чепчиках. А в
цілому ситуація стабілізується, ось-ось закриють «саркофаг». Оскільки загальна
картина піднімається в бік «прояснення», бюрократична машина починає набирати
обертів. Якщо раніше для вирішення питань вистачало 10 хвилин, то зараз може
піти і півдня, та ще з невідомим результатом. Охорона не може пропустити нас на
об'єкт – не вистачає якоїсь печатки.

* * *

Начальник гідроцеху Бузирев, пропонує нам новий транспорт – катер БМК-130, в
просторіччі іменований «бомбиль», кинутий кимось на березі за непотрібністю.
«Світить» він, правда, багато вище фону, зате виграємо в темпі пересування –
прямо через ставок-охолоджувач на БНС-3 (наше основне місце роботи). Сергій
Марц проводить огляд і приходить до висновку, що техніка в робочому стані,
паливом заправлений, можна використовувати з усім задоволенням.

Вантажимося на катер і в путь. Основні роботи проводяться на третій береговій.
У рідкісний момент відпочинку знайшли недалеко від місця робіт держаки від
лопат, розпиляли їх на цурки, пробило, однак, і почали грати «в городки».
Безпосереднє керівництво наше з самого початку не схвалювало подібного заняття,
але потім підключилося до гри. Психологічне навантаження треба знімати.

***

Залишили, як-то, Кошу охороняти розгорнуте спорядження, а самі вирушили на
обід. Повернувшись, знаходимо Колю в досить «задумливому» стані. Виявляється,
за час нашої відсутності, з'явився незнайомий чоловік в «афганці» (таку форму
носили керівники вище середньої ланки), і представившись співробітником «саме
тієї служби» (пропуски-посвідчення на грудях невеликі, хто його розбере, що там
написано?), поцікавився думкою професійного водолаза перспективою теракту на
станції, якщо диверсант-водолаз зайде з нейтральних вод Чорного моря і
підніметься під водою по Дніпру до Прип'яті. Коля, добра душа, швидко розніс в
пух і прах цю ідею-версію, і запропонував свій ефективний засіб вирішення
проблеми «малою кров'ю з використанням підручних матеріалів». У співрозмовника
витягнулося обличчя і, подякувавши Кошу, він швидко пішов. Ось тепер Коля і
гадає, що чи не «засланий це козачок»? 

(Для довідки: Коліна пропозиція могло привести до тих же наслідків, що нині на
Фукусімі).

* * *

Крім нашого загону брали участь у ліквідації колеги – водолази з інших
організацій Києва. Вони працювали в 30-кілометровій зоні: займалися обваловкою
Прип'яті, будували водозабори і водовипуски для вахтових селищ, прокладали
кабелі.

Сили загону, згодом, стали танути. Співробітників, які набрали річну допустиму
дозу, виводили із зони. Ми продовжували виконувати роботи на станції протягом
двох років, потім було прийнято рішення створити свою водолазну службу на ЧАЕС.
Пішли туди працювати С. Марц і Н. Кошмяков. До них приєднався і молодший брат
Печевистого – Сергій.

***

* * *

На сьогоднішній день [весна 2011 року. – В. Ф.] пішли з життя 7 наших
співробітників-ліквідаторів. 

На початку квітня цього року, ті, хто нині живе, відвідали ЧАЕС, проїхали по
знайомих до болю місцях. Мало нас залишилося. Вічна пам'ять тим, хто не
шкодуючи свого життя, встали на боротьбу з мирним атомом. І дай Бог здоров'я
живим.

Віктор Лєпнухов, травень 2011 року.

***

P.S.

До розповіді Віктора додаю фотографії 1986 року з його сімейного архіву та
фотозвіт з \enquote{ювілейної} експедиції в «Зону» (9 квітня 2011 року) членів КГО
«Товариство ветеранів розвідки Військово-Морського Флоту».

***
\ii{02_12_2021.fb.fb_group.story_kiev_ua.1.vodolazy_chernobyl.cmt}
