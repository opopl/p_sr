% vim: keymap=russian-jcukenwin
%%beginhead 
 
%%file 04_01_2022.yz.figurka_ot_maksima_jagudina.1.chempionaty_evropy.7.ostrava_2017
%%parent 04_01_2022.yz.figurka_ot_maksima_jagudina.1.chempionaty_evropy
 
%%url 
 
%%author_id 
%%date 
 
%%tags 
%%title 
 
%%endhead 
\subsubsection{Чемпионат Европы 2017 (Острава)}

\textbf{Каролина Костнер} пропустила два предыдущих чемпионата Европы, а вернувшись
показала высокий класс и завоевала юбилейную 10-ю европейскую медаль в своей
карьере.

\ii{04_01_2022.yz.figurka_ot_maksima_jagudina.1.chempionaty_evropy.7.ostrava_2017.pic.1}

16-летняя \textbf{Маша Сотскова}, проводившая свой первый международный
взрослый сезон, не справилась с волнением и дважды упала в произвольной
программе, а претендовать на медаль она могла только при чистых прокатах. На
последующем Чемпионате Мира Маша займет достойное 8 место и, на фоне неудачного
выступления Анны Погорилой, поможет сохранить три квоты у женщин на Олимпийский
сезон.

\ii{04_01_2022.yz.figurka_ot_maksima_jagudina.1.chempionaty_evropy.7.ostrava_2017.pic.2}

18-летняя \textbf{Анна Погорилая} продолжила серию уверенных прокатов - это был
лучший Чемпионат Европы в ее карьере, а перед этим она стала третьей в финале
Гран-при. Кто бы мог подумать, что спустя несколько месяцев Анну будет ждать
катастрофа на Чемпионате Мира, где она в произвольной программе сделает лишь
два тройных прыжка на плюсы и займет общее 13 место. Анна попробует найти в
себе мотивацию и остатки здоровья кататься дальше, но ее карьера закончится
после страшного проката на Скейт Канада, пересматривать который думаю вряд ли
кто-то решится.

\textbf{Евгения Медведева} в Остраве продолжила свою серию побед на турнирах,
которая достигла уже цифры 9. Последовавший Чемпионат Мира лишний раз показал,
что равных Евгении на протяжении двух последних сезонов просто никого нет.
