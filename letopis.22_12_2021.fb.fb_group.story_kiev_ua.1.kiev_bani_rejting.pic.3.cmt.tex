% vim: keymap=russian-jcukenwin
%%beginhead 
 
%%file 22_12_2021.fb.fb_group.story_kiev_ua.1.kiev_bani_rejting.pic.3.cmt
%%parent 22_12_2021.fb.fb_group.story_kiev_ua.1.kiev_bani_rejting
 
%%url 
 
%%author_id 
%%date 
 
%%tags 
%%title 
 
%%endhead 

\figCapA{3. Центральные бани на Малой Житомирской, 3-А.}

\iusr{Тамара Нарижная}
я бывала в этой бане в детстве вместе с мамой

\iusr{Валентина Зражевская}

Моя любимая баня, там было несколько парных, хотя и очень тесных, большая
душевая, много тазиков, я никогда ими не пользовалась, а этажом выше было
типа Кафе, наверное раньше там и были « нумера». А ещё там была парикмахерская
и замечательный мастер Наум, к нему нужно было записываться за пару недель.
Ходили в баню, старались, каждую неделю, не меньше 3-х раз в месяц. А потом
чистеньких и румяных фотографировали фотографы на Майдане. Это в 80 годы.

\iusr{Natali Slobodyanyuk}

Квартира нашей семьи была на Софиевской, а бани были через стенку! Но я ничего
не помню, а только воспоминания старших сестер! Теперь уже никто не
расскажет))))

\iusr{Белякова Людмила}

А мы ходили в баню возле парка Пушкина! Сейчас там банк.

\iusr{Татьяна Ткаченко}
\textbf{Белякова Людмила} 

Я жила на ул. Полевая и мы тоже ходили в баню напротив киностудии им. Довженко.
Замечательная была банька, мы ее вспоминаем до сих пор! @igg{fbicon.thumb.up.yellow} 

\iusr{Alexander Krivoshapka}
\textbf{Белякова Людмила} 

тоже ходили в номер с минибассейном. А не заметить ее было тяжело, потому что
вверху была большая надпись: \enquote{ЛАЗНЯ}

\iusr{Белякова Людмила}
\textbf{Alexander Krivoshapka} 

мы ходили в общую всей семьей каждую субботу. А потом шли через Пушкинский
парк. Там была шашлычая!!! Мы, простояв очередь, ели шашлыки! Для меня это был
праздник!!! Хоть их жарили в красном перце, но это было так вкусно!!! В баню
ходили до 1972 года пока квартиру получили. Какое это было время!!!

\iusr{Татьяна Трофимова}
И мы сюда ходили с мамой в детстве  @igg{fbicon.smile} 

\iusr{Тамара Яцык}
Где это? Малая Житомирская, что ли?

\iusr{Ирина Чикалова}
В эти бани - в экзотические парные с вениками - водили иностранных гостей

\iusr{Boris Gluzman}

Мало Житомирская 5. Строили эту баню после войны пленные немцы. Внизу были
персональные отделения. Большие общие отделения с лавочками на двоих, цинковыми
тазиками. Со двора примыкает к ней выразительно. Я жил на Мало Житомирской
номер 3.

\iusr{Boris Gluzman}
не выразительно, а вытрезвитель.

\iusr{Alex Gluzman}
Тазики называли шайками.

\iusr{Nina Ivanco}

Баня в которую ходила наша семья каждую субботу, когда жили на Стрелецкой.
Яркие воспоминания детства.

\iusr{Леся Сагайдачная}

Косметолог и мастер педикюра тут мои работали) и Сергей, парикмахер мужа
