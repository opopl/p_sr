% vim: keymap=russian-jcukenwin
%%beginhead 
 
%%file 24_08_2021.fb.bilchenko_evgenia.1.den_nezavisimosti_strana
%%parent 24_08_2021
 
%%url https://www.facebook.com/yevzhik/posts/4180457488655996
 
%%author Бильченко, Евгения
%%author_id bilchenko_evgenia
%%author_url 
 
%%tags bilchenko_evgenia,nezalezhnist,prazdnik
%%title БЖ. День рождения - грустный праздник.
 
%%endhead 
 
\subsection{БЖ. День рождения - грустный праздник.}
\label{sec:24_08_2021.fb.bilchenko_evgenia.1.den_nezavisimosti_strana}
 
\Purl{https://www.facebook.com/yevzhik/posts/4180457488655996}
\ifcmt
 author_begin
   author_id bilchenko_evgenia
 author_end
\fi

БЖ. День рождения - грустный праздник.

С тридцатилетием тебя, страна. Ты младше меня на десять лет, но очень много
успела. Потерять весь индустриальный сектор. Перечеркнуть социальные льготы
беднякам. Полностью отказаться от своей конституции, поправ права половины
своих граждан. Уничтожить много тысяч своих граждан разными способами: войной,
тюрьмой, травлей, забвением. Забыть всю историческую память и подменить ее
новой, рекламной. Удалить из гуманитарной сферы всю культуру, а из науки - всю
теорию. Отправить тысячи своих крестьян в гастарбайтеры и тысячи менеджеров - в
грантоеды. Потерять всякую независимость, став винтиком в машине внешнего
управления. Возненавидеть своего соседа и всех своих детей, не возжелавших его
ненавидеть.

\ifcmt
  pic https://scontent-cdt1-1.xx.fbcdn.net/v/t1.6435-9/240417962_4180457455322666_3621277209535605135_n.jpg?_nc_cat=106&_nc_rgb565=1&ccb=1-5&_nc_sid=8bfeb9&_nc_ohc=YQbThRZZm08AX-naDps&_nc_ht=scontent-cdt1-1.xx&oh=385a2ba32a3031e807c5e852d8d00ca9&oe=61496A8E
  width 0.4
\fi

Ты много успела, страна.

И с тобой у нас сложились весьма тяжёлые личные отношения. Когда-то до 2014 ты
была довольно добра ко мне. Ты дала мне золотую медаль, красный диплом, самого
молодого кандидата в двадцать шесть и самого юного доктора и профессора в
тридцать два, десятки изданных книг, сотни слушателей на сцене, шлейф студентов
в аудитории, весёлых интересных друзей. В роковом 2014 ты вступила на путь
забвения памяти, подростково топнув ногой по своей же груди. И, пока ты,
страна, подрывалась на собственной растяжке, я, как могла, была с тобой и
поддерживала тебя в самоубивании. 

Пока могла. Дальше стало невыносимо. И вот, страна, ты последовательно лишила
меня всего, чем я жила эти годы: науки, студенчества, кафедры, сцены,
репутации, студентов, друзей. В насмешку ты оставила меня живой, чтобы соседу
нечего было показывать по новостям и чтобы я дольше мучалась. В итоге, страна,
я отказалась от эмиграции в царство, которым ты одержима и которое предлагало
мне себя само, - Европа. Вот тогда, страна, я впервые ощутила, что ты - не я,
что я сильнее твоих фантазий. Наверное, русские действительно не сдаются,
что-то в этом есть.

Я знаю, страна, что я в меньшинстве. В отличие от твоей загрызающей саму же
себя оппозиции, у меня нет иллюзий относительно абстрактного выражения "народ в
оккупации". Нет никакой оккупации. Мы сами породили своих чудовищ, страна, и
сами же отравили их ядом своих отцов и детей. И даже моя мать, страна, цитируя
Маяковского, допытывается, почему я не отмечаю день флага.

А я, страна, не могу отмечать твоих дней. И день конституции для меня - это
навсегда день смерти деда. День же флага - день рождения матери. И всё. И, да,
я не буду смотреть парад. Наверное, когда-то я снова смогу тебя любить и выйти
с цветами на площадь, где будут портреты убитых деток. Наверно. Пока я хожу по
чужим улицам, курю чужие сигареты, пью чужой кофе, захожу в чужие кофе дать
чужим бариста чужие купюры на чай, зарабатываю чужие деньги, забираю чужую
трудовую из чужого места работы и смотрю в чужие глаза некогда родных людей,
которые говорят мне, потомку оседлых семи поколений в твоей утробе, что я -
пришлая. Так они говорили Югу, Востоку и ещё половине своих сограждан на Западе
и в Центре. И те поверили, страна.

С тридцатилетием тебя, детка. Я не знаю, что тебе пожелать. Ты превзошла все
ожидания.

\ii{24_08_2021.fb.bilchenko_evgenia.1.den_nezavisimosti_strana.cmt}
