% vim: keymap=russian-jcukenwin
%%beginhead 
 
%%file slova.zagadka
%%parent slova
 
%%url 
 
%%author_id 
%%date 
 
%%tags 
%%title 
 
%%endhead 
\chapter{Загадка}
\label{sec:slova.zagadka}

%%%cit
%%%cit_head
%%%cit_pic
%%%cit_text
Где в этом уравнении американцы и англичане, которые входили в антигитлеровскую
коалицию - \emph{загадка}. С таким же успехом можно говорить, что Советский Союз
принимал участие в операции "Оверлорд", когда англо-американские союзники
высадились в Нормандии.  Однако, если вспомнить празднование годовщины высадки
в Нормандии - на 75-летний юбилей летом 2019 года не позвали ни Путина, ни
Зеленского.  Да, разумеется, Советский Союз поддерживали по ленд-лизу те же
американцы и британцы. Однако это была взаимная история: именно Красная армия
на востоке сделала возможной высадку союзников на западе, отвлекая на себя
основные силы гитлеровцев и уничтожая их.  Причем, к тому моменту как
англо-американцы высадились в Нормандии, большая часть Украины уже была
освобождена
%%%cit_comment
%%%cit_title
\citTitle{Высадка союзников в Ужгороде. Кого Зеленский и Шмыгаль поблагодарили за освобождение Украины от нацистов}, 
Анна Копытько, strana.news, 28.10.2021
%%%endcit
