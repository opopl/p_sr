% vim: keymap=russian-jcukenwin
%%beginhead 
 
%%file 31_07_2021.fb.gluzman_semen.psihiatr.dissident.1.palachi
%%parent 31_07_2021
 
%%url https://www.facebook.com/permalink.php?story_fbid=796468544384136&id=100020629927113
 
%%author 
%%author_id gluzman_semen.psihiatr.dissident
%%author_url 
 
%%tags golodomor,istoria,kommunizm,palach,pamjat,ukraina
%%title СОЦСОРЕВНОВАНИЕ ПАЛАЧЕЙ
 
%%endhead 
 
\subsection{СОЦСОРЕВНОВАНИЕ ПАЛАЧЕЙ}
\label{sec:31_07_2021.fb.gluzman_semen.psihiatr.dissident.1.palachi}
 
\Purl{https://www.facebook.com/permalink.php?story_fbid=796468544384136&id=100020629927113}
\ifcmt
 author_begin
   author_id gluzman_semen.psihiatr.dissident
 author_end
\fi

СОЦСОРЕВНОВАНИЕ ПАЛАЧЕЙ.

В архиве Службы Безопасности Украины хранятся различные документы периода
сталинских репрессий. В их числе и такие, самые откровенные, самые
отвратительные – о вызове на социалистическое соревнование одним областным
Управлением НКВД другого. Речь там шла о планировании репрессий. Одни славные
чекисты обещали обезвредить конкретное число диверсантов, троцкистов, шпионов и
других врагов советского народа, а другие, в соседнем Управлении НКВД обещали в
конкретных группах врагов превысить показатели соседей…

\ifcmt
  pic https://scontent-cdt1-1.xx.fbcdn.net/v/t1.6435-9/227478241_796468491050808_8209089173793236394_n.jpg?_nc_cat=105&ccb=1-3&_nc_sid=8bfeb9&_nc_ohc=ojNRRrJEW74AX8Ix8AB&_nc_ht=scontent-cdt1-1.xx&oh=4c11f519544e14631fe9357278df5f49&oe=612CB105
  width 0.4
\fi

Страшное, дикое у нас прошлое. Вероятно, были социалистические соревнования и в
истреблении голодом крестьян Украины, Кубани и Кавказа. Американский
исследователь Голодомора Мейс назвал украинское общество постгеноцидным. Но
даже в относительно мягкий брежневский период санкции режима за распространение
сведений об уже далеком по времени Голодоморе были максимальными. Как правило,
такие «деяния» рассматривались советским идеологическим аппаратом как
проявление тяжелой психической болезни. В 70-ые годы прошлого века эта тема
была запретной и в среде коммунистической номенклатуры и офицеров КГБ. Это
прошлое советская власть скрывала и от своих.

В 1974 году в политическом лагере ВС 389-35, где я проводил своё несвободное
время, состоялась дискуссия, начало которой положил украинский диссидент Евген
Пронюк. Он предложил всем нам, активно сопротивляющимся ущемлению наших прав
администрацией зоны,  привычно реагирующим однодневными голодовками протеста на
всевозможные советские праздники, съезды КПСС и т.п., провести массовую акцию с
протестной голодовкой в память жертв Голодомора. Мнения наши разделились. Мы
спокойно принимали реакцию КГБ и администрации зоны на наши акции в виде
водворения в карцер, лишения права на свидания с родственниками и др., но в
этот раз мы задумались. Дискуссия была короткой, так как наши надзиратели
нервно спешили разогнать это наше антисоветское сборище. Главные слова успел
произнести наш мудрый Иван Алексеевич Свитлычный: «Цена за эту акцию будет для
нас очень дорогой. Когда мы протестуем по привычным для наших карателей
поводам, мы отделываемся сравнительно легкими наказаниями. Но в этом случае, я
уверен, будет жесткая реакция. А ведь мы не просто отбываем срок, мы – боремся.
Наша зона продуцирует много текстов, эти тексты читает весь цивилизованный мир.
Мы оставляем в истории память о наших стариках, отбывающих 25-летний срок
наказания. О солдатах УПА, «лесных братьях» из балтийского сопротивления. Мы
оставляем след в истории о себе. Глузман ведет ежедневную хронику о событиях в
зоне. И всё это прекратится, нас разбросают по другим лагерям, многих отправят
во Владимирскую тюрьму. Готовы ли мы платить такую цену за один день
символической голодовки протеста?»  Мы приняли аргументы Ивана Алексеевича.
Даже Евген Пронюк.

Сегодня мы живем в другом государстве. Правда о Голодоморе открыта. И не так уж
важно, что некоторые наши сограждане яростно отрицают, что Голодомор в Украине
был геноцидом. Имеют право, мы живем в демократической стране. Мы, советские
политические узники сделали для этого всё возможное. Некоторые из нас отдали
тоталитарному монстру свои жизни. Сегодня мне трудно принять аргументы этих
яростных последователей чужой правды. Вероятно потому, что я не историк, не
юрист.

Главное очевидно: советская власть загубила миллионы жизней. Сознательно,
целенаправленно, безнаказанно. Сын украинских крестьян Иван Свитлычный знал
это. Но тогда, в лагере он не думал о прошлом. Он хотел сохранить наши жизни и
возможность продолжения нашего лагерного сопротивления. Мудрый, добрый и
безукоризненно честный Иван Алексеевич. Трудно жить без него.
