% vim: keymap=russian-jcukenwin
%%beginhead 
 
%%file 19_12_2021.yz.maj_dnr.1.donbass_zavody
%%parent 19_12_2021
 
%%url https://zen.yandex.ru/media/id/5f8f226b1fe36c1d9e02a36b/v-donbasse-vse-zavody-porezali-na-metallolom-61bf710df8a3fd24528e7fdb
 
%%author_id 
%%date 
 
%%tags 
%%title В Донбассе все заводы порезали на металлолом?
 
%%endhead 
\subsection{В Донбассе все заводы порезали на металлолом?}
\label{sec:19_12_2021.yz.maj_dnr.1.donbass_zavody}

\Purl{https://zen.yandex.ru/media/id/5f8f226b1fe36c1d9e02a36b/v-donbasse-vse-zavody-porezali-na-metallolom-61bf710df8a3fd24528e7fdb}

Да, да! Так говорят и пишут украинские СМИ, правда, никогда не приводя в
доказательство хотя бы фоточку, но в это свято верят «пересичные громадяне»,
наверное, их утешает мысль о том, что не в одной Украине разрушена экономика,
мол, на фоне Донбасса Украина еще ого-го!.

\ii{19_12_2021.yz.maj_dnr.1.donbass_zavody.pic.1}

Вот этому вот «на фоне» однажды я посвящу отдельную статью и расскажу, почему я
считаю это слово обозначением слабости и поражения, но сейчас не об этом.

Во время недавнего десанта представителей российского бизнеса в Донбасс я
увязалась в одну из ознакомительных поездок с одной из делегаций. Вообще,
маршрутов оказалось несколько, но у меня был свой интерес: в этом году я была
на ДМЗ, куда очень хотела попасть еще раз, настолько огромное впечатление на
меня произвела струя раскаленного металла и вообще, весь завод. Но на видео,
выложенное тогда на мой канал в Ютубе, набежали полчища анонимных комментаторов
в стиле «Это все сказки и подделка», да так, что имея определенные претензии к
этой площадке по части ограничения или блокировки видео из Донбасса, я искренне
порадовалась тому, что Ютуб не пропускает особо пакостные выражения.

Короче говоря, я выбрала поездку на 4 предприятия: «Макеевский металлургический
завод», «Донецкий металлургический завод», «Донбассстандартметрологию» (где я
не была лет 10) и завод холодильников «Норд-Фрост». Так что теперь я могу
познакомить вас с этими предприятиями. Сегодня я не буду приводить вам цифры и
факты, впечатлять вас объемами и суммами, скажу только, что большинство наших
крупных предприятий сейчас находятся не просто на низком старте, а уже набирают
портфели заказов от российского бизнеса. И это лично в меня вселяет огромные
надежды на будущую большую работу для меня лично – пусть вам это не покажется
наглостью, но я хочу своими глазами увидеть и своим видоискателем запечатлеть,
как возрождается былая слава промышленного Донбасса.

\headTwo{Итак, первым был ММЗ – Макеевский металлургический завод имени С. М. Кирова.}

Основан в 1898 году, интересно, что некоторое время он назывался Завод «Юнион»,
а наш лучший телеканал «Юнион» тоже родился в Макеевке и такое имя каналу дали
не зря. Имеет богатейшую историю одного из самых крупных металлургических
заводов России и СССР – например, в 1940-е годы там выплавляли чугуна в два
раза больше, чем все метзаводы Польши вместе взятые, а в 1966 году завод был
награжден Орденом Ленина.

Но с 1991 года его слава пошла на убыль, но оставался в числе тех, кто приносил
Украине валютную выручку, и немалую. Правда, отданный украинской властью Ринату
Ахметову по хитрой схеме, завод избавился от всех социальных обязательств –
детсады, больницы, библиотеки, ДК, школы, словом, все, что не приносило
олигарху дохода, было навязано городской администрации, но то такое…

Война, как и экономическая и транспортная блокада, организованная Украиной в
2017 году, едва не добили его, но завод выстоял и работает, хотя и не во всех
цехах вы увидите рабочих. Но главное – благодаря открывшимся возможностям по
Указу Путина №657 завод может заработать в полную мощность.

Видео моего путешествия по заводу вот:
