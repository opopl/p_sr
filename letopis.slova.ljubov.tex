% vim: keymap=russian-jcukenwin
%%beginhead 
 
%%file slova.ljubov
%%parent slova
 
%%url 
 
%%author 
%%author_id 
%%author_url 
 
%%tags 
%%title 
 
%%endhead 
\chapter{Любовь}
\label{sec:slova.ljubov}

У, здавалося б, такій абсолютно політичній категорії як незалежність є
абсолютно чуттєва сторона – це \emph{любов}. Доброзичливість та \emph{любов}
приходить в життя звільнених від приниження та гноблення людей. Принаймні так
мало б бути.  Проте значна частина власників українських паспортів відчувають
взаємну \emph{нелюбов} та непотрібність у власній країні. Якщо бути об’єктивним
і більш точним: не люблять нас, але \emph{не любимо} і ми.  У незалежності
також є естетична сторона – це краса. Ми збудували некрасиву країну. І ми
розуміємо, що некрасиве виходить з \emph{нелюбові}. В українській мові є два
слова, котрі відображають естетичне: гарний і красивий. Перше більше
характеризує створене природою, а інше – створене людиною. Наприклад, жінка
може бути гарною, але не красивою. Країна у нас гарна, але не красива,
\citTitle{Нелюбов та некраса української незалежності}, Анатолій Якименко, analytics.hvylya.net, 10.06.2021

Про яку \emph{любов} ми говоримо. Не можна сказати, що тема \emph{любові} є
панівною в суспільному дискурсі, але заклики до \emph{любові} в Україні можна
почути часто. Вони лунають звідусіль: з церков, семінарів, тренінгів,
соціальних та політичних бордів й так далі. Бувають періоди, коли здається, що
у нас працює \emph{«любовна»} армія, яка складається з священиків різних
конфесій, коучів, менторів, фасилітаторів, лідерів громадської думки і,
звичайно, політиків. Серед останніх беззаперечним лідером тут є Юлія
Володимирівна. Згадаймо хоча б її промову на молитовному сніданку в США у
лютому 2018 року.  Отже, про що говорять всі ці люди? Безперечно, вони говорять
про \emph{безумовну любов}. Принаймні намагаються говорити саме про неї. І з
цього місця уже починаються проблеми, 
\citTitle{Нелюбов та некраса української незалежності}, Анатолій Якименко, analytics.hvylya.net, 10.06.2021

%%%cit
%%%cit_pic
%%%cit_text
Очень искренно и нежно. И мне кажется, что это одна из лучших песен о маме и
\emph{материнской любви}. И нельзя пройти мимо семейного дуэта мамы и дочки - Нины и
Тони Матвиенко, которые по своему не спели, а рассказали эту песню. Очень
нежно, глубоко и сердечно:
%%%cit_title
\citTitle{5 задушевных украинских песен, которые пели наши родители, а теперь поем мы}, 
Кино Вояж И Не Только, zen.yandex.ru, 07.06.2021
%%%endcit

%%%cit
%%%cit_head
%%%cit_pic
%%%cit_text
Помню ли я, простой горловчанин, 27 июля 2014 года? А помните ли вы, как
впервые поцеловались? Так вот, свой \enquote{поцелуй войны} я вряд ли когда-то забуду.
Кровавые коридоры городской хирургии, горящее админздание крупнейшей
энергетической компании Донбасса, несущие на руках окровавленных детей люди в
камуфляже, истошные крики женщин над телами погибших и ушедший за полчаса в
подвалы город. Достойное свидание со Смертью? Была ли ненависть? Была. Это со
временем она растворилась в анестезии рассуждений, осознании войны и
привыканиии к \enquote{труду} артиллеристов. Город, рожденный жить (как и тысячи других
мирных городов на планете), впервые ощутил то, как иногда государство может
страстно \enquote{\emph{любить}}
%%%cit_comment
%%%cit_title
\citTitle{Gorlovka.ua: Город, рожденный жить - Блоги}, 
Егор Воронов, gorlovka.ua, 29.07.2016
%%%endcit
