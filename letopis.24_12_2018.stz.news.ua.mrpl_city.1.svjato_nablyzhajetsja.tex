% vim: keymap=russian-jcukenwin
%%beginhead 
 
%%file 24_12_2018.stz.news.ua.mrpl_city.1.svjato_nablyzhajetsja
%%parent 24_12_2018
 
%%url https://mrpl.city/blogs/view/svyato-nablizhaetsya
 
%%author_id demidko_olga.mariupol,news.ua.mrpl_city
%%date 
 
%%tags 
%%title Свято наближається...
 
%%endhead 
 
\subsection{Свято наближається...}
\label{sec:24_12_2018.stz.news.ua.mrpl_city.1.svjato_nablyzhajetsja}
 
\Purl{https://mrpl.city/blogs/view/svyato-nablizhaetsya}
\ifcmt
 author_begin
   author_id demidko_olga.mariupol,news.ua.mrpl_city
 author_end
\fi

Незабаром настане найбільш очікуване свято для всіх дітлахів і більшості
дорослих, для тих, хто продовжує вірити в дива і любить дарувати подарунки.
Новий рік – це свято, яке, здається, об'єднує всіх. До заповітної 12 години
ночі, як правило, ми намагаємося виконати безліч завдань. Наприклад, знайти
мандарини та \enquote{вижити} в супермаркеті, порадувати близьких і себе.

\ii{24_12_2018.stz.news.ua.mrpl_city.1.svjato_nablyzhajetsja.pic.1}
\ii{24_12_2018.stz.news.ua.mrpl_city.1.svjato_nablyzhajetsja.pic.2}
\ii{24_12_2018.stz.news.ua.mrpl_city.1.svjato_nablyzhajetsja.pic.3}

Але так було далеко не завжди. Раніше в Україні головним зимовим святом було
Різдво. Наприкінці 1920-х років Різдво заборонили. У грудні 1935 року соратник
Сталіна Павло Постишев опублікував у газеті \enquote{Правда} статтю, де
запропонував повернути дітям ялинку. Цю пропозицію сприйняли як державний
наказ, і так Новий рік став головним радянським святом. У моєму сімейному
архіві та на просторах інтернету я знайшла безліч старих фото і чарівних
новорічних листівок. Пропоную подивитися на них разом і поринути в неймовірну
атмосферу свята минулих років.

\ii{24_12_2018.stz.news.ua.mrpl_city.1.svjato_nablyzhajetsja.pic.4}
\ii{24_12_2018.stz.news.ua.mrpl_city.1.svjato_nablyzhajetsja.pic.5}

\textbf{Читайте також:} \emph{Город покруче Киева: рождественские праздники в обновленном Мариуполе}%
\footnote{Город покруче Киева: рождественские праздники в обновленном Мариуполе, Анастасія Папуш, mrpl.city, 21.12.2018,\par%
\url{https://mrpl.city/news/view/gorod-pokruche-kieva-rozhdestvenskie-prazdniki-v-obnovlennom-mariupole-foto}
}

Найбільш поширеними були фото з родиною, друзями, сусідами. Адже разом
веселіше! І чи можна після 00:00 або впродовж 1 січня не заглянути кудись на
вогник.

\ii{24_12_2018.stz.news.ua.mrpl_city.1.svjato_nablyzhajetsja.pic.6}

Традиційним було фото з ялинкою як вдома, так і біля головної ялинки міста.

А новорічні листівки? Це ж ціла культура. Всі поспішали на пошту, щоб вибрати
найкрасивішу листівку, яку обов'язково відправляли близьким в інші міста і
країни, адже інтернету і смартфонів тоді ще не було. Цікаво, що, крім
розповсюджених зображень із зайченятами, ведмежатами, Дідом Морозом, його
чарівною онучкою, з 1950 року з'явилися листівки з зображенням людини, яка її
відправляла. Ось такий цікавий фотошоп, але ж родичам, мабуть, було дуже
приємно.

\ii{24_12_2018.stz.news.ua.mrpl_city.1.svjato_nablyzhajetsja.pic.7}

Фото зі святковими дитячими ранками теж були доволі популярні. Центральне місце
займали головні герої нового Року – Дід Мороз і Снігуронька і, звичайно, діти в
яскравих і казково гарних костюмах. Цікаво, ким сьогодні мріють стати діти?
Завдяки збереженим фото, здається, що раніше всі хотіли бути космонавтами.
Навіть на святковий ранок у дитячі садочки приходили справжні космонавти (чи
несправжні, але ж приходили).

\textbf{Читайте також:} \emph{Мариупольцы смогут проверить легальность новогодних елок с помощью приложения}%
\footnote{Мариупольцы смогут проверить легальность новогодних елок с помощью приложения, Роман Катріч, mrpl.city, 10.12.2018, \par\url{https://mrpl.city/news/view/mariupoltsy-smogut-proverit-legalnost-novogodnih-elok-s-pomoshhyu-prilozheniya} }

Підготовка до святкування Нового року стала окремим сюжетом на фото. Таких
світлин найменше, але все ж вони є. Це світлини з дітьми та їхніми батьками,
які прикрашають ялинку, з усміхненими та щасливими мешканцями, які поспішали
додому з подарунками в руках. Траплялися навіть фото з неймовірно довгими
чергами напередодні свята. Можливо, таким чином хотіли показати найбільш
затребувані товари. А черг найбільше було у дитячих та продуктових магазинах,
чим нікого не здивуєш і сьогодні.

\ii{24_12_2018.stz.news.ua.mrpl_city.1.svjato_nablyzhajetsja.pic.8}

Після невеликого екскурсу в минуле пропоную вам, дорогі читачі, продовжити
розглядати старі новорічні фото і листівки. Можливо, у когось збереглися старі
світлини батьків чи дідуся і бабусі біля маріупольської ялинки, чи фото з
родиною у новорічну ніч, чи унікальна листівка, яку тепер ніде більше не
знайдеш… Ці матеріали вже давно стали раритетом і можуть підняти всім настрій і
нагадати, що це свято користувалося і 10, і 50 років тому неабиякою
популярністю. Давайте ділитися спогадами в соціальних мережах та підіймати один
одному настрій.

\ii{24_12_2018.stz.news.ua.mrpl_city.1.svjato_nablyzhajetsja.pic.9}

\textbf{Читайте також:} \emph{Сколько выходных ждет мариупольцев в наступающем году?}%
\footnote{Сколько выходных ждет мариупольцев в наступающем году?, Анастасія Папуш, mrpl.city, 09.12.2018, \par %
\url{https://mrpl.city/news/view/skolko-vyhodnyh-zhdet-mariupoltsev-v-nastupayushhem-godu}}

\ii{24_12_2018.stz.news.ua.mrpl_city.1.svjato_nablyzhajetsja.pic.10}
\ii{24_12_2018.stz.news.ua.mrpl_city.1.svjato_nablyzhajetsja.pic.11}
\ii{24_12_2018.stz.news.ua.mrpl_city.1.svjato_nablyzhajetsja.pic.12}
\ii{24_12_2018.stz.news.ua.mrpl_city.1.svjato_nablyzhajetsja.pic.13}
