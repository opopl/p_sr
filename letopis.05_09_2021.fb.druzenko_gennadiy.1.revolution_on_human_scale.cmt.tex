% vim: keymap=russian-jcukenwin
%%beginhead 
 
%%file 05_09_2021.fb.druzenko_gennadiy.1.revolution_on_human_scale.cmt
%%parent 05_09_2021.fb.druzenko_gennadiy.1.revolution_on_human_scale
 
%%url 
 
%%author_id 
%%date 
 
%%tags 
%%title 
 
%%endhead 
\subsubsection{Коментарі}

\begin{itemize}
%%%fbauth
%%%fbauth_name
\iusr{Владимир Маглена}
%%%fbauth_url
%%%fbauth_place
%%%fbauth_id
%%%fbauth_front
%%%fbauth_desc
%%%fbauth_www
%%%fbauth_pic
%%%fbauth_pic portrait
%%%fbauth_pic background
%%%fbauth_pic other
%%%fbauth_tags
%%%fbauth_pubs
%%%endfbauth
 
Прекрасно. Очень всё красиво звучит )

%%%fbauth
%%%fbauth_name
\iusr{Ігор Шевченко}
%%%fbauth_url
%%%fbauth_place
%%%fbauth_id
%%%fbauth_front
%%%fbauth_desc
%%%fbauth_www
%%%fbauth_pic
%%%fbauth_pic portrait
%%%fbauth_pic background
%%%fbauth_pic other
%%%fbauth_tags
%%%fbauth_pubs
%%%endfbauth
 

Так, Гєна, бачу, що ти таки схильний до узурпаторства. Назву вечірки таки узурпував)!

\begin{itemize}
%%%fbauth
%%%fbauth_name
\iusr{Gennadiy Druzenko}
%%%fbauth_url
%%%fbauth_place
%%%fbauth_id
%%%fbauth_front
%%%fbauth_desc
%%%fbauth_www
%%%fbauth_pic
%%%fbauth_pic portrait
%%%fbauth_pic background
%%%fbauth_pic other
%%%fbauth_tags
%%%fbauth_pubs
%%%endfbauth
 
\textbf{Ігор Шевченко} революції - справа не зовсім демократична \Smiley[1.0][yellow]
\end{itemize}

%%%fbauth
%%%fbauth_name
\iusr{Yaroslav Znych}
%%%fbauth_url
%%%fbauth_place
%%%fbauth_id
%%%fbauth_front
%%%fbauth_desc
%%%fbauth_www
%%%fbauth_pic
%%%fbauth_pic portrait
%%%fbauth_pic background
%%%fbauth_pic other
%%%fbauth_tags
%%%fbauth_pubs
%%%endfbauth
 
Цікаво сформульовані результати мирної революції... Як будуть співіснувати RoofParty і антиолігархічна партія Успішна Україна?🤔

%%%fbauth
%%%fbauth_name
\iusr{Владимир Дубровский}
%%%fbauth_url
%%%fbauth_place
%%%fbauth_id
%%%fbauth_front
%%%fbauth_desc
%%%fbauth_www
%%%fbauth_pic
%%%fbauth_pic portrait
%%%fbauth_pic background
%%%fbauth_pic other
%%%fbauth_tags
%%%fbauth_pubs
%%%endfbauth
 

Пікетті абсолютно неправий, це лівацька маячня. Бідні точно не стають
біднішими, хоча, можливо, іноді багатіють повільніше. Взагалі навколо так
званої "соціальної справедливості" настільки багато лівацьких спекуляцій,
маніпуляцій та просто відвертої брехні, що я взагалі проти вживання цього
словосполучення у пристойному товаристві. Десь так, як з "класовою боротьбою",,
расизмом та гендерними забобонами

\begin{itemize}
%%%fbauth
%%%fbauth_name
\iusr{Роман Химич}
%%%fbauth_url
%%%fbauth_place
%%%fbauth_id
%%%fbauth_front
%%%fbauth_desc
%%%fbauth_www
%%%fbauth_pic
%%%fbauth_pic portrait
%%%fbauth_pic background
%%%fbauth_pic other
%%%fbauth_tags
%%%fbauth_pubs
%%%endfbauth
 
В Україні дещо кумедні уявлення про спектр політичної думки. Дуже сильно зсунуті до умовно правої частини
\ifcmt
  ig https://scontent-frx5-1.xx.fbcdn.net/v/t1.6435-9/241193858_4747702145260186_5467747498917213693_n.jpg?_nc_cat=110&_nc_rgb565=1&ccb=1-5&_nc_sid=dbeb18&_nc_ohc=ZgVByvziVd0AX9b6K2I&_nc_ht=scontent-frx5-1.xx&oh=062782aeb74eb5426dfb9bf38451a4fb&oe=615BA79E
  @width 0.3
\fi

%%%fbauth
%%%fbauth_name
\iusr{Gennadiy Druzenko}
%%%fbauth_url
%%%fbauth_place
%%%fbauth_id
%%%fbauth_front
%%%fbauth_desc
%%%fbauth_www
%%%fbauth_pic
%%%fbauth_pic portrait
%%%fbauth_pic background
%%%fbauth_pic other
%%%fbauth_tags
%%%fbauth_pubs
%%%endfbauth
 
\textbf{Vladimir Dubrovskiy} 

Ок, Пікетті - лівак (хоча на Заході його вважають
поважним науковцем). Але той самий висновок я знаходив у Снайдера в його \enquote{Road
to unfreedom} на прикладі США. Врешті-решт, це проста статистика: чим
ліберальніша економіка, тим більший розрив між бідними і багатими. Я навіть не
кажу, що це погано. Іноді - це передумови економічного ривка. Але це тільки
загострює відчуття соціальної несправедливості. Яке раніше чи пізніше вибухає.

%%%fbauth
%%%fbauth_name
\iusr{Роман Химич}
%%%fbauth_url
%%%fbauth_place
%%%fbauth_id
%%%fbauth_front
%%%fbauth_desc
%%%fbauth_www
%%%fbauth_pic
%%%fbauth_pic portrait
%%%fbauth_pic background
%%%fbauth_pic other
%%%fbauth_tags
%%%fbauth_pubs
%%%endfbauth
 
Снайдер це а) ситуативний та б) корисний лівак.

%%%fbauth
%%%fbauth_name
\iusr{Владимир Дубровский}
%%%fbauth_url
%%%fbauth_place
%%%fbauth_id
%%%fbauth_front
%%%fbauth_desc
%%%fbauth_www
%%%fbauth_pic
%%%fbauth_pic portrait
%%%fbauth_pic background
%%%fbauth_pic other
%%%fbauth_tags
%%%fbauth_pubs
%%%endfbauth
 

та немає такого закону, це мислевірус. Бо найбільше нерівності створює саме
держава, або прямо (екстрактивними інституціями), або нерівністю у правах
(почитайте Ернесто де Сото), або втручанням на боці багатих, як-от у США де
свіжонадруковані долари потрапляють прямісінько на Уолл-стріт. І ця нерівність
є більшою за природну, до якої призводить ліберальний ринок. А у природній
нічого особливо поганого немає, це просто закон природи

%%%fbauth
%%%fbauth_name
\iusr{Владимир Дубровский}
%%%fbauth_url
%%%fbauth_place
%%%fbauth_id
%%%fbauth_front
%%%fbauth_desc
%%%fbauth_www
%%%fbauth_pic
%%%fbauth_pic portrait
%%%fbauth_pic background
%%%fbauth_pic other
%%%fbauth_tags
%%%fbauth_pubs
%%%endfbauth
 
тут 
\href{https://zbruc.eu/node/59044}{%
Крах корупційного консенсусу: ризики, Володимир Дубровський, zbruc.eu, 23.11.2016%
}

я дещо про нерівність висловив

%%%fbauth
%%%fbauth_name
\iusr{Владимир Дубровский}
%%%fbauth_url
%%%fbauth_place
%%%fbauth_id
%%%fbauth_front
%%%fbauth_desc
%%%fbauth_www
%%%fbauth_pic
%%%fbauth_pic portrait
%%%fbauth_pic background
%%%fbauth_pic other
%%%fbauth_tags
%%%fbauth_pubs
%%%endfbauth
 
а лівак і поважний науковець - це сумісні речі, на превеликий жаль ((

%%%fbauth
%%%fbauth_name
\iusr{Павел Себастьянович}
%%%fbauth_url
%%%fbauth_place
%%%fbauth_id
%%%fbauth_front
%%%fbauth_desc
%%%fbauth_www
%%%fbauth_pic
%%%fbauth_pic portrait
%%%fbauth_pic background
%%%fbauth_pic other
%%%fbauth_tags
%%%fbauth_pubs
%%%endfbauth
 

отличие Украины от стран с диким капитализмом в том, что богатыми у нас
становятся не капиталисты, а чиновники. Такое пану Пикетти и не снилось.


%%%fbauth
%%%fbauth_name
\iusr{Gennadiy Druzenko}
%%%fbauth_url
%%%fbauth_place
%%%fbauth_id
%%%fbauth_front
%%%fbauth_desc
%%%fbauth_www
%%%fbauth_pic
%%%fbauth_pic portrait
%%%fbauth_pic background
%%%fbauth_pic other
%%%fbauth_tags
%%%fbauth_pubs
%%%endfbauth
 
\textbf{Pavel Sebastianovich} 

хіба Коломойський чи Ахметов колись були чиновниками? Насправді читаючи історію
Gilded Age в США, коли були чіткі ставки на голосування в Конгресі за той чи
інший законопроект, а моргани, рокфелери та карнегі заробляли неймовірні
статки, купуючи великим оптом суддів та політиків та створюючи монополії,
розумієш, що ми надто переоцінюємо свою унікальність \Smiley[1.0][yellow]


%%%fbauth
%%%fbauth_name
\iusr{Yaroslav Znych}
%%%fbauth_url
%%%fbauth_place
%%%fbauth_id
%%%fbauth_front
%%%fbauth_desc
%%%fbauth_www
%%%fbauth_pic
%%%fbauth_pic portrait
%%%fbauth_pic background
%%%fbauth_pic other
%%%fbauth_tags
%%%fbauth_pubs
%%%endfbauth
 
\textbf{Gennadiy Druzenko}, 

на жаль, Pavel Sebastianovich вперто не хоче бачити/розуміти прямого зв'язку
між олігархами, навіть гірше, олігархопаханатом і "госапаратом", як говорить
Pavel S. Аргументація ДУЖЕ\Laughey[1.0][white] переконлива: (майже пряма мова P.S.) підприємець
бачить/спілкується з олігархами? Ні!!! Він спілкується з пожежним, санітарним
інспектором, рядовим податківцем, суддею в місцевому суді - представником
госапарата😅... От в них і є ЗЛО!

%%%fbauth
%%%fbauth_name
\iusr{Yaroslav Znych}
%%%fbauth_url
%%%fbauth_place
%%%fbauth_id
%%%fbauth_front
%%%fbauth_desc
%%%fbauth_www
%%%fbauth_pic
%%%fbauth_pic portrait
%%%fbauth_pic background
%%%fbauth_pic other
%%%fbauth_tags
%%%fbauth_pubs
%%%endfbauth
 
\textbf{Vladimir Dubrovskiy} корисна розмова! Цікаво би почитати відредагований
матеріал в сьогоднішніх реаліях. Моє враження, що в Україні і світі все стало
ще більш квазі-ліберальним і популістичним.



\end{itemize}

%%%fbauth
%%%fbauth_name
\iusr{Max Mykhaylenko}
%%%fbauth_url
%%%fbauth_place
%%%fbauth_id
%%%fbauth_front
%%%fbauth_desc
%%%fbauth_www
%%%fbauth_pic
%%%fbauth_pic portrait
%%%fbauth_pic background
%%%fbauth_pic other
%%%fbauth_tags
%%%fbauth_pubs
%%%endfbauth
 

У нашій феодальній - з деякими елементами капіталізму - державі - питання
соціальної справедливості треба передати Церкві. Отримавши обов-язкову
десятину, вона частину віддаватиме жебракам. Давно слід припинити беззмістовні
фрікції в системі освіти і передати школи в управління Церкви, а університети
поділити між Церквою та армією, непотрібні закрити і перетворити на храми, або
ж казарми. Расистське питання закрите буллою Sublimus Dei папи Павла III у 1537
році. Гендерне питання вирішується звільненням жінок від трудової повинності
при розширенні їхніх політичних прав у дусі думки щодо цього питання папи
Бенедикта XV у 1919 році, яку підтвердив час.


%%%fbauth
%%%fbauth_name
\iusr{Олег Миколайович Баклан}
%%%fbauth_url
%%%fbauth_place
%%%fbauth_id
%%%fbauth_front
%%%fbauth_desc
%%%fbauth_www
%%%fbauth_pic
%%%fbauth_pic portrait
%%%fbauth_pic background
%%%fbauth_pic other
%%%fbauth_tags
%%%fbauth_pubs
%%%endfbauth
 

Овцы в волчей шкуре могут издалека или даже вблизи напугать, и заставить
затаиться или даже убежать подальше от них, думая, что это волки.

Но в конечном итоге овцы в волчей шкуре всегда лишь щиплют травку, а не атакуют
свою добычу, чтобы загрызть её и насытиться её плотью

Аналитик, даже если своими мыслями привлёк к себе внимание, получил широкую
известность и всерьёз может рассчитывать на миллионнные голоса избирателей или
даже победу на выборах - это не политик, это аналитик

Это овца, а не волк. Да, волчья шкура на этой овце есть, но не более того.

Овца всегда щиплет травку, пишет свои умные и правильные мысли, но никогда не
нападает на свою добычу, на действующую власть, чтобы отнять её и сделать
своей, чтобы загрызть свою добычу и полакомиться её плотью

P. S.

Написал, как полемический укол в адрес Эрика Земмура, но написанное в полной
мере относится (за редким исключением, точнее за ОДНИМ исключением) к
политическом движению "Украина 30+. Что дальше?"

Задайте вопрос, кто в этом движении, в его истоках, на учредительском собрании
говорит о взрослой политике, овцы (пусть даже в волчей шкуре неких политических
целей, программ, действий) или все таки волки

Думаю, овцы, и дальше пощипать травку конечного результата не будет.

Никогда даже самый талантливый Николло Маккиавели, гениально пишущий очередного
"Государя", сам Государем не становится

Зубы овцы плоские, ими можно щипать травку, но не впиваться в плоть и рвать её,
приканчивая свою добычу и сьедая потом её плоть.

Остры зубы есть лишь у волков

Можна сколько угодно считать себя бонапортистом или голлистом (это я снова
вернулся к полемическому уколу в адрес Эрика Земмура), но быть Наполеоном
Боннапартом или Шарлем де Голлем - это другое

Что касается меня, я скорее удар молнии.

Моё движение всегда зигзаг, всегда сверху, начинаю с наиболее общего и
эмоционального взгляда на Украину и украинцев, с понимания виденья и воли Бога
относительно Украины и нас, потом падаю на твердь наших реалий, и в своём ударе
освещаю тьму украинского понимания, каким должно быть наше настоящее и будущее.

Этот символизм отражён в моем личном знамени, к слову говоря

\begin{itemize}
%%%fbauth
%%%fbauth_name
\iusr{Max Mykhaylenko}
%%%fbauth_url
%%%fbauth_place
%%%fbauth_id
%%%fbauth_front
%%%fbauth_desc
%%%fbauth_www
%%%fbauth_pic
%%%fbauth_pic portrait
%%%fbauth_pic background
%%%fbauth_pic other
%%%fbauth_tags
%%%fbauth_pubs
%%%endfbauth
 

\textbf{Олег Миколайович Баклан} гм! Небезынтересно)

%%%fbauth
%%%fbauth_name
\iusr{Олег Миколайович Баклан}
%%%fbauth_url
%%%fbauth_place
%%%fbauth_id
%%%fbauth_front
%%%fbauth_desc
%%%fbauth_www
%%%fbauth_pic
%%%fbauth_pic portrait
%%%fbauth_pic background
%%%fbauth_pic other
%%%fbauth_tags
%%%fbauth_pubs
%%%endfbauth
 
\textbf{Max Mykhaylenko} 

Смотрю сейчас выступление Александра Дония на учредительном собрании
политического движения "Украина 30+. Что дальше?"

Эрик Земмур подрос, немного помолодел, заговорил на украинском языке без
пронапооеоновских и продеголлевских мыслей, и рассказывает, как

все плохо \Smiley[1.0][yellow]

Реальность в Украине качественно иная, прямо противоположная тому, как о ней
рассказывает Александр Доний

Из моих слов не следует, что в Украине "всё хорошо" или, как вариант мягче, в Украине "плохо не всё"

Моё возражение имеет другое содержание :

1. Александр Доний и все остальные участники в создаваемой политическом
движении "Украина 30+. Что дальше?" мыслят реалии Украины в координатах или-или
(или плохо, или хорошо), и выбирают для оценки вариант "плохо", а дальше,
автоматически, нацеливаются на "надо перейти от плохо к хорошо" и "мы это
сделаем, а если не мы, то кто?" и так далее

2. Это бессмысленное, уже устаревшее мышление на основе логики Аристотеля с её законом исключенного третьего, с этим самым или-или

3. Мыслить надо иначе, создав новый уровень мышления, как мышления на основе
логики Лотфи Заде с её ОГРАНИЧЕНИЕМ закона исключенного третьего, когда нет
или-или, а есть и-и

Мы не "или наравне/впереди, или позади" с вариантом "мы позади" стран
Центральной Европы, которые стартовали в тоже время, что и мы, они в 1989
(бархатная революция), мы в 1990 (революция на гранит)

Мы "и наравне/впереди, и позади" этих стран

Если шире, мы не "или слабы, или сильны ", не "или пророссийские, или
проевропейские", а сейчас не "или прокитайские, или проамериканской", а самое
главное мы не "или умны, или глупы", не "или действуем и достигает результата,
или бездействуем и ничего не достигает", мы все это и-и

У всего этого есть свой реальный смысл, и этот смысл скользит то в одну
сторону, то в другую, но никогда не становится каким-то одним смыслом

В этом состоянии надо строить абсолютно другое политическое движение, чем
делалось до сих пор, делается сейчас и делалось бы дальше, если бы я не
появился среди украинцев со своим уникальным и очень мощным мессежем о нашей
сути и нашем пути в наше будущее

Один из украинцев назвал меня украинским Наполеоном, возражая другому, который
обвинил меня в гигантских, наполеоновских планах: "Ты прав, это действительно
гигантские, наполеоновских планы. Но спроси себя, если Олег наш украинский
Наполеон, то какие у него должны быть планы, разве не наполеновские?"

Во-первых, мы имперская страна, а имперская Россия - это наша имперская ветвь,
отломленная от нашей имперской украинской сути. Этой сути не надо бояться, её
надо осознать и принять, а приняв объединить себя с Россией, но не путем
присоединения Украины к России как части России, а путем присоединения России к
Украине как части Украины, где RU приобретает значение Region Ukrainian. Наши
противоречия Восток-Запад, это не противоречия Россия-Украина в нашем
восточноукраинском и западноукраинском мышлении, чуть разном, но тем не менее
едином, а противоречие есть-нет явное проявление нашей имперский сути

Во-вторых, понимая свою имперскую суть и давая ей выход, мы можем и должны
пойти по пути деголленизации нашей внутренней и внешней политики, и, став на
такой путь, помочь нашим братьям европейцам, утверждая разными способами и там
деголленизации, как рестарт деголленизации во Франции, и деголленизации во всех
остальных европейских странах, в первую очередь в Германии, но именно как
деголленизации, а не гитлеризацию. Единоличное абсолютное правление, вождизм
нам и им нужен в стиле де Голля, а не в стиле Гитлера

В-третьих, нашу деголленизации мы можем и должны осуществить как восстановление
Украинского Гетьманського Государства, созданного/продолженного Павлом
Скоропадским в ответ на просьбу 3000 делегатов Всеукраинского Съезда Хлеборобов

Из локальных целей, я сейчас вижу три:

1. Занять резиденцию по переулку Тараса Шевченко, 16, купив её за 6 млн
долларов, и сделав нашим "Смольным"

2. Организовать наш бюджет, для начала 2 млрд долларов

3. Создать во Франции форум "Большой разговор в Париже", и провести там
стартовый "разговор номер один", полемику я-Земмур относительно рестарта
деголленизации Франции и старта деголленизации Украины, сотрудничества
деголленизированной Франции и деголленизированной Украины


%%%fbauth
%%%fbauth_name
\iusr{Олег Миколайович Баклан}
%%%fbauth_url
%%%fbauth_place
%%%fbauth_id
%%%fbauth_front
%%%fbauth_desc
%%%fbauth_www
%%%fbauth_pic
%%%fbauth_pic portrait
%%%fbauth_pic background
%%%fbauth_pic other
%%%fbauth_tags
%%%fbauth_pubs
%%%endfbauth
 
\textbf{Alexandre Melnik}
\end{itemize}

\end{itemize}

