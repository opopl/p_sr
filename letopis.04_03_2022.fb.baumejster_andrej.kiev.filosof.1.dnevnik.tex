% vim: keymap=russian-jcukenwin
%%beginhead 
 
%%file 04_03_2022.fb.baumejster_andrej.kiev.filosof.1.dnevnik
%%parent 04_03_2022
 
%%url https://www.facebook.com/andriibaumeister/posts/4816388301816024
 
%%author_id baumejster_andrej.kiev.filosof
%%date 
 
%%tags __feb_2022.vtorzhenie,dnevnik
%%title Девятый день войны. 4 марта. Продолжаю свой дневник
 
%%endhead 
 
\subsection{Девятый день войны. 4 марта. Продолжаю свой дневник}
\label{sec:04_03_2022.fb.baumejster_andrej.kiev.filosof.1.dnevnik}
 
\Purl{https://www.facebook.com/andriibaumeister/posts/4816388301816024}
\ifcmt
 author_begin
   author_id baumejster_andrej.kiev.filosof
 author_end
\fi

Девятый день войны. 4 марта. Продолжаю свой дневник. Позволю себе немного
философских наблюдений. Поэтому, если вам нужны резкие слова и оценки, не
читайте этот текст. Он - не для подхлестывания эмоций. Это первые попытки
осмысления. 

- Реальность распадается на осколки. Мы наблюдаем, как различные группы людей
поселились в своих особых мирах, в своих \enquote{пещерах}. Заселение произошло
давно, но даёт свои зрелые плоды только сейчас. 

Реальности не две, не три, их гораздо больше. И эти реальности нигде не
совпадают. Это не-пересекающиеся и не-взаимодействующие онтологии. В этих
различных мирах, хорошо обжитых, люди мыслят, оценивают, принимают решения. В
этих мирах уже сформированы все цепочки аргументов и собраны в когерентную
систему все наборы объяснительных схем.

Можно говорить, что все эти (квази) рациональные ресурсы выглядят как
безупречные и монолитные априорные построения. Все новые события и факты, как
твёрдые зерна, сразу поступают в эти мифологические мальницы и производят
правильную муку из которой выпекают правильный хлеб.

Весь мир видит обугленные дома, разбитые улицы, страдающих людей. А в этих
дьявольских мельницах совершается дьявольская работа. Всё \enquote{объяснено} и
\enquote{оправдано}.

Кажется, что обугленные дома, груды камней вместо улиц, искаженные горем и
отчаянием лица, - это \enquote{твёрдые факты} и \enquote{надёжные свидетели}. Но нет! К фактам
не пробиться. К событиям не подобрать ключей.

Даже язык становится нашим врагом. Семантика скорее запутывает, чем поясняет.
Освобождение, мир, ценности, свобода, гуманизм, нация, денацификация,
демилитаризация, пацификация, справедливость, воздаяние, возмездие, милосердие,
вера, божественные заповеди. Что нам делать с этими словами? Хочется сложить их
в глубине двора как древесные поленья. Или собрать метлой в мусорную кучу.

Нам нужны новые слова, а их нет. Срочно требуются новые объяснительные схемы,
но от них прячутся с паническим ужасом или отмахиваются с отчаянной ненавистью.
Нам нужно понимание и взаимопонимание, но все забыли, что это такое. И для чего
это используют. И с чем его едят. 

Меня упрекали в том, что в последнем видео я не объективен. Мои дорогие,
спросите самих себя: вам нужна объективность? Вы хотите объективности? Вы ищите
её? В неё ведь входят не только факты и события, но и оценки.

Эй, говорящие об истине сто раз на день! Вам это надо? Вы ищите истину? 

Мой ответ пока очень прост. Есть вещи бесспорные. И их нужно принимать всем:

- Именно Россия вторглась в Украину. Что это было \enquote{жизненно необходимо} и \enquote{по
другому было нельзя} - это оставьте для своей веры или изуверства. И вообще,
оставьте \enquote{на потом} свои супер-аналитические выкладки. Упивайтесь ими с утра до
вечера. Келейно. Кадя своим идолам и прославляя своих божков. Только не
называйте это христианством, пожалуйста.

Итак, именно Россия вторглась в Украину, а не Украина в Россию. 

- Армия РФ разрушает украинские города и села. Рушится вся инфраструктура
страны. Что это имеет какие-то там \enquote{оправдания} в терминах \enquote{возмездия} или
\enquote{справедливости}, - оставьте при себе. Хотите устраивать на эту тему
сектантские \enquote{радения} - на здоровье. Но не затемняйте и не подменяйте этими
умствованиями твёрдых фактов. Уничтожаются не российские, а украинские города.
Это \enquote{справедливо}, это \enquote{так надо}? Значит у вас что-то не так с головой.

- Гибнут мирные люди. Гибнут мирные украинцы, а не мирные россияне. Вы хотите
сказать, что гибель мирных украинцев необходима для жизни мирных россиян? Никто
из полноценных людей так думать не может.

Вы говорите, что украинские войска прикрываются мирными гражданами? Но
российские войска направляют бомбы и ракеты на мирные дома. Где просто живут
люди. Они не \enquote{живой щит} \enquote{неонацистов}, они просто живут в своих домах.

Вы говорите, что Украина несла угрозу России? Но именно Россия сейчас разрушает
Украину и угрожает применить тактическое вооружие. Это сбой логики: \enquote{вы можете
быть для нас опасными}, поэтому мы вас уничтожим или сломаем. 

Да, есть вещи спорные. Да, есть много непроясненных вопросов. Но верхом цинизма
и бесчеловечности затевать идеологические  споры под градом бомб и ракет. Когда
идеология прикрывает и оправдывает военные преступления она уже патологична и
аморальна.

Сейчас нужно признать базовые факты и сложить хотя бы контуры базовой
онтологии. Сконцентрироваться на бесспорных вещах. А о спорных вещах поговорим
потом. Когда войска чужой страны покинут нашу страну. И когда всеми будут
признаны бесспорные вещи.

Это трудно. Но идти другим путем - опасно и губительно. Для всех...
