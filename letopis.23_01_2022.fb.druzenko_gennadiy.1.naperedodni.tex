% vim: keymap=russian-jcukenwin
%%beginhead 
 
%%file 23_01_2022.fb.druzenko_gennadiy.1.naperedodni
%%parent 23_01_2022
 
%%url https://www.facebook.com/gennadiy.druzenko/posts/10158727023203412
 
%%author_id druzenko_gennadiy
%%date 
 
%%tags rossia,ugroza,ukraina
%%title НАПЕРЕДОДНІ
 
%%endhead 
 
\subsection{НАПЕРЕДОДНІ}
\label{sec:23_01_2022.fb.druzenko_gennadiy.1.naperedodni}
 
\Purl{https://www.facebook.com/gennadiy.druzenko/posts/10158727023203412}
\ifcmt
 author_begin
   author_id druzenko_gennadiy
 author_end
\fi

НАПЕРЕДОДНІ

Вихідні – слушний час для читання та роздумів. Особливо коли весь тиждень
готуєшся до найгіршого розвитку подій з вірою, що його вдасться уникнути.

Отже, добра новина, що Україні вдалось переламати тренд несубʼєктності. Браво
Dmytro Kuleba, Резніков Олексій, Валерій Залужний та інші причетні! Принаймні
деякі наші західні патрнери зрозуміли, що не тільки наше майбутнє залежить від
них, а й їхнє – від нас. Якщо Путін поставить Україну на коліна, це буде жирна
крапка у довгій історії домінування колективного Заходу. І деморалізований
\enquote{вільний світ} змушений буде перейти у тривалу позиційну оборону, визнавши
неефективність своєї базової моделі управління в порівнянні з автократичними
режимами.

\ii{23_01_2022.fb.druzenko_gennadiy.1.naperedodni.pic.1}

Тому Україна нарешті заговорила не як бідний родич, що ходить по світу з
простягнутою рукою, а як вершитель доль Західного світу. Бо Захід, який
поквапився попередити, що не надсилатиме в Україну свої війська, може перемогти
Путіна тільки українськими руками. Але для цього в цих руках має бути ефективна
зброя. І Захід в останні тижні почав її масовано надсилати в Україну.

А тим, хто гальмує процес озброєння України, насамперед Німеччині, приходиться
виправдовуватися і посилати у відставку своїх говірливих адміралів. Мене
неймовірно тішить, коли Президент України ставить на місце Президента США, що
плете нісенітниці про minor incursion, а Міністерство закордонних справ України
/ MFA of Ukraine викликає на килим німецького посла, аби вичитати її за
\enquote{одкровення} очільника ВМС Німеччини. Ще раз: браво усім причетним! Бо комплекс
меншовартості тривалий час був нашим найбільшим ворогом, що прирікав нас на
поразку.

За останній місяць ситуація з сучасними озброєннями кардинально змінилась – і
тепер нам є чим зустрічати непроханих "гостей" зі Сходу та Півночі. Докладніше
– в чудовому інтервʼю Yuriy Romanenko мого друга Arty Green:
https://youtu.be/B-Hkd0W8hJc

Тепер про погані новини. Достойна відповідь України та Заходу на неприйнятні
ультиматуми Путіна заганяє старого щура в глухий кут. Після того як Владіміра
Владіміровіча з його мріями про світ як "концерт великих держав" доволі
однозначно послали, йому – аби зберегти обличчя насамперед перед своїми елітами
– потрібно якось відповісти. В противному разі російські еліти зрозуміють, що
Акела промахнувся – і час відправляти його на пенсію (чи – в російських
традиціях – на той світ). А підвищувати ставки мирним шляхом далі фактично нема
куди. Залишаються тільки \enquote{військово-технічні засоби}. 

Саме тому Росія перекидає до Білорусі \enquote{Іскандери} з Далекого Сходу. Саме тому
Росія оголошує безпрецедентні навчання ВМФ, в яких будуть задіяні 140 кораблів
усіх російських флотів. Саме тому Росія активізувала кібератаки на критичну
українську інфраструктуру.

Тут Байден правий: "My guess is he will move in. He has to do something." У нас
чомусь звернули увагу на першу частину фрази. Але насправді важливіша друга: за
законами жанру Путін має \enquote{щось робити}, або визнати своє фіаско. Що матиме
руйнівні наслідки для його режиму.

Коли західні аналітики оцінюють втрати, яких зазнає Росія, зважившись на пряму
агресію проти України, вони часто-густо забувають оцінити, а яких втрат
(насамперед іміджевих) зазнає Путін, якщо він дослухається голосу розуму і
поверне війська у казарми. Не забуваймо, що легітимність Путіна ґрунтується на
наративі \enquote{піднімання Росії з колін}. Приборкання Чечні, \enquote{примус до миру}
Грузії, \enquote{повернення до рідної гавані} Криму, \enquote{підтримка законної влади} в Сирії
(та Венесуелі), повернення російських проксі до Африки... Все це в російській
свідомості – \enquote{вставання з колін}...

Проковтнути ляпас від Заходу, який на ультимативну вимогу залишити Україну сам
на сам із агресором став форсовано озброювати її, – в уявленні Путіна означає
вмить знову опинитись на колінах. І тому я не бачу іншого виходу із ситуації,
який би допоміг Путіну зберегти обличчя як всередині країни, так і перед своїми
партнерами-автократами, окрім як спробувати поставити на коліна Україну.
Показово покарати її за прозахідний курс і водночас показати неспроможність
колективного Заходу захистити тих, хто щиро прагне стати його частиною.

В цій ситуації Україна стає (вже стала) центральною ставкою у глобальній грі.
Яку Путін не може собі дозволити програти набагато більше, ніж Захід. Бо на
відміну від західних лідерів, для яких результат битви за Україну – це питання
престижу, для Путіна після максимального підняття ставок за останні місяці – це
питання життєздатності режиму, виплеканого ним за два десятиліття при владі.

І тут все вирішуватиме наша воля захищатись. Вперше ми не самотні у нашій борні
за нашу незалежність. Вперше наші західні партнери стали поважати наше право на
самозахист не глибокою стурбованістю, а масовими поставками високотехнологічної
сучасної зброї. Вперше на чолі Збройні Сили України стоять бойові та
патріотичні генерали, за яких не соромно. І які – на відміну від Муженка –
знищуватимуть новоявлених гіркіних та захищатимуть ДАП, а не виконуватимуть
політичні договорняки.

Безумовно, нам вкрай бракує ефективного політичного проводу, який здатний
мислити стратегічно. Кадрову еклектику Володимир Зеленський звів у абсолют.
Патріоти в уряді перемішані з відвертими колаборантами, а порядні професіонали
– з невиліковними корупціонерами. 

Але можливо саме тому нам і належить пройти через горнило справжньої війни?
Може її вогонь нарешті випече наші одвічні хвороби містечковості,
короткозорості та комплекс жертви? Бо саме війна народжує масштабні особистості
Вашингтонів, Ататюрків та де Голлів, які здатні закладати тривкі підвалини
державності своїх народів. 

Ціна, звичайно, буде великою. Але як написано на чудовому Korean War Memorial у
Вашингтоні, Freedom is not free. Свобода не буває задарма.
