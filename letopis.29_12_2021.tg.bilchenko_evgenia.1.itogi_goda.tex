% vim: keymap=russian-jcukenwin
%%beginhead 
 
%%file 29_12_2021.tg.bilchenko_evgenia.1.itogi_goda
%%parent 29_12_2021
 
%%url https://t.me/bilchenkozhenya/5458
 
%%author_id bilchenko_evgenia
%%date 
 
%%tags 
%%title Ну, что, родня, подведем итоги 2021 года?
 
%%endhead 
\subsection{Ну, что, родня, подведем итоги 2021 года?}
\label{sec:29_12_2021.tg.bilchenko_evgenia.1.itogi_goda}

\Purl{https://t.me/bilchenkozhenya/5458}
\ifcmt
 author_begin
   author_id bilchenko_evgenia
 author_end
\fi

Ну, что, родня, подведем итоги 2021 года?

\ii{29_12_2021.tg.bilchenko_evgenia.1.itogi_goda.pic.1}

Итак, в 2021 году я:

\begin{itemize}
  \item 1. Написала и издала два поэтических сборника: \enquote{Пьета} и \enquote{Зорге и Зорька: транслит по-русски}. Все проданы, продолжаю реализовывать пдф-ки за донаты.
  \item 2. Написала и издала 27 научных статей, включая 3 монографии, одна из которых - в престижном издательстве \enquote{Алетейя}.
  \item 3. Приняла участие в около 30 научных конференциях.
  \item 4. Продолжала вести \enquote{сепаратистские} лекции на Ютубе.
  \item 5. 18 января 2021 года написала в Фейсбуке пост, что Украина - колония Америки.
  \item 6. Пережила два суда прогрессивной украинской общественности.
  \item 7. Пережила буллинг в сети общей численностью около 10 тысяч информационных единиц. Его остатки длятся до сих пор в вялой форме.
  \item 8. Была отстранена от педагогической деятельности.
  \item 9. Моя страница на \enquote{Миротворце} пополнилась новыми красивыми фотографиями и преступлениями против украинской государственности.
  \item 10. Пережила три уличных приставания (одно с рукоприкладством).
  \item 11. Пережила около 50 интервью на русских и украинских медиа.
  \item 12. Отказалась от политубежища в Чехии и гранта в Финляндии. Простите, комрады, Томас, Катержина и Оксана.
  \item 13. 20 сентября была уволена из педагогического университета с самым высоким научным рейтингом по кафедре. Мне предложили место уборщицы.
  \item 14. Занесена в список службы занятости Украины как агент Кремля с волчьим билетом.
  \item 15. Начала работать копирайтером и овладела искусством редактуры и, вообще, ВСЕГО.
  \item 16. На Самайн поменяла страну проживания: выехала в Питер, продав свою занесенную на \enquote{Миротворец} трёхкомнатную квартиру в Киеве ради студии в центре центра града Петра. Ехали двое суток с приключениями. Вышла я в минус. 
  \item 17. Начала переживать обратную травлю со стороны своих новых сограждан, которые не поверили в мое покаяние.
  \item 18. Имела интересный опыт научного сотрудничества с РУДН в Москве.
  \item 19. Увидела живьём любимых поэтов и музыкантов. Начала офлайн тусить с прикольными людьми.
  \item 20. Получила горячую неожиданную и ожидаемую поддержку питерских читателей и была приглашена на выступления на все самые крутые литературные вечера.
  \item 21. Выступала мало где: Одесса (раз), Киев (раз), Питер (ой, раз, да ещё раз, да ещё много-много раз).
  \item 22. Научилась курить на лестнице и на двадцатиградусном морозе.
  \item 23.  Привилась спутником.
  \item 24. Начала лечить острый гастрит у нормального врача. Практически отказалась от спиртного.
  \item 25. Сползла с успокоительных. Сократила курение в два раза.
  \item 26. Получила дисплазию шейки матки и хронический сепсис из горла родом, на лечение нет денег.
\end{itemize}

Планы:

\begin{itemize}
  \item Аудиокнига \enquote{Штрафбат} - предложение от одного рэп-поэтри клуба.
  \item Уход за мужем, ссоры с мужем.
  \item Сольник поэтический.
\end{itemize}

Мечты:

\begin{itemize}
  \item СПбГУ, курсы по антиглобализму, полставки.
  \item Полис ОМС. 
  \item Лечение по госуслугам РФ, положенное мне по закону. Нужны большие деньги.
  \item Зарабатывать как писатель, независимый.
  \item Иметь деньги, чтобы издавать книги и ездить с ними по России.
  \item Показать мужу Прагу, для этого, опять-таки, иметь деньги.
  \item Платье белое, шерстяное, 100\% шерсть.
\end{itemize}

А у вас как год прошел?

На фото: я в январе 2021 г. в Киеве (слева), я сейчас (справа). Какая фотка
лучше?

