%%beginhead 
 
%%file 16_03_2023.fb.stomina_oksana.mariupol.1.zustrinemos_na_drami_druzi
%%parent 16_03_2023
 
%%url https://www.facebook.com/oksana.stomina/posts/pfbid02sr4fnyPbe3pW9fUA26YEZ9rUFsWcwKKP1kFRCYmtGMWVFJ5hwY66foXnQu1zNeFHl
 
%%author_id stomina_oksana.mariupol
%%date 16_03_2023
 
%%tags 
%%title Зустрінемося на драмі, друзі!
 
%%endhead 

\subsection{Зустрінемося на драмі, друзі!}
\label{sec:16_03_2023.fb.stomina_oksana.mariupol.1.zustrinemos_na_drami_druzi}

\Purl{https://www.facebook.com/oksana.stomina/posts/pfbid02sr4fnyPbe3pW9fUA26YEZ9rUFsWcwKKP1kFRCYmtGMWVFJ5hwY66foXnQu1zNeFHl}
\ifcmt
 author_begin
   author_id stomina_oksana.mariupol
 author_end
\fi

Але сьогодні я зроблю два дописи. І цей, другий, буде про театр, який ми любили
й пам'ятатимемо завжди. 

Лише кілька красномовних цікавинок про те, чим був театр для маріупольців.

Отже...

- Через директиву, згідно з якою будівлі театрів належало зводити лише в
обласних містах, наше місто мало залишатися без театру. Але містяни вимагали й
добилися свого. Та щоб обійти директиву, довелося зробити хід конем – назвати
маріупольський театр \enquote{Донецьким}...

- На першу виставу театру отримали запрошення всі будівельники, які зводили
театральну будівлю.

- Перша вистава в будівлі театру відбулася 2 листопаду 1960 року. А всього за
театральний сезон 1960-1961 рр. було поставлено 15 п'єс, відбулося понад 3700
вистав, театр прийняв 221. 600 глядачів і перевиконав план доходів на 54,6
тис. карбованців! 

-  Але були й важкі часи. В 90-х через проблеми з фінансуванням, працівники
театру зверталися по допомогу до глядачів. І ті – барабанний дріб – несли в
театр реквізит, допомагали робити декорації й приходили шити костюми абсолютно
безоплатно.

- Будівля нашого драматичного театру мала статус пам'ятки архітектури
державного значення і пам'ятки архітектури місцевого значення.

- Під час опитування маріупольців на питання \enquote{що для вас є центром міста}
абсолютна більшість відповіла \enquote{Театр}.

\enquote{Зустрінемось на драмі!} - казали ми один одному до 16 березня 2022 року.
Впевнена, що і після звільнення Маріуполя це місце ще довго носитиме таку
народну назву. Та поки зустрінемось, щоб згадати драм.

Сьогодні  16 березня, о 16.00, в фойе Національного академічного театру ім.
Івана Франка (пл. Івана Франка, 3) відкриється фотовиставка \enquote{Театр. Душа
Маріуполя}. 

Архівні фото – від періоду з відкриття маріупольського Драмтеатру до сьогодні,
коли від будівлі залишилися лише напівзруйновані стіни, світлини, які нагадують
про мирний час – акторів, вистав, інтер'єру будівлі, й фото часів блокади, коли
у театрі облаштували тимчасовий прихисток для людей під час обстрілів.

Зустрінемося на драмі, друзі! Я прийду.

%\ii{16_03_2023.fb.stomina_oksana.mariupol.1.zustrinemos_na_drami_druzi.cmt}
