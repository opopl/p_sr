% vim: keymap=russian-jcukenwin
%%beginhead 
 
%%file 25_05_2021.fb.fb_group.story_kiev_ua.3.staroje_foto
%%parent 25_05_2021
 
%%url https://www.facebook.com/groups/story.kiev.ua/permalink/1670043113192483/
 
%%author Киевские Истории
%%author_id fb_group.story_kiev_ua
%%author_url 
 
%%tags 
%%title Старое фото
 
%%endhead 
 
\subsection{Старое фото}
\label{sec:25_05_2021.fb.fb_group.story_kiev_ua.3.staroje_foto}
\Purl{https://www.facebook.com/groups/story.kiev.ua/permalink/1670043113192483/}
\ifcmt
 author_begin
   author_id fb_group.story_kiev_ua
 author_end
\fi

Anatol Zolotushkin
\url{https://www.facebook.com/groups/736908309839306/user/100000976367734/}

Приблизительно год назад со мной случилось маленькое чудо. Сын нашел на
странице киевлян старое фото, которое мы никогда не видали. Крайний справа там
мой тесть и его дедушка Писаренко Леонид Васильевич. Человек уникальной, но в
чем то обычной для советского человека судьбы. Отца, который был председателем
колхоза в украинском селе, репрессировали в 30-е годы. 

\ifcmt
  pic https://scontent-bos3-1.xx.fbcdn.net/v/t1.6435-0/s600x600/190347361_5509802935728864_3308999717864025019_n.jpg?_nc_cat=105&ccb=1-3&_nc_sid=825194&_nc_ohc=W7mb-FicmsMAX-lptIR&_nc_ht=scontent-bos3-1.xx&tp=7&oh=cdbc64d8cd4c95e2fee0d87e3252dea4&oe=60D23550
\fi

Когда началась война, Лёню не взяли в армию по молодости, но он убежал из дома,
чтобы воевать. Практически сразу попал в плен, сидел в концлагере. Когда
Красная армия освободила его, успел довоевать. После войны окончил в Киеве
техникум и женился на моей теще Маргарите Александровне. Его мать и сестры не
пожелали  общаться с невесткой-еврейкой. Жить им было негде. У тещиной мамы,
Анны Исаковны, была комнатушка в подвале. Он уехал с молодой женой в Воркуту и
работал механиком на угольной шахте. В Киев они приезжали на побывку к теще с
полными карманами денег. Не зря объектив фотографа запечатлел его в кафе \enquote{У
слоника} в Пионерском парке возле Владимирской горки.

Такой вот привет из прошлого и повод вспомнить хорошего человека.
