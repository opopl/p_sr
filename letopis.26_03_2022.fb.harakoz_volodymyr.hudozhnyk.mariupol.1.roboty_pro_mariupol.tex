%%beginhead 
 
%%file 26_03_2022.fb.harakoz_volodymyr.hudozhnyk.mariupol.1.roboty_pro_mariupol
%%parent 26_03_2022
 
%%url https://www.facebook.com/permalink.php?story_fbid=pfbid02BF8AJZiM66xnwmukRRYwwmoapBeKLSchry2g6fNrwiJcQgFDsb6Nu3X987YDY5ewl&id=100014630496277
 
%%author_id harakoz_volodymyr.hudozhnyk.mariupol
%%date 26_03_2022
 
%%tags mariupol,mariupol.art,isskustvo,kultura,hudozhnik,kartina,mariupol.pre_war
%%title Роботи про Маріуполь
 
%%endhead 

\subsection{Роботи про Маріуполь}
\label{sec:26_03_2022.fb.harakoz_volodymyr.hudozhnyk.mariupol.1.roboty_pro_mariupol}

\Purl{https://www.facebook.com/permalink.php?story_fbid=pfbid02BF8AJZiM66xnwmukRRYwwmoapBeKLSchry2g6fNrwiJcQgFDsb6Nu3X987YDY5ewl&id=100014630496277}
\ifcmt
 author_begin
   author_id harakoz_volodymyr.hudozhnyk.mariupol
 author_end
\fi

Мого Маріуполя - міста, де квітне акація, більше не існує. Ракові клітини
недодержави моксель, що намагається вкрасти нашу історію, з'їли здорове життя
міста.  Нема вже затишних вулиць старого Мариуполя - міста біля моря. Нема
вулиць з еклектикою кінця 18, 19 та початку 20 століть. Нема колись
провінційного затишного міста  з ароматом  травневого бузку та тонким запахом
квітучого винограду (спогади дитинства).

Переглянув свої роботи ї ось дещо знайшов  про Маріуполь та маріупольців.

\enquote{Поетка Валерія}, \enquote{Митець Володимир} ну і вулицями міста.

%\ii{26_03_2022.fb.harakoz_volodymyr.hudozhnyk.mariupol.1.roboty_pro_mariupol.cmt}
