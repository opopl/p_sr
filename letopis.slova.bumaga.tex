% vim: keymap=russian-jcukenwin
%%beginhead 
 
%%file slova.bumaga
%%parent slova
 
%%url 
 
%%author 
%%author_id 
%%author_url 
 
%%tags 
%%title 
 
%%endhead 
\chapter{Бумага}
\label{sec:slova.bumaga}

%%%cit
%%%cit_head
%%%cit_pic
%%%cit_text
Это день диагноза, день разрыва между реальностью и мечтами. День Конституции
в Украине напоминает вуду-обряд, когда люди каждый день, нарушающие все
основные положения этого документа, всем своим видом показывают, что эта
\emph{бумажка} имеет для них какое-то значение и вызывает у них какой-то трепет
и преклонение. Украина живет не по Конституции, которая изначально была
химерой, выросшей из борьбы Кучмы и Мороза, а по понятиям, которые нигде не
прописаны.  Именно это и есть главная проблема страны, потому что нельзя жить
по \emph{бумажным} понятиям, апеллировать к ним, но в реальности поступать
совершенно иначе.  Именно это порождает углубляющуюся шизофрению в украинском
обществе.  Поэтому День Конституции в Украине - это день диагноза, день разрыва
между реальностью и \enquote{мріями}. Это фиксация ключевой проблемы страны -
отрицание очевидного и выпячивание несуществующего. Это \emph{бумажный} замок
на песке, \emph{бумажный замок} несуществующих прав и обязанностей на зыбком
песке алчности, распила и самопожирания
%%%cit_comment
%%%cit_title
\citTitle{День Конституции в Украине напоминает вуду-обряд / Лента соцсетей / Страна}, 
Юрий Романенко, strana.ua, 28.06.2021
%%%endcit

