% vim: keymap=russian-jcukenwin
%%beginhead 
 
%%file 21_12_2021.stz.news.lnr.lug_info.1.konkurs_floristika
%%parent 21_12_2021
 
%%url https://lug-info.com/news/uchashiesya-i-pedagogi-predstavili-na-konkurs-sohranim-zhivuyu-el-bolee-tysyachi-rabot
 
%%author_id 
%%date 
 
%%tags 
%%title Жители ЛНР представили на конкурс "Сохраним живую ель!" более тысячи работ
 
%%endhead 
\subsection{Жители ЛНР представили на конкурс \enquote{Сохраним живую ель!} более тысячи работ}
\label{sec:21_12_2021.stz.news.lnr.lug_info.1.konkurs_floristika}

\Purl{https://lug-info.com/news/uchashiesya-i-pedagogi-predstavili-na-konkurs-sohranim-zhivuyu-el-bolee-tysyachi-rabot}

Жители ЛНР представили более тысячи творческих работ в рамках конкурса-выставки
новогодней флористики \enquote{Сохраним живую ель!}. Об этом сообщила
пресс-служба Министерства образования и науки ЛНР.

\ii{21_12_2021.stz.news.lnr.lug_info.1.konkurs_floristika.pic.1}

Состязание проводилось Республиканским центром эколого-натуралистического
творчества в следующих номинациях: \enquote{Стилизованная елка}, \enquote{Новогодняя
композиция}, \enquote{Творчество без границ}, \enquote{3D-плакат \enquote{В защиту елей!}} и
\enquote{Методическая копилка волшебства}. Его участниками стали воспитанники
дошкольных учреждений, школьники, студенты и педагоги. 

\ii{21_12_2021.stz.news.lnr.lug_info.1.konkurs_floristika.pic.2}

\enquote{Для участия поступило более 1000 работ. Конкурс-выставка стал популярным среди
обучающихся и педагогов всех образовательных организаций Луганской Народной
Республики}, - говорится в сообщении.

\ii{21_12_2021.stz.news.lnr.lug_info.1.konkurs_floristika.pic.3}

Работы конкурсантов оценивало жюри, в состав которого вошли специалисты
Министерства образования и науки ЛНР, Министерства природных ресурсов и
экологической безопасности ЛНР, Луганского центра народного творчества,
Республиканского центра научно-технического творчества, Республиканского центра
художественно-эстетического творчества и Республиканского центра
эколого-натуралистического творчества.

\enquote{Также республиканский природоохранный конкурс-выставка новогодней
флористики \enquote{Сохраним живую ель!} включает в себя благотворительную
акцию \enquote{Подарок другу}.  В рамках данной акции работы будут переданы
воспитанникам детских домов Луганской Народной Республики}, - добавили в
Минобразования.

Все желающие могут посетить выставку конкурсных работ до 25 января по адресу:
Луганск, улица Карла Либкнехта, 71.
