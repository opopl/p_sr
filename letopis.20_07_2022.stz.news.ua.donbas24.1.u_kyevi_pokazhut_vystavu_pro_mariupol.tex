% vim: keymap=russian-jcukenwin
%%beginhead 
 
%%file 20_07_2022.stz.news.ua.donbas24.1.u_kyevi_pokazhut_vystavu_pro_mariupol
%%parent 20_07_2022
 
%%url https://donbas24.news/news/u-kijevi-pokazut-vistavu-pro-mariupol
 
%%author_id demidko_olga.mariupol,news.ua.donbas24
%%date 
 
%%tags 
%%title У Києві покажуть виставу про Маріуполь
 
%%endhead 
 
\subsection{У Києві покажуть виставу про Маріуполь}
\label{sec:20_07_2022.stz.news.ua.donbas24.1.u_kyevi_pokazhut_vystavu_pro_mariupol}
 
\Purl{https://donbas24.news/news/u-kijevi-pokazut-vistavu-pro-mariupol}
\ifcmt
 author_begin
   author_id demidko_olga.mariupol,news.ua.donbas24
 author_end
\fi

\ifcmt
  ig https://i2.paste.pics/15feaefd9cd99d53c1b75a29bd34510b.png
  @wrap center
  @width 0.9
\fi

\begin{center}
\Large\em\bfseries\color{blue}
28 липня Театр авторської п'єси \enquote{Концепція} у Театрі на Подолі представить
виставу за документальними подіями \enquote{Обличчя кольору війна}
\end{center}

Презентаційний показ фрагменту документальної вистави \enquote{Обличчя кольору війна}
відбувся 24 червня. Цим показом був вражений і міський голова Києва. 

\begin{leftbar}
	\begingroup
		\bfseries
\qbem{Дуже емоційна вистава. Дівчата та хлопці не грали. Вони розказували, що
пережили}, — підкреслив Віталій Кличко.
	\endgroup
\end{leftbar}

\ii{20_07_2022.stz.news.ua.donbas24.1.u_kyevi_pokazhut_vystavu_pro_mariupol.pic.1.klichko}

Наразі театр готується показати повну виставу, прем'єра якої відбудеться \emph{28
липня}. І режисер, і актори дуже хвилюються, адже вони розповідатимуть історію,
яку пережив кожен з них, але на найбільш сучасній сцені Києва, тому це доволі
відповідальне та складне завдання. Художній керівник і режисер театру \textbf{Олексій
Гнатюк} розповів, які події охоплює спектакль і наголосив, що він зовсім не про
політику.

\begin{leftbar}
	\begingroup
		\bfseries
\enquote{У виставі ми розповідаємо правду. Те, що ми пережили з 24 лютого до 27
березня. Це документальний спектакль про маріупольців, про наш біль і
наші втрати}, — зазначив Олексій Гнатюк.
	\endgroup
\end{leftbar}

\ii{20_07_2022.stz.news.ua.donbas24.1.u_kyevi_pokazhut_vystavu_pro_mariupol.pic.2}

Завдяки виставі \enquote{Обличчя кольору війна}, кожен з акторів може розповісти свою
історію. Як на початку війни намагалися жити і займатися звичними справами.
Однак \enquote{прильотів} ставало все більше, на п'ятий день вимкнули світло. Були
змушені під обстрілами готувати їжу. Як ховалися у підвалах і втрачали
знайомих.

\begin{leftbar}
	\begingroup
		\bfseries
\qbem{Ця вистава досить емоційна і викликає в мене різні відчуття. З однієї
сторони я радію, що знову маю можливість вийти на сцену і можу показати
людям, як жили і що переживали маріупольці, показати якомога більшій
кількості людей. Але з іншого боку, дуже складно, адже ми не граємо, а
щоразу знову переживаємо ті болісні моменти, які з нами відбувалися в
нашому рідному місті. Ще не було ні однієї репетиції, щоб ми не
плакали. Плачуть всі: і діти, і жінки, і чоловіки. Найстрашніше те, що
вистава заснована на реальних подіях, але краще б це було просто
історією}, — поділилася актриса Катерина Калмикова.
	\endgroup
\end{leftbar}

\ii{20_07_2022.stz.news.ua.donbas24.1.u_kyevi_pokazhut_vystavu_pro_mariupol.pic.3}

У виставі задіяно 12 маріупольських акторів. Більшість з них перебували у
блокадному Маріуполі досить довгий час. Наразі\par\noindent Олексій Гнатюк робить все
можливе, щоб у акторів були сприятливі умови праці.

\begin{leftbar}
	\begingroup
		\bfseries
\qbem{Дуже багато організаційних питань необхідно вирішити. Все починаємо з
нуля. У акторів немає житла. Для роботи не вистачає технічного
оснащення. Але ми робимо все можливо, щоб наш театр жив і продовжував
існування вже у столиці}, — наголосив режисер. 
	\endgroup
\end{leftbar}

Двоє акторів (\textbf{Дмитро Гриценко} і \textbf{Ганна Лафазан}) тільки нещодавно приїхали і
зараз ми будемо проводити репетиції вже з ними. Сподіваємося, що вистава знайде
відгук у глядачів. Для нас це дуже важливо. Також у виставі будуть музичні і
танцювальні номери. 2 музиканти (\textbf{Микита Леонтьєв} та \textbf{Людмила Гричаненко}) з
Маріуполя і 1 (\textbf{Андрій Рахманінов}) з Києва.

\ii{20_07_2022.stz.news.ua.donbas24.1.u_kyevi_pokazhut_vystavu_pro_mariupol.pic.4.gitara}

\begin{leftbar}
	\begingroup
		\bfseries
\qbem{Наш режисер попросив нас написати монологи про пережиті події. Дуже
важко стримати свої емоції. Кожна репетиція повертає нас до тих
страшних днів, які ми пережили у Маріуполі. Але ми мусимо через
мистецтво показати те, що відбувалося у місті}, — розповів актор \emph{Євген
Сосновський}.
	\endgroup
\end{leftbar}

У Маріуполі Євген Сосновський виступав в колективі Театральної артілі
\enquote{ДрамКом}. Попри те, що в \enquote{Концепції} розпочав роботу порівняно нещодавно,
наголошує, що і з режисером, і з акторами працювати дуже комфортно. 

\begin{leftbar}
	\begingroup
		\bfseries
\qbem{Приємно вражає, наскільки чітко і злагоджено працює колектив. Водночас
організаційна робота, яку проводить режисер театру Олексій Гнатюк, є
колосальною}, — підкреслив Євген Сосновський. 
	\endgroup
\end{leftbar}

\ii{20_07_2022.stz.news.ua.donbas24.1.u_kyevi_pokazhut_vystavu_pro_mariupol.pic.5}

До колективу приєдналася і тринадцятирічна \textbf{Софія Алагірова}, яка раніше
виступала в Маріупольському Театрі ляльок. Для дівчинки дуже важливо знову
поринути в творче життя, адже це дуже відволікає від негативних думок. Мати
Софії, Лідія Алагірова активно підтримує доньку і вважає виставу \enquote{Обличчя
кольору війна} дуже актуальною. 

\begin{leftbar}
	\begingroup
		\bfseries
\qbem{Те, що відбувається творчий процес навіть, присвячений цим страшним
подіям, для маріупольців це, звичайно, новий подих. Тому що з бідою сам
на сам залишатися важко. А це можливість спробувати висловитися,
викричатися, звільнитися. Я радію, що у доньки з'явилася така
можливість. Коли я дивилася вперше, відчула страшний біль і нічого крім
болю та порожнечі. Дуже страшно дивитися, але я вважаю, що таку виставу
потрібно подивитися всім, особливо тим, хто й досі живе осторонь від
цих проблем}, — зазначила Лідія Алагірова. 
	\endgroup
\end{leftbar}

\ii{20_07_2022.stz.news.ua.donbas24.1.u_kyevi_pokazhut_vystavu_pro_mariupol.pic.6.pogljad}

У спектаклі заплановано показати світлини Євгена Сосновського, які йому вдалося
вивезти з Маріуполя. З виставою \enquote{Обличчя кольору війна} Театр авторської п'єси
\enquote{Концепція} планує гастролі як в українських містах, так і за кордоном.

\ii{20_07_2022.stz.news.ua.donbas24.1.u_kyevi_pokazhut_vystavu_pro_mariupol.pic.7.plakat}


Нагадаємо, раніше Донбас24 розповідав про \href{https://donbas24.news/news/lugancanin-recidivist-pisov-voyuvati-za-lnr-shhob-zarobiti-na-vstavni-zubi}{луганчанина-ре\hyp{}цидивіста, який пішов
воювати за \enquote{лнр}, щоб заробити на вставні зуби}.

ФОТО: Євгена Сосновського та з відкритих джерел.

\ii{insert.author.demidko_olga}
%\ii{20_07_2022.stz.news.ua.donbas24.1.u_kyevi_pokazhut_vystavu_pro_mariupol.txt}
