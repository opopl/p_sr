% vim: keymap=russian-jcukenwin
%%beginhead 
 
%%file 30_12_2020.news.ua.strana.1.uhazhivali_vsu_noch_mediki
%%parent 30_12_2020
 
%%url https://strana.ua/news/309538-stala-izvestna-istorija-foto-na-kotorom-mediki-spjat-na-polu-vozle-bolnoho-covid-19.html
 
%%author 
%%author_id 
%%author_url 
 
%%tags 
%%title Ухаживали всю ночь. В интернете прославилось фото из России с уставшими медиками на полу у кровати пациентки
 
%%endhead 
 
\subsection{Ухаживали всю ночь. В интернете прославилось фото из России с уставшими медиками на полу у кровати пациентки}
\label{sec:30_12_2020.news.ua.strana.1.uhazhivali_vsu_noch_mediki}
\Purl{https://strana.ua/news/309538-stala-izvestna-istorija-foto-na-kotorom-mediki-spjat-na-polu-vozle-bolnoho-covid-19.html}
\index[rus]{Коронавирус!Россия!Уставшие медики на полу у кровати пациентки, 30.12.2020}

\ifcmt
  pic https://strana.ua/img/article/3095/stala-izvestna-istorija-38_main.jpeg
	caption Фото с лежащими на полу медиками стала популярной в сети. Фото: facebook.com
  width 0.4
\fi

Интернет облетел снимок, на котором одетые в защитные костюмы обессиленные
медики спят прямо на полу возле кровати подключенного к аппарату ИВЛ больного
Covid-19.\Furl{https://strana.ua/news/309049-samoe-neobychnoe-posledstvie-koronavirusa-eto-zapakh-sery-ili-ryby.html} Оказалось, что фото сделано в России - в коронавирусном отделении в
Сосновом бору Ленинградской области. На него попали трое медиков из ночной
смены, которые прилегли на пол после того, как помогли больной Covid-19
пациентке преодолеть приступ панической атаки.

Об этом пишет "Газета.Ru".\Furl{https://www.gazeta.ru/social/news/2020/12/29/n_15428342.shtml?fbclid=IwAR3cPcUCBVLEVGZ0sVLSRwOrZTGFMBgv9aMmT5vLajh4NQOxwykTpxr1idQ&updated}

Один из героев фото - медик-студент Расуль - рассказал, что та ночная смена для
всех троих оказалась незапланированной. До этого они уже отработали полный
день.

"Так получилось, что пациентку охватила паническая атака, это можно сравнить с
утоплением, когда человек подсознательно пытается ухватиться за какие-то
предметы, барахтается. Она думала, что кислородная маска мешает дышать,
отрывала ее и катетеры. За ней нужно было следить. В два часа ночи в рабочей
беседе нам написали, что нужна помощь", - рассказал Расуль.

По его словам, он и его коллеги не спали на полу, а просто прилегли,
вымотавшись физически и морально. Но пациентку им удалось успокоить. Оставалось
ждать утра и следить за состоянием женщины, контролируя монитор, маску и пульс
пациентки.

"Мы под утро прилегли на холодный пол, потные в этих костюмах, которые не
дышат, чтобы как-то ощутить прохладу, чтобы хоть как-то отдохнуть", - говорит
медик. 

Расуль уточнил, что после работы в ночь ему снова пришлось выйти на смену днем,
поэтому отдохнуть получилось только к вечеру. Медик добавил, что пациентке,
которую они "охраняли", стало лучше. 

Ранее "Страна" писала о том, что Шмыгаль пообещал врачам европейские зарплаты в
2023 году.\Furl{https://strana.ua/news/307606-denis-shmyhal-v-2023-hodu-na-meditsinu-budet-vydeleno-5-ot-vvp.html}

Также мы сообщали, что в Киеве пьяный мужчина с гитарой ворвался в операционную
Больницы скорой помощи и напал на врачей.\Furl{https://kiev.strana.ua/307485-pjanyj-muzhchina-v-kieve-vorvalsja-v-operatsionnuju-s-hitaroj.html}


