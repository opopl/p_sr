% vim: keymap=russian-jcukenwin
%%beginhead 
 
%%file 25_04_2022.stz.pc.ua.dou.1.geroizm_pogljad_kapellana.4.tro_motyzhyn
%%parent 25_04_2022.stz.pc.ua.dou.1.geroizm_pogljad_kapellana
 
%%url 
 
%%author_id 
%%date 
 
%%tags 
%%title 
 
%%endhead 

\subsubsection{Історія від учасника ТрО в Мотижині}

Це історія мого знайомого, котрий зі 28 лютого по 4 березня захищав Мотижин від
окупантів.

Коли 28 лютого колони орків їхали з Житомирської траси, герої історії
зустрічали їх в чотирьох. Він, два його сини і їх друг. Вони розуміли, що це
дорога в один кінець. В них було декілька автоматів та один гранатомет.

Але щось сталося і окупанти вирішили їхати не по тій дорозі, а з іншої сторони
лісу. Звідкись понаїхали селяни з бензопилами та блоками з бетону. Вони
заблокували дорогу і тім самим загальмували колону. 7 одиниць техніки русні
залишилось в багнюці. Але найцікавіше почалось ввечорі, коли Герой історії
почав корегувати вогонь нашої артилерії по рашистам.

В ці 4 дні їх валили Піони, Байрактари, ССО з Джавелінами, винищувачі та
гелікоптери, завдяки співпраці з військовими героя цієї історії. Коли там вже
неможливо було залишатись, він зі своєю сім'єю евакуювався «дорогою життя.»

На фото нижче сім'я, котру вбили окупанти. Саме вони з героєм цієї історії
зупиняли колони рашистів, коли вони йшли на Київ. Якби вони там не зупинились,
моє місто Вишневе було би наступним. Але ми були вже готові, зустріти їх так,
що війна в Чечні показалася би їм раєм!
