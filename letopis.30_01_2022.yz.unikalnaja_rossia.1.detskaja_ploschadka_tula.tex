% vim: keymap=russian-jcukenwin
%%beginhead 
 
%%file 30_01_2022.yz.unikalnaja_rossia.1.detskaja_ploschadka_tula
%%parent 30_01_2022
 
%%url https://zen.yandex.ru/media/id/5e52a45be977e25b8eec5de4/primer-zdravogo-patriotizma-detskaia-ploscadka-s-figurkami-mestnogo-narodnogo-promysla-v-tule-61f66c395b8dd05028cc8800
 
%%author_id yz.unikalnaja_rossia
%%date 
 
%%tags deti,patriotizm,rossia,tula.rossia
%%title Пример здравого патриотизма: детская площадка с фигурками местного народного промысла в Туле
 
%%endhead 
 
\subsection{Пример здравого патриотизма: детская площадка с фигурками местного народного промысла в Туле}
\label{sec:30_01_2022.yz.unikalnaja_rossia.1.detskaja_ploschadka_tula}
 
\Purl{https://zen.yandex.ru/media/id/5e52a45be977e25b8eec5de4/primer-zdravogo-patriotizma-detskaia-ploscadka-s-figurkami-mestnogo-narodnogo-promysla-v-tule-61f66c395b8dd05028cc8800}
\ifcmt
 author_begin
   author_id yz.unikalnaja_rossia
 author_end
\fi

Гуляя по улицам Тулы, я обратил внимание на одну детскую площадку, которая
значительно отличается от своих \enquote{коллег}. Дело в том, что на ней установлено 11
композиций из стеклопластика, расписанных в стиле Филимоновской глиняной
игрушки.

Тульский край богат на различные бренды и символы. И это не только пряник,
самовар, оружие и гармонь. Это ещё и Филимоновская глиняная игрушка - древний
промысел родом из окрестностей города Одоев Тульской области.

\enquote{Огромные игрушки} и расположились на этой детской площадке.

Площадка появилась несколько лет назад в Октябрьском сквере на пересечении улиц
Октябрьской и Пузакова. Это довольно далеко от центра, ближе к северным
окраинам города, поэтому не все туристы, посещающие Тулу, доезжают или доходят
сюда.

Зато, если вы едете в город на машине из Москвы, площадка предстанет перед вами
значительно раньше центра - Московское шоссе как раз переходит в Туле в
Октябрьскую улицу.

\ii{30_01_2022.yz.unikalnaja_rossia.1.detskaja_ploschadka_tula.pic.1}

Персонажи Филимоновской игрушки - это различные животные, как фантастические,
так и вполне реальные. Кони, коты, козы, свиньи, рогатые, петухи и даже
черепахи. Также мастера запечатлевали и людей, иногда верхом на лошадях или
управляющих повозкой.

Традиционно цвета этого промысла всегда были яркими и наносились на белую
фигурку анилиновыми красками. 

Зима - не самое удачное время для посещения площадки, но как тут пройти мимо:)

Рассказывают, что промыслу филимоновской игрушки более 700 лет. Зародился он в
деревне Филимоново Одоевского района Тульской области, по которой и получил
название. 

\begin{zznagolos}
По легенде сама деревня была названа в честь деда Филимона - беглого
каторжника, знакомого с гончарным искусством. Он и научил жителей изготавливать
игрушки из местной чёрно-синей глины \enquote{синики}, которая встречается только в
окрестностях Филимоново. При обжиге глина становится светлой.	
\end{zznagolos}

\ii{30_01_2022.yz.unikalnaja_rossia.1.detskaja_ploschadka_tula.pic.2}

Идея с \enquote{Филимоновской площадкой} - потрясающая! Гуляющие и играющие на ней
ребятишки всегда могут лишний раз узнать про свой родной край, познакомиться
поближе с его символами и традициями. Такая же возможность есть и у взрослых,
пришедших сюда со своими детьми. Рядом с площадкой есть стенд с информацией о
промысле.

Кто знает, может после посещения этой площадки у кого-то появится желание
побывать в музее Филимоновской игрушки в Одоеве, или даже посвятить
исследованию и изучению родного региона свою жизнь.

Такая площадка - наглядный пример здравого и правильного патриотизма. 

Местные жители говорят, что в тёплое время года на площадке почти всегда много
детей. Филимоновские фигурки давно приглянулись многим юным тулякам.

Совсем рядом с площадкой находится скульптура, знакомящая с другими символами
Тульского региона - \enquote{Тульское чаепитие}. О ней я недавно писал:
\href{https://zen.yandex.ru/media/id/5e52a45be977e25b8eec5de4/tulskoe-chaepitie-uiutnaia-skulptura-v-kotoroi-izobrajeny-srazu-4-simvola-tuly-61edc81c1b287f5c6d20aea5}{%
\enquote{Тульское чаепитие}: уютная скульптура, в которой изображены сразу 4 символа Тулы, %
zen.yandex.ru, Уникальная Россия, 24.01.2022%
}

Ещё про Тулу:

\href{https://zen.yandex.ru/media/id/5e52a45be977e25b8eec5de4/pamiatnik-tulskomu-prianiku-odin-iz-samyh-sladkih-pamiatnikov-rossii-61ed6a1c9a54d91fbe111182}{%
Памятник тульскому прянику - один из самых сладких памятников России, %
zen.yandex.ru, Уникальная Россия, 23.01.2022%
}

\href{https://zen.yandex.ru/media/id/5e52a45be977e25b8eec5de4/edinstvennyi-pamiatnik-tesce-v-rossii-na-nem-esce-i-krest-zachemto-narisovali-61e1d8dbc69d7367a14a1d98}{%
Единственный \enquote{памятник тёще} в России. На нём ещё и крест зачем-то нарисовали, %
zen.yandex.ru, Уникальная Россия, 15.01.2022%
}

\href{https://zen.yandex.ru/media/id/5e52a45be977e25b8eec5de4/i-eto-tula-nu-i-nu-5e52ae1f5c1f4e253331ed4e}{%
И это Тула? Ну и ну!, %
zen.yandex.ru, Уникальная Россия, 23.02.2020%
}

Понравилась статья? Спасибо, я старался! Ставьте, дорогие друзья, лайк и
подписывайтесь на мой канал. Надеюсь, дальше будет только интереснее!

Отдельное спасибо, что дочитали до конца!

\ii{30_01_2022.yz.unikalnaja_rossia.1.detskaja_ploschadka_tula.cmt}
