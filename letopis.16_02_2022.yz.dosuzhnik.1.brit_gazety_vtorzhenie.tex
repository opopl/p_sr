% vim: keymap=russian-jcukenwin
%%beginhead 
 
%%file 16_02_2022.yz.dosuzhnik.1.brit_gazety_vtorzhenie
%%parent 16_02_2022
 
%%url https://zen.yandex.ru/media/dosuznik/ne-chitaite-pered-obedom-britanskih-gazet-620d5a6fff2d5a753e9652df
 
%%author_id yz.dosuzhnik
%%date 
 
%%tags rossia,ugroza,ukraina,vtorzhenie
%%title Не читайте перед обедом британских газет
 
%%endhead 
 
\subsection{Не читайте перед обедом британских газет}
\label{sec:16_02_2022.yz.dosuzhnik.1.brit_gazety_vtorzhenie}
 
\Purl{https://zen.yandex.ru/media/dosuznik/ne-chitaite-pered-obedom-britanskih-gazet-620d5a6fff2d5a753e9652df}
\ifcmt
 author_begin
   author_id yz.dosuzhnik
 author_end
\fi

Сегодня был день ожиданий. Американские и британские СМИ уточнили дату начала
российско-украинской войны, назначив её на 16 февраля. А английская газета The
Sun даже уточнила время атаки – 3 часа ночи по украинскому времени.

Видимо, газете не даёт покоя слава знаменитой соотечественницы Агаты Кристи,
написавшей детектив «Объявлено убийство». Но есть и другая версия: подвёл
Джеймс Бонд, предоставивший недостоверную информацию. Сегодня появилось очень
оригинальное опровержение этого «точного» прогноза:

\begin{zznagolos}
Холодное ясное небо над Киевом, где люди приготовились к воздушному налёту,
оставалось безмолвным.	
\end{zznagolos}

А как всё красиво было распланировано. CNN за несколько часов до
предполагаемого нападения даже название операции обнародовало - «Астролябия
Будды». Как такая крупная операция без названия!

\ifcmt
  tab_begin cols=2,no_fig,center

  ig https://avatars.mds.yandex.net/get-zen_doc/5352530/pub_620d5a6fff2d5a753e9652df_620d5b6b06cdcb4b8444819c/scale_1200
	ig https://avatars.mds.yandex.net/get-zen_doc/4361269/pub_620d5a6fff2d5a753e9652df_620d5bd69d79e264a3682335/scale_1200

  tab_end
\fi

Газетчики The Sun и на этом не остановились. Появилась статья о дальнейших
планах Путина с таким заголовком: «Есть опасность, что Путин, заняв Украину,
вторгнется в Литву, Латвию и Эстонию с захватом земель в стиле Гитлера». Ни
больше, ни меньше! Речь идёт о ядерной войне за Прибалтику.

Можно предположить, что к нападению готовилась чуть ли не половина населения
земли. Другая, включая меня, активно следила за событиями Олимпиады. Мне
хочется первой половине пожелать, немного перефразировав цитату профессора
Преображенского из «Собачьего сердца»: «Не читайте перед обедом британских
газет!».

Но непредсказуемая Россия в очередной раз наплевала на последнее британское
предупреждение и не явилась на войну! Крах всех планов. Вот и министр обороны
королевства вынужден объявить об отмене планируемого отпуска. Сколько теперь
придётся ждать?

Да и Украина в растерянности: никто не напал, дипломаты разъехались. Что
дальше? Глава МИД Дмитрий Кулеба просит провести совместное заседание
Постоянного совета ОБСЕ и Форума по сотрудничеству в области безопасности ОБСЕ
в течение 48 часов из-за "действий Москвы" у границы:

Мы еще раз призываем Россию учесть законные опасения государств-участников и
использовать инструменты ОБСЕ для разрядки напряженности, вызванной ее военными
действиями вблизи границы с Украиной. Заявлений об отводе войск недостаточно.
Нам нужна прозрачность и факты.

Двусмысленное заявление. Кто ведёт военные действия вблизи украинской границы:
Россия или ОБСЕ? Даже если предположить, что заявление относится к нашей
стране, непонятно, против кого мы воюем вблизи границ Украины? Серьёзнее надо
подходить к документам! А вот прозрачность и факты нам тоже очень нужны!

\ifcmt
  ig https://avatars.mds.yandex.net/get-zen_doc/5262161/pub_620d5a6fff2d5a753e9652df_620d5ce654590b2b43a3336c/scale_1200
  @wrap center
  @width 0.8
\fi

А вот заявление (хоть и в шутливой форме) представителя МИД России Захаровой я
поддерживаю:

\begin{zznagolos}
Просьба к средствам массовой дезинформации США и Британии: Bloomberg, The New
York Times, The Sun и т.д. — огласите график наших \enquote{вторжений} на
предстоящий год, хотелось бы спланировать отпуск.	
\end{zznagolos}

А раз война закончилась не начавшись, можно дальше следить за Олимпиадой. И актуальный анекдот:

\begin{zznagolos}
\obeycr
В Израиле:
- Беня, а шо там на Украине происходит?
- Украина с Россией воюет.
- И шо, как идут дела?
- Ну, Украина таки потеряла 2 миллиона гражданских, полуостров Крым, несколько вертолётов, Ил-76, пару десятков бронетранспортёров, убито много военных, и ещё на грани перехода к России две большие области на востоке.
- А у русских шо?
- Ты таки не поверишь, они вообще на войну не явились.	
\restorecr
\end{zznagolos}

Хочу сделать заявление.

В феврале Россия может напасть только 23. За эту дату выступают женщины, не
успевшие купить подарки мужикам. Надо перенаправить их ярость! Да и подумывают
о трофеях к 8 марта! Шутка!
