% vim: keymap=russian-jcukenwin
%%beginhead 
 
%%file slova.istoria
%%parent slova
 
%%url 
 
%%author 
%%author_id 
%%author_url 
 
%%tags 
%%title 
 
%%endhead 
\chapter{История}

В українській \emph{історії} цей міф про спартанців також проявився. Відомою
стала \emph{історія} з козацьких воєн, розказана Миколою Костомаровим, який
своєю чергою переповів німецького хроніста Йоахіма Пасторія та польського
шляхтича Станіслава Освенцима. Під час битви під Берестечком сталася цікава
подія, яку хроністи описали приблизно так: «Одна ватага козацька, зложена з
двох або трьох сот козаків, зробивши засіку на однім острівці, ставила мужній
опір так відважно, що Потоцький велів їм обіцяти життя, коли піддадуться. Але
вони не схотіли того прийняти. На знак повного відречення викинули з чересів
своїх гроші у воду, і так сильно почали відбиватися від наших, що кінець кінцем
піхота мусила на них гуртом наступати. А хоч розірвала (їх укріпленнє) і
розігнала з острову — але вони уступилися на болота, не хотячи піддатись, так
що там їх кожного доводилось добивати. А оден з них, добившися до човна, на
очах короля і всього війська дав докази мужності не хлопської! Кілька годин
відбивався він з того човна косою, зовсім не зважаючи на стрільбу — котра чи то
припадком його не трапляла — чи такий твердий був, що кулі його не брали. Аж
оден Мазур з Ціхановецького повіту, розібравшися на голо і по шию бредучи,
насамперед косою вдарив, а потім оден солдат пробив посередині пікою або списом
і добив його таким чином — на велику потіху і вдоволеннє короля, що довго
дивився на сю траґедію»,
\citTitle{Як метафори та міфи керують нашою уявою}, Богдан Шумилович, zaxid.net, 09.06.2021

Люди люблять виразні \emph{історії}, особливо якщо вони драматичні. Такою є \emph{історія}
про 300 спартанців, які захистили батьківщину від більшої кількісно армії
персів. Проте якщо ви дивилися фільм Зака Снайдера, то могло виникнути
відчуття, що перед вами радше ідеологічний соцреалізм, історія про боротьбу
американців зі злим Іраном, а не сюжет про давнє минуле.  У сонячний день
серпня 480 року до н. е. грецьке місто-держава Спарта святкувало одне з
найважливіших фестивалів року, тижневе свято «Карнея», яке було даниною поваги
Аполлону Карнейосу, богу зграй та стад домашньої худоби. На час празнику було
заборонено брати участь у будь-якій війні, щоб бог не зіпсував джерела їжі на
наступний рік. Зазвичай це не було проблемою, але тут з'явився перський цар
Ксеркс, що зібрався помститися за свого батька, який 10 років тому безуспішно
намагався захопити Грецію,
\citTitle{Як метафори та міфи керують нашою уявою}, Богдан Шумилович, zaxid.net, 09.06.2021

Люди схильні вірити у все, що заманеться – брехливим каналам Медведчука чи
«чесним» медіа Порошенка, незрозумілим сайтам типу «Страна», які поширюють
фейки та міфи. Чому, запитаєте ви? Дуже просто: реальність не має смислу,
\emph{історія} – це теж набір беззмістовних сюжетів. Смислу потоку \emph{історії} надають
\emph{історики} чи письменники, а в нашому світі – кінематографісти чи навіть блогери.
Хто створить більш переконливу картинку реальності, той і перемагає в битві за
уяву. І ця картинка не мусить бути правдивою. Окрім того, люди схильні
сприймати світ через стереотипні історії, і якщо ми прийняли якийсь міф чи
метафору з раннього віку, то її дуже складно змінити. Іншими словами, коли я
бачу подію чи якийсь феномен у реальності, я його співвідношу з уже наявним у
моїй голові образом. Себто коли польський шляхтич бачив, як б'ється на смерть
ватага козаків, то він побачив \emph{античну історію}, яку засвоїв ще дитиною від
своїх вчителів. Ми розуміємо реальність лише тоді, коли вона має для нас смисл!,
\citTitle{Як метафори та міфи керують нашою уявою}, Богдан Шумилович, zaxid.net, 09.06.2021

Новое время родило когорту историков-самоучек, которые умеют выжать сенсацию
нового знания буквально из одного документа. Даже целые книги пишут. Вот, к
примеру, наткнулся на очередной тг-канал с секретами тайных архивов кейджиби. А
там выложили уникальное свидетельство подлых замыслов Империи Зла - мишени
Осоавиахима. Враг на них носит каску Броди (Mk III), следовательно \enquote{в СССР
учились стрелять по британцам}. Но камон, друзья, такие же каски носили в
армии США. Значит в СССР готовились стрелять по американцам. А форма на парне
похожа на немецкую, так что СССР готовился первым напасть на Германию, а
британцев косить уже потом. И самое главное: в конце 1930-х годов в РККА было
решено создать стальной шлем для частей ПВО и гражданской обороны. За образец
взяли каску Броди и с 1938 года начали делать ее на Ленинградском металлическом
заводе.  Следовательно, в СССР готовились стрелять по своим! Страшная правда
про заградотряды ПВО раскрыта,
\citTitle{Новое время родило когорту историков-самоучек}, Дмитрий Заборин, strana.ua, 10.06.2021


Всім відомо, що \emph{історія} не знає умовного способу, що минуле взагалі краще не
чіпати окрім як з виховною чи пізнавальною метою. Всі знають і все одно
регулярно граються в \emph{альтернативну історію}. Бо це, зрештою, не так про минуле,
як про сьогодення. Яким би воно могло бути, \enquote{якби} – так, це закляття чарує хай
там що, надто в суспільствах, що ніяк не виборсаються з перехідного періоду.
Ось типова мрія про минуле: Галицько-Волинське князівство поступово стає одним
з наймогутніших східноєвропейських гравців, укладає кілька вигідних
регіональних пактів, звільняє українські землі від Орди і перетворюється на
Королівство Русь. Місцеве самоврядування та народні вольності посилюються, так
що в середині ХІХ століття Русь стає конституційною монархією і входить у
новітню добу цивілізованою високорозвиненою демократією,
\citTitle{Якби | Газета «День»}, Дмитро Десятерик, day.kyiv.ua, 09.06.2021


%%%cit
%%%cit_pic
%%%cit_text
Думаю, пора уже американскому ВПК платить России \enquote{откат}, ведь без
разрекламированной \enquote{российской угрозы} вооружение США перестали бы покупать и
миллионы рабочих мест в США пришлось бы сокращать))) Для Украины данная \emph{история}
показательна тем, что налаженное производство самолетов марки \enquote{Антонов}, это 80
000 рабочих мест НА УКРАИНЕ, а не в США или ЕС.  Об этом прекрасно знали и в ЕС
и в США, по этому \enquote{наши партнеры} руками продажных шкур во власти \enquote{Антонов}
практически уничтожили. Зато у нас теперь благодаря \enquote{нашим партнерам}, рабочие
места массово создают в зоне боевых действий на Донбассе, в виде \enquote{зарплат}
солдатам контрактникам за убийства украинцев
%%%cit_comment
%%%cit_title
\citTitle{Forbes призывает Европу срочно покупать американские противолодочные самолеты}, 
Дмитрий Василец, strana.ua, 14.06.2021
%%%endcit

%%%cit
%%%cit_pic
\ifcmt
  pic https://gdb.rferl.org/957CC1E1-23E0-4D73-B1CB-B9DDAA77B5EE_w1597_n_r1_st.jpg
	caption Остромирове Євангеліє – одна з найвидатніших пам’яток письменства періоду України-Русі. Переписане у 1056–1057 роках. Пам'ятка української мови
\fi
%%%cit_text
Перша міфологема називає тисячолітній рукопис «культурным достоянием России»
відповідно до старої імперської схеми: Київська Русь – колиска трьох братніх
народів, старший брат – російський народ, а відтак уся спадщина тисячолітньої
Русі – «достояние России».  На відміну від ідеологічних схем, \emph{історичні факти}
та наукові висновки у цій темі свідчать про інше. А саме: переписувач
Остромирового Євангелія – дяк Григорій – був киянином, замовник – Остромир –
походив з київського боярського роду і переписували та оздоблювали славетний
манускрипт у Києві, в 11-му столітті – тобто за великокнязівської доби
української історії. Цей період в \emph{офіційній історії} нашої країни сьогодні
називають Україна-Русь або Київська держава. Таку назву запровадив видатний
\emph{історик} Михайло Грушевський
%%%cit_comment
%%%cit_title
\citTitle{Чому написане у Києві Остромирове Євангеліє – «достояние России»?},
Ірина Костенко; Ірина Халупа, radiosvoboda.org, 13.06.2021
%%%endcit

%%%cit
%%%cit_pic
%%%cit_text
У Росії тих років, років молодості Путіна, було прийнято ігнорувати цілі пласти
\emph{української історії} або оголошувати їх частиною \emph{історії} російського народу. Так
і виходило, що на території Радянського Союзу єдиним \enquote{повноцінним} державним
утворенням була саме Росія. А інші народи, якщо і створювали держави, так потім
повинні були шукати в Росії захисту від ворога. І вже тому будь-які спроби
відділення від Російської імперії, а потім від Радянського Союзу, сприймалися
носіями таких поглядів як прояв справжньої невдячності
%%%cit_comment
%%%cit_title
\citTitle{Путін знову береться за історію України}, 
Віталій Портников, gazeta.ua, 14.06.2021
%%%endcit

%%%cit
%%%cit_head
%%%cit_pic
%%%cit_text
Таким образом, мы видим, что \emph{история} католической экспансии не закончена. Она
не может быть закончена, пока не будет достигнута главная её цель – отрыва
части русских от русского мира на всех уровнях, раздробления и ослабления
России 
%%%cit_comment
%%%cit_title
\citTitle{Католическая экспансия на русские земли: история и современность}, 
Светлана Пикта, teleskop.media, 14.06.2021
%%%endcit

%%%cit
%%%cit_head
%%%cit_pic
%%%cit_text
Страна, которая не имеет \emph{истории} — не имеет будущего. Страна, которая живет
прошлым, никогда не увидит ни настоящего, ни будущего. Страна, у которой нет
\emph{исторических корней}, будет оккупантом, будет поглощать и мимикрировать,
стараясь, стать похожей на народ, нацию, страну.  У Российской Федерации
\emph{история} начинается с 12 июня 1990 года. У СССР с 30 декабря 1922 года. \emph{История}
Российской Империи начинается с 2 ноября (22 октября) 1721 г. Далее \emph{история}
Московии, московско-хазарского улуса и московских болот. И, что самое главное,
это отдельная \emph{история}, отдельных государств, а не \enquote{\emph{история} России}, как пишут
российские \enquote{\emph{историки}}
%%%cit_comment
%%%cit_title
\citTitle{Страна без исторических корней всегда будет оккупантом}, 
Олена Степова, news.obozrevatel.com, 17.06.2021
%%%endcit

%%%cit
%%%cit_head
%%%cit_pic
%%%cit_text
Современная Россия отчаянно старается написать себе \emph{тысячилетнюю историю},
опираясь то на правопреемничество СССР, то на правопреемничество монархии, то и
во все, смешивая в кучу коммунизм и монархизм, не зря же, потомок царя Николая
лобызал дуло пушки на крейсере \enquote{Аврора} с криками \enquote{не забудем-повторим}. В
общем, лучшее в мире советское образование, дает о себе знать. СССР создавал
симулякры, особенно \emph{исторические}. Результат этого \emph{исторического коктейля}
страшен. В головах россиян смешано все, коммунизм, капитализм, монархизм, трусы
и крестик
%%%cit_comment
%%%cit_title
\citTitle{Страна без исторических корней всегда будет оккупантом}, 
Олена Степова, news.obozrevatel.com, 17.06.2021
%%%endcit
