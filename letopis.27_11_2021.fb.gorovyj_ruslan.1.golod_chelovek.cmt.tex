% vim: keymap=russian-jcukenwin
%%beginhead 
 
%%file 27_11_2021.fb.gorovyj_ruslan.1.golod_chelovek.cmt
%%parent 27_11_2021.fb.gorovyj_ruslan.1.golod_chelovek
 
%%url 
 
%%author_id 
%%date 
 
%%tags 
%%title 
 
%%endhead 
\subsubsection{Коментарі}
\label{sec:27_11_2021.fb.gorovyj_ruslan.1.golod_chelovek.cmt}

\begin{itemize} % {
\iusr{Oleksii Milchenko}

Росія так і залишилася втіленням цього зла, після розпаду тюрми народів. Але
суть нікуди не поділася, так і намагаються й надалі знищувати фізично та
ментально будь-які паростки свободи та здорового глузду.

\iusr{Маріна Омелянюк-Сахарук}

Пам'ятаймо!

\ifcmt
  ig https://scontent-frx5-1.xx.fbcdn.net/v/t39.30808-6/261534231_622440562128070_961608416849545690_n.jpg?_nc_cat=110&ccb=1-5&_nc_sid=dbeb18&_nc_ohc=dRMWUj19op0AX9vqUbG&_nc_ht=scontent-frx5-1.xx&oh=d8a5179de77082ae67d6e8a634d810d8&oe=61A70AF4
  @width 0.4
\fi

\iusr{Prymara Makeeva}
Знання не є вирішальним фактором, хоча і має вплив

\begin{itemize} % {
\iusr{Дмитро Вагабунд}
\textbf{Prymara Makeeva} а що на вашу думку вирішує

\iusr{Prymara Makeeva}
\textbf{Дмитро Вагабунд} аби знала, написала би. Можливо, це десь на рівні відчуттів, спроможності до емпатії. На разі не маю простого формулювання. Але багато людей знають, а не сприймають. Бачте, не маю, формули((

\iusr{Дмитро Вагабунд}
\textbf{Prymara Makeeva} у вас є якась спеціальність?

\iusr{Надія Тітова}
\textbf{Дмитро Вагабунд} трохи буде лонгрід, але то про знання/сприйняття.

Моя бабуся пережила 2 голодомора та роботи в Німеччині. Її сім'я (брати, дід,
тато) були репресовані у 20-х за непокору радянській владі. Її рік народження
підробили, аби вона була народження в СРСР, а не в Польщі. Вона прожила
справжньою комуністкою, вчителем в школі пропрацювала 50 років. Незалежний етап
України їй давався важко. На стільки сильно вона була залякана з дитинства аби
вижити! Щиро ненавиділа бандерівці. Аж на грані боязні. Казала, що ніколи їх не
бачила, але чула багато звірячого про їх вчинки. При цьому Її ж рідні брати й
були бендерівцями, і звірства вона бачила на свої очі від радянської влади. Але
ідеологія, що насаджується голодом і страхом - не обов'язково має бути
логічною. Просто вір, або вмри. Обирали перше. Не думати про це легше тим, хто
не хоче боротьби (через лінь, чи страх померти - не важливо).

\end{itemize} % }

\iusr{Дмитро Вагабунд}
Я вважаю що Нація пам'ятає все, але на жаль це дуже малий відсоток від кількості населення

\iusr{Ирина Белозорова}
А зараз що відбувається? Скоро усі усе згадають.

\iusr{Ruslan Gorovyi}
\textbf{Ирина Белозорова} що ви маєте на увазі?


\end{itemize} % }
