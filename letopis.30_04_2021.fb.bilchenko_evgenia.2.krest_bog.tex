% vim: keymap=russian-jcukenwin
%%beginhead 
 
%%file 30_04_2021.fb.bilchenko_evgenia.2.krest_bog
%%parent 30_04_2021
 
%%url https://www.facebook.com/yevzhik/posts/3849636171738131
 
%%author 
%%author_id 
%%author_url 
 
%%tags 
%%title 
 
%%endhead 
\subsection{БЖ. Страстная Пятница (2020).}
\label{sec:30_04_2021.fb.bilchenko_evgenia.2.krest_bog}
\Purl{https://www.facebook.com/yevzhik/posts/3849636171738131}


\ifcmt
  pic https://scontent-bos3-1.xx.fbcdn.net/v/t1.6435-0/p526x296/180224753_3849659525069129_2721117366839474272_n.jpg?_nc_cat=103&ccb=1-3&_nc_sid=730e14&_nc_ohc=pswJcaNM2dsAX-QeMbf&_nc_ht=scontent-bos3-1.xx&tp=6&oh=cce468e480b2779aadd30bb41e749031&oe=60B43C87
\fi


И вот как изменилось всё в моей поэзии за 12 лет. Текст на ту же тему, Страстной Пятницы, но прошлого года.

Когда они сняли Его с креста и положили в землю,

На этой земле всё так же сношались, пели, дрались, рожали.

Желающие свободы растили дурное зелье.

Желающие закона на камне секли скрижали.

Когда они сняли Его с креста, мир оставался прежним.
Никто и ничто в нём не собирались каяться и меняться.
Убивать в нём не стали реже, красть в нём не стали реже
И говорили всё так же сложно, сложно и непонятно.

Странный был Бог, однако: не чурался вина и яства.
Не потешался над бабами, детишками, стариками.
И всё позволял: позволял над собой смеяться,
Позволял себя бить, толкать и колоть штыками.

И даже в последнюю свою ночь, в том пригороде садовом,
Вёл себя Бог, на удивление, так по-детски:
Не присягал на верность, не сотрясал основы,
А звал - не победоносно, не выверенно, не веско.

Так, в детский сад попав, малыши кличут с работы мамку,
Которая привела, руку выдернула и вышла.
И я не знаю, что там цвело, в садике Гефсиманском,
Но на ум приходит одна черешня, черешня и черевишня.

И потом, когда сняли Его с креста, рыцарь не спас принцессу.
Самсон не сломал колонны - мощный, как терминатор.
Не поразила цареубийцу грозная длань Зевеса,
Космонавт не увидел царя небес через иллюминатор.

И люди, чьи лёгкие - жадные, как на безводье жабры, -
Продолжали шумно дышать и вздирать на дыбы,
Продолжали болеть, собой заражая, и заражаться,
Продолжали в обмен на веру требовать много рыбы.

Мир оставался прежним, но что-то в нём изменилось,
Если мы с тобою стоим вот так, одолевая время,
Одолевая самих себя, как самую злую мнимость,
Одолевая плоды трудов своих и своё же семя.

Само одоленье одолевая, так же, как все различья.
Правила логики нас едят, правила буквы лгут нам.
Но над болотом летит Любовь - выстраданная, птичья.
Через болото ползёт Любовь и собирает клюкву.

И, пока она её собирает, впихивая в котомку,
Пока снимают Его с креста, лихо терпя под дыхом,
Пятничный мир позади Него рушится громко-громко -
Мир Воскресения впереди строится тихо-тихо.

17 апреля 2020 г.

PS. И надпись на гараже мне тоже нравится. #Летовжив #Гроб #МарияМагдалина
