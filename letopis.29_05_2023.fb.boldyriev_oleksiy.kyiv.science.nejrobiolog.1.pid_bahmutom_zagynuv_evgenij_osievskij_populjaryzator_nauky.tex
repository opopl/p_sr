%%beginhead 
 
%%file 29_05_2023.fb.boldyriev_oleksiy.kyiv.science.nejrobiolog.1.pid_bahmutom_zagynuv_evgenij_osievskij_populjaryzator_nauky
%%parent 29_05_2023
 
%%url https://www.facebook.com/oleksiy.boldyriev/posts/pfbid02kUvsip9ovBzcbf6cbqYkg8gnFy9xBLTa3zrzDqKutUX76oN7mZXz3oCKKeXr3QRLl
 
%%author_id boldyriev_oleksiy.kyiv.science.nejrobiolog
%%date 29_05_2023
 
%%tags bahmut,science,nauka,nauka.populjarizacia,smert
%%title Під Бахмутом загинув популяризатор науки Євгеній Осієвський
 
%%endhead 

\subsection{Під Бахмутом загинув популяризатор науки Євгеній Осієвський}
\label{sec:29_05_2023.fb.boldyriev_oleksiy.kyiv.science.nejrobiolog.1.pid_bahmutom_zagynuv_evgenij_osievskij_populjaryzator_nauky}

\Purl{https://www.facebook.com/oleksiy.boldyriev/posts/pfbid02kUvsip9ovBzcbf6cbqYkg8gnFy9xBLTa3zrzDqKutUX76oN7mZXz3oCKKeXr3QRLl}
\ifcmt
 author_begin
   author_id boldyriev_oleksiy.kyiv.science.nejrobiolog
 author_end
\fi

Під Бахмутом загинув популяризатор науки Євгеній Осієвський.

Це він автор нещодавнього інтерв'ю про генетику та репресії від британського
популяризатора в \href{https://www.facebook.com/kunsht}{Куншт}, що викликало
бурхливі дискусії. Ми не були знайомі, проте ця новина так вразила мене, що
першою думкою було: продовжувати обговорювати це інтерв'ю тепер є неетичним та
неповагою до пам'яті оборонця України. Утім, тепер розумію, що обговорення і
навіть критика цього тексту є важливою для пам'яті Євгенія, бо кожен раз, коли
ми цитуватимемо його, ми можемо віддавати шану герою.

\ii{29_05_2023.fb.boldyriev_oleksiy.kyiv.science.nejrobiolog.1.pid_bahmutom_zagynuv_evgenij_osievskij_populjaryzator_nauky.pic.1}

А з перпендикулярного світу через соцмережі в ці ж дні лізе інший сюжет.

Російський популяризатор науки Александр Панчін страждає через безсонну ніч, бо
його не пустили прочитати лекцію в Молдові. З Александром я знайомий уже років
15, а якраз рівно 10 років тому зустрічався з ним на конференції в передмісті
Петербурга. Тому, хоч-не-хоч, сюжет цей зачіпає, тим більше, що безсонні ночі в
українських популяризаторів останній місяць уже без перерв. Доведеться й собі
докинути.

Панчін вважає себе лібералом і просвітником, хоча завжди обережний і балансує
на межі, як вчить його батько, завідувач лабораторії в підсанкційному інституті
РАН та відкривач особливого класу мембранних канальних білків, названих на його
честь. Тому його щиро дивує, чого існують якісь бар'єри для такого вільнодумця,
який не просто є співробітником того ж інституту \enquote{подвійного
призначення}, але й активним захисником арештованих сибірських фізиків -
розробників сумнозвісного \enquote{Кінжала}. 

Мені здається, я зрозумів природу цього здивованого нерозуміння.

Коли ми тусили у вконтактних навколонаукових групах у 2008-2012 роках,
російські вчені завжди любили глузувати з політкоректності американських та
європейських вчених щодо колись пригноблених ними народів. Ці всі квоти для
афроамериканців у США, велика кількість пакистанців та індійців у Британії,
стосунки між етнічними французами та громадянами Франції арабського та
африканського походження тощо - завжди викликали в росіян смішочки. От дурні,
чого вони принижуються (перед якимись чорними/темними/косоокими - якщо не
казалося, то читалося). Проте будь-які подібні коментарі на тему пригноблення
росіянами народів Сибіру чи Центральної Азії, сприймалися дуже серйозно, як
провокація міжетнічних конфліктів. Росіяни (та й українці) абсолютно щиро не
розуміли природу деколонізації та складних стосунків між колишніми
колонізаторами, які усвідомлюють відповідальність, і колонізованими, які
відчувають широкий спектр емоцій відносно минулого й теперішнього. Проте власну
колонізаторську політику росіяни заперечують \enquote{з порогу}, бо цей наратив
\enquote{братніх народів} росте з наративу \enquote{білих цивілізаторів} XIX століття без
будь-яких виправлень, які сталися в тій же Англії, Франції чи США століттям
пізніше.

Ось і тепер Панчін, освічений російський ліберал, має сліпу пляму в тому місці де: 

\begin{itemize} % {
\item Російська імперія захопила частину східнороманських земель та зробила
половину молдаван своїми підданими, відсуваючи їх від реальної влади в регіоні

\item Радянська влада створила \enquote{шароварну} Молдавію з кирилічним алфавітом і
переважною російською мовою спілкування

\item Російська федерація спровокувала війну на території держави Молдова та
впродовж 31 року окупує солідну частину її території з важливим промисловим
районом

\item Та ж федерація паралельно веде торговельні війни з маленькою країною, поширює
там свою пропаганду та з перемінним успіхом намагається встановити свій
маріонетковий режим
\end{itemize} % }

Панчін думає, що в Молдові \enquote{просто розмовляють російською мовою}, а тому він
може розповісти там щось про біотехнології чи ГМО. Він посилається на те, що
багато людей у Молдові хочуть його бачити, а тому гріх його не впускати. І він
не хоче впритул бачити, що ці люди хочуть слухати його, а не молодих
популяризаторів з Клуж-Напоки чи Бухаресту, тільки тому, що його кровожерлива
держава колонізувала цей народ понад 200 років - і продовжує це робити прямо
під час його вояжа. Звичайно, він не бачить тут не тільки своєї моральної
відповідальності, але навіть зв'язку між своїми лекціями й ситуацією в далекій
країні. І того, що гроші за лекцію він отримує виключно через привілей мешканця
метрополії.

Узагалі росіяни, скільки завгодно освічені, поводяться точно так же, як
середній американський консерватор з ядерного електорату Трампа, якого бісять
будь-які згадки про рабство та викликані ним привілеї та утиски убік частини
громадян США. Коротка дискусія 1993-1996 років у РФ на тему \enquote{а чи не повинні ми
дати свободу чеченцям/інгушам/дагестанцям/тощо після того, що ми з ними
зробили} закінчилася патріотичним угаром. Поки росіяни не впізнаватимуть у собі
колонізаторів, вони лишаться дикунами. У тезі, яку приписують Винниченку, що
\enquote{російський ліберал закінчується на українському питанні}, можна вставити
будь-який інший етнонім з поневолених росіянами за останніх 300 років.
\enquote{Ліберал} Панчін закінчився на молдовському.

Між перенесеним Євгенієм Осієвським на Донеччині й моїми безсонними київськими
ночами - зрозуміла всім тут нескінченна етична відстань. Між тими ж ночами та
однією ніччю дискомфорту російського ліберала-колонізатора - теж прірва. Тому
подібні порівняння бісять українців, але зате й дозволяють сформулювати оцінку
ситуації. Читаймо тексти Євгенія, не забуваймо наших героїв.

(Фото - з сайту \enquote{Куншт})
