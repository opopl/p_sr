% vim: keymap=russian-jcukenwin
%%beginhead 
 
%%file 13_12_2021.fb.fb_group.story_kiev_ua.2.raf_marshrutka.pic.3.cmt
%%parent 13_12_2021.fb.fb_group.story_kiev_ua.2.raf_marshrutka
 
%%url 
 
%%author_id 
%%date 
 
%%tags 
%%title 
 
%%endhead 
\iusr{Петр Кузьменко}

Очередь на маршрутку на Бесарабской площади.

\iusr{Любовь Белоцерковец}
А откуда фото было сделано? Мне кажется, что с той стороны высотки не было...
или память подводит?)))

\iusr{Петр Кузьменко}
\textbf{Любовь Белоцерковец} вот не знаю. Нашёл на просторах Интернета.

\iusr{Ксения Погорлецкая}
\textbf{Петр Кузьменко} На Подол маршрутки ходили редко, номер маршрута был, кажется, 201 или 202 и очередь была большая. Фото же сделано из окна поликлиники Ленинского района.

\iusr{Любовь Белоцерковец}
\textbf{Ксения Погорлецкая} совсем за поликлинику забыла)))

\iusr{Ирина Ицковская}
Маршрутка ехала на Подол

\iusr{Татьяна Буцько}
\textbf{Ирина Ицковская} і на Кловський узвіз. 15 копійок)

\iusr{Vitali Andrievski}
\textbf{Татьяна Буцько} на кловський узвiз зупинка була за будiвлею ринку. На фото ii не можна побачити.

\iusr{Татьяна Буцько}
\textbf{Vitali Andrievski} Спочатку вона була тут, це потім переннсли

\iusr{Ксения Погорлецкая}
\textbf{Татьяна Буцько} 

Маршрутка на Кловский Спуск никогда не ездила. Остановка во второй половине
60-х была внизу Крутого Спуска (где впоследствии было кафе \enquote{Харьков}), потом
(после аварии в которой погибли люди) остановку перенесли туда же откуда езди
ла на Подол и уже третий перенос - с у бокового выхода Бессарабского рынка
возле хлебного магазина.

\iusr{Татьяна Буцько}
Так, не Кловський, а Крутий. То я переплутала. Але ж на К))

\iusr{Tatiana Pani}
Снимок сделан из окна 9 поликлиники Ленинского района, сейчас Плаза Пинчука.

\ifcmt
  ig https://scontent-frx5-1.xx.fbcdn.net/v/t39.30808-6/266286950_429382115519560_4123660022792692402_n.jpg?_nc_cat=105&ccb=1-5&_nc_sid=dbeb18&_nc_ohc=bhRsjh39uzMAX89kky4&_nc_ht=scontent-frx5-1.xx&oh=00_AT9dEejHObjsWtOcXj3slXn5hoisHRhWLU8s1pQTG-Pm7w&oe=61C2910F
  @width 0.3
\fi

\iusr{Olga Poryadynska}
А в той час вже були маршрутки?

\iusr{Татьяна Буцько}
\textbf{Olga Poryadynska} так, були. Я правда знаю тільки з Бесарабки, бо користувалась

\iusr{Nataliia Shevchenko}

А ще ходила маршрутка з площі Богдана Хмельницького (Софіївської) через вул.
Жовтневої революції (Інститутську ) на Печерськ - 15 коп.

\iusr{Нина Кубанова}
15коп.
А рядом стоянка такси
Пару раз удавалось скинуться по 20копеек, чтоб не ждать маршрутку (красная площадь)

\iusr{Людмила Мозговая}
Там на такси была очередь...

\iusr{Люсянка Балашова}
в 1980 г, ехала от вокзала до Академгородка-15 коп.
