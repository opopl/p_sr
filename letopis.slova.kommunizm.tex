% vim: keymap=russian-jcukenwin
%%beginhead 
 
%%file slova.kommunizm
%%parent slova
 
%%url 
 
%%author 
%%author_id 
%%author_url 
 
%%tags 
%%title 
 
%%endhead 
\chapter{Коммунизм}
\label{sec:slova.kommunizm}

%%%cit
%%%cit_head
%%%cit_pic
%%%cit_text
Во-первых, латиноамериканцы плохо ассимилируются. Их дети, в отличие от
потомков русских, ментально останутся мексиканцами. И второе, от чего я орнул.
В Латинской Америке популярен \emph{коммунизм}. И нелегалы приносят его с собой
в Штаты, говорят критики. Так вот. Правда, по поводу \emph{коммунизма} говорят
республиканцы. У них тут своя проблема. Компартия в США может до сих пор и
есть, но не слишком популярна. А вот демократическая партия представлена везде.
В итоге латиноамериканцы топят на выборах за демов
%%%cit_comment
%%%cit_title
\citTitle{В США испугались массового наплыва коммунистов}, 
Максим Войтенко, strana.ua, 21.06.2021
%%%endcit


%%%cit
%%%cit_head
%%%cit_pic
%%%cit_text
\emph{Китайские коммунисты} к тому времени окончательно поссорились с советскими. И
американцы увидели в этом шанс открыть против Москвы \enquote{второй фронт} - сделать
Китай своим неофициальным союзником в холодной войне против СССР. Это резко
ухудшило геополитическую ситуацию для Советского Союза, так как пришлось
тратить огромные ресурсы для укрепления обороны Дальнего Востока.  С конца 70-х
годов, уже после смерти Мао, в Китая начались рыночные реформы, страна
открылась для внешних инвестиций. И позитивное отношение к этому американцев,
которые были заинтересованы в то время в укреплении КНР для противовеса
Советскому Союзу, сыграло в начавшемся инвестиционном буме не последнюю роль.
Но с конца 80-х годов отношение начало меняться
%%%cit_comment
%%%cit_title
\citTitle{Коммунистической партии Китая сто лет - Как появилась КПК}, 
Дмитрий Коротков, strana.ua, 28.06.2021
%%%endcit

%%%cit
%%%cit_head
%%%cit_pic
%%%cit_text
Там празднуют 100-летие \emph{Компартии} с огромной помпой.
Пока наши вятровичи ловят на улицах своей выдающейся разрухи рецидивистов в
шапке с советской кокардой, над Шанхаем, под литавры Интернационала, загорается
лазерный серп и молот ))) Это Китай так празднует 100-летие своей
\emph{Коммунистической} партии. С эпическим, как всё китайское, размахом: летающими
барабанщиками, пылающими небоскрёбами и океанами людей в кислотных касках.
Трудно представить количество бабла и свэга в одном квадратном метре этой
неомаоистской коачеллы. Как и нашего \emph{декоммунизатора}, пытающегося впаять
пятёрочку китайскому камраду, и не понимающего, что завтра зачем-то сам себя
оскопивший Ивасык-тэлэсык будет говорить «ни хао», а не «хай»
%%%cit_comment
%%%cit_title
\citTitle{Интересно, каково Вятровичу смотреть на торжество коммунизма в Китае? / Лента соцсетей / Страна}, 
Анатолий Ульянов, strana.ua, 05.07.2021
%%%endcit

%%%cit
%%%cit_head
%%%cit_pic
%%%cit_text
І чи здатна інша сторона, російськомовна, що вірить у російську пропаганду, у
\enquote{фашизм} на Майдані, в жахи про націоналістів зі смолоскипами, що зростала з
вірою в \emph{комуністичне} майбутнє, прийняти СРСР таким, яким він був для мільйонів,
– із жахами, що аж серце холоне, КДБ, де катували людей на рівні з нацистами, з
таборами болю, де гинули так само, як у концтаборах Третього рейху, з мільйоном
зґвалтованих німкень після перемоги в Другій світовій війні, з расизмом,
елементарним безправ’ям і зневагою навіть до найдрібніших потреб маленької
людини? 
%%%cit_comment
%%%cit_title
\citTitle{Українці не розуміють одне одного не через мову, а через небажання слухати, чути і сприймати}, 
Юлія Мендель, www.pravda.com.ua, 07.07.2021
%%%endcit

%%%cit
%%%cit_head
%%%cit_pic
\ifcmt
	width 0.4
  pic https://ic.pics.livejournal.com/maxim_nm/51556845/4080913/4080913_original.jpg
\fi
%%%cit_text
В 1991 году \emph{коммунистов} отстранили от власти и они оказались в весьма
непривычной для себя роли — роли оппозиции. Рейтинг \emph{коммунистов} в наевшимся
пустых обещаний обществе был невысоким, и им ничего не оставалось, как выходить
на уличные акции протеста. Вдруг выяснилось, что уличные акции могут быть не
только \enquote{проплачены Госдепом}, \enquote{организовываться недругами за бугром} и т.д., но
и представлять интересы обычных людей — по крайней мере, именно так заявляли
\emph{коммунисты}, что приходили на митинги. На фото — 1992 год, милицейские автозаки
на митинге \emph{коммунистов}
%%%cit_comment
%%%cit_title
\citTitle{Что получили люди сразу после распада СССР}, Максим Мирович, news.obozrevatel.com, 06.07.2021
%%%endcit
