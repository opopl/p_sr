% vim: keymap=russian-jcukenwin
%%beginhead 
 
%%file 30_05_2021.fb.savenkova_faina.1.verit_i_nadejatsja_donbass
%%parent 30_05_2021
 
%%url https://www.facebook.com/permalink.php?story_fbid=312313930389546&id=100048328254371
 
%%author 
%%author_id savenkova_faina
%%author_url 
 
%%tags 
%%title Верить и надеяться
 
%%endhead 
 
\subsection{Верить и надеяться}
\label{sec:30_05_2021.fb.savenkova_faina.1.verit_i_nadejatsja_donbass}
\Purl{https://www.facebook.com/permalink.php?story_fbid=312313930389546&id=100048328254371}
\ifcmt
 author_begin
   author_id savenkova_faina
 author_end
\fi

Верить и надеяться.

Каждый раз, когда меня просят рассказать о жизни под украинскими обстрелами, я
теряюсь. Не потому, что ребенок и не потому, что об этом нечего рассказать,
нет. Я просто не знаю, что от меня хотят услышать. Сухие и безразличные сводки
о жертвах и разрушениях? Конечно, нет. Для этого есть новости. Личные ощущения
и переживания? А вот с этим сложнее. Какая она – жизнь на войне? Наверное,
обычная, если ты почти не помнишь жизни мирной.

\ifcmt
  pic https://scontent-frt3-2.xx.fbcdn.net/v/t1.6435-9/192189848_312312057056400_4000598549717503962_n.jpg?_nc_cat=103&ccb=1-3&_nc_sid=730e14&_nc_ohc=_nJE_kxmK_cAX-erd11&_nc_ht=scontent-frt3-2.xx&oh=b4ef2b66f4baf2a317c07162c45eb4f9&oe=60D8A5F8

  caption фото Лидии Опренко
\fi

Многие могут ужаснуться из-за осознания того, что в двадцать первом веке в
географическом центре Европы есть ребята, которые не помнят летящие высоко в
небе самолеты, прогулки по вечернему городу вместе с родителями или еще
какую-нибудь милую ерунду, на которую не обращают внимания другие дети.

В нашу жизнь вносят свои корректировки бесполётная зона и комендантский час. И
когда читаешь о беспорядках в некоторых городах Европы из-за введения
комендантского часа в связи с коронавирусом, это вызывает недоумение: а что не
так? Всего лишь комендантский час, ничего страшного же не произошло, зачем так
беспокоиться?  Причина такого спокойствия на самом деле очень проста: всё
познается в сравнении, а нам сравнивать нечего. Мы – поколение, не помнящее
мирной жизни.

Мы – поколение, живущее по строгим правилам, несоблюдение которых может
привести к смерти. Мы научились определять на слух направление полета снарядов,
чтобы знать, когда стоит беспокоиться, а когда можно продолжать идти по своим
делам. Мы научились не пропускать мимо ушей лекции МЧС о правилах поведения во
время обстрелов, в случае обнаружения подозрительных предметов или прочие
рекомендации в различных ситуациях. И все равно никто не может дать гарантии,
что тебя случайно не заденет осколком из-за того, что тебе просто не повезло.
Странно? Страшно? Обыденность, с небольшой долей различий в зависимости от
интенсивности обстрелов территории.

Что такое жизнь под украинскими обстрелами? Это когда вечером 1 июня, в День
защиты детей, у мемориалов погибшим детям Донбасса в небо взмывают сотни
бумажных фонарей, чтобы осветить путь ангелам. Ведь малышам, запускающим
фонарики, сложно объяснить, почему у этих ангелов отняли их короткую жизнь,
лишив возможности повзрослеть и увидеть мир на нашей Родине. Теперь они могут
только наблюдать с небес и плакать вместе со взрослыми, утешая их.

Так получилось, что почти вся моя жизнь и воспоминания связаны с войной, из-за
чего у меня нет сожалений и печали о прошлом. Я живу настоящим и изредка думаю
о будущем, в котором есть место наивной и глупой мечте, вызывающей улыбку.
Вполне реальная, согревающая и почти осязаемая она позволяет не отчаиваться
даже в самые тяжелые времена. Я хочу, чтобы в небе Донбасса летали пассажирские
самолеты, а не бумажные фонарики. Любая мечта должна превращаться в реальность.

Так должно быть и я верю, что так будет.

