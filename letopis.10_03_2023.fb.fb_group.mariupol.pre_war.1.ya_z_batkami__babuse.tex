%%beginhead 
 
%%file 10_03_2023.fb.fb_group.mariupol.pre_war.1.ya_z_batkami__babuse
%%parent 10_03_2023
 
%%url https://www.facebook.com/groups/1233789547361300/posts/1424570758283177
 
%%author_id fb_group.mariupol.pre_war
%%date 10_03_2023
 
%%tags mariupol,mariupol.pre_war
%%title Я з батьками, бабусею і сестричкою напередодні мого першого класу, Маріуполь, 1973 рік
 
%%endhead 

\subsection{Я з батьками, бабусею і сестричкою напередодні мого першого класу, Маріуполь, 1973 рік}
\label{sec:10_03_2023.fb.fb_group.mariupol.pre_war.1.ya_z_batkami__babuse}

\Purl{https://www.facebook.com/groups/1233789547361300/posts/1424570758283177}
\ifcmt
 author_begin
   author_id fb_group.mariupol.pre_war
 author_end
\fi

%https://www.facebook.com/profile.php?id=100000943996304

Я з батьками, бабусею і сестричкою напередодні мого першого класу, Маріуполь,
1973 рік. Батько Іван Григорович тоді працював водієм на перших тролейбусних
лініях тодішнього Жданова. Жили ми за "1000 мєлочєй" у теперішній площі
Свободи, куди з мамою на зупинку виходили, аби перемовити з татом чи навіть
проїхатись з ним. Пишався цим, іноді їздив на батьковому тролейбусі на
передньому боковому до "Азовсталі", бо це був маршрут \#1. 

У 2019-му саме до кінцевої цього маршруту організував екологічну ходу з
вимогами щодо екологізації виробництва та закриття брудних виробництв типу
коксового цеху. Колись в ньому, тодішньому КХЗ, машиністом тепловозу працював
батько. Возив розпечений у сотні градусів кокс до тушильної камери, де кокс
заливали водою. Саме малим побачив там "ефект дракона" під оперуванням
батькових рук і команд: нестерпний жар і клуби отрутного пару, який маріупольці
щоденно спостерігали майже з усіх куточків міста. Мрії містян збулися: брудне
виробництво, екоцидне для моря, міста і всього живого закрилося і планується
відкритися у вигляді промислового і військового пантеону-музею з елементами
зоо- і дендропарку.

Батькові там нахамило начальство, тож він кинув посеред зміни свій тепловоз і
кілька днів забороняв відкривати двері на дзвінки, поки не забрав трудову з
роботи. Сам був смуглявий, веселий і музикальний, чим дуже нагадував Челентано.
Грав на баяні, був майстром на всі будівельні роботи, побудував чи оштукатурив,
обклав плиткою, мозаїкою сотні будинків. Нормально на шабашках заробляв. 

Змінив десятки робіт і професій. Останню я йому знайшов у ВОХРі Маріупольського
порту. Майже до 70 допрацював там, жодної крадіжки в його зміні, нач. порту
особисто з ним вітався. В'язав на спицях светри + сітки, розгадував до кожної
клітиночки складні кросворди, ганяв зі мною на мотоциклі чи моторолері,
мандрував. Навчав багатому: боротися, стріляти з пістолета, не боятись. Та
головне не лаяв, не сварив і жодного разу не бив, без нервів. На старості
чудив, був впертим до нестями: який в роботі – такий в житті. Хоча нема його,
та вдячний за науку і майстерність ~ передалось*)

Мамо Ніна Саввовна в той рік отримала почесну нагороду серед усіх педагогів
ждановських шкіл. "Побєдітєль соц. сорєвнованія 1973 года" – ця медалька
зберігалася майже до самої війни. Наступного року її не стало: надорвалася.
Батько казав, що колеги їй завидували, за спиною злословили, та то таке.
Вчителька фізики і труда в СШ N 9, працювала в дві зміни. Бував ще
дошкільнятком на її уроках серед верстатів, бо вела труди для хлопців. Важка
праця з підлітками, нервова. У моєму класі з першого по восьмий було 44 учня.
Всього шість класів паралельних. Зрозумів пізніше, чому зривалася. 

Писав колись про неї, а для себе вже старшим робив висновки: при перевантаженні
або роби подвиг, або... Ні, не вмри ~ розтягни виконання важкої праці, задачи,
якщо є така можливість. Можна делегувати комусь частину роботи, попросити
допомогу: краще бути живим, ніж "героїчно" мертвим. Мама була вкрай
відповідальна, і у неї це не вдавалося. 

Ще була вкрай порядною. Родичі розповідали, що у Слов'янську на курсах
підвищення кваліфікації познайомилася і покохала учителя. Призналась батькові.
Не бачив дома скандалів, але такі речі можна казати, якщо вирішили розлучитися.
Будьте мудрішими, не будіть у людині чи в собі Того, хто у вас спить. Вогонь 🔥
треба контролювати, при надмірному присипляти, окрім атаки, боротьби за великі
здобутки або життя.

Бабуся Олена Семенівна, у дівоцтві Кононенко. Як і батько, родом з Заволж'я,
село Вольне, що за кілька хвилин їзди від Сарай-Бату – ставки Золотої Орди.
Самий кордон Європи і Азії, жив там, потім на Кавказі після смерті матері. Як і
батько, смугла, весела, працьовита. Справжня майстриня і пастушка, навчала
випасати, доглядати і годувати худобу, гусей, кролів. Разом ліпили вареники з
вишні, шовковиці, полуниці, пасльону під назвою "дружба", які я збирав у саду. 

Навчила мене торгувати, інструктувала, куди бити, щоби одним ударом звалити з
ніг. У дитинстві вигравала на верблюжих скачках, тікала з класу у вікно на
сільські весілля: там музика, весело, цукерки і гроші дітям кидають. Якби не
чотири класи освіти – могла великими колективами керувати. Її всі любили й
поважали: пастухи і пастушки – вони такі!")

Двоюрідна сестричка Оля теж з Вольного. Симпатична ділова стрєкоза, на два роки
старше, трохи ніяковів з неї. Пам'ятаю, як спитала, що я можу приготувати. Хоча
мені нічого готувати не довіряли, але процес приготування яєчні описав
детально. Коли побачив її вже студентом в домі, де вона прала речі в тазику –
жахнувся: від юної привабливості в змарнілому обличчі нічого не залишилося.
Чоловік її, теж Сергій, на вигляд щупленький, але міцний, по п'яні натовп
казахів з другом міг розігнати. Сам бачив, коли під час поляни сказав, що один
з казахів у дворі мене х0х^0м обізвав. Сергій сам був українцем, одразу пішов у
двір. Казахів було більше, противник здоровань, та все ж реально чоловіка Олі
клінічно боялися, ледь бійку зупинив. Розумію, чому Оля була пригнічена: в
п'яному стані такому краще не грубити.

Це фото нагадало, що в перший клас я пішов, вміючи з чотирьох років читати і
рахувати, але емоційний, з відчуттям власної гідності. 44 учня в класі – це
мурашник, з постійним порушенням на перервах твоїх кордонів і стану. Що я там
зробив, фіг його знає, бо культурною, тихою і тендітною дитиною був. Але
вчителька мабуть вгледіла, як я комусь в ухо засадив, зняла з мене одного
октябряцький значок з кучерявим Ульяновим, замість нього почепила "їжачка" у
вигляді ріп'яха та до кінця уроків поставила в кут біля дошки. Пропедевтичний
садизм: соромно було, півдня ладоньку від лацкана з "їжачком" не відривав. 

Звали вчительку Римма Іванівна, а я тоді вже знав, де на карті знаходиться
Італія і її столиця, читав і дивився різне з історії та по-дитячому дивувався
фонетичній несумісності фашистського Риму та слов'янської Іванівни в одному
коктейлі. Сім рочків, а реально таке в голові"))

Писали ми не шаріковими ручками, а пір'євими, вмакаючи їх в чорнильниці, що
стояли у виїмках на партах з нахилом. Найважче мені, майбутньому журналісту,
давався рус.яз. Коли був диктант, ти вслухаєшся і довірливо відтворюєш у
швидкому темпі з російським аканням відчеканені слова і дивуєшся, чому в тебе
"карова дала малака" стільки помилок і велика одиниця з мінусом наприкінці. Ну
хоч не нижче нуля*) 

Мама з цього нервувала, іноді карала, я ревів. Диктувала мені не так швидко
нормальною запорізькою вимовою, бо з-під Гуляй-Поля родом, тож помилок майже не
робив. Коли померла, сильно змінився, пішов у себе. Сталося це у другому класі.
Перед цим забігав до неї в майстерню, аби відпросила мене від болючого уколу
під лопатку. Відпросила, але застав, як директорка жорстко старшокласників на
її уроці вгамовує. Жданов тоді махаловками і буйностю молоді був славетний,
перше місце в Союзі з молодіжної злочинності. Хвора вчителька, буває, чого ж
ви"(

В 34 роки у засвіти пішла, два роки перед цим хворіла. Символічно, що і я в 32
вперше по-справжньому до лікарні попав, свій ляк перед привівкою щоденними
уколами до сих пір компенсую. Тільки у живіт понад 20 тисяч за життя + ще
більше в руки, ноги й дупцю. Фобії, страхи, лякі треба переступати, переживати,
перемагати в справах. Очі бояться, а руки роблять!..

Не хвилюйтеся ~ переможемо "зараз і тут"! З літописів, біографій і власних
спостережень примітив: знакові події в житті ваших батьків і деякі від
дідів-бабів дивним чином накладаються в ті самі роки, на конкретні дати на ваше
життя. Карма, у нас це доля, в науці хроно-цикли та синхронність ключових подій
в житті кожного. Ось чому важливо знати ретельно свій родовід та шанувати своїх
предків. 

\#Вивчаємо!  \#Шануємо!!  \#Передбачаємо!!!
