% vim: keymap=russian-jcukenwin
%%beginhead 
 
%%file 21_04_2021.stz.news.ua.mrpl_city.1.ivan_papanaga
%%parent 21_04_2021
 
%%url https://mrpl.city/blogs/view/ivan-papanaga-probujte-eksperimentujte-stvoryujte
 
%%author_id demidko_olga.mariupol,news.ua.mrpl_city
%%date 
 
%%tags 
%%title Іван Папанага: "Пробуйте, експериментуйте, створюйте"
 
%%endhead 
 
\subsection{Іван Папанага: \enquote{Пробуйте, експериментуйте, створюйте}}
\label{sec:21_04_2021.stz.news.ua.mrpl_city.1.ivan_papanaga}
 
\Purl{https://mrpl.city/blogs/view/ivan-papanaga-probujte-eksperimentujte-stvoryujte}
\ifcmt
 author_begin
   author_id demidko_olga.mariupol,news.ua.mrpl_city
 author_end
\fi

\ii{21_04_2021.stz.news.ua.mrpl_city.1.ivan_papanaga.pic.1}

Нещодавно головний режисер Народного театру екстремального танцю \enquote{Legion} Іван
Іванович Папанага почав виконувати обов'язки директора ПК \enquote{Молодіжний}.
Впевнена, що не всі маріупольці знайомі з творчістю і колективом Івана, проте
цей чоловік є відомим маріупольським танцюристом і вже багато років надихає
своїм прикладом багатьох містян та допомагає молоді розкрити власний потенціал
і реалізувати мрії.

Народився Іван 17 червня 1983 року. Хореографією почав займатися з дитинства.
Танцювати любив всюди. Однак системно займатися вирішив, щоб сподобатися
дівчині, але сам незабаром на все життя закохався в Його Величність Танець, з
яким пов'язав і свою професію. Мама Вані була спортсменкою (чемпіонка з легкої
атлетики), тому у сина виховувала теж любов до спорту. Насправді в дитинстві
хлопцю довелося обирати між боротьбою і танцями, оскільки на все не вистачало
часу, він обрав танці. 

\ii{21_04_2021.stz.news.ua.mrpl_city.1.ivan_papanaga.pic.2}

З дитинства Вані говорили лікарі, що він не зможе бути
спортсменом, адже наслідки великої кількості травм давалися взнаки. Однак як
любить повторювати Іван: \enquote{те, що нас не вбиває, робить сильнішими!}. Попри всі
невтішні прогнози мужній хлопець продовжував відточувати танцювальну
майстерність.  Вчитель \emph{\textbf{Олександр Білоконь}} відіграв важливу роль в професійному
становленні юнака. Ваня став випускником маріупольського казацького ансамблю
народного танцю \enquote{Первоцвіт}. З 1998 по 2000 року Іван Папанага робить перші
кроки в професії хореографа – він – старший помічник хореографа в дитячому
ансамблі пісні і танцю \enquote{Первоцвіт} ПК ім. Карла Маркса. Влітку 2000 року Іван
працював хореографом в дитячій здравниці оздоровлення та відпочинку \enquote{Радужний}.

\ii{21_04_2021.stz.news.ua.mrpl_city.1.ivan_papanaga.pic.3}

Протягом 1998 – 2003 хлопець навчався у Львівському інституті культури.
Педагогічний склад цього закладу змінив ставлення та підхід Вані до класичного
танцю. У 1999 – 2000 роках юнак працює артистом балету в маріупольському
драматичному театрі. Можливо, це був виклик собі, але перемога була отримана, і
в постановках танцюриста відтепер можна було побачити унікальний симбіоз
народних елементів, акробатичних трюків та класичної хореографії. Саме цей
принцип (поєднання жанрів) Іван втілює в своїй роботі педагога-хореографа в
колективі, який створив сам – Театрі екстремального танцю \enquote{Legion}. Спочатку
колектив займався на вулицях, десь біля пам'ятників, у підворіттях шкіл, в
холах. Потім під свій дах у 1998 році їх впустив палац культури імені Карла
Маркса в Кальміуському (тоді – Іллічівському) районі, тобто Група брейкдансу
\enquote{Легіон} (початкова назва колективу) народилася там. Через три роки колектив
продовжив діяльність на базі Палацу культури \enquote{Іскра}. Директорка палацу \textbf{\emph{Тетяна
Іванівна Погоржельська}} наполягала, щоб вони були не просто групою брейкдансу, а
щоб стали чимось більш високим. І ось так утворився Театр екстремального танцю.
Потім через деякий час провели кілька концертів і у 2007 році отримали звання
\emph{народного}. Колектив є саме театром, адже всі номери дуже різноманітні, це не
просто спорт, акробатика, брейк. В кожному номері є свій сюжет, якась своя
родзинка, якийсь сенс і, найголовніше, драматургія.

Багато конкурсних вершин було швидко підкорено. І вже зовсім скоро Театр почали
запрошувати на масштабні фестивалі та конкурси в якості хедлайнера. У 2013 році
театр успішно пройшов кастинг в шоу талантів \enquote{Україна має талант} (отримали два
\enquote{так} від журі шоу). Один з учнів Івана Івановича увійшов  десятку найкращих у
танцювальному шоу талантів \enquote{Танцюють всі}.

\ii{21_04_2021.stz.news.ua.mrpl_city.1.ivan_papanaga.pic.4}

Іван Папанага залишався і буде залишатися \emph{\enquote{Граючим тренером}} завжди. І якщо
тренер – то з визнанням серед професіоналів: в 2008 році Іван стає призером
Першого Харківського конкурсу балетмейстерів.

Звісно, таланту буває замало місця, творчої інформації та простору прояву за
умови, якщо він постійно росте. Це – саме той випадок.

У 2008 – 2009 роках Іван працює артистом балету міжнародного театралізованого
музичного шоу \enquote{Ben Hur Live}.

З 2010 – 2018 році – постановник-хореограф танцювальних вистав у Китаї.

У 2016 – 2017 роках маріуполець успішно працює хореографом-постановником
новорічних і різдвяних вистав та кінофестивалю на Центральному телебаченні
Китаю CCTV.

\ii{21_04_2021.stz.news.ua.mrpl_city.1.ivan_papanaga.pic.5}

Але й дотепер Іван Іванович Папанага – головний режисер Народного театру
екстремального танцю \enquote{Legion}. Професійність, відданість справі, віра в
команду та любов до учнів, здається, тільки зміцнилися відстанню фундаменту на
якому впевнено стоїть театр. Кажуть, таке неможливо, та, мабуть, Іван залишав
своє серце в \enquote{Legion}, підкорюючи закордонні висоти. І воно
продовжувало битися, гріти, дарувати натхнення, любити свою дитину – свій
театр.

Зараз колектив налічує чотири філіали в Маріуполі та поза містом. У колективі
працють професійні учасники, перевірені часом. У лютому 2021 року колектив
прийшов в Центр культури \enquote{Лівобережний}. Як наголошує художня керівніця
ЦК \enquote{Лівобережний} \emph{\textbf{Тетяна Володимирівна Живолуга}}, на
таких керівників, як Іван Папанага, завжди йдуть діти. До того ж, головний
режисер Іван Іванович має великий педагогічний талант. За одне заняття учасники
вивчають чотири нових елементи. Загалом навчаються акробатиці, хіп-хопу,
брейкдансу та акторській майстерності. Кожен танець тренер завжди сам проживає
і цьому навчає своїх учнів – відчути і тоді передати глядачеві для того, щоб
був повний ефект від сприйняття такого продукту. У колектив можна приходити з 6
років, а вікових меж немає. Приходять і в 15, і у 18, і в 22 роки. Сам Іван
танцює протягом всього життя. На питання знайомих, коли він зупиниться, чоловік
відповідає: \emph{\enquote{мабуть, коли перестану ходити}}.

Поєднувати роботу режисера і керівника театру та виконуючого обов'язки
директора палацу культури складно. Іван Іванович намагається щодня проводити
час і зі своїм колективом, і вирішувати важливі завдання на керівній посаді.
Незважаючи на щільний графік, чоловік старається знаходити час і для своєї
сім'ї. Дружина Івана теж займається танцями і вже протягом багатьох років є
його головною опорою та підтримкою. Їхня донечка в свої чотири роки вже
займається гімнастикою.

Маріуполець  дуже хоче вивести майбутній ЦСМ \enquote{Готель Континенталь} на базі ПК
\enquote{Молодіжний} на новий рівень. Іван, наголошує, що він має стати майданчиком для
експериментів. Слова \enquote{пробуйте, експериментуйте, створюйте} повинні стати
девізом для ПК \enquote{Молодіжний}. Водночас маріуполець підтримує всі колективи
палацу і готовий сприяти їхньому розвитку й надалі.

\ii{21_04_2021.stz.news.ua.mrpl_city.1.ivan_papanaga.pic.6}

\begingroup
\em\textbf{Улюблена книга:} \enquote{Пророк з 8"б", або Вчора помилок не буде} К. Курбатова.

\textbf{Улюблений фільм:} фільми Леоніда Гайдая, \enquote{Весілля в Малинівці} (1967), \enquote{В бій
ідуть тільки \enquote{старики}} (1973), \enquote{Любов і голуби} (1984).

\textbf{Порада-побажання маріупольцям:} 

\begin{quote}
\enquote{Нічого не бійтесь. Викресліть зі свого життя
слово \enquote{неможливе}, йдіть крізь перепони до своєї мрії і у вас все вийде. Ніколи
не втрачайте свій шанс. Вірте в дива та не звертайте уваги на тих, хто не
вірить у вас!}.
\end{quote}
\endgroup
