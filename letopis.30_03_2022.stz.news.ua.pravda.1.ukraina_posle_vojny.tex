% vim: keymap=russian-jcukenwin
%%beginhead 
 
%%file 30_03_2022.stz.news.ua.pravda.1.ukraina_posle_vojny
%%parent 30_03_2022
 
%%url https://blogs.pravda.com.ua/authors/yakhno/624379ca2e9e9
 
%%author_id jahno_olesja,news.ua.pravda
%%date 
 
%%tags 
%%title Украина. После войны
 
%%endhead 
 
\subsection{Украина. После войны}
\label{sec:30_03_2022.stz.news.ua.pravda.1.ukraina_posle_vojny}
 
\Purl{https://blogs.pravda.com.ua/authors/yakhno/624379ca2e9e9}
\ifcmt
 author_begin
   author_id jahno_olesja,news.ua.pravda
 author_end
\fi

В каждом интервью я говорю о том, что Украина выйдет из этой войны сильнее,
умнее, красивее. Во всех смыслах. Во всех сферах. Несмотря на те разрушения,
которые Россия уже нам принесла, и ещё может принести (не удивлюсь, если там и
голод захотят устроить, по старой советской привычке).

Любая проблема начинается с головы. Ее разрешение – оттуда же. Развязав войну
(с 2014-го и по нарастающей), Россия своими же руками и своей же тупостью
обнулила все существующие внутренние противоречия и штампы, и, наконец,
показала всему миру, кем она есть.

Даже если оккупанты временно заходят в отдельные украинские города и проводят
какие-то референдумы, все это – уже неработающая схема. И перспективы
исторической у РФ в действующем формате – нет. Это не 30-40 годы XX столетия,
когда не было интернета. Информация сейчас распространяется молниеносно. И мир
российское нутро, черное, наконец, увидел. У Кремля не получится, например, как
когда-то с Андре Жидом и Лионом Фейхтвангером, которых пригласили в СССР,
обласкали, чтобы они писали хвалебные оды.

Россия, показав кем она есть, обнулила в самой Украине остатки ностальгии (у
тех, у кого она была) по прошлому и иллюзии относительно \enquote{братскости} в
будущем. Даже Гитлер не бомбил немецкоговорящих. А Путин показал свой \enquote{братский
оскал} и русскоговорящим городам Украины.

Хотя, надо сказать, достаточно долгое время после развала СССР Украина и Россия
как постсоветские государства переживали, во многом, схожие процессы.
Постсоветское пространство устроено так, что практически не поддается реформам.
Поэтому в России была сформирована сырьевая экономика, которая зависит от цен
на энергоресурсы. Поэтому в Украине была сформирована экспортно ориентированная
экономика, которая, во многом, зависела от внешнего спроса и цены на российский
газ.

При этом, с точки зрения менталитета/идентичности, политической системы и
движения государств, Украина и Россия – абсолютно разные. В РФ очень боялись
демонстрации этой разницы, прежде всего, на уровне идентичности – потому что
тогда бы пришлось ответить на вопрос и о собственной идентичности. И как только
Украина начала меняться и выходить постепенно из \enquote{постсоветской
трясины}, в России решили разрушить Украину. Но разрушат, в итоге, окончательно
не только существующие штампы, но и самих себя в перспективе. Последствия этой
войны будут по принципу \enquote{кругов на воде} и в разных сферах и масштабах.
Не удивлюсь, если УПЦ МП добровольно выйдет из РПЦ. А последствия будут
касаться всего постсоветского пространства (Беларуси, Молдовы, Грузии
особенно).

Ну а Украина поняла, что, кроме внешней поддержки, надо становиться сильной
изнутри. Быть сильной – значит неуязвимой. Украина показала, что может не
бояться российской армии, и побеждать.

И умной. Это российский подход – сила есть, ума не не надо. Мы себе этого не
можем позволить. Поэтому ещё один вывод – это важность действующих институций.
И тех, которые обеспечивают успех фронта в отражении агрессии. И тех, которые
обеспечивают работу тыла – и во время войны, и после войны.

Ну и красивой. В этой войне мы показали принципиальное отличие от оккупанта. И
смелостью, и единством, и чувством юмора. И мы говорим много о будущем, а не
носимся с портретами мертвых, как будто это праздник некромантии.

Часто вспоминаю два эссе Умбэрто Эко – \enquote{История красоты} и
\enquote{История уродства}.  Украина – это про красоту. Ну а кто себя показал
безобразно, как убийца и мародёр – видят все и знают все.
