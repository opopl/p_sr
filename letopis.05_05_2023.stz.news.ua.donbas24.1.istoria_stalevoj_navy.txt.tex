% vim: keymap=russian-jcukenwin
%%beginhead 
 
%%file 05_05_2023.stz.news.ua.donbas24.1.istoria_stalevoj_navy.txt
%%parent 05_05_2023.stz.news.ua.donbas24.1.istoria_stalevoj_navy
 
%%url 
 
%%author_id 
%%date 
 
%%tags 
%%title 
 
%%endhead 

Еліна Прокопчук (Маріуполь)
05_05_2023.elina_prokopchuk.donbas24.istoria_stalevoj_navy
Маріуполь,Україна,Мариуполь,Украина,Mariupol,Ukraine,Полон,Азовсталь,Валерія Суботіна,date.05_05_2023

Історія сталевої Нави — захисниця Маріуполя розповіла про весілля на Азовсталі та російський полон (ВІДЕО)

Майже рік Валерія Суботіна провела у неволі. Зараз дівчина вчиться жити далі з
пам'яттю про загиблого чоловіка

Історія кохання Валерії та Андрія Суботіних зворушила всю Україну. В цей час,
поки світ спостерігав за непохитністю захисників Маріуполя, на Азовсталь падали
бомби, а місто палало від ворожого вогню, у фортеці героїв було місце
найщирішим почуттям. Перебуваючи у пеклі війни, прикордонник Андрій зробив із
фольги обручку своїй коханій. Вона відповіла «так», і 5 травня, в день
народження полку «Азов», закохані одружилися. У пари було небагато часу разом —
вже 7 травня, виконуючи бойове завдання, Андрій Суботін загинув... А на Валерію
ще чекали нові випробовування — дні в оточеній Азовсталі, здача в російський
полон і майже рік у неволі. 10 квітня захисниця Маріуполя нарешті повернулася в
Україну.

Історію, сповнену болю, кохання та світлої пам'яті, вона розповіла ТСН.

Читайте також: Руїни замість будинків, розбиті «Азовсталь» та центр: Google
оновив карти Маріуполя (ФОТО)

Андрій з'явився, як диво

Валерія народилася та зростала у селищі під Шахтарськом. У цьому куточку
Донбасу, який потім окупували росіяни та назвали «исконно русской землей»,
споконвіку всі говорили українською, співали народних пісень та шанували
традиції. Маленька Лера теж говорила українською з близькими, але що таке
велика українська родина, ніколи не знала. З життя рано пішли її батьки, Леру
виховували дідусь та бабуся, які поховали власних синів — батька та дядька
дівчинки, які обидва були військовими льотчиками. Дідусь Лери багато хворів та
згодом помер, тож єдиною близькою людиною для дівчини довгий час залишалася
бабуся.

«У мене практично не було дитинства — не вистачало тепла та любові, було
недостатньо коштів. І коли мене ображали, я уявляла, що Бог сидить на великому
стільці, у нього велика ряса, і я, така маленька, заповзаю під той стілець і
лягаю спати. Бог зверху, і ніхто мене не може скривдити», — згадує через багато
років Валерія.

Потім був переїзд до Маріуполя, кар'єра у Маріупольському державному
університеті. Валерія Суботіна шість років працювала викладачем, стала
кандидаткою наук із соціальних комунікацій, писала вірші та до повномасштабної
війни встигла видати власну збірку поезій «Квіти та зброя».

У 2015 році Валерія залишила викладацьку діяльність та працювала у прес-службі
полку «Азов», а з 2017 року — очолила пресслужбу прикордонного загону Донецької
області. Там вона працювала до 2020 року, і саме там на неї чекало кохання.

«Андрій з'явився, як диво, — згадує Валерія. — Він працював у нашому
прикордонному загоні, був юристом, а я — пресофіцером. Я закохалася з першого
погляду! Це людина, з якою мені було максимально легко. Я завжди мріяла
зустріти доброго чоловіка, це така рідкість в світі! І тоді мені здалося, що
він мені був даний за все це дитинство, за все горе, яке було в моєму житті. Я
була з ним і дитиною, і жінкою, і була нереально щасливою».

Читайте також: Фотограф Євген Малолєтка переміг на World Press Photo — журі вразили світлини Маріуполя
«Розкопуйте Наву, вона жива»

З 2020 року до повномасштабної війни Валерія не служила у війську, але коли Росія вторглася на наші землі, дівчина за власним бажанням мобілізувалася і знову увійшла до пресслужби легендарного полку «Азов».

Валерія згадує: тоді з міста поїхала б лише за однієї умови. Вона хотіла вивезти з нього бабусю, а потім повернутися у Маріуполь. Та бабуся відмовилася, і Валерія одразу зателефонувала Калині, заступнику командира полку «Азов». Він сказав «приїзди». Так дівчина опинилася на Азовсталі. Андрій теж не поїхав з міста з прикордонниками і залишився захищати його з «Азовом».

15 квітня Валерія отримала поранення — на бункер, де вона знаходилася з
побратимами, впала авіабомба.

«Я почула голоси своїх побратимів: "Нава там, треба відкопувати, Нава жива".
Мене почали розкопувати, тягти, щось говорили про ноги. Я подумала, що в мене
немає ніг, бо я їх не відчувала. Виявилося, що їх просто привалило, — згадує
Нава. — Мене схопили Орест та Калина, понесли на руках у шпиталь, з голови
текла кров. Був сильний обстріл, і виявилося, що майже всі навколо мене мертві.
Ніхто не розумів, як так виявилося, що я була жива, це було диво».

Читайте також: Фільм про Маріуполь переміг на кінофестивалі у США 71-й день
війни став днем весілля

Ще у мирному Маріуполі Андрій хотів одружитися. Валерія чекала, коли відросте
волосся — мріяла зробити гарну зачіску, коли стане нареченою. Все змінила війна
— пропозицію коханий зробив Валерії з обручкою із фольги. Там, у підземеллях
Азовсталі, Валерія говорила, що ще буде час одружитися... Але, звісно, погодилася
на пропозицію і вони розписалися.

71-й день повномасштабного вторгнення став для пари днем весілля. Вони обрали
знакову дату — 5 травня — день народження полку «Азов».

"Не було якогось розпису, обміну клятвами, просто було щастя, що він прийшов",
— згадує дівчина.

Андрій Суботін загинув 7 травня, а Валерія дізналася про це 9-го. Та досі
дівчина відчуває, що він поруч. Тому розповідає усім, яким він був. Як сміливо
дбав про інших. Як носив їжу дітям у бункері. Як допомагав евакуювати поранених
побратимів. Як рятував життя інших і не боявся втратити своє.

«Андрій мені постійно снився у Таганрозі, я не знала, де було його тіло. І
одного разу мені наснилося, що він в Маріуполі. Він сказав — "мені тут добре, я
тут з хлопцями"», — згадує Валерія Суботіна.

Читайте також: Фільм про Маріуполь переміг на кінофестивалі у США Смерть була б
легшою за полон

«Для мене смерть була б набагато легшою, ніж все те, що довелося пережити. Це
була б можливість бути з моїм Андрієм. А я пішла в полон до людей, які в мене
його забрали, які забрали в мене все. Це було найважче», — розповідає Нава.

Досі дівчина не відчула, що повернулася додому. Вже не боїться прокидатися
зранку, але що б не робила, думає, як прокидаються в полоні наші захисники, що
вони їдять, яку мову чують.

«Не можу повернутися з полону, поки вони там», — говорить захисниця Маріуполя.

Перед поверненням в Україну не було ані речей, ані грошей. Валерія разом з
іншими захисницями думали: як жити далі, як виходити у порожнечу.

«Ми вийшли і зрозуміли, як Україна зустрічає своїх. Нас нагодували, дали багато
одягу, я не розуміла, навіщо мені два чи три светри! Ми співали гімн, до нас
бігло стільки людей», — говорить про свої перші радісні спогади Валерія.

Бабуся завжди вчила дівчину виглядати гарно. І весь час у полоні Нава мріяла
лише про гребінець для волосся! Разом з подругою вони уявляли, як бабуся
зустріне онуку і посварить за сивину. «Навіть тюрма — не привід ходити так», —
сказала б обов'язково жінка.

Вже після виходу з полону Валерія дізналася, що бабуся померла в окупованому
Маріуполі. А коли відкрила сумку, яку їй зібрала мама Андрія, побачила такий
омріяний гарний гребінець.

«Зараз в мене є батьки Андрія, які мене безмежно люблять. Такої любові я ніколи
не відчувала. Мені дуже б хотілося помінятися з Андрієм місцями, щоб вони могли
любити його живим. Але я намагаюся жити далі», — говорить Нава.

Після виходу з полону дівчина одразу зробила татуювання зі словом «сталева».
Розповідає: це про всю Азовсталь, про всіх наших сталевих, про весь «Азов» —
хлопців і дівчат, яких не зламати.

Наразі на честь річниці полку «Азов» та річниці весілля Валерії та Андрія про
пару вийшла пісня «Сталеве кохання». Леся Шиденко та Григорій Попович написали
слова, а виконала її Марія Бурмака. Пісня стане саундтреком документального
фільму Станіслава Сукненко «Азовсталь прихисток кохання».

Нагадаємо, раніше Донбас24 розповідав, що в Україні може з'явитися День
вшанування захисників Маріуполя.

Ще більше новин та найактуальніша інформація про Донецьку та Луганську області
в нашому телеграм-каналі Донбас24

ФОТО: з відкритих джерел
