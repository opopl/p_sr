% vim: keymap=russian-jcukenwin
%%beginhead 
 
%%file 18_04_2018.stz.news.ua.mrpl_city.1.knizhna_toloka_vernutsja_2019
%%parent 18_04_2018
 
%%url https://mrpl.city/blogs/view/pochemu-mariupolskaya-knizhnaya-toloka-dolzhna-vernutsya-v-2019-godu-i-kak-odin-festival-mozhet-obedinit-vsyu-ukrainu
 
%%author_id kuranova_roksana.mariupol,news.ua.mrpl_city
%%date 
 
%%tags 2018,2019,chtenie,festival,kniga,kultura,literatura,mariupol,mariupol.pre_war,toloka
%%title Почему "Мариупольская книжная толока" должна вернуться в 2019 году, и как один фестиваль может объединить всю Украину
 
%%endhead 
 
\subsection{Почему "Мариупольская книжная толока" должна вернуться в 2019 году, и как один фестиваль может объединить всю Украину}
\label{sec:18_04_2018.stz.news.ua.mrpl_city.1.knizhna_toloka_vernutsja_2019}
 
\Purl{https://mrpl.city/blogs/view/pochemu-mariupolskaya-knizhnaya-toloka-dolzhna-vernutsya-v-2019-godu-i-kak-odin-festival-mozhet-obedinit-vsyu-ukrainu}
\ifcmt
 author_begin
   author_id kuranova_roksana.mariupol,news.ua.mrpl_city
 author_end
\fi

Когда я впервые узнала о том, что
\href{https://mrpl.city/news/view/knizhnaya-toloka-vpervye-priedet-v-mariupol}{в %
Мариуполе будет проводиться книжный фестиваль}, тем более такого уровня, как
\enquote{Запорожская книжная толока}, то первое, что почувствовала – это страх. Почему
именно это чувство, а не прогнозируемое счастье для такого книжного червя, как
я? Страх за то, что не оценят по достоинству, страх за уровень организации,
несмотря на то что я была косвенно знакома с людьми, которые делали
\enquote{Запорожскую толоку}. Но побороть эту эмоцию было очень сложно. Со временем и
организаторы показали себя с лучшей стороны, и горожане с энтузиазмом
восприняли новость о первом литературном фестивале в Мариуполе. И на место
страха постепенно пришло любопытство и сладостное предвкушение.

\ii{18_04_2018.stz.news.ua.mrpl_city.1.knizhna_toloka_vernutsja_2019.pic.1}

\href{https://mrpl.city/news/view/knizhnaya-toloka-v-mariupole-kak-prohodit-pervyj-den-literaturnogo-festivalya-foto-plusvideo}{Утро
субботы}. СК \enquote{Ильичевец} и оживленный поток людей, как на \enquote{Центральном} рынке в
эти же дни круглый год. Ух, отлегло. Люди есть, очереди на ярмарке есть,
обсуждение литературы со всех сторон тоже имеется. Поразило разнообразие
публики, которую объединяло одно – предчувствие праздника на лицах и охапки
купленных книг в руках. \enquote{Толока} стала действительно местом встреч, здесь можно
было увидеть старых друзей, первых учителей, знакомых из других городов и
всех-всех-всех. Лично мои долгожданные встречи – это презентация Алексея Чупы,
Артема Полежаки, выступление Издрыка и \enquote{DRUMТИАТРА}, \enquote{урбаністичний майданчик}
с архитектром из Киева Алексеем Быковым, архитекторкой из Львова Христею
Колясою и кураторкой культурных мероприятий из Запорожья Натальей Лобач, лекция
\enquote{Пишем про женщин} и большие открытия фестиваля - поэты из Киева: Антон
Полунин, Иван Кулинский и Ута Плуто.

\ii{18_04_2018.stz.news.ua.mrpl_city.1.knizhna_toloka_vernutsja_2019.pic.2}

Признаюсь, в первый день было очень сложно собраться и успеть на все
мероприятия, которые запланировала накануне. Тем более что планировать что-либо
было бесполезно. По дороге на нужную локацию легко можно было зависнуть в
\enquote{Книголесье} на боях роботов или викторине по \enquote{Гарри Поттеру}, где у меня был
достойный 10-летний соперник. Или, проходя мимо оживленной дискуссии, невольно
остановиться под влиянием харизматичных спикеров. Что касается литературы, ее
было много, самой разной, но все-таки не хватало научно-популярной и
узкопрофильной профессиональной литературы, которую искали, например, мои
друзья. Очень много было патріотичних книг, содержание которых оставляло желать
лучшего. Ну, правда, too much. В любом вопросе, а тем более в литературе, я за
качество, а не количество и яркие жовто-блакитні обложечки. 

\ii{18_04_2018.stz.news.ua.mrpl_city.1.knizhna_toloka_vernutsja_2019.pic.3}

Именно \enquote{Книжная толока} показала, насколько Мариуполю не хватает нового,
масштабного выставочного центра. В городе не оказалось ни одной удобной
локации, чтобы разместить 9 площадок фестиваля и книжную ярмарку. СК
\enquote{Ильичевец} стал меньшим из зол, но, думаю, что не я одна заметила, как
неудобно были расположены 3-7 локации, куда просто не доходили люди. Часто
можно было встретить блуждающие по второму этажу спорткомплекса души. Думаю,
именно это сказалось на том, что на некоторых лекциях и презентациях было не
так много людей, как хотелось бы. Городской власти есть на что обратить
внимание и в каком направлении двигаться, если есть желание, чтобы в Мариуполе
подобные мероприятия проходили на высшем уровне.

\ii{18_04_2018.stz.news.ua.mrpl_city.1.knizhna_toloka_vernutsja_2019.pic.4}

Второй момент, который меня неприятно удивил – это идея нанять \enquote{Правый сектор}
для охраны мероприятия. Особенно учитывая, сколько мероприятий за последнее
время было сорвано в Украине праворадикальными организациями просто потому, что
им не нравились темы обсуждений. Набыченные, угрюмые хлопцы, которые
обсматривают посетителей и обсуждают между собой, как кого-то \enquote{ломают}, совсем
не вяжутся с атмосферой литературного фестиваля. До сих пор думаю о том, что
было бы, если бы на толоке подняли на дискуссиях вопрос о правах
ЛГБТ-сообщества, обсуждали гендерную идентичность или коснулись любой другой
темы, которая бы не вписывалась в их \enquote{традиционные ценности}.

\ii{18_04_2018.stz.news.ua.mrpl_city.1.knizhna_toloka_vernutsja_2019.pic.5}

Но это был первый \enquote{блин}, и однозначно он не стал комом. Ощущение праздника,
новые знакомства, литературные открытия, приобретенные книги, нереальная
атмосфера – все это было! Фестиваль задумывался как средство разрушить
стереотипы о Мариуполе как об исключительно промышленном городе и о
мариупольцах как о не читающих книги на украинском языке. Думаю, получилось. До
меня доходило много удивленных и восторженных отзывов как от гостей, так и от
местных жителей. Что касается участников фестиваля - кто был открыт и настроен,
чтобы познакомиться с городом и узнать поближе мариупольцев, тот получил свое.
Что ж, а те, кто изначально ехал подкреплять свой скептицизм, остались при
своем мнении, и тут уж никакая \enquote{Книжная толока} не поможет. \href{https://mrpl.city/news/view/vernetsya-li-knizhnaya-toloka-v-mariupol-ob-uspehah-i-perspektivah-foto}{Стоит ли проводить
этот фестиваль в Мариуполе ежегодно – однозначно ДА}. Есть и спрос, и огромное
множество фидбэков, и заинтересованность в будущей организации мероприятия, а,
как сказал директор фестиваля Игорь Гармаш, при таких условиях и финансирование
реально найти. Именно фестивали такого рода показывают, что Мариуполь открыт
новым людям, он ждет, чтобы его исследовали и переосмысливали. А еще «Книжная
толока» действительно объединяет украинцев, независимо от того, из какой части
нашей страны они родом.

\textbf{Так что, \enquote{Книжная толока}, возвращайся в 2019 году, Мариуполь ждет!}
