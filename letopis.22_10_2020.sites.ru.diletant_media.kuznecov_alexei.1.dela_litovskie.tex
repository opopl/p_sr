% vim: keymap=russian-jcukenwin
%%beginhead 
 
%%file 22_10_2020.sites.ru.diletant_media.kuznecov_alexei.1.dela_litovskie
%%parent 22_10_2020
 
%%url https://diletant.media/articles/45296446/
 
%%author Кузнецов, Алексей
%%author_id kuznecov_alexei
%%author_url 
 
%%tags 
%%title Дела литовские
 
%%endhead 
 
\subsection{Дела литовские}
\label{sec:22_10_2020.sites.ru.diletant_media.kuznecov_alexei.1.dela_litovskie}
\Purl{https://diletant.media/articles/45296446/}
\ifcmt
	author_begin
   author_id kuznecov_alexei
	author_end
\fi

\index[rus]{Русь!История!Литовская Русь, 22.10.2020}

В «Грозе» Островского один из гуляющих по бульвару высказывает предположение,
что «Литва на нас с неба упала». В переносном смысле так оно и есть.

\ifcmt
pic https://diletant.media/upload/iblock/f03/f03cfb76c5b45af30acee2a546150021.webp
\fi

\subsubsection{Иная Русь}

Учившиеся в советской школе и твёрдо знавшие, что за периодом Киевской Руси и
раздробленности следует этап Московского царства, из которого вырастет потом
Российская империя, мы затем вдруг с удивлением обнаружили, что в течение пяти
с половиной веков, начинаясь в иные годы практически сразу за Можайском,
существовало огромное государство. Брянск и Смоленск, Вязьма и Курск были
литовскими городами, большинство населения говорило на восточнославянских
диалектах, почти не отличавшихсяот тогдашнего московского; немалая часть его
исповедовала православие.

Литовские князья нередко роднились со своими московскими «партнёрами»:
например, Василий II Тёмный был внуком Дмитрия Донского и Витовта. В период
роста Московского княжества некоторые его соседи (Новгород, Тверь) всерьёз
размышляли, под чью «высокую руку» перейти — московского или литовского князя.
Иными словами, Литовская Русь до поры до времени не менее настойчиво, чем
Московская, претендовала на то, чтобы стать центром объединения «осколков»
Киевской Руси. 

\ifcmt
  pic https://diletant.media/upload/img_html_editor/45296446/13a/13acc8867edcac3ac86837498adab572.webp
  caption Великое княжество Литовское в XIV-XV веке. Источник: konturmap.ru
\fi

«Срединное» положение между Великим княжеством Московским и Польским
королевством, Крымским ханством и Ливонским орденом вынуждало литовские власти
постоянно маневрировать, заключать и расторгать союзы, а внутри
многоконфессиональной державы — поддерживать баланс сил. Так было при Гедимине
и Ольгерде, так было при Витовте и Ягайло, так было при Казимире IV
Ягайловиче, по-польски — Казимеже Ягеллончике.

Последнему приходилось особенно несладко. На западе у него, бывшего
одновременно польским королём и великим князем литовским, своевольничали
польские магнаты, считавшие короля «первым среди равных» (с акцентом на
последнем слове), в Восточном Поморье шла вялотекущая война с Тевтонским
орденом за выход к Балтике, на юго-востоке Орда предпринимала последние попытки
восстать из руин, на литовской территории православные князья посматривали в
сторону Москвы, чуть что, угрожая «отъехать».

В такой обстановке управление большой и сложной страной становилось подобно
искусству дирижёра симфонического оркестра. И Казимир в целом справлялся
успешно. 

\subsubsection{Уесть по-родственному}

Как и положено, монарху досаждали родственники, чем более близкие — тем
сильнее. Он платил им той же монетой, внимательно следя за тем, чтобы они не
усилились сверх меры. Ярким примером такого семейного противостояния стали
события начала 1480-х годов.

Князь слуцкий Михаил Олелькович, сын киевского князя Александра Владимировича
(Олелько по-польски — уменьшительное от Александр), приходился правнуком
Ольгерду, двоюродным братом великому князю Московскому Ивану III и двоюродным
племянником самому Казимиру.

С таким генеалогическим багажом он мог бы, как ему казалось, рассчитывать на
нечто большее, чем Слуцкое княжество — крупное, влиятельное, но лишь одно из
двух десятков. До поры до времени можно было надеяться, что оно лишь этап на
пути к Киеву, «отчине», доставшейся после смерти отца старшему брату Семёну.
Однако тут его ждало разочарование: в 1470-м его, как православного, отправили
князем в Новгород на усиление «литовской партии». 

\ifcmt
pic https://diletant.media/upload/medialibrary/8b9/8b99cea05d0a13115b2db8d4ee206a7a.webp
caption Казимир IV. Ян Матейко, 1890-е годы. Источник: журнал «Дилетант»
\fi

В строптивом городе-столице ещё одной «альтернативной Руси» прижиться ему не
удалось: аккурат перед его приездом скончался инициатор его приглашения (князь
в республиканско-олигархическом Новгороде был не наследственным правителем, а
приглашаемым должностным лицом) архиепископ Иона, и «московская партия»,
подпитываемая из Москвы духовно и денежно, начала брать верх. Неудивительно,
что, узнав о скоропостижной смерти брата, Михаил бросил Новгород и отправился
в Киев.

Вспоминать его добрым словом у новгородцев особых оснований не было: с него
началось распространение в городе ереси «жидовствующих», а внезапный отъезд
окончательно ослабил сторонников ориентации на Литву, и в решающей битве с
московской ратью на реке Шелони в июле 1471-го они остались без
квалифицированного военного руководителя. 

\ifcmt
  pic https://diletant.media/upload/medialibrary/7d4/7d4f247c634aaea221cda3350f1be985.webp
  caption Киевский князь Александр Владимирович. Источник: журнал «Дилетант»
\fi 

Хитрый Казимир тем временем воспользовался отсутствием Михаила и вероломно
изменил статус Слуцкого княжества, превратив его в воеводство, напрямую
подчиняющееся великому князю, и посадив там наместником шурина покойного князя
Мартина Гаштольда. Это вызвало не только приступ бешенства у Михаила, но и
недовольство киевлян; однако Казимиру удалось настоять на своём, как отмечает
хорошо информированный польский хронист: «Но наконец киевляне, боясь силы
короля, смирились». Внешне смирился и Олелькович, но всё запомнил; как показали
дальнейшие события, на память он не жаловался. 

\subsubsection{Чего они хотели?}

1480 год оказался важнейшим в истории Восточной Европы. Хан Большой Орды
(центрального «осколка» Золотой Орды) Ахмат, рассчитывая на поддержку своего
союзника, литовского князя, идёт «усмирять» мятежную Москву. Казимир от
нападения на восточного соседа воздерживается: его самого беспокоят союзные
Москве крымские татары, опасающиеся в свою очередь амбиций хана.

В результате Иван III освобождается от остатков ордынской зависимости (пока ещё
вряд ли понятно, что навсегда), возможно, Михаил Олелькович и ряд других
православных вельмож Литвы видят в этом свой шанс. Складывается заговор, помимо
слуцкого князя туда входят его двоюродный брат Фёдор Бельский и двоюродный дядя
Иван Гольшанский, а также князь Иван Глинский. 

\ifcmt
  pic https://diletant.media/upload/medialibrary/1b4/1b4b3120a778760b79d33db4d4528902.webp
  caption Ольгерд, великий князь литовский. Источник: welcome-belarus.ru
\fi

О целях заговора современники высказывали разные суждения. Софийская летопись
полагает, что его участники собирались с обширными землями отделиться и
присоединиться к Москве: «Того же лета бысть мятеж в Литовской земле: восхотеша
вотчичи Ольшанский да Олелькович, да князь Федор Бельский по Березыну реку
отсести на великого князя (Ивана III. — Авт.)».

Другая точка зрения предполагает у князей стремление самим занять
великокняжеский престол (с королевским в этом случае получалась более сложная и
долгая история, так как польско-литовская уния пока была личной и
автоматического соединения титулов не предусматривала).

Часть комментаторов (например, видный литовский государственный деятель 1520-х
годов Альбрехт Гаштольд) видит в этом попытку антикатолического переворота со
стороны «православной партии», другие усматривают лишь личные амбиции
заговорщиков. 

\ifcmt
  pic https://diletant.media/upload/medialibrary/7ef/7ef2a0623f6847e67e3f0e77a965f950.webp
  caption Иван III Васильевич. Парсуна XVII века. Источник: журнал «Дилетант»
\fi

Отдельным и крайне спорным остаётся вопрос о «руке Москвы»: была или нет, знал
Иван III или не знал. Прямых доказательств нет, косвенные имеются «в обе
стороны». Сама идея была, прямо скажем, незамысловатой — планировалось убить
Казимира во время свадьбы Фёдора Бельского, куда он был приглашён и подтвердил
намерение явиться. Однако дело вышло наружу; по наиболее распространённой
версии, о нём прознал старый недоброжелатель заговорщиков киевский наместник
Иван Ходкевич (помимо всего прочего понимавший, что ему, как сидящему «на
отчине» Михаила Олельковича, в случае успеха предприятия несдобровать).

Впрочем, есть версия, что доносчиком был брат Фёдора Бельского Семён, вероятно,
зарившийся на родовое имение — Белую крепость на Смоленщине (во всяком случае,
после провала заговора и поспешного бегства брата он её получил).

Наконец, несколько источников связывают дело с одним из слуг Казимира, который
якобы обнаружил в замке Бельского тайный склад оружия и сообразил, для чего оно
предназначено; слуги Фёдора Ивановича под пыткой выдали господина… 

\subsubsection{И что они получили}

Бельский бежал прямо с брачного ложа, жену его потом долгие годы пыталась
«выговорить» московская дипломатия, но тщетно, им отвечали, что она не хочет
воссоединяться с мужем; Глинскому тоже удалось «утечь». Гольшанского и Михаила
Олельковича судили. Интересно, что ни одного документа этого процесса ни в
подлиннике, ни в копии не сохранилось; при этом современники, писавшие об этом
деле, прямо или косвенно указывают, что их всё-таки судили.

Как это могло выглядеть? Судебная система Великого княжества Литовского в
современном понимании этого слова находилась в стадии становления. В 1468 году
был принят Судебник, до этого применялись нормы «Русской Правды»,
великокняжеских указов (привилеев) и обычаев. Все суды были сословными:
гродские, они же за́мковые, суды для горожан, вотчинные — для зависимых
крестьян, земские — для шляхты, копные рассматривали споры внутри крестьянской
общины. Над всем царил великокняжеский суд, назначавшийся для разбора
конкретных значительных дел. Действовал феодальный принцип суда равных:
шляхтичей судили шляхтичи, горожан — горожане. 

\ifcmt
  pic https://diletant.media/upload/img_html_editor/45296446/5a5/5a53c9b74de2fe94039124e36f1823b1.gif
  caption Свадьба Фёдора Бельского и княжны Анны. Источник: wikipedia.org
\fi

Из всего этого следует, что Казимир для суда над двумя князьями Гедиминовичами
должен был назначить специальное судебное присутствие из состава высшей
аристократии. Судя по всему, так и было сделано: Альбрехт Гаштольд указывает,
что одним из членов суда был его отец Мартин, воевода трокский (Троки — ныне
Тракай), а до этого, как мы помним, киевский.

Видимо, доказательств было достаточно (по понятиям того времени, разумеется), а
судьи были настроены вполне проказимировски, и факт заговора не вызвал
сомнений. В этом случае «Русская Правда» (Судебник 1468 года не касался
вопросов охраны княжеской жизни) предусматривала однозначное наказание —
смерть; князья Иван и Михаил были четвертованы 30 августа 1481 года. Однако же
— и это подчёркивается некоторыми источниками — кара не коснулась семей
казнённых (а могла бы, причём по закону), и в июле 1483 года вдове Михаила
княгине Анне была выдана специальная великокняжеская грамота, обязывавшая
вассалов покойного продолжать служить, как прежде: «И мы на ее жаданье
(просьбу. — Авт.), тым князем и бояром и слугам из их имений велели ей служити
и послушными быти, как князю Михайлу служили и его послушны были».

Любопытно, конечно, представить себе, как развивались бы события, удайся
заговорщикам устранить Казимира. Пофантазировать на эту тему можно — почему
нет? — но уж точно не сто́ит с уверенностью говорить об отказе Великого
княжества от унии с Польшей и переходе в состав Московской Руси. Скорее всего,
началась бы феодальная война вроде той, которая во второй четверти XV века
произошла «на Москве», только ещё и с религиозными осложнениями. И тут уж гадай
не гадай — не угадаешь…

Не так давно при раскопках Успенского собора Киево-Печерской лавры в крипте
Олельковичей были найдены останки человека, явно подвергшегося в своё время
четвертованию. Весьма вероятно, что они принадлежат мятежному князю Слуцкому. 
