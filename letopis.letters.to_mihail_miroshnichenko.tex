% vim: keymap=russian-jcukenwin
%%beginhead 
 
%%file letters.to_mihail_miroshnichenko
%%parent letters
 
%%url 
 
%%author_id 
%%date 
 
%%tags 
%%title 
 
%%endhead 

Добрый вечер, Михаил! Меня зовут Иван, я киевлянин, программист и живу и работаю
в Киеве. Много сейчас читаю про Мариуполь, у меня есть также несколько книжек
про Ваш замечательный Город. К сожалению, до войны не успел побывать в
Мариуполе, так что меня... с Мариуполем связывает только воспоминание о том,
как я когда то сел на поезд Киев-Мариуполь, и потом сошел на промежуточной
станции... Давно это уже было... Но... я принимаю близко к сердцу, все, что
произошло с Мариуполем и с мариупольцами. Страшная, невообразимая трагедия...
которая забрала ваших близких, родных, детей, родителей, раскидала вас,
мариупольцев, по всему миру... Тем не менее, я уверен, лучшие времена для
Мариуполя еще впереди, и отчаиваться точно не надо... Да, как я сказал выше, я
программист. В свободное время от работы... я также занимаюсь систематизацией
разных публикаций о войне, у меня есть отдельный проект Летописи Войны, который
я веду - пока что сам - довольно уже давно, собственно говоря, я сохраняю от
забвения посты на фейсбуке, потом раскидываю их по авторам, темам. Технология,
которую я использую, называется LaTeX (не знаю, имеете ли вы отношение к физике
или математике, но это та технология, которая повсеместно используется учеными
в научном мире для записи и публикации своих статей и исследований ). Так вот.
Я сохраняю в печатном виде, то, что я читаю, считаю важным для сохранения...
Про Мариуполь я в последнее время тоже сохраняю, и довольно много. И про
довоенный мирный Мариуполь, и для ужасы войны. У меня есть телеграм канал
https://t.me/kyiv_fortress_1, там уже довольно много выложено, так же как и на
этом фейсбук - аккаунте, также я постепенно выкладываю собранные  материалы - с
уважением к авторам - я всегда указываю, кто и когда что создал - на
https://archive.org - это интернет сайт, посвященный сохранению всего в
Интернете, ссылка на меня там https://archive.org/details/@kyiv_chronicler.
Зачем все это делается... знаете... Мариуполь физически то убили... но
Мариуполь не умер. Мариуполь остался в Духовном Пространстве, и это очень очень
важно... Но если не записывать... если этим не заниматься... есть риск, что
Мариуполь умрет духовно... и это будет уже навсегда. А духовная смерть - это...
еще страшнее, чем смерть физическая... Поэтому я этим и занимаюсь, хотя я и
киевлянин, вообще то говоря. Хочется, знаете, тоже приехать в наш Украинский
Возрожденный Мариуполь, как гость, в свое время ) Надеюсь, все о чем я пишу,
Вам будет интересно. 

Я вижу, Вы в последнее время выкладываете в фейсбуке свой дневник о блокадном Мариуполе,
буду рад, если Вы разрешите его тоже выложить в сохраненном виде. 
К сожалению, если не записывать систематически, а просто выкладывать в фейсбуке,
то есть риск, что (1) мало людей смогут узнать Вашу историю (2) фейсбук сам по себе очень плох
в плане сохранения памяти, из-за своей непредсказуемой политики (3) ну и конечно, нельзя исключить
такого, что фейсбук просто рухнет сам по себе, и все, что в нем есть, просто исчезнет навсегда. 

Отдельно хочу отметить ужасную ситуацию в Киеве и вообще по Украине - с книжками про Мариуполь - 
их очень мало, либо вообще нет. Об этом я более детально пишу в своем посте здесь 
https://www.facebook.com/ivan.ivan.kyiv/posts/pfbid02WyjjCrDz4v3PJuHKkKMm2Fxg1ninb3ZibGgekSsQuF2WYZ7npVwk7h1qEcLV9u4sl

Я послал Вам запрос на дружбу в фейсбуке, буду рад
подружиться с Вами здесь. 

С уважением, Иван.
