% vim: keymap=russian-jcukenwin
%%beginhead 
 
%%file 27_07_2020.fb.lnr.2
%%parent 27_07_2020
 
%%endhead 

\clearpage
\subsection{2 --- Как в ДНР работает организация по поиску пропавших без вести}
\label{sec:27_07_2020.fb.lnr.2}
\url{https://www.facebook.com/groups/LNRGUMO/permalink/2880707038707518/}

\index{LNRGUMO}

Представители Союза матерей без вести пропавших сыновей рассказали в интервью
News Front о продвижении поиска пропавших в Донбассе.

«Союз матерей без вести пропавших сыновей» --- это общественная организация,
которая осуществляет поиск людей по заявкам родственников, проживающих на
территории ДНР, а также граждан других государств.

Сергей Филипов, зам. председателя ОО «Союз матерей без вести пропавших сыновей»
рассказал, что данная организация была создана инициативной группой 22 июля
2018 года и зарегистрирована в феврале 2019 года. «Организация открыта для
всех, кто готов оказать содействие в поиске пропавших без вести».

О процессе поиска людей подробно сообщил Андрей Седлов, правозащитник «Союза
матерей без вести пропавших сыновей», отметив что в работе поисковика важен
первый контакт, в котором важно чтоб человек, который кого-то разыскивает
пропавшего доверил все, что знает, также важна надежда на то, что пропавший
человек обязательно будет найден.

Лариса Слюсарева, родственница пропавшей без вести рассказала о розыске
родственника: «пропала 6 мая этого года, она вышла из квартиры, где ее снимали
и направлялась к нам, по нашему адресу. Мы объездили все больницы, все что
можно и не нашли ее. Узнали об этой организации из интернета, решили
обратиться, в надежде что нам помогут в розыске.

Работа общественной организации «Союз матерей без вести пропавших сыновей»
заключается в содействии поиску людей пропавших без вести, а также в оказании
психологической и юридической помощи родственникам и близким. На горячую линию
организации за помощью обращаются не только жители ДНР, но и даже люди,
проживающие на территории Донбасса, временно не подконтрольной ДНР. В
организации им помогают в поиске людей, контакт с которыми был утерян во время
вооруженного конфликта.  Кроме этого у организации есть и другие направления
работы --- юридическое сопровождение, психологическая поддержка, а также связь с
бюро Республиканской судебно-медицинской экспертизы.

Авторская съемка военкора команды News Front Катерины Катиной.
