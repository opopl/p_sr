% vim: keymap=russian-jcukenwin
%%beginhead 
 
%%file 25_06_2018.fb.lesev_igor.1.po_erdoganu
%%parent 25_06_2018
 
%%url https://www.facebook.com/permalink.php?story_fbid=1967368339960926&id=100000633379839
 
%%author_id lesev_igor
%%date 
 
%%tags erdogan_redzhep_tajip,turcia
%%title По Эрдогану
 
%%endhead 
 
\subsection{По Эрдогану}
\label{sec:25_06_2018.fb.lesev_igor.1.po_erdoganu}
 
\Purl{https://www.facebook.com/permalink.php?story_fbid=1967368339960926&id=100000633379839}
\ifcmt
 author_begin
   author_id lesev_igor
 author_end
\fi

По Эрдогану.

Когда чел организовывает досрочку, очевидно, что он намерен ее выиграть. А в
случае с Турцией при почти тотальном контроле СМИ и даже интернет-пространства,
победа Эрдогана читалась без открытия книги.

Но теперь он еще и контролирует парламент, создавая суперпрезидентскую модель
правления. По сути, Турция становится такой себе вариацией Азербайджана или
Казахстана, где первое лицо становится божеством.

\ifcmt
  ig https://scontent-frx5-1.xx.fbcdn.net/v/t1.6435-9/35923084_1967368103294283_8166082690672492544_n.jpg?_nc_cat=111&ccb=1-5&_nc_sid=730e14&_nc_ohc=bUFc2u5XI94AX9GuUWH&_nc_ht=scontent-frx5-1.xx&oh=9a0cc1bcb7594b8f565e8d30186b131d&oe=61BBFC46
  @width 0.4
  %@wrap \parpic[r]
  @wrap \InsertBoxR{0}
\fi

Правда, Эрдогану по степени влиятельности в своей стране стать разновидностью
Алиева или Назарбаева будет сложнее. Не то, чтобы у турок большие традиции
демократического характера. Нет у них таких традиций, да и неоткуда взяться.

Но в Турции на протяжении всего прошлого века сложилась специфическая традиция
противостояния светскости и политического клерикализма. За первых всегда топили
военные, которые регулярно организовывали военные перевороты и ставили ручных
политиков в Анкаре, которые затем снова борзели и срывались с поводка.

После шестнадцатого года и неудавшегося переворота, военным переломили хребет.
И теперь против Эрдогана выступает такая себе местная навальщина,
сосредоточенная в самых развитых регионах на западе страны – Стамбул, Измир и в
целом, все «пригреческое» побережье.

А на востоке у Эрдогана свой «Кавказ» в виде диких курдов. Диких, по мнению
турок-националистов, естественно. В теории курды и либералы запада страны могли
бы создать ситуативный антиэрдогановский альянс, но по результатам оглашенных
выборов в парламенте, они УЖЕ пролетели.

И теперь Эрдоган будет ипать и западников, и сепаров. Заметьте, ситуация очень
похожая с украинскими реалиями, где восток – это «курды», а промайданные
патриоты – разновидность любителей ЕС из Измира и Стамбула.

Но на этом сходство между Турцией и Украиной заканчивается. Во-первых, в
Украине нет своего Эрдогана. А во-вторых, нет политических сил, отстаивающих в
первую очередь субъектность самой Украины. Есть или сторонники «возвращения
домой в русскую гавань», или «дам кому угодно бесплатно, но только не с
москалем».

Но у каждого свой путь. Турция имеет с Украиной очень схожее, почти
дублирующее, экономико-географическое положение. Только еще тридцать лет назад
турки были недоразвитой страной, единственной ценностью которой был контроль
над проливами. А Украина по выпуску промышленной продукции опережала Италию.

Но все стало на свои места. Турки умудряются одновременно получать на своих
условиях русские ЗРК и американские самолеты 5 поколения. Проводить СВОЮ
политику в Сирии, отличную от Москвы и Вашингтона. Доить Брюссель на бабло по
беженцам и на кую вертеть ЕС во всех вопросах, когда те лезут во внутреннюю
политику страны. Перехватывать у тупых болгар газопровод из России и
одновременно запускать дублирующий газопровод из Закавказья, нарушая
монопольные няшки «Газпрома».

Турки наглые, решительные и удачливые. И имеют то, что заслужили. Как и все
остальные. Включая Украину.

\ii{25_06_2018.fb.lesev_igor.1.po_erdoganu.cmt}
