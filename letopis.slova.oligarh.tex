% vim: keymap=russian-jcukenwin
%%beginhead 
 
%%file slova.oligarh
%%parent slova
 
%%url 
 
%%author 
%%author_id 
%%author_url 
 
%%tags 
%%title 
 
%%endhead 
\chapter{Олигарх}
\label{sec:slova.oligarh}

Власть \emph{олигархов} сможет преодолеть только обновленное общество,
Петр Олещук, strana.ua, 04.06.2021

Також \emph{олігархам} заборонять фінансувати політичні партії та брати участь у
приватизації. Звісно, вони, наче, і раніше нікого не фінансували. Усі наші
партії зі славного \emph{олігархічного} минулого живилися лише \enquote{внсками}. Але якщо
добре \enquote{копнути}, то \emph{олігахічні} сліди виявляться у багатьох. Власне, істерика
окремих політичних партій тут дуже показова,
\textbf{Власть \emph{олигархов} сможет преодолеть только обновленное общество},
Петр Олещук, strana.ua, 04.06.2021

Щодо самої концепції, то ми не повинні забувати про одну дуже просту річ.
\emph{Олігархи} у нас тому і \emph{олігархи}, що вони не просто інтегровані у систему, не
просто її очолюють - вони її сформували. Наприклад, коли у нас говорять про
\enquote{корумповані суди}, то це вони, у першу чергу, їх і корумпували. Коли
відбувався переділ власності, а суддя був досить зручним (і не найдорожчим)
інструментом вирішення проблеми. Радянські судді, які раніще просто
реалізовували волю політичного керівництва, радо стали високооплачуваним
інструментом \emph{олігархічного} контролю.  Сподівання якось вплинути на \emph{олігархів}
через українські суди - це щось типу віри у Санта-Клауса. Звісно, нічого
поганого у цьому немає, але і на ефект розраховувати не варто. Партії, ЗМІ, які
ЗМІ не є, бо не заробляють на інформації, суди... Все це - продукти
\emph{олігархічної} моделі. Ну і розраховувати, що всі ці інституції будуть
\enquote{боротися з олігархами} - це вірити у те, що \emph{олігархи} покарають себе
самі,
\textbf{Власть \emph{олигархов} сможет преодолеть только обновленное общество},
Петр Олещук, strana.ua, 04.06.2021

Реестр \emph{олигархов} не решит важнейших проблем нашей экономики.  Он не
поднимет Украину с уровня Сомали и Зимбабве, где она оказалась.  МЫ БЫСТРО
СКАТИЛИСЬ ДО УРОВНЯ СОМАЛИ И ЗИМБАБВЕ, И ЗАКОНОПРОЕКТ ОБ \emph{ОЛИГАРХАХ} УЖЕ
НЕ СПАСЕТ,
\textbf{Реестр \emph{олигархов} не решит важнейших проблем нашей экономики},
Александр Гончаров, strana.ua, 04.06.2021

Иначе и не могло быть. Законопроект об \emph{олигархах} – это новая дымовая
завеса от реальных и трудноразрешимых социально-экономических проблем. А
расплата за такие бестолковые действия – падение ВВП Украины, отсутствие прямых
иностранных инвестиций, сокращение доходов населения и дальнейший рост
безработицы. А если бы «Слуги народа» могли профессионально посмотреть на
структурные проблемы в государстве и экономике, которые они сами же и создают,
то недоинвестированность – это ключевая проблема,
\textbf{Реестр \emph{олигархов} не решит важнейших проблем нашей экономики},
Александр Гончаров, strana.ua, 04.06.2021

У президента нет полномочий выносить на референдум закон об \emph{олигархах},
Андрей Портнов, strana.ua, 06.05.2021

Ух, как я рад. Несказанно рад. Я мечтал об этом что называется с младых ногтей.
В моей стране, наконец, не будет \emph{олигархов}! Не будет дюжины скверных, жадных,
злых людей, порочащих высокое звание украинского гражданина. И будет один,
самый главный \emph{Олигарх}, добрый всевидящий и бесконечно щедрый. Законный,
избранный народом \emph{Олигарх}.  Но до этого, до утверждения соответствующего
закона, я уверен, что офис моего Президента опубликует для всеобщего обозрения
список активно работающих в моей стране государственных структур, общественных
организаций и фондов, целенаправленно и ежедневно борющихся с поедающей наш
бюджет и нашу государственность гидрой Коррупции, порожденной и питаемой теми
самими \emph{олигархами},
\textbf{Хочется увидеть полный список расходов на антикоррупционные органы},
Семен Глузман, strana.ua, 06.05.2021

