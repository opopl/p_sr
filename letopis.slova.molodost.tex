% vim: keymap=russian-jcukenwin
%%beginhead 
 
%%file slova.molodost
%%parent slova
 
%%url 
 
%%author 
%%author_id 
%%author_url 
 
%%tags 
%%title 
 
%%endhead 
\chapter{Молодость}
\label{sec:slova.molodost}

%%%cit
%%%cit_head
%%%cit_pic
%%%cit_text
С \emph{Днем молодежи} - неважно, празднуют ли его официально!  З \emph{Днем молоді}!  Не
перевірила і навіть не хочу перевіряти, чи є заоаз це свято в Україні!  Усе
наше покоління святкує!  Бажаю усім, друзі, чим більше часу, залишатися
\emph{молодими}!  І питання зовсім не у віці.  Душа, серце - це головні показники!
Чим добріші, відкритіші, тим довше люди залишаються \emph{молодими}!  Життєвого
оптимізму усім!  Світла, радості, любові усім!
%%%cit_comment
%%%cit_title
\citTitle{Чем люди добрее, чем более они открыты, тем дольше они остаются молодыми / Лента соцсетей / Страна}, 
Марина Ставнийчук, strana.ua, 27.06.2021
%%%endcit

%%%cit
%%%cit_head
%%%cit_pic
\ifcmt
  pic https://strana.ua/img/forall/u/0/36/2021-07-03_14h46_47.png
	width 0.4
\fi
%%%cit_text
Далее нам снова встречаются \emph{молодые люди}, которые в жизни говорят на
украинском языке, но понимают тех, кто общается на русском. 
\enquote{Мы на украинском языке говорим, а если [футболистам] на русском языке удобно -
пусть по-русски говорят. Никого не надо заставлять}, - говорит один парень. 
\enquote{Мне, конечно, на украинском языке, привычнее}, - добавляет его товарищ
%%%cit_comment
%%%cit_title
\citTitle{Что говорят украинцы о пресс-конференциях футболистов на русском языке. Опрос Страны}, 
Антонина Белоглазова, strana.ua, 03.07.2021
%%%endcit

