% vim: keymap=russian-jcukenwin
%%beginhead 
 
%%file 13_01_2022.stz.news.ua.den.1.insha_rossia
%%parent 13_01_2022
 
%%url https://day.kyiv.ua/uk/blog/polityka/chy-mozhlyva-insha-rosiya
 
%%author_id losєv_іgor
%%date 
 
%%tags rossia,demokratia,putin_vladimir
%%title Чи можлива «інша» Росія?
 
%%endhead 
\subsection{Чи можлива «інша» Росія?}
\label{sec:13_01_2022.stz.news.ua.den.1.insha_rossia}

\Purl{https://day.kyiv.ua/uk/blog/polityka/chy-mozhlyva-insha-rosiya}
\ifcmt
 author_begin
   author_id losєv_іgor
 author_end
\fi

\begin{zznagolos}
Російська демократія в її демократизмі ніколи не доходить до антиколоніального пафосу	
\end{zznagolos}

Справді, чи можлива «інша» Росія? Це питання є назвичайно важливим для України.
Чи можлива Росія демократична, цивілізована, миролюбна? Це залежить не тільки
від соціального устрою, але також від традицій, що формують менталітет нації.
Протягом століть цей російський менталітет формувався експансією, загарбанням
нових територій, поневоленням інших народів. Росіяни звикли такі явища вважати
нормальними та природними. Розпалися європейські колоніальні імперії, європейці
переживають каяття за свою минулу колоніальну політику, а в Росії в цьому сенсі
не змінилося нічого.

Нині не відчувається ніяких наслідків навіть більшовицької антиколоніальної
агітації 20-30 років ХХ ст., спрямованої проти царату. Імперська свідомість
(незалежно від соціально-політичної організації Росії) цю агітацію проковтнула
і виплюнула. Перемогла «етика» людожерів: «Коли мене хтось спіймав і з’їв — це
зло, а коли я когось спіймав і з’їв — це добро». Жодних слідів у нинішній РФ не
збереглося від єльцинських спроб демократизації. Що стосується відходу від
колонізаторської політики Росії, то він у ті часи був дуже непослідовним і
нерішучим. Уже тоді в 1991 році та пізніше в Москві намагалися відірвати шматки
російськомовних регіонів від відновлених і нових незалежних держав.

Ще тоді Росія розпочала боротьбу за Крим, Донбас та інші території так званої
Новоросії. Отже, Путін мав попередників серед російських демократів і
лібералів. У нас кажуть, що російська демократія закінчується на українському
питанні. Але вона такою ж мірою закінчується на грузинському питанні, на
молдовському питанні, на білоруському питанні, на балтійському питанні, на
вірменському і т.д. Російська демократія в її демократизмі ніколи не доходить
до антиколоніального пафосу. Її адепти максимально здатні на мрії лише про
«демократичну (варіант: ліберальну) імперію», котру можна було б тримати без
особливих звірств. Представником такої поширеної в російському лібералізмі
течії, наприклад, є відомий високопосадовий чиновник часів Єльцина і раннього
Путіна Анатолій Чубайс.

Віталій Портников, який багато років працював як журналіст у Москві, згадував,
що коли він після проголошення незалежності України приїхав у російську столицю
і відвідав там Верховну Раду (Державної Думи тоді ще не було) РРСФР, то
депутати бігали по парламенту і вимагали негайно відправити танки на Київ. Вони
хотіли це зробити не кваплячись, бо потім буде запізно... Ось така демократія,
ось така «інша» Росія...

Нині у нас є мода покладати якісь надії на лідера того, що залишилось від
російської опозиції, на Олексія Навального. Але ось яку характеристику йому дав
багаторазовий депутат державної думи РФ, а тепер політичний емігрант, який живе
в Україні, Ілля Пономарьов: «За світоглядом Навальний не відрізняється від
Путіна. Навальний — людина імперських переконань».

Такого ж погляду дотримується колишній глава ЦВК України Андрій Магера: «У нас
є стара пропагандистська парадигма: є поганий Путін і є добрий російський
народ. Це далеко не так. Путін — це колективний російський народ». Магера має
рацію, коли каже, що абсолютну більшість росіян влаштовує загарбання Криму,
події на Донбасі, постійний воєнний шантаж України і т.д.

Коли йдеться про всі злочини російських режимів, то треба пам’ятати, що
співучасником усіх цих дій був (пасивно чи активно) майже весь російський
народ. Ще один опозиціонер із Росії Андрій Іларіонов постійно каже, що в цій
війні Україна має розглядати російський народ як свого союзника. На превеликий
жаль, російський народ сьогодні не може бути не тільки союзником України, але
навіть союзником антипутінських сил у самій Росії. Путін уособлює багато того,
що дуже симпатично російському народові: ненависть до Заходу, зневагу до
демократії, люту ненависть до колишніх поневолених народів Росії, що  обрали
шлях незалежного розвитку і т.д. І водночас любов до найтемніших куточків
російської історії: до часів Івана Грозного, до деспотизму Петра I, до
репресивної монархії Ніколая I та інших аналогічних моментів минулого Росії.

Нічого не варті надії значного числа політологів України, що після відходу
Путіна від влади в РФ щось зміниться на краще. Не зміниться той народ, який
мріяв про Путіна і про добу відновлення СРСР, Російської імперії, про наступ на
Захід, про відновлення своєї особливої привілейованої ролі на всьому
євразійському просторі. Не тільки Путін сприймав розпад СРСР як «найбільшу
геополітичну катастрофу ХХ ст.»... «Какую страну мы потеряли!» — це не тільки
Путін. І все те, що відбувається сьогодні, це не тільки його особистий
реванш...  

Газета: №1-2, (2022)
