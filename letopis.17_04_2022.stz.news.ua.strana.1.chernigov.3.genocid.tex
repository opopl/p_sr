% vim: keymap=russian-jcukenwin
%%beginhead 
 
%%file 17_04_2022.stz.news.ua.strana.1.chernigov.3.genocid
%%parent 17_04_2022.stz.news.ua.strana.1.chernigov
 
%%url 
 
%%author_id 
%%date 
 
%%tags 
%%title 
 
%%endhead 

\subsubsection{\enquote{Это геноцид какой-то}}

Чернигов с первого дня войны находился под постоянными обстрелами со стороны
России. При этом все объекты инфраструктуры, которые пострадали от бомбежек —
гражданские. Черниговская ТЭЦ, школы, детские садики, отели, больницы, жилые
дома — вот далеко не полный перечень разрушений в городе. По оценке мэра,
уничтожено до 70\% застройки Чернигова. На полное восстановление инфраструктуры
уйдет минимум четыре года. 

Когда 23 марта российская армия разрушила мост через реку Десна, Чернигов
оказался окружен вражескими войсками. 

— По сути, мы стали островом, — описывает тот период войны Александр Ломако,
секретарь Черниговского горсовета.

После этого враг прицельно бил по складам с продуктами и заправкам с топливом,
чтобы в город не могли доставить помощь. Российские войска искусственно
доводили Чернигов до гуманитарной катастрофы. Три недели в Чернигове не было
света и воды, практически не было связи. Отопление не работало, а в это время
как раз ударили морозы. 

За стойкое сопротивление президент Украины Владимир Зеленский присвоил
Чернигову звание города-героя. Несмотря на гуманитарную катастрофу и постоянные
бомбардировки, в Чернигове все это время оставалось около 100 тысяч человек.
Сотрудники коммунальных служб под обстрелами чинили поврежденные электросети,
волонтеры развозили людям воду и еду. 

Многие из них получили ранения, а некоторые — погибли, пытаясь помочь другим.
За время бомбардировок умерло около 700 мирных жителей. Это только те жертвы,
которые уже удалось установить. Сколько точно людей погибло в Чернигове,
возможно, не удастся выяснить никогда. Многие числятся пропавшими без вести —
вероятно, эти люди заживо сгорели в местах применения тяжелых авиабомб.

— Чернигов — город-герой. Но в первую очередь, Чернигов — это город героев, —
говорит Александр Ломако. — Наша базовая больница была обстреляна, там не было
ни одного целого окна, не было света и воды. Но наши врачи под обстрелами
продолжали лечить людей. Бывали дни, когда привозили по 100 раненых. Волонтеры
продолжали печь хлеб, развозили его людям. Пожарные, полиция, водители, местная
власть, которая осталась на своем рабочем месте — все делали большое дело,
чтобы город выжил в эти 38 тяжелых дней окружения. 

\ii{17_04_2022.stz.news.ua.strana.1.chernigov.pic.5}

Черниговский футбольный стадион имени Юрия Гагарина российские войска
разбомбили ночью 11 марта. На этом стадионе базировался футбольный клуб
\enquote{Десна}. Враг с самолета сбросил сюда минимум 5 авиабомб. 

Одна из бомб оставила после себя огромную десятиметровую воронку посреди
футбольного поля. В сетке на воротах застряли куски бетона. Трибуны, на которых
еще недавно собирались болельщики, превратились в сплошное месиво из досок и
огрызков пластиковых сидений. 

Провалившиеся трибуны с жалостью рассматривает мужчина в ярко-оранжевом жилете
с эмблемой \enquote{Десна}. Он проработал на этом стадионе 21 год. 

— Я тут и за охранника был, и за сварщика, и за дворника. До последнего
оставался на дежурном посту. Но когда начали бомбить Чернигов, сотрудников
отсюда сняли, потому что опасно было. Когда обстреляли стадион, меня здесь не
было. А как русские ушли, мне позвонили и попросили охранять стадион от
мародеров. Так что я вернулся, — говорит Михаил.

Глядя на то, что осталось от стадиона, который видел столько спортивных
достижений, столько выкриков футбольных фанатов, столько труда, слез и пота
тренирующихся здесь спортсменов, мужчине трудно подобрать слова. 

— Я даже не знаю, как их назвать. Нелюди! Вот подходящее слово. Только нелюди
такое могут сделать.. Это не военный объект, не стратегический. Тут дети
занимались спортом, радовались жизни. Футбол, бокс, тяжелая атлетика… Брали с
шести лет, выращивали футболистов. Детская спортивная школа тут была на высоте.
У нас базировалась Сборная Украины по тяжелой атлетике. Я со всеми дружил, меня
все уважали. Я на работу ходил, как на праздник. А это все… Очень тяжело.
Тяжело понять, зачем это было сделано. Слов нет, — говорит Михаил, потирая
глаза в попытке сдержать слезы. 

\ii{17_04_2022.stz.news.ua.strana.1.chernigov.pic.6}

— Единственный вопрос, который меня сейчас гложет — что делать деткам, —
говорит Владимир Чуланов, директор детско-юношеского ФК \enquote{Десна}.

Мы общаемся под звуки дрели и ремонтных работ — рабочие заделывают дыры в стене
небольшого двухэтажного здания, где располагались кабинеты тренеров. Рядом —
большой синий автобус с надписью \enquote{Десна}, который раньше возил футболистов на
игры. Лобовое стекло треснуло, окна выбило взрывной волной. Желтые чехлы на
спинках сидений издали похожи на силуэты людей в балаклавах — будто футболисты
после очередного матча оглядываются назад, вспоминая свои победы и поражения на
поле. Теперь футбольное поле превратилось в поле боя. 

— Это геноцид какой-то. Сбрасывать сюда бомбы — это уму непостижимо. Вы видели
эти воронки? Это катастрофа. Сколько поколений жили, о чем-то мечтали, а они
просто забрали все это, — говорит Чуланов.

Прямо на его глазах с российского самолета сбросили бомбы на жилые дома в
центре Чернигова.

— Я живу за 100 метров оттуда. Самолет пролетел у меня над головой и через
секунду сбросил эти бомбы на улице Черновола. Бомб было пять или шесть, попали
в жилые дома, аптеку, больницу… Спасибо нашим ВСУ — они спасли наши жизни и
жизни наших детей. Потому что если бы русские сюда зашли, в Чернигове была бы
Буча №2. Для них человеческая жизнь ничего не стоит.

От бомбежки пострадала и детская библиотека в доме Музея украинских древностей
Василия Тарновского, примыкающая к стадиону. Эта постройка конца XIX века —
памятник истории местного значения, — устояла под обстрелами большевиков в 1918
и 1919 годах и во Вторую мировую войну под бомбами немецких нацистов. Но
российское вторжение Музей не пережил.

\ii{17_04_2022.stz.news.ua.strana.1.chernigov.pic.7}

— Вот посмотрите, как \enquote{вторая армия мира} воюет с книжками, — комментирует
сотрудник стадиона. На футбольном поле вместо мячей валяются обгоревшие тома
школьных учебников, атласов и энциклопедий — их отбросило взрывной волной после
удара по библиотеке. 

Многие здания устояли на перекошенных после обстрела шкафах. Можно разобрать
надписи на корешках книг: \enquote{Этика}, \enquote{Эстетика}, \enquote{Народоведение}, \enquote{Логика},
\enquote{История}... Деревянные балки торчат из дыры в потолке, упираясь в чей-то
резиновый тапок. Разбитая чашка, смятая пачка чая, на ошметках стола —
картотека с именами школьников, которые брали тут книги.

— Я этот музей и библиотеку знаю, я же в Чернигове родилась. Как жалко..., —
говорит пенсионерка Ольга. 

Она присела отдохнуть на автобусной остановке возле входа в музей. Только
лавочка, собственно, от остановки и осталась — все стекла выбило при бомбежке.
Поправив белую шапку и застегнув под горло темно-синий пуховик — холодно, —
Ольга принимается рассказывать, как ей жилось под обстрелами. 

— И стреляют, и стреляют, ночами не спишь. Мне было тяжело очень, я осталась
одна в квартире. В центре живу. Страшно... Один раз в подвал пошла, подумала, там
хоть какое-то общество. Как раз вышла из дома — и слышу, летит что-то. Оно там
дальше упало, но меня оглушило. Сказали, что у меня легкая контузия.
Представляете? Теперь плохо слышу, — рассказывает пенсионерка. 

Местные целый месяц провели в домах или подвалах, отрезанные от связи с внешним
миром. Многие жители Чернигова до сих пор не знают, живы ли их знакомые и
родные в других районах города. 

— У меня телефона считай нет — он старенький и не работает, так что я без
связи. Только на свои ножки надеюсь. Хочу вот сейчас проведать знакомую. Иду к
ней пешком. Когда меня оглушило, сказали, что это в девятиэтажку попало, а там
подруга моя живет. У меня уже душа болит. Думаю, надо пойти, узнать, жива она
или нет, — говорит Ольга.

В Чернигове больше месяца не было ни воды, ни света. Кое-где у жителей
оставался газ. Из-за осады практически невозможно было завезти в город продукты
питания. Соседи и просто случайные люди делились друг с другом едой.
Пенсионерка говорит, что приходилось экономить каждый кусочек. А питьевой воды
и вовсе было не найти. 

— Воду мы ходили набирать в реке Стрижень, кипятили ее. Я в основном сидела на
чае, ничего не готовила. А за хлебом такие очереди были большие... Я только взяла
пакет муки, а дома коржики на воде делала и с ними чай пила, — вспоминает
Ольга. 

Даже в магазин выйти было рискованно: в середине марта российские войска
обстреляли людей в очереди за хлебом. Тогда погибло 14 местных жителей. 

— Под такими обстрелами и за продуктами не пойдешь. Я раз вышла за хлебом, так
меня полицейский на улице остановил с автоматом. Спрашивает — куда идете? Да
вот кушать захотелось, говорю. \enquote{Так тут все магазины закрыты. Постойте тут,
подождите}. И вынес мне пакет с продуктами — все, что смог, говорит. Я аж
плакала. А уж как я эту еду потом экономила, — рассказывает пенсионерка, и от
этих воспоминаний о доброте незнакомца впервые за весь разговор по щекам у нее
текут слезы.

Это уже вторая война, которую приходится переживать этой женщине. 

— Я же и ту войну прошла (Вторую Мировую — Ред.). Я родилась, мне четыре годика
было, мы под бомбежку попали. А тут уже пора отходить от этого мира — и опять
война... Я даже не могла представить, что такое будет. Была нормальная жизнь.
Живешь себе, радуешься понемножку. А теперь думаю: есть у каждого человека
судьба. У меня — вот такая судьба. Так нечего грешить. Слава богу, выдержала.
Слава богу, не впустили русских в город, уже перестали стрелять — радуюсь. И
живу дальше.
