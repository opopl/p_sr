% vim: keymap=russian-jcukenwin
%%beginhead 
 
%%file 13_06_2018.fb.lesev_igor.1.singapur_vstrecha.cmt
%%parent 13_06_2018.fb.lesev_igor.1.singapur_vstrecha
 
%%url 
 
%%author_id 
%%date 
 
%%tags 
%%title 
 
%%endhead 
\subsubsection{Коментарі}
\label{sec:13_06_2018.fb.lesev_igor.1.singapur_vstrecha.cmt}

\begin{itemize} % {
\iusr{Евгений Отовчиц}
По поводу второго символа - верно. Вооружайтесь. А если будете вооружаться в магазине под вывеской "США", то вообще молодцы.
Американцам срочно нужно продавать оружие. Такая экономическая модель.

\iusr{Игорь Лесев}
все так, только американским оружием весь мир не напичкаешь, но ВПК планеты явно на подъеме

\iusr{Михаил Подоляк}

))) блестящее описание всего этого странного спектакля)... тебе надо бы к Ыну на полставки напрашиваться - вполне можно из парняга большого героя сделать)))

\begin{itemize} % {
\iusr{Игорь Лесев}
да, и домик в Чхонджине с двухразовым питанием в качестве подъемных

\iusr{Михаил Подоляк}

\textbf{Игорь Лесев} ))) но зато будет чистый безбензиновый воздух, отличная травяная диета, круглосуточная охрана и неспешные диалоги о собаках и американцах)))

\iusr{Игорь Лесев}

а если еще и порвать демонстративно свой паспорт, то +15 лет чудесного
северокорейского товарищества... ты умеешь делать заманчивые предлжения)) Кста,
ты ж там был, тебе и вторую полставки забирать))

\iusr{Михаил Подоляк}

\textbf{Игорь Лесев} ))) не/не, я вкусил всех экзотических прелестей... мне достаточно... и мой разрез глаз им не подходит))) и имя им больше Игорь нравится - он почти чучхеанское)))

\iusr{Игорь Лесев}
мой, значит, разрез подходит... интересное наблюдение

\iusr{Михаил Подоляк}

\textbf{Игорь Лесев} ))) ну да)... когда ты чуть винишка выпьешь корейского и корейских же ужастиков насмотришься за всю ночь)))... свой в доску для чучхеалов)))

\iusr{Игорь Лесев}
а потом еще завести разговоры о корейском мире и присоединении Хоккайдо... так можно будет и трехразовое питание выбить вместе с велосипедом
\end{itemize} % }

\iusr{Владислав Клочков}

в добродетель Америки не верю только вместе с тем - потомственный чучхист
подпишет пускай даже декларацию не обижать птичек-рыбок-мушек, только когда он
завербован этой самой Америкой. и то если давно. во время учебы в какой-нибудь
Швейцарии

\begin{itemize} % {
\iusr{Игорь Лесев}

добродетели в международных отношениях в принципе не существует, есть только
интересы и дураки. А истории с вербовками оставьте для украинских
политологов-экспертов, они на 5 канале в три предложения весь хитрый план
раскроют

\iusr{Владислав Клочков}

ну вот, я уже при3,14zdyachen к сми, стоит только вербовку помянуть... вместо
разговоров о каналах предложил бы подумать о том, как идеология
переделывает-переодевает людей, а если уж поколения... все равно что говорить с
рядовыми - только именно с рядовыми - адептами белого братства, путина, бандеры
или сталина. так что не о добродетелях тут, а об их противоположностях, о клише
государственных культов. как там начинать говорить с ким-ир-сыном,
кым-ир-внуком и прочими ким-ир-дальше? "Чувак, если ты способен на человеческую
речь, оставь чучхе-клише для твоего населения?" помню-помню, что все зло от
Вашингтона, а все-таки?


\iusr{Игорь Лесев}

не совсем улавливаю вашу мысль, но скажем так, чучхе, и вообще, вся система
власти в КНДР возможна только при полной изоляции страны. И Ким, который как
раз и поездил по миру, это понимает лучше нашего. Любое послабление и
либерализаци режима - это его демонтаж, а вместе с ним и конец самого Кима.
Поэтому, смею предположить, малолтений толстяк, у которого еще вся жизнь
впереди с неограниченной властью на подконтрольной территории, на это не пойдет

\iusr{Владислав Клочков}

ну вот, а говорите, не улавливаете... только у него уже нет выбора -
конвергенция с врагом, распад национального мифа, вхождение в состав Ю Кореи.
если повезет - эвакуация (коль скоро мы вспоминаем Хоннекера). плохо слушает
доброго доктора Трампампама - не пришьют ему новые ножки, нечем будет на№;\%
бежать по доорожке. те, кого при виде южнокорейских магазинов не одолеет
инфаркт (а такие выезды участились, одна олимпиада чего стоит), возглавят некий
слегка народный фронт... в общем вылитые 80-е в сесесере....

\end{itemize} % }

\iusr{Дмитрий Коломийченко}

Прямо роль личности в истории. По моему дилетантскому мнению, США при Обаме
исчерпали возможности дорогой гегемонии и пытаются перейти к дешевой (плата за
безопасность растёт). Сама плата имеет две формы: 1) больше в бюджет НАТО, 2)
пересмотр торговых отношений в пользу реального сектора США. Активизация старых
кризисов призвана компенсировать отрицательную для США инерцию возникшую на
фоне ближневосточной неудачи. Наезд на КНДР надо было чем-то закончить. Бомбить
слишком опасно, поэтому пришлось подписать какую-то бумагу для сохранения лица.
Государственный эгоизм довольно привычная для США позиция. От неё просто
отвыкли за годы глобализма. Странно, ожидать другого от такого интровертного
государства в нынешних обстоятельствах.

\iusr{Михайло Бойченко}

Игра на бирже не заменит мировую политику @igg{fbicon.wink}  жди ответа
ястребов

\begin{itemize} % {
\iusr{Игорь Лесев}
они мне лично ответят?)

\iusr{Михайло Бойченко}
Тьі еще недостаточно капитализирован @igg{fbicon.face.smiling.eyes.smiling}  еще нужно на биржевиков поработать @igg{fbicon.face.smiling.sunglasses} 
\end{itemize} % }

\iusr{Игорь Лесев}

Вдогонку к тексту. Чувак реально ломает геополитическую конструкцию в мире

\href{https://korrespondent.net/world/3980280-ssha-hotovy-vyvesty-voiska-s-koreiskoho-poluostrova}{%
США готовы вывести войска с Корейского полуострова, korrespondent.net, 14.06.2018%
}


\end{itemize} % }
