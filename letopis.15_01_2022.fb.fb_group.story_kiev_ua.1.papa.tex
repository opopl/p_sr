% vim: keymap=russian-jcukenwin
%%beginhead 
 
%%file 15_01_2022.fb.fb_group.story_kiev_ua.1.papa
%%parent 15_01_2022
 
%%url https://www.facebook.com/groups/story.kiev.ua/posts/1840817919448334
 
%%author_id fb_group.story_kiev_ua,gorodeckaja_lina
%%date 
 
%%tags kiev,pamjat,semja
%%title Просто вспомнился папа...
 
%%endhead 
 
\subsection{Просто вспомнился папа...}
\label{sec:15_01_2022.fb.fb_group.story_kiev_ua.1.papa}
 
\Purl{https://www.facebook.com/groups/story.kiev.ua/posts/1840817919448334}
\ifcmt
 author_begin
   author_id fb_group.story_kiev_ua,gorodeckaja_lina
 author_end
\fi

Просто вспомнился папа. И киевский дом,
И высокое солнце, которое в детстве
Безгранично сверкало. С годами потом
Замечаешь, что тучи живут по соседству.
Просто вспомнилась кухня и вид из окна
На днепровские кручи и Мост Пешеходный,
По судьбе в легком платье шагала весна.
Позже, к осени стали дороги просторны...
Просто вспомнилась мама на кухоньке той,
Удивительно, как она все успевала...
За окошком – деревья с резною листвой,
И стоят катера у речного вокзала.
Просто вспомнилась горка и фуникулер,
Две копейки проезд, и под небом беззвездным,
Пряный вкус чьих-то губ, ни о чем разговор,
Немудренное позже окажется сложным.
Просто вспомнился дом. На шестом этаже
Твоя крепость, уют, размышленья смешные,
И яичницу мама готовит уже,
И все живы пока, и еще молодые...
Просто вспомнился старый, в ромашках, балкон,
И обрывки мелодий царапают сердце,
Над январским рассветом тюльпановый звон
Перекличкою снов возвращает из детства.
Просто вспомнился папа...

© L.G.

***

Строчки в столбик, светлые, чуть грустные, но все же больше светлые...Смотрю на
это старое юное фото, и думаю, какой же крошечный был наш киевский балкон на
шестом этаже подольской девятиэтажки...  

Но мне он казался самым красивым. А уж пейзаж, который открывался оттуда... На
Пешеходный мост над Днепром, на Владимирскую горку и фуникулер, вагончики
которого я могла сосчитать, когда листья опадали, на сирень в мае, и
заснеженный двор в январе... Все было.))

Не знаю, какая в Киеве сегодня погода. У нас настоящая наша зима - идет дождь.
))

А стихи эти написались в канун дня рождения папы. 15 января. Мой  папа – Яков
Городецкий, Яшенька,  Яша – киевская безотцовщина  20-х годов, никогда не
знавший ласки отца, так как родился после его трагической гибели. Бабушка,
молодая и красивая  женщина,  так и не вышла замуж, сохранив верность  мужу, с
которым почти не была вместе...

О папе моем, наверное, надо бы однажды рассказать отдельно.

Да и стихотворение это, не о нем... Вернее, не только о нем.

Вашему вниманию, друзья.

Вечер добрый!
