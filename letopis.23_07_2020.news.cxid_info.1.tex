% vim: keymap=russian-jcukenwin
%%beginhead 
 
%%file 23_07_2020.news.cxid_info.1
%%parent 23_07_2020
 
%%endhead 
\subsection{С 27 июля на Донбассе должны прекратиться все обстрелы, работа снайперов, дронов и ДРГ}
\url{https://cxid.info/151893_s-iyulya-na-donbasse-dolzhny-prekratitsya-vse-obstrely-rabota-snaiperov-dronov-i-drg.html?fbclid=IwAR3_mu9raK4TzMWtfmXpFuvyhQNBCdDIWRBsZZSue4fik2bqtIZ5BOhZcAQ}
  
\vspace{0.5cm}
{\small\LaTeX section: \verb|23_07_2020.news.cxid_info.1| project: \verb|letopis| rootid: \verb|p_saintrussia|}
\vspace{0.5cm}
  
В рамках Минских переговоров Трехсторонняя контактная группа согласовала
введение с 27 июля дополнительных мер по обеспечению режима прекращения огня и
контроля за его соблюдением.

Об этом заявил полномочный представитель Луганска на Минских переговорах
Владислав Дейнего.

«Руководством вооруженных сил Украины и руководством Народной Милиции ЛНР и ДНР
будут изданы приказы в обеспечение режима прекращения огня, действующего с 21
июля 2019 года, которые должны установить дополнительные меры по его
обеспечению. Эти приказы должны быть введены в действие с 27 июля и будут
действовать в течение всего периода до полного всеобъемлющего урегулирования
конфликта», — говорится в заявлении.

Дейнего сообщил, что участниками Контактной группы были согласованы следующие
меры:

— запрет на наступательные и разведывательно-диверсионные действия, а также
запрет на использование любых видов летательных аппаратов сторон;

— запрет на применение огня, включая снайперский;

— запрет на размещение тяжелого вооружения в населенных пунктах и их
окрестностях, в первую очередь на объектах гражданской инфраструктуры, включая
школы, детские сады, больницы и помещения, открытые для общественности;

— эффективное применение дисциплинарных мер за нарушение режима прекращения
огня и уведомление о них координатора Контактной группы, который уведомляет об
этом всех ее участников;

— создание и задействование координационного механизма по реагированию на
нарушения режима прекращения огня при содействии СЦКК в действующем составе;

— ответный огонь в случае наступательных действий допускается только в том
случае, если он открыт по приказу соответствующего руководства вооруженных сил
Украины и руководства Народной Милиции ЛНР и ДНР после безуспешной попытки
использовать вышеупомянутый координационный механизм. Об издании таких приказов
будет уведомлена Контактная группа;

— вышеупомянутые меры не могут быть полностью или частично дезавуированы
никакими другими приказами, в том числе секретными.

Дейнего подчеркнул, что «данные обязательства подписаны представителями всех
сторон Минских переговоров, что подтвердила координатор Контактной группы
госпожа (Хайди) Грау».

Напомним, что участники Контактной группы по урегулированию конфликта в
Донбассе с осени 2014 года более 20 раз заявляли о достижении соглашений по
соблюдению «режима тишины» в регионе
