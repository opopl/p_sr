% vim: keymap=russian-jcukenwin
%%beginhead 
 
%%file 24_11_2021.fb.andrienko_julia.doneck.1.front.cmt
%%parent 24_11_2021.fb.andrienko_julia.doneck.1.front
 
%%url 
 
%%author_id 
%%date 
 
%%tags 
%%title 
 
%%endhead 
\subsubsection{Коментарі}

\begin{itemize} % {
\iusr{Svetlana Zadirieva}
Юля, плачу!!!! А у нас нет защиты... совсем.... Горжусь Вами, выносливые люди!!

\iusr{Ирина Ивановна}

А я вспоминаю видео, снятое Игорем Ивановым и горы орехов и фруктов,
привезённых мне именно этим подразделением. Их благородство и честь не знают
границ. Мои!)))

\begin{itemize} % {
\iusr{Юлия Андриенко}
подумалось, как тяжело их терять. Таких, как там, больше не делают. Пусть хранят их все святые, какие есть.
\end{itemize} % }

\iusr{Natali Romanovsky}
\textbf{Юлия Андриенко}, вы - редкий неравнодушный человек! Я приеду - сразу к вам!

\begin{itemize} % {
\iusr{Юлия Андриенко}
\textbf{Natali Romanovsky} очень буду рада!

\iusr{Natali Romanovsky}
\textbf{Юлия Андриенко} всем ребятам на передке-низкий поклон!

\iusr{Юлия Андриенко}
\textbf{Natali Romanovsky} обязательно передам.
\end{itemize} % }

\iusr{Алена Донецкая}
Честь имеют. Это про большинство из них

\iusr{Снежана Аэндо}

\obeycr
Донецкая степь... Сильный ветер. Сейчас минус пять... Ночью-десять.
И это уже третью зиму...
Костер не разжечь под прицелом. Окопы, землянки промерзли...
Война бьет метелью нам в спину.
Мы... ладно, привыкли. Мужчины. А тут, совсем недалече, в разрушенном доме бабуля,
Идет ей девятый десяток.
Эй, те, кто жег в Киеве шины, чьи в сердце у Беркута пули,
Кто вылез с майданных палаток.
Вы адское пламя раздули.
Мы выстоим... И победим вас... Фашистскую тварь уничтожим.
У нас нет иных вариантов.
За бабушек наших и деток... Щитом быть живым наша доля...
У мира других нет гарантов...
Снежана Аэндо
2016
\restorecr

\begin{itemize} % {
\iusr{Ирина Ивановна}
\textbf{Снежана Аэндо} спасибо, Снежана. Особенно \enquote{За бабушек наших и деток... Щитом быть живым наша доля...
У мира других нет гарантов...} - святая правда(
\end{itemize} % }

\iusr{Лидия Тарасова}
Спасибо, пиши ещё, Юля.

\iusr{Мария Крюкова}
Юлечка, вы так пишете от сердца, что это просто ваша миссия, простите за высокий штиль.

\end{itemize} % }
