% vim: keymap=russian-jcukenwin
%%beginhead 
 
%%file 23_11_2020.news.ua.fakty.smetanskaja_olga.1.marchuk_rab_molbert
%%parent 23_11_2020
 
%%url https://fakty.ua/361541-ivan-marchuk-ya-rab-prikovannyj-k-molbertu
 
%%author Сметанская, Ольга
%%author_id smetanskaja_olga
%%author_url 
 
%%tags 
%%title Иван Марчук: «Я раб, прикованный к мольберту»
 
%%endhead 
 
\subsection{Иван Марчук: «Я раб, прикованный к мольберту»}
\label{sec:23_11_2020.news.ua.fakty.smetanskaja_olga.1.marchuk_rab_molbert}
\Purl{https://fakty.ua/361541-ivan-marchuk-ya-rab-prikovannyj-k-molbertu}
\ifcmt
	author_begin
   author_id smetanskaja_olga
	author_end
\fi
\index[names.rus]{Марчук, Иван!Художник}

\ifcmt
	 pic https://fakty.ua/photos2/article/36/15/361541w540zc0.jpg?v=233022
	 caption Иван Марчук в Испании
\fi

11:34 — 23 ноября 2020
Ольга СМЕТАНСКАЯ, «ФАКТЫ»

\begin{leftbar}
	\bfseries
Несмотря на пандемию, жизнь продолжается. В Мадриде, в Музее авангардного
искусства, который находится в самом центре столицы Испании, в помещении
бывшего железнодорожного вокзала девятнадцатого века, открылась
персональная выставка украинского художника, вошедшего в список гениев
современности, Ивана Марчука. Его творчество сегодня широко известно
во всем мире.
\end{leftbar}

О своей поездке в Испанию и о выставке в эксклюзивном интервью
он рассказал «ФАКТАМ».

\subsubsection{«У меня брали интервью самые известные в мире искусствоведы. Один из них работал с Пабло Пикассо»}

— Иван Степанович, от всей души поздравляю вас с приятным событием!

— Спасибо. Я был в Испании девять дней. Моя ретроспективная выставка,
организованная при поддержке посольства, охватывает работы от 1960-х годов
по сегодняшний день. На ней представлено 184 мои картины. Распаковав
работы, испанцы были потрясены. Выставка получила огромный резонанс.
Столько отзывов! Буквально на третий день с просьбой организовать у себя
мою выставку обратились Валенсия и Андорра. Испанцы поймали «золотую
рыбку» и теперь хотят нарадоваться (улыбается).

— Сколько же времени продлится выставка?

— Два с половиной месяца. Это редкость. Пусть люди смотрят! Кроме того,
сейчас идет активная работа над каталогом. Все это для моей репутации
новая ступень.

\ifcmt
pic https://fakty.ua/user_uploads_new/images/articles/2020/11/23/361541/%D0%BD%D0%B0%D1%82%D1%8E%D1%80%D0%BC_n.jpg
caption Натюрморты художника всегда необычны (фото из альбома Ивана Марчука)
\fi


— Столько приятного на фоне пандемии!

— Да. Посетители выставки смогут увидеть, как я говорю, 12 разных
Марчуков. И они удивятся, что один художник может создавать такие разные
картины. В творчестве я ненасытный, голодный, одержимый! Постоянно думаю
о том, что еще показать из того, чего в мире еще не было.

Читайте также: Иван Марчук: «Уже работаю над новым циклом картин — душа,
как писал Шевченко, «оживає і сміється знову»

— Как проходило открытие выставки в новых реалиях?

— Все были в масках. Количество посетителей регулировалось. У меня брали
интервью самые известные в мире искусствоведы. Один из них работал с Пабло
Пикассо. Эти искусствоведы дадут свои резюме для каталога.

\subsubsection{«Мне очень понравились испанские вино и сыры»}

— Испания изменилась в связи с пандемией?

— А я до пандемии в этой стране не был, поэтому не знаю. Все люди
в масках. Мне удалось побывать в нескольких музеях. Конечно, в музее
Прадо. Особенно интересно было посетить Толедо. Вспоминается картина Эль
Греко, где он нарисовал это удивительное место. Толедо — чудо архитектуры.
К слову, в путешествиях она меня всегда интересует больше, чем живопись.
Архитектектура — лицо городов.

\ifcmt
pic https://fakty.ua/user_uploads_new/images/articles/2020/11/23/361541/%D0%9C%D0%B0%D1%80%D1%87%D1%83%D0%BA%20%D0%BC%D0%B0%D1%81%D0%BA%D0%B8_n.jpg
caption На открытии выставки все были в масках (фото из альбома Ивана Марчука)
\fi


— Как вас принимали в Мадриде?

— Прекрасно. Я жил в гостеприимной семье украинцев.

— Знаменитую испанскую паэлью попробовали?

— Мне очень понравились вино и сыры. Я же стремился для друзей, у которых
жил, приготовить что-то свое. Делал салаты, варил суп с чечевицей, так как
они там первое не готовят.

— Что-то купили в Испании?

— Знаете, я никогда ниоткуда не привожу сувениры. Даже конфеты не покупаю
за границей. Сам их не ем и дарить не хочу, так как сладкое вредно.

— Поездка вас вдохновила на новые произведения?

— Я не завишу от вдохновения. Вдохновить меня могут только женщины
(улыбается). Каждое утро иду в мастерскую и знаю, что должен что-то
делать. Сейчас вот, к примеру, рисую сумерки: солнце прощается с днем, оно
оставляет нас. И мне хочется передать это настроение…

— Счастливый вы человек, Иван Степанович!

— Смотря что считать счастьем. Если так, как его обычно понимают люди,
то по-человечески считаю себя самым несчастным человеком — я раб,
прикованный к мольберту!

— И все же что-то вас, наверное, в жизни радует?

— Радуюсь, когда вижу красивую женщину, которая мне улыбается, когда
со мной кто-то здоровается на улице, и я говорю: «Я вас не знаю»,
а в ответ слышу: «Зато мы все вас знаем!» По пути домой из мастерской
несколько человек обязательно со мной здороваются. Это приятно!

\ifcmt
pic https://fakty.ua/user_uploads_new/images/articles/2020/11/23/361541/%D0%BF%D0%B5%D0%B9%D0%B7%D0%B0%D0%B6%20%D1%81%D0%BD%D0%B5%D0%B35_n.jpg
caption Пейзажами Ивана Марчука можно любоваться бесконечно (фото из альбома Ивана Марчука)
\fi


\subsubsection{«Ценю удобство. Считаю, что лучшие брюки — джинсы»}

— Наверное, на открытии выставки в Мадриде у вас все хотели взять
автограф?

— Да, раздал их немало. На следующий день после открытия была встреча
с диаспорой. Присутствовало много молодежи. И все просили автографы
и сфотографироваться на память.

Читайте также: Иван Марчук: «Больно за Украину. Почему мы на райской
земле живем, как в пекле?»

— Смотреть на ваши картины можно бесконечно…

— Я весь мир уже ими очаровал (улыбается). И мне хочется, чтобы все больше
людей в разных странах увидели мои произведения. Искусство должно
принадлежать человечеству.

— Пандемия закрыла границы, но ваше творчество их не знает. Недавно в США
на одном из сайтов демонстрировалась коллекция одежды дизайнера Алены
Серебровой, созданная по мотивам вашего творчества.

— Очень красивые платья! Я бы хотел, чтобы мои картины печатались на целых
полотняных гектарах. Их нужно увеличивать! Маленькие картины я рисовал
в период, когда был запрещен, чтобы их можно было переслать.

— А вы любите красиво одеваться?

— Я ценю удобство. Считаю, что лучшие брюки — джинсы. Они у меня и для
парадного выхода, и для работы.

— Что бы вы пожелали читателям «ФАКТОВ»?

— Мир прекрасен и безграничен! Планета Земля — это чудо. Любите ее!
Интересно все: каждая травинка, деревце, цветок… Получайте удовольствие
от всего этого! Будьте добрыми, счастливыми и радостными. Любите
искусство!

Читайте также: В КГБ от меня требовали покаяться «в ошибках» и «признать
вину», — художник Иван Марчук

Фото в заголовке из альбома Ивана Марчука


