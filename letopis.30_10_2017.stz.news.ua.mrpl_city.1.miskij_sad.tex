% vim: keymap=russian-jcukenwin
%%beginhead 
 
%%file 30_10_2017.stz.news.ua.mrpl_city.1.miskij_sad
%%parent 30_10_2017
 
%%url https://mrpl.city/blogs/view/miskij-sad
 
%%author_id demidko_olga.mariupol,news.ua.mrpl_city
%%date 
 
%%tags 
%%title Міський сад
 
%%endhead 
 
\subsection{Міський сад}
\label{sec:30_10_2017.stz.news.ua.mrpl_city.1.miskij_sad}
 
\Purl{https://mrpl.city/blogs/view/miskij-sad}
\ifcmt
 author_begin
   author_id demidko_olga.mariupol,news.ua.mrpl_city
 author_end
\fi

\vspace{0.5cm}
\begin{raggedright}
\em
Цей сад багато зустрічав людей,\par
Тому він загадковий і щасливий.\par
Він обіймає затишком алей,\par
І прихистить від чергової зливи...\par

Він пам'ятає мудрість поколінь,\par
І нею Маріуполь захищає,\par
Присядь, погомони з ним, відпочинь.\par
Він знає щось таке, що ти не знаєш.\par
\end{raggedright}
\vspace{0.5cm}

Є в Маріуполі місце, де можна гуляти годинами: читати улюблену книгу, кататися
на ще живих атракціонах або ж просто випити чашечку міцної кави й насолодитися
культурним заходом. Це місце полюбляють і діти, і дорослі, складно знайти серед
маріупольців людину, яка не знає, де розташований Міський сад, але далеко не
всім відома його історія... 

Міський сад Маріуполя – центральний міський парк культури й відпочинку міста –
було закладено у далекому 1863 році на високому пагорбі. Вибір припав на це
місце тому, що тут росли численні фруктові дерева. У своєму розвитку парк
пройшов дві основні стадії. Перша – стадія деревонасаждення, друга – створення
великої пейзажної паркової зони.

\ii{30_10_2017.stz.news.ua.mrpl_city.1.miskij_sad.pic.1}

Цікавим є той факт, що великий Князь Костянтин Миколайович, син імператора
Російської імперії Миколи I (1796–1855 рр.) власноруч посадив у Міському саду
два дерева, коли відвідав Маріуполь у 1872 році.

У 1889 році за клопотанням Міської думи відомим маріупольським садівником і
громадським діячем \textbf{\emph{Георгієм Георгійовичем Псалті}} (1864 – 1940) в саду було
проведене повне і радикальне перепланування. Псалті, отримавши спеціальну
освіту у вищій школі садівництва Версаля, надав саду всі характерні риси
регулярного французького парку: прямі доріжки, що розходяться променями від
центру з оглядовим майданчиком, геометрично правильної форми клумби, засаджені
квітами; скомпоновані групи дерев, що доповнюють одна одну.

\ii{30_10_2017.stz.news.ua.mrpl_city.1.miskij_sad.pic.2}

Улюбленим місцем відпочинку маріупольців була ротонда – висока дерев'яна
альтанка, досить містка. Влітку в ротонді проходили вистави трупи В.
Шаповалова. Зафіксовано навіть випадок, коли, приїжджаючи, трупа ставила тут
оперу.

\ii{30_10_2017.stz.news.ua.mrpl_city.1.miskij_sad.pic.3}

Сад багатий на 17 різних видів дерев, вік деяких перевищує 100 років. Це дуби
й липи, софори й клени, каштани та берези, горіхи й верби, сосни й тополі,
шовковиці та акації. А на клумбах центральної площі з ранньої весни до пізньої
осені геометрично рівним килимом рясніють троянди, айстри й петунії.

У 1910 році після влаштування водопровідної системи у парку було встановлено
власний фонтан.

\ii{30_10_2017.stz.news.ua.mrpl_city.1.miskij_sad.pic.4}
\ii{30_10_2017.stz.news.ua.mrpl_city.1.miskij_sad.pic.5}

Різні оголошення, розклеєні на вулицях, повідомляли, що Міський сад відкритий
для \enquote{відпочинку і різних розваг поважній маріупольській публіці}. У
різний час тут полюбляли гуляти художники А. І. Куїнджі та К.Ф. Богаєвський,
письменники А. С. Серафимович і А. С. Новиков-Прибой, композитор Д. Б.
Кабалевський, артисти Б. Ф. Андреєв і М. І. Бернес та багато інших.

\ii{30_10_2017.stz.news.ua.mrpl_city.1.miskij_sad.pic.6}
\ii{30_10_2017.stz.news.ua.mrpl_city.1.miskij_sad.pic.7}

Міський сад зберіг унікальну споруду літнього театру, зведену у 1934 році. Річ
у тому, що будівля Зимового театру Маріуполя занепала, а до нашого міста на
гастролі зібралася трупа Ленінградського Великого драматичного театру.
Маріупольці взяли на себе зобов'язання звести за 45 днів, до приїзду трупи,
будівлю літнього театру, пристосовану для повноцінного проведення театральних
постановок і концертів, що у них, незважаючи на безліч труднощів, все ж таки
вийшло. Літній театр гармонійно вписався в архітектурний ансамбль парку.

\ii{30_10_2017.stz.news.ua.mrpl_city.1.miskij_sad.pic.8}

Друга світова війна дивом не зачепила цей чарівний куточок міста, зберігши
сусідами сімнадцять порід дерев, багаторічні кущі та трави. На території
міського саду розташовані братські могили, декілька пам'ятників воїнам,
загиблим в роки Громадянської та Другої світової війни.

У 1950-х у парку з'явилася алея просвітників, що простягається від в'їзної арки
до центральної площі, вона прикрашена двома рядами бюстів видатних людей, які
зробили величезний внесок у світову науку і культуру – О. Пушкіна, Л. Толстого,
Д. Менделєєва, М. Чернишевського та багатьох інших. Авторами скульптур є
маріупольський скульптор Іван Савелійович Баранніков і московський скульптор
Георгій Дмитрович Лавров.

\ii{30_10_2017.stz.news.ua.mrpl_city.1.miskij_sad.pic.9}

Південна частина саду виходить на крутий схил, з якого відкривається панорамний
вид на узбережжі Азовського моря, торговий порт і залізничне депо. Також
постійну увагу маріупольців і гостей міста привертають атракціони, розташовані
в міському саду.

Навесні 2012 року в саду був облаштований куточок закоханих. Були встановлені
ковані лавочка і ліхтарний стовп. Але найбільшою популярністю користується
міст, на якому фотографуються маріупольські молодята. Закохані пари перейняли
традицію, наявну в багатьох містах світу, – в знак любові та вірності вони
залишають на містку закритий замок зі своїми іменами.

На головній сцені парку постійно проходять концертні та розважальні програми,
організовані силами творчих колективів міських палаців культури. Тут можна
побачити театральні видовища, почути класичну музику чи взяти участь у міських
фестивалях.

\ii{30_10_2017.stz.news.ua.mrpl_city.1.miskij_sad.pic.10}

Які б події не розгорталися в країні та в місті, яку б назву не носив парк
(Міський сад, парк ім. Жданова, Дитячий парк), протягом свого довгого існування
він був і залишається осередком відпочинку, культури та просвітництва.

