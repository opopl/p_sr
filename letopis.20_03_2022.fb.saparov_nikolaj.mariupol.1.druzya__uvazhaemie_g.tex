%%beginhead

%%file 20_03_2022.fb.saparov_nikolaj.mariupol.1.druzya__uvazhaemie_g
%%parent 20_03_2022

%%url https://www.facebook.com/permalink.php?story_fbid=pfbid05V8FURKgX4YmDL5A2pRwp7AZRPKj4kMFwAJFFpnrV22wPc5jWAt6Cgor4WqMoHx2l&id=100015424741499

%%author_id saparov_nikolaj.mariupol
%%date 20_03_2022

%%tags mariupol,mariupol.war
%%title Друзья. Уважаемые горожане.  Я, как, наверное, и многие из Вас задумался - а что дальше?

%%endhead

\subsection{Друзья. Уважаемые горожане.  Я, как, наверное, и многие из Вас задумался - а что дальше?}
\label{sec:20_03_2022.fb.saparov_nikolaj.mariupol.1.druzya__uvazhaemie_g}

\Purl{https://www.facebook.com/permalink.php?story_fbid=pfbid05V8FURKgX4YmDL5A2pRwp7AZRPKj4kMFwAJFFpnrV22wPc5jWAt6Cgor4WqMoHx2l&id=100015424741499}
\ifcmt
 author_begin
   author_id saparov_nikolaj.mariupol
 author_end
\fi

Друзья. Уважаемые горожане.

Я, как, наверное, и многие из Вас задумался - а что дальше? Мы лишились
родных, близких, жилья, имущества, той непередаваемой атмосферы родного
города, нашего Марика, которую чувствуют только местные. И всего этого теперь
нет.

Мясорубка пережевала, искромсала в кровь, кости и бетон с кирпичом наши
судьбы, наши семьи, нашу общину, нашу громаду.

Многие бежали не глядя от этого ада и ужаса. Побросав всё, что было дорого и
близко. У меня всё время перед глазами эти тысячи машин с людьми, дающими
по-газам, как только проезжали ряд противотанковых  мин после блокпоста в
сторону Мелекино. Куда угодно, только быстрее и подальше от бомбежек и
перестрелок. Без денег, еды, воды. Многие - в никуда. Только с самыми родными
людьми и животными.

Очень многие никогда не вернутся. По разным причинам. Ни кто не может их
осудить после того,что они пережили и потеряли.

А я подумал, а кому мы нужны за пределами нашего родного города? Кто о нас
позаботится? Жалость к переселенцам пройдёт и мы будем мыкаться, с
протянутой рукой просить снисхождения. Сироты без своего дома. Да, разрушения
велики. А на новом месте у нас будет вилла и пальма у дома?

Я не агитирую ни за кого и ни за что. Вообще, не хочу  слышать о политике.
Страны, флаги, президенты, премьеры уже более 200 лет меняются, а Город с
большой буквы - Мариуполь, остаётся.

Я люблю своих детей, семью, дом, город, страну, Землю. Именно в таком порядке. И
,я бы очень хотел сохранить, в значении город, название Мариуполь.  Но, без
жителей, город будет кладбищем. Кладбищем разбитых надежд и жизней ! Хватит ли
у нас сил вернуться? Интересно было бы узнать Ваше мнение. Поделитесь. Подумаем
вместе.

Обещал написать по своей основной работе, но как-то нет желания. И так, негатива
выше крыши. Знаю, только, что умерших очень много и они неупокоены.

Считаю основной своей миссией организовать сбор тел, их идентификацию и
достойное захоронение. Это моя работа и мой долг. Все сотрудники «Скорботы»,
кто думает также, после окончания боевых действий, должны постараться
вернуться и помочь горожанам совершить погребение своих погибших и умерших.
Кто, в силу разных причин, не сможет или не захочет вернуться в ИСТЕРЗАННЫЙ
ГОРОД, я пойму и приму ваше решение. Документы и расчёт вышлю на указанный
Вами адрес. Тех, кто вернётся и поможет буду уважать ещё больше, чем раньше.
Потерявшим жильё, постараемся помочь расселив в сохранившихся помещениях.

Написал, а потом задумался. Ведь еще ничего не закончилось. Бои и разрушения
продолжаются.

Но очень хочется верить!!! Верить, что город не умер, а только тяжело ранен.
