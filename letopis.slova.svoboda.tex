% vim: keymap=russian-jcukenwin
%%beginhead 
 
%%file slova.svoboda
%%parent slova
 
%%url 
 
%%author 
%%author_id 
%%author_url 
 
%%tags 
%%title 
 
%%endhead 
\chapter{Свобода}

%%%cit
%%%cit_pic
\ifcmt
  pic https://avatars.mds.yandex.net/get-zen_doc/1641049/pub_60be21f47b0ba72b6f672a58_60be2b04e14854002e7c596f/scale_1200
\fi
%%%cit_text
Посольство США в Украине через Twitter позволило футбольной ассоциации страны
разместить на форме националистический лозунг \enquote{СУГС} и контур страны с Крымом.
Мужики, я восхищён. Королева восхищена. Все мы тут рыдаем. Следующий шаг -
добиться разрешения каждому футболисту сделать себе татуировку на спине, ну или
где удобно, где Украина с Крымом.  Или даже с Крымом и с Кубанью.  И во время
матча вся команда - р-раз! - и эту татуировку показывает стадиону, задирая (или
приспуская) форму. И русские плачут от страха и зависти.  «Крым - их! - кричат
русские в ужасе. - Смотрите! Видите! Вон их Крым!» Мы никогда не научимся быть
столь же \emph{свободными}
%%%cit_title
\citTitle{Футбольная ассоциация Украины разместила конутр страны с Крымом}, 
Захар Прилепин, zen.yandex.ru, 07.06.2021
%%%endcit

%%%cit
%%%cit_pic
%%%cit_text
Ох, труден в осмыслении триединый и (ныне трехглавый) русский народ. Одни себя
в пограничники записали, окраины от москалей стерегут, другие никак не решат:
из каких именно краев себе на холку «русь» варяжскую призвали и что это вообще
значит. А государство со столицей в Минске задачку на два умножило. Никак не
выяснят, почему она — Белая. Русь которая... Схематично проблему так
сформулируем: земли современного государства Белоруссия еще в Средневековье
назывались — Белая Русь. Историки XIX считали (по сей день имеют
последователей), что название происходит от цвета волос и повседневной одежды
белорусского народа (холстина неокрашенная).  Более пафосно звучит утверждение:
название «Белая» подчеркивает \emph{древнюю свободу}, кристальную чистоту и
незамутненность, независимость данных земель. Они длительное время не знали
«ига татаро-монгольского», польско-литовского и московского. Как там сказано...
