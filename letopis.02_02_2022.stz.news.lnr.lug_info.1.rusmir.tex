% vim: keymap=russian-jcukenwin
%%beginhead 
 
%%file 02_02_2022.stz.news.lnr.lug_info.1.rusmir
%%parent 02_02_2022
 
%%url https://lug-info.com/news/lnr-vmeste-s-rossiej-vspominaet-vseh-kto-pogib-radi-budushego-russkogo-mira-pasechnik
 
%%author_id news.lnr.lug_info
%%date 
 
%%tags lnr,pasechnik_leonid,rossia,rusmir,stalingrad
%%title ЛНР вместе с Россией вспоминает всех, кто погиб ради будущего Русского мира – Пасечник
 
%%endhead 
 
\subsection{ЛНР вместе с Россией вспоминает всех, кто погиб ради будущего Русского мира – Пасечник}
\label{sec:02_02_2022.stz.news.lnr.lug_info.1.rusmir}
 
\Purl{https://lug-info.com/news/lnr-vmeste-s-rossiej-vspominaet-vseh-kto-pogib-radi-budushego-russkogo-mira-pasechnik}
\ifcmt
 author_begin
   author_id news.lnr.lug_info
 author_end
\fi

Луганская Народная Республика в 79-ю годовщину завершения Сталинградской битвы
вместе с Россией вспоминает всех, кто погиб ради будущего Русского мира. Об
этом заявил глава ЛНР Леонид Пасечник.

\enquote{79 лет назад – 2 февраля 1943 – завершилась Сталинградская битва. Одно из
важнейших генеральных сражений Великой Отечественной войны. Этот День воинской
славы мы отмечаем вместе с братской Россией, вспоминая всех, кто погиб ради
мирного будущего Русского мира}, - написал Пасечник на своей странице в сети
микроблогов Twitter.

\ii{02_02_2022.stz.news.lnr.lug_info.1.rusmir.pic.1}

Сталинградская битва — боевые действия советских войск по обороне Сталинграда и
разгрому крупной стратегической группировки в междуречье Дона и Волги в ходе
Великой Отечественной войны, после которых вермахт потерял стратегическую
инициативу. К этой исторической битве относятся события периода с 17 июля 1942
года по 2 февраля 1943 года. Наряду со сражением на Курской дуге стала
переломным моментом в ходе Второй мировой войны. Сражение включало в себя
попытку вермахта захватить правобережье Волги в районе Сталинграда (современный
Волгоград) и сам город, противостояние советских войск и вермахта в городе, и
контрнаступление советских войск, в результате которого германские силы внутри
и около города были окружены и частью уничтожены, а частью захвачены в плен.

По приблизительным подсчетам, суммарные потери обеих сторон в этом сражении
превышают два миллиона человек. Для Советского Союза победа в Сталинградской
битве положила начало массовому изгнанию захватчиков с советской земли, за
которым последовали освобождение оккупированных территорий Европы и Победа над
Третьим рейхом в 1945 году.

Одним из солдат Красной Армии, особо проявивших себя в ожесточенных боях под
Сталинградом, был житель Первомайска, гвардии старший лейтенант Петр Болото. За
образцовое выполнение боевых заданий командования на фронте борьбы с немецкими
захватчиками и проявленные при этом отвагу и геройство ему было присвоено
звание Героя Советского Союза.

2 февраля является одним из Дней воинской славы России - День разгрома
советскими войсками немецко-фашистских войск в Сталинградской битве в 1943
году. 
