% vim: keymap=russian-jcukenwin
%%beginhead 
 
%%file 10_12_2021.fb.fb_group.story_kiev_ua.2.samyj_samyj_podol
%%parent 10_12_2021
 
%%url https://www.facebook.com/groups/story.kiev.ua/posts/1816124798584313
 
%%author_id fb_group.story_kiev_ua,kuzmenko_petr
%%date 
 
%%tags foto,gorod,kiev,kievljane,podol
%%title Самый прекрасный, самый добрый, самый важный, самый незабываемый, и вообще самый - самый...
 
%%endhead 
 
\subsection{Самый прекрасный, самый добрый, самый важный, самый незабываемый, и вообще самый - самый...}
\label{sec:10_12_2021.fb.fb_group.story_kiev_ua.2.samyj_samyj_podol}
 
\Purl{https://www.facebook.com/groups/story.kiev.ua/posts/1816124798584313}
\ifcmt
 author_begin
   author_id fb_group.story_kiev_ua,kuzmenko_petr
 author_end
\fi

Очень долго сомневался, разместить ли это фото, со своей ремаркой. И
фотографией Подола, сделанной дроном, давно никого не удивить. И
ностальгическим нытьём, по мнению многих строгих критиков группы. 

Но в одной этой фотографии, для меня, огромный пласт жизни. Самый прекрасный,
самый добрый, самый важный, самый незабываемый, и вообще самый - самый... 

\ii{10_12_2021.fb.fb_group.story_kiev_ua.2.samyj_samyj_podol.pic.1}

Начало жизни и радужное детство в коммуналке и дворе красивого монументального
старого шестиэтажного дома у подножия Андреевского спуска, бывшего для нас
просто старой подольской улицей, с патриархальным укладом жизни. Прямо и чуть
справа внизу прекрасная Покровская церковь, творение легендарного
Григоровича-Барского, и усадьба Стрельбицкого, представлявшие довольно унылое
зрелище в наши школьные годы. Мы играли там на продлёночной прогулке и даже на
большой перемене, выбегая из старого здания нашей 100-й школы, расположенного
сразу за ними. Низ 12-го трамплина, о котором я уже писал, и сохранившаяся
колокольня церкви Николая Доброго, между старым и новым домами пионеров в
зданиях бывших 17-й и 20-й школ. В этих местах мы учились, шалили, дружили,
занимались в кружках, иногда хулиганили, потом влюблялись и взрослели. 

Хорошо виден на фото Гостиный двор и сквер перед ним. Он помнит и памятник Зое
Космодемьянской, и мозаичную чайку над фонтанчиком, и площадку строительства
подольской ветки метрополитена. Ныне он украшен современным фонтаном и
великолепным памятником гетману Сагайдачному. Гастроном и универмаг в начале
улицы Жданова. Сейчас там тоже продмаг с пиццерией и \enquote{Пузата хата}
напротив, знаменуют исток пешеходной улицы Сагайдачного. Колокольня бывшей
церкви святой Екатерины была частью завода пищевой автоматики, а здание
райкомов компартии и комсомола так и осталось административным. 

Левее легендарная Киево-Могилянская академия, тогда Морполит - моя Альма-матер,
превратившая подольского босяка в дисциплинированного флотского офицера.
\enquote{Пищевик} так и остался для нас \enquote{Пищевиком}, сколько ни называй
его \enquote{Славутичем} и Музыкальным театром, позже. Видно дальше и
\enquote{Эру}. Другой такой дом быта в Киеве стоило очень поискать в то время.
Рядом каланча легендарной подольской \enquote{пожарки}, нынешнего музея
Чернобыля, напоминающего о страшной катастрофе 1986-го. 

Далее бывшая 15-я подольская больница, где ушла из жизни мама... Потом
Волошская, начало нашей Почайнинской, с древнейшей Ильинской церковью,
набережная, Днепр и гавань с бывшим речпортом, за элеватором и первым канатным
мостом на Рыбальский полуостров. 

Дальше Оболонь и туманный левый берег. Я неплохо их знаю, но они так и остались
слегка туманными и в жизни. Так уж получилось, что ваш покорный слуга остался
подолянином в душе на всю жизнь. И в какой уголок земли или моря ни заносила бы
меня и мою семью судьба, мы всегда рвались и стремимся домой, в этот маленький
и очень огромный и важный уголочек нашей планеты, что на этом фото, сделанном
бездушным дроном и его душевным оператором. Спасибо ему!  Пространная и
несколько тоскливая получилась ремарка. Простите, друзья...

\ii{10_12_2021.fb.fb_group.story_kiev_ua.2.samyj_samyj_podol.cmt}
