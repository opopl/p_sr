% vim: keymap=russian-jcukenwin
%%beginhead 
 
%%file 11_03_2022.stz.news.ua.strana.1.kiev_v_osade_itogi.5.uslovia
%%parent 11_03_2022.stz.news.ua.strana.1.kiev_v_osade_itogi
 
%%url 
 
%%author_id 
%%date 
 
%%tags 
%%title 
 
%%endhead 

\subsubsection{6 условий Путина}

О переговорах между Украиной и Россией вчера речь не шла. Единственная новость
на эту тему, появившаяся вчера в украинских СМИ, касалась полного пакета
требований Кремля к руководству Украины.

\enquote{Зеркало недели} опубликовало перечень из шести пунктов:

\begin{itemize} % {
\item Отказ от движения в НАТО. Нейтральный статус Украины. Одним из гарантов которого готова стать Россия.
    
\item Русский язык – второй государственный. Отмена всех ограничивающих его законов.
    
\item Признание Украиной Крыма русским.
    
\item Признание Украиной независимости \enquote{ДНР} и \enquote{ЛНР} в административных границах областей (включая ныне контролируемые Украиной территории).
    
\item \enquote{Денацификация}. Запрет деятельности, как сказано, ультранационалистических,
нацистских и неонацистских партий и общественных организаций, отмена
действующих законов о \enquote{героизации нацистов и неонацистов}.

\item \enquote{Демилитаризация Украины}. Полный отказ от наступательного вооружения.
\end{itemize} % }

Правда, советник главы Офиса президента Михаил Подоляк эту информацию опроверг,
назвав \enquote{российской пропагандой}.

Однако, то, что такие требования выдвигает России выглядит вполне
правдоподобным (тем более, что большую их часть Москва озвучивала и публично).
Просто в данном случае они детально расшифрованы.

Понятно, что практически ни на один из этих пунктов Киев идти не готов. Кроме
первого пункта о нейтральном статусе, о согласии на который высказались уже
практически все представители украинской власти. А некоторые пункты (например,
по запрету националистических формирований) вообще выглядят для украинской
власти невыполнимыми сейчас даже при желании. Да и желания такого нет. Потому
что для принятия этих пунктов украинская власть должна чувствовать себя
проигравшей стороной, близкой к разгрому. А она этого не чувствует и наоборот
полагает, что у противника резервы на исходе.

Россия также не готова пока поступаться даже частью своих требований
рассчитывая додавить Зеленского через активизацию боев на фронте. В таких
условиях, к сожалению, мирные переговоры сейчас стопорятся. Обе стороны
готовятся к новым боям, а все большее число мирных жителей нашей страны
готовятся к гуманитарной катастрофе.


