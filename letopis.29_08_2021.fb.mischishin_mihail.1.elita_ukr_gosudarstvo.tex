% vim: keymap=russian-jcukenwin
%%beginhead 
 
%%file 29_08_2021.fb.mischishin_mihail.1.elita_ukr_gosudarstvo
%%parent 29_08_2021
 
%%url https://www.facebook.com/mischishin/posts/4269772573058729
 
%%author_id mischishin_mihail
%%date 
 
%%tags elita,gosudarstvo,nezalezhnist,ukraina
%%title Украинская элита - неспособноcть создать свое государство
 
%%endhead 
 
\subsection{Украинская элита - неспособноcть создать свое государство}
\label{sec:29_08_2021.fb.mischishin_mihail.1.elita_ukr_gosudarstvo}
 
\Purl{https://www.facebook.com/mischishin/posts/4269772573058729}
\ifcmt
 author_begin
   author_id mischishin_mihail
 author_end
\fi

Почему украинская элита раз за разом оказывается неспособной создать свое
государство? В прошлые века и во время нынешнее?

\ifcmt
  pic https://scontent-frt3-2.xx.fbcdn.net/v/t1.6435-9/240684084_4269690273066959_1112682551771107061_n.jpg?_nc_cat=101&ccb=1-5&_nc_sid=730e14&_nc_ohc=U7R1VNvt3zsAX_CfP1N&_nc_ht=scontent-frt3-2.xx&oh=87c9fa3724bf17c2c17c5b84ebe523ce&oe=615BC123
  width 0.4
\fi

Некоторые привычно списывают это на несовершенство характера украинца. Наше
врожденное хуторянство, кумовство и местничество.

Но украинец - ничем не отличается от представителей других народов. Ни в
лучшую, ни в худшую сторону. По большому счету все люди одинаковы.

И вряд ли причиной нашей неспособности создать не фантом из герба и флага, а
свое, живое и развивающееся государство, которое обеспечивало бы свой народ не
только новыми войнами и новыми порциями патриотических лозунгов, но и реально
улучшало его жизнь, причем, не только верхов, но и низов, чем отличаются или
отличались ведущие державы мира, являются какие-то субъективные свойства
украинцев.

Ключи от каптерки в непобедимой и доблестной советской армии, да, пришлите,
пожалуйста, нам. А в остальном украинец ничем и не отличается от остальных,
более успешных народов.

Так в чем же причина поразительной и уже явно и не раз показанной нам самой
историей нашей неспособности создать свое государство?

Все дело в том, что нам его создает кто-то другой. Причем в своих интересах.

Разумеется, это дело заинтересованными особами стыдливо прикрыто ширмой нашои
споконвичнои боротьбы та вызвольных змагань. Но это все  херня, малята.

Это ритуальные танцы власти, скрывающей свою зависимость от тех заморских дядь, кто ее поставил.

И им не стыдно. И нам приятно. Херои, хуле!

Но вот эта приятная нашему самолюбию ширма, скрывающая то, что нашу державнисть
нам всегда обеспечивает кто-то другой, то ли Россия, то ли Запад - она мешает
нам понять одно из главных обстоятельств, которые не дают нам создать свое
полнокровное государство, для которого во главе угла стоял бы самый простой и
пересичный украинец.

Не на словах нашей продажной и управляемой власти. А на деле. И, чтоб мы все
это ощущали. Да, я - простой украинец. Русский, еврей, поляк, болгарин, венгр,
белорусс, литовец... И я в этой стране - самый главный.

Тогда Украина бы колосилась и набиралась сил от всех живущих на ее территории
народов. А не разделялась бы в себе, увядала и падала раз за разом в веках, как
любое разделенное царство в Евангелии.

Какова же реальная ситуация с нашей готовностью создать свое государство, и готова ли к этому наша элита?

Насколько сильна в ней и влиятельна государствообразующая прослойка, являющаяся костяком любого государства?

Увы, это вовсе не наши патриоты, гордо именующие себя державныкамы. Они-то как
раз раз за разом и показывают свою полнейшую неспособность создать и развивать
свое государство.

Свой особняк, своих детей, которые учатся в Англии, кума, брата, свата развить,
коль сам пробрался на хорошую госдолжность - это пожалуйста. 

Поговорить за Украину - это в любое время дня и ночи!

Создать свое государство, оторвав что-то от себя самого - нема дурных!

Почему же так происходит?

Герои, готовые умирать за свои страну и народ, у нас есть. А государство часто
носит фиктивный и декларативный характер.

Наша элита, по большому счету еще не родилась.

А то, что называет себя элитой в Украине, таковой не является. По крайней мере,
в части государственного строительства.

Нашу элиту Запад лепит из того, что было. И закрывает глаза на то, что
получилось. Так ему выгодно.

Распад СССР произошел в сроки, нужные Америке. Американцев поджимал
надвигающийся кризис, и им нужны были новые территории, освоение которых
отодвинуло бы его. Что, собственно, и произошло.

Вызревание национальных элит на постсоветском пространстве до способности
создать свое государство не имеет к борьбе СССР и Америки никакого отношения.
Она влияет на процесс, ускоряет его. Но в целом это - другое. И там свои
непререкаемые 9 месяцев для рождения. Иначе - выкидыш.

Но для цветной революции наша элита, наше украинское гуайдо - сойдеть.

И вот ее и объявляют способной создать свое государство.

Которое, правда, раз за разом не может существовать без иностранных штыков и
денег.

Разве не так?

Мы принимаем свою элиту, как данность.

Но она не готова создать свое государство.

У нее ещё младенческий хватательный рефлекс.

Она может растащить. Но не создать.

Кравчук - не элита. И Ющенко - не элита. И Кучма - не элита.

Это же можно сказать и про Януковича, Порошенко и Зеленского.

Причем качество украинской элиты от Кравчука к Зеленскому не растет. А,
наоборот, парадоксальным и катастрофическим образом ухудшается.

Даже партаппаратчики Кравчука - это небо и земля по сравнению с 95 кварталом Зеленского.

Украинские партаппаратчики, из которых идеолог Кравчук был далеко-далеко не
лучшим, создали страну, которая летала в космос. Нынешние лучшие люди Украины
отправляются в космос гораздо чаще. Но проще, без всего этого пафоса с
трансляцией  старта, и не окольным, а прямым путем. 

Борьба Запада с Россией, вроде бы опирающаяся на создание нацэлит на
постсоветском пространстве, не может создать качественную национальную элиту.

По определению.

Запад создаёт управляемые коррупционные колониальные элиты, послушные себе.

Они не в состоянии построить национальное государство в интересах большинства
народа, если только не кормить его одними лозунгами про свободу и демократию.
Как закон. Потому что им это и не нужно.

Они строят симулякр, которым являются сами.

Но зато они успешно умеют имитировать процесс, чтобы освоить как можно больше
западных денег и ресурсов своей страны.

Исключение - Польша, Венгрия и Россия. У них есть имперский опыт. И это
позволяет им строить свои сравнительно успешные государства.

У Украины такого имперского опыта нет. По крайней мере, невыдуманного.

И это возвращает нас к вопросу, что подобранная и обученная Западом украинская
элита, все эти длинноногие и политически грамотные Гарвардские мальчики и
девочки, не являются, собственно, элитой. И их задача - переподчинить Западу
украинские финансовые потоки и легализовать, наладить грабеж страны в интересах
Европы и Америки.

И со своей реальной задачей они справляются прекрасно!

Но это мало похоже на строительство государства Украина.

Понимаешь это только тогда, когда перестаешь смотреть телевизор и интернет.

Потому что в телевизоре и интернете украинское государство есть!

Оно есть даже на улице и на параде.

Но его гораздо меньше в нашей повседневной жизни, если не считать обилие
сине-желтого цвета вокруг - с точки зрения интересов простого украинца.

Чтобы убедиться в этом, достаточно взглянуть на то, что происходит в Нафтогазе.
И кто, и в чью пользу влияет на его решения об увеличении тарифов для населения
в 11 раз.

Как говаривает наш батюшка Байден, что делать в украинской энергетике -
спросите у Амо Хохштайна.

И поставь на его место нашего, украинца - будет все то же самое.

И даже хуже.

Для того, чтобы быть, надо вначале родиться.

Вот, собственно, все про украинское государственное строительство.

А наш сельский украинский дурень нехай думкою про незалежнисть багатие.
