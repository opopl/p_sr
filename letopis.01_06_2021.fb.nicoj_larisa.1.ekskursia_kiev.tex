% vim: keymap=russian-jcukenwin
%%beginhead 
 
%%file 01_06_2021.fb.nicoj_larisa.1.ekskursia_kiev
%%parent 01_06_2021
 
%%url https://www.facebook.com/nitsoi.larysa/posts/926021254630914
 
%%author Ницой, Лариса
%%author_id nicoj_larisa
%%author_url 
 
%%tags 
%%title Екскурсия по Киеву
 
%%endhead 
 
\subsection{Екскурсия по Киеву}
\label{sec:01_06_2021.fb.nicoj_larisa.1.ekskursia_kiev}
\Purl{https://www.facebook.com/nitsoi.larysa/posts/926021254630914}
\ifcmt
 author_begin
   author_id nicoj_larisa
 author_end
\fi

У дитинстві мені найбільше запам'яталися канікули, коли дідусь повіз мене на
екскурсію в Київ. Я тоді жила в районному містечку і приїхала до дідуся й
бабусі на літо в село. А він візьми, та й організуй таку поїздку. Сказав:
\enquote{Хочу показати внукам столицю}. Ми не мали свого авто, тож добиралися
потягом і ракетою по Дніпру... Дитячі враження я пам'ятаю й досі. 

Пройшло багато років. Сьогодні я живу в столиці і думаю, що багато батьків
залюбки відправили б своїх дітей на екскурсії до Києва, але у них виникає
закономірне питання, як це зробити й до кого їхати? Не у всіх у Києві живуть
родичі...

\ifcmt
  pic https://scontent-lga3-2.xx.fbcdn.net/v/t1.6435-9/195833183_926021227964250_5901035273101565299_n.jpg?_nc_cat=105&ccb=1-3&_nc_sid=8bfeb9&_nc_ohc=mdvOzrzPGcAAX-xLgfD&tn=ntrKbsW_7ChXu3v-&_nc_ht=scontent-lga3-2.xx&oh=84ce4cb97ff9e9e59f2ed0e14d3ec8ab&oe=60DDAF21
\fi

Приїздіть. Усіх зустріну.  Відправляйте дітей групами. Поселю в недорогому, але
затишному хостелі. Забезпечу триразове харчування. Ненадокучливо, але
пізнавально і весело проведу екскурсії. Сходимо в кіно, на боулінг чи на інші
розваги. Покатаємося в метро. Пообнімаємося в зоопарку з мавпами або єнотами.
Покажу історичні місця, які описані в підручниках.

Пишіть в особисті повідомлення або дзвоніть 068 002 88 90

\emph{Anatoly Parfeniuk}

І пані Фаріон, І пані Ніцой обидві публічні особи, патріотки, але наскільки
такими ось практичними вчинками теплішою, людянішою виглядає саме остання,
честь їй і хвала.

\emph{Лариса Ніцой}
\textbf{Anatoly Parfeniuk} Дякую, але Ірина моя товаришка і мені прикро, що Ви її попускаєте.

\emph{Наталія Жученя}

Дуже гарна ідея! Мені аж самій захотілося до вас на екскурсії! Я впевнена що з
вами буде цікаво не тільки дітям! Колись ми з чоловіком теж возили своїх двох
дітей до Києва. Але ж ми не могли показати їм місто так як то ви зробите.
Крутезна ініціатива!

\emph{Катерина Бровко}
Пані Лариса , Ви найкраща , порядна, щира українка.Дякую !
 · Reply · See Translation · 2d
\emph{Svitlana Zorii}
Пані Ларисо,ви щира і хороша людина.
А ваша ідея дуже актуальна...Дякуємо💛💕💙
\emph{Dmytro Pikula}

Ви - щире золото...

\emph{Mykola Rybak}

Поспілкувався з пані Ларисою, отримав заряд на тиждень💙💛

\emph{Віра Марголенко Коляса}

Цікавить ціна

\emph{Неля Романюк}

Яка гарна ідея. А яка вартість такої поїздки?

\emph{Tetyana Rabchevska}

Чудового Вам літа разом малими, в майбутньому великими , дякуючи також Вам, українцями.

\emph{Тетяна Матюхіна}
Цікавить ціна?
 · Reply · See Translation · 1d
Тетянка Дзей
Дякую, це приклад як змінити країну: почни з себе, тобто зроби, щось для людей. Дуже зацікавило.
 · Reply · See Translation · 2d
\emph{Roma Roman}
Гарна справа. Вдачі Вам.
 · Reply · See Translation · 1d · Edited
\emph{Лариса Мельник}
Ви - неперевершена, пані Ларисо!

\emph{Yuriy Nazarkevych}
Гарна ідея. Я теж пам'ятаю свою першу поїЗдку до Києва в п'ятому класі. Нас водив по столиці колишній директор нашої школи. Майже ,як Ваш дідусь.
 · Reply · See Translation · 2d
\emph{Константин Попов}
обiйми з мавпами це файно
 · Reply · See Translation · 2d
\emph{Nadiya Danik}
Гарна, порядна, щира, врівноважена, доброзичлива,вихована, чуйна, інтелегентна, ініціативна, людяна, патріотична.
Одним словом УКРАЇНКА, здоров'я вам шановна пані Лариси, творчої наснаги.

\emph{Игорь Алексеев}
Своего ребенка вам? Развеселила))) да никогда в жизни!!!
 · Reply · 1d
\emph{Олена Сухецька}
Дякую.
 · Reply · See Translation · 2d
\emph{Виталий Гаврилин}

Я и не сомневался, что пани Ницой из \enquote{районного мистэчка}. Всё согласно
директиве ОУН -- \enquote{сёла бросить на города, в городах очаг вражеской
власти}.  Проблема в том, что кожухи и солома не совместимы с наукой, культурой
и цивилизацией. Но это уже другой вопрос. Главное, чтобы атомные станции
закрыли безопасно, под наблюдением иностранных специалистов, а дальше
веселитесь как угодно.
