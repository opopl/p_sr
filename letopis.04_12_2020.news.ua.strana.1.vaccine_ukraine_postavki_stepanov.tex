% vim: keymap=russian-jcukenwin
%%beginhead 
 
%%file 04_12_2020.news.ua.strana.1.vaccine_ukraine_postavki_stepanov
%%parent 04_12_2020
 
%%url https://strana.ua/news/304693-maksim-stepanov-rasskazal-o-srokakh-podpisanija-sohlashenija-o-postavkakh-vaktsiny-ot-kovida.html
 
%%author 
%%author_id 
%%author_url 
 
%%tags 
%%title Вакцина от коронавируса. Минздрав назвал дату подписания первых соглашений и сроки поставок
 
%%endhead 
 
\subsection{Вакцина от коронавируса. Минздрав назвал дату подписания первых соглашений и сроки поставок}
\label{sec:04_12_2020.news.ua.strana.1.vaccine_ukraine_postavki_stepanov}
\Purl{https://strana.ua/news/304693-maksim-stepanov-rasskazal-o-srokakh-podpisanija-sohlashenija-o-postavkakh-vaktsiny-ot-kovida.html}

\index[names.rus]{Степанов, Максим!Министр здравоохранения Украины}
\index[names.rus]{Коронавирус!Украина!Сроки поставок вакцины}

\ifcmt
pic https://strana.ua/img/article/3046/93_main-v1607070811.jpeg
caption Максим Степанов анонсировал скорое подписание соглашения о поставках в Украину вакцины от коронавируса. Фото: "Страна"
\fi

В ходе часа вопросов к правительству на заседании Рады в пятницу, 4 декабря,\Furl{https://strana.ua/video/304676-zasedanie-verkhovnoj-rady-ukrainy-4-dekabrja-onlajn-transljatsija.html}
глава Минздрава Максим Степанов анонсировал подписание 7 декабря соглашения по
поставкам вакцин от коронавируса в Украину. 

\enquote{Украина получит вакцину, около 8 млн доз по глобальной инициативе COVAX. До 7
декабря у нас запланировано подписание соответствующих документов. Я думаю, что
срок получения этой вакцины - конец первого - начало второго квартала 2021
года. Кроме этого, мы находимся на стадии финализации договоренностей с
мировыми компаниями-производителями вакцин, которые уже завершили проведение
соответствующей третьей фазы клинических исследований. Это в том числе компания
Pfizer}, - заявил Максим Степанов. 

Напомним, в рамках программы COVAX Украине на безоплатной основе будут выделены
дозы вакцины\Furl{https://strana.ua/news/301994-privivka-ot-koronavirusa-v-ukraine-kohda-mozhno-budet-sdelat-tsena-proizvoditel-vaktsiny.html} - по одной единице на 20\% населения. Для курса иммунизации одну
человеку нужно две инъекции. Поэтому таким образом привить удастся 4 миллионам
украинцев - 10\% населения.

Также мы писали о том, что Кабмин внес вакцину от коронавируса в список
освобожденных от НДС лекарств.\Furl{https://strana.ua/news/304293-vaktsinu-ot-koronavirusa-budut-vvozit-v-ukrainu-bez-nds-reshenie-kabmina.html}
