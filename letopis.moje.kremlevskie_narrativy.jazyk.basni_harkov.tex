% vim: keymap=russian-jcukenwin
%%beginhead 
 
%%file moje.kremlevskie_narrativy.jazyk.basni_harkov
%%parent moje.kremlevskie_narrativy.jazyk
 
%%url 
 
%%author_id 
%%date 
 
%%tags 
%%title 
 
%%endhead 

\paragraph{Харьковския басни}
\label{sec:moje.kremlevskie_narrativy.jazyk.basni_harkov}

Также, наш Григорий Саввич, кроме красивой поэзии, писал также философско\hyp
мировоззренческую прозу, в которой можно найти осмысление самых разнообразных
жизненных ситуаций, выраженное через красивый поэтический язык, через язык
образов животных и явлений природы.  Мы имеем в виду его цикл под названием
\enquote{Харьковские басни}.  Позвольте здесь также привести небольшие тексты
из этого цикла.

\ifcmt
tab_begin cols=2
  @caption Памятник Григорию Сковороде в городе Киеве

  ig https://avatars.mds.yandex.net/get-altay/2776464/2a00000170f4a99f4b8134bb851eb03ec81d/XXL
  ig https://ic.pics.livejournal.com/s_a_plotnikov/85688553/71932/71932_original.jpg

tab_end
\fi

