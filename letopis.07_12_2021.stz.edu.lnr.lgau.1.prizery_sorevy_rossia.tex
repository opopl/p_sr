% vim: keymap=russian-jcukenwin
%%beginhead 
 
%%file 07_12_2021.stz.edu.lnr.lgau.1.prizery_sorevy_rossia
%%parent 07_12_2021
 
%%url http://lnau.su/novosti/sportsmeny-lgau-stali-prizerami-sorevnovanij-v-rossii
 
%%author_id 
%%date 07_12_2021
 
%%tags 
%%title Спортсмены ЛГАУ стали призерами соревнований в России
 
%%endhead 
\subsection{Спортсмены ЛГАУ стали призерами соревнований в России}
\label{sec:07_12_2021.stz.edu.lnr.lgau.1.prizery_sorevy_rossia}

\Purl{http://lnau.su/novosti/sportsmeny-lgau-stali-prizerami-sorevnovanij-v-rossii}

\textSelect{Студенты кафедры физического воспитания Луганского государственного аграрного
университета (ЛГАУ) с 1 по 3 декабря принимали участие в Кубке Ростовской
области по тяжелой атлетике, который проходил в городе Шахты, где заняли
призовые места.}

\ifcmt
  pic http://lnau.su/wp-content/uploads/2021/12/1-3-1536x319.jpg
  @width 0.8
\fi

В ЛГАУ активно работает кафедра физического воспитания, где занимаются студенты
университета, которые не просто на словах, а на деле показывают свои умения,
приобретенные благодаря работе преподавателей кафедры вуза.

Отметим, что большую поддержку обучающимся оказывает заведующий кафедрой
физического воспитания ЛГАУ \textSelect{Владимир Осипов}, который всегда болеет
за своих подопечных.

Луганский государственный аграрный университет в мероприятии представили двое
спортсменов:

\begin{itemize} % {
\item \textSelect{Никита Симонов}, студент первого курса факультета землеустройство и
кадастров, который занял I место в весовой категории до 81 кг, с результатом:
рывок 115 кг, а в толчке 155, что и обеспечило ему I место в весовой категории.
Отметим, что в его весовой категории было заявлено 10 спортсменов.

\item \textSelect{Игорь Отрутько}, студент первого курса биолого-технологического
факультета, выступал в весовой категории до 109 кг, и занял I место, с
результатом в рывке − 142 кг, а в толчке − 192 кг, что и помогло ему в упорной
борьбе занять I место в своей весовой категории.
\end{itemize} % }

Своими впечатлениями после соревнований поделился \textSelect{Никита Симонов}: 

\begin{zzquote}
−  В принципе результатом я доволен, но в упражнении – рывок, можно было
добавлять и добавлять. Штанга была очень легкой, но движение до конца не шло и,
в принципе, остался на первом подходе. В толчке было достаточно легко, я поднял
первый и второй подход, а на третьем немного расслабил спину и взял вес на
грудь из-за чего потянуло вперед, и я не справился с весом. В целом результатом
я доволен, но впереди достаточно много стартов, где можно себя реализовать, что
мы и будем делать.
\end{zzquote}

\ii{07_12_2021.stz.edu.lnr.lgau.1.prizery_sorevy_rossia.pic.1}

Кафедра физического воспитания ЛГАУ оснащена всеми видами инвентаря, студенты
университета имеют отличную возможность получать не только качественные знания,
но также развиваться физически и учувствовать в различных спортивных
мероприятиях.

Об участии в соревновании также рассказал и \textSelect{Игорь Отрутько}:

\begin{zzquote}
− Результатом я очень доволен, так как была сильная конкуренция, где я
столкнулся со своим земляком из Донецкой Народной Республики (ДНР), который
тоже очень сильный соперник. Мы показали достойный результат. Я думаю, что
зрителям и тренерам, которые наблюдали за нашим противостоянием, очень
понравилось. Хотелось бы выразить благодарность своим тренерам за их поддержку:
Сергею Романову, который является моим первым тренером и, конечно же, Андрею
Костину. Наши наставники прилагают все усилия, чтобы мы реализовывались как
спортсмены.  	
\end{zzquote}

Отметим, что спортсмены ЛГАУ не собираются останавливаться на достигнутом и уже
сейчас готовятся к выезду в город Иловайск на кубок ДНР по тяжелые атлетики
среди мужчин и женщин в честь заслуженного тренера ЗТУ СССР В.П. Соца, и
чемпионату ДНР среди мужчин и женщин в отдельных упражнениях.

\begin{zzquote} 
− После данных соревнований мы планируем поехать в город Курск,
куда готовится наша команда тяжелоатлетов, − отметил \textSelect{Андрей Костин.}	
\end{zzquote}

Также тренер команды пригласил всех школьников Республики поступать к нам в
университет и, конечно же, приходить заниматься на спортивную кафедру.

\begin{zzquote} 
− Ведь ЛГАУ − кузница чемпионов, − добавил \textSelect{Андрей Костин.}
\end{zzquote}

\textSelect{Пресс-центр университета, фото участников мероприятия.}

