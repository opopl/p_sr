% vim: keymap=russian-jcukenwin
%%beginhead 
 
%%file 06_09_2018.stz.news.ua.mrpl_city.1.elizaveta_birjukova_mariupol_spisok_shindlera
%%parent 06_09_2018
 
%%url https://mrpl.city/blogs/view/elizaveta-biryukova-chi-mariupolskij-spisok-shindlera-1
 
%%author_id demidko_olga.mariupol,news.ua.mrpl_city
%%date 
 
%%tags 
%%title Єлизавета Бірюкова, чи "Маріупольський список Шиндлера"
 
%%endhead 
 
\subsection{Єлизавета Бірюкова, чи \enquote{Маріупольський список Шиндлера}}
\label{sec:06_09_2018.stz.news.ua.mrpl_city.1.elizaveta_birjukova_mariupol_spisok_shindlera}
 
\Purl{https://mrpl.city/blogs/view/elizaveta-biryukova-chi-mariupolskij-spisok-shindlera-1}
\ifcmt
 author_begin
   author_id demidko_olga.mariupol,news.ua.mrpl_city
 author_end
\fi

\vspace{0.5cm}

\begin{raggedleft}
\em\enquote{Залишатися людиною в нелюдські моменти}. Єлизавета Бірюкова
\end{raggedleft}

\ii{06_09_2018.stz.news.ua.mrpl_city.1.elizaveta_birjukova_mariupol_spisok_shindlera.pic.1}

\begin{center}
\enquote{В каждом городе есть свои легенды,\par
На каждой улице есть свои персонажи,\par
В каждой голове есть свои мысли,\par
И я хочу все... узнать их однажды}. \par
\end{center}

Це слова з однієї пісні. Вони співзвучні моїм думкам. Скільки в Маріуполі ще
невідомих фактів, легенд?! Чи достатньо ми знаємо про людей, чиї вчинки,
самовідданість і героїзм вражають?! Я впевнена, що ми знаємо недостатньо і
Маріуполь зберігає в собі безліч унікальних біографій, цікавих фактів, які
будуть поступово розкриватися. Так сталося і з маріупольчанкою \textbf{\em Єлизаветою
Пилипівною Бірюковою}, біографічні відомості й фото якої представлені вперше. 

З нагоди 75-ї річниці визволення Маріуполя команда сайту MRPL.CITY вирішила
спробувати відновити факти з життя та діяльності Єлизавети Пилипівни, героїзм і
самовідданість якої сьогодні повинні знати та пам'ятати. Сподіваюся, що
знайомство з цією непересічною і дуже сильною жінкою стане для вас, дорогі
читачі, корисним і цікавим.

\ii{06_09_2018.stz.news.ua.mrpl_city.1.elizaveta_birjukova_mariupol_spisok_shindlera.pic.2}

У 2013 році українська письменниця, історик Олена Вікторівна Стяжкіна знайшла
матеріали по справі Єлизавети Бірюкової, молодої жінки з вищою технічною
освітою. Саме завдяки цій події громадськість вперше заговорила про невідомий
подвиг маріупольчанки. Під час окупації Маріуполя нацистами Єлизавета була
змушена залишитися в місті й працювати на рибоконсервному комбінаті. Але під
час роботи вона починає абсолютно системно рятувати людей. Пізніше вона була
засуджена за зв'язок з шефом комбінату і за те, що нібито на когось доносила.
Проте після її арешту люди, яких вона врятувала, проявивши справжню сміливість,
почали писати листи Вишинському, Берії та Сталіну. До цього часу історія життя
маріупольчанки, гідної захоплення, залишалася невідомою.

\ii{06_09_2018.stz.news.ua.mrpl_city.1.elizaveta_birjukova_mariupol_spisok_shindlera.pic.3}

Завдяки онучкам Єлизавети Пилипівни, Оксані Миколаївні та Наталі Миколаївні,
нам вдалося дізнатися багато унікальних деталей з біографії нашої героїні.
Водночас онучки поділилися світлинами з сімейного архіву.

Народилася Єлизавета 11 жовтня (27 вересня) 1914 року в Маріуполі, проживала по
вулиці Артема, 15 (тепер Куїнджі) в дружній родині. Батько – Пилип Семенович –
робочий столяр. Мама Єлизавети – Єфросинія Прокопівна, за розповіддю онучок,
мала знатне походження. А братик Костя, незважаючи на різницю у віці, став
найближчим і кращим другом для Лізи.

\ii{06_09_2018.stz.news.ua.mrpl_city.1.elizaveta_birjukova_mariupol_spisok_shindlera.pic.4}

Батьки зробили все можливе, щоб Ліза з братом отримали вищу освіту і ні від
кого не залежали. Єлизавета була чесною, справедливою і відповідальною
дівчиною. З юності привчала себе до праці, віддавалася роботі повністю, була
справжнім трудоголіком. У свої 27 років вона вже була дипломованим
інженером-технологом маріупольського рибоконсервного комбінату, працювала
старшим лаборантом.

Коли в червні 1941 року німецькі війська напали на Радянський Союз, ніхто в
Маріуполі не очікував, що довгий шлях до них в 1300 кілометрів подолають так
швидко. 1 вересня маріупольці почали зводити захисні споруди, а вже 8 жовтня
місто захопили. Мама Лізи виїхала з Маріуполя, брат вирушив на фронт. А батько
залишився з донькою, яка захворіла на скарлатину, що підтверджує лікарняний
лист, який зберігся в справі Єлизавети Пилипівни. Дівчина, з огляду на
обставини, була змушена продовжувати роботу на рибоконсервному комбінаті. Вона
добре володіла німецькою мовою, тому до неї часто зверталися німці, вважаючи
Лізу добросовісною і старанною робітницею.

\textbf{Читайте також:} \href{https://mrpl.city/blogs/view/mariupol-22-iyunya-1941-goda}{%
Мариуполь: 22 июня 1941 года, Сергій Буров, mrpl.city, 22.06.2018}

Завдяки матеріалам справи, знайденим О. Стяжкіною, вдалося дізнатися, що
Єлизавета Пилипівна намагалася зберегти молодь від поїздки до Німеччини. До
того ж Бірюкова робила все можливе, щоб прийняти на роботу всіх, кому
загрожувала небезпека з боку німецького командування: дружин комуністів, євреїв
(які мають дітей), комуністів і комсомольців. Також вона, ризикуючи власною
безпекою, звільняла військовополонених з табору, пояснюючи це тим, що їм треба
працювати на заводі, адже робітники потрібні завжди. До цих пір достовірно не
відомо, скільки насправді було врятовано маріупольців. Відповідно до матеріалів
справи й того, скільки всього робила Єлизавета Пилипівна, ця цифра може бути
більшою ніж 100 чи навіть 200 чоловік.

\ii{06_09_2018.stz.news.ua.mrpl_city.1.elizaveta_birjukova_mariupol_spisok_shindlera.pic.5}

Коли у вересні 1943 року радянські війська звільнили Маріуполь, дуже швидко на
столі в НКВС з'явився донос на Бірюкову, в якому наголошувалося, що вона
колабораціоністка і зрадниця Батьківщини. Нашу героїню звинувачували в
небажанні евакуюватися, у зв'язку з шефом комбінату німцем Бендером, який
висунув її на керівну посаду начальника виробництва рибкомбінату, у відправці
радянських громадян на каторжні роботи до нацистської Німеччини.

\ii{06_09_2018.stz.news.ua.mrpl_city.1.elizaveta_birjukova_mariupol_spisok_shindlera.pic.6}

Суд присудив Єлизаветі Пилипівні 20 років каторжних робіт і 5 років ураження в
правах. У тих складних обставинах все ж таки Єлизаветі Бірюковій пощастило не
бути розстріляною. Військовий трибунал постановив, що підсудна діяла за
вказівкою шефа рибоконсервного комбінату, і хоча вона була названа зрадницею,
проте отримала кваліфікацію \enquote{пособниця}, що і стало для неї порятунком. Однак
Єлизавета Пилипівна своєї вини не визнавала.

\ii{06_09_2018.stz.news.ua.mrpl_city.1.elizaveta_birjukova_mariupol_spisok_shindlera.pic.7}

У роки війни судетський німецький промисловець Оскар Шиндлер врятував майже
1200 євреїв під час Голокосту, надавши їм роботу на своїх заводах в Польщі та
Чехії. Єлизавета Бірюкова, як і Оскар Шиндлер, намагалася рятувати людей, які
не забули про вчинки цієї відважної та сміливої жінки. Завдяки матеріалам
справи відомо, що відпустити Єлизавету Пилипівну просили 11 осіб, але з огляду
на всі тогочасні обставини, ці люди, характеризуючи нашу героїню як рятівницю,
проявили неабияку сміливість.

Засуджена Єлизавета Бірюкова відбувала покарання у Воркуті, там, незважаючи на
важкі каторжні роботи, на всю несправедливість долі, вона продовжувала
зберігати людське обличчя. За розповіддю онучки, Оксани Миколаївни Кушнір, у
колонії бабусі написали портрет. Онучки знали, що бабуся була в колонії, але їм
нічого не було відомо про причини та про героїзм Єлизавети Пилипівни під час
окупації.

\ii{06_09_2018.stz.news.ua.mrpl_city.1.elizaveta_birjukova_mariupol_spisok_shindlera.pic.8}

І все ж завдяки листам батька Єлизавети, небайдужих маріупольців та її скарзі
головному військовому прокурору Збройних сил СРСР, яку вона написала, коли
отримала дозвіл на листування, справу Є. Бірюкової все ж таки переглянули.
Повторний суд відбувся у лютому 1948 році в м. Сталіно. Єлизавета Пилипівна все
ж таки була визнана винною в пособництві, формою якого назвали ефективну роботу
на керівній посаді рибоконсервного комбінату. Проте мірою покарання було
названо лише чотири роки й чотири місяці каторжних робіт, які на момент суду
були відбуті. Втомлена, але незламна жінка не погодилася з вироком і все життя
намагалася отримати реабілітацію.

\textbf{Читайте також:} 

\href{https://mrpl.city/blogs/view/pamyatniki-voinam-vizvolitelyam-mariupolya}{%
Пам'ятники Воїнам-визволителям Маріуполя, Ольга Демідко, mrpl.city, 09.05.2018}

Після колонії життя сміливої й відважної Єлизавети Пилипівни продовжувало
дивувати. Вона не боялася вступити в листування з архімандритом Вільнюського
монастиря, а це в роки тоталітаризму і проголошеного атеїзму. У 36 років
Єлизавета стала мамою. І це теж драматична та унікальна історія. Брат Костянтин
розповів, що знає нещасну і незаможну дівчину, але у неї на руках маленька
дитина – дівчинка 9 місяців. Костянтин запропонував Єлизаветі поглянути на
дитину. Але від побаченого Єлизавета Пилипівна не прийшла в захват. Дитина
погано одягнена, в ліжечку клопи. Онучка, Оксана Миколаївна, згадує слова
бабусі: \emph{\enquote{Єлизавета Пилипівна з батьками сиділи вдома, пили чай з печивом. З
роботи прийшов брат Костянтин і підкреслив: \enquote{Ви тут бенкетуєте, а дитя
помирає}. Ці слова справили досить сильне враження на Єлизавету і вона зробила
все можливе, щоб удочерити дівчинку, яку назвала Ірою}}.

\ii{06_09_2018.stz.news.ua.mrpl_city.1.elizaveta_birjukova_mariupol_spisok_shindlera.pic.9}

Єлизавета Пилипівна мала близькі й теплі стосунки з онуками як своїми, так і
брата. Онучка Оксана Миколаївна, згадуючи, зазначала: \enquote{Бабуся була дуже доброю,
м'якою, поблажливою, все прощала своїм онукам. Крім цього, з бабусею були
довірчі відносини. Вона дуже смачно готувала. Завжди шанувала традиції, збирала
разом всю родину. Прекрасно грала на фортепіано, дуже любила твір Бетховена \enquote{До
Елізи}. Все життя працювала. Сьогодні це назвали б кар'єризмом. Бабуся була
дуже прямою, справедливою. У чоловіках найбільше цінувала інтелект}.

\ii{06_09_2018.stz.news.ua.mrpl_city.1.elizaveta_birjukova_mariupol_spisok_shindlera.pic.10}

Єлизавета Пилипівна ніколи нікому з рідних не говорила про лихоліття, яке їй
довелося пережити. Не розповідала про врятованих нею маріупольців. Вона була
для онуків чудовою бабусею, для робочого колективу – цінним співробітником, але
ніхто не знав її до кінця. Ніхто з маріупольців і уявити не міг, наскільки
сильна, вольова жінка проживала в їхньому місті по вул. Артема, 15, наскільки
довго вона чекала реабілітації. Залишитися вірною самій собі, пересилити біль,
яку неможливо забути, і прожити гідне життя після важких випробувань,
несправедливості, жорстокості під силу далеко не кожному. Все ж Єлизавета
Пилипівна була реабілітована в 1992 році, але за фатальним збігом обставин і
незрозумілим сценарієм життя цього ж року вона померла.

\ii{06_09_2018.stz.news.ua.mrpl_city.1.elizaveta_birjukova_mariupol_spisok_shindlera.pic.11}

Справжній героїзм одразу ніколи не привертає увагу. І тільки згодом люди
оцінюють благородство і мужність душ, які ризикують заради порятунку ближніх...

\ii{06_09_2018.stz.news.ua.mrpl_city.1.elizaveta_birjukova_mariupol_spisok_shindlera.pic.12}

\textbf{Читайте по темі:} 

\href{https://mrpl.city/news/view/mariupolchanka-v-gody-natsistskoj-okkupatsii-spasla-150-chelovek-video}{%
Мариупольчанка в годы нацистской оккупации спасла 150 человек (ВИДЕО ИЗ АРХИВА), %
Ярослав Герасименко, mrpl.city, 14.04.2017}

\textbf{Єлизавета Бірюкова – це не легенда, а жива історія, яку необхідно знати. І наш
обов'язок - зберегти пам'ять про неї та передати її історію наступним
поколінням, адже тільки пам'ять захищає від повторення помилок...}

\emph{Джерело: \url{http://mrpl.city/}}

\ii{insert.author.demidko_olga}
