%%beginhead 
 
%%file 25_01_2023.fb.komarovskij_evgenij.likar.harkiv.1.kak_skazku_sdelat_bi
%%parent 25_01_2023
 
%%url https://www.facebook.com/komarovskiynet/posts/pfbid031QMGXJJZCGQ2ifeqmi8QxSuFrgJC8QDFb9xyDiM1tUh5N4RNN964etXwML1TqhvBl
 
%%author_id komarovskij_evgenij.likar.harkiv
%%date 25_01_2023
 
%%tags skazka,herson,ajbolit,chukovskij_kornej,barmalej,literatura,kniga,pobeda,dobro,zlo
%%title Как сказку сделать былью. Добрая история из Херсона
 
%%endhead 

\subsection{Как сказку сделать былью. Добрая история из Херсона}
\label{sec:25_01_2023.fb.komarovskij_evgenij.likar.harkiv.1.kak_skazku_sdelat_bi}

\Purl{https://www.facebook.com/komarovskiynet/posts/pfbid031QMGXJJZCGQ2ifeqmi8QxSuFrgJC8QDFb9xyDiM1tUh5N4RNN964etXwML1TqhvBl}
\ifcmt
 author_begin
   author_id komarovskij_evgenij.likar.harkiv
 author_end
\fi

Не все смотрят видео, поэтому расскажу эту историю буквами.

Как сказку сделать былью. Добрая история из Херсона. 

В июле 2022 стартовал наш проект «Казки українською з доктором Комаровським» 

- вот ссылка на плейлист, там на сегодня 11 сказок -
\href{https://www.youtube.com/watch?v=oKe31ZLlA6s}{Подолаєм Бармалея. Доктор Комаровський читає казку Корнія Чуковського, %
Доктор Комаровский, youtube, 29.07.2022%
}

\ii{25_01_2023.fb.komarovskij_evgenij.likar.harkiv.1.kak_skazku_sdelat_bi.video.1.youtube}

Первой записанной сказкой была «Подолаєм Бармалея» Чуковского. Это
малоизвестная сказка, написанная в 1942 г в разгар Сталинградской битвы, очень
пророческая, как будто срисованная с нашей действительности – там затеявший
войну Бармалей, там Айболит, которому не хватает снарядов, там прилетевший
из-за океана герой Иванко Васильченко и конечно же Победа добра над злом. 

Мои читатели сказку перевели, дети нарисовали более 300 иллюстраций, в общем
круто получилось - 

\href{https://www.youtube.com/watch?v=oKe31ZLlA6s}{%
Подолаєм Бармалея. Доктор Комаровський читає казку Корнія Чуковського, %
Доктор Комаровский, youtube, 29.07.2022%
}

Теперь собственно история.

Письмо.

Здравствуйте доктор. Привет Вам из героичного измученного Херсона.

... хочу сказать Вам большое спасибо за сказку \enquote{Подолаєм  Баомалея}. В месяцы
оккупации я пыталась объяснить своей 6- летней дочери происходящее и мы ее
прослушали.

В дальнейшем, дочь ежедневно спрашивала, когда же придет Иванко Васильченко нас
спасать, я обещала, что скоро) 11.11.2022 Херсон освободили ВСУ, не буду
подробно описывать то счастье, которое витало в нашем городе, вся Украина об
этом знает. 

Но вопрос дочери  - где же Иванко? Остался. Ей нужно было увидеть героя,
которого она так ждала. С совершенно странной просьбой я пристала к нашему
солдату, объяснив ситуацию (вырвала его из объятий херсонцев) Он согласился и
предсавился моей Кире Иванком, дал ей жменю конфет, обнял и сфотографировался!
Так для маленькой херсонки сказка стала правдой...

Жаль сейчас не так радужно, но мы верим и ждем победы!  *** А это фотография –
Кира и Иванко!
