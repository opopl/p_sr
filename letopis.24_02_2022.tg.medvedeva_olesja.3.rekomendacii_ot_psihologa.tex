% vim: keymap=russian-jcukenwin
%%beginhead 
 
%%file 24_02_2022.tg.medvedeva_olesja.3.rekomendacii_ot_psihologa
%%parent 24_02_2022
 
%%url https://t.me/Medvedeva_Olesya/7666
 
%%author_id medvedeva_olesja
%%date 
 
%%tags __feb_2022.vtorzhenie,chelovek,psihologia
%%title Рекомендации от психолога
 
%%endhead 
 
\subsection{Рекомендации от психолога}
\label{sec:24_02_2022.tg.medvedeva_olesja.3.rekomendacii_ot_psihologa}
 
\Purl{https://t.me/Medvedeva_Olesya/7666}
\ifcmt
 author_begin
   author_id medvedeva_olesja
 author_end
\fi

Рекомендации от психолога:

Что делать если стало очень страшно? И случился приступ паники? Как себе
помочь? 

Сердце бешено стучит, кружится голова, темнеет в глазах, закладывает уши,
бросает в пот, ноги слабеют, немеют руки, тошнит, не хватает воздуха. 

Вам кажется, что происходит что-то ужасное, что вы умираете или сейчас
потеряете сознание. Это приступ паники, паническая атака. 

Абсолютно здоровое и совершенно не опасное состояние организма. Что происходит
на самом деле? 

Наш мозг получил информацию об опасности. Реальной или выдуманной опасности –
не важно. Мозг убежден, что опасность есть. И он включает режим защиты. 

Стартует симпатическая нервная система. Выбрасывается адреналин. Сердце
начинает быстро прокачивать кровь к мышцам рук, ног. Чтобы мы могли драться или
быстро убегать. 

Мощный прилив крови ощущается как слабость. Отток крови от головы ощущается как
головокружение. Отключается режим пищеварения, чтобы не тратить энергию. Это
ощущается как тошнота, вздутие. 

Чтобы себя успокоить нужно:

А - Знать, что происходит с организмом. 

Просто включился режим защиты. Есть расслабленный режим отдыха организма, а это
режим защиты. Он абсолютно не опасен, а просто некомфортен.

B - Правильно дышать. 

Сейчас вы дышите часто и быстро. И это проблема. Напряжены мышцы груди. Может
возникать ощущение нехватки воздуха. Чтобы успокоить себя, нужно восстановить
дыхание. 

Техника №1

Задержать дыхание

Выдыхаем, а потом задерживаем дыхание — считаем по себя 1, 2, 3, 4. Медленно
вдыхаем — считая 1, 2, 3, 4. Медленно выдыхаем — считая 1, 2, 3, 4. Медленно
вдыхаем... 

Дышим...

Можно во время этого дыхания потягиваться или напрягать мышцы рук, ног. 

Это поможет быстрее вывести адреналин. 

Техника №2

Медленное дыхание животом. 

Когда человек дышит работает грудная клетка и живот. Ваша задача – дышать одним
животом. 

Начинаем дышать медленно и глубоко. Вдыхаем медленно – выдыхаем медленно. 

Представьте, что ваш живот это воздушный шар. Который надувается, сдувается. 

Грудь должна оставаться на месте. Работает только живот. Можно положить одну
руку на грудь, а другую на живот и контролировать процесс. 

Рука на груди остается на месте, а рука на животе поднимается, опускается,
поднимается, опускается. 

Дышим спокойно, медленно.  Несколько минут. Пока не отпустит.
