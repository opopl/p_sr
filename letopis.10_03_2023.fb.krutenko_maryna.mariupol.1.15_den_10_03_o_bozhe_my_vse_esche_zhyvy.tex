%%beginhead 
 
%%file 10_03_2023.fb.krutenko_maryna.mariupol.1.15_den_10_03_o_bozhe_my_vse_esche_zhyvy
%%parent 10_03_2023
 
%%url https://www.facebook.com/marinakrytenko/posts/pfbid0SHVoMdzdh8jvRsABD9XgvmRLN2gNPHedCz789DDbH5HXT3DG7SYQnMUeMvy9YmEhl
 
%%author_id krutenko_maryna.mariupol
%%date 10_03_2023
 
%%tags 10.03.2022,dnevnik,mariupol.war,mariupol
%%title ПЯТНАДЦАТЫЙ ДЕНЬ ВОЙНЫ 10.03.22 О, Боже! Прошло пятнадцать дней, а мы все ещё живы…..
 
%%endhead 

\subsection{ПЯТНАДЦАТЫЙ ДЕНЬ ВОЙНЫ 10.03.22 О, Боже! Прошло пятнадцать дней, а мы все ещё живы…..}
\label{sec:10_03_2023.fb.krutenko_maryna.mariupol.1.15_den_10_03_o_bozhe_my_vse_esche_zhyvy}

\Purl{https://www.facebook.com/marinakrytenko/posts/pfbid0SHVoMdzdh8jvRsABD9XgvmRLN2gNPHedCz789DDbH5HXT3DG7SYQnMUeMvy9YmEhl}
\ifcmt
 author_begin
   author_id krutenko_maryna.mariupol
 author_end
\fi

ПЯТНАДЦАТЫЙ ДЕНЬ ВОЙНЫ 10.03.22

О, Боже! Прошло пятнадцать дней, а мы все ещё живы.....

Утром мы проснулись позавтракали и пока с Олей собирались, решили послать
детей, чтоб они заняли очередь за питьевой водой (чем мы вообще думали? В этот
период времени вообще туго думалось). Дети пошли.....

Начался бой.... Дети были на пол пути к воде... мы переживали... Прибежали дети и
сказали, что не смоги перейти через улицу Митрополитскую, видели как летят пули
и падают снаряды. 

Отсиделись дома и мы с Женей пошли за водой. Возле водовоза стояли уже не 1000
чел, а 30. Над головой летал самолёт, наверное тот даже, что вчера бомбил
роддом. Люди с безумными глазами кричали в небо \enquote{не бомби нас} это напоминало
психиатрическую больницу.... Как можно пытаться докричатся к тому, кто тебя даже
не видит. Я понимала, что моя жизнь от меня сейчас не зависит. 

На этот раз все было хорошо. Мы добрались домой. 

Ближе к обеду была объявлена эвакуация. Снова слёзы, прощание и попытка
покинуть город-\enquote{ад}. 

Подъехала к \enquote{Ильичёвец}, машин было около 100, много полицейских которые ждали
ответа о переговорах между красным крестом и российской оккупационной властью.
Вторые никаких гарантий не давали. С одним военным мы разговорились, он сказал
что его жена с семьей и подругой пару дней назад рискнула и выехала на машине
через Милекино. 

-Возьмите с собой сигареты и тушенку, чтоб дать взятку российским военным,
помолитесь и езжайте, сказал он. 

Но сигарет у нас не было, никто из нас не курит. Тушенки тоже не было, к тому
времени две банки красной икры мы уже съели. Нам было страшно выезжать без
официального разрешения. 

Спустя час мы вернулись домой. 

Продолжение следует....
