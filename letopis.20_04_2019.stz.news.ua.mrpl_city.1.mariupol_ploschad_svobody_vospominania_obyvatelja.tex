% vim: keymap=russian-jcukenwin
%%beginhead 
 
%%file 20_04_2019.stz.news.ua.mrpl_city.1.mariupol_ploschad_svobody_vospominania_obyvatelja
%%parent 20_04_2019
 
%%url https://mrpl.city/blogs/view/mariupolskaya-ploshhad-svobody-vospominaniya-obyvatelya
 
%%author_id burov_sergij.mariupol,news.ua.mrpl_city
%%date 
 
%%tags 
%%title Мариупольская площадь Свободы: воспоминания обывателя
 
%%endhead 
 
\subsection{Мариупольская площадь Свободы: воспоминания обывателя}
\label{sec:20_04_2019.stz.news.ua.mrpl_city.1.mariupol_ploschad_svobody_vospominania_obyvatelja}
 
\Purl{https://mrpl.city/blogs/view/mariupolskaya-ploshhad-svobody-vospominaniya-obyvatelya}
\ifcmt
 author_begin
   author_id burov_sergij.mariupol,news.ua.mrpl_city
 author_end
\fi

\ii{20_04_2019.stz.news.ua.mrpl_city.1.mariupol_ploschad_svobody_vospominania_obyvatelja.pic.1}

В 20-х числах 1970 года жена получила однокомнатную квартиру в девятиэтажном
доме № 106 по проспекту Ленина, ныне называемом проспектом Мира. Жилище было
невелико, но уютное. Самым большим удобством был балкон на всю ширину комнаты и
кухни. Именно на нем можно было курить и одновременно наблюдать за состоянием
строительства соседних домов. Так на глазах выросли соседствующий дом № 104 и
находящаяся в поле зрения девятиэтажка по проспекту Строителей, 92. И дом, в
котором довелось жить в течение восемнадцати лет, и соседние дома, находились
вблизи от перекрестка проспектов Мира и Строителей. Застройка пространства
проходила на наших глазах. По четной стороне проспекта строй девятиэтажек
дополнился домами № 108 и № 110.

\textbf{Читайте также:} 

\href{https://mrpl.city/news/view/tsentr-mariupolya-ozhidaet-masshtabnaya-rekonstruktsiya-foto-plusvideo}{Центр Мариуполя ожидает масштабная реконструкция, Олена Онєгіна, mrpl.city, 28.03.2019}

Долгие годы казалось, что возведение \enquote{высоток} было обусловлено желанием
местных архитекторов как-то приукрасить городское пространство, уйти от унылого
однообразия микрорайонов, застроенных пятиэтажными малогабаритными домами. Но
прошло время, и по рассказам людей, бывших при власти, оказалось, что далеко не
всем населенным пунктам позволялось строить жилые здания выше пяти этажей.
Нашему городу повезло. Хотя В. М. Цыбулько не был руководителем города,
фактически же именно он руководил им. Он добился в Москве разрешения на
возведение жилых домов в девять этажей. При этом он вовсе не преследовал
эстетические цели. Бурный рост населения города требовал увеличения жилого
фонда. При сооружении \enquote{хрущевок} необходимое количество не вписывалось в
существующие границы города. 

Сейчас трудно вспомнить, когда сформировалась застройка вокруг перекрестка.
Довольно продолжительное время незанятой оставалась территория по нечетной
стороне проспекта Строителей от главной магистрали города до бывшего Дома
политпросвещения, а теперь одного из корпусов Мариупольского государственного
университета. Практически свободным остается пространство от улицы Леваневского
до уже упоминавшегося проспекта Строителей. Приходилось слышать, что, будто бы
площади зарезервировали для строительства здания цирка. Но в Киеве вроде бы
отказали. Город не областной, и население его составляло только чуть-чуть
больше полумиллиона. Когда возвели монумент Ленину, стало ясно, что в
непосредственной близости с ним никаких зданий строить не будут. Где-то в
середине 90-х или чуть позже на пустыре ближе к улице Леваневского расставил
свои аттракционы чехословацкий \enquote{Луна-парк}. Народ повалил туда семьями.
Невидалью тогда было получение какой-нибудь заграничной безделушки в награду за
выигрыш в тире или на каком-нибудь другом развлекательном устройстве. Маленьких
детей невозможно было вытащить из \enquote{Луна-парка}.

\textbf{Читайте также:} 

\href{https://mrpl.city/news/view/na-ploshhadi-svobody-v-mariupole-hotyat-postroit-fontan-za-24-mln-griven}{%
На площади Свободы в Мариуполе хотят построить фонтан за 24 млн гривен, Роман Катріч, mrpl.city, 25.02.2019}

В лихие 90-е, когда людям не платили зарплату, заменяя ее полотенцами,
куртками, покрывалами и носками, которые ни есть, ни продать было невозможно,
на площади вдоль нечетной стороны люди посадили огороды. Существовали они
довольно долго.

Сейчас уже не вспомнить, что раньше было построено - второй корпус
Гуманитарного университета или Греческий культурный центр, сооруженный в
классическом стиле. Одновременно с ним на первых порах шло сооружение, как
говорили, делового центра некого греческого бизнесмена. Но что-то произошло, и
возведение его прекратилось. Каркас здания долго стоял, продуваемый всеми
ветрами. Уже был построен ангар \enquote{Обжоры}, а потом каркас стал обрастать
строительными деталями и, наконец приобрел вид офиса солидной фирмы –
\enquote{Киевстар}. Впрочем, и здесь что-то напутано. Времени-то прошло много. Проспект
Строителей, объединивший улицы Приморского и Центрального районов, получил
название, не с ветра взятое. Это была дань уважения строителям, чьими руками и
умом были возведены новые цеха заводов, кварталы жилых домов, школы и многое
другое, чем пользуются мариупольцы по сей день...

Что-то не помнится, чтобы перекресток назывался площадью. Для обозначения места
встречи или сообщения о своем месторасположения в народе говорят кратко и емко:
\enquote{Тыщик}, заменяя название \enquote{1000 мелочей}. Между прочим, стоит молодому
поколению сообщить, что первоначально в этом магазине торговали инструментами,
метизами. Там можно было купить бельевую веревку и прищепки, свечку на случай
отключения электричества. Все, что нужно было для новоселов.

Теперь на карте Мариуполя появилось новое название - площадь Свободы. Что ж,
будем привыкать.

\textbf{Читайте также:}

\href{https://mrpl.city/news/view/rekonstruktsiya-glavnoj-ploshhadi-mariupolya-za-74-mln-grn-parkovka-peshehodnyj-fontan-zdanie-s-bufetom-i-dushem-foto}{Реконструкция главной площади Мариуполя за 74 млн грн: парковка, пешеходный фонтан, здание с буфетом и душем, mrpl.city, 03.04.2019}
