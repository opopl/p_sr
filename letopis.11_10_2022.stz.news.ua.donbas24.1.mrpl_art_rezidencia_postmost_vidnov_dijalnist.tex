% vim: keymap=russian-jcukenwin
%%beginhead 
 
%%file 11_10_2022.stz.news.ua.donbas24.1.mrpl_art_rezidencia_postmost_vidnov_dijalnist
%%parent 11_10_2022
 
%%url https://donbas24.news/news/mariupolska-art-rezidenciya-postmost-vidnovlyuje-svoyu-diyalnist
 
%%author_id demidko_olga.mariupol,news.ua.donbas24
%%date 
 
%%tags 
%%title Маріупольська арт-резиденція PostMost відновлює свою діяльність
 
%%endhead 
 
\subsection{Маріупольська арт-резиденція PostMost відновлює свою діяльність}
\label{sec:11_10_2022.stz.news.ua.donbas24.1.mrpl_art_rezidencia_postmost_vidnov_dijalnist}
 
\Purl{https://donbas24.news/news/mariupolska-art-rezidenciya-postmost-vidnovlyuje-svoyu-diyalnist}
\ifcmt
 author_begin
   author_id demidko_olga.mariupol,news.ua.donbas24
 author_end
\fi

\ii{11_10_2022.stz.news.ua.donbas24.1.mrpl_art_rezidencia_postmost_vidnov_dijalnist.pic.front}
\begin{center}
  \em\color{blue}\bfseries\Large
Маріупольські художники продовжують свою діяльність в Німеччині
\end{center}

У День художника, 9 жовтня, відновила \href{https://donbas24.news/news/u-kijevi-vidkrilasya-vistavka-mariupolskix-xudoznikiv-foto}{свою діяльність}%
\footnote{У Києві відкрилася виставка маріупольських художників, Алевтина Швецова, donbas24.news, 17.09.2022, \par\url{https://donbas24.news/news/u-kijevi-vidkrilasya-vistavka-mariupolskix-xudoznikiv-foto}}
арт-резиденція Маріуполя PostMost. Про це на своїй сторінці у Facebook
розповіла маріупольська художниця, яка наразі перебуває у Німеччині, \href{https://www.facebook.com/profile.php?id=100002254840110}{Олена Украінцева}.%
\footnote{\url{https://www.facebook.com/profile.php?id=100002254840110}}

\begin{leftbar}
\emph{\enquote{PostMost живий! І ми починаємо. Художниці з Маріуполя, які втекли від
війни, знайшли сили і можливість зібратися, щоб продовжити творити про
Маріуполь, наш Маріуполь, про нас. Ми живі та ми продовжуємо малювати і
розповідати про наше чудове місто, яке було і яке буде}}, — розповіла Олена. 
\end{leftbar}

\textbf{Читайте також:} \href{https://donbas24.news/news/mariupolski-mitci-u-stolici-pokazut-silu-ta-krasu-mista-mariyi-de-podivitisya}{\emph{Маріупольські митці у столиці покажуть силу та красу міста Марії: де подивитися}}%
\footnote{Маріупольські митці у столиці покажуть силу та красу міста Марії: де подивитися, Еліна Прокопчук, donbas24.news, 16.06.2022, \par%
\url{https://donbas24.news/news/mariupolski-mitci-u-stolici-pokazut-silu-ta-krasu-mista-mariyi-de-podivitisya}%
}

% 1 - olena 
% 2 - farby
\ii{11_10_2022.stz.news.ua.donbas24.1.mrpl_art_rezidencia_postmost_vidnov_dijalnist.pic.1_2}

\subsubsection{Кому належить ідея відновити проєкт?}

Ідея про продовження проєкту належить Олені Украінцевій. У травні вона стала
учасницею резиденції ArToll в Бедбург-Хау (землі Північна Рейн-Вестфалія) разом
з німецькими художниками. Історії про Маріуполь, його людей та художників, які
любили, берегли та покращували своє місто, дуже зворушили їх. Саме тоді Олені
запропонували організувати резиденцію із українськими художниками. Так і
з'явилася можливість реалізації ідеї.

Нагадаємо, що перша резиденція PostMost \enquote{Маріуполь. Місто морське} відбулася у
липні 2021 році і була присвячена Азовському морю. До Маріуполя завітали 9
художників з Києва, Львова та Одеси. Перед художниками стояла одна мета — за 9
днів перебування тут ознайомитися з містом та показати його у своїй фінальній
роботі. Виставка пройшла в арт-просторі Маріуполя \enquote{Гастроном} Центру сучасного
мистецтва \enquote{Континенталь}. 9 резидентів презентували близько 70 робіт. Картини,
фотографії, інсталяції, графічні роботи та навіть мозаїка — і всі вони були
присвячені Маріуполю.

\textbf{Читайте також:} \href{https://donbas24.news/news/mariupolskii-shhodennik-premjera-dokfilmu-ta-kartina-na-amerikanskomu-aukcioni-bonhams}{\emph{Маріупольський щоденник: прем'єра докфільму та картина на американському аукціоні Bonhams}}%
\footnote{Маріупольський щоденник: прем'єра докфільму та картина на американському аукціоні Bonhams, Алевтина Швецова, donbas24.news, 08.10.2022, \par%
\url{https://donbas24.news/news/mariupolskii-shhodennik-premjera-dokfilmu-ta-kartina-na-amerikanskomu-aukcioni-bonhams}%
}

\ii{11_10_2022.stz.news.ua.donbas24.1.mrpl_art_rezidencia_postmost_vidnov_dijalnist.pic.3_5}

\subsubsection{Хто тепер працюватиме в арт-резиденції?}

До проєкту долучилися художниці, які, рятуючись від війни, виїхали до ЄС. До
війни вони вже були учасницями проєкту PostMost. Це \textbf{\emph{українські художниці з
Маріуполя:}}

Олена Украінцева, Антоніна Дурнєва, \href{https://www.facebook.com/art.hilmenna}{Hilmena},%
\footnote{\url{https://www.facebook.com/art.hilmenna}} Simona, \href{https://www.facebook.com/profile.php?id=100083538824017}{Марина Черепченко},%
\footnote{\url{https://www.facebook.com/profile.php?id=100083538824017}}
Зугайрат Новікова; Інна Абрамова (художниця з Донецька) та художниця з
Німеччини — Іветта Бекер, яка дуже підтримує українських біженців.

Темою проєкту залишається Маріуполь (який є малою Батьківшиною художниць) та
загалом вся Україна.

\begin{leftbar}
\emph{\enquote{Звичайно, всі роботи пронизані війною, кожна художниця пише свій особистий
життєвий досвід за останні 7 місяців}}, — підкреслила кураторка
резиденції Олена Украінцева.
\end{leftbar}

\textbf{Читайте також:} \href{https://donbas24.news/news/xudoznik-denis-metelin-dolucivsya-do-blagodiinogo-projektu-zi-zboru-kostiv-dlya-zsu-video}{\emph{Художник Денис Метелін долучився до благодійного проєкту зі збору коштів для ЗСУ}}%
\footnote{Художник Денис Метелін долучився до благодійного проєкту зі збору коштів для ЗСУ, Алевтина Швецова, donbas24.news, 05.09.2022, \par%
\url{https://donbas24.news/news/xudoznik-denis-metelin-dolucivsya-do-blagodiinogo-projektu-zi-zboru-kostiv-dlya-zsu-video}%
}

\ii{11_10_2022.stz.news.ua.donbas24.1.mrpl_art_rezidencia_postmost_vidnov_dijalnist.pic.6_7}

\subsubsection{Де можна буде побачити роботи художниць?}

Роботи презентуватимуть безпосередньо в резиденції ArToll в Бедбург-Хау 22
жовтня. Потім виставка буде перевезена до Дортмунду у шахту-музей промисловості
Zeche Zollern, де 28 жовтня відбудеться презентація експозиції. Там роботи
пробудуть до весни. Також у Дортмунді до виставки додадуться принти робіт
Анастасії Шишкіної та \href{https://www.facebook.com/profile.php?id=100007568640200}{Олександра Малаховського},%
\footnote{\url{https://www.facebook.com/profile.php?id=100007568640200}} які перебувають наразі в
Україні та не мають можливості особисто взяти участь у проєкті.

\ii{11_10_2022.stz.news.ua.donbas24.1.mrpl_art_rezidencia_postmost_vidnov_dijalnist.pic.8_10}

Раніше Донбас24 розповідав, як у Німеччині створюються \href{https://archive.org/details/06_09_2022.olga_demidko.donbas24.futbolki_nimecchyna_mrpl}{\emph{футболки, присвячені Маріуполю}}.%
\footnote{Як у Німеччині створюються футболки, присвячені Маріуполю — деталі, Ольга Демідко, donbas24.news, 06.09.2022, %
\par\url{https://donbas24.news/news/yak-u-nimeccini-stvoryuyutsya-futbolki-prisvyaceni-mariupolyu-detali}, \par%
Internet Archive: \url{https://donbas24.news/news/yak-u-nimeccini-stvoryuyutsya-futbolki-prisvyaceni-mariupolyu-detali}%
}

Ще більше новин та найактуальніша інформація про Донецьку та Луганську області
в нашому телеграм-каналі Донбас24.

ФОТО: з особистого архіву Олени Украінцевої.

\ii{insert.author.demidko_olga}
%\ii{11_10_2022.stz.news.ua.donbas24.1.mrpl_art_rezidencia_postmost_vidnov_dijalnist.txt}
