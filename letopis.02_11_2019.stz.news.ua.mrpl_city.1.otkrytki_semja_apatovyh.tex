% vim: keymap=russian-jcukenwin
%%beginhead 
 
%%file 02_11_2019.stz.news.ua.mrpl_city.1.otkrytki_semja_apatovyh
%%parent 02_11_2019
 
%%url https://mrpl.city/blogs/view/starinnye-otkrytki-mariupolskoj-semi-apatovyh
 
%%author_id burov_sergij.mariupol,news.ua.mrpl_city
%%date 
 
%%tags 
%%title Старинные открытки мариупольской семьи Апатовых
 
%%endhead 
 
\subsection{Старинные открытки мариупольской семьи Апатовых}
\label{sec:02_11_2019.stz.news.ua.mrpl_city.1.otkrytki_semja_apatovyh}
 
\Purl{https://mrpl.city/blogs/view/starinnye-otkrytki-mariupolskoj-semi-apatovyh}
\ifcmt
 author_begin
   author_id burov_sergij.mariupol,news.ua.mrpl_city
 author_end
\fi

\ii{02_11_2019.stz.news.ua.mrpl_city.1.otkrytki_semja_apatovyh.pic.1}

История дореволюционного периода комбината имени Ильича, благодаря трудам
прежде всего - Д. Н. Грушевского, Р. И. Саенко, Л. Д. Яруцкого, Ю. Я. Некрасовского
и других авторов, представлена довольно подробно. Известны учредители \enquote{Никополь
- Мариупольского горного и металлургического общества} и \enquote{Русского Провиданса},
даты всех организационных этапов и этапов строительства, перечень цехов новых
заводов, построенных возле станции \enquote{Сартана} Екатерининской железной дороги,
периоды кризисов и стачечная борьба рабочих за свои права. А каков был быт
жителей?

В известной мере быт обитателей поселков с почтовым адресом – станция \enquote{Сартана}
Екатеринославской губернии - приоткрыл альбом со старинными открытками. \textbf{Борис
Леонидович Котлярчук}, к сожалению, рано ушедший из жизни, высококлассный
инженер, филофонист и историк отечественной эстрады, собрал в альбом почтовые
открытки, которыми обменивались в свое время его предки. Значительная их часть
имела отношение к его бабушке - \textbf{Александре Васильевне Апатовой}, в девичестве
Митрофановой. Именно она и складывала в какую-то заветную шкатулку милые сердцу
послания из ее молодости.

\textbf{Читайте также:} 

\href{https://mrpl.city/news/view/pervaya-v-mariupole-zhenshhina-voditel-avtobusa-ezdit-na-novom-marshrute-foto-plusvideo-pluspanorama-360}{% 
Первая в Мариуполе женщина-водитель автобуса ездит на новом маршруте, Богдан Коваленко, mrpl.city, 01.11.2019}

Александра Васильевна родилась в многодетной семье Василия и Анны Митрофановых.
Шурочка Митрофанова была очень миловидной девушкой. И на вечеринках, которые
устраивала молодежь колонии завода \enquote{Никополь}, у нее не было отбоя от
кавалеров. Вот признания во время игры в фанты: \enquote{Милый и симпатичный друг
Шурочка, я умираю, когда вы начинаете танцевать}, \enquote{Ваша улыбка точь-в-точь, как
здесь на открытке}, \enquote{Мадемуазель Митрофановой. Вы сегодня царица вечера! Только
на вас одной отдыхает взор. Своей красотой вы подчеркиваете безобразие других},
\enquote{Ваше кокетство доводит меня до безумия}, \enquote{Шура, вы сегодня интересная, я прямо
в вас влюбился, но, наверное, взаимностью пользоваться не буду}.

Но Александра Васильевна, Шурочка, предпочла всем \textbf{Ивана Ивановича Апатова}. Судя
по содержанию текстов на открытках, Иван Иванович работал в механическом отделе
завода \enquote{Никополь}. Он был специалистом своего дела, хорошо зарабатывал. Это
видно по нарядам его жены. На дошедшей до наших дней фотографии Александра
Васильевна запечатлена в шикарной шляпе, воротник ее пальто почти полностью
закрывает грудь, а в руках у нее муфта. И все это из дорогого в те отдаленные
годы меха бобра...

\ii{02_11_2019.stz.news.ua.mrpl_city.1.otkrytki_semja_apatovyh.pic.2}

Ивану Ивановичу приходилось бывать по делам в других городах. Иногда подолгу. В
альбоме есть открытка, направленная ему в Екатеринослав (нынешний Днепр), когда
он был по каким-то неотложным делам на Брянском заводе, теперь это -
Нижнеднепровский трубопрокатный завод. Иван Иванович знал, во всяком случае,
мог читать и понимать польский язык. Тому свидетельство послания, написанные
ему на этом языке. Какое-то время Апатов с женой и дочуркой \textbf{Людой} жил в селе
Родники Костромской губернии, где работал в механическом отделении текстильной
фабрики Красильщиковой. Именно из этого населенного пункта Российской империи
направила на станцию \enquote{Сартана} Екатеринославской губернии Александра Васильевна
поздравление с Рождеством Христовым и наступающим 1917 годом своим родителям.
На открытке стоит штемпель почтового отделения-отправителя - \enquote{\em 20 декабря 1916
года}.

Кстати, именно почтовые штемпели позволили определить период времени, когда
Александра Васильевна и ее близкие обменивались открытками – с октября 1908
года по декабрь 1917 года. Между прочим, эти же штемпели дают возможность
оценить работу почты в царской России. Так открытка, отправленная 11 декабря
1910 года из Тифлиса, прибыла в почтовое отделение станции \enquote{Сартана} 16 числа
того же месяца; весточка из Харькова от 16 марта 1915 года была в Сартане на
следующий день; послание из Екатеринослава (ныне Днепр), отмеченное 24 декабря
1915 года, было в Сартане уже 26 декабря. Чтобы не вдаваться в подробности,
сообщим, что почтовые отправления приходили в Сартану из г. Темникова
Тамбовской губернии (ныне Мордовии) на 9-й день, из Кисловодска – на 4-й, из
Москвы – также на 4-й, из Риги на 3-й, из Харькова – на 2-й день. Если учесть,
что большинство открыток отсылались с поздравлениями к Рождеству Христову и к
Пасхе, когда почта бывала максимально загружена, можно сказать, что результат
ее работы был весьма хорош.

\textbf{Читайте также:} \href{https://mrpl.city/blogs/view/vsmatrivayas-v-otkrytki}{%
Всматриваясь в открытки, Сергей Буров, mrpl.city, 01.04.2018}

Открытки, сохраненные Борисом Леонидовичем Котлярчуком, интересны не только
надписями, начертанными на них. Не менее интересны они сами по себе. На них
отображены рождественские и пасхальные сюжеты, несколько слащавые женские
портреты, а также пейзажи. А еще виды Москвы, Петербурга, Харькова, Риги, Ялты
и других городов. Есть и репродукции картин знаменитых художников, русских и
зарубежных. Но это уже другая тема.
