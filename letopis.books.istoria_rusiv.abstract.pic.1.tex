% vim: keymap=russian-jcukenwin
%%beginhead 
 
%%file books.istoria_rusiv.abstract.pic.1
%%parent books.istoria_rusiv.abstract
 
%%url 
 
%%author_id 
%%date 
 
%%tags 
%%title 
 
%%endhead 


\ifcmt
  tab_begin cols=2,no_fig,center

     pic https://2.bp.blogspot.com/-gfA3apJcpAg/USXYq49qmWI/AAAAAAAAFvQ/21CaHWP956w/s320/%D0%91%D0%BE%D0%B4..bmp
     @caption_begin
Осип Максимович Бодянский (1808-1877). Рисунок на дереве исполнен К. О. Брожем
(Брож, Карл Осипович (1836-1901), ксилограф и рисовальщик) с фотографии
сделанной за несколько лет до смерти О.  Бодянского. Гравировал портрет в Ницце
академик-гравер Его Величества Л. А. Серяков.
     @caption_end

     pic https://art16.ru/gallery2/d/80001-17/DSC06944.jpg
     @caption_begin
Бишбуа Л. П. А. (1801-1850), Дитц С. Ф. (1803-1873). Лубянская площадь

из серии \enquote{Виды Москвы}, 1846; бумага, литография, акварель;

Лубянская площадь. Впервые слово Лубянка упомянуто в летописи в 1480 году,
когда Иван III приказал новгородцам, выселенным в Москву после падения
республики, селиться в этом районе и именно они назвали этот район Лубянкой, в
честь Лубяниц - района Новгорода.

     @caption_end

  tab_end
\fi
