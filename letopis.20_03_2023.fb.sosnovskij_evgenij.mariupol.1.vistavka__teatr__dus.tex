%%beginhead 
 
%%file 20_03_2023.fb.sosnovskij_evgenij.mariupol.1.vistavka__teatr__dus
%%parent 20_03_2023
 
%%url https://www.facebook.com/evgeny.sosnovsky/posts/pfbid0r5oFy6mqSk7GPpbLRW78HuHJ8uZiQc9bQwQ4P4hvLvHMmggu37zTrr5bwXWbTcMnl
 
%%author_id sosnovskij_evgenij.mariupol
%%date 20_03_2023
 
%%tags mariupol,vystavka,teatr.mariupol.drama,mariupol.war,mariupol.pre_war,kultura,kiev,teatr.kiev.franka
%%title Виставка "Театр. Душа Маріуполя" у київському театрі iм. Івана Франка
 
%%endhead 

\subsection{Виставка \enquote{Театр. Душа Маріуполя} у київському театрі iм. Івана Франка}
\label{sec:20_03_2023.fb.sosnovskij_evgenij.mariupol.1.vistavka__teatr__dus}

\Purl{https://www.facebook.com/evgeny.sosnovsky/posts/pfbid0r5oFy6mqSk7GPpbLRW78HuHJ8uZiQc9bQwQ4P4hvLvHMmggu37zTrr5bwXWbTcMnl}
\ifcmt
 author_begin
   author_id sosnovskij_evgenij.mariupol
 author_end
\fi

Виставка \enquote{Театр. Душа Маріуполя} у київському театрі iм. Івана Франка.

В експозиції представлені унікальні архівні фотоматеріали – від періоду з
відкриття маріупольського Драмтеатру до його знищення рашистами.

Центральне місце займає 3D модель будівлі театру, виконана з елементів
конструктору LEGO Максимом Мітякіним, представником спільноти \enquote{УкрБрікс}, на
основі дизайну Олександра Ермолаєва. Проєктування моделі відбувалось в рамках
проекту МІНІМІСТО, у якому лего-архітектори відтворюють знищені рашистами
об'єкти культурної спадщини.

Київ, 19 березня 2023 року

%\ii{20_03_2023.fb.sosnovskij_evgenij.mariupol.1.vistavka__teatr__dus.cmt}
