% vim: keymap=russian-jcukenwin
%%beginhead 
 
%%file 05_04_2021.fb.romancha_viktoria.1.svet_glaza_prozrenie
%%parent 05_04_2021
 
%%url https://www.facebook.com/groups/420387464970776/permalink/1459031437773035/
 
%%author 
%%author_id 
%%author_url 
 
%%tags 
%%title 
 
%%endhead 

\subsection{УВАГА. Акція "ПОГЛЯНЬТЕ НАМ У ВІЧІ"}
\label{sec:05_04_2021.fb.romancha_viktoria.1.svet_glaza_prozrenie}
\url{https://www.facebook.com/groups/420387464970776/permalink/1459031437773035/}

УВАГА. Акція "ПОГЛЯНЬТЕ НАМ У ВІЧІ". 5 квітня 2021 р. у Києві, за адресою: узвіз Печерський, 3, починаючи з 08:00 год. відбудеться всеукраїнська акція протесту проти діяльності голови всеукраїнської громадської організації сліпих УТОС В. М. Більчича, його найближчого оточення та президії центрального правління УТОС. Саме представники вищого органу правління, узурпувавши владу центрального правління, недолуго здійснюють керівництво всією системою УТОС. Саме вони, прикриваючись людьми з інвалідністю по зору дбають лише про лобіювання своїх інтересів та збільшення власних статків в той час, коли Наша громадська організація знаходиться у глибокому занепаді.


\ifcmt
  pic https://scontent-ams4-1.xx.fbcdn.net/v/t1.6435-9/169228263_808986349824975_7473764972540547621_n.jpg?_nc_cat=104&ccb=1-3&_nc_sid=8bfeb9&_nc_ohc=G7TwEKPvlgsAX_2IRrF&_nc_ht=scontent-ams4-1.xx&oh=5c76dbf3d2fb8b0a2fc4b9257683f9d0&oe=608F8288
  width 0.4
\fi


       В акції беруть участь директори підприємств, представники територіальних первинних організацій, члени УТОС з усієї України, яким вже просто остогидла існуюча влада організації.
       МИ ВИМАГАЄМО:
- Зміни керівництва;
- Проведення незалежного аудиту фінансової діяльності центрального правління та інших структурних підрозділів УТОС;
- Забезпечення реалізації дієвої та прозорої державної фінансової підтримки;
- Формування належної та відповідальної кадрової політики на зайняття керівних посад;
- Здійснення перспективного та адаптивного планування розвитку УТОС в умовах сьогодення;
- Налагодження активної та ефективної взаємодії з вищими органами влади та органами місцевого самоврядування.
      НАГОЛОШУЄМО: Зазначені вимоги є найбільш актуальними, серед цілого ряду питань, реалізація яких вже давно назріла.  Адже важко уявити, що отримуючи державну фінансову підтримку близько 49 млн. грн. щорічно, існуюча верхівка керівних органів ЦП УТОС спромоглася довести саму організацію та членів УТОС до зубожіння. 
За відсутності стратегічного планування та належної і пропорційної фінансової підтримки, колись потужні промислові підприємства знижують темпи та обсяги свого виробництва, а деякі підприємства вже взагалі зупинили свою діяльність. Будівлі підприємств, будинки культури, будинки відпочинку, бібліотеки, гуртожитки, які також є власністю  УТОС, знаходяться у жахливому стані, а незрячим людям, членам організації доводиться жити і працювати в умовах холоду та плісняви, виконуючи рабську роботу за мізерну зарплатню. Натомість, ЦП УТОС формується та масштабно підтримується політика не розвитку, а так званого "проїдання", за якої перспективним напрямком, є лише напрямок здавання існуючих приміщень системи в оренду, а не використання їх за призначенням.          ДАЛІ МОВЧАТИ НЕ МОЖНА. Керівництво системи УТОС потребує перезавантаження, шляхом його повного оновлення. Ми всі рівні в наданих нам правах.
ДОСИТЬ МАНІПУЛЮВАТИ НЕЗРЯЧИМИ ТА СЛАБОЗОРИМИ ЛЮДЬМИ. Досить створювати видимість того, що УТОС піклується про своїх членів товариства, надаючи різноманітний спектр соціальних послуг, забезпечуючи їх моральний, психологічний, спортивний та духовний розвиток. Досить вищому керівництву УТОС жити за власними
законами, обіймати по дві керівні посади одразу та здійснювати інші зловживання. Давайте поважати організацію, яка існує з 1933 р., разом думати, надаючи підтримку залученню в УТОС молодих членів організації, піклуючись про їх інтелектуальний розвиток, соціальну та матеріальну
підтримку, створювати заходи для освітнього та інтелектуального розвитку та врешті решт створювати нові робочі місця, а не місця для оренди. Ми здатні виготовляти не лише прищіпки та монотонно перебирати шурупи. Ми здатні на більше та заслуговуємо на гідне життя.
      Просимо всіх, хто воліє допомогти незрячим та слабозорим людям у захисті їхніх прав та вимог.
МИ ПРОСИМО ВСІХ НЕБАЙДУЖИХ - громадський сектор, медіа, активістів, представників влади і всіх небайдужих громадян. ДОПОМОЖІТЬ. Нам нам дуже потрібна ваша допомога, підтримка й розуміння. ПОГЛЯНЬТЕ НАМ У ВІЧІ.
Володимир Зеленський, Дмитро Разумков
, Денис Шмигаль
, Арсен Аваков
, Sergii Marchenko
, Галина Третьякова
,
Marina Lazebna
, Віталій Кличко
, Валерій Сушкевич
, Татьяна Баранцова
.
