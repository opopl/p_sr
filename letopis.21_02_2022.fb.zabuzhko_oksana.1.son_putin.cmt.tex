% vim: keymap=russian-jcukenwin
%%beginhead 
 
%%file 21_02_2022.fb.zabuzhko_oksana.1.son_putin.cmt
%%parent 21_02_2022.fb.zabuzhko_oksana.1.son_putin
 
%%url 
 
%%author_id 
%%date 
 
%%tags 
%%title 
 
%%endhead 
\zzSecCmt

\begin{itemize} % {
\iusr{Oleksandr Bodak}

Цей епізод був у вашій книзі Let my people go. Але там, ви, здається, описували
сон з 2003 року, коли Путін вже президентом був.

\begin{itemize} % {
\iusr{Oksana Zabuzhko (Оксана Забужко)}
\textbf{Oleksandr Bodak} 

справді? от, таки треба буде перевірити дату - але згадала я його через
\enquote{простирадла}, які тільки тепер стали остаточно зрозумілими(((

\iusr{Людмила Савчак}
\textbf{Oksana Zabuzhko (Оксана Забужко)} а може в тих простирадлах їх втопили ??? І у сні і наяву!!!

\iusr{Oleksandr Bodak}
\textbf{Oksana Zabuzhko (Оксана Забужко)} знайшов. У книзі написано, що це запис від 16 травня 2004 року.

\begin{itemize} % {
\iusr{Людмила-Ладаслава Лобачова}
\textbf{Oleksandr Bodak} о, знаю, що візьму почитати у бібліотеці

\iusr{Орелі Боян}
\textbf{Oleksandr Bodak} чим сон скінчився?

\iusr{Oleksandr Bodak}
\textbf{Орелі Боян} там тільки те, що пані Оксана вже написала.

\iusr{Орелі Боян}
\textbf{Oleksandr Bodak} дякую.
\end{itemize} % }

\iusr{Олена Вековець}
А мені він в ті роки другом України увлявся, братні народи- оце все. Бррр..

\begin{itemize} % {
\iusr{Tatiana Vitalievna Lipnitska}
\textbf{Олена Вековець} 

а в ті роки Україна була майже такою ж бандитською як і Московія. То таки ми
були братські народи, але лише настільки, наскільки дві мафіозні сім'ї можуть
бути братніми  @igg{fbicon.shrug} 

\iusr{Iryna Semenova}
\textbf{Олена Вековець} 

с самого початку його ненавиділа, хоча багатьом здавався інтелектуалом...
благо, мене давно навчили: немає народів братів, є мирне співіснування

\iusr{Галина Бут}
\textbf{Олена Вековець} Яке щастя, що наш народ так змінюється !

\iusr{Михайло Пмв}
\textbf{Олена Вековець} Чому?

\iusr{Сергій Кулик}
\textbf{Олена Вековець} братні? Що тоді було під Крутами?

\iusr{Олена Вековець}
\textbf{Сергій Кулик} та не тільки під Крутами.
\end{itemize} % }

\end{itemize} % }

\iusr{Potaha Tsuaruko}

Ой а мне сон приснился когда майдан шёл в 2014 году: приезжаю я к себе домой в
Горловку из Парижа, а дома никого нет и такая зловещая тишина... Я на кухню чай
сделать, а в чайнике кровь. И в кране кровь. И в ванной и даже в туалете -
везде кровь...

Сегодня у меня праздник. Сегодня я верю, что в моем доме у меня на Родине
наконец-то перестанет литься кровь моих собратьев... Слава Богу за все!!!

\begin{itemize} % {
\iusr{Natalia Chmil}
\textbf{Potaha Tsuaruko} 

зараз не в складі України окремі райони Донецької і Луганської областей. Але
вони заявляють претензії на області повністю. Ви ці претензії підтримуєте?

\begin{itemize} % {
\iusr{Potaha Tsuaruko}
\textbf{Natalia Chmil} 

я только за то, чтобы прекратилась война братоубийственная и чтобы
русскоязычное население не угнеталось. Чтобы Запад не использовал украинский
народ против своих русских братьев. Чтобы из украинских детей не взращивали
врагов России. Я за Мир...

\iusr{Ніна Бажура}
\textbf{Potaha Tsuaruko} ви українка?

\iusr{Potaha Tsuaruko}
\textbf{Natalia Chmil} по паспорту?

\iusr{Natalia Chmil}
\textbf{Potaha Tsuaruko} я задала просте питання. Чому ви не можете відповісти?

\iusr{Oksana Kravets}
\textbf{Natalia Chmil} вона вам вже відповіла. Вона радіє за.... Росію. Все. То яка вона українка?

\iusr{Natalia Chmil}
\textbf{Oksana Kravets} 

вона за мір, і щоб не лилася кров. і радіє, що в ОРДЛО тепер Росія. Я просто
уточнюю: хоче відвоювати підконтрольну Україні частину донецької і Луганської
областей, як офіційно хоче динири-линири, чи вона проти офіційної позиції.
Просте запитання.

\iusr{Oksana Kravets}
\textbf{Natalia Chmil} то для вас просте, а для неї - ні. Воно ж за \enquote{мір, дружба, жвачка}.

\iusr{Natalia Chmil}
\textbf{Oksana Kravets} не відповідайте за когось. Людина має визначитись: за мір чи за офіційну позицію?

\iusr{Potaha Tsuaruko}

Не спорьте. За кого я - я определилась еще в 2014 году. До этого всегда считала
себя укаринкой. Пока Укараина была одной из стран большого русского мира. Вы
сами подтолкнули нас сделать выбор не в вашу пользу противопоставив все
украинское - русскому. Надеюсь, я ответила на ваш вопрос. Дисскуссия окончена.

Всем добра!

\iusr{Mashinskaya Anna}
\textbf{Natalia Chmil} 

Ну так, росіюшку любить, це видно. А от наскількі щирі оці туркотіння за мир,
цікаво. Але - я думаю це черговий бот, приставлений до сторінки Оксани
Стефанівни. Відповідно, ніякого діалогу не вийде. Просто буде користувати
популярний майданчик, щоб знов і знов срати в голови роспропагандою.

\iusr{Potaha Tsuaruko}

Анна, зайдите на мою страничку и вы увидите, что бот зарегестрирован с 2009
года и активно жил вне политики)

\iusr{Ольга Истомина}
\textbf{Mashinskaya Anna} 

не бот, але в голові одна російська пропаганда, оце її \enquote{большой русский
мир} все пояснює.

\end{itemize} % }

\iusr{Oksana Kravets}
\textbf{Natalia Chmil} ок

\iusr{Иван Василенко}
\textbf{Potaha Tsuaruko} 

то что у вас рашизм головного мозга это ясно, без уточнений! Вопрос только в
том, что ж вы в Италию збежали от Евромайдана?

\iusr{Potaha Tsuaruko}

Иван, вообще я не люблю отвечать на хамство, но ввиду ситуации я стараюсь
понять ваши эмоции и поэтому отвечу в этот раз. Я не от кого не бежала, а
уехала в Европу учиться еще в 2006 году. Прожила во Франции 11 лет, работая с
Германией и др. Европейскими странами, до того как мой будущий муж-итальянец
увёз меня к себе в Италию. У меня здесь семья. И я сейчас помогаю итальянским
компаниям-производителям работать с внешним рынком, так как разговариваю на 6ти
языках. С Украинским и Российским в том числе. Еще есть вопросы?)

\iusr{Potaha Tsuaruko}
\textbf{Ольга Истомина}, 

я имею возможность черпать информацию на 6ти языках, чтобы определиться со
своей позицией. Вы можете такое сказать о себе, к примеру? Или вы
ограничиваетесь укр. Новостями?)

\end{itemize} % }

\iusr{Тушті Лаліта}

А мені страшний сон снився теж років 20 тому, як злі клоуни заманили людей у
пастку і примушували стрибати з даху багатоповерхівки, надягаючи їм клоунські
маски, ніби це весело. Все відбувалося у вигляді розваги для натовпу, що
спостерів унизу. Ті, кого примушували стрибати, полотніли від жаху, але на них
були маски веселих клоунів. Мене теж збиралися скинути, але я врятувалася. Я
кричала уві сні. Досі пам'ятаю.

\begin{itemize} % {
\iusr{Андрей Аверьянов}
\textbf{Тушті Лаліта}

Так це ж сон не про Путіна, а про Зеленського та Коломойського. Згоден,
тлумачити сни то нелегка справа))

\iusr{Тушті Лаліта}
\textbf{Andrew Averianov} так, про нашого Клована, але теж пророчий.
\end{itemize} % }

\iusr{Olena Basistyuk}

А мені, як мали забрати Крим приснився павук, що лазив по килиму, а потім в
лапках в нього опинилася червона трикутна шматка, як піонерський галстук. А
павук був з головою рос. президента, ще малюнки такі були...( потім я
зрозуміла.. Крим має форму трикутника) А на сьогодні нічого не снилась, тільки
все якісь незнайомі чоловіки групками, в костюмах і в розмовах...

\begin{itemize} % {
\iusr{Галина Куліш}
\textbf{Olena Basistyuk} про все домовлено зарані. Всі лідери про це знали. Кожен собі щось виторгував.

\iusr{Nataliya Kurt}
\textbf{Галина Кулішза} тупість блок. Остогидли ви куріци.

\iusr{Ірина Андрусевич}
\textbf{Olena Basistyuk} сьогодні над Києвом була веселка
\end{itemize} % }

\iusr{Олена Тесленко-Мельничук}

Мені також декілька років назад сон наснився. Неначе стою я з якимись людьми, а
перед нами видно воду, яка наче насувається. І ми тривожимось, знаємо, що буде
повінь. Я кажу, що це ж Москва-ріка несеться. І я пам'ятаю, що було це тривожне
відчуття очікування, що от-от... Її було видно, але до повені не дійшло. Я
прокинулась і тоді ще думала, що росія попре. Зараз маленька надія, що, може,
таки до \enquote{повені} не дійде...

\begin{itemize} % {
\iusr{Tana Istama}
\textbf{Olena Teslenko-Melnichuk} вісім років гинуть наші воїни, тому повінь Є

\iusr{Вікторія Кашаюк}
\textbf{Olena Teslenko-Melnichuk} дійде(
\end{itemize} % }

\iusr{Инна Далина}

Теж у тіх літах (2004) мені теж снився раптово сон про пу. Начебто йду я по
червоній ковровій доріжці в кремлі (ніколи там не була в реалі) і розумію, що
це головний коридор до президентського кабінету. Поруч зі мною їде чоловік,
заходимо до кабінета, там сидить пу спиною до нас, обходимо його і я бачу, що
то не людина, а дуже реалістичний манекен. Чоловік знімає половину голови пу і
каже мені \enquote{бачиш? Що ми положимо туди, те й він зробе} і я розумію хто стоїть
поряд мене.

А ви кажете.. логіка, плани.

\iusr{Ігор Воронюк}
\textbf{Irina Dalina} 

Десь близько до вашого сенсу, мені Ху снився з мішком на голові, що не він
керує своїми діями. Є жінка яка направляє його.

\iusr{Натали Бордюг}

Якщо про сни - то мені сьогодні привидівся великий кровавий повний місяць (до війни)
Слава Україні))))

\begin{itemize} % {
\iusr{Галина Стефюк}
\textbf{Natali Bordyug} На сьогодні сни не збуваються.

\iusr{Натали Бордюг}
\textbf{Галина Стефюк} якби ще і війна була б не справжня)))

\iusr{Halia Karatushyna}
\textbf{Natali Bordyug} а чому ви смієтеся?

\iusr{Натали Бордюг}
Хіба це смішно ? вже не до сміху)
\end{itemize} % }

\iusr{Alexander Kovalevski}

2014 - го року, коли довелося покинути Крим, теж снився сон. Наче в небі над
морем з'явився російський прапор і почав звучати гімн СРСР - РФ. Гімн лунав все
гучніше, прапор все збільшувався і вже майже закрив все небо. Але якоїсь миті з
неба виникли дві здоровезні руки і подерли в дрібні клаптики той прапор і
покидали їх у море. І гімн той якось геть стих.

\iusr{Наталя Бурчак}
Дай Боже, щоб кінець вашого сну справдився! Слава нашим воїнам!

\iusr{Іра Кіщак}
Де сяде - там злізе. Є приказка така.
Боже, храни Україну!
***
Тарасові слова: \enquote{І на оновленій землі...}, \enquote{Як понесе з України
у синєє море...} Вона витримає, з нами Бог!


\end{itemize} % }
