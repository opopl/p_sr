% vim: keymap=russian-jcukenwin
%%beginhead 
 
%%file 27_07_2020.fb.lnr.1
%%parent 27_07_2020
 
%%endhead 

\subsection{1 --- Военные ДНР и ЛНР получили приказы о соблюдении режима тишины}
\label{sec:27_07_2020.fb.lnr.1}
\url{https://www.facebook.com/groups/LNRGUMO/permalink/2880785172033038/}

\index{ФБ!Группы!Луганская Народная Республика}
\index{LNRGUMO}

Военные самопровозглашенных Донецкой и Луганской народных республик (ДНР и
ЛНР) получили приказы о соблюдении режима тишины в связи с новым соглашением о
прекращении огня. Ранее участники Контактной группы утвердили пакет
дополнительных мер усиления и контроля действующего перемирия. С полуночи 27
июля в силу должны вступить в силу соответствующие приказы военного
руководства ДНР, ЛНР и Украины.

С 00:01 запрещено использование любого вооружения, включая стрелковое оружие,
сообщили в ДНР. В ЛНР заявили о введении дополнительных мер по усилению режима
прекращения огня.

К дополнительным мерам, в частности, относятся запрет наступательных и
разведывательно-диверсионных действий, запреты на использование любых видов
летательных аппаратов на применение огня, включая снайперский, на размещение
тяжелого вооружения в населенных пунктах и окрестностях. За нарушения режима
тишины вводятся дисциплинарные меры, планируется создание координационного
механизма для реагирования на нарушения.

Как пояснила “Ъ” глава миссии ДНР на переговорах в Минске Наталья Никонорова,
новым решением о прекращении огня документ назвать нельзя. «У нас с 21 июля
2019 года действует бессрочное перемирие, его никто не отменял»,— сказала она.
