% vim: keymap=russian-jcukenwin
%%beginhead 
 
%%file 09_10_2021.fb.fb_group.story_kiev_ua.1.zanjkoveckoj_8.cmt
%%parent 09_10_2021.fb.fb_group.story_kiev_ua.1.zanjkoveckoj_8
 
%%url 
 
%%author_id 
%%date 
 
%%tags 
%%title 
 
%%endhead 
\subsubsection{Коментарі}
\label{sec:09_10_2021.fb.fb_group.story_kiev_ua.1.zanjkoveckoj_8.cmt}

\begin{itemize} % {
\iusr{Наташа Котляренко}
А не во дворе этого дома снимали фильм "восток-запад"?

\begin{itemize} % {
\iusr{Анна Солдатова}
\textbf{Наташа Котляренко} да, там.

\iusr{Антонина Гурская}
\textbf{Наташа Котляренко} Ні, це було по Лютеранська 7/10 (10 номер по Заньковецькій) та ще частинку по Лютеранській вниз в бік Хрещатику. Ми тоді там жили, а нам забороняли у вікна виглядати.
\end{itemize} % }

\iusr{Владимир Глазков}
Сталінський (радянський) ампір.
Impire.

\begin{itemize} % {
\iusr{Анатолий Борозенец}
Так, він.

\iusr{Tetiana Samoilovych}
\textbf{Володимир Глазков} Отличные здания построили

\begin{itemize} % {
\iusr{Alexey Novozhylov}
\textbf{Tetiana Samoilovych} цікаве у вас розуміння отлічного... @igg{fbicon.beaming.face.smiling.eyes} 

\iusr{Владимир Глазков}
\textbf{Tetiana Samoilovych} дякую але в принципі за точністю не гнався

\iusr{Tetiana Samoilovych}
\textbf{Alexey Novozhylov} Так у кожного своэ уявлення про мiсце, де б вiн хтiв мати свiй дiм. Як на мене, то цегла й гарнi за розмiром кiмнати, це - ознака якостi

\iusr{Alexey Novozhylov}
\textbf{Tetiana Samoilovych} так, але приємніше здається ходити вулицями з якимось більш менш притомним ансамблем, щоб голова від такого ампіру не боліла.

\iusr{Tetiana Samoilovych}
\textbf{Alexey Novozhylov} тодi, мабуть, приэмнiше буде жити та ходити мiж панельними багатоповерхiвками у робiтничих районах без гарячоi води, опалення та з смiттям навкруги

\iusr{Alexey Novozhylov}
\textbf{Tetiana Samoilovych} зовсім ні, маю на увазі цивілізований світ, якийсь бульвар Saint-Germain, наприклад, або Mariahilfestrasse у Відні, або шо завгодно інше. Ви не втомлюйтесь шпацеруючи там, на відміну від цього совкового лайна.  @igg{fbicon.smile} 

\iusr{Tetiana Samoilovych}
\textbf{Alexey Novozhylov} Щось у це лайно тягне багатьох з цивiлiзованого свiту за грошима

\iusr{Alexey Novozhylov}
\textbf{Tetiana Samoilovych} ну, ми ж не такі, наші натури набагато витонченіше. @igg{fbicon.face.tears.of.joy}{repeat=3} 

\iusr{Tetiana Samoilovych}
\textbf{Alexey Novozhylov} Щось менi здаэться, що навряд чи, якщо так смiшно
\end{itemize} % }

\iusr{Evgenia Tymchenko}
\textbf{Володимир Глазков} 

Empire, бо em читається як носове а, тобто ам, а im читається французькою як
носове е, тобто ем.

\end{itemize} % }

\iusr{Serhii Yushko}
Краще б натомість театр відбудували тоді, а не будували це...

\iusr{Валентина Багинская}

\ifcmt
  ig https://scontent-mxp1-1.xx.fbcdn.net/v/t39.1997-6/s168x128/106015589_1000496863739351_8977234205464927799_n.png?_nc_cat=1&ccb=1-5&_nc_sid=ac3552&_nc_ohc=kPXjbi_Xt5sAX9rb6-M&_nc_oc=AQkjte5-UaGlJjjQk_28HGOh0Fs-cOoahqhBTH-IfoY-0RrQ9vimeowBnaJy3n2hfDM&_nc_ht=scontent-mxp1-1.xx&oh=6fb8882197a3f13c5c42e8a492468e3d&oe=6195678C
  @width 0.1
\fi

\iusr{Наталія Громова}
Ви так все цікаво описали, Дякую.

\iusr{Анатолий Борозенец}
\textbf{Наталія Громова} Будь ласка.

\iusr{Maryna Oumans'ka}

Будівля ресторану "Метро" значно красивіша за той будтинок нагорі. Оригінальна
і класна. А будинок №8 по Заньковецькій - банальний, хоч і, певна, зручний для
мешканців. А в тому театрі нf початку 20-х років працювала моя молода любима
баба - актрисою.

\begin{itemize} % {
\iusr{Анатолий Борозенец}

Не погоджусь ні щодо "значно красивіша", ні щодо "оригінальності" будівлі
ресторану. Та ще й перекрив простір з гарним краєвидом, який там планувався.
Проте, як кажуть, на колір і смак товариш не всяк. Наші оціночні думки можуть і
не співпадати. Це нормально )

\end{itemize} % }

\iusr{Наталія Кравчук}
Дякую, дуже цікаво))

\iusr{Анатолий Борозенец}
Будь ласка!

\iusr{Нина Дубчак}

Мое детство! В этом доме жила старшая сестра моей бабули. Как часто я там
бывала... Особенно любила рассматривать ордена и медали... Все мои дедушки
летчики, прошли войну и имели высокие звания и награды, а дедушка Валя и
бабушка Люся (старшая сестра моей бабули) жили в этом доме. Как давно это было
и как недавно ...

\end{itemize} % }
