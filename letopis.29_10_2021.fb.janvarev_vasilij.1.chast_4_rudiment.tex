% vim: keymap=russian-jcukenwin
%%beginhead 
 
%%file 29_10_2021.fb.janvarev_vasilij.1.chast_4_rudiment
%%parent 29_10_2021
 
%%url https://www.facebook.com/permalink.php?story_fbid=4501326519943920&id=100001998494582
 
%%author_id janvarev_vasilij
%%date 
 
%%tags cennosti,humanity,obschestvo,vojna
%%title Часть IV Рудимент
 
%%endhead 
 
\subsection{Часть IV Рудимент}
\label{sec:29_10_2021.fb.janvarev_vasilij.1.chast_4_rudiment}
 
\Purl{https://www.facebook.com/permalink.php?story_fbid=4501326519943920&id=100001998494582}
\ifcmt
 author_begin
   author_id janvarev_vasilij
 author_end
\fi

Часть IV Рудимент

С началом перехода человечества к постиндустриальному обществу идеи
завоевательной войны в развитых странах стали стремительно терять свою
популярность. Более того, колониальные страны начали отказываться от своих
заморских владений и в первую очередь по исключительно экономическим причинам.
Если раньше в колониях можно было заработать (хотя профит и постоянно
снижался), то после Второй мировой войны метрополии прямо дотировали колонии.
Например, за первые десять послевоенных лет государственные расходы и в
британских и французских колониях в Африке выросли в четыре раза, при резком
многократном росте их доли на инфраструктурные проекты и образование. И сегодня
территории в свое время не поспешившие прогнать колонизаторов, экономически
более развиты, чем их свободолюбивые соседи в первую очередь за счет финансовых
вливаний из метрополий.

\ifcmt
  ig https://scontent-frt3-1.xx.fbcdn.net/v/t39.30808-6/249012758_4501322799944292_1160999887357287254_n.jpg?_nc_cat=104&ccb=1-5&_nc_sid=730e14&_nc_ohc=rX9MWeqsskgAX_18MqW&_nc_ht=scontent-frt3-1.xx&oh=81fc2dd81f13737b97046f1b5fb18d6a&oe=619C01D8
  @width 0.4
  %@wrap \parpic[r]
  @wrap \InsertBoxR{0}
\fi

Куда же подевались ценности, ради которых человечество тысячелетиями воевало за
земли, отбирая их друг у друга? Собственно говоря, никуда не подевались и лежат
на том же месте. Изменилась структура экономики и деньги, которые могут быть
заработаны грабежом, в масштабах развитых стран мизерны. Израильский историк
Юваль Ной Харари подсчитал, что боевики ИГИЛ на Ближнем Востоке награбили на
500 млн долл и еще примерно столько же получили от продажи нефти. Но миллиард
долларов очень небольшая сумма, например, для Китая ВВП которого составляет
почти 15 триллионов долларов и вряд ли он ради нее будет воевать. А наиболее
ценные на сегодняшний день активы вообще невозможно захватить силой, поскольку
они в основе своей имеют не поля, рудники и скважины, а знания их сотрудников и
как точно отметил ученый, "в Силиконовой долине нет силиконовых приисков".

Более того, вновь приобретенные земли, как показывает практика, в современных
условиях не в состоянии прокормить даже сами себя. Для примера рассмотрим
территории, оказавшиеся под контролем Российской Федерации после 1991 года.
Даже отжатый без войны Крым является глубоко дотационным регионом,
потребовавший кроме того расходов на строительство моста (почти 4 млрд долл),
трассы "Таврида" (2,3 млрд долл), газопровода (270 млн долл), энергомоста (200
млн долл) и еще много чего. Содержание Абхазии и Южной Осетии в нынешнем году
"потянет" из федерального бюджета 13,76 млрд рублей (190 млн долл), на
Приднестровье уходит в среднем по 100 млн долл., а содержание ОРДЛО по
некоторым оценкам обходится в 1,5 млрд долл в год.

При этом все эти территории, за исключением Крыма, где по политическим
соображениям обеспечен более-менее сносный уровень жизни, представляют собой
весьма печальное зрелище. Зарплаты в Приднестровье ниже, чем в соседней
Молдове, которая и при СССР была одной из самых экономически отсталых
республик. В Южной Осетии, по заявлению главы парламентского комитета по
социальной политике и здравоохранению непризнанной республики Александра
Плиева, 80\% населения живет за чертой бедности. Донецк, в 2012-2013 годах
возглавлявший составленный Forbes рейтинг лучших украинских городов для ведения
бизнеса, сегодня может "похвастаться" средней зарплатой в 10-15 тысяч рублей
(140-210 долл), а ведь до войны зарплаты в шахтерской столице не уступали
киевским.

Таким образом, завоевательная война сегодня это исчезающий вид. Как говорилось
во второй части, в аграрном обществе присоединение новых территорий является
наиболее простым способом увеличить ВВП государства. В индустриальном, а тем
более постиндустриальном обществе нет смысла захватывать ресурсы – их можно
купить, пусть и недешево, но обратно продать уже готовую продукцию и это
гораздо выгоднее. В качестве примера можно привести Китай, гигантский
экономический скачок которого за последние сорок лет как то обошелся без
завоеваний. 

Так что сегодня привлекательными завоевания территорий могут быть или для
лидеров нищих недоразвитых бантустанов, для которых и три рубля деньги, или же
для засидевшихся у власти автократов, спекулирующих на остатках имперского
чувства у электората с целью поднять свой рейтинг.

\ii{29_10_2021.fb.janvarev_vasilij.1.chast_4_rudiment.cmt}
