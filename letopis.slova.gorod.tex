% vim: keymap=russian-jcukenwin
%%beginhead 
 
%%file slova.gorod
%%parent slova
 
%%url 
 
%%author 
%%author_id 
%%author_url 
 
%%tags 
%%title 
 
%%endhead 
\chapter{Город}

%%%cit
%%%cit_pic
%%%cit_text
\emph{Город} стоит на болотах. Он постоянно подтоплялся, и раз за разом настилалась
новая бревенчатая мостовая. До 28 слоев потом археологи насчитали! Нижние
подмокшие слои, лишенные, как и во всяком болоте, доступа кислорода,
превращались в идеальные капсулы, надежно сохраняющие органику. А береста, как
вы понимаете, это и есть органика.  Вспомните, сколько историй вам в школе
рассказывали о доисторических животных, попавших в трясину. Именно они лучше
всего сохранились. Для палеонтолога болота - это настоящее благословение.  Так
что в Новгороде найдено больше грамот не потому, что это был единственный оазис
просвещения на Руси, а потому что там просто-напросто были наиболее подходящие
условия для их сохранности. Московская береста, а также береста других русских
\emph{Городов} по большей части просто истлела
%%%cit_comment
%%%cit_title
\citTitle{Почему в Москве нашли всего четыре берестяные грамоты, а в Новгороде - свыше тысячи?}, 
Русичи, zen.yandex.ru, 18.05.2021
%%%endcit

%%%cit
%%%cit_head
%%%cit_pic
%%%cit_text
Тут можно долго беседовать со своим бессознательным, выясняя причины подобного
сюрреализма. Но даже не будучи фугасом, превратившимся в траву, за эти два года
я тоже пришел к простому выводу: Я - Человек, рожденный жить. И \emph{Город}, в
котором я живу, не создан для войны. Не создан подсчитывать \enquote{полученные
повреждения в ходе артобстрела}. Не создан считать \enquote{200}, \enquote{300} и \enquote{без вести
пропавших}. Не создан быть полигоном для утилизации боеприпасов и
тестостероновых \enquote{меряний} мужчинами, кто из них лучше умеет убивать. Этот \emph{Город}
рожден жить, любить, вдохновлять, производить и желать доброй ночи
%%%cit_comment
%%%cit_title
  <++>
%%%endcit
