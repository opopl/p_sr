% vim: keymap=russian-jcukenwin
%%beginhead 
 
%%file 26_01_2022.fb.sebastjanovich_pavel.1.kogo_dostalo
%%parent 26_01_2022
 
%%url https://www.facebook.com/p.sebastianovich/posts/5161623067205638
 
%%author_id sebastjanovich_pavel
%%date 
 
%%tags ideologia,obschestvo,politika,strana,ukraina
%%title В стране появился новый класс граждан. Это те, кого достало
 
%%endhead 
 
\subsection{В стране появился новый класс граждан. Это те, кого достало}
\label{sec:26_01_2022.fb.sebastjanovich_pavel.1.kogo_dostalo}
 
\Purl{https://www.facebook.com/p.sebastianovich/posts/5161623067205638}
\ifcmt
 author_begin
   author_id sebastjanovich_pavel
 author_end
\fi

Вчерашняя акция под ВР в очередной раз продемонстрировала - оторванные от жизни
рядовых граждан депутаты, как и весь остальной госаппарат не намерены выполнять
Конституцию и действовать на благо общества. У госаппарата совершенно другие
цели, идущие вразрез с целями народа. За 30 лет независимости мы создали чуждое
нам государство, служащее только самому себе. Разрыв в доходах между
высокопоставленными чиновниками, менеджерами госпредприятий, госбанков с одной
стороны, и врачами, учителями, пенсионерами с другой, превышает 100 раз. И эта
социальная несправедливость, этот разрыв в доходах ставит крест на перспективах
действующей власти.

\ifcmt
  ig https://scontent-frx5-1.xx.fbcdn.net/v/t39.30808-6/272437297_5161630283871583_113283244309525107_n.jpg?_nc_cat=100&ccb=1-5&_nc_sid=730e14&_nc_ohc=0huKAILgSzkAX8vNtIw&_nc_ht=scontent-frx5-1.xx&oh=00_AT9grFD1HJRwaLU8Y5lDSTX8FsvhI7jzxn9jYzAk6jEdDg&oe=61FB651C
	@wrap center
	@width 0.8
\fi

Прошлая проворовавшаяся власть, которую снесли в результате электорального
Майдана,  сегодня собирается во всевозможных конфигурациях, пытаясь опять
вскарабкаться  в высокие кресла. Экс-президенты и экс-министры, экс-премьеры и
экс-депутаты приобрели за время своего «правления» колоссальные финансовые
ресурсы, но обанкротились электорально. И сколько бы денег не было ими вложено
в свой имидж, проголосовавшие против них в 2019-м уже не станут голосовать за
них никогда.

В стране появился новый класс граждан. Это те, кого достало. Кого задолбало.
Кто не видит никаких перспектив ни с действующей властью, ни с прошлыми
хозяйственниками. Кто независимо от личного успеха/неуспеха, от личных доходов
ощущает эту удушливую атмосферу чиновничьего беззакония. Новые поколения,
родившиеся в независимой Украине, уже не хотят видеть на ключевых постах страны
некомпетентных проходимцев. Каждый опрос даёт один и тот же результат -
абсолютное большинство украинцев считает, что страна идёт не туда. И виновники
этого «не туда» - первые лица страны, нарушившие свои предвыборные обещания:
Президент, его окружение и лидеры правящей партии. Все они уйдут на ближайших
выборах в политическое небытие. Но кто придёт на смену?

Вопрос в следующем - есть ли партия, способная объединить в своих рядах
носителей идеи модернизации государства? Есть ли партия, способная сделать
общество более свободным, а экономику более здоровой? Рух «Украина 30+»
регулярно приглашает на свои собрания экспертов и практиков, предлагающих
решения конкретных проблем - в налоговой сфере, в правоохранительной, в
энергетике. Решения эти не востребованы действующими госуправленцами, так как
решения направлены на общественное благо, тогда как деятельность госаппарата
направлена на благо самого госаппарата, на благо самих управленцев.
Не могут
в справедливом обществе судья и прокурор получать зарплаты больше, чем солдат
на передовой или полицейский на дежурстве. Не могут врач и учитель получать
зарплаты ниже средней по стране. Не могут безнаказанно управляющие убыточными
госпредприятиями из года в год получать миллионные зарплаты, а общество гасить
их убытки. Не может сотрудник налоговой службы или социального фонда владеть
торговыми центрами, десятками квартир и земельных участков, и роскошными
автомобилями. Источники дохода чиновников, а не ФОПов, должны стать объектом
пристального внимания правоохранительных органов.

Сегодня рост ВВП, экспортной выручки, зарплат осуществляются не за счёт
экономического роста, а за счёт инфляции и роста мировых цен на сырьё и
энергоресурсы. Уровень жизни и покупательская способность украинцев при этом
постоянно падают. Показатель роста экономики год за годом находится на уровне
ниже среднемирового. Украина признана международными рейтинговыми агентствами
несвободной и коррумпированной страной. Люди не хотят тут жить и рожать детей.
Только за последние три года страну покинули более 2-х млн граждан, а
смертность в два раза превысила рождаемость. Система управления подчинена
старому, как мир, принципу отсталых авторитарных государств - своим всё, чужим
закон. 30 миллионов граждан оказались чужими на этом празднике жизни 1\%
населения. Своим - зарплаты в 300 000, чужим - налоги на землю и недвижимость.
Своим - полномочия, неприкосновенность, безбедное содержание, чужим - пенсия в
3 000 грн.

В стране созрели два противостоящих друг другу класса:

- класс бюрократии, управляющих госкомпаниям, госбанками, госимуществом, и
приближённых к ним, это класс застоя.

- класс с врождённым отвращением к застою, некомпетентности,
непрофессионализму, алчности, использованию госдолжности и госресурсов в личных
целях, ко всему тому, что мы наблюдаем в сегодняшнем госаппарате, это класс
модернизации.

Класс бюрократии хорошо организован, имеет в своих руках все рычаги влияния -
от создания и принятия антинародных законов до прямого силового воздействия на
оппонентов. Классу модернизации ещё предстоит осознать себя и объединиться

Задача новой политической силы - собрать в своих рядах носителей идей
модернизации Украины. Практика показала, что только сам носитель идеи будет
заниматься её реализацией. Никакие поднятые на флаг чужие идеи не могут быть
воплощены в жизнь, если их воплощением не занимается их носитель. Первейшая
задача пришедшей к власти новой силы - уменьшения размера госаппарата, как
количественно, так и в финансовом выражении. Именно разросшийся неэффективный
госаппарат прожирает общественные ресурсы и создаёт атмосферу удушливости и
безнадёжности. Приведение доходов чиновников и управляющих госпредприятиями к
средним доходам в частном секторе восстановит социальную справедливость и
снимет протестные настроения. Доходы врачей, учителей, полицейских,
военнослужащий также должны быть привязаны к средней зарплате в частном
секторе. Всем гражданам, независимо ни от каких условий, должен быть обеспечен
такой уровень дохода, такой уровень социального дивиденда, когда никто не
нуждается в еде и предметах первой необходимости. Наше общество владеет всеми
необходимыми ресурсами для обеспечения такого минимального уровня дохода для
каждого.

Задача новой политической силы - найти опору на свой класс и обеспечить ему
представительство во власти. Цель новой силы - свободное общество и здоровая
экономика. Задача новой силы - не упразднить госаппарат, не сделать его слабым,
а, уменьшив его размер, повысить его эффективность в достижении поставленной
политической силой цели.

Рух Україна 30+
