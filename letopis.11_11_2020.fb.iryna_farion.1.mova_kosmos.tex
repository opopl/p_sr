% vim: keymap=russian-jcukenwin
%%beginhead 
 
%%file 11_11_2020.fb.iryna_farion.1.mova_kosmos
%%parent 11_11_2020
 
%%url https://www.facebook.com/farioniryna/posts/2757280941212166
%%author 
%%tags 
%%title 
 
%%endhead 

\subsection{Диктант національної єдности}
\label{sec:11_11_2020.fb.iryna_farion.1.mova_kosmos}
\Purl{https://www.facebook.com/farioniryna/posts/2757280941212166}
\Pauthor{Фаріон, Ірина}
\index[rus]{Україна!Руський Язик!Диктант національної єдності}

Щойно тепер прочитала так званий диктант національної єдности....Давно не
читала такого пустого, примітивного, водянистого, нікудишнього тексту. Жодних
смислів. Жодних посилів. Жодних ідей. Попсове громаддя слів. А "національна
єдність" - це хіба нас усіх, ще не добитих українців, відправити в космос.

Малкович уже давно став речником космополітично-ліберальних ідей. Тому і пише
таке, що купи не тримається, а хіба гроші дає. Йому ж бо не під силу,
наприклад, написати про Героя московсько-української війни та воїнів-упівців,
що формувалися не на книжках Ролінг, а на блискучій історично-реалістичній
романістиці Лепкого чи Чайковського. Кожен пише лиш те, ким є сам. Не герой не
напише про Героя. Направду, нема більшого ворога для народу, як інтелігентна
бридь, вихована в умовах окупації і до смороду розкладена в умовах
демолібералізму. 

А вершиною  Малковичевого постмодернізму є угадування думок Божих.......свят,
свят, свят.  Дуже дякую шановній учительці Галині Воловець із Кам'янка-Буського
району, що, обурюючись, поділилася зі мною тим диктантом національного
приниження, бо інакше я б не звернула на це уваги, знаючи, хто готує цю
акцію.... 

Я б таке написати не змогла. Збридила б.
