% vim: keymap=russian-jcukenwin
%%beginhead 
 
%%file 07_12_2021.stz.news.nvr.voshod_info.1.ves_mir_s_ukrainoj
%%parent 07_12_2021
 
%%url https://voskhodinfo.su/nashi-avtori/lohmatyy/73622-ves-mir-s-ukrainoj-rzhavye-patrony-i-dobroe-naputstvie.html
 
%%author_id lebedev_sergej_lohmatyj
%%date 
 
%%tags 
%%title Весь мир с Украиной: ржавые патроны и доброе напутствие
 
%%endhead 
\subsection{Весь мир с Украиной: ржавые патроны и доброе напутствие}
\label{sec:07_12_2021.stz.news.nvr.voshod_info.1.ves_mir_s_ukrainoj}
\Purl{https://voskhodinfo.su/nashi-avtori/lohmatyy/73622-ves-mir-s-ukrainoj-rzhavye-patrony-i-dobroe-naputstvie.html}

\ifcmt
 author_begin
   author_id lebedev_sergej_lohmatyj
 author_end
\fi

Весь мир с нами! Так на всех углах заявляет украинская пропаганда, и даже
приводит массу примеров. Почитав у них, обыватель начинает склоняться к тому,
что украинская пропаганда права. И действительно: там Украину поддерживает
Франция, тут Германия, Британия обещала выделить батальон спецназовцев, Байден
рвет на себе майку с воплем: «держите меня семеро, сейчас SWIFT отключать
начну». И это без поддержки таких маститых русофобов, как Литва, Эстония,
Латвия, Грузия, Польша, Румыния.  Даже «президент» Белоруссии Тихановская
поддерживает Украину.

\ii{07_12_2021.stz.news.nvr.voshod_info.1.ves_mir_s_ukrainoj.pic.1}

Но как обстоят дела на самом деле? Давайте посмотрим на реальность, а не на
заявления. А в реальности Украина на каждом шагу что-то старается выпросить
хоть у США, хоть у Сомали: ну хоть что-то дайте. И в капиталистическом мире
нашли прекрасный вариант, как такую поддержку организовать и заработать на этом
неплохо.

Самый банальный пример – это утилизация военной техники, которая была списана.
Её отдают на Украину вместе с кредитом на покупку этой самой техники. То есть
США выделяют Украине помощь в «стопятьсот миллионов», отправляет ржавые корыта,
за которые Украина платит эти «стопятьсот миллионов». А потом Украина должна
отдать деньги вместе с процентами. Отдавать никто не собирается, ясно как божий
день. Но США таким образом (в лице Пентагон) имеют крупный гешефт. В Америке
утилизировать военный катер очень больших денег стоит. А тут Украина сама берет
на себя проблему утилизации, экономя Пентагону кучу денег, которые генералы по
старой доброй традиции пилят как в последний раз. Так еще и официальная
отговорка от всякой общественности есть: мы помогли Украине в войне с Россией!

Не секрет, что некоторые старые виды вооружения, попавшие на территорию
Украины, оказываются в африканских странах, с которыми торговать американцам
нельзя. Это оружие попадает в руки местных головорезов, которые пытаются шатать
режимы, а в это время наднациональные корпорации с неподконтрольной режимам
территории вывозят всякие земельные ресурсы, проводят бактериологические опыты
на местных и уходят от налогов, рассказывая о помощи голодающим детям Африки. В
общем, все при деле, все при заработке.

Украинское правительство продолжает погружать народ в глубочайшую яму кризиса,
рассказывая о самой сильной армии и всемирной поддержке в борьбе с
«агрессором». Но случаются такие казусы, что даже украинская пропаганда с ними
не справляется. Это показывает истинное отношение даже полудохлой Литвы к
украинскому правительству, армии и т.д. Вот с Литвы мы и начнем.

На днях на Украине опять хотели провести некие учения. «Воины света» собрались
на полигоне и распечатали очередную военную помощь, отправленную Литвой для
ВСУ. Как оказалось, патроны в «цинках» были ржавые, учение чуть не сорвалось. И
тогда командование приняло волевое решение – распечатать боеприпасы, которые
хранились для Донбасса. ВСУшники отстрелялись, учения окончены, а вот чем
воевать на Донбассе с «агрессором» никто не знает. Украинское командование дало
распоряжение проверить все остальные «подарки» из Литвы. А Литва… списала
ржавые патроны, вместо того, чтобы утилизировать их, тратя большие суммы на
перевозку, рытье котлованов, подрыв и т.д.

Канадцы очень «помогли» Украине. Риторика представителей Канады, а особенно
украинской диаспоры, проживающей на другой половине планеты, просто потрясает
своей воинственностью и поддержкой Украины в «войне» с Россией. Ну, раз
риторика есть, должны быть и хоть какие-то действия – так решили украинские
власти и обратились за военной помощью к канадским коллегам. Под статьей об
этой просьбе в канадской газете, местные жители написали несколько
комментариев. Основа их в том. Что не хотят канадцы лезть в это дело – помощь
Украине против России. Один из комментаторов предложил послать в ВСУ кофе и
пончики.

Канада уже оказывала помощь: поставили приборы ночного видения, тепловизоры и
прицелы. По прошествии пары недель представитель Канады захотел посмотреть, как
украинские солдаты тренируются с канадской техникой. Оказалось, что нескольких
приборов просто не хватает, а несколько… отсутствуют, и никто не знает, где
они. Дальше было совсем жестко: бойцы ДНР выложили видео своих тренировок, в
которых использовали те самые отсутствующие канадские приборы. Бойцы Ополчения
очень высоко оценили качество тепловизоров и попросили прислать еще.

Буквально на днях министр обороны Украины клянчил у главы Пентагона технику на
случай вторжения России. Резников умеет челобитную подавать — все было
аккуратно выписано на листочке в клеточку. Та были перечислены техника, системы
РЭБ и оружие, которое планировалось поставлять в Афганистан, до того, как США с
позором бежали из него, бросив своих «друзей» на растерзание талибам. Но как
заявили в США, они не желают нагнетать обстановку с РФ и бряцать оружием.
Потому Украина, получив пару списанных катеров и полторы тонны б/у резиновых
лодок, должна уже испытывать блаженство. Все-таки армии Афганистана Пентагон
доверял больше.

Скорее всего, от хлама большинство стран уже избавилось, а новое оружие Украине
не видать. Пусто стало на украинских складах боеприпасов, даже взорвать нечего.
Все хорошее разворовано и перепродано при Порошенко, а Зеленскому придется
воевать, по-другому скрыть колоссальную недостачу и объяснить экономических
хаос не удастся. Скорее всего, Украина готова пожертвовать еще несколькими
областями, чтобы уменьшить давление на экономику, которой уже и так практически
нет.

А чтобы ВСУшники и их «побратимы» не взбунтовались, их стоит спалить в огне
войны. По расчетам команды Зеленского, они смогут дотянуть до 20-х чисел января
2022-го. Потом наступит энергетический коллапс, влекущий за собой гибель
остатков инфраструктуры. Именно по этой причине украинские власти кричат о том,
что Россия нападет в то время.

Ну а пока усиливается риторика о всемирной поддержке Украины в войне с Россией.
Таким образом нагнетается внутренняя обстановка на Украине и придается
видимость храбрости ВСУ. Так было перед тем, как Грузия напала на Ю. Осетию и
Абхазию в 2008 году. Тогда тоже «весь мир был с Грузией» и войска стран НАТО
готовы были поддержать Саакашвили в войне с «оккупантом российским». Жаль, что
Саакашвили покинул Украину, не успев подсказать Зеленскому магазин в Киеве с
самыми вкусными галстуками.  

Сергей Лебедев (Лохматый)
