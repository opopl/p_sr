% vim: keymap=russian-jcukenwin
%%beginhead 
 
%%file 10_03_2022.fb.fb_group.story_kiev_ua.1.hroniki
%%parent 10_03_2022
 
%%url https://www.facebook.com/groups/story.kiev.ua/posts/1878227122374080
 
%%author_id fb_group.story_kiev_ua,stepanov_farid
%%date 
 
%%tags 
%%title ХРОНІКИ НЕОГОЛОШЕНОЇ ВІЙНИ, Або: день чотирнадцятий. Тринадцять днів, що змінили наше життя та світ
 
%%endhead 
 
\subsection{ХРОНІКИ НЕОГОЛОШЕНОЇ ВІЙНИ, Або: день чотирнадцятий. Тринадцять днів, що змінили наше життя та світ}
\label{sec:10_03_2022.fb.fb_group.story_kiev_ua.1.hroniki}
 
\Purl{https://www.facebook.com/groups/story.kiev.ua/posts/1878227122374080}
\ifcmt
 author_begin
   author_id fb_group.story_kiev_ua,stepanov_farid
 author_end
\fi

ХРОНІКИ НЕОГОЛОШЕНОЇ ВІЙНИ

Або: день чотирнадцятий. Тринадцять днів, що змінили наше життя та світ

\enquote{У перший день о 4-й ранку на нас полетіли крилаті ракети. Так, що всі
прокинулися – ми, діти, всі ми, живі люди, вся Україна. І відтоді не спить. Ми
всі стали до зброї. Ставши великою армією}

Йому випало бути Президентом у складні часи, коли питання стоїть: \enquote{Бути чи не
бути Україні}

\ii{10_03_2022.fb.fb_group.story_kiev_ua.1.hroniki.pic.1}

\enquote{На другий день відбивалися від атак у повітрі, на суші й на морі. І наші
героїчні прикордонники на острові Зміїний у Чорному морі розповіли всім про
фінал війни. А саме: куди зрештою вирушить ворог. Коли російський корабель
вимагав від наших хлопців скласти зброю, вони відповіли йому... Так міцно, як
не можна сказати в парламенті. І ми відчули силу. Велику силу нашого народу,
який до кінця гнатиме окупанта}

Йому пощастило відчути підтримку народу України, яку не мав жоден з його
попередників.

\enquote{На третій день російські війська не ховаючись били просто по людях, по
житлових будинках. Артилерією. Авіабомбами. І це остаточно показало нам,
показало світу, хто є хто. Хто великі люди, а хто – просто звірі}

Йому пропонували особисту безпеку, а він обрав бути разом зі своїм народом і
країною. 

\enquote{На четвертий день, коли полонених ми вже почали брати десятками, ми не
втратили гідності. І не знущалися над ними. Ми ставимося до них, як до людей.
Бо ми залишилися людьми на четвертий день цієї ганебної війни}

Ворог дуже бажає, щоб він тремтів та ховався. А він, ніби знущаючись,
звертається до нації, демонструючи, що він там, де і має бути - на своєму
робочому місці.

\enquote{На п'ятий день терор проти нас уже був відвертим. Проти міст, проти маленьких
містечок. Зруйновані райони. Бомби, бомби, бомби, знову бомби на будинки, на
школи, на лікарні. І це геноцид. Який не зламав нас. Мобілізував кожного й
кожну з нас. І дав нам відчуття великої правди}

Багато українців обирали його, віддаючи свій голос не за нього, а проти його
опонента. Сьогодні вони ним пишаються.

\enquote{На шостий день російські ракети впали на Бабин Яр. Це місце, де нацисти
стратили 100 тисяч людей у роки Другої світової війни. Через 80 років потому
Росія вбила їх удруге}

Навіть йому, людині із відчуттям гумору, було не смішно, коли його звинуватили
у нацизмі. Його, представника народу, що чи не найбільше постраждав від нацизму
у роки II Світової.

\enquote{На сьомий день ми зрозуміли, що вони руйнують навіть церкви. Бомбами! Знову
ракетами. Вони не знають святого й великого, як знаємо ми}

Як справжній чоловік, він викликав на чоловічу розмову того, кому і руку не
варто простягнути. Той відмовився. Бо, на відміну від нашого, той - чоловік
тільки за вторинними ознаками.

\enquote{На восьмий день світ побачив, що з російських танків стріляють по атомній
електростанції. Найбільшій у Європі. І світ почав розуміти, що це терор проти
всіх. Це великий терор}

Його обирали, як Президента, а отримали справжнього лідера.

\enquote{На дев'ятий день ми слухали зустріч країн НАТО. І без бажаного для нас
результату. Без мужності. Так ми відчули це – я не хочу нікого образити – ми
відчули, що альянси не діють. Навіть небо закрити не можуть. І тому гарантії
безпеки у Європі треба будувати з нуля}

Коли він зрозумів, що Україна залишилася практично один на один із ворогом, він
подякував союзникам за те, що зробили, що змогли.

\enquote{На десятий день беззбройні українці в окупованих містах повсюдно протестували,
масово. Зупиняючи бронетехніку голіруч. Ми стали незламними}

Були ті, хто називав його клоуном. Зараз вони звертаються до нього: \enquote{Мій
Президент}.

\enquote{На одинадцятий день, коли бомбили вже житлові райони, коли від вибухів усе
руйнувалося, коли евакуювали дітей з пошкодженої дитячої онколікарні... Ми
усвідомили: українці стали героями. Сотні тисяч людей. Цілі міста. Діти,
дорослі – всі}

Для лідерів світу від був молодим і недосвідченим новачком, що випадково
потрапив у вищу лігу. Сьогодні вони вважали би за честь потиснути йому руку.

\enquote{На дванадцятий, коли втрати російської армії вже перевищили 10 тисяч убитих, у
цьому рахунку з'явився й генерал. І це дало нам упевненість: за всі злочини, за
всі ганебні накази все ж таки буде відповідальність. Міжнародного суду або
української зброї}

За ці тринадцять днів він став дорослішим років на двадцять. Не знаю, чи
збільшилося у нього сивини, але рубців на серці точно стало більше.

\enquote{На тринадцятий день у блокованому російськими військами Маріуполі померла
дитина. Від зневоднення. Вони не пускають ні харчові продукти, ні воду не дають
людям. Просто заблокували – і вони у підвалах. Я думаю, всі чують: там у людей
немає води!}

4 червня 1940 року Прем'єр-міністр Великої Британії Вінстон Черчілль виголосив
перед Британським парламентом промову, яку потім історики будуть називати
однією з найвеличніших промов XX століття. 

\begin{zzquote}
"Ми не здамося й не програємо! Ми підемо до кінця.

Ми будемо боротися на морях, будемо битися в повітрі, ми будемо захищати нашу
землю, хоч би якою була ціна.

Ми будемо битися в лісах, на полях, на узбережжях, у містах і селах, на
вулицях, ми будемо битися на пагорбах..."
\end{zzquote}

Що він відчував? Страх? Відповідальність? Не знаю. Він просто повів за собою
націю до перемоги.

8 березня 2022 року Президент України Володимир Зеленський  виголосив перед
Британським парламентом промову, яку привітали оплесками: \enquote{... І я хочу від
себе додати: ми будемо битися на териконах, на березі Кальміуса та Дніпра! І ми
не здамося!}

Ми всі не здамося.

Слава Україні!

PS:

Все це не означає, що після нашої Перемоги, ми не будемо прискіпливо запитувати
у нашого Президента про виконання обіцянок, про здобутки і прорахунки. І зовсім
не означає, що у разі не виконання обіцяного, ми не оберемо іншого Президента.
Не тому, що ми невдячні. А тому, що ми - українці. І право вибору є для нас
священним.
