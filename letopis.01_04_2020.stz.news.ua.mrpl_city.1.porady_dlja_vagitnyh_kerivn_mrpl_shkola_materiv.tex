% vim: keymap=russian-jcukenwin
%%beginhead 
 
%%file 01_04_2020.stz.news.ua.mrpl_city.1.porady_dlja_vagitnyh_kerivn_mrpl_shkola_materiv
%%parent 01_04_2020
 
%%url https://mrpl.city/blogs/view/poradi-dlya-vagitnih-vid-kerivnitsi-mariupolskoi-shkoli-materiv
 
%%author_id demidko_olga.mariupol,news.ua.mrpl_city
%%date 
 
%%tags 
%%title Поради для вагітних від керівниці маріупольської Школи матерів
 
%%endhead 
 
\subsection{Поради для вагітних від керівниці маріупольської Школи матерів}
\label{sec:01_04_2020.stz.news.ua.mrpl_city.1.porady_dlja_vagitnyh_kerivn_mrpl_shkola_materiv}
 
\Purl{https://mrpl.city/blogs/view/poradi-dlya-vagitnih-vid-kerivnitsi-mariupolskoi-shkoli-materiv}
\ifcmt
 author_begin
   author_id demidko_olga.mariupol,news.ua.mrpl_city
 author_end
\fi

Весна видалася нелегкою не тільки для маріупольців чи загалом України. Багато
країн охопила пандемія жорстокого і небезпечного коронавірусу, здатного вразити
більшу частину людства. Вчені дійшли висновку, що найбільш уразлива група – це
літні люди.

Однак вразливими є і діти та люди, що мають хронічні захворювання. Останнім
часом у соцмережах шириться багато недостовірної інформації щодо коронавірусу
та його наслідків. Один з міфів стосується впливу цієї недуги на стан вагітних
та новонароджених. Саме тому я вирішила поспілкуватися з керівницею Школи
матерів, що працює при Перитинальному центрі Маріуполя \emph{\textbf{Оленою Іванівною
Осіпенко}}, яка дала декілька корисних порад для вагітних маріупольчанок.
По-перше, вона наголосила, що слід дуже серйозно віднестися до карантинних
заходів і ще більш відповідально ставитися до свого здоров'я. Наразі немає
підтвердженої інформації, що у вагітної жінки, яка має COVID-19, проходить
хвороба важче, ніж у будь-якої іншої дорослої людини.

\textbf{Читайте также:} \emph{Карантин — новый вызов для спортсмена}%
\footnote{Карантин — новый вызов для спортсмена, Георгий Федоренко, mrpl.city, 31.03.2020, \par%
\url{https://mrpl.city/blogs/view/karantin-novyj-vyzov-dlya-sportsmena}
}

Очікується, що велика частина вагітних, навіть якщо зараження і станеться,
будуть відчувати тільки помірні симптоми грипу або застуди. Тому коронавірус не
становить серйозної небезпеки для вагітних жінок, він не призводить до
ускладнень під час вагітності, а вірус не передається новонародженому.
Водночас, хоча ризики виглядають низькими, вагітні, в яких є хронічні
захворювання, такі як діабет, повинні бути більш уважні. Партнерські пологи
через карантин скасовані. Але Олена Іванівна підкреслює, що раніше про
партнерські пологи ніхто й не знав і пологи проходили без ускладнень, тому
сильно переживати з цього приводу не варто.

Оскільки Школа матерів зараз не працює Олена Іванівна підготувала декілька
корисних і важливих порад для вагітних:

\begin{itemize} % {
\item 1. \emph{Налаштуйте себе на позитив і попри всі обставини зберігайте спокій.}

\item 2. \emph{Дотримуйтеся загальних запобіжних заходів:} регулярно провітрюйте приміщення,
відкривайте широко вікна, робіть вологе прибирання.

\item 3. \emph{Частіше мийте руки з милом.} Якщо знаходитеся не вдома, користуйтеся
антисептиком і обробляйте ним руки, якщо немає можливості їх помити. Не
торкайтеся обличчя.

\item 4. \emph{Уникайте людей з явними симптомами ГРВІ.}

\item 5. \emph{Вдома – використовуйте окремі рушник і посуд.}

\item 6. \emph{Гуляйте на свіжому повітрі!} На вулиці намагайтеся не користуватися
громадським транспортом і уникайте скупчення людей.

\item 7. \emph{Носіть маску, але правильно!} Маска повинна щільно сидіти на обличчі, навколо
неї не повинно бути щілин, які відкривають доступ вірусам. Для кращого
прилягання можна відрегулювати медичну маску в районі носа, за допомогою
вбудованого в медичну маску носового фіксатора. Не чіпайте маску брудними
руками. Якщо маска одноразова – то вона одноразова. Ніколи не надягайте її
повторно. Зняли – і відразу ж викинули. Після того, як ви зняли маску – негайно
вимийте руки з милом. Саморобні марлеві маски потрібно щодня прати і прасувати
з обох сторін кожні дві години.

\item 8. \emph{Підберіть вітамінний комплекс.} Їжте більше сухофруктів, пийте узвар. Дуже
корисною є шипшина. 

\item 9. \emph{Обмежте приплив інформації.} Майбутня мати сприймає все прочитане близько до
серця і нервується. Бережіть себе, і замість \enquote{зависання} на форумах краще
думайте про своє щасливе майбутнє.

\item 10. \emph{Всі необхідні телефони лікарів завжди тримайте під рукою.} При появі перших
симптомів коронавірусу одразу проінформуйте свого лікаря. 
\end{itemize} % }

Наприкінці Олена Іванівна додала, що вагітність – це неймовірний і незабутній
етап в житті кожної жінки і ніякі пандемії цього ніколи не змінять.
