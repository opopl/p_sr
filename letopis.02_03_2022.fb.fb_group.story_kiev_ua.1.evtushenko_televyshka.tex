% vim: keymap=russian-jcukenwin
%%beginhead 
 
%%file 02_03_2022.fb.fb_group.story_kiev_ua.1.evtushenko_televyshka
%%parent 02_03_2022
 
%%url https://www.facebook.com/groups/story.kiev.ua/posts/1872646752932117
 
%%author_id fb_group.story_kiev_ua,veksler_sergej.vostochnyj_ierusalim
%%date 
 
%%tags 
%%title И сегодня Евтушенко. Пост этот на его языке
 
%%endhead 
 
\subsection{И сегодня Евтушенко. Пост этот на его языке}
\label{sec:02_03_2022.fb.fb_group.story_kiev_ua.1.evtushenko_televyshka}
 
\Purl{https://www.facebook.com/groups/story.kiev.ua/posts/1872646752932117}
\ifcmt
 author_begin
   author_id fb_group.story_kiev_ua,veksler_sergej.vostochnyj_ierusalim
 author_end
\fi

И сегодня Евтушенко.

Пост этот на его языке.

После того, как фашисты целились вчера в телевышку, установленную савецкой
властью на краю Бабьего Яра.

Вчера путинские фашисты стреляли по Бабьему Яру. Там, без памятника, покоятся
мои близкие.

Евгений Александрович Евтушенко. Євген Олександрович Євтушенко.

יבגני אלקסנדרוביץ' ייבטושנקו

Над Бабьим Яром памятников нет.
Крутой обрыв, как грубое надгробье.
Мне страшно.
Мне сегодня столько лет,
как самому еврейскому народу.

Мне кажется сейчас —
я иудей.
Вот я бреду по древнему Египту.
А вот я, на кресте распятый, гибну,
и до сих пор на мне — следы гвоздей.

Мне кажется, что Дрейфус —
это я.
Мещанство —
мой доносчик и судья.

Я за решеткой.
Я попал в кольцо.
Затравленный,
оплеванный,
оболганный.

И дамочки с брюссельскими оборками,
визжа, зонтами тычут мне в лицо.
Мне кажется —
я мальчик в Белостоке.
Кровь льется, растекаясь по полам.

Бесчинствуют вожди трактирной стойки
и пахнут водкой с луком пополам.
Я, сапогом отброшенный, бессилен.
Напрасно я погромщиков молю.

Под гогот:
«Бей жидов, спасай Россию!»-
насилует лабазник мать мою.

О, русский мой народ! —
Я знаю —
ты
По сущности интернационален.

Но часто те, чьи руки нечисты,
твоим чистейшим именем бряцали.
Я знаю доброту твоей земли.

Как подло,
что, и жилочкой не дрогнув,
антисемиты пышно нарекли
себя «Союзом русского народа»!

Мне кажется —
я — это Анна Франк,
прозрачная,
как веточка в апреле.

И я люблю.
И мне не надо фраз.
Мне надо,
чтоб друг в друга мы смотрели.

Как мало можно видеть,
обонять!
Нельзя нам листьев
и нельзя нам неба.

Но можно очень много —
это нежно
друг друга в темной комнате обнять.
Сюда идут?

Не бойся — это гулы
самой весны —
она сюда идет.
Иди ко мне.

Дай мне скорее губы.
Ломают дверь?
Нет — это ледоход...
Над Бабьим Яром шелест диких трав.

Деревья смотрят грозно,
по-судейски.
Все молча здесь кричит,
и, шапку сняв,
я чувствую,
как медленно седею.

И сам я,
как сплошной беззвучный крик,
над тысячами тысяч погребенных.

Я —
каждый здесь расстрелянный старик.

Я —
каждый здесь расстрелянный ребенок.

Ничто во мне
про это не забудет!

«Интернационал»
пусть прогремит,
когда навеки похоронен будет
последний на земле антисемит.
Еврейской крови нет в крови моей.
Но ненавистен злобой заскорузлой
я всем антисемитам,
как еврей,
и потому —
я настоящий русский!

ЕВГЕНИЙ ЕВТУШЕНКО. НАСТОЯЩИЙ РУССКИЙ.

2.3.22

Що у вас на думці?

Подорожник.

Ні, це російською мовою не буде тут. Подорожник - подорожник і українською мовою. 

Подорожник з Бабиного Яру він. У києвському садочку в Юдейських горах. Зараз,
другого березня. Наступного дня після обстрилу Бабиного Яру.

Він росте тепер і тут. З насіння, зібраного там. Не поруч з зпорудженою у
радянській час телевежою - а з тропок він тих, якими йшли люди 29 вересня 1941.
У березні 2022 Бабин Яр на прицілі путінських літаків...

Нестерпна біль. Немає їм виправди.
