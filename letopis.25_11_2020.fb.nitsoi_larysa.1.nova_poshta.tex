% vim: keymap=russian-jcukenwin
%%beginhead 
 
%%file 25_11_2020.fb.nitsoi_larysa.1.nova_poshta
%%parent 25_11_2020
 
%%url https://www.facebook.com/larysa.nitsoi/posts/3806803959351708
 
%%author 
%%author_id nitsoi_larysa
%%author_url 
 
%%tags 
%%title "Українською, будь ласка!"
 
%%endhead 
 
\subsection{\enquote{Українською, будь ласка!}}
\label{sec:25_11_2020.fb.nitsoi_larysa.1.nova_poshta}
\Purl{https://www.facebook.com/larysa.nitsoi/posts/3806803959351708}
\Pauthor{Ніцой, Лариса}

Дожилася!

Нова пошта. Прибігла здати посилочку.

	\obeycr
- Що в посилці? - питає оператор.
- Сувеніри, - відповідаю.
- Що саме?
- Е-е-е-е, - заскочена питанням (як би так відповісти одним словом), - ну-у-у, як сказати, хенд мейд.
- Українською, будь ласка! - каже оператор.
Я як засміюся.
- Чому ви? - підозріло дивиться на мене оператор, - щось не так?
- Так! Так! - киваю головою і продовжую сміятися. - Ви все правильно кажете! Українською буде \enquote{поробки}.
	\restorecr

Що мені тобі сказати, хлопче! Що я цього 30 років чекала? Не просто чекала.

Тратила життя на балачки, які часто закінчувалися сварками. Боролася, доводила,
переконувала...  Тратила нерви, енергію, здоров'я, не проводила час в насолоді
з дітьми, які зростали, поки мама бігала по мітингах, кабінетах, каналах тб...

Поливана помиями і оббріхана... Їздила по закордонах (не по пляжах і курортах
на відпочинок), щоб досліджувати, а як у них? Бо ж мене вважають у своїй країні
божевільною... А коли все ж розібралися, що то ж таки мова колонізатора - інша
напасть приключилася. Суцільна іноземщина в мову із заходу зайшла. І не
розуміють люди, що то також біда...  Я йому \enquote{хенд мейд}. А він мені
\enquote{українською, будь ласка}! Боже! Яка музика!

- Може у вас там скло, чи інший делікатний товар. Можете описати? - продовжує оператор.

Описала. Порозумілися. Пішла, сміючись. Мабуть, він мені здивовано дивився в спину. 

Дякую, тобі, хлопче! Знав би ти, як довго я до цього йшла, щоб почути серйозно сказане: \enquote{Українською, будь ласка!}

