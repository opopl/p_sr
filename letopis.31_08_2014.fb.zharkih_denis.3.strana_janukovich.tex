% vim: keymap=russian-jcukenwin
%%beginhead 
 
%%file 31_08_2014.fb.zharkih_denis.3.strana_janukovich
%%parent 31_08_2014
 
%%url https://www.facebook.com/permalink.php?story_fbid=1541109292769149&id=100006102787780
 
%%author_id zharkih_denis
%%date 
 
%%tags janukovych_viktor,strana,ukraina
%%title Что мне нравилось, а точнее, что мы потеряли с Януковичем
 
%%endhead 
 
\subsection{Что мне нравилось, а точнее, что мы потеряли с Януковичем}
\label{sec:31_08_2014.fb.zharkih_denis.3.strana_janukovich}
 
\Purl{https://www.facebook.com/permalink.php?story_fbid=1541109292769149&id=100006102787780}
\ifcmt
 author_begin
   author_id zharkih_denis
 author_end
\fi

Что мне нравилось, а точнее, что мы потеряли с Януковичем. 

Мой новый  френд Э. Парэссэ задал мне вопрос "Что было хорошо при власти
Януковича?" На этот ответ, конечно, есть клише и про отсутствие кровопролития,
и про спокойный доллар, и про наличие Крыма, и про наличие горячей воды. Но я
хочу ответить более нестандартно. Я хочу показать, как Украина теряла, и тот же
Крым, и воду горячую, и мирное небо над головой.

Реальным основанием Майдана был социальный взрыв, который был вызван плохим
управлением команды Януковича. Виктор Федорович и его "молодая команда" делали
из Украины Донецкую область, а именно пытались сосредоточить все деньги и
активы в своих руках, тем самым обезопасить себя от финансовых возможностей
оппонентов. Их не интересовало, собственно, управление страной и этими
активами, а именно владение и возможность перекрыть кран.Протестовать против
такой политики - святое дело. Но, извините, к евроинтеграции, которая выступила
поводом для Майдана, это не имеет никакого отношения. Сегодня осуждают жителей
Востока, которые вышли с российскими флагами, как бы призывая Россию вмешаться,
но несколько раньше люди выходили с флагами ЕС. Что мы имеем в этом случае? Мы
призываем кого-то разобраться с нашим политическим оппонентом, тем самым
демонстрируем инфантилизм, слабость и национальную незрелость. В ОБОИХ
СЛУЧАЯХ!. Более того, никто не запрещает обращаться к кому-нибудь за помощью,
но важно понимать, КАКАЯ ПОМОЩЬ ТРЕБУЕТСЯ. С этим полная неразбериха. ЕС дали
сигнал - мы хотим к вам, а нас Яукович не пускает. Точно такой же сигнал позже
получит Россия в отношении новой власти. И расскажите бабе Мане, что эту
идиотскую ситуацию начала Россия. Что, собственно, потеряло украинское общество
от такой постановки вопроса? Оно потеряло ПРАВО САМОМУ РЕШАТЬ, что и как делать
со своими политиками. Спасибо Майдану! Сегодня Порошенко и шага не сделает без
Запада, а вся Украина внимательно следит, что скажет Путин. Так вот, ДО МАЙДАНА
ТАКОГО НЕ БЫЛО! Мы начали терять свою независимость, когда делегировали право
решать наши внутренние вопросы ЕС, России и США, а не думать головой самим, и
не приходить к единому решению. 

Четкого понимания будущего страны Майдан так и не сформулировал, но на эту кашу
лозунгов ему навязали трех руководителей, которых народ назвал Трус, Балбес и
Бывалый. Представляли ли эти люди всю Украину? Естественно - нет. Это были
радикал Тягнибок, популист Кличко и представитель банкиров Яценюк. Эти люди не
могли представлять не то, что интересы всего народа, но даже интересы его
большинства. Люди на Майдане не поняли, что их дурят, поскольку основной целью
было свержение Януковича. Естественно, что Западу из этой троицы мог быть
полезен только банкир, который мог представлять интересы западного капитала в
доступе к украинским ресурсам. Но среднего украинца не представлял, собственно,
никто. Так украинское общество потеряло возможность представлять свои интересы
на Майдане. Кричать и выступать они могли много, но это просто был выпуск пара.
Украина УЖЕ была обречена на разрыв. Кремль молниеносно нашел представителей
своих интересов в Крыму. В Киеве были марионетки Запада, в Симферополе -
Москвы. Все зеркально,все по феншую. 

Но дело могло еще кончится с выигрышем для Украины, если бы мирное выступление
не перешло в беспорядки, а потом в восстание. Беспорядки и захват
административных зданий подняли радикальные элементы на небывалую высоту. Это
сломало прежнюю политическую систему Украины. Что делал средний политик для
победы? Он ПОДКУПАЛ электорат. Мог гречки дать, мог что-то пообещать, но в
целом, заигрывал с народом. Теперь с этим самым народом никто не церемонится.
Решают не слова, не финансы, не работа с людьми, а автомат. Такого в Украине
раньше не было. МЫ ПОТЕРЯЛИ ДЕМОКРАТИЧЕСКОЕ УСТРОЙСТВО СВОЕЙ СТРАНЫ. Потеряли
потому, что сжигали представителей власти (беркутовцев), издевались над
чиновниками и прочее. Ведь унижали не только банду Януковича, но и
ПРЕДСТАВИТЕЛЕЙ ГОСУДАРСТВА. Ломали ГОСУДАРСТВЕННЫЕ ИНСТИТУТЫ. А вместо этого
ставили людей с ружьем. Это сначала произошло на Западе Украины, а потом по той
же схеме на Востоке. Украинцы доказали, что они за деньги, а то и за бесплатно,
себе во вред, согласны громить институты власти своей родной страны. Ведь не
важно плохой там чиновник сидит    или хороший. Это все равно, что за рулем
сидит плохой водитель, так надо машину ему разбить, особенно, если разбивает ее
и курочит совладелец. Но часто в порыве гнева так и бывает. Разрушенное с двух
сторон государство стало беззащитно и начался передел собственности среди
олигархов, которые заменили государственные институты (уничтоженные украинцами)
своими фирмами. Олигархи создали свои армии и начали воевать друг с другом. По
власти они смело конкурируют с украинской государственностью, она для них
ничего уже не значит, их некому контролировать, руки коротки. 

Вариантов только два:  или этим олигархам выпишут трындюлей пророссийские силы,
которые по сути не олигархи, и играют по другим правилам (вспоминаем
Гражданскую войну прошлого века), или 
украинцы им сами отвесят трындюлей. Есть вариант, что под себя всех подомнет
один олигарх, он же и всех помирит, но это должен быть уже реальный
политический игрок. Их осталось два - Левочкин и Коломойский. Какой из
вариантов самый вероятный, думаю каждый понимает.
