% vim: keymap=russian-jcukenwin
%%beginhead 
 
%%file 08_09_2023.stz.news.ua.mrpl.0629.1.pochaty_zhyttja_z_nulja
%%parent 08_09_2023
 
%%url https://www.0629.com.ua/news/3657410/pocati-zitta-z-nula-mariupolcanka-vidkrila-u-lucku-majsternu-tetrapod-mrpl-i-sie-sporadzenna-dla-vijskovih
 
%%author_id news.ua.mrpl.0629
%%date 
 
%%tags 
%%title Почати життя "з нуля". Маріупольчанка відкрила у Луцьку майстерню "Тетрапод Мрпл" і шиє спорядження для військових
 
%%endhead 
 
\subsection{Почати життя \enquote{з нуля}. Маріупольчанка відкрила у Луцьку майстерню \enquote{Тетрапод Мрпл} і шиє спорядження для військових}
\label{sec:08_09_2023.stz.news.ua.mrpl.0629.1.pochaty_zhyttja_z_nulja}
 
\Purl{https://www.0629.com.ua/news/3657410/pocati-zitta-z-nula-mariupolcanka-vidkrila-u-lucku-majsternu-tetrapod-mrpl-i-sie-sporadzenna-dla-vijskovih}
\ifcmt
 author_begin
   author_id news.ua.mrpl.0629
 author_end
\fi

\ifcmt
  ig https://s.0629.com.ua/img/section/newsInternalIcon/is/2/photo2023-09-0812-12-4264faf37abe83e_64fb20093443e.jpg
  @wrap center
  @width 0.9
\fi

\begin{quote}
\em
Маріуполь помер для неї цієї зими. Не тоді, коли ховалась в заблокованому місті
від обстрілів, не тоді, коли залишала його, перетинаючи \enquote{місячній пейзажи}
розбомблених будинків, вириваючись із блокади. Місто померло для неї взимку,
коли побачила у новинах, як окупанти демонтують драматичний театр.
\end{quote}

\enquote{Ось в цю мить я відчула, що мого міста більше немає. І повертатися мені нема
куди}, - каже Оксана Макарова в інтерв'ю виданню 62.ua

В березні 2022 року вона, рятуючи себе та півторарічного сина, вирвалась із
заблокованого Маріуполя і опинилась у Луцьку. А вже зараз є керівницею
невеличкого підприємства з пошиву військового спорядження – \enquote{Тетрапод
Мрпл}.

\enquote{Звісно, тетрапод. Коли треба було вигадати назву нашому швейному виробництву,
одразу спало на думку це слово. Вона стало символом спротиву для багатьох
маріупольців. Тут, у Луцьку, не дуже розуміють, що це таке, але для мене
важливо зберігати зв'язок з містом, яке я так люблю}.

Оксана Макарова – маріупольчанка. До війни вона займалась підприємництвом, мала
невеличку майстерню, де виготовляла дивні вироби з епоксидної смоли. Але коли
народилася дитина, призупинила свій бізнес.

\subsubsection{Вижити в блокадному Маріуполі}

\enquote{Я закрила свій ФОП на початку лютого. Було абсолютно зрозуміло, що найближчим
часом буде вже не до бізнесу, - каже вона. – Мій чоловік – професійний
військовий, 25 років на службі. Олег був нагороджений Президентом України
медаллю за визволення Маріуполя у 2014 році. Він не просто штабний – весь цей
час на передовій, бойовий офіцер. І звісно, ми багато розмовляли про війну, про
те, що вона можлива. Олег казав, що краще поїхати з Маріуполя кудись подалі, бо
зрозуміло, що наше місто дуже цікавить росіян. Але ні він, ні я, ні колеги
чоловіка навіть не уявляли, що війна може бути ось такою, нищівною, тотальною,
страшною. Ніхто не думав, що місто будуть стирати із землі разом з тисячами
людей.

Ми були впевнені, що цивільних не зачеплять. Тому 24 лютого 2022 року чоловік
пішов у свій підрозділ, а ми з дитиною залишались у Маріуполі. Це ж дуже
непросто от так взяти, підхопитись, залишити квартиру, ремонт, все життя і
їхати в нікуди, у невідомість. Ми залишились вдома – і помилились...}

Квартира Оксани Макарової знаходилась на вулиці Куїнджі, в самому центрі
Маріуполя, другий будинок від драматичного театру. Це старий будинок,
побудований за сталінські часи, як і більшість сусідніх в тому районі. Їх
будували з бомбосховищами, тож було де укритися під час бомбардувань. Але
Оксана туди не спускалася.

\enquote{Ми з дитиною та собакою залишалися в квартирі. З початку повномасштабного
вторгнення в наш район везли людей з Лівобережжя. Підвали були забиті людьми.
Темні, холодні, необлаштовані. Люди сиділи на головах один у одного, потім
взагалі сваритися там почали. А я з маленькою дитиною. На вулиці мороз. Я
розуміла, що там дуже просто дитині підхопити запалення легенів, взагалі всі
один одного там заражали, хворіли. А ліків немає, лікарів немає. Тому ми з
синочком ховалися вдома. Накривалися ковдрами і перечікували обстріли. А потім
почались авіабомбардування...}

З вікон Оксаниної квартири було добре видно, як скидають літаки страшні бомби
на місто.

\enquote{Я бачила, як впала бомба на будинок з магазином \enquote{Шини}. Там
було бомбосховище, яке довго було зачинене. Потім чоловіки прийшли і силоміць
виламали двері, і люди туди спустилися. Але і туди прилетіла бомба. Я бачила,
як впала бомба на будинок з магазином \enquote{Манго}, там теж був підвал, де
переховувались люди. Не знаю, скільки їх там загинуло, але мені стало абсолютно
зрозуміло, що переховуватись у підвалі немає сенсу – тут просто лотерея,
пощастить – не пощастить.

На моїх очах літак скинув дві бомби на завод \enquote{Октябрь}. Цей завод знаходився
одразу навпроти нашого будинку. Одна бомба впала туди, а інша розтрощила кафе
\enquote{Артана} поруч. Від того вибуху в нашій квартирі повилітали всі вікна, все
посипалось на голови. Я стояла, дивилась на це і раптом зрозуміла, що досі не
накрила дитину собою. І раптом щось трапиться, я просто не встигну зреагувати.
Тоді мені стало ясно – треба уходити, бо якщо нас тут завалить, то нема кому
нас відкопувати.

До цього моменту я все чекала якогось коридору, чекала Червоного Хреста,
евакуації, бігала до драмтеатру із надією прочитати або почути оголошення про
евакуаційні автомобілі. Але після вибуху в \enquote{Октябрі} стало ясно: всі
мої очікування і сподівання марні. Треба діяти самостійно. Треба рятувати
дитину.

Тому я вивезла із гаража нашу машину, на щастя, вона була ціла, сіли туди – я з
дитиною, мої батьки, батьки чоловіка, собака французький бульдог – і поїхали до
друзів у Приморський район. Думали, там буде безпечніше. Це було 15 березня.
Але нас це не врятувало. Там неподалік була військова частина, і росіяни почали
тотально закидати бомбами та ракетами все навколо.

Тож я вирішила, що чекати більше не можна. Рано вранці 18 березня ми всі
завантажились у машину, 10 чоловік і собака, і поїхали з міста}.

Оксані дуже пощастило - вона встигла виїхати до початку роботи фільтраційних
пунктів. Їй, дружині військового, росіяни б цього не пробачили. Вони і так
питали на кожному блокпосту, де чоловік, на якого оформлений автомобіль.
Доводилось брехати, але проїхали. Спочатку опинились у Запоріжжі, а потім
рушили далі, на захід.

\enquote{Я не знала, куди їхати. У мене не було плану, - каже Оксана Макарова. – Я
просто написала у Фейсбуці, що ми живі, і не знаємо, що робити і куди їхати. І
на диво дуже швидко відгукнулись люди. Спочатку моя троюрідна сестра написала,
що вона у Луцьку, і можна їхати до неї. Ми дуже зрідка спілкувались до війни,
але вона одразу відгукнулась. А потім взагалі чужі люди написали, що у них під
Луцьком є великий триповерховий будинок. Правда, там вже 10 років ніхто не жив,
але будинок хороший, і вони готові нам його надати безкоштовно. Так ми з
родиною опинились у Луцьку}.

\subsubsection{Як почати життя \enquote{з  нуля} у Луцьку}

Коли Оксана Макарова трохи прийшла до тями, вона почала думати, як жити далі і
чим заробляти гроші.

\enquote{Стало зрозуміло, що все це – надовго, що швидкої перемоги не буде. І треба
якось жити тут і зараз. Тоді я стала шукати різні можливості, гранти, програми
допомоги, щоб відновити свій бізнес. Я вступила на курси СММ. Взагалі зараз
держава надає багато можливостей тим, хто хоче відновити свою діяльність в
іншому місті. Мені це все дуже допомогло.

\ii{08_09_2023.stz.news.ua.mrpl.0629.1.pochaty_zhyttja_z_nulja.pic.1}

Я отримала державний грант від \enquote{єРобота} на відкриття двох майстерень – з
виготовлення виробів з епоксидної смоли (це те, чим я займалася до декретної
відпустки) і майстерні з обробки шкіри (це для чоловіка, я сподіваюсь, після
повернення з війни він піде на заслужений відпочинок, і буде йому справа).

\ii{08_09_2023.stz.news.ua.mrpl.0629.1.pochaty_zhyttja_z_nulja.pic.2}

Але життя розпорядилося так, що у мене майже не залишається часу на улюблену
справу – епоксидку. Весь останній час я шию військове спорядження}.

Підприємство Оксани Макарової з шиття всього, що потрібно військовим - від
рюкзаків для снайперських гвинтівок до сідушек і каріматів - виникло випадково.

Влітку у відпустку з фронту приїхав чоловік. Його підрозділ переформатували,
тож був час на спілкування і на роздуми про майбутнє. А потім Олег повернувся
вже у новий підрозділ. І йому та побратимам дуже багато чого було потрібно із
військового спорядження.

\ii{08_09_2023.stz.news.ua.mrpl.0629.1.pochaty_zhyttja_z_nulja.pic.3}
\ii{08_09_2023.stz.news.ua.mrpl.0629.1.pochaty_zhyttja_z_nulja.pic.30}

\enquote{Я намагалась допомагати, але нічого не вдавалося купити. І гроші були, а
купити нема де. Кількість військових стрімко зросла в Україні, і на
підприємствах просто не встигали забезпечити потреби. А у мене вже на той
момент були дві машинки для шиття. Передали мені трохи якісної американської
тканини, і я спробувала пошити військову сумку самотужки. Вдалося, передала
чоловіку. У підрозділі побачили і стали просити і собі таке.

Так і виникла наша майстерня \enquote{Тетрапод Мрпл}.

В організації власної справи мені багато допомагали лучани. Грант я подавала
влітку, а гроші отримала з невеликою затримкою, у січні. Але знаєте, мені тут,
у Луцьку, постійно зустрічаються люди, які готові йти назустріч. Ще до того, як
я отримала гроші на оренду приміщення, мені передали приміщення. Ще грошей не
було на обладнання, а його мені вже поставили.

Навіть зараз, коли виникають проблеми зі сплатою, я кажу: \enquote{А можна я вам за 
приміщення через два місяці заплачу?} - і люди завжди йдуть назустріч}.

Оксана каже, що непросто було знайти гарних швачок. У Луцьку дуже велика
кількість швейних підприємств, конкуренція велика, багато майстринь поїхали за
кордон. Але Оксані вдалося зібрати команду.

\enquote{Зараз зі мною працюють 6 людей. Всі вони – дивовижні, дуже фахові. У нас
прекрасна команда, яка розуміє, що ми робимо і для кого. Бо до нас же хлопці
приходять, яким все треба на вчора. А нам з дівчатами ще тиждень для виконання
працювати. Ми викладаємось на 100\%, щоб відшивати все дуже швидко.

\ii{08_09_2023.stz.news.ua.mrpl.0629.1.pochaty_zhyttja_z_nulja.pic.4}

Нашим хлопцям не підходить \enquote{Китай}. Для військових потрібні якісні міцні
тканини, надійні шви. Все має бути точно за розміром до міліметра. Це важливо!
Тому ми використовуємо тільки американську фурнітуру, тканину. На щастя, і тут
я зустрічаю людей, які допомагають все закуповувати}.

Звісно, що у неї не було раніше такого досвіду, і вона не знала, як треба шити
наплічник для снайпера або штурмову панель з місцем для гідранта.

\enquote{Деякі моделі ми розробляємо самі. Наприклад, щоб пошити сумку для дронів, нам
приносили сам дрон, і ми його заміряли і робили розрахунки. А інколи допомагає
чоловік, який пояснює, що і як має виглядати}.

Оксана каже, що з тих пір, як вона перейшла на виробництво військового
спорядження, вона втратила можливість отримувати іноземні гранти на підтримку
виробництва, тому що західні партнери не фінансують військове виробництво. І з
урахуванням того, скільки всього її \enquote{Тетрапод Мрпл} постачає на фронт
безкоштовно, часто власниця майстерні опиняється у складному становищі.

\ii{08_09_2023.stz.news.ua.mrpl.0629.1.pochaty_zhyttja_z_nulja.pic.5}

\enquote{Дуже багато речей ми шиємо та поставляємо військовим безкоштовно. Хочеться ж
допомагати хлопцям щиро. І інколи через це виникають ситуації, коли я не
впевнена, а чи зможу взагалі сплатити своїм дівчатам заробітну плату (всі вони
отримують реальні, а не мінімальні зарплати, я сплачую всі податки). Звісно,
якось викручуємось. Але непросто все це. В окремі місяці у мене можуть руки
опускатися. В такі моменти я викроюю час для своєї улюбленої смоли. Це
творчість, яка рятує. У мене в роботах останнім часом з’явилось дуже багато
моря, і я розумію чому.

Коли бачу новини про корупцію у Міністерстві оборони, мене це дуже демотивує. Я
не можу зрозуміти, ну як це так – одні останнє віддають на фронт, а інші можуть
в той же час красти.

Я так не можу. Зараз я хочу лише одного – щоб ми перемогли, і мій чоловік
повернувся додому, до свого сина, якого він так чекав і через війну практично
не бачить. Я хочу, щоб моя дитина нарешті навчилася спокійно спати. Він досі не
спить нормально, кожну ніч він намагається сховатися під ковдрою від звуків
війни, які він і досі пам'ятає і чує у своїй голові... Син засинає тільки на
моєму серці. Так він засинав у Маріуполі, так засинає і зараз!

Дуже хочу, щоб в наш дім нарешті повернулось нормальне життя.

Тому я зараз працюю і не думаю про прибутки. Я працюю лише для того, щоб нашим
хлопцям на фронті було легше перемагати ворога}.

\ii{08_09_2023.stz.news.ua.mrpl.0629.1.pochaty_zhyttja_z_nulja.pic.6}
