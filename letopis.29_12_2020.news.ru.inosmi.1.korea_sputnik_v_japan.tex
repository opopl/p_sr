% vim: keymap=russian-jcukenwin
%%beginhead 
 
%%file 29_12_2020.news.ru.inosmi.1.korea_sputnik_v_japan
%%parent 29_12_2020
 
%%url https://inosmi.ru/politic/20201229/248851430.html
 
%%author 
%%author_id 
%%author_url 
 
%%tags 
%%title Yahoo news Japan (Япония): Южная Корея приглашает Путина и производит по лицензии «Спутник V»
 
%%endhead 
 
\subsection{Yahoo news Japan (Япония): Южная Корея приглашает Путина и производит по лицензии «Спутник V»}
\label{sec:29_12_2020.news.ru.inosmi.1.korea_sputnik_v_japan}
\Purl{https://inosmi.ru/politic/20201229/248851430.html}

\ifcmt
  pic https://cdn1.img.inosmi.ru/images/24865/82/248658276.jpg
\fi

\index[rus]{Вакцина!Спутник V!Россия, Корея, Япония, 29.12.2020}

\begin{leftbar}
  \begingroup
    \em\Large\bfseries\color{blue}
Российско-южнокорейское сближение: японские СМИ сообщают о том, что,
председатель парламента Южной Кореи передал спикеру Госдумы Володину
приглашение Путину посетить с визитом Сеул. Между тем, Южная Корея начала
производить по лицензии российскую вакцину «Спутник V». А вот японским
читателям такие новости не нравятся.
  \endgroup
\end{leftbar}

Как сообщает\Furl{https://news.yahoo.co.jp/articles/4a9046563559fcb5b34daaa7ed3cc766dc1e031d} крупный японский новостной портал Yahoo News Japan со ссылкой на
южнокорейскую газету JoongAng Ilbo, во время телефонной беседы с председателем
Госдумы Вячеславом Володиным 28 декабря, новый председатель южнокорейского
парламента (Национального собрания) Пак Бён Сок (Park Byeong Seug) выступил с
предложением об организации официального визита президента России Путина в
Южную Корею. Спикер южнокорейского парламента высказался также за ускорение
российско-южнокорейских переговоров по подписанию между странами Соглашения о
свободной торговле (свободном обмене товарами и услугами).

Во время беседы Пак Бён Сок особо отметил: «К сожалению, намеченные на этот год
двусторонние контакты высокопоставленных российских и южнокорейских чиновников
были перенесены из-за эпидемии коронавируса. Эти контакты остро необходимы с
точки зрения наполнения конкретным содержанием дальнейшего развития
плодотворных отношений между Южной Кореей и Россией. Мы приглашаем к себе
председателя Госдумы господина Володина, а также предлагаем рассмотреть вопрос
об осуществлении в подходящие сроки официального визита президента России
Путина в Южную Корею».

Вячеслав Володин ответил: «Независимо от процесса борьбы с коронавирусом мы
готовим мою поездку в Южную Корею. Я передам приглашение господина Пак Бён Сока
президенту Путину и расскажу о желании Южной Кореи видеть его у себя».

Председатель Госдумы России и председатель южнокорейского парламента высказали
обоюдное стремление к встрече в нынешнем году, знаменующем собой 30-летие
установления дипломатических отношений между двумя странами.

Тот же портал Yahoo News Japan со ссылкой уже на японское издание популярного
южнокорейского медиа-ресурса Wow! Korea дает новость\Furl{https://www.wowkorea.jp/news/korea/2020/1229/10282407.html} о том, что 29 декабря
рейсом южнокорейской авиакомпании Asiana из аэропорта Инчхон в Москву вылетел
грузовой борт OZ795, загруженный первой партией произведенной в Южной Корее по
лицензии России вакциной «Спутник V». Вакцину в Корее начала производить в
сотрудничестве с Россией крупная фармацевтическая компания Hangook Korus.

Речь идет о регулярной отправке в дальнейшем таких партий вакцины
южнокорейского производства в Россию. Это создаст условия для более активного и
эффективного ее распространения в мире. Asiana подписала с Hangook Korus
долгосрочный контакт о транспортном и логистическом обслуживании этих поставок
на российском направлении.

\ii{29_12_2020.news.ru.inosmi.1.korea_sputnik_v_japan.comments}


