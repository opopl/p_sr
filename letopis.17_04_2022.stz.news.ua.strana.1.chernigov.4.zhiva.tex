% vim: keymap=russian-jcukenwin
%%beginhead 
 
%%file 17_04_2022.stz.news.ua.strana.1.chernigov.4.zhiva
%%parent 17_04_2022.stz.news.ua.strana.1.chernigov
 
%%url 
 
%%author_id 
%%date 
 
%%tags 
%%title 
 
%%endhead 

\subsubsection{На входной двери написано: \enquote{Я еще жива}}

Больше всего разрушений пришлось на обычный жилой район в центре Чернигова —
улицу Черновола. Жилые многоэтажки тут превратились в памятник войне. Именно
эту бомбежку своими глазами видел футбольный тренер \enquote{Десны}.

\ii{17_04_2022.stz.news.ua.strana.1.chernigov.pic.8}

У каждого встреченного на улице человека есть своя история про 3-е марта, когда
из-за авианалета в считанные минуты погибло почти 50 человек. С самолета на
этот район сбросили шесть авиабомб. Попали в несколько жилых домов, аптеку и
больницу. 

— В 12 часов дня, в солнечный день, российские самолеты сбросили сюда шесть
авиабомб. Одна из них попала в областной кардиологический центр — он полностью
разрушен. Вторая ударила по аптеке, где в этот момент люди покупали лекарства,
— рассказывает Дмитрий Иванов, замглавы Черниговской ОГА.

Сразу после удара под завалами нашли 35 погибших, сейчас их количество возросло
до 50. Но жертв может быть больше: тела до сих пор продолжают доставать из-под
завалов. Вот на моих глазах в дом забегают спасатели: на шестом этаже 9-этажки
нашли женщину, которую задавило бетонной плитой. 

— Один мужчина в тот день подошел к больнице, ждал у порога свою жену, которая
там работала. Он позвонил ей и сказал: \enquote{Я на месте}. И тут прилетела бомба...
Этого человека мы нигде не можем найти. Он считается пропавшим без вести, —
сообщает Дмитрий Иванов. 

Припаркованные во дворах автомобили сгорели дотла. Одну из машин взрывом
перевернуло вверх дном, а сверху придавило куском бетона. 

Панельные стены в 9-этажке обрушились каменным водопадом. Под грудой бетона
погребены кусочки жизни тех, кто в один момент потерял свой дом. Вот обгоревшая
инструкция к тостеру, вот чемодан, вот картонная коробка из магазина косметики,
вот статуэтка Деда Мороза под елочку, вот чей-то ботинок, вот люстра, а вот
праздничная оберточная бумага, в которую кто-то кому-то заботливо заворачивал
подарок на день рождения.

На 8-м этаже на двух толстых проводах свисает то ли с потолка, то ли с пола
(разве теперь поймешь?) чья-то микроволновка. Так странно: стен в квартире
больше нет, а микроволновка — осталась. Теперь она колышется от сильного ветра,
как маятник, отсчитывающий время: тик-так, влево-вправо, тик-так... Для кого-то
время здесь остановилось навсегда, а для выживших — поделилось на \enquote{до} и
\enquote{после}. 

\ii{17_04_2022.stz.news.ua.strana.1.chernigov.pic.9}

На лавочке около разрушенной 9-этажки сидит мужчина и не спеша раскуривает
самокрутку. Когда сюда одна за другой прилетели шесть бомб, он был в подвале.
Его спасла чистая случайность. 

— Вышел покурить и по нужде захотелось. Думаю, поднимусь в квартиру. А сестра с
женщинами еще общалась в подвале, задерживалась. Ну, думаю подожду. Сел на
табуретку, и тут как рвануло! Аж подвал струснуло. Теперь в подвале ночуем. Из
нашего подъезда погибло 4 человека. И из того подъезда немало. На улице многих
убило, трупы лежали, — вспоминает мужчина, затягиваясь сигаретой.

— Помню, тогда как шарахнет — раз, два, и шесть штук таких! Нас в подвале много
было, человек 50. А как закончилось, чувствуем — дом трясет, в подвале пыль
стоит такая, что я тебя не вижу, а ты меня не видишь. Песок сыпался. Никогда не
забуду, как после бомбежки весь дом вышел из подвалов. Вышли, как на парад, —
пенсионерка срывается на плач, глядя на руины, в которые превратилась ее
квартира. — Остались бомжами, понимаете? Приходится кусок хлеба просить в свои
72 года. 

По ее словам, первое время после взрывов местные даже боялись к дому подходить:
\enquote{Думали, вдруг заминировано, еще подорвемся. Пугались каждого шороха}. 

Сейчас, когда в Чернигове уже долгое время тихо, местные жители начали
выбираться из укрытий и помогают убирать свой город. Заклеивают клеенкой
выбитые окна, разгребают мусор, встречают соседей, которых не видели много
дней. Высматривают в лицах прохожих своих знакомых, молясь о том, чтобы они
были живы.

\ii{17_04_2022.stz.news.ua.strana.1.chernigov.pic.10}

Многие сегодня впервые вышли посмотреть на то, во что российские бомбы
превратили их дома. 

В их числе — пенсионерка, живущая в доме напротив 9-этажки, которая пострадала
от бомбежки больше всего.

— Когда все это случилось, я была в квартире, в ванной. И мной крутило вот так,
— женщина энергично вертит руками по кругу, имитируя водоворот. — Все кружилось
вокруг меня. Если бы двери вырвало — все, хана. Я должна была умереть. Получила
контузию. Как я ее ласково называю, конфузия. Сегодня ночью бегала прятаться,
потому что опять показалось, что бомбят. Вот вам и конфузия, — говорит женщина,
вопреки войне не утратившая способность шутить. 

Ее квартира уцелела после бомбежки, но окна выбило взрывной волной. Сейчас они
заколочены досками. 

— Я как раз хотела ремонт делать. А теперь..., — женщина оглядывается на свою
квартиру. — Я вот тут на втором этаже живу, все окна забиты. Стекла вылетели от
взрывов... Я наверно, должна была от голода и холода умереть. Но не тут-то было.
На моей входной двери написано: \enquote{Я еще жива}. Потому что нечего было есть. Один
кусочек сала на три недели... 

В этом районе все жильцы знали друг друга в лицо, помогали с водой, едой,
давали, что могли.

Многие при той бомбежке потеряли близких.

— Горело все, падало, и люди еще остались под завалами, не всех достали. У меня
знакомая ушла в магазин, а дочь с мужем остались дома. А тут бомба… Завалило
обоих. Трагедия, — рассказывает пенсионерка. 

Ей уже за 80, и как многие пожилые украинцы, за свою долгую жизнь она второй
раз переживает подобное потрясение. Она родилась в 1949-м, и ее с семьей
переселили на Дальний Восток. \enquote{Вот это в начале жизни мне такое, и теперь еще и
в конце}, — говорит она, оглядывая кладбище сгоревших машин под подъездом.

Эта война лишила ее не только здоровья и покоя, но и разделила с собственной
семьей. 

— Мои дети живут в России. Они все понимают, но боятся высказываться — могут
все потерять. Дочка говорит: \enquote{Ты что от меня хочешь услышать?} А сын вообще
ничего не говорит. Вот и как мне дальше жить? Как общаться с этими детьми?
Звонит мне племянница, дочь моего брата, и говорит: \enquote{Так вы ж себя сами
бомбите, у вас там фашисты!...}. Знаете, я вообще человек культурный, у нас в
семье так принято. Но я такие маты ей рассказала, которые даже сама не знала, и
выключила телефон.

Чернигов, вопреки колоссальным разрушениям, пытается вернуться к мирной жизни.
Велосипедисты объезжают следы от разорвавшихся мин на асфальте. В центре
города, недалеко от разрушенной бомбой гостиницы \enquote{Украина}, сегодня
впервые за время войны открылось небольшое кафе. Здесь соблазнительно пахнет
кофе и свежей выпечкой. От посетителей нет отбоя. 

Каждый, кто входит в кафе, первым делом идет к витрине, где сложены три буханки
свежего ароматного хлеба. 

— Этот хлеб уже забронирован, свежий будет завтра, — извиняясь, говорит
бариста, и принимается готовить капучино. 

Посетителей становится все больше, и десять минут спустя парень уже не
справляется с количеством заказов, а очередь только растет. И тут за барной
стойкой появляется мужчина в военной форме. 

Украинский солдат вдруг становится рядом с баристой и объявляет: 

— Принимаю заказы на чай, — и подмигивает ошарашенному баристе. 

Делая первый глоток горячего чая, каждый в этом кафе думает о своем. Кто-то
вспоминает тех, кого не стало. Кто-то думает о том, что будет дальше. А кто-то
в этот момент понимает, что Чернигов — действительно \enquote{город героев}.
