% vim: keymap=russian-jcukenwin
%%beginhead 
 
%%file 17_02_2022.stz.news.ua.fraza.1.kak_zapadnyje_partnery_opuskajut_ukrainu.2.sluzhba_kvartal_95
%%parent 17_02_2022.stz.news.ua.fraza.1.kak_zapadnyje_partnery_opuskajut_ukrainu
 
%%url 
 
%%author_id 
%%date 
 
%%tags 
%%title 
 
%%endhead 

\subsubsection{«Служба безопасности Квартала 95» заявляет...}
\label{sec:17_02_2022.stz.news.ua.fraza.1.kak_zapadnyje_partnery_opuskajut_ukrainu.2.sluzhba_kvartal_95}

Обратило на себя внимание заявление СБУ о том, что новости о «вторжении Путина»
следует расценивать как «очередную волну гибридной войны».

«Сегодня отечественное информационное пространство испытывает беспрецедентное
влияние. Украина столкнулась с попытками системного нагнетания паники,
распространением фейковой информации и искажением реального положения вещей.
Все это в комплексе — не что иное, как очередная мощная волна гибридной
войны...

В то же время, мы прекрасно понимаем мотивы нынешнего информационного
нагнетания — посеять тревогу в украинском обществе, подорвать веру в
способность государства защитить своих граждан, расшатать наше единство...

Паника и дестабилизация играют в пользу только врагам, а не Украине. Мы все
должны мыслить критично и проверять любую информацию; руководствоваться данными
официальных источников, а не анонимных; научиться отличать правду от фейков», —
говорится в указанном заявлении СБУ и подчеркивается, что «такие проявления
гибридной войны» фиксируются в социальных сетях, отдельных СМИ, а также — «в
распространении некоторыми политиками нарративов страны-агрессора».

Трудно удержаться в рамках нормативной лексики, комментируя этот текст. Но все
же постараемся

Ни разу не оправдывая империалистическую политику Путина и его
похабно-демонстративные телодвижения возле наших границ в виде военных учений
именно в настоящий момент, обратим внимание на тот факт, что истерию в
информпространстве раздувала не Москва, а Запад. Подобные учения, в том числе и
на территории Беларуси, Россия проводит далеко не первый раз. В частности,
такие учения проводились и в минувшем году. Но тогда такой истерии не было Небо
над Украиной не закрывали, Черное море зоной военного конфликта не объявляли и
истерию в СМИ не раздували. Очевидно, тогда это «нагло-саксам» было невыгодно.
А может просто только сейчас додумались до того, чтобы поистерить в своих
интересах на фоне военных упражнений Москвы...

Наконец, именно сейчас Москва поставила в полный рост вопрос, как выражался
товарищ Ленин, о новом империалистическом переделе рынков сбыта и сфер влияния.
А Запад в ответ решил использовать Украину, которую не жалко. Ну, кой-какого
оружия разве что подбросили — стрелкового, противотанкового и немного
стингеров, которые от ракетных и авиаударов, как рыбке зонтик...

