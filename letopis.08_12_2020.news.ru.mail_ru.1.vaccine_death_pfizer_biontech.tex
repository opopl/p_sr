% vim: keymap=russian-jcukenwin
%%beginhead 
 
%%file 08_12_2020.news.ru.mail_ru.1.vaccine_death_pfizer_biontech
%%parent 08_12_2020
 
%%url https://news.mail.ru/incident/44474850/?frommail=1
 
%%author 
%%author_id 
%%author_url 
 
%%tags 
%%title Шесть человек умерли в ходе исследования вакцины Pfizer и BioNTech
 
%%endhead 
 
\subsection{Шесть человек умерли в ходе исследования вакцины Pfizer и BioNTech}
\label{sec:08_12_2020.news.ru.mail_ru.1.vaccine_death_pfizer_biontech}
\Purl{https://news.mail.ru/incident/44474850/?frommail=1}

В ходе исследований вакцины Pfizer и BioNTech умерли шесть участников. Об этом
во вторник, 8 декабря, сообщает The Wall Street Journal со ссылкой на
Управление по вопросам качества продовольствия и медикаментов (FDA) США.

«За отчетный период было шесть смертей (два летальных исхода в вакцинируемой
группе, четыре — в группе плацебо)», — говорится в документе, подготовленном к
заседанию FDA.

Указано, что в группе испытателей вакцины один участник, страдающий от
атеросклероза и ожирения, скончался через три дня после введения первой дозы
препарата. У другого участника произошла остановка сердца через 60 дней после
введения ему второй дозы, через три дня была зафиксирована его смерть.

В группе плацебо два человека умерли от инсульта и инфаркта, причина гибели еще
двух испытателей не установлена. Половина погибших во всех группах была старше
55 лет.

На 10 декабря назначено заседание Управления по вопросам качества
продовольствия и медикаментов США по вопросу одобрения применения вакцины,
разработанной Pfizer и германской BioNTech. Фирмы подали соответствующую заявку
в управление 20 ноября.

Вакцину от COVID-19 компании Pfizer начали доставлять на склады в США и Европе.
За день до этого президент США Дональд Трамп заявил, что поставки вакцины от
коронавируса в Соединенных Штатах начнутся на следующей неделе.

20 ноября компании BioNTech и Pfizer объявили, что финальные испытания
разработанной ими вакцины показали ее эффективность на 95\%. В начале этого же
месяца BioNTech и Pfizer рассказали об успешной третьей фазе клинических
испытаний разработанной ими вакцины BNT162b2 от коронавируса.

Тогда же глава Европейской комиссии Урсула фон дер Ляйен заявляла о намерении
ЕС подписать контракт с ними на закупку 300 млн доз вакцины от коронавируса.
