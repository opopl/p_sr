% vim: keymap=russian-jcukenwin
%%beginhead 
 
%%file 07_07_2020.stz.news.ua.mrpl_city.1.mrpl_teatralnyj_z_mtb
%%parent 07_07_2020
 
%%url https://mrpl.city/blogs/view/mariupol-teatralnij-z-mtb
 
%%author_id demidko_olga.mariupol,news.ua.mrpl_city
%%date 
 
%%tags 
%%title Маріуполь театральний з МТБ
 
%%endhead 
 
\subsection{Маріуполь театральний з МТБ}
\label{sec:07_07_2020.stz.news.ua.mrpl_city.1.mrpl_teatralnyj_z_mtb}
 
\Purl{https://mrpl.city/blogs/view/mariupol-teatralnij-z-mtb}
\ifcmt
 author_begin
   author_id demidko_olga.mariupol,news.ua.mrpl_city
 author_end
\fi

Дуже добре пам'ятаю, коли вперше закохалася в театр. Мені було 10 років і я
дивилася виставу \enquote{Приборкання норовливої}, при цьому  намагалася запам'ятати
кожну деталь. Того вечора я пообіцяла собі, що почну вивчати історію рідного
театру та вивчу імена всіх сучасних маріупольських режисерів і акторів. Вже
тоді я зрозуміла, що театральне мистецтво вимагає багато емоцій, сил та
величезної роботи актора над собою. 

Так сталося, що моя закоханість і з роками не пройшла. Спочатку вдалося видати
монографію \enquote{Ілюстрована історія театральної культури Маріуполя}, потім відкрити
арт-центр \enquote{Театр всередині тебе}, де проводилися творчі зустрічі і
майстер-класи з маріупольськими акторами і режисерами. У  2019 році я нарешті
захистила дисертацію, присвячену театральному життю Приазов'я. А тепер в \href{https://archive.org/details/01_07_2020.olga_demidko.mrpl_city.mne_eto_vazhno_poverjte_poslushajte}{День
народження єдиної  народної артистки України в Марі\hyp{}уполі Світлани Іванівни
Отченашенко}%
\footnote{\enquote{Мне это важно! Поверьте, послушайте}, Ольга Демідко, mrpl.city, 01.07.2020, \par%
\url{https://mrpl.city/blogs/view/mne-e-to-vazhno-poverte-poslushajte}, \par%
Internet Archive: \url{https://archive.org/details/01_07_2020.olga_demidko.mrpl_city.mne_eto_vazhno_poverjte_poslushajte}
}
розпочинаю нову рубрику на Маріупольському телебаченні \enquote{Маріуполь
театральний}. Такої рубрики не вистачало багатьом театралам.  І я розумію, що
це велика відповідальність, адже в Маріуполі наразі працюють як професійні, так
і самодіяльні театральні колективи, кожен з яких відрізняється неповторністю  і
унікальністю.

В інтерв'ю до 75-річчя Світлана Іванівна підкреслила, що в Маріуполі
відчувається особливе ставлення до театру. І це дійсно так. Можливо, це
обумовлено особливостями менталітету маріупольців. Можливо, тому що театр в
нашому місті є найстарішим на Лівобережній Україні... Але точно відомо, що
безліч цікавих і унікальних подій з театрального життя заслуговують більшої
уваги. Рубрика \enquote{Маріуполь театральний} створена для того, щоб театр став
ближчим для кожного маріупольця. Вона присвячена як театральним діячам міста,
так і ексклюзивним залаштункам маріупольських театрів.

Вистави маріупольських театральних колективів неодноразово давали відповіді
багатьом містянам на якісь внутрішні питання, допомагали відволіктися від
якихось проблем і робили життя трохи яскравішим. Думаю, настав час
познайомитися ближче, з тими, хто надихає і на власному приклад доводить, що
мистецтво здатне лікувати...

Дуже сподіваюся, що вдасться налагодити і зворотний зв'язок з глядачами, думки
яких завжди будуть враховуватися. Вже вийшли анонс і перший сюжет рубрики. В
анонсі взяли участь представники різних театральних колективів міста, а перший
сюжет містить ексклюзивний матеріал. Вперше видатна маріупольська артистка, що
сьогодні святкує свій ювілей, розповідає про особисте ставлення до Маріуполя,
який сьогодні вважає найбільш рідним і неповторним містом. Ідей для наступних
сюжетів безліч, планів ще більше. Залишається сподіватися, що \enquote{Маріуполь
театральний} знайде свого глядача і матеріали цієї рубрики стануть цікавими і
корисними не тільки для маріупольців, а й для театралів з різних міст України.
