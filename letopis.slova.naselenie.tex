% vim: keymap=russian-jcukenwin
%%beginhead 
 
%%file slova.naselenie
%%parent slova
 
%%url 
 
%%author 
%%author_id 
%%author_url 
 
%%tags 
%%title 
 
%%endhead 
\chapter{Население}
\label{sec:slova.naselenie}

%%%cit
%%%cit_head
%%%cit_pic
%%%cit_text
Зацитируем этот небольшой, но очень показательный документ из восьми пунктов.
Украинцы в нем определяются эпитетом \enquote{так называемые}. 
"Многоуважаемый партейгеноссе Розенберг!
По поручению фюрера я довожу до Вашего сведения его пожелание, чтобы Вы
соблюдали и проводили в жизнь в политике на оккупированных восточных
территориях следующие принципы.
\begin{itemize}
\item 1. Мы можем быть только заинтересованы в том, чтобы сокращать прирост \emph{населения}
оккупированных восточных областей путем абортов. Немецкие юристы ни в коем
случае не должны препятствовать этому. По мнению фюрера, следует разрешить на
оккупированных восточных территориях широкую торговлю предохранительными
средствами. Ибо мы нисколько не заинтересованы в том, чтобы \emph{ненемецкое
население} размножалось.
\item 2. Опасность, что население оккупированных восточных областей будет
размножаться сильнее, чем раньше, очень велика, ибо само собою понятно, его
благоустроенность пока намного лучше. Именно поэтому мы должны принять
необходимые меры против размножения ненемецкого \emph{населения}.
\item 3. Поэтому ни в коем случае не следует вводить немецкое обслуживание для
\emph{местного населения} оккупированных восточных областей. Например, ни при каких
условиях не должны производиться прививки и другие оздоровительные мероприятия
для \emph{ненемецкого населения}
\end{itemize}
%%%cit_comment
%%%cit_title
  \citTitle{22 июня - 80 лет нападения на СССР. Что немцы готовили для украинцев}, Максим Минин, strana.ua, 22.06.2021
%%%endcit


%%%cit
%%%cit_head
%%%cit_pic
%%%cit_text
В последнее время, всё чаще замечаю, что разные этно-политические (давайте
называть вещи своими именами) группы \emph{населения} Украины живут в разных
информационных и культурных реальностях, которые почти не пересекаются друг с
другом. Например, когда мы празднуем день рождения Булгакова, другая группа
молчит, так как в их культурной парадигме это просто обычный день. Точно так же
и нам наплевать, когда они отмечают день рождения какого-нибудь очередного
петлюровца, греко-католического епископа, или нацистского коллаборанта.
Последнее, конечно, вызывает возмущение, но больше по остаточному принципу. Все
эти действия происходят в другом, чуждом для нас, информационном и культурном
контексте. Контексте, от которого мы сознательно или подсознательно себя
отгородили, точно так же, как и его обитатели сознательно отгородили себя от
нашего контекста. Мы ходим по одним улицам, но живём в разных мирах
%%%cit_comment
%%%cit_title
\citTitle{Разные группы украинцев живут в разных информационных и культурных реальностях / Лента соцсетей / Страна}, 
Даниил Богатырев, strana.ua, 05.07.2021
%%%endcit

%%%cit
%%%cit_head
%%%cit_pic
%%%cit_text
Відносна слабкість громадянського суспільства в Україні пов’язана з тим, що в
країні існує аморальна більшість і моральна меншість, що їй протистоїть. Таку
диспозицію підтвердили вибори 2019 року. Виявилось, що у нас лише трохи більше
чверті \emph{населення} готові брати на себе відповідальність за свою державу, не
купуючись на порожні фантастичні обіцянки величезних життєвих благ, що впадуть
на голови громадян з неба, бо ніхто не пояснював, хто заплатить за банкет,
звідки все візьметься…
%%%cit_comment
%%%cit_title
\citTitle{Чи виникне  в Україні моральна більшість?}, 
Ігор Лосєв, day.kyiv.ua, 25.10.2021
%%%endcit

%%%cit
%%%cit_head
%%%cit_pic
%%%cit_text
В итоге по каждой проблеме народ впадает в истерику и поляризуется. Бесконечные
помаранчевые и белосиние, ватники и вышиватники, украиномовни и русскоязычные,
ваксеры и антиваксеры, те, кто за Усика и против Усика...  Манипулировать
\emph{населением} таким образом можно. Захватить и какое-то время удерживать
власть тоже можно. Эффективно управлять государством и построить цивилизованное
общество - нельзя!  Конспирология для Украины - избыточная роскошь. Никто не
способен навредить нам больше, чем мы сами себе.  Даже мировое правительство и
антихрист.  В стране объявляют карантин и красные зоны, но ничего из озвученных
мер не работает. Потому что не ради борьбы с вирусом были все эти истерики о
переполненых больницах, о тысячах случаев, когда несознательные и дикие
невакцинированые варвары вымирают целыми семьями, успев перед смертью заразить
миллионы!
%%%cit_comment
%%%cit_title
\citTitle{По каждой проблеме украинский народ впадает в истерику и поляризуется / Лента соцсетей / Страна}, 
Владислав Михеев, strana.news, 02.11.2021
%%%endcit

%%%cit
%%%cit_head
%%%cit_pic
\ifcmt
  pic https://strana.news/img/forall/u/0/36/249750605_1712255378973943_4688672848653586933_n.jpg
  @width 0.4
\fi
%%%cit_text
Никаких доказательств того, что парень первым бросился на людей с дубинками и в
форме, "стражники" не привели. Как и доказательств опьянения. Также ничего не
известно о том, чтобы против парня открывали дело за нападение.  Да и слабо
верится, что молодой человек будет голыми руками атаковать наряд "стражников".
После появления резонансного видео в "Мунварте" решили отреагировать на него.
Сегодня там опубликовали пост-ответ. В нем избитого парня назвали "животным" и
пообещали и впредь "жестко реагировать".  "В последние дни сетью
распространяется древнее обрезанное видео, где муниципалы вроде бы зверски
пытают маленького мальчика в трусиках за то, что он "просто поссал".  Хотим
успокоить любителей животных. Оно живо и здорово (в отличие от муниципала,
получившего травмы за то, что не позволил сцыкуну вести себя, как животное на
людях).  Вместо этого сознательную часть \emph{населения} нашего города мы
хотим заверить, что и дальше будем жестко реагировать на тех, кто будет
пытаться совершить насилие в отношении вас, а тем более в отношении наших
сотрудников, которые помешали правонарушению"
%%%cit_comment
%%%cit_title
\citTitle{Как члены Муниципальной варты Киева избили дубинкой киевлянина}, 
Анна Копытько, strana.news, 04.11.2021
%%%endcit

