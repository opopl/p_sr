% vim: keymap=russian-jcukenwin
%%beginhead 
 
%%file slova.devushka
%%parent slova
 
%%url 
 
%%author 
%%author_id 
%%author_url 
 
%%tags 
%%title 
 
%%endhead 
\chapter{Девушка}
\label{sec:slova.devushka}

%%%cit
%%%cit_pic
%%%cit_text
В Харькове 16-летний подросток жестоко избил \emph{девушку} кулаками по лицу,
находившиеся рядом люди никак не пытались ему помещать. Теперь подростку грозит
судимость.  Об этом сообщил pамминистра внутренних дел Антон Геращенко на своей
странице в Facebook.  Геращенко обнародовал видео инцидента. На кадрах видно,
как парень подходит к \emph{девушке} и бьет ее кулаками по лицу, она падает на асфальт
и пытается прикрыть голову руками, но его это не останавливает и он продолжает
ее избивать. При этом рядом на лавке сидят трое человек, еще несколько ходят
возле них, но никто не пытается защитить \emph{девушку}.  Подростка зовут  Тараник
Александр Игоревич, он родился 17 августа 2004 года. По словам замминистра, его
поймали сверстники и заставили на коленях просить прощения у избитых \emph{девушек}
%%%cit_comment
%%%cit_title
  
%%%endcit

%%%cit
%%%cit_pic
%%%cit_text
Месяц на небе, звезды сияют
Месяц на небе, звёзды сияют,
Тихо по морю, лодка плывёт.
\emph{Девушка} в лодке песнь напевает,
Казак услышит, сердце замрёт.
Та песня люба, та песня мила,
Всё про свиданья, всё про любовь.
Как мы любили и разошлись мы,
Теперь сошлись мы навеки вновь.
Ой, очи, очи, \emph{очи девичьи},
Темны вы как ночка, ясны как день.
Вы же мне очи жизнь сократили,
Где же научились сживать вы людей?
Месяц на небе, звёзды сияют,
Тихо по морю, лодка плывёт.
\emph{Девушка} в лодке песнь напевает,
Казак услышит, сердце замрёт
%%%cit_comment
%%%cit_title
\citTitle{Украинские Песни Русскими Словами}, 
БРАТИНА, zen.yandex.ru, 15.12.2020
%%%endcit

%%%cit
%%%cit_head
%%%cit_pic
%%%cit_text
История харьковской студентки Маргариты, которая пожаловалась, что ей запрещают
писать курсовую работу на русском языке, обрастает скандальными подробностями.
В соцсетях \emph{девушке} начали угрожать. \enquote{Стране} также стало известно, что после
записанного ролика Маргариту хотят уволить с работы.  \emph{Девушка} оказалась
сотрудницей Центра экстренной медицинской помощи Харьковской области. После
скандала руководство вызвало ее и требует писать заявление об уходе, иначе
угрожают обратиться в полицию и уволить по статье.  Интересно, что сегодня в
Раде появился законопроект \enquote{слуг народа} об уголовной ответственности за
надругательство над украинским языком. Документ направлен в том числе против
людей, которые высказывают недовольство украинизацией в социальных сетях – и за
последнее время таких недовольных в Украине все больше
%%%cit_comment
%%%cit_title
\citTitle{Языковой скандал с харьковской студенткой Маргаритой Мальцевой – все подробности}, 
Екатерина Терехова; Юлия Корзун, strana.ua, 18.06.2021
%%%endcit

%%%cit
%%%cit_head
%%%cit_pic
\ifcmt
  pic https://img.strana.ua/img/article/3393/pozhalovavshujusja-na-ukrainskij-10_main.jpeg
	caption Девушку хотят уволить за ее высказывания о языке. Скриншот из видео.
\fi
%%%cit_text
Запись разговора \emph{девушки} с директором центра Виктором Забаштой оказалась в
распоряжении \enquote{Страны}.  \enquote{Россия - действительно, страна,
которая показала себя не с лучшей стороны, это вражеская для нас страна, и язык
ее вражеский...  Просьба одна: идите с богом, туда, где не нужно изучать язык,
где не нужно вести документацию на государственном языке. Я с удивлением узнал,
что вы поступили в вуз и не знали, что там украинский язык, и к нам тоже...
Чтобы мы не были вынуждены обращаться в Нацполицию...} - сказал Забашта
%%%cit_comment
%%%cit_title
\citTitle{Пожаловавшуюся на украинский харьковскую студентку хотят уволить и угрожают полицией}, 
Наталья Полулях, strana.ua, 18.06.2021
%%%endcit

%%%cit
%%%cit_head
%%%cit_pic
%%%cit_text
А се, певно, була немала похвала, бо Тугар Вовк був мужчина, як дуб. Плечистий,
підсадкуватий, з грубими обрисами лиця і грубим, чорним волоссям, він і сам
подобав на одного з тих злющих тухольських медведів, яких їхав воювати. Але ж
бо й донька його Мирослава була \emph{дівчина}, якої пошукати. Не кажемо вже про її
уроду й красу, ані про її добре серце — в тім згляді багато її ровесниць могло
стати з нею нарівні, хоч і небагато могло перевищити її. Але в чім не мала вона
пари між своїми ровесницями, так се в природній свободі свого поводження, в
незвичайній силі мускулів, у смілості й рішучості, властивій тільки мужчинам,
що виросли в ненастанній боротьбі з супротивними обставинами. Зараз з першого
разу видно було, що Мирослава виросла на свободі, що виховання її було мужеське
і що в тім прегарно розвиненім \emph{дівочім} тілі живе сильний, великими здібностями
обдарований дух. Вона була в батька одиначка, а до того ще зараз при народженню
втратила матір
%%%cit_comment
%%%cit_title
\citTitle{Захар Беркут}, Іван Франко
%%%endcit

%%%cit
%%%cit_head
%%%cit_pic
\ifcmt
  tab_begin cols=2

     pic https://avatars.mds.yandex.net/get-zen_doc/4581585/pub_617da6d6eaf6465490b6ebfe_617e283593121f356cbac567/scale_1200

     pic https://avatars.mds.yandex.net/get-zen_doc/964926/pub_617da6d6eaf6465490b6ebfe_617e1cdb7971f707931a3867/scale_1200

  tab_end
\fi
%%%cit_text
Несколько минут до начала соревнований у \emph{девушек}. Я пока не знаю, как
сложатся прокаты. Но не могу не сказать о судействе в короткой программе. Сразу
скажу оно было жёстким. И справедливым. Но, как и в мужских стартах, не для
всех. Для кого-то слишком лояльно. И лояльное судейство было для Лизы
Туктамышевой.  В заявках, на исполняемый технический контент, произошли
"запланированные" изменения. Они, как по мне были очень предсказуемы. Алиса Лью
убрала четверной лутц из прыжкового контента на ПП, оставив один тройной
аксель. Видимо увидев, какие здесь судьи, фигуристка решила не рисковать
%%%cit_comment
%%%cit_title
\citTitle{Skate Canada. Девушки. Произвольная программа. Камила Валиева обновила мировой рекорд}, 
Непредвзятый спорт, zen.yandex.ru, 31.10.2021
%%%endcit

%%%cit
%%%cit_head
%%%cit_pic
%%%cit_text
Украинцы продолжают пользоваться всем, что придумано, построено и произведено в
СССР. Донашивают старые одёжки, основные фонды: заводы, шахты, электростанции,
школы и больницы. Надолго ещё хватит? Европы помогут? Мечты юной девственницы.
Опьянённой обёрткой брутального и богатого Запада.  Но ведь не вышло. Негодяй
вынес из дома \emph{девушки} всё ценное. Обрюхатил четырьмя детьми: коррупцией,
войной, некомпетентностью власти и дефолтом. Наградив ещё неприличной болезнью,
недолеченной Нюрнбергским трибуналом
%%%cit_comment
%%%cit_title
\citTitle{Как Украина была колонией СССР... Не наоборот ли?}, 
Исторические напёрстки, zen.yandex.ru, 02.11.2021
%%%endcit
