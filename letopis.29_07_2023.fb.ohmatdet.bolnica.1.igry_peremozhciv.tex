%%beginhead 
 
%%file 29_07_2023.fb.ohmatdet.bolnica.1.igry_peremozhciv
%%parent 29_07_2023
 
%%url https://www.facebook.com/ndslohmatdyt/posts/pfbid0mkMou6itHxisK61iLypFZvXhsUo4PrMCk6hbSsFGA6cb6Xd1ehYGp86ojAquLcY5l
 
%%author_id ohmatdet.bolnica
%%date 29_07_2023
 
%%tags 
%%title Ігри переможців: в Охматдиті започаткували нову традицію та встановили рекорд України
 
%%endhead 

\subsection{Ігри переможців: в Охматдиті започаткували нову традицію та встановили рекорд України}
\label{sec:29_07_2023.fb.ohmatdet.bolnica.1.igry_peremozhciv}

\Purl{https://www.facebook.com/ndslohmatdyt/posts/pfbid0mkMou6itHxisK61iLypFZvXhsUo4PrMCk6hbSsFGA6cb6Xd1ehYGp86ojAquLcY5l}
\ifcmt
 author_begin
   author_id ohmatdet.bolnica
 author_end
\fi

🇺🇦Ігри переможців: в Охматдиті започаткували нову традицію та встановили рекорд України. 

🔹Сьогодні, 29 липня, в Національній  дитячій спеціалізованій лікарні
\enquote{Охматдит} відбулися \enquote{Ігри переможців} — спортивне свято для дітей, які
перемогли важкі захворювання. Кожна наша дитина — переможець, герой та великий
приклад для нас усіх. 

🔸У рамках події був встановлений рекорд України — наймасовіший забіг дітей,
які подолали тяжкі хвороби. Рекорд зафіксував представник національного реєстру
рекордів України.💙💛

🔹Підтримати дітей завітало багато гостей: спортсмени, волонтери, меценати, артисти, благодійники, команда лікарні. 

Дякуємо кожному, хто долучився до створення події для дітей! Дякуємо за ці щасливі дитячі очі 🙏

🔸У змаганнях взяли участь діти, які подолали рак, після трансплантацій
кісткового мозку, діти з військовою травмою, рідкісними та хронічними
захворюваннями, діти, яким видалили пухлини мозку, пацієнти після інсультів. 

Всі ці дітки надзвичайно сильні, вони перемогли різні хвороби та повернулися до
звичайного життя. На \enquote{Іграх переможців} вони продемонстрували свої здібності,
активність та жагу до нових звершень.❤️

🔹В ході заходу діти виконували символічний забіг територією лікарні в компанії
генерального директора Охматдиту Володимира Жовніра та іменитих спортсменів,
зокрема олімпійських чемпіонів, міністра молоді і спорту і президента НОК
України Вадима Гутцайта, чемпіона світу Володимира Осніка, фристайліста
Олександра Абраменко, знаменитого футболіста і тренера Олега Протасова, також
до події долучились досвідчені хокеїсти Сергій Бабинець, Роман Благий і Артем
Євсейчик, підтримав подію також заступник командувача Об'єднаних Сил ЗСУ
полковник Євгеній Коваленко, Михайло Радуцький — голова комітету Верховної Ради
України з питань здоров'я нації та багато інших небайдужих, світлих і
талановитих людей!💫

🔸Пізніше спортсмени провели діткам майстер-класи. Серед спортивних локацій
були такі: футбол, хокей, карате, петанк, також були творчі майстер-класи і
аквагрим для бажаючих. Після спортивних змагань на всіх дітей чекали подарунки,
медалі та святковий концерт!👏

Дякуємо кожному, хто підтримує нашу лікарню! Разом з вами ми робимо все можливе для здоров'я і щастя наших дітей!🤝

📸 Більше фото і відео за посиланням: \url{https://photos.app.goo.gl/neYEL91J4KoAaBNX7}

📸 А тут фото від фотографки Ольги Руденко: \url{https://fex.net/ru/s/2ndot8z}

Командування Об'єднаних Сил ЗС України ДУ \enquote{Український медичний центр спортивної медицини Мінмолодьспорту України}
