% vim: keymap=russian-jcukenwin
%%beginhead 
 
%%file 27_09_2021.fb.nikonov_sergej.2.bilchenko_feminizm_kouching
%%parent 27_09_2021
 
%%url https://www.facebook.com/alexelsevier/posts/1590988594579783
 
%%author_id nikonov_sergej,bilchenko_evgenia
%%date 
 
%%tags bilchenko_evgenia,feminizm,poezia
%%title БЖ. Феминистический коучинг
 
%%endhead 
 
\subsection{БЖ. Феминистический коучинг}
\label{sec:27_09_2021.fb.nikonov_sergej.2.bilchenko_feminizm_kouching}
 
\Purl{https://www.facebook.com/alexelsevier/posts/1590988594579783}
\ifcmt
 author_begin
   author_id nikonov_sergej,bilchenko_evgenia
 author_end
\fi

\ifcmt
  ig https://scontent-yyz1-1.xx.fbcdn.net/v/t1.6435-9/243286961_1590987594579883_3461028784884720477_n.jpg?_nc_cat=108&ccb=1-5&_nc_sid=730e14&_nc_ohc=M5gBAlHxAtgAX9mA0M5&_nc_ht=scontent-yyz1-1.xx&oh=0c793093d246446d3263517c2dd5cc60&oe=61792C87
  @width 0.4
  %@wrap \parpic[r]
  @wrap \InsertBoxR{0}
\fi

Еще один стих поэтессы с подписью БЖ. На сей раз о семье. 
БЖ. Феминистический коучинг
Прочитала я, значит, совет от питерского психолога:
Как вернуть интерес мужа к жене, если оной холодно:
Затянуть поясок потуже, сделать фигурку груши,
Перед отходом ко сну с особым, блин, альтруизмом поговорить с мужем.
Пойти без ведома благоверного в парикмахерскую за дредами.
Заказать себе платье вместо сапог пролетарских брендовое.
Обрести новую славу за счёт экзотичных хобби.
На худой конец, начать разбираться в хоббитах.
А Христос говорил, что любят отнюдь не за популярность
И даже не за умение звезду доставать полярную.
Не за одёжу новую, не за причёску модную,
Не за искренность в виде трюка и не за стать свободную.
И поверила я Христу - больше даже, чем мужу.
Пуд грима не наложила, не сделала талью уже,
Не выменяв честь на должность, я даже и не накрасилась,
И на рекомендуемое психологом число поцелуйных разиков
Благоверного перед сном наплевала почти что хамски я.
Наверное, я - не жена, не нормальная баба, а рэпер Хаски:
Живая собака скороговорки, говорят, что с поэзией невротической,
Но я-то знаю, что я - птичка, Божья птичка, шизофреничка.
А птичке надобно только петь. Петь и клевать по лбу
Тех, кто рад меня закопать в фантиковом гробу,
Как в детстве мальчишки-садисты - с шейкой сломанной воробья...
Вот и вся психология этого - как его, мать его? - Бытия.
27 сентября 2021 г.
