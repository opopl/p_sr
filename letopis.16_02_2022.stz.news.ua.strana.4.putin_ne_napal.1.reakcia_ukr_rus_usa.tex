% vim: keymap=russian-jcukenwin
%%beginhead 
 
%%file 16_02_2022.stz.news.ua.strana.4.putin_ne_napal.1.reakcia_ukr_rus_usa
%%parent 16_02_2022.stz.news.ua.strana.4.putin_ne_napal
 
%%url 
 
%%author_id 
%%date 
 
%%tags 
%%title 
 
%%endhead 

\subsubsection{Как отреагировали в Украине, России и США}

Украина. Владимир Зеленский с утра записал видео, в котором поздравил всех с
новым праздником — Днем единения украинцев, который он накануне и учредил. Он
отметил, что все украинцы разные, но \enquote{объединяет нас единое желание – жить,
жить в мире, жить счастливо}.

\enquote{И на все это мы имеем полное право, потому что мы дома, мы в Украине. И никто
наш дом не будет любить так, как мы любим. И только мы вместе сможем наш дом
защитить. С днем единения, сине-желтые}, - сказал президент.

\ifcmt
  ig https://i2.paste.pics/0549b35d0848b5067ebaf1ccba95156d.png
  @wrap center
  @width 0.8
\fi

Итак, ночью Путин не напал, как обещали западные СМИ. Они уже возложили вину за
провал своего прогноза на «загадочность» Путина, который и «дальше заставляет
Запад теряться в догадках». Россия, впрочем, регулярно отрицала, что собирается
напасть на Украину.

Но, тем не менее, украинские власти уже готовы праздновать сегодня «перемогу».

Зеленский, объявивший сегодня день единения, уже с утра записал видео, в
котором сиял от радости и поздравил всех с праздником.

А на телеканалах по разнарядке Банковой уже запущен «марафон единения». Где
Зеленский и Ко будут рассказывать как «оставили агрессию» и «объединили народ».
«Вторжение» не состоялось, но пиар на нем никто не отменял.

\url{t.me/stranaua/24931}

История с новым праздником - это попытка Зеленского оседлать тему \enquote{вторжения} и
запустить \enquote{единение нации} вокруг своей персоны. То есть это чистый пиар,
который особенно показательный с учетом того, что ранее он многократно
опровергал угрозу этого самого \enquote{вторжения}. 

Россия. Утром в день \enquote{вторжения}, которого не случилось, официальный
представитель МИД РФ Мария Захарова выступила с троллинг-заявлением:

\enquote{Просьба к средствам массовой дезинформации США и Британии Bloomberg, The New
York Times, The Sun и т.д. - огласите график наших \enquote{вторжений} на предстоящий
год. Хотелось бы спланировать отпуск}, — написала Захарова в фейсбуке. 

\ifcmt
  ig https://strana.news/img/forall/u/10/91/%D0%A1%D0%BD%D0%B8%D0%BC%D0%BE%D0%BA_%D1%8D%D0%BA%D1%80%D0%B0%D0%BD%D0%B0_2022-02-16_%D0%B2_11.08_.47_.png
  @wrap center
  @width 0.8
\fi

Постоянный представитель России при Евросоюзе Владимир Чижов с утра также
заявил, что Россия не планирует "вторжение в Украину" в среду. По его словам,
эскалации ситуации не предвидится также на следующей неделе, через неделю, или
в следующем месяце.

"Я могу заверить, что в эту среду атаки не будет. На следующей неделе, или
через неделю, или в следующем месяце эскалации также не предвидится. Войны в
Европе редко начинаются в среду", - сказал Чижов в комментарии немецкому каналу
Welt.

Запад. На Западе, несмотря на залет с несостоявшимся \enquote{вторжением Путина},
продолжают его предсказывать. Но только уже \enquote{с открытой датой}.

Министр обороны Великобритании Бен Уоллес заявил, что бесполезно рассуждать о
датах возможного \enquote{вторжения} России в Украину.

\enquote{Я никогда не называл даты, потому что во всех разведывательных материалах,
которые я видел, самым важным являются не даты, а огромное количество войск (у
границы Украины - Ред.)}, - сказал Уоллес в интервью телеканалу Sky News.

То есть теперь все переходит в режим "Путин может напасть в любой день".

Об этом же заявили и в США. Госсекретарь США Блинкен теперь ждет агрессию
России \enquote{на этой неделе}.
 
\enquote{Акт агрессии на этой неделе абсолютно возможен, когда мы смотрим на то, что
происходит с размещением российских сил на территориях вокруг Украины}, -
заявил Блинкен.

Как оправдываются западные СМИ. Таблоид The Sun, который накануне заявил, что
\enquote{вторжение} состоится в три часа ночи, отредактировал свою знаменитую статью.
Теперь там говорится, что ночь, когда должна была состояться атака России,
прошла без происшествий.

\enquote{Холодное ясное небо над Киевом, где люди приготовились к воздушному налету,
оставалось безмолвным}, - констатирует газета.

Отсутствие атаки The Sun объяснил просто - \enquote{Путин и дальше заставляет Запад
теряться в догадках}. \enquote{Но напряжение остается высоким}, - заключает издание.

\ifcmt
  ig https://strana.news/img/forall/u/10/91/photo_2022-02-16_05-04-52.jpg
  @wrap center
  @width 0.8
\fi

Китай. В Китае заявили, что раздувание США угрозы войны нанесло удар по
экономике и социальной стабильности Украины. Об этом заявил официальный
представителяь МИД КНР.

\enquote{В течение последних нескольких дней американская сторона раздувала угрозу
войны, искусственно создавала напряженную атмосферу, что нанесло серьезный удар
по экономике, социальной стабильности и условиям жизни народа Украины, а также
усилило сопротивление в продвижении переговоров и диалога соответствующих
сторон}, - заявил официальный представитель МИД КНР Ван Вэньбинь.

