% vim: keymap=russian-jcukenwin
%%beginhead 
 
%%file 26_04_2023.stz.news.ua.donbas24.1.teatr_conception_linii_zhyttja
%%parent 26_04_2023
 
%%url https://donbas24.news/news/teatr-avtorskoyi-pjesi-conception-predstaviv-poeticnu-imprezu-liniyi-zittya
 
%%author_id demidko_olga.mariupol,news.ua.donbas24
%%date 
 
%%tags 
%%title Театр авторської п'єси Conception представив поетичну імпрезу "Лінії життя"
 
%%endhead 
 
\subsection{Театр авторської п'єси Conception представив поетичну імпрезу \enquote{Лінії життя}}
\label{sec:26_04_2023.stz.news.ua.donbas24.1.teatr_conception_linii_zhyttja}
 
\Purl{https://donbas24.news/news/teatr-avtorskoyi-pjesi-conception-predstaviv-poeticnu-imprezu-liniyi-zittya}
\ifcmt
 author_begin
   author_id demidko_olga.mariupol,news.ua.donbas24
 author_end
\fi

\ifcmt
  ig https://i2.paste.pics/905312eeee8582160ded05cfa43b8b4e.png
  @wrap center
  @width 0.9
\fi

\begin{center}
  \em\color{blue}\bfseries\Large
  Маріупольський театр у столиці виступив з новою постановкою
\end{center}

25 квітня у Києві в Національному музеї літератури відбулася чергова прем'єра
\href{https://donbas24.news/news/mariupolskii-teatr-conception-pokazav-u-ternopoli-svoyu-vistavu}{Театру
авторської п'єси \enquote{Conception}}, який продовжує підтримувати та
популяризувати культуру Маріуполя. Цього разу актори представили театралізовану
поетичну імпрезу авторів міста Марії \enquote{Лінії життя}.

\textbf{Читайте також:} \href{https://donbas24.news/news/u-kijevi-vidnovili-mariupolsku-vistavu}{%
У Києві відновили маріупольську виставу (ФОТО), Ольга Демідко, donbas24.news, 02.12.2022}

\begin{leftbar}
	\begingroup
		\bfseries
\enquote{Це чуттєва подорож у світ поезії та театру, що розкриває глибокі переживання міста Марії під час війни}, — йшлося в анонсі заходу. 
	\endgroup
\end{leftbar}

Зі сцени лунали вірші відомого поета та письменника, композитора, доктора
соціологічних наук, професора Маріупольського державного університету Богдана
Слющинського, який загинув у Маріуполі. Дуже вразили глядачів і щемливі вірші
актора театру Дмитра Гриценка. Також у програму увійшли вірші маріупольських
поетес Оксани Стоміної, Марії Сладкової, Олени Мирошниченко, Ірини Громової та
Ольги Демідко. Завдяки віршам цих авторів і театралізованому дійству актори
змогли майстерно передати настрій міста, його красу та складність, показавши,
як війна впливає на життя людей та як важливо зберігати людськість у будь-яких
умовах.

\ii{26_04_2023.stz.news.ua.donbas24.1.teatr_conception_linii_zhyttja.pic.1}

\begin{leftbar}
	\begingroup
		\bfseries
\qbem{Акторам було легко працювати з текстами маріупольських авторів, дуже
легкий слог, відчувалось, що писали люди з нашого регіону. Загалом для
театру це дуже цікавий досвід, адже це не просте прочитання віршів,
була і акторська робота, тому показана імпреза може вважатися
повноцінною виставою}, — зазначив режисер Театру авторської п'єси
\enquote{Conception} Олексій Гнатюк.
	\endgroup
\end{leftbar}

\textbf{Читайте також:} \href{https://donbas24.news/news/mariupolski-aktori-zigrali-blagodiinu-vistavu-v-kijevi-foto}{%
Маріупольські актори зіграли благодійну виставу в Києві (ФОТО), Артем Батечко, donbas24.news, 17.03.2023}

\ii{26_04_2023.stz.news.ua.donbas24.1.teatr_conception_linii_zhyttja.pic.2}

Актори Театру авторської пєси \enquote{Сonception} чутливо, щиро і правдиво читали
вірші маріупольських авторів, які акумулювали свій біль і горе від війни у
римовані рядки. Поетична імпреза \enquote{Лінії життя} змогла розкрити всю глибину
трагедії окупованого міста. Ці вірші дійсно чіпляють за живе, адже вони
розповідають правду про цю страшну, болючу і жорстоку війну.

\begin{leftbar}
	\begingroup
		\bfseries
\qbem{Дуже вдячні всьому колективу театру за цю важливу і чутливу поетичну
виставу. Ми сиділи і плакали, але усвідомлювали, що завдяки таким
віршам і постановкам наступні покоління зможуть дізнатися правду}, —
наголосила маріупольчанка Марія Макарова. 
	\endgroup
\end{leftbar}

\textbf{Читайте також:} \href{https://donbas24.news/news/zittya-pereselenske-mariupolskii-teatr-pidgotuvav-novu-premjeru}{%
\enquote{Життя переселенське} — маріупольський театр підготував нову прем'єру, %
Ольга Демідко, donbas24.news, 28.02.2023}

\ii{26_04_2023.stz.news.ua.donbas24.1.teatr_conception_linii_zhyttja.pic.3_4}

Театралізовану поетичну імпрезу \enquote{Лінії життя} заплановано показувати й надалі,
тому вона буде введена до репертуару театру.

\begin{leftbar}
	\begingroup
		\bfseries
\qbem{Така форма має бути в нашому репертуарі. Насправді, це буде
вистава-трансформер, яку можна буде показувати на різних локаціях}, —
додав Олексій Гнатюк.
	\endgroup
\end{leftbar}

\ii{26_04_2023.stz.news.ua.donbas24.1.teatr_conception_linii_zhyttja.pic.5}

Раніше \href{https://donbas24.news}{Донбас24} розповідав унікальні факти про \href{https://donbas24.news/news/unikalni-fakti-pro-teatralnu-kulturu-priazovya-do-vsesvitnyogo-dnya-teatru}{театральну культуру Маріуполя}.

Ще більше новин та найактуальніша інформація про Донецьку та Луганську області
в нашому \emph{\href{https://t.me/donbas24}{телеграм-каналі Донбас24}}.

Фото: з архіву Донбас24

\ii{insert.author.demidko_olga}
