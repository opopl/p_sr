% vim: keymap=russian-jcukenwin
%%beginhead 
 
%%file 11_12_2021.fb.lesev_igor.1.strategia_pustota
%%parent 11_12_2021
 
%%url https://www.facebook.com/permalink.php?story_fbid=4867411576623240&id=100000633379839
 
%%author_id lesev_igor
%%date 
 
%%tags gosudarstvo,ideologia,obschestvo,pustota,strana,strategia,ukraina
%%title О нашей стратегической пустоте
 
%%endhead 
 
\subsection{О нашей стратегической пустоте}
\label{sec:11_12_2021.fb.lesev_igor.1.strategia_pustota}
 
\Purl{https://www.facebook.com/permalink.php?story_fbid=4867411576623240&id=100000633379839}
\ifcmt
 author_begin
   author_id lesev_igor
 author_end
\fi

О нашей стратегической пустоте

У нас тут стали фантазировать, как американцы надавят на команду Блогера, и
Минск-2 таки засияет всеми прописанными там пунктами. Расскажу, почему этого не
будет. По крайней мере, не будет точно при Зеленском.

\ii{11_12_2021.fb.lesev_igor.1.strategia_pustota.pic.1}

Смотрите, как государственный проект Украина, может существовать только в двух
формах. Первая – это мультикультурная страна русских и украинцев. При этом,
«русские» и «украинцы» в данном случае далеко выходят за понятие этноса. В
наших реалиях это скорее цивилизационно-мировоззренческое понятие. В такой
Украине русские и украинцы равны во всем – от использования языка до
конкуренции идеологий.

Форм такой Украины может быть масса. Это может быть банальная федерация. Может
быть бельгийский вариант с проведением какой-то линии раздела, где уже
устанавливаются правила своей стаи. Может быть испанская версия. Может быть
даже полностью унитарная страна, но с полностью переписанной конституцией, в
которой права русских и украинцев полностью уравниваются.

И есть вторая форма Украины – этническая. Это Украина для украиноязычных
украинцев, которые равнее всех остальных, и в первую очередь, они равнее
русских. Какая это Украина расписывать смысла нет. Мы с вами в ней живем прямо
сейчас.

И вот Минские соглашения с их «особыми районами Донбасса». Слова, конечно же,
имеют значения. В начале было слово, а еще слово не воробей, ну и дальше вы это
всё знаете. Но все же если стол называть диваном, удобнее спать на нем не
будет. Минские соглашения – это не просто об автономии части Донбасса в составе
Украины. По тем словам, которые там прописаны – это конструкция союзного
договора между Киевом и Донецком/Луганском.

Да, для проекта мультикультурной Украины Минские не были бы ни дикими, ни даже
сколько-то заметными. Хотя будь Украина мультикультурной, и Минских соглашений
бы никаких в природе не существовало. Но у нас строится этнический проект
Украины. А он соотносится с автономным самоуправляемым образованием
приблизительно так же, как наркомания и здоровый образ жизни.

Согласитесь, трудно будет пояснить, почему в Донецке дети обучаются на русском
языке, а в Мариуполе только на украинском. Почему в Луганске могут студенты
обучаться на русском, а в Харькове это запрещено. Почему в Макеевке в
кинотеатрах есть русских дубляж, а во всей остальной Украине его нет. Вот эти
все «почему» выжигают всю эту этническую конструкцию и выставляют ее в глазах
сограждан уже «континентальной Украины» мракобесной.

При этом этнический проект Украины даже без проблемы Донбасса показывает свою
несостоятельность. Даже если бы в 14-м Москва позволила бы Киеву загасить
восстание на Донбассе, общую деградацию государственности это разве бы только
чуть притормозило. Постмайданная Украина с поразительной шизофреничностью
уничтожает все столбы-основания, на которых когда-то украинская
государственность и появилась.

Вот декоммунизация. Коммунизм – это всего лишь идеология. Одна из. И да,
когда-то у наших предков эта идеология была главенствующей. Часть нашей
истории. При этом массовые репрессии или голод 1933 – это не идеология
коммунизма, а политика той власти, которая называла себя коммунистической. Пол
Пот тоже называл себя коммунистом, и в его стране лопатами геноцидили свое же
население. А потом Пол Пота свергли вьетнамцы, которые тоже были коммунистами.
А вот Пиночет не любил коммунистов и задорно стрелял их пачками на стадионах.
Зато Пиночет был католиком. Стоит ли в таком случае носителям католической веры
приписывать массовые расстрелы людей в Чили?

Но именно при большевиках Украина впервые обрела черты институциональной
государственности. Все эти центральные рады и директории – пыль. ЛДНР
существует уже в несколько раз дольше. А все достижения Скоропадских и Петлюр
можно прописать одним словом – НИ-ЧЕ-ГО. Украинская государственность
состоялась благодаря В. Ленину, И. Сталину, ВКП(б)-КПСС и другим причастным.
Это они здесь проводили украинизацию и коренизацию, они нарисовали гигантские
границы УССР, они включали новые и новые земли в состав УССР, они проводили
здесь индустриализацию, они включили республику в страны-основатели ООН, и
именно они составляли такие конституции, при которых Украина имела юридическую
возможность выйти из Союза ССР.

Вот это всё – большевики. В одном месте – расстрелы людей, а где-то по
соседству новые заводы. Цензура и бесплатное образование. Запрет выезда за
рубеж и пионерские лагеря по всей стране. Солянка, которая не нарезается только
из позитива, или только из негатива. И именно из этой солянки в 1991 появилась
Украина. Появилась не потому, что где-то в 40-е в Карпатах бегало УПА. И не
потому что диаспора Канады регулярно надевала вышиванки. Ленин - ВКП(б) – КПСС
– Перестройка – ГКЧП – Беловежское соглашение – вот это все отцы-основатели
современной Украины. И вот когда в постмайданной Украине начался снос
памятников Ленину – это то же самое, как завалиться на кладбище и разбить
кувалдой памятную плиту своего прадеда.

Или дерусификация. Украина образца своего рождения 1991 года – это страна
русских и украинцев. Нет, есть, конечно же, и другие народы. Но если отсюда
вывести всех русских и украинцев, то Украина по плотности населения превратится
в Сахару. Поэтому и обобщаю до двух народо-единиц.

И именно Майданы – и первый, и особенно второй – разобщали два
народа-основателя Украины. Хотя первым это сделал все-таки Кучма с Конституцией
1996 года, где русский язык был юридически лишен равенства с украинским. Дальше
дерусификация сменилась русофобией, и так мы дотрахались до закона «о коренных
народах».

Не буду расписывать этические моменты этого процесса, и так уже не простыня, а
комплект постельного белья вышел. Отмечу только другой момент. Дерусификация
шагает нога в ногу с деиндустриализацией и дебилизацией. Помните броскую фразу
нашего отца-основателя, которому добили уже все памятники? Коммунизм – это
советская власть плюс электрификация всей страны. Вот русский язык – это
корабли и самолеты, а дерусификация – это то, что имеем сейчас плюс
невозможность строить корабли и самолеты. Да, кто-то может обидеться, но
картинка за окном именно такая и другой не будет.

Но давайте закругляться. Этническая Украина – вот она у нас под ногами. Люди
ведь писали в сталинские времена доносы и считали это правильным, наверняка
как-то обосновывали свою позицию. Возможно, даже им не было стыдно. Они так
помогали своему любимому советскому отечеству бороться с врагами народа. И
этническая Украина для украинцев – такая же самодостаточная ценность для
многих. Очень и очень многих.

Поэтому, и автономного Донбасса здесь быть не может. Вот вообще никак. Фигурка
не вписывается в конструкт. Нужен напильник или для фигурки, или для
конструкта. А последний уже построен, вылизан, выпестован и приобрел
иррациональную ценность. Ломать такое – это как в каком-то Гайсине в 1985-м
году свалить памятник Ленину. До приезда милиции местные отмудохали бы вандала.
Те самые местные, которые в 91-м проголосуют за независимость, затем поддержат
все Майданы, а потом ночью свалят таки памятник Ленину. Кстати, молодцы, что
ночью. Есть в этом что-то эдиповское.

И да, у Зеленского таки была уникальная возможность в 19-м чуть попилить
конструкт. Ради самой же Украины. Но пацанва выбрала обильную жратву, а в плане
идеологии и вовсе не заморачивались, перетянув себе всё тупиковое наследие
Петра. И теперь максимум, что можно из всего этого слепить – это боснийский
вариант под внешним управлением.

P.S. Заметьте, что в Донецке памятник Ленину по-прежнему стоит, хотя там вроде
бы у власти и не коммунисты. А на Бессарабке в Киеве на постаменте снесенному
памятнику Ленину по-прежнему пустота.

\url{https://t.me/Lesev_Igor}
\ii{11_12_2021.fb.lesev_igor.1.strategia_pustota.cmt}
