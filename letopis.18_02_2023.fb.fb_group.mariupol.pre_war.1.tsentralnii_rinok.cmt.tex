% vim: keymap=russian-jcukenwin
%%beginhead 
 
%%file 18_02_2023.fb.fb_group.mariupol.pre_war.1.tsentralnii_rinok.cmt
%%parent 18_02_2023.fb.fb_group.mariupol.pre_war.1.tsentralnii_rinok
 
%%url 
 
%%author_id 
%%date 
 
%%tags 
%%title 
 
%%endhead 

\qqSecCmt

\iusr{Заболотній Анатолій}

Дуже захоплююча розповідь!

\iusr{Zhanna Savelyeva}

Так, ринок це \enquote{розвантажка}. Любила у Маріуполі ранковий ринок на Лівому,
букіністів, барахолку, рибні ряди... Навіть, у Києві пощастило знайти свій
ринок. До 2018 року ринок на Виноградарі був майже стихійним, яскравим, щедрим
(з усіх куточків приїжджали торгувати - з Вінниці, з Каніва, з Білої церкви,
звідусіль). Там і мовні курси проходила:). І гриби, і сало, і щуки та їхня
ікра, городина усяка, чорниці (для мене щось нове, у Маріуполі не було
чорниць), все що завгодно... А потім монополізація та супермаркети почали
перемагати ринок: з частини ринку зробили парковку, потім усіх торговців м'ясом
притисли необхідністю торгувати із морозильного ларя (для багатьох це
непостльна витрата, ще й електроживлення треба), потім Ковід розігнав остаточно
чудовий цей ринок. Щось залишилося, але це вже не те, тільки на вихідні
приїздять селяни і небагато.

\begin{itemize} % {
\iusr{Віктор Кіпчарський}
\textbf{Zhanna Savelyeva} А я на новому місці свій ринок-розвантажку ще не знайшов...

\iusr{Zhanna Savelyeva}
\textbf{Віктор Кіпчарський} мені просто пощастило, ми - окраїна міста, взагалі, у Києві три ринки, один з яких Бессарабський з салом по 700грн. Жартую, але не дуже помиляюся). Ринки тут вбивають навмисне на користь супермаркетам(. Сумно
\end{itemize} % }

\iusr{Leonid Ehdelshteyn}

Я тоже обожал ходить на Центральный рынок. Ещё с детства. Я до 7.5 лет жил на
Артёма 48. Ходил на рынок вместе с бабушкой, на рынок мы ходили вместе, где-то
до 5 моих лет, пока она ещё сама выходила на улицу. Результат похода в мою
пользу, всегда был новая детская книжка, где она их покупала тогда, я по моему
тогдашнему возрасту не приметил, и что-нибудь в зубы пожевать: пирожок, чебурек
или плетёнку. С того времени у меня осталась неистребимая привычка, что-нибудь
покупать и есть на улице из уличной еды, где бы я не находился, в Париже ли, в
Гамбурге ли, на Андриатике в Кроатии, в Турции или Израиле. Впрочем, когда я
находился вне всей этой экзотики, я отлично обходился и пирожками-чебуреками с
Центрального рынка, во время еженедельной субботней закупки продуктов, а когда
два раза, каждый раз по пять лет, работал токарем вагонного депо, то, по дороге
на работу, и продукцией Слободского привокзального рынка, где признаться, ни
черта, собственно не было, но сигареты, корона - чёрный шоколад, пиво и вяленые
бычки-таранка, всё-таки всегда имелись в ассортименте. Там, кстати, принимал и
некоторые заказы на токарную работу от слободских мужичков и деловитых
армянских, державших на этом рынке пару прилавков.

А \enquote{железные ряды} Центрального рынка! Работая в вагонном депо, я закупал там
резцы, свёрла, метчики, плашки, алмазные круга для доводки резцов, фрезы.
подшипники и т.д. и т. п. Имел постоянных продавцов, из которых выделял Лёшу и
Толика. Зачем я это делал эти закупки за свои деньги, спросите меня? Ведь
какая-то инструментальная в вагонном депо всё-таки была? Вот именно, что
какая-то. 

Те заказы, которые мне делали рыбаки, на свои иномарки и катера, требовали
первосортный инструмент, который в этой инструментальной сроду не водился.

А ещё, в условиях дефицита, точнее почти полного отсутствия запчастей для
ремонта вагонов, особенно после того, как Донецк практически прекратил поставку
этих запчастей, с целью отжать у мариупольского депо все заказы в свою пользу,
прекратив давать даже сварную проволку для наплавки ремонтных деталей, даже
сменные керамические втулки для установки вагонных колёс, на вагонные оси, даже
амортизаторы для подвески генераторов, даже насосы для ручной прокачки воздуха
в отопительной системе вагонов, то я, опираясь на свой опыт токаря и технолога,
а ещё на \enquote{поставки} инструмента с Центрального рынка, а также на неликвиды из
старых запасов на складах вагонного депо, практически в одиночку решал и решил
все эти задачи как многие другие.

А куда мне было деваться? Поначалу, начальник депо, вызывал меня в кабинет и
ставил новую задачу, приговаривая: Думай, что нужно делать, а то уволю.

Это присказка у него была такая. А что тут было делать? На \enquote{Азовмаше} годами не
платили зарплату, а на комбинаты не было приёма. Закроется депо и что дальше?

Вот и думал; придумал, как изменить и изменил конструкцию станка, чтобы точить
посадочные шейки траверс после наплавки, которые были на 250 мм длиннее
межцентрового расстояния моего станка; догадался нагревать в горне или точить
сразу после наплавки траверсы, шплинтоны и шплинтонные втулки, наплавленные
высоколегированной сварочной проволокой из неликвидов со склада, при нагреве во
время наплавки или в кузнечном горне, сверхтвёрдая сварочная проволока
становилась мягкой, тем самым наплавленные поверхности переходили из
аустенитного состояния в ферритное и перлитное и появлялась возможность
обрабатывать их на станке при обычных режимах резания, спроектировал и сделал
приспособление для расточки корпусов топливных насосов, с последующей
постановкой в расточенные корпуса насоса, гильз, поршней и поршневых колец,
полученных литьём под давлением из спекаемого порошка; для подвески
генераторов, перетачивал резину, вынутую из амортизаторов для подвески немецких
конденционеров. Конденционеров уже давно не было, а амортизаторов для их
подвески на складе было, - завались, и тоже всё из неликвидов. Эх, было время!
А самое главное, были силы, на всё, за что не брался, куда всё делось вместе с
молодостью!

\begin{itemize} % {
\iusr{Віктор Кіпчарський}
\textbf{Leonid Ehdelshteyn} Вергуни... Присипані цукровою пудрою... А потім намагатись облизати язиком щоки та обсмоктувати липкі пальці...

\begin{itemize} % {
\iusr{Алінкіна Лариса}
\textbf{Віктор Кіпчарський} по 13 копійок

\iusr{Олена Сугак}
\textbf{Віктор Кіпчарський} самі смачнючи плєтьонки були на трамвайній зупинці вул Донбаська !

\iusr{Leonid Ehdelshteyn}
\textbf{Віктор Кіпчарський} 

Да-да. Я вставал с трамвая после смены, когда работал на \enquote{Азовмаше,} возле
этого киоска. Вергуны продавались в этом киоске, рядом с моим бывшим домом на
Артёма 48, когда я уже жил на Слободке. И не заходя домой, приходилось сразу
идти в институт, учился то на вечернем. И перекус вергунами перед началом
занятий. Именно по 13 копеек штука.

\end{itemize} % }

\iusr{Татьяна Иванова}
\textbf{Леонид Эдельштейн} 

да, такие токари в Германии наверное ценные работники! Ферритноперлитно
аустенитные вопросы - сочетание науки и техники! Поколение технарей уходит на
пенсию... Или просто уходит...

\begin{itemize} % {
\iusr{Leonid Ehdelshteyn}
\textbf{Татьяна Иванова} 

Пока только на пенсию. Уже 4 месяца немецкий пенсионер. Вышел на неё, родимую,
в 66 лет. Тут, в Германии, долго работают. А знание диаграммы железо - углерод,
пригодилось мне и в Германии. Мастер попросил на Stapler (погрузчике)
разгрузить машину. Я это сделал, но оставил включённой подачу на станке, так
как делал лабиринтную канавку в торце, под Dichtung (кольцевое уплотнение).
Прикинул, что успею разгрузить, пока резец пройдёт на нужную длину. Эту канавку
делают на самой малой подаче, и из-за большого диаметра, на малых оборотах
шпинделя. Обрабатывал канавку в корпусе, материал Hastelloy — сплав на основе
никеля, имеющий высокую стойкость к коррозии, жаропрочный и
труднообрабатываемый, твёрже не бывает, на диаметре 2000 мм. Ну, и слабо, по
запарке, зажал резцедержатель, который начал \enquote{плясать} вместе с резцом.

Меня нет, а он пляшет, ну и наплясал. Микрогеометрия канавки была страшной,
нужно было заплавлять. Заготовку, которая весом больше тонны, и из такой стали,
выбрасывать было стрёмно. Тем более, этот корпус делал весь участок, целую
неделю, тем более, оставались, только две окончательные операции, моя токарная
чистовая и торцешлифовальная. Тем более, что за деталью, уже вызвали самолёт из
Америки, который должен был прилететь послезавтра. Мастер был в шоке. А я нет.

Выслушав плач Ярославны, я предложил заплавить канавку прямо на станке. На что
он закричал:

- А потом, чем ты её угрызёшь? Это же Hastelloy!!!

- В горячем состоянии, я угрызу, что угодно.

Про перлиты-ферриты-аустениты, я распространяться не стал. Сам был не уверен.
Всё таки, в придачу, к материалу, ещё и диаметр.

Но сделал.

После снятия фасок, шлифовки торца, и постановки прокладки на место, вообще не
было видно, что канавку заплавляли и перетачивали.

А вообще-то, везде, в том числе и в Германии, я слышал одно и тоже: если ты
токарь, работай как токарь, если ты инженер - иди, и работай как инженер, что
ты всё время выёживаешься, будь кем-то одним.

Мой самый первый мастер, она пришла, в своё время в "Азовмаш" после техникума,
как токарь, у ней ничего не получалось на станке, тогда достала диплом и стала
работать мастером.

Так вот, она охарактеризовала меня и мою работу, тогда, когда мне было всего 20
лет, когда я пришёл на завод, сразу после армии, работать токарем: \enquote{У
Эдельштейна есть что-то от рационализатора, на этом он всё время и выезжает!} И
ещё: Я не ленивый. Наверное, в этом всё и дело.

\end{itemize} % }

\end{itemize} % }

\begin{center}
\begin{minipage}{\textwidth}
\iusr{Вика Ермолаева}

\ifcmt
  tab_begin cols=2,no_fig,center,separate

     pic https://scontent-frt3-2.xx.fbcdn.net/v/t39.30808-6/332087790_579426353818083_8085872636724233142_n.jpg?_nc_cat=111&ccb=1-7&_nc_sid=dbeb18&_nc_ohc=Nz4z8R3U_jMAX-qL_Gw&_nc_ht=scontent-frt3-2.xx&oh=00_AfA2wVbOUfg4kcEpNF0etayB5LGYYrQ5j1Yxr38FrAUp6g&oe=641A85D6
		 @caption Возле овощных рядов (Вика Ермолаева)

		 pic https://scontent-frt3-2.xx.fbcdn.net/v/t39.30808-6/332030731_5967466206678794_4324773187304624092_n.jpg?_nc_cat=111&ccb=1-7&_nc_sid=dbeb18&_nc_ohc=8ITt9qir1EEAX9FzfL8&_nc_ht=scontent-frt3-2.xx&oh=00_AfAhnkcqWUMLno0uhQImQdghRNOg1vaAOiP0xmITAKAeGg&oe=641B5994
		 @caption Апрель 2021, возле хоз. и строй товаров (Вика Ермолаева)

  tab_end
\fi

\end{minipage}
\end{center}

\iusr{Майя Коваленко}

Із сталого, традиційного: арьян в стаканчику. Колись у гранчаку. Ложка
видавалась. Зьїсти - посуд здати. Торт Наполєон тут же нарізані А потім пішли
вже скуповувати далі.

\iusr{Анна Супрунец}

У нас с мужем была традиция: каждый поход на Центральный рынок заканчивался в крытом, там продавали вкусные бургеры))

\iusr{Олена Сугак}

Як завжди - клас!

Сподобався \enquote{зирінг} 🤣зараз це актуально!

Пройшовся по магазинам позирінг, а на шопінг до секонд хенду..

\iusr{Leonid Ehdelshteyn}

Уже, живя в Германии, и приезжая в родной город, заходил на Центральный рынок,
попить арьяну с каймаком, здоровался со знакомыми продавцами, чьим постоянным
покупателем был многие разы, когда ещё был жителем Марика. Вот у этой я покупал
печенье, а у этой сливочное масло, у этого растительное, в павильоне. Они
отвечали мне на приветствие, удивлялись, что давно меня не видели, и каждый
спешил заниматься своими делами: они своими, а я своими. 

А вот и любимые \enquote{железные ряды.}

- Привет Толик, привет Лёха! Нужен ручной насос для велосипеда, с манометром.

- А что, в Германии нет?

- Может и есть, но я хочу хоть что-то купить здесь и привести туда.

- Сделаем. Завтра приходи. А если хочешь раньше, вон там, напротив нас, может
быть, даже сейчас есть.

- А я хочу купить у вас.

- Ну тогда до завтра.

Или удивительные встречи с людьми, из далёкого прошлого и именно на рынке.

Смотрю, стоят две красотки, одна моих лет, другая помладше, беседуют.

Подхожу, спрашиваю:

- Девушки, вы местные?

- А то ты не местный?

-???????????

- Что, не узнаёшь меня? Мы вместе в техникуме учились!

Вглядуюсь, узнаю. И тут же поправляю.

- Танька! Танька Л.! Не в техникуме, а в институте!

Отходим.

- Ты где?

- Я здесь, частный предприниматель. Вот мой магазин. А ты?

- А я... А я в Германии, работаю токарем.

- Ты же учился лучше всех!

- Ну, так получилось, знаешь, я думаю, что у меня всё нормально, меня как
токаря и там ценят.

- Заходи.

Захожу, послезавтра уезжать. Договариваюсь встретиться, обязательно
встретиться, в следующий раз, когда приеду. Проходят два ковидных года.
Наступает 2022. Решаю, что в этом году приеду обязательно. И ещё, нужно будет в
этом, 2022 году, обязательно повидать школьных одноклассников, с кем так хорошо
в 2013 году, отмечали 40 лет после окончания школы. И договориться со всеми о
новой встрече, ведь в следующем, 2023 году, - ровно 50 лет после её окончания.

А дальше было 24 февраля. А дальше была война.

\iusr{Ольга Чубарь}

Спасибо за теплые воспоминания!

Світлана Федоровська

Обожнювала там ходити і роздивлятися. Все було!

\iusr{Ирина Дубровина}

Самое интересное, общительное и общенародное место - это Центральный рынок!
Чего только не увидишь и чего только не услышишь ... и даже можно было
попробовать..., а барахолка - это же кладезь очень интересных вещей со своей
историей!!!
