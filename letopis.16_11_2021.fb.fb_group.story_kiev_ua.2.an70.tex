% vim: keymap=russian-jcukenwin
%%beginhead 
 
%%file 16_11_2021.fb.fb_group.story_kiev_ua.2.an70
%%parent 16_11_2021
 
%%url https://www.facebook.com/groups/story.kiev.ua/posts/1819967131533413
 
%%author_id fb_group.story_kiev_ua,nuribekova_vika
%%date 
 
%%tags aviacia,kb_antonov,ukraina
%%title Літак із майбутнього - Ан-70
 
%%endhead 
 
\subsection{Літак із майбутнього - Ан-70}
\label{sec:16_11_2021.fb.fb_group.story_kiev_ua.2.an70}
 
\Purl{https://www.facebook.com/groups/story.kiev.ua/posts/1819967131533413}
\ifcmt
 author_begin
   author_id fb_group.story_kiev_ua,nuribekova_vika
 author_end
\fi

Літак із майбутнього

16 грудня 1994 року літак короткого зльоту та посадки Ан-70 було вперше піднято
в повітря з аеродрому \enquote{Святошин} екіпажем у складі: Сергій Максимов - капітан,
Володимир Лисенко - другий пілот, Володимир Непочатих - штурман, Павло
Скотніков - бортінженер, Андрій Кострікин - радист, Михайло Березюк - провідний
інженер з льотних випробувань.

\begin{multicols}{2} % {
\setlength{\parindent}{0pt}

\ii{16_11_2021.fb.fb_group.story_kiev_ua.2.an70.pic.1}
\ii{16_11_2021.fb.fb_group.story_kiev_ua.2.an70.pic.1.cmt}

\ii{16_11_2021.fb.fb_group.story_kiev_ua.2.an70.pic.2}
\ii{16_11_2021.fb.fb_group.story_kiev_ua.2.an70.pic.3}
\ii{16_11_2021.fb.fb_group.story_kiev_ua.2.an70.pic.4}
\end{multicols} % }

@igg{fbicon.airplane} Історія створення Ан-70 розпочалась ще у 70-х роках, коли
КБ задумалось над створенням послідовника Ан-12. (Символично, що перший
дослідний літак Ан-12 вперше піднявся в повітря також 16 грудня, але у 1957 р.)

@igg{fbicon.airplane}  У 50-60х роках основу військово-транспортної авіації
складали літаки АНТОНОВ (Ан-22, Ан-26, Ан-22). В кінці 80х років, до цієї
преяди додався і Іл-76. Ідеологія  створення літака з турбо-гвинтовими
двигунами, який може злітати та сідати з будь-якої злітної смуги мав прийти на
заміну існуючого парку військово-транспорної авіації.

Технічне завдання на проектування почали розробляти в 1978 р. і затвердили
остаточний варіант завдання тільки в 1986. В цьому ж році почали розробляти
літак.

При створенні літака цього класу використовувались останні досягнення
аеродинаміки, міцності, авіатехнології, матеріалазнавства та радіоелектроніки
(БРЕО). 

У 1999 році Ан-70 (варіант Ан-7х) брав участь у европейському тендері на
середній транспортний літак, де переміг свого конкурента А400М. Проте у 2000
році, ФРГ прийнято політичне рішення на користь Airbus Military. 

@igg{fbicon.flag.ukraina} 24 серпня 2001 р.  взяв участь у повітрянному параді,
присвяченому 10-річчя Незалежності. 

@igg{fbicon.airplane} Літак пройшов глибоку модернізацію. Були враховані  результати випробувань,
які показали що потрібно покращити. 

Під час випробувань було встановлено 6світових та 15 національних рекордів, у
тому числі вантаж 55 тонн підняли на высоту 7350 м.

Це вдалось завдяки зусиллям київських конструкторів, інженерів, технологів,
робочих та випробувачів.

@igg{fbicon.flag.ukraina} 13 січня 2015 року Мінистерство Оборони України взяли
\enquote{70-ку} на озбооєння.

Внаслідок розриву зв'язків РФ наразі потребує імортозаміщення російських
комплектуючих і матеріалів, переобладнання та нових випробувань.

Ідеї, які використані у цьому проєкті, випередили свою епоху, а тому буде ще
довгі роки еталоном військових транспорників.

\ii{16_11_2021.fb.fb_group.story_kiev_ua.2.an70.cmt}
