% vim: keymap=russian-jcukenwin
%%beginhead 
 
%%file 20_11_2021.fb.bilchenko_evgenia.1.zvuk_vody
%%parent 20_11_2021
 
%%url https://www.facebook.com/yevzhik/posts/4457720254263050
 
%%author_id bilchenko_evgenia
%%date 
 
%%tags bilchenko_evgenia,kiev,kultura,poezia,recenzia,rossia,rusmir
%%title ЗВУК ВОДЫ: КОГДА ГОВОРИШЬ О БИЛЬЧЕНКО...
 
%%endhead 
 
\subsection{ЗВУК ВОДЫ: КОГДА ГОВОРИШЬ О БИЛЬЧЕНКО...}
\label{sec:20_11_2021.fb.bilchenko_evgenia.1.zvuk_vody}
 
\Purl{https://www.facebook.com/yevzhik/posts/4457720254263050}
\ifcmt
 author_begin
   author_id bilchenko_evgenia
 author_end
\fi

\ii{20_11_2021.fb.bilchenko_evgenia.1.zvuk_vody.pic.1}
\ii{20_11_2021.fb.bilchenko_evgenia.1.zvuk_vody.pic.2}

ЗВУК ВОДЫ: КОГДА ГОВОРИШЬ О БИЛЬЧЕНКО...
Трещины на асфальте: 
ночью было землетрясение.
Я выходила с клеем, 
заклеивала, честное слово.
Анна Долгарева 
(ниже - рецензия Владислав Сушков на мою книгу "Звук воды"... Влад, они всё отняли, всю мою воду из Лыбеди и Днепра вылакали, пришлось во рту, как девушке из древнего болгарского обряда в мать Неву переносить, дабы общий русский дождь пал с неба Бога-Отца и оросил, вскормил, очистил землю единую, братскую, давшую трещину, от нечисти заморской, от глобальной саранчи).
"Когда говоришь о Жене Бильченко aka БЖ, всегда есть опасность сорваться в пошлость романтических обобщений. Порой кажется, что она всей яркостью своего таланта провоцирует на это. Талантливый педагог и ученый, правозащитник, дивной мощи поэт из редкой категории «ни дня без текста»,– это, безусловно, всё она, БЖ, но суть прячется и теряется за этим набором штампов, как жизнь теряется за суетой больших дел. Семинары, фестивали, лекции, книги, выступления, поездки…
Как былинным богатырям, силы на всё ей даёт земля родная. Когда узнаёшь Женю Бильченко – узнаёшь Киев. Потому что она плоть от плоти этого города, она помнит и любит всю его пёструю мозаику, всю его историю от прошлого до будущего. В её поэтике – минимум роз и мимоз, зато в ней – правда и сила. Я люблю этот город, с его пацанами чёткими,/ С повисшими на передних старенькими девчонками,/ С православными и буддистами, с черепками и чётками,/ С ларьками и кабаками, где пьют не чокаясь. И когда мне понадобилось показать своей дочери Город, влюбить в него юную девочку с Русского Севера, я обратился именно к БЖ. Обратился, ещё не зная, что сам ник БЖ – суть киевский топоним, часть киевской географии и истории.
Поэзия Бильченко гармонично впитала в себя всех: древних святых и городских сумасшедших, поэтов и музыкантов, блаженных хиппарей и спешащих прохожих. Как в своё время ровесницу Жени, монументальную днепровскую Родину-Мать, хотели для пущей красоты покрыть сусальным золотом, да на наше счастье передумали,– так и стихи БЖ глянца не требуют. Они вырастают из всего, что рассыпает вокруг нас время, сочетая, казалось бы, несочетаемое: классические ямбы с динамикой рэпа, Восток с Западом, рок-н-рольный нерв с ясными нотами Лавры, Башлачёва с Летовым. Сила поэтического дара и энциклопедичность знаний дают возможность слить необъединимое в чисто звучащий сплав. В поэтике БЖ нет войны между традицией и модерном, нет разрыва между Львовом и Донецком, в её жилах пульсирует кровь Питера и Одессы, а университетский кофе оттеняет вечность киевских храмов. Потому разрушение этой действительности чисто физиологически выматывает поэта и задаёт основной тон её нынешних стихов. Сила и ясность в осознании Родины приходит, когда начинаешь ощущать её раны, как свои, – а для этого у Жени есть всё: талант, и душа, и родная земля. Как мало слов у неё всегда, у людоедки-пупсика./ «Не красавица», - говорит Шевчук, говорит Шевчук, ну, и пусть себе./ А ты верти свое колесо, смыкая конец с почином:/ От Мухосранска до Первомайска - Бог и тошнотики с капучино.
Женя, как и любой большой поэт, разорвана между своими ипостасями, питается от них, живёт и творит в поле этого напряжения. Наше время в изобилии и с некоторой словно бы садистической радостью подсовывает поводы для этой разорванности, – и Женя отвечает ему стихами, статьями, перформансами, литературными студиями. Они – не хобби, не источник заработка, не спасение от повседневности – для неё это просто её жизнь. И потому поэзия Жени Бильченко –наилучшее описание современности, изнутри и с высоты вечности сразу. Время породило БЖ, БЖ написала Время. Время смотрит на нас со страниц её книг. А это не только дар, но и крест. Когда отнимают отечество, когда отнимают отчество,/ Делай, что должен, делай - даже, когда не хочется./ Делай то, во что веришь,/ Даже, когда в нём прока нет./Делай, рискуя быть всенародно проклятым.
Однако времена не бывают плохими. А потому: когда думаете о Жене Бильченко – думайте о хорошем. О поэтах. О Городе. О единой душе.
Владислав Сушков, математик, поэт, переводчик, г. Сыктывкар, ноябрь 2019 г.

- - - - - -
БЖ. Звук воды
Звук, который я слышу, зачат глубоко-глубоко в земле:
Ниже слоёв керамики расписной и рогов оленьих,
Ниже спящих солдат Сталинграда, африканских песков зыбучих,
Грозового неба над Бучей и ставшего прошлым будущего.
Звук, который я слышу, - уже не звук, а, скорее, свет.
Когда он есть, тишина такая, что кажется - его нет.
Так, обнажив деревья, ноябрь белоснежной ваксой
Намечает крайнюю точку, где небо с землёй сливается.
Звук, который я слышу, - не в нас, это мы же сами.
Он каплет слезой дождя, тикает дедушкиными часами -
Бестактно, нерасторопно, нарушая сердечный ритм:
Выношу на кухню часы, а там - свечка в гробу горит.
Звук, который я слышу, - Крейцерово притяжение:
То ли играет музыка, то ли слышу её уже не я?
Плыву в радиоволне: то слева она, то справа...
Плыву и не знаю, что не умею плавать.
2019 г. \#вода
