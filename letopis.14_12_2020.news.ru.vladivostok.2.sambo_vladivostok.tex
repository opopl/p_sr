% vim: keymap=russian-jcukenwin
%%beginhead 
 
%%file 14_12_2020.news.ru.vladivostok.2.sambo_vladivostok
%%parent 14_12_2020
 
%%url https://www.newsvl.ru/sport/2020/12/14/195415/
 
%%author 
%%author_id 
%%author_url 
 
%%tags vladivostok
%%title Крупные дальневосточные соревнования по самбо прошли во Владивостоке после долгого перерыва
 
%%endhead 
 
\subsection{Крупные дальневосточные соревнования по самбо прошли во Владивостоке после долгого перерыва}
\label{sec:14_12_2020.news.ru.vladivostok.2.sambo_vladivostok}
\Purl{https://www.newsvl.ru/sport/2020/12/14/195415/}

\index[cities.rus]{Владивосток!Россия!Соревнования по самбо, 14.12.2020}

\ifcmt
pic https://static.vl.ru/news/1607908516782_default
\fi

В минувшие выходные в спорткомплексе «Олимпиец» состоялся чемпионат и
первенство Дальнего Востока по самбо и боевому самбо. Первое командное место в
неофициальном зачёте заняла сборная Приморья, которая завоевала 30 золотых, 17
серебряных и 18 бронзовых медалей.

Эти соревнования по самбо стали первыми в нашем регионе за последние полгода.
Ранее из-за пандемии коронавируса спортсменам было запрещено не только
участвовать в турнирах, но и какое-то время даже тренироваться.

«Яркая особенность этих соревнований — присутствие большого количества
спортсменов. Сборная каждого края привезла минимум по четыре участника в каждой
весовой категории. Наш турнир объединил около 450 спортсменов из различных
регионов Дальнего Востока: Амурской, Сахалинской областей, Хабаровского,
Камчатского, Забайкальского, Приморского краёв, Бурятии, Якутии. Все они
предоставили мандатной комиссии справку об отсутствии коронавирусной инфекции.
Также мы соблюдали все рекомендации Роспотребнадзора», - рассказал главный
судья соревнований Сергей Шкедов.

Участники были разделены на весовые категории. В зависимости от возраста групп
было от семи до девяти. Поединки в них проходили по олимпийской системе, когда
проигравший спортсмен выбывал, а победитель продолжал борьбу за медали.
Отметим, что победитель этих соревнований выполнял норматив мастера спорта и
получал путёвку на чемпионат или первенство России.

На эти соревнования во Владивосток приехал главный тренер сборных команд России
по самбо Дмитрий Трушкин. Задача специалиста была оценить уровень выступающих
спортсменов и посмотреть в деле кандидатов в сборную.

«Уровень соревнований высок. Тут есть участники и чемпионы мира. Большая
конкуренция среди юношей и девушек. Но мы придерживаемся незыблемого принципа
формирования сборной команды России: она должна состоять из победителей
чемпионатов и первенств России, а на эти соревнования можно попасть, выиграв
региональные соревнования», - сказал Дмитрий Трушкин.

\index[names.rus]{Храмова, Анастасия!Самбо, Россия, Владивосток, 14.12.2020}

Победитель и призёр чемпионатов и Кубков мира Анастасия Храмова из Владивостока
завоевала золото. Для неё это первый турнир после долгой паузы, связанной с
травмой. В финале спортсменка встретилась со своим партнёром по команде
Татьяной Туневой и сумела одолеть более молодую соперницу. По мнению девушек,
финальный поединок стал наиболее сложным.

«Мы занимаемся с Татьяной в одном зале и часто стоим в паре на тренировке,
знаем друг друга, кто и на что способен. Но даже если ты знаешь слабые места и
часто с этим партнёром работаешь, то нельзя недооценивать соперника», -
рассказала Анастасия Храмова.

Единственный мастер спорта международного класса по боевому самбо Николай
Ковтун в этот раз стал серебряным призёром чемпионата Дальнего Востока. В
финале он уступил чемпиону мира Шейх-Мансуру Хабибулаеву из
Петропавловска-Камчатского.

\index[names.rus]{Хабибулаев, Шейх-Мансур!Россия!Петропавловск-Камчатский, чемпион мира, самбо, 14.12.2020}

\index[names.rus]{Ковтун, Николай!Россия!Мастер международного класса по Самбо, Владивосток, 14.12.2020}

«Долгий перерыв сказывается, я и сам этой заразой переболел, - отметил Ковтун.
- Тяжело восстанавливаться функционально. Нужен год усердной работы и
тренировок, чтобы вернуться на прежний уровень».

В общей сложности спортсмены из Приморья завоевали 65 медалей и стали лидерами
неофициального зачёта этих соревнований.

