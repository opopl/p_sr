% vim: keymap=russian-jcukenwin
%%beginhead 
 
%%file 05_10_2019.stz.news.ua.mrpl_city.1.ivan_glotov
%%parent 05_10_2019
 
%%url https://mrpl.city/blogs/view/ko-dnyu-rabotnikov-obrazovaniya-ivan-glotov
 
%%author_id burov_sergij.mariupol,news.ua.mrpl_city
%%date 
 
%%tags 
%%title Ко дню работников образования: Иван Глотов
 
%%endhead 
 
\subsection{Ко дню работников образования: Иван Глотов}
\label{sec:05_10_2019.stz.news.ua.mrpl_city.1.ivan_glotov}
 
\Purl{https://mrpl.city/blogs/view/ko-dnyu-rabotnikov-obrazovaniya-ivan-glotov}
\ifcmt
 author_begin
   author_id burov_sergij.mariupol,news.ua.mrpl_city
 author_end
\fi

В старинном здании на Харлампиевской, 6 – сейчас его занимает частное учебное
заведение – с 1944 по 1958 год находилась женская средняя школа №2. Ее ученицы
давно стали бабушками, а некоторые даже прабабушками. Но когда они встречаются
друг с другом, то, заводя разговор о своих учителях, одним из первых вспоминают
\textbf{Ивана Федоровича Глотова}. И сразу же у одних из глубин памяти всплывают картины
школьных лет, как их учитель стремительно входил в класс, издали бросал журнал
на стол и на ходу громко восклицал: \enquote{Все учебники – на подоконники!}, а тут же
уже писал на доске уравнения реакций для каждого ряда парт отдельно.
Вспоминают, как при демонстрации химических опытов мог с улыбкой приговаривать:
\enquote{\em Какой прекрасный получился цвет раствора. Каждая девушка хотела бы иметь
платье такого цвета}. Другие переносятся мысленно в березовую рощицу, куда Иван
Федорович водил на экскурсию пятиклассниц, дополняя тем самым свои уроки
ботаники. Рощицы этой давно нет, на ее месте высится бывший кинотеатр
\enquote{Комсомолец} и скверик, засаженный деревьями разных пород. Третьи подчеркивают,
что при внешней суровости, он был человеком сопричастным к чужим бедам: всегда
интересовался о здоровье близких своих учеников, когда узнавал, что кто-то из
них болеет. Четвертые через годы и десятилетия пронесли в своей памяти поездку
в Киев на слет юннатов, организованную Иваном Федоровичем. И все безмерно
благодарны ему за переданные глубокие знания, особенно по химии: они без труда
получали \enquote{пятерки} по этому предмету при поступлении в вуз, да и при
последующей учебе, став студентами...

\ii{05_10_2019.stz.news.ua.mrpl_city.1.ivan_glotov.pic.1}

Конечно, ученикам этого учителя от Бога известна лишь внешняя сторона его
жизни. Ее подробности удалось воссоздать здесь благодаря документам из
семейного архива, а главное - воспоминаниям \textbf{Галины Ивановны Ткачевой}, дочери
Ивана Федоровича, любезно предоставленные автору этих строк.

\textbf{Читайте также:} 

\href{https://mrpl.city/news/view/grecheskoj-shkole-v-mariupole-kotoruyu-otkryval-leonid-kuchma-ispolnilos-20-let-foto}{%
Греческой школе в Мариуполе, которую открывал Леонид Кучма, исполнилось 20 лет, %
Богдан Коваленко, mrpl.city, 04.10.2019}

Иван Федорович родился 18 июня 1905 года в селе Старомайорское, по современному
территориальному делению этот населенный пункт входит в Великоновоселковской
район Донецкой области. В семье, кроме Ивана Федоровича, было еще два сына и
четыре сестры. В 1921 году он поступил в Мариупольский педагогический техникум.
После четырех лет обучения ему вручили свидетельство, где указано, что он –
Глотов Иван Федорович – получил квалификацию учителя младших классов с высшим
образованием. Молодой учитель был направлен в одну из школ родного ему района,
там он стал преподавателем и заведующим школой. Одновременно он учился заочно
на агробиологическом отделении Харьковского института социального воспитания. В
это же время Иван Федорович женился на Марии Ивановне Альянах, тоже
учительнице. В 1929 году у них родился первенец – сын Виталий.

В 1931 году он успешно окончил вуз и переехал в Мариуполь, чтобы занять
должности преподавателя физики и химии, а также заведующим учебной частью
рабфака Харьковского зоологического института при агробазе, находившейся вблизи
нашего города. Судя по благодарственным письмам выпускников этого учебного
заведения в адрес Ивана Федоровича, его педагогическая деятельность была весьма
успешной. Но в 1935 году это учебное заведение было упразднено и бывшему
рабфаковскому преподавателю пришлось перейти во 2-ю школу. Она занимала здание
дореволюционной частной женской гимназии Н. С. Дарий, а теперь в нем находится
городская налоговая администрация. В годы войны гитлеровцы сожгли школу. И
сразу после освобождения Мариуполя она переместилась в уцелевшее от пожара
помещение, о котором говорилось ранее. В середине 50-х годов ХХ века специально
для нее было построено школьное строение на улице Пушкина. Итак, 2-я школа
трижды меняла свое расположение, Иван Федорович же оставался на своем посту
преподавателя биологических дисциплин и химии вплоть до ухода на пенсию.

Ученики побаивались своего строгого наставника, но и любили. А повзрослев, уже
окончив школу, были безмерно признательны за те прочные знания, которые он
передал и должным образом оценили его требовательность. В семье дочери Ивана
Федоровича хранится много писем, открыток с теплыми словами искренней
благодарности учителю от бывших питомцев нескольких поколений. Вот строки
одного из таких писем: \enquote{\em Дорогой и любимый Иван Федорович! Выходя из стен этой
школы, мы надолго унесем в жизнь светлую и прекрасную память о вас.
Десятиклассницы. 18 июня 1948 года}.

Иван Федорович был великим тружеником: днем он работал в школе
общеобразовательной, вечером - в школе рабочей молодежи. Кстати, один из его
учеников-вечерников вспоминал: \enquote{\em Что греха таить, мы ходили на занятия не
столько за знаниями, сколько для того, чтобы получить со временем документ об
образовании, но наш строгий учитель без нажима сумел заставить учить, а позже и
полюбить химию}. 

А по воскресеньям и все лето во время каникул он занимался с ребятами в кружке
юных натуралистов или на школьных опытных полях, между прочим, созданных
благодаря хлопотам Ивана Федоровича. Он знал и любил живую природу, эти любовь
и знания сумел передать детям – своим и чужим. Работа юннатов продолжалась
практически круглый год. Зимой они сортировали семена, накануне весны
выращивали рассаду. Ну а летом ухаживали за посадками кукурузы, сорго, арахиса,
хлопка, овощей. Когда же наступала пора сбора урожая, Иван Федорович и его
подопечные неподдельно радовались плодами своих трудов. Вот что рассказала одна
из бывших его учениц: \enquote{\em Когда осенью мы выковыривали из земли клубни арахиса или
морковку, он с сияющей улыбкой разрешал их съесть, приговаривая: \enquote{Только мойте
морковку, ножиком счищайте}}.

\textbf{Читайте также:} 

\href{https://mrpl.city/news/view/mariupolskih-uchitelej-pozdravili-s-professionalnym-prazdnikom-foto}{%
Мариупольских учителей поздравили с профессиональным праздником, Олена Онєгіна, mrpl.city, 02.10.2019}

Не раз пришлось слышать историю о том, как Иван Федорович разводил тутового
шелкопряда. Вот что написала в своих воспоминаниях Галина Ивановна: \enquote{\em В 50-е
папа выписал из Узбекистана грену – так называются яйца шелкопряда. Первый раз
их прислали поздновато, так что часть этой грены не была роздана юннатам в
другие школы, всё осталось у нас дома. И вот из грены стали появляться
червячки, превратившиеся в больших прожорливых гусениц. У меня и сейчас перед
глазами папа, который ежедневно, утром и вечером, прикатывал на велосипеде
несколько огромных мешков веточек шелковицы с листьями. Посадки шелковицы тогда
росли у нас за городом. Для этой \enquote{армии} гусениц дома пришлось освободить две
комнаты от мебели}. Со временем шелкопряд переместился в кружок юннатов и в
школы. Сохранилась почетная грамота областного комитета комсомола, датированная
августом 1951 года, с таким текстом: \enquote{Этой грамотой награждается Глотов Иван
Федорович, руководитель кружка юных натуралистов за организацию работы на
пришкольном участке и за шелководство}.

Двор №19 на улице Семенишина, бывшей Малой Садовой. В глубине его, в небольшом
доме жила семья Глотовых. Вот что пишет Галина Ивановна об атмосфере, царившей
в нем: \enquote{\em Росли мы с братом в большой всеобъемлющей родительской любви, да и
любви всех родных и близких. Такой же любовью и обожанием были окружены и
внуки. Дома всегда было уютно и тепло, всегда много родственников, близких
людей. Папа очень любил своих братьев и сестер. После войны папа помогал семье
погибшего на фронте брата – у него осталось четверо детей, а также детям своих
сестер. Он помог им всем получить образование. Во время учебы племянники жили с
нами. И никогда не было тесно. Его мама, моя замечательная бабушка, тоже жила в
основном с нами. Дом наш был очень гостеприимным и хлебосольным. Дома у нас
росло много экзотических растений для нашей местности: инжир, гранат, большущий
лавр, огромное гуттаперчевое дерево, эвкалипты и многое другое}.

Иван Федорович еще до войны перенес тяжелую операцию, из-за которой плохо
слышал и был признан негодным для военной службы. Поэтому, когда началась
война, он не был призван в армию. А два его брата – Петр и Федор - ушли на
фронт и в боях за Родину сложили головы. Особенно трагичной оказалась судьба
Федора. Гвардии капитан Федор Глотов был убит 9 мая 1945 года в день окончания
войны.

Под влиянием учителя ряд его учеников избрали химию своей профессией и
отправились учиться в Киевский государственный университет, Московский
химико-технологический институт им. Менделеева, другие вузы, успешно их
окончив. Назовем имена некоторых из тех, кто стал высококвалифицированным
химиком: \textbf{Ольга Малоштан, Неонила Дмитренко, Лариса Тимошенко, Алла Гольцман,
Ирина Попова}, а также биологом – \textbf{Нонна Тоцкая, Юрий Солодовников}.

Преподаватель химии, физики и биологии, несмотря на внешний суровый вид, в
глубине души был лириком. Его радовали и умиляли первые всходы растений, суета
пчел в ульях, цветение деревьев. Втайне от посторонних глаз он писал стихи на
родной ему \emph{українській мові}. Иван Федорович Глотов с годами лишь изредка
позволял себе короткий отдых. В тот роковой день – 21 июля 1974 года – он
прилег, как он думал, вероятно, на несколько минут. Оказалось – навсегда.

\textbf{Читайте также:} 

\href{Читайте также: В Мариуполе школьник снимает видеоблог об истории города}{%
В Мариуполе школьник снимает видеоблог об истории города, Олена Онєгіна, mrpl.city, 05.10.2019}

\emph{Автор искренне благодарит Галину Ивановну Ткачеву, ее сына Сергея Евгеньевича
Ткачева, а также бывших воспитанников И. Ф. Глотова: Ольгу Сергееву, Валентину
Савину, Людмилу и Ларису Кричевских, Ольгу Малоштан, Тамару Фесенко, Веру
Самойлович, Семена Гольдберга и Анатолия Кирицева за помощь при подготовке
этого очерка.}
