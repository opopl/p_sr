% vim: keymap=russian-jcukenwin
%%beginhead 
 
%%file 07_10_2018.stz.news.ua.mrpl_city.1.mrpl_teatr_nimecka_okupacia_1
%%parent 07_10_2018
 
%%url https://mrpl.city/blogs/view/mariupolskij-teatr-v-roki-nimetskoi-okupatsii-chastina-1
 
%%author_id demidko_olga.mariupol,news.ua.mrpl_city
%%date 
 
%%tags 
%%title Маріупольський театр в роки німецької окупації. Частина 1.
 
%%endhead 
 
\subsection{Маріупольський театр в роки німецької окупації. Частина 1.}
\label{sec:07_10_2018.stz.news.ua.mrpl_city.1.mrpl_teatr_nimecka_okupacia_1}
 
\Purl{https://mrpl.city/blogs/view/mariupolskij-teatr-v-roki-nimetskoi-okupatsii-chastina-1}
\ifcmt
 author_begin
   author_id demidko_olga.mariupol,news.ua.mrpl_city
 author_end
\fi

Минуле театрального мистецтва Маріуполя має своєрідну історичну долю. Протягом
століть маріупольці вносили свою частку в скарбницю української театральної
культури. Найменш дослідженою на сьогодні залишається історія професійного й
аматорського театру на теренах Маріуполя періоду німецької окупації. До 75
річниці визволення Маріуполя від нацистських загарбників я вирішила висвітлити
розвиток \emph{театральної культури Маріуполя під час німецької окупації}. Оскільки
матеріалу виявилося багато, він буде розділений на \emph{2 частини}. Відомості
стосовно формування і розвитку театрального мистецтва Маріуполя у 1941 - 1943
рр. вдалося зібрати завдяки тогочасній періодичній пресі та працям українських
істориків \emph{В. Гайдабури і Д. Титаренка}.

До війни на теренах Північного Приазов'я діяли Маріупольський російський
музично-драматичний театр, обласний Державний грецький театр та численні
аматорські театральні й мистецькі гуртки. З перших днів війни в театрі була
створена фронтова бригада, яка виступала перед призовниками, пораненими
радянськими воїнами. 6 жовтня 1941 р. працівники Маріупольського театру
приготувалися до евакуації. Небагатьом акторам, серед них головному режисеру А.
Ходиреву і актору М. Рулеву, вдалося виїхати з міста.

\textbf{Читайте також:} \emph{Драматический театр Мариуполя открывает сезон}%
\footnote{Драматический театр Мариуполя открывает сезон, Анастасія Папуш, mrpl.city, 26.09.2018, \url{https://mrpl.city/news/view/dramaticheskij-teatr-mariupolya-otkryvaet-sezon}}

В умовах окупації 16 листопада 1941 р. у Маріуполі відкрився перший театральний
сезон. 9 березня 1942 р. у Маріупольському міському театрі відбувся великий
концерт, присвячений пам'яті Т. Шевченка. З нагоди дня народження Великого
Кобзаря театру було присвоєно його ім'я. Діяльність \textbf{\em Маріупольського
музично-драматичного театру ім. Т. Шевченка} висвітлюється на шпальтах
окупаційної преси – \emph{\enquote{Маріюпільської газети}}. 

\ii{07_10_2018.stz.news.ua.mrpl_city.1.mrpl_teatr_nimecka_okupacia_1.pic.1}

Заснування в місті професійного театру в період окупації стало результатом
впливу кількох важливих чинників: пропагандистські міркування; прагнення
окупаційних властей створити умови для відпочинку військовослужбовців вермахту
та намагання місцевого населення знайти додаткове джерело засобів для існування
завдяки праці в культурницьких інституціях. Разом з тим, незалежно від причин
перебування на окупованій території, місцева інтелігенція як примусово, так і
за добровільним вибором заради виживання вимушена була йти на співпрацю з
гітлерівською адміністрацією.

Серед маріупольських акторів були досить кваліфіковані особи з великим досвідом
роботи у театральних установах. Наприклад, директор Маріупольського театру
\emph{Андрій Ірій (Авраменко)} пройшов велику школу акторської та режисерської роботи,
зокрема в колективі, очолюваному Лесем Курбасом. Художній керівник і актор
театру \emph{Володимир Волгрик} до війни працював у Чернігівському театрі, здобув
звання заслуженого артиста республіки. Головним режисером у театрі працював І.
Тяк, режисери – С. Зав'ялова, В. Волгрик, головний художник театру – \emph{А.
Федоров}, балетмейстер – \emph{К. Вольська}.

У Маріупольському театрі ім. Т. Шевченка влаштовувалися вечори пам'яті
українських письменників, поетів, композиторів. Так, до святкування 129 роковин
з дня народження Тараса Григоровича Шевченка дирекція театру підготувала вечір
пам'яті великому українському поету. До програми свята увійшли: доповідь про
творчість Т. Шевченка; літературно-музичний монтаж; концерт. При театрі
працювали німецькі композитори, які \enquote{рекомендували} до постановки п'єси
українських та німецьких авторів. Проводилася добре продумана і спланована
агітаційно-пропагандистська робота серед населення антирадянського й
пронімецького змісту. Часто в театрі влаштовували народні читання, публічні
лекції, на яких обговорювалися враження про Німеччину та переваги
працевлаштування в Німеччині.

Дирекцію міського театру зобов'язали створити бригаду артистів для виступу на
фронті перед німецькими солдатами та українськими добровольцями. У програму
були включені українська народна музика, танці, українські та німецькі пісні.
Як повідомляв редактор \enquote{Маріюпільської газети} О. Озівський, 

\begin{quote}
\em\enquote{бригада під
керівництвом актора пана Ягупова побувала в м. Юзівці, Морфинському, Федірівці
та в багатьох інших місцевостях. У складі гастрольної групи: артисти-бояністи
Соловйов та Серебряков, балерини, солісти-співаки та інші. За 25 днів відбулося
понад 30 концертів, на яких було близько 15 000 німецьких вояків. Скрізь, де
відбувались концерти, глядачі дуже тепло приймали виступ акторів. Вояки з
великою зацікавленістю знайомились з українським мистецтвом, прохаючи акторів
ще приїздити до них}.
\end{quote}

Виконуючи як окупаційну повинність обслуговування
німецької армії, водночас театр мав можливість орієнтувати свою духовну енергію
на контакти з власним народом. На думку вченого-мистецтвознавця В. Гайдабури,
український театр періоду окупації, використовуючи легітимне становище для
найширшого спілкування із цивільним населенням, узяв на себе функції
мистецького механізму самозбереження національних традицій.

\textbf{Читайте також:} \emph{Театральному искусству Мариуполя 140 лет: о творчестве и достижениях актеров драмтеатра}%
\footnote{Театральному искусству Мариуполя 140 лет: о творчестве и достижениях актеров драмтеатра, Яна Іванова, mrpl.city, 26.03.2018, \url{https://mrpl.city/news/view/teatralnomu-iskusstvu-mariupolya-140-let-o-tvorchestve-i-dostizheniyah-akterov-dramteatra-foto}}

Відновив роботу театр ляльок, де ставилися вистави за казками Г. Андерсена, В.
Гауфа, братів Ґрімм та ін. Так, наприкінці травня 1942 р. у приміщенні театру
ім. Т. Шевченка відбулася перша вистава дитячого лялькового театру за казкою
Ганса Крістіана Андерсена \enquote{Свинопас}. Якщо в міському театрі йшов
вечірній спектакль, то квитки на нього служили пропуском, оскільки
комендантська година не скасовувалася.

Майже кожну прем'єру театру супроводжувала стаття місцевого журналіста-критика.
Як свідчать відгуки в пресі, програми вистав, гра акторів, музик, праця
диригентів і балетмейстерів викликала схвальні відгуки як цивільного населення,
так і окупантів. Присутність німців – європейського глядача, обізнаного з
мистецтвом, – орієнтувала театри на посилену відповідальність за результати
своєї праці. Перед прем'єрою спектаклю \enquote{Наймичка} у місцевій періодиці
зазначалось, що режисер вистави В. Волгрик додав до п'єси спеціальну музику,
також багато пісень і танців. Не зважаючи на відсутність матеріалів, художник
А. Федоров приготував декорації, що мають відбити побут часу, коли відбувались
події, наведені в п'єсі. У цілому вистава \enquote{Наймичка} І. Карпенка-Карого
приємно зворушила глядачів.

Маріупольський театр ім. Т. Шевченка працював дуже активно. 5 липня 1942 р.
відбувся балетний концерт. Через декілька днів \enquote{Маріюпільська газета}
відгукнулася на цю подію статтею. Переповівши в загальних рисах зміст концерту,
автор публікації зазначив: \emph{\enquote{Публіка з захопленням прийняла новий концерт}}.
Цікавими були акробатні вистави з дуже вдалими світловими ефектами та кілька
концертних і оркестрових номерів.

Репертуар Маріупольського музично-драматичного театру ім. Т. Шевченка складався
переважно з народно-побутових вистав, що виконувалися здебільшого українською
мовою. Всього за час окупації відбулося 18 вистав. Найбільш популярні твори, що
виконувалися на сцені театру: \emph{\enquote{Синочок}}, \emph{\enquote{Наталка
Полтавка}}, \emph{\enquote{Сватання на Гончарівці}}, \emph{\enquote{Пошились у
дурні}}, \emph{\enquote{Наймичка}}, \emph{\enquote{Маруся Богуславка}},
\emph{\enquote{Ревізор}}, \emph{\enquote{Рожеве павутиння}},
\emph{\enquote{Безприданниця}}, \emph{\enquote{Вій}} \emph{\enquote{Біда від
ніжного серця}}, \emph{\enquote{Закрита вистава}}, \emph{\enquote{Підступність
і кохання}}. Виходячи з репертуару, можна зробити висновок, що працівники
театру, прилаштовуючись до умов окупанта, водночас формували український
національно-культурний простір...

\textbf{Читайте також:} \emph{Серце культурного життя Маріуполя}%
\footnote{Серце культурного життя Маріуполя, Ольга Демідко, mrpl.city, 09.10.2017, \url{https://mrpl.city/blogs/view/sertse-kulturnogo-zhittya-mariupolya} } %
\footnote{Internet Archive: \url{https://archive.org/details/09_10_2017.olga_demidko.mrpl_city.serce_kulturnogo_zhyttja_mariupolja}}
