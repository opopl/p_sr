% vim: keymap=russian-jcukenwin
%%beginhead 
 
%%file 20_10_2021.fb.kapeljushnyj_anton.1.sobesedovanie_mova_programmist.cmt.in.kobelskiy.vinnica_2017
%%parent 20_10_2021.fb.kapeljushnyj_anton.1.sobesedovanie_mova_programmist.cmt
 
%%url 
 
%%author_id 
%%date 
 
%%tags 
%%title 
 
%%endhead 

\iusr{Dmytro Kobelskiy}
О, розумію. В мене, подібне у Вінниці було, але в 2017

\begin{itemize} % {
\iusr{Юлія Солонько}
\textbf{Dmytro Kobelskiy} яка компанія?

\iusr{Dmytro Kobelskiy}
\textbf{Юлія Солонько} 

низка компаній. Мейкап та Панама - посада копірайтера, там була лише російська
- і сказали, що я україномовний і в мене багато українізмів. Ще компанія
ЛеттіШопс, теж копірайтера. Скинув зразок тестового українською, а сказали, що
їм треба російською і мовляв мало досвіду і решту. Ще дві, невеликі, не згадаю
вже їх назв, але там одразу ж, що раз україномовний, то не підходить, бо в них
все російською і колектив російськомовний, клієнти і т.д. Це в 2017 році було

\iusr{Dmytro Kobelskiy}
\textbf{Юлія Солонько} ввечері буду вдома, познаходжу на пошті, покажу скріни

\iusr{Юлія Солонько}
\textbf{Dmytro}, ну одне питання коли продукт розрахований на певну країну/осередок/мову, тощо, але \textbf{\#ого}.

\iusr{Юлія Солонько}
\textbf{Dmytro Kobelskiy} нагадаю)

\iusr{Dmytro Kobelskiy}
\textbf{Юлія Солонько} мейкап та панама розраховані були на Україну.

\iusr{Юлія Солонько}
\textbf{Dmytro Kobelskiy} тоді \textbf{\#трулі\_жесть}

\iusr{Dmytro Kobelskiy}
\textbf{Юлія Солонько} да, я потім десь в 2019 бачив вакансію на мейкапі та панамі, що треба копірайтер, аби писав українською

\iusr{Dmytro Kobelskiy}
\textbf{Юлія Солонько} 

з телефону глянув, наче немає вибору української мови там. Може з компа буде 

\url{https://makeup.com.ua/product/10727/}

\url{https://makeup.com.ua/product/10727}

\iusr{Dmytro Kobelskiy}
\textbf{Юлія Солонько} 

а от панама українізувалась. Копірайтери там від однієї контори на мейкапі та
панамі \url{https://panama.ua/ua/}

\iusr{Алина Прокопенко}
\textbf{Dmytro Kobelskiy} на мейкапі є укр мова, але не скажу, чи завжди була, останні роки 3-4 є укр версія точно

\iusr{Dmytro Kobelskiy}
\textbf{Юлія Солонько}

\ifcmt
  ig https://scontent-frx5-1.xx.fbcdn.net/v/t39.30808-6/246362398_4635198099852296_7242248629765188289_n.jpg?_nc_cat=110&ccb=1-5&_nc_sid=dbeb18&_nc_ohc=wRV_psEA0UUAX-aeyZi&_nc_ht=scontent-frx5-1.xx&oh=2b7ddf5f07e81ad6b65e72463747e331&oe=61783273
  @width 0.4
\fi

\iusr{Dmytro Kobelskiy}
\textbf{Аліна Прокопенко} там наче з 2019 з'явилась

\iusr{Dmytro Kobelskiy}
\textbf{Аліна Прокопенко} в 2017 -все рсосійською було

\iusr{Dmytro Kobelskiy}
\textbf{Юлія Солонько} 

мейкап та панама - не можу знайти, але сутність та сама. А скиньте російською
мовою, а от в нас все російською. а Ваш текст - весь з укрїнізмами і Ви нам не
підходите.

А ще дві контори - там вживу, але майже те саме сказали, може навіть в більш
грубій формі

\end{itemize} % }

\iusr{Ольга Чех}
Позорище

\iusr{Марина Нємцева}
Я в а\#уї. Немає слів.

\iusr{Олег Слабоспицький}
\textbf{Anton Kapelushny} напиши скаргу \textbf{Уповноважений із захисту державної мови Тарас Кремінь} - це ж чисте порушення Закону.

\begin{itemize} % {
\iusr{Dmytro Kobelskiy}
\textbf{Олег Слабоспицький} не знаю, сам Кремінь російкомовний, я читав його наукові роботи до 2014 року, там вата ватою

\iusr{Олег Слабоспицький}
\textbf{Dmytro Kobelskiy} я теж був російськомовний половину життя і мені не подобалось коли мене виправляли на українську. І що тепер?
Він зараз єдиний відповідає згідно Закону за усунення порушень у сфері мови. І враховуючи мінімальний штат і мінімальне фінансування - дуже добре справляється.
Тому це дуже смішно накидувати на нього.

\iusr{Dmytro Kobelskiy}
\textbf{Олег Слабоспицький} не знаю, не здибався, а здибався з таким

\ifcmt
  ig https://scontent-frt3-1.xx.fbcdn.net/v/t39.30808-6/245644732_4627621723943267_1394127400786267874_n.jpg?_nc_cat=104&ccb=1-5&_nc_sid=dbeb18&_nc_ohc=u06Iu7Yg1DcAX9mk0qD&_nc_ht=scontent-frt3-1.xx&oh=033f9d495db3a6fdb249ecb249461500&oe=6178FB83
  @width 0.4
\fi

\iusr{Dmytro Kobelskiy}

\ifcmt
tab_begin cols=2,no_fig,width=0.1

  ig https://scontent-frt3-1.xx.fbcdn.net/v/t39.30808-6/246148596_4627622077276565_6535427845792148179_n.jpg?_nc_cat=102&ccb=1-5&_nc_sid=dbeb18&_nc_ohc=hLE7rE7930AAX9u8sKz&_nc_ht=scontent-frt3-1.xx&oh=558fcea234f6babc77b6b586c9b14fed&oe=61780FEB

	ig https://scontent-frt3-2.xx.fbcdn.net/v/t39.30808-6/247190074_4627622290609877_499107812008762632_n.jpg?_nc_cat=103&ccb=1-5&_nc_sid=dbeb18&_nc_ohc=3XpM7usHKmQAX-V6m4P&_nc_ht=scontent-frt3-2.xx&oh=adefb005f5919ceb8f68b77c9dfc0c7a&oe=61798346

tab_end
\fi

\iusr{Dmytro Kobelskiy}
\textbf{Олег Слабоспицький}

\ifcmt
tab_begin cols=2,no_fig,width=0.2

  ig https://scontent-frx5-1.xx.fbcdn.net/v/t39.30808-6/245962747_4627622557276517_7375118336401271961_n.jpg?_nc_cat=110&ccb=1-5&_nc_sid=dbeb18&_nc_ohc=lyoD3JSdm1UAX-gvDbC&_nc_ht=scontent-frx5-1.xx&oh=5ae7f4f9bd7654f604cb1b3f717babcf&oe=617834A4

  ig https://scontent-frx5-1.xx.fbcdn.net/v/t39.30808-6/246147797_4627622780609828_299464259573242435_n.jpg?_nc_cat=111&ccb=1-5&_nc_sid=dbeb18&_nc_ohc=XjCzXFDBfNQAX9vpui5&_nc_ht=scontent-frx5-1.xx&oh=afafd31330fa4dd2ff60d10c073d438a&oe=6179DF05

tab_end
\fi

\iusr{Dmytro Kobelskiy}
\textbf{Олег Слабоспицький} одна справа бути російськомовним. Але інша справа російськомовним, підтримувати партію регіонів і писати у ватному стилі - оце вже дещо інше

\iusr{Олег Слабоспицький}
\textbf{Dmytro Kobelskiy} ти це зараз серйозно? Ну звісно легше коменти писати ніж працювати на державній посаді. Лише є нюанс що коменти нічого системно не змінять.
Ніхто не каже, що Кремінь ідеальний. Але він дійсно має тверду позицію на захист Закону про мову. І його можна лише підтримувати у цьому. А не розводити срач про його минуле.

\iusr{Dmytro Kobelskiy}
\textbf{Олег Слабоспицький} ну, побачимо, побачимо. Ой, ну як особа, що працює в даний момент на державній посаді, скажу лише одне - дуже і дуже невтішно він працює

\iusr{Олег Слабоспицький}
\textbf{Dmytro Kobelskiy} ну так бери і допомагай щоб працював краще. Можна починати з написання скарг. Як показує досвід - більшість з них змінюють ситуацію навіть через банальний лист від Уповноваженого.




\end{itemize} % }
