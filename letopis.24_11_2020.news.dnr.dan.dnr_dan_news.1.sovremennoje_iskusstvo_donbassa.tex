% vim: keymap=russian-jcukenwin
%%beginhead 
 
%%file 24_11_2020.news.dnr.dan.dnr_dan_news.1.sovremennoje_iskusstvo_donbassa
%%parent 24_11_2020
 
%%url https://dan-news.info/obschestvo/sovremennoe-iskusstvo-donbassa-pravoslavnyj-konkurs-v-dnr-sobral-pochti-200-xudozhnikov.html
 
%%author Донецкое Агентво Новостей (ДАН)
%%author_id dnr_dan_news
%%author_url 
 
%%tags dnr,art
%%title Современное искусство Донбасса: над чем работают молодые художники в ДНР
 
%%endhead 
 
\subsection{Современное искусство Донбасса: над чем работают молодые художники в ДНР}
\label{sec:24_11_2020.news.dnr.dan.dnr_dan_news.1.sovremennoje_iskusstvo_donbassa}
\Purl{https://dan-news.info/obschestvo/sovremennoe-iskusstvo-donbassa-pravoslavnyj-konkurs-v-dnr-sobral-pochti-200-xudozhnikov.html}
\ifcmt
	author_begin
   author_id dnr_dan_news
	author_end
\fi

\ifcmt
pic https://dan-news.info/wp-content/uploads/2020/11/dan-news.info-2020-11-24_11-36-18_205490-anna-kolosyuk-d5nh6mcw52c-unsplash-1024x683.jpg
cpx Фото: Anna Kolosyuk/Unsplash
\fi

Донецк, 24 ноя – ДАН. Почти 200 работ молодых художников ДНР поступило на
республиканский конкурс «Веры нашей торжество», посвященный библейским сюжетам,
семье и традициям. Об этом сегодня на брифинге в ДАН сообщила заведующая
производственной практикой Донецкого художественного колледжа Светлана
Петрикина.

«В работах участники продемонстрировали любовь к родному краю, восхищение его
красотой, понимание вечных духовных ценностей, знание Священного Писания и
православной традиции», — сказала Петрикина.

По ее словам, в организационный комитет поступило 193 работы из Донецка,
Макеевки, Горловки, Енакиево, Иловайска, Кировского, Ждановки, Снежного,
Тореза, Харцызска, Шахтерска и Юнокоммунаровска. Первого ноября завершился
прием заявок от обучающихся художественных школ и соответствующих отделений
школ искусств, кружков и изостудий, воспитанников воскресных школ православной
церкви, студентов Донецкого художественного колледжа.

Петрикина добавила, что победители будут определены в декабре текущего года, а
лучшие работы войдут в тематический сборник из серии «Искусство Донбасса».

Напомним, что Республиканский конкурс молодых художников «Веры нашей торжество»
начался в июне 2020 года. Организаторы – министерство культуры ДНР и Донецкий
художественный колледж. Конкурс проходит в трех возрастных категориях от 8 до
25 лет по номинациям: библейские сюжеты; праздники православного календаря; мой
дом, семья, мой край.
