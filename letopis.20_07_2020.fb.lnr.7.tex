% vim: keymap=russian-jcukenwin
%%beginhead 
 
%%file 20_07_2020.fb.lnr.7
%%parent 20_07_2020
 
%%endhead 

\subsection{
На Украине продолжается истерика «активистов», вызванная законопроектом Максима Бужанского касательно русского языка.}
\url{https://www.facebook.com/groups/LNRGUMO/permalink/2861574897287399/}
  
\vspace{0.5cm}
{\small\LaTeX~section: \verb|20_07_2020.fb.lnr.7| project: \verb|letopis| rootid: \verb|p_saintrussia|}
\vspace{0.5cm}

На Украине продолжается истерика «активистов», вызванная законопроектом Максима
Бужанского касательно русского языка. Бужанский не предложил сделать русский
вторым государственным, он всего лишь предложил не так рьяно бороться с русским
языком. Законопроект говорит о том, чтобы не с этого года вводилось в школах
обязательное обучение на украинском, а с 2023-го.

Смысла в истерике нет никакого, но как говорил Портос: «Я дерусь… потому что
дерусь!», так и украинские «патриоты» находятся в состоянии перманентной
истерики и нужен всего лишь повод для того, чтобы начинать кричать на всю
планету. Причем громче всех кричат именно те, кто утверждает, что на Украине
при СССР была насильственная русификация, а насильственно этого делать нельзя.

Да, при Союзе был просто «ужас» в языковом вопросе: мои родители спокойно могли
написать отказ от изучения украинского языка и литературы, и я бы их не изучал.
Но родители не считали это нужным, потому в восьмом классе на экзамене по
украинскому у меня было 5/5. Кроме этого, наш класс выиграл конкурс по Т.
Шевченко, и на осенние каникулы мы поехали в Закарпатье: Яремча, Ясеня,
Ужгород.

Напомню: я рос в г. Николаеве, где на рынке быстрее услышишь армянский, чем
украинский, а уж просто в городе или транспорте вообще украинского не слышно.
Тот украинский, который я прекрасно знаю, отличается от современного
украинского больше, чем говор лондонца от говора австралийца. Современные
украинствующие сделали все, чтобы уничтожить мову, и теперь это даже не
галичанский суржик, а просто издевательство. Знаете, как по-украински теперь
Чебурашка? --- Гнэдысько!

Но оказалось, что русский язык, который не превратили в издевательство и
посмешище, является угрозой украинской независимости, государственному
суверенитету и многому другому. Но забавно другое --- эти «патриоты» пишут все
это по-русски! Я часто в комментариях с кастрюлеголовыми, прошу их писать
по-украински. Не пишут, так как, вероятно, не знают его или прекрасно понимают,
что русский --- это язык общения, а украинский --- просто прикол такой.

Ненавидеть Россию можно и на русском языке, с этим прекрасно справляется пятая
колонна в самой России, и прекрасно справляется вся верхушка украинской власти.
Все они прекрасно понимают, что русский --- это язык международного общения.
Английский знает не каждый гражданин Украины, а на приемлемом уровне — даже не
10\% граждан. А вот русский знают почти все.

Попытаться насильственно заставить изучать язык --- это признание того, что в
реальности он попросту никому не нужен. Ведь если сейчас дать выбор --- на
русском учиться или на украинском, то от украинского останутся только нелестные
воспоминания. На украинском языке нет ничего, ради чего этот язык стоит
изучать.

Украина за время «независимости» потеряла все возможности становления как
единого народа, пусть даже из разных наций. Граждане Украины смотрят фильмы на
русском языке, читают русскоязычные сайты. Технические и даже философские
науки, на украинском языке --- это просто мрак какой-то, потому изучают на
русском. Только фольклор, да «патриоты» пользуют украинский язык.

Русский язык --- язык прогресса, язык будущего. Обучение на украинском нужно лишь
для того, чтобы выращивать абсолютно бестолковых людей, которые по уровню
знаний и умению их применять, будут примерно в каменном веке. Украина
уничтожила свои промышленные производства, уничтожила науку, уничтожила все,
что является прогрессом. Носители украинского языка будут способны лишь на сбор
урожая и несложных процедур по сантехнической обработке помещений типа
«сортир».

Но украинская власть продолжает борьбу с прогрессом! С ним можно бороться,
можно вводить средневековые правила, благо сельские граждане к этому уже
готовы. Но продуктивно ли это? В итоге, украинская власть может оказаться там,
где всегда оказывались сильно «незалежные» гетманы --- в медном котле, где их
варили более удачливые проходимцы.

Лебедев Сергей (Лохматый) 
  
