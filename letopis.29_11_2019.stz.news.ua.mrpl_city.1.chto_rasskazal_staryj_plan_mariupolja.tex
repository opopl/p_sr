% vim: keymap=russian-jcukenwin
%%beginhead 
 
%%file 29_11_2019.stz.news.ua.mrpl_city.1.chto_rasskazal_staryj_plan_mariupolja
%%parent 29_11_2019
 
%%url https://mrpl.city/blogs/view/chto-rasskazal-staryj-plan-mariupolya
 
%%author_id burov_sergij.mariupol,news.ua.mrpl_city
%%date 
 
%%tags mariupol,mariupol.istoria
%%title Что рассказал старый план Мариуполя?
 
%%endhead 
 
\subsection{Что рассказал старый план Мариуполя?}
\label{sec:29_11_2019.stz.news.ua.mrpl_city.1.chto_rasskazal_staryj_plan_mariupolja}
 
\Purl{https://mrpl.city/blogs/view/chto-rasskazal-staryj-plan-mariupolya}
\ifcmt
 author_begin
   author_id burov_sergij.mariupol,news.ua.mrpl_city
 author_end
\fi

\ii{29_11_2019.stz.news.ua.mrpl_city.1.chto_rasskazal_staryj_plan_mariupolja.pic.1}

К фундаментальному труду коллектива сотрудников Мариупольского краеведческого
%музея \enquote{Мариуполь и его окрестности: взгляд из XXΙ века} приложен план нашего
музея \enquote{Мариуполь и его окрестности: взгляд из XXΙ века} приложен план нашего
города, помеченный 1930 годом. Всмотревшись в него внимательно и приложив к
нему сведения из справочника \enquote{Довідкова та адресові книга. Вся Україна та АМСР
на 1930 рік: Маріупіль}, можно получить довольно много информации о городе,
отделенном от нас, ныне живущих, разрушительной войной и почти девяноста
годами.

Итак, что бросается прежде всего в глаза. И завод имени Ильича, и морской порт
с прилегающими к ним жилыми кварталами находятся за пределами плана. На левом
берегу Кальмиуса обозначен лишь поселок Бузиновка, которому оставалось доживать
несколько месяцев. Еще только-только началось строительство \enquote{Азовстали},
поэтому его строящиеся объекты пока не нашли отражения на плане. Все
православные храмы Мариуполя отмечены значками: собор св. Харлампия и церковь
св. Екатерины - на Базарной площади (современная площадь Освобождения), св.
Марии Магдалины – в сквере, во имя Рождества Пресвятой Богородицы – на
Карасевке, Успения Пресвятой Богородицы – на Марьинске, Елено-Константиновская
- на Слободке, во имя всех Святых - на городском кладбище. Их печальная участь
– впереди. Через пять-шесть лет они будут сметены воинствующими безбожниками с
помощью взрывов с лица земли. Пойма речки Кальчик, территория, где сейчас
высятся дома пятого микрорайона, а также значительная часть земель, прилегающих
к Новоселовке, занята фруктовыми садами. К Городскому саду примыкает питомник
Зеленстроя.

\textbf{Читайте также:} \href{https://mrpl.city/blogs/view/mariupolskij-park-im-na-gurova-razmyshleniya-dosuzhego-passazhira-tramvaya}{%
Мариупольский парк им. Н.А. Гурова: размышления досужего пассажира трамвая}

Видно, как в ту пору был мал наш город. Во всяком случае, его центральная
часть. Западная оконечность кварталов между улицами Митрополитской и
Итальянской заканчивалась у Покровской площади, ее в народе называли Сенной или
Выгоном. Границей служила нечетная сторона улицы Франко (теперь - проспект
Металлургов). Ее четная сторона, обращенная на северо-восток, начиналась от
Митрополитской улицы.

На плане представлены промышленные предприятия. Это бойня на Правом берегу,
позже названная Мясокомбинатом, с трех сторон городского кладбища - по
кирпичному заводу, печи для выжигания извести у поселка Аджахи, на улице Артема
близ улицы Шевченко - две маслобойки. Неподалеку от них - чугунолитейный
заводик. На улице Торговой, 107 был Ультрамариновый завод, где изготавливалась
синька, из-за этого округа была окрашена в синий цвет, на этой же улице
располагались пивоваренный завод, две небольшие фабрики: одна по производству
мыла, другая –изготавливавшая колбасы, а также завод ковкого чугуна
\enquote{Универсал}, кстати, только он из вышеназванных и дожил до наших дней - теперь
это \enquote{Пожзащита}. Осталась на своем месте и кондитерская фабрика на Николаевской
улице. Правда, для нее в 50-е годы было построено новое здание. На углу улиц
Артема и Евпаторийской улиц находился эстампажный завод, в наши дни он
назывался бы кузнечно-прессовым, так как продукцией этого предприятия были
различные штампованные изделия, например, лопаты. Его \enquote{наследником} является
теперь фирма \enquote{Октябрь}.

\textbf{Читайте также:} \href{https://mrpl.city/blogs/view/vokrug-skvera-istoriya-tsentra-mariupolya}{%
Вокруг сквера: история центра Мариуполя}

Улица Энгельса, 68 – адрес графитового завода. По адресу Улица Фонтанная, 73
располагалось предприятие по перемолу полевого шпата, кварца, каолина и слюды,
на углу улиц Энгельса и Митрополитской – Мариупольский машиностроительный
завод, это на его базе было создано предприятие по производству медицинского
технологического оборудования, которое прекратило свое существование в начале
90-х годов. Оказывается, радиаторный завод, находящийся на улице Артема, занял
место тракторной базы. Вероятно, это было место, где селяне могли взять
напрокат трактора и другую сельхозтехнику. В то время в городе существовали
такие фабрики: швейно-чулочная, мебельная, макаронная, две – просто швейные. К
промышленным предприятиям можно отнести и электростанцию на Малофонтанной
улице. Она была взорвана немецко-фашистскими оккупантами в сентябре 1943 года,
а после войны не была восстановлена. Да, еще обозначена типография на площади
Свободы, теперь названной Театральной.

Об учебных заведениях. В той части города, которая отражена на плане, показано
двенадцать школ. Все они были построены в царское время. Здания некоторых из
них дожили до сегодняшних дней. Вот места их расположения. На бывшей улице
Октябрьской близ школы №36, на Новоселовке, на Фонтанной, 87 (ныне отдел
полиции), на Евпаторийской, 56 (Дом учителя), на Митрополитской, 5, (Управление
жилищно-коммунального хозяйства) на Торговой, 17, (бывший трест
\enquote{Ждановжилстрой}), на Итальянской, 59 (Городская налоговая инспекция), на
Слободке (бывшая железнодорожная школа). Ни одно из названных выше зданий не
используется сейчас по своему первоначальному назначению. В здании, в котором
находится сейчас Мариупольский профессиональный металлургический лицей (улица
Митрополитская, 61), в 1930 году помещался металлургический техникум. А в
индустриальном колледже на Георгиевской, 69 был техникум педагогический. 

Медицина на плане представлена городской больницей на Покровской площади
(теперь Горбольница №3), поликлиникой у Центрального сквера и тубдиспансером на
Георгиевской улице, 72. К этому можно добавить несколько адресов из справочника
\enquote{Довідкова та адресові книга}, упомянутого выше. Городская милиция находилась
по адресу - улица Карла Либкнехта (теперь Митрополитская), 25, окружной суд –
улица Энгельса, 24, биржа труда – на углу улиц Карла Либкнехта и Архипа
Куинджи), окружной военкомат улица ΙΙΙ-го Интернационала (ныне Торговая), 2 и
так далее.

\textbf{Читайте также:} \href{https://mrpl.city/blogs/view/kak-ulitsy-nazyvali}{Как улицы называли}

Что не говорите, а все-таки интересно рассматривать планы и справочники старого
Мариуполя, при этом как бы совершаешь путешествие в машине времени.
