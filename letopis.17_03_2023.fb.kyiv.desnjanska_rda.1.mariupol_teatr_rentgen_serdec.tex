%%beginhead 
 
%%file 17_03_2023.fb.kyiv.desnjanska_rda.1.mariupol_teatr_rentgen_serdec
%%parent 17_03_2023
 
%%url https://www.facebook.com/desn.rda/posts/pfbid028j7NvqGmhQF9fULm5xmT6GLCvtkuD3vC5V5okompQK8EvLsEx9fqVSskAjwGWvqtl
 
%%author_id kyiv.desnjanska_rda
%%date 17_03_2023
 
%%tags mariupol,kultura,kiev,teatr
%%title Актори маріупольського театру зіграли благодійну виставу "Рентген усміхнених сердець"
 
%%endhead 

\subsection{Актори маріупольського театру зіграли благодійну виставу \enquote{Рентген усміхнених сердець}}
\label{sec:17_03_2023.fb.kyiv.desnjanska_rda.1.mariupol_teatr_rentgen_serdec}

\Purl{https://www.facebook.com/desn.rda/posts/pfbid028j7NvqGmhQF9fULm5xmT6GLCvtkuD3vC5V5okompQK8EvLsEx9fqVSskAjwGWvqtl}
\ifcmt
 author_begin
   author_id kyiv.desnjanska_rda
 author_end
\fi

🎭Актори маріупольського театру зіграли благодійну виставу \enquote{Рентген усміхнених
сердець}

👉16 березня, у річницю трагічних подій😢, коли російська авіабомба зрівняла з
землею драматичний театр в Маріуполі, забравши життя сотень українців,
маріупольський театр авторської п'єси \enquote{Концепція} показав благодійну виставу
\enquote{Рентген усміхнених сердець}. ✔️Вистава пройшла у \href{https://www.facebook.com/lybra115}{Бібліотека 115 для дітей м.
Києва} нашого району, де актори майже рік тому знайшли прихисток та можливість
для подальшої реалізації своїх планів та розвитку. 

🎭З трагікомедії \enquote{Рентген усміхнених сердець} театр авторської п'єси розпочав
свою роботу у Маріуполі. Це була аншлагова вистава, яка мала неабиякий успіх та
підкорила багатьох маріупольців. Пєсу для постановки написав художній керівник
театру та режисер Олексій Гнатюк. При цьому назва спектаклю йому наснилася.

👉Вчора, глядачі спостерігали за всім, що відбувається на сцені, затамувавши
подих. Ця вистава нікого не залишила байдужим, адже сама історія є досить
повчальною та проникливою. 

🤝Присутні на виставі були заступниця голови Київської міської адміністрації
\href{https://www.facebook.com/starostenkodesnanka}{Ганна Старостенко},
керівник Деснянського району Дмитро Ратніков, депутати Київради
\href{https://www.facebook.com/ISHCHENKO.org.ua}{Іщенко Михайло - Депутат
Київради}, Віктор Грушко, Ігор Опадчий, Ілля Кушнір, Олена Овраменко, перший
заступник голови Деснянської РДА \href{https://www.facebook.com/profile.php?id=100003219216765}{Ірина Алєксєєнко} та заступник голови ДРДА
\href{https://www.facebook.com/profile.php?id=100050622292319}{Микола Загуменний}.

🗨Ганна Вікторівна Старостенко подякувала акторам за професійну гру та
зазначила, що не зважаючи на все жахіття війни, ці скрутні часи, мистецтво має
магічну силу: дозволяє хоч ненадовго забути про страшні пережиті моменти та дає
натхнення боротися далі. Ми обов'язково все відвоюємо, все відбудуємо і
обов'язково поїдемо дивитися цю та інші вистави в наш український
Маріуполь💛💙!

🗨«У такий важкий для кожного маріупольця час ми знайшли підтримку у Києві, що
став другою домівкою для нас. Саме тут, на Троєщині, у Дитячій бібліотеці №115
Деснянського району ми вперше зустрілися колективом після трагічних подій та
розпочали відбудовувати наш театр», - зазначила директорка Департаменту
культурно-громадського розвитку Маріупольської міської ради Діана Трима.

🫶Усі зібрані кошти з благодійного вечора підуть на потреби ЗСУ🪖. 

✔️За матеріалами: КМДА
