% vim: keymap=russian-jcukenwin
%%beginhead 
 
%%file 04_02_2022.yz.soullaway_soullaway.1.voronezh_nikto_ne_bombit
%%parent 04_02_2022
 
%%url https://zen.yandex.ru/media/id/5dbc544f2f1e4400aff32830/voronej-kotoryi-nikto-ne-bombit-61fbcf48ae3624671f63b65c
 
%%author_id yz.soullaway_soullaway
%%date 
 
%%tags rossia,voronezh
%%title Воронеж, который никто не бомбит
 
%%endhead 
 
\subsection{Воронеж, который никто не бомбит}
\label{sec:04_02_2022.yz.soullaway_soullaway.1.voronezh_nikto_ne_bombit}
 
\Purl{https://zen.yandex.ru/media/id/5dbc544f2f1e4400aff32830/voronej-kotoryi-nikto-ne-bombit-61fbcf48ae3624671f63b65c}
\ifcmt
 author_begin
   author_id yz.soullaway_soullaway
 author_end
\fi

Понятия не имею откуда взялось в интернете выражение на тему того, что кто-то
разбомбил Воронеж, но сдаётся мне эта история началась где-то в 2014 году.
Примерно тогда, когда начались у нас и разногласия с Украиной. Кто-то очень
хитрый решил запустить целую компанию по дискредитации наших городов. Но выбора
Воронежа с моей точки зрения тут был очень неудачным. Ведь люди хотя бы немного
увлекающиеся Россией в курсе, что это не то что не безнадёжный город, а
наоборот весьма перспективный. Да и не бывает городов с населением в миллион
жителей, которые находятся в упадке.

Уже чувствую, как тянется чья-то рука написать что-то об Омске, но спешу
разочаровать. Прежде чем мне что-то рассказывать об этих двух городах я
рекомендую вам туда съездить. Я лично бывал и в Воронеже, и в Омске и ничего не
видел там ужасающего. Более того города эти растут и туда активно тянется
молодежь. А именно молодежь залог процветания любого мегаполиса. Компанию же по
дискредитации наших городов явно запускали в угоду Украине. Вот только
расскажите мне о поведении молодежи в этой прекрасной стране. Правильно, там
пределом мечтаний молодого человека считается переезд на сбор какой-нибудь
клубники в Польшу. То есть в России люди хотят стать студентами и работать у
себя дома, а на Украине уехать из страны. Чувствуете разницу? По-моему, она
очевидна. Однако вернёмся к Воронежу.

Почему я вообще обратил на него внимание? Да, я не местный житель. То есть могу
смотреть на город беспристрастно. С ним меня ничего не связывает. Ну не считать
же связью прослушивание группы «Сектор газа» в детстве. Единственный момент,
который тут можно притянуть за уши это то, что Воронеж негласно считается
столицей Черноземья. А это в целом уже моя малая родина. И честно говоря
Воронеж мне чем-то напомнил мой родной Орёл. Только там улицы шире, дома выше и
чувствуется размах мегаполиса. Схожа там к слову и история. Город серьёзно
пострадал во время войны и его пришлось поднимать фактически из руин. Но если о
Волгограде у нас все знают, то Воронеж в этом плане незаслуженно обойдён
вниманием.

\ii{04_02_2022.yz.soullaway_soullaway.1.voronezh_nikto_ne_bombit.pic.1}

Однако по какой причине был выбран Воронеж как объект пропаганды я могу только
предполагать. Возможно, как раз по той причине, что он как-то не на слуху. Если
в том же Волгограде или Калининграде проводили матчи чемпионата мира по
футболу, то Воронеж словно выпал из всех СМИ. Обойдён он и вниманием туристов.
В Казань или Нижний Новгород народ активно едет и эти города не получится
преподнести с негативной стороны потому что многие там все-таки бывали. Воронеж
же загадочен для большинства жителей России.

Вообще меня берёт жуткая досада, что у многих моих соотечественников нет
интереса и любопытства к стране, в которой нам выпало жить. Более того
некоторыми недальновидными ребятами культивируется мысль о том, что у нас тут
всё однообразное и города похожи друг на друга. Я конечно объехал даже не
половину субъектов России, но точно могу заявить, что нет двух похожих городов
у нас. Тем более среди мегаполисов. И уж хотя бы эти 15 городов гигантов стоит
ради интереса посетить. Как минимум это полезно в познавательных целях. Да
подобный опыт туризма никогда не бывает лишним.

Да, от Воронежа не надо ждать красот Казани или даже Самары, но стоит
учитывать, что старый город был сметен войной. Новый же город вполне себе
крепкий, интересный и любопытный. На пару дней там застрять вполне себе благое
дело. И самое главное, что никаких разбитых улиц и упадка там нет. Это в
интернете любят нагнетать, а реальность-то всегда сильно отличается.

Так что если вы живёте где-то недалеко и думаете, чем бы заняться на майские
праздники, то самое время уже планировать поездку. В том числе и в Воронеж.
Познание своей станы дело благое. А уж если кто-то что-то пишет негативное о
каком-то конкретном месте, то это лишний повод его посетить. И убедиться в
очередной раз, что реальность абсолютно непохожа на какие-то слухи о нищей
России. У нас огромная и непостижимая страна. Иллюзии же о том, что здесь
ничего не происходит принадлежат тем людям, которые её не видели и застряли в
90-х. Только 90-е кончились два десятилетия назад. И уж Воронеж точно времени
не терял за эти годы.

\ifcmt
  ig https://avatars.mds.yandex.net/get-zen_doc/196516/pub_61fbcf48ae3624671f63b65c_61fbcf99af362c7c21a46cc7/scale_1200
  @wrap center
  @width 0.8
\fi

Единственное, что стоит отметить для справедливости так это печальную историю с
местным трамваем. Его власти загубили. Для такого мегаполиса это конечно же
печальная история. Но кто знает? Возможно когда-то придёт там к власти умный
человек и реанимирует все-таки трамвайную сеть.

У меня же на сегодня всё, спасибо вам за внимание и до новых встреч.
