% vim: keymap=russian-jcukenwin
%%beginhead 
 
%%file 06_07_2020.fb.fb_group.story_kiev_ua.1.fotografia_vospominania.pic.8
%%parent 06_07_2020.fb.fb_group.story_kiev_ua.1.fotografia_vospominania
 
%%url 
 
%%author_id 
%%date 
 
%%tags 
%%title 
 
%%endhead 

\ifcmt
  ig https://scontent-frt3-1.xx.fbcdn.net/v/t1.6435-9/106996597_3380589138641336_6569284068475982428_n.jpg?_nc_cat=108&ccb=1-5&_nc_sid=b9115d&_nc_ohc=5dSz7DRHlmMAX_qjFtH&_nc_ht=scontent-frt3-1.xx&oh=a6855d726d5b676cd3b382e7b0f95baa&oe=61B6CB47
  @width 0.4
\fi

\iusr{Ирина Петрова}
Ель была голубая)

\iusr{Alik Perlov}
А штаны  @igg{fbicon.face.tears.of.joy}{repeat=4} ???

\iusr{Ирина Петрова}

Шо??? У тебя таких не было??? Это были фланелевые штаниики веселого мышиного
цвета! Мяяяконькие, немаркие, всё можно было обтирать - и парапеты, и горки, шо
хочешь)))

\iusr{Татьяна Оксаненко}
Шароварики — было знатно!

\emph{Ирина Петрова}
\textbf{Татьяна Оксаненко} 

да! Шаровары!!! Точно!!! После снежных катаний с гор они дубели в камень, дома
сохли на батарее, подтекал лужи...

\iusr{Olena Klymenko}

Прекрасный рассказ, Ирина! Порадовали. Скверик возле театра Франко, и все
склоны над театром были основным местом наших, школьных прогулок. Я училась в
94 школе. Так что, это всё родные места.

\iusr{Ирина Петрова}
\textbf{Olen A Klymenko} о! Так мы одношкольники!!! Как приятно!!! В каких годах Вы учились? @igg{fbicon.heart.eyes} 

\iusr{Olena Klymenko}
\textbf{Ирина Петрова} С 1959-1969г, классная Анна Филимоновна Божко.

\iusr{Olena Klymenko}

Я вспомнила фамилию Бухтияров. Мне кажется, что с девочками из его класса я
играла в сборной по волейболу на первенство района!!!!

\iusr{Ирина Петрова}
\textbf{Olen A Klymenko} 

о! Как раз после вашего класса она пришла к нам! Значит, это класс Володи
Фадеичева, Жени Папуши, Сережи Яворского? Они в прошлом году отмечали 50-летие.
Вернее, это параллельный класс, у них была Раиса Романовна)

\iusr{Olena Klymenko}
\textbf{Ирина Петрова} 

Да, наш класс был 10-б. Хотя, в 10 классе Раиса Романовна долго болела, и
Михаил Григорьевич нас обьединял - и проводил подряд по 2-3 урока математики. Мы, 
смогли это оценить, только спустя годы. Многие поступили и успешно закончили
технические вузы.

\iusr{Ирина Петрова}
\textbf{Olen A Klymenko} 

о! Шваецкий - это легенда. Из бурого угля он смог сделать вполне пригодный
материал)))) а Вы знаете, что эти ребята встречаются каждый год!!!? В прошлом
году был Саша Самуджан.

\iusr{Olena Klymenko}
\textbf{Ирина Петрова} 

Да, знаю. Там не только наш выпуск, но и выпучкники разных лет. Так сложилось,
что Саша Самунджан, часто бывал на встречах с "а" классом. И Юра Глейзер,
раньше дружил с ребятами из "а" класса.

\iusr{Olena Klymenko}
\textbf{Ирина Петрова} 

Нет, после 20 лет - не была. Как то, встретились втроем, с Сашей Самунджаном и
Наташей Дудаковой - это все, кого Саша смог уговорить. Наверно, в 1999 году.

\iusr{Olena Klymenko}

Последнее воскресенье мая, если не ошибаюсь. Классы у нас были не большие.
Первые 20 лет, и наш класс собирался, но не было такого, как Володя Фадеичев
организатора.

\iusr{Ирина Петрова}
\textbf{Olen A Klymenko} а Вы бывали на встречах? Очень всё душевно)))

\iusr{Ирина Петрова}

Да, встречаются на площади, потом едут к Володе на дачу. Мы работали с Володей
вместе, в детстве играли на площади, как-то так и по жизни дружили. Я тоже
прихожу к ним, мне очень интересно слушать старших))) кое-что и я и рассказываю
о тех годах, когда они уже вышли из школы. В этом году встречи не
было... карантин...

\iusr{Olena Klymenko}

Когда так давно не видишь людей, а в памяти все 17 летние... трудно решиться
увидеть сегодня, слегка изменившихся друзей- одноклассников.

\iusr{Ирина Петрова}
Вооот! И чем чаще встречаешься, тем меньше видны перемены))))

\iusr{Olena Klymenko}
Конечно!!! @igg{fbicon.face.smiling.eyes.smiling} 

\iusr{Olena Klymenko}
А Вы в каком году закончили школу?

\iusr{Ирина Петрова}
В 1973 м
