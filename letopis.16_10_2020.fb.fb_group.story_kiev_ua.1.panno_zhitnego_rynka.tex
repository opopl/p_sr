% vim: keymap=russian-jcukenwin
%%beginhead 
 
%%file 16_10_2020.fb.fb_group.story_kiev_ua.1.panno_zhitnego_rynka
%%parent 16_10_2020
 
%%url https://www.facebook.com/groups/story.kiev.ua/posts/1486857991510997/
 
%%author_id fb_group.story_kiev_ua,petrova_irina.kiev
%%date 
 
%%tags gorod,kiev
%%title Панно Житнего рынка
 
%%endhead 
 
\subsection{Панно Житнего рынка}
\label{sec:16_10_2020.fb.fb_group.story_kiev_ua.1.panno_zhitnego_rynka}
 
\Purl{https://www.facebook.com/groups/story.kiev.ua/posts/1486857991510997/}
\ifcmt
 author_begin
   author_id fb_group.story_kiev_ua,petrova_irina.kiev
 author_end
\fi

Когда-то  давно, в начале 80-х, был у меня друг. Собирался стать историком и
мои мужем) Он первый показал мне панно Житнего рынка, рассказал о значении
элементов, объяснил логику размещения корабликов и гербов городов. Честно
скажу, не всё запомнила из его рассказа, да и мужем не стал, спросить
вотпрямщпас не у кого)

\ifcmt
  tab_begin cols=3

     pic https://scontent-frx5-1.xx.fbcdn.net/v/t1.6435-9/121728628_3710113035688943_1061174723780239924_n.jpg?_nc_cat=110&ccb=1-5&_nc_sid=b9115d&_nc_ohc=zN5BvhevnrYAX8qI3dV&_nc_ht=scontent-frx5-1.xx&oh=c76de2c745325009c937da502e0ce9e9&oe=61B4F68C

     pic https://scontent-frt3-1.xx.fbcdn.net/v/t1.6435-9/121694456_3710113269022253_2089952052412102457_n.jpg?_nc_cat=102&ccb=1-5&_nc_sid=b9115d&_nc_ohc=YwOduHPNpUQAX-b04yj&tn=lCYVFeHcTIAFcAzi&_nc_ht=scontent-frt3-1.xx&oh=451138d4ced35abe1e01658266fbcf24&oe=61B527FE

		 pic https://scontent-frt3-1.xx.fbcdn.net/v/t1.6435-9/121779997_3710113679022212_4207619584001025236_n.jpg?_nc_cat=107&ccb=1-5&_nc_sid=b9115d&_nc_ohc=s_fZ1rRN8mQAX8-FXcW&_nc_ht=scontent-frt3-1.xx&oh=27d3a71e8deec00b5a906ed6b2ba224c&oe=61B72B7F

  tab_end
\fi

Недавно была чудесная онлайн экскурсия Макса Дербенева и Макса Олейника как раз
на эту тему. Так всё всколыхнула! Решила съездить туда, сфотографировать и
рассказать об этом. А тут как раз и Олег Коваль пишет рассказ о Житнем. Сама
судьба!

\ifcmt
  tab_begin cols=3

     pic https://scontent-frx5-1.xx.fbcdn.net/v/t1.6435-9/121775388_3710136749019905_1089446120531147960_n.jpg?_nc_cat=105&ccb=1-5&_nc_sid=b9115d&_nc_ohc=tv85BCrk9GoAX_YVW8v&tn=lCYVFeHcTIAFcAzi&_nc_ht=scontent-frx5-1.xx&oh=60e3cc6e5a530d68c1a28498a8c38d26&oe=61B44935

     pic https://scontent-frx5-1.xx.fbcdn.net/v/t1.6435-9/121833614_3710136842353229_7883696197047204407_n.jpg?_nc_cat=105&ccb=1-5&_nc_sid=b9115d&_nc_ohc=LwiC2s7QGioAX-9cocl&_nc_ht=scontent-frx5-1.xx&oh=18e8a4e3d20fadede2979ee663f97a8e&oe=61B45AB0

		 pic https://scontent-frx5-1.xx.fbcdn.net/v/t1.6435-9/122035857_3710136959019884_4505433116617635625_n.jpg?_nc_cat=100&ccb=1-5&_nc_sid=b9115d&_nc_ohc=k8YFNipIlRcAX8elCD_&_nc_ht=scontent-frx5-1.xx&oh=98f2563e22b9366c4746ac12be9ac8d3&oe=61B6593F

  tab_end
\fi

многие знают, что фасады  рынка украшены монументальным декоративным панно «Из
варяг в греки» авторства заслуженного художника Украины Анатолия Домнича. Они
словно надвигаются на угол здания как на нос корабля, определяя его мысленный
вектор движения. На панно изображены два путешествия: одно - во времена
Киевской Руси, второе - в советские годы. И среди списка городов, через которые
проходят суда, есть Ленинград и Днепропетровск.

\ifcmt
  tab_begin cols=4

     pic https://scontent-frt3-2.xx.fbcdn.net/v/t1.6435-9/121722004_3710137082353205_7417511866160220112_n.jpg?_nc_cat=101&ccb=1-5&_nc_sid=b9115d&_nc_ohc=PssIUWR94q4AX_8-OeW&_nc_ht=scontent-frt3-2.xx&oh=919f344b866cc5ad9d4c84a52f41631b&oe=61B55829

     pic https://scontent-frt3-1.xx.fbcdn.net/v/t1.6435-9/121805558_3710137159019864_2529161701428614409_n.jpg?_nc_cat=102&ccb=1-5&_nc_sid=b9115d&_nc_ohc=nJ526yUpt5AAX_tjXLm&tn=lCYVFeHcTIAFcAzi&_nc_ht=scontent-frt3-1.xx&oh=3d66a7b1bc9bab6c706b1fcd1e21701b&oe=61B783E4

		 pic https://scontent-frx5-1.xx.fbcdn.net/v/t1.6435-9/121774772_3710137222353191_4467164948833188174_n.jpg?_nc_cat=105&ccb=1-5&_nc_sid=b9115d&_nc_ohc=BGQzOIF3KGIAX9HUh0u&tn=lCYVFeHcTIAFcAzi&_nc_ht=scontent-frx5-1.xx&oh=f4248f9fa4952abadc7d11720907504d&oe=61B4820B

		 pic https://scontent-frx5-1.xx.fbcdn.net/v/t1.6435-9/121814945_3710137382353175_5724362860449616907_n.jpg?_nc_cat=111&ccb=1-5&_nc_sid=b9115d&_nc_ohc=lcsNPYOGmEAAX90fnD5&_nc_oc=AQn1wmRgLPpCSG6r8fq7z5ajH9sIU72gLmABRrDRVgCUiwpPuGJsC3xyDMsXukHyL5s&_nc_ht=scontent-frx5-1.xx&oh=bbd5a0dcd169dc0252011b709ec18ae4&oe=61B6FECF

  tab_end
\fi

Начинается путешествие в великом Константинополе.  (т.е. начальная точка - из
"греков"). Заметьте, как идут стрелочки -треугольники - вверху - от греков к
варягам, внизу - "из варяг в греки". И кораблики так же ориентированы носами.

Понт Эвксинский ( Чёрное море)... через пороги (Березань, Хортица, остальные
прочитать не могу), потом Переяславль, и в Киев! Слева от знака Киева выделен
Печерск! А справа - Подол! Дальше - Любеч,  Смоленкск, Старая Русса,
Новгород...на углу часы с названием "Старое время" Очень любопытный циферблат.
Цифры обозначены буквами - Аз, Веди, Глагол, Добро с титлами - это обозначение
цифр. А вот почему пропущена одна буква? Кто знает? Только, чур, не гуглить)

\ifcmt
  tab_begin cols=3

     pic https://scontent-frt3-1.xx.fbcdn.net/v/t1.6435-9/121780224_3710137452353168_1070278283851084080_n.jpg?_nc_cat=104&ccb=1-5&_nc_sid=b9115d&_nc_ohc=zzpR3EO0d7sAX-J-DK2&_nc_ht=scontent-frt3-1.xx&oh=e6b37e046b5b37c1e07704dbc774b1cc&oe=61B77D99

     pic https://scontent-frx5-1.xx.fbcdn.net/v/t1.6435-9/121735562_3710137542353159_626189480989072883_n.jpg?_nc_cat=100&ccb=1-5&_nc_sid=b9115d&_nc_ohc=xP1hp24BYAAAX8u4Sxk&_nc_ht=scontent-frx5-1.xx&oh=c39062fdb9c851e9b2f12efb2eabdacb&oe=61B65232

		 pic https://scontent-frt3-1.xx.fbcdn.net/v/t1.6435-9/121807134_3710137685686478_1399543108466301399_n.jpg?_nc_cat=106&ccb=1-5&_nc_sid=b9115d&_nc_ohc=AOPGesjN_OcAX_2Ed3d&_nc_ht=scontent-frt3-1.xx&oh=7e22279dedac8533181290f377f0977d&oe=61B3BF41

  tab_end
\fi

За углом - время новое "Нові часи"! Тут с циферблатом всё понятно)

Черное море, порты - Констанца, Варна, Одесса, порты на Днепре - Херсон,
Запорожье, Днепропетровск.

Потом - Смоленск,  Новгород, Ладога, потом по Неве - в Ленинград, оттуда в
Балтийское море.

\ifcmt
  tab_begin cols=3

     pic https://scontent-frt3-2.xx.fbcdn.net/v/t1.6435-9/121803112_3710137752353138_2776124921510726990_n.jpg?_nc_cat=103&ccb=1-5&_nc_sid=b9115d&_nc_ohc=6-repbb_m2gAX9AARtH&_nc_ht=scontent-frt3-2.xx&oh=6860e1dbfdfb1e76d3954c47672d713d&oe=61B57966

     pic https://scontent-frx5-1.xx.fbcdn.net/v/t1.6435-9/121592908_3710137849019795_441830576329132548_n.jpg?_nc_cat=111&ccb=1-5&_nc_sid=b9115d&_nc_ohc=84xYzNrjja0AX_ta0lh&tn=lCYVFeHcTIAFcAzi&_nc_ht=scontent-frx5-1.xx&oh=52ae0eaabf1712226326a70f198c9fbd&oe=61B78F21

		 pic https://scontent-frt3-2.xx.fbcdn.net/v/t1.6435-9/121740112_3710138309019749_8978701604337542295_n.jpg?_nc_cat=101&ccb=1-5&_nc_sid=b9115d&_nc_ohc=rQoPy6J_KS8AX-V4tV0&_nc_ht=scontent-frt3-2.xx&oh=629031b00edb3708031e2efd481abfda&oe=61B58070

  tab_end
\fi

"Торговля есть единственно возможная экономическая связь между десятками
миллионов". Вот так!

Так много пустого места на втором этаже) здесь есть лифт. И, кстати, на рабочих
местах торговцев мясом и рыбой устроены краны с водой, чтобы мыть руки, не
отходя от рабочего места.

Буйство красок осени зашкаливает - горы гарбузов та кавунчиков. И еще есть тут
одно "местечко", где можно полакомиться обалденными чебуреками и янтыками. А
какой тут айран, ммм!!!

Ими я поделилась даже с сорокой-белобокой)

И еще один маленький штришок. Очень много встречается публикациях сожалений о
том, что старые подольские дома разрушаются, возводятся новые. Под лозунгом
"Всё пропало! гипс снимают! Клиент уезжает!" описывается это. как окончательное
разрушение города. Я соглашусь, что есть отдельные, довольно уродливые здания,
чуждые общему ансамблю старинного района. Но, на мой взгляд, есть и довольно
приемлемые новые решения. Во всяком случае - на фото - мне все же больше
нравится здание новое, чем старое...но, это на мой взгляд...
