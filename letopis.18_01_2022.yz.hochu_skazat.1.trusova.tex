% vim: keymap=russian-jcukenwin
%%beginhead 
 
%%file 18_01_2022.yz.hochu_skazat.1.trusova
%%parent 18_01_2022
 
%%url https://zen.yandex.ru/media/id/5f6b7c2f245af34c245197df/sasha-trusova-uporstvo-61e5c50f6e15de3443187ced
 
%%author_id yz.hochu_skazat
%%date 
 
%%tags devushka,figurist,figurnoje_katanie,krasota,rossia,sport,trusova_aleksandra
%%title Саша Трусова. Упорство
 
%%endhead 
\subsection{Саша Трусова. Упорство}
\label{sec:18_01_2022.yz.hochu_skazat.1.trusova}

\Purl{https://zen.yandex.ru/media/id/5f6b7c2f245af34c245197df/sasha-trusova-uporstvo-61e5c50f6e15de3443187ced}
\ifcmt
 author_begin
   author_id yz.hochu_skazat
 author_end
\fi

История с Аней Щербаковой на этом чемпионате затмила других фигуристок. А
хочется написать и о них. И начну я с Саши Трусовой. Многие каналы писали о её
баллах и о бронзе. Выскажу своё мнение. Саша отличается удивительным упорством.
Она поставила себе цель и к ней идёт. Через тернии, через непонимание
окружающих, через критику. Только вперёд. За упорство и бесстрашие я Сашу очень
уважаю.

\ii{18_01_2022.yz.hochu_skazat.1.trusova.pic.1}

На этом Чемпионате Саша пыталась сделать то, к чему идёт давно. Думаю, что
Этери Георгиевна сознательно разрешает Александре пробовать много четверных в
программе. Поди, останови, Ракету, когда она на свой курс нацелилась! Саше
нужно разрешать то, что она хочет. Иначе будет опять бегство в никуда. Поэтому
Саша пробует, примеривается. И надо отдать должное мудрости Этери Георгиевны,
которая стоически относится к Сашиным экспериментам.

\ii{18_01_2022.yz.hochu_skazat.1.trusova.pic.2}

Очень много мнений по поводу того, что Саша явно устала ко второй половине
программы. Но мы забываем, что, во-первых, чемпионат России был тяжелейшей
психологической нагрузкой не только для Ани, но и для Саши, и для Камилы.
Во-вторых. после ЧР были новогодние каникулы и перерыв в тренировках. Это точно
сказалось на всех девочках, не только на Саше. В-третьих, я думаю, что хейт,
полившийся на Аню, шокировал не только её, но и девочек тоже. Они же понимают,
что также в любой момент могут стать объектом такого же хейта. Хотя, скорее
всего, проплаченные журналюшки не будут трогать Сашу. У неё же контракт на шоу
у Рудковской-Плющенко. А им там нужна Русская Ракета во всей красе и с медалью
обязательно. 

Очень хочется, чтобы на время олимпиады Саша приостановила свои эксперименты с
пятью четверными и тройным акселем и катала какой-то разумный контент. Чтобы
было рискованно, но чистенько. Там судить будут жёще и тянуть на пьедестал не
только Луну Хендрикс, но и японок, и американок. Поэтому желаю Саше к олимпиаде
всё же включить голову и трезвый расчёт. А после Олимпиады можно прыгать
дальше)))

Но вообще, я восхищаюсь упорством нашей Ракеты. Знаю, что не все к Саше хорошо
относятся. Для меня же её смелость, прямота и огромная увлеченность своим делом
являются определяющим фактором. Поэтому желаю Саше только успехов! Пусть всё
получится! 

А в финале статьи - Сашино показательное выступление. Я не люблю показательные.
Но тут в очередной раз поразилась, насколько в \enquote{Хрустальном} точно подбирают
музыку. На 100\% Сашин вариант! Под её характер! 
