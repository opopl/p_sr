% vim: keymap=russian-jcukenwin
%%beginhead 
 
%%file 11_01_2022.fb.zharkih_denis.1.ljubov_i_politika
%%parent 11_01_2022
 
%%url https://www.facebook.com/permalink.php?story_fbid=3163721877174541&id=100006102787780
 
%%author_id zharkih_denis
%%date 
 
%%tags ljubov,politika,ukraina
%%title Любовь и политика
 
%%endhead 
 
\subsection{Любовь и политика}
\label{sec:11_01_2022.fb.zharkih_denis.1.ljubov_i_politika}
 
\Purl{https://www.facebook.com/permalink.php?story_fbid=3163721877174541&id=100006102787780}
\ifcmt
 author_begin
   author_id zharkih_denis
 author_end
\fi

Любовь и политика.

Любовь серьезно связана с развитием. И тут все равно какая любовь - к
противоположному полу или к Родине. Любовь к детям, например, серьезна связана
с развитием семьи и каждого ее члена в частности. Развивается, в первую очередь,
тот, кто любит, но он, как правило, развивает объект своей любви. У меня есть
сказка на эту тему, где линия любви молодых людей пересекается с линией любви
народа и правителя. 

К чему это я? Правитель, который любит свою страну, развивает ее. Развитие
страны и есть показатель любви правящей верхушки к народу. Но из этого следует,
что правящий класс Украину ненавидит, поскольку не дает ей развиваться. Да и в
народе давно притаилась неосознанная ненависть. Все это ведет к катастрофе
страны.

Славящие Украину негодяи ничуть не любят свою страну, поскольку они носятся с
символами, а не с реальной Украиной. Фетишист не способен на любовь, и идет в
фетишисты именно по этой причине. Носится с фетишем гораздо проще и менее
ответственно, чем любить живого человека.  Когда человек любит выдуманного
человека, выдуманную страну, то он любит не человека, не страну, а выдуманного
себя. И за криками \enquote{Слава Украине!} хорошо слышится \enquote{Слава мне великому!}. И,
вообще, это все производное от \enquote{Слава КПСС!} и еще более ранних ритуалов. 

Независимость во многих украинцах разбудила вовсе не лучшие, а худшие чувства -
зависть, жадность, озлобленность, ненависть. В стране планомерно убивают
любовь, как жить без любви? Без машины можно, без модной одежды - легко, без
отдыха за границей - не вопрос, без любви невозможно жить полноценной жизнью.
Мы становимся неполноценными не тогда, когда осознаем дефицит средств, а тогда,
когда лишаемся любви. И Украина плодит неполноценных граждан. Неполноценные это
вовсе не те, кто не говорит на государственном языке, а те кто сеет ненависть,
злобу, глупость. 

Вот этой ненавистью, злобой и глупостью убивают страну, даже когда при этом
признаются ей в любви. Не страну они любят, а свои больные фантазии. Вот
фантазии, как раз вовсю и развиваются. Жаль, что страна при этом гибнет.

2020

\ii{11_01_2022.fb.zharkih_denis.1.ljubov_i_politika.cmt}
