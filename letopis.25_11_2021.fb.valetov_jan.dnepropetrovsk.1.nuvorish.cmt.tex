% vim: keymap=russian-jcukenwin
%%beginhead 
 
%%file 25_11_2021.fb.valetov_jan.dnepropetrovsk.1.nuvorish.cmt
%%parent 25_11_2021.fb.valetov_jan.dnepropetrovsk.1.nuvorish
 
%%url 
 
%%author_id 
%%date 
 
%%tags 
%%title 
 
%%endhead 
\subsubsection{Коментарі}

\begin{itemize} % {
\iusr{Игорь Марьяш}
Так если бы козу через АМКУ, через закон об экологии и ренте на руду, то все получилось бы. И народ бы поддержал.
А так это не коза

\iusr{Николай Завадский}
\textbf{Игорь Марьяш} Зеленский со своей компанией живет в придуманном ими мире, и это бы еще полбеды. Беда в том, что они хотят, чтобы реальный мир был таким же, как придуманный...

\iusr{Volodymyr Chornyi}
Из грязи в князи.

\iusr{Андрей Подгайный}
Бидло - воно і в Африці Бидло.

\iusr{Игорь Мартыненко}
Мабуть час \#радникам повертати старі хештеги.
\#нє\_ной\_а\_пілотку\_міняй

\iusr{Iren Semivolos}
Ахметовські гроші не старі, просто кров'ю пахнуть вже не так сильно з часом

\iusr{Vadim Ryzhkov}
И опыт Саакашвили, который сидит в грузинской тюрьме, им не пример)

\iusr{Александр Кочетков}

Таки да. Респект поющим реквием! То есть, нам с тобой.

\iusr{Ruslan Pankevych}
Найгірше, що вони виїдуть при першому шухері, а країна залишиться ще з більшими проблемами і війною на порозі.

\iusr{Сергей Волнянский}
Логично! Согласен!

\iusr{Игорь Кольчик}

Деолигархизация - процесс запоздало необходимый. Долой карманные партии и
каналы! Не хочу, что бы судьбой моих внуков управлял самый хитровы@баный.
Надеюсь хватит потенциала у ЗЕ энд компани довести дело до конца.

\begin{itemize} % {
\iusr{Володимир Бердянський}
\textbf{Игорь Кольчик} ага, все партии и каналы должны принадлежать эль президентэ

\iusr{Игорь Кольчик}
\textbf{Володимир Бердянський} Вова, не передергивай. Партия должна принадлежать идее. За идею и голосует народ.

\iusr{Татьяна Альбертович}
\textbf{Володимир Бердянський} такой себе Фидель Кастро...

\iusr{Козюрман Ольга}
\textbf{Игорь Кольчик} за кино веселое  @igg{fbicon.wink} 

\iusr{Володимир Бердянський}
\textbf{Игорь Кольчик} Игорёчек, какой идее будут принадлежать партии, газеты и телеканалы после того, как их себе заберёт Эль Президенте?

\iusr{Игорь Кольчик}
Вовчик! Как заберёт? Задумайся над процессом. Или учи матчасть. Ненависть плохой советчик....

\iusr{Володимир Бердянський}
\textbf{Игорь Кольчик} ты предлагаешь, чтобы президент запретил финансирование партий и телеканалов капиталистами, вот ты и подумай, к чему это приведёт

\iusr{Игорь Кольчик}

Будут шоу, фильмы, детские программы, спорт и многое другое.... Если мне будет
нравится, я куплю их просмотр. Политические кастрирования и гильетины отомрут.
Останется обмен. Мнениями.

\iusr{Игорь Кольчик}
Ляшко и Гройсманы(ничего личного) не смогут навязывать мне и бабуле из пригорода свои фейсы и хотелки.

\iusr{Dembel Fox}
\textbf{Игорь Кольчик} , простите , а процесс деолигархизации это когда Коломойскому дают мародерствовать на энергорынке и Центрэнерго или когда сливают денежку из Укрнафты?

\iusr{Антон Овсиенко}
\textbf{Игорь Кольчик} Прям как в России.

\iusr{Игорь Кольчик}
Прощаю. Всех под контроль! Никаких поблажек. У меня, как и 40 миллионов украинцев, там родственников НЕТ!

\iusr{Володимир Бердянський}
\textbf{Игорь Кольчик} надои повышаются. Преступность побеждена. Все на майскую демонстрацию против козней Запада! Никакой радиации не существует!

\iusr{Игорь Кольчик}
Да. И срочно сообщить,. кто убил Кенеди! Спасибо за общение. Бред - без меня....

\iusr{Инна Тарнавская}
\textbf{Dembel Fox} вот хотела спросить, но забила, ибо в ответ будет витиеватый бред.  @igg{fbicon.face.grinning.smiling.eyes} 

\iusr{Володимир Бердянський}
Сплотимся же все вокруг нашего Господина! Великий Зе, сотрясатель вселенной, да святится Имя твое, на небе и на земле! Правь нами, распоряжайся нами, во веки веков! Ермакаминь!

\iusr{Dembel Fox}
\textbf{Инна Тарнавская} , ко мне на прием раз в месяц обязательно попадал городской сумасшедший ... я привык))

\iusr{Dembel Fox}
\textbf{Игорь Кольчик} , угу.. разум и экономика тоже без Вас...))

\iusr{Жанна Кондрашова}
\textbf{Игорь Кольчик} В России уже нет каналов принадлежащих олигархам. Возьмём их пример на вооружение? У нас какой будет кооператив ? « озеро» - это плагиат. Думаю кооператив « квартал95» ! Смеялась бы, если бы могла.

\iusr{Владимир Учень}
\textbf{Dembel Fox} А из Укрнафты сливают денежку? Вы это серьезно или так, бла-бла-бла?

\iusr{Володя Тернопиль}
\textbf{Игорь Кольчик} И за какую идею вы голосовали в 2019?

\iusr{Gregory Nudel}
\textbf{Игорь Кольчик} y этого дешёвого клоуна? Вы или неисправимый оптимист, или вам нужен психиатр

\iusr{Татьяна Пономаренко}
\textbf{Игорь Кольчик} , когда кто-то пишет с грубыми грамматическими ошибками создаётся впечатление, что общаешься с малограмотным, малообразованным, малочитающим человеком. Особенно смешно выглядит, когда Вы (то есть этот невежда) посылает других учить матчасть
\end{itemize} % }

\iusr{Эдуард Котов}

Свой "ум" и способности Ахметов показал в 2014 году в Донецке. С другой стороны
есть чувство, что если нынешняя украинская власть останется до конца срока
будет катастрофа. Противоборствующие стороны стоят друг друга. Может третья
будет?

\begin{itemize} % {
\iusr{Татьяна Пономаренко}
\textbf{Эдуард Котов} , а на кого «работали» медиаресурсы Ахметова с 2014 по 2019 год? Кто внёс весомый вклад в приходе к власти сегодняшней Команды? Я помню, у меня память хорошая!

\iusr{Игорь Марьяш}
\textbf{Татьяна Пономаренко} а на кого эти ресурсы работали с 2014го по 2019й?
\end{itemize} % }

\iusr{Володимир Козиненко}

Разумная экономика предполагает разделение занятий, т.е. разделение
государственной власти, бизнеса и финансов. Так или иначе, но Ахметов совмещает
пока эти занятия. Его деньги благодаря этому занятию. Не уверен что возможно с
ним договориться. Ошибка власти в том, что она действует против Ахметова, а не
против совмещения занятий и в этой связи не пользуется поддержкой народа.

\begin{itemize} % {
\iusr{Володимир Бердянський}
\textbf{Володимир Козиненко} невозможно заставить богатых людей не заниматься политикой, не превращая государство в коммунистическое

\iusr{Володимир Бердянський}
\textbf{Володимир Козиненко} зато можно отобрать у богаты и раздать.. себе. Что и происходило при Януковиче, и происходит вновь.

\iusr{Володимир Козиненко}

Я своими глазами видел как это делается. В Испании. Которая проводила реформы
всего на 10 лет раньше, чем начали в СССР. Заниматься политикой не запрещено.
Запрещено осуществлять деятельность с конфликтом интересов. Во всех формах.
Директора банков и члены их семей не могут заниматься бизнесом. Местный совет
не может принимать имущественные решения. Принимает мер. Но если он нарушит
закон, вертолеты прилетают из Мадрида и все это можно увидеть по телевизору.
Правда королевство Испания можно отнести к социалистическому. Действуют
социалистические нормы управления, распределения и потребления.


\iusr{Татьяна Пономаренко}
\textbf{Володимир Козиненко} , «реформы... всего на 10 лет раньше, чем начали в СССР». Запутали меня... А когда начали проводить реформы в СССР?
\end{itemize} % }

\iusr{Александра Романенко}

Рінату Леонідовичу, чи яктам його, ми зобов‘язані Віктором Федоровичем. Такі
про..би історія не вибачає. І тому шансів на олд мані в Україні він не має.

\begin{itemize} % {
\iusr{Наталия Филоненко}
\textbf{Александра Романенко} , а четырнадцатый год с гудком в уже оккупированном Донецке?

\iusr{Александра Романенко}
\textbf{Наталия Филоненко} Що означає з гудком?

\iusr{Наталия Филоненко}
\textbf{Александра Романенко} , была такая акция в « знак протеста» против « что- то пошло не так с гауляйтерами», под девизом « я за Украину» в означенное время гудели предприятия этого электорального заповедника. Ни уму , ни сердцу

\iusr{Олександр Костюченко}
\textbf{Nataliya Filonenko} гудок в Маріуполі був

\iusr{Наталия Филоненко}
\textbf{Олександр Костюченко} , можливо. Нюансів не пом‘ятаю. Донецьк вже був окупований.
\end{itemize} % }

\iusr{Оксана Левинсон}

Они не нувориши. Те рисковали, крутились, договаривались-добазаривались, и до
сладкой жизни дожили далеко не все. А это просто мелкое мудачье, которое не
прогрызало себе путь наверх - их вынесло наверх мутной волной. И так же смоет


\iusr{Андрій Скептик}
Ну а як їх не чіпати, щось можна в країні змінити?

\iusr{Людмила Левченко}

А как должен передвигаться президент?Без охраны, жить в многоквартирном
доме?У нас не Швеция, хотя и там умудрился убить премера, какой-то фанатик. А у
нас придурков хватает.

\begin{itemize} % {
\iusr{Ирина Ирина}
\textbf{Lyudmila Levchenko} например Порох, если на отдых летал, сам оплачивал частный самолёт, а не на вертолетах МЧС летал, и то ему те Мальдивы забыть не могут, а с этого придурка - как с гуся вода. По истине бог наказал Украину, отобрав мозги у большей части населения

\iusr{Dembel Fox}
\textbf{Людмила Левченко} ,
во-первых, президент не должен врать и не давать популистских предвыборных обещаний.
во-вторых, не возить челядь на ДР к завхозу за счет наших налогов

\iusr{Эдуард Котов}
\textbf{Людмила Левченко} Вы его обещания уже не помните?

\iusr{Malva Khortytska}
\textbf{Lyudmila Levchenko} нормально так полизькала

\iusr{Людмила Левченко}
\textbf{Ирина Ирина} О да, Порох-святой, тут и спорить не о чем @igg{fbicon.face.tears.of.joy} 

\iusr{Людмила Левченко}
Украинцы не могут жить без Олигархов!Кто на них с мечом, тот и от меча погибнет!

\iusr{Ирина Ирина}
\textbf{Lyudmila Levchenko} 

никто не говорил, что он святой. У него хватает недостатков, о которых
прекрасно знает его электорат. При этом его действия, направленные на
построение демократического европейского государства во многом перекрывают эти
недостатки. Ваш же бубочка открыто демонстрирует своё неуважение и к
государству, и к гражданам, вливает вам ежедневно в глаза урину и кладет на вас
то, чем играл на рояле. А вам хоть бы что, готовы принимать со всех сторон, как
он и говорил.


\iusr{Владимир Учень}
\textbf{Dembel Fox} Кого повез Президент к завхозу,на чем повез,за чей счёт? Укажите источник информации или Вы всего лишь разносчик мусора

\iusr{Dembel Fox}
\textbf{Владимир Учень} , похоже Вас Арестович учил учил, но так и не доучил анализировать информацию из открытых источников...

\iusr{Татьяна Пономаренко}
\textbf{Людмила Левченко} , а Вы не заметили, как с 2014 по 2019 год все олигархические каналы «мочили» Пороха? Почему же олигархи так старались очернить Пороха и привести к власти малограмотного КВН-щика?

\iusr{Татьяна Пономаренко}
\textbf{Владимир Учень} , передавали эту информацию во всех новостях. На вертолетах ДПС возили гостей на госдачу на ДР к Ермаку за деньки налогоплательщиков.

\iusr{Владимир Учень}
\textbf{Татьяна Пономаренко} "это" я читал. КТО сидел в вертолетах?КУДА они летали? "Журналист"предположил,а некая кучка сочувствующих подняла хай. А то,что это только предположение,никого не смутило?

\iusr{Владимир Учень}
\textbf{Dembel Fox} 

Кто такой Арестович и откуда он взялся знаю по-лучше Вас. А те "открытые"
источники, которыми Вы пользуетесь... Я Вам и не такое состряпаю. Это ведь
только предположения "журналиста", верно? А хай чего подняли?

\iusr{Dembel Fox}
\textbf{Владимир Учень}, знаете лучше ? При том , что мой хороший знакомый из одной компании с Кириллом ?
А что касается источников, то я смотрел не только УП, но и нелепую отмазку ОП.
Да, то что умеете фантазировать и уровень Вашей компетенции ( по крайней мере в энергетике) я уже увидел, жаль не успел размазать тонким слоем, Ян ответил.

\iusr{Владимир Учень}
\textbf{Dembel Fox} 

1. Не вижу пока ответа автора.
2. Об Арестовиче - из той же компании и даже раньше, когда он ещё только начинал свою "трудовую деятельность"

3. Я не крупный спец. в энергетике, а всего лишь довольно долго был подписан на
автора, хотя абсолютно с ним не знаком. Возможно, я ошибаюсь в деталях, но лично
для меня (повторяю, лично для меня), картина маслом. Тем более, что мое забрало
открыто.

\iusr{Dembel Fox}
\textbf{Владимир Учень} ,

1. Посмотрите в своем комменте. Ян ответил.

2. Лучше так лучше.

3. Зеленка- это только вершина айсберга и у Яна немного "специфический" ( я не
говорю, что он совсем не прав) угол зрения, поэтому по его постам не стоит
изучать состояние дел в энергетике.

\iusr{Владимир Учень}
\textbf{Dembel Fox} 

Да, я все увидел, принес автору свои извинения лично в его адрес, но не в
отношении олигарха. А информацию к размышлению об энергетике черпаю из
совершенно других источников


\end{itemize} % }

\iusr{Vadim Bas}
Всё так, но прогиб под Ахметова не защитан.

\iusr{Alexander Sipakov}
Зеле по-любому достанется и это гут!

\iusr{Евгений Языков}
Прекрасно сформулировано.

\iusr{Олег Бітюков}
Оконцовка реальная, как бы не прованговали как в "Ничейной земле"

\iusr{Леонид Власенко}
Подозреваю, что за РиЛе торчат уши сами знаете кого.

\iusr{Вадим Караваев}
Вспомнил БГ:
И чья в том вина, что арбатская пьянь
Пьет водку из чаш династии Тань?

\iusr{Tim Vlas}

Ненависть українців до мільярдерів, соціалістичні настрої взяли до уваги,
розрахунків? Який відсоток населення пам'ятає про 1кВт*ч за 2 і 4 копійки?

Ремейк соціалістичної революції буде фарсом, але можуть спробувати "по
приколу", якщо наважаться.


\iusr{Игорь Марьяш}
\textbf{Tim Vlas} Ринат Леонидович, пишите под своим именем

\iusr{Игорь Потысьев}

Находящиеся при власти действуют не против Ахметова, а в своих личных
интересах. Все их действия лишь предложение Ахметову учитывать эти интересы.
Действия Ахметова против власти - это, наоборот, предложение уменьшить аппетиты
"слуг народа".

P.S.

Ну и заодно попытка завести своего карманного президента. Это выйдет дешевле,
чем удовлетворять аппетиты чужого карманного


\iusr{Нина Черадионова}

Со всем согласна, кроме отсутствия при рождении "золотой ложечки во рту у
Рината Леонидовича"(с). Такого ангела-хранителя, как у него, ещё сыскать нужно.
Шутка ли, но куклы-марионетки-президенты меняются, а он, как стал теневым
президентом после ухода Леонида (!)Кучмы с официального поста, так и держится
бессменно до сих пор.

\iusr{Nikitta Marchenko}
Это по поводу статьи Ткача?

\iusr{Юрій Матюхін}
\textbf{Nikitta Marchenko} и это тоже

\iusr{Тім Желдак}

\obeycr
Вкотре жахнувся з рівня коментаторів.. але це наш виборець (((
Рідкісний випадок, коли згоден з ПивПавом
Шмаркля заслужив те, що має і в сотні разів гірше. Якщо його не можуть взяти на вила люди, яких він позбавляє майбутнього, нехай його змеле Рінат, якому він заліз в кишеню
\restorecr

\begin{itemize} % {
\iusr{Elena Baskina}
\textbf{Tim Zheldak} конечно лучше бы сразу денег взял или отжал бы пару заводив как предыдущие. Слепым надо быть, чтобы не видеть, что Ахметов и иже с ним никогда не встречали в Украине Президента, которого нельзя было бы купить и их это бесит

\iusr{Тім Желдак}
\textbf{Elena Baskina} зеленого наріка не можна купити???
Клоуна, який за 50 тисяч виплясував голим у вас на корпоративі???
Що у вас в голові ((((

\iusr{Игорь Потысьев}
\textbf{Тім Желдак} , иметь карманного президента выходит дешевле , чем удовлетворять аппетиты чужого карманного

\iusr{Татьяна Пономаренко}
\textbf{Elena Baskina} , нелогично. Если Ахметов покупал всех президентов, в том числе и пятого, то зачем ахметовские медиа все пять лет системно уничтожали репутацию ПП, и способствовали победе Зеленского?
\end{itemize} % }

\iusr{Лариса Прасолова}
Ой, цугцванг думаю для нас всех неотвратим

\iusr{Лариса Кузнецова}

Прям ода Ахметову получилась. Такой он душка по сравнению с теперешними.
Образец для подражания. А то, что он выжил и процветает за наш счет - дело
десятое, конечно.

\begin{itemize} % {
\iusr{Татьяна Пономаренко}
\textbf{Лариса Кузнецова} , \textbf{Лариса Шарикова}

\iusr{Лариса Кузнецова}
\textbf{Татьяна Пономаренко} Решили раскрыть свое настоящее имя?))
\end{itemize} % }

\iusr{Виталий Триль}

Чистый долг Метинвеста в 2021 году - 2,3 млрд долл , EBITDA материнской
компании Метинвест в Нидерландах за первый квартал 2021 -1462 млн долл .
Выручка компании 3,624 млрд долл за первый квартал 2021 года . При этом какая
сумма долгов компании срок исковой давности которой уже истек мне неизвестно .
Но : дворец Леопольда второго во Франции , виллы в Швейцарии и прочие пустяки в
виде офисов , самолетов и прочего .

\begin{itemize} % {
\iusr{Ян Валетов}
\textbf{Виталий Триль} обое рябое

\iusr{Виталий Триль}
\textbf{Ян Валетов} Ну понятно, Ахметов - наше всё, удобнее не замечать отчётность Метинвеста. И монопольное положение на рынке, что позволяет при таких доходах иметь такие долги.

\iusr{Гриша Савчук}
нам так разумеется не жить, а вообще как по мне большие деньги, большие проблемы, нафиг оно надо

\iusr{Семен Правденко}
\textbf{Виталий Триль} возможно, если б власть немного больше думала головой то и монополии такой бы во всех сферах не была б
\end{itemize} % }

\iusr{Malva Khortytska}
Все це дуже добре, якби не ахметівська людина, Ростислав Шурма, в замах у Єрмака

\iusr{Serhii Rybachok}

\href{https://youtu.be/jBkSHQB0srI}{%
Будем ждать, youtube, 24.07.2016%
}

\iusr{Elena Baskina}

Неужели Вы верите, что Ермак так праздновал свой день рождения, как разгоняют
сми? Он уже давно вырос из этих штанишек. Вот не верю я в это. Не та
обстановка, да и не стал бы он подкладывать свинью Президенту, не говоря уже о
том, что не нужно ему это с его теперешним темпом жизни, когда и семьи-то он
толком не видит.

\begin{itemize} % {
\iusr{Viktor Reshetnikov}
\textbf{Elena Baskina} , я заплакав...

\iusr{Elena Baskina}
\textbf{Viktor Reshetnikov} да хоть засмейтесь!

\iusr{Елена Сороконенко}
\textbf{Elena Baskina} нет, конечно он так не мог... Совсем не мог

\iusr{Olena Samus}

Президент вже давно зручно розмістився на тій клятій свині і частково завдяки
його Єрмаку. Що до темпу життя, то кварталівський темп був набагато інтенсивний
і продуктивний. Там майбутній \#потерпілий хоча б в живу виступав і ногами
дригав, зараз кофточки сушить і то виключно на відосиках. Який темп життя?


\iusr{Elena Baskina}
\textbf{Olena Samus} Вы мой комментарий ещё раз прочтите, я писала о Ермаке и его жизни, а Ваша злоба Вас настолько ослепляет, что Вы выплескиваете все зло на Президента, только дай Вам малейший повод. Вы сами себя не боитесь?

\iusr{Elena Baskina}
\textbf{Елена Сороконенко} только чёрная краски и негатив во всем, создатели фейков процветают на этой почве, а люди звереют...

\iusr{Елена Сороконенко}
\textbf{Elena Baskina} а какими красками ЗЕ лил грязь на всех остальных?

\iusr{Olena Samus}
\textbf{Elena Baskina} 

"Какаяразніца"? Ермак = Зеленський. З чого ви вирішили що я злюся? В цьому
випадку точно ні, в цьому випадку це просто смішно, як і "відрядження" в Оман,
інші відпустки, офшори і інша фігня, на критиці якій шостий став президентом.

А злюся я коли вже \#потерпілий приходить на брифінг і звинувачує в вбивстві
Шеремета таких людей як Ріф, Кузьменко і Дугарь. Коли \#потерпілий бреше і
викручується коли справа стосується безпеки країни. Коли його Єрмак несе повну
ахінею.

А з приводу дебільного і чергового святкування чогось там єрмакове, я не злюся,
я зловтішаюся над тими, хто за "оцевсе" голосували.


\iusr{Юрій Матюхін}
\textbf{Elena Baskina} когда думаешь не головой, а любишь сердцем, то приходишь к таким умозаключениям.

\iusr{Жанна Кондрашова}
\textbf{Elena Baskina} Как я вам завидую !! А вы где живёте , что такая благость на вас снисходит и наивность ? Ааа. Посмотрела. Во Франкфурте. Хм. Может к нам , сюда? Ну что- бы к реалиям поближе .

\iusr{Татьяна Пономаренко}
\textbf{Elena Baskina} , вижу, Вы очарованы игрой на рояле... о вкусах не спорят

\iusr{Elena Baskina}
\textbf{Zhanna Kondrashova} Это лучше, чем слепая злоба. Наивность-не порок, не завидуйте, «Хорошо там, где нас нет»

\iusr{Elena Baskina}
\textbf{Юрій Матюхін} Вы о Ермаке? Он мне в дети годится @igg{fbicon.grin} 

\iusr{Віталій Тронь}

Нагадаю, що напередодні локдауну Єрмак купив Мерс за 3 млн. грн. Всім нам в
морду, ага. 

\href{http://www.pravda.com.ua/news/2020/03/22/7244597/}{%
Єрмак напередодні карантину купив Mercedes за 3 мільйони, pravda.com.ua, 22.03.2020%
}

\iusr{Игорь Марьяш}
\textbf{Elena Baskina} Да и Татарова в ОП не он привёл, Шурму сейчас. Это ему враги подкинули

\iusr{Elena Baskina}
\textbf{Татьяна Пономаренко} да, я предпочитаю игру на рояле воровству и лицемерию

\iusr{Игорь Марьяш}
\textbf{Elena Baskina} воровство осталось

\iusr{Olena Samus}
\textbf{Elena Baskina} 

вивід грошей Приватбанку через свої офшори хіба не крадіжка? Прийти покладати
квіти загиблим на Майдані після того, як ти ржав над ними, хіба не ліцемірство?
Брехня, брехня без кінця і краю... без зупинки, без кордонів.

\iusr{Жанна Кондрашова}
\textbf{Elena Baskina} 

Вы не хотите верить , что Зеленский и Ермак праздновали ДР последнего в
резиденции , которая содержится на наши налоги. Зеленский , кстати , не отрицал
факта нахождения его там. И ещё пользовался вертолетом не по назначению . Это в
духе Брежнего . Гуляем за народные деньги. А главное - то , ВРЕТ . Вот врет ,
как дышит. Видишь ли он инспектировал готовность резиденции к приему каких- то
важных гостей . Мы что все идиоты ?? Президент воюющей страны с кучей проблем
летит на частном самолете Коломойского проверять чистые ли постели в номерах
??? Т.е. это для этого его украинцы выбирали? Мерзость , как она есть.

\end{itemize} % }

\iusr{Константин Рябченко}

Насчёт Ахметова не согласен, все олигархи черти. Впервые происходит то, что нам
обещали все папередники. И только у пацана хватило яиц поднять тарифы на ЖД
например или скинуть посипаку с главы Рады. Теперь или Ахметов станет нашим
Иванишвили на долгие годы или мы сдвинемся в сторону цивилизации

\begin{itemize} % {
\iusr{Елена Сороконенко}
\textbf{Константин Рябченко} "у пацана"... Металурги Ахметова часто в его возрасте уже пенсионеры (20лет стажа сталеварских к примеру)

\iusr{Константин Рябченко}
\textbf{Елена Сороконенко} ничего не понял?

\iusr{Елена Сороконенко}
\textbf{Константин Рябченко} ну вы ж кого-то в 43 годика "пацаном" считаете.

\iusr{Инна Тарнавская}
\textbf{Константин Рябченко} вот в этом ваша беда, что вообще мало что понимаете.

\iusr{Світлана Шипаєва}
\textbf{Константин Рябченко} 

только не надо про пацана. Нехай він скаже гоп, коли перескоче, ваш пацан. Ось
тоді і будемо оди йому співати, а поки що він соплі собі нехай підітре.

\end{itemize} % }

\iusr{Владимир Самойленко}
В Украине такие давно водились и о них знали. Скоробогатьками называли.

\iusr{Ольга Чала}
Ох:( чиста правда:((

\iusr{Borys Filatov}

Их проблема в том, что их не интересуют ничьи советы.

Даже умных и опытных людей, прошедших 90-е, банкротства, покушения, войны,
революции и прочие гримасы транзитивной демократии постсоветского периода.

Если что, то я не про Ахметова. А про тех, кто никому ничего не должен.

\begin{itemize} % {
\iusr{Ruslan Gorovyi}

бородаті діти... вони впевнені, шо до них в Україні нічого не було... тіко
динозаври... і то не скрізь

\iusr{Юрій Матюхін}
\textbf{Borys Filatov} ну а что же удивляться, тут как в той пословице який йшов, таких й зустрів. Єто тоже не про Ахметова )))

\iusr{Інгвар Тітус}
\textbf{Borys Filatov} 

Пам"ятаєте анегдот про три дипломи? Так от, про них мені розповідали ще на
початку 80-х, в муз. училищі... До де-кого ця майже притча не підходить. Мало
того, що дипломи там "липові", так і ..."матєріальчік так сєбє... Кєпочку мєрять
будєтє?"  @igg{fbicon.smile}  Це теж туди.... но з Одеси! :))))))))))))

\end{itemize} % }

\iusr{Анатолий Сагайдак}
Ода донецкой мафии  @igg{fbicon.beaming.face.smiling.eyes} 

\iusr{Ян Валетов}
\textbf{Анатолий Сагайдак} бред какой

\iusr{Анатолий Сагайдак}
\textbf{Ян Валетов} и не говорите. Я б такого не писал)

\iusr{Vitaly Nozhyn}
Сгорел сарай, гори и хата!

\iusr{Vadim Ryzhkov}

Вообще-то, нуворишами во Франции конца 18-начала 19 века называли людей,
которые обогатились благодаря революции. Они скупали конфискованные у дворян и
церкви земли и недвижимость, обогащались на военных поставках и т.д. В чем-то
сходство с Украиной есть))

\begin{itemize} % {
\iusr{Maksym Yeremenko}
\textbf{Vadim Ryzhkov} в использовании кризиса

\iusr{Татьяна Пономаренко}
\textbf{Vadim Ryzhkov} , а чем то, что Вы написали (именно про Францию) опровергает то, что написал автор? Нувориш - общеизвестное широко распространенное слово. До чого тут Ваше многозначительное «вообще-то»?

\iusr{Vadim Ryzhkov}
\textbf{Татьяна Пономаренко} Вас задело, что есть сходство с Украиной?
\end{itemize} % }

\iusr{Петро Закалата}

Вижу Ян что и вы пошли дорожкой проторенной Гордоном и Комаровском

\begin{itemize} % {
\iusr{Михаил Савченко}
\textbf{Петро Закалата} , а вам не кажется , что Зе и его команда , давно пошли по пути Януковича , и могут утащить за собой на дно всю страну ?

\iusr{Сергей Соломахин}
\textbf{Петро Закалата} На прошлой недели фейк от афериста Березы расписывал, сейчас гопнику оду сочинил....

\iusr{Петро Закалата}
\textbf{Михаил Савченко} 

коли вам "кажеться" - перехрестіться, дуже допомагає...а то , що Зе тягне "за
собой на дно всю страну", то с сліпому видно, як будуються дороги, злітні
смуги, аеропорти, мости, відновлюються і будуються школи, дитсадки, стадіони,
спортивні площадки...як Зе притиснув хвіст всім олігархам, через то й такий
проплачений вереск в ФБ.... як посмів замахнутися на білого й пухнастого
Ахметку??? "сивочолий гетьман" і той засц@в так ,що "продав" свої телеканали,
щоб не попасти в список олігархів...дави їх Президенте...підтримую..

\iusr{Михаил Савченко}
\textbf{Петро Закалата}

\ifcmt
  ig https://scontent-frt3-1.xx.fbcdn.net/v/t39.30808-6/260406473_4829348147115297_5981140970165206186_n.jpg?_nc_cat=107&ccb=1-5&_nc_sid=dbeb18&_nc_ohc=xhn1tdV4QHIAX85A0p6&_nc_ht=scontent-frt3-1.xx&oh=9f690d3998e1c67eb04e4baa1719fb2d&oe=61A83234
  @width 0.4
\fi

\iusr{Михаил Савченко}
\textbf{Петро Закалата} , вам тут мядальку прислали , из президентской ОПы . Носите спереди . Гордитесь . ДЛБЙБы .

\end{itemize} % }

\iusr{Ruslan Ultramarin}

"Здравствуй, племя младое, незнакомое!"
(частично завербованное врагом)  @igg{fbicon.face.smiling.eyes.smiling} 
То ли еще будет ) Скоро и "нувориши" из 90х-2000х выть начнут )))


\iusr{Tatyana Shestoyh}

Ахметов хочет и дальше грабить. Слово зарабатывать не подходит. Вопрос в
другом, нынешние хотят отодвинуть нуворишей чтоб встать на их место? Или
прекратить грабеж и создать цивилизованные условия?

\begin{itemize} % {
\iusr{Валентина Гаркуша}
\textbf{Tatyana Shestoyh} Зеленский и цивилизованные условия??? Не смешите. Одна история с оффшорами чего стоит. Продолжать можно долго. Такого беспредела, как сейчас, помноженного на вопиющую безграмотность , некомпетентность и наглость, Украина еще не знала. Однозначно, вариант номер один - именно отодвинуть.
\end{itemize} % }

\iusr{Сергей Перевозчиков}
Потерпите 2.5 года, и будет новый президент.

\begin{itemize} % {
\iusr{Анатолий Мельников}
\textbf{Сергей Перевозчиков} Новый? Если из пула приведенных рейтинговых то не приведи Господь.А других и не предлагают

\iusr{Игорь Потысьев}
\textbf{Сергей Перевозчиков} , лучше будет только ахметову
\end{itemize} % }

\iusr{Анатолий Елисеев}

То есть, Ян, вы поддерживаете и дальнейшее обогащение уже , по вашим меркам
"отмытого " Ахметова? К слову, обогатившегося и продолжающего свое обогащение
за счет народа? Пусть и дальше в стране процветает корупция? казнокрадство?
Продажные, брехливые СМИ? .. А во всем виноваты не Ахметовы, Фирташи и прочие
Порошенко? А Зеленский? Которому теперь и вы готовы подставить подножку..
Гордон, Комаровский, Голованов, Гавнопольский и прочие с радостью примут вас в
свои ряды... Печально..

\begin{itemize} % {
\iusr{Татьяна Пономаренко}
\textbf{Анатолий Елисеев} , как можно подставить ножку человеку, который ничего не умеет, кроме как играть членом на рояле и кривляться? Человек просто не успевал просыхать между корпоративами и выступлениями перед бандюками в банях... он вообще 42 не понимал, в какой стране он живет... а Вы его такими патриотическими мыслями наделяете... наивно... и грустно...
\end{itemize} % }

\iusr{Алексей Корецкий}

Цуцванг близко ... )))

А если серьёзно, грустно всё и гадко. Творят это они, а стыдно почему-то нам, у
кого ещё стыд остался.

А Ахметов, какой бы там не был, работает системно и имеет в этом изрядной опыт
не с одним уже президентом. И его топ менеджмент один из лучших в стране, во
всяком случае на уровень превосходит "команду ОП" вместе с карманным
правительством. Там такие горизонты только в мечтах, не умеют пока, а возможно
и не могут в принципе. Хотя ресурсы государства со всеми вытекающими за
президентом, только вытекает там со всех отверстий, даже для этого не
предназначенных.

Поэтому противостояние это по любому для государства ничем позитивным не
маячит. Но им в ОП похоже плевать давно, летят вертолёты МЧС, а щепки пускай
себе сидят ровно и не вякают (((

\begin{itemize} % {
\iusr{Юрій Матюхін}
\textbf{Алексей Корецкий} по поводу топ менеджеров Ахметова улыбнуло, один директор котельной - премьер в Зе команде, а второй , тот что в 27 лет Запорожсталью керував ( вумный такой наверно, раз в таком возрасте такое предприятие возглавил ), теперь зам Ермака по экономике. Самому не смешно ?

\iusr{Алексей Корецкий}
\textbf{Юрій Матюхін}
А где вы нашли смешное у меня?
Холдинг Ахметова приносит солидный и стабильный доход своему владельцу. Это и есть результат работы топ менеджеров. И не заметить это, нужно очень постараться. Разве реальная прибыль владельца многие годы это не показатель успешности его бизнес-команды?
Чего уж никак не скажешь о наших гос. институциях и государстве в целом.
\end{itemize} % }

\iusr{Pavlo Chaplygin}
На фразе "эскорты - нахуй" аж похолодело. Куда ж столько новых безработных девчуль девать?..

\iusr{Марина Кац}

Ото й сумно, що доводиться все це спостерігати зсереді, а той сраний попкорн
вже поперек горла. І найсумніше, що ті, хто голосував за Зеленського, навіть не
шкодують за свій вибір


\iusr{Max Chokan}
вспоминается анекдот из 90-х.
- чем же ты занимаешься, что столько бабла накосил?
- да вот, кроликов развожу.
- а мы, идиоты, лохов разводим.

\iusr{Люда Донев}

Супер! Отличный текст. Очень мудро и спокойно. И в самую тютельку. ППКС (и даже
под каждой буквой, почти). Дякую, ЯН, умница! Жаль только, услышат и поймут
немногие... Но, то таке. Делай, что должно и... не останавливайся на верном
пути. Обнимаю.  @igg{fbicon.index.pointing.up} ️@igg{fbicon.hand.victory} ️  @igg{fbicon.thumb.up.yellow} 

\iusr{Eugenio Tigrinian}

В Америке другая базовая культура. Она ИМХО куда как более эгалитарна.

Там тоже хватает распонтованных новых денег, но статусные понты куда как менее
характерны для протестантского населения.

\begin{itemize} % {
\iusr{Станислав С.}
\textbf{Eugenio Tigrinian} да ладно? На Трампа посмотрите. На Буша младшего.
У них свои понты просто.
Приём советской группы у себя в имении кстати это понт. Не потому что советские люди были в диковинку. Это не так. Туристов из СССР не было, а так народу из союза к 1990 году там было сотни тысяч, если не миллионы .

\iusr{Eugenio Tigrinian}
\textbf{Станислав С.}
Я бы не сказал, что приём на ранчо - это понты. Это southern hospitality ;))).
Трамп - таких мало.
П.С. Понравилось в Банффе и на озере Луизы?
Я там побывал в 98-м  @igg{fbicon.wink} .

\iusr{Станислав С.}
\textbf{Eugenio Tigrinian} да, понравилось, несмотря на все карантинные ограничения и межсезонье. Это далеко не первый мой визит в Альберту и Британскую Колумбию.
П.С. Думаете Яну показать фотографии как выглядит провинциальная автомагистраль построенная в 1960х? Фото виадуков над автомагистралью для перехода ЖИВОТНЫХ.

\iusr{Eugenio Tigrinian}
\textbf{Станислав С.}
Думаю, Ян их видел.
Да и у него в стране достоинство недавно победило, а наглоязычная цивилизация началась ИМХО с Великой Хартии 2015-го года.

\iusr{Станислав С.}
\textbf{Eugenio Tigrinian} я от него нахожусь менее чем в километре.

\iusr{Eugenio Tigrinian}
\textbf{Станислав С.}
От чего, от достоинства?
(на всякий случай упал в обмоРРРок)

\iusr{Станислав С.}
\textbf{Eugenio Tigrinian} от Яна. В смысле что страна та же, т даже локация  @igg{fbicon.smile} 

\iusr{Eugenio Tigrinian}
\textbf{Станислав С.}
Тогда и вы понимаете, что есть недавно победившее достоинство  @igg{fbicon.smile} .
Я по вашему профилю думал, что вы в Германии или Штатах ...

\iusr{Станислав С.}
\textbf{Eugenio Tigrinian} я там на Западе сумарно прожил почти 30 лет.
Про достоинство что победило я вообще ничего не понял.
Зачем Яныка свергали мне не понятно. При нем цены были в несколько раз ниже .

\iusr{Eugenio Tigrinian}
\textbf{Станислав С.}
А где, если не секрет?
И что привело вас обратно на Родину?
Про достоинство никто с рациональным умом не понимает.
Но Ян постоянно им машет, отбивяась от возможных атак Путина.

\iusr{Станислав С.}
\textbf{Eugenio Tigrinian} в общих чертах состояние здоровья тёщи, ныне уже покойной.
Пока вся работа на удаленке, мне без разницы откуда работать.

\iusr{Ян Валетов}
\textbf{Eugenio Tigrinian} Ян машет не достоинством. Яну не нравится, когда его страну оккупируют.

\iusr{Станислав С.}
\textbf{Ян Валетов} Вы о кораблях НАТО в украинских портах? Или об иностранцах в набсоветах? Или об иностранных гражданах в правительстве? Или о внешнем управлении?

\iusr{Станислав С.}
\textbf{Ян Валетов} вопрос на засыпку. Что делает переворот 2014 года революцией?

\iusr{Eugenio Tigrinian}
\textbf{Станислав С.}
Если не секрет, где больше всего нравится?
(штат, город)

\iusr{Eugenio Tigrinian}
\textbf{Ян Валетов}
Яну не нравится - я это понимаю.
Но он очччень много тут про достоинство излагает, размахивая оным ...

\iusr{Станислав С.}
\textbf{Eugenio Tigrinian} понравилось в плане жить на пенсии? Sunny Isles Beach, Florida. В плане не на пенсии -Мюнхен, Оттава.

\iusr{Eugenio Tigrinian}
\textbf{Станислав С.}
В Мюнхене бывал несколько раз, в Оттаве и Флориде - нет.
Больше Left Coast and NY/Boston.
Всё по северам, не пускает меня в себя тропический рай  @igg{fbicon.wink} .
А так - в Германии жить не смог бы, а Америка - это да  @igg{fbicon.smile} .

\iusr{Станислав С.}
\textbf{Eugenio Tigrinian} 

у меня в планах осесть на пенсии во Флориде. Но там же осядет большая часть
моего круга общения к 2040 году.

Мои дети к жизни в США относятся достаточно негативно.
Санни Айлз это скорее не тропический рай, это Флоридская Ривьера.
Тропический рай в том регионе это Флоридские Киз.

Оттава и Мюнхен приятны в плане работы, удобства жизни, социальной
защищенности, культурной жизни, географического положения, более чем терпимого
отношения к акценту в речи. Стандарты жизни высокие, много выше чем в США.
Качество еды, питьевой воды. Личная безопасность, культура в бытовом плане,
поведение водителей на дорогах.

\end{itemize} % }

\iusr{Анатолий Елисеев}

Получается, Ян, что эти владельцы рояльной фабрики, которых вы привели в
умилительный пример-на самом деле - потомки чикагских ганстеров или
безжалостных ковбоев , грабивших почтовые поезда и банки . Тогда понятно ваше
сравнение с таким , уже без крови на руках, уже платящим все до копеечки в
бюджет Украины, а не в оффшоры Ахметовым:) Тада да:(.


\iusr{Анатолий Елисеев}

Когда все узнали, что Ахметов платит Шустеру 300 000 не гривен, и не в год:)
сразу зачесались ручки у Гордона.. теперь вот.... @igg{fbicon.frown}  или не так, но все-равно
печалька:(

\begin{itemize} % {
\iusr{Анатолий Мельников}
\textbf{Анатолий Елисеев} И у доктора Комаровского.

\iusr{Петро Закалата}
в Кивы ещё раньше зачесалось... ясновидящий однака..

\iusr{Анатолий Елисеев}
\textbf{Петро Закалата} Кива- известный почесун руками:)
\end{itemize} % }

\iusr{Дмитрий Прияткин}

Ощущение такое, что это не люди вообще, а подменыши из Нижнего мира. В
ортопедических корсетах, чтобы по наружности не опознали...

\iusr{Ольга Столбченко}

Все правильно, только что нового или неожиданного для Вас в поведении этих
людей, которые буквально вломились во власть в 19 году? Они призваны были
уничтожить остатки нашей государственности, и успешно справляются.

\begin{itemize} % {

% -------------------------------------
\ii{fbauth.klimovskij_aleksej.krivoj_rog.ukraina}
% -------------------------------------

\textbf{Ольга Столбченко} 

в 13-м году открыли ящик.
То, что выскочило оттуда - захватило власть.
Вместо безидейных воришек пришли невменяшки, преступники и воришки с лозунгами.
Дальше выбор будет только из этих категорий

\iusr{Oleksii Slipchenko}
\textbf{Alexey Klimovsky} нечто зсхватило власть, не в 2013м, а в 2010м.

\iusr{Alexey Klimovsky}
\textbf{Oleksii Slipchenko} 

нечто захватило власть в 2004. С этого момента, когда президент типа незалэжной
державы звонит в посольство США и просит посла попросить мирных протестующих не
сильно буянить - говорить о "власти" в этой стране уже просто смешно.

Дальше был выбор - ущербные националисты или безидейные воришки.

В 10-м пришли безидейные воришки

В 13-м опять безумные, просто маргиналы с прапорами и воры с прапорами.

И собственно, кого вы сейчас ждёте?

\end{itemize} % }

\iusr{Александр Тарноруцкий}
Ян, у американских нуворишей тоже самое было в 30х годах прошлого века.

\iusr{Borys Natanczuk}

Так Ахметов же агент влияния Кремля в Украине, это же главное.

Деньги у не старые, но и бизнесы его морально устарели.

\iusr{Гузей Валерій}

Первая половина поста была такая, что я даже слезу пустил от умиления. И тут
вдруг - когнитивный диссонанс. Рекламный ролик дона Корлеоне  @igg{fbicon.face.confused} 

\begin{itemize} % {
\iusr{Eugenio Tigrinian}
\textbf{Гузей Валерій}
Ну хоть не Дона Кремлеоне ;)))
\end{itemize} % }

\iusr{Дмитрий Голышев}
Начинал красиво, \enquote{за здравие}, все остальное - прогиб

\iusr{Владимир В. Кулаков}
Хороший пост.
Стырю)

% -------------------------------------
\ii{fbauth.kofman_aleksandr.doneck.dnr.predsedatel.obschestvennaja_palata_dnr}
% -------------------------------------

Лови схему. Добыть на своих шахтах уголь. Продать его своим ТЭС, а энергию
через свои энергопоставляющие компании продать населению и предприятиям. Затем,
получить в государстве дотации на содержание своих же шахт. Вот тебе
стандартная ахметовская цепочка, благодаря которой дебилы вроде тебя и
окружающих кричали о дотационным Донбассе.

По сравнению с этим товарищем, Коломойский - образец ума, чистоты и невинности.

А у тебя судьба, Ян, обслуживать интересы капитала? Сперва Беня, теперь Ринат?

Смотри спину не сломай, для того, чтобы так изгибаться, нужно иметь весьма
гибкий позвоночник.

\begin{itemize} % {
\iusr{Ян Валетов}
\textbf{Александр Кофман} 

я об этих схемах знаю поболе тебя, потому что внутри нахожусь. В энергетике, с
перерывом правда, почти с 1996 года. Про вертикально-интегрированные корпорации
не тебе мне рассказывать, молодой ищщо.

И знаю я тему под другими углами. Дотации на уголь - это не изобретение Рината.
Иначе отрасль не работала.

И на его шахтах народ касками о землю не стучал. Зарплата как по часам,
оборудование новое, безопасность, ремонты.

Ну, вы теперь покажете как надо! Да, Саша? Как там с угледобычей в ДНР?

А насчет обслуживания интересов Рината... вот не смеши.

Ни он во мне не нуждается, ни я в нем не нуждаюсь.

\iusr{Сергей Соломахин}
\textbf{Ян Валетов} Вы не в курсе почему Запорожсталь все время убыточна?

\iusr{Ян Валетов}
\textbf{Сергей Соломахин} 

потому что в вертикально интегрированной структуре - а Запорожсталь ее часть -
можно регулировать прибыль и расходы в любой части схемы. Так как выгодно по
налогообложению.

\iusr{Сергей Соломахин}
\textbf{Ян Валетов} 

Правильно, но подло же, Ахметов когда приватизировал сырьевые предприятия (и
энергетические) обещал, что эффективный собственник будет наполнять бюджет
страны, а по факту занимался возвратом фиктивного НДС и выводом всей прибыли в
оффшеры. Кстати, оказывается молодой Шурма, будучи директором Запорожстали
вывел ее в прибыльное предприятие, за что тут же был уволен и Запорсталь снова
стала убыточной. Этот вор контролирует треть ВВП Украины, благодаря его схемам
в Украине один из самых высоких уровней налогообложения граждан. Мало того,
выжить в 90е с таким счастьем можно было только сотрудничая с ФСБ РФ.

\iusr{Виталий Белый}
\textbf{Сергей Соломахин} 

Дичь про Шурму. Его уволили за провал выборов и за то что обкрадывал Ахметова.
Насчёт прибыли они шли от того что он кучу цехов не основного производства
литейный механический и тд вывел в отдельный завод и списывал на него убытки.

\iusr{Сергей Соломахин}
\textbf{Виталий Белый} 

Так я же и написал, что сделатт прибыльное предприятие это значит обокрасть
Ахметова. Вы наверное думаете, что Ахметов озвучит эту причину увольнения? Вы
верите Ахметову, или ничего личного только деньги?

\iusr{Виталий Белый}
\textbf{Сергей Соломахин} прибыль предприятия уходила Ахметову. А обкрадывал Шурма рабочих предприятия. Платя копеечные зарплаты и не проводя текущие и капитальные ремонты оборудования.

\iusr{Сергей Соломахин}
\textbf{Виталий Белый} С прибыли приходилось платить налоги, а с убытков их не платят. Филипп Морис Украина платит налогов больше чем СКМ и ДТЭК вместе взятые.

% -------------------------------------
\ii{fbauth.savchenko_mihail.zaporozhie.ukraina}
% -------------------------------------

\textbf{Александр Кофман}, 

давно не видел клоунов. Увидел вас, зашёл на страницу .... Ну что сказать,
слабоумие и отвага )) Вы так смешны, в своих потугах казаться \enquote{рЭспубликой},
хотя вас даже Путин кинул, и к себе не взял ))

\iusr{Виталий Белый}
\textbf{Сергей Соломахин} Азовсталь и Ильича тоже эти все года показывали прибыль. И никого не увольняли за это. Шурма директором Зстали был 7 лет. И только недавно получается Ахметов прозрел)))

\iusr{Виталий Белый}
\textbf{Сергей Соломахин} 

Азовсталь и Ильича тоже эти все года показывали прибыль. И никого не увольняли
за это. Шурма директором Зстали был 7 лет. И только недавно получается Ахметов
прозрел)))

\ifcmt
  ig https://scontent-frx5-1.xx.fbcdn.net/v/t39.30808-6/259964592_6522475777827387_3476525207055637688_n.jpg?_nc_cat=111&ccb=1-5&_nc_sid=dbeb18&_nc_ohc=Cj2Xeaoh_1AAX93Yz87&_nc_ht=scontent-frx5-1.xx&oh=1cda0fa023c23b9f9ebf545c1f18aa81&oe=61A90325
  @width 0.4
\fi

\iusr{Dmytro Kochetkov}
\textbf{Михаил Савченко} есть пропагандоны на зарплате, а этот идейный, забесплатно подмахивает.

\iusr{Михаил Савченко}
\textbf{Dmytro Kochetkov} , этот замаскированный . Хитро*опый .

\iusr{Lagunovskii Eduard}
\textbf{Александр Кофман} 

вы честно считаете что ваши шахты рентабельны? Почему же тогда такие же шахты в
ростовской области закрылись? Просто для России Ростовская область это как
мелочь незаметная, не то что для Украины Донецк и Луганск. А Украина носилась с
Донбассом все это время, боялись социального взрыва. А он все равно произошёл.
Надо было закрыть все на хрен в 2005 году, а вместо этого юля думала что
приедет в Донецк в шарфике шахтёра и все её полюбят. Выбила закон о "шахтерском
труде", а те все равно за яныка проголосовали.


\iusr{Алекс Шорн}
\textbf{Lagunovskii Eduard} А уголь покупать в США как сейчас и гадать будут веерные отключения или пронесет. Гениально!

\iusr{Александр Кофман}
\textbf{Lagunovskii Eduard} с нынешней стоимостью угля, более чем рентабельны.

\iusr{Lagunovskii Eduard}
\textbf{Александр Кофман} очень в этом сомневаюсь. Столько людей там гибло. В нормальном мире человеческая жизнь ценится гораздо выше этого угля

\iusr{Олександр Костюченко}
\textbf{Виталий Белый} коли це вони показували прибуток?

\iusr{Виталий Белый}
\textbf{Олександр Костюченко} 

Акционеры металлургического комбината "Азовсталь", входящего в группу
"Метинвест", намерены направить на выплату дивидендов всю чистую прибыль по
итогам 2016 года и часть прибыли за 2014 год на выплату дивидендов.

Об этом сообщает "Интерфакс-Украина".

Согласно проекту решения собрания акционеров предлагается направить на выплату
дивидендов чистую прибыль за 2016 год в размере 558,417 млн грн, а также часть
чистой прибыли за 2014 год в размере 915 млн грн.

В целом на выплату дивидендов планируется направить 1,473 млрд грн.

\iusr{Володимир Скотников}
\textbf{Александр Кофман} 

Саша. Зато у вас Дыре наконец то такая жизнь какую заслужили. Это вам не при
Украине двери колбасой завязывать  @igg{fbicon.smile}  и с каждым годом все краше. Даже
российские нацисты после воевания у вас полностью исправляться - после того как
они увидели у вас \enquote{русскую Россию} они понимают что лучше уж Путин и те кто ими
правит  @igg{fbicon.smile} 

\end{itemize} % }

\iusr{Анатолий Елисеев}
\textbf{Андрей Катышев}

Нострадамус в 354 катрене писал о своем приемнике :

" V konce 20- nachale 21 stoletiy v Ukraine poyavitsya proplacheniy pizdobol
-Dmitriy Gordon, kotoriy svoimi prorochestvami zatrahaet vseh. Ne obrashayte
vnimaniya na etogo aferista i u vas vse bydet horosho....Tak je poshlite na her
i sleduyushih ushlyopkov : Shuster, Komarovskiy, Avakov, Yacenyc, Groysman s ih
sponsorami Poroshenko i Ahmetovim. ....Amin",

что в переводе с древнефламандского означает "Самым талантливым своим учеником
я считаю Диму Гордона и его команду. Только с ними в Украину придет
счастье.... Слава Украине."

\iusr{Владимир Учень}

Из этого поста торчат денежные уши больших денег, вложенных автором в \enquote{зелёную}
энергетику, и которые ему по какой-то (можно догадаться по какой!!!) причине
Кабмин пока что заморозил к выплате. Как ни странно, но заморозили выплату
только компании ДТЭК. Значит есть некая связь между компанией и автором. А вот
почему именно ДТЭК заморозили выплату, так это ответ властей на шантаж, если
попроще, то, как в старой советской поговорке: \enquote{на хитрую ж... есть х...} Ноги
этой истории растут из глубины и когда-то этот геморрой все равно надо
ликвидировать. Так вот \enquote{многоуважаемый нувориш} Ренат Леонидович, привыкший, что
ему, как обычно, пойдут навстречу в его \enquote{хотелках} ( иначе \enquote{положу} нахрен всю
энергетику страны) решил \enquote{кинуть} государство на 12 ярдов грн. По всей
видимости, это и стало крайней каплей в долгом и тяжёлом поиске компромиссов с
\enquote{нуворишем}. Уверен, что автор об этом знает, раз состоит в совместном бизнесе с
ДТЭК, но... надо возвращать бабки, а тут уже не до симпатий. Хотя это похоже на
лопнувшие банки, нахватавших депозитов от доверчивых граждан под офигительные
условия, а граждане не удосужились вникнуть и покопаться в подноготной этих
банков. Так что нужно было, по-видимому, хорошенько подумать, прежде чем
связываться с ДТЭК.

\begin{itemize} % {
\iusr{Ян Валетов}
\textbf{Владимир Учень} мне как раз все выплатили)
Можете фантазировать дальше)

\iusr{Владимир Учень}
\textbf{Ян Валетов} 

Если это действительно так (а не для поддержки собственного имиджа), я приношу
свои извинения-моя логическая цепочка где-то оказалась ошибочна, но только в
отношении Вас, но не в отношении к Ахметову.

\iusr{Ян Валетов}
\textbf{Владимир Учень} вы видели в своем тексте логическую цепочку? Серьезно?

\iusr{Владимир Учень}
\textbf{Ян Валетов} В тексте были уже выводы, а логическая цепочка выстраивалась раньше, чуть больше двух лет назад.

\end{itemize} % }

\iusr{Kirill Aronov}

Совсем так себе.
Ахметова нужно раскулачивать любыми методами а не договариваться. Закрыв глаза на то, что будет пакостить.
Сингапурский вариант, договаривались уже на Роттердам+.
Это по сути.
А привязать растроповича к ахметову - это так себе по форме.
Итого - ну никак(

\iusr{Dmitriy Mikhailov}

В квартире холодно( коммунальщикам по барабану), на работе холодно, на улице
холодно , ненавязщево подводят брать плакат с надписью ктото прийди.


\iusr{Alex Polakov}

Понятие \enquote{старые деньги} означает \enquote{надежно инвестированные, безинфляционные}.
\enquote{Старые деньги} редко приносят большие доходы, но приносят их в рассчетном
объеме в рассчетное время. Пока в Украине нет стабильной экономики, \enquote{старых
денег} в ней нет. Даже у Ахметова.

\begin{itemize} % {
\iusr{Станислав С.}
\textbf{Alex Polakov} скорее даже самого понятия старые деньги больше не существует. Просмотрите Форбс 400, там старыми деньгами даже не пахнет.

\iusr{Alex Polakov}
\textbf{Станислав С.} в этом и специфика старых денег, что миллиардеры калифорнии спят и видят, как бы обменять свои рейтинги в форбсе на старые деньги новой англии, и без вопросов, даже если в отношении 3 к 1. это не так уж просто.

\iusr{Станислав С.}
\textbf{Alex Polakov} не могу ни подтвердить ни опровергнуть. Я работал в США в долине и какое то время был женат на британской виконтессе и жил в английском Йоркшире и крутился в среде английских аристократов. Это вообще нигде не пересекаемые сообщества
\end{itemize} % }

\iusr{Геннадій Шашкевич}
Цунцванг...

\iusr{Григорій Акінфієв}
\textbf{Hennadii Shashkevych} Є, як зараз говорять, один "ньюанс"  @igg{fbicon.laugh.rolling.floor} 

\iusr{Гузей Валерій}
\textbf{Григорій Акінфієв} "ню анус" )))

\iusr{Лариса Рыбалка}

Частично согласна. Но.... Расчет на то, что многие забыли как делались его
деньги? Или слезы умиления от ролика по ТВ \enquote{дети Донбасса ищут новых
родителей}? А кто всё заварил?

\iusr{Вячеслав Ветлянский}

Ян скажи, как ты думаешь, сколько сегодня знают, кто такой Растропович?

\begin{itemize} % {
\iusr{Roman Khizhnyak}
\textbf{Вячеслав Ветлянский} кто это?

\iusr{Вячеслав Ветлянский}
\textbf{Roman Khizhnyak} я не про тебя. ты знаешь.

\iusr{Цезарь Томасик}
\textbf{Вячеслав Ветлянский} а заодно и МЭЙФЛАУЭР

\iusr{Roman Khizhnyak}
\textbf{Вячеслав Ветлянский} ))))

\iusr{Andrei Plekhanov}

В cвое время Ростропович был солистом Московской филармонии, а посему, как и
все прочие, был включен в бригаду по обслуживанию целинных и залежных земель.
Приезжают они на полевой стан — народ сидит на земле, фортепьяно нету.
Ростропович разволновался: «Как же я буду без аккомпанемента играть?» А
композитор Ян Френкель его успокоил: «Не волнуйся, Славочка, я хороший
аккордеонист, я тебе саккомпанирую — никто и не заметит!» Вот Ростропович
играет, Френкель на аккордеоне подыгрывает, как может... Вдруг где-то в конце
«зала» встает здоровенный целинник в робе и, перешагивая через сидящих,
движется к «сцене». Ростропович шепчет Френкелю: «Янек, что-то мне лицо его не
нравится, черт его знает, что у него на уме... Давай, играй побыстрее!» Однако
закончить не успели. Мужик дошел до концертантов, положил на струны виолончели
свою огромную ручищу и внушительным басом сказал Ростроновичу: «Браток, не
гунди — дай баян послушать!

\iusr{Катерина Доля}
підняли настрій:)

\iusr{Ярослав Гаврилишин}
\textbf{Вячеслав Ветлянский} Я даже в курсе кто такая Галина Вишневская ! @igg{fbicon.wink} 

\iusr{Alexey Martynov}
\textbf{Вячеслав Ветлянский} сколько вменяемых - столько и знают. Полагаю, процентов 15

\iusr{Вячеслав Ветлянский}
\textbf{Alexey Martynov} думаю 15, много. для этого нужна нормальная семья, школа-учителя, да и вуз нормальный. мне кажется, что их не больше 7-8 процетов.
\end{itemize} % }

\iusr{Станислав С.}

Факт Чекинг. В Сиэттле и во всем штате Вашингтон никогда не было фабрики по
изготовлению роялей.

\begin{itemize} % {
\iusr{Ян Валетов}
\textbf{Станислав С.} Я ненароком сказал что его фабрика была в Сиэттле? Или вы решили в очередной раз блеснуть умом?)

\iusr{Станислав С.}
\textbf{Ян Валетов}

\ifcmt
  ig https://scontent-frx5-1.xx.fbcdn.net/v/t39.30808-6/260812107_624447775259293_4958173175752526550_n.jpg?_nc_cat=111&ccb=1-5&_nc_sid=dbeb18&_nc_ohc=1GvfG1c9P7MAX-YksEg&_nc_oc=AQldDkMeT0XamHeJMsOISsBUIPylSO23gd8IEwCEL1fmVOeaI9RR3ZY2AV2NyeDiZTo&_nc_ht=scontent-frx5-1.xx&oh=5d82b438d5070a9a4269f753b4ca7e6f&oe=61A7A33B
  @width 0.4
\fi

\iusr{Ян Валетов}
\textbf{Станислав С.} 120 лет прошло.

\iusr{Станислав С.}
\textbf{Ян Валетов}

\ifcmt
  ig https://scontent-frx5-1.xx.fbcdn.net/v/t39.30808-6/259652654_624450865258984_2632879517369734662_n.jpg?_nc_cat=111&ccb=1-5&_nc_sid=dbeb18&_nc_ohc=-63SuVMtCXEAX8469cg&_nc_ht=scontent-frx5-1.xx&oh=8db3892dd50d8b73dcaba937e9a0806a&oe=61A8232A
  @width 0.4
\fi

\iusr{Станислав С.}
\textbf{Ян Валетов} 

при этом Билл Гейтс и Уоррен Баффет летают корпоративным джетом, на который и я
и вы можете купить перелёты. Около 5 тыс баксов за час перелёта. NetJet, если
нас кто-то из владельцев абонемента возьмёт .

\end{itemize} % }

\iusr{Борис Нечипоренко}

Прочитал кментарии, ничего не поняли, весь смысл статьи большинство сузили до
Ахметова! И чего удивляться действиям власти? Она как раз под таких! Жаль
беспросветно!


\iusr{Наталія Діхтяр}

Це ще треба мати не абиякий талант, щоб розсобачитись з усіма, хто тобі допоміг
залізти на Олімп... Величезні комплекси не дозволяють бачити очевидне.


\iusr{Денис Дуновский}

Как же Вы здорово пишете, Ян! Даже о таком мраке беспросветном.

\iusr{Станислав С.}
Ян Михайлович. Вы книги Э. Севелы про американцев читали?

\begin{itemize} % {
\iusr{Ян Валетов}
\textbf{Станислав С.} Да

\iusr{Станислав С.}
\textbf{Ян Валетов} мне очень напоминает меценатов из его книг.

\iusr{Ян Валетов}
У меня это живая история.
Футбольная команда в гостях)

\iusr{Станислав С.}
\textbf{Ян Валетов} значит его понты удались, если через много лет оно ещё сидит в памяти.
\end{itemize} % }

\iusr{Galina Malkina}

Основная доля капитала договороспобного выросла на немалых многолетних
многомиллиардных бюджетных деньгах. И при этом он почему-то не защитил свою
вотчину Донбасс. Возникает вопрос: с кем он договорился или, наоборот, не
договорился по Донбассу?

\begin{itemize} % {
\iusr{Станислав С.}
\textbf{Galina Malkina} один уже свою вотчину защитил. У него за это отжали Приватбанк  @igg{fbicon.wink} 
\end{itemize} % }

\iusr{Станислав Стасов}

Про НИХ уже много разного написано и переписано, от Л.Д. и до сейчас. А можно
все тоже только на пару ступенек ниже - о т.н. нуворишах местного, Днепровского
масштаба. Или ту все \enquote{Супер гуд}.


\iusr{Iryna Zheldak}

Цікаво, як швидко путін стане просто \enquote{не ангел, родился без золотой ложки во
рту и имеет некоторые особенности в биографии}, аби не \enquote{шмаркля}?

\iusr{Лилия Чуйкова}

\enquote{Крапивница} - как вы нежно это сказали. Аллергия на грани анафилактического
шока. Идиосинкразия. Рвотный рефлекс. Сказала бы \enquote{ненависть}, но это слишком
глубокое чувство для этих гопников.


\iusr{Евгения Игнатенко}
Благодарю.

\iusr{Людмила Пархоменко}
Ай, да Ян Валетов! Ай да Пушкин!
Это Ринат Л. старый нувориш?!
Я думала старый бандит. Рэкэтир по вашему интеллигентному.
Стыбрил немало у нас. И продолжает. В Донбассе обос₽ался, пришлось новый дворец под Киевом срочно воздвигнуть. На рояле перед ним играть надо? Ну так сыграй. Ты ж Мурку знаешь, и знаешь, каким местом надо играть.

\iusr{Viner Gazizullin}
Хаха, нувориш, за которым навсегда - шлейф преступлений из 20-летнего прошлого и поэтому - объявленный невъездным в США.

\iusr{Дмитрий Фоменко}
Конфликт должен привести к переменам. В какую сторону -вопрос, да и обе стороны хреновые. Но, возможно, таков путь..

\iusr{Виктория Худенко}

Как это вы, Ян Валетов, с придыханием \enquote{Ринат Леонидович...} Все украинские
бизнесмены такого типа, а уж тем более олигархи, нувориши. Только одни чуть
более помытые и причёсанные.

\iusr{Ireena Novikova}

Почтенный олигарх Ахметов за три месяца потратил более 2,8 миллиарда гривен на
недвижимость возле Женевы и так поиздержался, что уже устроил веерное
отключение электричества, в том числе и ВАК, где дело Роттердама
рассматривается. Грустно, что в НЕпочтенной компании оказываетесь, Валетов.

\ifcmt
  ig https://scontent-frx5-1.xx.fbcdn.net/v/t39.30808-6/260050814_3073736112948305_3909472267478280805_n.jpg?_nc_cat=100&ccb=1-5&_nc_sid=dbeb18&_nc_ohc=jREWIwIkIEsAX9FV-Ga&_nc_ht=scontent-frx5-1.xx&oh=04315665901b10a0d1441cbfc03b1fea&oe=61A7E802
  @width 0.4
\fi

\begin{itemize} % {
\iusr{Ян Валетов}
\textbf{Ireena Novikova} да я не в компании, я сам по себе. Мне так веселее и комфортнее.

\iusr{Виктория Худенко}
Ох и компашка!

\iusr{Viner Gazizullin}
\textbf{Ян Валетов} Стыдно продавать совесть
\end{itemize} % }

\iusr{Игорь Мастерной}
Не палевно защищаем Рината. Деньги не пахнут?

\begin{itemize} % {
\iusr{Ян Валетов}
\textbf{Игорь Мастерной} 

знаете, мне больше всего интересно, ну где? Где в этом тексте защита Рината? С
какого перепугу ее тут можно вычитать? И каким надо быть идиотом, чтобы не
понять, о чем написан пост?

Уровень комментаторов просто пугает своей компетентностью и умом.

\iusr{Игорь Мастерной}
\textbf{Ян Валетов}, ок напишите для нас идиотов  @igg{fbicon.wink}  посыл поста
Для себя я понял след. тезисы:
\enquote{Ругаться с Ринатом - не надо: Украине будет плохо и враг у ворот же},
\enquote{Ринат не такой страшный и ужасный, как кажется, а просто душка, который ездит без эскортов и эскортниц}

\iusr{Игорь Мастерной}
Да и вообще, человек заработал деньги, почти что рояли изготовлял на мебельной фабрике.

\iusr{Galina Malkina}
\textbf{Ян Валетов} Вы, как Влащенко Наталья, поставили себя над...Вас читают разные люди, но они не глупцы, все понимают и анализируют, рано или поздно вынесут свой вердикт

\iusr{Ян Валетов}
\textbf{Galina Malkina} (пожимая плечами) и что?

\iusr{Galina Malkina}
\textbf{Ян Валетов} (пожимая плечами) Действительно, и что?

\iusr{Ян Валетов}
\textbf{Galina Malkina} и ничего.
У меня есть своя точка зрения. Мне нужно подстроить ее под вашу?

\iusr{Galina Malkina}
\textbf{Ян Валетов} Под меня не нужно. Терпеть не могу, когда над...Приходится осаживать.

\iusr{Ян Валетов}
\textbf{Galina Malkina} у меня нет своей стороны в текущей ситуации.
Да, я над. Мне одинаково неприятны все участники тусовки. Глубоко антипатичны.
Что со мной не так?

\iusr{Galina Malkina}
\textbf{Ян Валетов} Вы также мне глубоко антипатичны...Заметьте, не я первая так сказала

\iusr{Ян Валетов}
\textbf{Galina Malkina} я вообще-то говорил не о вас, а о политической тусовке. Ну, ок. Хорошо, что мы все сразу и выяснили.)
\end{itemize} % }

\iusr{Петро Закалата}

Когда вы начинаете дышать ядом на Зеленского, это больше говорит о масштабах
вашей личности, а не об ущербности его.

Мальчик из Кривого Рога, из простой семьи, получивший очень среднее образование
- не в Лондоне и даже не в Киеве, сумел превратить команду КВН во
многомиллионный бизнес, научился продюсировать кино, сотрудничал с каналами
всех олигархов, договорившись о сотрудничестве на выгодных для себя условиях,
создал медиаимперию, и в результате, захотев власти, выиграл выборы в 40-ка
миллионной стране. Стал политиком номер один, как только захотел.

Давайте, дорогие мои личности и яркие индивидуальности, померяйтесь в ним
каким-нибудь органом, почешите свое ЧСВ.

Будет весело. Посмотрим, кто из нас лох, а кто молодец.

Он может нравиться или не нравиться, вы можете его обожествлять или ненавидеть.
Но он яркая личность, харизмат, чертовски удачливый и работоспособный.

Я не его поклонник, но стараюсь быть обьективным. Ян Валетов

\begin{itemize} % {
\iusr{Ян Валетов}
\textbf{Петро Закалата} и? Что не так?

\iusr{Zhanna Levchik}

я бы с вами согласилась не будь он президентом страны, которую необходимо
менять, у корой война семь лет. Не имел он права, с плохим образованием, без
единного понятия, балатироваться в президенты, а то, что его плохо образованный
народ избрал, так поверили в искренность героя сериала.

\iusr{Антон Овсиенко}
\textbf{Петро Закалата} Янукович тоже с низов пробился, так себе аргумент

\iusr{Dembel Fox}
\textbf{Петро Закалата}, 

мальчик никогда не стремился и не думал о политике, ну разве что, когда как
последняя бл@дь развлекал политиков на корпоративах и в саунах, мальчика сыграл
Карабас-Барабас (Беня) из-за того, что другой мальчик Пьеро ( Вакарчук),
которого собирался сыграть Пинчук, оказался более закомплексованным, и чего мы
должны писать кипятком от счасИя?))

\end{itemize} % }

\iusr{Дарья Тыльванчук}
Растропович? Это кто? Если это тот, о ком все подумали, то у вас, верно, опечатка.

\iusr{Вячеслав Валентинович}
Скоро увидим Яна на канале Украина

\begin{itemize} % {
\iusr{Ян Валетов}
\textbf{Вячеслав Валентинович} я там не раз бывал. И на 5-ом тоже.

\iusr{Viktor Shevchuk}
\textbf{Ian Valietov} Гарні канали, суцільна незаангажованість та об'єктивність. Напевно тому і захищаєте Ахметова, навіть не помічаючи цього.
\end{itemize} % }

\iusr{Александр Авдонин}
Розачарован (( Похоже, что парень не плохой, но очень глупый)

\iusr{Любовь Крылова}

"Потрясающая самоуверенность, плохое образование, недостаток воспитания,
жадность и презрение к окружающим в сочетании с большими деньгами - гремучая
смесь."- здорово вы подметили. И добавить нечего.


\iusr{Oleksiy Nesterenko}
Круто

\iusr{Сергей Писаренко}
А с чего взяли, что у Зе конфликт с Ра?

\iusr{Igor Green}
Bravo @igg{fbicon.hands.applause.yellow}{repeat=2} 

\iusr{Станислав С.}
С Nouveau riche все ясно. А какое ваше мнение о Nouveau pauvre?

\iusr{Den Romnov}
Путин у ворот )

\iusr{Лариса Николаева}

Интересна была бы история предков, зачинателей того олд состояния)

И нувориш с 30-летним стажем так себе пример, учитывая наши реалии. В Штатах
200 лет назад не было того, что получили наши нувориши в начале 90-х за почти
даром, некорректный пример. И период первоначального накопления капитала тоже к
нашим не подходит - просто тупо раздеребанили государственное добро.

\iusr{Vladimir Shohat Teitz}

Для того, чтобы появились \enquote{олд мани} нужны две вещи.

1) Протестантская религия, основными ценностями которой являются трудолюбие и бережливость.

2) Преемственность власти и стабильность государства. В совке за 100 лет
произошло миллион революций и переворотов, которые сопровождались грабежами и
переделом собственности.

Вот и попробуй, накопи из поколения в поколение.

Впрочем это не только в совке. Большинство людей, достаточно энергичных и
рисковых, чтобы заняться бизнесом, хотят иметь деньги здесь и сейчас, пока есть
здоровье и извиняюсь, стоит х\#й.


\iusr{Viktor Velychko}

Видимо, столоваться объедками самого щедрого сердца Донбасса сильно профитно.
Гордону платит \$300к в месяц. Вы не продешевили?

\begin{itemize} % {
\iusr{Ян Валетов}
\textbf{Viktor Velychko} сразу в бан.
\end{itemize} % }

\iusr{Николай Завадский}

как говорится "Посеешь ветер – пожнешь бурю", чего стоит чужой опыт, когда и
свой-то воспринимается, как ненужный. И ребята, не останавливаясь на
достигнутом, продолжают множить себе проблемы. И нам заодно.

\iusr{Ибрагим Габидулин}
Фейсом об тейбл!

\iusr{Pavlo Fomenko}
Наличие отсутствия или отсутствие наличия интеллигентности. Вот и результат.

\iusr{kirill nicolaev}

Ян. Буквально пару месяцев назад об этом писал человек который побывал изнутри.
Вот \url{https://lib.fontanka.ru/imenem-bratvy-1}

\iusr{Елена Концевая}

"Ахметов це є старі гроші". Ні, це нуворіш. І він ці гроші не заробив. І всю цю
олігархичну гидру треба руйнувати. А куди і з ким їздив хтось на гвинтокрилі
МЧС хто зна? Робити висновки на статті продажної УП? А чому люди так легко
хапають гачок?

\begin{itemize} % {
\iusr{Сергій Павлюченко}
\textbf{Елена Концевая} каким образом "руйнувати" и кто этим процессом будет руководить? Кто был никем сейчас у власти во всех коридорах. Результат налицо и именно об этом пост автора.

\iusr{Елена Концевая}
\textbf{Sergey Pavlyuchenko} ні, пане, автор допису говорить про инше. Читайте уважно. Він закликає не чіпляти ахметова.
Хто руйнує? Зеленський та його команда.

\iusr{Сергій Павлюченко}
\textbf{Елена Концевая} и я считаю, что лучше Ахметова руками Коломойского не трогать. А других рук у зелёных сегодня нет.

\iusr{Елена Концевая}
\textbf{Sergey Pavlyuchenko} руками коломойського? Це як? Ви не читали, що Приватбанк вручив позов коломойському на 5,5 млрд \$? Коломойському зараз самому непереливки.

\iusr{Сергій Павлюченко}
\textbf{Елена Концевая} вручить позов и получить по нему-это две большие разницы.

\iusr{Елена Концевая}
\textbf{Sergey Pavlyuchenko} 

до того щоб почати справу, треба зробити перший крок. І погодьтесь,
коломойський зараз сидить тихо. А ось Ахметов зробив для України погану річ -
змова щодо енергетики. І автор закликає ахметова не чіпляти. Ні, панове. Усі
обленерго Ахметов зобов'ян повернути державі. Законність цієї процедури маємо у
законодавстві України.


\iusr{Сергій Павлюченко}
\textbf{Елена Концевая} утопия, тем более в Украине.

\iusr{Елена Концевая}
\textbf{Sergey Pavlyuchenko} що дає вам підстави це стверджувати?


\end{itemize} % }

\iusr{Володимир Володимир}

Ось чому потрібно було залишити старі гроші в 19 му, а не вестися на халяву....
Із пана пан, а із хама пана довго чекати....

\iusr{Таиса Фаворская}

Потому мы и топчемся: шаг вперёд, два назад. Потому что некоторым Ахметова
больше жалко, чем себя.

\iusr{Таиса Фаворская}

Эта установка «чтобы не было хуже» и держит всю страну в стойле.
Договороспособный? Так пришло время проявить эту способность!

\iusr{Михаил Юрьевич}

Здравствуйте Ян, Вы можете написать мне в личку?! Я Вам пишу, меня бросает в
спам!

\iusr{Igor Stasenko}

можно вытащить быдло в цивилизацию. но нельзя вытащить быдло из быдла
@igg{fbicon.smile} 


\end{itemize} % }
