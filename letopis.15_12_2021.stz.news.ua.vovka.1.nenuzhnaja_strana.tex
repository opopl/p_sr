% vim: keymap=russian-jcukenwin
%%beginhead 
 
%%file 15_12_2021.stz.news.ua.vovka.1.nenuzhnaja_strana
%%parent 15_12_2021
 
%%url https://vovka.com.ua/?p=10320
 
%%author_id blog_obo_vsem_i_niachem
%%date 
 
%%tags 
%%title Никому не нужная страна
 
%%endhead 
\subsection{Никому не нужная страна}
\label{sec:15_12_2021.stz.news.ua.vovka.1.nenuzhnaja_strana}

\Purl{https://vovka.com.ua/?p=10320}

\ifcmt
 author_begin
   author_id blog_obo_vsem_i_niachem
 author_end
\fi

\begin{multicols}{2} % {
\setlength{\parindent}{0pt}

(Стиль розовых пони)

Блин, я себя ощущаю каким-то журналом «Огонек». Кто постарше должен помнить,
там вроде даже отдельная рубрика была «письма читателей», ну или как-то так.

Короче предыстория. Я ваще ФБ не перевариваю органически. Я не читаю личку, не
имею мессенджера, не листаю ленту. Я ваще не знаю ни одного селебрети из ФБ и
чем там сейчас дышит норот.

Я живу в другом измерении.

Но иногда вываливаюсь в это, в основном камменты почитать, или реже пописать в
них, если вы понимаете, о чем я. И так жэ иногда на меня выливаются сообщения,
которые пишут люди, вероятнее всего смотрящие мои видосы, потому шо таких
большинство среди моей подписоты.

Львиная доля сообщений это обвинения меня в работе на (подставить нужное), ну и
дальше по тексту. «За сколько продал свою жопу Кремлю» — мой самый любимый
вопрос-обвинение.

Я как представлю себе Кремль, скупающий жопы, всякий раз улыбаюсь.

Обычно я игнорирую все сообщения, начиная от теплых и заканчивая пожеланиями
гореть в аду. До лести мне дела нет, за теплые слова спасибо, ну а остальным
могу намекнуть, шо все там будем. Вот там и разберемся, кто на шо пожил.

Но недавнее письмо я в виде исключения не проигнорю, а на него отвечу.
Ва-первых, мне показалось что человек писал искренне, а ва-втарых, давно хотел
об этом написать. Я процитирую письмо с некими купюрами, ибо там, как мне
показалось личное. 

\begin{zzquote}
	
Доброго вечора! Мені подобається як ви інформуєте людей, і я з Вами згоден, що
в нас нажаль фінал надзвичайно плачевний. Скажіть на Вашу думку чи можливі
зміни в нашій державі на краще? Вся проблема в людях?

......
Мене просто цікавить Ваша думка, дякую!
\end{zzquote}

Начнем с того, что я никого не информирую. Такой задачи вообще не стоит. Я
просто веду себе свой дневничок и записываю туда свои мыслишки, которые из себя
мало чего ценного представляют.

Но как бы там ни было. Спрашивали? Отвечаем!

На тему «Возможны ли изменения в стране Украина к лучшему?» можно накатать
диссер.

Если кто не знает, это дета 150 страниц 12 шрифтом. Я прям ща, стока не выдам,
но по совокупности уверен, шо уже пробил эту планку.

Что бы ответить на этот вопрос, нужно провести одну небольшую аналогию.

Представьте себе торт «Медовик». Стопка коржей, промазанных сметанным кремом.
Если смотреть сверху, то однородный круглый продукт. В разрезе жэ нечто,
напоминающее зебру.

А теперь попробуем натянуть это на ситуацию.

Так сложилось, что 30 лет тому назад Украина пошла по странному эволюционному
пути.

Клептоманы, сменяя друг дружку у власти, в процессе тягания корыта по двору,
образовывали (каждый свой) новые коржи, сиречь слои торта.

Этот корж состоит из членов их семей, прихлебателей, родичей прихлебателей,
теток, дядек, лизоблюдов, всякой швали, которая живет по принципу
рыбы-прилипалы.

Как только кто-то из них дорывается до власти, вся эта сволота невероятно
распухает деньгами, а главное ЧСВ.

Чувством собственного величия.

Между коржами в качестве крема простые люди, пэрэсiчнi украïнцi.

Почему в качестве крема?

Вот эти представители коржей они вцепились в территорию. Они хоть что-нибудь из
себя представляют только на (в) Украине.

Уважение, почет, слава, звания, побрякушки, работают только до границы. До
терминала аэропорта «Борисполь».

Во всем остальном мире это обычные рядовые юниты.

Ну типа как Петр Алексеевич, который ненавидит жить в Испании. Тут его никто не
боится и перед ним не лизоблюдит. Тут всем поплевать на стоимость его туфлей и
корча. А в аптеку нада самому валить ножками. И его это дико бесит.

Люди выступают в виде крема, потому как они одновременно пластичны, ибо чуть
надавишь и разбежались, но в то жэ время связывают коржи между собой.

Без них вся эта конструкция просто нежизнеспособна, патаму шо должен жэ кто-то
ее обслуживать. Люди батрачат на вот этих хозяев страны, наполняя их карманы.
Но их карманы находятся за границей.

Зато каждый украинец, живущий в (на) Украине, просто потому что существует, уже
платит в бюджет. Это называется НДС, я уже рассказывал.

И получается так, что наливают в бюджет в основном обычные люди, с которого по
кругу кормятся все. От тех, кто сейчас у кормушки, до тех, кто был или будет у
нее в будущем.

То бишь доходы стабильно уменьшаются, расходы стабильно растут.

Надеюсь, более менее понятная аллегория.

Что из себя представляют нижние коржи торта Украина повествовать можно до новых
веников.

Я устану писать, а вы читать.

Кому сильно интересны срывы покровов, то самой годной по моему мнению
иллюстрацией стали откровения Ленечки Ройтмана. Бандюгана из Одессы, который
гордо именуется «русской мафией» в Вашингтонском обкоме.

Ну кто не в курсе или пропустил, кратенько пересказываю.

Если верить злым языкам, то одни киллеры вальнули своего работодателя, который
сколотил некий картель по зачистке неугодных на хозрасчете. А заодно выкинули
Леню, который там тоже трудился, на мороз. Отжали его доляху в акциях и бабло в
банке «Надра».

Ну, а как решает вопрос киллер с киллером? Понятное дело, шо не в шахматной
партии.

Но где-то потекло и готовящего зе тотал дэзмач вендетту Леонида принимают
бравые парни из ФБР, после чего он едет прямиком в тюрячку.

Это я сейчас рассказывал за от этих персонажей, если кто не знал:

\ii{15_12_2021.stz.news.ua.vovka.1.nenuzhnaja_strana.pic.1}

Леня оттянув срок в тюрячке откинулся и от обиды давай лить грязь на светлую
украинскую элиту.

Ознакомится с его пасквилями и тоннами наглой лжи можно вот тут и вот тут.

Но я прям уверен, шо бывший одессит Ленька все это придумал из зависти. Просто
он из своей замызганой Америки плачет каждый день в подушку за то, какую страну
он потерял.

Только вот в суд на него никто так и не подал, шо странно.

Хоть бы за клевету, не говоря уже за шота другое.

В 2014 Украина пробила днище и прошла точку ноль, что предопределило коридор в
котором она будет двигаться дальше.

Если кто не знает и думает, шо бодрые вояки УПА воевали с нацистами, так вот и
нет. Есть мнение, шо они шестерили нацистам в надежде шо те, признают их версию
Украины.

Нацисты эту конструкцию ваще в гробу видали, поскольку делили мир по простому
принципу. Мы типа годная нация, а вы тут все говно. В смысле унтерменши.

Ну так вот, гражданских недочеловеков отлавливали и отправляли на работу в
Гермашу, потому как негожэ панам вхаривать самотужки. А помогали им отлавливать
таких вот остарбайтеров полицейские силы, которые и состояли из всякого сброда
типа ОУН, шо (м), шо (б).

Из совка всего угнали примерно 5 лямов человеков. Из конкретно Украины шота там
2.4 ляма. Заценили циферки?

Ну а теперь слайды.

Злые языки клевещут, что с 2014 года с Украины свалило от 5 000 000 до 7 000
000 человеков.

Значит и шо мы имеем в сухом остатке? Крем поредел, коржи под собственным весом
сдавливают оставшихся.

Но и то было бы полбеды, но на старые коржи наслоились новые, потоньше, да
пожиже. Но сверху и так жэ требующие крем для связки и пропитки.

И что в результате?

Украина превратилась в страну, которая никому не нужна. В том состоянии, в
котором она есть на сейчас, имеется в виду.

Не, если ее отрихтовать, то прям милое дело. Но каждый видит этот процесс
по-своему.

Хозяева страны из первых слоев коржей выжали из советского наследства по
максимуму. Все. Больше красть нечего.

И тут фокус в том, что свалила-то самая работоспособная часть населения. А
пенсионеры и бедолаги, которым не повезло родиться в нормальной стране, но
которым тяжело сменить эту на шота лучшее, остались.

Это я про всяких там учителей, тех кто пошел в медицину не за деньгами, ну и
всех остальных потребителей из бюджета.

Переучиваться долго да и смысла особого нет, ибо просто так взять и шагнуть в
пропасть страшно.

И вот для хозяев страны они стали балластом. В их случае чугунной батареей на
груди.

Они ж такие бизнесмены, шо мама моя дорогая. Прям гении бизнеса.

Весь из бизнес план строится на недоплате налогов в казну и зарплаты тем, кто
батрачит на них за копейки.

А тут жэ вот какая незадача.

Количество тех, кто потенциально будет на них работать, упало, а тех, кто
требует бюджетных, денег рэзко возросло.

Они, как мне кажется, были бы сильно не против оставлять деньги у себя, в
пределах своей вотчины, которая обычно ограничивается областью — двумя, а не
кормить бесполезных с их жизненной позиции пенсионеров в масштабе всей страны.

Но это так сказать внутренние процессы. Но Украина жэ не сферический конь в
вакууме. На нее воздействуют и внешние силы.

Возьмем Европу. Продавив ассоциацию и зохватив рынок, выдав безвиз и получив
рабочую силу, им не сильно нужна страна клептоманов, которая постоянно клянчит
деньги.

Ну раз помогли, ну два. Но семь лет! Восьмой пошел, а воз все там.

Любое письмо в Брюссель выглядит как «Вышлите денег, здравствуйте…» И отказать
как-то не удобно. Каждый раз Украина рассказывает, шо если бы не она, то Путин
на танке уже б давно по Парижу катался.

Хотя лично мне не понятно зачем ему это? И если у Украины такая сильная армия,
что может на подступах к ЕС сдерживать орды русских варваров, то почему они
постоянно требуют помощи при малейшем шорохе с той стороны границы?

Ну стараться понять логику украинских Кулеб занятие не для слабонервных, так
что вернемся к Европе.

Лес, рынки и людишек они получили, а остаток хорошо бы было на кого-то
повесить. Например на русских. А пусть русские качают свой газ через украинскую
трубу в ЕС. Вот и папуасам баблишко будет перепадать. За транзит.

А на запросы «вышлите денег» всегда можно сказать «ну у вас жэ есть транзит,
какие деньги?» Сами, все сами.

Русским такое счастье даром не надобно. В 2013 они тоже видели в (на) Украине
рынок сбыта, но на сейчас это уже смешно. Украина импортирует даже тепличные
огурцы. Кстати из России.

Это вроде как так же рынок сбыта, но тогда вопрос был значительно ширше. Ну в
смысле ширее.

А сейчас смешно, потому как внезапно подорожал газ и продолжает дорожать по
любому поводу.

А Украина жэ отличается умом и сообразительностью. Вместо того, чтобы закупать
газ напрямую у агрессора, у которого она то ляктричество, то уголек покупала,
теперь вон огурцы, но конкретно газ она решила гордо покупать на бирже.

А на бирже кто? Спекулянты! Биржа она для того и придумана, чтобы
спекулировать. И эти спекулянты цену топлива чота не сильно хотят опускать.

Так что с большой долей вероятности, газ будет стоить космических денег минимум
пару лет. Но почти весь ЕС его получает по ценам прямых поставок и отказываться
от них собирается только к 2049 году.

А Украина, официально самая бедная страна в ЕС, покупает газ по максимально
возможной цене и крутит дули агрессору.

Это означает, шо работы будет сильно меньше, а значит и денег у людей. Потому
как на газ завязано так или иначе почти 90\% бизнеса.

Так что сбывать сюда шота вменяемое вряд ли получится.

Позиция русских по отношению к ЕС проста как тот жэ парниковый огурец. Вы
купите дополнительные объемы сверх контрактов, а мы их обязательно прокачаем
через Украину.

В ЕС обиженно сопят, но отлистывать не спешат.

А есть жэ исчо Вашингтонский обком и всякие там Канады.

Вон те вот подмахиватели Гитлеру в надежде, шо им перепадет, как увидели куда
дело катится, сразу сдались в плен бритам.

Их папа римский, как доблестных католиков, отмазал от выдачи проклятым красным
оккупантам и они осели в США и Канаде.

Называют себя диаспорой, собираются по выходным походить в вышиванках и
поспиваты писень. Но на этом все. Ни бабла подкинуть в каких-то внятных
количествах, ни тем более жить в (на) любимой нэньке очередь не стоит.

Зато наприсылали рехформаторов типа Супрун и Зварыча.

Правда они без дипломов из ВУЗов, но это такое. Для таких реформ дипломы не
нужны.

Главное помнить одно железобетонное правило: «Все, шо было в совке — плохо. В
любой непонятной ситуации делай ровно наоборот».

Нынешняя Украина их не устраивает советским прошлым.

Коммунисты ее оккупировали, но как-то странно. Вместо того, чтобы все выжать и
бросить, они зачем-то построили там разные больнички, школы, заводы.

И вот теперь, чтобы скинуть красное ярмо, хорошо бы было, пользуясь главным
правилом, все развалить, памятники снести, больнички закрыть, а лечиться можно
подорожником и содой, как написано в новых учебниках.

Причем адекватным людям, современным более менее, ну типа там АйТишнегам
всяким, нынешняя Украина тоже не вперлась. У них главный праздник Новый год.
Они то по сути и в Бога не особо-то веруют. Только если совсем припечет.
Привыкли так.

Так нет, жэ.

Им надо на голову нахлобучить католическое Рождество, а первого января надобно
походить с факелами и портретами человека, который никогда не имел украинского
паспорта.

Но он ваще всем украинцам украинец, почти как Павлик Морозов для пионэров. Как
там в песне поется? «Батько наш Бандера, Украина маты».

С факелами. В XXI веке. Толпой с портрэтами шляться по Киеву, в котором этот
персонаж ни разу не был. Громко орать как макаки в период гона, и оставлять за
собой горы мусора в центре столицы.

Вот адекватные люди и уходят во внутреннюю эмиграцию, наглухо отмораживаясь от
происходящего в стране. Ну или голосуют ногами, опять разбавляя крем.

А есть жэ исчо фонд Сороса. Поговаривают, что эта организация без мыла внедряет
в правительства своих представителей для преследования различных целей.

В основном работы по приумножению бабла транснациональных финансовых групп.

Но я уверен, шо это все грязная ложь.

Просто сердце обливается кровью и группы небезразличных людей, которые
высаживаются десантом в стране, где экономика валится в штопор, хотят поднять
ее с колен и заставить процветать, даже через силу.

Но под их прикрытием потянулась в страну первостатейная шушера типа Милованова,
который имеет аж 3 типа научные статьи и звание сопоставимое нашему доценту, но
в нем есть слово профессор.

Поэтому он бегает и всем рассказывает, шо он профессор экономики.

О которой знает по выжимке в пять листиков, потому как книжки умные читать это
долго и там многа букав.

Или Дэвида Брауна, в девичестве Арахамии, шо с Миловановым два сапога
кроссовки.

Оба тугие на столько, шо имеют пробелы в фундаментальных школьных знаниях, но
горят желанием изменить Украину, а то эта кривая какая-то.

Например, контуженный тротилом на рыбалке Милованов уверен, шо продавать надо
только прибыльные государственные предприятия, потому что убыточные никому не
вперлись.

Никто их не купит. Логично жэ! Гений.

А зачем продавать прибыльные предприятия? Так государство криворукий и
неэффективный собственник. Видимо он не в курсе, что в мире существуют
китайские госкомпании.

Ну и конечно жэ надо по-бырику распродать землицу. А то пока она не продана,
как-то неэффективно она используется.

Кстати, вчера Арахамия вращая глазами рассказывал, шо Украина такая канхфэтка
для инвестиций, патаму шо здесь зарплаты маленькие.

И это все в довесок всяким Шмыгалям и другим Гончарукам, который кретины на
столько, что не могут осознать, что они кретины. И то, что они несут и
вытворяют кажется им умными вещами.

Как вы думаете, при таких раскладах возможно ли улучшение уровня жизни? Я
лично, поскольку вы спросили мое мнение, не уверен.

И люди, о которых вы говорили в письме, тут не сильно влияют на ситуацию. Люди,
как люди. Только квартирный вопрос их испортил. Каждый в таком хаосе думает
только о своем. По всем слоям торта.

Нацарюваты и втэкты — вот самый годный план на жизнь. Ибо жить в этом не
возможно и просвета нет.

И вот эта вся ситуация должна вызреть. Пройти естественный процесс эволюции.
Какие-то коржи потеряв опору съедут, какие-то слипнуться, как-то оно
трансформируется и, скорее всего расползется.

Можно ли избежать этого?

Чисто теоретически да. Но это чистой воды теория, на практике мало применимая,
да и неизвестно что хуже.

Я говорю про революцию.

Только вот есть две проблемы.

Первая. Никто не знает точно когда она произойдет. Не верите? Был такой
картавый мужичонка в кепке, который еще надувное бревно таскал, так он в январе
1917 года писал следущее:

\begin{zzquote}
Мы, старики, может быть, не доживем до решающих битв этой грядущей революции.
Но я могу, думается мне, высказать с большой уверенностью надежду, что
молодежь, которая работает так прекрасно в социалистическом движении Швейцарии
и всего мира, что она будет иметь счастье не только бороться, но и победить в
грядущей пролетарской революции	
\end{zzquote}

Повторюсь, январь 17 года, если чо.

И вторая. Это должна быть настоящая революция. По взрослому. Без трусов. Без
отэтого «Ладно, ладно, вы победили, давайте договариваться». Только вот
проблема. Если полыхнет, то экономика точно не переживет. Уже сейчас те, кто
читают эти строки, не доживут до уровня 2013 года, даже если читающему 18 лет,
а прокашляет он до 80-ти.

Это физически не возможно.

Настоящая жэ революция, а не смена свиней у корыта, которая сметет касты,
перераспределит собственность, выпилит целые коржи из нашего тортика, она
откатит экономику в каменный век.

Можете понаблюдать за Африкой. Как не прогнали белых колонизаторов, так и
скатились в днище, когда производство тазика — цэ вжэ перемога.

А есть жэ пример Пакистана. Хоть и имеют они 130 ядреных бомб, тока шо воно им
дало? Сильно кто-то боится Пакистана? Кому он диктует свою непреклонную волю?

И вот положа руку на сердце, я не знаю, какое из зол меньшее. Я склоняюсь к
тому, что эволюция, а не революция. А как оно будет, время покажет.

Это мое личное мнение и я его не кому не навязываю, потому как могу шота
упускать из виду.

Простите за многа букав и за то, шо писал карандашом.

\ifcmt
	ig https://vovka.com.ua/wp-content/uploads/2020/04/las20sec.jpg
  @width 0.4
\fi

Там в самом низу страницы есть кнопочка «Donate». Если вы ее нажмете, то
отправитесь прямиком на страницу сервиса платежной системы PayPal, где есть
возможность перевести сколько-нибудь немножечко денежек. PayPal гарантирует
безопасность и прозрачность платежей.

Так же там есть ссылочка на Яндекс кошелек.

Ну и просто номер карточки для прямого пополнения.

А так же QR код от Capusta.Space.

Это не пожертвование. И я не отстал от поезда. Я не прошу подаяние и не пытаюсь
никого нажухать. Это деньги на развитие проекта.

Просто из-за всемирного движняка с коронавирусом я уже дважды остался без
основных источников дохода, на которые мы жили. Движняк прошел, уже второй на
подходе, а источники так и не восстановились. Теперь я вынужден заниматься не
тем, чем хочу и тем что мне интересно, а всякой дрянью, за которую платят.

Я трачу на этот бложик примерно по три часа в сутки, и если он не станет
приносить хоть какие-то деньги, то мне придется задвинуть его на задний план
своего ежедневного расписания, ибо надобно что-то есть, и на что-то жить.
Конечно, я не брошу его, но усталость не лучший стимул к творчеству.

На что пойдут деньги?

На оплату интернета. Ежемесячно это стоит около 50 евро, и скоро станет прям
роскошью. На хостинг и доменное имя. На рекламу. Про мой бложик знают полтора
человека, а алчная мордокнига никому ничего не показывает без денег. Может быть
сподоблюсь поменять ноутбук.

Если деньги будут появляться чуть более, чем в час по чайной ложке, то я
отключу гуглорекламу и это станет моим новым основным источником дохода.

В общем, если Вам нравится то, что я делаю — покормите котика. Если не нравится
— то тем более покормите котика. Сытый котик не мяукает.


\end{multicols} % }
