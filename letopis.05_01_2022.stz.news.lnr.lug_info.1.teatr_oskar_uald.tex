% vim: keymap=russian-jcukenwin
%%beginhead 
 
%%file 05_01_2022.stz.news.lnr.lug_info.1.teatr_oskar_uald
%%parent 05_01_2022
 
%%url https://lug-info.com/news/luganskij-teatr-v-ramkah-proekta-kul-turnaa-sreda-predstavil-tvorcestvo-oskara-uajl-da
 
%%author_id 
%%date 
 
%%tags 
%%title Луганский театр в рамках проекта "Культурная среда" представил творчество Оскара Уайльда
 
%%endhead 
\subsection{Луганский театр в рамках проекта \enquote{Культурная среда} представил творчество Оскара Уайльда}
\label{sec:05_01_2022.stz.news.lnr.lug_info.1.teatr_oskar_uald}

\Purl{https://lug-info.com/news/luganskij-teatr-v-ramkah-proekta-kul-turnaa-sreda-predstavil-tvorcestvo-oskara-uajl-da}
\ifcmt
 author_begin
   author_id  
 author_end
\fi

Луганский академический русский драматический театр имени Павла Луспекаева в
рамках проекта \enquote{Культурная среда} представил первую в новом году программу,
которая была посвящена творчеству ирландского писателя и поэта Оскара Уайльда.
Об этом с места события передает корреспондент ЛИЦ.

\ii{05_01_2022.stz.news.lnr.lug_info.1.teatr_oskar_uald.pic.1}

Автором очередного проекта \enquote{среды} выступила актриса театра Екатерина Митина.

Она вместе со своими коллегами окунула зрителей в событийную атмосферу жизни
писателя с его литературными образами, несущими особую эстетику, а слоганом
программы стала реплика самого Уайльда: \enquote{...мне нечего декларировать, кроме своей
гениальности}.

\ii{05_01_2022.stz.news.lnr.lug_info.1.teatr_oskar_uald.pic.2}

В программе приняли участие молодые актеры Владислав Солодилов, Артем Демянчук,
Руслан Махортов, Никита Вакуленко, Сергей Матвиенко, Анна Ладыгина, Анастасия
Воропаева, Анастасия Чепурова, Ольга Ткаченко, Дмитрий Иванов и Кирилл Галкин.

\ii{05_01_2022.stz.news.lnr.lug_info.1.teatr_oskar_uald.pic.3}

Митина отметила, что \enquote{через автобиографичные персонажи Уайльда, которые
развиваются, как и сам писатель, зритель сможет открыть свои собственные
утаенные грани}.

\ii{05_01_2022.stz.news.lnr.lug_info.1.teatr_oskar_uald.pic.4}

В программе артисты представили отрывки из нескольких произведений писателя, в
которых прослеживается трансформация мировоззрения самого автора на протяжении
его жизни.

По традиции, \enquote{Культурная среда} завершилась обсуждением увиденного и дискуссией
артистов со зрителями.

Заведующая литературной частью ЛАРДТ Ирина Цой отметила, что \enquote{творчество Оскара
Уайльда – это описание его жизни}.

\ii{05_01_2022.stz.news.lnr.lug_info.1.teatr_oskar_uald.pic.5}

\enquote{Это его самовыражение. Это удивительный талант, удивительный человек, который
из своей жизни смог создать такой блистательный роман. Его литература – это
высокая марка, и то, что мы сегодня прикоснулись к его творчеству,
познавательно для каждого из нас}, - сказала завлит.

Луганский академический русский драматический театр имени Луспекаева
располагается по адресу: Луганск, улица Коцюбинского, 9а.

Дополнительную информацию можно получить по телефонам: (0642) 50 20 31, (0642)
50 11 43, на сайте театра или на его страницах в социальных сетях \enquote{Фейсбук},
\enquote{ВКонтакте} и \enquote{Инстаграм}.

Напомним, в ноябре 2019 года ЛАРДТ имени Луспекаева презентовал проект
\enquote{Культурная среда}, который проводится в неформальной обстановке в фойе
учреждения культуры. Проект, с одной стороны, является экспериментальной
площадкой для апробации актерами новых театральных форм, с другой, дает
возможность прямого общения коллектива театра со зрителями.
