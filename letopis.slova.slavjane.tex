% vim: keymap=russian-jcukenwin
%%beginhead 
 
%%file slova.slavjane
%%parent slova
 
%%url 
 
%%author 
%%author_id 
%%author_url 
 
%%tags 
%%title 
 
%%endhead 
\chapter{Славяне}
\label{sec:slova.slavjane}

%%%cit
%%%cit_pic
\ifcmt
  pic https://avatars.mds.yandex.net/get-zen_doc/1687249/pub_5ff4cffafe4e686f6a93f46c_5ff4d882af142f0b17d231df/scale_1200
	caption Добрыня Сванельдич (он же Никитич) со своей сестрой Малушей, матерью Владимира Святославовича. Рисунок А. Рябушкина в книге \enquote{Русские былинные богатыри}, 1895 г.
\fi
%%%cit_text
То же самое можно сказать и о его ближайших потомках – князьях Игоре (сын),
Святославе (внук) и Владимире (правнук). Все они были не украинцами, а, скорее
всего, скандинавами. Но это по немецкой версии, а по версии российской – они
были смешавшимися с норманнами славянами с южных берегов Балтийского моря. Но
какие там могли жить \emph{славяне}? Там жили пруссы, эсты и прочие подобные
совсем не \emph{славянские народы}. Так что версия эта сомнительная.  Но вот у
современных украинцев вообще никакой версии нет. Вроде напрямую обзывать Рюрика
украинцем у них язык не поворачивается. Ну, это и естественно, так как
невесткой Рюрика была скандинавская же княгиня Ольга, а ее невесткой – матерью
Владимира – была сестра старшего дружинника князя Святослава и его ближайшего
советника Добрыни из рода Сванельдичей. Таким образом все данные указывают на
то, что ни по мужской, ни по женской линии у Владимира Святославовича не было
славян, а были одни викинги, они же норманны, они же русы
%%%cit_title
\citTitle{Почему киевский хан Владимир Креститель считается украинским князем, когда он даже славянином не был?},
Исторический Понедельник, zen.yandex.ru, 05.01.2021 
%%%endcit

%%%cit
%%%cit_head
%%%cit_pic
%%%cit_text
Но, повторимся, стратегически на отношение к \emph{славянам} эти маневры не влияли.
Как представителям \enquote{низшей расы} \emph{славянам} было уготовано в нацистской картине
мира бесправное положение, выселение с родной земли, политика по сокращению
численности, ассимиляция и, как итог, прекращение существования \emph{славянских}
этносов как таковых. Одним из первых идеологов нацистов, кто развивал идеи
\enquote{жизненного пространства} для немцев на востоке, был Альфред Розенберг. Он
родился в Российской империи в семье прибалтийских немцев, учился в Москве,
хорошо знал русский язык и поэтому считался среди нацистов видным
\enquote{специалистом} по \enquote{\emph{славянскому вопросу}}
%%%cit_comment
%%%cit_title
\citTitle{22 июня - 80 лет нападения на СССР. Что немцы готовили для украинцев}, 
Максим Минин, strana.ua, 22.06.2021
%%%endcit

%%%cit
%%%cit_head
%%%cit_pic
%%%cit_text
В \emph{славянских} культурах (в русской в частности) Слову придаётся огромное
значение. Но преимущественно оно используется, чтобы уязвить оппонента,
продемонстрировать его инаковость, ущербность и враждебность обществу, объявить
его врагом народа и уничтожить. До сих пор от людей, декларирующих своё желание
облагодетельствовать всё человечество, мы регулярно слышим: «Нет на вас
Сталина». Пожеланиями если не перевешать политических оппонентов, то отправить
их в ГУЛАГ навсегда, пестрит интернет.  Как всё это исправить? Хороший совет
был дан в советском мультфильме о Крошке Еноте: «А ты ему улыбнись». Вот только
примерно такой же завет оставил нам Господь. Вряд ли человечество, не
прислушавшееся к своему Создателю, послушает создателей детского мультфильма
%%%cit_comment
%%%cit_title
\citTitle{О Слове, Боге и разделённом человечестве и Крошке Еноте}, 
Ростислав Ищенко, ukraina.ru, 30.10.2021
%%%endcit

