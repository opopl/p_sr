% vim: keymap=russian-jcukenwin
%%beginhead 
 
%%file 26_01_2019.stz.news.ua.mrpl_city.1.delo_tabak
%%parent 26_01_2019
 
%%url https://mrpl.city/blogs/view/delo-tabak-kak-kurili-v-mariupole-so-vremen-kalmiusskoj-palanki
 
%%author_id burov_sergij.mariupol,news.ua.mrpl_city
%%date 
 
%%tags 
%%title Дело табак: как курили в Мариуполе со времен Кальмиусской паланки
 
%%endhead 
 
\subsection{Дело табак: как курили в Мариуполе со времен Кальмиусской паланки}
\label{sec:26_01_2019.stz.news.ua.mrpl_city.1.delo_tabak}
 
\Purl{https://mrpl.city/blogs/view/delo-tabak-kak-kurili-v-mariupole-so-vremen-kalmiusskoj-palanki}
\ifcmt
 author_begin
   author_id burov_sergij.mariupol,news.ua.mrpl_city
 author_end
\fi

\begin{quote}
\em	
КУРЕНИЕ - вдыхание дыма некоторых тлеющих растительных продуктов. Курение
табака — одна из наиболее распространенных вредных привычек (в Европе - с 16
в., в России - с 17 в.), отрицательно влияющая на здоровье курильщика и
окружающих его людей; способствует развитию болезней сердца, сосудов, желудка,
легких. Табачный дым содержит канцерогенные вещества.

Большая энциклопедия Кирилла и Мефодия, 2006.
\end{quote}

Как вы думаете, кто первым в наших краях начал курить табак? На этот вопрос еще
в прошлом веке дали ответ археологи. Делая раскопки у берегов Кальмиуса на
месте зимовников запорожских казаков, занимавшихся рыбной ловлей, ими были
найдены курительные трубки. Мало того, они откопали слипшиеся в монолит
обожженные заготовки трубок-люлек. Видимо, гончар хотел бросить брак в мусор, а
попал случайно в печь для обжига своей продукции. Значит, было немало любителей
подымить, коль было налажено массовое производство приспособлений для этого. А
пользовались люльками не только обитатели зимовников, но и казаки, несшие
службу в военном укреплении - административном центре Кальмиусской паланки. Оно
располагалось на территории нынешней площади Освобождения. Правду говорят:
какой же казак без люльки? Гоголевский Тарас Бульба даже в руки врагов попал,
пока искал свою любимую люльку.

\textbf{Читайте также:} 

\href{https://archive.org/details/14_04_2018.sergij_burov.mrpl_city.kogda_byl_osnovan_mariupol}{%
Когда был основан Мариуполь?, Сергей Буров, mrpl.city, 14.04.2018}

В 1780 году в Северное Приазовье прибыли греки – переселенцы из Крыма. Как у
них было насчет того, чтобы \enquote{подымить}? Этнографы отмечали, что курили не
только мужчины, но и пожилые женщины. Перенесемся к началу ХХ века. Полистаем
давно ставшую раритетом книгу под названием \enquote{Адрес-календарь: Весь Мариуполь},
которую отпечатал в 1910 году местный типограф С. А. Копкин. Там находим раздел
\enquote{Табачные торговли}, где перечислены четыре табачных магазина на Екатерининской
улице (Нерофиди, Фукса, Прика и Коминаровича) и один – на Торговой улице,
владельцем которого был некто Куркчи. Наверняка, кроме магазинов, были и
табачные киоски, да и без разносчиков папирос дело не обходилось. Торговали в
Мариуполе сигарами, курительными, нюхательными и жевательными табаками от
известных в Российской империи фирм: московской \enquote{Габая}, ростовской \enquote{Оноли},
керченской \enquote{Мисаксуди}, феодосийской \enquote{Стомболи}. Конечно же, были и папиросы.
Название одних из них увековечено в 1923 году Владимиром Маяковским в рекламе
для Моссельпрома. \emph{\enquote{Нами оставляются от старого мира только папиросы \enquote{Ира}}}, -
фраза, которая вошла во все собрания сочинений поэта.

После революции 1917 года и Гражданской войны табачные компании, как, впрочем,
и все другие частные предприятия, были национализированы и их папиросы и другие
изделия для любителей затянуться дымком получили новые названия. Но в тяжелые
годы усладой для мариупольских курильщиков были не \enquote{Красная звезда} или та же
\enquote{Ира}, а махорка и самосад. В предвоенные годы во всей стране (и в Мариуполе
тоже) появились в мягкой упаковке папиросы \enquote{Дымок}, \enquote{Спорт}, \enquote{Метро},
\enquote{Беломорканал} с изображением этого гидротехнического сооружения, построенного
заключенными, а также \enquote{Казбек} в картонной коробке с лихим джигитом на фоне
гор, обладание которыми было признаком начальствующего положения. Кстати,
\enquote{Казбек} частенько фигурировал в руках героев кинофильмов предвоенных лет. Но
предыдущее почерпнуто больше из книг, да из услышанного некогда от
представителей того поколения, что ушло в мир иной.

\textbf{Читайте также:} 

\href{https://mrpl.city/news/view/v-mariupole-pressu-pustili-na-samokrutki--foto-}{%
В Мариуполе прессу пустили на самокрутки?, mrpl.city, 08.09.2016}

А вот кое-что из \enquote{табачной} темы, сохранившееся в памяти автора этих строк.

Старый мариупольский базар, первые послевоенные годы. Вдоль стены сгоревшего
здания против мясных лавок на низеньких скамеечках сидят люди с инвалидностью и
старики. Перед ними небольшие мешочки: одни с махоркой, другие с искрошенным
самосадом, а иные со сплюснутыми густо-коричневыми пачками табака в листьях.
Здесь же старые газеты – материал для \enquote{самокруток} и \enquote{козьих
ножек}. Эти газеты предварительно тщательно просмотрены, чтобы, не дай Бог, не
попались номера с портретами руководителей партии и правительства, а тем более
генералиссимуса, иначе быть большущей беде: можно загреметь в Воркуту или
Магадан по известной статье. Рядом – нехитрый торговый инвентарь: стандартный
граненый стакан и стаканчик \enquote{на сто грамм}. Время от времени к
табачному ряду подходят покупатели, каждый из них выбирает стакан и табак или
махорку по средствам: маленький или большой. Продавец, получив деньги, ссыпает
содержимое своей мерки либо в подставленный кисет, либо прямо в карман.

Еще одно воспоминание. В доме № 15 на Харлампиевской улице, его местный люд
называл \enquote{домом с хлебным магазином}, в тех помещениях, где сейчас помещаются
отделения каких-то банков, находился центральный \enquote{Гастроном}. Он казался тогда,
а это было в самом конце сороковых годов прошлого века, огромным и невероятно
шикарным. В одном из его окон были выставлены фанерные щиты, на которых
мастерски масляными красками были изображены груды колбас, сочные окорока,
разрезанные головки голландского сыра и иной снеди, которой и в помине не было
ни в этом, ни, тем более, в иных магазинах города. А в другом окне стояла кукла
– джентльмен ростом около метра - одетая в серый в полоску коверкотовый костюм,
которая судорожным движением подносила к губам папиросу, при этом на конце ее
загоралась маленькая лампочка, имитируя тлеющий табак. Кукла стояла на фоне
огромного муляжа коробки \enquote{Казбека}.

\textbf{Читайте также:} 

\href{https://mrpl.city/blogs/view/zhit-horosho}{Жить хорошо!, Владимир Щербаков, mrpl.city, 18.10.2016}

Из школьных лет. В мужской средней школе № 4 имени И. А. Крылова (сейчас в ее
здании находится Дом учителя) внутри здания уборной не было, она находилась во
дворе. На перемене около нее собирались приобщенные к курению старшеклассники,
чтобы \enquote{потянуть} папироску. Иногда кто-то из них попадал в поле зрения Ивана
Терентьевича Сидорчука, директора школы. Когда раздавался звонок, ребята
стремглав мчались в школу, тут-то их и встречал директор... Замеченных в
нарушении школьных правил движением пальца манил к себе: \enquote{Давай цигарки}. И что
вы думаете? Отдавали. Следует попутно отметить, что в описываемую эпоху увидеть
девушку, даже студентку с папиросой было невозможно. Да, некоторые женщины
среднего возраста тогда курили, но только не на улице.

Наверное, настало время вспомнить о предпочтениях мариупольских курильщиков в
табачных изделиях. Если сейчас папиросы – некая экзотика, то в начале
пятидесятых годов ХХ века все было наоборот: на полках магазинов царили уже
упомянутые \enquote{Беломорканал} и \enquote{Казбек}, а также \enquote{Шахтерские}, ростовская \enquote{Наша
марка}. А вот первой \enquote{ласточкой} среди популярных сигарет, пожалуй, стала
\enquote{Прима}. Ее красная аккуратная глянцевая коробка с черной полоской была
обтянута тонким целлофаном. Изготовителем \enquote{Примы} была Рижская табачная
фабрика, а она была по тогдашним понятиям почти заграничной. Ее привозили
мариупольцы из Москвы, Ленинграда, Киева и той же Риги, которым по разным
надобностям приходилось посещать эти города. Вскоре рижской торговой маркой
воспользовались и другие фабрики Советского Союза. Правда, их коробочка
лишилась глянца и целлофановой обертки, а сигареты - приятного вкуса.

\vspace{0.5cm}
\begin{minipage}{0.9\textwidth}
\textbf{Читайте также:} 

\href{https://mrpl.city/news/view/mariupolets-vybil-synu-zuby-za-kurenie}{%
Мариуполец выбил сыну зубы за курение, Роман Катріч, mrpl.city, 21.10.2018}
\end{minipage}
\vspace{0.5cm}

\ii{26_01_2019.stz.news.ua.mrpl_city.1.delo_tabak.pic.1}

Лет через десять в нашем городе стали очень популярными, особенно среди
молодежи, болгарские \enquote{Шипка} и \enquote{Слънце}. Сообразительные
земляки сразу догадались, что здесь речь идет о нашем дневном светиле, и,
обращаясь к продавщицам, говорили ласково: \emph{\enquote{Дайте \enquote{Солнышко}}}.
И \enquote{Шипка}, и \enquote{Слънце}, как впрочем, и \enquote{Прима}, не имели
привычного сейчас фильтра. Но зато были относительно дешевы. О сигаретах с
фильтром поклонники курения узнали от моряков Азовского морского пароходства.
Пачка \enquote{Мальборо} с фильтром была больше, чем ее прямое предназначение,
это был символ принадлежности к некоему сообществу избранных, символом
продвинутости. Местные снобы готовы были отдать любые деньги, чтобы иметь
возможность в обществе себе подобных небрежно вынуть из кармана красно-белую
коробочку с легким ментоловым запахом.

Монополия заокеанского продукта длилась не так уж долго. \enquote{Шестнадцатая
советская республика}, так иронично называли Болгарию, стала ввозить к нам
несколько наименований сигарет с фильтром. Это \enquote{Родопи},
\enquote{Стюардесса}, \enquote{Ту-134}, \enquote{Интер}, \enquote{Опал},
\enquote{Вега} в мягкой упаковке и \enquote{БТ} в коробочке из тонкого картона.
Поспела и отечественная табачная промышленность. Ее произведения назвали
\enquote{Космос}, \enquote{Столичные}, \enquote{Ленинград}, \enquote{Киев},
\enquote{Таллин} и другие. Одно время все табачные отделы мариупольских
продовольственных магазинов заполонили кубинские сигареты \enquote{Партагас} и
\enquote{Лигерос}. Их особенность состояла в том, что с непривычки при первой
же затяжке запирало дыхание. Самое интересное, что были-то они не очень
крепкими.

\ii{26_01_2019.stz.news.ua.mrpl_city.1.delo_tabak.pic.2}

В перестроечную эпоху, когда языки развязались, кто-то из мемуаристов
обмолвился, что И. В. Сталин набивал свою знаменитую трубку табаком, разминая
папиросы \enquote{Герцоговина флор}. Московская фабрика \enquote{Ява} воспользовалась случаем и
стала массово изготавливать их. Помнится, когда приходилось ехать в
командировку в Москву, обязательно находился оригинал, который просил:
\emph{\enquote{Привези, пожалуйста, Герцоговину}}.

\textbf{Читайте также:} 

\href{https://mrpl.city/news/view/na-ves-zolota-mariupoltsam-pridetsya-dorozhe-platit-za-vrednye-privychki}{На вес золота: мариупольцам придется дороже платить за вредные привычки, Олена Онєгіна, mrpl.city, 24.11.2018}

Этот краткий экскурс в историю вовсе не призывает продолжить пагубное
пристрастию к вдыханию табачного дыма. Совсем наоборот. Для того и предпослан
тексту эпиграф, позаимствованный из электронной версии Большой энциклопедии
Кирилла и Мефодия.
