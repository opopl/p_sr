% vim: keymap=russian-jcukenwin
%%beginhead 
 
%%file 23_11_2020.fb.kosarenko_stas.zaporozhie.1.maidan
%%parent 23_11_2020
 
%%url https://www.facebook.com/s.v.kosarenko/posts/10225096097933700
 
%%author_id kosarenko_stas.zaporozhie
%%date 
 
%%tags maidan2,raskol,ukraina
%%title Майдан не был следствием раскола Украины, он стал её причиной
 
%%endhead 
 
\subsection{Майдан не был следствием раскола Украины, он стал её причиной}
\label{sec:23_11_2020.fb.kosarenko_stas.zaporozhie.1.maidan}
 
\Purl{https://www.facebook.com/s.v.kosarenko/posts/10225096097933700}
\ifcmt
 author_begin
   author_id kosarenko_stas.zaporozhie
 author_end
\fi

в историческом масштабе месяц-другой туда-сюда особого значения не имеют. Но,
пока еще кто-то что-то помнит и не успели переиздать учебники истории,
нацарапайте где-нибудь гвоздиком:

Майдан не был следствием раскола Украины, он стал её причиной. То есть, люди
начали вспоминать за єдину країну не тогда, когда вышли жечь покрышки и
стремиться в Европу, а когда Крым уже помахал ручкой, а Донбасс возмутился. И
не забывайте Яценюка (этого нацарапывать не нужно), вдруг обратившегося к
жителям Юго-Востока на русском языке. Это тоже случилось весной, когда часть
страны уже майдан возненавидела. Сложите правильную картинку у себя в голове и
подумайте, кто и зачем сделал так, что в Украине теперь территории процентов на
10 меньше, населения миллионов 35, из них не одна тысяча увечных, а война еще
идет. Если скажете, что это Путин, признайте, что майдан организовал тоже он.

\#\_четырнадцатый

\ii{23_11_2020.fb.kosarenko_stas.zaporozhie.1.maidan.cmt}
