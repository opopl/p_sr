%%beginhead 
 
%%file 01_01_2023.fb.galushko_denis.1.mi__khali_bakhmutom
%%parent 01_01_2023
 
%%url https://www.facebook.com/HalushkoDenis/posts/pfbid0USpQJ3tkzd2LHssFCUepHdv2yBDUaqetKirKkrY2VmbH12VJRPaRfvMxaHsG3Cqnl
 
%%author_id galushko_denis
%%date 01_01_2023
 
%%tags 
%%title Ми їхали Бахмутом
 
%%endhead 

\subsection{Ми їхали Бахмутом}
\label{sec:01_01_2023.fb.galushko_denis.1.mi__khali_bakhmutom}

\Purl{https://www.facebook.com/HalushkoDenis/posts/pfbid0USpQJ3tkzd2LHssFCUepHdv2yBDUaqetKirKkrY2VmbH12VJRPaRfvMxaHsG3Cqnl}
\ifcmt
 author_begin
   author_id galushko_denis
 author_end
\fi

\#запискидилетантазсу

31.12

Ми їхали Бахмутом,  заїхали на позицію подивитися чи ще вона ціла все добре,
зупинилися поміж домів:

- Мітя посигналь їм... 

Місцеві старі з радістю підбігли, до машини, думаю єдиною свіжою темою на
обговорення, на сьогодні, в них було святкування нового року серед руїн у
підвалі, поки вони чапали з іншого кінця вулиці до нашої машини, ми оглянули
місцевість і околиці, все ціле, на днях можна виходити і працювати. 

Старі зраділи, як діти, знаєте старші люди стають з віком сентиментальними, ви
можете це помітити по своїм батькам. 

Так і тут, ці два старика, яким цілий зам.мера, я сам здивувався, так в Бахмуті
є зам.мера і це жінка, довірила налагоджувати і роздавати гуманітарну допомогу
серед мешканців і вони з усією відповідальністю, взялися за цю справу, громада
майже в п'ятдесят людей, є діти. 

Ми не гуманітарка, ми військові, просто я знав, ціла позиція чи ні, ми
зустрінемо тут людей, а вони не знали, ми як новорічний сюрприз, як, як, так
блядь як дід мороз і нічого з цим тут вже не зробиш, вони тут так виховані, ми
привезли солодке і шкарпетки, трохи дали термух. Вони провели нас словами
«ї...ште етиих п....ов! І берегите себя»

Їдемо далі, в Бахмуті, тихо, відносно тихо, приходів по місту мало і вони десь
далеко на пагорбах, в околицях.  Від тої тиші понівечене і спустошене місто
набуває ще більш мерзенішого вигляду.

Війна, постріли, палаючі скелети будинків зі знесеними від вибухів дахами це
смерть, але і вона дає динаміку, в якій тобі вже якось спокійніше, а тиша на
війні, завжди моторошна хуйня, від тиши не знаєш що очікувати. 

Хлопці з точки евакуації пітвердили, другий день тихо, вчора теж, був обмін
полоненими, ми менше стріляли, вони менше огризались, штурмових дій не було. 

Ми на штабі одної з бригад, що на найгарячішому з напрямків оборони Бахмута,
екрани, трансляції, бою в реальному часі, ілюзія відносної тиші розвіюється як
цигарковий дим. 

Всі напружені, іде координація дій, на виїзді комбат, зам комбата всі в полі
під Бахмутом, комбриг передає ,перевести вогонь артилерії на інший напрямок. 

Ти бачиш тільки екран, де знімають реальне кіно про війну, де герої за яких
вболіваєш  нажаль гинуть в реальності, а є ворог, його ні хто не кликав, тому
він просто точка на екрані яку варто знищити. 

Я забираю необхідні дані і вшиваюся, на мене ні хто не звертає уваги, іде бій,
а мене на поверхні чекає машина і туди теж може прилетіти. 

На точці евакуації дали медикаментів, їм не потрібних, Академік просив передати
їх далі по Бахмуту в стаб. Центр.

А він переповнений ліками вивантажувати нема де, все є, повезли далі під
Бахмут в ще один стабілізаційний центр. Ми їхали порожняком, тож великі
коробки всі вмістилися в пікап. 

У дворі стоять машини евакуації, хамві не глушив двигун, вони везуть п'ятьох
далі на Краматорськ, потім важких до Дніпра, двигун не глушить щоб в
середині евакуаційної капсули холодної броні, було тепло, адже поранених
туди завантажують майже голими.  

Я забіг всередину по сходам, що з'єднані зі спуском для нош та візків, забіг
швидко, бо не хотів, щоб наша машина довго стояла на парковці евакуації, мене
зупинило смердюче гниле повітря, воно було ледь чутне, але цього було
достатньо щоб свідомість підняла всі спогади за минулий рік.

Ви були в морзі літом, ви були в морзі літом під час війни? А в полі? В якому
штиль, немає вітру, валяються чорні людськи трупи та дохла скотина... 

Цим літом я вимивав ще довго милом собі ніс, тепер цей сморід, може нагадати ці
нюанси, без яких не знати вам «романтики» війни. 

Знов наче в кіно, в приміщені нікого, коридором на мене вибігає вівчарка, перша
ідіотська думка: \enquote{- дякую блядь що без кістки у пащі}. 

Бачу офіцера, лікаря, доповів за ліки, просять чекати. 

Коридором на мене несуться ноші на колесах, перший важкий пішов у хамві,
загорнутий в золоту термо ковдру наче новорічний подарунок. 

І так наче новорічний подарунок, не інакше, ледь дихає, в крові і з трубками з
відкритими очима, не в мішку, це вже новорічне чудо, це прекрасно, хай так, але
він дихає.  

Я сконцентрований на своєму, вихожу на вулицю щоб не заважати, до мене
підходить водій одного з еваків, він навіть не вітається, не представляється. 

А просто починає:

- А цього зараз на Крам повезуть, згодом може далі на Дніпро, дійсно важкий, а
в мене вчора....

Я мовчу, знаєте, це так буденно для цього водія, я намагаюсь щось збагнути в
його не звязаній між контекстом мові, потім просто я вже збагнув, він без
вихідних по напрямку Бахмут і весь його екіпаж також, а лікарі захриплі катають
на колесах ліжка туди сюди навіть без слів, кому він ще виговориться, кому ще? 

Йому то треба, просто щоб не йобнутися мізками кінцево, я стою мовчу, йому
цього достатньо, що я просто періодично дивлюся йому в очі.

Це нормально, тут багато чого нормального, що за межами цього здається дивним. 

Він закінчив, вивозять чергового пораненого, просять допомоги, він побіг.

Згодом, він підійшов до мене, коли ми з лікарем вже вивантажували машину:

- О шкарпетки! А можна я шкарпеток візьму, а то хлопців привозять всіх в крові,
речі зрізають, потім хоч шкарпетки теплі будуть...

Різні речі, набувають тут інших цінностей, деякі стають такими дріб'язковими, а
деякі дрібниці стають дуже важливими. 

Ми вийшли з Бахмута, сьогодні не працювали, мали вільний час, 16:30 ми на базі,
сьогодні 31.12. кінець зв'язку.

