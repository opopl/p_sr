% vim: keymap=russian-jcukenwin
%%beginhead 
 
%%file 01_11_2021.fb.lesev_igor.1.ubijstvo_poljakov
%%parent 01_11_2021
 
%%url https://www.facebook.com/permalink.php?story_fbid=4737343512963381&id=100000633379839
 
%%author_id lesev_igor
%%date 
 
%%tags __oct_2021.smert.poljakov_anton.deputat,politika,poljakov_anton.ukr.deputat,ukraina
%%title Кто покрывает убийство Полякова?
 
%%endhead 
 
\subsection{Кто покрывает убийство Полякова?}
\label{sec:01_11_2021.fb.lesev_igor.1.ubijstvo_poljakov}
 
\Purl{https://www.facebook.com/permalink.php?story_fbid=4737343512963381&id=100000633379839}
\ifcmt
 author_begin
   author_id lesev_igor
 author_end
\fi

Кто покрывает убийство Полякова?

Вчера вышел «Хард» Влащенко со Скороход. Это было интересное, эмоциональное, а
в здоровом обществе должно было стать еще и резонансным интервью. Еще раз для
осознания. Действующий народный депутат заявляет, что его коллегу – другого на
тот момент действующего нардепа – убили. И дальше приводит массу железобетонных
аргументов, почему это именно так, а не как-то иначе.

Пару раз я встречал аналогии, что Поляков ддя Зеленского, это как Гонгадзе для
Кучмы. Ну это даже близко не так. Хотя бы потому, что Зеленский даже близко не
Кучма, и он не стоит того, чтобы против него выдумывали какую-то невероятно
сложную шахматную партию. Даже Великий комбинатор в шахматах умел хотя бы
правильно ходить фигурами, а с Зеленским можно играть только в Чапаева.

Если кто забыл, просто приведу несколько фактов по Гонгадзе, чтобы мы больше к
нему не возвращались. На момент убийства это был абсолютно безвестный
журналист. Это к истории о том, что он де критиковал Кучму и за это его убили.
Когда безвестность кого угодно критикует, на выходе выходит безвестная критика.

Никаких явных мотивов убийства безвестного Гонгадзе не было, но его убивают,
еще и максимально резонансно – отрезают голову. Именно по такому убийству он
становится известен всей Украине.

И, наконец, уже после убийства Гонгадзе появляются «пленки Мельниченко», на
которых слышно, как Кучму регулярно надрючивают против абсолютно неизвестного
ему журналиста. А дальше вы уже в курсе чем вся та история закончилась. Кучма
так и не решился пойти на третий срок.

Теперь по Полякову. Это все-таки был публичный человек, к тому же, один из
самых ярких критиков режима Зеленского. Вы скажете – но это же не причина его
убивать. Да, если власть адекватна, то, конечно же, не причина. Но если власть
неадекватна?

Итак, немного хронологии. Утро, когда стало известно о смерти Полякова,
ботоферма Офиса начинает вдруг разгонять тему, о том, что покойный
сожительствует с депутатом Скороход. Оба женаты, дети, но вот сожительствуют. А
еще что-то у них не клеится и Поляков в депрессии.

Дальше в эфир выходит грузинская декабристка Ясько и сообщает, что в крови
Полякова обнаружен метадон. И вот пазл вроде бы сходится. Парень в депрессии, а
тут подоспела наркота, не совмещение с алкоголем, ну и «не повезло». Типа, с
кем не бывает.

Но у этой легенды присутствует одна боевая несостыковка. Нет ни одного
свидетельства от людей, которые его хотя бы шапочно знали, что Поляков
употреблял наркоту. Окажись на его месте, скажем, народный депутат от «слуг»
Сашко Юрченко (желаю ему долгих лет жизни), никто ведь, согласитесь, такой
легенде даже бы не удивился. Ну потому что у персонажа есть определенная
нарко-история.

А это как репутация. Вот работает в коллективе, скажем, парень, который
регулярно потрахивает местных девчушек. И вроде бы он об этом не рассказывает,
и табличку на груди не носит, но все об этом знают. Другой неравнодушен к
стакану, и не надо к человеку даже в душу лезть, чтобы это увидеть через очень
непродолжительное время.

В публичной политике все точно так же. Согласно городским легендам, тот же
грузинский узник Ясько очень даже на «ты» с кокаином. Такие же городские
легенды бытуют о Тимошенко (имя можете подставить на свой вкус). И, конечно же,
эти городские легенды подобное судачат и о Зеленском. Мы все это где-то и
когда-то хотя бы пару раз слышали. Но я, например, ни разу не слышал, чтобы
Кличко употреблял наркоту. И о Полякове даже в чернушных материалах до его
смерти вы не найдете о нем ни одной нарко-истории.

Теперь к истории с таксистом. Она экстраординарна сама по себе, даже если бы
речь не шла о народном депутате. Менты останавливают машину, в которой
обнаружен труп человека. Таксист изначально дает неправдивые показания, что
якобы только-только посадил клиента. Затем всплывает история с пересадкой из
одной машины в другую. У самого таксиста оказывается еще есть и судимость. А
сама Скороход рассказывает о борозде на шее у Полякова и прямо говорит, что ему
наркоту вливали силой.

И что же? А ничего. Таксист и приятели-выпивохи Полякова по-прежнему вне
подозрений у ментов. У нас в стране открывают уголовные дела за ношение
футболки с надписью «СССР», и не открывают против фигурантов, которые явно
причастны к убийству народного депутата.

И еще два важных момента. Во-первых, где видео с камер последнего дня жизни
Полякова? Он ведь не в лесу гулял. Все происходило в Киеве, где даже в парках
есть камеры видеонаблюдения. И вот вдруг нам ничего не показывают.

Сравните с историей убийства мэра Кривого Рога Павлова. Нам тут же начали
парить версию о его самоубийстве и в сети стали появляться дозированные ролики
с камер видеонаблюдения. При этом, чертовски ужасного качества, как будто мэр
не может позволить купить себе нормальные камеры. Потом появляется «полная
версия», из которой так же ничего не ясно. Но главное даже не это. Эти видосы в
сеть могли слить только менты, или люди, которые первыми приехали в дом
Павлова. И сливали их только для того, чтобы убедить хомячков, что мэр сам себя
застрелил.

И сравните историю с «покушением» на Шефира. Кто-то видел видео, как
обстрелянная машина Шефира подъехала к АТБ, где по определению есть камеры
видеонаблюдения? И я не видел. Нам с вами показывают дозированно то, что
считают нужным, а когда история слишком палевная – не показывают вообще ничего.
Хотя в случае с Поляковым ментам только по одним камерам видеонаблюдения можно
было бы в первые часы отобразить все его передвижения и встречи накануне
смерти.

И во-вторых, обратите внимание на скорострельность похорон и Полякова, и того
же Павлова. Есть веские основания считать, что эти люди были убиты.
Железобетонные основания, на что указывает масса фактов. Но их хоронят на
третий (Поляков) и четвертый (Павлов) день. Не хочу показаться совсем циничным,
но это называется – зарывать улики в землю.

Вот просто для аналогии. Помните этого белорусского оппозиционера Шишова,
которого нашли повешенным в киевском парке? Почти все о его существовании
узнали именно после его смерти. Но его похоронили только на 55(!) день после
смерти. Была задача копать под Луку. Не накопали. Но если стоит задача что-то
найти, именно так долго и проводятся экспертизы. Не за два, и не за три дня.

Ну и последнее. Был ли какой-то реальный мотив убийства Полякова? В последней
своей передаче на эфире Антон в очередной раз рассказывал о зп в конвертах
«слугам народа». Это чистой воды коррупция и уголовщина, понимаете.

Но просто рассказывать – это такое дело. Мы, например, рассказываем о «большой
стройке», об откатах в 20\%, об отсутствии тендеров, о закрытом картеле и пр. Но
рассказывать и, скажем, обратиться с заявлением в прокуратуру – это,
согласитесь, очень разные вещи.

А Поляков накануне своего убийства подал заявление в НАБУ по этим самым
конвертам. Естественно, и эффективность НАБУ, и Сытника, и реально вынесенные
судебные дела – это отдельная история этого дорогого и бесполезного органа. Но
все-таки Поляков запустил маховик, который или при смене режима, или при
определенной конъюнктуре мог бы выстрелить. И вот теперь человек мертв, а Офис
Зеленского затирает нам, что он был наркоманом, и вообще, это был очень
несчастный случай.

\url{https://t.me/Lesev_Igor}
