% vim: keymap=russian-jcukenwin
%%beginhead 
 
%%file 23_12_2019.stz.news.ua.mrpl_city.1.iryna_matjushyna_mrpl_nash_budynok
%%parent 23_12_2019
 
%%url https://mrpl.city/blogs/view/irina-matyushina-mariupoltse-nash-budinok-i-pikluvatisya-pro-nogonasha-z-vami-vidpovidalnist
 
%%author_id demidko_olga.mariupol,news.ua.mrpl_city
%%date 
 
%%tags 
%%title Ірина Матюшина: "Маріуполь – це наш будинок і піклуватися про нього – наша з вами відповідальність!"
 
%%endhead 
 
\subsection{Ірина Матюшина: \enquote{Маріуполь – це наш будинок і піклуватися про нього – наша з вами відповідальність!}}
\label{sec:23_12_2019.stz.news.ua.mrpl_city.1.iryna_matjushyna_mrpl_nash_budynok}
 
\Purl{https://mrpl.city/blogs/view/irina-matyushina-mariupoltse-nash-budinok-i-pikluvatisya-pro-nogonasha-z-vami-vidpovidalnist}
\ifcmt
 author_begin
   author_id demidko_olga.mariupol,news.ua.mrpl_city
 author_end
\fi

\begin{quote}
\em Діяльність нашої наступної героїні є неймовірно потрібною і важливою.
Складно уявити, як склалася б доля багатьох жінок та їхніх дітей, які не
зустріли б самовіддану і щиру \textbf{Ірину Матюшину...} Це дивовижна жінка, яка має
велике серце, живий розум і добру душу. Вона по-справжньому глибока
особистість, яка обрала справою свого життя – надання допомоги і підтримки
іншим.
\end{quote}

\ii{23_12_2019.stz.news.ua.mrpl_city.1.iryna_matjushyna_mrpl_nash_budynok.pic.1}

Ірина не є корінною маріупольчанкою, оскільки народилася і виросла в Донецьку.
Її батько був шахтарем, а мама працювала в соціальній галузі. Маріуполь став її
другою домівкою в 1998 році, коли вона 20-річною дівчиною приїхала одним
вересневим днем на електричці, маючи квиток тільки в один кінець, тому що знала
що залишиться в цьому місті надовго. Проте тоді вона, звичайно, не знала
наскільки довго... Закінчивши навчання у виші з червоним дипломом (до, речі,
має два червоних дипломи) і, отримавши спеціальність економіста, дівчина хотіла
знайти справу, яка приносила б внутрішнє задоволення і стала, так би мовити,
\enquote{справою життя}.

З 2011 року Ірина розпочала реалізацію проєкту \textbf{\enquote{Маленька мама}} при
Маріупольському Благодійному Фонді \enquote{Пілігрим}. Він спрямований на допомогу
жінкам з дітьми, які опинилися в критичній ситуації. Іра разом з чоловіком
займаються реабілітацією людей з різними проблемами (з 1999 року. Проте у 2011
році сталася подія, яка дуже вплинула на подальшу діяльність Ірини. Саме того
року до них прийшла жінка з дитиною, яка жила останні пів року в Іллічівському
(сьогодні – Кальміуський) районі в покинутій машині на кладовищі. Дитина була
настільки занедбаною, що спочатку було складно зрозуміти, що це дівчинка. Жінка
потребувала в наданні високоспеціалізованої медичної допомоги. Коли Ірина
відвела її до лікарні, лікар дав зрозуміти, що жінці залишилося жити 3 місяці
максимум, тому дівчинці слід шукати родину. Для Ірини ці події та переживання
за долю жінки і її доньки стали переломним моментом. \emph{Вона пообіцяла собі, цій
жінці та Богу, що не покине її доньку та буде допомагати жінкам, які опинилися
в подібних складних життєвих обставинах з дітьми на руках..}. Доля цієї жінки
вплинула на все подальше життя Ірини. Вона удочерила дівчинку у листопаді 2011
році (жила вона в них з липня), а мами її не стало у вересні. З того часу наша
героїня займається допомогою іншим жінкам і кожного дня виконує дану собі
обіцянку. На сьогодні завдяки проєкту вдалося допомогти понад 290 жінкам,
приблизно 320 дітям і 130 переселенцям. 9 дітей допомогли повернути мамам з
інтернатно-дитбудинкової системи. З пологового будинку вдалося забрати 5
малюків разом з мамами. Наразі на утриманні проєкту перебувають 10 жінок і 16
дітей у віці від 1,5 року до 15 років. Ця важлива і потрібна ініціатива
допомагає жінкам зберегти біологічну сім'ю, отримати допомогу в скрутних
життєвих обставинах, запобігти соціальному сирітству та бродяжництву.

\ii{23_12_2019.stz.news.ua.mrpl_city.1.iryna_matjushyna_mrpl_nash_budynok.pic.2}

Гарні зміни в житті дорослих і дітей, якими опікується Ірина, – найкращі
мотиватори для неї. Часто випускники проєкту \enquote{Маленька мама} надсилають їй фото
зі своїми вже підрослими дітками... Тоді вона згадує, як одного разу забирала цю
жінку з пологового будинку і переконувала не відмовлятися від малюка. Або, як
забирала іншу жінку, промоклу разом з двомісячною дитиною, тому що вона блукала
по вулиці в надії знайти притулок... Таких історій в Ірини безліч. Вона
намагається підтримувати з усіма зв'язок і цікавиться як складається доля в
подальшому. Оскільки межі \enquote{Маленької мами} охоплюють вже не тільки Донецьку
область, дехто з учасників приїжджає здалеку, щоб ще раз висловити свою подяку
за своєчасний порятунок і трохи відпочити на морі.

Ірина пишається своєю командою, оскільки це дійсно люди, які є справжніми
альтруїстами. На їхнє надійне плече завжди можна покластися. Вони вже не тільки
команда, а й справжні друзі, частина великої і дружньої сім’ї. В команді є, як
чоловіки так і жінки. І у кожного своя унікальна історія, пов’язана зі
знайомством.

У майбутньому наша героїня планує вийти на новий рівень у розвитку проєкту, в
який входять реабілітаційний та адаптаційні центри. Думає про розвиток
соціального підприємництва всередині існуючих центрів. Безумовно, що такий
обсяг роботи неможливий без підтримки близьких і друзів. Головна людина, яка
завжди поруч – це чоловік Ірини. Для неї він найкращий радник і надійна опора.
Також у Іри багато друзів, як в Україні так і за її межами, вони завжди готові
допомогти і підтримати. З багатьма підтримує теплі відносини вже десятки років.

За роки життя в Маріуполі жінка полюбила місто і вже давно вважає його рідним
домом. Саме тут вона створила велику і щасливу сім’ю. Разом з коханим виховує
трьох чарівних доньок, які, безумовно, пишаються своєю мамою і підтримують її
діяльність.

\ii{23_12_2019.stz.news.ua.mrpl_city.1.iryna_matjushyna_mrpl_nash_budynok.pic.3}

Іра безмежно любить море, саме воно дає їй заспокоєння, надихає і прибирає
внутрішню втому. Вона з чоловіком часто гуляють вздовж моря іноді влаштовують
там невеликі сніданки. Також дуже полюбляють гулять в Міському парку.

\textbf{Улюблена книга:} \enquote{для мене найулюбленішою книгою залишається Біблія. За 25
років вона мені кожен раз відкривається по-новому і її актуальність невичерпна,
постійно знаходжу з неї нові відповіді для себе. З останніх прочитаних вразив і
не залишив байдужою \emph{роман американської письменниці  Енн Тайлер \enquote{Удочерити
Америку}}}.

\textbf{Улюблений фільм:} \emph{\enquote{Собаче серце} (1988 рік)}.

\textbf{Хобі:} 

\begin{quote}
\em\enquote{У нашій родині мій чоловік є великим колекціонером. У нас величезна
колекція годинників (настінні, камінні, наручні, кишенькові, механічні, жіночі,
чоловічі, дитячі). Влітку я люблю займатися своєю клумбою з квітами. Ще у нас є
улюблені вихованці – дві собаки: німецька вівчарка і аляскинський маламут.}
\end{quote}

\textbf{Порада маріупольцям:} 

\begin{quote}
\em\enquote{Маріупольцям дуже хочеться побажати бути активними, ніхто
крім нас не зробить наше місто кращим, чистішим, добрішим і затишнішим.
Маріуполь – це наш будинок і піклуватися про його облаштування – наша з вами
відповідальність!}
\end{quote}
