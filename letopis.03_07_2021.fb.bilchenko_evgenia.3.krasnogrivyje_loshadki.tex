% vim: keymap=russian-jcukenwin
%%beginhead 
 
%%file 03_07_2021.fb.bilchenko_evgenia.3.krasnogrivyje_loshadki
%%parent 03_07_2021
 
%%url https://www.facebook.com/yevzhik/posts/4032521036782976
 
%%author Бильченко, Евгения
%%author_id bilchenko_evgenia
%%author_url 
 
%%tags bilchenko_evgenia,poezia
%%title БЖ. Красногривые лошадки
 
%%endhead 
 
\subsection{БЖ. Красногривые лошадки}
\label{sec:03_07_2021.fb.bilchenko_evgenia.3.krasnogrivyje_loshadki}
\Purl{https://www.facebook.com/yevzhik/posts/4032521036782976}
\ifcmt
 author_begin
   author_id bilchenko_evgenia
 author_end
\fi

\noindent
БЖ. Красногривые лошадки.

\obeycr
Облака - это Божий сервер 
Облака - это русский север.
Облака - это огнь на серость:
Свет Фаворский, 
Поправший серу.
\smallskip
Облака - это гжель и алость.
Атомарность полей Эала.
Нескончаемая пиала:
Мама чай из неё пивала.
Облака - золотые линзы
На глазницах у Моны Лизы.
\smallskip
Облака - это кроль Алисы,
Тропкой скачущий к парадизу.
Облака - это голубое,
И тигриное, и любое.
\smallskip
Облака - это мы с тобою,
Занимаемые любовью.
Занимающиеся - мало:
Тело спело и перестало.
\smallskip
Вот он, Петр, - невесомый камень:
Церковь, ставшая Облаками.
\restorecr

Фото: Павел Паникин, лентикулярные облака в Саблино, \verb|#МойПитер|

\ifcmt
  pic https://scontent-mia3-1.xx.fbcdn.net/v/t1.6435-0/p526x296/210524717_4032521000116313_132971232416583712_n.jpg?_nc_cat=108&ccb=1-3&_nc_sid=8bfeb9&_nc_ohc=bgteSA5Ap7gAX8uZ8is&_nc_ht=scontent-mia3-1.xx&tp=6&oh=20c2fdfffaea71a3f28c4746e2a99bdf&oe=60E6AC08
  width 0.4
\fi
