% vim: keymap=russian-jcukenwin
%%beginhead 
 
%%file 11_09_2022.stz.news.ua.donbas24.1.potjag_do_peremogy.0.intro
%%parent 11_09_2022.stz.news.ua.donbas24.1.potjag_do_peremogy
 
%%url 
 
%%author_id 
%%date 
 
%%tags 
%%title 
 
%%endhead 

Героїзм українців у боротьбі з військами країни-окупанта художники відобразили
у малюнках на вагонах \enquote{Потягу до Перемоги}.
\href{https://donbas24.news/news/ukrzaliznicya-prisvyatila-potyag-okupovanim-mistam-cim-vin-osoblivii-foto}{Про
це раніше писав Донбас24}.\footnote{%
\href{https://donbas24.news/news/ukrzaliznicya-prisvyatila-potyag-okupovanim-mistam-cim-vin-osoblivii-foto}{\enquote{Укрзалізниця} присвятила потяг окупованим містам: чим він особливий (ФОТО), Веремєева Тетяна, donbas24.news, 24.08.2022}
}

Зараз стало відомо про зміну маршруту \enquote{Потяга до Перемоги}: він слідуватиме із
Запоріжжя до Львова. Сім вагонів, розписаних українськими митцями, мають в
ілюстраціях сакральний зміст і сенс, адже кожен вагон присвячений окремій
тимчасово окупованій території України та подвигам людей, які чинять опір
російським військам. Про малюнки на кожному вагоні Донбас24 розповість в цьому
матеріалі.

\ii{11_09_2022.stz.news.ua.donbas24.1.potjag_do_peremogy.pic.1}

\enquote{Потяг до перемоги} — спільна ініціатива \enquote{Укрзалізниці}, комунікаційної агенції
\href{https://www.instagram.com/grestodorchuk}{Gres Todorchuk} та сучасних українських митців під кураторством 
\href{https://www.facebook.com/kateryna.taylor.5}{Каті Тейлор}. 

\ifcmt
  ig https://gcdnb.pbrd.co/images/Cdt94ZSVO2Ht.png?o=1
  @wrap center
  @width 0.9
\fi

\begin{leftbar}
	\begingroup
		\bfseries
Олександр Камишін, голова правління \enquote{Укрзалізниці}:— \enquote{Потяг до перемоги}
— це наша подяка всім сміливим українцям, які чинять спротив окупантам.
Мета проєкту — показати, що Україна пам'ятає про тих, хто знаходиться в
тимчасовій окупації, що ми не забули ні про Херсонщину, ні про
Донеччину, ні про Крим. З 23 серпня потяг курсував маршрутом Київ—
Ужгород, а з 4 вересня курсуватиме від Запоріжжя до Львова. І ми
впевнені, що одного дня він прибуде на вокзали тих міст, які поки що
окуповані.
	\endgroup
\end{leftbar}

\ii{11_09_2022.stz.news.ua.donbas24.1.potjag_do_peremogy.pic.2}

