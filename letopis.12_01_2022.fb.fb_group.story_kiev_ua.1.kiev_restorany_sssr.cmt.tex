% vim: keymap=russian-jcukenwin
%%beginhead 
 
%%file 12_01_2022.fb.fb_group.story_kiev_ua.1.kiev_restorany_sssr.cmt
%%parent 12_01_2022.fb.fb_group.story_kiev_ua.1.kiev_restorany_sssr
 
%%url 
 
%%author_id 
%%date 
 
%%tags 
%%title 
 
%%endhead 
\zzSecCmt

\begin{itemize} % {
\iusr{Tetyana Kilesso}

Цукерки \enquote{Надія} я ніколи не куштувала. Відбивні з свинини так, замовляла вцьому
ресторані. А зараз замовила б заливну осетрину з лимоном.


\iusr{Светлана Манилова}

В упоминаемый Вами период в рестораны я еще не ходила. @igg{fbicon.smile} Впервые в \enquote{Столичном}
оказалась в новогоднюю ночь 1988 года и второй раз в июне 1989 года, когда
отмечали с группой окончание университета. Впечатления хорошие, как и
обслуживание с кухней.

\iusr{Вадим Ляшенко}

Многие блюда сейчас у нас в диковину.

\iusr{Светлана Ветрова}

Приємні спогади цін за порції можна було спокійно посидіти со спиртним на
червонец на двох але треба туди попасти а це іноді було скрутно

\iusr{Вадим Ляшенко}
 @igg{fbicon.thumb.up.yellow} 

\iusr{Катя Бекренева}

Не часто посещала рестораны) В основном на свадьбах) По хронологии: Пектораль,
два раза Краков, Млын (защита), Верховина, Украина (просто так), Братислава,
два раза на юбилеях, Столичный (просто так)

Это советское время. Понравилась кухня в Млыне.

\begin{itemize} % {
\iusr{Света Медецкая}
\textbf{Катя Бекренева} Верховина, которая на Павловской была???)))

\iusr{Катя Бекренева}
\textbf{Света Медецкая} нет, на 5-ой просеке, в Святошино
\end{itemize} % }

\iusr{Ирина Таировна}
Крутой был ресторан на то время,
Говорят,у директора ресторана сОбака ходила с золотым ошейником.

\iusr{Катя Бекренева}
\textbf{Ирина Таировна} и с бриллиантами) Шефиня выходила её выгуливать по ночам. Громкое дело было.

\iusr{Ирина Озерянская}

нас с чоловіком не пустили до цього ресторану, бо були у джинсах  @igg{fbicon.smile}  1981р.

\iusr{Таня Сидорова}

На мене з меню ні чого не справило враження. А інтер'єр був дуже простий.

\iusr{Басанько Любов}
Я была в ресторане \enquote{Столичном} после окончания института.
Это был 1965 год.
Сдавали, как сейчас помню, по 10 рублей.
Меню, конечно, не помню, но помню, что стол ломился от еды.
А как было весело!
Мы же тогда были совсем юные и беззаботные. Эх...

\iusr{Алла Вайнерман}
Наверное, нам так уже не жить!

\begin{itemize} % {
\iusr{Ольга Красюк}
\textbf{Алла Вайнерман} Конечно, не жить) мы ведь немножко стареем) в юности все иначе) но у каждого времени своя прелесть)

\iusr{Алла Вайнерман}
\textbf{Ольга Красюк}. Я о ценах, а вы о возрасте.
\end{itemize} % }

\iusr{Михайлина Голуб}

\enquote{Столичный} мы с друзьями иногда посещали в студенческие годы - 68-72-ой. Да,
студенты иногда могли себя это позволить. Цены там были демократичнее, чем,
например, в \enquote{Динамо} или \enquote{Театральном}. Правда, если честно сказать
обслуживание и кухня соответствовали этой демократичности. И еще в этом
ресторане в оркестре играл на трубе мой бывший однокласник Гриша. Вообще-то он
с блеском окончил музыкальную школу по класу скрипки, но она ему надоела. Он
окончил эстрадно-цирковое училище - разговорный жанр. По окончанию весь их курс
ра пределили в Архангельск. Там он с успехом выступал, но все же захотел
вернуться в Киев. А тут оказалось, что артисты-разговорники не нужны. А семью
кормить нужно было. Он очень быстро освоил трубу (вообще он играл на многих
инструментах) и стал, как тогда говорили, рестораным лабухом. Мы с
одноклассниками иногда приходили в \enquote{Столичный}, чтобы послушать его. А потом он
уехал за океан и в \enquote{Столичный} мы больше не ходили. И все-таки было жалко,
когда его здание снесли.

\begin{itemize} % {
\iusr{Ольга Красюк}
\textbf{Михайлина Голуб} Добрая история... жизненная..

\iusr{Михайлина Голуб}
\textbf{Ольга Красюк}  @igg{fbicon.grin}  @igg{fbicon.face.grinning.big.eyes}  @igg{fbicon.face.grinning.smiling.eyes}  @igg{fbicon.beaming.face.smiling.eyes}  @igg{fbicon.laugh.rolling.floor} 
\end{itemize} % }

\iusr{Ігор Кулаков}
Завжди всміхаюся, коли бачу підпис калькулятора

\iusr{Ed Edward}
Ресторан «Лейпциг».........?

\iusr{Наташа Гресько}

иногда с однокусниками обедали в Столичном: там были самые дешевые \enquote{комплексы}.
хотя обычно так далеко не забирались - ходили в Ленинградскую. или в Украину -
если \enquote{при деньгах}) а еще, там же Крыжинка была рядом!


\iusr{Катя Бекренева}

А вот самые вкусные и дефицитные конфеты нам приносила родственница, она
работала в ателье \enquote{Коммунар}. В открытой продаже таких я не встречала.


\iusr{Viktor Kozhevnikov}

В Столичном был фирменный салат и напиток Троянда. На основе его меню написали
книгу \enquote{современная украинская кухня}.

\begin{itemize} % {
\iusr{Леонид Красиловский}
\textbf{Viktor Kozhevnikov} да салат Столичный- типа Оливье но вместо колбасы и гороха было мясо и вроде как пекинской капусты ( которую ещё наверное не выращивали...)))
\end{itemize} % }

\iusr{Инна Валентиновна}

Мне все блюда в \enquote{Столичном} нравились: заливное, котлета по - киевски,
салат столичный и т. д. А что его снесли, я считаю преступлением перед нашей
памятью.  Никому он не мешал, разве что делкам строительного бизнеса.

\begin{itemize} % {
\iusr{Николай Гребенкин}
\textbf{Инна Валентиновна} 

А плюс к тому, это здание закрывало Крещатик от холодных северных ветров. Это
здание было щитом Крещатика @igg{fbicon.face.smiling.hearts} 

\iusr{Инна Валентиновна}
\textbf{Николай Гребенкин} согласна 100\%.
\end{itemize} % }

\iusr{Марина Набока}

конфеты \enquote{Надежда} даже не помню, а \enquote{птичье молоко} люблю до сих пор. Ресторан
Столичный тоже помню, неоднократно бывала и с друзьями, и просто поесть, на
Крещатике не сильно много было доступных мест общепита, а в Столичном часто
были банкеты и ресторан переставал быть доступным. Такой себе ресторан средней
руки, но нормально поесть можно было

\begin{itemize} % {
\iusr{Лариса Павловская}
\textbf{Marina Naboka} Я часто заходила в столовку на первом этаже, а в самом ресторане у нас свадьба была

\iusr{Марина Набока}
\textbf{Лариса Павловская} во-во, это было популярное место, я там тоже на чьей-то свадьбе была, уже и не помню чьей.

\iusr{Лариса Павловская}
\textbf{Marina Naboka} Жаль, что нет его уже
\end{itemize} % }

\iusr{Ольга Dzhun}

Обратитесь в ресторан Будьмо, руководство этого ресторана было у истоков
ресторанного дела в Киеве, они вам, думаю, помогут

\iusr{Анна Земко}

У нас був випуск інститутській у Столичному в червні 1990 року) Меню взагалі не
пам'ятаю( Памятаю якісь дивани півколами навколо столів (а може теж плутаю  @igg{fbicon.grin} )

\iusr{Надежда Коваленко}

А мы в \enquote{Столичном} свою свадьбу справляли.

Тогда, а это 27 апреля 1985 год, удивлялись в ресторане, что мы сами себе
торжество заказывали, без родителей. Сами платили, сами выбирали блюда. Хотя до
этого не посещали рестораны. Была мечта... вот такая, наивная и простая.

Там 3 зала было и общий танцпол. Оркестранты сказали, что мы самая красивая
пара ....Было очень приятно. А еще, нам предложили выбрать любую песню, чтоб
они для нас сыграли бесплатно, честь такой красивой пары.

Мы были молодыми и счастливыми... и не думали, что счастье долгим не бывает...
Чернобыль и все изменится

\begin{itemize} % {
\iusr{Ирина Мундирова}
\textbf{Надежда Коваленко} И у нас такая же история 02. 09. 1984 г.

\iusr{Людмила Зобенко}
\textbf{Надежда Коваленко} А какую же вы выбрали песню?

\iusr{Надежда Коваленко}

Я отдала предпочтение на выбор песни своей мвме, ту, которая ей на тот момент
нравилась.

Прошло столько лет, и не важна песня, важнее те эмоции, что мы все чувствовали
в тот момент, в тот день, в то время.... они/чувства/ незабываемы.

Все были живы, счастливы и молоды! А эмоции, песни тех лет - были прекрасны!
\end{itemize} % }

\iusr{Григорий Владимирович}

В 70- х участвовал в разработке \enquote{Дело Ёлкиной} ( Шахиня) в ресторане
Столичный.

\begin{itemize} % {
\iusr{Zoya Pshenichny}
\textbf{Григорий Владимирович} 

Думаю мало тех, кто помнит об этом деле.

А вообще классный был ресторан, у меня там был выпускной техникума.

Соседка рабо ала в кафе на первом этаже, кассиршей.

\begin{itemize} % {
\iusr{Larisa Kodryansky}
\textbf{Zoya Pshenichny} 

У меня была свадьба в кафе на первом этаже в 1972 году. Спустя год или два, уже
не помню, моего папу вызывали по делу \enquote{шахини}.

\iusr{Zoya Pshenichny}
\textbf{Larisa Kodryansky} 

Тогда многих вызывали свидетелями, особенно у кого были банкеты.

В музее МВД после дела \enquote{Шахини}, были выставлены конфискованные у нее вещи,
драгоценности.

\iusr{Larisa Kodryansky}
\textbf{Zoya Pshenichny} 

Нервы тогда потрепали изрядно. Я уже и не помню, чем закончилось дело \enquote{шахини}.

\end{itemize} % }

\iusr{Катя Бекренева}
\textbf{Григорий Владимирович} 

Я в году так 77-78 читала книгу, мне было тогда 15, не помню чьи воспоминания
об уголовных делах, написал кто-то из следователей, там было и дело Шахини
(шефини), врезалось в память.

\iusr{Надежда Владимир Федько}
\textbf{Григорий Владимирович} Пам'ятаю)

\end{itemize} % }

\end{itemize} % }
