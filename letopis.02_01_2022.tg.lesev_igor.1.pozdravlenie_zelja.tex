% vim: keymap=russian-jcukenwin
%%beginhead 
 
%%file 02_01_2022.tg.lesev_igor.1.pozdravlenie_zelja
%%parent 02_01_2022
 
%%url https://t.me/Lesev_Igor/259
 
%%author_id lesev_igor
%%date 
 
%%tags zelenskii_vladimir,pozdravlenie,novyj_god,ukraina,putin_vladimir
%%title По-прежнему под впечатлением от поздравления Блогера
 
%%endhead 
\subsection{По-прежнему под впечатлением от поздравления Блогера}
\label{sec:02_01_2022.tg.lesev_igor.1.pozdravlenie_zelja}

\Purl{https://t.me/Lesev_Igor/259}
\ifcmt
 author_begin
   author_id lesev_igor
 author_end
\fi

По-прежнему под впечатлением от поздравления Блогера. Осознание, что наш
президент иплан галактического масштаба не проходило ни под алкогольным
изливанием, ни под 2-ярварным похмельем.

Еще раз. Новогоднее поздравление, как и любое другое поздравление – с 8 мартам,
днем независимости и чем угодно другим – это дежурно-обыденный атрибут
политика. Вот как поздороваться при виде коллеги по офису. «Привет». И всё. Не
поздороваться/поздравить с праздником нельзя, потому что сочтут хамом. Но и
театрально вкладываться в это, чтобы Станиславский непременно поверил,
натуральное безумие.

Но у нашего дурака свое измерение реальности. 22 минуты пафоса с
постановочно-декоративным сидением «простых украинцев», где были военные,
спортсмены и доярки – это именно то, что выудили у Советского Союза и затем
назвали это совком. У Зеленского и его компашки полностью отсутствует чувство
меры. Они ебанулись окончательно. Пацанва живет в виртуальном мире, где
публичная политика – это найти 50 оттенков слова «привет».

Я уже писал о новогоднем поздравлении Путина. Оно идеальное. Я слова не мог
вспомнить из того поздравления сразу же, как оно завершилось. Пришел к столу,
что-то сказал дежурное и ба-бах – бой курантов и все уже забыли о том Путине,
потому что у каждого начался свой празднично-семейный концерт. Вот за это мы
все и любим Новый год. Минимум политики. Минимум навязчивости из телека. Потому
что, чем больше ИХ, тем больше наше раздражение. У Путина это как-то понимают,
а потому он и не залазит из телека в каждую кастрюльку с оливье.

Зато этого не понимали у Порошенко. Он раздражал. И не только потому, что я не
разделял его расистскую политику. Персонаж был лживым и искусственным. Он врал
уже со слова «привет». А потому и слова врунишки не воспринимались. К тому же,
эти слова всегда были заполитизированы. А это всегда неуместно к праздничному
столу.

Но Зеленский даже эту гадость сумел переплюнуть. Полное отсутствие реальности,
меры, адекватности. На дворе 2 января. Столько выпито и съедено, что просто
жуть. Уже все переварено и протрезвлено. А за персонажа все равно стыдно.
Уникальный дурак во всех своих начинаниях.
