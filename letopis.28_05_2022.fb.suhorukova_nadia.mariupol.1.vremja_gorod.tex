% vim: keymap=russian-jcukenwin
%%beginhead 
 
%%file 28_05_2022.fb.suhorukova_nadia.mariupol.1.vremja_gorod
%%parent 28_05_2022
 
%%url https://www.facebook.com/permalink.php?story_fbid=2422141691426285&id=100008914326381
 
%%author_id suhorukova_nadia.mariupol
%%date 
 
%%tags mariupol
%%title С 24 февраля в нашем городе было особенное время. Мариупольское
 
%%endhead 
 
\subsection{С 24 февраля в нашем городе было особенное время. Мариупольское}
\label{sec:28_05_2022.fb.suhorukova_nadia.mariupol.1.vremja_gorod}
 
\Purl{https://www.facebook.com/permalink.php?story_fbid=2422141691426285&id=100008914326381}
\ifcmt
 author_begin
   author_id suhorukova_nadia.mariupol
 author_end
\fi

\#мариуполь \#надежда Путь с пятого на первый во время бомбежки гораздо длиннее
полета на луну и обратно. С 24 февраля в нашем городе было особенное время.
Мариупольское. Оно практически не двигалось. Унылое и страшное время.
Бесконечный ад. 

\ii{28_05_2022.fb.suhorukova_nadia.mariupol.1.vremja_gorod.pic.1}

Иногда мы друг у друга спрашивали:  \enquote{Который час?} Как будто была
какая-то разница. Через две недели войны я путала дни недели и числа. Холодный
март лютовал как февраль. Мне казалось началась вечная зима. 

\ii{28_05_2022.fb.suhorukova_nadia.mariupol.1.vremja_gorod.pic.2}

В сутках было три времени  - утро, день и ночь. Страшнее всего было ночью. Она
в Мариуполе наступала сразу, как только темнело.  Тогда  начинались
нескончаемые  обстрелы. 

Мне кажется рашисты изводили город бомбежками и брали его измором. Наверное,
они думали,  если не дадут нам спать ни минуты, мы куда-нибудь исчезнем. Не
будем бегать, как запуганные мыши днём  по улицам в поисках воды и еды. 

\ii{28_05_2022.fb.suhorukova_nadia.mariupol.1.vremja_gorod.pic.3}

А может быть они просто развлекались. Как в компьютерной игре. Как будто город,
который они разрушали, многоэтажки в которые постоянно лупили ракетами, люди,
которых бомбили с самолетов -  ненастоящие. 

Мы для них были человечками в компьютерной игре. Они нас убивали и
посмеивались.  Им было плевать, что мы живые, нам страшно, нам больно, мы
пытаемся спастись. 

\ii{28_05_2022.fb.suhorukova_nadia.mariupol.1.vremja_gorod.pic.4}

Так стрелять по живым людям, как это делали рашисты, могут только монстры.
Каждый день они знакомили нас с  новым видом  оружия. Разбивали вдребезги
тысячи жизней. 

Недавно смотрела видео. Двое мужчин снимали из машины центр Мариуполя.  От
школы искусств и до Торговой. Один спросил: \enquote{Тут что кто-то выжил? Как
тут вообще можно было выжить?} 

Я тоже не понимаю, как мы  выжили? Как это вообще было возможно?

% 5,6,7
\ii{28_05_2022.fb.suhorukova_nadia.mariupol.1.vremja_gorod.pic.5}

Знаете, как рашисты затягивали петлю  на горле города? Они шли с разных сторон.
Мы сидели в своем района, слушали адские  звуки  и с ужасом  ждали, когда   они
подберутся совсем близко. 

% 8
\ii{28_05_2022.fb.suhorukova_nadia.mariupol.1.vremja_gorod.pic.6}

Мы знали, что  бомбят и обстреливают улицы рядом. Видели как по проспекту Мира
проехали танки с буквой зет на боку.  Дом через два перекрестка    заснул
вечным сном. Люди говорили, он сложился от авиаудара.   Под его  завалами
погибли не все. 

На поверхность доносились их стоны и крики. Достать их было некому. Перед этим
обстреляли базу спасателей.  Мертвые  лежали на асфальте возле здания ГСЧС. В
бомбоубежище рядом -  прятались  живые. 

Самые страшные ужасы рассказывал Витя. Сосед людей,  которые нас с мамой
приютили. Витя ходил  по городу и видел все своими глазами. Мертвые тела на
Черемушках, воронка, ставшая могилой для двухэтажного дома, оторванные
конечности.

Мы не хотели этого слышать. Если  укроешься с головой одеялом, то беда тебя не
заметит и пройдет мимо. Мы закрывались одеялом от войны. 

Мы понимали, что все очень плохо, но говорили себе: на нас не упадет ни одной
ракеты или бомбы. Все, что происходит снаружи - дурной сон. Война закончится
раньше. Орки не успеют приблизиться вплотную к нашему дому. 

Петля затянулась очень быстро.  Ракета попала в крышу дома, снаряды лупили по
всем   многоэтажкам  вокруг. Самолёты рассекали небо над нашим домом  и
наносили авиаудары. От них невозможно было спрятаться. 

Даже если бомба падает через дорогу, дом  откликается на удар  дрожанием и
плаваньем стен. Как будто гигантская  гитарная струна вибрирует и рвется, а
внутри тебя все замерзает. 

Эти фото из разных районов Мариуполя. Мой город теперь такой.

Сегодня оккупанты  продолжают бомбить Украину и убивать людей в их домах.

%\ii{28_05_2022.fb.suhorukova_nadia.mariupol.1.vremja_gorod.eng}

%\ii{28_05_2022.fb.suhorukova_nadia.mariupol.1.vremja_gorod.orig}
\ii{28_05_2022.fb.suhorukova_nadia.mariupol.1.vremja_gorod.cmt}
%\ii{28_05_2022.fb.suhorukova_nadia.mariupol.1.vremja_gorod.cmtx}
