% vim: keymap=russian-jcukenwin
%%beginhead 
 
%%file 15_10_2019.stz.news.ua.mrpl_city.1.oleksandr_magdalyc
%%parent 15_10_2019
 
%%url https://mrpl.city/blogs/view/mariupolets-oleksandr-magdalits-napoleglivo-jdit-do-svoei-mrii
 
%%author_id demidko_olga.mariupol,news.ua.mrpl_city
%%date 
 
%%tags 
%%title Маріуполець Олександр Магдалиць: "Наполегливо йдіть до своєї мрії!"
 
%%endhead 
 
\subsection{Маріуполець Олександр Магдалиць: \enquote{Наполегливо йдіть до своєї мрії!}}
\label{sec:15_10_2019.stz.news.ua.mrpl_city.1.oleksandr_magdalyc}
 
\Purl{https://mrpl.city/blogs/view/mariupolets-oleksandr-magdalits-napoleglivo-jdit-do-svoei-mrii}
\ifcmt
 author_begin
   author_id demidko_olga.mariupol,news.ua.mrpl_city
 author_end
\fi

\ii{15_10_2019.stz.news.ua.mrpl_city.1.oleksandr_magdalyc.pic.1}

Наступний наш герой має всі якості справжнього чоловіка. Він благородний,
високоморальний, самовідданий і чесний. \textbf{Олександр Магдалиць} вражає своєю
шляхетністю і лицарською поведінкою. Він є самодостатньою людиною, за плечима
якої великий життєвий досвід, а у серці – надія побудувати світле майбутнє.

Олександр народився в Маріуполі. Виростила його одна мама, яка виховувала Сашка
на цікавих і повчальних сімейних історіях. Мати була військовою і звикла
самотужки боротися з усіма життєвими труднощами. Вона працювала в
науково-дослідницьких судах, пізніше – в пологовому будинку. Водночас піднімала
одна сина. Олександра надихає приклад матері, завдяки якій він зрозумів, що у
житті головне – наполегливість і бажання працювати. До речі, працювати хлопець
почав у досить юному віці – з 13 років. Сашко отримав дві вищі освіти: юридичну
та фінансову. Завдяки спеціальній програмі маріуполець скористався унікальною
можливістю і почав працювати у виконкомі керівником відділу з охорони земель.
Пів року він працював там.

\ii{15_10_2019.stz.news.ua.mrpl_city.1.oleksandr_magdalyc.pic.2}

У 2013 році, коли назрівав воєнний конфлікт, вирішив, що слід опанувати навички
першої допомоги, тож разом із друзями пройшов навчання в \textbf{Товаристві Червоного
Хреста}. Під час військових подій наш герой не зміг залишатися осторонь. 9
травня 2014 року почалися розстріли. Олександр пам'ятає, як відтягував
поранених до машин \enquote{швидкої}. Намагався допомагати від Червоного Хреста
цивільним, які постраждали від воєнних дій. Бійцям часто допомагали просто від
себе, адже хлопцям не вистачало найнеобхіднішого. Збирали і відвозили, що
могли, аби ті були одягнені й не голодували. Згодом Олександр почав
співпрацювати з \textbf{Благодійним фондом \enquote{Карітас Маріуполь}}. Найбільше у нелегкий
час рухатися вперед допомагала пам'ять про сім'ю.

\textbf{Читайте також:} \emph{Волонтерство и благотворительность. Как отличить благотворителя от мошенника}%
\footnote{Волонтерство и благотворительность. Как отличить благотворителя от мошенника, Народный блогер, mrpl.city, 30.08.2019, \par%
\url{https://mrpl.city/blogs/view/volonterstvo-i-blagotvoritelnost-kak-otlichit-blagotvoritelya-ot-moshennika}
}

\ii{15_10_2019.stz.news.ua.mrpl_city.1.oleksandr_magdalyc.pic.3}

Однак Олександр допомагав не тільки людям. Завдяки його великому серцю і
притаманній небайдужості до всього живого він неодноразово рятував братів наших
менших. Зокрема, в одній із поїздок Сашко побачив собаку, якого контузило, і
він доживав своє на позиції біля Широкиного. Хлопець виміняв Лакі на два пакети
макаронів і відвіз у Маріуполь, де його врятувала група спеціалістів. Відтоді
маріуполець часто привозив врятованих тварин з зони бойових дій, що посприяло
створенню \textbf{притулку для постраждалих тварин \enquote{Чіп і Дейл}}. Притулок існував
завдяки коштам небайдужих українців і був закритий, коли прилаштували
останнього мешканця. Для Олександра зустрічі з врятованими тваринами завжди
приносили суцільну радість і були справжньою віддушиною.

\ii{15_10_2019.stz.news.ua.mrpl_city.1.oleksandr_magdalyc.pic.4}

Наразі Олександр працює лобістом благодійної організації \enquote{Ка\hyp{}рітас
Маріуполь}.  Ця діяльність подобається маріупольцю, адже цілком відповідає його
інтересам.  Робота найчастіше спрямована на досягнення мети, це може бути
ухвалення необхідного законопроекту або ж, навпаки, скасування якихось
нормативно-правових актів. На відпочинок зовсім не вистачає часу, адже
лобіювання вимагає постійного самовдосконалення. Зокрема, зараз Сашко
намагається осягнути швидкочитання, тому взяв за правило читати нову книгу
кожні 2 тижні. Також, оскільки дуже важливо навчитися стисло і швидко викладати
власні думки, у нагоді став копірайтинг. У лобістів творчість перетинається з
юриспруденцією. Головне – отримати згоду. Олександр пройшов курси з лобіювання
у \textbf{Олега Брагінського}, який має школу на території колишнього СНД і в
штатах.

\ii{15_10_2019.stz.news.ua.mrpl_city.1.oleksandr_magdalyc.pic.5}

Олександр дуже сподівається найближчим часом створити міцну сім'ю і стати
гарним батьком. А поки що найкращим відпочинком вважає відвідування лазні зі
справжніми однодумцями і заняття спортом. В людях найбільше цінує розум,
самодостатність і добре серце. Також наш герой дуже любить Азовське море, яке
завжди вражає своєю чарівністю і заспокоює від швидкого темпу життя.

\textbf{Порада маріупольцям:} 

\begin{quote}
\em\enquote{Йти до своєї мрії наполегливо! Не здаватися. Пам’ятайте, на все вистачить сил}. 
\end{quote}

\textbf{Читайте також:} \emph{Как начать закаляться и удивляться своему здоровью уже через пару месяцев: собственный пример}%
\footnote{Как начать закаляться и удивляться своему здоровью уже через пару месяцев: собственный пример, Народный блогер, %
mrpl.city, 28.01.2019, \par%
\url{https://mrpl.city/blogs/view/kak-nachat-zakalyatsya-i-udivlyatsya-svoemu-zdorovyu-uzhe-cherez-paru-mesyatsev-sobstvennyj-primer}
}
