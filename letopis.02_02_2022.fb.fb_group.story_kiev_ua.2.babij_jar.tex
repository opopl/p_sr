% vim: keymap=russian-jcukenwin
%%beginhead 
 
%%file 02_02_2022.fb.fb_group.story_kiev_ua.2.babij_jar
%%parent 02_02_2022
 
%%url https://www.facebook.com/groups/story.kiev.ua/posts/1872804149583044
 
%%author_id fb_group.story_kiev_ua,kirkevich_viktor.kiev
%%date 
 
%%tags __feb_2022.vtorzhenie
%%title Вчера ракетами стреляли по Бабьему Яру!
 
%%endhead 
 
\subsection{Вчера ракетами стреляли по Бабьему Яру!}
\label{sec:02_02_2022.fb.fb_group.story_kiev_ua.2.babij_jar}
 
\Purl{https://www.facebook.com/groups/story.kiev.ua/posts/1872804149583044}
\ifcmt
 author_begin
   author_id fb_group.story_kiev_ua,kirkevich_viktor.kiev
 author_end
\fi

\ii{02_02_2022.fb.fb_group.story_kiev_ua.2.babij_jar.pic.1}

На ленте своего друга Миши прочитал, что у него возникла ассоциация. О ней мне
писала и моя коллега Лена несколько дней тому назад. 2 марта 1953 года,
консилиум перепуганных медицинских светил диагностировал у руководителя СССР
кровоизлияние в мозг. Населению официально об этом сообщили только через трое
суток. Во многих фильмах и книгах появляются убедительные версии, что диктатору
«помогли» умереть его ближайшие помощники, в частности из-за его жестокой
политики и репрессий. В первую очередь против евреев. Это убийство руководства
Антифашистского еврейского комитета, Дела врачей и подготовки депортации всех
без исключения евреев с Европейской части страны в Сибирь. Об этом я подробно
пишу в своей книге «Продолжение «эсэсэсэрости» в Киеве». Политическим
преследованиям и «высшей мере» подвергались все народы СССР, особенно
украинские «националисты», за которыми гонялись по всей Западной Украине.
Правительственное окружение параноидного Сталина боялось и за себя!

Я неспроста начал с евреев. Вчера ракетами стреляли по Бабьему Яру! Поэтому
хочу напомнить, что есть еврейский праздник Пурим. Весёлый праздник победы над
врагом! Именно в Пурим 1953 года издал последний вздох Сталин… В этом году
праздник 17 марта. Даст Бог к празднику сдохнет Адольф Путинг, из-за
параноидной политики против многих народов мира. В одной только Сирии, погибли
тысячи россиян.

Безусловно, мы должны надеяться в первую очередь на собственные силы, на
присущую жителям Украины выдержку, на мужество наших доблестных защитников и
мудрость наших дипломатов.

Мы победим! Слава Украине!
