% vim: keymap=russian-jcukenwin
%%beginhead 
 
%%file 09_11_2020.news.ua.strana.1.mova_dnepr_professor
%%parent 09_11_2020
 
%%url https://strana.ua/news/299846-jazykovoj-skandal-v-dnepre-za-chto-uvolili-professora-hromova-iz-dpi.html
%%author 
%%tags 
%%title 
 
%%endhead 

\subsection{Казус профессора Громова. Как в Днепре за лекцию на русском языке довели до увольнения ученого}
\label{sec:09_11_2020.news.ua.strana.1.mova_dnepr_professor}
\Purl{https://strana.ua/news/299846-jazykovoj-skandal-v-dnepre-za-chto-uvolili-professora-hromova-iz-dpi.html}

\ifcmt
img_begin 
	url https://strana.ua/img/article/2998/46_main.jpeg
	caption Вуз в Днепре оказался в центре языкового скандала 
	width 0.7
img_end
\fi

В Днепре произошел языковой скандал. Руководство вуза "Днепровская политехника"
подвело под увольнение одного из самых заслуженных своих профессоров - философа
Валерия Громова.

Он читал лекцию на русском языке, но это возмутило одну из студенток, которая
подала официальную жалобу. После чего вуз бросился оправдываться, а профессор
написал заявление об отставке. 

И это только начало работы норм законов о тотальной украинизации, принятых
прошлой властью и не отмененных новой. 

"Страна" собрала подробности мовного скандала в Днепре.

\subsubsection{\enquote{Два языка - одна нация}}

В центре скандала оказался профессор Национального технического университета
"Днепровская политехника" Валерий Громов. Широкую огласку инцидент получил
после публикации в группе Facebook "Батьки SOS" 5 ноября (хотя само событие
произошло в 20-х числах октября).

Житель Днепра по имени Александр написал, что его дочь учится в "Днепровской
политехнике". Ее преподаватель с кафедры философии начал читать лекцию
по-русски, но студентка с места потребовала исполнять закон об образовании, где
предписано подавать материал на госязыке.

\begin{itemize}
\item - Можно на украинском, пожалуйста?
\item - Нет, я читаю по-русски.
\item - Согласно закону вы можете преподавать на украинском, вы должны.
\item - Если вам, вас не устраивает, пишите на меня жалобу.
\item - Хорошо.
\item - Два языка - одна нация, один - руины, а не Украина.
\end{itemize}

\Purl{https://youtu.be/d3r00K0Kg4s}

После этих слов студентка вышла из аудитории и написала жалобу на имя ректора.
Когда на следующий день на пару пришла комиссия и начали разбирать эту
ситуацию, четыре студента выступили в поддержку русского языка. В поддержку
украинского языка выступил только один студент. Остальные промолчали.

"Прошел месяц, а решения еще нет. Единственное известно, что его отстранили от
преподавания. Прошу огласки, ведь преподавателю может все сойти с рук!! Требуем
увольнения!!" - сообщил автор поста Александр.

\ifcmt
pic https://strana.ua/img/forall/u/0/34/%D0%A1%D0%BD%D0%B8%D0%BC%D0%BE%D0%BA(353).JPG
\fi

К сообщению отец студентки прикрепил фото преподавателя.

\ifcmt
img_begin 
	url https://strana.ua/img/forall/u/0/34/124021453_197724945198599_5568296798347047679_n.jpg
	caption Профессор Валерий Громов
	width 0.7
img_end
\fi

Позже выяснилось, что жалобу написала студентка София Лисковская. Ее начали
наперебой пиарить многие СМИ.

"Попросила, чтобы он преподавал на украинском языке, на что он сказал мне
категорическое "нет, я буду читать лекцию только на русском языке". Затем он
начал пропагандировать русский след, он говорил, что два языка - одна нация. И
что один (язык - Ред.) - руина, а не Украина, и мы наблюдаем это сейчас. Это
было большое возмущение. Я знаю многих людей, которые на войне защищают
Украину. И такое высказывание его... Они плюнули в лицо нам, сознательным
украинцам", - дала комментарий девушка.

\Purl{https://youtu.be/sGfJdoqfMu0}

\subsubsection{"Языковая политика должна быть другой". Позиция профессора}

Профессор Валерий Громов, который отказался читать лекцию на украинском языке,
дал комментарий СМИ. 

Он пояснил, что в "Днепровской политехнике" читал курсы: "Немецкая
идеалистическая философия", "Метафизика", "Эстетика", "Античная философия",
"Древневосточная философия", "Культура Древнего Востока" и "Ценностные
компетенции специалиста".

"Все это время, а в университете я проработал 27 лет, у меня не было проблем с
преподаванием материала на русском языке даже с теми студентами, которые
предпочли бы слушать преподавание на украинском языке. Мой украинский язык не
настолько безупречен, чтобы столь сложный материал по философии я мог излагать
по-украински. Читать философию, даже просто стараясь держаться интеллектуальной
стези какого-нибудь великого философа, значит мыслить, напрягаться, понимать",
- пояснил Громов.

Он отметил, что отстаивает русский язык в стране, где половина населения
говорит по-русски (хотя, судя по недавнему исследованию социальных сетей -
гораздо больше).

"На мой взгляд, хотя я хорошо осведомлен, насколько по-разному думают люди, в
Украине было бы гораздо больше ментального благополучия, если бы языковая
политика была другой. Моя точка зрения состоит в том, что украинский и русский
языки --- это наши языки, а не так, что украинский ваш, а русский наш. На
культуру независимой Украины могут плодотворно потрудиться все, нужно только
иметь добрую волю в духе универсальных моральных ценностей идти навстречу друг
другу.

Людям, сеющим раздоры и вскармливающим межнациональную рознь, кажется, что в их
жизни становится больше смысла и даже духовности по причине непримиримых
убеждений. Но в ХХІ веке, после столетий кровавой истории Европы, можно с
уверенностью сказать, что это путь в никуда. В конце концов, чему нас учили
Будда, Кришна, Христос и Мухаммед? Разве не учили они нас любви?", - говорит
профессор.

\subsubsection{"Сам уволился". Что говорят в вузе}

В Днепровской политехнике отреагировали на информацию, которая появилась в СМИ.
Там открестились от своего сотрудника с 27-летним стажем. И заверили, что это
"единичный инцидент". 

Что касается судьбы профессора, то на основании представленной студенткой
жалобы в университете создали комиссию. После чего профессор Громов сам написал
заявление об увольнении. 

"В итоге фигурант инцидента подал заявление на увольнение, которую
удовлетворили 23 октября. Ответ на жалобу направлен ее подателю...
Напоминаем, преподаватели и студенты НТУ "Днепровская политехника" защищали
суверенитет Украины в зоне АТО/ООС и активно занимались волонтерством. В
университете работает центр культуры украинского языка им. Олеся Гончара, часто
проводятся встречи с писателями и тематические мероприятия. Мы имеем четкую
позицию, быстро среагировали на инцидент и успели закрыть дело. Вопрос, почему
об этом эпизоде вспомнили через месяц, остается открытым", -
говорится на станице вуза в Facebook.

\ifcmt
pic https://strana.ua/img/forall/u/0/34/%D0%A1%D0%BD%D0%B8%D0%BC%D0%BE%D0%BA(352).JPG
\fi

При этом проректор по учебной работе университета Юрий Хоменко отметил, что на
преподавателя до этого никаких жалоб не было. После того, как профессору
предложили написать объяснительную в ответ на жалобу, тот от объяснения
отказался и написал заявление на увольнение.

Из показаний руководства "Политехники" может показаться, что Валерий Громов
совершил некий вздорный поступок и сам бросил заявление на стол. 

Однако очевидно, что причиной увольнения стала позиция вуза, который не защитил
своего преподавателя, а наоборот начал давление на него: путем создания
комиссий, требуя объяснительных и прочих унизительных для профессорского звания
процедур. 

При этом характерно, что за русский язык на лекции выступило большинство
студентов, которые принимали участие в обсуждении этого вопроса. Но в данном
случае - как и обычно в Украине - свою позицию продавило меньшинство. 

Правда, были и попытки заступиться за русский язык. Так, проректор Юрий Хоменко
заявил, что большинство учащихся вуза - выходцы из Донбасса. Поэтому часто
преподаватели переходят с ними на русский язык, если студенты просят. При этом
все методические материалы у них на украинском. 

Однако отца Софии Лисковской такой ответ не устроил. Он предложил русскоязычным
студентам ехать в Россию. 

"Я хоть плохо знаю географию, но по карте Украины, Донбасс - это наша
территория. Поэтому почему они решили, что надо идти на уступки, кому эти
уступки нужны? Если бы студенты хотели учиться на русском, они выбрали бы вузов
в другой стране. А если для университета эти люди иностранцы, то проводите им
лекции на английском, согласно ст.48 Закона Украины "О высшем образовании" -
сообщил отец студентки.

\ifcmt
pic https://strana.ua/img/forall/u/0/34/%D0%A1%D0%BD%D0%B8%D0%BC%D0%BE%D0%BA(354).JPG
\fi

Случай с Валерием Громовым показывает, как будут применять на практике
украинизаторские законы: при малейших жалобах националистов русскоязычнх
преподавателей будут выживать с работы. Невзирая на их стаж и заслуги.

Это, конечно, приведет в упадок и без того хромающее украинское образование.

В то же время ситуация в "Днепровской политехнике" показывает, как на местах
молчаливо саботируют языковой диктат из центра. И чем больше будет подобных
инцидентов, тем выше будет недовольство людей мовной диктатурой. С понятными
последствиями для властей, которые ее поддерживают. 

Это, видимо, понимают и у Зеленского, который может на следующих выборах в Раду
показать обвальное падение своей популярности. Сегодня, в обращении на день
украинского языка, президент внезапно призвал не травить людей, которые
общаются на других языках.

"Есть немало граждан, которые по многим причинам не могли ранее овладеть и до
сих пор не овладели украинским языком. Но буллинг, агрессия, принуждение -
точно не будут способствовать его изучению", - заявил президент. 

Правда, эти призывы никак не совпадают с той действительностью, которую
формирует и сам Зеленский. Он, вопреки обещаниям, оставил в силе нормы
украинизации, из-за которых уже началась настоящая травля русскоязычных
украинцев: от кассиров до профессоров. 

"Стоит напомнить, что ни президент Зеленский, ни его монобольшинство в Раде до
сих не предприняла никаких действий, чтоб отменить или хотя бы изменить закон о
тотальной украинизации, который полностью построен на принуждении к
использованию украинского языка во многих сферах.

Более того, созданный по этому закону институт Уполномоченного по госязыку уже
инициирует кампанию по принуждению учителей в школах к использованию
украинского языка. И это, судя по всему, только начало их работы.

Поэтому вместо того, чтоб писать проникновенные посты о недопустимости
буллинга, Зеленскому следовало бы сосредоточиться на искоренении
законодательной базы для этого буллинга.

Если, конечно, он как-то хочет реально решить проблему, а не пропиариться в
очередной раз на теме языка", - пишет телеграм-канал "Политика Страны".
  
\begin{center}
  \begin{fminipage}{0.7\textwidth}
Зеленский поздравил всех с днём украинской письменности, начав с того, что он
родился в русскоязычном Кривом Роге, а закончил тем, что приучать к
украинскому языку нужно без принуждения и булинга.

В тоже время, стоит напомнить, что ни президент Зеленский, ни его
монобольшинство в Раде до сих не предприняла никаких действий, чтоб
отменить или хотя бы изменить закон о тотальной украинизации, который
полностью построен на принуждении к использованию украинского языка
во многих сферах.

Более того, созданный по этому закону институт Уполномоченного по госязыку уже
инициирует кампанию по принуждению учителей в школах к использованию
украинского языка. И это, судя по всему, только начало их работы.

Поэтому вместо того, чтоб писать проникновенные посты о недопустимости
буллинга, Зеленскому следовало бы сосредоточиться на искоренении
законодательной базы для этого буллинга.

Если, конечно, он как-то хочет реально решить проблему, а не пропиариться в
очередной раз на теме языка.

\Purl{https://t.me/stranaua/9316}
  
  \end{fminipage}
\end{center}
  
