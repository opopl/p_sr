% vim: keymap=russian-jcukenwin
%%beginhead 
 
%%file 18_11_2020.news.ua.strana.1.battle_ljusenka_jura_ilovajskii.smysl_battla
%%parent 18_11_2020.news.ua.strana.1.battle_ljusenka_jura_ilovajskii
 
%%url 
%%author 
%%tags 
%%title 
 
%%endhead 

\subsubsection{В чем смысл баттла?}
\label{sec:18_11_2020.news.ua.strana.1.battle_ljusenka_jura_ilovajskii.smysl_battla}

Конфликт Арестовича и Бутусова - эхо войны, которая идет между Офисом
президента и командой Турчинова-Порошенко. У бывшего гаранта таким образом
консолидируют свой электорат, раскачивая тему \enquote{зрады} Зеленского.

И, надо сказать, тактика дала свой эффект: на местных выборах
\enquote{Евросолидарность} перетянула на себя часть националистически настроенных
избирателей \enquote{Слуги народа}. Активно в этой истории поучаствовал и Бутусов,
раскручивая, к примеру, громкую историю о госизмене Ермака по делу
\enquote{вагнеровцев}. 

Но вернемся к Арестовичу. Его взяли на работу в ТКГ вскоре после
увольнения умеренного экс-премьера Витольда Фокина.\Furl{https://strana.ua/news/282441-vitold-fokin-chto-izvestno-o-pervom-premere-ukrainy-kotoroho-mohut-otpravit-na-donbass.html} Того в свою очередь
пригласили накануне местных выборов, чтобы создать иллюзию прогресса в
минском процессе. И остановить отток юго-восточных избирателей к
\enquote{Оппозиционной платформе}. 

Однако после того, как Фокин в интервью \enquote{Стране}
заявил,\Furl{https://strana.ua/news/282441-vitold-fokin-chto-izvestno-o-pervom-premere-ukrainy-kotoroho-mohut-otpravit-na-donbass.html}
что нужно исполнять Минские соглашения во всей полноте - его убрали. А через
какое-то время поставили Арестовича, который до этого считался один из главных
ЛОМов условной \enquote{партии войны}. Что, видимо, по задумке Зеленского и Ермака,
должно было накал \enquote{зрады} снизить. 

Однако, как показала ситуация с Бутусовым, \enquote{партии войны} абсолютно без
разницы, какие декорации на Банковой возводят вокруг темы Донбасса. Атака
идет в принципе на весь процесс отношений с РФ в рамках минской площадки.

Цель при этом двоякая.

Во-первых - загнать президента за флажки, \enquote{красные линии}, прочерченные
Порошенко и Ко, ликвидировав даже теоретическую возможность каких-либо
компромиссов с Россией и \enquote{республиками}. 

Во-вторых - дискредитировать Зеленского среди промайданной части
избирателей, представив его едва ли не агентом Кремля. Хотя по факту его
политика по \enquote{Минску} мало чем отличается от времен Порошенко. У
последнего, впрочем, можно найти еще больше \enquote{зрады} - он не только
подписал Минские соглашения, но и продавил в Раде принятие закона об
особом статусе Донбасса. 

Поэтому - если Офис президента наймет для своего пиара по Донбассу даже не
Арестовича, а, к примеру, самого Парубия, отношение \enquote{партии войны} к
переговорам и к самому Зеленскому это никак не изменит.
