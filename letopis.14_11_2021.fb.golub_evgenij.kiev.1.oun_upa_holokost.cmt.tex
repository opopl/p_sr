% vim: keymap=russian-jcukenwin
%%beginhead 
 
%%file 14_11_2021.fb.golub_evgenij.kiev.1.oun_upa_holokost.cmt
%%parent 14_11_2021.fb.golub_evgenij.kiev.1.oun_upa_holokost
 
%%url 
 
%%author_id 
%%date 
 
%%tags 
%%title 
 
%%endhead 
\subsubsection{Коментарі}

\begin{itemize} % {
\iusr{Petr Oz}
Ждем издания русского перевода с предисловием Баумейстера

\begin{itemize} % {
\iusr{Евгений Голуб}
\textbf{Petr Oz} вначале будет украинский, насколько мне известно, а кто решиться сделать предисловие, то мне неведомо.

\iusr{Виталий Сидоркин}
\textbf{Petr Oz} LOL

\iusr{Gennadiy Druzenko}

\textbf{Eugene Golub} 

дискусія про це дослідження серед українських інтелектуалів вже триває. Бо для
притомних українців ця тема не є ані табу, ані секретом. Головне, аби передмова
була об’єктивнішою за власне дослідження. Бо у автора дуже специфічна репутація
в наукових колах  @igg{fbicon.smile} . Що втім аж ніяк не заперечує потребу перекласти і видати
його працю.

\iusr{Marina Cherenkova}
\textbf{Gennadiy Druzenko} А какая у автора репутация? Я Она публично оформлена? Или это слухи просто? Простите что спрашиваю, купить книгу хочу.

\iusr{Gennadiy Druzenko}

\textbf{Marina Cherenkova} 

є епічні історики, які як колись Гомер намагаються відтворити минувшину в усій
її розмаїтості та суперечності. До таких належать, безумовно, Джадт, Снайдер,
Грицак. Є ті, що як прожектор вихоплюють певну тему і все життя її розвивають.
Хімка з других. Він вихоплює прожектором своїх студій гріхи українського
націоналізму (що безумовно мали місце), але забуває прставити їх у ширший
історичний контекст. В’ятрович навпаки, одне слово  @igg{fbicon.smile} 

\iusr{Marina Cherenkova}
\textbf{Gennadiy Druzenko}, Снайдер? О, ясно, спасибо.

\iusr{Gennadiy Druzenko}
\textbf{Marina Cherenkova} always welcome  @igg{fbicon.smile}.

\iusr{Евгений Голуб}
\textbf{Gennadiy Druzenko} 

вниманию предлагается монография, солидная научная монография признанного
ученого. John-Paul испытал на себе все \enquote{прелести} cancel culture от украинской
диаспоры после 2010 года, когда он выразил сомнение в том, что Бандере стоит
присваивать звание Герой Украины. Эта же диаспора и постаралась создать автору
\enquote{репутацию}. Я готов услышать критику результатов его работы, опровержение
изложенных фактов, трактовок и пр. Когда же речь вместо этого речь заходит о
репутации автора, я понимаю, что или говорящий о \enquote{репутации} с работой автора
не знаком, или по существу возразить автору нечего.

\iusr{Евгений Голуб}
\textbf{Marina Cherenkova} 

я лично говорил с автором книги несколько дней назад. Он произвел на меня
впечатление честного и мужественного человека, дорого заплатившего за то, что
посмел вытащить на свет, рассмотреть и представить миру то, о чем украинской
диаспоре в Канаде хотелось бы никогда не вспоминать.


\iusr{Marina Cherenkova}
\textbf{Евгений Голуб},

Евгений, я бы все равно купила эту книгу. Читать люблю и сравнивать. Думаю мне
дико повезло, что я из Донбасса. Я бы не вынесла такого масштаба и такой
глубины травмы у целых поколений украинцев, которые живут в западной части.
Искренне сочувствую им.

\iusr{Gennadiy Druzenko}
\textbf{Eugene Golub} 

им мене, друже, переконуєш, наче я не написав, що книжку варто і прочитати, і
перекласти. Тільки зі стилистики твого допису, скидається, що тобі потрібна
зброя в боротьбі проти українського націоналізму, а не цікаве дослідження, яке
варто прочитати в контексті інших досліджень ОУН-УПА. Тобто я не заперечую
наукової вартості книжки Химки, я боюсь, що опоненти перетворення наукових
досліджень на інструмент пропаганди. Бо саме це й перечить мені в обох
екстренумах українського суспільства - це бажання бачити тільки ту частину
правди, яка близька їхньому світогляду.

\iusr{Олена Вонс}
\textbf{Marina Cherenkova} 

Травмы? Да они, земляки мои, одобряют и приветствуют \enquote{подвиги} своих предков.
Начнут свои камлания про \enquote{треба дивитися ширше, глибше} и проч.

\iusr{Евгений Голуб}
\textbf{Gennadiy Druzenko} 

ты же знаешь, как я ценю тебя за многие твои достоинства, и, в частности, за
умение сохранять взвешенность суждений, особенно суждений, сделанных публично
 @igg{fbicon.smile}. Ты отчасти прав, но только отчасти: действительно, я рассматриваю эту
книгу, как источник фактов аргументов против героизации УПА и ОУН. Для тех,
кому важно обладать этими фактами и аргументами. Бороться же с национализмом
логикой рассужений бессмысленно, так как это - убеждение. Разновидность тех
незамысловатых верований, что помогают человеку на время справиться с
одиночеством и получить эрзац смысла.

\end{itemize} % }

\iusr{Юрій Чорноморець}

На Амазоне также есть книги про участие русских националистов во всяких
карательных вещах и там есть поразительные цифры про 2,5 млн. таких русских
людей, служивших в подразделениях, которые участвовали в карательных операциях.

\begin{itemize} % {
\iusr{Евгений Голуб}
\textbf{Юрій Чорноморець} 

какое отношение Ваша реплика имеет к исследованию Химки? Представители
нацистких организаций, например, хорватов, болгар, венгров и других
стран-союзников Германии совершали зверства в отношении сербов, греков, евреев
в Греции, Хорватии, Украине и пр. Это в какой-то мере оправдывает глорификацию
отечественных нацистов? Каждая такая история ждет своего Химку или уже
дождалась. Нам свою историю нужно изучить, понять и сделать выводы.

\iusr{Юрій Чорноморець}
\textbf{Евгений Голуб} 

просто сегодняшняя власть в России осуждает всех националистов - и виной тому
сотрудничество с фашизмом - но молчит о лидерских позициях русских в деле этого
самого сотрудничества и также молчит о том, что сегодняшняя \enquote{консервативная
русская идеология путинизма} тождественна во всем фашизму итальянского типа. У
нас в Украине нет господства фашизма итальянского типа в политической теории и
практике. Нету и не будет. А в России есть и будет только дальше развиваться.

\iusr{Юрій Чорноморець}
\textbf{Евгений Голуб} 

и мне непонятно Ваше высказывание об оправдании глорификации каких-то
националистов. Я Химку поддерживал еще тогда, когда о нем никто не слышал из
тех, кому он теперь люб и дорог. Потому если мы уже двигаемся по пути
недопущения нового фашизма, то и надо в первую очередь смотреть туда, где и
коллаборантов было больше и где и сейчас фашизм есть - хотя пока что
относительно \enquote{мягкий} фашизм итальянского типа. Но думаю они скоро и до нацизма
доберутся - во всяком случае процесс уже в полном разгаре. И на этом фоне мы
можем долго осуждать что-то в истории и сегодняшнем дне украинского
национализма - но только если мы видим картину в целом, а не замолкаем о том,
что прямо рядом с нами чудовище полной несвободы, чудовище, растаптывающее
права и свободы человека и целых народов. Пока рассуждаем о недостатках Байдена
или ЕС или Украины - это все действительно есть. Но все молчим об монстре прямо
на нашем пороге.

\iusr{Роман Химич}
\textbf{Юрій Чорноморець} не хочу никого обидеть, но это классический whataboutism, популярнейший инструмент советской пропаганды

\iusr{Mikhael Barst}
Но все же глорификация тех, кто убивал, объявление их героями недопустимо

\iusr{Юрій Чорноморець}
\textbf{Mikhael Barst} 

Осуждение деятелей истории прошлого, которые совершали преступления, имеет
право быть при условии осуждения деятелей настоящего, которые сейчас совершают
аналогичные преступления и имеют аналогичную тоталитарную идеологию и
антигуманные практики, применяемые как к своим гражданам. Также сегодня мы
должны видеть где царит ксенофобия, оправдывается агрессия по отношению к
соседям и так далее. Мы может долго говорить правильная ли у нас политика
памяти или есть ошибки, но в РФ просто запрещают "Мемориал". При чем Мемориал
даже не поднимал многих проблем из прошлого, потому что этого бы уже РФ не
выдержала.

\iusr{Mikhael Barst}
\textbf{Юрій Чорноморець} 

Я, честно говоря, не могу понять, кто вы такой и почему вы ставите условия.
Какие, собственно, у вас полномочия решать, какие книги выпускать Химке и кого
критиковать. Для нормальных людей безусловно неприемлема деятельность
Украинской вспомогательной полиции, расстреливавшей евреев. Книги профессор
Химка будет выпускать и все другие будут делать то, что они хотят, это
демократия западного типа и вряд ли Химка попросит у вас визу на выпуск книги,
вы можете решать только что вам делать лично самому

\iusr{Роман Химич}
\textbf{Юрій Чорноморець} я думаю, стоит, всё таки, ознакомиться с предметом.
\url{https://en.wikipedia.org/wiki/Whataboutism}

\iusr{Rostyslav Pelekhovych}

> Для нормальных людей безусловно неприемлема деятельность Украинской
вспомогательной полиции, расстреливавшей евреев.

а "OUN-UPA" перекладається як "Украинская вспомогательная полиция"?

\iusr{Юрій Чорноморець}

я считаю, что все это разговор по сути. Или мы осуждаем всех и все, изучаем все
практики примирения и исторической памяти, или мы не осуждаем никого, забиваем
на политику памяти и примирения. И вообще требовать от Украины чтобы мы были
стерильно чисты в своем либерализме, пребывая на переднем крае борьбы с
современным реальным фашизмом - странно. И потом опять таки непонятно для нас
либеральные практики полного осуждения всех тоталитарных практик прошлого
актуальны или мы на самом деле не либералы? Потому что тот же Баумейстер совсем
не либерал... В отличие от Химки... И потому что мы хотим? Я понимаю, что
сказать что памятник Бандеры плохо и потому все преступления Путинского режима
против граждан РФ и других стран оправданы - легко. Но это все равно неправда,
это лицемерие. От того, что Байден в чем-то плох тоже не получается, что Путин
хорош. Чтобы сказать Украина плохая, Байден никакой, надо честно сказать обо
всем. Потому мне больше нравиться канал Ивана Яковины, который одинаково
критикует все, а не избирательно. Потому мне нравиться Грицак - одни критерии
оценки для всех событий прошлого, понимание контекста этих событий, полное
отрицание примеси идеологии - это правильно.

\iusr{Юрій Чорноморець}

И важнейшая вещь - социология. В РФ национализм и ксенофобию уровня укр. партии
Свобода проповедывали все сегодняшние парламентские партии - и за них народ
проголосовал так или иначе. Потому я не понимаю почему мы должны испытывать
вину за несколько процентов нашего населения голосующего за Свободу и не видеть
реальной опасности прямо рядом с нами. У нас что из-за русскоязычия сразу
должен пропасть разум, который позволяет анализировать факты и понимать
исторические реальности?


\iusr{Mikhael Barst}
\textbf{Юрій Чорноморець} 

Я не понимаю сам ваш подход, когда заявляете \enquote{мы делаем}, \enquote{..мы, Украина}. Вы
какой-то пост занимаете, что вы говорите от имени целой страны? В Украине, как
и в России, Канаде, США есть люди, высказывающие разные точки зрения. Те
примеры, которые вы приводите - Грицак, канал Ивана Яковины - это источники
обывателя, СМИ, не требующие специальной подготовки. Академическая среда, в
которой действует проф. Химка, совершенно другая. Цели её и методы совершенно
другие. Это не сегодняшнего дня политпросвет против Путина или против Бандеры,
а решение определенных академических задач в определенном контексте
исторической науки, которая поднимает какие-то вопросы уже десятилетиями. Он не
политик иметь партийную политческую репутацию, он исследователь с большим
научным стажем. В своей книге он анализирует Книгу докладов Михайло
Колодзинского, другие доклады УПА. Их источник - украинские архивы. Вы
говорите, что вы любите канал Ивана Яковины, так и смотрите его, Е. Голуб не
заставляет же вас покупать эту книгу, это другая дверь, понимаете,
академическая наука не для вас

\iusr{Юрій Чорноморець}
\textbf{Mikhael Barst} 

Упомянутые Вами люди Грицак и Яковина, Вами низко оцененные, ничем не хуже
Баумейстера как комментаторы политических процессов. То что я не имею права
говорить и обращать внимание что националистов у нас 2\% в отличие от русских
92\% - я понял. Это чудесно.


\iusr{Lara Alia}
\textbf{Юрій Чорноморець} 

уважаемый Юрий! Как вы легко с памятью расстаётесь, говоря \enquote{забъем}. Так уже
было.. у кого-то забили практически всех под ноль. Говорить об этом надо и по
возможности спокойно обсуждать, чтобы этот ужас не повторился.

\iusr{Виталий Сидоркин}
\textbf{Юрій Чорноморець} Так есть мнение, что русские белые офицеры начали антисемитскую компанию в Германии.

\iusr{Olga Golub}
\textbf{Юрій Чорноморець} 

вы имеете полное право говорить и обращать внимание на любые темы, так же как и
те, к кому обращён ваш странный лозунг - «или говорим обо всем или не говорим
ни о чем»

\end{itemize} % }

\iusr{Andrei Raschuk}
Они никуда не делись

\iusr{Александр Мороз}

Фамилии руководителей описанных подразделений очень похожи на те фамилии, что
делали революцию 1917... очередная хуцпа! Простым людям не нужен ненационализм
ни антисемитизм...


\iusr{Taras Tymo}

Серед багатьох істориків Химка \enquote{нерукопожатний} як маніпулятор і пропагандист.
ОУН-УПА це в нього якась травма, яку він постійно мусолить.

\begin{itemize} % {
\iusr{Роман Химич}
\textbf{Taras Tymo} серед яких саме істориків?

\iusr{Taras Tymo}
\textbf{Roman Khimich} а яка різниця - ви ж все одно їх затавруєте як \enquote{неправильних} )

\iusr{Роман Химич}
\textbf{Taras Tymo} побутова телепатія - потужний інструмент пізнання світу ))

\iusr{Taras Tymo}
\textbf{Roman Khimich} чому ж телепатія - достатньо ваш профіль глянути )

\iusr{Роман Химич}
\textbf{Taras Tymo} Argumentum ad hominem ultima ratio est ))
\end{itemize} % }

\iusr{Юрій Чорноморець}

Обратите внимание, что дискуссии среди украинских историков уже закончены,
итоги этих дискуссий уже подсуммированы Ярославом Грицаком. Химка -это голос до
того как черта была подведена, надо это понимать. Есть смысл понимать, что в
популярном дискурсе еще споры продолжаются, но это не означает, что в среде
профессиональных историков эти процессы еще продолжаются.

\begin{itemize} % {
\iusr{Роман Химич}
\textbf{Юрій Чорноморець} 

кто эти украинские историки? Вы не упустили, часом, приставку "про-"?
Нахмановичи и вятровичи, действительно, уже всё для себя выяснили и щедро
делятся плодами своих штудий с теми, кто готов их слушать.

Я, например, не готов

\iusr{Юрій Чорноморець}
\textbf{Роман Химич} Упомянутых я тоже не готов слушать. Не понимаю при чем тут они.

\iusr{Роман Химич}
\textbf{Юрій Чорноморець} тогда давайте выясним о каких историках идёт речь и чем именно закончились их дискуссии

\iusr{Юрій Чорноморець}
\textbf{Роман Химич} 

Дискусії в цілому і конретно щодо Химки були, ознайомитися з ними можна
починаючи з публікацій у цьому ліберальному журналі
\url{https://krytyka.com/ua/journals}

\iusr{Роман Химич}
\textbf{Юрій Чорноморець} ліберальному ))

\iusr{Роман Химич}
\textbf{Юрій Чорноморець} 

я и с этим либеральным часопысом, и с творчеством его головного редактора
знаком более чем хорошо. Моё о них мнение не оставляет места для Ваших тезисов,
если кратко

\href{https://www.facebook.com/roman.khimich/posts/3681437128553365}{%
Стаття Миколи Рябчука \enquote{Від Малоросії}, Роман Химич, facebook, 10.09.2020%
}

\iusr{Юрій Чорноморець}
\textbf{Роман Химич} допис против Рябчуа мені подобався і зараз подобається. При чому тут Грабович не знаю.

\iusr{Роман Химич}
\textbf{Юрій Чорноморець} Так Грабович, Рябчук та Критика в цілому є представниками "націонал-лібералізму". Це одна течія, один світогляд, один підхід

\iusr{Юрій Чорноморець}
\textbf{Роман Химич} 

Можливо. І я мрію, коли у росіян в парламенті, уряді та у президентському
кріслі будуть \enquote{націонал-ліберали}, а не політики, у всьому подібні до
італійського фашизму і можна буде спокійно говорити про те, які погані
укр. \enquote{націонал-ліберали} і як вони мало поступаються політиці ідентичності
сусідів та колаборантів.


\iusr{Taras Tymo}
\textbf{Roman Khimich} 

"Вы не упустили, часом, приставку "про-"? - а ви хотіли, щоб приставка була
"анти-"? Така мила відвертість... )


\iusr{Роман Химич}
\textbf{Taras Tymo} мені важко коментувати результати ваших чергових вправ в телепатії ))
У який спосіб ви прийшли висновку яку саме приставку я хотів? ))

\iusr{Роман Химич}
\textbf{Юрій Чорноморець} 

здається у нас різні уявлення про італійський фашизм та сучасний російський режим.

Скажу більше, якщо говорити про фашизм як такий, Україна суттєво випереджає РФ
в справі його легалізації. Ви в курсі національних особливостей боротьби із
пропагандою тоталітарних режимів? Можете пояснити, як так сталося, що в Україні
легалізовані абсолютна більшість символів нацизму та вся одразу фашитська та
неонацистська символіка?

\iusr{Алексей Овчаренко}
\textbf{Роман Химич} між символікою та соціальними процесами, так би мовити, досить велика відстань.

\iusr{Viktor Kaganovich}

— ваши историки - банальные этнонарциссисты и шарлатаны

— вот когда в России возникнет...


\end{itemize} % }

\iusr{Gleb Nemo}

«Эта выдающаяся книга является первым всеобъемлющим исследованием причастности
ОУН-УПА к Холокосту». «Украинские националисты и Холокост». «Химки проливает
свет на яростный антисемитизм, который пронизывал это движение...». Хочу
ответить, что «вообще-то термин «антисемит» совершенно не научен. Семиты — это
не только евреи, но и арабы и другие народности. Евреи постоянно ведут войну
против арабов, тем самым самих евреев можно смело назвать антисемитами»
(Истархов В.А. «Удар русских богов», Из-во ЛИО «Редактор», С.-Петербург
2001.—401 с. С.73)

Что касается украинского национализма, то согласно Статье 14 Конституции
Украины, в которой говориться, что «Земля є основним національним багатством,
що перебуває під особливою охороною держави», соответственно, нема
національності — нема «багатства». Мне кажется, я понимаю куда дует ветер.  @igg{fbicon.thinking.face} 

И ещё, Во всеобщей Декларации Прав Человека, утверждённой ООН, в статье 19
записано: «Каждый человек имеет право на свободу убеждений и а свободное
выражение их; это право включает свободу беспрепятственно придерживаться своих
убеждений и свободу искать, получать и РАСПРОСТРАНЯТЬ ИНФОРМАЦИЮ И ИДЕИ ЛЮБЫМИ
СРЕДСТВАМИ и независимо от государственных границ». Если мы имеем право иметь
любые убеждения, то почему нельзя быть антисемитом?

Джордано Бруно, Вольтер, Кант, Гёте, Шиллер, Генри Форд, историк Страбон,
Цицерон, Сенека, Тацит, Наполеон, Шопенгауэр, Джордж Вашингтон, Бисмарк, Дуглас
Рид, Герберт Уэллс, Черчилль, ведущие историки Египта, Рима, Греции,
Александрии, русские — Достоевский, Пушкин, Даль, Чехов, Гоголь,
Салтыков-Щедрин, Менделеев, Блок, Розанов, князь Святослав, князь Владимир ІІ,
Пётр I, князь Горчаков, Екатерина I, Императрица Елизавета Петровна, Николай I,
— все эти люди были антисемитами.

В Украине 7 октября 2021 года Владимиром Зеленским был подписан «Закон про
запобігання та протидію антисемітизму в Україні», к чему бы это?


\iusr{Сергей Лунин}
Да уж, на всякого ядрёного вышиватного д...ба найдётся не менее ядрёный ватный д...б.

\iusr{Олег Трибой}

книга, которая должна выйти в украинском и русском переводах! обязана выйти!
Поляки уже пережили \enquote{открытие темы} и очень болезненное переживание. Нас это
еще ждет.

\iusr{Андрей Лесто}
Хорошо бы эту книгу презентовать всем евреям, пропитанным духом украинского национализма.

\iusr{Yevgeniy Sova}
Вот этим для обязательного изучения. 

\href{https://uk.wikipedia.org/w/index.php?title=Жидобандерівці}{%
Жидобандерівці, uk.wikipedia.org%
}

\begin{multicols}{2}

«Жидобандерівці» — інтернет-мем, який виник 2006 року в українській блогосфері
як пародіювання стилю висловлювань російських шовіністів-українофобів.
Поширився у пародійній спільноті «Фофудья» після анонімного коментаря до посту
віртуального персонажу блогосфери \href{https://uk.wikipedia.org/wiki/Іван_Денікін}{Івана Денікіна}.

«Ізвініте что вьінужден пісать гадкімі малоросійскімі (ябьі сказал
мікроросійскімі) буквамі но злобньіе жидобандервци вирвалі все клавіши с
русскімі буквамі плоскорубцамі і сожглі іх.»

\headTwo{З 2014 року}

Набув більш широкої популярності в Україні в 2014 році на тлі російської
збройної агресії проти України та намагань російської пропаганди розколоти
українське громадянське суспільство. Жидобандерівцями називають себе громадяни
України з розвинутим почуттям гумору та необов'язково єврейським корінням, які
підтримали Євромайдан, і українських націоналістів — спадкоємців ідей Степана
Бандери. Назва виникла як реакція на звинувачення прихильників Майдану, зокрема
активістів «Правого сектора» в антисемітизмі, які активно поширювалися в
путінській Росії та транслювалися на інші країни[2].

Фактичною підставою для виникнення даного поняття стала підтримка єврейськими
колами України та окремими представниками єврейського капіталу патріотичних та
певною мірою українсько-націоналістичних рухів в Україні. Найбільш яскравим
проявом явища стала діяльність Ігоря Коломойського та його колеги Геннадія
Корбана щодо фінансування ДУК ПС та недопущення проросійського заколоту в
Дніпрі. В одному з епізодів цієї співпраці Коломойський з'явився на публіці в
майці з написом «Жидобандерівець».

\headTwo{Образ «жидобандерівців»}

Мем «жидобандерівців» в Україні став локальним трендом[3], був помічений у
вигляді символіки на майках — зображення менори з зубцями від Тризуба або
поєднання Тризуба з зіркою Давида.

\headTwo{Цікаві факти}

Одне з пропагандистських поєднань українських «буржуазних націоналістів» та
євреїв—«сіоністів», як антирадянських рухів, нібито підживлюваних Заходом у
ході «холодної війни», містилось в карикатурі українського журналу «Перець» за
червень 1981 року[4].


\end{multicols}

\end{itemize} % }
