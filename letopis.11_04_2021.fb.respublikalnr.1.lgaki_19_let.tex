% vim: keymap=russian-jcukenwin
%%beginhead 
 
%%file 11_04_2021.fb.respublikalnr.1.lgaki_19_let
%%parent 11_04_2021
 
%%url https://www.facebook.com/groups/respublikalnr/permalink/794789424490184/
 
%%author 
%%author_id 
%%author_url 
 
%%tags 
%%title 
 
%%endhead 

\subsection{В ЛГАКИ прошли мероприятия, приуроченные к 19-летию со дня основания вуза}
\label{sec:11_04_2021.fb.respublikalnr.1.lgaki_19_let}
\Purl{https://www.facebook.com/groups/respublikalnr/permalink/794789424490184/}

ГАЗЕТА "РЕСПУБЛИКА" (№14, 2021г). ДАТА. ГЕОМЕТРИЯ ПО МАТУСОВСКОМУ

В ЛГАКИ прошли мероприятия, приуроченные к 19-летию со дня основания вуза


\ifcmt
  pic https://scontent-bos3-1.xx.fbcdn.net/v/t1.6435-9/171901590_121092906737484_8862283202388457206_n.jpg?_nc_cat=104&ccb=1-3&_nc_sid=b9115d&_nc_ohc=rCKZuD2bfgAAX-CoyUp&_nc_ht=scontent-bos3-1.xx&oh=a2cc7edaabc538daed66755705736b26&oe=609C594C

	pic https://scontent-bos3-1.xx.fbcdn.net/v/t1.6435-9/172708328_121092936737481_2502112295634174226_n.jpg?_nc_cat=111&ccb=1-3&_nc_sid=b9115d&_nc_ohc=wBZ2RCNyzrcAX912wA9&_nc_ht=scontent-bos3-1.xx&oh=b87883f6c7a3c42024e061f422bec572&oe=609AB545

	pic https://scontent-bos3-1.xx.fbcdn.net/v/t1.6435-9/173208453_121092916737483_378962939225316552_n.jpg?_nc_cat=105&ccb=1-3&_nc_sid=b9115d&_nc_ohc=DBGr12crOPUAX_y_hTv&_nc_ht=scontent-bos3-1.xx&oh=ed87fb5eedd8754c7838b2eb794533af&oe=609C89D3

	pic https://scontent-bos3-1.xx.fbcdn.net/v/t1.6435-9/171969781_121092953404146_5144504408683294876_n.jpg?_nc_cat=109&ccb=1-3&_nc_sid=b9115d&_nc_ohc=0JY-22_j0EcAX_KSlXZ&_nc_ht=scontent-bos3-1.xx&oh=b401d71449cee452611c56f7e34a0578&oe=609C6B52

	pic https://scontent-bos3-1.xx.fbcdn.net/v/t1.6435-9/172567477_121092940070814_5522399662830333194_n.jpg?_nc_cat=104&ccb=1-3&_nc_sid=b9115d&_nc_ohc=U87Zqzejo0sAX9o5qNw&_nc_oc=AQnDm0V6MfuU1cF9bjGI4fJa7Qi8mnn9EG4Xy4DNMzdh8AOzE9DdP5sicB2VLFGU5XU&_nc_ht=scontent-bos3-1.xx&oh=50f9ac0af1cadac8c6b8a54bbdead4c9&oe=609D360D
\fi


Ежегодно Луганская государственная академия культуры и искусств имени Михаила
Матусовского выпускает более 300 профессионалов по 73 специальностям, которые
составляют кадровый потенциал ведущих учреждений культуры и средств массовой
информации Республики.

– Академия является флагманом в подготовке творческих кадров, надёжным звеном в
системе творческого воспитания и образования подрастающего поколения, а также
главным методическим центром всей системы творческого образования Республики, –
отметил на брифинге по случаю дня основания вуза Министр культуры, спорта и
молодёжи ЛНР Дмитрий Сидоров.

Одним из самых значимых шагов в научной деятельности вуза за прошедший год
стало создание при поддержке Министерства культуры, спорта и молодёжи ЛНР
научно-просветительского центра по изучению русской культуры Донбасса. Ректор
ЛГАКИ рассказал, что в ближайшее время центр презентует медиапроект «Луганск и
луганчане», а летом будут проведены фестиваль патриотической песни «С чего
начинается Родина» и открытый пленэр «Место силы» в Краснодоне.

За 7-летний период существования ЛНР ЛГАКИ было проведено более 200 конкурсов,
фестивалей, выставок и ко дню рождения Академии их количество увеличилось.
Педагоги и студенты вуза подготовили несколько мероприятий, объединённых общим
названием «Девятнадцатая диагональ».

– Многие спрашивают: почему диагональ? В геометрии диагональ – это прямая,
соединяющая две вершины. Искусство – это и есть вечное движение от одной
вершины к другой, – пояснил ректор ЛГАКИ Валерий Филиппов и добавил, что по
диагонали можно не только подняться вверх, но и стремительно скатиться вниз, а
чтобы этого не случилось, творческому человеку необходимо неустанно
развиваться, трудиться, находиться в постоянном поиске новых форм и решений.

Первым событием в череде запланированных стала художественная выставка работ
преподавателей Академии, презентация которой состоялась 5 апреля. Но
организовать обычный вернисаж может кто угодно, а «академики» любят
оригинальность и мыслят нестандартно. Поэтому главный идейный вдохновитель вуза
– Валерий Филиппов, предложил коллегам совместить два вида искусства и
запечатлеть образы, навеянные музыкой.

А чтобы гости смогли в полной мере оценить экспозицию, перед открытием выставки
провели перфоманс – представив зрителям каждый экспонат, будь то графика,
живопись или фотография, в сопровождении музыки, вдохновившей автора на
создание работы.

– Мы живём здесь – на земле, а мыслим небесными сферами. Наше вдохновение не от мира сего. Наша диагональ – от земли до неба, – поделилась декан факультета изобразительного и декоративно-прикладного искусства Наталья Феденко.

Продолжился марафон праздничных событий флешмобом «Птицы возвращаются домой».
Студенты решили совместить приятное с полезным и собрались 6 апреля у пятого
корпуса Академии, чтобы порадовать гостей праздника ярким представлением и
подарить крылатым вестникам весны новые «квартиры».

Следующий день был отведён для того, чтобы поблагодарить сотрудников и
студентов ЛГАКИ за их активность, креатив и трудолюбие. На торжественном приёме
ректора академии Матусовского раздавали грамоты и благодарности тем, кто
ратовал за общее дело – развитие родной альма-матер.

Завершающим аккордом стал отчётный концерт Академии Матусовского в большом зале
Дворца культуры имени Ленина.

Надежда ПЕРЕСВЕТ
Фото Ксении ЩЕРБИНЫ
ГАЗЕТА "РЕСПУБЛИКА" (№14, 2021г).
#газета #республика #День_рождения_вуза #ЛГАКИ
