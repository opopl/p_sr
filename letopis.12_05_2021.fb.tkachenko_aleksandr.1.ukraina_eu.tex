% vim: keymap=russian-jcukenwin
%%beginhead 
 
%%file 12_05_2021.fb.tkachenko_aleksandr.1.ukraina_eu
%%parent 12_05_2021
 
%%url https://www.facebook.com/oleksandr.tkachenko.ua/posts/4100992319968339
 
%%author 
%%author_id 
%%author_url 
 
%%tags 
%%title 
 
%%endhead 
\subsection{Україна об'єднана з Європою}
\Purl{https://www.facebook.com/oleksandr.tkachenko.ua/posts/4100992319968339}

Україна об'єднана з Європою: наша національна пам'ять та ідентичність – є
частиною європейської. Доєднуємось до онлайн марафону \verb|#миEUкраїна| та зберігаємо
українську культурну спадщину разом 

Згадаймо прислів'я: «Рим теж не одразу будувався». Тим не менш, Колізей, якому
вже майже 2000 років, збережений і до наших днів. Ми теж маємо зберегти наші
пам'ятки.

\ifcmt
  pic https://scontent-frt3-1.xx.fbcdn.net/v/t1.6435-9/186137370_4100984833302421_51982412010058980_n.jpg?_nc_cat=109&ccb=1-3&_nc_sid=730e14&_nc_ohc=imZeKkrXNSAAX9FDcJG&_nc_ht=scontent-frt3-1.xx&oh=8baf843c7c406c156205dd1fb4f11e4d&oe=60C685D6
\fi

Реалізація  програми «Велика Реставрація» важливий крок на шляху до збереження
культурного коду українців і трансляції європейських цінностей. Активна робота
команди \verb|#МКІП| над реставрацією культурних пам'яток розпочата. Наші інвестиції у
культуру працюватимуть на державу роками: на її славу та туристичну
привабливість. Окрім реставрації, у планах ремонт, будівництво та протиаварійні
роботи.


Сподіваюсь, що в подальшому охорона відреставрованих об'єктів та постійний
моніторинг підтримуватиме їх в належному стані. Охоронний статус надаватиметься
залежно від того, наскільки вагома роль кожної окремої пам'ятки для місцевого
чи національного значення.

Окрім порятунку та відновлення нашої спадщини, наповнюємо її новими змістами,
зокрема, туристичною привабливістю. Особливо це стосується невеликих містечок
та селищ, де культурні пам'ятки – основне джерело фінансового та культурного
розвитку регіону. Туристична привабливість України посилюється, в тому числі,
завдяки можливостям для подорожей – новими дорогами.

Загалом, маємо амбітні плани, тому 2021-й рік – лише початок нашого культурного
піднесення. Впевнені, запустивши цей процес – його уже не зупинити.

Тим часом передаємо естафету Міністерство оборони в онлайн-марафоні до Дня
Європи в Україні \verb|#миEUкраїна| у межах урядової інформаційної кампанії EUКраїна
за ініціативи Офіс Віцепрем'єрки з питань європейської та євроатлантичної
інтеграції, підтримки Представництва Європейського Союзу в Україні і проекту ЄС
«Association4U»

