% vim: keymap=russian-jcukenwin
%%beginhead 
 
%%file 04_05_2022.fb.topolja_taras.1.harkov_rusmir
%%parent 04_05_2022
 
%%url https://www.facebook.com/taras.topolya/posts/5143474432398922
 
%%author_id topolja_taras
%%date 
 
%%tags 
%%title Харків - всеодно тут було і залишається дуже багато тих, хто чекає... того самого русского міра
 
%%endhead 
 
\subsection{Харків - всеодно тут було і залишається дуже багато тих, хто чекає... того самого русского міра}
\label{sec:04_05_2022.fb.topolja_taras.1.harkov_rusmir}
 
\Purl{https://www.facebook.com/taras.topolya/posts/5143474432398922}
\ifcmt
 author_begin
   author_id topolja_taras
 author_end
\fi

Не Київ. 

Ранок - сімдесятий.

Настрій - на жаль, увы и ах...

Місяць - спостереження.

Харків це молоде по духу і ще досить проукраїнське місто. Люди освічені, багато
студентів. Але всеодно тут було і залишається дуже багато тих, хто чекає... того
самого русского міра.

Якщо ти йдеш по Харкову, вітаєшся, а від тебе відвертають голову, або взагалі
не дивляться в очі, знай це воно. Часто буває вдивляються в наші київські
номери і потім між собою емоційно обговорюють це, не подобається. 

В Черкаській Лозовій ми зайшли у двір, де на дверній ручці був російський герб,
а вдома повно літератури на тему. 

І це Харків!

Коли ми кажемо про колективну вину руSSкіх у цій війні, чомусь забуваємо про
наших домашніх. 

Але, не зважаючи на це, ми переможемо!

Слава ЗСУ!

Слава Україні!

\ii{04_05_2022.fb.topolja_taras.1.harkov_rusmir.cmt}
