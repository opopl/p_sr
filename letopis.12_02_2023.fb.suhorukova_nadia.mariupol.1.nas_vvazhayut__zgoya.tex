%%beginhead 
 
%%file 12_02_2023.fb.suhorukova_nadia.mariupol.1.nas_vvazhayut__zgoya
%%parent 12_02_2023
 
%%url https://www.facebook.com/permalink.php?story_fbid=pfbid0LikME5zzg4u2A4Ev6JvDdJyQ2LXv6HnPaomAE2tFwKxyybgERbxhcLB6RVtef2pfl&id=100087641497337
 
%%author_id suhorukova_nadia.mariupol
%%date 12_02_2023
 
%%tags mariupol
%%title Нас вважають ізгоями у Маріуполі
 
%%endhead 

\subsection{Нас вважають ізгоями у Маріуполі}
\label{sec:12_02_2023.fb.suhorukova_nadia.mariupol.1.nas_vvazhayut__zgoya}

\Purl{https://www.facebook.com/permalink.php?story_fbid=pfbid0LikME5zzg4u2A4Ev6JvDdJyQ2LXv6HnPaomAE2tFwKxyybgERbxhcLB6RVtef2pfl&id=100087641497337}
\ifcmt
 author_begin
   author_id suhorukova_nadia.mariupol
 author_end
\fi

Нас вважають ізгоями у Маріуполі.

Тому що ми поїхали.

Якийсь чоловік написав про це під одним з моїх дописів.

Він залишився в місті і запевняє, що в Маріуполі \enquote{все добре}.

Написав, що вже паспорт ерефії отримав.

Почувається чудово серед руїн, з мертвими сусідами та близькими, загрожує всім,
хто не з ним і славить країну-агресора.

Мені цікаво, чому у тих, хто за рашку, злість зашкалює?

Їх що там ненавистю заражають?

Чи може в окупації у когось  в голові щось змінюється?

А може, вони завжди такими були?

\enquote{Мариуполь разбит, там живут настоящие люди, которые любят Мариуполь, поверьте
даже в разбитых квартирах, без тепла, без света и газа, у людей всё сохранено
но они покинули Мариуполь, почему? Поехали в Европу в чужие комнаты, на
социалку, почему.....?}

\enquote{Значит так его любили, а город живой, зализал  свои раны и готов к новой жизни! И
не важно какая тут власть, нужно жить дальше, а так скулить и скитаться можно
бесконечно! Жизнь она продолжается, и так как было уже не будет, живите здесь и
сейчас!}

Тобто все одно с ким ти. Зі своєю країною чи з ворогом. 

Все одно, що вбили сотні тисяч, зруйнували все місто, знищили життя, Україну
орки продовжують бомбардувати та обстрілювати.

Але  \enquote{життя продовжується}...

Звідки вони беруть цю ідіотську банальну фразу?

Однакову. Видають, як під копірку.

Звісно, продовжується. Але хіба це життя? Це  -  існування.

Маріуполь обов'язково повернеться до України.

А ми до Маріуполя 

Як вони з нами житимуть?

Про що  розповідати будуть?

Ми ж завжди знатимемо, що вони не справжні.

Якщо щось трапиться – на них надії немає.

Їм однаково з ким.

Це ніби подруга – повія.

Їй все одно кого приймати.  І вона від тебе того ж чекає.

Ще дивується - чому ти інша? 

Я не вмію з такими  розмовляти. 

Раніше намагалася. Зараз не можу.

Блокую.

Дякую, тим, хто їм відповідає. Слова знаходять. Аж раптом до них щось дійде?

\enquote{Как жить рядом с теми, кто убил твоих родных, кто разрушил твой дом, кто
уничтожил твою жизнь, твои мечты растоптал? Оккупанстский хлеб глотку не дерёт?
И не надо говорить, что мы не любим Мариуполь, мы любим наш родной украинский
Мариуполь! И мы обязательно в него вернёмся!}

\enquote{Вы серьёзно?! Потому что мы не хотим жить с убийцами и окупантами!}

У місті люди різні залишились.

Є ті, у кого виходу не було, кому виїхати не вдалося. 

Хто чекає на повернення України. 

Уявляю, як їм страшно  дивиться на все, що там відбувається.

\enquote{А мы кто остался в Мариуполе, остались потому что не было транспорта,
не было возможности так как у меня лежачая мама, красный крест не приехал, в
частном доме не смогла её уберечь прилёт в окно. вот и все ,..а теперь находясь
в городе с дочерью и внуком (4 года), живём на мою пенсию по группе, везде
бешенные очереди и боль когда видишь что нет района стадиона, Московской, три
дома на Комсомольском бульваре и мой дом частный  на проспекте Ленинградский
который разбит и никто его не ремонтирует а средств на восстановление нет вот и
все ....сплошная боль и ужасы во сне}

Тим, хто чекає на повернення України - важко дуже.

Вони не слова пишуть, вони своїм болем діляться.

Їм відчути треба, що вони не одні. 

Що поруч не лише вороги та зрадники.

\enquote{Забыли добавить,что город вроде бы родной ,но чужой... Когда бомбили, было
только одно-выжить. Вода, дрова, приготовление еды, походы к родным - потому что
ту сторону бомбили( идёшь и думаешь, дом и родные целы? И эти мысли сводили с
ума, пока дойдешь.) А сейчас... Просто потерялась в этой жизни и своём родном
городе.}

\enquote{А мы кто уехал, очень вас любим и не капельки не осуждаем, а всегда думаем
как вам там холодно, голодно да,нам душевно очень тяжело но мы сыты и в тепле
,а вы продолжаете жить в этом аду и нам вас жалко ещё больше, не все люди могут
уехать, держитесь там без нас до времени!!! Любим !!!}

\enquote{Хочется поклониться Вам до земли, за Вашу поддержку. Я каждый день плачу, не
живу-существую}

Коли ми повернемося – головне не переплутати.

Щоб не всіх під одну гребінку.

Щоб не змішалися ті, у кого з ворогами \enquote{життя налагодилося} з людьми
справжніми.
