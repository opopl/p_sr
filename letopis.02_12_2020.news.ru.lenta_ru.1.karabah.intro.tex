% vim: keymap=russian-jcukenwin
%%beginhead 
 
%%file 02_12_2020.news.ru.lenta_ru.1.karabah.intro
%%parent 02_12_2020.news.ru.lenta_ru.1.karabah
 
%%url 
 
%%author 
%%author_id 
%%author_url 
 
%%tags 
%%title 
 
%%endhead 

\begin{leftbar}
	\bfseries
Азербайджанцы возвращаются в Карабах. Смогут ли они ужиться с армянами, победив
взаимную ненависть?
\end{leftbar}

\ifcmt
pic https://icdn.lenta.ru/images/2020/12/01/19/20201201191530335/detail_f0a8379da48b6239a73269580b4fdd08.jpg
caption Фото: Валерий Мельников РИА Новости
\fi

%В Нагорном Карабахе завершился процесс передачи территорий под контроль
%Азербайджана. Все прошло без эксцессов и в сроки, оговоренные в трехстороннем
%соглашении о перемирии между Москвой, Ереваном и Баку. Азербайджанцы не устают
%отмечать это событие уличными гуляньями, а президент Алиев уже пообещал вернуть
%обретенным землям былое величие и получить от Армении компенсацию за годы
%«оккупации». Армяне тоже выходят на улицы, но совершенно по другому поводу —
%для них потеря территорий в Карабахе стала национальной трагедией, и теперь они
%хотят свергнуть правительство, допустившее «предательство» национальных
%интересов. А в самом Карабахе между тем в полной неразберихе перемещаются сотни
%тысяч беженцев: азербайджанцы спустя годы возвращаются в некогда брошенные
%дома, армяне, напротив, покидают обжитые места и устремляются прочь. «Лента.ру»
%разбиралась, какие проблемы ожидают Армению и Азербайджан после очередного
%передела границ в этом взрывоопасном регионе.

%В Нагорном Карабахе завершился процесс передачи территорий под контроль
%Азербайджана. Все прошло без эксцессов и в сроки, оговоренные в трехстороннем
%соглашении о перемирии между Москвой, Ереваном и Баку. Азербайджанцы не устают
%отмечать это событие уличными гуляньями, а президент Алиев уже пообещал вернуть
%обретенным землям былое величие и получить от Армении компенсацию за годы
%«оккупации». Армяне тоже выходят на улицы, но совершенно по другому поводу —
%для них потеря территорий в Карабахе стала национальной трагедией, и теперь они
%хотят свергнуть правительство, допустившее «предательство» национальных
%интересов. А в самом Карабахе между тем в полной неразберихе перемещаются сотни
%тысяч беженцев: азербайджанцы спустя годы возвращаются в некогда брошенные
%дома, армяне, напротив, покидают обжитые места и устремляются прочь. «Лента.ру»
%разбиралась, какие проблемы ожидают Армению и Азербайджан после очередного
%передела границ в этом взрывоопасном регионе.

