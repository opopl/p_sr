% vim: keymap=russian-jcukenwin
%%beginhead 
 
%%file 08_06_2021.fb.filatova_elena.1.bogdan_hmelnickij
%%parent 08_06_2021
 
%%url https://www.facebook.com/permalink.php?story_fbid=316017416671741&id=100047904543554
 
%%author 
%%author_id filatova_elena
%%author_url 
 
%%tags hmelnickii_bogdan,istoria,kazaki,perejaslav_1654,rossia,rusmir,ukraina
%%title Богдан Хмельницкий
 
%%endhead 
 
\subsection{Богдан Хмельницкий}
\label{sec:08_06_2021.fb.filatova_elena.1.bogdan_hmelnickij}
\Purl{https://www.facebook.com/permalink.php?story_fbid=316017416671741&id=100047904543554}
\ifcmt
 author_begin
   author_id filatova_elena
 author_end
\fi

Одержав блистательные победы над польским войском под Жёлтыми Водами  5, 7 и 8
мая, а затем 16 мая – у Корсуня*, гетман Богдан Хмельницкий вернулся с войском
своим в Черкассы, и 8 июня (по нынешнему календарю – 18-го) 1648 года отправил
русскому царю Алексею Михайловичу послание с просьбой о принятии Гетманщины в
подданство. Письмо это замечательно во многих отношениях – если, конечно,
понимать суть времени, когда оно было написано.

\ifcmt
  pic https://scontent-mia3-1.xx.fbcdn.net/v/t1.6435-0/p526x296/197522139_316016510005165_3969561401849858395_n.jpg?_nc_cat=101&ccb=1-3&_nc_sid=730e14&_nc_ohc=fFw2fg1uOV4AX8hB6NI&_nc_ht=scontent-mia3-1.xx&tp=6&oh=4663f16fdfd36d9ee5c0869090b23283&oe=60CFE2F9
\fi

Понимать следует прежде всего то, что гетман Богдан-Зиновий Михайлович
Хмельницкий предлагал воссоединить не Украину с Россией, как о том
десятилетиями говорили присяжные историки, и что было закреплено на
государственном уровне во времена СССР, а Русь с Русью – ибо ничего
«украинского» на этих территориях никогда не было: даже много позже известный
предатель Филипп Орлик называл пресловутого Мазепу  Алкидом Российским, а отец
этого «Геракла» Адам-Степан Мазепа, один из соратников Богдана Хмельницкого,
носился с идеей создания не «Украины», а Великого княжества Русского (в составе
Речи Посполитой). Даже Лютый Ярема – Иеремия Вишневецкий, прямой противник
Хмельницкого, – носил титул «князя Русского», а самого гетмана Хмеля, на второй
день нового 1649 года триумфально въехавшего через Золотые Ворота в Киев,
называли «Моисеем веры русской», «защитником свободы русского народа», «новым
Маккавеем».

Историк Сергей Соловьёв предлагает смотреть на Хмельницкого прежде всего как на
казака – представителя природы «девственной ещё, детской, если угодно грубой».
Добавим сюда для полноты картины также шляхетское происхождение и образование –
чуть ли не лучшее по тем временам: иезуитский коллегиум во Львове либо Ярославе
(сейчас Польша). И твёрдость в вере, которой он не изменил, обучаясь у
католиков. Вероятно, именно такой и должна была быть сумма качеств руководителя
народного восстания, переросшего в Освободительную войну, принесшую вожделенное
воссоединение разорванных монголо-татарским нашествием частей Руси, прежде
единой.

Человек просвещённый, он знал, сколь печальные последствия имел опыт предыдущих
восстаний против польского гнёта: под предводительством Марка
Жмайло-Кульчицкого в 1625 году, Тараса Федоровича в 1630-м, Ивана Сулимы в
1635-м, Павлюка (Павла Бута) в 1637−1638-х,  Якова Острянина и Дмитрия Гуни в
том же 1638-м: за исключением первых двух случаев, когда судьба мятежных
гетманов осталась туманной, во всех других руководители восстаний и
присоединившаяся к ним казачья верхушка были свезены в Варшаву и там казнены
самыми лютыми способами: колесованием, посажением на кол, разрубанием на части,
вытягиванием жил, «прибитием гвоздями стоячими к доскам, облитым смолою, и
сожжением медленным огнём», «растерзанием железными когтями, похожими на
железную лапу» и т.д. Осознавал ли Богдан Хмельницкий перспективу в случае
поражения, пополнить своим именем этот длинный синодик жертв панской жестокости
и шляхетского произвола? Несомненно. Но он знал совершенно точно и о тех
многочисленных жертвах, которые терпел здесь и сейчас его народ от буквально
озверевших в своей лютости поляков, огнём и мечом укрощавших его бунт, топивших
его в крови. Прекрасно поставленная разведка доносила ему сведения о польских
расправах. В частности, Ярема Вишневецкий, «недавний отступник от православия,
с ненавистью ренегата к старой вере, вере хлопской», кричал палачам на
повсеместно творившихся расправах: «Мучьте их так, чтобы они чувствовали, что
умирают!» Лисянский полковник Максим Кривонос свидетельствовал: и головы
отсекали, и на кол сажали, «в каждом городе среди рынка виселица... а попам нашим
[князь Иеремия] буравом просверливал глаза». Девчатам поляки дарили красные
ленты: лоскуты кожи, сорванные с их плеч и шеи. Можно было спокойно читать
такие известия?

Да, это была и война за веру. Богдан Хмельницкий требовал, чтобы униаты и
католики убрались с Гетманщины, а православный епископ заседал в Сейме Речи
Посполитой на почётном месте, сразу после примаса Польши. Никогда поляки с этим
не соглашались. В откровенно горячую фазу религиозное противостояние перешло в
1651 году, когда папа римский Иннокентий Х послал к полякам своё благословение
и отпущение грехов, а короля Яна Казимира провозгласил защитником веры, передав
ему через своего легата мантию и освящённый меч. Коринфский митрополит Иоасаф,
в свою очередь, опоясал Хмельницкого мечом, освящённым на Гробе Господнем,
окропил войско святою водою и сам шёл на поляков при войске. Своё благословение
Богдану Хмельницкому передал и константинопольский патриарх Иоанникий II
(Линдиос).  Здесь мы говорим о событиях 1651 года, в котором Берестецкая битва,
состоявшаяся 18 июня, открыла ряд неудач казацкого войска того лета, имевших
достаточно тяжёлые последствия – поляками был опустошён взятый литовским
гетманом Радзивилом Киев, а «соборную церковь Богородицы каменную на посаде
ляхи разграбили всю, образа пожгли, церковь вся выгорела, одни стены остались».
Отчего такую ненависть питали поляки к Православию, поясняет тот факт, что
именно киевские иерархи церкви первыми осознали необходимость воссоединения
разделённого народа: ещё за четверть века до письма Богдана Хмельницкого о том
же, о принятии православного населения Малой Руси в подданство русского
государя, обращался (в 1622 году) православный митрополит Киевский и всея Руси
Исаия Копинский-Борисович. Двумя годами позже о том же писал в Москву
митрополит Иов (Борецкий). Он создал трактат «Протестация и благочестивая
юстификация», в котором было прямо заявлено: «с Москвой у нас одна вера и
богослужение, одно происхождение, язык и обычай».

В том, что это было не частное мнение князей церкви, убеждают книжные
источники, в том числе изданные Киево-Печерской (в лавре) типографией,
основанной в конце 1616-го или начале 1617 года её тогдашним архимандритом
Елисеем Плетенецким. Уже первая изданная здесь книга пестрит «российской»
лексикой: «Православному роду российскому, сыном Церкве Восточныа
взлюбленнейшим, душеспасителнаго здравиа», - напечатано в её предисловии от его
имени. Другая вступительная часть, от архидиакона Захарии Копыстенского,
начинается пояснением: «Се, правоверный христианине и всяк благоговейный
читателю, от нарочитых мест в России кииовских, сиречь от лаври Печерскиа
кииовскиа, благодатию Божиею, тщанием же и попечением преподобнейшаго во отцех
кир Елисеа Плетенецкого печерского кииовского архимандрита, происходит
напечатаннаа книга молебная Орологион, яже сказуется словенски Часослов». Далее
так: «Еже в роде нашем российском, во времена самодержца Владимера ...» В конце
«Часослова» помещены «Тропари и кондаки росским святым». В тропаре преподобный
Феодосий Печерский величается так: «...в молитвах яко бесплотен пребывая в
Росской земли яко светлое светило просия отче Феодосие...» и «Звезду росскую
днесь почтем». Тропарь и кондак святой княгине Ольге озаглавлены: «Преподобныа
Олги, в святом крещении Елены, княгини роской». В кондаке святой княгине
Ольге-Елене призывается: «Вспоем днесь благодателя всех Бога, прославльшаго в
Росии Елену богомудрую, молитвами ея, Христе, подаждь душам нашым грехом
оставление».

В разряде полемической литературы, изданной лаврской «друкарней», наиболее
значимой считается вышедшая в 1621 году «Книга о вере единой», автором которой
является, вероятно, Захария Копыстенский. В последующих во времени изданиях –
«Книга о правдивом единстве православных христиан», «Палинодия, или книга
обороны» – этот автор демонстрировал открытую симпатию московским князьям,
рассматривал Москву как оплот православной Руси, высказывал идеи единой
прародины трёх восточнославянских народов, неразрывности их исторических судеб,
близости разговорных языков, единства церковнославянского языка и
вероисповедания.

Поляки прекрасно понимали силу основы этого объединения. «Многоучёный и
красноглаголивый» Адам Кисель, в то время воевода киевский, главный
переговорщик с Хмельницким, уверяя последнего, что «нет в целом свете другого
государства, подобно нашему отечеству, правами и свободою», тем не менее честно
отписывал в Варшаву ещё в 1648 году: «...Кто может поручиться за них? Одна кровь,
одна религия!»

Широким ручьём польской крови, пролитой в возмездие за творимый в
Малороссии-Гетманщине разбой, Богдан Хмельницкий навеки отгородился от Польши
после битвы под Батогом. Ничего, кроме презрения, он к чванливой польской
шляхте не испытывал. Перина, Латина и Дитина - такие глумливые клички он дал
трём полководцам, стоявшим против него в битве под Пилявцами: князь Доминик
Заславський известен был богатством и изнеженностью, коронный хорунжий
Александр Конецпольський - молодостью, а региментарь Миколай Остроруг кичился
учёностью. Все они потерпели от него сокрушительное поражение. Стиль
Хмельницкого переняли и окружавшие его полковники. «Уже прошли те времена,
когда ляхи были нам страшны; мы под Пилявцами испытали, что это уже не те ляхи,
что прежде бывали. Это уже не Жолкевские, не Ходкевичи, это какие-то
Тхоржесвкие, да Заенчковские (Хорьковские – от хорька и Заечковские – от
зайца), дети, нарядившиеся в железо! Померли от страху, как только нас
увидели».

...Осенью 1653 года, через два месяца после Битвы под Батогом, Богдан Хмельницкий
снова, третий раз за время Освободительной войны обратился к Русскому царству
за протекторатом. 1 октября 1653 года в Москве специально по этому поводу был
собран Земский собор, оказавшийся последним в истории России, на котором было
решено: «гетмана Богдана Хмельницкого и все Войско Запорожское з городами и з
землями [в подданство] принять...». Так завершилась через пять лет и три с
половиной месяца эпопея одного из самых замечательных писем в мировой истории.
