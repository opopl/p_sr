%%beginhead 
 
%%file 10_01_2023.fb.fb_group.ukrainska_poezia_i_proza.1.sv_tl_i_pam_yat__mog
%%parent 10_01_2023
 
%%url https://www.facebook.com/groups/255425469158429/posts/758677292166575
 
%%author_id fb_group.ukrainska_poezia_i_proza
%%date 10_01_2023
 
%%tags 
%%title Світлій пам'яті мого односельчанина, сусіда Івана Варгаса. Спочивай з миром, друже
 
%%endhead 

\subsection{Світлій пам'яті мого односельчанина, сусіда Івана Варгаса. Спочивай з миром, друже}
\label{sec:10_01_2023.fb.fb_group.ukrainska_poezia_i_proza.1.sv_tl_i_pam_yat__mog}

\Purl{https://www.facebook.com/groups/255425469158429/posts/758677292166575}
\ifcmt
 author_begin
   author_id fb_group.ukrainska_poezia_i_proza
 author_end
\fi

Світлій пам'яті мого односельчанина, сусіда Івана Варгаса. Спочивай з миром, друже

\obeycr
Я нарешті вернувся до рідного дому, 
Попри тисячі верст,сотні днів і ночей,
Не відчувши,на диво,щоденної втоми.
Тут зустріло мене так багато людей .
Де не глянеш,навкруг односельці і рідні,
Ті,з якими прожив все свідоме життя .
Ще до того,коли рубежі наші східні 
Боронити пішов,як почалась війна.
Але радості в погляді їхнім немає, 
Очі сповнені сліз.Де Різдва торжество?
Вже провідна зоря в небі тихо згасає
Із собою забравши усе таїнство.
Не відбулося чуда.Заснув.Не устану.
На щиті побратими принесли мене.
Не загояться вже незагоєні рани,
Й друзі вже не підставлять широке плече .
Та залишаться ці віковічні простори
І зелені Карпати,що сплять у імлі.
Тож коли до країни ввірвалося горе,
Не вагався ні хвилі я щодо борні.
Не залишив я роду міцного коріння,
Не судилось мабУть .Шлях мій стелиться ввись.
В естафету прийміть серця мого горіння.
У молитві згадайте хоч часом,колись.
І прожийте життя,гідне честі і слави.
Те,яке вже не вдасться прожити мені,
Бо віддав я його в ім'я неньки-держави,
Щоби світло пролилось в прийдешні всі дні.
Olena Kindratyszyn
\restorecr
