% vim: keymap=russian-jcukenwin
%%beginhead 
 
%%file 25_05_2019.stz.news.ua.mrpl_city.1.guljanie_na_bolshoj_v_mariupole
%%parent 25_05_2019
 
%%url https://mrpl.city/blogs/view/gulyanie-na-bolshoj-v-mariupole
 
%%author_id burov_sergij.mariupol,news.ua.mrpl_city
%%date 
 
%%tags 
%%title Гуляние на Большой в Мариуполе
 
%%endhead 
 
\subsection{Гуляние на Большой в Мариуполе}
\label{sec:25_05_2019.stz.news.ua.mrpl_city.1.guljanie_na_bolshoj_v_mariupole}
 
\Purl{https://mrpl.city/blogs/view/gulyanie-na-bolshoj-v-mariupole}
\ifcmt
 author_begin
   author_id burov_sergij.mariupol,news.ua.mrpl_city
 author_end
\fi

\ii{25_05_2019.stz.news.ua.mrpl_city.1.guljanie_na_bolshoj_v_mariupole.pic.1}

Летними вечерами, когда спадала жара, вся молодежь, да и не только молодежь,
проживающая в старой части Мариуполя, устремлялась на \textbf{проспект Республики},
который для старшего поколения был \textbf{\enquote{Большой улицей}} или проще – \textbf{\enquote{Большой}}.
Улица эта была действительно самой большой в нашем благополучном городе, как по
протяженности, так и по ширине. Это четко видно на старинных картах. Можно не
сомневаться, что имя ей дал народ. Для поколения, подросшего в послевоенные
годы, - \textbf{\enquote{Проспектом}}, а для завсегдатаев с претензией на некоторую
исключительность – \textbf{\enquote{Бродом}}, т.е. \textbf{Бродвеем}. Однако употребление каждого из этих
названий было вполне понятно каждому мариупольскому аборигену. Исключение могли
составлять разве что совсем древние старушки, которые, услышав от внуков слово
\enquote{брод}, приставляли ладошку к уху, в большей степени сохранившему свою функцию,
переспрашивали: \emph{\enquote{Брод? Это не тот ли, что через Кальмиус за Правым берегом, на
дороге, ведущей на Донщину?}} Тут уж молодое поколение приходило в недоумение:
причем тут Кальмиус, Донщина, до которой еще надо было ехать километров двести,
если не больше. Им было невдомек, что их бабушки подразумевали под Донщиной
земли за Кальмиусом, которые в 30-е годы начали застраиваться цехами завода
\enquote{Азовсталь}, а со времен царствования Елизаветы Петровны до установления
советской власти принадлежали Всевеликому войску донскому.

Нравы и обычаи на Большой годами не менялись. Изменения наблюдались только в
одежде и, естественно, в смене поколений. Гулянье происходило по одной стороне,
чаще всего - по нечетной, от Торговой улицы и до сквера. Его участники
разделялись на два четко обозначенных медленно текущих людских потока: один -
вверх, к скверу, другой - вниз, в обратную сторону. В иной год потоки,
повинуясь неведомым закономерностям, перемещались на четную сторону.

\ii{25_05_2019.stz.news.ua.mrpl_city.1.guljanie_na_bolshoj_v_mariupole.pic.2}

Важной составляющей гуляющих были барышни, взявшие друг дружку \enquote{под крендель},
а также следовавшие за ними воздыхатели, перемещающиеся небольшими группами.
Проспект был местом встреч и знакомств. Когда знакомство происходило, вновь
образовавшаяся парочка оставалась там относительно короткое время, далее она
отправлялась по предложению ухажера либо в кино, либо кавалер провожал свою
избранницу домой. Осторожные, а главное, дальновидные молодые люди прежде, чем
развить свое ухаживание, пытались окольными путями выяснить, где проживает
приглянувшаяся ему барышня. Могло случиться, что на Слободке, в Парковом
поселке или в недрах Новоселовки. Провожание в эти благословенные местности
было чревато непредсказуемыми и, мягко выражаясь, не очень приятными
последствиями: можно было оказаться и \enquote{слегка} побитым за вторжение на чужую
территорию. Здесь надо было решать. Но бывало и так, что \enquote{любовь с первого
взгляда} заставляла представителей сильного пола пренебречь всеми
предосторожностями. Время от времени сквозь чинно перемещающуюся людскую массу
шныряли молодые люди, разыскивающие своих товарищей, а иногда - девушку,
понравившуюся в каком-то другом месте города, с которой из-за мимолетности
встречи не удалось познакомиться сразу...

Особенно большое скопление молодежи наблюдалось вечерами накануне выпускных
экзаменов по русской литературе в средних школах. Озабоченные юноши и девушки
перемещались по проспекту, периодически собираясь в небольшие группки,
перешептывались. Потом группки распадались и вновь собирались, частично или
полностью меняя состав. Целью этого броуновского движения молодых людей было
узнать темы экзаменационных сочинений. Эти темы присылали в скрепленных
сургучными печатями конвертах из областного центра, а может быть и из Киева,
вскрывались они непосредственно перед испытанием десятиклассников на прочность
знаний пройденного курса русской литературы, а заодно и грамотности. Но у
каждого поколения выпускников теплилась надежда, что кто-то заранее вскрыл
заветный конверт и знает пока еще \enquote{секретную} для других тему. Каждый год
надежда не сбывалась, и все-таки на следующий год уже новое поколение
десятиклассников спешило вечером накануне экзамена на Большую.

\textbf{Читайте также:}

\href{https://mrpl.city/news/view/poslednij-zvonok-v-mariupole-tantsevalnye-fleshmoby-slezy-na-glazah-shariki-v-nebesah-foto-plusvideo}{%
Последний звонок в Мариуполе: танцевальные флешмобы, слезы на глазах, шарики в небесах, mrpl.city, 24.05.2019}

В начале июля вечерами появлялись на Броде студенты, обучающиеся в Харькове,
Ленинграде или в Москве, успевшие уже в первые дни пребывания дома обгореть на
пляже. Встречались со своими одноклассниками, обменивались столичными
новостями, щеголяли модными словечками и китайскими темно-синими бостоновыми
брюками, которые в будущем носили, чуть ли не до пенсии. Удивлялись, что в
местном вузе, техникуме, медицинском и педагогическом училищах, а тем более в
средних школах, еще запрещают на вечерах крутить пластинки с танго и
фокстротами, а танцевать разрешено было падеспань, польку и вальс.

Несмотря на большое скопление публики, нельзя сказать, что было шумно во время
гулянья, если не считать шарканья о тротуар тысяч ног да грохота и звона
трамваев. Но трамваи в вечернюю пору появлялись относительно редко. \emph{\enquote{Какие
трамваи?}} - удивленно спросит наш современник. Ответим - обыкновенные. По
Проспекту до середины пятидесятых годов была проложена двухколейная трамвайная
линия. Она связывала порт, железнодорожный вокзал и, естественно, центр города
с Ильичевским районом, где и заканчивалась у рынка.

Конечно, мариупольцы занимались на главной улице не только тем, что фланировали
взад-вперед. Там было несколько притягательных мест для иных занятий.
Во-первых, кино. У парадных дверей, а еще больше у касс кинотеатров - один из
них, \enquote{Победа}, пока существует и по сей день, а второй, \enquote{Родина}, канул в Лету
- постоянно толпились люди. А незадолго до начала сеанса за несколько десятков
метров от входа в храм кинематографа прохожих одолевали вопросом: \emph{\enquote{Нет ли
лишнего билетика?}}. Кинотеатр \enquote{Победа} был притягательным не только тем, что на
его экране демонстрировались новые кинокартины, но и возможностью перекусить
перед сеансом в буфете пирожным, украшенным розочкой и прилепленным к нему
зеленым листочком, запивая ситро. А потом можно было подняться на второй этаж и
послушать небольшой ансамбль, состоящий из скрипача Александра Мухина,
пианистки Галины Соболевской, виолончелиста Николая Никаро-Карпенко и других
местных одаренных музыкантов.

Во-вторых, погребок на углу проспекта Республики и улицы Советской. В наши дни
его вход, кажется, обозначен вывеской \enquote{Мебель}, а раньше на ее месте
красовалась другая, на которой четко было выведено: \enquote{Главвино}. В просторечии
же погребок именовался \enquote{Зверинцем}. Это название прилепилось к нему из-за
скульптурных изображений львиных голов, прикрепленных к стене. Из пастей львов
торчали бронзовые краны, открывая их, \enquote{хозяйки} заведения источали в стаканы
вино, красное или белое, в зависимости от вкусов и пристрастий жаждущих. Для
одних погребок был источником радости, а для городских властей - причиной
головной боли. Переусердствовавшие поклонники напитка, воспетого Омаром
Хайямом, отнюдь не украшали главный проспект города, тем более, когда они
устраивались \enquote{отдыхать} под скамейки, установленные неподалеку. Сейчас уже не
вспомнить, когда \enquote{Главвино} было перенесено в другой погребок, на угол улицы
Донбасской на пересечении с Советской. Надпись на вывеске стала короче –
\enquote{Вино}. Местные острословы расшифровывали это слово так: \emph{\enquote{Всесоюзный институт
народного образования}}.

В-третьих, Клуб металлургов, известный нашим современникам, как \href{https://mrpl.city/blogs/dk-molodezhnyj}{Дворец культуры
\enquote{Молодежный}}. Там тоже можно было посмотреть кинофильм, правда, не первой
свежести, побывать на спектакле народного театра или концертах самодеятельного
хора, танцевального коллектива или оркестра народных инструментов. А уж, если
сильно повезет, то - попасть на встречу с какой-нибудь знаменитостью, например,
с композиторами Юрием Милютиным или Дмитрием Кабалевским, киноартистами Борисом
Андреевым или Марком Бернесом. Для любителей поплясать в нижнем фойе
устраивались танцы.

Летние гулянья по главной улице города не были феноменом, присущим только
Мариуполю. Вспомним главные улицы других южных городов: Дерибасовскую в Одессе,
Большую Садовую в Ростове-на-Дону. Трудно сказать - сохранился ли этот обычай в
упомянутых выше городах, но у нас он исчез. Почему? Неведомо никому.

\vspace{0.5cm}
\begin{minipage}{0.9\textwidth}
\textbf{Читайте также:} 
	
\href{https://archive.org/details/09_06_2018.sergij_burov.mrpl_city.park_kultury_i_otdyha_sezon_1958_goda}{Парк культуры и отдыха: сезон 1958 года, Сергей Буров, mrpl.city, 09.06.2018}
\end{minipage}
\vspace{0.5cm}
