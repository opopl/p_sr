% vim: keymap=russian-jcukenwin
%%beginhead 
 
%%file 17_08_2021.fb.marshalok_igor.ukraina.1.bozhe_ukrainu_hrani
%%parent 17_08_2021
 
%%url https://www.facebook.com/permalink.php?story_fbid=1161631821031800&id=100015549784426
 
%%author_id marshalok_igor.ukraina
%%date 
 
%%tags nezalezhnist,ukraina
%%title Боже, великий, єдиний, нам Україну храни!
 
%%endhead 
 
\subsection{Боже, великий, єдиний, нам Україну храни!}
\label{sec:17_08_2021.fb.marshalok_igor.ukraina.1.bozhe_ukrainu_hrani}
 
\Purl{https://www.facebook.com/permalink.php?story_fbid=1161631821031800&id=100015549784426}
\ifcmt
 author_begin
   author_id marshalok_igor.ukraina
 author_end
\fi

Боже, великий, єдиний, нам Україну храни!

24 серпня 1991 року увійшло в історію нашої держави як дата проголошення
Незалежності України. У світі народилася нова велика держава – наша Україна. 30
років тому від вічнозелених Карпат до сивого Дніпра мільйони голосів сказали:
«Так, бути Незалежній Україні». Як на довгій ниві життя людського, чимало
довелося перетерпіти нашій молодій державі: і грози, і сльози, і болі, і
радощі, і успіхи, і невдачі, гібридну війну і Covid-19. Але український народ
вірить, що попереду ще чимало щасливих і радісних днів, незважаючи на тимчасові
труднощі.

\ifcmt
  pic https://scontent-frt3-1.xx.fbcdn.net/v/t39.30808-6/239468156_1161631737698475_5599982894684159847_n.jpg?_nc_cat=108&ccb=1-5&_nc_sid=8bfeb9&_nc_ohc=os2aDB9s5HkAX-l35b2&_nc_ht=scontent-frt3-1.xx&oh=0ecac517236da034f2a17a7a69347c0e&oe=61A409F2
  @width 0.8
\fi

Україна святкує День Незалежності. До цієї події наш народ йшов чотири
століття. Було все: війни, зради, заколоти, підступність ворогів і сусідів, але
ми вистояли, бо народ, який боровся за свою волю, має право жити на своїй
власній квітучій і рідній землі.

Цій світлині – 31 рік. На ній бачимо ТИТАНІВ українського державотворення;
Вячеслава Черновола і Богдана Котика, Івана Геля і Ростислава Братуня, Романа
Лубківського і Сашка Кривенко, Василя Шпіцера, брати Горині і майже весь
національно свідомий Львів.

Так все розпочиналось. Всі знаємо  коли святкувати народження суверенної
української держави, та не всім відомо, що так було не завжди. Спочатку датою
торжества обрали 16 липня, саме в цей день відбулося прийняття Декларації про
державний суверенітет України. Такий документ підписали депутати Верховної
Ради УРСР у 1990 році.

Ухваленням \enquote{Декларації про державний суверенітет України},
проголосованої Верховною Радою України 1990 року «під завісу перебудови в
СРСР», стало початком новітнього конституційного процесу в Україні.

Мрія мати власну державу знову визрівала роками у творчості
діячів культури-шістдесятників, у політичних акціях протесту
правозахисників та дисидентів, у прагненні мільйонів українців.
Спроба у Москві 19 серпня 1991 року «оновити» Радянський Союз у
сталінських традиціях стала поштовхом до розгортання
національно-демократичного руху за незалежність у більшості
республік СРСР. Українці об’єдналися у масові мітинги з вимогою
виходу зі складу СРСР.

Невдовзі відбулися видатні історичні події – проголошення незалежності
України 24 серпня 1991 року та всенародний референдум 1 грудня 1991 року, що
підтвердив проголошений Акт незалежності України.

На підтвердження Акту проголошення незалежності Верховна Рада України вирішила
провести 1 грудня 1991 р. республіканський референдум. Він був потрібен, щоб
нейтралізувати політичні спекуляції противників української незалежності,
особливо у східних і південних областях республіки, які заявляли, що народ
буцімто не підтримує Акт про незалежність. Союзне керівництво на чолі з
президентом СРСР М. Горбачовим, не втрачаючи надій на укладення нового союзного
договору, вело активну роботу в цьому напрямі. Крім того, світове
співтовариство не поспішало з визнанням самостійності України, вичікуючи, як
розгортатимуться події. 

На всеукраїнському референдумі кожен громадянин мав
чітко відповісти «Так, підтверджую», або «Ні, не підтверджую» на питання: «Чи
підтверджуєте Ви Акт проголошення незалежності України?» Із 37885,6 тис.
громадян України, котрі були внесені до списків для таємного голосування, взяли
участь у голосуванні 31891,7 тис. (84,18\%). Із них позитивно відповіли 28804,1
тис. виборців (90,92\%). Зокрема, у Криму відповіли на запитання референдуму
«Так, підтверджую» 54,19\% громадян, у Севастополі — 57,07\%; у Донецькій,
Луганський, Одеській, Харківській областях підтвердили Акт про незалежність
понад 80\% виборців; в Івано-Франківській, Львівській, Тернопільській,
Волинській, Рівненській, Житомирській, Київській, Хмельницькій, Черкаській,
Вінницькій областях за незалежність проголосувало понад 95\%, у решті областей —
понад 90\% громадян. За результатами референдуму вже ні в кого не могло бути
сумніву, чи український народ хоче мати свою самостійну державу. 

Втретє за 350 років від Визвольної війни середини XVII ст. Україна здобула
самостійність.

Головним завданням стало збереження незалежності й суверенності Української
держави, недопущення помилок, зроблених на двох попередніх етапах існування
державності, — у середині XVII ст. і в 1917—1920 рр

Тож мусимо бути мужніми, пильними і завбачливими, щоб не пустити до нашого
порога тих, хто ще й досі ненавидить нашу молоду державу, нашу мову, та,
зрештою, і нас, українців. Чи вистоїмо? Повинні. Мусимо. Все залежить від нас.
Намагаймося бути єдиними в своїх помислах і діях, коли позбудемося рабської
покірливості, поступливості та байдужості.

Поки що живемо у скруті, наші фабрики і заводи простоюють. Розкрадено,
розділено, розтринькано все те багатство, яке своїми руками створили діди і
батьки. Але і в нас є шанс на порятунок, на вихід з кризи, на те, що людина
праці одержуватиме  гідну зарплатню, що не даватимемо хабарів чиновникам, що
наші можновладці не будуть думати тільки про себе, а хоч трішки про народ. На
жаль, сьогоднішні події в державі засвідчують, що Верховна Рада не готова чи не
хоче стояти на захисті людини, а отже, зневажається основний закон - Конситуацію. 
Для них закон – одне, для нас, виходить, інше.

Чому людина, яка чесно працює, знаходиться за межею бідності, а тим часом у
Парламенті більше 250 мільйонерів?Працювати є над чим.   Мрія українців про
квітучу, розвинуту, багату державу гаснутиме доти, допоки наша земля не
виплекає української національно-свідомої  влади.

Ніхто нам не забезпечить успішне майбутнє - тільки ми цього повинні домогтися
самі. Віриться, що на цей раз ми достойно вийдемо із складної ситуації нашого
сучасного становища. Лишень тоді, сподіваюся, почнеться відродження національної
свідомості і духовності, відродження української землі, мови.

Перед Незалежною, суверенною державою Україна великий, але не вторований
шлях. Тому труднощів не оминути. Але крок, зроблений 30 років тому - 24
серпня 1991 року   відкрив перед нами великі перспективи. Наша місія – не
втратити того, що маємо і досягти більшого. Маємо пам’ятати тих, хто був у
витоків Незалежності,цінувати працю шахтаря, який, наче на фронт,
спускаються в шахти, щоб прогодувати свої сім’ї; селянина, який продовжують
вирощувати хліб на своїй землі;воїна-захисника,який і вдень і вночі
відстоює і боронить нашу Незалежність, молодь, яка хоче працювати заради
добробуту власних родин, а отже – країни загалом, матерів і дружин, які
працюють поза кордонами і також примножують бюджет держави, маємо дбати і
про середній бізнес, який  продовжує гнутися під тиском фіскальних органів.
Розрив між бідними і багатими за роки незалежності стабільно росте,
високотехнологічна виробнича галузь майже не модернізується, натомість
колосальні прибутки приносить не виробництво, а посередницькі і фінансові
операції.

30 років – це розквіт молодості, а молодість – це повнота життя. Це свято –
привід для кожного з нас, незалежно від віку, досвіду, політичних переконань,
відчути себе молодим, а отже – сповненим рішучості та енергії до праці та
самоствердження. Україна – це всі ми, зі своїми турботами, орієнтирами,
перемогами та невдачами. Сьогодні просимо одного: «Боже, великий, єдиний, нам
Україну храни! Дай нам віри й волі, дай нам терпіння й одностайності, дай нам
любові, злагоди і взаєморозуміння.»

І наостанок. Нехай кожен із нас у День Незалежності принесе в офіру молитву за
рідну землю, за її розквіт, щастя та долю, за усвідомлення, що кожен із нас є
частиною українства. А всім людям доброї волі бажаю мирного неба, здоров'я і
друзів тих, що треба!

Ігор МАРШАЛОК

Р.S. На сайті Мінфіну з явилася інформація, що на заходи, які намічаються на
честь святкування 30-ї річниці Незалежності України планується  витратити з
бюджету не менше 4,5 млрд.грн. Як на мене, а чи не доцільніше б була частину
коштів скерувати на погашення заборгованості по зарплатні, чи скерувати їх на
придбання якісної вакцини проти COVID-19?

\ii{17_08_2021.fb.marshalok_igor.ukraina.1.bozhe_ukrainu_hrani.cmt}
