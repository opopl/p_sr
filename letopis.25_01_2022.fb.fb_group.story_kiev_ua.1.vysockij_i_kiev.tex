% vim: keymap=russian-jcukenwin
%%beginhead 
 
%%file 25_01_2022.fb.fb_group.story_kiev_ua.1.vysockij_i_kiev
%%parent 25_01_2022
 
%%url https://www.facebook.com/groups/story.kiev.ua/posts/1847477845449008
 
%%author_id fb_group.story_kiev_ua,denisova_oksana.kiev.ukraina.gid
%%date 
 
%%tags kiev,vysockij_vladimir.bard
%%title Высоцкий и Киев
 
%%endhead 
 
\subsection{Высоцкий и Киев}
\label{sec:25_01_2022.fb.fb_group.story_kiev_ua.1.vysockij_i_kiev}
 
\Purl{https://www.facebook.com/groups/story.kiev.ua/posts/1847477845449008}
\ifcmt
 author_begin
   author_id fb_group.story_kiev_ua,denisova_oksana.kiev.ukraina.gid
 author_end
\fi

Этот мальчик на фотографии – Володя Высоцкий. Здесь ему 12 лет и в этом
возрасте он впервые приехал в Киев к своей бабушке, которая много лет прожила в
квартире на улице Франко. 

Бабушка Высоцкого была удивительной женщиной, она любила менять имена и
умудрилась даже изменить дату своего рождения. Когда-то юная Дебора Бронштейн
приехала в Киев из Житомира и вышла замуж за Вольфа (Владимира) Высоцкого. В
свидетельстве о рождении она была записана как Дебора и дата ее рождения 1891
год. Но в свидетельстве о браке она уже фигурировала как Дора и дату рождения
она немного изменила и стала чуть-чуть моложе – 1893 год. Она себя называла то
Ириадой, то Ириной, была она женщиной удивительно веселой и остроумной.
Работала она косметологом в салоне на Крещатике и ее очень любили клиентки – за
красоту и веселый нрав, за умение всегда быть в хорошем настроении, за любовь к
розыгрышам. С мужем она развелась, и старший сын Семен воспитывался в Москве у
отца, а младший – Алексей – жил с ней в Киеве. Оказавшись в оккупированном
Киеве, понимая, что евреям грозит реальная опасность, она тоже не растерялась и
умудрилась за несколько дней оформить церковный брак со своим вторым мужем и
стала Дарьей Алексеевной Семененко.

Володя очень любил приезжать в Киев к своей удивительной бабушке. И она,
кстати, была единственной в семье, кто его поддержал, когда Высоцкий решил
бросить Московский инженерно-строительный институт и поступить в школу-студию
МХАТ. Бабушка умерла в 1970 году на руках у внука, который почувствовав что-то
неладное, прилетел со съемок в Одессе и провел последний день с бабушкой.

А впервые на гастроли в Киев Высоцкий приехал с театром в 1971 году и мне
всегда обидно, что она чуть-чуть не дожила и не увидела любимого внука на
театральной сцене. А Владимир Высоцкий всегда говорил, что его чувство юмора и
умение и желание хохмить – это у него от его удивительной киевской бабушки!

С днем рождения, Владимир Высоцкий!
