% vim: keymap=russian-jcukenwin
%%beginhead 
 
%%file 13_09_2021.fb.nikonov_sergej.4.bilchenko_lekcia
%%parent 13_09_2021
 
%%url https://www.facebook.com/alexelsevier/posts/1581046488907327
 
%%author_id nikonov_sergej,bilchenko_evgenia
%%date 
 
%%tags bilchenko_evgenia,lekcia,nauka
%%title Евгения Бильченко -  Моя самая потрясающая лекция на открытом воздухе
 
%%endhead 
 
\subsection{Евгения Бильченко -  Моя самая потрясающая лекция на открытом воздухе}
\label{sec:13_09_2021.fb.nikonov_sergej.4.bilchenko_lekcia}
 
\Purl{https://www.facebook.com/alexelsevier/posts/1581046488907327}
\ifcmt
 author_begin
   author_id nikonov_sergej,bilchenko_evgenia
 author_end
\fi

Жизнь продолжается. Преподаватель Евгения Витальевна Бильченко провела лекцию
перед 1 студентом... А я об этом сообщаю. Молчаливый наборщик текста.  К
оригинальному тексту приложена песня на стихи Егора Летова ...ть на мое лицо.
Фейсбук не позволяет добавлять аудиозаписи.

Евгения

\ii{13_09_2021.fb.nikonov_sergej.4.bilchenko_lekcia.pic}

Моя самая потрясающая лекция на открытом воздухе. Рассказывала студенту про
русскую семиотическую школу и европейский антиглобализм. Как встарь... Только
вместо кафедры - скамейка, вместо коридора - парк, вместо зарплаты - вкусный
суп, вместо крыши аудитории - пропитавший нас до нитки холодный осенний ливень.
Было радостно и спокойно. Обнаружила, что сейчас в вузах Украины нельзя в
работах по театру упоминать Станиславского, потому что он русский.

Декан пединститута имени Сергея Стерненко любил пугать нас, что он нас выгонит,
и мы будем читать лекции на бульваре Шевченко. Иван Иванович, не на бульваре, а
в парке КПИ, где стоит ещё советская техника со звездами, которую вы разрушили,
а нового ничего не построили.

Да пребудут с нами Якобсон, Лотман, Межуев и Тульчинский! \#выкуси

\ii{13_09_2021.fb.nikonov_sergej.4.bilchenko_lekcia.cmt}
