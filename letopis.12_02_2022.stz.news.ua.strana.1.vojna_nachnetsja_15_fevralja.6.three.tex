% vim: keymap=russian-jcukenwin
%%beginhead 
 
%%file 12_02_2022.stz.news.ua.strana.1.vojna_nachnetsja_15_fevralja.6.three
%%parent 12_02_2022.stz.news.ua.strana.1.vojna_nachnetsja_15_fevralja
 
%%url 
 
%%author_id 
%%date 
 
%%tags 
%%title 
 
%%endhead 

\subsubsection{3. Контролируемый конфликт}
\label{sec:12_02_2022.stz.news.ua.strana.1.vojna_nachnetsja_15_fevralja.6.three}

Третий вариант развития событий - управляемый конфликт или \enquote{военный
договорняк}. Когда состоится \enquote{незначительное вторжение} России (например, с
введением войск на Донбасс, а затем и неких половинчатых санкций) и в качестве
итога - возвращение к сценарию №1 и выходу на какое-то глобальное соглашение с
Москвой.

После чего Байден с лаврами миротворца, предотвратившего ядерную войну, летит,
как голубь мира, домой и триумфально заводит демократов в Конгресс.

\enquote{Незначительное вторжение} - этот фраза самого Байдена, которую тот
неосторожно использовал на одной из своих пресс-конференций. Он дал понять, что
в таком случае и санкции будут меньше. 

Тогда эта фраза вызвала огромное недоумение и скандал, а в Белом доме сказанное
\enquote{сонным Джо} начали опровергать. Но нельзя исключать, что президент
случайно проговорился и выдал истинные планы. 

Впрочем, у этого сценария тоже есть ограничитель - высокий уровень взаимного
недоверия сторон. Без доверия сложно разыграть такую комбинацию - на каждом
этапе высока вероятность, что партнер может элементарно \enquote{кинуть}. 

Например, после \enquote{незначительного вторжения} ввести весьма значительные
санкции и никаких соглашений с Путиным не подписывать. 
