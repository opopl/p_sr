% vim: keymap=russian-jcukenwin
%%beginhead 
 
%%file 03_12_2021.fb.tolkachev_aleksej.1.299_rokiv_skovoroda.cmt
%%parent 03_12_2021.fb.tolkachev_aleksej.1.299_rokiv_skovoroda
 
%%url 
 
%%author_id 
%%date 
 
%%tags 
%%title 
 
%%endhead 
\subsubsection{Коментарі}

\begin{itemize} % {
\iusr{Альбіна Паренюк}
Я б приєдналась!

\iusr{Serhiy Kandul}
Петр Приступов, подаймо свої пропозиції!

\iusr{Тарас Єрмашов}

Можна зробити виставку (артефактів, видань і т.п.), аналогічну до зробленої
минулої зими в \enquote{Укр. домі} на честь 150-річчя Лесі Українки - 

\href{https://www.facebook.com/events/779040659377151/}{%
Мистецький проєкт «Леся Українка: 150 імен»%
}

\begin{multicols}{2}

Відкриття ювілейного року поетеси та драматургині відбудеться в Українському
Домі та включатиме: експозицію, перформанс, змістовні дискусії та стане
справжнім культурно-медійним епіцентром.  ⠀

І саме на експозиції ви зможете вперше побачити повне нецензуроване видання
творів Лесі Українки у 14 томах.
⠀
З нагоди 150-річчя на Мистецькому проєкті вас чекає:

@igg{fbicon.round.pushpin}  Косач Talks — Інтелектуально-дискусійна програма в
межах проєкту «Леся Українка: 150 імен», що триватиме з 25 по 28 лютого
включно.  Часові слоти: 13:00, 15:00, 17:00, тривалість — півтори/дві години
Модератори: Анастасія Євдокимова, Богдана Романцова, Євгеній Стасіневич
Куратор: Євгеній Стасіневич

@igg{fbicon.round.pushpin}  Трансляція Re:post-опери «LE» від Влада Троїцького та музичної формації NOVA OPERA

@igg{fbicon.round.pushpin}  Особисте життя Лесі Українки: дитинство, родина, почуття, подорожі, дружба.

@igg{fbicon.round.pushpin}  Лінія життя — основні дати життя і творчості поетеси, культурні і політичні події у світі.

@igg{fbicon.round.pushpin}  Образи Лариси Косач-Квітки: галерея портретів, скульптур.
⠀
За детальною афішею слідкуйте на офіційній сторінці Міністерства культури та
інформаційної політики України, Державного агентства України з питань мистецтв
та мистецької освіти та Національного центру «Український дім» ⠀

Чекаємо вас

25 лютого вхід за запрошеннями.
З 26 лютого по 8 березня 2021 року для всіх бажаючих.

Розклад роботи:

будні з 11.00 до 20.00,

вихідні з 10.00 до 20.00.

@igg{fbicon.collision}  Вхід безкоштовний за реєстрацією на сайті \url{150imenlesi.org/registration}

@igg{fbicon.collision}  Реєстрація для організованих груп \url{https://bit.ly/3ssvu3j}

Український дім, вул. Хрещатик 2

Просимо вас звернути увагу, що під час проведення культурних подій продовжують
діяти карантинні обмеження, яких необхідно дотримуватися: одягати распіратор
або захисну маску, що закриває ніс та рот; дотримуватися соціальної дистанції
між людьми, не більше однієї особи на 4 кв. м площі та дотримуватися заходів
особистої гігієни.

\end{multicols}


\iusr{Алена Яремчук}

Я б хотіла приєднатись. Дуже цікаво. Читала сценарій про його життя в \textbf{Коронація
Слова}, і булав захваті ... Треба знімати кіно.

\iusr{Алена Яремчук}
Онлайн флеш моб видатних зірок блогерів України з його висловлюваннями...

\iusr{Mike Kaufman-Portnikov}
Я готовий!

\end{itemize} % }
