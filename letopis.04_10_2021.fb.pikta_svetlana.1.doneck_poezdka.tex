% vim: keymap=russian-jcukenwin
%%beginhead 
 
%%file 04_10_2021.fb.pikta_svetlana.1.doneck_poezdka
%%parent 04_10_2021
 
%%url https://www.facebook.com/svetilkin.lanna/posts/4784953621537048
 
%%author_id pikta_svetlana
%%date 
 
%%tags donbass,doneck,pikta_svetlana,poezdka,reportazh,vojna
%%title Это мой третий приезд в послевоенный Донецк
 
%%endhead 
 
\subsection{Это мой третий приезд в послевоенный Донецк}
\label{sec:04_10_2021.fb.pikta_svetlana.1.doneck_poezdka}
 
\Purl{https://www.facebook.com/svetilkin.lanna/posts/4784953621537048}
\ifcmt
 author_begin
   author_id pikta_svetlana
 author_end
\fi

1
- Куда ты пристёгиваешься?? - смеётся волонтёр Андрей, - Как ты выпрыгивать будешь из машины?
Недоумённо смотрю на него. "Выпрыгивать? Из машины?" 
Пытаюсь понять, шутит Андрей или нет
- Ну да, если попадём под обстрел, как ты будешь выпрыгивать?

Андрей хохочет, по-доброму, я никогда не была ещё в красной зоне, он это знает.
У меня по хребту пробегает холодный электрический разряд, я вдруг понимаю: всё
серьёзно. 

\ifcmt
  ig https://scontent-lga3-2.xx.fbcdn.net/v/t1.6435-9/244415074_4784911161541294_1045493111560880570_n.jpg?_nc_cat=100&ccb=1-5&_nc_sid=8bfeb9&_nc_ohc=_7t_33fx83kAX-O6VHU&_nc_ht=scontent-lga3-2.xx&oh=3c51379d5a91e821adaa33a68efb5b82&oe=6182A2EB
  @width 0.4
  %@wrap \parpic[r]
  @wrap \InsertBoxR{0}
\fi

Это мой третий приезд в послевоенный Донецк. Добиться , чтобы взяли на линию
разграничения официально, как корреспондента, мне не удалось. Два прошлых
приезда в этом смысле прошли даром. Конечно, общение с дончанами бесценно и
приятно было видеть, что город, на окраинах которого ад, сажает розы и
поддерживает центр и главные улицы в идеальной чистоте, но хотелось увидеть
главное: что творится там, под ежедневными обстрелами в течение семи лет. И
поэтому у меня созрело решение: ехать с волонтёрами, как частное лицо. 

Дальше последовали два дня, за которые я смогла прожить десять жизней -
столько, сколько было "подопечных" в эти два дня у волонтёра Андрея, и
переосмыслить одну - свою. 

Итак, мы едем в сторону Петровки и Александровки. Это тот же Донецк, буквально 20 минут от центра. 
- Не нервничай! Всё будет хорошо. - успокаивает меня Андрей. - Я же слежу за сводками. 
"Разве заметно, что я нервничаю?" - невничаю я. 
Ещё задолго до блокпоста дорога пустеет, машины попадаются всё реже.  Я начинаю замечать, что фактически все ворота здесь - в дырках от осколков. Толстый металл дырявый, как паутина. Неужели нет ничего прочного в этой жизни?
Мы везём продуктовые наборы и деньги тем людям, которые не могут по каким-то причинам выехать из опасной зоны. 
Первый адрес на улице  Мамина-Сибиряка. Украинские силы отсюда в 700 метрах. Въезжаем в густые заросли амброзии. Навстречу выскакивает бабушка Валентина, сухая и лёгкая, машет на нас руками, сюда, мол, нельзя. 
- Вы к моему дому по этой улице не подъезжайте. Вот сюда сверните. Тут вас нацики заметят. 
Бежит впереди машины, показывая дорогу. 
- Может я выйду и пойду, а она сядет? Мне неловко, что она бежит, а я сижу. - предлагаю я.
- Сиди! - командует Андрей. - Сама никуда не ходи здесь, запомни. 
Да. Здесь нужно точно знать, куда идёшь и зачем. Дисциплина в этих условиях спасает жизнь, а умение слышать и исполнять советы опытных - бесценно. 
Оборачиваюсь к нашей попутчице , она здешняя и ориентирует нас "по адресам" - кому, где и какая нужна помощь.
- Алёна, а как же дети местные, они же могут выбежать на заминированные поля. 
- Местные? Не выбегут. Они всё знают с рождения.
Притихаю. Как могут дети знать о страхе смерти? Что они должны пережить, чтобы осознать эту опасность?
Амброзия карябаяет стёкла, она здесь по крышу машины. Едем по узенькой дорожке , "кошачьими тропами", тихонечко,  к бабушке Вале. 
Подъезжаем и видим дом, у которого вместо крыши - целлофановая плёнка, брезент и ещё Бог знает что. В новых вставленных окнах свежие дырки от осколков. Во дворе чистота, розы, малина, яблони. 
- Собачка вся в осколках, в будку попало в прошлый раз. Вот эта яблоня нас часто защищает, бедная. А малину всю посекли, нелюди, - оборачивается в сторону украинских позиций бабушка и вытирает рот краем платка. Совсем, как моя бабушка. 
Заходим в дом. Огромные дыры в потолке. Дыры в стёклах. 
- Почему нам не дают стройматериалы, чтобы мы отремонтировались? - недоумевает бабушка. - Мы с 14 года каждый день под обстрелами!
Наша проводница и попутчица Алёна говорит мне тихо, чтобы баба Валя не слышала:
- Потому что толку ноль, "прилетает" каждый день. Наша улица чуть дальше - нам отремонтировать помогли, даже пьяницам. А здесь вряд ли будут что-то делать. Толку!
Оставив продукты и деньги, собираемся уходить. 
- Погодите, постойте , дайте я вам хоть яблочек дам. Суёт нам пакет с яблоками, видимо, с той самой яблони, что защищает их от осколков, потом пакет петрушки, потом тащит ящик моркови. 
Мы поспешно ретируемся, а я напоследок вдруг неожиданно для себя обнимаю бабу Валю и целую её в седую макушку. 
Садимся в машину и потихоньку выезжаем.
- Вот, Света, мы заехали туда, где волки ср..ть боятся - говорит Андрей , и мы едем по следующему адресу.
