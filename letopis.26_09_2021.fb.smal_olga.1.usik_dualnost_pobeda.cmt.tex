% vim: keymap=russian-jcukenwin
%%beginhead 
 
%%file 26_09_2021.fb.smal_olga.1.usik_dualnost_pobeda.cmt
%%parent 26_09_2021.fb.smal_olga.1.usik_dualnost_pobeda
 
%%url 
 
%%author_id 
%%date 
 
%%tags 
%%title 
 
%%endhead 
\subsubsection{Коментарі}

\begin{itemize} % {
\iusr{Дарья Ткаченко}
Это как во фразе «мы за вас, конечно, рады, но не от всего сердца» @igg{fbicon.thinking.face}  @igg{fbicon.face.grinning.sweat} 

\begin{itemize} % {
\iusr{Ольга Смаль}
\textbf{Дарья Ткаченко} тут муки скорее: вроде и погордиться хочется (радость-то внутри искренняя), а вроде-как с очень своим специфическими взглядами Александр, а вроде и уникальный спортсмен:))).

\iusr{Дарья Ткаченко}
\textbf{Olga Glushchenko Smal} , да, понимаю - я только искренне поаплодировала труду, который сделала его команда и он, но не более @igg{fbicon.face.relieved} хотя так или иначе пояса теперь будут в Украине, и победителей не судят, наверное

\iusr{Ольга Смаль}
\textbf{Дарья Ткаченко} а я не про себя писала пост, я про наблюдения в нашем сегменте фейсбука @igg{fbicon.smile} . Я бой пока не смотрела.

\iusr{Olena Kovtun}
\textbf{Ольга Смаль}, а что у него со взглядами?

\iusr{Ольга Смаль}
\textbf{Olena Kovtun} 

ой:). Ну путаница. Люблю Украину не могу, вот вам стих в подарок, тут же мы —
один народ с россией-славяне, тут же дружба с РПЦ и тд. И да, он вообще оч
религиозный товарищ. Ну его право на самом деле. Он считает себя украинцем, и
что раз поднимается благодаря ему украинский флаг, он действует в интересах
Украины. Делаем поправку на то, что он из Симферополя и даже сейчас вроде живет
в Симферополе: там у них своя история.


\iusr{Olena Kovtun}
\textbf{Ольга Смаль}, поняла, спасибо. Спортсмены - вообще отдельная категория товарищей. Но за выступление под нашим флагом можно это и простить

\iusr{Ольга Смаль}
\textbf{Olena Kovtun} 

именно. У них образ жизни: тренировка-сборы-соревнования-режим. Во-первых не
все обязаны думать одинаково (мы не в советском же союзе), во-вторых — не все
такие осознанные по всем вопросам. Другое дело, что их менеджмент, их
пиар-службы, руководители из комитетов должны следить, чтобы лучше спортсмены
иногда ничего не говорили с учетом войны с агрессором Россией, с количеством
жертв, аннексией, оккупацией и тд. Тысячи, миллионов людей могут быть задеты,
не понять всего этого и вот этой политической «слепоты».

\iusr{Анастасия Громовская}
\textbf{Ольга Смаль} 

живет он с семьей в Киеве) и несмотря на все нюансы, он очень цельный - есть
свои взгляды, мнение - и это тоже достойно уважения, никакой хейт его убеждений
не изменяет. Мне многое не близко, но он безумно крутой спортсмен и хороший
человек, и просто очень крутой папа троих детей @igg{fbicon.biceps.flexed} 


\iusr{Ольга Смаль}
\textbf{Анастасия Громовская} 

возможно хороший, да: я лично не знакома. Имеет право на свои взгляды —
бесспорно и я уважаю стержень в людях. Просто ситуация у нас непростая и
информационная война с Россией никуда не делась. Его высказывания (и не толко
его) могут использовать в плохих целях, поэтому позиция позицией, но бывает,
что-то может и навредить, поэтому лучше аккуратнее с высказываниями. Но за
прекрасный бой, пояса, и украинский флаг победный ему в любом случае спасибо
большое.

\iusr{Анастасия Громовская}
\textbf{Ольга Смаль} здесь согласна, да @igg{fbicon.hands.raising} 

\end{itemize} % }

% -------------------------------------
\ii{fbauth.sanin_mihail.kiev.ukraina.vokalist.kompania.narod}
% -------------------------------------

А если попробовать не гордиться не своим достижением (как наши северо-восточные
"великие" соседи), а просто порадоваться успеху талантливого спортсмена?
Зае...ла вот эта жажда гордости за то, к чему не имеешь никакого отношения!
Ведь это не мы часами потели на тренировках, чтобы выйти на ринг за чемпионские
пояса, а Усик. Мне просто понравился бокс...

\begin{itemize} % {
\iusr{Ольга Смаль}
\textbf{Михаил Санин} 

ну я вообще бой еще не смотрела, но я очень рада, что он выиграл. Все пишут,
что это было безоговорочно прекрасно. Спортсмены представляют страны и на
трусах у Усика — именно украинский флаг, а не что-то другое. Пояса едут в
Украину, поэтому логично, что украинские фанаты гордятся его победой:). Я лично
для себя давно поняла, что у спортсменов свои расклады в голове и они вообще
чуть оторваны от реальности в виду образа жизни: требовать от них какой-то
сверхсознательности не стоит. Они могут гениально делать свое дело, прославлять
свою страну, класть на это здоровье, ну и спасибо им за это. Если еще и со мной
во взглядах совпадают, прекрасно. Нет — так нет.

\iusr{Oleksii Parkhomenko}

Главное чтобы они занимались своим делом, то в чем они профи. И не шли в
политику и рассуждать на каналы. А то все походы спортсменов, актеров, певцов в
политику заканчиваются очень плохо для людей. Народный кумир превращается в
дурачка. Хотя конечно возможны исключения , но редко и только после годов
работы в новой области .

\iusr{Ольга Смаль}
\textbf{Oleksii Parkhomenko} 

в нашем случае в политику идут все подряд:))). Я не считаю, что спортсмен или
актер это диагноз и что они не могут быть хорошими политиками, если у них
хорошая команда:). В мировой истории есть и неплохие примеры. Но каждый случай
надо рассматривать индивидуально:).

\end{itemize} % }

% -------------------------------------
\ii{fbauth.gurjeva_larisa.kiev.ukraina}
% -------------------------------------

Пацаны много глупостей наговорили конечно, и он, и Вася. А еще больше
раздражали своими контактами с РПЦ. Но хейтить их за это глупо, если все-таки
вспомнить, сколько они сделали для Украины, и ОИ, и во время проф.карьеры. Для
меня вообще повышенная религиозность - некий маркер, неважно, какая парафия. Ну
и что? Да, было немножко неприятно, я его с Чисорой и не смотрела, как ранее,
затаив дыхание, решила поспать, позже посмотрев обзор. Нынешняя ситуация -
совсем другая. 

Во-первых, Джошуа, один из сильнейших бойцов в истории хеви вейта, во-вторых,
Усик, который все-таки искусственно поднялся в эту категорию и только потому,
что выиграл все в крузервейте, там не осталось соперников, в-третьих, перипетии
этого боя, сколько времени ему не давали ход, пытылись откупиться, использовали
пандемию, уж очень Джошуа хотелось выйти против Фьюри.  А тут обязательная
защита пояса. И Усик уперся всеми рогами, многомесячно настаивая на
обязательной защите пояса, это вызывает уважение и потому болеть против?! Как
мне вчера написали - пусть валит в свой Крым. 

Ну, во-первых, мог давно свалить, если бы хотел. Потому это несправедливо.
Судьба у него все-таки сложная, болел туберкулезом в детстве, и боец он
нереальный. Что никто не подсказывает, чтобы следил за языком и не все
вываливал в инсту, как и Вася - это плохо. 

Ну а наши - так еще распнут радостно и с удовольствием. Я болела вчера и
поддерживала бы в любом случае. Проиграть такому противнику совсем не зазорно.
Сильно волновалась, потому что понимала, бой примерно равный, как там судьи
сложат по раундам - хз, по очкам может быть всякое, потому в конце 12 прям
завопила, разбудив мужа, появился очень серьезный шанс. И это очень круто.  Но
реванш будет тяжелым, Джошуа в истории с Руисом показал, что работу над
ошибками делать умеет.

\begin{itemize} % {
\iusr{Ольга Смаль}
\textbf{Лариса Гурьева} 

я вообще ничего не понимаю в боксе как в спорте,только понимаю, что это очень
талантливый спортсмен, а я ВСЕГДА болею за наших спортсменов: они кладут
здоровье на алтарь спорта и благодаря им наши флаги подымают по всему миру.
Насколько я читала, он и сейчас живет в Крыму. Я бой еще не смотрела, но читала
десятки восторженных отзывов.  @igg{fbicon.smile}  ps. Да, было бы неплохо, если бы публичные
люди следили за тем, что говорят, а если сами не могут, чтобы их пресс-службы в
свою очередь следили за высказываниями, чтобы не создавать напряжения. По
поводу религиозности вот этой нарочитой это вообще, да, отдельная тема.


\iusr{Лариса Гурьева}
\textbf{Ольга Смаль} возможно мама, но он с семьей давно в Киеве, и построил дом где-то в Ворзеле. И жена, и дети с ним в Украине.

\iusr{Ольга Смаль}
Лариса Гурьева а, возможно, точно не буду утверждать: просто где-то попадалась инфо.

\iusr{Лариса Гурьева}
\textbf{Ольга Смаль} 

у него похоже, она не нарочитая как раз. Он год в детстве пролежал в больнице и
тогда впервые пришел в церковь в возрасте 8 лет, так как подслушал врачей, что
он не жилец. Священник очень удивился, когда такой маленький пацан начал его
расспрашивать. В общем там много было страданий, это его жизнь, его право так
относится к этой теме, но похоже он искренен в этом. И все бы ничего, но когда
пошла война патриархатов - он еще парой высказываний подлил масла в огонь. А
потом началась волна хейта, как по мне - обе стороны были не на высоте, что он,
что хейтеры.

\iusr{Ольга Смаль}
\textbf{Лариса Гурьева} 

Я не об искренности веры, а о том, как он демонстрирует свою веру. Можно быть
глубоко верующим, но никто об этом может не узнать, наверное. Он всегда делает
на этом акцент и это часть его перформанса. Безусловно это его право и его
выбор, но да, на фоне войны все эти высказывания разного характера были
неуместны, не вовремя, некорректны. Но в любом случае он остается выдающимся
спортсменом и большой фигурой в нашем спорте, что бы кто не говорил, и как бы
не относился к его запутанным взглядам. Было приятно увидеть видео кличко после
боя, где они втроем: он, Шевченко, Усик. Очень долго мы были страной «Ааааа,
Shevchenko””, “aaaa, Klitchko”, теперь будем «аааа, Usik” зарубежом. Эти люди
популяризировали нас как страну, за что им большое спасибо.

\end{itemize} % }

\iusr{Артем Бальсанко}

Ситуация аналогичная бою Льюиса против Рахмана. Джошуа, как свойственно всем
талантливым неграм, очень сильно недооценил ноги Усика и переоценил свой удар.
Такое было миллион раз в разных весовых категориях. Мозли, Майвезер...
Продолжать можно долго.

Реванша Усик не переживёт. Джошуа умный парень и в реванше сделает ставку на
вязкий грязный бокс, плюс будет больше бить в корпус и не будет пытаться
догнать Усика и идти на размен.

После половины боя с работой в корпус, Усик опустит руки и пропустит удар
Джошуа, а это все... Держать удар Усик не умеет...

Радоваться нечему. Скорость в руках Саша потерял, удара для супертяжа у него
нет, ноги только спасают, но это пока и это соперники могут снивелировать.

\begin{itemize} % {
\iusr{Ольга Смаль}
\textbf{Артем Бальсанко} я ничего не понимаю в боксе как в спорте, время покажет:). А пока радуюсь за него.
\end{itemize} % }

\end{itemize} % }
