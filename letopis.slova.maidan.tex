% vim: keymap=russian-jcukenwin
%%beginhead 
 
%%file slova.maidan
%%parent slova
 
%%url 
 
%%author 
%%author_id 
%%author_url 
 
%%tags 
%%title 
 
%%endhead 
\chapter{Майдан}
\label{sec:slova.maidan}

В це відео слід тикати носиками як щенят, що нашкодили, тицяють у їхні какахи,
всіх наших (і не наших по ходу також) \enquote{білих пальтечок}, всіх тих, кому
наш \emph{Майдан} видавався \enquote{надто брутальним}, кому не подобався запах
\enquote{коктейлів} та шин, хто засуджував стрілянину та закликав нас до
ненасилля та всякого гандизму. А особливо тих, хто до останнього намагався про
щось \enquote{домовитись} із Овочем, вів нескінченні \enquote{перемовини}, коли
у нас на вулицях вже йшла війна.  Якби ми тоді їх послухались - були б зараз на
місці Протасевича, без варіантів.  Бо ось так, як на сьогоднішньому відео, і
лише так, закінчується \enquote{гандізм}, якщо твої опоненти - не британські
джентльмени, а породження ГУЛАГу. Проти цих потвор - лише вогонь на ураження,
\textbf{Хто на місці Протасевича не був, будь ласка, стуліть писки},
Євген Дикий, gazeta.ua, 04.06.2021

Сохранившие разум в \emph{майданном} безумии. В августе 2017 года в телепередаче
«Время покажет» на «Первом канале» киевлянка Светлана Пикта, мама трех детей,
которая тогда была на шестом месяце беременности (в декабере она родила двойню
— сына и дочь), на личном примере рассказала, как живется на Украине людям,
разумом которых не овладел \emph{майданный морок}. Светлана Пикта высказывала свое
мнение в социальных сетях: «Мы не относимся к украинским гражданам, которые
писают от восторга от режима Порошенко. Наша позиция не совпадает с
\emph{промайдановской} позицией. Мы пацифисты, христиане. Мы – многодетная семья»,
\textbf{Мужественные киевляне стали нашими земляками}, rweek.ru, 20.07.2018


%%%cit
%%%cit_head
%%%cit_pic
%%%cit_text
Про український характер у порівнянні з московитами а ляхами мовлено у праці
Володимира Антоновича ТРИ НАЦІОНАЛЬНІ ТИПИ НАРОДНІ:
\url{http://www.litforum.com.ua/index.php?r=7&a=5631} Українські \emph{майдани}
1990, 2004, 2013 були вибухом саме такої української стихії.  Вороги не тямлять
ані української нації, ані українського характеру, ані української стихії. Саме
тому вони не мають стовідсоткового успіху в Україні.  Жодна окупація в Україні
не досягнула і не досягне успіху, бо се вербовий народ (як влучно зауважила у
романі ВІДЛУННЯ відома україеська письмачка Лариса Денисенко), він гнеться але
не ламаєсь, і вельми вельми живучий.  Коли буде \emph{четвертий майдан}?
Скоро.  Десь 2024 рок
%%%cit_comment
%%%cit_title
\citTitle{Роман Кухарук - УКРАЇНСЬКА СТИХІЯ  А ВОРОГИ - Блог}, 
, romankuxaruk.com.ua, 08.04.2021
%%%endcit

