% vim: keymap=russian-jcukenwin
%%beginhead 
 
%%file 08_11_2020.fb.dnr.kiselev_v_evgenii.1.marik_death
%%parent 08_11_2020
 
%%url https://www.facebook.com/permalink.php?story_fbid=830847947753363&id=100024844966275
%%author 
%%tags 
%%title 
 
%%endhead 

\subsection{Андрей Панькин (Марик) - прощание - Донецк}
\Purl{https://www.facebook.com/permalink.php?story_fbid=830847947753363&id=100024844966275}
\Pauthor{Якубова, Татьяна}
\Pauthor{Худжец, Давид}

\index[names.rus]{Панькин, Андрей!ДНР}

В Донецке простились с легендарным комбатом Андреем Панькиным (позывной Марик).

8 ноября десятки людей собрались на Лозовском кладбище в Куйбышевском районе
Донецка. Здесь сослуживцы, родственники и друзья попрощались с легендарным
командиром стрелкового батальона 11-го полка защитником Республики Андреем
Панькиным (позывной Марик).  Андрей Панькин — уроженец Мариуполя, на защиту он
Донбасса встал практически с самого начала вооружённого конфликта, в мае 2014
года. Свой путь начинал рядовым бойцом, рассказывают сослуживцы комбата.

\ifcmt
pic https://scontent-waw1-1.xx.fbcdn.net/v/t1.0-9/124348330_830847721086719_5929885157143033926_n.jpg?_nc_cat=104&ccb=2&_nc_sid=8bfeb9&_nc_ohc=JN6CcmzAqV8AX_M28ca&_nc_ht=scontent-waw1-1.xx&oh=b57d4a496548b1da13e2a2362cf2af2e&oe=5FD3261E
\fi

«С первых дней в батальоне «Восток», всегда был в самых опасных ситуациях,
никогда не боялся рисковать, на той же промзоне Марик очень ответственно себя
проявлял во всех ситуациях. Начинал от пехотинца, от рядового, дошёл до
командира стрелкового батальона. Очень ответственный был, жаль таких терять», —
сказал ИА «Новороссия» бывший сослуживец Андрея Панькина Владислав Шинкарь. Он
служил с Мариком в одном отделении.

Андрею Панькину было 33 года, и семь из них он защищал Республику. За это время
он был удостоен множества наград — получил орден «За воинскую доблесть» первой
степени, медали «За боевые заслуги», звание кавалера двух Георгиевских крестов.

«В 2017 году мы с ним познакомились. При каких обстоятельствах? Как у нас
водится, часто мы знакомимся, участвуя в каких-то боевых действиях. Он тогда
ещё командиром роты был. Очень грамотный был командир», — рассказывает
сослуживец Андрея Панькина под позывным Север.

О боевом пути Марика в качестве командира роты вспоминают — твёрдость характера
проявлялась им в каждом бою.

«Его на самые угрожаемые участки ставили, и он везде справлялся. Батальон
вывели, оставили его роту, когда Грек погиб, там его рота держала по факту. Они
не только удерживали, но и оказывали достойное сопротивление. За Пески ходили,
ему группу дали диверсионную, и через промзону проходили, отважный он. Характер
твёрдый».

На службе Марик действительно нашёл свое место в жизни, говорит супруга Андрея
Юлия Панькина.  «Ранения были, много чего… Да, он был очень смелым, по-другому
и не скажешь, мужественным. Он, ни секунды не раздумывая, шёл на службу,
несмотря на то, что у нас были маленькие дети на тот момент. Мы всегда его
поддерживали, и мы очень гордимся тем, что у меня такой муж, у них такой отец.
Сын, кстати, тоже подумывает о том, чтобы пойти по его стопам — хочет поступить
в суворовское училище. Он — живой пример чести, доблести и мужества, боевой дух
в нём жил всегда. Его очень уважали, он всегда для своих бойцов, сослуживцев,
неважно, какой чин, он всех поддерживал, он за своих ребят горой стоял. У него
детей по факту намного больше».

Вместе с Юлией, 10-летним сыном и 9-летней дочерью Андрея 8 ноября на Лозовском
кладбище Донецка с легендарным комбатом попрощались его сослуживцы и друзья.
Двумя днями ранее Марик приехал на позиции Народной милиции ДНР, по которым
вели обстрел украинские вооружённые формирования. С половины восьмого утра за
13 минут по ним было выпущено 22 выстрела из СПГ, семь «восемьдесят вторых»
мин, два выстрела из гранатомёта. Об этом сообщал основатель батальона «Восток»
Александр Ходаковский в своём телеграм-канале. И спустя несколько часов стало
известно: спасая своего раненого бойца, Андрей Панькин погиб сам.

«Понимаете, когда комбат вытаскивает бойца, это говорит о многом». Сослуживцы
Марика уверены: по-другому он бы не смог.

Репортаж: Татьяна Якубова \& Давид Худжец
