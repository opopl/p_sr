% vim: keymap=russian-jcukenwin
%%beginhead 
 
%%file slova.strategia
%%parent slova
 
%%url 
 
%%author 
%%author_id 
%%author_url 
 
%%tags 
%%title 
 
%%endhead 
\chapter{Стратегия}
\label{sec:slova.strategia}

%%%cit
%%%cit_head
%%%cit_pic
\ifcmt
  pic https://odnarodyna.org/sites/default/files/styles/adaptive/public/article/2021/%D0%A1%D1%83%D1%89%D0%BD%D0%BE%D1%81%D1%82%D0%B8%20%D0%BC%D0%BD%D0%BE%D0%B6%D0%B8%D1%82.jpg?itok=ddZnY3nV
	width 0.4
\fi
%%%cit_text
«Триумфом дезы», вероятно, можно было назвать прошедший ещё в мае
пропагандистский форум «Украина 30», посвящённый «\emph{Стратегии} национальной
безопасности», которая стала идеей фикс для Зеленского и его челяди. О нём не
очень охотно говорили украинские СМИ, и понятно почему: у них небольшой выбор –
либо встраиваться в информационную политику команды Зе, либо повторить судьбу
трёх закрытых по велению и хотению президента оппозиционных телеканалов (и,
добавлю, ещё множества изданий, особенно в интернет-пространстве, расправа с
которыми была не столь заметной).  \emph{Стратегия}, как и большинство
идефиксов Зеленского, зародилась в недрах Совета национальной безопасности и
обороны Украины (СНБО) и была утверждена соответствующим указом гаранта 14
сентября 2020 года
%%%cit_comment
%%%cit_title
\citTitle{Как борьба с «дезой» на Украине множит сущности}, Георгий Бородин, odnarodyna.org, 07.01.2021
%%%endcit

%%%cit
%%%cit_head
%%%cit_pic
%%%cit_text
\emph{Стратегия} нынешней мировой войны состоит в неперсонифицированной вражде. Врагом
объявлена глобальная эпидемия коронавируса, которая позволяет разделить
общества на враждебные стороны (масочников и антимасочников, ваксеров и
антиваксеров, за тотальный цифровой контроль и против тотального цифрового
контроля и т.д.).  Как только угроза эпидемии в ходе вакцинации стала
уменьшаться, на смену ей пришла другая глобальная угроза — климатические
изменения, которые со временем так же будут должны породить враждебные стороны
в каждом обществе (экологистов и антиэкологистов, углеродников и
антиуглеродников и т.д.).  Если климатической угрозы не хватит, есть угроза
вулканов, космической катастрофы и агрессивных пришельцев. Ну и всегда можно
еще что-то придумать: креативный класс на службе у буржуазии справится с этим
%%%cit_comment
%%%cit_title
\citTitle{Враг}, Сергей Дацюк, analytics.hvylya.net, 18.11.2021
%%%endcit

%%%cit
%%%cit_head
%%%cit_pic
%%%cit_text
Чого не вистачає Україні для справжнього прориву у своєму розвитку?  Якось
колишній канцлер ФРН Конрад Аденауер сказав: «Усі під одним небом живемо, але
горизонт у всіх різний».  Сам Аденауер був людиною з широким
\emph{стратегічним} «горизонтом», і в його команді були люди з широким
\emph{стратегічним} «горизонтом».  На чолі нашої держави вже 20 років немає
людей з широким \emph{стратегічним} «горизонтом». Всі мислять тактичними
категоріями, а не \emph{стратегічно} (системно на велику часову перспективу).
Не плутайте мрії і фантазії про майбутнє з реалістичним, раціональним, якісним
системним \emph{стратегічним плануванням}.  У владі немає людей зі
\emph{стратегічним баченням}, які живуть у стрічному просторі щоденно, роками,
десятиліттями.  \emph{Стратег} будує міст, щоб організувати сполучення з іншим
берегом. А тактик напише стратегію, яка складається з чималої кількості звичних
для нього маленьких місточків, кладок, бродів (замість одного надійного мосту)
%%%cit_comment
%%%cit_title
\citTitle{Чего не хватает Украине для настоящего прорыва в своем развитии? / Лента соцсетей / Страна}, 
Владимир Воля, strana.news, 04.01.2022
%%%cit_url
\href{https://strana.news/opinions/370202-cheho-ne-khvataet-ukraine-dlja-nastojashcheho-proryva-v-svoem-razvitii.html}{link}
%%%endcit
