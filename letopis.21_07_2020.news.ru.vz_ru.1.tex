% vim: keymap=russian-jcukenwin
%%beginhead 
 
%%file 21_07_2020.news.ru.vz_ru.1
%%parent 21_07_2020
 
%%endhead 
\subsection{Канадцам стыдно за пригретых на груди украинцев}
\url{https://vz.ru/world/2020/7/21/1050838.html?utm_campaign=vz&utm_medium=referral&utm_source=push&fbclid=IwAR2iYKuGiuIpHZ0R2ycAFPHJks4c2h-YKaVjWIdIFPYw6AbuAEipeGfrtHo}
  
\vspace{0.5cm}
{\small\LaTeX~section: \verb|21_07_2020.news.ru.vz_ru.1| project: \verb|letopis| rootid: \verb|p_saintrussia|}
\vspace{0.5cm}

В Канаде начинаются тектонические процессы. Общественное мнение, прежде
поддерживавшее ``национально-освободительную борьбу украинцев с Россией'', вдруг
обнаружило, что опекает нацистов --- как современных, так и дававших клятву
верности еще живому Гитлеру. Поводом для этого стала безобразная акция в духе
тех, что прокатились по Америке в рамках борьбы с расизмом.

Кто бы мог подумать, что «памятникопад» и другие формы борьбы с расизмом в
рамках так называемой новой этики смогут принести России и русскому миру
внешнеполитическую пользу. Но именно это сейчас и произошло.

Эта история началась в маленьком городе Оквилл (провинция Онтарио, Канада), где
расположено кладбище Святого Владимира --- того самого Владимира, который крестил
Русь и которого современные украинские историки считают «украинским
государственным деятелем». Это важное уточнение, поскольку кладбище считается
украинским, а одной из его достопримечательностей является памятник «Борцам за
свободу Украины»

Надпись на нем гласит: «Светлая память участникам боев под Бродами в 1944
году». Другими словами, монумент посвящен легионерам дивизии СС «Галичина», и
только им. Называть вещи своими именами помешала политкорректность.

21 июня текущего года памятник подвергся вандализму: кто-то написал на нем
белой краской «Нацистский военный памятник». Подчеркнув, что вандализм на
кладбищах --- явление отвратительное и оправданий не имеющее, заметим вот что:
если учесть разницу во времени между Онтарио и Москвой, можно предположить, что
надпись появилась в ночь на 22 июня, когда гитлеровская Германия напала на
СССР.

Полиция провозгласила акт вандализма «преступлением на почве ненависти», что в
законодательстве Канады серьезно утяжеляет состав. Из каких мотивов это было
сделано, сказать трудно, возможно, без всякой задней мысли: подобный довесок к
обвинению полагается применять, если «акт ненависти» направлен против
«идентифицируемой группы». Как мы убедились ранее, идентифицировать «борцов за
свободу Украины» в данном случае очень просто.

Прошел месяц. Вандалов не нашли. Но полицию Онтарио как будто подменили --- она
отзывает свои предыдущие заявления, рассыпается в извинениях и клянется, что не
поддерживает нацизм и не имела в виду «ничего такого». Более того, мэр Оквилла
Роб Бертон и шеф муниципальной полиции Стив Таннер призывают демонтировать
украинский монумент. «Лично я шокирован и удивлен, что такой памятник вообще
существует», --- заявил Таннер в Twitter.

Все дело в том, что за последний месяц журналисты и ученые преподали канадцам
несколько уроков истории, касающихся «Галичины», СС в целом, личной клятвы
легионеров Гитлеру и участия украинских националистов в Холокосте.

Мы об этом писали многократно и повторяться не будем. Лучше расскажем другую
историю --- историю украинцев Канады, без которой на вопрос Стива Таннера будет
сложно ответить.

Первые украинцы переселились в Канаду еще в конце XIX века, где осваивали
медвежьи углы нынешней провинции Манитоба. Если точнее, это были подданные
Австро-Венгерской империи, называвшие себя русскими людьми. В агитационных
брошюрах «Про вольные земли» и «Об эмиграции», написанных видным переселенцем,
бывшим профессором Львовского университета Иосифом Олеськовым, украинцы даже не
упоминаются.

Вторую волну украинской иммиграции дала Польша 1920-1930-х годов. Тогда же у
канадских украинцев появились политические движения и общественные объединения,
содействующие переезду соотечественников.

Третья волна пришла по итогам Второй мировой войны и содержала огромное
количество украинских коллаборационистов, бежавших в западную зону оккупации от
Красной армии. В Канаду их активно зазывала уже оформившаяся диаспора, подчас
готовая даже оплатить дорогу, а премьер-министр Макензи Кинг этому ничуть не
препятствовал. Украинцы были для него уже привычной частью канадского пейзажа,
а вот евреев он недолюбливал, стремясь ограничить им возможности для переезда.

Да, Канада воевала на стороне союзников. Но еще до того Макензи Кинг поддался
очарованию Гитлера, с которым встречался лично. Он видел в фюрере ни много ни
мало «возможного спасителя человечества».

Не хотелось бы, чтобы об этом деятеле думали только как о примитивном расисте –
политик он серьезный и заслуженный, его портрет на канадском «полтиннике»
возник не случайно. Однако участие украинских националистов в истреблении
евреев, судя по всему, не было в глазах Макензи Кинга серьезным компроматом.

В современной Канаде, переживающей уже четвертую --- «незалежную» --- волну,
проживает третья по численности украинская диаспора (вторая, разумеется, в
России) --- более 1,2 млн человек, или 4\% населения страны. Украинский язык
указывают в качестве родного только 0,5\% канадцев, но русский, к примеру, всего
0,3\%. В упомянутой выше провинции Онтарио, где расположены столица Оттава и
крупнейший город Торонто, русских тоже вдвое меньше, чем украинцев (а украинцев
порядка 350 тысяч).

В общем, нет повода удивляться ни памятнику «Галичине», ни тому, что
правительство Канады занимает жесткую проукраинскую позицию по всем вопросам –
даже более жесткую, чем США. Украинскими корнями могут похвастаться многие
бизнесмены и политики страны, включая Христю Фриланд --- главу МИД Канады в
предыдущем правительстве Джастина Трюдо.

Дед Фриланд был коллаборационистом и издателем погромной газеты. Мы знаем это
благодаря дяде Фриланд --- историку Джону-Полу Химке, специализирующемся на
украинском национализме. Это, впрочем, не помешало Фриланд заявить, что ее
происхождение оболгали российские журналисты и блогеры, а премьер Трюдо то ли
повелся на это из-за недостатка знаний по истории, то ли решил, что русским
априори верить нельзя, то ли включил циника, но выслал российских дипломатов «в
знак протеста против дискредитации министра иностранных дел».

Возможно, позднее в биографию Фриланд все-таки вчитались, раскрыв ее глупую
ложь, возможно, Трюдо до сих пор пребывает в неведении, но в его втором
правительстве украинка пост главы МИД не сохранила. Что, впрочем, не имеет
никакого отношения к истории с кладбищем в Онтарио --- над местной властью глава
МИД не властен. Столь резкая перемена настроений продиктована другой причиной:
солидарностью либералов с движением BLM, под атакой которого в равной степени
находятся полиция и «расистские памятники».

Нельзя сказать, что о монументах, поставленных в честь нацистов, в Канаде
раньше не слышали. На памятник в Оквилле посольство России обратило внимание
канадцев еще в 2017 году (а заодно и на бюст Романа Шухевича в городе
Эдмонтоне). Тогда это было заглушено украинским воем о «российской пропаганде»,
а теперь, видимо, придется признать, что «российская пропаганда» говорила
чистую правду, тогда как дезинформацию распространяла именно украинская
сторона.

Эта сторона привыкла к тому, что в Канаде ее буквально вылизывают, как
умученное Россией национальное меньшинство, но сейчас украинскую национальную
гордость внезапно начали хлестать по лицу ее кровавым исподним. Вдруг
выяснилось, что пролезть в эпоху «новой этики» вместе с ненавистью к русским
еще можно, а вместе с дивизией СС «Галичина» уже нельзя.

Значимость этого вопроса значительно больше, чем исторические обидки сельского
соборного люда с его простой и жирной пищей. Канадская диаспора украинцев
является консолидированной силой, настойчиво продвигающей в Оттаве свою
повестку --- от поставок оружия Киеву до введения новых санкций против Москвы.
Все это обложено этнопатриотизмом и национальными мифами о борьбе с Мордором,
прежде воспринимаемыми в либеральной Канаде некритично. 

Трудно сказать, упадет ли после событий на кладбище политическое влияние
Конгресса украинцев Канады (UCC), который уже много лет в меру сил пытается
вставлять палки в колеса России и поддерживать «патриотов» на Украине деньгами
и лоббистскими усилиями. Это хитрые люди, которые обязательно переобуются,
когда поймут, что целиком их национальный миф теперь уже не спасти и как
минимум галицийскими эсэсовцами придется пожертвовать.

Но у канадцев действительно стали открываться глаза на то, что они пригрели на
груди нацистов.

Как и в случае с монументами генералам-конфедератам в США, памятник «Борцам за
свободу Украины» внезапно стал восприниматься как куда более позорное явление,
чем направленный на него вандализм (произошедший, напомним, на кладбище).

Кстати говоря, «украинские патриоты» навлекли этот скандал на свою голову сами
– в полном соответствии с их исторической планидой неудачников. Не исключено,
что полиция до сих пор расследовала бы «преступление на почве ненависти» к
эсэсовцам, поскольку не стала публиковать фотографий памятника и надписи на
нем. Но к месту событий прибыли какие-то украинские видеоблогеры --- посмотреть и
поужасаться. Именно благодаря их кадрам в Канаде узнали, что акция вандалов
имела антинацистскую направленность --- как и сотни других подобных акций за
последний месяц.

Остается надеяться, что жители страны с кленовым листом на флаге продолжат
удовлетворять свое любопытство и прямо завтра, 22 июля, обратят внимание на
современную Украину. Наверняка там с военными почестями и с вопиющей для нищей
страны пышностью будут отмечать очередную годовщину разгрома той самой дивизии
СС «Галичина» под теми самыми Бродами. Тогда канадцы увидят воочию и расистов,
и неонацистов --- зачастую тех самых людей, которых еще вчера чествовали как
«героев Майдана», «демократических активистов» и «борцов за свободу Украины».

Чем больше обывателей во всем западном мире, искренне ненавидящих нацизм или
пытающихся встроиться в «новую этику», разглядят клыки под их мягкими
славянскими подбородочками, тем меньше русскоязычных людей в Донбассе умрет от
обстрелов и последствий блокады, организованных современными фанатами Бандеры,
Шухевича и «Галичины».
  
