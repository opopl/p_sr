% vim: keymap=russian-jcukenwin
%%beginhead 
 
%%file 03_01_2022.stz.news.ua.den.1.tragichni_momenty_suchasnosti
%%parent 03_01_2022
 
%%url https://day.kyiv.ua/uk/blog/suspilstvo/tragichni-momenty-nashoyi-suchasnosti
 
%%author_id losєv_іgor
%%date 
 
%%tags 
%%title Трагічні моменти нашої сучасності
 
%%endhead 
\subsection{Трагічні моменти нашої сучасності}
\label{sec:03_01_2022.stz.news.ua.den.1.tragichni_momenty_suchasnosti}

\Purl{https://day.kyiv.ua/uk/blog/suspilstvo/tragichni-momenty-nashoyi-suchasnosti}
\ifcmt
 author_begin
   author_id losєv_іgor
 author_end
\fi

Відчуття цієї трагічності, загроза відчувається скрізь. Резонують усі
українські телеканали. Напруженість на кордоні збігається зі спробами деяких
політиків із влади загострити внутрішню ситуацію та фактично спровокувати
громадянську війну, не ту, про яку розповідають казки у Москві, а справжню. Нас
тією чи іншою мірою намагаються захищати дружні іноземні держави, а от
вітчизняна влада демонструє олімпійський спокій. Судячи з усього, те, що
відбувається, її не турбує. Ворог №1 для Зеленського – Порошенко, а Путін – це
партнер із переговорів, з яким прагнуть зустрітися... Здається, Захід не бачить у
Києві серйозних та відповідальних людей, з якими можна мати справу.

На «Прямому» журналіст Сайчук процитував американську газету, де говорилося, що
«...президент України – це в минулому комічний актор, який оточив себе радниками
зі своєї колишньої комедійної трупи».

У програмі \enquote{Говорить великий Львів} виступив багаторазовий нардеп Тарас
Стецьків, який поділився своїми побоюваннями. США та Росія можуть домовитися за
наш рахунок, шантаж приносить Росії успіх. Першою жертвою таких домовленостей
стане Україна. Зрештою, не здають лише тих, хто вміє себе захищати. А в нас на
Банковій захищати Україну явно не поспішають, підписали нове безглузде
перемир'я з агресором на Донбасі. Воно протрималося 10-12 годин, а потім знову
почалися звичайні обстріли нашої території.

Голова Тернопільської обласної ради пан Головко переконаний: «Росія перейде в
наступ тоді, коли буде впевнена, що справжнього опору не буде». А ця
впевненість залежить від поведінки Банкової... На думку пана Головка: «На жаль,
Європу Москва налякала, а ми не дотримувалися принципу: «Хочеш миру – готуйся
до війни». Ми не повинні розраховувати на союзників, а лише на себе».

Маніакальна «миротворчість» Зеленського наближає велику війну до України.
Балаканина про мир і реальне досягнення миру - це різні речі.

У студії провели голосування: «Чи готові ви взяти до рук зброю у разі
широкомасштабного наступу Росії?»

80\% глядачів каналу відповіли \enquote{так}, 20\% – \enquote{ні}.

Політолог, який був присутній у студії, сказав: «Нас врятує не миротворча
демагогія, а сильна українська армія». Банально, але вірно. На його думку, не
варто розраховувати на розвал Росії, нічого не роблячи для цього. Слова
політолога нагадали одну дискусію Володимира Леніна із товаришами по партії. Ті
переконували лідера, що не варто докладати особливих зусиль до руйнування
Російської імперії. Вона, мовляв, прогнила і сама впаде. На це вождь відповів:
Імперія сама не впаде. Її треба впустити».

Дуже різко виступив колишній перший заступник генерального прокурора України
Микола Голомша. Він гнівно запитував: Де наші диверсанти на території ворога?

І пропонував програму дій: «У нас багато військових підприємств, вони мають
працювати у 3-4 зміни. Потрібно припинити торгівлю зброєю у світі та озброювати
свою армію. Ми маємо показати, що вміємо не лише захищатися, а й атакувати. Як
ми можемо припустити, щоб нам хтось диктував на нашій території?

Чудовий виступ. Але ось на екрані з'явився колишній генерал СБУ Вовк і почав
лякати патріотичних українців відповідальністю за «пропаганду війни». Так, у
Радянському Союзі був такий пропагандистський закон, за яким, здається, ніхто
не був притягнутий до відповідальності. СРСР усю свою історію лише такою
пропагандою і займався. Як і РФ нині... А пропаганда оборонної війни в країні,
яка стала жертвою іноземної збройної агресії, є справою святою і праведною.
Шкода, що генерал Вовк цього не розуміє.

Багато хто в студії був переконаний, що наступ РФ на Україну неминучий. У 2022
році виповнюється 100 років з моменту створення СРСР, руйнацію якого Путін
вважає «найбільшою катастрофою ХХ століття». Путін дуже любить історичні дати,
згадаємо 2014 рік - 100 років від дня початку Першої світової війни.

Банкова ощасливила нас указом про те, що від імені України можуть виступати
лише три особи: Президент, прем'єр-міністр та міністр закордонних справ. Ну, це
не нове, так прийнято у всьому цивілізованому світі. Дивно тільки, що після
такого указу з такими заявами виступає Андрій Єрмак, який готує нас до
імплементації формули Штайнмайєра в українське законодавство, до перетворення
України на конфедерацію з подальшим розчленуванням та знищенням. Хоча Єрмак
клянеться, що «нічого страшного» у формулі Штайнмаєра немає. Але автономія
ОРДЛО у складі України призведе саме до такого катастрофічного результату.

Хто такий Єрмак? Де контора, яку він очолює, що прописана в Конституції
України?

Багато помічників Президента не мають статусу співробітника офісу, одержують
гроші в якихось інших структурах.

Тобто юридично вони ніхто, але при цьому допущені до найважливіших державних
справ, і ні за що не відповідають.

У розмові з ведучим «Еспресо» російський політолог-емігрант Андрій
Піонтковський сказав, що Китай підштовхує Росію до конфлікту із Заходом, що
призведе до розпаду Росії та переходу її східних територій до Китаю.

Кримінальну справу проти Порошенка Піонтковський назвав «жахливою помилкою». На
його думку, тут Банкова явно діє на користь Кремля, можливо, має місце якась
домовленість. До речі, так вважає і колишній посол України в США Валерій Чалий:
«Умовою прямих переговорів Путіна із Зеленським є арешт в Україні Петра
Порошенка». Росія вже заявила, що не обговорюватиме Донбас на цих переговорах.
Раніше вона відмовлялася обговорювати лише Крим. Результат зусиль пана
Зеленського? То що тоді обговорюватимуть? Залишається лише одна тема для
обговорення – капітуляція Зеленського та України.

На каналі «Україна-24» окрім набридлих і діючих на нерви скарг доктора
Комаровського на те, як йому докучає українська мова (чому б йому не
перебратися в країну, де української мови взагалі немає, в Росію? Не
переберуться, ці панове хочуть, щоб їм побудували Росію тут, в Україні. Ці
прихильники мовно-культурної політики «російського світу» залишаються
стратегічним резервом Путіна в нашій країні), був ще й дивовижний виступ Дмитра
Гордона. Він закликав молодь їхати з неблагополучної України за кордон, туди,
де краще. Замість боротьби за найкращу Україну молодих українців кличуть тікати
зі своєї країни. Ось такий патріотизм... Відроджується стародавнє гасло
космополітів: «Ubi bene, ibi patria» – «Де добре, там і батьківщина». Головним
аргументом був наступний: «Життя одне». Так можна виправдати будь-який злочин:
«Один раз живемо, візьмемо від життя все». Саме цим принципом керувалися ті,
хто ці 30 років грабував, обкрадав і зраджував Україну, «один раз живемо, отже,
все можна...»
