% vim: keymap=russian-jcukenwin
%%beginhead 
 
%%file 25_04_2022.stz.pc.ua.dou.1.geroizm_pogljad_kapellana.6.zvit
%%parent 25_04_2022.stz.pc.ua.dou.1.geroizm_pogljad_kapellana
 
%%url 
 
%%author_id 
%%date 
 
%%tags 
%%title 
 
%%endhead 

\subsubsection{Звіт}

Після моєї останньої статті, ми зібрали майже 100 000 грн, що було великою
підтримкою для нас. Всього з початку війни з різних джерел було зібрано більш
ніж 430 000 грн.

Нам вдалося організувати та доставити 5 бронежилетів, 6 бронепластин, 10
тактичних американських аптечок, військову форму, розгрузки. Завезли десятки
тон гуманітарних вантажів з Польщі, бус для евакуації та капеланських потреб в
ХСП та багато чого іншого.

З початку другої фази війни, а саме з 27 лютого 2022 року, коли я став
капеланом, було витрачено:

\begin{itemize} % {
\item Паливо і авто 71 814
\item Кеш 28 039
\item Їжа 43 901
\item Медицина 20 191
\item Одяг та взуття 17 483
\item Зв'язок + інтернет 700
\item Ночівля у Львові та на кордоні (3 рази) 7 000
\item Ремонт авто після ДТП по дорозі на евакуацію в Ніжин 33 500 (сума буде більшою)
\item Ремонт іншого авто 9 000
\item Евакуація Чернігів 17 000
\item Евакуація Донбас 90 000
\item Допомога ЗСУ 26 150
\item Допомога ТрО 20 150
\item Волонтери 2 000
\item Допомога 10 000 грн герою \href{https://youtu.be/HPqmJA3MbU0}{цього відео від Hromadske}, який був в полоні у окупантів. Я знаю його особисто.
\end{itemize} % }

Всього 396 928 грн або майже 13 000 доларів. Може здатись, що це багато, але це
лише мала частка того, що ще треба зробити.

Якщо у вас є можливість допомогти нам в тому, що ми робимо, не зволікайте,
шерьте цю статтю поміж друзями та знайомими. У мене є карта монобанку, донатьте
туди. Для тих, хто не в Україні, у мене є PayPal та Binance гаманець. Якщо ви
працюєте без готівки, у мене є реквізити фонду. Напишіть мені у
\href{https://www.linkedin.com/in/vicchern}{LinkedIn} або Телеграм @VicChern.

Якщо у вас є контакти волонтерів, які закупають амуніцію та інше спорядження
для ЗСУ, давайте прямі контакти. Ми все доставимо, куди це потрібно.

Дякую, що дочитали до кінця. Дякую за ваші молитви, вашу підтримку та відкриті
до нас сердця.

Все буде Україна. Перемога за нами!
