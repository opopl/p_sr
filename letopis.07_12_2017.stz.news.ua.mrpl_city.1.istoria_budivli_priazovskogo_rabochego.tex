% vim: keymap=russian-jcukenwin
%%beginhead 
 
%%file 07_12_2017.stz.news.ua.mrpl_city.1.istoria_budivli_priazovskogo_rabochego
%%parent 07_12_2017
 
%%url https://mrpl.city/blogs/view/istoriya-budivli-priazovskogo-rabochego
 
%%author_id demidko_olga.mariupol,news.ua.mrpl_city
%%date 
 
%%tags 
%%title Історія будівлі "Приазовского рабочего"
 
%%endhead 
 
\subsection{Історія будівлі \enquote{Приазовского рабочего}}
\label{sec:07_12_2017.stz.news.ua.mrpl_city.1.istoria_budivli_priazovskogo_rabochego}
 
\Purl{https://mrpl.city/blogs/view/istoriya-budivli-priazovskogo-rabochego}
\ifcmt
 author_begin
   author_id demidko_olga.mariupol,news.ua.mrpl_city
 author_end
\fi

\ii{07_12_2017.stz.news.ua.mrpl_city.1.istoria_budivli_priazovskogo_rabochego.pic.1}

На центральній вулиці міста – проспекті Миру розташована всім знайома будівля
\enquote{Приазовского рабочего}. Однак не всім маріупольцям відомо, що цей будинок має
щасливу долю і багату історію. Адже це двоповерховий особняк, який пережив
разом зі своїми мешканцями, містом і країною революції та війни, зберігає у
своїх стінах до цього часу багато історій і унікальних таємниць. 

\ii{07_12_2017.stz.news.ua.mrpl_city.1.istoria_budivli_priazovskogo_rabochego.pic.2}

У 2017 році особняку виповнилося 115 років. Це легко встановити, досить
вдивитися в сталеву в'язь огорожі балкона, щоб розглянути цифри \enquote{1902}. Довгий
час не вдавалося з'ясувати, кому належав цей респектабельний особняк на
головній вулиці міста. Архів інвентаризаційного бюро згорів в роки Другої
світової війни, а разом з ним, здавалося, і всі відомості про будинок. Проте
завдяки дослідженню краєзнавця Сергія Давидовича Бурова і допомозі доцента
ПДТУ Сергія Сергійовича Данилова, чий батько поділився достовірними
відомостями про господаря будинку, вдалося з'ясувати, що особняк належав
Гавриїлу Ісидоровичу Гофу, впливовому маріупольському купцю. Сім'я Гофів була
однією з найбагатших родин Маріуполя. У місті вони були власниками 24
будинків. Гавриїл Гоф помер у 1911 р. бездітним, і його власність перейшла
племінниці Ганні Іллівні Гоф. На старовинному маріупольському цвинтарі можна
знайти сімейний склеп Гофів, де зберігся напис: \enquote{Гавриїл Ісидорович Гоф.
Народ. 6 серпня 1838 р. Помер 9 листопада 1911 р. Мир праху твоєму дорогий
дядько}.

Племінниця в особняку жила недовго, адже вдалося встановити, що в різний час у
будівлі працювали ювелірний магазин Брауде, тютюновий магазин, а в роки війни
нацисти тут влаштували гуртожиток для офіцерів.

Після війни в приміщенні розмістили шкільні класи, де у кілька змін навчалися
діти, які пережили два роки окупації. Поступово будівлі міських шкіл стали
відновлювати й переводити туди школярів, а їхнє місце зайняли учні
Маріупольського педагогічного училища. Для майбутніх педагогів знайшли інше
приміщення, а у звільнені кімнати вселилася редакція газети \enquote{Приазовский
рабочий}. Сталося це на початку квітня 1951 р. Перший поверх займав книжковий і
канцелярський магазин, здається, перший після окупації. Коли сусідній будинок №
23 був відновлений, в нього перевели книгарню, а його місце зайняв магазин
наочних посібників.

Наприкінці 60-х рр. ΧΧ ст. головним редактором \enquote{Приазовского рабочего} став
Іван Васильович Коробов – людина енергійна і діяльна. При ньому були зроблені
капітальний ремонт і перепланування внутрішніх приміщень особняка. З того часу
перший поверх перейшов у повне розпорядження редакції. Сьогодні редакція газети
успішно продовжує свою діяльність. Директором і головним редактором є
Токарський Микола Миколайович. Заступник директора, редактор відділу економіки,
міських новин і права – Сухоруков Віктор Костянтинович. Газета є однією з
найстаріших в Україні і має найбільший тираж в Донецькій області. Видається з
березня 1918 р.

\ii{07_12_2017.stz.news.ua.mrpl_city.1.istoria_budivli_priazovskogo_rabochego.pic.3}

У пресклубі редакції – саме за вікном з балконом – зберігся старий камін з
кахлями, покритими біло-блакитною глазур'ю, який колись прикрашав вітальню
Гавриїла Ісидоровича Гофа. Цікаво, що використання спеціальної кахлі мало не
лише декоративне, а й практичне значення. У пустотах за нею зберігалося тепле
повітря, і самі керамічні кахлі, які значно товстіші ніж ті, до яких ми звикли,
довго залишались теплими. Окрім того, їхню глазуровану поверхню значно легше
було очистити від сажі й пилу, ніж побілену піч.

\ii{07_12_2017.stz.news.ua.mrpl_city.1.istoria_budivli_priazovskogo_rabochego.pic.4}

Особняк побудований у стилі монументального класицизму. Багата ліпнина, що
прикрашає фасад, привертає до себе не меншу увагу, ніж кований балкон.
Унікальними для нашого міста є морський і фруктовий орнамент (відповідно, між
вікнами 1-го і 2-го поверхів та у верхній частині фасаду).

\ii{07_12_2017.stz.news.ua.mrpl_city.1.istoria_budivli_priazovskogo_rabochego.pic.5}

Також у будинку зберіглася старовинна драбина, по якій протягом 115 років
піднімаються і спукаються маріупольці різних епох, не замислюючись, що кожного
дня ми можемо скористатися унікальною можливістю і опинитися в старовинному
особняку міста.

Ми рідко замислюємося про те, яке значення мають архітектурні пам'ятники. Тим
часом, якщо запитати у перехожих, які асоціації у нього викликають слова
\enquote{Франція} і \enquote{Париж}, то більшість згадає про Ейфелеву вежу. У той же час
неможливо уявити собі Відень, Прагу або Рим без старовинних будівель. Багато
туристів долають тисячі кілометрів, щоб насолодитися цією неповторною красою,
відчути атмосферу минулих століть, дізнатися більше про минуле і краще
зрозуміти сьогодення. Зберегти архітектурну спадщину важливо не тільки як
данину історії, це важливо для кожного з нас. Маю велику надію, що і в
Маріуполі незабаром зміниться ставлення до архітектурної спадщини, яка формує
атмосферу міста і є його головним обличчям.
