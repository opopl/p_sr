% vim: keymap=russian-jcukenwin
%%beginhead 
 
%%file 10_01_2022.fb.zabuzhko_oksana.1.chudo_na_dnipri.cmt
%%parent 10_01_2022.fb.zabuzhko_oksana.1.chudo_na_dnipri
 
%%url 
 
%%author_id 
%%date 
 
%%tags 
%%title 
 
%%endhead 
\zzSecCmt

\begin{itemize} % {
\iusr{Volodymyr Viatrovych}

Нас було багато завдяки тому, що тривалий час Майдан тримався ненасильницьких
засобів боротьби. Саме завдяки цим ненависним деяким радикалам «ліхтарикам» і
«танцям». Такий Майдан приваблював не лише світ, створюючи позитивний образ
протесту, але й інших українців готових долучитися до нього. Радикалізація
протесту звужує його мобілізаційні можливості.

Але з іншого боку Майдан набравши розгону як ненасильницький спромігся дати
силову відповідь на насильство влади. І це сталося не надто рано (Казахстан) і
не надто пізно (Білорусь). Ми справді пройшли по лезу.

\begin{itemize} % {
\iusr{Mila Oliynyk}
\textbf{Volodymyr Viatrovych} білоруси теж були ненасильницькі з рожевими поні. І?

\iusr{Volodymyr Viatrovych}
\textbf{Mila Oliynyk} прочитайте комент до кінця. В ньому і про готовність відповісти на силові дії влади.

\iusr{Maria Zayats}
\textbf{Volodymyr Viatrovych} так, але і люди в своїй більшості вийшли якраз реагуючи на насильство над студентами зі сторони режимної влади януковича – хто зна чи вийшло б стільки людей, якби не трагічні події 30 листопада.

\iusr{Volodymyr Viatrovych}
\textbf{Maria Zayats} 

насильство 30.11 мало викликати негайну силову відповідь Майдану за планом
влади. І це мало стати кінцем Майдану. Але цього не сталося. І навіть спроби
«пришвидшити» перетікання протесту 1.12 теж не вдалися. Тому коли Майдан таки
відповів силою це не змінило позитивного ставлення до нього з боку світу та
українців

\iusr{Мирослав Логийко}
\textbf{Volodymyr Viatrovych}

 @igg{fbicon.thumb.up.yellow}  але все ж таки без парламенту з проукраїнськими та проєвропейськими
парламентарями, які не втекли, а стали представляти Майдан у Раді, нічого б не
вийшло.

\begin{itemize} % {
\iusr{Volodymyr Viatrovych}
\textbf{Miroslav Logiyko} згоден. І опозиція в парламенті вчасно і вдало зіграла свою роль

\iusr{Мирослав Логийко}
\textbf{Volodymyr Viatrovych} 

В 2014-му році у України вийшло не тільки завдяки майдану і добровольцям, а й
завдяки Парламенту України.

У Парламенті України були проукраїнські та проєвропейські політики та партії,
які втримали державу і заклали фундамент нової.

Саме Парламент України, після втечі януковича і його влади, оголосив початок
проведення антитерористичної операції.

Парламент України провів вибори і легітимізував владу.

І тепер саме парламент України є ціллю путіна-зеленського.

ОПЗЖ вже заявили про свої наміри кинути мандати, а це в свою чергу дозволяє
зеленському розпустити парламент.

\url{https://youtu.be/Kccmy-TqmgM}

Зеленський з його монобільшістю та проросійскою підтримкою це підхватить.

\url{https://m.facebook.com/story.php?story_fbid=4932810226762610&id=100001010458434}

Цього разу Путін не повторить своєї помилки 2014-го.

Главный вывод по событиям в Казахстане - что бы Западу было кому помогать, в
том числе оружием, нужна организованная оппозиция в стране и лидер у этой
оппозиции.

Запад не будет поставлять орудие и помощь разроздненым группам сопротивления.

Именно этого сейчас добивается путин/зеленський - убрать лидера оппозиции, как
объединяющую фигуру, и уничтожить парламентаризм как таковой, что бы не было
альтернатив.

Уже год про это пишу:

\url{https://m.facebook.com/story.php?story_fbid=4932810226762610&id=100001010458434}
\end{itemize} % }

\iusr{Ntina Ntoubrova}
\textbf{Volodymyr Viatrovych} 

Треба ще й врахувати, що це для українців другий майдан (а для білорусів -
перший), і що в нас на відміну від росіян та казахів - дуже компактна територія
без значних перешкод в пересуванні із наспунктів навіть прикордоння - до Києва.
Ну і наостанок - на нас перших цю спецоперацію випробували.

\begin{itemize} % {
\iusr{Marina Mischenko}
\textbf{Ntina Ntoubrova} 

Третій. Перший був \enquote{Революція на граніті}, голодування студентів. Після
першого Майдана був офіційно прийнятий Блакитно-жовтий Державний прапор

\end{itemize} % }

\iusr{Микола Миколайович Попелуха}
\textbf{Volodymyr Viatrovych} 

Головне все таки, що майданівці не грабували магазини чи перехожих! і навіть і
не думали про це.. По собі суджу.. Якось дивно було б уявити себе з кросівками
найк чи рибок бігучим стримголов по Хрещатику!))) Ясно, що в Алмаати
грабіжниками були так звані тітушки від спецслужб, які і організували і
спровокували мародерство... тітушки, думаю були, як від токаєва-назарбаєва, так
і від путлєра...

Правосєкі і ультрас, дійсно вчиняли насильство на Груші, але Вони не грабували
магазини і всіляко це пресікали!!! Сбербанк Росії зправа від барикади в бік
Європейської площі, що по Хрещатику і якраз на розі з Інститутською працював
вплоть до розстрілу на Інститутській!!!

\begin{itemize} % {
\iusr{Игорь Швец}
\textbf{Микола Миколайович Попелуха}, не ускладнюйте, той хто зовні організовував хороводи радикалів з надією, що вони почнуть мародерствувать не розрахував на те, що там українці, свого не віддадуть, чуже не візьмуть. Інакше б вони поставили на тих хто міг би зміг зайнятися мародерством. В Казахстані це врахували.

\iusr{Тетяна Костенко}
\textbf{Микола Миколайович Попелуха} Всі кафешки, бутіки, супермаркет працювали весь час в Глобусі.
\end{itemize} % }

\iusr{Taras Zarubko}
\textbf{Volodymyr Viatrovych} ніколи би не подумав, що Вятрович таку дурню напише!

\begin{itemize} % {
\iusr{Volodymyr Viatrovych}
\textbf{Taras Zarubko} дуже аргументоване заперечення  @igg{fbicon.smile} 

\iusr{Taras Zarubko}
\textbf{Volodymyr Viatrovych} Історику про історію...

10 грудня, дрімаємо коло бочок з дошками, які з одного кінця горять- тліють....

Аж тут пішла чутка про штурм! Хапаєм дошки, палаючі з одного кінця і біжимо
вверх по Інститутській! Виявилося, що беркутівський кийок не тягне проти
палаючої дошки і вони відступають з барикади....

То тільки був початок подій тієї ночі.....

Кажете мирний протест, ліхтарики - телефончики???

\iusr{Степан Лазар}
\textbf{Taras Zarubko} о експерд...

\iusr{Taras Zarubko}
\textbf{Степан Лазар} \enquote{експерт} пишеться от так!
Можеш звертатися! Багато що тобі розповім про Майдан ! На дивані, дивлячись
телевізор, ти багато що пропустив!
\end{itemize} % }

\iusr{Ольга Багалій}
Згодна. але навіть в цей час певна організація людей на Майдані була. І це врятувало від провокацій.

\iusr{Taras Zarubko}
\textbf{Olga Bagaliy} Я думаю, що не організація врятувала від провокацій і погромів, а внутрішня ідейність учасників!

\iusr{Xenia Gerke}
Про який силовий Майдан ви говорите? 100 майданівців убиті режимом Янука - скількох януків убив Майдан?

\begin{itemize} % {
\iusr{Taras Zarubko}
\textbf{Xenia Gerke} 18 \enquote{силовиків}.

\iusr{Xenia Gerke}
\textbf{Taras Zarubko} І вони не в Сотні. Бо виконували злочинні накази.

\iusr{Taras Zarubko}
\textbf{Xenia Gerke} Ви запитали, я відповів.

\iusr{Майя Копернак}
\textbf{Xenia Gerke} Майдан убив головного ворога - КПСС, нажаль, не всі це помітили...
\end{itemize} % }

\iusr{Борис Курдибаха}
\textbf{Volodymyr Viatrovych} 

ліберали на тракторі заважали швидко вирішити проблему неадекватів у владі,
тому жертви на Майдані і потім на фронті.

Воюють радикали.

\iusr{Roman Novatchinski}
\textbf{Volodymyr Viatrovych} 

а чим радикали не вгодили, може вже буде показувати які ми толєрасти і
завалювати ворога горою наших трупів, може краще нехай вже буде гора трупів, але
наших ворогів... питання до вас як публічної особи, хто дозволяв приземлитися
літакам з кацапії з їхнім \enquote{спєцназом}, чому не судять Кучму, де розслідування по
розстрілу на Майдані... може мій батько мав танцювати і світити ліхтарі перед
москалями замість того щоб піти в ліс в далеких 40х...

\begin{itemize} % {
\iusr{Taras Zarubko}
\textbf{Roman Novatchinski} Слава Україні!
Мій дід загинув в УПА!

\iusr{Roman Novatchinski}
\textbf{Taras Zarubko} вічна Слава і Пошана, мого зловили в 50 му через зраду, дали 25 років, відсидів 6ть.

\iusr{Taras Zarubko}
\textbf{Roman Novatchinski} 

В мене на Майдані побратим був Вася Фіцич з Н.Березова. Його батько брав участь
в знаменитому бою березівської сотні, коли погромили полк НКВС, більше 300
вбитих ворогів!

\end{itemize} % }

\iusr{Олександр Гринечко}
\textbf{Volodymyr Viatrovych} Пане Володимире майже у всьому я з вами згідний, тільки от думаю що якби підчас тих подій загинув хтось із представників влади а вийшло навпаки і за це ще ніхто не поніс покарання,тоді б х оч хтось боявся народу.

\iusr{Lesia Buzhuk}
\textbf{Volodymyr Viatrovych} 

пройшли, але адекватний свій захист був завжди. \enquote{Чудом} була ефективна
спонтанна взаємодія \enquote{радикальної} і \enquote{мирної} частин у найкритичніші моменти,
можливим стало саме завдяки достатній кількості мотивованих патріотів, які мали
досвід кількарічної діяльності - і гуманітарної, і прикладної військової та
давали приклад іншим. І готовності всіх учасників оперативно діяти у
правильному напрямку, власне завдяки отій патріотичній тяглості, що передається
через покоління... Результат невпинної і часто невидимої роботі тих, хто, як
зазначає Оксана Стефанівна, ніколи не демобілізувався від боротьби за Україну
та зумів передати головне нам прийдешнім. Щодо \enquote{тактичності} по часу -
цілковито Божа допомога, бо мирна бездіяльність дала перевагу ворогам

\iusr{Тарас Гірший}
\textbf{Lesia Buzhuk} сто відсотків

\iusr{Тетяна Носир}
\textbf{Volodymyr Viatrovych} 

одним словом, на Майдані було більше (набагато більше) розумних людей! Вони
своїм прикладом стримували дуже радикально налаштованих і тому Майдан відбувся.

\iusr{Anton Beletsky}
\textbf{Volodymyr Viatrovych} 

це насправді певний прояв рабської, совкової ментальності, людей бьють, а вони танці танцюють.

\enquote{мирний майдан} це був вимушений крок через те що люди не були готові
психологічно дати справжню справедливу реакцію.

В мініаполісі коли поліція вбила людину, то люди спалили відділок поліції і
постріляли поліцейських і всім чхати було на \enquote{ооой давайте без провокаці}

\iusr{Люба Пивоварник}
\textbf{Volodymyr Viatrovych} 

нажаль \enquote{благадаря} таким утопістам ми продовжуємо мешкати в
кацапсько-єврейській резервації з навязливою \enquote{незалежністью}....

\end{itemize} % }

\iusr{Олег Гуменюк}

Що за сварки в коментарях? Чому ми знову дуріємо?

Ближче до теми яку авторка подала в порівнянні і на перспективу і вже як
приклад для інших. Але ще нам багато треба працювати над собою у несприйнятті
популізму і толерантності між собою.

Дякую, пані Оксані!


\iusr{Юрий Поляков}

Варіант парадоксального розуміння того, як проходить навчання держав (пробачте,
але є і така думка).

 @igg{fbicon.flag.triangular}  Грузія – перший курс (революція троянд), другий курс (курс внутрішнього
путіна-іванішвілі)

 @igg{fbicon.flag.triangular}  Молдова – перший курс (погром парламенту), другий курс (курс внутрішнього
путіна-додона), третій курс (євро-орієнтація)

 @igg{fbicon.flag.triangular} Киргизія - перший курс (погроми парламенту), другий курс (курс внутрішнього
путіна-...)

 @igg{fbicon.flag.triangular} Казахстан - перший курс (погроми)

 @igg{fbicon.flag.triangular} Україна - перший курс (євро-орієнтація), другий курс (курс внутрішнього
путіна-януковича),... п'ятий курс (пошуки провідника до українського дива)

 @igg{fbicon.flag.triangular} Білорусь - зразу вищий курс (самрт-революція - не противитись злу насиллям),
кандидатська (дао-очікування природного само-замикання зла)

 @igg{fbicon.flag.triangular} Росія - зразу вищий курс (самрт-революція - не противитись злу насиллям),
кандидатська (дао-очікування природного само-замикання зла)

/На фото гуру Кличко хоче зупинити вогняну фазу революції..., щоб відбулось
само-замикання зла/

\ifcmt
  ig https://scontent-frt3-2.xx.fbcdn.net/v/t39.30808-6/271701037_4788105047934078_9187218432163742412_n.jpg?_nc_cat=103&ccb=1-5&_nc_sid=dbeb18&_nc_ohc=IEn6IuqZMQIAX_ZHcxr&_nc_ht=scontent-frt3-2.xx&oh=00_AT-QX_4oCvg5OavmHK-lf3qF_UaD6nOPSxPU3BORKr33Ug&oe=61E29A55
  @width 0.3
\fi

\end{itemize} % }
