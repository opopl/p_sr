% vim: keymap=russian-jcukenwin
%%beginhead 
 
%%file 25_11_2021.fb.kochetova_irina.lugansk.lnr.1.sajt_lnr_dnr_prestuplenia_ukr
%%parent 25_11_2021
 
%%url https://www.facebook.com/irina.kocetova/posts/10209275451919187
 
%%author_id kochetova_irina.lugansk.lnr
%%date 
 
%%tags dnr,donbass,lnr,prestuplenie.vojennoje,sajt,ukraina,vojna
%%title ДНР и ЛНР - создание сайта для сбора данных о военных преступлениях Украины
 
%%endhead 
 
\subsection{ДНР и ЛНР - создание сайта для сбора данных о военных преступлениях Украины}
\label{sec:25_11_2021.fb.kochetova_irina.lugansk.lnr.1.sajt_lnr_dnr_prestuplenia_ukr}
 
\Purl{https://www.facebook.com/irina.kocetova/posts/10209275451919187}
\ifcmt
 author_begin
   author_id kochetova_irina.lugansk.lnr
 author_end
\fi

ДНР и ЛНР объявили о создании сайта для сбора данных о военных преступлениях
Украины Донецк, 25 ноя – ДАН. Сайт по сбору данных о военных преступлениях
Украины и пропавших без вести в результате войны в Донбассе начал работу.
Ресурс был представлен сегодня на онлайн-конференции «Опаленная память
Донбасса: военные преступления украинской армии и новые данные о массовых
убийствах гражданского населения».

\ifcmt
  ig https://external-frt3-2.xx.fbcdn.net/safe_image.php?d=AQF01GXOXul-V_KP&w=500&h=261&url=https%3A%2F%2Fdan-news.info%2Fstorage%2Fc%2F2019%2F04%2F16%2F1623062572_074373_72.jpg&cfs=1&ext=jpg&_nc_oe=6f1ca&_nc_sid=06c271&ccb=3-5&_nc_hash=AQHngaR56zZeLIN4
  @width 0.4
  %@wrap \parpic[r]
  @wrap \InsertBoxR{0}
\fi

\href{https://dan-news.info/obschestvo/dnr-i-lnr-objavili-o-sozdanii-sajta-dlja-sbora-dannyh-o-voennyh-prestuplenijah/}{%
ДНР и ЛНР объявили о создании сайта для сбора данных о военных преступлениях Украины, dan-news.info, 25.11.2021%
}

«Сайт призван консолидировать наши усилия, дать людям достоверную информацию о
количестве погибших, раненых за время конфликта, показать ход нашей работы,
рассказать о том, что привело к образованию массовых захоронений. Ведь есть
большое количество пропавших без вести. На сайте будут отображены истории
людей, переживших украинскую агрессию, потерявших близких и родных с 2014
года», - рассказал член Общественной палаты ЛНР Сергей Белов. Сайт доступен по
адресу \url{donbasstragedy.info}. Сейчас в нем два раздела: фотогалерея и окно
обратной связи. Каждый житель региона может рассказать свою историю, поделиться
информацией о факте массового или спонтанного захоронения, прислать фото.
Ресурс только начал заполняться. По словам Белова, в дальнейшем его планируется
сделать на нескольких языках, включая французский и немецкий.Напомним, что этим
летом в ДНР и ЛНР активизировали поиск пропавших без вести. Для этого были
созданы специальные органы. В них включены представители руководства Республик,
сотрудники силовых структур, экспертные организации. В задачи входит
организация работ по выяснению судеб и местонахождения пропавших без вести,
сбор прижизненной информации о них, составление единой базы пропавших, умерших,
а также родственников пропавших, координация действий по эксгумации,
идентификации тел и передачи их родственникам. 

В ДНР эксперты провели первую эксгумацию 19 августа - на кладбище «Овсяное» в
Снежном. Всего же с 2014 года эксгумированы останки почти 150 неизвестных жертв
конфликта в Донбассе.
