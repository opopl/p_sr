% vim: keymap=russian-jcukenwin
%%beginhead 
 
%%file 27_05_2019.1.serial_chernobyl
%%parent 27_05_2019
 
%%url https://strana.today/opinions/203195-serial-chernobyl-polon-umelykh-namerennykh-iskazhenij.html
 
%%author_id tkachev_jurij
%%date 
 
%%tags avaria_na_chaes,chaes,chernobyl,film
%%title Сериал Чернобыль полон умелых, намеренных искажений
 
%%endhead 
 
\subsection{Сериал Чернобыль полон умелых, намеренных искажений}
\label{sec:27_05_2019.1.serial_chernobyl}
 
\Purl{https://strana.today/opinions/203195-serial-chernobyl-polon-umelykh-namerennykh-iskazhenij.html}
\ifcmt
 author_begin
   author_id tkachev_jurij
 author_end
\fi

Но мир, да и украинцы, будут судить об аварии именно по нему...

Посмотрел я первую серию HBO-шного \enquote{Чернобыля}. Ну, что я вам могу сказать:
американцы добились ещё одной блестящей пропагандистской победы. Первый
полномасштабный сериал о Чернобыле снят ими, и, конечно же, снят точно так, как
он должен быть снят с точки зрения интересов Соединённых Штатов Америки.

Сериал снят действительно мастерски. Колоссальное, кропотливейшее внимание к
деталям, даже к мелочам. Даже барабаны-сепараторы в правильный жёлтый цвет
покрасили. Быт. Одежда. Причёски. Архитектура и геометрия станции. Сама история
с плёнками академика Легасова и его самоубийство, с кадров которого начинается
сериал. Проделана действительно колоссальная работа по тому, чтобы уточнить
детали.

А затем - не менее колоссальная работа по тому, чтобы домешать в это нужные
дозы пропагандистского яда, исказить факты для создания должного
пропагандистского эффекта, выставить события в максимально чёрном свете, убрать
из этой истории всё великое и героическое и заменить его подлостью,
беспечностью и самодурством.

Просто примеры.

Академик Легасов, разумеется, не прятал никакие плёнки в мусорном баке,
опасаясь КГБ. Некоторые из них вообще прямо адресовались конкретным
специалистам, например, коллегам Дёмину и Сухоручкину, одна кассета - это
вообще интервью с журналистом Адамовским. Они были расшифрованы, их стенограмы
напечаталны, их может прочесть любой желающий. Да, там есть вопрос, что часть
записей была стёрта. Это - есть. Но никакой истории с мусорным ведром не было.
Но ведь как хорошо смотрится, а? Советский учёный кончает с собой, не в силах
вынести груз вины, но перед этим рассказывает ПРАВДУ и прячет её от совестких
властей!

Работники станции после аварии не бродили по территории, как обкуренные лемуры,
впадая в истерику и плача горючими слезами. Они проделали колоссальную,
героическую и крайне важную работу по недопущению распространения пожара в
машинном зале и нераспространению аварии на третий энергоблок. Именно они
погасили пожар в машзале (пожарные занимались крышей, и только крышей). Слили
горючее масло из генераторов. Вытеснили взрывоопасный водород азотом.

Эту работу, конечно, не покажут. Работу выживших, таких, как начсмены
турбинного цеха Давлетбаев, и погибших, таких как его подчинённые Новик,
Перчук, Бражник - её показывать не надо. Зато вдоволь смакуется напрасный труд
Акимова и Топтунова, действительно ошибочно подававших воду в реактор, будучи
уверенными, что он цел.

Акимов в сериале показан безвольным слабаком и трусом, полностью подавленным
деспотическим Дятловым. Это тоже ложь. Акимов был, безусловно, мужественным и
волевым человеком. Он совершил ошибку, но это была чисто научная ошибка. И он
дорого заплатил за её последствия. Посмертно унижать его - подлость.

Ещё детали. В фильме сильно обожжённого кипятком Шашёнка кто-то (Перевозченко?)
бросает просто посреди коридора, чтобы помочь Кудрявцеву и Проскурякову
пробраться в реакторный зал. Это, конечно, ложь. Во-первых, Перевозченко
Шашёнка так и не нашёл. Это сделали другие люди, и они, конечно, не бросали
раненого товарища посреди коридора: его доставили в санчасть, а потом увезли в
больницу.

Момент с пожарным, который поднял кусок графита, и через пару минут получил
радиационный ожог. Самое удивительное, что похожая история была. Но только там
не кусок графита был, а радиоактивная железяка, пробившая колесо, и пожарный,
действительно, минут 10 с ней возился, вытаскивая её. И апликацинный ожог,
разумеется, проявился через пару дней.

Ну и феерия - это заседание горкома Припяти в помещении бункера станции. Что
это за нахер? ТАкого не было и такого не могло быть даже теоретически. Что это
за демонический старец, отдающий начальнику станции (ШТА?) приказ изолировать
город (ШТА?).

В реальности ничего такого не было. Город никто не закрывал. Некоторые его
жители стали покидать его в первую же ночь. Некоторых из них впоследствии
арестовали, но не за то, что уехали, а за то, что уехали в то время, когда
должны были быть на станции - проще говоря, дезертировали.

Феерический диалог:

Член горкома: в городе воздух светится (ШТА?) Дятлов: эффект Вавилова-Черенкова
(да-а-а?!), встречается даже при небольшой радиации (что, нахер?!!).

Авторы, вероятно, считают зрителей безграмотными тупыми нелюбопытными идиотами.
Что же, судя по буре восторгов, они знают, с кем имеют дело.

А чудесная сцена с Ситниковым, где его под дулом автомата отправляют
рассматривать реактор? Реально под дулом автомата: его сопровождает какой-то
солдатик. Конечно же ничего такого не было (или иначе где этот солдатик в
списке погибших, как Ситников?). Ситников пошёл на крышу по приказу Фомина, но
пошёл сам и вполне добровольно - он и сам понимал, что лучшего способа оценить
состояние реактора нет.

Ещё сцена с Легасовым, которому ранним утром звонит Щербина и которого тот
пытается убедить начать эвакуацию города, но ему говорят, мол, твоё дело
свинячье, начальство умнее. Не было такого диалога и быть не могло. Послушайте
на эту тему хотя бы те же плёнки Легасова, который прямо говорит: решения по
эвакуации принимались совершенно вовремя и правильно, а он на тот момент ничего
советовать не мог банально по той причине, что не знал ни об уровнях радиации,
ни о состоянии реактора. И не звонил ему никакой Щербина, а об аварии он узнал
на заседании партактива Института Курчатова. Этот момент в плёнках Легасова
описан подробно, как и момент с эвакуацией. Авторы фильма явно были знакомы с
плёнками (сужу по другим деталям) но этот момент осознанно переврали. Ну
конечно, злые совки предпочли убить жителей, чтобы не нагнетать панику. Они же
всегда так делают.

К слову, я так понимаю, из Легасова попытаются сделать эдакого
героя-разоблачителя. Ну да, ну да. Только вот повесился он, кстати, потому, что
его собратья-учёные как раз затравили за то, что он слишком уж поддерживал
официальную версию, в которую многие не верили.

Ошибки, большие и малые, в фильме на каждом шагу. При этом, повторюсь, авторы
предельно точны в деталях и мелочах, когда это всё равно с точки зрения
идеологии. Искажения возникают лишь тогда, когда события надо драматизировать.
Это не небрежность, наоборот: умная, подлая и сознательная манипуляция.

Но мы можем написать об этом тысячу раз. Весь мир, и в том числе наши
собственные сограждане, будут судить о Чернобыле именно по этому сериалу.
Который, конечно, мы должны были снять первыми. И именно поэтому факт выхода
фильма - наше огромное пропагандистское поражение и внушительный успех наших
недоброжелателей.
