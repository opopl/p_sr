% vim: keymap=russian-jcukenwin
%%beginhead 
 
%%file people.dostojevskij_fedor_pisatel
%%parent people
 
%%url 
 
%%author_id 
%%date 
 
%%tags 
%%title 
 
%%endhead 

%\ii{11_11_2021.fb.teatr_lavreneva_chernomor_flot_rf.1.dostojevskij_200}
%\ii{11_11_2021.fb.baumejster_andrej.kiev.filosof.1.200_let_dostoevskij}

%%%cit
%%%cit_head
%%%cit_pic
%%%cit_text
Достоевский учился в Петербурге в Главном инженерном училище. Был полевым
инженером-подпоручиком, но с военной службы уволился достаточно скоро,
поскольку все его мысли были посвящены литературе.  Власти считали Федора
Достоевского преступником, который распространял преступные факты о религии. В
возрасте 28 лет он был приговорен к смертной казни путем повешения. Но все же
его помиловали и отправили на каторгу в Омск.  После освобождения служил в
батальоне в Семипалатинске.  В возрасте 36 лет обвенчался в Русской
православной церкви с Марией Исаевой. Их брак продлился семь лет, детей не
было.  Вторая жена Анна (на фото ниже) родила четверых детей: Софию (она
скончалась через несколько месяцев после рождения), Любовь, Федора и Алексея.
Их потомки сейчас живут в Санкт-Петербурге
%%%cit_comment
%%%cit_title
\citTitle{200 лет Достоевскому. Как великий писатель стал идолом Вуди Аллена и пугалом для Андруховича}, 
Екатерина Терехова; Анна Копытько, strana.news, 11.11.2021
%%%endcit
