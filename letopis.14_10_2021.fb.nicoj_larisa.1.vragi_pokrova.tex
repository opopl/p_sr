% vim: keymap=russian-jcukenwin
%%beginhead 
 
%%file 14_10_2021.fb.nicoj_larisa.1.vragi_pokrova
%%parent 14_10_2021
 
%%url https://www.facebook.com/nitsoi.larysa/posts/1008193959746976
 
%%author_id nicoj_larisa
%%date 
 
%%tags bandera_stepan,nicoi_larisa,prazdnik.pokrova,ukraina
%%title Доброго дня (насправді, для вас не доброго) усі вороги України
 
%%endhead 
 
\subsection{Доброго дня (насправді, для вас не доброго) усі вороги України}
\label{sec:14_10_2021.fb.nicoj_larisa.1.vragi_pokrova}
 
\Purl{https://www.facebook.com/nitsoi.larysa/posts/1008193959746976}
\ifcmt
 author_begin
   author_id nicoj_larisa
 author_end
\fi

Доброго дня (насправді, для вас не доброго) усі вороги України, в тому числі й
по той бік блокпостів і порєбриків. Сьогодні Ненька святкує День Покрови
(матінки-захисниці), День нашого славного лицарства-козацтва, День не скореної
вами УПА і День наших доблесних коханих любимих Захисників України, які
сьогодні в окопах чи вдома. 

У цей славний день шлю вам, запоребрикові поторочі, палкий «привіт» зі славного
українського міста Стрий.

Знаєте, що ото за будівля у мене за спиною? То Стрийська школа імені
Митрополита Андрея Шептицького, на стіні якої, прямо над нами, висить пам’ятна
дошка про їхнього учня Юрія Дяковського. 

\ifcmt
  ig https://scontent-lga3-1.xx.fbcdn.net/v/t1.6435-9/245386214_1008193583080347_1503371740387802888_n.jpg?_nc_cat=101&ccb=1-5&_nc_sid=730e14&_nc_ohc=nMjIm7_UEIIAX9eANOa&_nc_ht=scontent-lga3-1.xx&oh=28b524bab77389aa24bec18a45271939&oe=619368CE
  @width 0.4
  %@wrap \parpic[r]
  @wrap \InsertBoxR{0}
\fi

Пам’ятаєте, хто такий Юра Дяковський? Ви йому 2014-го в Слов’янську розрізали
живіт, напхали камінням і втопили. Він міг урятуватися, якби заговорив, як ви й
примушували, по-вашому, по-запорєбриковому, але він виявився незламним. Ви його
катували, але він відмовився. Ви вбили його. Він став героєм.

Знаєте, що ото за жінка стоїть біля мене на фото? То його вчителька, пані
Оксана, яка навчила Юру бути патріотом України, трагічно пережила його загибель
і теж шле вам "вітання" з дуже палкими побажаннями.

Знаєте, що ото за дітки навколо нас? То її учні, і цих учнів з року в рік стає
все більше й більше. Вчителька вчить їх теж бути патріотами України. Разом з
учителькою і батьками вони приходять до шкільної стіни з пам’ятною Юриною
дошкою і моляться за нього, за Україну і за всіх українських героїв. 

Ці діти живуть з Юриним ім’ям на вустах. Коли після зустрічі зі мною про
книжечки, про літературних героїв, учителька запропонувала зробити колективне
фото в класі, то самі ж діти вигукнули: «А давайте сфотографуємося біля Юри!»
Одягли курточки і побігли на їхнє звичне доглянуте місце біля дошки з героєм.
Вони там часто збираються. 

Які ж ви дурні, антиукраїнські поторочі, що вбили Юрія Дяковського і багатьох
синів і дочок України. Ви не врахували, що Юра жив з ім’ям Бандери в душі, а ці
діти тепер живуть у серці не лише з ім’ям Степана Бандери, а ще й з іменем Юри
Дяковського. Як гадаєте (якщо у вас є чим гадати), це відбирає у них сили чи
додає? Авжеж додає і підсилює. Так що, українська приказка «за одного битого
сімох небитих дають» модернізувалася завдяки вам, і оті семеро з приказки, які
приходять на місце одного битого – їх уже не семеро, бо їхня кількість кожного
дня зростає в геометричній прогресії не лише в Стрию, а по всій  Україні.

Так що бажаю вам, поторочі, у наше славне свято Дня захисника України (у свята
ж прийнято щось бажати), якщо ви ще живі, щоб у вас зник сон. Живіть і
прислухайтеся. Може вам смішно з моїх слів. То пам’ятайте, у нас є ще одна
приказка: «Сміється той, хто сміється останній».

Ну а всім українцям, побратимам і посестрам, захисникам і захисницям, малим і
великим побажаю у цей святковий день преміцного здоров’я (нам ще перемогу
здобувати), і як завжди, гукну: «Слава Україні!» І як казав колись Бандера:
«Гукне один, а відгукнуться тисячі».

\ii{14_10_2021.fb.nicoj_larisa.1.vragi_pokrova.cmt}
