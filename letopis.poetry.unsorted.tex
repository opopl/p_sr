% vim: keymap=russian-jcukenwin
%%beginhead 
 
%%file poetry.unsorted
%%parent poetry
 
%%url 
 
%%author 
%%author_id 
%%author_url 
 
%%tags 
%%title 
 
%%endhead 

%https://strana.ua/opinions/328704-ukrainskoe-obshchestvo-hluboko-bolno-i-nuzhdaetsja-vo-vrachevanii.html
Этот поезд в огне,
И нам не на что больше жать.
Этот поезд в огне,
И нам некуда больше бежать.

Эта земля была нашей,
Пока мы не увязли в борьбе,
Она умрет, если будет ничьей.
Пора вернуть эту землю себе.


Людей неинтересных в мире нет.
Их судьбы — как истории планет.
У каждой все особое, свое,
и нет планет, похожих на нее.
А если кто-то незаметно жил
и с этой незаметностью дружил,
он интересен был среди людей
самой неинтересностью своей.
У каждого — свой тайный личный мир.
Есть в мире этом самый лучший миг.
Есть в мире этом самый страшный час,
но это все неведомо для нас.
И если умирает человек,
с ним умирает первый его снег,
и первый поцелуй, и первый бой...
Все это забирает он с собой.
Да, остаются книги и мосты,
машины и художников холсты,
да, многому остаться суждено,
но что-то ведь уходит все равно!
Таков закон безжалостной игры.
Не люди умирают, а миры.
Людей мы помним, грешных и земных.
А что мы знали, в сущности, о них?
Что знаем мы про братьев, про друзей,
что знаем о единственной своей?
И про отца родного своего
мы, зная все, не знаем ничего.
Уходят люди... Их не возвратить.
Их тайные миры не возродить.
И каждый раз мне хочется опять
от этой невозвратности кричать.
