% vim: keymap=russian-jcukenwin
%%beginhead 
 
%%file 20_08_2018.stz.news.ua.mrpl_city.1.240_ricchja_vasyl_altuhov
%%parent 20_08_2018
 
%%url https://mrpl.city/blogs/view/vasil-altuhov
 
%%author_id demidko_olga.mariupol,news.ua.mrpl_city
%%date 
 
%%tags 
%%title До 240-річчя Маріуполя: Василь Алтухов
 
%%endhead 
 
\subsection{До 240-річчя Маріуполя: Василь Алтухов}
\label{sec:20_08_2018.stz.news.ua.mrpl_city.1.240_ricchja_vasyl_altuhov}
 
\Purl{https://mrpl.city/blogs/view/vasil-altuhov}
\ifcmt
 author_begin
   author_id demidko_olga.mariupol,news.ua.mrpl_city
 author_end
\fi

\ii{20_08_2018.stz.news.ua.mrpl_city.1.240_ricchja_vasyl_altuhov.pic.1}

Вже незабаром відбудеться святкування 240-річчя Маріуполя. Влада готує місто до
перевтілення, містяни чекають на зміни, а історики і краєзнавці намагаються
представити увазі маріупольців найбільш цікаві і яскраві сторінки з історії
рідного Маріуполя. А оскільки історію створювали і будуть створювати люди, мова
піде про людину неординарну, сильну і унікальну у всіх відношеннях. Газети
називали його господарем килима, бійцівською славою Жданова, першим чемпіоном,
нащадком Івана Піддубного, зразком сильної волі, працьовитості та
наполегливості. Впевнена, що біографія \emph{\textbf{Василя Івановича Алтухова}} – 2-кратного
чемпіону світу з класичної боротьби, 8-кратного чемпіону України – може і
вразити, і надихнути водночас.

Про Василя Івановича писали журналісти, краєзнавці, спортсмени і просто
небайдужі маріупольці. Йому присвятили свої статті музейний працівник Наталя
Тарасова, суддя республіканської категорії Е. Соколок, майстер спорту СРСР В.
Криштопа, журналісти В. Андрєєв, В. Андрушкевич, Я. Димов, Л. Довбня, А.
Зоненко, А. Лукашов, А. Потапов.

\textbf{Читайте також:} \emph{В Мариуполе прошел чемпионат по стрельбе из боевого оружия}%
\footnote{В Мариуполе прошел чемпионат по стрельбе из боевого оружия (ВИДЕО), Яна Іванова, mrpl.city, 19.08.2018, \url{https://mrpl.city/news/view/v-mariupole-proshel-chempionat-po-strelbe-iz-boevogo-oruzhiya-video}}

З Алтухова Василя Івановича в Маріуполі почалася класична боротьба. Його шлях у
великому спорті був пройдений надзвичайно стрімко і результативно. Дитинство
Васі пройшло в Макіївці. У шахтарському селищі хлопчик годинами проводив на
вулиці, \enquote{досліджував} навколишні терикони і далекі лісопосадки. І хто знає, як
би склалася доля цього шибайголови, якби не зустріч з \emph{\textbf{Августом Вєдєнєєвим}},
майстром спорту з класичної боротьби. Чимось сподобався йому впертий, жорсткий
та сильний хлопець. Запросив займатися спортом, а в розмові наголосив: \emph{\enquote{Будеш
ходити – людиною станеш, не будеш – пропадеш}}. Ці слова досі пам'ятні
спортсмену. А тоді – дуже вразили, потрапили на благодатний ґрунт. Василь
закінчив школу-інтернат з однією четвіркою (з історії СРСР), що було більше
пов'язано з ідеологічним підґрунтям, ніж зі знаннями. Сім'я у нашого героя була
великою (9 дітей). Мати до кожного сина і кожної доньки знаходила
індивідуальний підхід. Вона говорила лише українською, тому не дивно, що
хлопець любив з дитинства Україну, яка завжди асоціювалася у нього з обличчям
матері. Проте батько був партійним робітником, хоча наприкінці життя зізнався
синові, що вся ця партія нічого не значить.

\ii{20_08_2018.stz.news.ua.mrpl_city.1.240_ricchja_vasyl_altuhov.pic.2}

Тренер Василя, Август Валер'янович, не помилився в своєму вихованці, побачивши
задатки великого спортсмена. Уже через рік після початку тренувань Василь став
чемпіоном України, майстром спорту, а незабаром – і чемпіоном СРСР серед
юнаків. Мати чекала сина з кожного тренування, сподіваючись, що Василь
повернеться здоровим і неушкодженим. Однак Василь Іванович довів всій сім'ї, що
він знає свою справу і збирається лише перемагати.

\textbf{Читайте також:} \emph{В Мариуполе дети и взрослые сражались на яхтах за кубок Лепорского}
\footnote{В Мариуполе дети и взрослые сражались на яхтах за кубок Лепорского (ФОТО), Яна Іванова, mrpl.city, 19.08.2018, \url{https://mrpl.city/news/view/v-mariupole-deti-i-vzroslye-srazhalis-na-yahtah-za-kubok-leporskogo-foto}}

У 1962 році доля закинула його в Маріуполь (тоді Жданов). Трудова біографія
почалася в цеху водопостачання і заводу імені Ілліча. Вдень – робота, а
вечорами – тренування в СК \enquote{Азов} у \textbf{\em Ф. І. Бабенка}. Через два роки завод
направляє Василя на навчання в Жданівський металургійний інститут. І тут він не
розлучається зі спортом. Його тренує \textbf{\emph{Г. Д. Патріча}}, згодом заслужений тренер
України. Обидва тренери Василя - і А. Ведєнєв, і Г. Патріча відіграли виняткову
роль у становленні спортсмена. Перший навчив Василя чесності і дав ази
борцівської грамоти, сміливості, другий – усього, чого він досяг у спорті в
зрілі роки.

\ii{20_08_2018.stz.news.ua.mrpl_city.1.240_ricchja_vasyl_altuhov.pic.3}

Завдяки Василю для студентства настали веселі дні. Василь Іванович, борець
багатосторонній і азартний, не пропускав жодного Панаїру, завжди перемагав і
повертався в альма-матер з баранчиком, на радість друзям-товаришам. Студентські
роки виявилися плідними у всіх відносинах, а особливо в спорті. Юнак
відрізнявся неабиякою наполегливістю і працьовитістю. Не дарма Василя Івановича
за роботу в студентському будівельному загоні нагородили Почесною грамотою
Президії Верховної Ради Калмицької АРСР. Після гучної перемоги в Будапешті
нашого героя включили до складу збірної команди СРСР. І протягом 10 років він
залишався членом цієї команди. У матчевих зустрічах на рівні національних
збірних СРСР – Угорщина, СРСР – Японія, СРСР – НДР він знову і знову доводив,
що є одним з найсильніших борців країни. У газетах з'являлися його фото, про
нього багато писали.

У квітні 1967 року найбільший міжнародний турнір в Будапешті зібрав кращих
борців з 11 країн Європи. Радянський спортсмен, майстер спорту з Маріуполя В.
Алтухов здобув на турнірі 6 блискучих перемог і виграв золоту медаль.

\ii{20_08_2018.stz.news.ua.mrpl_city.1.240_ricchja_vasyl_altuhov.pic.4}

Сутички з його участю були видовищні та захоплюючі. А за його поєдинком з
олімпійським чемпіоном – югославом Симичем зал стежив з особливою увагою.
Молодий спортсмен, борець атакуючого стилю розвинув такий наступальний темп, що
іменитий противник попросив його \enquote{зменшити обороти}. Але у Василя була мета –
тільки перемога, і поєдинок закінчився перемогою маріупольця. Серед переможених
нашим земляком був і чемпіон світу – угорець Ласло Силаї. На цих змаганнях
Алтухов перевиконав всі спортивні нормативи і став першим в Маріуполі і у всій
Донецькій області майстром спорту СРСР міжнародного класу. У місто, що стало
йому рідним, він повернувся чемпіоном.

А через кілька місяців він знову завоював чемпіонський титул – тепер уже на IV
спартакіаді України і отримав путівку на спартакіаду народів СРСР. Ювілейна
спартакіада за своїми масштабами, за кількістю учасників перевершила навіть
Олімпійські ігри. Кращими спортсменами країни були представлені всі види
спорту, зокрема, і класична боротьба. Жеребкування звело В. Алтухова, що
виступав в середній вазі, із заслуженими майстрами спорту Радянського Союзу –
неодноразовим чемпіоном світу і СРСР Анатолієм Кіровим і неодноразовим
чемпіоном СРСР Валентином Олійником. Алтухов виграв за балами у Кірова, внічию
закінчив поєдинок з Олійником, переміг ще в двох сутичках і посів почесне
четверте місце. У своєму місті Василь Іванович був визнаний кращим спортсменом.

У 1970-му Василь Алтухов в черговий раз виявляється в центрі уваги любителів
спорту. Першими великими змаганнями тоді стала Всесоюзна універсіада. Беручи
участь в ній, маріупольський студент переміг усіх суперників, а чемпіона СРСР
Муханова він поклав на лопатки за 15 секунд, встановивши таким чином своєрідний
рекорд турніру. Золоту медаль він виборов задовго до закінчення універсіади і
став в СРСР першим чемпіоном країни в ювілейному році. У 1975 році Василь
Іванович став переможцем шостої Спартакіади УРСР. Перемоги супроводжували
Алтухова Василя протягом всього життя і всім здавалося, що це не викликало
особливих труднощів. Однак це - кропітка повсякденна праця, високі фізичні
навантаження, від яких тіло, як свинцем, наливалося втомою. Окрім тренувань на
килимі були кроси, плавання, веслування, заняття зі штангою. У 2002 році Василю
Івановичу вдалося отримати чергову перемогу у третьому чемпіонаті світу серед
ветеранів. Запам'яталися на все життя Василю Івановичу слова старого чабана,
який пам'ятав виступи Василя на Панаїрі 60-70-х років: \emph{\enquote{Ех, ще б разок
подивитися, як ти борешся, а там і помирати можна}}. Ці слова стали найвищою
оцінкою його спортивної майстерності.

\ii{20_08_2018.stz.news.ua.mrpl_city.1.240_ricchja_vasyl_altuhov.pic.5}

Залишивши великий спорт, він став тренувати сільських хлопчаків. Свою школу
створював власними руками. Здобув величезний авторитет у місцевих жителів. Про
це говорить звання Почесного громадянина Першотравневого району. Для нього це
була найбільша нагорода. Василь Іванович був скупий на похвали, часом дуже
вимогливий. \emph{\enquote{Більше зібраності! Більше завзяття!}} – наставляв він борців.
Загалом тренерський шлях для Василя Івановича був ближчим, ніж кар'єра
спортсмена.

\ii{20_08_2018.stz.news.ua.mrpl_city.1.240_ricchja_vasyl_altuhov.pic.6}

У 1986 році Василь Алтухов у віці 47 років зустрів кохання всього життя.
Дружина Лариса стала надійною опорою і головною помічницею свого чоловіка. Син
Олександр з дитинства брав участь у багатьох спортивних змаганнях, але вирішив
обрати власний шлях і стати юристом, в чому Василь Іванович його підтримав,
адже головне, щоб син залишався чесним і людяним. А професія – його власний
вибір.

\textbf{Читайте також:} \emph{Мариупольский стронгмен в День независимости сразится с сильнейшими европейцами}%
\footnote{Мариупольский стронгмен в День независимости сразится с сильнейшими европейцами (ФОТО), Ярослав Герасименко, mrpl.city, 18.08.2018, \url{https://mrpl.city/news/view/mariupolskij-strongmen-v-den-nezavisimosti-srazitsya-s-silnejshimi-evropejtsami-foto-1}}

\ii{20_08_2018.stz.news.ua.mrpl_city.1.240_ricchja_vasyl_altuhov.pic.7}

Сьогодні Василю Івановичу 75 років. Він виглядає дуже бадьоро і пишається тим,
що нікому ні разу не вдалося покласти його на лопатки. В його оселі стільки
нагород, дипломів, медалей, кубків, що можна створювати окрему виставкову залу.
У Маріуполі найбільш улюбленим місцем залишився Міський сад. Любить джаз.
Улюблені фільми: \enquote{Чудова сімка} і \enquote{Любов під в'язами}. У Василя Івановича є
чому повчитися і тим, хто не займається спортом. Його сила духу, постійний
оптимізм і наполегливість у всьому – вражають. Василь Іванович для багатьох вже
став легендарним спортсменом і, безперечно, за свою діяльність на користь
рідного краю, за примноження спортивної слави міста, заслуговує присвоєння
звання почесного громадянина Маріуполя.

Історія Маріуполя складається з багатьох фактів, подій, людських вчинків, і я
дуже радію, що така непересічна і унікальна людина як Алтухов Василь Іванович є
частиною історії Маріуполя. Зі свого боку Василь Іванович побажав всім
маріупольцям: \emph{\enquote{Знайти себе і любити себе, тільки тоді все буде}}.

\clearpage
