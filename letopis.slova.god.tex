% vim: keymap=russian-jcukenwin
%%beginhead 
 
%%file slova.god
%%parent slova
 
%%url 
 
%%author_id 
%%date 
 
%%tags 
%%title 
 
%%endhead 
\chapter{Год}

%%%cit
%%%cit_head
%%%cit_pic
\ifcmt
  pic https://avatars.mds.yandex.net/get-zen_doc/5265785/pub_60f29964cf9db26dda7a8641_60f569696169c246f95e2670/scale_1200
  @width 0.4
\fi
%%%cit_text
И таких карт можно еще много найти. То есть "московия" появляется на картах
именно во время и благодаря антироссийской литовско-польской пропаганде, что в
принципе довольно наглядно видно по тому, когда вообще впервые появились карты
с термином "московия". И любой, кто прежде всего человек увлекающийся историей,
а не нацист, это прекрасно знает.  В 1586 \emph{году} была отлита Царь Пушка с
недвусмысленной надписью на ней
%%%cit_comment
%%%cit_title
\citTitle{300-летию \enquote{московии} посвящается}, Илья Duke, zen.yandex.ru, 19.07.2021
%%%endcit
