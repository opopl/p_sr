% vim: keymap=russian-jcukenwin
%%beginhead 
 
%%file 12_01_2021.fb.fb_group.story_kiev_ua.1.kukla_cyganka_1941.cmt
%%parent 12_01_2021.fb.fb_group.story_kiev_ua.1.kukla_cyganka_1941
 
%%url 
 
%%author_id 
%%date 
 
%%tags 
%%title 
 
%%endhead 
\zzSecCmt

\begin{itemize} % {
\iusr{Арт Юрковская}

А еще если вспомнить еврейские погромы 1905 г. и петлюровские.

\begin{itemize} % {
\iusr{Мария Каменцова}
\textbf{Арт Юрковская} . К сожалению, о столь далеких временах у меня нет историй, бабушки и дедушки не сохранили... Эту я из мамы тоже с трудом вытянула!  @igg{fbicon.smile} 

\iusr{Elena Samoylenko}
\textbf{Art Yurkovska} зачем так далеко. 50-е годы 20 века. Борьба передового пролетариата с космополитизмом. Дело врачей.
\end{itemize} % }

\iusr{Надежда Кохан}
Это изумительная история!

\begin{itemize} % {
\iusr{Мария Каменцова}
\textbf{Надежда Кохан} . Благодарю вас. Я долго думала и сомневалась, стоит ли ее публиковать, но все же решилась...  @igg{fbicon.smile} 

\begin{itemize} % {
\iusr{Надежда Кохан}
\textbf{Мария Каменцова} Вы не против, если я размещу эту историю на израильском сайте? Вдруг найдётся семья Лии?

\iusr{Мария Каменцова}
\textbf{Надежда Кохан} . 

С огромной радостью! Мама отказалась, она была уже очень старая (мы поздние
дети), любила покой, который она, несомненно, заслужила, – не захотела
поднимать эту историю, я записала ее для себя, как и много других, я их еще
покажу тут... Кроме того, она просто не знала (или забыла?..) фамилию этих
людей. Но было бы интересно что-либо о них узнать! Все же связь тут,
получается, у нас с ними КРОВНАЯ...

\iusr{Ольга Гондза}
\textbf{Мария Каменцова} вот и у меня так же получилось. Но думаю - главное, что об этом знаем мы и наши дети.

\iusr{Мария Каменцова}
\textbf{Ольга Гондза} . Мы хранители этой информации, и наше дело – постараться ее передать, обнародовать и сохранить...  @igg{fbicon.smile} 

\iusr{Надежда Кохан}
\textbf{Мария Каменцова} я сегодня поделилась Вашей историей. Будем надеяться, что у нее будет продолжение @igg{fbicon.hands.pray} 

\iusr{Мария Каменцова}
\textbf{Надежда Кохан} . Я думаю, будет трудно найти какие-либо следы этой чуть ли не столетней истории. Но, как говорится, чудеса – это по части Бога!  @igg{fbicon.smile} 

\iusr{Надежда Кохан}
\textbf{Мария Каменцова} всегда есть надежда на Чудо!

\iusr{Мария Каменцова}
\textbf{Надежда Кохан} . 

А кроме того, я рассказала только об одной семье, которая нашла у нас тогда
приют... Их, рассказывала мама, было множество! Всех прятали, давали какую-то
одежду ( в Бабий яр евреев доставляли, в холодное осеннее время, полностью
голыми), делились куском хлеба или ложкой горохового супа (почему-то только
горох тогда был в относительном изобилии!), помогали искать родственников или
знакомых, которые соглашались приютить и спасти... Дети бегали по городу с
записками, добирались и в близлежащие поселки!  @igg{fbicon.smile} 

\end{itemize} % }

\iusr{Наталия Слободян}
\textbf{Надежда Кохан},и если будет продолжение, пожалуйста, опубликуйте в группе.

\iusr{Надежда Кохан}
\textbf{Наталия Слободян} это обязательно!!!

\iusr{Мария Каменцова}
\textbf{Надежда Кохан} . 

Тут меня в комментах спросили (еще и с гневом!), почему спасенная семья не
попыталась связаться с нами, оказать помощь со своей стороны... Думаю, ответ
будет и для вас интересен!  @igg{fbicon.smile}  " Тогда другое время было... Даже простая
переписка с родственниками или знакомыми из-за границы могла привести к
печальным последствиям для \enquote{советских граждан}, и я знаю такие случаи лично.
Письма могли не доставляться, какие-либо попытки связи – блокироваться,
позвонить из-за границы было попросту невозможно. А в 1991 году, после развала
Союза и снятия всех ограничений, мы переехали и сменили адрес. Найти нас было
уже трудно. Старшее поколение спасенных к тому времени могло уже уйти
естественным путем, следующее – не знать всего точно... Да и чем, собственно,
они могли нам тут помочь?! Мы вовсе в помощи не нуждались, и возможно, не
приняли бы её! Мама, как я написала в конце рассказа, отказалась искать этих
людей из-за давности событий, или по иной причине...  @igg{fbicon.smile}  Не стоит сердиться на
семью тети Лии, скорее всего, они не смогли связаться с нами по каким-то
серьезным причинам... Там, в Израиле или в Америке, нередки теракты, они могли
попросту погибнуть или пострадать! Не думаю, что причина в неблагодарности.
Могло, к сожалению, случиться всякое... А вам спасибо за ваше горячее
неравнодушие!" @igg{fbicon.heart.suit}

\iusr{Ірина Оснач}
Думаю, варто уточнити, в якій саме школі працювала тьотя Лія. В 70-й?

\end{itemize} % }

\iusr{Helen Terliga}
Какие сообразительные и смелые дети

\iusr{Мария Каменцова}
\textbf{Helen Terliga} . Мамина жертва была велика. Но спасенные жизни того стоили...

\iusr{Larisa Kirova}
Чудесная история... На таких людях держится мир

\begin{itemize} % {
\iusr{Мария Каменцова}
\textbf{Larisa Kirova} . Сколько будущих детей было спасено ценой ужасной гибели одной куклы...  @igg{fbicon.wink} 

\iusr{Larisa Kirova}
\textbf{Мария Каменцова} это правда
\end{itemize} % }

\iusr{Римма Майская}
Очень интересная история, действительно на таких людях мир держится!!

\iusr{Мария Каменцова}
\textbf{Римма Майская} . Сколько будущих детей было спасено ценой ужасной гибели одной куклы...  @igg{fbicon.wink} 

\iusr{Anna Kolesova}
Страшные времена были и спасибо вашей семье за спасение людей. Тот, кто спас одну жизнь, спасает весь мир.

\iusr{Мария Каменцова}
\textbf{Anna Kolesova} . Каждый человек – целая Вселенная.  @igg{fbicon.smile} 

\iusr{Маргарита Марченко}
Спасибо за рассказ!

\iusr{Анна Анна}

в процессе чтения, потеряла Ваш рассказ, лента прошла... Искала с целью
дочитать его. Спасибо, это очень интересная история!

\iusr{Мария Каменцова}
\textbf{Анна Анна} . 

Маленькие герои и праведники потом живут целую большую жизнь, не думая о том,
что они совершили подвиг и прошли по грани бытия и смерти...

\iusr{Irina Sokol}

Есть очень классный рассказ писательницы которая приехала в США в 1994 году и
пишет по-английски, ее зовут Лара Вапняр, рассказ называется «Евреи в моем
доме» или может «В моем доме-евреи», поищите может есть перевод на русский,
если нет, то я как-нибудь найду время, переведу, и поставлю в группе.

\begin{itemize} % {
\iusr{Мария Каменцова}
\textbf{Irina Sokol}. Было бы очень интересно прочитать! Наверное, тема совпадает с моей!  @igg{fbicon.smile} 

\iusr{Irina Sokol}
\textbf{Maria Kamentsova} 

ну рассказов про это время много, просто у неё очень классно написано. Я найду
время и переведу. Уже заказала книгу в библиотеке, к сожалению на интернете
даже английской версии нет чтоб куда-то в транслит вставить и облегчить задачу,
так что буду старым способом с листа переводить.


\iusr{Людмила Мозговая}
\textbf{Irina Sokol} заранее спасибо.

\iusr{Vera Faynberg}
\textbf{Irina Sokol} Спасибо! только что нашла на Google. I will try to download it.

\begin{itemize} % {
\iusr{Irina Sokol}
\textbf{Vera Faynberg} на английском или на русском? Если на русском ставьте в группу, если на английском, пришлите мне пожалуйста, я переведу для участников.

\iusr{Vera Faynberg}
\textbf{Irina Sokol} In English, но я тоже попробую поискать на русском. Если найду, пошлю Вам.

\iusr{Irina Sokol}
\textbf{Vera Faynberg} спасибо, если найдёте на русском ставьте в группу, а пока перейдите мне пожалуйста на английском, я заброшу в Google translate и отредактирую.

\iusr{Vera Faynberg}
\textbf{Irina Sokol} попробую это сделать сейчас.

\iusr{Vera Faynberg}
\textbf{Irina Sokol} I just sent it to you in messenger. Let me know please if you see it.

\end{itemize} % }

\end{itemize} % }

\iusr{Сергей Сереженко}
Очень впечатляет когда ходишь по этой улице сегодня

\begin{itemize} % {
\iusr{Мария Каменцова}
\textbf{Сергей Сереженко}. 

Осталось название, но не улица. Все старые, красивые и прочные дома уничтожены
вместе с вековыми садами. Кстати, до 80-х годов у нас на улице охотно снимали
историческое кино...  @igg{fbicon.wink} 

\iusr{Мария Каменцова}
\textbf{Сергей Сереженко}. А дом, о котором я рассказываю, выглядел вот так...

\ifcmt
  ig https://scontent-lhr8-1.xx.fbcdn.net/v/t1.6435-9/137578251_3820065778046171_536999766791301165_n.jpg?_nc_cat=108&ccb=1-5&_nc_sid=dbeb18&_nc_ohc=W7McA516wekAX-E5-RG&_nc_ht=scontent-lhr8-1.xx&oh=00_AT86ISkc7U9_AotluyBZp3OuAWRBQWS55oreCL74DU40UA&oe=62164E0B
  @width 0.3
\fi

\begin{itemize} % {
\iusr{Марина Шкотт}
\textbf{Мария Каменцова} Я помню Ваш дом. Мы жили на Герцена 15. Сейчас там Шевченковское РОВД.

\iusr{Мария Каменцова}
\textbf{Марина Шкотт} . Ой, мало места!!! Наши герценовские объявились, как здорово!!!  @igg{fbicon.heart.sparkling}  Я и сейчас дружу с Таней Овдий из 13-го номера!!!

\iusr{Оксана Тернавская}
\textbf{Марина Шкотт} а до Шевченковского РОВД там была казарма КВИРТУ. А вот наша улица немного позже, когда уже сносили...Наш я отметила красной стрелкой

\ifcmt
  ig https://scontent-lhr8-1.xx.fbcdn.net/v/t1.6435-9/138986095_3831358860219126_3930798636462104107_n.jpg?_nc_cat=111&ccb=1-5&_nc_sid=dbeb18&_nc_ohc=J5zRRWMwsSQAX8FEflu&_nc_ht=scontent-lhr8-1.xx&oh=00_AT-81wsVS9Gzjm14jxSe1UvE4wh9klhmTcYOUAbSW_Ivrw&oe=621504C7
  @width 0.4
\fi

\iusr{Марина Шкотт}
Добрый вечер.

Да, нас \enquote{сносило} училище КВИРТу. Построили казармы. А Вы жили на
противоположной стороне? С той стороны я помню Зеленчуков, и учителя
физкультуры ( школа 70)

\iusr{Оксана Тернавская}

Марина Шкотт наш дом - Герцена, 21. Это он на фото 1976 года. Эту фотографию я
сделала лично в 5-м классе))) Зеленчуков я тоже помню) А рядом с нашим стоял
дом 19, там жили семьи Михлевских и Вишневских, такие полные сестры. В 17-м
жила тётя Римма, она хорошо стригла и делала причёски, сейчас живет в
Австралии. А её племянница Таня с Машенькой (автором поста) в одном доме.

\ifcmt
  ig https://scontent-lhr8-1.xx.fbcdn.net/v/t1.6435-9/139479356_3832170233471322_845410591669392954_n.jpg?_nc_cat=107&ccb=1-5&_nc_sid=dbeb18&_nc_ohc=VCpR0SUiBs0AX_vQaJx&_nc_ht=scontent-lhr8-1.xx&oh=00_AT_POv1gdw9PtsYhIjGfhzrvMM41qezF-Lqla3Rv4HfjfA&oe=62141E71
  @width 0.3
\fi

\iusr{Оксана Тернавская}
\textbf{Людмила Калишевская} о моем

\iusr{Оксана Тернавская}
\textbf{Людмила Калишевская} был, уже нет... На его месте вот это теперь

\ifcmt
  ig https://scontent-lhr8-1.xx.fbcdn.net/v/t1.6435-9/139282159_3832187670136245_4878058137615433463_n.jpg?_nc_cat=106&ccb=1-5&_nc_sid=dbeb18&_nc_ohc=9Wj98WZMpuEAX8d8xsP&_nc_ht=scontent-lhr8-1.xx&oh=00_AT8puwqZeHNxB3g_6FKT5U9vQlvbYwzLRn0TEtrqZ-nw0Q&oe=62141015
  @width 0.3
\fi

\iusr{Марина Шкотт}
\textbf{Оксана Тернавская}

Вишневских тётю Раю и ее сестру помню. Тёти Раи мужа помню. Звали Августин.
С тётей Риммой дружила моя мама, а Таня Соколова сейчас живёт на Бережанской.
Мне кажется, что пост о доме 21 писала не ее Маша.

\iusr{Оксана Тернавская}
\textbf{Людмила Калишевская} 

наши тоже все ещё бы стояли, такие стены были толстенные и крепкие, потолки
высокие. Нас сносил ЦК КПУ, хотели себе там построить дома, но СССР
развалился...

\iusr{Оксана Тернавская}
\textbf{Марина Шкотт} 

писала моя Маша, Мария Каменцова, это моя сводная двоюродная сестра, её дед,
овдовев с маленькой дочкой, её мамой, на руках, перед войной женился на моей
бабушке - тоже на тот момент вдове, Мишиной Екатерине Павловне, а потом они уже
вместе купили этот дом. А познакомились на Лукьяновском кладбище - могилы их
жены и мужа соответственно находились рядом...


\iusr{Оксана Тернавская}
\textbf{Людмила Калишевская} 

мы от этого дома получили 5 квартир и 20 000 руб. компенсации. Я в 90-х
просчитывала и сравнивала, даже 10 соток земли учла, у меня получилось тоже
самое

\iusr{Оксана Тернавская}
А Маша моя, Каменцова, живет в одном доме с Таней Соколовой и её Машей))) Вот такая земля круглая)

\iusr{Оксана Тернавская}
\textbf{Марина Шкотт} 

Ваша мама, безусловно, знала и мою маму, так как мама тоже приятельствовала с
тетей Риммой и всегда меня водила к ней на модельную стрижку. Просто мы
приезжали в Киев к бабушке в 70-х г.г. только на лето, так как папа у меня был
военным и служил на Северном флоте, где я и училась в школе, только в 3-м и 5-м
классах - в Киеве в 70-й школе. А мамино детство и юность прошли на ул.
Герцена, всех своих ровесников она знала


\iusr{Марина Шкотт}
\textbf{Оксана Тернавская}

Спасибо Вам за ответ. Очень приятно вспоминать места, где прошло детство. А
вдвойне приятно, когда можно и повспоминать с соседями. А еще я не могу найти
Инну Павленко. Они жили в 11 номере. Их снесли немного раньше нас, и переехали
они на Березняки. Мы бывали у них в гостях, когда была жива ее мама, тётя Ира.
Потом как-то потерялись. Вы ничего о них не знаете?


\iusr{Мария Каменцова}
\textbf{Марина Шкотт} . Если помните Таню Овдий из 13-го номера, то она у меня в друзьях, легко найти – друзей у меня немного, и все НАСТОЯЩИЕ!  @igg{fbicon.smile} 

\iusr{Марина Шкотт}
\textbf{Мария Каменцова}
Ок, спасибо. Сейчас посмотрю @igg{fbicon.smile} 

\iusr{Alexey Paschenko}
\textbf{Марина Шкотт} Олег Зеленчук на этой фото

\ifcmt
  ig https://scontent-lhr8-1.xx.fbcdn.net/v/t1.6435-9/139908343_4203826626294139_1257286631440479942_n.jpg?_nc_cat=108&ccb=1-5&_nc_sid=dbeb18&_nc_ohc=HTPoNAsTiHkAX971PRR&_nc_ht=scontent-lhr8-1.xx&oh=00_AT-idy8I90jb54FvTs8nVzct5i1qKgU7Jrvt5C2F2OK0Eg&oe=621621A7
  @width 0.4
\fi

\iusr{Марина Шкотт}
\textbf{Alexey Paschenko}
Да, во втором ряду. Классная фотка.

\iusr{Мария Каменцова}
\textbf{Alexey Paschenko} . Хо! Зеленчук за моей подружкой ухаживал, я его хорошо знала! Здорово! Земля круглая, а мир тесен! :)))
\end{itemize} % }

\end{itemize} % }

\end{itemize} % }
