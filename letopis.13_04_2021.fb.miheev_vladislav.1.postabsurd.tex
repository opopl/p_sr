% vim: keymap=russian-jcukenwin
%%beginhead 
 
%%file 13_04_2021.fb.miheev_vladislav.1.postabsurd
%%parent 13_04_2021
 
%%url https://www.facebook.com/vladislav.mikheev.5/posts/3986504394749639
 
%%author 
%%author_id 
%%author_url 
 
%%tags 
%%title 
 
%%endhead 
\subsection{Принуждение к постабсурду}
\label{sec:13_04_2021.fb.miheev_vladislav.1.postabsurd}
\Purl{https://www.facebook.com/vladislav.mikheev.5/posts/3986504394749639}

Принуждение к абсурду - вот, что лежит в основе нарождающейся общественной
формации. Но это какой-то другой, не освоенный культурой прошлых веков абсурд.

И это принуждение намного изощренней предыдущих форм эксплуатации.

Грубая сила, экономический интерес, великая идея - все это меркнет перед
нынешней властью абсурда. 

Бьют не в биологию или психологию - бьют в экзистенциальный центр. В онтологию,
в антропологию, в Dasein.

Человек разумный должен совершить добровольное насилие над собственным разумом.
Принять без-умие, без-мыслие и без-смыслие как естественное состояние,
игнорируя не только логику и здравый смысл, но даже и интуицию.

Мы имеем дело с когнитивной инволюцией, которая притворяется  революционной.

У Павича есть замечательный постмодернисткий сборник - \enquote{Вывернутая перчатка}.
Отныне изнанка бытия отождествлена с бытием. 

Даже инстинкт самосохранения и продолжения рода вывернут наизнанку - они
обратились против самих себя. Любовь и голод больше не правят миром, как
когда-то казалось Шиллеру. Им правит абсурд.

Но это вовсе не тот абсурд Камю или Сартра, в котором Делез видел путь поиска
нового смысла в ХХ веке. 

Это уже некий постабсурд - умудрившийся вывернуть наизнанку даже самого себя.
Он не просто по ту сторону ницшеанского \enquote{добра и зла}. 

Он по ту сторону и пути, и поиска, и смысла -  по ту сторону человека. 

Но именно в этом  измерении \enquote{изнаночного бытия}, заявленного как \enquote{новая норма},
человеку отныне и предлагается обитать.
