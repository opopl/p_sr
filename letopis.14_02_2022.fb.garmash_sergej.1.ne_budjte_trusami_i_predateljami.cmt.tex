% vim: keymap=russian-jcukenwin
%%beginhead 
 
%%file 14_02_2022.fb.garmash_sergej.1.ne_budjte_trusami_i_predateljami.cmt
%%parent 14_02_2022.fb.garmash_sergej.1.ne_budjte_trusami_i_predateljami
 
%%url 
 
%%author_id 
%%date 
 
%%tags 
%%title 
 
%%endhead 
\zzSecCmt

\begin{itemize} % {
\iusr{Сергей Гармаш}

Переклад українською:

Останнім часом ми всі живемо з відчуттям людини, якій сказали, що вона хвора на
рак і має померти найближчими днями. При цьому лікарі, які повідомили про це,
для переконливості залишили ліки, обіцяли перемогти онкозахворювання в
майбутньому, і пішли спостерігати за агонією по телевізору. \enquote{Приреченому}
делікатно дали зрозуміти, що сподіватися він може або на приховані резерви
свого організму, або на Бога. \enquote{Медицина тут безсила!}, - сказав лікар у Білому,
- \enquote{але не втрачайте надії}. І звалив, демонструючи безнадійність вашого
становища.

І ось, ви стоїте біля вікна зі своєю нежиттю, намагаєтеся зрозуміти, звідки
такий діагноз, що взагалі відбувається, і куди змилися ті, хто давав клятву
Гіппократа, або, як там його, – Будапештського меморандуму…

І, в принципі, ви нормально себе почуваєте, але коли всі навколо кажуть, що із
дня на день ти помреш, - поступово починаєш перейматися цією істерією. І тоді
постає питання: що робити? А чіткої відповіді вам ніхто не дає. Тільки
медсестри, прагнучи зберегти ситуацію у лікарні керованої, переконують, що \enquote{все
буде добре}. Але пояснити втечу лікарів, їхній фатальний діагноз і впевненість
у своєму \enquote{добре} вони теж не можуть.

І ось тоді приходить розуміння висловлювання, що очікування смерті, може бути
гірше за саму смерть... А найнеприємніше, що ти почуваєшся зрадженим тими, кого
вважав якщо не друзями, то добрими лікарями...

Те, що відбувається зараз з диппредставництвами наших західних та інших
\enquote{партнерів} – це якесь марення. Абсолютно алогічне. Воно показує не те,
що імпотентність, а просто відсутність світової системи безпеки. Весь світ знає
не лише намір Путіна напасти, а навіть дати цього нападу! І що? Де ООН,
створена спеціально для запобігання війнам; де ОБСЄ – організація з БЕЗПЕКИ в
Європі?...  Якщо всі знають, що буде війна, то чому замість запобіжних дій
просто біжать із потенційного поля бою?

Якщо ви дійсно так вірите в реальність загрози, то чому чекаєте, поки вона
відбудеться, а не запобігаєте її своїм санкціям, наприклад? Скільки українців
мають віддати свої життя за те, щоб ви вважали це достатнім для початку
реальних дій проти війни, яку ви, до речі, вважаєте реальнішою, ніж ми? Жодної
логіки! Це чи інформаційна війна, в якій нас призначили жертвами, чи деградація
всіх західних демократичних інституцій!

Евакуюючи посольства з Києва \enquote{цивілізований Захід}, цей \enquote{захисник} цінностей
\enquote{свободи та демократії}, начебто розчищає Путіну плацдарм для наступу, усуває
перешкоди, стимулює його до нападу! Замість, навпаки, посилити свою
дипломатичну присутність в Україні і цим убезпечити її, Захід збирає валізи,
бажаючи нам доброго дня під російськими бомбами...

Хоча ви ж чудово знаєте, що регулярні російські війська, наступом яких ви нас
лякаєте, – це не банди моджахедів, і у разі їхніх дій проти ваших громадян,
Російська держава вступає в конфлікт з вами, а Путін не може собі дозволити
воювати з усім Заходом!

Тому, якщо у Західних столицях справді хочуть миру в Європі, то необхідно не
евакуюватися з Києва, а навпаки, скликати у столиці України представницький та
численний Конгрес за мир та проводити його доки не настане деескалація! Він має
стати запобіжником від путінської агресії!

Я закликаю громадян іноземних держав з українським корінням або тим, кому
просто небайдужа доля України та світу в Європі, прилітати до Києва і проводити
тут якнайбільше заходів, що показують солідарність із нашою країною. Діаспоро,
ти потрібна Україні!

Я закликаю західні уряди повернути своїх дипломатів до Києва і навіть збільшити
їхню кількість. Не будьте трусами та зрадниками!

Ми ж всі чудово розуміємо, що можлива війна Росії з Україною буде не за
територію, і не за населення чи природні ресурси! Це буде війна за цінності, за
західні демократичні, ліберальні цінності, які ненавидить Кремль і які уособлює
Україна. Тому сьогодні, або всі, хто поділяє ці цінності, повинні бути разом!
Або зніміть білі халати і перестаньте брехати, що свобода і демократія для вас
щось означають!

\begin{itemize} % {
\iusr{Вячеслав Редько}
\textbf{Сергей Гармаш}, в данном случае важно содержание

\iusr{Олена Вільна - Сова}
\textbf{Вячеслав Редько}, але Мова викладення змісту, ВАЖЛИВА на разі, як ніколи!

\iusr{Olena Oleksiivna}
Дивно, що друзі(?)не хвилюються, що можуть втратити Україну(((

\iusr{Вячеслав Редько}
\textbf{Олена Вільна - Сова}, мова завжди важлива.

\iusr{Татьяна Кулицкая}
\textbf{Sergiy Garmash} добре було б на англійську перекласти!
\end{itemize} % }

\iusr{Andrey Polgorodnik}

Прекрасный текст до момента \enquote{Я вас призываю}... Вы прекрасно отдаете себе отчет
в том, что это крик в пропасть.

\begin{itemize} % {
\iusr{Сергей Гармаш}
\textbf{Andrey Polgorodnik} 

это не просто \enquote{крик}, это конкретные предложения, план действий. И если его
подхватят западные СМИ, то он может стать реальным.

\iusr{Олександр Приходько}
\textbf{Sergiy Garmash} може це і є такий собі протокол лікування, який допоможе отримати імунітет до наступних хвороб)

\iusr{Tana Istama}
\textbf{Олександр Приходько} і я так само подумала, ну, як ще можна зупинити маніяка?

\iusr{Олександр Приходько}
\textbf{Tana Istama} справа не тільки в маньяці, а і в змозі налаштувати організм до боротьби з ним
\end{itemize} % }

\iusr{Валентина Дар}

Вот это посыл! РЕСПЕКТ!! Пора прекращать двойные стандарты в дипломатии и
политике, пора называть вещи своими именам и действовать на упреждение, а то
больше бла... бла... бла...

\iusr{Tatyana Malyarenko}

Это информационная война, в которой Украиной пожертвовали. Нужно научиться
отстаивать свои интересы и перед Западом тоже, а не - дякую, дякую, боже,
сохрани королеву. Она не украинская королева

\iusr{Max Chokan}

запад не может быть более проукраинским чем мы сами. если наш посол в
великобритании позволяет себе легко отказываться от наших ценностей в качестве
компромисса, а президент стесняется четко назвать агрессора агрессором и
укоряет партнеров, вместо того, чтобы солидаризироваться с ними, то чего ждать
от самих европейцев? что они будут вместо нас бороться за наши ценности?

кстати, на фоне циркулирующих слухов, что президент вывез свою семью, неплохо
бы ему эти слухи опровергнуть, появиться с семьей где-нибудь в публичном месте.
ну чтобы у запада был повод не эвакуировать свои посольства.

\begin{itemize} % {
\iusr{Екатерина Золотарёва}
\textbf{Max Chokan} господи, какая чушь и ересь))

\iusr{Екатерина Золотарёва}
\textbf{Max Chokan} 

и семья его в Киеве, о чем он и сказал. А тебе надо прекратить питаться
исключительно порохоскотными фейками, вы там уже ничем от русских по уровню
развития не отличаетесь.

\end{itemize} % }

\iusr{Віталій Імамов}

Захід наразі хоче досягти деескалації за рахунок Києва (поступки по Мінську)

П'ятничні зливи від анонімів на Заході були направлені суто на підігруванні
паніки в нас.

Треба просто не піддатися цій накачці кілька днів.

РФ, на прикладі нового проекту про визнання \enquote{республік} + заява Лукашенко що
війська залишаться після навчань - показують, що росіяни вирішили почекати, щоб
збити інформаційну хвилю американців та спробувати виставити їх неадекватами.

Стоїмо. Спокійно

\iusr{Віталій Імамов}

Плюс полюбовна розмова міністрів оборони України та Білорусі - це теж про
ситуативне відкладання ескалації. Взагалі вкрай прагматичний хід України

\iusr{Анна Малиночка}

Во-первых, мы не олицетворяем западные либеральные ценности. С нашей-то
нетерпимостью, архаичностью и коррупцией. Во-вторых, Запад не просил нас ничего
олицетворять — мы сами вызвались. Ну может поматросил чуток, но и жениться не
обещал. А в третьих, похоже Запад и сам уже давно ничего не олицетворяет.

В общем нам надо срочно перестать думать шо мы чего-то олицетворяем, и искать
способы, как себя сохранить. А желательно преумножить.

\iusr{Mariia Stepanets}

Пришло время нам повзрослеть и научиться брать ответственность за свою жизнь на
себя. Все к лучшему. Нам надо запомнить бегство наших «партнеров» из Киева,
сделать выводы и жить дальше без иллюзий

\iusr{Ольга Иванова}

А как принимать информацию о чартерных рейсах наших богатых людей? Нам говорят
с их тв о патриотизме, а сами вывозят со страны свои семьи. Значит у них дети,
родители, а у нас нет. Противно все это наблюдать

\iusr{Зіновій Музичак}

Простий, зрозумілий і реальний для втілення алгоритм різкого підвищення
боєздатності нашої країни в частині \enquote{люди}

Перед Україною стоїть загроза бути знищеною зовнішнім і внутрішнім ворогами.
Зовнішній це Росія, внутрішній це корупція.

Щоб Україна вистояла перед зовнішнім ворогом, потрібне швидке підвищення
обороноздатності (реформування) Української армії. Для цього потрібні гроші.

В цій пропозиції не буду зупинятись на всіх грошах, які акумулює українська
корупція, а тільки зупинюсь на грошах, які акумулює корупція у військово
мобілізаційній сфері.

Тут я пропоную гроші, які зараз акумулює корупція у військово мобілізаційній
сфері ПЕРЕНАПРАВИТИ на статтю бюджету ПІДВИЩЕННЯ ОБОРОНОЗДАТНОСТІ.

В Україні є багато громадян з ПАЦИФІСТСЬКИМИ настроями, які хочуть жити в
демократичній державі, але не хочуть її захищати.

Тому пропоную ввести ПОДАТОК НА ПАЦИФІЗМ. Ті ж хабарі, які зараз осідають в
кишенях корупціонерів, пропоную акумулювати в державному бюджеті.

Найголовніше правило - ЗАКОН повинен СТОСУВАТИСЬ ВСІХ, бути ЗРОЗУМІЛИЙ ВСІМ, і
враховувати БАЖАННЯ І МОЖЛИВОСТІ ВСІХ.

Українські збройні сили повинні складатись і наступного:

1. ПРОФЕСІЙНА КОНТРАКТНА АРМІЯ, яка є зараз в Україні.

2. ЗАГАЛЬНООБОВ'ЯЗКОВА СТРОКОВА служба.

Призов на строкову службу в Україні був і є вибірковим. Служать строкову службу
не всі. Служать вибірково (Служать ті, хто \enquote{НЕ ВІДКУПИВСЯ}). А це менше 20\%
від загальної кількості що повинні були б служити.

В той же час, щоб зараз збільшити кількість військових строкової служби,
потрібно відкривати нові військові частини (для цього потрібні гроші).

Я не закликаю відкривати нові військові частини. Краще зменшити строк строкової
служби і зосередитись виключно на ВІЙСЬКОВІЙ ПІДГОТОВЦІ строковиків.

За небажання служити строкову службу, треба буде заплатити РАЗОВИЙ податок.

Разовий податок повинен бути невеликий (менший мінімальної зарплати за період
служби), і корелюватись між кількістю призову і кількістю, що підлягає призову
у цьому році.

На той час, коли ровесники служать строкову службу, платник податку не служить.
Але після часу служби, всіх зараховують \enquote{в ЗАПАС} (хто служив і хто заплатив
податок).

3. Військова підготовка в ЗАПАСІ.

Встановити цикл, на протязі якого, кожний громадянин, що знаходиться в ЗАПАСІ
повинен пройти ПЕРЕПІДГОТОВКУ (певний термін) в спеціальному навчальному
центрі.

Такі НАВЧАЛЬНІ ЦЕНТРИ треба відкривати. Для цього потрібні гроші, які
появляться в результаті введення ПОДАТКУ НА ПАЦИФІЗМ.

Ті хто не бажає проходити таку перепідготовку, платить РІЧНИЙ податок і спить
спокійно на протязі року.

РІЧНИЙ податок повинен також корелюватись між планом перепідготовки (кількістю
що можуть прийняти НАВЧАЛЬНІ ЦЕНТРИ) і кількістю громадян, що перебувають в
ЗАПАСІ. РІЧНИЙ податок також повинен бути не високий.

4. Стосовно СИСТЕМИ ТЕРИТОРІАЛЬНОЇ ОБОРОНИ, то не можна допустити щоб ТрО стала
формою \enquote{відкосити} молодих і здорових військовозобов'язаних від
загальної військової підготовки в ЗАПАСІ. Я б запропонував включати в СИСТЕМУ
ТЕРИТОРІАЛЬНОЇ ОБОРОНИ військовозобов'язаних, які старші за 50 років, а також
таких тих, хто отримав поранення, чи психологічні травми (полон) під час
бойових дій. А також громадян, які не хочуть проходити загальну військову
підготовку в ЗАПАСІ, але готові заплатити невеликий податок, щоб служити в ТрО.

Слід пам'ятати, що у вищезгаданих військових службах (СТРОКОВА; Підготовка в
ЗАПАСІ; ТЕРИТОРІАЛЬНА ОБОРОНА) потрібно проводити активну АГІТАЦІЙНУ роботу з
метою укомплектування ПРОФЕСІЙНУ КОНТРАКТНУ АРМІЮ.

Слід зрозуміти, що громадянин, який є підприємцем, висококваліфікованим
спеціалістом, чи навіть хорошим працівником, принесе державі на своєму робочому
місці більше користі, ніж його служба в армії. А критерієм його фаховості буде
його здатність заплатити Податок на ПАЦИФІЗМ.

Це ж стосується і громадян, які хочуть працювати за кордоном. Слід не
допускати, щоб Вони платили хабарі корупціонерам і їхали на роботу за кордон.
Краще нехай сплачують податок на ПАЦИФІЗМ і з чистою совістю їдуть.

Стосовно \enquote{релігійних переконань}, то кожна справжня побожна людина з
радістю заплатить податок, бо Вони чесні і старанні у праці, та й релігійна
громада їм допоможе.

Це ж стосується людей, про яких колись говорили, що \enquote{мають БРОНЬ}. Ці
люди висококваліфіковані і заплатити податок для них не буде важко. Та й
підприємство, яке зацікавлене в них, їм допоможе

\iusr{Александр Малюк}
\textbf{Зіновій Музичак} В мене схожі думки, відмінності лише в деталях.

\iusr{Dmytro Khokhlov}
Це необхідно перекласти всіма мовами і розповсюдити серед основних закордоних змі.

\iusr{Евгений Сергеевич}
Прекрасный текст! Абсолютно верный ход мыслей

\iusr{Nikita Podobedov}
Ну хоть не говорят, что надо просто перестать стрелять!

\iusr{Василь Лаптійчук}
Именно.

\iusr{Ruslan Sem}

післявоєнний світопорядок рухнув ще в 2014 році (і ще раніше в Молдові,
Грузії), зараз так - судоми...

\iusr{Виктор Щербина}
Хто куди, а я в госпталь. До людей в білих халатах.

\iusr{Дмитрий Рыжаков}

Может это все попытки склонить Украину к условиям выдвигаемым россией?
Пристайко вот обмолвился, что мы бы могли рассмотреть отказ от нато. И весь
этот коллективный запад не может понять, почему в Украине нет паники и не бегут
за туалетной бумагой и гречкой люди.

\iusr{Вікторія Матусевич}

А Президента з його командою ні до чого не хочеш призвати?

\iusr{Татьяна Оприщенко}

Видимо, врачами руководит режиссер. А больному пора забить на всё, заняться
здоровым образом жизни и перестать думать о болезни

\iusr{Татьяна Нумова}

Возникает логический вопрос, а где эти западные ценности? и за что мы вообще
боремся??? за этих подлых перебежчиков?

\iusr{Вячеслав Редько}
Браво, Сергей!!!

\iusr{Сергей Рогожин}

Шумиха про \enquote{вот-вот нападёт} - это элементы гибридной войны. Раньше расия вела
гибридную войну против Украины, теперь Америка ведёт гибридную войну против
расии.

Даже если расия реально собиралась напасть, то теперь она вынуждена испуганно
оглядываться и оправдываться \enquote{да это не мы, да мы ваще не...}.

И теперь реальное вторжение невозможно.

А с расией только так и надо: язык силы, угрозы и информационные манёвры.

Да и Европа начала чухаться и повнимательнее присматриваться к Украине.

\iusr{Володимир Неплюєв}

Все ж шансів у нас більше ніж онкохворого, і від нашого спротиву залежить і
світова реакція.

\iusr{Andriy Fedyanin}
++

\iusr{Людмила Пухова}

И деградация с последующей ликвидацией, и к сожалению без вазелина
@igg{fbicon.face.wink.tongue}{repeat=3} 

\iusr{Eugen Cherevko}

Плачем - горю не помочь. Если разобраться - в ситуации есть 2 составляющих.

1. То, что происходит и может произойти - есть отложенный штраф за все, что мы
делали с 1991г. Не хотели вступать в НАТО (в отличие от стран Прибалтики).
Держали в Крыму флот РФ (как же, взвамен дешевый газ!) Голосовали за
Януковичей. Мы не хотели \enquote{рвать} связи с Рашей. Не проводили
болезненных рыночных реформ, как это было в Польше, в Прибалтике. За все это -
голосовало каждый раз (ну почти!) наше большинство. Ну да, платить приходится
всем. Это - железный закон. А возьмем хотя бы последние выборы: у нас хватило
ума проголосовать за человека который считал, что достаточно просто перестать
стрелять. Выбирали между войной и позором. Выбрали позор. А война, как мы видим
- \enquote{не забарилася}. Ладно, проехали, а машины времени - пока не
придумали. Так что, по первому пункту дискутировать не буду..

2. Так что же делать? Первое: не ныть. И запомнить: нам - никто ничего не
должен. Второе: не прогибаться. Прогнемся сейчас - просто отсрочим беду на то
время, когда будем к ней готовы еще меньше. То, что от нас требует Раша - чтоб
мы подготовили себя к более комфортному для нее (и более страшного для нас) -
съедению.

Значит, делаем сейчас - все, что должно. С холодной головой. Кто считает, что
лучше убежать - не вопрос. Не осуждаю. Быть может, и сам бы сбежал, ежели б
было куда. Или, спрятался, было бы где. И вообще, нужно понимать, что польский,
словацкий, венгерский и румынский кордоны будут скорее всего закрыты. Так что
можно замерзнуть или умереть с голоду прямо на границе. Или в пробке на шоссе.
Или, наоборот, на пустом шоссе с пустым баком, быть ограбленым и прирезанным
любителем легкой наживы. Такие персонажи - часто материализуются в смутные
времена.

Не хочешь? Если так - Наилучшее для тебя убежище - Украинская Армия. ВСУ. Не
можешь бегать с автоматом, боишься \enquote{джавелина}? Вытаскивай раненых,
подноси снаряды и патроны, носи нашим украинским солдатам еду и сигареты. Дело
для тебя - найдется, не сомневайся.

Вот такая картина. А будет ли нам кто-то помогать? Будут! Но лишь в одном
случае: если сопротивление будет реальным. Чем эффективнее - тем больше
помощь. Решим сдаться - разведут руками. 100\%. Мол, они сами это приняли и
пусть так живут. А жизнь эта, скажу я вам, так чтоб была хорошей - так то вряд
ли.

Вот такая, други картина.

Ну а пронесет - тогда Слава Богу. Но это - пока другая история. \enquote{Вдову Клико}
для встречи 2023г. - пока покупать рано.


\iusr{Олег Янков}

По Вашей метафоре \enquote{обречённому} остаётся надеяться на скрытые резервы или на
Бога. Я разъясню эту надежду с точки зрения Сознания. Нынешняя угроза войны -
это следствие разрушительного мышления людей во власти при всех президентах и
разрушительные процессы в обществе (коллективное сознание). Это привело к
отсутствию Единства в обществе, разрушенной экономике, ресурсной зависимости,
слабой армии и всех структур власти, очень низкого статуса страны на
международной арене. У любого агрессивного соседа возникает соблазн у
нерадивого хозяина что-то забрать. Болезнь будет прогрессировать с угрозой для
жизни, если пациент не захочет страстно жить. Для этого нужно включить инстинкт
самосохранения и начать менять себя, а не за лекарствами по аптекам бегать.

Рецепт: 

1. Обществу выйти из самообмана. Ложь всегда разрушает. 

2. Признать, что все политики врут обществу. Значит все законы, принятые
лжецами, потенциально несут вред обществу и на службу в госаппарат они наняли
потенциальных лжецов. 

3. Обществу сформировать запрос на наличие навыка честности перед обществом у
политиков и потребовать обязательное его наличие у депутатов и президента,
которые перенесут эту обязательность у чиновников. 

4.  Обществу сформировать независимую от президента структуру, которая будет
обучать навыку мышления честности перед обществом политиков во власти и вести
контроль за его применением. 

5. Общество даёт этой структуре полномочия рекомендовать ЦИК освобождать от
статуса депутат человека, которые 5 раз проявил действия во вред обществу (в
том числе и ложь). После внедрения в Украине таких механизмов будут запущены
процессы, которые изменят общество и страну.

\iusr{Igor Nesterenko}

Не будет никакой большой войны. Это видно и по многим признакам проходящих
логистических операций ВС противника и по логике его мышления. Нет ни сил, ни
средств, ни даже понимания у него, как выйти из создавшейся по своей очередной
тупости ситуации. Завтрак одинокого подполковника в Брисбене уже никогда не
перейдет в банкет по случаю победы, хотя бы потому, что несколько лет он и так
принимает пищу в тоскливом одиночестве. Позади ланч на двоих с уменьшенной
нанокопией, позади кофе между лыжными спусками. Даже натурные съемки в
алтайских предгорьях и ночевка в палатке не радует. Собственно старику все не в
радость. По кремлевским подвалам распускают сплетни, что осталось Карабасу
тыкать носом в холодный очаг до весны 23-го. Не стоит поддаваться психозу,
приходящему из СМИ. Во-первых потому, что от каждого из вас ничего не зависит.
Создано грандиозное поле пропагандистской войны, на котором победы и потери
принадлежат политиканам различных мастей и пропагандонам. К тому же сравнение в
начале текста далеко не корректное. Как сказал один подводник с \enquote{Комсомольска},
попав в отделение радиологии, - у вас здесь как на затонувшей подводной лодке,
-полная безнадега... Точь в точь как у подполковника, волею случая напялившего
на себя мундир известного мировой истории ефрейтора. А армия наша справится.

\iusr{Андрей Дуденко}
Очень похоже на недавние события в Афганистане. Нет?

\iusr{Елена Ковтуненко}
Просто супер! подписываюсь под каждым словом!

\iusr{Руслан Сидорович}

Просто эпично!! Запад виноват, чо вернул войну, которая идёт с 14-го года в
повестку дня мировой прессы? Запад виноват, что согласно протоколу вывозит
семьи своих представительств? Запад мало помог? Какие ценности олицетворяет
Украина?? Бенины майнинговые фермы? Сваты, кумы, кварталы?? Вымогатели
каке-то.. требуют от запада.. Напоминает, ка к один дурачек полгода назад
вкинул на Байдена, почему Украина ещё не в НАТО!!



\end{itemize} % }
