% vim: keymap=russian-jcukenwin
%%beginhead 
 
%%file 13_09_2021.fb.bondarenko_pavel.1.moskva_istoria
%%parent 13_09_2021
 
%%url https://www.facebook.com/urtab/posts/4262788200495213
 
%%author_id bondarenko_pavel
%%date 
 
%%tags dnepropetrovsk,herson,istoria,moskva,novorossia,odessa,ukraina
%%title Як Москва писала для нас нашу історію? Будь ласка
 
%%endhead 
 
\subsection{Як Москва писала для нас нашу історію? Будь ласка}
\label{sec:13_09_2021.fb.bondarenko_pavel.1.moskva_istoria}
 
\Purl{https://www.facebook.com/urtab/posts/4262788200495213}
\ifcmt
 author_begin
   author_id bondarenko_pavel
 author_end
\fi

Як Москва писала для нас нашу історію? Будь ласка.

Сьогодні місто Дніпро відзначає день свого заснування. Сталося це нібито 13
вересня 1776 року за повелінням  Катерини ІІ.

Це брехня. Дніпро набагато старший.

На цьому місці, на лівому березі Дніпра існувало давнє козацьке місто Самара
(Стара Самара), центр однойменної паланки Війська Запорізького. Після того, як
московський цар порушив своє "царське нерушиме слово" перед українцями й уклав
сепаратний Андрусівський мир з поляками, розділивши Україну, у Самару прийшли
московські загарбники, вигнали з міста козаків і заснували тут Богородицьку
фортецю. Сталося це у 1667-1668 роках.

\ifcmt
  pic https://scontent-mia3-2.xx.fbcdn.net/v/t1.6435-9/241841131_4262785923828774_8237284884976067134_n.jpg?_nc_cat=105&_nc_rgb565=1&ccb=1-5&_nc_sid=730e14&_nc_ohc=bSjGjvVrh5YAX9jPiKO&_nc_ht=scontent-mia3-2.xx&oh=fde638a28a6b6c87ef09ad0ce57c092d&oe=61838723
  @width 0.4
  %@wrap \parpic[r]
\fi

Про старовинне козацьке місто зараз нагадує один з районів міста Дніпра -
Самарський.

А навпроти Самари, на правому березі Дніпра стояло інше старе козацьке місто,
центр Кодацької паланки Війська Запорізького - Новий Кодак. Воно відоме
принаймні з середини ХVІІ століття. 

Його теж було знищено московитами під час "заснування" Катеринослава. 

Про це стародавнє козацьке місто нагадує інший район міста Дніпра, який
називається Новокодацький.

Обидва цих міста позначено на старовинних мапах. Обидва стараннями московської
пропаганди забуті й навряд чи усім мешканцям Дніпра про них узагалі відомо. Для
них це лише назви адміністративних одиниць міста.

Історія з Катеринославом-Дніпропетровськом-Дніпром один в один нагадує історію
Луганська-Ворошиловграда-Луганська. Той, відповідно до написаної у Москві
історії, теж було "засновано" Катериною ІІ 1795 році нібито у "Дикому полі".

Насправді Луганськ є лише продовженням історії старих поселень Кальміуської
паланки - Кам'яного Броду та Великої Вергунки. Про це нагадують назви одного з
районів міста Луганська - Камя`нобрідського, та селища Велика Вергунка.

Усі ці вигадки про "заснування міст" за часів Катерини ІІ мають одну мету:
змусити українців забути власну історію. Забути про те, що Маріуполь
засновувала не Катерина, а що був він раніше містом-фортецею Війська
Запорізького Домахою. Що Донецьк стоїть на місці старого запорізького поселення
Олександрівка. Що Торецьк (колишній Дзержинськ) засновано не імперцями у 1804
році, а богуславським козаком Антоном Щербиною на півтора століття раніше. Це
місто до 1936 року і називалося Щербинівкою. Що Лисичанськ то козацькі
зимівники Лисичий Байрак на Дінці та Вище на Дінці. Що Одеса то місто Коцюбіїв,
а Херсон - то місто Білховичі.

І далі за списком.

Вигадки московських імперських "істориків" мають на меті обґрунтування
претензій Москви на ці землі. На оту вигадану у ХІХ столітті "Новоросію", за
яку нині воюються кремлівські найманці та "іхтамнєти". Так брехня проростає
війною та кров`ю. І українці мусять це знати.

Мусять знати свою історію і дніпряни. І згадувати про Самар і Новий Кодак, коли
нинішнім вечором питимуть пиво під урочистий, дорогий і красивий феєрверк.

P.S.

Народ, який не знає або забув своє минуле, не має майбутнього - сказав
розумничка Платон.

Цікавитеся минулим нашої землі та народу? Для вас невеличка допомога -
розслідування найбільш важливих і таємничих сторінок історії, автором яких маю
честь бути. Для широкого кола читачів.

Класичні паперові книжки (автограф гарантовано) ви можете замовити за цим посиланням: 

\url{http://pavlopraviy.blogspot.com/2019/03/blog-post_16.html}

Електронні версії за символічну платню тут: 

\url{https://pavlopraviy.blogspot.com/2018/10/blog-post.html}
