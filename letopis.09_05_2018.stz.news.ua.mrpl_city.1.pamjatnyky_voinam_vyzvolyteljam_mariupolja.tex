% vim: keymap=russian-jcukenwin
%%beginhead 
 
%%file 09_05_2018.stz.news.ua.mrpl_city.1.pamjatnyky_voinam_vyzvolyteljam_mariupolja
%%parent 09_05_2018
 
%%url https://mrpl.city/blogs/view/pamyatniki-voinam-vizvolitelyam-mariupolya
 
%%author_id demidko_olga.mariupol,news.ua.mrpl_city
%%date 
 
%%tags 
%%title Пам'ятники Воїнам-визволителям Маріуполя
 
%%endhead 
 
\subsection{Пам'ятники Воїнам-визволителям Маріуполя}
\label{sec:09_05_2018.stz.news.ua.mrpl_city.1.pamjatnyky_voinam_vyzvolyteljam_mariupolja}
 
\Purl{https://mrpl.city/blogs/view/pamyatniki-voinam-vizvolitelyam-mariupolya}
\ifcmt
 author_begin
   author_id demidko_olga.mariupol,news.ua.mrpl_city
 author_end
\fi

Зараз ми вступаємо в період, коли для багатьох людей Друга світова війна – вже
легендарна історія, все менше залишається сучасників тих складних часів. У
зв'язку з цим виникає потреба у зверненні до пам'ятників, присвячених тим
подіям. Маріупольці пам'ятають і славлять подвиги воїнів-визволителів міста: на
честь з'єднань і героїв, що принесли довгоочікувану перемогу, встановлено
меморіальні дошки та пам'ятники, що прославляють переможців та їхні ратні
подвиги...

Операція зі звільнення Маріуполя мала величезне військове, політичне та
економічне значення. Німці зробили все можливе, щоб перетворити територію міста
в добре укріплений район. У звільненні Маріуполя були задіяні дві стрілецькі
дивізії: 221-ша (командуючий – генерал-майор Іван Іванович Блажевич) і 130-та
Таганрозька стрілецька дивізія (командуючий – генерал-майор Костянтин
Васильович Сичов), маріупольський десант (384-й окремий батальйон морської
піхоти, командир лейтенант К. Ф. Ольшанський), 4-й гвардійський механізований
корпус під командуванням генерал-лейтенанта Т. І. Танасчишина, 9-та винищувальна
авіаційна дивізія і частини Кубанського козачого кавалерійського корпусу.

5 вересня 1943 р. воїнами Південного фронту під командуванням
генерал-полковника Ф. І. Толбухіна почалася операція зі звільнення Маріуполя.

Уздовж моря з боями наближалися до Маріуполя воїни 130-ї стрілецької
Таганрозької дивізії під командуванням полковника К. В. Сичова.

У північно-східному напрямку вела бої 221-ша стрілецька дивізія під
командуванням полковника І.І. Блажевича. Звільнивши від нацистів територію
заводу ім. Ілліча і прилеглих до нього селищ, бійці дивізії з півночі увійшли в
місто.

7 і 8 вересня противник чинив сильний опір, безперервно контрактуя частини
221-ї стрілецької дивізії. Атака радянських бійців завершилася успіхом.
Мужність і відвагу проявила санінструктор 695-го стрілецького полку П.
Ковальова. Нехтуючи смертельною небезпекою, вона надавала допомогу пораненим і
під вогнем винесла з поля бою понад сорок солдатів і офіцерів. Сама присутність
на позиціях цієї мужньої дівчини вселяла в бійців впевненість у перемозі, і
багато хто, незважаючи на поранення, не покидали поле бою. Складно уявити
зараз, через стільки років, які тяготи довелося перенести тендітним дівочим
плечам... Пригадуються рядки Ю. Друніної:

\begin{center}
\em\bfseries

Я только раз видала рукопашный,

Раз наяву. И тысячу - во сне.

Кто говорит, что на войне не страшно,

Тот ничего не знает о войне.	
\end{center}

За героїзм, проявлений у цих боях, старшина медичної служби Парасковія
Ковальова була нагороджена орденом Червоного Прапору.

До 15 години 10 вересня 1943 р. Маріуполь був повністю очищений від окупантів.

За мужність і відвагу, проявлені при звільненні міста, 130-та стрілецька
дивізія була нагороджена орденом Червоного Прапора, а 221-ша дивізія отримала
почесне найменування \enquote{Маріупольська}.

\ii{09_05_2018.stz.news.ua.mrpl_city.1.pamjatnyky_voinam_vyzvolyteljam_mariupolja.pic.1}

7 вересня 1985 р. на пр. Будівельників відкрито \textbf{пам'ятний знак} \textbf{\enquote{Воїнам 221-ї
стрілецької дивізії}}. Знак становить собою вертикально стоячий природний камінь
у вигляді обеліска, округлений потужними металевими конструкціями, що
символізують міць армії й героїзм воїнів. Автор: архітектор В. Пантов.

На згадку про бойові подвиги воїнів 130-ї та 221-ї стрілецьких дивізій на
Комсомольському бульварі 6 листопада 1975 р. відкрито \textbf{пам'ятний знак і запалено
Вічний вогонь}. Пам'ятний знак являє собою 76-міліметрову артилерійську гармату,
встановлену на прямокутному постаменті. На меморіальній дошці напис:
\emph{\enquote{Воїнам-визволителям 221-ї Маріупольської та 130-ї Таганрозької дивізії}}.

\textbf{\enquote{Пам'ятник воїнам-визволителям МіГ-17}}. В основі стріли пам'ятна дошка з
написом: \enquote{Воїнам-визволителям вдячні жданівці присвячують}.

\ii{09_05_2018.stz.news.ua.mrpl_city.1.pamjatnyky_voinam_vyzvolyteljam_mariupolja.pic.2}

На честь подвигу танкістів 4-го гвардійського механізованого корпусу під
командуванням генерал-лейтенанта Т.І. Танасчишіна, які брали участь у
визволенні Маріуполя від нацистських загарбників 10 вересня 1943 р.,  на вулиці
Карпинського встановлено \textbf{пам'ятний знак – бойовий танк Т-34 № 121}. На
постаменті закріплено дві меморіальні дошки з написами:

\emph{\enquote{Безсмертній мужності та героїзму радянських танкістів присвячується}}.

\emph{\enquote{Споруджений на честь 30-річчя Перемоги радянського народу у Великій
Вітчизняній війні 9.05.1975 р.}}. Архітектор – А. Клюєв.

\ii{09_05_2018.stz.news.ua.mrpl_city.1.pamjatnyky_voinam_vyzvolyteljam_mariupolja.pic.3}

У Маріуполі встановлено пам'ятник і морякам-десантникам – \enquote{Бронекатер № 346},
який представлено на постаменті у вигляді стилізованих хвиль. Бронекатер має
свою історію. Побудований він в місті Зеленодольську на Волзі на початку
радянсько-німецької війни. Бойова біографія катера почалася під Сталінградом і
тривала на Азовському морі, в районі міст Маріуполь, Бердянськ, Темрюк, в
Керченській протоці.

У 1944 р. катер залізницею переправлений на Дніпро, де продовжив свій бойовий
шлях. Далі був Дністер, а закінчився його бойовий шлях на параді Дунайської
флотилії в Будапешті в 1945 р.

Пам'ятник відкритий 10 вересня 1974 р. Автори постаменту: скульптор В.Л.
Пацевич, архітектор В. С. Соломін.

\ii{09_05_2018.stz.news.ua.mrpl_city.1.pamjatnyky_voinam_vyzvolyteljam_mariupolja.pic.4}

На честь 55-ї річниці Перемоги у радянсько-німецькій війні у травні 2000 р. на
площі Фонтанів відкрито монумент. Центральне місце в ньому займає
воїн-переможець з лавровою гілкою у витягнутій руці. Монумент виконаний із
залізобетону з використанням листової міді. Автори: художники Л. Кузьмінков і
В. Константінов.

\ii{09_05_2018.stz.news.ua.mrpl_city.1.pamjatnyky_voinam_vyzvolyteljam_mariupolja.pic.5}

Всі пам'ятники Маріуполя, присвячені визволителям нашого міста від нацистських
окупантів, сьогодні нагадують нам про ті далекі, але страшні події, про
героїчну волю наших визволителів і про їхнє прагнення до свободи. Є події, над
якими не владний час, які назавжди залишаються в пам'яті народній. Такою подією
стала Друга світова війна. І сьогодні ми продовжуємо вшановувати пам'ять тих,
хто подарував свободу своїм нащадкам! 

\href{https://youtu.be/czLn_KCfByU}{%
Памятники Мариуполя, Мариупольское телевидение, youtube, 08.05.2018%
}

\ii{09_05_2018.stz.news.ua.mrpl_city.1.pamjatnyky_voinam_vyzvolyteljam_mariupolja.pic.6.video}

\ii{insert.author.demidko_olga}
