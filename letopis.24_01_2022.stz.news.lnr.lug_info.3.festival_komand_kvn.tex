% vim: keymap=russian-jcukenwin
%%beginhead 
 
%%file 24_01_2022.stz.news.lnr.lug_info.3.festival_komand_kvn
%%parent 24_01_2022
 
%%url https://lug-info.com/news/festival-komand-kvn-sobral-detej-i-molodezh-iz-luganska-lutugino-i-pereval-skogo-rajona
 
%%author_id news.lnr.lug_info
%%date 
 
%%tags donbass,festival,kvn,lnr,molodezh,zhizn
%%title Фестиваль команд КВН собрал детей и молодежь из Луганска, Лутугино и Перевальского района
 
%%endhead 
 
\subsection{Фестиваль команд КВН собрал детей и молодежь из Луганска, Лутугино и Перевальского района}
\label{sec:24_01_2022.stz.news.lnr.lug_info.3.festival_komand_kvn}
 
\Purl{https://lug-info.com/news/festival-komand-kvn-sobral-detej-i-molodezh-iz-luganska-lutugino-i-pereval-skogo-rajona}
\ifcmt
 author_begin
   author_id news.lnr.lug_info
 author_end
\fi

Фото: Марина Сулименко / ЛИЦ

Семь команд из Луганска, Лутугино и Перевальского района приняли участие в
фестивале КВН \enquote{Студенческий марафон}, который прошел в Луганском
государственном университете имени Владимира Даля. Об этом с места события
передает корреспондент ЛИЦ.

\ii{24_01_2022.stz.news.lnr.lug_info.3.festival_komand_kvn.pic.1}

Организатором юмористического состязания стала общественная организация (ОО)
\enquote{Молодая гвардия}, а участниками - школьники, студенты и представители
творческих объединений.

\ii{24_01_2022.stz.news.lnr.lug_info.3.festival_komand_kvn.pic.2}

Команды выступили в конкурсах \enquote{Приветствие} и \enquote{Разминка}. Свои шутки они
посвятили ситуации на международной арене, народным традициям, взаимоотношениям
с близкими людьми, учебе и другим темам.

\ii{24_01_2022.stz.news.lnr.lug_info.3.festival_komand_kvn.pic.3}

Одной из особенностей фестиваля стало участие в нем команды \enquote{Исключение из
правил}, в состав которой вошли учащиеся младших классов школы № 34 поселка
Червоный Прапор Перевальского района.

\ii{24_01_2022.stz.news.lnr.lug_info.3.festival_komand_kvn.pic.4}

\enquote{С шутками нам помогали заместитель директора школы и учитель. Нам очень
понравилось участвовать. Это дает нам радость и веселье. Зрители смеялись,
хлопали. Мы справились со своим заданием}, - рассказал участник команды Назар.

\ii{24_01_2022.stz.news.lnr.lug_info.3.festival_komand_kvn.pic.5}

Кроме того, экспериментальным юмором зрителей удивляла \enquote{Невидимая сборная} из
Луганска. Вместо личного присутствия ее участники подготовили для публики
аудиовыступление.

\ii{24_01_2022.stz.news.lnr.lug_info.3.festival_komand_kvn.pic.6}

Победителем фестиваля стала команда \enquote{Перевальск}, второе и третье места заняли
\enquote{Сборная Перевальского района} и \enquote{Сборная Лутугино}. В мероприятии также
участвовали луганские команды Old school и \enquote{Своим ходом}. Все участники
получили призы и грамоты от организаторов и спонсоров фестиваля.

Выступления команд оценивало жюри, в состав которого вошли депутат Народного
Совета ЛНР Иван Санаев, советник министра образования и науки ЛНР Константин
Кучер, начальник управления спорта и молодежи Министерства культуры, спорта и
молодежи ЛНР Олег Шеренешев, председатель ОО "Молодая гвардия" Даниил
Степанков, чемпион Высшей лиги КВН Валдес Романов и начальник организационного
отдела аппарата республиканского исполнительного комитета общественного
движения \enquote{Мир Луганщине} Сергей Гербеев.

\enquote{Нам на почту поступило более 20 заявок от команд, которые изъявили желание
принять участие в фестивале. Затем были редакторские просмотры, отборы. К
сожалению, не все команды попали на это мероприятие. Но ребята, которые
выступают сегодня, показали высокий уровень подготовки в предыдущих турах}, -
рассказал Степанков.

Члены жюри отметили, что фестиваль КВН стал еще одной площадкой для
популяризации игры среди молодежи ЛНР.

\enquote{Никаких особых критериев оценивания нет, главное - чтобы было смешно. В
Республике очень сильные команды. Они выступают в Российской Федерации, выходят
на более высокий уровень}, - сказал Романов.
