% vim: keymap=russian-jcukenwin
%%beginhead 
 
%%file 19_07_2020.fb.lnr.4
%%parent 19_07_2020
 
%%endhead 
\subsection{БАКТЕРИОЛОГИЧЕСКАЯ БОМБА США В СРЕДНЕЙ АЗИИ}
\url{https://www.facebook.com/groups/LNRGUMO/permalink/2860185447426344/}
  
\vspace{0.5cm}
{\small\LaTeX~section: \verb|19_07_2020.fb.lnr.4| project: \verb|letopis| rootid: \verb|p_saintrussia|}
\vspace{0.5cm}


Международное радио Китая об американской лаборатории биологической войны в
Казахстане.

В последние дни на территории Казахстана распространяется неизвестная инфекция,
которая гораздо опаснее COVID-19, и уровень ее летальности выше, чем от
коронавируса. Министерство здравоохранения Казахстана и другие агентства
проводят сравнительные исследования и пока не могут определить природу вируса,
вызывающего пневмонию.  Президент Казахстана Касым-Жомарт Токаев заявил, что
необходимо признать ситуацию с пневмонией чрезвычайной.

При этом министр здравоохранения Алексей Цой уточнил, что за первое полугодие
2020 года в Казахстане было зарегистрировано 98 546 случаев заболевания
пневмонией.

По сравнению с прошлым годом данный показатель вырос на 55,4%.  Цой добавил,
что за прошлый месяц число зафиксированных смертей от пневмонии составило 628
случаев.

Он заявил, что у половины из всех пациентов с пневмонией в Казахстане —
отрицательный тест на коронавирус.  Местные врачи считают это последствиями все
того же коронавируса нового типа. Однако некоторые ученые считают, что
заболевание вызвано не этим вирусом.

По их мнению, если бы причиной новой пневмонии был коронавирус, то это можно
было бы легко зафиксировать, а значит, опасность возникновения неизвестной
ранее инфекции все же присутствует.

То, что в Казахстане, расположенном в центре континента Евразия, обнаружили
неизвестный вирус страшнее COVID-19, вызвало у местных жителей интерес к работе
Центральной референс-лаборатории (ЦРЛ) Казахстана в Алматы, построенной и
работающей при поддержке Департамента обороны США.

Ссылаясь на слова секретаря Совета безопасности РФ Николая Патрушева, агентство
«Спутник» в России сообщает, что в настоящее время США ввели в эксплуатацию
многочисленные биологические лаборатории по всему миру, в том числе в
Казахстане.

Строительство Центральной референс-лаборатории в Алматы началось 1 апреля 2010
года на средства США.

В свое время Вашингтон объявил, что построит в этом городе самую большую
биологическую лабораторию в Центральной Азии, функция которой заключается в
исследовании и сохранении чрезвычайно опасных вирусов.

В статье, опубликованной 11 июля на сайте «Stanradar» Кыргызстана, говорится,
что Соединенные Штаты являются реальным владельцем казахстанской биологической
лаборатории.

Задачей лаборатории в Алматы является сбор данных ДНК местного населения,
животных и растений, а также возбудителей опасных заболеваний.

В статье также отмечается, что Казахстан — удобный полигон для проведения
испытаний и опытов как над жителями, так и над сельскохозяйственными и дикими
животными. Рядом — границы России, Китая, Кыргызстана и Узбекистана.

И особо опасные заболевания в случае их сознательного или случайного
распространения могут свободно достичь территории этих стран.

«Все больше жителей Казахстана узнают о природе биологической лаборатории США в
их стране», — заявил 11 июля сопредседатель Социалистического движения
Казахстана Айнур Курманов в интервью Интернет-газете «ZONAkz».

«Во время пандемии коронавируса опасения граждан по поводу работы ЦРЛ в Алматы
только усилились.

Поэтому пребывание на нашей земле американских военных специалистов и
биологических объектов, построенных на деньги Пентагона, — это уже серьёзная
геополитическая проблема.

Я её называю бактериологической бомбой США в Центральной Азии, которая уже
начала действовать», — отметил Айнур Курманов.

По сообщению агентства «Спутник» в России, США создали более 200 биологических
лабораторий по всему миру, в том числе на территории таких стран, как Украина,
Грузия, Армения, Казахстан, Афганистан и др.

В правительстве РФ считают, что Соединенные Штаты пытаются скрыть истинную цель
создания их биологических лабораторий в странах мира.

США создали множество биологических лабораторий на территории бывшего
Советского Союза, что вызывает серьезную обеспокоенность у населения
соответствующих государств и соседних стран.

«В МИД РФ обратили внимание на усиление американского биологического
присутствия за пределами США, в том числе и на территории бывших советских
республик.

Нельзя исключать, что в подобных референс-лабораториях американцами в третьих
странах ведутся работы по созданию и модифицированию различных возбудителей
опасных заболеваний, в том числе и в военных целях», — подчеркнула официальный
представитель МИД России Мария Захарова.

В то же время секретарь Совета безопасности РФ Николай Патрушев призвал страны
мира усилить контроль за распространением эпидемии и исследованиями в области
биобезопасности.  Perevodika

P.S. Colonel Cassad

Можно только представить накал истерики в Штатах, если бы РФ имела развернутую
лабораторию биологической войны на Кубе или в Венесуэле, а в самих США начался
рост нетипичных заболеваний, вроде недавно обнаруженной бубонной чумы. Но пока
что наблюдаем за этим исключительно у своих границ.  Китаю разумеется выгодно
поддерживать российские обвинения, так как это позволяет переводить стрелки на
США в вопросе обвинений, связанных с первоисточником коронавируса, который
изучался в биологических лабораториях США на территории Украины и Средней Азии.
США разумеется будут продолжать делать вид, что эти обвинения беспочвенные и у
Китая, и РФ нет прямых доказательств причастности США, и сама деятельность этих
лабораторий будет продолжаться, в том числе и на территории формального
союзника РФ по ОДКБ (это такой же оксюморон, как если бы идентичная китайская
или российская лаборатория функционировала на территории стран НАТО).
  
