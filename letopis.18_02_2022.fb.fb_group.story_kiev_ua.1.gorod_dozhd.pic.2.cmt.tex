% vim: keymap=russian-jcukenwin
%%beginhead 
 
%%file 18_02_2022.fb.fb_group.story_kiev_ua.1.gorod_dozhd.pic.2.cmt
%%parent 18_02_2022.fb.fb_group.story_kiev_ua.1.gorod_dozhd
 
%%url 
 
%%author_id 
%%date 
 
%%tags 
%%title 
 
%%endhead 

\iusr{Татьяна Сирота}

Богдана Хмельницкого,15.

Здание Природоведческого музея.

Строилось для Ольгинской гимназии.

В 1909 году молодому архитектору Павлу Федотовичу Алешину было поручено
подготовить проект строительства здания для Ольгинской гимназии, которая, на
тот момент стала нуждаться в расширении.

В июле 1914 года началось строительтво на углу улиц Владимирской и
Фундуклеевской (ныне Богдана Хмельницкого). Однако, в связи с началом мировой
войны, а также последующих революции и гражданской войны строительство было
приостановлено. И разместиться в новом здании Ольгинской гимназии так и не
пришлось.

\iusr{Ия Рудзицкая}

Когда-то музей украшали картины замечательного художника Геннадия Наумовича
гликмана. Интересно, они сохранились?
