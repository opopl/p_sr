% vim: keymap=russian-jcukenwin
%%beginhead 
 
%%file 26_12_2021.fb.fb_group.story_kiev_ua.2.istoria_kievljanina_glava_12_ljonja
%%parent 26_12_2021
 
%%url https://www.facebook.com/groups/story.kiev.ua/posts/1826926044170855
 
%%author_id fb_group.story_kiev_ua,kabysh_sergej.kiev
%%date 
 
%%tags kiev,kievljane
%%title О многом понемногу. (История киевлянина). Глава 12. Лёня
 
%%endhead 
 
\subsection{О многом понемногу. (История киевлянина). Глава 12. Лёня}
\label{sec:26_12_2021.fb.fb_group.story_kiev_ua.2.istoria_kievljanina_glava_12_ljonja}
 
\Purl{https://www.facebook.com/groups/story.kiev.ua/posts/1826926044170855}
\ifcmt
 author_begin
   author_id fb_group.story_kiev_ua,kabysh_sergej.kiev
 author_end
\fi

О многом понемногу.  (История киевлянина).  Глава 12. Лёня.

С Лёней Серёга учился в техникуме в одной группе. Как–то на первом курсе они
шли вдвоём от метро «Университет» вдоль ботанического сада к техникуму.
Неожиданно их оттеснили к забору два парня, на голову выше их и на три, четыре
года постарше.

\begin{multicols}{2} % {
\setlength{\parindent}{0pt}

\ii{26_12_2021.fb.fb_group.story_kiev_ua.2.istoria_kievljanina_glava_12_ljonja.pic.1}
\ii{26_12_2021.fb.fb_group.story_kiev_ua.2.istoria_kievljanina_glava_12_ljonja.pic.1.cmt}
\end{multicols} % }

- Молодёжь, сдавайте мелочь, - сказали они.

- С чего бы это? – спросил Лёня.

- Чтобы по голове не получить, - ответили ему.

Лёня до техникума занимался боксом, а Серёга ходил на борьбу, но сейчас они
стояли молча, как телята. Они опасались, что, если ввязаться в драку и попасть
в милицию, их обязательно исключат из техникума. Лёня с Серёгой отдали какую-то
мелочь, и парни ушли.

- Если бы я зарядил в челюсть того, что стоял ближе ко мне, ты бы своего
ударил? – спросил Лёня.

- Нет, - честно признался Серёга, - я не могу бить людей в лицо. Я бы
попробовал провести приём.

- Это неправильно, - сказал Лёня, - нечего валяться по асфальту. В таких
случаях надо вырубать и быстро уходить. Чего уж теперь говорить. Я побоялся
вылететь из техникума.

- Я тоже, - сказал Серёга.

* * *

В мае Лёня предложил ребятам отпраздновать свой день рождения на Гидропарке.
Его мама, женщина, приятная во всех отношениях, поехала вместе со всеми, чтобы
помочь организовать праздник. Солнце, горячий песок, чистый Днепр, большая
миска салата и полная сумка выпивки. Что ещё надо для веселья? Все молоды,
здоровы и убеждены в счастливом будущем. Этот день рождения запомнился как
кусочек счастья.

* * *

После техникума Лёня с Серёгой остались работать в Киеве и их пути время от
времени пересекались. В начале девяностых Лёня позвонил Серёге.

- Слушай, Серёга, я одно место знаю. Положишь сто долларов, а через месяц
заберёшь двести. Гарантировано!

- Нет, Лёня, я в эти игры не играю. Бесплатный сыр только в мышеловке.

- Зря это ты, но, как хочешь, - сказал Лёня.

Через некоторое время прошла информация, что Лёня прилично поднялся, прикупил
несколько квартир, женился и у него родился сын. А вскоре Лёня пропал. Чаще
всего ходили слухи, что он поехал в Донецк разбираться с долгами, и там его,
скорее всего, закопали.

* * *

Лет через десять Лёня вышел на связь. Он объявился в Италии, недалеко от
Неаполя. Когда Серёга собирался в Рим, он запланировал автобусную экскурсию в
Неаполь, в надежде повидать Лёню.

Они встретились на набережной Неаполя, оба плотные, с седеющими короткими
бородами.

- Вы братья? – спрашивали экскурсанты из Серёгиной группы.

- Можно и так сказать, - отвечал Серёга.

Им повезло. В автобусе были свободные места, и экскурсовод разрешил Лёне
съездить с группой в Помпеи и обратно. В дороге они успели поговорить.

\obeycr
- Как ты здесь оказался? – спросил Серёга.
- Надо было срочно уезжать из Украины, попал сюда, мне здесь понравилось, так и живу.
- У тебя здесь семья?
- Живу с женщиной.
- Так ты почти двадцать лет в Киеве не был?
- Сначала нельзя было, опасно, а сейчас, если поеду, то не смогу вернуться сюда. Я нелегал.
- А как твои в Киеве?
- Мама недавно умерла, святая была женщина. Сын в мединституте учится, всё хорошо.
- Так твои к тебе приезжали?
- Нет, но надеюсь.
- А сам в Киев собираешься?
- Хотелось бы, когда-нибудь.
\restorecr

Гуляя по развалинам Помпеи, они обошли скульптуру мальчика, которую помнили по
обложке советского учебника истории древнего мира за пятый класс. Лёня
восторженно рассказывал какая прекрасная в Италии медицина, погода, еда, вино и
природа, но в его глазах неумело пряталась печаль.

На набережной Неаполя они обменялись сувенирами и подарками, обнялись и
простились. Лёня передал друзьям столько подарков, что Серёге стало неловко.

* * *
Через пару лет, когда у Серёги появился вайбер, Лёня позвонил.

\obeycr
- Серёга, приезжай! Я всё устрою. Прекрасно отдохнёшь и совсем не дорого.
- Ты сам то как?
- Недавно делал ремонт для серьёзного человека. Он обещал помочь с легализацией.
- Да ты за двадцать лет давно мог бы легализоваться и в Киев приезжать!
- Ты понимаешь, если легализоваться, то придётся налоги платить.
- Лёня, ё-моё!  Нам уже за пятьдесят. Давно пора знать, чего ты хочешь!
- Не всё так просто...
\restorecr

* * *
Из переписки одногруппников в FB.
Саня: Жаль, что с Лёней так получилось.
Лёня: Саня, чего тебе жаль? На всё воля Божья.
Серёга: Воля Божья оставляет человеку свободу выбора. Чего уж теперь говорить?
* * *
