% vim: keymap=russian-jcukenwin
%%beginhead 
 
%%file slova.deti
%%parent slova
 
%%url 
 
%%author 
%%author_id 
%%author_url 
 
%%tags 
%%title 
 
%%endhead 
\chapter{Дети}
\label{sec:slova.deti}

\enquote{Сегодня у нас проходит мероприятие под названием \enquote{Самый
\emph{детский} день!}. Мы его так назвали, потому что в этот день вспоминают о
всех \emph{детках}, а у нас \emph{детки} особенные, которые требуют особого
внимания, особого к себе отношения, особой любви. Спасибо огромное за помощь и
поддержку нашим постоянным друзьям, которые помогают нашим \emph{деткам}}, -
поблагодарила Ильина, \textbf{Праздник для особых \emph{детей} \enquote{Самый
\emph{детский} день!} прошел в Луганске}, lug-info.com, 31.05.2021

\ifcmt
  pic http://img.lug-info.com/cache/1/5/1622459163_921.jpg/1000wm.jpg
\fi

Весь в крови и орет на пионеров. Тилль Линдеманн показал новое видео про СССР -
\enquote{Я ненавижу детей}, strana.ua, 01.06.2021

Як уберегти \emph{дітей} від небезпеки? Чи потребують змін правила організації
шкільних екскурсій, походів?  2 червня екскурсійна група у складі 44 учнів
львівської школи, трьох вчителів і двох батьків, приїхала на водоспад Кам'янка,
що у Карпатах, неподалік міста Сколе. \emph{12-річний школяр}, роблячи селфі,
виліз на каміння, послизнувся, впав униз водоспаду, зачепившись за дерево.
\emph{Дитина} перебувала під водою. Рятувальники, щоб витягнути \emph{школяра},
розпиляли дерево. На жаль, хлопець помер до приїзду медиків,
\textbf{Загибель 12-річного школяра змусила посилити заходи безпеки біля водоспаду у Карпатах},
radiosvoboda.org, 03.06.2021

«Чи дерево зіграло фатальну роль, чи спричинило смерть \emph{дитини}? Якби \emph{дитина} не
зайшла занадто близько до водоспаду, не переступила за обгородження, цього б не
сталося. Якби \emph{дитина} впала з цієї висоти просто у воду, то вдарилась би у
камінь і теж міг бути смертельний випадок. На цьому водоспаді рятувальники
вперше надавали допомогу, це перша така рятувальна операція, на жаль,
трагічна», – каже Радіо Свобода Віталій Туровцев, керівник пресслужби Головного
управління Державної служби з надзвичайних ситуацій у Львівській області,
\textbf{Загибель 12-річного школяра змусила посилити заходи безпеки біля водоспаду у Карпатах},
radiosvoboda.org, 03.06.2021

«Нічого кращого, як бути з \emph{дітьми}, говорити з ними, ділитися своїм життєвим
досвідом немає. \emph{Діти} добре засвоюють приклад батьків, як ті дали собі раду при
зустрічі з небезпекою. Шкільні екскурсії корисні для \emph{дітей}, бо це спілкування,
живий досвід взаємодопомоги, здобування досвіду походів, це здружує, виробляє
риси взаємовиручки, що основне у нашому у житті. Звісно, важко передбачити
поведінку підлітка, це вік турбулентності, то велика, то мала \emph{дитина}, а що
ввімкнеться, яка позиція, у який час, у який момент – точно не знаємо. Певні
вікові особливості закодовані природою, але те, що треба спілкуватися з \emph{дітьми},
мандрувати з ними – це дуже добре, це моменти задоволення і щастя. На жаль,
трагічні випадки будуть і від цього нікуди не подінешся, але оберігати і
застерігати, безумовно, потрібно», – наголосила психолог Галина Католик,
\textbf{Загибель 12-річного школяра змусила посилити заходи безпеки біля водоспаду у Карпатах},
radiosvoboda.org, 03.06.2021

Скільки людей щодня відвідують це відоме туристичне місце у Карпатах керівник
парку не сказав. Вхід на територію платний, але другий день після загибелі
\emph{школяра}, каса не працювала. Смерть \emph{дитини} не відлякала туристів. Сюди сьогодні
постійно приїжджали, як дорослі, так \emph{дитячі} екскурсійні групи,
\textbf{Загибель 12-річного школяра змусила посилити заходи безпеки біля водоспаду у Карпатах},
radiosvoboda.org, 03.06.2021

\enquote{На последнем совещании Леонид Иванович (Пасечник) сказал, что вопрос
\emph{детства} – самый главный. Это и подтверждает своими действиями. Сейчас мы
занимаемся распределением инвалидных колясок и передачей наиболее нуждающимся},
- уточнила уполномоченный по правам \emph{ребенка}. Она также выразила
признательность за доставку груза и его хранение Народной милиции ЛНР и
общественной организации инвалидов \enquote{Путь добра}, 
\textbf{Детский омбудсмен передала воспитаннице лутугинского интерната инвалидную коляску},
lug-info.com, 03.06.2021

\enquote{По поручению главы Луганской Народной Республики Леонида Пасечника 3 июня
уполномоченный по правам \emph{ребенка} ЛНР Юлия Назаренко передала \emph{девочке} с
нарушением опорно-двигательного аппарата, воспитаннице Государственного
образовательного учреждения Луганской Народной Республики \enquote{Лутугинская
специальная (коррекционная) школа-интернат} Валерии Веремеенко инвалидную
коляску для передвижения в помещении}, - говорится в сообщении,
\textbf{Детский омбудсмен передала воспитаннице лутугинского интерната инвалидную коляску},
lug-info.com, 03.06.2021

Государственное унитарное предприятие \enquote{Почта ЛНР} в рамках
благотворительной акции \enquote{Лучики добра} собрало 19 тыс. руб. для \emph{детских}
домов и интернатов. Об этом сообщили на предприятии.  \enquote{За период с
февраля по май 2021 года, благодаря совместным усилиям почтовиков и всех
неравнодушных жителей ЛНР, удалось собрать свыше 19 тыс. руб., из которых на
сумму свыше 9 тыс. руб. была оформлена подписка на \emph{детские} республиканские
периодические печатные издания, а на сумму около 10 тыс. руб. были осуществлены
благотворительные платежи на расчетные счета республиканских \emph{детских} социальных
учреждений}, - говорится в сообщении,
\textbf{\enquote{Почта ЛНР} в рамках акции собрала для детских домов и интернатов 19 тыс. руб.},
lug-info.com, 03.06.2021

\begingroup
Колектив Національного культурного центру України у м. Москві вітає з
Міжнародним днем захисту \emph{дітей}!, 
\textbf{Наші діти!}, ukrcentr.ru, 01.06.2021

Внаслідок російської агресії на сході України загинули 240 \emph{дітей}, -
Україна в ОБСЄ, day.kiev.ua, 03.06.2021
\endgroup

Гибель людей на Донбассе говорит о том, что наше общество больное. И, что
человек, что общество, приступают к лечению, когда чувствуют себя больными.
Именно чувствуют, но для этого надо эти самые чувства иметь. К сожалению, мы
продолжаем развивать бесчувственное общество потребителей, паразитов, не
имеющих чувств, а только желудок.  Вакциной от бесчувствия является творчество,
искусство, красота. И в этом плане творчество маленькой \emph{девочки} Фаины
Савенковой лучик света в этом огромном хаосе преступных желаний политиков и
безразличия масс. В творчестве Фаины нет пафоса, нет конъюнктуры, нет
романтизации войны - это творчество чистого \emph{Ребенка} повзрослевшего во время
войны более большинства взрослых,
\textbf{Гибель людей на Донбассе говорит о том, что наше общество больное},
Денис Жарких, strana.ua, 04.06.2021

Учреждения образования Луганска открыли 11 летних онлайн-лагерей для своих
воспитанников. Об этом сообщила пресс-служба столичной администрации со ссылкой
на заместителя начальника управления образования Анжелику Прядкину.  \enquote{У нас
открылись пришкольные онлайн-лагеря, как это было в прошлом году. Но если
минувшим летом их было 7, то теперь работают 11 лагерей, расположенных в разных
частях города, более чем для 300 детей}, - сказала она,
\citTitle{Луганский Информационный Центр — Учреждения образования Луганска
запустили 11 летних онлайн-лагерей – мэрия}, , lug-info.com, 07.06.2021

Активисты общественного движения (ОД) \enquote{Мир Луганщине} благоустроили
\emph{детскую площадку} в Луганске. Об этом сообщила пресс-служба движения.
\enquote{В Луганске 9 июня активисты общественного движения \enquote{Мир
Луганщине} в рамках акции \enquote{Начни с \emph{детства}} провели субботник на
\emph{детской} площадке, которая находится в квартале Жукова}, - говорится в
сообщении.  Пресс-служба отметила, что акция проходит уже второй год подряд в
период летних каникул под патронатом Луганского территориального отделения ОД
\enquote{Мир Луганщине}. В ней участвуют активисты проектов \enquote{Молодая
гвардия} и \enquote{Дружина} ОД \enquote{Мир Луганщине}: они убирают мусор с
\emph{детских} площадок, зачищают и красят турники, качели, лавочки, горки.
\enquote{Главная задача – сделать для \emph{детей} города Луганска пусть
небольшие, но красочные и уютные \emph{детские} площадки. Мы надеемся, что наша
работа им понравится}, – сказал член Молодежного совета при администрации
Луганска Максим Ефименко,
\citTitle{Луганский Информационный Центр — Активисты ОД \enquote{Мир Луганщине}
благоустроили детскую площадку в Луганске}, , lug-info.com, 09.06.2021

Алексей Попович
Вот  ....... реву как девчонка ... Плачу по нашему чистому \emph{детству}...
комментарий, Валерия Ланская \enquote{Прекрасное далёко}, youtube

Андрей Погорелый
Реквием по Советскому Союзу. Гимн счастливого Детства!!!
комментарий, Валерия Ланская \enquote{Прекрасное далёко}, youtube

%%%cit
%%%cit_pic
%%%cit_text
Знаете, я обожаю своё поколение.  Мы застали пластинки и бобины, перекручивали
на карандаше кассеты и сидели под стационарным телефоном в ожидании звонка; ещё
помним дискеты и диафильмы, но наравне с сегодняшними тинейджерами свободно
осваиваем все новинки техники.  Мы ещё любим Высоцкого и Магомаева, но и не
впадаем в эпилепсию от современных ритмов.  Более того, у нас и здесь нашлись
«любимчики».  Мы точно знаем, что в этом мире больше никогда не будет таких как
Майкл Джексон, Queen и Led Zeppelin, и никому до сих пор так и не удалось
заменить нам утраченных Белоусова, Талькова и Цоя.  Это мы в \emph{детстве} часами
выглаживали колючую советскую школьную форму и красный галстук, а сегодня также
трепетно заботимся о замшевой обуви
%%%cit_comment
%%%cit_title
\citTitle{Советскому поколению повезло родиться до того, как у детей отобрали право выбора}, 
Елена Лукаш, strana.ua, 13.06.2021
%%%endcit

%%%cit
%%%cit_head
%%%cit_pic
%%%cit_text
Если вы думаете, что украинский шпрехен-фюрер заточен только на выискивание
русской речи, то ничего подобного. Омбудсмен также наехал на венгерский гимн в
Закарпатье. Ну низяяяяяя, чего не понятно, только тотальная мова.  С гордостью
мовный шпрехен-фюрер в отчете отразил, что за 2020 год закрыто 300 русских
школ, но тут же тяжело вздохнул, указав, что в частных \emph{детских} садах как
говорили на русском так и говорят, и ничьи хотелки на это повлиять не могут, а
в одном частном \emph{детском саду} на Днепре вообще прямо отказались от мовы
заявив, что русский язык - это язык элиты
%%%cit_comment
%%%cit_title
\citTitle{Speak на мове please}, 
Terra Incognita, zen.yandex.ru, 30.04.2021
%%%endcit

%%%cit
%%%cit_head
%%%cit_pic
\ifcmt
  tab_begin cols=3

     width 0.3
     pic https://sharij.net/wp-content/uploads/2021/06/1-48.jpg?x35786

     pic https://sharij.net/wp-content/uploads/2021/06/2-22.jpg?x35786

     pic https://sharij.net/wp-content/uploads/2021/06/3-16.jpg?x35786

  tab_end
\fi
%%%cit_text
Материалы переданы в архив в рамках проекта «Без срока давности». В частности,
в них задокументированы случаи с вырезанием органов, испытаниями действий
препаратов и выкачиванием крови для раненых немецких военных.  Также
рассекречены сведения о нечеловеческих условиях содержания узников и убийствах
на территории концлагеря «Красный». Кроме того, раскрыты новые обстоятельства
преступлений против мирных жителей в Крыму, Запорожской и Николаевской областях
УССР. В числе жертв были и \emph{дети}
%%%cit_comment
%%%cit_title
\citTitle{В ФСБ рассекретили документы об экспериментах нацистских врачей в Крыму: вырезали органы и выкачивали кровь}, 
Светлана Комаренко, sharij.net, 22.06.2021
%%%endcit

%%%cit
%%%cit_head
%%%cit_pic
\ifcmt
  pic https://strana.ua/img/forall/u/11/33/%D0%B4%D0%B5%D0%BD%D0%B8%D1%81.png
  caption Ксения Драгунская, российский драматург, сценарист, прозаик и искусствовед
  width 0.4
\fi
%%%cit_text
Драгунская родилась в семье Виктора Драгунского, автора известного
\emph{детского} цикла \enquote{Денискины рассказы}.  Дебют Ксении в качестве
драматурга состоялся в 1993 году: ее пьесу \enquote{Земля Октября} напечатали в
журнале \enquote{Современная драматургия}. Произведение было переведено на
немецкий язык и поставлено на радио в Германии.  Драгунская также стала
автором пьес \enquote{Секрет русского камамбера утрачен навсегда-навсегда},
\enquote{Яблочный вор}, \enquote{Эдит Пиаф. Мой легионер}. К началу 2018 года
она создала более тридцати пьес.  В 2010 году вышел ее автобиографический роман
\enquote{Заблуждение велосипеда}. Также она писала киносценарии
%%%cit_comment
%%%cit_title
\citTitle{Умерла писательница Ксения Драгунская, у которой начался перитонит}, Наталья Полулях, strana.ua, 01.07.2021
%%%endcit

%%%cit
%%%cit_head
%%%cit_pic
\ifcmt
  pic https://img.strana.ua/img/article/3414/v-ukraine-umer-3_main.jpeg
  width 0.4
  caption Ирине Химич было 10 лет. Фото: Facebook/ Діна Химич 
\fi
%%%cit_text
\enquote{Ирочка умерла. Вчера остановилось ее сердце. Извини, солнышко, что не смогла в
этот раз тебя спасти}, - написала Дина Химич.
Ирина Химич болела прогерией (синдром Хатчинсона-Гилфорда). Это редкое
генетическое и неизлечимое заболевания у \emph{детей}, во время которого происходит
преждевременное старение организма. За год такие пациенты биологически стареют
на 8-10 лет. Ирине поставили диагноз, когда ей было пять лет. 
Ирина Химич из-за болезни не могла вести активный образ жизни. Она рисовала
картины, которые экспонировались на ее персональных выставках
%%%cit_comment
%%%cit_title
\citTitle{В Украине умер единственный ребенок с синдромом преждевременного старения организма}, 
Игорь Кулик, strana.ua, 01.07.2021
%%%endcit

%%%cit
%%%cit_head
%%%cit_pic
%%%cit_text
Недавно поймал себя на мысли о том, что я понятия не имею, на каких книгах
воспитываются \emph{дети} \enquote{истинных} (украиноязычных, родившихся в сёлах западной
Украины) украинцев. Я воспитывался на русских былинах, сказках, стихах Пушкина,
классической музыке на аудиокасетах. Подозреваю, что обо всём этом они тоже не
имеют ни малейшего понятия
%%%cit_comment
%%%cit_title
\citTitle{Разные группы украинцев живут в разных информационных и культурных реальностях / Лента соцсетей / Страна}, 
Даниил Богатырев, strana.ua, 05.07.2021
%%%endcit

%%%cit
%%%cit_head
%%%cit_pic
\ifcmt
  pic https://strana.ua/img/forall/u/0/36/2021-07-05_16h48_18.png
	width 0.5
\fi
%%%cit_text
С предыдущим мнением не согласна местная жительница, для которой переименование
города в Нью-Йорк было \emph{мечтой детства}, так как это историческое название
поселка.  \enquote{Я очень рада. Это было моей мечтой с \emph{детства}. Мои
дедушки-бабушки-прабабушки, родители, мы и мое поколение из школы идем и
[спрашиваем], как ты пойдешь - переулком или через Нью-Йорк. То есть он у нас
был на слуху. Я очень рада событию}, - рассказала нам женщина.  Она еще не
обсуждала эту новость со знакомыми, но допускает, что мнения людей могут быть
различны
%%%cit_comment
%%%cit_title
\citTitle{Гетьман в Нью-Йорке. Как Порошенко устроил черешня-тур на Донбасс}, 
Оксана Малахова, strana.ua, 06.07.2021
%%%endcit

%%%cit
%%%cit_head
%%%cit_pic
\ifcmt
  pic https://img.strana.ua/img/article/3423/na-kamchatke-hde-56_main.jpeg
	width 0.4
	caption Ан-26 разбился накануне. Фото: t.me/bazabazon 
\fi
%%%cit_text
Во время спецоперации по ликвидации последствий трагедии пассажирского
авиалайнера Ан-26 на Камчатке обнаружили уже девять тел. Среди них спасатели
опознали \emph{ребенка} - им оказался 17-летний Артём Тищенко.  Об этом пишет
Telegram-канал Mash в среду, 7 июля
%%%cit_comment
%%%cit_title
\citTitle{На месте крушения Ан-26 в РФ извлекли девять тел и опознали ребенка. Фото}, 
Елена Вьюн, strana.ua, 07.07.2021
%%%endcit

%%%cit
%%%cit_head
%%%cit_pic
%%%cit_text
Недавним примером такого движения стали результаты экзаменов, показавшие, что
\emph{российские дети} просто не желают учиться. И бесполезно обвинять в этом
затравленных поставщиков образовательных услуг учителей или не имеющую времени
на воспитание \emph{ребенка} семью. Это диагноз всему российскому обществу, на примере
которого \emph{дети} видят, что затраты времени и усилий на приобретение знаний не
дают никаких преимуществ в достижении социального успеха и материального
достатка, а для получения их нужны совершенно другие личностные качества и
индивидуальные умения, а в последнее время вообще никаких, кроме улыбки удачи
%%%cit_comment
%%%cit_title
\citTitle{Революция Духа – единственный путь спасения России}, 
Юрий Барбашов, voskhodinfo.su, 30.06.2021
%%%endcit

%%%cit
%%%cit_head
%%%cit_pic
%%%cit_text
Чи потрібна націонал-патріотові така справжня Україна? З двома мовами, що
співіснують, – українською і українською російською? З тими, хто сумує за СРСР,
але так само любить своїх \emph{дітей} і так само хоче жити краще?  З тими,
хто, попри ідеологічні суперечки, боронить Україну проти Росії на Донбасі?
%%%cit_comment
%%%cit_title
\citTitle{Українці не розуміють одне одного не через мову, а через небажання слухати, чути і сприймати}, 
Юлія Мендель, www.pravda.com.ua, 07.07.2021
%%%endcit

%%%cit
%%%cit_head
%%%cit_pic
%%%cit_text
Майор.
Очень долго думал, стоит об этом писать или нет. Потому что есть темы, где
удобней не топтаться. Но все же напишу, потому что это касается всех нас.
На днях на Донбассе погиб украинский военный. Майор Богдан. И так вышло, что
это сын одной из лучших подруг моей мамы тети Оли. Она его всегда называла
Бодя. И все мое подростковое \emph{детство} в Гайсине прошло рядом с Богданом. Не то,
чтобы мы с ним дружили, но дружили наши мамы. А значит все дни рождения,
дежурные походы в гости – это все проходило у нас совместно.
Когда я закончил школу – я старше Богдана на 4 года – и уехал из Гайсина, с ним
мы практически больше не виделись. Помню, Богдан как-то даже стал заочной
причиной скандала между мной и мамой. Уже студентом приехав на пару дней в
Гайсин, я увидел, что не хватает каких-то моих машинок и солдатиков.
Оказывается мама дала их поиграть (с возвратом и он, действительно, потом их
вернул) Богдану. А я орал, что это моя частная собственность и вообще, какого
хера
%%%cit_comment
%%%cit_title
\citTitle{Наши ребята гибнут там, где у власти нет ни стратегической, ни тактической цели}, 
Игорь Лесев, strana.ua, 08.07.2021
%%%endcit

%%%cit
%%%cit_head
%%%cit_pic
%%%cit_text
Вспоминая \emph{детство}, выходец из Елисаветградского уезда Лев Троцкий в
книге «Моя жизнь» писал: «Ни одна из столиц мира — ни Париж, ни Нью-Йорк не
произвела на меня впоследствии такого впечатления, как Елисаветград, с его
тротуарами, зелеными крышами, балконами, магазинами, городовыми и красными
шарами на ниточках. В течение нескольких часов я широко раскрытыми глазами
глядел в лицо цивилизации»
%%%cit_comment
%%%cit_title
\citTitle{Игорь Тамм. Нобелевский лауреат из Елизаветграда}, Дмитрий Губин, ukraina.ru, 08.07.2021
%%%endcit

%%%cit
%%%cit_head
%%%cit_pic

\ifcmt
  tab_begin cols=2
		 caption Вчитель Геннадій Іванченко

		 pic https://life.pravda.com.ua/images/doc/1/d/1d472b8-ivanchenko-1.jpg
     caption Геннадій Іванченко із Приазов’я вже 34 роки працює в школі, а понад 20 років очолює маріупольський осередок національної скаутської організації \enquote{Пласт}. Фото: Андрій Болотнов

     pic https://life.pravda.com.ua/images/doc/2/b/2b19364-ivanchenko-2.jpg
		 caption Своїх учнів Геннадій постійно заохочує до соціально важливих проєктів. Фото: Андрій Болотнов

  tab_end
\fi

%%%cit_text
Учитель за сім років своєї волонтерської діяльності об’їхав усю лінію
розмежування,  неодноразово бував на передових позиціях, підтримуючи
захисників: привозив харчі, необхідні речі й щоразу обов’язково передавав листи
та малюнки від \emph{дітей}: 
\enquote{У Маріуполі багато людей знають мої проукраїнські настрої. Тому в нашу
школу часто приводять \emph{дітей} батьки-патріоти.  Ці \emph{діти} знають і
люблять своє місто, вивчають історію України і бачать своє майбутнє в контексті
країни. Вони черпають інформацію з різних джерел, постійно поповнюючи свій
багаж знань..} Своїх учнів Геннадій постійно заохочує до соціально важливих
проєктів: \enquote{Ми з \emph{дітьми} влаштовуємо концерти для військових,
організовуємо зустрічі з діячами культури та письменниками, відвідуємо
патріотичні заходи, а нещодавно збирали книги для культурного осередку у
Маріуполі} 
%%%cit_comment
%%%cit_title
\citTitle{Не бачили України без війни: як учителі в прифронтових зонах інтегрують дітей в українське суспільство}, 
Жанна Дуфіна; Дар'я Городецька, life.pravda.com.ua, 09.07.2021
%%%endcit

%%%cit
%%%cit_head
%%%cit_pic
%%%cit_text
На вопрос о том, сталкивается ли он с давлением со стороны сверстников, Михаил
говорит, что притеснения бывают в форме... русскоязычных песен: "Когда мы
приходим к кому-то в гости, и кто-то из моих друзей напевает какую-то русскую
песню, меня это ужасно бесит, я начинаю носиться, как безумный, и кричать: "Для
чего ты это слушаешь? Это же дерьмо!" Меня просто разрывает".  Мальчик также
раскритиковал русскоязычных учителей. И призвал выгонять из школы \emph{детей}
и педагогов за переход на русский
%%%cit_comment
%%%cit_title
\citTitle{Песни на русском – дерьмо".  Как на канале Порошенко с ребенком о национализме говорили}, 
Анна Копытько, strana.ua, 27.07.2021
%%%endcit
