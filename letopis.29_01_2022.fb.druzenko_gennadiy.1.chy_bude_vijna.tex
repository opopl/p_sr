% vim: keymap=russian-jcukenwin
%%beginhead 
 
%%file 29_01_2022.fb.druzenko_gennadiy.1.chy_bude_vijna
%%parent 29_01_2022
 
%%url https://www.facebook.com/gennadiy.druzenko/posts/10158737380763412
 
%%author_id druzenko_gennadiy
%%date 
 
%%tags napadenie,rossia,ugroza,ukraina,vojna
%%title ЧИ БУДЕ ВІЙНА?
 
%%endhead 
 
\subsection{ЧИ БУДЕ ВІЙНА?}
\label{sec:29_01_2022.fb.druzenko_gennadiy.1.chy_bude_vijna}
 
\Purl{https://www.facebook.com/gennadiy.druzenko/posts/10158737380763412}
\ifcmt
 author_begin
   author_id druzenko_gennadiy
 author_end
\fi

ЧИ БУДЕ ВІЙНА?

Напевно це зараз найбільш популярне питання в Україні. Накал затятих
суперечкок, нападе Путін чи не нападе, які виплеснулись у телефонні розмови,
теревені за обідом, діалоги в маршрутках та бої у коментах під фейсбучними
дописами напевно перевершив навіть накал переманетних боїв адептів Петро
Порошенко та прихильників Володимир Зеленський.

\ifcmt
  ig https://scontent-frt3-2.xx.fbcdn.net/v/t39.30808-6/272772416_10158737373958412_1065299288335861216_n.jpg?_nc_cat=103&ccb=1-5&_nc_sid=730e14&_nc_ohc=hyhd2cY6W5AAX-iTb5X&_nc_ht=scontent-frt3-2.xx&oh=00_AT9UzuYZpSzoUD1bWafWuC0DT8Q9Brw97mzC1xLK_vla4A&oe=61FC5860
	@wrap center
	@width 0.6
\fi

Іронія полягає в тому, що єдина розумна, поміркована, зважена, прагматична та
твереза відповідь на це питання – МИ НЕ ЗНАЄМО. Решта – ворожіння на кавовій
гущі. Бо будь-які прагматичні розрахунки можуть більш-менш точно відповісти на
запитання, наскільки сторони готові до війни, які здобутки та втрати понесе
агресор та його жертва у разі нападу, які ресурси з обох боків можуть бути
кинуті у топку війни тощо...

Але біда в тому, що люди не є раціональними акторами, які зважують всі за і
проти, а потім діють. Їхні розрахунки спотворюють емоції, вони завжди ухвалюють
рішення, виходячи з неповної інформації, ми завжди є заручниками свого минулого
досвіду. Врешті-решт людям властиво помилятись. Якщо б можновладці не
помилялись, ніколи б не трапилась самогубча для Європи Перша світова війна
(раджу перечитати \enquote{The Guns of August: The Outbreak of World War I} by Barbara
W. Tuchman – твір, який Президент Кеннеді розіслав усім членам Конгресу та
американським генералам під час Карибської кризи).

Дуже переконливо описує принципову непередбачуваність майбутнього співець
чорних лебедів Насім Талеб: 

"Перший елемент тріади — патологічна віра в те, що наш світ зрозуміліший,
пояснюваний і, відповідно, передбачуваніший, ніж є насправді.

Дорослі весь час повторювали, що війна, яка протривала майже сімнадцять років,
закінчиться «за кілька днів». Вони були цілком упевнені у своїх прогнозах. Про
це ж свідчила й велика кількість біженців, котрі пересиджували війну в готелях
та інших тимчасових помешканнях на Кіпрі, у Греції, у Франції. Один мій дядько
розповідав, що тридцять років тому багаті палестинці, тікаючи в Ліван, вважали
цей варіант винятково тимчасовим (більшість із тих, хто досі живий, усе ще там,
хоч минуло вже шістдесят років). Але на питання, чи не затягнеться наша війна,
він відповів: «Ні, що ти. У нас усе по-іншому; завжди було по-іншому». Чужий
приклад його ніби не стосувався."

Дивним чином відповідь на питання, нападе чи не нападе Путін, потрібно шукати у
Сократа та Фукуями, який нещодавно нагадав нам про \enquote{третю частину душі} –
тимос. У своїй праці \enquote{Ідентичність} Фукуяма повертає нас до великого відкриття,
який Платон записав у своїй \enquote{Державі}: \enquote{За понад два тисячоліття до початку
епохи сучасної економіки Сократ із Главконом зрозуміли те, що для неї
залишилося непоміченим. Бажання і розум — складові частини людської психіки
(душі), але третій складник, тимос, діє цілком незалежно від них,} – і
продовжує: \enquote{ця третя частина душі, тимос, — царина сучасної політики
ідентичності}. 

Відчуття скривдженості та часто ресентименту рухає і нами і росіянами. Ми
воліємо проклянути \enquote{материнську утробу} імперії (за Снайдером, визрівання
модерних європейських націй в лоні імперій було правилом, а не девіацією) та
відцуратися від радянської спадщини з усіма її злочинами та здобутками, наче
українці не були ані співтворцями Російської імперії, ані співтворцями СРСР.

Росіяни, навпаки, не можуть зрозуміти, як трапилось так, що у них під носом
раптом зʼявилась нація, яка чудово розуміє їхню мову і говорить нею, з якою
триста років вони прожили в одній державі, представники якої обіймали найвищі
посади і в Російській імперії і в СРСР, творили імперську та радянську
культуру, воювали пліч-о-пліч зі спільними ворогами, але з незбагненою для
росіян затятістю прагне вирватись з гравітації нашого спільного минулого наче
його і не було.

І росіяни якщо не розуміють, то відчувають, що без України вони ніколи знову не
стануть тою \enquote{великою Росією}, якою були десь від часів створення імперії
(початок 17 століття) до розпаду СРСР (кінець 20-го). 

Френсіс Фукуяма правий, що \enquote{рушійна сила сучасної політики ідентичності — це
прагнення маргіналізованих груп до визнання їх рівними. Але таке бажання може
легко перетворитися на потребу бути визнаними як вища група. Це значною мірою
стосується націоналізму й національної ідентичності, а також деяких форм
екстремістської релігійної політики сучасності.} І в цьому плані і в нас і в
росіян однаковий рушій.

Тільки якщо ми прагнемо визнати нас рівними серед рівних, то росіяни – вищими
за інших. Принаймні у \enquote{своїй зоні впливу}. І ці два несумісних тимоса
підштовхують наші кораблі до лобового зіткнення. Відбудеться воно чи ні, ми не
знаємо. Це точно вирішуватимуть не економіка чи раціональні розрахунки. Точніше
не тільки вони. Але для мене самоочевидною істиною є те, що до цього зіткнення
слід готуватись. Краще підстелити соломку там, де не впав, аніж впасти там, де
не підстелив...

Si vis pacem para bellum!
